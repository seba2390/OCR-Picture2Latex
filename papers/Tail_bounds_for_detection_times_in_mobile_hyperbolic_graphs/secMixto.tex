
In this section we define the set $\dt$ similarly as in the radial section, without the restriction that $|\theta_0| \le \tfrac{\pi}{2}$, since also points that belong to the half disk of $B_O(R)$ opposite to $Q$ are starting positions from which particles can detect (mainly thanks to the angular movement component). As in previous sections, on a high level, this set $\dt$ is chosen in such a way that at least a constant part of the measure of the overall detection probability comes from~$\dt$. The precise definition of $\dt$ (given in the two following theorems) depends on the fact whether $\ss$ is small or large. For $\ss$ large, we can moreover provide uniform lower bounds of detection for every element in $\dt$ and matching uniform upper bounds of detection for every element in $\ndt$. To make this intuition precise, we have the following two theorems, depending on whether $\ss$ is small or large: 

\begin{theorem}\label{thm:mixedSmall} 
Assume $\beta \le 1/2$,  $\ss=\Omega((e^{\beta R}/n)^2)\cap O(1)$. Let
\begin{equation}\label{mix:lower:defD}
\dt := \{ x_0=(r_0,\theta_0)\in B_O(R) \mid |\theta_0|\leq \phi(r_0) +\sqrt{\ss}e^{-\beta r_0} \}. 
\end{equation}
Then 
\[
\int_{\dt} \PP_{x_0}(T_{det}\leq\ss)d\mu(x_0)=\Theta(\mu(\dt))=\Theta(n e^{-\beta R}\sqrt{\ss})
\]
and 
\[
\int_{\ndt} \PP_{x_0}(T_{det}\leq\ss)d\mu(x_0)=O(\mu(\dt)).
\]
\end{theorem}
\begin{theorem}\label{thm:mixedLarge}
Let $\KA > 1$. Define then $\KR:=\KA$ if $\alpha < 2\beta$ and $\KR:=e^{\KA^2}$ if $\alpha \ge 2\beta$. Let 
\[
\dt(\KA):=\{x_0=(r_0,\theta_0)\in B_O(R) \mid |\theta_0|\leq\phi(R)+\KA\phis \vee |\theta_0|\leq \phi(r_0)+\KR e^{-(\beta\wedge\frac12) r_0}\}, 
\]
where 
\[\phis:=\begin{cases}
(\ss^{\frac{1}{\alpha}}/e^R)^{\beta\wedge\frac{1}{2}}, 
% & \text{if $\alpha<2\beta$ and $\beta\leq\frac12$},\\[2pt]
 & \text{if $\alpha<2\beta$},\\[2pt]
e^{-\beta R}\sqrt{\ss \log\ss}, & \text{if $\alpha=2\beta$},\\[2pt]
e^{-\beta R}\sqrt{\ss},&\text{if $\alpha>2\beta$}.
\end{cases}
\]
Furthermore, define $$\mathfrak{Z} := \begin{cases}
e^{\alpha R}, &
\text{if $\alpha < 2\beta$,} \\
e^{\alpha R}/(\alpha R), &
\text{if $\alpha=2\beta$,} \\
e^{2\beta R}, & \text{if $\alpha > 2\beta$.} 
\end{cases} $$
Then, for $\ss=\Omega(1)\cap O(\mathfrak{Z})$ we have 
\[
\int_{\ndt(\KA)} \PP_{x_0}(T_{det}\leq\ss)d\mu(x_0)=O(\mu(\dt(\KA))).
\]
Furthermore, under the additional assumption $\ss=\omega(1)$ we have
%and also $\ss=O(\mathfrak{Z})$, the following hold:
$$
\inf_{x_0\in\dt(\KA)}\P_{x_0}(T_{det}\leq \ss) =
\begin{cases}
e^{-O(\KA^2)},
& \text{ if $\alpha\geq2\beta$,} \\[2pt]
\Omega(\KA^{-\alpha/(\beta\wedge\frac12)}),
& \text{ if $\alpha<2\beta$,} 
\end{cases}
$$
and
\[
\sup_{x_0\in\ndt(\KA)}\P_{x_0}(T_{det}\leq \ss) =
\begin{cases}
e^{-\Omega(\KA^2)},
& \text{ if $\alpha\geq2\beta$,} \\[2pt]
O(\KA^{-\alpha/(\beta\wedge\frac12)}),
& \text{ if $\alpha<2\beta$.} 
\end{cases}
\]
\end{theorem}


\begin{figure}
    \centering
    
\begin{tikzpicture}[x=1cm,y=1cm,scale=0.65,
     decoration={markings,
       mark=at position 1 with {\arrow[scale=1.5,black]{latex}};
      }]
      
      \def\c{12}
      \def\barr{4}
      \def\radp{6.8}
      \def\radb{2.95}
      \def\delta{1.5}
      \def\angb{45}
      \def\phip{200}
      \def\phiro{266.5}
      \def\phiss{185}
      \def\angabs{-71.4};
      \node[inner sep=0] (O) at (\c,\c) {};
      \node[inner sep=0] (P) at (2,4) {};
      \node[inner sep=0] (Q) at (\c,0) {};
  \draw[fill=gray!20] (8.204cm,0.616cm) --(10.262cm,6.803cm) -- (10.178cm,7.003cm) -- (10.092cm,7.208cm) -- (10.005cm,7.419cm) -- (9.918cm,7.636cm) -- (9.831cm,7.859cm) -- (9.747cm,8.090cm) -- (9.664cm,8.328cm) -- (9.586cm,8.575cm) -- (9.513cm,8.830cm) -- (9.447cm,9.094cm) -- (9.391cm,9.367cm) -- (9.346cm,9.649cm) -- (9.315cm,9.940cm) -- (9.301cm,10.238cm) -- (9.307cm,10.543cm) -- (9.336cm,10.852cm) -- (9.391cm,11.162cm) -- (9.476cm,11.470cm) -- (9.593cm,11.771cm) -- (9.744cm,12.059cm) -- (9.930cm,12.326cm) -- (10.150cm,12.565cm) -- (10.402cm,12.767cm) -- (10.631cm,12.900cm) -- (10.678cm,12.923cm) -- (10.727cm,12.944cm) -- (10.775cm,12.963cm) -- (10.824cm,12.981cm) -- (10.873cm,12.997cm) -- (10.923cm,13.011cm) -- (10.972cm,13.024cm) -- (11.022cm,13.035cm) -- (11.072cm,13.044cm) -- (11.122cm,13.052cm) -- (11.172cm,13.058cm) -- (11.222cm,13.062cm) -- (11.271cm,13.064cm) -- (11.321cm,13.064cm) -- (11.370cm,13.063cm) -- (11.418cm,13.060cm) -- (11.466cm,13.054cm) -- (11.513cm,13.048cm) -- (11.560cm,13.039cm) -- (11.606cm,13.028cm) -- (11.651cm,13.016cm) -- (11.695cm,13.002cm) -- (11.738cm,12.987cm) -- (11.780cm,12.969cm) -- (11.820cm,12.950cm) -- (11.860cm,12.930cm) -- (11.898cm,12.908cm) -- (11.934cm,12.884cm) -- (11.969cm,12.859cm) -- (12.002cm,12.833cm) -- (11.998cm,12.833cm) -- (12.031cm,12.859cm) -- (12.066cm,12.884cm) -- (12.102cm,12.908cm) -- (12.140cm,12.930cm) -- (12.180cm,12.950cm) -- (12.220cm,12.969cm) -- (12.262cm,12.987cm) -- (12.305cm,13.002cm) -- (12.349cm,13.016cm) -- (12.394cm,13.028cm) -- (12.440cm,13.039cm) -- (12.487cm,13.048cm) -- (12.534cm,13.054cm) -- (12.582cm,13.060cm) -- (12.630cm,13.063cm) -- (12.679cm,13.064cm) -- (12.729cm,13.064cm) -- (12.778cm,13.062cm) -- (12.828cm,13.058cm) -- (12.878cm,13.052cm) -- (12.928cm,13.044cm) -- (12.978cm,13.035cm) -- (13.028cm,13.024cm) -- (13.077cm,13.011cm) -- (13.127cm,12.997cm) -- (13.176cm,12.981cm) -- (13.225cm,12.963cm) -- (13.273cm,12.944cm) -- (13.322cm,12.923cm) -- (13.369cm,12.900cm) -- (13.416cm,12.876cm) -- (13.463cm,12.851cm) -- (13.509cm,12.824cm) -- (13.554cm,12.796cm) -- (13.598cm,12.767cm) -- (13.642cm,12.736cm) -- (13.685cm,12.704cm) -- (13.728cm,12.671cm) -- (13.769cm,12.637cm) -- (13.810cm,12.601cm) -- (13.850cm,12.565cm) -- (13.889cm,12.527cm) -- (13.927cm,12.489cm) -- (13.964cm,12.450cm) -- (14.000cm,12.409cm) -- (14.035cm,12.368cm) -- (14.070cm,12.326cm) -- (14.103cm,12.283cm) -- (14.136cm,12.240cm) -- (14.167cm,12.195cm) -- (14.198cm,12.150cm) -- (14.227cm,12.105cm) -- (14.256cm,12.059cm) -- (14.283cm,12.012cm) -- (14.310cm,11.965cm) -- (14.336cm,11.917cm) -- (14.360cm,11.869cm) -- (14.384cm,11.820cm) -- (14.407cm,11.771cm) -- (14.429cm,11.722cm) -- (14.449cm,11.672cm) -- (14.469cm,11.622cm) -- (14.488cm,11.571cm) -- (14.507cm,11.521cm) -- (14.524cm,11.470cm) -- (14.540cm,11.419cm) -- (14.556cm,11.368cm) -- (14.570cm,11.317cm) -- (14.584cm,11.265cm) -- (14.597cm,11.214cm) -- (14.609cm,11.162cm) -- (14.620cm,11.110cm) -- (14.630cm,11.059cm) -- (14.640cm,11.007cm) -- (14.649cm,10.955cm) -- (14.657cm,10.903cm) -- (14.664cm,10.852cm) -- (14.671cm,10.800cm) -- (14.677cm,10.748cm) -- (14.682cm,10.697cm) -- (14.686cm,10.645cm) -- (14.690cm,10.594cm) -- (14.693cm,10.543cm) -- (14.696cm,10.492cm) -- (14.698cm,10.441cm) -- (14.699cm,10.390cm) -- (14.700cm,10.339cm) -- (14.700cm,10.288cm) -- (14.699cm,10.238cm) -- (14.698cm,10.188cm) -- (14.697cm,10.138cm) -- (14.695cm,10.088cm) -- (14.692cm,10.038cm) -- (14.689cm,9.989cm) -- (14.685cm,9.940cm) -- (14.681cm,9.891cm) -- (14.677cm,9.842cm) -- (14.672cm,9.793cm) -- (14.666cm,9.745cm) -- (14.661cm,9.697cm) -- (14.654cm,9.649cm) -- (14.648cm,9.601cm) -- (14.641cm,9.554cm) -- (14.634cm,9.507cm) -- (14.626cm,9.460cm) -- (14.618cm,9.413cm) -- (14.609cm,9.367cm) -- (14.601cm,9.321cm) -- (14.592cm,9.275cm) -- (14.582cm,9.229cm) -- (14.573cm,9.184cm) -- (14.553cm,9.094cm) -- (14.487cm,8.830cm) -- (14.414cm,8.575cm) -- (14.336cm,8.328cm) -- (14.253cm,8.090cm) -- (14.169cm,7.859cm) -- (14.082cm,7.636cm) -- (13.995cm,7.419cm) -- (13.908cm,7.208cm) -- (13.822cm,7.003cm) -- (13.738cm,6.803cm) -- (15.845cm,0.633cm) --(15.475cm,0.514cm) --(15.101cm,0.408cm) --(14.724cm,0.313cm) --(14.344cm,0.231cm) --(13.961cm,0.161cm) --(13.577cm,0.104cm) --(13.190cm,0.059cm) --(12.803cm,0.027cm) --(12.415cm,0.007cm) --(12.026cm,0.000cm) --(11.637cm,0.005cm) --(11.249cm,0.024cm) --(10.861cm,0.054cm) --(10.475cm,0.097cm) --(10.090cm,0.153cm) --(9.707cm,0.221cm) --(9.327cm,0.302cm) --(8.949cm,0.394cm) --(8.575cm,0.499cm) --(8.204cm,0.616cm) --cycle;

\draw[fill=gray!40] (11.941cm,0.000cm) -- (11.902cm,1.200cm) -- (11.842cm,2.401cm) -- (11.748cm,3.604cm) -- (11.607cm,4.811cm) -- (11.404cm,6.030cm) -- (11.136cm,7.278cm) -- (10.842cm,8.591cm) -- (10.798cm,8.820cm) -- (10.758cm,9.051cm) -- (10.725cm,9.285cm) -- (10.698cm,9.521cm) -- (10.681cm,9.760cm) -- (10.675cm,9.999cm) -- (10.681cm,10.239cm) -- (10.704cm,10.477cm) -- (10.744cm,10.711cm) -- (10.804cm,10.938cm) -- (10.885cm,11.154cm) -- (10.988cm,11.356cm) -- (11.113cm,11.538cm) -- (11.260cm,11.696cm) -- (11.426cm,11.825cm) -- (11.608cm,11.921cm) -- (11.801cm,11.980cm) -- (12.000cm,12.000cm) --(12.199cm,11.980cm) -- (12.392cm,11.921cm) -- (12.574cm,11.825cm) -- (12.740cm,11.696cm) -- (12.887cm,11.538cm) -- (13.012cm,11.356cm) -- (13.115cm,11.154cm) -- (13.196cm,10.938cm) -- (13.256cm,10.711cm) -- (13.296cm,10.477cm) -- (13.319cm,10.239cm) -- (13.325cm,9.999cm) -- (13.319cm,9.760cm) -- (13.302cm,9.521cm) -- (13.275cm,9.285cm) -- (13.242cm,9.051cm) -- (13.202cm,8.820cm) -- (13.158cm,8.591cm) -- (13.112cm,8.366cm) -- (13.063cm,8.144cm) -- (13.013cm,7.924cm) -- (12.963cm,7.707cm) -- (12.913cm,7.492cm) -- (12.864cm,7.278cm) -- (12.815cm,7.067cm) -- (12.768cm,6.857cm) -- (12.723cm,6.649cm) -- (12.679cm,6.441cm) -- (12.636cm,6.235cm) -- (12.596cm,6.030cm) -- (12.557cm,5.825cm) -- (12.521cm,5.621cm) -- (12.486cm,5.418cm) -- (12.453cm,5.215cm) -- (12.422cm,5.013cm) -- (12.393cm,4.811cm) -- (12.366cm,4.609cm) -- (12.340cm,4.408cm) -- (12.316cm,4.206cm) -- (12.293cm,4.005cm) -- (12.272cm,3.805cm) -- (12.252cm,3.604cm) -- (12.158cm,2.401cm) -- (12.098cm,1.200cm) -- (12.059cm,0.000cm) -- (12.059cm,0.000cm) -- (12.045cm,0.000cm) -- (12.030cm,0.000cm) -- (12.015cm,0.000cm) -- (12.000cm,0.000cm) -- (11.985cm,0.000cm) -- (11.970cm,0.000cm) -- (11.955cm,0.000cm) -- (11.941cm,0.000cm) -- cycle;
\draw[fill=black] (O) ++(\phiss:\radp-1) circle (0.07) node[right, below] {$x_{\ss}$};
\draw[fill=black] (O) ++(\phip:\radp) circle (0.07) node[above left] {$x_0$};
\draw[dashed] (O)  ++(\phip:0) -- ++(\phip:\c);

      \draw[dotted] (O) -- (Q);
      \draw[dotted] (O) -- ++(\angabs:\c+2);
      \draw[dotted] (\c,-0.75) -- (\c,-2);
      \begin{scope}[xshift=340,yshift=340]
        \draw[dash dot] (\phip-\angb:\c) -- (\phip-\angb:\radb) arc (\phip-\angb:\phip+\angb:\radb) -- (\phip+\angb:\c);
        \draw[postaction={decorate}] (\phip:\radp) arc (\phip:\phip+\angb:\radp);
        \draw[postaction={decorate}] (\phip+\angb:\radp) arc (\phip+\angb:\phip:\radp) node[midway,xshift=7.5pt,yshift=5pt] {$_{\widehat{\theta}_0}$};
        \draw[dashed] (0,0) -- (\phip+\angb:\radb);
        \draw[postaction={decorate}] (\phiro:\radp) arc (\phiro:\phip+\angb:\radp) node[midway, above, xshift=5pt] {$_{\frac12\widehat{\theta}_0}$};
        \draw[postaction={decorate}] (\phip+\angb:\radp) arc (\phip+\angb:\phiro:\radp);
        \draw[postaction={decorate}] (270:\radp+\delta) arc (270:\phip:\radp+\delta) node[midway, above] {$_{\theta_0}$};
        %\draw[postaction={decorate}] (\phip:\radp+\delta) arc (\phip:270:\radp+\delta);
        \draw[postaction={decorate}] (270:\c+1.5) arc (270:360+\angabs:\c+1.5) node[midway, below] {$_{\phi(R)+\kappa_A\phi^{(\ss)}}$};
        \draw[postaction={decorate}] (360+\angabs:\c+1.5) arc (360+\angabs:270:\c+1.5);
        \draw[postaction={decorate}] (\phiro:\c+1.5) arc (\phiro:270:\c+1.5) node[midway, below] {$_{\phi(r_0)}$};
        \draw[postaction={decorate}] (270:\c+1.5) arc (270:\phiro:\c+1.5);
     % \draw[fill=black] (\phiro:\radp) circle (0.08) node[above, xshift=5pt] {$_{B_0}$};       
        \draw[postaction={decorate}] (0,0) -- (\phiro:\radp) node[midway, left, xshift=2pt] {$_{r_0}$};
      \draw[dotted] (0,0) -- (\phiro:\c+2);
      \draw[dashed] (0,0) circle (5.4);
      \draw[postaction={decorate}] (0,0) -- (-50:5.4) node[midway, right] {$_{r''}$};
      \draw[dashed] (0,0) circle (0.87);
      \draw (0,0) -- (30:1.37);
      \draw[postaction={decorate}] (30:1.57) -- (30:0.87) node [right, xshift=2pt,yshift=-1pt] {$_{r'}$};
      \draw[postaction={decorate}] (0,0) -- (180:\radb) node [midway, above, yshift=-2pt] {$_{\widehat{r}_0}$};
      \end{scope}
      \draw[thick,black] (\c,\c) circle (\c);
      \draw[fill=black] (Q) circle (0.07);
      \draw[fill=black] (O) circle (0.07);
      \node[below] at (Q) {$_Q$};      
      \node[above] at (O) {$_O$}; 
	\end{tikzpicture}
    \caption{(a) The lightly shaded area corresponds to $\dt(\KA)\setminus B_{Q}(R)$ as defined in Theorem~\ref{thm:mixedLarge} and the strongly shaded region represents $B_Q(R)$. (b) The smallest (respectively largest) circumference whose
    boundary is a dashed line is of radius $r'$
    ($r''$, respectively). 
    (c) The region whose boundary is the dashed-dot line corresponds to the box $\mathcal{B}(x_0):=[\widehat{r}_0,R]\times [\theta_0-\widehat{\theta}_0,\theta_0+\widehat{\theta}_0]$ when $\ss$ is large.}
    \label{fig:mixto}
\end{figure}
%\cml{If the picture contains the box, then it should be placed in the upper bounds section no?}
%\cmk{Below I only refer to item (a) of the picture. In the upper bound secton I will refer to item (b), and so on until all of its aspects are introduced. Sounds ok?}\cml{ok then!}\dmc{Maybe we could have three figures: the one only with the grey area here, only part (a). Then, once the box is described, also the box (which is (a) and (c). And then finally the radii $r'$ and $r''$ which also (b)}\dmc{I thought the old version had the right $\hat{\theta}_0$ that is half of the width of the box around $x_0$}

We refer to Figure~\ref{fig:mixto}(a) for an illustration of $\dt(\KA)$ as defined in the last theorem.

\medskip
The following subsection is dedicated to proving the lower bounds of the two theorems just stated (in fact, as can be seen from the proof, the assumptions about $\ss$ are slightly milder in the proofs). The subsequent subsection then deals with the corresponding upper bounds.


\subsection{Lower bounds}\label{sec:Lower}
%
In this subsection we prove the lower bounds of Theorem~\ref{thm:mixedSmall} and~\ref{thm:mixedLarge}.

\subsubsection{The case $\ss$ small}
We start with the proof of the lower bound of Theorem~\ref{thm:mixedSmall}. 
We establish the following proposition, which proves the  lower bound of Theorem~\ref{thm:mixedSmall}. %\dmc{You are right that the bound $\ss \in \Omega(n^{4\beta-2})$ is not needed for the proof. But if $\ss$ is even smaller than the right hand side is less than constant. Should we still put it? For me ok, so I left for now your suggestion of statement}\cmk{I would remove the superflous assumption.}\dmc{Ok, took out the lower bound on $\ss=\Omega(n^{4\beta-2})$}
\begin{proposition}\label{generalschico_lower}
 Fix $\beta\le\frac{1}{2}$ and take $\dt$ as in~\eqref{mix:lower:defD}. If $\ss=O(1)$, then
\[\int_{\dt}\P_{x_0}(T_{det}\leq \ss)d\mu(x_0) 
\;=\; \Omega(ne^{-\beta R}\sqrt{\ss}).\]
\end{proposition}
\begin{proof}
Consider $x_0$ with $|\theta_0| \le \phi(r_0)+\sqrt{\ss} e^{-\beta r_0}$ and $r_0 \ge c$ for some constant $c > 0$. Note that for such choice of $r_0$ we have $\coth(r_0)\leq\coth(c)=O(1)$ and thus within time $O(1)$, with probability bounded away from zero, the radial coordinate changes by at most $1$. Conditioning on this, we consider the angular movement variance $1$ Brownian motion $B_{\II{\ss}}$ where now $\II{\ss}:=\int_0^\ss\cosech^2(\beta r_{\ss})d\ss\geq (\sqrt{\ss}/\sinh(\beta(r_0+1))^2$. As in the proof of Theorem~\ref{thm:mainangular}, define the exit time $H_{[-a,b]}$ from the interval $[-a,b]$ where
 $a:=\phi(r_0+1)-|\theta_0|$ and $b:=2\pi-\phi(r_0+1)-|\theta_0|$, and as in~\eqref{eqn:angular-exitt} in the proof of Theorem~\ref{thm:mainangular},
$$
\mathbb{P}_{x_0}(T_{det} \le \ss) \ge \mathbb{P}(H_{[-a,b]} \le \II{\ss})=\Omega\big(\Phi\big((\phi(r_0+1)-|\theta_0|)\sinh(\beta (r_0+1))/\sqrt{\ss}\big)\big).
$$
Since $\phi(\cdot)>0$ and $\sinh(x)\leq e^{x}$ we conclude that $\PP_{x_0}(T_{det}\le \ss)
  =\Omega(\Phi(-|\theta_0|e^{\beta (r_0+1)}/\sqrt{\ss}))$.
Integrating over the elements of $\dt$ satisfying $|\theta_0| \le \sqrt{\ss} e^{-\beta r_0}$, 
%\dmc{here I can afford on purpose not to put $\phi(r_0)$,} 
we have
$$
\int_{\dt} \mathbb{P}_{x_0}(T_{det} \le \ss) d\mu(x_0) =
\Omega(ne^{-\alpha R})\int_{1}^R \int_0^{\sqrt{\ss} e^{-\beta \widehat{r}_0}} 
\Phi(-\theta_0 e^{\beta \widehat{r}_0}/\sqrt{\ss})\sinh(\alpha \widehat{r}_0) d\theta_0 d\widehat{r}_0. 
$$
Performing the change of variables $y_0:=\theta_0 e^{\beta \widehat{r}_0}/\sqrt{\ss}$, since $\alpha>\beta$, and using $\sinh(x)=\Theta(e^x)$ for $x=\Omega(1)$, we get for the latter
$$
\Omega(ne^{-\alpha R})\int_0^{1} \Phi(-y_0)\sqrt{\ss}dy_0\int_{1}^R e^{(\alpha-\beta)\widehat{r}_0} d\widehat{r}_0=\Omega(ne^{-\beta R}\sqrt{\ss})\int_0^{1} \Phi(-y_0) dy_0.
$$
Note that the last integral is $\Omega(1)$, and thus the stated result follows.
%\cmk{Where is the hypothesis $\ss=\Omega(n^{4\beta-2})$ used?}\dmc{It is not needed in the proof, but I put it for the statement: if $\ss$ is right of this order, the conclusion gives a constant right hand size. If $\ss$ is even smaller  }
\end{proof}


\subsubsection{The case $\ss$ large}
We now prove the lower bound of Theorem~\ref{thm:mixedLarge}. In fact, we will only need to assume the milder condition $\ss \ge Ce^{2\KA^2}$ with $C$ being a large enough constant. %\cmk{Is this phrase really needed here? Why?}\dmc{Ok now?}. 
We start by showing that particles that start close to the origin detect quickly with at least constant probability:
\begin{lemma}\label{lem:closetoorigin}
Let $\tau > 0$, and let $c^*:=C^*/\beta$ for some arbitrarily large constant $C^* > 0$. Then,
$$
 \inf_{x_0\in B_O(c^*)}\PP_{x_0}(T_{det}\leq \tau)=\Omega(1).
$$
\end{lemma}
\begin{proof}
Let $T_{h}$ be the smallest time $t$ such that $r_t=2c^*$ and notice that, calling $\PP^r$ the law of the radial component of the process, for any point $x_0=(r_0, \theta_0)$ with~$r_0 \le c^*$, it holds that
$\PP^r_{r_0}(T_{h}\ge \tau)\,\geq\,\PP^r_{c^*}(T_{h} \ge \tau).$
Now, fix any realization $\{r_s\}_{s\geq 0}$ of the radial component of the process such that~$T_h\ge\tau $ and observe that for any such realization, detection is guaranteed if $|\theta_s-\theta_0|>2\pi$ for some $0 \le s \le \tau$. Under the event $T_{h}\ge\tau$ the radial coordinate at any time $s \le \tau$ is at most $2c^*$. Thus, the angular movement $\theta_\tau-\theta_0$ has a normal distribution centered at zero with variance at least
\[\int_0^{\tau}\sinh^{-2}(\beta r_s)ds\,\geq\, \tau\sinh^{-2}(2C^*)  =\Omega(1).\]
Thus, with constant probability, within time~$\tau$ the angular movement covers an angle of $2\pi$. The lemma follows. 
\end{proof}
%We may and will thus in the following always assume that $r_0 > c^*$ for $c^*$ as in the statement of Lemma~\ref{lem:closetoorigin}. 


We now deal with the remaining starting points $x_0\in\dt$.
Before doing so we establish a simple fact concerning the random variables defined for nonnegative integer values of $k$ as follows:
\begin{equation}\label{mixedLower:def:Ik}
I_k := \int_0^\ss {\bf 1}_{(0,k]}(r_s)ds.
\end{equation}
Recall that $\pi(r):=\frac{\alpha\sinh(\alpha r)}{\cosh(\alpha R)-1}$, $0\leq r\leq R$, is the stationary distribution of the process~$\{r_s\}_{s\geq 0}$.
\begin{fact}\label{fact:stationary}
If $k\geq\frac{1}{\alpha}\log 2$, 
then $\EE_{\pi}(I_k) \ge \tfrac14\ss e^{-\alpha (R-k)}$.
\end{fact}
\begin{proof}
Suppose $r_0$  is distributed according to the stationary distribution $\pi$. 
Since $\cosh(x)-1\leq \frac12 e^x$, by definition of $I_k$, we have
\[
\EE_{\pi}(I_k) = \int_0^{\ss}\PP_{\pi}(r_s\leq k)ds =\ss\pi((0,k]) = \ss\cdot\frac{\cosh(\alpha k)-1}{\cosh(\alpha R)-1}
\geq 2\ss e^{-\alpha R}(\cosh(\alpha k)-1).
\]
The desired conclusion follows observing that the map $x\mapsto 1+e^{-2x}-2e^{-x}$ is non-decreasing, and thus for $x\geq\log 2$, we have 
$\cosh(x)-1=\frac12 e^x(1+e^{-2x}-2e^{-x})\geq\frac14 e^{x}$.
\end{proof}

We are now ready to state and prove the proposition dealing with particles whose initial position are points satisfying the first condition in the definition of $\dt$:
\begin{proposition}\label{mixedLowerBoundschico}
Let $\KA > 1$. There is a sufficiently large constant $C>0$ such that if $\ss\ge Ce^{2\KA^2}$ and $\ss=O(\mathfrak{Z})$, then for any $x_0=(r_0, \theta_0)$ with $|\theta_0| \le \phi(R)+\KA \phis$, we have the following: 
\begin{enumerate}[(i)]

\item\label{mixedLowerBoundschico:imt1}  If $\alpha \ge 2\beta$, then $\displaystyle 
\PP_{x_0}(T_{det} \leq \ss)=\Omega(\tfrac{1}{\KA}e^{-\frac12\KA^2}) =\Omega(e^{-0.7\KA^2})$. %\cmk{I don't understand, what issue?}\dmc{I mean we don't have matching upper and lower bounds, let's ignore this. Ok, I commented out the asymptotics with respect to $\KR$ and put $\KA$ only. Also merged cases (ii) and (iii) from before according to your suggestion} 

\item\label{mixedLowerBoundschico:imt2} 
%If $\alpha < 2\beta$ and $\beta \ge 1/2$, then \dmc{I think when merging conditions, I forgot to update here. It should be just: "If $\alpha < 2\beta$ then $\PP_{x_0}(T_{det} \leq \ss) = \Omega(\KA^{-(2 \wedge \frac{1}{\beta})\alpha})$, right? we could also separate cases as in Proposition 32. Or merge them therein, as you wish. But it should be consistent}
If $\alpha < 2\beta$, then %\dmc{It was $\wedge$ before, but it should be $\vee. $, I changed it. Check} 
$\displaystyle
\PP_{x_0}(T_{det} \leq \ss) = \Omega(\KA^{-\alpha/(\beta\wedge\frac12)})$.
%$\displaystyle
%\PP_{x_0}(T_{det} \leq \ss) = %\Omega(\KA^{-2\alpha}).%=\Omega(\KR^{-2\alpha}).
%$
\end{enumerate}
\end{proposition}
To prove the just stated proposition we use the following standard fact several times, so we state it explicitly.
%\dmc{I say it is standard, but important: $\eta > 0$ is a uniform bound. It is Feller's lemma 7.1, commented still in the preliminaries. I think we need only $\kappa > 1$ (strictly bigger than $1$) to have this uniform bound.}\cmk{As far as I can tell, I think the bound is correct for $\kappa\geq 1$ even taking $\eta=\frac32$ provided you are willing to accept a multiplicative $\frac{1}{\sqrt{2\pi}}$ factor. Check www.johndcook.com/blog/norm-dist-bounds/}\dmc{We can put the exact lower bound given in your link of $1/(\sqrt{2\pi}) \frac{t}{t^2+1}e^{-t^2/2}$ for $\P(X > t)$. So where is $3/2$ in the exponent? In any case, it is not matching upper bounds...}\cmk{Just use that $1+t^2\leq e^{t^2}$ and you get a lowe bound of $\frac{t}{\sqrt{2\pi}e^{-\frac32 t^2}}$. You may them either assume $t\geq \sqrt{2\pi}$ and use the lower bound $e^{-\frac32 t^2}$ or use $t\geq 1$ and use the lower bound $\frac{1}{\sqrt{2\pi}}e^{-\frac32 t^2}$.}\dmc{Ok, adapted it below}
\begin{fact}[\cite{gordon41}]\label{mixedLower:fact:brownian}
Given the radial component $\{r_{s}\}_{s\geq 0}$ of a particle's trajectory,  the angular component $\{\theta_s\}_{s\geq 0}$ law is that of a Brownian motion indexed by $\II{\ss}:=\int_0^{\ss}\cosech^2(\beta r_s)ds$. 
If $\II{\ss} \ge \sigma^2 > 0$ and $\kappa>0$, then 
%\dmc{(should we just cite www.johndcook.com/blog/norm-dist-bounds/ ?}\cmk{Let me first try to find a more standard reference.}
\[
\PP(\sup_{0\leq s\leq \ss} B_{\II{s}}\geq \kappa \sigma \mid \{r_s\}_{s\geq 0}) \geq  \frac{\kappa}{\sqrt{2\pi}(\kappa^2+1)}e^{-\frac12\kappa^2}=\Omega\Big(\frac{1}{\kappa}e^{-\frac12\kappa^2}\Big).
\]
\end{fact}
Now we proceed with the pending proof.
\begin{proof}[Proof of Proposition~\ref{mixedLowerBoundschico}.]
Assume first $\beta \ge 1/2$
(and thus necessarily $\alpha < 2\beta$). In this case the radial movement dominates and the proof of~\eqref{mixedLowerBoundschico:imt2} is very similar to the one given for $x_0 \in \dt$ where $|\theta_0|\le \phi(R)+\kappa\phis$: assume first that $\ss$ (and $\KA$) is such that $|\theta_0|\le \phi(R)+\KA \phis \le \pi/2 - c$ for some $c > 0$. Define $\overline{r}_0$ be as $\rabs_0$ in the proof of Proposition~\ref{prop:rad-upperBnd} (that is $\overline{r}_0$ corresponds to the absorption radius in case there was radial movement only; since $\phi(R)+\KA \phis \le \pi/2 - c$, we have $\overline{r}_0=\Omega(1)$). By the same argument as given in the proof of Proposition~\ref{prop:rad-upperBnd}, with $\KA$ playing the role of $\kappa$ therein, with probability $\Omega(\KA^{-2\alpha})$, there exists a time moment $T \le \ss$, at which a radial value of $\overline{r}_0$ is reached. In this case, by symmetry of the angular movement, with probability at least $1/2$, either the angle at time $T$ satisfies $|\theta_T| \le |\theta_0|$ or there exists $t \le T$ where the angle $\theta_t=0$ and we detected already by time $t$, and hence in this case $Q$ is detected by time $T$ with probability $\Omega(\KA^{-2\alpha})$. Assume then that~$\ss$ (and $\KA$) is such that $\pi/2 - c \le \phi(R)+\KA \phis$. In this case, let $\overline{r}_0$ be equal  (assuming there were radial movement only) to the absorption radius $\rabs_0$ that would correspond to an angle exactly $|\theta_0|=\pi/2 -c$. Note that in this case $\overline{r}_0=\Theta(1)$. By the proof of Proposition~\ref{prop:rad-upperBnd}, with probability $\Omega(\KA^{-2\alpha})$ a radial value of $\overline{r}_0$ is reached at a time moment  $T \le \frac12\ss$. In this case, with probability at least $1/2$, as before, either there existed a moment $t$ with $\theta_t=0$, and we would have detected $Q$, or $|\theta_T|\le|\theta_0|$. In this case, by Lemma~\ref{lem:closetoorigin}, since $\overline{r}_0=\Theta(1)$, with constant probability we detect $Q$ in time $\frac12\ss=\Omega(1)$, and hence also in this case $Q$ is detected by time $\ss$ with probability $\Omega(\KA^{-2\alpha})$.


Consider thus $x_0 \in \dt$ with $|\theta_0| \le \phi(R)+\KA\phi^{(\ss)}$ under the assumption $\beta < 1/2$. Recall that we also may assume that $r_0 \ge c^*$ for $c^*$ as in the statement of Lemma~\ref{lem:closetoorigin} since other values of $r_0$ were already dealt with in said lemma.
First, note that for $x_0 \in B_Q(R)$ the lower bound trivially holds. 
So, henceforth let $x_0 \in \dt \setminus B_Q(R)$. Since by hypothesis $\ss\in O(\mathfrak{Z})=O(e^{\alpha R})$, we may assume that $\ss < \exp(\alpha(R-\frac{1}{\alpha}\log 2 - 1+c))$ for $c>0$ sufficiently large. %By hypothesis, taking $C$ sufficiently large, \dm{with room to spare}, we may also assume that $\ss \ge e^{\alpha c}$.
Let $\rho:=R-\frac{1}{\alpha}\log\ss+c$ and note that $\frac{1}{\alpha}\log 2+1 < \rho$ by the previous assumption on $\ss$. 
%\cmk{Also, you did not answer my question regarding where do we use the upper bound on $\rho$.}\dmc{I think you're right about this, I removed it. We use  below only that $g(\overline{\rho} $ is always bounded away from $R$, so I think no need for this, indeed}.
%\dmc{Changes from here on, since change in the strategy}  
Denote by $T_{\rho-1}$ the random variable corresponding to the first time the process reaches $B_O(\rho-1)$. Recall from Part~\eqref{radial:itm:phi2} of Lemma~\ref{lemmaradial}, with $\yabs_0=\rho-1$ and $\auxY=R$, that for the radial component $\{r_s\}_{s\geq 0}$ of the movement, and for any $r_0 \in [\rho-1, R]$,
	\[\EE_{r_0}(T_{\rho-1})\leq\frac{4}{\alpha^2}e^{\alpha(R-\rho+1)}
	=\frac{4}{\alpha^2} \ss e^{-\alpha(c-1)}.
	\]
	By Markov's inequality, for $c$ sufficiently large so that  $\frac{4}{\alpha^2}e^{-\alpha(c-1)} \le \frac14$, it follows that the event $\mathcal{A}$ corresponding to the process reaching $B_O(\rho-1)$ 
	before time $\frac12\ss$ happens with
	probability
	\[\PP(\mathcal{A})=1-\PP_{r_0}(T_{\rho-1}\geq \tfrac12 \ss) \ge \tfrac{1}{2}.\]
	Now, let $\overline{\rho}:=\max\{\rho-\beta \log \KA, c^*\}.$ Moreover, define $\mathcal{B}$ as the event that starting from radius $\rho-1$ we hit the radial value $\overline{\rho}$ before hitting the radial value $R$.
	Define $g(r):=\log(\tanh(\frac12\alpha R)/\tanh(\frac12\alpha r))$ as in Fact~\ref{fct:radial-varphi2} and observe that as argued therein,
	since $\rho=\Omega(1)$ we have $g(\rho-1)=O(e^{-\alpha \rho})$ and because
	$\overline{\rho}=R-\Omega(1)$ we have $g(\overline{\rho})=\Omega(e^{-\alpha \overline{\rho}})$. Recall from Part~\eqref{radial:itm:phi3} of Lemma~\ref{lemmaradial} that $g(\rho-1)/g(\overline{\rho})$ is the probability that starting from radius $\rho-1$ we hit the radial value $\overline{\rho}$ before hitting the radial value $R$, and we have
$$
\PP(\mathcal{B})=\frac{g(\rho-1)}{g(\overline{\rho})}=\Omega(e^{\alpha(\overline{\rho}-\rho)})=
%e^{\alpha(\max\{\rho-\frac{1}{\beta}\log \KA, c^*\}-\rho)}=
\Omega(\KA^{-\alpha/\beta}).%=\Omega(\KR^{-\alpha/\beta}).
$$
 From Part~\eqref{radial:itm:phi4} of Lemma~\ref{lemmaradial} we obtain 
$$
\EE_{\rho-1}(T_{\overline{\rho}} \mid T_{\overline{\rho}} < T_R) \le \frac{2}{\alpha}(\beta \log \KA+1)+\frac{2}{\alpha^2} \le \frac{\ss}{8},
$$
where the last inequality follows (with room to spare) by our assumption of $\ss$.
Thus, by Markov's inequality, conditionally under $\mathcal A \cap \mathcal B$,
%\cmk{Well, my point is that event $\mathcal{A}$ ocurring tells us we have reached $\rho-1$, it does not say we have reached $\rho$.}\dmc{Good catch, I adapted it}
with probability at least $\frac12$, we reach $\overline{\rho}$ before time $\frac34\ss$. Let $\mathcal C$ be the corresponding event. Conditional under $\mathcal C$, with constant probability the particle stays one unit of time inside $B_O(\overline{\rho}+1)$ before time $\ss$, and let this event be called $\mathcal D$. Conditional under $\mathcal D$, the angular component's law during one unit of time inside $B_O(\overline{\rho}+1)$ is $B_{\II{\ss}}$ with $$\II{\ss}\ge \cosech^2(\beta(\overline{\rho}+1)) \ge e^{-2\beta(\overline{\rho}+1)} \geq e^{-2\beta (R-\frac{1}{\alpha}\log\ss+c-\frac{1}{\beta}\log \KA +1)}=:\sigma^2.$$
Note that $\sigma = e^{-\beta R}\ss^{\frac{\beta}{\alpha}}\KA e^{-\beta(c+1)}$ for the absolute constant $c > 0$ independent of $\KA$ from above, and recall also that
 $|\theta_0| \le \phi(R)+\KA \ss^{\frac{\beta}{\alpha}}e^{-\beta R} \le 2\KA \ss^{\frac{\beta}{\alpha}}e^{-\beta R}$ by assumptions on $\beta < 1/2$ and $\ss$, with room to spare. 
Let $\mathcal{E}$ be the event that when reaching $B_O(\overline{\rho}+1)$ the angle at the origin spanned by the particle and $Q$ is (in absolute value) is at most $|\theta_0|$ or there was a moment $t$ before reaching $B_O(\overline{\rho}+1)$ with $\theta_t=0$ (and thus we detected at time $t$). Note that by symmetry of the angular movement, $\PP(\mathcal{E})\ge \frac12$.
 Conditional under $\mathcal{D} \cap \mathcal{E}$, by Fact~\ref{mixedLower:fact:brownian},  since $\KA > 1$, for some $c_1=c_1(c)$ we have
\[
\P_{x_0}(T_{det}\leq\ss)\geq \PP(\sup_{0\leq s\leq \ss} B_{\II{s}}\geq c_1 \sigma \mid \{r_s\}_{s\geq 0}) =\Omega(1),
\]
 and since $\mathcal{D} \cap \mathcal{E}$ holds with probability $\Omega(\KA^{-\alpha/\beta})$, we have
 $
\P_{x_0}(T_{det}\leq\ss)= \Omega(\KA^{-\alpha/\beta})$,
which finishes the proof of the case $\alpha < 2\beta$ and $\beta < 1/2$ (and thus the proof of~\eqref{mixedLowerBoundschico:imt2}).


 We now deal with the remaining $\alpha \ge 2\beta$ case (and therefore necessarily $\beta < 1/2$). Assume first $\alpha > 2\beta$. This argument is analogous to the one for angular movement: we repeat it for convenience. 
%Since $|\theta_0|\le \kappa e^{-\beta R}\sqrt{\ss}+\phi(r_0)$, 
Given the trajectory of $\{r_{s}\}_{s\geq 0}$, recall that the angular component's law is that of a Brownian motion $B_{\II{\ss}}$, where $\II{\ss}:=\int_0^{\ss} \cosech^2(\beta r_s)ds \ge 4\ss e^{-2\beta R} =: \sigma^2$.
Hence, using Fact~\ref{mixedLower:fact:brownian}, since $\KA > 1$ and using that $0.2x^2\ge \log x$ for all $x\ge 1$,
%\[
%\PP(\sup_{0\leq s\leq \ss} B_{\II{s}}\geq \KA \sigma \mid \{r_s\}_{s\geq 0}) \geq  %e^{-\eta \KA^2},
%\]
%and so
$$
\P_{x_0}(T_{det}\leq\ss)\geq \PP(\sup_{0\leq s\leq \ss} B_{\II{s}}\geq \KA \sigma \mid \{r_s\}_{s\geq 0}) =\Omega\big(-\tfrac{1}{\KA}e^{-\frac12\KA^2}\big)={\Omega(e^{-0.7\KA^2})}, 
$$
showing the result in the case $\alpha > 2\beta$.


Finally, suppose $\alpha=2\beta$ (therefore $\beta > \frac14$).
First, suppose that the starting point $r_0$ of the radial component is distributed according to the stationary distribution of the process, that is, $\pi(r):=\frac{\alpha\sinh(\alpha r)}{\cosh(\alpha R)-1}$ with $0\leq r\leq R$. 
%\dmc{At the end of the proof I remove this assumption, it was missing before}
We will see that the contribution to  $\II{\ss}:=\int_0^{\ss} \cosech^2(\beta r_s) ds$ of the time spent around different radial values is roughly the same, forcing a logarithmic correction. To bound $\II{\ss}$ from below, let $\overline{k}=R-\lfloor\tfrac{1}{\alpha}\log (\ss/4) \rfloor$, which by our hypothesis $\ss= O(\mathfrak{Z})=O(e^{\alpha R}/(\alpha R))$ implies $\overline{k}=\omega(1)$ (in particular it implies also $\overline{k} \ge \frac{1}{\alpha}\log 2$). 
Note that 
\[
\II{\ss} = \int_0^{\overline{k}}\cosech^2(\beta r_s)ds
  + \!\sum_{k=\overline{k}+1}^R \int_{k-1}^{k}\cosech^2(\beta r_s)ds
      \geq 4\Big(I_{\overline{k}}e^{-2\beta \overline{k}}+\!\sum_{k=\overline{k}+1}^R (I_k - I_{k-1})e^{-2\beta k}\Big).
\]
Hence, using that $\beta\geq\frac14$ implies that $4(1-e^{-2\beta})\geq 1$,
\begin{equation}%\label{eqn:hlowbound}
    \II{\ss} 
    \geq 4\Big(\sum_{k=\overline{k}}^{R-1}I_ke^{-2\beta k}(1-e^{-2\beta}) +I_R e^{-2\beta R} \Big)
    \ge \sum_{k=\overline{k}}^R e^{-2\beta k}I_{k}.
\end{equation}
 So, recalling that $\overline{k}\ge \frac{1}{\alpha}\log 2$, by Fact~\ref{fact:stationary}, for all $k \in \{\overline{k},\ldots,R\}$, we have $\EE_{\pi}(I_k)\geq \frac14 \ss e^{-\alpha(R-k)}$,
which gives an estimate for the value of the $I_k$ variables. Moreover, by Corollary~\ref{radial:cor:coupling}, %\cmk{This $C$ is related to the $C$ in the statement of Propostion 31, but it is not used/mentioned in what follows. The $C$ plays a role in the necessary hypothesis for applying Proposition 31, that is, $\ss\geq Ce^{\alpha(\auxY-k)}$. In fact, shouldn't we justify why this condition is satisfied?}\dmc{Indeed. I try. I don't want to put the $C$ of the proposition, just the $\widetilde{C}$, this is enough, I think. In that proposition the $C$ there in the statement is redundant I think, need to check again, but plan to remove it} 
since $k \ge \overline{k}$ and by definition of $\overline{k}$ we have $\ss \ge 4e^{\alpha (R-\overline{k})} \ge 4e^{\alpha (R-k)}$. Since $\cosh(x)-1=\frac12 e^x(1-e^{-x})^2\le \frac12 e^x$, using once more that $k\ge\overline{k}\ge\frac{1}{\alpha}\log 2$, we obtain %\dmc{if you agree above it is $e^{-\alpha(R-k)}(1-e^{-\alpha k})^2$, and $\frac14$ is enough instead of $\frac{1}{12}$, I think} 
$\pi((0,k])\ge e^{-\alpha(R-k)}(1-e^{-\frac12\alpha k})^2\ge \frac{1}{4}e^{-\alpha(R-k)}$, so $\ss\ge 1/\pi((0,k])$ and the assumptions of Corollary~\ref{radial:cor:coupling} are satisfied. Hence, there exist $\widetilde{c}, \widetilde{\eta} \in (0,1)$, such that for each $\overline{k} \le k \le R$, we have that the expectation of the indicator $Z_k$ of the event $\{I_k\geq \widetilde{c}\ss e^{-\alpha (R-k)}\}$ is at least~$\widetilde{\eta}$. Let $Z:=\sum_{k=\overline{k}}^R Z_k$ and that $\EE(Z) \ge (R-\overline{k}+1)\widetilde{\eta}$. Define then the event 
$\mathcal{E}:=\{Z>\frac{\widetilde{\eta}}{2}(R-\overline{k}+1)\}$. Since $\EE(Z)\le (\PP(\mathcal{E})+\frac{\widetilde{\eta}}{2}\PP(\overline{\mathcal{E}}))(R-\overline{k}+1)$,
it must be the case that $\PP(\mathcal{E})\ge\frac{\widetilde{\eta}}{2}/(1-\frac{\widetilde{\eta}}{2})\ge\frac{\widetilde{\eta}}{2}$.
Thus, for a fixed realization of $\{r_s\}_{s\geq 0}$ satisfying $\mathcal{E}$, since $\alpha=2\beta$, we have 
\[
%\frac{1}{4(1-e^{-2\beta})}\II{\ss}
\sum_{k=\overline{k}}^{R}e^{-2\beta k}I_k 
\geq\sum_{k=\overline{k}}^{R}e^{-2\beta k}\widetilde{c}\ss e^{-\alpha (R-k)}Z_k
= \widetilde{c}\ss e^{-2\beta R}\sum_{k=\overline{k}}^R Z_k 
> \widetilde{c}\ss e^{-2\beta R}\frac{\widetilde{\eta}}{2}(R-\overline{k}+1).
\]
Note that  $R-\overline{k}+1\ge \frac{1}{\alpha}\log (\ss/4)$ by definition of $\overline{k}$, so 
$
\sum_{k=\overline{k}}^R e^{-2\beta k}I_k\geq \frac{\widetilde{c}}{2\alpha}\widetilde{\eta}\ss e^{-2\beta R}\log(\ss/4)) \ge \frac{\widetilde{c}}{3\alpha}\widetilde{\eta}\ss e^{-2\beta R}\log\ss,
$
where we used the assumption that $\ss$ is at least a sufficiently large constant.
Thus, under $\mathcal{E}$ the angular movement dominates stochastically
a Brownian motion $B_{\frac{\widetilde{c}}{3\alpha}\widetilde{\eta}\ss e^{-2\beta R}\log\ss}$ with standard deviation 
$e^{-\beta R}\sqrt{(\widetilde{c}\ss \widetilde{\eta}/(3\alpha))\log\ss}=:\sigma$. By Fact~\ref{mixedLower:fact:brownian}, %\dmc{Should we put $3/2$ instead of $c$?} 
\[
\PP(\sup_{0\leq s\leq \ss} B_{\II{s}}\geq \KA \sigma \mid \{r_s\}_{s\geq 0})=\Omega(\tfrac{1}{\KA}e^{-\frac12\KA^2}).
\]
Thus, for $x_0 \in \dt$ with $|\theta_0| \le \phi(R)+\KA \phis\le 2\KA e^{-\beta R}\sqrt{\ss \log \ss}$, since $\P_{x_0}(\mathcal{E})\geq \frac{\widetilde{\eta}}{2}$ and using that $0.2x^2\ge \log x$ for all $x\ge 1$,
\begin{equation}\label{startstationary}
\P_{x_0}(T_{det}\le \ss)\geq \tfrac{\widetilde{\eta}}{2}\P_{x_0}(T_{det} \le \ss \mid \mathcal{E})=\Omega(\tfrac{1}{\KA}e^{-\frac12\KA^2})=\Omega(e^{-0.7\KA^2}).%\dm{=\KR^{-c}}.%\Omega(\Phi(-\kappa)).
\end{equation}

It remains to show thus that with positive probability the trajectory starting with a fixed initial radius for $x_0 \in \dt$ can be coupled in such a way that the probability of detection of the target by time $\ss$ can be bounded from below by the probability of detection when the radius is chosen according to the stationary distribution $\pi(\cdot)$. Denote by $\widehat{r}_t$ the radial component at time $t$ when starting according to the stationary distribution.
Consider the event $\mathcal{A}$ that the initial radial value $\widehat{r}_0$ is at most one unit away from the 
boundary of $B_O(R)$, that is, $\mathcal{A}:=\{\widehat{r}_0\in [R-1,R]\}$.
Clearly, $\PP(\mathcal{A})=\pi([R-1,R])=\Omega(1)$. 
Let $\mathcal{B}$ be the event that $\{\widehat{r}_s\}_{s\geq 0}$ starting from $R-1$ hits $R$ by time $\frac12\ss$. 
Conditional under $\mathcal{A}$, since the time to hit $R$ is clearly dominated by the time a standard  Brownian motion (corresponding to a one-dimensional radial movement) hits $R$ starting from $R-1$, by our lower bound on $\ss$, we have that $\PP(\mathcal{B}\mid \mathcal{A})=\Omega(1)$.  Note that conditional under $\mathcal{A}\cap\mathcal{B}$ either the trajectories starting with a fixed value of $r_0$ on the one hand and with $\widehat{r}_0$ according to the stationary distribution $\pi(\cdot)$ on the other hand must cross by time $\frac12\ss$ (and they can be coupled naturally from then on for a time interval of $\frac12\ss$), or $r_t \le \widehat{r}_t$ for any $0 \le t \le \frac12\ss$, and during an initial time period of length $\frac12\ss$ the detection probability starting from $r_0$ stochastically dominates the one when starting from $\widehat{r}_0$. Thus, with probability $\PP(\mathcal{A}\cap\mathcal{B})=\Omega(1)$, the process $\{r_s\}_{s\geq 0}$ can be successfully coupled with $\{\widehat{r}_s\}_{s\geq 0}$, and since we aim for a lower bound, we assume that $\mathcal{A}\cap\mathcal{B}$ holds. Conditional under $\mathcal{A}\cap\mathcal{B}$, we may thus apply the reasoning yielding~\eqref{startstationary} with $\ss$ replaced by $\frac12\ss$, and the result follows.
%For the second bound, note that by definition of $\dt$, the set of $x_0$ with $|\theta_0| \le \kappa \phis\added[id=mk]{+\phi(r_0)}$ is contained in $\dt$ (which is a sector of $B_O(R)$, and the corollary follows by combining Proposition~\ref{uniformlowerbound} together with angular symmetry of the distribution of the particles.
\end{proof}



%\dmc{This was originally before, now it is after the first condition}
We still have to deal with particles whose initial location are points satisfying the second condition in the definition of $\dt$. The next proposition does this.
\begin{proposition}\label{prop:uniformlowerboundN2}
Let $\KR >1$ and assume that $\ss\ge Ce^{2\KA^2}$ for $C$ large enough.
%\dmc{This assumption is enough for all cases, but not needed in some, of course}. 
Then, for any $x_0=(r_0,\theta_0)$ with 
$(|\theta_0|-\phi(r_0))e^{(\beta\wedge\frac12)r_0} \le \KR$, we have the following: 
%\[\PP_{x_0}(T_{det}\leq \ss)= \Omega(\KR^{- (\alpha/\beta \wedge  2\alpha)})\]
\begin{enumerate}[(i)]

\item\label{prop:uniformLowerboundN2:itm1} If $\alpha \ge 2\beta$, then 
$\displaystyle 
\PP_{x_0}(T_{det} \leq \ss)=\Omega(\KR^{-\frac{\alpha}{\beta}})
=\Omega(e^{-{\frac{\alpha}{\beta}\KA^2}}).
$
\item\label{prop:uniformLowerboundN2:itm2} If $\alpha < 2\beta$, % and $\beta \ge 1/2$,
then  
$\displaystyle
\PP_{x_0}(T_{det} \leq \ss) = \Omega(\KR^{-\alpha/(\beta\wedge\frac12)})=\Omega(\KA^{-\alpha/(\beta\wedge\frac12)}).
$
%\item\label{prop:uniformLowerboundN2:itm3} If $\alpha < 2 \beta$ and $\beta < 1/2$, then 
%$\displaystyle
%\PP_{x_0}(T_{det} \leq \ss) =\Omega(\KR^{-\alpha/\beta})=\Omega(\KA^{-\alpha/\beta}).
%$
\end{enumerate}
\end{proposition}
\begin{proof}
Thanks to Lemma~\ref{lem:closetoorigin} we may, and will, assume throughout the proof that
$r_0>c^*$ for an arbitrarily large $c^*$. Under said condition, the proof argument formalizes the following intuitive fact: Either there is a chance for the particle to move radially towards the origin so that the boundary of $B_Q(R)$ is reached and detection happens, or there is a chance to move radially towards the origin, enough so that the particle stays long enough in such a region close enough to the origin so that the relatively large angle the particle traverses makes detection probable.

We first assume $\beta<1/2$.
%address the proof of~\eqref{prop:uniformLowerboundN2:itm1} and~\eqref{prop:uniformLowerboundN2:itm3} together. Note that $\alpha \ge 2\beta$ implies $\beta < 1/2$, and so in both cases we have $\beta < 1/2$.  
Observe that we may assume $|\theta_0|\ge 2\phi(r_0)$, since otherwise during one time unit there is constant probability that the radial value is at most $\min\{R,r_0+1\}$, and conditional under this event, during this time unit with constant probability an angular movement of standard deviation at least $e^{-\beta (r_0+1)}$ is performed, thus covering with constant probability an angular distance of $e^{-\beta(r_0+1)} \ge 2\phi(r_0)$ in the case of $\beta < 1/2$ for $r_0 > c^*$. We may and will thus below replace the condition $(|\theta_0|-\phi(r_0))e^{(\beta\wedge\frac12)r_0} \le \KR$ by $|\theta_0|e^{(\beta\wedge\frac12)r_0} \le 2\KR$. We consider the worst-case scenario, i.e., $|\theta_0|=2\KR e^{-\beta r_0}$, or equivalently $r_0=\frac{1}{\beta}\log(2\KR/|\theta_0|)$. %\dmc{I think contrary to  Prop 26, here this special treatment is needed sinc we go for a lower bound $g(r_0)/g(\overline{r}_0)$ below} 
We restrict our discussion to vertices with $r_0 < R-\log \KA$, as other vertices were already considered before: indeed, for values of $r_0 \ge R-\log \KA$, in case~\eqref{prop:uniformLowerboundN2:itm1}
we have $2\KR e^{-\beta r_0} = 2e^{\KA^2}e^{-\beta r_0} \le e^{\KA^2}e^{-\beta R}\KA \le \KA \sqrt{\ss}e^{-\beta R}=\KA \phis$, where the second inequality follows from our assumption of $\ss\ge Ce^{2\KA^2}$ with $C$ large. Hence such values of $x_0=(\theta_0, r_0)$ satisfy the first condition of $\dt$ and were treated in Proposition~\ref{mixedLowerBoundschico}. In case~\eqref{prop:uniformLowerboundN2:itm2}, for values of $r_0 \ge R-\log \KA$, we have $2\KR e^{-\beta r_0}=2\KA e^{-\beta r_0}\le 2\KA^2 e^{-\beta R}\le \KA \phis$, where again the second inequality follows from the same assumption on $\ss \ge Ce^{2\KA^2}$ (again with room to spare), and again this case was already dealt with in Proposition~\ref{mixedLowerBoundschico}.
Define $g(r):=\log(\tanh(\frac12 \alpha R)/\tanh(\frac12 \alpha r))$ as in Fact~\ref{fct:radial-varphi2} and let $\overline{r}_0:=\max\{r_0-\frac{1}{\beta}\log \KR, c^*\}$. By arguments given in said fact, since we have restricted our discussion to the case where $r_0=R-\Omega(1)$, we have $g(r_0)=\Omega(e^{-\alpha r_0})$ and, since $\overline{r}_0\geq c^*$ with $c^*$ large, we also have $g(\overline{r}_0)=O(e^{-\alpha \overline{r}_0})$. %\dmc{Please check that the preceding is ok - I hope so. if not, I have to change a bit the preceding paragraph and exclude other values of $r_0$ already}
Recall from Part~\eqref{radial:itm:phi3} of Lemma~\ref{lemmaradial} that $g(r_0)/g(\overline{r}_0)$ is the probability that starting from radius $r_0$ we hit the radial value $\overline{r}_0$ before hitting the radial value $R$, and we have
$$
\frac{g(r_0)}{g(\overline{r}_0)}=\Omega(e^{\alpha(\overline{r}_0-r_0)})=
\Omega(e^{\alpha(\max\{r_0-\frac{1}{\beta}\log \KR, c^*\}-r_0)})=\Omega(\KR^{-\alpha/\beta}).
%=\Omega(e^{-\dm{\eta}\KA^2})
$$
Thus, by Part~\eqref{radial:itm:phi4} of Lemma~\ref{lemmaradial} with $\auxy_0:=\overline{r}_0$ and $\auxY:=R$, we have that $\EE_{r_0} (T_{\overline{r}_0} \mid T_{\overline{r}_0} < T_R) \le \frac{2}{\alpha\beta}\log \KR + \frac{2}{\alpha^2} \le \frac14\ss$ by our lower bound hypothesis of $\ss$ with $C:=C(\beta)$ sufficiently large. 
%\dmc{For now $C$ depends on $\beta$, if we don't want this, we have to put $\beta \ge \varepsilon' > 0$ as assumption.}
By Markov's inequality,  conditional under $T_{\overline{r}_0} < T_R$, starting at radius $r_0$, with probability at least $1/2$, the value of $\overline{r}_0$ is hit by time $\frac12\ss$. In this case, if $\overline{r}_0=c^*$, then by Lemma~\ref{lem:closetoorigin} the target is detected with constant probability in constant time, and we are done. If $\overline{r}_0 =r_0-\frac{1}{\beta}\log \KR > c^*$, then with constant probability, during the ensuing unit time interval, the radial coordinate $r$ is always at most $\overline{r}_0+1$. Call this event $\mathcal{A}$. By symmetry of the angular movement, with probability at least $1/2$, either the angle $\theta$ at the hitting time of~$\overline{r}_0$ is in absolute value at most $|\theta_0|$ or there was a time moment $t$ with $\theta_t=0$ before, and we had detected $Q$ by time $t$ already. Call this event~$\mathcal{B}$. Conditional under $\mathcal{A} \cap \mathcal{B}$, the variance of the angular movement after reaching $\overline{r}_0$ during the following unit time interval is at least 
$
e^{-2\beta (r_0-\frac{1}{\beta}\log\KR+1)} =e^{-2\beta (r_0+1)}\KR^2$,
so recalling that by our worst-case assumption $|\theta_0|=2\KR e^{-\beta r_0}$,  the standard deviation is thus at least $e^{-\beta (r_0+1)}\KR=e^{-\beta}|\theta_0|/2$. Hence, with constant probability, independently of $\KR$ (and also independently of $\beta$), in this case an angle of $|\theta_0|$ is covered during a unit time interval, and by our lower bound assumption of $\ss$, the particle is detected by time $\ss$, finishing also this case and %both the proof 
%of Part~\eqref{prop:uniformLowerboundN2:itm1} and Part~\eqref{prop:uniformLowerboundN2:itm3} 
establishing the proposition when $\beta<1/2$ 
by plugging in the respective relations between $\KA$ and $\KR$ in order to obtain the last equality of the the two parts of the statement.

Next, we consider Part~\eqref{prop:uniformLowerboundN2:itm2} which encompasses a case similar to the one analyzed in Section~\ref{sec:radial}, so we give only a short sketch of how it is handled. By hypothesis and case assumption $|\theta_0| \le \phi(r_0)+\KR e^{-\frac12r_0}$. Since $\KR >1$, we thus have $|\theta_0|\le 3\KR e^{-\frac12r_0}$. Again, it suffices to show the statement for the worst-case scenario, that is, we may assume $|\theta_0|= 3\KR e^{-\frac12r_0}$. We may assume that $r_0 < R- \KR$, since for vertices with $r_0 \ge R-\KR$, using our assumption of $\ss\ge Ce^{2\KA^2}$ (with room to spare) we have
$3\KR e^{-\frac12r_0} \le 3\KR e^{\KR} e^{-\frac12 R} \le \KA \ss^{1/(2\alpha) e^{-\frac12 R}}=\KA \phis$, and such cases were already dealt with in the proof of Part~\eqref{mixedLowerBoundschico:imt2}  of Proposition~\ref{mixedLowerBoundschico}. By Lemma~\ref{lem:closetoorigin} we may also assume $r_0 > c^*$ for arbitrarily large $c^*$.
Define $\overline{r}_0:=\max\{r_0-2 \log \KR-2, c^{*}\}$. 
As before, recall from Part~\eqref{radial:itm:phi3} of Lemma~\ref{lemmaradial} that $g(r_0)/g(\overline{r}_0)$ is the probability that starting from radius $r_0$ we hit the radial value $\overline{r}_0$ before hitting the radial value $R$. Since $r_0=R-\Omega(1)$ and since $\overline{r}_0 \ge c^*$, we have
$$
\frac{g(r_0)}{g(\overline{r}_0)}=\Omega(e^{\alpha(\overline{r}_0-r_0)})=
\Omega(e^{\alpha(\max\{r_0-2\log \KR-2, c^*\}-r_0)})=\Omega(\KR^{-2\alpha}),
%=\Omega(e^{-\dm{\eta}\KA^2})
$$
and let $\mathcal{A}$ be the event that this indeed happens. 
By Part~\eqref{radial:itm:phi4} of Lemma~\ref{lemmaradial} with $\auxy_0:=\overline{r}_0$ and $\auxY:=R$, we have that $\EE_{x_0}(T_{\overline{r}_0}\mid \mathcal{A}) \le \frac{4}{\alpha}\log \KR +\frac{4}{\alpha}+ \frac{2}{\alpha^2} \le \frac14\ss$ by our lower bound hypothesis on $\ss$. By Markov's inequality, with probability at least $1/2$, $T_{\overline{r}_0} < \frac12\ss$. Let $\mathcal{B}$ be the event to reach the radius $\overline{r}_0$ by time $\frac12\ss$.
We have
$$
\PP_{x_0}(\mathcal{B}) \ge \PP_{x_0}(\mathcal{B} \mid \mathcal{A})\PP_{x_0}(\mathcal{A}) =\Omega(\KR^{-2\alpha}).
$$

 Let $\mathcal{C}$ be the event that one of the following events occurs: Either at the moment $T$ when reaching $\overline{r}_0$ the angle made at the origin between the particle and $Q$ is in absolute value at most $|\theta_0|$, or there was a moment $t \le T$ where $\theta_t=0$ (and we already detected $Q$ by time $t$). Note that by symmetry of the angular movement, $\PP_{x_0}(\mathcal{C} \mid \mathcal{B})=\PP_{x_0}(\mathcal{C}) \ge 1/2$. Hence, $\PP_{x_0}(\mathcal{C} \cap \mathcal{B})=\Omega(\KR^{-2\alpha})$. To conclude observe first the following:
 If it is the case that $\overline{r}_0=c^*$, then conditional under $\mathcal{C} \cap \mathcal{B}$, by Lemma~\ref{lem:closetoorigin}, with constant probability, in time $O(1)$ the particle $Q$ is detected, and since $\frac12\ss+O(1) \le \ss$, we are done. In particular, if 
we had $|\theta_0| > \pi/2 - c$ for some sufficiently small $c > 0$, then, since $|\theta_0| \le 3\KR e^{-\beta r_0}$, we must have $r_0 < c^*$, and therefore clearly also $\overline{r}_0=c^*$. If on the other hand we had  $|\theta_0| \le \pi/2 - c$ for some arbitrarily small $c > 0$ and also $\overline{r}_0=r_0-2\log \KR$, then observe that since $\overline{r}_0 > c^*$ with $c^*$ large enough, we have
$\phi(\overline{r}_0) \ge e 1.99\KR e^{-r_0/2} \ge 3\KR e^{-r_0/2} \ge |\theta_0|$, and $Q$ is detected by time $T$, since $\phi(\overline{r}_0)$ corresponds to the maximum angle at the origin between a particle at radius $\overline{r}_0$ and $Q$ that guarantees detection. 
\end{proof}

The uniform lower bound of Theorem~\ref{thm:mixedLarge} now follows directly by combining Proposition~\ref{mixedLowerBoundschico} and Proposition~\ref{prop:uniformlowerboundN2}. The integral lower bound of Theorem~\ref{thm:mixedLarge} follows directly from Proposition~\ref{mixedLowerBoundschico} applied with $\KA$ being any fixed constant and the fact that $\mu(\dt)=\Omega(n\phis)$ (by considering the condition $|\theta_0| \le \phi(R)+\KA \phis$ only), and the proofs of the lower bounds of Theorem~\ref{thm:mixedSmall} and Theorem~\ref{thm:mixedLarge} are finished. %\dmc{Marcos, I would leave out the remaining from here on}
%We conclude the section stating the following corollary which shows that the contribution of the measure of $\dt$ that comes from points satisfying the first condition in the definition of $\dt$ is asymptotically already of the same order as the matching upper bound shown below in Section~\ref{sec:Upper}, \dm{as we claimed in Theorem~\ref{thm:mixed}.}
%\begin{corollary}\label{cor:schico}Fix $\beta>0$, $\KA > 1$, and take $\dt$ as in~\eqref{mix:eqn:defDs}. There is a sufficiently large $C>0$ such that if $\ss \ge C e^{2\KA^2}$ %\dmc{This assumption is enough for all cases, but not needed in some, of course} satisfies the conditions of Theorem~\ref{thm:intro-mixed}, then 
%\[\int_{\dt}\P_{x_0}(T_{det}\leq \ss)d\mu(x_0) \;=\; \Omega(e^{-\frac12R}\phis).\]
%\end{corollary}
%\cmk{Shouldn't we remove the $\KA$'s from the corollary above? We are goint to fix the all $\kappa$'s in the integral bounds, right? Also, shouldn't we state somewhere what the value of $\mu(\dt)$ is (similarly to what was done in previous sections)?}