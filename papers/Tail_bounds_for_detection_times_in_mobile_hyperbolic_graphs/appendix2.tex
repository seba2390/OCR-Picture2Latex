



Here we prove Lemma \ref{lem:cotapower} that gives a large deviation result for sums of independent heavy tailed random variables. The lemma is a more precise version of the results in \cite{Omelchenko2019} (although here we treat only with positive random variables which are bounded away from zero), and the proof follows closely what was done there.

\begin{lemma}
Let $S_m=\sum_{i=1}^m Z_i$ where $\{Z_i\}_{i\in\NN}$ is a sequence of i.i.d. absolutely continuous random variables taking values in $[1,\infty)$ such that there are $V,\gamma>0$ with
\[1-F_Z(x)=\PP(Z_i\geq x)\leq Vx^{-\gamma}\] 
for all $x>0$. Then there are $\auxc,\auxl>0$ depending on $V$, $\gamma$ and $\EE(Z_1)$ (if it exists) alone such that:
\begin{itemize}
    \item If $\gamma<1$, then for all $L>\auxl$,
    $\displaystyle\PP(S_m\geq Lm^{\frac{1}{\gamma}})\leq \auxc L^{-\gamma}$.
    \item If $\gamma=1$, then for all $L>\auxl$, $\displaystyle\PP(S_m\geq Lm\log(m))\leq \left(\frac{\auxc}{L\log(m)}\right)^{1-\frac{\auxl}{L}}$.
    \item If $\gamma>1$, then for all $L>\auxl$,
    $\displaystyle\PP(S_m\geq Lm)\,\leq\,\auxc L^{-\gamma}m^{-((\gamma-1)\wedge\frac{\gamma}{2})}$.
\end{itemize}
\end{lemma}
\begin{remark}
We remark that tighter results can be obtained for $\gamma \ge 1$ below, but the current result is sufficient for our purposes.
\end{remark}
\begin{proof}
We proceed as in \cite{Omelchenko2019} by assuming first that $\gamma\leq 1$. Fix $x>0$ and define the event $B=\{\forall 1\leq i< m,\,Z_i\leq x\}$ so that 
\[
    \PP(S_m\geq x)\;\leq\;\PP(\overline{B})+\PP(S_m\geq x\,|\,B)\PP(B)\;\leq\;mVx^{-\gamma}+\PP(S^{(x)}_m\geq x)\PP(B)
\]
where $S^{(x)}$ is the sum of $m$ i.i.d. random variables with c.d.f. $\frac{F_Z(y)}{F_Z(x)}$ for any $y \in [0,x]$. We can thus use a Chernoff bound to deduce
\[
    \PP(S_m\geq x)\;\leq\;mVx^{-\gamma}+e^{-\lambda x}\EE\big(e^{\lambda S^{(x)}_m}\big)\PP(B)\;=\;mVx^{-\gamma}+e^{-\lambda x}\left(\int_1^xe^{\lambda y}dF_Z(y)\right)^m,
\]
where we used independence of the random variables to conclude that $\PP(B)=(F_Z(x))^m$. Define now $M=\frac{2\gamma}{\lambda}$ for some $\lambda$ to be chosen below so that $M<x$ holds. Hence,
\[R(\lambda,x)\,:=\,\int_1^xe^{\lambda y}dF_Z(y)\,=\,\int_1^Me^{\lambda y}dF_Z(y)\,+\,\int_M^xe^{\lambda y}dF_Z(y),\]
which we bound separately. First notice that for some constant $C_1=C_1(\gamma, V)$ we have
\begin{align*}
    \int_1^Me^{\lambda y}dF_Z(y)&\leq\,e^{\lambda M}F_Z(M)-\lambda\int_1^Me^{\lambda y}F_Z(y)dy\\[3pt]
    &\leq\,e^{\lambda M}F_Z(M)-e^{\lambda M}+1+\lambda\int_1^Me^{\lambda y}(1-F_Z(y))dy\\[3pt]
    &\leq\,e^{\lambda M}F_Z(M)-e^{\lambda M}+1+\lambda Ve^{\lambda M}\int_1^My^{-\gamma}dy\;\leq\,1+C_1Q(\lambda),
\end{align*}
where $Q(\lambda)=\lambda^{\gamma}$ if $\gamma<1$ and $Q(\lambda)=-\lambda\log(\frac{\lambda}{2\gamma})$ if $\gamma=1$. For the integral between $M$ and $x$ observe that
\begin{align*}
    \int_M^xe^{\lambda y}dF_Z(y)&\leq\,e^{\lambda M}(1-F_Z(M))+\lambda\int_M^xe^{\lambda y}(1-F_Z(y))dy\\[3pt]
    &\leq\,Ve^{\lambda M}M^{-\gamma}+\lambda V\int_M^xe^{\lambda y}y^{-\gamma}dy\\[3pt]
    &=\,Ve^{2\gamma}\left(\frac{\lambda}{2\gamma}\right)^{\gamma}+ Ve^{\lambda x}x^{-\gamma}\int_0^{\lambda(x-M)}e^{-w}\left(1-\frac{w}{\lambda x}\right)^{-\gamma}dw,
\end{align*}
where in the last line we used the change of variables $w=\lambda(x-y)$. Now, since $M = 2\gamma/\lambda$, the function $f(w)=e^{\frac{w}{2}}(1-\frac{w}{\lambda x})^{\gamma}$ has a positive derivative for $w \in [0, \lambda(x-M)]$. Hence for $w\in[0,\lambda(x-M)]$ we have  $(1-\frac{w}{\lambda x})^{-\gamma}\leq e^{w/2}$  and hence the last integral is therefore smaller than $\int_0^\infty e^{-w/2}dw=2$, giving
\[\int_M^xe^{\lambda y}dF_Z(y)\,\leq\,Ve^{2\gamma}\left(\frac{\lambda}{2\gamma}\right)^{\gamma}+ 2Ve^{\lambda x}x^{-\gamma}\,=\,C_2\lambda^{\gamma}+C_3e^{\lambda x}x^{-\gamma}\]
for some constants $C_2, C_3$ depending only on $V$ and $\gamma$. Putting together both bounds for $R(\lambda,x)$ we arrive at
\begin{align*}\PP(S_m\geq x)&\leq\;mVx^{-\gamma}+e^{-\lambda x}\left(1+C_1Q(\lambda)+C_2\lambda^{\gamma}+ C_3e^{\lambda x}x^{-\gamma}\right)^m\\[3pt]
&\leq\;mVx^{-\gamma}+\exp\left(-\lambda x+mC_1Q(\lambda)+mC_2\lambda^{\gamma}+ mC_3e^{\lambda x}x^{-\gamma}\right).
\end{align*}
Our aim at this point to choose $\lambda$ such that the term on the right is small, which is achieved when taking
\[\lambda=\frac{1}{x}\log\left(\frac{x^{\gamma}}{m}\right),\]
so that $me^{\lambda x}x^{-\gamma}=1$. Assume first that $\gamma<1$ so $x=Lm^{\frac{1}{\gamma}}$ for $L$ large, for which $\lambda=\frac{\gamma\log(L)}{Lm^{\frac{1}{\gamma}}}$ is small, while $\lambda x=\gamma\log(L)$ is large so the assumption $M<x$ is justified. Now, since $\gamma<1$, $Q(\lambda)=\lambda^{\gamma}$ and hence we have
\[-\lambda x+mC_1Q(\lambda)+mC_2\lambda^{\gamma}+ mC_3e^{\lambda x}x^{-\gamma}\,=\,-\gamma\log(L)+(C_1+C_2)\left(\frac{\gamma\log(L)}{L}\right)^{\gamma}+C_3,\]
and since we are assuming $L$ large, we have $(\frac{\gamma\log(L)}{L})^{\gamma}\leq 1$ which finally gives
\begin{align*}\PP(S_m\geq x)&\leq\;mVx^{-\gamma}+\exp\left(-\lambda x+mC_1Q(\lambda)+mC_2\lambda^{\gamma}+ mC_3e^{\lambda x}x^{-\gamma}\right)\\[3pt]
&\leq\;VL^{-\gamma}+\exp\left(-\gamma\log(L)+C_1+C_2+C_3\right)\;=\;\auxc L^{-\gamma},
\end{align*}
which proves the first point of the theorem. Suppose now that $\gamma=1$ so that $x=Lm\log(m)$ for $L\geq \auxl$ for some $\auxl$ large, and hence $\lambda=\frac{1}{Lm\log(m)}\log(L\log(m))$ is small, while $\lambda x=\log(L\log(m))$ is large, so again the assumption $M<x$ is justified. For this choice of $\gamma$ we have $mQ(\lambda)=-m\lambda\log(\frac{\lambda}{2})$ which we can bound as 
\begin{align*}
-m\lambda\log(\tfrac{\lambda}{2})&=\frac{\log(L\log(m))}{L\log(m)}\log\left(\frac{2Lm\log(m)}{\log(L\log(m))}\right)\\[3pt]&\leq\frac{(\log(2L\log(m))^2}{L\log(m)}+\frac{\log(L\log(m))}{L}\leq C_4+\frac{\log(L\log(m))}{L}
\end{align*}
for some constant $C_4$, and hence we arrive at
\begin{align*}\PP(S_m\geq x)&\leq\;mVx^{-1}+\exp\left(-\lambda x+mC_1Q(\lambda)+mC_2\lambda+ mC_3e^{\lambda x}x^{-1}\right)\\[3pt]
&\leq\;\frac{V}{L\log(m)}+C_5\exp\Big({-}\log(L\log(m))+\frac{C_1}{L}\log(L\log(m))\Big)\;\leq\;\Big(\frac{\auxc}{L\log(m)}\Big)^{1-\frac{\auxl}{L}}
\end{align*}
for some constant $C_5$, and where the last inequality holds by choosing $\auxl$ larger than $C_1$ and also by choosing $\auxc$ large enough.

Suppose now that $\gamma>1$ so that $E_0:=\EE(Z_1)$ exists. In this case we can perform a similar computation to the one before to deduce that 
\[
    \PP(S_m-mE_0\geq x)\;\leq\;mVx^{-\gamma}+e^{-\lambda x}\Big(e^{-\lambda E_0}\int_1^xe^{\lambda y}dF_Z(y)\Big)^m,
\]
and we can divide the integral $\int_1^xe^{\lambda y}dF_Z(y)$ as before so that
\[\int_1^xe^{\lambda y}dF_Z(y)\,=\,\int_1^Me^{\lambda y}dF_Z(y)+\int_M^xe^{\lambda y}dF_Z(y)\]
where again $M=\frac{2\gamma}{\lambda}$ (and for our choice of small $\lambda$ below again we have $M < x$). Now, the main difference in this case is the treatment of the first term, for which we have
\begin{align*}
    \int_1^Me^{\lambda y}dF_Z(y)&=\,\int_1^MdF_Z(y)+\lambda\int_1^MydF_Z(y)+\int_1^M\left(e^{\lambda y}-1-\lambda y\right)dF_Z(y)\\[3pt]&\leq\,1+\lambda E_0-\left(e^{\lambda y}-1-\lambda y\right)(1-F_Z(y))\bigg|^M_1+\lambda\int_1^M\left(e^{\lambda y}-1\right)(1-F_Z(y))dy\\[3pt]&\leq 1+\lambda E_0+\left(e^{\lambda }-1-\lambda\right)+\lambda V\int_1^M\left(e^{\lambda y}-1\right)y^{-\gamma}dy\\[3pt]&\leq 1+\lambda E_0+\left(e^{\lambda }-1-\lambda\right)+\frac{\lambda V}{\gamma-1}\left(e^{\lambda}-1\right)+\frac{\lambda^2 V}{\gamma-1}e^{\lambda M}\int_1^My^{1-\gamma}dy\\[3pt]&\leq 1+\lambda E_0+\lambda^2+\frac{2\lambda^2 V}{\gamma-1}+C_1 W(\lambda),
\end{align*}
for some value $C_1$ depending on $\gamma$ alone, where we used that $\gamma>1$, that $\lambda$ is small, but also $\lambda M=2\gamma$, and where 
\[W(\lambda)=\left\{\begin{array}{cl}\lambda^{\gamma}&\text{ if }\gamma<2\\[3pt]-\lambda^2\log(\lambda)&\text{ if }\gamma=2\\[3pt]\lambda^2&\text{ if }\gamma>2\end{array}\right.\]
Since $\lambda$ is small we conclude that the $W(\lambda)$ is at least of the same order as the terms containing $\lambda^2$ and hence
\[\int_1^Me^{\lambda y}dF_Z(y)\;\leq\;1+\lambda E_0+3W(\lambda).\]
Treating the integral $\int_M^xe^{\lambda y}dF_Z(y)$ as in the case $\gamma\leq 1$ we finally obtain 
\begin{align*}\PP(S_m-mE_0\geq x)&\leq\;mVx^{-\gamma}+e^{-\lambda x-\lambda mE_0}\left(1+\lambda E_0+3W(\lambda)+C_2\lambda^{\gamma}+ C_3e^{\lambda x}x^{-\gamma}\right)^m\\[4pt]
&\leq\;mVx^{-\gamma}+\exp\left(-\lambda x+mC_4W(\lambda)+ mC_3e^{\lambda x}x^{-\gamma}\right).
\end{align*}
Now, since we are interested in the probability $\PP(S_m\geq Lm)$ for $L$ larger than some $\auxl$ which we can take larger than $2E_0$ we have
\[\PP(S_m\geq Lm)\,\leq\,\PP(S_m-mE_0\geq Lm/2),\]
and hence we can take $x=Lm/2$. For $\gamma<2$ we choose $\lambda=\frac{1}{x}\log(\frac{x^\gamma}{m})$ as before (which is small) for which $mC_3e^{\lambda x}x^{-\gamma}=C_3$ and hence
\[\PP(S_m-mE_0\geq x)\;\leq\;2^\gamma VL^{-\gamma}m^{1-\gamma}+\exp\left(-\log(L^\gamma m^{\gamma-1}/2^{\gamma})+3C_5m\lambda^\gamma+ C_3\right),  \]
but $m\lambda^\gamma=\frac{2^\gamma\log^\gamma(L^\gamma m^{\gamma-1}2^{-\gamma})}{L^\gamma m^{\gamma-1}}\leq 1$ for $L^\gamma m^{\gamma-1}$ large enough, and hence
\[\PP(S_m-mE_0\geq x)\;\leq\;\auxc L^{-\gamma}m^{1-\gamma}.\]
Suppose now that $\gamma\geq 2$ and choose $\lambda=\frac{\gamma}{x}\log(\frac{x}{\sqrt{m}})$ for which we have $mC_3e^{\lambda x}x^{-\gamma}=C_3m^{1-\frac{\gamma}{2}}\leq C_3$, giving 
\[\PP(S_m-mE_0\geq x)\;\leq\;2^\gamma VL^{-\gamma}m^{1-\gamma}+\exp\left(-\log(L^\gamma m^{\frac{\gamma}{2}}/2^{\gamma})+C_6mW(\lambda)+ C_3\right).\]
Now, if $\gamma=2$, then $W(\lambda)=\lambda^2\log(1/\lambda)$ so for some constant $C_7$
\[mW(\lambda)=\frac{16}{L^2m}\log^2(\tfrac{L \sqrt{m}}{2})\log\left(\frac{Lm}{4\log(L\sqrt{m}/2)}\right)\leq\frac{C_1}{L^2m}\log^3(L^2m)\leq 1\]
for large $L^2m$, while if $\gamma>2$, $W(\lambda)=\lambda^2$, and so 
\[mW(\lambda)=\frac{2\gamma^2}{L^2m}\log^2(\tfrac{L \sqrt{m}}{2})\leq 1\]
for large $L^2m$. In any case scenario, we obtain
\[\PP(S_m-mE_0\geq x)\;\leq\;2^\gamma VL^{-\gamma}m^{1-\gamma}+\auxc L^{-\gamma}m^{-\frac{\gamma}{2}},\]
but for $\gamma\geq 2$ we have $\frac{\gamma}{2}\leq\gamma-1$ and hence the second term dominates the first, giving the result.
\end{proof}