%==============================================================================
\section{Introduction}
\label{sec:intro}
%==============================================================================

Remote code disclosure attacks threaten computer systems.  Remote attackers
exploiting buffer overread vulnerabilities~\cite{Overread:EuroSec09} can not
only steal intellectual property (e.g., proprietary application code, for reverse
engineering), but also leak code to locate gadgets for
advanced code reuse attacks~\cite{JIT-ROP:Oakland13}, thwarting code layout
diversification defenses like Address Space Layout Randomization
(ASLR)~\cite{ASLR:PaX01}.
Embedded Internet-of-Things (IoT) devices exacerbate the situation; many of these
microcontroller-based systems have the same Internet connectivity as
desktops and servers but rarely employ protections against
attacks~\cite{IoT:DAC15,ThermoSpy:BlackHatUSA14}.
Given the ubiquity of these embedded devices in industrial production
and in our lives, making them immune
to code disclosure attacks is crucial.

Recent research~\cite{XnR:CCS14,HideM:CODASPY15,Readactor:Oakland15,%
LR2:NDSS16,ExOShim:ICCWS16,KHide:CNS16,NORAX:Oakland17,kR^X:EuroSys17,%
XOM-Switch:BlackHatAsia18,IskiOS:ArXiv19,uXOM:UsenixSec19} implements
\emph{execute-only memory (XOM)} to defend against code disclosure
attacks.  Despite being unable to prevent code pointer leakage from
data regions such as heaps and stacks, XOM enforces memory
protection on the code region so that
instruction fetching is allowed but reading or writing instructions as
data is disallowed.  This simple and effective defense, however, is not
natively available on low-end microcontrollers.  For example, the
ARMv7-M and ARMv8-M architectures used in mainstream devices support memory
protection but not
execute-only (XO) permissions~\cite{ARMv7-M:Manual,ARMv8-M:Manual}.
uXOM~\cite{uXOM:UsenixSec19} implements
XOM on ARM embedded systems but incurs
significant performance and code size overhead (7.3\% and 15.7\%,
respectively) as it
transforms most load instructions into special unprivileged load
instructions.  Given embedded systems' real-time constraints and limited
memory resources, a practically ideal XOM solution should have
\emph{close-to-zero performance penalty} and
\emph{minimal memory overhead}.

This paper presents \emph{\System}, a fast and novel XOM
system for ARMv7-M and ARMv8-M devices using a memory protection unit
(MPU) and the Data Watchpoint and Tracing (DWT)
unit~\cite{ARMv7-M:Manual,ARMv8-M:Manual}.  {\System} uses the MPU to enforce
\emph{write} protection on code and uses the
unique \emph{address range matching capability}
of the DWT unit to control \emph{read} access to the code region.  On a
matched access, the DWT unit generates a debug monitor exception
indicating an illegal code read, while unmatched accesses execute
normally without slowdown. As {\System} disallows all read
accesses to the code segment, it includes a minimal compiler change that
removes all data embedded in the code segment.

We built a prototype of {\System} and
evaluated it on an ARMv7-M board with two benchmark suites and five
real-world embedded applications.  Our results show that {\System} adds
\emph{negligible performance overhead} of 0.33\% and only has a
\emph{small code size
increase} of 5.89\% while providing strong protection against code disclosure
attacks.

To summarize, our contributions are:
\begin{itemize}
\item
  {\System}: a novel method of utilizing the ARMv7-M and ARMv8-M
  debugging facilities to implement XOM.
  To the best of our knowledge, this is the first use of ARM debug
  features for security purposes.

\item
  A prototype implementation of {\System} on ARMv7-M.

\item
  An evaluation of {\System}'s performance and code size impact on the
  BEEBS benchmark suite, the CoreMark-Pro benchmark suite, and five
  real-world embedded applications, showing that {\System} only incurs 0.33\%
  run-time overhead and 5.89\% code size overhead.
\end{itemize}

The rest of the paper is organized as follows.  Section~\ref{sec:bg}
provides background information on ARMv7-M and ARMv8-M.
Section~\ref{sec:threat} describes our threat model and assumptions.
Sections~\ref{sec:design} and \ref{sec:impl} present the design and
implementation of {\System}, respectively.  Section~\ref{sec:eval}
reports on our evaluation of {\System}, Section~\ref{sec:related}
discusses related work, and Section~\ref{sec:conc} concludes the paper
and discusses future work.
