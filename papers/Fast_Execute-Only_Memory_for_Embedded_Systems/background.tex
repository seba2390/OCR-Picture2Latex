%==============================================================================
\section{Background}
\label{sec:bg}
%==============================================================================

{\System} targets ARMv7-M and ARMv8-M architectures, which cover a wide
range of embedded devices on the market, and it leverages unique
features of these architectures.  This section provides
important background material on the instruction sets, execution modes,
address space layout, memory protection mechanisms, and on-chip debug
support found in ARMv7-M and ARMv8-M.

%------------------------------------------------------------------------------
\subsection{Instruction Sets and Execution Modes}
\label{sec:bg:isa}
%------------------------------------------------------------------------------

ARMv7-M~\cite{ARMv7-M:Manual} and ARMv8-M~\cite{ARMv8-M:Manual} are the
mainstream M-profile ARM architectures for embedded microcontrollers.
Unlike ARM's A and R profiles, they only support the Thumb instruction
set which is a mixture of 16-bit and 32-bit densely-encoded Thumb
instructions.

ARMv7-M~\cite{ARMv7-M:Manual} supports two execution modes:
unprivileged mode and
privileged mode.  An ARMv7-M processor always executes
exception handlers in privileged mode, while application code is allowed
to execute in either mode.  Code running in unprivileged mode can raise the
current execution mode to
privileged mode using a supervisor call instruction ({\tt SVC}).
This is typically how ARMv7-M realizes system calls.  However,
embedded applications usually run in privileged mode to reduce
the cost of system calls.

ARMv8-M inherits all the features of ARMv7-M and adds a security
extension called TrustZone-M~\cite{ARMv8-M:Manual} that isolates
software into a secure world and a non-secure world; this effectively
doubles the execution modes as software can be executing in either world,
privileged or unprivileged.

%------------------------------------------------------------------------------
\subsection{Address Space Layout}
\label{sec:bg:layout}
%------------------------------------------------------------------------------

\begin{figure}[tb]
  \centering
  \resizebox{1.0\columnwidth}{!}{%
    \includegraphics{figs/address-space}
  }
  \caption{Memory Layout of ARMv7-M and ARMv8-M Architectures}
  \label{fig:address-space}
\end{figure}

Both ARMv7-M~\cite{ARMv7-M:Manual} and ARMv8-M~\cite{ARMv8-M:Manual}
architectures operate on a single 32-bit
physical address space and use memory-mapped I/O to access external devices and
peripherals.  As Figure~\ref{fig:address-space} shows, the address space
is generally divided into eight consecutive 512~MB regions; the
{\tt Code} region maps flash memory/ROM that contains code and read-only
data, the {\tt SRAM} region typically contains heaps and stacks, and
the {\tt System} region holds memory-mapped system registers including a
Private Peripheral Bus (PPB) subregion.  The PPB subregion contains all
critical system registers such as MPU configuration registers and the
Vector Table Offset Register {\tt VTOR}.
All other regions are for
memory-mapped peripherals and external devices.
Note that ARMv7-M and ARMv8-M do not have special privileged
instructions to access system registers mapped in the {\tt System}
region; instead, they can be modified by regular load and store
instructions.

%------------------------------------------------------------------------------
\subsection{Memory Protection Unit}
\label{sec:bg:mpu}
%------------------------------------------------------------------------------

ARMv7-M and ARMv8-M devices do not have a memory management unit (MMU)
that supports virtual memory; instead, they support an
optional MPU that can be configured to enforce
region-based access control on physical
memory~\cite{ARMv7-M:Manual,ARMv8-M:Manual}.  A typical ARMv7-M device
supports up to 8 MPU regions, each of which is configurable with a
base address, a power-of-two size from 32~bytes to 4~GB, and separate access
permissions (R, W, and X) for privileged and unprivileged
modes.  With TrustZone-M, ARMv8-M has separate MPU
configurations for secure and non-secure worlds~\cite{ARMv8-M:Manual}.
MPU configuration registers are in the PPB region.

There are, however, limitations on how one can configure access permissions
for an MPU region.  First, the privileged access permission cannot be
more restrictive than the unprivileged one; this prohibits an MPU region
with, for example, unprivileged read-write and privileged read-only
permissions.  Second, the PPB region is always privileged-accessible,
unprivileged-inaccessible, and non-executable regardless of the MPU
configuration.  Third, and most importantly, the MPU does not have the
execute-only permission necessary to support XOM; an MPU region is
executable only if it is configured as both readable and executable.

%------------------------------------------------------------------------------
\subsection{Debug Support}
\label{sec:bg:dwt}
%------------------------------------------------------------------------------

Debug support is another processor feature that ARMv7-M and ARMv8-M
devices can optionally support.  Of all components in the architecture's
debug support, we focus on the DWT
unit~\cite{ARMv7-M:Manual,ARMv8-M:Manual} which provides groups of debug
registers called \emph{DWT
comparators} that support instruction/data address matching, PC value
tracing, cycle counters, and many other functionalities.
Most importantly, a DWT comparator enables monitoring of read accesses over a
specified address range; if the processor reads from
or writes to an address within a specified range, the DWT comparator
will halt the software execution or generate a debug monitor exception.
If, instead, the access does not fall into the specified range,
execution proceeds as normal, and performance is unaffected. When
multiple DWT comparators are configured for data address range matching,
an access that hits any of them will trap.

On ARMv7-M, a DWT comparator can be configured to match an address range
by programming its base address with a mask that specifies a power-of-two
range size~\cite{ARMv7-M:Manual}.  ARMv8-M implements DWT address range
matching by using two consecutively numbered DWT
comparators~\cite{ARMv8-M:Manual},
where the first one specifies the lower bound of the address
range and the second one specifies the upper bound.
