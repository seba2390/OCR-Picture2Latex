% ****** Start of file apssamp.tex ******
%
%   This file is part of the APS files in the REVTeX 4.2 distribution.
%   Version 4.2a of REVTeX, December 2014
%
%   Copyright (c) 2014 The American Physical Society.
%
%   See the REVTeX 4 README file for restrictions and more information.
%
% TeX'ing this file requires that you have AMS-LaTeX 2.0 installed
% as well as the rest of the prerequisites for REVTeX 4.2
%
% See the REVTeX 4 README file
% It also requires running BibTeX. The commands are as follows:
%
%  1)  latex apssamp.tex
%  2)  bibtex apssamp
%  3)  latex apssamp.tex
%  4)  latex apssamp.tex
%

\documentclass[%
prx,
reprint,
%preprint,
superscriptaddress,
% groupedaddress,
%unsortedaddress,
%runinaddress,
%frontmatterverbose, 
%preprint,
%preprintnumbers,
nofootinbib,
%nobibnotes,
%bibnotes,
 amsmath,amssymb,
 aps,
%pra,
%prb,
%rmp,
%prstab,
%prstper,
%floatfix,
% fixfoot,%Reference the same footnote multiple times
% bigfoot,
% onecolumn,
]{revtex4-2}

% \usepackage[nomarkers,nofiglist,noheads]{endfloat}

\usepackage{footmisc}

% The lineno packages adds line numbers. Start line numbering with
% \begin{linenumbers}, end it with \end{linenumbers}. Or switch it on
% for the whole article with \linenumbers.
\usepackage[columnwise]{lineno}
% \linenumbers%Have uncommented for line numbering

\usepackage{graphicx}% Include figure files
\usepackage{dcolumn}% Align table columns on decimal point
\usepackage{bm}% bold math
\usepackage{bbm}
\usepackage{bbold}
%\usepackage{hyperref}% add hypertext capabilities
%\usepackage[mathlines]{lineno}% Enable numbering of text and display math
%\linenumbers\relax % Commence numbering lines

%\usepackage[showframe,%Uncomment any one of the following lines to test 
%%scale=0.7, marginratio={1:1, 2:3}, ignoreall,% default settings
%%text={7in,10in},centering,
%%margin=1.5in,
%%total={6.5in,8.75in}, top=1.2in, left=0.9in, includefoot,
%%height=10in,a5paper,hmargin={3cm,0.8in},
%]{geometry}
\usepackage{xcolor}
\usepackage{scrextend}%Repeat a footnote multiple times
\usepackage{multirow}
\usepackage{makecell}%For newlines within table cells
\usepackage{relsize}%Larger integrals

\definecolor{redcolor}{rgb}{.7,0.,0.}
\def\ega#1{{\color{redcolor} EGA: #1}}

\definecolor{greencolour}{rgb}{0.,.7,0.}
\def\jmm#1{{\color{greencolour} JMM: #1}}

%Allow the use of "\change{ changed text }" To indicate changes in the manuscript:
% \definecolor{changecolour}{rgb}{0.5, 0.0, 0.5}%A shade of purple
\definecolor{changecolour}{rgb}{0, 0.0, 0}%Black
\def\change#1{{\color{changecolour} #1}}

%An easy way to have multiple rows in a cell of a table
\newenvironment{innertable}[1]{\begin{tabular}{@{}#1@{}}}{\end{tabular}}

\renewcommand{\arraystretch}{1.3}%More space between rows of table

\newcommand{\xin}{{{\bf x}^{\text{input}}}}
\newcommand{\yin}{{{\bf y}^{\text{input}}}}
\newcommand{\xset}{\{{\bf x}\}}

% % % From:
% % https://aty.sdsu.edu/bibliog/latex/floats.html
% %
% % % Alter some LaTeX defaults for better treatment of figures:
% % See p. 105 of "TeX Unbound" for suggested values.
% % See pp. 199-200 of Lamport's "LaTeX" book for details.
% %
% %   General parameters, for ALL pages:
% \renewcommand{\topfraction}{0.9}	% max fraction of floats at top
% \renewcommand{\bottomfraction}{0.8}	% max fraction of floats at bottom
% %   Parameters for TEXT pages (not float pages):
% \setcounter{topnumber}{2}
% \setcounter{bottomnumber}{2}
% \setcounter{totalnumber}{2}     % 2 may work better
% \setcounter{dbltopnumber}{2}    % for 2-column pages
% \renewcommand{\dbltopfraction}{0.9}	% fit big float above 2-col. text
% \renewcommand{\textfraction}{0.07}	% allow minimal text w. figs
% %   Parameters for FLOAT pages (not text pages):
% \renewcommand{\floatpagefraction}{0.7}	% require fuller float pages
% % N.B.: floatpagefraction MUST be less than topfraction !!
% \renewcommand{\dblfloatpagefraction}{0.7}	% require fuller float pages

\begin{document}

% Each figure on its own page
\makeatletter
\@fpsep\textheight
\makeatother

\interfootnotelinepenalty=10000%Prevent footnote split across pages from disappearing

\renewcommand{\theequation}{S\arabic{equation}}\renewcommand{\thefigure}{S\arabic{figure}}
\makeatletter
\def\@biblabel#1{[S#1]}
\makeatother
\newcommand{\scite}[1]{\textrm{[S\citealp{#1}]}}

% \interfootnotelinepenalty=10000%Do not split footnotes over multiple pages

\preprint{APS/123-QED}

\title{Supplemental material: Non-parametric power-law surrogates}
% \thanks{A footnote to the article title}%

\author{Jack Murdoch Moore}
%\email{jackmoore@tongji.edu.cn}
%\affiliation{%
%School of Physics Science and Engineering, Tongji University, Shanghai 200092, People's Republic of China}%
\author{Gang Yan}%
%\email{gyan@tongji.edu.cn (corresponding author)}
%\affiliation{%
%School of Physics Science and Engineering, Tongji University, Shanghai 200092, People's Republic of China}
%\affiliation{%
%Shanghai Institute of Intelligent Science and Technology, Tongji University, Shanghai, 200092, P. R. China}
%\affiliation{%
%Center for Excellence in Brain Science and Intelligence Technology, Chinese Academy of Sciences, Shanghai, 200031, P. R. China
%}
\author{Eduardo G. Altmann}%
%\email{eduardo.altmann@sydney.edu.au}%
% \affiliation{%
% School of Mathematics and Statistics, University of Sydney, New South Wales 2006, Australia
% }%

%\date{\today}% It is always \today, today,
             %  but any date may be explicitly specified

\begin{abstract}
\end{abstract}

%\keywords{Suggested keywords}%Use showkeys class option if keyword
                              %display desired
\maketitle

% \jmm{I am not sure that Footnote~\ref{fn.rate-reject-empirical} (Page~\pageref{fn.rate-reject-empirical}) is advisable.}

%\tableofcontents

% % \section{Surrogate methods}
% A surrogate method takes an input sequence $\xin = x_1, \ldots, x_N$ of length $N$ and produces from it one or more synthetic datasets ${\bf x}_n$ of the same length. In the main paper, we show results for typical and constrained i.i.d. power-law surrogates, bootstrapping, shuffling, Markov order one power-law surrogates and ordinal pattern power-law surrogates. In this supplemental document, we expand slightly the range of surrogate methods considered, and present results also for amplitude adjusted Fourier transform (AAFT)~\scite{theiler1992testing} and iterated AAFT (IAAFT)~\scite{schreiber1996improved} surrogates. %In some cases we also present results which would arise for sequences independently generated from the true underlying process (from which the input sequence $\xin$ arose), and for sequences identical with the input sequence $\xin$ originally observed.

\section{Generating constrained surrogates by fixing likelihood}

In this section we demonstrate properties of constrained surrogates chosen uniformly at random from a set $\mathcal{U} \left( \xin \right)$ satisfying conditions C1-C3 of Sec.~II~A of the main paper, focusing on the case in which the likelihood under the null hypothesis is one of the constrained properties. We show that, under the null hypothesis, this prescription is equivalent to conditioning on the outcome of the map $\mathcal{U} : \xin \mapsto \mathcal{U} \left( \xin \right)$, and that the corresponding conditional probability is independent of model parameters. 

% {\bf Claim.} 
% The statistic $S = \mathcal{U} \left( {\bf X} \right)$ is a sufficient statistic for discrete random variable ${\bf X}$, and our prescription for generating constrained surrogates is equivalent to conditioning on $S$.

\noindent{\bf Proof.} 

Let ${\bf X}$ be a discrete random variable with domain $\mathcal{D}$ and probability distribution $P_{\alpha}\left({\bf x}\right)$, where $\alpha$ denotes the model parameter(s). Let $S = \mathcal{U} \left( {\bf X} \right)$, and let $P_{\alpha}\left({\bf x} \mid s\right)$ denote the conditional probability of ${\bf X}$ given $S$. We will show that for any value of the model parameter $\alpha$, and for any ${\bf x} \in \mathcal{D}$,
\begin{equation}
P_{\alpha}\left({\bf x} \mid s\right) = \begin{cases}{\left\lvert \mathcal{U}\left({\bf x}\right)\right\rvert}^{-1},&s = \mathcal{U}\left({\bf x}\right)\\
0,&s \neq \mathcal{U}\left({\bf x}\right)
\end{cases},\notag
\end{equation}
where $\left\lvert \mathcal{A} \right\rvert$ denotes the cardinality of a set $\mathcal{A}$. This expression for the conditional probability $P_{\alpha}\left({\bf x} \mid s\right)$ is independent of $\alpha$. In addition, the probability corresponds to choosing uniformly at random from the set $\mathcal{U}\left({\bf x}\right)$, showing that, under the null hypothesis, the way in which we produce constrained surrogates is indeed equivalent to conditioning on $\mathcal{U}$.

Let $Q_{\alpha} \left(s\right)$ denote the probability distribution of $S = \mathcal{U}\left({\bf X}\right)$, and let $P_{\alpha}({\bf x}, s)$ denote the joint distribution of ${\bf X}, S$. Let ${\bf x},{\bf y} \in \mathcal{D}$. If $\mathcal{U}\left({\bf x}\right) \neq \mathcal{U}\left({\bf y}\right)$ then $P_{\alpha}\left({\bf x} \mid \mathcal{U}\left({\bf y}\right)\right) = 0$. Hence, we can restrict our attention to the case
\begin{equation}\label{eq.joint}
\mathcal{U}\left({\bf x}\right) = \mathcal{U}\left({\bf y}\right),
\end{equation}%
in which circumstance
\begin{align}
P_{\alpha}\left( {\bf x}, \mathcal{U}\left({\bf y}\right) \right) &= P_{\alpha}({\bf x}, \mathcal{U}\left({\bf x}\right))\notag\\
&= P_{\alpha}\left({\bf x}\right).\label{eq.characteristic}
\end{align}%
First we note that, because the likelihood given by $\mathcal{L}_{{\bf x}}\left( \alpha \right) = P_{\alpha} \left( {\bf x}\right)$ is one of the constrained properties $K$, condition C3 implies
\begin{align}
\forall {\bf z} \in \mathcal{U}\left({\bf x}\right),\qquad P_{\alpha}\left( {\bf z} \right) &= P_{\alpha}\left( {\bf x} \right).\label{eq.prob}
\end{align}%
Now we will show that, by conditions C1 and C2,
\begin{align}\label{eq.partition}
\forall {\bf z} \in \mathcal{D},\qquad \mathcal{U}\left({\bf z}\right) = \mathcal{U}\left({\bf x}\right) \Leftrightarrow {\bf z} \in \mathcal{U}\left({\bf x}\right).
\end{align}%
($\Rightarrow$): For any ${\bf z} \in \mathcal{D}$, if $\mathcal{U}\left({\bf z}\right) = \mathcal{U}\left({\bf x}\right)$ then, by condition C1, ${\bf z} \in \mathcal{U}\left({\bf z}\right) = \mathcal{U}\left({\bf x}\right)$. ($\Leftarrow$): Conversely, for any ${\bf z} \in \mathcal{D}$, if ${\bf z} \in \mathcal{U}\left({\bf x}\right)$ then, by condition C2, $\mathcal{U}\left({\bf z}\right) = \mathcal{U}\left({\bf x}\right)$. Result (\ref{eq.partition}) implies that
\begin{align}
Q_{\alpha}(\mathcal{U}\left({\bf x}\right)) &= \sum_{\substack{{\bf z} \in \mathcal{D}\\ \mathcal{U}\left({\bf z}\right) = \mathcal{U}\left({\bf x}\right)}} P_{\alpha}\left({\bf z}\right)\notag\\
&= \sum_{{\bf z} \in \mathcal{U}\left({\bf x}\right)} P_{\alpha}\left({\bf z}\right),\notag
\end{align}%
showing, together with Eq.~(\ref{eq.prob}), that
\begin{align}
Q_{\alpha}(\mathcal{U}\left({\bf x}\right)) &= \sum_{{\bf z} \in \mathcal{U}\left({\bf x}\right)} P_{\alpha}\left({\bf x}\right)\notag\\
&= \left\lvert\mathcal{U}\left({\bf x}\right)\right\rvert P_{\alpha}\left({\bf x}\right).\label{eq.uniform}
\end{align}
If $P_{\alpha}\left({\bf x}\right) \neq 0$,
\begin{align}
P_{\alpha}\left({\bf x} \mid \mathcal{U}\left({\bf y}\right)\right) &= \frac{P_{\alpha}\left({\bf x}, \mathcal{U}\left({\bf y}\right)\right)}{Q_{\alpha}(\mathcal{U}\left({\bf y}\right))}\notag\\&= \frac{P_{\alpha}\left({\bf x}\right)}{Q_{\alpha}(\mathcal{U}\left({\bf x}\right))}&&\textrm{By Eq. (\ref{eq.joint}, \ref{eq.characteristic})}\notag\\
&= \frac{P_{\alpha}\left({\bf x}\right)}{\left\lvert\mathcal{U}\left({\bf x}\right)\right\rvert P_{\alpha}\left({\bf x}\right)}&&\textrm{By Eq. (\ref{eq.uniform})}\notag\\
&= {\left\lvert\mathcal{U}\left({\bf x}\right)\right\rvert}^{-1}.\notag
\end{align}%
We can neglect the case $P_{\alpha}\left({\bf x}\right) = 0$ because in such a situation, by Eq. (\ref{eq.uniform}), $Q_{\alpha}(\mathcal{U}\left({\bf x}\right)) = 0$ and the conditional probability $P_{\alpha}\left({\bf x} \mid \mathcal{U}\left({\bf y}\right)\right)$ is not defined.
$\square$

%\clearpage

% \section{Distributions of values}
% In this section we illustrate using histograms the performance of different surrogate methods in: (1) randomizing input sequences; and (2) preserving power-law characteristics. To do so, wWe take as input $\xin$ a single sequence of length $N = 1,024$ arising i.i.d from a discrete power-law with lower cut-off $x_{\min} = 12$ and scale exponent $\alpha = 2.5$. In Fig.~\ref{fig:histograms}, for each surrogate method, we juxtapose a histogram for $\xin$ with the median and 90\% confidence intervals of 100 surrogates generated from $\xin$. 


\begin{figure*}[p]
\centering
\includegraphics[width=0.6\textwidth]{histograms-comparing-surrogates-original_100surr.pdf}
\caption{
{Bootstrap, shuffle, AAFT, and IAAFT surrogates preserve power-law characteristics but cannot generate previously unobserved values; typical and constrained power-law surrogates preserve statistical characteristics of input power-law data while randomising specific values.}
% Constrained power-law surrogates preserve statistical characteristics of input power-law data while randomising specific values.
Each panel shows: (1) the probability distribution of an input sequence $\xin$ arising from $N = 1024$ i.i.d. power-law random variables with scale exponent $\alpha = 2.5$ and lower cut-off $x_{\min} = 12$ (opaque orange line); (2) the median probability over 100 independent surrogate realisations from this input sequence $\xin$ (opaque purple line); and (3) the 90\% confidence interval of the probability (transparent purple region). The surrogates considered are: (a) the true underlying process (i.i.d. power-law with scale exponent $\alpha = 2.5$ and lower cut-off $x_{\min} = 12$); (b) typical i.i.d. power-law; (c) constrained i.i.d. power-law; (d) bootstrap; (e) shuffle; (f) constrained Markov order one power-law; (g) constrained length $m + 1 = 16$ ordinal pattern power-law; (h) amplitude adjusted Fourier transform (AAFT); and (i) iterated amplitude adjusted Fourier transform (IAAFT).% \jmm{Are the following statements appropriate in a caption?} Bootstrapping can modify the frequency with which previously observed values appear, but cannot generate previously unseen states. Shuffle, AAFT, and IAAFT surrogates are all unable to either achieve novel values or change the frequency of previously observed values. Typical i.i.d. power-law surrogates and constrained i.i.d., Markov order one and ordinal pattern power-law surrogate methods are all effective at generating previously unobserved values while preserving the power-law characteristics of the original dataset.
}
\label{fig:histograms}
\end{figure*}

%\clearpage

\begin{figure*}[p]
\centering
\includegraphics[width=0.80\textwidth]{cond-entropy-order-1.pdf}
\caption{
{Among all randomizing surrogates, only the} Markov order one power-law surrogate preserves the conditional entropy of order one of the hypothesised Markov states. Each panel shows 100 markers, each of which corresponds to an independently generated observed  input sequence $\xin$ comprising $N = 1024$ independently and identically distributed (i.i.d.) power-law random variables with scale exponent $\alpha = 2.5$ and lower cut-off $x_{\min} = 1$. The $x$-coordinate ($y$-coordinate) of each marker is the conditional entropy of order one of hypothesised Markov states of the corresponding observed input sequence (a single surrogate generated from the corresponding observed input sequence). The surrogates considered are: (a) exact reproduction of the observed sequence (trivially producing markers along the line of identity $y = x$); (b) the true underlying process (i.i.d. power-law with scale exponent $\alpha = 2.5$); (c) typical i.i.d. power-law; (d) constrained i.i.d. power-law; (e) bootstrap; (f) shuffle; (g) constrained Markov order one power-law; (h) constrained ordinal pattern power-law; (i) amplitude adjusted Fourier transform (AAFT); and (j) iterated amplitude adjusted Fourier transform (IAAFT). {The conditional entropy considered is relative to the Markov states, based on log-binning (main paper, Appendix A), defining the constrained Markov order one power-law surrogate algorithm which we employ.}% \jmm{Are the following statements appropriate in a caption?} The constrained Markov order one power-law surrogate precisely preserves the conditional entropy of order one. This is not true of any other randomizing surrogate method, although constrained ordinal pattern power-law, IAAFT and shuffle surrogates consistently show points close to the line of identity on which the conditional entropy of order one of the input sequence and the surrogate realization are equal.
}
\label{fig:cond_ent_one}
\end{figure*}

%\clearpage

% \section{Expectation and variance}

% In this section we explore the expectation and variance of sample statistics with respect to surrogate realisation or input sequence $\xin$. As explained in the main paper, for constrained surrogates and any statistic $s$,
% \begin{equation}\label{eq.var}
% \mathbb{V} \left[ s\right] = \mathbb{V} \left[\mathbb{E}_{\textrm{surr.}} \left[ s\right] \right] + \mathbb{E} \left[\mathbb{V}_{\textrm{surr.}} \left[ s\right] \right],
% \end{equation}
% where $\mathbb{E}$ ($\mathbb{V}$) denotes the expectation (variance) over sequences from the original generative process and $\mathbb{E}_{\textrm{surr.}}$ ($\mathbb{V}_{\textrm{surr.}}$) denotes the expectation (variance) over surrogate sequences generated from a single input sequence $\xin$. 

% We consider independently generating power-law sequences, and for each surrogate sequence estimating the sample maximum, coefficient of variation (a common measure of inequality or heterogeneity~\scite{champernowne1998economic}), and index of dispersion~\scite{perry1979power}. 

% Figure \ref{fig:mean_var_statistics} shows that, as also noted in the main paper, constrained surrogates provide an unbiased estimate in all cases, in contrast to bootstrapping and the typical approach (Fig.~\ref{fig:mean_var_statistics}a-c). 

% As predicted by Eq.~(4) of the main paper, taking the mean over many constrained surrogates reduces finite variance in estimates of the expectation of the sample coefficient of variation. In contrast, because the sample maximum and sample index of dispersion have infinite variance, taking the mean over many constrained surrogate realisations may not reduce its variance, although it usually reduces the variance we estimate from a finite number of independently realised power-law sequences $\xin$. Typical surrogates can also reduce variance, but may introduce bias at the same time. 
% The variance of a statistic over many typical surrogate realisations can be similar to the total variance of that statistic. In contrast, as expected from Eq.~(4) of the main paper, for finite variance statistics, the variance over constrained surrogates is lower in expectation than the total statistical variance. The sample maximum and index of dispersion increase with sample length $N$, but the number of instances of sequences used to calculate the variance in these statistics does not change, instead remaining fixed at 100 and 10,000 for $\mathbb{V}_{\text{surr.}}$ and $\mathbb{V}$ respectively. For this reason, the variances $\mathbb{V}_{\text{surr.}}$ and $\mathbb{V}$ of these statistics tend to increase as sample length $N$ increases.

\begin{figure*}[p]
\includegraphics[width=0.50\textwidth]{ee-ve-ev_supp_power-law_min-val-1_10000tests_100surr_0-3-1-2-4.pdf}
\caption{
Constrained surrogates do not bias sample statistics and can reduce their variance. (a-c) Expectation $\mathbb{E}$ and (d-f) variance $\mathbb{V}$ of estimated expectation $\mathbb{E}_{\text{surr.}}$ of sample statistics. (g-i) Expectation $\mathbb{E}$ of estimated variance $\mathbb{V}_{\text{surr.}}$ of sample statistics.
$\mathbb{E}$ and $\mathbb{V}$ are calculated using $10,000$ independent input sequences $\xin$, each comprising $N$ values drawn i.i.d. from a power-law distribution with $\alpha = 2.5$.
$\mathbb{E}_{\text{surr.}}$ and $\mathbb{V}_{\text{surr.}}$ are estimated using $100$ surrogates, each from the same $\xin$.  The surrogates considered are either generated independently from the true underlying process (solid gray), exactly the input sequence $\xin$ originally observed (dot-dashed black), typical, constrained or bootstrapped. Surrogates have the same length $N$ as the original sequence (horizontal axis). Results for the sample maximum, coefficient of variation, and index of dispersion are shown in each column and {are} computed in sequences of length $N$. 
{Constrained surrogates provide an unbiased estimate in all cases, in contrast to bootstrapping and the typical approach.
As predicted by Eq.~(4) of the main paper, taking the mean $\mathbb{E}_{\text{surr.}}$ over many constrained surrogates reduces finite variance $\mathbb{V}$ in estimates of the expectation of the sample coefficient of variation. Because the sample maximum and sample index of dispersion of a power-law with scale exponent three or less have infinite variance, taking the mean $\mathbb{E}_{\text{surr.}}$ over many constrained surrogate realisations may not reduce its variance $\mathbb{V}$, although it usually reduces the variance we estimate from a finite number of independently realised power-law sequences $\xin$. Typical surrogates can also reduce variance $\mathbb{V}$, but may introduce bias at the same time. 
The variance $\mathbb{V}_{\text{surr.}}$ of a statistic over many typical surrogate realisations can be similar to the total variance of that statistic. As expected from Eq.~(4) of the main paper, for finite variance statistics, the variance $\mathbb{V}_{\text{surr.}}$ over constrained surrogates is lower in expectation than the total statistical variance.
The variances $\mathbb{V}_{\text{surr.}}$ and $\mathbb{V}$ of these statistics tend to increase as sample length $N$ increases. This occurs because the sample maximum, coefficient of variation and index of dispersion increase with sample length $N$, but the number of instances of sequences used to calculate variance does not change.%, instead remaining fixed at 100 and 10,000 for $\mathbb{V}_{\text{surr.}}$ and $\mathbb{V}$ respectively.
}
}
\label{fig:mean_var_statistics}
\end{figure*}

%\clearpage

% \section{Hypothesis tests}

% In this section we consider hypothesis testing, presenting results for more combinations of surrogate method, discriminating statistic and dataset than in the main paper. First, in Fig. \ref{fig:hyp_test}, we consider data which are indeed power-law, but also Markov of order one or two (main paper, Appendix A). In addition to the results already visible in Fig.~10 of the main paper, Fig. \ref{fig:hyp_test} presents results for AAFT and IAAFT surrogates and the outcome of using mean, variance or maximum as a discriminating statistic. Whichever discriminating statistic is employed, typical and constrained i.i.d. power-law surrogates lead to rates of rejection of the power-law hypothesis much higher than the nominal size of the test. When shuffle, AAFT or IAAFT surrogates (which only alter the order of a sequence, but not the frequencies of distinct values) are used together with KS distance, mean, variance or maximum (statistics independent of the order), the rate of rejection trivially remains close to the nominal size. In contrast, when a conditional entropy is used as discriminating statistic together with AAFT or IAAFT surrogates, rejection rates are uniformly high for long samples, showing that these surrogate methods cannot capture this aspect of the correlation properties of Markov order one or two power-law time series. For discussion of hypothesis tests which employ constrained Markov order one power-law surrogates and use a conditional entropy as discriminating statistic, please see the discussion of Fig.~10 in the main paper. Constrained ordinal pattern power-law surrogates do not reject at a rate substantially higher than the nominal size when either the KS distance or the conditional entropy of order two is used as a discriminating statistic. When the mean, variance, maximum or conditional entropy of order one is used instead, this class of surrogate still leads to a rate of rejection lower than that observed for typical or constrained i.i.d. power-law surrogates. These patterns suggest that constrained ordinal pattern power-law surrogates can partly capture the statistical properties of the considered Markov order one and two power-law datasets.

\begin{figure*}[p]
\centering
\includegraphics[width=0.90\textwidth]{Markov-power-law_6stats_nom-size-0-1_num-surr-9_num-tests_1000_pow_min-val-1-1_7surr_9N4-1024.pdf}
\caption{%Amplitude adjusted Fourier transform (AAFT) and iterated amplitude adjusted Fourier transform (IAAFT) cannot prevent correlation-based rejection of power-law hypotheses. 
Constrained correlated power-law surrogates reduce correlation-based rejection of power-law hypotheses. 
Rates of rejection estimated from $1,000$ hypothesis tests, each using a sample of length $N$ from a power-law which has scale exponent $\alpha = 2.5$, lower cut-off $x_{\min} = 1$ and Markov order (a-f) one or (g-l) two. Tests use nine typical {i.i.d.} power-law, constrained {i.i.d.} power-law, shuffled, constrained Markov order one power-law, constrained length $m + 1 = 16$ ordinal pattern power-law, AAFT or IAAFT surrogates, have a nominal size 10\%, and use as test statistic (from left to right) the KS distance, mean, variance, maximum, conditional entropy of order one, and conditional entropy of order two. Markov power-law surrogates use the same Markov states as the original time series. Error bars show standard error and are at most only slightly larger than the line width. 
% Whichever discriminating statistic is employed, typical and constrained i.i.d. power-law surrogates lead to rates of rejection of the power-law hypothesis much higher than the nominal size of the test. 
{When shuffle, AAFT or IAAFT surrogates (which only alter the order of a sequence, but not the frequencies of distinct values) are used together with KS distance, mean, variance or maximum (statistics independent of the order), the rate of rejection trivially remains close to the nominal size. When a conditional entropy is used as discriminating statistic together with AAFT or IAAFT surrogates, rejection rates are uniformly high for long samples. Constrained ordinal pattern power-law surrogates do not reject at a rate substantially higher than the nominal size when either the KS distance or the conditional entropy of order two is used as a discriminating statistic. When the mean, variance, maximum or conditional entropy of order one is used instead, constrained ordinal pattern surrogates lead to a rate of rejection lower than that observed for typical or constrained i.i.d. power-law surrogates.}
}
\label{fig:hyp_test}
\end{figure*}

%\clearpage

% In Fig.~\ref{fig:hyp_test_empirical_data_various_L} we apply typical i.i.d., constrained i.i.d and constrained ordinal pattern power-law surrogates to correlated empirical data to examine how the rate of rejection of a power-law hypothesis depends on the hypothesised length of correlations $m$. We consider numbers of customers affected by blackouts, intensities of solar flares and energy release by earthquakes. 
% Whichever test statistic, surrogate method or length of correlations is considered, the hypothesis that the number of people affected by blackouts follows a power-law is never rejected with rate substantially higher than the nominal size of the test (This is not surprising given the short length, $N = 57$, of the time series). 
% In contrast, the rate of rejection of a power-law hypothesis for solar flare data is consistently higher than the nominal rate of rejection, except in cases in which the KS-distance is used as a discriminating statistic together with a constrained ordinal pattern power-law surrogate. 
% The rate of rejection for records of earthquake energy release shows still different behaviour. When mean, variance or maximum is used as the discriminating statistic, the rate of rejection is never much higher than the nominal size of the test. When instead the KS distance or a conditional entropy is employed, the rate of rejection is substantially higher than the nominal size when typical or constrained i.i.d. power-law surrogates are used. For these statistics, as the hypothesised length of correlations $m$ increases, the rate of rejection decreases until, when ordinal patterns of length $m + 1 = 16$ are used, it is no longer substantially larger than the nominal size of the test\footnote{For the conditional entropy of order one, the rate of rejection is $0.129$, slightly greater than the nominal size of 0.1. % which, for a Bernoulli random variable with success probability equal to the nominal size of the test, $\alpha = 0.1$, is slightly less than one standard deviation, $\sqrt{\alpha (1 - \alpha)} = 0.3$, above the expected value of $alpha$. 
% To evaluate the significance of this, and the extent to which it is inconsistent with the null hypothesis, one should consider that multiple hypothesis tests are being performed and there is limited sample size (the 59,555 observations can make at most 58 non-overlapping sequences of length $N = 1,024$). %: that the data are power-law with correlations captured by ordinal patterns of length $m + 1$. %The probability of observing sample mean $0.129 n$ or higher for a Binomial random variable describing $n$ trials with success probability $p = 0.1$ is greater than 0.10 (0.05) as as long as $n < 156$ ($n < 280$), but the $59,555$ samples in the earthquake dataset (main paper, Table II) could make at most $n = 58$ non-overlapping samples of length $N = 1,024$. 
% In any case, for the characterization of extreme events, the conditional entropy is of less direct interest than discriminating statistics such as the mean, maximum or variance. \label{fn.rate-reject-empirical}}. This is our rationale for emphasising ordinal patterns of length $m + 1 = 16$ throughout the main paper and this supplemental material.

\begin{figure*}[p]
\centering
\includegraphics[width=0.90\textwidth]{flares-earthquakes-blackouts.pdf}
\caption{For sequences of energy released by earthquakes, but not intensity of solar flares, {constraining ordinal patterns of length $m + 1 = 16$ avoids} rejection of a power-law hypothesis at a rate {substantially} higher than the nominal size. Rates of rejection estimated from $1,000$ hypothesis tests, each using a sample of length (a-f) {$N = 57$ of} number of customers affected by earthquakes, (g-l) {$N = 1,024$ of} intensities of solar flares, and (m-r) {$N = 1,024$ of} energy released by earthquakes. Each input sequence $\xin$ considered begins at an independently and randomly chosen point in the full empirical sequence of values above the fitted lower cut-off $\hat{x}_{\min}$. Tests use nine typical {i.i.d.}, constrained {i.i.d.}, or constrained length $m + 1$ ordinal pattern (OP) power-law surrogates, have a nominal size 10\%, and use as test statistic (from left to right) the KS distance, mean, variance, maximum, conditional entropy of order one, and conditional entropy of order two. A solid gray line is drawn at the nominal size.
% \jmm{Are the following statements appropriate in a caption?}
% Whichever test statistic, surrogate method or length of correlations is considered, the hypothesis that the number of people affected by blackouts follows a power-law is never rejected with rate substantially higher than the nominal size of the test. The rate of rejection of a power-law hypothesis for solar flare data is consistently higher than the nominal rate of rejection, except in cases in which the KS-distance is used as a discriminating statistic together with a constrained ordinal pattern power-law surrogate. 
% For records of earthquake energy release, when mean, variance or maximum is used as the discriminating statistic, the rate of rejection is never much higher than the nominal size of the test. When instead the KS distance or a conditional entropy is employed, the rate of rejection is substantially higher than the nominal size when typical or constrained i.i.d. power-law surrogates are used. For these statistics, as the hypothesised length of correlations $m$ increases, the rate of rejection decreases until, when ordinal patterns of length $m + 1 = 16$ are used, it is no longer substantially larger than the nominal size of the test.
}
\label{fig:hyp_test_empirical_data_various_L}
\end{figure*}

%\clearpage

% Finally, in Fig.~\ref{fig:hyp_test_empirical_data} we provide additional information about how the choice of surrogate method impacts the conclusions we make about correlated empirical datasets. In contrast to Fig.~12 of the main paper, Fig.~\ref{fig:hyp_test_empirical_data} shows the outcome of employing AAFT and IAAFT surrogates, uses the mean, variance or conditional entropy of order one as a test statistic, and presents results for the number of customers affected by blackouts% as well as solar flare intensity and energy released by earthquakes
% . 
% %
% Whichever test statistic, surrogate method or sample length $N$ is considered, the hypothesis that the number of people affected by blackouts is a power-law is never rejected with a rate higher than 30\%, and as the sample length $N$ increases, this rate consistently comes to be at most slightly larger than the nominal size of the test. This low rate of rejection is not surprising given the short length, $N = 57$, of the blackout data. 
% %
% When shuffle, AAFT or IAAFT surrogates are used together with KS distance, mean, variance or maximum, as noted above, the rate of rejection trivially remains close to the nominal size. In contrast, when a conditional entropy is used as discriminating statistic together with AAFT or IAAFT surrogates, rejection rates are high for long samples of intensities of solar flares and earthquake energy release, showing that these surrogate methods cannot capture this aspect of the correlation properties of these datasets.
% %
% Employing a constrained Markov order one power-law surrogate consistently leads to low rates of rejection of a power-law hypothesis for intensity of solar flares or energy released by earthquakes on the basis of KS-distance, mean, variance or maximum. However, the type of order one Markov constraints which we consider do not explain the observed order one or two Markov properties (captured by the conditional entropy of order one or two) of sequences of solar flare intensities. In contrast, for sequences of earthquake energies these Markov constraints substantially decrease rates of rejection when using the conditional entropy of order one or two and so appear to be able to at least partly explain the observed memory properties. For solar flare intensities, using ordinal pattern surrogates only slightly reduces the rate of rejection when the conditional entropy of order one or two is used as a discriminating statistic, and actually increases the rate of rejection when using the mean, variance or maximum.  When correlations in earthquake intensities are accommodated by constraining ordinal patterns, the discriminating statistics considered consistently lead to low rates of rejection of a power-law model.
% %
% Please see the discussion of Fig.~12 in the main paper for further consideration of hypothesis tests based on KS distance, maximum or conditional entropy of order two together with shuffling or typical i.i.d., constrained i.i.d., constrained Markov order one or constrained ordinal pattern power-law surrogates. 

\begin{figure*}[p]
\centering
\includegraphics[width=0.90\textwidth]{flares_earthquakes_blackouts_with_aaft_iaaft_6stats_nom-size-0-1_num-surr-9_num-tests_1000_min-deg-1-1_7.pdf}
\caption{{Incorporating correlation constraints impacts the rate of rejection of power-law hypotheses for empirical observations across a wide range of discriminating statistics.} 
Rates of rejection estimated from $1,000$ hypothesis tests of sequences of length $N$ of (a-f) number of customers affected by blackouts, (g-l) intensities of solar flares, and (m-r) energy released by earthquakes. Each input sequence $\xin$ considered begins at an independently and randomly chosen point in the full empirical sequence of values above the fitted lower cut-off $\hat{x}_{\min}$.  Tests use nine typical {i.i.d.} power-law, constrained {i.i.d.} power-law, shuffled, constrained Markov order one power-law, {constrained} length 16 ordinal pattern power-law, AAFT or IAAFT surrogates, have a nominal size 10\%, and use as test statistic (from left to right) the KS distance, mean, variance, maximum, conditional entropy of order one, and conditional entropy of order two. Markov power-law surrogates use the same Markov states as the original time series.
{When shuffle, AAFT or IAAFT surrogates (which only alter the order of a sequence, but not the frequencies of distinct values) are used together with KS distance, mean, variance or maximum (statistics independent of the order) the rate of rejection trivially remains close to the nominal size.}
% \jmm{Are these statements appropriate for a caption?}
% Whichever test statistic, surrogate method or sample length $N$ is considered, the hypothesis that the number of people affected by blackouts is a power-law is never rejected with a rate higher than 30\%, and as the sample length $N$ increases, this rate consistently comes to be at most slightly larger than the nominal size of the test.
% When a conditional entropy is used as discriminating statistic together with AAFT or IAAFT surrogates, rejection rates are high for long samples of intensities of solar flares and earthquake energy release.
% Employing a constrained Markov order one power-law surrogate consistently leads to low rates of rejection of a power-law hypothesis for intensity of solar flares or energy released by earthquakes on the basis of KS-distance, mean, variance or maximum. The type of order one Markov constraints which we consider do not explain the observed order one or two Markov properties (captured by the conditional entropy of order one or two) of sequences of solar flare intensities. For sequences of earthquake energies these Markov constraints substantially decrease rates of rejection when using the conditional entropy of order one or two and so appear to be able to at least partly explain the observed memory properties. For solar flare intensities, using ordinal pattern surrogates only slightly reduces the rate of rejection when the conditional entropy of order one or two is used as a discriminating statistic, and increases the rate of rejection when using the mean, variance or maximum. When correlations in earthquake intensities are accommodated by constraining ordinal patterns, the discriminating statistics considered consistently lead to low rates of rejection of a power-law model.
}
\label{fig:hyp_test_empirical_data}
\end{figure*}

%\clearpage

% \section{Computational time}

% In this section, we compare the computational cost of different surrogate methods. In Fig.~\ref{fig:comp_time} we juxtapose the mean time $T$ required to produce 100 surrogates from an i.i.d. power-law sequence with lower cut-off $x_{\min} = 1$ and scale exponent $\alpha = 2.5$. Bootstrap and shuffle surrogates are the fastest methods and, at large sample length $N$, constrained i.i.d. power-law, AAFT and IAAFT surrogates have comparable times. However, the apparent computational time of AAFT and IAAFT surrogate methods would be increased by communication costs resulting from our choice to call an existing MATLAB implementation from within Python. Typical i.i.d. power-law surrogates require less (more) computational time than constrained i.i.d. power-law surrogates for large (small) values of sample length $N$. For sample length $N$ greater than or equal to eight, the computational time required to produce constrained Markov order one or ordinal pattern surrogates using a Metropolis algorithm is more than the time required for other surrogate methods.

% We also examine the scaling of computational time $T$ with sample length $N$, based on the model
% \begin{equation}
% T = K N^\eta.\label{eq.time-scaling}
% \end{equation}
% To do so, we fit the scale exponent $\eta$ using least squares regression of $\log T$ to $\log N$ for $N \geq 64$, using the average value of $T$ over the ten trials considered. In the legend of Fig.~\ref{fig:comp_time} we present our estimates of scale exponent $\eta$. Over the considered range, the computational time required to produce constrained Markov order one or ordinal pattern surrogates using a Metropolis algorithm scales approximately as the number of transitions. The exponent $\eta$ for typical (constrained) i.i.d. power-law surrogates is slightly smaller (larger) than $\eta = 1$. Other methods appear to scale approximately linearly in the sample length $N$, with $\eta \approx 1$. 

\begin{figure*}[p]
\centering
\includegraphics[width=0.50\textwidth]{comp-time_min-val-1_100surr_10trials.pdf}
\caption{
{Computational cost of surrogate methods as a function of sample size $N$. The computational cost of constrained (typical)} {i.i.d.} power-law surrogates scales {slightly more than (slightly less than) linearly} with {sample} length, while {the computational cost of constrained} Markov {order} and ordinal pattern power-law {surrogates} scales approximately linearly with the number of transitions. The mean over ten independent trials of the computational time $T$ required to produce 100 surrogates. Each input sequence $\xin$ arises independently from a single realisation of length $N$ of an i.i.d. power-law {process} with scale exponent $\alpha = 2.5$ {and lower cut-off $x_{\min} = 1$}. Typical {i.i.d.} power-law, constrained {i.i.d.} power-law, bootstrap, shuffle, Markov order one, ordinal pattern, amplitude adjusted Fourier transform (AAFT) and iterated amplitude adjusted Fourier transform (IAAFT) surrogates are considered. In the case of the constrained Markov order one power-law and ordinal pattern power-law surrogates, which use a Metropolis algorithm, we consider separately time required for $N^2$ and $10^5$ transitions per surrogate. The legend presents the exponent $\eta$ with which computational time $T$ scales with sample length $N$, according to {the model $T = K N^\eta$. The exponent $\eta$ is estimated using least squares regression of $\log T$ to $\log N$ for $N \geq 64$, using the average value of $T$ over the ten trials considered. The AAFT and IAAFT surrogate methods are MATLAB implementations called from within Python, which leads to communication costs which increase their apparent computational time.}
}
\label{fig:comp_time}
\end{figure*}

%\clearpage

%\begin{acknowledgements}
%JMM and GY are supported by National Natural Science Foundation of China (Grant No. 11875043), Shanghai Science and Technology Committee (Grant No. 18ZR1442000). %We thank Michael Small, Thomas J\"ungling, D\'ebora Cristina Corr\^ea and Yinqi Xuan for careful reading of the manuscript.
%\end{acknowledgements}

% \bibliography{ref}% Produces the bibliography via BibTeX.



\end{document}
%
% ****** End of file apssamp.tex ******
