% This is "sig-alternate.tex" V2.1 April 2013
% This file should be compiled with V2.5 of "sig-alternate.cls" May 2012
%
% This example file demonstrates the use of the 'sig-alternate.cls'
% V2.5 LaTeX2e document class file. It is for those submitting
% articles to ACM Conference Proceedings WHO DO NOT WISH TO
% STRICTLY ADHERE TO THE SIGS (PUBS-BOARD-ENDORSED) STYLE.
% The 'sig-alternate.cls' file will produce a similar-looking,
% albeit, 'tighter' paper resulting in, invariably, fewer pages.
%
% ----------------------------------------------------------------------------------------------------------------
% This .tex file (and associated .cls V2.5) produces:
%       1) The Permission Statement
%       2) The Conference (location) Info information
%       3) The Copyright Line with ACM data
%       4) NO page numbers
%
% as against the acm_proc_article-sp.cls file which
% DOES NOT produce 1) thru' 3) above.
%
% Using 'sig-alternate.cls' you have control, however, from within
% the source .tex file, over both the CopyrightYear
% (defaulted to 200X) and the ACM Copyright Data
% (defaulted to X-XXXXX-XX-X/XX/XX).
% e.g.
% \CopyrightYear{2007} will cause 2007 to appear in the copyright line.
% \crdata{0-12345-67-8/90/12} will cause 0-12345-67-8/90/12 to appear in the copyright line.
%
% ---------------------------------------------------------------------------------------------------------------
% This .tex source is an example which *does* use
% the .bib file (from which the .bbl file % is produced).
% REMEMBER HOWEVER: After having produced the .bbl file,
% and prior to final submission, you *NEED* to 'insert'
% your .bbl file into your source .tex file so as to provide
% ONE 'self-contained' source file.
%
% ================= IF YOU HAVE QUESTIONS =======================
% Questions regarding the SIGS styles, SIGS policies and
% procedures, Conferences etc. should be sent to
% Adrienne Griscti (griscti@acm.org)
%
% Technical questions _only_ to
% Gerald Murray (murray@hq.acm.org)
% ===============================================================
%
% For tracking purposes - this is V2.0 - May 2012

%\documentclass[conference]{IEEEtran}
\documentclass[sigconf]{acmart}
%\documentclass{sig-alternate-05-2015}
%\documentclass[preprint,nocopyrightspace,numbers]{sigplanconf}

%\usepackage{xifthen}
%\usepackage{multirow}
%\usepackage{graphicx}
%\usepackage[cmex10]{amsmath}
\usepackage{verbatim}
%\usepackage{url}
\usepackage{amsmath}
%\usepackage[usenames, dvipsnames]{color}
%\makeatletter
%\patchcmd{\maketitle}{\@copyrightspace}{}{}{}
%\makeatother
%\usepackage[hidelinks]{hyperref} 
%\usepackage{etoolbox}
%\usepackage{array}% for fancier tabular
%\usepackage{mathtools}
\usepackage{listings}
\usepackage{xcolor}
%\usepackage[T1]{fontenc}
%\usepackage{bera}
\usepackage{caption}
\usepackage{setspace}

%\makeatletter
%\patchcmd{\maketitle}{\@copyrightspace}{}{}{}
%\makeatother

\usepackage{subcaption}
\PassOptionsToPackage{hyphens}{url}
%\usepackage[hyphens]{url}
%\usepackage{hyperref}
\usepackage{booktabs}
\usepackage{pdfpages}
\setcopyright{none}
%\acmConference[]{}{}{}
%\acmDOI{}
%\acmISBN{}
%\acmPrice{}
%\acmYear{}
\settopmatter{printacmref=false, printfolios=false}
\renewcommand\footnotetextcopyrightpermission[1]{} % removes footnote with conference information in first column
\pagestyle{plain} % removes running headers

\usepackage[font=small,labelfont=bf]{caption}

\begin{document}

% Copyright
%\setcopyright{acmcopyright}
%\setcopyright{acmlicensed}
%\setcopyright{rightsretained}
%\setcopyright{usgov}
%\setcopyright{usgovmixed}
%\setcopyright{cagov}
%\setcopyright{cagovmixed}


% DOI
%\doi{10.475/123_4}

% ISBN
%\isbn{123-4567-24-567/08/06}

%Conference
%\conferenceinfo{PLDI '13}{June 16--19, 2013, Seattle, WA, USA}

%\acmPrice{\$15.00}

%
% --- Author Metadata here ---
%\conferenceinfo{WOODSTOCK}{'97 El Paso, Texas USA}
%\CopyrightYear{2007} % Allows default copyright year (20XX) to be over-ridden - IF NEED BE.
%\crdata{0-12345-67-8/90/01}  % Allows default copyright data (0-89791-88-6/97/05) to be over-ridden - IF NEED BE.
% --- End of Author Metadata ---

%\title{Application-Level Modeling to Analyze Application Resiliency to Transient Faults}
%\title{High Performance Data Persistency for HPC without Checkpointing}
%\title{High Performance Persistent Memory for HPC}
\title{High Performance Data Persistence in Non-Volatile Memory \\ for Resilient High Performance Computing}
%\titlenote{A full version of this paper is available as
%\textit{Author's Guide to Preparing ACM SIG Proceedings Using
%\LaTeX$2_\epsilon$\ and BibTeX} at
%\texttt{www.acm.org/eaddress.htm}}}
\author{Yingchao Huang}
\affiliation{%
  \institution{University of California, Merced}
}
\email{yhuang46@ucmerced.edu}

\author{Kai Wu}
\affiliation{
	\institution{University of California, Merced}
}
\email{kwu42@ucmerced.edu}

\author{Dong Li}
\affiliation{
	\institution{University of California, Merced}
}
\email{dli35@ucmerced.edu}

\begin{comment}
\begin{abstract}
Checkpoint is the most common method to enable resilient HPC. However, using checkpoint, we face dilemmas between HPC resilience, recomputation cost, and checkpoint cost. The fundamental reason that accounts for the dilemmas is the cost of data copying inherent in checkpoint. In this paper we explore how to build resilient HPC with emerging non-volatile memory (NVM) as main memory. We introduce a variety of schemes and optimization techniques to leverage high performance and non-volatility of NVM to establish a consistent state for application critical data objects as the traditional checkpoint mechanism. We demonstrate that using NVM to implement traditional checkpoint and address the dilemmas is not feasible because of large data copying overhead, even though NVM is expected to have superior performance. We reveal that NVM is feasible to establish data persistency much more frequently than the traditional checkpoint with ignorable runtime overhead, which fundamentally addresses the dilemmas rooted in checkpoint.
\end{abstract}
\end{comment}

\begin{abstract}
Resilience is a major design  goal  for HPC. Checkpoint is the most common method to enable resilient HPC.  Checkpoint periodically saves critical data objects to non-volatile storage to enable data persistence. However, using checkpoint, we face dilemmas between resilience, recomputation and checkpoint cost. The reason that accounts for the dilemmas is the cost of data copying inherent in checkpoint. In this paper we explore how to build resilient HPC with non-volatile memory (NVM) as main memory and address the dilemmas. We introduce a variety of optimization techniques that leverage high performance and non-volatility of NVM to enable high performance data persistence for data objects in applications. With NVM we avoid data copying; we optimize cache flushing needed to ensure consistency between caches and NVM. We demonstrate that using NVM is feasible to establish data persistence frequently with small overhead (4.4\% on average) to achieve highly resilient HPC and minimize recomputation.
%which addresses the dilemmas rooted in checkpoint.
\vspace{-15pt}
\end{abstract}

\maketitle

\input text/intro   %1page
\input text/background    %0.5page
\input text/prelim_design  %1page
\input text/design	 %3pages
%\input text/impl	 %1page
\input text/evaluation   %2pages
\input text/related_work  %1
\input text/conclusions   %0.5page

% The following two commands are all you need in the
% initial runs of your .tex file to
% produce the bibliography for the citations in your paper.
\bibliographystyle{ACM-Reference-Format}
\bibliography{li}  % sigproc.bib is the name of the Bibliography in this case
% You must have a proper ".bib" file
%  and remember to run:
% latex bibtex latex latex
% to resolve all references

%\includepdf[pages=-]{figures/artifacts-appendix-sc17.pdf}
\end{document}
