\section{ELT Traces}
\subsection{Trace of ELT Matrices}
\begin{defn}
Let $\R$ be a commutative ELT ring, and take $A\in \left(\overline{\R}\right)^{n\times n}$, $A=\left(a_{i,j}\right)$. The \textbf{trace} of $A$ is
$$\tr\left(A\right)=\sum_{i=1}^{n}a_{i,i}$$
\end{defn}

\begin{lem}
The ELT trace satisfies the following relations:
\begin{enumerate}
\item $\forall A,B\in \left(\overline{\R}\right)^{n\times n}:\tr\left(A+B\right)=\tr\left(A\right)+\tr\left(B\right)$.
\item $\forall\alpha\in\overline{\R}\;\forall A\in \left(\overline{\R}\right)^{n\times n}:\tr\left(\alpha A\right)=\alpha\tr\left(A\right)$.
\item $\forall A,B\in \left(\overline{\R}\right)^{n\times n}:\tr\left(AB\right)=\tr\left(BA\right)$.
\end{enumerate}
\end{lem}
\begin{proof}
These properties can be proved just as the classical theory.
\end{proof}

\subsection{ELT Nilpotent Matrices}
\begin{defn}
Let $\R$ be a commutative ELT ring. A matrix $A\in \left(\overline{\R}\right)^{n\times n}$ is called \textbf{ELT nilpotent}, if there exists $m\in\N$ such that $A^m\in\zeroset{\left(\overline{\R}\right)^{n\times n}}$.
\end{defn}

Similarly to the classical theory, one would expect that the trace of a nilpotent matrix would be of layer zero; however, this is wrong. For this
reason, we define the essential trace in the next subsubsection.

\begin{example}
Let $\R=\ELT{\mathbb{R}}{\mathbb{C}}$, and consider the following matrix: $\displaystyle{A=\begin{pmatrix}\layer{0}{1} & \layer{1}{0}\\
\layer{0}{0} & \layer{0}{1}
\end{pmatrix}}$.
Then $\tr\left(A\right)=\layer{0}{2}$, while
$$A^{2}=\begin{pmatrix}\layer{0}{1} & \layer{1}{0}\\
\layer{0}{0} & \layer{0}{1}
\end{pmatrix}\begin{pmatrix}\layer{0}{1} & \layer{1}{0}\\
\layer{0}{0} & \layer{0}{1}
\end{pmatrix}=\begin{pmatrix}\layer{1}{0} & \layer{1}{0}\\
\layer{0}{0} & \layer{1}{0}
\end{pmatrix}.$$
Therefore, ELT nilpotent matrices don't necessarily  have zero-layered trace.
\end{example}

Another interesting example is an ELT nilpotent matrix, whose determinant is not of layer zero.
\begin{example}
Let $\R$ be a commutative ELT ring, and take $a,b,c,d\in\R$ such that
$$\t\left(a^2\right)<\t\left(bc\right)<\t\left(d^2\right)$$
and $s\left(d\right)=0$. Consider the matrix $\displaystyle{A=\begin{pmatrix}a&b\\c&d\end{pmatrix}}$. We have
$$A^2=\begin{pmatrix}a^2+bc&ab+bd\\ac+cd&bc+d^2\end{pmatrix}=\begin{pmatrix}bc&bd\\cd&d^2\end{pmatrix}$$
and
$$A^3=\begin{pmatrix}abc+bcd&abd+bd^2\\bc^2+cd^2&bcd+d^3\end{pmatrix}=\begin{pmatrix}bcd&bd^2\\cd^2&d^3\end{pmatrix}$$
which is of layer zero, since $s\left(d\right)=0$.

But we may choose $a,b,c,d$ such that $\t\left(bc\right)>\t\left(ad\right)$ and $s\left(bc\right)\neq0$; in that case, $A$ is quasi-invertible and ELT nilpotent.
\end{example}

\subsection{The Essential Trace}\label{sec:etr}

Before defining the new notion of trace, we give an important definition, which is significant in our construction of the new trace.
\begin{defn}
Let $\R$ be a commutative ELT ring, and let $\displaystyle{p\left(\lambda\right)=\sum_{i=1}^nh_i\in\overline{\R}\left[\lambda\right]}$ be an ELT polynomial, where each $h_i$ is a monomial. For a monomial $h$, define $\displaystyle{p_h\left(\lambda\right)=\sum_{h_i\neq h}h_i}$.
\begin{enumerate}
\item The monomial $h$ is called \textbf{inessential at a point $a\in\R$}, if $p\left(a\right)=p_h\left(a\right)$ and $\t\left(h\left(a\right)\right)<\t\left(p\left(a\right)\right)$.
    If $h$ is inessential at every point of $\R$, it is called \textbf{inessential}.
\item The monomial $h$ is called \textbf{essential at a point $a\in\R$}, if $p\left(a\right)=h\left(a\right)$ and $\t\left(p_h\left(a\right)\right)<\t\left(p\left(a\right)\right)$.
    If $h$ is essential at some point of $\R$, it is called \textbf{essential}.
\item The monomial $h$ is called \textbf{quasi-essential at a point $a\in\R$}, if it is neither inessential at $a$ nor essential at $a$.
    If $h$ is quasi-essential at some point of $\R$, it is called \textbf{quasi-essential}.
\end{enumerate}
\end{defn}

Throughout the rest of the section, we need a stronger assumption on our ELT algebras. We require them to be \textbf{divisible ELT fields}, that is ELT algebras of the form $\R=\ELT{\F}{\L}$, where $\F$ is a divisible group and $\L$ is a field.

\begin{defn}
Let $\R$ be a divisible ELT field, and let $A\in \left(\overline{\R}\right)^{n\times n}$. We define
\begin{eqnarray*}
L\left(A\right) & = & \left\{ \ell\ge1\middle|\forall k\leq n:\,\frac{\t\left(c_{\ell}\right)}{\ell}\ge\frac{\t\left(c_k\right)}{k}\right\} \\
\mu\left(A\right) & = & \min\left(L\left(A\right)\right)
\end{eqnarray*}
We write $\mu$ for $\mu\left(A\right)$, if $A$ is given.
\end{defn}

\begin{lem}\label{lem:etr-justification}
Let $\R$ be a divisible ELT field, and let $p\left(\lambda\right)=\lambda^{n}+{\displaystyle \sum_{i=1}^{n}\alpha_{i}\lambda^{n-i}}\in\overline{\R}\left[\lambda\right]$ be an ELT polynomial. Then the first monomial after $\lambda^{n}$ that is not inessential is $\alpha_{\mu}\lambda^{n-\mu}$.
\end{lem}
\begin{proof}
We need to find the monomial for which the intersection between $\lambda^{n}$ and $\alpha_{\ell}\lambda^{n-\ell}$ is maximal (in the sense that its tangible value is maximal).

First, we compute the tangible value of the intersection:
$$\lambda^{n}=\alpha_{\ell}\lambda^{n-\ell}\Rightarrow\t\left(\lambda\right)=\frac{\t\left(\alpha_\ell\right)}{\ell}$$
The tangible value of the value of the polynomial at that point is $\frac{n\t\left(\alpha_\ell\right)}{\ell}$.

So, if
$$\forall k\leq n:\,\frac{\t\left(\alpha_\ell\right)}{\ell}\ge\frac{\t\left(\alpha_k\right)}{k}$$
$\ell$ satisfies our conditions. Take such $\ell$ minimal, which is $\mu$, and we are done.
\end{proof}

\begin{rem}
If $\left|L\left(A\right)\right|\ge2$, then $\alpha_{\mu}\lambda^{n-\mu}$ is only quasi-essential in $p_A\left(\lambda\right)$.
\end{rem}

\begin{defn}
Let $\R$ be a divisible ELT field, and let $A\in \left(\overline{\R}\right)^{n\times n}$ be a matrix. Assume $p_{A}\left(\lambda\right)=\lambda^{n}+{\displaystyle \sum_{i=1}^{n}\alpha_{i}\lambda^{n-i}}$. $\alpha_{\mu}$ is called the \textbf{dominant characteristic coefficient}. The \textbf{essential trace} of $A$, denoted $\etr\left(A\right)$, is given by the formula
$$\etr\left(A\right)=\left\{\begin{matrix}\tr\left(A\right)&\minus\tr\left(A\right)\lambda^{n-1}\textnormal{ is essential in }p_A\left(\lambda\right)\\\layer{\left(\frac{\t\left(\alpha_\mu\right)}{\mu}\right)}{0}&\textnormal{Otherwise}\end{matrix}\right.$$
\end{defn}

\begin{lem}\label{lem:tr-zero-implies-etr-zero}
If $s\left(\tr\left(A\right)\right)=0$, then $s\left(\etr\left(A\right)\right)=0$.
\end{lem}
\begin{proof}
If $\minus \tr\left(A\right)\lambda^{n-1}$ is essential in $p_A\left(\lambda\right)$, then $\etr\left(A\right)=\tr\left(A\right)$.
which is of layer zero. Otherwise,~$\minus \tr\left(A\right)\lambda^{n-1}$ is not essential in $p_A\left(\lambda\right)$, and thus, by the definition of essential trace, $s\left(\etr\left(A\right)\right)=0$.
\end{proof}

\begin{defn}
Let $A\in \left(\overline{\R}\right)^{n\times n}$ be a matrix over a commutative ELT ring $\R$. A \textbf{path from $i$ to~$j$} is an expression of the form
$$s=a_{i,i_{2}}a_{i_{2},i_{3}}\dots a_{i_{k},j}$$
where $1\leq i_{1},\dots,i_{k+1}\leq n$. If $i=j$, we call it a \textbf{multicycle}. The \textbf{length} of $s$ is $\left|s\right|=k$. The \textbf{tangible average value} of $s$ is $\displaystyle{\frac{\t\left(s\right)}{\left|s\right|}}$. A \textbf{simple cycle} is a multicycle from $i$ to $i$, such that $i_{j}\neq i_{j'}$ for $j\neq j'$ (with $i_{1}=i$).
\end{defn}

\begin{fact}
Every multicycle can be written as a product of simple cycles.
\end{fact}

\begin{lem}\label{lem:i,j-ele-of-A^k}
The $\left(i,j\right)$ element in $A^{k}$ is the sum of all paths from $i$ to $j$. That is,
$$\left(A^{k}\right)_{i,j}=\sum_{1\leq i_{2},\dots,i_{k}\leq n}a_{i,i_{2}}a_{i_{2},i_{3}}\dots a_{i_{k},j}$$
\end{lem}
\begin{proof}
By induction on $k$, where the case $k=1$ is clear. If the assertion is true for some $k$, then
$$\left(A^{k+1}\right)_{i,j}=\sum_{\ell=1}^{n}a_{i,\ell}A_{\ell,j}^{k}=\sum_{j=1}^{n}\,\sum_{1\leq i_{2},\dots,i_{k}\leq n}a_{i,\ell}a_{\ell,i_{2}}a_{i_{2},i_{3}}\dots a_{i_{k},j}=\sum_{1\leq i_{2},\dots,i_{k+1}\leq n}a_{i,i_{2}}a_{i_{2},i_{3}}\dots a_{i_{k+1},j}$$
\end{proof}

\begin{lem}\label{lem:coeff-of-char-poly}
The coefficient of $\lambda^{n-k}$ in $p_{A}\left(\lambda\right)$ is
$$\left(\minus\right)^{k}\sum_{1\leq i_{1},\dots,i_{k}\leq n}\,\sum_{\sigma\in S_{k}}\layer 0{\sgn\sigma}\, a_{i_{1},i_{\sigma\left(1\right)}}a_{i_{2},i_{\sigma\left(2\right)}}\dots a_{i_{k},i_{\sigma\left(k\right)}}$$
\end{lem}
\begin{proof}
We must choose $n-k$ indices from which we ``take'' $\lambda$ in the expansion of $\det\left(\lambda I+\minus A\right)$; we are left with a $k\times k$ submatrix, with rows $i_{1},\dots,i_{k}$ from $A$. Its determinant is the inner sum.
\end{proof}

\begin{lem}\label{lem:Mulcyc-contr-to-etr}
Any multicycle contributing to the dominant characteristic coefficient must be a simple cycle.
\end{lem}
\begin{proof}
Otherwise, assume it is not a simple cycle. Since it can be written as a product of simple cycles, at least one of which, $s$, would have $\frac{\t\left(s\right)}{\left|s\right|}\ge\etr\left(A\right)$, and a shorter length. Thus, $s$ would give a dominant characteristic coefficient of lower degree, in contradiction to our assumption.
\end{proof}

\begin{lem}
If $s\left(\etr\left(A+B\right)\right)\neq0$, then $\etr\left(A+B\right)=\etr\left(A\right)+\etr\left(B\right)$.
\end{lem}
\begin{proof}
The assumption can only happen if $\minus\tr\left(A+B\right)\lambda^{n-1}$ is essential in $p_{A+B}\left(\lambda\right)$, and $\etr\left(A+B\right)=\tr\left(A+B\right)$.

The multicycles of $A+B$ are products of sums of elements from $A$ and from $B$. In particular, every multicycle of $A$ and of $B$ is a part of a multicycle of $A+B$.

We know that $\tr\left(A+B\right)=\tr\left(A\right)+\tr\left(B\right)$; if $\t\left(\tr\left(A+B\right)\right)=\t\left(\tr\left(A\right)\right)$, then $\t\left(\tr\left(A\right)\right)>$every average value of a multicycle of $A$; so $\etr\left(A\right)=\tr\left(A\right)$. We are almost finished:
\begin{casenv}
\item If $\t\left(\tr\left(A+B\right)\right)=\t\left(\tr\left(A\right)\right)=\t\left(\tr\left(B\right)\right)$, then $\etr\left(A\right)=\tr\left(A\right)$, $\etr\left(B\right)=\tr\left(B\right)$, and $$\etr\left(A+B\right)=\tr\left(A+B\right)=\tr\left(A\right)+\tr\left(B\right)=\etr\left(A\right)+\etr\left(B\right)$$

\item If $\t\left(\tr\left(A+B\right)\right)=\t\left(\tr\left(A\right)\right)>\t\left(\tr\left(B\right)\right)$, then $\etr\left(A\right)=\tr\left(A\right)$, and $$\etr\left(A+B\right)=\tr\left(A+B\right)=\tr\left(A\right)+\tr\left(B\right)=\tr\left(A\right)=\etr\left(A\right)$$
    In particular, $\t\left(\etr\left(A\right)\right)>\t\left(\etr\left(B\right)\right)$; otherwise, there would have been a multicycle from $B$ with $\t\left(\textnormal{average value}\right)\ge\t\left(\etr\left(A\right)\right)=\t\left(\etr\left(A+B\right)\right)$, and thus $s\left(\etr\left(A+B\right)\right)=0$, which is a contradiction. So
    $$\etr\left(A+B\right)=\etr\left(A\right)=\etr\left(A\right)+\etr\left(B\right)$$
\end{casenv}
\end{proof}

\begin{example}
In general, it is not true that $\etr\left(A+B\right)\vDash\etr\left(A\right)+\etr\left(B\right)$. For example, take in $\left(\ELT{\mathbb{R}}{\mathbb{C}}\right)^{2\times 2}$ the following matrices: $A=\left(\begin{matrix}\layer{0}{1}&\layer{0}{1}\\-\infty&\layer{0}{1}\end{matrix}\right)$, $B=A^t$. Then $\etr\left(A\right)=\etr\left(B\right)=\layer{0}{2}$. However, $A+B=\left(\begin{matrix}\layer{0}{2}&\layer{0}{1}\\\layer{0}{1}&\layer{0}{2}\end{matrix}\right)$, and $p_{A+B}\left(\lambda\right)=\lambda^2+\layer{0}{-4}\lambda+\layer{0}{3}$. The monomial $\layer{0}{-4}\lambda$ is quasi-essential, and thus $\etr\left(A+B\right)=\layer{0}{0}$.
\end{example}

\begin{lem}
$\etr\left(AB\right)=\etr\left(BA\right)$.
\end{lem}
\begin{proof}
By \Lref{lem:Mulcyc-contr-to-etr}, it is enough to check only simple cycles. Assume
$$\left(AB\right)_{i_{1},i_{2}}\left(AB\right)_{i_{2},i_{3}}\dots\left(AB\right)_{i_{k},i_{1}}$$
contributes to the dominant characteristic coefficient, where $i_{j}\neq i_{j'}$ for $j\neq j'$. So there are $\ell_{1},\dots,\ell_{k}$ such that
$$s=a_{i_{1},\ell_{1}}b_{\ell_{1},i_{2}}a_{i_{2},\ell_{2}}b_{\ell_{2},i_{3}}\dots a_{i_{k},\ell_{k}}b_{\ell_{k},i_{1}}$$
contributes to the dominant characteristic coefficient, i.e. $\t\left(s\right)=\etr\left(AB\right)$.

If $\ell_{j}=\ell_{j'}$ for $j<j'$, write
\begin{eqnarray*}
s & = & \left(a_{i_{1},\ell_{1}}b_{\ell_{1},i_{2}}\dots a_{i_{j},\ell_{j}}b_{\ell_{j},i_{j'+1}}\dots a_{i_{k},\ell_{k}}b_{\ell_{k},i_{1}}\right)\left(b_{\ell_{j},i_{j+1}}a_{i_{j+1},\ell_{j+1}}\dots a_{i_{j'},\ell_{j'}}\right)=\\
 & = & \underbrace{\left(a_{i_{1},\ell_{1}}b_{\ell_{1},i_{2}}\dots a_{i_{j},\ell_{j}}b_{\ell_{j},i_{j'+1}}\dots a_{i_{k},\ell_{k}}b_{\ell_{k},i_{1}}\right)}_{s_{1}}\underbrace{\left(a_{i_{j+1},\ell_{j+1}}\dots a_{i_{j'},\ell_{j'}}b_{\ell_{j},i_{j+1}}\right)}_{s_{2}}
\end{eqnarray*}
Since $\t\left(s\right)=\t\left(\etr\left(AB\right)\right)$, $\t\left(s_{1}\right)\ge\t\left(\etr\left(AB\right)\right)$ or $\t\left(s_{2}\right)\ge\t\left(\etr\left(AB\right)\right)$. Since $s_{1}$ and $s_{2}$ are shorter, we get a contradiction.

So $\ell_{j}\neq\ell_{j'}$ for $j\neq j'$. So
$$s=b_{\ell_{1},i_{2}}a_{i_{2},\ell_{2}}b_{\ell_{2},i_{3}}\dots a_{i_{k},\ell_{k}}b_{\ell_{k},i_{1}}a_{i_{1},\ell_{1}}$$
is a part of the simple cycle
$$\left(BA\right)_{\ell_{1},\ell_{2}}\left(BA\right)_{\ell_{2},\ell_{3}}\dots\left(BA\right)_{\ell_{k},\ell_{1}}$$
in $BA$. So every simple cycle contributing to $\etr\left(AB\right)$ also contributes to $\etr\left(BA\right)$.

By symmetry, the opposite is true as well; so $\etr\left(AB\right)=\etr\left(BA\right)$.
\end{proof}

\begin{lem}\label{lem:multicycle-ge-tr}
If there is a multicycle $s=a_{i_1,i_{2}}a_{i_{2},i_{3}}\dots a_{i_{k},i_1}$ of length $k\ge 2$ such that $$\t\left(\tr\left(A\right)^k\right)<\t\left(s\right)$$
then $\minus\tr\left(A\right)\lambda^{n-1}$ is not essential in $p_{A}\left(\lambda\right)$ (meaning, it is either inessential or quasi-essential). In other words, $\mu\left(A\right)\ge 2$.
\end{lem}
\begin{proof}
Firstly, we may assume that $k\leq n$; otherwise, write $s$ as a product of multicycle, $s=s_1\dots s_j$. Since
$$\t\left(\tr\left(A\right)^k\right)<\t\left(s\right)=\sum_{\ell=1}^j\t\left(s_\ell\right)$$
At least one $s_\ell$ should satisfy $\t\left(\tr\left(A\right)^{\left|s_\ell\right|}\right)<\t\left(s_\ell\right)$, and we may replace $k$ by $\ell$.

Therefore, we assume $k\leq n$. Write $p_{A}\left(\lambda\right)=\lambda^{n}+{\displaystyle \sum_{i=1}^{n}}\alpha_{i}\lambda^{n-i}$. By \Lref{lem:coeff-of-char-poly},
$$c_k=a_{i_1,i_{2}}a_{i_{2},i_{3}}\dots a_{i_{k},i_1}+\cdots=s+\cdots$$
In particular, $\t\left(c_k\right)\ge\t\left(s\right)$. Also recall that $c_1=\minus\tr\left(A\right)$. Therefore,
$$\t\left(c_1\right)=\t\left(\tr\left(A\right)\right)=\frac{\t\left(\tr\left(A\right)^k\right)}{k}< \frac{\t\left(s\right)}{k}\leq\frac{\t\left(c_k\right)}{k}$$
An thus $\mu\left(A\right)\ge k\ge 2$, as required.
\end{proof}

\begin{lem}\label{lem:ELT-nilpotent-etr}
If $A$ is ELT nilpotent, and if $s\left(\tr\left(A\right)\right)\neq0$, then $\minus\tr\left(A\right)\lambda^{n-1}$ is not essential in $p_{A}\left(\lambda\right)$. In other words, $\minus\tr\left(A\right)\lambda^{n-1}$ is either quasi-essential or inessential in $p_{A}\left(\lambda\right)$.
\end{lem}
\begin{proof}
For this proof, write $a\leq_{\t}b$ if $\t\left(a\right)\leq\t\left(b\right)$, $a<_{\t}b$ if $\t\left(a\right)<\t\left(b\right)$, and $a\equiv_{\t}b$ if $\t\left(a\right)=\t\left(b\right)$.

We take some $a_{i,i}$ such that $a_{i,i}\equiv_{\t}\tr\left(A\right)$ and $s\left(a_{i,i}\right)\ne0$; without loss of generality, let $i=1$. Write $p_{A}\left(\lambda\right)=\lambda^{n}+{\displaystyle \sum_{i=1}^{n}}\alpha_{i}\lambda^{n-i}$.

By \Lref{lem:i,j-ele-of-A^k}, $\left(A^{k}\right)_{1,1}=a_{1,1}^{k}+\cdots$. Take $k$ minimal such that $s\left(A^{k}\right)=0$. In particular, $s\left(\left(A^{k}\right)_{1,1}\right)=0$, so we have two cases:

\begin{casenv}
\item There is a multicycle $s=a_{1,i_{2}}a_{i_{2},i_{3}}\dots a_{i_{k},1}$ with $a_{1,1}^{k}<_{\t}s$. Firstly, we may assume that $k\leq n$; otherwise, write $s$ as a product of multicycle, $s=s_1\dots s_j$. Since
    $$\t\left(\tr\left(A\right)^k\right)<\t\left(s\right)=\sum_{\ell=1}^j\t\left(s_\ell\right)$$
    At least one $s_\ell$ should satisfy $\t\left(\tr\left(A\right)^{\left|s_\ell\right|}\right)<\t\left(s_\ell\right)$, and we may replace $k$ by $\ell$.

    Therefore, we assume $k\leq n$. Write $p_{A}\left(\lambda\right)=\lambda^{n}+{\displaystyle \sum_{i=1}^{n}}\alpha_{i}\lambda^{n-i}$. By \Lref{lem:coeff-of-char-poly},
    $$c_k=a_{i_1,i_{2}}a_{i_{2},i_{3}}\dots a_{i_{k},i_1}+\cdots=s+\cdots$$
    In particular, $\t\left(c_k\right)\ge\t\left(s\right)$. Also recall that $c_1=\minus\tr\left(A\right)$. Therefore,
    $$\t\left(c_1\right)=\t\left(\tr\left(A\right)\right)=\frac{\t\left(\tr\left(A\right)^k\right)}{k}< \frac{\t\left(s\right)}{k}\leq\frac{\t\left(c_k\right)}{k}$$
    An thus $\mu\left(A\right)\ge k\ge 2$, as required.

\item There is no multicycle with $a_{1,1}^{k}<_{\t}a_{1,i_{2}}a_{i_{2},i_{3}}\dots a_{i_{k},1}$. Then there has to be a multicycle for which $a_{1,1}^{k}\equiv_{\t}a_{1,i_{2}}a_{i_{2},i_{3}}\dots a_{i_{k},1}=s$ (since $s\left(\left(A^{k}\right)_{1,1}\right)=0$, and $a_{1,1}^k$ is a summand in the sum defining $\left(A^{k}\right)_{1,1}$).

    By writing $s$ as a product of simple cycles, we may assume that each simple cycle $a_{j_{1},j_{2}}a_{j_{2},j_{3}}\dots a_{j_{\ell},j_{1}}\equiv_{\t}a_{1,1}^{\ell}$ (Otherwise, we are finished by the first case).

    Since $\ell<n$, we get that $\left|L\left(A\right)\right|\ge2$, meaning $\minus\tr\left(A\right)\lambda^{n-1}$ is quasi-essential in $p_{A}\left(\lambda\right)$.
\end{casenv}
\end{proof}

\begin{cor}\label{cor:etr-of-nilpotent}
If $A$ is ELT nilpotent, then $s\left(\etr\left(A\right)\right)=0$.
\end{cor}
\begin{proof}
There are two cases:
\begin{enumerate}
\item If $s\left(\tr\left(A\right)\right)=0$, then $s\left(\etr\left(A\right)\right)=0$ by \Lref{lem:tr-zero-implies-etr-zero}.
\item Otherwise, $s\left(\tr\left(A\right)\right)\neq 0$; but then $s\left(\etr\left(A\right)\right)=0$ by \Lref{lem:ELT-nilpotent-etr}.
\end{enumerate}
\end{proof}
