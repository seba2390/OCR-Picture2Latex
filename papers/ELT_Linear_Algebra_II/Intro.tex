\section{Introduction}

Tropical linear algebra, also known as Max-Plus linear algebra, has been studied for more than 50 years (ref.~\cite{B}). While tropical geometry mainly deals with geometric combinatorial problems, tropical linear algebra deals with algebraic non-linear combinatorial problems (for instance, the assignment problem \cite{K}). Tropical linear algebra may also be used as a mean to study the tropical algebraic geometry (for instance, the tropical resultant). Notable work in this field can be found at \cite{B}, \cite{DSS}, \cite{IR4}, \cite{IR3} and \cite{S}.\\

In our previous paper (\cite{BS}) we introduced a new structure, which we call exploded layered tropical algebra (or ELT algebra for short). This structure is a generalization of the work of Izhakian and Rowen (\cite{IR1}), and is similar to Parker's exploded structure (\cite{PR}). The layers enable us to use ``classical language'' even when dealing with tropical questions.\\

Our work in this paper can be divided into two main parts. The first one uses the theory of semirings with a negation map to study the ELT structure. We formulate and prove an ELT version of the transfer principles written in \cite{Akian2008}, and use them to study ELT matrix theory, such as the ELT adjoint matrix (\Tref{thm:Trans-Princ} and \Tref{thm:Trans-Princ-Surpass}).\\

The second part of our work deals with a new notion of trace. Whereas the trace can be defined as in the classical theory, it lacks some important properties in the ELT theory. For example, the trace of an ELT nilpotent matrix need not be of layer zero. We define the essential trace of an ELT matrix (\sSref{sec:etr}) to deal with such cases.\\

\subsection{ELT Algebras}

\begin{defn}
Let $\L$ be a semiring, and $\F$ a totally ordered semigroup. An \textbf{ELT algebra} is the pair $\R=\ELT{\F}{\L}$, whose elements are denoted $\layer a{\ell}$ for $a\in\F$ and $\ell\in\L$, together with the semiring (without zero) structure:
\begin{enumerate}
\item $\layer{a_{1}}{\ell_{1}}+\layer{a_{2}}{\ell_{2}}:=\begin{cases}
\layer{a_{1}}{\ell_{1}} & a_{1}>a_{2}\\
\layer{a_{2}}{\ell_{2}} & a_{1}<a_{2}\\
\layer{a_{1}}{\ell_{1}+_\L\ell_{2}} & a_{1}=a_{2}
\end{cases}$.
\item $\layer{a_{1}}{\ell_{1}}\cdot\layer{a_{2}}{\ell_{2}}:=\layer{\left(a_{1}+_\F a_{2}\right)}{\ell_{1}\cdot_\L\ell_{2}}$.
\end{enumerate}
We write . For $\layer{a}{\ell}$, $\ell$ is called the \textbf{layer}, whereas $a$ is called the \textbf{tangible value}.
\end{defn}

ELT algebras originate from \cite{PR}, and are also discussed in \cite{Sheiner2015}.\\

Let $\R$ be an ELT algebra. We write $s:\R\rightarrow\L$ for the projection on the first component (the \textbf{sorting map}):
$$s\left(\layer a{\ell}\right)=\ell$$
We also write $\t:\R\rightarrow\F$ for the projection on the second component:
$$\t\left(\layer a{\ell}\right)=a$$
We denote the \textbf{zero-layer subset}
$$\zeroset{\R}=\left\{\alpha\in \R\middle| s\left(\alpha\right)=0\right\}$$
and
$$\R^{\times}=\left\{\alpha\in\R\middle|s\left(\alpha\right)\neq 0\right\}=\R\setminus\zeroset{\R}$$\\

We note some special cases of ELT algebras.
\begin{example}
Let $\left(G,\cdot\right)$ be a totally ordered group. We denote by $G_{\max}$ the max-plus algebra defined over $G$, i.e.\ the set $G$ endowed with the operation
$$a\oplus b=\max\left\{a,b\right\},\;\;a\odot b=a\cdot b.$$
Then $G_{\max}$ is equivalent to the trivial ELT algebra with $\F=G$ and $\L=\left\{1\right\}$.
\end{example}

\begin{example}
Zur Izhakian's supertropical algebra (\cite{IZ}) is equivalent to an ELT algebra with a layering set $\L=\left\{1,\infty\right\}$, where
$$1+1=\infty,\;\;1+\infty=\infty+1=\infty,\;\;\infty+\infty=\infty$$
and
$$1\cdot 1=1,\;\;1\cdot\infty=\infty\cdot 1=\infty,\;\;\infty\cdot\infty=\infty.$$

The supertropical "ghost" elements $a^\nu$ correspond to $\layer{a}{\infty}$ in the ELT notation, whereas the tangible elements $a$ correspond to $\layer{a}{1}$.
\end{example}

We define a partial order relation $\vDash$ on $\R$ in the following way:
$$x\vDash y\Longleftrightarrow \exists z\in\zeroset{\R}:x=y+z$$

\begin{lem}[{\cite[Lemma 0.4]{BS}}]
$\vDash$ is a partial order relation on $\R$.
\end{lem}

Let us point out some important elements in any ELT algebra $\R$:
\begin{enumerate}
\item $\layer{0}{1}$, which is the multiplicative identity of $\R$.
\item $\layer{0}{0}$, which is idempotent for both operations of $\R$.
\item $\layer{0}{-1}$, which has the role of ``$-1$'' in our theory.
\end{enumerate}

Note that $\zero\cdot\layer{a}{\ell}=\layer{a}{0}$. Therefore, $\zeroset{\R}=\zero\,\R$. In particular, $\zeroset{\R}$ is an ideal of $\R$.\\

Throughout this paper, unless otherwise noted, we work under more general assumptions than in~\cite{BS}. Out underlying ELT algebras $\R=\ELT{\F}{\L}$ will be \textbf{commutative ELT rings}, meaning that $\F$ is an abelian group, and $\L$ is a commutative ring.

\subsection{The Element $-\infty$}\label{sub:the-element-minf}

As in the tropical algebra, ELT algebras lack an additive identity. Therefore, we adjoin a formal element to the ELT algebra $\R$, denoted by $-\infty$, which satisfies $\forall\alpha\in\R$:
$$\begin{array}{c}
-\infty+\alpha=\alpha+-\infty=\alpha\\
-\infty\cdot\alpha=\alpha\cdot-\infty=-\infty
\end{array}$$
We also define $s\left(-\infty\right)=0$. We denote $\overline{\R}=\R\cup\left\{-\infty\right\}$.\\

We note that $\overline{\R}$ is now a semiring, with the following property:
$$\alpha+\beta=-\infty\Longrightarrow\alpha=\beta=-\infty$$
Such a semiring is called an \textbf{antiring}. Antirings are dealt with in \cite{Tan2007} and \cite{Dolzan2008}.

\subsection{Non-Archimedean Valuations and Puiseux Series}

We recall the definition of a (non-Archimedean) valuation (see \cite{Efrat2006} and \cite{Tignol2015}).

\begin{defn}
Let $K$ be a field, and let $\left(\Gamma,+,\ge\right)$ be an abelian totally ordered group. Extend $\Gamma$ to $\Gamma\cup\left\{\infty\right\}$ with $\gamma<\infty$ and $\gamma+\infty=\infty+\gamma=\infty$ for all $\gamma\in\Gamma$. A function $v:K\to\Gamma\cup\left\{\infty\right\}$ is called a \textbf{valuation}, if the following properties hold:
\begin{enumerate}
  \item $v\left(x\right)=\infty\Longleftrightarrow x=0$.
  \item $\forall x,y\in K: v\left(xy\right)=v\left(x\right)+v\left(y\right)$.
  \item $\forall x,y\in K: v\left(x+y\right)\ge\min\left\{v\left(x\right), v\left(y\right)\right\}$.
\end{enumerate}
\end{defn}

Given a valuation $v$ over a field $K$, we recall some basic properties:
\begin{enumerate}
  \item $v\left(1\right)=0$.
  \item $\forall x\in K:v\left(-x\right)=v\left(x\right)$.
  \item $\forall x\in K^{\times}:v\left(x^{-1}\right)=-v\left(x\right)$.
  \item If $v\left(x+y\right)>\min\left\{v\left(x\right), v\left(y\right)\right\}$, then $v\left(x\right)=v\left(y\right)$. (For this reason, the equality between the valuation of two elements is central in out theory.)
\end{enumerate}

One may associate with $v$ the \textbf{valuation ring}
$$\mathcal{O}_v=\left\{x\in K\middle|v\left(x\right)\ge 0\right\}$$
This is a local ring with the unique maximal ideal
$$\mathfrak{m}_v=\left\{x\in K\middle|v\left(x\right)>0\right\}$$
The quotient $k_v=\quo{\mathcal{O}_v}{\mathfrak{m}_v}$ is called the \textbf{residue field} of the valuation.\\

Let us present another key construction related to valuations. For $\gamma\in\Gamma$, let $D_{\ge\gamma}=\left\{x\in K\middle|v\left(x\right)\ge\gamma\right\}$ and $D_{>\gamma}=\left\{x\in K\middle|v\left(x\right)>\gamma\right\}$. It is easily seen that $D_{\ge\gamma}$ is an abelian additive group, and that $D_{>\gamma}$ is a subgroup of $D_{\ge\gamma}$. Note that for $\gamma=0$, $D_{\ge 0}=\mathcal{O}_v$ and $D_{>0}=\mathfrak{m}_v$. Set $D_\gamma=\quo{D_{\ge\gamma}}{D_{>\gamma}}$. The \textbf{associated graded ring} of $K$ with respect to $v$ is
$$\gr_v\left(K\right)=\bigoplus_{\gamma\in\Gamma}D_\gamma$$
Given $\gamma_1,\gamma_2\in\Gamma$, the multiplication in $K$ induces a well-defined multiplication $D_{\gamma_1}\times D_{\gamma_2}\to D_{\gamma_1+\gamma_2}$ given by
$$\left(x_1+D_{>\gamma_1}\right)\cdot\left(x_2+D_{>\gamma_2}\right)=x_1x_2+D_{>\left(\gamma_1+\gamma_2\right)}$$
This multiplication can be extended to a multiplication map in $\gr_v\left(K\right)$, endowing it with a structure of a graded ring.\\

We will now focus on Puiseux series, which is the central example for our theory. The field of \textbf{Puiseux series} with coefficients in a field $K$ and exponents in an abelian ordered group $\Gamma$ is
$$K\{\{t\}\}=\left\{\sum_{i\in I}\alpha_i t^i\middle|\alpha_i\in K, I\subseteq \Gamma\text{ is well-ordered}\right\}$$
The resulting set, equipped with the natural operations, is a field; in addition, if $K$ is algebraically closed and $\Gamma$ is divisible, then $K\{\{t\}\}$ is also algebraically closed.\\

Assuming $\Gamma$ is also totally ordered, one may define a valuation on the field of Puiseux series $v:K\{\{t\}\}\to\Gamma\cup\left\{\infty\right\}$ as follows: $v\left(0\right)=\infty$, and
$$v\left(\sum_{i\in I}\alpha_i t^i\right)=\min\left\{i\in I\middle|\alpha_i\neq 0\right\}$$

Let us examine the associated graded ring with respect to this valuation. For each $\gamma\in\Gamma$, we first claim that $D_\gamma\cong K$. Indeed, the kernel of the homomorphism $f:D_{\ge\gamma}\to K$ defined by
$$f\left(\sum_{\gamma\leq i\in I}\alpha_i t^i\right)=\alpha_\gamma$$
is precisely $D_{>\gamma}$ (since $D_{>\gamma}$ is the subgroup of $D_{\ge\gamma}$ of Puiseux series whose minimal degree is bigger than $\gamma$).

\subsection{ELT Algebras and Puiseux Series}

Let $\R=\ELT{\F}{\L}$ be an ELT algebra. In \cite{BS} we introduced the \textbf{EL tropicalization} function $\ELTrop:\L\{\{t\}\}\rightarrow\overline{\R}$, which is defined in the following way: if $x\in \L\{\{t\}\}\backslash\left\{0\right\}$ has a leading monomial $\ell t^{a}$, then
$$\ELTrop\left(x\right)=\layer{\left(-a\right)}{\ell}.$$
In addition, $\ELTrop\left(0\right)=-\infty$.

\begin{lem}[{\cite[Lemma 0.9]{BS}}]\label{lem:ELTrop-prop}
The following properties hold:
\begin{enumerate}
\item $\forall x,y\in\L\{\{t\}\}:\ELTrop\left(x\right)+\ELTrop\left(y\right)\vDash\ELTrop\left(x+y\right)$.
\item $\forall\alpha\in\L\,\forall x\in\L\{\{t\}\}:\ELTrop\left(\alpha x\right)=\layer 0{\alpha}\, \ELTrop\left(x\right)$.
\item $\forall x,y\in\L\{\{t\}\}:\ELTrop\left(x\right)\ELTrop\left(y\right)\vDash\ELTrop\left(xy\right)$.
\end{enumerate}
\end{lem}
We remark that in the case in which $\R$ is an ELT integral domain, meaning $\L$ is an integral domain, we have $\ELTrop\left(x\right)\ELTrop\left(y\right)=\ELTrop\left(xy\right)$ for all $x,y\in\L\{\{t\}\}$. \\

Let us examine the relation $x\vDash\ELTrop\left(y\right)$ a bit more deeply. If $x=\ELTrop\left(y\right)$, it means that $x$ can be lifted to a Puiseux series which has $x$ as its leading monomial. Otherwise, we have that $x$ is of layer zero, and its tangible value is bigger than the tangible value of $\ELTrop\left(y\right)$; so one may say that $x$ can also be lifted to a Puisuex series with leading coefficient $x$, where we allow it to have a zero coefficient in its leading monomial.

\subsection{Semirings with a Negation Map and ELT Rings}\label{sub:ELT-symmetrized}

Semirings need not have additive inverses to all of the elements. While some of the theory of rings can be copied ``as-is'' to semirings, there are many facts about rings which use the additive inverses of the elements. The idea of negation maps on semirings (sometimes called symmetrized semirings) is to imitate the additive inverse map. Semirings with negation maps are discussed in \cite{Akian1990}, \cite{Gaubert1992}, \cite{Gaubert1997}, \cite{Akian2008}, \cite{Akian2014}, \cite{Rowen2016}.

\begin{defn}
Let $R$ be a semiring. A map $(-):R\to R$ is a \textbf{negation map} (or a \textbf{symmetry})
if the following properties hold:
\begin{enumerate}
\item $\forall a,b\in R:(-)\left(a+b\right)=(-)a+(-)b$.
\item $(-)0_R=0_R$.
\item $\forall a,b\in R:(-)\left(a\cdot b\right)=a\cdot\left((-)b\right)=\left((-)a\right)\cdot b$.
\item $\forall a\in R:(-)\left((-)a\right)=a$.
\end{enumerate}
We say that $\left(R,(-)\right)$ is a \textbf{semiring with a negation map} (or a \textbf{symmetrized semiring}). If $(-)$ is clear from the context, we will not mention it.
\end{defn}

We give several examples of semirings with negation maps:
\begin{itemize}
\item A trivial example of a negation map (over any semiring) is $(-)a=a$.
\item If $R$ is a ring, it has a negation map $(-)a=-a$.
\item If $\R$ is an ELT algebra, we have a negation map given by $(-)a=\minus a$.
\end{itemize}

The last example is the central example for our theory, since it shows that any ELT algebra is equipped with a natural negation map. Thus, the theory of semirings with negation maps can be used when dealing with ELT algebras.\\

We now present several notations from this theory:
\begin{itemize}
\item $a+(-)a$ is denoted $a^\circ$.
\item $R^\circ=\left\{a^\circ\middle|a\in R\right\}$.
\item We define two partial orders on $R$:
\begin{itemize}
\item The relation $\symmdash$ defined by
$$a\symmdash b\Leftrightarrow \exists c\in R^\circ:a=b+c$$
\item The relation $\nabla$ defined by
$$a\nabla b\Leftrightarrow a+(-)b\in R^\circ$$
\end{itemize}
\end{itemize}

If $\R$ is an ELT algebra, then some of these notations have already been defined. For example, $a^\circ=\zero a$, $\R^\circ=\zeroset{\R}$ and the relation $\symmdash$ is the relation $\vDash$.
