%%%%%%%%%%%%%%%%%%%%%%%%%%%%%%%%%%%%%%%%%%%%%%%%%%%%%%%%%%%%%%%%%%
%%%%%%%% ICML 2017 EXAMPLE LATEX SUBMISSION FILE %%%%%%%%%%%%%%%%%
%%%%%%%%%%%%%%%%%%%%%%%%%%%%%%%%%%%%%%%%%%%%%%%%%%%%%%%%%%%%%%%%%%

% Use the following line _only_ if you're still using LaTeX 2.09.
%\documentstyle[icml2017,epsf,natbib]{article}
% If you rely on Latex2e packages, like most moden people use this:
\documentclass{article}

% use Times
\usepackage{times}
% For figures
\usepackage{graphicx} % more modern
%\usepackage{epsfig} % less modern
\usepackage{subfigure} 

% For citations
\usepackage{natbib}

% For algorithms
\usepackage{algorithm}
\usepackage{algorithmic}

% As of 2011, we use the hyperref package to produce hyperlinks in the
% resulting PDF.  If this breaks your system, please commend out the
% following usepackage line and replace \usepackage{icml2017} with
% \usepackage[nohyperref]{icml2017} above.
\usepackage{hyperref}

% Packages hyperref and algorithmic misbehave sometimes.  We can fix
% this with the following command.
\newcommand{\theHalgorithm}{\arabic{algorithm}}

\usepackage{amsfonts,amsmath,amssymb,amscd,dsfont,mathrsfs,amsthm}

\usepackage{todonotes}

%\usepackage{authblk}
%\setlength{\textwidth}{\paperwidth}

% Employ the following version of the ``usepackage'' statement for
% submitting the draft version of the paper for review.  This will set
% the note in the first column to ``Under review.  Do not distribute.''
%\usepackage{icml2017} 

% Employ this version of the ``usepackage'' statement after the paper has
% been accepted, when creating the final version.  This will set the
% note in the first column to ``Proceedings of the...''
\usepackage[accepted]{icml2017}

\newtheorem{lemma}{Lemma}
\newtheorem{proposition}{Proposition}
\newtheorem{theorem}{Theorem}
\newtheorem{definition}{Definition}
\newtheorem{corollary}{Corollary}
\newtheorem{remark}{Remark}

\newcommand{\argmin}{\mathop{^\rm argmin}}
\newcommand{\argmax}{\mathop{\rm argmax}}
\newcommand{\rank}{\mathop{\sf rank}}
\newcommand{\iprod}[2]{\langle #1, #2 \rangle}
\newcommand{\lmin}{\lambda_{\min}}
\newcommand{\norm}[1]{\left\|#1\right\|}
\newcommand{\mge}{\succeq}
\newcommand{\mi}{{-1}}
\newcommand{\R}{\mathbb{R}}\newcommand{\B}{\mathbb{B}}\newcommand{\Id}{\mathbf{I}}
\newcommand{\zerovec}{\mathbf{0}}
\newcommand{\onevec}{\mathbf{1}}
\newcommand{\defeq}{:=}
\newcommand{\secref}[1]{Section~\ref{#1}}
\newcommand{\lemref}[1]{Lemma~\ref{#1}}
\newcommand{\proref}[1]{Proposition~\ref{#1}}
\newcommand{\thmref}[1]{Theorem~\ref{#1}}
\newcommand{\assref}[1]{Assumption~\ref{#1}}
\newcommand{\eqnref}[1]{(\ref{#1})}
\newcommand{\algref}[1]{Algorithm~\ref{#1}}

\newcommand{\Prob}{\mathbb{P}}
\newcommand{\E}{\mathbb{E}}
\newcommand{\V}{\mathbb{V}}
\newcommand{\tr}[1]{\textrm{tr}\left(#1\right)}
%\newcommand{\dt}[1]{\textrm{det}\left(#1\right)}

\newcommand{\Evt}{\mathcal{E}}
\newcommand{\EvtG}{\Evt_G}
\newcommand{\EvtD}{\Evt_\Delta}
\newcommand{\EvtX}{\Evt_X}

\newcommand{\lihong}[1]{[[\textbf{LL:} #1]]}
\renewcommand{\lihong}[1]{}

\newcommand{\RN}[1]{%
  \textup{\uppercase\expandafter{\romannumeral#1}}%
}

\newtheorem{assumption}{Assumption}

\sloppy

% The \icmltitle you define below is probably too long as a header.
% Therefore, a short form for the running title is supplied here:
\icmltitlerunning{Generalized Linear Contextual Bandits}

\begin{document} 

\twocolumn[
\icmltitle{Provably Optimal Algorithms for Generalized Linear Contextual Bandits}

% It is OKAY to include author information, even for blind
% submissions: the style file will automatically remove it for you
% unless you've provided the [accepted] option to the icml2017
% package.

% list of affiliations. the first argument should be a (short)
% identifier you will use later to specify author affiliations
% Academic affiliations should list Department, University, City, Region, Country
% Industry affiliations should list Company, City, Region, Country

% you can specify symbols, otherwise they are numbered in order
% ideally, you should not use this facility. affiliations will be numbered
% in order of appearance and this is the preferred way.
\icmlsetsymbol{equal}{*}

\begin{icmlauthorlist}
\icmlauthor{Lihong Li}{msr}
\icmlauthor{Yu Lu}{yale}
\icmlauthor{Dengyong Zhou}{msr}
\end{icmlauthorlist}

\icmlaffiliation{msr}{Microsoft Research, Redmond, WA 98052}
\icmlaffiliation{yale}{Department of Statistics, Yale University, New Haven, CT, USA}

\icmlcorrespondingauthor{Lihong Li}{lihongli@microsoft.com}
\icmlcorrespondingauthor{Yu Lu}{yu.lu@yale.edu}
\icmlcorrespondingauthor{Dengyong Zhou}{denzho@microsoft.com}

% You may provide any keywords that you 
% find helpful for describing your paper; these are used to populate 
% the "keywords" metadata in the PDF but will not be shown in the document
\icmlkeywords{contextual bandit, exploration, generalized linear model}

\vskip 0.3in
]

% this must go after the closing bracket ] following \twocolumn[ ...

% This command actually creates the footnote in the first column
% listing the affiliations and the copyright notice.
% The command takes one argument, which is text to display at the start of the footnote.
% The \icmlEqualContribution command is standard text for equal contribution.
% Remove it (just {}) if you do not need this facility.

\printAffiliationsAndNotice{}  % leave blank if no need to mention equal contribution
%\printAffiliationsAndNotice{\icmlEqualContribution} % otherwise use the standard text.
%\footnotetext{hi}

\begin{abstract}
Contextual bandits are widely used in Internet services from news recommendation to advertising, and to Web search. Generalized linear models (logistical regression in particular) have demonstrated stronger performance than linear models in many applications where rewards are binary. However, most theoretical analyses on contextual bandits so far are on linear bandits.  In this work, we propose an upper confidence bound based algorithm for generalized linear contextual bandits, which achieves an $\tilde{O}(\sqrt{dT})$ regret over $T$ rounds with $d$ dimensional feature vectors. This regret matches the minimax lower bound, up to logarithmic terms, and improves on the best previous result by a $\sqrt{d}$ factor, assuming the number of arms is fixed.
A key component in our analysis is to establish a new, sharp finite-sample confidence bound for maximum-likelihood estimates in generalized linear models, which may be of independent interest.  We also analyze a simpler upper confidence bound algorithm, which is useful in practice, and prove it to have optimal regret for certain
cases.
%\todo{Check the last statement}
\end{abstract} 

\begin{abstract}
Real-world planning problems, including autonomous driving and sustainable energy applications like carbon storage and resource exploration, have recently been modeled as partially observable Markov decision processes (POMDPs) and solved using approximate methods. To solve high-dimensional POMDPs in practice, state-of-the-art methods use online planning with problem-specific heuristics to reduce planning horizons and make the problems tractable. Algorithms that learn approximations to replace heuristics have recently found success in large-scale fully observable domains. The key insight is the combination of online Monte Carlo tree search with offline neural network approximations of the optimal policy and value function. In this work, we bring this insight to partially observed domains and propose \textit{BetaZero}, a belief-state planning algorithm for high-dimensional POMDPs. BetaZero learns offline approximations that replace heuristics to enable online decision making in long-horizon problems. We address several challenges inherent in large-scale partially observable domains; namely challenges of transitioning in stochastic environments, prioritizing action branching with a limited search budget, and representing beliefs as input to the network. To formalize the use of all limited search information we train against a novel $Q$-weighted policy vector target. We test BetaZero on various well-established benchmark POMDPs found in the literature and a real-world, high-dimensional problem of critical mineral exploration. Experiments show that BetaZero outperforms state-of-the-art POMDP solvers on a variety of tasks.\footnote{{Code:
https://github.com/sisl/BetaZero.jl}}
\end{abstract}

\section{Introduction}
Optimizing sequential decisions in real-world settings is challenging due to the inherent uncertainty about the true state of the environment.
Modeling such problems as partially observable Markov decision processes (POMDPs) has shown recent success in autonomous driving \citep{wray2021pomdps}, robotics \citep{lauri2022partially}, and aircraft collision avoidance \citep{kochenderfer2012next}.
Solving large or continuous POMDPs require approximations in the form of state-space discretizations or modeling assumptions, e.g., assuming full observability.
Although these approximations are necessary when making fast decisions in a short time horizon, scaling these solutions to long-horizon problems is challenging \cite{shani2013survey}.
Recently, POMDPs have been used to model large-scale information gathering problems such as carbon capture and storage (CCS) \citep{corso2022pomdp,wang2023optimizing}, remediation for groundwater contamination \cite{wang2022sequential}, and critical mineral exploration for battery metals \citep{mern2023intelligent} and are solved using online tree search methods such as DESPOT \cite{ye2017despot} and POMCPOW \citep{sunberg2018online}.
The performance of these online methods rely on heuristics for action selection (to reduce search tree expansion) and heuristics to estimate the value function (to avoid expensive rollouts and reduce tree search depth).
Without heuristics, online methods have difficulty planning for long-term information acquisition to reason about uncertain future events.
Thus, algorithms to solve high-dimensional POMDPs need to be designed to learn heuristic approximations to enable decision making in long-horizon problems.

\begin{figure}[t]
    \centering
    \resizebox{\linewidth}{!}{
        \definecolor{almostwhite}{RGB}{240,240,240}
\ifdefined\shiftforposter\else
    \newcommand{\shiftforposter}{0mm}
\fi
\tikzset{
    >={Latex[width=1.5mm,length=1.5mm]},
    end/.style = {circle, minimum width=2mm, minimum height=2mm},
    block/.style = {rectangle, fill=white, draw=black, minimum width=40mm, minimum height=10mm},
    duplicate/.style = {rectangle, fill=none, draw=gray, dashed, minimum width=4cm},
}

\trimbox{0 0 0 5mm}{ % left, bottom, right, top
\begin{tikzpicture}
% \begin{tikzpicture}[background rectangle/.style={fill=white}, show background rectangle]

\node (eval) [block, double copy shadow={fill=almostwhite, draw=black, shadow xshift=1mm, shadow yshift=1mm}, label={[yshift=-(\shiftforposter/2)]below:\textbf{policy evaluation}}, label={[anchor=south west, xshift=\shiftforposter+3mm, yshift=\shiftforposter+1.5mm]north west:{$n$ parallel MCTS simulations}}] {\resizebox{5cm}{!}{\ifdefined\shiftforposter\else
    \newcommand{\shiftforposter}{0mm}
\fi
\tikzset{
    nodes={draw=\primarycolor, fill=\secondarycolor, circle}, >=latex, -, level distance=11mm,
    level 1/.style={sibling distance=24mm},
    level 2/.style={sibling distance=10mm, level distance=16mm},
    every node/.style={draw=\primarycolor, fill=\secondarycolor, text=\primarycolor, thin, minimum size=7mm, scale=1.2},
    every path/.append style={draw=\primarycolor},
    norm/.style={edge from parent/.style={thin,draw}},
    emph1/.style={edge from parent/.style={line width=1.6pt}},
    emph2/.style={edge from parent/.style={line width=2.0pt,draw}},
    emph3/.style={edge from parent/.style={line width=2.4pt,draw}},
    emph4/.style={edge from parent/.style={line width=1.6pt,draw}}, % rollout
    semiselected/.style={line width=2pt},
    invisible/.style={draw=none, fill=none},
    selected/.style={line width=2pt},
    root/.style={label=\textsc{#1}},
    baseline=(selection.base),
    state/.style={circle, norm, -},
    action/.style={rectangle, norm, sibling distance=2in},
    rollout-end/.style={draw=none, rectangle, minimum height=0mm, text=\primarycolor},
    rollout-edge/.style={->, decorate, decoration={snake, amplitude=1mm, post length=10pt, segment length=12pt}, dotted, line cap=round, dash pattern=on 0pt off 2\pgflinewidth},
}
\newcommand{\spacing}{3.5cm}
\trimbox{4mm 12mm 0 2mm}{ % left, bottom, right, top
\begin{tikzpicture}
    % \node [root={Selection/Branching}, state, selected] (selection) {$b$}
    % \node [root={Selection}, state, selected, label={[label distance=10mm]below:\small{$\left(\text{where } \tilde{b}\leftarrow\textsc{Representation}(b)\right)$}}] (selection) {$b$}
    \node [root={Selection}, state, selected, label={[label distance=10mm]below:\small{$\left(\text{where } \tilde{b}\leftarrow\phi(b)\right)$}}] (selection) {$b$}
        % child [emph1, ->] { node [action, selected, label={[label distance=-0mm]below:\small{\shortstack{$a\leftarrow\textsc{Select}\bigl(P_\theta(\tilde{b}, \cdot)\bigr)$}}}] {$a$}
        child [emph1, ->] { node [action, selected, label={[label distance=-(1mm-\shiftforposter)]right:\small{$\sim P_\theta(\tilde{b})$}}] {$a$}
            child [] { edge from parent[invisible] node[invisible] {} }
            child [] { edge from parent[invisible] node[invisible] {} }
        }
        % child [state] {node [action] {}}
        child [state] {node [action] {}
            % child [state] { node {} }
            % child [state] { node {} }
        };

    \node [root=Expansion, state, right=\spacing of selection] (expansion) {$b$}
        child { node [action, semiselected] {$a$}
            child [state, emph1, ->] { node [semiselected, label={[label distance=-10.5mm]below:\small{\shortstack{$b'\leftarrow\textsc{Update}(b,a,o)$}}}] {$b'$} edge from parent node[right, draw=none, fill=none, text=\primarycolor, pos=0.63, xshift=\shiftforposter] {\ \small{\shortstack[l]{$s\phantom{'}\sim b\phantom{T()}$\\$s' \sim T(s,a)$\\$o\phantom{'} \sim O(a,s')$}}}}
            % child [] { edge from parent[invisible] node[invisible] {} }
        }
        % child [state] {node [action] {}}
        child [state] {node [action] {}
            % child [state] { node {} }
            % child [state] { node {} }
        };

    % \node [root=Simulation/Rollout, state, right of=expansion] (simulation) {$b$}
    \node [root=Simulation, state, right=\spacing of expansion] (simulation) {$b$}
        child { node [action] {$a$}
            child [state] { node [state, selected] {$b'$}
                child [state, emph4, draw=none, level distance=0.6in] { node [rollout-end, invisible] {$r + \gamma V_\theta(\tilde{b}')$}
                    edge from parent [rollout-edge]
                }
            }
            % child [] { edge from parent[invisible] node[invisible] {} }
        }
        % child [state] {node [action] {}}
        child [state] {node [action] {}
            % child [state] { node {} }
            % child [state] { node {} }
        };

    \node [root=Backpropagation, state, selected, right=\spacing of simulation] (backprop) {$b$}
        child [<-,emph1] { node [action, selected] {$a$}
            child [state, emph1, <-] { node [state, selected] {$b'$} edge from parent node[right, pos=0.65, xshift=\shiftforposter, draw=none, fill=none, text=\primarycolor] {\small{$Q$-value}}}
            % child [] { edge from parent[invisible] node[invisible] {} }
        }
        % child [state] {node [action] {}}
        child [state] {node [action] {}
            % child [state] { node {} }
            % child [state] { node {} }
        };
\end{tikzpicture}
}
}};

\node (improve) [block, below=17mm of eval, label={[yshift=(\shiftforposter/2)]above:\textbf{policy improvement}}] {$f'_\theta = \textsc{Train}(f_\theta, \mathcal{D})$};

% \node (input) [left=10mm of eval, label={below:\shortstack{\scriptsize initial network\\\scriptsize$\mathbf{p}=P_\theta(\tilde{b})$, $v = V_\theta(\tilde{b})$}}] {$(\mathbf{p},v) = f_\theta(\tilde{b})$};

\node (input) [left=10mm of eval, label={[xshift=(-\shiftforposter/2),yshift=-\shiftforposter]below:\shortstack{initial network}}] {$(\mathbf{p},v) = f_\theta(\tilde{b})$\hspace*{\shiftforposter}};

\draw[->] (input.east) -- (eval.west);

\draw[->] ($(eval.east)+(2mm,0mm)$) -| ++(4mm,0mm) |- (improve.east) node [pos=0.25, xshift=\shiftforposter, label={right:\shortstack{$\mathcal{D}=\left\{\big\{(b_t, \pi_t, g_t)\big\}_{t=1}^T\right\}_{i=1}^n$\\{collected data}}}] {};

\draw[-] (improve.west) -| ($(eval.west)+(-4mm,0)$) node [pos=0.75, xshift=-\shiftforposter, label={left:$f_\theta = f'_\theta$}] {};


% \node (input) [] {input size = $32 \times 32 \times 2$};

% \node (input_norm) [block, below of=input] {normalization}; 
% \node (conv1) [block, below of=input] {$\text{convolution}(5,5) \Rightarrow \text{relu}$};
% \node (conv2) [block, below of=conv1] {$\text{convolution}(5,5) \Rightarrow \text{relu}$};
% \node (flatten) [block, below of=conv2] {flatten};


% % \node (dense1) [block, below=6mm of input_norm] {fully connected};
% \node (bnorm1) [block, below=6mm of flatten] {fc $\Rightarrow$ batch norm};
% \node (dense1) [block, below=6mm of bnorm1] {relu};
% \node (dropout1) [block, below of=dense1] {dropout};

% \path let \p1=(bnorm1.north), \p2=(dropout1.south) in node (duplicate_block) [fit={($(\x1,\y1)+(0mm,1mm)$) ($(\x2,\y2)+(0mm,0mm)$)}, duplicate, minimum width=35mm]{};
% \node at ($(duplicate_block.south west)+(3mm,2mm)$) [font={\small}, text=gray] {$\times 2$};

% \node (vbnorm1) [block, below left=8mm and -15mm of duplicate_block, label={above:{\tiny value head}}] {fc $\Rightarrow$ batch norm};
% \node (vdense1) [block, below of=vbnorm1] {relu};
% \node (vdropout) [block, below of=vdense1] {dropout};
% \node (vdense2) [block, below of=vdropout] {relu};
% \node (vnorm) [block, below of=vdense2, label={below:$V_\theta(\tilde{b})$}] {denormalization};

% \node (pbnorm1) [block, below right=8mm and -15mm of duplicate_block, label={above:{\tiny policy head}}] {fc $\Rightarrow$ batch norm};
% \node (pdense1) [block, below of=pbnorm1] {relu};
% \node (pdropout) [block, below of=pdense1] {dropout};
% \node (pdense2) [block, below of=pdropout] {relu};
% \node (psoftmax) [block, below of=pdense2, label={below:$P_\theta(\tilde{b},\cdot)$}] {softmax};

% \draw[->] (input.south) -- (conv1.north);
% \draw[->] (conv1.south) -- (conv2.north) node [pos=0.4, label={[label distance=-2mm]right:{\tiny $\ell=64$}}] {};
% \draw[->] (conv2.south) -- (flatten.north) node [pos=0.4, label={[label distance=-2mm]right:{\tiny $\ell=128$}}] {};
% \draw[->] (flatten.south) -- (bnorm1.north) node [pos=0.4, label={[label distance=-2mm]right:{\tiny $\ell=73728$}}] {};
% \draw[->] (bnorm1.south) -- (dense1.north) node [pos=0.4, label={[label distance=-2mm]right:{\tiny $\ell=256$}}] {};
% \draw[->] (dense1.south) -- (dropout1.north) node [pos=0.4, label={[label distance=-2mm]right:{\tiny $\ell=256$}}] {};

% \draw[->] ($(dropout1.south)+(-0.2mm,0)$) -| ++(0,-3mm) -| ($(vbnorm1.north)+(0,2.7mm)$);
% \draw[->] ($(dropout1.south)+(0.2mm,0)$) -| ++(0,-3mm) -| ($(pbnorm1.north)+(0,2.7mm)$);

% \draw[->] (vbnorm1.south) -- (vdense1.north) node [pos=0.4, label={[label distance=-2mm]right:{\tiny $\ell=256$}}] {};
% \draw[->] (vdense1.south) -- (vdropout.north) node [pos=0.4, label={[label distance=-2mm]right:{\tiny $\ell=256$}}] {};
% \draw[->] (vdropout.south) -- (vdense2.north) node [pos=0.4, label={[label distance=-2mm]right:{\tiny $\ell=256$}}] {};
% \draw[->] (vdense2.south) -- (vnorm.north) node [pos=0.4, label={[label distance=-2mm]right:{\tiny $\ell=1$}}] {};

% \draw[->] (pbnorm1.south) -- (pdense1.north) node [pos=0.4, label={[label distance=-2mm]right:{\tiny $\ell=256$}}] {};
% \draw[->] (pdense1.south) -- (pdropout.north) node [pos=0.4, label={[label distance=-2mm]right:{\tiny $\ell=256$}}] {};
% \draw[->] (pdropout.south) -- (pdense2.north) node [pos=0.4, label={[label distance=-2mm]right:{\tiny $\ell=256$}}] {};
% \draw[->] (pdense2.south) -- (psoftmax.north) node [pos=0.4, label={[label distance=-2mm]right:{\tiny $\ell=|\mathcal{A}|=38$}}] {};



% \node (falsification) [sectionstyle, label={above:\textsc{Falsification}}] {
% \begin{tikzpicture}
%     \node (start) [startstyle] {};

%     \node (success) [successstyle, above right=3mm and 25mm of start, label={right:{\color{darkgreen}success}}] {};
%     \node (failure) [failurestyle, below right=3mm and 25mm of start, label={right:{\color{darkred}failure}}] {};

%     \draw [semithick,->] plot [smooth] coordinates {(start.east) ($(start.east)+(3mm,0mm)$)  ($(start.east)+(6mm,-3mm)$) ($(start.east)+(13mm,1mm)$) ($(start.east)+(20mm,-10mm)$) (failure.south west)};
% \end{tikzpicture}
% };

% \draw [gray] ($(falsification.east) + (0.4cm,-2)$) -- ($(falsification.east) + (0.4cm,2)$);

% \node (mostlikely) [sectionstyle, right=of falsification, label={[align=center]above:\textsc{Most-likely}\\\textsc{Failure Analysis}}] {
% \begin{tikzpicture}
%     \node (start) [startstyle] {};
    
%     \node (success) [successstyle, above right=3mm and 25mm of start, label={right:{\color{darkgreen}success}}] {};   
%     \node (failure) [failurestyle, below right=3mm and 25mm of start, label={right:{\color{darkred}failure}}] {};

%     \draw [semithick,->] plot [smooth] coordinates {(start.east) ($(start.east)+(10mm,1mm)$) (failure.north west)};
% \end{tikzpicture}
% };

% \draw [gray] ($(mostlikely.east) + (0.4cm,-2)$) -- ($(mostlikely.east) + (0.4cm,2)$);

% \node (failureprob) [sectionstyle, right=of mostlikely, label={[align=center]above:\textsc{Failure Probability}\\\textsc{Estimation}}] {
% \begin{tikzpicture}
%     \node (start) [startstyle] {};
    
%     \node (success) [successstyle, above right=3mm and 25mm of start, label={right:{\color{darkgreen}success}}] {};
%     \node (failure) [failurestyle, below right=3mm and 25mm of start, label={right:{\color{darkred}failure}}] {};

%     \draw [semithick,gray4,->] plot [smooth] coordinates {(start.east) ($(start.east)+(3mm,0mm)$)  ($(start.east)+(6mm,-3mm)$) ($(start.east)+(13mm,1mm)$) ($(start.east)+(20mm,-10mm)$) (failure.south west)};
%     \draw [semithick,gray4,->] plot [smooth] coordinates {(start.east) ($(start.east)+(5mm,-2mm)$) ($(start.east)+(16mm,-12mm)$) (failure.south)};
%     \draw [semithick,gray3,->] plot [smooth] coordinates {(start.east) ($(start.east)+(15mm,6mm)$) (failure.north)};
%     \draw [semithick,gray2,->] plot [smooth] coordinates {(start.east) ($(start.east)+(5mm,-1mm)$) ($(start.east)+(15mm,-9mm)$) (failure.south west)};
%     \draw [semithick,gray1,->] plot [smooth] coordinates {(start.east) ($(start.east)+(6mm,-1mm)$) ($(start.east)+(15mm,-6mm)$) (failure.west)};
%     \draw [semithick,->] plot [smooth] coordinates {(start.east) ($(start.east)+(10mm,1mm)$) (failure.north west)};
% \end{tikzpicture}
% };

\end{tikzpicture}
} % trimbox
    }
    \caption{The \textit{BetaZero} POMDP policy iteration algorithm.}
    \label{fig:betazero-alg}
\end{figure}

\begin{figure*}[t!]
    \centering
    \resizebox{0.65\textwidth}{!}{
        \ifdefined\shiftforposter\else
    \newcommand{\shiftforposter}{0mm}
\fi
\tikzset{
    nodes={draw=\primarycolor, fill=\secondarycolor, circle}, >=latex, -, level distance=11mm,
    level 1/.style={sibling distance=24mm},
    level 2/.style={sibling distance=10mm, level distance=16mm},
    every node/.style={draw=\primarycolor, fill=\secondarycolor, text=\primarycolor, thin, minimum size=7mm, scale=1.2},
    every path/.append style={draw=\primarycolor},
    norm/.style={edge from parent/.style={thin,draw}},
    emph1/.style={edge from parent/.style={line width=1.6pt}},
    emph2/.style={edge from parent/.style={line width=2.0pt,draw}},
    emph3/.style={edge from parent/.style={line width=2.4pt,draw}},
    emph4/.style={edge from parent/.style={line width=1.6pt,draw}}, % rollout
    semiselected/.style={line width=2pt},
    invisible/.style={draw=none, fill=none},
    selected/.style={line width=2pt},
    root/.style={label=\textsc{#1}},
    baseline=(selection.base),
    state/.style={circle, norm, -},
    action/.style={rectangle, norm, sibling distance=2in},
    rollout-end/.style={draw=none, rectangle, minimum height=0mm, text=\primarycolor},
    rollout-edge/.style={->, decorate, decoration={snake, amplitude=1mm, post length=10pt, segment length=12pt}, dotted, line cap=round, dash pattern=on 0pt off 2\pgflinewidth},
}
\newcommand{\spacing}{3.5cm}
\trimbox{4mm 12mm 0 2mm}{ % left, bottom, right, top
\begin{tikzpicture}
    % \node [root={Selection/Branching}, state, selected] (selection) {$b$}
    % \node [root={Selection}, state, selected, label={[label distance=10mm]below:\small{$\left(\text{where } \tilde{b}\leftarrow\textsc{Representation}(b)\right)$}}] (selection) {$b$}
    \node [root={Selection}, state, selected, label={[label distance=10mm]below:\small{$\left(\text{where } \tilde{b}\leftarrow\phi(b)\right)$}}] (selection) {$b$}
        % child [emph1, ->] { node [action, selected, label={[label distance=-0mm]below:\small{\shortstack{$a\leftarrow\textsc{Select}\bigl(P_\theta(\tilde{b}, \cdot)\bigr)$}}}] {$a$}
        child [emph1, ->] { node [action, selected, label={[label distance=-(1mm-\shiftforposter)]right:\small{$\sim P_\theta(\tilde{b})$}}] {$a$}
            child [] { edge from parent[invisible] node[invisible] {} }
            child [] { edge from parent[invisible] node[invisible] {} }
        }
        % child [state] {node [action] {}}
        child [state] {node [action] {}
            % child [state] { node {} }
            % child [state] { node {} }
        };

    \node [root=Expansion, state, right=\spacing of selection] (expansion) {$b$}
        child { node [action, semiselected] {$a$}
            child [state, emph1, ->] { node [semiselected, label={[label distance=-10.5mm]below:\small{\shortstack{$b'\leftarrow\textsc{Update}(b,a,o)$}}}] {$b'$} edge from parent node[right, draw=none, fill=none, text=\primarycolor, pos=0.63, xshift=\shiftforposter] {\ \small{\shortstack[l]{$s\phantom{'}\sim b\phantom{T()}$\\$s' \sim T(s,a)$\\$o\phantom{'} \sim O(a,s')$}}}}
            % child [] { edge from parent[invisible] node[invisible] {} }
        }
        % child [state] {node [action] {}}
        child [state] {node [action] {}
            % child [state] { node {} }
            % child [state] { node {} }
        };

    % \node [root=Simulation/Rollout, state, right of=expansion] (simulation) {$b$}
    \node [root=Simulation, state, right=\spacing of expansion] (simulation) {$b$}
        child { node [action] {$a$}
            child [state] { node [state, selected] {$b'$}
                child [state, emph4, draw=none, level distance=0.6in] { node [rollout-end, invisible] {$r + \gamma V_\theta(\tilde{b}')$}
                    edge from parent [rollout-edge]
                }
            }
            % child [] { edge from parent[invisible] node[invisible] {} }
        }
        % child [state] {node [action] {}}
        child [state] {node [action] {}
            % child [state] { node {} }
            % child [state] { node {} }
        };

    \node [root=Backpropagation, state, selected, right=\spacing of simulation] (backprop) {$b$}
        child [<-,emph1] { node [action, selected] {$a$}
            child [state, emph1, <-] { node [state, selected] {$b'$} edge from parent node[right, pos=0.65, xshift=\shiftforposter, draw=none, fill=none, text=\primarycolor] {\small{$Q$-value}}}
            % child [] { edge from parent[invisible] node[invisible] {} }
        }
        % child [state] {node [action] {}}
        child [state] {node [action] {}
            % child [state] { node {} }
            % child [state] { node {} }
        };
\end{tikzpicture}
}

    }
    \caption{The four stages of MCTS in \textit{BetaZero} using a value $V_\theta$ and policy $P_\theta$ network (the \textit{policy evaluation} step in \cref{fig:betazero-alg}).}
    \label{fig:mcts-betazero}
\end{figure*}

\paragraph{Related work.}
Algorithms to solve high-dimensional, fully observable Markov decision processes (MDPs) learn approximations to replace problem-specific heuristics.
\citeauthor{silver2018general} introduced the \textit{AlphaZero} algorithm for large, deterministic MDPs and showed considerable success in games such as Go, chess, shogi, and Atari \cite{silver2018general, schrittwieser2020mastering}.
The success is attributed to the combination of online Monte Carlo tree search (MCTS) and a neural network that approximates the optimal value function and the offline policy.
Extensions of AlphaZero and the model-free variant \textit{MuZero} \cite{schrittwieser2020mastering} have already addressed several challenges when applying to broad classes of MDPs.
For large or continuous action spaces, \citeauthor{hubert2021learning} introduced a policy improvement algorithm called \textit{Sampled MuZero} that samples an action set of an \textit{a priori} fixed size every time a node is expanded.
\citeauthor{antonoglou2021planning} introduced \textit{Stochastic MuZero} that extends MuZero to games with stochastic transitions but assumes a finite set of possible next states so that each transition can be associated with a chance outcome.
Applying these algorithms to large or continuous spaces remains challenging.

To handle partial observability in stochastic games, \citeauthor{ozair2021vector} combine VQ-VAEs with MuZero to encode future discrete observations into latent variables.
Other approaches incorporate partial observability by inputting action-observation histories directly into the network \cite{kimura2020development, vinyals2019grandmaster}.
Similarly, \citeauthor{igl2018deep} introduce a method to learn a belief representation within the network when the agent is only given access to histories.
Their work focuses on the \textit{reinforcement learning} (RL) domain and they show that a belief distribution can be represented as a latent state in the learned model.
The FORBES algorithm \cite{chen2022flow} builds a normalizing flow-based belief and learns a policy through an actor-critic RL algorithm.
Methods to learn the belief are necessary when a prior belief model is not available.
When such models \textit{do} exist, as is the case with many geological applications, using the models can be valuable for long-term planning.
\citeauthor{hoel2019combining} apply AlphaGo Zero \cite{silver2017mastering} to an autonomous driving POMDP using the most-likely state as the network input but overlook significant belief uncertainty information.
Planning under uncertainty is crucial for learning effective POMDP policies.


\paragraph{Planning vs. reinforcement learning.}
In POMDP \textit{planning}, models of the transitions, rewards, and observations are known.
In contrast, in the model-based \textit{reinforcement learning} (RL) domain, these models are learned along with a policy or value function \cite{sutton2018reinforcement}.
A difference between these settings is that RL algorithms reset the agent and learn through experience, while planning algorithms, like MCTS, must consider future trajectories from any state.
When RL problems have deterministic state transitions, they can be cast as a planning problem by replaying the full state trajectory along a tree path, which may be prohibitively expensive for long-horizon problems.
Both settings are closely related and pose interesting research challenges.
Specifically, sequential planning over given models in high-dimensional, long-horizon POMDPs remains challenging \cite{lauri2022partially}.


\paragraph{Online POMDP planning.}
Existing online POMDP planning algorithms rely on problem-specific heuristics for good performance.
\citeauthor{sunberg2018online} introduced the POMCPOW planning algorithm that iteratively builds a particle set belief within the tree, designed for continuous observations.
In practice, POMCPOW relies on heuristics for value function estimation and action selection (e.g., work from \citeauthor{mern2023intelligent}).
\citeauthor{wu2021adaptive} introduced AdaOPS that adaptively approximates the belief through particle filtering and maintains value function bounds that are initialized with heuristics (e.g., solving the MDP or using expert policies).
The major limitation of existing solvers is the reliance on heuristics to make long-horizon POMDPs tractable, which may not scale to high-dimensional problems.

Our work avoids heuristics and uses the insight of combining online MCTS with offline neural network approximations to solve high-dimensional POMDPs.
We propose the \textit{BetaZero} belief-state planning algorithm that plans in belief space 
using no problem-specific heuristics, only using information from the search.
We address partially observable challenges in large discrete action spaces and continuous state and observation spaces.
In stochastic environments, BetaZero uses progressive widening \cite{couetoux2011continuous} to limit belief-state expansion---only using information available in the tree.
When planning in belief space, computationally expensive belief updates limit the search budget in practice.
Therefore, information from the policy network is used to prioritize branching on promising actions.
We introduce a novel $Q$-weighted policy vector target that formalizes the use of all the information seen during the limited search.
To capture state uncertainty, the network receives a parametric approximation of the belief $\tilde{b}$ as input.
BetaZero uses the learned policy network $P_\theta$ to reduce search breadth and the learned value estimate $V_\theta$ to reduce search depth to enable long-horizon online planning (\cref{fig:mcts-betazero}).



\section{Background}

A partially observable Markov decision process (POMDP) is a model for sequential decision making problems where the true state is unobservable.
Defined by the tuple $\langle \mathcal{S}, \mathcal{A}, \mathcal{O}, T, R, O, \gamma \rangle$, 
POMDPs are an extension to the Markov decision process (MDP) used in reinforcement learning and planning with the addition of an observation space $\mathcal{O}$ (where $o \in \mathcal{O}$) and observation model $O(o \mid a, s')$.
Given a current state $s \in \mathcal{S}$ and taking an action $a \in \mathcal{A}$, the agent transitions to a new state $s'$ using the transition model $s' \sim T(\cdot \mid s, a)$.
Without access to the true state, an observation is received $o \sim O(\cdot \mid a, s')$ and used to update the belief $b$ over the possible next states $s'$ to get the posterior
\begin{equation}    
    b'(s') \propto O(o \mid a, s')\int_{s \in \mathrlap{\mathcal{S}}} T(s' \mid s, a)b(s) \,ds.
\end{equation}
The non-parametric \textit{particle set} belief can represent a broad range of distributions \cite{thrun2005probabilistic} and \citeauthor{lim2023optimality} show that optimality guarantees exist in finite-sample particle-based POMDP approximations.
Despite choosing to study particle-based beliefs, our work generalizes well to problems with parametric beliefs.

A stochastic POMDP policy $\pi(a \mid b)$ is defined as the distribution over actions given the current belief $b$.
After taking an action $a \sim \pi(\cdot \mid b)$, the agent receives a reward $r$ from the environment according to the reward function $R: \mathcal{S} \times \mathcal{A} \to \mathbb{R}$ or $R: \mathcal{S} \times \mathcal{A} \times \mathcal{S} \to \mathbb{R}$ using the next state.

\paragraph{Belief-state MDPs.}
In \textit{belief-state} MDPs, the POMDP is converted to an MDP by treating the belief as a state \cite{dmbook}. % TODO: \cite{kaelbling1998planning}
The reward function then becomes a weighted sum of the state-based reward:
\begin{align}
    R_b(b,a) &= \int_{s \in \mathrlap{\mathcal{S}}} b(s)R(s,a) \,ds \label{eq:reward} \\
             &\approx \sum_{s \in b} b(s)R(s,a)
\end{align}
The belief-state MDP shares the same action space as the POMDP and operates over a belief space $\mathcal{B}$ that is the simplex over the state space $\mathcal{S}$. The belief-MDP defines a new belief-state transition function $b' \sim T_b(\cdot \mid b, a)$ as:
\begin{align}
    s &\sim b(\cdot)            \qquad &&s' \sim T(\cdot \mid s, a) \label{eq:tb_update} \\
    o &\sim O(\cdot \mid a, s') \qquad &&b' \leftarrow \textsc{Update}(b, a, o) \nonumber
\end{align}
where the belief update can be done using a particle filter \cite{gordon1993novel}.
Therefore, the belief-state MDP is defined by the tuple $\langle \mathcal{B}, \mathcal{A}, T_b, R_b, \gamma \rangle$ with the finite-horizon discount factor $\gamma \in [0,1)$ that controls the effect that future rewards have on the current action.

The objective to solve belief-MDPs is to find a policy $\pi$ that maximizes the \textit{value function} from an initial belief $b_0$:
\begin{equation}
    V^\pi(b_0) = \mathbb{E}_\pi \left[\sum_{t=1}^T \gamma^t R_b(b_t,a_t) \bigmid\mid b_t \sim T_b, a_t \sim \pi \right]
\end{equation}
Instead of explicitly constructing a policy over all beliefs,
online planning algorithms estimate the next best action through a planning procedure, often a best-first tree search.


\subsection{Monte Carlo tree search (MCTS)}
Monte Carlo tree search \cite{coulom2007efficient,browne2012survey} is an online, recursive, best-first tree search algorithm to determine the approximately optimal action to take from a given root state of an MDP.
Extensions to MCTS have been applied to POMDPs through several algorithms: \textit{partially observable Monte Carlo planning} (POMCP) treats the state nodes as histories $h$ of action-observation trajectories \cite{silver2010pomcp}, \textit{POMCP with observation widening} (POMCPOW) construct weighted particle sets at the state nodes and extends POMCP to fully continuous domains \cite{sunberg2018online}, and \textit{particle filter trees} (PFT) and \textit{information particle filter trees} (IPFT) treat the POMDP as a belief-state MDP and plan directly over the belief-state nodes using a particle filter \cite{ fischer2020information}.
All variants of MCTS execute the following four key steps.
In this section we use $s$ to represent the state, the history $h$, and the belief state $b$ and refer to them as ``the state''.


\begin{enumerate}
    \item \textbf{Selection.}\quad 
    During \textit{selection}, an action is selected from the children of a state node based on criteria that balances exploration and exploitation and is considered a multi-armed bandit problem \cite{munos2014bandits}.
    The \textit{upper-confidence tree} algorithm (UCT) \cite{kocsis2006bandit} is a commonly used criterion that selects an action that maximizes the upper-confidence bound
    \begin{equation}
        Q(s, a) + c\sqrt{\frac{\log N(s)}{N(s,a)}} \label{eq:uct}
    \end{equation}
    where $Q(s,a)$ is the $Q$-value estimate for state-action pair $(s,a)$, the visit count for that state-action pair is $N(s,a)$, the total visit count is $N(s) = \sum_a N(s,a)$ for the children $a \in A(s)$, and $c \ge 0$ is an exploration constant.
    \citeauthor{rosin2011multi} introduced the \textit{upper-confidence tree with predictor} algorithm (PUCT), modified by \citeauthor{silver2017mastering}, where a predictor $P(s,a)$ guides the exploration towards promising branches and selects an action that maximizes
    \begin{equation}
        Q(s,a) + c\left(P(s,a)\frac{\sqrt{N(s)}}{1 + N(s,a)}\right). \label{eq:puct}
    \end{equation}

    When dealing with a small discrete action space, the selection step often first branches on all actions and then selects based on some criterion; as is the case for AlphaZero that first expands on all legal moves \cite{silver2016mastering, silver2018general}.
    For large or continuous action spaces, algorithms attempt to balance branching and selection.
   
    
    \item \textbf{Expansion.}\quad 
    In the \textit{expansion} step, the selected action is taken in simulation and the transition model $T(s' \mid s, a)$ is sampled to determine the next state $s'$.
    When the transitions are deterministic, the child node is always a single state.
    If the transition dynamics are stochastic, techniques to balance the branching factor such as progressive widening \cite{couetoux2011continuous} and state abstraction refinement \cite{sokota2021monte} have been proposed.
    
    
    \item \textbf{Rollout/Simulation.}\quad 
    In the \textit{rollout} step, also called the \textit{simulation} step due to recursively simulating the MCTS tree expansion, the value is estimated through the execution of a rollout policy until termination or using heuristics to approximate the value function from the given state $s'$.
    \citeauthor{silver2017mastering} replace the expensive rollouts done by AlphaGo with a value network lookup in AlphaGo Zero and AlphaZero \cite{silver2016mastering, silver2017mastering, silver2018general}.
    
    
    \item \textbf{Backpropagation.}\quad 
    Finally, during the \textit{backpropagation} step, the $Q$-value estimate from the rollout is propagated up the path in the tree as a running average.
\end{enumerate}


\paragraph{Root node action selection.}\label{par:root_selection}
After repeating the four steps of MCTS, the final action is selected from the children $a \in A(s)$ of the root state $s$ and executed in the environment.
One way to select the best root node action, referred to as the \textit{robust child} \cite{schadd2009monte, browne2012survey}, selects the action with the highest visit count as $\argmax_a N(s,a)$.
Sampling from the normalized counts, exponentiated by an exploratory temperature $\tau$, is also common \cite{silver2017mastering}.
Another method uses the highest estimated $Q$-value as $\argmax_a Q(s,a)$.
Both criteria have been shown to have problem-based trade-offs \cite{browne2012survey}.


\paragraph{Double progressive widening.}\label{sec:dpw}
To handle stochastic state transitions and large or continuous state and action spaces, double progressive widening (DPW) balances between sampling new nodes to expand on or selecting from existing nodes already in the tree \cite{couetoux2011continuous}.
Two hyperparameters $\alpha \in [0,1]$ and $k \ge 0$ control the branching factor.
If the number of actions tried from state $s$ is less than $kN(s)^\alpha$, then a new action is sampled from the action space and added as a child of node $s$.
Likewise, if the number of expanded states from state-action node $(s,a)$ is less than $kN(s,a)^\alpha$, then a new state is sampled from the transition function $s' \sim T(\cdot \mid s, a)$ and added as a child of that state-action node.
If the state widening condition is not met, then a next state is sampled from the existing children.
To encouraging widening, let $k \to \infty$ and $\alpha \to 1$, e.g., in highly stochastic problems \cite{sokota2021monte}, and to discourage widening let $k \to 1$ and $\alpha \to 0$ \cite{moss2020adaptive}.

Note that in the following sections we will refer to the belief state as $b$ and the true (hidden) state as $s$.



\section{Proposed Algorithm: BetaZero}\label{sec:betazero}

We introduce the \textit{BetaZero} POMDP planning algorithm that replaces heuristics with learned approximations of the optimal policy and value function.
BetaZero is a belief-space policy iteration algorithm with two offline steps:
\begin{enumerate}
    \item \textbf{Policy evaluation}: Evaluate the current value and policy network through $n$ parallel episodes of MCTS and collect training data: $\mathcal{D} = \left\{\{(b_t, \pi_t, g_t)\}_{t=1}^T\right\}_{i=1}^n$
    \item \textbf{Policy improvement}: Improve the estimated value function and policy by retraining the neural network parameters $\theta$ with data from the most recent MCTS simulations.
\end{enumerate}
The policy vector over actions $\vect{p} = P_\theta(\tilde{b}, \cdot)$ and the value $v = V_\theta(\tilde{b})$ are combined into a single network with two output heads $(\vect{p}, v) = f_\theta(\tilde{b})$; we refer to $P_\theta$ and $V_\theta$ separately.

During \textit{policy evaluation}, training data is collected from the outer POMDP loop.
The belief $b_t$ and the tree policy $\pi_t$ are collected for each time step $t$.
At the end of each episode, the returns
$g_t = \sum_{i=t}^T \gamma^{(i-t)} r_i$
are computed from the set of observed rewards for all time steps up to a terminal horizon $T$.
Traditionally, MCTS algorithms use a tree policy $\pi_t$ that is proportional to the root node visit counts of its children actions
\(
    \pi_t(b_t, a) \propto N(b_t,a)^{1/\tau}\label{eq:policy_counts}
\).
The counts are sampled after exponentiating with a temperature $\tau$ to encourage exploration but evaluated online with $\tau \to 0$ to return the maximizing action \cite{silver2017mastering}.
In certain settings, relying solely on visit counts may overlook crucial information.


\paragraph{Policy vector as $Q$-weighted counts.}
When planning over belief states, expensive belief updates occur in the tree search and thus a limited MCTS budget may be used.
Therefore, the visit counts may not converge towards an optimal strategy as the budget may be spent on exploration.
\citeauthor{danihelka2022policy} and \citeauthor{czech2021improving} suggest using knowledge of the $Q$-values from search in MCTS action selection.
Using only tree information, we incorporate $Q$-values and train against the policy $\pi_t(b_t, a)$ proportional to
\begin{equation}
    \Biggl(\biggl(\frac{\exp Q(b_t, a)}{\sum_{a'} \exp Q(b_t, a')}\biggr)^{z_q}\biggl(\frac{N(b_t,a)}{\sum_{a'} N(b_t,a')}\biggr)^{z_n}\Biggr)^{1/\tau}\label{eq:policy_q_weight}
\end{equation}
which is then normalized to get a valid probability distribution.
\Cref{eq:policy_q_weight} simply weights the visit counts by the softmax $Q$-value distribution with parameters $z_q \in [0,1]$ and $z_n \in [0,1]$ defining the influence of the values and the visit counts, respectively.
If $z_q=z_n=1$, then the influence is equal and if $z_q=z_n=0$, then the policy becomes uniform.
Once the tree search finishes, the final action is selected as $a \sim \pi_t(b_t, \cdot)$ and returns the $\argmax$ when $\tau \to 0$.

\begin{figure}[h!]
    \centering
    \resizebox{\linewidth}{!}{
        \ifdefined\shiftforposterbarlabels\else
    \newcommand{\shiftforposterbarlabels}{2mm}
\fi
\tikzset{
    nodes={draw=black, fill=white, circle}, >=latex, -, level distance=12mm,
    level 1/.style={sibling distance=15mm},
    % level 2/.style={sibling distance=12mm, level distance=16mm},
    every node/.style={draw=black, fill=white, text=black, thin, minimum size=7mm},
    every path/.append style={draw=black},
    norm/.style={edge from parent/.style={thin,draw}},
    emph1/.style={edge from parent/.style={line width=1.6pt,draw=grays1}},
    emph2/.style={edge from parent/.style={line width=2.0pt,draw=grays2}},
    emph3/.style={edge from parent/.style={line width=2.4pt,draw=grays3}},
    emph4/.style={edge from parent/.style={line width=2.8pt,draw=grays4}},
    emph5/.style={edge from parent/.style={line width=3.6pt,draw=grays4}}, % note 4
    semiselected/.style={line width=2pt},
    invisible/.style={draw=none, fill=none},
    selected/.style={line width=2pt},
    root/.style={label=\textsc{#1}},
    baseline=(selection-root.base),
    state/.style={circle, norm, -},
    action/.style={rectangle, norm, sibling distance=2in},
    rollout-end/.style={draw=none, rectangle, minimum height=0mm, text=black},
    rollout-edge1/.style={->, decorate, decoration={snake, amplitude=1mm, post length=8pt, segment length=12pt}, dotted, line cap=round, dash pattern=on 0pt off 4\pgflinewidth},
    rollout-edge2/.style={->, decorate, decoration={snake, amplitude=1mm, post length=8pt, segment length=12pt}, dotted, line cap=round, dash pattern=on 0pt off 4\pgflinewidth},
    rollout-edge3/.style={->, decorate, decoration={snake, amplitude=1mm, post length=8pt, segment length=12pt}, dotted, line cap=round, dash pattern=on 0pt off 4\pgflinewidth},
}

\trimbox{0 5mm 0 2mm}{ % left, bottom, right, top
\scalebox{2}{%
\begin{tikzpicture}%[child anchor=center]
    \node [state, label={[align=center]above:\includegraphics[scale=0.3]{figures/q-weighting/bar_visit_counts.pdf}}] (selection-root) {$b$}
        child [state, emph3, ->] { 
            node [action, label={[label distance=-\shiftforposterbarlabels]below:\tiny\scalebox{0.8}{\shortstack{$N{=}10$\\$Q{=}0$}}}] {$a_1$}
            % edge from parent node[left, pos=0.55, draw=none, fill=none, text=\primarycolor] {\tiny{$N{=}10$}}
        }
        child [state, emph3, ->] {
            node [action, label={[label distance=-\shiftforposterbarlabels]below:\tiny\scalebox{0.8}{\shortstack{$N{=}10$\\$Q{=}1$}}}] {$a_2$}
            % edge from parent node[right, pos=0.7, xshift=-0.5mm, draw=none, fill=none, text=\primarycolor] {\tiny{$N{=}10$}}
        }
        child [state, emph3, ->] {
            node [action, label={[label distance=-\shiftforposterbarlabels]below:\tiny\scalebox{0.8}{\shortstack{$N{=}10$\\$Q{=}2$}}}] {$a_3$}
            % edge from parent node[right, pos=0.55, draw=none, fill=none, text=\primarycolor] {\tiny{$N{=}10$}}
        };
\end{tikzpicture}
\hspace{5mm}
\begin{tikzpicture}
    \node [state, label={[align=center]above:\includegraphics[scale=0.3]{figures/q-weighting/bar_q_weighted.pdf}}] {$b$}
       child [state, emph1, ->] { 
            node [action, label={[label distance=-\shiftforposterbarlabels]below:\tiny\scalebox{0.8}{\shortstack{$N{=}10$\\$Q{=}0$}}}] {$a_1$}
            % edge from parent node[left, pos=0.55, draw=none, fill=none, text=\primarycolor] {\tiny{$N{=}10$}}
        }
        child [state, emph2, ->] {
            node [action, label={[label distance=-\shiftforposterbarlabels]below:\tiny\scalebox{0.8}{\shortstack{$N{=}10$\\$Q{=}1$}}}] {$a_2$}
            % edge from parent node[right, pos=0.7, xshift=-0.5mm, draw=none, fill=none, text=\primarycolor] {\tiny{$N{=}10$}}
        }
        child [state, emph5, ->] {
            node [action, label={[label distance=-\shiftforposterbarlabels]below:\tiny\scalebox{0.8}{\shortstack{$N{=}10$\\$Q{=}2$}}}] {$a_3$}
            % edge from parent node[right, pos=0.55, draw=none, fill=none, text=\primarycolor] {\tiny{$N{=}10$}}
        };
\end{tikzpicture}
\hspace{0.1cm}
} % scalebox
} % trimbox
    }
    \caption{An illustrative example of when using a small MCTS budget with high exploration and collecting policy data based purely on visit counts (left) would perform worse than weighting the counts based on $Q$-values (right).
    In this example, MCTS had a budget of $30$ iterations and visited each action $10$ times.
    When training on the visit-counts, we miss useful $Q$-value information and select the worst performing action $a_1$.
    When using both the $Q$-values and visit counts, we incorporate both what the tree search \textit{focused on} and the \textit{values it found}.
    This also accounts for uncertainty by weighting the $Q$-value by the number of samples in the estimate.
    An ablation study in the appendix tests this idea.}
    \label{fig:q-weighting}
\end{figure}


\paragraph{Loss function.}
Using the latest collected data, the \textit{policy improvement} step retrains the policy network head using the cross-entropy loss $\mathcal{L}_{P_\theta}(\pi_t, \vect{p}_t) = -\pi_t^\top \log \vect{p}_t$.
The value network head is simultaneously trained to fit the returns $g_t$ using mean-squared error (MSE) or mean-absolute error (MAE) to predict the value of the belief $b_t$.
The choice of value loss function $\mathcal{L}_{V_\theta}$ depends on the characteristics of the return distribution.
In sparse reward problems,
MAE is a better choice as the distribution is closer to Laplacian \cite{hodson2022root}.
When the reward is distributed closer to Gaussian, then MSE is more suitable \cite{chai2014root}.
The final loss function combines the value and policy losses with added $L_2$-regularization scaled by $\lambda$:
\begin{equation}
    \ell_{\beta_0} = \mathcal{L}_{V_\theta}(g_t, v_t) + \mathcal{L}_{P_\theta}(\pi_t, \vect{p}_t) + \lambda\norm{\theta}^2
\end{equation}


\paragraph{Prioritized action widening.}
Planning in belief space explicitly handles state uncertainty but may incur computational overhead when performing belief updates, therefore we avoid trying all actions at every belief-state node.
Action progressive widening has been used successfully in the context of continuous action spaces \cite{moerland2018a0c} and large discrete action spaces \cite{yee2016monte}. 
\citeauthor{mern2021improved} show that prioritizing actions can improve MCTS performance in large discrete action spaces and \citeauthor{browne2012survey} found action progressive widening to be effective in cases where favorable actions were tried first.

In BetaZero, we use \cref{alg:betazero-action-pw} to select actions through progressive widening and use information from the learned policy network $P_\theta$ to sample new actions.
This way, we can first focus the expansion on promising actions, then make the final selection based on PUCT.
In the appendix, we perform an ablation to measure the effect of using the policy $P_\theta$ to prioritize actions when widening the tree.


\begin{figure}[t!]
    \begin{algorithm}[H]
    \small
    \caption{BetaZero action progressive widening.}
    \label{alg:betazero-action-pw}
    \begin{algorithmic}[1]
        \Function{ActionSelection$(f_\theta, b)$}{}
            \State $\tilde{b} \leftarrow \phi(b)$ \GrayComment{belief representation \cref{eq:belief_rep}}
            \If {$| A(b) | \le kN(b)^\alpha\phantom{\tilde{b}}$} \GrayComment{action progressive widening}
                \State $a \sim P_\theta(\tilde{b}, \cdot)$ \GrayComment{prioritized from network}
                % \State $a \sim \begin{cases}
                %     P_\theta(\tilde{b}, \cdot) & \text{w/ prob. $1-\epsilon$}\\
                %     \mathcal{A}(b) & \text{otherwise}
                % \end{cases}$
                \State $N(b,a) \leftarrow N_0(b,a)$
                \State $Q(b,a) \leftarrow Q_0(b,a)$ \label{line:init_q} \GrayComment{bootstrap initial $Q$-value}
                \State $A(b) \leftarrow A(b) \cup \{a\}$ \GrayComment{add to visited actions $A(b)$}
            \EndIf
            \State \Return ${\displaystyle\argmax_{a \in A(b)}}\ Q(b,a) + c\Bigl(P_\theta(\tilde{b},a)\frac{\sqrt{N(b)}}{1 + N(b,a)}\Bigr)$ \GrayComment{PUCT}
        \EndFunction
    \end{algorithmic}
\end{algorithm}

\end{figure}


\begin{figure}[b!]
    \begin{algorithm}[H]
    \small
    \caption{BetaZero belief-state progressive widening.}
    \label{alg:betazero-state-pw}
    \begin{algorithmic}[1]
        \Function{BeliefStateExpansion$(b,a)$}{}
            \If {$|B(b,a)| \le kN(b,a)^\alpha$} \GrayComment{belief progressive widening} % \textbf{or} $B(b,a) = \emptyset$}
                \State $s \sim b(\cdot)$
                \State $s' \sim T(\cdot \mid s, a)$
                \State $o \sim O(\cdot \mid a, s')$
                \hspace*{4em}%
                \rlap{\raisebox{\dimexpr.5\normalbaselineskip+.5\jot}{\smash{$\left.\begin{array}{@{}c@{}}\\{}\\{}\\{}\end{array}\color{commentgray}\right\}%
                \color{commentgray}\begin{tabular}{l}$b' \sim T_b(\cdot \mid s, a)$\\from \cref{eq:tb_update}\end{tabular}$}}}
                \State $b' \leftarrow \textsc{Update}(b,a,o)$
                \State $B(b,a) \leftarrow B(b,a) \cup \{b'\}$ \GrayComment{add to visited beliefs}
            \Else
                \State $b' \sim B(b,a)$ \GrayComment{sample from belief-states in the tree}
            \EndIf
            \State $r \leftarrow R(b, a)$ or $r \leftarrow R(b, a, b')$
            \State \Return $b', r$
        \EndFunction
    \end{algorithmic}
\end{algorithm}
\end{figure}

\paragraph{Stochastic belief-state transitions.}
A challenge with~partially observable domains is handling non-deterministic belief-state transitions in the tree search.
The belief-state transition function $T_b$ consists of several stochastic components and the belief---which is a probability distribution over states---is continuous.
To address this, we use progressive widening from \citeauthor{couetoux2011continuous} (\cref{alg:betazero-state-pw}).
Other methods for state expansion, like state abstraction refinement from \citeauthor{sokota2021monte}, rely on problem-specific distance metrics between states to perform a nearest neighbor search.
Progressive widening avoids problem-specific heuristics and uses only the information in the search tree to provide artificially limited belief-state branching which
is important as the belief updates can be computationally expensive, thus limiting the MCTS search budget in practice.


\paragraph{Parametric belief representation.}
Although a particle set belief is not parametrically defined, approximating the belief as summary statistics (e.g., mean and std of the state particles) may capture enough information for value and policy estimation to be used during planning (further analyzed in the appendix).
Approximating the particle set parametrically is easy to implement and computationally inexpensive.
We show that the approximation works well across various problems and, unsurprisingly, using only the mean state is inadequate. 
We represent the particle set $b$ parametrically as:
\begin{equation}
    \phi(b) \defeq \big[\mu(b), \sigma(b)\big] \label{eq:belief_rep}
\end{equation}
BetaZero plans over the full belief $b$ in the tree and only converts to the belief representation $\tilde{b} = \phi(b)$ for network evaluations.
Other algorithms (e.g., FORBES from \citeauthor{chen2022flow}) could instead be used to learn this belief representation.

\paragraph{Bootstrapping initial $Q$-values.} When a new state-action node is added to the tree, initial $Q$-values can be bootstrapped using the estimate from the value network $V_\theta$:
\begin{equation}
    \small
    Q_0(b, a) \defeq R_b(b, a) + \gamma V_\theta(\phi(b')) \,\, \text{where} \,\, b' \sim T_b(\cdot \mid b, a)
\end{equation}
Bootstrapping occurs in \cref{alg:betazero-action-pw} (line \ref{line:init_q}) and incurs an additional belief update through the belief-state transition $T_b$ and may be opted only during online execution.
The bootstrapped estimate is more robust \cite{kumar2019stabilizing} and can be useful to initialize online search.
Note that MuZero also uses bootstrapping \cite{schrittwieser2020mastering}.

\Cref{alg:betazero-mcts} details MCTS for BetaZero and the full BetaZero algorithm is in the appendix (algorithms \ref{alg:betazero}--\ref{alg:mcts-top-lvl}).

\begin{figure}[b!]
    % Monte Carlo tree search (MCTS) 4-stage diagram: selection, expansion, rollout, and backpropagation.
\tikzset{
    nodes={draw, circle}, >=latex, -, level distance=0.6in,
    every node/.style={draw=black, thin, minimum size=7mm, scale=1.2},
    level 1/.style={sibling distance=20mm},
    level 2/.style={sibling distance=10mm, level distance=16mm},
    norm/.style={edge from parent/.style={black,thin,draw}},
    emph1/.style={edge from parent/.style={line width=1.6pt,draw}},
    emph2/.style={edge from parent/.style={line width=2.0pt,draw}},
    emph3/.style={edge from parent/.style={line width=2.4pt,draw}},
    emph4/.style={edge from parent/.style={line width=1.6pt,draw}}, % rollout
    semiselected/.style={line width=1.6pt},
    invisible/.style={draw=none},
    selected/.style={line width=1.6pt},
    root/.style={label=\textsc{#1}},
    baseline=(selection-root.base),
    state/.style={circle, norm,-},
    action/.style={rectangle, norm},
    rollout-end/.style={rectangle, draw=none, minimum height=0mm},
    rollout-edge/.style={->, decorate, decoration={snake, amplitude=1mm, post length=10pt, segment length=12pt}, dotted, line cap=round, dash pattern=on 0pt off 2\pgflinewidth},
}
\begin{tikzpicture}
    \node [root=Selection, state, selected] (selection-root) {$s$}
        child [emph1, ->] { node [action, selected] {$a$}
            % child [state] { node {} }
            % child [] { edge from parent[invisible] node[invisible] {} }
        }
        % child [state] {node [action] {}}
        child [state] {node [action] {}
            % child [state] { node {} }
            % child [state] { node {} }
        };
\end{tikzpicture}
\hspace{10mm}
\begin{tikzpicture}
    \node [root=Expansion, state] {$s$}
        child { node [action, semiselected] {$a$}
            % child [state] { node {} }
            child [state, emph1, ->] { node [semiselected, label={[label distance=-0.5mm]right:\scriptsize{$\sim T(\cdot \mid s, a)$}}] {$s'$} }
        }
        % child [state] {node [action] {}}
        child [state] {node [action] {}
            % child [state] { node {} }
            % child [state] { node {} }
        };
\end{tikzpicture}
\hspace{3mm}
\begin{tikzpicture}
    \node [root=Rollout, state] {$s$}
        child { node [action] {$a$}
            % child [state] { node {} }
            child [state] { node [state, selected] {$s'$}
                child [state, emph4, level distance=0.74in] { node [rollout-end, invisible] {$r + \gamma V(s')$}
                    edge from parent [rollout-edge]
                }
            }
        }
        % child [state] {node [action] {}}
        child [state] {node [action] {}
            % child [state] { node {} }
            % child [state] { node {} }
        };
\end{tikzpicture}
\hspace{5mm}
\begin{tikzpicture}
    \node [root=Backpropagation, state, selected] {$s$}
        child [<-,emph1] { node [action, selected] {$a$}
            % child [state] { node {} }
            child [state, emph1, <-] { node [state, selected] {$s'$} edge from parent node[right, pos=0.6, draw=none, fill=none, text=black] {\scriptsize{$Q$-value}}}
        }
        % child [state] {node [action] {}}
        child [state] {node [action] {}
            % child [state] { node {} }
            % child [state] { node {} }
        };
\end{tikzpicture}

\end{figure}

\begin{figure*}[hb!]
    \centering
    \includegraphics[width=0.8\textwidth]{figures/results/value_and_policy_plots_lightdark_betazero.pdf}
    \caption{\textsc{LightDark}$(10)$ value and policy plots: BetaZero (top) and value iteration (bottom) over belief mean and std. High uncertainty (horizontal axis) makes the agent localize \texttt{up} near $y=10$, then moves \texttt{down} and \texttt{stops} at the origin.}
    \label{fig:lightdark_value_policy}
\end{figure*}

\section{Experiments}\label{sec:experiments}

Three benchmark problems were chosen to evaluate the performance of BetaZero.
Appendices further describe the POMDPs, network architectures, and experiment design.


\begin{table}[ht]
    \centering
    \begin{threeparttable}
        \begin{small}
        \begin{tabular}{@{}lrrr@{}}
            \toprule
                                                   &  $|\mathcal{S}|$              &  $|\mathcal{A}|$  &  $|\mathcal{O}|$  \\
            \midrule
            $\text{LightDark}(5 \text{ and } 10)$  &  $|\mathbb{R}|$               &  $3$              &  $|\mathbb{R}|$  \\
            $\text{RockSample}(15,15)$             &  $7{,}372{,}800$              &  $20$             &  $3$  \\
            $\text{RockSample}(20,20)$             &  $419{,}430{,}400$            &  $25$             &  $3$  \\
            Mineral Exploration                    &  $|\mathbb{R}^{32\times32}|$  &  $38$             &  $|\mathbb{R}_{\ge 0}|$  \\
            \bottomrule
        \end{tabular}
        \end{small}
    \end{threeparttable}
    \caption{POMDP state, action, and observation spaces.}\label{tab:spaces}
\end{table}

In \textsc{LightDark$(y)$} \cite{platt2010belief}, the goal of the agent is to execute a \texttt{stop} action at the origin while receiving noisy observations of its true location.
The noise is minimized in the ``light'' region ${y=5}$.
We also benchmark against a harder version with the light region at ${y=10}$ from \citeauthor{sunberg2018online} and restrict the agent to only three actions: move \texttt{up} or \texttt{down} by one, or \texttt{stop} (removing actions of moving ten units).
The modified problem requires information gathering over longer horizons.
 
\textsc{RockSample$(n,k)$} \cite{smith2012heuristic} is a scalable information gathering problem where an agent moves in an $n \times n$ grid to observe $k$ rocks to sample only the ``good'' rocks.
Well-established POMDP benchmarks go up to $n=15$ and $k=15$; we also test a harder version with $n=20$ and $k=20$ to show the scalability of BetaZero, noting that \citeauthor{cai2021hyp} evaluated this in the multi-agent case.

In the real-world \textsc{Mineral Exploration} problem \cite{mern2023intelligent} the agent drills over a $32\times32$ region to determine if a subsurface ore body should be mined or abandoned. % (where $50\%$ of cases are economical to mine).
The agent receives a continuous ore quality observation at the drill locations to build its belief.
Drilling incurs a penalty and if chosen to mine then the agent is rewarded or penalized  based on an economic threshold of the extracted ore mass.
The problem is challenging due to reasoning over limited observations with sparse rewards.

We baseline BetaZero against several online POMDP algorithms, namely AdaOPS \cite{wu2021adaptive}, POMCPOW \cite{sunberg2018online}, and DESPOT \cite{ye2017despot}.
In LightDark, we solve for an approximately optimal policy using \textit{local approximation value iteration} (LAVI) \cite{dmubook} over a discretized parametric belief space.
For a fair comparison, parameters were set to roughly match the total number of simulations experienced.


\section{Results and Discussion}
\Cref{fig:lightdark_value_policy} compares the raw BetaZero value and policy network with \textit{value iteration} for \textsc{LightDark}$(10)$.
Qualitatively, BetaZero learns an accurate optimal policy and value function in areas where training data was collected.
Despite unrepresented value function regions (gray), BetaZero remains nearly optimal as those beliefs don't occur during execution.
Out-of-distribution methods could quantify this uncertainty, e.g., an ensemble of networks \cite{salehi2022unified}.


\begin{figure}[t!]
    \centering
    \resizebox{\linewidth}{!}{
        % Recommended preamble:
% \usetikzlibrary{arrows.meta}
% \usetikzlibrary{backgrounds}
% \usepgfplotslibrary{patchplots}
% \usepgfplotslibrary{fillbetween}
% \pgfplotsset{%
%     layers/standard/.define layer set={%
%         background,axis background,axis grid,axis ticks,axis lines,axis tick labels,pre main,main,axis descriptions,axis foreground%
%     }{
%         grid style={/pgfplots/on layer=axis grid},%
%         tick style={/pgfplots/on layer=axis ticks},%
%         axis line style={/pgfplots/on layer=axis lines},%
%         label style={/pgfplots/on layer=axis descriptions},%
%         legend style={/pgfplots/on layer=axis descriptions},%
%         title style={/pgfplots/on layer=axis descriptions},%
%         colorbar style={/pgfplots/on layer=axis descriptions},%
%         ticklabel style={/pgfplots/on layer=axis tick labels},%
%         axis background@ style={/pgfplots/on layer=axis background},%
%         3d box foreground style={/pgfplots/on layer=axis foreground},%
%     },
% }

\begin{tikzpicture}[/tikz/background rectangle/.style={fill={rgb,1:red,1.0;green,1.0;blue,1.0}, fill opacity={1.0}, draw opacity={1.0}}, show background rectangle]
\begin{axis}[point meta max={nan}, point meta min={nan}, legend cell align={left}, legend columns={2}, title={Online performance in \textsc{RockSample}$(15,15)$}, title style={at={{(0.5,1)}}, anchor={south}, font={{\fontsize{15 pt}{19.5 pt}\selectfont}}, color={rgb,1:red,0.0;green,0.0;blue,0.0}, draw opacity={1.0}, rotate={0.0}, align={center}}, legend style={color={rgb,1:red,0.0;green,0.0;blue,0.0}, draw opacity={1.0}, line width={1}, solid, fill={rgb,1:red,1.0;green,1.0;blue,1.0}, fill opacity={1.0}, text opacity={1.0}, font={{\fontsize{13 pt}{17.2 pt}\selectfont}}, text={rgb,1:red,0.0;green,0.0;blue,0.0}, cells={anchor={west}}, at={(0.98, 0.98)}, anchor={north east}}, axis background/.style={fill={rgb,1:red,1.0;green,1.0;blue,1.0}, opacity={1.0}}, anchor={north west}, xshift={20.0mm}, yshift={-5.0mm}, width={146.215mm}, height={48.01mm}, scaled x ticks={false}, xlabel={online planning iterations}, x tick style={color={rgb,1:red,0.0;green,0.0;blue,0.0}, opacity={1.0}}, x tick label style={color={rgb,1:red,0.0;green,0.0;blue,0.0}, opacity={1.0}, rotate={0}}, xlabel style={at={(ticklabel cs:0.5)}, anchor=near ticklabel, at={{(ticklabel cs:0.5)}}, anchor={near ticklabel}, font={{\fontsize{14 pt}{18.2 pt}\selectfont}}, color={rgb,1:red,0.0;green,0.0;blue,0.0}, draw opacity={1.0}, rotate={0.0}}, xmode={log}, log basis x={10}, xmajorgrids={true}, xmin={10.0}, xmax={1.0e7}, xticklabels={{$10^{1}$,$10^{2}$,$10^{3}$,$10^{4}$,$10^{5}$,$10^{6}$,$10^{7}$}}, xtick={{10.0,100.0,1000.0,10000.0,100000.0,1.0e6,1.0e7}}, xtick align={inside}, xticklabel style={font={{\fontsize{14 pt}{18.2 pt}\selectfont}}, color={rgb,1:red,0.0;green,0.0;blue,0.0}, draw opacity={1.0}, rotate={0.0}}, x grid style={color={rgb,1:red,0.0;green,0.0;blue,0.0}, draw opacity={0.1}, line width={0.5}, solid}, xticklabel pos={left}, x axis line style={color={rgb,1:red,0.0;green,0.0;blue,0.0}, draw opacity={1.0}, line width={1}, solid}, scaled y ticks={false}, ylabel={returns}, y tick style={color={rgb,1:red,0.0;green,0.0;blue,0.0}, opacity={1.0}}, y tick label style={color={rgb,1:red,0.0;green,0.0;blue,0.0}, opacity={1.0}, rotate={0}}, ylabel style={at={(ticklabel cs:0.5)}, anchor=near ticklabel, at={{(ticklabel cs:0.5)}}, anchor={near ticklabel}, font={{\fontsize{14 pt}{18.2 pt}\selectfont}}, color={rgb,1:red,0.0;green,0.0;blue,0.0}, draw opacity={1.0}, rotate={0.0}}, ymajorgrids={true}, ymin={2.1906999999999996}, ymax={19.479300000000002}, yticklabels={{$4$,$6$,$8$,$10$,$12$,$14$,$16$,$18$}}, ytick={{4.0,6.0,8.0,10.0,12.0,14.0,16.0,18.0}}, ytick align={inside}, yticklabel style={font={{\fontsize{14 pt}{18.2 pt}\selectfont}}, color={rgb,1:red,0.0;green,0.0;blue,0.0}, draw opacity={1.0}, rotate={0.0}}, y grid style={color={rgb,1:red,0.0;green,0.0;blue,0.0}, draw opacity={0.1}, line width={0.5}, solid}, yticklabel pos={left}, y axis line style={color={rgb,1:red,0.0;green,0.0;blue,0.0}, draw opacity={1.0}, line width={1}, solid}, colorbar={false}]
    \addplot+[line width={0}, draw opacity={0}, fill={rgb,1:red,0.7686;green,0.3059;blue,0.3216}, fill opacity={0.1}, mark={none}, forget plot]
        coordinates {
            (10.0,16.93)
            (100.0,18.11)
            (1000.0,18.3)
            (1000.0,17.61)
            (100.0,17.39)
            (10.0,16.259999999999998)
            (10.0,16.93)
        }
        ;
    \addplot+[line width={0}, draw opacity={0}, fill={rgb,1:red,0.7686;green,0.3059;blue,0.3216}, fill opacity={0.1}, mark={none}, forget plot]
        coordinates {
            (10.0,16.93)
            (100.0,18.11)
            (1000.0,18.3)
            (1000.0,18.990000000000002)
            (100.0,18.83)
            (10.0,17.6)
            (10.0,16.93)
        }
        ;
    \addplot[color={rgb,1:red,0.7686;green,0.3059;blue,0.3216}, name path={8a703a3d-016d-4836-81e2-e9e785df520a}, legend image code/.code={{
    \draw[fill={rgb,1:red,0.7686;green,0.3059;blue,0.3216}, fill opacity={0.1}] (0cm,-0.1cm) rectangle (0.6cm,0.1cm);
    }}, draw opacity={1.0}, line width={2}, solid, mark={*}, mark size={2.25 pt}, mark repeat={1}, mark options={color={rgb,1:red,0.7686;green,0.3059;blue,0.3216}, draw opacity={1.0}, fill={rgb,1:red,1.0;green,1.0;blue,1.0}, fill opacity={1.0}, line width={0.75}, rotate={0}, solid}]
        table[row sep={\\}]
        {
            \\
            10.0  16.93  \\
            100.0  18.11  \\
            1000.0  18.3  \\
        }
        ;
    \addlegendentry {BetaZero}
    \addplot[color={rgb,1:red,0.7686;green,0.3059;blue,0.3216}, name path={cca4d01f-76a8-4505-983f-3cc86491141c}, draw opacity={0.1}, line width={1}, solid, forget plot]
        table[row sep={\\}]
        {
            \\
            10.0  17.6  \\
            100.0  18.83  \\
            1000.0  18.990000000000002  \\
        }
        ;
    \addplot[color={rgb,1:red,0.7686;green,0.3059;blue,0.3216}, name path={ee1bbd94-1962-4c9f-85c9-13080faa0b79}, draw opacity={0.1}, line width={1}, solid, forget plot]
        table[row sep={\\}]
        {
            \\
            10.0  16.259999999999998  \\
            100.0  17.39  \\
            1000.0  17.61  \\
        }
        ;
    \addplot+[line width={0}, draw opacity={0}, fill={rgb,1:red,0.5059;green,0.4471;blue,0.698}, fill opacity={0.1}, mark={none}, forget plot]
        coordinates {
            (10.0,10.96)
            (1.0e7,10.96)
            (1.0e7,9.98)
            (10.0,9.98)
            (10.0,10.96)
        }
        ;
    \addplot+[line width={0}, draw opacity={0}, fill={rgb,1:red,0.5059;green,0.4471;blue,0.698}, fill opacity={0.1}, mark={none}, forget plot]
        coordinates {
            (10.0,10.96)
            (1.0e7,10.96)
            (1.0e7,11.940000000000001)
            (10.0,11.940000000000001)
            (10.0,10.96)
        }
        ;
    \addplot[color={rgb,1:red,0.5059;green,0.4471;blue,0.698}, name path={e981e9fb-39d0-4237-ac7e-c386e30aab9e}, legend image code/.code={{
    \draw[fill={rgb,1:red,0.5059;green,0.4471;blue,0.698}, fill opacity={0.1}] (0cm,-0.1cm) rectangle (0.6cm,0.1cm);
    }}, draw opacity={1.0}, line width={1}, dashed]
        table[row sep={\\}]
        {
            \\
            10.0  10.96  \\
            1.0e7  10.96  \\
        }
        ;
    \addlegendentry {Raw Policy $P_\theta$}
    \addplot+[line width={0}, draw opacity={0}, fill={rgb,1:red,0.298;green,0.4471;blue,0.6902}, fill opacity={0.1}, mark={none}, forget plot]
        coordinates {
            (10.0,2.85)
            (100.0,4.16)
            (1000.0,4.9)
            (10000.0,6.6)
            (100000.0,11.18)
            (1.0e6,11.31)
            (1.0e7,11.1)
            (1.0e7,10.549999999999999)
            (1.0e6,10.73)
            (100000.0,10.62)
            (10000.0,6.35)
            (1000.0,4.44)
            (100.0,3.71)
            (10.0,2.68)
            (10.0,2.85)
        }
        ;
    \addplot+[line width={0}, draw opacity={0}, fill={rgb,1:red,0.298;green,0.4471;blue,0.6902}, fill opacity={0.1}, mark={none}, forget plot]
        coordinates {
            (10.0,2.85)
            (100.0,4.16)
            (1000.0,4.9)
            (10000.0,6.6)
            (100000.0,11.18)
            (1.0e6,11.31)
            (1.0e7,11.1)
            (1.0e7,11.65)
            (1.0e6,11.89)
            (100000.0,11.74)
            (10000.0,6.85)
            (1000.0,5.36)
            (100.0,4.61)
            (10.0,3.02)
            (10.0,2.85)
        }
        ;
    \addplot[color={rgb,1:red,0.298;green,0.4471;blue,0.6902}, name path={701e7188-16e7-4ef8-85e7-9d5dcda770b9}, legend image code/.code={{
    \draw[fill={rgb,1:red,0.298;green,0.4471;blue,0.6902}, fill opacity={0.1}] (0cm,-0.1cm) rectangle (0.6cm,0.1cm);
    }}, draw opacity={1.0}, line width={2}, solid, mark={*}, mark size={2.25 pt}, mark repeat={1}, mark options={color={rgb,1:red,0.298;green,0.4471;blue,0.6902}, draw opacity={1.0}, fill={rgb,1:red,1.0;green,1.0;blue,1.0}, fill opacity={1.0}, line width={0.75}, rotate={0}, solid}]
        table[row sep={\\}]
        {
            \\
            10.0  2.85  \\
            100.0  4.16  \\
            1000.0  4.9  \\
            10000.0  6.6  \\
            100000.0  11.18  \\
            1.0e6  11.31  \\
            1.0e7  11.1  \\
        }
        ;
    \addlegendentry {POMCPOW}
    \addplot[color={rgb,1:red,0.298;green,0.4471;blue,0.6902}, name path={216d847e-55bb-4787-9ced-3f16985f8f17}, draw opacity={0.1}, line width={1}, solid, forget plot]
        table[row sep={\\}]
        {
            \\
            10.0  3.02  \\
            100.0  4.61  \\
            1000.0  5.36  \\
            10000.0  6.85  \\
            100000.0  11.74  \\
            1.0e6  11.89  \\
            1.0e7  11.65  \\
        }
        ;
    \addplot[color={rgb,1:red,0.298;green,0.4471;blue,0.6902}, name path={3282b7b8-c30f-4def-bacd-195f887c6920}, draw opacity={0.1}, line width={1}, solid, forget plot]
        table[row sep={\\}]
        {
            \\
            10.0  2.68  \\
            100.0  3.71  \\
            1000.0  4.44  \\
            10000.0  6.35  \\
            100000.0  10.62  \\
            1.0e6  10.73  \\
            1.0e7  10.549999999999999  \\
        }
        ;
    \addplot+[line width={0}, draw opacity={0}, fill={rgb,1:red,0.3333;green,0.6588;blue,0.4078}, fill opacity={0.1}, mark={none}, forget plot]
        coordinates {
            (10.0,9.96)
            (1.0e7,9.96)
            (1.0e7,9.31)
            (10.0,9.31)
            (10.0,9.96)
        }
        ;
    \addplot+[line width={0}, draw opacity={0}, fill={rgb,1:red,0.3333;green,0.6588;blue,0.4078}, fill opacity={0.1}, mark={none}, forget plot]
        coordinates {
            (10.0,9.96)
            (1.0e7,9.96)
            (1.0e7,10.610000000000001)
            (10.0,10.610000000000001)
            (10.0,9.96)
        }
        ;
    \addplot[color={rgb,1:red,0.3333;green,0.6588;blue,0.4078}, name path={5fdd37aa-8fd8-4299-b29a-5fe8ead9b56f}, legend image code/.code={{
    \draw[fill={rgb,1:red,0.3333;green,0.6588;blue,0.4078}, fill opacity={0.1}] (0cm,-0.1cm) rectangle (0.6cm,0.1cm);
    }}, draw opacity={1.0}, line width={1}, dashed]
        table[row sep={\\}]
        {
            \\
            10.0  9.96  \\
            1.0e7  9.96  \\
        }
        ;
    \addlegendentry {Raw Value $V_\theta$}
\end{axis}
\end{tikzpicture}

    }
    \caption{Performance of POMCPOW with heuristics up to $10$ million online iterations plateaus, indicating that extending online searches alone misses valuable offline experience.}
    \label{fig:online}
\end{figure}


\begin{table*}[t!]
    \centering
    \begin{threeparttable}
        \begin{adjustbox}{max width=\textwidth}
        \begin{tabular}{@{}lrrrrrrrrrr@{}}
            \arrayrulecolor{black} % revert
            \toprule
                &  \multicolumn{2}{c}{$\text{LightDark}(5)$}  &  \multicolumn{2}{c}{$\text{LightDark}(10)$}  &  \multicolumn{2}{c}{$\text{RockSample}(15,15)$}  &  \multicolumn{2}{c}{$\text{RockSample}({20,20})$}  &  \multicolumn{2}{c}{Mineral Exploration} \\
            \arrayrulecolor{lightgray}
            \cmidrule{2-11}
            \arrayrulecolor{black} % revert
                &  returns  &  time [s]  &  returns  &  time [s]  &  returns  &  time [s]  &  returns  &  time [s]  &  returns  &  time [s] \\
            \midrule
            \arrayrulecolor{white}
            BetaZero  &  $\mathBF{4.47 \pm 0.28}$  &  \tcolor{$[\num{2274},\,\num{0.014}]$}  &  $\mathBF{16.77 \pm 1.28}$  &  \tcolor{$[\num{2740},\,\num{0.331}]$}  &  $\num{20.15 \pm 0.71}$  &  \tcolor{$[\num{5701},\,\num{0.477}]$}  &  $\mathBF{13.09 \pm 0.55}$  &  \tcolor{$[\num{7081},\,\num{1.109}]$}  &  $\mathBF{10.67 \pm 2.25}$  &  \tcolor{$[\num{22505},\,\num{5.126}]$}  \\
            \midrule
            Raw Policy $P_\theta$  &  $\num{4.44 \pm 0.28}$  &  \tcolor{$[\num{2274},\,\num{0.004}]$}  &  $\num{13.74 \pm 1.33}$  &  \tcolor{$[\num{2740},\,\num{0.004}]$}  &  $\num{10.96 \pm 0.98}$  &  \tcolor{$[\num{5701},\,\num{0.018}]$}  &  $\num{2.03 \pm 0.34}$  &  \tcolor{$[\num{7081},\,\num{0.084}]$}  &  $\num{8.67 \pm 2.52}$  &  \tcolor{$[\num{22505},\,\num{0.533}]$}  \\
            \midrule
            Raw Value $V_\theta$\tnote{*}  &  $\num{3.16 \pm 0.4}$  &  \tcolor{$[\num{2274},\,\num{0.008}]$}  &  $\num{12.7 \pm 1.46}$  &  \tcolor{$[\num{2740},\,\num{0.009}]$}  &  $\num{9.96 \pm 0.65}$  &  \tcolor{$[\num{5701},\,\num{0.158}]$}  &  $\num{3.57 \pm 0.40}$  &  \tcolor{$[\num{7081},\,\num{0.204}]$}  &  $\num{9.75 \pm 2.42}$  &  \tcolor{$[\num{22505},\,\num{1.420}]$}  \\
            % 
            \arrayrulecolor{black}\midrule
            % 
            \tworow{AdaOPS}  &  $\num{3.78 \pm 0.27}$  &  \tworow{\tcolor{$[\num{68},\,\num{0.089}]$}}  &  \tworow{$\num{5.22 \pm 1.77}$}  &  \tworow{\tcolor{$[\num{81},\,\num{0.510}]$}}  &  $\mathBF{20.67 \pm 0.72}$  &  \tworow{\tcolor{$[\num{7},\,\num{2.768}]$}}  &  \tworow{---}  &  \tworow{---}  &  \tworow{$\num{3.33 \pm 1.95}$}  &  \tworow{\tcolor{$[\num{5},\,\num{0.112}]$}}  \\
                             &  \lit{$\num{3.79 \pm 0.07}$}  &  &  &  & \lit{$\num{17.16 \pm 0.21}$}  &  &  &  &  &  \\
            \arrayrulecolor{white}\midrule
            AdaOPS (fixed bounds)  &  $\num{3.7 \pm 0.25}$  &  \tcolor{$[\num{0},\,\num{0.039}]$}  &  $\num{4.98 \pm 2.01}$  &  \tcolor{$[\num{0},\,\num{0.573}]$}  &  $\num{13.37 \pm 0.71}$  &  \tcolor{$[\num{0},\,\num{1.349}]$}  &  $\num{11.66 \pm 0.49}$  &  \tcolor{$[\num{1},\,\num{1.458}]$}  &  \sameresults  &  \sameresults  \\
            % 
            \arrayrulecolor{grays1}\midrule
            % 
            \tworow{POMCPOW}  &  $\num{3.21 \pm 0.38}$  &  \tworow{\tcolor{$[\num{59},\,\num{0.189}]$}}  &  \tworow{$\num{0.68 \pm 0.41}$}  &  \tworow{\tcolor{$[\num{70},\,\num{1.261}]$}}  &  $\num{11.14 \pm 0.59}$  &  \tworow{\tcolor{$[\num{0},\,\num{0.929}]$}}  &  \tworow{$\num{10.22 \pm 0.47}$}  &  \tworow{\tcolor{$[\num{0},\,\num{1.480}]$}}  &  \tworow{$\num{9.43 \pm 2.19}$}  &  \tworow{\tcolor{$[\num{0},\,\num{6.728}]$}}  \\
                              &  \lit{$\num{3.23 \pm 0.11}$}  &  &  &  &  \lit{$\num{10.40 \pm 0.18}$}  &  &  &  &  &  \\
            \arrayrulecolor{white}\midrule
            POMCPOW (no heuristics)  &  $\num{1.96 \pm 0.58}$  &  \tcolor{$[\num{0},\,\num{0.099}]$}  &  $\num{-5.9 \pm 5.78}$  &  \tcolor{$[\num{0},\,\num{0.742}]$}  &  $\num{10.17 \pm 0.61}$  &  \tcolor{$[\num{0},\,\num{1.485}]$}  &  $\num{4.03 \pm 0.44}$  &  \tcolor{$[\num{0},\,\num{5.173}]$}  &  $\num{5.38 \pm 2.15}$  &  \tcolor{$[\num{0},\,\num{5.915}]$}  \\
            % 
            \arrayrulecolor{grays1}\midrule
            % 
            \tworow{DESPOT}  &  $\num{2.37 \pm 0.37}$  &  \tworow{\tcolor{$[\num{0},\,\num{0.008}]$}}  &  \tworow{$\num{0.43 \pm 0.36}$}  &  \tworow{\tcolor{$[\num{0},\,\num{0.046}]$}}  &  $\num{18.44 \pm 0.69}$  &  \tworow{\tcolor{$[\num{7},\,\num{3.822}]$}}  &  \tworow{---}  &  \tworow{---}  &  \tworow{$\num{5.29 \pm 2.17}$}  &  \tworow{\tcolor{$[\num{5},\,\num{0.283}]$}}  \\
                             &  \lit{$\num{2.50 \pm 0.10}$}  &  &  &  &  \lit{$\num{15.67 \pm 0.20}$}  &  &  &  &  &  \\
            \arrayrulecolor{white}\midrule
            DESPOT (fixed bounds)  &  $\num{2.70 \pm 0.50}$  &  \tcolor{$[\num{0},\,\num{0.008}]$}  &  $\num{0.49 \pm 0.30}$  &  \tcolor{$[\num{0},\,\num{0.025}]$}  &  $\num{4.29 \pm 0.45}$  &  \tcolor{$[\num{0},\,\num{5.091}]$}  &  $\num{0.00 \pm 0.00}$  &  \tcolor{$[\num{0},\,\num{5.179}]$}  &  \sameresults  &  \sameresults \\
            % 
            \arrayrulecolor{black}\midrule
            % \tnote{$\dagger$}
            Approx. Optimal  &  $\num{4.09 \pm 0.33}$  &  \tcolor{$[\num{267},\,\num{0.037}]$}  &  $\num{14.16 \pm 1.39}$  &  \tcolor{$[\num{260},\,\num{0.025}]$}  &  ---  &  ---  &  ---  &  ---  &  $\num{11.9 \pm 0.18}$  &  N/A  \\
            \arrayrulecolor{black} % revert
            \bottomrule
        \end{tabular}
        \end{adjustbox}
        \begin{scriptsize}
            \begin{tablenotes}
                \item[*] {One-step look-ahead over all actions using only the value network with $5$ observations per action. All results report the mean return $\pm$ standard error over $100$ seeds.}
                \item[\phantom{$\dagger$}] {Entries with ``---'' indicate they failed to run on that domain, entries with \textdoublequotes{} are the same as the ones above, and entries in \litdesc{parentheses} are from the literature.}
            \end{tablenotes}
        \end{scriptsize}
    \end{threeparttable}
    \caption{Results comparing \textit{BetaZero} to various state-of-the-art POMDP solvers (reporting returns and [\textit{offline}, \textit{online}] timing).}\label{tab:results}
\end{table*}
    

\Cref{tab:results} shows that BetaZero outperforms state-of-the-art algorithms in most cases, with larger improvements when baseline algorithms do not rely on heuristics.
Timing results show BetaZero incurs a significant offline penalty, further elaborated in the limitations section below.


\newcommand*{\SHOWPLOTS}{}
\ifdefined\SHOWPLOTS
    \begin{figure}[b!]
        \centering
        \input{data/minex_data}% load minex data files (BetaZero)
        \renewcommand*{\algcaption}{BetaZero}
        \resizebox{\linewidth}{!}{%
            % Defining by hand the axis
\begin{tikzpicture}[x={(0cm,-0.5cm)},y={(0cm,0.5cm)},z={(0cm,4cm)}, line join=round, tdplot_main_coords]

% Defining hsb color to have a color scale
\colorlet{highcolorbar}[hsb]{gray} % teal, black
\colorlet{lowcolorbar}[hsb]{white} % lime, white


\definecolor{viridis_yellow}{HTML}{fde725} 
\definecolor{viridis_blue}{HTML}{440154}
\definecolor{highground}{HTML}{e1e697} % black
\definecolor{lowground}{HTML}{44342a} % white
\colorlet{highcolormap}[hsb]{highground}
\colorlet{lowcolormap}[hsb]{lowground}


\ifdefined\SHOWMINIAXES
    % Drawing the system of axes
    \draw[-stealth] (0,0,0) -- (1,0,0) node [black,below left] {$x$};
    \draw[-stealth] (0,0,0) -- (0,1,0) node [black,above left,yshift=-1mm] {$y$};
    \draw[-stealth] (0,0,0) -- (0,0,1) node [black,left] {$p$};
\fi

\node at (3,0,3.5) {\Huge\textsc\algcaption{}};

% Write unit on x and y
\foreach \p in {2,...,\maxX}{ % NOTE: start at 2 not 0 (skip min and max)
    \ifdefined\SHOWLABELS
        \draw {(\p,-\step/2,0)} node[right, gray] {\p}; % x-axis label
    \fi
    \ifdefined\SHOWGRID
        % Draw the grid
        \foreach \q in {2,...,\maxY}{ % NOTE: start at 2 not 0 (skip min and max)
            \draw[lightgray] (\p-\step,\q-\step,0) -- (\p+\step,\q-\step,0) -- (\p+\step,\q+\step,0) -- (\p-\step,\q+\step,0) -- (\p-\step,\q-\step,0);
        }
    \fi
}

\ifdefined\SHOWLABELS
    % y-axis label
    \foreach \p in {2,...,\maxY}{
        \draw {(\maxX+2.5*\step,\p,0)} node[left, gray] {$\p$};
    }
\fi

\ifdefined\SHOWMAP
    % Draw map on the bottom
    \foreach \p in {2,...,\maprows}{ % NOTE: start at 2 not 0 (skip min and max)
        \pgfplotstablegetelem{\p}{[index] 0}\of{\maptable}    % The order in which the bars are drawn is determined by the order of the lines in the data file.
        \pgfmathsetmacro{\x}{\pgfplotsretval/\scale}
        \pgfplotstablegetelem{\p}{[index] 1}\of{\maptable}    % And as the drawings just pile up, the last one just goes on top of the previous drawings.
        \pgfmathsetmacro{\y}{\pgfplotsretval/\scale}
        \pgfplotstablegetelem{\p}{[index] 2}\of{\maptable}    % The order here works with chosen view angle, if you change the angle, you might have to change it.
        \pgfmathsetmacro{\z}{\pgfplotsretval/\scale}

        \pgfmathsetmacro{\w}{\step + 0.05/\scale} % half the width of the bars (plus some padding to cover grid area)

        \pgfmathtruncatemacro{\teinte}{100-(((\z-\minZS)/(\maxZS-\minZS))*100)}
        \colorlet{colour}[rgb]{lowcolormap!\teinte!highcolormap}

        % Bottom heatmap / state (bottom face)
        \fill[colour] (\x-\w,\y-\w,0) -- (\x-\w,\y+\w,0) -- (\x+\w,\y+\w,0) -- (\x+\w,\y-\w,0) -- (\x-\w,\y-\w,0);
    }
\fi


\ifdefined\SHOWOUTSIDEBOX
    \draw[black, line width=0.25mm, line join=round] (\step,\step,0) -- (\maxX+\step,\step,0) -- (\maxX+\step,\maxY+\step,0) -- (\step,\maxY+\step,0) -- (\step,\step,0);
\fi


% Plot vertical probability bars
\foreach \p in {2,...,\rows}{ % NOTE: start at 2 not 0 (skip min and max)
        \pgfplotstablegetelem{\p}{[index] 0}\of{\firsttable}    % The order in which the bars are drawn is determined by the order of the lines in the data file.
        \pgfmathsetmacro{\x}{\pgfplotsretval/\scale}
        \pgfplotstablegetelem{\p}{[index] 1}\of{\firsttable}    % And as the drawings just pile up, the last one just goes on top of the previous drawings.
        \pgfmathsetmacro{\y}{\pgfplotsretval/\scale}
        \pgfplotstablegetelem{\p}{[index] 2}\of{\firsttable}    % The order here works with chosen view angle, if you change the angle, you might have to change it.
        \pgfmathsetmacro{\z}{\Zceiling*(\pgfplotsretval - (\minZ-\epsZ))/(\maxZ - (\minZ-\epsZ))}

        \pgfmathsetmacro{\w}{\step} % half the width of the bars

        % \ifnum0<\z % TODO: Only plot non-zero actions
            \pgfmathtruncatemacro{\teinte}{100*(1 - \z/\Zceiling)}
            \colorlet{colour}[rgb]{lowcolorbar!\teinte!highcolorbar}

            % Unseen faces from orginal view, but if you change the angle ....
            % \fill[colour] (\x-\w,\y-\w,\z) -- (\x-\w,\y+\w,\z) -- (\x-\w,\y+\w,0) -- (\x-\w,\y-\w,0) -- (\x-\w,\y-\w,\z);
            % \draw[black] (\x-\w,\y-\w,\z) -- (\x-\w,\y+\w,\z) -- (\x-\w,\y+\w,0) -- (\x-\w,\y-\w,0) -- (\x-\w,\y-\w,\z);
            % \fill[colour] (\x-\w,\y+\w,\z) -- (\x+\w,\y+\w,\z) -- (\x+\w,\y+\w,0) -- (\x-\w,\y+\w,0) -- (\x-\w,\y+\w,\z);
            % \draw[black](\x-\w,\y+\w,\z) -- (\x+\w,\y+\w,\z) -- (\x+\w,\y+\w,0) -- (\x-\w,\y+\w,0) -- (\x-\w,\y+\w,\z);

            % Visible faces from original view
            \fill[colour] (\x+\w,\y+\w,\z) -- (\x+\w,\y-\w,\z) -- (\x+\w,\y-\w,0) -- (\x+\w,\y+\w,0) -- (\x+\w,\y+\w,\z);
            \draw[black](\x+\w,\y+\w,\z) -- (\x+\w,\y-\w,\z) -- (\x+\w,\y-\w,0) -- (\x+\w,\y+\w,0) -- (\x+\w,\y+\w,\z);

            \fill[colour!60!gray] (\x+\w,\y-\w,\z) -- (\x-\w,\y-\w,\z) -- (\x-\w,\y-\w,0) -- (\x+\w,\y-\w,0) -- (\x+\w,\y-\w,\z);
            \draw[black](\x+\w,\y-\w,\z) -- (\x-\w,\y-\w,\z) -- (\x-\w,\y-\w,0) -- (\x+\w,\y-\w,0) -- (\x+\w,\y-\w,\z);

            % Top face
            \pgfmathparse{\actionx==\x && \actiony==\y}
            % \ifnum\pgfmathresult=1
                % Show action on topface
                % \fill[top color=actioncolor!40!gray, bottom color=actioncolor!80!gray] (\x-\w,\y-\w,\z) -- (\x-\w,\y+\w,\z) -- (\x+\w,\y+\w,\z) -- (\x+\w,\y-\w,\z) -- (\x-\w,\y-\w,\z);
            % \else
                \fill[top color=colour!40!gray, bottom color=colour!80!gray] (\x-\w,\y-\w,\z) -- (\x-\w,\y+\w,\z) -- (\x+\w,\y+\w,\z) -- (\x+\w,\y-\w,\z) -- (\x-\w,\y-\w,\z);
            % \fi

            \draw[black] (\x-\w,\y-\w,\z) -- (\x-\w,\y+\w,\z) -- (\x+\w,\y+\w,\z) -- (\x+\w,\y-\w,\z) -- (\x-\w,\y-\w,\z);
        % \fi
}

\colorlet{defaultbar}{lightgray} % brown


\ifthenelse{\equal{\actioninfo}{abandon}}
{
    \colorlet{nocolor}{red}
    \colorlet{yescolor}{defaultbar}
}{
    \ifthenelse{\equal{\actioninfo}{mine}}
    {
        \colorlet{nocolor}{defaultbar}
        \colorlet{yescolor}{red}    
    }{
        \colorlet{nocolor}{defaultbar}
        \colorlet{yescolor}{defaultbar}    
    }
}


\begin{axis}[
    tuftelike,
    xbar,
    at={(2cm,-2.7cm)}, 
    % at={(2cm,-4cm)}, % When using {65} pitch angle
    y=-0.5cm,
    bar width=0.3cm,
    bar shift=0pt,
    xmin=0,
    enlarge y limits={abs=0.45cm},
    xlabel={probability},
    symbolic y coords={abandon,mine},
    every axis plot/.append style={
        ytick=data,
    },
    x tick label style={
        /pgf/number format/.cd,
            fixed,
            fixed zerofill,
            precision=2,
        /tikz/.cd
    },
    nodes near coords, nodes near coords align={horizontal},
]
\addplot[nocolor!20!black, fill=nocolor!80!white] table[col sep=comma,header=false] {
\nodecision,abandon
};
\addplot[yescolor!20!black, fill=yescolor!80!white] table[col sep=comma,header=false] {
\yesdecision,mine
};
\end{axis}

\end{tikzpicture}
        }
        \hfill
        \input{data/minex_data_pomcpow}% load minex data files (POMCPOW)
        \renewcommand*{\algcaption}{POMCPOW}
        \resizebox{\linewidth}{!}{%
            % Defining by hand the axis
\begin{tikzpicture}[x={(0cm,-0.5cm)},y={(0cm,0.5cm)},z={(0cm,4cm)}, line join=round, tdplot_main_coords]

% Defining hsb color to have a color scale
\colorlet{highcolorbar}[hsb]{gray} % teal, black
\colorlet{lowcolorbar}[hsb]{white} % lime, white


\definecolor{viridis_yellow}{HTML}{fde725} 
\definecolor{viridis_blue}{HTML}{440154}
\definecolor{highground}{HTML}{e1e697} % black
\definecolor{lowground}{HTML}{44342a} % white
\colorlet{highcolormap}[hsb]{highground}
\colorlet{lowcolormap}[hsb]{lowground}


\ifdefined\SHOWMINIAXES
    % Drawing the system of axes
    \draw[-stealth] (0,0,0) -- (1,0,0) node [black,below left] {$x$};
    \draw[-stealth] (0,0,0) -- (0,1,0) node [black,above left,yshift=-1mm] {$y$};
    \draw[-stealth] (0,0,0) -- (0,0,1) node [black,left] {$p$};
\fi

\node at (3,0,3.5) {\Huge\textsc\algcaption{}};

% Write unit on x and y
\foreach \p in {2,...,\maxX}{ % NOTE: start at 2 not 0 (skip min and max)
    \ifdefined\SHOWLABELS
        \draw {(\p,-\step/2,0)} node[right, gray] {\p}; % x-axis label
    \fi
    \ifdefined\SHOWGRID
        % Draw the grid
        \foreach \q in {2,...,\maxY}{ % NOTE: start at 2 not 0 (skip min and max)
            \draw[lightgray] (\p-\step,\q-\step,0) -- (\p+\step,\q-\step,0) -- (\p+\step,\q+\step,0) -- (\p-\step,\q+\step,0) -- (\p-\step,\q-\step,0);
        }
    \fi
}

\ifdefined\SHOWLABELS
    % y-axis label
    \foreach \p in {2,...,\maxY}{
        \draw {(\maxX+2.5*\step,\p,0)} node[left, gray] {$\p$};
    }
\fi

\ifdefined\SHOWMAP
    % Draw map on the bottom
    \foreach \p in {2,...,\maprows}{ % NOTE: start at 2 not 0 (skip min and max)
        \pgfplotstablegetelem{\p}{[index] 0}\of{\maptable}    % The order in which the bars are drawn is determined by the order of the lines in the data file.
        \pgfmathsetmacro{\x}{\pgfplotsretval/\scale}
        \pgfplotstablegetelem{\p}{[index] 1}\of{\maptable}    % And as the drawings just pile up, the last one just goes on top of the previous drawings.
        \pgfmathsetmacro{\y}{\pgfplotsretval/\scale}
        \pgfplotstablegetelem{\p}{[index] 2}\of{\maptable}    % The order here works with chosen view angle, if you change the angle, you might have to change it.
        \pgfmathsetmacro{\z}{\pgfplotsretval/\scale}

        \pgfmathsetmacro{\w}{\step + 0.05/\scale} % half the width of the bars (plus some padding to cover grid area)

        \pgfmathtruncatemacro{\teinte}{100-(((\z-\minZS)/(\maxZS-\minZS))*100)}
        \colorlet{colour}[rgb]{lowcolormap!\teinte!highcolormap}

        % Bottom heatmap / state (bottom face)
        \fill[colour] (\x-\w,\y-\w,0) -- (\x-\w,\y+\w,0) -- (\x+\w,\y+\w,0) -- (\x+\w,\y-\w,0) -- (\x-\w,\y-\w,0);
    }
\fi


\ifdefined\SHOWOUTSIDEBOX
    \draw[black, line width=0.25mm, line join=round] (\step,\step,0) -- (\maxX+\step,\step,0) -- (\maxX+\step,\maxY+\step,0) -- (\step,\maxY+\step,0) -- (\step,\step,0);
\fi


% Plot vertical probability bars
\foreach \p in {2,...,\rows}{ % NOTE: start at 2 not 0 (skip min and max)
        \pgfplotstablegetelem{\p}{[index] 0}\of{\firsttable}    % The order in which the bars are drawn is determined by the order of the lines in the data file.
        \pgfmathsetmacro{\x}{\pgfplotsretval/\scale}
        \pgfplotstablegetelem{\p}{[index] 1}\of{\firsttable}    % And as the drawings just pile up, the last one just goes on top of the previous drawings.
        \pgfmathsetmacro{\y}{\pgfplotsretval/\scale}
        \pgfplotstablegetelem{\p}{[index] 2}\of{\firsttable}    % The order here works with chosen view angle, if you change the angle, you might have to change it.
        \pgfmathsetmacro{\z}{\Zceiling*(\pgfplotsretval - (\minZ-\epsZ))/(\maxZ - (\minZ-\epsZ))}

        \pgfmathsetmacro{\w}{\step} % half the width of the bars

        % \ifnum0<\z % TODO: Only plot non-zero actions
            \pgfmathtruncatemacro{\teinte}{100*(1 - \z/\Zceiling)}
            \colorlet{colour}[rgb]{lowcolorbar!\teinte!highcolorbar}

            % Unseen faces from orginal view, but if you change the angle ....
            % \fill[colour] (\x-\w,\y-\w,\z) -- (\x-\w,\y+\w,\z) -- (\x-\w,\y+\w,0) -- (\x-\w,\y-\w,0) -- (\x-\w,\y-\w,\z);
            % \draw[black] (\x-\w,\y-\w,\z) -- (\x-\w,\y+\w,\z) -- (\x-\w,\y+\w,0) -- (\x-\w,\y-\w,0) -- (\x-\w,\y-\w,\z);
            % \fill[colour] (\x-\w,\y+\w,\z) -- (\x+\w,\y+\w,\z) -- (\x+\w,\y+\w,0) -- (\x-\w,\y+\w,0) -- (\x-\w,\y+\w,\z);
            % \draw[black](\x-\w,\y+\w,\z) -- (\x+\w,\y+\w,\z) -- (\x+\w,\y+\w,0) -- (\x-\w,\y+\w,0) -- (\x-\w,\y+\w,\z);

            % Visible faces from original view
            \fill[colour] (\x+\w,\y+\w,\z) -- (\x+\w,\y-\w,\z) -- (\x+\w,\y-\w,0) -- (\x+\w,\y+\w,0) -- (\x+\w,\y+\w,\z);
            \draw[black](\x+\w,\y+\w,\z) -- (\x+\w,\y-\w,\z) -- (\x+\w,\y-\w,0) -- (\x+\w,\y+\w,0) -- (\x+\w,\y+\w,\z);

            \fill[colour!60!gray] (\x+\w,\y-\w,\z) -- (\x-\w,\y-\w,\z) -- (\x-\w,\y-\w,0) -- (\x+\w,\y-\w,0) -- (\x+\w,\y-\w,\z);
            \draw[black](\x+\w,\y-\w,\z) -- (\x-\w,\y-\w,\z) -- (\x-\w,\y-\w,0) -- (\x+\w,\y-\w,0) -- (\x+\w,\y-\w,\z);

            % Top face
            \pgfmathparse{\actionx==\x && \actiony==\y}
            % \ifnum\pgfmathresult=1
                % Show action on topface
                % \fill[top color=actioncolor!40!gray, bottom color=actioncolor!80!gray] (\x-\w,\y-\w,\z) -- (\x-\w,\y+\w,\z) -- (\x+\w,\y+\w,\z) -- (\x+\w,\y-\w,\z) -- (\x-\w,\y-\w,\z);
            % \else
                \fill[top color=colour!40!gray, bottom color=colour!80!gray] (\x-\w,\y-\w,\z) -- (\x-\w,\y+\w,\z) -- (\x+\w,\y+\w,\z) -- (\x+\w,\y-\w,\z) -- (\x-\w,\y-\w,\z);
            % \fi

            \draw[black] (\x-\w,\y-\w,\z) -- (\x-\w,\y+\w,\z) -- (\x+\w,\y+\w,\z) -- (\x+\w,\y-\w,\z) -- (\x-\w,\y-\w,\z);
        % \fi
}

\colorlet{defaultbar}{lightgray} % brown


\ifthenelse{\equal{\actioninfo}{abandon}}
{
    \colorlet{nocolor}{red}
    \colorlet{yescolor}{defaultbar}
}{
    \ifthenelse{\equal{\actioninfo}{mine}}
    {
        \colorlet{nocolor}{defaultbar}
        \colorlet{yescolor}{red}    
    }{
        \colorlet{nocolor}{defaultbar}
        \colorlet{yescolor}{defaultbar}    
    }
}


\begin{axis}[
    tuftelike,
    xbar,
    at={(2cm,-2.7cm)}, 
    % at={(2cm,-4cm)}, % When using {65} pitch angle
    y=-0.5cm,
    bar width=0.3cm,
    bar shift=0pt,
    xmin=0,
    enlarge y limits={abs=0.45cm},
    xlabel={probability},
    symbolic y coords={abandon,mine},
    every axis plot/.append style={
        ytick=data,
    },
    x tick label style={
        /pgf/number format/.cd,
            fixed,
            fixed zerofill,
            precision=2,
        /tikz/.cd
    },
    nodes near coords, nodes near coords align={horizontal},
]
\addplot[nocolor!20!black, fill=nocolor!80!white] table[col sep=comma,header=false] {
\nodecision,abandon
};
\addplot[yescolor!20!black, fill=yescolor!80!white] table[col sep=comma,header=false] {
\yesdecision,mine
};
\end{axis}

\end{tikzpicture}
        }
        \renewcommand*{\algcaption}{} % reset
        \caption{Mineral exploration policies: BetaZero prioritizes uncertainty, matching heuristics from \citeauthor{mern2023intelligent}.}
        \label{fig:minex}
    \end{figure}
\fi

In \textsc{RockSample}$(n,k)$, BetaZero is comparable to AdaOPS and DESPOT which compute an upper bound using QMDP.
QMDP computes the optimal utility of the fully observable MDP over all $k-1$ rock combinations, which scales exponentially in $n$.
For larger state spaces, like \textsc{RockSample}$(20,20)$, the QMDP solution is intractable.
Thus, fixed bounds are used assuming an optimistic $V_\text{max}$ \cite{adaops2021review}
while BetaZero scales well to these higher dimensional problems. % and learn approximations to replace heuristics.
Indicated in \cref{tab:results}, the raw networks alone perform well but outperform when combined with online planning, enabling reasoning with current information.


If online algorithms ran for a large number of iterations, one might expect to see convergence to the optimal policy.
In practice, this may be an intractable number as \cref{fig:online} shows POMCPOW has not reached the required number of iterations for RockSample.
The advantage of BetaZero is that it can generalize from a more diverse set of experiences.


The inability of existing online algorithms to plan over long horizons is also evident in the mineral exploration POMDP (\cref{fig:minex}).
POMCPOW ran for one million online iterations without a value estimator heuristic and BetaZero ran online for $100$ iterations (using about $850{,}000$ offline simulations).
In the figure, the probability of selecting a drilling location is shown as vertical bars for each action, overlaid on the initial belief uncertainty (i.e., the standard deviation of the belief in subsurface ore quality).
BetaZero learned to take actions in areas of the belief space with high uncertainty (which matches the domain-specific heuristic developed for the mineral exploration problem from \citeauthor{mern2023intelligent}), while POMCPOW fails to distinguish between the actions and resembles a uniform policy.

\paragraph{Limitations.}
It is standard for POMDP planning algorithms to assume known models but this may limit the applicability to certain problems where reinforcement learning may be better suited.
We chose a simplified belief representation to allow for further research innovations in using other parametric and non-parametric representations.
Other limitations include compute resource requirements for training neural networks and parallelizing MCTS simulations.
We designed BetaZero to use a single GPU for training and to scale based on available CPUs.
Certain POMDPs may not require the training burden, especially when known heuristics perform well.
BetaZero is useful for long-horizon, high-dimensional continuous POMDPs but may be unnecessary when offline training is computationally limited.
BetaZero is designed for problems where the simulation cost is the dominating factor compared to offline training time.

\section{Conclusions}
We propose the \textit{BetaZero} belief-state planning algorithm for POMDPs; designed to learn from offline experience to inform online decisions.
Planning in belief space explicitly handles state uncertainty and learning offline approximations to replace heuristics enables effective online planning in long-horizon POMDPs.
BetaZero can also scale to larger problems where certain heuristics break down.
Results suggest that BetaZero can solve large-scale POMDPs and learns to plan in belief space using zero heuristics.


\clearpage

% In the unusual situation where you want a paper to appear in the
% references without citing it in the main text, use \nocite
%\nocite{langley00}

\bibliography{bandit}
\bibliographystyle{icml2017}

%\end{document} 

\clearpage

\onecolumn

\appendix

\begin{center}
\large
Supplementary for \\
\Large
Provably Optimal Algorithms for Generalized Linear Contextual Bandits
\vspace{10mm}
\end{center}

\onecolumn


% \tableofcontents{}

% \newpage

\section*{Supplementary Material}
\addcontentsline{toc}{section}{Supplementary Material}


Throughout this discussion, 
we will make frequently use 
of the following standard results
concerning the exponential concentration 
of random variables:

\begin{lemma}[Hoeffding's inequality for independent RVs~\citep{hoeffding1994probability}] Let $Z_1, Z_2, \ldots, Z_n$ be independent bounded random variables with $Z_i \in [a,b]$ for all $i$, then 
    \begin{align*}
        \prob\left( \frac{1}{n} \sum_{i=1}^n (Z_i - \Expo{Z_i}) \ge t \right) \le \exp{\left( -\frac{2nt^2}{(b-a)^2} \right) }
    \end{align*} 
    and 
    \begin{align*}
        \prob\left( \frac{1}{n} \sum_{i=1}^n (Z_i - \Expo{Z_i}) \le -t \right) \le \exp{\left( -\frac{2nt^2}{(b-a)^2} \right) }
    \end{align*} 
    for all $t \ge 0$. 
\end{lemma}

\begin{lemma}[Hoeffding's inequality for sampling with replacement~\citep{hoeffding1994probability}] \label{lem:hoeffding_sampling} Let $\calZ = (Z_1, Z_2, \ldots, Z_N)$ be a finite population of $N$ points with $Z_i \in [a.b]$ for all $i$. Let $X_1, X_2, \ldots X_n$ be a random sample drawn without replacement from $\calZ$. Then for all $t \ge 0$, we have 
    \begin{align*}
        \prob\left( \frac{1}{n} \sum_{i=1}^n (X_i - \mu ) \ge t \right) \le \exp{\left( -\frac{2nt^2}{(b-a)^2} \right) }
    \end{align*} 
    and 
    \begin{align*}
        \prob\left( \frac{1}{n} \sum_{i=1}^n (X_i - \mu ) \le -t \right) \le \exp{\left( -\frac{2nt^2}{(b-a)^2} \right) } \,,
    \end{align*} 
    where $\mu = \frac{1}{N} \sum_{i=1}^{N} Z_i$. 
\end{lemma}

We now discuss one condition that generalizes the exponential concentration to dependent random variables.
\begin{condition}[Bounded difference inequality] \label{cond:BDC} Let $\calZ$ be some set and $\phi: \calZ^n \to \Real$. We say that $\phi$ satisfies the bounded difference assumption if 
there exists $c_1, c_2, \ldots c_n \ge 0$ s.t. for all $i$, we have 
\begin{align*}
    \sup_{Z_1,Z_2, \ldots,Z_n, Z_i^\prime \in \calZ^{n+1} } \abs{\phi (Z_1, \ldots, Z_i, \ldots, Z_n ) - \phi (Z_1, \ldots, Z_i^\prime, \ldots, Z_n ) } \le c_i \,.
\end{align*} 
\end{condition}

\begin{lemma}[McDiarmid’s inequality~\citep{mcdiarmid1989}] \label{lem:McDiarmid} Let $Z_1, Z_2, \ldots, Z_n$ be independent random variables on set $\calZ$ and $\phi : \calZ^n \to \Real$ satisfy bounded difference inequality (\codref{cond:BDC}). Then for all $t>0$, we have 
    \begin{align*}
        \prob\left( \phi(Z_1, Z_2, \ldots, Z_n) - \Expo{\phi(Z_1, Z_2, \ldots, Z_n)} \ge t \right) \le \exp{\left( -\frac{2t^2}{\sum_{i=1}^n c_i^2} \right) } 
    \end{align*} 
    and 
    \begin{align*}
        \prob\left( \phi(Z_1, Z_2, \ldots, Z_n) - \Expo{\phi(Z_1, Z_2, \ldots, Z_n)} \le -t \right) \le \exp{\left( -\frac{2t^2}{\sum_{i=1}^n c_i^2} \right) } \,.
    \end{align*} 
\end{lemma}


\section{Proofs from \secref{sec:ERM_training}}\label{app:proof_erm}

\textbf{Additional notation {} {}} Let $m_1$ be the number of mislabeled points ($\wt S_M$) and $m_2$ be the number of correctly labeled points ($\wt S_C$). Note $m_1 + m_2 = m$. 


\subsection{Proof of \thmref{thm:error_ERM}}


\begin{proof}[Proof of \lemref{lem:fit_mislabeled}] 
    The main idea of our proof is to regard 
    the clean portion of the data 
    ($S \cup \wt S_C$) as fixed.   
    Then, there exists an (unknown) classifier $f^*$ 
    that minimizes the expected risk
    calculated on the (fixed) clean data
    and (random draws of) the mislabeled data $\wt S_M$. 
    % 
    % 
    Formally, 
    \begin{align}
    f^* \defeq \argmin_{f \in \calF} \error_{\widecheck {\calD}} (f) \,, \label{eq:modified_ERM}
    \end{align}
    where $$\widecheck \calD = \frac{n}{m+n} \calS + \frac{m_2}{m+n} \wt \calS_C  + \frac{m_1}{m+n}\calDm \,.$$ 
    Note here that $\widecheck \calD$ is a combination 
    of the \emph{empirical distribution} 
    over correctly labeled data $S \cup \wt S_C$
    and the (population) distribution 
    over mislabeled data $\calDm$.
    Recall that 
    \begin{align}
    \wh f \defeq \argmin_{f \in \calF} \error_{\calS \cup \wt S} (f) \,. \label{eq:orig_ERM}
    \end{align}
    % 
    % 
    Since, $\widehat f$ minimizes 0-1 error 
    on $S \cup \wt S$, using ERM optimality on \eqref{eq:orig_ERM},  
    we have 
    \begin{align}
        \error_{\calS \cup \wt \calS}(\widehat f) \le \error_{
            \calS \cup \wt \calS}(f^*) \,.    \label{eq:step1}
    \end{align}
    Moreover, since $f^*$ is independent of $\wt S_M$, using Hoeffding's bound,
    % \footnote{For a fully rigorous argument,
    % refer to the complete proof in App.~\ref{app:proof_erm}.} 
    we have with probability at least $1-\delta$ that
    \begin{align}
      \error_{\wt \calS_M}(f^*) \le \error_{ \calDm}(f^*) +  \sqrt{\frac{\log(1/\delta)}{2 m_1}} \,. \label{eq:step2} 
    \end{align}
    %$ 
    %for some constant $c_1\le 1/2$. 
    Finally, since $f^*$ is the optimal classifier on $\widecheck \calD$, 
    we have 
    \begin{align}
        \error_{\widecheck \calD}(f^*) \le \error_{\widecheck \calD}(\widehat f) \,. \label{eq:step3}
    \end{align}
    Now to relate \eqref{eq:step1} and \eqref{eq:step3}, we multiply \eqref{eq:step2} by $\frac{m_1}{m+n}$ and add $\frac{n}{m+n} \error_{\calS} (f)  + \frac{m_2}{m+n} \error_{\wt \calS_C} (f)$ both the sides. Hence, 
    we can rewrite \eqref{eq:step2} as follows: 
    \begin{align}
        \error_{\calS \cup \wt\calS}(f^*) \le \error_{ \widecheck \calD}(f^*) +  \frac{m_1}{m+n}\sqrt{\frac{\log(1/\delta)}{2 m_1}} \,. \label{eq:step4} 
    \end{align}
    Now we combine equations \eqref{eq:step1}, \eqref{eq:step4}, and \eqref{eq:step3}, to get 
    \begin{align}
        \error_{\calS \cup \wt \calS}(\wh f) \le \error_{\widecheck \calD}(\wh f) +  \frac{m_1}{m+n}\sqrt{\frac{\log(1/\delta)}{2 m_1}} \,, 
    \end{align}
    which implies 
    \begin{align}
        \error_{ \wt \calS_M}(\wh f) \le \error_{\calDm}(\wh f) + \sqrt{\frac{\log(1/\delta)}{2 m_1}} \,. \label{eq:lemma1_final}
    \end{align}
    Since $\wt S$ is obtained by randomly labeling an unlabeled dataset, we assume $2m_1 \approx m$ \footnote{Formally, with probability at least $1-\delta$, we have  $(m - 2m_1)\le \sqrt{m\log(1/\delta)/2}$.}. Moreover, using $\error_{\calDm} = 1 - \error_{\calD}$ we obtain the desired result.   
    % Combining the above steps and using the fact 
    % that $\error_\calD = 1- \error_{\calDm} $, 
    % we obtain the desired result.
\end{proof}

\begin{proof}[Proof of \lemref{lem:mislabeled_error}]
    Recall $\error_{\wt S} (f) = \frac{m_1}{m} \error_{\wt S_M}(f) + \frac{m_2}{m} \error_{\wt S_C}(f)$. Hence, we have 
    \begin{align}
        2\error_{\wt S}(f) - \error_{\wt S_M}(f) - \error_{\wt S_C}(f) &= \left(\frac{2m_1}{m} \error_{\wt S_M}(f) - \error_{\wt S_M}(f)\right) + \left(\frac{2m_2}{m} \error_{\wt S_C}(f) - \error_{\wt S_C}(f)\right) \\ &= \left(\frac{2m_1}{m} - 1\right) \error_{\wt S_M}(f) + \left(\frac{2m_2}{m} - 1 \right)\error_{\wt S_C} (f) \,.
    \end{align} 
    Since the dataset is labeled uniformly at random, with probability at least $1-\delta$, we have  $\left(\frac{2m_1}{m} - 1\right) \le \sqrt{\frac{\log(1/\delta)}{2m}}$. Similarly, we have with probability at least $1-\delta$, $\left(\frac{2m_2}{m} - 1\right) \le \sqrt{\frac{\log(1/\delta)}{2m}}$. Using union bound, with probability at least $1-\delta$, we have
    % \begin{align}
    %     2\error_{\wt S} - \error_{\wt S_M}(f) - \error_{\wt S_C}(f) \le \sqrt{\frac{\log(2/\delta)}{2m}} \left(\error_{\wt S_M}(f) + \error_{\wt S_C}(f) \right) \le 2\sqrt{\frac{\log(2/\delta)}{2m}} \,. \label{eq:lemma2_final}
    % \end{align}
    \begin{align}
        2\error_{\wt S} - \error_{\wt S_M}(f) - \error_{\wt S_C}(f) \le \sqrt{\frac{\log(2/\delta)}{2m}} \left(\error_{\wt S_M}(f) + \error_{\wt S_C}(f) \right) \,. \label{eq:lemma2_prefinal}
    \end{align}
    With re-arranging $\error_{\wt S_M}(f) + \error_{\wt S_C}(f)$ and using the inequality $ 1- a\le \frac{1}{1+a} $, we have  
    \begin{align}
        2\error_{\wt S} - \error_{\wt S_M}(f) - \error_{\wt S_C}(f) \le 2\error_{\wt \calS} \sqrt{\frac{\log(2/\delta)}{2m}}  \,. \label{eq:lemma2_final}
    \end{align}

    % We obtain the desired result by using 
\end{proof}

\begin{proof}[Proof of \lemref{lem:clear_error}]
% Recall 0-1 error on each point  $(x,y) \in S \cup \wt S$ is given by $\I{ f(x)\ne y}$.
In the set of correctly labeled points $S \cup \wt S_C$, we have $S$ as a random subset of $S \cup \wt S_C$. Hence, using Hoeffding's inequality for sampling without replacement (\lemref{lem:hoeffding_sampling}), we have with probability at least $1-\delta$
\begin{align}
    \error_{\wt \calS_C} (\wh f)- \error_{\calS \cup \wt \calS_C}( \wh f) \le  \sqrt{\frac{\log(1/\delta)}{2m_2}} \,.
\end{align}
Re-writing $\error_{\calS \cup \wt \calS_C}( \wh f)$ as $\frac{m_2}{m_2 + n} \error_{\wt \calS_C }(\wh f) + \frac{n}{m_2 + n} \error_{\calS }(\wh f)$, we have with probability at least $1-\delta$
\begin{align}
   \left(\frac{n}{n+m_2}\right) \left(\error_{\wt \calS_C} (\wh f)- \error_{\calS}( \wh f) \right) \le  \sqrt{\frac{\log(1/\delta)}{2m_2}} \,.
\end{align}
As before, assuming $2m_2 \approx m$, we have with probability at least $1-\delta$ 
\begin{align}
    \error_{\wt \calS_C} (\wh f)- \error_{\calS}( \wh f) \le \left(1+\frac{m_2}{n}\right)  \sqrt{\frac{\log(1/\delta)}{m}} \le \left(1 + \frac{m}{2n}\right) \sqrt{\frac{\log(1/\delta)}{m}} \,. \label{eq:lemma3_final}
\end{align} 
\end{proof}

\begin{proof}[Proof of \thmref{thm:error_ERM}] 
    Having established these core intermediate results, we can now combine above three lemmas to prove the main result. 
    In particular, we bound the population error on clean data ($\error_\calD(\wh f)$) as follows:  
    \begin{enumerate}[(i)]
        \item First, use \eqref{eq:lemma1_final}, to obtain an upper bound on the population error on clean data, i.e., with probability at least $1-\delta/4$, we have
        \begin{align}
            \error_{ \calD} (\wh f) \le 1 - \error_{ \wt \calS_M}(\wh f) + \sqrt{\frac{\log(4/\delta)}{m}} \,. 
        \end{align}
        \item  Second, use \eqref{eq:lemma2_final}, to relate the error on the mislabeled fraction with error on clean portion of randomly labeled data and error on whole randomly labeled dataset, i.e., with probability at least $1-\delta/2$, we have 
        \begin{align}
            - \error_{\wt S_M}(f) \le \error_{\wt S_C}(f) - 2\error_{\wt S}  + 2\error_{\wt S} \sqrt{\frac{\log(4/\delta)}{2m}}  \,. 
        \end{align} 
        \item Finally, use \eqref{eq:lemma3_final} to relate the error on the clean portion of randomly labeled data and error on clean training data, i.e., with probability $1-\delta/4$, we have 
        \begin{align}
            \error_{\wt \calS_C} (\wh f)\le - \error_{\calS}( \wh f) + \left(1 + \frac{m}{2n} \right) \sqrt{\frac{\log(4/\delta)}{m}} \,. 
        \end{align} 
    \end{enumerate}

    Using union bound on the above three steps, we have with probability at least $1-\delta$: 
    \begin{align}
        \error_\calD (\wh f) \le \error_{\calS}(\wh f)   + 1 - 2\error_{\wt \calS}(\wh f)   + \left(\sqrt{2} \error_{\wt S} + 2 + \frac{m}{2n}\right)  \sqrt{\frac{\log(4/\delta)}{m}} \,.
    \end{align}
    % Note that $(1/\sqrt{2} + 2.5)$ is a loose constant. In experiments, we use the ratio $\frac{m}{n}$
    %  the exact error $\error_{\wt \calS}(\wh f)$ 
    % to evaluate R.H.S.    
\end{proof}

\subsection{Proof of \propref{prop:rademacher}}

\begin{proof}[Proof of \propref{prop:rademacher}]
    For a classifier $ f: \calX \to \{-1, 1\}$, we have $1 - 2\,\indict{ f(x) \ne y} = y \cdot f(x)$. Hence, by definition of $\error$, we have 
    \begin{align}
        1 -2\error_{\wt \calS}(f) = \frac{1}{m}\sum_{i=1}^m y_i \cdot f(x_i) \le \sup_{f \in \calF} \, \frac{1}{m} \sum_{i=1}^m y_i \cdot f(x_i)  \,. \label{eq:error_rademacher}
    \end{align}
    Note that for fixed inputs $(x_1, x_2, \ldots, x_m)$ in $\wt S$, $(y_1, y_2, \ldots y_m)$ are random labels. Define $\phi_1 (y_1, y_2, \ldots, y_m) \defeq \sup_{f \in \calF} \, \frac{1}{m} \sum_{i=1}^m y_i \cdot f(x_i)$. We have the following bounded difference condition on $\phi_1$. For all i, 
    \begin{align}
        \sup_{y_1, \ldots y_m, y_i^\prime \in \{-1, 1\}^{m+1} } \abs{ \phi_1 (y_1,\ldots, y_i, \ldots, y_m) - \phi_1 (y_1,\ldots, y_i^\prime, \ldots, y_m)  } \le 1/m \,. \label{cond1_rademacher}
    \end{align} 
    
    Similarly, we define $\phi_2 (x_1, x_2, \ldots, x_m) \defeq \Expt{ y_i \sim_U \{-1, 1\}  }{ \sup_{f \in \calF} \, \frac{1}{m}  \sum_{i=1}^m y_i \cdot f(x_i)}$. We have the following bounded difference condition on $\phi_2$. 
    For all i,
    \begin{align}
        \sup_{x_1, \ldots x_m, x_i^\prime \in \calX^{m+1} } \abs{ \phi_2 (x_1,\ldots, x_i, \ldots, x_m) - \phi_1 (x_1,\ldots, x_i^\prime, \ldots, x_m)  } \le 1/m \,. \label{cond2_rademacher}
    \end{align}
    Using McDiarmid’s inequality (\lemref{lem:McDiarmid}) twice 
    with Condition \eqref{cond1_rademacher} and \eqref{cond2_rademacher}, 
    with probability at least $1-\delta$, we have
    \begin{align}
        \sup_{f \in \calF} \, \frac{1}{m} \sum_{i=1}^m y_i \cdot f(x_i)  - \Expt{x,y}{\sup_{f \in \calF} \, \frac{1}{m} \sum_{i=1}^m y_i \cdot f(x_i) } \le \sqrt{\frac{2\log(2/\delta)}{m}} \,. \label{eq:final_rademacher}
    \end{align} 
    Combining \eqref{eq:error_rademacher} and \eqref{eq:final_rademacher}, we obtain the desired result. 
\end{proof}


\subsection{Proof of \thmref{thm:error_regularized_ERM}}

Proof of \thmref{thm:error_regularized_ERM} follows similar to the proof of \thmref{thm:error_ERM}. Note that the same results in \lemref{lem:fit_mislabeled}, \lemref{lem:mislabeled_error}, and \lemref{lem:clear_error} hold in the regularized ERM case. However, the arguments in the proof of \lemref{lem:fit_mislabeled} change slightly. Hence, we state the lemma for regularized ERM and prove it here for completeness. 

\begin{lemma} \label{lem:lemma1_reg}
    Assume the same setup as \thmref{thm:error_regularized_ERM}. 
    Then for any $\delta >0$, with probability at least  $1-\delta$ 
    over the random draws of mislabeled data $\wt S_M$, we have 
    \begin{align}
        \error_\calD(\widehat f)  \le 1 -\error_{\wt \calS_M}(\widehat f) + \sqrt{\frac{\log(1/\delta)}{m}}\,. 
    \end{align} 
\end{lemma}
\begin{proof}
    The main idea of the proof remains the same, i.e. regard 
    the clean portion of the data 
    ($S \cup \wt S_C$) as fixed.   
    Then, there exists a classifier $f^*$ 
    that is optimal over draws 
    of the mislabeled data $\wt S_M$. 

    
    Formally, 
    \begin{align}
    f^* \defeq \argmin_{f \in \calF} \error_{\widecheck {\calD}} (f)  + \lambda R(f) \,, \label{eq:modified_ERM_reg}
    \end{align}
    where $$\widecheck \calD = \frac{n}{m+n} \calS + \frac{m_1}{m+n} \wt \calS_C  + \frac{m_2}{m+n}\calDm \,.$$ That is, $\widecheck \calD$ a combination of 
    the \emph{empirical distribution} 
    over correctly labeled data $S \cup \wt S_C$
    % in $S\cup \wt S$ 
    and the (population) distribution 
    over mislabeled data $\calDm$.
    Recall that 
    \begin{align}
    \wh f \defeq \argmin_{f \in \calF} \error_{\calS \cup \wt S} (f) + \lambda R(f) \,. \label{eq:orig_ERM_reg}
    \end{align}
    % 
    % 
    Since, $\widehat f$ minimizes 0-1 error 
    on $S \cup \wt S$, using ERM optimality on \eqref{eq:orig_ERM},  
    we have 
    \begin{align}
        \error_{\calS \cup \wt \calS}(\widehat f) + \lambda R(\wh f) \le \error_{
            \calS \cup \wt \calS}(f^*) + \lambda R(f^*) \,.    \label{eq:step1_reg}
    \end{align}
    Moreover, since $f^*$ is independent of $\wt S_M$, using Hoeffding's bound,
    % \footnote{For a fully rigorous argument,
    % refer to the complete proof in App.~\ref{app:proof_erm}.} 
    we have with probability at least $1-\delta$ that
    \begin{align}
      \error_{\wt \calS_M}(f^*) \le \error_{ \calDm}(f^*) +  \sqrt{\frac{\log(1/\delta)}{2 m_1}} \,. \label{eq:step2_reg} 
    \end{align}
    %$ 
    %for some constant $c_1\le 1/2$. 
    Finally, since $f^*$ is the optimal classifier on $\widecheck \calD$, 
    we have 
    \begin{align}
        \error_{\widecheck \calD}(f^*) + \lambda R(f^*) \le \error_{\widecheck \calD}(\widehat f) + \lambda R(\wh f) \,. \label{eq:step3_reg}
    \end{align}
     Now to relate \eqref{eq:step1_reg} and \eqref{eq:step3_reg}, we can re-write the \eqref{eq:step2_reg} as follows: 
    \begin{align}
        \error_{\calS \cup \wt\calS}(f^*) \le \error_{ \widecheck \calD}(f^*) +  \frac{m_1}{m+n}\sqrt{\frac{\log(1/\delta)}{2 m_1}} \,. \label{eq:step4_reg} 
    \end{align}
    After adding $\lambda R(f^*)$ on both sides in \eqref{eq:step4_reg}, we combine equations \eqref{eq:step1_reg}, \eqref{eq:step4_reg}, and \eqref{eq:step3_reg}, to get 
    \begin{align}
        \error_{\calS \cup \wt \calS}(\wh f) \le \error_{\widecheck \calD}(\wh f) +  \frac{m_1}{m+n}\sqrt{\frac{\log(1/\delta)}{2 m_1}} \,, 
    \end{align}
    which implies 
    \begin{align}
        \error_{ \wt \calS_M}(\wh f) \le \error_{\calDm}(\wh f) + \sqrt{\frac{\log(1/\delta)}{2 m_1}} \,. \label{eq:lemma_reg_final}
    \end{align}
    Similar as before, since $\wt S$ is obtained by randomly labeling an unlabeled dataset, we assume 
    $2m_1 \approx m$. Moreover, using $\error_{\calDm} = 1 - \error_{\calD}$ we obtain the desired result. 
\end{proof}
% \begin{proof}[Proof of ]
    
% \end{proof}

\subsection{Proof of \thmref{thm:multiclass_ERM}}

To prove our results in the multiclass case,
we first state and prove lemmas
parallel to those
% We first state and prove lemmas 
% parallel 
% to the three lemmas 
used in the proof of balanced binary case. 
We then combine these results 
% in the three lemmas 
to obtain the result in \thmref{thm:multiclass_ERM}. 

Before stating the result, 
we define mislabeled distribution $\calDm$ for any $\calD$.
While $\calDm$ and $\calD$ share 
the same marginal distribution over inputs $\calX$,
the conditional distribution over labels $y$ 
given an input $x\sim \calD_\calX$ is changed as follows:
For any $x$, the Probability Mass Function (PMF) over $y$ is defined as:  
$p_{\calDm} (\cdot \vert x) \defeq \frac{1 - p_{\calD}(\cdot \vert x)}{k - 1}$, where $ p_{\calD}(\cdot \vert x)$ is the PMF over $y$ for the distribution $\calD$. 

\begin{lemma} \label{lem:fit_mislabeled_multi}
    Assume the same setup as \thmref{thm:multiclass_ERM}. 
    Then for any $\delta >0$, with probability at least  $1-\delta$ 
    over the random draws of mislabeled data $\wt S_M$, we have 
    \begin{align}
        \error_\calD(\widehat f)  \le (k-1)\left(1 -\error_{\wt \calS_M}(\widehat f)\right) + (k-1)\sqrt{\frac{\log(1/\delta)}{m}}\,. \label{eq:lemma1_multi}
    \end{align}   
\end{lemma} 

\begin{proof}
   
    The main idea of the proof remains the same.
    We begin by regarding the clean portion of the data 
    ($S \cup \wt S_C$) as fixed. 
    Then, there exists a classifier $f^*$ 
    that is optimal over draws 
    of the mislabeled data $\wt S_M$. 
    
    However, in the multiclass case,
    we cannot as easily relate the population error on mislabeled data 
    to the population accuracy on clean data.   
    While for binary classification, 
    % we could upper bound $\error_{\wt \calS_M}$ 
    % with $1-\error_\calD$ 
    we could lower bound the population accuracy $1-\error_\calD$
    with the empirical error on mislabeled data $\error_{\wt \calS_M}$ 
    (in the proof of \lemref{lem:fit_mislabeled}), 
    for multiclass classification, 
    error on the mislabeled data 
    and accuracy on the clean data 
    in the population 
    are not so directly related.  
    To establish \eqref{eq:lemma1_multi},
    we break the error on the 
    (unknown) mislabeled data 
    into two parts: one term corresponds 
    to predicting the true label on mislabeled data, 
    and the other corresponds to predicting 
    neither the true label 
    nor the assigned (mis-)label.  
    Finally, we relate these errors to their
    population counterparts to establish \eqref{eq:lemma1_multi}. 
    
    Formally, 
    \begin{align}
    f^* \defeq \argmin_{f \in \calF} \error_{\widecheck {\calD}} (f)  + \lambda R(f) \,, \label{eq:modified_ERM_reg2}
    \end{align}
    where $$\widecheck \calD = \frac{n}{m+n} \calS + \frac{m_1}{m+n} \wt \calS_C  + \frac{m_2}{m+n}\calDm \,.$$ 
    That is, $\widecheck \calD$ is a combination 
    of the \emph{empirical distribution} 
    over correctly labeled data $S \cup \wt S_C$
    % in $S\cup \wt S$ 
    and the (population) distribution 
    over mislabeled data $\calDm$.
    Recall that 
    \begin{align}
    \wh f \defeq \argmin_{f \in \calF} \error_{\calS \cup \wt S} (f) + \lambda R(f) \,. \label{eq:orig_ERM_reg2}
    \end{align}
    % 
    % 
    Following the exact steps from the proof of \lemref{lem:lemma1_reg}, 
    with probability at least $1-\delta$, we have  
    \begin{align}
        \error_{ \wt \calS_M}(\wh f) \le \error_{\calDm}(\wh f) + \sqrt{\frac{\log(1/\delta)}{2 m_1}} \,. \label{eq:lemma1_final_multi_prev}
    \end{align}
    Similar to before, since $\wt S$ is obtained 
    by randomly labeling an unlabeled dataset, 
    we assume 
    $\frac{k}{k-1} m_1 \approx m$. 
    
    Now we will relate $\error_{\calDm} (\wh f)$ with $\error_{\calD}(\wh f)$. 
    Let $y^T$ denote the (unknown) true label 
    for a mislabeled point $(x, y)$ 
    (i.e., label before replacing it with a mislabel). 
    \begin{align*}    
         \Expt{(x, y) \in \sim \calDm}{\indict{ \wh f(x) \ne y }}  &= \underbrace{\Expt{(x, y) \in \sim \calDm}{\indict{ \wh f(x) \ne y \land \wh f(x) \ne y^T}}}_{\RN{1}} \\ &\qquad \qquad + \underbrace{\Expt{(x, y) \in \sim \calDm}{\indict{ \wh f(x) \ne y \land \wh f(x) = y^T}}}_{\RN{2}} \,. \numberthis \label{eq:excess_term}
    \end{align*}
    Clearly, term 2 is one minus the accuracy 
    on the clean unseen data, i.e.,
    \begin{align}
        \RN{2} = 1 - \Expt{{x,y} \sim \calD}{ \indict{ \wh f(x) \ne y}} = 1- \error_{\calD}(\wh f) \,. \label{eq:term1}    
    \end{align}
    Next, we relate term 1 with the error on the unseen clean data. 
    We show that term 1 is equal to the error on the unseen clean data 
    scaled by $\frac{k-2}{k-1}$,
    where $k$ is the number of labels.
    Using the definition of mislabeled distribution $\calDm$,  
    we have 
    \begin{align}
        \RN{1} = \frac{1}{k-1} \left( \Expt{(x, y) \in \sim \calD}{ \sum_{i \in \calY \land i\ne y}  \indict{ \wh f(x) \ne i \land \wh f(x) \ne y}} \right) = \frac{k-2}{k-1} \error_{\calD}(\wh f) \,.\label{eq:term2}
    \end{align}    

    Combining the result in \eqref{eq:term1}, \eqref{eq:term2} and \eqref{eq:excess_term}, we have 
    \begin{align}
        \error_{\calDm}(\wh f) = 1- \frac{1}{k-1} \error_{\calD}(\wh f) \,.\label{eq:combine_terms}
    \end{align}
    Finally, combining the result in \eqref{eq:combine_terms} 
    with equation \eqref{eq:lemma1_final_multi_prev}, 
    we have with probability $1-\delta$, 
    \begin{align}
      \error_{\calD}(\wh f) \le  (k-1) \left( 1- \error_{ \wt \calS_M}(\wh f) \right)  + (k-1) \sqrt{\frac{k \log(1/\delta)}{ 2(k-1)m}} \,. \label{eq:lemma1_final_multi}
    \end{align}
\end{proof}

\begin{lemma} \label{lem:mislabeled_error_multi}
    Assume the same setup as \thmref{thm:multiclass_ERM}. 
    Then for any $\delta >0$, 
    with probability at least $1-\delta$ 
    over the random draws of $\wt S$, we have  
    % \begin{align}
        $$\abs{k\error_{\wt \calS}(\widehat f) - \error_{\wt \calS_C}(\widehat f) -  (k-1)\error_{\wt \calS_M}(\widehat f) } \le  2k\sqrt{\frac{\log(4/\delta)}{2m}}\,. $$ % \label{eq:lemma2}
    % \end{align}   
    %  for some constant $c_3 \le 1.0\,$.
\end{lemma} 


\begin{proof}
    Recall $\error_{\wt S} (f) = \frac{m_1}{m} \error_{\wt S_M}(f) + \frac{m_2}{m} \error_{\wt S_C}(f)$. Hence, we have 
    \begin{align*}
        k\error_{\wt S}(f) - (k-1)\error_{\wt S_M}(f) - \error_{\wt S_C}(f) &= (k-1)\left(\frac{k m_1}{(k-1) m} \error_{\wt S_M}(f) - \error_{\wt S_M}(f)\right) \\ & \qquad \qquad + \left(\frac{km_2}{m} \error_{\wt S_C}(f) - \error_{\wt S_C}(f)\right) \\ &= k \left[ \left(\frac{m_1}{m} - \frac{k-1}{k}\right) \error_{\wt S_M}(f) + \left(\frac{m_2}{m} - \frac{1}{k} \right) \error_{\wt S_C} (f) \right] \,.
    \end{align*} 
    Since the dataset is randomly labeled, 
    we have with probability at least $1-\delta$, 
    $\left(\frac{m_1}{m} - \frac{k-1}{k}\right) \le \sqrt{\frac{\log(1/\delta)}{2m}}$. 
    Similarly, we have with probability at least $1-\delta$, 
    $\left(\frac{m_2}{m} - \frac{1}{k}\right) \le \sqrt{\frac{\log(1/\delta)}{2m}}$. 
    Using union bound, we have with probability at least $1-\delta$
    % \begin{align}
    %     2\error_{\wt S} - \error_{\wt S_M}(f) - \error_{\wt S_C}(f) \le \sqrt{\frac{\log(2/\delta)}{2m}} \left(\error_{\wt S_M}(f) + \error_{\wt S_C}(f) \right) \le 2\sqrt{\frac{\log(2/\delta)}{2m}} \,. \label{eq:lemma2_final}
    % \end{align}
    \begin{align}
        k\error_{\wt S}(f) - (k-1)\error_{\wt S_M}(f) - \error_{\wt S_C}(f)  \le k \sqrt{\frac{\log(2/\delta)}{2m}} \left(\error_{\wt S_M}(f) + \error_{\wt S_C}(f) \right) \,. \label{eq:lemma2_final_multi}
    \end{align}

    % We obtain the desired result by using 
\end{proof}

\begin{lemma} \label{lem:clear_error_multi}
    Assume the same setup as \thmref{thm:multiclass_ERM}. 
    Then for any $\delta >0$, with probability at least $1-\delta$ 
    over the random draws of $\wt S_C$ and $S$, we have 
    % \begin{align}
        $$\abs{\error_{\wt \calS_C}(\widehat f) - \error_{\calS}(\widehat f) } \le 1.5 \sqrt{\frac{k\log(2/\delta)}{2m}}\,.$$ %\label{eq:lemma3}
    % \end{align}   
    % for some constant $c_2 \le 1.2\,$.
\end{lemma} 
\begin{proof}
    % Recall 0-1 error on each point  $(x,y) \in S \cup \wt S$ is given by $\I{ f(x)\ne y}$.
    In the set of correctly labeled points $S \cup \wt S_C$,
    we have $S$ as a random subset of $S \cup \wt S_C$. 
    Hence, using Hoeffding's inequality 
    for sampling without replacement 
    (\lemref{lem:hoeffding_sampling}), 
    we have with probability at least $1-\delta$
    \begin{align}
        \error_{\wt \calS_c} (\wh f)- \error_{\calS \cup \wt \calS_C}( \wh f) \le  \sqrt{\frac{\log(1/\delta)}{2m_2}} \,.
    \end{align}
    Re-writing $\error_{\calS \cup \wt \calS_C}( \wh f)$ 
    as $\frac{m_2}{m_2 + n} \error_{\wt \calS_C }(\wh f) + \frac{n}{m_2 + n} \error_{\calS }(\wh f)$, 
    we have with probability at least $1-\delta$
    \begin{align}
       \left(\frac{n}{n+m_2}\right) \left(\error_{\wt \calS_c} (\wh f)- \error_{\calS}( \wh f) \right) \le  \sqrt{\frac{\log(1/\delta)}{2m_2}} \,.
    \end{align}
    As before, assuming $km_2 \approx m$, 
    we have with probability at least $1-\delta$ 
    \begin{align}
        \error_{\wt \calS_c} (\wh f)- \error_{\calS}( \wh f) \le \left(1+\frac{m_2}{n}\right)  \sqrt{\frac{k\log(1/\delta)}{2m}} \le \left( 1 + \frac{1}{k}\right) \sqrt{\frac{k\log(1/\delta)}{2m}} \,. \label{eq:lemma3_final_multi}
    \end{align} 
\end{proof}

\begin{proof}[Proof of \thmref{thm:multiclass_ERM}] 
    Having established these core intermediate results, 
    we can now combine above three lemmas. 
    In particular, we bound the population error 
    on clean data ($\error_\calD(\wh f)$) as follows:  
    \begin{enumerate}[(i)]
        \item First, use \eqref{eq:lemma1_final_multi}, 
        to obtain an upper bound on the population error on clean data, 
        i.e., with probability at least $1-\delta/4$, we have
        \begin{align}
            \error_{ \calD} (\wh f) \le (k-1)\left(1 - \error_{ \wt \calS_M}(\wh f) \right) + (k-1) \sqrt{\frac{k\log(4/\delta)}{2(k-1)m}} \,. 
        \end{align}
        \item  Second, use \eqref{eq:lemma2_final_multi}
        to relate the error on the mislabeled fraction 
        with error on clean portion of randomly labeled data 
        and error on whole randomly labeled dataset, 
        i.e., with probability at least $1-\delta/2$, we have 
        \begin{align}
            - (k-1)\error_{\wt S_M}(f) \le \error_{\wt S_C}(f) - k\error_{\wt S}  + k\sqrt{\frac{\log(4/\delta)}{2m}}  \,. 
        \end{align} 
        \item Finally, use \eqref{eq:lemma3_final_multi} 
        to relate the error on the clean portion of randomly labeled data 
        and error on clean training data, 
        i.e., with probability $1-\delta/4$, we have 
        \begin{align}
            \error_{\wt \calS_C} (\wh f)\le - \error_{\calS}( \wh f) + \left(1 + \frac{m}{kn} \right) \sqrt{\frac{k\log(4/\delta)}{2m}} \,. 
        \end{align} 
    \end{enumerate}

    Using union bound on the above three steps, 
    we have with probability at least $1-\delta$: 
    \begin{align}
        \error_\calD (\wh f) \le \error_{\calS}(\wh f) + (k-1) - k\error_{\wt \calS}(\wh f)   + (\sqrt{k(k-1)} + k + \sqrt{k} + \frac{m}{n\sqrt{k}})  \sqrt{\frac{\log(4/\delta)}{2m}} \,.\label{eq:multiclass_ERM_final}
    \end{align}
    Simplifying the term in RHS of \eqref{eq:multiclass_ERM_final}, 
    we get the desired result. 
    % Note that since $\frac{m}{n\sqrt{k}}$ 
    % is much smaller than the sum of the other terms
    % the other terms in summation, 
    % we ignore $\frac{m}{n\sqrt{k}}$  
    % Z: ??? --- great
    % that 
    % them
    in the final bound. 
    % we ignore that in the final bound. 
    % Note that $(1/\sqrt{2} + 2.5)$ is a loose constant. In experiments, we use the ratio $\frac{m}{n}$
    %  the exact error $\error_{\wt \calS}(\wh f)$ 
    % to evaluate R.H.S.    
\end{proof}

\newpage
\section{Proofs from \secref{sec:linear_models}}\label{app:proof_gd}
We suppose that the parameters of the linear function 
are obtained via gradient descent on 
the following $L_2$ regularized problem: 
\begin{align}
    % n in denominator is avoided deliberately
    \calL_S(w; \lambda) \defeq \sum_{i=1}^n{(w^Tx_i - y_i)^2} + \lambda \norm{w}{2}^2 \,, \label{eq:l2_MSE_app}   
\end{align}
where $\lambda\ge0$ is a regularization parameter. 
We assume access to a clean dataset 
$S = \{(x_i, y_i)\}_{i=1}^n \sim \calD^n$ 
and randomly labeled dataset 
$\wt S = \{(x_i, y_i)\}_{i=n+1}^{n+m} \sim \wt \calD^m$. 
Let $\bX = [x_1, x_2, \cdots, x_{m+n}]$ 
and $\by = [y_1, y_2, \cdots, y_{m+n}]$. 
Fix a positive learning rate $\eta$ such that 
$\eta \le 1/\left(\norm{\bX^T\bX}{\text{op}} + \lambda^2\right)$ 
and an initialization $w_0 = 0$. 
% \todos{Assumption made for simplicty}. 
Consider the following gradient descent iterates 
to minimize objective \eqref{eq:l2_MSE_app} on $S \cup \wt S$:
\begin{align}
w_t = w_{t-1} - \eta \grad_w \calL_{S \cup \wt S} (w_{t-1}; \lambda) \quad \forall t=1,2,\ldots \label{eq:GD_iterates_app}
\end{align} 
Then we have $\{ w_t\}$ converge to the limiting solution 
$\wh w = \left( \bX^T\bX+\lambda \boldsymbol{I}\right)^{-1}\bX^T\by$. Define $\widehat f (x) \defeq f(x ; \wh w) $.  

% \subsection{\textcolor{red}{Errata}}

% We wish to correct the following error in the body:
% \codref{cond:error_stability} is not enough 
% to guarantee the result in \thmref{thm:linear}. 
% We now present a slightly stronger condition 
% called \emph{hypothesis stability} 
% under which we obtain a result 
% similar to \thmref{thm:linear}. 

% This error doesn't change the main arguments of the proof,
% where we show that the empirical train error 
% is less than or equal to the leave-one-out error.
% We need a stronger condition to relate leave-one-out error 
% with the population error of the original classifier. 
% Specifically, while \codref{cond:error_stability} 
% relates the average population error of leave-one-out classifiers 
% with the population error of the original classifier, 
% we need the new condition to show the concentration 
% of the empirical leave-one-out error 
% and average population error of leave-one-out classifiers. 
% main takeaway 

% Note that the new condition, 
% while being stronger than the previous one, 
% still doesn't imply generalization \citep{bousquet2002stability,elisseeff2003leave,abou2019exponential}. 
% Overall, the main results in \secref{sec:ERM_training} 
% and takeaways of the paper remain unaffected by the error.  

% We now present the new condition 
% and a corrected statement of \thmref{thm:linear}. 
% Recall, for a given training set $S \sim \calD^n $, 
% we use $S_{(i)}$ to denote the training set $S$ 
% with the $i^{\text{th}}$ point removed.

% \begin{condition}[Hypothesis Stability] 
%     \label{cond:hypothesis_stability}
%     We have $\beta$ hypothesis stability 
%     if our training algorithm $\calA$ satisfies the following: 
%     \begin{align*}
%     % ${\sum_{i=1}^n \frac{\error_{\calD}( f(\calA, S_{(i)}))}{n} - \error_\calD(f(\calA, S))} \le \beta\,$.
%     \forall i \in \{1,2,\ldots, n\}, \quad  \Expt{\calS, (x,y) \in \calD}{ \abs{\error\left( f(x) ,y  \right) - \error\left( f_{(i)}(x), y \right) }} \le \frac{\beta}{n} \,,
%     \end{align*}
%     where $f_{(i)} \defeq f(\calA, S_{(i)})$ and $ f \defeq f(\calA, S)$.
% \end{condition}

% \begin{theorem}[Correct statement of \thmref{thm:linear}] \label{thm:new_linear}
%     Assume that this gradient descent algorithm satisfies \codref{cond:hypothesis_stability}
%     with $\beta=\calO(1)$.  
%     Then for any $\delta >0$, with probability at least $1-\delta$ 
%     over the random draws of datasets $\wt S$ and $S$, we have:
%     \begin{align}
%         \error_\calD(\widehat f) \le \error_\calS(\widehat f) + 1 - 2 \error_{\wt\calS}(\widehat f) + \left(\frac{1}{\sqrt{2}} + 1.5 \right) \sqrt{\frac{\log(4/\delta)}{m}} + \sqrt{\frac{4}{\delta}\left(\frac{1}{m} +\frac{3\beta}{m+n} \right)}  \,. \label{eq:gd_error}
%     \end{align} 
%     % for some constant $c\le 3.2$.
% \end{theorem}

\subsection{Proof of \thmref{thm:linear}}
We use a standard result from linear algebra, 
namely the Shermann-Morrison formula 
\citep{sherman1950adjustment} for matrix inversion:  

\begin{lemma}[\citet{sherman1950adjustment}] \label{lem:sherman}
    Suppose $\bA \in \Real^{n \times n}$ 
    is an invertible square matrix 
    and $u,v \in \Real^n$ are column vectors. 
    Then $\bA + uv^T$ is invertible iff $1 + v^T \bA u \ne 0$ 
    and in particular
    \begin{align}
        (\bA + u v^T)^{-1} = \bA^{-1}  - \frac{\bA^{-1} uv^T \bA^{-1} }{ 1 + v^T \bA^{-1} u} \,.
    \end{align}   
\end{lemma}
\newcommand\byy[1]{\by_{\left(#1\right)}}
\newcommand\bXX[1]{\bX_{\left(#1\right)}}
\newcommand\ff[1]{\wh f_{\left(#1\right)}}

For a given training set $S \cup \wt S_C$, 
define leave-one-out error 
on mislabeled points in the training data 
as $$\error_{\text{LOO}(\wt S_M) } = \frac{\sum_{(x_i, y_i) \in \wt S_M} \error( f_{(i)}( x_i), y_i)}{ \abs{\wt S_M }} \,, $$
where $f_{(i)} \defeq f(\calA, (S \cup \wt S)_{(i)})$. 
To relate empirical leave-one-out error and population error 
with hypothesis stability condition, 
we use the following lemma:   

\begin{lemma}[\citet{bousquet2002stability}] \label{lem:stability_error}
    For the leave-one-out error, we have
    \begin{align}
        \Expo{ \left( \error_{\calDm}(\wh f) -\error_{\text{LOO}(\wt S_M) } \right)^2 } \le \frac{1}{2m_1}+  \frac{3\beta}{n + m}\,.
    \end{align}   
    % where $ f \defeq f(\calA, S \cup \wt S) $.
\end{lemma}

Proof of the above lemma is similar 
to the proof of Lemma 9 in \citet{bousquet2002stability} 
and can be found in \appref{app:proof_lem_error}. 
% 
% Before presenting the result, we introduce some notation. 
Before presenting the proof of \thmref{thm:linear}, 
we introduce some more notation. 
Let $\bX_{(i)}$ denote the matrix of covariates 
with the $i^{\text{th}}$ point removed. 
Similarly, let $\by_{(i)}$ be the array of responses 
with the $i^{\text{th}}$ point removed. 
Define the corresponding regularized GD solution 
as $\wh w_{(i)} = \left( \bXX{i}^T\bXX{i}+\lambda \boldsymbol{I}\right)^{-1}\bXX{i}^T\byy{i}$. 
Define $\ff{i}(x) \defeq f(x ; \wh w_{(i)}) $.

\begin{proof}[Proof of \thmref{thm:linear}]
    Because squared loss minimization does not imply 0-1 error minimization, 
    we cannot use arguments from \lemref{lem:fit_mislabeled}. 
    This is the main technical difficulty. 
    To compare the 0-1 error at a train point with an unseen point, 
    we use the closed-form expression for $\widehat{w}$ 
    and Shermann-Morrison formula 
    to upper bound training error 
    with leave-one-out cross validation error. 
    
    The proof is divided into three parts: 
    In part one, we show that 0-1 error 
    on mislabeled points in the training set 
    is lower than the error obtained 
    by leave-one-out error at those points. 
    In part two, we relate this leave-one-out error 
    with the population error on mislabeled distribution
    using \codref{cond:hypothesis_stability}.
    While the empirical leave-one-out error is an unbiased estimator 
    of the average population error of leave-one-out classifiers, 
    we need hypothesis stability 
    to control the variance 
    of empirical leave-one-out error. 
    Finally, in part three, we show 
    that the error on the mislabeled training points 
    can be estimated with just the randomly labeled 
    and clean training data (as in proof of \thmref{thm:error_ERM}).  

    \textbf{Part 1 {} {}} First we relate training error with leave-one-out error.        
    For any training point $(x_i, y_i)$ in $\wt S \cup S$, we have 
    \begin{align}
        \error(\wh f(x_i), y_i ) &= \indict{ y_i \cdot x_i^T \wh w < 0 } = \indict{ y_i \cdot x_i^T \left( \bX^T\bX+\lambda \boldsymbol{I}\right)^{-1}\bX^T\by < 0 } \\
        &= \indict{ y_i \cdot x_i^T \underbrace{\left( \bXX{i}^T\bXX{i} + x_i ^T x_i +\lambda \boldsymbol{I}\right)^{-1}}_{\RN{1}} (\bXX{i}^T\byy{i} + y_i \cdot x_i) < 0 } \,.
    \end{align}
    Letting $\bA = \left(\bXX{i}^T\bXX{i} +\lambda \boldsymbol{I}\right)$ 
    and using \lemref{lem:sherman} on term 1, we have 
    \begin{align}
        \error(\wh f(x_i), y_i ) &= \indict{ y_i \cdot x_i^T \left[\bA^{-1} -  \frac{\bA^{-1} x_i x_i^T \bA^{-1}}{ 1 + x_i ^T \bA^{-1} x_i } \right] (\bXX{i}^T\byy{i} + y_i \cdot x_i) < 0 } \\
        &= \indict{ y_i \cdot\left[ \frac{ x_i^T \bA^{-1} ( 1 + x_i ^T \bA^{-1} x_i ) -  x_i^T \bA^{-1} x_i x_i^T \bA^{-1}}{ 1 + x_i ^T \bA ^{-1}x_i } \right] (\bXX{i}^T\byy{i} + y_i \cdot x_i) < 0 } \\
        &= \indict{ y_i \cdot\left[ \frac{ x_i^T \bA^{-1}}{ 1 + x_i ^T \bA ^{-1}x_i } \right] (\bXX{i}^T\byy{i} + y_i \cdot x_i) < 0 } \,.
    \end{align}

    Since $1 + x_i^T \bA^{-1} x_i > 0$, we have 
    \begin{align}
        \error(\wh f(x_i), y_i ) &= \indict{ y_i \cdot x_i^T \bA^{-1} (\bXX{i}^T\byy{i} + y_i \cdot x_i) < 0 } \\
        &= \indict{ x_i^T \bA^{-1} x_i +  y_i \cdot x_i^T \bA^{-1} (\bXX{i}^T\byy{i}) < 0 } \\
        &\le \indict{ y_i \cdot x_i^T \bA^{-1} (\bXX{i}^T\byy{i}) < 0 } = \error(\ff{i}(x_i), y_i ) \,.\label{eq:LOO_error}
    \end{align}

    Using \eqref{eq:LOO_error}, we have 
    \begin{align}
        \error_{\wt \calS_M } (\wh f) \le \error_{\text{LOO} (\wt S_M)} \defeq \frac{\sum_{(x_i, y_i) \in \wt S_M} \error(\ff{i}(x_i), y_i ) }{\abs{\wt \calS_M}}\label{eq:LOO_error_final} \,.
    \end{align}
    \textbf{Part 2 {}{}} We now relate RHS in \eqref{eq:LOO_error_final} 
    with the population error on mislabeled distribution. 
    To do this, we leverage \codref{cond:hypothesis_stability} 
    and \lemref{lem:stability_error}. 
    In particular, we have 

    \begin{align}
        \Expt{\calS \cup \wt \calS_M }{ \left(\error_{\calDm}(\wh f) - \error_{\text{LOO} (\wt S_M)}\right)^2 } \le \frac{1}{2m_1} + \frac{3\beta}{m+n} \,.
    \end{align}

    Using Chebyshev's inequality, with probability at least $1-\delta$, we have 
    \begin{align}
        \error_{\text{LOO} (\wt S_M)} \le  \error_{\calDm}(\wh f)   + \sqrt{\frac{1}{\delta}\left(\frac{1}{2m_1} +\frac{3\beta}{m+n} \right)} \,. \label{eq:final_mislabeled_linear}
    \end{align}
    

    \textbf{Part 3 {}{}} Combining \eqref{eq:final_mislabeled_linear} and \eqref{eq:LOO_error_final}, we have 

    \begin{align}
        \error_{\wt \calS_M } (\wh f) \le \error_{\calDm}(\wh f)   + \sqrt{\frac{1}{\delta}\left(\frac{1}{2m_1} +\frac{3\beta}{m+n} \right)} \,. \label{eq:linear_parallel_lem1}
    \end{align}

    Compare \eqref{eq:linear_parallel_lem1} with \eqref{eq:lemma1_final} 
    in the proof of \lemref{lem:fit_mislabeled}. 
    We obtain a similar relationship 
    between $\error_{\wt \calS_M }$ and $\error_{\calDm}$ 
    but with a polynomial concentration 
    instead of exponential concentration. 
    In addition, since we just use concentration arguments 
    to relate mislabeled error to the errors
    on the clean and unlabeled portions 
    of the randomly labeled data, 
    we can directly use the results 
    in \lemref{lem:mislabeled_error} and \lemref{lem:clear_error}. 
    Therefore, combining results in \lemref{lem:mislabeled_error}, \lemref{lem:clear_error}, and \eqref{eq:linear_parallel_lem1} with union bound, 
    we have with probability at least $1-\delta$
    \begin{align}
        \error_\calD(\widehat f) \le \error_\calS(\widehat f) + 1 - 2 \error_{\wt\calS}(\widehat f) + \left(\sqrt{2}\error_{\wt\calS}(\widehat f) + 1 + \frac{m}{2n} \right) \sqrt{\frac{\log(4/\delta)}{m}} + \sqrt{\frac{4}{\delta}\left(\frac{1}{m} +\frac{3\beta}{m+n} \right)}  \,.
    \end{align}
    

       
\end{proof}

\subsection{Extension to multiclass classification} \label{app:multiclass_linear}
For multiclass problems with squared loss minimization, as standard practice, we consider one-hot encoding for the underlying label, i.e., a class label $c \in [k]$ is treated as $(0, \cdot, 0,1,0, \cdot, 0) \in \Real^k$ (with $c$-th coordinate being 1).  As before, we suppose that the parameters of the linear function 
are obtained via gradient descent on the following $L_2$ regularized problem: 
\begin{align}
    % n in denominator is avoided deliberately
    \calL_S(w; \lambda) \defeq \sum_{i=1}^n\norm{w^Tx_i - y_i}{2}^2 + \lambda \sum_{j=1}^k \norm{w_j}{2}^2 \,, \label{eq:l2_multiclass_MSE_app}   
\end{align}
where $\lambda\ge0$ is a regularization parameter. 
We assume access to a clean dataset 
$S = \{(x_i, y_i)\}_{i=1}^n \sim \calD^n$ 
and randomly labeled dataset 
$\wt S = \{(x_i, y_i)\}_{i=n+1}^{n+m} \sim \wt \calD^m$. 
Let $\bX = [x_1, x_2, \cdots, x_{m+n}]$ 
and $\by = [e_{y_1}, e_{y_2}, \cdots, e_{y_{m+n}}]$. 
Fix a positive learning rate $\eta$ such that 
$\eta \le 1/\left(\norm{\bX^T\bX}{\text{op}} + \lambda^2\right)$ 
and an initialization $w_0 = 0$. 
% \todos{Assumption made for simplicty}. 
Consider the following gradient descent iterates 
to minimize objective \eqref{eq:l2_MSE_app} on $S \cup \wt S$:
\begin{align}
{w_j}^t = {w_j}^{t-1} - \eta \grad_{w_j} \calL_{S \cup \wt S} (w^{t-1}; \lambda) \quad \forall t=1,2,\ldots \text{ and } j=1,2,\ldots,k  \,. \label{eq:GD_multi_iterates_app}
\end{align} 
Then we have $\{ {w_j}^t\}$ for all $j =1,2,\cdots, k$ converge to the limiting solution 
$\wh w_j = \left( \bX^T\bX+\lambda \boldsymbol{I}\right)^{-1}\bX^T\by_j$. Define $\widehat f (x) \defeq f(x ; \wh w) $.  

\begin{theorem}\label{thm:multi_linear}
    Assume that this gradient descent algorithm satisfies \codref{cond:hypothesis_stability}
    with $\beta=\calO(1)$.  
    Then for a multiclass classification problem wth $k$ classes, for any $\delta >0$, with probability at least $1-\delta$, we have:
    \begin{align*}
        \error_\calD(\widehat f) \le \error_\calS(\widehat f) &+ (k-1)\left(1 - \frac{k}{k-1} \error_{\wt\calS}(\widehat f) \right) \\ &+ \left(k + \sqrt{k} + \frac{m}{n\sqrt{k}} \right) \sqrt{\frac{\log(4/\delta)}{2m}} + \sqrt{k(k-1)} \sqrt{\frac{4}{\delta}\left(\frac{1}{m} +\frac{3\beta}{m+n} \right)}  \,. \numberthis \label{eq:gd_multi_error}
    \end{align*} 
    % for some constant $c\le 3.2$.
\end{theorem}
\begin{proof}
    The proof of this theorem is divided into two parts. In the first part, we relate the error on the mislabeled samples with the population error on the mislabeled data. Similar to the proof of \thmref{thm:linear}, we use Shermann-Morrison formula to upper bound training error with leave-one-out error on each $\wh w^j$. Second part of the proof follows entirely from the proof of \thmref{thm:multiclass_ERM}. In essence, the first part derives an equivalent of \eqref{eq:lemma1_final_multi_prev} for GD training with squared loss and then the second part follows from the proof  of \thmref{thm:multiclass_ERM}. 
    
    \textbf{Part-1:} Consider a training point $(x_i,y_i)$ in $\wt S \cup S $. For simplicity, we use $c_i$ to denote the class of $i$-th point and use $y_i$ as the corresponding one-hot embedding. Recall error in multiclass point is given by $\error(\wh f(x_i), y_i ) = \indict{ c_i \not \in \argmax x_i^T \wh w }$. Thus, there exists a $j \ne c_i \in [k]$, such that we have
     \begin{align}
        \error(\wh f(x_i), y_i ) &= \indict{ c_i \not \in \argmax x_i^T \wh w } = \indict{ x_i^T \wh w_{c_i} < x_i^T \wh w_{j}  } \\ &= \indict{ x_i^T \left( \bX^T\bX+\lambda \boldsymbol{I}\right)^{-1}\bX^T\by_{c_i} < x_i^T \left( \bX^T\bX+\lambda \boldsymbol{I}\right)^{-1}\bX^T\by_{j} } \\
        &= \indict{ x_i^T \underbrace{\left( \bXX{i}^T\bXX{i} + x_i ^T x_i +\lambda \boldsymbol{I}\right)^{-1}}_{\RN{1}} \left(\bXX{i}^T{\by_{c_i}}_{(i)} + x_i - \bXX{i}^T{\by_{j}}_{(i)}\right) < 0 } \,.
    \end{align}
    Letting $\bA = \left(\bXX{i}^T\bXX{i} +\lambda \boldsymbol{I}\right)$ 
    and using \lemref{lem:sherman} on term 1, we have 
    \begin{align}
        \error(\wh f(x_i), y_i ) &= \indict{ x_i^T \left[\bA^{-1} -  \frac{\bA^{-1} x_i x_i^T \bA^{-1}}{ 1 + x_i ^T \bA^{-1} x_i } \right]  \left(\bXX{i}^T{\by_{c_i}}_{(i)} + x_i - \bXX{i}^T{\by_{j}}_{(i)}\right) < 0 } \\
        &= \indict{ \left[ \frac{ x_i^T \bA^{-1} ( 1 + x_i ^T \bA^{-1} x_i ) -  x_i^T \bA^{-1} x_i x_i^T \bA^{-1}}{ 1 + x_i ^T \bA ^{-1}x_i } \right]  \left(\bXX{i}^T{\by_{c_i}}_{(i)} + x_i - \bXX{i}^T{\by_{j}}_{(i)}\right) < 0 } \\
        &= \indict{ \left[ \frac{ x_i^T \bA^{-1}}{ 1 + x_i ^T \bA ^{-1}x_i } \right]  \left(\bXX{i}^T{\by_{c_i}}_{(i)} + x_i - \bXX{i}^T{\by_{j}}_{(i)}\right) < 0} \,.
    \end{align}
    Since $1 + x_i^T \bA^{-1} x_i > 0$, we have 
    \begin{align}
        \error(\wh f(x_i), y_i ) &= \indict{ x_i^T \bA^{-1}  \left(\bXX{i}^T{\by_{c_i}}_{(i)} + x_i - \bXX{i}^T{\by_{j}}_{(i)}\right) < 0 } \\
        &= \indict{ x_i^T \bA^{-1} x_i +  x_i^T \bA^{-1}  \bXX{i}^T{\by_{c_i}}_{(i)}  - x_i^T\bA^{-1}  \bXX{i}^T{\by_{j}}_{(i)} < 0 } \\
        &\le \indict{  x_i^T \bA^{-1}  \bXX{i}^T{\by_{c_i}}_{(i)}  - x_i^T\bA^{-1}  \bXX{i}^T{\by_{j}}_{(i)} < 0  } = \error(\ff{i}(x_i), y_i ) \,.\label{eq:LOO_error_multi}
    \end{align}
    Using \eqref{eq:LOO_error_multi}, we have 
    \begin{align}
        \error_{\wt \calS_M } (\wh f) \le \error_{\text{LOO} (\wt S_M)} \defeq \frac{\sum_{(x_i, y_i) \in \wt S_M} \error(\ff{i}(x_i), y_i ) }{\abs{\wt \calS_M}}\label{eq:LOO_error_multi_final} \,.
    \end{align}
    
    We now relate RHS in \eqref{eq:LOO_error_final} 
    with the population error on mislabeled distribution. 
    Similar as before, to do this, we leverage \codref{cond:hypothesis_stability} 
    and \lemref{lem:stability_error}. Using  \eqref{eq:final_mislabeled_linear} and \eqref{eq:LOO_error_multi_final}, we have 
    \begin{align}
        \error_{\wt \calS_M } (\wh f) \le \error_{\calDm}(\wh f)   + \sqrt{\frac{1}{\delta}\left(\frac{1}{2m_1} +\frac{3\beta}{m+n} \right)} \,. \label{eq:linear_multi_parallel_lem1}
    \end{align}
    
    We have now derived a parallel to \eqref{eq:lemma1_final_multi_prev}. Using the same arguments in the proof of \lemref{lem:fit_mislabeled_multi}, we have 
    \begin{align}
      \error_{\calD}(\wh f) \le  (k-1) \left( 1- \error_{ \wt \calS_M}(\wh f) \right)  + (k-1)\sqrt{\frac{k}{\delta(k-1)}\left(\frac{1}{2m_1} +\frac{3\beta}{m+n} \right)}  \,. \label{eq:lemma1_linear_final_multi}
    \end{align}
    
    \textbf{Part-2:} We now combine the results in \lemref{lem:mislabeled_error_multi} and \lemref{lem:clear_error_multi} to obtain the final inequality in terms of quantities that can be computed from just the randomly labeled and clean data. Similar to the binary case, we obtained a polynomial concentration instead of exponential concentration. Combining \eqref{eq:lemma1_linear_final_multi} with \lemref{lem:mislabeled_error_multi} and \lemref{lem:clear_error_multi}, we have with probability at least $1-\delta$
    \begin{align*}
        \error_\calD(\widehat f) \le \error_\calS(\widehat f) &+ (k-1)\left(1 - \frac{k}{k-1} \error_{\wt\calS}(\widehat f) \right) \\ &+ \left(k + \sqrt{k} + \frac{m}{n\sqrt{k}} \right) \sqrt{\frac{\log(4/\delta)}{2m}} + \sqrt{k(k-1)} \sqrt{\frac{4}{\delta}\left(\frac{1}{m} +\frac{3\beta}{m+n} \right)}  \,. \numberthis \label{eq:gd_multi_error_proof}
    \end{align*} 
\end{proof}

\subsection{Discussion on \codref{cond:hypothesis_stability}} \label{app:discuss_cond1}
The quantity in LHS of \codref{cond:hypothesis_stability} 
measures how much the function learned by the algorithm 
(in terms of error on unseen point) will change 
when one point in the training set is removed. 
% Discussion on exponential concentration and stronger condition. 
% Notice that hypothesis stability implies error stability, i.e., \codref{cond:error_stability} \citep{bousquet2002stability}.  
% In summary, while error stability allowed us 
% to relate the average population error 
% of the leave-one-out classifiers 
% with the population error of the original classifier, 
We need hypothesis stability condition 
to control the variance of the empirical leave-one-out error to show concentration of average leave-one-error with the population error. 

Additionally, we note that while the dominating term in the RHS of \thmref{thm:linear} matches with the dominating term in ERM bound in \thmref{thm:error_ERM}, there is a polynomial concentration term 
(dependence on $1/\delta$ instead of $\log(\sqrt{1/\delta})$) 
in \thmref{thm:linear}. 
Since with hypothesis stability, 
we just bound the variance, 
the polynomial concentration is due 
to the use of Chebyshev's inequality 
instead of an exponential tail inequality
(as in \lemref{lem:fit_mislabeled}).
Recent works have highlighted that 
a slightly stronger condition than hypothesis stability 
can be used to obtain an exponential concentration 
for leave-one-out error \citep{abou2019exponential},
but we leave this for future work for now. 
% We leave 
% However, the constants 

% we also want to highlight  

\subsection{Formal statement and proof of \propref{prop:early_stop}} \label{app:formal_early_stop}

Before formally presenting the result, 
we will introduce some notation.  
By $\calL_{S}(w)$, we denote 
the objective in \eqref{eq:l2_MSE_app} with $\lambda=0$. 
Assume Singular Value Decomposition (SVD) of $\bX$
as $\sqrt{n} \bU \bS^{1/2} \bV^T$. 
Hence $\bX^T \bX = \bV \bS \bV^T$.
Consider the GD iterates defined in \eqref{eq:GD_iterates_app}. 
% 
We now derive closed form expression 
for the $t^\text{th}$ iterate of gradient descent:  
% 
\begin{align}
    w_t = w_{t-1} + \eta \cdot \bX^T (\by - \bX w_{t-1}) = (\bI - \eta \bV \bS \bV^T )w_{k-1} + \eta \bX^T \by \,.
\end{align}
Rotating by $\bV^T$, we get 
\begin{align}
    \wt w_t = (\bI - \eta\bS )\wt w_{k-1} + \eta \wt \by \label{eq:GD_recur},
\end{align}
where $\wt w_t = \bV^T w_t $ and $\wt \by = \bV^T \bX^T \by$. 
Assuming the initial point $w_0 = 0$ 
and applying the recursion in \eqref{eq:GD_recur}, we get
\begin{align}
    \wt w_t = \bS ^{-1} ( \bI - (\bI - \eta \bS)^k ) \wt \by \,, 
\end{align} 
Projecting solution back to the original space, we have 
\begin{align}
     w_t = \bV \bS ^{-1} ( \bI - (\bI - \eta \bS)^k ) \bV^T \bX^T \by \,. 
\end{align} 
% We will work with this GD solution at any iterate $t$ in the next proposition. 
Define $f_t(x) \defeq f(x;w_t)$ 
as the solution at the $t^{\text{th}}$ iterate. 
Let $\wt w_{\lambda} = \argmin_{w} \calL_\calS (w;\lambda) = (\bX^T \bX + \lambda \bI)^{-1} \bX^T \by = \bV (\bS + \lambda \bI )^{-1} \bV^T \bX^T \by $. 
% ) \,,$ for all $t=1,2,\ldots\,.$ 
and define $\wt f_\lambda(x) \defeq f(x;\wt w_\lambda)$ as the regularized solution. 
Assume $\kappa$ be the condition number 
of the population covariance matrix 
and let $s_\text{min}$ be the minimum positive 
singular value of the empirical covariance matrix. 
Our proof idea is inspired from recent work 
on relating gradient flow solution 
and regularized solution 
for regression problems \citep{ali2018continuous}. 
We will use the following lemma in the proof: 
\begin{lemma} \label{lem:ineq_soln}
    For all $x \in [0,1]$ and for all $ k \in \mathbb{N}$, 
    we have (a) $ \frac{kx}{1+kx} \le 1- (1-x)^k$ 
    and (b) $ 1- (1-x)^k \le 2 \cdot \frac{kx}{kx+1} $.
    %  where $g(c)$ is a constant dependent on $c$. For $c = 1$, $g(c) = 2.0$.   
\end{lemma}
\begin{proof}
    % [Proof of \lemref{lem:ineq_soln}]
    % Part (a) is easy. 
    Using $ (1-x)^k \le \frac{1}{1+kx}$, we have part (a). 
    For part (b), we numerically maximize 
    $\frac{ (1+kx ) (1 - (1-x)^k) }{kx}$ 
    for all $k\ge 1$ and for all $x \in [0, 1]$.  
\end{proof}

% 
% Next, 

\begin{prop}[Formal statement of \propref{prop:early_stop}] \label{prop:formal_early_stop}
Let $\lambda = \frac{1}{t\eta}$. 
For a training point $x$, we have 
\begin{align*}
    \Expt{x \sim \calS}{(f_t(x) - \wt f_\lambda(x))^2} &\le c(t,\eta) \cdot \Expt{x \sim \calS}{f_t(x)^2} \,, %\label{eq:early_stop}
\end{align*}
where $c(t, \eta) \defeq \min( 0.25, \frac{1}{s_\text{min}^2 t^2 \eta^2})$. 
Similarly for a test point, we have 
\begin{align*}
    \Expt{x \sim \calD_\calX}{(f_t(x) - \wt f_\lambda(x))^2} &\le \kappa \cdot c(t,\eta) \cdot \Expt{x \sim \calD_\calX}{f_t(x)^2} \,. %\label{eq:early_stop}
\end{align*}
\end{prop} 

\begin{proof}
    %%%%%%%%%%%%% 
    We want to analyze the expected squared difference output 
    of regularized linear regression 
    with regularization constant $\lambda = \frac{1}{\eta t}$ 
    and the gradient descent solution at the $t^\text{th}$ iterate. 
    We separately expand the algebraic expression 
    for squared difference at a training point and a test point. 
    % We start by considering the difference  
    Then the main step is to show that 
    $\left[ \bS ^{-1} ( \bI - (\bI - \eta \bS)^k )  - (\bS + \lambda \bI )^{-1}\right] \preceq c(\eta, t) \cdot \bS ^{-1} ( \bI - (\bI - \eta \bS)^k ) $.

    %%%%%%%%%%%%%
    
   \textbf{Part 1 {} {}} 
    First, we will analyze the squared difference 
    of the output at a training point 
    (for simplicity, we refer to $S \cup \wt S$ as $S$), i.e., 
    \begin{align}
        \Expt{ x \sim \calS }{\left(f_t(x) - \wt f_\lambda (x)\right)^2} &= \norm{\bX w_t - \bX \wt w_\lambda}{2}^2\\ &=   \norm{\bX \bV \bS ^{-1} ( \bI - (\bI - \eta \bS)^t ) \bV^T \bX^T \by - \bX \bV (\bS + \lambda \bI )^{-1} \bV^T \bX^T \by }{2}^2 \\
        &= \norm{\bX \bV \left(\bS ^{-1} ( \bI - (\bI - \eta \bS)^t ) - (\bS + \lambda \bI )^{-1} \right) \bV^T \bX^T \by  }{2} \\
        &=  \by^T \bV \bX \left( \underbrace{\bS ^{-1} ( \bI - (\bI - \eta \bS)^t ) - (\bS + \lambda \bI )^{-1}}_{\RN{1}} \right)^2 \bS \bV^T \bX^T \by \label{eq:train_GD_rel} \,.
        %  (\bX \bV \bS ^{-1} ( \bI - (\bI - \eta \bS)^k ) \bV^T \bX^T \by)^T \bX \bV \bS ^{-1} ( \bI - (\bI - \eta \bS)^k ) \bV^T \bX^T \by
    \end{align}
    We now separately consider term 1. 
    Substituting $\lambda = \frac{1}{t \eta}$, 
    we get
    \begin{align}
        \bS ^{-1} ( \bI - (\bI - \eta \bS)^t ) - (\bS + \lambda \bI )^{-1} &= \bS^{-1} \left( ( \bI - (\bI - \eta \bS)^t ) - (\bI + \bS^{-1} \lambda )^{-1}\right) \\
        &= \underbrace{\bS^{-1} \left( ( \bI - (\bI - \eta \bS)^t ) - (\bI + ( \bS t \eta)^{-1}  )^{-1}\right)}_{\bA} \,.
    \end{align}

    We now separately bound the diagonal entries in matrix $\bA$. 
    With $s_i$, we denote $i^{\text{th}}$ diagonal entry of $\bS$.
    Note that since $ \eta\le 1/\norm{S}{\text{op}}$, 
    for all $i$, $\eta s_i  \le 1$.  
    Consider $i^{\text{th}}$ diagonal term (which is non-zero) 
    of the diagonal matrix $\bA$, we have 
    \begin{align}
        \bA_{ii} = \frac{1}{s_i} \left(  1 - (1 - s_i \eta)^t - \frac{t \eta s_i}{1 + t \eta s_i } \right) &=  \frac{1 - (1 - s_i \eta)^t}{s_i} \left( \underbrace{ 1 - \frac{t \eta s_i}{(1 + t \eta s_i)(1 - (1 - s_i \eta)^t)}}_{\RN{2}} \right) \\ 
         &\le \frac{1}{2}\left[ \frac{1 - (1 - s_i \eta)^t}{ s_i} \right] \tag*{(Using \lemref{lem:ineq_soln} (b))} \,.
    \end{align} 
    Additionally, we can also show the following upper bound on term 2: 
    \begin{align}
         1 - \frac{t \eta s_i}{(1 + t \eta s_i)(1 - (1 - s_i \eta)^t)} &= \frac{(1 + t \eta s_i)(1 - (1 - s_i \eta)^t) - t \eta s_i }{(1 + t \eta s_i)(1 - (1 - s_i \eta)^t)} \\
         & \le  \frac{ 1 -  (1 - s_i \eta)^t - t \eta s_i (1 - s_i \eta)^t}{(1 + t \eta s_i)(1 - (1 - s_i \eta)^t)} \\
         & \le \frac{1}{t\eta s_i} \,. \tag{Using \lemref{lem:ineq_soln} (a)}
        %  &\le \frac{1}{2}\left[ \frac{1 - (1 - s_i \eta)^t}{ s_i} \right] \tag*{(Using \lemref{lem:ineq_soln})} \,.
    \end{align} 

    Combining both the upper bounds 
    on each diagonal entry $\bA_{ii}$, we have 
    \begin{align}
    \bA \preceq c_1(\eta, t) \cdot \bS^{-1} ( \bI - (\bI - \eta \bS)^t ) \,, \label{eq:upperbound_diagonal}
    \end{align}
    where $c_1(\eta, t ) = \min(0.5, \frac{1}{t s_i \eta })$. Plugging this into \eqref{eq:train_GD_rel}, we have 
    \begin{align}
        \Expt{ x \sim \calS }{\left(f_t(x) - \wt f_\lambda (x)\right)^2} &\le c(\eta, t) \cdot \by^T \bV \bX  \left( \bS^{-1} ( \bI - (\bI - \eta \bS)^t ) \right)^2 \bS \bV^T \bX^T \by \\
        &=   c(\eta, t) \cdot \by^T \bV \bX  \left( \bS^{-1} ( \bI - (\bI - \eta \bS)^t ) \right) \bS \left( \bS^{-1} ( \bI - (\bI - \eta \bS)^t ) \right) \bV^T \bX^T \by \\
        & =  c(\eta, t) \cdot \norm{\bX w_t}{2}^2 \\
        &= c(\eta, t) \cdot  \Expt{ x \sim \calS }{\left(f_t(x) \right)^2} \,,
    \end{align}
    where $c(\eta, t ) = \min(0.25, \frac{1}{t^2 s^2_i \eta^2 })$.

    \textbf{Part 2 {} {}} With $\bSigma$, 
    we denote the underlying true covariance matrix. 
    We now consider the squared difference of output at an unseen point: 
    \begin{align}
        \Expt{ x \sim \calD_{\calX} }{\left(f_t(x) - \wt f_\lambda (x)\right)^2} &= \Expt{x \sim \calD_{\calX}}{\norm{x^T w_t - x^T \wt w_\lambda}{2}} \\
        &=   \norm{x^T \bV \bS ^{-1} ( \bI - (\bI - \eta \bS)^t ) \bV^T \bX^T \by - x^T \bV (\bS + \lambda \bI )^{-1} \bV^T \bX^T \by }{2} \\
        &= \norm{x^T \bV \left(\bS ^{-1} ( \bI - (\bI - \eta \bS)^t ) - (\bS + \lambda \bI )^{-1} \right) \bV^T \bX^T \by  }{2} \\
        &= \by^T \bV \bX \left( \bS ^{-1} ( \bI - (\bI - \eta \bS)^t ) - (\bS + \lambda \bI )^{-1} \right) \bV^T \bSigma \bV \\ &\qquad \qquad \qquad \qquad \qquad \left( (\bI - (\bI - \eta \bS)^t ) - (\bS + \lambda \bI )^{-1} \right) \bV^T \bX^T \by \\
        &\le \sigma_{\text{max}} \cdot \by^T \bV \bX \left( \underbrace{\bS ^{-1} ( \bI - (\bI - \eta \bS)^t ) - (\bS + \lambda \bI )^{-1}}_{\RN{1}} \right)^2 \bV^T \bX^T \by \,, \label{eq:test_GD_rel}
        %  (\bX \bV \bS ^{-1} ( \bI - (\bI - \eta \bS)^k ) \bV^T \bX^T \by)^T \bX \bV \bS ^{-1} ( \bI - (\bI - \eta \bS)^k ) \bV^T \bX^T \by
    \end{align}
    where $\sigma_{\text{max}}$ is the maximum eigenvalue 
    of the underlying covariance matrix $\bSigma$. 
    Using the upper bound on term 1 in \eqref{eq:upperbound_diagonal}, 
    we have 
    \begin{align}
        \Expt{ x \sim \calD_{\calX} }{\left(f_t(x) - \wt f_\lambda (x)\right)^2} &\le \sigma_{\text{max}} \cdot c(\eta, t) \cdot \by^T \bV \bX  \left( \bS^{-1} ( \bI - (\bI - \eta \bS)^t ) \right)^2 \bV^T \bX^T \by \\
        &=   \kappa \cdot c(\eta, t) \cdot \sigma_{\text{min}}\cdot \norm{\bV \left( \bS^{-1} ( \bI - (\bI - \eta \bS)^t ) \right) \bV^T \bX^T \by}{2}^2 \\
        &\le \kappa \cdot c(\eta, t) \cdot \left[ \bV \left( \bS^{-1} ( \bI - (\bI - \eta \bS)^t ) \right) \bV^T \bX^T \right]^T \bSigma \\
        &\qquad \qquad \qquad \qquad \qquad \left[ \bV \left( \bS^{-1} ( \bI - (\bI - \eta \bS)^t ) \right) \bV^T \bX^T \right] \by \\
        & = \kappa \cdot c(\eta, t) \cdot \Expt{x \sim \calD_{\calX}}{\norm{x^T w_t}{2}} \,.
    \end{align}
% 
% 
    % Since $ \eta\le 1/\norm{S}{\text{op}}$, invoking \lemref{lem:ineq_soln} to upper bound term 1 with
\end{proof}

\subsection{Extension to deep learning} \label{appsubsec:ext_DL}
Under \asmpref{appsubsec:justifying_assumption1}, we present the formal result parallel to \thmref{thm:multiclass_ERM}. 
\begin{theorem} \label{thm:multiclass_ERM_algoA}
    Consider a multiclass classification problem 
    with $k$ classes. Under \asmpref{asmp:deep_models}, 
    for any $\delta >0$, with probability at least $1-\delta$,
    we have
    \vspace{-10pt}
    \begin{align*}
        \error_\calD(\widehat f)  \le \error_\calS(\widehat f) + (k-1) \left(1 - \tfrac{k}{k-1} \error_{\wt\calS}(\widehat f)\right) + c\sqrt{\frac{\log(\frac{4}{\delta})}{2m}} \,,\numberthis \label{eq:multiclass_ERM_deep}
    % \vspace{-20pt}
    \end{align*}
    for some constant $c \le ((c+1) k+\sqrt{k} + \frac{m}{n\sqrt{k}})$.
\end{theorem}

The proof follows exactly as in step (i) to (iii) in \thmref{thm:multiclass_ERM}.  

\subsection{Justifying~\asmpref{asmp:deep_models}} \label{appsubsec:justifying_assumption1}

Motivated by the analysis on linear models, we now discuss alternate (and weaker) conditions that imply \asmpref{asmp:deep_models}. 
We need hypothesis stability (\codref{cond:hypothesis_stability}) and the following assumption relating training error and leave-one-error: 

\begin{assumption} \label{asmp:loo_error}
Let $\wh f$ be a model obtained by training with algorithm $\calA$ on a mixture of clean $S$ and randomly labeled data $\wt S$. Then we assume we have 
\begin{align*}
    \error_{\wt \calS_M} (\wh f) \le  \error_{\text{LOO} (\wt S_M)} \,, 
\end{align*}
for all $(x_i, y_i) \in  \wt S_M$ where $\wh f_{(i)} \defeq f(\calA, S \cup {{}\wt S_M}_{(i)})$ and  $\error_{\text{LOO} (\wt S_M)} \defeq  \frac{\sum_{(x_i, y_i) \in \wt S_M} \error(\ff{i}(x_i), y_i ) }{\abs{\wt \calS_M}}$.  
\end{assumption}

% we assume this to extend our result (parallel to \thmref{thm:multi_linear}) for deep models. 
Intuitively, this assumption states that the error on a (mislabeled) datum $(x,y)$ included in the training set is less than the error on that datum $(x,y)$ obtained by a model trained on the training set $S - \{(x,y)\}$. We proved this for linear models trained with GD in the proof of \thmref{thm:multi_linear}. 
% 
\codref{cond:hypothesis_stability} with $\beta = \calO(1)$ and \asmpref{asmp:loo_error} together with \lemref{lem:stability_error} implies \asmpref{asmp:deep_models} with a polynomial residual term (instead of logarithmic in $1/\delta$): 
\begin{align}
     \error_{\calS_M} (\wh f) \le  \error_{\calDm}(\wh f)   + \sqrt{\frac{1}{\delta}\left(\frac{1}{m} +\frac{3\beta}{m+n} \right)} \,.
\end{align}
% Note that this  

\newpage 
\section{Additional experiments and details}\label{app:exp}
\newcommand\tab[1][1cm]{\hspace*{#1}}

\subsection{Datasets} \label{sec:app_dataset}

\textbf{Toy Dataset {} {}} Assume fixed constants $\mu$ and $\sigma$. For a given label $y$, we simulate features $x$ in our toy classification setup as follows: 
\begin{align*}
    x \defeq \texttt{concat} \left[ x_1, x_2\right] \quad \text{where} \quad  x_1 \sim  \calN( y \cdot \mu, \sigma^2 I_{d \times d}) \ \  \text{and} \ \  x_1 \sim  \calN( 0, \sigma^2 I_{d \times d}) \,.
\end{align*}  
% where $y$ is the true label and $x$ is the corresponding feature vector. 
In experiements throughout the paper, we fix dimention $d=100$, $\mu = 1.0 $, and $\sigma = \sqrt{d}$. Intuitively, $x_1$ carries the information about the underlying label and $x_2$ is additional noise independent of the underlying label. 

\textbf{CV datasets {} {}} We use MNIST~\citep{lecun1998mnist} and CIFAR10~\cite{krizhevsky2009learning}. 
% For binary tasks, 
We produce a binary variant from the multiclass classification problem by mapping classes $\{0,1,2,3,4\}$ to label $1$ and $\{ 5,6,7,8,9\}$ to label $-1$. For CIFAR dataset, we also use the standard data augementation of random crop and horizontal flip. PyTorch code is as follows: 

\texttt{(transforms.RandomCrop(32, padding=4),\\
\tab transforms.RandomHorizontalFlip())}

\textbf{NLP dataset {} {}} We use IMDb Sentiment analysis~\citep{maas2011learning} corpus.  

\subsection{Architecture Details} 

All experiments were run on NVIDIA GeForce RTX 2080 Ti GPUs. We used PyTorch~\citep{NEURIPS2019a9015} and Keras with Tensorflow~\citep{abadi2016tensorflow} backend for experiments. 
% , ELMo embeddings~\citep{Peters:2018}, and Hugging Face Transformers~\citep{wolf-etal-2020-transformers}. 

\textbf{Linear model {} {}} For the toy dataset, we simulate a linear model with scalar output and the same number of parameters as the number of dimensions.   

\textbf{Wide nets {} {}} To simulate the NTK regime, we experiment with $2-$layered wide nets. The PyTorch code for 2-layer wide MLP is as follows: 


\texttt{ nn.Sequential( \\
\tab     nn.Flatten(),\\
\tab    nn.Linear(input\_dims, 200000, bias=True),\\
\tab    nn.ReLU(),\\
\tab    nn.Linear(200000, 1, bias=True)\\
\tab     )}


We experiment both (i) with the second layer fixed at random initialization; (ii)  and updating both layers' weights.     

\textbf{Deep nets for CV tasks {} {}} We consider a 4-layered MLP. The PyTorch code for 4-layer MLP is as follows: 

\texttt{ nn.Sequential(nn.Flatten(), \\
\tab        nn.Linear(input\_dim, 5000, bias=True),\\
\tab        nn.ReLU(),\\
\tab        nn.Linear(5000, 5000, bias=True),\\
\tab        nn.ReLU(),\\
\tab        nn.Linear(5000, 5000, bias=True),\\
\tab        nn.ReLU(),\\
% \tab        nn.Linear(5000, 5000, bias=True),\\
% \tab        nn.ReLU(),\\
\tab        nn.Linear(1024, num\_label, bias=True)\\
\tab        )}

For MNIST, we use $1000$ nodes instead of $5000$ nodes in the hidden layer. 
% 
We also experiment with convolutional nets. In particular, we use ResNet18 \citep{he2016deep}. Implementation adapted from:  \url{https://github.com/kuangliu/pytorch-cifar.git}. 

\textbf{Deep nets for NLP {} {}} We use a simple LSTM model with embeddings intialized with ELMo embeddings~\citep{Peters:2018}. Code adapted from: \url{https://github.com/kamujun/elmo_experiments/blob/master/elmo_experiment/notebooks/elmo_text_classification_on_imdb.ipynb} 

We also evaluate our bounds with a BERT model. In particular, we fine-tune an off-the-shelf uncased BERT model~\citep{devlin2018bert}. Code adapted from Hugging Face Transformers~\citep{wolf-etal-2020-transformers}: \url{https://huggingface.co/transformers/v3.1.0/custom_datasets.html}. 


\subsection{Additonal experiments}

\textbf{Results with SGD on underparameterized linear models {} {}} 

\begin{figure*}[h]
    \centering 
    % \vspace{-15pt}
    % \includegraphics[width=0.9\linewidth]{example-image-a}
    \includegraphics[width=0.3\linewidth]{figures/lowdim-Gaussian-SGD.pdf}
    % \includegraphics[width=0.9\linewidth]{figures/{CIFAR10_rn=0.1_lr=0.2_wd=0.005}.png}
    \vspace{-5pt}
    \caption{ 
    % Predicted lower bound 
    % on different
    We plot the accuracy and corresponding bound 
    (RHS in \eqref{eq:erm}) at $\delta = 0.1$
    for toy binary classification task. 
    Results aggregated over $3$ seeds. 
    % i.e., $1-\error$ where $\error$ is the term in the RHS of \eqref{eq:erm}
    Accuracy vs fraction of unlabeled data (w.r.t clean data) 
    in the toy setup with a linear model trained with SGD. Results parallel to \figref{fig:error_binary}(a) with SGD.  }
    \label{fig:error_binary_linear}
    \vspace{-5pt}
\end{figure*}

\textbf{Results with wide nets on binary MNIST {} {}}

\begin{figure*}[h]
    \centering 
    % \vspace{-15pt}
    % \includegraphics[width=0.9\linewidth]{example-image-a}
    \subfigure[GD with MSE loss]{\includegraphics[width=0.3\linewidth]{figures/MNIST-GD_MSE.pdf}} \hfil
    \subfigure[SGD with CE loss]{\includegraphics[width=0.3\linewidth]{figures/MNIST-SGD_CE.pdf}}
    \subfigure[SGD with MSE loss]{\includegraphics[width=0.3\linewidth]{figures/MNIST-SGD_MSE-first-layer.pdf}}
    % \includegraphics[width=0.9\linewidth]{figures/{CIFAR10_rn=0.1_lr=0.2_wd=0.005}.png}
    \vspace{-5pt}
    \caption{ 
    % Predicted lower bound 
    % on different
    We plot the accuracy and corresponding bound 
    (RHS in \eqref{eq:erm}) at $\delta = 0.1$ 
    for binary MNIST classification. 
    Results aggregated over $3$ seeds. 
    % i.e., $1-\error$ where $\error$ is the term in the RHS of \eqref{eq:erm}
    Accuracy vs fraction of unlabeled data 
    for a 2-layer wide network on binary MNIST with both the layers training in (a,b) and only first layer training in (c). 
    Results parallel to \figref{fig:error_binary}(b) .  }
    \label{fig:error_binary_MNIST}
    \vspace{-5pt}
\end{figure*}

% \begin{figure*}[h]
%     \centering 
%     % \vspace{-15pt}
%     % \includegraphics[width=0.9\linewidth]{example-image-a}
%     \subfigure[GD with MSE loss]{\includegraphics[width=0.3\linewidth]{figures/MNIST.pdf}} \hfil
    
%     \subfigure[SGD with CE loss]{\includegraphics[width=0.3\linewidth]{figures/MNIST.pdf}}
%     % \includegraphics[width=0.9\linewidth]{figures/{CIFAR10_rn=0.1_lr=0.2_wd=0.005}.png}
%     \vspace{-5pt}
%     \caption{ 
%     % Predicted lower bound 
%     % on different
%     We plot the accuracy and corresponding bound 
%     (RHS in \eqref{eq:erm}) at $\delta = 0.1$
%     for binary MNIST classification. 
%     Results aggregated over $3$ seeds. 
%     % i.e., $1-\error$ where $\error$ is the term in the RHS of \eqref{eq:erm}
%     Accuracy vs fraction of unlabeled data 
%     for a 2-layer wide network on binary MNIST with just the first layer training. 
%     Results parallel to \figref{fig:error_binary}(b) with only the first layer training.  }
%     \label{fig:error_binary_MNIST}
%     \vspace{-5pt}
% \end{figure*}

\textbf{Results on CIFAR 10 and MNIST {} {}} 
% 
We plot epoch wise error curve for results in \tabref{table:multiclass}(\figref{fig:error_epoch_CIFAR10} and \figref{fig:error_epoch_MNIST}). We observe the same trend as in \figref{fig:error_CIFAR10}. Additionally, we plot an \emph{oracle bound} obtained by tracking the error on mislabeled data which nevertheless were predicted as true label. To obtain an exact emprical value of the oracle bound, we need underlying true labels for the randomly labeled data. 
% Note that our bound in \thmref{thm:multiclass_ERM}, lower bounds the accuracy as predicted by the oracle bound. 
While with just access to extra unlabeled data we cannot calculate oracle bound, we note that the oracle bound is very tight and never violated in practice underscoring an importamt aspect of generalization in multiclass problems. This highlight that even a stronger conjecture may hold in multiclass classification, i.e., error on mislabeled data (where nevertheless true label was predicted) lower bounds the population error on the distribution of mislabeled data and hence, the error on (a specific) mislabeled portion predicts the population accuracy on clean data. 
% 
On the other hand, the dominating term of in \thmref{thm:multiclass_ERM} is loose when compared with the oracle bound. The main reason, we believe is the pessimistic upper bound in \eqref{eq:lemma1_final_multi_prev} in the proof of \lemref{lem:fit_mislabeled_multi}. We leave an investigation on this gap for future. 
% of fit 

% However, oracle bound highlights two . One,  



\begin{figure}[h]
    \centering 
    % \vspace{-15pt}
    % \includegraphics[width=0.9\linewidth]{example-image-a}
    \subfigure[MLP]{\includegraphics[width=0.3\linewidth]{figures/CIFAR10-FNN.pdf}} \hfil
    \subfigure[ResNet]{\includegraphics[width=0.3\linewidth]{figures/CIFAR10-Resnet.pdf}}
    % \includegraphics[width=0.9\linewidth]{figures/{CIFAR10_rn=0.1_lr=0.2_wd=0.005}.png}
    % \vspace{-10pt}
    \caption{ Per epoch curves for CIFAR10 corresponding results in \tabref{table:multiclass}. As before, we just plot the dominating term in the RHS of \eqref{eq:multiclass_ERM} as predicted bound. Additionally, we also plot the predicted lower bound by the error on mislabeled data which nevertheless were predicted as true label. We refer to this as ``Oracle bound''. See text for more details. 
    % 
    % except for the stopping point. 
    % The bound predicted by RATT (RHS in \eqref{eq:multiclass_ERM}) is vacuous. 
    }\label{fig:error_epoch_CIFAR10}
    % \vspace{-15pt}
\end{figure}


\begin{figure}[h]
    \centering 
    % \vspace{-15pt}
    % \includegraphics[width=0.9\linewidth]{example-image-a}
    \subfigure[MLP]{\includegraphics[width=0.3\linewidth]{figures/MNIST-FNN.pdf}} \hfil
    \subfigure[ResNet]{\includegraphics[width=0.3\linewidth]{figures/MNIST-Resnet.pdf}}
    % \includegraphics[width=0.9\linewidth]{figures/{CIFAR10_rn=0.1_lr=0.2_wd=0.005}.png}
    % \vspace{-10pt}
    \caption{ Per epoch curves for MNIST corresponding results in \tabref{table:multiclass}. As before, we just plot the dominating term in the RHS of \eqref{eq:multiclass_ERM} as predicted bound. Additionally, we also plot the predicted lower bound by the error on mislabeled data which nevertheless were predicted as true label. We refer to this as ``Oracle bound''. See text for more details. 
    % 
    % except for the stopping point. 
    % The bound predicted by RATT (RHS in \eqref{eq:multiclass_ERM}) is vacuous. 
    }\label{fig:error_epoch_MNIST}
    % \vspace{-15pt}
\end{figure}

\textbf{Results on CIFAR 100 {} {}} 
% 
On CIFAR100, our bound in \eqref{eq:multiclass_ERM} yields vacous bounds. However, the oracle bound as explained above yields tight guarantees in the initial phase of the learning (i.e., when learning rate is less than $0.1$) (\figref{fig:error_CIFAR100}).  

\begin{figure}[h]
    \centering 
    % \vspace{-15pt}
    % \includegraphics[width=0.9\linewidth]{example-image-a}
    \includegraphics[width=0.3\linewidth]{figures/CIFAR100-Resnet.pdf}
    % \includegraphics[width=0.9\linewidth]{figures/{CIFAR10_rn=0.1_lr=0.2_wd=0.005}.png}
    % \vspace{-10pt}
    \caption{ Predicted lower bound by the error on mislabeled data which nevertheless were predicted as true label with ResNet18 on CIFAR100. We refer to this as ``Oracle bound''. See text for more details. 
    % 
    % except for the stopping point. 
    The bound predicted by RATT (RHS in \eqref{eq:multiclass_ERM}) is vacuous. 
    }\label{fig:error_CIFAR100}
    % \vspace{-15pt}
\end{figure}


% \paragraph{Experiments on CIFAR100} 


% \subsection{Model Selection using RATT}


\subsection{Hyperparameter Details}


\textbf{\figref{fig:error_CIFAR10} {} {}} We use clean training dataset of size $40,000$. We fix the amount of unlabeled data at $20\%$ of the clean size, i.e. we include additional $8,000$ points with randomly assigned labels. We use test set of $10,000$ points. For both MLP and ResNet, we use SGD with an initial learning rate of $0.1$ and momentum $0.9$. We fix the weight decay parameter at $5\times 10^{-4}$. After $100$ epochs, we decay the learning rate to $0.01$. We use SGD batch size of $100$. 

\textbf{\figref{fig:error_binary} (a) {} {}} We obtain a toy dataset according to the process described in \secref{sec:app_dataset}. We fix $d=100$ and create a dataset of $50,000$ points with balanced classes. Moreover, we sample additional covariates with the same procedure to create randomly labeled dataset. For both SGD and GD training, we use a fixed learning rate $0.1$.    

\textbf{\figref{fig:error_binary} (b) {} {}} Similar to binary CIFAR, we use clean training dataset of size $40,000$ and fix the amount of unlabeled data at $20\%$ of the clean dataset size. To train wide nets, we use a fixed learning of $0.001$ with GD and SGD. We decide the weight decay parameter and the early stopping point that maximizes our generalization bound (i.e. without peeking at unseen data ).  We use SGD batch size of $100$. 

\textbf{\figref{fig:error_binary} (c) {} {}} With IMDb dataset, we use a clean dataset of size $20,000$ and as before, fix the amount of unlabeled data at $20\%$ of the clean data. To train ELMo model, we use Adam optimizer with a fixed learning rate $0.01$ and weight decay $10^{-6}$ to minimize cross entropy loss. We train with batch size $32$ for 3 epochs. To fine-tune BERT model, we use Adam optimizer with learning rate $5\times 10^{-5}$ to minimize cross entropy loss. We train with a batch size of $16$ for 1 epoch.    

\textbf{\tabref{table:multiclass} {} {}} For multiclass datasets, we train both MLP and ResNet with the same hyperparameters as described before. We sample a clean training dataset of size $40,000$ and fix the amount of unlabeled data at $20\%$ of the clean size. We use SGD with an initial learning rate of $0.1$ and momentum $0.9$. We fix the weight decay parameter at $5\times 10^{-4}$. After $30$ epochs for ResNet and after $50$ epochs for MLP, we decay the learning rate to $0.01$.  We use SGD with batch size $100$. 
For \figref{fig:error_CIFAR100}, we use the same hyperparameters as 
CIFAR10 training, except we now decay learning rate after $100$ epochs. 


In all experiments, to identify the best possible accuracy on just the clean data, we use the exact same set of hyperparamters except the stopping point. We choose a stopping point that maximizes test performance. 

\subsection{Summary of experiments }

\begin{center}
    \begin{table}[H] 
        \centering
        \begin{tabular}{|c|c|c|c|} 
        \hline
        Classification type & Model category & Model & Dataset  \\ [0.5ex] 
        \hline
        \hline
        \multirow{10}{*}{Binary} & Low dimensional & Linear model & Toy Gaussain dataset  \\
                        \cline{2-4}
                         & Overparameterized 
                        %  & Linear model & Toy Gaussain dataset \\
                        %  \cline{3-4}
                        %  & & 2-layer wide net& Toy Gaussain dataset \\
                        %  \cline{3-4}
                         & \multirow{2}{*}{2-layer wide net} & \multirow{2}{*}{Binary MNIST} \\
                         & linear nets & &  
                         \\
                         \cline{2-4}                 
                         & \multirow{6}{*}{Deep nets} & \multirow{2}{*}{MLP} & Binary MNIST \\
                         \cline{4-4}
                         & &  & Binary CIFAR \\
                         \cline{3-4}
                         &  & \multirow{2}{*}{ResNet} & Binary MNIST \\
                         \cline{4-4}
                         & &  & Binary CIFAR \\
                         \cline{3-4}
                         &  & ELMo-LSTM model & IMDb Sentiment Analysis \\
                         \cline{3-4}
                         & & BERT pre-trained model & IMDb Sentiment Analysis \\
        \hline
        \multirow{5}{*}{Multiclass} & \multirow{5}{*}{Deep nets} & \multirow{2}{*}{MLP} & MNIST \\
                        \cline{4-4} 
                        & & & CIFAR10 \\                   
                        \cline{3-4}
                         &   & \multirow{3}{*}{ResNet} & MNIST \\
                         \cline{4-4}
                         &   & & CIFAR10 \\
                         \cline{4-4}
                         &   & & CIFAR100 \\
        \hline
        \end{tabular}
        % \caption{Summary of experiments performed} \label{table:experiments}
    \end{table}    
    % \footnotetext[6]{We use both MSE loss and cross-entropy loss.}
    % \footnotetext[6]{We try 2 variants: one with a fixed first layer and the other with both layers trainable.}
\end{center}

\newpage
\section{Proof of \lemref{lem:stability_error}} \label{app:proof_lem_error}

\begin{proof}[Proof of \lemref{lem:stability_error}]
    Recall, we have a training set $S \cup \wt S_C$. We defined leave-one-out error on mislabeled points as $$\error_{\text{LOO}(\wt S_M) } = \frac{\sum_{(x_i, y_i) \in \wt S_M} \error( f_{(i)}( x_i), y_i)}{ \abs{\wt S_M }} \,, $$
    where $f_{(i)} \defeq f(\calA, (S \cup \wt S)_{(i)})$. Define $S^\prime \defeq S \cup \wt S$. Assume $(x,y)$ and $(x^\prime,y^\prime)$ as i.i.d. samples from ${\calDm}$. 
    Using Lemma 25 in \citet{bousquet2002stability}, we have
    \begin{align*}
        \Expo{ \left( \error_{\calDm}(\wh f) -\error_{\text{LOO}(\wt S_M) } \right)^2 } \le & \Expt{ S^\prime, (x,y), (x^\prime,y^\prime) }{ \error(\wh f(x), y ) \error(\wh f(x^\prime), y^\prime )} - 2 \Expt{ S^\prime, (x,y) }{ \error(\wh f(x), y ) \error(f_{(i)}(x_i), y_i )} \\
        & + \frac{m_1-1}{m_1}\Expt{ S^\prime }{  \error(f_{(i)}(x_i), y_i )  \error(f_{(j)}(x_j), y_j )} + \frac{1}{m_1} \Expt{ S^\prime }{  \error(f_{(i)}(x_i), y_i ) } \,. \numberthis \label{eq:main_reln}
    \end{align*}
    We can rewrite the equation above as : 
    \begin{align*}
        \Expo{ \left( \error_{\calDm}(\wh f) -\error_{\text{LOO}(\wt S_M) } \right)^2 } \le &  \, \underbrace{\Expt{ S^\prime, (x,y), (x^\prime,y^\prime) }{ \error(\wh f(x), y ) \error(\wh f(x^\prime), y^\prime ) - \error(\wh f(x), y ) \error(f_{(i)}(x_i), y_i )}}_{\RN{1}} \\
        & + \underbrace{\Expt{ S^\prime }{  \error(f_{(i)}(x_i), y_i )  \error(f_{(j)}(x_j), y_j ) -  \error(\wh f(x), y ) \error(f_{(i)}(x_i), y_i )}}_{\RN{2}} \\ &+ \underbrace{\frac{1}{m_1} \Expt{ S^\prime }{  \error(f_{(i)}(x_i), y_i ) - \error(f_{(i)}(x_i), y_i )  \error(f_{(j)}(x_j), y_j ) }}_{\RN{3}} \,. \numberthis \label{eq:main_reln2}
    \end{align*}
    
    We will now bound term $\RN{3}$.  Using Cauchy-Schwarz's inequality, we have
    
    \begin{align}
        \Expt{ S^\prime }{  \error(f_{(i)}(x_i), y_i ) - \error(f_{(i)}(x_i), y_i )  \error(f_{(j)}(x_j), y_j ) }^2 &\le  \Expt{ S^\prime }{  \error(f_{(i)}(x_i), y_i ) }^2 \Expt{S^\prime}{1 -   \error(f_{(j)}(x_j), y_j ) }^2 \\
        &\le \frac{1}{4} \,.\label{eq:term1_lem12}
    \end{align}
    
    Note that since $(x_i,y_i)$, $(x_j ,y_j )$, $(x,y)$, and $(x^\prime, y^\prime)$ are all from same distribution $\calDm$, we directly incorporate the bounds on term $\RN{1}$ and $\RN{2}$ from the proof of Lemma 9 in \citet{bousquet2002stability}. Combining that with \eqref{eq:term1_lem12} and our definition of hypothesis stability in \codref{cond:hypothesis_stability}, we have the required claim. 
    
    
    % We now re-write term $\RN{1}$ as
    % \begin{align*}
    %         &\Expt{S^\prime, (x,y), (x^\prime,y^\prime) }{ \error(\wh f(x), y ) \error(\wh f(x^\prime), y^\prime ) - \error(\wh f(x), y ) \error(f_{(i)}(x_i), y_i )} \\ & \qquad = \Expt{ S^\prime, (x,y), (x^\prime,y^\prime) }{ \error(\wh f(x), y ) \error(\wh f  (x^\prime), y^\prime ) - \error(\wh f ^\prime(x), y ) \error(f_{(i)}(x^\prime), y^\prime )} \tag{Exchanging $(x_i, y_i)$ with $(x^\prime, y^\prime)$ in the second term} \\
    %         & \qquad = \Expt{ S^\prime, (x,y), (x^\prime,y^\prime) }{  \left(\error(\wh f(x), y )-  \error(f_{(i)}(x), y ) \right) \error(\wh f  (x^\prime), y^\prime )  } \\
    %         & \qquad  + \Expt{ S^\prime, (x,y), (x^\prime,y^\prime) }{  \left(\error(f_{(i)}(x), y ) -\error(\wh f ^\prime(x), y ) \right) \error(\wh f  (x^\prime), y^\prime )}  \\
    %         & \qquad +\Expt{ S^\prime, (x,y), (x^\prime,y^\prime) }{  \left( \error(\wh f  (x^\prime), y^\prime ) -  \error(f_{(i)}(x^\prime), y^\prime ) \right) \error(\wh f ^\prime(x), y ) }  \,, \numberthis \label{eq:term1_final}
    % \end{align*}
    % where $\wh f^\prime$ is the classifier obtained by training on $ S^\prime_{(i)} \cup \{ (x^\prime, y^\prime) \} $. Similarly we can re-write term $\RN{2}$ as 
    % \begin{align*}
    %     & \Expt{ S^\prime }{  \error(f_{(i)}(x_i), y_i )  \error(f_{(j)}(x_j), y_j ) -  \error(\wh f(x), y ) \error(f_{(i)}(x_i), y_i )} \\
    %     &\quad  = \Expt{ S^\prime, (x,y), (x^\prime,y^\prime)}{  \error(f^{\prime\prime}_{(i)}(x), y )  \error(f_{(j)}^{\prime}(x^\prime), y^\prime ) -  \error(\wh f(x), y ) \error(f_{(i)}(x_i), y_i )} \tag{Exchanging $(x_i, y_i)$ with $(x, y)$ and $(x_j, y_j)$ with $(x^\prime, y^\prime)$ in the first term}\\
    %     &\quad = \Expt{ S^\prime, (x,y), (x^\prime,y^\prime)}{  \error(f^{\prime\prime}_{(j)}(x), y )  \error(f_{(i)}^{\prime}(x^\prime), y^\prime ) -  \error(\wh f^\prime (x), y ) \error(f^\prime_{(j)}(x^\prime), y^\prime )} \tag{Exchanging $(x_i, y_i)$ and $(x_j, y_j)$ and then replacing $(x_j, y_j)$ with $(x^\prime, y^\prime)$ in the second term} \\
    %     & \quad = \Expt{ S^\prime, (x,y), (x^\prime,y^\prime) }{  \left( \error(f_{(i)}^{\prime}(x^\prime), y^\prime )   -  \error(\wh f^{\prime\prime}  (x^\prime), y^\prime ) \right)  \error(f^{\prime\prime}_{(j)}(x), y )   } \\
    %     & \quad  + \Expt{ S^\prime, (x,y), (x^\prime,y^\prime) }{  \left( \error(f^{\prime\prime}_{(j)}(x), y )  -\error(\wh f ^\prime(x), y ) \right) \error(\wh f^{\prime\prime}  (x^\prime), y^\prime )  }  \\
    %     & \quad+ \Expt{ S^\prime, (x,y), (x^\prime,y^\prime) }{  \left( \error(\wh f^{\prime\prime}  (x^\prime), y^\prime )  -  \error(f^\prime_{(j)}(x^\prime), y^\prime ) \right)  \error(\wh f^\prime (x), y ) }   \\
    %     & \quad = \Expt{ S^\prime, (x,y), (x^\prime,y^\prime) }{  \left( \error(f_{(i)}^{\prime}(x^\prime), y^\prime )   -  \error(\wh f (x^\prime), y^\prime ) \right)  \error(f_{(i)}(x_j), y_j )   } \\
    %     & \quad  + \Expt{ S^\prime, (x,y), (x^\prime,y^\prime) }{  \left( \error(f^{\prime\prime}_{(j)}(x), y )  -\error(\wh f (x), y ) \right) \error(\wh f^{\prime\prime}  (x_j), y_j )  }  \\
    %     & \quad+ \Expt{ S^\prime, (x,y), (x^\prime,y^\prime) }{  \left( \error(\wh f^{\prime\prime}  (x^\prime), y^\prime )  -  \error(f^\prime_{(j)}(x^\prime), y^\prime ) \right)  \error(\wh f^\prime (x^\prime), y^\prime ) }  \,, \numberthis \label{eq:term2_final}
    % \end{align*}
    % where $f^{\prime\prime}_{(j)}$ is trained on $S^\prime_{(j,i)} \cup {(x,y)}$, $f^{\prime}_{(i)}$ is trained on $S^\prime_{(j,i)} \cup {(x^\prime,y^\prime)}$, and $\wh f^{\prime\prime} $ is trained on $S^\prime_{(j)} \cup {(x,y)}$. Note in the last line we replaced $(x,y)$ by $(x_j, y_j)$ in the first term, replaced $(x^\prime,y^\prime)$ by $(x_j, y_j)$ in the second term and exchanged $(x_i,y_i)$ with $(x_j,y_j)$ and also $(x,y)$ and $(x^\prime, y^\prime)$
    
    
\end{proof}


% 
% 16th Century Version Control 
% 

% \onecolumn

% \section*{Supplementary Material}
% We will be using the following standard results
% on exponential concentration of random variables 
% all throughout the discussion:

% \begin{lemma}[Hoeffding's inequality for independent RVs~\citep{hoeffding1994probability}] Let $Z_1, Z_2, \ldots, Z_n$ be independent bounded random variables with $Z_i \in [a,b]$ for all $i$, then 
%     \begin{align*}
%         \prob\left( \frac{1}{n} \sum_{i=1}^n (Z_i - \Expo{Z_i}) \ge t \right) \le \exp{\left( -\frac{2nt^2}{(b-a)^2} \right) }
%     \end{align*} 
%     and 
%     \begin{align*}
%         \prob\left( \frac{1}{n} \sum_{i=1}^n (Z_i - \Expo{Z_i}) \le -t \right) \le \exp{\left( -\frac{2nt^2}{(b-a)^2} \right) }
%     \end{align*} 
%     for all $t \ge 0$. 
% \end{lemma}

% \begin{lemma}[Hoeffding's inequality for sampling with replacement~\citep{hoeffding1994probability}] \label{lem:hoeffding_sampling} Let $\calZ = (Z_1, Z_2, \ldots, Z_N)$ be a finite population of $N$ points with $Z_i \in [a.b]$ for all $i$. Let $X_1, X_2, \ldots X_n$ be a random sample drawn without replacement from $\calZ$. Then for all $t \ge 0$, we have 
%     \begin{align*}
%         \prob\left( \frac{1}{n} \sum_{i=1}^n (X_i - \mu ) \ge t \right) \le \exp{\left( -\frac{2nt^2}{(b-a)^2} \right) }
%     \end{align*} 
%     and 
%     \begin{align*}
%         \prob\left( \frac{1}{n} \sum_{i=1}^n (X_i - \mu ) \le -t \right) \le \exp{\left( -\frac{2nt^2}{(b-a)^2} \right) } \,,
%     \end{align*} 
%     where $\mu = \frac{1}{N} \sum_{i=1}^{N} Z_i$. 
% \end{lemma}

% We now discuss one condition that generalizes the exponential concentration to dependent random variables.
% \begin{condition}[Bounded difference inequality] \label{cond:BDC} Let $\calZ$ be some set and $\phi: \calZ^n \to \Real$. We say that $\phi$ satisfies the bounded difference assumption if 
% there exists $c_1, c_2, \ldots c_n \ge 0$ s.t. for all $i$, we have 
% \begin{align*}
%     \sup_{Z_1,Z_2, \ldots,Z_n, Z_i^\prime in \calZ^{n+1} } \abs{\phi (Z_1, \ldots, Z_i, \ldots, Z_n ) - \phi (Z_1, \ldots, Z_i^\prime, \ldots, Z_n ) } \le c_i \,.
% \end{align*} 
% \end{condition}

% \begin{lemma}[McDiarmid’s inequality~\citep{mcdiarmid1989}] \label{lem:McDiarmid} Let $Z_1, Z_2, \ldots, Z_n$ be independent random variables on set $\calZ$ and $\phi : \calZ^n \to \Real$ satisfy bounded difference assumption (\codref{cond:BDC}). Then for all $t>0$, we have 
%     \begin{align*}
%         \prob\left( \phi(Z_1, Z_2, \ldots, Z_n) - \Expo{\phi(Z_1, Z_2, \ldots, Z_n)} \ge t \right) \le \exp{\left( -\frac{2t^2}{\sum_{i=1}^n c_i^2} \right) } 
%     \end{align*} 
%     and 
%     \begin{align*}
%         \prob\left( \phi(Z_1, Z_2, \ldots, Z_n) - \Expo{\phi(Z_1, Z_2, \ldots, Z_n)} \le -t \right) \le \exp{\left( -\frac{2t^2}{\sum_{i=1}^n c_i^2} \right) } \,
%     \end{align*} 
% \end{lemma}


% \section{Proofs from \secref{sec:ERM_training}}\label{app:proof_erm}

% \textbf{Additional notation {} {}} Let $m_1$ be the number of mislabeled points ($\wt S_M$) and $m_2$ be the number of correctly labeled points ($\wt S_C$). Note $m_1 + m_2 = m$. 


% \subsection{Proof of \thmref{thm:error_ERM}}


% \begin{proof}[Proof of \lemref{lem:fit_mislabeled}] 
%     The main idea of our proof is to regard 
%     the clean portion of the data 
%     ($S \cup \wt S_C$) as fixed.   
%     Then, there exists a classifier $f^*$ 
%     that is optimal over draws 
%     of the mislabeled data $\wt S_M$. 
% % 
%     % 
%     Formally, 
%     \begin{align}
%     f^* \defeq \argmin_{f \in \calF} \error_{\widecheck {\calD}} (f) \,, \label{eq:modified_ERM}
%     \end{align}
%     where $$\widecheck \calD = \frac{n}{m+n} \calS + \frac{m_1}{m+n} \wt \calS_C  + \frac{m_2}{m+n}\calDm \,.$$ That is, $\widecheck \calD$ a combination of 
%     the \emph{empirical distribution} 
%     over correctly labeled data $S \cup \wt S_C$
%     % in $S\cup \wt S$ 
%     and the (population) distribution 
%     over mislabeled data $\calDm$.
%     Recall that 
%     \begin{align}
%     \wh f \defeq \argmin_{f \in \calF} \error_{\calS \cup \wt S} (f) \,. \label{eq:orig_ERM}
%     \end{align}
%     % 
%     % 
%     Since, $\widehat f$ minimizes 0-1 error 
%     on $S \cup \wt S$, using ERM optimality on \eqref{eq:orig_ERM},  
%     we have 
%     \begin{align}
%         \error_{\calS \cup \wt \calS}(\widehat f) \le \error_{
%             \calS \cup \wt \calS}(f^*) \,.    \label{eq:step1}
%     \end{align}
%     Moreover, since $f^*$ is independent of $\wt S_M$, using Hoeffding's bound,
%     % \footnote{For a fully rigorous argument,
%     % refer to the complete proof in App.~\ref{app:proof_erm}.} 
%     we have with probability at least $1-\delta$ that
%     \begin{align}
%       \error_{\wt \calS_M}(f^*) \le \error_{ \calDm}(f^*) +  \sqrt{\frac{\log(1/\delta)}{2 m_1}} \,. \label{eq:step2} 
%     \end{align}
%     %$ 
%     %for some constant $c_1\le 1/2$. 
%     Finally, since $f^*$ is the optimal classifier on $\widecheck \calD$, 
%     we have 
%     \begin{align}
%         \error_{\widecheck \calD}(f^*) \le \error_{\widecheck \calD}(\widehat f) \label{eq:step3}
%     \end{align}
%      Now to relate \eqref{eq:step1} and \eqref{eq:step3}, we can re-write the \eqref{eq:step2} as follows: 
%     \begin{align}
%         \error_{\calS \cup \wt\calS}(f^*) \le \error_{ \widecheck \calD}(f^*) +  \frac{m_1}{m+n}\sqrt{\frac{\log(1/\delta)}{2 m_1}} \,. \label{eq:step4} 
%     \end{align}
%     Now we combine equations \eqref{eq:step1}, \eqref{eq:step4}, and \eqref{eq:step3}, to get 
%     \begin{align}
%         \error_{\calS \cup \wt \calS}(\wh f) \le \error_{\widecheck \calD}(\wh f) +  \frac{m_1}{m+n}\sqrt{\frac{\log(1/\delta)}{2 m_1}} \,, 
%     \end{align}
%     which implies 
%     \begin{align}
%         \error_{ \wt \calS_M}(\wh f) \le \error_{\calDm}(\wh f) + \sqrt{\frac{\log(1/\delta)}{2 m_1}} \,. \label{eq:lemma1_final}
%     \end{align}
%     Since $\wt S$ is obtained by randomly labeling an unlabeled dataset, we assume $2m_1 \approx m$ \footnote{Formally, with probability at least $1-\delta$, we have  $(m - 2m_1)\le \sqrt{m\log(1/\delta)/2}$ }. Moreover, using $\error_{\calDm} = 1 - \error_{\calD}$ we obtain the desired result.   
%     % Combining the above steps and using the fact 
%     % that $\error_\calD = 1- \error_{\calDm} $, 
%     % we obtain the desired result.
% \end{proof}

% \begin{proof}[Proof of \lemref{lem:mislabeled_error}]
%     Recall $\error_{\wt S} (f) = \frac{m_1}{m} \error_{\wt S_M}(f) + \frac{m_2}{m} \error_{\wt S_C}(f)$. Hence, we have 
%     \begin{align}
%         2\error_{\wt S}(f) - \error_{\wt S_M}(f) - \error_{\wt S_C}(f) &= \left(\frac{2m_1}{m} \error_{\wt S_M}(f) - \error_{\wt S_M}(f)\right) + \left(\frac{2m_2}{m} \error_{\wt S_C}(f) - \error_{\wt S_C}(f)\right) \\ &= \left(\frac{2m_1}{m} - 1\right) \error_{\wt S_M}(f) + \left(\frac{2m_2}{m} - 1 \right)\error_{\wt S_C} (f) \,.
%     \end{align} 
%     Since the dataset is randomly labeled, with probability at least $1-\delta$, we have  $\left(\frac{2m_1}{m} - 1\right) \le \sqrt{\frac{\log(1/\delta)}{2m}}$. Similarly, we have with probability at least $1-\delta$, $\left(\frac{2m_2}{m} - 1\right) \le \sqrt{\frac{\log(1/\delta)}{2m}}$. Using union bound, we have with probability at least $1-\delta$
%     % \begin{align}
%     %     2\error_{\wt S} - \error_{\wt S_M}(f) - \error_{\wt S_C}(f) \le \sqrt{\frac{\log(2/\delta)}{2m}} \left(\error_{\wt S_M}(f) + \error_{\wt S_C}(f) \right) \le 2\sqrt{\frac{\log(2/\delta)}{2m}} \,. \label{eq:lemma2_final}
%     % \end{align}
%     \begin{align}
%         2\error_{\wt S} - \error_{\wt S_M}(f) - \error_{\wt S_C}(f) \le \sqrt{\frac{\log(2/\delta)}{2m}} \left(\error_{\wt S_M}(f) + \error_{\wt S_C}(f) \right) \,. \label{eq:lemma2_prefinal}
%     \end{align}
%     With re-arranging $\error_{\wt S_M}(f) + \error_{\wt S_C}(f)$ and using the inequality $ 1- a\le \frac{1}{1+a} $, we have  
%     \begin{align}
%         2\error_{\wt S} - \error_{\wt S_M}(f) - \error_{\wt S_C}(f) \le 2\error_{\wt \calS} \sqrt{\frac{\log(2/\delta)}{2m}}  \,. \label{eq:lemma2_final}
%     \end{align}

%     % We obtain the desired result by using 
% \end{proof}

% \begin{proof}[Proof of \lemref{lem:clear_error}]
% % Recall 0-1 error on each point  $(x,y) \in S \cup \wt S$ is given by $\I{ f(x)\ne y}$.
% In the set of correctly labeled points $S \cup \wt S_C$, we have $S$ as a random subset of $S \cup \wt S_C$. Hence, using Hoeffding's inequality for sampling without replacement (\lemref{lem:hoeffding_sampling}), we have with probability at least $1-\delta$
% \begin{align}
%     \error_{\wt \calS_c} (\wh f)- \error_{\calS \cup \wt \calS_C}( \wh f) \le  \sqrt{\frac{\log(1/\delta)}{2m_2}} \,.
% \end{align}
% Re-writing $\error_{\calS \cup \wt \calS_C}( \wh f)$ as $\frac{m_2}{m_2 + n} \error_{\wt \calS_C }(\wh f) + \frac{n}{m_2 + n} \error_{\calS }(\wh f)$, we have with probability at least $1-\delta$
% \begin{align}
%   \left(\frac{n}{n+m_2}\right) \left(\error_{\wt \calS_c} (\wh f)- \error_{\calS}( \wh f) \right) \le  \sqrt{\frac{\log(1/\delta)}{2m_2}} \,.
% \end{align}
% As before, assuming $2m_2 \approx m$, we have with probability at least $1-\delta$ 
% \begin{align}
%     \error_{\wt \calS_c} (\wh f)- \error_{\calS}( \wh f) \le \left(1+\frac{m_2}{n}\right)  \sqrt{\frac{\log(1/\delta)}{m}} \le 1.5 \sqrt{\frac{\log(1/\delta)}{m}} \,. \label{eq:lemma3_final}
% \end{align} 
% \end{proof}

% \begin{proof}[Proof of \thmref{thm:error_ERM}] 
%     Having established these core intermediate results, we can now combine above three lemmas to prove the main result. 
%     In particular, we bound the population error on clean data ($\error_\calD(\wh f)$) as follows:  
%     \begin{enumerate}[(i)]
%         \item First, use \eqref{eq:lemma1_final}, to obtain an upper bound on the population error on clean data, i.e., with probability at least $1-\delta/4$, we have
%         \begin{align}
%             \error_{ \calD} (\wh f) \le 1 - \error_{ \wt \calS_M}(\wh f) + \sqrt{\frac{\log(4/\delta)}{m}} \,. 
%         \end{align}
%         \item  Second, use \eqref{eq:lemma2_final}, to relate the error on the mislabeled fraction with error on clean portion of randomly labeled data and error on whole randomly labeled dataset, i.e., with probability at least $1-\delta/2$, we have 
%         \begin{align}
%             - \error_{\wt S_M}(f) \le \error_{\wt S_C}(f) - 2\error_{\wt S}  + \sqrt{\frac{\log(4/\delta)}{2m}}  \,. 
%         \end{align} 
%         \item Finally, use \eqref{eq:lemma3_final} to relate the error on the clean portion of randomly labeled data and error on clean training data, i.e., with probability $1-\delta/4$, we have 
%         \begin{align}
%             \error_{\wt \calS_C} (\wh f)\le - \error_{\calS}( \wh f) + \left(1 + \frac{m}{2n} \right) \sqrt{\frac{\log(4/\delta)}{m}} \,. 
%         \end{align} 
%     \end{enumerate}

%     Using union bound on the above three steps, we have with probability at least $1-\delta$: 
%     \begin{align}
%         \error_\calD (\wh f) \le \error_{\calS}(\wh f)   + 1 - 2\error_{\wt \calS}(\wh f)   + (1/\sqrt{2} + 2.5)  \sqrt{\frac{\log(4/\delta)}{m}} \,.
%     \end{align}
%     Note that $(1/\sqrt{2} + 2.5)$ is a loose constant. In experiments, we use the ratio $\frac{m}{n}$
%     %  the exact error $\error_{\wt \calS}(\wh f)$ 
%     to evaluate R.H.S.    
% \end{proof}

% \subsection{Proof of \propref{prop:rademacher}}

% \begin{proof}[Proof of \propref{prop:rademacher}]
%     For a classifier $ f: \calX \to \{-1, 1\}$, we have $1 - 2\,\indict{ f(x) \ne y} = y \cdot f(x)$. Hence, by definition of $\error$, we have 
%     \begin{align}
%         1 -2\error_{\wt \calS}(f) = \frac{1}{m}\sum_{i=1}^m y_i \cdot f(x_i) \le \sup_{f \in \calF} \, \frac{1}{m} \sum_{i=1}^m y_i \cdot f(x_i)  \,. \label{eq:error_rademacher}
%     \end{align}
%     Note that for fixed inputs $(x_1, x_2, \ldots, x_m)$ in $\wt S$, $(y_1, y_2, \ldots y_m)$ are random labels. Define $\phi_1 (y_1, y_2, \ldots, y_m) \defeq \sup_{f \in \calF} \, \frac{1}{m} \sum_{i=1}^m y_i \cdot f(x_i)$. We have the following bounded difference condition on $\phi_1$. For all i, 
%     \begin{align}
%         \sup_{y_1, \ldots y_m, y_i^\prime \in \{-1, 1\}^{m+1} } \abs{ \phi_1 (y_1,\ldots, y_i, \ldots, y_m) - \phi_1 (y_1,\ldots, y_i^\prime, \ldots, y_m)  } \le 1/m \,. \label{cond1_rademacher}
%     \end{align} 
    
%     Similarly define $\phi_2 (x_1, x_2, \ldots, x_m) \defeq \Expt{ y_i \sim_U \{-1, 1\}  }{ \sup_{f \in \calF} \, \frac{1}{m}  \sum_{i=1}^m y_i \cdot f(x_i)}$. We have the following bounded difference condition on $\phi_2$. For all i,
%     \begin{align}
%         \sup_{x_1, \ldots x_m, x_i^\prime \in \calX^{m+1} } \abs{ \phi_2 (x_1,\ldots, x_i, \ldots, x_m) - \phi_1 (x_1,\ldots, x_i^\prime, \ldots, x_m)  } \le 1/m \,. \label{cond2_rademacher}
%     \end{align}
%     Using McDiarmid’s inequality (\lemref{lem:McDiarmid}) twice with Condition \eqref{cond1_rademacher} and \eqref{cond2_rademacher}, with probability at least $1-\delta$, we have
%     \begin{align}
%         \sup_{f \in \calF} \, \frac{1}{m} \sum_{i=1}^m y_i \cdot f(x_i)  - \Expt{x,y}{\sup_{f \in \calF} \, \frac{1}{m} \sum_{i=1}^m y_i \cdot f(x_i) } \le \sqrt{\frac{2\log(2/\delta)}{m}} \label{eq:final_rademacher}
%     \end{align} 
%     Combining \eqref{eq:error_rademacher} and \eqref{eq:final_rademacher}, we obtain the desired result. 
% \end{proof}


% \subsection{Proof of \thmref{thm:error_regularized_ERM}}

% Proof of \thmref{thm:error_regularized_ERM} follows similar to the proof of \thmref{thm:error_ERM}. Note that the same results in \lemref{lem:fit_mislabeled}, \lemref{lem:mislabeled_error}, and \lemref{lem:clear_error} hold in the regularized ERM case. However, the arguments in the proof of \lemref{lem:fit_mislabeled} changes slightly. Hence, we state and prove a lemma parallel to \lemref{lem:fit_mislabeled} for completeness. 

% \begin{lemma} \label{lem:lemma1_reg}
%     Assume the same setup as \thmref{thm:error_regularized_ERM}. 
%     Then for any $\delta >0$, with probability at least  $1-\delta$ 
%     over the random draws of mislabeled data $\wt S_M$, we have 
%     \begin{align}
%         \error_\calD(\widehat f)  \le 1 -\error_{\wt \calS_M}(\widehat f) + \sqrt{\frac{\log(1/\delta)}{m}}\,. 
%     \end{align} 
% \end{lemma}
% \begin{proof}
%     The main idea of the proof remains the same, i.e. regard 
%     the clean portion of the data 
%     ($S \cup \wt S_C$) as fixed.   
%     Then, there exists a classifier $f^*$ 
%     that is optimal over draws 
%     of the mislabeled data $\wt S_M$. 

    
%     Formally, 
%     \begin{align}
%     f^* \defeq \argmin_{f \in \calF} \error_{\widecheck {\calD}} (f)  + \lambda R(f) \,, \label{eq:modified_ERM_reg}
%     \end{align}
%     where $$\widecheck \calD = \frac{n}{m+n} \calS + \frac{m_1}{m+n} \wt \calS_C  + \frac{m_2}{m+n}\calDm \,.$$ That is, $\widecheck \calD$ a combination of 
%     the \emph{empirical distribution} 
%     over correctly labeled data $S \cup \wt S_C$
%     % in $S\cup \wt S$ 
%     and the (population) distribution 
%     over mislabeled data $\calDm$.
%     Recall that 
%     \begin{align}
%     \wh f \defeq \argmin_{f \in \calF} \error_{\calS \cup \wt S} (f) + \lambda R(f) \,. \label{eq:orig_ERM_reg}
%     \end{align}
%     % 
%     % 
%     Since, $\widehat f$ minimizes 0-1 error 
%     on $S \cup \wt S$, using ERM optimality on \eqref{eq:orig_ERM},  
%     we have 
%     \begin{align}
%         \error_{\calS \cup \wt \calS}(\widehat f) + \lambda R(\wh f) \le \error_{
%             \calS \cup \wt \calS}(f^*) + \lambda R(f^*) \,.    \label{eq:step1_reg}
%     \end{align}
%     Moreover, since $f^*$ is independent of $\wt S_M$, using Hoeffding's bound,
%     % \footnote{For a fully rigorous argument,
%     % refer to the complete proof in App.~\ref{app:proof_erm}.} 
%     we have with probability at least $1-\delta$ that
%     \begin{align}
%       \error_{\wt \calS_M}(f^*) \le \error_{ \calDm}(f^*) +  \sqrt{\frac{\log(1/\delta)}{2 m_1}} \,. \label{eq:step2_reg} 
%     \end{align}
%     %$ 
%     %for some constant $c_1\le 1/2$. 
%     Finally, since $f^*$ is the optimal classifier on $\widecheck \calD$, 
%     we have 
%     \begin{align}
%         \error_{\widecheck \calD}(f^*) + \lambda R(f^*) \le \error_{\widecheck \calD}(\widehat f) + \lambda R(\wh f) \label{eq:step3_reg}
%     \end{align}
%      Now to relate \eqref{eq:step1_reg} and \eqref{eq:step3_reg}, we can re-write the \eqref{eq:step2_reg} as follows: 
%     \begin{align}
%         \error_{\calS \cup \wt\calS}(f^*) \le \error_{ \widecheck \calD}(f^*) +  \frac{m_1}{m+n}\sqrt{\frac{\log(1/\delta)}{2 m_1}} \,. \label{eq:step4_reg} 
%     \end{align}
%     After adding $\lambda R(f^*)$ on both sides in \eqref{eq:step4_reg}, we combine equations \eqref{eq:step1_reg}, \eqref{eq:step4_reg}, and \eqref{eq:step3_reg}, to get 
%     \begin{align}
%         \error_{\calS \cup \wt \calS}(\wh f) \le \error_{\widecheck \calD}(\wh f) +  \frac{m_1}{m+n}\sqrt{\frac{\log(1/\delta)}{2 m_1}} \,, 
%     \end{align}
%     which implies 
%     \begin{align}
%         \error_{ \wt \calS_M}(\wh f) \le \error_{\calDm}(\wh f) + \sqrt{\frac{\log(1/\delta)}{2 m_1}} \,. \label{eq:lemma_reg_final}
%     \end{align}
%     Similar as before, since $\wt S$ is obtained by randomly labeling an unlabeled dataset, we assume 
%     $2m_1 \approx m$. Moreover, using $\error_{\calDm} = 1 - \error_{\calD}$ we obtain the desired result. 
% \end{proof}
% % \begin{proof}[Proof of ]
    
% % \end{proof}

% \subsection{Proof of \thmref{thm:multiclass_ERM}}

% We first state and prove lemmas parallel to three lemmas used in the proof of balanced binary case. Then we combine the results in the three lemmas to obtain the result in \thmref{thm:multiclass_ERM}. 

% Before stating the result, we define mislabeled distribution $\calDm$ for any $\calD$. While $\calDm$ and $\calD$ share 
% the same marginal distribution over $\calX$, 
% the distribution over labels $y$ 
% given an input $x\sim \calD_\calX$ is changed.
% In particular, for any $x$, the pdf over $y$ is changed to:  
% $p_{\calDm} (\cdot \vert x) \defeq \frac{1 - p_{\calD}(\cdot \vert x)}{k - 1}$.

% \begin{lemma} \label{lem:fit_mislabeled_multi}
%     Assume the same setup as \thmref{thm:multiclass_ERM}. 
%     Then for any $\delta >0$, with probability at least  $1-\delta$ 
%     over the random draws of mislabeled data $\wt S_M$, we have 
%     \begin{align}
%         \error_\calD(\widehat f)  \le (k-1)\left(1 -\error_{\wt \calS_M}(\widehat f)\right) + (k-1)\sqrt{\frac{\log(1/\delta)}{m}}\,. \label{eq:lemma1_multi}
%     \end{align}   
% \end{lemma} 

% \begin{proof}
%     The main idea of the proof remains the same, i.e. regard 
%     the clean portion of the data 
%     ($S \cup \wt S_C$) as fixed. 
%     Then, there exists a classifier $f^*$ 
%     that is optimal over draws 
%     of the mislabeled data $\wt S_M$. 
    
%     However, we need to be careful while relating population error on mislabeled data with population accuracy on clean data.   
%     While for binary classification,  we could upper bound $\error_{\wt \calS_M}$ 
%     with $1-\error_\calD$  (in the proof of \lemref{lem:fit_mislabeled}), 
%     for multiclass classification, 
%     error on the mislabeled data 
%     and accuracy on the clean data 
%     in the population 
%     are not so directly related.  
%     To establish \eqref{eq:lemma1_multi},
%     we break the error on the 
%     (unknown) mislabeled data 
%     into two parts: one term corresponds 
%     to predicting the true label on mislabeled data, 
%     and the other corresponds to predicting 
%     neither the true label 
%     nor the assigned (mis-)label.  
%     Finally, we relate these errors to their
%     population counterparts to establish \eqref{eq:lemma1_multi}. 
    
%     Formally, 
%     \begin{align}
%     f^* \defeq \argmin_{f \in \calF} \error_{\widecheck {\calD}} (f)  + \lambda R(f) \,, \label{eq:modified_ERM_reg2}
%     \end{align}
%     where $$\widecheck \calD = \frac{n}{m+n} \calS + \frac{m_1}{m+n} \wt \calS_C  + \frac{m_2}{m+n}\calDm \,.$$ That is, $\widecheck \calD$ a combination of 
%     the \emph{empirical distribution} 
%     over correctly labeled data $S \cup \wt S_C$
%     % in $S\cup \wt S$ 
%     and the (population) distribution 
%     over mislabeled data $\calDm$.
%     Recall that 
%     \begin{align}
%     \wh f \defeq \argmin_{f \in \calF} \error_{\calS \cup \wt S} (f) + \lambda R(f) \,. \label{eq:orig_ERM_reg2}
%     \end{align}
%     % 
%     % 
%     Following the exact steps from the proof of \lemref{lem:lemma1_reg}, with probability at least $1-\delta$, we have  
%     \begin{align}
%         \error_{ \wt \calS_M}(\wh f) \le \error_{\calDm}(\wh f) + \sqrt{\frac{\log(1/\delta)}{2 m_1}} \,. \label{eq:lemma1_final_multi_prev}
%     \end{align}
%     Similar to before, since $\wt S$ is obtained by randomly labeling an unlabeled dataset, we assume 
%     $\frac{k}{k-1} m_1 \approx m$. 
    
%     Now we will relate $\error_\calDm (\wh f)$ with $\error_{\calD}(\wh f)$. Let $y^T$ denote the (unknown) true label for a mislabeled point $(x, y)$ (i.e., label before replacing it with a mislabel). 
%     \begin{align}    
%          \Expt{(x, y) \in \sim \calDm}{\indict{ \wh f(x) \ne y }}  &= \underbrace{\Expt{(x, y) \in \sim \calDm}{\indict{ \wh f(x) \ne y \land \wh f(x) \ne y^T}}}_{\RN{1}} + \underbrace{\Expt{(x, y) \in \sim \calDm}{\indict{ \wh f(x) \ne y \land \wh f(x) = y^T}}}_{\RN{2}} \,. \label{eq:excess_term}
%     \end{align}
%     Clearly, term 2 is one minus the accuracy on the clean unseen data, i.e. 
%     \begin{align}
%         \RN{2} = 1 - \Expt{{x,y} \sim \calD}{ \indict{ \wh f(x) \ne y}} = 1- \error_{\calD}(\wh f) \,. \label{eq:term1}    
%     \end{align}
%     Next, we  relate term 1 with the error on the unseen clean data. We show that term 1 is equal to the error on the unseen clean data scaled by $\frac{k-2}{k-1}$ where $k$ is the number of labels. Using the definition of mislabeled distribution $\calDm$,  we have 
%     \begin{align}
%         \RN{1} = \frac{1}{k-1} \left( \Expt{(x, y) \in \sim \calD}{ \sum_{i \in \calY \land i\ne y}  \indict{ \wh f(x) \ne i \land \wh f(x) \ne y}} \right) = \frac{k-2}{k-1} \error_{\calD}(\wh f) \,.\label{eq:term2}
%     \end{align}    

%     Combining the result in \eqref{eq:term1}, \eqref{eq:term2} and \eqref{eq:excess_term}, we have 
%     \begin{align}
%         \error_{\calDm}(\wh f) = 1- \frac{1}{k-1} \error_{\calD}(\wh f) \,.\label{eq:combine_terms}
%     \end{align}
%     Finally, combining the result in \eqref{eq:combine_terms} with equation \eqref{eq:lemma1_final_multi_prev}, we have with probability $1-\delta$, 
%     \begin{align}
%       \error_{\calD}(\wh f) \le  (k-1) \left( 1- \error_{ \wt \calS_M}(\wh f) \right)  + (k-1) \sqrt{\frac{k \log(1/\delta)}{ 2(k-1)m}} \,. \label{eq:lemma1_final_multi}
%     \end{align}
% \end{proof}

% \begin{lemma} \label{lem:mislabeled_error_multi}
%     Assume the same setup as \thmref{thm:multiclass_ERM}.  Then for any $\delta >0$, with probability at least $1-\delta$ over the random draws of $\wt S$, we have  
%     % \begin{align}
%         $$\abs{k\error_{\wt \calS}(\widehat f) - \error_{\wt \calS_C}(\widehat f) -  (k-1)\error_{\wt \calS_M}(\widehat f) } \le  2k\sqrt{\frac{\log(4/\delta)}{2m}}\,. $$ % \label{eq:lemma2}
%     % \end{align}   
%     %  for some constant $c_3 \le 1.0\,$.
% \end{lemma} 


% \begin{proof}
%     Recall $\error_{\wt S} (f) = \frac{m_1}{m} \error_{\wt S_M}(f) + \frac{m_2}{m} \error_{\wt S_C}(f)$. Hence, we have 
%     \begin{align}
%         k\error_{\wt S}(f) - (k-1)\error_{\wt S_M}(f) - \error_{\wt S_C}(f) &= (k-1)\left(\frac{k m_1}{(k-1) m} \error_{\wt S_M}(f) - \error_{\wt S_M}(f)\right) + \left(\frac{km_2}{m} \error_{\wt S_C}(f) - \error_{\wt S_C}(f)\right) \\ &= k \left[ \left(\frac{m_1}{m} - \frac{k-1}{k}\right) \error_{\wt S_M}(f) + \left(\frac{m_2}{m} - \frac{1}{k} \right) \error_{\wt S_C} (f) \right] \,.
%     \end{align} 
%     Since the dataset is randomly labeled, we have with probability at least $1-\delta$, $\left(\frac{m_1}{m} - \frac{k-1}{k}\right) \le \sqrt{\frac{\log(1/\delta)}{2m}}$. Similarly, we have with probability at least $1-\delta$, $\left(\frac{m_2}{m} - \frac{1}{k}\right) \le \sqrt{\frac{\log(1/\delta)}{2m}}$. Using union bound, we have with probability at least $1-\delta$
%     % \begin{align}
%     %     2\error_{\wt S} - \error_{\wt S_M}(f) - \error_{\wt S_C}(f) \le \sqrt{\frac{\log(2/\delta)}{2m}} \left(\error_{\wt S_M}(f) + \error_{\wt S_C}(f) \right) \le 2\sqrt{\frac{\log(2/\delta)}{2m}} \,. \label{eq:lemma2_final}
%     % \end{align}
%     \begin{align}
%         k\error_{\wt S}(f) - (k-1)\error_{\wt S_M}(f) - \error_{\wt S_C}(f)  \le k \sqrt{\frac{\log(2/\delta)}{2m}} \left(\error_{\wt S_M}(f) + \error_{\wt S_C}(f) \right) \,. \label{eq:lemma2_final_multi}
%     \end{align}

%     % We obtain the desired result by using 
% \end{proof}

% \begin{lemma} \label{lem:clear_error_multi}
%     Assume the same setup as \thmref{thm:multiclass_ERM}. 
%     Then for any $\delta >0$, with probability at least $1-\delta$ 
%     over the random draws of $\wt S_C$ and $S$, we have 
%     % \begin{align}
%         $$\abs{\error_{\wt \calS_C}(\widehat f) - \error_{\calS}(\widehat f) } \le 1.5 \sqrt{\frac{k\log(2/\delta)}{2m}}\,.$$ %\label{eq:lemma3}
%     % \end{align}   
%     % for some constant $c_2 \le 1.2\,$.
% \end{lemma} 
% \begin{proof}
%     % Recall 0-1 error on each point  $(x,y) \in S \cup \wt S$ is given by $\I{ f(x)\ne y}$.
%     In the set of correctly labeled points $S \cup \wt S_C$, we have $S$ as a random subset of $S \cup \wt S_C$. Hence, using Hoeffding's inequality for sampling without replacement (\lemref{lem:hoeffding_sampling}), we have with probability at least $1-\delta$
%     \begin{align}
%         \error_{\wt \calS_c} (\wh f)- \error_{\calS \cup \wt \calS_C}( \wh f) \le  \sqrt{\frac{\log(1/\delta)}{2m_2}} \,.
%     \end{align}
%     Re-writing $\error_{\calS \cup \wt \calS_C}( \wh f)$ as $\frac{m_2}{m_2 + n} \error_{\wt \calS_C }(\wh f) + \frac{n}{m_2 + n} \error_{\calS }(\wh f)$, we have with probability at least $1-\delta$
%     \begin{align}
%       \left(\frac{n}{n+m_2}\right) \left(\error_{\wt \calS_c} (\wh f)- \error_{\calS}( \wh f) \right) \le  \sqrt{\frac{\log(1/\delta)}{2m_2}} \,.
%     \end{align}
%     As before, assuming $km_2 \approx m$, we have with probability at least $1-\delta$ 
%     \begin{align}
%         \error_{\wt \calS_c} (\wh f)- \error_{\calS}( \wh f) \le \left(1+\frac{m_2}{n}\right)  \sqrt{\frac{k\log(1/\delta)}{2m}} \le \left( 1 + \frac{1}{k}\right) \sqrt{\frac{k\log(1/\delta)}{2m}} \,. \label{eq:lemma3_final_multi}
%     \end{align} 
% \end{proof}

% \begin{proof}[Proof of \thmref{thm:multiclass_ERM}] 
%     Having established these core intermediate results, we can now combine above three lemmas. 
%     In particular, we bound the population error on clean data ($\error_\calD(\wh f)$) as follows:  
%     \begin{enumerate}[(i)]
%         \item First, use \eqref{eq:lemma1_final_multi}, to obtain an upper bound on the population error on clean data, i.e., with probability at least $1-\delta/4$, we have
%         \begin{align}
%             \error_{ \calD} (\wh f) \le (k-1)\left(1 - \error_{ \wt \calS_M}(\wh f) \right) + (k-1) \sqrt{\frac{k\log(4/\delta)}{2(k-1)m}} \,. 
%         \end{align}
%         \item  Second, use \eqref{eq:lemma2_final_multi}, to relate the error on the mislabeled fraction with error on clean portion of randomly labeled data and error on whole randomly labeled dataset, i.e., with probability at least $1-\delta/2$, we have 
%         \begin{align}
%             - (k-1)\error_{\wt S_M}(f) \le \error_{\wt S_C}(f) - k\error_{\wt S}  + k\sqrt{\frac{\log(4/\delta)}{2m}}  \,. 
%         \end{align} 
%         \item Finally, use \eqref{eq:lemma3_final_multi} to relate the error on the clean portion of randomly labeled data and error on clean training data, i.e., with probability $1-\delta/4$, we have 
%         \begin{align}
%             \error_{\wt \calS_C} (\wh f)\le - \error_{\calS}( \wh f) + \left(1 + \frac{m}{kn} \right) \sqrt{\frac{k\log(4/\delta)}{2m}} \,. 
%         \end{align} 
%     \end{enumerate}

%     Using union bound on the above three steps, we have with probability at least $1-\delta$: 
%     \begin{align}
%         \error_\calD (\wh f) \le \error_{\calS}(\wh f) + (k-1) - k\error_{\wt \calS}(\wh f)   + (\sqrt{k(k-1)} + k + \sqrt{k} + \frac{m}{n\sqrt{k}})  \sqrt{\frac{\log(4/\delta)}{2m}} \,.
%     \end{align}
%     % Note that $\frac{m}{n\sqrt{k}}$ is much smaller than the other terms in addition. Hence, we ignore this in the final bound. 
%     % Note that $(1/\sqrt{2} + 2.5)$ is a loose constant. In experiments, we use the ratio $\frac{m}{n}$
%     %  the exact error $\error_{\wt \calS}(\wh f)$ 
%     % to evaluate R.H.S.    
% \end{proof}

% \newpage
% \section{Proofs from \secref{sec:linear_models}}\label{app:proof_gd}

% We suppose that the parameters of the linear function 
% are obtained via gradient descent on 
% the following $L_2$ regularized problem: 
% \begin{align}
%     % n in denominator is avoided deliberately
%     \calL_S(w; \lambda) \defeq \sum_{i=1}^n{(w^Tx_i - y_i)^2} + \lambda \norm{w}{2}^2 \,, \label{eq:l2_MSE_app}   
% \end{align}
% where $\lambda\ge0$ is a regularization parameter. 
% We assume access to a clean dataset 
% $S = \{(x_i, y_i)\}_{i=1}^n \sim \calD^n$ 
% and randomly labeled dataset 
% $\wt S = \{(x_i, y_i)\}_{i=n+1}^{n+m} \sim \wt \calD^m$. 
% Let $\bX = [x_1, x_2, \cdots, x_{m+n}]$ 
% and $\by = [y_1, y_2, \cdots, y_{m+n}]$. 
% Fix a positive learning rate $\eta$ such that 
% $\eta \le 1/\left(\norm{\bX^T\bX}{\text{op}} + \lambda^2\right)$ 
% and an initialization $w_0 = 0$. 
% % \todos{Assumption made for simplicty}. 
% Consider the following gradient descent iterates 
% to minimize objective \eqref{eq:l2_MSE_app} on $S \cup \wt S$:
% \begin{align}
% w_t = w_{t-1} - \eta \grad_w \calL_{S \cup \wt S} (w_{t-1}; \lambda) \quad \forall t=1,2,\ldots \label{eq:GD_iterates_app}
% \end{align} 
% Then we have $\{ w_t\}$ converge to the limiting solution 
% $\wh w = \left( \bX^T\bX+\lambda \boldsymbol{I}\right)^{-1}\bX^T\by$. Define $\widehat f (x) \defeq f(x ; \wh w) $.  

% \subsection{\textcolor{red}{Errata}}

% We wish to correct the following error in the body: \codref{cond:error_stability} is not enough to guarantee the result in \thmref{thm:linear}. We now present a slightly stronger condition called \emph{hypothesis stability} under which we obtain a result similar to \thmref{thm:linear}. 

% This error doesn't change the main arguments of the proof where we show that the empirical train error is less than or equal to the leave-one-out error. We need a stronger condition to relate leave-one-out error with the population error of the original classifier. Specifically, while \codref{cond:error_stability} relates the average population error of leave-one-out classifiers with the population error of the original classifier, we need the new condition to show the concentration of the empirical leave-one-out error and  average population error of leave-one-out classifiers. 
% % main takeaway 

% Note that the new condition, while being stronger than the previous one, still doesn't imply generalization~\cite{bousquet2002stability,elisseeff2003leave,abou2019exponential}. Overall, the main results in \secref{sec:ERM_training} and takeaways of the paper remain unaffected by the error.  

% We now present the new condition and a corrected statement of \thmref{thm:linear}. Recall, for a given training set $S \sim \calD^n $, 
% we use $S_{(i)}$ to denote the training set $S$ 
% with the $i^{\text{th}}$ point removed.

% \begin{condition}[Hypothesis Stability] 
%     \label{cond:hypothesis_stability}
%     We have $\beta$ hypothesis stability 
%     if our training algorithm $\calA$ satisfies the following: 
%     \begin{align*}
%     % ${\sum_{i=1}^n \frac{\error_{\calD}( f(\calA, S_{(i)}))}{n} - \error_\calD(f(\calA, S))} \le \beta\,$.
%     \forall i \in \{1,2,\ldots, n\}, \quad  \Expt{\calS, (x,y) \in \calD}{ \abs{\error\left( f(x) ,y  \right) - \error\left( f_{(i)}(x), y \right) }} \le \frac{\beta}{n} \,,
%     \end{align*}
%     where $f_{(i)} \defeq f(\calA, S_{(i)})$ and $ f \defeq f(\calA, S)$.
% \end{condition}

% \begin{theorem}[Correct statement of \thmref{thm:linear}] \label{thm:new_linear}
%     Assume that this gradient descent algorithm satisfies \codref{cond:hypothesis_stability}
%     with $\beta=\calO(1)$.  
%     Then for any $\delta >0$, with probability at least $1-\delta$ 
%     over the random draws of datasets $\wt S$ and $S$, we have:
%     \begin{align}
%         \error_\calD(\widehat f) \le \error_\calS(\widehat f) + 1 - 2 \error_{\wt\calS}(\widehat f) + \left(\frac{1}{\sqrt{2}} + 1.5 \right) \sqrt{\frac{\log(4/\delta)}{m}} + \sqrt{\frac{4}{\delta}\left(\frac{1}{m} +\frac{3\beta}{m+n} \right)}  \,. \label{eq:gd_error}
%     \end{align} 
%     % for some constant $c\le 3.2$.
% \end{theorem}

% \subsection{Proof of \thmref{thm:new_linear}}
% We use a standard result from linear algebra, namely Shermann-Morrison formula~\citep{sherman1950adjustment} for matrix inversion:  

% \begin{lemma}[\citet{sherman1950adjustment}] \label{lem:sherman}
%     Suppose $\bA \in \Real^{n \times n}$ is an invertible square matrix and $u,v \in \Real^n$ are column vectors. Then $\bA + uv^T$ is invertible iff $1 + v^T \bA u \ne 0$ and in particular
%     \begin{align}
%         (\bA + u v^T)^{-1} = \bA^{-1}  - \frac{\bA^{-1} uv^T \bA^{-1} }{ 1 + v^T \bA^{-1} u} \,.
%     \end{align}   
% \end{lemma}
% \newcommand\byy[1]{\by_{\left(#1\right)}}
% \newcommand\bXX[1]{\bX_{\left(#1\right)}}
% \newcommand\ff[1]{\wh f_{\left(#1\right)}}

% For a given training set $S \cup \wt S_C$, define leave-one-out error on mislabeled points in the training data as $$\error_{\text{LOO}(\wt S_M) } = \frac{\sum_{(x_i, y_i) \in \wt S_M} \error( f_{(i)}( x_i), y_i)}{ \abs{\wt S_M }} \,, $$
% where $f_{(i)} \defeq f(\calA, (S \cup \wt S)_{(i)})$. To relate empirical leave-one-out error and population error with hypothesis stability condition, we use the following lemma:   

% \begin{lemma}[\citet{bousquet2002stability}] \label{lem:stability_error}
%     For the leave-one-out error, we have
%     \begin{align}
%         \Expo{ \left( \error_{\calDm}(\wh f) -\error_{\text{LOO}(\wt S_M) } \right)^2 } \le \frac{1}{2m_1}+  \frac{3\beta}{n + m}\,.
%     \end{align}   
%     % where $ f \defeq f(\calA, S \cup \wt S) $.
% \end{lemma}

% Proof of the above lemma is similar to the proof of  Lemma 9 in \citet{bousquet2002stability} and can be found in \appref{app:proof_lem_error}. 
% % 
% % Before presenting the result, we introduce some notation. 
% Before presenting the proof of \thmref{thm:new_linear}, we introduce some more notation. Let $\bX_{(i)}$ denote the matrix of covariates with $i^{\text{th}}$ point removed. Similarly let $\by_{(i)}$ be the array of responses with $i^{\text{th}}$ point removed. Define the corresponding regularized GD solution as $\wh w_{(i)} = \left( \bXX{i}^T\bXX{i}+\lambda \boldsymbol{I}\right)^{-1}\bXX{i}^T\byy{i}$. Define $\ff{i}(x) \defeq f(x ; \wh w_{(i)}) $.

% \begin{proof}[Proof of \thmref{thm:new_linear}]
%     Because squared loss minimization does not imply 0-1 error minimization, we cannot use arguments from \lemref{lem:fit_mislabeled}. This is the main technical difficulty. To compare the 0-1 error at a train point with an unseen point, 
%     we use the closed-form expression for $\widehat{w}$ and Shermann-Morrison formula to upper bound training error with leave-one-out cross validation error. 
    
%     The proof is divided into three parts: In part one, we show that 0-1 error on mislabeled points in the training set is lower than the error obtained by leave-one-out error at those points. In part two, we relate this leave-one-out error with the population error on mislabeled distribution using \codref{cond:hypothesis_stability}. While the empirical leave-one-out error is unbiased estimator of the average population error of leave-one-out classifiers, we need hypothesis stability to control the variance of empirical leave-one-out error. Finally in part three, we show that the error on the mislabeled training points can be estimated with just the randomly labeled and  clean training data (as in proof of \thmref{thm:error_ERM}).  

%     \textbf{Part 1 {} {}} First we relate training error with leave-one-out error.        
%     For any 
%     training point $(x_i, y_i)$ in $\wt S \cup S$, we have 
%     \begin{align}
%         \error(\wh f(x_i), y_i ) &= \indict{ y_i \cdot x_i^T \wh w < 0 } = \indict{ y_i \cdot x_i^T \left( \bX^T\bX+\lambda \boldsymbol{I}\right)^{-1}\bX^T\by < 0 } \\
%         &= \indict{ y_i \cdot x_i^T \underbrace{\left( \bXX{i}^T\bXX{i} + x_i ^T x_i +\lambda \boldsymbol{I}\right)^{-1}}_{\RN{1}} (\bXX{i}^T\byy{i} + y \cdot x_i) < 0 }
%     \end{align}
%     Letting $\bA = \left(\bXX{i}^T\bXX{i} +\lambda \boldsymbol{I}\right)$ and using \lemref{lem:sherman} on term 1, we have 
%     \begin{align}
%         \error(\wh f(x_i), y_i ) &= \indict{ y_i \cdot x_i^T \left[\bA^{-1} -  \frac{\bA^{-1} x_i x_i^T \bA^{-1}}{ 1 + x_i ^T \bA^{-1} x_i } \right] (\bXX{i}^T\byy{i} + y \cdot x_i) < 0 } \\
%         &= \indict{ y_i \cdot\left[ \frac{ x_i^T \bA^{-1} ( 1 + x_i ^T \bA^{-1} x_i ) -  x_i^T \bA^{-1} x_i x_i^T \bA^{-1}}{ 1 + x_i ^T \bA ^{-1}x_i } \right] (\bXX{i}^T\byy{i} + y \cdot x_i) < 0 } \\
%         &= \indict{ y_i \cdot\left[ \frac{ x_i^T \bA^{-1}}{ 1 + x_i ^T \bA ^{-1}x_i } \right] (\bXX{i}^T\byy{i} + y \cdot x_i) < 0 } \,.
%     \end{align}

%     Since $1 + x_i^T \bA^{-1} x_i > 0$, we have 
%     \begin{align}
%         \error(\wh f(x_i), y_i ) &= \indict{ y_i \cdot x_i^T \bA^{-1} (\bXX{i}^T\byy{i} + y \cdot x_i) < 0 } \\
%         &= \indict{ x_i^T \bA^{-1} x_i +  y_i \cdot x_i^T \bA^{-1} (\bXX{i}^T\byy{i}) < 0 } \\
%         &\le \indict{ y_i \cdot x_i^T \bA^{-1} (\bXX{i}^T\byy{i}) < 0 } = \error(\ff{i}(x_i), y_i ) \,.\label{eq:LOO_error}
%     \end{align}

%     Using \eqref{eq:LOO_error}, we have 
%     \begin{align}
%         \error_{\wt \calS_M } (\wh f) \le \error_{\text{LOO} (S_M)} \defeq \frac{\sum_{(x_i, y_i) \in \wt S_M} \error(\ff{i}(x_i), y_i ) }{\abs{\wt \calS_M}}\label{eq:LOO_error_final}
%     \end{align}
%     \textbf{Part 2 {}{}} We now relate RHS in \eqref{eq:LOO_error_final} with the population error on mislabeled distribution. To do this, we leverage \codref{cond:hypothesis_stability} and \lemref{lem:stability_error}. In particular, we have 

%     \begin{align}
%         \Expt{\calS \cup \wt \calS_M }{ \left(\error_{\calDm}(\wh f) - \error_{\text{LOO} (S_M)}\right)^2 } \le \frac{1}{2m_1} + \frac{3\beta}{m+n} \,.
%     \end{align}

%     Using Chebyshev's inequality, with probability at least $1-\delta$, we have 
%     \begin{align}
%         \error_{\text{LOO} (S_M)} \le  \error_{\calDm}(\wh f)   + \sqrt{\frac{1}{\delta}\left(\frac{1}{2m_1} +\frac{3\beta}{m+n} \right)} \,. \label{eq:final_mislabeled_linear}
%     \end{align}
    

%     \textbf{Part 3 {}{}} Combining \eqref{eq:final_mislabeled_linear} and \eqref{eq:LOO_error_final}, we have 

%     \begin{align}
%         \error_{\wt \calS_M } (\wh f) \le \error_{\calDm}(\wh f)   + \sqrt{\frac{1}{\delta}\left(\frac{1}{2m_1} +\frac{3\beta}{m+n} \right)} \,. \label{eq:linear_parallel_lem1}
%     \end{align}

%     Compare \eqref{eq:linear_parallel_lem1}, with \eqref{eq:lemma1_final} in the proof of \lemref{lem:fit_mislabeled}. We obtain a similar relationship between $\error_{\wt \calS_M }$ and $\error_{\calDm}$ but with a polynomial concentration instead of exponential concentration. 
%     In addition, since we just use concentration arguments to relate mislabeled error with the error on clean portion and unlabeled portion, we can directly use the results in \lemref{lem:mislabeled_error} and \lemref{lem:clear_error}. Therefore, combining results in \lemref{lem:mislabeled_error}, \lemref{lem:clear_error}, and \eqref{eq:linear_parallel_lem1} with union bound, we have with probability at least $1-\delta$

%     \begin{align}
%         \error_\calD(\widehat f) \le \error_\calS(\widehat f) + 1 - 2 \error_{\wt\calS}(\widehat f) + \left(\frac{1}{\sqrt{2}} + 1.5 \right) \sqrt{\frac{\log(4/\delta)}{m}} + \sqrt{\frac{4}{\delta}\left(\frac{1}{m} +\frac{3\beta}{m+n} \right)}  \,.
%     \end{align}
    

       
% \end{proof}

% \subsection{Discussion on \codref{cond:hypothesis_stability}}

% The quantity in LHS of \codref{cond:hypothesis_stability} measures how much the function learned by the algorithm (in terms of error on unseen point) will change when one point in the training set is removed. 
% % Discussion on exponential concentration and stronger condition. 
% Notice that hypothesis stability implies error stability, i.e., \codref{cond:error_stability} ~\cite{bousquet2002stability}.  In summary, while error stability allowed us to relate the average population error of the leave-one-out classifiers with the population error of the original classifier, we need hypothesis stability condition to control the variance of the empirical leave-one-out error. 

% Additionally, we note that while the dominating term in the RHS of \thmref{thm:new_linear} matches with the dominating term in ERM bound in \thmref{thm:error_ERM}, there is a polynomial concentration term (dependence on $1/\delta$ instead of $\log(\sqrt{1/\delta})$) in  \thmref{thm:new_linear}. 
% Since with hypothesis stability, we just bound the variance,  the polynomial concentration is due to the use of Chebyshev's inequality instead of an exponential tail inequality (as in \lemref{lem:fit_mislabeled}).
% Recent works have highlighted that slightly stronger condition than hypothesis stability can be used to obtained an exponential concentration for leave-one-out error~\citep{abou2019exponential}, but we leave this for future work for now. 
% % We leave 
% % However, the constants 

% % we also want to highlight  

% \subsection{Formal statement and proof of  of \propref{prop:early_stop}}

% Before formally presenting the result, we will introduce some notation.  By $\calL_{S}(w)$, we denote 
% the objective in \eqref{eq:l2_MSE_app} with $\lambda=0$. 
% Assume Singular Value Decomposition (SVD) of $\bX$  as $\sqrt{n} \bU \bS^{1/2} \bV^T$. Hence $\bX^T \bX = \bV \bS \bV^T$.
% Consider the GD iterates defined in \eqref{eq:GD_iterates_app}. 
% % 
% We now derive closed form expression for the $t^\text{th}$ iterate of gradient descent:  
% % 
% \begin{align}
%     w_t = w_{t-1} + \eta \cdot \bX^T (\by - \bX w_{t-1}) = (\bI - \eta \bV \bS \bV^T )w_{k-1} + \eta \bX^T \by \,.
% \end{align}
% Rotating by $\bV^T$, we get 
% \begin{align}
%     \wt w_t = (\bI - \eta\bS )\wt w_{k-1} + \eta \wt \by \,, \label{eq:GD_recur}
% \end{align}
% where $\wt w_t = \bV^T w_t $ and $\wt \by = \bV^T \bX^T \by$. Assuming the initial point $w_0 = 0$ and applying the recursion in \eqref{eq:GD_recur}, we get
% \begin{align}
%     \wt w_t = \bS ^{-1} ( \bI - (\bI - \eta \bS)^k ) \wt \by \,, 
% \end{align} 
% Projecting solution back to the original space, we have 
% \begin{align}
%      w_t = \bV \bS ^{-1} ( \bI - (\bI - \eta \bS)^k ) \bV^T \bX^T \by \,, 
% \end{align} 
% % We will work with this GD solution at any iterate $t$ in the next proposition. 
% Define $f_t(x) \defeq f(x;w_t)$ as the solution at the $t^{\text{th}}$ iterate. 
% Let $\wt w_{\lambda} = \argmin_{w} \calL_\calS (w;\lambda) = (\bX^T \bX + \lambda \bI)^{-1} \bX^T \by = \bV (\bS + \lambda \bI )^{-1} \bV^T \bX^T \by $. 
% % ) \,,$ for all $t=1,2,\ldots\,.$ 
% and define $\wt f_\lambda(x) \defeq f(x;\wt w_\lambda)$ as the regularized solution. 
% Assume $\kappa$ be the condition number of the population covariance matrix 
% and 
% let $s_\text{min}$ be the minimum positive singular value of the empirical covariance matrix. Our proof idea is inspired from recent work on relating gradient flow solution and regularized solution for regression problems \citep{ali2018continuous}. We will use the following lemma in the proof: 
% \begin{lemma} \label{lem:ineq_soln}
%     For all $x \in [0,1]$ and for all $ k \in \mathbb{N}$, we have (a) $ \frac{kx}{1+kx} \le 1- (1-x)^k$ and (b) $ 1- (1-x)^k \le 2 \cdot \frac{kx}{kx+1} $.
%     %  where $g(c)$ is a constant dependent on $c$. For $c = 1$, $g(c) = 2.0$.   
% \end{lemma}
% \begin{proof}
%     % [Proof of \lemref{lem:ineq_soln}]
%     % Part (a) is easy. 
%     Using $ (1-x)^k \le \frac{1}{1+kx}$, we have part (a). For part (b), we numerically maximize $\frac{ (1+kx ) (1 - (1-x)^k) }{kx}$ for all $k\ge 1$ and for all $x \in [0, 1]$.  
% \end{proof}

% % 
% % Next, 

% \begin{prop}[Formal statement of \propref{prop:early_stop}] \label{prop:formal_early_stop}
% Let $\lambda = \frac{1}{t\eta}$. For a training point $x$, we have 
% \begin{align*}
%     \Expt{x \sim \calS}{(f_t(x) - \wt f_\lambda(x))^2} &\le c(t,\eta) \cdot \Expt{x \sim \calS}{f_t(x)^2} \,, %\label{eq:early_stop}
% \end{align*}
% where $c(t, \eta) \defeq \min( 0.25, \frac{1}{s_\text{min}^2 t^2 \eta^2})$. Similarly for a test point, we have 
% \begin{align*}
%     \Expt{x \sim \calD_\calX}{(f_t(x) - \wt f_\lambda(x))^2} &\le \kappa \cdot c(t,\eta) \cdot \Expt{x \sim \calD_\calX}{f_t(x)^2} \,. %\label{eq:early_stop}
% \end{align*}
% \end{prop} 

% \begin{proof}
%     %%%%%%%%%%%%% 
%     We want to analyze the expected squared difference output of regularized linear regression with regularization constant $\lambda = \frac{1}{\eta t}$ and gradient descent solution at $t^\text{th}$ iterate. We separately expand the algebraic expression for squared difference at a training point and a test point. 
%     % We start by considering the difference  
%     Then the main step is to show that  $\left[ \bS ^{-1} ( \bI - (\bI - \eta \bS)^k )  - (\bS + \lambda \bI )^{-1}\right] \preceq c(\eta, t) \cdot \bS ^{-1} ( \bI - (\bI - \eta \bS)^k ) $.

%     %%%%%%%%%%%%%
    
%   \textbf{Part 1 {} {}} 
%     First, we will analyze the squared difference of output at a training point (for simplicity, we refer to $S \cup \wt S$ as $S$), i.e. 
%     \begin{align}
%         \Expt{ x \sim \calS }{\left(f_t(x) - \wt f_\lambda (x)\right)^2} &= \norm{\bX w_t - \bX \wt w_\lambda}{2}^2 =   \norm{\bX \bV \bS ^{-1} ( \bI - (\bI - \eta \bS)^t ) \bV^T \bX^T \by - \bX \bV (\bS + \lambda \bI )^{-1} \bV^T \bX^T \by }{2}^2 \\
%         &= \norm{\bX \bV \left(\bS ^{-1} ( \bI - (\bI - \eta \bS)^t ) - (\bS + \lambda \bI )^{-1} \right) \bV^T \bX^T \by  }{2} \\
%         &=  \by^T \bV \bX \left( \underbrace{\bS ^{-1} ( \bI - (\bI - \eta \bS)^t ) - (\bS + \lambda \bI )^{-1}}_{\RN{1}} \right)^2 \bS \bV^T \bX^T \by \label{eq:train_GD_rel}
%         %  (\bX \bV \bS ^{-1} ( \bI - (\bI - \eta \bS)^k ) \bV^T \bX^T \by)^T \bX \bV \bS ^{-1} ( \bI - (\bI - \eta \bS)^k ) \bV^T \bX^T \by
%     \end{align}
%     We now separately consider term 1. Substituting $\lambda = \frac{1}{t \eta}$, we get
%     \begin{align}
%         \bS ^{-1} ( \bI - (\bI - \eta \bS)^t ) - (\bS + \lambda \bI )^{-1} &= \bS^{-1} \left( ( \bI - (\bI - \eta \bS)^t ) - (\bI + \bS^{-1} \lambda )^{-1}\right) \\
%         &= \underbrace{\bS^{-1} \left( ( \bI - (\bI - \eta \bS)^t ) - (\bI + ( \bS t \eta)^{-1}  )^{-1}\right)}_{\bA}
%     \end{align}

%     We now separately bound the diagonal entries in matrix $\bA$. 
%     With $s_i$, we denote $i^{\text{th}}$ diagonal entry of $\bS$. Note that since $ \eta\le 1/\norm{S}{\text{op}}$, for all $i$, $\eta s_i  \le 1$.  Consider $i^{\text{th}}$ diagonal term (which is non-zero) of the diagonal matrix $\bA$, we have 
%     \begin{align}
%         \bA_{ii} = \frac{1}{s_i} \left(  1 - (1 - s_i \eta)^t - \frac{t \eta s_i}{1 + t \eta s_i } \right) &=  \frac{1 - (1 - s_i \eta)^t}{s_i} \left( \underbrace{ 1 - \frac{t \eta s_i}{(1 + t \eta s_i)(1 - (1 - s_i \eta)^t)}}_{\RN{2}} \right) \\ 
%          &\le \frac{1}{2}\left[ \frac{1 - (1 - s_i \eta)^t}{ s_i} \right] \tag*{(Using \lemref{lem:ineq_soln} (b))} \,.
%     \end{align} 
%     Additionally, we can also show the following upper bound on term 2: 
%     \begin{align}
%          1 - \frac{t \eta s_i}{(1 + t \eta s_i)(1 - (1 - s_i \eta)^t)} &= \frac{(1 + t \eta s_i)(1 - (1 - s_i \eta)^t) - t \eta s_i }{(1 + t \eta s_i)(1 - (1 - s_i \eta)^t)} \\
%          & \le  \frac{ 1 -  (1 - s_i \eta)^t - t \eta s_i (1 - s_i \eta)^t}{(1 + t \eta s_i)(1 - (1 - s_i \eta)^t)} \\
%          & \le \frac{1}{t\eta s_i} \,. \tag{Using \lemref{lem:ineq_soln} (a)}
%         %  &\le \frac{1}{2}\left[ \frac{1 - (1 - s_i \eta)^t}{ s_i} \right] \tag*{(Using \lemref{lem:ineq_soln})} \,.
%     \end{align} 

%     Combining both the upper bounds on each diagonal entry $\bA_{ii}$, we have 
%     \begin{align}
%     \bA \preceq c_1(\eta, t) \cdot \bS^{-1} ( \bI - (\bI - \eta \bS)^t ) \,, \label{eq:upperbound_diagonal}
%     \end{align}
%     where $c_1(\eta, t ) = \min(0.5, \frac{1}{t s_i \eta })$. Plugging this into \eqref{eq:train_GD_rel}, we have 
%     \begin{align}
%         \Expt{ x \sim \calS }{\left(f_t(x) - \wt f_\lambda (x)\right)^2} &\le c(\eta, t) \cdot \by^T \bV \bX  \left( \bS^{-1} ( \bI - (\bI - \eta \bS)^t ) \right)^2 \bS \bV^T \bX^T \by \\
%         &=   c(\eta, t) \cdot \by^T \bV \bX  \left( \bS^{-1} ( \bI - (\bI - \eta \bS)^t ) \right) \bS \left( \bS^{-1} ( \bI - (\bI - \eta \bS)^t ) \right) \bV^T \bX^T \by \\
%         & =  c(\eta, t) \cdot \norm{\bX w_t}{2}^2 \\
%         &= c(\eta, t) \cdot  \Expt{ x \sim \calS }{\left(f_t(x) \right)^2} \,,
%     \end{align}
%     where $c(\eta, t ) = \min(0.25, \frac{1}{t^2 s^2_i \eta^2 })$.

%     \textbf{Part 2 {} {}} With $\bSigma$, we denote the underlying true covariance matrix. We now consider the squared difference of output at an unseen point: 
%     \begin{align}
%         \Expt{ x \sim \calD_{\calX} }{\left(f_t(x) - \wt f_\lambda (x)\right)^2} &= \Expt{x \sim \calD_{\calX}}{\norm{x^T w_t - x^T \wt w_\lambda}{2}} \\
%         &=   \norm{x^T \bV \bS ^{-1} ( \bI - (\bI - \eta \bS)^t ) \bV^T \bX^T \by - x^T \bV (\bS + \lambda \bI )^{-1} \bV^T \bX^T \by }{2} \\
%         &= \norm{x^T \bV \left(\bS ^{-1} ( \bI - (\bI - \eta \bS)^t ) - (\bS + \lambda \bI )^{-1} \right) \bV^T \bX^T \by  }{2} \\
%         &= \by^T \bV \bX \left( \bS ^{-1} ( \bI - (\bI - \eta \bS)^t ) - (\bS + \lambda \bI )^{-1} \right) \bV^T \bSigma \bV \\ &\qquad \qquad \qquad \qquad \qquad \left( (\bI - (\bI - \eta \bS)^t ) - (\bS + \lambda \bI )^{-1} \right) \bV^T \bX^T \by \\
%         &\le \sigma_{\text{max}} \cdot \by^T \bV \bX \left( \underbrace{\bS ^{-1} ( \bI - (\bI - \eta \bS)^t ) - (\bS + \lambda \bI )^{-1}}_{\RN{1}} \right)^2 \bV^T \bX^T \by \,, \label{eq:test_GD_rel}
%         %  (\bX \bV \bS ^{-1} ( \bI - (\bI - \eta \bS)^k ) \bV^T \bX^T \by)^T \bX \bV \bS ^{-1} ( \bI - (\bI - \eta \bS)^k ) \bV^T \bX^T \by
%     \end{align}
%     where $\sigma_{\text{max}}$ is the maximum eigenvalue of the underlying covariance matrix $\bSigma$. Using the upper bound on term 1 in \eqref{eq:upperbound_diagonal}, we have 
%     \begin{align}
%         \Expt{ x \sim \calD_{\calX} }{\left(f_t(x) - \wt f_\lambda (x)\right)^2} &\le \sigma_{\text{max}} \cdot c(\eta, t) \cdot \by^T \bV \bX  \left( \bS^{-1} ( \bI - (\bI - \eta \bS)^t ) \right)^2 \bV^T \bX^T \by \\
%         &=   \kappa \cdot c(\eta, t) \cdot \sigma_{\text{min}}\cdot \norm{\bV \left( \bS^{-1} ( \bI - (\bI - \eta \bS)^t ) \right) \bV^T \bX^T \by}{2}^2 \\
%         &\le \kappa \cdot c(\eta, t) \cdot \left[ \bV \left( \bS^{-1} ( \bI - (\bI - \eta \bS)^t ) \right) \bV^T \bX^T \right]^T \bSigma \\
%         &\qquad \qquad \qquad \qquad \qquad \left[ \bV \left( \bS^{-1} ( \bI - (\bI - \eta \bS)^t ) \right) \bV^T \bX^T \right] \by \\
%         & = \kappa \cdot c(\eta, t) \cdot \Expt{x \sim \calD_{\calX}}{\norm{x^T w_t}{2}} \,.
%     \end{align}
% % 
% % 
%     % Since $ \eta\le 1/\norm{S}{\text{op}}$, invoking \lemref{lem:ineq_soln} to upper bound term 1 with
% \end{proof}


% \newpage
% \section{Additional experiments and details}\label{app:exp}
% \newcommand\tab[1][1cm]{\hspace*{#1}}

% \subsection{Datasets} \label{sec:app_dataset}

% \textbf{Toy Dataset {} {}} Assume fixed constants $\mu$ and $\sigma$. For a given label $y$, we simulate features $x$ in our toy classification setup as follows: 
% \begin{align*}
%     x \defeq \texttt{concat} \left[ x_1, x_2\right] \quad \text{where} \quad  x_1 \sim  \calN( y \cdot \mu, \sigma^2 I_{d \times d}) \ \  \text{and} \ \  x_1 \sim  \calN( 0, \sigma^2 I_{d \times d}) \,.
% \end{align*}  
% % where $y$ is the true label and $x$ is the corresponding feature vector. 
% In experiements throughout the paper, we fix dimention $d=100$, $\mu = 1.0 $, and $\sigma = \sqrt{d}$. Intuitively, $x_1$ carries the information about the underlying label and $x_2$ is additional noise independent of the underlying label. 

% \textbf{CV datasets {} {}} We use MNIST~\citep{lecun1998mnist} and CIFAR10~\cite{krizhevsky2009learning}. 
% % For binary tasks, 
% We produce a binary variant from the multiclass classification problem by mapping classes $\{0,1,2,3,4\}$ to label $1$ and $\{ 5,6,7,8,9\}$ to label $-1$. For CIFAR dataset, we also use the standard data augementation of random crop and horizontal flip. PyTorch code is as follows: 

% \texttt{(transforms.RandomCrop(32, padding=4),\\
% \tab transforms.RandomHorizontalFlip())}

% \textbf{NLP dataset {} {}} We use IMDb Sentiment analysis~\citep{maas2011learning} corpus.  

% \subsection{Architecture Details} 

% All experiments were run on NVIDIA GeForce RTX 2080 Ti GPUs. We used PyTorch~\citep{NEURIPS2019a9015} and Keras with Tensorflow~\citep{abadi2016tensorflow} backend for experiments. 
% % , ELMo embeddings~\citep{Peters:2018}, and Hugging Face Transformers~\citep{wolf-etal-2020-transformers}. 

% \textbf{Linear model {} {}} For the toy dataset, we simulate a linear model with scalar output and the same number of parameters as the number of dimensions.   

% \textbf{Wide nets {} {}} To simulate the NTK regime, we experiment with $2-$layered wide nets. The PyTorch code for 2-layer wide MLP is as follows: 


% \texttt{ nn.Sequential( \\
% \tab     nn.Flatten(),\\
% \tab    nn.Linear(input\_dims, 200000, bias=True),\\
% \tab    nn.ReLU(),\\
% \tab    nn.Linear(200000, 1, bias=True)\\
% \tab     )}


% We experiment both (i) with the first layer fixed at random initialization; (ii)  and updating both layers' weights.     

% \textbf{Deep nets for CV tasks {} {}} We consider a 4-layered MLP. The PyTorch code for 4-layer MLP is as follows: 

% \texttt{ nn.Sequential(nn.Flatten(), \\
% \tab        nn.Linear(input\_dim, 5000, bias=True),\\
% \tab        nn.ReLU(),\\
% \tab        nn.Linear(5000, 5000, bias=True),\\
% \tab        nn.ReLU(),\\
% \tab        nn.Linear(5000, 5000, bias=True),\\
% \tab        nn.ReLU(),\\
% % \tab        nn.Linear(5000, 5000, bias=True),\\
% % \tab        nn.ReLU(),\\
% \tab        nn.Linear(1024, num\_label, bias=True)\\
% \tab        )}

% For MNIST, we use $1000$ nodes instead of $5000$ nodes in the hidden layer. 
% % 
% We also experiment with convolutional nets. In particular, we use ResNet18 \citep{he2016deep}. Implementation adapted from:  \url{https://github.com/kuangliu/pytorch-cifar.git}. 

% \textbf{Deep nets for NLP {} {}} We use a simple LSTM model with embeddings intialized with ELMo embeddings~\citep{Peters:2018}. Code adapted from: \url{https://github.com/kamujun/elmo_experiments/blob/master/elmo_experiment/notebooks/elmo_text_classification_on_imdb.ipynb} 

% We also evaluate our bounds with a BERT model. In particular, we fine-tune an off-the-shelf uncased BERT model~\citep{devlin2018bert}. Code adapted from Hugging Face Transformers~\citep{wolf-etal-2020-transformers}: \url{https://huggingface.co/transformers/v3.1.0/custom_datasets.html}. 


% \subsection{Additonal experiments}

% 1. SGD with linear models on cross entropy and MSE loss. 

% 2. CE loss and SGD. GD with MSE loss 

% 3. Binary MNIST with MLP. multiclass MNIST  

% \textbf{Results on CIFAR 10 {} {}} 
% % 
% We plot epoch wise error curve for results in \tabref{table:multiclass}. We observe the same trend as in \figref{fig:error_CIFAR10}. Additionally, we plot an \emph{oracle bound} obtained by tracking the error on mislabeled data which nevertheless were predicted as true label. To obtain an exact emprical value of the oracle bound, we need underlying true labels for the randomly labeled data. 
% % Note that our bound in \thmref{thm:multiclass_ERM}, lower bounds the accuracy as predicted by the oracle bound. 
% While with just access to extra unlabeled data we cannot calculate oracle bound, we note that the oracle bound is very tight and never violated in practice underscoring an importamt aspect of generalization in multiclass problems. This highlight that even a stronger conjecture may hold in multiclass classification, i.e., error on mislabeled data (where nevertheless true label was predicted) lower bounds the population error on the distribution of mislabeled data and hence, the error on (a specific) mislabeled portion predicts the population accuracy on clean data. 
% % 
% On the other hand, the dominating term of in \thmref{thm:multiclass_ERM} is loose when compared with the oracle bound. The main reason, we believe is the pessimistic upper bound in \eqref{eq:lemma1_final_multi_prev} in the proof of \lemref{lem:fit_mislabeled_multi}. We leave an investigation on this gap for future. 
% % of fit 

% % However, oracle bound highlights two . One,  



% \begin{figure}[h]
%     \centering 
%     % \vspace{-15pt}
%     % \includegraphics[width=0.9\linewidth]{example-image-a}
%     \includegraphics[width=0.4\linewidth]{figures/CIFAR10-FNN.pdf} \hfil
%     \includegraphics[width=0.4\linewidth]{figures/CIFAR10-Resnet.pdf}
%     % \includegraphics[width=0.9\linewidth]{figures/{CIFAR10_rn=0.1_lr=0.2_wd=0.005}.png}
%     % \vspace{-10pt}
%     \caption{ Per epoch curves for CIFAR10 corresponding results in \tabref{table:multiclass}. As before, we just plot the dominating term in the RHS of \eqref{eq:multiclass_ERM} as predicted bound. Additionally, we also plot the predicted lower bound by the error on mislabeled data which nevertheless were predicted as true label. We refer to this as ``Oracle bound''. See text for more details. 
%     % 
%     % except for the stopping point. 
%     % The bound predicted by RATT (RHS in \eqref{eq:multiclass_ERM}) is vacuous. 
%     }\label{fig:error_epoch_CIFAR10}
%     % \vspace{-15pt}
% \end{figure}


% \textbf{Results on CIFAR 100 {} {}} 
% % 
% On CIFAR100, our bound in \eqref{eq:multiclass_ERM} yields vacous bounds. However, the oracle bound as explained above yields tight guarantees in the initial phase of the learning (i.e., when learning rate is less than $0.1$). 

% \begin{figure}[h]
%     \centering 
%     % \vspace{-15pt}
%     % \includegraphics[width=0.9\linewidth]{example-image-a}
%     \includegraphics[width=0.4\linewidth]{figures/CIFAR100-Resnet.pdf}
%     % \includegraphics[width=0.9\linewidth]{figures/{CIFAR10_rn=0.1_lr=0.2_wd=0.005}.png}
%     % \vspace{-10pt}
%     \caption{ Predicted lower bound by the error on mislabeled data which nevertheless were predicted as true label with ResNet18 on CIFAR100. We refer to this as ``Oracle bound''. See text for more details. 
%     % 
%     % except for the stopping point. 
%     The bound predicted by RATT (RHS in \eqref{eq:multiclass_ERM}) is vacuous. 
%     }\label{fig:error_CIFAR100}
%     % \vspace{-15pt}
% \end{figure}


% % \paragraph{Experiments on CIFAR100} 



% \subsection{Hyperparameter Details}


% \textbf{\figref{fig:error_CIFAR10} {} {}} We use clean training dataset of size $40,000$. We fix the amount of unlabeled data at $20\%$ of the clean size, i.e. we include additional $8,000$ points with randomly assigned labels. We use test set of $10,000$ points. For both MLP and ResNet, we use SGD with an initial learning rate of $0.1$ and momentum $0.9$. We fix the weight decay parameter at $5\times 10^{-4}$. After $100$ epochs, we decay the learning rate to $0.01$. We use SGD batch size of $100$. 

% \textbf{\figref{fig:error_binary} (a) {} {}} We obtain a toy dataset according to the process described in \secref{sec:app_dataset}. We fix $d=100$ and create a dataset of $50,000$ points with balanced classes. Moreover, we sample additional covariates with the same procedure to create randomly labeled dataset. For both SGD and GD training, we use a fixed learning rate $0.1$.    

% \textbf{\figref{fig:error_binary} (b) {} {}} Similar to binary CIFAR, we use clean training dataset of size $40,000$ and fix the amount of unlabeled data at $20\%$ of the clean dataset size. To train wide nets, we use a fixed learning of $0.001$ with GD and SGD. We decide the weight decay parameter and the early stopping point that maximizes our generalization bound (i.e. without peeking at unseen data ).  We use SGD batch size of $100$. 

% \textbf{\figref{fig:error_binary} (c) {} {}} With IMDb dataset, we use a clean dataset of size $20,000$ and as before, fix the amount of unlabeled data at $20\%$ of the clean data. To train ELMo model, we use Adam optimizer with a fixed learning rate $0.01$ and weight decay $10^{-6}$ to minimize cross entropy loss. We train with batch size $32$ for 3 epochs. To fine-tune BERT model, we use Adam optimizer with learning rate $5\times 10^{-5}$ to minimize cross entropy loss. We train with a batch size of $16$ for 1 epoch.    

% \textbf{\tabref{table:multiclass} {} {}} For multiclass datasets, we train both MLP and ResNet with the same hyperparameters as described before. We sample a clean training dataset of size $40,000$ and fix the amount of unlabeled data at $20\%$ of the clean size. We use SGD with an initial learning rate of $0.1$ and momentum $0.9$. We fix the weight decay parameter at $5\times 10^{-4}$. After $30$ epochs for ResNet and after $50$ epochs for MLP, we decay the learning rate to $0.01$.  We use SGD with batch size $100$. 
% For \figref{fig:error_CIFAR100}, we use the same hyperparameters as 
% CIFAR10 training, except we now decay learning rate after $100$ epochs. 


% In all experiments, to identify the best possible accuracy on just the clean data, we use the exact same set of hyperparamters except the stopping point. We choose a stopping point that maximizes test performance. 

% \subsection{Summary of experiments }

% \begin{center}
%     \begin{table}[H] 
%         \centering
%         \begin{tabular}{|c|c|c|c|} 
%         \hline
%         Classification type & Model category & Model & Dataset  \\ [0.5ex] 
%         \hline
%         \hline
%         \multirow{9}{*}{Binary} & Low dimensional & Linear model & Toy Gaussain dataset  \\
%                         \cline{2-4}
%                          & \multirow{1}{*}{Overparameterized linear nets} 
%                         %  & Linear model & Toy Gaussain dataset \\
%                         %  \cline{3-4}
%                         %  & & 2-layer wide net& Toy Gaussain dataset \\
%                         %  \cline{3-4}
%                          & 2-layer wide net & Binary MNIST \\
%                          \cline{2-4}                 
%                          & \multirow{6}{*}{Deep nets} & \multirow{2}{*}{MLP} & Binary MNIST \\
%                          \cline{4-4}
%                          & &  & Binary CIFAR \\
%                          \cline{3-4}
%                          &  & \multirow{2}{*}{ResNet} & Binary MNIST \\
%                          \cline{4-4}
%                          & &  & Binary CIFAR \\
%                          \cline{3-4}
%                          &  & ELMo-LSTM model & IMDb Sentiment Analysis \\
%                          \cline{3-4}
%                          & & BERT pre-trained model & IMDb Sentiment Analysis \\
%         \hline
%         \multirow{5}{*}{Multiclass} & \multirow{5}{*}{Deep nets} & \multirow{2}{*}{MLP} & MNIST \\
%                         \cline{4-4} 
%                         & & & CIFAR10 \\                   
%                         \cline{3-4}
%                          &   & \multirow{3}{*}{ResNet} & MNIST \\
%                          \cline{4-4}
%                          &   & & CIFAR10 \\
%                          \cline{4-4}
%                          &   & & CIFAR100 \\
%         \hline
%         \end{tabular}
%         % \caption{Summary of experiments performed} \label{table:experiments}
%     \end{table}    
%     % \footnotetext[6]{We use both MSE loss and cross-entropy loss.}
%     % \footnotetext[6]{We try 2 variants: one with a fixed first layer and the other with both layers trainable.}
% \end{center}

% \newpage
% \section{Proof of \lemref{lem:stability_error}} \label{app:proof_lem_error}

% \begin{proof}[Proof of \lemref{lem:stability_error}]
%     Recall, we have a training set $S \cup \wt S_C$. We defined leave-one-out error on mislabeled points as $$\error_{\text{LOO}(\wt S_M) } = \frac{\sum_{(x_i, y_i) \in \wt S_M} \error( f_{(i)}( x_i), y_i)}{ \abs{\wt S_M }} \,, $$
%     where $f_{(i)} \defeq f(\calA, (S \cup \wt S)_{(i)})$. Define $S^\prime \defeq S \cup \wt S$. Assume $(x,y)$ and $(x^\prime,y^\prime)$ as i.i.d. samples from ${\calDm}$. 
%     Using Lemma 25 in \citet{bousquet2002stability}, we have
%     \begin{align*}
%         \Expo{ \left( \error_{\calDm}(\wh f) -\error_{\text{LOO}(\wt S_M) } \right)^2 } \le & \Expt{ S^\prime, (x,y), (x^\prime,y^\prime) }{ \error(\wh f(x), y ) \error(\wh f(x^\prime), y^\prime )} - 2 \Expt{ S^\prime, (x,y) }{ \error(\wh f(x), y ) \error(f_{(i)}(x_i), y_i )} \\
%         & + \frac{m_1-1}{m_1}\Expt{ S^\prime }{  \error(f_{(i)}(x_i), y_i )  \error(f_{(j)}(x_j), y_j )} + \frac{1}{m_1} \Expt{ S^\prime }{  \error(f_{(i)}(x_i), y_i ) } \,. \numberthis \label{eq:main_reln}
%     \end{align*}
%     We can rewrite the equation above as : 
%     \begin{align*}
%         \Expo{ \left( \error_{\calDm}(\wh f) -\error_{\text{LOO}(\wt S_M) } \right)^2 } \le &  \, \underbrace{\Expt{ S^\prime, (x,y), (x^\prime,y^\prime) }{ \error(\wh f(x), y ) \error(\wh f(x^\prime), y^\prime ) - \error(\wh f(x), y ) \error(f_{(i)}(x_i), y_i )}}_{\RN{1}} \\
%         & + \underbrace{\Expt{ S^\prime }{  \error(f_{(i)}(x_i), y_i )  \error(f_{(j)}(x_j), y_j ) -  \error(\wh f(x), y ) \error(f_{(i)}(x_i), y_i )}}_{\RN{2}} \\ &+ \underbrace{\frac{1}{m_1} \Expt{ S^\prime }{  \error(f_{(i)}(x_i), y_i ) - \error(f_{(i)}(x_i), y_i )  \error(f_{(j)}(x_j), y_j ) }}_{\RN{3}} \,. \numberthis \label{eq:main_reln2}
%     \end{align*}
    
%     We will now bound term $\RN{3}$.  Using Schwarz's inequality, we have
    
%     \begin{align}
%         \Expt{ S^\prime }{  \error(f_{(i)}(x_i), y_i ) - \error(f_{(i)}(x_i), y_i )  \error(f_{(j)}(x_j), y_j ) }^2 &\le  \Expt{ S^\prime }{  \error(f_{(i)}(x_i), y_i ) }^2 \Expt{S^\prime}{1 -   \error(f_{(j)}(x_j), y_j ) }^2 \\
%         &\le \frac{1}{4} \label{eq:term1_lem12}
%     \end{align}
    
%     Note that since $(x_i,y_i)$, $(x_j ,y_j )$, $(x,y)$, and $(x^\prime, y^\prime)$ are all from same distribution $\calDm$, we directly incorporate the bounds on term $\RN{1}$ and $\RN{2}$ from proof of Lemma 9 in \citet{bousquet2002stability}. Combining that with \eqref{eq:term1_lem12} and our definition of hypothesis stability in \codref{cond:hypothesis_stability}, we have the required claim. 
    
    
%     % We now re-write term $\RN{1}$ as
%     % \begin{align*}
%     %         &\Expt{S^\prime, (x,y), (x^\prime,y^\prime) }{ \error(\wh f(x), y ) \error(\wh f(x^\prime), y^\prime ) - \error(\wh f(x), y ) \error(f_{(i)}(x_i), y_i )} \\ & \qquad = \Expt{ S^\prime, (x,y), (x^\prime,y^\prime) }{ \error(\wh f(x), y ) \error(\wh f  (x^\prime), y^\prime ) - \error(\wh f ^\prime(x), y ) \error(f_{(i)}(x^\prime), y^\prime )} \tag{Exchanging $(x_i, y_i)$ with $(x^\prime, y^\prime)$ in the second term} \\
%     %         & \qquad = \Expt{ S^\prime, (x,y), (x^\prime,y^\prime) }{  \left(\error(\wh f(x), y )-  \error(f_{(i)}(x), y ) \right) \error(\wh f  (x^\prime), y^\prime )  } \\
%     %         & \qquad  + \Expt{ S^\prime, (x,y), (x^\prime,y^\prime) }{  \left(\error(f_{(i)}(x), y ) -\error(\wh f ^\prime(x), y ) \right) \error(\wh f  (x^\prime), y^\prime )}  \\
%     %         & \qquad +\Expt{ S^\prime, (x,y), (x^\prime,y^\prime) }{  \left( \error(\wh f  (x^\prime), y^\prime ) -  \error(f_{(i)}(x^\prime), y^\prime ) \right) \error(\wh f ^\prime(x), y ) }  \,, \numberthis \label{eq:term1_final}
%     % \end{align*}
%     % where $\wh f^\prime$ is the classifier obtained by training on $ S^\prime_{(i)} \cup \{ (x^\prime, y^\prime) \} $. Similarly we can re-write term $\RN{2}$ as 
%     % \begin{align*}
%     %     & \Expt{ S^\prime }{  \error(f_{(i)}(x_i), y_i )  \error(f_{(j)}(x_j), y_j ) -  \error(\wh f(x), y ) \error(f_{(i)}(x_i), y_i )} \\
%     %     &\quad  = \Expt{ S^\prime, (x,y), (x^\prime,y^\prime)}{  \error(f^{\prime\prime}_{(i)}(x), y )  \error(f_{(j)}^{\prime}(x^\prime), y^\prime ) -  \error(\wh f(x), y ) \error(f_{(i)}(x_i), y_i )} \tag{Exchanging $(x_i, y_i)$ with $(x, y)$ and $(x_j, y_j)$ with $(x^\prime, y^\prime)$ in the first term}\\
%     %     &\quad = \Expt{ S^\prime, (x,y), (x^\prime,y^\prime)}{  \error(f^{\prime\prime}_{(j)}(x), y )  \error(f_{(i)}^{\prime}(x^\prime), y^\prime ) -  \error(\wh f^\prime (x), y ) \error(f^\prime_{(j)}(x^\prime), y^\prime )} \tag{Exchanging $(x_i, y_i)$ and $(x_j, y_j)$ and then replacing $(x_j, y_j)$ with $(x^\prime, y^\prime)$ in the second term} \\
%     %     & \quad = \Expt{ S^\prime, (x,y), (x^\prime,y^\prime) }{  \left( \error(f_{(i)}^{\prime}(x^\prime), y^\prime )   -  \error(\wh f^{\prime\prime}  (x^\prime), y^\prime ) \right)  \error(f^{\prime\prime}_{(j)}(x), y )   } \\
%     %     & \quad  + \Expt{ S^\prime, (x,y), (x^\prime,y^\prime) }{  \left( \error(f^{\prime\prime}_{(j)}(x), y )  -\error(\wh f ^\prime(x), y ) \right) \error(\wh f^{\prime\prime}  (x^\prime), y^\prime )  }  \\
%     %     & \quad+ \Expt{ S^\prime, (x,y), (x^\prime,y^\prime) }{  \left( \error(\wh f^{\prime\prime}  (x^\prime), y^\prime )  -  \error(f^\prime_{(j)}(x^\prime), y^\prime ) \right)  \error(\wh f^\prime (x), y ) }   \\
%     %     & \quad = \Expt{ S^\prime, (x,y), (x^\prime,y^\prime) }{  \left( \error(f_{(i)}^{\prime}(x^\prime), y^\prime )   -  \error(\wh f (x^\prime), y^\prime ) \right)  \error(f_{(i)}(x_j), y_j )   } \\
%     %     & \quad  + \Expt{ S^\prime, (x,y), (x^\prime,y^\prime) }{  \left( \error(f^{\prime\prime}_{(j)}(x), y )  -\error(\wh f (x), y ) \right) \error(\wh f^{\prime\prime}  (x_j), y_j )  }  \\
%     %     & \quad+ \Expt{ S^\prime, (x,y), (x^\prime,y^\prime) }{  \left( \error(\wh f^{\prime\prime}  (x^\prime), y^\prime )  -  \error(f^\prime_{(j)}(x^\prime), y^\prime ) \right)  \error(\wh f^\prime (x^\prime), y^\prime ) }  \,, \numberthis \label{eq:term2_final}
%     % \end{align*}
%     % where $f^{\prime\prime}_{(j)}$ is trained on $S^\prime_{(j,i)} \cup {(x,y)}$, $f^{\prime}_{(i)}$ is trained on $S^\prime_{(j,i)} \cup {(x^\prime,y^\prime)}$, and $\wh f^{\prime\prime} $ is trained on $S^\prime_{(j)} \cup {(x,y)}$. Note in the last line we replaced $(x,y)$ by $(x_j, y_j)$ in the first term, replaced $(x^\prime,y^\prime)$ by $(x_j, y_j)$ in the second term and exchanged $(x_i,y_i)$ with $(x_j,y_j)$ and also $(x,y)$ and $(x^\prime, y^\prime)$
    
    
% \end{proof}

\end{document} 


% This document was modified from the file originally made available by
% Pat Langley and Andrea Danyluk for ICML-2K. This version was
% created by Lise Getoor and Tobias Scheffer, it was slightly modified  
% from the 2010 version by Thorsten Joachims & Johannes Fuernkranz, 
% slightly modified from the 2009 version by Kiri Wagstaff and 
% Sam Roweis's 2008 version, which is slightly modified from 
% Prasad Tadepalli's 2007 version which is a lightly 
% changed version of the previous year's version by Andrew Moore, 
% which was in turn edited from those of Kristian Kersting and 
% Codrina Lauth. Alex Smola contributed to the algorithmic style files.  
