% Comment out to revert to include line numbers
\newcommand*{\ARXIV}{}%

\documentclass[10.5pt]{article}
%%%%%%%%%%%%%%%%%%%% Line numbers, and patch for ams math environs
\usepackage[displaymath, mathlines]{lineno}

\ifdefined\ARXIV
  \pdfoutput=1
\else
  \linenumbers
\fi

\usepackage{benstyle}

\geometry{margin=1.05in}
\newcommand*{\pbk}{\vspace{1pc}\noindent}

\newcommand*\patchAmsMathEnvironmentForLineno[1]{%
  \expandafter\let\csname old#1\expandafter\endcsname\csname #1\endcsname
  \expandafter\let\csname oldend#1\expandafter\endcsname\csname end#1\endcsname
  \renewenvironment{#1}%
     {\linenomath\csname old#1\endcsname}%
     {\csname oldend#1\endcsname\endlinenomath}}% 
\newcommand*\patchBothAmsMathEnvironmentsForLineno[1]{%
  \patchAmsMathEnvironmentForLineno{#1}%
  \patchAmsMathEnvironmentForLineno{#1*}}%
\AtBeginDocument{%
\patchBothAmsMathEnvironmentsForLineno{equation}%
\patchBothAmsMathEnvironmentsForLineno{align}%
\patchBothAmsMathEnvironmentsForLineno{flalign}%
\patchBothAmsMathEnvironmentsForLineno{alignat}%
\patchBothAmsMathEnvironmentsForLineno{gather}%
\patchBothAmsMathEnvironmentsForLineno{multline}%
}
%%%%%%%%%%%%%%%%%%%%

\usepackage[usenames,dvipsnames,svgnames,table]{xcolor}
\usepackage{mathtools,ifthen}
\usepackage{array,multirow,booktabs} % for fancier tables
%\usepackage{showkeys}
\usepackage{lipsum}
\usepackage{bbm}

\newcommand{\annote}[1]{{\leavevmode\color{RoyalBlue}{#1}}}
\newcommand{\anrev}[1]{{\leavevmode\color{BrickRed}{#1}}}
\renewcommand{\anrev}[1]{#1}
\newcommand{\com}[1]{{\color{red}#1}}
\newcommand{\V}[1]{\boldsymbol{#1}}
\newcommand{\mathd}{\mathrm{d}}
\newcommand{\dx}[1]{\mathd #1}

\newcommand{\edits}[1]{{\leavevmode\color{BrickRed}{#1}}}
\renewcommand{\edits}[1]{#1}

%
%\setlength{\parindent}{0pt}
%\setlength{\parskip}{1ex plus 0.5ex minus 0.2ex}
%

% For leftstackrel
\newlength{\leftstackrelawd}
\newlength{\leftstackrelbwd}
\def\leftstackrel#1#2{\settowidth{\leftstackrelawd}%
{${{}^{#1}}$}\settowidth{\leftstackrelbwd}{$#2$}%
\addtolength{\leftstackrelawd}{-\leftstackrelbwd}%
\leavevmode\ifthenelse{\lengthtest{\leftstackrelawd>0pt}}%
{\kern-.5\leftstackrelawd}{}\mathrel{\mathop{#2}\limits^{#1}}}

\newcommand{\naive}{na\"{\i}ve}

% Hackity-hack for condensed spacing of bibliography
\let\oldbibliography\thebibliography
\renewcommand{\thebibliography}[1]{%
  \oldbibliography{#1}%
  \setlength{\itemsep}{0pt}%
}

\begin{document}

\title{Compressed sensing with sparse corruptions: Fault-tolerant sparse collocation approximations}
\author{
  Ben Adcock\footnote{B. Adcock and A. Bao acknowledge the support of the Alfred P. Sloan Foundation and the Natural Sciences and Engineering Research Council of Canada through grant 611675.} \\ Department of Mathematics \\ Simon Fraser University \\ Burnaby, BC, Canada 
  \and 
  Anyi Bao\footnotemark[1] \\ Department of Mathematics \\ Simon Fraser University \\ Burnaby, BC, Canada \\[20pt]
  \and 
  John D.\ Jakeman\footnote{J.D.Jakeman's work was supported by DARPA EQUiPS.} \\ Computer Science Research Institute \\ Sandia National Laboratories \\ Albuquerque, NM, USA\\[10pt]
  %\hspace*{40pt} Undisclosed author\footnote{This author's institution requires internal approval of authorship for publicly released documents. This approval is pending and proper authorship will be credited in future versions of this manuscript.}\hspace*{40pt}\\[20pt]
  \and 
  Akil Narayan\footnote{A. Narayan is partially supported by NSF DMS-1720416, AFOSR FA9550-15-1-0467, and DARPA EQUiPS N660011524053} \\ Department of Mathematics and \\ Scientific Computing and Imaging (SCI) Institute \\ University of Utah \\ Salt Lake City, UT, USA
}

\maketitle
\begin{abstract}
  The recovery of approximately sparse or compressible coefficients in a polynomial chaos expansion is a common goal in many modern parametric uncertainty quantification (UQ) problems. However, relatively little effort in UQ has been directed toward theoretical and computational strategies for addressing the sparse \textit{corruptions} problem, where a small number of measurements are highly corrupted. Such a situation has become pertinent today since modern computational frameworks are sufficiently complex with many interdependent components that may introduce hardware and software failures, some of which can be difficult to detect and result in a highly polluted simulation result. 
  
  In this paper we present a novel compressive sampling-based theoretical analysis for a regularized $\ell^1$ minimization algorithm that aims to recover sparse expansion coefficients in the presence of measurement corruptions. Our recovery results are uniform (the theoretical guarantees hold for all compressible signals and compressible corruptions vectors), and prescribe algorithmic regularization parameters in terms of a user-defined \textit{a priori} estimate on the ratio of measurements that are believed to be corrupted. We also propose an iteratively reweighted optimization algorithm that automatically refines the value of the regularization parameter, and empirically produces superior results. Our numerical results test our framework on several medium-to-high dimensional examples of solutions to parameterized differential equations, and demonstrate the effectiveness of our approach.
\end{abstract}

%\tableofcontents

% \leavevmode
% \\
% \\
% \\
% \\
% \\
\section{Introduction}
\label{introduction}

AutoML is the process by which machine learning models are built automatically for a new dataset. Given a dataset, AutoML systems perform a search over valid data transformations and learners, along with hyper-parameter optimization for each learner~\cite{VolcanoML}. Choosing the transformations and learners over which to search is our focus.
A significant number of systems mine from prior runs of pipelines over a set of datasets to choose transformers and learners that are effective with different types of datasets (e.g. \cite{NEURIPS2018_b59a51a3}, \cite{10.14778/3415478.3415542}, \cite{autosklearn}). Thus, they build a database by actually running different pipelines with a diverse set of datasets to estimate the accuracy of potential pipelines. Hence, they can be used to effectively reduce the search space. A new dataset, based on a set of features (meta-features) is then matched to this database to find the most plausible candidates for both learner selection and hyper-parameter tuning. This process of choosing starting points in the search space is called meta-learning for the cold start problem.  

Other meta-learning approaches include mining existing data science code and their associated datasets to learn from human expertise. The AL~\cite{al} system mined existing Kaggle notebooks using dynamic analysis, i.e., actually running the scripts, and showed that such a system has promise.  However, this meta-learning approach does not scale because it is onerous to execute a large number of pipeline scripts on datasets, preprocessing datasets is never trivial, and older scripts cease to run at all as software evolves. It is not surprising that AL therefore performed dynamic analysis on just nine datasets.

Our system, {\sysname}, provides a scalable meta-learning approach to leverage human expertise, using static analysis to mine pipelines from large repositories of scripts. Static analysis has the advantage of scaling to thousands or millions of scripts \cite{graph4code} easily, but lacks the performance data gathered by dynamic analysis. The {\sysname} meta-learning approach guides the learning process by a scalable dataset similarity search, based on dataset embeddings, to find the most similar datasets and the semantics of ML pipelines applied on them.  Many existing systems, such as Auto-Sklearn \cite{autosklearn} and AL \cite{al}, compute a set of meta-features for each dataset. We developed a deep neural network model to generate embeddings at the granularity of a dataset, e.g., a table or CSV file, to capture similarity at the level of an entire dataset rather than relying on a set of meta-features.
 
Because we use static analysis to capture the semantics of the meta-learning process, we have no mechanism to choose the \textbf{best} pipeline from many seen pipelines, unlike the dynamic execution case where one can rely on runtime to choose the best performing pipeline.  Observing that pipelines are basically workflow graphs, we use graph generator neural models to succinctly capture the statically-observed pipelines for a single dataset. In {\sysname}, we formulate learner selection as a graph generation problem to predict optimized pipelines based on pipelines seen in actual notebooks.

%. This formulation enables {\sysname} for effective pruning of the AutoML search space to predict optimized pipelines based on pipelines seen in actual notebooks.}
%We note that increasingly, state-of-the-art performance in AutoML systems is being generated by more complex pipelines such as Directed Acyclic Graphs (DAGs) \cite{piper} rather than the linear pipelines used in earlier systems.  
 
{\sysname} does learner and transformation selection, and hence is a component of an AutoML systems. To evaluate this component, we integrated it into two existing AutoML systems, FLAML \cite{flaml} and Auto-Sklearn \cite{autosklearn}.  
% We evaluate each system with and without {\sysname}.  
We chose FLAML because it does not yet have any meta-learning component for the cold start problem and instead allows user selection of learners and transformers. The authors of FLAML explicitly pointed to the fact that FLAML might benefit from a meta-learning component and pointed to it as a possibility for future work. For FLAML, if mining historical pipelines provides an advantage, we should improve its performance. We also picked Auto-Sklearn as it does have a learner selection component based on meta-features, as described earlier~\cite{autosklearn2}. For Auto-Sklearn, we should at least match performance if our static mining of pipelines can match their extensive database. For context, we also compared {\sysname} with the recent VolcanoML~\cite{VolcanoML}, which provides an efficient decomposition and execution strategy for the AutoML search space. In contrast, {\sysname} prunes the search space using our meta-learning model to perform hyperparameter optimization only for the most promising candidates. 

The contributions of this paper are the following:
\begin{itemize}
    \item Section ~\ref{sec:mining} defines a scalable meta-learning approach based on representation learning of mined ML pipeline semantics and datasets for over 100 datasets and ~11K Python scripts.  
    \newline
    \item Sections~\ref{sec:kgpipGen} formulates AutoML pipeline generation as a graph generation problem. {\sysname} predicts efficiently an optimized ML pipeline for an unseen dataset based on our meta-learning model.  To the best of our knowledge, {\sysname} is the first approach to formulate  AutoML pipeline generation in such a way.
    \newline
    \item Section~\ref{sec:eval} presents a comprehensive evaluation using a large collection of 121 datasets from major AutoML benchmarks and Kaggle. Our experimental results show that {\sysname} outperforms all existing AutoML systems and achieves state-of-the-art results on the majority of these datasets. {\sysname} significantly improves the performance of both FLAML and Auto-Sklearn in classification and regression tasks. We also outperformed AL in 75 out of 77 datasets and VolcanoML in 75  out of 121 datasets, including 44 datasets used only by VolcanoML~\cite{VolcanoML}.  On average, {\sysname} achieves scores that are statistically better than the means of all other systems. 
\end{itemize}


%This approach does not need to apply cleaning or transformation methods to handle different variances among datasets. Moreover, we do not need to deal with complex analysis, such as dynamic code analysis. Thus, our approach proved to be scalable, as discussed in Sections~\ref{sec:mining}.
\section{Notation}
\label{sec:notation}

Let bold functions $\mathbf{f}$ represent the row-wise vectorized versions of their scalar counterparts $f$: $\mathbf{f}(\mathbf{x}) = [f(x_1), f(x_2) \dots f(x_n)]^T$. 

Let $I_a(z)$ be the indicator function for $z=a$.

Let $P(Z^*)$ be a distribution that places probability $\frac{1}{n}$ on each element of  a set of samples $\{z_1, z_2 \dots z_n\}$. Let $z^*$ denote a sample from $Z^*$ and $\bm{z}^*$ denote a set of $n$ such samples (a bootstrap sample).

\subsection{Notation for Observational data}

Observational datasets are represented by a set of $n$ tuples of observations $(x_i,w_i,y_i)$. For each subject $i$, $x_i \in \mathcal{R}^{p}$ is a vector of observed pre-treatment covariates, $w_i \in \{0,1\}^n$ is a binary indicator of treatment status, and $y_i \in \mathcal{R}^n$ is a real-valued outcome. Let capital letters $X$, $W$, and $Y$ represent the corresponding random variables. Let $\mathcal{S}_0$ be the set of indices of untreated subjects, and $\mathcal{S}_1$ the set of indices of treated subjects: $\mathcal{S}_w = \{i | w_i = w\}$.  Denote a series of I.I.D. realizations of a random variable $z \sim P(Z)$ as $\bm{z}$ so that the full observational dataset can be written as $d = (\mathbf{x}, \mathbf{w}, \mathbf{y})$ where $(x_i,w_i,y_i) \overset{\text{I.I.D.}}{\sim} P(X,W,Y)$. 

The average treatment effect is defined as 

\begin{equation}
\tau = E_{X,Y}[Y|X,W=1] - E_{X,Y}[Y|X,W=0]
\label{eq:effect}
\end{equation}

which is the expected difference between what the outcome would have been had a subject received the treatment and what the outcome would have been had a subject not received the treatment, averaged over all subjects in the population. %The outcomes under each of these conditions are referred to as potential outcomes. Let those be denoted as $f_w(x,u) = f(x,u,w) = E[Y_w | X, U]$. %Note that $f_w(\mathbf{x}, \mathbf{u}) + \bm{\eta} \in \mathcal{R}^n$, but $\mathbf{y}_w \in \mathcal{R}^{n_w}$. The latter does not include potential outcomes under treatment $w$ that were not observed.

\section{Theory for the sparse corruptions problem}\label{sec:theory}

%\subsection{Notation}
%\bull{
%\item $m$ -- number of measurements
%\item $N$ -- length of sparse vector
%\item $x$ -- sparse vector in $\bbC^N$
%\item $c$ -- corruptions vector in $\bbC^m$
%\item $A$ -- $m \times N$ measurement matrix
%\item $n$ -- noise vector in $\bbC^m$
%\item $\epsilon$ -- noise bound
%\item $\lambda$ -- nonnegative weighting parameter for the corruptions vector
%\item $\hat{x}$, $\hat{c}$ -- solutions of the optimization problem
%\item $S$ -- subset of indices corresponding to $x$
%\item $T$ -- subset of indices corresponding to $c$
%\item $s$ -- sparsity of $x$
%\item $k$ -- sparsity of $c$
%\item $\Sigma_{s}$ -- set of $s$-sparse vectors in $\bbC^N$
%\item $\Sigma_{k}$ -- set of $k$-sparse vectors in $\bbC^m$
%\item $\sigma_{s}(x)_1$ -- best $s$-term approximation error, measured in the $\ell^1$ norm
%\item $\sigma_{k}(c)_1$ -- best $k$-term approximation error, measured in the $\ell^1$ norm
%}

\begin{table}
  \begin{center}
  \resizebox{\textwidth}{!}{
    \renewcommand{\tabcolsep}{0.4cm}
    \renewcommand{\arraystretch}{1.3}
    {\scriptsize
    \begin{tabular}{@{}cp{0.8\textwidth}@{}}
      \toprule
      $m$ & number of measurements \\
      $N$ & length of sparse vector \\
      $x$ & sparse vector in $\bbC^N$ \\
      $c$ & corruptions vector in $\bbC^m$\\
      $A$ & $m \times N$ measurement matrix \\
      $n$ & noise vector in $\bbC^m$ \\
      $\epsilon$ & noise bound \\
      $\lambda$ & non-negative weighting parameter for the corruptions vector \\
      $\hat{x}$, $\hat{c}$ & solutions of the optimization problem \\
      $S$ & subset of $\left\{1, \ldots, N \right\}$, indices corresponding to $x$ \\
      $T$ & subset of $\left\{1, \ldots, m \right\}$, indices corresponding to $c$ \\
      $s$ & sparsity of $x$ \\
      $k$ & sparsity of $c$ \\
      $\Sigma_{s}$ & set of $s$-sparse vectors in $\bbC^N$ \\
      $\Sigma_{k}$ & set of $k$-sparse vectors in $\bbC^m$ \\
      $\sigma_{s}(x)_1$ & best $s$-term approximation error, measured in the $\ell^1$ norm \\
      $\sigma_{k}(c)_1$ & best $k$-term approximation error, measured in the $\ell^1$ norm \\
    \bottomrule
    \end{tabular}
  }
    \renewcommand{\arraystretch}{1}
    \renewcommand{\tabcolsep}{12pt}
  }
  \end{center}
  \caption{Notation used throughout this article.}\label{tab:notation}
\end{table}

We recall and summarize our notation for the sparse corruptions problem in Table \ref{tab:notation}. Our previous discussion was framed for real-valued signals $x$ and measurements $y$, but we now generalize to the complex-valued setting. This adds generality with no additional mathematical difficulty.

We follow a familiar path for deriving conditions on $m$ such that $\ell^1$ optimization problems recover sparse solutions (see, for example, \cite{FoucartRauhutCSbook}). Section \ref{sec:robust-nsp} defines an appropriate robust Null Space Property (NSP) for the matrix $A$ in the sparse corruptions setting. Under this property, we show that the recovery estimates \eqref{eq:recovery-summary} hold. In order to construct matrices $A$ that satisfy the robust NSP, Section \ref{sec:rip} generalizes the concept of the Restricted Isometry Property (RIP) for matrices to the sparse corruptions setting. That section shows that matrices satisfying the RIP for the sparse corruptions problem also satisfy the robust NSP.  Sections \ref{sec:rip-2} and \ref{sec:bos} show that if the dictionary elements $\phi_j$ form a bounded orthonormal system, then under the condition \eqref{eq:m-bound}, the matrix $A$ satisfies the RIP with high probability.  Finally, using these various results, we discuss a theoretically-optimal choice for $\lambda$ in Section \ref{ss:lambda-strategy}.

\subsection{The Robust Null Space Property for the sparse corruptions problem}\label{sec:robust-nsp}

%\GR{I have removed the first paragraph here (essentially a repeat of what's written above).  Also, I removed the subsubsection headings, which I felt were unnecessary.}

%In this section we show that a certain \textit{robust null space property} (robust NSP) is sufficient for stable and robust recovery.  Note that the robust NSP was originally introduced by Rauhut [] for the recovery of sparse vectors.  Below we introduce a generalization that allows for (i) the incorporation of a sparse corruptions term $c$, and (ii) the weighting parameter $\lambda$ used in the optimization problem \R{l1_lambda_recovery}.

%\subsubsection{Definitions}

The following two definitions are generalizations of robust null space properties (cf. \cite[Definition 4.17]{FoucartRauhutCSbook} and \cite[Definition 4.21]{FoucartRauhutCSbook}, respectively), and prescribe classes of matrices whose kernels do not contain sparse vectors.

\defn{
Let $1 \leq s \leq N$, $1 \leq k \leq m$ and $\lambda > 0$.  A matrix $A \in \bbC^{m \times N}$ satisfies the $\ell^1$-robust null space property of order $(s,k)$ with weight $\lambda$ if there exist constants $0 < \rho <1$ and $\tau > 0$ such that
\bes{
\| x_{S} \|_{1} + \lambda \| c_{T} \|_{1} \leq \rho \left ( \| x_{S^c} \|_{1} + \lambda \| c_{T^c} \|_{1} \right ) + \tau \| A x + c \|_{2},\quad \forall x \in \bbC^N,\ c \in \bbC^m,
}
for all sets $S \subseteq \{1,\ldots,N\}$ and $T \subseteq \{1,\ldots,m\}$ with $| S | \leq s$ and $|T| \leq k$. Above, $S^c$ is the complement of $S$ in $\{1, \ldots, N\}$, and similarly for $T^c$.
}

\defn{
Let $1 \leq s \leq N$, $1 \leq k \leq m$ and $\lambda > 0$.  A matrix $A \in \bbC^{m \times N}$ satisfies the $\ell^2$-robust null space property of order $(s,k)$ with weight $\lambda$ if there exist constants $0 < \rho <1$ and $\tau > 0$ such that
\be{
\label{l2_rNSP_def}
\sqrt{\| x_{S} \|^2_{2} + \| c_{T} \|^2_{2}} \leq \frac{\rho}{\sqrt{s+\lambda^2 k}} \left ( \| x_{S^c} \|_{1} + \lambda \| c_{T^c} \|_{1} \right ) + \tau \| A x + c \|_{2},\quad \forall x \in \bbC^N,\ c \in \bbC^m,
}
for all sets $S \subseteq \{1,\ldots,N\}$ and $T \subseteq \{1,\ldots,m\}$ with $| S | \leq s$ and $|T| \leq k$.
}

%\rem{
%This can be generalized to an $\ell^p$-robust NSP for $1 \leq p \leq 2$.  This would add some generality to the error bounds below.
%}
%\GR{Remark removed here.  I think I wrote this mainly for our own benefit as a reminder.  I don't have the time now to work out the $\ell^p$-robust NSP, but can do this for the journal version.}


%\subsubsection{The $\ell^2$-robust null space property implies stable and robust recovery}

These definitions yield the following two results:
%With this in hand, we now have the following two results:

\lem{
\label{l:21_RNSP}
If $A \in \bbC^{m \times N}$ satisfies the $\ell^2$-robust null space property of order $(s,k)$ with weight $\lambda >0$ and constants $0 < \rho <1$, $\tau >0$ then it satisfies the $\ell^1$-robust null space property of order $(s,k)$ with weight $\lambda >0$ and constants $\rho$, $ \tau\sqrt{s+\lambda^2 k}$.
}
\prf{
Observe that
\bes{
\| x_{S} \|_{1} + \lambda \| c_{T} \|_{1} \leq \sqrt{s} \| x_{S} \|_{2} + \lambda \sqrt{k} \| c_{T} \|_{2} \leq \sqrt{s+\lambda^2 k} \sqrt{\| x_{S} \|^2_{2} + \| c_{T} \|^2_{2} }.
}
We now use the definition of the $\ell^2$-robust null space property.
}



\thm{
\label{t:rNSP_stable_robust}
Let $1 \leq s \leq N$, $1 \leq k \leq m$ and $\lambda > 0$ and suppose that $A \in \bbC^{m \times N}$ satisfies the $\ell^2$-robust null space property of order $(s,k)$ with weight $\lambda$.  Let $x \in \bbC^N$, $c \in \bbC^m$, $y \in \bbC^m$ and $\epsilon > 0$ be such that $\| A x + c - y \|_{2} \leq \epsilon$, and suppose that  $(\hat{x},\hat{c})$ is a minimizer of
\bes{
\min_{z \in \bbC^N, d \in \bbC^m} \| z \|_{1} + \lambda \| d \|_{1}\ \mbox{subject to $\| A z + d - y \|_{2} \leq \epsilon$}.
}
%where $y \in \bbC^m$ satisfies $\| A x + c - y \|_{2} \leq \eta$ for some $\epsilon > 0$.
Then
\be{
\label{l1_err_bound}
\| x - \hat{x} \|_{1} + \lambda \| c - \hat{c} \|_{1} \leq C_1 \left( \sigma_{s}(x)_1 + \lambda \sigma_{k}(c)_1 \right) + C_2\sqrt{s+\lambda^2 k} \epsilon,
}
and
\be{
\label{l2_err_bound}
\| x - \hat{x} \|_{2} + \| c - \hat{c} \|_{2} \leq C_3 \left ( 1 + \eta^{1/4} \right )\left( \frac{\sigma_{s}(x)_1}{\sqrt{s}} + \frac{\sigma_{k}(c)_1}{\sqrt{k}} \right) + C_4 \left ( 1 + \eta^{1/4} \right ) \epsilon,
}
where the constants $C_1,C_2,C_3,C_4$ depend on $\rho$ and $\tau$ only and $\eta$ is given by
\be{
\label{eta_def}
\eta = \eta_{s,k}(\lambda) = \frac{s+\lambda^2k}{\min \{ s , \lambda^2 k \} }.
}
}
%
%This theorem suggests that the choice $\lambda = \sqrt{s/k}$ is a good one, at least from the view of the theory.  We will see further evidence for this below when we look at the RIP.
%\GR{Removed comment here (a leftover from the notes).  All discussion on $\eta$ is now placed later in the Discussion subsubsection.}


\prf{
We first prove \R{l1_err_bound}.
Lemma \ref{l:21_RNSP} implies that $A$ satisfies the $\ell^1$-robust null space property.  Let $S \subseteq \{1,\ldots,N\}$, $|S| \leq s$ and $T \subseteq \{1,\ldots,m\}$, $|T| \leq k$ be such that $\| x_{S^c} \|_{1} = \sigma_{s}(x)_1$ and $\| c_{T^c} \|_{1} = \sigma_{k}(c)_1$.  Then, if $v = x - \hat{x}$ and $e = c - \hat{c}$ we have
\eas{
\| x \|_{1} +\lambda \| c \|_{1} + \| v_{S^c} \|_{1} + \lambda \| e_{T^c} \|_{1} &\leq 2 \| x_{S^c} \|_{1} + \| x_{S} \|_{1} + \lambda \left ( 2 \| c_{T^c} \|_{1} + \| c_{T} \|_1 \right ) + \| \hat{x}_{S^c} \|_{1} + \lambda \| \hat{c}_{T^c} \|_{1}
\\
& \leq 2 \| x_{S^c} \|_{1} + \| v_{S} \|_{1} + \| \hat{x} \|_{1} + \lambda \left ( 2 \| c_{T^c} \|_{1} + \| e_{T} \|_{1} + \| \hat{c} \|_{1} \right ).
}
Rearranging now gives
\eas{
\| v_{S^c} \|_{1} + \lambda \| e_{T^c} \|_{1} \leq & \left ( 2 \| x_{S^c} \|_{1} + \| v_{S} \|_{1} \right ) + \lambda \left ( 2 \| c_{T^c} \|_{1} + \| e_{T} \|_{1} \right )
\\
& + \left ( \| \hat{x} \|_{1} + \lambda \| \hat{c} \|_{1} \right ) - \left ( \| x \|_{1} + \lambda \| c \|_{1} \right )
\\
& \leq 2 \left ( \| x_{S^c} \|_{1} + \lambda \| c_{T^c} \|_{1} \right ) + \left ( \| v_{S} \|_{1} + \lambda \| e_{T} \|_{1} \right ),
}
where in the second inequality we note that $\| x \|_{1} + \lambda \| c \|_{1} \geq \| \hat{x} \|_{1} + \lambda \| \hat{c} \|_{1}$ since $(x,c)$ is feasible and $(\hat{x},\hat{c})$ is a minimizer.    The $\ell^1$-robust null space property now implies that
\bes{
\| v_{S^c} \|_{1} + \lambda \| e_{T^c} \|_{1} \leq \frac{2}{1-\rho} \left ( \| x_{S^c} \|_{1} + \lambda \| c_{T^c} \|_{1} \right ) + \frac{\tau\sqrt{s+\lambda^2 k}}{1-\rho} \| A v + e \|_{2},
}
and since $\| x_{S^c} \|_{1} = \sigma_{s}(x)_1$, $\| c_{T^c} \|_{1} = \sigma_{k}(c)_1$ and
\be{
\label{Ave_eps_bound}
\| A v + e \|_{2} \leq \| A \hat{x} + \hat{c} - y \|_{2} + \| A x + c - y \|_{2} \leq 2 \epsilon,
}
we deduce that
\be{
\label{ve_ST_comp_bound}
\| v_{S^c} \|_{1} + \lambda \| e_{T^c} \|_{1} \leq\frac{2}{1-\rho} \left ( \sigma_{s}(x)_1+ \lambda \sigma_{k}(c)_1 \right ) + \frac{2 \tau}{1-\rho} \sqrt{s+\lambda^2 k}\epsilon.
}
Finally, to complete the proof of \R{l1_err_bound} we argue as follows:
\eas{
\| v \|_{1} + \lambda \| e\|_{1} & \leq \| v_{S} \|_{1} + \lambda \| e_{T} \|_{1} + \| v_{S^c} \|_{1} + \lambda \| e_{T^c} \|_{1}
\\
& \leq (1+\rho) \left ( \| v_{S^c} \|_{1} + \lambda \| e_{T^c} \|_{1} \right ) + \tau \sqrt{s+\lambda^2 k} \| A v + e \|_{2}
\\
& \leq 2 \frac{1+\rho}{1-\rho} \left (  \sigma_{s}(x)_1+ \lambda \sigma_{k}(c)_1 \right ) + \frac{4}{1-\rho} \tau \sqrt{s+\lambda^2 k} \epsilon.
}
Here, we use the $\ell^1$-robust null space property in the second step, and \R{Ave_eps_bound} and \R{ve_ST_comp_bound} in the third step.

We now consider \R{l2_err_bound}.  Writing $v = x - \hat{x}$ and $e = c - \hat{c}$ as before, let $S$ be the index of the largest $s$ elements of $v$ in absolute value and $T$ be the index set of the largest $k$ elements of $e$ in absolute value.  Define
\bes{
\theta_{v} = \min_{i \in S} | v_i|,\quad \theta_{e} = \min_{j \in T} |e_j |,\qquad \theta = \max \{ \theta_v , \theta_e / \lambda \}.
}
Then
\bes{
\| v_{S^c} \|^2_{2} + \| e_{T^c} \|^2_2 = \sum_{i \notin S} | v_i |^2 + \sum_{j \notin T } |e_j |^2 \leq \theta_v \sum_{i \notin S} | v_i | + \theta_e \sum_{j \notin T } |e_j | \leq \theta \left ( \| v_{S^c} \|_{1} + \lambda \| e_{T^c} \|_{1} \right ).
}
Now observe that $\theta_{v} \leq \| v_{S} \|_{2} / \sqrt{s}$ and $\theta_{e} \leq \| e_{T} \|_{2} / \sqrt{k}$, and therefore
\bes{
\theta \leq \frac{\sqrt{\|v_S\|^2_2+\|e_T\|^2_2}}{\min \{ \sqrt{s} , \lambda \sqrt{k} \}} \leq \frac{1}{\min \{ \sqrt{s} , \lambda \sqrt{k} \}} \left ( \frac{\rho}{\sqrt{s+\lambda^2 k}} \left ( \| v_{S^c} \|_{1} + \lambda \| e_{T^c} \|_1 \right ) + 2 \tau \epsilon \right ),
}
where in the second step we use the $\ell^2$-robust null space property and \R{Ave_eps_bound}.  Combining this with the previous estimate and using the definition of $\eta$ gives
\eas{
\| v_{S^c} \|^2_{2} + \| e_{T^c} \|^2_2  \leq &  \frac{1}{\min \{ \sqrt{s} , \lambda \sqrt{k} \}} \left ( \frac{\rho}{\sqrt{s+\lambda^2 k}} \left ( \| v_{S^c} \|_{1} + \lambda \| e_{T^c} \|_1 \right )^2 + 2 \tau \epsilon \left ( \| v_{S^c} \|_{1} + \lambda \| e_{T^c} \|_1 \right ) \right )
\\
= &
  \sqrt{\eta} \left[ \rho w^2 + 2 \tau \epsilon w \right],
% \leq & \frac{1}{\min \{ \sqrt{s} , \lambda \sqrt{k} \}} \left ( \frac{\rho+1/2}{\sqrt{s+\lambda^2 k}} \left ( \| v_{S^c} \|_{1} + \lambda \| e_{T^c} \|_1 \right )^2 + 2  \tau^2 \sqrt{s+\lambda^2 k} \epsilon^2 \right ),
}
where we have defined the non-negative scalar $w$ as
  \begin{align*}
    w \coloneqq \frac{\left\| v_{S^c} \right\|_1 + \lambda \left\| e_{T^c} \right\|_1}{\sqrt{s + \lambda^2 k}}
  \end{align*}
  Completing the square with respect to $w$ under the brackets yields
  \begin{align*}
    \| v_{S^c} \|^2_{2} + \| e_{T^c} \|^2_2 \leq \rho \sqrt{\eta} \left[ \left(w + \frac{\tau \epsilon}{\sqrt{\rho}} \right)^2 - \frac{\tau^2 \epsilon^2}{\rho} \right] 
                                            \leq \rho \sqrt{\eta} \left(w + \frac{\tau \epsilon}{\sqrt{\rho}}\right)^2
  \end{align*}
Using the $\ell^2$-robust NSP on the pair $(v,e)$ along with the above estimate, we have
\begin{align}\nonumber
  \frac{1}{\sqrt{2}} \left( \| v\|_2 + \|e\|_2 \right) \leq \sqrt{\| v \|^2_{2} + \| e \|^2_{2}} &= \sqrt{\| v_S \|^2_2 + \| e_T \|^2_2 + \| v_{S^c} \|^2_2 + \| e_{T^c} \|^2_2}
\\\nonumber
&\leq \sqrt{\| v_S \|^2_2 + \| e_T \|^2_2} + \sqrt{\| v_{S^c} \|^2_2 + \| e_{T^c} \|^2_2}
\\\nonumber
&\leq \rho w + 2 \tau \epsilon + \sqrt{\rho} \eta^{1/4} \left(w + \frac{\tau \epsilon}{\sqrt{\rho}}\right) \\\label{eq:rNSP_stable_robust-temp}
&= \sqrt{\rho} \left( \sqrt{\rho} + \eta^{1/4}\right) w + \tau \left(2 + \eta^{1/4} \right) \epsilon
\end{align}
We note that
\begin{align*}
  w &= \frac{ \left\| v_{S^c} \right\|_1 + \lambda \left\| e_{T^c} \right\|_1}{\sqrt{s + \lambda^2 k}} \leq \frac{ \left\| v \right\|_1 + \lambda \left\| e \right\|_1}{\sqrt{s + \lambda^2 k}} \\
    &\leftstackrel{\eqref{l1_err_bound}}{\leq} C_1 \left[ \frac{\sigma_s(x)_1}{\sqrt{s + \lambda^2 k}} + \lambda \frac{\sigma_k(c)_1}{\sqrt{s + \lambda^2 k}} \right] + C_2 \epsilon
  \leq C_1 \left[ \frac{\sigma_s(x)_1}{\sqrt{s}} + \frac{\sigma_k(c)_1}{\sqrt{k}} \right] + C_2 \epsilon
\end{align*}
Combining the above with \eqref{eq:rNSP_stable_robust-temp} proves \eqref{l2_err_bound}.
%where in the second step we use the inequality $a b \leq a^2/2+b^2/2$.  Since $\sqrt{a+b} \leq \sqrt{a} + \sqrt{b}$, we deduce that
%\bes{
%\sqrt{\| v_{S^c} \|^2_{2} + \| e_{T^c} \|^2_2} \leq \frac{1}{\sqrt{\min \{ \sqrt{s} , \lambda \sqrt{k} \}}} \left ( \frac{\sqrt{\rho+1/2}}{(s+\lambda^2 k)^{1/4}} \left ( \| v_{S^c} \|_{1} + \lambda \| e_{T^c} \|_1 \right ) + \sqrt{2} \tau (s+\lambda^2 k)^{1/4} \epsilon \right ).
%}
%Now applying the $\ell^2$-robust null space property once more we deduce that
%\eas{
%\sqrt{\| v \|^2_{2} + \| e \|^2_{2}} =& \sqrt{\| v_S \|^2_2 + \| e_T \|^2_2 + \| v_{S^c} \|^2_2 + \| e_{T^c} \|^2_2}
%\\
%\leq & \sqrt{\| v_S \|^2_2 + \| e_T \|^2_2} + \sqrt{\| v_{S^c} \|^2_2 + \| e_{T^c} \|^2_2}
%\\
%\leq & \left ( \frac{\rho}{\sqrt{s+\lambda^2 k}} \left ( \| v_{S^c} \|_{1} + \lambda \| e_{T^c} \|_1 \right ) + 2 \tau \epsilon \right )
%\\
%& + \frac{1}{\sqrt{\min \{ \sqrt{s} , \lambda \sqrt{k} \}}} \left ( \frac{\sqrt{\rho+1/2}}{(s+\lambda^2 k)^{1/4}} \left ( \| v_{S^c} \|_{1} + \lambda \| e_{T^c} \|_1 \right ) + \sqrt{2} \tau (s+\lambda^2 k)^{1/4} \epsilon \right )
%\\
% = & \left ( \frac{\rho}{\sqrt{s+\lambda^2 k}}  + \frac{1}{\sqrt{\min \{ \sqrt{s} , \lambda \sqrt{k} \}}}  \frac{\sqrt{\rho+1/2}}{(s+\lambda^2 k)^{1/4}} \right ) \left ( \| v_{S^c} \|_{1} + \lambda \| e_{T^c} \|_1 \right )
%\\
%& +  \tau \left ( 2 + \frac{\sqrt{2}  (s+\lambda^2 k)^{1/4}}{\sqrt{\min \{ \sqrt{s} , \lambda \sqrt{k} \}}} \right ) \epsilon
%}
%Using the definition of $\eta$ now gives
%\bes{
%  \sqrt{\| v \|^2_{2} + \| e \|^2_{2}} \leq \left ( \rho + \sqrt{\rho+1/2} \eta^{1/4} \right ) \frac{\| v \|_{1} + \lambda \| e \|_1}{\sqrt{s+\lambda^2k}} + \left ( 2 + \sqrt{2} \eta^{1/4} \right ) \epsilon \annote{\tau}.
%}
%To get the result we now use \R{l1_err_bound} to notice that
%\bes{
% \frac{\| v \|_{1} + \lambda \| e \|_1}{\sqrt{s+\lambda^2k}} \leq C_1 \frac{\sigma_{s}(x)_1 + \lambda \sigma_k(c)_1 }{\sqrt{s+\lambda^2 k}} + C_2 \epsilon \leq C_1 \left ( \frac{\sigma_s(x)_1}{\sqrt{s}} + \frac{\sigma_k(c)_1}{\sqrt{k}} \right ) + C_2 \epsilon,
%}
%as required.
}
%\annote{(The new computation above makes the constants a bit sharper, which is pretty irrelevant, but it also does something qualitatively satisfying: in the previous estimates, $C_3\sim 1$ when $\rho \downarrow 0$. However, the new computation has $C_3 \sim \sqrt{\rho}$ for small $\rho$. Thus there is some consistency of the constants $\rho$ and $\tau$ defined in \eqref{l2_rNSP_def} and $C_3$ and $C_4$ above.)}


\subsection{The Restricted Isometry Property for the sparse corruptions problem}\label{sec:rip}


The robust NSP is typically difficult to prove directly.  Hence we now introduce the Restricted Isometry Property (RIP) for the sparse corruptions problem, and show that it implies the robust NSP.  Note that this has been defined previously in \cite[Defn.\ 2.1]{LiCorruptionsConstrApprox}.

\defn{
\label{d:RIPcorruptions}
Let $1 \leq s \leq N$, $1 \leq k \leq m \leq N$ and $A \in \bbC^{m \times N}$.  The $(s,k)^{\rth}$ Restricted Isometry Constant (RIC) $\delta = \delta_{s,k}$ of the matrix $A$ is the smallest constant such that
\bes{
(1-\delta) \left ( \| x \|^2_{2} + \| c \|^2_{2} \right ) \leq \| A x  + c \|^2_{2} \leq (1+\delta) \left ( \| x \|^2_{2} + \| c \|^2_{2} \right )
}
for all $x \in \Sigma_{s}$ and $c \in \Sigma_{k}$.  If $0 < \delta_{s,k} < 1$ then we say that $A$ has the Restricted Isometry Property (RIP) of order $(s,k)$.
}


%\subsubsection{Main result}
Our first result is the following:

\lem{
\label{l:RIP_implies_rNSP}
Let $1 \leq s \leq N$, $1 \leq k \leq m \leq N$, $\lambda > 0$ and $A \in \bbC^{m \times N}$.  If $A$ satisfies the RIP of order $(2s,2k)$ with constant
\be{
\label{delta_cond}
\delta_{2s,2k} < \frac{1}{\sqrt{1 + \left ( \frac{1}{2 \sqrt{2}} + \sqrt{\eta} \right )^2 }},
}
where $\eta$ is as in \R{eta_def}, then $A$ satisfies the $\ell^2$-robust NSP of order $(s,k)$ with weight $\lambda$ and constants $0 < \rho < 1$ and $\tau > 0$ depending only on $\delta_{2s,2k}$.
}
The proof of this result is given next. Combining this lemma with Theorem \ref{t:rNSP_stable_robust} now yields our main result:
\thm{
\label{t:RIP_stable_robust}
Let $1 \leq s \leq N$, $1 \leq k \leq m$ and $\lambda > 0$ and suppose that $A \in \bbC^{m \times N}$ satisfies the RIP of order $(2s,2k)$ with constant $\delta_{2s, 2k}$ satisfying \eqref{delta_cond}
%\be{
%\label{delta_cond2}
%\delta_{2s,2k} < \frac{1}{\sqrt{1 + \left ( \frac{1}{2 \sqrt{2}} + \sqrt{\eta} \right )^2 }},
%}
and $\eta$ as in \eqref{eta_def}.
%\bes{
%\eta = \eta_{s,k}(\lambda) = \frac{s+\lambda^2k}{\min \{ s , \lambda^2 k \} }.
%}
Let $x \in \bbC^N$, $c \in \bbC^m$, $y \in \bbC^m$ and $\epsilon > 0$ be such that $\| A x + c - y \|_{2} \leq \epsilon$, and suppose that  $(\hat{x},\hat{c})$ is a minimizer of
\bes{
\min_{z \in \bbC^N, d \in \bbC^m} \| z \|_{1} + \lambda \| d \|_{1}\ \mbox{subject to $\| A z + d - y \|_{2} \leq \epsilon$},
}
%where $y \in \bbC^m$ satisfies $\| A x + c - y \|_{2} \leq \eta$ for some $\epsilon > 0$.  
Then
\begin{align*}
  \| x - \hat{x} \|_{1} + \lambda \| c - \hat{c} \|_{1} &\leq C_1 \left( \sigma_{s}(x)_1 + \lambda \sigma_{k}(c)_1 \right) + C_2\sqrt{s+\lambda^2 k} \epsilon,\\ 
  \| x - \hat{x} \|_{2} + \| c - \hat{c} \|_{2} &\leq C_3 \left ( 1 + \eta^{1/4} \right )\left( \frac{\sigma_{s}(x)_1}{\sqrt{s}} + \frac{\sigma_{k}(c)_1}{\sqrt{k}} \right) + C_4 \left ( 1 + \eta^{1/4} \right ) \epsilon,
\end{align*}
where the constants $C_1,C_2,C_3,C_4$ depend on $\delta_{2s,2k}$ only.%, and
%\bes{
%%\label{eta_def}
%\eta = \eta_{s,k}(\lambda) = \frac{s+\lambda^2k}{\min \{ s , \lambda^2 k \} }.
%}
}










%\subsubsection{Proof of Lemma \ref{l:RIP_implies_rNSP}}\label{sec:lemma-proof}


We now prove Lemma \ref{l:RIP_implies_rNSP}.  We first require the following:
\lem{
\label{l:disjoint_inner_product_RIP}
Let $1 \leq s \leq N$, $1 \leq k \leq m \leq N$, and let $A \in \bbC^{m \times N}$ satisfy the RIP of order $(2s, 2k)$ with constant $\delta_{2s, 2k}$.  Suppose that $x \in \Sigma_{s}$ and $c \in \Sigma_{k}$ are such that
\bes{
\nm{A x + c}^2_{2} - \left ( \| x \|^2_{2} + \| c \|^2_{2} \right ) = t  \left ( \| x \|^2_{2} + \| c \|^2_{2} \right ),
}
for some $t$ with $0 \leq |t| \leq \delta_{2s,2k}$. If $z \in \Sigma_s$ and $d \in \Sigma_k$ are orthogonal to $x$ and $c$, respectively, then
\bes{
\left | \ip{A x + c}{A z + d} \right | \leq \sqrt{\delta^2_{2s,2k} - t^2} \sqrt{\| x \|^2_{2} + \| c \|^2_{2}} \sqrt{\| z \|^2_{2} + \| d\|^2_{2}}.
}
}
\prf{
Assume that $\| x \|^2_{2} + \| c \|^2_{2} = \| z \|^2_{2} + \| d\|^2_{2} = 1$ without loss of generality.  Let $\alpha,\beta \in \bbR$ and $\gamma \in \bbC$ and notice that $\alpha x + \gamma z, \beta x - \gamma z \in \Sigma_{2s}$ and $\alpha c + \gamma d , \beta c - \gamma d \in \Sigma_{2k}$.  Therefore
\eas{
\nm{A (\alpha x + \gamma z ) +( \alpha c + \gamma d) }^2_{2} &\leq \left ( 1 + \delta_{2s,2k} \right ) \left ( \nm{\alpha x + \gamma z }^2_2 + \nm{\alpha c + \gamma d }^2_{2} \right )
\\
& = \left ( 1 + \delta_{2s,2k} \right ) \left (  \alpha^2 \left ( \nm{x}^2_2 + \nm{c}^2_2 \right ) + |\gamma|^2 \left ( \nm{z}^2_2 + \nm{d}^2_2 \right ) \right )
\\
& = \left ( 1 + \delta_{2s,2k} \right )  \left ( \alpha^2 + | \gamma |^2 \right ).
}
Note that in the second step we use orthogonality of the vectors $x$ and $z$ and $c$ and $d$.  Similarly,
\bes{
\nm{A (\beta x - \gamma z ) +( \beta c - \gamma d) }^2_{2} \geq \left ( 1 - \delta_{2s,2k} \right ) \left ( \beta^2 + |\gamma|^2 \right ).
}
Subtracting the second equation from the first gives
\ea{
\nm{A (\alpha x + \gamma z ) +( \alpha c + \gamma d) }^2_{2} -& \nm{A (\beta x - \gamma z ) +( \beta c - \gamma d) }^2_{2} \nn
\\
& \leq \left ( 1 + \delta_{2s,2k} \right )  \left ( \alpha^2 + | \gamma |^2 \right ) - \left ( 1 - \delta_{2s,2k} \right ) \left ( \beta^2 + |\gamma|^2 \right ) \nn
\\
& = \delta_{2s,2k} \left ( \alpha^2 + \beta^2 + 2 | \gamma |^2 \right ) + \alpha^2 - \beta^2. \label{diff_upper_bd}
}
On the other hand
\eas{
\nm{A (\alpha x + \gamma z ) +( \alpha c + \gamma d) }^2_{2} - &\nm{A (\beta x - \gamma z ) +( \beta c - \gamma d) }^2_{2}
\\
=& \ \alpha^2 \nm{A x + c }^2_{2} + | \gamma |^2 \nm{A z + d }^2_{2} + 2 \Re \ip{\alpha(Ax+c)}{\gamma(A z + d ) }
\\
& \ - \beta^2 \nm{A x + c }^2_{2} - | \gamma |^2 \nm{A z + d }^2_{2} + 2 \Re \ip{\beta(Ax+c)}{\gamma(Az+d)}
\\
=& \left ( \alpha^2 - \beta^2 \right ) \nm{A x + c }^2_{2} + 2 (\alpha+\beta) \Re \left ( \bar{\gamma} \ip{Ax+c}{Az+d} \right )
\\
= & \left ( \alpha^2 - \beta^2 \right ) (1+t)  +  2 (\alpha+\beta) \Re \left ( \bar{\gamma} \ip{Ax+c}{Az+d} \right ).
}
Combining this with \R{diff_upper_bd} gives
\bes{
\left ( \alpha^2 - \beta^2 \right ) (1+t)  +  2 (\alpha+\beta) \Re \left ( \bar{\gamma} \ip{Ax+c}{Az+d} \right ) \leq \delta_{2s,2k} \left ( \alpha^2 + \beta^2 + 2 | \gamma |^2 \right ) + \alpha^2 - \beta^2.
}
Now let $\gamma$ be such that $| \gamma | = 1$ and $\Re \left ( \bar{\gamma} \ip{Ax+c}{Az+d} \right ) = | \ip{Ax+c}{Az+d}|$.  Then, after rearranging, we get
\bes{
| \ip{Ax+c}{Az+d}| \leq \frac{\left ( \delta_{2s,2k} - t\right ) \alpha^2 +\left ( \delta_{2s,2k} +t\right ) \beta^2 + 2 \delta_{2s,2k} }{2(\alpha+\beta)} .
}
We now seek values $\alpha$ and $\beta$ which minimize the right-hand side of this expression.  If $t = \delta_{2s,2k}$ then the minimal value $0$ is attained by setting $\beta =0 $ and letting $\alpha \rightarrow \infty$.  Conversely, if $t < \delta_{2s,2k}$ the minimal value is attained when $\alpha = \sqrt{\frac{\delta_{2s,2k}+t}{\delta_{2s,2k}-t}}$ and $\beta = \frac{1}{\alpha}$.  This gives
\bes{
| \ip{Ax+c}{Az+d}| \leq \sqrt{\delta^2_{2s,2k} - t^2},
}
which completes the proof.
}


\prf{[Proof of Lemma \ref{l:RIP_implies_rNSP}]
Let $x \in \bbC^N$ and $c \in \bbC^m$.  To prove the $\ell^2$-robust NSP for $A$ it is enough to show that \R{l2_rNSP_def} holds when $S = S_0$ is the index set of the $s$ largest coefficients of $x$ in absolute value and $T = T_0$ is the set of the $k$ largest values of $c$ in absolute value.  Given $S_0$, let $S_1$ be the index set of the next $s$ largest coefficients of $x$ in absolute value, $S_2$ be the index set of the next $s$ largest coefficients and so on.  Define $T_1,T_2,\ldots$ in a similar way.  We now have the following:
\ea{
\nm{A x_{S_0} + c_{T_0}}^2 &= \ip{A x_{S_0} + c_{T_0} }{A x_{S_0} + c_{T_0} } \nn
\\
& = \ip{A x_{S_0} + c_{T_0} }{A x + c} - \sum_{j \geq 1} \ip{A x_{S_0} + c_{T_0} }{A x_{S_j} + c_{T_j}}.\label{RIP_sum_split}
}
Let $0 \leq |t| \leq \delta_{2s,2k}$ be such that
\be{
\label{t_def}
\nm{A x_{S_0} + c_{T_0}}^2_2 = (1+t) \left ( \nm{x_{S_0}}^2_2 + \nm{c_{T_0}}^2_2 \right ),
}
and note that this gives
\be{
\label{RIP_sum_split_term1}
\left | \ip{A x_{S_0} + c_{T_0} }{A x + c} \right | \leq \sqrt{1+t} \sqrt{ \nm{x_{S_0}}^2_2 + \nm{c_{T_0}}^2_2 } \nm{A x + c}_{2}.
}
For the second term of \R{RIP_sum_split}, we use the disjointness of $S_0$ and $S_j$ and $T_0$ and $T_j$ for $j \geq 1$ in combination with Lemma \ref{l:disjoint_inner_product_RIP} to get
\ea{
\left | \sum_{j \geq 1} \ip{A x_{S_0} + c_{T_0} }{A x_{S_j} + c_{T_j}} \right | &\leq \sqrt{\delta^2_{2s,2k} - t^2} \sqrt{ \nm{x_{S_0}}^2_2 + \nm{c_{T_0}}^2_2 }  \sum_{j \geq 1}  \sqrt{ \nm{x_{S_j}}^2_2 + \nm{c_{T_j}}^2_2 } \nn
\\
& \leq \sqrt{\delta^2_{2s,2k} - t^2} \sqrt{ \nm{x_{S_0}}^2_2 + \nm{c_{T_0}}^2_2 }  \left ( \sum_{j \geq 1} \| x_{S_j} \|_{2} + \sum_{j \geq 1} \nm{c_{T_j}}_2 \right ). \label{RIP_sum_split_term2}
}
Let $x^{+}_{j}$ and $x^{-}_{j}$ be the largest entries of $x_{S_j}$ in absolute value.  Then, by \cite[Lem.\ 6.14]{FoucartRauhutCSbook}, we have
\eas{
\sum_{j \geq 1} \| x_{S_j} \|_{2} & \leq \sum_{j \geq 1} \left ( \frac{\nm{x_{S_j}}_1}{\sqrt{s}} + \frac{\sqrt{s}}{4} \left ( x^{+}_{j} - x^{-}_{j} \right ) \right )
\\
& \leq \frac{\nm{x_{S^c_0}}_1}{\sqrt{s}} + \frac{\sqrt{s}}{4} \sum_{j \geq 1} \left ( x^{+}_{j} - x^{+}_{j+1} \right )
%\\
%& 
\leq \frac{\nm{x_{S^c_0}}_1}{\sqrt{s}} + \frac{\sqrt{s}}{4} x^{+}_{1}
%\\
%& 
\leq \frac{\nm{x_{S^c_0}}_1}{\sqrt{s}} + \frac{1}{4} \| x_{S_0} \|_{2}.
}
Similarly,
\bes{
\sum_{j \geq 1} \nm{c_{T_j}}_2 \leq \frac{\nm{c_{T^c_0}}_1}{\sqrt{k}} + \frac{1}{4} \nm{c_{T_0}}_2 \leq  \frac{\lambda \nm{c_{T^c_0}}_1}{\lambda \sqrt{k}} + \frac{1}{4} \nm{c_{T_0}}_2 ,
}
which gives
\bes{
\sum_{j \geq 1} \| x_{S_j} \|_{2} + \sum_{j \geq 1} \nm{c_{T_j}}_2 \leq \frac{1}{\min \left \{ \sqrt{s} , \lambda \sqrt{k} \right \} } \left ( \nm{x_{S^c_0}}_1 + \lambda \nm{c_{T^c_0}}_1 \right ) + \frac14 \left ( \| x_{S_0} \|_{2} + \nm{c_{T_0}}_2 \right ).
}
Therefore, combining this with \R{RIP_sum_split}, \R{t_def}, \R{RIP_sum_split_term1} and \R{RIP_sum_split_term2} yields
\eas{
(1+t) &\sqrt{\nm{x_{S_0}}^2_2 + \nm{c_{T_0}}^2_2} \leq  \sqrt{1+t} \nm{A x + c}_{2}
\\
& + \sqrt{\delta^2_{2s,2k} - t^2} \left ( \frac{1}{\min \left \{ \sqrt{s} , \lambda \sqrt{k} \right \} } \left ( \nm{x_{S^c_0}}_1 + \lambda \nm{c_{T^c_0}}_1 \right ) + \frac14 \left ( \| x_{S_0} \|_{2} + \nm{c_{T_0}}_2 \right ) \right ).
}
Consider the function $g(t) = \frac{\delta^2_{2s,2k} - t^2}{(1+t)^2}$, where $0 \leq t \leq \delta_{2s,2k}$.  This function attains its maximum value at $t = - \delta^2_{2s,2k}$ and takes value $\frac{\delta^2_{2s,2k}}{1-\delta^2_{2s,2k}}$ there.  Additionally $\frac{1}{\sqrt{1+t}} \leq \frac{1}{\sqrt{1-\delta_{2s,2k}}}$.  Hence we get
\eas{
\sqrt{\nm{x_{S_0}}^2_2 + \nm{c_{T_0}}^2_2} &\leq \frac{1}{\sqrt{1-\delta_{2s,2k}}} \nm{Ax+c}_2
\\
 +& \frac{\delta_{2s,2k}}{\sqrt{1-\delta^2_{2s,2k}}} \left ( \frac{1}{\min \left \{ \sqrt{s} , \lambda \sqrt{k} \right \} } \left ( \nm{x_{S^c_0}}_1 + \nm{c_{T^c_0}}_1 \right ) + \frac14 \left ( \| x_{S_0} \|_{2} + \nm{c_{T_0}}_2 \right ) \right ).
}
After noting that $\| x_{S_0} \|_{2} + \nm{c_{T_0}}_2 \leq \sqrt{2} \sqrt{\nm{x_{S_0}}^2_2 + \nm{c_{T_0}}^2_2}$ and rearranging, we obtain
\bes{
  \sqrt{\nm{x_{S_0}}^2_2 + \nm{c_{T_0}}^2_2} \leq \frac{\rho}{\sqrt{s + \lambda^2 k}} \left ( \nm{x_{S^c_0}}_1 + \nm{c_{T^c_0}}_1 \right ) + \tau \nm{A x + c}_{2},
}
where
\be{\label{eq:rho-tau-def}
\rho = \frac{2 \sqrt{2} \delta_{2s,2k}}{2\sqrt{2} \sqrt{1-\delta^2_{2s,2k}} - \delta_{2s,2k}} \sqrt{\eta} ,\quad \tau = \frac{2 \sqrt{2} \sqrt{1+\delta_{2s,2k}} }{2\sqrt{2} \sqrt{1-\delta^2_{2s,2k}} - \delta_{2s,2k}}.
}
To complete the proof we note that $\rho, \tau > 0$ provided $\delta_{2s,2k} < \sqrt{8/9}$.  This holds by assumption, since $\eta \geq 2$ and therefore the condition \R{delta_cond} implies that $\delta_{2s,2k} < \sqrt{8/33} < \sqrt{8/9}$.  Also, after rearranging we see that $\rho < 1$ if
\bes{
\left ( 1 + \left ( \frac{1}{2 \sqrt{2}} + \sqrt{\eta} \right )^2 \right ) \delta^2_{2s,2k} < 1,
}
which again holds by assumption.
}

%\annote{Here is an alternative derivation of a different version of the above. We essentially seek to show an improved version of \eqref{eq:delta-condition-1}. The punch line is that if we choose $\lambda \geq \sqrt{\frac{s}{k}}$, then we only need 
%  \begin{align}\label{eq:delta-condition-2}
%    \delta_{2s, 2k} < \sqrt{\frac{8}{17 + 4\sqrt{2}}} \approx 0.59,
%  \end{align}
%  which is even better than \eqref{eq:delta-condition-1}.
%  
%  Consider a sharper definition of $\rho$, 
%  \begin{align*}
%  \rho = \frac{2 \sqrt{2} \delta_{2s,2k}}{2\sqrt{2} \sqrt{1-\delta^2_{2s,2k}} - \delta_{2s,2k}} \sqrt{\nu} ,\quad \tau = \frac{2 \sqrt{2} \sqrt{1+\delta_{2s,2k}} }{2\sqrt{2} \sqrt{1-\delta^2_{2s,2k}} - \delta_{2s,2k}},
%  \end{align*}
%  where $\nu \coloneqq \frac{1}{\min \left\{s, \lambda^2 k \right\}}$. The only difference between the above and \eqref{eq:rho-tau-def} is that $\eta$ has been replaced by $\nu$. Most of the same conclusions hold: since $\rho = \frac{\delta_{2s,2k}}{\sqrt{1 + \delta_{2s,2k}}} \sqrt{\nu} \tau$, then both $\rho, \tau > 0$ when $2\sqrt{2} \sqrt{1-\delta^2_{2s,2k}} - \delta_{2s,2k} > 0$, i.e., when 
%  \begin{align*}
%    \delta_{2s,2k} < \sqrt{\frac{8}{9}}.
%  \end{align*}
%  Under this condition, we can ensure $\rho < 1$ if in addition
%  \begin{align}\label{eq:delta-nu-condition}
%    \delta_{2s,2k} < \frac{1}{\sqrt{1 + \left(\frac{1}{2\sqrt{2}} + \sqrt{\nu}\right)^2}}.
%  \end{align}
%  This condition on $\delta_{2s,2k}$ is strictly better (looser) than \eqref{delta_cond} since $\nu < \eta$. Since the ultimate goal is to allow $\delta$ to be as large as possible, we seek to minimize $\nu$. Thus, we seek to choose $\lambda$ such that 
%  \begin{align*}
%    \max \left\{ \frac{1}{s}, \frac{1}{\lambda^2 k} \right\}
%  \end{align*}
%  is minimized. This happens when $\lambda \geq \sqrt{\frac{s}{k}}$, and this ensures that $\nu = \frac{1}{s} \leq 1$. The condition \eqref{eq:delta-condition-2} is equivalent to 
%  \begin{align*}
%    \delta_{2s,2k} < \frac{1}{\sqrt{1 + \left(\frac{1}{2\sqrt{2}} + 1\right)^2}} \leq \frac{1}{\sqrt{1 + \left(\frac{1}{2\sqrt{2}} + \frac{1}{\sqrt{s}} \right)^2}} = \frac{1}{\sqrt{1 + \left(\frac{1}{2\sqrt{2}} + \sqrt{\nu} \right)^2}}
%  \end{align*}
%  Thus, assuming \eqref{eq:delta-condition-1} ensures \eqref{eq:delta-nu-condition} and hence the $\ell^2$-robust NSP.
%}


\rem{
\label{r:RIPinlevels}
The RIP for the sparse corruptions problem is a special case of the RIP in levels (RIPL), introduced in \cite{BastounisHansen}.  The RIPL applies to vectors that are sparse in levels; namely, having different amounts of sparsity in different (but fixed) sections of the vector.  In the context of the sparse corruptions problem, this corresponds to the concatenated vector $z = [ x ; c ]$, which is $s$-sparse in its first $N$ entries and $k$-sparse in its remaining $m$ entries.  As a general tool, sparsity in levels has been used in the context of compressive imaging \cite{AHPRBreaking,OptimalSamplingQuest,AsymptoticCS}, radar \cite{Dorsch2016} and multi-sensor acquisition \cite{AdcockChunParallel}.  It is interesting that the same model also occurs naturally in the, seemingly unrelated, sparse corruptions problem.  We note in passing that Theorems \ref{t:rNSP_stable_robust} and \ref{t:RIP_stable_robust} follow a similar approach to that of \cite{BastounisHansen} with some changes made to incorporate the weighted optimization problem.  
}


\subsection{Matrices that satisfy the RIP for sparse corruptions}\label{sec:rip-2}

We first recall the classical RIP for sparse vectors:

\defn{
Let $1 \leq s \leq N$ and $A \in \bbC^{m \times N}$.  The $s^{\rth}$ Restricted Isometry Constant (RIC) $\delta = \delta_{s}$ of the matrix $A$ is the smallest constant such that
\bes{
(1-\delta)\nm{x}^2_2 \leq \| A x \|^2_{2} \leq (1+\delta) \nm{x}^2_2,
}
for all $x \in \Sigma_{s}$.  If $0 < \delta_{s} < 1$ then we say that $A$ has the Restricted Isometry Property (RIP) of order $s$.
}

To distinguish it from the RIP for the sparse corruptions problem (Definition \ref{d:RIPcorruptions}), we shall refer to this as the \textit{RIP for sparse vectors}.

\lem{
\label{l:deltask_deltas_sigmask}
Let $1 \leq s \leq N$, $1 \leq k \leq m$, $A \in \bbC^{m \times N}$ and define
\be{
\label{sigma_def}
\sigma_{s,k} = \max_{\substack{S \subseteq \{1,\ldots,N\}, |S| = s \\ T \subseteq \{1,\ldots,m \}, |T| = k }} \| A_{S,T} \|_{2},
}
where $A_{S,T} \in \bbC^{|T| \times |S|}$ is the submatrix of $A$ with entries $\{ A_{ij} \}_{i \in T, j \in S}$.  Suppose that $A$ has the RIP for sparse vectors with constant $\delta_s$ and that $\sigma_{s,k} < \sqrt{1-\delta_{s}}$.  Then $A$ has the RIP of order $(s,k)$ for the sparse corruptions problem with constant
\bes{
\delta_{s,k} =  \frac{\delta_s + \sqrt{\delta^2_s+4 \sigma^2_{s,k}}}{2}.
}
In other words,
\bes{
\left ( 1 - \delta_{s,k} \right ) \left ( \nm{x}^2_{2} + \nm{c}^2_2 \right ) \leq \nm{A x + c}^2_{2} \leq \left ( 1 + \delta_{s,k} \right ) \left ( \nm{x}^2_{2} + \nm{c}^2_2 \right )
}
for all $x \in \Sigma_{s}$ and $c \in \Sigma_k$.
}
\prf{
Let $x \in \Sigma_{s}$ and $c \in \Sigma_k$ and write $S = \supp(x)$ and $T = \supp(c)$.  Then
\bes{
\nm{A x + c}^2_{2} = \nm{A x}^2_{2} + \nm{c}^2_{2} + 2 \Re \ip{A_{S,T} x}{c}.
}
By Young's inequality
\bes{
2\left | \ip{A_{S,T} x}{c} \right | \leq 2 \| A_{S,T} \|_{2} \nm{x}_2 \nm{c}_2 \leq \| A_{S,T} \|_{2} \left ( \nm{x}^2_2 / \epsilon + \epsilon \nm{c}^2_2 \right ),
}
for any $\epsilon > 0$.  Hence
\bes{
\left ( 1 - \delta_s - \sigma_{s,k}/\epsilon \right ) \nm{x}^2_2 + \left ( 1 - \sigma \epsilon \right ) \nm{c}^2_{2} \leq \nm{A x + c }^2_{2} \leq \left ( 1 + \delta_s + \sigma_{s,k}/\epsilon \right ) \nm{x}^2_2 + \left ( 1 + \sigma \epsilon \right ) \nm{c}^2_{2}.
}
Solving the equation $\delta_s + \sigma_{s,k}/\epsilon = \sigma_{s,k} \epsilon$ yields the value $\epsilon = \frac{\delta_s + \sqrt{\delta^2_s + 4 \sigma^2_{s,k}}}{2 \sigma}$, and substituting this value of $\epsilon$ into the previous expression yields the proof.
}

This result shows that any matrix satisfying the RIP for sparse vectors also satisfies the RIP for the sparse corruptions problem, provided the all $k \times s$ submatrices have small spectral norm.

\subsubsection{Gaussian random matrices}
Gaussian random matrices in the context of the sparse corruptions problem were considered in \cite{LiCorruptionsConstrApprox}.  The following result essentially recaps the main result for this case given therein.  We include a short proof for completeness:

\thm{
\label{t:GaussCorruption}
Let $0 < \delta , \epsilon < 1$, $1 \leq s \leq$, $1 \leq k \leq m$ and suppose that
\ea{
m &\gtrsim \delta^{-2} \left ( s \cdot \log(2N/s) + \log(2 \epsilon^{-1}) \right ), \label{mGaussRIP}
\\
m &\gtrsim \delta^{-2} \cdot k \cdot \log(\delta^{-1}). \label{mGaussSigma}
}
Let $A \in \bbC^{m \times N}$ be a matrix whose entries are independent Gaussian random variables with mean zero and variance $1$.  Then with probability at least $1-\epsilon$, the matrix $\frac{1}{\sqrt{m}} A$ has the RIP for the sparse corruptions problem of order $(s,k)$ with constant $\delta_{s,k} \leq \delta$.
}
\prf{
Lemma \ref{l:deltask_deltas_sigmask} asserts that $A$ has the RIP of order $(s,k)$ for the sparse corruptions problem with constant $\delta_{s,k} \leq \delta$ provided (i) $A$ has the RIP of order $s$ with $\delta_{s} \leq \delta/\sqrt{2}$ and (ii) the constant $\sigma_{s,k}$ defined in \R{sigma_def} satisfies $\sigma_{s,k} \leq \delta/(2 \sqrt{2})$.  Hence, by the union bound it suffices to show that \R{mGaussRIP} and \R{mGaussSigma} imply both (i) and (ii) separately with probabilities at least $1-\epsilon/2$.  Due to a standard result in compressed sensing (see, for example, \cite[Thm.\ 9.2]{FoucartRauhutCSbook}), property (i) holds with probability at least $1-\epsilon/2$ whenever the condition \R{mGaussRIP} is satisfied.  We now consider property (ii).  First, notice that $\sigma_{s,k}$ is increasing in $k$.  Therefore, we may assume that $ k \asymp \delta^2 \cdot m$, i.e.\ $k \gtrsim \delta^{2} \cdot m$ and $k \lesssim \delta^{2} \cdot m$.  Fix subsets $S \subseteq \{1,\ldots,N\}$ and $T \subseteq \{1,\ldots,m\}$ with $|S| = s$ and $|T|  =k$.  Then, due to a known result for singular values of random Gaussian matrices (see, for example, \cite[Cor.\ 5.35]{Vershynin:bookCh}), we have
\bes{
\bbP \left ( \nm{A_{S,T}}_2 \geq \sqrt{s} + \sqrt{k} + t \right ) \leq 2 \exp(-t^2/2).
}
The conditions \R{mGaussRIP} and \R{mGaussSigma} imply that $\sqrt{s/m} \leq \delta  /(6\sqrt{2})$ and $\sqrt{k/m} \leq \delta  /(6\sqrt{2})$.  Hence, by the union bound
\eas{
\bbP \left ( \sigma_{s,k} > \delta / (2 \sqrt{2}) \right )  \leq \left ( \begin{array}{c} N \\ s \end{array} \right ) \left ( \begin{array}{c} m \\ k \end{array} \right ) \exp(-m\delta^2/48) \leq \left ( \frac{\E N}{s} \right )^s \left ( \frac{\E m}{k} \right )^k \exp(-m\delta^2/48).
}
In particular, $\bbP \left ( \sigma_{s,k} > \delta / (2 \sqrt{2}) \right )  \leq \epsilon/2$ provided
\bes{
m \geq 48 \cdot \delta^{-2} \left ( s \log(\E N/s) + k \log(\E m/k) + \log(2 \epsilon^{-1}) \right ).
}
Since $k \asymp \delta^2 \cdot m$, we have $\log(\E m / k) \lesssim \log(2 \delta^{-1})$.  Hence this condition is implied by \R{mGaussRIP} and \R{mGaussSigma}.  This establishes property (ii) and completes the proof.
}

This result asserts that Gaussian random matrices can recover a fixed fraction $k/m \leq c$ of corruptions (see \R{mGaussSigma}) and (up to constants) the same level of sparsity $s$ as in the uncorrupted case (see \R{mGaussRIP}).




\subsubsection{Bounded orthonormal systems}\label{sec:bos}

Gaussian random matrices, while mathematically appealing, are of little relevance to multivariate approximation using Polynomial Chaos expansions.  In this case, a more suitable framework is that of bounded orthonormal systems (see, for example, \cite[Chpt.\ 12]{FoucartRauhutCSbook}):

Let $D$ be a domain with a probability measure $\nu$ and $\phi_1,\ldots,\phi_N$ be an orthonormal system of complex-value functions in $L^2(D)$.  Recall that this system is bounded if
\bes{
\nm{\phi_i }_{L^\infty} = \sup_{\xi \in D} | \phi_i(\xi) | \leq K
}
Given such a system, we construct the measurement matrix $A$ as
\be{
\label{ABOS}
A = \frac{1}{\sqrt{m}} \left \{ \phi_j(\xi_i) \right \}^{m,N}_{i=1,j=1} \in \bbC^{m \times N},
}
where $t_i$ are drawn independently at random according to the probability measure $\nu$.  

\thm{
\label{t:RIP_BOS}
Let $A \in \bbC^{m \times N}$ be the matrix of a bounded orthonormal system, $1 \leq s \leq N$ and $0 < \delta , \epsilon < 1$.  If
\bes{
m \gtrsim \delta^{-2} \cdot s \cdot \left ( \log^3(2s) \cdot \log(2N) + \log(\epsilon^{-1}) \right ),
}
then $A$ satisfies the RIP for sparse vectors with probability at least $1-\epsilon$.
}

We remark in passing that the logarithmic dependence in $s$ can be improved by one power, at the expense of a larger factor in $\delta^{-1}$ \cite{ChkifaDownwardsCS}.  However, this may not be best for the purposes of this paper, since in view of Theorem \ref{t:RIP_stable_robust}, $\delta^{-2}$ scales linearly in the parameter $\eta$ (see next).   

The following lemma estimates the constant $\sigma_{s,k}$ for matrices of the form \R{ABOS}:
\lem{
\label{l:BOSsigmaEst}
Let $A \in \bbC^{m \times N}$ be the matrix of a bounded orthonormal system, $1 \leq s,k \leq N$ and $\sigma_{s,k}$ be as in \R{sigma_def}.  Then
\bes{
\sigma_{s,k} \leq \sqrt{\frac{K^2 s k}{m}}.
}
}
\prf{
Fix subsets $S \subseteq \{1,\ldots,N\}$, $|S|=s$ and $T \subseteq \{1,\ldots,m\}$, $|T|=k$ and let $x \in \bbC^N$ and $c \in \bbC^m$ with $\supp(x) = S$ and $\supp(c) = T$.  Then
\eas{
\left | c^* A x \right |^2 &= \frac{1}{\sqrt{m}} \left | \sum_{i \in T} \overline{c_i} \sum_{j \in S} \phi_j(t_i) x_j \right | 
\\
& \leq \frac{1}{\sqrt{m}} \max_{i =1,\ldots,m} \left | \sum_{j \in S} \phi_j(t_i) x_j \right | \sum_{i \in T} | c_i | 
%\\
%& 
\leq \frac{K}{\sqrt{m}}  \| x \|_{1} \| c \|_{1}
%\\
%& 
\leq \sqrt{\frac{K^2 s k}{m}} \| x \|_{2} \| c \|_{2}.
}
Hence $\| P_{T} A P_{S} \|_{2} \leq \sqrt{\frac{K^2 s k}{m}}$.  This now gives the result.
}

With this in hand, we now deduce the following result:
\thm{
\label{t:BOS-RIP}
Let $1 \leq s \leq N$, $1 \leq k \leq m$, $0 < \delta,\epsilon < 1$ and suppose that
\be{
\label{mRIP1}
m \gtrsim \delta^{-2} \cdot K^2 \cdot s \cdot \left ( \log^3(2s) \cdot \log(2N) + \log(\epsilon^{-1}) \right ),
}
and 
\bes{
\label{mRIP2}
m \geq 8 \cdot \delta^{-2} \cdot K^2 \cdot s \cdot k.
}
Then, with probability at least $1-\epsilon$, $A$ has the RIP of order $(s,k)$ for the sparse corruptions problem with constant $\delta_{s,k} \leq \delta$.
}
\prf{
Theorem \ref{t:RIP_BOS} and \R{mRIP1} imply that $A$ has the RIP of order $s$ with $\delta_{s} \leq \delta/\sqrt{2}$ with probability at least $1-\epsilon$.  Moreover, Lemma \ref{l:BOSsigmaEst} and \R{mRIP2} imply that $\sigma_{s,k} \leq \delta/(2 \sqrt{2})$.  We now apply Lemma \ref{l:deltask_deltas_sigmask}.
%
%Lemma \ref{l:deltask_deltas_sigmask} asserts that $A$ has the RIP of order $(s,k)$ for the sparse corruptions problem with constant $\delta_{s,k} \leq \delta$ provided (i) $A$ has the RIP of order $s$ with $\delta_{s} \leq \delta/\sqrt{2}$ and (ii) the constant $\sigma_{s,k}$ defined in \R{sigma_def} satisfies $\sigma_{s,k} \leq \delta/(2 \sqrt{2})$.  Property (i) follows immediately from Theorem \ref{t:RIP_BOS} and \R{mRIP1}, and property (ii) follows from Lemma \ref{l:BOSsigmaEst} and \R{mRIP2}.
}

\rem{
This result asserts that the number of corruptions that can be tolerated is a fraction of $m/s$.  This is inferior to the case of Gaussian random measurements, where Theorem \ref{t:GaussCorruption} gives that a fraction of $m$ corruptions are permitted.  We conjecture, however, that a similar estimate can be proved for the bounded orthonormal systems case -- indeed, a nonuniform recovery result of this form was proved in \cite{LiCorruptionsConstrApprox} for the case of exactly sparse coefficients $x$ and corruptions $c$ with random sign sequences -- albeit with a substantially more sophisticated argument than the proof of Theorem \ref{t:GaussCorruption}.  In particular, while estimates for the singular values of matrices of bounded orthonormal systems are known \cite{Vershynin:bookCh}, they are more stringent than those for Gaussian random matrices.  Using these estimates and arguing via the union bound (as in the proof of Theorem \ref{t:GaussCorruption}) unfortunately results in an estimate similar to \R{mRIP2}.  We also note in passing that while there exist RIP estimates for quite general matrices under the sparsity in levels model \cite{LiAdcockRIP} (see Remark \ref{r:RIPinlevels}), these unfortunately do not apply to the setup of the sparse corruptions problem.
We therefore leave the problem of improving \R{mRIP2} for future work.  
}
%\GR{
%For your information: I have tried to derive a better estimate for $\sigma_{s,k}$ by using arguments similar to those used in the proof of the standard RIP for bounded orthonormal systems.  On a high-level, these types of proofs proceed as follows: i) show that $\bbE(\sigma_{s,k}) \lesssim 1$, ii) given (i), show that $\sigma_{s,k} \leq \sigma$ with high probability.  Right now I'm stuck in one key technical detail within part (i).  Part (ii) goes through without a problem.  If you're interested I can share my notes with you.
%}


%\subsubsection{Case studies: high-dimensional polynomial approximation}
%\GR{
%We suggestion here is we give the overall results in the context of Legendre/Chebyshev polynomial approximations.
%}


%\subsubsection{Discussion}
%Unfortunately, this lemma gives that the sufficient condition for recovery in the corruptions problem of the form
%\bes{
%m \gtrsim s(s+k) \times \mbox{log factors}
%}
%which is obviously suboptimal.  The aim is to improve this to just $s+k$.  An idea is to setup a similar approach as is used to prove the RIP for BOS.


\subsection{Strategy for choosing $\lambda$}\label{ss:lambda-strategy}
Regardless of the matrix $A$, our main theorems (Theorems \ref{t:RIP_stable_robust} and \ref{t:RIP_BOS}) suggest an optimal strategy for choosing the parameter $\lambda$.  Notice that the restricted isometry constant $\delta$ enters into the measurement condition in Theorem \ref{t:RIP_BOS} as $\delta^{-2}$.  Since Theorem \ref{t:RIP_stable_robust} requires that \R{delta_cond} holds, the measurement condition contains a factor that is at least as large as 
\bes{
 1 + \left ( \frac{1}{2 \sqrt{2}} + \sqrt{\eta} \right )^2.
}
We wish to minimize this factor so as to reduce the measurement condition as much as possible.
This can be done by minimizing $\eta$, which in turn yields the theoretically-optimal scaling
\be{
\label{eq:lambda-opt}
\lambda = \sqrt{\frac{s}{k}}.
}
Notice that this gives the value $\eta = 2$.  In particular, the condition \R{delta_cond} becomes
\be{\label{eq:delta-condition-1}
\delta_{2s,2k} < \sqrt{8/33} \approx 0.492,
}
with right-hand side independent of $s$ and $k$.  We remark in passing that the choice \R{eq:lambda-opt} is implicitly made in \cite{LiCorruptionsConstrApprox}.  However, the condition given in \cite[Lem.\ 2.3]{LiCorruptionsConstrApprox} is $\delta_{2s,2k} < 1/18 \approx 0.056$  which is significantly more stringent than \R{eq:lambda-opt}.  Moreover, \cite{LiCorruptionsConstrApprox} only considers exact sparsity, whereas Theorem \ref{t:RIP_stable_robust} also treats the case of stable recovery of inexactly sparse coefficients and corruptions.

%
%We will see later that for a typical matrix $A$ to satisfy the RIP with constant $\delta$ we require $m$ to scale like $\delta^{-2}$ in terms of $\delta$.  In other words, the measurement condition contains a factor
%\bes{
% 1 + \left ( \frac{1}{2 \sqrt{2}} + \sqrt{\eta} \right )^2.
%}
%Hence, the larger $\eta$ is the worse the measurement condition.  Seeking to minimize $\eta$, we therefore set $\lambda = \sqrt{s/k}$ which gives $\eta = 2$.  In this case, we note that the condition \R{delta_cond} becomes
%\be{\label{eq:delta-condition-1}
%\delta_{2s,2k} < \sqrt{8/33} \approx 0.492.
%}
%Note that this better than Li's result, which is $\delta_{2s,2k} < 1/18 \approx 0.056$ (see Lemma 2.3 of \cite{LiCorruptionsConstrApprox}).  Moreover, Theorem \ref{t:RIP_stable_robust} applies to all values of $\lambda$, unlike Li's result, and treats inexact sparsity.

%!TEX ROOT = ../../centralized_vs_distributed.tex

\section{{\titlecap{the centralized-distributed trade-off}}}\label{sec:numerical-results}

\revision{In the previous sections we formulated the optimal control problem for a given controller architecture
(\ie the number of links) parametrized by $ n $
and showed how to compute minimum-variance objective function and the corresponding constraints.
In this section, we present our main result:
%\red{for a ring topology with multiple options for the parameter $ n $},
we solve the optimal control problem for each $ n $ and compare the best achievable closed-loop performance with different control architectures.\footnote{
\revision{Recall that small (large) values of $ n $ mean sparse (dense) architectures.}}
For delays that increase linearly with $n$,
\ie $ f(n) \propto n $, 
we demonstrate that distributed controllers with} {few communication links outperform controllers with larger number of communication links.}

\textcolor{subsectioncolor}{Figure~\ref{fig:cont-time-single-int-opt-var}} shows the steady-state variances
obtained with single-integrator dynamics~\eqref{eq:cont-time-single-int-variance-minimization}
%where we compare the standard multi-parameter design 
%with a simplified version \tcb{that utilizes spatially-constant feedback gains
and the quadratic approximation~\eqref{eq:quadratic-approximation} for \revision{ring topology}
with $ N = 50 $ nodes. % and $ n\in\{1,\dots,10\} $.
%with $ N = 50 $, $ f(n) = n $ and $ \tau_{\textit{min}} = 0.1 $.
%\autoref{fig:cont-time-single-int-err} shows the relative error, defined as
%\begin{equation}\label{eq:relative-error}
%	e \doteq \dfrac{\optvarx-\optvar}{\optvar}
%\end{equation}
%where $ \optvar $ and $ \optvarx $ denote the the optimal and sub-optimal scalar variances, respectively.
%The performance gap is small
%and becomes negligible for large $ n $.
{The best performance is achieved for a sparse architecture with  $ n = 2 $ 
in which each agent communicates with the two closest pairs of neighboring nodes. 
This should be compared and contrasted to nearest-neighbor and all-to-all 
communication topologies which induce higher closed-loop variances. 
Thus, 
the advantage of introducing additional communication links diminishes 
beyond}
{a certain threshold because of communication delays.}

%For a linear increase in the delay,
\textcolor{subsectioncolor}{Figure~\ref{fig:cont-time-double-int-opt-var}} shows that the use of approximation~\eqref{eq:cont-time-double-int-min-var-simplified} with $ \tilde{\gvel}^* = 70 $
identifies nearest-neighbor information exchange as the {near-optimal} architecture for a double-integrator model
with ring topology. 
This can be explained by noting that the variance of the process noise $ n(t) $
in the reduced model~\eqref{eq:x-dynamics-1st-order-approximation}
is proportional to $ \nicefrac{1}{\gvel} $ and thereby to $ \taun $,
according to~\eqref{eq:substitutions-4-normalization},
making the variance scale with the delay.

%\mjmargin{i feel that we need to comment about different results that we obtained for CT and DT double-intergrator dynamics (monotonic deterioration of performance for the former and oscillations for the latter)}
\revision{\textcolor{subsectioncolor}{Figures~\ref{fig:disc-time-single-int-opt-var}--\ref{fig:disc-time-double-int-opt-var}}
show the results obtained by solving the optimal control problem for discrete-time dynamics.
%which exhibit similar trade-offs.
The oscillations about the minimum in~\autoref{fig:disc-time-double-int-opt-var}
are compatible with the investigated \tradeoff~\eqref{eq:trade-off}:
in general, 
the sum of two monotone functions does not have a unique local minimum.
Details about discrete-time systems are deferred to~\autoref{sec:disc-time}.
Interestingly,
double integrators with continuous- (\autoref{fig:cont-time-double-int-opt-var}) ad discrete-time (\autoref{fig:disc-time-double-int-opt-var}) dynamics
exhibits very different trade-off curves,
whereby performance monotonically deteriorates for the former and oscillates for the latter.
While a clear interpretation is difficult because there is no explicit expression of the variance as a function of $ n $,
one possible explanation might be the first-order approximation used to compute gains in the continuous-time case.
%which reinforce our thesis exposed in~\autoref{sec:contribution}.

%\begin{figure}
%	\centering
%	\includegraphics[width=.6\linewidth]{cont-time-double-int-opt-var-n}
%	\caption{Steady-state scalar variance for continuous-time double integrators with $ \taun = 0.1n $.
%		Here, the \tradeoff is optimized by nearest-neighbor interaction.
%	}
%	\label{fig:cont-time-double-int-opt-var-lin}
%\end{figure}
}

\begin{figure}
	\centering
	\begin{minipage}[l]{.5\linewidth}
		\centering
		\includegraphics[width=\linewidth]{random-graph}
	\end{minipage}%
	\begin{minipage}[r]{.5\linewidth}
		\centering
		\includegraphics[width=\linewidth]{disc-time-single-int-random-graph-opt-var}
	\end{minipage}
	\caption{Network topology and its optimal {closed-loop} variance.}
	\label{fig:general-graph}
\end{figure}

Finally,
\autoref{fig:general-graph} shows the optimization results for a random graph topology with discrete-time single integrator agents. % with a linear increase in the delay, $ \taun = n $.
Here, $ n $ denotes the number of communication hops in the ``original" network, shown in~\autoref{fig:general-graph}:
as $ n $ increases, each agent can first communicate with its nearest neighbors,
then with its neighbors' neighbors, and so on. For a control architecture that utilizes different feedback gains for each communication link
	(\ie we only require $ K = K^\top $) we demonstrate that, in this case, two communication hops provide optimal closed-loop performance. % of the system.}

Additional computational experiments performed with different rates $ f(\cdot) $ show that the optimal number of links increases for slower rates: 
for example, 
the optimal number of links is larger for $ f(n) = \sqrt{n} $ than for $ f(n) = n $. 
\revision{These results are not reported because of space limitations.}
% \vspace{-0.5em}
\section{Conclusion}
% \vspace{-0.5em}
Recent advances in multimodal single-cell technology have enabled the simultaneous profiling of the transcriptome alongside other cellular modalities, leading to an increase in the availability of multimodal single-cell data. In this paper, we present \method{}, a multimodal transformer model for single-cell surface protein abundance from gene expression measurements. We combined the data with prior biological interaction knowledge from the STRING database into a richly connected heterogeneous graph and leveraged the transformer architectures to learn an accurate mapping between gene expression and surface protein abundance. Remarkably, \method{} achieves superior and more stable performance than other baselines on both 2021 and 2022 NeurIPS single-cell datasets.

\noindent\textbf{Future Work.}
% Our work is an extension of the model we implemented in the NeurIPS 2022 competition. 
Our framework of multimodal transformers with the cross-modality heterogeneous graph goes far beyond the specific downstream task of modality prediction, and there are lots of potentials to be further explored. Our graph contains three types of nodes. While the cell embeddings are used for predictions, the remaining protein embeddings and gene embeddings may be further interpreted for other tasks. The similarities between proteins may show data-specific protein-protein relationships, while the attention matrix of the gene transformer may help to identify marker genes of each cell type. Additionally, we may achieve gene interaction prediction using the attention mechanism.
% under adequate regulations. 
% We expect \method{} to be capable of much more than just modality prediction. Note that currently, we fuse information from different transformers with message-passing GNNs. 
To extend more on transformers, a potential next step is implementing cross-attention cross-modalities. Ideally, all three types of nodes, namely genes, proteins, and cells, would be jointly modeled using a large transformer that includes specific regulations for each modality. 

% insight of protein and gene embedding (diff task)

% all in one transformer

% \noindent\textbf{Limitations and future work}
% Despite the noticeable performance improvement by utilizing transformers with the cross-modality heterogeneous graph, there are still bottlenecks in the current settings. To begin with, we noticed that the performance variations of all methods are consistently higher in the ``CITE'' dataset compared to the ``GEX2ADT'' dataset. We hypothesized that the increased variability in ``CITE'' was due to both less number of training samples (43k vs. 66k cells) and a significantly more number of testing samples used (28k vs. 1k cells). One straightforward solution to alleviate the high variation issue is to include more training samples, which is not always possible given the training data availability. Nevertheless, publicly available single-cell datasets have been accumulated over the past decades and are still being collected on an ever-increasing scale. Taking advantage of these large-scale atlases is the key to a more stable and well-performing model, as some of the intra-cell variations could be common across different datasets. For example, reference-based methods are commonly used to identify the cell identity of a single cell, or cell-type compositions of a mixture of cells. (other examples for pretrained, e.g., scbert)


%\noindent\textbf{Future work.}
% Our work is an extension of the model we implemented in the NeurIPS 2022 competition. Now our framework of multimodal transformers with the cross-modality heterogeneous graph goes far beyond the specific downstream task of modality prediction, and there are lots of potentials to be further explored. Our graph contains three types of nodes. while the cell embeddings are used for predictions, the remaining protein embeddings and gene embeddings may be further interpreted for other tasks. The similarities between proteins may show data-specific protein-protein relationships, while the attention matrix of the gene transformer may help to identify marker genes of each cell type. Additionally, we may achieve gene interaction prediction using the attention mechanism under adequate regulations. We expect \method{} to be capable of much more than just modality prediction. Note that currently, we fuse information from different transformers with message-passing GNNs. To extend more on transformers, a potential next step is implementing cross-attention cross-modalities. Ideally, all three types of nodes, namely genes, proteins, and cells, would be jointly modeled using a large transformer that includes specific regulations for each modality. The self-attention within each modality would reconstruct the prior interaction network, while the cross-attention between modalities would be supervised by the data observations. Then, The attention matrix will provide insights into all the internal interactions and cross-relationships. With the linearized transformer, this idea would be both practical and versatile.

% \begin{acks}
% This research is supported by the National Science Foundation (NSF) and Johnson \& Johnson.
% \end{acks}

\subsubsection*{Acknowledgments}
B. Adcock thanks Simone Brugiapaglia and Xiaodong Li for helpful discussions.  The authors acknowledge an anonymous referee whose report led to the investigations outlined in Remark \ref{rem:weights}.%\BLUE{I need to add my grants somewhere (possibly here), as I suspect do you as well.  For me, can you please add: ``BA and AB acknowledge the support of the Alfred P. Sloan Foundation and the Natural Sciences and Engineering Research Council of Canada through grant 611675.''.}

Sandia National Laboratories is a multimission laboratory managed and
operated by National Technology and Engineering Solutions of Sandia, LLC., a
wholly owned subsidiary of Honeywell International, Inc., for the
U.S. Department of Energy's National Nuclear Security Administration
under contract DE-NA-0003525. The views expressed in the article do not necessarily represent the views of the U.S. Department of Energy or the United States Government.

\bibliographystyle{abbrv}
\bibliography{CSCorruptionsBib}

\end{document}

%%% Local Variables:
%%% mode: latex
%%% TeX-master: CSCorruptionsv1
%%% End:
