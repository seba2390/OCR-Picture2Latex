\documentclass[10pt]{amsart}

\usepackage{amsmath,graphicx,bbm}
\usepackage{amsthm,verbatim}
\usepackage[footnotesize,bf]{caption}
\usepackage[bottom=1in,left=1in,right=1in,top=1in]{geometry}
\setlength\parskip{10pt}

\usepackage{color}
\usepackage{subfigure}
\usepackage[usenames,dvipsnames,svgnames,table]{xcolor}

\usepackage{lipsum}

\newcommand{\mathd}{\mathrm{d}}
\newcommand{\dfdx}[2]{\frac{\mathd #1}{\mathd #2}}
\newcommand{\dx}[1]{\mathd #1}
\newcommand{\pfpx}[2]{\frac{\partial #1}{\partial #2}}
\newcommand{\N}{\mathbbm{N}}
\newcommand{\R}{\mathbbm{R}}
\newcommand{\Z}{\mathbbm{Z}}
\newcommand{\nchoosek}[2]{\left(\begin{array}{c} #1 \\ #2 \end{array}\right)}

%\newcommand{\edits}[1]{{\bf \textcolor{blue}{#1}}}
\newcommand{\edits}[1]{{\leavevmode\color{BrickRed}{#1}}}

\newcommand{\referee}[1]{{\fontfamily{cmss}\selectfont \color{black!50!white}{#1}}\newline}

\begin{document}
\thispagestyle{empty}
\begin{flushright}
University of Utah\\
Scientific Computing and Imaging Institute \\
72 S. Campus Drive, Room 3750\\
Salt Lake City, UT 84112 \\
\end{flushright}
\vskip 15pt

\noindent 
Society for Industrial and Applied Mathematics \\
3600 University City Science Center \\
Philadelphia, PA 19104 \vskip 15pt

\noindent Dear editor,
\vskip 10pt

We would like to submit a revision for the manuscript ``Compressed sensing with sparse corruptions: Fault-tolerant sparse collocation approximations" to the SIAM/ASA Journal on Uncertainty Quantification. We have included in an attached document our detailed response to the referee reports.

This submission considers the problem of performing robust and accurate compressive sampling approximation in the presence of corrupted measurements. The corruptions problem is distinct from standard notions of ``noisy" measurements in that a small fraction of measurements are highly polluted, i.e., those measurements contain extremely large deviations, far outside the realm of small noise perturbations. The problem of corrupted measurements surfaces in large-scale computational frameworks where ``soft" failure mechanisms occasionally cause undetected and unpredictable corruptions of simulation quantities of interest.

Standard $\ell^1$ optimization algorithms that are common in compressive sampling procedures fail to obtain accurate accurate approximations in the presence of corrupted measurements. This work provides novel mathematical analysis and numerical investigation of an iteratively reweighted and regularized $\ell^1$ minimization algorithm that restores accurate and robust approximation results when measurements are corrupted. A distinct feature of our algorithm is that the corrupted measurements are identified along with the corresponding corruption values. Our mathematical analysis proves convergence of the algorithm under a general sparse corruptions model and considerably improves upon previous convergence guarantees in the compressive sampling literature, and numerical examples on medium-dimensional (4-10) problems show robustness and accuracy of the algorithm in the presence of corruptions. To our knowledge, this manuscript provides the first targeted analysis and investigation in the UQ community of sparse approximation algorithms with corrupted measurements.

We thank you for the time you invest in the review process, and we hope to hear a favorable response from you soon.

\noindent Sincerely, \\[50pt]
Ben Adcock\\
{\tt ben\_adcock@sfu.ca}\\[5pt]
Anyi Bao\\
{\tt cbao@sfu.ca}\\[5pt]
John D. Jakeman\\
{\tt jdjakem@sandia.gov}\\[5pt]
Akil Narayan\\
{\tt akil@sci.utah.edu}

\end{document}
