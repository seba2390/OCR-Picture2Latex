\section{Introduction}


Tate introduced a $p$-adic analytic geometry 
so-called the rigid geometry. 
In the original definition by Tate, 
a rigid analytic space is not a topological space but 
a Grothendieck topological space (cf. \cite{BGR84}). 
To remedy this situation, 
Huber established the theory of adic spaces. 
He introduced a topological space $\Spa\,A$, 
called an affinoid spectrum, associated to an affinoid ring $A=(A^{\rhd}, A^+)$, 
where $A^{\rhd}$ is an f-adic ring and 
$A^+$ is a certain open subring of $A^{\rhd}$. 
An adic space is obtained by gluing affinoid spectra. 

Although Huber introduced a structure presheaf $\MO_A$ 
on $\Spa\,A$ for an arbitrary affinoid ring $A$, 
it is not a sheaf in general 
(cf. \cite{BV}, \cite[the example after Proposition 1.6]{Hub94}, \cite{Mih16}). 
The reason for this is that 
the definition of $\MO_A$ depends on the completion, 
which is a transcendental operation. 
Thus it is natural to ask whether 
the theory can be more well-behaved 
after replacing the completion by a more algebraic operation. 
For example, if $(A, I)$ is a pair consisting of a ring $A$ and an ideal $I$ of $A$, 
then we can associate the Zariskian ring $A^{\Zar}$ and 
the henselisation $A^h$ with respect to $I$. 
The henselisation $A^h$ is known as an algebraic approximation of the $I$-adic completion $\widehat{A}$, 
whilst the associated Zariskian ring $A^{\Zar}$ is closer to the original ring $A$ than $A^h$. 
The purpose of this paper is to establish the Zariskian version 
of Huber's theory. 
Therefore, the first step is to introduce a notion of Zariskian f-adic rings. 

\begin{dfn}[Definition~\ref{d-zar}, Remark \ref{r-zar1}, Definition~\ref{d-zar-top}]\label{intro-d-zar}
Let $A$ be an f-adic ring. 
\begin{enumerate}
\item 
We set $S_A^{\Zar}:=1+A^{\circ\circ}$. 
It is easy to show that $S_A^{\Zar}$ is a multiplicative subset of $A$. 
We set $A^{\Zar}:=(S_A^{\Zar})^{-1}A$. 
Both $A^{\Zar}$ and the natural ring homomorphism $\alpha:A \to A^{\Zar}$ 
are called the {\em Zariskisation} of $A$. 
We say that $A$ is {\em Zariskian} if $\alpha:A \to A^{\Zar}$ 
is bijective. 
\item 
For a ring of definition $A_0$ of $A$ and 
an ideal of definition $I_0$ of $A_0$, 
we equip $A^{\Zar}$ with the group topology 
defined by the images of $\{I_0^kA_0^{\Zar}\}_{k\in \Z_{>0}}$. 
We can show that 
this topology does not depend on the choice of $A_0$ and $I_0$ 
(cf. Lemma~\ref{l-top-indep}). 
\end{enumerate}
\end{dfn}

We will prove that $A^{\Zar}$ satisfies some reasonable properties. 
For instance, $A^{\Zar}$ is a Zariskian f-adic ring (Theorem~\ref{t-zar-zar}) and 
$A^{\Zar}$ has the same completion as the one of $A$ 
(Theorem~\ref{t-comp-factor}). 
However one might consider that 
the definition of the topology of $A^{\Zar}$ 
is somewhat artificial. 
The following theorem asserts that 
our definition given above can be characterised 
in a category-theoretic way, 
i.e. $A^{\Zar}$ is an initial object of 
the category of Zariskian f-adic $A$-algebras. 


\begin{thm}[Theorem~\ref{t-zar-univ}]\label{intro-t-zar-univ}
Let $A$ be an f-adic ring and 
let $\alpha:A \to A^{\Zar}$ be the Zariskisation of $A$. 
Then, for any continuous ring homomorphism $\varphi:A \to B$ 
to a Zariskian f-adic ring $B$, 
there exists a unique continuous ring homomorphism 
$\psi:A^{\Zar} \to B$ such that $\varphi=\alpha \circ \psi$. 
\end{thm}





For an affinoid ring $A=(A^{\rhd}, A^+)$, 
we introduce a presheaf $\MO_A^{\Zar}$ on $\Spa\,A$ 
in the same way as in Huber's theory. 
The presheaf $\MO_A$ introduced by Huber is not a sheaf in general, 
whilst the Zariskian version $\MO^{\Zar}_A$ 
is always a sheaf. 

\begin{thm}[Theorem~\ref{t-sheafy}]
For an affinoid ring $A=(A^{\rhd}, A^+)$, the presheaf $\MO_A^{\Zar}$ on $\Spa\,A$ is a sheaf. 
\end{thm}

Then one might be tempted to hope 
the Tate acyclicity in general. 
Unfortunately this is not the case. 


\begin{thm}[Theorem~\ref{t-non-TA}]
There exists an affinoid ring $A=(A^{\rhd}, A^+)$ such that 
$H^1(\Spa\,A, \MO_A^{\Zar}) \neq 0$. 
\end{thm}


Although the Zariskian structure sheaf $\MO_A^{\Zar}$ 
does not behave nicely 
to establish a theory of coherent sheaves, 
the Zariskian rings might be still useful. 
For instance, if $A$ is a noetherian ring equipped with an $\m$-adic topology for some maximal ideal $\m$, 
then the Zariskisation $A^{\Zar}$ is nothing but 
the local ring $A_{\m}$ at $\m$. 
Therefore, in this situation, 
$A$ is Zariskian if and only if its $\m$-adic completion 
$A \to \widehat{A}$ is faithfully flat. 
The flatness of completion is a thorny problem 
for non-noetherian rings, 
whilst we prove that the completion is actually faithfully flat, 
under the assumption that $A$ is Zariskian and the completion is flat. 
More generally, we obtain the following result. 


\begin{thm}[Corollary~\ref{c-ff-criterion}]\label{intro-t-ff}
Let $\varphi:A \to B$ be a continuous ring homomorphism 
of Zariskian f-adic rings. 
Assume that the induced map $\widehat{\varphi}:\widehat{A} \to \widehat{B}$ 
is an isomorphism of topological rings. 
Then the following hold. 
\begin{enumerate}
\item 
Any maximal ideal of $A$ is contained in the image of 
the induced map $\Spec\,B \to \Spec\,A$. 
\item 
If $\varphi$ is flat, then $\varphi$ is faithfully flat. 
\end{enumerate}
\end{thm}


Theorem~\ref{intro-t-ff} is a consequence of 
the following characterisation of Zariskian f-adic rings. 


\begin{thm}[Theorem~\ref{t-characterise}]
Let $(A, A^+)$ be an affinoid ring. 
Then $A$ is Zariskian if and only if 
any maximal ideal of $A$ is contained in the image of 
the natural map 
$$\theta:\Spa\,(A, A^+) \to \Spec\,A, \quad v \mapsto \Ker(v).$$
\end{thm}

For a Zariskian affinoid ring $(A, A^+)$, 
the above theorem claims that 
the image of $\theta$ contains all the maximal ideals, 
however the map $\theta$ 
is not surjective in general (Theorem~\ref{t-non-surje}). 


\medskip

\textbf{Acknowledgement:} 
The author was funded by EPSRC. 
He would like to thank the referee for reading the paper carefully and for giving many constructive comments. 