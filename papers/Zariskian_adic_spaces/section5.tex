



\section{Images to affine spectra}

In this section, we study the image of the natural map 
$$\theta:\Spa\,(A, A^+) \to \Spec\,A, \quad v \mapsto \Ker(v)$$
for an affinoid ring $(A, A^+)$. 
In Subsection~\ref{ss-characterise}, 
we prove that $A$ is Zariskian 
if and only if $\text{Im}(\theta)$ contains all the maximal ideals of $A$ (Theorem~\ref{t-characterise}). 
We conclude some criteria for faithful flatness 
(Corollary~\ref{c-ff-criterion}, Corollary~\ref{c-ff-cplt}). 
In Subsection~\ref{ss-non-surje}, 
we find a Zariskian affinoid ring $(A, A^+)$ such that 
$\theta$ is not surjective (Theorem \ref{t-non-surje}). 



\subsection{Dominance onto maximal spectra}\label{ss-characterise}


The purpose of this subsection is to show Theorem~\ref{t-characterise}. 






\begin{thm}\label{t-characterise}
Let $(A, A^+)$ be an affinoid ring. 
Then the following are equivalent. 
\begin{enumerate}
\item 
$A$ is Zariskian. 
\item 
An arbitrary maximal ideal of $A$ is contained in the image of 
the natural map 
$$\Spa\,(A, A^+) \to \Spec\,A, \quad v \mapsto \Ker(v).$$
\end{enumerate}
\end{thm}

\begin{proof}
Assume that (1) holds. 
Thanks to Lemma \ref{l-zar-nonempty}(2), 
we can find a point $v \in \Spa\,(A, A^+)$ such that $\m=\Ker(v)$. 
Hence, (2) holds. 






Assume that (1) does not hold, i.e. $A$ is not Zariskian. 
We can find an element 
$x \in (1+A^{\circ\circ}) \setminus A^{\times}$. 
Then there exists a maximal ideal $\m$ containing $x$. 
Consider a commutative diagram of the induced maps: 
$$\begin{CD}
\Spa\,(A^{\Zar}, (A^+)^{\Zar}) @>g >> \Spec\,A^{\Zar}\\
@VV\beta V @VV \alpha V\\
\Spa\,(A, A^+) @>f >> \Spec\,A.
\end{CD}$$
Since the natural map $\beta$ is bijective by Theorem~\ref{t-comp-factor} and \cite[Proposition 3.9]{Hub93}, 
we have that 
$$\text{Im}(f)=\text{Im}(f \circ \beta)=\text{Im}(\alpha \circ g) 
\subset \text{Im}(\alpha).$$
By the choice of $\m$, we have that $\m \not\in \text{Im}(\alpha)$. 
Thus it holds that $\m \not\in \text{Im}(f)$, hence (2) does not hold. 
\end{proof}


\begin{cor}\label{c-ff-criterion}
Let $\varphi:(A, A^+) \to (B, B^+)$ be a continuous ring homomorphism 
of affinoid rings. 
Assume that $A$ is Zariskian and 
the induced map $\Spa\,(B, B^+) \to \Spa\,(A, A^+)$ is surjective. 
Then the following hold. 
\begin{enumerate}
\item 
Any maximal ideal of $A$ is contained in the image of 
the induced map $\Spec\,B \to \Spec\,A$. 
\item 
If $\varphi$ is flat, then $\varphi$ is faithfully flat. 
\end{enumerate}
\end{cor}

\begin{proof}
We have a natural commutative diagram: 
$$\begin{CD}
\Spa\,(B, B^+) @>\theta_B >> \Spec\,B\\
@VV\varphi^{\flat}V @VV\varphi^{\sharp}V\\
\Spa\,(A, A^+) @>\theta_A >> \Spec\,A.
\end{CD}$$
Therefore, (1) holds by Theorem~\ref{t-characterise}. 
We obtain the assertion (2) by (1) and \cite[Theorem 7.3(ii)]{Mat89}. 
\end{proof}



\begin{cor}\label{c-ff-cplt}
Let $\varphi:A \to B$ be a continuous ring homomorphism 
of f-adic rings. 
Assume that $A$ is Zariskian and 
the induced map $\widehat{\varphi}:\widehat{A} \to \widehat{B}$ 
is an isomorphism of topological rings. 
Then the following hold. 
\begin{enumerate}
\item 
Any maximal ideal of $A$ is contained in the image of 
the induced map $\Spec\,B \to \Spec\,A$. 
\item 
If $\varphi$ is flat, then $\varphi$ is faithfully flat. 
\end{enumerate}
\end{cor}


\begin{proof}
It follows from \cite[Proposition 3.9]{Hub93} that 
$\varphi^{\flat}:\Spa\,(B, B^{\circ}) \to \Spa\,(A, A^{\circ})$ is bijective. 
Therefore, the assertions follow from Corollary~\ref{c-ff-criterion}. 
\end{proof}



\begin{cor}
Let $(A, A^+)$ be an affinoid ring. 
Set $X:=\Spa\,(A, A^+)$. 
Let $X=\bigcup_{i \in I} U_i$ be 
a finite open cover where $U_i$ is a rational subset 
of $\Spa\,(A, A^+)$ for any $i \in I$. 
Then the induced ring homomorphism 
$$\rho:\MO_X^{\Zar}(X) \to \prod_{i \in I}\MO_X^{\Zar}(U_i)$$
is faithfully flat. 
\end{cor}

\begin{proof}
Since the induced map 
$$\rho^{\flat}:
\Spa\,\left(\prod_{i \in I}\MO_X^{\Zar}(U_i), \prod_{i \in I}\MO_X^{\Zar, +}(U_i)\right)
\to 
\Spa\,\left(\MO_X^{\Zar}(X), \MO_X^{\Zar, +}(X)\right)$$
is the same as the open cover $\coprod_{i \in I} U_i \to X$, 
we see that $\rho^{\flat}$ is surjective. 
Since $\rho$ is flat, it is faithfully flat by Corollary~\ref{c-ff-criterion}(ii). 
\end{proof}

\subsection{Non-surjective example}\label{ss-non-surje}

The purpose of this subsection is 
to show Theorem~\ref{t-non-surje}, 
which asserts that there exists a Zariskian f-adic ring $R$ such that 
the natural map 
$$\Spa\,(R, R^+) \to \Spec\,R,\quad v \mapsto \Ker(v)$$
is not surjective for any ring of integral elements $R^+$ of $R$. 
Our example is based on \cite[Remark (2) after Theorem 10.17]{AM69}, 
hence let us start by recalling its construction. 



\begin{nota}\label{n-AM}
Let $A:=\{f:\R \to \R\,|\,f\text{ is of class }C^{\infty}\}.$ 
For 
$$\varphi:A \to \R,\quad f \mapsto f(0),$$
we set $\m:=\Ker(\varphi)$. 
Since $\varphi$ is a surjective ring homomorphism to a field $\R$, 
it holds that $\m$ is a maximal ideal of $A$. 
By Taylor's theorem, we get $\m=xA$. 
It follows again from Taylor's theorem that 
$$\bigcap_{n=1}^{\infty} \m^n=\{f \in A\,|\,f(0)=f'(0)=f''(0)=\cdots=0\}.$$
We equip $A$ with the $\m$-adic topology. 
Since $\m$ is a finitely generated ideal, we have that $A$ is an f-adic ring. 
Recall that $A^{\Zar}=(1+\m)^{-1}A$ and 
let $\alpha:A \to A^{\Zar}$ be the natural continuous ring homomorphism, 
where the topology on $A^{\Zar}$ coincides with the $\m A^{\Zar}$-adic topology. 
We can directly check that 
$$\Ker(\alpha)=\{f \in A\,|\, f|_U=0\text{ for some open neighbourhood } U\text{ of } 0 \in \R\}.$$
In particular, it follows that 
$e^{-1/x^2} \in \left(\bigcap_{n=1}^{\infty} \m^n\right) \setminus \Ker(\alpha)$ 
and 
$$\alpha(e^{-1/x^2}) \in \left(\bigcap_{n=1}^{\infty} \m^n A^{\Zar}\right) \setminus \{0\}.$$
\end{nota}



\begin{lem}\label{l-non-surje}
We use Notation~\ref{n-AM}. 
Then the following hold. 
\begin{enumerate}
\item $A$ is reduced. 
\item $A^{\Zar}$ is reduced. 
\item There exists a prime ideal $\p$ of $A^{\Zar}$ 
such that 
$$\bigcap_{n=1}^{\infty} \m^n A^{\Zar} \not\subset \p.$$ 
\item The closed immersion 
$$\pi^{\sharp}:\Spec\,((A^{\Zar})^{\hd}) \to \Spec\,A^{\Zar}$$
is not surjective, where $\pi^{\sharp}$ is the morphism 
induced by the natural surjective ring homomorphism 
$\pi:A^{\Zar} \to (A^{\Zar})^{\hd}$.
\end{enumerate}
\end{lem}

\begin{proof}
We first show (1). 
Take $f \in A$ and assume that $f^n=0$ for some positive integer $n$. 
Since $\R$ is reduced, we have that 
the equation $f(a)^n=0$ implies that $f(a)=0$ for any $a \in \R$. 
Therefore, we get $f=0$. 
Thus (1) holds. 
The assertion (2) follows from (1) (cf. \cite[Corollary 3.12]{AM69}). 




We now show (3). 
It follows from (2) and $\bigcap_{n=1}^{\infty} \m^n A^{\Zar} \neq 0$ that 
$$\bigcap_{n=1}^{\infty} \m^n A^{\Zar} \not\subset \{0\}=\sqrt{0}=\bigcap_{\p \in \Spec\,A^{\Zar}} \p.$$
In particular, there exists a prime ideal $\p$ of $A^{\Zar}$ 
such that $\bigcap_{n=1}^{\infty} \m^n A^{\Zar} \not\subset \p$. 
Thus (3) holds. 


Let us show (4). 
Since $(A^{\Zar})^{\hd}=A^{\Zar}/\left( \bigcap_{n=1}^{\infty} \m^n A^{\Zar}\right)$, 
the image of $\pi^{\sharp}$ consists of the prime ideals of $A^{\Zar}$ 
containing $\bigcap_{n=1}^{\infty} \m^n A^{\Zar}$. 
Thus (4) follows from (3). 
\end{proof}




\begin{thm}\label{t-non-surje}
There exists a Zariskian f-adic ring $R$ such that the natural map 
$$\theta:\Spa\,(R, R^+) \to \Spec\,R, \quad v \mapsto \Ker(v)$$
is not surjective for any ring of integral elements $R^+$ of $R$. 
\end{thm}

\begin{proof}
We use Notation~\ref{n-AM}. 
Set $R:=A^{\Zar}$. 
Fix an arbitrary ring of integral elements $R^+$ of $R$. 
We obtain a commutative diagram of natural maps: 
$$\begin{CD}
\Spa\,(R^{\hd}, (R^+)^{\hd}) @>\theta^{\hd}>> \Spec\,R^{\hd}\\
@VV\pi^{\flat}V @VV\pi^{\sharp} V\\
\Spa\,(R, R^+) @>\theta >> \Spec\,R.
\end{CD}$$
It follows from \cite[Proposition 3.9]{Hub93} that $\pi^{\flat}$ is bijective. 
Since $\pi^{\sharp}$ is not surjective by Lemma~\ref{l-non-surje}, 
neither is $\theta$. 
\end{proof}


