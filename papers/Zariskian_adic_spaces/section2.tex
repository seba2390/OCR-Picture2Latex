
\section{Preliminaries}

In this section, we summarise notation and basic results. 



\subsection{Notation}

\begin{enumerate}
\item 
Throughout the paper, a {\em ring} is 
always assumed to be commutative and to have a unity element 
with respect to the multiplication. 
We say that $P$ is a {\em pseudo-ring}, 
if $P$ satisfies all the axioms of rings except for the existence of a multiplicative identity. 
A subset $Q$ of a pseudo-ring is a {\em pseudo-subring} of $P$
if $Q$ is an additive subgroup of $P$ such that 
$q_1q_2 \in Q$ for all $q_1, q_2 \in Q$. 
In this case, we consider $Q$ as a pseudo-ring. 
For example, any ideal of a ring $A$ is a pseudo-subring of $A$. 
\item 
We will freely use the notation and terminology of 
\cite{Hub93} and \cite{Hub94}. 
In particular, given a topological ring $A$, 
we denote by 
$A^{\circ}$ (resp. $A^{\circ\circ}$) the subset of $A$ consisting of 
the power-bounded (resp. topological nilpotent) elements. 
If $A$ is an f-adic ring, then $A^{\circ}$ is an open subring of $A$ 
and $A^{\circ\circ}$ is an open ideal of $A^{\circ}$. 
A map $\varphi:A \to B$ of topological rings 
is an {\em isomorphism of topological rings} 
if $\varphi$ is bijective and both $\varphi$ and $\varphi^{-1}$ 
are continuous ring homomorphisms. 
\item 
For a partially ordered set $I$ and an element $i_0 \in I$, 
we set $I_{\geq i_0}:=\{i \in I\,|\, i\geq i_0\}$. 
We define $I_{>i_0}$ in the same way. 
\item 
For a topological space $X$, 
the topology on $X$ is {\em trivial} 
if any open subset of $X$ is equal to $X$ or the empty set. 
\end{enumerate}





\begin{dfn}\label{d-OBI}
Let $A$ and $B$ be topological rings. 
\begin{enumerate}
\item 
We define $\mathfrak O_A$ as 
the set of the open subrings of $A$. 
If $f:A \to B$ is a continuous ring homomorphism, 
we define $f^*:\mathfrak O_{B} \to \mathfrak O_{A}$ 
by $B_0 \mapsto f^*(B_0):=f^{-1}(B_0)$. 
If $f^*$ is bijective, then the inverse map is denoted by $f_*$. 
\item 
Let $\mathfrak B_{A}$ be the set of the bounded open subrings 
of $A$. 
\item 
Let $\mathfrak I_{A}$ be the set of 
the open subrings of $A$ 
that are contained in $A^{\circ}$ and integrally closed in $A$. 
\end{enumerate}
\end{dfn}



\subsection{Group topologies}

Let $P$ be a pseudo-ring. 
Let $\{P_n\}_{n \in \Z_{\geq 0}}$ be a set of additive subgroups of $P$ 
such that $P_{0} \supset P_{1} \supset \cdots$. 
Then we call the {\em group topology} on $P$ 
induced by $\{P_n\}_{n \in \Z_{\geq 0}}$ is a topology on $P$ 
such that given a subset $U$ of $P$, 
$U$ is an open subset of $P$ if and only if 
for any $x \in U$, there exists $n_x \in \Z_{\geq 0}$ such that 
$x+P_{n_x} \subset U$. 
We can directly check that $P$ is a topological group 
with respect to the additive structure. 
The following lemma gives a criterion 
for the continuity of the multiplication map. 

\begin{lem}\label{l-top-criterion}
Let $P$ be a pseudo-ring. 
Let $\{P_n\}_{n \in \Z_{\geq 0}}$ be a set of a pseudo-subrings of $P$ 
such that $P_0 \supset P_1 \supset \cdots$. 
We equip $P$ with the group topology induced by $\{P_n\}_{n \in \Z_{\geq 0}}$. 
Then the following are equivalent. 
\begin{enumerate}
\item 
$P$ is a topological pseudo-ring, i.e. the multiplication map 
$$\mu:P \times P \to P,\quad (x, y) \mapsto xy$$
is continuous. 
\item 
For any element $x \in P$ and any non-negative integer $n$, 
there exists a non-negative integer $m$ satisfying the inclusion  
$$
xP_m:=\{xy \in P\,|\, y \in P_m\} \subset P_n.
$$
\end{enumerate}
\end{lem}


\begin{proof}
Assume (1). 
Take an element $x \in P$ and a non-negative integer $n$. 
By (1), the composite map 
$$\theta:P \to P \times P \to P, \quad y \mapsto (x, y) \mapsto xy$$
is continuous. 
In particular, we get an inclusion $P_m \subset \theta^{-1}(P_n)$ 
for some $m \in \Z_{\geq 0}$. 
Therefore we obtain $x P_m=\theta(P_m) \subset \theta(\theta^{-1}(P_n)) \subset P_n$, 
hence (2) holds. 
Thus (1) implies (2). 



Assume (2). 
Fix $z \in P$ and $n \in \Z_{> 0}$. 
It suffices to show that $\mu^{-1}(z+P_n)$ is 
an open subset of $P \times P$. 
If $\mu^{-1}(z+P_n)$ is empty, then there is nothing to show. 
Pick $(x, y) \in \mu^{-1}(z+P_n)$, i.e. $xy \in z+P_n$. 
By (2), we can find a positive integer $m$ 
such that $m \geq n$, $xP_m \subset P_n$ and $yP_m \subset P_n$. 
Take $x', y' \in P_m$. 
We get 
$$\mu(x+x', y+y')=xy+x'y+xy'+x'y' \in (z+P_n)+P_n+P_n+P_m \subset z+P_n.$$
Therefore it holds that 
$$(x+P_m) \times (y+P_m) \subset \mu^{-1}(z+P_n),$$ 
hence (1) holds. Thus (2) implies (1).  
\end{proof}

\subsection{Quotients by ideals}


The materials treated in this subsection 
appear in literature (cf. \cite[(1.4.1)]{Hub96}). 

\subsubsection{Quotients of f-adic rings}\label{ss-quot-fad}

Let $A$ be an f-adic ring. 
Let $J$ be an ideal of $A$ and let $\pi:A \to A/J$ be the natural ring homomorphism. 
For a ring of definition $A_0$ of $A$ and 
an ideal of definition $I_0$ of $A_0$, 
we equip $A/J$ with the group topology induced by $\{\pi(I_0^nA_0)\}_{n \in \Z_{>0}}$. 
We can check directly by definition that 
the topology of $A/J$ does not depend on the choices of $A_0$ and $I_0$. 
It follows from Lemma~\ref{l-top-criterion} that $A/J$ is a topological ring. 
Again by definition, 
we have that $A/J$ is an f-adic ring and $\pi:A \to A/J$ is adic. 
We call $A/J$ the {\em quotient f-adic ring} of $A$ by $J$. 

The following lemma should be well-known for experts, 
however we include the proof for the sake of completeness. 

\begin{lem}\label{l-2quot-top}
With the notation as above, the topology on $A/J$ coincides with the quotient topology induced by $\pi$.  
\end{lem}

\begin{proof}
To avoid confusion, we call the topology on $A/J$ defined above 
{\em the f-adic topology} in this proof. 
Fix a ring of definition $A_0$ of $A$ and an ideal of definition $I_0$ of $A_0$. 

Let $V$ be an open subset of $A/J$ with respect to the quotient topology. 
Pick $x \in \pi^{-1}(V)$. 
Since $\pi^{-1}(V)$ is an open subset of $A$, 
we can find a positive integer $n_x$ 
satisfying $x+I_0^{n_x}A_0 \subset \pi^{-1}(V)$. 
Thus we get an equation 
$\pi^{-1}(V)=\bigcup_{x \in \pi^{-1}(V)} (x+I_0^{n_x}A_0),$ which implies 
$$V=\pi\left(\pi^{-1}(V)\right)=\bigcup_{x \in \pi^{-1}(V)} \left(\pi(x)+\pi(I_0^{n_x}A_0)\right).$$
Therefore, $V$ is an open subset of $A/J$ with respect to the f-adic topology. 

Let $V$ be an open subset of $A/J$ with respect to the f-adic topology. 
For any $y \in V$, we can find a positive integer $n_y$ such that 
$y+\pi(I_0^{n_y}A_0) \subset V$. 
Thus we get an equation $V=\bigcup_{y \in V} (y+\pi(I_0^{n_y}A_0)),$ 
which implies 
$$\pi^{-1}(V)=\bigcup_{y \in V} \pi^{-1}(y+\pi(I_0^{n_y}A_0))
=\bigcup_{y \in V} (y'+J+I_0^{n_y}A_0),$$
where $y'$ is an element of $A$ satisfying $\pi(y')=y$. 
Thus $\pi^{-1}(V)$ is an open subset of $A$. 
Therefore, we have that $V$ is an open subset of $A/J$ with respect to the quotient topology. 
\end{proof}

\subsubsection{Quotients of affinoid rings}\label{ss-quot-aff}


Let $A=(A^{\rhd}, A^+)$ be an affinoid ring. 
For an ideal $J^{\rhd}$ of $A^{\rhd}$, 
we equip $A^{\rhd}/J^{\rhd}$ with the quotient topology. 
Thanks to Subsection~\ref{ss-quot-fad}, 
we have that $A^{\rhd}/J^{\rhd}$ is an f-adic ring and 
the induced ring homomorphism $\pi:A^{\rhd} \to A^{\rhd}/J^{\rhd}$ is adic. 
We set $(A/J^{\rhd})^+$ to be the integral closure of $\pi(A^+)$ in 
$A^{\rhd}/J^{\rhd}$. 
Then the pair $A/J^{\rhd}:=(A^{\rhd}/J^{\rhd}, (A/J^{\rhd})^+)$ 
is an affinoid ring and 
$\pi:A \to A/J^{\rhd}$ is an adic ring homomorphism of affinoid rings. 
We call $A/J^{\rhd}$ the {\em quotient affinoid ring} of $A$ by $J^{\rhd}$. 


\begin{rem}
We use the same notation as above. 
Then $A/J^{\rhd}$ satisfies the following universal property: 
for any continuous ring homomorphism $\varphi:A \to B$ 
to an affinoid ring $B$ such that $\varphi(J^{\rhd})=0$, 
there exists a unique continuous ring homomorphism 
$\psi:A/J^{\rhd} \to B$ such that $\varphi=\psi \circ \pi$. 
\end{rem}


\subsection{Hausdorff quotient}

Let $A$ be a topological ring. 
Then the closure $\overline{\{0\}}$ of the zero ideal in $A$ 
is an ideal of $A$ and 
the residue ring $A^{\hd}:=A/\overline{\{0\}}$, 
equipped with the quotient topology, is a Hausdorff topological ring. 
Both of $A^{\hd}$ and $A \to A^{\hd}$ are 
called the {\em Hausdorff quotient} of $A$. 
For a continuous ring homomorphism $\varphi:A \to B$, 
we get a commutative diagram of continuous ring homomorphisms: 
$$\begin{CD}
A @>>> B\\
@VVV @VVV\\
A^{\hd} @>>> B^{\hd}.
\end{CD}$$

\begin{rem}\label{r-hd-univ}
Let $A$ be a topological ring and let $\varphi:A \to B$ 
be a continuous ring homomorphism to a Hausdorff topological ring $B$. 
Then there exists a unique continuous ring homomorphism 
$\psi:A^{\hd} \to B$ such that $\varphi=\psi \circ \pi$, 
where $\pi:A \to A^{\hd}$ is the natural map. 
\end{rem}



\begin{rem}\label{r-quot-open}
Let $A$ be a topological ring and 
let $\pi:A \to A^{\hd}$ be its Hausdorff quotient. 
Clearly, the induced map $\pi^*:\mathfrak O_{A^{\hd}} \to \mathfrak O_{A}$ is bijective and its inverse map $\pi_*:=(\pi^*)^{-1}$ 
satisfies $\pi_*(A_0)=\pi(A_0)$ for any $A_0 \in \mathfrak O_{A}$. 
Moreover we have that $\pi_*(A_0)$ is 
canonically isomorphic to $A_0^{\hd}$. 
\end{rem}


Let $A$ be an f-adic ring. 
For a ring of definition $A_0$ of $A$ and 
an ideal of definition $I_0$ of $A_0$, 
we have that $\overline{\{0\}} \subset I_0^nA_0$. 
Thus its image $\pi(I_0^nA_0)$ is an open subset of $A^{\hd}$. 
Since $\{\pi(I_0^nA_0)\}_{n \in \Z_{>0}}$ is a fundamental system 
of open neighbourhoods of $0 \in A^{\hd}$ by Lemma~\ref{l-2quot-top}, 
we have that $A^{\hd}$ is an f-adic ring. 
Moreover, the natural ring homomorphism $A \to A^{\hd}$ is adic. 


\begin{prop}\label{p-hd-open}
Let $A$ be an f-adic ring and let $\pi:A \to A^{\hd}$ 
be the Hausdorff quotient of $A$. 
Then the following hold. 
\begin{enumerate}
\item 
It holds that $\pi_*(\mathfrak B_A)\subset \mathfrak B_{A^{\hd}}$ 
and 
$\pi^*(\mathfrak B_{A^{\hd}}) \subset \mathfrak B_A$. 
\item 
It holds that 
$\pi(A^{\circ})=(A^{\hd})^{\circ}$ and 
$\pi^{-1}((A^{\hd})^{\circ})=A^{\circ}$. 
\item 
It holds that $\pi_*(\mathfrak I_A)\subset \mathfrak I_{A^{\hd}}$ 
and 
$\pi^*(\mathfrak I_{A^{\hd}}) \subset \mathfrak I_A$.
\end{enumerate}
\end{prop}


\begin{proof}
We now show (1). 
The former inclusion $\pi_*(\mathfrak B_A)\subset \mathfrak B_{A^{\hd}}$ 
follows from \cite[Lemma 1.8(i)]{Hub93}. 
Let us prove the latter one $\pi^*(\mathfrak B_{A^{\hd}}) \subset \mathfrak B_A$. 
Take $B_1 \in \mathfrak B_{A^{\hd}}$. 
For any positive integer $n_1$, 
the boundedness of $B_1$ enables us 
to find a positive integer $n_2$ such that 
$$\pi(I_0^{n_2}A_0) \cdot B_1 \subset \pi(I_0^{n_1}A_0).$$
This inclusion immediately induces the following inclusion
$$I_0^{n_2}A_0 \cdot \pi^{-1}(B_1) \subset I_0^{n_1}A_0,$$
which implies $\pi^*(\mathfrak B_{A^{\hd}}) \subset \mathfrak B_A$. 
Thus (1) holds. 
The assertion (2) follows from (1) and \cite[Corollary 1.3(iii)]{Hub93}. 

We now show (3). 
Let us prove the former inclusion 
$\pi_*(\mathfrak I_A)\subset \mathfrak I_{A^{\hd}}$. 
Take $A^+ \in \mathfrak I_A$. 
It follows from (2) that $\pi(A^+) \subset (A^{\hd})^{\circ}$. 
Take an element $x \in A$ whose image 
$\pi(x)$ is integral over $\pi(A^+)$. 
We get 
$$x^n+a_1x^{n-1}+\cdots+a_n \in \overline{\{0\}}$$
for some $a_1, \cdots, a_n \in A^+$. 
By $\overline{\{0\}} \subset A^+$, we have that $x$ is integral over $A^+$. 
Since $A^+$ is integrally closed in $A$, 
we obtain $x \in A^+$. 
Hence, it holds that $\pi(x) \in \pi(A^+)$. 
Therefore, we get $\pi(A^+) \in \mathfrak I_{A^{\hd}}$ and 
$\pi_*(\mathfrak I_A) \subset \mathfrak I_{A^{\hd}}$. 
This completes the proof of the inclusion $\pi_*(\mathfrak I_A)\subset \mathfrak I_{A^{\hd}}$. 

Let us prove the other inclusion 
$\pi^*(\mathfrak I_{A^{\hd}}) \subset \mathfrak I_A$. 
Take $B^+ \in \mathfrak I_{A^{\hd}}$. 
It follows from (2) that 
$$\pi^{-1}(B^+) \subset \pi^{-1}((A^{\hd})^{\circ})=A^{\circ}.$$ 
Take an element $x \in A$ which is integral over $\pi^{-1}(B^+)$. 
We get 
$$x^n+a_1x^{n-1}+\cdots+a_n=0$$
for some $a_1, \cdots, a_n \in \pi^{-1}(B^+)$. 
Since $B^+$ is integrally closed in $A^{\hd}$, 
we have that $\pi(x) \in B^+$ and $x \in \pi^{-1}(B^+)$. 
Therefore, we get $\pi^{-1}(B^+) \in \mathfrak I_{A}$ and 
$\pi^*(\mathfrak I_{A^{\hd}}) \subset \mathfrak I_{A}$. 
Thus (3) holds. 
\end{proof}





\subsection{Completion}


Let $A$ be a topological ring. 
Its completion $A \to \widehat{A}$ 
is defined in \cite[Ch II \S 3 Section 3, Ch. III \S 6 Sect. 3]{Bou89} 
(cf. \cite[Ch. 0, Section 7.1(c)]{FK}). 
When we treat f-adic rings, 
their completions can be constructed 
by the classical method using Cauchy sequences 
(cf. \cite[Section 10.1]{AM69}). 
Even for general topological rings, 
we can construct the completions by using Cauchy nets in a similar way. 

If the topology of $A$ coincides with the $I$-adic topology for some finitely generated ideal $I$, 
then its completion $\widehat{A}$ is constructed 
also by the inverse limit $\varprojlim_n A/I^n$. 

\begin{rem}
Let $A$ be a ring, $I$ a finitely generated ideal of $A$, and $M$ an $A$-module. 
Then the inverse limit $\widehat{M}:=\varprojlim_n M/I^nM$ is complete 
with respect to the projective limit topology and $M/I^nM \xrightarrow{\simeq} \widehat{M}/I^n\widehat{M}$ 
by \cite[Ch. 0, Lemma 7.2.8 and Proposition 7.2.16]{FK}. 
If $I$ is not finitely generated, 
there exist counterexamples to these assertions (\cite[Ch. 0, Example 7.2.10]{FK}). 
\end{rem}



Let $A$ be an f-adic ring and let $\gamma:A \to \widehat{A}$ be the completion. 
For any open subring $A_0$ of $A$, 
the induced continuous ring homomorphsim $\widehat{A_0} \to \widehat{A}$ 
is injective and open, hence we consider $\widehat{A_0}$ 
as an open subring of $\widehat{A}$. 
In other words, we get a map 
$\gamma_1:\mathfrak O_A \to \mathfrak O_{\widehat{A}}$ 
defined by $\gamma_1(A_0)=\overline{\gamma(A_0)}$ for any $A_0 \in \mathfrak O_A$. 
For a ring of definition $A_0$ of $A$ and 
an ideal of definition $I_0$ of $A_0$, 
it follows from \cite[Lemma 1.6(i)]{Hub93} that $\{I_0^n\widehat{A_0}\}$ 
is a fundamental system of open neighbourhoods of $0 \in \widehat{A}$. 
In particular, 
the completion $\gamma:A \to \widehat{A}$ is adic. 
The completion $\gamma$ uniquely factors through the Hausdorff quotient: 
$$\gamma:A \xrightarrow{\pi}  A^{\hd} \xrightarrow{\gamma'}\widehat{A}.$$
Thanks to \cite[Corollary 1.9(ii)]{Hub93}, 
also the induced map $\gamma'$ is an adic ring homomorphism. 



\begin{lem}\label{l-complete-open}
Let $A$ be an f-adic ring and 
let $\gamma:A \to \widehat{A}$ be the completion. 
Consider the following two maps: 
\begin{eqnarray*}
\gamma_1:\mathfrak O_{A} \to \mathfrak O_{\widehat{A}},&& 
A_0 \mapsto \overline{\gamma(A_0)}\\
\gamma^*:\mathfrak O_{\widehat{A}} \to \mathfrak O_{A},&& 
B_0 \mapsto \gamma^{-1}(B_0).
\end{eqnarray*}
Then both $\gamma_1 \circ \gamma^*$ and $\gamma^*\circ \gamma_1$ are 
the identity maps. 
\end{lem}

According to Definition~\ref{d-OBI}(1), we set $\gamma_*:=\gamma_1$. 

\begin{proof}
By Remark~\ref{r-quot-open}, 
we may assume that $A$ is Hausdorff and 
$A$ is a subring of $\widehat{A}$. 

First we show that $\gamma^* \circ \gamma_1$ is the identity map. 
Take $A_0 \in \mathfrak O_{A}$. 
We show $A_0=\overline{A_0} \cap A$. 
It suffices to prove that $A_0 \supset \overline{A_0} \cap A$. 
Fix $\alpha \in \overline{A_0} \cap A$. 
There is a Cauchy sequence 
$\{a_n\}_{n \in \Z_{>0}} \subset A_0$ converging to $\alpha \in A$. 
Since $A_0$ is a closed subset of $A$, it follows that $\alpha \in A_0$. 
Therefore we get $A_0 \supset \overline{A_0} \cap A$, 
hence $\gamma^* \circ \gamma_1$ is the identity map. 



Second we show that $\gamma_1 \circ \gamma^*$ is the identity map. 
Take $B_0 \in \mathfrak O_{\widehat{A}}$. 
We prove $B_0=\overline{B_0 \cap A}$. 
It suffices to show that $B_0 \subset \overline{B_0 \cap A}$. 
Take $\beta \in B_0$. 
There exists a Cauchy sequence 
$\{b_n\}_{n \in \Z_{>0}} \subset A$ converging to $\beta$. 
Since $B_0$ is an open subset of $\widehat{A}$, we can find $N \in \Z_{>0}$ 
such that $b_n-\beta \in B_0$ for any $n \geq  N$. 
Thus we get $b_n=\beta+(b_n-\beta) \in B_0$ for any $n \in \Z_{\geq N}$. 
In particular, the shifted sequence $b_N, b_{N+1}, \cdots$ 
is contained in $A \cap B_0$. 
Therefore, we get $\beta \in \overline{B_0 \cap A}$, 
hence $\gamma_1 \circ \gamma^*$ is the identity map. 
\end{proof}


\begin{lem}\label{l-sub-bdd-bdd}
Let $B$ be an f-adic ring and 
let $A$ be an f-adic subring of $B$ 
whose topology coincides with the induced topology from $B$. 
Set $j:A \hookrightarrow B$ to be the inclusion map. 
Then the inclusion 
$j^*(\mathfrak B_{B}) \subset \mathfrak B_{A}$ holds. 
\end{lem}

\begin{proof}
Let $B_0$ be a bounded subset of $B$. 
It suffices to show that $B_0 \cap A$ is a bounded subset of $A$. 
Let $U$ be an open neighbourhood of $0 \in A$. 
There exists an open subset $U_B$ of $B$ such that $U_B \cap A=U$. 
Since $B_0$ is bounded, 
we can find an open neighbourhood $V_B$ of $0 \in B$ such that 
$B_0 \cdot V_B \subset U_B$. 
We get  
$$(B_0 \cap A)\cdot (V_B \cap A) \subset 
(B_0 \cdot V_B) \cap A \subset U_B \cap A=U,$$
which implies that $B_0 \cap A$ is bounded. 
\end{proof}



\begin{lem}\label{l-complete-open2}
Let $A$ be an f-adic ring and 
let $\gamma:A \to \widehat{A}$ be the completion of $A$. 
Let $\gamma_*:\mathfrak O_A \to \mathfrak O_{\widehat{A}}$ and $\gamma^*:\mathfrak O_{\widehat{A}} \to \mathfrak O_{A}$ be 
the induced maps (cf. Lemma~\ref{l-complete-open}). 
Then the following assertions hold. 
\begin{enumerate}
\item If $A_0 \in \mathfrak B_{A}$, then $\gamma_*(A_0) 
\in \mathfrak B_{\widehat{A}}$.
\item If $B_0 \in \mathfrak B_{\widehat{A}}$, then $\gamma^*(B_0) \in \mathfrak B_{A}$. 
\end{enumerate}
\end{lem}

\begin{proof}
By Proposition~\ref{p-hd-open}, we may assume that $A$ is Hausdorff and 
$A$ is a subring of $\widehat{A}$. 
The assertion (1) follows from the fact that 
$\gamma_*(A_0)=\overline{A_0}$ is bounded (\cite[Lemma 1.6(i)]{Hub93}). 
The assertion (2) holds by Lemma~\ref{l-sub-bdd-bdd}. 
\end{proof}


\begin{lem}\label{l-pbdd-completion}
Let $A$ be an f-adic ring. 
Then the natural map $\widehat{A^{\circ}} \to (\widehat{A})^{\circ}$ is an isomorphism of topological rings.
\end{lem}

\begin{proof}
Let $\gamma:A \to \widehat{A}$ be the completion of $A$. 

Note that the natural map $\varphi:\widehat{A^{\circ}} \to (\widehat{A})^{\circ}$ in the statement is 
constructed as follows. 
Since the completion $\gamma:A \to \widehat{A}$ is adic by 
\cite[Lemma 1.6]{Hub93}, 
it follows from \cite[Corollary 1.3(iii), Lemma 1.8(i)]{Hub93} that $\gamma(A^{\circ}) \subset (\widehat{A})^{\circ}$. 
Taking the completion of $\gamma|_{A^{\circ}}:A^{\circ} \to (\widehat{A})^{\circ}$, 
we obtain a natural continuous ring homomorphism 
$\varphi:\widehat{A^{\circ}} \to \widehat{(\widehat{A})^{\circ}}=(\widehat{A})^{\circ}$, 
where the equation $\widehat{(\widehat{A})^{\circ}}=(\widehat{A})^{\circ}$ 
follows from the fact that $(\widehat{A})^{\circ}$ is 
an open, hence a closed, subring of $\widehat{A}$. 


By the construction of $\varphi$, 
it follows that $\varphi$ is open and injective, 
hence we can consider $\widehat{A^{\circ}}$ as an open subring of $(\widehat{A})^{\circ}$. 
It suffices to show $\widehat{A^{\circ}} \supset (\widehat{A})^{\circ}$. 
Take $\alpha \in (\widehat{A})^{\circ}$. 
Let $B_0$ be a bounded open subring of $(\widehat{A})^{\circ}$ containing $\alpha$. 
We set $A_0:=\gamma^{-1}(B_0)$. 
Then $A_0$ is a bounded open subring of $A$ such that  $\widehat{A_0}=B_0$ by Lemma~\ref{l-complete-open2}. 
Thus we obtain $\alpha \in \widehat{A_0} \subset \widehat{A^{\circ}}$, 
as desired. 
\end{proof}



\begin{lem}\label{l-integral-completion}
Let $B$ be an f-adic ring and let $A$ be an open subring of $B$. 
Let $\widehat{A}$ and $\widehat{B}$ be the completions of $A$ and $B$, 
respectively. 
Set $\gamma:B \to \widehat{B}$ be the induced map. 
Then the following hold. 
\begin{enumerate}
\item The equation $\widehat{B}=\gamma(B) \cdot \widehat{A}$ holds, 
where $\gamma(B) \cdot \widehat{A}$ denotes the smallest subring of $\widehat{B}$ containing $\gamma(B) \cup \widehat{A}$. 
\item If $A \subset B$ is an integral extension, then so is $\widehat{A} \subset \widehat{B}$. 
\end{enumerate}
\end{lem}

\begin{proof}
We first show (1). Set $C:=\gamma(B) \cdot \widehat{A}$. 
Since $\widehat{A}$ is an open subring of $\widehat{B}$, 
so is $C$. 
In particular, $C$ is a closed subset of $\widehat{B}$ containing $\gamma(B)$. 
Thus $\widehat{B}=C$, hence (1) holds. 
The assertion (2) immediately follows from (1). 
\end{proof}


\begin{lem}\label{l-complete-open3}
Let $A$ be an f-adic ring. 
Let $\gamma:A \to \widehat{A}$ be the completion and 
let $\gamma_*:\mathfrak O_A \to \mathfrak O_{\widehat{A}}$ and $\gamma^*:\mathfrak O_{\widehat{A}} \to \mathfrak O_{A}$ be 
the induced maps (cf. Lemma~\ref{l-complete-open}). 
Then the following hold. 
\begin{enumerate}
\item If $A^+ \in \mathfrak I_{A}$, then $\gamma_*(A^+) 
\in \mathfrak I_{\widehat{A}}$.
\item If $B^+ \in \mathfrak I_{\widehat{A}}$, then $\gamma^*(B^+) \in \mathfrak I_{A}$. 
\end{enumerate}
\end{lem}

\begin{proof} 
By Proposition~\ref{p-hd-open}, 
we may assume that $A$ is Hausdorff 
and $A$ is an subring of $\widehat{A}$. 

We first show (1). 
Since $\gamma_*(A^+)=\widehat{A^+}$ and 
$\widehat{A^+} \subset \widehat{A^{\circ}} =(\widehat{A})^{\circ}$ 
(Lemma~\ref{l-pbdd-completion}), 
it suffices to show that $\widehat{A^+}$ 
is integrally closed in  $\widehat{A}$. 
Let $\beta \in \widehat{A}$ be an element which is integral over $\widehat{A^+}$. 
We can write 
$$\beta^n+\alpha_1\beta^{n-1}+\cdots+\alpha_n=0$$
for some $\alpha_i \in \widehat{A^+}$. 
There exist sequences $\{a_{i, k}\}_{k \in \Z_{>0}}\subset A^+$ and 
$\{b_k\}_{k \in \Z_{>0}} \subset A$ converging to $\alpha_i$ and $\beta$, 
respectively. 
We set 
$$b_k^n+a_{1, k}b_k^{n-1}+\cdots+a_{n, k}=:c_k \in A.$$
We see that the sequence $c_1, c_2, \cdots$ converges to zero. 
Since $A^+$ is an open subset of $A$, 
we can assume that $\{c_k\}_{k \in \Z_{>0}} \subset A^+$. 
In particular, each $b_k$ is integral over $A^+$. 
As $A^+$ is integrally closed in $A$, 
it follows that $\{b_k\}_{k \in \Z_{>0}} \subset A^+$. 
Therefore, we get $\beta \in \widehat{A^+}$, hence (1) holds. 



We now show (2). 
Set $\gamma^*(B^+)=\gamma^{-1}(B^+)=:A_1$. 
Since Lemma~\ref{l-pbdd-completion} implies 
$$B^+ \subset (\widehat{A})^{\circ}=\widehat{A^{\circ}}=\gamma_*(A^{\circ}),$$
we have that $A_1=\gamma^*(B^+) \subset \gamma^*\gamma_*(A^{\circ})=A^{\circ}$. 
Thus it suffices to prove that $A_1$ is integrally closed in $A$. 
Let $A_2$ be the integral closure of $A_1$ in $A$. 
It follows from $A_1 \subset A_2$ that $A_2$ is an open subring of $A$. 
Since $A_2$ is integral over $A_1$, 
it holds by Lemma \ref{l-integral-completion}(2) that 
$\gamma_*(A_2)=\overline{A_2}$ is integral over $\gamma_*(A_1)=\gamma_*\gamma^*(B^+)=B^+$. 
Thanks to $B^+ \in \mathfrak I_{\widehat{A}}$, 
we get $\overline{A_2} \subset B^+$. 
By Lemma~\ref{l-complete-open}, we have that  
$$A_2 =\gamma^*\gamma_*A_2=\gamma^{-1}(\overline{A_2}) \subset \gamma^{-1}(B^+)=A_1,$$ 
as desired. Thus (2) holds. 
\end{proof}




\begin{lem}\label{l-common-comp}
Let $A$ be a topological ring and let $\theta:A \to \widehat A$ be the completion of $A$. 
Let $\varphi:A \to B$ be a continuous ring homomorphism 
to a topological ring $B$ that satisfies the following properties:  
\begin{enumerate}
\item There exists a continuous 
ring homomorphism $\psi:B \to \widehat{A}$ 
such that $\theta=\psi \circ \varphi$. 
\item For any open neighbourhood $V$ of $0 \in B$, 
there exists an open neighbourhood $W$ of $0 \in \widehat{A}$ such that $\psi^{-1}(W) \subset V$. 
\end{enumerate}
Then the induced map $\widehat{\varphi}:\widehat{A} \to \widehat{B}$ 
is an isomorphism of topological rings. 
\end{lem}

\begin{proof}
We set $\theta_A:=\theta$ and let 
$\theta_B:B \to \widehat{B}$ be the completion, 
so that we get a commutative diagram: 
$$\begin{CD}
A @>\varphi >> B @>\psi >> \widehat{A}\\
@VV\theta_A V @VV\theta_B V @V\simeq V \widehat{\theta_A}V\\
\widehat{A} @>\widehat{\varphi}>> \widehat{B} @>\widehat{\psi} >> \widehat{\widehat{A}}.
\end{CD}$$

\begin{step}\label{s-common-comp1}
$\widehat{\psi}$ is injective. 
\end{step}

\begin{proof}(of Step~\ref{s-common-comp1}) 
Take $\beta \in \widehat{B}$ such that $\widehat{\psi}(\beta)=0$. 
There is a Cauchy net $\{b_i\}_{i \in I} \subset B$ whose image 
$\{\theta_B(b_i)\}_{i \in I}$ converges to $\beta$. 
Since $\{\psi(b_i)\}_{i \in I}$ is a Cauchy net of $\widehat{A}$, 
it converges to an element $\alpha$ of $\widehat{A}$. 
We have that 
$$0=\widehat{\psi}(\beta)=\widehat{\psi}(\lim_{i \in I} \theta_{B}(b_i))
=\widehat{\theta_A}(\alpha),$$
which implies $\alpha=0$. 
Take an open neighbourhood $V$ of $0 \in B$. 
It follows from (2) that there exists an open neighbourhood $W$ 
of $0 \in \widehat{A}$ such that $\psi^{-1}(W) \subset V$. 
Since $0=\alpha=\varinjlim_{i \in I}\psi(b_i)$, 
there exists an index $i_0 \in I$ such that $\psi(b_i) \in W$ 
for any $i \in I_{\geq i_0}$. 
In particular, 
we get $b_i \in \psi^{-1}(W) \subset V$ for any $i \in I_{\geq i_0}$. 
This implies that $\{b_i\}_{i \in I}$ converges to zero, 
hence $\beta=\theta_B(\lim_{i \in I} b_i)=0$, as desired. 
This completes the proof of Step~\ref{s-common-comp1}. 
\end{proof}


\begin{step}\label{s-common-comp2}
$\theta_B=\widehat{\varphi} \circ \psi$.
\end{step}

\begin{proof}(of Step~\ref{s-common-comp2}) 
Since $\widehat{\psi}$ is injective by Step~\ref{s-common-comp1}, 
it suffices to show that 
$\widehat{\psi} \circ \theta_B=\widehat{\psi} \circ 
\widehat{\varphi} \circ \psi$, which follows from 
$$\widehat{\psi} \circ 
\widehat{\varphi} \circ \psi=\widehat{\theta_A} \circ \psi=\widehat{\psi} \circ \theta_B,$$
where the first equation is guaranteed by (1). 
This completes the proof of Step~\ref{s-common-comp2}.
\end{proof}

\begin{step}\label{s-common-comp3}
$\widehat{\varphi}$ is bijective. 
\end{step}

\begin{proof}(of Step~\ref{s-common-comp3}) 
Since $\widehat{\psi} \circ \widehat{\varphi}$ is bijective, 
we have that $\widehat{\varphi}$ is injective. 

It suffices to show that $\widehat{\varphi}$ is surjective. 
Take $\beta \in \widehat{B}$. 
We can find a Cauchy net $\{b_i\}_{i \in I} \subset B$ 
such that its image $\{\theta_B(b_i)\}_{i \in I}$ converges to $\beta$. 
Since $\psi:B \to \widehat{A}$ is continuous, 
also $\{\psi(b_i)\}_{i \in I}$ is a Cauchy net. 
Thus we can find an element $\alpha \in \widehat{A}$ 
with $\alpha=\lim_{i \in I}\psi(b_i)$. 
We have that 
$$\widehat{\varphi}(\alpha)=\widehat{\varphi}\left(\lim_{i \in I}\psi(b_i)\right)=\lim_{i \in I}\theta_B(b_i)=\beta,$$
where the second equation holds by Step~\ref{s-common-comp2}. 
Thus $\widehat{\varphi}$ is surjective. 
This completes the proof of Step~\ref{s-common-comp3}.
\end{proof}

\begin{step}\label{s-common-comp4}
$\widehat{\varphi}$ is an open map. 
\end{step}

\begin{proof}(of Step~\ref{s-common-comp4}) 
Let $U$ be an open subset of $\widehat{A}$. 
Since both $\widehat{\varphi}$ and $\widehat{\psi}$ are bijective by Step~\ref{s-common-comp3}, 
we get an equation: 
$$\widehat{\varphi}(U)=\widehat{\psi}^{-1}(\widehat{\psi}(\widehat{\varphi}(U))).$$
Since $\widehat{\psi}(\widehat{\varphi}(U))$ is an open subset and 
$\widehat{\psi}$ is continuous, 
$\widehat{\varphi}(U)$ is an open set. 
This completes the proof of Step~\ref{s-common-comp4}.
\end{proof}
Step~\ref{s-common-comp3} 
and Step~\ref{s-common-comp4} 
complete the proof of Lemma~\ref{l-common-comp}. 
\end{proof}

\subsection{Topological tensor products}

In this subsection, we introduce tensor products 
for the category of f-adic rings (Theorem~\ref{t-top-tensor}). 


\begin{dfn}\label{d-tensor}
Let $\mathcal R$ be a category. 
For two arrows $\varphi:R \to A$ and $\psi:R \to B$ in $\mathcal R$, 
we say that $(T, f, g)$ is a {\em tensor product} in $\mathcal R$ of $(\varphi, \psi)$ or of a diagram 
$$\begin{CD}
R @>\varphi >> A\\
@VV\psi V\\
B,
\end{CD}$$
if the following hold: 
\begin{enumerate}
\item 
$T$ is an object of $\mathcal R$, 
\item 
$f:A \to T$ and $g:B \to T$ are arrows of $\mathcal R$ 
such that $f \circ \varphi=g \circ \psi$, and
\item 
for a commutative diagram of arrows of $\mathcal R$ 
$$\begin{CD}
R @>\varphi >> A\\
@VV\psi V @VVf'V\\
B @>g'>>C,
\end{CD}$$
there exists a unique arrow $\theta:T \to C$ in $\mathcal R$ 
satisfying $f'=\theta \circ f$ and $g'=\theta \circ g'$. 
\end{enumerate}
We often call $T$ {\em the tensor product} of $(\varphi, \psi)$ 
in $\mathcal R$ if no confusion arises. 
\end{dfn}

\begin{dfn}\label{d-ring-tensor}
The tensor products in the category of rings are called 
{\em ring-theoretic tensor products}. 
\end{dfn}

\begin{dfn}\label{d-categories}
\begin{enumerate}
\item 
Let $(\text{TopRing})$ be the category of topological rings  
whose arrows are the continuous ring homomorphisms. 
\item 
Let $(\text{FadRing})$ be the full subcategory of $(\text{TopRing})$ 
whose objects are f-adic rings.  
\item 
Let $(\text{FadRing})^{\text{ad}}$ 
be the category of f-adic rings 
whose arrows are the adic ring homomorphisms. 
\end{enumerate}
\end{dfn}


\begin{lem}\label{l-underlying}
Let $\mathcal R$ be one of 
the categories 
$({\rm TopRing})$ and $({\rm FadRing})$. 
Let $\varphi:R \to A$ and $\psi:R \to B$ be two arrows 
of $\mathcal R$. 
Assume that there exists a tensor product $(T, f, g)$ in $\mathcal R$ 
of $(\varphi, \psi)$, 
then the induced ring homomorphism $A \otimes_R B \to T$ 
from the ring-theoretic tensor product $A \otimes_R B$ is bijective. 
\end{lem}

\begin{proof}
We only treat the case where $\mathcal R=(\text{FadRing})$, 
as both the proofs are the same. 
Take a ring $C$ and a commutative diagram of ring homomorphisms: 
$$\begin{CD}
R @>\varphi >> A\\
@VV\psi V @VVf'V\\
B @>g'>> C.
\end{CD}$$
We equip $C$ with the trivial topology. 
Then $C$ is an f-adic ring such that $f'$ and $g'$ are continuous. 
Since $(T, f, g)$ is a tensor product in $(\text{FadRing})$ of $(\varphi, \psi)$, 
there exists a unique continuous ring homomorphism 
$\theta:T \to C$ such that $f'=\theta \circ f$ and $g'=\theta \circ g$. 
Take another ring homomorphism $\widetilde{\theta}:T \to C$ 
satisfying $f'=\widetilde \theta \circ f$ and $g'=\widetilde \theta \circ g'$. 
Then $\widetilde \theta$ is automatically continuous, 
hence it follows from Definition~\ref{d-tensor} that 
$\theta=\widetilde \theta$. 
Therefore the underlying ring $T$ satisfies the universal property 
characterising ring-theoretic tensor products. 
Thus the induced ring homomorphism $A \otimes_R B \to T$ 
is bijective. 
\end{proof}





\begin{thm}\label{t-top-tensor}
Let $\varphi:R \to A$ and $\psi:R \to B$ 
be adic ring homomorphisms of f-adic rings. 
Then the following hold:  
\begin{enumerate}
\item 
There exists a tensor product $(T, f, g)$ in $({\rm FadRing})$ 
of $(\varphi, \psi)$. 
\item 
$(T, f, g)$ is a tensor product in $({\rm FadRing})^{{\rm ad}}$ 
of $(\varphi, \psi)$. 
\item 
The induced ring homomorphism $A \otimes_R B \to T$ 
from the ring-theoretic tensor product $A \otimes_R B$ is bijective. 
\item 
The induced continuous ring homomorphisms $f:A \to T$ and $g:B \to T$ are adic. 
\end{enumerate}
\end{thm}

An f-adic ring $T$ satisfying the above properties 
is called a {\em topological tensor product} of $(\varphi, \psi)$. 
Because of (3), we often denote it by $A \otimes_R B$. 


\begin{proof}
As a ring, we set $T:=A \otimes_R B$. 
Let 
$$\begin{CD}
R @>\varphi >> A\\
@VV\psi V @VVfV\\
B @>g>> T
\end{CD}$$
be the induced commutative diagram of ring homomorphisms. 
Let $A_0$ and $B_0$ be rings of definition of $A$ and $B$, respectively. Since $\varphi^{-1}(A_0) \cap \psi^{-1}(B_0)$ is an open subring of $R$, 
we can find a ring of definition $R_0$ of $R$ 
such that $R_0 \subset \varphi^{-1}(A_0) \cap \psi^{-1}(B_0)$. 
We have the natural ring homomorphism 
$$\rho:A_0 \otimes_{R_0} B_0 \to A \otimes_R B.$$
We take an ideal of definition $I_0$ of $R_0$. 
It follows from \cite[Lemma 1.8(ii)]{Hub93} that $I_0A_0$ and $I_0B_0$ are 
ideals of definition of $A_0$ and $B_0$, respectively. 
We equip $T=A \otimes_R B$ with the group topology defined by 
$$\left\{\rho\left(I_0^k\cdot(A_0 \otimes_{R_0} B_0)\right)\right\}_{k \in \Z_{>0}}.$$ 
We can check that this topology does not depend 
on the choices of $R_0, A_0,  B_0$ and $I_0$. 


\begin{claim}
The ring $T$, equipped with the group topology defined as above, 
is an f-adic ring such that the induced ring homomorphisms $f:A \to T$ and $g:B \to T$ 
are adic.  
\end{claim}

\begin{proof}(of Claim) 
To show that $T$ is a topological ring, 
we prove that the property (2) of Lemma~\ref{l-top-criterion} holds. 
Take $\xi \in A \otimes_R B$ and $n \in \Z_{>0}$. 
We can write $\xi=\sum_{i=1}^r a_i \otimes_R b_i$ 
for some $a_i \in A$ and $b_i \in B$. 
Since $A$ and $B$ are topological rings, 
Lemma~\ref{l-top-criterion} enables us to find $m_1 \in \Z_{>0}$ satisfying  
$$a_i I_0^{m_1}A_0 \subset I_0^nA_0,\quad b_i I_0^{m_1}B_0 \subset I_0^nB_0$$ 
for any $i \in \{1, \cdots, r\}$. 
In particular, we obtain 
$$\xi \cdot \rho\left(I_0^{2m_1}\cdot(A_0 \otimes_{R_0} B_0)\right) 
\subset \rho\left(I_0^{2n}\cdot(A_0 \otimes_{R_0} B_0)\right),$$
hence (2) of Lemma~\ref{l-top-criterion} holds for $T$. 
Therefore, $T$ is a topological ring. 

Moreover, it follows from definition of the topology on $T$ 
that $T$ is an f-adic ring and both of $f$ and $g$ are adic. 
This completes the proof of Claim. 
\end{proof}

We will show the following property: 
\begin{enumerate}
\item[$(1)'$] $(T, f, g)$ is a tensor product in $({\rm FadRing})$ of $(\varphi, \psi)$. 
\end{enumerate}

For the time being, 
let us check that the statement of the theorem holds if $(1)'$ holds. 
We have that (1) follows from $(1)'$. 
We obtain (3) and (4) by the construction of $T$ and Claim, respectively. 
It follows from $(1)'$ and \cite[Corollary 1.9(ii)]{Hub93} that (2) automatically holds. 
Therefore, the statement of Theorem \ref{t-top-tensor} holds if $(1)'$ holds. 

\medskip 

Thus it suffices to show $(1)'$. 
Take a commutative diagram 
$$\begin{CD}
R @>\varphi >> A\\
@VV\psi V @VVf'V\\
B @>g'>> C 
\end{CD}$$
in $({\rm FadRing}).$ 
Since $T$ is a ring-theoretic tensor product, 
there exists a unique ring homomorphism $\theta:T \to C$ 
such that $f'=\theta \circ f$ and $g'=\theta \circ g$. 
It is enough to show that $\theta$ is continuous. 

Let $C_1$ be an open pseudo-subring of $C$. 
Since $f'$ and $g'$ are continuous, 
we can find a positive integer $k$ such that 
$I_0^kA_0 \subset f'^{-1}(C_1)$ and $I_0^kB_0 \subset g'^{-1}(C_1)$. 
Then we have that 
$$\{x \otimes_R 1\,|\, x \in I_0^kA_0\} \cup 
\{1 \otimes_R y\,|\, y \in I_0^kB_0\} \subset \theta^{-1}(C_1).$$ 
Since $\theta^{-1}(C_1)$ is a pseudo-subring of $T$, 
we get 
$$\rho(I_0^{2k}(A_0 \otimes_{R_0} B_0)) \subset \theta^{-1}(C_1).$$ 
Therefore $\theta$ is continuous. 
Thus $(1)'$ holds, 
hence so does the statement of Theorem~\ref{t-top-tensor}. 
\end{proof}




\subsection{Structure sheaves of Zariskian schemes}

\subsubsection{Affine case}\label{ss-FK-affine}

In this subsection, we summarise some results from \cite{FK} for later use. 
Let $A$ be a ring and let $I$ be an ideal of $A$. 
For an $A$-module $M$, recall that $\widetilde M$ 
is the quasi-coherent sheaf on $\Spec\,A$ satisfying 
$M=\Gamma(\Spec\,A, \widetilde M)$. 
We define a sheaf $M^{\diamondsuit}$ on a topological space $V(I)$ by 
$$M^{\diamondsuit}:=i^{-1}(\widetilde M).$$
The following theorem is nothing but \cite[Ch. I, Proposition B.1.4]{FK}, 
however we give a proof of it 
since the proof of \cite[Ch. I, Proposition B.1.4]{FK} omits some of arguments. 



\begin{thm}\label{t-FK-sheafy}
Let $A$ be a ring and let $I$ be an ideal of $A$. 
For any $A$-module $M$ and element $f \in A$, 
the equation 
$$\Gamma(V(I) \cap D(f), M^{\diamondsuit})=M \otimes_A (1+IA_f)^{-1}A_f$$
holds. 
\end{thm}

\begin{proof}
Take elements $f_1, \cdots, f_n \in A$ such that 
$V(I) \subset D(f_1) \cup \cdots \cup D(f_n)$. 
For any $A$-module $N$ and any element $g \in A$, we set 
$$N_g^Z:=N \otimes_A (1+IA_g)^{-1}A_g$$
and $N^Z:=N_1^Z$. 
By the proof of \cite[Ch. I, Proposition B.1.4]{FK}, 
it suffices to show that the sequence 
\begin{equation}\label{zar-sch-sheafy}
0 \to M^Z \xrightarrow{\varphi} 
\prod_{1 \leq i\leq r} M_{f_i}^{Z} \xrightarrow{\psi} 
\prod_{1 \leq i<j \leq r} M_{f_if_j}^{Z}
\end{equation}
is exact, where $\varphi$ is defined by $\varphi(m)=(m \otimes_A 1, \cdots, m \otimes_A 1)$ for any $m \in M$ and 
$\psi$ is defined by the difference. 




Replacing $A$ and $M$ by $A^Z$ and $M^Z$ respectively, 
we may assume that $A=(1+I)^{-1}A$. 
Thanks to the inclusion $V(I) \subset D(f_1) \cup \cdots \cup D(f_r)$, 
the images of the elements $f_1, \cdots, f_r$ to $A/I$ 
generate $A/I$. 
In particular, we can find elements $g_1, \cdots, g_r \in A$ such that 
$\sum_{i=1}^rg_if_i=1+z$ for some $z \in I$. 
Since $1+z \in 1+I \subset A^{\times}$, 
we may assume that $z=0$, i.e. the equation 
\begin{equation}\label{e-FK-sheafy1}
\sum_{i=1}^rg_if_i=1
\end{equation}
holds in $A$.  

\setcounter{step}{0}
\begin{step}\label{s-domain}
The sequence (\ref{zar-sch-sheafy}) is exact if $M=A/\p$ for some prime ideal of $A$. 
\end{step}

\begin{proof}(of Step~\ref{s-domain}) 
We may assume that $M=A$ and $A$ is an integral domain. 
Since $A$ is an integral domain, it is clear that $\varphi$ is injective. 
Take $(\xi_1, \cdots, \xi_r) \in \prod_{1 \leq i\leq r}(1+IA_{f_i})^{-1}A_{f_i}$ 
such that $\psi((\xi_1, \cdots, \xi_r))=0$. 
For each $i \in \{1, \cdots, r\}$, we can write 
$$\xi_i=\frac{\frac{a_i}{f_i^{n_i}}}{1+\frac{x_i}{f_i^{m_i}}}.$$
for some $a_i\in A$, $x_i \in I$ and $n_i, m_i \in \Z_{> 0}$. 
We may assume that there is a positive integer $n$ such that $n=n_i=m_i$ 
for any $i \in \{1, \cdots, r\}$. 
Moreover, replacing $f_i^n$ by $f_i$, 
the problem is reduced to the case where $n=1$. 
Thus we obtain 
$$\xi_i=\frac{\frac{a_i}{f_i}}{1+\frac{x_i}{f_i}}.$$
Since $\psi((\xi_1, \cdots, \xi_r))=0$, we get an equation
$$\frac{\frac{a_i}{f_i}}{1+\frac{x_i}{f_i}}=\frac{\frac{a_j}{f_j}}{1+\frac{x_j}{f_j}}$$
in $(1+IA_{f_if_j})^{-1}A_{f_if_j}$ for any $i, j\in \{1, \cdots, r\}$. 
Since $A$ is an integral domain, we get an equation 
\begin{equation}\label{e-FK-sheafy2}
(f_j+x_j)a_i=(f_i+x_i)a_j
\end{equation}
in $A$ for any $i, j\in \{1, \cdots, r\}$. 
We set 
$$a:=\sum_{j=1}^ra_jg_j.$$
Then, it holds that 
$$(f_i+x_i)a=\sum_{j=1}^r(f_i+x_i)a_jg_j=\sum_{j=1}^r(f_j+x_j)a_ig_j
=a_i\left(1+\sum_{j=1}^rx_jg_j\right),$$
where the second and third equations hold by 
(\ref{e-FK-sheafy2}) and (\ref{e-FK-sheafy1}) respectively. 
Since $1+\sum_{j=1}^rx_jg_j \in 1+IA \subset A^{\times}$, 
we get $(1+\sum_{j=1}^rx_jg_j)^{-1} \in A$. 
Thus, for 
$$\widetilde a:=\left(1+\sum_{j=1}^r x_jg_j\right)^{-1}a,$$
it holds that $\varphi(\widetilde a)=(\xi_1, \cdots, \xi_r)$, as desired. 
This completes the proof of Step~\ref{s-domain}. 
\end{proof}

\begin{step}\label{s-thickening}
Let $0 \to M_1 \to M_2 \to M_3 \to 0$ be an exact sequence of $A$-modules. 
If 
(\ref{zar-sch-sheafy}) is exact for the cases $M =M_1$ and $M=M_3$, then 
(\ref{zar-sch-sheafy}) is exact also for the case $M=M_2$. 
\end{step}


\begin{proof}(of Step~\ref{s-thickening}) 
We have the following commutative diagram 
$$\begin{CD}
@. 0 @. 0 @. 0 @.\\
@. @VVV @VVV @VVV\\
0 @>>> M_1^{Z} @>>> \prod_{1\leq i \leq r} (M_1)_{f_i}^Z @>>> \prod_{1 \leq i<j\leq r} (M_1)_{f_if_j}^Z\\
@. @VVV @VVV @VVV\\
0 @>>> M_2^{Z} @>>> \prod_{1\leq i \leq r} (M_2)_{f_i}^Z @>>> \prod_{1 \leq i<j\leq r} (M_2)_{f_if_j}^Z\\
@. @VVV @VVV @VVV\\
0 @>>> M_3^{Z} @>>> \prod_{1\leq i \leq r} (M_3)_{f_i}^Z @>>> \prod_{1 \leq i<j\leq r} (M_3)_{f_if_j}^Z\\
@. @VVV @VVV @VVV\\
@. 0 @. 0 @. 0 @.\\
\end{CD}$$
Since $A \to A^{Z}_f=(1+IA_f)^{-1}A_f$ is flat for any element $f \in A$, 
all the vertical sequences are exact. 
Moreover, also the upper and lower horizontal sequences are exact by assumption. 
It follows from a diagram chase that the middle horizontal sequence is exact, as desired. 
This completes the proof of Step~\ref{s-thickening}.
\end{proof}


\begin{step}\label{s-noether}
The sequence (\ref{zar-sch-sheafy}) is exact if $A$ is a noetherian ring and $M$ is a finitely generated $A$-module. 
\end{step}

\begin{proof}(of Step~\ref{s-noether}) 
Thanks to \cite[Theorem 6.5]{Mat89}, 
we can find a descending sequence of $A$-submodules of $M$: 
$$M=:M_0 \supset M_1 \supset \cdots \supset M_{\ell-1} \supset M_{\ell}=0$$
such that for any $k \in \{0, \cdots, \ell-1\}$, 
there exists a prime ideal $\p_k$ of $A$ such that $M_k/M_{k+1} \simeq A/\p_k$. 
By Step~\ref{s-thickening}, we may assume that $M=A/\p$ for some prime ideal $\p$ of $A$. 
The assertion of Step~\ref{s-noether} holds by Step~\ref{s-domain}. 
\end{proof}



\begin{step}\label{s-general}
The sequence (\ref{zar-sch-sheafy}) is exact without any additional assumptions. 
\end{step}

\begin{proof}(of Step~\ref{s-general}) 
We only show the exactness on $\prod_{1 \leq i\leq r} M_{f_i}^{Z}$, as the remaining case is easier. 
Take $(\xi_1, \cdots, \xi_r) \in \prod_{1 \leq i\leq r} M_{f_i}^{Z}$ 
such that $\psi((\xi_1, \cdots, \xi_r))=0$. 
Since 
$$M^Z_{f_i}=M \otimes_A (1+IA_{f_i})^{-1}A_{f_i},$$
we can write 
$$\xi_i=\frac{\frac{\mu_i}{f_i^{n_i}}}{1+\frac{x_i}{f_i^{m_i}}}$$
for some $x_i \in I$, $\mu_i \in M$, $n_i, m_i \in \Z_{>0}$. 
Since $\psi((\xi_1, \cdots, \xi_r))=0$, 
for any $i, j \in \{1, \cdots, r\}$, an equation  
\begin{equation}\label{e-psi-relation}
\begin{split}
\,\,\,\,&(f_if_j)^{k_{ij}}((f_if_j)^{\ell_{ij}}+y_{ij})(f_i^{m_i}f_j^{m_j}+f_i^{m_i}x_j)f_j^{n_j}\mu_i \\
=&(f_if_j)^{k_{ij}}((f_if_j)^{\ell_{ij}}+y_{ij})(f_i^{m_i}f_j^{m_j}+f_j^{m_j}x_i)f_i^{n_i}\mu_j
\end{split}
\end{equation}
holds in $M$ for some $y_{ij} \in I$, $\ell_{ij}, k_{ij} \in \Z_{>0}$. 



Let $B_1$ be the finitely generated $\Z$-subalgebra of $A$ generated by $\{f_i\} \cup \{g_i\} \cup \{a_i\} \cup \{x_i\} \cup \{y_{ij}\}$. 
Let 
$$J:=\sum_{i=1}^r B_1x_i+\sum_{1 \leq i, j \leq r} B_1y_{ij} \subset B_1.$$
Let $B:=(1+J)^{-1}B_1$. 
We have a natural ring homomorphism $B \to A$. 
In particular, $M$ is an $B$-module. 
Let $N$ be the finitely generated $B$-submodule of $M$ generated by $\mu_1, \cdots, \mu_r$. 
Thanks to Step~\ref{s-noether}, the sequence 
$$0 \to N \xrightarrow{\varphi'} \prod^r_{i=1} N \otimes_B (1+JB_{f_i})^{-1}B_{f_i}
\xrightarrow{\psi'} \prod_{1 \leq i, j \leq r} N \otimes_B (1+JB_{f_if_j})^{-1}B_{f_if_j}$$
is exact. 
Take an element 
$$\eta:=\left(\frac{\frac{\mu_1}{f_1^{n_1}}}{1+\frac{x_1}{f_1^{m_1}}}, \cdots, \frac{\frac{\mu_r}{f_r^{n_r}}}{1+\frac{x_r}{f_i^{m_r}}}\right) \in 
\prod^r_{i=1} N \otimes_B (1+JB_{f_i})^{-1}B_{f_i}.$$
It follows from (\ref{e-psi-relation}) that $\psi'(\eta)=0$. 
Thus, we can find an element $\mu \in N$ such that $\varphi'(\mu)=\eta$. 
For any $i \in \{1, \cdots, r\}$, the equation 
$$f_i^{n_i+q_i}(f_i^{p_i}+z_i)(f_i^{m_i}+x_i)\mu=f_i^{m_i+q_i}(f_i^{p_i}+z_i)\mu_i$$
holds in $N$ for some $p_i, q_i \in \Z_{>0}$ and $z_i \in J$. 
Therefore, we get 
$$\varphi(\mu)=\left(\frac{\frac{\mu_1}{f_1^{n_1}}}{1+\frac{x_1}{f_1^{m_1}}}, \cdots, \frac{\frac{\mu_r}{f_r^{n_r}}}{1+\frac{x_r}{f_i^{m_r}}}\right),$$
as desired. 
This completes the proof of Step~\ref{s-general}. 
\end{proof}
Step~\ref{s-general} completes the proof of Theorem~\ref{t-FK-sheafy}. 
\end{proof}

\subsubsection{Global case}\label{ss-FK-global}

Let $X$ be a scheme and let $V$ be a closed subset of $X$ equipped with the induced topology from $X$. 
Set $i:V \hookrightarrow X$ to be the induced continuous map. 
For a quasi-coherent sheaf $F$ on $X$ and an affine open subset $U$ of $X$ with $A:=\Gamma(U, \MO_X)$, 
it follows from Theorem~\ref{t-FK-sheafy} that 
\begin{equation}\label{e-FK-sheafy}
\Gamma(V \cap U, i^{-1}(F))=\Gamma(U, F) \otimes_A (1+IA)^{-1}A, 
\tag{\ref{ss-FK-global}.1}
\end{equation}
where $I$ is an ideal of $A$ such that 
the closed subset $V(I)$ of $U=\Spec\,A$ corresponding to $I$ 
is equal to $V \cap U$. 


