


\section{Zariskian adic spaces}

In this section, we introduce Zariskian adic spaces. 
In Subsection~\ref{ss-affinoid}, 
we establish foundations for the affinoid case. 
In particular we define a presheaf $\MO_A^{\Zar}$ on $\Spa\,A$ 
for any affinoid ring $A$. 
In Subsection~\ref{ss-structure}, 
we prove that $\MO_A^{\Zar}$ is actually a sheaf. 
In Subsection~\ref{ss-zar-ad-sp}, 
we define Zariskian adic spaces. 
In Subsection~\ref{ss-cex-Tate}, 
we exhibit examples that violate the Tate acyclicity. 

\subsection{Affinoid case}\label{ss-affinoid}

In this subsection, we introduce a presheaf $\MO_A^{\Zar}$ on $\Spa\,A$ 
for any affinoid ring $A$. 
This is a Zariskian analogue of Huber's affinoid adic spaces 
introduced in \cite[Section 1]{Hub94}. 
Indeed, most of our arguments are quite similar 
to the ones in \cite[Section 1]{Hub94}. 

In (\ref{sss-rat-localise}), we consider 
what the definition of $\MO_A^{\Zar}(U)$ should be for rational subsets $U$. 
In (\ref{sss-zar-str}), we define $\MO_A^{\Zar}$  
based on results obtained in (\ref{sss-rat-localise}). 


\subsubsection{Rational localisation}\label{sss-rat-localise}



Let $A=(A^{\rhd}, A^+)$ be an affinoid ring and 
let $U$ be a rational subset of $\Spa\,A$. 
Then there exist elements $f_1, \cdots, f_r, g \in A^{\rhd}$ such that 
$(f_1, \cdots, f_r, g)$ is an open ideal of $A^{\rhd}$ and 
$$U=R\left(\frac{f_1, \cdots, f_r}{g}\right).$$
Take a ring of definition $A_0$ of $A^{\rhd}$ 
and an ideal of definition $I_0$ of $A_0$. 
We set $B^{\rhd}:=A^{\rhd}[1/g]=(A^{\rhd})_g$ and  
$B_0:=A_0\left[f_1/g, \cdots, f_r/g\right] \subset B^{\rhd}.$ 
We equip $B^{\rhd}$ with the group topology induced by $\{I^k_0 B_0\}_{k \in \Z_{>0}}$. 
We can check that this topology does not depend on the choice 
of $A_0$ and $I_0$. 
Set $B^+$ to be the integral closure of 
$B^+[f_1/g, \cdots, f_r/g]$ in $B^{\rhd}$. 
We set 
$$A\left(\frac{f_1, \cdots, f_r}{g}\right):=(B^{\rhd}, B^+).$$
It follows that $A\left(\frac{f_1, \cdots, f_r}{g}\right)$ is 
an affinoid ring and the induced ring homomorphism 
$A \to A\left(\frac{f_1, \cdots, f_r}{g}\right)$ 
is adic. 
Let $A\left(\frac{f_1, \cdots, f_r}{g}\right)^{\Zar}$ be 
the Zariskisation of $A\left(\frac{f_1, \cdots, f_r}{g}\right)$ and 
let 
$$\varphi=\varphi_U:A \xrightarrow{\alpha} A\left(\frac{f_1, \cdots, f_r}{g}\right) 
\xrightarrow{\theta} A\left(\frac{f_1, \cdots, f_r}{g}\right)^{\Zar}=:Z_A(U)$$
be the induced adic ring homomorphisms. 
Although it is not clear that $Z_A(U)$ does not depend on the choices 
of $f_1, \cdots, f_r, g$, 
we shall later see in Lemma~\ref{l-rat-universal1} that this is actually true. 
To this end, we first establish an auxiliary result: 
Lemma~\ref{l-zar-nonempty}. 





\begin{lem}\label{l-zar-nonempty}
Let $A=(A^{\rhd}, A^+)$ be a Zariskian affinoid ring. 
Then the following hold. 
\begin{enumerate}
\item $(A^{\rhd})^{\times}$ is an open subset of $A^{\rhd}$. 
\item If $\m$ is a maximal ideal of $A^{\rhd}$, 
then there exists a point $v \in \Spa\,A$ such that $\m=\Ker(v)$. 
\end{enumerate}
\end{lem}

\begin{proof}
First we show (1). 
Since $1+(A^{\rhd})^{\circ\circ}$ is an open subset of $A^{\rhd}$, 
so is 
$$(A^{\rhd})^{\times}=\bigcup_{x \in (A^{\rhd})^{\times}} x \cdot (1+(A^{\rhd})^{\circ\circ}).$$ 
Thus the assertion (1) holds. 

Second we show (2). 
Since $(A^{\rhd})^{\times}$ is an open subset of $A^{\rhd}$ by (1),  
the inclusion 
$\m \subset A^{\rhd} \setminus (A^{\rhd})^{\times}$ 
implies 
$\overline{\m} \subset A^{\rhd} \setminus (A^{\rhd})^{\times}$, 
where $\overline{\m}$ denotes the closure of $\m$ in $A^{\rhd}$. 
Since $\overline{\m}$ is again an ideal of $A^{\rhd}$ 
such that $\m \subset \overline{\m} \subsetneq A^{\rhd}$, 
we have that $\m=\overline{\m}$. 
We set $B^{\rhd}:=A^{\rhd}/\m$ to be the topological ring 
equipped with the quotient topology induced from $A^{\rhd}$. 
Since $\m=\overline{\m}$, it holds that $B^{\rhd}$ is Hausdorff. 
Let $B^+$ be the integral closure of the image of $A^+$. 
Then $B:= (B^{\rhd}, B^+)$ is an affinoid ring 
(cf. Subsection \ref{ss-quot-aff}). 
It follows from \cite[Proposition 3.6(i)]{Hub93} that 
$\Spa\,B \neq \emptyset$. 
Since we have a natural continuous ring homomorphism $A \to B$, 
any element of the image of $\Spa\,B \to \Spa\,A$ 
is an element $v \in \Spa\,A$ as in the statement. 
Thus (2) holds. 
\end{proof}


\begin{lem}\label{l-rat-universal1}
Let $A=(A^{\rhd}, A^+)$ be an affinoid ring. 
Let $U$ be a rational subset of $\Spa\,A$ 
and let $f_1, \cdots, f_r, g \in A^{\rhd}$ be elements 
such that $(f_1, \cdots, f_r, g)$ is an open ideal of $A^{\rhd}$ and 
$$U=R\left(\frac{f_1, \cdots, f_r}{g}\right).$$
If $\psi:A \to C$ is a continuous ring homomorphism 
to a Zariskian affinoid ring 
such that the image of the induced map $\Spa\,(\psi):\Spa\,C \to \Spa\,A$ 
is contained in $U$,  
then there exists a unique continuous ring homomorphism 
$$\psi':A\left(\frac{f_1, \cdots, f_r}{g}\right)^{\Zar} \to C$$ 
such that $\psi=\psi' \circ \varphi$.  
\end{lem}

\begin{proof}
Let $\psi:A \to C$ be a continuous ring homomorphism 
to a Zariskian affinoid ring $C=(C^{\rhd}, C^+)$ 
satisfying ${\rm Im}\,(\Spa\,(\psi)) \subset U$. 
We now show the following two assertions. 
\begin{enumerate}
\item[(i)] $\psi(g) \in (C^{\rhd})^{\times}$. 
\item[(ii)] $\psi(f_1)/\psi(g), \cdots, \psi(f_r)/\psi(g) \in C^+$. 
\end{enumerate}

First we prove (i). 
By $\Image(\Spa\,(\psi)) \subset U=R\left(\frac{f_1, \cdots, f_r}{g}\right)$, 
we have that $v(\psi(g)) \neq 0$ for any $v \in \Spa\,C$. 
It follows from Lemma~\ref{l-zar-nonempty}(2) that 
$\psi(g)$ is not contained in any maximal ideal of $C^{\rhd}$. 
Thus (i) holds. 

Second we show (ii). 
Again by $\Image(\Spa\,(\psi)) \subset U=R\left(\frac{f_1, \cdots, f_r}{g}\right)$, 
we have that $v(\psi(f_i)/\psi(g)) \leq 1$ 
for any $v \in \Spa\,C$ and any $i \in \{1, \cdots, r\}$. 
By \cite[Lemma 3.3(i)]{Hub93}, we get $\psi(f_i)/\psi(g) \in C^+$ 
for any $i \in \{1, \cdots, r\}$. 
Thus (ii) holds. 

\medskip

By (i) and (ii), 
there exists a unique ring homomorphism 
$\psi_1:A\left(\frac{f_1, \cdots, f_r}{g}\right) \to C$ 
inducing the following factorisation:  
$$\psi:A \xrightarrow{\alpha} A\left(\frac{f_1, \cdots, f_r}{g}\right) \xrightarrow{\psi_1} C.$$
It follows from \cite[(1.2)(ii)]{Hub94} that $\psi_1$ is continuous. 
Taking the Zariskisation of $\psi_1$, we get a unique factorisation of $\psi_1$: 
$$\psi:A \to A\left(\frac{f_1, \cdots, f_r}{g}\right) \xrightarrow{\theta} A\left(\frac{f_1, \cdots, f_r}{g}\right)^{\Zar} \xrightarrow{\psi'} C.$$
We are done. 
\end{proof}





\begin{lem}\label{l-rat-universal2}
Let $A$ be an affinoid ring and let $U$ be a rational subset of $\Spa\,A$. 
Then the following assertions hold. 
\begin{enumerate}
\item 
If $V$ is a rational subset of $\Spa\,A$ such that $V \subset U$, then 
there is a unique continuous ring homomorphism 
$Z_A(U) \to Z_A(V)$ that commutes with 
$\varphi_V:A \to Z_A(V)$ and $\varphi_U:A \to Z_A(U)$. 
\item 
The induced map $\Spa\,(\varphi_U):\Spa\,Z_A(U) \to \Spa\,A$ 
is an open injective map whose image is equal to $U$. 
If $W$ is a rational subset of $\Spa\,Z_A(U)$, then 
so is $\Spa\,(\varphi_U)(W)$. 
If $V$ is a rational subset of $\Spa\,A$ contained in $U$, then 
also $(\Spa\,(\varphi_U))^{-1}(V)$ is a rational subset. 
\item 
Set $B:=Z_A(U)$ and $g:=\Spa(\varphi_U):\Spa\,B \to \Spa\,A$. 
Let $V$ be a rational subset of $\Spa\,A$ contained in $U$. 
Then there exists a unique continuous ring homomorphism 
$r:Z_A(V) \to Z_B(g^{-1}(V))$ such that the following diagram is commutative: 
$$\begin{CD}
Z_B(g^{-1}(V)) @<<< Z_A(V)\\
@AAA @AAA\\
B @<<< A.
\end{CD}$$
Furthermore, $r$ is an isomorphism of topological rings. 
\end{enumerate}
\end{lem}

\begin{proof}
The assertion (1) holds by Lemma~\ref{l-rat-universal1}. 
The assertion (2) follows from \cite[Lemma 1.5(ii)]{Hub94} 
and Theorem \ref{t-comp-factor}. 
As for (3), 
we can apply the same argument as in \cite[Lemma 1.5(iii)]{Hub94} 
after replacing $F_A(-)$ and \cite[Lemma 1.3]{Hub94} by $Z_A(-)$ and 
Lemma~\ref{l-rat-universal1}, respectively. 
\end{proof}


\subsubsection{Zariskian structure presheaves}\label{sss-zar-str}

Let $A=(A^{\rhd}, A^+)$ be an affinoid ring. 
For any open subset $V$ of $\Spa\,A$, we set 
$$\Gamma(V, \MO_A^{\Zar}):=\MO_A^{\Zar}(V):=
\varprojlim_{\substack{U \subset V,\\ 
U:\text{ rational}\\ 
\text{subset}}} Z_A(U)^{\rhd},$$
where the inverse limit is taken in the category of $A^{\rhd}$-algebras. 
We equip $\MO_A^{\Zar}(V)$ with the inverse limit topology. 
If $V_1$ and $V_2$ are open subsets of $\Spa\,A$ satisfying 
$V_1 \supset V_2$, 
then the induced ring homomorphism  $\MO_A^{\Zar}(V_1) \to \MO_A^{\Zar}(V_2)$ is continuous. 
Thus $\MO^{\Zar}_A$ is a presheaf of topological rings. 

Fix $x \in \Spa\,A$. 
Set 
$$\MO^{\Zar}_{A, x}:=\varinjlim_{x \in U} \MO^{\Zar}_A(U),$$ 
where the direct limit is taken in the category of rings. 
We obtain a natural isomorphism of rings: 
$$\varinjlim_{\substack{x \in U,\\ 
U:\text{ rational}\\ 
\text{subset}}} \MO^{\Zar}_A(U) \xrightarrow{\simeq} 
\varinjlim_{x \in U} \MO^{\Zar}_A(U).$$
For any rational subset $U$ with $x \in U$, 
the valuation $x:A^{\rhd} \to \Gamma_x \cup \{0\}$ 
is uniquely extended to a valuation $v_U:\MO_A^{\Zar}(U) \to \Gamma_x \cup \{0\}$. 
Thus the set of valuations $\{v_U\}_{x \in U}$ define a valuation 
$$v_x:\MO_{A, x} \to \Gamma_x \cup \{0\}.$$
For any open subset $U$ of $\Spa\,A$, we set 
$$\MO_A^{\Zar, +}(U):=\{f \in \MO^{\Zar}_A(U)\,|\,v_x(f) \leq 1\text{ for any }x \in U\}.$$
Then $\MO_A^{\Zar, +}$ is a presheaf of rings on $\Spa\,A$. 
For any $x \in \Spa\,A$, 
we set $\MO_{A, x}^{\Zar, +}(U)$ to be the stalk of $\MO_A^{\Zar, +}$ at $x$. 


\begin{prop}\label{p-zar-LRS}
The following hold. 
\begin{enumerate}
\item 
For any $x \in \Spa\,A$, 
the stalk $\MO_{A, x}^{\Zar}$ is a local ring 
whose maximal ideal is equal to $\Ker(v_x)$. 
\item 
For any $x \in \Spa\,A$, 
the stalk $\MO_{A, x}^{\Zar, +}$ is a local ring. 
It holds that 
$\MO_{A, x}^{\Zar, +}=\{f \in \MO_{A, x}^{\Zar}\,|\, v_x(f) \leq 1\}$ and 
its maximal ideal is equal to $\{f \in \MO_{A, x}^{\Zar}\,|\, v_x(f) < 1\}$. 
\item 
For any open subset $U$ of $\Spa\,A$ and $f, g \in \MO_A^{\Zar}(U)$, 
the set $\{x \in U\,|\,v_x(f) \leq v_x(g) \neq 0\}$ is 
an open subset of $\Spa\,A$. 
\item 
 For any rational subset $U$ of $\Spa\,A$, 
 it holds that $\MO_A^{\Zar}(U)=Z_A(U)^{\rhd}$ and 
$\MO_A^{\Zar, +}(U)=Z_A(U)^+$. 
\end{enumerate}
\end{prop}

\begin{proof}
We can apply the same argument as in \cite[Proposition 1.6]{Hub94} 
after replacing \cite[Lemma 1.5]{Hub94} by Lemma~\ref{l-rat-universal2}. 
\end{proof}


Let $A=(A^{\rhd}, A^+)$ be an affinoid ring. 
Let $M$ be an $A^{\rhd}$-module. 
For any open subset $V$ of $\Spa\,A$, we set 
$$\Gamma(V, M \otimes \MO_A^{\Zar}):=
\varprojlim_{\substack{U \subset V,\\ 
U:\text{ rational}\\ 
\text{subset}}}M \otimes_A Z_A(U)^{\rhd},$$
where the inverse limit is taken in the category of $A^{\rhd}$-modules. 
We have that $M \otimes \MO_A^{\Zar}$ is a presheaf of $\MO_A$-modules. 



\subsection{Structure sheaves}\label{ss-structure}

The purpose of this subsection is to show that the structure presheaf $\MO_A^{\Zar}$ is actually a sheaf (Theorem~\ref{t-sheafy}). 
To this end, we start with an auxiliary result (Lemma \ref{l-rat-covering}) 
that assures the existence of refined rational coverings. 


\begin{lem}\label{l-rat-covering}
Let $A=(A^{\rhd}, A^+)$ be a Zariskian affinoid ring and 
let $\mathcal U$ be an open cover of $\Spa\,A$. 
Then there exist elements $f_0, \cdots, f_n \in A^{\rhd}$ 
such that $\sum_{i=0}^n A^{\rhd} f_i=A^{\rhd}$ and 
the induced open cover $\left\{R\left(\frac{f_0, \cdots, f_n}{f_i}\right)\right\}_{0 \leq i \leq n}$ of $\Spa\,A$ is a refinement of $\mathcal U$. 
\end{lem}




\begin{proof}
Let $\gamma:A^{\rhd} \to \widehat{A^{\rhd}}$ be the completion. 
Thanks to \cite[Lemma 2.6]{Hub94}, we can find a refinement 
$\mathcal U$ by a rational covering 
$\left\{R\left(\frac{f_0, \cdots, f_n}{f_i}\right)\right\}_{0 \leq i \leq n}$ 
for some elements $f_0, \cdots, f_n, a_0, \cdots, a_n \in \widehat{A^{\rhd}}$ 
such that $\sum_{i=0}^n a_if_i=1$. 
Fix a ring of definition $A_0$ of $A^{\rhd}$ and an ideal of definition $I_0$ of $A_0$. 
By \cite[Lemma 3.10]{Hub93}, 
we can find elements $f'_0, \cdots, f'_n \in A^{\rhd}$ 
whose images by $\gamma$ are sufficiently close to $f_0, \cdots, f_n$, 
so that 
$$-1+\sum_{i=0}^na_i\gamma(f'_i) \in I_0\widehat{A_0}$$ 
and 
$$R\left(\frac{f_0, \cdots, f_n}{f_i}\right)=
R\left(\frac{f'_0, \cdots, f'_n}{f'_i}\right)$$
for any $i \in \{0, 1, \cdots, n\}$. 
Take elements $a'_i \in A^{\rhd}$ whose images by $\gamma$ 
are sufficiently close to $a_i$, so that 
$$-1+\sum_{i=0}^n\gamma(a'_if'_i) \in I_0\widehat{A_0}.$$ 
Therefore, we get 
$$-1+\sum_{i=0}^na'_if'_i \in \gamma^{-1}(I_0\widehat{A_0})=I_0A_0.$$
Since $A^{\rhd}$ is Zariskian, we get $\sum_{i=0}^nA^{\rhd} f'_i=A^{\rhd}$ as desired. 
\end{proof}


\begin{thm}\label{t-sheafy}
Let $A=(A^{\rhd}, A^+)$ be an affinoid ring and let $M$ be an $A^{\rhd}$-module. 
Then the presheaf $M \otimes \MO_A^{\Zar}$ on $\Spa\,A$ is a sheaf. 
\end{thm}


\begin{proof}
Replacing $A$ by its Zariskisation, 
we may assume that $A$ is Zariskian. 
Set $X:=\Spa\,A$. 
Take elements $f_0, \cdots, f_n \in A^{\rhd}$ such that 
$\sum_{k=0}^n A^{\rhd} f_k=A^{\rhd}$. 
Set $U_k:=R\left(\frac{f_0, \cdots, f_n}{f_k}\right)$ and 
$U_{k_1k_2} :=U_{k_1} \cap U_{k_2}$ for any $0 \leq k, k_1, k_2 \leq n$. 
Thanks to Lemma~\ref{l-rat-covering}, 
it suffices to show that the sequence 
{\small
\begin{equation}\label{e-sheafy}
0 \to (M \otimes \MO_A^{\Zar})(X) \xrightarrow{\varphi} \prod_{0\leq k \leq n} (M \otimes \MO_A^{\Zar})(U_k) 
\xrightarrow{\psi} \prod_{0\leq k_1<k_2 \leq n} (M \otimes \MO_A^{\Zar})(U_{k_1k_2})
\end{equation}
}
is exact. 
Fix a ring of definition $A_0$ of $A^{\rhd}$ and 
an ideal of definition $I_0$ of $A_0$.
We have that 
$$(M \otimes \MO_A^{\Zar})(X)=M,$$
$$(M \otimes \MO_A^{\Zar})(U_k)
=M \otimes_{A^{\rhd}} \left(1+I_0A_0\middle[\frac{f_0}{f_k}, \cdots, \frac{f_n}{f_k}\middle]\right)^{-1}A^{\rhd}\left[\frac{1}{f_k}\right],$$
$$(M \otimes \MO_A^{\Zar})(U_{k_1k_2})=M \otimes_{A^{\rhd}}
\left(1+I_0A_0\middle[\middle\{\frac{f_{\ell_1}f_{\ell_2}}{f_{k_1}f_{k_2}}\middle\}_{0 \leq \ell_1, \ell_2 \leq n}\middle]\right)^{-1}A^{\rhd}\left[\frac{1}{f_{k_1}f_{k_2}}\right].$$
Let $J_0$ be the $A_0$-submodule of $A^{\rhd}$ defined by 
$$J_0:=\sum_{k=0}^n A_0f_k \subset A^{\rhd}.$$
For $J:=J_0A^{\rhd}=A^{\rhd}$, we get a commutative diagram of schemes: 
$$\begin{CD}
\Proj(\bigoplus_{d=0}^{\infty} J^d) @>\pi >> \Spec\,A^{\rhd}\\
@VV\beta V @VV\alpha V\\
\Proj(\bigoplus_{d=0}^{\infty} J_0^d) @>\pi_0>> \Spec\,A_0,
\end{CD}$$
where $J^0:=A^{\rhd}$ and $J_0^0:=A_0$. 
By $J=A^{\rhd}$, we have that $\pi$ is an isomorphism. 
We get a morphism 
$$\sigma:=\beta \circ \pi^{-1}:\Spec\,A^{\rhd} \to 
\Proj\left(\bigoplus_{d=0}^{\infty} J_0^d\right).$$
It holds that $\beta$ is an affine morphism satisfying $\beta^{-1}(D_+(f_k))=D_+(f_k)$ for any $k \in \{0, \cdots, n\}$. 
Since the equation $\pi^{-1}(D(f_k))=D_+(f_k)$ holds, 
we obtain $\sigma^{-1}(D_+(f_k))=D(f_k)$. 
Thus, for any non-empty subset $K$ of $\{0, \cdots, n\}$, 
if we set 
$$f_K:=\prod_{k \in K} f_i,$$ 
then it holds that  
\begin{equation}\label{e-sheafy2}
\Gamma(D_+\left(f_K\right), \sigma_*\widetilde M)=
\Gamma(D\left(f_K\right), \widetilde M)=M \otimes_{A^{\rhd}} A^{\rhd}\left[\frac{1}{f_K}\right].
\end{equation}



\begin{claim}
For any non-empty subset $K$ of $\{0, \cdots, n\}$, 
the equation
\begin{equation}\label{e-sheafy3}
\Gamma\left(D_+(f_K), 
\MO_{\Proj(\bigoplus_{d=0}^{\infty} J_0^d)}\right)
=A_0\left[\middle\{\frac{f_{i_1}\cdots f_{i_{|K|}}}{f_K}\middle\}_{0 \leq i_j \leq n}\right] 
\end{equation}
holds, where the right hand side is defined as 
the $A_0$-subalgebra of $A^{\rhd}[1/f_K]$ generated by 
the set 
$\{f_{i_1}\cdots f_{i_{|K|}}f_K^{-1}\,|\, 0 \leq i_j \leq n\}$. 
\end{claim}

\begin{proof}(of Claim) 
Consider the following $A_0$-algebra homomorphisms of 
graded $A_0$-algebras preserving degrees 
\begin{eqnarray*}
A_0[t_0, \cdots, t_n] &\xrightarrow{\rho}& \bigoplus_{d=0}^{\infty} J_0^d 
\hookrightarrow A^{\rhd}[s]\\
t_k &\mapsto & f_ks
\end{eqnarray*}
where $A_0[t_0, \cdots, t_n]$ is the polynomial ring over $A_0$ and 
$f_ks$ is the element of degree one 
which is the same element as $f_k$. We set 
$$t_K:=\prod_{k \in K} t_k.$$
Taking localisations, we obtain $A_0$-algebra homomorphisms of 
graded $A_0$-algebras preserving degrees: 
\begin{eqnarray*}
A_0[t_0, \cdots, t_n, t_K^{-1}] &\xrightarrow{\rho_K}& 
\left(\bigoplus_{d=0}^{\infty} J_0^d\right)\left[\frac{1}{f_Ks^{|K|}}\right] 
\overset{q}\hookrightarrow \left(A^{\rhd}\left[\frac{1}{f_K}\right]\right)[s, s^{-1}].
\end{eqnarray*}
Since $\rho$ is surjective, so is $\rho_K$. 
It follows from \cite[Ch. II, Proposition 2.5(b)]{Har77} that 
the left hand side 
$\Gamma\left(D_+(f_K), \MO_{\Proj(\bigoplus_{d=0}^{\infty} J_0^d)}\right)$ 
of (\ref{e-sheafy3}) is nothing but the degree zero part of 
the graded ring $(\bigoplus_{d=0}^{\infty} J_0^d)[1/f_Ks^{|K|}]$, 
which is equal to 
$\rho_K\left(A_0[\{\frac{t_{i_1} \cdots t_{i_{|K|}}}{t_K}\}_{0 \leq i_j \leq n}]\right)$ 
because $\rho_K$ is surjective and preserving degrees. 
Embedding this ring via $q$, we get 
\begin{eqnarray*}
\Gamma\left(D_+(f_K), \MO_{\Proj(\bigoplus_{d=0}^{\infty} J_0^d)}\right)
&=&\rho_K\left(A_0\middle[\middle\{\frac{t_{i_1} \cdots t_{i_{|K|}}}{t_K}\middle\}_{0 \leq i_j \leq n}\middle]\right)\\
 &\xrightarrow{q,\, \simeq}& A_0\left[\middle\{\frac{f_{i_1} \cdots f_{i_{|K|}}}{f_K}\middle\}_{0 \leq i_j \leq n}\right]
\end{eqnarray*}
where the right hand side is the $A_0$-subalgebra of $A^{\rhd}[1/f_K]$ 
generated by $\{f_{i_1} \cdots f_{i_{|K|}}f_K^{-1}\,|\,0 \leq i_j \leq n\}$. 
This completes the proof of Claim. 
\end{proof}





Let 
$\iota: W=\pi^{-1}_0(V(I_0)) \hookrightarrow \Proj(\bigoplus_{d=0}^{\infty} J_0^d)$ be the inclusion map and we equip $W$ the induced topology. 
We set $\mathcal D(g):=D(g) \cap W$ for any homogeneous element $g$ 
of positive degree. 
For $\mathcal M:=\iota^{-1}\sigma_*\widetilde{M},$ 
we have that 
\begin{eqnarray*}
&&\Gamma(\mathcal D_+(f_k), \mathcal M)\\
&=&M \otimes_{A^{\rhd}} A^{\rhd}\left[\frac{1}{f_k}\right] 
\otimes_{A_0\left[\frac{f_0}{f_k},\cdots, \frac{f_n}{f_k}\right]}
\left(A_0\left[\frac{f_0}{f_k},\cdots, \frac{f_n}{f_k}\right]\right)^{Z}\\
&=&M \otimes_{A^{\rhd}} \left(1+I_0A_0\left[\frac{f_0}{f_k},\cdots, \frac{f_n}{f_k}\right]\right)^{-1} A^{\rhd}\left[\frac{1}{f_k}\right]\\
&=& (M \otimes \MO_A^{\Zar})(U_k).
\end{eqnarray*}
where we set 
{\small 
$$\left(A_0\left[\frac{f_0}{f_k},\cdots, \frac{f_n}{f_k}\right]\right)^{Z}
:=\left(1+I_0A_0\left[\frac{f_0}{f_k},\cdots, \frac{f_n}{f_k}\right]\right)^{-1}
\left(A_0\left[\frac{f_0}{f_k},\cdots, \frac{f_n}{f_k}\right]\right)$$
}
and the first equation follows from 
(\ref{e-sheafy2}), (\ref{e-sheafy3}) and 
the equation (\ref{e-FK-sheafy}) in Subsection~\ref{ss-FK-global}. 
By the same argument, we get 
$$\Gamma( \mathcal D_+(f_{k_1}f_{k_2}), \mathcal M
)= (M \otimes \MO_A^{\Zar})(U_{k_1k_2}).$$ 
Therefore, the sequence (\ref{e-sheafy}) coincides 
with the following sequence 
$$0 \to \Gamma(W, \mathcal M) 
\to \prod_{0 \leq k\leq n}  \Gamma(\mathcal D_+(f_k), \mathcal M) 
\to \prod_{0\leq k_1<k_2\leq n}  \Gamma(\mathcal D_+(f_{k_1}f_{k_2}), \mathcal M).$$
This is an exact suquence, since $\mathcal M=\iota^{-1}\sigma_*\widetilde M$ is a sheaf. 
Thus also (\ref{e-sheafy}) is exact. 
\end{proof}



\subsection{Zariskian adic spaces}\label{ss-zar-ad-sp}

Let $\mathcal V$ be the category of 
the triples $(X, \MO_X, (v_x)_{x \in X})$, 
where $X$ is a topological space, $\MO_X$ is a sheaf of topological rings, 
and each $v_x$ is a valuation of the stalk $\MO_{X, x}$. 
An arrow $f:X \to Y$ is a morphism in $\mathcal V$ 
if $f$ is a continuous map such that 
$f^{\sharp}:\MO_Y \to f_*\MO_X$ 
is a continuous ring homomorphism and that 
the valuation $v_{f(x)}$ is equivalent to the composition of 
$v_x$ and $f^{\sharp}_x:\MO_{Y, f(x)} \to \MO_{X, x}$.  

\begin{dfn}\label{d-zar-adic-sp}
For an affinoid ring $A$, 
the object 
$$(\Spa\,A, \MO_A^{\Zar}, (v_x)_{x \in \Spa\,A})$$ 
of $\mathcal V$ is called the {\em Zariskian adic space associated with} $A$. 
An {\em affinoid Zariskian adic space} is an object of $\mathcal V$ 
which is isomorphic to the Zariskian adic space associated with an affinoid ring. 

A {\em Zariskian adic space} is an object $(X, \MO_X, (v_x)_{x \in X})$ 
such that any point $x \in X$ has an open neighbourhood $U$ of $X$ 
to which the restriction $(U, \MO_X|_U, (v_x)_{x \in U})$ is 
an affinoid Zariskian adic space. 
A {\em morphism} of adic spaces is a morphism in $\mathcal V$. 
\end{dfn}




\subsection{Examples violating the Tate acylicity}\label{ss-cex-Tate}



\begin{nota}\label{n-cex}
Fix a prime number $p$ such that $p \neq 2$. 
In the following, we equip $\Q$ and $\Z_{(p)}$ 
with the $p$-adic topologies. 
We equip $\Q[t]$ the group topology induced by $\{p^n\Z_{(p)}[t]\}_{n \in \Z_{>0}}$. 
Then the pair
$$A:=(\Q[t], \Z_{(p)}[t])$$
is an affinoid ring.  
Let 
$$U_1:=R\left(\frac{1}{t+1}\right), \quad U_2:=R\left(\frac{1}{t-1}\right)$$
be rational subsets of $\Spa\,A$. 
We get 
$$U_1 \cap U_2=R\left(\frac{1}{t^2-1}\right)$$
We consider the following map 
\begin{eqnarray*}
\rho: \MO_A^{\Zar}(U_1) \times \MO_A^{\Zar}(U_2) &\to & 
\MO_A^{\Zar}(U_1 \cap U_2)\\
(f, g)&\mapsto &f|_{U_1 \cap U_2}-g|_{U_1 \cap U_2}.
\end{eqnarray*}
\end{nota}

\begin{lem}\label{l-cex-cover}
We use Notation~\ref{n-cex}. 
Then 
$$\Spa\,A=R\left(\frac{1}{t+1}\right) 
\cup R\left(\frac{1}{t-1}\right).$$ 
\end{lem}

\begin{proof}
Take $v \in \Spa(\Q[t], \Z_{(p)}[t]) \setminus R(\frac{1}{t+1})$. 
Then we have that $v(t+1) < 1=v(2)$, which implies 
$$v(t-1)=v(t+1-2)=1.$$
Thus we get $v \in R(\frac{1}{t-1})$, as desired. 
\end{proof}


\begin{prop}\label{p-non-surje}
We use Notation~\ref{n-cex}. 
Then $\rho$ is not surjective.
\end{prop}


\begin{proof}
The map $\rho$ can be written as 
$$\rho:\Q\left[t, \frac{1}{t+1}\right]^{\Zar} \times \Q\left[t, \frac{1}{t-1}\right]^{\Zar} 
\to \Q\left[t, \frac{1}{t^2-1}\right]^{\Zar}.$$
We embed all the rings appearing above into $\Q(t)$. 
We have that 
$$\frac{1}{1+\frac{p}{t^2-1}} \in 
\left(1+p\Z_{(p)}\left[t, \frac{1}{t^2-1}\right]\right)^{-1}
\Q\left[t, \frac{1}{t^2-1}\right]=\Q\left[t, \frac{1}{t^2-1}\right]^{\Zar}.$$
Assume that this element is in the image of $\rho$. 
Let us derive a contradiction. 
We can write 
$$ \frac{1}{1+\frac{p}{t^2-1}}=\frac{g_1(t, \frac{1}{t+1})}{1+pf_1(t, \frac{1}{t+1})}+
\frac{g_2(t, \frac{1}{t-1})}{1+pf_2(t, \frac{1}{t-1})}$$
for some $f_i(X, Y) \in \Z_{(p)}[X, Y]$ and $g_i(X, Y) \in \Q[X, Y]$. 
Taking the multiplication with the product of the denominators, 
we get 
{\small 
\begin{equation}\label{e-cex1}
\left(1+pf_1\middle(t, \frac{1}{t+1}\middle)\middle)\middle(1+pf_2\middle(t, \frac{1}{t-1}\middle)\right)=\left(1+\frac{p}{t^2-1}\right)g\left(t, \frac{1}{t^2-1}\right)
\end{equation}
}
for some $g(t, \frac{1}{t^2-1}) \in \Q[t, \frac{1}{t^2-1}]$. 
We have that $g(t, \frac{1}{t^2-1}) \in \Z_{(p)}[t, \frac{1}{t^2-1}]$. 
Indeed, otherwise we can find a positive integer $\nu$ such that 
$p^{\nu}g(t, \frac{1}{t^2-1}) \in \Z_{(p)}[t, \frac{1}{t^2-1}]$ and its modulo $p$ reduction 
is not zero, which contradicts the fact that $\F_p[t, \frac{1}{t^2-1}]$ 
is an integral domain, where $\F_p:=\Z/p\Z$. 




\begin{claim}
$(1+\frac{p}{t^2-1})\Z_{(p)}[t, \frac{1}{t^2-1}]$ is a prime ideal of $\Z_{(p)}[t, \frac{1}{t^2-1}]$. 
\end{claim}

\begin{proof}(of Claim) 
We have that $t^2-1+p$ 
is an irreducible polynomial over $\Q$, which in turn implies that 
$$(t^2-1+p)\Z_{(p)}[t]$$
is a prime ideal of $\Z_{(p)}[t]$. 
In particular, we get $t^2-1 \not\in (t^2-1+p)\Z_{(p)}[t]$, 
since if $t^2-1 \in (t^2-1+p)\Z_{(p)}[t]$, then the residue ring 
$$\Z_{(p)}[t]/(t^2-1+p)\Z_{(p)}[t] \simeq \Z_{(p)}[t]/(t^2-1, p)\Z_{(p)}[t] \simeq \F_p[t]/(t^2-1)$$
is not an integral domain. 
For $S:=\{(t^2-1)^r\}_{r \geq 0}$, we have that 
{\small 
$$S^{-1}(\Z_{(p)}[t]/(t^2-1+p)\Z_{(p)}[t]) \simeq 
\Z_{(p)}\left[t, \frac{1}{t^2-1}\middle]\middle/(t^2-1+p)\Z_{(p)}\middle[t, \frac{1}{t^2-1}\right]$$
}
is an integral domain, as the image of $t^2-1$ in $\Z_{(p)}[t]/(t^2-1+p)\Z_{(p)}[t]$ is nonzero. 
Therefore, Claim holds. 
\end{proof}

Let us go back to the proof  of Proposition \ref{p-non-surje}. 
By Claim and (\ref{e-cex1}), one of 
$$1+pf_1\left(t, \frac{1}{t+1}\right)\quad \text{and}\quad 
1+pf_2\left(t, \frac{1}{t-1}\right)$$
is contained in $(1+\frac{p}{t^2-1})\Z_{(p)}[t, \frac{1}{t^2-1}]$. 
By symmetry, we may assume that the former case occurs. 
Then we can write 
\begin{equation}\label{e-cex2}
1+pf_1\left(t, \frac{1}{t+1}\right)=\left(1+\frac{p}{t^2-1}\right)
h\left(t, \frac{1}{t^2-1}\right)
\end{equation}
for some $h(X, Y) \in \Z_{(p)}[X, Y]$. 
For the additive $(t-1)$-adic valuation $v_{t-1}:\Q(t) \to \Z$ 
satisfying $v_{t-1}(t-1)=1$, 
it follows from (\ref{e-cex2}) that 
$v_{t-1}(h(t, \frac{1}{t^2-1}))\geq 1$, i.e. 
we can write 
\begin{equation}\label{e-cex3}
h\left(t, \frac{1}{t^2-1}\right)=(t-1)h_1\left(t, \frac{1}{t+1}\right)+...+(t-1)^kh_k\left(t, \frac{1}{t+1}\right)
\end{equation}
for some $h_1,\cdots, h_k \in \Z_{(p)}[t, \frac{1}{t+1}]$. 
Combining (\ref{e-cex2}) and (\ref{e-cex3}), we obtain 
\begin{equation}\label{e-cex4}
1+pf_1\left(t, \frac{1}{t+1}\right)=\left(t-1+\frac{p}{t+1}\right)
\widetilde{h}\left(t, \frac{1}{t+1}\right),
\end{equation}
for 
$$\widetilde{h}\left(t, \frac{1}{t+1}\right):=
h_1\left(t, \frac{1}{t+1}\right)+...+(t-1)^{k-1}h_k\left(t, \frac{1}{t+1}\right) 
\in \Z_{(p)}\left[t, \frac{1}{t+1}\right].$$
Substituting $t=1$ for (\ref{e-cex4}), 
we get an equation of rational numbers: 
\begin{equation}\label{e-cex5}
1+pf_1\left(1, \frac{1}{2}\right)=\frac{p}{2} \times 
\widetilde{h}\left(1, \frac{1}{2}\right).
\end{equation}
Since all of $\frac{1}{2}, f_1(1, \frac{1}{2})$ and 
$\widetilde{h}(1, \frac{1}{2})$ are contained in $\Z_{(p)}$, 
the $p$-adic valuations of the both hand sides of (\ref{e-cex5})
are different, which is absurd. 
This completes the proof of Proposition \ref{p-non-surje}.
\end{proof}
\begin{thm}\label{t-non-TA}
We use Notation~\ref{n-cex}. 
Then it holds that 
$$H^1(\Spa\,A, \MO_A^{\Zar}) \neq 0.$$ 
\end{thm}

\begin{proof}
Set $X:=\Spa\,A$. 
Let $j_1:U_1 \to X$, $j_2:U_2 \to X$ and $j_3:U_1 \cap U_2 \to X$ 
be the open immersions. 
Lemma~\ref{l-cex-cover} induces the following Mayer--Vietoris exact sequence: 
$$0 \to \MO^{\Zar}_A \to (j_1)_*(\MO^{\Zar}_A|_{U_1})
\times (j_2)_*(\MO^{\Zar}_A|_{U_2}) 
\to (j_3)_*(\MO^{\Zar}_A|_{U_1 \cap U_2}) \to 0.$$
It follows from Proposition~\ref{p-non-surje} that  
$H^1(X, \MO_A^{\Zar}) \neq 0$, as desired. 
\end{proof}




