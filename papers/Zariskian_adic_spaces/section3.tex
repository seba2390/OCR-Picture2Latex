



\section{Zariskian f-adic rings}

In Subsection~\ref{ss-zar-def}, 
we introduce Zariskian f-adic rings and Zariskisation of f-adic rings. 
In Subsection \ref{ss-Zariskisation}, 
we show that the Zariskisation of any f-adic ring is Zariskian. 
In Subsection~\ref{ss-zar-haus} (resp. \ref{ss-zar-comp}), 
we discuss the relation between Zariskisations and 
Hausdorff quotients (resp. completions). 
In Subsection~\ref{ss-zar-aff}, 
we introduce Zariskian affinoid rings. 

For some corresponding results for the case of adic rings, 
we refer to \cite[Ch. 0, Section 7.3(b) and Ch. I, Section B.1]{FK}. 

\subsection{Definition}\label{ss-zar-def}

In this subsection, we introduce Zariskian f-adic rings and 
Zariskisation of f-adic rings. 


\subsubsection{Definition as rings}\label{sss-zar-def1}

Let us first define Zariskisations of f-adic rings 
without topological structures. 
In (\ref{sss-zar-def2}), we shall introduce topologies on them. 

\begin{dfn}\label{d-zar}
Let $A$ be an f-adic ring. 
We set $S_A^{\Zar}:=1+A^{\circ\circ}$. 
Since $A^{\circ\circ}$ is an ideal of $A^{\circ}$, 
it holds that $S_A^{\Zar}$ is a multiplicative subset of $A$. 
We set $A^{\Zar}:=(S_A^{\Zar})^{-1}A$, 
which is called the {\em Zariskisation} of $A$. 
Also the natural ring homomorphism $A \to A^{\Zar}$ is 
called the {\em Zariskisation} of $A$. 
We say that $A$ is {\em Zariskian} if $S_A^{\Zar} \subset A^{\times}$. 
\end{dfn}



\begin{rem}\label{r-zar1}
Given an f-adic ring $A$, 
it is obvious that $A$ is Zariskian if and only if the natural ring homomorphism 
$A \to A^{\Zar}$ is bijective. 
\end{rem}

\begin{rem}\label{r-zar2}
If $\varphi:A \to B$ be a continuous ring homomorphism, 
then we have that $A^{\circ\circ} \subset B^{\circ\circ}$ 
and $\varphi(S_A^{\Zar}) \subset S_B^{\Zar}$. 
In particular, we get a commutative diagram of ring homomorphisms:
$$\begin{CD}
A @>\varphi >> B\\
@VVV @VVV\\
A^{\Zar} @>\varphi^{\Zar}>> B^{\Zar}, 
\end{CD}$$ 
where the vertical arrows are the natural ring homomorphisms. 
\end{rem}


\begin{lem}\label{l-zar-sub}
Let $A$ be an f-adic ring and let $A_0$ be an open subring of $A$. 
Then $A$ is Zariskian if and only if $A_0$ is Zariskian. 
\end{lem}

\begin{proof}
Assume that $A_0$ is Zariskian. 
Take $y \in A^{\circ\circ}$. 
We can find a positive integer $n$ such that $y^n \in A_0^{\circ\circ}$. 
Since $A_0$ is Zariskian, we have that 
$$1-y^n \in 1+A_0^{\circ\circ} \subset A_0^{\times} \subset A^{\times}.$$
This implies $1-y \in A^{\times}$. 
Thus $A$ is Zariskian. 

Conversely, assume that $A$ is Zariskian. 
Take $y \in A_0^{\circ\circ}$. 
We have that 
$$1+y \in 1+A_0^{\circ\circ} \subset 1+A^{\circ\circ} \subset A^{\times},$$
where the last inclusion holds because $A$ is Zariskian. 
Thus, there exists $w \in A$ such that $(1+y)w=1$. 
It suffices to show that $w \in A_0$. 
Let us prove $wy^m \in A_0$ for any $m \in \Z_{\geq 0}$ 
by descending induction on $m$. 
Since $y \in A_0^{\circ\circ}$, we get $wy^m \in A_0$ for $m \gg 0$. 
Thanks to the equation 
$$y^mw+y^{m+1}w=y^m \in A_0,$$
if $y^{m+1}w \in A_0$, then $y^mw \in A_0$. 
Therefore, it follows that $w \in A_0$. 
Thus $A_0$ is Zariskian. 
\end{proof}



\subsubsection{Definition as f-adic rings}\label{sss-zar-def2}

For any f-adic ring $A$, 
we introduce a topology on the ring $A^{\Zar}$ (Definition~\ref{d-zar-top}), 
so that also $A^{\Zar}$ is an f-adic ring. 
To this end, we need two auxiliary results: Lemma~\ref{l-two-S-inv} 
and Lemma~\ref{l-top-indep}. 
Furthermore, we show that $A^{\Zar}$ is 
an initial object of the category of Zariskian f-adic $A$-algebras 
(Theorem~\ref{t-zar-univ}). 



\begin{lem}\label{l-two-S-inv}
Let $A$ be an f-adic ring. 
Let $P$ be an open pseudo-subring of $A^{\circ\circ}$. 
Then the natural ring homomorphism 
$\theta:(1+P)^{-1}A \to A^{\Zar}$ is bijective. 
\end{lem}

\begin{proof}
Take $z \in S_A^{\Zar}$. 
Since $S_A^{\Zar}=1+A^{\circ\circ}$, 
we can write $z=1-y$ for some $y \in A^{\circ\circ}$. 
It follows from $y \in A^{\circ\circ}$ that 
we get $y^k \in P$ for some positive integer $k$. 
Therefore, for any $a \in A$, we get 
$$\theta\left(\frac{a(1+y+\cdots+y^{k-1})}{1-y^k}\right)=\frac{a}{1-y}=\frac{a}{z}.$$
Thus $\theta$ is surjective. 

Take $a \in A$ and $w \in 1+P$ such that $\theta(a/w)=0$. 
This implies that $za=0$ for some $z \in S_A^{\Zar}$. 
For $y \in A^{\circ\circ}$ and $k \in \Z_{>0}$ such that 
$z=1-y$ and $y^k \in P$, 
we have that 
$(1-y^k)a=0$ and $1-y^k \in 1+P$. 
Therefore the equation $a/w=0$ holds, hence $\theta$ is injective. 
\end{proof}

\begin{rem}\label{r-FK-def}
If $A$ is an f-adic ring 
whose topology coincides with the $I$-adic topology 
for some ideal $I$ of $A$, 
then Lemma~\ref{l-two-S-inv} implies 
the following assertions. 
\begin{enumerate}
\item 
The natural ring homomorphism 
$(1+I)^{-1}A \to A^{\Zar}$ is bijective. 
\item 
$A$ is Zariskian if and only if $1+I \subset A^{\times}$. 
\end{enumerate}
\end{rem}


\begin{rem}\label{r-two-S-inv}
For an f-adic ring $A$ and an open subring $A'$ of $A$, 
we can consider $(A')^{\Zar}$ as a subring of $A^{\Zar}$ 
by Lemma~\ref{l-two-S-inv}. 
\end{rem}




\begin{lem}\label{l-top-indep}
Let $A$ be an f-adic ring. 
For any $i \in \{1, 2\}$, 
let $A_i$ be a ring of definition of $A$ and 
let $I_i$ be an ideal of definition of $A_i$. 
Then, for any positive integer $n_1$, 
there exists a positive integer $n_2$ such that 
the inclusion 
$$I_2^{n_2}A_2^{\Zar} \subset I_1^{n_1}A_1^{\Zar}$$
holds as subsets of $A^{\Zar}$ (cf. Remark~\ref{r-two-S-inv}). 
\end{lem}

\begin{proof}
Replacing $(A_2, I_2)$ by $(A_1 \cdot A_2, I_2A_1 \cdot A_2)$ \cite[Corollary 1.3(i)]{Hub93}, 
we may assume that $A_1 \subset A_2$. 
Fix a positive integer $n_1$. 
There is a positive integer $n_2$ such that $I_2^{n_2}A_2 \subset I_1^{n_1}A_1$. 
Then we get 
$$I_2^{n_2}A_2^{\Zar}=I_2^{n_2}(S^{\Zar}_{A_2})^{-1}A_2=I_2^{n_2}(S^{\Zar}_{A_1})^{-1}A_2 \subset I_1^{n_1}(S^{\Zar}_{A_1})^{-1}A_1=I_1^{n_1}A_1^{\Zar},$$
where the second equation holds, 
since the inclusion $A_1 \subset A_2$ enables us to apply Lemma \ref{l-two-S-inv}. \end{proof}


\begin{dfn}\label{d-zar-top}
Let $A$ be an f-adic ring and let $A^{\Zar}$ be its Zariskisation. 
For a ring of definition $A_0$ of $A$ and 
an ideal of definition $I_0$ of $A_0$, 
we equip $A^{\Zar}$ with the group topology 
defined by $\{I_0^kA_0^{\Zar}\}_{k\in \Z_{>0}}$. 
Thanks to Lemma~\ref{l-top-indep}, 
this topology does not depend on the choice of $A_0$ and $I_0$. 
It is easy to check that $A^{\Zar}$ is an f-adic ring (cf. Lemma \ref{l-top-criterion}). 
\end{dfn}



\begin{rem}\label{r-zar-adic}
For any f-adic ring $A$, the natural ring homomorphism $A \to A^{\Zar}$ is adic. 
\end{rem}

\begin{lem}\label{l-zar-cont}
Let $\varphi:A \to B$ be a continuous ring homomorphism of f-adic rings and 
let $\varphi^{\Zar}:A^{\Zar} \to B^{\Zar}$ be the induced ring homomorphism 
(cf. Remark~\ref{r-zar2}). 
Then $\varphi^{\Zar}$ is continuous. 
Moreover, if $\varphi$ is adic, so is $\varphi^{\Zar}$. 
\end{lem}

\begin{proof}
Take a ring of definition $A_0$ (resp. $B_0$) of $A$ (resp. $B$) 
and an ideal of definition $I_0$ (resp. $J_0$) of $A_0$ (resp. $B_0$). 
Fix a positive integer $m$. 
Since $\varphi$ is continuous, we can find a positive integer $n$ such that 
$$I_0^n \subset \varphi^{-1}(J_0^m).$$
It follows from Lemma \ref{l-two-S-inv} that 
$A_0^{\Zar}=(1+I_0^n)^{-1}A_0$ and $B_0^{\Zar}=(1+J_0^m)^{-1}B_0$. 
In particular, any element $\zeta \in I_0^nA_0^{\Zar}$ can be written 
by $\zeta=(1+x)^{-1}x'$ for some $x, x' \in I_0^n$. 
Then we get 
\[
\varphi^{\Zar}(\zeta)=\varphi^{\Zar}\left(\frac{x'}{1+x} \right)=\frac{\varphi(x')}{1+\varphi(x)}
\in J_0^m\cdot (1+J_0^m)^{-1}B_0= J_0^mB_0^{\Zar}.
\]
Hence, it holds that $I_0^nA^{\Zar}_0 \subset (\varphi^{\Zar})^{-1}(J_0^mB^{\Zar}_0).$ 
Therefore, $\varphi^{\Zar}$ is continuous. 

If $\varphi$ is adic, then \cite[Corollary 1.9(ii)]{Hub93} 
implies that $\varphi^{\Zar}$ is adic. 
\end{proof}


\begin{thm}\label{t-zar-univ}
Let $A$ be an f-adic ring and 
let $\theta:A \to A^{\Zar}$ be the Zariskisation. 
Then, for any continuous ring homomorphism $\varphi:A \to B$ 
to a Zariskian f-adic ring $B$, 
there exists a unique continuous ring homomorphism 
$\psi:A^{\Zar} \to B$ such that $\varphi=\psi \circ \theta$. 
\end{thm}

\begin{proof}
The assertion follows from Remark~\ref{r-zar1} and Lemma~\ref{l-zar-cont}. 
\end{proof}



\subsection{Zariskisations are Zariskian}\label{ss-Zariskisation}

In this subsection, 
we show that the Zariskisation of any f-adic ring is Zariskian 
(Theorem~\ref{t-zar-zar}). 




\begin{lem}\label{l-frac-indep}
Let $A$ be an f-adic ring. 
Take elements $a_1, a_2 \in A$ and $s_1, s_2 \in S^{\Zar}_A$. 
If $a_1 \in A^{\circ\circ}$ and 
the equation $a_1/s_1=a_2/s_2$ holds in $A^{\Zar}$, 
then $a_2 \in A^{\circ\circ}$. 
\end{lem}

\begin{proof}
We have the equation $s_3s_2a_1=s_3s_1a_2$ in $A$ for some $s_3 \in S^{\Zar}_A$. 
As we can write $s_i=1+y_i$ for some $y_i \in A^{\circ\circ}$, 
we get 
$$(1+y_3)(1+y_2)a_1=(1+y_3)(1+y_1)a_2.$$
Thanks to $a_1 \in A^{\circ\circ}$ and the fact that $A^{\circ\circ}$ is an ideal of $A^{\circ}$, 
we can find $y \in A^{\circ\circ}$ such that 
$$(1+y)a_2 \in A^{\circ\circ}.$$




We show that $y^na_2 \in A^{\circ\circ}$ for any non-negative integer $n$ 
by descending induction on $n$. 
If $n \gg 0$, then it follows from $y \in A^{\circ\circ}$ that 
$y^na_2 \in A^{\circ\circ}$. 
Assume that $y^{k+1}a_2 \in A^{\circ\circ}$ for a non-negative integer $k$.  
Since 
$$y^ka_2+y^{k+1}a_2=y^k(1+y)a_2 \in A^{\circ\circ},$$
we have that $y^ka_2 \in A^{\circ\circ}$. 
Therefore, we get $a_2 \in A^{\circ\circ}$, as desired. 
\end{proof}

\begin{prop}\label{p-zariski-calcu}
Let $A$ be an f-adic ring. 
Then the equation 
$$(A^{\Zar})^{\circ\circ} = (S^{\Zar}_A)^{-1}A^{\circ\circ}$$
holds, where $(S^{\Zar}_A)^{-1}A^{\circ\circ}:=\{s^{-1}y \in A^{\Zar}\,|\,y \in A^{\circ\circ}, s \in S^{\Zar}_A\}$. 
\end{prop}

\begin{proof}
First, we show $(A^{\Zar})^{\circ\circ} \supset (S^{\Zar}_A)^{-1}A^{\circ\circ}$. 
Take $z \in (S^{\Zar}_A)^{-1}A^{\circ\circ}$ 
and we prove $z \in (A^{\Zar})^{\circ\circ}$. 
We can write 
$$z=\frac{y_1}{1+y_2} \in A^{\Zar}$$ 
for some $y_1, y_2 \in A^{\circ\circ}$. 
Let $A_0$ be a ring of definition of $A$ satisfying $y_1, y_2 \in A_0$. 
We see that $y_1, y_2 \in A_0^{\circ\circ}$. 
Take an ideal of definition $I_0$ of $A_0$. 
After replacing $z$ by a sufficiently large power $z^N$, 
we may assume that $y_1 \in I_0$. 
In particular, we have that 
$$z \in I_0A_0^{\Zar} \subset (A^{\Zar})^{\circ\circ},$$
where the inclusion follows from the definition of the topology on $A^{\Zar}$ (cf. Definition \ref{d-zar-top}). 
Thus the inclusion $(A^{\Zar})^{\circ\circ} \supset (S^{\Zar}_A)^{-1}A^{\circ\circ}$ 
holds. 

Second, we show $(A^{\Zar})^{\circ\circ} \subset (S^{\Zar}_A)^{-1}A^{\circ\circ}$. 
Take $z \in (A^{\Zar})^{\circ\circ}$. 
We can write  $z=a/s$ for some $a \in A$ and $s \in S^{\Zar}_A$. 
Let $A_0$ be a ring of definition of $A$ and 
let $I_0$ be an ideal of definition of $A_0$. 
Since $z^n \in I_0A_0^{\Zar}$ for some $n \in \Z_{>0}$, 
we can write 
$$z^n=\sum_{\ell=1}^m \frac{y_{1\ell}}{1+y_{2\ell}}$$ 
for some $y_{1\ell} \in I_0, y_{2\ell} \in A_0^{\circ\circ}$. 
In particular, we get $y_{1\ell}, y_{2\ell} \in A^{\circ\circ}$. 
Hence, we obtain $a^n/s^n=z^n \in (S^{\Zar}_A)^{-1}A^{\circ\circ}$. 
It follows from Lemma~\ref{l-frac-indep} that $a^n \in A^{\circ\circ}$. 
Since $A^{\circ\circ}=\sqrt{A^{\circ\circ}}$, we get $a \in A^{\circ\circ}$. 
Therefore, we have that $z=a/s \in (S^{\Zar}_A)^{-1}A^{\circ\circ}$, as desired. 
\end{proof}



\begin{thm}\label{t-zar-zar}
Let $A$ be an f-adic ring. 
Then the Zariskisation $A^{\Zar}$ of $A$ is Zariskian. 
\end{thm}

\begin{proof}
Take $z \in 1+(A^{\Zar})^{\circ\circ}$. 
By Proposition~\ref{p-zariski-calcu}, we can write 
$$z=1+\frac{y_1}{1+y_2}=\frac{1+y_1+y_2}{1+y_2}$$ 
for some $y_1, y_2 \in A^{\circ\circ}$. 
Since $1+y_1+y_2 \in S_A^{\Zar}$, 
we have that $z \in (A^{\Zar})^{\times}$. 
Hence, $A^{\Zar}$ is Zariskian. 
\end{proof}




\subsection{Relation to Hausdorff quotient}\label{ss-zar-haus}

In this subsection, we discuss the relation between 
Hausdorff quotients and Zariskisations. 
It is easy to find a Hausdorff f-adic ring that is not Zariskian 
(e.g. the integer ring $\Z$ with the $p$-adic topology). 
On the other hand, if a Zariskian f-adic ring $A$ 
is noetherian and its topology coincides with the $I$-adic topology for some ideal $I$, 
then $A$ is Hausdorff (Example~\ref{e-sep-zar}(1)). 
Unfortunately the same statement is false in general 
(Example~\ref{e-sep-zar}(3)). 
On the other hand, these two operations commute each other (Proposition~\ref{p-hd-zar}). 



\begin{lem}\label{l-zar-sep}
Let $A$ be an f-adic ring and let $\alpha:A \to A^{\Zar}$ 
be the natural ring homomorphism. 
Then the inclusion $\Ker(\alpha) \subset \overline{\{0\}}$ holds.
\end{lem}

\begin{proof}
Take $x \in \Ker(\alpha)$. 
Then $(1-y)x=0$ for some $y \in A^{\circ\circ}$, which implies 
$x=y^kx$ for any positive integer $k$. 
Hence, we get $x \in \overline{\{0\}}$. 
\end{proof}


\begin{ex}\label{e-sep-zar}
Let $A$ be a ring and let $I$ be an ideal of $A$. 
We equip $A$ with the $I$-adic topology. 
\begin{enumerate}
\item 
If $A$ is a noetherian ring, then it follows from \cite[Theorem 10.17]{AM69} 
that 
$$\bigcap_{n=1}^{\infty} I^n=\{ x \in A\,|\,(1+y)x=0\text{ for some }y\in I\}.$$
This equation implies that the kernels of 
$A \to A^{\hd}$ and $A \to (1+I)^{-1}A$ coincide. 
In particular, if $A$ is noetherian and Zariskian, 
then $A$ is Hausdorff. 
\item 
In general, the inclusion 
$$\bigcap_{n=1}^{\infty} I^n \supset \{ x \in A\,|\,(1+y)x=0\text{ for some }y\in I\}$$
holds by Lemma~\ref{l-zar-sep}. 
\item 
By \cite[Remark (2) after Theorem 10.17]{AM69}, 
there exist a ring $B$ and an ideal $J$ of $B$ such that 
$$\bigcap_{n=1}^{\infty} J^n \supsetneq \{ x \in B\,|\,(1+y)x=0\text{ for some }y\in J\}.$$
We equip $B$ with the $J$-adic topology. 
Then the Zariskisation $B^{\Zar}=(1+J)^{-1}B$ of $B$ 
is not Hausdorff. 
We shall further study this example in Subsection~\ref{ss-non-surje}. 
\end{enumerate}
\end{ex}







\begin{lem}\label{l-zar-quot}
Let $A$ be an f-adic ring and let $J$ be an ideal of $A$. 
Let $A/J$ be the quotient f-adic ring of $A$ by $J$ (cf. Subsection \ref{ss-quot-fad}). 
If $A$ is Zasikian, then also $A/J$ is Zariskian. 
\end{lem}


\begin{proof}
Let $\pi:A \to A/J$ be the induced ring homomorphism. 
Fix a ring of definition $A_0$ of $A$ 
and an ideal of definition $I_0$ of $A_0$. 
We have that 
$$1+\pi(I_0A_0)=\pi(1+I_0A_0) \subset \pi(1+A^{\circ\circ}) 
\subset \pi(A^{\times}) \subset (A/J)^{\times}.$$  
Therefore, $A/J$ is Zariskian by Lemma~\ref{l-two-S-inv}. 
\end{proof}




\begin{lem}\label{l-hd-zar}
Let $A$ be a Hausdorff f-adic ring. 
Then the following hold. 
\begin{enumerate}
\item The natural ring homomorphism $A \to A^{\Zar}$ is injective. 
\item $A^{\Zar}$ is Hausdorff. 
\end{enumerate}
\end{lem}

\begin{proof}
The assertion (1) holds by Lemma~\ref{l-zar-sep}. 
Let us show (2). 
Take $\zeta \in \bigcap_{n \in \Z_{>0}} I_0^n A_0^{\Zar}$. 
Since $\zeta \in A_0^{\Zar}$, 
we can write $\zeta=a/s$ for some $a \in A_0$ and $s \in S_{A_0}^{\Zar}$. 
It suffices to show that $a \in \bigcap_{n \in \Z_{>0}} I_0^n$. 
Fix $n \in \Z_{>0}$. 
Since $\zeta \in I_0^nA_0^{\Zar}$, we get 
$a/s=x_n/s_n$ for some $x_n \in I_0^n$ and $s_n \in 1+I_0$. 
In particular, we get an equation 
$$ts_na=tsx_n \in I_0^n$$
for some $t \in 1+I_0$. 
We can write $ts_n=1-y$ for some $y \in I_0$, hence 
$$(1-y^{n})a=(1+y+\cdots+y^{n-1})tsx_n \in I_0^n.$$
As $y^n a \in I_0^n$, we get $a \in I_0^n$. 
Thus (2) holds. 
\end{proof}



\begin{prop}\label{p-hd-zar}
Let $A$ be an f-adic ring. 
Then both the natural continuous ring homomorphisms 
$$\theta_1:A^{\hd} \otimes_A A^{\Zar} \to (A^{\hd})^{\Zar} 
\quad\text{and}\quad
\theta_2:A^{\hd} \otimes_A A^{\Zar} \to (A^{\Zar})^{\hd}$$
are isomorphisms of topological rings, 
where $A^{\hd} \otimes_A A^{\Zar}$ denotes the topological tensor product (cf. Theorem~\ref{t-top-tensor}). 
\end{prop}

\begin{proof}
Fix a ring of definition $A_0$ of $A$ and 
an ideal of definition $I_0$ of $A_0$. 

\setcounter{step}{0}

\begin{step}\label{s-theta1}
The map 
$\theta_1:A^{\hd} \otimes_A A^{\Zar} \to (A^{\hd})^{\Zar}$ is an isomorphism 
of topological rings. 
\end{step}

\begin{proof}(of Step~\ref{s-theta1}) 
Let $N:=\Ker(A \to A^{\hd})$. 
Since the image of $A^{\circ\circ}$ by $A \to A^{\hd}$ is equal to $(A^{\hd})^{\circ\circ}$, we get 
$$A^{\hd} \otimes_A A^{\Zar} =(A/N) \otimes_A (S_A^{\Zar})^{-1}A
= (S^{\Zar}_{A^{\hd}})^{-1}(A/N)=(A^{\hd})^{\Zar},$$
hence the ring homomorphism $\theta_1$ is bijective. 
For any $k \in \Z_{\geq 0}$, the images of 
$I_0^k(A_0^{\hd} \otimes_{A_0} A_0^{\Zar})$ and $I_0^k(A_0^{\hd})^{\Zar}$ 
in $(A^{\hd})^{\Zar}$ coincide via $\theta_1$. 
Therefore, $\theta_1$ is an open map. 
This completes the proof of Step~\ref{s-theta1}. 
\end{proof}








\begin{step}\label{s-zh-z}
$(A^{\Zar})^{\hd}$ is Zariskian. 
\end{step}

\begin{proof}(of Step~\ref{s-zh-z}) 
The assertion follows from Lemma~\ref{l-zar-quot}. 
\end{proof}



\begin{step}\label{s-initial}
The map $\theta_2:A^{\hd} \otimes_A A^{\Zar} \to (A^{\Zar})^{\hd}$ is an isomorphism of topological rings. 
\end{step}

\begin{proof}(of Step~\ref{s-initial}) 
Let $\mathcal R$ be the category of 
Hausdorff Zariskian f-adic $A$-algebras 
whose arrows are continuous $A$-algebra homomorphisms. 
It follows from Lemma \ref{l-hd-zar}(2) that $(A^{\hd})^{\Zar}$ is Hausdorff. 
In particular, Step~\ref{s-theta1} implies that 
$A^{\hd} \otimes_A A^{\Zar}$ is a Hausdorff Zariskian f-adic $A$-algebra. 
Then we see that $A^{\hd} \otimes_A A^{\Zar}$ is 
an initial object of $\mathcal R$ 
by Remark~\ref{r-hd-univ}, Theorem~\ref{t-top-tensor} 
and Theorem~\ref{t-zar-univ}. 
It suffices to show that $(A^{\Zar})^{\hd}$ is an initial object of $\mathcal R$. 
It follows from Step~\ref{s-zh-z} that 
$(A^{\Zar})^{\hd}$ is an object of $\mathcal R$. 
By Remark~\ref{r-hd-univ} and Theorem~\ref{t-zar-univ}, 
$(A^{\Zar})^{\hd}$ is an initial object of $\mathcal R$, as desired. 
\end{proof}
Step~\ref{s-theta1} and Step~\ref{s-initial} complete 
the proof of Proposition~\ref{p-hd-zar}. 
\end{proof}




\subsection{Relation to completion}\label{ss-zar-comp}



The main result of this subsection 
is Theorem~\ref{t-comp-factor} 
which asserts that any f-adic ring and its Zariskisation 
has the same completion. 
We start with the following basic result. 


\begin{lem}\label{l-comp-zar}
Let $A$ be a complete f-adic ring. 
Then $A$ is Zariskian. 
\end{lem}

\begin{proof}
Take $y \in A^{\circ\circ}$. 
Since $1+y+y^2+\cdots \in \widehat{A}$, we get $1-y \in A^{\times}$.  
\end{proof}



\begin{thm}\label{t-comp-factor}
Let $A$ be an f-adic ring. 
Let $\alpha:A \to A^{\Zar}$ be its Zariskisation and 
let $\gamma:A \to \widehat{A}$ be its completion. 
Then the following hold. 
\begin{enumerate}
\item 
There exists a unique continuous 
ring homomorphism $\beta:A^{\Zar} \to \widehat{A}$ such that $\gamma=\beta \circ \alpha$. 
\item 
For any non-negative integer $k$, 
ring of definition $A_0$ of $A$ and ideal of definition $I_0$ of $A_0$, 
the equation 
$$I_0^kA_0^{\Zar}=\beta^{-1}(I_0^k \widehat{A_0})$$
holds.   
\item 
The induced map $\widehat{\alpha}:\widehat{A} \to \widehat{A^{\Zar}}$ 
is an isomorphism of topological rings. 
\end{enumerate}
\end{thm}

\begin{proof}
The assertion (1) holds by Theorem~\ref{t-zar-univ} 
and Lemma~\ref{l-comp-zar}. 

We now show (2). 
Taking the Hausdorff quotients of $A \to A^{\Zar}$, 
we may assume that $A$ is Hausdorff and 
both $A$ and $A^{\Zar}$ are subrings of $\widehat{A}$ (Proposition~\ref{p-hd-zar}). 
It suffices to show the equation: 
$I_0^kA_0^{\Zar}=I_0^k \widehat{A_0} \cap A^{\Zar}.$ 
Since the inclusion $I_0^kA_0^{\Zar} \subset I_0^k \widehat{A_0} \cap A^{\Zar}$ is clear, 
it is enough to prove the opposite one: 
$I_0^kA_0^{\Zar} \supset I_0^k \widehat{A_0} \cap A^{\Zar}$. 

We have the following: 
$$I_0^k = I_0^k \widehat{A_0} \cap A \subset A \subset \widehat{A}.$$ 
Applying the functor $(-) \otimes_{A_0} A_0^{\Zar}$, we get  
{\small 
\begin{equation}\label{t-comp-factor1}
I_0^k \otimes_{A_0} A_0^{\Zar} = 
(I_0^k\widehat{A_0} \cap A) \otimes_{A_0} A_0^{\Zar} \subset A \otimes_{A_0} A_0^{\Zar} \subset \widehat{A} \otimes_{A_0} A_0^{\Zar}.
\end{equation}
}
Consider the induced ring isomorphism: 
$$\theta:\widehat{A} \otimes_{A_0} A_0^{\Zar} \xrightarrow{\simeq} \widehat{A}, \quad x \otimes y \mapsto xy.$$

\begin{claim}
The inclusion 
$$I_0^k\widehat{A_0} \cap A^{\Zar} \subset 
\theta((I_0^k\widehat{A_0} \cap A) \otimes_{A_0} A_0^{\Zar}).$$
holds. 
\end{claim}

\begin{proof}(of Claim) 
Take $\zeta \in I_0^k\widehat{A_0} \cap A^{\Zar}$. 
By Lemma~\ref{l-two-S-inv}, 
we can find $s \in 1+I_0$ such that $s\zeta \in I_0^k\widehat{A_0} \cap A$. 
In particular, we get 
$$s\zeta \otimes s^{-1} \in (I_0^k\widehat{A_0} \cap A) \otimes_{A_0} A_0^{\Zar},$$
which implies 
$$\zeta= \theta(s\zeta \otimes s^{-1}) \in \theta((I_0^k\widehat{A_0} \cap A) \otimes_{A_0} A_0^{\Zar}).$$
This completes the proof of Claim. 
\end{proof}

We obtain 
$$I_0^k\widehat{A_0} \cap A^{\Zar} \subset 
\theta((I_0^k\widehat{A_0} \cap A) \otimes_{A_0} A_0^{\Zar})
=\theta(I_0^k \otimes_{A_0} A_0^{\Zar})=I^k_0A_0^{\Zar},$$
where the inclusion holds by Claim and 
the first equation follows from (\ref{t-comp-factor1}). 
Thus (2) holds. 






Thanks to (1) and (2), we can apply Lemma~\ref{l-common-comp}, 
hence the assertion (3) holds. 
This completes the proof of Theorem~\ref{t-comp-factor}. 
\end{proof}

\begin{cor}\label{c-zar-bdd-open}
Let $A$ be an f-adic ring. 
Let $\alpha:A \to A^{\Zar}$ be the Zariskisation of $A$. 
Then the following hold. 
\begin{enumerate}
\item 
The induced map $\alpha^*:\mathfrak O_{A^{\Zar}} \to \mathfrak O_{A}$ is bijective, and the inverse map $(\alpha^*)^{-1}=:\alpha_*$ satisfies $\alpha_*(A_0)=A_0^{\Zar}$ for any $A_0 \in \mathfrak O_{A}$. 
\item 
It holds that 
$\alpha_*(\mathfrak B_A)=\mathfrak B_{A^{\Zar}}$ 
and 
$\alpha^*(\mathfrak B_{A^{\Zar}})=\mathfrak B_{A}$. 
\item 
It holds that 
$\alpha_*(\mathfrak I_{A})=\mathfrak I_{A^{\Zar}}$ 
and 
$\alpha^*(\mathfrak I_{A^{\Zar}})=\mathfrak I_{A}$. 
\end{enumerate}
\end{cor}

\begin{proof}
Take the Zariskisation and the completion of $A$ (cf. Theorem~\ref{t-comp-factor}(1)): 
$$\gamma:A \xrightarrow{\alpha} A^{\Zar} \xrightarrow{\beta} \widehat{A}.$$

We show (1).   
By Lemma~\ref{l-complete-open}, 
we see that $\beta^*$ and $\gamma^*$ are bijective, hence so is $\alpha^*$. 
Take $A_0 \in \mathfrak O_{A}$. 
By Theorem~\ref{t-comp-factor}(3), 
we have that $\gamma_*(A_0)=\beta_*(A_0^{\Zar})$, 
which implies that 
$$\alpha_*(A_0)=\beta^*\beta_*\alpha_*(A_0)=\beta^*\gamma_*(A_0)=\beta^*\beta_*(A_0^{\Zar})=A_0^{\Zar}.$$
Thus (1) holds. 



The assertion (2) (resp. (3)) follows from Lemma \ref{l-complete-open2} 
(resp. Lemma \ref{l-complete-open3}). 
\end{proof}





\subsection{Zariskian affinoid rings}\label{ss-zar-aff}

In this subsection, 
we introduce Zariskian affinoid rings and Zariskisation of affinoid rings. 

\begin{dfn}\label{d-zar-affinoid}
Let $A=(A^{\rhd}, A^+)$ be an affinoid ring. 
We define the {\em Zariskisation} $A^{\Zar}$ of $A$ 
by $A^{\Zar}:=((A^{\rhd})^{\Zar}, (A^+)^{\Zar})$. 
\end{dfn}

\begin{rem}\label{r-aff-zar-c}
We use the same notation as in Definition~\ref{d-zar-affinoid}. 
Then the following hold. 
\begin{enumerate}
\item 
By Definition~\ref{d-zar-top}, 
$(A^{\rhd})^{\Zar}$ is an f-adic ring. 
It follows from Corollary~\ref{c-zar-bdd-open}(3) that 
$(A^+)^{\Zar} \in \mathfrak I_{A^{\Zar}}$. 
Thus 
$$A^{\Zar}=((A^{\rhd})^{\Zar}, (A^+)^{\Zar})$$ 
is an affinoid ring. 
\item 
By Theorem~\ref{t-comp-factor}, there exist natural continuous $A$-algebra homomorphisms: 
$$\gamma:A \xrightarrow{\alpha} A^{\Zar} \xrightarrow{\beta} \widehat{A},$$
where $\beta$ and $\gamma$ are the completions. 
Moreover, all of $\alpha$, $\beta$ and $\gamma$ are adic. 
\end{enumerate}
\end{rem}




\begin{dfn}
An affinoid ring $A$ is {\em Zariskian} if 
the natural homomorphism $A \to A^{\Zar}$ is an isomorphism 
of affinoid rings  
(cf. Remark~\ref{r-aff-zar-c}(2)). 
\end{dfn}

\begin{rem}\label{r-zar-zar}
Let $A$ be an affinoid ring. 
Then its Zariskisation $A^{\Zar}$ is 
Zariskian by Theorem~\ref{t-zar-zar}. 
\end{rem}




\begin{prop}\label{p-zar-univ}
Let $A$ be an affinoid ring and 
let $\alpha:A \to A^{\Zar}$ be the Zariskisation of $A$. 
Then, for a continuous ring homomorphism $\varphi:A \to B$ 
to a Zariskian affinoid ring $B$, 
there  exists a unique continuous ring homomorphism 
$\psi:A^{\Zar} \to B$ of affinoid rings such that 
$\varphi=\psi \circ \alpha$. 
\end{prop}

\begin{proof}
The assertion follows from Theorem~\ref{t-zar-univ}. 
\end{proof}


\begin{lem}\label{l-zar-quot2}
Let $A=(A^{\rhd}, A^+)$ be an f-adic ring and let $J^{\rhd}$ 
be an ideal of $A^{\rhd}$. 
Let $A/J^{\rhd}$ be the quotient affinoid ring of $A$ by $J^{\rhd}$ (cf. Subsection \ref{ss-quot-aff}). 
If $A$ is Zariskian, then also $A/J^{\rhd}$ is Zariskian. 
\end{lem}

\begin{proof}
The assertion immediately follows from Lemma~\ref{l-zar-quot}. 
\end{proof}
