\pdfoutput=1
\documentclass[microtype]{gtpart}
%\documentclass{amsart}
%\usepackage{microtype}
\usepackage{graphicx}
\usepackage[mathscr]{eucal}
\usepackage{dutchcal}
\usepackage{amssymb}
\usepackage[dvipsnames]{xcolor}
\usepackage{enumerate, cite}
%\usepackage{overpic}
%\usepackage[left]{showlabels}
%\usepackage{pinlabel}
\usepackage[margin=1.4in]{geometry}
%\usepackage{fancyhdr}
%\usepackage{fourier}
%\usepackage[matha,mathb]{mathabx} % for connect sum \#
\usepackage{hyperref}


%%%%%%%%%%%%%%%%%%%%%%%%%%%%%%%%%%%%%%%%%%%%%%
%  Begin user defined commands


\newcommand{\br}{\mathbb{R}}
\newcommand{\bc}{\mathbb C}
\newcommand{\bz}{\mathbb Z}
\newcommand{\bn}{\mathbb N}
\newcommand{\bq}{\mathbb Q}
\newcommand{\bh}{\mathbb H}
\newcommand{\bS}{\mathbb S}
\newcommand{\bd}{\mathbb D}

\newcommand{\cE}{\mathcal E}
\newcommand{\cF}{\mathcal F}
\newcommand{\cc}{\mathcal C}
\newcommand{\cm}{\mathscr M}
\newcommand{\cac}{\mathcal{AC}}





\newcommand{\al}{\alpha}
\newcommand{\be}{\beta}
\newcommand{\si}{\sigma}
\newcommand{\ep}{\epsilon}
\newcommand{\ve}{\varepsilon}
\newcommand{\vp}{\varphi}
\newcommand{\Si}{\Sigma}
\newcommand{\sig}{\sigma}


\newcommand{\ssm}{\smallsetminus}
\newcommand{\into}{\hookrightarrow}
\newcommand{\mr}{\mathring}

\newcommand{\sE}{\mathscr E}
\newcommand{\Enp}{\s\varepsilon_{\mathrm{np}}}
\newcommand{\Eno}{\s\varepsilon_{\mathrm{no}}}
\newcommand{\Ep}{\varepsilon_{\mathrm{p}}}
\newcommand{\Enpp}{\varepsilon_{\mathrm{npp}}}
\newcommand{\sU}{\mathscr U}
\newcommand{\sV}{\mathscr V}
\newcommand{\sW}{\mathscr W}
\newcommand{\sP}{\mathscr P}

\DeclareMathOperator{\arcsinh}{arcsinh}
\DeclareMathOperator{\mcg}{MCG}
\DeclareMathOperator{\pmcg}{PMCG}
\DeclareMathOperator{\mmcg}{MCG_{\mathscr M}}
\DeclareMathOperator{\isom}{Isom}
\DeclareMathOperator{\Ends}{\cE}
\DeclareMathOperator{\Mod}{Mod}
\DeclareMathOperator{\Homeo}{Homeo}
\DeclareMathOperator{\PHomeo}{PHomeo}
\DeclareMathOperator{\Diffeo}{Diffeo}
\DeclareMathOperator{\Aut}{Aut}
\DeclareMathOperator{\Fin}{Fin}
\DeclareMathOperator{\supp}{supp}
\DeclareMathOperator{\fix}{fix}


\renewcommand{\co}{\colon\thinspace}



%  End user defined commands
%%%%%%%%%%%%%%%%%%%%%%%%%%%%%%%%%%%%%%%%%%%%%%


%%%%%%%%%%%%%%%%%%%%%%%%%%%%%%%%%%%%%%%%%%%%%%
% These establish different environments for stating Theorems, Lemmas, Remarks, etc.

\newtheorem{Thm}{Theorem}[section]
\newtheorem{Thm*}{Theorem}
\newtheorem{Prop}[Thm]{Proposition}
\newtheorem{Lem}[Thm]{Lemma}
\newtheorem{Cor}[Thm]{Corollary}
\newtheorem{Conj}[Thm*]{Conjecture}
\newtheorem*{Question}{Question}

\newtheorem*{MainThm1}{Theorem}
\newtheorem{Cor*}{Corollary}



\theoremstyle{definition}
\newtheorem{Def}[Thm]{Definition}
\newtheorem{Ex}[Thm]{Example}
\newtheorem*{Ex*}{Examples}
\newtheorem{Rem}[Thm]{Remark}
\newtheorem*{Problem}{Problem}

\numberwithin{equation}{section}


% For Comments
\newcommand{\nv}[1]{\color{Cerulean}  \{NV: #1\}\color{black}}
\newcommand{\mb}[1]{\color{BrickRed} \{MB: #1\}\color{black}}


%  \pagestyle{fancy} 
%  \fancyhead{} 
%  \fancyhead[LE]{}
%  \fancyhead[LO]{\bfseries}
%  \fancyhead[R]{\bfseries \thepage}
%  \fancyfoot{}
%  \fancypagestyle{plain}

% End environments 
%%%%%%%%%%%%%%%%%%%%%%%%%%%%%%%%%%%%%%%%%%%%%%%


\title{Orientation-preserving homeomorphisms of\\Euclidean space are commutators}%: \\ automatic continuity, normal generation, and purity}

\author{Megha Bhat}
\address{Department of Mathematics \\ CUNY Graduate Center \\ New York, NY 10016}
\email{mbhat@gradcenter.cuny.edu}

\author{Nicholas G. Vlamis}
\address{Department of Mathematics \\ CUNY Graduate Center \\ New York, NY 10016, and \newline Department of Mathematics \\ CUNY Queens College \\ Flushing, NY 11367}
\email{nicholas.vlamis@qc.cuny.edu}




\begin{document}  

\begin{abstract}
We prove that every orientation-preserving homeomorphism of Euclidean space can be expressed as a commutator.  
We establish an analogous result for annuli. 
\end{abstract}


\maketitle

\vspace{-0.5in}

%-------------------
% Introduction
%-------------------

\section{Introduction}

In 1951, Ore \cite{OreSome} initiated the investigation of groups in which every element can be expressed as a commutator.
In particular, he proved that this holds for finite alternating groups and the symmetric group on \( \bn \).
Much more recently, for each \( n \in \bn \), Tsuboi \cite{TsuboiHomeomorphism} showed that the group of orientation-preserving homeomorphisms of the \( n \)-sphere \( \mathbb S^n \) has this property.
At roughly the same time, Basmajian--Maskit \cite{BasmajianSpace} proved that every orientation-preserving isometry of \( \mathbb S^{n-1} \), \( \br^n \), and \( \mathbb H^n \) can be expressed as commutator for \( n \geq 3 \), where \( \mathbb H^n \) is hyperbolic \( n \)-space.

More than 50 years earlier, Anderson \cite{AndersonAlgebraic} showed that each element of \( \Homeo^+(\mathbb S^2) \) and \( \Homeo^+(\mathbb S^3) \) can be expressed as the product of two commutators. %, and Fisher \cite{FisherGroup} proved the same for \( \Homeo^+(\mathbb R^2) \) and \( \Homeo^+(\br^3) \). 
Using the Generalized Sch\"onflies Theorem and the Annulus Theorem (see the preliminaries below), the techniques used by Anderson in dimension two can be extended to show that, for all \( n \in \bn \), every element of \( \Homeo^+(\mathbb S^n) \) can be expressed as a product of two commutators. 
The same result holds for \( \Homeo^+(\br^n) \) and \( \Homeo_0(\mathbb S^{n-1} \times \br) \) (a proof for \( n = 2 \), using ideas of Le Roux--Mann \cite{LeRouxStrong}, can be found in \cite{VlamisHomeomorphism}; the same proof technique can be used to establish the result in the higher dimensional cases as well). 
Above, \( \Homeo_0(M) \) denotes the connected component of the identity of \( \Homeo(M) \) equipped with the compact-open topology.
We note that \( \Homeo_0(\mathbb S^n) \) coincides with \( \Homeo^+(\mathbb S^n) \),  \( \Homeo_0(\br^n) \)  with  \( \Homeo^+(\br^n) \), and \( \Homeo_0(\mathbb S^n \times \br) \) with the subgroup of \( \Homeo^+(\mathbb S^n \times \br) \) stabilizing each end of \( \mathbb S^n \times \br \) (see the preliminaries below). 

Given this history and  the work of Tsuboi and Basmajian--Maskit, it is natural to ask if every element of \( \Homeo_0(\br^n) \) and \( \Homeo_0(\mathbb S^n \times \br) \) can be expressed as a commutator: our theorem answers this in the affirmative.

\begin{MainThm1}
For each \( n \in \bn \), every element of \( \Homeo_0(\mathbb R^n) \) and \( \Homeo_0(\mathbb S^n \times \br) \) can be expressed as a commutator. 
\end{MainThm1}

The proof of our theorem uses the same philosophy as Tsuboi's argument in the spherical case: 
A group element \( f \) can be expressed as a commutator if and only if there exists a group element \( g \) such that \( gf \) and \( g \) are conjugate. 
Tsuboi's idea is to start with an orientation-preserving homeomorphism \( f \) of a sphere and construct a homeomorphism \( g \) having strong enough hyperbolic dynamics with respect to \( f \) so that \( g\circ f \) exhibits the same dynamics as \( g \).
He then uses this dynamical picture to guarantee \( g\circ f \) and \( g \) are conjugate.

%Any homeomorphism of \( \br^n \) or \( \mathbb S^{n-1} \times \br \) can be viewed as a homeomorphism of \( \mathbb S^n \); however, Tsuboi's proof cannot be ported over to these cases: the requirement of fixing certain points on the sphere breaks his argument. 
%However, we are able to use the same dynamical philosophy.
%
%Before we move to the proof, we state two corollaries. 
%It is readily checked that every commutator in \( \Homeo(\br^n) \) (resp., \( \Homeo(\mathbb S^n \times \br) \)) is contained in \( \Homeo_0(\br^n) \) (resp., \( \Homeo_0(\mathbb S^n \times \br) \)); this observation yields the following corollary. 
%
%\begin{Cor*}
%For \( n \in \bn \), \( \Homeo(\br^n) \) and \( \Homeo(\mathbb S^n \times \br) \) have commutator width one. \mb{With our current definition of commutator width which seems to also be the standard definition, this would imply the false(?) statement that every homeomorphism of \(\mathbb{R}^n\) and \(\mathbb S^n \times \mathbb{R}\) is a commutator.}
%\nv{No, because commutator width is defined with respect to \( [G,G] \), so it's not required that every element of \( G \) be perfect.}
%\qed
%\end{Cor*}


The discussion above is a particular instance of a more general phenomenon.
A \emph{word} is an element in a finite-rank free group.
Given \( r \in \bn \) and a word \( {w \in \bz x_1 * \bz x_2 *\cdots*\bz x_r} \), we write \( w = w(x_1, x_2, \ldots, x_r) \) and view \( w \) as an expression in the variables \( x_1, \ldots, x_r \).
Then, given a group \( G \), we have a substitution map \( w \co G^r \to G \) given by \( (g_1, \ldots, g_r) \) maps to \( w(g_1, \ldots, g_r) \) in \( G \). 
Let \( w(G) \) be the subgroup of \( G \) generated by the set \( {G_w = \{ w(\mathbf g), w(\mathbf g)^{-1} : \mathbf g \in G^r \}} \).
Then, \( w(G) \) is a normal subgroup of \( G \) (in fact, it is characteristic).
The \emph{\( w \)-width} of \( G \) is the smallest natural number \( p \) such that every element of \( w(G) \) can be expressed as a product of \( p \) elements in \( G_w \); if such a number does not exist then the \( w \)-width is said to be infinite (see \cite{SegalWords} for more details). 
For example, if \( w = xyx^{-1}y^{-1} \in \bz x * \bz y \), then the \( w \)-width of a group is known as its \emph{commutator width}. 
In this language, Tsuboi's theorem implies that \( \Homeo_0(\mathbb S^n) \) has commutator width one, and our theorem implies   \( \Homeo_0(\br^n) \) and \( \Homeo_0(\mathbb S^n \times \br ) \) also have commutator width one. 

A group \( G \) is \emph{uniformly simple} if there exists \( n \in \bn \) such that for all \( g, f \in G \) with \( g \) nontrivial, \( f \) can be expressed as the product of at most \( n \) conjugates of \( g \) and \( g^{-1} \).
Anderson \cite{AndersonAlgebraic} showed that \( \Homeo_0(\mathbb S^2) \) is uniformly simple, and his argument can be extended to other dimensions using the generalized Sch\"onflies theorem and the annulus theorem; in fact, he showed that \( n \) can be chosen to be eight.
Therefore, given any nontrivial word \( w \), \( w(\Homeo_0(\mathbb S^n)) = \Homeo_0(\mathbb S^n) \) and the \( w \)-width of \( \Homeo_0(\mathbb S^n) \) is at most eight. 
In light of Tsuboi's result, it is natural to ask for which words \( \Homeo_0(\mathbb S^n) \) has width one. 

\begin{Problem}
Characterize the words for which the word width of \( \Homeo_0(\mathbb S^n) \) (resp., \( \Homeo_0(\br^n) \), \( \Homeo_0(\mathbb S^n \times \br) \)) is equal to one. 
\end{Problem}

%The analgous question for \( \mathrm{SL}(n,\bz) \) was recently answered by Avni--Meiri \cite{AvniWords}, and we refer the reader to their introduction for a history of such questions. 

%It also appears to meaningful to ask the question above for \( \Homeo_0(\br^n) \) and \( \Homeo_0(\mathbb S^n \times \br) \). 
%These groups are strongly bounded \cite{LeRouxStrong}, but they are not simple; however,  their germs at infinity are simple \cite{LingAlgebraic} (see also \cite{MannAutomatic}).
%\nv{Not sure how I feel about this paragraph.}


As a corollary to our theorem, we show that the \( w \)-width is one for a particular class of words.

\begin{Cor*}
\label{cor:2}
Let \( n, r \in \bn \) with \( r > 1 \), and let \( G \) denote any one of \( \Homeo_0(\mathbb S^n) \), \( \Homeo_0(\br^n) \), or \( \Homeo_0(\mathbb S^n \times \br) \).
If \( p_1, p_2, \ldots, p_r \in \bz \ssm \{0\} \) and \( g \in G \), then there exists \( g_1, g_2, \ldots, g_r \in G \) such that \( g = \prod_{i=1}^r g_i^{\circ p_i} \). %= g_1^{\circ p_1} \circ g_2^{\circ p_2} \circ \cdots \circ g_r^{\circ p_r} \). 
\end{Cor*}

The proof of Corollary~\ref{cor:2} will be given at the end of the note. 


\subsection*{Acknowledgements}
The second author is supported by NSF DMS-2212922 and PSC-CUNY Award \#65331-00 53.




\section{Preliminaries}

Before we begin, we will need the Generalized Sch\"onflies Theorem, the Annulus Theorem, and the characterizations of \( \Homeo_0(\mathbb S^n)\), \( \Homeo_0(\mathbb R^n) \), and \( \Homeo_0(\mathbb S^n \times \br) \)  given in the introduction.
An \( (n-1) \)-dimensional submanifold \( N \) of an \( n \)-manifold \( M \) is \emph{locally flat} if each point of \( N \) has an open neighborhood \( U \) in \( M \) such that the pair \( (U, U \cap N) \) is homeomorphic to \( (\br^n, \br^{n-1}) \).
Additionally, if  \( X \subset M \) is a closed subset with nonempty interior, then we say \( X \) is \emph{locally flat} if \( \partial X \) is a locally flat \( (n-1) \)-dimensional submanifold of \( M \).
In an \( n \)-manifold \( M \), we use the terminology \emph{locally flat annulus} to refer to a locally flat closed subset of \( M \) that is homeomorphic to \( \mathbb S^{n-1} \times [0,1] \). 


\begin{Thm}[Generalized Sch\"onflies Theorem {\cite{Brown1,Brown2}}]
If \( \Sigma \) is a locally flat {\( (n-1) \)-dimensional} sphere in \( \mathbb S^n \), then the closure of each component of \( \mathbb S^n \ssm \Sigma \) is homeomorphic to the closed \( n \)-ball. 
\qed
\end{Thm}

\begin{Thm}[Annulus Theorem {\cite{Kirby,Quinn}}]
The closure of the region co-bounded by two disjoint locally flat \( (n-1) \)-dimensional spheres in \( \mathbb S^n \) is a locally flat annulus.
\qed
\end{Thm}

A homeomorphism of \( \br^n \) is \emph{stable} if it can be factored as a composition of homeomorphisms each of which restricts to the identity on an open subset of \( \br^n \). 
The annulus theorem is equivalent to the stable homeomorphism theorem, which states that every homeomorphism of \( \br^n \) is stable \cite{BrownStable}.
From this, together with Alexander's trick, one readily deduces that every orientation-preserving homeomorphism of \( \mathbb R^n \) (resp., \( \mathbb S^n \)) is isotopic to the identity.  
It is also possible to deduce from the stable homeomorphism theorem that every orientation-preserving homeomorphism of \( \mathbb S^n \times \br \) that stabilizes the topological ends is isotopic to the identity; however, it is easier to see this fact by using  the fragmentation lemma (such a proof can be found for \( n =1 \) in \cite{VlamisHomeomorphism}, which can be generalized to higher dimensions).
The fragmentation lemma is a stronger version of the stable homeomorphism theorem that gives control of the open sets being fixed by each homeomorphism in the factorization; it is deduced from the work of Edwards--Kirby \cite{EdwardsDeformations}\footnote{The authors learned of the fragmentation lemma from \cite{MannAutomatic}.}.
We record these facts in the following statement.

\begin{Thm}
For \( n \in \bn \), every orientation-preserving homeomorphism of \( \mathbb S^n \) (resp., \( \br^n \)) is isotopic to the identity, and every orientation-preserving homeomorphism of \( \mathbb S^n \times \br \) stabilizing the ends is isotopic to the identity. 
\end{Thm}
%\begin{proof} 
%The statement is obvious for \(\mathbb{R}^n\) from the fact that compositions preserve isotopy. The statement for \(\mathbb{S}^n\) can be reduced to the one for \(\mathbb{R}^n\) by composing the given homeomorphism with a homeomorphism isotopic to the identity such that the composition fixes a point. We will give a proof for \( \mathbb S^n \times \mathbb{R} \). \\
%Let \(h\) be any orientation-preserving homeomorphism  of  \( \mathbb S^n \times \br \) stabilizing the ends. Fix a point \( p \in \mathbb{R}^{n+1} \) and identify \( \mathbb S^n \times \br \) with \( \mathbb{R}^{n+1} \ssm\ \{p\} \). Then  \(h \) can be written as the restriction of an orientation-preserving homeomorphism \( h' \) of  \( \mathbb{R}^{n+1} \) that fixes \(p\). 
%This \(h'\) must be stable as stated above, so we have \( h' = \prod_{i=1}^k h_i \) where each \( h_i \in \) Homeo\(_0  (\mathbb{R}^{n+1}) \) that restricts to the identity outside the non-empty open ball \( U_i \) of \( \mathbb{R}^{n+1} \). \\ %\( h_i(p)=x_i \) for \( 1 \leq i \leq k \)
%Suppose \(h_k(p)=x_k\). Choose \( \gamma_k \in \) Homeo\(_0  (\mathbb{R}^{n+1}) \) such that \( \gamma_k \) restricts to the identity outside the open ball \( V_k \subset U_k \) and \( \gamma_k (x_k) =p \).  Let \(h'_k=\gamma_k \circ h_k\). For \( 1 < i < k \), choose \(\gamma_i \in \) Homeo\(_0  (\mathbb{R}^{n+1})\) that restricts to the identity outside \(V_i \subset U_i\) and \(\gamma_i \circ \prod_{j=i}^k h_j (p)=p\), and let \( h'_i=\gamma_i \circ h_i \circ \gamma_{i+1}^{-1} \). Finally, let \( h'_1=h_1 \circ \gamma_2^{-1} \). Then each \(h'_i\) fixes \(p\) and \(\prod_{i=1}^k h'_i = \prod_{i=1}^k h_i =h' \). Further, the point \(p\) is in all the \(U_i\), hence their intersection is nonempty and we can choose \(V_i\) small enough so that each \(h'_i\) restricts to the identity outside a non-empty open ball. Using Alexander's trick on the closure of the ball we get that \(h'_i\) is isotopic to the identity. Each map in this isotopy can be made to fix a point, which we will choose to be \(p\). Thus the restrictions of the \(h'_i\) to \(\mathbb{R}^{n+1}-\{p\}\) are homeomorphisms of \(\mathbb{S}^n\times\mathbb{R}\) that are isotopic to the identity whose composition is the given \(h\). 
%\end{proof}

The following lemma is required to glue together homeomorphisms defined on disjoint pieces of a sphere, especially in the change of coordinates corollary below and the construction of the conjugating map in Proposition~\ref{prop:conjugate}.

\begin{Lem}
\label{lem:isotopy}
Let \( \iota_1, \iota_2 \co \mathbb S^{n-1} \to \mathbb S^{n} \) be embeddings such that \( \Sigma_1 := \iota_1(\mathbb S^{n-1}) \) and \( \Sigma_2 := \iota_2(\mathbb S^{n-1}) \) are locally flat and disjoint. 
%If \( \tau \in \Homeo_0(\mathbb S^n) \) such that \( \tau(\Sigma) \cap \Sigma = \varnothing \), t
Then, there exists an isotopy \( \vp \co \mathbb S^{n-1} \times [0,1] \to \mathbb S^n \) between \( \iota_1 \) and \( \iota_2 \) whose image is the locally flat annulus co-bounded by \( \Sigma_1 \) and \( \Sigma_2 \). %; in particular, \( (\tau\circ \vp)(x,0) = \vp(x,1) \). 
\end{Lem}

\begin{proof}
Let \( A \) be the locally flat annulus co-bounded by \( \Sigma_1 \) and \( \Sigma_2 \). 
Choose an embedding \( \iota_A \co \mathbb S^{n-1} \times [0,1] \to \mathbb S^n \) whose image is \( A \) and such that \( \iota_A|_{\mathbb S^{n-1}\times\{0\}} = \iota_1 \).
Let \( \iota = \iota_A|_{\mathbb S^{n-1}\times\{1\}} \).
Then,  \( \tau = \iota^{-1}\circ  \iota_2 \) is an orientation-preserving homeomorphism of \( \mathbb S^{n-1} \); hence, \( \tau \) is isotopic to the identity. 
Choose an isotopy \( H\co \mathbb S^{n-1} \times [0,1] \to \mathbb S^{n-1} \) such that \( H(x,0) = x \) and \( H(x,1) = \tau(x) \). 
Define \( h \co \mathbb S^{n-1} \times [0,1] \to \mathbb S^{n-1} \times [0,1] \) by \( h(x,t) = (H(x,t), t) \).
It is readily checked that \( h \) is a homeomorphism and that \( \vp:=  \iota_A\circ h \) is the desired map.
\end{proof}

We record as a corollary the main ways in which we will use the results above, each of which being a type of change of coordinates.

\begin{Cor}[Three change of coordinates principles]
\label{cor:coordinates}
Let \( n \in \bn \).
\begin{enumerate}[(1)]

\item Given any two locally flat \( (n-1) \)-dimensional spheres \( \Sigma \) and \( \Sigma' \) in \( \br^n \),  there exists a compactly supported ambient homeomorphism mapping \( \Sigma \) onto \( \Sigma' \).

\item Given two locally flat \( n \)-dimensional spheres \( \Sigma \) and \( \Sigma' \) in \( \mathbb S^n \times [0,1] \) each separating \( \mathbb S^n \times \{0\} \) from \( \mathbb S^n \times \{1\} \), there exists an ambient homeomorphism mapping \( \Sigma \) onto \( \Sigma' \) and fixing \( \mathbb S^n \times \{0,1\} \) pointwise. 

\item Let \( x, y \in \mathbb S^n \).
Given a sequence of locally flat annuli \( \{A_k\}_{k\in\bz} \) with pairwise-disjoint interiors  such that \( A_k \) and \( A_{k+1} \) share a boundary sphere and such that \( \mathbb S^n \ssm\{x,y\} = \bigcup_{k\in\bz} A_k \), there exists \( \tau \in \Homeo^+(\mathbb S^n) \) such that \( \tau(x) = x \), \( \tau(y) = y \), and \( \tau(A_k) = A_{k+1} \). \qed
\end{enumerate}
\end{Cor}


We can now introduce the notion of a topologically loxodromic homeomorphism. 
Tsuboi uses the terminology ``topologically hyperbolic homeomorphism'', but the definition given here is seemingly more stringent than the one he gives, and so we introduce a slight modification in our nomenclature to recognize this potential difference. 



\begin{Def}
An orientation-preserving homeomorphism \( \tau \co \mathbb S^n \to \mathbb S^n \) is \emph{topologically loxodromic} if there exists \( \tau^+, \tau^- \in \mathbb S^n \) fixed by \( \tau \) and a sequence of  locally flat annuli \( \{ A_n\}_{n\in\bz} \) with pairwise-disjoint interiors such that 
\begin{enumerate}[(i)]
\item \( \mathbb S^n \ssm \{\tau_\pm\} = \bigcup_{k\in\bz} A_k \),
\item \( A_k \) shares a boundary component with \( A_{k+1} \), 
\item \( \tau(A_k) = A_{k+1} \), and
\item \( \lim_{k\to\pm\infty} \tau^k(x) = \tau^\pm \) for all \( x \in \mathbb S^n \ssm \{\tau^+, \tau^-\} \).
\end{enumerate}
We the say the sequence of annuli \( \{A_n\}_{n\in\bn} \) is \emph{suited} to \( \tau \), and we call \( \tau^+ \) the \emph{sink} of \( \tau \) and \( \tau^- \) the \emph{source}. 
Note that every homeomorphism of \( \br^n \) and \( \mathbb S^{n-1} \times \br \) can be viewed as a homeomorphism of \( \mathbb S^n \) that fixes one or two points, respectively, corresponding to the ends, and so we say a homeomorphism of either of these spaces is topologically loxodromic if it is a topologically loxodromic homeomorphism as a homeomorphism of \( \mathbb S^n \). 
\end{Def}

The main motivating examples of topologically loxodromic homeomorphisms are loxodromic M\"obius transformations of \( \mathbb S^n \),  dilations of \( \mathbb R^n \), and translations of \( \mathbb S^n \times \br \) of the form \( (x,t) \mapsto (x, t+t_0) \) for some \( t_0 \in \br \).  
Of course the latter two examples are just loxodromic M\"obius transformations in different coordinates. 

%Given \( g \in \Homeo(\mathbb S^n) \), let \( \fix(g) = \{ x \in \mathbb S^n : g(x) = x \} \). 

In the next proposition, we prove that any two topologically loxodromic homeomorphisms of \( \mathbb S^n \) are conjugate.  
This is the key fact in Tsuboi's argument, and the following results appear as part of Tsuboi's proof in reference to particular maps.
We pull the ideas out here and formalize them in the general setting. 
As a corollary, we provide the analogous statement for \( \br^n \) and \( \mathbb S^n \times \br \). 

%\begin{Lem}
%\label{lem:isotopy}
%Let \( \iota \co \mathbb S^{n-1} \to \mathbb S^{n} \) be an embedding such that \( \Sigma := \iota(\mathbb S^{n-1}) \) is locally flat. 
%If \( \tau \in \Homeo_0(\mathbb S^n) \) such that \( \tau(\Sigma) \cap \Sigma = \varnothing \), then there exists an embedding \( \vp \co \mathbb S^{n-1} \times [0,1] \to \mathbb S^n \) whose image is the locally flat annulus co-bounded by \( \Sigma \) and \( \tau(\Sigma) \) and such that \( \vp(x,0) = \iota(x) \) and \( \vp(x,1) = (\tau\circ \iota)(x) \); in particular, \( (\tau\circ \vp)(x,0) = \vp(x,1) \). 
%\end{Lem}
%
%\begin{proof}
%Let \( A \) be the locally flat annulus co-bounded by \( \Sigma \) and \( \tau(\Sigma) \). 
%Choose an embedding \( \iota_A \co \mathbb S^{n-1} \times [0,1] \to \mathbb S^n \) whose image is \( A \) and such that \( \iota_A|_{\mathbb S^{n-1}\times\{0\}} = \iota \).
%Let \( \iota_1 = \iota_A|_{\mathbb S^{n-1}\times\{1\}} \).
%Then,  \( \tau^* = \iota_1^{-1}\circ \tau \circ \iota \) is an orientation-preserving homeomorphism of \( \mathbb S^{n-1} \); hence, \( \tau^* \) is isotopic to the identity. 
%Choose an isotopy \( H\co \mathbb S^{n-1} \times [0,1] \to \mathbb S^{n-1} \) such that \( H(x,0) = x \) and \( H(x,1) = \tau^*(x) \). 
%Define \( h \co \mathbb S^{n-1} \times [0,1] \to \mathbb S^{n-1} \times [0,1] \) by \( h(x,t) = (H(x,t), t) \).
%It is readily checked that \( h \) is a homeomorphism and that \( \vp:=  \iota_A\circ h \) is the desired map.
%\nv{We should check!} 
%\end{proof}

\begin{Prop}
\label{prop:conjugate}
Any two topologically loxodromic homeomorphisms in \( \Homeo_0(\mathbb S^n) \) are conjugate in \( \Homeo_0(\mathbb S^n) \). 
%If \( \tau, \sigma \in \Homeo^+(\mathbb S^n) \) are topologically loxodromic, then \( \tau = h \circ\sigma \circ h^{-1} \) for some \( h \in \Homeo^+(\mathbb S^n) \) satisfying \( \fix(\sigma) \cap \fix(\tau) \subset \fix(h) \). 
%\nv{I think this statement about fix is redundant.}
\end{Prop}

\begin{proof}
Let \( \tau, \sigma \in \Homeo_0(\mathbb S^n) \) be topologically loxodromic. 
%Let \( x_\pm \) and \( y_\pm \) be the fixed points of \( \tau \) and \( \sigma \), respectively, and 
Let \( \{ A_n\}_{n\in\bz} \) and \( \{ B_n\}_{n\in\bz} \) be sequences of locally flat annuli suited to \( \tau \) and \( \sigma \), respectively. 
Let \( \Sigma_A \) be the component of \( A_0 \) such that \( \partial A_0 = \Sigma_A \cup \tau(\Sigma_A) \), and similarly, define \( \Sigma_B \) so that \( \partial B_0 = \Sigma_B \cup \sigma(\Sigma_B) \). 

Fix embeddings \( \iota_A, \iota_B \co \mathbb S^{n-1} \to \mathbb S^n \) such that the images of \( \iota_A \) and \( \iota_B \) are \( \Sigma_A \) and \( \Sigma_B \), respectively. 
Then, applying Lemma~\ref{lem:isotopy}, we get embeddings \( \vp_A, \vp_B \co \mathbb S^{n-1}\times [0,1] \to \mathbb S^n \) whose images are \( A_0 \) and \( B_0 \), respectively, and such that \( \vp_A \) (resp., \( \vp_B \)) is an isotopy between \( \iota_A \) and \( \tau\circ \iota_A \) (resp., \( \iota_B \) and \( \sigma \circ \iota_B \)).  
Set \( \vp = \vp_A \circ \vp_B^{-1} \). 

%Choose a homeomorphism \( \iota_A \co \mathbb S^{n-1} \times [0,1] \to A_0 \), and let \( \iota_0, \iota_1 \co \mathbb S^{n-1} \to \mathbb S^n \) be given by \( \iota_i(x) = \iota_A(x, i) \) for \( i \in \{1,2\} \). 
%Then,  \( \tau^* = \iota_1^{-1}\circ \tau \circ \iota_0 \) is an orientation-preserving homeomorphism of \( \mathbb S^{n-1} \), and hence isotopic to the identity. 
%Choose such an isotopy \( H \) such that \( H(x,0) = x \) and \( H(x,1) = \tau^*(x) \). 
%Define \( h_A \co \mathbb S^{n-1} \times [0,1] \to \mathbb S^{n-1} \times [0,1] \) by \( h(x,t) = (H(x,t), t) \).
%It is readily checked that \( h_A \) is a homeomorphism and \( \vp_A :=  \iota_A\circ h_A \) is an isotopy between \( \iota_0 \) and \( \tau\circ \iota_0 \) that is a homeomorphism onto its image \( A_0 \). 
%\nv{We should check!} 
%We can similarly define the embedding \( \vp_B \co \mathbb S^{n-1} \times [0,1] \to S^n \) whose image is \( B_0 \) and that also serves as an isotopy between the embeddings \( \iota' \) and \( \sigma \circ \iota' \) for some fixed embedding \( \iota' \co \mathbb S^{n-1} \to \Sigma_B \).
%Finally, let \( \vp = \vp_A \circ \vp_B^{-1} \). 

%Choose homeomorphisms \( \psi_A \co \mathbb S^{n-1} \to \Sigma_A \) and \( \psi_B \co \mathbb S^{n-1} \to \Sigma_B \).
%Then, the embeddings \( \psi_A \) and \( \tau\circ \psi_A \) of \( \mathbb S^{n-1} \) into \( \mathbb S^n \) are isotopic; let \( \Psi_A \co \mathbb S^{n-1} \times [0,1]  \to A_0 \) be such an isotopy.
%Similarly, let \( \Psi_B \co \mathbb S^{n-1} \times [0,1] \to B_0 \) be an isotopy between the embeddings of \( \mathbb S^{n-1} \) into \( \mathbb S^n \) given by \( \psi_B \) and \( \sigma \circ \psi_B \). 
%Finally, let \( \vp \co B_0 \to A_0 \) be given by \( \vp = \Psi_A \circ \Psi_B^{-1} \). 
%\nv{I'm using some less trivially obvious applications of the preliminary theorems here.  See \href{https://math.stackexchange.com/questions/3164454/are-any-two-n-balls-in-mathbbrn-isotopic}{Stack Exchange}. This also makes me think the sphere minus Cantor set case might be harder than I was expecting.}

Define \( h \co \mathbb S^n \to \mathbb S^n \) by \( h(\sigma_\pm) = \tau_\pm \) and \( h(x) = (\tau^n \circ \vp \circ \sigma^{-n}) (x) \) for \( x \in B_n \).
It is now readily checked that \( h \) is well-defined for the elements of \( B_n \cap B_{n+1} \), establishing that \( h \) is a well-defined orientation-preserving homeomorphism of \( \mathbb S^n \).
Now, for \( x \in A_n \),
\begin{align*}
(h\circ \sigma \circ h^{-1})(x) &= \left[(\tau^{n+1}\circ \vp \circ \sigma^{-(n+1)})\circ \sigma \circ (\sigma^n \circ \vp^{-1} \circ \tau^{-n})\right](x)\\
	&= \tau(x).
\end{align*}
Hence, \( \tau = h \circ \sigma \circ h^{-1} \).
\end{proof}

\begin{Cor}
\label{cor:conjugate}
Let \( G \) denote either \( \Homeo_0(\br^n) \) or \( \Homeo_0(\mathbb S^n \times \br) \).
Any two topologically loxodromic homeomorphisms in \( G \) that share a sink or a source are conjugate in \( G \).
\end{Cor}

\begin{proof}
Viewing two given topologically loxodromic homeomorphisms in \( G \) as homeomorphisms of \( \mathbb S^n \) with a shared sink, a shared source, or both, the conjugating map constructed in Proposition~\ref{prop:conjugate} will preserve any shared source or sink, and hence the conjugation occurs in the appropriate group.
\end{proof}


\section{Proofs}

We prove the theorem first for annuli then for Euclidean spaces.  

\begin{Thm}
\label{thm:annulus}
For \( n \in \bn \), every element of \( \Homeo_0(\mathbb S^{n} \times \br) \) is a commutator. 
\end{Thm}

\begin{proof}
Let \( f \in \Homeo_0(\mathbb S^{n} \times \br) \). 
Let \( \Sigma_0 = \mathbb S^{n} \times \{0\} \), and set \( t_0 = 0 \). 
Let \( n_0 \in \bn \) such that \( A_0 := \mathbb S^{n} \times [-n_0, n_0] \) contains \( \Sigma_0 \cup f(\Sigma_0) \) in its interior. 
Now, choose \( t_1, n_1 \in \bn \) such that \( \Sigma_{1} = \mathbb S^{n} \times \{t_1\} \)  satisfies \( \Sigma_{1} \cup f(\Sigma_{1}) \) is contained in the interior of \( A_{1} := \mathbb S^{n} \times [n_0, n_1] \). 
Continuing in this fashion in both directions, we construct sequences \( \{n_k\}_{k\in\bz} \) and \( \{t_k\}_{k\in\bz} \) of integers such that, by setting \( \Sigma_k =  \mathbb S^{n} \times \{t_k\} \) and \( A_k = \mathbb S^{n} \times [n_k, n_{k+1}] \), we have \( \Sigma_k \cup f(\Sigma_k) \) is contained in the interior of \( A_k \). 

Now, let \( g' \in \Homeo_0(\mathbb S^{n} \times \br) \) such that \( g'(A_k) = A_{k+1} \).
Note, by construction, \( \bigcup_{k\in\bz} A_k \) is all of \( \mathbb S^n \times \br \), and so \( g' \) is topologically loxodromic. 
As both \( (g'\circ f)(\Sigma_k) \) and \( \Sigma_{k+1} \) are locally flat annuli contained in the interior of \( A_{k+1} \), we can choose \( h_k \in \Homeo_0(\mathbb S^{n} \times \br) \) that is supported in the interior of \( A_{k+1} \) and satisfies \( h_k(g'(f(\Sigma_k))) = \Sigma_{k+1} \). 
The sequence \( \{h_k\}_{k\in \bz} \) consists of homeomorphisms with pairwise-disjoint supports, and hence, we can define \( h = \prod_{k\in\bz} h_k \).
Set \( g = h\circ g' \). 

As the support of each \( h_k \) is contained in the interior of \( A_{k+1} \), we have that \[ g(A_k) = h_k(g'(A_k)) = h_k(A_{k+1}) = A_{k+1}; \] in particular, \( g \) is topologically loxodromic.
Let \( B_k \) be the locally flat annulus co-bounded by \( \Sigma_k \) and \( \Sigma_{k+1} \).
Then, as \( (g\circ f)(\Sigma_k) = \Sigma_{k+1} \), we have \( (g\circ f)(B_k) = B_{k+1} \).
Therefore, \( g\circ f \) is topologically loxodromic.
Moreover, \( g \) and \( g\circ f \) share the same sink, and hence they are  conjugate  by Corollary~\ref{cor:conjugate}.
This establishes that \( f \) can be expressed as a commutator.  
\end{proof}


%We finish with considering Euclidean spaces of dimension at least two. 

\begin{Thm}
For \( n \in \bn \), every element of \( \Homeo_0(\mathbb R^n) \) is a commutator. 
\end{Thm}

\begin{proof}
Letting \( \mathbb S^0 \) be a singleton, the proof  in the annulus case with \( n = 0 \) shows that every element of \( \Homeo_0(\br) \) can be expressed as a commutator. 
We may now assume that \( n > 1 \).

Fix \( z \in \mathbb S^n \), and identify \( \Homeo_0(\br^n) \) with the stabilizer of \( z \) in \( \Homeo_0( \mathbb S^n) \).  
Let \( f \) be an element of \( \Homeo_0(\mathbb S^n) \) such that \( f(z) = z \). 
We will show that \( f \) can be expressed as a commutator of a pair of elements in \( \Homeo_0(\mathbb S^n) \) that each fix \( z \). 
The identity homeomorphism is clearly a commutator, so we may assume that \( f \) is not the identity, and therefore there exists \( y \in \mathbb S^n \) such that \( f(y) \neq y \). 
By continuity, there exists a locally flat ball \( D_0 \) centered at \( y \) such that \( f(D_0) \cap D_0 = \varnothing \). 
Choose a locally flat ball \( D_1 \) centered at \( z \) that is disjoint from \( D_0 \cup f(D_0) \). 
Again by continuity, by shrinking \( D_1 \) if necessary, we may assume that \( f(D_1) \) is also disjoint from \( D_0 \cup f(D_0) \). 

First, in \( D_1 \), we proceed identically as we did in the annulus case. 
Choose a locally flat  annulus \( A_1 \) such that \( \partial D_1 \cup f(\partial D_1) \) is contained in the interior of \( A_1 \) and such that \( A_1 \) is disjoint from \( D_0 \cup f(D_0) \). 
Now choose a locally flat ball \( D_{2} \) centered at \( z \) such that \( D_2 \cup f(D_2) \) is disjoint from \( A_1 \). 
Choose a locally flat  annulus \( A_2 \) such that \( A_1 \) and \( A_2 \) share a boundary component and \( \partial D_2 \cup f(\partial D_2) \) is contained in the interior of \( A_2 \). 
Continuing in this fashion, we build a sequence of locally flat  annuli \( \{ A_k\}_{k\in\bn} \) and locally flat balls \( \{D_k\}_{k\in\bn} \) such that \( A_k \) and \( A_{k+1} \) share a boundary, \( \partial D_k \cup f(\partial D_k) \) is contained in the interior of \( A_k \), and \( \bigcap_{k\in\bn} D_k = \{ z \} \). 
Note that the last condition is not a priori guaranteed, but at each stage we are free to choose \( D_k \) to have radius less than \( 1/k \), forcing the intersection of the \( D_k \) to be \( \{z\} \). 

Let \( g_0' \) be a topologically loxodromic homeomorphism of \( \mathbb S^n \) that fixes \( z \), that maps \( A_{k+1} \) to \( A_{k} \) for \( k \in \bn \), and that maps \( f(\partial D_1) \) to \( \partial D_0 \). 
Then, \( (g_0'\circ f)(\partial D_{k+1}) \) and \( \partial D_{k} \) are locally flat spheres in the annulus \( A_k \), and so we may choose  \( h_k \in \Homeo( \mathbb S^n ) \) supported in the interior of \( A_{k} \) such that \( h_k(g_0'(f(\partial D_{k+1}))) = \partial D_{k} \). 
As the \( h_k \) have pairwise disjoint support, we can define \( h = \prod_{k\in\bn} h_k \). 
Let \( g_0 = h \circ g_0' \). 
Then, for \( k \in \bn\cup\{0\} \), setting \( T_{k} \) to be the locally flat  annulus bounded by \( \partial D_{k} \) and \( \partial D_{k+1} \), we have \( (g_0\circ f)(T_{k+1}) = T_{k} \). 

At this point, \( g_0 \circ f \) behaves like a topologically loxodromic homeomorphism when restricted to \( D_1 \), and so now we have to edit \( g_0 \) in the complement of \( D_1 \). 
This portion of the argument is a version of Tsuboi's argument for spheres. 
Let \( D' = f(D_0) \) and \( D = g_0(D') \), and note that \( f(D) \subset D' \) and \( g_0(f(D)) \subset D \). 
Moreover, observe that \( D' \subset T_0 = g_0(A_1) \) and \( D \subset g_0(T_0) \); hence, modifying \( g_0 \) in either \( D' \) or \( D \) will not change the fact that \( g_0 \) is topologically loxodromic. 

Fix \( x \in D \).
We may assume that \( g_0(f(x)) = x \); indeed, if not, then we may post compose \( g_0 \) with a homeomorphism of \( \mathbb S^n \) supported in \( D \) that maps \( g_0(f(x)) \) to \( x \). 
And, as just noted above, this edited version of \( g_0 \) remains topologically loxodromic. 
We will now edit \( g_0 \)  in \( D' \) so that \( g_0 \circ f \) will be topologically loxodromic with \( z \) its sink and \( x \) its source.

Let \( \Sigma_0 = \partial D \), and let \( \Sigma_1 = g_0(f(\Sigma_0)) \).
Note \( \Sigma_0 \cap \Sigma_1 = \varnothing \), and therefore \( \Sigma_0 \) and \( \Sigma_1 \) co-bound a locally flat  annulus, which we label \( T_{-1} \). 
Let \( B_2 \) be a ball centered at \( x \) disjoint from \( \Sigma_1 \).
Then, we can choose a locally flat ball \( B_2' \) in \( D' \) centered at \( f(x) \) such that \( B_2' \cap f(\Sigma_1) = \varnothing \) and  \( g_0(f(B_2')) \subset B_1 \). 
%We can then find a compact annulus \( C_1 \) such that \( f(\Sigma_1) \cup \partial B_2' \) is contained in the interior of \( C_1 \) and \( C_1 \cap f(\Sigma_0) = \varnothing \). 
We can then choose \( \sigma_1 \in \Homeo(\mathbb S^n) \) such that \( \sigma_1 \) is supported in the ball bounded by \( f(\Sigma_0) \) that contains \( x \) and such that \( \sigma_1(f(\Sigma_1)) = \partial B_2' \). 
Set \( \Sigma_2 = g_0(\sigma_1(f(\Sigma_1))) \), set \( T_{-2} \) to be the  locally flat annulus co-bounded by \( \Sigma_1 \) and \( \Sigma_2 \), and set  \( g_1 = g_0 \circ \sigma_1 \).
Then, \( (g_1\circ f)(T_k) = T_{k-1} \) for all \( k \in \bn \cup \{0, -1\} \). 
Note that the choice of \( B_2' \) depends on the choice of \( B_2 \), but each of these balls can be made arbitrarily small.

%For clarity, let us describe the next step. 
%Choose a ball \( B_3 \) centered at \( x \) disjoint from \( \Sigma_2 \).
%Then, choose a locally flat ball \( B_3' \) in \( D' \) centered at \( f(x) \) such that \( B_3' \cap f(\Sigma_2) = \varnothing \) and \( g_0(f(B_3')) \subset B_3 \). 
%Choose a compact annulus \( C_2 \) containing \( f(\Sigma_2) \cup B_3'' \) in its interior and that is disjoint from \( f(\Sigma_2) \). 
%Choose \( \sigma_2 \in \Homeo(\mathbb S^n) \) supported in \( C_2 \) such that \( \sigma_2(f(\Sigma_2)) = \partial B_3' \). 
%Note that \( C_1 \) and \( C_2 \) might have nontrivial intersection, but the intersection is contained in the annulus \( f(T_{-2}) \). 
%Set \( \Sigma_3 = g_1(\sigma_2(f(\Sigma_2))) \), set \( T_{-3} \) to be the compact locally flat annulus co-bounded by \( \Sigma_2 \) and \( \Sigma_3 \), and set \( g_2 = g_1 \circ \sigma_2 \)\( (= g_0\circ \sigma_1 \circ \sigma_2) \). 
%Then, \( (g_2\circ f)(T_k) = T_{k-1} \) for all \( k \in \bn \cup \{0, -1, -2\} \). 


Continuing in this fashion, for each \( m \in \bn \), we obtain a  locally flat annulus \( T_{-m} \) and a homeomorphism \( \sigma_m \) of \( \mathbb S^n \) supported in a ball of radius \( 1/m \) centered at \( f(x) \) such that
\begin{enumerate}[(1)]
\item \( \mathbb S^n \ssm \{ x,z\} = \bigcup_{k\in\bz} T_k \),
\item \( (g_m\circ f)(T_k) = T_{k-1} \) for every integer \( k \geq -m \), and
\item \( g_m = g_{m-1}\circ \sigma_m = g_0 \circ (\sigma_1 \circ \sigma_{2} \circ \cdots \circ \sigma_m) \).
\end{enumerate}
These properties guarantee that \( \lim_{m\to\infty} g_m \) exists, call it \( g \), and that \( (g \circ f)(T_k)= T_{k-1} \) for every \( k \in \bz \); hence, \( g\circ f \) is topologically loxodromic. 
Moreover, as already noted,  \( g \) is topologically loxodromic as \( g \) agrees with \( g_0 \) outside of \( D \cup D' \). 
And, as \( g \) and \( g\circ f \) share the same sink, they are conjugate in \( \Homeo_0(\br^n) \) by Corollary~\ref{cor:conjugate}.
Thus, \( f \) can be expressed as a commutator in \( \Homeo_0(\br^n) \).
\end{proof}

We finish by providing a proof of Corollary~\ref{cor:2}.

\begin{proof}[Proof of Corollary~\ref{cor:2}]
Without loss of generality, it is enough to prove the statement with \( p_i \in \bn \) for each \( i \in \{1, \ldots, r\} \), as we can always replace an element with its inverse in the decomposition to switch the sign of the exponent. 

Fix \( g \in G \).
We have shown in the above proofs that there exist \( f, h \in G \) such that \( g = f\circ h \) and both \( f \) and \( h \) are topologically loxodromic. 
Applying the same techniques to \( f \), we can write \( f = f_1 \circ f_2 \) with both \( f_1 \) and \( f_2 \) topologically loxodromic. 
Therefore, continuing this splitting as many times as necessary, we can write \( g = \prod_{i=1}^r f_i \) with \( f_i \) topologically loxodromic. 
Now, every power of a topologically loxodromic homeomorphism is itself topologically loxodromic. 
Therefore, by Proposition~\ref{prop:conjugate} and Corollary~\ref{cor:conjugate}, \( f_i^{\circ p_i} \) is conjugate to \( f_i \). 
Choose \( h_i \) such that \( f_i = h_i \circ f_i^{\circ p_i} \circ h_i^{-1} \). 
Then, setting \( g_i  = h_i \circ f_i \circ h_i^{-1} \), we have \( f_i = g_i^{\circ p_i} \), and hence, \( g = \prod_{i=1}^r g_i^{\circ p_i} \). 
\end{proof}


\bibliographystyle{amsalpha}
\bibliography{commutators}


\end{document}
