It required approximately 2 (4) hours to train the 2D (3D) model for 200 epochs using the aforementioned GPU cards. The time required for generating the whole sCT volume of a patient was approximately 5.5 s for both models.  

Figure~\ref{fig3} shows transverse slices of sCTs generated by the 2D and 3D models along with the corresponding slices of the normalized T1-weighted MR images and deformed CTs from three patients. As shown in the difference maps, both models gave accurate HU value predictions for most regions, especially soft tissues, but had difficulty generating accurate HU values near the body contour and bone outlines. 

\begin{figure}
\begin{center}
\begin{tabular}{c}
\includegraphics[height=18cm]{figure3.pdf}
\end{tabular}
\end{center}
\caption 
{\label{fig3}
Transverse slices of the normalized MRIs (row 1), the dCTs (row 2), the 2D model sCTs (row3) and the 3D model sCTs (row 4) from three patients. The last two rows show the difference maps between the 2D model sCTs and the dCTs (row 5), and the difference maps between the 3D model sCTs and the dCTs (row 6). The color bar is associated with all images except normalized MRIs.} 
\end{figure} 

The MAE, including all patients, is shown in Figure~\ref{fig4} as a function of dCT values. The MAE was calculated in 25 HU bins.  Both models behaved similarly, with similar MAE curves for most HU values except that the 2D model yielded greater MAEs than the 3D model within (-650, -200) HU, and vice-versa within (850,1600) HU. 
\begin{figure}
\begin{center}
\begin{tabular}{c}
\includegraphics[height=8cm]{figure4.pdf}
\end{tabular}
\end{center}
\caption 
{\label{fig4}
MAE of voxels within body masks from all patients as a function of dCT values, calculated in 25 HU bins.} 
\end{figure}

Table~\ref{tab1} summarizes the statistics of the voxel-wise and geometric metrics averaged across all patients and shows the results of Wilcoxon signed-rank tests. The maximum MAEs within the body were 56.5 HU and 53.1 HU for 2D and 3D models, respectively. The minimum bone-region DSCs were 0.70 and 0.72 for the 2D and 3D models, respectively. As shown in Tab. 1, p-values of all Wilcoxon signed-rank tests were less than 0.05, except for recall. We also performed paired t-tests which yielded the same hypothesis-testing results. 
  

\begin{table}[ht]
\begin{center}
\begin{tabular}{|c|c|c|c|c|}
\hline
\multicolumn{2}{|c|}{} & 2D model & 3D model & P value\\
\hline
\multirow{3}{*}{MAE [HU]} & whole body & 40.5 $\pm$ 5.4 & 37.6 $\pm$ 5.1 & 3.90 $\times$ 10$^{-4}$\\
& soft tissue & 28.9 $\pm$ 4.7 & 26.2 $\pm$ 4.5 & 2.54 $\times$ 10$^{-4}$\\
& bone & 159.7 $\pm$ 22.5 & 154.3 $\pm$ 22.3 & 0.010\\ \hline
\multicolumn{2}{|c|}{DSC} & 0.81 $\pm$ 0.04 & 0.82 $\pm$ 0.04 & 0.048\\
\hline
\multicolumn{2}{|c|}{recall} & 0.85 $\pm$ 0.04 & 0.84 $\pm$ 0.04 & 0.60\\
\hline
\multicolumn{2}{|c|}{precision}  & 0.77 $\pm$ 0.09 & 0.80 $\pm$ 0.08 & 1.7 $\times$ 10$^{-3}$\\
\hline
\end{tabular}
\caption{\label{tab1} Comparison of the MAEs for different HU-thresholded regions, and the DSC, recall, and precision for the bone region for the 2D and 3D models. Results were averaged across the 20-patient cohort. The rightmost column shows the results of Wilcoxon signed-rank tests on the difference of four reported metrics between the two models.} 
\end{center}
\end{table} 
