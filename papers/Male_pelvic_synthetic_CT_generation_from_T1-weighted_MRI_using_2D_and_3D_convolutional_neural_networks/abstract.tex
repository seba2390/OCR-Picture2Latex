\textbf{Purpose:} The improved soft tissue contrast of magnetic resonance imaging (MRI) compared to computed tomography (CT) makes it a useful imaging modality for radiotherapy treatment planning. Even when MR images are used for treatment planning, standard clinical practice currently also requires a CT for dose calculation and x-ray based patient positioning. This increases workloads, introduces uncertainty due to the required inter-modality image registrations, and involves unnecessary irradiation. While it would be beneficial to use exclusively MR images, a method needs to be employed to estimate a synthetic CT (sCT) for generating electron density maps and patient positioning reference images. We investigated deep learning approaches, 2D and 3D convolutional neural network (CNN) methods, to generate a male pelvic sCT using a T1-weighted MR image and evaluate their performance using geometric and voxel-wise metrics. \\
\textbf{Method:}
A retrospective study was performed using CTs and T1-weighted MR images of 20 prostate cancer patients. The proposed 2D CNN model, which contained 27 convolutional layers, was modified from the SegNet for better performance.  3D version of the CNN model was also developed. Both CNN models were trained from scratch to map intensities of T1-weighted MR images to CT Hounsfield Unit (HU) values. Each sCT was generated in a five-fold-cross-validation framework and compared with the corresponding CT using voxel-wise mean absolute error (MAE). The sCT geometric accuracy was evaluated by comparing bony structures in the CTs and the sCTs using dice similarity coefficient (DSC), recall, and precision. Wilcoxon signed-rank tests were performed to evaluate the differences between the 2D and 3D CNN models.  \\ 
\textbf{Result:}
Generating pelvic sCT datasets required approximately 5.5 s using the proposed deep learning methods. The MAE averaged across all patients were 40.5 $\pm$ 5.4 HU and 37.6 $\pm$ 5.1 HU for the 2D and 3D CNN models, respectively. The DSC, recall, and precision of the bony structures were 0.81 $\pm$ 0.04, 0.85 $\pm$ 0.04, and 0.77 $\pm$ 0.09 for the 2D CNN model, and 0.82 $\pm$ 0.04, 0.84 $\pm$ 0.04, and 0.80 $\pm$ 0.08 for the 3D CNN model, respectively. P values of the Wilcoxon signed-rank tests were less than 0.05 except for recall, which was 0.6. \\
\textbf{Conclusion:}
The 2D and 3D CNN models generated accurate pelvic sCTs for the 20 patients using T1-weighted MR images. The evaluation metrics and statistical tests indicated that the 3D model was able to generate sCTs with better MAE, bone DSC, and bone precision. The accuracy of the dose calculation and patient positioning using generated sCTs will be tested and compared for the two models in the future. 