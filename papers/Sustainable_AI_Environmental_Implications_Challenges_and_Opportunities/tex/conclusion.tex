\section{Conclusion}
\label{sec:conclusion}

This paper is the first effort to explore the environmental impact of the super-linear trends for AI growth from a holistic perspective, spanning \textit{data}, \textit{algorithms}, and \textit{system hardware}. We characterize the carbon footprint of AI computing by examining the model development cycle across industry-scale ML use cases at Facebook and, at the same time, considering the life cycle of system hardware. Furthermore, we capture the operational and manufacturing carbon footprint of AI computing and present an end-to-end analysis for \textit{what} and \textit{how} hardware-software design and at-scale optimization can help reduce the overall carbon footprint of AI. We share the key challenges and chart out important directions across all dimensions of AI---data, algorithms, systems, metrics, standards, and best experimentation practices.
% AI researchers and technology developers must take a deliberate approach to assess the broader impact of proposed research. 
Advancing the field of machine intelligence must not in turn make climate change worse. We must develop AI technologies with a deeper understanding of the societal and environmental implications.
%We hope the messages and insights presented in this paper can inspire the community to advance the field of AI in an environmentally-responsible manner.

%Advances in AI are currently driven by AI research that seeks to improve accuracy (or related measures) through the use of massive computational power while disregarding cost. To our knowledge, this is the first effort to explore the environmental impact of these trends from a holistic perspective; with operational emissions spanning across the  ML lifecycle and the embody cost from systems hardware. We present optimization efforts within \fb across the stack of hardware, infrastructure, platform and algorithms to achieve significant reduction in energy and carbon emissions. But with the growing demand for AI, there is a lot more we can do. We outline the challenges and opportunities to designing and building sustainable AI technologies, spanning Data, Algorithm, and System Hardware.