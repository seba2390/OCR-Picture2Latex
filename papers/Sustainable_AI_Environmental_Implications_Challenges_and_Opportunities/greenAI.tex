%\documentclass{sig-alternate}
%\documentclass[conference,a4paper]{IEEEtran}
\documentclass{IEEEtran}
\setlength{\paperheight}{11in}
\setlength{\paperwidth}{8.5in}
\usepackage{microtype}
\usepackage{graphicx}
\usepackage{subfigure}
\usepackage{booktabs} % for professional tables
\usepackage{xcolor}
\usepackage{comment}
\usepackage{wrapfig}
\usepackage{url}
\usepackage{xurl}

\newcommand{\fb}{Facebook}
\newenvironment{itemize*}%
  {\begin{itemize}%
    \setlength{\itemsep}{0pt}%
    \setlength{\parskip}{0pt}}%
  {\end{itemize}}
% hyperref makes hyperlinks in the resulting PDF.
% If your build breaks (sometimes temporarily if a hyperlink spans a page)
% please comment out the following usepackage line and replace
% \usepackage{mlsys2022} with \usepackage[nohyperref]{mlsys2022} above.
\usepackage{hyperref}

% Attempt to make hyperref and algorithmic work together better :
\newcommand{\theHalgorithm}{\arabic{algorithm}}


\usepackage{authblk}
\newlength{\bibitemsep}\setlength{\bibitemsep}{.2\baselineskip plus .05\baselineskip minus .05\baselineskip}
\newlength{\bibparskip}\setlength{\bibparskip}{0pt}
\let\oldthebibliography\thebibliography
\renewcommand\thebibliography[1]{%
  \oldthebibliography{#1}%
  \setlength{\parskip}{\bibitemsep}%
  \setlength{\itemsep}{\bibparskip}%
}



\title{Sustainable AI: Environmental Implications, Challenges and Opportunities
}


\author{\normalsize{Carole-Jean Wu, Ramya Raghavendra, Udit Gupta, Bilge Acun, Newsha Ardalani, Kiwan Maeng,\\ Gloria Chang, Fiona Aga Behram, James Huang, Charles Bai, Michael Gschwind, Anurag Gupta, Myle Ott,\\ Anastasia Melnikov, Salvatore Candido, David Brooks, Geeta Chauhan, Benjamin Lee, Hsien-Hsin S. Lee,\\ Bugra Akyildiz, Maximilian Balandat, Joe Spisak, Ravi Jain, Mike Rabbat, Kim Hazelwood}}

\affil{Facebook AI}

% Just remember to make sure that the TOTAL number of authors
% is the number that will appear on the first page PLUS the
% number that will appear in the \additionalauthors section.

\pagenumbering{arabic}

\begin{document}
\maketitle
%\thispagestyle{firstpage}
\pagestyle{plain}



%%%%%% -- PAPER CONTENT STARTS-- %%%%%%%%

\begin{abstract}
This paper explores the environmental impact of the super-linear growth trends for AI from a holistic perspective, spanning \textit{Data}, \textit{Algorithms}, and \textit{System Hardware}. 
We characterize the carbon footprint of AI computing by examining the model development cycle across industry-scale machine learning use cases and, at the same time, considering the life cycle of system hardware.
Taking a step further, we capture the operational and manufacturing carbon footprint of AI computing and present an end-to-end analysis for \textit{what} and \textit{how} hardware-software design and at-scale optimization can help reduce the overall carbon footprint of AI.
Based on the industry experience and lessons learned, we share the key challenges and chart out important development directions across the many dimensions of AI. 
We hope the key messages and insights presented in this paper can inspire the community to advance the field of AI in an environmentally-responsible manner.
\end{abstract}


\section{Introduction}  \label{sec:introduction}

\newcommand\inexpIntro[3]{#1?(#2,#3).}
\newcommand\rinexpIntro[3]{*#1?(#2,#3).}
\newcommand\outexpIntro[3]{#1!(#2,#3).}
\newcommand\outatomIntro[3]{#1!(#2,#3)}

We propose a fully automated method for proving termination of \(\pi\)-calculus processes.
Although there have been a lot of studies on termination analysis for the \(\pi\)-calculus
and related calculi~\cite{Deng06IC,Demangeon07,SangiorgiTermination,KobayashiHybrid,Yoshida04IC,DBLP:journals/jlp/DemangeonHS10,Venet98SAS}, most of them have been rather theoretical,
and there have been surprisingly little efforts in developing  fully automated termination
verification methods and tools based on them. To our knowledge,
Kobayashi's \typical{}~\cite{TyPiCal,KobayashiHybrid} is the only exception that
can prove termination of \(\pi\)-calculus processes (extended with natural numbers)
fully automatically, but its termination analysis is quite limited (see Section~\ref{sec:relatedwork}).

Our method is based on a reduction to termination analysis for sequential programs:
we translate a \(\pi\)-calculus process \(P\) to a sequential program \(S_P\), so that
if \(S_P\) is terminating, so is \(P\). The reduction allows us to use
powerful, mature methods and tools
for termination analysis of sequential programs~\cite{heizmann2016ultimate,freqterm,DBLP:conf/lics/PodelskiR04,Kuwahara2014Termination,DBLP:journals/cacm/CookPR11}.

The idea of the translation is to convert a chain of communications on replicated input
channels to a chain of recursive function calls of the target sequential program.
Let us consider the following Fibonacci process:
\begin{align*}
    & \rinexpIntro{\fib}{n}{r}
        \ifexp{n<2}{ \soutatom{r}{1} \\ &\quad}
                   { \nuexp{s_1} \nuexp{s_2} (\outatomIntro{\fib}{n-1}{s_1} \PAR \outatomIntro{\fib}{n-2}{s_2} \PAR \sinexp{s_1}{x}\sinexp{s_2}{y}\soutatom{r}{x+y}) \\}
    & \PAR \outatomIntro{\fib}{m}{r}
\end{align*}
Here, the process
$\rinexpIntro{\fib}{n}{r} \ldots$ is a function server that computes the \(n\)-th Fibonacci number
in parallel and returns the result to \(r\),
and $\outatom{\fib}{m}{r}$ sends a request for computing the \(m\)-th Fibonacci number;
those who are not familiar with the syntax of the \(\pi\)-calculus may wish to consult
Section~\ref{sec:targetlanguage} first.
To prove that the process above is terminating for any integer \(m\),
it suffices to show that there is no infinite chain of communications on $\fib$:
\[
    \fib(m,r) \to \fib(m_1,r_1) \to \fib(m_2,r_2) \to \cdots.
\]
We convert the process above to the following program:\footnote{The actual translation
  given later is a little more complex.}
\begin{verbatim}
 let rec fib(n) = if n<2 then () else (fib(n-1) [] fib(n-2)) in
 fib(m)
\end{verbatim}
Here, \texttt{[]} represents the non-deterministic choice.
Note that, although the calculation of Fibonacci numbers is not preserved,
for each chain of communications on \texttt{fib}, there is a corresponding
sequence of recursive calls:
\[
\mathtt{fib}(m) \to \mathtt{fib}(m_1) \to \mathtt{fib}(m_2) \to \cdots.
\]
Thus, the termination of the sequential program above implies the termination of
the original process.
As shown in the example above, (i) each communication on a replicated input channel
is converted to a function call, (ii) each communication on a non-replicated input
channel is just removed (or, in the actual translation, replaced by a call of
a trivial function defined by \(f(\seq{x})=(\,)\)), and (iii) parallel composition
is replaced by a non-deterministic choice.
We formalize the translation outlined above and prove its correctness.

The basic translation sketched above sometimes loses too much information.
For example, consider the following process:
\begin{align*}
    & \rinexpIntro{\pre}{n}{r} \soutatom{r}{n-1} \\
    & \PAR \rinexpIntro{f}{n}{r} \ifexp{n<0}{ \soutatom{r}{1} }
                                       { \nuexp{s} (\outatomIntro{\pre}{n}{s} \PAR \sinexp{s}{x}\outatomIntro{f}{x}{r}) } \\
    & \PAR \outatomIntro{f}{m}{r}
\end{align*}
The translation sketched above would yield:
\begin{verbatim}
  let pred(n) = n-1 in
  let rec f(n) = if n<0 then () else (pred(n) [] f(*)) in
  f(m)
\end{verbatim}
Here, \texttt{*} represents a non-deterministic integer: since we have removed
the input $\sinatom{s}{x}$, we do not have information about the value of \( x \).
As a result, the sequential program above is non-terminating, although the original
process is terminating.
To remedy this problem, we also refine the basic translation above by using a refinement
type system for the \(\pi\)-calculus. Using the refinement type system,
we can infer that the value of \(x\) in the original process is less than \(n\),
so that we can refine the definition of \texttt{f} to:
\begin{verbatim}
 let rec f(n) = ... else (pred(n) [] let x=* in assume(x<n);f(x))
\end{verbatim}
The target program is now terminating, from which
we can deduce that the original process is also terminating.
We have implemented an automated tool based on the refined translation above.

The contributions of this paper are summarized as follows.
\begin{itemize}
\item The formalization of the basic translation from the \(\pi\)-calculus
  (extended with integers) to sequential programs, and a proof of its correctness.
\item The formalization of a refined translation based on a refinement type system.
\item An implementation of the refined translation, including automated refinement type
  inference based on CHC solving, and experiments to evaluate the effectiveness of
  our method.
\end{itemize}

The rest of this paper is structured as follows.
Section~\ref{sec:targetlanguage} introduces the source and target languages
of our translation.
Section~\ref{sec:approach} 
formalizes the basic translation, and proves its correctness.
Section~\ref{sec:refinement} refines the basic translation by using a refinement type system.
Section~\ref{sec:implementation} reports an implementation and experiments.
Section~\ref{sec:relatedwork} discusses related work,
and Section~\ref{sec:conclusion} concludes the paper.

\section{Model Development Phases and AI System Hardware Life Cycle}
\label{sec:model-life-cycle-analysis}


Figure~\ref{fig:ml_lifecycle} depicts the major development phases for ML --- \textbf{Data Processing}, \textbf{Experimentation}, \textbf{Training}, and \textbf{Inference} (Section~\ref{sec:ml-model-lifecycle}) --- over the life cycle of AI system hardware (Section~\ref{sec:ml-hardware-lifecycle}).
Driven by distinct objectives of AI research and advanced product development, infrastructure is designed and built specifically to maximize data storage and ingestion efficiency for the phase of \textbf{Data Processing}, developer efficiency for the phase of \textbf{Experimentation}, training throughput efficiency for the phase of \textbf{Training}, and tail-latency bounded throughput efficiency for \textbf{Inference}.


\subsection{Machine Learning Model Development Cycle}
\label{sec:ml-model-lifecycle}


% Model development cycle
ML researchers extract features from data during the \textbf {Data Processing} phase and apply weights to individual features based on feature importance to the model optimization objective.
During \textbf{Experimentation}, the researchers design, implement and evaluate the quality of proposed algorithms, model architectures, modeling techniques, and/or training methods for determining model parameters.
This model exploration process is computationally-intensive. A large collection of diverse ML ideas are explored simultaneously at-scale. 
Thus, during this phase, we observe unique system resource requirements from the large pool of training experiments. 
Within \fb's ML research cluster, 50\% (p50) of ML training experiments take up to 1.5 GPU days while 99\% (p99) of the experiments complete within 24 GPU days. There are a number of large-scale, trillion parameter models which require over 500 GPUs days.
% Mean: 1.5 GPU days, P95: 4.3 GPU days, P99: 24 GPU days. 

Once a ML solution is determined as promising, it moves into \textbf{Training} where the ML solution is evaluated using extensive production data --- data that is \textit{more recent}, is \textit{larger in quantity}, and contains \textit{richer features}.
The process often requires additional hyper-parameter tuning. 
Depending on the ML task requirement, the models can be trained/re-trained at different frequencies. For example, models supporting \fb's \textit{Search} service were trained at an hourly cadence whereas the \textit{Language Translation} models were trained weekly~\cite{Hazelwood:hpca:2018}.
A p50 production model training workflow takes 2.96 GPU days while a training workflow at p99 can take up to 125 GPU days.
% XLMG trained on Azure is 94K hours 
% longest training at prod is 42K hours

Finally, for \textbf{Inference}, the best-performing model is deployed, producing trillions of daily predictions to serve billions of users worldwide.
% https://ai.\fb.com/blog/pytorch-builds-the-future-of-ai-and-machine-learning-at-\fb/
%\textbf{Inference}:
%Trained models are further optimized for deployment, where 
The total compute cycles for inference predictions are expected to exceed the corresponding training cycles for the deployed model. 
% to reach the state-of-the-art quality. At \fb, we observe a rough power capacity breakdown of 10:20:70 for AI infrastructures devoted to \textbf{Experimentation}, \textbf{Training}, and \textbf{Inference} [Figure~\ref{fig:ml_lifecycle}(a)].
% To be made available: \footnote{Additional carbon footprint from AI can be incurred from model training in public clouds, such as Google Cloud Platform, AWS, Microsoft Azure.}.
% At \fb, production model inference consumes the largest amount of the total AI power capacity at approximately \textcolor{blue}{70\%} with the remaining \textcolor{blue}{30\%} contributed by \textbf{Experimentation} and \textbf{Training}.

%\begin{figure}[t]
%    \centering
%    \includegraphics[width=\linewidth]{mlsys2022greenai/images/Power Resource 
%    \caption{\textcolor{blue}{Resource requirement for AI model development cycle}}
%    \label{figure:resource-breakdown}
%    \vspace{-0.4cm}
%\end{figure}

%\textbf{ML Model Life Cycle:} 
%From the perspective of an \textit{individual} machine learning task, we start with \textit{\underline{data}} from which important \textit{\underline{features}} are extracted. Weights are applied to individual features depending on feature importance to the model optimization objective. At the \textit{\underline{training}} stage is when model exploration takes place --- machine learning algorithms, model architectures, modeling techniques, training algorithms are applied to determine model parameters. Further \textit{\underline{evaluation}} and optimization is performed on the trained model for \textit{\underline{deployment}} to produce predictions for the task. Data used for model inference is logged, forming a virtuous cycle of continual improvement for the machine learning pipeline. 


\subsection{Machine Learning System Life Cycle}
\label{sec:ml-hardware-lifecycle}


%\textbf{System Life Cycle:}
Life Cycle Analysis (LCA) is a common methodology to assess the carbon emissions over the product life cycle. There are four major phases: \textit{manufacturing}, \textit{transport}, \textit{product use}, and \textit{recycling}\footnote{Recycling is an important domain, for which the industry is developing a circular economy model to up-cycle system components --- design with recycling in mind.}. From the perspective of AI's carbon footprint analysis, \textit{\underline{manufacturing}} and \textit{\underline{product use}} are the focus. Thus, in this work, we consider the overall carbon footprint of AI by including \textit{manufacturing} --- carbon emissions from building %domain-specific 
infrastructures specifically for AI (i.e., \textit{embodied carbon footprint}) and \textit{product use} --- carbon emissions from the use of AI (i.e., \textit{operational carbon footprint}).

%There are two major sources contributing to the overall carbon footprint of AI computing: \textit{embodied} energy consumption required from the manufacturing of AI system hardware, such as CPUs, GPUs, TPUs, or Application Specific Integrated Circuit (ASIC) specialized hardware, and \textit{operational} energy consumption from the aforementioned model development phases. 
While quantifying the exact breakdown between operational and embodied carbon footprint is a complex process, we estimate the significance of embodied carbon emissions using \fb’s Greenhouse Gas (GHG) emission statistics\footnote{Facebook  Sustainability Data: \url{https://sustainability.fb.com/report/2020-sustainability-report/}.}. 
\textit{In this case, more than 50\% of \fb’s emissions owe to its value chain --- Scope 3 of \fb's GHG emission}. As a result, a significant embodied carbon cost is paid upfront for every system component brought into \fb's fleet of datacenters, where AI is the biggest growth driver.

\section{AI Computing's Carbon Footprint}
\label{sec:ai-carbon-footprint}


\begin{figure}[t]
    \centering
    \includegraphics[width=\linewidth]{images/cfcharacterizationv8.pdf}
    %\vspace{-0.5cm}
    \caption{The carbon footprint of the LM model is dominated by Inference whereas, for RM1 -- RM5, the carbon footprint of Training versus Inference is roughly equal. The average carbon footprint for ML training tasks at Facebook is 1.8 times larger than that of Meena used in modern conversational agents and 0.3 times of GPT-3's carbon footprint. Carbon footprint for inference tasks is included for models that are used in production. Note: the operational carbon footprint of AI does not correlate with the number of model parameters. The OSS large-scale ML tasks are based on the vanilla model architectures from~\cite{Patterson:arxiv:2021} and may not be reflective of production use cases.}
    \label{figure:cf-characterization}
    %\vspace{-0.25cm}
\end{figure}



\subsection{Carbon Footprint Analysis for Industry-Scale ML Training and Deployment}


% In this section, we perform a carbon footprint analysis for industry-scale machine learning model training and deployment use cases at \fb.


Figure~\ref{figure:cf-characterization} illustrates the operational carbon emissions for model training and inference across the ML tasks. 
We analyze six representative machine learning models in production at \fb\footnote{In total, the six models account for a vast majority of compute resources for the overall inference predictions at \fb, serving billions of users world wide.}.
\textbf{LM} refers to \fb's Transformer-based Universal Language Model for text translation~\cite{XLM-r}.
\textbf{RM1} -- \textbf{RM5} represent five unique deep learning recommendation and ranking models for various Facebook  products~\cite{Naumov:arxiv:2019,Gupta:hpca:2020}. 

 We compare the carbon footprint of \fb's production ML models with seven large-scale, open-source (OSS) models: BERT-NAS, T5, Meena, GShard-600B, Switch Transformer, and GPT-3. 
 Note, we present the operational carbon footprint of the OSS model training from~\cite{Strubell:arxiv:2019,Patterson:arxiv:2021}. The operational carbon footprint results can vary based on the exact AI systems used and the carbon intensity of the energy mixture. Models with more parameters do not necessarily result in longer training time nor higher carbon emissions. Training the Switch Transformer model equipped with 1.5 trillion parameters~\cite{Fedus:switch-transformer:2021} produces significantly less carbon emission than that of GPT-3 (750 billion parameters)~\cite{brown:arxiv:2020}. This illustrates the carbon footprint advantage of operationally-efficient model architectures.
 %We compare the carbon footprint of important \fb ML tasks with other open-source (OSS) large scale ML models~\cite{Strubell:arxiv:2019,Patterson:arxiv:2021}: BERT-NAS, T5, Meena, GShard-600B, Switch Transformer, and GPT-3. We followed a methodology similar to the ones outlined in the above papers:  We use measured average power for host types to compute overall power consumed, we compute carbon intensity in a location-based manner based on the energy mix, and assume a PUE of 1.1  for \fb data centers\footnote{A significant portion of BERT-NAS carbon footprint is from NAS that may not be considered in other model training. Based on \fb's renewable energy and sustainability programs~\cite{\fb-sustainability-report}, the operational carbon footprint of \fb AI computing has been completely neutralized. The operational carbon footprint for the \textit{OSS Large-Scale ML Models} is from~\cite{Strubell:arxiv:2019,Patterson:arxiv:2021} and can vary depending on specifics of the AI systems.}.
 
 \begin{figure}[t]
    \centering
    \includegraphics[width=\linewidth]{images/Embodied-Operational-CF-v1.pdf}
    %\vspace{-0.5cm}
    \caption{When considering the overall life cycle of ML models and systems in this analysis, manufacturing carbon cost is roughly 50\% of the (location-based) operational carbon footprint of large-scale ML tasks (Figure~\ref{figure:cf-characterization}). Taking into account carbon-free energy, such as solar, the operational energy consumption can be significantly reduced, leaving the manufacturing carbon cost as the dominating source of AI's carbon footprint.}
    \label{figure:ops-vs-embodied}
    %\vspace{-0.25cm}
\end{figure}

\vspace{+0.2cm}
\noindent{\textit{Both \textbf{Training} and \textbf{Inference} can contribute significantly to the overall carbon footprint of machine learning tasks at \fb.
The exact breakdown between the two phases varies across ML use cases.}}
\vspace{+0.2cm}

The overall operational carbon footprint is categorized into \textit{offline training}, \textit{online training}, and \textit{inference}. 
Offline training encompasses both experimentation and training models with historical data.
Online training is particularly relevant to recommendation models where parameters are continuously updated based on recent data.
The inference footprint represents the emission from serving production traffic.
The online training and inference emissions are considered over the period of offline training.
 For recommendation use cases, we find the carbon footprint is split evenly between training and inference. On the other hand, the carbon footprint of LM is dominated by the inference phase, using much higher inference resources (65\%) as compared to training (35\%).
  %\textcolor{blue}{For recommendation use cases we find the carbon footprint is split evenly between training and inference. On the other hand,  for XLM-R use cases the breakdown varies across the model scales.}



 %We categorize the overall carbon footprint into \textit{offline training} which includes experimentation, \textit{online training} which updates model parameters continuously with latest data (relevant for ranking models), and \textit{inference} where the models are running inference at production-scale traffic. The online training and inference emissions are normalized by the offline training duration  for each model. 
 
 %We compare the carbon footprint of important \fb ML tasks with other open-source (OSS) large scale ML models~\cite{Strubell:arxiv:2019,Patterson:arxiv:2021}: BERT-NAS, T5, Meena, GShard-600B, Switch Transformer, and GPT-3. We followed a methodology similar to the ones outlined in the above papers:  We use measured average power for host types to compute overall power consumed, we compute carbon intensity in a location-based manner based on the energy mix, and assume a PUE of 1.1  for \fb data centers\footnote{A significant portion of BERT-NAS carbon footprint is from NAS that may not be considered in other model training. Based on \fb's renewable energy and sustainability programs~\cite{\fb-sustainability-report}, the operational carbon footprint of \fb AI computing has been completely neutralized. The operational carbon footprint for the \textit{OSS Large-Scale ML Models} is from~\cite{Strubell:arxiv:2019,Patterson:arxiv:2021} and can vary depending on specifics of the AI systems.}.
%\footnote{https://engineering.fb.com/2014/03/14/data-center-engineering/open-sourcing-pue-wue-dashboards/}

\vspace{+0.2cm}
\noindent{\textit{Both \textbf{operational} and \textbf{embodied carbon emissions} can contribute significantly to the overall footprint of ML tasks}.}
\vspace{+0.2cm}

\textbf{Operational Carbon Footprint:} 
Across the life cycle of the Facebook  models shown in Figure~\ref{figure:cf-characterization}, the average carbon footprint is 1.8$\times$ higher than that of the open-source Meena model~\cite{google-meena} and one-third of GPT-3's training footprint.
%The average carbon footprint for large scale machine learning tasks at \fb is 1.6 times larger than that of the open-source large scale ML model training for Meena used in modern conversational agents~\cite{google-meena} and 0.3 times of GPT-3's carbon footprint.
%We consider the overall life cycle of ML models and systems in this analysis.
To quantify the emissions of \fb's models we measure the total energy consumed, assume location-based carbon intensities for energy mixes,\footnote{Renewable energy and sustainability programs of \fb~\cite{facebook-sustainability-report}.} and use a data center Power Usage Effectiveness (PUE) of 1.1. 
%The operational carbon footprint is a product of the operational energy consumption and carbon intensity of the geographical location where data centers are located for model training and deployment.
% DLRM training: 344 g CO2/kWh; DLRM serving: 280 g CO2/kWh
% OSS workloads: US average mix is 0.429 kg of CO 2 e/KWh
In addition to model-level and hardware-level optimizations, \fb's renewable energy procurement~\cite{facebook-sustainability-report} programs mitigates these emissions. 
%based on \fb's renewable energy procurement programs~\cite{\fb-sustainability-report}, this operational carbon footprint has been completely neutralized. 



\textbf{Embodied Carbon Footprint:} To quantify the embodied carbon footprint of AI hardware, we use LCA (Section~\ref{sec:ml-hardware-lifecycle}). 
% with similar compute, memory, and storage capabilities.
We assume GPU-based AI training systems have similar embodied footprint
as the production footprint of Apple's 28-core CPU with dual AMD Radeon GPUs (2000kg CO$_2$)~\cite{appleMacProMax}.
For CPU-only systems, we assume half the embodied emissions.
Based on the characterization of model training and inference at \fb, we assume an average utilization of 30-60\% over the 3- to 5-year lifetime for servers.
%We approximate the embodied carbon footprint from the manufacturing footprint of system hardware based on the life cycle analysis of large-scale Apple desktop with 28-core CPU and dual AMD Radeon GPUs~\cite{}. 
%We assume the server operates at 30\% of the potential maximum utilization for 24 hours a day, 365 days a year, and for 3 years.
Figure~\ref{figure:ops-vs-embodied} presents the overall carbon footprint for the large scale ML tasks at \fb, spanning both operational and embodied carbon footprint. Based on the assumptions of location-based renewable energy availability, the split between the embodied and (location-based) operational carbon footprint is roughly 30\% / 70\% for the large scale ML tasks. Taking into account carbon-free energy, such as solar, the operational carbon footprint can be significantly reduced, leaving the manufacturing carbon cost as the dominating source of AI's carbon footprint.


%\textit{Operational carbon footprint dominates the overall carbon footprint of model training and inference for XLM-R while embodied carbon footprint plays a more significant role for recommendation and ranking tasks (DLRM1 -- DLRM5)}. 
%Transformer model training can more effectively leverage hardware parallelism in GPU accelerators. This translates into higher resource utilization. 
%On the other hand, \textcolor{blue}{DLRM-X} exhibits much lower resource utilization for training. This is when embodied carbon footprint starts dominating the overall carbon footprint. 


%FOr each model, emissions are broken down into offline training (including experimentation), online training (for ranking models that are continuously retrained), and inference costs. For comparison, we show the carbon emissions reported by other OSS machine learning tasks. 

\begin{figure}[t]
    \centering
    \includegraphics[width=1\linewidth]{images/Optimizationbreakdown.pdf}
    %\vspace{-0.5cm}
    %mlsys2022greenai/images/CO2-breakdown.pdf}
    \caption{Optimization is an iterative process — we have achieved an average of 20\% operational energy footprint reduction every 6 months across the machine learning hardware-software stack.}  
    % In total, this has led to 25.8\% operational energy footprint reduction over a two-year time period.
    %Carbon footprint improvement from state-of-the-art machine learning accelerators (M times), at-scale infrastructure optimization (N times), machine learning platform optimization (O times), and algorithmic efficiency improvement (P times).} \textcolor{blue}{add another figure to show the accumulated CO2 reduction benefit from H1-Yr1 to ....}}
    \label{fig:hw-sw-optimization}
    %\vspace{-0.25cm}
\end{figure}


\subsection{Carbon Footprint Optimization from Hardware-Software Co-Design}
\label{sec:hw-sw-optimization}


%In this section, we review optimization techniques that enable the lower carbon footprint at the granularity of individual machine learning models. 

%\subsubsection{Optimization across AI Model Development and System Stack over Time}
%\label{sec:continuous-optimization}

% Figure 7: Overview showing we need optimization across ML, platform, HW, Infrastructure. The amount of benefit from each will be application specific. 

\vspace{+0.2cm}
\noindent{\textit{Optimization is an iterative process --- we reduce the power footprint across the machine learning hardware-software stack by ~20\% every 6 months. But at the same time, AI infrastructure continued to scale out. The net effect, with Jevon's Paradox, is a 28.5\% operational power footprint reduction over two years (Figure~\ref{fig:jevon-paradox}).}}
\vspace{+0.2cm}

\if 0 
\textit{Optimization is an iterative process --- we achieve an average of 20\% operational energy footprint reduction every 6 months across the machine learning hardware-software stack. However, over the same time period, AI infrastructure capacity continued to expand. The net effect,with Jevon's Paradox, led to the 28.5\% operational energy footprint reduction over two years (Figure~\ref{fig:jevon-paradox}).}
\fi

\textbf{Optimization across AI Model Development and System Stack over Time:}
Figure~\ref{fig:hw-sw-optimization} shows the operational power footprint reduction across \fb's AI fleet over two years. 
The improvement come from four areas of optimizations: \textit{model} (e.g., designing resource-efficient models), \textit{platform} (e.g., PyTorch's support for quantization), \textit{infrastructure} (e.g., data center optimization and low-precision hardware), and \textit{hardware} (e.g., domain-specific acceleration).
Each bar illustrates the operational power reduction across \fb's AI fleet over 6-month period from each of the optimization areas.
The optimizations in aggregate provide, on average, a 20\% reduction in operational power consumption every six months. 
The compounded benefits highlight the need for cross-stack optimizations. 
%Figure~\ref{fig:hw-sw-optimization} shows the carbon footprint reduction potential when optimization techniques are pursued for ML use cases in production using model-level optimization (Model), platform level optimization (Platform), such as PyTorch quantization support, at-scale infrastructure optimization (Infrastructure), and more efficient system hardware (Hardware) over the two-year time frame at \fb. 
%Each bar shows the operational carbon footprint reduction obtained across \fb's AI fleet over a 6-month period, broken down by the contributions coming from each layer of the optimization stack. 

% Hardware optimization improvement comes from servers and ML accelerators with higher performance-per-Watt energy efficiency. Infrastructure optimizations refer to techniques that improve utilization of the large scale training and serving infrastructures and at-scale infrastructure resource management. Platform efficiency refers to efficiency improvement from deep learning frameworks, such as high-performance PyTorch operators, memory-friendly data layout optimization. Model efficiency refers to algorithmic optimizations that reduce the overall computation requirement for a machine learning task by using resource-efficient modeling technique or achieving model consolidation in favor of foundational models~\cite{}. Here we use the operational energy footprint as a proxy for carbon footprint improvement achieved at the scale of \fb over the two-year time period. 


\begin{figure}[t]
    \centering
    \includegraphics[width=\linewidth]{images/HW-SW-TextRay.pdf}
    %\vspace{-0.5cm}
    \caption{For the cross-lingual ML task (LM), the operational energy footprint can be significantly reduced by more than $800\times$ using \textit{platform-level caching}, \textit{GPUs}, \textit{low precision data format}, and \textit{additional algorithmic optimization}.}
    %\textcolor{red}{Udit: It may be interesting to add in the impacts of renewable energy procurement as well (going from 300g CO2 per kWh down to around 40 g CO2 per kWh with solar).}}
    \label{fig:textray-optimization}
    %\vspace{-0.25cm}
\end{figure}


%\subsubsection{Case Study: Optimizing XLM-R's Carbon Footprint}
%\label{sec:XLM-R-carbon-footprint}
\textbf{Optimizing the Carbon Footprint of \mbox{LMs}:} 
We dive into a specific machine learning task at \fb: language translation using a Transformer-based architecture (\mbox{LM}).
\mbox{LM} is designed based on the state-of-the-art cross-lingual understanding through self-supervision. 
%using 24 layers and 600 million parameters. 
Figure~\ref{fig:textray-optimization} analyzes the power footprint improvements over a collection of optimization steps for \mbox{LM}: \textit{platform-level caching}, \textit{GPU acceleration}, \textit{low precision format on accelerator}, and \textit{model optimization}. 
In aggregate the optimizations reduce the infrastructure resources required to serve \mbox{LM} at scale by over 800$\times$.
We outline the optimization benefits from each area below.
%\textcolor{blue}{Describe the optimization steps that lead to an overall of more than 800 times resource reduction in carbon footprint.}
%With traditional chip-multiprocessor server based inference, achieving the target latency-bounded queries-per-second throughput would have required a fleet of CPU servers at a power budget --- using $800\times$ more resources than what is required in an optimal setting.
\begin{itemize}
    \item \textbf{Platform-Level Caching.} Starting with a CPU server baseline, application-level caching improves power efficiency by 6.7$\times$. These improvements are a result of pre-computing and caching frequently accessed embeddings for language translation tasks. Using DRAM and Flash storage devices as caches, these pre-computed embeddings can be shared across applications and use cases.
    \item \textbf{GPU acceleration.} In addition to caching, deploying LM across GPU-based specialized AI hardware unlocks an additional 10.1$\times$ energy efficiency improvement.

    \item \textbf{Algorithmic optimization.} Finally, algorithmic optimizations provide an additional 12$\times$ energy efficiency reduction. Halving precision (e.g., going from 32-bit to 16-bit operations) provides a 2.4$\times$ energy efficiency improvement on GPUs. Another 5$\times$ energy efficiency gain can be achieved by using custom operators to schedule encoding steps within a single kernel of the Transformer module, such as~\cite{faster-transformer}. 

\end{itemize}


%Pre-computing embeddings by caching frequently 

%By storing computed results in the DRAM and flash memory, we achieve approximately $6.69\times$ higher power efficiency. With an additional L0 cache on each CPU host to cache the most frequently-accessed embedding portions can further boost inference efficiency. Caching at the application level is plausible because our model produces embeddings that are repeatedly consumed by different down-stream applications over the course of several days. By analyzing down-stream traffic, we are able to identify optimal caching strategies and more effectively retain computed embeddings in a storage hierarchy of DRAM and Flash. The computed embeddings can be shared across different applications over time.


%\textit{Platform-Level Caching.} By storing computed results in the DRAM and flash memory, we achieve approximately $6.69\times$ higher power efficiency. With an additional L0 cache on each CPU host to cache the most frequently-accessed embedding portions can further boost inference efficiency. Caching at the application level is plausible because our model produces embeddings that are repeatedly consumed by different down-stream applications over the course of several days. By analyzing down-stream traffic, we are able to identify optimal caching strategies and more effectively retain computed embeddings in a storage hierarchy of DRAM and Flash. The computed embeddings can be shared across different applications over time.



%\textit{GPUs and Algorithmic Optimization.} With GPUs, we can unlock an additional $10.1\times$ and $24.2\times$ times energy efficiency improvement using 32-bit and 16-bit data formats, respectively. Platform-level caching optimization on CPUs and acceleration using GPUs in combination reduces the carbon cost of serving large-scale language models by $162\times$.
%Another 5 times energy efficiency gain can be achieved by 
%optimizing the implementation of the Transformer encoder operator, 
%using a custom operator to schedule the encoding steps performed by the Transformer module in a single kernel, and optimizes padding across batch element boundaries~\cite{faster-transformer}.

% Based on the average miles driven in the US (about 12,500km), this corresponds to the yearly power consumption of about 20,500 Tesla Model 3. 
%Despite the significantly reduced environmental footprint achieved through hardware-software co-design, we see a variety of XLM-R models in production, spanning from commerce to digital assistant use cases. Taking into account the scale, AI's environmental footprint could add up quickly without these optimizations across the application, hardware selection, representation and algorithmic optimizations. 

%This is great but we need to continue optimizing carbon footprint. 
%One route is efficiency at-scale optimizations which \fb has been engaging with very heavily.
%Show utilization case study. 
%This is great! But we need to go beyond efficiency as well… (leads into Challenges, and Optimizaitons Section --- Section 4)


\textbf{Optimizing the Carbon Footprint of \mbox{RMs}:}
The LM analysis is used as an example to highlight the optimization opportunities available with judicious cross-stack, hardware/software optimization. In addition to optimizing the carbon footprint for the language translation task,
%In Appendix~\ref{sec:DLRM-carbon-footprint}, 
we describe additional optimization techniques tailored for ranking and recommendation use cases. 

%We use XLM-R as an example to highlight the potential of carbon footprint reduction that can be achieved through careful cross-stack optimization, AI accelerators, and hardware-software co-design and optimization. In Appendix~\ref{sec:DLRM-carbon-footprint}, we describe additional optimization techniques tailored for the DLRM tasks. 
%When applied to all ML models at scale, we can make step-function carbon footprint reduction.
%, reducing both operational and embodied carbon cost of AI. %In fact, the scale of \fb data center fleets and workloads offers additional at-scale optimization opportunities. 


A major infrastructure challenge faced by deep learning RM training and deployment (\textbf{RM1} -- \textbf{RM5}) is the fast-rising memory capacity and bandwidth demands (Figure~\ref{fig:data-model-system-growth}). There are two primary sub-nets in a RM: the dense fully-connected (FC) network and the sparse embedding-based network. The FC network is constructed with multi-layer perceptions (MLPs), thus computationally-intensive. The embedding network is used to project hundreds of sparse, high-dimensional features to low-dimension vectors. It can easily contribute to over 95\% of the total model size. For a number of important recommendation and ranking use cases, the embedding operation dominates the inference execution time~\cite{Gupta:hpca:2020,Ke:isca:2020}.

To tackle the significant memory capacity and bandwidth requirement, we deploy model quantization for RMs~\cite{deng:ieee-micro-2021}. 
Quantization offers two primary efficiency benefits: the low-precision data representation reduces the amount of computation requirement and, at the same time, lowers the overall memory capacity need. 
By converting 32-bit floating-point numerical representation to 16-bit, we can reduce the overall RM2 model size by 15\%. This has led to 20.7\% reduction in memory bandwidth consumption.
Furthermore, the memory capacity reduction enabled by quantization unblocks novel systems with lower on-chip memory. For example, for RM1, quantization has enabled RM deployment on highly power-efficient systems with smaller on-chip memory, leading to an end-to-end inference latency improvement of 2.5 times. 


\subsection{Machine Learning Infrastructures at Scale}

\textbf{ML Accelerators:}
GPUs are the de-facto training accelerators at \fb, contributing to significant power capacity investment in the context of \fb's fleet of datacenters. However, GPUs can be severely under-utilized during both the ML Experimentation and Training phases (Figure~\ref{fig:gpu-utilization})~\cite{Wesolowski:ieee-micro:2021}. To amortize the upfront embodied carbon cost of every accelerator deployed into \fb’s datacenters, maximizing accelerator utilization is a must. 
% From the perspective of environmental sustainability, ML accelerators such as GPUs offer significantly higher training throughput performance per unit carbon footprint cost. Taking ResNet-50 as an example, GPUs, such as NVIDIA v100s, offer \textcolor{blue}{M} times training throughput per unit carbon footprint as compared with conventional CPUs, such as \textcolor{blue}{Y}. 
\begin{figure}[t]
    \centering
    \includegraphics[width=\linewidth]{images/CarbonFootprintGrowthofAI.pdf}
    %\vspace{-0.5cm}
    \caption{The iterative optimization process has led to 28.5\% operational energy footprint reduction over the two-year time period (Section~\ref{sec:hw-sw-optimization}). Despite the significant operational power footprint reduction, we continue to see the overall electricity demand for AI to increase over time --- an example of \textit{Jevon's Paradox}, where efficiency improvement stimulates additional novel AI use cases.}
    \label{fig:jevon-paradox}
    %\vspace{-0.25cm}
\end{figure}

\textbf{Efficiency of Scale:} The higher throughput performance density achieved with ML accelerators reduces the total number of processors deployed into datacenter racks. This leads to more effective amortization of shared infrastructure overheads. Furthermore, datacenter capacity is not only limited by physical space but also power capacity --- higher operational power efficiency directly reduces the inherited carbon cost from manufacturing of IT infrastructures and datacenter buildings. 

%In Appendix~\ref{sec:at-scale-optimization-fb-appendix}, we further discuss optimization opportunities for \fb's fleet of datacenters. 

\textbf{At-Scale Efficiency Optimization for Facebook Data Centers:}
Servers in Facebook data center fleets are customized for internal workloads only --- machine learning tasks~\cite{Hazelwood:hpca:2018} or not~\cite{Sriraman:isca:2019,Sriraman:asplos:2020}. Compared to public cloud providers, this puts Facebook at a unique position for at-scale resource management design and optimization. First, \fb customizes server SKUs --- compute, memcached, storage tiers and ML accelerators --- to maximize performance and power efficiency. Achieving a Power Usage Effectiveness (PUE) of about 1.10, \fb's data centers are about 40\% more efficient than small-scale, typical data centers. 
% (\url{https://sustainability.fb.com/report-page/data-centers/}).

Furthermore, the large-scale deployment of servers of different types provides an opportunity to build performance measurement and optimization tools to ensure high utilization of the underlying infrastructure. For data center fleets in different geographical regions where the actual server utilization exhibits a diurnal pattern, Auto-Scaling frees the over-provisioned capacity during off-peak hours, by up to 25\% of the web tier’s machines~\cite{Tang:osdi:2020}. By doing so, it provides opportunistic server capacity for others to use, including offline ML training. Furthermore, static power consumption plays a non-trivial role in the context of the overall data center electricity footprint. This motivates more effective processor idle state management.


\textbf{Carbon-Free Energy:} Finally, over the past years, \fb has invested in carbon free energy sources to neutralize its operational carbon footprint~\cite{facebook-sustainability-report}. 
Reaching net zero emissions entails matching every unit of energy consumed by data centers with 100\% renewable energy purchased by \fb. Remaining emissions are offset with various sustainability programs, further reducing the operational carbon footprint of AI computing at \fb. As Section~\ref{sec:systems} will later show, \textit{more can be done}.  



\subsection{Going Beyond Efficiency Optimization}

Despite the opportunities for optimizing energy efficiency and reducing environmental footprint at scale, there are many reasons why we must care about scaling AI in a more environmentally-sustainable manner. AI growth is multiplicative beyond current industrial use cases. Although domain-specific architectures improve the operational energy footprint of AI model training by more than 90\%~\cite{Patterson:arxiv:2021}, these architectures require more system resources, leading to larger embodied carbon footprints.


\begin{figure}[t]
    \centering
    \includegraphics[width=\linewidth]{images/ChasingCarbon-projection.pdf}
    \vspace{-0.5cm}
    \caption{As accelerator utilization improves over time, both operational and embodied carbon footprints of AI improve. Carbon-free energy helps reduce the operational carbon footprint, making embodied carbon cost the dominating factor. 
    To reduce the rising carbon footprint of AI computing at-scale, we must complement efficiency and utilization optimization with novel approaches to reduce the remaining embodied carbon footprint of AI systems.}
    \label{figure:server-utilization}
    \vspace{-0.25cm}
\end{figure}
While shifting model training and inference to data centers with carbon-free energy sources can reduce emissions, the solution may not scale to all AI use cases.
Infrastructure for carbon free energy is limited by rare metals and materials, and takes significant economic resources and time to build. 
Furthermore, the carbon footprint of federated learning and optimization use cases at the edge is estimated to be similar to that of training a Transformer Big model (Figure~\ref{fig:fl_carbon}). As on-device learning becomes more ubiquitously adopted to improve data privacy, we expect to see more computation being shifted away from data centers to the edge, where access to renewable energy may be limited. The edge-cloud space for AI poses interesting design opportunities (Section~\ref{sec:systems}). 

%In addition to carbon reduction potential of further efficiency improvements at-scale, we must also look beyond efficiency optimization to tackle AI's growing carbon footprint.
%In particular, we find solely optimizing efficiency provides limited carbon reduction potential compared to the exponentially rising AI data set size, model size, and infrastructure capacity. 

\textit{The growth of AI in all dimensions outpaces the efficiency improvement at-scale.}
Figure~\ref{figure:server-utilization} illustrates that, as GPU utilization is improved (x-axis) for LM training on GPUs, both embodied and operational carbon emissions will reduce. Increasing GPU utilization up to 80\%, the overall carbon footprint decreases by 3$\times$. 
Powering AI services with renewable energy sources 
%as compared with using the average energy mix in the United States 
can further reduce the overall carbon footprint by a factor of 2. Embodied carbon cost becomes the dominating source of AI's overall carbon footprint. To curb the rising carbon footprint of AI computing at-scale (Figure~\ref{fig:jevon-paradox} and Figure~\ref{figure:server-utilization}), \textit{we must look beyond efficiency optimization and complement efficiency and utilization optimization with efforts to tackle the remaining embodied carbon footprint of AI systems.}

% The analysis focuses on GPU-based and CPU-based AI training on top and bottom, respectively. 
% We assume the GPU-based server has comparable embodied emissions as Apple's desktop with 28-core CPU and dual AMD Radeon GPU's~\cite{}; embodied carbon footprint for the CPU-based configuration is scaled based on the hardware compute, memory, and storage capacity.
% Based on characterizations of \fb's GPU-based and CPU-based training workloads we find the baseline utilization to be 33\% and 20\%, respectively.



\section{A Sustainability Mindset for AI}
\label{sec:optimization-opportunities}

%\begin{figure}[t]
%    \centering
%    \includegraphics[width=1.0\linewidth]{mlsys2022greenai/images/Optimization 
%    \vspace{-0.8cm}
%    \caption{Across the ML development cycle, the solution space for designing AI technologies in an environmentally-responsible manner is wide open --- there is a general under-investment in the \textit{AI algorithm and data efficiency} space (Sections~\ref{sec:data} and~\ref{sec:model}) beyond system design and optimization (Section~\ref{sec:systems}).}
%    \vspace{-0.4cm}
%    \label{fig:optimization-opportunity}
%\end{figure}

To tackle the environmental implications of AI's exponential growth (Figure~\ref{fig:data-model-system-growth}), the first key step requires ML practitioners and researchers to develop and adopt an \textit{sustainability mindset}. The solution space is wide open---while there are significant efforts looking at \textit{AI system and infrastructure efficiency} optimization, the \textit{AI data, experimentation, and training algorithm efficiency} space (Sections~\ref{sec:data}  and~\ref{sec:experimentation-training}) beyond system design and optimization (Section~\ref{sec:systems}) is less well explored. 
%Navigating the large experimentation space for AI efficiently can translate into noticeable environmental footprint reduction. 
We cannot optimize what cannot be measured --- telemetry to track the carbon footprint of AI technologies must be adopted by the community (Section~\ref{sec:metrics}). 
We synthesize a number of important directions to scale AI in a \textit{sustainable} manner and to minimize the environmental impact of AI for the next decades.

%\subsection{An Efficiency Mindset for AI}
%\label{sec:efficiiency-mindset}
The field of AI is currently primarily driven by research that seeks to maximize model accuracy --- \textit{progress} is often used synonymously with improved prediction quality. This endless pursuit of higher accuracy over the decade of AI research has significant implications in computational resource requirement and environmental footprint.
% achieved through the use of massive computational power while disregarding any associated resources or environmental footprint.
To develop AI technologies responsibly, \textit{we must achieve competitive model accuracy at a fixed or even reduced computational and environmental cost}. 
Despite the recent calls-to-action~\cite{Strubell:arxiv:2019,Lacoste:arxiv:2019,Henderson:arxiv:2020,Bender:facct:2021,Patterson:arxiv:2021}, the overall community remains under-invested in research that aims at deeply understanding and minimizing the cost of AI. We conjecture the factors that may have contributed to the current state in Appendix~\ref{sec:appendix-efficiiency-mindset}.
% There are several factors that may have contributed to the current state:
%%%%%% APPENDIX %%%%%%
%\begin{itemize}
%\vspace{-0.35cm}
%\setlength\itemsep{0em}
%    \item {\bf Lack of incentives:} Over 90\% of the ML publications only focus on model accuracy improvements at the expense of efficiency~\cite{Schwartz:arxiv:2019}. Challenges\footnote{Efficient Open-Domain Question Answering (\url{https://efficientqa.github.io/}), SustaiNLP: Simple and Efficient Natural Language Processing  (\url{https://sites.google.com/view/sustainlp2020/home}), and WMT: Machine Translation Efficiency Task (\url{http://www.statmt.org/wmt21/efficiency-task.html}).} incentivize investment into efficient approaches.
%    \item {\bf Lack of common tools:} There is no standard telemetry in place to provide accurate, reliable energy and carbon footprint measurement. The measurement methodology is complex --- factors, such as datacenter infrastructures, hardware architectures, energy sources, can perturb the final measure easily.
%    \item {\bf Lack of normalization factors:} Algorithmic progress in ML is often presented in some measure of model accuracy, e.g., BLEU, points, ELO, cross-entropy loss, but without considering resource requirement as a normalization factor, e.g., the number of CPU/GPU/TPU hours used, the overall energy consumption and/or carbon footprint required.   
%    \item {\bf Platform fragmentation:} Implementation details can have a significant impact on real-world efficiency, but best practices remain elusive and platform fragmentation prevents performance and efficiency portability across model development.
%    \vspace{-0.35cm}
%\end{itemize}
%%%%%% APPENDIX %%%%%%
To bend the exponential growth curve of AI and its environmental footprint, we must build a future where efficiency is an evaluation criterion for publishing ML research on computationally-intensive models beyond accuracy-related measures. 

%and have an option to focus on advancing the state of AI through data, algorithmic, and hardware-software design efficiency improvement.

%We need to develop AI research that yields novel results while taking into account their computational (and thus carbon) cost, encouraging a reduction in resources spent. Despite a few calls to action largely coming from academia % [_Str19_ (https://arxiv.org/pdf/1906.02243.pdf),_Sch20_ (https://arxiv.org/pdf/1910.09700.pdf),_Hen20_ (https://jmlr.org/papers/volume21/20-312/20-312.pdf), _Bender21_ (https://dl.acm.org/doi/10.1145/3442188.3445922),_Pat21_ (https://arxiv.org/pdf/2104.10350.pdf)],
%the community remains underinvested in research aimed at deeply understanding and reducing the cost of AI. There are several factors that contribute to this state of affairs including 1) Lack of uniform measure of efficiency  and algorithmic progress within various AI disciplines,  2) Lack of common tools to collect and measure energy and CO2 across heterogeneous hardware and platforms, 3) Lack of incentives to focus on measures of performance other than accuracy %[_Schwartz19_ (https://arxiv.org/pdf/1907.10597.pdf)]. 

%We can build a future where efficiency would be an evaluation criterion for publishing ML research on computationally intensive models besides accuracy and related measures, thus encouraging advances across the board (as the most sustainable energy is the energy you don’t use.)

%There is a huge opportunity for \fb to be a thought leader in advancing AI research in a cost-efficient manner. We can build a future where efficiency would be an evaluation criterion for publishing AI research on computationally intensive models besides accuracy and related measures, thus encouraging advances across the board (as the most sustainable energy is the energy you don’t use).  When measures of efficiency are widely accepted as important evaluation metrics alongside accuracy for AI, then practitioners will have the option of focusing on the efficiency of their models leading to a positive impact on the environment. 


\subsection{Data Utilization Efficiency}
\label{sec:data}

\textbf{Data Scaling and Sampling:} \textit{No data is like more data} --- data scaling is the de-facto approach to increase model quality, where the primary factor for accuracy improvement is driven by the size and quality of training data, instead of algorithmic optimization. However, data scaling has significant environmental footprint implications. 
To keep the model training time manageable, overall system resources must be scaled with the increase in the data set size, resulting in larger embodied carbon footprint and operational carbon footprint from the data storage and ingestion pipeline and model training. 
Alternatively, if training system resources are kept fixed, data scaling increases training time, resulting in a larger operational energy footprint. 

When designed well, however, data scaling, sampling and selection strategies can improve the competitive analysis for ML algorithms, reducing the environmental footprint of the process (Appendix~\ref{sec:appendix-data-efficiency}). For instance, Sachdeva et al. demonstrated that intelligent data sampling with merely 10\% of data sub-samples can effectively preserve the relative ranking performance of different recommendation algorithms~\cite{Sachdeva:arxiv:2021}. This ranking performance is achieved with an average of 5.8 times execution time speedup, leading to significant operating carbon footprint reduction.
% Paul:arxiv:2021,Killamsetty:arxiv:2021

\textbf{Data Perishability:} Understanding key characteristics of data is fundamental to efficient data utilization for AI applications. \textit{Not all data is created equal} and data collected over time loses its predictive value gradually. 
Understanding the rate at which data loses its predictive value has strong implications on the resulting carbon footprint.
For example, natural language data sets can lose half of their predictive value in the time period of less than 7 years (the half-life time of data)~\cite{valavi:hbs:2020}. The exact half-life period is a function of context. If we were able to predict the half-life time of data, we can devise effective sampling strategies to subset data at different rates based on its half-life. 
By doing so, the resource requirement for the data storage and ingestion pipeline can be significantly reduced~\cite{Zhao:arxiv:2021} --- lower training time (operational carbon footprint) as well as storage needs (embodied carbon footprint).  


\subsection{Experimentation and Training Efficiency}
\label{sec:experimentation-training}

The experimentation and training phases are closely coupled (Section~\ref{sec:model-life-cycle-analysis}).   
%, in that experimentation typically revolves around changes to the training pipeline (e.g., model architecture, data pre-processing/augmentation, optimizer and hyperparameter search), which then get rolled out during the training phase. 
There is a natural trade-off between the investment in experimentation and the subsequent training cost (Section~\ref{sec:ai-carbon-footprint}).
\textbf{\emph{Neural architecture search} (NAS) and \emph{hyperparameter optimization} (HPO)} are techniques that automate the design space exploration. Despite their capability to discover higher-performing neural networks, NAS and HPO can be extremely resource-intensive, involving training many models, especially when using simple approaches. Strubell et al. show that grid-search NAS can incur over $3000\times$ environmental footprint overhead~\cite{Strubell:arxiv:2019}.
%However, much more sample-efficient NAS and HPO methods exist~\citep{Turner2021bbox,Ren2021NASsurvey}, and utilizing those can translate directly into carbon footprint reduction.
Utilizing much more sample-efficient NAS and HPO methods~\cite{Turner2021bbox,Ren2021NASsurvey} can translate directly into carbon footprint improvement.
% it is important to not only educate ML researchers and practitioners about these improved optimization tools, but also to incorporate methods into the full ML life cycle as part of standard, easy-to-use tool kits. 
In addition to reducing the number of training experiments, one can also reduce the training time of each experiment.
% \textbf{Early stopping:} F
%For many models / hyperparameter configurations, it is evident early on during training that the resulting model will perform poorly. 
By detecting and \emph{stopping under-performing training workflows early}, unnecessary training cycles can be eliminated. 
% \textbf{Check-pointing:} 
%Furthermore, 
%in practice training jobs often fail, e.g. due to bugs in unrelated pipeline code or infrastructure reliability issues. 
%routinely \emph{check-pointing model state} avoids the large cost of re-training ``from scratch''~\cite{Maeng:arxiv:2021,Eisenman:arxiv:2021}.
    
% \textbf{Multi-Objective Optimization:} 
%There are nontrivial trade-offs between model quality and system resource requirement --- minor accuracy tolerance can lead to significant computational savings. 
%However, these trade-offs are not known a priori, and they are often hard to encode in a single scalar objective. 
%Rather than optimizing a pre-defined cost function,  
\textbf{\emph{Multi-objective optimization}} explores the Pareto frontier of efficient model quality and system resource trade-offs. If used early in the model exploration process, it enables more informed decisions about \textit{which} model to train fully and deploy given certain infrastructure capacity. 
% 
Beyond model accuracy and timing performance~\cite{Song:kdd:2020,Joglekar:kdd:2020,Tan:arxiv:2020,eriksson2021latencyNAS}, energy and carbon footprint can be directly incorporated into the cost function as optimization objectives %into the NAS design space 
to enable discovery of environmentally-friendly models. 
Furthermore, when training is decoupled from NAS, sub-networks tailoring to specialized system hardware can be selected \textit{without additional training}~\cite{cai:arxiv:2020,Stamoulis:arxiv:2019,Chen:arxiv:2021,Mellor:arxiv:2021}. Such approaches can significantly reduce the overall training time, however, at the expense of increased embodied carbon footprint.

Developing \textbf{\textit{resource-efficient model architectures}} fundamentally reduce the overall system capacity need of ML tasks.  
% Feeding the demand for continual accuracy improvements, modern DNNs encompass trillions of parameters and memory capacity requirements exceed terabyte scale. 
% Large-scale training infrastructures are being developed from the ground up~\cite{Jouppi:cacm:2020,Mudigere:scaling-training:2021} in conjunction with a variety of new parallel training approaches~\cite{Rajbhandari:zero:2021} in order to sustain the chase of model accuracy. However, 
%In many large scale training cases, accelerator utilization is low at only 30-50\%; see Figure~\ref{fig:gpu-utilization}. For example, neural recommendation models exhibit orders-of-magnitude lower compute-to-memory ratios~\cite{Gupta:hpca:2020} compared to multi-layer perceptron, convolutional, and recurrent neural networks. Meanwhile, neural recommendation models require significantly higher memory capacity and bandwidth~\cite{Acun:hpca:2021,Ke:isca:2020}. 
From the systems perspective, accelerator memory is scarce. 
However, DNNs, such as neural recommendation models, require significantly higher memory capacity and bandwidth~\cite{Acun:hpca:2021,Ke:isca:2020}. 
%and embodied carbon cost is starting to outweigh the operational footprint. 
This motivates researchers to develop memory-efficient model architectures. For example, the Tensor-Train compression technique (TT-Rec) achieves more than 100$\times$ memory capacity reduction with negligible training time and accuracy trade-off~\cite{yin:mlsys:2021}. Similarly, the design space trade-off between memory capacity requirement, training time, and model accuracy is also explored in Deep Hash Embedding (DHE)~\cite{kang:kdd:2021}. While training time increases lead to higher operational carbon footprint, in the case of TT-Rec and DHE, the memory-efficient model architectures require significantly lower memory capacity while better utilizing the computational capability of training accelerators, resulting in lower embodied carbon footprint. 
% From the sustainability perspective, innovating in the space of resource-efficient algorithms and model architectures is a promising direction to address AI's ever-increasing infrastructure demands.

Developing \textbf{\textit{efficient training algorithms}} is a long-time objective of research in optimization and numerical methods~\cite{nemirovskij1983problem}. 
%Whereas in traditional optimization it is possible, in theory, to properly set hyperparameters based on problem-specific constants (which are either known, or can be estimated online),  hyperparameter tuning is an essential part of life in optimization for ML. 
Evaluations of optimization methods should account for \textit{all} experimentation efforts required to tune optimizer hyperparameters, not just the method performance after tuning~\cite{choi2019empirical,sivaprasad2020optimizer}. 
%When comparisons are performed on a fixed hardware platform, this can also serve as a reasonable proxy for algorithm-based energy efficiency improvement.
In addition, 
%large-scale ML training experiments leverage multiple GPUs in parallel. 
significant research has gone into algorithmic approaches to efficiently scale training~\cite{goyal2017accurate,ott2018scaling} by reducing communication cost via compression~\cite{alistarh2017qsgd,vogels2019powersgd}, pipelining~\cite{huang2019gpipe}, and sharding~\cite{rajbhandari2020zero,rasley2020deepspeed}. The advances have enabled efficient scaling to larger models and larger datasets. 
%by making large-scale training more efficient. 
%There remains much more research for efficient distributed training. 
We expect efficient training methods to continue as an important domain.
While this paper has focused on supervised learning relying labeled data,   
%(e.g., machine translation and recommendation models, where training data is labeled), 
algorithmic efficiency extends to other learning paradigms including self-supervised and semi-supervised learning (Appendix~\ref{sec:ssl}).

% % \textbf{Operations-aware design:} 
% Traditionally, development of ML pipeline  components is rather siloed - feature selection, model design and training, and inference optimization are considered separate steps, often handled by different teams. In such a setting, it is challenging to design the model and associated pipelines in a way that considers both training and operational efficiency of the deployed model from the start. For instance, removing some features may not substantially affect model performance or training costs (e.g., because the data has already been collected), but may save large amounts of storage and data ingestion costs during deployment, resulting in much lower operational costs over the lifetime of the model. In order to fully understand and harness these inter-dependencies, we need to consider the various design stages more holistically going forward.

\subsection{Efficient, Environmentally-Sustainable AI Infrastructure and System Hardware}
\label{sec:systems}

To amortize the embodied carbon footprint, model developers and system architects must \textit{maximize the utilization of accelerator and system resources} when in use and \textit{prolong the lifetime of AI infrastructures}. 
Existing practices such as the move to domain-specific architectures at cloud scale~\cite{Jouppi:isca:2017,AWS-inferentia,Microsoft-graphcore} reduce AI computing’s footprint by consolidating computing resources at scale and by operating the shared infrastructures more environmentally-friendly with carbon free energy\footnote{We discuss additional important directions for building environmentally-sustainable systems in Appendix~\ref{sec:appendix-system-efficiency}, including datacenter infrastructure disaggregation; fault tolerant, resilient AI systems.}. %~\cite{greenest_cloud}.

%%%%%% APPENDIX %%%%%%
%\textcolor{blue}{Appendix} \textbf{Disaggregating Machine Learning Pipeline Stages:} As depicted in Figure~\ref{fig:ml_lifecycle}, the overall training throughput efficiency for large scale ML models depends on the throughput performance of both \textit{data ingestion and pre-processing} and \textit{model training}. Disaggregating the data ingestion and pre-processing stage of the machine learning pipeline from model training is the de-facto approach for industry-scale machine learning model training. This allows training accelerator, network and storage I/O bandwidth utilization to scale independently, thereby increasing the overall model training throughput by 56\%~\cite{Zhao:arxiv:2021}. Disaggregation with well-designed check-pointing support~\cite{Maeng:arxiv:2021,Eisenman:arxiv:2021} improves training fault tolerance as well. By doing so, failure on nodes that are responsible for data ingestion and pre-processing can be recovered efficiently without requiring re-runs of the entire training experiment. From a sustainability perspective, disaggregating the data storage and ingestion stage from model training maximizes infrastructure efficiency by \textit{using less system resources to achieve higher training throughput}, resulting in lower embodied carbon footprint. By increasing fault tolerance, the operational carbon footprint is reduced at the same time.  
%%%%%% APPENDIX %%%%%%

\begin{figure}[t]
    \centering
    \includegraphics[width=\linewidth]{images/GPU-utilization.pdf}
    %\vspace{-0.8cm}
    \caption{A vast majority of model experimentation (over tens of thousands of training workflows) utilizes GPUs at only 30-50\%, leaving room for utilization and efficiency improvements.}
    \label{fig:gpu-utilization}
    %\vspace{-0.6cm}
\end{figure}

\textbf{Accelerator Virtualization and Multi-Tenancy Support:} Figure~\ref{fig:gpu-utilization} illustrates the utilization of GPU accelerators in \fb's research training infrastructure. A significant portion of machine learning model experimentation utilizes GPUs at only 30-50\%, leaving significant room for improvements to efficiency and overall utilization. Virtualization and workload consolidation technologies can help maximize accelerator utilization~\cite{GPU-vm}. Google's TPUs have also recently started supporting virtualization~\cite{TPU-vm}. Multi-tenancy for AI accelerators is gaining traction as an effective way to improve resource utilization, thereby amortizing the upfront embodied carbon footprint of customized system hardware for AI at the expense of potential operational carbon footprint increase~\cite{Gschwind:jrd:2017,Ghodrati:micro:2020,Kao:arxiv:2021,Jeon:usenix:2019,Yu:arxiv:2019}.

\textbf{Environmental Sustainability as a Key AI System Design Principle:}
Today, servers are designed to optimize performance and power efficiency. 
However, system design with a focus on operational energy efficiency optimization does not always produce the most environmentally-sustainable solution~\cite{jain:mobicom:2002,Chang:hotpower:2010,Gupta:HPCA:2021}.
With the rising embodied carbon cost and the exponential demand growth of AI, system designers and architects must re-think fundamental system hardware design principles to minimize computing’s footprint end-to-end, considering the entire hardware and ML model development life cycle. In addition to the respective performance, power, and cost profiles, the environmental footprint characteristics of processors over the generations of CMOS technologies, DDRx and HBM memory technologies, SSD/NAND-flash/HDD storage technologies can be orders-of-magnitude different~\cite{Bardon:iedm:2020}. Thus, designing AI systems with the least environmental impact requires explicit consideration of environmental footprint characteristics at the design time. 


\textbf{The Implications of General-Purpose Processors, General-Purpose Accelerators, Reconfigurable Systems, and ASICs for AI:} 
There is a wide variety of system hardware choices for AI from general-purpose processors (CPUs), general-purpose accelerators (GPUs or TPUs), field-programmable gate arrays (FPGAs)~\cite{Putnam:ieee-micro-2015}, to application-specific integrated circuit (ASIC), such as Eyeriss~\cite{7551407}. 
The exact system deployment choice can be multifaceted --- 
the cadence of ML algorithm and model architecture evolution, the diversity of ML use cases and the respective system resource requirements, and the maturity of the software stack. 
While ML accelerator deployment brings a step-function improvement in \textit{operational energy efficiency}, it may not necessarily reduce the carbon footprint of AI computing overall. This is because of the upfront embodied carbon footprint associated with the different system hardware choices. 
From the environmental sustainability perspective, the optimal point depends on the compounding factor of operational efficiency improvement over generations of ML algorithms/models, deployment lifetime and embodied carbon footprint of the system hardware. Thus, to design for environmental sustainability, one must strike a careful balance between \textit{efficiency} and \textit{flexibility} and, at the same time, consider environmental impact as a key design dimension for next-generation AI systems.



%%%%%% APPENDIX %%%%%%
%\textbf{Fault-Tolerant AI Systems and Hardware:}
%One way to amortize the rising embodied carbon cost of AI infrastructures is to extend hardware lifetime. However, hardware ages --- depending on the wear-out characteristics, increasingly more errors can surface over time and result in \textit{silent data corruption}, leading to erroneous computation, model accuracy degradation, non-deterministic ML execution, or fatal system failure. In a large fleet of processors, silent data corruption can occur frequently enough to have disruptive impact on service productivity~\cite{Dixit:arxiv:2021,Hochschild:hotos:2021}. Decommissioning an AI system entirely because of hardware faults is expensive from the perspective of resource and environmental footprints. System architects can design differential reliability levels for micro architectural components on an AI system depending on the ML model execution characteristics. Alternatively, algorithmic fault tolerance can be built into deep learning programming frameworks to provide a code execution path that is cognizant of hardware wear-out characteristics.
%%%%%% APPENDIX %%%%%%

%\subsubsection{Increase system/hardware life time
%One way to amortize embodied carbon emissions is to extend hardware lifetimes. Today, servers are maintained in \fb’s data centers for ~3 years before being recycled for newer hardware. This enables \fb services to enjoy the doubling performance from Moore’s Law. However, Moore’s Law is coming to an end so there is much less incentive for \fb to keep up with the 3-year server upgrade cycle. Elixir %(https://www.internalfb.com/intern/wiki/Elixir_FAQ/) is exploring server lifetime extension.
    
%Developing fault-tolerance AI systems can help amortize the upfront embodied carbon emissions from manufacturing and infrastructure capacity. Example: Silent Error - Service Level Mitigation, Cores That Don’t Count %(https://sigops.org/s/conferences/hotos/2021/papers/hotos21-s01-hochschild.pdf).
    
%Fault and failure rate characteristics of CPUs, GPUs, DRAM, HBM, SSD/HDD disks are distinct [Large Scale Studies of Memory, Storage, and Network Failures in a Modern Data Center (https://arxiv.org/abs/1901.03401)]. When system failures occur, monolithic systems require the entire system be de-commissioned while systems designed with modularity in mind can be repaired — reducing e-waste. The Open Compute Project (OCP) specifies standard system interface, enabling customized \fb hardware infrastructures that are efficient, flexible, and scalable. Aligning with \fb’s sustainability mission, OCP kick-started the sustainability pillar to collaborate with industry partners defining sustainability standards, such as Life Cycle Analysis (LCA) metrics % (https://ocp-all.groups.io/g/Sustainability-Metrics-LCA), for computing infrastructures, aiming to build a circular economy around environmentally-sustainable computing. 

%\textbf{Programming Framework Support for Carbon-Efficient Model Architectures:}
%\textcolor{blue}{[Udit] Can we provide an illustrating example for different networks will be selected for a machine learning task when carbon cost is the optimization objective?}
%%\subsection{Optimize deep learning frameworks to scale model efficiency improvement across a wide variety of deep learning use cases}
%% Deep learning frameworks, compilers and toolchains can incorporate environmental footprint costs, such as carbon emission, as an optimization target or as part of a multi-objective optimization function. This enables model developers and architects to discover carbon-efficient model architectures, tailoring to the resource specification of ML systems.

%\begin{figure}[t]
%    \centering
%    \includegraphics[width=0.75\linewidth]{mlsys2022greenai/images/carbon aware scheduling.png}
%    \caption{\textcolor{blue}{Opportunity for 24/7 carbon aware scheduling.}}
%    \label{fig:cas}
%    \vspace{-0.4cm}
%\end{figure}


\textbf{Carbon-Efficient Scheduling for AI Computing At-Scale:}
As the electricity consumption of hyperscale data centers continues to rise, data center operators have devoted significant investment to neutralize operational carbon footprint.
By operating large-scale computing infrastructures with carbon free energy, technology companies
%, such as \fb~\cite{\fb_sustainability}, Google~\cite{google_sustainability}, and Microsoft~\cite{microsoft_sustainability}, 
are taking an important step to address the environmental implications of computing.
%~\cite{facebook_sustainability,google_sustainability,microsoft_sustainability}
\textit{More can be done however}.

As the renewable energy proportion in the electricity grid increases, fluctuations in 
energy generation will increase due to the intermittent nature of renewable energy sources (i.e. wind, solar). Elastic carbon-aware workload scheduling techniques
can be used in and across datacenters to predict and exploit the intermittent energy generation patterns~\cite{radovanovic2021carbon}. However such scheduling algorithms might require server over-provisioning to allow for flexibility of shifting workloads to times when carbon-free energy is available.
Furthermore, any additional server capacity comes with manufacturing carbon cost
which needs to be incorporated into the design space. 
%in carbon aware scheduling algorithm studies.
Alternatively, energy storage (e.g. batteries, pumped hydro, flywheels, molten salt) can be used to store renewable
energy during peak generation times for use during low generation times. 
There is an interesting design space to achieve 24/7 carbon-free AI computing.
%However, energy storage infrastructure is limited and expensive, and trade-offs
%of using storage versus shifting workloads using extra server capacities needs to
%be further studied in order to utilize renewable energy in the most effective way.


%\subsubsection{Enable carbon aware scheduling for AI workloads}
%Carbon aware scheduling considers the availability of renewable energy capacities when determining which data center computation should take place where and when. While Fblearner supports geography-aware, elastic scheduling for ML training by leveraging idle compute resources during off-peak hours, Fblearner does not consider carbon-intensity nor renewable energy availabilities currently. %https://fb.quip.com/BCCAEAnVbL8
    
%To mitigate the intermittent nature of renewable energy availability, scalable and cost-effective energy storage solutions are essential but can come with tradeoff, such as energy conversion loss, infrastructure availability, additional upfront infrastructure cost. Energy storage fuels, such as hydrogen, require complex electrochemical reactions driven by low-cost catalysts. To accelerate the discovery of effective catalysts, the OpenCatalyst %(https://fb.workplace.com/groups/2447831298797573/permalink/2757154297865270/) 
%project enables the community to find new ways to store renewable energy %(https://ai.\fb.com/blog/\fb-and-carnegie-mellon-launch-the-open-catalyst-project-to-find-new-ways-to-store-renewable-energy).
%Furthermore, to enable datacenter-scale computing be sourced with renewable energy in real-time, it requires fundamental changes to how we design systems and software to adopt to the intermittent nature of renewable energy availability — both temporally and spatially. 
    
%Carbon-free, renewable energy introduces interesting system and architectural optimization opportunities. For example, computation sprinting %(http://acg.cis.upenn.edu/sprinting/) 
%with turbo-boosting processor performance envelope or by activating dark silicon 
% (https://ieeexplore.ieee.org/document/6307773) 
%can be realized with more advanced, potentially more power-hungry data center and/or processor spot-cooling technologies. The additional cooling power requirement is carbon free and furthers the amortization of the embodied carbon emission of silicon.

\begin{figure}[t]
    \centering
    \includegraphics[width=\linewidth]{images/FL-footprint.pdf}
    %\vspace{-0.5cm}
    \caption{Federated learning and optimization can result in a non-negligible amount of carbon emissions, equivalent to the carbon footprint of training $Transformer_{Big}$~\cite{Patterson:arxiv:2021}. % With the increasing demand for on-device learning over billions of client devices and limited access to renewable energy at the edge, the carbon footprint of on-device learning can add up to a dire amount quickly.
    FL-1 and FL-2 represent two production FL applications.
    P100-Base represents the carbon footprint of $Transformer_{Big}$ training on P100 GPU
    whereas TPU-base is $Transformer_{Big}$ training on TPU. P100-Green and TPU-Green consider renewable energy at the cloud (Methodology detail in Appendix~\ref{sec:appendix-system-efficiency}).}
    %We compare the carbon footprint of FL-1 and FL-2 with the carbon footprint of Transformer Big on P100 GPU (P100-Base) and TPU (TPU-base). Although the models used in FL-1 and FL-2 are small (around 10MB), the carbon footprint of federated learning is comparable with training Transformer Big. We also show the carbon footprint considering renewable energy availability at the cloud. (P100-Green, TPU-Green).}
    \label{fig:fl_carbon}
    %\vspace{-0.25cm}
\end{figure}

\textbf{On-Device Learning}
On-device AI is becoming more ubiquitously adopted to enable model personalization~\cite{tinytl, fl_personalization,Bonawitz:arxiv:2019} while improving data privacy~\cite{gboard_prediction, gboard_ctr, gboard_emoji,huba2021papaya}, yet its impact in terms of carbon emission is often overlooked.
%
On-device learning emits non-negligible carbon. Figure~\ref{fig:fl_carbon} illustrates that the operational carbon footprint for training a small ML task using \emph{federated learning} (FL) is comparable to that of training an orders-of-magnitude larger Transformer-based model in a centralized setting.
%
As FL trains local models on client devices and periodically aggregates the model parameters for a global model, without collecting raw user data~\cite{gboard_prediction},
%
the FL process can emit non-negligible carbon at the edge due to both computation and wireless communication. 
%

%%%%%% APPENDIX %%%%%%
%To estimate the carbon emission, we used a similar methodology to~\cite{flcarbon}. We collected the 90-day log data for federated learning production use cases at \fb, which recorded the time spent on computation, data downloading, and data uploading per client device. We multiplied the computation time with the estimated device power and upload/download time with the estimated router power, and omitted other energy, as in~\cite{flcarbon}. We assumed a device power of 3W and a router power of 7.5W~\cite{phone_ml_energy, flcarbon}.
%Model training on client edge devices is inherently less energy-efficient because of the high wireless communication overheads, sub-optimal training data distribution in individual client devices~\cite{flcarbon}, large degree of system heterogeneity among client edge devices, and highly-fragmented edge device architectures that make system-level optimization significantly more challenging~\cite{wu:hpca:2019}. Note, the wireless communication energy cost takes up a significant portion of the overall energy footprint of federated learning, making energy footprint optimization on communication important.
%%%%%% APPENDIX %%%%%%

It is important to reduce AI's environmental footprint at the edge. With the ever-increasing demand for on-device use cases over billions of client devices, such as teaching AI to understand the physical environment from the first-person perception~\cite{grauman:2021:ego4d} or personalizing AI tasks, the carbon footprint for on-device AI can add up to a dire amount quickly. Also, renewable energy is far more limited for client devices compared to datacenters.
%
Optimizing the overall energy efficiency of FL and on-device AI is an important first step~\cite{kim:micro:2021,kang:asplos:2017,kim:micro:2020,yang:arxiv:2017,Stamoulis:iccad:2018}. Reducing embodied carbon cost for edge devices is also important, as 
%the dominating environmental footprint of client devices is unique, where 
manufacturing carbon cost accounts for 74\% of the total footprint~\cite{Gupta:HPCA:2021} of client devices.
%This is primarily because c
It is particularly challenging to amortize the embodied carbon footprint because client devices are often under-utilized~\cite{gao:ispass:2015}. 
%This distinct usage characteristics expose novel design dimensions for on-device computing. 
%
%
%When training is offloaded to the edge, it is hard to quantify and/or control their carbon emission, as each user's device hardware and their energy source is something that cannot be measured or controlled easily.


%\textcolor{blue}{Appendix} \textbf{Optimizing Efficiency vs. Optimizing Environmental Sustainability:}
%\textcolor{blue}{An example showing how optimizing efficiency may not always lead to environmentally-sustainable AI solutions.}


\section{Call-to-Action}

\subsection{Development of Easy-to-Adopt Telemetry for Assessing AI's Environmental Footprint}
\label{sec:metrics}

While the open source community has started building tools to enable automatic measurement of AI training's environmental footprint~\cite{Lacoste:arxiv:2019,Henderson:arxiv:2020,codecarbon,Lottick:2019} and the ML research community requiring a broader impact statement for the submitted research manuscript, more can be done in order to incorporate efficiency and sustainability into the design process.
Enabling carbon accounting methodologies and telemetry that is easy to adopt is an important step to quantify the significance of our progress in developing AI technologies in an environmentally-responsible manner. While assessing the novelty and quality of ML solutions, it is crucial to consider sustainability metrics including \textit{energy consumption} and \textit{carbon footprint} along with measures of \textit{model quality} and \textit{system performance}. %We must consider these metrics for the entire ML life cycle . 
% starting from \textif{Data} from which important features are extracted, \textit{Experimentation} and \textit{Training} where model explorations --- machine learning algorithms, model architectures, modeling techniques, training algorithms to determine parameters of models --- take place. Trained models are further optimized for \textit{Inference} deployment.

% \subsubsection{Accounting methods and component-level carbon footprint cost}
\textbf{Metrics for AI Model and System Life Cycles:} Standard carbon footprint accounting methods for AI's overall carbon footprint are at a nascent stage. We need simple, easy-to-adopt metrics to make fair and useful comparisons between AI innovations. Many different aspects must be accounted for, including the life cycles of both AI models (\textit{Data}, \textit{Experimentation}, \textit{Training}, \textit{Deployment}) and system hardware (\textit{Manufacturing} and \textit{Use}) (Section~\ref{sec:model-life-cycle-analysis}). 

In addition to incorporating an efficiency measure as part of leader boards for various ML tasks, data~\cite{kiela2021dynabench}, models\footnote{Papers with code: \url{https://paperswithcode.com/sota/image-classification-on-imagenet}}, training algorithms~\cite{hernandez2020efficiency}, environmental impact must also be considered and adopted by AI system hardware developers.
For example, MLPerf~\cite{Mattson:ieee-micro:2020,Reddi:ieee-micro:2021,mlperf:mobile} is the industry standard for ML system performance comparison. 
%Since the MLPerf Training~\cite{mlperf-training} and Inference~\cite{mlperf-inference} benchmark suites in 2018 and 2019, respectively, 
The industry has witnessed significantly higher system performance speedup, outstripping what is enabled by Moore's Law~\cite{mlperf-training,mlperf-inference}. Moreover, 
%the ML Commons Algorithmic Efficiency Working Group is currently 
an algorithm efficiency benchmark is under development\footnote{\url{https://github.com/mlcommons/algorithmic-efficiency/}}. 
%to measure progress in efficiency due to algorithmic advances (e.g., sampling, better optimizers)
 The MLPerf benchmark standards can advance the field of AI in an environmentally-competitive manner by enabling the measurement of energy and/or carbon footprint.
%Furthermore, depending on the carbon intensities of energy sources that fuel the AI model life cycle, the exact carbon emissions can vary. 

%The following metrics are crucial to assess the novelty and quality of machine learning solutions: 
%\begin{itemize}
%\vspace{-0.35cm}
%\setlength\itemsep{0em}
%    \item \textbf{Model quality} determines if a DNN is accurate enough to perform a given machine learning task;
%    \item \textbf{Latency/throughput} determines if the DNN trains fast enough to achieve the service-level objective and/or %can provide inference prediction under a pre-specified latency constraint;
%    \item \textbf{Power/energy consumption} is primarily determined by the AI system hardware; and
%    \item \textbf{Carbon footprint}, including both operational and manufacturing carbon emissions, determines the %environmental cost for a machine learning solution.
%\vspace{-0.35cm}
%\end{itemize}

% renewable energy availabilities. We are starting to tackle this direction by building first-order analytical models for rapid research system exploration undertaking detailed LCAs for \fb infrastructures and customized systems 
% [First-order Analytical Model for Embodied Carbon Footprint %(https://fb.workplace.com/groups/462049494322506/permalink/1047937865733663/)]. https://fb.quip.com/BOXAEAyhakM
% FAIR Cluster dashboard provides AI training energy consumption and estimated carbon emission data %(https://fb.workplace.com/notes/louis-martin/estimating-the-fair-cluster-gpu-co2-emissions/601001143813363). 
% To enable GPU system efficiency optimization, FAIR Cluster dashboard will provide detailed per-GPU SM-level utilization, GPU power consumption and temperature logging in the coming days.% https://fb.quip.com/HDcAEA4lK2B https://fb.quip.com/CVXAEAZreF4
% TCO versus carbon footprint: Many academic research studies often use Thermal Design Power (TDP) as a proxy for data center total cost of ownerships (TCO). As sustainability includes both operational and infrastructure capacity related components, understanding the differences between sustainability and data center TCO will better illuminate how the two can are aligned or differ.  
    
% \textbf{Machine Learning System Performance Benchmarks:}
%% \textcolor{blue}{MLPerf -- DawnBench -- Fathom -- TDB}    


\textbf{Carbon Impact Statements and Model Cards:} 
%An important step in raising awareness of AI's carbon footprint is to require carbon footprint disclosures. 
We believe it is important for all published research papers to disclose the operational \textit{and} embodied carbon footprint of proposed design; we are only at the beginning of this journey\footnote{\url{https://2021.naacl.org/ethics/faq/\#-if-my-paper-reports-on-experiments-that-involve-lots-of-compute-timepower}}. Note, while embodied carbon footprints for AI hardware may not be readily available, describing hardware platforms, the number of machines, total runtime used to produce results presented in a research manuscript is an important first step.
%Examples: Towards the Systematic Reporting of the Energy and Carbon Footprints of Machine Learning %(https://arxiv.org/abs/2002.05651).
%\textbf{Model cards:} %(https://arxiv.org/abs/1810.03993): 
In addition, new models must be associated with a model card that, among other aspects of data sets and models~\cite{Mitchell:fat:2019}, describes the model’s overall carbon footprint to train and conduct inference. 

%\subsubsection{Policy and guideline}
%    As hardware manufacturing and infrastructure account for a large fraction of \fb’s environmental footprint it is crucial to consider the role of future resource planning and provisioning decisions for AI infrastructure. Capacity that is ineffectively used incurs high environmental costs. An external example of this includes Microsoft's internal tax to “hold business divisions financially responsible for reducing their carbon emissions” 
%(https://blogs.microsoft.com/on-the-issues/2019/04/15/were-increasing-our-carbon-fee-as-we-double-down-on-sustainability/). 


\section{Key Takeaways}
\label{sec:takeaways}

%Advances in AI are currently driven by AI research that seeks to improve accuracy (or related measures) through the use of massive computational power while disregarding physical resource requirements. 

\textbf{The Growth of AI:} Deep learning has witnessed an exponential growth in training data, model parameters, and system resources over the recent years (Figure~\ref{fig:data-model-system-growth}).
The amount of data for AI has grown by $2.4\times$, leading to $3.2\times$ increase in the data ingestion bandwidth demand at \fb.
%To learn valuable information from the data, model sizes for a variety of ML tasks have also increased to achieve higher model quality. 
\fb's recommendation model sizes have increased by $20\times$ between 2019 and 2021.
The explosive growth in AI use cases has driven $2.9\times$ and $2.5\times$ capacity increases for AI training and inference at Facebook over the recent 18 months, respectively.
The environmental footprint of AI is staggering (Figure~\ref{figure:cf-characterization}, Figure~\ref{figure:ops-vs-embodied}).

    
\textbf{A Holistic Approach:} To ensure an environmentally-sustainable growth of AI, we must consider the AI ecosystem holistically going forward.
We must look at the machine learning pipelines end-to-end --- data collection, model exploration and experimentation, model training, optimization and run-time inference (Section~\ref{sec:model-life-cycle-analysis}). 
The frequency of training and scale of each stage of the ML pipeline must be considered to understand salient bottlenecks to sustainable AI.
From the system's perspective, the life cycle of model development and system hardware, including \textit{manufacturing} and \textit{operational use}, must also be accounted for. 

\textbf{Efficiency Optimization:} 
%\textcolor{blue}{summarize Section 3. 
Optimization across the axes of algorithms, platforms, infrastructures, hardware can significantly reduce the operational carbon footprint for the Transformer-based universal translation model by $810\times$. Along with other efficiency optimization at-scale, this has translated into 25.8\% operational energy footprint reduction over the two-year period. 
%Optimization is an iterative process. 
\textit{More must be done to bend the environmental impact from the exponential growth of AI} (Figure~\ref{fig:jevon-paradox} and Figure~\ref{figure:server-utilization}).

\textbf{An Sustainability Mindset for AI:} Optimization beyond efficiency across the software and hardware stack at scale is crucial to enabling future sustainable AI systems.
To develop AI technologies responsibly, we must achieve competitive model accuracy at a fixed or even reduced computational and environmental cost. We chart out potentially high-impact research and development directions across the \textit{data}, \textit{algorithms and model}, \textit{experimentation} and \textit{system hardware}, and \textit{telemetry} dimensions for AI at datacenters and at the edge (Section~\ref{sec:optimization-opportunities}). 

We must take a deliberate approach when developing AI research and technologies, considering the environmental impact of innovations and taking a responsible approach to technology development~\cite{wu:arxiv:2021}. That is, we need AI to be green and environmentally-sustainable.

%\fb has the opportunity to lead the industry in creation of environmentally responsible AI. 
%We will drive towards this goal through: 
%\begin{enumerate}
%    \item transparent reporting of metrics of the broader energy and carbon %footprints of ML in both research and production settings
%    \item community building through external partnerships to drive organic shift towards efficiency mindset
%    \item incentive mechanisms to encourage reduction in resources
%    \item internal dogfooding to drive down cost and demonstrate benefits of developing AI in an environmentally responsible manner.
%\end{enumerate}


\begin{comment}
\begin{figure}
\includegraphics[width=\linewidth]{figs/beyond_tss_lesion.pdf}
\caption[]{End-to-End runtime lesion study of the entire MNIST dataset and the FMA featurized music dataset. Each of DROP's contributions provides a runtime improvement.}
\label{fig:beyond_lesion}
\end{figure}
\end{comment}



\section{Conclusion}
\label{sec:conclusion}

Advanced data analytics techniques must scale to rising data volumes. 
DR techniques offer a powerful toolkit when processing these datasets, with PCA frequently outperforming popular techniques in exchange for high computational cost. 
In response, we propose DROP, a new dimensionality reduction optimizer. 
DROP combines progressive sampling, progress estimation, and online aggregation to identify high quality low dimensional bases via PCA without processing the entire dataset by balancing the runtime of downstream tasks and achieved dimensionality. 
Thus, DROP provides a first step in bridging the gap between quality and efficiency in end-to-end DR for downstream \red{analytics}. 

%We revisit canonical operators for time series dimensionality reduction and the measurement study of~\cite{keogh-study}, and show that PCA is more effective than popular alternatives in the data mining literature often by a margin of over $2\times$ on average on gold-standard time series benchmark data sets with respect to output data dimension. More surprisingly, we empirically demonstrate that a small number of samples are sufficient to accurately characterize directions of maximum variance and obtain a high-quality low-dimensional transformation.




\section*{Acknowledgement}

We would like to thank 
Nikhil Gupta,
Lei Tian, 
Weiyi Zheng, 
Manisha Jain,
Adnan Aziz,
and Adam Lerer for their feedback on many iterations of this draft, and in-depth technical discussions around building efficient infrastructure and platforms;  
Adina Williams,
Emily Dinan,
Mona Diab,
Ashkan Yousefpour for the valuable discussions and insights on AI and environmental responsibility; 
Mark Zhou,
Niket Agarwal,
Jongsoo Park,
Michael Anderson,
Xiaodong Wang; 
Yatharth Saraf,
Hagay Lupesco, Jigar Desai, Joelle Pineau, 
Ram Valliyappan, Rajesh Mosur,
Ananth Sankarnarayanan and
Eytan Bakshy for their leadership and vision without which this work would not have been possible. 


% the insightful discussions and valuable context for sustainable computing efforts across industry.

% Acknowledgement
% Yatharth Saraf
% Hagay Lupesco
% Nikhil Gupta
% Eytan Bakshy
% Ram Valliyappan
% Ananth Sankarnarayanan
% Adina Williams
% Emily Dinan
% Mark Zhou
% Niket Agarwal
% Jongsoo Park
% Michael Anderson
% Xiaodong Wang
% Ashkan Yousefpour
% Adnan Aziz
% Lei Tian 
% Mona Diab 



%%%%%%% -- PAPER CONTENT ENDS -- %%%%%%%%


%%%%%%%%% -- BIB STYLE AND FILE -- %%%%%%%%
\bibliographystyle{ieeetr}
\bibliography{greenai}

%%%%%%%%%%%%%%%%%%%%%%%%%%%%%%%%%%%%
\newpage
\appendix


\section{An Sustainability Mindset for AI}
\label{sec:appendix-efficiiency-mindset}

Despite the recent calls-to-action~\cite{Strubell:arxiv:2019,Lacoste:arxiv:2019,Henderson:arxiv:2020,Bender:facct:2021}, the overall community remains under-invested in research that aims at deeply understanding and minimizing the cost of AI.
There are several factors that may have contributed to the current state of AI:

\begin{itemize}

\setlength\itemsep{0em}
    \item {\bf Lack of incentives:} Over 90\% of the ML publications only focus on model accuracy improvements at the expense of efficiency~\cite{Schwartz:arxiv:2019}. Challenges\footnote{Efficient Open-Domain Question Answering (\url{https://efficientqa.github.io/}), SustaiNLP: Simple and Efficient Natural Language Processing  (\url{https://sites.google.com/view/sustainlp2020/home}), and WMT: Machine Translation Efficiency Task (\url{http://www.statmt.org/wmt21/efficiency-task.html}).} incentivize investment into efficient approaches.
    \item {\bf Lack of common tools:} There is no standard telemetry in place to provide accurate, reliable energy and carbon footprint measurement. The measurement methodology is complex --- factors, such as datacenter infrastructures, hardware architectures, energy sources, can perturb the final measure easily.
    \item {\bf Lack of normalization factors:} Algorithmic progress in ML is often presented in some measure of model accuracy, e.g., BLEU, points, ELO, cross-entropy loss, but without considering resource requirement as a normalization factor, e.g., the number of \\CPU/GPU/TPU hours used, the overall energy consumption and/or carbon footprint required.   
    \item {\bf Platform fragmentation:} Implementation details can have a significant impact on real-world efficiency, but best practices remain elusive and platform fragmentation prevents performance and efficiency portability across model development.

\end{itemize}



\section{Additional Opportunities for AI Research and Development}
\label{sec:additional-opportunities}
\vspace{0.4cm}

\begin{figure}[t]
    \centering
    \includegraphics[width=\linewidth]{images/Model-Data-Scaling.pdf}
    %{mlsys2022greenai/images/Data Scaling.png}
    \vspace{-1cm}
    \caption{Model quality of recommendation use cases improves as we scale up the amount of data and/or the number of model parameters (e.g., embedding cardinality or dimension), leading to higher energy and carbon footprint. Maximizing model accuracy for the specific recommendation use case comes with significant energy cost --- Roughly 4$\times$ energy saving can be achieved with only 0.004 model quality degradation (green vs. yellow stars).}
    \label{fig:energy-scaling}
    \vspace{-0.4cm}
\end{figure}

\subsection{Data Utilization Efficiency}
\label{sec:appendix-data-efficiency}

Figure~\ref{fig:energy-scaling} depicts energy footprint reduction potential when data and model scaling is performed in tandem. The x-axis represents the energy footprint required per training step whereas the y-axis represents model error. 
The \textbf{blue} solid lines capture model size scaling (through embedding hash scaling) while the training data set size is kept fixed. 
Each line corresponds to a different data set size, in an increasing order from top to bottom.
The points within each line represent different model (embedding) sizes, in an increasing order from left to right. 
The \textbf{red} dashed lines capture data scaling while the model size is kept fixed. 
Each line corresponds to a different embedding hash size, in an increasing order from left to right.
The points within each line represent different data sizes, in an increasing order from top to bottom. 
The dashed black line captures the performance scaling trend as we scale data and model sizes in tandem. This represents the energy-optimal scaling approach.

%0.79 vs. 0.794
Scaling data sizes or model sizes independently deviates from the energy-optimal trend.
We highlight two energy-optimal settings along the Pareto-frontier curve.
The yellow star uses the scaling setting of \textit{Data scaling 2$\times$} and \textit{Model scaling 2$\times$} whereas the green star adopts the setting of \textit{Data scaling 8$\times$} and \textit{Model scaling 16$\times$}.
The yellow star consumes roughly 4$\times$ lower energy as compared to the green star with only 0.004 model quality degradation in Normalized Entropy. 
%for 0.8\% NE gain) highlights two different energy-optimal solutions at two different accuracy targets and energy cost. Note that while the green point is only 0.5\% better in performance, it requires almost 4x more energy.
Overall model quality performance has a (diminishing) power-law relationship with the corresponding energy consumption and the power of the power law is extremely small (0.002-0.004). This means achieving higher model quality through model-data scaling for recommendation use cases incurs significant energy cost.

%\subsection{Resource-Efficient Modeling Techniques}
%\label{sec:appendix-algo-efficiency}

\subsection{Efficient, Environmentally-Sustainable AI Systems}
\label{sec:appendix-system-efficiency}

\textbf{Disaggregating Machine Learning Pipeline Stages:} As depicted in Figure~\ref{fig:ml_lifecycle}, the overall training throughput efficiency for large-scale ML models depends on the throughput performance of both \textit{data ingestion and pre-processing} and \textit{model training}. Disaggregating the data ingestion and pre-processing stage of the machine learning pipeline from model training is the de-facto approach for industry-scale machine learning model training. This allows training accelerator, network and storage I/O bandwidth utilization to scale independently, thereby increasing the overall model training throughput by 56\%~\cite{Zhao:arxiv:2021}. Disaggregation with well-designed check-pointing support~\cite{Maeng:arxiv:2021,Eisenman:arxiv:2021} improves training fault tolerance as well. By doing so, failure on nodes that are responsible for data ingestion and pre-processing can be recovered efficiently without requiring re-runs of the entire training experiment. From a sustainability perspective, disaggregating the data storage and ingestion stage from model training maximizes infrastructure efficiency by \textit{using less system resources to achieve higher training throughput}, resulting in lower embodied carbon footprint. By increasing fault tolerance, the operational carbon footprint is reduced at the same time.  

\textbf{Fault-Tolerant AI Systems and Hardware:}
One way to amortize the rising embodied carbon cost of AI infrastructures is to extend hardware lifetime. However, hardware ages --- depending on the wear-out characteristics, increasingly more errors can surface over time and result in \textit{silent data corruption}, leading to erroneous computation, model accuracy degradation, non-deterministic ML execution, or fatal system failure. In a large fleet of processors, silent data corruption can occur frequently enough to have disruptive impact on service productivity~\cite{Dixit:arxiv:2021,Hochschild:hotos:2021}. Decommissioning an AI system entirely because of hardware faults is expensive from the perspective of resource and environmental footprints. System architects can design differential reliability levels for micro architectural components on an AI system depending on the ML model execution characteristics. Alternatively, algorithmic fault tolerance can be built into deep learning programming frameworks to provide a code execution path that is cognizant of hardware wear-out characteristics.

%\textbf{AI Computing with Renewable Energy:} There exist a number of challenges and opportunities as data centers increasingly draw on renewable energy sources. Hyperscale data center operators
%---such as \fb~\cite{}, Google~\cite{google_sustainability}, and Microsoft~\cite{microsoft_sustainability}---
%have made significant strides to address the environmental implications of computing. Operators have achieved "net zero" by matching data center energy consumption against renewable energy generation. First, operators collaborate with utility providers, investing in wind or solar generation and securing contracts to purchase the resulting carbon-free energy. Second, as contracted renewable energy is generated, the operator receives renewable energy credits that are matched against data center energy consumption. 

%At present, the generation and consumption of renewable energy is matched annually and more can be done. In future, data centers should schedule computation based on the hourly availability and carbon intensity of its energy sources (\textit{i.e.}, 24/7 carbon-free). As the proportion of renewable energy in the electricity grid increases, data centers must be able to manage increased fluctuations in energy generation due to the intermittent nature of renewable energy sources (\textit{i.e.}, wind, solar). 

%Ensuring renewable energy generation matches data center consumption in each and every hour of the day requires a coordinated, multi-pronged strategy. First, operators must continue to invest in carbon-free energy, ensuring that its contracted mix of solar and wind assets provide significant supply across times of day and seasons of the year. Second, operators should plan to store carbon-free energy, charging and discharging batteries when renewable generation is high and low, respectively. Third, operators should provision additional server capacity so that flexible workloads can be shifted to times when carbon-free energy, whether from the generation or batteries, is more abundant. 

%System architects should explore the rich, multi-dimensional design space to determine the least expensive path to 24/7 matching. The most efficient solution coordinates the provision of renewable generation, battery capacity, and server capacity. Decisions made in one dimension will affect others. A complementary mix of wind and solar may require less battery capacity. Greater battery capacity may reduce demands on workload shifting. The various solutions differ in total cost of ownership as well as the embodied carbon of infrastructure such as batteries or servers. Although there has been significant prior work in data center scheduling based on signals from the power grid \cite{radovanovic2021carbon, liu14sigmetrics, wierman14igcc}, we have yet to see a holistic approach to data center design and management in an era of renewable, carbon-free energy.  

\textbf{On-Device Learning:} 
Federated learning and optimization can result in a non-negligible amount of carbon emissions at the edge, similar to the carbon footprint of training $Transformer_{Big}$~\cite{Patterson:arxiv:2021}.
Figure~\ref{fig:fl_carbon} shows that the federated learning and optimization process emits non-negligible carbon at the edge due to both computation and wireless communication during the process. 
To estimate the carbon emission, we used a similar methodology to~\cite{flcarbon}. We collected the 90-day log data for federated learning production use cases at \fb, which recorded the time spent on computation, data downloading, and data uploading per client device. We multiplied the computation time with the estimated device power and upload/download time with the estimated router power, and omitted other energy. We assumed a device power of 3W and a router power of 7.5W~\cite{phone_ml_energy, flcarbon}.
Model training on client edge devices is inherently less energy-efficient because of the high wireless communication overheads, sub-optimal training data distribution in individual client devices~\cite{flcarbon}, large degree of system heterogeneity among client edge devices, and highly-fragmented edge device architectures that make system-level optimization significantly more challenging~\cite{wu:hpca:2019}. Note, the wireless communication energy cost takes up a significant portion of the overall energy footprint of federated learning, making energy footprint optimization on communication important.


\subsection{Efficiency and Self-Supervised Learning}
\label{sec:ssl}

\emph{Self-supervised learning} (SSL) have received much attention in the research community in recent years. SSL methods train deep neural networks without using explicit supervision in the form of human-annotated labels for each training sample. Having humans annotate data is a time-consuming, expensive, and typically noisy process. SSL methods are typically used to train \emph{foundation models} --- models that can readily be fine-tuned using a small amount of labeled data on a down-stream task~\cite{bommasani2021opportunities}. SSL methods have been extremely successful for pre-training large language models, becoming the de-facto standard, and they have also attracted great interest in computer vision.

When comparing supervised and self-supervised methods, there is a glaring trade-off between having labels and the amount of computational overhead involved in pre-training. For example, Chen et al. report achieving 69.3\% top-1 validation accuracy with a ResNet-50 model after SSL pre-training for 1000 epochs on the ImageNet dataset and using the linear evaluation protocol, freezing the pre-trained feature extractor, and fine-tuning a linear classifier on top for 60 epochs using the full ImageNet dataset with all labels~\cite{chen2020simple}. In contrast, the same model typically achieves at least 76.1\% top-1 accuracy after 90 epochs of fully-supervised training. Thus, in this example, using labels and supervised training is worth a roughly 10$\times$ reduction in training effort, measured in terms of number of passes over the dataset.

Recent work suggests that incorporating even a small amount of labeled data can significantly bridge this gap. Assran et al. describe an approach called \emph{Predicting view Assignments With Support samples} (PAWS) for semi-supervised pre-training inspired by SSL~\cite{assran2021semi}. With access to labels for just 10\% of the training images in ImageNet, a ResNet-50 achieves 75.5\% top-1 accuracy after just 200 epochs of PAWS pre-training. Running on 64 V100 GPUs, this takes roughly 16 hours. Similar observations have recently been made for language model pre-training as well~\cite{dery2021should}.

Self-supervised pre-training potentially has advantages in that a single foundation model can be trained (expensive) but then fine-tuned (inexpensive), amortizing the up front cost across many tasks~\cite{bommasani2021opportunities}. Substantial additional research is needed to better understand the cost-benefit trade-offs for this paradigm.

\end{document}
