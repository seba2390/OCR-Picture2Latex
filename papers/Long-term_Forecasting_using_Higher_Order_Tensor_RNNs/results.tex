\begin{figure}
    \centering

    \begin{subfigure}[b]{0.3\textwidth}
        \includegraphics[width=\textwidth]{Figure/f1.png}
        \caption{oscillatory}
        \label{fig:f1}
    \end{subfigure}
    ~ %add desired spacing between images, e. g. ~, \quad, \qquad, \hfill etc.
      %(or a blank line to force the subfigure onto a new line)
    \begin{subfigure}[b]{0.3\textwidth}
        \includegraphics[width=\textwidth]{Figure/df1.png}
        \caption{oscillatory}
        \label{fig:df1}
    \end{subfigure}
    \begin{subfigure}[b]{0.3\textwidth}
        \includegraphics[width=\textwidth]{Figure/foo.jpg}
        \caption{product peak}
        \label{fig:f2}
    \end{subfigure}\\ % f1


        \begin{subfigure}[b]{0.3\textwidth}
        \includegraphics[width=\textwidth]{Figure/f2.png}
        \caption{product peak}
        \label{fig:f2}
    \end{subfigure}
    ~ %add desired spacing between images, e. g. ~, \quad, \qquad, \hfill etc.
      %(or a blank line to force the subfigure onto a new line)
    \begin{subfigure}[b]{0.3\textwidth}
        \includegraphics[width=\textwidth]{Figure/df2.png}
        \caption{product peak}
        \label{fig:df2}
    \end{subfigure}
    ~ %add desired spacing between images, e. g. ~, \quad, \qquad, \hfill etc.
    %(or a blank line to force the subfigure onto a new line)
    \begin{subfigure}[b]{0.3\textwidth}
        \includegraphics[width=\textwidth]{Figure/df2_3.png}
        \caption{product peak}
        \label{fig:df2}
    \end{subfigure}\\ % f2
            \begin{subfigure}[b]{0.3\textwidth}
        \includegraphics[width=\textwidth]{Figure/f3.png}
        \caption{corner peak}
        \label{fig:f3}
    \end{subfigure}
    ~ %add desired spacing between images, e. g. ~, \quad, \qquad, \hfill etc.
      %(or a blank line to force the subfigure onto a new line)
    \begin{subfigure}[b]{0.3\textwidth}
        \includegraphics[width=\textwidth]{Figure/df3.png}
        \caption{corner peak}
        \label{fig:df3}
    \end{subfigure}
    ~ %add desired spacing between images, e. g. ~, \quad, \qquad, \hfill etc.
    %(or a blank line to force the subfigure onto a new line)
    \begin{subfigure}[b]{0.3\textwidth}
        \includegraphics[width=\textwidth]{Figure/foo.jpg}
        \caption{corner peak}
        \label{fig:df3_7}
    \end{subfigure}\\ % f3 g


    \begin{subfigure}[b]{0.3\textwidth}
        \includegraphics[width=\textwidth]{Figure/f4.png}
        \caption{Gaussian}
        \label{fig:f4}
    \end{subfigure}
    \begin{subfigure}[b]{0.3\textwidth}
        \includegraphics[width=\textwidth]{Figure/df4.png}
        \caption{Gaussian}
        \label{fig:df1}
    \end{subfigure}
    \begin{subfigure}[b]{0.3\textwidth}
        \includegraphics[width=\textwidth]{Figure/df4_8.png}
        \caption{Gaussian}
        \label{fig:f4}
    \end{subfigure}\\ % f4

        \begin{subfigure}[b]{0.3\textwidth}
        \includegraphics[width=\textwidth]{Figure/f5.png}
        \caption{continuous}
        \label{fig:f5}
    \end{subfigure}
    ~ %add desired spacing between images, e. g. ~, \quad, \qquad, \hfill etc.
      %(or a blank line to force the subfigure onto a new line)
    \begin{subfigure}[b]{0.3\textwidth}
        \includegraphics[width=\textwidth]{Figure/df5.png}
        \caption{continuous}
        \label{fig:df5}
    \end{subfigure}
    ~ %add desired spacing between images, e. g. ~, \quad, \qquad, \hfill etc.
    %(or a blank line to force the subfigure onto a new line)
    \begin{subfigure}[b]{0.3\textwidth}
        \includegraphics[width=\textwidth]{Figure/df5_1.png}
        \caption{continuous}
        \label{fig:df4}
    \end{subfigure}\\ % f5

            \begin{subfigure}[b]{0.3\textwidth}
        \includegraphics[width=\textwidth]{Figure/f6.png}
        \caption{discontinuous}
        \label{fig:f6}
    \end{subfigure}
    ~ %add desired spacing between images, e. g. ~, \quad, \qquad, \hfill etc.
      %(or a blank line to force the subfigure onto a new line)
    \begin{subfigure}[b]{0.3\textwidth}
        \includegraphics[width=\textwidth]{Figure/df6.png}
        \caption{discontinuous}
        \label{fig:df6}
    \end{subfigure}
    ~ %add desired spacing between images, e. g. ~, \quad, \qquad, \hfill etc.
    %(or a blank line to force the subfigure onto a new line)
    \begin{subfigure}[b]{0.26\textwidth}
        \includegraphics[width=\textwidth]{Figure/df6_5.png}
        \caption{discontinuous}
        \label{fig:f6}
    \end{subfigure}\\ % f6 g

    \caption{Pictures of animals}\label{fig:animals}
\end{figure}



















% \begin{table}[t]
% \caption{Real-world data long forecasting RMSE with respect to the forecasting horizon}
% \label{sample-table}
% \begin{center}
% \begin{tabular}{lllll}
% \multicolumn{1}{c}{\bf MODEL}  &\multicolumn{4}{c}{\bf RMSE (1e-2)}
% \\ \hline \\
%         & t=20 & t=40 & t=60 & t=80 \\
% LSTM      & 0.00928109   & 0.0223645    & 0.11187873  &  0.14012846\\
% MLSTM     &0.00689099    & 0.02217738 & 0.11175835 & 0.13953213\\
% \tlstm{}     &0.00654901 &  0.02197221 &  0.1117889 & 0.13957831\\
% \\ \hline
% \end{tabular}
% \end{center}
% \end{table}





%\begin{figure}
%    \centering
%    \begin{subfigure}[b]{0.2\textwidth}
%        \includegraphics[width=\textwidth]{Figure/trnn_r2_1.png}
%    \end{subfigure}
%    \begin{subfigure}[b]{0.2\textwidth}
%        \includegraphics[width=\textwidth]{Figure/trnn_r4_1.png}
%
%    \end{subfigure}
%    \begin{subfigure}[b]{0.2\textwidth}
%        \includegraphics[width=\textwidth]{Figure/trnn_r8_1.png}
%
%    \end{subfigure}
%       \begin{subfigure}[b]{0.2\textwidth}
%        \includegraphics[width=\textwidth]{Figure/trnn_r16_1.png}
%
%    \end{subfigure}\\ %example 1
%
%        \begin{subfigure}[b]{0.2\textwidth}
%        \includegraphics[width=\textwidth]{Figure/trnn_r2_2.png}
%        \caption{TRNN $r=2$}
%        \label{fig:f1}
%    \end{subfigure}
%    \begin{subfigure}[b]{0.2\textwidth}
%        \includegraphics[width=\textwidth]{Figure/trnn_r4_2.png}
%        \caption{TRNN $r=4$}
%        \label{fig:df1}
%    \end{subfigure}
%    \begin{subfigure}[b]{0.2\textwidth}
%        \includegraphics[width=\textwidth]{Figure/trnn_r8_2.png}
%        \caption{TRNN $r=8$}
%        \label{fig:f2}
%    \end{subfigure}
%       \begin{subfigure}[b]{0.2\textwidth}
%        \includegraphics[width=\textwidth]{Figure/trnn_r16_2.png}
%        \caption{TRNN $r=16$}
%        \label{fig:f2}
%    \end{subfigure}\\ %example 2
%
%   \caption{long-term (right 2) predictions (red) for Tensor-RNN  versus the ground truth (blue) with respect to different ranks.}\label{fig:long-term}
%\end{figure}

% \begin{figure}
%     \end{subfigure}
%        \begin{subfigure}[b]{0.2\textwidth}
%         \includegraphics[width=\textwidth]{Figure/trnn_r16_1.png}

%     \end{subfigure}\\ %example 1

%         \begin{subfigure}[b]{0.2\textwidth}
%         \includegraphics[width=\textwidth]{Figure/trnn_r2_2.png}
%         \caption{TRNN $r=2$}
%         \label{fig:f1}
%     \end{subfigure}
%     \begin{subfigure}[b]{0.2\textwidth}
%         \includegraphics[width=\textwidth]{Figure/trnn_r4_2.png}
%         \caption{TRNN $r=4$}
%         \label{fig:df1}
%     \end{subfigure}
%     \begin{subfigure}[b]{0.2\textwidth}
%         \includegraphics[width=\textwidth]{Figure/trnn_r8_2.png}
%         \caption{TRNN $r=8$}
%         \label{fig:f2}
%     \end{subfigure}
%        \begin{subfigure}[b]{0.2\textwidth}
%         \includegraphics[width=\textwidth]{Figure/trnn_r16_2.png}
%         \caption{TRNN $r=16$}
%         \label{fig:f2}
%     \end{subfigure}\\ %example

%    \caption{long-term (right 2) predictions (red) for \trnn{}  versus the ground truth (blue) with respect to different ranks.}\label{fig:long-term}
% \end{figure}
