\documentclass[twoside,11pt]{article}
% about 20 main file, 30 total 
% Any additional packages needed should be included after jmlr2e.
% Note that jmlr2e.sty includes epsfig, amssymb, natbib and graphicx,
% and defines many common macros, such as 'proof' and 'example'.
%
% It also sets the bibliographystyle to plainnat; for more information on
% natbib citation styles, see the natbib documentation, a copy of which
% is archived at http://www.jmlr.org/format/natbib.pdf

\usepackage{jmlr2e}

\usepackage{algorithm}
\usepackage{algorithmic}
%\usepackage{breakurl}
%\usepackage[breaklinks]{hyperref}
\usepackage{wrapfig}
\usepackage{graphicx} % \texttt{\texttt{}}more modern

 % and their extensions so you won't have to specify these with
 % every instance of \includegraphics
\DeclareGraphicsExtensions{.pdf,.jpeg,.png}

\usepackage{wrapfig}
\usepackage{subfiles}
\usepackage[subpreambles=false]{standalone}

\usepackage{microtype}
\usepackage{graphicx}
\usepackage{subcaption}
\usepackage{booktabs} % for professional tables
\usepackage{url}
\usepackage{graphicx}
\usepackage{amsmath}
\usepackage{amssymb}
\usepackage{wrapfig}
\usepackage{color}

\newcommand{\fix}{\marginpar{FIX}}
\newcommand{\new}{\marginpar{NEW}}


\usepackage{enumitem}
%\iclrfinalcopy % Uncomment for camera-ready version, but NOT for submission.

\usepackage[boxed,lined,noend,algoruled]{algorithm2e}
\usepackage{amsmath,amssymb}
%\usepackage{kpfonts}
\usepackage{tikz}
\usepackage{tkz-euclide}
\usepackage{todonotes,color}
\usepackage{url}
\usepackage{subfig}
\usepackage{pgfplots}
\usepackage{xcolor}
\usepackage{xspace}
\usepackage{framed}
\usepackage[T1]{fontenc}
% \usepackage[capitalise]{cleveref} 


% \varprod
\DeclareSymbolFont{largesymbolsA}{U}{txexa}{m}{n}
\DeclareMathSymbol{\varprod}{\mathop}{largesymbolsA}{16}

\DeclareMathAlphabet\mathbfcal{OMS}{cmsy}{b}{n}
\def\mathbi#1{\textbf{\em #1}}

\newcommand{\set}[2]{\left\{#1 \; \left|\;\; #2 \right.\right\}}

\newcommand{\tchak}{\vdash}

\newcommand{\exact}{\textbf{exact}}
\newcommand{\hcenter}{\textbf{center}}
\newcommand{\cons}{\textbf{cons}}
\newcommand{\worst}{\textbf{worst}}
\newcommand{\avg}{\textbf{avg}}
\newcommand{\compact}{\textbf{compact}}
\newcommand{\heur}{\textbf{heur}}

% \newcommand{\dmin}{d^{min}}
\newcommand{\class}{\mathcal{C}}
\newcommand{\itpaths}[1]{\F^{#1\rightarrow t}}
\newcommand{\itpathsless}[2]{\F^{#1\rightarrow t}_{#2}}
\newcommand{\posparent}{v}
\renewcommand{\S}{\mathbi{S}}
\newcommand{\Metric}{\mathcal{M}}
\newcommand{\w}{{w}}
\newcommand{\C}{\mathcal{C}}
\newcommand{\K}{\mathcal{K}}
\newcommand{\G}{\mathcal{G}}
\newcommand{\GG}{\mathbb{G}}
\newcommand{\VV}{\mathbb{V}}
\newcommand{\EE}{\mathbb{E}}
\newcommand{\Z}{\mathbb{Z}}
\newcommand{\R}{\mathbb{R}}
\newcommand{\Edir}{\Vec{\EE}}
\newcommand{\CC}{\mathcal{C}}
\newcommand{\U}{\mathcal{U}}
\newcommand{\E}{\mathcal{E}}
\newcommand{\W}{\mathcal{W}^p}
\newcommand{\WW}{\mathcal{W}}
%\newcommand{\UU}{\mathcal{U}}
\newcommand{\UU}{\mathcal{U}}
\newcommand{\tUU}{\widetilde{\mathcal{U}}}
\newcommand{\tF}{\widetilde{F}}
\newcommand{\tomega}{\tilde{\omega}}
\newcommand{\tx}{\tilde{x}}
\newcommand{\tmu}{\tilde{\mu}}
\newcommand{\ttheta}{\tilde{\theta}}
\newcommand{\tk}{\tilde{k}}
\newcommand{\tu}{\tilde{u}}
\newcommand{\F}{\mathcal{F}}
\newcommand{\NP}{\mathcal{NP}}
\newcommand{\X}{\mathcal{X}}
\newcommand{\grandO}{\mathcal{O}}
\renewcommand{\u}{u}
\renewcommand{\v}{v}
\newcommand{\bc}{\beta}
\newcommand{\dd}{\hat{d}}
\newcommand{\oa}{\overline{a}}
%\renewcommand{\up}{{u}^+}
\newcommand{\up}{{u}^+}
\newcommand{\un}{{u}^-}
\newcommand{\cp}{{c}^+}
\newcommand{\cn}{{c}^-}
\newcommand{\dbb}{d^{bb}}
\newcommand{\ddet}{D}
\newcommand{\drob}{\mathcal{D}}
\newcommand{\nU}[1]{{\sigma_{#1}}}
\newcommand{\NU}{\sigma}
\newcommand{\opt}{opt}
\newcommand{\diaminst}{\Delta}
\newcommand{\level}{s}
\newcommand{\VF}{V(F)}
\newcommand{\roots}{R}
\renewcommand{\root}{r}

\newcommand{\dmax}{d^{max}}
\newcommand{\costmax}{\cost^{max}}
\newcommand{\Fmax}{F^{max}}
\newcommand{\cost}{c}
\newcommand{\costtilde}{\tilde{c}}
\newcommand{\Ex}{\mathbb{E}}
\newcommand{\abs}[1]{\left\lvert #1\right\rvert}
\newcommand{\card}[1]{\left\lvert #1\right\rvert}
\newcommand{\floor}[1]{\left\lfloor#1\right\rfloor}
\newcommand{\ceil}[1]{\left\lceil#1\right\rceil}
\newcommand{\distany}[2]{d(#1,#2)}
%\newcommand{\dist}[2]{\left\Vert#2 - #1\right\Vert_2}
\newcommand{\dist}[2]{\distany{#1}{#2}}
\newcommand{\distbis}[2]{d'(#1,#2)}
\newcommand{\norme}[1]{\left\Vert#1\right\Vert_2}
\DeclareMathOperator*{\conv}{conv}
\DeclareMathOperator*{\diam}{diam}
\DeclareMathOperator*{\ext}{ext}
%\DeclareMathOperator*{\argmin}{arg\,min}
%\DeclareMathOperator*{\argmax}{arg\,max}

%\newtheorem{theorem}{Theorem}
%\newtheorem{acknowledgement}{Acknowledgement}
%\newtheorem{axiom}{Axiom}
%\newtheorem{case}{Case}
%\newtheorem{claim}{Claim}
%\newtheorem{conclusion}{Conclusion}
%\newtheorem{condition}{Condition}
%\newtheorem{conjecture}{Conjecture}
%\newtheorem{corollary}{Corollary}
%\newtheorem{criterion}{Criterion}
%\newtheorem{definition}{Definition}
%\newtheorem{example}{Example}
%\newtheorem{exercise}{Exercise}
%\newtheorem{lemma}{Lemma}
%\newtheorem{notation}{Notation}
%\newtheorem{problem}{Problem}
%\newtheorem{property}{Property}
%\newtheorem{proposition}{Proposition}
%\newtheorem{remark}{Remark}
%\newtheorem{solution}{Solution}
%\newtheorem{summary}{Summary}

%\theoremstyle{TH}
\newtheorem{observation}{Observation}

\newcommand{\JO}[1]{~\\[4pt]\todo[color=blue!40,noinlinepar,size=\small]{\textbf{JO:} #1}}
\newcommand{\MP}[1]{~\\[4pt]\todo[color=green!40,noinlinepar,size=\small]{\textbf{MP:} #1}}
\newcommand{\MB}[1]{~\\[4pt]\todo[color=red!40,noinlinepar,size=\small]{\textbf{MB:} #1}}

\newcommand{\jo}[1]{{\color{blue}jo : #1}}
\renewcommand{\mp}[1]{{\color{brown}mp : #1}}
\newcommand{\mb}[1]{{\color{red}mb : #1}}

\newcommand{\SP}{\textsc{sp}\xspace}
\newcommand{\MST}{\textsc{mst}\xspace}
\newcommand{\MINEWCP}{\textsc{min-ewcp}\xspace}
\newcommand{\DELTASP}{$\Delta${-\textsc{sp}}\xspace}
\newcommand{\DELTAMST}{$\Delta${-\textsc{mst}}\xspace}
\newcommand{\DELTATSP}{$\Delta${-\textsc{tsp}}\xspace}
%\newcommand{\PBFAMILY}{\ensuremath{\textsc{min-subgraph}}\xspace}
\newcommand{\PBFAMILY}{\ensuremath{\mathcal{S}}\xspace}
\newcommand{\EVALC}{\ensuremath{\textsc{adversarial}}\xspace}
\newcommand{\MAXCUT}{\ensuremath{\textsc{max-cut}}\xspace}
\newcommand{\LISTCOL}{\ensuremath{\textsc{list-col}}\xspace}
\newcommand{\MAXDIVERSITY}{\ensuremath{\textsc{max-diversity}}\xspace}
\newcommand{\ROBUST}[1]{\ensuremath{\textsc{LocRob}\mbox{\footnotesize-#1}}\xspace}
\newcommand{\CONSERVATIVE}[1]{\ensuremath{\textsc{cons}\mbox{\footnotesize-#1}}\xspace}
\newcommand{\additionalInput}{\alpha}
\newcommand{\OPT}{\mbox{\textsc{opt}}}

\newcommand{\NPH}{$\cal{NP}$-hard\xspace}
\newcommand{\PEQNP}{$\cal{P}=\cal{NP}$\xspace}
\newcommand{\PTAS}{$\cal{PTAS}$\xspace}
\newcommand{\WONEH}{${\cal W}[1]$-hard\xspace}
\newcommand{\WONEHness}{${\cal W}[1]$-hardness\xspace}
\newcommand{\FPT}{$\cal{FPT}$\xspace}
\newcommand{\FPTAS}{$\cal{FPTAS}$\xspace}

%%notations fptas
\DeclareMathOperator*{\Val}{Val}
\newcommand{\nval}{n_{val}}
\newcommand{\nprof}{n_{\P}}
\renewcommand{\P}{\mathcal{P}} %ens de profils



\newcommand{\defproblem}[4]{\par
 \vspace{3mm}
\noindent\fbox{
 \begin{minipage}{0.96\textwidth}

   \begin{tabular*}{\textwidth}{@{\extracolsep{\fill}}lr} #1 &  \vspace{1mm} \\ \end{tabular*}
{\textbf{Input:} #2
\vspace{1mm}\\%
\textbf{Output:} #3
  \vspace{1mm}\\%
 \textbf{Goal:}} #4

 \end{minipage}
 }
 \vspace{3mm}\par
}

\usepackage{hyperref}


% Attempt to make hyperref and algorithmic work together better:
%\newcommand{\theHalgorithm}{\arabic{algorithm}}


% Definitions of handy macros can go here

\newcommand{\dataset}{{\cal D}}
\newcommand{\fracpartial}[2]{\frac{\partial #1}{\partial  #2}}

% Heading arguments are {volume}{year}{pages}{date submitted}{date published}{paper id}{author-full-names}

\jmlrheading{1}{2019}{1-48}{4/00}{10/00}{yu18}{Rose Yu, Stephan Zheng, Anima Anandkumar, Yisong Yue}

% Short headings should be running head and authors last names

\ShortHeadings{Higher-Order Tensor RNNs}{Yu et al.}
\firstpageno{1}

\begin{document}

\title{Long-Term Forecasting using  Higher-Order Tensor RNNs}

\author{\name Rose Yu \email rose@caltech.edu 
       \AND
       \name Stephan Zheng \email stephan@caltech.edu 
       \AND
      \name Anima  Anandkumar \email anima@caltech.edu 
       \AND
       \name Yisong Yue \email yyue@caltech.edu \\
        \addr Department of Computing and Mathematical Sciences\\
       California Institute of Technology\\
       Pasadena, CA 91125, USA}

\editor{Francis Bach, David Blei and Bernhard Sch{\"o}lkopf}

\maketitle

\begin{abstract}%   <- trailing '%' for backward compatibility of .sty file
  We present Higher-Order Tensor RNN (\trnn{}), a novel family of neural sequence architectures for multivariate forecasting in environments with nonlinear dynamics.
  %
  Long-term forecasting in such systems is highly challenging, since there exist long-term temporal dependencies, higher-order correlations and sensitivity to error propagation.
  %
  Our proposed recurrent architecture addresses these issues by learning the nonlinear dynamics directly using higher-order moments and higher-order state transition functions.
  %
  Furthermore, we decompose the higher-order structure using the tensor-train decomposition to reduce the number of parameters while preserving the model performance.
  %
  We theoretically establish the approximation guarantees and the variance bound for \trnn{} for general sequence inputs. We also
  demonstrate $5 \sim 12\%$  improvements for long-term prediction over general RNN and LSTM architectures on a range of simulated environments with nonlinear dynamics, as well on real-world time series data.
  %
\end{abstract}

\begin{keywords}
   Time Series, Forecasting, Tensor, RNNs, Nonlinear Dynamics
\end{keywords}


\section{Introduction}
\label{intro}
Reinforcement learning has achieved great success in areas such as Game-playing \citep{silver2018general,vinyals2019grandmaster}, robotics \cite{kober2013reinforcement}, large language models \citep{ouyang2022training}, etc.
However, due to safety concerns or physical limitations, in some real-world reinforcement learning problems, we must consider additional constraints that may influence the optimal policy and the learning process \citep{garcia2015comprehensive}.
% For example, a robotic arm must not take actions that may cause harm to itself or the environments.
A standard framework to handle such cases is the constrained Markov Decision Process (CMDP) \citep{altman1999constrained}.
Within the CMDP framework, the agent has to maximize
the expected cumulative reward while
obeying a finite number of constraints, which are usually in the form of expected cumulative cost criteria.

However, we are sometimes concerned with the problem with a continuum of constraints.
For example,
the constraints we meet might be time-evolving or subject to uncertain parameters, which
cannot be formulated as an ordinary CMDP
(see Examples \ref{Example_Time_Evolving} and  \ref{Example_Uncertain}).
In this paper we would study a generalized CMDP  
to address the above problem.  Because the constraints are not only infinite-number but also lie
in a continuous set,
the generalization is not trivial. Fortunately, we find that we can borrow the idea behind semi-infinite programming (SIP) \citep{remez1934determination, hettich1993semi} to deal with the semi-infinite constraints.
Accordingly, we propose \emph{semi-infinitely constrained Markov decision processes} (SICMDPs)
as a novel complement to the ordinary CMDP framework.
%More specifically,  an SICMDP model %, we consider 
%contains a continuum of constraints whereas an ordinary CMDP contains a finite number of constraints. 

%This generalization is natural but not trivial. However, we can brows the idea  
%The idea is quite natural and can be backtracked
%to the practice of extending linear programming to linear semi-infinite programming (LSIP) %\cite{remez1934determination, GobernaLSIO1998}.
%In addition, 
%As a complementary approach to the ordinary CMDP framework, 
%SICMDP can be used to model these problems  which cannot be described by a finite number of constraints
%that are not covered by .
%For example,
%the restrictions we consider can be time-evolving or subject to uncertain parameters
%, thus
%cannot be described by a finite number of constraints but a continuum of constraints 
%(see Examples \ref{Example_Time_Evolving} and  \ref{Example_Uncertain}).

We also present two reinforcement learning algorithms to solve SICMDPs called SI-CRL and SI-CPO, respectively.
SI-CRL is a model-based reinforcement learning algorithm designed for tabular cases, and SI-CPO is a policy optimization algorithm for non-tabular cases.
% and analyze its performance both theoretically and empirically.
The main challenge is that we need to deal with a continuum of constraints, thus reinforcement learning algorithms for ordinary CMDPs do not work anymore.
In SI-CRL, we tackle this difficulty by first transforming the reinforcement learning problem to an equivalent LSIP problem, which can then be solved using methods in the LSIP literature like the dual exchange methods \citep{Hu1990,reemtsen1998numerical}.
In SI-CPO, we resort to the idea of cooperative stochastic approximation developed in \cite{lan2020algorithms, wei2020comirror}.
As far as we know, we are the first to introduce tools from semi-infinitely programming (SIP) into the reinforcement learning community for solving constrained reinforcement learning problems.

% To the best of our knowledge, we are the first to apply tools from semi-infinitely programming (SIP) to solve reinforcement learning problems.
Furthermore, we give theoretical analysis for both SI-CRL and SI-CPO.
We decompose the error of SI-CRL into two parts: the statistical error from approximating the true SICMDP with an offline dataset and the optimization error due to the fact that the solution of the LSIP problem obtained by the dual exchange method is inexact.
On the optimization side, we show that the iteration complexity of SI-CRL is $O\left(\left\{\mathrm{diam}(Y)L\sqrt{|\gS|^2|\gA|m}/\left[(1-\gamma)\epsilon\right]\right\}^m\right)$.
On the statistical side, we show that the sample complexity of SI-CRL is $\widetilde O\left(\frac{|S|^2|A|^2}{\epsilon^2(1-\gamma)^3}\right)$ if the offline dataset is generated by a generative model, and $\widetilde O\left(\frac{|S||A|}{\nu_{\min} \epsilon^2(1-\gamma)^3}\right)$ if the dataset is generated by a probability measure $\nu$ as considered in \cite{chen2019information}.
Here $\widetilde O$ means that all logarithm terms are discarded.
For SI-CPO, things become a little more complicated because other than the statistical error and the optimization error, we also need to consider the function approximation error, which comes from imperfect policy parametrizations.
It is shown if the function approximation error can be controlled to $O(\epsilon)$ order, the iteration complexity of SI-CPO is $\widetilde{O}\left(\frac{1}{\epsilon^2(1-\gamma)^6}\right)$ and the sample complexity of SI-CPO is $\widetilde{O}(\frac{1}{\epsilon^4(1-\gamma)^{10}})$.
Here our iteration complexity bound is equivalent to a typical $\widetilde O(1/\sqrt{T})$ global convergence rate.

We perform a set of numerical experiments to illustrate the SICMDP model and validate our proposed algorithms.
Specifically, we examine two numerical examples, namely the discharge of sewage and ship route planning.
Through the discharge of sewage example, we show the advantage of the SICMDP framework over the CMDP baseline obtained by naive discretization in modeling realistic sequential decision-making problems.
Moreover, we demonstrate the effectiveness of the SI-CRL and SI-CPO algorithms in such tabular environments. 
In the ship route planning example, we illustrate the benefits of the SICMDP framework and the ability of the SI-CPO algorithm to address complex continuous control tasks involving continuous state spaces with modern deep reinforcement learning techniques.

% In summary, our contributions are listed as follows.
% First, we present the SICMDP model, which can be viewed as a generalization of the ordinary CMDP model.
% Second, we propose an algorithm to perform reinforcement learning for SICMDPs, which is called SI-CRL, and we believe that we are the first to apply tools from SIP
% to solve reinforcement learning problems.
% Third, we give a theoretical analysis of SI-CRL and identify both its sample complexity and iteration complexity.
% In addition, we perform numerical experiments to illustrate the SICMDP model and validate the SI-CRL algorithm.
% \{This paragraph can be removed!!! \}





%
\section{Related Work}
\label{related}
\textbf{Related work}:
% Object detection related datasets/algo in non-medical domain
% Locally labeled CXR dataset
A few CXR datasets have localized abnormality annotations \cite{shih2019augmenting,filice2020crowdsourcing,jaeger2014two} that are curated manually. These are high quality gold standard ground truth datasets but tend to be smaller in scale (< 30,000 images) and have a narrow coverage, with typically only 1-2 labels. In addition, since most labeling efforts only have abnormality semantics attached, no direct relationships with the affected anatomical locations are available. 

%MEHDI: repeated concepts from above. I am removing the following: 

%The lack of anatomic semantics in the annotation is a limitation for complex multi-modal clinical reasoning work, e.g., differential diagnosis, since clinicians often integrate information along anatomical lines, and for downstream report generation tasks, which often requires describing not only the abnormality but also correctly communicate the location of the abnormalities (and medical devices) to the receiving clinicians. 

Two recent CXR datasets have labels for anatomies described in the reports. In \cite{datta2020dataset}, a small manually annotated dataset (2000 reports) included 10 abnormalities that are individually associated with 29 unique spatial locations (anatomies) at the report level. Another CXR dataset has automatically extracted abnormality and anatomy labels as disconnected concepts that are only correlated at the study level from  160,000 reports using a supervised NLP algorithm \cite{bustos2020padchest}. This was trained on a smaller set of manually annotated data. Neither datasets contain localized annotations for the associated CXR images, nor any comparison relation annotations between sequential exams, both of which are available in the Chest ImaGenome dataset. In Table \ref{tab:related}, we present a comparison of our Chest ImagGenome dataset with other datasets available in the literature.

% Table -- Kashyap

% MEdical imaging datasets to go here: Discussed that we will only focus on cxr datasets that are available for this paper. 
% \caption{\color{red} Kashyap, feel free to continue with the table. We should remove the questionmarks and add a line for our dataset (since all others are not graph). For longer text, using abbreviations and explaining them in the caption often works better. If fill in the values is not possible, it is better to remove the table altogether.}


\begin{table}[t!]
\caption{Summary of existing chest X-ray datasets}
\resizebox{\textwidth}{!}{%
\begin{tabular}{@{}lllllllll@{}}
\toprule
\textbf{Dataset} & \textbf{Annotation Level} & \textbf{Annotation Method} & \textbf{Num Labels} & \textbf{Anatomy Labeled} & \textbf{Graph} & \textbf{Dataset Size} & \textbf{Temporal Labels} & \textbf{Reports} \\ \midrule
SIIM-ACR Pneumothorax Segmentation \cite{filice2020crowdsourcing} & Segmentation & Manual + augmented & 1 & No & No & 12,047 & No & No \\
RSNA Pneumonia Detection Challenge   \cite{shih2019augmenting} & Bounding Boxes & Manual & 1 & No & No & 30,000 & No & No \\
Indiana University Chest X-ray collection \cite{demner2016preparing} & Global & Automated & 10 & No & No & 3,813 & No & Yes \\
NIH CXR dataset \cite{wang2017chestx} & Global & Automated & 14 & No & No & 112,120 & No & No \\
PLCO \cite{team2000prostate} & Global & Automated & 24 & Yes & No & 236,000 & Yes & No \\
Stanford CheXpert \cite{irvin2019chexpert} & Global & Automated & 14 & No & No & 224,316 & No & No \\
MIMIC-CXR \cite{johnson2019mimic} & Global & Automated & 14 & No & No & 377,110 & No & Yes \\
Dutta \cite{datta2020dataset} & Global & Manual & 10 & Yes & Yes & 2,000 & No & Yes \\
PadChest \cite{bustos2020padchest} & Global & Manual + automated & 297 & Yes & No & 160,868 & No & Yes \\
Montgomery County Chest X-ray   \cite{jaeger2014two} & Segmentation & Manual & 1 & Yes & No & 138 & No & No \\
Shenzen Hospital Chest X-ray   \cite{jaeger2014two} & Segmentation & Manual & 1 & Yes & No & 662 & No & No \\  \hline \hline
\textbf{Chest ImaGenome} & Bounding Boxes & Automated & 131 & Yes & Yes & 242,072 & Yes & Yes \\
\bottomrule
\end{tabular}%
}
\label{tab:related}
\vspace{-0.4cm}
\end{table}
% removed (Derived from MIMIC-CXR \cite{johnson2019mimic}) % makes table really small

%
\section{Higher-Order Tensor RNNs}
\label{trnn}
\paragraph{Forecasting Nonlinear Dynamics}
%
Our goal is to learn an efficient forecasting model for \ti{continuous multivariate time series} in environments with nonlinear dynamics.
%
The state $\V{x}_t \in \mathR^d$ of such systems evolves over time using a set of \ti{nonlinear} differential equations:
%
\eq{\label{eq:dynamics}
\brckcur{\xi^i\brck{\V{x}_t, \fr{d\V{x}}{dt}, \fr{d^2\V{x}}{dt^2}, \ldots; \phi} = 0 }_i,
}
%
where $\xi^i$ can be an arbitrary (smooth) function of the state $\V{x}_t$ and its derivatives. 
% 
Continuous time dynamics are usually described by differential equations while  difference equations are employed for discrete time. 
% 
In continuous time, a classic example is the first-order Lorenz attractor, whose realizations showcase the ``butterfly-effect'', a characteristic set of double-spiral orbits. 
% 
In discrete-time, a non-trivial example is the 1-dimensional Genz dynamics, whose difference equation is:
%
\eq{\label{eq:genzprodpeak}
	x_{t+1} = \brck{c^{-2} + (x_t + w)^2}^{-1}, \hspace{10pt}  c,w \in [0,1],
}
where $x_t$ denotes the system state at time $t$ and $c,w$ are the parameters. Due to the nonlinear nature of the dynamics, such systems exhibit higher-order correlations, long-term dependencies and sensitivity to error propagation, and thus form a challenging setting for forecasting.
% 
% \ryedit{Add visualization of Genz dynamics here?}

Given a sequence of initial states $\V{x}_0\ldots \V{x}_t$, the forecasting problem aims to learn a dynamics model $F$ that outputs a sequence of future states $\V{x}_{t+1} \ldots \V{x}_T$. 
\eq{\label{eq:forecast}
F: \brck{\V{x}_0\ldots \V{x}_t} \mapsto \brck{\V{y}_{t} \ldots \V{y}_T},
%
\hspace{10pt} \V{y}_t = \V{x}_{t+1},
}
The system is governed by some unknown dynamics. Hence, accurately approximating the dynamics is critical to learning a good forecasting model and making predictions for long time horizons.
 
 
\begin{figure*}[t]
\begin{center}
\begin{minipage}[t]{0.62\linewidth}
\centering		\includegraphics[width=\linewidth]{Figure/tlstm.png}
\caption{\trnn{} within a seq2seq model. Both encoder  and decoder contain higher-order recurrent cells. The augmented state $\V{s}_{t-1}$ (grey) takes in past $L$ hidden states (blue) and forms a higher-order tensor. \trnn{} (red)  factorizes the tensor and outputs the next hidden state.}
\label{fig:seq2seq}
\end{minipage}
\hspace{0.02\linewidth}
\begin{minipage}[t]{0.33\linewidth}
\centering		\includegraphics[width=\linewidth]{Figure/tensor_train.png}
\caption{A \trnn{} cell. The augmented state $\V{s}_{t-1}$ (grey) forms a higher-order tensor, which is then factorized to output the next hidden state.}
\label{fig:ttrnn}
\end{minipage}
\end{center}
\vspace{-5mm}
\end{figure*}

\paragraph{First-order Markovian Models}
%
In deep learning, popular approaches such as recurrent neural networks (RNNs) employ first-order hidden-state models to approximate the dynamics. An RNN with a single  cell recursively  computes a hidden state $\V{h}_t$ using the most recent hidden state $\V{h}_{t-1}$, generating  the output $\V{y}_t$ from the hidden state $\V{h}_t$ :
%
\eq{\label{eq:rnn}
\V{h}_t = f(\V{x}_t, \V{h}_{t-1}; \theta_f),\hspace{10pt} \V{y}_t = g(\V{h}_t; \theta_g),
}
%
where $f$ is the state transition function, $g$ is the output  function and $\{\theta_f, \theta_g\}$ are the corresponding  model parameters. A common parametrization scheme for \refn{eq:rnn} applies a nonlinear  activation function such as sigmoid $\sigma$ to a linear map of $\V{x}_t$ and $\V{h}_{t-1}$ as:
%
\eq{
\V{h}_t &= \sigma(W^{hx} \V{x}_t + W^{hh} \V{h}_{t-1} + \V{b}^h), \quad
\V{x}_{t+1} = W^{xh} \V{h}_t + \V{b}^x,
}
where $W^{hx}, W^{xh}$ and $W^{hh}$ are  the transition weight matrices and $\V{b}^h, \V{b}^x$ are the biases.

RNNs have many different variations, including LSTMs \citep{hochreiter1997long} and GRUs \citep{chung2014empirical}. 
% For instance, LSTM cells use a memory-state, which mitigate the ``exploding gradient'' problem and allow RNNs to propagate information over longer time horizons.
%
Although a RNN can approximate any function in theory, its hidden state $\V{h}_t$  only depends on the previous state $\V{h}_{t-1}$ and the input $\V{x}_t$. Such models do not explicitly capture higher-order dynamics and only  implicitly encode long-term dependencies between all historical states $\V{h}_{0} \ldots \V{h}_{t}$. This limits the representation power of RNNs, especially for forecasting in environments with nonlinear dynamics. Hence, instead of using a wide RNN with many hidden units, we exploit the recurrent cell to design higher-order tensor RNNs that can approximate complex non-linear governing equations. 


% \paragraph{The Debate Between Deep and Shallow}
% While both the deep and shallow networks preserve the universal approximation property, the folk wisdom is that shallow networks memorize well but generalize poorly. Theoretically, \cite{mhaskar2017and} have shown deep networks to have lower number of sample complexity. Empirically, hierarchical architectures such as residual networks \cite{he2016deep} and dense networks \cite{huang2017densely} are quite successful. 

% We take an analogous approach in our design principle but focus on the temporal dimension. We study the universal approximation property of RNNs for representing the underlying dynamics. 

\subsection{Higher-Order Non-Markovian Models}

To effectively learn nonlinear dynamics with higher-order temporal dependency, we propose a family of models that generalizes standard RNNs: higher-order recurrent neural networks, or  \trnn{}. 
%
We design \trnn{}s with two goals in mind: explicitly modeling 1) $L$-order Markov processes with $L$ steps of temporal memory and 2) polynomial interactions between the hidden states $\V{h}_{\cdot}$ and $\V{x}_t$.

First, we consider longer ``history'': we keep length $L$ historic states: $\V{h}_{t},\cdots, \V{h}_{t-L}$:
%
\eq{
\V{h}_t = f( \V{x}_t , \V{h}_{t-1}, \cdots, \V{h}_{t-L}; \theta_f)
\label{eqn:high_order_markov}
}
%
where $f$ represents the state transition function.  In principle, early work \citep{giles1989higher} has shown that with a large enough hidden state size, such recurrent structures are capable of approximating any dynamical system.

Second, to learn the nonlinear dynamics $\xi$ efficiently, we also use higher-order moments to approximate the state transition function.
%
We use an augmented state $\V{s}$, where we mute the subscript of $\V{s}_{t-1}$ for notation simplicity.:
\begin{equation}
	% \V{s}_{t-1} \otimes \cdots \otimes \V{s}_{t-1} \quad
	\V{s}^T = [1 \hspace{5pt} \V{h}_{t-1}^\top \hspace{5pt} \ldots \hspace{5pt} \V{h}_{t-L}^\top ]
\end{equation}
which concatenates $L$ previous hidden states.
% 
To compute $\V{h}_t$, we construct a $P$-dimensional transition \ti{weight tensor} to model degree-$P$ polynomial interactions between hidden states:
%Hence, the \trnn{} with standard RNN cell is defined by:
%
\begin{align}
[\V{h}_{t}]_\alpha = \phi(W^{hx}_\alpha\V{x}_t+  
    \sum_{i_1,\cdots, i_p}\T{W}_{\alpha i_1 \cdots i_{P}}  \underbrace{\V{s}_{i_1} \otimes\cdots\otimes \V{s}_{i_p} }_{P} )\nonumber
%
\label{eqn:tensor_rnn}
\end{align}
%
where $\alpha$ indices the hidden dimension, $i_\cdot$ indices the higher-order terms and $P$ is the total  polynomial order. We included the bias unit $1$ in $\V{s}$ to account for the first order term, so that  $\V{s}_{i_1} \otimes\cdots\otimes \V{s}_{i_p} = [1, \V{h}_t, \V{h}_t\V{h}_{t-1},\cdots]$ can include all  polynomial expansions of hidden states up to order $P$. 

%
% where $\otimes$ is the tensor product.
%
%
The \trnn{} with LSTM cell, or ``\tlstm{}'', is defined analogously as:
\begin{align}
[\V{i}_t, \V{g}_t, &\V{f}_t, \V{o}_t]_\alpha = \sigma (W^{hx}_\alpha \V{x}_t + \sum_{i_1,\cdots, i_p}\T{W}_{\alpha i_1 \cdots i_{P}}  \underbrace{\V{s}_{i_1} \otimes\cdots\otimes \V{s}_{i_P} }_{P} ), \\
%
& \V{c}_t = \V{c}_{t-1} \circ \V{f}_t +  \V{i}_t\circ \V{g}_t,
%
\qquad
%
\V{h}_t = \V{c}_t \circ \V{o}_t \nonumber
%
\end{align}
%
where $\circ$ denotes the Hadamard product. Note that the bias units are again included.

 \trnn{} is a basic  unit that can be incorporated in most of the existing recurrent neural architectures such as convolutional RNN \citep{xingjian2015convolutional} and hierarchical RNN \citep{chung2016hierarchical}. In this work, we use  \trnn{} as a module for sequence-to-sequence (seq2seq) framework \citep{sutskever2014sequence} in order to perform long-term forecasting.


As shown in Figure \ref{fig:seq2seq}, seq2seq models consist of an encoder-decoder pair. The encoder takes an input sequence and learns a hidden representation. The decoder initializes with this hidden representation and generates an output sequence. Both the encoder and the decoder contain multiple layers of higher-order tensor recurrent cells (red).
% 
The augmented state $\V{s}_{t-1}$ (grey) concatenates the past $L$ hidden states;
% 
the \trnn{} cell takes $\V{s}_{t-1}$ and outputs the next hidden state. 
% 
The encoder encodes the initial states $x_{0}, \ldots, x_{t}$ and the decoder predicts $x_{t+1}, \ldots, x_{T}$. 
% 
For each time step $t$, the decoder uses its previous prediction $\V{y}_t$ as an input.
%
\subsection{Dimension Reduction with Tensor-Train}
% 
Unfortunately, due to the ``curse of dimensionality'', the number of parameters in $\T{W}_\alpha$ with hidden size $H$ grows exponentially as $O(HL^P)$, which makes the higher-order model prohibitively large to train. To overcome this difficulty, we  utilize   \textit{tensor networks} to approximate the weight tensor. Such networks encode a structural decomposition of tensors into low-dimensional components and have been shown to provide the most general approximation to smooth tensors \citep{orus2014practical}.
%
The most commonly used tensor networks are \textit{linear tensor networks} (LTN), also known as \textit{tensor-trains} in numerical analysis or \textit{matrix-product states} in quantum physics \citep{oseledets2011tensor}.

A tensor train model decomposes a $P$-dimensional tensor $\T{W}$ into a network of sparsely connected low-dimensional tensors $\{\T{A}^p \in \R^{r_{p-1} \times n_p \times r_{p}} \}$ as:
%
\begin{equation*}
\T{W}_{i_1 \cdots i_P} =
\sum_{\alpha_1 \cdots \alpha_{P-1}}
\T{A}^1_{\alpha_0 i_1 \alpha_1}%
\T{A}^2_{\alpha_1 i_2 \alpha_2}%
\cdots%
\T{A}^P_{\alpha_{P-1} i_P \alpha_P}
\nonumber
\end{equation*}
%
 with $ \alpha_0 = \alpha_P = 1$, as depicted in Figure (\ref{fig:ttrnn}). When $r_0 = r_{P} = 1$ the $\{r_p\}$ are called the tensor-train rank.
%
With tensor-train decomposition, we can reduce the number of parameters of \trnn{} from $(HL+1)^{P}$ to $(HL+1)R^2P$, with $R = \max_p{r_p}$ as the upper bound on the tensor-train rank.
%
Thus, a major benefit of tensor-train is that they \textit{do not} suffer from the curse of dimensionality, which is in sharp contrast to many classical tensor decomposition models, such as the Tucker decomposition.

%\aacomment{theory should be a separate section}



%
\section{Approximation Theorem for HOT-RNNs}
\label{thm}
\begin{figure}
    \centering
    \includegraphics[width=0.6\textwidth]{graphs/convex-concordance} %0.45
    \caption{Strong convexity v.s.~self-concordance. Black curve: population risk; colored dot: reference point; colored dashed curve: quadratic approximation at the corresponding reference point.}
    \label{fig:convex_concordance}
\end{figure}

\subsection{Preliminaries}
\label{sub:preliminary}

\myparagraph{Notation}
We denote by $\grad(\theta; z) := \nabla_\theta \score(\theta; z)$ the gradient of the loss at $z$ and $H(\theta; z) := \nabla_\theta^2 \score(\theta; z)$ the Hessian at $z$.
Their population versions are $\grad(\theta) := \Expect[\grad(\theta; Z)]$ and $H(\theta) := \Expect[H(\theta; Z)]$, respectively.
We assume standard regularity assumptions so that $\grad(\theta) = \nabla_\theta L(\theta)$ and $H(\theta) = \nabla_\theta^2 L(\theta)$.
We write $H_\star := H(\theta_\star)$.
Note that the two optimality conditions then read $\grad(\theta_\star) = 0$ and $H_\star \succ 0$.
It follows that $\lambda_\star := \lambda_{\min}(H_\star) > 0$ and $\lambda^\star := \lambda_{\max}(H_\star) > 0$.
Furthermore, we let $G(\theta; z) := S(\theta; z) S(\theta; z)^\top$ and $G(\theta) := \Expect[\grad(\theta; Z)\grad(\theta; Z)^\top]$ be the autocorrelation matrices of the gradient.
We write $G_\star := G(\theta_\star)$.
We define their empirical quantities as $L_n(\theta) := n^{-1} \sum_{i=1}^n \score(\theta; Z_i)$, $\grad_n(\theta) := n^{-1} \sum_{i=1}^n \grad(\theta; Z_i)$, $H_n(\theta) := n^{-1} \sum_{i=1}^n H(\theta; Z_i)$, and $G_n(\theta) := n^{-1} \sum_{i=1}^n G(\theta; Z_i)$.
The first step of our analysis is to localize the estimator to a \emph{Dikin ellipsoid} at $\theta_\star$ of radius $r$, i.e.,
\begin{align*}
    \Theta_r(\theta_\star) := \left\{\theta \in \Theta: \norm{\theta - \theta_\star}_{H_\star} < r \right\},
\end{align*}
where, given a positive semi-definite matrix $J$, we let $\norm{x}_J := \norm{J^{1/2} x}_2 = \sqrt{x^\top J x}$.

\myparagraph{Effective dimension}
A quantity that plays a central role in our analysis is the \emph{effective dimension}.
\begin{definition}
\label{def:effective_dim}
    We define the effective dimension to be
    \begin{align}
        d_\star := \Tr( H_\star^{-1/2} G_\star H_\star^{-1/2} ).
    \end{align}
\end{definition}
The effective dimension appears recently in non-asymptotic analyses of (penalized) M-estimation; see, e.g., \citep{spokoiny2017penalized,ostrovskii2021finite}.
It better characterizes the complexity of the parameter space $\Theta$ than the parameter dimension $d$.
When the model is well-specified, it can be shown that $H_\star = G_\star$ and thus $d_\star = d$.
When the model is misspecified, it can be much smaller than $d$ depending on the spectra of $H_\star$ and $G_\star$.
Moreover, it is closely connected to classical asymptotic theory of M-estimation under model misspecification---it is the trace of the limiting covariance matrix of $\sqrt{n}H_n(\theta_n)^{1/2}(\theta_n - \theta_\star)$;
see \Cref{sub:discussion} for a thorough discussion.

\myparagraph{Generalized self-concordance}
We will use the notion of \emph{self-concordance} from convex optimization in our analysis.
Self-concordance originated from the analysis of the interior-point and Newton-type convex optimization methods \citep{yurii1994interior}.
It was later modified by \citet{bach2010self}, which we call the \emph{pseudo self-concordance}, to derive finite-sample bounds for the generalization properties of the logistic regression.
Recently, \citet{sun2019generalized} proposed the \emph{generalized self-concordance} which unifies these two notions.
For a function $f: \reals^d \to \reals$, we define $D_x f(x)[u] := \frac{\D}{\D t} f(x + tu) |_{t = 0}$, $D_x^2 f(x)[u, v] := D_x (D_x f(x)[u])[v]$ for $x, u, v \in \reals^d$, and $D_x^3 f(x)[u, v, w]$ similarly.
\begin{definition}[Generalized self-concordance]
\label{def:general_self_concordance}
    Let $\calX \subset \reals^d$ be open and $f: \calX \rightarrow \reals$ be a closed convex function.
    For $R > 0$ and $\nu > 0$, we say $f$ is $(R, \nu)$-generalized self-concordant on $\calX$ if
    \begin{align*}
        \abs{D_x^3 f(x) [u, u, v]} \le R \norm{u}_{\nabla^2 f(x)}^2 \norm{v}_{\nabla^2 f(x)}^{\nu-2} \norm{v}_2^{3-\nu}
    \end{align*}
    with the convention $0/0 = 0$ for the case $\nu < 2$ and $\nu > 3$.
    Recall that $\norm{u}_{\nabla^2 f(x)}^2 := u^\top \nabla^2 f(x) u$.
\end{definition}

\myparagraph{Remark}
When $\nu = 2$ and $\nu = 3$, this definition recovers the pseudo self-concordance and the standard self-concordance, respectively.

In contrast to strong convexity which imposes a gross lower bound on the Hessian, generalized self-concordance specifies the rate at which the Hessian can vary, leading to a finer control on the Hessian.
Concretely, it allows us to bound the Hessian in a neighborhood of $\theta_\star$ with the Hessian at $\theta_\star$, which is key to controlling $H_n(\theta_n)$.
We illustrate the difference between them in \Cref{fig:convex_concordance}.
As we will see in \Cref{sub:main_results}, thanks to the generalized self-concordance, we are able to remove the direct dependency on $\lambda_\star$ in our confidence set.
To the best of our knowledge, this is the first work extending classical results for M-estimation to generalized self-concordant losses.

\myparagraph{Concentration of Hessian}
One key result towards deriving our bounds is the concentration of empirical Hessian, i.e., $(1 - c_n(\delta))H(\theta) \preceq H_n(\theta) \preceq (1 + c_n(\delta)) H(\theta)$ with probability at least $1 - \delta$.
When the loss function is of the form $\ell(\theta; z) := \ell(y, \theta^\top x)$ (e.g., GLMs), the empirical Hessian reads $H_n(\theta) = n^{-1} \sum_{i=1}^n \ell''(Y_i, \theta^\top X_i) X_i X_i^\top$ where $\ell''(y, \bar y) := \D^2 \ell(y, \bar y) / \D \bar y^2$, which is of the form of a sample covariance.
Assuming $X$ to be sub-Gaussian, \citet{ostrovskii2021finite} obtained a concentration bound for $H_n(\theta_\star)$ with $c_n(\delta) = O(\sqrt{(d + \log{(1/\delta)})/n})$ via the concentration bound for sample covariance \citep[Thm.~5.39]{vershynin2010introduction}.
For general loss functions, such a special structure cannot be exploited.
We overcame this challenge by the matrix Bernstein inequality \citep[Thm.~6.17]{wainwright2019high}, obtaining a sharper concentration bound with $c_n(\delta) := O(\sqrt{\log{(d/\delta)}/n})$.
Note that the matrix Bernstein inequality has been used to control the empirical Hessian of kernel ridge regression with random features \citep[Prop.~6]{rudi2017generalization} and later extended to regularized empirical risk minimization \citep[Lem.~30]{marteau2019beyond}.
However, their results require the regularization parameter to be strictly positive (otherwise the bounds are vacuous) and the sample Hessian to be bounded.
On the contrary, our technique allows for zero regularization and unbounded Hessian as long as the Hessian satisfies a matrix Bernstein condition.
Moreover, combining generalized self-concordance with matrix Bernstein, we are able to show the concentration of $H_n(\theta_n)$ around $H_\star$ for general losses, which is itself a novel result.

\subsection{Assumptions}
\label{sub:assumption}

Our key assumption is the generalized self-concordance of the loss function.
\begin{assumption}[Generalized self-concordance]
\label{asmp:self_concordance}
    For any $z \in \calZ$, the scoring rule $\score(\cdot; z)$ is $(R, \nu)$-generalized self-concordant for some $R > 0$ and $\nu \ge 2$.
    Moreover, $\risk(\cdot)$ is also $(R, \nu)$-generalized self-concordant.
\end{assumption}

\myparagraph{Remark}
If $\score(\cdot; z)$ is generalized self-concordant with $\nu = 2$, so is $\risk(\cdot)$.

Many loss functions in statistical machine learning satisfy this assumption.
We give in \Cref{sub:examples} examples from generalized linear models and score matching.


In order to control the empirical gradient $\grad_n(\theta)$, we assume that the normalized gradient at $\theta_\star$ is sub-Gaussian.
\begin{assumption}[Sub-Gaussian gradient]
\label{asmp:sub_gaussian}
    There exists a constant $K_1 > 0$ such that the normalized gradient at $\theta_\star$ is sub-Gaussian with parameter $K_1$, i.e., $\lVert G_\star^{-1/2} \grad(\theta_\star; Z) \rVert_{\psi_2} \le K_1$.
    Here $\norm{\cdot}_{\psi_2}$ is the sub-Gaussian norm whose definition is recalled in \Cref{sec:tools}.
\end{assumption}

When the loss function is of the form $\ell(\theta; z) = \ell(y, \theta^\top x)$, we have $S(\theta; Z) = \ell'(Y, \theta^\top X) X$.
As a result, \Cref{asmp:sub_gaussian} holds true if (i) $\ell'(Y, \theta_\star^\top X)$ is sub-Gaussian and $X$ is bounded or (ii) $\ell'(Y, \theta_\star^\top X)$ is bounded and $X$ is sub-Gaussian.
For least squares with $\ell(y, \theta^\top x) = \frac12 (y - \theta^\top x)^2$, the derivative $\ell'(Y, \theta_\star^\top X) = \theta_\star^\top X - Y$ is the negative residual.
\Cref{asmp:sub_gaussian} is guaranteed if the residual is sub-Gaussian and $X$ is bounded.
For logistic regression with $\ell(y, \theta^\top x) = -\log{\sigma(y\cdot \theta^\top x)}$ where $\sigma(u) = (1 + e^{-u})^{-1}$, the derivative $\ell'(Y, \theta_\star^\top X) = [\sigma(Y \cdot \theta_\star^\top X) - 1]Y \in [-1, 1]$ is bounded.
Thus, \Cref{asmp:sub_gaussian} is guaranteed if $X$ is sub-Gaussian.

In order to control the empirical Hessian, we assume that the Hessian of the loss function satisfies the matrix Bernstein condition in a neighborhood of $\theta_\star$.

\begin{assumption}[Matrix Bernstein of Hessian]
\label{asmp:bernstein}
    There exist constants $K_2, r > 0$ such that, for any $\theta \in \Theta_{r}(\theta_\star)$, the standardized Hessian
    \begin{align*}
        H(\theta)^{-1/2} H(\theta; Z) H(\theta)^{-1/2} - I_d
    \end{align*}
    satisfies a Bernstein condition (defined in \Cref{sec:tools}) with parameter $K_2$. Moreover,
    \begin{align*}
        \sigma_H^2 := \sup_{\theta \in \Theta_{r}(\theta_\star)} \norm{\Var\left( H(\theta)^{-\frac12}H(\theta; Z)H(\theta)^{-\frac12} \right)}_2 < \infty,
    \end{align*}
    where $\norm{\cdot}_2$ is the spectral norm and $\Var(J) := \Expect[JJ^\top] - \Expect[J] \Expect[J]^\top$.
    By convention, we let $\Theta_0(\theta_\star) = \{\theta_\star\}$.
\end{assumption}

\subsection{Main Results}
\label{sub:main_results}

We now give simplified versions of our main theorems.
We use $C_\nu$ to represent a constant depending only on $\nu$ that may change from line to line; and $C_{K_1, \nu}$ similarly.
We use $\lesssim$ and $\gtrsim$ to hide constants depending only on $K_1, K_2, \sigma_H, \nu$.
The precise versions can be found in \Cref{sec:proofs}.
Recall that $\lambda_\star := \lambda_{\min}(H_\star)$ and $\lambda^\star := \lambda_{\max}(H_\star)$.
\begin{theorem}\label{thm:risk_bound_generalized}
    Let $\nu \in [2, 3)$.
    Under \Cref{asmp:self_concordance,asmp:sub_gaussian,asmp:bernstein} with $r = 0$, it holds that,
    whenever
    \begin{align*}
        n \gtrsim \log{(2d/\delta)} + \lambda_\star^{-1} \left[ R^2 d_\star \log{(e/\delta)} \right]^{1/(3-\nu)},
    \end{align*}
    the empirical risk minimizer $\theta_n$ uniquely exists and satisfies, with probability at least $1 - \delta$,
    \begin{align}\label{eq:conf_bound}
        \norm{\theta_n - \theta_\star}^2_{H_\star} \lesssim \log{(e/\delta)} \frac{d_\star}{n}.
    \end{align}
\end{theorem}

With a local matrix Bernstein condition, we can replace $H_\star$ by $H_n(\theta_n)$ in \eqref{eq:conf_bound} and obtain a finite-sample version of the Wald confidence set.
\begin{theorem}\label{thm:conf_set}
    Let $\nu \in [2, 3)$.
    Suppose the same assumptions in \Cref{thm:risk_bound_generalized} hold true.
    Furthermore, suppose that \Cref{asmp:bernstein} holds with $r = C_\nu \lambda_\star^{(3-\nu)/2} / R$.
    Let $\calC_{\text{Wald}, n}(\delta)$ be
    \begin{align}\label{eq:my_conf_set}
        \left\{\theta \in \Theta: \norm{\theta - \theta_n}_{H_n(\theta_n)}^2 \le C_{K_1,\nu} \frac{d_\star}{n} \log{\frac{e}{\delta}} \right\}.
    \end{align}
    Then we have $\Prob(\theta_\star \in \calC_{\text{Wald}, n}(\delta)) \ge 1 - \delta$ whenever
    \begin{align}\label{eq:n_large_enough}
        n \gtrsim \log{\frac{2d}{\delta}} + d\log{n} + \lambda_\star^{-1}\left[ R^2 d_\star \log{\frac{e}{\delta}} \right]^{\frac1{3-\nu}}.
    \end{align}
\end{theorem}

\myparagraph{Remark}
In the precise versions of \Cref{thm:risk_bound_generalized,thm:conf_set}, the term $d_\star \log{(e/\delta)}$ in the bounds \eqref{eq:conf_bound} and \eqref{eq:my_conf_set} should be replaced by $d_\star + \log{(e/\delta)} \lVert G_\star^{1/2} H_\star^{-1} G_\star^{1/2} \rVert_2$, which almost match the misspecified Cram\'er-Rao lower bound \citep[e.g.,][Thm.~1]{fortunati2016misspecified} up to a constant factor.

\Cref{thm:conf_set} suggests that the tail probability of $\norm{\theta_n - \theta_\star}_{H_n(\theta_n)}^2$ is governed by a $\chi^2$ distribution with $d_\star$ degrees of freedom, which coincides with the asymptotic result.
In fact, according to \citet{huber1967under}, under suitable regularity assumptions, it holds that $\sqrt{n} H_n(\theta_n)^{1/2}(\theta_n - \theta_\star) \rightarrow_d W \sim \mathcal{N}(0, H_\star^{-1/2} G_\star H_\star^{-1/2})$ which implies that
\begin{align*}
    n(\theta_n - \theta_\star)^\top H_n(\theta_n) (\theta_n - \theta_\star) \rightarrow_d W^\top W.
\end{align*}
This induces an asymptotic confidence set with a similar form of \eqref{eq:my_conf_set} and radius $O(\Expect[W^\top W] / n) = O(d_\star / n)$.
Our result characterizes the \emph{critical sample size} enough to enter the asymptotic regime.

From \Cref{thm:conf_set} we can also derive a finite-sample version of the LR confidence set.
\begin{corollary}\label{cor:lr_conf_set}
    Let $\nu \in [2, 3)$.
    Suppose the same assumptions in \Cref{thm:conf_set} hold true.
    Let $\calC_{\text{LR}, n}(\delta)$ be
    \begin{align}\label{eq:lr_conf_set}
        \left\{\theta \in \Theta: 2[L_n(\theta) - L_n(\theta_n)] \le C_{K_1,\nu} \frac{d_\star}{n} \log{\frac{e}{\delta}} \right\}.
    \end{align}
    Then we have $\Prob(\theta_\star \in \calC_{\text{LR}, n}(\delta)) \ge 1 - \delta$ whenever
    \begin{align*}
        n \gtrsim \log{\frac{2d}{\delta}} + d\log{n} + \lambda_\star^{-1}\left[ R^2 d_\star \log{\frac{e}{\delta}} \right]^{\frac1{3-\nu}}.
    \end{align*}
\end{corollary}

We give the proof sketches of \Cref{thm:risk_bound_generalized}, \Cref{thm:conf_set}, and \Cref{cor:lr_conf_set} here and defer their full proofs to \Cref{sec:proofs}.
We discuss in~\Cref{sub:discussion} 
how our proof techniques and theoretical results complement and improve on previous works.

We start by showing the existence and uniqueness of $\theta_n$.
The next result shows that $\theta_n$ exists and is unique whenever the quadratic form $\grad_n(\theta_\star)^\top H_n^{-1}(\theta_\star) \grad_n(\theta_\star)$ is small.
Note that this quantity is also known as Rao's score statistic for goodness-of-fit testing.
This result also localizes $\theta_n$ to a neighborhood of the target parameter $\theta_\star$.
\begin{proposition}\label{prop:localization}
    Under \Cref{asmp:self_concordance},
    if $\norm{\grad_n(\theta_\star)}_{H_n^{-1}(\theta_\star)} \le C_{\nu} [\lambda_{\min}(H_n(\theta_\star))]^{(3-\nu)/2} / (R n^{\nu/2-1})$,
    then the estimator $\theta_n$ uniquely exists and satisfies
    \begin{align*}
        \norm{\theta_n - \theta_\star}_{H_n(\theta_\star)} \le 4 \norm{\grad_n(\theta_\star)}_{H_n^{-1}(\theta_\star)}.
    \end{align*}
\end{proposition}

The main tool used in the proof of \Cref{prop:localization} is a strong convexity type result for generalized self-concordant functions recalled in \Cref{sec:tools}.
In order to apply \Cref{prop:localization}, we need to control $\norm{\grad_n(\theta_\star)}_{H_n^{-1}(\theta_\star)}$.
This result is summarized in the following proposition.

\begin{proposition}\label{prop:score}
    Under \Cref{asmp:sub_gaussian,asmp:bernstein} with $r = 0$, it holds that, with probability at least $1 - \delta$,
    \begin{align*}
        \norm{S_n(\theta_\star)}_{H_n^{-1}(\theta_\star)}^2 \lesssim \frac{d_\star}n \log{(e/\delta)}
    \end{align*}
    whenever $n \gtrsim \log{(2d/\delta)}$.
\end{proposition}

The proof of \Cref{prop:score} consists of two steps: (a) lower bound $H_n(\theta_\star)$ by $H_\star$ up to a constant using the Bernstein inequality and (b) upper bound $\norm{\grad_n(\theta_\star)}_{H^{-1}(\theta_\star)}$ using a concentration inequality for isotropic random vectors, where the tools are recalled in \Cref{sec:tools}.
Combining them implies that $\norm{\grad_n(\theta_\star)}_{H^{-1}(\theta_\star)}$ can be arbitrarily small and thus satisfies the requirement in \Cref{prop:localization} for sufficiently large $n$.
This not only proves the existence and uniqueness of the empirical risk minimizer $\theta_n$ but also provides an upper bound for $\norm{\theta_n - \theta_\star}_{H_n(\theta_\star)}$ through $\norm{\grad_n(\theta_\star)}_{H_n^{-1}(\theta_\star)}$.

In order to prove \Cref{thm:conf_set}, it remains to upper bound $H_n(\theta_n)$ by $H_\star$ up to a constant factor.
This can be achieved by the following result.
\begin{proposition}\label{prop:emp_hess_est}
    Under \Cref{asmp:self_concordance,asmp:bernstein} with $r = C_\nu \lambda_\star^{(\nu-3)/2} / R$, it holds that, with probability at least $1 - \delta$,
    \begin{align*}
        \frac1{2C_\nu} H_\star \preceq H_n(\theta) \preceq \frac32 C_\nu H_\star, \;\mbox{for all } \theta \in \Theta_{r}(\theta_\star),
    \end{align*}
    whenever $n \gtrsim \left\{ \log{(2d/\delta)} + d (\nu/2-1) \log{n}\right\}$.
\end{proposition}

Finally, \Cref{cor:lr_conf_set} follows from \Cref{thm:conf_set} and the Taylor expansion: there exists $\bar \theta_n \in \mbox{Conv}\{\theta_n, \theta_\star\}$ such that
\begin{align*}
    2[L_n(\theta_\star) - L_n(\theta_n)] = \norm{\theta_n - \theta_\star}_{H_n(\bar \theta_n)},
\end{align*}
where we have used $\nabla L_n(\theta_n) = 0$.

\subsection{Approximating the effective dimension}

One downside of \Cref{thm:conf_set,cor:lr_conf_set} is that $d_\star$ depends on the unknown data distribution.
Alternatively, we use the following empirical counterpart
\begin{align*}
    d_n := \Tr\left(H_n(\theta_n)^{-1/2} G_n(\theta_n) H_n(\theta_n)^{-1/2} \right).
\end{align*}
The next result implies that we do not lose much if we replace $d_\star$ by $d_n$.
This result is novel and of independent interest since one also needs to estimate $d_\star$ in order to construct asymptotic confidence sets under model misspecification.

\begin{customasmp}{2'}\label{asmp:subG_local}
    There exist constants $r, K_1 > 0$ such that, for any $\theta \in \Theta_r(\theta_\star)$, we have $\norm{G(\theta)^{-1/2} S(\theta; Z)}_{\psi_2} \le K_1$.
\end{customasmp}

\begin{assumption}\label{asmp:lip}
    There exists $r > 0$ such that $M := \Expect[M(Z)] < \infty$, where $M(z)$ is defined as
    \begin{align*}
        \sup_{\theta_1 \neq \theta_2 \in \Theta_r(\theta_\star)} \frac{\norm{G_\star^{-1/2} [G(\theta_1; z) - G(\theta_2; z)] G_\star^{-1/2}}_2}{\norm{\theta_1 - \theta_2}_{H_\star}}.
    \end{align*}
\end{assumption}

\myparagraph{Remark}
\Cref{asmp:lip} is a Lipschitz-type condition for $G(\theta; z)$. This assumption was previously used by \citep[Assumption 3]{mei2018landscape} to analyze non-convex risk landscapes. 

\begin{proposition}\label{prop:d_n}
    Let $\nu \in [2, 3)$.
    Under Asms.~\ref{asmp:self_concordance}, \ref{asmp:subG_local}, \ref{asmp:bernstein} and \ref{asmp:lip} with $r = C_\nu \lambda_\star^{(3-\nu)/2}/R$, it holds that
    \begin{align*}
        \frac1{C_\nu} d_\star \le d_n \le C_\nu d_\star,
    \end{align*}
    with probability at least $1 - \delta$,
    whenever $n$ is large enough (see \Cref{sub:appendix:consist_dn} for the precise condition).
\end{proposition}

\myparagraph{Remark}
The precise version of $\Cref{prop:d_n}$ in \Cref{sub:appendix:consist_dn} implies that $d_n$ is a consistent estimator of $d$.

With \Cref{prop:d_n} at hand, we can obtain finite-sample confidence sets involving $d_n$, which can be computed from data.
We illustrate it with the Wald confidence set.
\begin{corollary}\label{cor:wald_conf_set}
    Suppose the same assumptions in \Cref{prop:d_n} hold true.
    Let $\calC_{\text{Wald}, n}'(\delta)$ be
    \begin{align*}
        \left\{ \theta \in \Theta: \norm{\theta - \theta_\star}_{H_n(\theta_n)}^2 \le C_{K_1, \nu} \log{(e/\delta)} \frac{d_n}{n} \right\}.
    \end{align*}
    Then we have $\Prob(\theta_\star \in \calC_{\text{Wald}, n}'(\delta)) \ge 1 - \delta$ whenever $n$ satisfies the same condition as in \Cref{prop:d_n}.
\end{corollary}

\subsection{Discussion}
\label{sub:discussion}

\myparagraph{Fisher information and model misspecification}
When the model is well-specified, the autocorrelation matrix $G(\theta)$ coincides with the well-known Fisher information $\mathcal{I}(\theta) := \Expect_{Z \sim P_\theta}[S(\theta; Z)S(\theta; Z)^\top]$ at $\theta_\star$.
The Fisher information plays a central role in mathematical statistics and, in particular, M-estimation; see \citep{pennington2018spectrum,kunstner2019limitations,ash2021gone,soen2021variance} for recent developments in this line of research.
It quantifies the amount of information a random variable carries about the model parameter.
Under a well-specified model, it also coincides with the Hessian matrix $H(\theta)$ at the optimum which captures the local curvature of the population risk.
When the model is misspecified, the Fisher information deviates from the Hessian matrix.
In the asymptotic regime, this discrepancy is reflected in the limiting covariance of the weighted M-estimator which admits a sandwich form $H_\star^{-1/2} G_\star H_\star^{-1/2}$; see, e.g., \cite[Sec.~4]{huber1967under}.

\myparagraph{Effective dimension}
The counterpart of the sandwich covariance in the non-asymptotic regime is the effective dimension $d_\star$; see, e.g., \citep{spokoiny2017penalized,ostrovskii2021finite}.
Our bounds also enjoy the same merit---its dimension dependency is via the effective dimension.
When the model is well-specified, the effective dimension reduces to $d$, recovering the same rate of convergence $O(d/n)$ as in classical linear regression; see, e.g., \cite[Prop.~3.5]{bach2021learning}.
When the model is misspecified, the effective dimension provides a characterization of the problem complexity which is adapted to both the data distribution and the loss function via the matrix $H_\star^{-1/2} G_\star H_\star^{-1/2}$.
To gain a better understanding of the effective dimension $d_\star$, we summarize it in \Cref{tab:decay} in \Cref{sec:proofs} under different regimes of eigendecay, assuming that $G_\star$ and $H_\star$ share the same eigenvectors.
It is clear that, when the spectrum of $G_\star$ decays faster than the one of $H_\star$, the dimension dependency can be better than $O(d)$.
In fact, it can be as good as $O(1)$ when the spectrum of $G_\star$ and $H_\star$ decay exponentially and polynomially, respectively.

\myparagraph{Comparison to classical asymptotic theory}
Classical asymptotic theory of M-estimation is usually based on two assumptions: (a) the model is well-specified and (b) the sample size $n$ is much larger than the parameter dimension $d$.
These assumptions prevent it from being applicable to many real applications where the parametric family is only an approximation to the unknown data distribution and the data is of high dimension involving a large number of parameters.
On the contrary, our results do not require a well-specified model, and the dimension dependency is replaced by the effective dimension $d_\star$ which captures the complexity of the parameter space.
Moreover, they are of non-asymptotic nature---they hold true for any $n$ as long as it exceeds some constant factor of $d_\star$.
This allows the number of parameters to potentially grow with the same size.

\myparagraph{Comparison to recent non-asymptotic theory}
Recently, \citet{spokoiny2012parametric} achieved a breakthrough in finite-sample analysis of parametric M-estimation.
Although fully general, their results require strong global assumptions on the deviation of the empirical risk process and are built upon advanced tools from empirical process theory.
Restricting ourselves to generalized self-concordant losses, we are able to provide a more transparent analysis with neater assumptions only in a neighborhood of the optimum parameter $\theta_\star$.
Moreover, our results maintain some generality, covering several interesting examples in statistical machine learning as provided in \Cref{sub:examples}.

\citet{ostrovskii2021finite} also considered self-concordant losses for M-estimation.
However, their results are limited to generalized linear models whose loss is (pseudo) self-concordant and admits the form $\ell(\theta; Z) := \ell(Y, \theta^\top X)$.
While sharing the same rate $O(d_\star / n)$, our results are more general than theirs in two aspects.
First, the loss need not be of the form $\ell(Y, \theta^\top X)$, encompassing the score matching loss in \Cref{ex:score_matching} below.
Second, we go beyond pseudo self-concordance via the notion of generalized self-concordance.
Moreover, they focus on bounding the excess risk rather than providing confidence sets, and they do not study the estimation of $d_\star$.

Pseudo self-concordant losses have been considered for semi-parametric models \citep{liu2022orthogonal}.
However, they focus on bounding excess risk and require a localization assumption on $\theta_n$. Here we prove the localization result in \Cref{prop:localization} and we focus on confidence sets.

\myparagraph{Regularization}
Our results can also be applied to regularized empirical risk minimization by including the regularization term in the loss function.
Let $\theta_{n}^\lambda$ and $\theta_{\star}^\lambda$ be the minimizers of the \emph{regularized} empirical and population risk, respectively.
Let $d_\star^\lambda := \Tr\big((H_\star^\lambda)^{-1/2} G_\star^{\lambda} (H_\star^\lambda)^{-1/2}\big)$ where $H_\star^{\lambda}$ and $G_\star^{\lambda}$ are the regularized Hessian and the autocorrelation matrix of the regularized gradient at $\theta_\star^\lambda$, respectively.
Then our results characterize the concentration of $\theta_{n}^\lambda$ around $\theta_{\star}^\lambda$:
\begin{align*}
    \norm{\theta_n^{\lambda} - \theta_\star^\lambda}_{H_\star^\lambda}^2 \le O(d_\star^\lambda / n).
\end{align*}
This result coincides with \citet[Thm.~2.1]{spokoiny2017penalized}.
If the goal is to estimate the unregularized population risk minimizer $\theta_\star$, then we need to pay an additional error $\norm{\theta_\star^\lambda - \theta_\star}_{H_\star^\lambda}^2$ which is referred to as the modeling bias \citep[Sec.~2.5]{spokoiny2017penalized}.
One can invoke a so-called \emph{source condition} to bound the modeling bias and a \emph{capacity condition} to bound $d_\star^\lambda$.
An optimal value of $\lambda$ can be obtained by balancing between these two terms \cite[see, e.g.,][]{marteau2019beyond}.

For instance, let $Z := (X, Y)$ where $X \in \reals^d$ with $\Expect[XX^\top] = I_d$ and $Y \in \reals$.
Consider the regularized squared loss
$ \score^\lambda(\theta; z) := 1/2\, (y - \theta^\top x)^2 + 1/2\, \theta^\top U \theta$
where $U = \diag\{\mu_1, \dots, \mu_d\}$.
The regularized effective dimension is then~\citep[Sec.~2.1]{spokoiny2017penalized} of order 
$ O\big( \sum_{k=1}^d 1/(1 + \mu_k) \big)$
which can be much smaller than $d$ if $\{\mu_k\}$ is increasing.


\section{Variance Bound for \trnn{}}
\label{gen}
\begin{figure}[t]
  \centering
  \includegraphics[width=\linewidth]{figs/exps/gen_fig.pdf}
  \caption{Qualitative comparisons on the self-augmented dataset for \editb \  learning. The editing prompt is "S* as a police, looking at the camera". "w/o edit" and "w/o recon" denote for the encoder is trained without \editb \ learning objective and without reconstruction learning, respectively. We show that the generated images can not follow the prompt properly without the \editb \ learning. Meanwhile, the face similarity will be lower without the reconstruction learning on FFHQ.}
  \label{fig:exp_self_aug}
\end{figure}
%
\section{Experiments}
\label{exp}
\subsection{Unsupervised Grammar Induction}

\subsubsection{Setup}\label{sec:LM_setup}
\paragraph{Baselines and Evaluation.} 
For comparison, we include six recent strong models for unsupervised parsing with available open source implementations: StructFormer \cite{DBLP:conf/acl/ShenTZBMC20}, Ordered Neurons~\cite{DBLP:conf/iclr/ShenTSC19}, URNNG~\cite{dblp:conf/naacl/kimrykdm19}, DIORA~\cite{dblp:conf/naacl/drozdovvyim19}, C-PCFG~\cite{kim-etal-2019-compound}, and R2D2~\cite{hu-etal-2021-r2d2}. 
To observe the marginal gain from pretraining, we also include Fast-R2D2 without pretraining denoted as Fast-R2D2$_{\rm w/o}$.
Following~\newcite{htut-etal-2018-grammar}, we train all systems on a training set consisting only of raw text, and evaluate and report the results on an annotated test set. 
As an evaluation metric, we adopt sentence-level unlabeled $F_1$ computed using the script from \newcite{kim-etal-2019-compound}.
We compare against the non-binarized gold trees per convention.
The results of Fast-R2D2 are obtained from 3 runs of each model with different random seeds in pre-training.
The best checkpoint for each system is picked based on scores on the validation set. 
Fast-R2D2 is pretrained with span constraints for the word level but without span constraints for the word-piece level.
To support word-piece level evaluation, 
we convert gold trees to word-piece level trees 
by simply breaking each terminal node into a non-terminal node with its word-pieces as terminals, e.g., (NN discrepancy) into (NN (WP disc) (WP \#\#re) (WP \#\#pan) (WP \#\#cy)).

\paragraph{Environment.} EFLOPS~\cite{DBLP:conf/hpca/DongCZYWFZLSPGJ20} is a highly scalable distributed training system designed by Alibaba. With its optimized hardware architecture and co-designed supporting software tools, including ACCL~\cite{DBLP:journals/micro/DongWFCPTLLRGGL21} and KSpeed (the high-speed data-loading service), it could easily be extended to 10K nodes (GPUs) with linear scalability.

\paragraph{Hyperparameters.} The tree encoder of our model uses 4-layer Transformers with 768-dimensional embeddings, 
3,072-dimensional hidden layer representations, and 12 attention heads. 
The top-down parser of our model uses a 4-layer bidirectional LSTM with 128-dimensional embeddings and 256-dimensional hidden layer. The sampling number $K$ is set to be 256.
Training is conducted using Adam optimization with weight decay using a learning rate of $5 \times 10^{-5}$ for the tree encoder and $1 \times 10^{-2}$ for the top-down parser.
The batch size is set to 64 per GPU for $m$=$4$, though we also limit the maximum total length for each batch, such that excess sentences are moved to the next batch. The limit is set to 1,536. It takes about 120 hours for 60 epochs of training with $m$=$4$ on 8 A100 GPUs.

\paragraph{Data.}  For English, to fully leverage the scalability of Fast-R2D2, we pretrain Fast-R2D2 on WikiText103~\cite{DBLP:conf/iclr/MerityX0S17}
and then fine-tune the model on the Penn Treebank (PTB)~\cite{marcus-etal-1993-building}
for 10 epochs with the same objective.
WikiText103 is split at the sentence level, and sentences longer than 200 after tokenization are discarded (about 0.04‰ of the original data). 
The total number of sentences is 4,089,500, and the average sentence length is 26.97.
For Chinese, we use a subset of Chinese Wikipedia (Simplified Characters) for pretraining, specifically the first 10,000,000 sentences shorter than 150 characters and then fine-tune on Chinese Penn Treebank (CTB) 8.0~\cite{ctb8}.
We test our approach on PTB WSJ data with the standard splits (2--21 for training, 22 for validation, 23 for test) and the same preprocessing as in recent work \cite{kim-etal-2019-compound}, where we discard punctuation and lower-case all tokens. 
To explore the universality of the model across languages, we further evaluate using the CTB,
on which we also remove punctuation.
Note that in all settings, the training and fine-tuning is conducted entirely on raw unannotated text.

\subsubsection{Results and Discussion}

\begin{table}
\newcommand{\invzero}{\hphantom{0}}
\begin{center}
\setlength{\tabcolsep}{3.pt}
\resizebox{0.45\textwidth}{!}{
\begin{tabular}{@{}l|l|l|l|l@{}}
                    &  eval & mem. & \multicolumn{1}{c|}{WSJ}  & \multicolumn{1}{c}{CTB}  \\
Model               & gran. & cplx  &  $F_1(\mu)$ & $F_1(\mu)$\\ \hline \hline
Left Branching (W)  & WD & $O(n)$& \invzero 8.15  & 11.28 \\
Right Branching (W) & WD & $O(n)$& 39.62 & 27.53 \\
Random Trees (W)    & WD & $O(n)$ & 17.76 & 20.17 \\
\hline
URNNG (W)           & WD & $O(n^3)$& 45.4$^\dag$ & ~~--- \\
ON-LSTM (W)         & WD & $O(n)$  & 47.7$^\dag$ & 24.73 \\
DIORA (W)           & WD & $O(n^3)$& 51.4 & ~~---  \\
StructFormer (W)    & WD & $O(n^2)$& 54.0$^\ddagger$ & ~~--- \\
C-PCFG (W)          & WD & $O(n^3)$& 55.2$^\dag$ & 49.95 \\ \hline
R2D2 (WP)           & WD & $O(n)$ & 48.11 & 44.85  \\
Fast-R2D2$^*$(W)$_{\rm w/o}$ & WD & $O(n)$ & 48.24 & 45.24 \\
Fast-R2D2$^*$(WP)$_{\rm w/o}$ & WD & $O(n)$ & 48.89 & 45.26 \\
Fast-R2D2$^*$(WP)  & WD & $O(n)$ & \textbf{57.22} & \textbf{53.13} \\
\hline \hline
R2D2 (WP)           & WP & $O(n)$  & 52.28 & 63.94 \\ 
Fast-R2D2(WP)      & WP & $O(n)$ & 50.20 & \textbf{67.79} \\
Fast-R2D2$^*$(WP)  & WP & $O(n)$& \textbf{53.88} & 67.74 \\ \hline
\end{tabular}
}
\end{center}
\caption{Unsupervised parsing results with words (W) or word-pieces (WP) as input. ``eval gran." is short for evaluation granularity.
        Values marked with $^{\dag}$ are taken from \newcite{kim-etal-2019-compound}, while $^{\ddagger}$ denotes values taken from \newcite{DBLP:conf/acl/ShenTZBMC20}.
        The bottom three systems are all pre-trained or trained 
        at the word-piece level \textbf{without} span constraints and are measured against word-piece level golden trees. ${\rm w/o}$ means without pretraining.}
\label{tbl:constituency_parsing}
\end{table}


Table~\ref{tbl:constituency_parsing} shows the results of all systems with words (W) and word-pieces (WP) as input on the WSJ and CTB test sets. 
When we evaluate all systems on word-level golden trees, 
our Fast-R2D2 performs substantially better than R2D2 across both datasets.
We denote as Fast-R2D2 the method of using the parser to guide the pruning and selecting the best tree using the chart table and as Fast-R2D2$^*$ the system that uses the top-down parser for tree induction with subsequent R2D2 encoding.
Interestingly, the results suggest that Fast-R2D2$^*$ outperforms Fast-R2D2, especially on the WSJ test set.
Additionally, pretrained Fast-R2D2$^*$
outperforms the models specifically designed for grammar induction.

\begin{table}[!htb]
\small
\begin{center}
\setlength{\tabcolsep}{3.5pt}
\resizebox{0.48\textwidth}{!}{ %
\begin{tabular}{@{}ll| l l l l l l@{}}
 & Model  & WD & NNP & VP & SBAR\\\hline \hline
\multirow{5}{*}{\rotatebox[origin=c]{90}{WSJ}} & DIORA (WP)  & 94.63 & 77.83 & 17.30 & 22.16\\
& C-PCFG (W)                  & ~~--- & ~~--- & 41.7$^\dag$ & 56.1$^\dag$ \\
& C-PCFG (WP)                  & 87.35 & 66.44 & 23.63 & 40.40 \\
& R2D2 (WP)    & \textbf{99.76} & \textbf{86.76} & 24.74 & 39.81\\
& Fast-R2D2$^*$ (WP) & 97.67 & 83.44 & \textbf{63.80} & \textbf{65.68} \\ \hline \hline
\multirow{3}{*}{\rotatebox[origin=c]{90}{CTB}} & C-PCFG(WP) &89.34 & 46.74 & 39.53 & ~~---\\
 & R2D2 (WP) & 97.16 & 67.19 & 37.90 & ~~---\\
 & Fast-R2D2$^*$ (WP) & \textbf{97.80} & \textbf{68.57} & \textbf{46.59} & ~~---
 \\ \hline \hline
\end{tabular}
}
\end{center}
\caption{Recall of constituents and words. WD means word.  Values with $^{\dag}$ are taken from \newcite{kim-etal-2019-compound}.}
\label{tbl:unsupervised_chunking}
\end{table}

Following \newcite{dblp:conf/naacl/kimrykdm19} and \newcite{drozdov-etal-2020-unsupervised},
we also compute the recall of constituents when evaluating on word-piece level golden trees.
Besides standard constituents, we also compare the recall of word-piece chunks and proper noun chunks. 
Proper noun chunks are extracted by finding adjacent unary nodes with the same parent and tag NNP. 
Table~\ref{tbl:unsupervised_chunking} reports the recall scores for constituents and words on the WSJ and CTB test sets. 
Compared with the R2D2 baseline, 
our Fast-R2D2 performs slightly worse for small semantic units, 
but significantly better over larger semantic units (such as VP and SBAR) on the WSJ test set.
On the CTB test set, our Fast-R2D2 outperforms R2D2 on all constituents. 

From Tables~\ref{tbl:constituency_parsing}~and~\ref{tbl:unsupervised_chunking}, 
we conclude that Fast-R2D2 overall obtains better results than R2D2 on CTB, while faring slightly worse than R2D2 only for small semantic units on WSJ. We conjecture that this difference stems from differences in  tokenization between Chinese and English. 
Chinese is a character-based language without complex morphology, where collocations of characters are consistent with the language, making it easier for the top-down parser to learn them well. 
In contrast, word-pieces for English are built based on statistics, and individual word-pieces are not necessarily natural semantic units. Thus, there may not be sufficient semantic self-consistency, such that it is harder for a top-down parser with a small number of parameters to fit it well.

\subsection{Downstream Tasks}
We next consider the effectiveness of Fast-R2D2 in downstream tasks. This experiment is not intended to advance the state-of-the-art on the GLUE benchmark but rather to assess to what extent our approach performs respectably against the dominant inductive bias as in conventional sequential Transformers.

\subsubsection{Setup}
\paragraph{Data and Baseline.}
We fine-tune pretrained models on several datasets,
including SST-2, CoLA, QQP, and MNLI from the GLUE benchmark~\cite{wang2018glue}.
As sequential Transformers with their dominant inductive bias remain the norm for numerous NLP tasks, 
we mainly compare Fast-R2D2 with \bert~\cite{devlin2018} as a representative pretrained model based on a sequential Transformer. 
We did not include recursive models such as Gumbel-Tree-LSTMs~\cite{DBLP:conf/aaai/ChoiYL18} and CRvNN~\cite{DBLP:conf/icml/ChowdhuryC21} among our baselines, as they are not pretrained models.
In order to compare the two forms of inductive bias fairly and efficiently,
we pretrain \bert models with 4 layers and 12 layers as well as our Fast-R2D2 from scratch on the WikiText103 corpus following Section~\ref{sec:LM_setup}. 
Considering that longer inputs in the pre-training stage are helpful for BERT’s downstream task performance, we use the original corpus that is not split into sentences as inputs.
For simplicity, Fast-R2D2 is fine-tuned without span constraints.
Following the common settings, we add an MLP layer over the root representation of the R2D2 encoder for single-sentence classification. 
For cross-sentence tasks such as QQP and MNLI, we feed the root representations of the two sentences into the pretrained tree encoder of R2D2 as left and right inputs, 
and also add a new task ID as another input term to the R2D2 encoder. 
Then we feed the hidden output of the new task ID into another MLP layer to predict the final label.
We train all systems across the four datasets for 10 epochs 
with a learning rate of $5\times 10^{-5}$, batch size $64$, and maximum input length $200$.
We validate each model in each epoch and report the best results on development sets.

\begin{table}
\begin{center}
\setlength{\tabcolsep}{1.5pt}
\resizebox{0.48\textwidth}{!}{
\begin{tabular}{l|c|r r|r r}
\multirow{4}{*}{Model} & \multirow{4}{*}{Para.} & \multicolumn{2}{c|}{Single sent.} & \multicolumn{2}{c}{Cross sent.} \\
 &  & \begin{tabular}[c]{@{}l@{}}SST-2\\ (Acc.)\end{tabular} & \begin{tabular}[c]{@{}l@{}}CoLA\\ (Mcc.)\end{tabular} & \begin{tabular}[c]{@{}l@{}}QQP\\ (F1)\end{tabular} & \begin{tabular}[c]{@{}l@{}}MNLI\\m/mm\\ (Acc.)\end{tabular}            \\ \hline \hline
\bert (4L)  & 52M & 84.98 & 17.07 & 84.01 & 73.73/74.63 \\
\bert (12L) & 116M & 90.25 & 40.72 & 87.13 & 80.00/80.41 \\ \hline
R2D2        & 52M & 89.33 & 34.79 & 84.27 &  69.35/68.72 \\ \hline
Fast-R2D2$^\dag$& {\multirow{2}{*}{\begin{tabular}[c]{@{}c@{}}\\52M/\\ 10M\end{tabular}}} & 87.50 & 8.67 & 83.97 & 69.53/69.50 \\
Fast-R2D2$^*\dag$& {} & 88.30 & 10.14 & 84.07 & 69.36/69.11 \\
Fast-R2D2  & {} & 90.25 & 38.45 & 84.35 & 69.36/68.80 \\ 
Fast-R2D2$^*$& {} & 90.71 & 40.11 & 84.32 & 69.64/69.57\\
\hline \hline
\end{tabular}
}
\end{center}
\caption{Downstream results. All systems are pretrained from scratch on WikiText103.
        Para.\ describes the number of parameters for each model. Fast-R2D2 contains the R2D2 encoder and top-down parser, two components with 52M and 10M parameters, respectively.
        Mcc.\ stands for Matthew's correlation coefficient.
        Fast-R2D2 with $\dag$ are models fine-tuned without $\mathcal{L}_\mathrm{bilm}$ for an ablation study.
    }\vspace{-10pt}
\label{tbl:classification}
\end{table}
\subsubsection{Results and Discussion}
Table~\ref{tbl:classification} shows the corresponding scores on SST-2, CoLA, QQPl, and MNLI. 
In terms of the parameter size, our Fast-R2D2 model has 52M and 10M parameters for the R2D2 encoder and top-down parser, respectively.
It is clear that 12-layer \bert is significantly better than 4-layer \bert.
As mentioned in Section~\ref{sec:downstream}, Fast-R2D2 has two options to construct the final tree and representation for a given input sentence:
Fast-R2D2$^*$ uses the output tree from the top-down parser, while Fast-R2D2 uses the best tree inferred by the R2D2 encoder.
Similar to the results for unsupervised parsing, Fast-R2D2$^*$ in classification tasks again outperforms Fast-R2D2.
We hypothesize that trees generated by the top-down parser without Gumbel noise are more stable and reasonable.
Fast-R2D2 significantly outperforms 4-layer \bert and achieves competitive results compared to 12-layer \bert in single sentence classification tasks such as SST-2 and CoLA, but still performs significantly worse in the cross-sentence tasks. 
We believe this is an expected result, as there is no cross-attention mechanism in the inductive bias of Fast-R2D2. 
However, the performance of Fast-R2D2 on classification tasks shows that the inductive bias of R2D2 has higher parameter utilization than sequentially applied Transformers.
Importantly, we demonstrate that a Recursive Neural Network variant with an unsupervised parser can achieve comparable results to pretrained sequential Transformers even with fewer parameters and interpretable intermediate results, 
Hence, our Fast-R2D2 framework provides an alternative for NLP tasks.

\subsection{Speed Evaluation}
To assess the time cost, we mainly compare sequential Transformers and Fast-R2D2 in forced encoding on various sequence length ranges. We randomly select 1,000 sentences for each range from WikiText103 and report the average time consumption on a single A100 GPU. \bert is based on the open source Transformers library\footnote{\url{https://github.com/huggingface/transformers}} and R2D2 is based on the official code in \newcite{hu-etal-2021-r2d2}.\footnote{\url{https://github.com/alipay/StructuredLM_RTDT/tree/r2d2}}

\begin{table}% [htb!]
\small
\begin{center}
\setlength{\tabcolsep}{3.pt}
\resizebox{0.45\textwidth}{!}{
\begin{tabular}{l|rrrr}
\multirow{2}{*}{Model} & \multicolumn{4}{c}{Sequence Length Ranges} \\\cline{2-5}
      & \multicolumn{1}{c|}{0--50} & \multicolumn{1}{l|}{50--100} & \multicolumn{1}{l|}{100--200} & 200--500 \\ 
\hline
\bert (12L) & \multicolumn{1}{r|}{1.36}     & \multicolumn{1}{r|}{1.46}       & \multicolumn{1}{r|}{1.62}        & 2.38 \\ \hline
R2D2  & \multicolumn{1}{r|}{38.06}     & \multicolumn{1}{r|}{173.74}       & \multicolumn{1}{r|}{555.95}        &    ---     \\
Fast-R2D2  & \multicolumn{1}{r|}{4.67} & \multicolumn{1}{r|}{14.91} & \multicolumn{1}{r|}{39.73} & 150.26 \\
Fast-R2D2* & \multicolumn{1}{r|}{1.28} & \multicolumn{1}{r|}{2.96}  & \multicolumn{1}{r|}{5.56}  & 10.70 \\ 
\hline \hline
\end{tabular}
}
\end{center}
\caption{Inference time in seconds for various systems to process 1,000 sentences with a batch size of 50.}
\label{tbl:speed_test}
\end{table}

Table~\ref{tbl:speed_test} shows the inference time in seconds for different systems to process 1,000 sentences with a batch size of 50.
Running R2D2 is time-consuming, since the heuristic pruning method involves substantial memory exchanges between GPU and CPU. 
In Fast-R2D2, we alleviate this problem by using model-guided pruning to accelerate the chart table processing,
in conjunction with a code implementation in CUDA, Fast-R2D2 reduces the inference time significantly. 
Fast-R2D2$^{*}$ further improves the inference speed by running forced encoding in parallel over the binary tree generated by the parser, which is about 30--50 times faster than R2D2 in various ranges. 
Although there is still a gap in speed compared to sequential Transformers, Fast-R2D2$^{*}$ is sufficiently fast for most NLP tasks while producing interpretable intermediate representations.

%
\section{Discussion}
\label{disc}
\section{Summary and discussion}\label{s:disc}

In this work we introduced an attack that allows an adversary to violate the 
  guarantees of MPM systems by leveraging users' friends.
We also proposed several mitigations, but our proposals satisfy only two out 
  of three desirable properties: privacy (leak no information), low 
  communication overhead (i.e., clients need not send many messages per round), 
  and low latency (friends get to talk to each other often).
The most pragmatic of our solutions requires bounding the maximum number of 
  friends that a client can have.

Even with our mitigations, compromised friends are a liability and can be
  used to learn sensitive information through other means.
For example, if a user is uncharacteristically slow to respond to a 
  compromised friend's message (a user's response pattern could be constructed over
  many prior interactions), this anomaly in itself leaks information.
We believe that understanding the impact of this type of attack in
  practice is a promising avenue for future work.


% Acknowledgements should go at the end, before appendices and references

\acks{We would like to acknowledge support for this project
from the National Science Foundation (NSF grant IIS-9988642)
and the Multidisciplinary Research Program of the Department
of Defense (MURI N00014-00-1-0637). }

% Manual newpage inserted to improve layout of sample file - not
% needed in general before appendices/bibliography.

\vskip 0.2in
\newpage
\bibliography{jmlr}

\newpage

\appendix
\section*{Appendix A.}
\label{app:theorem}
%\appendices
\section{Pseudocode for Algorithm of Section~\ref{subsec:instantly_decodable}}
\label{app:pseudocode}

\algdef{SE}[EVENT]{Event}{EndEvent}[1]{\textbf{upon event}\ #1\ \algorithmicdo}{\algorithmicend\ \textbf{event}}%
\algtext*{EndEvent}

\begin{algorithm}
\caption{Coding for Three Users with Feedback}\label{alg:three_users}
\begin{algorithmic}[1]
\State \textbf{Initialize}: $r_i \gets 0, \forall i \in \mathcal{U}$  % \Comment{Packets received by each user}
\ForAll {$t \in [N]$}
    \While {$\nexists \ i \in \mathcal{U} \ s.t.\ Z_i = 0$}
%    \State Send $S(t)$ until at least one user receives it
    	\State Transmit $S(t)$
    \EndWhile
%    \If {$\exists \ i \in \mathcal{U}, T \in \mathbb{N} \ s.t.\ N_i(T) = 0$}
%    \If {$\exists \ i \in \mathcal{U} \ s.t.\ N_i = 0$}
        \State $Q_{\mathcal{E}} \gets Q_{\mathcal{E}} \cup \{S(t)\}$
        \State $r_j\gets r_j + 1 \qquad \forall \ j \ s.t.\  Z_j = 0$
%    \EndIf
\EndFor

%\While{$\exists \ \textrm{distinct} i, j, k \in \mathcal{U}, i\neq j \neq k \ s.t.\ Q_i \neq \varnothing \ \textbf{and} \ \Q{j,k} \neq \varnothing$}
\While{$\exists \ \textrm{distinct} \ i, j, k \in \mathcal{U}, \ s.t.\ Q_i \neq \varnothing \ \textbf{and} \ \Q{j,k} \neq \varnothing$}
    \While {$\nexists \ l \in \mathcal{U} \ s.t.\ Z_l = 0$}
%	\State Let $q_i \in Q_i$, $q_{j,k} \in \Q{j,k}$ 
%       \State Transmit $q_i \oplus q_{j,k}$ until at least one user receives it  
       \State Transmit $q_i \oplus q_{j,k}$ where $q_i \in Q_i$, $q_{j,k} \in \Q{j,k}$ 
    \EndWhile
    \State $r_l\gets r_l + 1 \qquad \forall \ l \ s.t.\  Z_l = 0$
    \If {$Z_i = 0$}
    	\State $Q_i \gets Q_i \setminus \{q_i\}$
    \EndIf
    \If {$Z_j = 0 \ \textbf{and} \ Z_k = 1$}
    	\State $Q_k \gets Q_k \cup \{q_{j, k}\}$
       \State $\Q{j,k} \gets \Q{j, k} \setminus \{q_{j,k}\}$
    \ElsIf {$Z_j = 1 \ \textbf{and} \ Z_k = 0$}
    	\State $Q_j \gets Q_j \cup \{q_{j, k}\}$
	\State $\Q{j,k} \gets \Q{j, k} \setminus \{q_{j,k}\}$
    \ElsIf {$Z_j = 0 \ \textbf{and} \ Z_k = 0$}
    	\State $\Q{j,k} \gets \Q{j, k} \setminus \{q_{j,k}\}$
    \EndIf    
\EndWhile

\While{$Q_i \neq \varnothing \  \forall \ i \in \mathcal{U}$}
	\While {$\nexists \ l \in \mathcal{U} \ s.t.\ Z_l = 0$}
		\State Transmit $q_1 \oplus q_{2} \oplus q_{3}$, where $q_i \in Q_i \ \forall i \in \mathcal{U}$
	\EndWhile
	\ForAll {$l \in \mathcal{U} \ s.t.\ Z_l = 0$}
		\State $Q_l \gets Q_l \setminus \{q_l\}$
		\State $z_l \gets z_l + 1$
	\EndFor
\EndWhile

\Event{$ \exists \ i \ s.t.\ r_i \geq 1- d_i  $} %\Comment{One user finished}
%	\State Let $j, k \in \mathcal{U} \setminus \{i\}, \ s.t.\ j \neq k$
	\State $Q_j \gets Q_j \cup \Q{i,j} \qquad \forall \ j \in \mathcal{U} \setminus \{i\}$
	\State \textbf{run} two-user algorithm of Section~\ref{sec:two_users}
\EndEvent
\end{algorithmic}
\end{algorithm}


% Note: in this sample, the section number is hard-coded in. Following
% proper LaTeX conventions, it should properly be coded as a reference:

%In this appendix we prove the following theorem from
%Section~\ref{sec:textree-generalization}:



\end{document}
