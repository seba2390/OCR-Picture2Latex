\documentclass[twoside,11pt]{article}
% about 20 main file, 30 total 
% Any additional packages needed should be included after jmlr2e.
% Note that jmlr2e.sty includes epsfig, amssymb, natbib and graphicx,
% and defines many common macros, such as 'proof' and 'example'.
%
% It also sets the bibliographystyle to plainnat; for more information on
% natbib citation styles, see the natbib documentation, a copy of which
% is archived at http://www.jmlr.org/format/natbib.pdf

\usepackage{jmlr2e}

\usepackage{algorithm}
\usepackage{algorithmic}
%\usepackage{breakurl}
%\usepackage[breaklinks]{hyperref}
\usepackage{wrapfig}
\usepackage{graphicx} % \texttt{\texttt{}}more modern

 % and their extensions so you won't have to specify these with
 % every instance of \includegraphics
\DeclareGraphicsExtensions{.pdf,.jpeg,.png}

\usepackage{wrapfig}
\usepackage{subfiles}
\usepackage[subpreambles=false]{standalone}

\usepackage{microtype}
\usepackage{graphicx}
\usepackage{subcaption}
\usepackage{booktabs} % for professional tables
\usepackage{url}
\usepackage{graphicx}
\usepackage{amsmath}
\usepackage{amssymb}
\usepackage{wrapfig}
\usepackage{color}

\newcommand{\fix}{\marginpar{FIX}}
\newcommand{\new}{\marginpar{NEW}}


\usepackage{enumitem}
%\iclrfinalcopy % Uncomment for camera-ready version, but NOT for submission.

% \usepackage{bm}
% \usepackage{amsmath}
% \usepackage{amsfonts}
% \usepackage{amssymb}
% \usepackage{amsthm}
% \usepackage{mathtools}
\usepackage{color}

\newcommand*\diff{\mathop{}\!\mathrm{d}}

\newcommand{\loss} {\mathcal{L}}
\newcommand{\W} {\mathcal{W}}

\newcommand\smallO{
	\mathchoice
	{{\scriptstyle\mathcal{O}}}% \displaystyle
	{{\scriptstyle\mathcal{O}}}% \textstyle
	{{\scriptscriptstyle\mathcal{O}}}% \scriptstyle
	{\scalebox{.7}{$\scriptscriptstyle\mathcal{O}$}}%\scriptscriptstyle
}

\newcommand{\R}{{\mathbb R}}   %% mathbb not working?
\newcommand{\E}{{\mathbb E}}   %% mathbb not working?

\newcommand{\skt}{{\mathcal{S}}}
\newcommand{\tr}{{\text{tr}}}
\newcommand{\vt}{{\text{vec}}}


\newcommand{\eat}[1]{}
% \newtheorem{theorem}{Theorem}[section]
% \newtheorem{proposition}[theorem]{Proposition}


% \newtheorem{definition}[theorem]{Definition}
% \newtheorem{lemma}[theorem]{Lemma}
% \newtheorem{fact}[theorem]{Fact}
% \newtheorem{corollary}[theorem]{Corollary}

%\newenvironment{proof}[1][Proof]{\begin{trivlist}
%\item[\hskip \labelsep {\bfseries #1}]}{\end{trivlist}}
%\newenvironment{definition}[1][Definition]{\begin{trivlist}
%\item[\hskip \labelsep {\bfseries #1}]}{\end{trivlist}}
% \newenvironment{example}[1][Example]{\begin{trivlist}
% \item[\hskip \labelsep {\bfseries #1}]}{\end{trivlist}}
% \newenvironment{remark}[1][Remark]{\begin{trivlist}
% \item[\hskip \labelsep {\bfseries #1}]}{\end{trivlist}}

% %
%\newcommand{\qed}{\nobreak \ifvmode \relax \else
%      \ifdim\lastskip<1.5em \hskip-\lastskip
%      \hskip1.5em plus0em minus0.5em \fi \nobreak
%      \vrule height0.75em width0.5em depth0.25em\fi}

\newcommand{\real}{\mathbb{R}}
\newcommand{\ex}{\mathbb{E}}
\newcommand{\pq}[1]{\left( #1 \right)}
\newcommand{\Loss}{\mathcal{L}}
\newcommand{\M}[1]{{\mathbf{#1}}} % matrix
\newcommand{\T}[1]{{\mathcal{#1}}} % tensor
\newcommand{\V}[1]{{\mathbf{#1}}} % vector
\newcommand{\SM}[1]{\mathbf{\hat{#1}}}
\newcommand{\TM}[1]{\mathbf{\bar{#1}}}
\newcommand{\krp}{\odot}
\newcommand{\PR}{\mathbb{P}}
\newcommand{\MG}[2]{\mathcal{N}(#1,#2)}
\newcommand{\OO}[1]{\mathcal{O}(#1)}
\newcommand{\OOt}[1]{\tilde{\mathcal{O}}(#1)}
\newcommand{\NI} [1]{ \N{ #1}_{1}}
\newcommand{\Nsp}[1]{ \N{ #1}_{2}}
\newcommand{\N}[1]{ {\left\| #1 \right\|}}
\newcommand{\Nf} [1]{ \N{ #1}_{\textsc{F}}}
\newcommand{\Mij}[3]{#1_{{#2},{#3}}}
\newcommand{\Mcij}[3]{{\{#1\}}_{{#2},{#3}}}
\newcommand{\SV}[2]{\sigma_{#1}(#2)}
\newcommand{\EV}[2]{\lambda_{#1}(#2)}
\newcommand{\numberthis}{\addtocounter{equation}{1}\tag{\theequation}}
\newcommand{\abs}[1]{ {\left| #1 \right|}}

\newcommand{\ryedit}[1]{{\color{magenta} #1}}
\newcommand{\rycomment}[1]{\ryedit{[RY: #1]}}
\newcommand{\stzedit}[1]{{\color{red} #1}}
\newcommand{\stzcomment}[1]{\stzedit{[DH: #1]}}
\newcommand{\yledit}[1]{{\color{blue} #1}}
\newcommand{\ylcomment}[1]{\yledit{[YL: #1]}}


%\newcommand{\real}{\mathbb{R}}
\newcommand{\kp}{\otimes}
\newcommand{\hp}{*}
%\newcommand{\krp}{\odot}
\newcommand{\op}{\circ }
\newcommand{\cpd}[1]{{\left\llbracket {#1} \right\rrbracket}}

%\newcommand{\dhedit}[1]{{#1}}
%\newcommand{\dhcomment}[1]{}
%\newcommand{\yledit}[1]{#1}
%\newcommand{\ylcomment}[1]{}




%% ===============================================================
%% Colors
%% ===============================================================

%% ===============================================================
%% Shortcuts and symbols
%% ===============================================================

\newcommand{\doublesep}{\hspace{2pt}||\hspace{2pt}}
\newcommand{\commentthis}[1]{\begin{comment}#1\end{comment}}
\newcommand{\refn}[1]{(\ref{#1})}
\newcommand{\hsp}{\hspace{20pt}}
\newcommand{\vecc}[1]{\begin{array}\left(#1\right\end{array})}

\newcommand{\gde}[4]{\frac{\partial g_{{#1}{#2}}}{\partial {#3}^{#4}}}
\newcommand{\chr}[3]{\Gamma^{#1}{}_{#2 #3}}
\newcommand{\pderaaaa}[2]{\frac{\partial}{\partial {#1}^{#2}}}
\newcommand{\pdere}[1]{\frac{\partial^2}{\partial {#1}^2}}
% \newcommand{\tr}{\operatorname{tr}}
\newcommand{\propa}[2]{\frac{1}{#1^2-#2^2+i\epsilon}}
\newcommand{\brck}[1]{\left(#1\right)}
\newcommand{\brcksq}[1]{\left[#1\right]}
\newcommand{\brckcur}[1]{\left\{#1\right\}}
\newcommand{\brckc}[1]{\left[#1\right\}}
\newcommand{\brcka}[1]{\langle #1\rangle}
\newcommand{\norm}[1]{||#1||}

\newcommand{\bra}[1]{\langle #1|}
\newcommand{\ket}[1]{| #1\rangle}
\newcommand{\fr}[2]{\frac{#1}{#2}}
\newcommand{\vdelim}{\hspace{4pt}|\hspace{4pt}}
\newcommand{\ol}[1]{\overline{#1}}
\newcommand{\cl}[1]{\begin{center} #1\end{center}}

\newcommand{\lra}{\longrightarrow}
\newcommand{\lmt}{\longmapsto}
\newcommand{\llra}{\longleftrightarrow}
\newcommand{\Llra}{\Longleftrightarrow}

\newcommand{\be}{\begin{equation}}
\newcommand{\ee}{\end{equation}}
\newcommand{\bali}{\begin{eqnarray*}}
\newcommand{\eali}{\end{eqnarray*}}
\newcommand{\eq}[1]{\begin{align}#1\end{align}}
\newcommand{\eqn}[1]{\begin{align*}#1\end{align*}}
\newcommand{\pmat}[1]{\begin{pmatrix}#1\end{pmatrix}}
\newcommand{\vecleft}[1]{\begin{pmatrix*}[l]#1\end{pmatrix*}}
\newcommand{\iitem}[1]{\begin{itemize}#1\end{itemize}}
\newcommand{\enumm}[1]{\begin{enumerate}#1\end{enumerate}}
\newcommand{\eps}{\epsilon}
\newcommand{\zbar}{\overline{z}}

\newcommand{\calA}{\mathcal{A}}
\newcommand{\calB}{\mathcal{B}}
\newcommand{\calC}{\mathcal{C}}
\newcommand{\calD}{\mathcal{D}}
\newcommand{\calE}{\mathcal{E}}
\newcommand{\calF}{\mathcal{F}}
\newcommand{\calG}{\mathcal{G}}
\newcommand{\calH}{\mathcal{H}}
\newcommand{\calI}{\mathcal{I}}
\newcommand{\calJ}{\mathcal{J}}
\newcommand{\calK}{\mathcal{K}}
\newcommand{\calL}{\mathcal{L}}
\newcommand{\calM}{\mathcal{M}}
\newcommand{\calN}{\mathcal{N}}
\newcommand{\calO}{\mathcal{O}}
\newcommand{\calP}{\mathcal{P}}
\newcommand{\calQ}{\mathcal{Q}}
\newcommand{\calR}{\mathcal{R}}
\newcommand{\calS}{\mathcal{S}}
\newcommand{\calT}{\mathcal{T}}
\newcommand{\calU}{\mathcal{U}}
\newcommand{\calV}{\mathcal{V}}
\newcommand{\calW}{\mathcal{W}}
\newcommand{\calX}{\mathcal{X}}
\newcommand{\calY}{\mathcal{Y}}
\newcommand{\calZ}{\mathcal{Z}}

\newcommand{\frakg}{\mathfrak{g}}
\newcommand{\Sym}{\trm{Sym}}
\newcommand{\mathR}{\mathbb{R}}
\newcommand{\mathC}{\mathbb{C}}
\newcommand{\mathQ}{\mathbb{Q}}
\newcommand{\mathZ}{\mathbb{Z}}

\newcommand{\tder}[2]{\frac{d #1}{d #2}}
\newcommand{\pder}[2]{\frac{\partial #1}{\partial #2}}
\newcommand{\pdl}[1]{\partial_{#1}}
\newcommand{\pdu}[1]{\partial^{#1}}
\newcommand{\pd}{\partial}
\newcommand{\pdbar}{\ol{\partial}}
\newcommand{\fder}[2]{\frac{\delta #1}{\delta #2}}

\newcommand{\trm}[1]{\textnormal{#1}}
\newcommand{\tsc}[1]{\textsc{#1}}
\newcommand{\tb}[1]{\textbf{#1}}
\newcommand{\ti}[1]{\textit{#1}}

\newcommand{\mm}{\tb{m}}
\newcommand{\PP}{\tb{P}}
\newcommand{\ww}{\tb{w}}
% \newcommand{\argm}[1]{\textrm{argmax}_{#1}}
\newcommand{\argm}[1]{\textrm{argmax}_{#1}}
\newcommand{\argmi}[1]{\textrm{argmin}_{#1}}


%% ===============================================================
%% Custom commands (paper specific)
%% ===============================================================
\newcommand{\es}[1]{x_{#1}}
\newcommand{\ob}[2]{o^{#1}_{#2}}
\newcommand{\hist}[2]{h^{#1}_{#2}}
\newcommand{\ac}[2]{a^{#1}_{#2}}
\newcommand{\acc}[1]{\tb{a}_{#1}}
\newcommand{\st}[2]{s^{#1}_{#2}}
\newcommand{\stt}[1]{\tb{s}_{#1}}
\newcommand{\ep}[1]{f_{#1}}
\newcommand{\rw}[2]{r^{#1}_{#2}}
\newcommand{\rww}[1]{\tb{r}_{#1}}
\newcommand{\ms}[2]{m^{#1}_{#2}}
\newcommand{\mss}[1]{\tb{m}_{#1}}


\newcommand{\pol}[2]{\pi^{#1}_{#2}\xspace}
\newcommand{\val}[1]{V^{#1}\xspace}
\newcommand{\ev}[1]{\mathbb{E}\brcksq{#1}\xspace}

\newcommand{\trnn}{\texttt{HOT-RNN}}
\newcommand{\tlstm}{HOT-LSTM}

\usepackage{hyperref}


% Attempt to make hyperref and algorithmic work together better:
%\newcommand{\theHalgorithm}{\arabic{algorithm}}


% Definitions of handy macros can go here

\newcommand{\dataset}{{\cal D}}
\newcommand{\fracpartial}[2]{\frac{\partial #1}{\partial  #2}}

% Heading arguments are {volume}{year}{pages}{date submitted}{date published}{paper id}{author-full-names}

\jmlrheading{1}{2019}{1-48}{4/00}{10/00}{yu18}{Rose Yu, Stephan Zheng, Anima Anandkumar, Yisong Yue}

% Short headings should be running head and authors last names

\ShortHeadings{Higher-Order Tensor RNNs}{Yu et al.}
\firstpageno{1}

\begin{document}

\title{Long-Term Forecasting using  Higher-Order Tensor RNNs}

\author{\name Rose Yu \email rose@caltech.edu 
       \AND
       \name Stephan Zheng \email stephan@caltech.edu 
       \AND
      \name Anima  Anandkumar \email anima@caltech.edu 
       \AND
       \name Yisong Yue \email yyue@caltech.edu \\
        \addr Department of Computing and Mathematical Sciences\\
       California Institute of Technology\\
       Pasadena, CA 91125, USA}

\editor{Francis Bach, David Blei and Bernhard Sch{\"o}lkopf}

\maketitle

\begin{abstract}%   <- trailing '%' for backward compatibility of .sty file
  We present Higher-Order Tensor RNN (\trnn{}), a novel family of neural sequence architectures for multivariate forecasting in environments with nonlinear dynamics.
  %
  Long-term forecasting in such systems is highly challenging, since there exist long-term temporal dependencies, higher-order correlations and sensitivity to error propagation.
  %
  Our proposed recurrent architecture addresses these issues by learning the nonlinear dynamics directly using higher-order moments and higher-order state transition functions.
  %
  Furthermore, we decompose the higher-order structure using the tensor-train decomposition to reduce the number of parameters while preserving the model performance.
  %
  We theoretically establish the approximation guarantees and the variance bound for \trnn{} for general sequence inputs. We also
  demonstrate $5 \sim 12\%$  improvements for long-term prediction over general RNN and LSTM architectures on a range of simulated environments with nonlinear dynamics, as well on real-world time series data.
  %
\end{abstract}

\begin{keywords}
   Time Series, Forecasting, Tensor, RNNs, Nonlinear Dynamics
\end{keywords}


\section{Introduction}
\label{intro}
% !TEX root = ../arxiv.tex

Unsupervised domain adaptation (UDA) is a variant of semi-supervised learning \cite{blum1998combining}, where the available unlabelled data comes from a different distribution than the annotated dataset \cite{Ben-DavidBCP06}.
A case in point is to exploit synthetic data, where annotation is more accessible compared to the costly labelling of real-world images \cite{RichterVRK16,RosSMVL16}.
Along with some success in addressing UDA for semantic segmentation \cite{TsaiHSS0C18,VuJBCP19,0001S20,ZouYKW18}, the developed methods are growing increasingly sophisticated and often combine style transfer networks, adversarial training or network ensembles \cite{KimB20a,LiYV19,TsaiSSC19,Yang_2020_ECCV}.
This increase in model complexity impedes reproducibility, potentially slowing further progress.

In this work, we propose a UDA framework reaching state-of-the-art segmentation accuracy (measured by the Intersection-over-Union, IoU) without incurring substantial training efforts.
Toward this goal, we adopt a simple semi-supervised approach, \emph{self-training} \cite{ChenWB11,lee2013pseudo,ZouYKW18}, used in recent works only in conjunction with adversarial training or network ensembles \cite{ChoiKK19,KimB20a,Mei_2020_ECCV,Wang_2020_ECCV,0001S20,Zheng_2020_IJCV,ZhengY20}.
By contrast, we use self-training \emph{standalone}.
Compared to previous self-training methods \cite{ChenLCCCZAS20,Li_2020_ECCV,subhani2020learning,ZouYKW18,ZouYLKW19}, our approach also sidesteps the inconvenience of multiple training rounds, as they often require expert intervention between consecutive rounds.
We train our model using co-evolving pseudo labels end-to-end without such need.

\begin{figure}[t]%
    \centering
    \def\svgwidth{\linewidth}
    \input{figures/preview/bars.pdf_tex}
    \caption{\textbf{Results preview.} Unlike much recent work that combines multiple training paradigms, such as adversarial training and style transfer, our approach retains the modest single-round training complexity of self-training, yet improves the state of the art for adapting semantic segmentation by a significant margin.}
    \label{fig:preview}
\end{figure}

Our method leverages the ubiquitous \emph{data augmentation} techniques from fully supervised learning \cite{deeplabv3plus2018,ZhaoSQWJ17}: photometric jitter, flipping and multi-scale cropping.
We enforce \emph{consistency} of the semantic maps produced by the model across these image perturbations.
The following assumption formalises the key premise:

\myparagraph{Assumption 1.}
Let $f: \mathcal{I} \rightarrow \mathcal{M}$ represent a pixelwise mapping from images $\mathcal{I}$ to semantic output $\mathcal{M}$.
Denote $\rho_{\bm{\epsilon}}: \mathcal{I} \rightarrow \mathcal{I}$ a photometric image transform and, similarly, $\tau_{\bm{\epsilon}'}: \mathcal{I} \rightarrow \mathcal{I}$ a spatial similarity transformation, where $\bm{\epsilon},\bm{\epsilon}'\sim p(\cdot)$ are control variables following some pre-defined density (\eg, $p \equiv \mathcal{N}(0, 1)$).
Then, for any image $I \in \mathcal{I}$, $f$ is \emph{invariant} under $\rho_{\bm{\epsilon}}$ and \emph{equivariant} under $\tau_{\bm{\epsilon}'}$, \ie~$f(\rho_{\bm{\epsilon}}(I)) = f(I)$ and $f(\tau_{\bm{\epsilon}'}(I)) = \tau_{\bm{\epsilon}'}(f(I))$.

\smallskip
\noindent Next, we introduce a training framework using a \emph{momentum network} -- a slowly advancing copy of the original model.
The momentum network provides stable, yet recent targets for model updates, as opposed to the fixed supervision in model distillation \cite{Chen0G18,Zheng_2020_IJCV,ZhengY20}.
We also re-visit the problem of long-tail recognition in the context of generating pseudo labels for self-supervision.
In particular, we maintain an \emph{exponentially moving class prior} used to discount the confidence thresholds for those classes with few samples and increase their relative contribution to the training loss.
Our framework is simple to train, adds moderate computational overhead compared to a fully supervised setup, yet sets a new state of the art on established benchmarks (\cf \cref{fig:preview}).

%
\section{Related Work}
\label{related}
\section{Related Work}\label{sec:related}
 
The authors in \cite{humphreys2007noncontact} showed that it is possible to extract the PPG signal from the video using a complementary metal-oxide semiconductor camera by illuminating a region of tissue using through external light-emitting diodes at dual-wavelength (760nm and 880nm).  Further, the authors of  \cite{verkruysse2008remote} demonstrated that the PPG signal can be estimated by just using ambient light as a source of illumination along with a simple digital camera.  Further in \cite{poh2011advancements}, the PPG waveform was estimated from the videos recorded using a low-cost webcam. The red, green, and blue channels of the images were decomposed into independent sources using independent component analysis. One of the independent sources was selected to estimate PPG and further calculate HR, and HRV. All these works showed the possibility of extracting PPG signals from the videos and proved the similarity of this signal with the one obtained using a contact device. Further, the authors in \cite{10.1109/CVPR.2013.440} showed that heart rate can be extracted from features from the head as well by capturing the subtle head movements that happen due to blood flow.

%
The authors of \cite{kumar2015distanceppg} proposed a methodology that overcomes a challenge in extracting PPG for people with darker skin tones. The challenge due to slight movement and low lighting conditions during recording a video was also addressed. They implemented the method where PPG signal is extracted from different regions of the face and signal from each region is combined using their weighted average making weights different for different people depending on their skin color. 
%

There are other attempts where authors of \cite{6523142,6909939, 7410772, 7412627} have introduced different methodologies to make algorithms for estimating pulse rate robust to illumination variation and motion of the subjects. The paper \cite{6523142} introduces a chrominance-based method to reduce the effect of motion in estimating pulse rate. The authors of \cite{6909939} used a technique in which face tracking and normalized least square adaptive filtering is used to counter the effects of variations due to illumination and subject movement. 
The paper \cite{7410772} resolves the issue of subject movement by choosing the rectangular ROI's on the face relative to the facial landmarks and facial landmarks are tracked in the video using pose-free facial landmark fitting tracker discussed in \cite{yu2016face} followed by the removal of noise due to illumination to extract noise-free PPG signal for estimating pulse rate. 

Recently, the use of machine learning in the prediction of health parameters have gained attention. The paper \cite{osman2015supervised} used a supervised learning methodology to predict the pulse rate from the videos taken from any off-the-shelf camera. Their model showed the possibility of using machine learning methods to estimate the pulse rate. However, our method outperforms their results when the root mean squared error of the predicted pulse rate is compared. The authors in \cite{hsu2017deep} proposed a deep learning methodology to predict the pulse rate from the facial videos. The researchers trained a convolutional neural network (CNN) on the images generated using Short-Time Fourier Transform (STFT) applied on the R, G, \& B channels from the facial region of interests.
The authors of \cite{osman2015supervised, hsu2017deep} only predicted pulse rate, and we extended our work in predicting variance in the pulse rate measurements as well.

All the related work discussed above utilizes filtering and digital signal processing to extract PPG signals from the video which is further used to estimate the PR and PRV.  %
The method proposed in \cite{kumar2015distanceppg} is person dependent since the weights will be different for people with different skin tone. In contrast, we propose a deep learning model to predict the PR which is independent of the person who is being trained. Thus, the model would work even if there is no prior training model built for that individual and hence, making our model robust. 

%
%
\section{Higher-Order Tensor RNNs}
\label{trnn}
\paragraph{Forecasting Nonlinear Dynamics}
%
Our goal is to learn an efficient forecasting model for \ti{continuous multivariate time series} in environments with nonlinear dynamics.
%
The state $\V{x}_t \in \mathR^d$ of such systems evolves over time using a set of \ti{nonlinear} differential equations:
%
\eq{\label{eq:dynamics}
\brckcur{\xi^i\brck{\V{x}_t, \fr{d\V{x}}{dt}, \fr{d^2\V{x}}{dt^2}, \ldots; \phi} = 0 }_i,
}
%
where $\xi^i$ can be an arbitrary (smooth) function of the state $\V{x}_t$ and its derivatives. 
% 
Continuous time dynamics are usually described by differential equations while  difference equations are employed for discrete time. 
% 
In continuous time, a classic example is the first-order Lorenz attractor, whose realizations showcase the ``butterfly-effect'', a characteristic set of double-spiral orbits. 
% 
In discrete-time, a non-trivial example is the 1-dimensional Genz dynamics, whose difference equation is:
%
\eq{\label{eq:genzprodpeak}
	x_{t+1} = \brck{c^{-2} + (x_t + w)^2}^{-1}, \hspace{10pt}  c,w \in [0,1],
}
where $x_t$ denotes the system state at time $t$ and $c,w$ are the parameters. Due to the nonlinear nature of the dynamics, such systems exhibit higher-order correlations, long-term dependencies and sensitivity to error propagation, and thus form a challenging setting for forecasting.
% 
% \ryedit{Add visualization of Genz dynamics here?}

Given a sequence of initial states $\V{x}_0\ldots \V{x}_t$, the forecasting problem aims to learn a dynamics model $F$ that outputs a sequence of future states $\V{x}_{t+1} \ldots \V{x}_T$. 
\eq{\label{eq:forecast}
F: \brck{\V{x}_0\ldots \V{x}_t} \mapsto \brck{\V{y}_{t} \ldots \V{y}_T},
%
\hspace{10pt} \V{y}_t = \V{x}_{t+1},
}
The system is governed by some unknown dynamics. Hence, accurately approximating the dynamics is critical to learning a good forecasting model and making predictions for long time horizons.
 
 
\begin{figure*}[t]
\begin{center}
\begin{minipage}[t]{0.62\linewidth}
\centering		\includegraphics[width=\linewidth]{Figure/tlstm.png}
\caption{\trnn{} within a seq2seq model. Both encoder  and decoder contain higher-order recurrent cells. The augmented state $\V{s}_{t-1}$ (grey) takes in past $L$ hidden states (blue) and forms a higher-order tensor. \trnn{} (red)  factorizes the tensor and outputs the next hidden state.}
\label{fig:seq2seq}
\end{minipage}
\hspace{0.02\linewidth}
\begin{minipage}[t]{0.33\linewidth}
\centering		\includegraphics[width=\linewidth]{Figure/tensor_train.png}
\caption{A \trnn{} cell. The augmented state $\V{s}_{t-1}$ (grey) forms a higher-order tensor, which is then factorized to output the next hidden state.}
\label{fig:ttrnn}
\end{minipage}
\end{center}
\vspace{-5mm}
\end{figure*}

\paragraph{First-order Markovian Models}
%
In deep learning, popular approaches such as recurrent neural networks (RNNs) employ first-order hidden-state models to approximate the dynamics. An RNN with a single  cell recursively  computes a hidden state $\V{h}_t$ using the most recent hidden state $\V{h}_{t-1}$, generating  the output $\V{y}_t$ from the hidden state $\V{h}_t$ :
%
\eq{\label{eq:rnn}
\V{h}_t = f(\V{x}_t, \V{h}_{t-1}; \theta_f),\hspace{10pt} \V{y}_t = g(\V{h}_t; \theta_g),
}
%
where $f$ is the state transition function, $g$ is the output  function and $\{\theta_f, \theta_g\}$ are the corresponding  model parameters. A common parametrization scheme for \refn{eq:rnn} applies a nonlinear  activation function such as sigmoid $\sigma$ to a linear map of $\V{x}_t$ and $\V{h}_{t-1}$ as:
%
\eq{
\V{h}_t &= \sigma(W^{hx} \V{x}_t + W^{hh} \V{h}_{t-1} + \V{b}^h), \quad
\V{x}_{t+1} = W^{xh} \V{h}_t + \V{b}^x,
}
where $W^{hx}, W^{xh}$ and $W^{hh}$ are  the transition weight matrices and $\V{b}^h, \V{b}^x$ are the biases.

RNNs have many different variations, including LSTMs \citep{hochreiter1997long} and GRUs \citep{chung2014empirical}. 
% For instance, LSTM cells use a memory-state, which mitigate the ``exploding gradient'' problem and allow RNNs to propagate information over longer time horizons.
%
Although a RNN can approximate any function in theory, its hidden state $\V{h}_t$  only depends on the previous state $\V{h}_{t-1}$ and the input $\V{x}_t$. Such models do not explicitly capture higher-order dynamics and only  implicitly encode long-term dependencies between all historical states $\V{h}_{0} \ldots \V{h}_{t}$. This limits the representation power of RNNs, especially for forecasting in environments with nonlinear dynamics. Hence, instead of using a wide RNN with many hidden units, we exploit the recurrent cell to design higher-order tensor RNNs that can approximate complex non-linear governing equations. 


% \paragraph{The Debate Between Deep and Shallow}
% While both the deep and shallow networks preserve the universal approximation property, the folk wisdom is that shallow networks memorize well but generalize poorly. Theoretically, \cite{mhaskar2017and} have shown deep networks to have lower number of sample complexity. Empirically, hierarchical architectures such as residual networks \cite{he2016deep} and dense networks \cite{huang2017densely} are quite successful. 

% We take an analogous approach in our design principle but focus on the temporal dimension. We study the universal approximation property of RNNs for representing the underlying dynamics. 

\subsection{Higher-Order Non-Markovian Models}

To effectively learn nonlinear dynamics with higher-order temporal dependency, we propose a family of models that generalizes standard RNNs: higher-order recurrent neural networks, or  \trnn{}. 
%
We design \trnn{}s with two goals in mind: explicitly modeling 1) $L$-order Markov processes with $L$ steps of temporal memory and 2) polynomial interactions between the hidden states $\V{h}_{\cdot}$ and $\V{x}_t$.

First, we consider longer ``history'': we keep length $L$ historic states: $\V{h}_{t},\cdots, \V{h}_{t-L}$:
%
\eq{
\V{h}_t = f( \V{x}_t , \V{h}_{t-1}, \cdots, \V{h}_{t-L}; \theta_f)
\label{eqn:high_order_markov}
}
%
where $f$ represents the state transition function.  In principle, early work \citep{giles1989higher} has shown that with a large enough hidden state size, such recurrent structures are capable of approximating any dynamical system.

Second, to learn the nonlinear dynamics $\xi$ efficiently, we also use higher-order moments to approximate the state transition function.
%
We use an augmented state $\V{s}$, where we mute the subscript of $\V{s}_{t-1}$ for notation simplicity.:
\begin{equation}
	% \V{s}_{t-1} \otimes \cdots \otimes \V{s}_{t-1} \quad
	\V{s}^T = [1 \hspace{5pt} \V{h}_{t-1}^\top \hspace{5pt} \ldots \hspace{5pt} \V{h}_{t-L}^\top ]
\end{equation}
which concatenates $L$ previous hidden states.
% 
To compute $\V{h}_t$, we construct a $P$-dimensional transition \ti{weight tensor} to model degree-$P$ polynomial interactions between hidden states:
%Hence, the \trnn{} with standard RNN cell is defined by:
%
\begin{align}
[\V{h}_{t}]_\alpha = \phi(W^{hx}_\alpha\V{x}_t+  
    \sum_{i_1,\cdots, i_p}\T{W}_{\alpha i_1 \cdots i_{P}}  \underbrace{\V{s}_{i_1} \otimes\cdots\otimes \V{s}_{i_p} }_{P} )\nonumber
%
\label{eqn:tensor_rnn}
\end{align}
%
where $\alpha$ indices the hidden dimension, $i_\cdot$ indices the higher-order terms and $P$ is the total  polynomial order. We included the bias unit $1$ in $\V{s}$ to account for the first order term, so that  $\V{s}_{i_1} \otimes\cdots\otimes \V{s}_{i_p} = [1, \V{h}_t, \V{h}_t\V{h}_{t-1},\cdots]$ can include all  polynomial expansions of hidden states up to order $P$. 

%
% where $\otimes$ is the tensor product.
%
%
The \trnn{} with LSTM cell, or ``\tlstm{}'', is defined analogously as:
\begin{align}
[\V{i}_t, \V{g}_t, &\V{f}_t, \V{o}_t]_\alpha = \sigma (W^{hx}_\alpha \V{x}_t + \sum_{i_1,\cdots, i_p}\T{W}_{\alpha i_1 \cdots i_{P}}  \underbrace{\V{s}_{i_1} \otimes\cdots\otimes \V{s}_{i_P} }_{P} ), \\
%
& \V{c}_t = \V{c}_{t-1} \circ \V{f}_t +  \V{i}_t\circ \V{g}_t,
%
\qquad
%
\V{h}_t = \V{c}_t \circ \V{o}_t \nonumber
%
\end{align}
%
where $\circ$ denotes the Hadamard product. Note that the bias units are again included.

 \trnn{} is a basic  unit that can be incorporated in most of the existing recurrent neural architectures such as convolutional RNN \citep{xingjian2015convolutional} and hierarchical RNN \citep{chung2016hierarchical}. In this work, we use  \trnn{} as a module for sequence-to-sequence (seq2seq) framework \citep{sutskever2014sequence} in order to perform long-term forecasting.


As shown in Figure \ref{fig:seq2seq}, seq2seq models consist of an encoder-decoder pair. The encoder takes an input sequence and learns a hidden representation. The decoder initializes with this hidden representation and generates an output sequence. Both the encoder and the decoder contain multiple layers of higher-order tensor recurrent cells (red).
% 
The augmented state $\V{s}_{t-1}$ (grey) concatenates the past $L$ hidden states;
% 
the \trnn{} cell takes $\V{s}_{t-1}$ and outputs the next hidden state. 
% 
The encoder encodes the initial states $x_{0}, \ldots, x_{t}$ and the decoder predicts $x_{t+1}, \ldots, x_{T}$. 
% 
For each time step $t$, the decoder uses its previous prediction $\V{y}_t$ as an input.
%
\subsection{Dimension Reduction with Tensor-Train}
% 
Unfortunately, due to the ``curse of dimensionality'', the number of parameters in $\T{W}_\alpha$ with hidden size $H$ grows exponentially as $O(HL^P)$, which makes the higher-order model prohibitively large to train. To overcome this difficulty, we  utilize   \textit{tensor networks} to approximate the weight tensor. Such networks encode a structural decomposition of tensors into low-dimensional components and have been shown to provide the most general approximation to smooth tensors \citep{orus2014practical}.
%
The most commonly used tensor networks are \textit{linear tensor networks} (LTN), also known as \textit{tensor-trains} in numerical analysis or \textit{matrix-product states} in quantum physics \citep{oseledets2011tensor}.

A tensor train model decomposes a $P$-dimensional tensor $\T{W}$ into a network of sparsely connected low-dimensional tensors $\{\T{A}^p \in \R^{r_{p-1} \times n_p \times r_{p}} \}$ as:
%
\begin{equation*}
\T{W}_{i_1 \cdots i_P} =
\sum_{\alpha_1 \cdots \alpha_{P-1}}
\T{A}^1_{\alpha_0 i_1 \alpha_1}%
\T{A}^2_{\alpha_1 i_2 \alpha_2}%
\cdots%
\T{A}^P_{\alpha_{P-1} i_P \alpha_P}
\nonumber
\end{equation*}
%
 with $ \alpha_0 = \alpha_P = 1$, as depicted in Figure (\ref{fig:ttrnn}). When $r_0 = r_{P} = 1$ the $\{r_p\}$ are called the tensor-train rank.
%
With tensor-train decomposition, we can reduce the number of parameters of \trnn{} from $(HL+1)^{P}$ to $(HL+1)R^2P$, with $R = \max_p{r_p}$ as the upper bound on the tensor-train rank.
%
Thus, a major benefit of tensor-train is that they \textit{do not} suffer from the curse of dimensionality, which is in sharp contrast to many classical tensor decomposition models, such as the Tucker decomposition.

%\aacomment{theory should be a separate section}



%
\section{Approximation Theorem for HOT-RNNs}
\label{thm}
%\paragraph{Theoretical Analysis}
%
A significant benefit of using \trnn{} is that we can theoretically characterize its expressiveness  for approximating the underlying dynamics.   The main idea is to analyze a class of functions that satisfies certain regularity conditions. For such functions, tensor-train representations preserve the weak differentiability and yield a compact representation.


The following theorem characterizes the representation power of \trnn{}, viewed as a one-layer hidden neural network, in terms of 1) the regularity of the target function $f$, 2) the dimension of the input space, 3) the tensor train rank and 4) the order of the tensor:
%
\begin{theorem}
Let the target function $f\in \mathcal{H}^k_\mu$ be a H\"older continuous function defined on a input  domain $\mathcal{I} =I_1\times \cdots \times I_d$, with  bounded derivatives up to order $k$ and finite Fourier magnitude distribution $C_f$. A single layer \trnn{} with $h$ hidden units, $\hat{f}$ can approximate $f$ with approximation error $\epsilon$ at most:
%
\eq{
%
\epsilon \leq \fr{1}{h} \brck{C_f^2 \frac{d-1}{(k-1)(r+1)^{k-1}} + C(k)p^{-k} }
%
}
%
where $C_f = \int |\omega|_1 |\hat{f}(\omega) d \omega|$, $d$ is the dimension of the function, i.e., the size of the state space, $r$ is the tensor-train rank, $p$ is the degree of the higher-order polynomials i.e., the order of the tensor, and $C(k)$ is the coefficient of the spectral expansion of $f$.
\label{eqn:thm}
\end{theorem}

\textbf{Remarks}: The result above shows that the number of weights required to approximate the target function $f$ is dictated by its regularity (i.e., its H\"older-continuity order $k$). The expressiveness of \trnn{} is driven by the selection of the rank $r$ and the polynomial degree $p$; moreover, it improves for functions with increasing regularity.  Compared with ``first-order'' regular RNNs, \trnn{}s are  exponentially more powerful for large rank: if the order  $p$ increases, we require fewer hidden units $h$. 
% 
% The results also provide intuitions for choosing the hidden size and the rank for optimal storage. \ryedit{don't know what storage means}
%
% as long as we are given a state transitions $(\V{x}_t, \V{s}_t) \mapsto \V{s}_{t+1}$ (e.g. the state transition function learned by the  encoder).

\ti{Proof sketch}: 
For the full proof, see the Appendix. 
% 
We design \trnn{} to approximate the underlying system dynamics. The target function $f(\V{x})$ represents the state transition function, as in \refn{eqn:tensor_rnn}.
%
We first show that if $f$ preserves weak derivatives, then it has a compact tensor-train representation. Formally, let us assume that $f$ is a Sobolev function: $f\in\mathcal{H}^k_\mu$, defined on the input space $\T{I}= I_1 \times I_2\times \cdots I_d $, where each $I_i$ is a set of vectors. The space $\mathcal{H}^k_\mu$ is defined as the functions that have bounded derivatives up to some order $k$ and are $L_\mu$-integrable.
%
\begin{eqnarray}
\mathcal{H}^k_\mu =  \left\{  f  \in L_\mu(\T{I}):\sum_{i\leq k}\|D^{(i)}f\|^2   < +\infty \right\},
\end{eqnarray}
%
where $D^{(i)}f$ is the $i$-th weak derivative of $f$ and $\mu \geq 0$.\footnote{A weak derivative generalizes the derivative concept for (non)-differentiable functions and is implicitly defined as: e.g. $v\in L^1([a,b])$ is a weak derivative of $u\in L^1([a,b])$ if for all smooth $\varphi$ with $\varphi(a) = \varphi(b) = 0$: $\int_a^bu(t)\varphi'(t) = -\int_a^bv(t)\varphi(t)$.}
% 
% The space $\mathcal{H}^k_\mu$ is equipped with a norm $\|f\|^2_{k,\mu} =\sum_{|i|\leq k}\|D^{(i)}f\|^2$ and a semi-norm $|f|^2_{k,\mu} =\sum_{|i|=k}\|D^{(i)}f\|^2 $. For notation simplicity, we muted the subscript $\mu$ and used $\|\cdot\| $ for $\|\cdot \|_{L_{\mu}}$.
% f
It is known that any Sobolev function admits a Schmidt decomposition: $f(\cdot) = \sum_{i =0}^\infty \sqrt{\lambda_i } \gamma (\cdot)_i \otimes \phi (\cdot)_i $, where $\{\lambda \}$ are the eigenvalues and $\{\gamma\}, \{ \phi\}$ are the associated eigenfunctions.
%
Hence, for $\V{x}\in\calI$, we can represent the target function $f(\V{x}) $ as an infinite summation of products of a set of basis functions:
%
\begin{align}
&f(\V{x}) = \sum_{\alpha_0,\cdots,\alpha_d=1}^\infty
%
% \T{A}^1_{\alpha_0, x_1, \alpha_1}
\T{A}^1 (x_1)_{\alpha_0 \alpha_1}
%
\cdots
%
\T{A}^d(x_d)_{\alpha_{d-1}	 \alpha_d},
\label{eqn:ftt}
\end{align}
%
where  $ \{ \T{A}^j(x_j)_{\alpha_{j-1} \alpha_j} \}$ are basis functions over each input dimension.
% = \sqrt{\lambda^{d-1}_{\alpha_{d-1}}} \phi (x_d)_{\alpha_{d-1}}
These basis functions satisfy $\langle \T{A}^j(\cdot)_{im}, \T{A}^j (\cdot)_{in} \rangle = \delta_{mn}$ for all $j$.
% 
If we truncate \eqref{eqn:ftt} to a low dimensional subspace ($\V{r}<\infty$), we obtain a functional approximation of the state transition function $f(\V{x})$.  
% 
This approximation is also known as the \ti{functional tensor-train} (FTT):
%
\begin{align}
&f_{FTT}(\V{x}) = \sum_{\alpha_0,\cdots,\alpha_d}^\mathbf{r}
%
\T{A}^1(x_1)_{\alpha_0\alpha_1}
%
\cdots
%
\T{A}^d(x_d)_{\alpha_{d-1}\alpha_d},
\end{align}

In practice, \trnn{} implements a polynomial expansion of the states using $[\V{s}, \V{s}^{\otimes 2}, \cdots, \V{s}^{\otimes P}]$, where $P$ is the degree of the polynomial. The final function represented by \trnn{} is a polynomial approximation of the functional tensor-train function $f_{FTT}$. 

Given a target  function $f(\V{x}) = f(\V{s}\otimes \dots \otimes \V{s})$, we can express it using FTT and the polynomial expansion of the states $\V{s}$. This allows us to characterize \trnn{} using a family of functions that it can represent.  Combined with the classic neural network approximation theory \cite{barron1993universal}, we can bound the approximation error for \trnn{} with one hidden layer. The above results applies to the full family of \trnn{}s, including those using  vanilla RNN or LSTM as the recurrent cell.

One can think of the universal approximation result in Theorem \ref{eqn:thm} bounds the estimation bias of the model: $f-\mathbb{E} [\hat{f}]$, where the expectation is taken over training sets. While a large neural network can approximate any function, training as a large neural network will be hard given a finite data set, demonstrating bias-variance trade-off. In the next section, we provide bounds for the estimation variance. 
 
%
% Given a target function $f$, and a neural network with one hidden layer and sigmoid activation function, the following lemma describes the classic result of  describing the error between  $f$ and the single hidden-layer neural network that approximates it best:
%
% \begin{lemma}[NN Approximation \cite{barron1993universal}]
% 	Given a function $f$ with finite Fourier magnitude distribution $C_f$, there exists a  neural network of $n$ hidden units $f_n$, such that
% 	\begin{eqnarray}
% 		\| f - f_n\| \leq \frac{C_f}{\sqrt{n}}
% 	\end{eqnarray}
% 	where $C_f = \int |\omega|_1  | \hat{f}(\omega) | d \omega$ with Fourier representation $f(x)=\int e^{i\omega x}\hat{f}(\omega) d\omega$.
% 	\label{lemma:nn}
% \end{lemma}
% 
% We can  generalize Barron's approximation result in  lemma \ref{lemma:nn} to \trnn{}.  



% As the rank of the tensor-train and the polynomial order increase, the required size of the hidden units become smaller, up to a constant that depends on the regularity of the underlying dynamics $f$.
%
% This theorem applies to  the entire family of \trnn{}s, including those using vanilla RNN or LSTM as the recurrent cell, as long as we are given a state transition function $(\V{x}_t, \V{s}_t) \mapsto \V{s}_{t+1}$. (e.g. the state transition function learned by the  encoder).
%
% In this case, Theorem \ref{eqn:thm} bounds the estimation error when approximating a state-transition function $f:\V{s}_t \mapsto \V{x}_{t+1}$ using a tensor-train layer.
%
% However, the general case where the state-encoder is not known.
% For instance, when using $\V{x} = [\V{s} \V{s} \V{s}]$, we recover the.tensor transition function in
%
% $N$ refers to number of state transitions (or examples) seen during training. The fewer transitions we observe, the less information we get about the true dynamics, and we need to memorize more information for the same approximation error.
% \stzedit{If we assume that the Does this apply to LSTMs as well?}


\section{Variance Bound for \trnn{}}
\label{gen}
\begin{figure}[t]
  \centering
  \includegraphics[width=\linewidth]{figs/exps/gen_fig.pdf}
  \caption{Qualitative comparisons on the self-augmented dataset for \editb \  learning. The editing prompt is "S* as a police, looking at the camera". "w/o edit" and "w/o recon" denote for the encoder is trained without \editb \ learning objective and without reconstruction learning, respectively. We show that the generated images can not follow the prompt properly without the \editb \ learning. Meanwhile, the face similarity will be lower without the reconstruction learning on FFHQ.}
  \label{fig:exp_self_aug}
\end{figure}
%
\section{Experiments}
\label{exp}
\section{Experiment \& Analysis}
\label{sec:exp}
In this section, we first introduce the experimental set-up. Then, we report the performances of baselines and the proposed steep slope loss on ImageNet, followed by comprehensive analyses. 
% At last, we present comprehensive analyses to help better understand the efficacy of the proposed loss.

\noindent\textbf{Experimental Set-Up}.
We use ViT B/16 \cite{Dosovitskiy_ICLR_2021} and ResNet-50 \cite{He_CVPR_2016} as the classifiers, and the respective backbones are used as the oracles' backbones. We denote the combination of oracles and classifiers as \textlangle O, C\textrangle. There are four combinations in total, \ie \textlangle ViT, ViT\textrangle, \textlangle ViT, RSN\textrangle, \textlangle RSN, ViT\textrangle, and \textlangle RSN, RSN\textrangle, where RSN stands for ResNet.
In this work, we adopt three baselines, \ie the cross entropy loss \cite{Cox_JRSS_1972}, focal loss \cite{Lin_ICCV_2017}, and TCP confidence loss \cite{Corbiere_NIPS_2019}, for comparison purposes.

The experiment is conducted on ImageNet \cite{Deng_CVPR_2009}, which consists of 1.2 million labeled training images and 50000 labeled validation images. It has 1000 visual concepts. Similar to the learning scheme in \cite{Corbiere_NIPS_2019}, the oracle is trained with training samples and evaluated on the validation set. During the training process of the oracle, the classifier works in the evaluation mode so training the oracle would not affect the parameters of the classifier. Moreover, we conduct the analyses about how well the learned oracle generalizes to the images in the unseen domains. To this end, we apply the widely-used style transfer method \cite{Geirhos_ICLR_2019} and the functional adversarial attack method \cite{Laidlaw_NeurIPS_2019} to generate two variants of the validation set, \ie stylized validation set and adversarial validation set. \REVISION{Also, we adopt ImageNet-C \cite{Hendrycks_ICLR_2018} for evaluation, which is used for evaluating robustness to common corruptions.}
% Then, the oracle trained with regular training samples would be evaluated with the samples that are in the two unseen domains.

% To understand how the learned oracle work on unseen domains, the oracle is learned with training samples and is evaluated with three types of samples, the samples on the same domain as training samples and the samples on two unseen domains. We base our experiments on ImageNet \cite{Deng_CVPR_2009}, a widely-used large-scale dataset. Except for the training set and the validation set, we use the stylized ImageNet validation set \cite{Geirhos_ICLR_2019} and an ImageNet validation set that are perturbed by the functional adversarial attack technique \cite{Laidlaw_NeurIPS_2019}.
% (Introduce models here.)
% (Introduce hyperpaprameters here.)

The oracle's backbone is initialized by the pre-trained classifier's backbone and trained by fine-tuning using training samples and the trained classifier.
% As the oracle's backbone is initialized by the pre-trained classifier's backbone, the training process of the oracles is equivalent to the process of fine-tuning the initialized oracles.
Training the oracles with all the loss functions uses the same hyperparameters, such as learning rate, weight decay, momentum, batch size, etc.
The details for the training process and the implementation are provided in \appref{sec:implementation}.

For the focal loss, we follow \cite{Lin_ICCV_2017} to use $\gamma=2$,  which leads to the best performance for object detection.
For the proposed loss, we use $\alpha^{+}=1$ and $\alpha^{-}=3$ for the oracle that is based on ViT's backbone, while we use $\alpha^{+}=2$ and $\alpha^{-}=5$ for the oracle that is based on ResNet's backbone.

Following \cite{Corbiere_NIPS_2019}, we use FPR-95\%-TPR, AUPR-Error, AUPR-Success, and AUC as the metrics.
FPR-95\%-TPR is the false positive rate (FPR) when true positive rate (TPR) is equal to 95\%. 
AUPR is the area under the precision-recall curve. 
Specifically, AUPR-Success considers the correct prediction as the positive class, whereas AUPR-Error considers the incorrect prediction as the positive class.
AUC is the area under the receiver operating characteristic curve, which is the plot of TPR versus FPR.
Moreover, we use TPR and true negative rate (TNR) as additional metrics because they assess overfitting issue, \eg TPR=100\% and TNR=0\% imply that the trustworthiness predictor is prone to view all the incorrect predictions to be trustworthy. %due to overfitting.

% \noindent\textbf{Baselines \& Metrics}.
% We adopt widely-used loss functions, \ie cross entropy and focal loss, as the baselines. To comprehensively understand and measure oracles' performance, we use KL divergence and Bhattacharya coefficient to measure the correlation between two feature distributions, use true positive rate (TPR), true negative rate (TNR), accuracy (Acc=$(TP+TN)/Total$), F1 score, precision (P), and recall (R) to measure the efficacy of predicting trustworthiness. Specifically, we add Acc\textsubscript{P} and Acc\textsubscript{N} to understand how much TP and TN contribute to Acc. This is useful when the model overfits the data, \ie classifying all the images as either positives or negatives. Moreover, to differentiate the accuracy of classification from the accuracy of predicting trustworthiness, we denote the classifier's accuracy as C-Acc, and the oracle's accuracy as O-Acc.

% \begin{table}[!t]
	\centering
	\caption{\label{tbl:avg_perf}
	    Averaged performance over the regular ImageNet validation set, the stylized ImageNet validation set, and the adversarial ImageNet validation set. The oracle is trained with the cross entropy (CE) loss, the focal loss, and the proposed steep slope (SS) loss on the ImageNet training set. The resulting oracles w.r.t. each loss are evaluated on the three validation sets. The classifier is used in the evaluation mode in the experiment. $d_{KL}$ represents KL divergence, while $c_{B}$ represents Bhattacharyya coefficient.
	}
	\adjustbox{width=1\columnwidth}{
	\begin{tabular}{L{7ex} C{8ex} C{8ex} C{8ex} C{10ex} C{8ex} C{8ex} C{8ex} C{8ex} C{8ex} C{10ex} C{8ex}}
		\toprule
		Loss & C-Acc & $d_{KL}\uparrow$ & $c_{B}\downarrow$ & TPR & TNR & O-Acc & O-Acc\textsubscript{P} & O-Acc\textsubscript{N} & F1 & P & R  \\
		\cmidrule(lr){1-1} \cmidrule(lr){2-2} \cmidrule(lr){3-3} \cmidrule(lr){4-4} \cmidrule(lr){5-5} \cmidrule(lr){6-6} \cmidrule(lr){7-7} \cmidrule(lr){8-8} \cmidrule(lr){9-9} \cmidrule(lr){10-10} \cmidrule(lr){11-11} \cmidrule(lr){12-12}
		& \multicolumn{11}{c}{Oracle: ViT, classifier: ViT} \\
		\cmidrule(lr){2-12}
		CE & 35.74 & 0.5138 & 0.9983 & 99.98 & 0.04 & 35.78 & 35.74 & 0.04 & 0.4382 & 0.3575 & 0.8444 \\
        Focal & 35.74 & 0.5224 & 0.9972 & 99.23 & 1.30 & 36.22 & 35.43 & 0.78 & 0.4374 & 0.3579 & 0.8403 \\
        SS & 35.74 & 1.0875 & 0.9302 & 73.62 & 47.23 & 63.94 & 29.84 & 34.10 & 0.4727 & 0.4430 & 0.5964 \\ \midrule
		& \multicolumn{11}{c}{Oracle: ResNet, classifier: ViT} \\
		\cmidrule(lr){2-12}
		CE & & & & & & & & & & & \\
		Focal & & & & & & & & & & & \\
		SS & & & & & & & & & & & \\
		\bottomrule	
	\end{tabular}}
\end{table}

% \begin{figure}[!t]
% 	\centering
% 	\subfloat{\includegraphics[width=0.32\textwidth]{fig/sigmoid_imagenet_trfeat}    } \hfill
% 	\subfloat{\includegraphics[width=0.32\textwidth]{fig/focal_imagenet_trfeat}    } \hfill
% 	\subfloat{\includegraphics[width=0.32\textwidth]{fig/steep_imagenet_trfeat}    } \\
% 	\subfloat{\includegraphics[width=0.32\textwidth]{fig/sigmoid_imagenet_valfeat}    } \hfill
% 	\subfloat{\includegraphics[width=0.32\textwidth]{fig/focal_imagenet_valfeat}    } \hfill
% 	\subfloat{\includegraphics[width=0.32\textwidth]{fig/steep_imagenet_valfeat}    } \\
% 	\subfloat{\includegraphics[width=0.32\textwidth]{fig/sigmoid_imagenet_valfeat_sty}    } \hfill
% 	\subfloat{\includegraphics[width=0.32\textwidth]{fig/focal_imagenet_valfeat_sty}    } \hfill
% 	\subfloat{\includegraphics[width=0.32\textwidth]{fig/steep_imagenet_valfeat_sty}    } \\
% 	\subfloat{\includegraphics[width=0.32\textwidth]{fig/sigmoid_imagenet_valfeat_adv}    } \hfill
% 	\subfloat{\includegraphics[width=0.32\textwidth]{fig/focal_imagenet_valfeat_adv}    } \hfill
% 	\subfloat{\includegraphics[width=0.32\textwidth]{fig/steep_imagenet_valfeat_adv}    }
% 	\caption{\label{fig:distribution}
%     	Feature distributions w.r.t. the cross entropy (first column), focal (second column), and the proposed steep slope (third column) losses on the ImageNet training set (second row), ImageNet validation set (first row), stylized ImageNet validation set (third row), and adversarial ImageNet validation set (fourth row). CE stands for cross entropy, while SS stands for steep slope.
%     % 	\REVISION{\textit{Baseline} indicates ResNet GEM.}
%     	}
% \end{figure}

% \begin{table}[!t]
	\centering
	\caption{\label{tbl:perf_rsn_vit}
	    Performances on the regular ImageNet validation set, the stylized ImageNet validation set, and the adversarial ImageNet validation set. In this experiment, ResNet-50 is used for the oracle backbone while ViT is used for the classifier. The classifier is used in the evaluation mode in the experiment.
	}
	\adjustbox{width=1\columnwidth}{
	\begin{tabular}{L{8ex} C{8ex} C{8ex} C{8ex} C{10ex} C{8ex} C{8ex} C{8ex}}
		\toprule
		Loss & Acc$\uparrow$ & FPR-95\%-TPR$\downarrow$ & AURP-Error$\uparrow$ & AURP-Success$\uparrow$ & AUC$\uparrow$ & TPR$\uparrow$ & TNR$\uparrow$ \\
		\cmidrule(lr){1-1} \cmidrule(lr){2-2} \cmidrule(lr){3-3} \cmidrule(lr){4-4} \cmidrule(lr){5-5} \cmidrule(lr){6-6} \cmidrule(lr){7-7} \cmidrule(lr){8-8}
		& \multicolumn{7}{c}{Regular validation set} \\
		\cmidrule(lr){1-1} \cmidrule(lr){2-8}
		CE & 83.90 & 92.58 & 14.59 & 85.57 & 53.78 & 100.00 & 0.00 \\
		Focal & 83.90 & 94.92 & 14.87 & 85.26 & 52.49 & 100.00 & 0.00 \\
		TCP & 83.90 & 91.63 & 14.17 & 86.06 & 55.37 & 100.00 & 0.00 \\
% 		SS & 83.90 & 89.86 & 11.99 & 89.49 & 62.75 & 67.74 & 48.98 \\
		SS & 83.90 & 88.63 & 11.75 & 89.87 & 64.11 & 95.41 & 10.48 \\
% 		& 83.90 & 88.63 & 11.75 & 89.87 & 64.11 & 95.41 & 10.48 \\
%         & 83.90 & 88.72 & 11.76 & 89.85 & 64.01 & 91.10 & 18.51 \\
% 		& 83.90 & 88.25 & 11.54 & 90.24 & 65.23 & 88.23 & 24.25 \\ rsn152
		\midrule
		& \multicolumn{7}{c}{Stylized validation set} \\
		\cmidrule(lr){1-1} \cmidrule(lr){2-8}
		CE & 15.94 & 86.54 & 79.32 & 22.00 & 61.74 & 100.00 & 0.00 \\
		Focal & 15.94 & 95.04 & 85.18 & 14.94 & 48.20 & 100.00 & 0.00 \\
		TCP & 15.94 & 90.82 & 80.69 & 19.96 & 58.34 & 100.00 & 0.00 \\
		SS & 15.94 & 93.80 & 82.10 & 18.19 & 54.18 & 56.94 & 48.88 \\
% 		& 15.94 & 94.27 & 83.09 & 16.97 & 52.35 & 92.13 & 9.03 \\ 52
% 		& 15.94 & 95.76 & 84.28 & 15.74 & 49.02 & 93.83 & 5.52 \\ a62
        \midrule
		& \multicolumn{7}{c}{Adversarial validation set} \\
		\cmidrule(lr){1-1} \cmidrule(lr){2-8}
        CE & & & & & & &  \\
        Focal & & & & & & &  \\
        TCP & & & & & & & \\
        SS & & & & & & &  \\
		\bottomrule	
	\end{tabular}}
\end{table}



% \begin{figure}[!t]
	\centering
	\subfloat[\textlangle ViT, ViT\textrangle]{\includegraphics[width=0.24\textwidth]{fig/risk/risk_vit_vit}} \hfill
	\subfloat[\textlangle ViT, RSN\textrangle]{\includegraphics[width=0.24\textwidth]{fig/risk/risk_vit_rsn}} \hfill
	\subfloat[\textlangle RSN, ViT\textrangle]{\includegraphics[width=0.24\textwidth]{fig/risk/risk_rsn_vit}}
	\hfill
	\subfloat[\textlangle RSN, RSN\textrangle]{\includegraphics[width=0.24\textwidth]{fig/risk/risk_rsn_rsn}}
% 	\subfloat[O: ViT, C: ViT, Loss: TCP]{\includegraphics[width=0.24\textwidth]{fig/tsne/tsne_tcp}    } \hfill
% 	\subfloat[O: ViT, C: ViT, Loss: SS]{\includegraphics[width=0.24\textwidth]{fig/tsne/tsne_steep}    } 
    \\
	\caption{\label{fig:anal_risk}
    	Curves of risk vs. coverage. Selective risk represents the percentage of errors in the remaining validation set for a given coverage.
    	The curves correspond to the oracles used in \tabref{tbl:all_perf_w_std}.
    % 	\REVISION{\textit{Baseline} indicates ResNet GEM.}
    	}
\end{figure}

% \begin{figure}[!t]
	\centering
	\subfloat[O: ViT, C: ViT, Loss: CE]{\includegraphics[width=0.24\textwidth]{fig/tsne/tsne_ce}    } \hfill
	\subfloat[O: ViT, C: ViT, Loss: Focal]{\includegraphics[width=0.24\textwidth]{fig/tsne/tsne_focal}    } \hfill
	\subfloat[O: ViT, C: ViT, Loss: TCP]{\includegraphics[width=0.24\textwidth]{fig/tsne/tsne_tcp}    } \hfill
	\subfloat[O: ViT, C: ViT, Loss: SS]{\includegraphics[width=0.24\textwidth]{fig/tsne/tsne_steep}    } \\
	\caption{\label{fig:anal_tsne}
    	Analysis of t-SNE.
    % 	\REVISION{\textit{Baseline} indicates ResNet GEM.}
    	}
\end{figure}

% \begin{table}[!t]
% 	\centering
% 	\caption{\label{tbl:noise}
% 	    Correctness of oracle on the ImageNet validation set. The oracles are trained with the ImageNet training set. The underlined architecture indicates the architecture of Bayesian network. Leave-out rate indicates the proportion of samples that are ruled out by the oracle. Ideally, it should be equivelant to 1-Acc.
% 	}
% 	\adjustbox{width=1\columnwidth}{
% 	\begin{tabular}{L{10ex} C{12ex} C{12ex} C{9ex} C{9ex} C{9ex} C{9ex} C{9ex} C{9ex} C{9ex}}
% 		\toprule
% 		Dataset & Oracle & Classifier & Acc & O-Acc & O-TP & O-FP & F1 & Precision & Recall \\
% 		\cmidrule(lr){1-1} \cmidrule(lr){2-2} \cmidrule(lr){3-3} \cmidrule(lr){4-4} \cmidrule(lr){5-5} \cmidrule(lr){6-6} \cmidrule(lr){7-7} \cmidrule(lr){8-8} \cmidrule(lr){9-9} \cmidrule(lr){10-10}
% 		Regular & ViT-sigm            & ViT & 83.90 & 83.93 & 83.41 & 15.57 & 0.9121 & 0.8426 & 0.9941    \\
% 		Regular & ViT-Gauss            & ViT & 83.90 & 83.95 & 83.26 & 15.41 & 0.9121 & 0.8438 & 0.9924    \\
% 		Regular & ViT-exp            & ViT & 83.90 & 82.11 &  &  &  &  &     \\  \midrule
% 		Stylized & ViT-sigm            & ViT & 15.93 & 20.62 & 15.36 & 78.79 & 0.2790 & 0.1631 & 0.9639    \\
% 		Stylized & ViT-Gauss            & ViT & 15.93 & 46.28 & 13.01 & 50.79 & 0.3263 & 0.2039 & 0.8163    \\
% 		Stylized & ViT-exp            & ViT & 15.93 & 72.23 &  &  &  &  &     \\ \midrule
% 		Adv & ViT-sigm            & ViT & 7.41 & 11.23 & & & 0.1307 & 0.0762 & 0.5336    \\
% 		Adv & ViT-Gauss            & ViT & 7.41 & 11.15 & 7.14 & 88.79 & 0.1270 & 0.0744 & 0.5088 \\
% 		Adv & ViT-exp            & ViT & 7.41 & 32.57 &  &  &  &  &     \\ 
% 		\bottomrule	
% 	\end{tabular}}
% \end{table}

% \begin{table}[!t]
	\centering
	\caption{\label{tbl:all_perf}
	    Performance on the ImageNet validation set. The averaged scores are computed over three runs. The oracles are trained with the ImageNet training samples. The classifier is used in the evaluation mode in the experiment. Acc is the classification accuracy (\%) and is helpful to understand the proportion of correct predictions. \textit{SS} stands for the proposed steep slope loss.
	   % For example, Acc=83.90\% implies that 83.90\% of predictions is trustworthy and 16.10\% of predictions is untrustworthy.
	}
	\adjustbox{width=1\columnwidth}{
	\begin{tabular}{C{10ex} L{9ex} C{8ex} C{10ex} C{8ex} C{8ex} C{8ex} C{8ex} C{8ex}}
		\toprule
		\textbf{\textlangle O, C\textrangle} & \textbf{Loss} & \textbf{Acc$\uparrow$} & \textbf{FPR-95\%-TPR$\downarrow$} & \textbf{AUPR-Error$\uparrow$} & \textbf{AUPR-Success$\uparrow$} & \textbf{AUC$\uparrow$} & \textbf{TPR$\uparrow$} & \textbf{TNR$\uparrow$} \\
		\cmidrule(lr){1-1} \cmidrule(lr){2-2} \cmidrule(lr){3-3} \cmidrule(lr){4-4} \cmidrule(lr){5-5} \cmidrule(lr){6-6} \cmidrule(lr){7-7} \cmidrule(lr){8-8} \cmidrule(lr){9-9} 
		\multirow{4}{*}{\textlangle ViT, ViT \textrangle} & CE & 83.90 & 93.01 & \textbf{15.80} & 84.25 & 51.62 & \textbf{99.99} & 0.02 \\
		 & Focal \cite{Lin_ICCV_2017} & 83.90 & 93.37 & 15.31 & 84.76 & 52.38 & 99.15 & 1.35 \\
		 & TCP \cite{Corbiere_NIPS_2019} & 83.90 & 88.38 & 12.96 & 87.63 & 60.14 & 99.73 & 0.00 \\
		 & SS & 83.90 & \textbf{80.48} & 10.26 & \textbf{93.01} & \textbf{73.68} & 87.52 & \textbf{38.27} \\
		\midrule
		\multirow{4}{*}{\textlangle ViT, RSN\textrangle} & CE & 68.72 & 93.43 & 30.90 & 69.13 & 51.24 & \textbf{99.90} & 0.20 \\
		 & Focal \cite{Lin_ICCV_2017} & 68.72 & 93.94 & \textbf{30.97} & 69.07 & 51.26 & 93.66 & 7.71 \\
		 & TCP \cite{Corbiere_NIPS_2019} & 68.72 & 83.55 & 23.56 & 79.04 & 66.23 & 94.25 & 0.00 \\
		 & SS & 68.72 & \textbf{77.89} & 20.91 & \textbf{85.39} & \textbf{74.31} & 68.32 & \textbf{67.53} \\
        \midrule
		\multirow{4}{*}{\textlangle RSN, ViT\textrangle} & CE & 83.90 & 93.29 & 14.74 & 85.40 & 53.43 & \textbf{100.00} & 0.00 \\
		 & Focal \cite{Lin_ICCV_2017} & 83.90 & 94.60 & \textbf{14.98} & 85.13 & 52.37 & \textbf{100.00} & 0.00 \\
		 & TCP \cite{Corbiere_NIPS_2019} & 83.90 & 91.93 & 14.12 & 86.12 & 55.55 & \textbf{100.00} & 0.00 \\
         & SS & 83.90 & \textbf{88.70} & 11.69 & \textbf{90.01} & \textbf{64.34} & 96.20 & \textbf{9.00} \\
% 		RSN & ViT & SS & 83.90 & 89.86 & 11.99 & 89.49 & 62.75 & 67.74 & 48.98 \\
        \midrule
        \multirow{4}{*}{\textlangle RSN, RSN\textrangle} & CE & 68.72 & 94.84 & 29.41 & 70.79 & 52.36 & \textbf{100.00} & 0.00 \\
		 & Focal \cite{Lin_ICCV_2017} & 68.72 & 95.16 & \textbf{29.92} & 70.23 & 51.43 & 99.86 & 0.08 \\
		 & TCP \cite{Corbiere_NIPS_2019} & 68.72 & 88.81 & 24.46 & 77.79 & 62.73 & 99.23 & 0.00 \\
         & SS & 68.72 & \textbf{86.21} & 22.53 & \textbf{81.88} & \textbf{67.92} & 79.20 & \textbf{42.09} \\
		\bottomrule	
	\end{tabular}}
\end{table}
\begin{table}[!t]
	\centering
	\caption{\label{tbl:all_perf_w_std}
	    Performance on the ImageNet validation set. The mean and the standard deviation of scores are computed over three runs. The oracles are trained with the ImageNet training samples. The classifier is used in the evaluation mode. Acc is the classification accuracy and is helpful to understand the proportion of correct predictions. \textit{SS} stands for the proposed steep slope loss.
	   % For example, Acc=83.90\% implies that 83.90\% of predictions is trustworthy and 16.10\% of predictions is untrustworthy.
	}
	\adjustbox{width=1\columnwidth}{
	\begin{tabular}{C{10ex} L{10ex} C{8ex} C{10ex} C{10ex} C{10ex} C{10ex} C{10ex} C{10ex}}
		\toprule
		\textbf{\textlangle O, C\textrangle} & \textbf{Loss} & \textbf{Acc$\uparrow$} & \textbf{FPR-95\%-TPR$\downarrow$} & \textbf{AUPR-Error$\uparrow$} & \textbf{AUPR-Success$\uparrow$} & \textbf{AUC$\uparrow$} & \textbf{TPR$\uparrow$} & \textbf{TNR$\uparrow$} \\
		\cmidrule(lr){1-1} \cmidrule(lr){2-2} \cmidrule(lr){3-3} \cmidrule(lr){4-4} \cmidrule(lr){5-5} \cmidrule(lr){6-6} \cmidrule(lr){7-7} \cmidrule(lr){8-8} \cmidrule(lr){9-9} 
		\multirow{4}{*}{\textlangle ViT, ViT \textrangle} & CE & 83.90 & 93.01$\pm$0.17 & \textbf{15.80}$\pm$0.56 & 84.25$\pm$0.57 & 51.62$\pm$0.86 & \textbf{99.99}$\pm$0.01 & 0.02$\pm$0.02 \\
		 & Focal \cite{Lin_ICCV_2017} & 83.90 & 93.37$\pm$0.52 & 15.31$\pm$0.44 & 84.76$\pm$0.50 & 52.38$\pm$0.77 & 99.15$\pm$0.14 & 1.35$\pm$0.22 \\
		 & TCP \cite{Corbiere_NIPS_2019} & 83.90 & 88.38$\pm$0.23 & 12.96$\pm$0.10 & 87.63$\pm$0.15 & 60.14$\pm$0.47 & 99.73$\pm$0.02 & 0.00$\pm$0.00 \\
		 & SS & 83.90 & \textbf{80.48}$\pm$0.66 & 10.26$\pm$0.03 & \textbf{93.01}$\pm$0.10 & \textbf{73.68}$\pm$0.27 & 87.52$\pm$0.95 & \textbf{38.27}$\pm$2.48 \\
		\midrule
		\multirow{4}{*}{\textlangle ViT, RSN\textrangle} & CE & 68.72 & 93.43$\pm$0.28 & 30.90$\pm$0.35 & 69.13$\pm$0.36 & 51.24$\pm$0.63 & \textbf{99.90}$\pm$0.04 & 0.20$\pm$0.00 \\
		 & Focal \cite{Lin_ICCV_2017} & 68.72 & 93.94$\pm$0.51 & \textbf{30.97}$\pm$0.36 & 69.07$\pm$0.35 & 51.26$\pm$0.62 & 93.66$\pm$0.29 & 7.71$\pm$0.53 \\
		 & TCP \cite{Corbiere_NIPS_2019} & 68.72 & 83.55$\pm$0.70 & 23.56$\pm$0.47 & 79.04$\pm$0.91 & 66.23$\pm$1.02 & 94.25$\pm$0.96 & 0.00$\pm$0.00 \\
		 & SS & 68.72 & \textbf{77.89}$\pm$0.39 & 20.91$\pm$0.05 & \textbf{85.39}$\pm$0.16 & \textbf{74.31}$\pm$0.21 & 68.32$\pm$0.41 & \textbf{67.53}$\pm$0.62 \\
        \midrule
		\multirow{4}{*}{\textlangle RSN, ViT\textrangle} & CE & 83.90 & 93.29$\pm$0.53 & 14.74$\pm$0.17 & 85.40$\pm$0.20 & 53.43$\pm$0.28 & \textbf{100.00}$\pm$0.00 & 0.00$\pm$0.00 \\
		 & Focal \cite{Lin_ICCV_2017} & 83.90 & 94.60$\pm$0.53 & \textbf{14.98}$\pm$0.21 & 85.13$\pm$0.24 & 52.37$\pm$0.51 & \textbf{100.00}$\pm$0.00 & 0.00$\pm$0.00 \\
		 & TCP \cite{Corbiere_NIPS_2019} & 83.90 &91.93$\pm$0.49 & 14.12$\pm$0.12 & 86.12$\pm$0.15 & 55.55$\pm$0.46 & \textbf{100.00}$\pm$0.00 & 0.00$\pm$0.00 \\
         & SS & 83.90 & \textbf{88.70}$\pm$0.08 & 11.69$\pm$0.04 & \textbf{90.01}$\pm$0.10 & \textbf{64.34}$\pm$0.16 & 96.20$\pm$0.73 & \textbf{9.00}$\pm$1.32 \\
% 		RSN & ViT & SS & 83.90 & 89.86 & 11.99 & 89.49 & 62.75 & 67.74 & 48.98 \\
        \midrule
        \multirow{4}{*}{\textlangle RSN, RSN\textrangle} & CE & 68.72 & 94.84$\pm$0.27 & 29.41$\pm$0.18 & 70.79$\pm$0.19 & 52.36$\pm$0.41 & \textbf{100.00}$\pm$0.00 & 0.00$\pm$0.00 \\
		 & Focal \cite{Lin_ICCV_2017} & 68.72 & 95.16$\pm$0.19 & \textbf{29.92}$\pm$0.38 & 70.23$\pm$0.44 & 51.43$\pm$0.50 & 99.86$\pm$0.05 & 0.08$\pm$0.03 \\
		 & TCP \cite{Corbiere_NIPS_2019} & 68.72 & 88.81$\pm$0.24 & 24.46$\pm$0.12 & 77.79$\pm$0.29 & 62.73$\pm$0.14 & 99.23$\pm$0.14 & 0.00$\pm$0.00 \\
         & SS & 68.72 & \textbf{86.21}$\pm$0.44 & 22.53$\pm$0.03 & \textbf{81.88}$\pm$0.10 & \textbf{67.92}$\pm$0.11 & 79.20$\pm$2.50 & \textbf{42.09}$\pm$3.77 \\
		\bottomrule	
	\end{tabular}}
\end{table}

\noindent\textbf{Performance on Large-Scale Dataset}. 
The result on ImageNet are reported in \tabref{tbl:all_perf_w_std}. We have two key observations. Firstly, training with the cross entropy loss, focal loss, and TCP confidence loss lead to overfitting the imbalanced training samples, \ie the dominance of trustworthy predictions. Specifically, TPR is higher than 99\% while TNR is less than 1\% in all cases. Secondly, the performance of predicting trustworthiness is contingent on both the oracle and the classifier. When a high-performance model (\ie ViT) is used as the oracle and a relatively low-performance model (\ie ResNet) is used as the classifier, cross entropy loss and focal loss achieve higher TNRs than the loss functions with the other combinations. In contrast, the two losses with \textlangle ResNet, ViT\textrangle~ lead to the lowest TNRs (\ie 0\%). %, compared to the cases with the other combinations.

Despite the combinations of oracles and classifiers, the proposed steep slope loss can achieve significantly higher TNRs than using the other loss functions, while it achieves desirable performance on FPR-95\%-TPR, AUPR-Success, and AUC. This verifies that the proposed loss is effective to improve the generalizability for predicting trustworthiness. Note that the scores of AUPR-Error and TPR yielded by the proposed loss are lower than that of the other loss functions. Recall that AUPR-Error aims to inspect how easy to detect failures and depends on the negated trustworthiness confidences w.r.t. incorrect predictions \cite{Corbiere_NIPS_2019}. The AUPR-Error correlates to TPR and TNR. When TPR is close to 100\% and TNR is close to 0\%, it indicates the oracle is prone to view all the predictions to be trustworthy. In other words, almost all the trustworthiness confidences are on the right-hand side of $p(o=1|\theta,\bm{x})=0.5$. Consequently, when taking the incorrect prediction as the positive class, the negated confidences are smaller than -0.5. On the other hand, the oracle trained with the proposed loss intends to yield the ones w.r.t. incorrect predictions that are smaller than 0.5. In general, the negated confidences w.r.t. incorrect predictions are greater than the ones yielded by the other loss functions. In summary, a high TPR score and a low TNR score leads to a high AUPR-Error.

To intuitively understand the effects of all the loss functions, we plot the histograms of trustworthiness confidences w.r.t. true positive (TP), false positive (FP), true negative (TN), and false negative (FN) in \figref{fig:histogram_part}. The result confirms that the oracles trained with the baseline loss functions are prone to predict overconfident trustworthiness for incorrect predictions, while the oracles trained with the proposed loss can properly predict trustworthiness for incorrect predictions.

% On the other hand, the proposed steep slope loss show better generalizability over the three domains, where TPR is 73.62\% and TNR is 47.23\%. Secondly, the learned oracles exhibit consistent separability over the three domains through the lens of KL divergence and Bhttacharya coefficient. This is aligned with the intuition that a model that work well on a domain is likely to work well on other domains. 

\begin{figure}[!t]
	\centering
	\subfloat[\textlangle ViT, ViT\textrangle + CE]{\includegraphics[width=0.24\textwidth]{fig/hist/ce_vit_vit_val}    } \hfill
	\subfloat[\textlangle ViT, ViT\textrangle + Focal]{\includegraphics[width=0.24\textwidth]{fig/hist/focal_vit_vit_val}    } \hfill
	\subfloat[\textlangle ViT, ViT\textrangle + TCP]{\includegraphics[width=0.24\textwidth]{fig/hist/tcp_vit_vit_val}    } \hfill
	\subfloat[\textlangle ViT, ViT\textrangle +  SS]{\includegraphics[width=0.24\textwidth]{fig/hist/ss_vit_vit_val}    } \\
	\subfloat[\textlangle ViT, RSN\textrangle + CE]{\includegraphics[width=0.24\textwidth]{fig/hist/ce_vit_rsn_val}    } \hfill
	\subfloat[\textlangle ViT, RSN\textrangle + Focal]{\includegraphics[width=0.24\textwidth]{fig/hist/focal_vit_rsn_val}    } \hfill
	\subfloat[\textlangle ViT, RSN\textrangle + TCP]{\includegraphics[width=0.24\textwidth]{fig/hist/tcp_vit_rsn_val}    } \hfill
	\subfloat[\textlangle ViT, RSN\textrangle + SS]{\includegraphics[width=0.24\textwidth]{fig/hist/ss_vit_rsn_val}    } \\
% 	\subfloat[\textlangle RSN, ViT\textrangle + CE]{\includegraphics[width=0.24\textwidth]{fig/hist/ce_rsn_vit_val}    } \hfill
% 	\subfloat[\textlangle RSN, ViT\textrangle + Focal]{\includegraphics[width=0.24\textwidth]{fig/hist/focal_rsn_vit_val}    } \hfill
% 	\subfloat[\textlangle RSN, ViT\textrangle + TCP]{\includegraphics[width=0.24\textwidth]{fig/hist/tcp_rsn_vit_val}    } \hfill
% 	\subfloat[\textlangle RSN, ViT\textrangle + SS]{\includegraphics[width=0.24\textwidth]{fig/hist/ss_rsn_vit_val}    } \\
% 	\subfloat[\textlangle RSN, RSN\textrangle + CE]{\includegraphics[width=0.24\textwidth]{fig/hist/ce_rsn_rsn_val}    } \hfill
% 	\subfloat[\textlangle RSN, RSN\textrangle + Focal]{\includegraphics[width=0.24\textwidth]{fig/hist/focal_rsn_rsn_val}    } \hfill
% 	\subfloat[\textlangle RSN, RSN\textrangle + TCP]{\includegraphics[width=0.24\textwidth]{fig/hist/tcp_rsn_rsn_val}    } \hfill
% 	\subfloat[\textlangle RSN, RSN\textrangle + SS]{\includegraphics[width=0.24\textwidth]{fig/hist/ss_rsn_rsn_val}    } \\
	\caption{\label{fig:histogram_part}
    	Histograms of trustworthiness confidences w.r.t. all the loss functions on the ImageNet validation set.
    	The oracles that are used to generate the confidences are the ones used in \tabref{tbl:all_perf_w_std}. The histograms generated with \textlangle RSN, ViT\textrangle and \textlangle RSN, RSN\textrangle are provided in \appref{sec:histogram}.
    % 	appendix \ref{sec:histogram}.
    % 	the cross entropy (first column), focal loss (second column), TCP confidence loss (third column), and the proposed steep slope loss (fourth column) on the ImageNet validation set.
    % 	\REVISION{\textit{Baseline} indicates ResNet GEM.}
    	}
    \vspace{-1ex}
\end{figure}

% \begin{wrapfigure}{r}{0.5\textwidth}
\begin{table}[!t]
	\centering
	\caption{\label{tbl:perf_mnist}
	    Performance on MNIST and CIFAR-10.
	   % We use the official TCP code, but find out that there are several bugs and we couldn't reproduce the performance reported in their paper, not even close. Below are the best results by fixing a few bugs, according to the technical details in the paper.
	}
	\adjustbox{width=1\columnwidth}{
	\begin{tabular}{C{12ex} L{15ex} C{8ex} C{10ex} C{8ex} C{8ex} C{8ex} C{8ex} C{8ex}}
		\toprule
		\textbf{Dataset} & \textbf{Loss} & \textbf{Acc$\uparrow$} & \textbf{FPR-95\%-TPR$\downarrow$} & \textbf{AUPR-Error$\uparrow$} & \textbf{AUPR-Success$\uparrow$} & \textbf{AUC$\uparrow$} & \textbf{TPR$\uparrow$} & \textbf{TNR$\uparrow$} \\
		\cmidrule(lr){1-1} \cmidrule(lr){2-2} \cmidrule(lr){3-3} \cmidrule(lr){4-4} \cmidrule(lr){5-5} \cmidrule(lr){6-6} \cmidrule(lr){7-7} \cmidrule(lr){8-8} \cmidrule(lr){9-9}
		\multirow{6}{*}{MNIST} & MCP \cite{Hendrycks_ICLR_2017} & 99.10 & 5.56 & 35.05 & \textbf{99.99} & 98.63 & 99.89 & \textbf{8.89} \\
		& MCDropout \cite{Gal_ICML_2016} & 99.10 & 5.26 & 38.50 & \textbf{99.99} & 98.65 & - & - \\
		& TrustScore \cite{Jiang_NIPS_2018} & 99.10 & 10.00 & 35.88 & 99.98 & 98.20 & - & - \\
		& TCP \cite{Corbiere_NIPS_2019} & 99.10 & 3.33 & \textbf{45.89} & \textbf{99.99} & 98.82 & 99.71 & 0.00 \\
		& TCP$\dagger$ & 99.10 & 3.33 & 45.88 & \textbf{99.99} & 98.82 & 99.72 & 0.00 \\
		& SS & 99.10 & \textbf{2.22} & 40.86 & \textbf{99.99} & \textbf{98.83} & \textbf{100.00} & 0.00 \\
		\midrule
		\multirow{6}{*}{CIFAR-10} & MCP \cite{Hendrycks_ICLR_2017} & 92.19 & 47.50 & 45.36 & 99.19 & 91.53 & 99.64 & 6.66 \\
		& MCDropout \cite{Gal_ICML_2016} & 92.19 & 49.02 & 46.40 & \textbf{99.27} & 92.08 & - & - \\
		& TrustScore \cite{Jiang_NIPS_2018} & 92.19 & 55.70 & 38.10 & 98.76 & 88.47 & - & - \\
		& TCP \cite{Corbiere_NIPS_2019} & 92.19 & 44.94 & 49.94 & 99.24 & 92.12 & \textbf{99.77} & 0.00 \\
		& TCP$\dagger$ & 92.19 & 45.07 & 49.89 & 99.24 & 92.12 & 97.88 & 0.00 \\
		& SS & 92.19 & \textbf{44.69 }& \textbf{50.28}  & 99.26 & \textbf{92.22} & 98.46 & \textbf{28.04} \\
		\bottomrule	
	\end{tabular}}
\end{table}
% \end{wrapfigure}

% \begin{figure}[!t]
% 	\centering
% 	\subfloat[Official TCP  plot]{\includegraphics[width=0.45\textwidth]{fig/hist/tcphp_mnist_tefeat}    } \hfill
% 	\subfloat[Proposed with pretrained baseline ]{\includegraphics[width=0.45\textwidth]{fig/hist/steephp_mnist_tefeat}    } \\
% 	\subfloat[TCP with trained baseline]{\includegraphics[width=0.45\textwidth]{fig/hist/tcplp_mnist_tefeat}    } \hfill
% 	\subfloat[Proposed with trained baseline ]{\includegraphics[width=0.45\textwidth]{fig/hist/steeplp_mnist_tefeat}    }
% 	\caption{
%     	Reproduction and comparison.
%     	}
% \end{figure}

\noindent\textbf{Separability between Distributions of Correct Predictions and Incorrect Predictions}.
As observed in \figref{fig:histogram_part}, the confidences w.r.t. correct and incorrect predictions follow Gaussian-like distributions.
Hence, we can compute the separability between the distributions of correct and incorrect predictions from a probabilistic perspective.
% There are two common tools to achieve the goal, \ie Kullback–Leibler (KL) divergence \cite{Kullback_AMS_1951} and Bhattacharyya distance \cite{Bhattacharyya_JSTOR_1946}.
Given the distribution of correct predictions {\small $\mathcal{N}_{1}(\mu_{1}, \sigma^{2}_{1})$} and the distribution of correct predictions {\small $\mathcal{N}_{2}(\mu_{2}, \sigma^{2}_{2})$}, we use the average Kullback–Leibler (KL) divergence {\small $\bar{d}_{KL}(\mathcal{N}_{1}, \mathcal{N}_{2})$} \cite{Kullback_AMS_1951} and Bhattacharyya distance {\small $d_{B}(\mathcal{N}_{1}, \mathcal{N}_{2})$} \cite{Bhattacharyya_JSTOR_1946} to measure the separability. 
More details and the quantitative results are reported in \appref{sec:separability}. 
In short, the proposed loss leads to larger separability than the baseline loss functions. 
This implies that the proposed loss is more effective to differentiate incorrect predictions from correct predictions.

\noindent\textbf{Performance on Small-Scale Datasets}.
We also provide comparative experimental results on small-scale datasets, \ie MNIST \cite{Lecun_IEEE_1998} and CIFAR-10 \cite{Krizhevsky_TR_2009}.
\REVISION{The results are reported in \tabref{tbl:perf_mnist}.}
% The experiment details and results are reported in \appref{sec:mnist}.
The proposed loss outperforms TCP$\dagger$ on metric FPR-95\%-TPR on both MNIST and CIFAR-10, and additionally achieved good performance on metrics AUPR-Error and TNR on CIFAR-10.
This shows the proposed loss is able to adapt to relatively simple data.
\REVISION{More details can be found in \appref{sec:mnist}.}

\noindent\textbf{Generalization to Unseen Domains}.
In practice, the oracle may run into the data in the domains that are different from the ones of training samples.
Thus, it is interesting to find out how well the learned oracles generalize to the unseen domain data.
% To this end, we apply a style transfer method \cite{Geirhos_ICLR_2019} and the functional adversarial attack method \cite{Laidlaw_NeurIPS_2019} to generate the stylized ImageNet validation set and the adversarial ImageNet validation set.
Using the oracles trained with the ImageNet training set (\ie the ones used in \tabref{tbl:all_perf_w_std}), we evaluate it on the stylized ImageNet validation set \cite{Geirhos_ICLR_2019}, adversarial ImageNet validation set \cite{Laidlaw_NeurIPS_2019}, and corrupted ImageNet validation set \cite{Hendrycks_ICLR_2018}.
% and evaluated on the two variants of the validation set.
\textlangle ViT, ViT\textrangle~ is used in the experiment.

The results on the stylized ImageNet, adversarial ImageNet, and ImageNet-C are reported in \tabref{tbl:perf_vit_vit}, \REVISION{More results on ImageNet-C are reported in \tabref{tbl:perf_imagenetc}}.
As all unseen domains are different from the one of the training set, the classification accuracies are much lower than the ones in \tabref{tbl:all_perf_w_std}. 
The adversarial validation set is also more challenging than the stylized validation set \REVISION{and the corrupted validation set}.
As a result, the difficulty affects the scores across all metrics.
The oracles trained with the baseline loss functions are still prone to recognize the incorrect prediction to be trustworthy.
The proposed loss consistently improves the performance on FPR-95\%-TPR, AUPR-Sucess, AUC, and TNR.
Note that the adversarial perturbations are computed on the fly \cite{Laidlaw_NeurIPS_2019}. Instead of truncating the sensitive pixel values and saving into the images files, we follow the experimental settings in \cite{Laidlaw_NeurIPS_2019} to evaluate the oracles on the fly.
Hence, the classification accuracies w.r.t. various loss function are slightly different but are stably around 6.14\%.

% Also, we report the performances on each domain in \tabref{tbl:perf_vit_vit} and \tabref{tbl:perf_rsn_vit}.
% They shows that the cross entropy and focal loss work well on the regular validation set, but work poorly on the stylized and adversarial validation sets. This confirms the overfitting resulted from the learning with the cross entropy and focal loss.

\begin{table}[!t]
	\centering
	\vspace{-1ex}
	\caption{\label{tbl:perf_vit_vit}
	   % Histograms of trustworthiness confidences w.r.t. all the loss functions on the stylized ImageNet validation set (stylized val) and the adversarial ImageNet validation set (adversarial val). \textlangle ViT, ViT\textrangle is used in the experiment and the domains of the two validation sets are different from the one of the training set that is used for training the oracle.
	    Performance on the stylized ImageNet validation set, the adversarial ImageNet validation set, and one (Defocus blur) of validation sets in ImageNet-C. Defocus blus is at at the highest level of severity.
	    \textlangle ViT, ViT\textrangle~ is used in the experiment and the domains of the two validation sets are different from the one of the training set that is used for training the oracle. The corresponding histograms are available in \appref{sec:histogram}. More results on ImageNet-C can be found in \tabref{tbl:perf_imagenetc}.
	   % In this experiment, ViT is used for both the oracle backbone and the classifier. The oracle is trained with the CE loss, the focal loss, and the proposed steep slope loss on the ImageNet training set. The resulting oracles w.r.t. each loss are evaluated on the three validation sets. The classifier is used in the evaluation mode in the experiment.
	}
	\adjustbox{width=1\columnwidth}{
	\begin{tabular}{C{15ex} L{10ex} C{8ex} C{10ex} C{8ex} C{8ex} C{8ex} C{8ex} C{8ex}}
		\toprule
		\textbf{Dataset} & \textbf{Loss} & \textbf{Acc$\uparrow$} & \textbf{FPR-95\%-TPR$\downarrow$} & \textbf{AUPR-Error$\uparrow$} & \textbf{AUPR-Success$\uparrow$} & \textbf{AUC$\uparrow$} & \textbf{TPR$\uparrow$} & \textbf{TNR$\uparrow$} \\
		\cmidrule(lr){1-1} \cmidrule(lr){2-2} \cmidrule(lr){3-3} \cmidrule(lr){4-4} \cmidrule(lr){5-5} \cmidrule(lr){6-6} \cmidrule(lr){7-7} \cmidrule(lr){8-8} \cmidrule(lr){9-9}
% 		& \multicolumn{7}{c}{Regular validation set} \\
% 		\cmidrule(lr){1-1} \cmidrule(lr){2-8}
% 		CE & 83.90 & 92.83 & 15.08 & 84.99 & 52.78 & 100.00 & 0.01 \\
% 		Focal & 83.90 & 92.68 & 14.69 & 85.46 & 53.47 & 99.06 & 1.61 \\
% 		TCP & 83.90 & 88.07 & 12.86 & 87.80 & 60.45 & 99.72 & 1.02 \\
% % 		TCP & 83.90 & 86.45 & 12.12 & 88.95 & 63.39 & 99.07 & 3.06 \\
% 		SS & 83.90 & 80.89 & 10.31 & 92.90 & 73.31 & 88.44 & 35.64 \\
% 		\midrule
% 		& \multicolumn{7}{c}{Stylized validation set} \\
% 		\cmidrule(lr){1-1} \cmidrule(lr){2-8}
		\multirow{4}{*}{Stylized \cite{Geirhos_ICLR_2019}} & CE & 15.94 & 95.52 & 84.18 & 15.86 & 49.07 & \textbf{99.99} & 0.02 \\
		& Focal \cite{Lin_ICCV_2017} & 15.94 & 95.96 & \textbf{85.90} & 14.30 & 46.01 & 99.71 & 0.25 \\
		& TCP \cite{Corbiere_NIPS_2019} & 15.94 & 93.42 & 80.17 & 21.25 & 57.29 & 99.27 & 0.00 \\
% 		& TCP & 15.94 & 93.19 & 78.53 & 24.52 & 60.31 & 95.41 & 6.24 \\
		& SS & 15.94 & \textbf{89.38} & 75.08 & \textbf{34.39} & \textbf{67.68} & 44.42 & \textbf{81.22} \\
        \midrule
% 		& \multicolumn{7}{c}{Adversarial validation set} \\
% 		\cmidrule(lr){1-1} \cmidrule(lr){2-8}
        \multirow{4}{*}{Adversarial \cite{Laidlaw_NeurIPS_2019}} & CE & 6.14 & 94.35 & \textbf{93.70} & 6.32 & 51.28 & \textbf{99.97} & 0.06 \\
        & Focal \cite{Lin_ICCV_2017} & 6.15 & 93.67 & 93.48 & 6.56 & 52.39 & 99.06 & 1.43 \\
        & TCP \cite{Corbiere_NIPS_2019} & 6.11 & 93.94 & 92.77 & 7.55 & 55.81 & 99.71 & 0.00 \\
        & SS  & 6.16 & \textbf{90.07} & 90.09 & \textbf{13.07} & \textbf{65.36} & 87.10 & \textbf{24.33} \\ \midrule
        \multirow{4}{*}{Defocus blur \cite{Hendrycks_ICLR_2018}} & CE & 31.83 & 94.46 & \textbf{68.56} & 31.47 & 50.13 & \textbf{99.15} & 1.07 \\
		& Focal \cite{Lin_ICCV_2017} & 31.83 & 94.98  & 66.87 & 33.24 & 51.28 & 96.70 & 3.26 \\
		& TCP \cite{Corbiere_NIPS_2019} & 31.83 & 93.50 & 64.67 & 36.05 & 54.27 & 96.71 & 4.35 \\
		& SS & 31.83 & \textbf{90.18} & 57.95 & \textbf{48.80} & \textbf{64.34} & 77.79 & \textbf{37.29} \\
		\bottomrule	
	\end{tabular}}
\end{table}

\begin{figure}[!b]
	\centering
	\subfloat[]{\includegraphics[width=0.32\textwidth]{fig/risk/risk_vit_vit} \label{fig:risk_vit}} \hfill
	\subfloat[]{\includegraphics[width=0.30\textwidth]{fig/analysis/loss} \label{fig:abl_loss}} \hfill
	\subfloat[]{\includegraphics[width=0.32\textwidth]{fig/analysis/tpr_tnr} \label{fig:abl_tpr_tnr}} 
	\caption{\label{fig:anal_abl}
    	Analyses based on \textlangle ViT, ViT\textrangle. (a) are the curves of risk vs. coverage. Selective risk represents the percentage of errors in the remaining validation set for a given coverage. (b) are the curves of loss vs. $\alpha^{-}$. (c) are TPR and TNR against various $\alpha^{-}$.
    	}
\end{figure}

\noindent\textbf{Selective Risk Analysis}.
Risk-coverage curve is an important technique for analyzing trustworthiness through the lens of the rejection mechanism in the classification task \cite{Corbiere_NIPS_2019,Geifman_NIPS_2017}. 
In the context of predicting trustworthiness, selective risk is the empirical loss that takes into account the decisions, \ie to trust or not to trust the prediction. 
Correspondingly, coverage is the probability mass of the non-rejected region. As can see in \figref{fig:risk_vit}, the proposed loss leads to significantly lower risks, compared to the other loss functions.
We present the risk-coverage curves w.r.t. all the combinations of oracles and classifiers in \appref{sec:risk}.
They consistently exhibit similar pattern.

\noindent\textbf{Ablation Study}.
In contrast to the compared loss functions, the proposed loss has more hyperparameters to be determined, \ie $\alpha^{+}$ and $\alpha^{-}$.
As the proportion of correct predictions is usually larger than that of incorrect predictions, we would prioritize $\alpha^{-}$ over $\alpha^{+}$ such that the oracle is able to recognize a certain amount of incorrect predictions.
In other words, we first search for $\alpha^{-}$ by freezing $\alpha^{+}$, and then freeze $\alpha^{-}$ and search for $\alpha^{+}$.
\figref{fig:abl_loss} and \ref{fig:abl_tpr_tnr} show how the loss, TPR, and TNR vary with various $\alpha^{-}$. In this analysis, the combination \textlangle ViT, ViT\textrangle~ is used and $\alpha^{+}=1$.
We can see that $\alpha^{-}=3$ achieves the optimal trade-off between TPR and TNR.
We follow a similar search strategy to determine $\alpha^{+}=2$ and $\alpha^{-}=5$ for training the oracle with ResNet backbone.
% With the classifier ViT and the ViT based oracle, we show how the performance vary when $\alpha^{+}$ and $\alpha^{-}$ change.  

\noindent\textbf{Effects of Using $z=\bm{w}^{\top}\bm{x}^{out}+b$}.
Using the signed distance as $z$, \ie $z = \frac{\bm{w}^{\top} \bm{x}^{out}+b}{\|\bm{w}\|}$, has a geometric interpretation as shown in \figref{fig:workflow_a}.
However, the main-stream models \cite{He_CVPR_2016,Tan_ICML_2019,Dosovitskiy_ICLR_2021} use $z=\bm{w}^{\top}\bm{x}^{out}+b$. 
Therefore, we provide the corresponding results in appendix \ref{sec:appd_z}, which are generated by the proposed loss taking the output of the linear function as input.
In comparison with the results of using $z = \frac{\bm{w}^{\top} \bm{x}^{out}+b}{\|\bm{w}\|}$, using $z=\bm{w}^{\top}\bm{x}^{out}+b$ yields comparable scores of FPR-95\%-TPR, AUPR-Error, AUPR-Success, and AUC.
Also, TPR and TNR are moderately different between $z = \frac{\bm{w}^{\top} \bm{x}^{out}+b}{\|\bm{w}\|}$ and $z=\bm{w}^{\top}\bm{x}^{out}+b$, when $\alpha^{+}$ and $\alpha^{-}$ are fixed.
This implies that TPR and TNR are sensitive to $\|\bm{w}\|$. 
% \REVISION{We discuss the reason in \appref{sec:effect_normalization}.}
% 
\REVISION{
This is because the normalization by $\|w\|$ would make $z$ more dispersed in value than the variant without normalization. 
In other words, the normalization leads to long-tailed distributions while no normalization leads to short-tailed distributions. 
Given the same threshold, TNR (TPR) is determined by the location of the distribution of negative (positive) examples and the extent of short/long tails. 
Our analysis on the histograms generated with and without $\|w\|$ normalization verifies this point.
}

% \noindent\textbf{Learning with Class Weights}. We witness the imbalancing characteristics in the learning task for predicting trustworthiness. Table xx shows that one of most common learning techniques with imbalanced data, \ie using class weights, is not effective. The reason is that applying class weights to the loss function, \eg cross entropy, it only scale up the graph along y-axis. However, the long tail regions still slow down the move of the features w.r.t. false positive or false negative towards the well-classified regions.

% \noindent\textbf{Separability between Distributions of Correct Predictions and Incorrect Predictions}.
% As observed in \figref{fig:histogram_part}, the confidences w.r.t. correct and incorrect predictions follow Gaussian-like distributions.
% Hence, we can compute the separability between the distributions of correct and incorrect predictions from a probabilistic perspective.
% % There are two common tools to achieve the goal, \ie Kullback–Leibler (KL) divergence \cite{Kullback_AMS_1951} and Bhattacharyya distance \cite{Bhattacharyya_JSTOR_1946}.
% Given the distribution of correct predictions $\mathcal{N}_{1}(\mu_{1}, \sigma^{2}_{1})$ and the distribution of correct predictions $\mathcal{N}_{2}(\mu_{2}, \sigma^{2}_{2})$, we use the average Kullback–Leibler (KL) divergence $\bar{d}_{KL}(\mathcal{N}_{1}, \mathcal{N}_{2})$ \cite{Kullback_AMS_1951} and Bhattacharyya distance $d_{B}(\mathcal{N}_{1}, \mathcal{N}_{2})$ \cite{Bhattacharyya_JSTOR_1946} to measure the separability. More details and the quantitative results are reported in \appref{sec:separability}. In short, the proposed loss leads to larger separability than the baseline loss functions. This implies that the proposed loss is more effective to differentiate incorrect predictions from correct predictions.

\noindent\textbf{Steep Slope Loss vs. Class-Balanced Loss}.
We compare the proposed loss to the class-balanced loss \cite{Cui_CVPR_2019}, which is based on a re-weighting strategy.
The results are reported in \appref{sec:cbloss}.
Overall, the proposed loss outperforms the class-balanced loss, which implies that the imbalance characteristics of predicting trustworthiness is different from that of imbalanced data classification.

% KL divergence is used to measure the difference between two distributions \cite{Cantu_Springer_2004,Luo_TNNLS_2020}, while Bhattacharyya distance is used to measure the similarity of two probability distributions. Given two Gaussian distributions $\mathcal{N}_{1}(\mu_{1}, \sigma^{2}_{1})$ and $\mathcal{N}_{2}(\mu_{2}, \sigma^{2}_{2})$, we use the averaged KL divergence, \ie $\bar{d}_{KL}(\mathcal{N}_{1}, \mathcal{N}_{2}) = (d_{KL}(\mathcal{N}_{1}, \mathcal{N}_{2}) + d_{KL}(\mathcal{N}_{2}, \mathcal{N}_{1}))/2$, where $d_{KL}(\mathcal{N}_{1}, \mathcal{N}_{2})=\log\frac{\sigma_{2}}{\sigma_{1}}+\frac{\sigma_{1}^{2}+(\mu_{1}-\mu_{2})^{2}}{2\sigma_{2}^{2}}-\frac{1}{2}$ is not symmetrical. On the other hand, Bhattacharyya distance is defined as $d_{B}(\mathcal{N}_{1}, \mathcal{N}_{2})=\frac{1}{4}\ln \left( \frac{1}{4} \left( \frac{\sigma^{2}_{1}}{\sigma^{2}_{2}}+\frac{\sigma^{2}_{2}}{\sigma^{2}_{1}}+2 \right) \right) + \frac{1}{4} \left( \frac{(\mu_{1}-\mu_{2})^{2}}{\sigma^{2}_{1}+\sigma^{2}_{2}} \right)$. A larger $\bar{d}_{KL}$ or $d_{B}$ indicates that the two distributions are further away from each other.


% We hypothesize that $x$ w.r.t. positive and negative samples both follow Gaussian distributions. The discriminativeness of features is an important characteristic that correlates to the performance, \eg accuracy. We are interested in measures of separability of feature distributions, which reflect the discriminativeness from a probabilistic perspective. There are two common tools to achieve the goal, \ie Kullback–Leibler (KL) divergence \cite{Kullback_AMS_1951} and Bhattacharyya distance \cite{Bhattacharyya_JSTOR_1946}. Usually, KL divergence is used to measure the difference between two distributions \cite{Cantu_Springer_2004,Luo_TNNLS_2020}, while Bhattacharyya distance is used to measure the similarity of two probability distributions. Given two Gaussian distributions $\mathcal{N}_{1}(\mu_{1}, \sigma^{2}_{1})$ and $\mathcal{N}_{2}(\mu_{2}, \sigma^{2}_{2})$, we use an averaged KL divergence as in this work, \ie $\bar{d}_{KL}(\mathcal{N}_{1}, \mathcal{N}_{2}) = (d_{KL}(\mathcal{N}_{1}, \mathcal{N}_{2}) + d_{KL}(\mathcal{N}_{2}, \mathcal{N}_{1}))/2$, where $d_{KL}(\mathcal{N}_{1}, \mathcal{N}_{2})$ is the KL divergence between $\mathcal{N}_{1}$ and $\mathcal{N}_{2}$ (not symmetrical). On the other hand, Bhattacharyya distance is defined as $d_{B}(\mathcal{N}_{1}, \mathcal{N}_{2})=\frac{1}{4}\ln \left( \frac{1}{4} \left( \frac{\sigma^{2}_{1}}{\sigma^{2}_{2}}+\frac{\sigma^{2}_{2}}{\sigma^{2}_{1}}+2 \right) \right) + \frac{1}{4} \left( \frac{(\mu_{1}-\mu_{2})^{2}}{\sigma^{2}_{1}+\sigma^{2}_{2}} \right)$. In this work, we use Bhattacharyya coefficient that measures the amount of overlap between two distributions, instead of Bhattacharyya distance. Bhattacharyya coefficient is defined as $c_{B}(\mathcal{N}_{1}, \mathcal{N}_{2}) = \exp(-d_{B}(\mathcal{N}_{1}, \mathcal{N}_{2}))$. $c_{B} \in [0,1]$, where 1 indicates a full overlap and 0 indicates no overlap.

% \noindent\textbf{Semantics Difference between Predicting Trustworthiness and Classification}. As we use ViT for both the oracle and classifier, it is interesting to find out what features are leaned for predicting trustworthiness, in comparison to the features learned for classification. Hence, we compute the $l_{1}$ and $l_{2}$ distances between the features generated by the learned oracle and the features generated by the pre-trained classifier. The features are the inputs to the last layer of ViT, \ie 768-dimensional vectors.

% The mean and standard deviation of distances over all the samples in the training and validation sets are provided in \tabref{tbl:anal_diff}. Note that a smaller distance indicates higher similarity between two features. Overall, the mean of distances w.r.t. the three loss functions are large, but the focal loss yields the smallest averaged distance, which implies that the oracle learned with the focal yields the most similar features as the ones generated by the pre-trained classifier. One of possible reasons is that the focal loss prohibits the oracle training.

% Comparison of classifier backbone and oracle backbone

% Per class accuracy, precision, recall, F1

% \noindent\textbf{Taking Features as Input}
% \figref{fig:anal_featinput} shows the distributions of discriminative features generated by a multi-layer perceptron (MLP),, which plays as an oracle. The MLP takes the features generated by the classifier, instead of images, as input. The MLP-based oracle is training on the training set and is evaluated on the validation set. The figure shows that the oracle barely distinguish between positives and negatives. Because all the features are on the right-hand side of the decision boundary $x=0$.

% focal loss vs proposed

% \begin{figure}[!t]
	\centering
	\subfloat{\includegraphics[width=0.32\textwidth]{fig/analysis/anal_featinput_ce}    } \hfill
	\subfloat{\includegraphics[width=0.32\textwidth]{fig/analysis/anal_featinput_focal}    } \hfill
	\subfloat{\includegraphics[width=0.32\textwidth]{fig/analysis/anal_featinput_ss}    } \\
	\caption{\label{fig:anal_featinput}
    	Analysis of taking the features of the classifier as input to the oracle on the ImageNet validation set. In this experiment, ViT is used for both the oracle backbone and the classifier. The features are 768-dimensional vectors. The classifier is used in the evaluation mode in the experiment.
    % 	\REVISION{\textit{Baseline} indicates ResNet GEM.}
    	}
\end{figure}

% \begin{table}[!t]
	\centering
	\caption{\label{tbl:anal_diff}
	    Analysis of the difference of the output features between the classifier backbone and the oracle backbone in terms of $l_{1}$ and $l_{2}$ distances. The common backbone is ViT. The oracle backbone is trained for predicting trustworthiness, while the classifier backbone is pre-trained for classification.
	}
	\adjustbox{width=1\columnwidth}{
	\begin{tabular}{L{7ex} C{14ex} C{14ex} C{14ex} C{14ex}}
		\toprule
		& \multicolumn{2}{c}{Training} & \multicolumn{2}{c}{Validation} \\
		\cmidrule(lr){2-3} \cmidrule(lr){4-5}
		Loss & $l_{1}$ & $l_{2}$ & $l_{1}$ & $l_{2}$ \\
		\cmidrule(lr){1-1} \cmidrule(lr){2-2} \cmidrule(lr){3-3} \cmidrule(lr){4-4} \cmidrule(lr){5-5}
		CE & 74.0674$\pm$23.9773 & 3.4074$\pm$1.0967 & 78.4107$\pm$24.9338 & 3.6051$\pm$1.1402 \\
        Focal & 29.0901$\pm$8.5641 & 1.3527$\pm$0.3933 & 30.6497$\pm$8.9262 & 1.4240$\pm$0.4100 \\
        SS & 70.1997$\pm$32.8220 & 3.2129$\pm$1.4973 & 77.3162$\pm$33.4536 & 3.5378$\pm$1.5271 \\
		\bottomrule	
	\end{tabular}}
\end{table}

% \noindent\textbf{Ablation Study}. With the classifier ViT and the ViT based oracle, we show how the performance vary when $\alpha^{+}$ and $\alpha^{-}$ change.  

% \noindent\textbf{Generalization to Unseen Classifier}.
% As the oracle is trained by observing what a classifier predicts the label for an image, the knowledge learned in this way highly correlates to the behaviours of the classifier. It is interesting to know how the knowledge learned by the oracle generalizes to other unseen classifiers. To this end, we use the ViT based oracle that is trained with a ViT classifier to predict the trustworthiness of a ResNet-50 on the adversarial validation set, which is the most challenging in the three sets. 
% For the proposed loss, we use $\alpha^{+}=1$ and $\alpha^{-}=3$ for the oracle that is based on ViT's backbone, while we use $\alpha^{+}=2$ and $\alpha^{-}=5$ for the oracle that is based on ResNet's backbone.


%
\section{Discussion}
\label{disc}
\section{Summary and discussion}\label{s:disc}

In this work we introduced an attack that allows an adversary to violate the 
  guarantees of MPM systems by leveraging users' friends.
We also proposed several mitigations, but our proposals satisfy only two out 
  of three desirable properties: privacy (leak no information), low 
  communication overhead (i.e., clients need not send many messages per round), 
  and low latency (friends get to talk to each other often).
The most pragmatic of our solutions requires bounding the maximum number of 
  friends that a client can have.

Even with our mitigations, compromised friends are a liability and can be
  used to learn sensitive information through other means.
For example, if a user is uncharacteristically slow to respond to a 
  compromised friend's message (a user's response pattern could be constructed over
  many prior interactions), this anomaly in itself leaks information.
We believe that understanding the impact of this type of attack in
  practice is a promising avenue for future work.


% Acknowledgements should go at the end, before appendices and references

\acks{We would like to acknowledge support for this project
from the National Science Foundation (NSF grant IIS-9988642)
and the Multidisciplinary Research Program of the Department
of Defense (MURI N00014-00-1-0637). }

% Manual newpage inserted to improve layout of sample file - not
% needed in general before appendices/bibliography.

\vskip 0.2in
\newpage
\bibliography{jmlr}

\newpage

\appendix
\section*{Appendix A.}
\label{app:theorem}
\appendix

\section{Experimental details and more results}
\label{sec:app_exp}
We run all the experiments on Nvidia RTX 2080 Ti GPUs and V100 GPUs. Table~\ref{tab:app_testbed} summarizes the set of images used in each figure or table in the main paper.  

\captionsetup[table]{font=small}
\begin{table}[H]
    \small
    \centering
    \begin{tabular}{|p{2.5cm}|p{10cm}|}
    \toprule
         {\bf Figure/Table} & {\bf Comments}	\\
    \midrule
        Figure~\ref{fig:BN_var}a & We’ve tuned hyperparams for the attack (see Appendix~\ref{sec:app_hyperparam}) and carried out evaluations on the whole CIFAR-subset. The first sampled batch of size 16 from CIFAR-subset was used in Figure~\ref{fig:BN_var}a to demonstrate the quality of recovery for low-resolution images when BatchNorm statistics are not assumed to be known.  \\
        \midrule
        Figure~\ref{fig:BN_var}b & We’ve tuned hyperparams for the attack (see Appendix~\ref{sec:app_hyperparam}) and carried out evaluations on the whole ImageNet-subset. The best-reconstructed image in ImageNet-subset was used in Figure 1b to demonstrate the quality of recovery for high-resolution images when BatchNorm statistics are not assumed to be known.\\
        \midrule
        Figure~\ref{fig:batch_label_dist} & Percentages of class labels per batch were evaluated over the entire CIFAR10 dataset, for a random seed.	\\
        \midrule
        Figure~\ref{fig:reconstructed_labels} & The first sampled batch of size 16 was used in Figure~\ref{fig:reconstructed_labels} to demonstrate the quality of recovery when labels are not assumed to be known.	\\
        \midrule
        Table~\ref{tab:exp_summary} and Figure~\ref{fig:vis_recon} & We’ve tuned hyperparams for the attack and carried out evaluations on the whole CIFAR-subset. Table~\ref{tab:exp_summary} summarizes the performance of the attack on the whole CIFAR-subset and  Figure~\ref{fig:vis_recon} shows example images.\\
    \bottomrule
    \end{tabular}
    \caption{Summary of experimental testbed for each evaluation.}
    \label{tab:app_testbed}
\end{table}


\subsection{Hyper-parameters}
\label{sec:app_hyperparam}



\paragraph{Training.} For all experiments, we train ResNet-18 for 200 epochs, with a batch size of 128. We use SGD with momentum 0.9 as the optimizer. The initial learning rate is set to 0.1 by default, except for gradient pruning with $p=0.99$ and $p=0.999$. where we set the initial learning rate to 0.02. We decay the learning rate by a factor of 0.1 every 50 epochs.

\paragraph{The attack.}  We report the performance under different $\alpha_{\rm TV}$'s (Figure~\ref{fig:BN_tv_tune}) and $\alpha_{\rm BN}$'s (Figure~\ref{fig:BN_reg_tune}).

\begin{figure}[H]
\captionsetup[subfigure]{labelfont=scriptsize, textfont=tiny}
    \centering
    \subfloat[Original]{\includegraphics[width=0.12\textwidth]{imgs/appendix/TV/original.png}}
    \subfloat[$\alpha_{\rm TV}$=0]{\includegraphics[width=0.12\textwidth]{imgs/appendix/TV/tv_0.png}}
    \subfloat[$\alpha_{\rm TV}$=1e-3]{\includegraphics[width=0.12\textwidth]{imgs/appendix/TV/tv_1e-3.png}}
    \subfloat[$\alpha_{\rm TV}$=5e-3]{\includegraphics[width=0.12\textwidth]{imgs/appendix/TV/tv_5e-3.png}}
    \subfloat[$\alpha_{\rm TV}$=1e-2]{\includegraphics[width=0.12\textwidth]{imgs/appendix/TV/tv_1e-2.png}}
    \subfloat[$\alpha_{\rm TV}$=5e-2]{\includegraphics[width=0.12\textwidth]{imgs/appendix/TV/tv_5e-2.png}}
    \subfloat[$\alpha_{\rm TV}$=1e-1]{\includegraphics[width=0.12\textwidth]{imgs/appendix/TV/tv_1e-1.png}}
    \subfloat[$\alpha_{\rm TV}$=5e-1]{\includegraphics[width=0.12\textwidth]{imgs/appendix/TV/tv_5e-1.png}}
    
    \caption{Attacking a single CIFAR-10 images in $\rm BN_{exact}$ setting, with different coefficients for the total variation regularizer ($\alpha_{\rm TV}$'s). $\alpha_{\rm TV}$=1e-2 gives the best reconstruction.}
    \label{fig:BN_tv_tune}
\end{figure}


\begin{figure}[H]
\vspace{-5mm}
\captionsetup[subfigure]{labelfont=scriptsize, textfont=tiny}
    \centering
    \subfloat[Original]{\includegraphics[width=0.16\textwidth]{imgs/assumptions/BN/original.png}}
    \subfloat[$\alpha_{\rm BN}$=0]{\includegraphics[width=0.16\textwidth]{imgs/assumptions/BN/reconstructed_train_train_bn=0.png}}
    \subfloat[$\alpha_{\rm BN}$=5e-4]{\includegraphics[width=0.16\textwidth]{imgs/assumptions/BN/reconstructed_train_train_bn=5e-4.png}}
    \subfloat[$\alpha_{\rm BN}$=1e-3]{\includegraphics[width=0.16\textwidth]{imgs/assumptions/BN/reconstructed_train_train_bn=1e-3.png}}
    \subfloat[$\alpha_{\rm BN}$=5e-3]{\includegraphics[width=0.16\textwidth]{imgs/assumptions/BN/reconstructed_train_train_bn=5e-3.png}}
    \subfloat[$\alpha_{\rm BN}$=1e-2]{\includegraphics[width=0.16\textwidth ]{imgs/assumptions/BN/reconstructed_train_train_bn=1e-2.png}}
    \caption{Attacking a batch of 16 CIFAR-10 images in $\rm BN_{infer}$ setting, with different coefficients for the BatchNorm regularizer ($\alpha_{\rm BN}$'s). $\alpha_{\rm TV}$=1e-3 gives the best reconstruction.}
    \label{fig:BN_reg_tune}
\end{figure}


\subsection{Details and more results for Section~\ref{sec:assumption}}

\paragraph{Attacking a single ImageNet image.} We launched the attack on ImageNet using the objective function in Eq.~\ref{eq:objective}, where $\alpha_{\rm TV}=0.1$, $\alpha_{\rm BN}=0.001$. We run the attack for 24,000 iterations using Adam optimizer, with initial learning rate 0.1, and decay the learning rate by a factor of $0.1$ at 
$3/8,5/8,7/8$ of training. We rerun the attack 5 times and present the best results measured by LPIPS in Figure~\ref{fig:BN_var}.

\paragraph{Qualitative and quantitative results for a more realistic attack.} We also present results of a more realistic attack in Table~\ref{tab:exp_summary_realistic} and Figure~\ref{fig:vis_recon_realistic}, where the attacker does {\em not} know BatchNorm statistics but knows the private labels. We assume the private labels to be known in this evaluation, because for those batches whose distribution of labels is uniform, the restoration of labels should still be quite accurate~\citep{yin2021see}.
As shown, in the evaluated setting, the attack is no longer effective when the batch size is 32 and Intra-InstaHide with $k=4$ is applied. The accuracy loss to stop the realistic attack is only around $3\%$ (compared to around $7\%$ to stop the strongest attack) .


\begin{figure}[H]
\captionsetup[subfigure]{font=small}
  \centering
  \subfloat{\includegraphics[width=\textwidth]{imgs/Compare_16_32.png}}
  \caption{Reconstruction results under different defenses for a more realistic setting (when the attacker knows private labels but does not know BatchNorm statistics). We also present the results for the strongest attack from Figure~\ref{fig:vis_recon} for comparison. Using Intra-InstaHide with $k=4$ and batch size 32 seems to stop the realistic attack.}
  \label{fig:vis_recon_realistic}
\end{figure}

\captionsetup[table]{font=small}
\begin{table}[H] 
  \scriptsize
  \setlength{\tabcolsep}{2.6pt}
  \renewcommand{\arraystretch}{0.95}
  \begin{tabular}{|l|c|c|c|c|c|c|c|c|c|c|c|c|c|c|c|c|}
  \toprule
   &  \multirow{2}{*}{\bf None} & \multicolumn{6}{c|}{\multirow{2}{*}{\bf GradPrune ($p$)}} & \multicolumn{2}{c|}{\multirow{2}{*}{\bf MixUp ($k$)}} & \multicolumn{2}{c|}{\multirow{2}{*}{\bf Intra-InstaHide ($k$)}} & \multicolumn{2}{c|}{\bf GradPrune ($p=0.9$)}\\
   & & \multicolumn{6}{c|}{} & \multicolumn{2}{c|}{} & \multicolumn{2}{c|}{} & {\bf  + MixUp } & {\bf  + Intra-InstaHide}\\
  \midrule
   {\bf Parameter}  & - & 0.5 & 0.7 & 0.9 & 0.95 & 0.99 & 0.999 & 4 & 6 & 4 & 6 & $k=4$ & $k=4$ \\
   \midrule
   {\bf Test Acc.} & 93.37 & 93.19 & 93.01 & 90.57 & 89.92 & 88.61 & 83.58 &  92.31 & 90.41 & 90.04 & 88.20 & 91.37 & 86.10 \\
   \midrule
  {\bf Time (train)} & $1\times$ & \multicolumn{6}{c|}{$1.04\times$} & \multicolumn{2}{c|}{$1.06\times$} & \multicolumn{2}{c|}{$1.06\times$} & \multicolumn{2}{c|}{$1.10\times$} \\
  \midrule
  \multicolumn{14}{|c|}{\bf Attack batch size $= 16$, the strongest attack} \\
  \midrule
  {\bf Avg. LPIPS $\downarrow$}  & 0.41  & 0.41  & 0.42  & 0.46  & 0.48  & 0.50  & 0.55         & 0.50  & 0.49  & 0.69  & 0.69  & 0.62  & \best{0.73}\\
  {\bf Best LPIPS $\downarrow$}  & 0.21  & 0.22  & 0.27  & 0.29  & 0.30  & 0.29  & 0.48         & 0.31  & 0.28  & 0.56  & 0.56  & 0.37  & \best{0.65}\\
  {(LPIPS std.)}                 & 0.09  & 0.08  & 0.07  & 0.06  & 0.06  & 0.06  & 0.04         & 0.10  & 0.10  & 0.06  & 0.07  & 0.10  & 0.05\\
  \midrule
   \multicolumn{14}{|c|}{\bf Attack batch size $= 16$, attacker knows private labels but does not know BatchNorm statistics} \\
   \midrule
   {\bf Avg. LPIPS $\downarrow$}  & 0.49 & 0.51 & 0.48 & 0.51 & 0.52 & 0.56 & 0.60 & 0.71 & 0.71 & \best{0.75} & \best{0.75} & 0.74 &  0.74\\
   {\bf Best LPIPS $\downarrow$}  & 0.30 & 0.33 & 0.31 & 0.33 & 0.34 & 0.39 & 0.44 & 0.48 & 0.53 & \best{0.65} & 0.63 & 0.61 &  0.63\\
   {(LPIPS std.)}                 & 0.08 & 0.09 & 0.08 & 0.08 & 0.07 & 0.07 & 0.05 & 0.08 & 0.07 & 0.04 & 0.05 & 0.08 &  0.05\\
   \midrule
   \multicolumn{14}{|c|}{\bf Attack batch size $= 32$, the strongest attack} \\
  \midrule
  {\bf Avg. LPIPS $\downarrow$}  & 0.45  & 0.46  & 0.48  & 0.52  & 0.54  & 0.58  & 0.63         & 0.50  & 0.49  & 0.69  & 0.69  & 0.62  & \best{0.73}\\
   {\bf Best LPIPS $\downarrow$}  & 0.18  & 0.18  & 0.22  & 0.31  & 0.43  & 0.48  & 0.54         & 0.31  & 0.28  & 0.56  & 0.56  & 0.37  & \best{0.65}\\
   {(LPIPS std.)}                 & 0.11  & 0.11  & 0.09  & 0.07  & 0.05  & 0.04  & 0.04         & 0.10  & 0.10  & 0.06  & 0.07  & 0.10  & 0.05\\
    \midrule
   \multicolumn{14}{|c|}{\bf Attack batch size $= 32$, attacker knows private labels but does not know BatchNorm statistics} \\
   \midrule
   {\bf Avg. LPIPS $\downarrow$}  & 0.48 & 0.50 & 0.53 & 0.53 & 0.55 & 0.60 & 0.63 & 0.73 & 0.72 & 0.76 & 0.76 & 0.76 & \best{0.77} \\
   {\bf Best LPIPS $\downarrow$}  & 0.29 & 0.32 & 0.32 & 0.31 & 0.40 & 0.41 & 0.55 & 0.63 & 0.60 & \best{0.68} & 0.63 & 0.66 & 0.65\\
   {(LPIPS std.)}                 & 0.08 & 0.07 & 0.07 & 0.08 & 0.08 & 0.06 & 0.04 & 0.06 & 0.06 & 0.04 & 0.05 & 0.06 & 0.05\\
  \bottomrule
  \end{tabular}
  \vspace{2mm}
%   \subfloat{\includegraphics[width=0.98\textwidth]{imgs/Compare_16_32.png}}
  \caption{\small Utility-security trade-off of different defenses for a more realistic setting (when the attacker knows private labels but does not know BatchNorm statistics). We also present the results for the strongest attack from Table~\ref{tab:exp_summary} for comparison. We evaluate the attack on 50 CIFAR-10 images and report the LPIPS score ($\downarrow$: lower values suggest more privacy leakage).
  We mark the least-leakage defense measured by the metric in \best{green}.} 
  \label{tab:exp_summary_realistic}
\end{table}

\iffalse
\paragraph{Qualitative and quantitative results for private labels unknown.} Apart from the example in Figure~\ref{fig:reconstructed_labels} with batch size being 16, we provide another example for how unknown labels affect reconstruction quality in Figure~\ref{fig:assumption2_app}, with batch size being 32. We also provide quantitative measurements in Figure~\ref{tab:assumption2_app1} and~\ref{tab:assumption2_app2}.

\begin{figure}[H]
    \centering
    \subfloat[Reconstructions with and without private labels]{
    \includegraphics[width=0.95\textwidth]{imgs/assumptions/label_known_unknown_32.png}
    \label{fig:assumption2_app}
    }\\
    \subfloat[Batch size = 16]{
        \setlength{\tabcolsep}{4pt}
        \small
        \begin{tabular}[b]{|c|c|c|}
                \toprule
                  & {\bf Labels known} &  {\bf Labela unknown} \\
                \midrule
                %  {\bf Avg. PSNR $\uparrow$} & 12.45 & 12.01  \\
                %  {\bf Best PSNR $\uparrow$} & 17.42 & 14.85    \\
                 {\bf Avg. LPIPS $\downarrow$} & 0.44 & 0.58 \\
                 {\bf Best LPIPS $\downarrow$} & 0.25 & 0.32    \\
                \bottomrule
            \end{tabular}
        \label{tab:assumption2_app1}
        }
        \subfloat[Batch size = 32]{
        \setlength{\tabcolsep}{4pt}
        \small
        \begin{tabular}[b]{|c|c|c|}
                \toprule
                  & {\bf Labels known} &  {\bf Labels unknown} \\
                \midrule
                %  {\bf Avg. PSNR $\uparrow$} & 13.01 & 12.16 \\
                %  {\bf Best PSNR $\uparrow$} & 17.09 & 14.62    \\
                 {\bf Avg. LPIPS $\downarrow$} & 0.41 & 0.62 \\
                 {\bf Best LPIPS $\downarrow$} & 0.21 & 0.39    \\
                \bottomrule
            \end{tabular}
        \label{tab:assumption2_app2}
        }
    \caption{A reconstructed batch of 32 images with and without private labels known (a). We also provide quantitative measurements of reconstructions with batch size 16 (b) and 32 (c) ($\downarrow$: lower values suggest more leakage). The gradient inversion attack is weakened when private labels are not available.}
\end{figure}
\fi


% \iffalse
\subsection{More results for the strongest attack}

\paragraph{Full version of Figure~\ref{fig:vis_recon}.} Figure~\ref{fig:vis_recon_full} provides more examples for reconstructed images by the strongest attack under different defenses and batch sizes. 



\begin{figure}[H]
\captionsetup[subfigure]{font=small}
  \centering
  \vspace{-12mm}
  \subfloat[Batch size $=1$]{\includegraphics[width=\linewidth]{imgs/recon_vis_bs=1_BN_exact_small.png}}\\
  \vspace{-3mm}
  \subfloat[Batch size $=16$]{\includegraphics[width=\linewidth]{imgs/recon_vis_bs=16_BN_exact_small.png}}\\
  \vspace{-3mm}
  \subfloat[Batch size $=32$]{\includegraphics[width=\linewidth]{imgs/recon_vis_bs=32_BN_exact_small.png}}\\
  \vspace{-2mm}
  \caption{Reconstruction results under different defenses with batch size 1 (a), 16 (b) and 32 (c). Full version of Figure~\ref{fig:vis_recon}.}
  \label{fig:vis_recon_full}
  \vspace{-2mm}
\end{figure}


\paragraph{Results with MNIST dataset.} We’ve repeated our main evaluation of defenses and attacks (Table~\ref{tab:exp_summary}) on MNIST dataset~\citep{deng2012mnist} with a simple 6-layer ConvNet model. Note that the simple ConvNet does not contain BatchNorm layers. We evaluate the following defenses on the MNIST dataset with a 6-layer ConvNet architecture against the strongest attack (private labels known):

\begin{itemize}
    \item GradPrune (gradient pruning): gradient pruning sets gradients of small magnitudes to zero. We vary the pruning ratio $p$ in \{0.5, 0.7, 0.9, 0.95, 0.99, 0.999, 0.9999\}.
    \item MixUp: we vary $k$ in \{4,6\}, and set the upper bound of a single coefficient to 0.65 (coefficients sum to 1).
    \item Intra-InstaHide: we vary $k$ in \{4,6\}, and set the upper bound of a single coefficient to 0.65 (coefficients sum to 1). 
    \item A combination of GradPrune and MixUp/Intra-InstaHide.
\end{itemize}

We run the evaluation against the strongest attack and batch size 1 to estimate the upper bound of privacy leakage. Specifically, we assume the attacker knows private labels, as well as the indices of mixed images and mixing coefficients for MixUp and Intra-InstaHide. 

\begin{figure}[t]
    \centering
    \includegraphics[width=0.95\linewidth]{imgs/appendix/recon_vis_MNIST.png}
    \caption{Reconstruction results of MNIST digits under different defenses with the strongest atttack and batch size 1.}
    \label{fig:vis_recon_MNIST}
    \vspace{-5mm}
\end{figure}

For MNIST with a simple 6-layer ConvNet, defending the strongest attack with gradient pruning may require the pruning ratio $p\geq 0.9999$. MixUp with $k=4$ or $k=6$ are not sufficient to defend the gradient inversion attack. Combining MixUp ($k=4$) with gradient pruning ($p=0.99$) improves the defense, however, the reconstructed digits are still highly recognizable. Intra-InstaHide alone ($k=4$ or $k=6$) gives a bit better defending performance than MixUp and GradPrune. Combining InstaHide ($k=4$) with gradient pruning ($p=0.99$) further improves the defense and makes the reconstruction almost unrecognizable. 




\subsection{More results for encoding-based defenses}
We visualize the whole reconstructed dataset under MixUp and Intra-InstaHide defenses with different batch sizes in Figure~\ref{fig:encode_bs1}, \ref{fig:encode_bs16} and \ref{fig:encode_bs32}.  Sample results of the original and the reconstructed batches are provided in Figure~\ref{fig:mixup_vs_instahide}.

\begin{figure}[H]
    \centering
    \includegraphics[width=0.95\textwidth]{imgs/appendix/mixup_vs_instahide.png}
    \caption{Original and reconstructed batches of 16 images under MixUp and Intra-InstaHide defenses. We visualize both the original and the absolute images for the Intra-InstaHide defense. Intra-InstaHide makes pixel-wise matching harder.}
    \label{fig:mixup_vs_instahide}
    \vspace{-5mm}
\end{figure}

\begin{figure}[H]
\captionsetup[subfigure]{labelfont=scriptsize, textfont=tiny}
    \centering
    \subfloat[Original]{\includegraphics[width=0.23\textwidth]{imgs/decode_res/InstaHide/bs1_k4/originals.png}} \hspace{1mm}
    \subfloat[MixUp, $k$=4]{\includegraphics[width=0.23\textwidth]{imgs/decode_res/Mixup/bs1_k4/grad_decode.png}} \hspace{1mm}
    \subfloat[MixUp, $k$=6]{\includegraphics[width=0.23\textwidth]{imgs/decode_res/Mixup/bs1_k6/grad_decode.png}} \hspace{1mm}
    \subfloat[MixUp+GradPrune, $k$=4, $p$=0.9]{\includegraphics[width=0.23\textwidth]{imgs/decode_res/Mixup/bs1_k4_gradprune/grad_decode.png}}
    
    \subfloat[Original]{\includegraphics[width=0.23\textwidth]{imgs/decode_res/InstaHide/bs1_k4/originals.png}} \hspace{1mm}
    \subfloat[InstaHide, $k$=4]{\includegraphics[width=0.23\textwidth]{imgs/decode_res/InstaHide/bs1_k4/grad_decode.png}} \hspace{1mm}
    \subfloat[InstaHide, $k$=6]{\includegraphics[width=0.23\textwidth]{imgs/decode_res/InstaHide/bs1_k6/grad_decode.png}} \hspace{1mm}
    \subfloat[InstaHide+GradPrune, $k$=4, $p$=0.9]{\includegraphics[width=0.23\textwidth]{imgs/decode_res/InstaHide/bs1_k4_gradprune/grad_decode.png}}
    \caption{Reconstrcuted dataset under MixUp and Intra-InstaHide against the strongest attack (batch size is 1).}
    \label{fig:encode_bs1}
    \vspace{-10mm}
\end{figure}


\begin{figure}[H]
\captionsetup[subfigure]{labelfont=scriptsize, textfont=tiny}
    \centering
    \subfloat[Original]{\includegraphics[width=0.23\textwidth]{imgs/decode_res/InstaHide/bs1_k4/originals.png}} \hspace{1mm}
    \subfloat[MixUp, $k$=4]{\includegraphics[width=0.23\textwidth]{imgs/decode_res/Mixup/bs16_k4/grad_decode.png}} \hspace{1mm}
    \subfloat[MixUp, $k$=6]{\includegraphics[width=0.23\textwidth]{imgs/decode_res/Mixup/bs16_k6/grad_decode.png}} \hspace{1mm}
    \subfloat[MixUp+GradPrune, $k$=4, p=0.9]{\includegraphics[width=0.23\textwidth]{imgs/decode_res/Mixup/bs16_k4_gradprune/grad_decode.png}}

    
    \subfloat[Original]{\includegraphics[width=0.23\textwidth]{imgs/decode_res/InstaHide/bs1_k4/originals.png}} \hspace{1mm}
    \subfloat[InstaHide, $k$=4]{\includegraphics[width=0.23\textwidth]{imgs/decode_res/InstaHide/bs16_k4/grad_decode.png}} \hspace{1mm}
    \subfloat[InstaHide, $k$=6]{\includegraphics[width=0.23\textwidth]{imgs/decode_res/InstaHide/bs16_k6/grad_decode.png}} \hspace{1mm}
    \subfloat[InstaHide+GradPrune, $k$=4, $p$=0.9]{\includegraphics[width=0.23\textwidth]{imgs/decode_res/InstaHide/bs16_k4_gradprune/grad_decode.png}}
    \caption{Reconstrcuted dataset under MixUp and Intra-InstaHide against the strongest attack (batch size is 16).}
    \label{fig:encode_bs16}
\end{figure}



\begin{figure}[H]
\captionsetup[subfigure]{labelfont=scriptsize, textfont=tiny}
    \centering
    \subfloat[Original]{\includegraphics[width=0.23\textwidth]{imgs/decode_res/InstaHide/bs1_k4/originals.png}} \hspace{1mm}
    \subfloat[MixUp, $k$=4]{\includegraphics[width=0.23\textwidth]{imgs/decode_res/Mixup/bs32_k4/grad_decode.png}} \hspace{1mm}
    \subfloat[MixUp, $k$=6]{\includegraphics[width=0.23\textwidth]{imgs/decode_res/Mixup/bs32_k6/grad_decode.png}} \hspace{1mm}
    \subfloat[MixUp+GradPrune, $k$=4, $p$=0.9]{\includegraphics[width=0.23\textwidth]{imgs/decode_res/Mixup/bs32_k4_gradprune/grad_decode.png}}

    
    \subfloat[Original]{\includegraphics[width=0.23\textwidth]{imgs/decode_res/InstaHide/bs1_k4/originals.png}} \hspace{1mm}
    \subfloat[InstaHide, $k$=4]{\includegraphics[width=0.23\textwidth]{imgs/decode_res/InstaHide/bs32_k4/grad_decode.png}} \hspace{1mm}
    \subfloat[InstaHide, $k$=6]{\includegraphics[width=0.23\textwidth]{imgs/decode_res/InstaHide/bs32_k6/grad_decode.png}} \hspace{1mm}
    \subfloat[InstaHide+GradPrune, $k$=4, $p$=0.9]{\includegraphics[width=0.23\textwidth]{imgs/decode_res/InstaHide/bs32_k4_gradprune/grad_decode.png}}
    \caption{Reconstrcuted dataset under MixUp and Intra-InstaHide against the strongest attack (batch size is 32).}
    \label{fig:encode_bs32}
\end{figure}






We briefly recall the framework of statistical inference via empirical risk minimization.
Let $(\bbZ, \calZ)$ be a measurable space.
Let $Z \in \bbZ$ be a random element following some unknown distribution $\Prob$.
Consider a parametric family of distributions $\calP_\Theta := \{P_\theta: \theta \in \Theta \subset \reals^d\}$ which may or may not contain $\Prob$.
We are interested in finding the parameter $\theta_\star$ so that the model $P_{\theta_\star}$ best approximates the underlying distribution $\Prob$.
For this purpose, we choose a \emph{loss function} $\score$ and minimize the \emph{population risk} $\risk(\theta) := \Expect_{Z \sim \Prob}[\score(\theta; Z)]$.
Throughout this paper, we assume that
\begin{align*}
     \theta_\star = \argmin_{\theta \in \Theta} L(\theta)
\end{align*}
uniquely exists and satisfies $\theta_\star \in \text{int}(\Theta)$, $\nabla_\theta L(\theta_\star) = 0$, and $\nabla_\theta^2 L(\theta_\star) \succ 0$.

\myparagraph{Consistent loss function}
We focus on loss functions that are consistent in the following sense.

\begin{customasmp}{0}\label{asmp:proper_loss}
    When the model is \emph{well-specified}, i.e., there exists $\theta_0 \in \Theta$ such that $\Prob = P_{\theta_0}$, it holds that $\theta_0 = \theta_\star$.
    We say such a loss function is \emph{consistent}.
\end{customasmp}

In the statistics literature, such loss functions are known as proper scoring rules \citep{dawid2016scoring}.
We give below two popular choices of consistent loss functions.

\begin{example}[Maximum likelihood estimation]
    A widely used loss function in statistical machine learning is the negative log-likelihood $\score(\theta; z) := -\log{p_\theta(z)}$ where $p_\theta$ is the probability mass/density function for the discrete/continuous case.
    When $\Prob = P_{\theta_0}$ for some $\theta_0 \in \Theta$,
    we have $L(\theta) = \Expect[-\log{p_\theta(Z)}] = \kl(p_{\theta_0} \Vert p_\theta) - \Expect[\log{p_{\theta_0}(Z)}]$ where $\kl$ is the Kullback-Leibler divergence.
    As a result, $\theta_0 \in \argmin_{\theta \in \Theta} \kl(p_{\theta_0} \Vert p_\theta) = \argmin_{\theta \in \Theta} L(\theta)$.
    Moreover, if there is no $\theta$ such that $p_\theta \txtover{a.s.}{=} p_{\theta_0}$, then $\theta_0$ is the unique minimizer of $L$.
    We give in \Cref{tab:glms} a few examples from the class of generalized linear models (GLMs) proposed by \citet{nelder1972generalized}.
\end{example}

\begin{example}[Score matching estimation]
    Another important example appears in \emph{score matching} \citep{hyvarinen2005estimation}.
    Let $\bbZ = \reals^\tau$.
    Assume that $\Prob$ and $P_\theta$ have densities $p$ and $p_\theta$ w.r.t the Lebesgue measure, respectively.
    Let $p_\theta(z) = q_\theta(z) / \Lambda(\theta)$ where $\Lambda(\theta)$ is an unknown normalizing constant. We can choose the loss
    \begin{align*}
        \score(\theta; z) := \Delta_z \log{q_\theta(z)} + \frac12 \norm{\nabla_z \log{q_\theta(z)}}^2 + \text{const}.
    \end{align*}
    Here $\Delta_z := \sum_{k=1}^p \partial^2/\partial z_k^2$ is the Laplace operator.
    Since \cite[Thm.~1]{hyvarinen2005estimation}
    \begin{align*}
        L(\theta) = \frac12 \Expect\left[ \norm{\nabla_z q_\theta(z) - \nabla_z p(z)}^2 \right],
    \end{align*}
    we have, when $p = p_{\theta_0}$, that $\theta_0 \in \argmin_{\theta \in \Theta} L(\theta)$.
    In fact, when $q_\theta > 0$ and there is no $\theta$ such that $p_\theta \txtover{a.s.}{=} p_{\theta_0}$, the true parameter $\theta_0$ is the unique minimizer of $L$ \cite[Thm.~2]{hyvarinen2005estimation}.
\end{example}

\myparagraph{Empirical risk minimization}
Assume now that we have an i.i.d.~sample $\{Z_i\}_{i=1}^n$ from $\Prob$.
To learn the parameter $\theta_\star$ from the data, we minimize the empirical risk to obtain the \emph{empirical risk minimizer}
\begin{align*}
    \theta_n \in \argmin_{\theta \in \Theta} \left[ L_n(\theta) := \frac1n \sum_{i=1}^n \score(\theta; Z_i) \right].
\end{align*}
This applies to both maximum likelihood estimation and score matching estimation. 
In \Cref{sec:main_results}, we will prove that, with high probability, the estimator $\theta_n$ exists and is unique under a generalized self-concordance assumption.

\begin{figure}
    \centering
    \includegraphics[width=0.45\textwidth]{graphs/logistic-dikin} %0.4
    \caption{Dikin ellipsoid and Euclidean ball.}
    \label{fig:logistic_dikin}
\end{figure}

\myparagraph{Confidence set}
In statistical inference, it is of great interest to quantify the uncertainty in the estimator $\theta_n$.
In classical asymptotic theory, this is achieved by constructing an asymptotic confidence set.
We review here two commonly used ones, assuming the model is well-specified.
We start with the \emph{Wald confidence set}.
It holds that $n(\theta_n - \theta_\star)^\top H_n(\theta_n) (\theta_n - \theta_\star) \rightarrow_d \chi_d^2$, where $H_n(\theta) := \nabla^2 L_n(\theta)$.
Hence, one may consider a confidence set $\{\theta: n(\theta_n - \theta)^\top H_n(\theta_n) (\theta_n - \theta) \le q_{\chi_d^2}(\delta) \}$ where $q_{\chi_d^2}(\delta)$ is the upper $\delta$-quantile of $\chi_d^2$.
The other is the \emph{likelihood-ratio (LR) confidence set} constructed from the limit $2n [L_n(\theta_\star) - L_n(\theta_n)] \rightarrow_d \chi_d^2$, which is known as the Wilks' theorem \citep{wilks1938large}.
These confidence sets enjoy two merits: 1) their shapes are an ellipsoid (known as the \emph{Dikin ellipsoid}) which is adapted to the optimization landscape induced by the population risk; 2) they are asymptotically valid, i.e., their coverages are exactly $1 - \delta$ as $n \rightarrow \infty$.
However, due to their asymptotic nature, it is unclear how large $n$ should be in order for it to be valid.

Non-asymptotic theory usually focuses on developing finite-sample bounds for the \emph{excess risk}, i.e., $\Prob(L(\theta_n) - L(\theta_\star) \le C_n(\delta)) \ge 1 - \delta$.
To obtain a confidence set, one may assume that the population risk is twice continuously differentiable and $\lambda$-strongly convex.
Consequently, we have $\lambda \norm{\theta_n - \theta_\star}_2^2 / 2 \le L(\theta_n) - L(\theta_\star)$ and thus we can consider the confidence set $\calC_{\text{finite}, n}(\delta) := \{\theta: \norm{\theta_n - \theta}_2^2 \le 2C_n(\delta)/\lambda\}$.
Since it originates from a finite-sample bound, it is valid for fixed $n$, i.e., $\Prob(\theta_\star \in \calC_{\text{finite}, n}(\delta)) \ge 1 - \delta$ for all $n$; however, it is usually conservative, meaning that the coverage is strictly larger than $1 - \delta$.
Another drawback is that its shape is a Euclidean ball which remains the same no matter which loss function is chosen.
We illustrate this phenomenon in \Cref{fig:logistic_dikin}.
Note that a similar observation has also been made in the bandit literature \citep{faury2020improved}.

We are interested in developing finite-sample confidence sets.
However, instead of using excess risk bounds and strong convexity, we construct in \Cref{sec:main_results} the Wald and LR confidence sets in a non-asymptotic fashion, under a generalized self-concordance condition.
These confidence sets have the same shape as their asymptotic counterparts while maintaining validity for fixed $n$.
These new results are achieved by characterizing the critical sample size enough to enter the asymptotic regime.


% Note: in this sample, the section number is hard-coded in. Following
% proper LaTeX conventions, it should properly be coded as a reference:

%In this appendix we prove the following theorem from
%Section~\ref{sec:textree-generalization}:



\end{document}
