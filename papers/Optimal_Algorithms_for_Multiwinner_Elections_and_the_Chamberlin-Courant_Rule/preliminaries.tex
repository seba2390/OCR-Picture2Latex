\section{Preliminaries}
\label{secLprelim}
%\kn{A few conventions: Let's use \g{} and \b{}. Let's not italicize Latin phrases. Let's use ``xxx'' for quotation marks. (Not "xxx" or ``xxx".) Let's use $k^{\text{th}}$ and $(k + 1)^{\text{st}}$. Let's use $A \setminus B$.}

We consider the problem of selecting a subset of cardinality $k$ from a set $\C$ of $m$ candidates. We call this subset a \emph{committee}. A set $\V$ of $n$ voters express their preferences on the candidates ordinally. Each voter $v$ has a bijective ranking function $r_v : \C \rightarrow \{1, 2, \ldots, m\}$, and $v$ prefers those $c$'s with smaller $r_v(c)$. For example, the top-ranked candidate of $v$, denoted by $c_{\mathrm{top}(v)}$, satisfies $r_v\left(c_{\mathrm{top}(v)}\right) = 1$, and the bottom-ranked $c_{\mathrm{bot}(v)}$ satisfies $r_v\left(c_{\mathrm{bot}(v)}\right) = m$.

In $s$-Borda score, the cost for a voter $v$ of a committee $T$ is the sum of her ranks of the top $s$ candidates in $T$: $r_v(T) = \min_{Q \subseteq T, |Q| = s} \sum_{c \in Q} r_v(c)$. Further, the $s$-Borda score ($s \leq k$) of a committee $T$ is the average cost for all voters:
\[
r_{\V}(T) = \frac{1}{n} \sum_{v \in \V} r_v(T) = \frac{1}{n} \sum_{v \in \V} \left(\min_{Q \subseteq T, |Q| = s} \sum_{c \in Q} r_v(c)\right).
\]
In particular, when $s = 1$, $r_{\V}(T) = \frac{1}{n} \sum_{v \in \V} \min_{c \in T} r_v(c)$.

Fix a voter $v$ and look at her ranking on $\C$. If we pick a random size-$k$ subset of $\C$, the $t^{\text{th}}$ smallest rank is $t \cdot \frac{m + 1}{k + 1}$ in expectation. (See Appendix~\ref{app:folklore} for a proof of this well-known fact.)  
 Therefore, the expected performance of a random committee is
\[
\E_{T \subseteq \C}[r_{\V}(T)] = \frac{1}{n} \sum_{v \in \V} \E_{T \subseteq \C}[r_{v}(T)] = \sum_{t = 1}^s t \cdot \frac{m + 1}{k + 1} = \frac{s(s + 1)}{2} \cdot \frac{m + 1}{k + 1}.
\]
Define the benchmark $\rand$ to be the expected performance of a random committee when $s = 1$ as $\rand = \frac{m + 1}{k + 1}$. We will justify this benchmark in the subsequent sections.

We consider two simple committee-selection rules: \g{} and \b{}. These algorithms run in $k$ iterations, during which they build sets $\varnothing = T_0 \subsetneq T_1 \subsetneq \cdots \subsetneq T_k$, and declare $T_k$ as the selected committee.

In the $j^{\text{th}}$ iteration, \g{} picks candidate $c_j \in \C \setminus T_{j - 1}$ that minimizes $r_{\V}(T_{j - 1} \cup \{c_j\})$, and let $T_j = T_{j - 1} \cup \{c_j\}$. \b{}~\cite{Banzhaf,Heuristics}, on the other hand, picks candidate $c_j \in \C \setminus T_{j - 1}$ to minimize
\[
\sum_{\substack{S \subseteq \C: |S| = k \\ S \supseteq T_{j - 1} \cup \{c_j\}}} r_{\V}(S)
\]
in the $j^{\text{th}}$ iteration, and then sets $T_j = T_{j - 1} \cup \{c_j\}$. In other words, it greedily picks the candidate that minimizes the final score if the rest of the committee is chosen uniformly at random. Both \g{} and \b{} can run in polynomial time~\cite{Heuristics}.

Throughout the paper, we use \rand{}, \g{} and \b{} to denote either the algorithms or their performances, which should be clear from the context.
