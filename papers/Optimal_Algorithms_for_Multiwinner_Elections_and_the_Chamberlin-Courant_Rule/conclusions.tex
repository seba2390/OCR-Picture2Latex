\section{Conclusion}
%In this paper, we have presented an improved lower bound for the score achievable by polynomial time algorithms for the Chamberlin-Courant (CC) rule (also called the $1$-Borda score), and its generalizations. This enabled us to perform a tight analysis of natural greedy heuristics for this problem, and thereby perform a fine-grained analytic comparison of their performance. 
Our work opens some interesting directions for further research. One open question is to extend our results to the stronger notion of approximate core stability under the CC rule for which the best known result is a $16$-approximation~\cite{ChengJMW20,JiangMW20}. It would be interesting to explore if our techniques can help improve the approximation factor via a simple-to-implement procedure. 

We conjecture that there is a lower bound of $\Omega(s^{3/2}) \cdot \rand$ on the score achievable by poly-time algorithms for $s$-Borda, i.e., that the algorithm in Section~\ref{sec:lp} is almost optimal. This will require a non-trivial strengthening of known hardness results for maximum multicover~\cite{Barman}.  It would also be interesting to explore if there are greedy rules that can match these bounds.

In the same vein, another interesting question is to map the landscape of approximation ratios  for generalizations such as  committee scoring rules. The work of~\cite{Byrka} shows strong positive results when voters assign a smooth set of weights to all candidates in the committee, while our work considers the case where the weights are concentrated on higher-ranked candidates. There is a large middle ground where the approximability of this problem is poorly understood. 

