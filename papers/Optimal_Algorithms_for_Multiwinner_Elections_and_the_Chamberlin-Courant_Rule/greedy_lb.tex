
\paragraph{Construction.} In the sequel, we will prove the above theorem. In the instance we construct, $m$, $n$, and $k$ are all sufficiently large. For convenience of illustration, we scale down the ranks by a factor of $m$: now the ranks are $\frac{1}{m}, \frac{2}{m}, \ldots, \frac{m - 1}{m}, 1$. As $m \to \infty$, $\frac{1}{m} \to 0$, so the set of ranking $\{\frac{1}{m}, \frac{2}{m}, \ldots, 1\}$ will become dense in $[0, 1]$, and thus we regard the ranking as being continuous from $0$ to $1$. Our goal becomes to construct an instance in which \g{} gives $r_{\V}(T_k) > 1.962 \cdot \frac{1}{k + 1}$. %We also think of the voters as being divided by $n$ and forming a continuum from $0$ to $1$. These voters are located on a circle (the base in Fig.~\ref{fig:cylinder}) with angular position ranging from $0$ to $2 \pi$.

There are sufficiently many voters, enabling us to view them as a continuum from $0$ to $1$, forming a circle (the base in Fig.~\ref{fig:cylinder}) with angular position ranging from $0$ to $2 \pi$. Imagine that each voter writes down her favorite, her second favorite, \ldots, her least favorite candidate in that order vertically. The result is the side of a cylinder with height $1$, as depicted in Fig.~\ref{fig:cylinder}. Each point on the side identifies a candidate, whose distance to the top, $d$, indicates the corresponding voter ranks him as her $(dm)^{\text{th}}$ favorite candidate (i.e., the candidate has a rank of $d$ in the voter's preference after scaling).

%Fix an integer $N$, and take a small-enough constant $a \in (0, 1)$, and define $f(t) = a\varphi^{\frac{t}{N}}$, where $\varphi$ denotes the golden ratio $\frac{\sqrt{5} - 1}{2} \approx 0.618$. 

\begin{figure}[H]
\centering
\begin{tikzpicture}[scale = 1.0, yscale = 2.5, xscale = 4.0]
\draw[domain=-1.002:1.002, variable=\x, samples=1000, smooth, very thick, dotted] plot ({\x}, {0.1+0.2*((1-\x*\x)^2)^0.25});
\draw[domain=-1.002:1.002, variable=\x, samples=1000, smooth, very thick] plot ({\x}, {0.1-0.2*((1-\x*\x)^2)^0.25});
\draw[domain=-1.002:1.002, variable=\x, samples=1000, smooth, ultra thick, color3] plot ({\x}, {2+0.2*((1-\x*\x)^2)^0.25});
\draw[domain=-1.002:1.002, variable=\x, samples=1000, smooth, ultra thick, color3] plot ({\x}, {2-0.2*((1-\x*\x)^2)^0.25});

\draw[domain=0.1:2, variable=\y, samples=20, smooth, very thick] plot ({-1}, {\y});
\draw[domain=0.1:2, variable=\y, samples=20, smooth, very thick] plot ({1}, {\y});


\draw[domain=-1:1, variable=\x, color1, samples=200, smooth, very thick] plot ({\x}, {2*(1-0.4*exp(-0.076587*(asin(\x)/180*pi+pi/2))) - 0.2*(1-\x*\x)^0.5});
\draw[domain=-1:1, variable=\x, color1, samples=200, smooth, very thick, dotted] plot ({\x}, {2*(1-0.4*((5^0.5-1)/2)^0.5*exp(-0.076587*(asin(-\x)/180*pi+pi/2))) + 0.2*(1-\x*\x)^0.5});

\draw[domain=-1:1, variable=\x, color1, samples=200, smooth, very thick] plot ({\x}, {2*(1-0.4*((5^0.5-1)/2)^1.0*exp(-0.076587*(asin(\x)/180*pi+pi/2))) - 0.2*(1-\x*\x)^0.5});
\draw[domain=-1:1, variable=\x, color1, samples=200, smooth, very thick, dotted] plot ({\x}, {2*(1-0.4*((5^0.5-1)/2)^1.5*exp(-0.076587*(asin(-\x)/180*pi+pi/2))) + 0.2*(1-\x*\x)^0.5});

\draw[domain=-1:1, variable=\x, color1, samples=200, smooth, very thick] plot ({\x}, {2*(1-0.4*((5^0.5-1)/2)^2.0*exp(-0.076587*(asin(\x)/180*pi+pi/2))) - 0.2*(1-\x*\x)^0.5});
\draw[domain=-1:1, variable=\x, color1, samples=200, smooth, very thick, dotted] plot ({\x}, {2*(1-0.4*((5^0.5-1)/2)^2.5*exp(-0.076587*(asin(-\x)/180*pi+pi/2))) + 0.2*(1-\x*\x)^0.5});

\draw[domain=-1:1, variable=\x, color1, samples=200, smooth, very thick] plot ({\x}, {2*(1-0.4*((5^0.5-1)/2)^3.0*exp(-0.076587*(asin(\x)/180*pi+pi/2))) - 0.2*(1-\x*\x)^0.5});
\draw[domain=-1:1, variable=\x, color1, samples=200, smooth, very thick, dotted] plot ({\x}, {2*(1-0.4*((5^0.5-1)/2)^3.5*exp(-0.076587*(asin(-\x)/180*pi+pi/2))) + 0.2*(1-\x*\x)^0.5});

\draw[domain=-1:1, variable=\x, color1, samples=200, smooth, very thick] plot ({\x}, {2*(1-0.4*((5^0.5-1)/2)^4.0*exp(-0.076587*(asin(\x)/180*pi+pi/2))) - 0.2*(1-\x*\x)^0.5});
\draw[domain=-1:1, variable=\x, color1, samples=200, smooth, very thick, dotted] plot ({\x}, {2*(1-0.4*((5^0.5-1)/2)^4.5*exp(-0.076587*(asin(-\x)/180*pi+pi/2))) + 0.2*(1-\x*\x)^0.5});


\draw[domain=-0.29:0.29, variable=\x, color2, samples=200, smooth, ultra thick] plot ({\x}, {2*(1-0.4*exp(-0.076587*(asin(\x)/180*pi+pi/2))) - 0.2*(1-\x*\x)^0.5});


\draw[domain=-0.02:0.02, variable=\y, color2, samples=20, smooth, very thick] plot ({-0.29}, {\y+2*(1-0.4*exp(-0.076587*(asin(-0.29)/180*pi+pi/2))) - 0.2*(1-(-0.29)*(-0.29))^0.5});

\draw[domain=-0.02:0.02, variable=\y, color2, samples=20, smooth, very thick] plot ({-0.10}, {\y+2*(1-0.4*exp(-0.076587*(asin(-0.10)/180*pi+pi/2))) - 0.2*(1-(-0.10)*(-0.10))^0.5});

\draw[domain=-0.02:0.02, variable=\y, color2, samples=20, smooth, very thick] plot ({0.10}, {\y+2*(1-0.4*exp(-0.076587*(asin(0.10)/180*pi+pi/2))) - 0.2*(1-(0.10)*(0.10))^0.5});

\draw[domain=-0.02:0.02, variable=\y, color2, samples=20, smooth, very thick] plot ({0.29}, {\y+2*(1-0.4*exp(-0.076587*(asin(0.29)/180*pi+pi/2))) - 0.2*(1-(0.29)*(0.29))^0.5});


\draw[domain=-0.188:0.188, variable=\x, color2, samples=200, smooth, ultra thick] plot ({\x}, {2*(1-0.4*((5^0.5-1)/2)^2.0*exp(-0.076587*(asin(\x)/180*pi+pi/2))) - 0.2*(1-\x*\x)^0.5});


\draw[domain=-0.02:0.02, variable=\y, color2, samples=20, smooth, very thick] plot ({-0.188}, {\y + 2*(1-0.4*((5^0.5-1)/2)^2.0*exp(-0.076587*(asin(-0.188)/180*pi+pi/2))) - 0.2*(1-(-0.188)*(-0.188))^0.5});

\draw[domain=-0.02:0.02, variable=\y, color2, samples=20, smooth, very thick] plot ({-0.114}, {\y + 2*(1-0.4*((5^0.5-1)/2)^2.0*exp(-0.076587*(asin(-0.114)/180*pi+pi/2))) - 0.2*(1-(-0.114)*(-0.114))^0.5});

\draw[domain=-0.02:0.02, variable=\y, color2, samples=20, smooth, very thick] plot ({-0.0382}, {\y + 2*(1-0.4*((5^0.5-1)/2)^2.0*exp(-0.076587*(asin(-0.0382)/180*pi+pi/2))) - 0.2*(1-(-0.0382)*(-0.0382))^0.5});

\draw[domain=-0.02:0.02, variable=\y, color2, samples=20, smooth, very thick] plot ({0.0382}, {\y + 2*(1-0.4*((5^0.5-1)/2)^2.0*exp(-0.076587*(asin(0.0382)/180*pi+pi/2))) - 0.2*(1-(0.0382)*(0.0382))^0.5});

\draw[domain=-0.02:0.02, variable=\y, color2, samples=20, smooth, very thick] plot ({0.114}, {\y + 2*(1-0.4*((5^0.5-1)/2)^2.0*exp(-0.076587*(asin(0.114)/180*pi+pi/2))) - 0.2*(1-(0.114)*(0.114))^0.5});

\draw[domain=-0.02:0.02, variable=\y, color2, samples=20, smooth, very thick] plot ({0.188}, {\y + 2*(1-0.4*((5^0.5-1)/2)^2.0*exp(-0.076587*(asin(0.188)/180*pi+pi/2))) - 0.2*(1-(0.188)*(0.188))^0.5});


\draw [very thick, ->] (1.2, 2) -- (1.2, 0.1);

\node [color3] at (-0.9, 2.2) {Voters};
\node [color1] at (-0.5, 1.25) {Critical Candidates};
\node [color2] at (0, 1) {Lower-Layer Candidates};
\node [color2] at (0, 1.45) {Higher-Layer Candidates};
\node at (1.45, 1.4) {Decreasing};
\node at (1.45, 1.28) {Preferences};

\node at (1.40, 2.0) {Rank: $0$};
\node at (1.40, 0.1) {Rank: $1$};

\draw [decorate, very thick, color2, decoration={brace,amplitude=5pt}] (-1, {2-0.8}) -- (-1, {2-0.8*(5^0.5-1)/2});
\draw [decorate, very thick, color2, decoration={brace,amplitude=5pt}] (-1, {2-0.8*(5^0.5-1)/2}) -- (-1, {2-0.8*((5^0.5-1)/2)^2});

\node [color2] at (-1.25, {(2-0.8+2-0.8*(5^0.5-1)/2)/2}) {$1^{\text{st}}$ Layer};
\node [color2] at (-1.25, {(2-0.8*(5^0.5-1)/2+2-0.8*((5^0.5-1)/2)^2)/2}) {$2^{\text{nd}}$ Layer};



\end{tikzpicture}
\caption{Construction of the Bad Instance for \g{}}
\label{fig:cylinder}
\end{figure}


We divide the set of candidates into two types -- {\em critical} and {\em dummy}. The former set has size $k \ll m$, and the latter has size $m-k$. Our proof will show that \g{} will choose the critical candidates in a fixed order, and will not choose any dummy candidate.

The critical candidates are present in $\ell$ ``layers'' as shown in the red spiral in Fig.~\ref{fig:cylinder}, where $\ell$ is sufficiently large. This figure shows the ranks of the critical candidates in the voters' profiles. We parametrize this spiral by $\theta$, which maps to the voter at the corresponding angular position $2\pi\theta$. We place critical candidates in order, where each candidate appears a number of times consecutively on the spiral. Therefore, each voter has one critical candidate from each layer $t = 0, 1, \ldots, \ell$ in the spiral part of its ranking.

In the $t^{\text{th}}$ layer, the parameter $\theta$ lies in $[t - 1, t)$. The critical candidate when the parameter is $\theta$ has rank $g(\theta) = a \varphi^{\theta}$ for the voter at angular position $2\pi\theta$. Here, $\varphi$ denotes the golden ratio $\frac{\sqrt{5} - 1}{2} \approx 0.618$, and $a$ is a sufficiently small constant so that rounding to the nearest integer does not change the analysis. This critical candidate is placed for a certain length $h(\theta)$ on the spiral, which means this candidate appears at rank $g(\theta)$ for voters in the range $\left[2\pi\theta, 2\pi(\theta + h(\theta))\right]$. In our construction, $h(\theta)$ will be very small, so that we will say this candidate appears $h(\theta)$ times at rank $g(\theta)$ for parameter $\theta$. The greater $\theta$ is, the smaller $h(\theta)$ has to be, and we will calculate its expression later. 

For the convenience of analysis, at the layer $t = 0$, that is, for $\theta \in [-1,0)$, there is a special candidate appearing on the spiral throughout the layer. This special candidate is picked first by \g{}. Other than its appearance on the spiral, any critical candidate is placed at the very bottom, i.e., rank $1$, for the other voters. Denote the total number of critical candidates by $k$. Then we have $m - k$ dummy candidates. These dummy candidates are symmetrically placed at other ranks. We copy each voter $(m - k)!$ times, once for each possible permutation of the dummy candidates to place in the remaining ranks. %(See Remark at the end of the proof for how to make the construction polynomial size in $m$.)

The idea of this construction is to trick \g{} into picking every critical candidate on the spiral in order, while in fact, lower-layer critical candidates have no contribution to the objective once higher-layer ones have been selected. The following analysis computes the optimal parameters to realize this plan.




\paragraph{Not Choosing a Dummy Candidate.} We first ensure \g{} does not choose a dummy candidate in this instance by setting $h(\theta)$ properly. We assume that \g{} chooses critical candidates in increasing order of $\theta$, and we will justify this assumption later.


To simplify notation, denote $X = \int_{0}^1 a\varphi^{\theta}\d \theta$ and $Y = \int_{0}^1 a^2\varphi^{2\theta}\d \theta$. Computing these explicitly:
\[
X = \frac{a}{\ln\varphi}(\varphi - 1), \qquad Y = \frac{a^2}{2\ln\varphi}(\varphi^2 - 1) = X^2 \frac{(\varphi + 1)\ln\varphi}{2(\varphi - 1)}.
\]

Using this notation, consider the critical candidate at the beginning of the first layer, that is, at $\theta = 0$. Since \g{} chooses the candidate at layer $t = 0$, the decrease in score due to this critical candidate is:
\begin{equation}
\label{eq:improve_critical}
h(0) \cdot (g(-1) - g(0)) = h(0) \cdot a \cdot \left(\frac{1}{\varphi} - 1\right) = h(0) \cdot a \cdot \varphi.
\end{equation}
where we have used that since $\varphi$ is the golden ratio, $\varphi + \varphi^2 = 1$.

Now consider the dummy candidates. Just after \g{} has chosen the special candidate at layer $t = 0$, each such candidate improves the rank of $g(\theta-1)$ fraction of voters at $\theta \in [0, 1)$. This is because we placed all permutations of the dummy candidates with each voter $\theta$, and \g{} has already chosen the special candidate. By the same reasoning, conditioned on improvement, the average improvement is $g(\theta-1)/2$. Therefore, the decrease in score due to a dummy candidate is: 
\begin{equation}
\label{eq:improve_dummy}
\int_0^{1} \frac{g^2(\theta-1)}{2} \d \theta = \frac{a^2}{2\varphi^2} \int_0^{1} \varphi^{2\theta} \d \theta = \frac{1}{2\varphi^2} \cdot Y.
\end{equation}
Since we want \g{} to choose the critical candidate, we need to set
\[
h(0) = \frac{Y}{2\varphi^3 a}.
\]
By the symmetry of the spiral, an identical calculation now holds for all $\theta > 0$. To make \g{} choose the critical candidate at this location (assuming it has chosen critical candidates for smaller values of $\theta$), we need:
\[
h(\theta) = \frac{Y}{2\varphi^3 a} \varphi^{\theta}.
\]
Note that $h(\theta)$ depends linearly on $a$, so that for very small $a$, we can pretend this set of voters lies exactly at $\theta$. Further, $h(\theta)$ is decreasing with $\theta$.
%to make the spiral symmetric and the calculation identical at any step of \g{}.

\paragraph{Choosing Critical Candidates in Order.} We now show that \g{} chooses the critical candidates following the order on the spiral.
\begin{lemma}
\label{lem:greedyopt2}
\g{} chooses the critical candidates in increasing order of $\theta$.
\end{lemma}
\begin{proof}
The calculation is identical at any step of \g{}, so we focus on the step where \g{} is at the beginning of the first layer, that is, considering the critical candidate at $\theta = 0$. Recall that \g{} has chosen the special candidate at layer $t = 0$. The previous analysis showed that the critical candidate at $\theta = 0$ yields decrease of $\frac{Y}{2 \varphi^2}$. For critical candidates in the same layer $t = 1$ (that is, for $\theta \in [0, 1)$), the contribution of the candidate at $\theta$ is
\[
h(\theta) \cdot (g(\theta-1) - g(\theta)) = \frac{Y}{2\varphi^3} \varphi^{\theta} \left(\varphi^{\theta - 1} - \varphi^{\theta}\right) = \frac{Y}{2\varphi^2} \cdot \varphi^{2\theta},
\]
which decreases with $\theta$, so that the current candidate, $\theta = 0$, offers the best decrease. Here, we have used that since $\varphi$ is the golden ratio, $\varphi^2 + \varphi = 1$.

For $t \ge 1$, suppose we instead considered a candidate $t +\theta$ for $\theta \in [0,1)$ located in layer $t+1$. Conditioned on having chosen layer $t = 0$, this candidate gives a contribution of
\begin{align*}
h(t + \theta) \cdot (g(\theta-1) - g(t + \theta)) &\leq h(t) \cdot (g(-1) - g(t))\\
&\leq \max\big(h(2) \cdot g(-1), \ h(1) \cdot (g(-1) - g(1))\big)\\
&= \max\left(\frac{Y}{2\varphi^2}, \ \frac{Y}{2\varphi^2a} \cdot a\left(\frac{1}{\varphi} - \varphi\right)\right) = \frac{Y}{2\varphi^2},
\end{align*}
where the first inequality uses that $h(\theta)$ is decreasing in $\theta$, and that $\varphi < 1$.

Therefore, \g{} will pick the critical candidate at $\theta = 0$ instead of another candidate at the same or a higher layer. Since the argument is identical at each $\theta$, \g{} picks critical candidates in order on the spiral.
\end{proof}

\paragraph{The Lower Bound.} So far we have shown that \g{} chooses critical candidates in increasing order of layers and does not choose dummy candidates. We finally put it all together and show the following bound, which completes the proof of Theorem~\ref{thm:greedy_lb}. 

\begin{proof}[Proof of Theorem~\ref{thm:greedy_lb}]
The number of critical candidates on the $t^{\text{th}}$ layer ($\theta \in [t - 1, t)$) is
\[
\int_{t - 1}^t \frac{1}{h(\theta)} \d \theta = \frac{2 \varphi^3 a}{Y} \int_{t - 1}^t \varphi^{-\theta} \d \theta = \frac{2 a}{\varphi^{t-3} Y} \int_{-1}^0 \varphi^{-\theta} \d \theta  = \frac{2 a}{\varphi^{t-3} Y} \int_{0}^1 \varphi^{\theta} \d \theta = \frac{2 X}{\varphi^{t - 3} Y}.
\]
Therefore, when it is done with the $\ell^{\text{th}}$ layer, the number of candidates \g{} has picked is
\[
k = \frac{2 X}{\varphi^{\ell - 3} Y} (1 + \varphi + \varphi^2 + \cdots + \varphi^{\ell - 1}) \rightarrow \frac{2 X}{(1 - \varphi)\varphi^{\ell - 3} Y}
\]
when $\ell$ is large. Meanwhile, the $1$-Borda score of \g{} is
\[
r_{\V}(T_k) = \int_{0}^1 g(\ell - 1 + \theta) \d \theta = \varphi^{\ell - 1} X.
\]
Therefore, the approximation ratio is
\[
(k+1) r_{\V}(T_k) \ge \frac{2 X}{(1 - \varphi)\varphi^{\ell - 3} Y} \cdot \varphi^{\ell - 1} X = \frac{2 \varphi^2 X^2}{(1 - \varphi) Y} = \frac{2\varphi^2}{(1 - \varphi)} \cdot \frac{2(\varphi - 1)}{(\varphi + 1) \ln \varphi} = -\frac{4 \varphi^2}{(\varphi + 1) \ln \varphi} > 1.962. \qedhere
\]
\end{proof}


%\paragraph{Remark} \kn{Is this still correct? I'm not sure... (This is referred to in the construction part.)} The current construction has $n$ which is exponential in $m$. However, using the same proof as Lemma~\ref{lem:random_construction}, if we sample $\mbox{poly}(m,1/\varepsilon)$ voters, the improvement generated by each candidate is approximated to within a factor of $(1+\varepsilon)$ at all steps of \g{} with high probability. This is sufficient to make \g{}  choose critical candidates in the correct order. Note that analysis of \g{} remains the same even for finite $N = \mbox{poly}(m)$, and the limiting case is just to simplify the analysis. This makes the overall construction polynomial size in the number of candidates $m$ and we omit the straightforward details.
