\documentclass{article}
\usepackage{amsmath,amsthm,amssymb}
\usepackage[utf8]{inputenc}
\usepackage{fullpage}
\usepackage{url}
\usepackage{graphicx}
\usepackage{xcolor}
\usepackage{tikz}
\usepackage{optidef}
\usepackage{float}
\usepackage[ruled]{algorithm2e} % For algorithms
\usetikzlibrary{decorations.pathreplacing}

% --------------------------------------------------------------------
\usepackage[only,llbracket,rrbracket]{stmaryrd}
\SetSymbolFont{stmry}{bold}{U}{stmry}{m}{n}

\usepackage{amsmath,amsthm,amssymb,mathtools,nicefrac}
\usepackage[bb=boondox]{mathalfa}
\usepackage{xspace}
\usepackage{enumitem}
\usepackage{bbding}
\usepackage[sort&compress]{natbib}

\makeatletter
\def\NAT@spacechar{~}% NEW
\makeatother

% --------------------------------------------------------------------
\newcommand{\notation}[4][0]{\newcommand{#2}[#1]{#3}}

\newcommand{\rname}[1]{[\textsc{#1}]}

% --------------------------------------------------------------------
% TODOs

\usepackage{todonotes}
\usepackage{setspace}

\setlength{\marginparwidth}{3cm}

\newcounter{todocnt}
\newcommand{\todox}[3][]{%
\refstepcounter{todocnt}{%
\setstretch{0.7}%
\todo[color={red!100!green!33},size=\small,#1]{%
  \textbf{[\uppercase{#2}\thetodocnt]:}~#3}}}

\newcommand{\gb}[2][]{\todox[#1]{GB}{#2}}
\newcommand{\py}[2][]{\todox[#1]{PY}{#2}}
\newcommand{\jh}[2][]{\todox[#1]{JH}{#2}}
\newcommand{\te}[2][]{\todox[#1]{TE}{#2}}

% --------------------------------------------------------------------
% English

\def\ie{i.e.\xspace}
\def\eg{e.g.\xspace}

% --------------------------------------------------------------------
% Local labels

\newcounter{localc}

\newcommand*\lclabel[1]{\label{exer:\thelocalc:#1}}
\newcommand*\lcref[1]{\ref{exer:\thelocalc:#1}}
\newcommand*\lceqref[1]{\eqref{exer:\thelocalc:#1}}

\newcommand{\steplocal}{\stepcounter{localc}}

% --------------------------------------------------------------------
% Standard symbols

\newcommand{\eqdef}{\mathrel{\stackrel{\scriptscriptstyle \triangle}{=}}}
\newcommand{\iffdef}{\mathrel{\stackrel{\scriptscriptstyle \triangle}{\iff}}}

\newcommand{\inv}[1]{#1^{\raisebox{.2ex}{$\scriptscriptstyle-\!1$}}}

\newcommand{\proj}[1]{\pi_{#1}}
\newcommand{\fst}{\proj{1}}
\newcommand{\snd}{\proj{2}}

\newcommand{\rR}{\mathrel{\mathcal{R}}}
\newcommand{\rS}{\mathrel{\mathcal{S}}}

\newcommand{\oR}{\mathcal{R}}
\newcommand{\oS}{\mathcal{S}}

\newcommand{\simplies}{\mathrel{\Rightarrow}}

\newcommand{\true}{{\mathrel{\top}}}
\newcommand{\false}{{\mathrel{\perp}}}

\newcommand{\fun}[1]{\lambda\,{#1} .\,}
\newcommand{\card}[1]{{|{#1}|}}
\newcommand{\rcomp}[1]{\overline{#1}}

\newcommand{\relrestr}[3]{\mathrel{{#3}_{|_{{#1} \!\times\! {#2}}}}}

\newcommand{\qt}[2]{{#1}_{/_{\!#2}}}
\newcommand{\qtc}[2]{{[#1]}_{#2}}

\newcommand{\rrefl}[1]{\mathrel{#1^=}}
\newcommand{\rsym}[1]{\mathrel{\inv{#1}}}

\def\ssrc{\mathop{\top}}
\def\sdst{\mathop{\perp}}

\newcommand{\cset}[1]{\mathcal{E}(#1)}

% --------------------------------------------------------------------
% \bigtimes
\DeclareFontFamily{U}{mathx}{\hyphenchar\font45}
\DeclareFontShape{U}{mathx}{m}{n}{
      <5> <6> <7> <8> <9> <10>
      <10.95> <12> <14.4> <17.28> <20.74> <24.88>
      mathx10
      }{}
\DeclareSymbolFont{mathx}{U}{mathx}{m}{n}
\DeclareMathSymbol{\bigtimes}{1}{mathx}{"91}

% --------------------------------------------------------------------
% Standard sets

\newcommand{\rset}[1]{\ensuremath{\mathbb{#1}}}

\newcommand{\BB}{{\{0, 1\}}}
\newcommand{\NN}{\rset{N}}
\newcommand{\ZZ}{\rset{Z}}
\newcommand{\QQ}{\rset{Q}}
\newcommand{\RR}{\rset{R}}
\newcommand{\RRP}{\rset{R}^+}

\renewcommand{\setminus}{\mathrel{-}}

\newcommand{\I}[1]{\mathcal{I}_{#1}}

\newcommand{\intv}[2]{{[{#1}, {#2}]}}
\newcommand{\indicator}[1]{\raisebox{.25ex}{$\chi_{#1}$}}

% ---------------------------------------------------------------------
% Distributions

\newcommand{\DistOp}{{\mathbb{D}}}
\newcommand{\FDistOp}{{\DistOp^{\scriptscriptstyle =1}}}
\newcommand{\Exp}{\mathbb{E}}
\newcommand{\ExpD}{\mathbb{E}}
\newcommand{\PrS}{\mathbb{P}}

\newcommand{\Dist}{\DistOp}
\newcommand{\FDist}{\FDistOp}

\newcommand{\dnull}[1][]{{{\mathbb{0}}^{#1}}}
\newcommand{\dunit}[2][]{{{\mathbb{1}}^{#1}_{#2}}}
\newcommand{\dnormed}[1]{{#1}^{=1}}
\newcommand{\dlet}[3]{\ExpD_{{#1} \sim {#2}} [{#3}]}
\newcommand{\dslet}[2]{\ExpD_{#1} [{#2}]}
\newcommand{\dclet}[4]{\dlet {#1} {\drestr {#2} {#3}} {#4}}
\newcommand{\dsclet}[3]{\dslet {\drestr {#1} {#2}} {#3}}
\newcommand{\dmargin}[2]{{#1} \circ \inv{#2}}
\newcommand{\dlift}[1]{#1^{\sharp}}
\newcommand{\mlift}[1]{{\overline{#1}}} %FIXME
\newcommand{\dproj}[1]{\dlift{\proj{#1}}}
\newcommand{\dfst}{\dproj{1}}
\newcommand{\dsnd}{\dproj{2}}
\newcommand{\drestr}[2]{{#1}_{|{#2}}}
\newcommand{\rdrestr}[2]{\drestr {#2} {#1}}
\newcommand{\dlim}[2]{\lim_{{#1}\infty}\,#2}
\newcommand{\dprod}{\mathbin{\star}}
\newcommand{\dprodeq}{\mathbin{\overline{\star}}}
\newcommand{\djoin}[2]{{#1} \mathbin{\Join} {#2}}
\newcommand{\dswap}[1]{#1^{\leftrightarrow}}
\newcommand{\dleq}{\mathrel{\preceq}}
\newcommand{\dgeq}{\mathrel{\succeq}}
\newcommand{\dlt}{\mathrel{\prec}}
\newcommand{\dgt}{\mathrel{\succ}}
\newcommand{\lap}[1]{\mathcal{L}_{#1}}

\newcommand{\E}[3]{\Exp_{{#1} \sim {#2}} [{#3}]}
\newcommand{\cE}[4]{\Exp_{{#1} \sim {#2}} [{#3} \mid {#4}]}
\newcommand{\sE}[2]{\Exp_{{#1}} [{#2}]}

\renewcommand{\P}[3]{\PrS_{{#1} \sim {#2}} [{#3}]}
\newcommand{\sP}[2]{\PrS_{#1} [{#2}]}
\newcommand{\iP}[1]{\PrS [{#1}]}

\newcommand{\mass}[1]{|{#1}|}
\DeclareMathOperator{\supp}{supp}

\newcommand{\wtn}[2]{\langle {#1, #2} \rangle}

\newcommand{\dcoupled}[3]
  {{#1} \mathrel{\blacktriangleleft} \langle {#2} \mathrel{\&} {#3} \rangle}

\newcommand{\dlifted}[4]
  {{#1} \mathrel{\blacktriangleleft_{#2}} \langle {#3} \mathrel{\&} {#4} \rangle}

\newcommand{\dacoupled}[6]
  {\langle {#1, #2} \rangle \mathrel{\blacktriangleleft}_{#3,#4}
     \langle {#5} \mathrel{\&} {#6} \rangle}

\newcommand{\dalifted}[7]
  {\langle {#1, #2} \rangle \mathrel{\blacktriangleleft_{#3,#4}^{#5}}
     \langle {#6} \mathrel{\&} {#7} \rangle}

% --------------------------------------------------------------------
% Assertions

\newcommand{\pre}{\phi}
\newcommand{\post}{\psi}

% --------------------------------------------------------------------
% Sid'ed

\def\ms{\mspace{-1.5mu}}

\newcommand{\lmark}{{\scriptscriptstyle \vartriangleleft}}
\newcommand{\rmark}{{\scriptscriptstyle \vartriangleright}}
\newcommand{\pmark}{{\scriptscriptstyle \Join}}

\newcommand{\lside}{_{\ms\lmark}}
\newcommand{\rside}{_{\ms\rmark}}

\newcommand{\LPVars}[1][]{\PVars[#1]^{\ms\lmark}}
\newcommand{\RPVars}[1][]{\PVars[#1]^{\ms\rmark}}
\newcommand{\PPVars}[1][]{\PVars[#1]^{\ms\pmark}}

\newcommand{\LSExpr}[1][]{\SExpr[#1]^{\ms\lmark}}
\newcommand{\RSExpr}[1][]{\SExpr[#1]^{\ms\rmark}}
\newcommand{\PSExpr}[1][]{\SExpr[#1]^{\ms\pmark}}

% --------------------------------------------------------------------
% Semantics

\newcommand{\sem}[1]{{\llbracket {#1} \rrbracket}}
\newcommand{\semop}[1]{{\overline{#1}}}

% --------------------------------------------------------------------
% Logics

\newcommand{\lift}[1]{\mathrel{#1^\sharp}}
\newcommand{\alifttoplas}[2]{\mathrel{#1^{(1)}_{#2}}}
\newcommand{\alifticalp}[2]{\mathrel{#1^{(2)}_{#2}}}
\newcommand{\aliftnew}[2]{\mathrel{#1^{(\star)}_{#2}}}

\newcommand{\symlifttoplas}[2]{\mathrel{\overline{#1}^{(1)}_{#2}}}
\newcommand{\symliftnew}[2]{\mathrel{\overline{#1}^{(\star)}_{#2}}}

% --------------------------------------------------------------------
% Refs

% \iflipics
% \theoremstyle{plain}
% \newtheorem{conj}[theorem]{Conjecture}
% \theoremstyle{definition}
% \newtheorem{defprop}[theorem]{Defn.-Lemma}
% \newtheorem{nottion}[theorem]{Notation}
% \else
% \theoremstyle{plain}
% \newtheorem{theorem}{Theorem}
% \newtheorem{lemma}[theorem]{Lemma}
% \newtheorem{corollary}[theorem]{Corollary}
% \newtheorem{conj}[theorem]{Conjecture}
% \theoremstyle{definition}
% \newtheorem{definition}[theorem]{Definition}
% \newtheorem{defprop}[theorem]{Defn.-Lemma}
% \newtheorem{example}[theorem]{Example}
% \newtheorem{nottion}[theorem]{Notation}
% \theoremstyle{remark}
% \newtheorem*{remark}{Remark}
% \fi

\usepackage{hyperref}

\iflipics
\else
\usepackage[capitalise]{cleveref}

\crefname{section}{section}{sections}
\Crefname{section}{Section}{Sections}

\crefname{lemma}{lemma}{lemmas}
\Crefname{lemma}{Lemma}{Lemmas}

\crefname{theorem}{theorem}{theorems}
\Crefname{theorem}{Theorem}{Theorems}

\crefname{definition}{definition}{definitions}
\Crefname{definition}{Definition}{Definitions}
\fi

\renewcommand{\epsilon}{\varepsilon}

%%% Local Variables:
%%% mode: latex
%%% TeX-master: "main"
%%% End:


\title{Optimal Algorithms for Multiwinner Elections and the Chamberlin-Courant Rule}

\newcommand*\samethanks[1][\value{footnote}]{\footnotemark[#1]}
\author{Kamesh Munagala\thanks{Computer Science Department, Duke University. Email: \texttt{kamesh@cs.duke.edu}, \texttt{zeyu.shen@duke.edu}, \texttt{knwang@cs.duke.edu}.} \and Zeyu Shen\samethanks[1] \and Kangning Wang\samethanks[1]}
\date{}

\begin{document}

\maketitle

\begin{abstract}
We consider the algorithmic question of choosing a subset of candidates of a given size $k$ from a set of $m$ candidates, with knowledge of voters' ordinal rankings over all candidates. We consider the well-known and classic scoring rule for achieving diverse representation: the Chamberlin-Courant (CC) or $1$-Borda rule, where the score of a committee is the average over the voters, of the rank of the best candidate in the committee for that voter; and its generalization to the average of the top $s$ best candidates, called the $s$-Borda rule. 

Our first result is an improved analysis of the natural and well-studied greedy heuristic. We show that greedy achieves a $\left(1 - \frac{2}{k+1}\right)$-approximation to the maximization (or satisfaction) version of CC rule, and a $\left(1 - \frac{2s}{k+1}\right)$-approximation to the $s$-Borda score. This significantly improves the existing submodularity-based analysis of the greedy algorithm that only shows a $(1-1/e)$-approximation. Our result also improves on the best known approximation algorithm for this problem. We achieve this result by showing that the average dissatisfaction score for the greedy algorithm is at most $2\frac{m+1}{k+1}$ for the CC rule, and at most $2s^2 \frac{m+1}{k+1}$ for $s$-Borda. We show these dissatisfaction score bounds are tight up to constants, and even the constant factor of $2$ in the case of the CC rule is almost tight.

For the dissatisfaction (or minimization) version of the problem, it is known that the average dissatisfaction score of the best committee cannot be approximated in polynomial time to within any constant factor when $s$ is a constant (under standard computational complexity assumptions). As our next result, we strengthen this to show that the score of $\frac{m+1}{k+1}$ can be viewed as an \emph{optimal benchmark} for the CC rule, in the sense that it is essentially the best achievable score of any polynomial-time algorithm even when the optimal score is a polynomial factor smaller. We show that another well-studied algorithm for this problem, called the Banzhaf rule, attains this benchmark. 

We finally show that for the $s$-Borda rule, when the optimal value is small, these algorithms can be improved by a factor of $\tilde \Omega(\sqrt{s})$ via LP rounding. Our upper and lower bounds are a significant improvement over previous results, and taken together, not only enable us to perform a finer comparison of greedy algorithms for these problems, but also provide analytic justification for using such algorithms in practice.
\end{abstract}

% Paper body
% \leavevmode
% \\
% \\
% \\
% \\
% \\
\section{Introduction}
\label{introduction}

AutoML is the process by which machine learning models are built automatically for a new dataset. Given a dataset, AutoML systems perform a search over valid data transformations and learners, along with hyper-parameter optimization for each learner~\cite{VolcanoML}. Choosing the transformations and learners over which to search is our focus.
A significant number of systems mine from prior runs of pipelines over a set of datasets to choose transformers and learners that are effective with different types of datasets (e.g. \cite{NEURIPS2018_b59a51a3}, \cite{10.14778/3415478.3415542}, \cite{autosklearn}). Thus, they build a database by actually running different pipelines with a diverse set of datasets to estimate the accuracy of potential pipelines. Hence, they can be used to effectively reduce the search space. A new dataset, based on a set of features (meta-features) is then matched to this database to find the most plausible candidates for both learner selection and hyper-parameter tuning. This process of choosing starting points in the search space is called meta-learning for the cold start problem.  

Other meta-learning approaches include mining existing data science code and their associated datasets to learn from human expertise. The AL~\cite{al} system mined existing Kaggle notebooks using dynamic analysis, i.e., actually running the scripts, and showed that such a system has promise.  However, this meta-learning approach does not scale because it is onerous to execute a large number of pipeline scripts on datasets, preprocessing datasets is never trivial, and older scripts cease to run at all as software evolves. It is not surprising that AL therefore performed dynamic analysis on just nine datasets.

Our system, {\sysname}, provides a scalable meta-learning approach to leverage human expertise, using static analysis to mine pipelines from large repositories of scripts. Static analysis has the advantage of scaling to thousands or millions of scripts \cite{graph4code} easily, but lacks the performance data gathered by dynamic analysis. The {\sysname} meta-learning approach guides the learning process by a scalable dataset similarity search, based on dataset embeddings, to find the most similar datasets and the semantics of ML pipelines applied on them.  Many existing systems, such as Auto-Sklearn \cite{autosklearn} and AL \cite{al}, compute a set of meta-features for each dataset. We developed a deep neural network model to generate embeddings at the granularity of a dataset, e.g., a table or CSV file, to capture similarity at the level of an entire dataset rather than relying on a set of meta-features.
 
Because we use static analysis to capture the semantics of the meta-learning process, we have no mechanism to choose the \textbf{best} pipeline from many seen pipelines, unlike the dynamic execution case where one can rely on runtime to choose the best performing pipeline.  Observing that pipelines are basically workflow graphs, we use graph generator neural models to succinctly capture the statically-observed pipelines for a single dataset. In {\sysname}, we formulate learner selection as a graph generation problem to predict optimized pipelines based on pipelines seen in actual notebooks.

%. This formulation enables {\sysname} for effective pruning of the AutoML search space to predict optimized pipelines based on pipelines seen in actual notebooks.}
%We note that increasingly, state-of-the-art performance in AutoML systems is being generated by more complex pipelines such as Directed Acyclic Graphs (DAGs) \cite{piper} rather than the linear pipelines used in earlier systems.  
 
{\sysname} does learner and transformation selection, and hence is a component of an AutoML systems. To evaluate this component, we integrated it into two existing AutoML systems, FLAML \cite{flaml} and Auto-Sklearn \cite{autosklearn}.  
% We evaluate each system with and without {\sysname}.  
We chose FLAML because it does not yet have any meta-learning component for the cold start problem and instead allows user selection of learners and transformers. The authors of FLAML explicitly pointed to the fact that FLAML might benefit from a meta-learning component and pointed to it as a possibility for future work. For FLAML, if mining historical pipelines provides an advantage, we should improve its performance. We also picked Auto-Sklearn as it does have a learner selection component based on meta-features, as described earlier~\cite{autosklearn2}. For Auto-Sklearn, we should at least match performance if our static mining of pipelines can match their extensive database. For context, we also compared {\sysname} with the recent VolcanoML~\cite{VolcanoML}, which provides an efficient decomposition and execution strategy for the AutoML search space. In contrast, {\sysname} prunes the search space using our meta-learning model to perform hyperparameter optimization only for the most promising candidates. 

The contributions of this paper are the following:
\begin{itemize}
    \item Section ~\ref{sec:mining} defines a scalable meta-learning approach based on representation learning of mined ML pipeline semantics and datasets for over 100 datasets and ~11K Python scripts.  
    \newline
    \item Sections~\ref{sec:kgpipGen} formulates AutoML pipeline generation as a graph generation problem. {\sysname} predicts efficiently an optimized ML pipeline for an unseen dataset based on our meta-learning model.  To the best of our knowledge, {\sysname} is the first approach to formulate  AutoML pipeline generation in such a way.
    \newline
    \item Section~\ref{sec:eval} presents a comprehensive evaluation using a large collection of 121 datasets from major AutoML benchmarks and Kaggle. Our experimental results show that {\sysname} outperforms all existing AutoML systems and achieves state-of-the-art results on the majority of these datasets. {\sysname} significantly improves the performance of both FLAML and Auto-Sklearn in classification and regression tasks. We also outperformed AL in 75 out of 77 datasets and VolcanoML in 75  out of 121 datasets, including 44 datasets used only by VolcanoML~\cite{VolcanoML}.  On average, {\sysname} achieves scores that are statistically better than the means of all other systems. 
\end{itemize}


%This approach does not need to apply cleaning or transformation methods to handle different variances among datasets. Moreover, we do not need to deal with complex analysis, such as dynamic code analysis. Thus, our approach proved to be scalable, as discussed in Sections~\ref{sec:mining}.
%!TEX root = hopfwright.tex
%

In this section we systematically recast the Hopf bifurcation problem in Fourier space. 
We introduce appropriate scalings, sequence spaces of Fourier coefficients and convenient operators on these spaces. 
To study Equation~\eqref{eq:FourierSequenceEquation} we consider Fourier sequences $ \{a_k\}$ and fix a Banach space in which these sequences reside. It is indispensable for our analysis that this space have an algebraic structure. 
The Wiener algebra of absolutely summable Fourier series is a natural candidate, which we use with minor modifications. 
In numerical applications, weighted sequence spaces with algebraic and geometric decay have been used to great effect to study periodic solutions which are $C^k$ and analytic, respectively~\cite{lessard2010recent,hungria2016rigorous}. 
Although it follows from Lemma~\ref{l:analytic} that the Fourier coefficients of any solution decay exponentially, we choose to work in a space of less regularity. 
The reason is that by working in a space with less regularity, we are better able to connect our results with the global estimates in \cite{neumaier2014global}, see Theorem~\ref{thm:UniqunessNbd2}.


%
%
%\begin{remark}
%	Although it follows from Lemma~\ref{l:analytic} that the Fourier coefficients of any solution decay exponentially, we choose to work in a space of less regularity, namely summable Fourier coefficients. This allows us to draw SOME MORE INTERESTING CONCLUSION LATER.
%	EXPLAIN WHY WE CHOOSE A NORM WITH ALMOST NO DECAY!
%	% of s Periodic solutions to Wright's equation are known to be real analytic and so their  Fourier coefficients must decay geometrically [Nussbaum].
%	% We do not use such a strong result;  any periodic solution must be continuously differentiable, by which it follows that $ \sum | c_k| < \infty$.
%\end{remark}


\begin{remark}\label{r:a0}
There is considerable redundancy in Equation~\eqref{eq:FourierSequenceEquation}. First, since we are considering real-valued solutions $y$, we assume $\c_{-k}$ is the complex conjugate of $\c_k$. This symmetry implies it suffices to consider Equation~\eqref{eq:FourierSequenceEquation} for $k \geq 0$.
Second, we may effectively ignore the zeroth Fourier coefficient of any periodic solution \cite{jones1962existence}, since it is necessarily equal to $0$. 
%In \cite{jones1962existence}, it is shown that if $y \not\equiv -1$ is a periodic solution of~\eqref{eq:Wright} with frequency $\omega$, then $ \int_0^{2\pi/\omega} y(t) dt =0$. 
		The self contained argument is as follows. 
		As mentioned in the introduction, any periodic solution to Wright's equation must satisfy $ y(t) > -1$ for all $t$. 
	By dividing Equation~\eqref{eq:Wright} by $(1+y(t))$, which never vanishes, we obtain
	\[
	\frac{d}{dt} \log (1 + y(t)) = - \alpha y(t-1).
	\]  
	Integrating over one period $L$ we derive the condition 
	$0=\int_0^L y(t) dt $.
	Hence $a_0=0$ for any periodic solution. 
	It will be shown in Theorem~\ref{thm:FourierEquivalence1} that a related argument implies that we do not need to consider Equation~\eqref{eq:FourierSequenceEquation} for $k=0$.
\end{remark}

%%%
%%%
%%%\begin{remark}\label{r:c0} 
%%%In \cite{jones1962existence}, it is shown that if $y \not\equiv -1$ is a periodic solution of~\eqref{eq:Wright} with frequency $\omega$, then $ \int_0^{2\pi/\omega} y(t) dt =0$. 
%%%PERHAPS TOO MUCH DETAIL HERE. The self contained argument is as follows.
%%%If $y \not\equiv -1$ then $y(t) \neq -1$ for all $t$, since if $y(t_0)=-1$ for some $t_0 \in \R$ then $y'(t_0)=0$ by~\eqref{eq:Wright} and in fact by differentiating~\eqref{eq:Wright} repeatedly one obtains that all derivatives of $y$ vanish at $t_0$. Hence $y \equiv -1$ by Lemma~\ref{l:analytic}, a contradiction. Now divide~\eqref{eq:Wright} by $(1+y(t))$, which never vanishes, to obtain
%%%\[
%%%  \frac{d}{dt} \log |1 + y(t)| = - \alpha y(t-1).
%%%\]  
%%%Integrating over one period we obtain $\int_0^L y(t) dt =0$.
%%%\end{remark}



%Furthermore, the condition that $y(t)$ is real forces $\c_{-k} = \overline{\c}_{k}$.  
%
We define the spaces of absolutely summable Fourier series
\begin{alignat*}{1}
	\ell^1 &:= \left\{ \{ \c_k \}_{k \geq 1} : 
    \sum_{k \geq 1} | \c_k| < \infty  \right\} , \\
	\ell^1_\bi &:= \left\{ \{ \c_k \}_{k \in \Z} : 
    \sum_{k \in \Z} | \c_k| < \infty  \right\} .
\end{alignat*} 
We identify any semi-infinite sequence $ \{ \c_k \}_{k \geq 1} \in \ell^1$ with the bi-infinite sequence $ \{ \c_k \}_{k \in \Z} \in \ell^1_\bi$ via the conventions (see Remark~\ref{r:a0})
\begin{equation}
  \c_0=0 \qquad\text{ and }\qquad \c_{-k} = \c_{k}^*. 
\end{equation}
In other word, we identify $\ell^1$ with the set
\begin{equation*}
   \ell^1_\sym := \left\{ \c \in \ell^1_\bi : 
	\c_0=0,~\c_{-k}=\c_k^* \right\} .
\end{equation*}
On $\ell^1$ we introduce the norm
\begin{equation}\label{e:lnorm}
  \| \c \| = \| \c \|_{\ell^1} := 2 \sum_{k = 1}^\infty |\c_k|.
\end{equation}
The factor $2$ in this norm is chosen to have a Banach algebra estimate.
Indeed, for $\c, \tilde{\c} \in \ell^1 \cong \ell^1_\sym$ we define
the discrete convolution 
\[
\left[ \c * \tilde{\c} \right]_k = \sum_{\substack{k_1,k_2\in\Z\\ k_1 + k_2 = k}} \c_{k_1} \tilde{\c}_{k_2} .
\]
Although $[\c*\tilde{\c}]_0$ does not necessarily vanish, we have $\{\c*\tilde{\c}\}_{k \geq 1} \in \ell^1 $ and 
\begin{equation*}
	\| \c*\tilde{\c} \| \leq \| \c \| \cdot  \| \tilde{\c} \| 
	\qquad\text{for all } \c , \tilde{\c} \in \ell^1, 
\end{equation*}
hence $\ell^1$ with norm~\eqref{e:lnorm} is a Banach algebra.

By Lemma~\ref{l:analytic} it is clear that any periodic solution of~\eqref{eq:Wright} has a well-defined Fourier series $\c \in \ell^1_\bi$. 
The next theorem shows that in order to study periodic orbits to Wright's equation we only need to study Equation~\eqref{eq:FourierSequenceEquation} 
for $k \geq 1$. For convenience we introduce the notation 
\[
G(\alpha,\omega,\c)_k=
( i \omega k + \alpha e^{ - i \omega k}) \c_k + \alpha \sum_{k_1 + k_2 = k} e^{- i \omega k_1} \c_{k_1} \c_{k_2} \qquad \text{for } k \in \N.
\]
We note that we may interpret the trivial solution $y(t)\equiv 0$ as a periodic solution of arbitrary period.
\begin{theorem}
\label{thm:FourierEquivalence1}
Let $\alpha>0$ and $\omega>0$.
If $\c \in \ell^1 \cong \ell^1_{\sym}$ solves
$G(\alpha,\omega,\c)_k =0$  for all $k \geq 1$,
then $y(t)$ given by~\eqref{eq:FourierEquation} is a periodic solution of~\eqref{eq:Wright} with period~$2\pi/\omega$.
Vice versa, if $y(t)$ is a periodic solution of~\eqref{eq:Wright} with period~$2\pi/\omega$ then its Fourier coefficients $\c \in \ell^1_\bi$ lie in $\ell^1_\sym \cong \ell^1$ and solve $G(\alpha,\omega,\c)_k =0$ for all $k \geq 1$.
\end{theorem}

\begin{proof}	
	If $y(t)$ is a periodic solution of~\eqref{eq:Wright} then it is real analytic by Lemma~\ref{l:analytic}, hence its Fourier series $\c$ is well-defined and $\c \in \ell^1_{\sym}$ by Remark~\ref{r:a0}.
	Plugging the Fourier series~\eqref{eq:FourierEquation} into~\eqref{eq:Wright} one easily derives that $\c$ solves~\eqref{eq:FourierSequenceEquation} for all $k \geq 1$.

To prove the reverse implication, assume that $\c \in \ell^1_\sym$ solves
Equation~\eqref{eq:FourierSequenceEquation} for all $k \geq 1$. Since $\c_{-k}
= \c_k^*$, Equation \eqref{eq:FourierSequenceEquation} is also satisfied for
all $k \leq -1$. It follows from the Banach algebra property and
\eqref{eq:FourierSequenceEquation} that $\{k \c_k\}_{k \in \Z} \in \ell^1_\bi$,
hence $y$, given by~\eqref{eq:FourierEquation}, is continuously differentiable.
% (and by bootstrapping one infers that $\{k^m c_k \} \in \ell^1_\bi$, 
% hence $y \in C^m$ for any $m \geq 1$).
	Since~\eqref{eq:FourierSequenceEquation} is satisfied for all $k \in \Z \setminus \{0\}$ (but not necessarily for $k=0$) one may perform the inverse Fourier transform on~\eqref{eq:FourierSequenceEquation} to conclude that
	$y$ satisfies the delay equation 
\begin{equation}\label{eq:delaywithK}
   	y'(t) = - \alpha y(t-1) [ 1 + y(t)] + C
\end{equation}
	for some constant $C \in \R$. 
   Finally, to prove that $C=0$ we argue by contradiction.
   Suppose $C \neq 0$. Then $y(t) \neq -1$ for all $t$.
   Namely, at any point where $y(t_0) =-1$ one would have $y'(t_0) = C$
   which has fixed sign,   hence it would follow that $y$ is not periodic
   ($y$ would not be able to cross $-1$ in the opposite direction, 
   preventing $y$  from being periodic).  
  We may thus divide~\eqref{eq:delaywithK} through by $1 + y(t)$ and obtain 
\begin{equation*}
	\frac{d}{dt} \log | 1 + y(t) | = - \alpha y(t-1) + \frac{C}{1+y(t)} .
\end{equation*}
	By integrating both sides of the equation over one period $L$ and by using that $\c_0=0$, we 
	obtain
	\[
	 C \int_0^L \frac{1}{1+y(t)} dt =0.
	\]
	Since the integrand is either strictly negative or strictly positive, this implies that $C=0$. Hence~\eqref{eq:delaywithK} reduces to~\eqref{eq:Wright},
	and $y$ satisfies Wright's equation. 
\end{proof}






To efficiently study Equation~\eqref{eq:FourierSequenceEquation}, we introduce the following linear operators on $ \ell^1$:
\begin{alignat*}{1}
   [K \c ]_k &:= k^{-1} \c_k  , \\ 
   [ U_\omega \c ]_k &:= e^{-i k \omega} \c_k  .
\end{alignat*}
The map $K$ is a compact operator, and it has a densely defined inverse $K^{-1}$. The domain of $K^{-1}$ is denoted by
\[
  \ell^K := \{ \c \in \ell^1 : K^{-1} \c \in \ell^1 \}.  
\]
The map $U_{\omega}$ is a unitary operator on $\ell^1$, but
it is discontinuous in $\omega$. 
With this notation, Theorem~\ref{thm:FourierEquivalence1} implies that our problem of finding a SOPS to~\eqref{eq:Wright} is equivalent to finding an $\c \in \ell^1$ such that
\begin{equation}
\label{e:defG}
  G(\alpha,\omega,\c) :=
  \left( i \omega K^{-1} + \alpha U_\omega \right) \c + \alpha \left[U_\omega \, \c \right] * \c  = 0.
\end{equation}


%In order for the solutions of Equation \ref{eq:FHat} to be isolated we need to impose a phase condition. 
%If there is a sequence $ \{ c_k \} $ which satisfies  Equation \ref{eq:FHat}, then $ y( t + \tau) = \sum_{k \in \Z} c_k e^{ i k \omega (t + \tau)}$ satisfies Wright's equation at parameter $\alpha$. 
%Fix $ \tau = - Arg[c_1] / \omega$ so that $ c_1  e^{ i \omega \tau} $ is a nonnegative real number. 
%By Proposition \ref{thm:FourierEquivalence1} it follows that $\{ c'_k \} =  \{c_k e^{ i \omega k \tau }   \}$ is a solution to Equation \ref{eq:FHat}, and furthermore that $ c'_1 = \epsilon$ for some $ \epsilon \geq 0$. 


Periodic solutions are invariant under time translation: if $y(t)$ solves Wright's equation, then so does $ y(t+\tau)$ for any $\tau \in \R$. 
We remove this degeneracy by adding a phase condition. 
Without loss of generality, if $\c \in \ell^1$ solves Equation~\eqref{e:defG}, we may assume that $\c_1 = \epsilon$ for some 
\emph{real non-negative}~$\epsilon$:
\[
  \ell^1_{\epsilon} := \{\c \in \ell^1 : \c_1 = \epsilon \} 
  \qquad \text{where } \epsilon \in \R,  \epsilon \geq 0.
\]
In the rest of our analysis, we will split elements $\c \in \ell^1$ into two parts: $\c_1$ and $\{\c_{k}\}_{k \geq 2}$.  
We define the basis elements $\e_j \in \ell^1$ for $j=1,2,\dots$ as
\[
  [\e_j]_k = \begin{cases}
  1 & \text{if } k=j, \\
  0 & \text{if } k \neq j.
  \end{cases}
\]
We note that $\| \e_j \|=2$. 
Then we can decompose
% We define
% \[
%   \tilde{\epsilon} := (\epsilon,0,0,0,\dots) \in \ell^1
% \]
% and
% For clarity when referring to sequences $\{c_{k}\}_{k \geq 2}$, we make the following definition:
% \[
% \ell^1_0  := \{ \tc \in \ell^1 : \tc_1 = 0 \}.
% \]
% With the
any $\c \in \ell^1_\epsilon$ uniquely as
\begin{equation}\label{e:aepsc}
  \c= \epsilon \e_1 + \tc \qquad \text{with}\quad 
  \tc \in \ell^1_0 := \{ \tc \in \ell^1 : \tc_1 = 0 \}.
\end{equation}
We follow the classical approach in studying Hopf bifurcations and consider 
$\c_1 = \epsilon$ to be a parameter, and then find periodic solutions with Fourier modes in $\ell^1_{\epsilon}$.
This approach rewrites the function $G: \R^2 \times \ell^K \to \ell^1$ as a function $\tilde{F}_\epsilon : \R^2 \times \ell^K_0 \to \ell^1$, where 
we denote 
\[
\ell^K_0 := \ell^1_0 \cap \ell^K.
\]
% I AM ACTUALLY NOT SURE IF YOU WANT TO DEFINE THIS WITH RANGE IN $\ell^1$
% OR WITH DOMAIN IN $\ell^1_0$ ?? IT SEEMS TO DEPEND ON WHICH GLOBAL STATEMENT YOU WANT/NEED TO MAKE!?
\begin{definition}
We define the $\epsilon$-parameterized family of  functions $\tilde{F}_\epsilon: \R^2 \times \ell^K_0  \to \ell^1$ 
by 
\begin{equation}
\label{eq:fourieroperators}
\tilde{F}_{\epsilon}(\alpha,\omega, \tc) := 
\epsilon [i \omega + \alpha e^{-i \omega}] \e_1 + 
( i \omega K^{-1} + \alpha U_{\omega}) \tc + 
\epsilon^2 \alpha e^{-i \omega}  \e_2  +
\alpha \epsilon L_\omega \tc + 
\alpha  [ U_{\omega} \tc] * \tc ,
\end{equation}
where
$L_\omega : \ell^1_0 \to \ell^1$ is given by
\[
   L_{\omega} := \sigma^+( e^{- i \omega} I + U_{\omega}) + \sigma^-(e^{i \omega} I + U_{\omega}),
\]
with $I$ the identity and  $\sigma^\pm$ the shift operators on $\ell^1$:
\begin{alignat*}{2}
\left[ \sigma^- a \right]_k &:=  a_{k+1}  , \\
\left[ \sigma^+ a \right]_k &:=  a_{k-1}  &\qquad&\text{with the convention } \c_0=0.
\end{alignat*}
The operator $ L_\omega$ is discontinuous in $\omega$ and $ \| L_\omega \| \leq 4$. 
\end{definition} 

%The maps $ \sigma^{+}$ and $ \sigma^-$ are shift up and shift down operators respectively. 
We reformulate Theorem~\ref{thm:FourierEquivalence1}  in terms of the map  $\tilde{F}$. 
We note that it follows from Lemma~\ref{l:analytic} and 
%\marginpar{Reformulate}
%one's choice of  
Equation~\eqref{eq:FourierSequenceEquation}  
%or Equation ~\eqref{eq:fourieroperators},
that the Fourier coefficients of any periodic solution of~\eqref{eq:Wright} lie in $\ell^K$.
These observations are summarized in the following theorem.
\begin{theorem}
\label{thm:FourierEquivalence2}
	Let $ \epsilon \geq 0$,  $\tc \in \ell^K_0$, $\alpha>0$ and $ \omega >0$. 
	Define $y: \R\to \R$ as 
\begin{equation}\label{e:ytc}
	y(t) = 
	\epsilon \left( e^{i \omega t }  + e^{- i \omega t }\right) 
	+  \sum_{k = 2}^\infty   \tc_k e^{i \omega k t }  + \tc_k^* e^{- i \omega k t } .
\end{equation}
%	and suppose that $ y(t) > -1$. 
	Then $y(t)$ solves~\eqref{eq:Wright} if and only if $\tilde{F}_{\epsilon}( \alpha , \omega , \tc) = 0$. 
	Furthermore, up to time translation, any periodic solution of~\eqref{eq:Wright} with period $2\pi/\omega$ is described by a Fourier series of the form~\eqref{e:ytc} with $\epsilon \geq 0$ and $\tc \in \ell^K_0$.
\end{theorem}


%We note that for $\epsilon>0$ such solutions are truly periodic, while for $\epsilon=0$ a zero of $\tilde{F}_\epsilon$ may either correspond to a periodic solution or to the trivial solution $y(t) \equiv 0$. 



% \begin{proof}
%  By Proposition \ref{thm:FourierEquivalence1}, it suffices to show that $\tilde{F}(\alpha,\omega,c) =0$ is equivalent to Equation \ref{eq:FourierSequenceEquation} being satisfied for $k \geq 1$.
%  Since Equation \ref{eq:FourierSequenceEquation} is equivalent to Equation \ref{eq:FHat}, we expand  Equation \ref{eq:FHat} by writing $ \hat{c} = \hat{\epsilon } + c$  where $ \hat{\epsilon} := (\epsilon,0,0,\dots) \in \ell^1$ as below:
%  \begin{equation}
%  0=  \left( i \omega K^{-1} + \alpha U_\omega \right) (\hat{\epsilon}+ c) + \alpha \left[U_\omega \, (\hat{\epsilon}+ c) \right] * (\hat{\epsilon}+ c) \label{eq:Intial}
%  \end{equation}
%  The RHS of Equation \ref{eq:Intial} is $ \tilde{F}(\alpha,\omega,c)$, so the theorem is proved.
% \end{proof}



Since we want to analyze a Hopf bifurcation, we will want to solve $\tilde{F}_\epsilon = 0$ for small values of~$\epsilon$. 
However, at the bifurcation point, $ D \tilde{F}_0(\pp  ,\pp , 0)$ is not invertible.
In order for our asymptotic analysis to be non-degenerate,
we work with a rescaled version of the problem. To this end, for any $\epsilon >0$, we rescale both $\tc$ and $\tilde{F}$ as follows. Let $\tc = \epsilon c$ and 
\begin{equation}\label{e:changeofvariables}
  \tilde{F}_\epsilon (\alpha,\omega,\epsilon c) = \epsilon F_\epsilon (\alpha,\omega,c).
\end{equation}
For $\epsilon>0$ the problem then reduces to finding zeros of 
\begin{equation}
\label{eq:FDefinition}
	F_\epsilon(\alpha,\omega, c) := 
	[i \omega + \alpha e^{-i \omega}] \e_1 + 
	( i \omega K^{-1} + \alpha U_{\omega}) c + 
	\epsilon \alpha e^{-i \omega} \e_2  +
	\alpha \epsilon L_\omega c + 
	\alpha \epsilon [ U_{\omega} c] * c.
\end{equation}
We denote the triple $(\alpha,\omega,c) \in \R^2 \times \ell^1_0$ by $x$.
To pinpoint the components of $x$ we use the projection operators
\[
   \pi_\alpha x = \alpha, \quad \pi_\omega x = \omega, \quad 
  \pi_c x = c \qquad\text{for any } x=(\alpha,\omega,c).
\]

After the change of variables~\eqref{e:changeofvariables} we now have an invertible Jacobian $D F_0(\pp  ,\pp , 0)$ at the bifurcation point.
On the other hand, for $\epsilon=0$ the zero finding problems for $\tilde{F}_\epsilon$ and $F_\epsilon$ are not equivalent. 
However, it follows from the following lemma that any nontrivial periodic solution having $ \epsilon=0$ must have a relatively large size when $ \alpha $ and $ \omega $ are close to the bifurcation point. 

\begin{lemma}\label{lem:Cone}
	Fix $ \epsilon \geq 0$ and $\alpha,\omega >0$. 
	Let
	\[
	b_* :=  \frac{\omega}{\alpha} - \frac{1}{2} - \epsilon  \left(\frac{2}{3}+ \frac{1}{2}\sqrt{2 + 2 |\omega-\pp| } \right).
	\]
Assume that $b_*> \sqrt{2} \epsilon$. 
Define
% \begin{equation*}%\label{e:zstar}
% 	z^{\pm}_* :=b_* \pm \sqrt{(b_*)^2- \epsilon^2 } .
% \end{equation*}
% \note[J]{Proposed change to match Lemma E.4}
\begin{equation}\label{e:zstar}
z^{\pm}_* :=b_* \pm \sqrt{(b_*)^2- 2 \epsilon^2 } .
\end{equation}
If there exists a $\tc \in \ell^1_0$ such that $\tilde{F}_\epsilon(\alpha, \omega,\tc) = 0$, then \\
\mbox{}\quad\textup{(a)} either $ \|\tc\| \leq  z_*^-$ or $ \|\tc\| \geq z_*^+  $.\\
\mbox{}\quad\textup{(b)} 
$ \| K^{-1} \tc \| \leq (2\epsilon^2+ \|\tc\|^2) / b_*$. 
\end{lemma}
\begin{proof}
	The proof follows from Lemmas~\ref{lem:gamma} and~\ref{lem:thecone} in Appendix~\ref{appendix:aprioribounds}, combined with the observation that
$\frac{\omega}{\alpha} - \gamma \geq b_*$,
% \[
%   \frac{\omega}{\alpha} - \gamma \geq b_*
%  \qquad\text{for all }
% | \alpha - \pp| \leq r_\alpha \text{ and } 
%   | \omega - \pp| \leq r_\omega.
% \]
with $\gamma$ as defined in Lemma~\ref{lem:gamma}.
\end{proof}

\begin{remark}\label{r:smalleps}
We note that for $\alpha < 2\omega$
\begin{alignat*}{1}
z^+_* &\geq   \frac{2 \omega - \alpha}{\alpha} 
- \epsilon \left(4/3+\sqrt{2 + 2 |\omega-\pp| } \, \right) + \cO(\epsilon^2)
\\[1mm]
z^-_* & \leq   \cO(\epsilon^2)
\end{alignat*}
for small $\epsilon$. 
Hence Lemma~\ref{lem:Cone} implies that for values of $(\alpha,\omega)$ near $(\pp,\pp)$ any solution has either $\|\tc\|$ of order 1 or $\|\tc\| =  \cO(\epsilon^2)$. 
The asymptotically small term bounding $z_*^-$ is explicitly calculated in Lemma~\ref{lem:ZminusBound}. 
A related consequence is that for $\epsilon=0$ there are no nontrivial solutions 
of $\tilde{F}_0(\alpha,\omega,\tc)=0$ with 
$\| \tc \| < \frac{2 \omega - \alpha}{\alpha} $. 
\end{remark}

\begin{remark}\label{r:rhobound}
In Section~\ref{s:contraction} we will work on subsets of $\ell^K_0$ of the form
\[
  \ell_\rho := \{ c \in \ell^K_0 : \|K^{-1} c\| \leq \rho \} .
\]
Part (b) of Lemma~\ref{lem:Cone} will be used in Section~\ref{s:global} to guarantee that we are not missing any solutions by considering $\ell_\rho$ (for some specific choice of $\rho$) rather than the full space $\ell^K_0$.
In particular, we infer from Remark~\ref{r:smalleps} that  small solutions (meaning roughly that $\|\tc\| \to 0$ as $\epsilon \to 0$)
satisfy $\| K^{-1} \tc \| = \cO(\epsilon^2)$.
\end{remark}

The following theorem guarantees that near the bifurcation point the problem of finding all periodic solutions is equivalent to considering the rescaled problem $F_\epsilon(\alpha,\omega,c)=0$.
\begin{theorem}
\label{thm:FourierEquivalence3}
\textup{(a)} Let $ \epsilon > 0$,  $c \in \ell^K_0$, $\alpha>0$ and $ \omega >0$. 
	Define $y: \R\to \R$ as 
\begin{equation}\label{e:yc}
	y(t) = 
	\epsilon \left( e^{i \omega t }  + e^{- i \omega t }\right) 
	+ \epsilon  \sum_{k = 2}^\infty   c_k e^{i \omega k t }  + c_k^* e^{- i \omega k t } .
\end{equation}
%	and suppose that $ y(t) > -1$. 
	Then $y(t)$ solves~\eqref{eq:Wright} if and only if $F_{\epsilon}( \alpha , \omega , c) = 0$.\\
\textup{(b)}
Let $y(t) \not\equiv 0$ be a periodic solution of~\eqref{eq:Wright} of period $2\pi/\omega$
 with Fourier coefficients $\c$.
Suppose $\alpha < 2\omega$ and $\| \c \| < \frac{2 \omega - \alpha}{\alpha} $.
Then, up to time translation, $y(t)$ is described by a Fourier series of the form~\eqref{e:yc} with $\epsilon > 0$ and $c \in \ell^K_0$.
\end{theorem}

\begin{proof}
Part (a) follows directly from Theorem~\ref{thm:FourierEquivalence2} and the  change of variables~\eqref{e:changeofvariables}.
To prove part (b) we need to exclude the possibility that there is a nontrivial solution with $\epsilon=0$. The asserted bound on the ratio of $\alpha$ and $\omega$ guarantees, by Lemma~\ref{lem:Cone} (see also Remark~\ref{r:smalleps}), that indeed $\epsilon>0$ for any nontrivial solution. 
\end{proof}

We note that in practice (see Section~\ref{s:global}) a bound on $\| \c \|$ is derived from a bound on $y$ or $y'$ using Parseval's identity.

\begin{remark}\label{r:cone}
It follows from Theorem~\ref{thm:FourierEquivalence3} and Remark~\ref{r:smalleps} that for values of $(\alpha,\omega)$ near $(\pp,\pp)$ any reasonably bounded solution satisfies $\| c\| =  O(\epsilon)$ as well as $\|K^{-1} c \| = O(\epsilon)$ asymptotically (as $\epsilon \to 0$).
These bounds will be made explicit (and non-asymptotic) for specific choices of the parameters in Section~\ref{s:global}.
\end{remark}

% We are able to rule out such large amplitude solutions using global estimates such as those in \cite{neumaier2014global}.
% Hence, near the bifurcation point, the problem of describing periodic solutions of~\eqref{eq:Wright} reduces to studying the family of zeros finding problems $F_\epsilon=0$.





%Specifically, if a solution having $ \epsilon = 0$ does in fact correspond to a nontrivial periodic solution and $\alpha  < 2\omega $, then $ \| \tilde{c} \| > 2 \omega \alpha^{-1} -1$. 
%%PERHAPS THIS NEEDS A FORMULATION AS A THEOREM AS WELL?
%%IN OTHER WORDS: ARE WE SURE WE HAVE FOUND ALL ZEROS OF $\tilde{F}_0$, I.E. ALL SOLUTIONS WITH $\epsilon=0$ NEAR THE BIFURCATION POINT? AFTER RESCALING THESE ARE INVISIBLE?
%%THERE SHOULD BE A STATEMENT ABOUT THIS SOMEWHERE! EITHER HERE OR SOME





We finish this section by defining a curve of approximate zeros $\bx_\epsilon$ of $F_\epsilon$ 
(see \cite{chow1977integral,hassard1981theory}). 
%(see \cite{chow1977integral,morris1976perturbative,hassard1981theory}). 


\begin{definition}\label{def:xepsilon}
Let
\begin{alignat*}{1}
	\balpha_\epsilon &:= \pp + \tfrac{\epsilon^2}{5} ( \tfrac{3\pi}{2} -1)  \\
	\bomega_\epsilon &:= \pp -  \tfrac{\epsilon^2}{5} \\
	\bc_\epsilon 	 &:= \left(\tfrac{2 - i}{5}\right) \epsilon \,  \e_2 \,.
\end{alignat*}
We define the approximate solution 
$ \bx_\epsilon := \left( \balpha_\epsilon , \bomega_\epsilon  , \bc_\epsilon \right)$
for all $\epsilon \geq 0$.
\end{definition}

We leave it to the reader to verify that both 
 $F_\epsilon(\pp,\pp,\bc_{\epsilon})=\cO(\epsilon^2)$ and $F_\epsilon(\bx_\epsilon)=\cO(\epsilon^2)$.
%%%	
%%%	
%%%	}{Better like this?}
%%%\annote[J]{ $F_\epsilon(\bx_0)=\cO(\epsilon^2)$ and $F_\epsilon(\bx_\epsilon)=\cO(\epsilon^2)$.}{I think we'd still need the $ \bar{c}_\epsilon$ term in $\bar{x}_0$ to be of order $ \epsilon$.}
%%%\remove[JB]{We show in Proposition A.1
%%%%\ref{prop:ApproximateSolutionWorks} 
%%% that any $ x \in \R^2 \times \ell^1_0$ which is $ \cO(\epsilon^2)$ close to $ \bar{x}_\epsilon $ will yield the estimate $F_\epsilon(x) = \cO(\epsilon^2)$.
%%%Hence choosing $\{ \pp , \pp, \bar{c}_\epsilon\}$ as our approximate solution would also have been a natural choice for performing an $\cO(\epsilon^2)$ analysis and would have simplified several of our calculations.
%%%However,} 
%%%
We choose to use the more accurate approximation 
for the $ \alpha$ and $ \omega $ components to improve our final quantitative results. 














%
% Values for $ (\alpha, \omega,c)$ which approximately solve $\tilde{F}(\alpha,\omega,c) = 0$  are computed in  \cite{chow1977integral,morris1976perturbative,hassard1981theory} and are as follows:
%  \begin{eqnarray}
%  \tilde{\alpha}( \epsilon) &:=& \pi /2 + \tfrac{\epsilon^2}{5} ( \tfrac{3\pi}{2} -1) \nonumber \\
%  \tilde{\omega}( \epsilon) &:=& \pi /2 -  \tfrac{\epsilon^2}{5} \label{eq:ScaleApprox} \\
%  \tc(\epsilon) 	  &:=& \{ \left(\tfrac{2 - i}{5}\right)  \epsilon^2 , 0,0, \dots \} \nonumber
%  \end{eqnarray}
% In Appendix \ref{sec:OperatorNorms} we illustrate an alternative method for deriving this approximation.
%
%
%
%
% We want to solve $ \tilde{F}(\alpha , \omega, \hat{c}) =0$ for small values of $ \epsilon$.
% However $ D \tilde{F}(\alpha , \omega , c)$ is not invertible at $ ( \pp , \pp , 0)$ when $ \epsilon = 0$.
% In order for our asymptotic analysis to be non-degenerate, we need to make the change of variables $ c \mapsto \epsilon c$.
% Under this change of variables, we define the function $ F$ below so that $ \tilde{F}(\alpha , \omega , \epsilon c) =\epsilon  F( \alpha , \omega , c)$.
%
%
%
% \begin{definition}
% Construct an $\epsilon$-parameterized family of densely defined functions  $F : \R^2 \oplus \ell^1 / \C \to \ell^1$ by:
% \begin{equation}
% \label{eq:FDefinition}
% 	F(\alpha,\omega, c) :=
% 	[i \omega + \alpha e^{-i \omega}]_1 +
% 	( i \omega K^{-1} + \alpha U_{\omega}) c +
% 	[\epsilon \alpha e^{-i \omega}]_2  +
% 	\alpha \epsilon L_\omega c +
% 	\alpha \epsilon [ U_{\omega} c] * c.
% \end{equation}
% \end{definition}

%%
%%
%%\begin{corollary}
%%	\label{thm:FourierEquivalence3}
%%	Fix $ \epsilon > 0$, and $ c \in \ell^1 / \C $, and $ \omega >0$. Define $y: \R\to \R$ as 
%%	\[
%%	y(t) = 
%%	\epsilon \left( e^{i \omega t }  + e^{- i \omega t }\right) 
%%	+  \epsilon  \left( \sum_{k = 2}^\infty   c_k e^{i \omega k t }  + \overline{c}_k e^{- i \omega k t } \right) 
%%	\]
%%	and suppose that $ y(t) > -1$. 
%%	Then $y(t)$ solves Wright's equation at parameter $ \alpha > 0 $ if and only if $ F( \alpha , \omega , c) = 0$ at parameter $ \epsilon$. 
%%	
%%	
%%	
%%\end{corollary}
%%
%%
%%\begin{proof}
%%	Since $ \tilde{F}(\alpha,\omega, \epsilon c) = \epsilon F( \alpha , \omega , c)$, the result follows from Theorem \ref{thm:FourierEquivalence2}.
%%\end{proof}

% If we can find $(\alpha , \omega, c)$ for which $ F( \alpha , \omega,c)=0$ at parameter $\epsilon$, then $ \tilde{F}(\alpha ,\omega, c)=0$.
% By Theorem \ref{thm:FourierEquivalence2} this amounts to finding a periodic solution to Wright's equation.
% Lastly, because we have performed the change of variables $ c \mapsto \epsilon c$, we need to  apply this change of variables to our approximate solution as well.
%
% \begin{definition}
% 	Define the approximate solution $ x( \epsilon) = \left\{ \alpha(\epsilon ) , \omega ( \epsilon ) , c(\epsilon) \right\}$ as below,  where $c(\epsilon) = \{ c_2( \epsilon) , 0 ,0 , \dots\} $.
% 	We may also write $ x_\epsilon = x(\epsilon) $.
% 	\begin{eqnarray}
% 	\alpha( \epsilon) &:=& \pi /2 + \tfrac{\epsilon^2}{5} ( \tfrac{3\pi}{2} -1) \nonumber \\
% 	\omega( \epsilon) &:=& \pi /2 -  \tfrac{\epsilon^2}{5} \label{eq:Approx} \\
% 	c_2(\epsilon) 	  &:=& \left(\tfrac{2 - i}{5}\right) \epsilon \nonumber
% 	\end{eqnarray}
%
% \end{definition}

\section{Analysis of \g{} for $1$-Borda}
\label{sec:greedy}
In this section, we analyze the performance of \g{}, evaluated with respect to the benchmark \rand{}. Throughout this section, we only consider the $1$-Borda score, i.e., $s = 1$. We first show an upper bound that $\g \leq 2 \cdot \rand$, and then present an almost-matching lower-bound instance where $\g > 1.962 \cdot \rand$.

\subsection{Upper Bound}
\label{sec:greedy_ub}
Now we show $\g \leq 2 \cdot \rand$ as an upper bound. We first present the following lemma, which gives a lower bound on the improvement at each iteration.
\begin{lemma}
Let $T_t$ and $T_{t + 1}$ be the set of candidates produced by \g{} in the $t^{\text{th}}$ and $(t + 1)^{\text{st}}$ iterations, and $r_{\V}(T_t)$, $r_{\V}(T_{t + 1})$ be their respective score. We have:
\[
r_{\V}(T_t) - r_{\V}(T_{t + 1}) \geq \frac{\sum_{v \in \V} r_v(T_t)(r_v(T_t) - 1)}{2n(m - t)}.
\]
\label{lem:minimum_improvement}
\end{lemma}
\begin{proof}
For a candidate $c \notin T_t$, define $\Delta_c := r_{\V}(T_t) - r_{\V}(T_t \cup \{c\})$, i.e., the current marginal contribution of $c$ to the $1$-Borda score. Taking the sum of $\Delta_c$ over $c \notin T_t$:
\[
\sum_{c \in \C \setminus T_t} \Delta_c = \frac{1}{n} \sum_{v \in \V} \sum_{j = 1}^{r_v(T_t) - 1} j = \frac{\sum_{v \in \V} r_v(T_t)(r_v(T_t) - 1)}{2n}.
\]

\g{} chooses $c^* = \argmax_c \Delta_c$ at the $(t + 1)^{\text{st}}$ iteration, giving us
\[
r_{\V}(T_t) - r_{\V}(T_{t + 1}) = \Delta_{c^*} \geq \frac{1}{m - t} \sum_{c \in \C \setminus T_t} \Delta_c = \frac{\sum_{v \in \V} r_v(T_t)(r_v(T_t) - 1)}{2n(m - t)}. \qedhere
\]
\end{proof}

Now we prove our upper bound of $2$.
\begin{theorem}
$\g \leq 2 \cdot \rand$.
\label{thm:ub_greedy_1}
\end{theorem}
\begin{proof}
We prove by induction. As the base case where $k = 1$, $\g \leq m < m + 1 =  2 \cdot \rand$.
Now suppose that the claim holds for some $k - 1$ and we will prove that it also holds for $k$. By induction hypothesis, we have:
\[
r_{\V}(T_{k - 1}) \leq 2 \cdot \frac{m + 1}{k}.
\]

If $r_{\V}(T_{k - 1}) \leq 2 \cdot \frac{m + 1}{k + 1}$, then $r_{\V}(T_k) \leq r_{\V}(T_{k - 1}) \leq 2 \cdot \frac{m + 1}{k + 1}$ finishes the proof. Thus, we only need to consider the following case:
\[
2 \cdot \frac{m + 1}{k + 1} < r_{\V}(T_{k - 1}) \leq 2 \cdot \frac{m + 1}{k}.
\]

We now have the following, where the first inequality is by Lemma~\ref{lem:minimum_improvement} and second by Cauchy-Schwarz inequality:
\begin{align*}
r_{\V}(T_{k - 1}) - r_{\V}(T_k) &\geq \frac{\sum_{v \in \V} r_v(T_{k - 1})(r_v(T_{k - 1}) - 1)}{2n(m - k + 1)} \\
&\geq \frac{\frac{1}{n}(\sum_{v \in \V}r_v(T_{k - 1}))^2 - \sum_{v \in \V}r_v(T_{k - 1})}{2n(m - k + 1)} \\
&= \frac{(\sum_{v \in \V}r_v(T_{k - 1}))^2}{2n^2(m + 1)} \cdot \frac{m + 1}{m - k + 1} \cdot \frac{\sum_{v \in \V}(r_v(T_{k - 1}) - 1)}{\sum_{v \in \V}r_v(T_{k - 1})}.
\end{align*}

Since $r_{\V}(T_{k - 1}) \geq 2 \cdot \frac{m + 1}{k + 1}$ by assumption, we have:
\begin{align*}
\frac{m + 1}{m - k + 1} \cdot \frac{\sum_{v \in \V}(r_v(T_{k - 1}) - 1)}{\sum_{v \in \V}r_v(T_{k - 1})} &\geq \frac{m + 1}{m - k + 1} \cdot \frac{2 \cdot \frac{m + 1}{k + 1} - 1}{2 \cdot \frac{m + 1}{k + 1}}\\
&= \frac{2(m + 1) - k - 1}{2(m + 1) - 2k} \geq 1.
\end{align*}
Combining the previous two inequalities, we therefore have:
\[
r_{\V}(T_{k - 1}) - r_{\V}(T_k) \geq \frac{(\sum_{v \in \V}r_v(T_{k - 1}))^2}{2n^2(m + 1)} = \frac{r_{\V}^2(T_{k - 1})}{2(m + 1)},
\]
which is equivalent to:
\[
r_{\V}(T_k) \leq - \frac{1}{2(m + 1)} r_{\V}^2(T_{k - 1}) + r_{\V}(T_{k - 1}).
\]

Notice that the right hand side is a quadratic function in $r_{\V}(T_{k - 1})$, which is monotonically increasing for $r_{\V}(T_{k - 1}) \leq m + 1$. Since $r_{\V}(T_{k - 1}) \leq 2 \cdot \frac{m + 1}{k} \leq m + 1$, the right hand side reaches its maximum at $2 \cdot \frac{m + 1}{k}$. Thus, we have:
\[
r_{\V}(T_k) \leq - \frac{1}{2(m + 1)}\cdot \left(\frac{2(m + 1)}{k}\right)^2 + \frac{2(m + 1)}{k} \leq \frac{2(m + 1)}{k + 1},
\]
which concludes our induction.
\end{proof}

\paragraph{Proof of Theorem~\ref{thm:greedymax}.} For the maximization version, the above result implies \g{} achieves score at least $(m+1) \cdot \left(1- \frac{2}{k+1} \right)$. Since the maximum possible score is $m$, this implies that \g{} is a $\left( 1 - \frac{2}{k+1} \right)$-approximation.

\subsection{Lower Bound}
\label{sec:greedy_lb}
Now we complement our result with a lower-bound example for \g.

\begin{theorem}
\label{thm:greedy_lb}
There exists an instance in which $r_{\V}(T_k) > 1.962\cdot \rand$.
\end{theorem}

\paragraph{Construction.} In the sequel, we will prove the above theorem. In the instance we construct, $m$, $n$, and $k$ are all sufficiently large. For convenience of illustration, we scale down the ranks by a factor of $m$: now the ranks are $\frac{1}{m}, \frac{2}{m}, \ldots, \frac{m - 1}{m}, 1$. As $m \to \infty$, $\frac{1}{m} \to 0$, so the set of ranking $\{\frac{1}{m}, \frac{2}{m}, \ldots, 1\}$ will become dense in $[0, 1]$, and thus we regard the ranking as being continuous from $0$ to $1$. Our goal becomes to construct an instance in which \g{} gives $r_{\V}(T_k) > 1.962 \cdot \frac{1}{k + 1}$. %We also think of the voters as being divided by $n$ and forming a continuum from $0$ to $1$. These voters are located on a circle (the base in Fig.~\ref{fig:cylinder}) with angular position ranging from $0$ to $2 \pi$.

There are sufficiently many voters, enabling us to view them as a continuum from $0$ to $1$, forming a circle (the base in Fig.~\ref{fig:cylinder}) with angular position ranging from $0$ to $2 \pi$. Imagine that each voter writes down her favorite, her second favorite, \ldots, her least favorite candidate in that order vertically. The result is the side of a cylinder with height $1$, as depicted in Fig.~\ref{fig:cylinder}. Each point on the side identifies a candidate, whose distance to the top, $d$, indicates the corresponding voter ranks him as her $(dm)^{\text{th}}$ favorite candidate (i.e., the candidate has a rank of $d$ in the voter's preference after scaling).

%Fix an integer $N$, and take a small-enough constant $a \in (0, 1)$, and define $f(t) = a\varphi^{\frac{t}{N}}$, where $\varphi$ denotes the golden ratio $\frac{\sqrt{5} - 1}{2} \approx 0.618$. 

\begin{figure}[H]
\centering
\begin{tikzpicture}[scale = 1.0, yscale = 2.5, xscale = 4.0]
\draw[domain=-1.002:1.002, variable=\x, samples=1000, smooth, very thick, dotted] plot ({\x}, {0.1+0.2*((1-\x*\x)^2)^0.25});
\draw[domain=-1.002:1.002, variable=\x, samples=1000, smooth, very thick] plot ({\x}, {0.1-0.2*((1-\x*\x)^2)^0.25});
\draw[domain=-1.002:1.002, variable=\x, samples=1000, smooth, ultra thick, color3] plot ({\x}, {2+0.2*((1-\x*\x)^2)^0.25});
\draw[domain=-1.002:1.002, variable=\x, samples=1000, smooth, ultra thick, color3] plot ({\x}, {2-0.2*((1-\x*\x)^2)^0.25});

\draw[domain=0.1:2, variable=\y, samples=20, smooth, very thick] plot ({-1}, {\y});
\draw[domain=0.1:2, variable=\y, samples=20, smooth, very thick] plot ({1}, {\y});


\draw[domain=-1:1, variable=\x, color1, samples=200, smooth, very thick] plot ({\x}, {2*(1-0.4*exp(-0.076587*(asin(\x)/180*pi+pi/2))) - 0.2*(1-\x*\x)^0.5});
\draw[domain=-1:1, variable=\x, color1, samples=200, smooth, very thick, dotted] plot ({\x}, {2*(1-0.4*((5^0.5-1)/2)^0.5*exp(-0.076587*(asin(-\x)/180*pi+pi/2))) + 0.2*(1-\x*\x)^0.5});

\draw[domain=-1:1, variable=\x, color1, samples=200, smooth, very thick] plot ({\x}, {2*(1-0.4*((5^0.5-1)/2)^1.0*exp(-0.076587*(asin(\x)/180*pi+pi/2))) - 0.2*(1-\x*\x)^0.5});
\draw[domain=-1:1, variable=\x, color1, samples=200, smooth, very thick, dotted] plot ({\x}, {2*(1-0.4*((5^0.5-1)/2)^1.5*exp(-0.076587*(asin(-\x)/180*pi+pi/2))) + 0.2*(1-\x*\x)^0.5});

\draw[domain=-1:1, variable=\x, color1, samples=200, smooth, very thick] plot ({\x}, {2*(1-0.4*((5^0.5-1)/2)^2.0*exp(-0.076587*(asin(\x)/180*pi+pi/2))) - 0.2*(1-\x*\x)^0.5});
\draw[domain=-1:1, variable=\x, color1, samples=200, smooth, very thick, dotted] plot ({\x}, {2*(1-0.4*((5^0.5-1)/2)^2.5*exp(-0.076587*(asin(-\x)/180*pi+pi/2))) + 0.2*(1-\x*\x)^0.5});

\draw[domain=-1:1, variable=\x, color1, samples=200, smooth, very thick] plot ({\x}, {2*(1-0.4*((5^0.5-1)/2)^3.0*exp(-0.076587*(asin(\x)/180*pi+pi/2))) - 0.2*(1-\x*\x)^0.5});
\draw[domain=-1:1, variable=\x, color1, samples=200, smooth, very thick, dotted] plot ({\x}, {2*(1-0.4*((5^0.5-1)/2)^3.5*exp(-0.076587*(asin(-\x)/180*pi+pi/2))) + 0.2*(1-\x*\x)^0.5});

\draw[domain=-1:1, variable=\x, color1, samples=200, smooth, very thick] plot ({\x}, {2*(1-0.4*((5^0.5-1)/2)^4.0*exp(-0.076587*(asin(\x)/180*pi+pi/2))) - 0.2*(1-\x*\x)^0.5});
\draw[domain=-1:1, variable=\x, color1, samples=200, smooth, very thick, dotted] plot ({\x}, {2*(1-0.4*((5^0.5-1)/2)^4.5*exp(-0.076587*(asin(-\x)/180*pi+pi/2))) + 0.2*(1-\x*\x)^0.5});


\draw[domain=-0.29:0.29, variable=\x, color2, samples=200, smooth, ultra thick] plot ({\x}, {2*(1-0.4*exp(-0.076587*(asin(\x)/180*pi+pi/2))) - 0.2*(1-\x*\x)^0.5});


\draw[domain=-0.02:0.02, variable=\y, color2, samples=20, smooth, very thick] plot ({-0.29}, {\y+2*(1-0.4*exp(-0.076587*(asin(-0.29)/180*pi+pi/2))) - 0.2*(1-(-0.29)*(-0.29))^0.5});

\draw[domain=-0.02:0.02, variable=\y, color2, samples=20, smooth, very thick] plot ({-0.10}, {\y+2*(1-0.4*exp(-0.076587*(asin(-0.10)/180*pi+pi/2))) - 0.2*(1-(-0.10)*(-0.10))^0.5});

\draw[domain=-0.02:0.02, variable=\y, color2, samples=20, smooth, very thick] plot ({0.10}, {\y+2*(1-0.4*exp(-0.076587*(asin(0.10)/180*pi+pi/2))) - 0.2*(1-(0.10)*(0.10))^0.5});

\draw[domain=-0.02:0.02, variable=\y, color2, samples=20, smooth, very thick] plot ({0.29}, {\y+2*(1-0.4*exp(-0.076587*(asin(0.29)/180*pi+pi/2))) - 0.2*(1-(0.29)*(0.29))^0.5});


\draw[domain=-0.188:0.188, variable=\x, color2, samples=200, smooth, ultra thick] plot ({\x}, {2*(1-0.4*((5^0.5-1)/2)^2.0*exp(-0.076587*(asin(\x)/180*pi+pi/2))) - 0.2*(1-\x*\x)^0.5});


\draw[domain=-0.02:0.02, variable=\y, color2, samples=20, smooth, very thick] plot ({-0.188}, {\y + 2*(1-0.4*((5^0.5-1)/2)^2.0*exp(-0.076587*(asin(-0.188)/180*pi+pi/2))) - 0.2*(1-(-0.188)*(-0.188))^0.5});

\draw[domain=-0.02:0.02, variable=\y, color2, samples=20, smooth, very thick] plot ({-0.114}, {\y + 2*(1-0.4*((5^0.5-1)/2)^2.0*exp(-0.076587*(asin(-0.114)/180*pi+pi/2))) - 0.2*(1-(-0.114)*(-0.114))^0.5});

\draw[domain=-0.02:0.02, variable=\y, color2, samples=20, smooth, very thick] plot ({-0.0382}, {\y + 2*(1-0.4*((5^0.5-1)/2)^2.0*exp(-0.076587*(asin(-0.0382)/180*pi+pi/2))) - 0.2*(1-(-0.0382)*(-0.0382))^0.5});

\draw[domain=-0.02:0.02, variable=\y, color2, samples=20, smooth, very thick] plot ({0.0382}, {\y + 2*(1-0.4*((5^0.5-1)/2)^2.0*exp(-0.076587*(asin(0.0382)/180*pi+pi/2))) - 0.2*(1-(0.0382)*(0.0382))^0.5});

\draw[domain=-0.02:0.02, variable=\y, color2, samples=20, smooth, very thick] plot ({0.114}, {\y + 2*(1-0.4*((5^0.5-1)/2)^2.0*exp(-0.076587*(asin(0.114)/180*pi+pi/2))) - 0.2*(1-(0.114)*(0.114))^0.5});

\draw[domain=-0.02:0.02, variable=\y, color2, samples=20, smooth, very thick] plot ({0.188}, {\y + 2*(1-0.4*((5^0.5-1)/2)^2.0*exp(-0.076587*(asin(0.188)/180*pi+pi/2))) - 0.2*(1-(0.188)*(0.188))^0.5});


\draw [very thick, ->] (1.2, 2) -- (1.2, 0.1);

\node [color3] at (-0.9, 2.2) {Voters};
\node [color1] at (-0.5, 1.25) {Critical Candidates};
\node [color2] at (0, 1) {Lower-Layer Candidates};
\node [color2] at (0, 1.45) {Higher-Layer Candidates};
\node at (1.45, 1.4) {Decreasing};
\node at (1.45, 1.28) {Preferences};

\node at (1.40, 2.0) {Rank: $0$};
\node at (1.40, 0.1) {Rank: $1$};

\draw [decorate, very thick, color2, decoration={brace,amplitude=5pt}] (-1, {2-0.8}) -- (-1, {2-0.8*(5^0.5-1)/2});
\draw [decorate, very thick, color2, decoration={brace,amplitude=5pt}] (-1, {2-0.8*(5^0.5-1)/2}) -- (-1, {2-0.8*((5^0.5-1)/2)^2});

\node [color2] at (-1.25, {(2-0.8+2-0.8*(5^0.5-1)/2)/2}) {$1^{\text{st}}$ Layer};
\node [color2] at (-1.25, {(2-0.8*(5^0.5-1)/2+2-0.8*((5^0.5-1)/2)^2)/2}) {$2^{\text{nd}}$ Layer};



\end{tikzpicture}
\caption{Construction of the Bad Instance for \g{}}
\label{fig:cylinder}
\end{figure}


We divide the set of candidates into two types -- {\em critical} and {\em dummy}. The former set has size $k \ll m$, and the latter has size $m-k$. Our proof will show that \g{} will choose the critical candidates in a fixed order, and will not choose any dummy candidate.

The critical candidates are present in $\ell$ ``layers'' as shown in the red spiral in Fig.~\ref{fig:cylinder}, where $\ell$ is sufficiently large. This figure shows the ranks of the critical candidates in the voters' profiles. We parametrize this spiral by $\theta$, which maps to the voter at the corresponding angular position $2\pi\theta$. We place critical candidates in order, where each candidate appears a number of times consecutively on the spiral. Therefore, each voter has one critical candidate from each layer $t = 0, 1, \ldots, \ell$ in the spiral part of its ranking.

In the $t^{\text{th}}$ layer, the parameter $\theta$ lies in $[t - 1, t)$. The critical candidate when the parameter is $\theta$ has rank $g(\theta) = a \varphi^{\theta}$ for the voter at angular position $2\pi\theta$. Here, $\varphi$ denotes the golden ratio $\frac{\sqrt{5} - 1}{2} \approx 0.618$, and $a$ is a sufficiently small constant so that rounding to the nearest integer does not change the analysis. This critical candidate is placed for a certain length $h(\theta)$ on the spiral, which means this candidate appears at rank $g(\theta)$ for voters in the range $\left[2\pi\theta, 2\pi(\theta + h(\theta))\right]$. In our construction, $h(\theta)$ will be very small, so that we will say this candidate appears $h(\theta)$ times at rank $g(\theta)$ for parameter $\theta$. The greater $\theta$ is, the smaller $h(\theta)$ has to be, and we will calculate its expression later. 

For the convenience of analysis, at the layer $t = 0$, that is, for $\theta \in [-1,0)$, there is a special candidate appearing on the spiral throughout the layer. This special candidate is picked first by \g{}. Other than its appearance on the spiral, any critical candidate is placed at the very bottom, i.e., rank $1$, for the other voters. Denote the total number of critical candidates by $k$. Then we have $m - k$ dummy candidates. These dummy candidates are symmetrically placed at other ranks. We copy each voter $(m - k)!$ times, once for each possible permutation of the dummy candidates to place in the remaining ranks. %(See Remark at the end of the proof for how to make the construction polynomial size in $m$.)

The idea of this construction is to trick \g{} into picking every critical candidate on the spiral in order, while in fact, lower-layer critical candidates have no contribution to the objective once higher-layer ones have been selected. The following analysis computes the optimal parameters to realize this plan.




\paragraph{Not Choosing a Dummy Candidate.} We first ensure \g{} does not choose a dummy candidate in this instance by setting $h(\theta)$ properly. We assume that \g{} chooses critical candidates in increasing order of $\theta$, and we will justify this assumption later.


To simplify notation, denote $X = \int_{0}^1 a\varphi^{\theta}\d \theta$ and $Y = \int_{0}^1 a^2\varphi^{2\theta}\d \theta$. Computing these explicitly:
\[
X = \frac{a}{\ln\varphi}(\varphi - 1), \qquad Y = \frac{a^2}{2\ln\varphi}(\varphi^2 - 1) = X^2 \frac{(\varphi + 1)\ln\varphi}{2(\varphi - 1)}.
\]

Using this notation, consider the critical candidate at the beginning of the first layer, that is, at $\theta = 0$. Since \g{} chooses the candidate at layer $t = 0$, the decrease in score due to this critical candidate is:
\begin{equation}
\label{eq:improve_critical}
h(0) \cdot (g(-1) - g(0)) = h(0) \cdot a \cdot \left(\frac{1}{\varphi} - 1\right) = h(0) \cdot a \cdot \varphi.
\end{equation}
where we have used that since $\varphi$ is the golden ratio, $\varphi + \varphi^2 = 1$.

Now consider the dummy candidates. Just after \g{} has chosen the special candidate at layer $t = 0$, each such candidate improves the rank of $g(\theta-1)$ fraction of voters at $\theta \in [0, 1)$. This is because we placed all permutations of the dummy candidates with each voter $\theta$, and \g{} has already chosen the special candidate. By the same reasoning, conditioned on improvement, the average improvement is $g(\theta-1)/2$. Therefore, the decrease in score due to a dummy candidate is: 
\begin{equation}
\label{eq:improve_dummy}
\int_0^{1} \frac{g^2(\theta-1)}{2} \d \theta = \frac{a^2}{2\varphi^2} \int_0^{1} \varphi^{2\theta} \d \theta = \frac{1}{2\varphi^2} \cdot Y.
\end{equation}
Since we want \g{} to choose the critical candidate, we need to set
\[
h(0) = \frac{Y}{2\varphi^3 a}.
\]
By the symmetry of the spiral, an identical calculation now holds for all $\theta > 0$. To make \g{} choose the critical candidate at this location (assuming it has chosen critical candidates for smaller values of $\theta$), we need:
\[
h(\theta) = \frac{Y}{2\varphi^3 a} \varphi^{\theta}.
\]
Note that $h(\theta)$ depends linearly on $a$, so that for very small $a$, we can pretend this set of voters lies exactly at $\theta$. Further, $h(\theta)$ is decreasing with $\theta$.
%to make the spiral symmetric and the calculation identical at any step of \g{}.

\paragraph{Choosing Critical Candidates in Order.} We now show that \g{} chooses the critical candidates following the order on the spiral.
\begin{lemma}
\label{lem:greedyopt2}
\g{} chooses the critical candidates in increasing order of $\theta$.
\end{lemma}
\begin{proof}
The calculation is identical at any step of \g{}, so we focus on the step where \g{} is at the beginning of the first layer, that is, considering the critical candidate at $\theta = 0$. Recall that \g{} has chosen the special candidate at layer $t = 0$. The previous analysis showed that the critical candidate at $\theta = 0$ yields decrease of $\frac{Y}{2 \varphi^2}$. For critical candidates in the same layer $t = 1$ (that is, for $\theta \in [0, 1)$), the contribution of the candidate at $\theta$ is
\[
h(\theta) \cdot (g(\theta-1) - g(\theta)) = \frac{Y}{2\varphi^3} \varphi^{\theta} \left(\varphi^{\theta - 1} - \varphi^{\theta}\right) = \frac{Y}{2\varphi^2} \cdot \varphi^{2\theta},
\]
which decreases with $\theta$, so that the current candidate, $\theta = 0$, offers the best decrease. Here, we have used that since $\varphi$ is the golden ratio, $\varphi^2 + \varphi = 1$.

For $t \ge 1$, suppose we instead considered a candidate $t +\theta$ for $\theta \in [0,1)$ located in layer $t+1$. Conditioned on having chosen layer $t = 0$, this candidate gives a contribution of
\begin{align*}
h(t + \theta) \cdot (g(\theta-1) - g(t + \theta)) &\leq h(t) \cdot (g(-1) - g(t))\\
&\leq \max\big(h(2) \cdot g(-1), \ h(1) \cdot (g(-1) - g(1))\big)\\
&= \max\left(\frac{Y}{2\varphi^2}, \ \frac{Y}{2\varphi^2a} \cdot a\left(\frac{1}{\varphi} - \varphi\right)\right) = \frac{Y}{2\varphi^2},
\end{align*}
where the first inequality uses that $h(\theta)$ is decreasing in $\theta$, and that $\varphi < 1$.

Therefore, \g{} will pick the critical candidate at $\theta = 0$ instead of another candidate at the same or a higher layer. Since the argument is identical at each $\theta$, \g{} picks critical candidates in order on the spiral.
\end{proof}

\paragraph{The Lower Bound.} So far we have shown that \g{} chooses critical candidates in increasing order of layers and does not choose dummy candidates. We finally put it all together and show the following bound, which completes the proof of Theorem~\ref{thm:greedy_lb}. 

\begin{proof}[Proof of Theorem~\ref{thm:greedy_lb}]
The number of critical candidates on the $t^{\text{th}}$ layer ($\theta \in [t - 1, t)$) is
\[
\int_{t - 1}^t \frac{1}{h(\theta)} \d \theta = \frac{2 \varphi^3 a}{Y} \int_{t - 1}^t \varphi^{-\theta} \d \theta = \frac{2 a}{\varphi^{t-3} Y} \int_{-1}^0 \varphi^{-\theta} \d \theta  = \frac{2 a}{\varphi^{t-3} Y} \int_{0}^1 \varphi^{\theta} \d \theta = \frac{2 X}{\varphi^{t - 3} Y}.
\]
Therefore, when it is done with the $\ell^{\text{th}}$ layer, the number of candidates \g{} has picked is
\[
k = \frac{2 X}{\varphi^{\ell - 3} Y} (1 + \varphi + \varphi^2 + \cdots + \varphi^{\ell - 1}) \rightarrow \frac{2 X}{(1 - \varphi)\varphi^{\ell - 3} Y}
\]
when $\ell$ is large. Meanwhile, the $1$-Borda score of \g{} is
\[
r_{\V}(T_k) = \int_{0}^1 g(\ell - 1 + \theta) \d \theta = \varphi^{\ell - 1} X.
\]
Therefore, the approximation ratio is
\[
(k+1) r_{\V}(T_k) \ge \frac{2 X}{(1 - \varphi)\varphi^{\ell - 3} Y} \cdot \varphi^{\ell - 1} X = \frac{2 \varphi^2 X^2}{(1 - \varphi) Y} = \frac{2\varphi^2}{(1 - \varphi)} \cdot \frac{2(\varphi - 1)}{(\varphi + 1) \ln \varphi} = -\frac{4 \varphi^2}{(\varphi + 1) \ln \varphi} > 1.962. \qedhere
\]
\end{proof}


%\paragraph{Remark} \kn{Is this still correct? I'm not sure... (This is referred to in the construction part.)} The current construction has $n$ which is exponential in $m$. However, using the same proof as Lemma~\ref{lem:random_construction}, if we sample $\mbox{poly}(m,1/\varepsilon)$ voters, the improvement generated by each candidate is approximated to within a factor of $(1+\varepsilon)$ at all steps of \g{} with high probability. This is sufficient to make \g{}  choose critical candidates in the correct order. Note that analysis of \g{} remains the same even for finite $N = \mbox{poly}(m)$, and the limiting case is just to simplify the analysis. This makes the overall construction polynomial size in the number of candidates $m$ and we omit the straightforward details.


\iffalse

\paragraph{Construction.} In the sequel, we will prove the above theorem. In the instance we construct, $m$, $n$, and $k$ are all sufficiently large. For convenience of illustration, we scale down the ranks by a factor of $m$: now the ranks are $\frac{1}{m}, \frac{2}{m}, \ldots, \frac{m - 1}{m}, 1$. As $m \to \infty$, $\frac{1}{m} \to 0$, so the set of ranking $\{\frac{1}{m}, \frac{2}{m}, \ldots, 1\}$ will become dense in $[0, 1]$, and thus we regard the ranking as being continuous from $0$ to $1$. Our goal becomes to construct an instance in which \g{} gives $r_{\V}(T_k) > 1.962 \cdot \frac{1}{k + 1}$.

Fix an integer $N$, and take a small-enough constant $a \in (0, 1)$, and define $f(t) = a\varphi^{\frac{t}{N}}$, where $\varphi$ denotes the golden ratio $\frac{\sqrt{5} - 1}{2} \approx 0.618$. 

We divide the set of candidates into two types -- {\em critical} and {\em dummy}. The former set has size $k$, and the latter has size $m-k$. Our proof will show that \g{} will choose the critical candidates in a fixed order, and will not choose dummy candidates. 

The critical candidates are present in $\ell$ ``layers'' as shown in the red spiral in Fig.~\ref{fig:cylinder}, where $\ell$ is sufficiently large. This figure shows the ranks of the critical candidates in the voters' profiles. In layer $i$, there are $k_i$ candidates each appearing $n/k_i$ times. We will compute $k_i$ below. Let $S_i$ denote the set of $k_i$ critical candidates in layer $i$. Let $Q$ denote the set of all permutations of the $(m-k)$ dummy candidates. The set of voters is $[N] \times S_1 \times \cdots \times S_{\ell} \times Q$. For voter $\langle t, c_1, c_2, \ldots, c_{\ell}, \pi \rangle$, the critical candidate in layer $i$ is $c_i \in S_i$, and its rank is $f(t + (i-1)N)$. The remaining candidates from $S_1 \cup \cdots \cup S_{\ell}$ have the lowest possible rank $1$ for this voter. The dummy candidates appear in the ranking in the remaining slots according to the permutation $\pi$.

In this instance, $n = N (m-k)! \prod_{i=1}^{\ell} k_i$, and $k = \sum_{i=1}^{\ell} k_i$. (See Remark at the end of the proof for how to make the construction polynomial size in $m$.) For analytic convenience, take the limit $N \rightarrow \infty$ and assume $[N]$ is a continuum $[0,1]$. Therefore, for a voter with first coordinate $x \in [0,1]$, a candidate in layer $i$ appears with rank $g_i(x) = a\varphi^{x + (i-1)}$. Note that in this instance, the candidates in each layer are symmetric from the perspective of \g{}, and so are the dummy candidates.


%\begin{figure}[H]
\centering
\begin{tikzpicture}[scale = 1.0, yscale = 2.5, xscale = 4.0]
\draw[domain=-1.002:1.002, variable=\x, samples=1000, smooth, very thick, dotted] plot ({\x}, {0.1+0.2*((1-\x*\x)^2)^0.25});
\draw[domain=-1.002:1.002, variable=\x, samples=1000, smooth, very thick] plot ({\x}, {0.1-0.2*((1-\x*\x)^2)^0.25});
\draw[domain=-1.002:1.002, variable=\x, samples=1000, smooth, ultra thick, color3] plot ({\x}, {2+0.2*((1-\x*\x)^2)^0.25});
\draw[domain=-1.002:1.002, variable=\x, samples=1000, smooth, ultra thick, color3] plot ({\x}, {2-0.2*((1-\x*\x)^2)^0.25});

\draw[domain=0.1:2, variable=\y, samples=20, smooth, very thick] plot ({-1}, {\y});
\draw[domain=0.1:2, variable=\y, samples=20, smooth, very thick] plot ({1}, {\y});


\draw[domain=-1:1, variable=\x, color1, samples=200, smooth, very thick] plot ({\x}, {2*(1-0.4*exp(-0.076587*(asin(\x)/180*pi+pi/2))) - 0.2*(1-\x*\x)^0.5});
\draw[domain=-1:1, variable=\x, color1, samples=200, smooth, very thick, dotted] plot ({\x}, {2*(1-0.4*((5^0.5-1)/2)^0.5*exp(-0.076587*(asin(-\x)/180*pi+pi/2))) + 0.2*(1-\x*\x)^0.5});

\draw[domain=-1:1, variable=\x, color1, samples=200, smooth, very thick] plot ({\x}, {2*(1-0.4*((5^0.5-1)/2)^1.0*exp(-0.076587*(asin(\x)/180*pi+pi/2))) - 0.2*(1-\x*\x)^0.5});
\draw[domain=-1:1, variable=\x, color1, samples=200, smooth, very thick, dotted] plot ({\x}, {2*(1-0.4*((5^0.5-1)/2)^1.5*exp(-0.076587*(asin(-\x)/180*pi+pi/2))) + 0.2*(1-\x*\x)^0.5});

\draw[domain=-1:1, variable=\x, color1, samples=200, smooth, very thick] plot ({\x}, {2*(1-0.4*((5^0.5-1)/2)^2.0*exp(-0.076587*(asin(\x)/180*pi+pi/2))) - 0.2*(1-\x*\x)^0.5});
\draw[domain=-1:1, variable=\x, color1, samples=200, smooth, very thick, dotted] plot ({\x}, {2*(1-0.4*((5^0.5-1)/2)^2.5*exp(-0.076587*(asin(-\x)/180*pi+pi/2))) + 0.2*(1-\x*\x)^0.5});

\draw[domain=-1:1, variable=\x, color1, samples=200, smooth, very thick] plot ({\x}, {2*(1-0.4*((5^0.5-1)/2)^3.0*exp(-0.076587*(asin(\x)/180*pi+pi/2))) - 0.2*(1-\x*\x)^0.5});
\draw[domain=-1:1, variable=\x, color1, samples=200, smooth, very thick, dotted] plot ({\x}, {2*(1-0.4*((5^0.5-1)/2)^3.5*exp(-0.076587*(asin(-\x)/180*pi+pi/2))) + 0.2*(1-\x*\x)^0.5});

\draw[domain=-1:1, variable=\x, color1, samples=200, smooth, very thick] plot ({\x}, {2*(1-0.4*((5^0.5-1)/2)^4.0*exp(-0.076587*(asin(\x)/180*pi+pi/2))) - 0.2*(1-\x*\x)^0.5});
\draw[domain=-1:1, variable=\x, color1, samples=200, smooth, very thick, dotted] plot ({\x}, {2*(1-0.4*((5^0.5-1)/2)^4.5*exp(-0.076587*(asin(-\x)/180*pi+pi/2))) + 0.2*(1-\x*\x)^0.5});


\draw[domain=-0.29:0.29, variable=\x, color2, samples=200, smooth, ultra thick] plot ({\x}, {2*(1-0.4*exp(-0.076587*(asin(\x)/180*pi+pi/2))) - 0.2*(1-\x*\x)^0.5});


\draw[domain=-0.02:0.02, variable=\y, color2, samples=20, smooth, very thick] plot ({-0.29}, {\y+2*(1-0.4*exp(-0.076587*(asin(-0.29)/180*pi+pi/2))) - 0.2*(1-(-0.29)*(-0.29))^0.5});

\draw[domain=-0.02:0.02, variable=\y, color2, samples=20, smooth, very thick] plot ({-0.10}, {\y+2*(1-0.4*exp(-0.076587*(asin(-0.10)/180*pi+pi/2))) - 0.2*(1-(-0.10)*(-0.10))^0.5});

\draw[domain=-0.02:0.02, variable=\y, color2, samples=20, smooth, very thick] plot ({0.10}, {\y+2*(1-0.4*exp(-0.076587*(asin(0.10)/180*pi+pi/2))) - 0.2*(1-(0.10)*(0.10))^0.5});

\draw[domain=-0.02:0.02, variable=\y, color2, samples=20, smooth, very thick] plot ({0.29}, {\y+2*(1-0.4*exp(-0.076587*(asin(0.29)/180*pi+pi/2))) - 0.2*(1-(0.29)*(0.29))^0.5});


\draw[domain=-0.188:0.188, variable=\x, color2, samples=200, smooth, ultra thick] plot ({\x}, {2*(1-0.4*((5^0.5-1)/2)^2.0*exp(-0.076587*(asin(\x)/180*pi+pi/2))) - 0.2*(1-\x*\x)^0.5});


\draw[domain=-0.02:0.02, variable=\y, color2, samples=20, smooth, very thick] plot ({-0.188}, {\y + 2*(1-0.4*((5^0.5-1)/2)^2.0*exp(-0.076587*(asin(-0.188)/180*pi+pi/2))) - 0.2*(1-(-0.188)*(-0.188))^0.5});

\draw[domain=-0.02:0.02, variable=\y, color2, samples=20, smooth, very thick] plot ({-0.114}, {\y + 2*(1-0.4*((5^0.5-1)/2)^2.0*exp(-0.076587*(asin(-0.114)/180*pi+pi/2))) - 0.2*(1-(-0.114)*(-0.114))^0.5});

\draw[domain=-0.02:0.02, variable=\y, color2, samples=20, smooth, very thick] plot ({-0.0382}, {\y + 2*(1-0.4*((5^0.5-1)/2)^2.0*exp(-0.076587*(asin(-0.0382)/180*pi+pi/2))) - 0.2*(1-(-0.0382)*(-0.0382))^0.5});

\draw[domain=-0.02:0.02, variable=\y, color2, samples=20, smooth, very thick] plot ({0.0382}, {\y + 2*(1-0.4*((5^0.5-1)/2)^2.0*exp(-0.076587*(asin(0.0382)/180*pi+pi/2))) - 0.2*(1-(0.0382)*(0.0382))^0.5});

\draw[domain=-0.02:0.02, variable=\y, color2, samples=20, smooth, very thick] plot ({0.114}, {\y + 2*(1-0.4*((5^0.5-1)/2)^2.0*exp(-0.076587*(asin(0.114)/180*pi+pi/2))) - 0.2*(1-(0.114)*(0.114))^0.5});

\draw[domain=-0.02:0.02, variable=\y, color2, samples=20, smooth, very thick] plot ({0.188}, {\y + 2*(1-0.4*((5^0.5-1)/2)^2.0*exp(-0.076587*(asin(0.188)/180*pi+pi/2))) - 0.2*(1-(0.188)*(0.188))^0.5});


\draw [very thick, ->] (1.2, 2) -- (1.2, 0.1);

\node [color3] at (-0.9, 2.2) {Voters};
\node [color1] at (-0.5, 1.25) {Critical Candidates};
\node [color2] at (0, 1) {Lower-Layer Candidates};
\node [color2] at (0, 1.45) {Higher-Layer Candidates};
\node at (1.45, 1.4) {Decreasing};
\node at (1.45, 1.28) {Preferences};

\node at (1.40, 2.0) {Rank: $0$};
\node at (1.40, 0.1) {Rank: $1$};

\draw [decorate, very thick, color2, decoration={brace,amplitude=5pt}] (-1, {2-0.8}) -- (-1, {2-0.8*(5^0.5-1)/2});
\draw [decorate, very thick, color2, decoration={brace,amplitude=5pt}] (-1, {2-0.8*(5^0.5-1)/2}) -- (-1, {2-0.8*((5^0.5-1)/2)^2});

\node [color2] at (-1.25, {(2-0.8+2-0.8*(5^0.5-1)/2)/2}) {$1^{\text{st}}$ Layer};
\node [color2] at (-1.25, {(2-0.8*(5^0.5-1)/2+2-0.8*((5^0.5-1)/2)^2)/2}) {$2^{\text{nd}}$ Layer};



\end{tikzpicture}
\caption{Construction of the Bad Instance for \g{}}
\label{fig:cylinder}
\end{figure}


%\paragraph{Continuous Interpretation.} The above construction has the following continuous interpretation as a distribution over ranks. 

%\begin{itemize}
%    \item  Assume the voters lie on a continuum in $[0,1]$ and so do the ranks, with $0$ being the lowest rank. As mentioned above, this will be a good approximation for large $m,n$.
%    \item The rank of voter $x \in [0,1]$ for its critical candidate in layer $i \in \{1,2, \ldots, \ell\}$ is $g_i(x) = a\varphi^{x - (i-1)}$, where $a$ is a small constant. 
%    \item Independently of other layers, for layer $i$ and for each voter $x \in [0,1]$, one of the $k_i$ critical candidates chosen uniformly at random appears at rank $g_i(x)$, and the remaining $k_i-1$ critical candidates appear at rank $1$.
%    \item Independently of the critical candidates and other voters, for each voter, the order of dummy candidates is a uniformly random permutation over them. 
%\end{itemize}

%This yields a distribution over ranks over the set of voters. We obtain the discrete version by treating this distribution as a collection of samples, and making the final voter set the the union of the sets of voters, one set from each sample.  %As $\gamma \rightarrow \infty$, the behavior of \g{} on this discrete version will approximate its behavior on the continuous version. In the sequel, we analyze the continuous version.

%To interpret this continuous version, if a candidate appears with mass $y$ at a particular rank for a voter, in the underlying instance, there are copies of this voter and this candidate appears at this rank for a $y$ fraction of those voters. Therefore, if a set of candidates appears at this rank with total mass of $1$, we split these candidates in the appropriate fractions among  the underlying voter copies. 


\paragraph{Computing the $k_i$.} We first assume that \g{} chooses critical candidates in increasing order of layers. We will justify this assumption later. Recall $k_i$ is the number of candidates in the $i^{\text{th}}$ layer. Let $K_i = \sum_{j \le i} k_j$ be the total number of chosen candidates after we have chosen the $i^{\text{th}}$ layer. Let $T_{K_i}$ denote the set of chosen candidates after we have chosen the $i^{\text{th}}$ layer. We will now compute $k_i$ in order to make \g{} prefer critical to dummy candidates.

%In the analysis below, we will take the limit as $m \rightarrow \infty$ and divide the ranks throughout by $m$. Therefore, we wish to show that \g{} achieves $r_{\V}(T_k) > 1.962 \cdot \frac{1}{k + 1}$.

To simplify notation, denote $X = \int_{x = 0}^1 a\varphi^x\d x$ and $Y = \int_{x = 0}^1 a^2\varphi^{2x}\d x$. Computing these explicitly:
$$X = \frac{a}{\ln\varphi}(\varphi - 1), \qquad Y = \frac{a^2}{2\ln\varphi}(\varphi^2 - 1) = X^2 \frac{(\varphi + 1)\ln\varphi}{2(\varphi - 1)}.$$

Using this notation, the score of candidate set $T_{K_i}$ is given by
\begin{equation}
    \label{eq:improve}
    \sum_{v \in \V}r_v(T_{K_i})= \int_{x = 0}^1 g_i(x) \d x = \varphi^{i - 1}\int_{x = 0}^1 a\varphi^x\d x = \varphi^{i - 1} X.
    \end{equation}

The decrease in score had all the candidates in layer $i$ were chosen is therefore:
$$\sum_{c \in i^{\text{th}} \text{ layer}} [r_{\V}(T_{K_{i - 1}}) - r_{\V}(T_{K_{i - 1}}\cup \{c\})]= r_{\V}(T_{K_{i - 1}}) - r_{\V}(T_{K_i}) = (\varphi^{i - 2} - \varphi^{i - 1})X.$$

The candidates in $T_{K_i}$ are symmetric in that each of these gives the same decrease in score regardless of which other candidates in this layer have been chosen. Therefore, each critical candidate in layer $i$ gives a decrease of exactly $(\varphi^{i - 2} - \varphi^{i - 1}) \frac{X}{k_i}$.

Now consider the dummy candidates. Just after \g{} chooses $T_{K_{i-1}}$, each such candidate, if it improves the rank, appears at average rank $g_{i-1}(x)/2$ on a fraction $g_{i-1}(x)$ of voters. This upper bounds the decrease even after some candidates in $T_{K_i}$ are chosen. Therefore, the decrease in score due to a dummy candidate while \g{} is considering layer $i$ is at most: \kn{$\sum_{v \in \V}\frac{r_v^2(T_{K_{i-1}})}{2} = \int_{x = 0}^1 g_{i - 1}^2(x) \d x = \frac{\varphi^{2i-4}}{2}\int_{x = 0}^1 a^2\varphi^{2x} \d x$ ?}
\[
\sum_{v \in \V}\frac{r_v^2(T_{K_{i-1}})}{2} = \frac{\varphi^{2i-2}}{2}\int_{x = 0}^1 a^2\varphi^{2x} \d x = \frac{\varphi^{2i - 2}}{2} Y.
\]

Therefore, for \g{} to choose a critical candidate at layer $i$ over any dummy candidate, we need to have
\[
\frac{\varphi^{2i - 2}}{2} Y \le (\varphi^{i - 2} - \varphi^{i - 1}) \frac{X}{k_i}.
\]
Setting this to equality, and assuming \g{} breaks ties to favor critical candidates, this yields 
\begin{equation} \label{eq:ki}
k_i = \frac{(\varphi^{i - 2} - \varphi^{i - 1})X}{\frac{\varphi^{2i - 2}}{2}Y} = \frac{-4(\varphi - 1)^2}{X(\varphi + 1)\ln\varphi} \cdot \frac{1}{\varphi^i},
\end{equation}
and it follows that:
\begin{equation}\label{eq:k}
k = \sum_{i = 1}^\ell k_i = \frac{-4(\varphi - 1)^2}{X(\varphi + 1)\ln\varphi} \sum_{i = 1}^\ell \frac{1}{\varphi^i} = \frac{4(\varphi - 1)}{X(\varphi + 1)\ln\varphi} \left(\frac{1}{\varphi^{\ell - 1}} - \varphi \right).
\end{equation}
We assume $a$ is small enough that rounding $k_i$'s to the nearest integer does not change the analysis.

\paragraph{Order of Choosing Candidates in \g{}.} We now show that \g{} chooses the critical candidates in increasing order of layers. 

\begin{lemma}
\label{lem:greedyopt2}
\g{}  chooses the critical candidates from the first layer to the $\ell^{\text{th}}$ layer in order, and will not skip a layer in the process.
\end{lemma}
\begin{proof}
First, we notice that, when $a$ is small enough, \g{} will start with choosing the candidates in the first layer. Now, suppose that \g{} has finished choosing candidates in the $i^{\text{th}}$ layer, we show that in the next step, it will choose candidates in the $(i + 1)^{\text{st}}$ layer instead of candidates in the $(i + 2)^{\text{nd}}$ layer, and using the same argument, it will not choose candidates in layers $(i + 3)$, $(i + 4)$, etc, before layer $(i + 1)$.

Recall that we use $k_i$ to denote the number of candidates in the $i^{\text{th}}$ layer,  $K_i$ to denote the total number of chosen candidates after we have chosen the $i^{\text{th}}$ layer, $T_{K_i}$ to denote the corresponding set of chosen candidates. Let $z = \frac{Y}{2} = \frac{X^2(\varphi + 1)\ln\varphi}{4(\varphi - 1)}$. 

Assume \g{} has chosen $T_{K_i}$. By Eq~(\ref{eq:improve}) and~(\ref{eq:ki}), for any candidate in the $(i + 1)^{\text{st}}$ layer, the decrease in score after choosing it is exactly:
$$ \frac{r_{\V}(T_{K_{i}}) - r_{\V}(T_{K_{i+1}})}{k_{i+1}} = \frac{\varphi^{i - 1} - \varphi^i}{k_{i+1}} X = z\varphi^{2i}.$$

Similarly, if we choose a candidate in the $(i + 2)^{\text{nd}}$ layer instead, the worst-case decrease will be given by assuming \g{} has not chosen any candidate from $T_{K_i}$, so this decrease is at most:
$$ \frac{r_{\V}(T_{K_{i}}) - r_{\V}(T_{K_{i+2}})}{k_{i+2}} = \frac{\varphi^{i - 1} - \varphi^{i+1}}{k_{i+2}} X = z \left(\varphi^{2i+1} +  \varphi^{2i+2} \right).$$

Since $\varphi$ is the golden ratio, we have $\varphi^2 + \varphi = 1$, and thus we have:
$$z\varphi^{2i+1} + z\varphi^{2i+2} = z\varphi^{2i},$$
which implies that after \g{} has chosen the $i^{\text{th}}$ layer, choosing a candidate in the $(i + 1)^{\text{st}}$ layer provides the optimal decrease in score provided we break ties in its favor. This shows that \g{}  will choose candidates from the first layer to the $\ell^{\text{th}}$ layer in order.
\end{proof}

\paragraph{The Lower Bound.} So far we have shown that \g{} chooses critical candidates in increasing order of layers and does not choose dummy candidates. We finally put it all together and show the following bound, which completes the proof of Theorem~\ref{thm:greedy_lb}. 

\begin{lemma}
\label{lem:greedyopt1}
For $\ell$ sufficiently large, $r_{\V}(T_k) > 1.962 \cdot \frac{1}{k + 1}$.
\end{lemma}
\begin{proof}
Using Eq~(\ref{eq:improve}) and Eq~(\ref{eq:k}), we now have:
$$r_{\V}(T_k) \cdot(k + 1) > \varphi^{\ell - 1}X \cdot \frac{4(\varphi - 1)}{X(\varphi + 1)\ln\varphi}\left(\frac{1}{\varphi^{\ell - 1}} - \varphi\right) = \frac{4(\varphi - 1)}{(\varphi + 1)\ln\varphi}(1 - \varphi^{\ell}) > 1.962,$$
i.e., $r_{\V}(T_k) > 1.962 \cdot \frac{1}{k + 1}$.
\end{proof}


\paragraph{Remark} The current construction has $n$ which is exponential in $m$. However, using the same proof as Lemma~\ref{lem:random_construction}, if we sample $\mbox{poly}(m,1/\varepsilon)$ voters, the improvement generated by each candidate is approximated to within a factor of $(1+\varepsilon)$ at all steps of \g{} with high probability. This is sufficient to make \g{}  choose critical candidates in the correct order. Note that analysis of \g{} remains the same even for finite $N = \mbox{poly}(m)$, and the limiting case is just to simplify the analysis. This makes the overall construction polynomial size in the number of candidates $m$ and we omit the straightforward details.

%\begin{figure}[!htb]
%\centering
%\includegraphics[width = 0.8\textwidth]{election.jpg}
%\caption{$n!$ Copies of the Election Graph}
%\end{figure}

\fi

\section{Verification benchmark}
\label{sec:benchmark}

% Relative to the tokamak core, the characteristic time and spatial scales
% are compressed.  However, type-I ELMs still have the instability time
% scale associated with the fast crash is an order of magnitude faster
% than the transport-time scale associated with the processes that govern
% the build up of the pedestal structure.   This separation of time scales
% still allows the standard decomposition of studying the linear
% instabilities about an equilibrium that is used in core modes as well.
% 
% Like the core modes, these long-wavelength instabilities are dominated
% by the stiffness in the ideal MHD terms, even for the cases when they
% may be strictly ideal stable.  Multiple numerical methods have been
% developed to handle this stiffness for both linear and nonlinear codes.
% For the nonlinear codes, one numerical advantage is to 
% separate the fields into steady-state (e.g. the reconstructed fields)
% and time-dependent parts.  The pure steady-state terms are analytically
% eliminated resulting in the largest terms in the system to be removed
% from the numerical computations.
% 
% Although typically only MHD-force balance (a
% Grad-Shafranov solution) is strictly enforced for the steady state, in practice
% all fields associated are time independent. This effectively assumes the
% presence of implicit (in the sense that they are calculable but not calculated)
% sources, fluxes and sinks.  With these assumptions, if the code is run on a
% MHD-stable case, the fields do not change.  Alternatively, the initial fields
% are self-consistently modified by the presence of unstable modes. 
% {\bf SEK: OK -- I think this is a better place to put the discussion, in
%   the end this is confusing unless there is an appendix to explain
%   things in detail.  We need to discuss whether we want to add it.  I
% think not as this is really an EHO discussion.}
% 
% There is no technical reason to make this time-scale decomposition - the NIMROD
% code has the capability to compute the extended-MHD evolution of the
% reconstructed fields. However, it is well-known that physical mechanisms
% outside the scope of the extended-MHD model mediate tokamak transport such as
% neoclassical bootstrap current, toroidal viscosity, and poloidal flow damping;
% neutral beam and RF drive; turbulence; and coupling to the scrape-off layer
% (SOL), neutrals, impurities and the material boundary. Including these effects
% requires explicit calculation of the sources, fluxes and sinks. These
% transport-type calculations are possible and are becoming practical (e.g.
% \cite{held15}), but this sort of integrated modeling remains in the future.

We begin with a study of a high resolution, lower-single-null, JT-60U-like
equilibrium (`Meudas-1'), which was originally employed in a benchmark of the
MARG2D and ELITE codes \cite{Aiba07}, including a
close approach to the X-point \cite{Snyder09}.
This extends previous benchmarks~\cite{Burke10} of ELITE and NIMROD as it
includes diverted magnetic topology and a higher edge safety factor
($q_{95}=6.74$, the safety factor at 95\% of the normalized poloidal flux) that
leads to increased resolution requirements. An ideal-MHD limit is achieved in
NIMROD by using flat density and resistivity profiles inside the last closed
flux surface (LCFS) with small resistivity, $S=10^8$ where $S$ is the Lundquist
number  ($S=\tau_R/\tau_A$), $\tau_A$ is the Alfvén time ($\tau_A=R_o/v_A$),
$v_A$ is the Alfvén velocity ($B/\sqrt{m_i n_i \mu_0}$), $\tau_R$ is the
resistive diffusion time ($\tau_R=R_o^2 \mu_0/\eta$), $R_o=2.936 m$ is the
radius of the magnetic axis, $\eta$ is the electrical resistivity, $\mu_0$ is
the permeability of free space, $m_\alpha$ is a species mass (the $\alpha$
subscript denotes ions or electrons in this work), and $n_\alpha$ is a species
density. The deuteron mass ($m_i = 3.34\times 10^{-27} kg$) is used. In order to
reproduce the vacuum response model outside the LCFS that is used by ELITE, a
low density ($0.01$ of the core density) and high resistivity ($10^7$ times the
core resistivity) is prescribed beyond the LCFS (more details on these
approximations are in Ref.~\cite{Burke10}).  

\begin{figure}
  \centering
  \includegraphics[width=8cm]{ELITEComparison}
  \vspace{-4mm}
  \caption{[Color online]
  Growth rates for the `Meudas-1' benchmark. ELITE with $\Gamma=5/3$ and
  $\Gamma=0$ are compared against results from NIMROD with $\Gamma=5/3$).
  Associated NIMROD data available in Ref.~\cite{king16Z}.}
  \label{fig:ELITEComp}
\end{figure}

\begin{figure}
  \centering
  \includegraphics[width=8cm]{idealConv}
  \vspace{-4mm}
  \caption{[Color online]
  Spectral convergence of the NIMROD code for the ideal-like parameters. 
  The maximum polynomial degree (P) of the basis functions composing the 
  spectral elements in increased in each subsequent line plotted.
  Associated NIMROD data available in Ref.~\cite{king16Z}.}
  \label{idealConv}
\end{figure}

The normalized growth rates ($\gamma \tau_A$ where the linearized mode grows as
$\text{exp}[\gamma t]$) vs.~toroidal mode number ($n_\phi$) from NIMROD and ELITE are
plotted in Fig.~\ref{fig:ELITEComp}.  There is good agreement between the
codes except for $n_\phi$=4 where there is a 27\% relative difference. All
other cases have a relative difference of less than 8\% with typical
differences of 5\%. The NIMROD convergence in terms of the maximum polynomial
order of the spectral elements is shown in Fig.~\ref{idealConv}. Convergence is
most challenging at high wavenumbers where the resolution requirements are most
stringent (the poloidal mesh is composed of $72\times512$ spectral elements).
%SEK: Great point, but total troll bait for reviewers
These cases converge from the unstable side where the growth rate decreases with
enhanced resolution. Thus the excellent agreement between NIMROD and ELITE at
high $n_\phi$ in Fig.~\ref{fig:ELITEComp} may be spurious and indicate that
slightly more resolution is required for $n_\phi$>25, however, the 
shown growth rates are likely within 5\% of their converged values.
Studying nearly ideal cases with extended MHD codes such as NIMROD is challenging 
given the vanishingly small dissipation operators, and convergence is achieved
more quickly with the additional non-ideal terms in the extended-MHD equations,
as in the cases in Sec.~\ref{sec:xMHD}.

% Relative to modeling with extended MHD, ideal-MHD convergence is more challenging 
% given the vanishingly small dissipation operators and convergence is 
% achieved more quickly with all other model equations shown in this work.

\begin{figure}
  \includegraphics[width=8cm]{meudas_n11_BR}
  \caption{[Color online]
  Poloidal cross section of the radial magnetic field component of the
  $n_\phi=11$ peeling-ballooning mode from the `Meudas-1' benchmark case. }
  \vspace{-4mm}
  \label{meudas_n11_BR}
\end{figure}

Figure \ref{meudas_n11_BR} shows a poloidal cross section of the magnetic
($B_R$) eigenmode.  The mode develops an `interference-pattern' structure near
the X-point when inboard and outboard finger-like structures overlap. The
finite-element-mesh nodes are superimposed atop the smallest-scale sub-figure.
As established by Fig.~\ref{idealConv}, this simulation is spatially
and temporally converged. The high resolution required to resolve these
high-$q_{95}$, diverted cases
motivated development of memory-scaling improvements in the NIMROD code.

\section{Lower Bound on Committee-Monotone Algorithms for $1$-Borda}
\label{sec:monotone}
Consider the $1$-Borda score. A nice property of \g{} is that it is committee-monotone: In each iteration, the candidate chosen by \g{} only depends on which candidates have been chosen in previous iterations and not on $k$, and thus when $k$ increases, the committee selected by \g{} includes all the candidates \g{} used to select. On the other hand, \b{} does not satisfy committee monotonicity, as the candidates chosen by \b{} does depend on $k$.

This naturally brings up the question: Is there a committee-monotone algorithm which is optimal with respect to the benchmark \rand{}? We answer this question in the negative, by presenting a lower bound of $1.015$ for all committee-monotone algorithms.

%This sounds like a tradeoff between performance and monotonicity: on one hand, \b{} has better performance with respect to our benchmark than \g{} and on the other hand, \g{} satisfies committee monotonocity while \b{} does not. This brings up the natural question: is it possible for committee monotone algorithms to be optimal, in the sense of our benchmark? In the following theorem, we answer this question in the negative. We present a lower bound of $1.015$ for all committee monotone algorithms with respect to our benchmark, thus showing a separation between monotonicity and optimality.

\begin{theorem}
For any large enough $m$, there exist instances with $m$ candidates where any committee-monotone algorithm \alg{} satisfies $r_{\V}(T_k) > 1.015 \cdot \rand$ for some value $k \in \{1, 2\}$. Here, $T_k$ is the set of candidates \alg{} chooses when the size of this set is $k$.
\label{thm:monotonicity}
\end{theorem}
\begin{proof}
The construction goes as follows: There are two types of candidates, $X$ and $Y$. Candidates of type $X$ are ranked between $[am, bm]$ by every voter and candidates of type $Y$ are ranked between $[1, am] \cup [bm, m]$ by every voter, where $0 < a < b < 1$ are two parameters. We construct sufficiently many voters so that all candidates of the same type are symmetric (by having all permutations of candidates of type $X$ and those of type $Y$). We want to find proper $a$ and $b$, so that when $k = 1$, the optimal choice is to choose a candidate of type $X$, while when $k = 2$, the optimal choice is to choose two candidates both of type $Y$. This means that no committee-monotone algorithm can produce optimal choice for both the first iteration and the second iteration. We optimize over $a$ and $b$ to find the maximum lower bound.

In particular, the search procedure goes as follows. Let $r_{\V}(Y)$ denote the $1$-Borda score of choosing a candidate of type $Y$; $r_{\V}(XX)$ denote the score of choosing two candidates both of type $X$; and $r_{\V}(XY)$ denote the score of choosing a candidate of type $X$ and a candidate of type $Y$. We can easily see that, when $m$ goes to infinity, up to an $o(1)$ additive error:
\[
\begin{cases}
\frac{1}{m + 1}r_{\V}(Y) = \frac{a}{2} \cdot \Pr[Y \text{ is at } [1, am]] + \frac{1 + b}{2} \cdot \Pr[Y \text{ is at } [bm, m]] = \frac{a}{2} \cdot \frac{a}{1 - (b - a)} + \frac{1 + b}{2} \cdot \frac{1 - b}{1 - (b - a)}\\
\frac{1}{m + 1}r_{\V}(XX) = \frac{2a + b}{3}\\
\frac{1}{m + 1}r_{\V}(XY) = \frac{a}{2} \cdot \Pr[Y \text{ is at } [1, am]] + \frac{a + b}{2} \cdot \Pr[Y \text{ is at } [bm, m]] = \frac{a}{2} \cdot \frac{a}{1 - (b - a)} + \frac{a + b}{2} \cdot \frac{1 - b}{1 - (b - a)}
\end{cases}.
\]

A committee-monotone algorithm either chooses $Y$ in the first iteration, or chooses $XX$ or $XY$ in the first two iterations. Thus, we maximize $\min\left(\frac{2}{m + 1}r_{\V}(Y), \frac{3}{m + 1}r_{\V}(XX), \frac{3}{m + 1}r_{\V}(XY)\right)$ (note that the value on the numerator corresponds to the value of $k + 1$) over $0 < a < b < 1$, and find that, for $a = 0.377$ and $b = 0.552$, it achieves a lower bound greater than $1.015$.
\end{proof}

\section{Connection to the Core}
\label{sec:core}
We now consider the relationship between the core and $1$-Borda score. In particular, we show that the core achieves a stronger notion of proportionality than $1$-Borda: any $\alpha$-approximate core solution has $1$-Borda score at most $\alpha \cdot \frac{k + 1}{k} \cdot \rand$, while neither the optimal solution $\opt$ nor the algorithms \g{} and \b{} lies in an $o(k)$-approximate core.

Recall that we say a committee $T$ of size $k$ is in the $\alpha$-approximate core if there is no blocking candidate strictly preferred by at least $\alpha \cdot \frac{n}{k}$ voters over $T$. See Eq~(\ref{eq:core}) for a formal definition. In this section, we investigate the relationship between $1$-Borda and the core.

First, we present in the following theorem the implication of the core for $1$-Borda score.

\begin{theorem}
Any committee $T$ in the $\alpha$-approximate core satisfies $r_{\V}(T) \leq \alpha \cdot \frac{k + 1}{k} \cdot \rand$.
\label{thm:core_to_borda}
\end{theorem}
\begin{proof}
As $T$ is in the $\alpha$-approximate core, there is no deviation of size $\frac{\alpha n}{k}$, i.e., there is no candidate ranked above all candidates in $T$ for $\frac{\alpha n}{k}$ voters. Therefore,
\[
\frac{1}{m - k} \sum_{v \in \V} (r_v(T) - 1) \leq \frac{\alpha n}{k}
\]
by a counting argument. Thus,
\[
r_{\V}(T) = \frac{1}{n} \sum_{v \in \V} r_v(T) \leq (m - k) \cdot \frac{\alpha}{k} = \alpha \cdot \frac{k + 1}{k} \cdot \frac{m + 1}{k + 1}. \qedhere
\]
\end{proof}

Naturally we ask: Does the reverse statement -- a good approximation to $\rand$ for the $1$-Borda score gives a good approximation to the core -- hold as well? It turns out that the answer is no.

\begin{example}
Let $n = 3 \cdot (m - 2)!$, where $m$ is sufficiently large. $c_1$ and $c_2$ are two ``critical'' candidates, and the remaining $m - 2$ are ``dummy'' candidates. For the first $\frac{n}{3}$ voters, $c_1$ is their top choices and $c_2$ is their second choices. For the second $\frac{n}{3}$ voters, $c_1$ is their bottom choices and $c_2$ is their top choices. For the last $\frac{n}{3}$ voters, $c_1$ is their bottom choice and $c_2$ is their second bottom choice. We fill the rest of their preferences with all permutations of the dummy candidates. This example is illustrated in Figure~\ref{fig:not_core}.
\label{ex:not_core}
\end{example}

\begin{figure}
\centering
\begin{tikzpicture}[yscale = 3.75, xscale = 5]

\draw [decorate, very thick, color3, decoration={brace,amplitude=10pt}] (-1, 2) -- (-0.37, 2);
\draw [decorate, very thick, color3, decoration={brace,amplitude=10pt}] (-0.315, 2) -- (0.315, 2);
\draw [decorate, very thick, color3, decoration={brace,amplitude=10pt}] (0.37, 2) -- (1, 2);

\draw [very thick, color1] (-1, 1.9) -- (-0.37, 1.9);
\draw [very thick, color1] (-1, 1.88) -- (-1, 1.92);
\draw [very thick, color1] (-0.37, 1.88) -- (-0.37, 1.92);

\draw [very thick, color2] (-1, 1.8) -- (-0.37, 1.8);
\draw [very thick, color2] (-1, 1.78) -- (-1, 1.82);
\draw [very thick, color2] (-0.37, 1.78) -- (-0.37, 1.82);

\draw [thick] (-1, 1.7) -- (-0.37, 1.7) -- (-0.37, 1.0) -- (-1, 1.0) -- (-1, 1.7);



\draw [very thick, color1] (-0.315, 1.0) -- (0.315, 1.0);
\draw [very thick, color1] (-0.315, 0.98) -- (-0.315, 1.02);
\draw [very thick, color1] (0.315, 0.98) -- (0.315, 1.02);

\draw [very thick, color2] (-0.315, 1.9) -- (0.315, 1.9);
\draw [very thick, color2] (-0.315, 1.88) -- (-0.315, 1.92);
\draw [very thick, color2] (0.315, 1.88) -- (0.315, 1.92);

\draw [thick] (-0.315, 1.8) -- (0.315, 1.8) -- (0.315, 1.1) -- (-0.315, 1.1) -- (-0.315, 1.8);



\draw [very thick, color1] (0.37, 1.0) -- (1, 1.0);
\draw [very thick, color1] (0.37, 0.98) -- (0.37, 1.02);
\draw [very thick, color1] (1, 0.98) -- (1, 1.02);

\draw [very thick, color2] (0.37, 1.1) -- (1, 1.1);
\draw [very thick, color2] (0.37, 1.08) -- (0.37, 1.12);
\draw [very thick, color2] (1, 1.08) -- (1, 1.12);

\draw [thick] (0.37, 1.9) -- (1, 1.9) -- (1, 1.2) -- (0.37, 1.2) -- (0.37, 1.9);


\draw [very thick, ->] (1.2, 1.9) -- (1.2, 1);

\node [color3] at (-0.685, 2.2) {First $\frac{n}{3}$ Voters};
\node [color3] at (0, 2.2) {Second $\frac{n}{3}$ Voters};
\node [color3] at (0.685, 2.2) {Last $\frac{n}{3}$ Voters};

\node [color1] at (-0.685, 1.95) {$c_1$};
\node [color2] at (-0.685, 1.85) {$c_2$};
\node at (-0.685, 1.35) {Dummy Candidates};


\node [color1] at (0, 1.05) {$c_1$};
\node [color2] at (0, 1.95) {$c_2$};
\node at (0, 1.45) {Dummy Candidates};


\node [color1] at (0.685, 1.05) {$c_1$};
\node [color2] at (0.685, 1.15) {$c_2$};
\node at (0.685, 1.55) {Dummy Candidates};


\node at (1.40, 1.45) {Preferences};
\node at (1.40, 1.55) {Decreasing};

\end{tikzpicture}
\caption{Illustration of Voters' Preferences in Example~\ref{ex:not_core}}
\label{fig:not_core}
\end{figure}


\begin{theorem}
The solutions of \opt{}, \g{} and \b{} do not lie in an $o(k)$-approximate core in Example~\ref{ex:not_core}. 
\label{thm:borda_to_core}
\end{theorem}
\begin{proof}
Let $k = \sqrt{m} - 1$ in Example~\ref{ex:not_core}. We show all of \opt{}, \g{} and \b{} chooses $c_2$ and $k - 1$ dummy candidates. In this solution, $\frac{n}{3}$ voters could deviate to $c_1$, showing that it does not lie in a $\frac{k}{3}$-approximate core.

\paragraph{Proof for \opt{}}  
%Clearly, \opt{} chooses $c_2$. \kn{Why?} Next, we compare the score for choosing $c_2$ and $k - 1$ dummy candidates with choosing $c_1$, $c_2$ and $k - 2$ dummy candidates. Let $T_{k - 1}$ be a set of candidates consisting of $c_2$ and $k - 2$ dummy candidates. Then, we have:
We compare the resulting $s$-Borda score for all possible schemes: choosing $c_1$, $c_2$, and $k - 2$ dummy candidates; choosing $c_1$ and $k - 1$ dummy candidates; choosing $c_2$ and $k - 1$ dummy candidates; and choosing $k$ dummy candidates. Let $D_j$ be a set consisting of $j$ dummy candidates. Then, we have:
\begin{align*}
r_{\V}(D_{k - 2} \cup \{c_1\} \cup \{c_2\}) = \frac{m + 1}{3(k - 1)} + \frac{2}{3}, &\quad r_{\V}(D_{k - 1} \cup \{c_1\}) = \frac{2(m + 1)}{3k} + \frac{1}{3},\\
r_{\V}(D_{k - 1} \cup \{c_2\}) = \frac{m + 1}{3k} + 1, &\quad r_{\V}(D_k) = \frac{m + 1}{k + 1}.
\end{align*}

For $k = \sqrt{m} - 1$, we have:
$$r_{\V}(D_{k - 1} \cup \{c_2\}) < r_{\V}(D_{k - 2} \cup \{c_1\}\cup\{c_2\}) <  r_{\V}(D_{k - 1} \cup \{c_1\}) < r_{\V}(D_k).$$

Thus, \opt{} chooses $c_2$ and $k - 1$ dummy candidates without choosing $c_1$.

\paragraph{Proof for \g{}}
For the first iteration, \g{} chooses $c_2$. We will show that, for the next $\sqrt{m} - 2$ iterations, \g{} chooses the dummy candidates and does not choose $c_1$. Suppose we have chosen $j - 1$ candidates, where $j - 1 \leq \sqrt{m} - 1$, and the current set of candidates is $T_{j - 1}$. Then, we have:
\[
r_{\V}(T_{j - 1}) - r_{\V}(T_{{j - 1}} \cup \{c_1\}) = \frac{1}{3},
\]
\[
r_{\V}(T_{j - 1}) - r_{\V}(T_{j - 1} \cup \{c_j\}) = \frac{1}{3}\left(\frac{m + 1}{j} - \frac{m + 1}{j - 1}\right) = \frac{m + 1}{3j(j - 1)} > \frac{1}{3}, \forall c_j \in \C \setminus T_{j - 1}, c_j \neq c_1.
\]
which shows that for the $\sqrt{m} - 2$ iterations after the first iteration, \g{} chooses dummy candidates.

\paragraph{Proof for \b{}}
Let $T_j$ be the set of candidates produced by \b{} after $j$ iterations. Recall that by \b{}, in the $j^{\text{th}}$ iteration, we pick $c_j \in \C \setminus T_{j - 1}$ that minimizes:
\[
\sum_{\substack{S \subseteq \C: |S| = k \\ S \supseteq T_{j - 1} \cup \{c_j\}}} r_{\V}(S).
\]

Clearly, \b{} chooses $c_2$ in the first iteration, because, as we have shown in the proof for \opt{}, for $k = \sqrt{m} - 1$, choosing $c_2$ always yields better result than not choosing $c_2$. 

Then, we show that \b{} chooses dummy candidates for the next $\sqrt{m} - 2$ iterations. Assume at $(j - 1)^{\text{th}}$ iteration, we have chosen $j - 2$ dummy candidates and $c_2$. As we have shown in the proof for \opt{}, for $k = \sqrt{m} - 1$, we have $r_{\V}(T_{k - 1} \cup \{c_j\}) < r_{\V}(T_{k - 1} \cup \{c_1\})$, $\forall c_j \in \C \setminus T_{k - 1}, c_j \neq c_1$, where $T_{k - 1}$ is a set consisting of $c_2$ and $k - 2$ dummy candidates. This implies that the candidate that minimizes the above objective is dummy candidate but not $c_1$. Thus, for the $j^{\text{th}}$ iteration, \b{} also chooses a dummy candidate, and by inductive principle, \b{} chooses $c_2$ and $k - 1$ dummy candidates in $k$ iterations.
\end{proof}

Theorem~\ref{thm:core_to_borda} and Theorem~\ref{thm:borda_to_core} together establishes that the core achieves a stronger notion of proportionality than $1$-Borda.


\section{The $s$-Borda Score}
In this section, we extend our analysis of the greedy algorithms to $s$-Borda score, and show how to significantly improve on the \g{} and \b{} rules via linear programming. 

Recall that $\rand = \frac{m+1}{k+1}$ and choosing a random committee of size $k$ yields expected score $\frac{s(s + 1)}{2} \cdot \rand$. As a derandomization, \b{} has score at most this value similar to Theorem~\ref{thm:banzhaf_better_than_rand}. Let $\opt$ denote the best possible $s$-Borda score. Considering the instance with one voter for each permutation of candidates as its preference ordering, we have the following proposition:

\begin{proposition}
For any $s$, $m$ and $k$, there exists instances where $\opt =\frac{s(s + 1)}{2} \cdot \rand$.
\end{proposition}

We first consider a natural extension of \g{} in the $1$-Borda case. In Appendix~\ref{app:sborda}, we show that it achieves an $s$-Borda score at most $2s^2\cdot \rand$ (Theorem~\ref{thm:s_borda}), which is within a factor of $\frac{4s}{s + 1}$ of the \b{} rule. We then show that this bound cannot be improved even when $\opt$ is small. However, unlike the $1$-Borda case, there is no fundamental barrier to an improved algorithm when $\opt$ is small, and we present such an algorithm in Section~\ref{sec:lp}.

\subsection{The \g{} Algorithm} 
\label{sec:sborda_greedy}
\label{sec:sborda_g_ub}
\label{sec:sborda_g_lb}
The \g{} algorithm follows exactly the same procedure as for $1$-Borda, except that we now compute the score based on $s$-Borda. We present an upper bound for \g{} in the following theorem. Since the proof is very similar to the $s=1$ case, we present it in Appendix~\ref{app:sborda}.

\begin{theorem}[Proved in Appendix~\ref{app:sborda}]
\label{thm:s_borda}
$\g{} \leq 2s^2\cdot \rand$.
\end{theorem}

\paragraph{Lower Bound for Small $\opt$.} 
In general, $\opt = \Omega(s^2) \cdot \rand$, in which case the analysis of greedy is tight to within a constant factor. The question we now ask is: Does \g{} always perform better when $\opt$ is small? We answer this in the negative.

\begin{theorem}
\label{thm:sbordalb}
There exists an instance where $\opt = O(s^2) = o(1) \cdot \rand$, while the score of \g{} is $\Omega(s^2)\cdot \rand$.
\end{theorem}
To prove this lower bound, we use the following instance.
%We now present an example in which \opt{} = $O(s^2) = o(1) \cdot \rand$, while the score of \g{} is still $\Omega(s^2)\cdot \rand$. %This example points to such general cases in which \g{} performs badly while \opt{} is small. We now present an example in which the $s$-Borda score of \opt{} is $O(s^2) = o(1)\cdot \rand$, while the score of \g{} is $\Omega(s^2)\cdot \rand$. 
%This shows that \g{} can perform as bad as random even when \opt{} is small and thus motivates the improved guarantee in Section~\ref{sec:lp}.
\begin{example}
%Similar as before, we consider this example through an election graph. 
Let $m = \omega(k)$, $k = \omega(s)$, and $n = \frac{k}{s}(m - k)!$. There are $k$ ``critical'' candidates $c_1, c_2, \ldots, c_k$, while the remaining $m - k$ are ``dummy'' candidates. Candidate $c_{i(k/s) + j}$ is the $(i + 1)^{\text{st}}$ choice of the $j^{\text{th}}$ $\frac{k}{s}$ voters, $\forall i \in \{0, 1, \ldots, s - 1\}, j \in \{1, 2, \ldots, \frac{k}{s}\}$. Aside from the first $s$ rows, the critical candidates lie at the very bottom. For each group of $\frac{k}{s}$ voters, we fill the rest of the preferences with all permutations of the dummy candidates. This example is illustrated in Figure~\ref{figure:greedy_bad}. %We can make the number of voters $\mbox{poly}(m)$ by sampling permutations; see remark at the end of Section~\ref{sec:greedy_lb}.
\label{example:greedy_bad}
\end{example}

\begin{figure}
\centering
\begin{tikzpicture}[yscale = 3.75, xscale = 5]

\draw [decorate, very thick, color3, decoration={brace,amplitude=10pt}] (-1, 2) -- (-0.5, 2);
\node [color3] at (-0.75, 2.2) {First $\frac{sn}{k}$ Voters};
\draw [decorate, very thick, color3, decoration={brace,amplitude=10pt}] (-0.45, 2) -- (0.05, 2);
\node [color3] at (-0.2, 2.2) {Second $\frac{sn}{k}$ Voters};
\draw [decorate, very thick, color3, decoration={brace,amplitude=10pt}] (0.5, 2) -- (1, 2);
\node [color3] at (0.75, 2.2) {Last $\frac{sn}{k}$ Voters};

\draw [very thick, color1] (-1, 1.9) -- (-0.5, 1.9);
\draw [very thick, color1] (-1, 1.88) -- (-1, 1.92);
\draw [very thick, color1] (-0.5, 1.88) -- (-0.5, 1.92);
\node [color1] at (-0.75, 1.95) {$c_1$};

\draw [very thick, color1] (-0.45, 1.9) -- (0.05, 1.9);
\draw [very thick, color1] (-0.45, 1.88) -- (-0.45, 1.92);
\draw [very thick, color1] (0.05, 1.88) -- (0.05, 1.92);
\node [color1] at (-0.2, 1.95) {$c_2$};

\draw [very thick, color1] (0.5, 1.9) -- (1, 1.9);
\draw [very thick, color1] (0.5, 1.88) -- (0.5, 1.92);
\draw [very thick, color1] (1, 1.88) -- (1, 1.92);
\node [color1] at (0.75, 1.95) {$c_{k/s}$};

\node [color1] at (0.275, 1.9) {$\ldots$};

\draw [very thick, color1] (-1, 1.8) -- (-0.5, 1.8);
\draw [very thick, color1] (-1, 1.78) -- (-1, 1.82);
\draw [very thick, color1] (-0.5, 1.78) -- (-0.5, 1.82);
\node [color1] at (-0.75, 1.85) {$c_{(k/s)+1}$};

\draw [very thick, color1] (-0.45, 1.8) -- (0.05, 1.8);
\draw [very thick, color1] (-0.45, 1.78) -- (-0.45, 1.82);
\draw [very thick, color1] (0.05, 1.78) -- (0.05, 1.82);
\node [color1] at (-0.2, 1.85) {$c_{(k/s) + 2}$};

\draw [very thick, color1] (0.5, 1.8) -- (1, 1.8);
\draw [very thick, color1] (0.5, 1.78) -- (0.5, 1.82);
\draw [very thick, color1] (1, 1.78) -- (1, 1.82);
\node [color1] at (0.75, 1.85) {$c_{2k/s}$};

\node [color1] at (0.275, 1.8) {$\ldots$};

\node [color1] at (-0.75, 1.7){$\vdots$};

\node [color1] at (-0.2, 1.7){$\vdots$};

\node [color1] at (0.75, 1.7){$\vdots$};

\node [color1] at (0.275, 1.7){$\vdots$};

\draw [very thick, color1] (-1, 1.5) -- (-0.5, 1.5);
\draw [very thick, color1] (-1, 1.48) -- (-1, 1.52);
\draw [very thick, color1] (-0.5, 1.48) -- (-0.5, 1.52);
\node [color1] at (-0.75, 1.55) {$c_{(s-1)(k/s)+1}$};

\draw [very thick, color1] (-0.45, 1.5) -- (0.05, 1.5);
\draw [very thick, color1] (-0.45, 1.48) -- (-0.45, 1.52);
\draw [very thick, color1] (0.05, 1.48) -- (0.05, 1.52);
\node [color1] at (-0.2, 1.55) {$c_{(s-1)(k/s) + 2}$};

\draw [very thick, color1] (0.5, 1.5) -- (1, 1.5);
\draw [very thick, color1] (0.5, 1.48) -- (0.5, 1.52);
\draw [very thick, color1] (1, 1.48) -- (1, 1.52);
\node [color1] at (0.75, 1.55) {$c_{k}$};

\node [color1] at (0.275, 1.5) {$\ldots$};

\draw [thick] (-1, 1.45) -- (1, 1.45) -- (1, 0.85) -- (-1, 0.85) -- (-1, 1.45);
\node at (0, 1.15) {Dummy Candidates};

\draw [thick] (-1, 0.8) -- (1, 0.8) -- (1, 0.5) -- (-1, 0.5) -- (-1, 0.8);
\node at (0, 0.65) {Critical Candidates};

\draw [very thick, ->] (1.2, 1.9) -- (1.2, 0.5);

\node at (1.40, 1.30) {Decreasing};
\node at (1.40, 1.20) {Preferences};

\end{tikzpicture}
\caption{Illustration of Voters' Preferences in Example~\ref{example:greedy_bad}}
\label{figure:greedy_bad}
\end{figure}

In this instance, \opt{} is clearly $O(s^2)$ by choosing all the critical candidates. We now show that \g{} achieves its worst-case bound even on this instance.

\begin{proposition}
In Example~\ref{example:greedy_bad}, $\g{} = \Omega(s^2)\cdot \rand$.
\end{proposition}
\begin{proof}
For the first $s$ iterations, \g{} chooses dummy candidates: as $m \gg k$, choosing a critical candidate adds $\frac{s - 1}{s}(m + 1)$ to the score, while choosing a dummy candidate adds only $\frac{1}{2}(m + 1)$.

Then, we show that, for the first $\frac{k}{2}$ iterations, \g{} chooses dummy candidates. Assume at $(j - 1)^{\text{th}}$ iteration, where $s \leq j - 1 < \frac{k}{2}$, we have chosen $j - 1$ dummy candidates, and the set of candidates is $T_{j - 1}$. Then, we have:
\[
r_{\V}(T_{j - 1}) - r_{\V}(T_{j - 1} \cup \{c_\mathrm{critical}\}) \leq \frac{s}{k}\cdot \frac{s}{j}(m + 1) = \frac{s^2}{kj}(m + 1),
\]
and
\[
r_{\V}(T_{j - 1}) - r_{\V}(T_{j - 1} \cup \{c_\mathrm{dummy}\}) = \frac{s(s + 1)}{2j}(m + 1) - \frac{s(s + 1)}{2(j + 1)}(m + 1) = \frac{s(s + 1)}{2j(j + 1)}(m + 1),
\]
where $c_\mathrm{critical}$ is some critical candidate and $c_\mathrm{dummy}$ is some dummy candidate. This is because if we choose a critical candidate, then for $\frac{s}{k}$ fraction of the voters, the bottom-ranked dummy candidate will be dropped, while the critical candidate will be added. Since we have chosen $j - 1$ dummy candidates, the bottom-ranked dummy candidate has average rank $\frac{s}{j}(m + 1)$. In other words, for $\frac{s}{k}$ fraction of the voters, we drop a candidate at rank $\frac{s}{j}(m + 1)$ and gain a candidate at the top, while for the other voters, the top $s$ candidates remain unchanged. If we choose a dummy candidate instead, the average score goes from $\frac{s(s + 1)}{2j}(m + 1)$ to $\frac{s(s + 1)}{2(j + 1)}(m + 1)$.

For $j \leq \frac{k}{2}$, we have:
$$\frac{s(s + 1)}{2j(j + 1)}(m + 1) > \frac{s^2}{kj}(m + 1),$$
and thus \g{} chooses a dummy candidate in the $j^{\text{th}}$ iteration as well. Thus, by inductive principle, \g{} chooses dummy candidates for at least $\frac{k}{2}$ iterations.

However, this implies that we can choose at most $\frac{k}{2}$ critical candidates. Suppose for the $i^{\text{th}}$ $\frac{s}{k}$ voters, there are $x_i$ critical candidates among the top $s$ candidates. We have:
$$\sum_{i = 1}^{k/s}x_i \leq \frac{k}{2}.$$

Let $T_k$ denote the final set of candidates. As we choose at most $\frac{k}{2}$ critical candidates, at least $\frac{k}{2}$ candidates must be chosen, and we derive a lower bound for $r_\V(T_k)$ based on this. We have:
\begin{align*}
r_{\V}(T_k) &\geq \frac{s}{k}\left(\sum_{i = 1}^{k/s}\sum_{j = 1}^{s - x_i}j\right)\cdot \rand \geq \frac{s}{2k}\left(\sum_{i = 1}^{k/s}(s - x_i)^2\right) \cdot \rand \\
& \geq \frac{s}{2k} \frac{\left(\sum_{i = 1}^{k/s}(s - x_i)\right)^2}{k/s}\cdot \rand \geq \frac{s^2}{8}\cdot \rand.
\end{align*}

Recall that given $n$ voters whose preference structures include all permutations of the $m$ candidates, when we choose $k$ candidates out of them, the average contribution of the $i^{\text{th}}$-ranked candidates for each voter to $r_\V(T_k)$ is $i\cdot \rand$. The first inequality is by applying the above fact on each set of $\frac{s}{k}$ voters whose preference structures include all permutations. The third inequality is by Cauchy-Schwarz inequality. The last inequality is because $\sum_{i = 1}^{k/s}x_i \leq \frac{k}{2}$. Thus, we can conclude that $r_{\V}(T_k) = \Omega(s^2)\cdot \rand$.
\end{proof}


This shows that \g{} can perform as bad as random even when \opt{} is small and thus motivates the improved guarantee in Section~\ref{sec:lp}.


\subsection{An Improved Algorithm via LP Rounding}
\label{sec:lp}
As mentioned above, \g{} can hit its worst-case bound of $\Omega(s^2)\cdot \rand$ even when \opt{} is actually small. We know that for the case of $1$-Borda, no polynomial-time algorithm can do better. Now the question is, can a different algorithm do better in the case of $s$-Borda for $s = \omega(1)$? We answer this question in the affirmative by presenting an algorithm that is based on dependent rounding of an LP relaxation combined with uniform random sampling, which provides nontrivial improvement when $\opt$ is small. In particular, it achieves expected score at most $3\cdot \opt + O(s^{3/2}\log s)\cdot \rand$. 

\subsubsection{LP-Rounding-Based Algorithm}
The following linear program (based on~\cite{cornuejols1983,LinV,CharikarGTS,fault,Byrka}) is a natural relaxation for the $s$-Borda problem.
\begin{mini*}
{}{\sum_{i = 1}^n\sum_{\ell = 1}^s\sum_{j=1}^{m} x_{ij}^{\ell} \cdot r_{v_i}(c_j),}{}{}
\addConstraint{\sum_{j=1}^m\:} {y_j}{=k}
\addConstraint{\sum_{\ell = 1}^k\:}{x_{ij}^{\ell}}{\leq y_j,}{\mkern46mu i \in \{1, \ldots, n\}, \:\:j\in\{1 ,\ldots, m\}} 
\addConstraint{\sum_{j = 1}^m\:}{x_{ij}^{\ell}}{\geq 1,}{\mkern53mu i \in \{1, \ldots, n\},\:\: \ell\in \{1, \ldots, k\}}
\addConstraint{y_j, \:}{x_{ij}^{\ell}}{\in [0, 1],}{\mkern26mu i \in \{1, \ldots, n\}, \:\: j \in \{1, \ldots, m\}, \:\: \ell \in \{1, \ldots, k\}.}
\end{mini*}

Variable $y_j$ denotes how much candidate $c_j$ is chosen; integral values $1$ and $0$ mean choosing and not choosing candidate $c_j$, respectively. The first constraint encodes choosing exactly $k$ candidates. We copy each voter $k$ times, and the $\ell^{\text{th}}$ copy of the voter $v_i$ is assigned to the $\ell^{\text{th}}$-preferred chosen candidate. Variable $x_{ij}^{\ell}$ denotes how much the $\ell^{\text{th}}$ copy of voter $v_i$ is assigned to candidate $c_j$. The second constraint prevents a voter from being assigned to a candidate that is not chosen. The third constraint ensures that each copy of the voter is assigned to some candidate. The objective function computes the $s$-Borda score.

We will use dependent rounding~\cite{dependent} to round this LP solution. There is a catch though: Dependent rounding can cause a deficit in around $\tilde{O}(\sqrt{s})$ candidates from the top $s$ that are fractionally chosen by the LP. Since any solution must account for the top $s$ scores, we need to ensure these ``deficit'' candidates do not increase the score too much. Towards this end, we scale down the LP solution, and choose enough candidates uniformly at random so that these candidates can absorb the deficit. However, such scaling creates a further deficit that will have to be absorbed by random sampling. We find that the right trade-off is achieved by scaling down by a factor of $(1-\frac{1}{\sqrt{s}})$.

Without further ado, the overall algorithm works as follows:
\begin{enumerate}
    \item Solve the above linear program and let $\tilde{y}$ denote the optimal solution.
    \item For $j = 1,2,\ldots, m$, let $y_j = (1 - \frac{1}{\sqrt{s}}) \tilde{y}_j$. Note that $\sum_{j=1}^m y_j = k (1 - \frac{1}{\sqrt{s}})$.
    \item Apply dependent rounding~\cite{dependent} on the variables $\{y_j\}$ so that exactly $k (1-\frac{1}{\sqrt{s}})$ candidates are chosen. Let $T_1$ denote the set of chosen candidates.
    \item Finally choose a set $T_2$ of $\frac{k}{\sqrt{s}}$ candidates uniformly at random from $\C \setminus S$ and output $T = T_1 \cup T_2$. 
\end{enumerate}

We will show the following theorem:

\begin{theorem}
\label{thm:lp}
When $m = \omega(k)$, $k = \omega(s^{3/2})$, and $s = \omega(1)$, we have:
$$\E[r_{\V}(T)] \leq 3\opt +O(s^{3/2}\log s)\cdot \rand.$$
\end{theorem}


\subsubsection{Analysis: Proof of Theorem~\ref{thm:lp}}
First consider dependent rounding on $\{y_j\}$. Let $Y_j$ denote the random variable which returns $1$ if $y_j$ is rounded to $1$ and $0$ if $y_j$ is rounded to $0$. Note that $\E[Y_j] = y_j$ for all candidates $j \in \C$. The following lemma is an easy consequence of Chernoff bounds:

\begin{lemma}
\label{lem:bound}
For any subset of candidates $\{c_{j_1}, \ldots, c_{j_\ell}\}$, let $W = \sum_{t = 1}^\ell y_{j_t}$. If $W = \omega(1)$, then 
$$\Pr\left[\sum_{t = 1}^\ell Y_{j_t} \in \left(W - 9\sqrt{W\log W}, W + 9\sqrt{W\log W}\right)\right] \geq 1 - \frac{2}{W^3}.$$
\end{lemma}
%\begin{proof}
%Since  $\E[\sum_{t = 1}^\ell Y_{j_t}] = W$, by Chernoff Bounds applied to negatively dependent random variables, we have: $\Pr\left[ |\sum_{t = 1}^\ell Y_{j_t} - W | \geq 2 \sqrt{W\log W}\right] \leq \frac{2}{W^3}.$
%\qedhere
%\end{proof}

 We now compute the expected $s$-Borda score for each voter. Towards this end, we partition the candidates into buckets with geometrically decreasing sum of $y_i$ values, and account for the expected score generated by dependent rounding in each bucket against the LP value of the subsequent bucket. Lemma~\ref{lem:bound} will ensure the number of candidates chosen from each bucket is close to the LP value, and the deficit gets taken care of by the uniformly randomly chosen candidates.

For simplicity of notation, let $\eta = \log_2 \frac{\sqrt{s}}{2}$. Fix a voter $v_i$, and suppose its preference order is $c_{i_1} \succ c_{i_2} \succ \ldots \succ c_{i_m}$. Recall that $\{\tilde{y}, x\}$ is the LP solution. The values $x_{ij}$ in the LP are set as follows: Consider the prefix of the ordering such that $\sum_{t=1}^{\ell} \tilde{y}_{i_t} \le s$ and $\sum_{t=1}^{\ell+1} \tilde{y}_{i_t} > s$. The LP sets $x_{i i_t} = \tilde{y}_{i_t}$ for $t \le \ell$, and sets $x_{i i_{\ell+1}} = s - \sum_{t=1}^{\ell} \tilde{y}_{i_t}$. The contribution of $v_i$ to the LP objective is therefore
\begin{equation}
    \label{eq:opt1}
\opt_i =  \sum_{t=1}^{\ell} r_{v_i}(c_{i_t}) \tilde{y}_{i_t} + r_{v_i}(c_{i_{\ell+1}}) \left(s - \sum_{t=1}^{\ell} \tilde{y}_{i_t} \right) \ge \sum_{t=1}^{\ell} r_{v_i}(c_{i_t}) \tilde{y}_{i_t}.
\end{equation}

Consider the first $\ell$ candidates in the above ordering. We have $\sum_{j=1}^{\ell} \tilde{y}_j \ge s - 1$, so that
\begin{equation}
    \label{eq:sumy}
\sum_{j=1}^{\ell} y_j \ge (s-1) \left(1 - \frac{1}{\sqrt{s}} \right) \ge s - (\sqrt{s}+1) \ge s - 2 \sqrt{s}.
\end{equation}
We split these $\ell$ candidates into sets $\C_1, \ldots, \C_{\eta}$ as follows: We walk down the preference order of $v_i$. We take $\C_1$ as the set of candidates whose $y$-values sum to $\frac{s}{2}$;  $\C_2$ as the next set of candidates whose  $y$-values sum to $\frac{s}{4}$, and so on until $\C_{\eta}$, whose sum of $y$-values is $\frac{s}{2^\eta} = 2\sqrt{s}$.  Now the sum of $y$-values of all candidates in $\{\C_1, \ldots, \C_{\eta}\}$ is exactly $s - 2\sqrt{s}$. Formally, we define 
\[
\theta_0 = 0, \qquad \theta_h = \min\left\{q \ \bigg| \ \sum_{t = 1}^q y_{i_t} \geq \left(1 - \frac{1}{2^h}\right)s \right\}, \:\:\forall h \in \{1, \ldots, \eta\},
\]
and correspondingly define the sets $\{\C_1, \ldots, \C_{\eta}\}$ as:
$$\C_h = \{c_{i_{q}} \mid \theta_{h - 1} < q \leq \theta_h\}, \:\: \forall h \in \:\{1, \ldots, \eta\}.$$

For all $h \in \{1, \ldots, \eta\}$, let $y_{\C_h} = \sum_{c_j \in \C_h}y_j$ and  $Y_{\C_h} = \sum_{c_j \in \C_h}Y_j$. Note that $y_{\C_h}$ decreases by a factor of $2$ as $h$ increases. Now consider the outcome of the dependent rounding procedure for each of the sets $\C_1, \ldots \C_{\eta}$. We say the rounding fails for $v_i$ if there exists $h \in \{1, \ldots, \eta-1\}$ such that the number $Y_{\C_h}$ of chosen candidates  in $\C_h$ is not in range $y_{\C_h} \pm 9 \sqrt{y_{\C_h}\log y_{\C_h}}$. We will not consider $\C_{\eta}$ when defining failure, and will deal with this set separately. 

Let $\F$ denote the failure event. We now bound the probability of the event $\F$ for voter $v_i$.
\begin{lemma}
$\Pr[\F] \leq s^{-3/2}.$
\end{lemma}
\begin{proof}
By union bound applied to Lemma~\ref{lem:bound}, we have:
\begin{align*}
\Pr[\F] & \leq \sum_{h = 1}^{\eta-1}\frac{2}{y_{\C_h}^3} \leq \frac{3}{y_{\C_{\eta-1}}^3} \le s^{-3/2},
\end{align*}
where we have used that $\{y_{\C_h}\}$ is a geometrically decreasing sequence, and that $y_{\C_{\eta-1}} = 4 \sqrt{s}$. 
\end{proof}

We are now ready to compute the expected score for $v_i$ in our algorithm. Recall that $T$ denotes the set of chosen candidates and $\opt_i$ denotes the $s$-Borda score for $v_i$ in the LP solution. Let $\mathsc{Bad}$ denote the expected $s$-Borda score for $v_i$ in the event $\F$, and $\mathsc{Good}$ denote the expected score otherwise. We will bound these separately below.

\begin{lemma}
\label{lem:bad}
$\mathsc{Bad} \le O(s^{5/2})\cdot \rand$.
\end{lemma}
\begin{proof}
If $\F$ happens, the final solution is still at least as good as choosing the $\frac{k}{\sqrt{s}}$ random candidates in Step (4) of the algorithm. Note that since we assumed $k = \omega(s^{3/2})$, we have $\frac{k}{\sqrt{s}} \ge s$, so that we will have chosen enough random candidates to fill up at least $s$ positions for computing $s$-Borda score. Further, since we assume that $m =  \omega(k)$, the score of the solution will at most double had we assumed these $\frac{k}{\sqrt{s}}$ candidates are chosen randomly from the entire set of $m$ candidates instead of from the remaining $m - k(1 - \frac{1}{\sqrt{s}})$ candidates after dependent rounding. Thus, we have:
$$\mathsc{Bad} \leq 2 \E_{T \subseteq C, |T| = k/\sqrt{s}}[r_{v_i}(T)] \le 2 \frac{s(s+1)}{2} \frac{m+1}{\frac{k}{\sqrt{s}} + 1}\leq 4s^{3/2}(s + 1)\cdot \rand,$$
which yields that $\mathsc{Bad} \leq O(s^{5/2})\cdot \rand$.
\end{proof}

\begin{lemma}
\label{lem:good}
$\mathsc{Good} \leq 3\opt_i +O(s^{3/2}\log s)\cdot \rand$.
\end{lemma}
\begin{proof}
Suppose $\F$ does not happen. Denote the set of candidates chosen by the algorithm from $\{\C_1, \ldots, \C_{\eta-1}\}$ as $T_1$, and the randomly chosen $\frac{k}{\sqrt{s}}$ candidates as $T_2$. Therefore $T = T_1 \cup T_2$. From Eq~(\ref{eq:sumy}), we have 
$$\sum_{h=1}^{\eta-1} y_{\C_h} = s - 2\sqrt{s} - y_{\C_{\eta}} = s - 4\sqrt{s}.$$ 
Since $\F$ does not happen, we have:
$$|T_1| \geq s - 4 \sqrt{s} - \sum_{h = 1}^{\eta-1} \sqrt{y_{\C_h}\log y_{\C_h}} \geq s - 4 \sqrt{s} - \sum_{h = 1}^{\eta-1} \sqrt{\frac{s}{2^h}\log s} \geq s - 4 \sqrt{s} - (\sqrt{2} + 1)\sqrt{s\log s}.$$

Denote $u = 4 \sqrt{s} + (\sqrt{2} + 1)\sqrt{s\log s} = O(\sqrt{s\log s})$, so that $|T_1| \ge s - u$.  The quantity $u$ is the total ``deficit'' in candidates from the top $s$ that is caused by scaling the LP and dependent rounding. We make up this deficit using the set $T_2$. Specifically, consider the subsets, 
$$T_1^* = \argmin_{Q \subseteq T_1, |Q| = s - u}\sum_{c \in Q}r_{v_i}(c) \qquad \mbox{and} \qquad T_2^* = \argmin_{Q \subseteq T_2, |Q| = u}\sum_{c \in Q}r_{v_i}(c).$$ 
Note that $T_1^* \subseteq T_1$, and $T_2^* \subseteq T_2$. We will evaluate the score of these subsets of candidates, which will be an upper bound on the score of the algorithm. Towards this end, we define $\mu_h$ as the scaled LP score of $\C_{h}$, that is:
$$\mu_h = \frac{\sum_{j = \theta_{h - 1} + 1}^{\theta_h}r_{v_i}(c_{i_j})y_{i_j}}{y_{\C_h}}, \:\: \forall h \in \{1, \ldots, \eta\}.$$

Since $y_i \le \tilde{y}_i$, combining the previous inequality with Eq~(\ref{eq:opt1}), we have:
$$\sum_{h = 1}^{\eta}\mu_hy_{\C_h} \le \opt_i.$$

Since $y_{\C_h} \ge 2\sqrt{s} = \omega(1)$ for all $h \in \{1,2,\ldots, \eta\}$, we have:
$$|T_1 \cap \C_h| \le y_{\C_h} + 9 \sqrt{y_{\C_h} \log y_{\C_h}} \le \frac{3}{2} y_{\C_h}.$$ 

Since $\mu_h > r_{v_i}(c), \forall c \in \C_{h - 1}$ and since $y_{\C_h} \le 2 y_{\C_{h+1}}$, we can bound the expected score of $T_1^*$ as:
$$\sum_{c \in T_1^*}r_{v_i}(c) \leq \sum_{h = 1}^{\eta-1} \frac{3}{2} \cdot y_{\C_h}\mu_{h + 1} \le \sum_{h = 1}^{\eta-1} 3\cdot y_{\C_{h + 1}}\mu_{h + 1} \le 3\opt_i.$$

We can again assume that the $\frac{k}{\sqrt{s}}$ random candidates are chosen randomly from the entire set of $m$ candidates. This yields a bound on the score of $T_2^*$ as:
$$\sum_{c \in T_2^*}r_{v_i}(c) \leq \sum_{t = 1}^u t \cdot(\sqrt{s}\cdot \rand) = O(s^{3/2}\log s)\cdot \rand.$$
where we used $u = O(\sqrt{s\log s})$ to derive $\sum_{t = 1}^u t = O(s\log s)$. 

Therefore, we can bound $\mathsc{Good}$ as follows:
\[\mathsc{Good} \leq \sum_{c \in T_1^*}r_{v_i}(c) + \sum_{c \in T_2^*}r_{v_i}(c) \leq 3\opt_i + O(s^{3/2}\log s)\cdot \rand.\qedhere\]
\end{proof}

Synthesizing the bounds from Lemmas~\ref{lem:bad} and~\ref{lem:good}, we can conclude that:
\begin{align*}
\E[r_{v_i}(T)] &= \mathsc{Bad}\cdot \Pr[\F] + \mathsc{Good}\cdot (1-\Pr[\F])\\
%&\leq \mathsc{Bad}\cdot \Pr[\F] + \mathsc{Good} \\
&\leq  s^{-3/2} \cdot O(s^{5/2})\cdot \rand + 3\opt_i + O(s^{3/2}\log s)\cdot \rand \\
&= 3\opt_i + O(s^{3/2}\log s)\cdot \rand.
\qedhere
\end{align*}

Taking expectation over all voters, this yields Theorem~\ref{thm:lp}.

\section{Conclusions}
\label{sec:conclusions}

In this paper, we apply shared-workload techniques at the \sql level for
improving the throughput of \qaasl systems without incurring in additional
query execution costs. Our approach is based on query rewriting for grouping
multiple queries together into a single query to be executed in one go. This
results in a significant reduction of the aggregated data access done by the
shared execution compared to executing queries independently.

%execution times and costs of the shared scan operator when
%varying query selectivity and predicate evaluation. We observed that for
%\athena, although the cost only depends on the amount of data read, it is
%conditioned to its ability to use its statistics about the data. In some cases
%a wrong query execution plan leads to higher query execution costs, which the
%end-user has to pay. 

%\bigquery's minimum query execution cost is determined by
%the input size of a query.  However, the query cost can increase depending not
%just in the amount of computation it requires, but in the mix of resources the
%query requires.  

We presented a cost and runtime evaluation of the shared operator driving data access costs. 
Our experimental study using the TPC-H benchmark confirmed the benefits of our
query rewrite approach. Using a shared execution approach reduces significantly
the execution costs. For \athena, we are able to make it 107x cheaper and for
\bigquery, 16x cheaper taking into account Query 10 which we cannot execute,
but 128x if it is not taken into account. Moreover, when having queries that do
not share their entire execution plan, i.e., using a single global plan, we
demonstrated that it is possible to improve throughput and obtain a 10x cost
reduction in \bigquery.

%We followed the TPC systems pricing guideline for
%computing how expensive is to have a TPC-H workload working on the evaluated
%\qaasl systems. The result is that even though we are able to reduce overall
%costs a TPC-H workload in 15x for \bigquery (128x excluding query 10 which we
%could not optimize) and in 107x for \athena, the overall price is at least 10x
%more expensive than the cheapest system price published by the TPC.

There are multiple ways to extend our work. The first is
to implement a full \sql-to-\sql translation layer to encapsulate the proposed
per-operator rewrites.  Another one is to incorporate the initial work on
building a cost-based optimizer for shared execution
\cite{Giannikis:2014:SWO:2732279.2732280} as an external component for \qaasl
systems.  Moreover, incorporating different lines of work (e.g., adding
provenance computation \cite{GA09} capabilities) also based on query
rewriting is part of our future work to enhance our system.


\section*{Acknowledgments}
We thank Brandon Fain for several discussions, and the anonymous reviewers for their suggestions. This work is supported by NSF grant CCF-1637397, ONR award N00014-19-1-2268, and DARPA award FA8650-18-C-7880.


\bibliographystyle{plain}
\bibliography{ref}


\appendix
\chapter{Supplementary Material}
\label{appendix}

In this appendix, we present supplementary material for the techniques and
experiments presented in the main text.

\section{Baseline Results and Analysis for Informed Sampler}
\label{appendix:chap3}

Here, we give an in-depth
performance analysis of the various samplers and the effect of their
hyperparameters. We choose hyperparameters with the lowest PSRF value
after $10k$ iterations, for each sampler individually. If the
differences between PSRF are not significantly different among
multiple values, we choose the one that has the highest acceptance
rate.

\subsection{Experiment: Estimating Camera Extrinsics}
\label{appendix:chap3:room}

\subsubsection{Parameter Selection}
\paragraph{Metropolis Hastings (\MH)}

Figure~\ref{fig:exp1_MH} shows the median acceptance rates and PSRF
values corresponding to various proposal standard deviations of plain
\MH~sampling. Mixing gets better and the acceptance rate gets worse as
the standard deviation increases. The value $0.3$ is selected standard
deviation for this sampler.

\paragraph{Metropolis Hastings Within Gibbs (\MHWG)}

As mentioned in Section~\ref{sec:room}, the \MHWG~sampler with one-dimensional
updates did not converge for any value of proposal standard deviation.
This problem has high correlation of the camera parameters and is of
multi-modal nature, which this sampler has problems with.

\paragraph{Parallel Tempering (\PT)}

For \PT~sampling, we took the best performing \MH~sampler and used
different temperature chains to improve the mixing of the
sampler. Figure~\ref{fig:exp1_PT} shows the results corresponding to
different combination of temperature levels. The sampler with
temperature levels of $[1,3,27]$ performed best in terms of both
mixing and acceptance rate.

\paragraph{Effect of Mixture Coefficient in Informed Sampling (\MIXLMH)}

Figure~\ref{fig:exp1_alpha} shows the effect of mixture
coefficient ($\alpha$) on the informed sampling
\MIXLMH. Since there is no significant different in PSRF values for
$0 \le \alpha \le 0.7$, we chose $0.7$ due to its high acceptance
rate.


% \end{multicols}

\begin{figure}[h]
\centering
  \subfigure[MH]{%
    \includegraphics[width=.48\textwidth]{figures/supplementary/camPose_MH.pdf} \label{fig:exp1_MH}
  }
  \subfigure[PT]{%
    \includegraphics[width=.48\textwidth]{figures/supplementary/camPose_PT.pdf} \label{fig:exp1_PT}
  }
\\
  \subfigure[INF-MH]{%
    \includegraphics[width=.48\textwidth]{figures/supplementary/camPose_alpha.pdf} \label{fig:exp1_alpha}
  }
  \mycaption{Results of the `Estimating Camera Extrinsics' experiment}{PRSFs and Acceptance rates corresponding to (a) various standard deviations of \MH, (b) various temperature level combinations of \PT sampling and (c) various mixture coefficients of \MIXLMH sampling.}
\end{figure}



\begin{figure}[!t]
\centering
  \subfigure[\MH]{%
    \includegraphics[width=.48\textwidth]{figures/supplementary/occlusionExp_MH.pdf} \label{fig:exp2_MH}
  }
  \subfigure[\BMHWG]{%
    \includegraphics[width=.48\textwidth]{figures/supplementary/occlusionExp_BMHWG.pdf} \label{fig:exp2_BMHWG}
  }
\\
  \subfigure[\MHWG]{%
    \includegraphics[width=.48\textwidth]{figures/supplementary/occlusionExp_MHWG.pdf} \label{fig:exp2_MHWG}
  }
  \subfigure[\PT]{%
    \includegraphics[width=.48\textwidth]{figures/supplementary/occlusionExp_PT.pdf} \label{fig:exp2_PT}
  }
\\
  \subfigure[\INFBMHWG]{%
    \includegraphics[width=.5\textwidth]{figures/supplementary/occlusionExp_alpha.pdf} \label{fig:exp2_alpha}
  }
  \mycaption{Results of the `Occluding Tiles' experiment}{PRSF and
    Acceptance rates corresponding to various standard deviations of
    (a) \MH, (b) \BMHWG, (c) \MHWG, (d) various temperature level
    combinations of \PT~sampling and; (e) various mixture coefficients
    of our informed \INFBMHWG sampling.}
\end{figure}

%\onecolumn\newpage\twocolumn
\subsection{Experiment: Occluding Tiles}
\label{appendix:chap3:tiles}

\subsubsection{Parameter Selection}

\paragraph{Metropolis Hastings (\MH)}

Figure~\ref{fig:exp2_MH} shows the results of
\MH~sampling. Results show the poor convergence for all proposal
standard deviations and rapid decrease of AR with increasing standard
deviation. This is due to the high-dimensional nature of
the problem. We selected a standard deviation of $1.1$.

\paragraph{Blocked Metropolis Hastings Within Gibbs (\BMHWG)}

The results of \BMHWG are shown in Figure~\ref{fig:exp2_BMHWG}. In
this sampler we update only one block of tile variables (of dimension
four) in each sampling step. Results show much better performance
compared to plain \MH. The optimal proposal standard deviation for
this sampler is $0.7$.

\paragraph{Metropolis Hastings Within Gibbs (\MHWG)}

Figure~\ref{fig:exp2_MHWG} shows the result of \MHWG sampling. This
sampler is better than \BMHWG and converges much more quickly. Here
a standard deviation of $0.9$ is found to be best.

\paragraph{Parallel Tempering (\PT)}

Figure~\ref{fig:exp2_PT} shows the results of \PT sampling with various
temperature combinations. Results show no improvement in AR from plain
\MH sampling and again $[1,3,27]$ temperature levels are found to be optimal.

\paragraph{Effect of Mixture Coefficient in Informed Sampling (\INFBMHWG)}

Figure~\ref{fig:exp2_alpha} shows the effect of mixture
coefficient ($\alpha$) on the blocked informed sampling
\INFBMHWG. Since there is no significant different in PSRF values for
$0 \le \alpha \le 0.8$, we chose $0.8$ due to its high acceptance
rate.



\subsection{Experiment: Estimating Body Shape}
\label{appendix:chap3:body}

\subsubsection{Parameter Selection}
\paragraph{Metropolis Hastings (\MH)}

Figure~\ref{fig:exp3_MH} shows the result of \MH~sampling with various
proposal standard deviations. The value of $0.1$ is found to be
best.

\paragraph{Metropolis Hastings Within Gibbs (\MHWG)}

For \MHWG sampling we select $0.3$ proposal standard
deviation. Results are shown in Fig.~\ref{fig:exp3_MHWG}.


\paragraph{Parallel Tempering (\PT)}

As before, results in Fig.~\ref{fig:exp3_PT}, the temperature levels
were selected to be $[1,3,27]$ due its slightly higher AR.

\paragraph{Effect of Mixture Coefficient in Informed Sampling (\MIXLMH)}

Figure~\ref{fig:exp3_alpha} shows the effect of $\alpha$ on PSRF and
AR. Since there is no significant differences in PSRF values for $0 \le
\alpha \le 0.8$, we choose $0.8$.


\begin{figure}[t]
\centering
  \subfigure[\MH]{%
    \includegraphics[width=.48\textwidth]{figures/supplementary/bodyShape_MH.pdf} \label{fig:exp3_MH}
  }
  \subfigure[\MHWG]{%
    \includegraphics[width=.48\textwidth]{figures/supplementary/bodyShape_MHWG.pdf} \label{fig:exp3_MHWG}
  }
\\
  \subfigure[\PT]{%
    \includegraphics[width=.48\textwidth]{figures/supplementary/bodyShape_PT.pdf} \label{fig:exp3_PT}
  }
  \subfigure[\MIXLMH]{%
    \includegraphics[width=.48\textwidth]{figures/supplementary/bodyShape_alpha.pdf} \label{fig:exp3_alpha}
  }
\\
  \mycaption{Results of the `Body Shape Estimation' experiment}{PRSFs and
    Acceptance rates corresponding to various standard deviations of
    (a) \MH, (b) \MHWG; (c) various temperature level combinations
    of \PT sampling and; (d) various mixture coefficients of the
    informed \MIXLMH sampling.}
\end{figure}


\subsection{Results Overview}
Figure~\ref{fig:exp_summary} shows the summary results of the all the three
experimental studies related to informed sampler.
\begin{figure*}[h!]
\centering
  \subfigure[Results for: Estimating Camera Extrinsics]{%
    \includegraphics[width=0.9\textwidth]{figures/supplementary/camPose_ALL.pdf} \label{fig:exp1_all}
  }
  \subfigure[Results for: Occluding Tiles]{%
    \includegraphics[width=0.9\textwidth]{figures/supplementary/occlusionExp_ALL.pdf} \label{fig:exp2_all}
  }
  \subfigure[Results for: Estimating Body Shape]{%
    \includegraphics[width=0.9\textwidth]{figures/supplementary/bodyShape_ALL.pdf} \label{fig:exp3_all}
  }
  \label{fig:exp_summary}
  \mycaption{Summary of the statistics for the three experiments}{Shown are
    for several baseline methods and the informed samplers the
    acceptance rates (left), PSRFs (middle), and RMSE values
    (right). All results are median results over multiple test
    examples.}
\end{figure*}

\subsection{Additional Qualitative Results}

\subsubsection{Occluding Tiles}
In Figure~\ref{fig:exp2_visual_more} more qualitative results of the
occluding tiles experiment are shown. The informed sampling approach
(\INFBMHWG) is better than the best baseline (\MHWG). This still is a
very challenging problem since the parameters for occluded tiles are
flat over a large region. Some of the posterior variance of the
occluded tiles is already captured by the informed sampler.

\begin{figure*}[h!]
\begin{center}
\centerline{\includegraphics[width=0.95\textwidth]{figures/supplementary/occlusionExp_Visual.pdf}}
\mycaption{Additional qualitative results of the occluding tiles experiment}
  {From left to right: (a)
  Given image, (b) Ground truth tiles, (c) OpenCV heuristic and most probable estimates
  from 5000 samples obtained by (d) MHWG sampler (best baseline) and
  (e) our INF-BMHWG sampler. (f) Posterior expectation of the tiles
  boundaries obtained by INF-BMHWG sampling (First 2000 samples are
  discarded as burn-in).}
\label{fig:exp2_visual_more}
\end{center}
\end{figure*}

\subsubsection{Body Shape}
Figure~\ref{fig:exp3_bodyMeshes} shows some more results of 3D mesh
reconstruction using posterior samples obtained by our informed
sampling \MIXLMH.

\begin{figure*}[t]
\begin{center}
\centerline{\includegraphics[width=0.75\textwidth]{figures/supplementary/bodyMeshResults.pdf}}
\mycaption{Qualitative results for the body shape experiment}
  {Shown is the 3D mesh reconstruction results with first 1000 samples obtained
  using the \MIXLMH informed sampling method. (blue indicates small
  values and red indicates high values)}
\label{fig:exp3_bodyMeshes}
\end{center}
\end{figure*}

\clearpage



\section{Additional Results on the Face Problem with CMP}

Figure~\ref{fig:shading-qualitative-multiple-subjects-supp} shows inference results for reflectance maps, normal maps and lights for randomly chosen test images, and Fig.~\ref{fig:shading-qualitative-same-subject-supp} shows reflectance estimation results on multiple images of the same subject produced under different illumination conditions. CMP is able to produce estimates that are closer to the groundtruth across different subjects and illumination conditions.

\begin{figure*}[h]
  \begin{center}
  \centerline{\includegraphics[width=1.0\columnwidth]{figures/face_cmp_visual_results_supp.pdf}}
  \vspace{-1.2cm}
  \end{center}
	\mycaption{A visual comparison of inference results}{(a)~Observed images. (b)~Inferred reflectance maps. \textit{GT} is the photometric stereo groundtruth, \textit{BU} is the Biswas \etal (2009) reflectance estimate and \textit{Forest} is the consensus prediction. (c)~The variance of the inferred reflectance estimate produced by \MTD (normalized across rows).(d)~Visualization of inferred light directions. (e)~Inferred normal maps.}
	\label{fig:shading-qualitative-multiple-subjects-supp}
\end{figure*}


\begin{figure*}[h]
	\centering
	\setlength\fboxsep{0.2mm}
	\setlength\fboxrule{0pt}
	\begin{tikzpicture}

		\matrix at (0, 0) [matrix of nodes, nodes={anchor=east}, column sep=-0.05cm, row sep=-0.2cm]
		{
			\fbox{\includegraphics[width=1cm]{figures/sample_3_4_X.png}} &
			\fbox{\includegraphics[width=1cm]{figures/sample_3_4_GT.png}} &
			\fbox{\includegraphics[width=1cm]{figures/sample_3_4_BISWAS.png}}  &
			\fbox{\includegraphics[width=1cm]{figures/sample_3_4_VMP.png}}  &
			\fbox{\includegraphics[width=1cm]{figures/sample_3_4_FOREST.png}}  &
			\fbox{\includegraphics[width=1cm]{figures/sample_3_4_CMP.png}}  &
			\fbox{\includegraphics[width=1cm]{figures/sample_3_4_CMPVAR.png}}
			 \\

			\fbox{\includegraphics[width=1cm]{figures/sample_3_5_X.png}} &
			\fbox{\includegraphics[width=1cm]{figures/sample_3_5_GT.png}} &
			\fbox{\includegraphics[width=1cm]{figures/sample_3_5_BISWAS.png}}  &
			\fbox{\includegraphics[width=1cm]{figures/sample_3_5_VMP.png}}  &
			\fbox{\includegraphics[width=1cm]{figures/sample_3_5_FOREST.png}}  &
			\fbox{\includegraphics[width=1cm]{figures/sample_3_5_CMP.png}}  &
			\fbox{\includegraphics[width=1cm]{figures/sample_3_5_CMPVAR.png}}
			 \\

			\fbox{\includegraphics[width=1cm]{figures/sample_3_6_X.png}} &
			\fbox{\includegraphics[width=1cm]{figures/sample_3_6_GT.png}} &
			\fbox{\includegraphics[width=1cm]{figures/sample_3_6_BISWAS.png}}  &
			\fbox{\includegraphics[width=1cm]{figures/sample_3_6_VMP.png}}  &
			\fbox{\includegraphics[width=1cm]{figures/sample_3_6_FOREST.png}}  &
			\fbox{\includegraphics[width=1cm]{figures/sample_3_6_CMP.png}}  &
			\fbox{\includegraphics[width=1cm]{figures/sample_3_6_CMPVAR.png}}
			 \\
	     };

       \node at (-3.85, -2.0) {\small Observed};
       \node at (-2.55, -2.0) {\small `GT'};
       \node at (-1.27, -2.0) {\small BU};
       \node at (0.0, -2.0) {\small MP};
       \node at (1.27, -2.0) {\small Forest};
       \node at (2.55, -2.0) {\small \textbf{CMP}};
       \node at (3.85, -2.0) {\small Variance};

	\end{tikzpicture}
	\mycaption{Robustness to varying illumination}{Reflectance estimation on a subject images with varying illumination. Left to right: observed image, photometric stereo estimate (GT)
  which is used as a proxy for groundtruth, bottom-up estimate of \cite{Biswas2009}, VMP result, consensus forest estimate, CMP mean, and CMP variance.}
	\label{fig:shading-qualitative-same-subject-supp}
\end{figure*}

\clearpage

\section{Additional Material for Learning Sparse High Dimensional Filters}
\label{sec:appendix-bnn}

This part of supplementary material contains a more detailed overview of the permutohedral
lattice convolution in Section~\ref{sec:permconv}, more experiments in
Section~\ref{sec:addexps} and additional results with protocols for
the experiments presented in Chapter~\ref{chap:bnn} in Section~\ref{sec:addresults}.

\vspace{-0.2cm}
\subsection{General Permutohedral Convolutions}
\label{sec:permconv}

A core technical contribution of this work is the generalization of the Gaussian permutohedral lattice
convolution proposed in~\cite{adams2010fast} to the full non-separable case with the
ability to perform back-propagation. Although, conceptually, there are minor
differences between Gaussian and general parameterized filters, there are non-trivial practical
differences in terms of the algorithmic implementation. The Gauss filters belong to
the separable class and can thus be decomposed into multiple
sequential one dimensional convolutions. We are interested in the general filter
convolutions, which can not be decomposed. Thus, performing a general permutohedral
convolution at a lattice point requires the computation of the inner product with the
neighboring elements in all the directions in the high-dimensional space.

Here, we give more details of the implementation differences of separable
and non-separable filters. In the following, we will explain the scalar case first.
Recall, that the forward pass of general permutohedral convolution
involves 3 steps: \textit{splatting}, \textit{convolving} and \textit{slicing}.
We follow the same splatting and slicing strategies as in~\cite{adams2010fast}
since these operations do not depend on the filter kernel. The main difference
between our work and the existing implementation of~\cite{adams2010fast} is
the way that the convolution operation is executed. This proceeds by constructing
a \emph{blur neighbor} matrix $K$ that stores for every lattice point all
values of the lattice neighbors that are needed to compute the filter output.

\begin{figure}[t!]
  \centering
    \includegraphics[width=0.6\columnwidth]{figures/supplementary/lattice_construction}
  \mycaption{Illustration of 1D permutohedral lattice construction}
  {A $4\times 4$ $(x,y)$ grid lattice is projected onto the plane defined by the normal
  vector $(1,1)^{\top}$. This grid has $s+1=4$ and $d=2$ $(s+1)^{d}=4^2=16$ elements.
  In the projection, all points of the same color are projected onto the same points in the plane.
  The number of elements of the projected lattice is $t=(s+1)^d-s^d=4^2-3^2=7$, that is
  the $(4\times 4)$ grid minus the size of lattice that is $1$ smaller at each size, in this
  case a $(3\times 3)$ lattice (the upper right $(3\times 3)$ elements).
  }
\label{fig:latticeconstruction}
\end{figure}

The blur neighbor matrix is constructed by traversing through all the populated
lattice points and their neighboring elements.
% For efficiency, we do this matrix construction recursively with shared computations
% since $n^{th}$ neighbourhood elements are $1^{st}$ neighborhood elements of $n-1^{th}$ neighbourhood elements. \pg{do not understand}
This is done recursively to share computations. For any lattice point, the neighbors that are
$n$ hops away are the direct neighbors of the points that are $n-1$ hops away.
The size of a $d$ dimensional spatial filter with width $s+1$ is $(s+1)^{d}$ (\eg, a
$3\times 3$ filter, $s=2$ in $d=2$ has $3^2=9$ elements) and this size grows
exponentially in the number of dimensions $d$. The permutohedral lattice is constructed by
projecting a regular grid onto the plane spanned by the $d$ dimensional normal vector ${(1,\ldots,1)}^{\top}$. See
Fig.~\ref{fig:latticeconstruction} for an illustration of the 1D lattice construction.
Many corners of a grid filter are projected onto the same point, in total $t = {(s+1)}^{d} -
s^{d}$ elements remain in the permutohedral filter with $s$ neighborhood in $d-1$ dimensions.
If the lattice has $m$ populated elements, the
matrix $K$ has size $t\times m$. Note that, since the input signal is typically
sparse, only a few lattice corners are being populated in the \textit{slicing} step.
We use a hash-table to keep track of these points and traverse only through
the populated lattice points for this neighborhood matrix construction.

Once the blur neighbor matrix $K$ is constructed, we can perform the convolution
by the matrix vector multiplication
\begin{equation}
\ell' = BK,
\label{eq:conv}
\end{equation}
where $B$ is the $1 \times t$ filter kernel (whose values we will learn) and $\ell'\in\mathbb{R}^{1\times m}$
is the result of the filtering at the $m$ lattice points. In practice, we found that the
matrix $K$ is sometimes too large to fit into GPU memory and we divided the matrix $K$
into smaller pieces to compute Eq.~\ref{eq:conv} sequentially.

In the general multi-dimensional case, the signal $\ell$ is of $c$ dimensions. Then
the kernel $B$ is of size $c \times t$ and $K$ stores the $c$ dimensional vectors
accordingly. When the input and output points are different, we slice only the
input points and splat only at the output points.


\subsection{Additional Experiments}
\label{sec:addexps}
In this section, we discuss more use-cases for the learned bilateral filters, one
use-case of BNNs and two single filter applications for image and 3D mesh denoising.

\subsubsection{Recognition of subsampled MNIST}\label{sec:app_mnist}

One of the strengths of the proposed filter convolution is that it does not
require the input to lie on a regular grid. The only requirement is to define a distance
between features of the input signal.
We highlight this feature with the following experiment using the
classical MNIST ten class classification problem~\cite{lecun1998mnist}. We sample a
sparse set of $N$ points $(x,y)\in [0,1]\times [0,1]$
uniformly at random in the input image, use their interpolated values
as signal and the \emph{continuous} $(x,y)$ positions as features. This mimics
sub-sampling of a high-dimensional signal. To compare against a spatial convolution,
we interpolate the sparse set of values at the grid positions.

We take a reference implementation of LeNet~\cite{lecun1998gradient} that
is part of the Caffe project~\cite{jia2014caffe} and compare it
against the same architecture but replacing the first convolutional
layer with a bilateral convolution layer (BCL). The filter size
and numbers are adjusted to get a comparable number of parameters
($5\times 5$ for LeNet, $2$-neighborhood for BCL).

The results are shown in Table~\ref{tab:all-results}. We see that training
on the original MNIST data (column Original, LeNet vs. BNN) leads to a slight
decrease in performance of the BNN (99.03\%) compared to LeNet
(99.19\%). The BNN can be trained and evaluated on sparse
signals, and we resample the image as described above for $N=$ 100\%, 60\% and
20\% of the total number of pixels. The methods are also evaluated
on test images that are subsampled in the same way. Note that we can
train and test with different subsampling rates. We introduce an additional
bilinear interpolation layer for the LeNet architecture to train on the same
data. In essence, both models perform a spatial interpolation and thus we
expect them to yield a similar classification accuracy. Once the data is of
higher dimensions, the permutohedral convolution will be faster due to hashing
the sparse input points, as well as less memory demanding in comparison to
naive application of a spatial convolution with interpolated values.

\begin{table}[t]
  \begin{center}
    \footnotesize
    \centering
    \begin{tabular}[t]{lllll}
      \toprule
              &     & \multicolumn{3}{c}{Test Subsampling} \\
       Method  & Original & 100\% & 60\% & 20\%\\
      \midrule
       LeNet &  \textbf{0.9919} & 0.9660 & 0.9348 & \textbf{0.6434} \\
       BNN &  0.9903 & \textbf{0.9844} & \textbf{0.9534} & 0.5767 \\
      \hline
       LeNet 100\% & 0.9856 & 0.9809 & 0.9678 & \textbf{0.7386} \\
       BNN 100\% & \textbf{0.9900} & \textbf{0.9863} & \textbf{0.9699} & 0.6910 \\
      \hline
       LeNet 60\% & 0.9848 & 0.9821 & 0.9740 & 0.8151 \\
       BNN 60\% & \textbf{0.9885} & \textbf{0.9864} & \textbf{0.9771} & \textbf{0.8214}\\
      \hline
       LeNet 20\% & \textbf{0.9763} & \textbf{0.9754} & 0.9695 & 0.8928 \\
       BNN 20\% & 0.9728 & 0.9735 & \textbf{0.9701} & \textbf{0.9042}\\
      \bottomrule
    \end{tabular}
  \end{center}
\vspace{-.2cm}
\caption{Classification accuracy on MNIST. We compare the
    LeNet~\cite{lecun1998gradient} implementation that is part of
    Caffe~\cite{jia2014caffe} to the network with the first layer
    replaced by a bilateral convolution layer (BCL). Both are trained
    on the original image resolution (first two rows). Three more BNN
    and CNN models are trained with randomly subsampled images (100\%,
    60\% and 20\% of the pixels). An additional bilinear interpolation
    layer samples the input signal on a spatial grid for the CNN model.
  }
  \label{tab:all-results}
\vspace{-.5cm}
\end{table}

\subsubsection{Image Denoising}

The main application that inspired the development of the bilateral
filtering operation is image denoising~\cite{aurich1995non}, there
using a single Gaussian kernel. Our development allows to learn this
kernel function from data and we explore how to improve using a \emph{single}
but more general bilateral filter.

We use the Berkeley segmentation dataset
(BSDS500)~\cite{arbelaezi2011bsds500} as a test bed. The color
images in the dataset are converted to gray-scale,
and corrupted with Gaussian noise with a standard deviation of
$\frac {25} {255}$.

We compare the performance of four different filter models on a
denoising task.
The first baseline model (`Spatial' in Table \ref{tab:denoising}, $25$
weights) uses a single spatial filter with a kernel size of
$5$ and predicts the scalar gray-scale value at the center pixel. The next model
(`Gauss Bilateral') applies a bilateral \emph{Gaussian}
filter to the noisy input, using position and intensity features $\f=(x,y,v)^\top$.
The third setup (`Learned Bilateral', $65$ weights)
takes a Gauss kernel as initialization and
fits all filter weights on the train set to minimize the
mean squared error with respect to the clean images.
We run a combination
of spatial and permutohedral convolutions on spatial and bilateral
features (`Spatial + Bilateral (Learned)') to check for a complementary
performance of the two convolutions.

\label{sec:exp:denoising}
\begin{table}[!h]
\begin{center}
  \footnotesize
  \begin{tabular}[t]{lr}
    \toprule
    Method & PSNR \\
    \midrule
    Noisy Input & $20.17$ \\
    Spatial & $26.27$ \\
    Gauss Bilateral & $26.51$ \\
    Learned Bilateral & $26.58$ \\
    Spatial + Bilateral (Learned) & \textbf{$26.65$} \\
    \bottomrule
  \end{tabular}
\end{center}
\vspace{-0.5em}
\caption{PSNR results of a denoising task using the BSDS500
  dataset~\cite{arbelaezi2011bsds500}}
\vspace{-0.5em}
\label{tab:denoising}
\end{table}
\vspace{-0.2em}

The PSNR scores evaluated on full images of the test set are
shown in Table \ref{tab:denoising}. We find that an untrained bilateral
filter already performs better than a trained spatial convolution
($26.27$ to $26.51$). A learned convolution further improve the
performance slightly. We chose this simple one-kernel setup to
validate an advantage of the generalized bilateral filter. A competitive
denoising system would employ RGB color information and also
needs to be properly adjusted in network size. Multi-layer perceptrons
have obtained state-of-the-art denoising results~\cite{burger12cvpr}
and the permutohedral lattice layer can readily be used in such an
architecture, which is intended future work.

\subsection{Additional results}
\label{sec:addresults}

This section contains more qualitative results for the experiments presented in Chapter~\ref{chap:bnn}.

\begin{figure*}[th!]
  \centering
    \includegraphics[width=\columnwidth,trim={5cm 2.5cm 5cm 4.5cm},clip]{figures/supplementary/lattice_viz.pdf}
    \vspace{-0.7cm}
  \mycaption{Visualization of the Permutohedral Lattice}
  {Sample lattice visualizations for different feature spaces. All pixels falling in the same simplex cell are shown with
  the same color. $(x,y)$ features correspond to image pixel positions, and $(r,g,b) \in [0,255]$ correspond
  to the red, green and blue color values.}
\label{fig:latticeviz}
\end{figure*}

\subsubsection{Lattice Visualization}

Figure~\ref{fig:latticeviz} shows sample lattice visualizations for different feature spaces.

\newcolumntype{L}[1]{>{\raggedright\let\newline\\\arraybackslash\hspace{0pt}}b{#1}}
\newcolumntype{C}[1]{>{\centering\let\newline\\\arraybackslash\hspace{0pt}}b{#1}}
\newcolumntype{R}[1]{>{\raggedleft\let\newline\\\arraybackslash\hspace{0pt}}b{#1}}

\subsubsection{Color Upsampling}\label{sec:color_upsampling}
\label{sec:col_upsample_extra}

Some images of the upsampling for the Pascal
VOC12 dataset are shown in Fig.~\ref{fig:Colour_upsample_visuals}. It is
especially the low level image details that are better preserved with
a learned bilateral filter compared to the Gaussian case.

\begin{figure*}[t!]
  \centering
    \subfigure{%
   \raisebox{2.0em}{
    \includegraphics[width=.06\columnwidth]{figures/supplementary/2007_004969.jpg}
   }
  }
  \subfigure{%
    \includegraphics[width=.17\columnwidth]{figures/supplementary/2007_004969_gray.pdf}
  }
  \subfigure{%
    \includegraphics[width=.17\columnwidth]{figures/supplementary/2007_004969_gt.pdf}
  }
  \subfigure{%
    \includegraphics[width=.17\columnwidth]{figures/supplementary/2007_004969_bicubic.pdf}
  }
  \subfigure{%
    \includegraphics[width=.17\columnwidth]{figures/supplementary/2007_004969_gauss.pdf}
  }
  \subfigure{%
    \includegraphics[width=.17\columnwidth]{figures/supplementary/2007_004969_learnt.pdf}
  }\\
    \subfigure{%
   \raisebox{2.0em}{
    \includegraphics[width=.06\columnwidth]{figures/supplementary/2007_003106.jpg}
   }
  }
  \subfigure{%
    \includegraphics[width=.17\columnwidth]{figures/supplementary/2007_003106_gray.pdf}
  }
  \subfigure{%
    \includegraphics[width=.17\columnwidth]{figures/supplementary/2007_003106_gt.pdf}
  }
  \subfigure{%
    \includegraphics[width=.17\columnwidth]{figures/supplementary/2007_003106_bicubic.pdf}
  }
  \subfigure{%
    \includegraphics[width=.17\columnwidth]{figures/supplementary/2007_003106_gauss.pdf}
  }
  \subfigure{%
    \includegraphics[width=.17\columnwidth]{figures/supplementary/2007_003106_learnt.pdf}
  }\\
  \setcounter{subfigure}{0}
  \small{
  \subfigure[Inp.]{%
  \raisebox{2.0em}{
    \includegraphics[width=.06\columnwidth]{figures/supplementary/2007_006837.jpg}
   }
  }
  \subfigure[Guidance]{%
    \includegraphics[width=.17\columnwidth]{figures/supplementary/2007_006837_gray.pdf}
  }
   \subfigure[GT]{%
    \includegraphics[width=.17\columnwidth]{figures/supplementary/2007_006837_gt.pdf}
  }
  \subfigure[Bicubic]{%
    \includegraphics[width=.17\columnwidth]{figures/supplementary/2007_006837_bicubic.pdf}
  }
  \subfigure[Gauss-BF]{%
    \includegraphics[width=.17\columnwidth]{figures/supplementary/2007_006837_gauss.pdf}
  }
  \subfigure[Learned-BF]{%
    \includegraphics[width=.17\columnwidth]{figures/supplementary/2007_006837_learnt.pdf}
  }
  }
  \vspace{-0.5cm}
  \mycaption{Color Upsampling}{Color $8\times$ upsampling results
  using different methods, from left to right, (a)~Low-resolution input color image (Inp.),
  (b)~Gray scale guidance image, (c)~Ground-truth color image; Upsampled color images with
  (d)~Bicubic interpolation, (e) Gauss bilateral upsampling and, (f)~Learned bilateral
  updampgling (best viewed on screen).}

\label{fig:Colour_upsample_visuals}
\end{figure*}

\subsubsection{Depth Upsampling}
\label{sec:depth_upsample_extra}

Figure~\ref{fig:depth_upsample_visuals} presents some more qualitative results comparing bicubic interpolation, Gauss
bilateral and learned bilateral upsampling on NYU depth dataset image~\cite{silberman2012indoor}.

\subsubsection{Character Recognition}\label{sec:app_character}

 Figure~\ref{fig:nnrecognition} shows the schematic of different layers
 of the network architecture for LeNet-7~\cite{lecun1998mnist}
 and DeepCNet(5, 50)~\cite{ciresan2012multi,graham2014spatially}. For the BNN variants, the first layer filters are replaced
 with learned bilateral filters and are learned end-to-end.

\subsubsection{Semantic Segmentation}\label{sec:app_semantic_segmentation}
\label{sec:semantic_bnn_extra}

Some more visual results for semantic segmentation are shown in Figure~\ref{fig:semantic_visuals}.
These include the underlying DeepLab CNN\cite{chen2014semantic} result (DeepLab),
the 2 step mean-field result with Gaussian edge potentials (+2stepMF-GaussCRF)
and also corresponding results with learned edge potentials (+2stepMF-LearnedCRF).
In general, we observe that mean-field in learned CRF leads to slightly dilated
classification regions in comparison to using Gaussian CRF thereby filling-in the
false negative pixels and also correcting some mis-classified regions.

\begin{figure*}[t!]
  \centering
    \subfigure{%
   \raisebox{2.0em}{
    \includegraphics[width=.06\columnwidth]{figures/supplementary/2bicubic}
   }
  }
  \subfigure{%
    \includegraphics[width=.17\columnwidth]{figures/supplementary/2given_image}
  }
  \subfigure{%
    \includegraphics[width=.17\columnwidth]{figures/supplementary/2ground_truth}
  }
  \subfigure{%
    \includegraphics[width=.17\columnwidth]{figures/supplementary/2bicubic}
  }
  \subfigure{%
    \includegraphics[width=.17\columnwidth]{figures/supplementary/2gauss}
  }
  \subfigure{%
    \includegraphics[width=.17\columnwidth]{figures/supplementary/2learnt}
  }\\
    \subfigure{%
   \raisebox{2.0em}{
    \includegraphics[width=.06\columnwidth]{figures/supplementary/32bicubic}
   }
  }
  \subfigure{%
    \includegraphics[width=.17\columnwidth]{figures/supplementary/32given_image}
  }
  \subfigure{%
    \includegraphics[width=.17\columnwidth]{figures/supplementary/32ground_truth}
  }
  \subfigure{%
    \includegraphics[width=.17\columnwidth]{figures/supplementary/32bicubic}
  }
  \subfigure{%
    \includegraphics[width=.17\columnwidth]{figures/supplementary/32gauss}
  }
  \subfigure{%
    \includegraphics[width=.17\columnwidth]{figures/supplementary/32learnt}
  }\\
  \setcounter{subfigure}{0}
  \small{
  \subfigure[Inp.]{%
  \raisebox{2.0em}{
    \includegraphics[width=.06\columnwidth]{figures/supplementary/41bicubic}
   }
  }
  \subfigure[Guidance]{%
    \includegraphics[width=.17\columnwidth]{figures/supplementary/41given_image}
  }
   \subfigure[GT]{%
    \includegraphics[width=.17\columnwidth]{figures/supplementary/41ground_truth}
  }
  \subfigure[Bicubic]{%
    \includegraphics[width=.17\columnwidth]{figures/supplementary/41bicubic}
  }
  \subfigure[Gauss-BF]{%
    \includegraphics[width=.17\columnwidth]{figures/supplementary/41gauss}
  }
  \subfigure[Learned-BF]{%
    \includegraphics[width=.17\columnwidth]{figures/supplementary/41learnt}
  }
  }
  \mycaption{Depth Upsampling}{Depth $8\times$ upsampling results
  using different upsampling strategies, from left to right,
  (a)~Low-resolution input depth image (Inp.),
  (b)~High-resolution guidance image, (c)~Ground-truth depth; Upsampled depth images with
  (d)~Bicubic interpolation, (e) Gauss bilateral upsampling and, (f)~Learned bilateral
  updampgling (best viewed on screen).}

\label{fig:depth_upsample_visuals}
\end{figure*}

\subsubsection{Material Segmentation}\label{sec:app_material_segmentation}
\label{sec:material_bnn_extra}

In Fig.~\ref{fig:material_visuals-app2}, we present visual results comparing 2 step
mean-field inference with Gaussian and learned pairwise CRF potentials. In
general, we observe that the pixels belonging to dominant classes in the
training data are being more accurately classified with learned CRF. This leads to
a significant improvements in overall pixel accuracy. This also results
in a slight decrease of the accuracy from less frequent class pixels thereby
slightly reducing the average class accuracy with learning. We attribute this
to the type of annotation that is available for this dataset, which is not
for the entire image but for some segments in the image. We have very few
images of the infrequent classes to combat this behaviour during training.

\subsubsection{Experiment Protocols}
\label{sec:protocols}

Table~\ref{tbl:parameters} shows experiment protocols of different experiments.

 \begin{figure*}[t!]
  \centering
  \subfigure[LeNet-7]{
    \includegraphics[width=0.7\columnwidth]{figures/supplementary/lenet_cnn_network}
    }\\
    \subfigure[DeepCNet]{
    \includegraphics[width=\columnwidth]{figures/supplementary/deepcnet_cnn_network}
    }
  \mycaption{CNNs for Character Recognition}
  {Schematic of (top) LeNet-7~\cite{lecun1998mnist} and (bottom) DeepCNet(5,50)~\cite{ciresan2012multi,graham2014spatially} architectures used in Assamese
  character recognition experiments.}
\label{fig:nnrecognition}
\end{figure*}

\definecolor{voc_1}{RGB}{0, 0, 0}
\definecolor{voc_2}{RGB}{128, 0, 0}
\definecolor{voc_3}{RGB}{0, 128, 0}
\definecolor{voc_4}{RGB}{128, 128, 0}
\definecolor{voc_5}{RGB}{0, 0, 128}
\definecolor{voc_6}{RGB}{128, 0, 128}
\definecolor{voc_7}{RGB}{0, 128, 128}
\definecolor{voc_8}{RGB}{128, 128, 128}
\definecolor{voc_9}{RGB}{64, 0, 0}
\definecolor{voc_10}{RGB}{192, 0, 0}
\definecolor{voc_11}{RGB}{64, 128, 0}
\definecolor{voc_12}{RGB}{192, 128, 0}
\definecolor{voc_13}{RGB}{64, 0, 128}
\definecolor{voc_14}{RGB}{192, 0, 128}
\definecolor{voc_15}{RGB}{64, 128, 128}
\definecolor{voc_16}{RGB}{192, 128, 128}
\definecolor{voc_17}{RGB}{0, 64, 0}
\definecolor{voc_18}{RGB}{128, 64, 0}
\definecolor{voc_19}{RGB}{0, 192, 0}
\definecolor{voc_20}{RGB}{128, 192, 0}
\definecolor{voc_21}{RGB}{0, 64, 128}
\definecolor{voc_22}{RGB}{128, 64, 128}

\begin{figure*}[t]
  \centering
  \small{
  \fcolorbox{white}{voc_1}{\rule{0pt}{6pt}\rule{6pt}{0pt}} Background~~
  \fcolorbox{white}{voc_2}{\rule{0pt}{6pt}\rule{6pt}{0pt}} Aeroplane~~
  \fcolorbox{white}{voc_3}{\rule{0pt}{6pt}\rule{6pt}{0pt}} Bicycle~~
  \fcolorbox{white}{voc_4}{\rule{0pt}{6pt}\rule{6pt}{0pt}} Bird~~
  \fcolorbox{white}{voc_5}{\rule{0pt}{6pt}\rule{6pt}{0pt}} Boat~~
  \fcolorbox{white}{voc_6}{\rule{0pt}{6pt}\rule{6pt}{0pt}} Bottle~~
  \fcolorbox{white}{voc_7}{\rule{0pt}{6pt}\rule{6pt}{0pt}} Bus~~
  \fcolorbox{white}{voc_8}{\rule{0pt}{6pt}\rule{6pt}{0pt}} Car~~ \\
  \fcolorbox{white}{voc_9}{\rule{0pt}{6pt}\rule{6pt}{0pt}} Cat~~
  \fcolorbox{white}{voc_10}{\rule{0pt}{6pt}\rule{6pt}{0pt}} Chair~~
  \fcolorbox{white}{voc_11}{\rule{0pt}{6pt}\rule{6pt}{0pt}} Cow~~
  \fcolorbox{white}{voc_12}{\rule{0pt}{6pt}\rule{6pt}{0pt}} Dining Table~~
  \fcolorbox{white}{voc_13}{\rule{0pt}{6pt}\rule{6pt}{0pt}} Dog~~
  \fcolorbox{white}{voc_14}{\rule{0pt}{6pt}\rule{6pt}{0pt}} Horse~~
  \fcolorbox{white}{voc_15}{\rule{0pt}{6pt}\rule{6pt}{0pt}} Motorbike~~
  \fcolorbox{white}{voc_16}{\rule{0pt}{6pt}\rule{6pt}{0pt}} Person~~ \\
  \fcolorbox{white}{voc_17}{\rule{0pt}{6pt}\rule{6pt}{0pt}} Potted Plant~~
  \fcolorbox{white}{voc_18}{\rule{0pt}{6pt}\rule{6pt}{0pt}} Sheep~~
  \fcolorbox{white}{voc_19}{\rule{0pt}{6pt}\rule{6pt}{0pt}} Sofa~~
  \fcolorbox{white}{voc_20}{\rule{0pt}{6pt}\rule{6pt}{0pt}} Train~~
  \fcolorbox{white}{voc_21}{\rule{0pt}{6pt}\rule{6pt}{0pt}} TV monitor~~ \\
  }
  \subfigure{%
    \includegraphics[width=.18\columnwidth]{figures/supplementary/2007_001423_given.jpg}
  }
  \subfigure{%
    \includegraphics[width=.18\columnwidth]{figures/supplementary/2007_001423_gt.png}
  }
  \subfigure{%
    \includegraphics[width=.18\columnwidth]{figures/supplementary/2007_001423_cnn.png}
  }
  \subfigure{%
    \includegraphics[width=.18\columnwidth]{figures/supplementary/2007_001423_gauss.png}
  }
  \subfigure{%
    \includegraphics[width=.18\columnwidth]{figures/supplementary/2007_001423_learnt.png}
  }\\
  \subfigure{%
    \includegraphics[width=.18\columnwidth]{figures/supplementary/2007_001430_given.jpg}
  }
  \subfigure{%
    \includegraphics[width=.18\columnwidth]{figures/supplementary/2007_001430_gt.png}
  }
  \subfigure{%
    \includegraphics[width=.18\columnwidth]{figures/supplementary/2007_001430_cnn.png}
  }
  \subfigure{%
    \includegraphics[width=.18\columnwidth]{figures/supplementary/2007_001430_gauss.png}
  }
  \subfigure{%
    \includegraphics[width=.18\columnwidth]{figures/supplementary/2007_001430_learnt.png}
  }\\
    \subfigure{%
    \includegraphics[width=.18\columnwidth]{figures/supplementary/2007_007996_given.jpg}
  }
  \subfigure{%
    \includegraphics[width=.18\columnwidth]{figures/supplementary/2007_007996_gt.png}
  }
  \subfigure{%
    \includegraphics[width=.18\columnwidth]{figures/supplementary/2007_007996_cnn.png}
  }
  \subfigure{%
    \includegraphics[width=.18\columnwidth]{figures/supplementary/2007_007996_gauss.png}
  }
  \subfigure{%
    \includegraphics[width=.18\columnwidth]{figures/supplementary/2007_007996_learnt.png}
  }\\
   \subfigure{%
    \includegraphics[width=.18\columnwidth]{figures/supplementary/2010_002682_given.jpg}
  }
  \subfigure{%
    \includegraphics[width=.18\columnwidth]{figures/supplementary/2010_002682_gt.png}
  }
  \subfigure{%
    \includegraphics[width=.18\columnwidth]{figures/supplementary/2010_002682_cnn.png}
  }
  \subfigure{%
    \includegraphics[width=.18\columnwidth]{figures/supplementary/2010_002682_gauss.png}
  }
  \subfigure{%
    \includegraphics[width=.18\columnwidth]{figures/supplementary/2010_002682_learnt.png}
  }\\
     \subfigure{%
    \includegraphics[width=.18\columnwidth]{figures/supplementary/2010_004789_given.jpg}
  }
  \subfigure{%
    \includegraphics[width=.18\columnwidth]{figures/supplementary/2010_004789_gt.png}
  }
  \subfigure{%
    \includegraphics[width=.18\columnwidth]{figures/supplementary/2010_004789_cnn.png}
  }
  \subfigure{%
    \includegraphics[width=.18\columnwidth]{figures/supplementary/2010_004789_gauss.png}
  }
  \subfigure{%
    \includegraphics[width=.18\columnwidth]{figures/supplementary/2010_004789_learnt.png}
  }\\
       \subfigure{%
    \includegraphics[width=.18\columnwidth]{figures/supplementary/2007_001311_given.jpg}
  }
  \subfigure{%
    \includegraphics[width=.18\columnwidth]{figures/supplementary/2007_001311_gt.png}
  }
  \subfigure{%
    \includegraphics[width=.18\columnwidth]{figures/supplementary/2007_001311_cnn.png}
  }
  \subfigure{%
    \includegraphics[width=.18\columnwidth]{figures/supplementary/2007_001311_gauss.png}
  }
  \subfigure{%
    \includegraphics[width=.18\columnwidth]{figures/supplementary/2007_001311_learnt.png}
  }\\
  \setcounter{subfigure}{0}
  \subfigure[Input]{%
    \includegraphics[width=.18\columnwidth]{figures/supplementary/2010_003531_given.jpg}
  }
  \subfigure[Ground Truth]{%
    \includegraphics[width=.18\columnwidth]{figures/supplementary/2010_003531_gt.png}
  }
  \subfigure[DeepLab]{%
    \includegraphics[width=.18\columnwidth]{figures/supplementary/2010_003531_cnn.png}
  }
  \subfigure[+GaussCRF]{%
    \includegraphics[width=.18\columnwidth]{figures/supplementary/2010_003531_gauss.png}
  }
  \subfigure[+LearnedCRF]{%
    \includegraphics[width=.18\columnwidth]{figures/supplementary/2010_003531_learnt.png}
  }
  \vspace{-0.3cm}
  \mycaption{Semantic Segmentation}{Example results of semantic segmentation.
  (c)~depicts the unary results before application of MF, (d)~after two steps of MF with Gaussian edge CRF potentials, (e)~after
  two steps of MF with learned edge CRF potentials.}
    \label{fig:semantic_visuals}
\end{figure*}


\definecolor{minc_1}{HTML}{771111}
\definecolor{minc_2}{HTML}{CAC690}
\definecolor{minc_3}{HTML}{EEEEEE}
\definecolor{minc_4}{HTML}{7C8FA6}
\definecolor{minc_5}{HTML}{597D31}
\definecolor{minc_6}{HTML}{104410}
\definecolor{minc_7}{HTML}{BB819C}
\definecolor{minc_8}{HTML}{D0CE48}
\definecolor{minc_9}{HTML}{622745}
\definecolor{minc_10}{HTML}{666666}
\definecolor{minc_11}{HTML}{D54A31}
\definecolor{minc_12}{HTML}{101044}
\definecolor{minc_13}{HTML}{444126}
\definecolor{minc_14}{HTML}{75D646}
\definecolor{minc_15}{HTML}{DD4348}
\definecolor{minc_16}{HTML}{5C8577}
\definecolor{minc_17}{HTML}{C78472}
\definecolor{minc_18}{HTML}{75D6D0}
\definecolor{minc_19}{HTML}{5B4586}
\definecolor{minc_20}{HTML}{C04393}
\definecolor{minc_21}{HTML}{D69948}
\definecolor{minc_22}{HTML}{7370D8}
\definecolor{minc_23}{HTML}{7A3622}
\definecolor{minc_24}{HTML}{000000}

\begin{figure*}[t]
  \centering
  \small{
  \fcolorbox{white}{minc_1}{\rule{0pt}{6pt}\rule{6pt}{0pt}} Brick~~
  \fcolorbox{white}{minc_2}{\rule{0pt}{6pt}\rule{6pt}{0pt}} Carpet~~
  \fcolorbox{white}{minc_3}{\rule{0pt}{6pt}\rule{6pt}{0pt}} Ceramic~~
  \fcolorbox{white}{minc_4}{\rule{0pt}{6pt}\rule{6pt}{0pt}} Fabric~~
  \fcolorbox{white}{minc_5}{\rule{0pt}{6pt}\rule{6pt}{0pt}} Foliage~~
  \fcolorbox{white}{minc_6}{\rule{0pt}{6pt}\rule{6pt}{0pt}} Food~~
  \fcolorbox{white}{minc_7}{\rule{0pt}{6pt}\rule{6pt}{0pt}} Glass~~
  \fcolorbox{white}{minc_8}{\rule{0pt}{6pt}\rule{6pt}{0pt}} Hair~~ \\
  \fcolorbox{white}{minc_9}{\rule{0pt}{6pt}\rule{6pt}{0pt}} Leather~~
  \fcolorbox{white}{minc_10}{\rule{0pt}{6pt}\rule{6pt}{0pt}} Metal~~
  \fcolorbox{white}{minc_11}{\rule{0pt}{6pt}\rule{6pt}{0pt}} Mirror~~
  \fcolorbox{white}{minc_12}{\rule{0pt}{6pt}\rule{6pt}{0pt}} Other~~
  \fcolorbox{white}{minc_13}{\rule{0pt}{6pt}\rule{6pt}{0pt}} Painted~~
  \fcolorbox{white}{minc_14}{\rule{0pt}{6pt}\rule{6pt}{0pt}} Paper~~
  \fcolorbox{white}{minc_15}{\rule{0pt}{6pt}\rule{6pt}{0pt}} Plastic~~\\
  \fcolorbox{white}{minc_16}{\rule{0pt}{6pt}\rule{6pt}{0pt}} Polished Stone~~
  \fcolorbox{white}{minc_17}{\rule{0pt}{6pt}\rule{6pt}{0pt}} Skin~~
  \fcolorbox{white}{minc_18}{\rule{0pt}{6pt}\rule{6pt}{0pt}} Sky~~
  \fcolorbox{white}{minc_19}{\rule{0pt}{6pt}\rule{6pt}{0pt}} Stone~~
  \fcolorbox{white}{minc_20}{\rule{0pt}{6pt}\rule{6pt}{0pt}} Tile~~
  \fcolorbox{white}{minc_21}{\rule{0pt}{6pt}\rule{6pt}{0pt}} Wallpaper~~
  \fcolorbox{white}{minc_22}{\rule{0pt}{6pt}\rule{6pt}{0pt}} Water~~
  \fcolorbox{white}{minc_23}{\rule{0pt}{6pt}\rule{6pt}{0pt}} Wood~~ \\
  }
  \subfigure{%
    \includegraphics[width=.18\columnwidth]{figures/supplementary/000010868_given.jpg}
  }
  \subfigure{%
    \includegraphics[width=.18\columnwidth]{figures/supplementary/000010868_gt.png}
  }
  \subfigure{%
    \includegraphics[width=.18\columnwidth]{figures/supplementary/000010868_cnn.png}
  }
  \subfigure{%
    \includegraphics[width=.18\columnwidth]{figures/supplementary/000010868_gauss.png}
  }
  \subfigure{%
    \includegraphics[width=.18\columnwidth]{figures/supplementary/000010868_learnt.png}
  }\\[-2ex]
  \subfigure{%
    \includegraphics[width=.18\columnwidth]{figures/supplementary/000006011_given.jpg}
  }
  \subfigure{%
    \includegraphics[width=.18\columnwidth]{figures/supplementary/000006011_gt.png}
  }
  \subfigure{%
    \includegraphics[width=.18\columnwidth]{figures/supplementary/000006011_cnn.png}
  }
  \subfigure{%
    \includegraphics[width=.18\columnwidth]{figures/supplementary/000006011_gauss.png}
  }
  \subfigure{%
    \includegraphics[width=.18\columnwidth]{figures/supplementary/000006011_learnt.png}
  }\\[-2ex]
    \subfigure{%
    \includegraphics[width=.18\columnwidth]{figures/supplementary/000008553_given.jpg}
  }
  \subfigure{%
    \includegraphics[width=.18\columnwidth]{figures/supplementary/000008553_gt.png}
  }
  \subfigure{%
    \includegraphics[width=.18\columnwidth]{figures/supplementary/000008553_cnn.png}
  }
  \subfigure{%
    \includegraphics[width=.18\columnwidth]{figures/supplementary/000008553_gauss.png}
  }
  \subfigure{%
    \includegraphics[width=.18\columnwidth]{figures/supplementary/000008553_learnt.png}
  }\\[-2ex]
   \subfigure{%
    \includegraphics[width=.18\columnwidth]{figures/supplementary/000009188_given.jpg}
  }
  \subfigure{%
    \includegraphics[width=.18\columnwidth]{figures/supplementary/000009188_gt.png}
  }
  \subfigure{%
    \includegraphics[width=.18\columnwidth]{figures/supplementary/000009188_cnn.png}
  }
  \subfigure{%
    \includegraphics[width=.18\columnwidth]{figures/supplementary/000009188_gauss.png}
  }
  \subfigure{%
    \includegraphics[width=.18\columnwidth]{figures/supplementary/000009188_learnt.png}
  }\\[-2ex]
  \setcounter{subfigure}{0}
  \subfigure[Input]{%
    \includegraphics[width=.18\columnwidth]{figures/supplementary/000023570_given.jpg}
  }
  \subfigure[Ground Truth]{%
    \includegraphics[width=.18\columnwidth]{figures/supplementary/000023570_gt.png}
  }
  \subfigure[DeepLab]{%
    \includegraphics[width=.18\columnwidth]{figures/supplementary/000023570_cnn.png}
  }
  \subfigure[+GaussCRF]{%
    \includegraphics[width=.18\columnwidth]{figures/supplementary/000023570_gauss.png}
  }
  \subfigure[+LearnedCRF]{%
    \includegraphics[width=.18\columnwidth]{figures/supplementary/000023570_learnt.png}
  }
  \mycaption{Material Segmentation}{Example results of material segmentation.
  (c)~depicts the unary results before application of MF, (d)~after two steps of MF with Gaussian edge CRF potentials, (e)~after two steps of MF with learned edge CRF potentials.}
    \label{fig:material_visuals-app2}
\end{figure*}


\begin{table*}[h]
\tiny
  \centering
    \begin{tabular}{L{2.3cm} L{2.25cm} C{1.5cm} C{0.7cm} C{0.6cm} C{0.7cm} C{0.7cm} C{0.7cm} C{1.6cm} C{0.6cm} C{0.6cm} C{0.6cm}}
      \toprule
& & & & & \multicolumn{3}{c}{\textbf{Data Statistics}} & \multicolumn{4}{c}{\textbf{Training Protocol}} \\

\textbf{Experiment} & \textbf{Feature Types} & \textbf{Feature Scales} & \textbf{Filter Size} & \textbf{Filter Nbr.} & \textbf{Train}  & \textbf{Val.} & \textbf{Test} & \textbf{Loss Type} & \textbf{LR} & \textbf{Batch} & \textbf{Epochs} \\
      \midrule
      \multicolumn{2}{c}{\textbf{Single Bilateral Filter Applications}} & & & & & & & & & \\
      \textbf{2$\times$ Color Upsampling} & Position$_{1}$, Intensity (3D) & 0.13, 0.17 & 65 & 2 & 10581 & 1449 & 1456 & MSE & 1e-06 & 200 & 94.5\\
      \textbf{4$\times$ Color Upsampling} & Position$_{1}$, Intensity (3D) & 0.06, 0.17 & 65 & 2 & 10581 & 1449 & 1456 & MSE & 1e-06 & 200 & 94.5\\
      \textbf{8$\times$ Color Upsampling} & Position$_{1}$, Intensity (3D) & 0.03, 0.17 & 65 & 2 & 10581 & 1449 & 1456 & MSE & 1e-06 & 200 & 94.5\\
      \textbf{16$\times$ Color Upsampling} & Position$_{1}$, Intensity (3D) & 0.02, 0.17 & 65 & 2 & 10581 & 1449 & 1456 & MSE & 1e-06 & 200 & 94.5\\
      \textbf{Depth Upsampling} & Position$_{1}$, Color (5D) & 0.05, 0.02 & 665 & 2 & 795 & 100 & 654 & MSE & 1e-07 & 50 & 251.6\\
      \textbf{Mesh Denoising} & Isomap (4D) & 46.00 & 63 & 2 & 1000 & 200 & 500 & MSE & 100 & 10 & 100.0 \\
      \midrule
      \multicolumn{2}{c}{\textbf{DenseCRF Applications}} & & & & & & & & &\\
      \multicolumn{2}{l}{\textbf{Semantic Segmentation}} & & & & & & & & &\\
      \textbf{- 1step MF} & Position$_{1}$, Color (5D); Position$_{1}$ (2D) & 0.01, 0.34; 0.34  & 665; 19  & 2; 2 & 10581 & 1449 & 1456 & Logistic & 0.1 & 5 & 1.4 \\
      \textbf{- 2step MF} & Position$_{1}$, Color (5D); Position$_{1}$ (2D) & 0.01, 0.34; 0.34 & 665; 19 & 2; 2 & 10581 & 1449 & 1456 & Logistic & 0.1 & 5 & 1.4 \\
      \textbf{- \textit{loose} 2step MF} & Position$_{1}$, Color (5D); Position$_{1}$ (2D) & 0.01, 0.34; 0.34 & 665; 19 & 2; 2 &10581 & 1449 & 1456 & Logistic & 0.1 & 5 & +1.9  \\ \\
      \multicolumn{2}{l}{\textbf{Material Segmentation}} & & & & & & & & &\\
      \textbf{- 1step MF} & Position$_{2}$, Lab-Color (5D) & 5.00, 0.05, 0.30  & 665 & 2 & 928 & 150 & 1798 & Weighted Logistic & 1e-04 & 24 & 2.6 \\
      \textbf{- 2step MF} & Position$_{2}$, Lab-Color (5D) & 5.00, 0.05, 0.30 & 665 & 2 & 928 & 150 & 1798 & Weighted Logistic & 1e-04 & 12 & +0.7 \\
      \textbf{- \textit{loose} 2step MF} & Position$_{2}$, Lab-Color (5D) & 5.00, 0.05, 0.30 & 665 & 2 & 928 & 150 & 1798 & Weighted Logistic & 1e-04 & 12 & +0.2\\
      \midrule
      \multicolumn{2}{c}{\textbf{Neural Network Applications}} & & & & & & & & &\\
      \textbf{Tiles: CNN-9$\times$9} & - & - & 81 & 4 & 10000 & 1000 & 1000 & Logistic & 0.01 & 100 & 500.0 \\
      \textbf{Tiles: CNN-13$\times$13} & - & - & 169 & 6 & 10000 & 1000 & 1000 & Logistic & 0.01 & 100 & 500.0 \\
      \textbf{Tiles: CNN-17$\times$17} & - & - & 289 & 8 & 10000 & 1000 & 1000 & Logistic & 0.01 & 100 & 500.0 \\
      \textbf{Tiles: CNN-21$\times$21} & - & - & 441 & 10 & 10000 & 1000 & 1000 & Logistic & 0.01 & 100 & 500.0 \\
      \textbf{Tiles: BNN} & Position$_{1}$, Color (5D) & 0.05, 0.04 & 63 & 1 & 10000 & 1000 & 1000 & Logistic & 0.01 & 100 & 30.0 \\
      \textbf{LeNet} & - & - & 25 & 2 & 5490 & 1098 & 1647 & Logistic & 0.1 & 100 & 182.2 \\
      \textbf{Crop-LeNet} & - & - & 25 & 2 & 5490 & 1098 & 1647 & Logistic & 0.1 & 100 & 182.2 \\
      \textbf{BNN-LeNet} & Position$_{2}$ (2D) & 20.00 & 7 & 1 & 5490 & 1098 & 1647 & Logistic & 0.1 & 100 & 182.2 \\
      \textbf{DeepCNet} & - & - & 9 & 1 & 5490 & 1098 & 1647 & Logistic & 0.1 & 100 & 182.2 \\
      \textbf{Crop-DeepCNet} & - & - & 9 & 1 & 5490 & 1098 & 1647 & Logistic & 0.1 & 100 & 182.2 \\
      \textbf{BNN-DeepCNet} & Position$_{2}$ (2D) & 40.00  & 7 & 1 & 5490 & 1098 & 1647 & Logistic & 0.1 & 100 & 182.2 \\
      \bottomrule
      \\
    \end{tabular}
    \mycaption{Experiment Protocols} {Experiment protocols for the different experiments presented in this work. \textbf{Feature Types}:
    Feature spaces used for the bilateral convolutions. Position$_1$ corresponds to un-normalized pixel positions whereas Position$_2$ corresponds
    to pixel positions normalized to $[0,1]$ with respect to the given image. \textbf{Feature Scales}: Cross-validated scales for the features used.
     \textbf{Filter Size}: Number of elements in the filter that is being learned. \textbf{Filter Nbr.}: Half-width of the filter. \textbf{Train},
     \textbf{Val.} and \textbf{Test} corresponds to the number of train, validation and test images used in the experiment. \textbf{Loss Type}: Type
     of loss used for back-propagation. ``MSE'' corresponds to Euclidean mean squared error loss and ``Logistic'' corresponds to multinomial logistic
     loss. ``Weighted Logistic'' is the class-weighted multinomial logistic loss. We weighted the loss with inverse class probability for material
     segmentation task due to the small availability of training data with class imbalance. \textbf{LR}: Fixed learning rate used in stochastic gradient
     descent. \textbf{Batch}: Number of images used in one parameter update step. \textbf{Epochs}: Number of training epochs. In all the experiments,
     we used fixed momentum of 0.9 and weight decay of 0.0005 for stochastic gradient descent. ```Color Upsampling'' experiments in this Table corresponds
     to those performed on Pascal VOC12 dataset images. For all experiments using Pascal VOC12 images, we use extended
     training segmentation dataset available from~\cite{hariharan2011moredata}, and used standard validation and test splits
     from the main dataset~\cite{voc2012segmentation}.}
  \label{tbl:parameters}
\end{table*}

\clearpage

\section{Parameters and Additional Results for Video Propagation Networks}

In this Section, we present experiment protocols and additional qualitative results for experiments
on video object segmentation, semantic video segmentation and video color
propagation. Table~\ref{tbl:parameters_supp} shows the feature scales and other parameters used in different experiments.
Figures~\ref{fig:video_seg_pos_supp} show some qualitative results on video object segmentation
with some failure cases in Fig.~\ref{fig:video_seg_neg_supp}.
Figure~\ref{fig:semantic_visuals_supp} shows some qualitative results on semantic video segmentation and
Fig.~\ref{fig:color_visuals_supp} shows results on video color propagation.

\newcolumntype{L}[1]{>{\raggedright\let\newline\\\arraybackslash\hspace{0pt}}b{#1}}
\newcolumntype{C}[1]{>{\centering\let\newline\\\arraybackslash\hspace{0pt}}b{#1}}
\newcolumntype{R}[1]{>{\raggedleft\let\newline\\\arraybackslash\hspace{0pt}}b{#1}}

\begin{table*}[h]
\tiny
  \centering
    \begin{tabular}{L{3.0cm} L{2.4cm} L{2.8cm} L{2.8cm} C{0.5cm} C{1.0cm} L{1.2cm}}
      \toprule
\textbf{Experiment} & \textbf{Feature Type} & \textbf{Feature Scale-1, $\Lambda_a$} & \textbf{Feature Scale-2, $\Lambda_b$} & \textbf{$\alpha$} & \textbf{Input Frames} & \textbf{Loss Type} \\
      \midrule
      \textbf{Video Object Segmentation} & ($x,y,Y,Cb,Cr,t$) & (0.02,0.02,0.07,0.4,0.4,0.01) & (0.03,0.03,0.09,0.5,0.5,0.2) & 0.5 & 9 & Logistic\\
      \midrule
      \textbf{Semantic Video Segmentation} & & & & & \\
      \textbf{with CNN1~\cite{yu2015multi}-NoFlow} & ($x,y,R,G,B,t$) & (0.08,0.08,0.2,0.2,0.2,0.04) & (0.11,0.11,0.2,0.2,0.2,0.04) & 0.5 & 3 & Logistic \\
      \textbf{with CNN1~\cite{yu2015multi}-Flow} & ($x+u_x,y+u_y,R,G,B,t$) & (0.11,0.11,0.14,0.14,0.14,0.03) & (0.08,0.08,0.12,0.12,0.12,0.01) & 0.65 & 3 & Logistic\\
      \textbf{with CNN2~\cite{richter2016playing}-Flow} & ($x+u_x,y+u_y,R,G,B,t$) & (0.08,0.08,0.2,0.2,0.2,0.04) & (0.09,0.09,0.25,0.25,0.25,0.03) & 0.5 & 4 & Logistic\\
      \midrule
      \textbf{Video Color Propagation} & ($x,y,I,t$)  & (0.04,0.04,0.2,0.04) & No second kernel & 1 & 4 & MSE\\
      \bottomrule
      \\
    \end{tabular}
    \mycaption{Experiment Protocols} {Experiment protocols for the different experiments presented in this work. \textbf{Feature Types}:
    Feature spaces used for the bilateral convolutions, with position ($x,y$) and color
    ($R,G,B$ or $Y,Cb,Cr$) features $\in [0,255]$. $u_x$, $u_y$ denotes optical flow with respect
    to the present frame and $I$ denotes grayscale intensity.
    \textbf{Feature Scales ($\Lambda_a, \Lambda_b$)}: Cross-validated scales for the features used.
    \textbf{$\alpha$}: Exponential time decay for the input frames.
    \textbf{Input Frames}: Number of input frames for VPN.
    \textbf{Loss Type}: Type
     of loss used for back-propagation. ``MSE'' corresponds to Euclidean mean squared error loss and ``Logistic'' corresponds to multinomial logistic loss.}
  \label{tbl:parameters_supp}
\end{table*}

% \begin{figure}[th!]
% \begin{center}
%   \centerline{\includegraphics[width=\textwidth]{figures/video_seg_visuals_supp_small.pdf}}
%     \mycaption{Video Object Segmentation}
%     {Shown are the different frames in example videos with the corresponding
%     ground truth (GT) masks, predictions from BVS~\cite{marki2016bilateral},
%     OFL~\cite{tsaivideo}, VPN (VPN-Stage2) and VPN-DLab (VPN-DeepLab) models.}
%     \label{fig:video_seg_small_supp}
% \end{center}
% \vspace{-1.0cm}
% \end{figure}

\begin{figure}[th!]
\begin{center}
  \centerline{\includegraphics[width=0.7\textwidth]{figures/video_seg_visuals_supp_positive.pdf}}
    \mycaption{Video Object Segmentation}
    {Shown are the different frames in example videos with the corresponding
    ground truth (GT) masks, predictions from BVS~\cite{marki2016bilateral},
    OFL~\cite{tsaivideo}, VPN (VPN-Stage2) and VPN-DLab (VPN-DeepLab) models.}
    \label{fig:video_seg_pos_supp}
\end{center}
\vspace{-1.0cm}
\end{figure}

\begin{figure}[th!]
\begin{center}
  \centerline{\includegraphics[width=0.7\textwidth]{figures/video_seg_visuals_supp_negative.pdf}}
    \mycaption{Failure Cases for Video Object Segmentation}
    {Shown are the different frames in example videos with the corresponding
    ground truth (GT) masks, predictions from BVS~\cite{marki2016bilateral},
    OFL~\cite{tsaivideo}, VPN (VPN-Stage2) and VPN-DLab (VPN-DeepLab) models.}
    \label{fig:video_seg_neg_supp}
\end{center}
\vspace{-1.0cm}
\end{figure}

\begin{figure}[th!]
\begin{center}
  \centerline{\includegraphics[width=0.9\textwidth]{figures/supp_semantic_visual.pdf}}
    \mycaption{Semantic Video Segmentation}
    {Input video frames and the corresponding ground truth (GT)
    segmentation together with the predictions of CNN~\cite{yu2015multi} and with
    VPN-Flow.}
    \label{fig:semantic_visuals_supp}
\end{center}
\vspace{-0.7cm}
\end{figure}

\begin{figure}[th!]
\begin{center}
  \centerline{\includegraphics[width=\textwidth]{figures/colorization_visuals_supp.pdf}}
  \mycaption{Video Color Propagation}
  {Input grayscale video frames and corresponding ground-truth (GT) color images
  together with color predictions of Levin et al.~\cite{levin2004colorization} and VPN-Stage1 models.}
  \label{fig:color_visuals_supp}
\end{center}
\vspace{-0.7cm}
\end{figure}

\clearpage

\section{Additional Material for Bilateral Inception Networks}
\label{sec:binception-app}

In this section of the Appendix, we first discuss the use of approximate bilateral
filtering in BI modules (Sec.~\ref{sec:lattice}).
Later, we present some qualitative results using different models for the approach presented in
Chapter~\ref{chap:binception} (Sec.~\ref{sec:qualitative-app}).

\subsection{Approximate Bilateral Filtering}
\label{sec:lattice}

The bilateral inception module presented in Chapter~\ref{chap:binception} computes a matrix-vector
product between a Gaussian filter $K$ and a vector of activations $\bz_c$.
Bilateral filtering is an important operation and many algorithmic techniques have been
proposed to speed-up this operation~\cite{paris2006fast,adams2010fast,gastal2011domain}.
In the main paper we opted to implement what can be considered the
brute-force variant of explicitly constructing $K$ and then using BLAS to compute the
matrix-vector product. This resulted in a few millisecond operation.
The explicit way to compute is possible due to the
reduction to super-pixels, e.g., it would not work for DenseCRF variants
that operate on the full image resolution.

Here, we present experiments where we use the fast approximate bilateral filtering
algorithm of~\cite{adams2010fast}, which is also used in Chapter~\ref{chap:bnn}
for learning sparse high dimensional filters. This
choice allows for larger dimensions of matrix-vector multiplication. The reason for choosing
the explicit multiplication in Chapter~\ref{chap:binception} was that it was computationally faster.
For the small sizes of the involved matrices and vectors, the explicit computation is sufficient and we had no
GPU implementation of an approximate technique that matched this runtime. Also it
is conceptually easier and the gradient to the feature transformations ($\Lambda \mathbf{f}$) is
obtained using standard matrix calculus.

\subsubsection{Experiments}

We modified the existing segmentation architectures analogous to those in Chapter~\ref{chap:binception}.
The main difference is that, here, the inception modules use the lattice
approximation~\cite{adams2010fast} to compute the bilateral filtering.
Using the lattice approximation did not allow us to back-propagate through feature transformations ($\Lambda$)
and thus we used hand-specified feature scales as will be explained later.
Specifically, we take CNN architectures from the works
of~\cite{chen2014semantic,zheng2015conditional,bell2015minc} and insert the BI modules between
the spatial FC layers.
We use superpixels from~\cite{DollarICCV13edges}
for all the experiments with the lattice approximation. Experiments are
performed using Caffe neural network framework~\cite{jia2014caffe}.

\begin{table}
  \small
  \centering
  \begin{tabular}{p{5.5cm}>{\raggedright\arraybackslash}p{1.4cm}>{\centering\arraybackslash}p{2.2cm}}
    \toprule
		\textbf{Model} & \emph{IoU} & \emph{Runtime}(ms) \\
    \midrule

    %%%%%%%%%%%% Scores computed by us)%%%%%%%%%%%%
		\deeplablargefov & 68.9 & 145ms\\
    \midrule
    \bi{7}{2}-\bi{8}{10}& \textbf{73.8} & +600 \\
    \midrule
    \deeplablargefovcrf~\cite{chen2014semantic} & 72.7 & +830\\
    \deeplabmsclargefovcrf~\cite{chen2014semantic} & \textbf{73.6} & +880\\
    DeepLab-EdgeNet~\cite{chen2015semantic} & 71.7 & +30\\
    DeepLab-EdgeNet-CRF~\cite{chen2015semantic} & \textbf{73.6} & +860\\
  \bottomrule \\
  \end{tabular}
  \mycaption{Semantic Segmentation using the DeepLab model}
  {IoU scores on the Pascal VOC12 segmentation test dataset
  with different models and our modified inception model.
  Also shown are the corresponding runtimes in milliseconds. Runtimes
  also include superpixel computations (300 ms with Dollar superpixels~\cite{DollarICCV13edges})}
  \label{tab:largefovresults}
\end{table}

\paragraph{Semantic Segmentation}
The experiments in this section use the Pascal VOC12 segmentation dataset~\cite{voc2012segmentation} with 21 object classes and the images have a maximum resolution of 0.25 megapixels.
For all experiments on VOC12, we train using the extended training set of
10581 images collected by~\cite{hariharan2011moredata}.
We modified the \deeplab~network architecture of~\cite{chen2014semantic} and
the CRFasRNN architecture from~\cite{zheng2015conditional} which uses a CNN with
deconvolution layers followed by DenseCRF trained end-to-end.

\paragraph{DeepLab Model}\label{sec:deeplabmodel}
We experimented with the \bi{7}{2}-\bi{8}{10} inception model.
Results using the~\deeplab~model are summarized in Tab.~\ref{tab:largefovresults}.
Although we get similar improvements with inception modules as with the
explicit kernel computation, using lattice approximation is slower.

\begin{table}
  \small
  \centering
  \begin{tabular}{p{6.4cm}>{\raggedright\arraybackslash}p{1.8cm}>{\raggedright\arraybackslash}p{1.8cm}}
    \toprule
    \textbf{Model} & \emph{IoU (Val)} & \emph{IoU (Test)}\\
    \midrule
    %%%%%%%%%%%% Scores computed by us)%%%%%%%%%%%%
    CNN &  67.5 & - \\
    \deconv (CNN+Deconvolutions) & 69.8 & 72.0 \\
    \midrule
    \bi{3}{6}-\bi{4}{6}-\bi{7}{2}-\bi{8}{6}& 71.9 & - \\
    \bi{3}{6}-\bi{4}{6}-\bi{7}{2}-\bi{8}{6}-\gi{6}& 73.6 &  \href{http://host.robots.ox.ac.uk:8080/anonymous/VOTV5E.html}{\textbf{75.2}}\\
    \midrule
    \deconvcrf (CRF-RNN)~\cite{zheng2015conditional} & 73.0 & 74.7\\
    Context-CRF-RNN~\cite{yu2015multi} & ~~ - ~ & \textbf{75.3} \\
    \bottomrule \\
  \end{tabular}
  \mycaption{Semantic Segmentation using the CRFasRNN model}{IoU score corresponding to different models
  on Pascal VOC12 reduced validation / test segmentation dataset. The reduced validation set consists of 346 images
  as used in~\cite{zheng2015conditional} where we adapted the model from.}
  \label{tab:deconvresults-app}
\end{table}

\paragraph{CRFasRNN Model}\label{sec:deepinception}
We add BI modules after score-pool3, score-pool4, \fc{7} and \fc{8} $1\times1$ convolution layers
resulting in the \bi{3}{6}-\bi{4}{6}-\bi{7}{2}-\bi{8}{6}
model and also experimented with another variant where $BI_8$ is followed by another inception
module, G$(6)$, with 6 Gaussian kernels.
Note that here also we discarded both deconvolution and DenseCRF parts of the original model~\cite{zheng2015conditional}
and inserted the BI modules in the base CNN and found similar improvements compared to the inception modules with explicit
kernel computaion. See Tab.~\ref{tab:deconvresults-app} for results on the CRFasRNN model.

\paragraph{Material Segmentation}
Table~\ref{tab:mincresults-app} shows the results on the MINC dataset~\cite{bell2015minc}
obtained by modifying the AlexNet architecture with our inception modules. We observe
similar improvements as with explicit kernel construction.
For this model, we do not provide any learned setup due to very limited segment training
data. The weights to combine outputs in the bilateral inception layer are
found by validation on the validation set.

\begin{table}[t]
  \small
  \centering
  \begin{tabular}{p{3.5cm}>{\centering\arraybackslash}p{4.0cm}}
    \toprule
    \textbf{Model} & Class / Total accuracy\\
    \midrule

    %%%%%%%%%%%% Scores computed by us)%%%%%%%%%%%%
    AlexNet CNN & 55.3 / 58.9 \\
    \midrule
    \bi{7}{2}-\bi{8}{6}& 68.5 / 71.8 \\
    \bi{7}{2}-\bi{8}{6}-G$(6)$& 67.6 / 73.1 \\
    \midrule
    AlexNet-CRF & 65.5 / 71.0 \\
    \bottomrule \\
  \end{tabular}
  \mycaption{Material Segmentation using AlexNet}{Pixel accuracy of different models on
  the MINC material segmentation test dataset~\cite{bell2015minc}.}
  \label{tab:mincresults-app}
\end{table}

\paragraph{Scales of Bilateral Inception Modules}
\label{sec:scales}

Unlike the explicit kernel technique presented in the main text (Chapter~\ref{chap:binception}),
we didn't back-propagate through feature transformation ($\Lambda$)
using the approximate bilateral filter technique.
So, the feature scales are hand-specified and validated, which are as follows.
The optimal scale values for the \bi{7}{2}-\bi{8}{2} model are found by validation for the best performance which are
$\sigma_{xy}$ = (0.1, 0.1) for the spatial (XY) kernel and $\sigma_{rgbxy}$ = (0.1, 0.1, 0.1, 0.01, 0.01) for color and position (RGBXY)  kernel.
Next, as more kernels are added to \bi{8}{2}, we set scales to be $\alpha$*($\sigma_{xy}$, $\sigma_{rgbxy}$).
The value of $\alpha$ is chosen as  1, 0.5, 0.1, 0.05, 0.1, at uniform interval, for the \bi{8}{10} bilateral inception module.


\subsection{Qualitative Results}
\label{sec:qualitative-app}

In this section, we present more qualitative results obtained using the BI module with explicit
kernel computation technique presented in Chapter~\ref{chap:binception}. Results on the Pascal VOC12
dataset~\cite{voc2012segmentation} using the DeepLab-LargeFOV model are shown in Fig.~\ref{fig:semantic_visuals-app},
followed by the results on MINC dataset~\cite{bell2015minc}
in Fig.~\ref{fig:material_visuals-app} and on
Cityscapes dataset~\cite{Cordts2015Cvprw} in Fig.~\ref{fig:street_visuals-app}.


\definecolor{voc_1}{RGB}{0, 0, 0}
\definecolor{voc_2}{RGB}{128, 0, 0}
\definecolor{voc_3}{RGB}{0, 128, 0}
\definecolor{voc_4}{RGB}{128, 128, 0}
\definecolor{voc_5}{RGB}{0, 0, 128}
\definecolor{voc_6}{RGB}{128, 0, 128}
\definecolor{voc_7}{RGB}{0, 128, 128}
\definecolor{voc_8}{RGB}{128, 128, 128}
\definecolor{voc_9}{RGB}{64, 0, 0}
\definecolor{voc_10}{RGB}{192, 0, 0}
\definecolor{voc_11}{RGB}{64, 128, 0}
\definecolor{voc_12}{RGB}{192, 128, 0}
\definecolor{voc_13}{RGB}{64, 0, 128}
\definecolor{voc_14}{RGB}{192, 0, 128}
\definecolor{voc_15}{RGB}{64, 128, 128}
\definecolor{voc_16}{RGB}{192, 128, 128}
\definecolor{voc_17}{RGB}{0, 64, 0}
\definecolor{voc_18}{RGB}{128, 64, 0}
\definecolor{voc_19}{RGB}{0, 192, 0}
\definecolor{voc_20}{RGB}{128, 192, 0}
\definecolor{voc_21}{RGB}{0, 64, 128}
\definecolor{voc_22}{RGB}{128, 64, 128}

\begin{figure*}[!ht]
  \small
  \centering
  \fcolorbox{white}{voc_1}{\rule{0pt}{4pt}\rule{4pt}{0pt}} Background~~
  \fcolorbox{white}{voc_2}{\rule{0pt}{4pt}\rule{4pt}{0pt}} Aeroplane~~
  \fcolorbox{white}{voc_3}{\rule{0pt}{4pt}\rule{4pt}{0pt}} Bicycle~~
  \fcolorbox{white}{voc_4}{\rule{0pt}{4pt}\rule{4pt}{0pt}} Bird~~
  \fcolorbox{white}{voc_5}{\rule{0pt}{4pt}\rule{4pt}{0pt}} Boat~~
  \fcolorbox{white}{voc_6}{\rule{0pt}{4pt}\rule{4pt}{0pt}} Bottle~~
  \fcolorbox{white}{voc_7}{\rule{0pt}{4pt}\rule{4pt}{0pt}} Bus~~
  \fcolorbox{white}{voc_8}{\rule{0pt}{4pt}\rule{4pt}{0pt}} Car~~\\
  \fcolorbox{white}{voc_9}{\rule{0pt}{4pt}\rule{4pt}{0pt}} Cat~~
  \fcolorbox{white}{voc_10}{\rule{0pt}{4pt}\rule{4pt}{0pt}} Chair~~
  \fcolorbox{white}{voc_11}{\rule{0pt}{4pt}\rule{4pt}{0pt}} Cow~~
  \fcolorbox{white}{voc_12}{\rule{0pt}{4pt}\rule{4pt}{0pt}} Dining Table~~
  \fcolorbox{white}{voc_13}{\rule{0pt}{4pt}\rule{4pt}{0pt}} Dog~~
  \fcolorbox{white}{voc_14}{\rule{0pt}{4pt}\rule{4pt}{0pt}} Horse~~
  \fcolorbox{white}{voc_15}{\rule{0pt}{4pt}\rule{4pt}{0pt}} Motorbike~~
  \fcolorbox{white}{voc_16}{\rule{0pt}{4pt}\rule{4pt}{0pt}} Person~~\\
  \fcolorbox{white}{voc_17}{\rule{0pt}{4pt}\rule{4pt}{0pt}} Potted Plant~~
  \fcolorbox{white}{voc_18}{\rule{0pt}{4pt}\rule{4pt}{0pt}} Sheep~~
  \fcolorbox{white}{voc_19}{\rule{0pt}{4pt}\rule{4pt}{0pt}} Sofa~~
  \fcolorbox{white}{voc_20}{\rule{0pt}{4pt}\rule{4pt}{0pt}} Train~~
  \fcolorbox{white}{voc_21}{\rule{0pt}{4pt}\rule{4pt}{0pt}} TV monitor~~\\


  \subfigure{%
    \includegraphics[width=.15\columnwidth]{figures/supplementary/2008_001308_given.png}
  }
  \subfigure{%
    \includegraphics[width=.15\columnwidth]{figures/supplementary/2008_001308_sp.png}
  }
  \subfigure{%
    \includegraphics[width=.15\columnwidth]{figures/supplementary/2008_001308_gt.png}
  }
  \subfigure{%
    \includegraphics[width=.15\columnwidth]{figures/supplementary/2008_001308_cnn.png}
  }
  \subfigure{%
    \includegraphics[width=.15\columnwidth]{figures/supplementary/2008_001308_crf.png}
  }
  \subfigure{%
    \includegraphics[width=.15\columnwidth]{figures/supplementary/2008_001308_ours.png}
  }\\[-2ex]


  \subfigure{%
    \includegraphics[width=.15\columnwidth]{figures/supplementary/2008_001821_given.png}
  }
  \subfigure{%
    \includegraphics[width=.15\columnwidth]{figures/supplementary/2008_001821_sp.png}
  }
  \subfigure{%
    \includegraphics[width=.15\columnwidth]{figures/supplementary/2008_001821_gt.png}
  }
  \subfigure{%
    \includegraphics[width=.15\columnwidth]{figures/supplementary/2008_001821_cnn.png}
  }
  \subfigure{%
    \includegraphics[width=.15\columnwidth]{figures/supplementary/2008_001821_crf.png}
  }
  \subfigure{%
    \includegraphics[width=.15\columnwidth]{figures/supplementary/2008_001821_ours.png}
  }\\[-2ex]



  \subfigure{%
    \includegraphics[width=.15\columnwidth]{figures/supplementary/2008_004612_given.png}
  }
  \subfigure{%
    \includegraphics[width=.15\columnwidth]{figures/supplementary/2008_004612_sp.png}
  }
  \subfigure{%
    \includegraphics[width=.15\columnwidth]{figures/supplementary/2008_004612_gt.png}
  }
  \subfigure{%
    \includegraphics[width=.15\columnwidth]{figures/supplementary/2008_004612_cnn.png}
  }
  \subfigure{%
    \includegraphics[width=.15\columnwidth]{figures/supplementary/2008_004612_crf.png}
  }
  \subfigure{%
    \includegraphics[width=.15\columnwidth]{figures/supplementary/2008_004612_ours.png}
  }\\[-2ex]


  \subfigure{%
    \includegraphics[width=.15\columnwidth]{figures/supplementary/2009_001008_given.png}
  }
  \subfigure{%
    \includegraphics[width=.15\columnwidth]{figures/supplementary/2009_001008_sp.png}
  }
  \subfigure{%
    \includegraphics[width=.15\columnwidth]{figures/supplementary/2009_001008_gt.png}
  }
  \subfigure{%
    \includegraphics[width=.15\columnwidth]{figures/supplementary/2009_001008_cnn.png}
  }
  \subfigure{%
    \includegraphics[width=.15\columnwidth]{figures/supplementary/2009_001008_crf.png}
  }
  \subfigure{%
    \includegraphics[width=.15\columnwidth]{figures/supplementary/2009_001008_ours.png}
  }\\[-2ex]




  \subfigure{%
    \includegraphics[width=.15\columnwidth]{figures/supplementary/2009_004497_given.png}
  }
  \subfigure{%
    \includegraphics[width=.15\columnwidth]{figures/supplementary/2009_004497_sp.png}
  }
  \subfigure{%
    \includegraphics[width=.15\columnwidth]{figures/supplementary/2009_004497_gt.png}
  }
  \subfigure{%
    \includegraphics[width=.15\columnwidth]{figures/supplementary/2009_004497_cnn.png}
  }
  \subfigure{%
    \includegraphics[width=.15\columnwidth]{figures/supplementary/2009_004497_crf.png}
  }
  \subfigure{%
    \includegraphics[width=.15\columnwidth]{figures/supplementary/2009_004497_ours.png}
  }\\[-2ex]



  \setcounter{subfigure}{0}
  \subfigure[\scriptsize Input]{%
    \includegraphics[width=.15\columnwidth]{figures/supplementary/2010_001327_given.png}
  }
  \subfigure[\scriptsize Superpixels]{%
    \includegraphics[width=.15\columnwidth]{figures/supplementary/2010_001327_sp.png}
  }
  \subfigure[\scriptsize GT]{%
    \includegraphics[width=.15\columnwidth]{figures/supplementary/2010_001327_gt.png}
  }
  \subfigure[\scriptsize Deeplab]{%
    \includegraphics[width=.15\columnwidth]{figures/supplementary/2010_001327_cnn.png}
  }
  \subfigure[\scriptsize +DenseCRF]{%
    \includegraphics[width=.15\columnwidth]{figures/supplementary/2010_001327_crf.png}
  }
  \subfigure[\scriptsize Using BI]{%
    \includegraphics[width=.15\columnwidth]{figures/supplementary/2010_001327_ours.png}
  }
  \mycaption{Semantic Segmentation}{Example results of semantic segmentation
  on the Pascal VOC12 dataset.
  (d)~depicts the DeepLab CNN result, (e)~CNN + 10 steps of mean-field inference,
  (f~result obtained with bilateral inception (BI) modules (\bi{6}{2}+\bi{7}{6}) between \fc~layers.}
  \label{fig:semantic_visuals-app}
\end{figure*}


\definecolor{minc_1}{HTML}{771111}
\definecolor{minc_2}{HTML}{CAC690}
\definecolor{minc_3}{HTML}{EEEEEE}
\definecolor{minc_4}{HTML}{7C8FA6}
\definecolor{minc_5}{HTML}{597D31}
\definecolor{minc_6}{HTML}{104410}
\definecolor{minc_7}{HTML}{BB819C}
\definecolor{minc_8}{HTML}{D0CE48}
\definecolor{minc_9}{HTML}{622745}
\definecolor{minc_10}{HTML}{666666}
\definecolor{minc_11}{HTML}{D54A31}
\definecolor{minc_12}{HTML}{101044}
\definecolor{minc_13}{HTML}{444126}
\definecolor{minc_14}{HTML}{75D646}
\definecolor{minc_15}{HTML}{DD4348}
\definecolor{minc_16}{HTML}{5C8577}
\definecolor{minc_17}{HTML}{C78472}
\definecolor{minc_18}{HTML}{75D6D0}
\definecolor{minc_19}{HTML}{5B4586}
\definecolor{minc_20}{HTML}{C04393}
\definecolor{minc_21}{HTML}{D69948}
\definecolor{minc_22}{HTML}{7370D8}
\definecolor{minc_23}{HTML}{7A3622}
\definecolor{minc_24}{HTML}{000000}

\begin{figure*}[!ht]
  \small % scriptsize
  \centering
  \fcolorbox{white}{minc_1}{\rule{0pt}{4pt}\rule{4pt}{0pt}} Brick~~
  \fcolorbox{white}{minc_2}{\rule{0pt}{4pt}\rule{4pt}{0pt}} Carpet~~
  \fcolorbox{white}{minc_3}{\rule{0pt}{4pt}\rule{4pt}{0pt}} Ceramic~~
  \fcolorbox{white}{minc_4}{\rule{0pt}{4pt}\rule{4pt}{0pt}} Fabric~~
  \fcolorbox{white}{minc_5}{\rule{0pt}{4pt}\rule{4pt}{0pt}} Foliage~~
  \fcolorbox{white}{minc_6}{\rule{0pt}{4pt}\rule{4pt}{0pt}} Food~~
  \fcolorbox{white}{minc_7}{\rule{0pt}{4pt}\rule{4pt}{0pt}} Glass~~
  \fcolorbox{white}{minc_8}{\rule{0pt}{4pt}\rule{4pt}{0pt}} Hair~~\\
  \fcolorbox{white}{minc_9}{\rule{0pt}{4pt}\rule{4pt}{0pt}} Leather~~
  \fcolorbox{white}{minc_10}{\rule{0pt}{4pt}\rule{4pt}{0pt}} Metal~~
  \fcolorbox{white}{minc_11}{\rule{0pt}{4pt}\rule{4pt}{0pt}} Mirror~~
  \fcolorbox{white}{minc_12}{\rule{0pt}{4pt}\rule{4pt}{0pt}} Other~~
  \fcolorbox{white}{minc_13}{\rule{0pt}{4pt}\rule{4pt}{0pt}} Painted~~
  \fcolorbox{white}{minc_14}{\rule{0pt}{4pt}\rule{4pt}{0pt}} Paper~~
  \fcolorbox{white}{minc_15}{\rule{0pt}{4pt}\rule{4pt}{0pt}} Plastic~~\\
  \fcolorbox{white}{minc_16}{\rule{0pt}{4pt}\rule{4pt}{0pt}} Polished Stone~~
  \fcolorbox{white}{minc_17}{\rule{0pt}{4pt}\rule{4pt}{0pt}} Skin~~
  \fcolorbox{white}{minc_18}{\rule{0pt}{4pt}\rule{4pt}{0pt}} Sky~~
  \fcolorbox{white}{minc_19}{\rule{0pt}{4pt}\rule{4pt}{0pt}} Stone~~
  \fcolorbox{white}{minc_20}{\rule{0pt}{4pt}\rule{4pt}{0pt}} Tile~~
  \fcolorbox{white}{minc_21}{\rule{0pt}{4pt}\rule{4pt}{0pt}} Wallpaper~~
  \fcolorbox{white}{minc_22}{\rule{0pt}{4pt}\rule{4pt}{0pt}} Water~~
  \fcolorbox{white}{minc_23}{\rule{0pt}{4pt}\rule{4pt}{0pt}} Wood~~\\
  \subfigure{%
    \includegraphics[width=.15\columnwidth]{figures/supplementary/000008468_given.png}
  }
  \subfigure{%
    \includegraphics[width=.15\columnwidth]{figures/supplementary/000008468_sp.png}
  }
  \subfigure{%
    \includegraphics[width=.15\columnwidth]{figures/supplementary/000008468_gt.png}
  }
  \subfigure{%
    \includegraphics[width=.15\columnwidth]{figures/supplementary/000008468_cnn.png}
  }
  \subfigure{%
    \includegraphics[width=.15\columnwidth]{figures/supplementary/000008468_crf.png}
  }
  \subfigure{%
    \includegraphics[width=.15\columnwidth]{figures/supplementary/000008468_ours.png}
  }\\[-2ex]

  \subfigure{%
    \includegraphics[width=.15\columnwidth]{figures/supplementary/000009053_given.png}
  }
  \subfigure{%
    \includegraphics[width=.15\columnwidth]{figures/supplementary/000009053_sp.png}
  }
  \subfigure{%
    \includegraphics[width=.15\columnwidth]{figures/supplementary/000009053_gt.png}
  }
  \subfigure{%
    \includegraphics[width=.15\columnwidth]{figures/supplementary/000009053_cnn.png}
  }
  \subfigure{%
    \includegraphics[width=.15\columnwidth]{figures/supplementary/000009053_crf.png}
  }
  \subfigure{%
    \includegraphics[width=.15\columnwidth]{figures/supplementary/000009053_ours.png}
  }\\[-2ex]




  \subfigure{%
    \includegraphics[width=.15\columnwidth]{figures/supplementary/000014977_given.png}
  }
  \subfigure{%
    \includegraphics[width=.15\columnwidth]{figures/supplementary/000014977_sp.png}
  }
  \subfigure{%
    \includegraphics[width=.15\columnwidth]{figures/supplementary/000014977_gt.png}
  }
  \subfigure{%
    \includegraphics[width=.15\columnwidth]{figures/supplementary/000014977_cnn.png}
  }
  \subfigure{%
    \includegraphics[width=.15\columnwidth]{figures/supplementary/000014977_crf.png}
  }
  \subfigure{%
    \includegraphics[width=.15\columnwidth]{figures/supplementary/000014977_ours.png}
  }\\[-2ex]


  \subfigure{%
    \includegraphics[width=.15\columnwidth]{figures/supplementary/000022922_given.png}
  }
  \subfigure{%
    \includegraphics[width=.15\columnwidth]{figures/supplementary/000022922_sp.png}
  }
  \subfigure{%
    \includegraphics[width=.15\columnwidth]{figures/supplementary/000022922_gt.png}
  }
  \subfigure{%
    \includegraphics[width=.15\columnwidth]{figures/supplementary/000022922_cnn.png}
  }
  \subfigure{%
    \includegraphics[width=.15\columnwidth]{figures/supplementary/000022922_crf.png}
  }
  \subfigure{%
    \includegraphics[width=.15\columnwidth]{figures/supplementary/000022922_ours.png}
  }\\[-2ex]


  \subfigure{%
    \includegraphics[width=.15\columnwidth]{figures/supplementary/000025711_given.png}
  }
  \subfigure{%
    \includegraphics[width=.15\columnwidth]{figures/supplementary/000025711_sp.png}
  }
  \subfigure{%
    \includegraphics[width=.15\columnwidth]{figures/supplementary/000025711_gt.png}
  }
  \subfigure{%
    \includegraphics[width=.15\columnwidth]{figures/supplementary/000025711_cnn.png}
  }
  \subfigure{%
    \includegraphics[width=.15\columnwidth]{figures/supplementary/000025711_crf.png}
  }
  \subfigure{%
    \includegraphics[width=.15\columnwidth]{figures/supplementary/000025711_ours.png}
  }\\[-2ex]


  \subfigure{%
    \includegraphics[width=.15\columnwidth]{figures/supplementary/000034473_given.png}
  }
  \subfigure{%
    \includegraphics[width=.15\columnwidth]{figures/supplementary/000034473_sp.png}
  }
  \subfigure{%
    \includegraphics[width=.15\columnwidth]{figures/supplementary/000034473_gt.png}
  }
  \subfigure{%
    \includegraphics[width=.15\columnwidth]{figures/supplementary/000034473_cnn.png}
  }
  \subfigure{%
    \includegraphics[width=.15\columnwidth]{figures/supplementary/000034473_crf.png}
  }
  \subfigure{%
    \includegraphics[width=.15\columnwidth]{figures/supplementary/000034473_ours.png}
  }\\[-2ex]


  \subfigure{%
    \includegraphics[width=.15\columnwidth]{figures/supplementary/000035463_given.png}
  }
  \subfigure{%
    \includegraphics[width=.15\columnwidth]{figures/supplementary/000035463_sp.png}
  }
  \subfigure{%
    \includegraphics[width=.15\columnwidth]{figures/supplementary/000035463_gt.png}
  }
  \subfigure{%
    \includegraphics[width=.15\columnwidth]{figures/supplementary/000035463_cnn.png}
  }
  \subfigure{%
    \includegraphics[width=.15\columnwidth]{figures/supplementary/000035463_crf.png}
  }
  \subfigure{%
    \includegraphics[width=.15\columnwidth]{figures/supplementary/000035463_ours.png}
  }\\[-2ex]


  \setcounter{subfigure}{0}
  \subfigure[\scriptsize Input]{%
    \includegraphics[width=.15\columnwidth]{figures/supplementary/000035993_given.png}
  }
  \subfigure[\scriptsize Superpixels]{%
    \includegraphics[width=.15\columnwidth]{figures/supplementary/000035993_sp.png}
  }
  \subfigure[\scriptsize GT]{%
    \includegraphics[width=.15\columnwidth]{figures/supplementary/000035993_gt.png}
  }
  \subfigure[\scriptsize AlexNet]{%
    \includegraphics[width=.15\columnwidth]{figures/supplementary/000035993_cnn.png}
  }
  \subfigure[\scriptsize +DenseCRF]{%
    \includegraphics[width=.15\columnwidth]{figures/supplementary/000035993_crf.png}
  }
  \subfigure[\scriptsize Using BI]{%
    \includegraphics[width=.15\columnwidth]{figures/supplementary/000035993_ours.png}
  }
  \mycaption{Material Segmentation}{Example results of material segmentation.
  (d)~depicts the AlexNet CNN result, (e)~CNN + 10 steps of mean-field inference,
  (f)~result obtained with bilateral inception (BI) modules (\bi{7}{2}+\bi{8}{6}) between
  \fc~layers.}
\label{fig:material_visuals-app}
\end{figure*}


\definecolor{city_1}{RGB}{128, 64, 128}
\definecolor{city_2}{RGB}{244, 35, 232}
\definecolor{city_3}{RGB}{70, 70, 70}
\definecolor{city_4}{RGB}{102, 102, 156}
\definecolor{city_5}{RGB}{190, 153, 153}
\definecolor{city_6}{RGB}{153, 153, 153}
\definecolor{city_7}{RGB}{250, 170, 30}
\definecolor{city_8}{RGB}{220, 220, 0}
\definecolor{city_9}{RGB}{107, 142, 35}
\definecolor{city_10}{RGB}{152, 251, 152}
\definecolor{city_11}{RGB}{70, 130, 180}
\definecolor{city_12}{RGB}{220, 20, 60}
\definecolor{city_13}{RGB}{255, 0, 0}
\definecolor{city_14}{RGB}{0, 0, 142}
\definecolor{city_15}{RGB}{0, 0, 70}
\definecolor{city_16}{RGB}{0, 60, 100}
\definecolor{city_17}{RGB}{0, 80, 100}
\definecolor{city_18}{RGB}{0, 0, 230}
\definecolor{city_19}{RGB}{119, 11, 32}
\begin{figure*}[!ht]
  \small % scriptsize
  \centering


  \subfigure{%
    \includegraphics[width=.18\columnwidth]{figures/supplementary/frankfurt00000_016005_given.png}
  }
  \subfigure{%
    \includegraphics[width=.18\columnwidth]{figures/supplementary/frankfurt00000_016005_sp.png}
  }
  \subfigure{%
    \includegraphics[width=.18\columnwidth]{figures/supplementary/frankfurt00000_016005_gt.png}
  }
  \subfigure{%
    \includegraphics[width=.18\columnwidth]{figures/supplementary/frankfurt00000_016005_cnn.png}
  }
  \subfigure{%
    \includegraphics[width=.18\columnwidth]{figures/supplementary/frankfurt00000_016005_ours.png}
  }\\[-2ex]

  \subfigure{%
    \includegraphics[width=.18\columnwidth]{figures/supplementary/frankfurt00000_004617_given.png}
  }
  \subfigure{%
    \includegraphics[width=.18\columnwidth]{figures/supplementary/frankfurt00000_004617_sp.png}
  }
  \subfigure{%
    \includegraphics[width=.18\columnwidth]{figures/supplementary/frankfurt00000_004617_gt.png}
  }
  \subfigure{%
    \includegraphics[width=.18\columnwidth]{figures/supplementary/frankfurt00000_004617_cnn.png}
  }
  \subfigure{%
    \includegraphics[width=.18\columnwidth]{figures/supplementary/frankfurt00000_004617_ours.png}
  }\\[-2ex]

  \subfigure{%
    \includegraphics[width=.18\columnwidth]{figures/supplementary/frankfurt00000_020880_given.png}
  }
  \subfigure{%
    \includegraphics[width=.18\columnwidth]{figures/supplementary/frankfurt00000_020880_sp.png}
  }
  \subfigure{%
    \includegraphics[width=.18\columnwidth]{figures/supplementary/frankfurt00000_020880_gt.png}
  }
  \subfigure{%
    \includegraphics[width=.18\columnwidth]{figures/supplementary/frankfurt00000_020880_cnn.png}
  }
  \subfigure{%
    \includegraphics[width=.18\columnwidth]{figures/supplementary/frankfurt00000_020880_ours.png}
  }\\[-2ex]



  \subfigure{%
    \includegraphics[width=.18\columnwidth]{figures/supplementary/frankfurt00001_007285_given.png}
  }
  \subfigure{%
    \includegraphics[width=.18\columnwidth]{figures/supplementary/frankfurt00001_007285_sp.png}
  }
  \subfigure{%
    \includegraphics[width=.18\columnwidth]{figures/supplementary/frankfurt00001_007285_gt.png}
  }
  \subfigure{%
    \includegraphics[width=.18\columnwidth]{figures/supplementary/frankfurt00001_007285_cnn.png}
  }
  \subfigure{%
    \includegraphics[width=.18\columnwidth]{figures/supplementary/frankfurt00001_007285_ours.png}
  }\\[-2ex]


  \subfigure{%
    \includegraphics[width=.18\columnwidth]{figures/supplementary/frankfurt00001_059789_given.png}
  }
  \subfigure{%
    \includegraphics[width=.18\columnwidth]{figures/supplementary/frankfurt00001_059789_sp.png}
  }
  \subfigure{%
    \includegraphics[width=.18\columnwidth]{figures/supplementary/frankfurt00001_059789_gt.png}
  }
  \subfigure{%
    \includegraphics[width=.18\columnwidth]{figures/supplementary/frankfurt00001_059789_cnn.png}
  }
  \subfigure{%
    \includegraphics[width=.18\columnwidth]{figures/supplementary/frankfurt00001_059789_ours.png}
  }\\[-2ex]


  \subfigure{%
    \includegraphics[width=.18\columnwidth]{figures/supplementary/frankfurt00001_068208_given.png}
  }
  \subfigure{%
    \includegraphics[width=.18\columnwidth]{figures/supplementary/frankfurt00001_068208_sp.png}
  }
  \subfigure{%
    \includegraphics[width=.18\columnwidth]{figures/supplementary/frankfurt00001_068208_gt.png}
  }
  \subfigure{%
    \includegraphics[width=.18\columnwidth]{figures/supplementary/frankfurt00001_068208_cnn.png}
  }
  \subfigure{%
    \includegraphics[width=.18\columnwidth]{figures/supplementary/frankfurt00001_068208_ours.png}
  }\\[-2ex]

  \subfigure{%
    \includegraphics[width=.18\columnwidth]{figures/supplementary/frankfurt00001_082466_given.png}
  }
  \subfigure{%
    \includegraphics[width=.18\columnwidth]{figures/supplementary/frankfurt00001_082466_sp.png}
  }
  \subfigure{%
    \includegraphics[width=.18\columnwidth]{figures/supplementary/frankfurt00001_082466_gt.png}
  }
  \subfigure{%
    \includegraphics[width=.18\columnwidth]{figures/supplementary/frankfurt00001_082466_cnn.png}
  }
  \subfigure{%
    \includegraphics[width=.18\columnwidth]{figures/supplementary/frankfurt00001_082466_ours.png}
  }\\[-2ex]

  \subfigure{%
    \includegraphics[width=.18\columnwidth]{figures/supplementary/lindau00033_000019_given.png}
  }
  \subfigure{%
    \includegraphics[width=.18\columnwidth]{figures/supplementary/lindau00033_000019_sp.png}
  }
  \subfigure{%
    \includegraphics[width=.18\columnwidth]{figures/supplementary/lindau00033_000019_gt.png}
  }
  \subfigure{%
    \includegraphics[width=.18\columnwidth]{figures/supplementary/lindau00033_000019_cnn.png}
  }
  \subfigure{%
    \includegraphics[width=.18\columnwidth]{figures/supplementary/lindau00033_000019_ours.png}
  }\\[-2ex]

  \subfigure{%
    \includegraphics[width=.18\columnwidth]{figures/supplementary/lindau00052_000019_given.png}
  }
  \subfigure{%
    \includegraphics[width=.18\columnwidth]{figures/supplementary/lindau00052_000019_sp.png}
  }
  \subfigure{%
    \includegraphics[width=.18\columnwidth]{figures/supplementary/lindau00052_000019_gt.png}
  }
  \subfigure{%
    \includegraphics[width=.18\columnwidth]{figures/supplementary/lindau00052_000019_cnn.png}
  }
  \subfigure{%
    \includegraphics[width=.18\columnwidth]{figures/supplementary/lindau00052_000019_ours.png}
  }\\[-2ex]




  \subfigure{%
    \includegraphics[width=.18\columnwidth]{figures/supplementary/lindau00027_000019_given.png}
  }
  \subfigure{%
    \includegraphics[width=.18\columnwidth]{figures/supplementary/lindau00027_000019_sp.png}
  }
  \subfigure{%
    \includegraphics[width=.18\columnwidth]{figures/supplementary/lindau00027_000019_gt.png}
  }
  \subfigure{%
    \includegraphics[width=.18\columnwidth]{figures/supplementary/lindau00027_000019_cnn.png}
  }
  \subfigure{%
    \includegraphics[width=.18\columnwidth]{figures/supplementary/lindau00027_000019_ours.png}
  }\\[-2ex]



  \setcounter{subfigure}{0}
  \subfigure[\scriptsize Input]{%
    \includegraphics[width=.18\columnwidth]{figures/supplementary/lindau00029_000019_given.png}
  }
  \subfigure[\scriptsize Superpixels]{%
    \includegraphics[width=.18\columnwidth]{figures/supplementary/lindau00029_000019_sp.png}
  }
  \subfigure[\scriptsize GT]{%
    \includegraphics[width=.18\columnwidth]{figures/supplementary/lindau00029_000019_gt.png}
  }
  \subfigure[\scriptsize Deeplab]{%
    \includegraphics[width=.18\columnwidth]{figures/supplementary/lindau00029_000019_cnn.png}
  }
  \subfigure[\scriptsize Using BI]{%
    \includegraphics[width=.18\columnwidth]{figures/supplementary/lindau00029_000019_ours.png}
  }%\\[-2ex]

  \mycaption{Street Scene Segmentation}{Example results of street scene segmentation.
  (d)~depicts the DeepLab results, (e)~result obtained by adding bilateral inception (BI) modules (\bi{6}{2}+\bi{7}{6}) between \fc~layers.}
\label{fig:street_visuals-app}
\end{figure*}


\end{document}
