\section{Lower Bound on Committee-Monotone Algorithms for $1$-Borda}
\label{sec:monotone}
Consider the $1$-Borda score. A nice property of \g{} is that it is committee-monotone: In each iteration, the candidate chosen by \g{} only depends on which candidates have been chosen in previous iterations and not on $k$, and thus when $k$ increases, the committee selected by \g{} includes all the candidates \g{} used to select. On the other hand, \b{} does not satisfy committee monotonicity, as the candidates chosen by \b{} does depend on $k$.

This naturally brings up the question: Is there a committee-monotone algorithm which is optimal with respect to the benchmark \rand{}? We answer this question in the negative, by presenting a lower bound of $1.015$ for all committee-monotone algorithms.

%This sounds like a tradeoff between performance and monotonicity: on one hand, \b{} has better performance with respect to our benchmark than \g{} and on the other hand, \g{} satisfies committee monotonocity while \b{} does not. This brings up the natural question: is it possible for committee monotone algorithms to be optimal, in the sense of our benchmark? In the following theorem, we answer this question in the negative. We present a lower bound of $1.015$ for all committee monotone algorithms with respect to our benchmark, thus showing a separation between monotonicity and optimality.

\begin{theorem}
For any large enough $m$, there exist instances with $m$ candidates where any committee-monotone algorithm \alg{} satisfies $r_{\V}(T_k) > 1.015 \cdot \rand$ for some value $k \in \{1, 2\}$. Here, $T_k$ is the set of candidates \alg{} chooses when the size of this set is $k$.
\label{thm:monotonicity}
\end{theorem}
\begin{proof}
The construction goes as follows: There are two types of candidates, $X$ and $Y$. Candidates of type $X$ are ranked between $[am, bm]$ by every voter and candidates of type $Y$ are ranked between $[1, am] \cup [bm, m]$ by every voter, where $0 < a < b < 1$ are two parameters. We construct sufficiently many voters so that all candidates of the same type are symmetric (by having all permutations of candidates of type $X$ and those of type $Y$). We want to find proper $a$ and $b$, so that when $k = 1$, the optimal choice is to choose a candidate of type $X$, while when $k = 2$, the optimal choice is to choose two candidates both of type $Y$. This means that no committee-monotone algorithm can produce optimal choice for both the first iteration and the second iteration. We optimize over $a$ and $b$ to find the maximum lower bound.

In particular, the search procedure goes as follows. Let $r_{\V}(Y)$ denote the $1$-Borda score of choosing a candidate of type $Y$; $r_{\V}(XX)$ denote the score of choosing two candidates both of type $X$; and $r_{\V}(XY)$ denote the score of choosing a candidate of type $X$ and a candidate of type $Y$. We can easily see that, when $m$ goes to infinity, up to an $o(1)$ additive error:
\[
\begin{cases}
\frac{1}{m + 1}r_{\V}(Y) = \frac{a}{2} \cdot \Pr[Y \text{ is at } [1, am]] + \frac{1 + b}{2} \cdot \Pr[Y \text{ is at } [bm, m]] = \frac{a}{2} \cdot \frac{a}{1 - (b - a)} + \frac{1 + b}{2} \cdot \frac{1 - b}{1 - (b - a)}\\
\frac{1}{m + 1}r_{\V}(XX) = \frac{2a + b}{3}\\
\frac{1}{m + 1}r_{\V}(XY) = \frac{a}{2} \cdot \Pr[Y \text{ is at } [1, am]] + \frac{a + b}{2} \cdot \Pr[Y \text{ is at } [bm, m]] = \frac{a}{2} \cdot \frac{a}{1 - (b - a)} + \frac{a + b}{2} \cdot \frac{1 - b}{1 - (b - a)}
\end{cases}.
\]

A committee-monotone algorithm either chooses $Y$ in the first iteration, or chooses $XX$ or $XY$ in the first two iterations. Thus, we maximize $\min\left(\frac{2}{m + 1}r_{\V}(Y), \frac{3}{m + 1}r_{\V}(XX), \frac{3}{m + 1}r_{\V}(XY)\right)$ (note that the value on the numerator corresponds to the value of $k + 1$) over $0 < a < b < 1$, and find that, for $a = 0.377$ and $b = 0.552$, it achieves a lower bound greater than $1.015$.
\end{proof}
