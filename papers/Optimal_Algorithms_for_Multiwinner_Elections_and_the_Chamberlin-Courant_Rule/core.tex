\section{Connection to the Core}
\label{sec:core}
We now consider the relationship between the core and $1$-Borda score. In particular, we show that the core achieves a stronger notion of proportionality than $1$-Borda: any $\alpha$-approximate core solution has $1$-Borda score at most $\alpha \cdot \frac{k + 1}{k} \cdot \rand$, while neither the optimal solution $\opt$ nor the algorithms \g{} and \b{} lies in an $o(k)$-approximate core.

Recall that we say a committee $T$ of size $k$ is in the $\alpha$-approximate core if there is no blocking candidate strictly preferred by at least $\alpha \cdot \frac{n}{k}$ voters over $T$. See Eq~(\ref{eq:core}) for a formal definition. In this section, we investigate the relationship between $1$-Borda and the core.

First, we present in the following theorem the implication of the core for $1$-Borda score.

\begin{theorem}
Any committee $T$ in the $\alpha$-approximate core satisfies $r_{\V}(T) \leq \alpha \cdot \frac{k + 1}{k} \cdot \rand$.
\label{thm:core_to_borda}
\end{theorem}
\begin{proof}
As $T$ is in the $\alpha$-approximate core, there is no deviation of size $\frac{\alpha n}{k}$, i.e., there is no candidate ranked above all candidates in $T$ for $\frac{\alpha n}{k}$ voters. Therefore,
\[
\frac{1}{m - k} \sum_{v \in \V} (r_v(T) - 1) \leq \frac{\alpha n}{k}
\]
by a counting argument. Thus,
\[
r_{\V}(T) = \frac{1}{n} \sum_{v \in \V} r_v(T) \leq (m - k) \cdot \frac{\alpha}{k} = \alpha \cdot \frac{k + 1}{k} \cdot \frac{m + 1}{k + 1}. \qedhere
\]
\end{proof}

Naturally we ask: Does the reverse statement -- a good approximation to $\rand$ for the $1$-Borda score gives a good approximation to the core -- hold as well? It turns out that the answer is no.

\begin{example}
Let $n = 3 \cdot (m - 2)!$, where $m$ is sufficiently large. $c_1$ and $c_2$ are two ``critical'' candidates, and the remaining $m - 2$ are ``dummy'' candidates. For the first $\frac{n}{3}$ voters, $c_1$ is their top choices and $c_2$ is their second choices. For the second $\frac{n}{3}$ voters, $c_1$ is their bottom choices and $c_2$ is their top choices. For the last $\frac{n}{3}$ voters, $c_1$ is their bottom choice and $c_2$ is their second bottom choice. We fill the rest of their preferences with all permutations of the dummy candidates. This example is illustrated in Figure~\ref{fig:not_core}.
\label{ex:not_core}
\end{example}

\begin{figure}
\centering
\begin{tikzpicture}[yscale = 3.75, xscale = 5]

\draw [decorate, very thick, color3, decoration={brace,amplitude=10pt}] (-1, 2) -- (-0.37, 2);
\draw [decorate, very thick, color3, decoration={brace,amplitude=10pt}] (-0.315, 2) -- (0.315, 2);
\draw [decorate, very thick, color3, decoration={brace,amplitude=10pt}] (0.37, 2) -- (1, 2);

\draw [very thick, color1] (-1, 1.9) -- (-0.37, 1.9);
\draw [very thick, color1] (-1, 1.88) -- (-1, 1.92);
\draw [very thick, color1] (-0.37, 1.88) -- (-0.37, 1.92);

\draw [very thick, color2] (-1, 1.8) -- (-0.37, 1.8);
\draw [very thick, color2] (-1, 1.78) -- (-1, 1.82);
\draw [very thick, color2] (-0.37, 1.78) -- (-0.37, 1.82);

\draw [thick] (-1, 1.7) -- (-0.37, 1.7) -- (-0.37, 1.0) -- (-1, 1.0) -- (-1, 1.7);



\draw [very thick, color1] (-0.315, 1.0) -- (0.315, 1.0);
\draw [very thick, color1] (-0.315, 0.98) -- (-0.315, 1.02);
\draw [very thick, color1] (0.315, 0.98) -- (0.315, 1.02);

\draw [very thick, color2] (-0.315, 1.9) -- (0.315, 1.9);
\draw [very thick, color2] (-0.315, 1.88) -- (-0.315, 1.92);
\draw [very thick, color2] (0.315, 1.88) -- (0.315, 1.92);

\draw [thick] (-0.315, 1.8) -- (0.315, 1.8) -- (0.315, 1.1) -- (-0.315, 1.1) -- (-0.315, 1.8);



\draw [very thick, color1] (0.37, 1.0) -- (1, 1.0);
\draw [very thick, color1] (0.37, 0.98) -- (0.37, 1.02);
\draw [very thick, color1] (1, 0.98) -- (1, 1.02);

\draw [very thick, color2] (0.37, 1.1) -- (1, 1.1);
\draw [very thick, color2] (0.37, 1.08) -- (0.37, 1.12);
\draw [very thick, color2] (1, 1.08) -- (1, 1.12);

\draw [thick] (0.37, 1.9) -- (1, 1.9) -- (1, 1.2) -- (0.37, 1.2) -- (0.37, 1.9);


\draw [very thick, ->] (1.2, 1.9) -- (1.2, 1);

\node [color3] at (-0.685, 2.2) {First $\frac{n}{3}$ Voters};
\node [color3] at (0, 2.2) {Second $\frac{n}{3}$ Voters};
\node [color3] at (0.685, 2.2) {Last $\frac{n}{3}$ Voters};

\node [color1] at (-0.685, 1.95) {$c_1$};
\node [color2] at (-0.685, 1.85) {$c_2$};
\node at (-0.685, 1.35) {Dummy Candidates};


\node [color1] at (0, 1.05) {$c_1$};
\node [color2] at (0, 1.95) {$c_2$};
\node at (0, 1.45) {Dummy Candidates};


\node [color1] at (0.685, 1.05) {$c_1$};
\node [color2] at (0.685, 1.15) {$c_2$};
\node at (0.685, 1.55) {Dummy Candidates};


\node at (1.40, 1.45) {Preferences};
\node at (1.40, 1.55) {Decreasing};

\end{tikzpicture}
\caption{Illustration of Voters' Preferences in Example~\ref{ex:not_core}}
\label{fig:not_core}
\end{figure}


\begin{theorem}
The solutions of \opt{}, \g{} and \b{} do not lie in an $o(k)$-approximate core in Example~\ref{ex:not_core}. 
\label{thm:borda_to_core}
\end{theorem}
\begin{proof}
Let $k = \sqrt{m} - 1$ in Example~\ref{ex:not_core}. We show all of \opt{}, \g{} and \b{} chooses $c_2$ and $k - 1$ dummy candidates. In this solution, $\frac{n}{3}$ voters could deviate to $c_1$, showing that it does not lie in a $\frac{k}{3}$-approximate core.

\paragraph{Proof for \opt{}}  
%Clearly, \opt{} chooses $c_2$. \kn{Why?} Next, we compare the score for choosing $c_2$ and $k - 1$ dummy candidates with choosing $c_1$, $c_2$ and $k - 2$ dummy candidates. Let $T_{k - 1}$ be a set of candidates consisting of $c_2$ and $k - 2$ dummy candidates. Then, we have:
We compare the resulting $s$-Borda score for all possible schemes: choosing $c_1$, $c_2$, and $k - 2$ dummy candidates; choosing $c_1$ and $k - 1$ dummy candidates; choosing $c_2$ and $k - 1$ dummy candidates; and choosing $k$ dummy candidates. Let $D_j$ be a set consisting of $j$ dummy candidates. Then, we have:
\begin{align*}
r_{\V}(D_{k - 2} \cup \{c_1\} \cup \{c_2\}) = \frac{m + 1}{3(k - 1)} + \frac{2}{3}, &\quad r_{\V}(D_{k - 1} \cup \{c_1\}) = \frac{2(m + 1)}{3k} + \frac{1}{3},\\
r_{\V}(D_{k - 1} \cup \{c_2\}) = \frac{m + 1}{3k} + 1, &\quad r_{\V}(D_k) = \frac{m + 1}{k + 1}.
\end{align*}

For $k = \sqrt{m} - 1$, we have:
$$r_{\V}(D_{k - 1} \cup \{c_2\}) < r_{\V}(D_{k - 2} \cup \{c_1\}\cup\{c_2\}) <  r_{\V}(D_{k - 1} \cup \{c_1\}) < r_{\V}(D_k).$$

Thus, \opt{} chooses $c_2$ and $k - 1$ dummy candidates without choosing $c_1$.

\paragraph{Proof for \g{}}
For the first iteration, \g{} chooses $c_2$. We will show that, for the next $\sqrt{m} - 2$ iterations, \g{} chooses the dummy candidates and does not choose $c_1$. Suppose we have chosen $j - 1$ candidates, where $j - 1 \leq \sqrt{m} - 1$, and the current set of candidates is $T_{j - 1}$. Then, we have:
\[
r_{\V}(T_{j - 1}) - r_{\V}(T_{{j - 1}} \cup \{c_1\}) = \frac{1}{3},
\]
\[
r_{\V}(T_{j - 1}) - r_{\V}(T_{j - 1} \cup \{c_j\}) = \frac{1}{3}\left(\frac{m + 1}{j} - \frac{m + 1}{j - 1}\right) = \frac{m + 1}{3j(j - 1)} > \frac{1}{3}, \forall c_j \in \C \setminus T_{j - 1}, c_j \neq c_1.
\]
which shows that for the $\sqrt{m} - 2$ iterations after the first iteration, \g{} chooses dummy candidates.

\paragraph{Proof for \b{}}
Let $T_j$ be the set of candidates produced by \b{} after $j$ iterations. Recall that by \b{}, in the $j^{\text{th}}$ iteration, we pick $c_j \in \C \setminus T_{j - 1}$ that minimizes:
\[
\sum_{\substack{S \subseteq \C: |S| = k \\ S \supseteq T_{j - 1} \cup \{c_j\}}} r_{\V}(S).
\]

Clearly, \b{} chooses $c_2$ in the first iteration, because, as we have shown in the proof for \opt{}, for $k = \sqrt{m} - 1$, choosing $c_2$ always yields better result than not choosing $c_2$. 

Then, we show that \b{} chooses dummy candidates for the next $\sqrt{m} - 2$ iterations. Assume at $(j - 1)^{\text{th}}$ iteration, we have chosen $j - 2$ dummy candidates and $c_2$. As we have shown in the proof for \opt{}, for $k = \sqrt{m} - 1$, we have $r_{\V}(T_{k - 1} \cup \{c_j\}) < r_{\V}(T_{k - 1} \cup \{c_1\})$, $\forall c_j \in \C \setminus T_{k - 1}, c_j \neq c_1$, where $T_{k - 1}$ is a set consisting of $c_2$ and $k - 2$ dummy candidates. This implies that the candidate that minimizes the above objective is dummy candidate but not $c_1$. Thus, for the $j^{\text{th}}$ iteration, \b{} also chooses a dummy candidate, and by inductive principle, \b{} chooses $c_2$ and $k - 1$ dummy candidates in $k$ iterations.
\end{proof}

Theorem~\ref{thm:core_to_borda} and Theorem~\ref{thm:borda_to_core} together establishes that the core achieves a stronger notion of proportionality than $1$-Borda.

