
The primary objective of this work is to compare the performances of the perturbed underdamped Langevin dynamics (\ref{eq:perturbed_underdamped}) and the unperturbed dynamics (\ref{eq:langevin}) according to the criteria outlined in Section \ref{sec:comparison} and to find suitable choices for the matrices $J_{1}$, $J_{2}$, $M$ and $\Gamma$ that improve the performance of the sampler.  We begin our investigations of (\ref{eq:perturbed_underdamped}) by establishing ergodicity and exponentially fast return to equilibrium, and by studying the overdamped limit of~\eqref{eq:perturbed_underdamped}. As the latter turns out to be nonreversible and therefore in principle superior to the usual overdamped limit~\eqref{eq:overdamped},e.g.~\cite{Hwang2005}, this calculation provides us with further motivation to study the proposed dynamics.
\\\\
For the bulk of this work, we focus on the particular case when the target measure is Gaussian, i.e. when the potential is given by $V(q)=\frac{1}{2}q^{T}Sq$
with a symmetric and positive definite precision matrix $S$ (i.e. the covariance matrix is given by $S^{-1}$). In this
case, we advocate the following conditions for the choice of parameters:\begin{subequations}
	\label{eq:optimal parameters}
	\begin{align}
	M & =S,\label{eq:M=00003DS}\\
	\Gamma & =\gamma S,\\
	SJ_{1}S & =J_{2},\label{eq: perturbation condition}\\
	\mu & =\nu.
	\end{align}
\end{subequations}
Under the above choices \eqref{eq:optimal parameters}, we show that the large perturbation limit $\lim_{\mu\rightarrow\infty} \sigma_f^2$ exists and is finite and we provide an explicit expression for it (see Theorem \ref{cor:limit_asym_var}). From this expression, we derive an algorithm for finding optimal choices for $J_1$ in the case of quadratic observables (see Algorithm \ref{alg:optimal J general}).
\\\\
If the friction coefficient is not too small ($\gamma > \sqrt {2}$), and under certain mild nondegeneracy conditions, we prove that adding a small perturbation will always decrease the asymptotic variance for observables of the form $f(q)=q\cdot Kq+l\cdot q+C$:
\[
\left. \frac{\mathrm{d}}{\mathrm{d}\mu}\sigma_{f}^{2}\right\rvert_{\mu=0}=0\quad\text{and }\quad \left. \frac{\mathrm{d}^{2}}{\mathrm{d}\mu^{2}}\sigma_{f}^{2}\right\rvert_{\mu=0}<0,
\]
see Theorem \ref{cor:small pert unit var}. 
In fact, we conjecture that this statement is true for arbitrary observables
$f\in L^{2}(\pi)$, but we have not been able to prove this. The dynamics (\ref{eq:perturbed_underdamped})
(used in conjunction with the conditions (\ref{eq:M=00003DS})-(\ref{eq: perturbation condition}))
proves to be especially effective when the observable is antisymmetric
(i.e. when it is invariant under the substitution $q\mapsto-q$) or when it
has a significant antisymmetric part. In particular, in Proposition~\ref{prop:antisymmetric observables} we show that under certain conditions on the spectrum of $J_1$, for any antisymmetric observable $f\in L^{2}(\pi)$ it holds that  $\lim_{\mu\rightarrow\infty}\sigma_{f}^{2}=0$.
\\\\
Numerical experiments and analysis show that departing significantly
from~\ref{eq: perturbation condition} in fact possibly decreases
the performance of the sampler. This is in stark contrast to~\eqref{eq:nonreversible_overdamped}, where it is not possible to increase the asymptotic variance by \emph{any} perturbation.  For that reason, until now it seems practical to use (\ref{eq:perturbed_underdamped})  as a sampler only when a reasonable estimate of the global covariance of the target distribution is available. In the case of Bayesian inverse problems and diffusion bridge sampling, the target measure $\pi$ is given with respect to a Gaussian prior. We demonstrate the effectiveness of our approach in these applications, taking the prior Gaussian covariance as $S$ in (\ref{eq:M=00003DS})-(\ref{eq: perturbation condition}).
% In the case when the target measure is highly nonlinear, it it tempting
% to choose $J_{1}$ and $J_{2}$ in a position-dependent way, such
% that (\ref{eq: perturbation condition}) is satisfied locally (where
% $S$ then encodes the local covariance structure of the target, for
% instance being the expected Fisher information). This approach (which
% can be regarded as a manifold version of (\ref{eq: Perturbed Underdamped Langevin})
% and is thus similar to Riemannian manifold Monte Carlo,\cite{RiemannHMC})
% will be developed in a forthcoming publication. 
\begin{remark}
	In \cite[Rem. 3]{LelievreNierPavliotis2013} another modification of (\ref{eq:langevin})
	was suggested (albeit with the simplifications $\Gamma=\gamma\cdot I$
	and $M=I$):
\end{remark}
\begin{align}
\mathrm{d}q_{t} & =(1-J)M^{-1}p_{t}\mathrm{d}t ,\nonumber \\
\mathrm{d}p_{t} & =-(1+J)\nabla V(q_{t})\mathrm{d}t-\Gamma M^{-1}p_{t}\mathrm{d}t+\sqrt{2\Gamma}\mathrm{d}W_{t},\label{eq: JJ perturbation}
\end{align}
$J$ again denoting an antisymmetric matrix. However, under the change
of variables $p\mapsto(1+J)\tilde{p}$ the above equations transform
into 
\begin{align*}
\mathrm{d}q_{t} & =\tilde{M}^{-1}p_{t}\mathrm{d}t,\\
\mathrm{d}\tilde{p_{t}} & =-\nabla V(q_{t})\mathrm{d}t-\tilde{\Gamma}\tilde{M}^{-1}\tilde{p}_{t}\mathrm{d}t+\sqrt{2\tilde{\Gamma}}\mathrm{d}\tilde{W}_{t},
\end{align*}
where $\tilde{M}=(1+J)^{-1}M(1-J)^{-1}$ and $\tilde{\Gamma}=(1+J)^{-1}\Gamma(1-J)^{-1}$.
Since any observable $f$ depends only on $q$ (the $p$-variables
are merely auxiliary), the estimator $\pi_T(f)$ as well as its associated convergence characteristics (i.e. asymptotic
variance and speed of convergence to equilibrium) are invariant under this transformation.
Therefore, (\ref{eq: JJ perturbation}) reduces to the underdamped
Langevin dynamics (\ref{eq:langevin}) and does not represent an independent approach to sampling. Suitable choices
of $M$ and $\Gamma$ will be discussed in Section \ref{sec:arbitrary covariance}.

\subsection{Properties of Perturbed Underdamped Langevin Dynamics}
\label{sec:hypocoercivity}

In this section we study some of the properties of the perturbed underdamped dynamics (\ref{eq:perturbed_underdamped}). First, note that its generator is given by
\begin{equation}
\label{eq:generator}
\mathcal{L}=\underbrace{\underbrace{M^{-1}p\cdot\nabla_{q}-\nabla_{q}V\cdot\nabla_{p}}_{\mathcal{L}_{ham}}\underbrace{-\Gamma M^{-1}p\cdot\nabla_{p}+\Gamma : D^2_{p}}_{\mathcal{L}_{therm}}}_{\mathcal{L}_0} \underbrace{-\mu J_{1}\nabla V \cdot \nabla_{q} - \nu J M^{-1} p \cdot \nabla_{p}}_{\mathcal{L}_{pert}},
\end{equation}	
decomposed into the perturbation $\mathcal{L}_{pert}$ and the unperturbed operator $\mathcal{L}_0$, which can be further split into the Hamiltonian part $\mathcal{L}_{ham}$ and the thermostat (Ornstein-Uhlenbeck) part $\mathcal{L}_{therm}$, see \cite{pavliotis2014stochastic,Free_energy_computations,LS2016}.

\begin{lemma}
\label{lem:hypoellipticity}
	The infinitesimal generator $\gen$~\eqref{eq:generator} is hypoelliptic.
\end{lemma}
\begin{proof}
	See Appendix \ref{app:hypocoercivity}.\qed
\end{proof}

An immediate corollary of this result and of Theorem \ref{theorem:invariance_theorem} is that the perturbed underdamped Langevin process \eqref{eq:perturbed_underdamped} is ergodic with unique invariant distribution $\widehat{\pi}$ given by \eqref{eq:augmented target}.
\\\\
As explained in Section \ref{sec:comparison}, the exponential decay estimate \eqref{eq:hypocoercive estimate} is crucial for our approach, as in particular it guarantees the well-posedness of the Poisson equation \eqref{eq:poisson_general}. 
From now on, we will therefore make the following assumption on the potential $V,$ required to prove exponential decay in $L^2(\pi)$:

\begin{assumption}
	\label{ass:bounded+Poincare}
	Assume that the Hessian of $V$ is \emph{bounded} and that the target measure $\pi(\mathrm{d}q) = \frac{1}{Z}e^{-V}\mathrm{d}q$ satisfies a \emph{Poincare inequality}, i.e. there exists a constant $\rho>0$ such that 
	\begin{equation}
	\int_{\mathbb{R}^d}\phi^2\mathrm{d}\pi \le \rho \int_{\mathbb{R}^d} \vert \nabla \phi \vert ^2 \mathrm{d}\pi, 
	\end{equation}
	holds for all $\phi \in L_{0}^2(\pi)\cap H^1(\pi)$.
\end{assumption}
Sufficient conditions on the potential so that Poincar\'{e}'s inequality holds, e.g. the Bakry-Emery criterion, are presented in~\cite{bakry2013analysis}.
\begin{theorem}
	\label{theorem:Hypocoercivity}Under Assumption \ref{ass:bounded+Poincare} there exist constants $C\ge 1$ and $\lambda>0$ such that the semigroup $(P_t)_{t\ge0}$ generated by $\gen$ satisfies exponential decay in $L^2(\pi)$ as in \eqref{eq:hypocoercive estimate}.
\end{theorem}
\begin{proof}
	See Appendix \ref{app:hypocoercivity}.
\end{proof}
\begin{remark}
	The proof uses the machinery of hypocoercivity developed in \cite{villani2009hypocoercivity}.
	However, it seems likely that using the framework of \cite{DolbeaultMouhotSchmeiser2015},
	the assumption on the boundedness of the Hessian of $V$ can be substantially
	weakened.
\end{remark}

\subsection{The Overdamped Limit}
\label{sec:overdamped}

In this section we develop a connection between the perturbed underdamped
Langevin dynamics (\ref{eq:perturbed_underdamped}) and
the nonreversible overdamped Langevin dynamics (\ref{eq:nonreversible_overdamped}). The analysis is very similar to the one presented in \cite[Section 2.2.2]{Free_energy_computations} and we will be brief. For convenience in this section we will perform the analysis on the $d$-dimensional torus $\mathbb{T}^d \cong (\mathbb{R} / \mathbb{Z})^d$, i.e. we will assume $q \in \mathbb{T}^d$.
Consider the following scaling of (\ref{eq:perturbed_underdamped}):
\begin{subequations}
\begin{eqnarray}
\mathrm{d}q_{t}^{\epsilon} & = &  \frac{1}{\epsilon}M^{-1}p_{t}^{\epsilon},\mathrm{d}t-\mu J_{1}\nabla_{q}V(q_{t})\mathrm{d}t, \\
\mathrm{d}p_{t}^{\epsilon} & = & -\frac{1}{\epsilon}\nabla_{q}V(q_{t}^{\epsilon})\mathrm{d}t-\frac{1}{\epsilon^{2}}\nu J_{2}M^{-1}p_{t}^{\epsilon}\mathrm{d}t-\frac{1}{\epsilon^{2}}\Gamma M^{-1}p_{t}^{\epsilon}\mathrm{d}t+\frac{1}{\epsilon}\sqrt{2\Gamma}\mathrm{d}W_{t},
\end{eqnarray}
\label{eq:rescaling}
\end{subequations}
valid for the small mass/small momentum regime 
\begin{equation*}
M  \rightarrow\epsilon^{2}M, \quad   p_{t}  \rightarrow\epsilon p_{t}.
\end{equation*}
Equivalently, those modifications can be obtained from subsituting
$\Gamma\rightarrow\epsilon^{-1}\Gamma$ and $t\mapsto\epsilon^{-1}t$,
and so in the limit as $\epsilon\rightarrow0$ the dynamics (\ref{eq:rescaling})
describes the limit of large friction with rescaled time. It turns
out that as $\epsilon\rightarrow0$, the dynamics (\ref{eq:rescaling})
converges to the limiting SDE 
\begin{equation}
\mathrm{d}q_{t}=-(\nu J_{2}+\Gamma)^{-1}\nabla_{q}V(q_{t})\mathrm{d}t-\mu J_{1}\nabla_{q}V(q_{t})\mathrm{d}t+(\nu J_{2}+\Gamma)^{-1}\sqrt{2\Gamma}\mathrm{d}W_{t}.\label{eq:overdamped limit}
\end{equation}
The following proposition makes this statement precise.
\begin{proposition}
	\label{prop: overdamped limit}Denote by $(q_{t}^{\epsilon},p_{t}^{\epsilon})$
	the solution to (\ref{eq:rescaling}) with (deterministic) initial
	conditions $(q_{0}^{\epsilon},p_{0}^{\epsilon})=(q_{init},p_{init})$
	and by $q_{t}^{0}$ the solution to (\ref{eq:overdamped limit}) with
	initial condition $q_{0}^{0}=q_{init}.$ For any $T>0$, $(q_{t}^{\epsilon})_{0\le t\le T}$
	converges to $(q_{t}^{0})_{0\le t\le T}$ in $L^{2}(\Omega,C([0,T]),\mathbb{T}^{d})$
	as $\epsilon\rightarrow0$, i.e. 
	\[
	\lim_{\epsilon\rightarrow0}\mathbb{E}\big(\sup_{0\le t\le T}\vert q_{t}^{\epsilon}-q_{t}^{0}\vert^{2}\big)=0.
	\]
\end{proposition}
\begin{remark}
	By a refined analysis, it is possible to get information on the rate of convergence; see, e.g.~\cite{PavlSt03,PavSt05a}.
\end{remark}
The limiting SDE (\ref{eq:overdamped limit}) is nonreversible due to the term $-\mu J_1 \nabla_q V(q_t)\mathrm{d}t$ and also because the
matrix $(\nu J_{2}+\Gamma)^{-1}$ is in general neither symmetric
nor antisymmetric.
This result, together with the fact that nonreversible perturbations
of overdamped Langevin dynamics of the form \eqref{eq:nonreversible_overdamped} are by now well-known to have improved
performance properties, motivates further investigation of the dynamics
(\ref{eq:perturbed_underdamped}).

\begin{remark}
	The limit we described in this section respects the invariant distribution,
	in the sense that the limiting dynamics (\ref{eq:overdamped limit})
	is ergodic with respect to the measure $\pi(dq)=\frac{1}{Z}e^{-V}\mathrm{d}q.$
	To see this, we have to check that (we are using the notation $\nabla$ instead of $\nabla_q$) 
	\[
	\mathcal{L}^{\dagger}(e^{-V})=-\nabla\cdot\big((\nu J_{2}+\Gamma)^{-1}\nabla e^{-V}\big)+\nabla\cdot(\mu J_{1}\nabla e^{-V})+\nabla\cdot\big((\nu J_{2}+\Gamma)^{-1}\Gamma(-\nu J_{2}+\Gamma)^{-1}\nabla e^{-V}\big)=0,
	\]
	where $\mathcal{L}^{\dagger}$ refers to the $L^{2}(\mathbb{R}^{d})$-adjoint
	of the generator of (\ref{eq:overdamped limit}), i.e. to the associated Fokker-Planck operator. Indeed, the term
	$\nabla\cdot(\mu e^{-V}J_{1}\nabla V)$ vanishes because of the
	antisymmetry of $J_{1}.$ Therefore, it remains to show that 
	\[
	\nabla\cdot\big((\nu J_{2}+\Gamma)^{-1}\Gamma(-\nu J_{2}+\Gamma)^{-1}-(\nu J_{2}+\Gamma)^{-1}\big)\nabla e^{-V}\big)=0,
	\]
	i.e. that the matrix $(\nu J_{2}+\Gamma)^{-1}\Gamma(-\nu J_{2}+\Gamma)^{-1}-(\nu J_{2}+\Gamma)^{-1}$
	is antisymmetric. Clearly, the first term is symmetric and furthermore
	it turns out to be equal to the symmetric part of the second term:
		\begin{eqnarray*}
 \frac{1}{2}\big((\nu J_{2}+\Gamma)^{-1}+(-\nu J_{2}+\Gamma)^{-1}\big) & = &
	  =\frac{1}{2}\big((\nu J_{2}+\Gamma)^{-1}(-\nu J_{2}+\Gamma)(-\nu J_{2}+\Gamma)^{-1}  \\ && + (\nu J_{2}+\Gamma)^{-1}(\nu J_{2}+\Gamma)(-\nu J_{2}+\Gamma)^{-1}\big)\\
	& = & (\nu J_{2}+\Gamma)^{-1}\Gamma(-\nu J_{2}+\Gamma)^{-1},
		\end{eqnarray*}
	so $\pi$ is indeed invariant under the limiting dynamics (\ref{eq:overdamped limit}).
\end{remark}
	

