% \documentclass{svjour2}                    % onecolumn
\documentclass[onecollarge]{svjour2}       % onecolumn "king-size"

\smartqed  % flush right qed marks, e.g. at end of proof
%
\usepackage{graphicx}
\usepackage{color}
\usepackage{amsmath}
%\usepackage{amsthm}
\usepackage{amssymb}
% \usepackage{amsthm}
\usepackage{esint}
%\usepackage{hyperref}
\usepackage{mathtools}
\usepackage{float}
\usepackage{amsmath,amsfonts,amssymb,latexsym,epsfig}
%\usepackage[notref,notcite]{showkeys}
\usepackage[most]{tcolorbox}
\usepackage[colorinlistoftodos,prependcaption,textsize=tiny]{todonotes}
\usepackage{todonotes}
\usepackage{subcaption}
\usepackage{enumitem}
\usepackage{ntheorem}
%%%%%%%%%%%%%%%%%%%%%%%% referenc.tex %%%%%%%%%%%%%%%%%%%%%%%%%%%%%%
% sample references
% %
% Use this file as a template for your own input.
%
%%%%%%%%%%%%%%%%%%%%%%%% Springer-Verlag %%%%%%%%%%%%%%%%%%%%%%%%%%
%
% BibTeX users please use
% \bibliographystyle{}
% \bibliography{}
%

\biblstarthook{References may be \textit{cited} in the text either by number (preferred) or by author/year.\footnote{Make sure that all references from the list are cited in the text. Those not cited should be moved to a separate \textit{Further Reading} section or chapter.} The reference list should ideally be \textit{sorted} in alphabetical order -- even if reference numbers are used for the their citation in the text. If there are several works by the same author, the following order should be used: 
\begin{enumerate}
\item all works by the author alone, ordered chronologically by year of publication
\item all works by the author with a coauthor, ordered alphabetically by coauthor
\item all works by the author with several coauthors, ordered chronologically by year of publication.
\end{enumerate}
The \textit{styling} of references\footnote{Always use the standard abbreviation of a journal's name according to the ISSN \textit{List of Title Word Abbreviations}, see \url{http://www.issn.org/en/node/344}} depends on the subject of your book:
\begin{itemize}
\item The \textit{two} recommended styles for references in books on \textit{mathematical, physical, statistical and computer sciences} are depicted in ~\cite{science-contrib, science-online, science-mono, science-journal, science-DOI} and ~\cite{phys-online, phys-mono, phys-journal, phys-DOI, phys-contrib}.
\item Examples of the most commonly used reference style in books on \textit{Psychology, Social Sciences} are~\cite{psysoc-mono, psysoc-online,psysoc-journal, psysoc-contrib, psysoc-DOI}.
\item Examples for references in books on \textit{Humanities, Linguistics, Philosophy} are~\cite{humlinphil-journal, humlinphil-contrib, humlinphil-mono, humlinphil-online, humlinphil-DOI}.
\item Examples of the basic Springer style used in publications on a wide range of subjects such as \textit{Computer Science, Economics, Engineering, Geosciences, Life Sciences, Medicine, Biomedicine} are ~\cite{basic-contrib, basic-online, basic-journal, basic-DOI, basic-mono}. 
\end{itemize}
}

\begin{thebibliography}{99.}%
% and use \bibitem to create references.
%
% Use the following syntax and markup for your references if 
% the subject of your book is from the field 
% "Mathematics, Physics, Statistics, Computer Science"
%
% Contribution 
\bibitem{science-contrib} Broy, M.: Software engineering --- from auxiliary to key technologies. In: Broy, M., Dener, E. (eds.) Software Pioneers, pp. 10-13. Springer, Heidelberg (2002)
%
% Online Document
\bibitem{science-online} Dod, J.: Effective substances. In: The Dictionary of Substances and Their Effects. Royal Society of Chemistry (1999) Available via DIALOG. \\
\url{http://www.rsc.org/dose/title of subordinate document. Cited 15 Jan 1999}
%
% Monograph
\bibitem{science-mono} Geddes, K.O., Czapor, S.R., Labahn, G.: Algorithms for Computer Algebra. Kluwer, Boston (1992) 
%
% Journal article
\bibitem{science-journal} Hamburger, C.: Quasimonotonicity, regularity and duality for nonlinear systems of partial differential equations. Ann. Mat. Pura. Appl. \textbf{169}, 321--354 (1995)
%
% Journal article by DOI
\bibitem{science-DOI} Slifka, M.K., Whitton, J.L.: Clinical implications of dysregulated cytokine production. J. Mol. Med. (2000) doi: 10.1007/s001090000086 
%
\bigskip

% Use the following (APS) syntax and markup for your references if 
% the subject of your book is from the field 
% "Mathematics, Physics, Statistics, Computer Science"
%
% Online Document
\bibitem{phys-online} J. Dod, in \textit{The Dictionary of Substances and Their Effects}, Royal Society of Chemistry. (Available via DIALOG, 1999), 
\url{http://www.rsc.org/dose/title of subordinate document. Cited 15 Jan 1999}
%
% Monograph
\bibitem{phys-mono} H. Ibach, H. L\"uth, \textit{Solid-State Physics}, 2nd edn. (Springer, New York, 1996), pp. 45-56 
%
% Journal article
\bibitem{phys-journal} S. Preuss, A. Demchuk Jr., M. Stuke, Appl. Phys. A \textbf{61}
%
% Journal article by DOI
\bibitem{phys-DOI} M.K. Slifka, J.L. Whitton, J. Mol. Med., doi: 10.1007/s001090000086
%
% Contribution 
\bibitem{phys-contrib} S.E. Smith, in \textit{Neuromuscular Junction}, ed. by E. Zaimis. Handbook of Experimental Pharmacology, vol 42 (Springer, Heidelberg, 1976), p. 593
%
\bigskip
%
% Use the following syntax and markup for your references if 
% the subject of your book is from the field 
% "Psychology, Social Sciences"
%
%
% Monograph
\bibitem{psysoc-mono} Calfee, R.~C., \& Valencia, R.~R. (1991). \textit{APA guide to preparing manuscripts for journal publication.} Washington, DC: American Psychological Association.
%
% Online Document
\bibitem{psysoc-online} Dod, J. (1999). Effective substances. In: The dictionary of substances and their effects. Royal Society of Chemistry. Available via DIALOG. \\
\url{http://www.rsc.org/dose/Effective substances.} Cited 15 Jan 1999.
%
% Journal article
\bibitem{psysoc-journal} Harris, M., Karper, E., Stacks, G., Hoffman, D., DeNiro, R., Cruz, P., et al. (2001). Writing labs and the Hollywood connection. \textit{J Film} Writing, 44(3), 213--245.
%
% Contribution 
\bibitem{psysoc-contrib} O'Neil, J.~M., \& Egan, J. (1992). Men's and women's gender role journeys: Metaphor for healing, transition, and transformation. In B.~R. Wainrig (Ed.), \textit{Gender issues across the life cycle} (pp. 107--123). New York: Springer.
%
% Journal article by DOI
\bibitem{psysoc-DOI}Kreger, M., Brindis, C.D., Manuel, D.M., Sassoubre, L. (2007). Lessons learned in systems change initiatives: benchmarks and indicators. \textit{American Journal of Community Psychology}, doi: 10.1007/s10464-007-9108-14.
%
%
% Use the following syntax and markup for your references if 
% the subject of your book is from the field 
% "Humanities, Linguistics, Philosophy"
%
\bigskip
%
% Journal article
\bibitem{humlinphil-journal} Alber John, Daniel C. O'Connell, and Sabine Kowal. 2002. Personal perspective in TV interviews. \textit{Pragmatics} 12:257--271
%
% Contribution 
\bibitem{humlinphil-contrib} Cameron, Deborah. 1997. Theoretical debates in feminist linguistics: Questions of sex and gender. In \textit{Gender and discourse}, ed. Ruth Wodak, 99--119. London: Sage Publications.

\end{thebibliography}

%\part{\part{\part{title}}}
\usepackage{blkarray}

\newcommand{\E}{{\mathbb E}}
\newcommand{\cL}{\mathcal L}
\DeclareMathOperator{\Var}{Var}
\DeclareMathOperator{\Law}{Law}
\DeclareMathOperator{\Tr}{Tr}
\DeclareMathOperator{\Hess}{Hess}
\DeclareMathOperator{\diag}{diag}
\DeclareMathOperator{\Span}{span}
\DeclareMathOperator{\im}{im}
\DeclareMathOperator{\dist}{dist}

\DeclareMathOperator{\R}{\mathbb{R}}
\DeclareMathOperator{\Prob}{\mathbb{P}}
\DeclareMathOperator{\gen}{\mathcal{L}}

\newtheorem{assumption}{Assumption}

\newtheorem{algorithm}{Algorithm}


\newcommand{\inner}[2]{\left\langle #1, #2 \right\rangle}
\newcommand{\norm}[1]{\left|{#1}\right|} % A norm |x|
\newcommand{\Norm}[1]{\left\lVert{#1}\right\rVert} % A norm ||x||

%%%%%%%%%%Review Macros%%%%%%%%%%%
\newcommand{\andrew}[1]{\todo[linecolor=green,backgroundcolor=green!25,bordercolor=green]{A:#1}}

\newcommand{\notate}[1]{\textcolor{blue}{\textbf{[#1]}}}


\begin{document}

\title{Using Perturbed Underdamped Langevin Dynamics to Efficiently Sample
from Probability Distributions}

\titlerunning{Perturbed Underdamped Langevin Dynamics}        % if too long for running head

\author{A. B. Duncan         \and
        N. N{\"u}sken \and  
        G. A. Pavliotis
}

\authorrunning{Duncan, N{\"u}sken, Pavliotis} % if too long for running head

\begingroup
\institute{A. B. Duncan \at
              School of Mathematical and Physical Sciences, University of Sussex, Falmer, Brighton, BN1 9RH United Kingdom
              \email{Andrew.Duncan@sussex.ac.uk}           %  \\
%             \emph{Present address:} of F. Author  %  if needed
           \and
           N. N{\"u}sken \at
 Imperial College London, Department of Mathematics, South Kensington Campus, London SW7 2AZ,England \email{n.nusken14@imperial.ac.uk}
            \and
           G. A. Pavliotis \at
             Imperial College London, Department of Mathematics, South Kensington Campus, London SW7 2AZ,England \email{g.pavliotis@imperial.ac.uk}}

\date{Received: date / Accepted: date}


\maketitle
\begin{abstract}
In this paper we introduce and analyse Langevin samplers that consist of perturbations of the standard underdamped Langevin dynamics. The perturbed dynamics is such that its invariant measure is the same as that of the unperturbed dynamics. We show that appropriate choices of the perturbations can lead to samplers that have improved properties, at least in terms of reducing the asymptotic variance. We present a detailed analysis of the new Langevin sampler for Gaussian target distributions. Our theoretical results are supported by numerical experiments with non-Gaussian target measures.
\end{abstract}


\section{Introduction and Motivation}
\label{sec:introduction}
% \leavevmode
% \\
% \\
% \\
% \\
% \\
\section{Introduction}
\label{introduction}

AutoML is the process by which machine learning models are built automatically for a new dataset. Given a dataset, AutoML systems perform a search over valid data transformations and learners, along with hyper-parameter optimization for each learner~\cite{VolcanoML}. Choosing the transformations and learners over which to search is our focus.
A significant number of systems mine from prior runs of pipelines over a set of datasets to choose transformers and learners that are effective with different types of datasets (e.g. \cite{NEURIPS2018_b59a51a3}, \cite{10.14778/3415478.3415542}, \cite{autosklearn}). Thus, they build a database by actually running different pipelines with a diverse set of datasets to estimate the accuracy of potential pipelines. Hence, they can be used to effectively reduce the search space. A new dataset, based on a set of features (meta-features) is then matched to this database to find the most plausible candidates for both learner selection and hyper-parameter tuning. This process of choosing starting points in the search space is called meta-learning for the cold start problem.  

Other meta-learning approaches include mining existing data science code and their associated datasets to learn from human expertise. The AL~\cite{al} system mined existing Kaggle notebooks using dynamic analysis, i.e., actually running the scripts, and showed that such a system has promise.  However, this meta-learning approach does not scale because it is onerous to execute a large number of pipeline scripts on datasets, preprocessing datasets is never trivial, and older scripts cease to run at all as software evolves. It is not surprising that AL therefore performed dynamic analysis on just nine datasets.

Our system, {\sysname}, provides a scalable meta-learning approach to leverage human expertise, using static analysis to mine pipelines from large repositories of scripts. Static analysis has the advantage of scaling to thousands or millions of scripts \cite{graph4code} easily, but lacks the performance data gathered by dynamic analysis. The {\sysname} meta-learning approach guides the learning process by a scalable dataset similarity search, based on dataset embeddings, to find the most similar datasets and the semantics of ML pipelines applied on them.  Many existing systems, such as Auto-Sklearn \cite{autosklearn} and AL \cite{al}, compute a set of meta-features for each dataset. We developed a deep neural network model to generate embeddings at the granularity of a dataset, e.g., a table or CSV file, to capture similarity at the level of an entire dataset rather than relying on a set of meta-features.
 
Because we use static analysis to capture the semantics of the meta-learning process, we have no mechanism to choose the \textbf{best} pipeline from many seen pipelines, unlike the dynamic execution case where one can rely on runtime to choose the best performing pipeline.  Observing that pipelines are basically workflow graphs, we use graph generator neural models to succinctly capture the statically-observed pipelines for a single dataset. In {\sysname}, we formulate learner selection as a graph generation problem to predict optimized pipelines based on pipelines seen in actual notebooks.

%. This formulation enables {\sysname} for effective pruning of the AutoML search space to predict optimized pipelines based on pipelines seen in actual notebooks.}
%We note that increasingly, state-of-the-art performance in AutoML systems is being generated by more complex pipelines such as Directed Acyclic Graphs (DAGs) \cite{piper} rather than the linear pipelines used in earlier systems.  
 
{\sysname} does learner and transformation selection, and hence is a component of an AutoML systems. To evaluate this component, we integrated it into two existing AutoML systems, FLAML \cite{flaml} and Auto-Sklearn \cite{autosklearn}.  
% We evaluate each system with and without {\sysname}.  
We chose FLAML because it does not yet have any meta-learning component for the cold start problem and instead allows user selection of learners and transformers. The authors of FLAML explicitly pointed to the fact that FLAML might benefit from a meta-learning component and pointed to it as a possibility for future work. For FLAML, if mining historical pipelines provides an advantage, we should improve its performance. We also picked Auto-Sklearn as it does have a learner selection component based on meta-features, as described earlier~\cite{autosklearn2}. For Auto-Sklearn, we should at least match performance if our static mining of pipelines can match their extensive database. For context, we also compared {\sysname} with the recent VolcanoML~\cite{VolcanoML}, which provides an efficient decomposition and execution strategy for the AutoML search space. In contrast, {\sysname} prunes the search space using our meta-learning model to perform hyperparameter optimization only for the most promising candidates. 

The contributions of this paper are the following:
\begin{itemize}
    \item Section ~\ref{sec:mining} defines a scalable meta-learning approach based on representation learning of mined ML pipeline semantics and datasets for over 100 datasets and ~11K Python scripts.  
    \newline
    \item Sections~\ref{sec:kgpipGen} formulates AutoML pipeline generation as a graph generation problem. {\sysname} predicts efficiently an optimized ML pipeline for an unseen dataset based on our meta-learning model.  To the best of our knowledge, {\sysname} is the first approach to formulate  AutoML pipeline generation in such a way.
    \newline
    \item Section~\ref{sec:eval} presents a comprehensive evaluation using a large collection of 121 datasets from major AutoML benchmarks and Kaggle. Our experimental results show that {\sysname} outperforms all existing AutoML systems and achieves state-of-the-art results on the majority of these datasets. {\sysname} significantly improves the performance of both FLAML and Auto-Sklearn in classification and regression tasks. We also outperformed AL in 75 out of 77 datasets and VolcanoML in 75  out of 121 datasets, including 44 datasets used only by VolcanoML~\cite{VolcanoML}.  On average, {\sysname} achieves scores that are statistically better than the means of all other systems. 
\end{itemize}


%This approach does not need to apply cleaning or transformation methods to handle different variances among datasets. Moreover, we do not need to deal with complex analysis, such as dynamic code analysis. Thus, our approach proved to be scalable, as discussed in Sections~\ref{sec:mining}.

\section{Construction of General Langevin Samplers}
\label{sec:background}
\section{Background and Motivation}

\subsection{IBM Streams}

IBM Streams is a general-purpose, distributed stream processing system. It
allows users to develop, deploy and manage long-running streaming applications
which require high-throughput and low-latency online processing.

The IBM Streams platform grew out of the research work on the Stream Processing
Core~\cite{spc-2006}.  While the platform has changed significantly since then,
that work established the general architecture that Streams still follows today:
job, resource and graph topology management in centralized services; processing
elements (PEs) which contain user code, distributed across all hosts,
communicating over typed input and output ports; brokers publish-subscribe
communication between jobs; and host controllers on each host which
launch PEs on behalf of the platform.

The modern Streams platform approaches general-purpose cluster management, as
shown in Figure~\ref{fig:streams_v4_v6}. The responsibilities of the platform
services include all job and PE life cycle management; domain name resolution
between the PEs; all metrics collection and reporting; host and resource
management; authentication and authorization; and all log collection. The
platform relies on ZooKeeper~\cite{zookeeper} for consistent, durable metadata
storage which it uses for fault tolerance.

Developers write Streams applications in SPL~\cite{spl-2017} which is a
programming language that presents streams, operators and tuples as
abstractions. Operators continuously consume and produce tuples over streams.
SPL allows programmers to write custom logic in their operators, and to invoke
operators from existing toolkits. Compiled SPL applications become archives that
contain: shared libraries for the operators; graph topology metadata which tells
both the platform and the SPL runtime how to connect those operators; and
external dependencies. At runtime, PEs contain one or more operators. Operators
inside of the same PE communicate through function calls or queues. Operators
that run in different PEs communicate over TCP connections that the PEs
establish at startup. PEs learn what operators they contain, and how to connect
to operators in other PEs, at startup from the graph topology metadata provided
by the platform.

We use ``legacy Streams'' to refer to the IBM Streams version 4 family. The
version 5 family is for Kubernetes, but is not cloud native. It uses the
lift-and-shift approach and creates a platform-within-a-platform: it deploys a
containerized version of the legacy Streams platform within Kubernetes.

\subsection{Kubernetes}

Borg~\cite{borg-2015} is a cluster management platform used internally at Google
to schedule, maintain and monitor the applications their internal infrastructure
and external applications depend on. Kubernetes~\cite{kube} is the open-source
successor to Borg that is an industry standard cloud orchestration platform.

From a user's perspective, Kubernetes abstracts running a distributed
application on a cluster of machines. Users package their applications into
containers and deploy those containers to Kubernetes, which runs those
containers in \emph{pods}. Kubernetes handles all life cycle management of pods,
including scheduling, restarting and migration in case of failures.

Internally, Kubernetes tracks all entities as \emph{objects}~\cite{kubeobjects}.
All objects have a name and a specification that describes its desired state.
Kubernetes stores objects in etcd~\cite{etcd}, making them persistent,
highly-available and reliably accessible across the cluster. Objects are exposed
to users through \emph{resources}. All resources can have
\emph{controllers}~\cite{kubecontrollers}, which react to changes in resources.
For example, when a user changes the number of replicas in a
\code{ReplicaSet}, it is the \code{ReplicaSet} controller which makes sure the
desired number of pods are running. Users can extend Kubernetes through
\emph{custom resource definitions} (CRDs)~\cite{kubecrd}. CRDs can contain
arbitrary content, and controllers for a CRD can take any kind of action.

Architecturally, a Kubernetes cluster consists of nodes. Each node runs a
\emph{kubelet} which receives pod creation requests and makes sure that the
requisite containers are running on that node. Nodes also run a
\emph{kube-proxy} which maintains the network rules for that node on behalf of
the pods. The \emph{kube-api-server} is the central point of contact: it
receives API requests, stores objects in etcd, asks the scheduler to schedule
pods, and talks to the kubelets and kube-proxies on each node. Finally,
\emph{namespaces} logically partition the cluster. Objects which should not know
about each other live in separate namespaces, which allows them to share the
same physical infrastructure without interference.

\subsection{Motivation}
\label{sec:motivation}

Systems like Kubernetes are commonly called ``container orchestration''
platforms. We find that characterization reductive to the point of being
misleading; no one would describe operating systems as ``binary executable
orchestration.'' We adopt the idea from Verma et al.~\cite{borg-2015} that
systems like Kubernetes are ``the kernel of a distributed system.'' Through CRDs
and their controllers, Kubernetes provides state-as-a-service in a distributed
system. Architectures like the one we propose are the result of taking that view 
seriously.

The Streams legacy platform has obvious parallels to the Kubernetes
architecture, and that is not a coincidence: they solve similar problems.
Both are designed to abstract running arbitrary user-code across a distributed
system.  We suspect that Streams is not unique, and that there are many
non-trivial platforms which have to provide similar levels of cluster
management.  The benefits to being cloud native and offloading the platform
to an existing cloud management system are: 
\begin{itemize}
    \item Significantly less platform code.
    \item Better scheduling and resource management, as all services on the cluster are 
        scheduled by one platform.
    \item Easier service integration.
    \item Standardized management, logging and metrics.
\end{itemize}
The rest of this paper presents the design of replacing the legacy Streams 
platform with Kubernetes itself.



\section{Perturbation of Underdamped Langevin Dynamics}
\label{sec:perturbed_langevin}

The primary objective of this work is to compare the performances of the perturbed underdamped Langevin dynamics (\ref{eq:perturbed_underdamped}) and the unperturbed dynamics (\ref{eq:langevin}) according to the criteria outlined in Section \ref{sec:comparison} and to find suitable choices for the matrices $J_{1}$, $J_{2}$, $M$ and $\Gamma$ that improve the performance of the sampler.  We begin our investigations of (\ref{eq:perturbed_underdamped}) by establishing ergodicity and exponentially fast return to equilibrium, and by studying the overdamped limit of~\eqref{eq:perturbed_underdamped}. As the latter turns out to be nonreversible and therefore in principle superior to the usual overdamped limit~\eqref{eq:overdamped},e.g.~\cite{Hwang2005}, this calculation provides us with further motivation to study the proposed dynamics.
\\\\
For the bulk of this work, we focus on the particular case when the target measure is Gaussian, i.e. when the potential is given by $V(q)=\frac{1}{2}q^{T}Sq$
with a symmetric and positive definite precision matrix $S$ (i.e. the covariance matrix is given by $S^{-1}$). In this
case, we advocate the following conditions for the choice of parameters:\begin{subequations}
	\label{eq:optimal parameters}
	\begin{align}
	M & =S,\label{eq:M=00003DS}\\
	\Gamma & =\gamma S,\\
	SJ_{1}S & =J_{2},\label{eq: perturbation condition}\\
	\mu & =\nu.
	\end{align}
\end{subequations}
Under the above choices \eqref{eq:optimal parameters}, we show that the large perturbation limit $\lim_{\mu\rightarrow\infty} \sigma_f^2$ exists and is finite and we provide an explicit expression for it (see Theorem \ref{cor:limit_asym_var}). From this expression, we derive an algorithm for finding optimal choices for $J_1$ in the case of quadratic observables (see Algorithm \ref{alg:optimal J general}).
\\\\
If the friction coefficient is not too small ($\gamma > \sqrt {2}$), and under certain mild nondegeneracy conditions, we prove that adding a small perturbation will always decrease the asymptotic variance for observables of the form $f(q)=q\cdot Kq+l\cdot q+C$:
\[
\left. \frac{\mathrm{d}}{\mathrm{d}\mu}\sigma_{f}^{2}\right\rvert_{\mu=0}=0\quad\text{and }\quad \left. \frac{\mathrm{d}^{2}}{\mathrm{d}\mu^{2}}\sigma_{f}^{2}\right\rvert_{\mu=0}<0,
\]
see Theorem \ref{cor:small pert unit var}. 
In fact, we conjecture that this statement is true for arbitrary observables
$f\in L^{2}(\pi)$, but we have not been able to prove this. The dynamics (\ref{eq:perturbed_underdamped})
(used in conjunction with the conditions (\ref{eq:M=00003DS})-(\ref{eq: perturbation condition}))
proves to be especially effective when the observable is antisymmetric
(i.e. when it is invariant under the substitution $q\mapsto-q$) or when it
has a significant antisymmetric part. In particular, in Proposition~\ref{prop:antisymmetric observables} we show that under certain conditions on the spectrum of $J_1$, for any antisymmetric observable $f\in L^{2}(\pi)$ it holds that  $\lim_{\mu\rightarrow\infty}\sigma_{f}^{2}=0$.
\\\\
Numerical experiments and analysis show that departing significantly
from~\ref{eq: perturbation condition} in fact possibly decreases
the performance of the sampler. This is in stark contrast to~\eqref{eq:nonreversible_overdamped}, where it is not possible to increase the asymptotic variance by \emph{any} perturbation.  For that reason, until now it seems practical to use (\ref{eq:perturbed_underdamped})  as a sampler only when a reasonable estimate of the global covariance of the target distribution is available. In the case of Bayesian inverse problems and diffusion bridge sampling, the target measure $\pi$ is given with respect to a Gaussian prior. We demonstrate the effectiveness of our approach in these applications, taking the prior Gaussian covariance as $S$ in (\ref{eq:M=00003DS})-(\ref{eq: perturbation condition}).
% In the case when the target measure is highly nonlinear, it it tempting
% to choose $J_{1}$ and $J_{2}$ in a position-dependent way, such
% that (\ref{eq: perturbation condition}) is satisfied locally (where
% $S$ then encodes the local covariance structure of the target, for
% instance being the expected Fisher information). This approach (which
% can be regarded as a manifold version of (\ref{eq: Perturbed Underdamped Langevin})
% and is thus similar to Riemannian manifold Monte Carlo,\cite{RiemannHMC})
% will be developed in a forthcoming publication. 
\begin{remark}
	In \cite[Rem. 3]{LelievreNierPavliotis2013} another modification of (\ref{eq:langevin})
	was suggested (albeit with the simplifications $\Gamma=\gamma\cdot I$
	and $M=I$):
\end{remark}
\begin{align}
\mathrm{d}q_{t} & =(1-J)M^{-1}p_{t}\mathrm{d}t ,\nonumber \\
\mathrm{d}p_{t} & =-(1+J)\nabla V(q_{t})\mathrm{d}t-\Gamma M^{-1}p_{t}\mathrm{d}t+\sqrt{2\Gamma}\mathrm{d}W_{t},\label{eq: JJ perturbation}
\end{align}
$J$ again denoting an antisymmetric matrix. However, under the change
of variables $p\mapsto(1+J)\tilde{p}$ the above equations transform
into 
\begin{align*}
\mathrm{d}q_{t} & =\tilde{M}^{-1}p_{t}\mathrm{d}t,\\
\mathrm{d}\tilde{p_{t}} & =-\nabla V(q_{t})\mathrm{d}t-\tilde{\Gamma}\tilde{M}^{-1}\tilde{p}_{t}\mathrm{d}t+\sqrt{2\tilde{\Gamma}}\mathrm{d}\tilde{W}_{t},
\end{align*}
where $\tilde{M}=(1+J)^{-1}M(1-J)^{-1}$ and $\tilde{\Gamma}=(1+J)^{-1}\Gamma(1-J)^{-1}$.
Since any observable $f$ depends only on $q$ (the $p$-variables
are merely auxiliary), the estimator $\pi_T(f)$ as well as its associated convergence characteristics (i.e. asymptotic
variance and speed of convergence to equilibrium) are invariant under this transformation.
Therefore, (\ref{eq: JJ perturbation}) reduces to the underdamped
Langevin dynamics (\ref{eq:langevin}) and does not represent an independent approach to sampling. Suitable choices
of $M$ and $\Gamma$ will be discussed in Section \ref{sec:arbitrary covariance}.

\subsection{Properties of Perturbed Underdamped Langevin Dynamics}
\label{sec:hypocoercivity}

In this section we study some of the properties of the perturbed underdamped dynamics (\ref{eq:perturbed_underdamped}). First, note that its generator is given by
\begin{equation}
\label{eq:generator}
\mathcal{L}=\underbrace{\underbrace{M^{-1}p\cdot\nabla_{q}-\nabla_{q}V\cdot\nabla_{p}}_{\mathcal{L}_{ham}}\underbrace{-\Gamma M^{-1}p\cdot\nabla_{p}+\Gamma : D^2_{p}}_{\mathcal{L}_{therm}}}_{\mathcal{L}_0} \underbrace{-\mu J_{1}\nabla V \cdot \nabla_{q} - \nu J M^{-1} p \cdot \nabla_{p}}_{\mathcal{L}_{pert}},
\end{equation}	
decomposed into the perturbation $\mathcal{L}_{pert}$ and the unperturbed operator $\mathcal{L}_0$, which can be further split into the Hamiltonian part $\mathcal{L}_{ham}$ and the thermostat (Ornstein-Uhlenbeck) part $\mathcal{L}_{therm}$, see \cite{pavliotis2014stochastic,Free_energy_computations,LS2016}.

\begin{lemma}
\label{lem:hypoellipticity}
	The infinitesimal generator $\gen$~\eqref{eq:generator} is hypoelliptic.
\end{lemma}
\begin{proof}
	See Appendix \ref{app:hypocoercivity}.\qed
\end{proof}

An immediate corollary of this result and of Theorem \ref{theorem:invariance_theorem} is that the perturbed underdamped Langevin process \eqref{eq:perturbed_underdamped} is ergodic with unique invariant distribution $\widehat{\pi}$ given by \eqref{eq:augmented target}.
\\\\
As explained in Section \ref{sec:comparison}, the exponential decay estimate \eqref{eq:hypocoercive estimate} is crucial for our approach, as in particular it guarantees the well-posedness of the Poisson equation \eqref{eq:poisson_general}. 
From now on, we will therefore make the following assumption on the potential $V,$ required to prove exponential decay in $L^2(\pi)$:

\begin{assumption}
	\label{ass:bounded+Poincare}
	Assume that the Hessian of $V$ is \emph{bounded} and that the target measure $\pi(\mathrm{d}q) = \frac{1}{Z}e^{-V}\mathrm{d}q$ satisfies a \emph{Poincare inequality}, i.e. there exists a constant $\rho>0$ such that 
	\begin{equation}
	\int_{\mathbb{R}^d}\phi^2\mathrm{d}\pi \le \rho \int_{\mathbb{R}^d} \vert \nabla \phi \vert ^2 \mathrm{d}\pi, 
	\end{equation}
	holds for all $\phi \in L_{0}^2(\pi)\cap H^1(\pi)$.
\end{assumption}
Sufficient conditions on the potential so that Poincar\'{e}'s inequality holds, e.g. the Bakry-Emery criterion, are presented in~\cite{bakry2013analysis}.
\begin{theorem}
	\label{theorem:Hypocoercivity}Under Assumption \ref{ass:bounded+Poincare} there exist constants $C\ge 1$ and $\lambda>0$ such that the semigroup $(P_t)_{t\ge0}$ generated by $\gen$ satisfies exponential decay in $L^2(\pi)$ as in \eqref{eq:hypocoercive estimate}.
\end{theorem}
\begin{proof}
	See Appendix \ref{app:hypocoercivity}.
\end{proof}
\begin{remark}
	The proof uses the machinery of hypocoercivity developed in \cite{villani2009hypocoercivity}.
	However, it seems likely that using the framework of \cite{DolbeaultMouhotSchmeiser2015},
	the assumption on the boundedness of the Hessian of $V$ can be substantially
	weakened.
\end{remark}

\subsection{The Overdamped Limit}
\label{sec:overdamped}

In this section we develop a connection between the perturbed underdamped
Langevin dynamics (\ref{eq:perturbed_underdamped}) and
the nonreversible overdamped Langevin dynamics (\ref{eq:nonreversible_overdamped}). The analysis is very similar to the one presented in \cite[Section 2.2.2]{Free_energy_computations} and we will be brief. For convenience in this section we will perform the analysis on the $d$-dimensional torus $\mathbb{T}^d \cong (\mathbb{R} / \mathbb{Z})^d$, i.e. we will assume $q \in \mathbb{T}^d$.
Consider the following scaling of (\ref{eq:perturbed_underdamped}):
\begin{subequations}
\begin{eqnarray}
\mathrm{d}q_{t}^{\epsilon} & = &  \frac{1}{\epsilon}M^{-1}p_{t}^{\epsilon},\mathrm{d}t-\mu J_{1}\nabla_{q}V(q_{t})\mathrm{d}t, \\
\mathrm{d}p_{t}^{\epsilon} & = & -\frac{1}{\epsilon}\nabla_{q}V(q_{t}^{\epsilon})\mathrm{d}t-\frac{1}{\epsilon^{2}}\nu J_{2}M^{-1}p_{t}^{\epsilon}\mathrm{d}t-\frac{1}{\epsilon^{2}}\Gamma M^{-1}p_{t}^{\epsilon}\mathrm{d}t+\frac{1}{\epsilon}\sqrt{2\Gamma}\mathrm{d}W_{t},
\end{eqnarray}
\label{eq:rescaling}
\end{subequations}
valid for the small mass/small momentum regime 
\begin{equation*}
M  \rightarrow\epsilon^{2}M, \quad   p_{t}  \rightarrow\epsilon p_{t}.
\end{equation*}
Equivalently, those modifications can be obtained from subsituting
$\Gamma\rightarrow\epsilon^{-1}\Gamma$ and $t\mapsto\epsilon^{-1}t$,
and so in the limit as $\epsilon\rightarrow0$ the dynamics (\ref{eq:rescaling})
describes the limit of large friction with rescaled time. It turns
out that as $\epsilon\rightarrow0$, the dynamics (\ref{eq:rescaling})
converges to the limiting SDE 
\begin{equation}
\mathrm{d}q_{t}=-(\nu J_{2}+\Gamma)^{-1}\nabla_{q}V(q_{t})\mathrm{d}t-\mu J_{1}\nabla_{q}V(q_{t})\mathrm{d}t+(\nu J_{2}+\Gamma)^{-1}\sqrt{2\Gamma}\mathrm{d}W_{t}.\label{eq:overdamped limit}
\end{equation}
The following proposition makes this statement precise.
\begin{proposition}
	\label{prop: overdamped limit}Denote by $(q_{t}^{\epsilon},p_{t}^{\epsilon})$
	the solution to (\ref{eq:rescaling}) with (deterministic) initial
	conditions $(q_{0}^{\epsilon},p_{0}^{\epsilon})=(q_{init},p_{init})$
	and by $q_{t}^{0}$ the solution to (\ref{eq:overdamped limit}) with
	initial condition $q_{0}^{0}=q_{init}.$ For any $T>0$, $(q_{t}^{\epsilon})_{0\le t\le T}$
	converges to $(q_{t}^{0})_{0\le t\le T}$ in $L^{2}(\Omega,C([0,T]),\mathbb{T}^{d})$
	as $\epsilon\rightarrow0$, i.e. 
	\[
	\lim_{\epsilon\rightarrow0}\mathbb{E}\big(\sup_{0\le t\le T}\vert q_{t}^{\epsilon}-q_{t}^{0}\vert^{2}\big)=0.
	\]
\end{proposition}
\begin{remark}
	By a refined analysis, it is possible to get information on the rate of convergence; see, e.g.~\cite{PavlSt03,PavSt05a}.
\end{remark}
The limiting SDE (\ref{eq:overdamped limit}) is nonreversible due to the term $-\mu J_1 \nabla_q V(q_t)\mathrm{d}t$ and also because the
matrix $(\nu J_{2}+\Gamma)^{-1}$ is in general neither symmetric
nor antisymmetric.
This result, together with the fact that nonreversible perturbations
of overdamped Langevin dynamics of the form \eqref{eq:nonreversible_overdamped} are by now well-known to have improved
performance properties, motivates further investigation of the dynamics
(\ref{eq:perturbed_underdamped}).

\begin{remark}
	The limit we described in this section respects the invariant distribution,
	in the sense that the limiting dynamics (\ref{eq:overdamped limit})
	is ergodic with respect to the measure $\pi(dq)=\frac{1}{Z}e^{-V}\mathrm{d}q.$
	To see this, we have to check that (we are using the notation $\nabla$ instead of $\nabla_q$) 
	\[
	\mathcal{L}^{\dagger}(e^{-V})=-\nabla\cdot\big((\nu J_{2}+\Gamma)^{-1}\nabla e^{-V}\big)+\nabla\cdot(\mu J_{1}\nabla e^{-V})+\nabla\cdot\big((\nu J_{2}+\Gamma)^{-1}\Gamma(-\nu J_{2}+\Gamma)^{-1}\nabla e^{-V}\big)=0,
	\]
	where $\mathcal{L}^{\dagger}$ refers to the $L^{2}(\mathbb{R}^{d})$-adjoint
	of the generator of (\ref{eq:overdamped limit}), i.e. to the associated Fokker-Planck operator. Indeed, the term
	$\nabla\cdot(\mu e^{-V}J_{1}\nabla V)$ vanishes because of the
	antisymmetry of $J_{1}.$ Therefore, it remains to show that 
	\[
	\nabla\cdot\big((\nu J_{2}+\Gamma)^{-1}\Gamma(-\nu J_{2}+\Gamma)^{-1}-(\nu J_{2}+\Gamma)^{-1}\big)\nabla e^{-V}\big)=0,
	\]
	i.e. that the matrix $(\nu J_{2}+\Gamma)^{-1}\Gamma(-\nu J_{2}+\Gamma)^{-1}-(\nu J_{2}+\Gamma)^{-1}$
	is antisymmetric. Clearly, the first term is symmetric and furthermore
	it turns out to be equal to the symmetric part of the second term:
		\begin{eqnarray*}
 \frac{1}{2}\big((\nu J_{2}+\Gamma)^{-1}+(-\nu J_{2}+\Gamma)^{-1}\big) & = &
	  =\frac{1}{2}\big((\nu J_{2}+\Gamma)^{-1}(-\nu J_{2}+\Gamma)(-\nu J_{2}+\Gamma)^{-1}  \\ && + (\nu J_{2}+\Gamma)^{-1}(\nu J_{2}+\Gamma)(-\nu J_{2}+\Gamma)^{-1}\big)\\
	& = & (\nu J_{2}+\Gamma)^{-1}\Gamma(-\nu J_{2}+\Gamma)^{-1},
		\end{eqnarray*}
	so $\pi$ is indeed invariant under the limiting dynamics (\ref{eq:overdamped limit}).
\end{remark}
	



\section{Sampling from a Gaussian Distribution}
\label{sec:Gaussian}

In this section we study in detail the performance of the Langevin sampler~\eqref{eq:perturbed_underdamped} for Gaussian target densities, first considering the case of unit covariance. In particular, we study the optimal choice for the parameters in the sampler, the exponential decay rate and the asymptotic variance. We then extend our results to Gaussian target densities with arbitrary covariance matrices.

\subsection{Unit covariance - small perturbations}
\label{sec:small perturbations}

In our study of the dynamics given by \eqref{eq:perturbed_underdamped}
we first consider the simple case when $V(q)=\frac{1}{2}\vert q\vert^{2}$,
i.e. the task of sampling from a Gaussian measure with unit covariance.
We will assume $M=I$, $\Gamma=\gamma I$ and $J_{1}=J_{2}=:J$
(so that the $q-$ and $p-$dynamics are perturbed in the same way,
albeit posssibly with different strengths $\mu$ and $\nu$). Using
these simplifications, (\ref{eq:perturbed_underdamped})
reduces to the linear system 
\begin{align}
\mathrm{d}q_{t} & =p_{t}\mathrm{d}t-\mu Jq_{t}\mathrm{d}t\nonumber, \\
\mathrm{d}p_{t} & =-q_{t}\mathrm{d}t-\nu Jp_{t}\mathrm{d}t-\gamma p_{t}\mathrm{d}t+\sqrt{2\gamma}\mathrm{d}W_{t}.\label{eq:unit covariance}
\end{align}
The above dynamics are of Ornstein-Uhlenbeck type, i.e. we can write
\begin{equation}
\mathrm{d}X_{t}=-BX_{t}\mathrm{d}t+\sqrt{2Q}\mathrm{d}\bar{W}_{t}\label{eq:OU process}
\end{equation}
with $X=(q,p)^{T}$, 
\begin{equation}
B=\left(\begin{array}{cc}
\mu J & -I\\
I & \gamma I+\nu J
\end{array}\right),\label{eq:drift matrix}
\end{equation}
\begin{equation}
Q=\left(\begin{array}{cc}
\boldsymbol{0} & \boldsymbol{0}\\
\boldsymbol{0} & \gamma I
\end{array}\right)\label{eq:diffusion matrix}
\end{equation}
and $(\bar{W}_{t})_{t\ge0}$ denoting a standard Wiener process on
$\mathbb{R}^{2d}$. The generator of (\ref{eq:OU process}) is then
given by 
\begin{equation}
\mathcal{L}=-Bx\cdot\nabla+\nabla^{T}Q\nabla.\label{eq:OU generator}
\end{equation}
We will consider quadratic observables of the form 
\[
f(q)=q\cdot Kq+l\cdot q+C,
\]
with $K\in\mathbb{R}_{sym}^{d\times d}$, $l\in\mathbb{R}^{d}$ and
$C\in\mathbb{R}$, however it is worth recalling that the asymptotic variance $\sigma^2_f$ does not depend on $C$. We also stress that $f$ is assumed to be independent of
$p$ as those extra degrees of freedom are merely auxiliary. Our
aim will be to study the associated asymptotic variance $\sigma_{f}^{2}$, see equation (\ref{eq:asymptoticvariance}), in particular its
dependence on the parameters $\mu$ and $\nu$.  This dependence is encoded in the
function 
\begin{alignat*}{1}
\Theta:\quad\mathbb{R}^{2} & \rightarrow\mathbb{R}\\
(\mu,\nu) & \mapsto\sigma_{f}^{2},
\end{alignat*}
assuming a fixed observable $f$ and perturbation matrix $J$. In
this section we will focus on small perturbations, i.e. on the behaviour
of the function $\Theta$ in the neighbourhood of the origin. Our
main theoretical tool will be the Poisson equation \eqref{eq:poisson_general}, see the proofs in Appendix \ref{app:Gaussian_proofs}. Anticipating the forthcoming analysis, let us already state our main result, showing that in the neighbourhood of the origin, the function $\Theta$ has favourable properties along the diagonal $\mu=\nu$ (note that the perturbation strengths in the first and second line of \eqref{eq: unit covariance-1-1} coincide):

\begin{theorem}
	\label{cor:small pert unit var}Consider the dynamics 
\begin{align}
\mathrm{d}q_{t} & =p_{t}\mathrm{d}t-\mu Jq_{t}\mathrm{d}t,\nonumber \\
\mathrm{d}p_{t} & =-q_{t}\mathrm{d}t-\mu Jp_{t}\mathrm{d}t-\gamma p_{t}\mathrm{d}t+\sqrt{2\gamma}\mathrm{d}W_{t},\label{eq: unit covariance-1-1}
\end{align}
with $\gamma>\sqrt{2}$ and an observable of the form $f(q)=q\cdot Kq+l\cdot q+C$.
If at least one of the conditions $[J,K]\neq0$ and $l\notin\ker J$
is satisfied, then the asymptotic variance of the unperturbed sampler
is at a local maximum independently of $K$ and $J$ (and $\gamma$,
as long as $\gamma>\sqrt{2}$), i.e. 
\[
\left. \partial_{\mu}\sigma_{f}^{2} \right\rvert_{\mu =0}=0
\]
and 
\[
\left. \partial_{\mu}^{2}\sigma_{f}^{2}\right\rvert_{\mu = 0}<0.
\]
\end{theorem}



\subsubsection{\label{sub:Purely-quadratic-observables}Purely quadratic observables}

Let us start with the case $l=0$, i.e. $f(q)=q\cdot Kq+C$. The following
holds: 
\begin{proposition}
	\label{thm: local quadratic observable}The function $\Theta$ satisfies
	\begin{equation}
	\left. \nabla\Theta\right\rvert_{(\mu,\nu)=(0,0)}=0\label{eq:gradTheta}
	\end{equation}
	and 
	\begin{equation}
	\left. \Hess\Theta\right\rvert_{(\mu,\nu)=(0,0)}=\left(\begin{array}{cc}
	-(\gamma+\frac{1}{\gamma^{3}}+\gamma^{3})\left(\Tr(JKJK)-\Tr(J^{2}K^{2})\right) & (\frac{1}{\gamma^{3}}+\frac{1}{\gamma}-\gamma)\Tr(J^{2}K^{2})\\
	-\frac{2}{\gamma}\Tr(JKJK) & +(-\frac{1}{\gamma^{3}}+\frac{1}{\gamma}+\gamma)\Tr(JKJK)\\
	(\frac{1}{\gamma^{3}}+\frac{1}{\gamma}-\gamma)\Tr(J^{2}K^{2}) & (\frac{1}{\gamma^{3}}-\frac{1}{\gamma})\Tr(J^{2}K^{2})\\
	+(-\frac{1}{\gamma^{3}}+\frac{1}{\gamma}+\gamma)\Tr(JKJK) & -(\frac{1}{\gamma^{3}}+\frac{1}{\gamma})\Tr(JKJK)
	\end{array}\right).\label{eq:HessTheta}
	\end{equation}
\end{proposition}
\begin{proof}
	See Appendix \ref{app:Gaussian_proofs}.
	\qed
\end{proof}
The above proposition shows that the unperturbed dynamics represents a
critical point of $\Theta$, independently of the choice of $K$,
$J$ and $\gamma$. In general though, $\Hess\Theta\vert_{(\mu,\nu)=(0,0)}$
can have both positive and negative eigenvalues. In particular, this
implies that an unfortunate choice of the perturbations will actually
increase the asymptotic variance of the dynamics (in contrast to the situation
of perturbed \emph{overdamped }Langevin dynamics, where any nonreversible
perturbation leads to an improvement in asymptotic variance as detailed
in \cite{asvar_Hwang} and \cite{duncan2016variance}). Furthermore, the nondiagonality
of $\Hess\Theta\vert_{(\mu,\nu)=(0,0)}$ hints at the fact that the
interplay of the perturbations $J_{1}$ and $J_{2}$ (or rather their
relative strengths $\mu$ and $\nu$) is crucial for the performance
of the sampler and, consequently, the effect of these perturbations cannot be satisfactorily
studied independently. 
\begin{example}
	Assuming $J^{2}=-I$ and $[J,K]=0$ it follows that
	\[
	\left. \partial_{\mu}^{2}\Theta\right\rvert_{\mu=0}=\left. \partial_{\nu}^{2}\Theta\right\rvert_{\mu=0}=\frac{1}{\gamma}\Tr(K^{2})>0,
	\]
	for all nonzero $K$. Therefore in this case, a small perturbation of $J_{1}$ only or $J_{2}$ only will increase  the asymptotic variance, uniformly over all choices of $K$ and $\gamma$.
\end{example}
However, it turns out that it is possible to construct an improved sampler
by combining both perturbations in a suitable way. Indeed,
the function $\Theta$ can be seen to have good properties along $\mu=\nu$. We set $\mu(s)=s$, $\nu(s):=s$ and compute

\begin{align*}
\left. \frac{\mathrm{d}^{2}}{\mathrm{d}s^{2}}\Theta\right\rvert_{s=0} & =(1,1)\cdot\Hess\Theta\vert_{(\mu,\nu)=(0,0)}(1,1)\\
& =-(\gamma+\frac{1}{\gamma^{3}}+\gamma^{3})\left(\Tr(JKJK)-\Tr(J^{2}K^{2})\right)-\frac{2}{\gamma}\Tr(JKJK)\\
& +2\cdot\left((\frac{1}{\gamma^{3}}+\frac{1}{\gamma}-\gamma)\Tr(J^{2}K^{2})+(-\frac{1}{\gamma^{3}}+\frac{1}{\gamma}+\gamma)\Tr(JKJK)\right)\\
& +(\frac{1}{\gamma^{3}}-\frac{1}{\gamma})\Tr(J^{2}K^{2})-(\frac{1}{\gamma^{3}}+\frac{1}{\gamma})\Tr(JKJK)\\
& =\big(\gamma-\frac{4}{\gamma^{3}}-\gamma^{3}-\frac{1}{\gamma}\big)\cdot(\Tr(JKJK)-\Tr(J^{2}K^{2}))\le0.
\end{align*}
The last inequality follows from 
\[
\gamma-\frac{4}{\gamma^{3}}-\gamma^{3}-\frac{1}{\gamma}<0
\]
and 
\[
\Tr(JKJK)-\Tr(J^{2}K^{2})\ge0
\]
(both inequalities are proven in the Appendix, Lemma \ref{lem:basic_inequalities}), where the last inequality
is strict if $[J,K]\neq0$. Consequently, choosing both perturbations
to be of the same magnitude ($\mu=\nu$) and assuring that $J$ and
$K$ do not commute always leads to a smaller asymptotic variance,
independently of the choice of $K$, $J$ and $\gamma$. We state
this result in the following corrolary:
\begin{corollary}
	\label{cor:unit covariance quadratic obs}Consider the dynamics 
	\begin{align}
	\mathrm{d}q_{t} & =p_{t}\mathrm{d}t-\mu Jq_{t}\mathrm{d}t\nonumber, \\
	\mathrm{d}p_{t} & =-q_{t}\mathrm{d}t-\mu Jp_{t}\mathrm{d}t-\gamma p_{t}\mathrm{d}t+\sqrt{2\gamma}\mathrm{d}W_{t},\label{eq: unit covariance-1}
	\end{align}
 and a quadratic observable $f(q)=q\cdot Kq+C$. If $[J,K] \neq 0,$
	then the asymptotic variance of the unperturbed sampler is at a local
	maximum independently of K, $J$ and $\gamma$, i.e. 
	\[
	\left. \partial_{\mu}\sigma_{f}^{2}\right\rvert_{\mu=0}=0
	\]
	and 
	\[
	\left. \partial_{\mu}^{2}\sigma_{f}^{2}\right\rvert_{\mu=0}<0.
	\]
\end{corollary}
\begin{remark}
	As we will see in Section \ref{sec:large perturbations}, more precisely Example \ref{ex:commutation quadratic observables}, if $[J,K]=0,$
	the asymptotic variance is constant as a function of $\mu$, i.e.
	the perturbation has no effect.\end{remark}
\begin{example}
\label{ex:opposed perturbation}
	Let us set $\mu(s):=s$ and $\nu(s):=-s$ (this corresponds to a small
	perturbation with $J\nabla V(q_{t})\mathrm{d}t$ in $q$ and $-Jp_{t}\mathrm{d}t$
	in $p$). In this case we get 
	\[
	\left. \frac{\mathrm{d}^{2}\Theta}{\mathrm{d}s^{2}} \right\rvert_{s=0}=\underbrace{-\frac{1}{2}\cdot\frac{\gamma^{4}+3\gamma^{2}+5}{\gamma}\left(\Tr(JKJK)-\Tr(J^{2}K^{2})\right)}_{\le0}\underbrace{-4\frac{\Tr(J^{2}K^{2})}{\gamma}}_{\ge0},
	\]
	which changes its sign depending on $J$ and $K$ as the first term
	is negative and the second is positive. Whether the perturbation improves the performance of the sampler in terms of asymptotic variance therefore depends on the specifics of the observable and the perturbation in this case.
	
\end{example}

\subsubsection{Linear observables}

Here we consider the case $K=0$, i.e. $f(q)=l\cdot q+C$, where
again $l\in\mathbb{R}^{d}$ and $C\in\mathbb{R}$. We have the following
result: 
\begin{proposition}
	\label{thm:linear_full_J}The function $\Theta$ satisfies 
	\[
	\nabla\Theta\vert_{(\mu,\nu)=(0,0)}=0
	\]
	and 
	\[
	\Hess\Theta\vert_{(\mu,\nu)=(0,0)}=\left(\begin{array}{cc}
	-2\gamma^{3}\vert Jl\vert^{2} & 2\gamma\vert Jl\vert^{2}\\
	2\gamma\vert Jl\vert^{2} & 0
	\end{array}\right).
	\]
\end{proposition}
\begin{proof}
	See Appendix \ref{app:Gaussian_proofs}.
	\qed
\end{proof}
	Let us assume that $l\notin\ker J$. Then $\partial_{\mu}^{2}\Theta\vert_{\mu,\nu=0}<0$,
	and hence Theorem \ref{thm:linear_full_J} shows that a small perturbation
	by $\mu J\nabla V(q_{t})\mathrm{d}t$ alone always results in an improvement
	of the asymptotic variance. However, if we combine both perturbations
	$\mu J\nabla V(q_{t})\mathrm{d}t$ and $\nu Jp_{t}\mathrm{d}t$, then
	the effect depends on the sign of 
	\[
	\left(\begin{array}{cc}
	\mu & \nu\end{array}\right)\left(\begin{array}{cc}
	-2\gamma^{3}\vert Jl\vert^{2} & 2\gamma\vert Jl\vert^{2}\\
	2\gamma\vert Jl\vert^{2} & 0
	\end{array}\right)\left(\begin{array}{c}
	\mu\\
	\nu
	\end{array}\right)=-(2\mu^{2}\gamma^{3}-4\mu\nu\gamma)\vert Jl\vert^{2}.
	\]
	This will be negative if $\mu$ and $\nu$ have different signs, and
	also if they have the same sign and $\gamma$ is big enough.
Following Section \ref{sub:Purely-quadratic-observables}, we require
$\mu=\nu$. We then end up with the requirement 
\[
2\mu^{2}\gamma^{3}-4\mu\nu\gamma>0,
\]
which is satisfied if $\gamma>\sqrt{2}$

Summarizing the results of this section, for observables of the form
$f(q)=q\cdot Kq+l\cdot q+C$, choosing equal perturbations ($\mu=\nu$)
with a sufficiently strong damping $(\gamma>\sqrt{2}$) always leads
to an improvement in asymptotic variance under the conditions $[J,K]\neq0$
and $l\notin\ker J$. This is finally the content of Theorem \ref{cor:small pert unit var}.
\begin{figure}
	\begin{subfigure}[b]{0.5 \textwidth}
		\includegraphics[width=\textwidth]{q_equal}
		\caption{Equal perturbations: $\mu=\nu$}
		\label{fig:asym_quad}
	\end{subfigure}
	\hfill
	\begin{subfigure}[b]{0.5 \textwidth}
		\includegraphics[width=\textwidth]{q_09}
		\caption{Approximately equal perturbations: $\mu=0.9\nu$}
		\label{fig:no_limit1}
	\end{subfigure}
	\hfill
	\begin{subfigure}[b]{0.5 \textwidth}
		\includegraphics[width=\textwidth]{q_opp}
		\caption{Opposing perturbations: $\mu=-\nu$ 
			\label{fig:no_limit2}}
	\end{subfigure}
	\hfill
	\begin{subfigure}[b]{0.5 \textwidth}
		\includegraphics[width=\textwidth]{l_equalg25.pdf}
		\caption{Equal perturbations: $\mu=\nu$ (sufficiently large friction $\gamma$)}
		\label{fig:lin_large_friction}
	\end{subfigure}
	\hfill
	\begin{subfigure}[b]{0.5 \textwidth}
		\includegraphics[width=\textwidth]{l_equal_g1.pdf}
		
		\caption{Equal perturbations: $\mu=\nu$ (small friction $\gamma$)}
		\label{fig:lin_small_friction}
	\end{subfigure}
	\caption{Asymptotic variance for linear and quadratic observables, depending on relative perturbation and friction strengths}
	\label{fig:linear and quadratic observables}
\end{figure}

Let us illustrate the results of this section by plotting the asymptotic variance as a function of the perturbation strength $\mu$ (see Figure \ref{fig:linear and quadratic observables}), making the choices $d=2$, $l=(1,1)^{T}$,
\begin{equation}
K=\left(\begin{array}{cc}
2 & 0\\
0 & 1
\end{array}\right)
\quad \text{and} \quad
J=\left(\begin{array}{cc}
0 & 1\\
-1 & 0
\end{array}\right).
\end{equation}
The asymptotic variance has been computed according to \eqref{eq:Gaussian asymvar}, using \eqref{eq:Lyapunov equation} and \eqref{eq:linear condition} from Appendix \ref{app:Gaussian_proofs}. The graphs confirm the results summarized in Corollary \ref{cor:small pert unit var} concerning the asymptotic variance in the neighbourhood of the unperturbed dynamics ($\mu = 0$). Additionally, they give an impression of the global behaviour, i.e. for larger values of $\mu$.

Figures \ref{fig:asym_quad}, \ref{fig:no_limit1} and \ref{fig:no_limit2}  show the asymptotic variance associated with the quadratic observable $f(q)=q\cdot K q$. In accordance with Corollary \ref{cor:unit covariance quadratic obs}, the asymptotic variance is at a  local maximum at zero perturbation in the case $\mu=\nu$ (see Figure \ref{fig:asym_quad}). For increasing perturbation strength, the graph shows that it decays monotonically
and reaches a limit for $\mu\rightarrow\infty$ (this limiting behaviour will be explored analytically in Section \ref{sec:large perturbations}). If the condition $\mu=\nu$ is only approximately satisfied (Figure \ref{fig:no_limit1}), our numerical examples still exhibits decaying asymptotic variance in the neighbourhood of the critical point. In this case, however, the asymptotic variance diverges for growing values of the perturbation $\mu$. If the perturbations are opposed ($\mu=-\nu$) as in Example \ref{ex:opposed perturbation}, it is possible for certain observables that the unperturbed dynamics represents a global minimum. Such a case is observed in Figure \ref{fig:no_limit2}. In Figures \ref{fig:lin_large_friction} and \ref{fig:lin_small_friction} the observable $f(q)=l\cdot q$ is considered. If the damping is sufficiently strong ($\gamma > \sqrt{2}$), the unperturbed dynamics is at a local maximum of the asymptotic variance (Figure \ref{fig:lin_large_friction}). Furthermore, the asymptotic variance approaches zero as $\mu \rightarrow \infty$ (for a theoretical explanation see again Section \ref{sec:large perturbations}). The graph in Figure \ref{fig:lin_small_friction} shows that the assumption of $\gamma$ not being too small cannot be dropped from Corollary \ref{cor:small pert unit var}. Even in this case though the example shows decay of the asymptotic variance for large values of $\mu$.    
\subsection{Exponential decay rate}
\label{sec:exp_decay}
Let us denote by $\lambda^{*}$ the \emph{optimal exponential decay rate} in \eqref{eq:hypocoercive estimate}, i.e.
\begin{equation}
\lambda^{*}=\sup\{\lambda > 0 \, \vert \, \text{There exists } C\ge 1 \text{ such that } \eqref{eq:hypocoercive estimate} \text{ holds}\}.
\end{equation}
Note that $\lambda^{*}$ is well-defined and positive by Theorem \ref{theorem:Hypocoercivity}. We also define the \emph{spectral bound} of the generator $\gen$ by
\begin{equation}
s(\gen)=\inf(\text{Re}\,\sigma(-\gen)\setminus\{0\}).
\end{equation} 
In \cite{Metafune_formula} it is proven that the Ornstein-Uhlenbeck semigroup $(P_t)_{t\ge0}$ considered in this section is differentiable (see Proposition 2.1). In this case (see Corollary 3.12 of \cite{Engel2000Semigroup}), it is known that the exponential decay rate and the spectral bound coincide, i.e. $\lambda^{*}=s(\gen)$, whereas in general only $\lambda^{*}\le s(\gen)$ holds.
In this section we will therefore analyse the spectral properties of the generator
(\ref{eq:OU generator}). In particular, this leads to some intuition
of why choosing equal perturbations ($\mu=\nu$) is crucial for the
performance of the sampler.

In \cite{Metafune_formula} (see also \cite{OPP12}), it was proven that
the spectrum of $\mathcal{L}$ as in (\ref{eq:OU generator}) in $L^{2}(\widehat{\pi})$
is given by 
\begin{equation}
\sigma(\mathcal{L})=\left\{-\sum_{j=1}^{r}n_{j}\lambda_{j}:\, n_{j}\in\mathbb{N},\lambda_{j}\in \sigma(B)\right\}.\label{eq:Metafune formula}
\end{equation}
Note that $\sigma(\mathcal{L})$ only depends on the drift matrix
$B$. In the case where $\mu=\nu$,
the spectrum of $B$ can be computed explicitly. 
\begin{lemma}
	\label{lem:drift matrix properties}Assume $\mu=\nu$. Then the spectrum
	of $B$ is given by
	\begin{equation}
	\sigma(B)=\left\{\mu\lambda+\sqrt{\big(\frac{\gamma}{2}\big)^{2}-1}+\frac{\gamma}{2}\vert\lambda\in\sigma(J)\}\cup\{\mu\lambda-\sqrt{\big(\frac{\gamma}{2}\big)^{2}-1}+\frac{\gamma}{2}\vert\lambda\in\sigma(J)\right\}.\label{eq:spectrum of B}
	\end{equation}
\end{lemma}
\begin{proof}
	We will compute $\sigma\big(B-\frac{\gamma}{2}I\big)$ and then use
	the identity
	\begin{equation}
	\sigma(B)=\left\{\lambda+\frac{\gamma}{2}\vert\lambda\in\sigma\left(B-\frac{\gamma}{2}I\right)\right\}.\label{eq:shift spectrum}
	\end{equation}
	We have 
	\begin{align*}
	\det\left(B-\frac{\gamma}{2}I-\lambda I\right) & =\det\left(\left(\mu J-\frac{\gamma}{2}I-\lambda I\right)\left(\mu J+\frac{\gamma}{2}I-\lambda I\right)+I\right)\\
	& =\det\left((\mu J-\lambda I)^{2}-\left(\frac{\gamma}{2}\right)^{2}I+I\right)\\
	& =\det\left(\left(\mu J-\lambda I+\sqrt{\left(\frac{\gamma}{2}\right)^{2}-1} I\right)\cdot\left(\mu J-\lambda I-\sqrt{\left(\frac{\gamma}{2} \right)^{2}-1} I\right)\right)\\
	& =\det\left(\mu J-\lambda I+\sqrt{\left(\frac{\gamma}{2}\right)^{2}-1} I\right)\cdot\det\left(\mu J-\lambda I-\sqrt{\left(\frac{\gamma}{2}\right)^{2}-1} I\right),
	\end{align*}
	where $I$ is understood to denote the identity matrix of appropriate dimension.
	The above quantity is zero if and only if 
	\[
	\lambda-\sqrt{\left(\frac{\gamma}{2}\right)^{2}-1}\in\sigma(\mu J)
	\]
	or 
	\[
	\lambda+\sqrt{\left(\frac{\gamma}{2}\right)^{2}-1}\in\sigma(\mu J).
	\]
	Together with (\ref{eq:shift spectrum}), the claim follows.
	\qed
\end{proof}
Using formula \eqref{eq:Metafune formula}, in Figure \ref{fig:good_spectrum} we show a sketch of the spectrum $\sigma(-\mathcal{L}$)
for the case of equal perturbations ($\mu=\nu)$ with the convenient
choices $n=1$ and $\gamma=2.$ Of course, the eigenvalue at $0$ is
associated to the invariant measure since $\sigma(-\mathcal{L})=\sigma(-\mathcal{L}^{\dagger})$
and $\mathcal{L}^{\dagger}\widehat{\pi}=0$, where $\mathcal{L}^{\dagger}$ denotes the Fokker-Planck operator, i.e. the $L^2(\mathbb{R}^{2d})$-adjoint of $\mathcal{L}$. The arrows indicate the movement
of the eigenvalues as the perturbation $\mu$ increases in accordance
with Lemma \ref{lem:drift matrix properties}. Clearly, the spectral
bound of $\gen$ is not affected by the perturbation. 
Note that the eigenvalues on the real axis stay invariant under the
perturbation. The subspace of $L_{0}^{2}(\widehat{\pi})$ associated to
those will turn out to be crucial for the characterisation of the
limiting asymptotic variance as $\mu\rightarrow\infty$.

To illustrate the suboptimal properties of the perturbed dynamics
when the perturbations are not equal, we plot the spectrum of the
drift matrix $\sigma(B)$ in the case when the dynamics is only perturbed
by the term $\nu J_{2}p\mathrm{d}t$ (i.e. $\mu=0$) for $n=2$, $\gamma=2$ and
\begin{equation}
J_2=\left(\begin{array}{cc}
0 & -1\\
1 & 0
\end{array}\right),
\end{equation} 
(see Figure \ref{fig:bad_spectrum}). Note that the full spectrum $\sigma(-\mathcal{L})$
can be inferred from (\ref{eq:Metafune formula}). For $\nu=0$ we have that the spectrum $\sigma(B)$
only consists of the (degenerate) eigenvalue $1$. For increasing
$\nu$, the figure shows that the degenerate eigenvalue splits up
into four eigenvalues, two of which get closer to the imaginary axis as $\nu$ increases, leading to a smaller spectral
bound and therefore to a decrease in the speed of convergence to equilibrium.
Figures (\ref{fig:good_spectrum}) and (\ref{fig:bad_spectrum}) give an intuitive explanation
of why the fine-tuning of the perturbation strengths is crucial.

\begin{figure}
	\begin{subfigure}[b]{0.45 \textwidth}
		\includegraphics[width=\textwidth]{spectrum2.pdf}
		\caption{$\sigma(-\gen)$ in the case $\mu=\nu$. The arrows indicate the movement of the spectrum as the perturbation strength $\mu$ increases.\label{fig:good_spectrum}}
	\end{subfigure}
	\hfill
	\begin{subfigure}[b]{0.45 \textwidth}
		\includegraphics[width=\textwidth]{moving2.pdf}
		\caption{$\sigma(B)$ in the case $J_{1}=0$, i.e. the dynamics is only perturbed
			by $-\nu J_{2}p\mathrm{d}t$. The arrows indicate the movement of
			the eigenvalues as $\nu$ increases.\label{fig:bad_spectrum}}
	\end{subfigure}
	.	\caption{Effects of the perturbation on the spectra of $-\gen$ and $B$.}
	\label{fig:examples-introduction}
\end{figure}

\subsection{Unit covariance - large perturbations}
\label{sec:large perturbations}

In the previous subsection we observed that for the particular perturbation $J_1 = J_2$ and $\mu = \nu$, i.e. 
\begin{align}
\mathrm{d}q_{t} & =p_{t}\mathrm{d}t-\mu Jq_{t}\mathrm{d}t\nonumber \\
\mathrm{d}p_{t} & =-q_{t}\mathrm{d}t-\mu Jp_{t}\mathrm{d}t-\gamma p_{t}\mathrm{d}t+\sqrt{2\gamma}\,\mathrm{d}W_{t},\label{eq: unit covariance perfect perturbation}
\end{align}
the perturbed Langevin dynamics demonstrated an improvement in performance for $\mu$ in a neighbourhood of $0$, when the observable is linear or quadratic.  Recall that this dynamics is ergodic with respect to a standard Gaussian measure $\widehat{\pi}$ on $\mathbb{R}^{2d}$ with marginal $\pi$  with respect to the $q$--variable.  In the following we shall consider only observables that do not depend on $p$. Moreover, we assume without loss of generality that $\pi(f)=0$. For such an observable we will write $f\in L^2_0(\pi)$ and assume the canonical embedding $L^2_0(\pi)\subset L^2(\widehat{\pi})$.  The infinitesimal generator of (\ref{eq: unit covariance perfect perturbation})
is given by 
\begin{equation}
\label{eq:generator_equal}
\mathcal{L}=\underbrace{p\cdot\nabla_{q}-q\cdot\nabla_{p}+\gamma(-p\cdot\nabla_{p}+\Delta_{p})}_{\mathcal{L}_{0}}+\mu\underbrace{(-Jq\cdot\nabla_{q}-Jp\cdot\nabla_{p})}_{\mathcal{A}}=:\mathcal{L}_{0}+\mu\mathcal{A},
\end{equation}
where we have introduced the notation $\mathcal{L}_{pert}=\mu \mathcal{A}$. In the sequel, the adjoint of an operator $B$ in $L^2(\widehat{\pi})$ will be denoted by $B^{*}$. In the rest of this section we will make repeated use of the Hermite polynomials
\begin{equation}
g_{\alpha}(x)=(-1)^{\vert\alpha\vert}e^{\frac{\vert x\vert^{2}}{2}}\nabla^{\alpha}e^{-\frac{\vert x\vert^{2}}{2}},\quad\alpha\in\mathbb{N}^{2d},\label{eq: Hermite polynomials}
\end{equation}
invoking the notation $x=(q,p)\in\mathbb{R}^{2d}$. For $m\in\mathbb{N}_{0}$
define the spaces 
\[
\label{eq:Hermite spaces}
H_{m}=\Span\{g_{\alpha}:\,\vert\alpha\vert=m\},
\]
with induced scalar product 
\[
\langle f,g\rangle_{m}:=\langle f,g\rangle_{L^2(\widehat{\pi})},\quad f,g\in H_{m}.
\]
The space $(H_{m},\langle\cdot,\cdot\rangle_{m})$ is then a real Hilbert
space with (finite) dimension
\[
\dim H_{m}=\left(\begin{array}{c}
m+2d-1\\
m
\end{array}\right).
\]
The following result (Theorem \ref{thm:L2 decomposition}) holds for operators of the form
\begin{equation}
\label{eq:OU_operator}
\mathcal{L}=-Bx\cdot\nabla+\nabla^{T}Q\nabla,
\end{equation}
where the quadratic drift and diffusion matrices $B$ and $Q$ are such that $\mathcal{L}$ is the generator of an ergodic stochastic process (see \cite[Definition 2.1]{Arnold2014} for precise conditions on $B$ and $Q$ that ensure ergodicity). The generator of the SDE \eqref{eq: unit covariance perfect perturbation} is given by \eqref{eq:OU_operator} with $B$ and $Q$  as in equations \eqref{eq:drift matrix}
and \eqref{eq:diffusion matrix}, respectively.  The following result provides an orthogonal decomposition of $L^{2}(\widehat{\pi})$ into invariant subspaces of the operator $\mathcal{L}$.
\begin{theorem}{\cite[Section 5]{Arnold2014}.}
	\label{thm:L2 decomposition}The following holds:
	\begin{enumerate}[label=(\alph*)]
		\item The space $L^{2}(\widehat{\pi})$ has a decomposition into mutually orthogonal
		subspaces:
		\[
		L^{2}(\widehat{\pi})=\bigoplus_{m\in\mathbb{N}_{0}}H_{m}.
		\]
		
		\item For all $m\in\mathbb{N}_{0}$, $H_{m}$ is invariant under $\mathcal{L}$
		as well as under the semigroup $(e^{-t\mathcal{L}})_{t\ge0}$. 
		\item The spectrum of $\mathcal{L}$ has the following decomposition:
		\[
		\sigma(\mathcal{L})=\bigcup_{m\in\mathbb{N}_{0}}\sigma(\mathcal{L}\vert_{H_{m}}),
		\]
		where 
		\begin{equation}
		\sigma(\mathcal{L}\vert_{H_{m}})=\left\lbrace\sum_{j=1}^{2d}\alpha_{j}\lambda_{j}:\,\vert\alpha\vert=m,\,\lambda_{j}\in\sigma(B)\right\rbrace.\label{eq:spectrum on subspaces}
		\end{equation}
		
	\end{enumerate}
\end{theorem}
\begin{remark}
	Note that by the ergodicity of the dynamics, $\ker\mathcal{L}$ consists of constant functions and so $\ker\mathcal{L}=H_{0}$. Therefore, $L^2_0(\widehat{\pi})$ has the decomposition
	\[
	L_{0}^{2}(\widehat{\pi})=L^{2}(\widehat{\pi})/\ker\mathcal{L}=\bigoplus_{m\ge1}H_{m}.
	\]
	\end{remark}
Our first main result of this section is an expression for the asymptotic
variance in terms of the unperturbed operator $\mathcal{L}_{0}$ and
the perturbation $\mathcal{A}$:
\begin{proposition}
	\label{prop:asymvar_op_formula}
	Let $f\in L_{0}^{2}(\pi)$  (so in particular $f=f(q)$).
	Then the associated asymptotic variance is given by 
	\begin{equation}
	\label{eq:asymvar_op_formula}
	\sigma_{f}^{2}=\langle f,-\mathcal{L}_{0}(\mathcal{L}_{0}^{2}+\mu^{2}\mathcal{A}^{*}\mathcal{A})^{-1}f\rangle_{L^{2}(\widehat{\pi})}.
	\end{equation}
	
\end{proposition}
\begin{remark}
The proof of the preceding Proposition will show that $\mathcal{L}_{0}^{2}+\mu^{2}\mathcal{A}^{*}\mathcal{A}$ is invertible on $L^2_0(\widehat{\pi})$ and that $(\mathcal{L}_{0}^{2}+\mu^{2}\mathcal{A}^{*}\mathcal{A})^{-1}f \in \mathcal{D}(\mathcal{L}_0)$ for all $f \in L^2_0(\widehat{\pi})$. 
\end{remark}
To prove Proposition \ref{prop:asymvar_op_formula} we will make use of the \emph{generator
	with reversed perturbation} 
\[
\mathcal{L}_{-}=\mathcal{L}_{0}-\mu\mathcal{A}
\]
and the \emph{momentum flip operator} 
\begin{align*}
P:L_{0}^{2}(\widehat{\pi}) & \rightarrow L_{0}^{2}(\widehat{\pi})\\
\phi(q,p) & \mapsto\phi(q,-p).
\end{align*}
Clearly, $P^{2}=I$ and $P^{*}=P$. Further properties of $\mathcal{L}_{0}$,
$\mathcal{A}$ and the auxiliary operators $\mathcal{L}_{-}$ and
$P$ are gathered in the following lemma:
\begin{lemma}
	\label{operator lemma}
	For all $\phi, \psi \in C^{\infty}(\mathbb{R}^{2d})\cap L^2(\widehat{\pi})$ the following holds:
	\begin{enumerate}[label=(\alph*)]
		\item \label{it:oplem1} The generator $\mathcal{L}_{0}$ is symmetric in $L^2(\widehat{\pi})$ with respect to $P$:
		\[
		\langle  \phi, P\mathcal{L}_{0}P \psi\rangle_{L^2(\widehat{\pi})}=\langle \mathcal{L}_{0} \phi, \psi \rangle_{L^2(\widehat{\pi})}.
		\]
		
		\item \label{it:oplem2} The perturbation $\mathcal{A}$ is skewadjoint in $L^{2}(\widehat{\pi})$:
		\[ 
		\mathcal{A}^{*} = -\mathcal{A}.
		\]
		
		\item \label{it:oplem3} The operators $\mathcal{L}_{0}$ and $\mathcal{A}$ commute:
		\[
		[\mathcal{L}_{0},\mathcal{A}]\phi=0.
		\]
		
		\item \label{it:oplem4} The perturbation $\mathcal{A}$ satisfies
		\[
		P\mathcal{A}P\phi=\mathcal{A}\phi.
		\]
		
		\item \label{it:oplem5} $\mathcal{L}$ and $\mathcal{L}_{-}$ commute,
		\begin{equation*}
		[\mathcal{L},\mathcal{L}_{-}]\phi = 0,
		\end{equation*}
		
		 and the following relation holds:
		\begin{equation}
		\langle \phi ,P\mathcal{L}P\psi\rangle_{L^{2}(\widehat{\pi})}=\langle\mathcal{L}_{-}\phi,\psi\rangle_{ L^{2}(\widehat{\pi})}.\label{eq:L+L-}
		\end{equation}
		\item \label{it:oplem6} 
		The operators $\mathcal{L}$, $\mathcal{L}_0$, $\mathcal{L}_{-}$, $\mathcal{A}$ and $P$ leave the Hermite spaces $H_m$ invariant.
	\end{enumerate}
\end{lemma}
\begin{remark}
	The claim \ref{it:oplem3} in the above lemma is crucial for our approach, which
	itself rests heavily on the fact that the $q-$ and $p-$perturbations
	match ($J_{1}=J_{2}$).
\end{remark}
\begin{proof}[of Lemma \ref{operator lemma}]
	To prove \ref{it:oplem1}, consider the following
	decomposition of $\mathcal{L}_{0}$ as in (\ref{eq:generator}):
	\[
	\mathcal{L}_{0}=\underbrace{p\cdot\nabla_{q}-q\cdot\nabla_{p}}_{\mathcal{L}_{ham}}+\underbrace{\gamma\left(- p\cdot\nabla_{p}+ \Delta_{p}\right)}_{\mathcal{L}_{therm}}.
	\]
	By partial integration it is straightforward to see that 
	\begin{equation*}
	\langle\phi,\mathcal{L}_{ham}\psi\rangle_{ L^{2}(\widehat{\pi})}=-\langle\mathcal{L}_{ham}\phi,\psi\rangle_{ L^{2}(\widehat{\pi})}
	\end{equation*}
	and
	\begin{equation*}
	 \langle \phi,\mathcal{L}_{therm}\psi\rangle_{ L^{2}(\widehat{\pi})}=\langle\mathcal{L}_{therm}\phi,\psi\rangle_{ L^{2}(\widehat{\pi})},
	 \end{equation*}
	 for all $\phi,\psi \in C^{\infty}(\mathbb{R}^{2d})\cap L^2(\widehat{\pi})$,
	  i.e. $\mathcal{L}_{ham}$ and $\mathcal{L}_{therm}$
	are antisymmetric and symmetric in $L^{2}(\widehat{\pi})$ respectively.
	Furthermore, we immediately see that $P\mathcal{L}_{ham}P\phi=-\mathcal{L}_{ham}\phi$ and $P\mathcal{L}_{therm}P\phi = \mathcal{L}_{therm}\phi$, so that
	\[
	\langle \phi,P\mathcal{L}_{0}P\psi\rangle_{ L^{2}(\widehat{\pi})}=\langle\phi,-\mathcal{L}_{ham}\psi+\mathcal{L}_{therm}\psi\rangle_{ L^{2}(\widehat{\pi})}=\langle\mathcal{L}_{0}\phi,\psi\rangle_{ L^{2}(\widehat{\pi})}.
	\]
	We note that this result holds in the more general setting of Section \ref{sec:perturbed_langevin} for the infinitesimal generator \eqref{eq:generator}.  The claim \ref{it:oplem2} follows by noting that the flow vector field $b(q,p)=(-Jq,-Jp)$ associated to $\mathcal{A}$ is divergence-free with respect to $\widehat{\pi}$, i.e. $\nabla \cdot(\widehat{\pi}b)=0$. Therefore, $\mathcal{A}$ is the generator of a strongly continuous unitary semigroup on $L^2(\widehat{\pi})$ and hence skewadjoint by Stone's Theorem.
  To prove \ref{it:oplem3} we use the decomposition $\mathcal{L}_{0}=\mathcal{L}_{ham}+\mathcal{L}_{therm}$ to obtain
	\begin{equation}
	\label{eq:oplemmac_proof}
	[\mathcal{L}_{0},\mathcal{A}]\phi=[\mathcal{L}_{ham},\mathcal{A}]\phi+[\mathcal{L}_{therm},\mathcal{A}]\phi,\quad \phi \in C^\infty(\mathbb{R}^{2d})\cap L^2(\widehat{\pi}).
	\end{equation}
	The first term of \eqref{eq:oplemmac_proof} gives 
	\begin{align*}
	[p\cdot\nabla_{q}-q\cdot\nabla_{p}&,-Jq\cdot\nabla_{q} -Jp\cdot\nabla_{p}]\phi\\
	& =\big([p\cdot\nabla_{q},-Jq\cdot\nabla_{q}]+[p\cdot\nabla_{q},-Jp\cdot\nabla_{p}]+[-q\cdot\nabla_{p},-Jq\cdot\nabla_{q}] \\
	& \qquad +[-q\cdot\nabla_{p},-Jp\cdot\nabla_{p}]\big)\phi\\
	&= Jp\cdot\nabla_{q}\phi-Jp\cdot\nabla_{q}\phi+Jq\cdot\nabla_{p}\phi-Jq\cdot\nabla_{p}\phi=0.
	\end{align*}
	The second term of \eqref{eq:oplemmac_proof} gives 
	\begin{equation}
	\label{eq:term1}
	[-p\cdot\nabla_{p}+\Delta_{p},\mathcal{A}]\phi =[-p\cdot\nabla_{p},-Jp\cdot\nabla_{p}]\phi+[\Delta_{p},-Jp\cdot\nabla_{p}]\phi,
	\end{equation}
	since $Jq\cdot\nabla_{q}$ commutes with $p\cdot\nabla_{p}+\Delta_{p}$. Both  terms in \eqref{eq:term1} are clearly zero due the antisymmetry of $J$ and the symmetry of the Hessian $D^2_p \phi$. 
	\\\\
	The claim \ref{it:oplem4} follows from a short calculation similar to the proof of  \ref{it:oplem1}.  To prove \ref{it:oplem5}, note that the fact that $\mathcal{L}$ and $\mathcal{L}_{-}$ commute follows from \ref{it:oplem3}, as 
	\[
	[\mathcal{L},\mathcal{L}_{-}]\phi=[\mathcal{L}_{0}+\mu\mathcal{A},\mathcal{L}_{0}-\mu\mathcal{A}]\phi=-2\mu[\mathcal{L}_{0},\mathcal{A}]\phi=0,\quad \phi \in C^{\infty}\cap L^2(\widehat{\pi}),
	\]
	while the property $\langle \phi ,P\mathcal{L}_{0}P\psi\rangle_{L^{2}(\widehat{\pi})}=\langle\mathcal{L}_{-}\phi,\psi\rangle_{ L^{2}(\widehat{\pi})}$ follows from properties \ref{it:oplem1}, \ref{it:oplem2} and \ref{it:oplem4}. Indeed,
	\begin{subequations}
	\begin{eqnarray*}
	\langle \phi,P\mathcal{L}P\psi\rangle_{ L^{2}(\widehat{\pi})}& = & \langle \phi, P(\mathcal{L}_{0}+\mu\mathcal{A})P\psi\rangle_{ L^{2}(\widehat{\pi})}=\langle\phi,\left(P\mathcal{L}_{0}P+\mu\mathcal{A}\right)\psi\rangle_{ L^{2}(\widehat{\pi})} \\	
	 & = & \langle (\mathcal{L}_{0}-\mu\mathcal{A})\phi,\psi\rangle_{ L^{2}(\widehat{\pi})}=\langle\mathcal{L}_{-}\phi,\psi\rangle_{ L^{2}(\widehat{\pi})}, 
\end{eqnarray*}
\end{subequations}
	as required. To prove \ref{it:oplem6} first notice that $\mathcal{L}$, $\mathcal{L}_0$ and $\mathcal{L}_{-}$ are of the form \eqref{eq:OU_operator} and therefore leave the spaces $H_m$ invariant by Theorem \ref{thm:L2 decomposition}. It follows immediately that also $\mathcal{A}$ leaves those spaces invariant. The fact that $P$ leaves the spaces $H_m$ invariant follows directly by inspection of \eqref{eq: Hermite polynomials}.
	\qed
\end{proof}
Now we proceed with the proof of Proposition  \ref{prop:asymvar_op_formula}:
\begin{proof}[of Proposition \ref{prop:asymvar_op_formula}] Since the potential $V$ is quadratic, Assumption \ref{ass:bounded+Poincare} clearly holds and thus Lemma \ref{lemma:variance} ensures that $\mathcal{L}$ and $\mathcal{L}_{-}$ are invertible on $L^2_{0}(\widehat{\pi})$ with 
\begin{equation}
\label{eq:Laplace transform}
\mathcal{L}^{-1}=\int_0^\infty e^{-t\mathcal{L}}\mathrm{d}t,
\end{equation}
	and analogously for $\mathcal{L}_{-}^{-1}$.
	 In particular, the asymptotic variance can be written as 
	 \begin{equation*}
	 \sigma_{f}^{2}=\langle f,(-\mathcal{L})^{-1}f\rangle_{L^{2}(\widehat{\pi})}.
	 \end{equation*}
	  Due to the respresentation \eqref{eq:Laplace transform} and Theorem \ref{thm:L2 decomposition}, the inverses of $\mathcal{L}$ and $\mathcal{L}_{-}$ leave the Hermite spaces $H_m$ invariant. We will prove the claim from Proposition \ref{prop:asymvar_op_formula} under the assumption that $Pf=f$ which includes the case 
	$f=f(q)$. For the following calculations we will assume $f\in H_m$ for fixed $m \ge 1$. Combining statement \ref{it:oplem6} with \ref{it:oplem1} and \ref{it:oplem5} of Lemma \ref{operator lemma} (and noting that $H_m \subset C^\infty(\mathbb{R}^{2d})\cap L^2(\widehat{\pi})$) we see that 
	\begin{equation}
	\label{eq:PLPL-}
	P\mathcal{L}P=\mathcal{L}_{-}^{*}
	\end{equation}
	 and 
	 \begin{equation}
	 P\mathcal{L}_{0}P=\mathcal{L}_{0}^{*}
	 \end{equation}
	  when restricted to $H_m$. Therefore, the following calculations are justified:
	\begin{align*}
	\langle f,(-\mathcal{L})^{-1}f\rangle_{L^{2}(\widehat{\pi})} &=\frac{1}{2}\langle f,(-\mathcal{L})^{-1}f\rangle_{L^{2}(\widehat{\pi})}+\langle f,(-\mathcal{L}^{*})^{-1}f\rangle_{L^{2}(\widehat{\pi})}\\
	&=\frac{1}{2}\langle f,(-\mathcal{L})^{-1}f\rangle_{L^{2}(\widehat{\pi})}+\langle Pf,(-\mathcal{L}^{*})^{-1}Pf\rangle_{L^{2}(\widehat{\pi})}\\
	&=\frac{1}{2}\langle f,(-\mathcal{L})^{-1}f\rangle_{L^{2}(\widehat{\pi})}+\langle f,(-\mathcal{L}_{-})^{-1}f\rangle_{L^{2}(\widehat{\pi})}\\
	&=\frac{1}{2}\langle f,\left((-\mathcal{L})^{-1}+(-\mathcal{L}_{-})^{-1}\right)f\rangle_{L^{2}(\widehat{\pi})},
	\end{align*}
	where in the third line we have used the assumption $Pf=f$ and in
	the fourth line the properties $P^{2}=I$, $P^{*}=P$ and equation
	(\ref{eq:PLPL-}).   Since $\mathcal{L}$ and $\mathcal{L}_{-}$ commute on $H_m$ according to Lemma
	\ref{operator lemma}\ref{it:oplem5},\ref{it:oplem6} we can write
	\begin{equation*}
	(-\mathcal{L})^{-1}+(-\mathcal{L}_{-})^{-1}  =\mathcal{L}_{-}(-\mathcal{L}\mathcal{L}_{-})^{-1}+\mathcal{L}(-\mathcal{L}\mathcal{L}_{-})^{-1}
	=-2\mathcal{L}_{0}(\mathcal{L}\mathcal{L}_{-})^{-1}
	\end{equation*}
	for the restrictions on $H_m$, 
	using $\mathcal{L}+\mathcal{L}_{-}=2\mathcal{L}_{0}$. We also have
	\begin{alignat*}{1}
	\mathcal{L}\mathcal{L}_{-} & =(\mathcal{L}_{0}+\mu\mathcal{A})(\mathcal{L}_{0}-\mu\mathcal{A}) =\mathcal{L}_{0}^{2}+\mu^{2}\mathcal{A}^{*}\mathcal{A},
	\end{alignat*}
	since $\mathcal{L}_{0}$ and $\mathcal{A}$ commute. We thus arrive at the formula
	\begin{equation}
	\label{eq:av_formula_Hm}
	\sigma_{f}^{2}=\langle f,-\mathcal{L}_{0}(\mathcal{L}_{0}^{2}+\mu^{2}\mathcal{A}^{*}\mathcal{A})^{-1}f\rangle_{L^{2}(\widehat{\pi})}, \quad f\in H_m.
	\end{equation}
	Now since $(\mathcal{L}_{0}^{2}+\mu^{2}\mathcal{A}^{*}\mathcal{A})^{-1}f = (\mathcal{L}\mathcal{L}_{-})^{-1}f \in \mathcal{D}(\mathcal{L}_{0})$ for all $f\in L^2(\widehat{\pi})$, it follows that the operator $-\mathcal{L}_{0}(\mathcal{L}_{0}^{2}+\mu^{2}\mathcal{A}^{*}\mathcal{A})^{-1}$ is bounded. We can therefore extend formula \eqref{eq:av_formula_Hm} to the whole of $L^2(\widehat{\pi})$ by continuity, using the fact that $L^2_0(\widehat{\pi})=\bigoplus_{m\ge 1}H_m$. 
	\qed
\end{proof}
Applying Proposition \ref{prop:asymvar_op_formula} we can analyse the behaviour
of $\sigma_{f}^{2}$ in the limit of large perturbation strength $\mu\rightarrow\infty$.
To this end, we introduce the orthogonal decomposition
\begin{equation}
\label{eq:kernel decomposition}
L_{0}^{2}(\pi)=\ker (Jq\cdot \nabla_q) \oplus\ker (Jq\cdot \nabla_q)^{\perp},
\end{equation}
where $Jq\cdot\nabla_q$ is understood as an unbounded operator acting on $L_0^2(\pi)$, obtained as the smallest closed extension of $Jq\cdot \nabla_q$ acting on $C^{\infty}_c(\mathbb{R}^d)$. In particular, $\ker (Jq\cdot \nabla_q)$ is a closed linear subspace of $L^2_0(\pi)$.   
Let $\Pi$ denote the $L_{0}^{2}(\pi)$-orthogonal projection onto
$\ker (Jq\cdot \nabla_q)$. We will write $\sigma_{f}^{2}(\mu)$ to
stress the dependence of the asymptotic variance on the perturbation
strength. The following result shows that for large perturbations,
the limiting asymptotic variance is always smaller than the asymptotic
variance in the unperturbed case. Furthermore, the limit is given as
the asymptotic variance of the projected observable $\Pi f$ for the
unperturbed dynamics.
\begin{theorem}
	\label{prop:large pert}
	Let $f\in L_{0}^{2}(\pi)$, then
	\[
	\lim_{\mu\rightarrow\infty}\sigma_{f}^{2}(\mu)=\sigma_{\Pi f}^{2}(0)\le\sigma_{f}^{2}(0).
	\]
\end{theorem}
\begin{remark}
	Note that the fact that the limit exists and is finite is nontrivial.
	In particular, as Figures \ref{fig:no_limit1} and \ref{fig:no_limit2} demonstrate, it is often
	the case that $\lim_{\mu\rightarrow\infty}\sigma_{f}^{2}(\mu)=\infty$
	if the condition $\mu=\nu$ is not satisfied.
\end{remark}
\begin{remark}
	\label{rem:projection}
	The projection $\Pi$ onto $\ker(Jq\cdot\nabla_q)$ can be understood in terms of Figure \ref{fig:good_spectrum}. Indeed, the eigenvalues on the real axis (highlighted by diamonds) are not affected by the perturbations. Let us denote by $\tilde{\Pi}$ the projection onto the span of the eigenspaces of those eigenvalues. As $\mu \rightarrow \infty$, the limiting asymptotic  variance is given as the asymptotic variance associated to the unperturbed dynamics of the projection $\tilde{\Pi}f$. If we denote by $\Pi_0$ the projection of $L^2(\widehat{\pi})$ onto $L^2_0(\pi)$, then we have that $\Pi\Pi_0=\Pi_0\tilde{\Pi}$. 
\end{remark}
\begin{proof}[of Theorem \ref{prop:large pert}]
	Note that $\mathcal{L}_{0}$ and $\mathcal{A}^{*}\mathcal{A}$ leave the Hermite spaces $H_m$ invariant and their restrictions to those spaces commute 
	(see Lemma \ref{operator lemma}, \ref{it:oplem2}, \ref{it:oplem3} and \ref{it:oplem6}). Furthermore, as the Hermite spaces $H_m$ are finite-dimensional, those operators have discrete spectrum. As $\mathcal{A}^{*}\mathcal{A}$
	is nonnegative self-adjoint, there exists an orthogonal
	decomposition $L_{0}^{2}(\pi)=\bigoplus_{i}W_{i}$  into eigenspaces of the operator $-\mathcal{L}_{0}(\mathcal{L}_{0}^{2}+\mu^{2}\mathcal{A}^{*}\mathcal{A})^{-1}$,
	the decomposition $\bigoplus W_i$ being finer then $\bigoplus H_m$ in the sense that every $W_i$ is a subspace of some $H_m$. 
	 Moreover,
	\[
	-\mathcal{L}_{0}(\mathcal{L}_{0}^{2}+\mu^{2}\mathcal{A}^{*}\mathcal{A})^{-1}\vert_{W_{i}}=-\mathcal{L}_{0}(\mathcal{L}_{0}^{2}+\mu^{2}\lambda_{i})^{-1}\vert_{W_i},
	\]
	where $\lambda_{i}\ge0$ is the eigenvalue of $\mathcal{A}^{*}\mathcal{A}$
	associated to the subspace $W_{i}$. Consequently, formula (\ref{eq:asymvar_op_formula})
	can be written as 
	\begin{equation}
	\label{eq:asymvar_spectral}
	\sigma_{f}^{2}=\sum_{i}\langle f_{i},-\mathcal{L}_{0}(\mathcal{L}_{0}^{2}+\mu^{2}\lambda_{i})^{-1}f_{i}\rangle_{L^{2}(\widehat{\pi})},
	\end{equation}
	where $f=\sum_{i}f_{i}$ and $f_{i}\in W_{i}$. Let us assume now
	without loss of generality that $W_{0}=\ker\mathcal{A}^{*}\mathcal{A}$,
	so in particular $\lambda_{0}=0$. Then clearly 
	\[
	\lim_{\mu\rightarrow\infty}\sigma_{f}^{2}=2\langle f_{0},-\mathcal{L}_{0}(\mathcal{L}_{0}^{2})^{-1}f_{0}\rangle_{L^{2}(\widehat{\pi})}=2\langle f_{0},(-\mathcal{L}_{0})^{-1}f_{0}\rangle_{L^{2}(\widehat{\pi})}=\sigma_{f_{0}}^{2}(0).
	\]
	Now note that $W_{0}=\ker\mathcal{A}^{*}\mathcal{A}=\ker\mathcal{A}$ due
	to $\ker\mathcal{A}^{*}=(\im\mathcal{A})^{\perp}$.  It remains to show that  $\sigma_{\Pi f}^{2}(0)\le\sigma_{f}^{2}(0)$.  To see this, we write 
	\begin{align*}
	\sigma_{f}^{2}(0) & =2\langle f,(-\mathcal{L}_{0})^{-1}f\rangle_{L^{2}(\widehat{\pi})}=2\langle\Pi f+(1-\Pi)f,(-\mathcal{L}_{0})^{-1}\big(\Pi f+(1-\Pi)f\big)\rangle_{L^{2}(\widehat{\pi})}\\
	& =\sigma_{\Pi f}^{2}(0)+\sigma_{(1-\Pi)f}^{2}(0)+R,
	\end{align*}
	where 
	\[
	R=2\langle\Pi f,(-\mathcal{L}_{0})^{-1}(1-\Pi)f\rangle_{L^{2}(\widehat{\pi})}+2\langle(1-\Pi)f,(-\mathcal{L}_{0})^{-1}\Pi f\rangle_{L^{2}(\widehat{\pi})}.
	\]
	Note that since we only consider observables that do not depend on $p$, $\Pi f\in \ker (Jq\cdot \nabla_q)$ and $(1-\Pi)f\in\bigoplus_{i\ge1}W_{i}$.
	Since $\mathcal{L}_{0}$ commutes with $\mathcal{A}$, it follows
	that $(-\mathcal{L}_{0})^{-1}$ leaves both $W_{0}$ and $\bigoplus_{i\ge1}W_{i}$
	invariant. Therefore, as the latter spaces are orthogonal to each
	other, it follows that $R=0$, from which the result follows. 
	\qed
\end{proof}
From Theorem \ref{prop:large pert} it follows that in the limit as $\mu \rightarrow \infty$, the asymptotic variance $\sigma_f^2(\mu)$ is not decreased by the perturbation if $f \in \ker(Jq \cdot \nabla_q)$. In fact, this result also holds true non-asymptotically, i.e. observables in $\ker(Jq \cdot \nabla_q)$ are not affected at all by the perturbation:
\begin{lemma}
	\label{lem:invariant observables}
	Let $f\in \ker (Jq\cdot \nabla_q)$. Then
	\begin{equation*}
	\sigma^2_f(\mu) = \sigma^2_f(0)
	\end{equation*}
	for all $\mu \in \mathbb{R}$.
\end{lemma}
\begin{proof}
	From $f \in  \ker (Jq\cdot \nabla_q)$ it follows immediately that $f \in \ker \mathcal{A}^{*}\mathcal{A}$. Then the claim follows from the expression \eqref{eq:asymvar_spectral}.
	\qed
\end{proof}
\begin{example}
	\label{ex:commutation quadratic observables}
	Recall the case of observables of the form $f(q)=q\cdot Kq+l\cdot q+C$
	with $K\in\mathbb{R}_{sym}^{d\times d}$, $l\in\mathbb{R}^{d}$ and
	$C\in\mathbb{R}$ from Section \ref{sec:small perturbations}. If $[J,K]=0$
	and $l\in\ker J$, then $f\in\ker (Jq\cdot \nabla_q)$ as 
	\[
	Jq\cdot\nabla_{q}(q\cdot Kq+l\cdot q+C)=2Jq\cdot Kq+Jq\cdot l=q\cdot(KJ-JK)q-q\cdot Jl=0.
	\]
	From the preceding lemma it follows that $\sigma_{f}^{2}(\mu)=\sigma_{f}^{2}(0)$
	for all $\mu\in\mathbb{R},$ showing that the assumption in Theorem
	\ref{cor:small pert unit var} does not exclude nontrivial cases.
\end{example}
The following result shows that the dynamics (\ref{eq: unit covariance perfect perturbation})
is particularly effective for antisymmetric observables (at least
in the limit of large perturbations):
\begin{proposition}
	\label{prop:antisymmetric observables}Let $f\in L_{0}^{2}(\pi)$
	satisfy $f(-q)=-f(q)$ and assume that $\ker J=\{0\}$.
	Furthermore, assume that the eigenvalues of $J$ are rationally independent, i.e. 
	\begin{equation}
	\label{eq:rat_indp_spectrum}
	\sigma(J)=\{\pm i\lambda_{1},\pm i\lambda_{2},\ldots,\pm i\lambda_d\}
	\end{equation}
	with $\lambda_{i}\in\mathbb{R}_{>0}$ and  $\sum_i k_i \lambda_i \neq 0$ for all $(k_1,\ldots,k_d)\in\mathbb{Z}^d\setminus(0,\ldots,0)$. Then $\lim_{\mu\rightarrow\infty}\sigma_{f}^{2}(\mu)=0$.
\end{proposition} 
\begin{proof}
	[of Proposition \ref{prop:antisymmetric observables}]
	The claim would
	immediately follow from $f\in\ker(Jq\cdot\nabla)^{\perp}$ according to Theorem \ref{prop:large pert}, but that does not seem to be so easy to prove directly. Instead, we again make
	use of the Hermite polynomials.
	
	Recall from the proof of Proposition \ref{prop:asymvar_op_formula} that $\gen$ is invertible on $L_{0}^{2}(\widehat{\pi})$ and its inverse leaves the Hermite spaces $H_m$ invariant. Consequently, the
	asymptotic variance of an observable $f\in L_{0}^{2}(\widehat{\pi})$ can be written as  
	\begin{subequations}
	\begin{eqnarray}
	\sigma_{f}^{2} & = & \langle f,(-\mathcal{L})^{-1}f\rangle_{L^{2}(\widehat{\pi})} \\
	& = & \sum_{m=1}^{\infty}\langle\Pi_{m}f,(-\mathcal{L}\vert_{H_{m}})^{-1}\Pi_{m}f\rangle_{L^2(\widehat{\pi})},\label{eq:asymvar decomposition} 
	\end{eqnarray}
	\end{subequations}
	where $\Pi_{m}:L_{0}^{2}(\widehat{\pi})\rightarrow H_{m}$ denotes the orthogonal
	projection onto $H_{m}$. From (\ref{eq: Hermite polynomials}) it
	is clear that $g_{a}$ is symmetric for $\vert\alpha\vert$ even and
	antisymmetric for $\vert\alpha\vert$ odd. Therefore, from $f$ being
	antisymmetric it follows that 
	\[
	f\in\bigoplus_{m\ge1,m\,\text{odd}}H_{m}.
	\]
	In view of (\ref{eq:spectrum of B}), (\ref{eq:spectrum on subspaces}) and (\ref{eq:rat_indp_spectrum})
	the spectrum of $\mathcal{L}_{\vert H_{m}}$ can be written as 
	\begin{subequations}
	\begin{eqnarray}
	\sigma(\mathcal{L}\vert_{H_{m}}) & = &\left\lbrace \mu\sum_{j=1}^{2d}\alpha_{j}\beta_{j}+C_{\alpha,\gamma}:\,\vert\alpha\vert=m,\,\beta_{j}\in\sigma(J)\right\rbrace  \nonumber \\
	& = & \left\lbrace i\mu\sum_{j=1}^{d}(\alpha_{j}-\alpha_{j+d})\lambda_{j}+C_{\alpha,\gamma}:\,\vert\alpha\vert=m\right\rbrace  \label{eq:spec_grow}
	\end{eqnarray}
	\end{subequations}
	with appropriate real constants $C_{\alpha,\gamma}\in\mathbb{R}$ that depend
	on $\alpha$ and $\gamma$, but not on $\mu$. For $\vert\alpha\vert=\sum_{j=1}^{2d} \alpha_j=m$ odd, we have that
	\begin{equation}
	\label{eq:nonzero}
	\sum_{j=1}^{d}(\alpha_{j}-\alpha_{j+d})\lambda_{j} \neq 0.
	\end{equation}
	Indeed, assume to the contrary that the above expression is zero. Then it follows that $\alpha_j = \alpha_{j+d}$ for all $j=1,\ldots,d$ by rational independence of $\lambda_1,\ldots,\lambda_d$.
	From \eqref{eq:spec_grow} and \eqref{eq:nonzero} it is clear that
	\begin{equation*}
	\sup \left\lbrace r>0 : B(0,r) \cap \sigma(\mathcal{L}\vert_{H_m}) = \emptyset \right\rbrace \xrightarrow{\mu \rightarrow \infty} \infty,
	\end{equation*}
	where $B(0,r)$ denotes the ball of radius $r$ centered at the origin in $\mathbb{C}$.
	Consequently, the spectral radius of $(-\mathcal{L}\vert_{H_m})^{-1}$ and hence $(-\mathcal{L}\vert_{H_m})^{-1}$ itself converge to zero as $\mu \rightarrow \infty$. The result then follows from (\ref{eq:asymvar decomposition}). \qed\end{proof}
\begin{remark}
	The idea of the preceding proof can be explained using Figure \ref{fig:good_spectrum} and Remark \ref{rem:projection}. Since the real eigenvalues correspond to Hermite polynomials of even order, antisymmetric observables are orthogonal to the associated subspaces. The rational independence condition on the eigenvalues of $J$ prevents cancellations  that would lead to further eigenvalues on the real axis.
\end{remark}
The following corollary gives a version of the converse of Proposition \ref{prop:antisymmetric observables} and provides further intuition into the mechanics of the variance reduction achieved by the perturbation.
\begin{corollary}
	Let $f\in L_{0}^{2}(\pi)$ and assume that $lim_{\mu\rightarrow\infty}\sigma_{f}^{2}(\mu)=0$. Then 
	\[
	\int_{B(0,r)}f\mathrm{dq=0}
	\]
	for all $r\in(0,\infty)$, where $B(0,r)$ denotes the ball centered at $0$ with radius $r$.
\end{corollary}
\begin{proof}
	According to Theorem \ref{prop:large pert},  $\lim_{\mu\rightarrow\infty}\sigma_{f}^{2}(\mu)=0$  implies $\sigma_{\Pi f}^{2}(0)=0$. We can write 
	\begin{subequations}
	\begin{eqnarray}
	\sigma_{\Pi f}^{2}(0) & = & \langle \Pi f, (-\mathcal{L}_0)^{-1}\Pi f \rangle_{ L^{2}(\widehat{\pi})} \nonumber \\
	& = & \frac{1}{2}\langle \Pi f, \left((-\mathcal{L}_0)^{-1}+(-\mathcal{L}^{*}_0)^{-1}\right)\Pi f \rangle_{ L^{2}(\widehat{\pi})} \nonumber
	\end{eqnarray}
	\end{subequations}
	and recall from the proof of Proposition \ref{prop:asymvar_op_formula} that $(-\mathcal{L}_0)^{-1}$ and $(-\mathcal{L}^{*}_0)^{-1}$ leave the Hermite spaces $H_m$ invariant. Therefore  
	\begin{equation}
	\ker \left((-\mathcal{L}_0)^{-1}+(-\mathcal{L}^{*}_0)^{-1}\right) = {0}
	\end{equation}
	in $L^2_0(\widehat{\pi})$, and in particular $\sigma_{\Pi f}^{2}(0)=0$ implies $\Pi f = 0$, which in turn shows that
	  $f\in\ker(Jq\cdot\nabla)^{\perp}$. Using $\ker(Jq\cdot\nabla)^{\perp}=\overline{\im(Jq\cdot\nabla)}$,
	it follows that there exists a sequence $(\phi_n)_n\in C_c^{\infty}(\mathbb{R}^d)$ such that $Jq\cdot\nabla\phi_n \rightarrow f$ in $L^2(\pi)$. Taking a subsequence if necessary, we can assume that the convergence is pointwise $\pi$-almost everywhere and that the sequence is pointwise bounded by a function in $L^1(\pi)$. 
	Since $J$ is antisymmetric, we have that $Jq\cdot\nabla\phi_n=\nabla\cdot(\phi_n Jq)$.
	Now Gauss's theorem yields
	\[
	\int_{B(0,r)}f\mathrm{d}q=\int_{B(0,r)}\nabla\cdot(\phi Jq)\mathrm{d}q=\int_{\partial B(0,r)}\phi Jq\cdot\mathrm{d}n,
	\]
	where $n$ denotes
	the outward normal to the sphere $\partial B(0,r)$. This quantity
	is zero due to the orthogonality of $Jq$ and $n$, and so the result
	follows from Lebesgue's dominated convergence theorem.\qed
\end{proof}
\subsection{Optimal Choices of $J$ for Quadratic Observables}

Assume $f\in L_{0}^{2}(\pi)$ is given by $f(q)=q\cdot Kq+l\cdot q -\Tr K$,  with $K\in\mathbb{R}_{sym}^{d\times d}$ and $l\in\mathbb{R}^{d}$ (note that the constant term is chosen such that $ \pi(f)=0 $). Our objective is to choose $J$ in such a way that $\lim_{\mu\rightarrow\infty}\sigma_{f}^{2}(\mu)$ becomes as small as possible. To stress the dependence on the choice
of $J$, we introduce the notation $\sigma_{f}^{2}(\mu,J)$. Also, we denote the orthogonal projection onto $(\ker J)^{\perp}$ by $\Pi^{\perp}_{\ker J}$.
\begin{lemma}
	\label{lem:lin_observables}{(Zero variance limit for linear observables).} Assume $K=0$ and $\Pi^{\perp}_{\ker J}l=0$. Then 
	\[
	\lim_{\mu\rightarrow\infty}\sigma_{f}^{2}(\mu,J)=0.
	\]
\end{lemma}
\begin{proof}
	According to Proposition \ref{prop:large pert}, we have to show that
	$\Pi f=0$, where $\Pi$ is the $L^{2}(\pi)$-orthogonal projection
	onto $\ker(Jq\cdot\nabla)$. Let us thus prove that 
	\[
	f\in\ker(Jq\cdot\nabla)^{\perp}=\overline{\im(Jq\cdot\nabla)^{*}}=\overline{\im(Jq\cdot\nabla)},
	\]
	where the second identity uses the fact that $(Jq\cdot\nabla)^{*}=-Jq\cdot\nabla$.
	Indeed, since $\Pi^{\perp}_{\ker J}=0$, by Fredholm's alternative there exists $u \in \mathbb{R}^d$ such that $Ju=l$. Now define $\phi\in L_{0}^{2}(\pi)$
	by $\phi(q)=-u\cdot q,$ leading to 
	\[
	f=Jq\cdot\nabla\phi,
	\]
	so the result follows.\qed
\end{proof}

\begin{lemma}
	\label{lem:optimal_perturbation}{(Zero variance limit for purely quadratic observables.)} Let $l=0$ and consider the decomposition $K=K_{0}+K_{1}$ into the traceless part $K_{0}=K-\frac{\Tr K}{d}\cdot I$ and the
	trace-part $K_{1}=\frac{\Tr K}{d}\cdot I.$ For the corresponding
	decomposition of the observable 
	\[
	f(q)=f_{0}(q)+f_{1}(q)=q\cdot K_{0}q+q\cdot K_{1}q-\Tr K
	\]
	the following holds:
	\begin{enumerate}[label=(\alph*)]
		\item There exists an antisymmetric matrix $J$ such that  $\lim_{\mu\rightarrow\infty}\sigma_{f_{0}}^{2}(\mu,J)=0,$
		and there is an algorithmic way (see Algorithm \ref{alg:optimal J}) to compute an appropriate $J$ in terms
		of $K$.
		\item The trace-part is not effected by the perturbation, i.e. $\sigma_{f_{1}}^{2}(\mu,J)=\sigma_{f_{1}}^{2}(0)$ for all $\mu\in\mathbb{R}$.
	\end{enumerate}
\end{lemma}
\begin{proof}
	To prove the first claim, according to Theorem \ref{prop:large pert}
	it is sufficient to show that $f_{0}\in\ker(Jq\cdot\nabla)^{\perp}=\overline{\im(Jq\cdot\nabla)}$.
	Let us consider the function $\phi(q)=q\cdot Aq$, with $A\in\mathbb{R}_{sym}^{d\times d}$. It holds that
	\begin{equation*}
	Jq\cdot\nabla\phi=q\cdot(J^{T}Aq)=q\cdot[A,J]q.
	\end{equation*}
	The task of finding an antisymmetric matrix $J$ such that 
	\begin{equation}
	\label{eq:quad_var_reduction}
	\lim_{\mu\rightarrow\infty}\sigma_{f_{0}}^{2}(\mu,J)=0
	\end{equation}
	can therefore be accomplished by constructing an antisymmetric matrix
	$J$ such that there exists a symmetric matrix $A$ with the property
	$K_{0}=[A,J]$.  Given any traceless matrix $K_{0}$ there exists
	an orthogonal matrix $U\in O(\mathbb{R}^{d})$ such that $UK_{0}U^{T}$
	has zero entries on the diagonal, and that $U$ can be obtained in
	an algorithmic manner (see for example \cite{alg_zero_diag} or \cite[Chapter 2, Section 2, Problem 3]{Horn_Johnson_Matrix_Analysis}; for the reader's convenience we have summarised the algorithm in Appendix \ref{tracefree}.) Assume
	thus that such a matrix $U\in O(\mathbb{R}^{d})$ has been found and choose real numbers $a_1,\ldots,a_d \in \mathbb{R}$ such that $a_{i}\neq a_{j}$ if $i\neq j$.
	We now set
	\begin{equation}
	\bar{A}=\diag(a_{1},\ldots,a_{n}),
	\end{equation}
	and 
	\begin{equation}
	\bar{J}_{ij}= 
	\begin{cases}
	\frac{(UK_{0}U^{T})_{ij}}{a_{i}-a_{j}} & \text{if } i\neq j, \\
	0 & \text{if } i=j. \\ 
	\end{cases}
	\end{equation}
	Observe that since $UK_{0}U^{T}$ is symmetric, $\bar{J}$ is antisymmetric. 
	A short calculation shows that $[\bar{A},\bar{J}]= UK_{0}U^{T}$.
	We can thus define $A=U^{T}\bar{A}U$ and $J=U^{T}\bar{J}U$ to obtain $[A,J]=K_0$. Therefore, the $J$ constructed in this way indeed satisfies \eqref{eq:quad_var_reduction}.  For the second claim, note that $f_{1}\in\ker(Jq\cdot\nabla)$, since
	\begin{equation}
	\label{eq:constant_trace}
	Jq\cdot\nabla\left(q\cdot\frac{\Tr K}{d}q\right)=2\frac{\Tr K}{d}q\cdot Jq=0
	\end{equation}
	because of the antisymmetry of $J$. The result then follows from
	Lemma \ref{lem:invariant observables}.\qed
\end{proof}
We would like to stress that the perturbation $J$ constructed in the previous lemma is far from unique due to the freedom of choice of $U$ and $a_1,\ldots,a_d \in \mathbb{R}$ in its proof. However, it is asymptotically optimal:
\begin{corollary}
	\label{cor:optimality}
	In the setting of Lemma \ref{lem:optimal_perturbation} the following holds:
	\[
	\min_{J^T=-J}\left(\lim_{\mu\rightarrow\infty} \sigma^2_{f}(\mu,J)\right)=\sigma^2_{f_1}(0).
	\]
\end{corollary}
\begin{proof}
	The claim follows immediately since $f_{1}\in\ker(Jq\cdot\nabla)$ for arbitrary antisymmetric $J$ as shown in \eqref{eq:constant_trace}, and therefore the contribution of the trace part $f_1$ to the asymptotic variance cannot be reduced by any choice of $J$ according to Lemma \ref{lem:invariant observables}.	
\end{proof}
As the proof of Lemma \ref{lem:optimal_perturbation} is constructive, we obtain the following algorithm for determining optimal perturbations for quadratic observables:
\begin{algorithm}
	\label{alg:optimal J}
	Given $K\in\mathbb{R}_{sym}^{d\times d}$,
	determine an optimal antisymmetric perturbation $J$ as follows:
	\begin{enumerate}
		\item Set $K_{0}=K-\frac{\Tr K}{d}\cdot I.$
		\item Find $U\in O(\mathbb{R}^{d})$ such that $UK_{0}U^{T}$ has zero entries
		on the diagonal (see Appendix \ref{tracefree}).
		\item Choose $a_{i}\in\mathbb{R},$ $i=1,\ldots d$ such that $a_{i}\neq a_{j}$
		for $i\neq j$ and set 
		\[
		\bar{J}_{ij}=\frac{(UK_{0}U^{T})_{ij}}{a_{i}-a_{j}}
		\]
		for $i\ne j$ and $\bar{J}_{ii}=0$ otherwise.
		\item
		Set $J=U^{T}\bar{J}U$.
	\end{enumerate}
\end{algorithm}
\begin{remark}
	In \cite{duncan2016variance}, the authors consider the task of finding optimal perturbations $J$ for the nonreversible overdamped Langevin dynamics given in \eqref{eq:nonrev_overdamped_J}. In the Gaussian case this optimization problem turns out be equivalent to the one considered in this section. Indeed, equation (39) of \cite{duncan2016variance} can be rephrased as 
	\begin{equation*}
	f \in \ker(Jq\cdot \nabla)^{\perp}.
	\end{equation*}
	Therefore, Algorithm \ref{alg:optimal J} and its generalization Algorithm \ref{alg:optimal J general} (described in Section~\ref{sec:arbitrary covariance})  can be used without modifications to find optimal perturbations of overdamped Langevin dynamics.
\end{remark}

\subsection{Gaussians with Arbitrary Covariance and Preconditioning}
\label{sec:arbitrary covariance}

In this section we  extend the results of the preceding sections to the case
when the target measure $\pi$ is given by a Gaussian with arbitrary
covariance, i.e. $V(q)=\frac{1}{2}q\cdot Sq$ with $S\in\mathbb{R}_{sym}^{d\times d}$ symmetric and positive definite. The
dynamics (\ref{eq:perturbed_underdamped}) then takes the
form 

\begin{align}
\mathrm{d}q_{t} & =M^{-1}p_{t}\mathrm{d}t-\mu J_{1}Sq_{t}\mathrm{d}t\nonumber, \\
\mathrm{d}p_{t} & =-Sq_{t}\mathrm{d}t-\nu J_{2}M^{-1}p_{t}\mathrm{d}t-\Gamma M^{-1}p_{t}\mathrm{d}t+\sqrt{2\Gamma}\mathrm{d}W_{t}.\label{eq:Underdamped Langevin Gaussian}
\end{align}
The key observation is now that the choices $M=S$ and $\Gamma=\gamma S$
together with the transformation $\widetilde{q}=S^{1/2}q$ and $\widetilde{p}=S^{-1/2}p$
lead to the dynamics
\begin{align}
\mathrm{d}\widetilde{q}_{t} & =\widetilde{p}_{t}\mathrm{d}t-\mu S^{1/2}J_{1}S^{1/2}\widetilde{q}_{t}\mathrm{d}t,\nonumber \\
\mathrm{d}\widetilde{p}_{t} & =-\widetilde{q}_{t}\mathrm{d}t-\mu S^{-1/2}J_{2}S^{-1/2}\widetilde{p}_{t}\mathrm{d}t-\gamma\widetilde{p}_{t}\mathrm{d}t+\sqrt{2\gamma}\mathrm{d}W_{t},\label{eq:Underdamped Langevin transformed}
\end{align}
which is of the form (\ref{eq:unit covariance}) if $J_{1}$ and
$J_{2}$ obey the condition $SJ_{1}S=J_{2}$ (note that both $S^{1/2}J_{1}S^{1/2}$
and $S^{-1/2}J_{2}S^{-1/2}$ are of course antisymmetric). Clearly
the dynamics (\ref{eq:Underdamped Langevin transformed}) is ergodic
with respect to a Gaussian measure with unit covariance, in the following
denoted by $\widetilde{\pi}$. The connection between the asymptotic variances
associated to (\ref{eq:Underdamped Langevin Gaussian}) and (\ref{eq:Underdamped Langevin transformed})
is as follows: 
\\\\
For an observable $f\in L_{0}^{2}(\pi)$ we can write 
\[
\sqrt{T}\bigg(\frac{1}{T}\int_{0}^{T}f(q_{s})\mathrm{d}s-\pi(f)\bigg)=\sqrt{T}\bigg(\frac{1}{T}\int_{0}^{T}\widetilde{f}(\widetilde{q}_{s})\mathrm{d}s-\widetilde{\pi}(\widetilde{f})\bigg),
\]
where $\widetilde{f}(q)=f(S^{-1/2}q)$. Therefore, the asymptotic variances
satisfy
\begin{equation}
\sigma_{f}^{2}=\widetilde{\sigma}_{\widetilde{f}}^{2},\label{eq:asymvar transform}
\end{equation}
where $\widetilde{\sigma}_{\widetilde{f}}^{2}$ denotes the asymptotic variance
of the process $(\widetilde{q}_{t})_{t\ge0}$. Because of this, the results
from the previous sections generalise to (\ref{eq:Underdamped Langevin Gaussian}),
subject to the condition that the choices $M=S$, $\Gamma=\gamma S$
and $SJ_{1}S=J_{2}$ are made. We formulate our results in this general
setting as corollaries:
\begin{corollary}
	\label{cor:small_pert_general}
	Consider the dynamics 
	\begin{align}
	\mathrm{d}q_{t} & =M^{-1}p_{t}\mathrm{d}t-\mu J_{1}\nabla V(q_{t})\mathrm{d}t,\nonumber \\
	\mathrm{d}p_{t} & =-\nabla V(q_{t})\mathrm{d}t-\mu J_{2}M^{-1}p_{t}\mathrm{d}t-\Gamma M^{-1}p_{t}\mathrm{d}t+\sqrt{2\Gamma}\mathrm{d}W_{t},\label{eq: perturbed Langevin corollary}
	\end{align}
	with $V(q)=\frac{1}{2}q\cdot Sq$. Assume that $M=S$, $\Gamma=\gamma S$
	with $\gamma > \sqrt{2}$ and $SJ_{1}S=J_{2}$. Let $f\in L^{2}(\pi)$ be an observable of the form 
	\begin{equation}
	f(q)=q\cdot Kq+l\cdot q+C\label{eq:quadratic observable}
	\end{equation}
	with $K\in\mathbb{R}_{sym}^{d\times d}$, $l\in\mathbb{R}^{d}$ and
	$C\in\mathbb{R}$. If at least one of the conditions $KJ_{1}S\neq SJ_{1}K$
	and $l \notin \ker J$ is satisfied, then the asymptotic variance is at a local maximum for the unperturbed sampler, i.e.
	\[
	\left. \partial_{\mu}\sigma_{f}^{2}\right\rvert_{\mu=0}=0\qquad \mbox{ and } \qquad 	\left. \partial_{\mu}^{2}\sigma_{f}^{2}\right\rvert_{\mu=0}<0.
	\]
\end{corollary}
\begin{proof}
	Note that 
	\[
	\widetilde{f}(q)=f(S^{-1/2}q)=q\cdot S^{-1/2}KS^{-1/2}q+S^{-1/2}l\cdot q+C=q\cdot\widetilde{K}q+\widetilde{l}\cdot q+C
	\]
	is again of the form (\ref{eq:quadratic observable}) (where in the
	last equality, $\widetilde{K}=S^{-1/2}KS^{-1/2}$ and $\widetilde{l}=S^{-1/2}l$
	have been defined). From (\ref{eq:Underdamped Langevin transformed}),
	(\ref{eq:asymvar transform}) and Theorem \ref{cor:small pert unit var}
	the claim follows if at least one of the conditions $[\widetilde{K},S^{1/2}J_{1}S^{1/2}]\neq0$
	and $\widetilde{l}\notin\ker(S^{1/2}J_{1}S^{1/2})$ is satisfied. The
	first of those can easily seen to be equivalent to 
	\[
	S^{-1/2}(KJS-SJK)S^{-1/2}\neq0,
	\]
	which is equivalent to $KJ_{1}S\neq SJ_{1}K$ since $S$ is nondegenerate.
	The second condition is equivalent to 
	\[
	S^{1/2}J_{1}l\neq0,
	\]
	which is equivalent to $J_{1}l\neq0,$ again by nondegeneracy of $S$. \qed \end{proof}
\begin{corollary}
	\label{cor:limit_asym_var}
	Assume the setting from the previous corollary and denote by $\Pi$
	the orthogonal projection onto $\ker(J_{1}Sq\cdot\nabla)$. For $f\in L^{2}(\pi)$
	it holds that
	\[
	\lim_{\mu\rightarrow\infty}\sigma_{f}^{2}(\mu)=\sigma_{\Pi f}^{2}(0)\le\sigma_{f}^{2}(0).
	\]
\end{corollary}
\begin{proof}
	Theorem \ref{prop:large pert} implies 
	\[
	\lim_{\mu\rightarrow\infty}\widetilde{\sigma}_{\widetilde{f}}^{2}(\mu)=\widetilde{\sigma}_{\widetilde{\Pi}\widetilde{f}}^{2}(0)\le\widetilde{\sigma}_{\widetilde{f}}^{2}(0)
	\]
	for the transformed system (\ref{eq:Underdamped Langevin transformed}).
	Here $\widetilde{f}(q)=f(S^{-1/2}q)$ is the transformed observable and
	$\widetilde{\Pi}$ denotes $L^{2}(\pi)$-orthogonal projection onto $\ker(S^{1/2}J_{1}S^{1/2}q\cdot\nabla)$.
	According to (\ref{eq:asymvar transform}), it is sufficient to show
	that $(\Pi f)\circ S^{-1/2}=\widetilde{\Pi}\widetilde{f}$. This however follows
	directly from the fact that the linear transformation $\phi\mapsto\phi\circ S^{1/2}$
	maps $\ker(S^{1/2}J_{1}S^{1/2}q\cdot\nabla)$ bijectively onto $\ker(J_{1}Sq\cdot\nabla)$.\qed
\end{proof}
Let us also reformulate Algorithm \ref{alg:optimal J} for the case of a Gaussian with arbitrary covariance.
\begin{algorithm}
	\label{alg:optimal J general}Given $K,S\in\mathbb{R}_{sym}^{d\times d}$
	with $f(q)=q\cdot Kq$ and $V(q)=\frac{1}{2}q\cdot Sq$ (assuming $S$ is nondegenerate), determine optimal perturbations $J_{1}$
	and $J_{2}$ as follows:
	\begin{enumerate}
		\item Set $\widetilde{K}=S^{-1/2}KS^{-1/2}$ and $\widetilde{K}_{0}=\widetilde{K}-\frac{\Tr\widetilde{K}}{d}\cdot I$.
		\item Find $U\in O(\mathbb{R}^{d})$ such that $U\widetilde{K}_{0}U^{T}$ has
		zero entries on the diagonal (see Appendix \ref{tracefree}).
		\item Choose $a_{i}\in\mathbb{R}$, $i=1,\ldots,d$ such that $a_{i}\ne a_{j}$
		for $i\ne j$ and set 
		\[
		\bar{J}_{ij}=\frac{(U\widetilde{K}_{0}U^{T})_{ij}}{a_{i}-a_{j}}.
		\]
		\item
		Set $\widetilde{J}=U^{T}\bar{J}U$.
		\item Put $J_{1}=S^{-1/2}\widetilde{J}S^{-1/2}$ and $J_{2}=S^{1/2}JS^{1/2}$.
	\end{enumerate}
\end{algorithm}
Finally, we obtain the following optimality result from Lemma \ref{lem:lin_observables} and Corollary \ref{cor:optimality}.
\begin{corollary}
	Let $f(q)=q\cdot Kq+l\cdot q-\Tr K$ and assume that $\Pi^{\perp}_{\ker J}l=0$.
	Then 
	\[
	\min_{J_1^T=-J_1,\, J_2=SJ_1 S}\left(\lim_{\mu\rightarrow\infty} \sigma^2_{f}(\mu,J_1,J_2)\right)=\sigma^2_{f_1}(0),
	\]
	where $f_{1}(q)=q\cdot K_{1}q$, $K_{1}=\frac{\Tr(S^{-1}K)}{d}S$.
	Optimal choices for $J_{1}$ and $J_{2}$ can be obtained using Algorithm \ref{alg:optimal J general}.
\end{corollary}
\begin{remark}
	Since in Section \ref{sec:small perturbations} we analysed the case
	where $J_{1}$ and $J_{2}$ are proportional, we are not able to drop
	the restriction $J_{2}=SJ_{1}S$ from the above optimality
	result. Analysis of completely arbitrary perturbations will be the
	subject of future work. 
\end{remark}

\begin{remark}
	The choices $M=S$ and $\Gamma=\gamma S$ have been introduced to
	make the perturbations considered in this article lead to samplers that perform well in terms of reducing the asymptotic variance. However, adjusting
	the mass and friction matrices according to the target covariance
	in this way (i.e. $M=S$ and $\Gamma=\gamma S$) is a popular way of preconditioning the dynamics, see for instance \cite{GirolamiCalderhead2011} and, in particular mass-tensor molecular dynamics~\cite{Bennett1975267}. Here we will present an argument why such a preconditioning
	is indeed beneficial in terms of the convergence rate of the dynamics.
	Let us first assume that $S$ is diagonal, i.e. $S=\diag(s^{(1)},\ldots,s^{(d)})$
	and that $M=\diag(m^{(d)},\ldots,m^{(d)})$ and $\Gamma=\diag(\gamma^{(d)},\ldots,\gamma^{(d)})$
	are chosen diagonally as well. Then (\ref{eq:Underdamped Langevin Gaussian})
	decouples into one-dimensional SDEs of the following form: 
	\begin{align}
	\mathrm{d}q_{t}^{(i)} & =\frac{1}{m^{(i)}}p_{t}^{(i)}\mathrm{d}t,\nonumber \\
	\mathrm{d}p_{t}^{(i)} & =-s^{(i)}q_{t}^{(i)}\mathrm{d}t-\frac{\gamma^{(i)}}{m^{(i)}}p_{t}^{(i)}\mathrm{d}t+\sqrt{2\gamma^{(i)}}\mathrm{d}W_{t},\quad i=1,\ldots,d.\label{eq:decoupled Langevin}
	\end{align}
	Let us write those Ornstein-Uhlenbeck processes as 
	\begin{equation}
	\mathrm{d}X_{t}^{(i)}=-B^{(i)}X_{t}^{(i)}\mathrm{d}t+\sqrt{2Q^{(i)}}\mathrm{d}W_{t}^{(i)}\label{eq: decoupled OU process}
	\end{equation}
	with 
	\[
	B^{(i)}=\left(\begin{array}{cc}
	0 & -\frac{1}{m^{(i)}}\\
	s^{(i)} & \frac{\gamma^{(i)}}{m^{(i)}}
	\end{array}\right)\,\text{and }\,Q^{(i)}=\left(\begin{array}{cc}
	0 & 0\\
	0 & \gamma^{(i)}
	\end{array}\right).
	\]
	As in Section \ref{sec:exp_decay}, the rate of the exponential decay of (\ref{eq: decoupled OU process}) is equal to $\min\text{Re}\,\sigma(B^{(i)})$. A short calculation shows that the eigenvalues of $B^{(i)}$ are given by  
	\[
	\lambda_{1,2}^{(i)}=\frac{\gamma^{(i)}}{2m^{(i)}}\pm\sqrt{\bigg(\frac{\gamma^{(i)}}{2m^{(i)}}\bigg)^{2}-\frac{s^{(i)}}{m^{(i)}}}.
	\]
	Therefore, the rate of exponential decay is maximal when 
	\begin{equation}
	\bigg(\frac{\gamma^{(i)}}{2m^{(i)}}\bigg)^{2}-\frac{s^{(i)}}{m^{(i)}}=0,\label{eq:gm constraint}
	\end{equation}
	in which case it is given by 
	\[
	(\lambda^{(i)})^{*}=\sqrt{\frac{s^{(i)}}{m^{(i)}}}.
	\]
	Naturally, it is reasonable to choose $m^{(i)}$ in such a way that
	the exponential rate $(\lambda^{(i)})^{*}$ is the same for all $i$, leading
	to the restriction $M=cS$ with $c>0$. Choosing $c$ small will result in fast convergence to equilibrium,
	but also make the dynamics (\ref{eq:decoupled Langevin}) quite stiff,
	requiring a very small timestep $\Delta t$ in a discretisation scheme.
	The choice of $c$ will therefore need to strike a balance between
	those two competing effects. The constraint (\ref{eq:gm constraint})
	then implies $\Gamma=2cS$.	 By a coordinate transformation, the preceding argument also applies if $S$, $M$ and $\Gamma$ are diagonal in the same basis, and of course $M$ and $\Gamma$ can always be chosen that way.
	Numerical experiments show that it is possible to increase the rate of convergence to equilibrium even further by choosing $M$ and $\Gamma$ nondiagonally with respect to $S$ (although
	only by a small margin). A clearer understanding of this is a topic of further investigation.
\end{remark}


%\section{Numerical implementation}
%\label{sec:implementation}
%\section{Implementation: Ring Abstraction}
\label{sec:implement}
\subsection{Distributed \mbox{$G_t$} in QMC Solver}
\label{distributedG4}
Before introducing the communication phase of the ring abstraction layer,
it is important to understand how the authors distributed the large device array $G_t$ across MPI ranks.
%
Original $G_t$ was compared, and $G^d_t$ versions were distributed
(Figure~\ref{fig:compare_original_distributed_g4}). 


In the original $G_t$ implementation, the measurements---which were computed by matrix-matrix multiplication---are distributed statically and independently over the MPI ranks to avoid
inter-node communications. Each MPI rank keeps its partial copy of $G_{t,i}$ to accumulate 
measurements within a rank, where $i$ is the rank index. 
After all the measurements are finished, a reduction step is 
taken to accumulate $G_{t,i}$ across all MPI ranks into a final and complete
$G_t$ in the root MPI rank. The size of the $G_{t,i}$ in each rank is 
the same size as the final and complete $G_t$. 

With the distributed $G^d_t$ implementation, this large device array 
$G_t$ was evenly partitioned across all MPI ranks; each portion of it is local to each MPI rank.
%
Instead of keeping its partial copy of $G_t$, 
each rank now keeps an instance of $G^d_{t,i}$ to accumulate measurements 
of a portion or sub-slice of the final and complete $G_t$, where the notation
$d$ in $G^d_t$  refers to the distributed version, and $i$ means the $i$-th rank.
%
The $G^d_{t,i}$ size in each rank is 
reduced to $1/p$ of the size of the final and complete $G_t$, comparing the same configuration 
in original $G_t$ implementation, where $p$ is the number of MPI ranks used. 
%
For example, in Figure~\ref{fig:distributed_g4}, there are four ranks, and rank $i$
now only keeps $G^d_{t,i}$, which is one-fourth the size of the original $G_t$ array size.
% and contains values indexing from range of $[0, ..., N/4)$ in original $G_t$ array where $N$ is the 
% number of entries in  $G_t$  when viewed as a one-dimensional array.

To compute the final and complete $G^d_{t,i}$ for the distributed $G^d_t$ implementation, 
each rank must see every $G_{\sigma,i}$ from all ranks. 
%
In other words, each rank must broadcast the
locally generated $G_{\sigma,i}$ to the remainder of the other ranks at every measurement step. 
%
To efficiently perform this ``all-to-all'' broadcast, a ring abstraction layer was built (Section. \ref{section:ring_algorithm}), which circulates
all $G_{\sigma,i}$ across all ranks.

% over high-speed GPUs interconnect (GPUDirect RDMA) to facilitate the communication phase.

% \begin{figure}
% \centering
% \subfloat[Original $G_t$ implementation.]
%     {\includegraphics[width=\columnwidth]{original_g4.pdf}}\label{fig:original_g4}

% \subfloat[Distributed $G_t$ implementation.]
%     {\includegraphics[width=0.99\columnwidth]{distributed_g4.pdf} \label{fig:distributed_g4}}

% \caption{Comparison of the original $G_t$ vs. the distributed $G^d_t$ implementation. Each rank contains one GPU resource.}
% \label{fig:compare_original_distributed_g4} 
% \end{figure} 

\begin{figure}
\centering
     \begin{subfigure}[b]{\columnwidth}
         \centering
         \includegraphics[width=\textwidth]{images/original_g4.pdf}
         \caption{Original $G_t$ implementation.}
         \label{fig:original_g4}
     \end{subfigure}
     
    \begin{subfigure}[b]{\columnwidth}
         \centering
         \includegraphics[width=\textwidth]{images/distributed_g4.pdf}
         \caption{Distributed $G_t$ implementation.}
         \label{fig:distributed_g4}
     \end{subfigure}
     
\caption{Comparison of the original $G_t$ vs. the distributed $G^d_t$ implementation. Each rank contains one GPU resource.}
\label{fig:compare_original_distributed_g4}
\end{figure}

\subsection{Pipeline Ring Algorithm}
\label{section:ring_algorithm}
A pipeline ring algorithm was implemented that broadcasts the $G_{\sigma}$ 
array circularly during every measurement. 
%
The algorithm (Algorithm \ref{alg:ring_algorithm_code}) is 
visualized in Figure~\ref{fig:ring_algorithm_figure}.

\begin{algorithm}
\SetAlgoLined
    generateGSigma(gSigmaBuf)\; \label{lst:line:generateG2}
    updateG4(gSigmaBuf)\;       \label{lst:line:updateG4}
    %\texttt{\\}
    {$i\leftarrow 0$}\;         \label{lst:line:initStart}
    {$myRank \leftarrow worldRank$}\;          \label{lst:line:initRankId}
    {$ringSize \leftarrow mpiWorldSize$}\;      \label{lst:line:initRingSize}
    {$leftRank \leftarrow (myRank - 1 + ringSize) \: \% \: ringSize $}\;
    {$rightRank \leftarrow (myRank + 1 + ringSize) \: \% \: ringSize $}\;
    sendBuf.swap(gSigmaBuf)\;           \label{lst:line:initEnd}
    \While{$i< ringSize$}{
        MPI\_Irecv(recvBuf, source=leftRank, tag = recvTag, recvRequest)\; \label{lst:line:Irecv}
        MPI\_Isend(sendBuf, source=rightRank, tag = sendTag, sendRequest)\; \label{lst:line:Isend}
        MPI\_Wait(recvRequest)\;        \label{lst:line:recevBuffWait}
        
        updateG4(recvBuf)\;             \label{lst:line:updateG4_loop}
        
        MPI\_Wait(sendRequest)\;        \label{lst:line:sendBuffWait}
        
        sendBuf.swap(recvBuf)\;         \label{lst:line:swapBuff}
        i++\;
    }
\caption{Pipeline ring algorithm}
\label{alg:ring_algorithm_code}
\end{algorithm}

\begin{figure}
	\centering
	\includegraphics[width=\columnwidth, trim=0 5cm 0 0, clip]{images/ring_algorithm.pdf}
	\caption{Workflow of ring algorithm per iteration. }
	\label{fig:ring_algorithm_figure}
\end{figure}

At the start of every new measurement, a single-particle Green's function $G_{\sigma}$
 (Line~\ref{lst:line:generateG2}) is generated 
and then used to update $G^d_{t,i}$ (Line~\ref{lst:line:updateG4})
via the formula in Eq.~(\ref{eq:G4}).
%
% Different from original method that performs 
% full matrix-matrix multiplication (Equation~(\ref{eq:G4})), the current ring algorithm only performs partial
% matrix-matrix multiplication that contributes to $G^d_{t,i}$, a subslice of the final and complete $G_t$.
%
Between Lines \ref{lst:line:initStart} to \ref{lst:line:initEnd}, the algorithm 
initializes the indices
of left and right neighbors and prepares the sending message buffer from the
previously generated $G_{\sigma}$ buffer. 
%
The processes are organized as a ring so that the first and last rank are considered to be neighbors to each other. 
%
A \textit{swap} operation is used to avoid unnecessary memory copies for \textit{sendBuf} preparation.
%
A walker-accumulator thread allocates an additional \textit{recvBuf} buffer of the same size 
as \textit{gSigmaBuf} to hold incoming 
\textit{gSigmaBuf} buffer from \textit{leftRank}. 

The \textit{while} loop is the core part of the pipeline ring algorithm. 
%
For every iteration, each thread in a rank 
receives a $G_{\sigma}$ buffer from its left neighbor rank and sends a $G_{\sigma}$ buffer to its right neighbor rank. 
A synchronization step (Line~\ref{lst:line:recevBuffWait}) is performed
afterward to ensure that each rank receives a new buffer to update the 
local $G^d_{t,i}$ (Line~\ref{lst:line:updateG4_loop}). 
%
Another synchronization step
follows to ensure that all send requests are finalized 
(Line~\ref{lst:line:sendBuffWait}). Lastly, another \textit{swap} operation is used to exchange
content pointers between \textit{sendBuf} and \textit{recvBuf} to avoid unnecessary memory copy and prepare
for the next iteration of communication.
%
In the multi-threaded version (Section~\ref{subsec:multi-thread}), the thread of index, \textit{i}, only communicates with
	threads of index, \textit{i}, in neighbor ranks, and each thread allocates two buffers: \textit{sendBuff} and \textit{recvBuff}.

The \textit{while} loop will be terminated after $\mbox{\textit{ringSize}} - 1$ steps. By that time, 
each locally generated $G_{\sigma,i}$ will have traveled across all MPI ranks and
updated $G^d_{t,i}$ in all ranks. Eventually, each $G_{\sigma,i}$ reaches
to the left neighbor of its birth rank. For example, $G_{\sigma,0}$ generated from rank $0$ will end 
in last rank in the ring communicator.

Additionally, if the $G_t$ is too large to be stored in one node, 
it is optional to accumulate all $G^d_{t,i}$
at the end of all measurements. 
%
Instead, a parallel write into the file system could be taken.

\subsubsection{Sub-Ring Optimization.}

A sub-ring optimization strategy is further proposed to reduce message communication
times if the large device array $G_t$ can fit in fewer devices. 
%
The sub-ring algorithm is visualized in Figure~\ref{fig:subring_algorithm_figure}.

For the ring algorithm (Section~\ref{section:ring_algorithm}), the size of the ring communicator
(\textit{mpiWorldSize}) is set to the same size of the global \mbox{\texttt{MPI\_COMM\_WORLD}}, and thus the size of $G_t$ is equally 
distributed across all MPI ranks.

However, to complete the update to $G^d_{t,i}$ in one measurement, 
one $G_{\sigma,i}$
must travel \textit{mpiWorldSize} ranks. In total, 
there are \textit{mpiWorldSize} numbers of $G_{\sigma,i}$
being sent and received concurrently in one measurement 
in the global
\mbox{\texttt{MPI\_COMM\_WORLD}} 
communicator. If the size of $G^d_{t,i}$ is relatively small per rank, then this will cause high communication overhead.

If $G_t$ can be distributed and fitted in fewer devices, then a shorter travel distance is required 
for $G_{\sigma,i}$, thus reducing the communication overhead. One reduction
step was performed at the end of all measurements to accumulate $G^d_{t,s_i}$, 
where $s_i$ means $i$-th rank on the $s$-th sub-ring.

At the beginning of MPI initialization, the global \mbox{\texttt{MPI\_COMM\_WORLD}} was partitioned  into several new sub-ring communicators by using \mbox{\texttt{MPI\_Comm\_split}}. 
% where each new communicator represents conceptually a subring. 
The new
communicator information was passed to the DCA++ concurrency class by substituting the original global 
\mbox{\texttt{MPI\_COMM\_WORLD}} with this new communicator. Now, only a few minor modifications
are needed to transform the ring algorithm (Algorithm~\ref{alg:ring_algorithm_code})
to sub-ring Algorithm~\ref{alg:sub_ring_algorithm}. In Line~\ref{lst:line:initRankId}, \textit{myRank} is 
initialized to \textit{subRingRank} instead of \textit{worldRank}, where 
\textit{subRingRank} is the rank index in the local sub-ring communicator. 
%
In Line~\ref{lst:line:initRingSize}, \textit{ringSize} is initialized to \textit{subRingSize}
instead of \textit{mpiWorldSize}, where \textit{subRingSize} is the
size of the new communicator.
%
The general ring algorithm is a special case for the sub-ring algorithm because the
\textit{subRingSize} of the general ring algorithm is equal to \textit{mpiWorldSize}, and
there is only one sub-ring group throughout all MPI ranks.


\LinesNumberedHidden
\begin{algorithm}
    {$\mbox{\textit{myRank}} \leftarrow \mbox{\textit{subRingRank}}$}\;         
    {$\mbox{\textit{ringSize}} \leftarrow \mbox{\textit{subRingSize}}$}\;      
\caption{Modified ring algorithm to support sub-ring communication}
\label{alg:sub_ring_algorithm}
\end{algorithm}


\begin{figure}
	\centering
	\includegraphics[width=\columnwidth, trim=0 5cm 0 0, clip]{images/subring_alg.pdf}
	\caption{Workflow of sub-ring algorithm per iteration. Every consecutive $S$ rank forms a sub-ring communicator, 
	and no communication occurs between sub-ring communicators until all measurements are finished. Here, $S$ is the number of ranks in a sub-ring.}
	\label{fig:subring_algorithm_figure}
\end{figure}

\subsubsection{Multi-Threaded Ring Communication.}
\label{subsec:multi-thread}
To take advantage of the multi-threaded QMC model already in DCA++, 
multi-threaded ring communication support was further implemented in the ring algorithm.
%
Figure~\ref{fig:dca_overview} shows that in the original DCA++ method,
each walker-accumulator
thread in a rank is independent of each other, and all the threads in a 
rank synchronize only after all rank-local measurements are finished.
%
Moreover, during every measurement, each walker-accumulator thread
generates its own thread-private $G_{\sigma, i}$ to update $G_t$. 
%

The multi-threaded ring algorithm now allows concurrent message exchange so that threads of same rank-local thread index exchange their thread-private $G_{\sigma, i}$. 
%
Conceptually, there are $k$ parallel and independent rings, where $k$ 
is number of threads per rank, because threads of the same local thread ID
form a closed ring. 
%
For example, a thread of index $0$ in rank $0$ will send its $G_\sigma$ to 
the thread of index $0$ in rank $1$ and receive another $G_\sigma$ from thread index of $0$ 
from last rank in the ring algorithm.
%

The only changes in the ring algorithm are offsetting the tag values 
(\texttt{recvTag} and \texttt{sendTag}) by the thread index value. For example,
Lines~\ref{lst:line:Irecv} and ~\ref{lst:line:Isend} from 
Algorithm~\ref{alg:ring_algorithm_code} are modified to Algorithm~\ref{alg:multi_threaded_ring}.

\LinesNumberedHidden
\begin{algorithm}
        MPI\_Irecv(recvBuf, source=leftRank, tag = recvTag + threadId, recvRequest)\; 
        MPI\_Isend(sendBuf, source=rightRank, tag = sendTag + threadId, sendRequest)\;
\caption{Modified ring algorithm to support multi-threaded ring}
\label{alg:multi_threaded_ring}
\end{algorithm}

To efficiently send and receive $G_\sigma$, each thread
will allocate one additional \textit{recvBuff} to hold incoming 
\textit{gSigmaBuf} buffer from \textit{leftRank} and perform send/receive efficiently.
%
In the original DCA++ method, there are $k$ numbers of buffers of $G_\sigma$ 
size per rank, and in the multi-threaded ring method, there are $2k$
numbers of buffers of $G_\sigma$ size per rank, where $k$ is number of 
threads per rank.


\section{Numerical Experiments: Diffusion Bridge Sampling}
\label{sec:numerics}
\section{Numerical implementation and solution}
In this section, we explain how to implement and solve the minimization problem \eqref{eq:dynamic_recon} numerically which, depending on the amount of time steps $T$, can be challenging. 
We derive a general primal-dual algorithm for its solution, before we line out some strategies to reduce computational costs and speed up the implementation at the end of this section. 

\subsection{Gradients and sampling operators}
\label{subsec:implementation of operators}

In order to use the discrete total variation already defined in Section \ref{subsubsec:TV}, we need a discrete gradient operator that maps an image $u \in \C^N$ to its gradient $\nabla u \in \C^{N \times 2}$. 
Following \cite{ChambollePock}, we implement the gradient by standard forward differences. Moreover, we  will use its discrete adjoint, the negative divergence $-\diverg$, defined by the identity $\langle \nabla u, w \rangle_{\C^{N \times 2}} = - \langle u, \diverg(w) \rangle_{\C^N}$.
The inner product on the gradient space $\C^{N\times 2}$ is defined in a straightforward way as 
\begin{align*}
	\langle v,w \rangle_{\C^{N \times 2}} = \Real(v_1^* w_1) + \Real(v_2^* w_2),
\end{align*}
%
for $v,w \in \C^{N \times 2}$.
For $v \in \C^{N \times 2}$, the (isotropic) 1-norm is defined by 
\begin{align*}
	\|v \|_1 := \sum_{i=1}^N \sqrt{|(v_i)_1|^2 + |(v_i)_2|^2},
\end{align*}
and accordingly the dual $\infty$-norm for $w \in \C^{N \times 2}$ is given by 
\begin{align*}
	\| w \|_\infty := \max_{i=1,\cdots,N} |w_i| = \max_{i=1,\cdots,N} ~ \sqrt{|(w_i)_1|^2 + |(w_i)_2|^2}.
\end{align*}

The sampling operators $\Kcal_t \colon \C^N \to \C^{M_t}$ (and analogously for $\Kcal_0 \colon \C^{N_0} \to \C^{M_0}$) we consider are either a standard fast Fourier transform (FFT) on a Cartesian grid, followed by a projection onto the sampled frequencies, or a non-uniform fast Fourier transform (NUFFT), in case the sampled frequencies are not located on a Cartesian grid \cite{Fessler:NUFFT}. \\

\noindent {\it Fourier transform on a Cartesian grid - the simulated data case}\\
\noindent
In the numerical study on artificial data we use a simple version of the Fourier transform and sampling operator (the same can also be found in e.g. \cite{Ehrhardt2016,Rasch2017}). 
We discretize the image domain on the unit square using an (equi-spaced) Cartesian grid with $N_1 \times N_2$ pixels such that the discrete grid points are given by 
\begin{align*}
 \Omega_N = \left\{ \left(\frac{n_1}{N_1-1}, \frac{n_2}{N_2-1} \right) ~\Big|~ n_1 = 0, \dots, N_1-1, \; n_2 = 0, \dots N_2-1 \right\}.
\end{align*}
We proceed analogously with the $k$-space, i.e. the location of the $(m_1,m_2)$-th Fourier coefficient is given by $(m_1/(N_1-1), m_2/(N_2-1))$. Then, we arrive at the following formula for the (standard) Fourier transform $\Fcal$ applied to $u \in \C^{N_1 \times N_2}$:
\begin{align*}
 (\Fcal u)_{m_1,m_2} = \frac{1}{N_1 N_2} \sum_{n_1=0}^{N_1-1} \sum_{n_2=0}^{N_2-1} u_{n_1,n_2} e^{-2\pi i \left(\frac{n_1 m_1}{N_1} + \frac{n_2 m_2}{N_2} \right)},
\end{align*}
where $ m_1 = 0, \dots, N_1-1, m_2 = 0, \dots, N_2 -1$.
For simplicity, we use a vectorized version such that $\Fcal \colon \C^N \to \C^N$ with $N = N_1 \cdot N_2$.
We then employ a simple sampling operator $\Scal_t \colon \C^N \to \C^{M_t}$ which discards all Fourier frequencies which are not located on the desired sampling geometry at time $t$ (i.e. the chosen spokes). 
More precisely, following \cite{Ehrhardt2016}, if we let $\Pcal_t \colon \{1,\dots,M_t \} \to \{1,\dots,N\}$ be an injective mapping which chooses $M_t$ Fourier coefficients from the $N$ coefficients available, we can define the sampling operator $\Scal$ applied to $f \in \C^N$ as
\begin{align*}
 \Scal_t \colon \C^N \to \C^{M_t}, \quad (\Scal_t f)_k = f_{\Pcal_t(k)}.
\end{align*}
The full forward operator $\Kcal_t$ can hence be expressed as 
\begin{align}\label{eq:forward_op_art}
 \Kcal_t \colon \C^N \xrightarrow{\Fcal} \C^N \xrightarrow{\Scal_t} \C^{M_t}.
\end{align}
The corresponding adjoint operator of $\Kcal_t$ is given by  
\begin{align*}
 \Kcal_t^* \colon \C^{M_t} \xrightarrow{\Scal_t^*} \C^N \xrightarrow{\Fcal^{-1}} \C^N,
\end{align*}
where $\Fcal^{-1}$ denotes the standard inverse Fourier transform and $\Scal_t^*$ `fills' the missing frequencies with zeros, i.e. 
\begin{align*}
 (\Scal_t^*z)_l = \sum_{k=1}^{M_t} z_k \delta_{l,\Pcal_t(k)}, \qquad \text{for } l = 1, \dots, N.
\end{align*}
For the prior $u_0$, we choose a full Cartesian sampling, which corresponds to $\Pcal_0$ being the identity. For the subsequent dynamic scan, we set up $\Pcal_t$ such that it chooses the frequencies located on (discrete) spokes through the center of the $k$-space.     
It is important to notice that this implies that the locations of the (discretized) spokes are still located on a Cartesian grid, which allows to employ a standard fast Fourier transform (FFT) followed by the above projection onto the desired frequencies. 
This is not the case for the operators we use for real data. \\

\noindent {\it Non-uniform Fourier transform - the real data case}\\ 
\noindent 
In contrast to the above (simplified) setup for artificial data, in many real world application the measured $k$-space frequencies $\xi_m$ in \eqref{eq:fourier_transform} are {\it not} located on a Cartesian grid. 
While this is not a problem with respect to the formula itself, it however excludes the possibility to employ a fast Fourier transform, 
%numerically
which usually reduces the computational costs of an $N$-point Fourier transform from an order of $O(N^2)$ to $O(N \log N)$.
To get to a similar order of convergence also for non-Cartesian samplings, it is necessary to employ the concept of non-uniform fast Fourier transforms (NUFFT) \cite{Fessler:NUFFT,Fessler:code,Matej2004,Nguyen:1999,Strohmer2000}. 
We only give a quick intuition here and for further information we refer the reader to the literature listed above. 
The main idea is to use a (weighted) and oversampled standard Cartesian $K$-point FFT $\Fcal$, $K \geq N$ followed by an interpolation $\Scal$ in $k$-space onto the desired frequencies $\xi_m$. 
Note that the oversampling takes place in $k$-space.
The operator $\Kcal_t$ for time $t$ can hence again be expressed as a concatenation of a $K$-point FFT and a sampling operator 
\textbf{\begin{align*}
 \Kcal_t \colon \C^N \xrightarrow{\Fcal} \C^N \xrightarrow{\Scal_t} \C^{M_t}.
\end{align*}}
For our numerical experiments with the experimental DCE-MRI data, the sampling operator $\Scal_t$ and its adjoint were taken from the NUFFT package \cite{Fessler:code}.

\subsection{Numerical solution}
%
Due to the nondifferentiablity and the involved operators we apply a primal-dual method \cite{ChambollePock} to solve the minimization problem \eqref{eq:dynamic_recon}. 
We first line out how to solve the (simple) TV-regularized problem for the prior (\ref{tvu0}) and then extend the approach to the dynamic problem. 
Interestingly, the problem for the prior already provides all the ingredients needed for the numerical solution of the dynamic problem, which can then be done in a very straightforward way. 
We consider the problem 
\begin{equation} \label{tvu0}
	\min_{u_0} ~ \frac{\alpha_0}{2} \| \Kcal_0 u_0 - f_0 \|_{\C^{M_0}}^2 + \| \nabla u_0 \|_1, 
\end{equation}
with $u_0 \in \C^{N_0}$.
Dualizing both terms leads to its primal-dual formulation 
\begin{align}\label{eq:tv_pd}
	\min_{u_0} \max_{y_1,y_2} ~ \langle y_1, \Kcal_0 u_0 - f_0 \rangle_{C^{M_0}} - \frac{1}{2 \alpha_0} \|y_1 \|_{\C^{M_0}}^2 + \langle y_2, \nabla u_0 \rangle_{\C^{N_0 \times 2}} + \chi_{C}(y_2),
\end{align}
where $y_1 \in \C^{M_0}$ and $\chi_{C}$ denotes the characteristic function of the set  
\begin{align*}
	C := \{ y \in \C^{N_0 \times 2} ~|~ \|y \|_\infty \leq 1 \}.
\end{align*}
% 
The primal-dual algorithm in \cite{ChambollePock} now essentially consists in performing a proximal gradient descent on the primal variable $u_0$ and a proximal gradient ascent on the dual variables $y_1$ and $y_2$, where the gradients are taken with respect to the linear part, the proximum with respect to the nonlinear part. 
We hence need to compute the proximal operators for the nonlinear parts in \eqref{eq:tv_pd} to obtain the update steps for $u_0$ and $y_1,y_2$. 
It is easy to see that the proximal operator for $\phi(y_1) = \frac{1}{2 \alpha} \|y_1\|_{\C^{M_0}}^2$ is given by 
\begin{align}\label{eq:prox_dual_l2}
	y_1 = \prox_{\sigma \phi} (r) \Leftrightarrow y_1 = \frac{\alpha r}{\alpha + \sigma}.
\end{align}
The proximal operators for the update of $y_2$ are given by a simple projection onto the set $C$, i.e. 
\begin{align}\label{eq:prox_proj}
	y_2 = \proj_C (r) \Leftrightarrow (y_2)_i = r_i / \max (|r_i|,1) \quad \text{for all } i.
\end{align}
Putting everything together leads to Algorithm \ref{alg:prior}.
\begin{algorithm}[t!] 
\caption{\textbf{Reconstruction of the prior}}
{
\begin{algorithmic}[1]
\Require step sizes $\tau,\sigma > 0$, data $f_0$, parameter $\alpha_0$
\Ensure $u_0^0 = \bar{u}_0^0 = \Kcal_0^*f_0, ~ y_1^0 = y_2^0 = 0$
	\While{$\sim$ stop crit}
    	\State {\it Dual updates}
        \State $y_1^{k+1} = (\alpha_0 \left[ y_1^k + \sigma (\Kcal_0 \bar{u}_0^k - f_0)\right]) / (\alpha_0 + \sigma)$
          \State $y_2^{k+1} = \proj_C \left(y_2^k + \sigma \nabla \bar{u}_0^k\right)$
          \State {\it Primal updates}
          \State $u_0^{k+1} =  u_0^k - \tau \left[ \Kcal_0^* y_1^{k+1} - \diverg(y_2^{k+1}) \right]$
          \State {\it Overrelaxation}
          \State $\bar{u}_0^{k+1}= 2 u_0^{k+1} - u_0^k$
	\EndWhile\\
\Return $u_0 = u_0^k$
\end{algorithmic}
}
\label{alg:prior}
\end{algorithm}
\ \\

The numerical realization of the dynamic problem is now straightforward.
In order to deal with the infimal convolution, we use its definition and introduce an additional auxiliary variable yielding 
\begin{alignat*}{4}
	&\min_{\ubold}&& ~ &&\sum_{t=1}^T \frac{\alpha_t}{2} \| \Kcal_t u_t - f_t \|_{\C^{M_t}}^2 + \sum_{t=1}^{T-1} \frac{\gamma_t}{2} \|u_{t+1} - u_t \|_{\C_N}^2 + \sum_{t=1}^T w_t \TV(u_t) \\
	& && + && \sum_{t=1}^T (1-w_t) \ICBTV^{p_0}(u_t,u_0) \\    
   = &\min_{\ubold,\zbold}&& ~ &&\sum_{t=1}^T \frac{\alpha_t}{2} \| \Kcal_t u_t - f_t \|_{\C^{M_t}}^2 + \sum_{t=1}^{T-1} \frac{\gamma_t}{2} \|u_{t+1} - u_t \|_{\C_N}^2 +\sum_{t=1}^T w_t \| \nabla u_t \|_1\\
   & && + &&\sum_{t=1}^T (1-w_t) \left[ \| \nabla (u_t - z_t) \|_1 + \| \nabla z_t \|_1  - \langle p_0,u_t \rangle_{\C^N} + \langle 2 p_0, z_t \rangle_{\C^N} \right]
\end{alignat*}
where $\ubold = [u_1, \dots, u_T] \in \C^{N \times T}$ and $\zbold = [z_1, \dots, z_T] \in \C^{N \times T}$.
Introducing a dual variable $\ybold$ for all the terms containing an operator, leads to the primal-dual formulation 
\begin{alignat}{4}
\label{eq:primal_dual}
	&\min_{\ubold,\zbold} \max_{\ybold} && ~ && \sum_{t=1}^T \left(\langle y_{t,1}, \Kcal_t u_t - f_t \rangle_{\C^{M_t}} - \frac{1}{2 \alpha_t} \|y_{t,1} \|_{\C^M}^2 \right) + \sum_{t=1}^{T-1} \frac{\gamma_t}{2} \| u_{t+1} - u_t \|_{\C^N}^2 \notag \\
    & && + && \sum_{t=1}^T \left(\langle y_{t,2}, \nabla u_t \rangle_{\C^{N \times 2}} + \langle y_{t,3}, \nabla (u_t - z_t) _{\C^{N \times 2}} + \langle y_{t,4}, \nabla z_t \rangle_{\C^{N \times 2}}\right) \notag \\
    & && - && \sum_{t=1}^T \langle (1-w_t)p_0,u_t \rangle_{\C^N} + \sum_{t=1}^T \langle 2(1-w_t)p_0,z_t \rangle_{\C^N}  \notag \\
    & && + && \sum_{t=1}^T \left(\chi_{C_{t,2}}(y_{t,2}) + \chi_{C_{t,3}}(y_{t,3}) + \chi_{C_{t,4}}(y_{t,4})\right)
\end{alignat}
where $\ybold = [\ybold_1, \dots, \ybold_T]$, $\ybold_t = [y_{t,1}, \dots, y_{t,4}]$, and for all $t = 1, \dots, T$, $u_{t} \in \C^{M_t}$ and
\begin{align*}
	&C_{t,2} := \{ y \in \C^{M \times 2} ~|~ \|y \|_{\infty} \leq w_t \}, \\
    &C_{t,3} := \{ y \in \C^{M \times 2} ~|~ \|y \|_{\infty} \leq (1-w_t) \}, \\
    &C_{t,4} := \{ y \in \C^{M \times 2} ~|~ \|y \|_{\infty} \leq (1-w_t) \}. \\
\end{align*}
%
\begin{algorithm}[t!]
\caption{\textbf{Dynamic reconstruction with structural prior}}
{
\begin{algorithmic}[1]
\Require step sizes $\tau,\sigma > 0$, subgradient $p_0$, for all $t=1,\dots,T$: data $f_t$, parameters $\alpha_t$, $w_t$, $\gamma_t$
\Ensure for all $t=1,\dots,T$: $u_t^0 = \bar{u}_t^0 = \Kcal_t^*f_t, ~ z_t^0 = \bar{z}_t^0 = 0, ~ y_{t,1}^0 = y_{t,2}^0 = y_{t,3}^0 = y_{t,4}^0 = 0$
	\While{$\sim$ stop crit}
    	\For{t=1,\dots,T} 
          \State {\it Dual updates}
          \State $y_{t,1}^{k+1} = \frac{\alpha_t \left[ y_{t,1} + \sigma (\Kcal_t \bar{u}_t^k - f_t)\right]}{\alpha_t + \sigma}$
          \State $y_{t,2}^{k+1} = \proj_{C_2}\left(y_{t,2}^k + \sigma \nabla \bar{u}_t^k\right)$
          \State $y_{t,3}^{k+1} = \proj_{C_3}\left(y_{t,3}^k + \sigma \nabla (\bar{u}_t^k - \bar{z}_t^k) \right)$
          \State $y_{t,4}^{k+1} = \proj_{C_4}\left(y_{t,4}^k + \sigma \nabla \bar{z}_t^k \right)$
          \State {\it Primal updates}
          \State $u_t^{k+1} =  \frac{u_t^k - \tau \left[ \Kcal_t^* y_{t,1}^{k+1} - \diverg(y_{t,2}^{k+1}) - \diverg(y_{t,3}^{k+1}) - (1-w_t)p_0 \right] + \tau \gamma_t u_{t+1}^k + \tau \gamma_{t-1} u_{t-1}^k}{\tau (\gamma_t + \gamma_{t+1}) +1}$
          \State $z_t^{k+1} - \tau \left[ 2(1-w_t) p_0 + \diverg(y_{t,3}^{k+1}) - \diverg(y_{t,4}^{k+1}) \right]$
          \State {\it Overrelaxation}
          \State $(\bar{u}_t^{k+1}, \bar{z}_t^{k+1}) = 2 (u_t^{k+1}, z_t^{k+1}) - (u_t^k,z_t^k)$
    	\EndFor
	\EndWhile\\
\Return for all $t = 1,\dots,T$: $u_t = u_t^k$
\end{algorithmic}
}
\label{alg:fmri}
\end{algorithm}
%
To solve the problem, we again perform a proximal gradient descent on the primal variables $\ubold$ and $\zbold$, and a proximal gradient ascent on the dual variables $\ybold$, where the gradients are taken with respect to the linear parts, the proximum with respect to the nonlinear parts. 
We hence need to compute the proximal operators for the nonlinear parts in \eqref{eq:primal_dual} to obtain the update steps for $u_t,z_t$ and $\ybold_t$ for every $t = 1, \dots, T$. 
The proximal operators for $\phi_t(y_{t,1}) = \frac{1}{2 \alpha_t} \| y_{t,1} \|_{\C^{M_t}}^2$ can be computed exactly as in \eqref{eq:prox_dual_l2}.
The proximal operators for the updates of $y_{t,j}$, $j = 2,3,4$, are given by projections onto the sets $C_{t,j}$ similar to \eqref{eq:prox_proj}.
For the squared norm related to the time regularization, we notice that for every $1< t < T$, $u_t$ only interacts with the previous and the following time step, i.e. $u_{t-1}$ and  $u_{t+1}$. 
Hence, analogously to $\phi_t$, the proximum for 
\begin{align*}
	\psi_t(u_t) = \frac{\gamma_{t-1}}{2} \|u_t - u_{t-1}\|_{\C^N}^2 + \frac{\gamma_t}{2} \|u_{t+1} - u_t\|_{\C^N}^2
\end{align*}
is given by 
\begin{align*}
	u_t = \prox_{\tau \psi_t} (r) \Leftrightarrow u_t = \frac{r + \tau \gamma_t u_{t+1} + \tau \gamma_{t-1} u_{t-1}}{\tau (\gamma_t + \gamma_{t-1}) + 1}. 
\end{align*}
The two odd updates for $t = 1$ and $t = T$ can be obtained by the same formula by simply setting $\gamma_0 = 0$ and $\gamma_T = 0$, respectively.
Putting everything together, we obtain Algorithm \ref{alg:fmri}.\\

\subsection{Step sizes and stopping criteria}
We quickly discuss the choice of the step sizes $\tau, \sigma$ and stopping criteria for Algorithm \ref{alg:fmri}. 
In most standard applications it stands to reason to choose the step sizes according to the condition $\tau \sigma \| L \|^2 < 1$ ($L$ denotes the collection of all operators) such that convergence of the algorithm is guaranteed \cite{ChambollePock}.
However, depending on $T$, i.e. the number of time frames we consider, the norm of the operator $L$ 
can be very costly to compute, or too large such that the condition $\tau \sigma \| L \|^2 < 1$ only permits extremely small step sizes. 
For practical use, we instead simply choose $\tau$ and $\sigma$ reasonably ''small`` and track both the energy of the problem and the primal-dual residual \cite{Goldstein:Adaptive} to monitor convergence. 
For the sake of brevity, we do not write down the primal-dual residual for Algorithm \ref{alg:fmri} and instead refer the reader to \cite{Goldstein:Adaptive} for its definition. 
The implementation is then straightforward.
We hence stop the algorithm if both, the relative change in energy between consecutive iterates and the primal-dual residual, have dropped below a certain threshold.

\subsection{Practical considerations}
It is clear that for a large number of time frames $T$ Algorithm \ref{alg:fmri} starts to require an increasing amount of time to return reliable results and for reasonably ''large`` step sizes $\tau$ and $\sigma$ it is even doubtful whether we can obtain convergence. 
In practice, it is hence necessary to divide the time series $\Tbold = \{1, \dots, T\}$ into $l$ smaller bits of consecutive time frames. More precisely, choose numbers $1 \leq T_1 < \dots < T_l = T$ such that $\Tbold = \Tbold_1 \cup \Tbold_2 \cup \dots \cup \Tbold_l$ with $\Tbold = \{1, \dots, T_1, T_1+1, \dots, T_2, \dots, T_{l-1}+1, \dots, T_l\}$.
We can then perform the reconstruction separately for all $\Tbold_i$. 
In order to keep the ''continuity`` between $\Tbold_i$ and $\Tbold_{i+1}$, we can include the last frame of $\Tbold_i$ into the reconstruction of $\Tbold_{i+1}$ by letting $\gamma_{T_i} \neq 0$ and choosing $u_{T_i}$ as the respective last frame of $\Tbold_i$.
This divides the overall problem into smaller and easier subproblems, which can be solved faster.
In practice, we observed that a size of five to ten frames per subset $\Tbold_i$ is a reasonable choice, which essentially gives very similar results as doing a reconstruction for the entire time series $\Tbold$.









\section{Outlook and Future Work}
\label{sec:outlook}

A new family of Langevin samplers was introduced in this paper. These new SDE samplers consist of perturbations of the underdamped Langevin dynamics (that is known to be ergodic with respect to the canonical measure), where auxiliary drift terms in the equations for both the position and the momentum are added, in a way that the perturbed family of dynamics is ergodic with respect to the same (canonical) distribution. These new Langevin samplers were studied in detail for Gaussian target distributions where it was shown, using tools from spectral theory for differential operators, that an appropriate choice of the perturbations in the equations for the position and momentum can improve the performance of the Langvin sampler, at least in terms of reducing the asymptotic variance. The performance of the perturbed Langevin sampler to non-Gaussian target densities was tested numerically on the problem of diffusion bridge sampling.

The work presented in this paper can be improved and extended in several directions. First, a rigorous analysis of the new family of Langevin samplers for non-Gaussian target densities is needed. The analytical tools developed in~\cite{duncan2016variance} can be used as a starting point. Furthermore, the study of the actual computational cost and its minimization by an appropriate choice of the numerical scheme and of the perturbations in position and momentum would be of interest to practitioners. In addition, the analysis of our proposed samplers can be facilitated by using tools from symplectic and differential geometry. Finally, combining the new Langevin samplers with existing variance reduction techniques such as zero variance MCMC, preconditioning/Riemannian manifold MCMC can lead to sampling schemes that can be of interest to practitioners, in particular in molecular dynamics simulations. All these topics are currently under investigation.


\begin{comment}
\notate{There needs to be a conclusion to the paper}
\subsection{'Symplectic manifold MCMC'}

The generator of the unperturbed Langevin dynamics (\ref{eq:langevin})
given by 
\[
\mathcal{L}_{0}=M^{-1}p\cdot\nabla_{q}-\nabla V(q)\cdot\nabla_{p}-\Gamma M^{-1}p\cdot\nabla_{p}+\nabla\Gamma\nabla
\]
is often written as 
\[
\mathcal{L}_{0}=\{H,\cdot\}-\Gamma M^{-1}p\cdot\nabla_{p}+\nabla\Gamma\nabla,
\]
where the Hamiltonian is expressed in terms of the \emph{Hamiltonian
	\[
	H(q,p)=V(q)+\frac{1}{2}p^{T}M^{-1}p
	\]
}and the \emph{Poisson bracket
	\[
	\{A,B\}=\big((\nabla_{q}A)^{T}(\nabla_{p}A)^{T}\Pi_{0}\left(\begin{array}{c}
	\nabla_{q}B\\
	\nabla_{p}B
	\end{array}\right).
	\]
}Here $\Pi_{0}$ denotes the $2d\times2d$-matrix 
\[
\Pi_{0}=\left(\begin{array}{cc}
\boldsymbol{0} & -I\\
I & \boldsymbol{0}
\end{array}\right)
\]
and $A,B:\mathbb{R}^{2d}\rightarrow\mathbb{R}^{2d}$ are sufficiently
regular functions. Now observe that the generator (\ref{eq:generator})
can be expressed in the same way if $\Pi_{0}$ is replaced by 
\[
\tilde{\Pi}=\left(\begin{array}{cc}
\mu J_{1} & -I\\
I & \nu J_{2}
\end{array}\right).
\]
Therefore, the perturbations under investigation in this paper can
be interpreted more abstractly as a change of Poisson structure. In
this framework, the unperturbed Langevin dynamics (\ref{eq:langevin})
should be thought of as evolving in $\mathbb{R}^{2d}$ equipped with
the canonical symplectic structure associated to $\Pi_{0}$. The perturbed
dynamics (\ref{eq:perturbed_underdamped}) then represent
a process in $\mathbb{R}^{2d}$ equipped with the symplectic structure
given rise to $\tilde{\Pi}$. This alternative viewpoint has multiple
advantages. For instance, the underlying symplectic structure suggests
efficient numerical integrators for the perturbed dynamics. Moreover,
this formulation naturally allows for the possibility of introducing
point-dependent Poisson structures connented to perturbations $J_{1}(q,p)$
and $J_{2}(q,p)$ that depend on $q$ and $p$. In this way, it might
be possible to extend the results of this paper to target measures
with locally different covariance structures or targets that are far
away from Gaussian. Lastly, this formulation interacts nicely with
\emph{Riemannian manifold Monte Carlo }approaches (see \cite{GirolamiCalderhead2011}),
where the metric structure of the underlying manifold instead of the
symplectic structure is changed. All of those connections will be
explored further in a subsequent publication.


\end{comment}


\section*{Acknowledgments}
 AD was supported by the EPSRC under grant No. EP/J009636/1. NN is supported by EPSRC through a Roth Departmental Scholarship. GP is partially supported by the EPSRC under grants No. EP/J009636/1, EP/L024926/1, EP/L020564/1 and EP/L025159/1. Part of the work reported in this paper was done while NN and GP were visiting the Institut Henri Poincar\'{e} during the Trimester Program "Stochastic Dynamics Out of Equilibrium". The hospitality of the Institute and of the organizers of the program is greatly acknowledged. 

\appendix
\section{Estimates for the Bias and Variance}
\label{app:proofs}

\begin{proof}[of Lemma \ref{lemma:bias}]
Suppose that $(P_t)_{t \ge 0}$ satisfies \eqref{eq:hypocoercive estimate}.  Let $\pi_0$ be an initial distribution of $(X_t)_{t \ge 0}$ such that $\pi_0 \ll \pi$ and $h = \frac{d\pi_0}{d\pi} \in L^2(\pi)$.  Slightly abusing notation, we denote by $\pi_0 P_t$ the law of $X_t$ given $X_0 \sim \pi$.  Then
\begin{align*}
	\lVert \pi_0 P_t - \pi \rVert_{TV} = \left\lVert P_t^* h - 1 \right\rVert_{L^1(\pi)} \leq \left\lVert P_t^* \right\rVert_{L^2(\pi)\rightarrow L^2(\pi)} \left\lVert h - 1 \right\rVert_{L^2(\pi)} \leq Ce^{-\lambda t}\left\lVert h - 1 \right\rVert_{L^2(\pi)},
\end{align*}
where $P_t^*$ denotes the $L^2(\pi)$-adjoint of $P_t$.  Since $f$ is assumed to be bounded, we immediately obtain
$$
  \norm{\mathbb{E}[f(X_t) | X_0 \sim \pi_0] - \pi(f)} \leq C\Norm{f}_{L^\infty}e^{-\lambda t}\left(\mbox{Var}_{\pi}\left[\frac{d\pi_0}{d\pi}\right]\right)^{1/2},
$$
and so, for $X_0 \sim \pi_0$,
$$
	\norm{\pi_T(f) - \pi(f)} \leq \frac{C}{\lambda T}{\left(1 - e^{-\lambda t}\right)}\lVert f \rVert_{L^\infty}\left(\mbox{Var}_{\pi}\left[\frac{d\pi_0}{d\pi}\right]\right)^{1/2},
$$
as required.
  \qed 
\end{proof}

\begin{proof}[of Lemma \ref{lemma:variance}]
Given $f \in L^2(\pi)$, for fixed $T > 0$, 
\begin{equation}
  \chi_T(x) := \int_{0}^{T} \left(\pi(f) - P_t f(x)\right)\mathrm{d}t.
\end{equation}
Then we have that $\chi_T \in \mathcal{D}(\mathcal{L})$ and $\mathcal{L}\chi_T  = f - P_T f$, moreover
\begin{align*}
  \Norm{\chi_{T} - \chi_{T'}}_{L^2(\pi)} &= \Norm{\int^{T'}_{T} P_t(f)-\pi(f)\,\mathrm{d}t}_{L^2(\pi)}\\
  &\leq C\Norm{f}_{L^2(\pi)}\int_{T}^{T'}e^{-\lambda t}\,\mathrm{d}t,
\end{align*}
so that $\lbrace \chi_T \rbrace_{T \geq 0}$ is a Cauchy sequence in $L^2(\pi)$ converging to $\chi = \int_0^\infty \left(\pi(f) - P_tf\right)\mathrm{d}t$.  Since $\mathcal{L}$ is closed and
$$
  (\mathcal{L}\chi_T, \chi_T) \rightarrow (f-\pi(f), \chi),\quad T \rightarrow \infty,
$$
in $L^2(\pi)$, it follows that $\chi\in\mathcal{D}(\mathcal{L})$ and $\mathcal{L}\chi = f - \pi(f)$.  Moreover,
$$
  \Norm{\chi}_{L^{2}(\pi)} \leq \int_0^\infty \Norm{P_t(f) - \pi(f)}_{L^2(\pi)}\,\mathrm{d}t \leq K_{\lambda}\Norm{f-\pi(f)}_{L^2(\pi)},\quad 
$$
where $K_{\lambda} = C\int_0^\infty e^{-\lambda t}\,\mathrm{d}t$. 
Since we assume that $f$ is smooth, the coefficients are smooth and $\gen$ is hypoelliptic, then $\mathcal{L}\chi = f-\pi(f)$ implies that  $\chi \in C^{\infty}(\R^d)$, and thus we can apply It\^{o}'s formula to $\chi(X_t)$ to obtain:
$$
\frac{1}{T}\int_0^T \left[f(X_t) - \pi(f)\right]\,\mathrm{d}t = \frac{1}{T}\left[\chi(X_0) - \chi(X_T)\right] + \frac{1}{T}\int_0^T \nabla\chi(X_t)\sigma(X_t)\,\mathrm{d}W_t.
$$
One can check that the conditions of \cite[Theorem 7.1.4]{EthierKu86} hold.  In particular, the following central limit theorem follows
$$
	\frac{1}{\sqrt{T}}\int_0^T \nabla\chi(X_t)\sigma(X_t)\,\mathrm{d}W_t \xrightarrow{d} \mathcal{N}(0,2\sigma^2_f),\quad \mbox{ as } T \rightarrow \infty.
$$
By Theorem \ref{theorem:invariance_theorem}, the generator $\gen$ has the form
$$
	\gen = \pi^{-1}\nabla\cdot\left(\pi\Sigma \nabla \cdot \right) + \gamma\cdot\nabla,
$$
where $\nabla\cdot(\pi \gamma) = 0$.  It follows that
\begin{equation}
\label{eq:variance_equation}
\sigma^2_f = \inner{\Sigma \nabla\chi}{\nabla\chi}_{L^2(\pi)} = -\inner{\gen \chi}{\chi}_{L^2(\pi)}  = \inner{\chi}{f}_{L^2(\pi)} < \infty.
\end{equation}
First suppose that $X_0 \sim \pi$.  Then $(\chi(X_t))_{t\geq 0}$ is a stationary process, and so 
$$
	\frac{1}{\sqrt{T}}\left(\chi(X_0) - \chi(X_T)\right) \rightarrow 0,\quad \mbox{a.s as } T \rightarrow \infty.
$$
From which \eqref{eq:CLT} follows.  More generally, suppose that $X_0 \sim \pi_0$, where $\pi_0(x) = h(x)\pi(x)$ for $h \in L^2(\pi)$.  If $f \in L^\infty(\pi)$, then by \eqref{lemma:bias},
\begin{align*}
	|\chi(x)| &\leq \int_0^\infty |\pi(f) - P_t f(x)|\,dt \\
			  &\leq \int_0^\infty \lVert f\rVert_{L^\infty}\lVert\pi - \pi_0 P_t\rVert_{TV}\,dt \\
			  & \leq \frac{C}{\lambda }\lVert f\rVert_{L^\infty}\left(\mbox{Var}_{\pi}\left[\frac{d\pi_0}{d\pi}\right]\right)^{1/2},
\end{align*} 
so that $\chi \in L^\infty(\pi)$.  Therefore $\frac{1}{\sqrt{T}}(\chi(X_0)- \chi(X_T)) \xrightarrow{p} 0$ as $T \rightarrow \infty$, and so \eqref{eq:CLT} holds in this case, similarly.
\end{proof}



\section{Proofs of Section \ref{sec:perturbed_langevin}}
\label{app:hypocoercivity}

\begin{proof} of Lemma \ref{lem:hypoellipticity}
	We first note that $\gen$ in \eqref{eq:generator} can be written in the ``sum of squares'' form:
	$$
	\mathcal{L}=A_{0}+\frac{1}{2}\sum_{k=1}^{d}A_{k}^{2},
	$$
	where 
	$$
	A_{0}=M^{-1}p\cdot\nabla_{q}-\nabla_{q}V\cdot\nabla_{p}-\mu J_{1}\nabla_{q}V\cdot\nabla_{q}-\nu J_{2}M^{-1}p\cdot\nabla_{p}-\Gamma M^{-1}p\cdot\nabla_{p}
	$$
	and
	$$
	A_{k}=e_{k}\cdot\Gamma^{1/2}\nabla_{p}, \quad k = 1,\ldots, d.
	$$
	Here $\lbrace e_k \rbrace_{k=1,\ldots, d}$ denotes the standard Euclidean basis and $\Gamma^{1/2}$ is the unique positive definite square root of the matrix $\Gamma$.   The relevant commutators turn out to be
	\[
	[A_{0},A_{k}]=e_{k}\cdot\Gamma^{1/2} M^{-1}(\Gamma\nabla_{p}-\nabla_{q}-\nu J_{2}\nabla_{p}), \quad k = 1,\ldots, k.
	\]
	Because $\Gamma$ has full rank on $\R^d$, it follows that 
	\[
	\Span\{A_{k}:k=1,\ldots d\}=\Span\{\partial_{p_{k}}:k=1,\ldots,d\}.
	\]
	Since
	\[
	e_{k}\cdot\Gamma^{1/2} M^{-1}(\Gamma\nabla_{p}-\nu J_{2}\nabla_{p})\in\Span\{A_{j}:j=1,\ldots d\},\quad k=1,\ldots,d,
	\]
	and $\Span(\lbrace \Gamma^{1/2}M^{-1}\nabla_q \,:\, k=1,\ldots, d \rbrace) = \Span\{\partial_{q_{k}}:k=1,\ldots,d\}$, it follows that
	\[
	\Span(\lbrace A_{k}:k=0,1,\ldots,d\rbrace\cup\{[A_{0},A_{k}]:k=1,\ldots,d\}) = \R,
	\]
	so the assumptions of H\"{o}rmander's theorem hold.\qed
\end{proof}

\subsection{The overdamped limit}

The following is a technical lemma required for the proof of Proposition \ref{prop: overdamped limit}:
\begin{lemma}
	\label{lem:bounded p}Assume the conditions from Proposition \ref{prop: overdamped limit}.
	Then for every $T>0$ there exists $C>0$ such that 
	\[
	\mathbb{E} \left( \sup_{0\le t\le T}\vert p_{t}^{\epsilon}\vert^{2} \right) \le C.
	\]
	
\end{lemma}
\begin{proof}
	Using variation of constants, we can write the second line of (\ref{eq:rescaling})
	as 
	\[
	p_{t}^{\epsilon}=e^{-\frac{t}{\epsilon^{2}}(\nu J_{2}+\Gamma)M^{-1}}p_{0}-\frac{1}{\epsilon}\int_{0}^{t}e^{-\frac{(t-s)}{\epsilon^{2}}(\nu J_{2}+\Gamma)M^{-1}}\nabla_{q}V(q_{s}^{\epsilon})\mathrm{d}s+\frac{1}{\epsilon}\sqrt{2\Gamma}\int_{0}^{t}e^{-\frac{(t-s)}{\epsilon^{2}}(\nu J_{2}+\Gamma)M^{-1}}\mathrm{d}W_{s}.
	\]
	We then compute 
	\begin{align}
	\mathbb{E}\sup_{0\le t\le T}\vert p_{t}^{\epsilon}\vert^{2} & =\sup_{0\le t\le T}\left\lvert e^{-\frac{t}{\epsilon^{2}}(\nu J_{2}+\Gamma)M^{-1}}p_{0}\right\rvert^{2}+\frac{1}{\epsilon^{2}}\mathbb{E}\sup_{0\le t\le T}\left\lvert\int_{0}^{t}e^{-\frac{(t-s)}{\epsilon^{2}}(\nu J_{2}+\Gamma)M^{-1}}\nabla_{q}V(q_{s}^{\epsilon})\mathrm{d}s\right\rvert^{2}\nonumber \\
	& +\frac{1}{\epsilon^{2}}\mathbb{E}\sup_{0\le t\le T}\left\lvert\sqrt{2\Gamma}\int_{0}^{t}e^{-\frac{(t-s)}{\epsilon^{2}}(\nu J_{2}+\Gamma)M^{-1}}\mathrm{d}W_{s}\right\rvert^{2}\nonumber \\
	& -\frac{1}{\epsilon}\mathbb{E}\sup_{0\le t\le T}\left(e^{-\frac{t}{\epsilon^{2}}(\nu J_{2}+\Gamma)M^{-1}}p_{0}\cdot\int_{0}^{t}e^{-\frac{(t-s)}{\epsilon^{2}}(\nu J_{2}+\Gamma)M^{-1}}\nabla_{q}V(q_{s}^{\epsilon})\mathrm{d}s\right)\label{eq:P^2}\\
	& +\frac{1}{\epsilon}\mathbb{E}\sup_{0\le t\le T}\left(e^{-\frac{t}{\epsilon^{2}}(\nu J_{2}+\Gamma)M^{-1}}p_{0}\cdot\frac{1}{\epsilon}\sqrt{2\Gamma}\int_{0}^{t}e^{-\frac{(t-s)}{\epsilon^{2}}(\nu J_{2}+\Gamma)M^{-1}}\mathrm{d}W_{s}\right)\nonumber \\
	& -\frac{1}{\epsilon^{2}}\mathbb{E}\sup_{0\le t\le T}\left(\int_{0}^{t}e^{-\frac{(t-s)}{\epsilon^{2}}(\nu J_{2}+\Gamma)M^{-1}}\nabla_{q}V(q_{s}^{\epsilon})\mathrm{d}s\cdot\sqrt{2\Gamma}\int_{0}^{t}e^{-\frac{(t-s)}{\epsilon^{2}}(\nu J_{2}+\Gamma)M^{-1}}\mathrm{d}W_{s}\right).\nonumber 
	\end{align}
	Clearly, the first term on the right hand side of (\ref{eq:P^2})
	is bounded. For the second term, observe that 
	\begin{equation}
	\frac{1}{\epsilon^{2}}\mathbb{E}\sup_{0\le t\le T}\left\lvert\int_{0}^{t}e^{-\frac{(t-s)}{\epsilon^{2}}(\nu J_{2}+\Gamma)M^{-1}}\nabla_{q}V(q_{s}^{\epsilon})\mathrm{d}s\right\rvert^{2}\le\frac{1}{\epsilon^{2}}\sup_{0\le t\le T}\int_{0}^{t}\left\lVert e^{-\frac{(t-s)}{\epsilon^{2}}(\nu J_{2}+\Gamma)M^{-1}}\right\rVert^{2}\mathrm{d}s\label{eq:estimate1}
	\end{equation}
	since $V \in C^1(\mathbb{T}^d)$ and therefore $\nabla_{q}V$ is bounded. By the basic matrix exponential estimate
	$\Vert e^{-t(\nu J_{2}+\Gamma)M^{-1}}\Vert\le Ce^{-\omega t}$ for
	suitable $C$ and $\omega$, we see that (\ref{eq:estimate1}) can
	further be bounded by 
	\[
	\frac{1}{\epsilon^{2}}C\sup_{0\le t\le T}\int_{0}^{t}e^{-2\omega\frac{(t-s)}{\epsilon^{2}}}\mathrm{d}s=\frac{C}{2\omega}\left(1-e^{-2\omega\frac{T}{\epsilon^{2}}}\right),
	\]
	so this term is bounded as well. The third term is bounded by the
	Burkholder\textendash Davis\textendash Gundy inequality and a similar
	argument to the one used for the second term applies. The cross terms can
	be bounded by the previous ones, using the Cauchy-Schwarz inequality
	and the elementary fact that $\sup(ab)\le\sup a\cdot\sup b$ for $a,b>0$, so the
	result follows. \qed
\end{proof}

\begin{proof}
	[of Proposition \ref{prop: overdamped limit}] Equations (\ref{eq:rescaling})
	can be written in integral form as
	\[
	(\nu J_{2}+\Gamma)q_{t}^{\epsilon}=(\nu J_{2}+\Gamma)q_{0}^{\epsilon}+\frac{1}{\epsilon}\int_{0}^{t}(\nu J_{2}+\Gamma)M^{-1}p_{s}^{\epsilon}\mathrm{d}s-\mu\int_{0}^{t}(\nu J_{2}+\Gamma)J_{1}\nabla_{q}V(q_{s}^{\epsilon})\mathrm{d}s
	\]
	and 
	\begin{equation}
	-\int_{0}^{t}\nabla V(q_{s}^{\epsilon})\mathrm{d}s-\frac{1}{\epsilon}\int_{0}^{t}(\nu J_{2}+\Gamma)M^{-1}p_{s}^{\epsilon}\mathrm{d}s+\sqrt{2\Gamma}W(t)=\epsilon(p_{t}^{\epsilon}-p_{0}),\label{eq:rescaled p equation-1}
	\end{equation}
	where the first line has been multiplied by the matrix $\nu J_{2}+\Gamma$.
	Combining both equations yields
	\[
	q_{t}^{\epsilon}=q_{0}^{\epsilon}-\int_{0}^{t}(\nu J_{2}+\Gamma)\nabla_{q}V(q_{s}^{\epsilon})\mathrm{d}s-\epsilon(\nu J_{2}+\Gamma)^{-1}(p_{t}^{\epsilon}-p_{0})-\mu\int_{0}^{t}J_{1}\nabla_{q}V(q_{s})\mathrm{d}s+(\nu J_{2}+\Gamma)^{-1}\sqrt{2\Gamma}W_{t}.
	\]
	Now applying Lemma \ref{lem:bounded p} gives the desired result,
	since the above equation differs from the integral version of (\ref{eq:overdamped limit})
	only by the term $\epsilon(\nu J_{2}+\Gamma)^{-1}(p_{t}^{\epsilon}-p_{0})$
	which vanishes in the limit as $\epsilon\rightarrow0$. \qed
\end{proof}

\subsection{Hypocoercivity}

The objective of this section is to prove that the perturbed dynamics
(\ref{eq:perturbed_underdamped}) converges to equilibrium
exponentially fast, i.e. that the associated semigroup $(P_t)_{t\ge0}$ satisfies the estimate \eqref{eq:hypocoercive estimate}. We we will be using the theory of hypocoercivity outlined in
\cite{villani2009hypocoercivity} (see also the exposition in \cite[Section 6.2]{pavliotis2014stochastic}).
We provide a brief review of the theory of hypocoercivity.
\\\\
Let $(\mathcal{H},\langle\cdot,\cdot\rangle)$ be a real separable
Hilbert space and consider two unbounded operators $A$ and $B$ with
domains $D(A)$ and $D(B)$ respectively, $B$ antisymmetric. Let
$S\subset\mathcal{H}$ be a dense vectorspace such that $S\subset D(A)\cap D(B)$,
i.e. the operations of $A$ and $B$ are authorised on $S$. The theory
of hypocoercivity is concerned with equations of the form 
\begin{equation}
\label{eq:abstract fp equation}
\partial_{t}h+Lh=0,
\end{equation}
and the associated semigroup $(P_t)_{t\ge0}$ generated by $L=A^{*}A-B$. Let
us also introduce the notation $K=\ker L$. With the choices $\mathcal{H}=L^{2}(\widehat{\pi})$,
$A=\sigma\nabla_{p}$ and $B=M^{-1}p\cdot\nabla_{q}-\nabla_{q}V\cdot\nabla_{p}-\mu J_{1}\nabla_{q}V\cdot\nabla_{q}-\nu J_{2}M^{-1}p\cdot\nabla_{p},$
it turns out that $L$ is the (flat) $L^2(\mathbb{R}^{2d})$-adjoint of the generator $\mathcal{L}$ given in \eqref{eq:generator} and therefore equation \eqref{eq:abstract fp equation} is the Fokker-Planck equation associated to the dynamics \eqref{eq:perturbed_underdamped}. 
	In many situations of practical interest, the operator $A^{*}A$ is
	coercive only in certain directions of the state space, and therefore
	exponential return to equilibrium does not follow in general. In our
	case for instance, the noise acts only in the $p$-variables and therefore
	relaxation in the $q$-variables cannot be concluded a priori. However,
	intuitively speaking, the noise gets transported through the equations
	by the Hamiltonian part of the dynamics. This is what the theory of
	hypocoercivity makes precise. Under some conditions on the interactions
	between $A$ and $B$ (encoded in their iterated commutators), exponential
	return to equilibrium can be proved.
To state the main abstract theorem, we need the following definitions: 
\begin{definition}
	(Coercivity) Let $T$ be an unbounded operator on $\mathcal{H}$ with
	domain $D(T)$ and kernel $K$. Assume that there exists another Hilbert
	space $(\tilde{\mathcal{H}},\langle\cdot,\cdot\rangle_{\tilde{\mathcal{H}}})$,
	continuously and densely embedded in $K^{\perp}$. The operator is
	said to be $\lambda$-coercive if 
	\[
	\langle Th,h\rangle_{\tilde{\mathcal{H}}}\ge\lambda\Vert h\Vert_{\tilde{\mathcal{H}}}^{2}
	\]
	for all $h\in K^{\perp}\cap D(T)$. 
\end{definition}

\begin{definition}
	An operator $T$ on $\mathcal{H}$ is said to be relatively bounded
	with respect to the operators $T_{1},\ldots,T_{n}$ if the intersection
	of the domains $\cap D(T_{j})$ is contained in $D(T)$ and there
	exists a constant $\alpha>0$ such that 
	\[
	\Vert Th\Vert\le\alpha(\Vert T_{1}h\Vert+\ldots+\Vert T_{n}h\Vert)
	\]
	holds for all $h\in D(T)$. 
\end{definition}
We can now proceed to the main result of the theory.
\begin{theorem}{\cite[Theorem 24]{villani2009hypocoercivity}}
	\label{thm: hypocoercivity abstract}Assume there exists $N\in\mathbb{N}$
	and possibly unbounded operators $$C_{0},C_{1},\ldots,C_{N+1},R_{1},\ldots,R_{N+1},Z_{1},\ldots,Z_{N+1},$$
	such that $C_{0}=A$, 
	\begin{equation}
	[C_{j},B]=Z_{j+1}C_{j+1}+R_{j+1}\quad(0\le j\le N),\quad C_{N+1}=0,\label{eq:iterated commutators}
	\end{equation}
	and for all $k=0,1,\ldots,N$ 
	\begin{enumerate}[label=(\alph*)]
		\item \label{it:hypo1} $[A,C_{k}]$ is relatively bounded with respect to $\{C_{j}\}_{0\le j\le k}$
		and $\{C_{j}A\}_{0\le j\le k-1}$, 
		\item \label{it:hypo2} $[C_{k},A^{*}]$ is relatively bounded with respect to $I$ and $\{C_{j}\}_{0\le j\le k}$
		,
		\item \label{it:hypo3} $R_{k}$ is relatively bounded with respect to $\{C_{j}\}_{0\le j\le k-1}$
		and $\{C_{j}A\}_{0\le j\le k-1}$ and
		\item \label{it:hypo4} there are positive constants $\lambda_{i}$, $\Lambda_{i}$ such that
		$\lambda_{j}I\le Z_{j}\le\Lambda_{j}I$.
	\end{enumerate}
	Furthermore, assume that $\sum_{j=0}^{N}C_{j}^{*}C_{j}$ is $\kappa$-coercive
	for some $\kappa>0$.  Then, there exists $C\ge0$ and $\lambda>0$ such that 
	\begin{equation}
	\Vert P_t\Vert_{\mathcal{H}^{1}/K\rightarrow \mathcal{H}^{1}/K}\le Ce^{-\lambda t},\label{eq:hypocoercivity estimate}
	\end{equation}
	where $\mathcal{H}^{1}\subset\mathcal{H}$ is the subspace associated
	to the norm
	\begin{equation}
	\label{eq:abstractH1_norm}
	\Vert h\Vert_{\mathcal{H}^{1}}=\sqrt{\Vert h\Vert^{2}+\sum_{k=0}^{N}\Vert C_{k}h\Vert^{2}}
	\end{equation}
	and $K=\ker(A^{*}A-B)$. \end{theorem}
\begin{remark}
	Property (\ref{eq:hypocoercivity estimate}) is called \emph{hypocoercivity
		of $L$ on $\mathcal{H}^{1}:=(K^{\perp},\Vert\cdot\Vert_{\mathcal{H}^{1}})$.}
\end{remark}
If the conditions of the above theorem hold, we also get a regularization
result for the semigroup $e^{-tL}$ (see  \cite[Theorem A.12]{villani2009hypocoercivity}):
\begin{theorem}
	\label{thm:hypocoercive regularisation}Assume the setting and notation
	of Theorem \ref{thm: hypocoercivity abstract}. Then there exists
	a constant $C>0$ such that for all $k=0,1,\ldots,N$ and $t \in (0,1]$ the following
	holds:
	\[
	\Vert C_{k}P_{t}h\Vert\le C\frac{\Vert h\Vert}{t^{k+\frac{1}{2}}},\quad h\in\mathcal{H}.
	\]
\end{theorem}
\begin{proof}
	[of Theorem \ref{theorem:Hypocoercivity}]. We pove the claim by verifying
	the conditions of Theorem \ref{thm: hypocoercivity abstract}. Recall
	that $C_{0}=A=\sigma\nabla_{p}$ and 
	\[
	B=M^{-1}p\cdot\nabla_{q}-\nabla_{q}V\cdot\nabla_{p}-\mu J_{1}\nabla_{q}V\cdot\nabla_{q}-\nu J_{2}M^{-1}p\cdot\nabla_{p}.
	\]
	A quick calculation shows that 
	\[
	A^{*}=\sigma M^{-1}p-\sigma\nabla_{p},
	\]
	so that indeed 
	\[
	A^{*}A=\Gamma M^{-1}p\cdot\nabla_{p}-\nabla^{T}\Gamma\nabla=\mathcal{L}_{therm}
	\]
	and 
	\[
	A^{*}A-B=-\mathcal{L}^{*}.
	\]
	We make the choice $N=1$ and calculate the commutator 
	\[
	[A,B]=\sigma M^{-1}(\nabla_{q}+\nu J_{2}\nabla_{p}).
	\]
	Let us now set $C_{1}=\sigma M^{-1}\nabla_{q}$, $Z_{1}=1$ and $R_{1}=\nu\sigma M^{-1}J_{2}\nabla_{p}$,
	such that (\ref{eq:iterated commutators}) holds for $j=0$. Note
	that $[A,A]=0$\footnote{This is not true automatically, since $[A,A]$ stands for the array
		$([A_{j},A_{k}])_{jk}$.}, $[A,C_{1}]=0$ and $[A^{*},C_{1}]=0$. Furthermore, we have that
	\[
	[A,A^{*}]=\sigma M^{-1}\sigma.
	\]
	We now compute 
	\[
	[C_{1},B]=-\sigma M^{-1}\nabla^{2}V\nabla_{p}+\mu\sigma M^{-1}\nabla^{2}VJ_{1}\nabla_{q}
	\]
	and choose $R_{2}=[C_{1},B]$, $Z_{2}=1$ and recall that $C_{2}=0$
	by assumption (of Theorem \ref{thm: hypocoercivity abstract}). With those choices, assumptions \ref{it:hypo1}-\ref{it:hypo4} of Theorem \ref{thm: hypocoercivity abstract} are fulfilled. Indeed, assumption \ref{it:hypo1} holds trivially 
	since all relevant commutators are zero. Assumption \ref{it:hypo2} follows from the fact that $[A,A^{*}]=\sigma M^{-1}\sigma$ is clearly bounded
	relative to $I$. To verify assumption \ref{it:hypo3}, let us start with the
	case $k=1$. It is necessary to show that $R_{1}=\nu\sigma M^{-1}J_{2}\nabla_{p}$
	is bounded relatively to $A=\sigma\nabla_{p}$ and $A^{2}$.
	This is obvious since the $p$-derivatives appearing in $R_{1}$ can
	be controlled by the $p$-derivatives appearing in $A$. For $k=2,$
	a similar argument shows that $R_{2}=-\sigma M^{-1}\nabla^{2}V\nabla_{p}+\mu\sigma M^{-1}\nabla^{2}VJ_{1}\nabla_{q}$
	is bounded relatively to $A=\sigma\nabla_{p}$ and $C_{1}=\sigma M^{-1}\nabla_{q}$
	because of the assumption that $\nabla^{2}V$ is bounded. Note that it
	is crucial for the preceding arguments to assume that the matrices
	$\sigma$ and $M$ have full rank. Assumption \ref{it:hypo4} is trivially satisfied,
	since $Z_{1}$ and $Z_{2}$ are equal to the identity.  It remains to show that 
	\[
	T:=\sum_{j=0}^{N}C_{j}^{*}C_{j}
	\]
	is $\kappa$-coercive for some $\kappa>0$.  It is straightforward
	to see that the kernel of $T$ consists of constant functions and
	therefore 
	\[
	(\ker T)^{\perp}=\{\phi\in L^{2}(\mathbb{R}^{2d},\widehat{\pi}):\quad\widehat{\pi}(\phi)=0\}.
	\]
	Hence, $\kappa$-coercivity of $T$ amounts to the functional inequality
	\[
	\int_{\mathbb{R}^{2d}}\big(\vert\sigma M^{-1}\nabla_{q}\phi\vert^{2}+\vert\sigma\nabla_{p}\phi\vert^{2}\big)\mathrm{d}\widehat{\pi}\ge\kappa\bigg(\int_{\mathbb{R}^{2d}}\phi^{2}\mathrm{d}\widehat{\pi}-\left(\int_{\mathbb{R}^{2d}}\phi\mathrm{d}\widehat{\pi}\right)^{2}\bigg),\quad\phi\in H^{1}(\widehat{\pi}).
	\]
	Since the transformation $\phi\mapsto\psi$, $\psi(q,p)=\phi(\sigma^{-1}Mq,\sigma^{-1}p)$ is bijective on $H^{1}(\mathbb{R}^{2d},\widehat{\pi})$, the above is equivalent to 
	\[
	\int_{\mathbb{R}^{2d}}\big(\vert\nabla_{q}\psi\vert^{2}+\vert\nabla_{p}\psi\vert^{2}\big)\mathrm{d}\widehat{\pi}\ge\kappa\bigg(\int_{\mathbb{R}^{2d}}\psi^{2}\mathrm{d}\widehat{\pi}-\left(\int_{\mathbb{R}^{2d}}\psi\mathrm{d}\widehat{\pi}\right)^{2}\bigg),\quad\psi\in H^{1}(\widehat{\pi}),
	\]
	i.e. a Poincar\'{e} inequality for $\widehat{\pi}$. Since $\widehat{\pi}=\pi\otimes\mathcal{N}(0,M),$
	coercivity of $T$ boils down to a Poincar\'{e} inequality for $\pi$
	as in Assumption \ref{ass:bounded+Poincare}. This concludes the proof of the hypocoercive decay estimate
	(\ref{eq:hypocoercivity estimate}). Clearly, the abstract $\mathcal{H}^{1}$-norm from \eqref{eq:abstractH1_norm}
	is equivalent to the Sobolev norm $H^{1}(\widehat{\pi})$, and therefore it follows that there exist constants $C\ge0$ and $\lambda\ge0$ such that 
	\begin{equation}
	\label{eq:H1_decay}
	\Vert P_{t} f \Vert_{H^1(\widehat{\pi})}  \le C e^{-\lambda t} \Vert f \Vert _{H^1(\widehat{\pi})},
	\end{equation}
	for all $f \in H^1(\widehat{\pi})\setminus K$, where $K=\ker T$ consists of constant functions.  Let us now lift this estimate to $L^2(\widehat{\pi})$. There exist a constant $\tilde{C}\ge0$ such that 
	\begin{equation}
	\Vert h \Vert_{H^1(\widehat{\pi})} \le \tilde{C} \sum_{k=0}^2 \Vert C_k h \Vert_{L^2(\widehat{\pi})},
	\quad f \in H^1(\widehat{\pi}).
	\end{equation}
	Therefore, Theorem \ref{thm:hypocoercive regularisation} implies 
	\begin{equation}
	\label{eq:H1L2_reg}
	\Vert P_{1} f \Vert_{H^1(\widehat{\pi})} \le \tilde{C} \Vert f \Vert_{L^2(\widehat{\pi})},
	\quad f \in L^2(\widehat{\pi}), 
	\end{equation}
	for $t=1$ and a possibly different constant $\tilde{C}$. Let us now assume that $t\ge1$ and $f \in L^2(\widehat{\pi})\setminus K$.	It holds that
	\begin{equation}
	\Vert P_t f \Vert_{L^2(\widehat{\pi})} \le \Vert P_t f \Vert_{H^1(\widehat{\pi})} = \Vert P_{t-1}P_{1} f \Vert_{H^1(\widehat{\pi})}
	\le C e^{-\lambda (t-1)} \Vert P_{1} f \Vert_{H^1(\widehat{\pi})}, 
	\end{equation}
	where the last inequality follows from \eqref{eq:H1_decay}. Now applying \eqref{eq:H1L2_reg} and gathering constants results in 
	\begin{equation}
	\Vert P_t f\Vert_{L^2(\widehat{\pi})} \le C e^{-\lambda t}\Vert f \Vert_{L^2(\widehat{\pi})}, \quad f \in L^2(\widehat{\pi})\setminus K.
	\end{equation} 
	Note that although we assumed $t\ge1$, the above estimate also holds for $t\ge0$ (although possibly with a different constant $C$) since $\Vert P_t \Vert_{L^2(\widehat{\pi})\rightarrow L^2(\widehat{\pi})}$ is bounded on $[0,1]$. 
	\qed  
\end{proof}


\section{Asymptotic Variance of Linear and Quadratic Observables in the Gaussian Case}
\label{app:Gaussian_proofs}

We begin by deriving a formula for the asymptotic variance of observables
of the form 
\[
f(q)=q\cdot Kq+l\cdot q-\Tr K,
\]
with $K\in\mathbb{R}_{sym}^{d\times d}$ and $l\in\mathbb{R}^{d}$.
Note that the constant term is chosen such that $\widehat{\pi}(f)=0$.
The following calculations are very much along the lines of \cite[Section  4]{duncan2016variance}. Since the Hessian of $V$ is bounded and the target measure $\pi$ is Gaussian, Assumption \ref{ass:bounded+Poincare} is satisfied and exponential decay of the semigroup $(P_t)_{t\ge0}$ as in \eqref{eq:hypocoercive estimate} follows by  Theorem \ref{theorem:Hypocoercivity}. According to Lemma \ref{lemma:variance}, the
asymptotic variance is then given by 
\begin{equation}
\sigma_{f}^{2}=\langle \chi,f\rangle_{L^{2}(\widehat{\pi})},
\end{equation}
where $\chi$ is the solution to the Poisson equation 
\begin{equation}
\label{eq:Poisson equation Gauss}
-\mathcal{L}\chi=f,\quad\widehat{\pi}(\chi)=0.
\end{equation}
Recall that 
\[
\mathcal{L}=-Bx\cdot\nabla+\nabla^{T}Q\nabla=-x\cdot A\nabla+\nabla^{T}Q\nabla
\]
is the generator as in (\ref{eq:OU generator}), where for later convenience
we have defined $A=B^{T}$, i.e.
\begin{equation}
A=\left(\begin{array}{cc}
-\mu J & I\\
-I & \gamma I -\nu J
\end{array}\right) \in \mathbb{R}^{2d \times 2d}.\label{eq:Amatrix}
\end{equation}
In the sequel we will solve (\ref{eq:Poisson equation Gauss}) analytically.  First, we introduce the notation 
\[
\bar{K}=\left(\begin{array}{cc}
K & \boldsymbol{0}\\
\boldsymbol{0} & \boldsymbol{0}
\end{array}\right)\in\mathbb{R}^{2d\times2d}
\]
and 
\[
\bar{l}=\left(\begin{array}{c}
l\\
\boldsymbol{0}
\end{array}\right)\in\mathbb{R}^{2d},
\]
such that by slight abuse of notation $f$ is given by 
\[
f(x)=x\cdot\bar{K}x+\bar{l}\cdot x-\Tr\bar{K}.
\]
By uniqueness (up to a constant) of the solution to the Poisson equation \eqref{eq:Poisson equation Gauss} and
linearity of $\mathcal{L}$, $g$ has to be a quadratic polynomial,
so we can write 
\[
g(x)=x\cdot Cx+D\cdot x-\Tr C,
\]
where $C\in\mathbb{R}_{sym}^{2d\times2d}$ and $D\in\mathbb{R}^{2d}$ (notice that $C$ can be chosen to be symmetrical since $x \cdot C x$ does not depend on the antisymmetric part of $C$).
Plugging this ansatz into (\ref{eq:Poisson equation Gauss}) yields
\[
-\mathcal{L}g(x)=x\cdot A\big(2Cx+D\big)-\gamma\Tr_{p}C=x\cdot\bar{K}x+\bar{l}\cdot x-\Tr\bar{K},
\]
where 
\[
\Tr_{p}C=\sum_{i=n+1}^{2n}C_{ii}
\]
denotes the trace of the momentum component of $C$. Comparing different
powers of $x$, this leads to the conditions
\begin{subequations}
\begin{eqnarray}
AC+CA^{T} & =\bar{K},
\label{eq:Lyapunov equation}\\
AD & =\bar{l},
\label{eq:linear condition}\\
\gamma\Tr_{p}C & =\Tr\bar{K}.
\label{eq:trace condition}
\end{eqnarray}
\end{subequations}
Note that (\ref{eq:trace condition}) will be satisfied eventually
by existence and uniqueness of the solution to (\ref{eq:Poisson equation Gauss}).
Then, by the calculations in \cite{duncan2016variance}, the asymptotic variance is given by 
\begin{equation}
\sigma_{f}^{2}=2\Tr(C\bar{K})+D\cdot\bar{l}.
\label{eq:Gaussian asymvar}
\end{equation}

\begin{proof}[of Proposition \ref{thm: local quadratic observable}]. According to
	(\ref{eq:Gaussian asymvar}) and (\ref{eq:Lyapunov equation}), the
	asymptotic variance satisfies 
	\[
	\sigma_{f}^{2}=2\Tr(C\bar{K}),
	\]
	where the matrix $C$ solves 
	\begin{equation}
	AC+CA^{T}=\bar{K}\label{eq: lyap equation}
	\end{equation}
	and $A$ is given as in (\ref{eq:Amatrix}). We will use the notation
	\[
	C(\mu,\nu)=\left(\begin{array}{cc}
	C_{1}(\mu,\nu) & C_{2}(\mu.\nu)\\
	C_{2}^{T}(\mu.\nu) & C_{3}(\mu,\nu)
	\end{array}\right)
	\]
	and the abbreviations $C(0):=C(0,0)$, $C^{\mu}(0):=\partial_{\mu}C\vert_{\mu,\nu=0}$
	and $C^{\nu}(0):=\partial_{\nu}C\vert_{\mu,\nu=0}$.  Let us first determine $C(0)$, i.e. the solution to the equation
	\[
	\left(\begin{array}{cc}
	\boldsymbol{0} & I\\
	-I & \gamma I
	\end{array}\right)C(0)+C(0)\left(\begin{array}{cc}
	\boldsymbol{0} & I\\
	-I & \gamma I
	\end{array}\right)^{T}=\left(\begin{array}{cc}
	K & \boldsymbol{0}\\
	\boldsymbol{0} & \boldsymbol{0}
	\end{array}\right).
	\]
	This leads to the following system of equations,
	\begin{subequations}
	\begin{eqnarray}
	C_{2}(0)+C_{2}(0)^{T} & =K,
	\label{eq:C(0) 1}\\
	-C_{1}(0)+\gamma C_{2}(0)+C_{3}(0) & =\boldsymbol{0},
	\label{eq:C(0) 2}\\
	-C_{1}(0)+\gamma C_{2}(0)^{T}+C_{3}(0) & =\boldsymbol{0},
	\label{eq:C(0) 3}\\
	-C_{2}(0)-C_{2}(0)^{T}+2\gamma C_{3}(0) & =\boldsymbol{0}.\\ \label{eq:C(0) 4}
	\end{eqnarray}
	\end{subequations}
	Note that equations (\ref{eq:C(0) 2}) and (\ref{eq:C(0) 3}) are
	equivalent by taking the transpose. Plugging (\ref{eq:C(0) 1}) into
	(\ref{eq:C(0) 4}) yields 
	\begin{equation}
	C_{3}(0)=\frac{1}{2\gamma}K.\label{eq:C3(0) result}
	\end{equation}
	Adding (\ref{eq:C(0) 2}) and (\ref{eq:C(0) 3}), together with (\ref{eq:C(0) 1})
	and (\ref{eq:C3(0) result}) leads to 
	\[
	C_{1}(0)=\frac{1}{2\gamma}K+\frac{\gamma}{2}K.
	\]
	Solving (\ref{eq:C(0) 2}) we obtain, 
	\[
	C_{2}(0)=\frac{1}{2}K,
	\]
	so that
	\begin{equation}
	C(0)=\left(\begin{array}{cc}
	\frac{1}{2\gamma}K+\frac{\gamma}{2}K & \frac{1}{2}K\\
	\frac{1}{2}K & \frac{1}{2\gamma}K
	\end{array}\right).\label{eq:C(0) result}
	\end{equation}
	Taking the $\mu$-derivative of (\ref{eq: lyap equation}) and setting
	$\mu=\nu=0$ yields 
	\begin{equation}
	A^{\mu}(0)C(0)+A(0)C^{\mu}(0)+C^{\mu}(0)A(0)^{T}+C(0)A^{\mu}(0)^{T}=\boldsymbol{0}.\label{eq:mu lyap}
	\end{equation}
	Notice that 
	\begin{align*}
	A^{\mu}(0)C(0)+C(0)A^{\mu}(0)^{T}\\
	=\left(\begin{array}{cc}
	-J & \boldsymbol{0}\\
	\boldsymbol{0} & \boldsymbol{0}
	\end{array}\right)C(0)+C(0)\left(\begin{array}{cc}
	J & \boldsymbol{0}\\
	\boldsymbol{0} & \boldsymbol{0}
	\end{array}\right)\\
	=\left(\begin{array}{cc}
	\big(\frac{1}{2\gamma}+\frac{\gamma}{2}\big)[K,J] & -\frac{1}{2}JK\\
	\frac{1}{2}KJ & \boldsymbol{0}
	\end{array}\right).
	\end{align*}
	With computations similar to those in the derivation of (\ref{eq:C(0) result})
	(or by simple substitution), equation (\ref{eq:mu lyap}) can be solved
	by 
	\begin{equation}
	C^{\mu}(0)=\left(\begin{array}{cc}
	-\big(\frac{\gamma^{2}}{4}+\frac{1}{4\gamma^{2}}+\frac{1}{4}\big)[K,J] & \frac{1}{2\gamma}JK-\frac{\gamma}{4}[K,J]\\
	-\frac{1}{2\gamma}KJ-\frac{\gamma}{4}[K,J] & -\big(\frac{1}{4\gamma^{2}}+\frac{1}{4}\big)[K,J]
	\end{array}\right).\label{eq:C^mu}
	\end{equation}
	We employ a similar strategy to determine $C^{\nu}(0)$: Taking the
	$\nu$-derivative in equation (\ref{eq: lyap equation}), setting
	$\mu=\nu=0$ and inserting $C(0)$ and $A(0)$ as in (\ref{eq:C(0) result})
	and (\ref{eq:Amatrix}) leads to the equation
	\[
	\left(\begin{array}{cc}
	\boldsymbol{0} & I\\
	-I & \gamma I
	\end{array}\right)C^{\nu}(0)+C^{\nu}(0)\left(\begin{array}{cc}
	\boldsymbol{0} & I\\
	-I & \gamma I
	\end{array}\right)=\left(\begin{array}{cc}
	\boldsymbol{0} & -\frac{1}{2}KJ\\
	\frac{1}{2}JK & -\frac{1}{2\gamma}[K,J]
	\end{array}\right),
	\]
	which can be solved by 
	\begin{equation}
	C^{\nu}(0)=\left(\begin{array}{cc}
	\big(-\frac{1}{4\gamma^{2}}+\frac{1}{4}\big)[K,J] & \frac{1}{\gamma}\big(-\frac{1}{2}KJ+\frac{1}{4}[K,J]\big)\\
	\frac{1}{\gamma}\big(\frac{1}{2}KJ-\frac{1}{4}[K,J]\big) & -\frac{1}{4\gamma^{2}}[K,J]
	\end{array}\right).\label{eq:C^nu}
	\end{equation}
	Note that $\Tr (C\bar{K})=\Tr(C_{1}K)$, and so 
	\begin{alignat*}{1}
	\partial_{\mu}\Theta\vert_{\mu,\nu=0} & =2\Tr(C_{1}^{\mu}(0)K)=\\
	& =-\big(\frac{\gamma^{2}}{4}+\frac{1}{4\gamma^{2}}+\frac{1}{4}\big)\cdot\Tr([K,J]K)=0,
	\end{alignat*}
	since clearly $\Tr([K,J],K)=\Tr(KJK)-\Tr(JK^{2})=0$. In the same
	way it follows that 
	\[
	\partial_{\nu}\Theta\vert_{\mu,\nu=0}=0,
	\]
	proving (\ref{eq:gradTheta}). 
	\\\\
	Taking the second $\mu$-derivative of (\ref{eq: lyap equation})
	and setting $\mu=\nu=0$ yields
	\[
	2A^{\mu}(0)C^{\mu}(0)+A(0)C^{\mu\mu}(0)+C^{\mu\mu}(0)A(0)^{T}+2C^{\mu}(0)A^{\mu}(0)^{T}=\boldsymbol{0},
	\]
	employing the notation $C^{\mu\mu}(0)=\partial_{\mu}^{2}C\vert_{\mu,\nu=0}$
	and noticing that $\partial_{\mu}^{2}A=0$. Using (\ref{eq:C^mu})
	we calculate
	\[
	A^{\mu}(0)C^{\mu}(0)+C^{\mu}(0)A^{\mu}(0)^{T}=\left(\begin{array}{cc}
	\big(\frac{\gamma^{2}}{4}+\frac{1}{4\gamma^{2}}+\frac{1}{4}\big)[J,[K,J]] & -\frac{1}{2\gamma}J^{2}K+\frac{\gamma}{4}J[K,J]\\
	-\frac{1}{2\gamma}KJ^{2}-\frac{\gamma}{4}[K,J]J & \boldsymbol{0}
	\end{array}\right).
	\]
	As before, we make the ansatz
	\[
	C^{\mu\mu}(0)=\left(\begin{array}{cc}
	C_{1}^{\mu\mu}(0) & C_{2}^{\mu\mu}(0)\\
	\left(C_{2}^{\mu\mu}(0)\right)^T & C_{3}^{\mu\mu}(0)
	\end{array}\right),
	\]
	leading to the equations
	\begin{subequations}
	\begin{eqnarray}
	C_{2}^{\mu\mu}(0)+C_{2}^{\mu\mu}(0)^{T} & = &-\big(\frac{\gamma^{2}}{4}+\frac{1}{4\gamma^{2}}+\frac{1}{4}\big)[J,[K,J]]\label{eq:C''1}\\
	-C_{1}^{\mu\mu}(0)+\gamma C_{2}^{\mu\mu}(0)+C_{3}^{\mu}(0) & =&\frac{1}{\gamma}J^{2}K-\frac{\gamma}{2}J[K,J]\label{eq:C''2}\\
	-C_{1}^{\mu\mu}(0)+\gamma C_{2}^{\mu\mu}(0)^{T}+C_{3}^{\mu}(0) & =&\frac{1}{\gamma}KJ^{2}+\frac{\gamma}{2}[K,J]J\label{eq:C''3}\\
	-C_{2}^{\mu\mu}(0)-C_{2}^{\mu\mu}(0)^{T}+2\gamma C_{3}^{\mu\mu}(0) & =&\boldsymbol{0}.
	\label{eq:C''4}
	\end{eqnarray}
	\end{subequations}
	Again, (\ref{eq:C''2}) and (\ref{eq:C''3}) are equivalent by taking
	the transpose. Plugging (\ref{eq:C''1}) into (\ref{eq:C''4}) and
	combing with (\ref{eq:C''2}) or (\ref{eq:C''3}) gives
	\[
	C_{1}^{\mu\mu}(0)=\big(\frac{\gamma}{4}+\frac{1}{4\gamma^{3}}+\frac{\gamma^{3}}{4}\big)(2JKJ-J^{2}K-KJ^{2})-\frac{1}{\gamma}JKJ.
	\]
	Now 
	\[
	\partial_{\mu}^{2}\Theta\vert_{\mu,\nu=0}=2\Tr(C_{1}^{\mu\mu}(0)K)=-(\gamma+\frac{1}{\gamma^{3}}+\gamma^{3})\big(\Tr(JKJK)-\Tr(J^{2}K^{2})\big)-\frac{2}{\gamma}\Tr(JKJK)
	\]
	gives the first part of (\ref{eq:HessTheta}). We proceed in the same
	way to determine $C_{1}^{\nu\nu}(0)$. Analogously, we get 
	\[
	A^{\nu}(0)C^{\nu}(0)+C^{\nu}(0)A^{\nu}(0)^{T}=\left(\begin{array}{cc}
	\boldsymbol{0} & \frac{1}{\gamma}(KJ^{2}-\frac{1}{2}[K,J]J)\\
	\frac{1}{\gamma}(JKJ-\frac{1}{2}J[K,J]) & \frac{1}{2\gamma^{2}}([K,J]J-J[K,J])
	\end{array}\right).
	\]
	Solving the resulting linear matrix system (similar to (\ref{eq:C''1})-(\ref{eq:C''4}))
	results in
	\[
	C_{1}^{\nu\nu}(0)=\big(\frac{1}{4\gamma^{3}}-\frac{1}{4\gamma}\big)(KJ^{2}+J^{2}K)-\big(\frac{1}{2\gamma^{3}}+\frac{1}{2\gamma}\big)JKJ,
	\]
	leading to 
	\[
	\partial_{\nu}^{2}\Theta\vert_{\mu,\nu=0}=2\Tr(C_{1}^{\nu\nu}(0)K)=\big(\frac{1}{\gamma^{3}}-\frac{1}{2\gamma}\big)\Tr(J^{2}K^{2})\big)-\big(\frac{1}{2\gamma^{3}}+\frac{1}{2\gamma}\big)\Tr(JKJK).
	\]
	To compute the cross term $C_{1}^{\mu\nu}(0)$ we take the mixed derivative
	$\partial_{\mu\nu}^{2}$ of (\ref{eq: lyap equation}) and set $\mu=\nu=0$
	to arrive at 
	\[
	A^{\mu}(0)C^{\nu}(0)+A^{\nu}(0)C^{\mu}(0)+A(0)C^{\mu\nu}(0)+C^{\mu\nu}(0)A(0)^{T}+C^{\mu}(0)A^{\nu}(0)^{T}+C^{\nu}(0)A^{\mu}(0)^{T}=\boldsymbol{0}.
	\]
	Using $\eqref{eq:C^mu}$ and (\ref{eq:C^nu}) we see that 
	\begin{multline*}
	A^{\mu}(0)C^{\nu}(0)+A^{\nu}(0)C^{\mu}(0)+C^{\mu}(0)A^{\nu}(0)^{T}+C^{\nu}(0)A^{\mu}(0)^{T}\\
	=\left(\begin{array}{cc}
	\big(\frac{1}{4\gamma^{2}}-\frac{1}{4}\big)[J,[K,J]] & \frac{1}{\gamma}JKJ-\frac{1}{4\gamma}J[K,J]-\frac{\gamma}{4}[K,J]J\\
	\frac{1}{2\gamma}JKJ+\frac{1}{2\gamma}KJ^{2}+\frac{\gamma}{4}J[K,J]-\frac{1}{4\gamma}[K,J]J & \big(\frac{1}{4\gamma^{2}}+\frac{1}{4}\big)[J,[K,J]]
	\end{array}\right).
	\end{multline*}
	The ensuing linear matrix system yields the solution
	\[
	C_{1}^{\mu\nu}(0)=\big(-\frac{1}{4\gamma^{3}}+\frac{\gamma}{4}-\frac{1}{4\gamma}\big)[J,[K,J]]+\frac{1}{\gamma}JKJ,
	\]
	leading to 
	\begin{equation}
	\partial_{\mu\nu}^{2}\Theta\vert_{\mu,\nu=0}=2\Tr(C_{1}^{\mu\nu}(0)K)=\big(\frac{1}{\gamma^{3}}+\frac{1}{\gamma}-\gamma\big)\Tr(J^{2}K^{2})+\big(-\frac{1}{\gamma^{3}}+\frac{1}{\gamma}+\gamma\big)\Tr(JKJK).
	\end{equation}
	This completes the proof.
	\qed 
    \end{proof}
\begin{proof}
	[Proof of Proposition \ref{thm:linear_full_J}] By (\ref{eq:linear condition})
	and (\ref{eq:Gaussian asymvar}) the function $\Theta$ satisfies
	\[
	\Theta(\mu,\nu)=\bar{l}\cdot A^{-1}\bar{l}.
	\]
	Recall the following formula for blockwise inversion of matrices using the Schur complement:
	\begin{equation}
	\left(\begin{array}{cc}
	U & V\\
	W & X
	\end{array}\right)^{-1}=\left(\begin{array}{cc}
	(U-VX^{-1}W)^{-1} & \ldots\\
	\ldots & \ldots
	\end{array}\right),\label{eq: blockwise inversion}
	\end{equation}
	provided that $X$ and $U-VX^{-1}W$ are invertible. Using this, we obtain 
	\[
	\Theta(\mu,\nu)=l\cdot\big(-\mu J+(\gamma-\nu J)^{-1}\big)l.
	\]
	Taking derivatives, setting $\mu=\nu=0$ and using the fact that $J^{T}=-J$
	leads to the desired result.
	\qed
\end{proof}

\begin{lemma}
	\label{lem:basic_inequalities}
	The following holds: 
	\begin{enumerate}[label=(\alph*)]
		\item \label{it:gaussian_lem1}$\gamma-\frac{4}{\gamma^{3}}-\gamma^{3}-\frac{1}{\gamma}<0$
		for $\gamma\in(0,\infty)$. 
		\item \label{it:gaussian_lem2} Let $J=-J^{T}$ and $K=K^{T}$. Then $\Tr(JKJK)-\Tr(J^{2}K^{2})\ge0$.
		Furthermore, equality holds if and only if $[J,K]=0.$ 
	\end{enumerate}
\end{lemma}
\begin{proof}
	To show \ref{it:gaussian_lem1} we note that $\gamma-\frac{4}{\gamma^{3}}-\gamma^{3}-\frac{1}{\gamma}<\gamma-\frac{4}{\gamma^{3}}-\gamma^{3}=\gamma(1 - \frac{4}{\gamma^4}-\gamma^2)$.  The function $f(\gamma):=1 - \frac{4}{\gamma^4}-\gamma^2$
	has a unique global maximum on $(0,\infty)$ at $\gamma_{min}=8^{1/6}$
	with $f(\gamma_{min})=-2$, so the result follows.
	\\\\
	For \ref{it:gaussian_lem2} we note that $[J,K]^{T}=[J,K],$ and that $[J,K]^{2}$ is symmetric
	and nonnegative definite. We can write 
	\[
	\Tr([J,K]^{2})=\sum_{i}\lambda_{i}^{2},
	\]
	with $\lambda_{i}$ denoting the (real) eigenvalues of $[J,K]$. From
	this it follows that $\Tr([J,K]^{2})\ge0$ with equality if and only
	if $[J,K]=0$. Now expand 
	\[
	\Tr([J,K]^{2})=2\big(\Tr(JKJK)-\Tr(J^{2}K^{2}),
	\]
	which implies the advertised claim. \qed
\end{proof}


\section{Orthogonal Transformation of Tracefree Symmetric Matrices into a Matrix with Zeros on the Diagonal}
\label{tracefree}

Given a symmetric matrix $K\in\mathbb{R}_{sym}^{d\times d}$ with
$\Tr K=0$, we seek to find an orthogonal matrix $U\in O(\mathbb{R}^{d})$
such that $UKU^{T}$ has zeros on the diagonal. This is a crucial
step in Algorithms \ref{alg:optimal J} and \ref{alg:optimal J general}
and has been addressed in various places in the literature (see for
instance \cite{alg_zero_diag} or \cite{Bhatia1997}, Chapter 2,
Section 2). For the convenience of the reader, in the following we
summarize an algorithm very similar to the one in \cite{alg_zero_diag}.
\\\\
Since $K$ is symmetric, there exists an orthogonal matrix $U_{0}\in O(\mathbb{R}^{d})$
such that $U_{0}KU_{0}^{T}=\diag(\lambda_{1},\ldots,\lambda_{d})$.
Now the algorithm proceeds iteratively, orthogonally transforming
this matrix into one with the first diagonal entry vanishing, then
the first two diagonal entries vanishing, etc, until after $d$ steps
we are left with a matrix with zeros on the diagonal. Starting with
$\lambda_{1}$, assume that $\lambda_{1}\neq0$ (otherwise proceed
with $\lambda_{2}$). Since $\Tr(K)=\Tr(U_{0}KU_{0}^{T})=\sum\lambda_{i}=0$,
there exists $\lambda_{j}$, $j\in\{2,\ldots,d\}$ such that $\lambda_{1}\lambda_{j}<0$
(i.e. $\lambda_{1}$ and $\lambda_{j}$ have opposing signs). We now
apply a rotation in the $1j$-plane to transform the first diagonal
entry into zero. More specifically, let 
\[
U_{1}=\begin{blockarray}{ccccccccc}
 ~ & ~ & ~ & ~ & j & ~ & ~ & ~ & ~\\
\begin{block}{(cccccccc)c}
\cos\alpha & 0 & \ldots & 0 & -\sin\alpha & 0 & \ldots & 0 & ~\\
0 & 1 & ~ & ~ & 0 & ~ & ~ & \vdots & ~\\
\vdots & ~ & \ddots & ~ & ~ & ~ & ~ & ~ & ~\\
0 & ~ & ~ & 1 & 0 & ~ & ~ & ~ & ~\\
\sin\alpha & 0 & ~ & 0 & \cos\alpha & 0 & ~ & ~ & j\\
0 & ~ & ~ & ~ & 0 & 1 & ~ & \vdots & ~\\
\vdots & ~ & ~ & ~ & ~ & ~ & \ddots & 0 & ~\\
0 & 0 & \hdots & ~ & ~ & \hdots & 0 & 1 & ~ \\
\end{block}\end{blockarray} \in O(\mathbb{R}^{d})
\]
with $\alpha=\arctan\sqrt{-\frac{\lambda_{1}}{\lambda_{j}}}.$ We
then have $(U_{1}U_{0}KU_{0}^{T}U_{1}^{T})_{11}=0$. Now the same
procedure can be applied to the second diagonal entry $\text{\ensuremath{\lambda}}_{2}$,
leading to the matrix $U_{2}U_{1}U_{0}KU_{0}^{T}U_{1}^{T}U_{2}^{T}$
with 
\[
(U_{2}U_{1}U_{0}KU_{0}^{T}U_{1}^{T}U_{2}^{T})_{11}=(U_{2}U_{1}U_{0}KU_{0}^{T}U_{1}^{T}U_{2}^{T})_{22}=0
\]
Iterating this process, we obtain that $U_{d}\ldots U_{1}U_{0}KU_{0}^{T}U_{1}^{T}\ldots U_{d}^{T}$
has zeros on the diagonal, so $U_{d}\ldots U_{1}U_{0}\in O(\mathbb{R}^{d})$
is the required orthogonal transformation. 


\bibliographystyle{alpha}
\documentclass[conference]{IEEEtran}

\usepackage{amsmath}
\usepackage{amsfonts}
\usepackage{amssymb}
\usepackage{amsmath,amsthm}
\usepackage{color}
\usepackage{lmodern}
\usepackage{pgfplots}
\usepackage{algorithm}
\usepackage[noend]{algpseudocode}
\usepackage{graphicx}
\usepackage[export]{adjustbox}

\newtheorem{mydef}{Definition}
\newtheorem{myprop}{Proposition}

%reduce space below images
%\setlength{\dbltextfloatsep}{.5cm}

\newenvironment{example}
{\let\oldqedsymbol=\qedsymbol
	\renewcommand{\qedsymbol}{$\centerdot$}
	\begin{proof}[\bfseries\upshape Example]}
	{\end{proof}
	\renewcommand{\qedsymbol}{\oldqedsymbol}}


\begin{document}

\title{Improving the coding speed of erasure codes with polynomial ring transforms}

\author{
\IEEEauthorblockN{Jonathan Detchart,     J\'er\^ome Lacan}
\IEEEauthorblockA{\\Universit\'e de Toulouse ISAE-SUPAERO, Toulouse, France\\\{jonathan.detchart, jerome.lacan\}@isae-supaero.fr}
} % end 

\maketitle

\begin{abstract}
Erasure codes are widely used in today's storage systems to cope with failures. Most of them use the finite field arithmetic. In this paper, we propose an implementation and a coding speed evaluation of an original method called PYRIT (PolYnomial RIng Transform) to perform operations between elements of a finite field into a bigger ring by using fast transforms between these two structures. Working in such a ring is much easier than working in a finite field. Firstly, it reduces the coding complexity by design. Secondly, it allows simple but efficient \texttt{xor}-based implementations by unrolling the operations thanks to the properties of the ring structure. We evaluate this proposition for Maximum Distance Separable erasure codes and we show that our method has better performances than common codes. Compared to the best known implementations, the coding speeds are increased by a factor varying from $1.5$ to $2$.
\end{abstract}

\section{Introduction}
\label{sec:intro}
In today's storage systems, erasure codes are widely used and provide reliability to failures. They are used by RAID solutions \cite{Anvin:2009}, Cloud storage \cite{Huang:2012:ECW:2342821.2342823}, or as an elementary building block in large scale coding systems \cite{xorbas}. They replace replication by reducing the amount of extra storage needed to tolerate the same amount of erasures (\cite{Weatherspoon2002}). 
But this kind of technique is limited by the complexity of the arithmetic used. Most of the complexity of erasure codes consists in making linear combinations over a finite field. To speedup these linear combinations, several solutions have been presented:

Recently, the use of SIMD instructions have been proposed to drastically increase the speed of erasure codes, particularly by optimizing the multiplication of a large region of elements of a finite field by a constant element: \cite{Li:2008:PNC:1491269.1493312, 4770564, pgm:13:sfg}. In \cite{xor-luby}, the authors consider the elements of a finite field of characteristic 2 as a binary matrix where the entries represent a \texttt{xor} between two parts of data. In \cite{isit}, we extended an idea proposed in \cite{ITOH1989} by defining other transforms to speedup the coding process in the context of erasure codes. The contribution of this paper is essentially theoretical. 

In this paper, we propose an extension to other finite fields and a complete performance analysis of the method introduced in \cite{isit}, called PYRIT (PolYnomial RIng Transform), replacing the multiplication in a finite field by the multiplication in a ring by using transforms between particular finite fields and polynomial rings. We show that an element of a field can have several representations in a bigger ring, and that the choice of this element can have an impact on the performances.
We also show that using a ring to perform multiplications allows to reduce the complexity of the coding process. It also allows some optimizations in the implementation which are not possible when using a classic \texttt{xor} based implementation. 
We compare our implementation with other implementations and show that we are faster than the best known implementations for both single and multithreading.

\section{Correspondence between field and ring}
\label{sec:algebra}

\subsection{Algebraic context}

Let us recall some known properties about polynomial rings and fields:

Let $p(x)$ be an irreducible polynomial of degree $w$. The field  $\mathbb{F}_{2}[x]/(p(x))$ is denoted by $\mathbb{F}_{2^w}$. The polynomial $x^n+1$ is not irreducible, and thus $\mathbb{F}_{2}[x]/(x^n+1)$ is a ring. Let us denote it by $R_{2,n}$.

Let us now consider the factorization of $x^n+1$ into irreducible polynomials: $x^n+1=p_1^{u_1}(x)p_2^{u_2}(x)\ldots p_l^{u_l}(x)$.
If $n$ is odd, it can be shown that $u_1=u_2=\ldots=u_l=1$ (see \cite{poli1992error}). In this document, we only consider this case.

The proofs of the following propositions can be found in \cite{poli1992error} or \cite{MacWilliams19711}.

\begin{myprop}
	\label{prop:directSumDecomposition}
	The ring $R_{2,n}$ is equal to the direct sum of the principal ideals of  $\mathbb{F}_{2}[x]$ $A_i=( (x^n+1)/p_i(x) )$ for $i=1,\ldots,l$. 
	Each ideal contains an unique idempotent $\theta_i(x)$. 	
\end{myprop}

\begin{myprop}
	\label{prop:isomorphism}	
	For each $i=1,\ldots,l$, $A_i$ is isomorphic to the finite field $B_i=\mathbb{F}_{2}[x]/(p_i(x))$. The isomorphism is :
	\begin{equation}
	\phi_i  :  \begin{array}{ccc}
	B_i & \rightarrow & A_i \\
	b(x) & \rightarrow & \bar{b}(x)=b(x)\theta_i(x) \
	\end{array}
	\end{equation}  	
	and the inverse isomorphism is :
	\begin{equation}
	\phi_i^{-1}  :  \begin{array}{ccc}
	A_i & \rightarrow & B_i \\
	a(x) & \rightarrow & a(x)\textrm{ mod }p_i(x)\\
	\end{array}
	\end{equation}  	
\end{myprop}

We will use this kind of morphism to transform each finite field element from the source vectors and  the generator matrix of the erasure code into ring elements to perform xor operations and apply reverse transforms on the generated vectors.

\subsection{The sparse transform for generator matrices}
\label{sec:sparse}
One of the main challenges in the erasure code construction is the choice of the generator matrix. Since we here only consider systematic codes, the parity part of the matrix must verify two main properties:
\begin{itemize}
	\item it must be as sparse as possible. Indeed, the number of \texttt{xor} done on the data is defined by the number of ones in the binary form of the generator matrix,
	\item any square submatrix must be invertible to verify the MDS property.
\end{itemize} 

For the first point, we use an interesting property of the correspondences between the ring and the field. The number of ring elements is $2^n$, which is greater than the number of the elements of the field: $2^w$.  

In fact, the important point behind the use of a ring is that one of its ideals, $A_1$, is isomorphic to the field. A naive approach to multiply two elements of the field $u(x)$ and $v(x)$ would consist in sending them in $A_1$ by applying the isomorphism $\phi_1$ to obtain $\bar{u}(x)$ and $\bar{v}(x)$. The second step would consist in multiplying  $\bar{u}(x)$ and $\bar{v}(x)$ and apply the inverse isomorphism $\phi_1^{-1}$ on the result.

However, we can observe that the structure of the ring (which is decomposed as a direct sum of ideals) allows to consider that the operations on the ring can be decomposed into "parallel" and independent operations in each ideal. It follows that in the function from the field to the ring, we can add other ring elements which belong to the other ideals. These do not interfere with the operations in $A_1$ which are the ones important for the field. 

To be more precise, let us define a function $\xi$ from the field to the ring which is such that :
for any $u(x)$ in the field, $\xi(u(x))=\bar{u}(x)+\hat{u}(x)$, where $\bar{u}(x)=\phi_1(u(x)) \in A_1$ and $\hat{u}(x)$ is an element of the ring which does not have a component in $A_1$ (see Proposition \ref{prop:directSumDecomposition}). 


Then, to multiply the field elements $u(x)$ and $v(x)$, we can compute the product $\xi(u(x)).\xi(v(x))$ which is equal to $\bar{u}(x).\bar{v}(x)+\hat{u}(x).\hat{v}(x)$. The application $\phi^{-1}$ to this result removes the part outside $A_1$ and then outputs $u(x).v(x)$ in the field. This means that, to perform computations with element $u(x)$, we can use any element of the form $ \bar{u}(x)+\hat{u}(x)$ in the ring, where $\bar{u}(x)=\phi_1(u(x))$ and $\hat{u}(x)$ is an element of the ring which does not have a component in $A_1$. The number of ring elements which do not have a component in $A_1$ is equal to $2^{n-w}$. 

The interest of this property is that the complexity of the multiplication in the ring is not the same for all the elements. Indeed, for the generator matrix in the binary form, the complexity depends on the number of ones. So this transfom allows to choose the element in the ring having the lowest weight among the ring elements corresponding to a given field element. We call this operation the \texttt{sparse transform}.

\subsection{Pyrit using AOP}
\label{sec:aop}
All One Polynomials (AOP) of degree $w$ are equal to $x^w+x^{w-1}+x^{w-2}+\ldots+x+1$. The AOP of degree $w$ is irreducible over $\mathbb{F}_{2}$  if and only if $w+1$ is a prime and $w$ generates $\mathbb{F}^*_{w+1}$, where $\mathbb{F}^*_{w+1}$ is the multiplicative group in $\mathbb{F}_{w+1}$ \cite{Wah1984}. The values $w+1$, such that $w$ is an irreducible AOP is the sequence A001122 in \cite{A001122}.

The use of AOPs for fast operations in finite fields was studied by \cite{ITOH1989} and then by \cite{Silverman1999} in the context of hardware implementations of large finite field operations. 

Irreducible AOP of degree $w$ appears in the factorization of $x^{w+1}+1$ which is equal to $( x^w+x^{w-1}+x^{w-2}+\ldots+x+1).(x+1)$. Some of the previous propositions can be specified for these polynomials. 

Let $p(x)$ be the AOP of degree $w$. Then:
\begin{itemize}
	\item  the ring $R_{2,w+1}$ is equal to the direct sum of the principal ideals of  $\mathbb{F}_{2}[x]$ $A_1$ generated by $x+1$  and $A_2$ generated by $p(x)$. 
	The idempotent of $A_1$  and $A_2$  are respectively $p(x)+1$ and $p(x)$.
	\item  The isomorphism between $B_1=\mathbb{F}_{2}[x]/(p(x))=\mathbb{F}_{2^w}$ and $A_1$ is equal to 
	\begin{equation}
	\phi_1  :  \begin{array}{ccc}
	B_1 & \rightarrow & A_1 \\
	b(x) & \rightarrow & \bar{b}(x)=b(x)(p(x)+1) \
	\end{array}
	\end{equation}  	
	and the inverse isomorphism is:
	\begin{equation}
	\phi_i^{-1}  :  \begin{array}{ccc}
	A_i & \rightarrow & B_i \\
	\bar{b}(x) & \rightarrow & \bar{b}(x)\textrm{ mod }p(x)\\
	\end{array}
	\end{equation} 
\end{itemize}

An interesting property of $A_1$ is that it is equal to the set of polynomials of even weight. Indeed, since an element of $A_1$ is a multiple of $x+1$ (modulo $x^{w+1}+1$), it contains an even number of monomials. And since the number of elements of $A_1$, which is equal to $2^w$, corresponds to the number of polynomials of even weight, $A_1$ is equal to the set of even weight polynomials.  


For example, by considering the finite field $\mathbb{F}_{2^4}$ defined by the irreducible All-One Polynomial $p(x) = 1 + x + x^2 + x^3 + x^4$, we can define $R_{2,5} = \mathbb{F}_{2}[x]/(x^5 + 1)$ as the quotient ring of polynomials of the polynomial $\mathbb{F}_2[x]$ quotiented by the ideal generated by the polynomial $x^5 -1$.

The polynomial $x^5+1$ is the product of the irreducible polynomials $p_1(x)=x^4 + x^3 + x^2 + x + 1$ and $p_2(x)=x+1$. The ring $R_{2,5}$ is the direct sum of the ideal $A_1$ generated by the polynomial $(x^5 + 1) /p_1(x) = p_2(x)$ and the ideal $A_2$ generated by the $(x^5 + 1) / p_2(x) = p_1(x)$. In others words, any element $u(x)$ of $R_{2,5}$ can be written in a unique way as the sum of two components $u_1(x)+u_2(x)$, where $u_1(x)\in A_1$ and $u_2(x)\in A_2$.
	It can be verified that $A_1$ (resp. $A_2$) contains one and only one idempotent $\theta_1(x)=x^4 + x^3 + x^2 + x$ (resp. $\theta_2(x)=x^4 + x^3 + x^2 + x+1$). A construction of this idempotent is given in \cite[Chap. 8, Theorem 6]{MWSl77}.

Since $\mathbb{F}_q[x]/(p_i(x))$ is isomorphic to $B_i = \mathbb{F}_{q^{w_i}}$ where $p_i(x)$ is of degree $w_i$, $R_{2,5}$ is isomorphic to the following cartesian product $R_{2,5} \simeq B_1 \otimes B_2$ with $B_1=\mathbb{F}_2[x]/(p(x))=\mathbb{F}_{2^4}$ and $B_2=\mathbb{F}_2[x]/(x-1)=\mathbb{F}_2$. The image by the isomorphism of  $b(x)\in B_1$ into $A_1$  is $\bar{b}(x)=\phi_1(b(x))= b(x)\theta_1(x) $. On the other side, the image of the element  $\bar{b}(x)$ of $A_1$ by the inverse isomorphism is equal to $b(x)=\phi_i^{-1}(\bar{b}(x))=\bar{b}(x)\textrm{ mod }p_1(x)$. 

	In $R_{2,5}$, let us consider $a(x)=1+x^2$ and $b(x)=x+x^4$ and their respective matrix and vector representations:
	\begin{displaymath}
	a(x) \longrightarrow \includegraphics[valign=m]{figures/mat1x3.eps}, \ b(x)  \longrightarrow    \includegraphics[valign=m]{figures/vectorxx4.eps}
	\end{displaymath}
	where the filled squares represent the ones and the empty squares represent the zeros.
	
	To compute the multiplication, we just have to perform the matrix vector multiplication:
	\begin{displaymath}
	a(x).b(x) \longrightarrow \includegraphics[valign=m]{figures/mat1x3.eps} * \includegraphics[valign=m]{figures/vectorxx4.eps} = \includegraphics[valign=m]{figures/vectorx3x4.eps} \longrightarrow  c(x)= x^3+x^4  
	\end{displaymath}
	We can verify that we obtain the same result with the multiplication of polynomials in the ring:
	\begin{eqnarray}
	a(x).b(x)& = & (1+x^2).(x+x^4)\textrm{ mod }(x^5+1) \nonumber\\
%	& = & x+x^3+x^4+x^6\textrm{ mod }(x^5+1) \nonumber\\
%	& = & x+x^3+x^4+x \nonumber\\
	& = & x^3+x^4 \nonumber
	\end{eqnarray}




	The field $\mathbb{F}_{2^4}=\mathbb{F}_{2}[x]/(x^4+x^3+x^2+x+1)$ is isomorphic to the ideal $A_1$ of the ring $R_{2,5}$ generated by $(x+1)$. Let us consider the element $u(x)=x^2$ of the field. According to Proposition \ref{prop:isomorphism}, its image $\bar{u}(x)$ in $A_1$ is equal to $u(x)\theta_1(x)=x^2.(x+x^2+x^3+x^4)=1+x+x^3+x^4$.
	According to the previous paragraph, to perform multiplications with this element in the ring, we can use any element of the form $\bar{u}(x)+r(x)$ where $r(x)$ does not have a component in $A_1$. The $2^{n-w}=2$ elements which have this property are $0$ and $p(x)=1+x+x^2+x^3+x^4$. So we can consider the binary matrices associated to $\bar{u}(x)$ and $\bar{u}(x)+p(x)$:
	\begin{displaymath}
	\bar{u}(x) \longrightarrow \includegraphics[valign=m]{figures/matrix1xx2x3x4.eps} \ \ \   		
	\bar{u}(x)+p(x) \longrightarrow \includegraphics[valign=m]{figures/matrixx2.eps}   
	\end{displaymath}
	We can observe that the matrix associated to  $\bar{u}(x)+p(x)$ is more sparse than the one associated to $\bar{u}(x)$. So this matrix is chosen to perform the multiplication corresponding to $u(w)$ in the ring.

The function which makes the correspondence between the elements of the field and the sparsest matrices among their corresponding ones is denoted by $\phi_S$. You can see an example of the binary representation for a generator matrix in figure 1. Note that in the example, the elements of the ring are represented by a 5x5 binary matrix. However, depending on the transforms applied to the data, some operations are useless, and we can remove the corresponding row or column. So the elements are represented by 4x5 or 5x4 matrices (see \ref{parity_transform} and \ref{emb_transform}).

\begin{figure}
\label{fig:gen_mat}
\begin{center}
\includegraphics[width=.8\linewidth]{transform.eps}
\end{center}
\caption{binary generator matrices in a field and a ring}
\end{figure}

Now, we have to define the function that sends the data vector from the field into the ring. Even if the function  $\phi_S$ can be used, it is not optimal because the \texttt{xor}-based representation handles the data by blocks and not sequentially. Thus, it is not efficient to access the binary representation of each field element in order to determine the sparsest corresponding ring element. 

\subsubsection{The parity transform}
\label{parity_transform}
The first function we propose is to compute simple parity bits on the data blocks. In this case, the inverse operation just consists in removing the parity bits from the results of the matrix vector multiplication. As we explained in \cite{isit}, this transform is not an isomorphism, but just a bijection. That means operations in the field and in the ring are not compatible, and both encoder and decoder must use the parity transform to perform coding operations.

For an irreducible AOP of degree $w$, the usual isomorphism that sends the finite field elements into the ideal $A_1$ of the ring $R_{2,w+1}$ consists in multiplying the polynomial by the idempotent. However, we showed that $A_1$ is the set of polynomials of even weight. It follows that the function $\phi_P$ which adds a single parity bit to the vector corresponding to the finite field element (in order to have an even weight) can be used to make the correspondence between the field and the ideal. In the following equations, suppose $b(x) = a+b.x+c.x^2+d.x^3$ as an element of $\mathbb{F}_{2^4}$.

	\begin{displaymath}
		\phi_P : b(x) \rightarrow (a+b+c+d). x^4 + b(x)
	\end{displaymath}
For the inverse function, the vector resulting from the matrix vector multiplication contains elements of the ideal because the data vector belongs to the ideal and the matrix elements, obtained with $\phi_S$, belongs to the ring. So the finite field elements can be obtained by just removing the parity bit of each ring elements. We call this function $\phi_P^{-1}$. 

	\begin{displaymath}
		\phi_P^{-1} : e.x^4 + b(x) \rightarrow b(x)
	\end{displaymath}
To summarize, the $\phi_P$ function adds a single parity bit to the finite field elements and the inverse function removes it from the ring elements. These two functions can be very efficiently performed on the \texttt{xor}-based representation of the data. We can note that $\phi_P^{-1}$ is not an isomorphism. This implies that both the encoder and the decoder must use $\phi_P$ and $\phi_P^{-1}$.

\subsubsection{The embedding transform}
\label{emb_transform}

The second function we propose is a simple embedding, denoted by $\phi_E$ from the field in the ring. In other words, the polynomial corresponding to the finite field element is simply "padded" with $n-w$ zeros and considered as an element of the ring. 

	\begin{displaymath}
		\phi_E : b(x) \rightarrow b(x) + 0.x^4
	\end{displaymath}

For the inverse function, we use the the traditional inverse isomorphism $\phi^{-1}$ presented in Proposition \ref{prop:isomorphism} which corresponds to the computation modulo $p(x)$ where $p(x)$ is the irreducible polynomial. Note that these functions correspond to the ones proposed by \cite{ITOH1989}. According to the embedding function, the elements of the field are just padded by one $0$ and considered as elements of the ring. 

The inverse function is the computation of the remainder modulo $p(x)$. Since $p(x)$ is AOP, this operation consists in adding the last bit (the coefficient of the monomial of degree $w$) to all the other coefficients. 

	\begin{displaymath}
		\phi_E^{-1} : e.x^4 + b(x) \rightarrow (a+e)+(b+e).x+(c+e).x^2+(d+e).x^3
	\end{displaymath}



\subsection{Pyrit using ESP}
\label{sec:esp}

Similar approach can be used with $\mathbb{F}_{2^6}$. First, let's recall the ESP definition.
\begin{mydef}[\cite{ITOH1989}]
	
	\label{def:esp}	
	A polynomial $g(x)=x^{sr}+x^{s(r-1)}+x^{s(r-2)}+\ldots + x^s + 1=p(x^s)$, where $p(x)$ is an AOP of degree $r$, is called s-equally spaced polynomial ($s$-ESP) of degree $sr$.
	
\end{mydef}

According to \cite[Theorem 3]{ITOH1989}, $g(x)$ is irreducible if and only if $p(x)$ is irreducible and for some integer $t$, $s=(r+1)^{t-1}$ and $2^{r(r+1)^{t-2}}\not=1$ mod $(r+1)^t$.  

The first values of the pair $(r,s)$ for which the ESP is irreducible are $(2,3), (2,9), (4,5), (2,27),\ldots$ 

The ESP $g(x)=x^{sr}+x^{s(r-1)}+\ldots + x^s + 1$ divides the polynomials $x^{s.(r+1)}+ 1$ because $x^{s.(r+1)} + 1 = g(x).(x^s+1)$. Thus, according to Proposition \ref{prop:directSumDecomposition}, if the ESP $g(x)$ is irreducible, the field $\mathbb{F}_{2}[x]/(g(x))=\mathbb{F}_{2^{r.s}}$ is isomorphic to the ideal $A_1$ generated by $ (x^{s.(r+1)}-1)/g(x) = x^s+1$. 

It can be shown that the idempotent $\theta_1(x)$ of $A_1$ is equal to $g(x)+1$. 
% proof : this can be shown by showing that $g(x)^2=g(x)$, thus $g(x)$ is an idempotent and thus $g(x)+1$ also. Since $(g(x)+1)g(x)=g(x)+g(x)=0$ then $g(x)+1$ is the idempotent of $A$
Thus, the idempotent between the field $B=\mathbb{F}_{2}[x]/(g(x))=\mathbb{F}_{2^w}$ and the ideal $A_1$ are the following: 
\begin{equation}
\phi  :  \begin{array}{ccc}
B_1 & \rightarrow & A_1 \\
b(x) & \rightarrow & \bar{b}(x)=b(x)(g(x)+1) 
\end{array}
\end{equation}  	

and the inverse isomorphism is:
\begin{equation}
\phi^{-1}  :  \begin{array}{ccc}
A_1 & \rightarrow & B_1 \\
\bar{b}(x) & \rightarrow & \bar{b}(x)\textrm{ mod }g(x)\\
\end{array}
\end{equation} 

Like AOPs, ideals associated to ESP have an interesting parity property. Indeed, any element is a multiple of the generator polynomial $x^s+1$. So the element $a(x)$ is equal to $u(x).(x^s+1)=(\sum_{i=0}^{{sr}-1}u_i x^i).(x^s+1)$. $u(x)$ can also be expressed under the form $\sum_{j=0}^{s-1}x^jv_j(x^s)$ where $v_j(x)$ is a polynomial of degree $r-1$, for $j=0,\ldots,s-1$. Thus, we have  $u(x).(x^s+1)= \sum_{j=0}^{s-1}x^jv_j(x^s).(x^s+1)=\sum_{j=0}^{s-1}x^j v'_j(x^s)$ where $v'_j(x)=v_j(x).(x+1)$, for $j=0,\ldots,s-1$. Like for the AOP, this implies that the weight of each $v'_j$ is even. This means that any element of $A_1$ can be seen as the interleaving of $s$ even weight elements of length $r$. The number of elements which verify this property is exactly the number of elements of $A_1$, so this property characterizes the elements of $A_1$. 

In our case, we have $s=3$ and $r=2$. The ESP $g(x)=x^6+x^3+1$ is irreducible and allows to represent the finite field  $\mathbb{F}_{2^6}$. Its elements are sent onto the ring $R_{2,9}= \mathbb{F}_{2}[x]/(x^9+1)$ to perform fast operations.

\subsubsection{The parity transform}

As we said, the ideal corresponding to a finite field determined by an ESP is the set of elements which can be seen as an interleaving of $s$ even weight words of length $r$. Like AOP, we propose to add a single parity bit to each "interleaved" word of length $r$ in order to verify the parities. In the following equations, let's suppose $b(x) = a+b.x+c.x^2+d.x^3+e.x^4+f.x^5$ as an element of $\mathbb{F}_{2^6}$.

	\begin{displaymath}
		\phi_P : b(x) \rightarrow b(x) + (a+d).x^6+(b+e).x^7+(c+f).x^8
	\end{displaymath}

For the inverse function, for the same reasons as for AOPs, we can just remove the parity bits from the vector obtained after the matrix vector multiplication.

	\begin{displaymath}
		\phi_P^{-1} : b(x) + (a+d).x^6+(b+e).x^7+(c+f).x^8 \rightarrow b(x)
	\end{displaymath}

\subsubsection{The embedding transform}

Like for AOPs, the embedding function for ESPs is direct. 

	\begin{displaymath}
		\phi_E : b(x) \rightarrow b(x) + 0.x^6+0.x^7+0.x^8
	\end{displaymath}

The inverse function consists in computing the remainder modulo $p(x)$, an ESP of degree $r.s$. Thanks to the form of ESP, we can observe that this operation is equivalent to \texttt{xor}ing the last block of $s$ bits to the $r$ blocks of $s$ bits:

	\begin{equation*}
	\phi_E^{-1} :  \begin{array}{c}
	b(x) + g.x^6+h.x^7+i.x^8  \rightarrow \\
	(a+g)+(b+h).x+(c+i).x^2+\\
	(d+g).x^3+(e+h).x^4+(f+i).x^5
	\end{array}
	\end{equation*} 

\section{Implementation}

\begin{figure*}[htb!]
	\begin{center}% note that \centering uses less vspace...
		\begin{tikzpicture}
		\begin{axis}[
		ymajorgrids,
		width=0.38\textwidth,
		height=0.15\textheight,
		xtick pos=left,
		ytick pos=left,
		%legend columns=-1,
		legend entries={Pyrit (encoding), Pyrit (decoding), ISA-L (encoding), ISA-L (decoding)},
		legend to name=named,
		xmode = log,
		axis x line*=bottom,
		axis y line*=left,
		ymin=0,
		ymax=20,
		xmin=0,
		ytick={5,10,15,20},
		xtick={128,2048,32768,524288,8388608},
		xticklabels={128 B, 2 KB, 32 KB, 512 KB, 8 MB},
		extra x ticks={128,256,512,1024,2048,4096, 8192, 16384, 32768, 65536, 131072, 262144, 524288, 1048576, 2097152, 4194304, 8388608},
		extra x tick labels={,,,,,,,,,,,,},
		%xlabel=Symbol size,
		%ylabel=Coding Speed (GB/s),
		title={\textbf{(12,8) coding}}]
		\addplot+[red, mark=*, mark options={fill=red}] table [x=size, y=t1,, col sep=comma] {encoding_12_8_emb.csv};
		\addplot+[pink, mark=*, mark options={fill=pink}] table [x=size, y=t1,, col sep=comma] {decoding_12_8_emb.csv};
		\addplot+[blue, mark=square*, mark options={fill=blue}] table [x=size, y=t1,, col sep=comma] {encoding_isa_12_8.csv};
		\addplot+[green, solid, mark=square*, mark options={fill=green}] table [x=size, y=t1,, col sep=comma] {decoding_isa_12_8.csv};
		\end{axis}
		\end{tikzpicture}
		%
		\begin{tikzpicture}
		\begin{axis}[
		ymajorgrids,
		width=0.38\textwidth,
		height=0.15\textheight,
		xtick pos=left,
		ytick pos=left,
		%legend columns=-1,
		legend entries={Pyrit (encoding), Pyrit (decoding), ISA-L (encoding), ISA-L (decoding)},
		legend to name=named,
		xmode = log,
		axis x line*=bottom,
		axis y line*=left,
		ymin=0,
		ymax=4,
		xmin=0,
		ytick={1,2,3,4},
		xtick={128,2048,32768,524288,8388608},
		xticklabels={128 B, 2 KB, 32 KB, 512 KB, 8 MB},
		extra x ticks={128,256,512,1024,2048,4096, 8192, 16384, 32768, 65536, 131072, 262144, 524288, 1048576, 2097152, 4194304, 8388608},
		extra x tick labels={,,,,,,,,,,,,},
		%xlabel=Symbol size,
		%ylabel=Coding Speed (GB/s),
		title={\textbf{(60,40) coding}}]
		\addplot+[red, mark=*, mark options={fill=red}] table [x=size, y=t1,, col sep=comma] {encoding_60_40_emb.csv};
		\addplot+[pink, mark=*, mark options={fill=pink}] table [x=size, y=t1,, col sep=comma] {decoding_60_40_emb.csv};
		\addplot+[blue, mark=square*, mark options={fill=blue}] table [x=size, y=t1,, col sep=comma] {encoding_isa_60_40.csv};
		\addplot+[green, solid, mark=square*, mark options={fill=green}] table [x=size, y=t1,, col sep=comma] {decoding_isa_60_40.csv};
		\end{axis}
		\end{tikzpicture}		
		%
		\begin{tikzpicture}
		\begin{axis}[
		ymajorgrids,
		width=0.38\textwidth,
		height=0.15\textheight,
		xtick pos=left,
		ytick pos=left,
		%legend columns=-1,
		legend columns=4,
		legend entries={Pyrit (encoding), Pyrit (decoding), ISA-L (encoding), ISA-L (decoding)},
		legend to name=named,
		xmode = log,
		axis x line*=bottom,
		axis y line*=left,
		ymin=0,
		ymax=8,
		xmin=0,
		ytick={1,2,3,4,5,6,7,8},
		xtick={128,2048,32768,524288,8388608},
		xticklabels={128 B, 2 KB, 32 KB, 512 KB, 8 MB},
		extra x ticks={128,256,512,1024,2048,4096, 8192, 16384, 32768, 65536, 131072, 262144, 524288, 1048576, 2097152, 4194304, 8388608},
		extra x tick labels={,,,,,,,,,,,,},	
		%xlabel=Symbol size,
		%ylabel=Coding Speed (GB/s),
		title={\textbf{(60,20) coding}}]
		\addplot+[red, mark=*, mark options={fill=red}] table [x=size, y=t1,, col sep=comma] {encoding_60_20_par.csv};
		\addplot+[pink, mark=*, mark options={fill=pink}] table [x=size, y=t1,, col sep=comma] {decoding_60_20_par.csv};
		\addplot+[blue, mark=square*, mark options={fill=blue}] table [x=size, y=t1,, col sep=comma] {encoding_isa_60_20.csv};
		\addplot+[green, solid, mark=square*, mark options={fill=green}] table [x=size, y=t1,, col sep=comma] {decoding_isa_60_20.csv};
		\end{axis}
		\end{tikzpicture}
		%
		%\ref{named}
		\caption{Coding speed (GB/s) comparison of MDS erasure codes depending on the block size}
		\label{bench}
	\end{center}
\end{figure*}

We have implemented our method into an MDS erasure code making linear combinations over two finite fields ($\mathbb{F}_{2^4}$ and $\mathbb{F}_{2^6}$) using generalized Cauchy matrices.
The code is written in C, except the coding part which is written  in inline assembly. Indeed, by applying the transforms, we need to generate additional data before doing the \texttt{xor} operations. This data is never used after these xor's. In order to avoid useless memory writes, we compute this additional data into the CPU core by using SIMD registers without writing back the result into the DRAM memory.

For the field $\mathbb{F}_{2^w}$, we use $w$ SIMD registers to load $w$ chunks of a source block, and $w$ other registers to load $w$ chunks of the destination block.

Assume we are working on $\mathbb{F}_{2^4}$ using the SSE SIMD instruction set. We use four 128-bit registers to load 64 bytes of data at a time, which is generally the size of a cache line. Four other registers are used to load the coded data in the same way, and one last register is used to perform the ring transform.

Once the data is loaded into the registers, depending on the finite field and the transform, we use one or three registers to compute the additional data to transform the field elements into ring elements. Then, we do the \texttt{xor} operations before applying the reverse transform to go back from the ring to the field.
As the binary matrices representing the constant elements are only composed by diagonals, we do not need to read the operations to carry out from the memory: for each bit equal to 1 in the constant element, $w$ independent \texttt{xor} operations are unrolled. These binary matrices are built with the "sparse transform" introduced in \ref{sec:sparse}.

Depending on the code parameters, different transforms can be used. Indeed, when $k>=r$, the embedding transform is faster, when $k<r$, the parity transform is faster \cite{isit}. Of course, the corresponding reverse transform must be used.

\section{Performance evaluation}
\label{sec:perf}

In this section, we measure the performances of the encoding and decoding processes. 
For our tests, we used a machine with a 3.40 GHz Intel Core i7-6700 (Skylake architecture) and 16GB of DRAM.
The CPU has 4 cores with hyper threading, 2*32 kB of L1 cache (32 kB of data + 32 kB of instructions) and 256 kB of L2 cache per core, and 8 MB of shared L3 cache.
It supports SSE, AVX and AVX2 instructions sets. 

We defined 3 use cases: a $(12,8)$ over $\mathbb{F}_{2^4}$ with embedding transform, a $(60,40)$ over $\mathbb{F}_{2^6}$ with embedding transform and a $(60,20)$ over $\mathbb{F}_{2^6}$ with parity transform.

We first focused on the excellent Jerasure library \cite{jerasure} because it provides a lot of different methods to perform MDS erasure codes.
In	\cite{pgm:13:sfg}, the authors show that the \textit{split tables} method using SIMD instructions is the fastest implementation for erasure codes based on $\mathbb{F}_{2^w}$ and have the same performances when $w=4$ and $w=8$. 
As we partially did our implementation in assembly code, we fairly compared it with erasure code of the Intel ISA-L library \cite{isal}, another assembly code implementing an erasure code using the \textit{split tables} method with SIMD instructions sets.

As far as we know, ISA-L is the fastest MDS erasure code available because it implements one of the best methods to perform multiplications in a finite field, and it is written in assembly. 

We compared our codec with ISA-L on the 3 different use cases. We also studied the impact of multithreading on the coding speed.


\subsection{Performance analysis using 1 thread}

Figure 2 shows the performances of our codec using  $\mathbb{F}_{2^4}$ or  $\mathbb{F}_{2^6}$ compared to ISA-L. Both codecs are configured to use AVX2 instructions. We encode and decode data by varying the block size for two code parameters: $(12,8)$, $(60,40)$ and $(60,20)$. We repeat the measurements 1000 times for each point.

For the 3 cases, Pyrit is faster than ISA-L for both encoding and decoding. Some reasons are that Pyrit does not use lookup tables like split tables, and needs less instructions to perform the multiplications.
Note that when the total amount of data manipulated ($(k+r)*(symbol size)$) is greater than the L3 cache ($8MB$), the coding speed is constant for both codecs.

\subsection{Performance analysis using multithreading}

\begin{figure*}
	\begin{center}% note that \centering uses less vspace...
		\begin{tikzpicture}
		\begin{axis}[
		ymajorgrids,
		width=0.55\textwidth,
		height=0.18\textheight,
		xtick pos=left,
		ytick pos=left,
		%legend columns=-1,
		legend entries={Pyrit (1t), Pyrit (2t), Pyrit (4t), Pyrit (8t), ISA-L (1t), ISA-L (2t),ISA-L (4t), ISA-L (8t)},
		legend to name=named,
		xmode = log,
		axis x line*=bottom,
		axis y line*=left,
		ymin=0,
		ymax=20,
		xmin=0,
		ytick={5,10,15,20},
		xtick={128,2048,32768,524288,8388608},
		xticklabels={128 B, 2 KB, 32 KB, 512 KB, 8 MB},
		extra x ticks={128,256,512,1024,2048,4096, 8192, 16384, 32768, 65536, 131072, 262144, 524288, 1048576, 2097152, 4194304, 8388608},
		extra x tick labels={,,,,,,,,,,,,},
		%xlabel=Symbol size,
		%ylabel=Coding Speed (GB/s),
		title={\textbf{(60,40) encoding}}]
		\addplot+[red, mark=*, mark options={fill=red}] table [x=size, y=t1,, col sep=comma] {encoding_60_40_emb.csv};
		\addplot+[pink, mark=*, mark options={fill=orange}] table [x=size, y=t2,, col sep=comma] {encoding_60_40_emb.csv};
		\addplot+[pink, mark=*, mark options={fill=yellow}] table [x=size, y=t4,, col sep=comma] {encoding_60_40_emb.csv};
		\addplot+[pink, mark=*, mark options={fill=pink}] table [x=size, y=t8,, col sep=comma] {encoding_60_40_emb.csv};
		\addplot+[blue, mark=square*, mark options={fill=blue}] table [x=size, y=t1,, col sep=comma] {encoding_isa_60_40.csv};
		\addplot+[green, solid, mark=square*, mark options={fill=green}] table [x=size, y=t2,, col sep=comma] {encoding_isa_60_40.csv};
		\addplot+[brown, solid, mark=square*, mark options={fill=brown}] table [x=size, y=t4,, col sep=comma] {encoding_isa_60_40.csv};
		\addplot+[black, solid, mark=square*, mark options={fill=black}] table [x=size, y=t8,, col sep=comma] {encoding_isa_60_40.csv};		
		\end{axis}
		\end{tikzpicture}		
		%
		\begin{tikzpicture}
		\begin{axis}[
		ymajorgrids,
		width=0.55\textwidth,
		height=0.18\textheight,
		xtick pos=left,
		ytick pos=left,
		%legend columns=-1,
		legend columns=8,
		legend entries={Pyrit (1t), Pyrit (2t), Pyrit (4t), Pyrit (8t), ISA-L (1t), ISA-L (2t),ISA-L (4t), ISA-L (8t)},
		legend to name=named,
		xmode = log,
		axis x line*=bottom,
		axis y line*=left,
		ymin=0,
		ymax=20,
		xmin=0,
		ytick={5,10,15,20},
		xtick={128,2048,32768,524288,8388608},
		xticklabels={128 B, 2 KB, 32 KB, 512 KB, 8 MB},
		extra x ticks={128,256,512,1024,2048,4096, 8192, 16384, 32768, 65536, 131072, 262144, 524288, 1048576, 2097152, 4194304, 8388608},
		extra x tick labels={,,,,,,,,,,,,},	
		%xlabel=Symbol size,
		%ylabel=Coding Speed (GB/s),
		title={\textbf{(60,20) encoding}}]
		\addplot+[red, mark=*, mark options={fill=red}] table [x=size, y=t1,, col sep=comma] {encoding_60_20_par.csv};
		\addplot+[red, mark=*, mark options={fill=orange}] table [x=size, y=t2,, col sep=comma] {encoding_60_20_par.csv};
		\addplot+[pink, mark=*, mark options={fill=yellow}] table [x=size, y=t4,, col sep=comma] {encoding_60_20_par.csv};
		\addplot+[pink, mark=*, mark options={fill=pink}] table [x=size, y=t8,, col sep=comma] {encoding_60_20_par.csv};
		\addplot+[blue, mark=square*, mark options={fill=blue}] table [x=size, y=t1,, col sep=comma] {encoding_isa_60_20.csv};
		\addplot+[green, solid, mark=square*, mark options={fill=green}] table [x=size, y=t2,, col sep=comma] {encoding_isa_60_20.csv};
		\addplot+[brown, solid, mark=square*, mark options={fill=brown}] table [x=size, y=t4,, col sep=comma] {encoding_isa_60_20.csv};
		\addplot+[black, solid, mark=square*, mark options={fill=black}] table [x=size, y=t8,, col sep=comma] {encoding_isa_60_20.csv};	
		\end{axis}
		\end{tikzpicture}
		%
		\\
		\begin{tikzpicture}
		\begin{axis}[
		ymajorgrids,
		width=0.55\textwidth,
		height=0.18\textheight,
		xtick pos=left,
		ytick pos=left,
		%legend columns=-1,
		legend entries={Pyrit (1t), Pyrit (2t), Pyrit (4t), Pyrit (8t), ISA-L (1t), ISA-L (2t),ISA-L (4t), ISA-L (8t)},
		legend to name=named,
		xmode = log,
		axis x line*=bottom,
		axis y line*=left,
		ymin=0,
		ymax=20,
		xmin=0,
		ytick={5,10,15,20},
		xtick={128,2048,32768,524288,8388608},
		xticklabels={128 B, 2 KB, 32 KB, 512 KB, 8 MB},
		extra x ticks={128,256,512,1024,2048,4096, 8192, 16384, 32768, 65536, 131072, 262144, 524288, 1048576, 2097152, 4194304, 8388608},
		extra x tick labels={,,,,,,,,,,,,},
		%xlabel=Symbol size,
		%ylabel=Coding Speed (GB/s),
		title={\textbf{(60,40) decoding}}]
		\addplot+[red, mark=*, mark options={fill=red}] table [x=size, y=t1,, col sep=comma] {decoding_60_40_emb.csv};
		\addplot+[pink, mark=*, mark options={fill=orange}] table [x=size, y=t2,, col sep=comma] {decoding_60_40_emb.csv};
		\addplot+[pink, mark=*, mark options={fill=yellow}] table [x=size, y=t4,, col sep=comma] {decoding_60_40_emb.csv};
		\addplot+[pink, mark=*, mark options={fill=pink}] table [x=size, y=t8,, col sep=comma] {decoding_60_40_emb.csv};
		\addplot+[blue, mark=square*, mark options={fill=blue}] table [x=size, y=t1,, col sep=comma] {decoding_isa_60_40.csv};
		\addplot+[green, solid, mark=square*, mark options={fill=green}] table [x=size, y=t2,, col sep=comma] {decoding_isa_60_40.csv};
		\addplot+[brown, solid, mark=square*, mark options={fill=brown}] table [x=size, y=t4,, col sep=comma] {decoding_isa_60_40.csv};
		\addplot+[black, solid, mark=square*, mark options={fill=black}] table [x=size, y=t8,, col sep=comma] {decoding_isa_60_40.csv};		
		\end{axis}
		\end{tikzpicture}		
		%
		\begin{tikzpicture}
		\begin{axis}[
		ymajorgrids,
		width=0.55\textwidth,
		height=0.18\textheight,
		xtick pos=left,
		ytick pos=left,
		%legend columns=-1,
		legend columns=4,
		legend entries={Pyrit (1t), Pyrit (2t), Pyrit (4t), Pyrit (8t), ISA-L (1t), ISA-L (2t),ISA-L (4t), ISA-L (8t)},
		legend to name=named,
		xmode = log,
		axis x line*=bottom,
		axis y line*=left,
		ymin=0,
		ymax=30,
		xmin=0,
		ytick={5,10,15,20,25,30},
		xtick={128,2048,32768,524288,8388608},
		xticklabels={128 B, 2 KB, 32 KB, 512 KB, 8 MB},
		extra x ticks={128,256,512,1024,2048,4096, 8192, 16384, 32768, 65536, 131072, 262144, 524288, 1048576, 2097152, 4194304, 8388608},
		extra x tick labels={,,,,,,,,,,,,},	
		%xlabel=Symbol size,
		%ylabel=Coding Speed (GB/s),
		title={\textbf{(60,20) decoding}}]
		\addplot+[red, mark=*, mark options={fill=red}] table [x=size, y=t1,, col sep=comma] {decoding_60_20_par.csv};
		\addplot+[red, mark=*, mark options={fill=orange}] table [x=size, y=t2,, col sep=comma] {decoding_60_20_par.csv};
		\addplot+[pink, mark=*, mark options={fill=yellow}] table [x=size, y=t4,, col sep=comma] {decoding_60_20_par.csv};
		\addplot+[pink, mark=*, mark options={fill=pink}] table [x=size, y=t8,, col sep=comma] {decoding_60_20_par.csv};
		\addplot+[blue, mark=square*, mark options={fill=blue}] table [x=size, y=t1,, col sep=comma] {decoding_isa_60_20.csv};
		\addplot+[green, solid, mark=square*, mark options={fill=green}] table [x=size, y=t2,, col sep=comma] {decoding_isa_60_20.csv};
		\addplot+[brown, solid, mark=square*, mark options={fill=brown}] table [x=size, y=t4,, col sep=comma] {decoding_isa_60_20.csv};
		\addplot+[black, solid, mark=square*, mark options={fill=black}] table [x=size, y=t8,, col sep=comma] {decoding_isa_60_20.csv};	
		\end{axis}
		\end{tikzpicture}		
		%\ref{named}
		\label{mt}
		\caption{Coding speed (GB/s) comparison depending on the block size and the number of threads}
		\label{bench}
	\end{center}
\end{figure*}

We modified both ISA-L and Pyrit to support multithreading coding. To do that, we split the data to encode or decode, and send a different part to all the threads, synchronized with pthread barriers. That means that each thread is writing and reading a different part of the symbols to code. Figure 3 shows the results for 1,2,4 and 8 threads using ISA-L and Pyrit (over $\mathbb{F}_{2^4}$).

Both codecs have increased performances using multiple threads, even if Pyrit is still faster than ISA-L.
But, when the number of threads is greater than the number of cores, for $8$ threads on $4$ cores, the performance of the ISA-L codec decreases whereas this is not the case for Pyrit, even if the gain is low.

\subsection{Results}

For any coding parameters, the Pyrit method performs much faster than ISA-L. Even when the data does not fit into the CPU caches, our method drastically improves the performances.
Making linear combinations over $\mathbb{F}_{2^4}$ or $\mathbb{F}_{2^6}$ using Pyrit gives the same performances for the encoding process. This can be explained since we use a generalized Cauchy matrix which is sparse but generic matrix. For the decoding process, as the inverse matrix depends on the loss pattern, we have a slightly dense matrix using $\mathbb{F}_{2^6}$, so working in $\mathbb{F}_{2^4}$ is faster.
The only constraint is the number of symbols which is respectively $16$ and $64$ for $\mathbb{F}_{2^4}$ and $\mathbb{F}_{2^6}$. With ISA-L, it is possible to have $256$ symbols. Nevertheless, when $n<=64$, Pyrit is the fastest method.

\section{Conclusion}
\label{sec:ccl}

In this paper, we have presented a new method to accelerate both coding and decoding processes of erasure codes. By using transforms between a finite field and a polynomial ring, sparse generator matrices can be obtained. This allows to significantly reduce the complexity of matrix vector multiplication. 
We presented two fast erasure codes implementations using $\mathbb{F}_{2^4}$ and $\mathbb{F}_{2^6}$ and evaluated the performances for several use cases. For each cases, Pyrit has better performances. The performance analysis was done for MDS erasure codes, but Pyrit can be used by every code using matrix vector multiplication over a finite field.

\bibliographystyle{IEEEtran}
	\bibliography{biblio}
\end{document}

\end{document}
