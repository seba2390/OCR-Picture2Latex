% \documentclass{svjour2}                    % onecolumn
\documentclass[onecollarge]{svjour2}       % onecolumn "king-size"

\smartqed  % flush right qed marks, e.g. at end of proof
%
\usepackage{graphicx}
\usepackage{color}
\usepackage{amsmath}
%\usepackage{amsthm}
\usepackage{amssymb}
% \usepackage{amsthm}
\usepackage{esint}
%\usepackage{hyperref}
\usepackage{mathtools}
\usepackage{float}
\usepackage{amsmath,amsfonts,amssymb,latexsym,epsfig}
%\usepackage[notref,notcite]{showkeys}
\usepackage[most]{tcolorbox}
\usepackage[colorinlistoftodos,prependcaption,textsize=tiny]{todonotes}
\usepackage{todonotes}
\usepackage{subcaption}
\usepackage{enumitem}
\usepackage{ntheorem}
@book{meyerOffin,
  author =  {K. R. Meyer and D. C. Offin},
  title =   {Introduction to Hamiltonian Dynamical Systems and the N-Body Problem},
  editor =    {},
  publisher = {Springer International Publishing},
  year =      {2017},
  volume =    {90},
  number =    {},
  series =    {Applied Mathematical Sciences},
  address =   {},
  edition =   {3rd},
  month =     {},
}

@article{Llibre2021,
  author =  {J. Llibre and D. Pa?ca and C. Valls},
  title =   {The circular restricted 4-body problem with three equal primaries in the collinear central configuration of the 3-body problem},
  journal = {Celestial Mechanics and Dynamical Astronomy},
  year =    {2021},
  volume =  {133},
  number =  {53},
  pages =   {},
  doi =     {https://doi.org/10.1007/s10569-021-10052-6},
}

@article{AlvarezRamirez2015,
  author =  {M. Alvarez-Ram�rez and J.E.F. Skea and T.J. Stuchi},
  title =   {Nonlinear stability analysis in a equilateral restricted four-body problem},
  journal = {Astrophys Space Sci},
  year =    {2015},
  volume =  {358},
  number =  {3},
  pages =   {},
  doi =     {https://doi.org/10.1007/s10509-015-2333-4},
}

@article{Corbera2014,
  author =  {M. Corbera and J. Llibre},
  title =   {Central configurations of the $4$-body problem with masses $m_1=m_2>m_3=m_4>0$ and small},
  journal = {Appl Math Comput},
  year =    {2014},
  volume =  {246},
  number =  {},
  pages =   {121-147},
  doi =     {https://doi.org/10.1016/j.amc.2014.07.109},
}

@article{AlvarezRamirez2013,
  author =  {M. Alvarez-Ram\'irez and J. Llibre},
  title =   {The symmetric central configuration of the $4$-body problem with masses $m_1=m_2\ne m_3=m_4$},
  journal = {Appl Math Comput},
  year =    {2013},
  volume =  {219},
  number =  {},
  pages =   {5996-6001},
  doi =     {https://doi.org/10.1016/j.amc.2012.12.036},
}

@article{Long2002,
  author =  {Y. Long and S. Sun},
  title =   {Four-Body Central Configurations with some Equal Masses},
  journal = {Arch Rational Mech Anal},
  year =    {2002},
  volume =  {162},
  number =  {},
  pages =   {25-44},
  doi =     {https//doi.org/10.1007/s002050100183},
}

@article{Simo,
  author =  {C. Sim\'o},
  title =   {Relative Equilibrium Solutions in the Four Body Problem},
  journal = {Celestial Mechanics},
  year =    {1978},
  volume =  {18},
  number =  {},
  pages =   {165-184},
  doi =     {},
}

@article{Roy,
  author =  {A. E. Roy and B. A. Steves},
  title =   {Some Special restricted four-body problems-{\rm II}: From Caledonia to Copenhagen},
  journal = {Planet Space Sci},
  year =    {1998},
  volume =  {46},
  number =  {11-12},
  pages =   {1475-1486},
  doi =     {},
}

@article{Shoaib,
  author =  {M. Shoaib and I. Faye},
  title =   {Collinear Equilibrium Solutions of Four-Body Problems},
  journal = {J Astrophys Astr},
  volume =  {32},
  year =    {2011},
  number =  {},
  pages =   {411-423},
  doi =     {},
}

@article{Davis,
  author =  {J. Davis},
  title =   {Planetary Society asteroid hunters help find rare type of double asteroid},
  journal = {The Planetary Society},
  year =    {2018},
  number =  {},
  pages =   {},
  note =     {Online article at https://www.planetary.org/articles/shoemaker-winners-2017-ye5},
}

@manual{nasa,
  author =  {C. Cofield and J. Wendel},
  title =   {Observatories Team Up To Reveal Rare Double Asteroid},
  organization = {NASA},
  year =    {2018},
  note =    {Online article at https://www.nasa.gov/feature/jpl/observatories-team-up-to-reveal-rare-double-asteroid},
}

@misc{Johnston,
  author =  {W. R. Johnston},
  title =   {Asteroids with Satellites},
  year =    {2023},
  note =    {Online Database at https://www.johnstonsarchive.net/astro/asteroidmoons.html},
}

@article{Monteiro,
  author =  {F. Monteiro and E. Rond\'on and D. Lazzaro and J. Oey and M Evangelista-Santana and P. Avcoverde and M DeCicco and J. S. Silva-Cabrera and T. Rodrigues and L. B. Santos},
  title =   {Physical characterization of equal-mass binary near-Earth asteroid 2017 YE5: a possible dormant Jupiter-family comet},
  journal = {Monthly Notices of the Royal Astronomical Society},
  year =    {2021},
  volume =  {507},
  number =  {},
  pages =   {5403-5414},
  doi =     {https://doi.org/10.1093/mnras/stab2408},
}

@article{Scheeres,
  author =  {D. J. Scheeres and J. Bellerose},
  title =   {The Restricted Hill Full $4$-Body Problem: application to spacecraft motion about binary asteroids},
  journal = {Dyn Syst},
  year =    {2005},
  volume =  {20},
  number =  {1},
  pages =   {23-44},
  doi =     {https://doi.org/10.1080/1468936042000281321},
}

@article{Wang,
  author =  {H.S. Wang and X. Y. Hou},
  title =   {Forced Hovering orbit above the primary in the binary asteroid system},
  journal = {Celest Mech Dyn Astron},
  year =    {2022},
  volume =  {134},
  number =  {},
  pages =   {},
  doi =     {https://doi.org/10.1007/s10569-022-10098-0},
}

@article{Lu,
  author =  {J. Lu H. Shang and B. Wei},
  title =   {Accelerating binary asteroid system propagation via nested interpolation method},
  journal = {Celest Mech Dyn Astron},
  year =    {2023},
  volume =  {135},
  number =  {},
  pages  =  {},
  doi    =  {https://doi.org/10.1007/s10569-023-10123-w},
}

@article{Espitia,
  author =  {D. Espitia and E. A. Quintero and I. D. Arellano-Ram\'irez},
  title =   {Determination of Orbital Elements and Ephemerides using the Geometric Laplace's Method},
  journal = {J Astron Space Sci},
  year =    {2020},
  volume =  {37},
  number =  {3},
  pages =   {171-185},
  doi =     {https://doi.org/10.5140/JASS.2020.37.3.171},
}

@article{Stoica,
  author =  {J. Llibre and C. Stoica},
  title =   {Comet- and Hill-type periodic orbits in restricted $(N+1)$-body problems},
  journal = {J Differential Equations},
  year =    {2011},
  volume =  {250},
  number =  {},
  pages =   {1747-1766},
  doi =     {doi:10.1016/j.jde.2010.08.005},
}

@unpublished{Bakker,
  author =  {L. F. Bakker and J. Murri and S. Simmons},
  title =   {A model for the Binary Asteroid 2017 YE5},
  year =    {2023},
  Note =    {In preparation},
}

@unpublished{Cochran,
  author =  {L. F. Bakker and S. Cochran},
  title =   {Periodic solutions in the planar $5$-body problem with two free bodies},
  year =    {2021},
  note =    {Undergraduate Research},
}

@misc{Freeman,
  author = {N. J. Freeman},
  title  = {Investigations of a Binary Asteroid Dynamical Model},
  year   = {2023},
  month  = {April},
  note   = {Senior Thesis, Department of Physics and Astronomy, Brigham Young University},
}

@book{Szebehely,
  author    = {V. Szebehely},
  title     = {The Theory of Orbits},
  publisher = {Academic Press},
  address   = {New York},
  year      = {1967},
}



%\part{\part{\part{title}}}
\usepackage{blkarray}

\newcommand{\E}{{\mathbb E}}
\newcommand{\cL}{\mathcal L}
\DeclareMathOperator{\Var}{Var}
\DeclareMathOperator{\Law}{Law}
\DeclareMathOperator{\Tr}{Tr}
\DeclareMathOperator{\Hess}{Hess}
\DeclareMathOperator{\diag}{diag}
\DeclareMathOperator{\Span}{span}
\DeclareMathOperator{\im}{im}
\DeclareMathOperator{\dist}{dist}

\DeclareMathOperator{\R}{\mathbb{R}}
\DeclareMathOperator{\Prob}{\mathbb{P}}
\DeclareMathOperator{\gen}{\mathcal{L}}

\newtheorem{assumption}{Assumption}

\newtheorem{algorithm}{Algorithm}


\newcommand{\inner}[2]{\left\langle #1, #2 \right\rangle}
\newcommand{\norm}[1]{\left|{#1}\right|} % A norm |x|
\newcommand{\Norm}[1]{\left\lVert{#1}\right\rVert} % A norm ||x||

%%%%%%%%%%Review Macros%%%%%%%%%%%
\newcommand{\andrew}[1]{\todo[linecolor=green,backgroundcolor=green!25,bordercolor=green]{A:#1}}

\newcommand{\notate}[1]{\textcolor{blue}{\textbf{[#1]}}}


\begin{document}

\title{Using Perturbed Underdamped Langevin Dynamics to Efficiently Sample
from Probability Distributions}

\titlerunning{Perturbed Underdamped Langevin Dynamics}        % if too long for running head

\author{A. B. Duncan         \and
        N. N{\"u}sken \and  
        G. A. Pavliotis
}

\authorrunning{Duncan, N{\"u}sken, Pavliotis} % if too long for running head

\begingroup
\institute{A. B. Duncan \at
              School of Mathematical and Physical Sciences, University of Sussex, Falmer, Brighton, BN1 9RH United Kingdom
              \email{Andrew.Duncan@sussex.ac.uk}           %  \\
%             \emph{Present address:} of F. Author  %  if needed
           \and
           N. N{\"u}sken \at
 Imperial College London, Department of Mathematics, South Kensington Campus, London SW7 2AZ,England \email{n.nusken14@imperial.ac.uk}
            \and
           G. A. Pavliotis \at
             Imperial College London, Department of Mathematics, South Kensington Campus, London SW7 2AZ,England \email{g.pavliotis@imperial.ac.uk}}

\date{Received: date / Accepted: date}


\maketitle
\begin{abstract}
In this paper we introduce and analyse Langevin samplers that consist of perturbations of the standard underdamped Langevin dynamics. The perturbed dynamics is such that its invariant measure is the same as that of the unperturbed dynamics. We show that appropriate choices of the perturbations can lead to samplers that have improved properties, at least in terms of reducing the asymptotic variance. We present a detailed analysis of the new Langevin sampler for Gaussian target distributions. Our theoretical results are supported by numerical experiments with non-Gaussian target measures.
\end{abstract}


\section{Introduction and Motivation}
\label{sec:introduction}
\section{Introduction}  \label{sec:introduction}

\newcommand\inexpIntro[3]{#1?(#2,#3).}
\newcommand\rinexpIntro[3]{*#1?(#2,#3).}
\newcommand\outexpIntro[3]{#1!(#2,#3).}
\newcommand\outatomIntro[3]{#1!(#2,#3)}

We propose a fully automated method for proving termination of \(\pi\)-calculus processes.
Although there have been a lot of studies on termination analysis for the \(\pi\)-calculus
and related calculi~\cite{Deng06IC,Demangeon07,SangiorgiTermination,KobayashiHybrid,Yoshida04IC,DBLP:journals/jlp/DemangeonHS10,Venet98SAS}, most of them have been rather theoretical,
and there have been surprisingly little efforts in developing  fully automated termination
verification methods and tools based on them. To our knowledge,
Kobayashi's \typical{}~\cite{TyPiCal,KobayashiHybrid} is the only exception that
can prove termination of \(\pi\)-calculus processes (extended with natural numbers)
fully automatically, but its termination analysis is quite limited (see Section~\ref{sec:relatedwork}).

Our method is based on a reduction to termination analysis for sequential programs:
we translate a \(\pi\)-calculus process \(P\) to a sequential program \(S_P\), so that
if \(S_P\) is terminating, so is \(P\). The reduction allows us to use
powerful, mature methods and tools
for termination analysis of sequential programs~\cite{heizmann2016ultimate,freqterm,DBLP:conf/lics/PodelskiR04,Kuwahara2014Termination,DBLP:journals/cacm/CookPR11}.

The idea of the translation is to convert a chain of communications on replicated input
channels to a chain of recursive function calls of the target sequential program.
Let us consider the following Fibonacci process:
\begin{align*}
    & \rinexpIntro{\fib}{n}{r}
        \ifexp{n<2}{ \soutatom{r}{1} \\ &\quad}
                   { \nuexp{s_1} \nuexp{s_2} (\outatomIntro{\fib}{n-1}{s_1} \PAR \outatomIntro{\fib}{n-2}{s_2} \PAR \sinexp{s_1}{x}\sinexp{s_2}{y}\soutatom{r}{x+y}) \\}
    & \PAR \outatomIntro{\fib}{m}{r}
\end{align*}
Here, the process
$\rinexpIntro{\fib}{n}{r} \ldots$ is a function server that computes the \(n\)-th Fibonacci number
in parallel and returns the result to \(r\),
and $\outatom{\fib}{m}{r}$ sends a request for computing the \(m\)-th Fibonacci number;
those who are not familiar with the syntax of the \(\pi\)-calculus may wish to consult
Section~\ref{sec:targetlanguage} first.
To prove that the process above is terminating for any integer \(m\),
it suffices to show that there is no infinite chain of communications on $\fib$:
\[
    \fib(m,r) \to \fib(m_1,r_1) \to \fib(m_2,r_2) \to \cdots.
\]
We convert the process above to the following program:\footnote{The actual translation
  given later is a little more complex.}
\begin{verbatim}
 let rec fib(n) = if n<2 then () else (fib(n-1) [] fib(n-2)) in
 fib(m)
\end{verbatim}
Here, \texttt{[]} represents the non-deterministic choice.
Note that, although the calculation of Fibonacci numbers is not preserved,
for each chain of communications on \texttt{fib}, there is a corresponding
sequence of recursive calls:
\[
\mathtt{fib}(m) \to \mathtt{fib}(m_1) \to \mathtt{fib}(m_2) \to \cdots.
\]
Thus, the termination of the sequential program above implies the termination of
the original process.
As shown in the example above, (i) each communication on a replicated input channel
is converted to a function call, (ii) each communication on a non-replicated input
channel is just removed (or, in the actual translation, replaced by a call of
a trivial function defined by \(f(\seq{x})=(\,)\)), and (iii) parallel composition
is replaced by a non-deterministic choice.
We formalize the translation outlined above and prove its correctness.

The basic translation sketched above sometimes loses too much information.
For example, consider the following process:
\begin{align*}
    & \rinexpIntro{\pre}{n}{r} \soutatom{r}{n-1} \\
    & \PAR \rinexpIntro{f}{n}{r} \ifexp{n<0}{ \soutatom{r}{1} }
                                       { \nuexp{s} (\outatomIntro{\pre}{n}{s} \PAR \sinexp{s}{x}\outatomIntro{f}{x}{r}) } \\
    & \PAR \outatomIntro{f}{m}{r}
\end{align*}
The translation sketched above would yield:
\begin{verbatim}
  let pred(n) = n-1 in
  let rec f(n) = if n<0 then () else (pred(n) [] f(*)) in
  f(m)
\end{verbatim}
Here, \texttt{*} represents a non-deterministic integer: since we have removed
the input $\sinatom{s}{x}$, we do not have information about the value of \( x \).
As a result, the sequential program above is non-terminating, although the original
process is terminating.
To remedy this problem, we also refine the basic translation above by using a refinement
type system for the \(\pi\)-calculus. Using the refinement type system,
we can infer that the value of \(x\) in the original process is less than \(n\),
so that we can refine the definition of \texttt{f} to:
\begin{verbatim}
 let rec f(n) = ... else (pred(n) [] let x=* in assume(x<n);f(x))
\end{verbatim}
The target program is now terminating, from which
we can deduce that the original process is also terminating.
We have implemented an automated tool based on the refined translation above.

The contributions of this paper are summarized as follows.
\begin{itemize}
\item The formalization of the basic translation from the \(\pi\)-calculus
  (extended with integers) to sequential programs, and a proof of its correctness.
\item The formalization of a refined translation based on a refinement type system.
\item An implementation of the refined translation, including automated refinement type
  inference based on CHC solving, and experiments to evaluate the effectiveness of
  our method.
\end{itemize}

The rest of this paper is structured as follows.
Section~\ref{sec:targetlanguage} introduces the source and target languages
of our translation.
Section~\ref{sec:approach} 
formalizes the basic translation, and proves its correctness.
Section~\ref{sec:refinement} refines the basic translation by using a refinement type system.
Section~\ref{sec:implementation} reports an implementation and experiments.
Section~\ref{sec:relatedwork} discusses related work,
and Section~\ref{sec:conclusion} concludes the paper.


\section{Construction of General Langevin Samplers}
\label{sec:background}

% Panoptic segmentation

% 3D segmentation

% Multi-object tracking

% Online 3D panoptic:

% PanopticFusion: (IROS 2019)
% https://arxiv.org/pdf/1903.01177.pdf
%
% - most similar to ours
% - PSPNet + M-RCNN + 2D fusion
% - volumetric mapping, 
% - greedy matching with IoU -> optimal only with 0.5 threshold
% - voxel & class weighting
% - CRF regularisation
%
% - good:
%
% - bad:
%  - CRF post-processing step
%  - greedy data-association
%    - can't be tuned for lower overlap ratios -> has to have high framerate, large changes in viewpoint could break this
%    - IoU: sensitive to 2D labels projecting over object borders (CRF and voxel weighting seem to alleviate this)

% Voxblox++: (Robotics & automation letters 2019)
% https://arxiv.org/pdf/1903.00268.pdf
% https://github.com/ethz-asl/voxblox-plusplus
%
% - M-RCNN + geometric segmentation + fusion 
% - data association of geometric segments with 3D overlap (no. points inside volume), fixed threshold for min number of points
% - instance label is assigned to a segment based on highest overlap
% - only one detected segment per reference label, as in PanopticFusion and Ours
% - TSDF Integration 
%
% good: 
% - because of geometric segmentation objects with no associated semantic class can also be segmented
% bad:
% - two different object segment types -> confusing, overly complicated ?
% - quite inaccurate (fixed below)

% Reconstructing Interactive 3D Scenes by Panoptic Mapping and CAD Model Alignments (ICRA 2021)
% https://arxiv.org/pdf/2103.16095.pdf
% https://github.com/hmz-15/Interactive-Scene-Reconstruction
%
% - based heavily on Voxblox++, much more accurate
% - Scene-graph ("contact graph") for mapping object relations
% - Search & replace voxels with CAD models, with geometrical and physical constraints
% - Object 6D pose
% - Format for robot interaction
%
% - Segmentation: bilateral fusion of geomatric and semantic segments -> reduce segmentation noise compared to Voxblox++
% - Fusion: triplet count improves consistency over Voxblox++ pairwise count strategy (take semantic label into account in addition to instance and geometry)
% - Fusion: instance labels are also combined if there is enough overlap with common geometric label for long enough time
%   - this means multiple detections can match the same reference unlike ours, voxblox++ and PanopticFusion ?
%

% Panoptic-MOPE: (ROBOTICS AND AUTOMATION LETTERS 2020)
% https://ieeexplore.ieee.org/stamp/stamp.jsp?tp=&arnumber=8977356
% https://github.com/hoangcuongbk80/Object-RPE/tree/panoptic-mope
%
% - novel RGB-D semantic segmentation model + M-RCNN
% - camera tracking based on "addaptively weighted optimization of geometric, appearance, and semantic cues"
% - surfel map: 
%   - how does it scale ? authors satate they tested on room-sized environments, but could be applied in larger scale as well ...
%     - could maybe be applied as VO in a SLAM algorithm ...
%   - demo only on a small pallet + surroundings, might not be applicable in large-scale SLAM

% US VS THEM:
%
% - based heavily on PanopticFusion, with modifications:
%   - instead of greedy data-association (which seems to be the case in others as well), we solve LAP (JPDA?)
%     - overlap threshold can be tuned, which renders the algorithm more flexible
%     - could be extended to dynamic tracking ?
%   - multiple options for association likelihood
%   - outlier rejection (either clustering or probabilistic)
%   - test different options for decreasing processing time
%   - no post-processing
%
% - model-agnostic:
%   - completely separated from segmentation
%   - does not care how point clouds are obtained -> applicable for LIDAR segmentation (e.g. EfficientLPS) as well
%
% - also agnostic to localisation method
%   - could, however, be utilised to find landmark locations / poses

% More compact version of this paragraph to introduction to save space?
%Panoptic segmentation -- proposed in \cite{panoptic_segmentation} -- aims to solve the unified task of semantic- and instance segmentation. Semantic classes are separated to \textit{stuff} -- amorphous, unquantifiable regions like sky, road or floor -- and \textit{things} -- quantifiable objects. The distinction between the two can vary depending on the application, but a semantic class can only belong to one or another. The article also proposes a unified panoptic evaluation metric, coined \textbf{Panoptic Quality} (PQ). Many 2D approaches to panoptic segmentation -- \textit{e.g.} \cite{panopticfpn,seamless,panoptic_deeplab,efficientps} -- have since been proposed. Deep neural networks for performing semantic- or instance segmentation directly on the 3D reconstruction -- \textit{e.g.} on \cite{scannet,s3dis,paris_lille_3d} -- have also been proposed, but since they require the reconstructed 3D scene, they are mostly offline approaches and therefore out of scope for this work. Some recent works also apply panoptic segmentation to point clouds -- \textit{e.g.} methods in the SemanticKITTI panoptic segmentation competition \cite{semantic_kitti} -- mostly aimed at segmenting LiDAR output. They are suitable for online processing, but similar to RGB-D images require a method for tracking object instances persistent in both time and space. In fact, our proposed method, as well as some others mentioned in this work, could use segmented LiDAR point clouds as an input similarly to RGB-D images.

PanopticFusion \cite{panopticfusion} is the first work to propose online integration of panoptic image segmentations to a 3D reconstruction. They integrate point clouds generated from segmented images to a TSDF voxel volume \cite{tsdf,voxblox} by greedily matching detected segments with the reconstruction and regulating each voxel's corresponding instance with a weighting function. Semantic labels are inferred in a bayesian manner based on confidence scores provided by the segmentation model. They also apply a Conditional Random Field (CRF) to regularise the reconstruction, improving results significantly. Voxblox++ \cite{voxblox++} -- introduced later the same year -- is a similar approach that also integrates segmented RGB-D images into a TSDF volume. It leverages geometric segmentation of depth images to improve instance segmentation accuracy. Both geometric and semantic segments are used to compute a pair-wise weight, which is used to greedily match them with segments in the reconstruction. Because of the geometric segmentation, the method allows segmentation of objects with no known semantic class in addition to objects recognised by the instance segmentation model. 

Recently, \cite{interactive_3d_scenes} built upon the idea of Voxblox++. They apply Voxblox++ for 3D instance integration, with two small but effective modifications: the pair-wise weight is replaced by a triplet weight that also takes semantic labels into account in the fusion, and -- in addition to geometric segments -- instance segments are fused if they overlap by a significant amount. The article introduces a method for searching and aligning CAD models to reconstructed objects based on geometry and semantic class, as well as geometrical and physical rules. With the CAD models, a contact graph and interactive virtual scene are reconstructed to allow a robot to simulate its interaction with the environment. SceneGraphFusion \cite{scenegraphfusion} is another approach that forms a scene graph online from a stream of RGB-D images, but unlike the above-mentioned approach, it generates the graph with a deep neural network, after which the panoptic labels for geometrically segmented portions of the 3D reconstruction are produced a side product.

Panoptic-MOPE \cite{panoptic_mope} is another recent approach, which integrates sequences of RGB-D images into a surfel reconstruction. Unlike other mentioned approaches -- which assume the camera pose either known or estimated elsewhere -- it also tracks camera movements based on geometric-, appearance- and semantic cues. The method also applies a novel RGB-D panoptic segmentation model. Although it is only tested on room-sized environments, the authors claim it could be scaled to larger environments as well.

\section{Perturbation of Underdamped Langevin Dynamics}
\label{sec:perturbed_langevin}

The primary objective of this work is to compare the performances of the perturbed underdamped Langevin dynamics (\ref{eq:perturbed_underdamped}) and the unperturbed dynamics (\ref{eq:langevin}) according to the criteria outlined in Section \ref{sec:comparison} and to find suitable choices for the matrices $J_{1}$, $J_{2}$, $M$ and $\Gamma$ that improve the performance of the sampler.  We begin our investigations of (\ref{eq:perturbed_underdamped}) by establishing ergodicity and exponentially fast return to equilibrium, and by studying the overdamped limit of~\eqref{eq:perturbed_underdamped}. As the latter turns out to be nonreversible and therefore in principle superior to the usual overdamped limit~\eqref{eq:overdamped},e.g.~\cite{Hwang2005}, this calculation provides us with further motivation to study the proposed dynamics.
\\\\
For the bulk of this work, we focus on the particular case when the target measure is Gaussian, i.e. when the potential is given by $V(q)=\frac{1}{2}q^{T}Sq$
with a symmetric and positive definite precision matrix $S$ (i.e. the covariance matrix is given by $S^{-1}$). In this
case, we advocate the following conditions for the choice of parameters:\begin{subequations}
	\label{eq:optimal parameters}
	\begin{align}
	M & =S,\label{eq:M=00003DS}\\
	\Gamma & =\gamma S,\\
	SJ_{1}S & =J_{2},\label{eq: perturbation condition}\\
	\mu & =\nu.
	\end{align}
\end{subequations}
Under the above choices \eqref{eq:optimal parameters}, we show that the large perturbation limit $\lim_{\mu\rightarrow\infty} \sigma_f^2$ exists and is finite and we provide an explicit expression for it (see Theorem \ref{cor:limit_asym_var}). From this expression, we derive an algorithm for finding optimal choices for $J_1$ in the case of quadratic observables (see Algorithm \ref{alg:optimal J general}).
\\\\
If the friction coefficient is not too small ($\gamma > \sqrt {2}$), and under certain mild nondegeneracy conditions, we prove that adding a small perturbation will always decrease the asymptotic variance for observables of the form $f(q)=q\cdot Kq+l\cdot q+C$:
\[
\left. \frac{\mathrm{d}}{\mathrm{d}\mu}\sigma_{f}^{2}\right\rvert_{\mu=0}=0\quad\text{and }\quad \left. \frac{\mathrm{d}^{2}}{\mathrm{d}\mu^{2}}\sigma_{f}^{2}\right\rvert_{\mu=0}<0,
\]
see Theorem \ref{cor:small pert unit var}. 
In fact, we conjecture that this statement is true for arbitrary observables
$f\in L^{2}(\pi)$, but we have not been able to prove this. The dynamics (\ref{eq:perturbed_underdamped})
(used in conjunction with the conditions (\ref{eq:M=00003DS})-(\ref{eq: perturbation condition}))
proves to be especially effective when the observable is antisymmetric
(i.e. when it is invariant under the substitution $q\mapsto-q$) or when it
has a significant antisymmetric part. In particular, in Proposition~\ref{prop:antisymmetric observables} we show that under certain conditions on the spectrum of $J_1$, for any antisymmetric observable $f\in L^{2}(\pi)$ it holds that  $\lim_{\mu\rightarrow\infty}\sigma_{f}^{2}=0$.
\\\\
Numerical experiments and analysis show that departing significantly
from~\ref{eq: perturbation condition} in fact possibly decreases
the performance of the sampler. This is in stark contrast to~\eqref{eq:nonreversible_overdamped}, where it is not possible to increase the asymptotic variance by \emph{any} perturbation.  For that reason, until now it seems practical to use (\ref{eq:perturbed_underdamped})  as a sampler only when a reasonable estimate of the global covariance of the target distribution is available. In the case of Bayesian inverse problems and diffusion bridge sampling, the target measure $\pi$ is given with respect to a Gaussian prior. We demonstrate the effectiveness of our approach in these applications, taking the prior Gaussian covariance as $S$ in (\ref{eq:M=00003DS})-(\ref{eq: perturbation condition}).
% In the case when the target measure is highly nonlinear, it it tempting
% to choose $J_{1}$ and $J_{2}$ in a position-dependent way, such
% that (\ref{eq: perturbation condition}) is satisfied locally (where
% $S$ then encodes the local covariance structure of the target, for
% instance being the expected Fisher information). This approach (which
% can be regarded as a manifold version of (\ref{eq: Perturbed Underdamped Langevin})
% and is thus similar to Riemannian manifold Monte Carlo,\cite{RiemannHMC})
% will be developed in a forthcoming publication. 
\begin{remark}
	In \cite[Rem. 3]{LelievreNierPavliotis2013} another modification of (\ref{eq:langevin})
	was suggested (albeit with the simplifications $\Gamma=\gamma\cdot I$
	and $M=I$):
\end{remark}
\begin{align}
\mathrm{d}q_{t} & =(1-J)M^{-1}p_{t}\mathrm{d}t ,\nonumber \\
\mathrm{d}p_{t} & =-(1+J)\nabla V(q_{t})\mathrm{d}t-\Gamma M^{-1}p_{t}\mathrm{d}t+\sqrt{2\Gamma}\mathrm{d}W_{t},\label{eq: JJ perturbation}
\end{align}
$J$ again denoting an antisymmetric matrix. However, under the change
of variables $p\mapsto(1+J)\tilde{p}$ the above equations transform
into 
\begin{align*}
\mathrm{d}q_{t} & =\tilde{M}^{-1}p_{t}\mathrm{d}t,\\
\mathrm{d}\tilde{p_{t}} & =-\nabla V(q_{t})\mathrm{d}t-\tilde{\Gamma}\tilde{M}^{-1}\tilde{p}_{t}\mathrm{d}t+\sqrt{2\tilde{\Gamma}}\mathrm{d}\tilde{W}_{t},
\end{align*}
where $\tilde{M}=(1+J)^{-1}M(1-J)^{-1}$ and $\tilde{\Gamma}=(1+J)^{-1}\Gamma(1-J)^{-1}$.
Since any observable $f$ depends only on $q$ (the $p$-variables
are merely auxiliary), the estimator $\pi_T(f)$ as well as its associated convergence characteristics (i.e. asymptotic
variance and speed of convergence to equilibrium) are invariant under this transformation.
Therefore, (\ref{eq: JJ perturbation}) reduces to the underdamped
Langevin dynamics (\ref{eq:langevin}) and does not represent an independent approach to sampling. Suitable choices
of $M$ and $\Gamma$ will be discussed in Section \ref{sec:arbitrary covariance}.

\subsection{Properties of Perturbed Underdamped Langevin Dynamics}
\label{sec:hypocoercivity}

In this section we study some of the properties of the perturbed underdamped dynamics (\ref{eq:perturbed_underdamped}). First, note that its generator is given by
\begin{equation}
\label{eq:generator}
\mathcal{L}=\underbrace{\underbrace{M^{-1}p\cdot\nabla_{q}-\nabla_{q}V\cdot\nabla_{p}}_{\mathcal{L}_{ham}}\underbrace{-\Gamma M^{-1}p\cdot\nabla_{p}+\Gamma : D^2_{p}}_{\mathcal{L}_{therm}}}_{\mathcal{L}_0} \underbrace{-\mu J_{1}\nabla V \cdot \nabla_{q} - \nu J M^{-1} p \cdot \nabla_{p}}_{\mathcal{L}_{pert}},
\end{equation}	
decomposed into the perturbation $\mathcal{L}_{pert}$ and the unperturbed operator $\mathcal{L}_0$, which can be further split into the Hamiltonian part $\mathcal{L}_{ham}$ and the thermostat (Ornstein-Uhlenbeck) part $\mathcal{L}_{therm}$, see \cite{pavliotis2014stochastic,Free_energy_computations,LS2016}.

\begin{lemma}
\label{lem:hypoellipticity}
	The infinitesimal generator $\gen$~\eqref{eq:generator} is hypoelliptic.
\end{lemma}
\begin{proof}
	See Appendix \ref{app:hypocoercivity}.\qed
\end{proof}

An immediate corollary of this result and of Theorem \ref{theorem:invariance_theorem} is that the perturbed underdamped Langevin process \eqref{eq:perturbed_underdamped} is ergodic with unique invariant distribution $\widehat{\pi}$ given by \eqref{eq:augmented target}.
\\\\
As explained in Section \ref{sec:comparison}, the exponential decay estimate \eqref{eq:hypocoercive estimate} is crucial for our approach, as in particular it guarantees the well-posedness of the Poisson equation \eqref{eq:poisson_general}. 
From now on, we will therefore make the following assumption on the potential $V,$ required to prove exponential decay in $L^2(\pi)$:

\begin{assumption}
	\label{ass:bounded+Poincare}
	Assume that the Hessian of $V$ is \emph{bounded} and that the target measure $\pi(\mathrm{d}q) = \frac{1}{Z}e^{-V}\mathrm{d}q$ satisfies a \emph{Poincare inequality}, i.e. there exists a constant $\rho>0$ such that 
	\begin{equation}
	\int_{\mathbb{R}^d}\phi^2\mathrm{d}\pi \le \rho \int_{\mathbb{R}^d} \vert \nabla \phi \vert ^2 \mathrm{d}\pi, 
	\end{equation}
	holds for all $\phi \in L_{0}^2(\pi)\cap H^1(\pi)$.
\end{assumption}
Sufficient conditions on the potential so that Poincar\'{e}'s inequality holds, e.g. the Bakry-Emery criterion, are presented in~\cite{bakry2013analysis}.
\begin{theorem}
	\label{theorem:Hypocoercivity}Under Assumption \ref{ass:bounded+Poincare} there exist constants $C\ge 1$ and $\lambda>0$ such that the semigroup $(P_t)_{t\ge0}$ generated by $\gen$ satisfies exponential decay in $L^2(\pi)$ as in \eqref{eq:hypocoercive estimate}.
\end{theorem}
\begin{proof}
	See Appendix \ref{app:hypocoercivity}.
\end{proof}
\begin{remark}
	The proof uses the machinery of hypocoercivity developed in \cite{villani2009hypocoercivity}.
	However, it seems likely that using the framework of \cite{DolbeaultMouhotSchmeiser2015},
	the assumption on the boundedness of the Hessian of $V$ can be substantially
	weakened.
\end{remark}

\subsection{The Overdamped Limit}
\label{sec:overdamped}

In this section we develop a connection between the perturbed underdamped
Langevin dynamics (\ref{eq:perturbed_underdamped}) and
the nonreversible overdamped Langevin dynamics (\ref{eq:nonreversible_overdamped}). The analysis is very similar to the one presented in \cite[Section 2.2.2]{Free_energy_computations} and we will be brief. For convenience in this section we will perform the analysis on the $d$-dimensional torus $\mathbb{T}^d \cong (\mathbb{R} / \mathbb{Z})^d$, i.e. we will assume $q \in \mathbb{T}^d$.
Consider the following scaling of (\ref{eq:perturbed_underdamped}):
\begin{subequations}
\begin{eqnarray}
\mathrm{d}q_{t}^{\epsilon} & = &  \frac{1}{\epsilon}M^{-1}p_{t}^{\epsilon},\mathrm{d}t-\mu J_{1}\nabla_{q}V(q_{t})\mathrm{d}t, \\
\mathrm{d}p_{t}^{\epsilon} & = & -\frac{1}{\epsilon}\nabla_{q}V(q_{t}^{\epsilon})\mathrm{d}t-\frac{1}{\epsilon^{2}}\nu J_{2}M^{-1}p_{t}^{\epsilon}\mathrm{d}t-\frac{1}{\epsilon^{2}}\Gamma M^{-1}p_{t}^{\epsilon}\mathrm{d}t+\frac{1}{\epsilon}\sqrt{2\Gamma}\mathrm{d}W_{t},
\end{eqnarray}
\label{eq:rescaling}
\end{subequations}
valid for the small mass/small momentum regime 
\begin{equation*}
M  \rightarrow\epsilon^{2}M, \quad   p_{t}  \rightarrow\epsilon p_{t}.
\end{equation*}
Equivalently, those modifications can be obtained from subsituting
$\Gamma\rightarrow\epsilon^{-1}\Gamma$ and $t\mapsto\epsilon^{-1}t$,
and so in the limit as $\epsilon\rightarrow0$ the dynamics (\ref{eq:rescaling})
describes the limit of large friction with rescaled time. It turns
out that as $\epsilon\rightarrow0$, the dynamics (\ref{eq:rescaling})
converges to the limiting SDE 
\begin{equation}
\mathrm{d}q_{t}=-(\nu J_{2}+\Gamma)^{-1}\nabla_{q}V(q_{t})\mathrm{d}t-\mu J_{1}\nabla_{q}V(q_{t})\mathrm{d}t+(\nu J_{2}+\Gamma)^{-1}\sqrt{2\Gamma}\mathrm{d}W_{t}.\label{eq:overdamped limit}
\end{equation}
The following proposition makes this statement precise.
\begin{proposition}
	\label{prop: overdamped limit}Denote by $(q_{t}^{\epsilon},p_{t}^{\epsilon})$
	the solution to (\ref{eq:rescaling}) with (deterministic) initial
	conditions $(q_{0}^{\epsilon},p_{0}^{\epsilon})=(q_{init},p_{init})$
	and by $q_{t}^{0}$ the solution to (\ref{eq:overdamped limit}) with
	initial condition $q_{0}^{0}=q_{init}.$ For any $T>0$, $(q_{t}^{\epsilon})_{0\le t\le T}$
	converges to $(q_{t}^{0})_{0\le t\le T}$ in $L^{2}(\Omega,C([0,T]),\mathbb{T}^{d})$
	as $\epsilon\rightarrow0$, i.e. 
	\[
	\lim_{\epsilon\rightarrow0}\mathbb{E}\big(\sup_{0\le t\le T}\vert q_{t}^{\epsilon}-q_{t}^{0}\vert^{2}\big)=0.
	\]
\end{proposition}
\begin{remark}
	By a refined analysis, it is possible to get information on the rate of convergence; see, e.g.~\cite{PavlSt03,PavSt05a}.
\end{remark}
The limiting SDE (\ref{eq:overdamped limit}) is nonreversible due to the term $-\mu J_1 \nabla_q V(q_t)\mathrm{d}t$ and also because the
matrix $(\nu J_{2}+\Gamma)^{-1}$ is in general neither symmetric
nor antisymmetric.
This result, together with the fact that nonreversible perturbations
of overdamped Langevin dynamics of the form \eqref{eq:nonreversible_overdamped} are by now well-known to have improved
performance properties, motivates further investigation of the dynamics
(\ref{eq:perturbed_underdamped}).

\begin{remark}
	The limit we described in this section respects the invariant distribution,
	in the sense that the limiting dynamics (\ref{eq:overdamped limit})
	is ergodic with respect to the measure $\pi(dq)=\frac{1}{Z}e^{-V}\mathrm{d}q.$
	To see this, we have to check that (we are using the notation $\nabla$ instead of $\nabla_q$) 
	\[
	\mathcal{L}^{\dagger}(e^{-V})=-\nabla\cdot\big((\nu J_{2}+\Gamma)^{-1}\nabla e^{-V}\big)+\nabla\cdot(\mu J_{1}\nabla e^{-V})+\nabla\cdot\big((\nu J_{2}+\Gamma)^{-1}\Gamma(-\nu J_{2}+\Gamma)^{-1}\nabla e^{-V}\big)=0,
	\]
	where $\mathcal{L}^{\dagger}$ refers to the $L^{2}(\mathbb{R}^{d})$-adjoint
	of the generator of (\ref{eq:overdamped limit}), i.e. to the associated Fokker-Planck operator. Indeed, the term
	$\nabla\cdot(\mu e^{-V}J_{1}\nabla V)$ vanishes because of the
	antisymmetry of $J_{1}.$ Therefore, it remains to show that 
	\[
	\nabla\cdot\big((\nu J_{2}+\Gamma)^{-1}\Gamma(-\nu J_{2}+\Gamma)^{-1}-(\nu J_{2}+\Gamma)^{-1}\big)\nabla e^{-V}\big)=0,
	\]
	i.e. that the matrix $(\nu J_{2}+\Gamma)^{-1}\Gamma(-\nu J_{2}+\Gamma)^{-1}-(\nu J_{2}+\Gamma)^{-1}$
	is antisymmetric. Clearly, the first term is symmetric and furthermore
	it turns out to be equal to the symmetric part of the second term:
		\begin{eqnarray*}
 \frac{1}{2}\big((\nu J_{2}+\Gamma)^{-1}+(-\nu J_{2}+\Gamma)^{-1}\big) & = &
	  =\frac{1}{2}\big((\nu J_{2}+\Gamma)^{-1}(-\nu J_{2}+\Gamma)(-\nu J_{2}+\Gamma)^{-1}  \\ && + (\nu J_{2}+\Gamma)^{-1}(\nu J_{2}+\Gamma)(-\nu J_{2}+\Gamma)^{-1}\big)\\
	& = & (\nu J_{2}+\Gamma)^{-1}\Gamma(-\nu J_{2}+\Gamma)^{-1},
		\end{eqnarray*}
	so $\pi$ is indeed invariant under the limiting dynamics (\ref{eq:overdamped limit}).
\end{remark}
	



\section{Sampling from a Gaussian Distribution}
\label{sec:Gaussian}

In this section we study in detail the performance of the Langevin sampler~\eqref{eq:perturbed_underdamped} for Gaussian target densities, first considering the case of unit covariance. In particular, we study the optimal choice for the parameters in the sampler, the exponential decay rate and the asymptotic variance. We then extend our results to Gaussian target densities with arbitrary covariance matrices.

\subsection{Unit covariance - small perturbations}
\label{sec:small perturbations}

In our study of the dynamics given by \eqref{eq:perturbed_underdamped}
we first consider the simple case when $V(q)=\frac{1}{2}\vert q\vert^{2}$,
i.e. the task of sampling from a Gaussian measure with unit covariance.
We will assume $M=I$, $\Gamma=\gamma I$ and $J_{1}=J_{2}=:J$
(so that the $q-$ and $p-$dynamics are perturbed in the same way,
albeit posssibly with different strengths $\mu$ and $\nu$). Using
these simplifications, (\ref{eq:perturbed_underdamped})
reduces to the linear system 
\begin{align}
\mathrm{d}q_{t} & =p_{t}\mathrm{d}t-\mu Jq_{t}\mathrm{d}t\nonumber, \\
\mathrm{d}p_{t} & =-q_{t}\mathrm{d}t-\nu Jp_{t}\mathrm{d}t-\gamma p_{t}\mathrm{d}t+\sqrt{2\gamma}\mathrm{d}W_{t}.\label{eq:unit covariance}
\end{align}
The above dynamics are of Ornstein-Uhlenbeck type, i.e. we can write
\begin{equation}
\mathrm{d}X_{t}=-BX_{t}\mathrm{d}t+\sqrt{2Q}\mathrm{d}\bar{W}_{t}\label{eq:OU process}
\end{equation}
with $X=(q,p)^{T}$, 
\begin{equation}
B=\left(\begin{array}{cc}
\mu J & -I\\
I & \gamma I+\nu J
\end{array}\right),\label{eq:drift matrix}
\end{equation}
\begin{equation}
Q=\left(\begin{array}{cc}
\boldsymbol{0} & \boldsymbol{0}\\
\boldsymbol{0} & \gamma I
\end{array}\right)\label{eq:diffusion matrix}
\end{equation}
and $(\bar{W}_{t})_{t\ge0}$ denoting a standard Wiener process on
$\mathbb{R}^{2d}$. The generator of (\ref{eq:OU process}) is then
given by 
\begin{equation}
\mathcal{L}=-Bx\cdot\nabla+\nabla^{T}Q\nabla.\label{eq:OU generator}
\end{equation}
We will consider quadratic observables of the form 
\[
f(q)=q\cdot Kq+l\cdot q+C,
\]
with $K\in\mathbb{R}_{sym}^{d\times d}$, $l\in\mathbb{R}^{d}$ and
$C\in\mathbb{R}$, however it is worth recalling that the asymptotic variance $\sigma^2_f$ does not depend on $C$. We also stress that $f$ is assumed to be independent of
$p$ as those extra degrees of freedom are merely auxiliary. Our
aim will be to study the associated asymptotic variance $\sigma_{f}^{2}$, see equation (\ref{eq:asymptoticvariance}), in particular its
dependence on the parameters $\mu$ and $\nu$.  This dependence is encoded in the
function 
\begin{alignat*}{1}
\Theta:\quad\mathbb{R}^{2} & \rightarrow\mathbb{R}\\
(\mu,\nu) & \mapsto\sigma_{f}^{2},
\end{alignat*}
assuming a fixed observable $f$ and perturbation matrix $J$. In
this section we will focus on small perturbations, i.e. on the behaviour
of the function $\Theta$ in the neighbourhood of the origin. Our
main theoretical tool will be the Poisson equation \eqref{eq:poisson_general}, see the proofs in Appendix \ref{app:Gaussian_proofs}. Anticipating the forthcoming analysis, let us already state our main result, showing that in the neighbourhood of the origin, the function $\Theta$ has favourable properties along the diagonal $\mu=\nu$ (note that the perturbation strengths in the first and second line of \eqref{eq: unit covariance-1-1} coincide):

\begin{theorem}
	\label{cor:small pert unit var}Consider the dynamics 
\begin{align}
\mathrm{d}q_{t} & =p_{t}\mathrm{d}t-\mu Jq_{t}\mathrm{d}t,\nonumber \\
\mathrm{d}p_{t} & =-q_{t}\mathrm{d}t-\mu Jp_{t}\mathrm{d}t-\gamma p_{t}\mathrm{d}t+\sqrt{2\gamma}\mathrm{d}W_{t},\label{eq: unit covariance-1-1}
\end{align}
with $\gamma>\sqrt{2}$ and an observable of the form $f(q)=q\cdot Kq+l\cdot q+C$.
If at least one of the conditions $[J,K]\neq0$ and $l\notin\ker J$
is satisfied, then the asymptotic variance of the unperturbed sampler
is at a local maximum independently of $K$ and $J$ (and $\gamma$,
as long as $\gamma>\sqrt{2}$), i.e. 
\[
\left. \partial_{\mu}\sigma_{f}^{2} \right\rvert_{\mu =0}=0
\]
and 
\[
\left. \partial_{\mu}^{2}\sigma_{f}^{2}\right\rvert_{\mu = 0}<0.
\]
\end{theorem}



\subsubsection{\label{sub:Purely-quadratic-observables}Purely quadratic observables}

Let us start with the case $l=0$, i.e. $f(q)=q\cdot Kq+C$. The following
holds: 
\begin{proposition}
	\label{thm: local quadratic observable}The function $\Theta$ satisfies
	\begin{equation}
	\left. \nabla\Theta\right\rvert_{(\mu,\nu)=(0,0)}=0\label{eq:gradTheta}
	\end{equation}
	and 
	\begin{equation}
	\left. \Hess\Theta\right\rvert_{(\mu,\nu)=(0,0)}=\left(\begin{array}{cc}
	-(\gamma+\frac{1}{\gamma^{3}}+\gamma^{3})\left(\Tr(JKJK)-\Tr(J^{2}K^{2})\right) & (\frac{1}{\gamma^{3}}+\frac{1}{\gamma}-\gamma)\Tr(J^{2}K^{2})\\
	-\frac{2}{\gamma}\Tr(JKJK) & +(-\frac{1}{\gamma^{3}}+\frac{1}{\gamma}+\gamma)\Tr(JKJK)\\
	(\frac{1}{\gamma^{3}}+\frac{1}{\gamma}-\gamma)\Tr(J^{2}K^{2}) & (\frac{1}{\gamma^{3}}-\frac{1}{\gamma})\Tr(J^{2}K^{2})\\
	+(-\frac{1}{\gamma^{3}}+\frac{1}{\gamma}+\gamma)\Tr(JKJK) & -(\frac{1}{\gamma^{3}}+\frac{1}{\gamma})\Tr(JKJK)
	\end{array}\right).\label{eq:HessTheta}
	\end{equation}
\end{proposition}
\begin{proof}
	See Appendix \ref{app:Gaussian_proofs}.
	\qed
\end{proof}
The above proposition shows that the unperturbed dynamics represents a
critical point of $\Theta$, independently of the choice of $K$,
$J$ and $\gamma$. In general though, $\Hess\Theta\vert_{(\mu,\nu)=(0,0)}$
can have both positive and negative eigenvalues. In particular, this
implies that an unfortunate choice of the perturbations will actually
increase the asymptotic variance of the dynamics (in contrast to the situation
of perturbed \emph{overdamped }Langevin dynamics, where any nonreversible
perturbation leads to an improvement in asymptotic variance as detailed
in \cite{asvar_Hwang} and \cite{duncan2016variance}). Furthermore, the nondiagonality
of $\Hess\Theta\vert_{(\mu,\nu)=(0,0)}$ hints at the fact that the
interplay of the perturbations $J_{1}$ and $J_{2}$ (or rather their
relative strengths $\mu$ and $\nu$) is crucial for the performance
of the sampler and, consequently, the effect of these perturbations cannot be satisfactorily
studied independently. 
\begin{example}
	Assuming $J^{2}=-I$ and $[J,K]=0$ it follows that
	\[
	\left. \partial_{\mu}^{2}\Theta\right\rvert_{\mu=0}=\left. \partial_{\nu}^{2}\Theta\right\rvert_{\mu=0}=\frac{1}{\gamma}\Tr(K^{2})>0,
	\]
	for all nonzero $K$. Therefore in this case, a small perturbation of $J_{1}$ only or $J_{2}$ only will increase  the asymptotic variance, uniformly over all choices of $K$ and $\gamma$.
\end{example}
However, it turns out that it is possible to construct an improved sampler
by combining both perturbations in a suitable way. Indeed,
the function $\Theta$ can be seen to have good properties along $\mu=\nu$. We set $\mu(s)=s$, $\nu(s):=s$ and compute

\begin{align*}
\left. \frac{\mathrm{d}^{2}}{\mathrm{d}s^{2}}\Theta\right\rvert_{s=0} & =(1,1)\cdot\Hess\Theta\vert_{(\mu,\nu)=(0,0)}(1,1)\\
& =-(\gamma+\frac{1}{\gamma^{3}}+\gamma^{3})\left(\Tr(JKJK)-\Tr(J^{2}K^{2})\right)-\frac{2}{\gamma}\Tr(JKJK)\\
& +2\cdot\left((\frac{1}{\gamma^{3}}+\frac{1}{\gamma}-\gamma)\Tr(J^{2}K^{2})+(-\frac{1}{\gamma^{3}}+\frac{1}{\gamma}+\gamma)\Tr(JKJK)\right)\\
& +(\frac{1}{\gamma^{3}}-\frac{1}{\gamma})\Tr(J^{2}K^{2})-(\frac{1}{\gamma^{3}}+\frac{1}{\gamma})\Tr(JKJK)\\
& =\big(\gamma-\frac{4}{\gamma^{3}}-\gamma^{3}-\frac{1}{\gamma}\big)\cdot(\Tr(JKJK)-\Tr(J^{2}K^{2}))\le0.
\end{align*}
The last inequality follows from 
\[
\gamma-\frac{4}{\gamma^{3}}-\gamma^{3}-\frac{1}{\gamma}<0
\]
and 
\[
\Tr(JKJK)-\Tr(J^{2}K^{2})\ge0
\]
(both inequalities are proven in the Appendix, Lemma \ref{lem:basic_inequalities}), where the last inequality
is strict if $[J,K]\neq0$. Consequently, choosing both perturbations
to be of the same magnitude ($\mu=\nu$) and assuring that $J$ and
$K$ do not commute always leads to a smaller asymptotic variance,
independently of the choice of $K$, $J$ and $\gamma$. We state
this result in the following corrolary:
\begin{corollary}
	\label{cor:unit covariance quadratic obs}Consider the dynamics 
	\begin{align}
	\mathrm{d}q_{t} & =p_{t}\mathrm{d}t-\mu Jq_{t}\mathrm{d}t\nonumber, \\
	\mathrm{d}p_{t} & =-q_{t}\mathrm{d}t-\mu Jp_{t}\mathrm{d}t-\gamma p_{t}\mathrm{d}t+\sqrt{2\gamma}\mathrm{d}W_{t},\label{eq: unit covariance-1}
	\end{align}
 and a quadratic observable $f(q)=q\cdot Kq+C$. If $[J,K] \neq 0,$
	then the asymptotic variance of the unperturbed sampler is at a local
	maximum independently of K, $J$ and $\gamma$, i.e. 
	\[
	\left. \partial_{\mu}\sigma_{f}^{2}\right\rvert_{\mu=0}=0
	\]
	and 
	\[
	\left. \partial_{\mu}^{2}\sigma_{f}^{2}\right\rvert_{\mu=0}<0.
	\]
\end{corollary}
\begin{remark}
	As we will see in Section \ref{sec:large perturbations}, more precisely Example \ref{ex:commutation quadratic observables}, if $[J,K]=0,$
	the asymptotic variance is constant as a function of $\mu$, i.e.
	the perturbation has no effect.\end{remark}
\begin{example}
\label{ex:opposed perturbation}
	Let us set $\mu(s):=s$ and $\nu(s):=-s$ (this corresponds to a small
	perturbation with $J\nabla V(q_{t})\mathrm{d}t$ in $q$ and $-Jp_{t}\mathrm{d}t$
	in $p$). In this case we get 
	\[
	\left. \frac{\mathrm{d}^{2}\Theta}{\mathrm{d}s^{2}} \right\rvert_{s=0}=\underbrace{-\frac{1}{2}\cdot\frac{\gamma^{4}+3\gamma^{2}+5}{\gamma}\left(\Tr(JKJK)-\Tr(J^{2}K^{2})\right)}_{\le0}\underbrace{-4\frac{\Tr(J^{2}K^{2})}{\gamma}}_{\ge0},
	\]
	which changes its sign depending on $J$ and $K$ as the first term
	is negative and the second is positive. Whether the perturbation improves the performance of the sampler in terms of asymptotic variance therefore depends on the specifics of the observable and the perturbation in this case.
	
\end{example}

\subsubsection{Linear observables}

Here we consider the case $K=0$, i.e. $f(q)=l\cdot q+C$, where
again $l\in\mathbb{R}^{d}$ and $C\in\mathbb{R}$. We have the following
result: 
\begin{proposition}
	\label{thm:linear_full_J}The function $\Theta$ satisfies 
	\[
	\nabla\Theta\vert_{(\mu,\nu)=(0,0)}=0
	\]
	and 
	\[
	\Hess\Theta\vert_{(\mu,\nu)=(0,0)}=\left(\begin{array}{cc}
	-2\gamma^{3}\vert Jl\vert^{2} & 2\gamma\vert Jl\vert^{2}\\
	2\gamma\vert Jl\vert^{2} & 0
	\end{array}\right).
	\]
\end{proposition}
\begin{proof}
	See Appendix \ref{app:Gaussian_proofs}.
	\qed
\end{proof}
	Let us assume that $l\notin\ker J$. Then $\partial_{\mu}^{2}\Theta\vert_{\mu,\nu=0}<0$,
	and hence Theorem \ref{thm:linear_full_J} shows that a small perturbation
	by $\mu J\nabla V(q_{t})\mathrm{d}t$ alone always results in an improvement
	of the asymptotic variance. However, if we combine both perturbations
	$\mu J\nabla V(q_{t})\mathrm{d}t$ and $\nu Jp_{t}\mathrm{d}t$, then
	the effect depends on the sign of 
	\[
	\left(\begin{array}{cc}
	\mu & \nu\end{array}\right)\left(\begin{array}{cc}
	-2\gamma^{3}\vert Jl\vert^{2} & 2\gamma\vert Jl\vert^{2}\\
	2\gamma\vert Jl\vert^{2} & 0
	\end{array}\right)\left(\begin{array}{c}
	\mu\\
	\nu
	\end{array}\right)=-(2\mu^{2}\gamma^{3}-4\mu\nu\gamma)\vert Jl\vert^{2}.
	\]
	This will be negative if $\mu$ and $\nu$ have different signs, and
	also if they have the same sign and $\gamma$ is big enough.
Following Section \ref{sub:Purely-quadratic-observables}, we require
$\mu=\nu$. We then end up with the requirement 
\[
2\mu^{2}\gamma^{3}-4\mu\nu\gamma>0,
\]
which is satisfied if $\gamma>\sqrt{2}$

Summarizing the results of this section, for observables of the form
$f(q)=q\cdot Kq+l\cdot q+C$, choosing equal perturbations ($\mu=\nu$)
with a sufficiently strong damping $(\gamma>\sqrt{2}$) always leads
to an improvement in asymptotic variance under the conditions $[J,K]\neq0$
and $l\notin\ker J$. This is finally the content of Theorem \ref{cor:small pert unit var}.
\begin{figure}
	\begin{subfigure}[b]{0.5 \textwidth}
		\includegraphics[width=\textwidth]{q_equal}
		\caption{Equal perturbations: $\mu=\nu$}
		\label{fig:asym_quad}
	\end{subfigure}
	\hfill
	\begin{subfigure}[b]{0.5 \textwidth}
		\includegraphics[width=\textwidth]{q_09}
		\caption{Approximately equal perturbations: $\mu=0.9\nu$}
		\label{fig:no_limit1}
	\end{subfigure}
	\hfill
	\begin{subfigure}[b]{0.5 \textwidth}
		\includegraphics[width=\textwidth]{q_opp}
		\caption{Opposing perturbations: $\mu=-\nu$ 
			\label{fig:no_limit2}}
	\end{subfigure}
	\hfill
	\begin{subfigure}[b]{0.5 \textwidth}
		\includegraphics[width=\textwidth]{l_equalg25.pdf}
		\caption{Equal perturbations: $\mu=\nu$ (sufficiently large friction $\gamma$)}
		\label{fig:lin_large_friction}
	\end{subfigure}
	\hfill
	\begin{subfigure}[b]{0.5 \textwidth}
		\includegraphics[width=\textwidth]{l_equal_g1.pdf}
		
		\caption{Equal perturbations: $\mu=\nu$ (small friction $\gamma$)}
		\label{fig:lin_small_friction}
	\end{subfigure}
	\caption{Asymptotic variance for linear and quadratic observables, depending on relative perturbation and friction strengths}
	\label{fig:linear and quadratic observables}
\end{figure}

Let us illustrate the results of this section by plotting the asymptotic variance as a function of the perturbation strength $\mu$ (see Figure \ref{fig:linear and quadratic observables}), making the choices $d=2$, $l=(1,1)^{T}$,
\begin{equation}
K=\left(\begin{array}{cc}
2 & 0\\
0 & 1
\end{array}\right)
\quad \text{and} \quad
J=\left(\begin{array}{cc}
0 & 1\\
-1 & 0
\end{array}\right).
\end{equation}
The asymptotic variance has been computed according to \eqref{eq:Gaussian asymvar}, using \eqref{eq:Lyapunov equation} and \eqref{eq:linear condition} from Appendix \ref{app:Gaussian_proofs}. The graphs confirm the results summarized in Corollary \ref{cor:small pert unit var} concerning the asymptotic variance in the neighbourhood of the unperturbed dynamics ($\mu = 0$). Additionally, they give an impression of the global behaviour, i.e. for larger values of $\mu$.

Figures \ref{fig:asym_quad}, \ref{fig:no_limit1} and \ref{fig:no_limit2}  show the asymptotic variance associated with the quadratic observable $f(q)=q\cdot K q$. In accordance with Corollary \ref{cor:unit covariance quadratic obs}, the asymptotic variance is at a  local maximum at zero perturbation in the case $\mu=\nu$ (see Figure \ref{fig:asym_quad}). For increasing perturbation strength, the graph shows that it decays monotonically
and reaches a limit for $\mu\rightarrow\infty$ (this limiting behaviour will be explored analytically in Section \ref{sec:large perturbations}). If the condition $\mu=\nu$ is only approximately satisfied (Figure \ref{fig:no_limit1}), our numerical examples still exhibits decaying asymptotic variance in the neighbourhood of the critical point. In this case, however, the asymptotic variance diverges for growing values of the perturbation $\mu$. If the perturbations are opposed ($\mu=-\nu$) as in Example \ref{ex:opposed perturbation}, it is possible for certain observables that the unperturbed dynamics represents a global minimum. Such a case is observed in Figure \ref{fig:no_limit2}. In Figures \ref{fig:lin_large_friction} and \ref{fig:lin_small_friction} the observable $f(q)=l\cdot q$ is considered. If the damping is sufficiently strong ($\gamma > \sqrt{2}$), the unperturbed dynamics is at a local maximum of the asymptotic variance (Figure \ref{fig:lin_large_friction}). Furthermore, the asymptotic variance approaches zero as $\mu \rightarrow \infty$ (for a theoretical explanation see again Section \ref{sec:large perturbations}). The graph in Figure \ref{fig:lin_small_friction} shows that the assumption of $\gamma$ not being too small cannot be dropped from Corollary \ref{cor:small pert unit var}. Even in this case though the example shows decay of the asymptotic variance for large values of $\mu$.    
\subsection{Exponential decay rate}
\label{sec:exp_decay}
Let us denote by $\lambda^{*}$ the \emph{optimal exponential decay rate} in \eqref{eq:hypocoercive estimate}, i.e.
\begin{equation}
\lambda^{*}=\sup\{\lambda > 0 \, \vert \, \text{There exists } C\ge 1 \text{ such that } \eqref{eq:hypocoercive estimate} \text{ holds}\}.
\end{equation}
Note that $\lambda^{*}$ is well-defined and positive by Theorem \ref{theorem:Hypocoercivity}. We also define the \emph{spectral bound} of the generator $\gen$ by
\begin{equation}
s(\gen)=\inf(\text{Re}\,\sigma(-\gen)\setminus\{0\}).
\end{equation} 
In \cite{Metafune_formula} it is proven that the Ornstein-Uhlenbeck semigroup $(P_t)_{t\ge0}$ considered in this section is differentiable (see Proposition 2.1). In this case (see Corollary 3.12 of \cite{Engel2000Semigroup}), it is known that the exponential decay rate and the spectral bound coincide, i.e. $\lambda^{*}=s(\gen)$, whereas in general only $\lambda^{*}\le s(\gen)$ holds.
In this section we will therefore analyse the spectral properties of the generator
(\ref{eq:OU generator}). In particular, this leads to some intuition
of why choosing equal perturbations ($\mu=\nu$) is crucial for the
performance of the sampler.

In \cite{Metafune_formula} (see also \cite{OPP12}), it was proven that
the spectrum of $\mathcal{L}$ as in (\ref{eq:OU generator}) in $L^{2}(\widehat{\pi})$
is given by 
\begin{equation}
\sigma(\mathcal{L})=\left\{-\sum_{j=1}^{r}n_{j}\lambda_{j}:\, n_{j}\in\mathbb{N},\lambda_{j}\in \sigma(B)\right\}.\label{eq:Metafune formula}
\end{equation}
Note that $\sigma(\mathcal{L})$ only depends on the drift matrix
$B$. In the case where $\mu=\nu$,
the spectrum of $B$ can be computed explicitly. 
\begin{lemma}
	\label{lem:drift matrix properties}Assume $\mu=\nu$. Then the spectrum
	of $B$ is given by
	\begin{equation}
	\sigma(B)=\left\{\mu\lambda+\sqrt{\big(\frac{\gamma}{2}\big)^{2}-1}+\frac{\gamma}{2}\vert\lambda\in\sigma(J)\}\cup\{\mu\lambda-\sqrt{\big(\frac{\gamma}{2}\big)^{2}-1}+\frac{\gamma}{2}\vert\lambda\in\sigma(J)\right\}.\label{eq:spectrum of B}
	\end{equation}
\end{lemma}
\begin{proof}
	We will compute $\sigma\big(B-\frac{\gamma}{2}I\big)$ and then use
	the identity
	\begin{equation}
	\sigma(B)=\left\{\lambda+\frac{\gamma}{2}\vert\lambda\in\sigma\left(B-\frac{\gamma}{2}I\right)\right\}.\label{eq:shift spectrum}
	\end{equation}
	We have 
	\begin{align*}
	\det\left(B-\frac{\gamma}{2}I-\lambda I\right) & =\det\left(\left(\mu J-\frac{\gamma}{2}I-\lambda I\right)\left(\mu J+\frac{\gamma}{2}I-\lambda I\right)+I\right)\\
	& =\det\left((\mu J-\lambda I)^{2}-\left(\frac{\gamma}{2}\right)^{2}I+I\right)\\
	& =\det\left(\left(\mu J-\lambda I+\sqrt{\left(\frac{\gamma}{2}\right)^{2}-1} I\right)\cdot\left(\mu J-\lambda I-\sqrt{\left(\frac{\gamma}{2} \right)^{2}-1} I\right)\right)\\
	& =\det\left(\mu J-\lambda I+\sqrt{\left(\frac{\gamma}{2}\right)^{2}-1} I\right)\cdot\det\left(\mu J-\lambda I-\sqrt{\left(\frac{\gamma}{2}\right)^{2}-1} I\right),
	\end{align*}
	where $I$ is understood to denote the identity matrix of appropriate dimension.
	The above quantity is zero if and only if 
	\[
	\lambda-\sqrt{\left(\frac{\gamma}{2}\right)^{2}-1}\in\sigma(\mu J)
	\]
	or 
	\[
	\lambda+\sqrt{\left(\frac{\gamma}{2}\right)^{2}-1}\in\sigma(\mu J).
	\]
	Together with (\ref{eq:shift spectrum}), the claim follows.
	\qed
\end{proof}
Using formula \eqref{eq:Metafune formula}, in Figure \ref{fig:good_spectrum} we show a sketch of the spectrum $\sigma(-\mathcal{L}$)
for the case of equal perturbations ($\mu=\nu)$ with the convenient
choices $n=1$ and $\gamma=2.$ Of course, the eigenvalue at $0$ is
associated to the invariant measure since $\sigma(-\mathcal{L})=\sigma(-\mathcal{L}^{\dagger})$
and $\mathcal{L}^{\dagger}\widehat{\pi}=0$, where $\mathcal{L}^{\dagger}$ denotes the Fokker-Planck operator, i.e. the $L^2(\mathbb{R}^{2d})$-adjoint of $\mathcal{L}$. The arrows indicate the movement
of the eigenvalues as the perturbation $\mu$ increases in accordance
with Lemma \ref{lem:drift matrix properties}. Clearly, the spectral
bound of $\gen$ is not affected by the perturbation. 
Note that the eigenvalues on the real axis stay invariant under the
perturbation. The subspace of $L_{0}^{2}(\widehat{\pi})$ associated to
those will turn out to be crucial for the characterisation of the
limiting asymptotic variance as $\mu\rightarrow\infty$.

To illustrate the suboptimal properties of the perturbed dynamics
when the perturbations are not equal, we plot the spectrum of the
drift matrix $\sigma(B)$ in the case when the dynamics is only perturbed
by the term $\nu J_{2}p\mathrm{d}t$ (i.e. $\mu=0$) for $n=2$, $\gamma=2$ and
\begin{equation}
J_2=\left(\begin{array}{cc}
0 & -1\\
1 & 0
\end{array}\right),
\end{equation} 
(see Figure \ref{fig:bad_spectrum}). Note that the full spectrum $\sigma(-\mathcal{L})$
can be inferred from (\ref{eq:Metafune formula}). For $\nu=0$ we have that the spectrum $\sigma(B)$
only consists of the (degenerate) eigenvalue $1$. For increasing
$\nu$, the figure shows that the degenerate eigenvalue splits up
into four eigenvalues, two of which get closer to the imaginary axis as $\nu$ increases, leading to a smaller spectral
bound and therefore to a decrease in the speed of convergence to equilibrium.
Figures (\ref{fig:good_spectrum}) and (\ref{fig:bad_spectrum}) give an intuitive explanation
of why the fine-tuning of the perturbation strengths is crucial.

\begin{figure}
	\begin{subfigure}[b]{0.45 \textwidth}
		\includegraphics[width=\textwidth]{spectrum2.pdf}
		\caption{$\sigma(-\gen)$ in the case $\mu=\nu$. The arrows indicate the movement of the spectrum as the perturbation strength $\mu$ increases.\label{fig:good_spectrum}}
	\end{subfigure}
	\hfill
	\begin{subfigure}[b]{0.45 \textwidth}
		\includegraphics[width=\textwidth]{moving2.pdf}
		\caption{$\sigma(B)$ in the case $J_{1}=0$, i.e. the dynamics is only perturbed
			by $-\nu J_{2}p\mathrm{d}t$. The arrows indicate the movement of
			the eigenvalues as $\nu$ increases.\label{fig:bad_spectrum}}
	\end{subfigure}
	.	\caption{Effects of the perturbation on the spectra of $-\gen$ and $B$.}
	\label{fig:examples-introduction}
\end{figure}

\subsection{Unit covariance - large perturbations}
\label{sec:large perturbations}

In the previous subsection we observed that for the particular perturbation $J_1 = J_2$ and $\mu = \nu$, i.e. 
\begin{align}
\mathrm{d}q_{t} & =p_{t}\mathrm{d}t-\mu Jq_{t}\mathrm{d}t\nonumber \\
\mathrm{d}p_{t} & =-q_{t}\mathrm{d}t-\mu Jp_{t}\mathrm{d}t-\gamma p_{t}\mathrm{d}t+\sqrt{2\gamma}\,\mathrm{d}W_{t},\label{eq: unit covariance perfect perturbation}
\end{align}
the perturbed Langevin dynamics demonstrated an improvement in performance for $\mu$ in a neighbourhood of $0$, when the observable is linear or quadratic.  Recall that this dynamics is ergodic with respect to a standard Gaussian measure $\widehat{\pi}$ on $\mathbb{R}^{2d}$ with marginal $\pi$  with respect to the $q$--variable.  In the following we shall consider only observables that do not depend on $p$. Moreover, we assume without loss of generality that $\pi(f)=0$. For such an observable we will write $f\in L^2_0(\pi)$ and assume the canonical embedding $L^2_0(\pi)\subset L^2(\widehat{\pi})$.  The infinitesimal generator of (\ref{eq: unit covariance perfect perturbation})
is given by 
\begin{equation}
\label{eq:generator_equal}
\mathcal{L}=\underbrace{p\cdot\nabla_{q}-q\cdot\nabla_{p}+\gamma(-p\cdot\nabla_{p}+\Delta_{p})}_{\mathcal{L}_{0}}+\mu\underbrace{(-Jq\cdot\nabla_{q}-Jp\cdot\nabla_{p})}_{\mathcal{A}}=:\mathcal{L}_{0}+\mu\mathcal{A},
\end{equation}
where we have introduced the notation $\mathcal{L}_{pert}=\mu \mathcal{A}$. In the sequel, the adjoint of an operator $B$ in $L^2(\widehat{\pi})$ will be denoted by $B^{*}$. In the rest of this section we will make repeated use of the Hermite polynomials
\begin{equation}
g_{\alpha}(x)=(-1)^{\vert\alpha\vert}e^{\frac{\vert x\vert^{2}}{2}}\nabla^{\alpha}e^{-\frac{\vert x\vert^{2}}{2}},\quad\alpha\in\mathbb{N}^{2d},\label{eq: Hermite polynomials}
\end{equation}
invoking the notation $x=(q,p)\in\mathbb{R}^{2d}$. For $m\in\mathbb{N}_{0}$
define the spaces 
\[
\label{eq:Hermite spaces}
H_{m}=\Span\{g_{\alpha}:\,\vert\alpha\vert=m\},
\]
with induced scalar product 
\[
\langle f,g\rangle_{m}:=\langle f,g\rangle_{L^2(\widehat{\pi})},\quad f,g\in H_{m}.
\]
The space $(H_{m},\langle\cdot,\cdot\rangle_{m})$ is then a real Hilbert
space with (finite) dimension
\[
\dim H_{m}=\left(\begin{array}{c}
m+2d-1\\
m
\end{array}\right).
\]
The following result (Theorem \ref{thm:L2 decomposition}) holds for operators of the form
\begin{equation}
\label{eq:OU_operator}
\mathcal{L}=-Bx\cdot\nabla+\nabla^{T}Q\nabla,
\end{equation}
where the quadratic drift and diffusion matrices $B$ and $Q$ are such that $\mathcal{L}$ is the generator of an ergodic stochastic process (see \cite[Definition 2.1]{Arnold2014} for precise conditions on $B$ and $Q$ that ensure ergodicity). The generator of the SDE \eqref{eq: unit covariance perfect perturbation} is given by \eqref{eq:OU_operator} with $B$ and $Q$  as in equations \eqref{eq:drift matrix}
and \eqref{eq:diffusion matrix}, respectively.  The following result provides an orthogonal decomposition of $L^{2}(\widehat{\pi})$ into invariant subspaces of the operator $\mathcal{L}$.
\begin{theorem}{\cite[Section 5]{Arnold2014}.}
	\label{thm:L2 decomposition}The following holds:
	\begin{enumerate}[label=(\alph*)]
		\item The space $L^{2}(\widehat{\pi})$ has a decomposition into mutually orthogonal
		subspaces:
		\[
		L^{2}(\widehat{\pi})=\bigoplus_{m\in\mathbb{N}_{0}}H_{m}.
		\]
		
		\item For all $m\in\mathbb{N}_{0}$, $H_{m}$ is invariant under $\mathcal{L}$
		as well as under the semigroup $(e^{-t\mathcal{L}})_{t\ge0}$. 
		\item The spectrum of $\mathcal{L}$ has the following decomposition:
		\[
		\sigma(\mathcal{L})=\bigcup_{m\in\mathbb{N}_{0}}\sigma(\mathcal{L}\vert_{H_{m}}),
		\]
		where 
		\begin{equation}
		\sigma(\mathcal{L}\vert_{H_{m}})=\left\lbrace\sum_{j=1}^{2d}\alpha_{j}\lambda_{j}:\,\vert\alpha\vert=m,\,\lambda_{j}\in\sigma(B)\right\rbrace.\label{eq:spectrum on subspaces}
		\end{equation}
		
	\end{enumerate}
\end{theorem}
\begin{remark}
	Note that by the ergodicity of the dynamics, $\ker\mathcal{L}$ consists of constant functions and so $\ker\mathcal{L}=H_{0}$. Therefore, $L^2_0(\widehat{\pi})$ has the decomposition
	\[
	L_{0}^{2}(\widehat{\pi})=L^{2}(\widehat{\pi})/\ker\mathcal{L}=\bigoplus_{m\ge1}H_{m}.
	\]
	\end{remark}
Our first main result of this section is an expression for the asymptotic
variance in terms of the unperturbed operator $\mathcal{L}_{0}$ and
the perturbation $\mathcal{A}$:
\begin{proposition}
	\label{prop:asymvar_op_formula}
	Let $f\in L_{0}^{2}(\pi)$  (so in particular $f=f(q)$).
	Then the associated asymptotic variance is given by 
	\begin{equation}
	\label{eq:asymvar_op_formula}
	\sigma_{f}^{2}=\langle f,-\mathcal{L}_{0}(\mathcal{L}_{0}^{2}+\mu^{2}\mathcal{A}^{*}\mathcal{A})^{-1}f\rangle_{L^{2}(\widehat{\pi})}.
	\end{equation}
	
\end{proposition}
\begin{remark}
The proof of the preceding Proposition will show that $\mathcal{L}_{0}^{2}+\mu^{2}\mathcal{A}^{*}\mathcal{A}$ is invertible on $L^2_0(\widehat{\pi})$ and that $(\mathcal{L}_{0}^{2}+\mu^{2}\mathcal{A}^{*}\mathcal{A})^{-1}f \in \mathcal{D}(\mathcal{L}_0)$ for all $f \in L^2_0(\widehat{\pi})$. 
\end{remark}
To prove Proposition \ref{prop:asymvar_op_formula} we will make use of the \emph{generator
	with reversed perturbation} 
\[
\mathcal{L}_{-}=\mathcal{L}_{0}-\mu\mathcal{A}
\]
and the \emph{momentum flip operator} 
\begin{align*}
P:L_{0}^{2}(\widehat{\pi}) & \rightarrow L_{0}^{2}(\widehat{\pi})\\
\phi(q,p) & \mapsto\phi(q,-p).
\end{align*}
Clearly, $P^{2}=I$ and $P^{*}=P$. Further properties of $\mathcal{L}_{0}$,
$\mathcal{A}$ and the auxiliary operators $\mathcal{L}_{-}$ and
$P$ are gathered in the following lemma:
\begin{lemma}
	\label{operator lemma}
	For all $\phi, \psi \in C^{\infty}(\mathbb{R}^{2d})\cap L^2(\widehat{\pi})$ the following holds:
	\begin{enumerate}[label=(\alph*)]
		\item \label{it:oplem1} The generator $\mathcal{L}_{0}$ is symmetric in $L^2(\widehat{\pi})$ with respect to $P$:
		\[
		\langle  \phi, P\mathcal{L}_{0}P \psi\rangle_{L^2(\widehat{\pi})}=\langle \mathcal{L}_{0} \phi, \psi \rangle_{L^2(\widehat{\pi})}.
		\]
		
		\item \label{it:oplem2} The perturbation $\mathcal{A}$ is skewadjoint in $L^{2}(\widehat{\pi})$:
		\[ 
		\mathcal{A}^{*} = -\mathcal{A}.
		\]
		
		\item \label{it:oplem3} The operators $\mathcal{L}_{0}$ and $\mathcal{A}$ commute:
		\[
		[\mathcal{L}_{0},\mathcal{A}]\phi=0.
		\]
		
		\item \label{it:oplem4} The perturbation $\mathcal{A}$ satisfies
		\[
		P\mathcal{A}P\phi=\mathcal{A}\phi.
		\]
		
		\item \label{it:oplem5} $\mathcal{L}$ and $\mathcal{L}_{-}$ commute,
		\begin{equation*}
		[\mathcal{L},\mathcal{L}_{-}]\phi = 0,
		\end{equation*}
		
		 and the following relation holds:
		\begin{equation}
		\langle \phi ,P\mathcal{L}P\psi\rangle_{L^{2}(\widehat{\pi})}=\langle\mathcal{L}_{-}\phi,\psi\rangle_{ L^{2}(\widehat{\pi})}.\label{eq:L+L-}
		\end{equation}
		\item \label{it:oplem6} 
		The operators $\mathcal{L}$, $\mathcal{L}_0$, $\mathcal{L}_{-}$, $\mathcal{A}$ and $P$ leave the Hermite spaces $H_m$ invariant.
	\end{enumerate}
\end{lemma}
\begin{remark}
	The claim \ref{it:oplem3} in the above lemma is crucial for our approach, which
	itself rests heavily on the fact that the $q-$ and $p-$perturbations
	match ($J_{1}=J_{2}$).
\end{remark}
\begin{proof}[of Lemma \ref{operator lemma}]
	To prove \ref{it:oplem1}, consider the following
	decomposition of $\mathcal{L}_{0}$ as in (\ref{eq:generator}):
	\[
	\mathcal{L}_{0}=\underbrace{p\cdot\nabla_{q}-q\cdot\nabla_{p}}_{\mathcal{L}_{ham}}+\underbrace{\gamma\left(- p\cdot\nabla_{p}+ \Delta_{p}\right)}_{\mathcal{L}_{therm}}.
	\]
	By partial integration it is straightforward to see that 
	\begin{equation*}
	\langle\phi,\mathcal{L}_{ham}\psi\rangle_{ L^{2}(\widehat{\pi})}=-\langle\mathcal{L}_{ham}\phi,\psi\rangle_{ L^{2}(\widehat{\pi})}
	\end{equation*}
	and
	\begin{equation*}
	 \langle \phi,\mathcal{L}_{therm}\psi\rangle_{ L^{2}(\widehat{\pi})}=\langle\mathcal{L}_{therm}\phi,\psi\rangle_{ L^{2}(\widehat{\pi})},
	 \end{equation*}
	 for all $\phi,\psi \in C^{\infty}(\mathbb{R}^{2d})\cap L^2(\widehat{\pi})$,
	  i.e. $\mathcal{L}_{ham}$ and $\mathcal{L}_{therm}$
	are antisymmetric and symmetric in $L^{2}(\widehat{\pi})$ respectively.
	Furthermore, we immediately see that $P\mathcal{L}_{ham}P\phi=-\mathcal{L}_{ham}\phi$ and $P\mathcal{L}_{therm}P\phi = \mathcal{L}_{therm}\phi$, so that
	\[
	\langle \phi,P\mathcal{L}_{0}P\psi\rangle_{ L^{2}(\widehat{\pi})}=\langle\phi,-\mathcal{L}_{ham}\psi+\mathcal{L}_{therm}\psi\rangle_{ L^{2}(\widehat{\pi})}=\langle\mathcal{L}_{0}\phi,\psi\rangle_{ L^{2}(\widehat{\pi})}.
	\]
	We note that this result holds in the more general setting of Section \ref{sec:perturbed_langevin} for the infinitesimal generator \eqref{eq:generator}.  The claim \ref{it:oplem2} follows by noting that the flow vector field $b(q,p)=(-Jq,-Jp)$ associated to $\mathcal{A}$ is divergence-free with respect to $\widehat{\pi}$, i.e. $\nabla \cdot(\widehat{\pi}b)=0$. Therefore, $\mathcal{A}$ is the generator of a strongly continuous unitary semigroup on $L^2(\widehat{\pi})$ and hence skewadjoint by Stone's Theorem.
  To prove \ref{it:oplem3} we use the decomposition $\mathcal{L}_{0}=\mathcal{L}_{ham}+\mathcal{L}_{therm}$ to obtain
	\begin{equation}
	\label{eq:oplemmac_proof}
	[\mathcal{L}_{0},\mathcal{A}]\phi=[\mathcal{L}_{ham},\mathcal{A}]\phi+[\mathcal{L}_{therm},\mathcal{A}]\phi,\quad \phi \in C^\infty(\mathbb{R}^{2d})\cap L^2(\widehat{\pi}).
	\end{equation}
	The first term of \eqref{eq:oplemmac_proof} gives 
	\begin{align*}
	[p\cdot\nabla_{q}-q\cdot\nabla_{p}&,-Jq\cdot\nabla_{q} -Jp\cdot\nabla_{p}]\phi\\
	& =\big([p\cdot\nabla_{q},-Jq\cdot\nabla_{q}]+[p\cdot\nabla_{q},-Jp\cdot\nabla_{p}]+[-q\cdot\nabla_{p},-Jq\cdot\nabla_{q}] \\
	& \qquad +[-q\cdot\nabla_{p},-Jp\cdot\nabla_{p}]\big)\phi\\
	&= Jp\cdot\nabla_{q}\phi-Jp\cdot\nabla_{q}\phi+Jq\cdot\nabla_{p}\phi-Jq\cdot\nabla_{p}\phi=0.
	\end{align*}
	The second term of \eqref{eq:oplemmac_proof} gives 
	\begin{equation}
	\label{eq:term1}
	[-p\cdot\nabla_{p}+\Delta_{p},\mathcal{A}]\phi =[-p\cdot\nabla_{p},-Jp\cdot\nabla_{p}]\phi+[\Delta_{p},-Jp\cdot\nabla_{p}]\phi,
	\end{equation}
	since $Jq\cdot\nabla_{q}$ commutes with $p\cdot\nabla_{p}+\Delta_{p}$. Both  terms in \eqref{eq:term1} are clearly zero due the antisymmetry of $J$ and the symmetry of the Hessian $D^2_p \phi$. 
	\\\\
	The claim \ref{it:oplem4} follows from a short calculation similar to the proof of  \ref{it:oplem1}.  To prove \ref{it:oplem5}, note that the fact that $\mathcal{L}$ and $\mathcal{L}_{-}$ commute follows from \ref{it:oplem3}, as 
	\[
	[\mathcal{L},\mathcal{L}_{-}]\phi=[\mathcal{L}_{0}+\mu\mathcal{A},\mathcal{L}_{0}-\mu\mathcal{A}]\phi=-2\mu[\mathcal{L}_{0},\mathcal{A}]\phi=0,\quad \phi \in C^{\infty}\cap L^2(\widehat{\pi}),
	\]
	while the property $\langle \phi ,P\mathcal{L}_{0}P\psi\rangle_{L^{2}(\widehat{\pi})}=\langle\mathcal{L}_{-}\phi,\psi\rangle_{ L^{2}(\widehat{\pi})}$ follows from properties \ref{it:oplem1}, \ref{it:oplem2} and \ref{it:oplem4}. Indeed,
	\begin{subequations}
	\begin{eqnarray*}
	\langle \phi,P\mathcal{L}P\psi\rangle_{ L^{2}(\widehat{\pi})}& = & \langle \phi, P(\mathcal{L}_{0}+\mu\mathcal{A})P\psi\rangle_{ L^{2}(\widehat{\pi})}=\langle\phi,\left(P\mathcal{L}_{0}P+\mu\mathcal{A}\right)\psi\rangle_{ L^{2}(\widehat{\pi})} \\	
	 & = & \langle (\mathcal{L}_{0}-\mu\mathcal{A})\phi,\psi\rangle_{ L^{2}(\widehat{\pi})}=\langle\mathcal{L}_{-}\phi,\psi\rangle_{ L^{2}(\widehat{\pi})}, 
\end{eqnarray*}
\end{subequations}
	as required. To prove \ref{it:oplem6} first notice that $\mathcal{L}$, $\mathcal{L}_0$ and $\mathcal{L}_{-}$ are of the form \eqref{eq:OU_operator} and therefore leave the spaces $H_m$ invariant by Theorem \ref{thm:L2 decomposition}. It follows immediately that also $\mathcal{A}$ leaves those spaces invariant. The fact that $P$ leaves the spaces $H_m$ invariant follows directly by inspection of \eqref{eq: Hermite polynomials}.
	\qed
\end{proof}
Now we proceed with the proof of Proposition  \ref{prop:asymvar_op_formula}:
\begin{proof}[of Proposition \ref{prop:asymvar_op_formula}] Since the potential $V$ is quadratic, Assumption \ref{ass:bounded+Poincare} clearly holds and thus Lemma \ref{lemma:variance} ensures that $\mathcal{L}$ and $\mathcal{L}_{-}$ are invertible on $L^2_{0}(\widehat{\pi})$ with 
\begin{equation}
\label{eq:Laplace transform}
\mathcal{L}^{-1}=\int_0^\infty e^{-t\mathcal{L}}\mathrm{d}t,
\end{equation}
	and analogously for $\mathcal{L}_{-}^{-1}$.
	 In particular, the asymptotic variance can be written as 
	 \begin{equation*}
	 \sigma_{f}^{2}=\langle f,(-\mathcal{L})^{-1}f\rangle_{L^{2}(\widehat{\pi})}.
	 \end{equation*}
	  Due to the respresentation \eqref{eq:Laplace transform} and Theorem \ref{thm:L2 decomposition}, the inverses of $\mathcal{L}$ and $\mathcal{L}_{-}$ leave the Hermite spaces $H_m$ invariant. We will prove the claim from Proposition \ref{prop:asymvar_op_formula} under the assumption that $Pf=f$ which includes the case 
	$f=f(q)$. For the following calculations we will assume $f\in H_m$ for fixed $m \ge 1$. Combining statement \ref{it:oplem6} with \ref{it:oplem1} and \ref{it:oplem5} of Lemma \ref{operator lemma} (and noting that $H_m \subset C^\infty(\mathbb{R}^{2d})\cap L^2(\widehat{\pi})$) we see that 
	\begin{equation}
	\label{eq:PLPL-}
	P\mathcal{L}P=\mathcal{L}_{-}^{*}
	\end{equation}
	 and 
	 \begin{equation}
	 P\mathcal{L}_{0}P=\mathcal{L}_{0}^{*}
	 \end{equation}
	  when restricted to $H_m$. Therefore, the following calculations are justified:
	\begin{align*}
	\langle f,(-\mathcal{L})^{-1}f\rangle_{L^{2}(\widehat{\pi})} &=\frac{1}{2}\langle f,(-\mathcal{L})^{-1}f\rangle_{L^{2}(\widehat{\pi})}+\langle f,(-\mathcal{L}^{*})^{-1}f\rangle_{L^{2}(\widehat{\pi})}\\
	&=\frac{1}{2}\langle f,(-\mathcal{L})^{-1}f\rangle_{L^{2}(\widehat{\pi})}+\langle Pf,(-\mathcal{L}^{*})^{-1}Pf\rangle_{L^{2}(\widehat{\pi})}\\
	&=\frac{1}{2}\langle f,(-\mathcal{L})^{-1}f\rangle_{L^{2}(\widehat{\pi})}+\langle f,(-\mathcal{L}_{-})^{-1}f\rangle_{L^{2}(\widehat{\pi})}\\
	&=\frac{1}{2}\langle f,\left((-\mathcal{L})^{-1}+(-\mathcal{L}_{-})^{-1}\right)f\rangle_{L^{2}(\widehat{\pi})},
	\end{align*}
	where in the third line we have used the assumption $Pf=f$ and in
	the fourth line the properties $P^{2}=I$, $P^{*}=P$ and equation
	(\ref{eq:PLPL-}).   Since $\mathcal{L}$ and $\mathcal{L}_{-}$ commute on $H_m$ according to Lemma
	\ref{operator lemma}\ref{it:oplem5},\ref{it:oplem6} we can write
	\begin{equation*}
	(-\mathcal{L})^{-1}+(-\mathcal{L}_{-})^{-1}  =\mathcal{L}_{-}(-\mathcal{L}\mathcal{L}_{-})^{-1}+\mathcal{L}(-\mathcal{L}\mathcal{L}_{-})^{-1}
	=-2\mathcal{L}_{0}(\mathcal{L}\mathcal{L}_{-})^{-1}
	\end{equation*}
	for the restrictions on $H_m$, 
	using $\mathcal{L}+\mathcal{L}_{-}=2\mathcal{L}_{0}$. We also have
	\begin{alignat*}{1}
	\mathcal{L}\mathcal{L}_{-} & =(\mathcal{L}_{0}+\mu\mathcal{A})(\mathcal{L}_{0}-\mu\mathcal{A}) =\mathcal{L}_{0}^{2}+\mu^{2}\mathcal{A}^{*}\mathcal{A},
	\end{alignat*}
	since $\mathcal{L}_{0}$ and $\mathcal{A}$ commute. We thus arrive at the formula
	\begin{equation}
	\label{eq:av_formula_Hm}
	\sigma_{f}^{2}=\langle f,-\mathcal{L}_{0}(\mathcal{L}_{0}^{2}+\mu^{2}\mathcal{A}^{*}\mathcal{A})^{-1}f\rangle_{L^{2}(\widehat{\pi})}, \quad f\in H_m.
	\end{equation}
	Now since $(\mathcal{L}_{0}^{2}+\mu^{2}\mathcal{A}^{*}\mathcal{A})^{-1}f = (\mathcal{L}\mathcal{L}_{-})^{-1}f \in \mathcal{D}(\mathcal{L}_{0})$ for all $f\in L^2(\widehat{\pi})$, it follows that the operator $-\mathcal{L}_{0}(\mathcal{L}_{0}^{2}+\mu^{2}\mathcal{A}^{*}\mathcal{A})^{-1}$ is bounded. We can therefore extend formula \eqref{eq:av_formula_Hm} to the whole of $L^2(\widehat{\pi})$ by continuity, using the fact that $L^2_0(\widehat{\pi})=\bigoplus_{m\ge 1}H_m$. 
	\qed
\end{proof}
Applying Proposition \ref{prop:asymvar_op_formula} we can analyse the behaviour
of $\sigma_{f}^{2}$ in the limit of large perturbation strength $\mu\rightarrow\infty$.
To this end, we introduce the orthogonal decomposition
\begin{equation}
\label{eq:kernel decomposition}
L_{0}^{2}(\pi)=\ker (Jq\cdot \nabla_q) \oplus\ker (Jq\cdot \nabla_q)^{\perp},
\end{equation}
where $Jq\cdot\nabla_q$ is understood as an unbounded operator acting on $L_0^2(\pi)$, obtained as the smallest closed extension of $Jq\cdot \nabla_q$ acting on $C^{\infty}_c(\mathbb{R}^d)$. In particular, $\ker (Jq\cdot \nabla_q)$ is a closed linear subspace of $L^2_0(\pi)$.   
Let $\Pi$ denote the $L_{0}^{2}(\pi)$-orthogonal projection onto
$\ker (Jq\cdot \nabla_q)$. We will write $\sigma_{f}^{2}(\mu)$ to
stress the dependence of the asymptotic variance on the perturbation
strength. The following result shows that for large perturbations,
the limiting asymptotic variance is always smaller than the asymptotic
variance in the unperturbed case. Furthermore, the limit is given as
the asymptotic variance of the projected observable $\Pi f$ for the
unperturbed dynamics.
\begin{theorem}
	\label{prop:large pert}
	Let $f\in L_{0}^{2}(\pi)$, then
	\[
	\lim_{\mu\rightarrow\infty}\sigma_{f}^{2}(\mu)=\sigma_{\Pi f}^{2}(0)\le\sigma_{f}^{2}(0).
	\]
\end{theorem}
\begin{remark}
	Note that the fact that the limit exists and is finite is nontrivial.
	In particular, as Figures \ref{fig:no_limit1} and \ref{fig:no_limit2} demonstrate, it is often
	the case that $\lim_{\mu\rightarrow\infty}\sigma_{f}^{2}(\mu)=\infty$
	if the condition $\mu=\nu$ is not satisfied.
\end{remark}
\begin{remark}
	\label{rem:projection}
	The projection $\Pi$ onto $\ker(Jq\cdot\nabla_q)$ can be understood in terms of Figure \ref{fig:good_spectrum}. Indeed, the eigenvalues on the real axis (highlighted by diamonds) are not affected by the perturbations. Let us denote by $\tilde{\Pi}$ the projection onto the span of the eigenspaces of those eigenvalues. As $\mu \rightarrow \infty$, the limiting asymptotic  variance is given as the asymptotic variance associated to the unperturbed dynamics of the projection $\tilde{\Pi}f$. If we denote by $\Pi_0$ the projection of $L^2(\widehat{\pi})$ onto $L^2_0(\pi)$, then we have that $\Pi\Pi_0=\Pi_0\tilde{\Pi}$. 
\end{remark}
\begin{proof}[of Theorem \ref{prop:large pert}]
	Note that $\mathcal{L}_{0}$ and $\mathcal{A}^{*}\mathcal{A}$ leave the Hermite spaces $H_m$ invariant and their restrictions to those spaces commute 
	(see Lemma \ref{operator lemma}, \ref{it:oplem2}, \ref{it:oplem3} and \ref{it:oplem6}). Furthermore, as the Hermite spaces $H_m$ are finite-dimensional, those operators have discrete spectrum. As $\mathcal{A}^{*}\mathcal{A}$
	is nonnegative self-adjoint, there exists an orthogonal
	decomposition $L_{0}^{2}(\pi)=\bigoplus_{i}W_{i}$  into eigenspaces of the operator $-\mathcal{L}_{0}(\mathcal{L}_{0}^{2}+\mu^{2}\mathcal{A}^{*}\mathcal{A})^{-1}$,
	the decomposition $\bigoplus W_i$ being finer then $\bigoplus H_m$ in the sense that every $W_i$ is a subspace of some $H_m$. 
	 Moreover,
	\[
	-\mathcal{L}_{0}(\mathcal{L}_{0}^{2}+\mu^{2}\mathcal{A}^{*}\mathcal{A})^{-1}\vert_{W_{i}}=-\mathcal{L}_{0}(\mathcal{L}_{0}^{2}+\mu^{2}\lambda_{i})^{-1}\vert_{W_i},
	\]
	where $\lambda_{i}\ge0$ is the eigenvalue of $\mathcal{A}^{*}\mathcal{A}$
	associated to the subspace $W_{i}$. Consequently, formula (\ref{eq:asymvar_op_formula})
	can be written as 
	\begin{equation}
	\label{eq:asymvar_spectral}
	\sigma_{f}^{2}=\sum_{i}\langle f_{i},-\mathcal{L}_{0}(\mathcal{L}_{0}^{2}+\mu^{2}\lambda_{i})^{-1}f_{i}\rangle_{L^{2}(\widehat{\pi})},
	\end{equation}
	where $f=\sum_{i}f_{i}$ and $f_{i}\in W_{i}$. Let us assume now
	without loss of generality that $W_{0}=\ker\mathcal{A}^{*}\mathcal{A}$,
	so in particular $\lambda_{0}=0$. Then clearly 
	\[
	\lim_{\mu\rightarrow\infty}\sigma_{f}^{2}=2\langle f_{0},-\mathcal{L}_{0}(\mathcal{L}_{0}^{2})^{-1}f_{0}\rangle_{L^{2}(\widehat{\pi})}=2\langle f_{0},(-\mathcal{L}_{0})^{-1}f_{0}\rangle_{L^{2}(\widehat{\pi})}=\sigma_{f_{0}}^{2}(0).
	\]
	Now note that $W_{0}=\ker\mathcal{A}^{*}\mathcal{A}=\ker\mathcal{A}$ due
	to $\ker\mathcal{A}^{*}=(\im\mathcal{A})^{\perp}$.  It remains to show that  $\sigma_{\Pi f}^{2}(0)\le\sigma_{f}^{2}(0)$.  To see this, we write 
	\begin{align*}
	\sigma_{f}^{2}(0) & =2\langle f,(-\mathcal{L}_{0})^{-1}f\rangle_{L^{2}(\widehat{\pi})}=2\langle\Pi f+(1-\Pi)f,(-\mathcal{L}_{0})^{-1}\big(\Pi f+(1-\Pi)f\big)\rangle_{L^{2}(\widehat{\pi})}\\
	& =\sigma_{\Pi f}^{2}(0)+\sigma_{(1-\Pi)f}^{2}(0)+R,
	\end{align*}
	where 
	\[
	R=2\langle\Pi f,(-\mathcal{L}_{0})^{-1}(1-\Pi)f\rangle_{L^{2}(\widehat{\pi})}+2\langle(1-\Pi)f,(-\mathcal{L}_{0})^{-1}\Pi f\rangle_{L^{2}(\widehat{\pi})}.
	\]
	Note that since we only consider observables that do not depend on $p$, $\Pi f\in \ker (Jq\cdot \nabla_q)$ and $(1-\Pi)f\in\bigoplus_{i\ge1}W_{i}$.
	Since $\mathcal{L}_{0}$ commutes with $\mathcal{A}$, it follows
	that $(-\mathcal{L}_{0})^{-1}$ leaves both $W_{0}$ and $\bigoplus_{i\ge1}W_{i}$
	invariant. Therefore, as the latter spaces are orthogonal to each
	other, it follows that $R=0$, from which the result follows. 
	\qed
\end{proof}
From Theorem \ref{prop:large pert} it follows that in the limit as $\mu \rightarrow \infty$, the asymptotic variance $\sigma_f^2(\mu)$ is not decreased by the perturbation if $f \in \ker(Jq \cdot \nabla_q)$. In fact, this result also holds true non-asymptotically, i.e. observables in $\ker(Jq \cdot \nabla_q)$ are not affected at all by the perturbation:
\begin{lemma}
	\label{lem:invariant observables}
	Let $f\in \ker (Jq\cdot \nabla_q)$. Then
	\begin{equation*}
	\sigma^2_f(\mu) = \sigma^2_f(0)
	\end{equation*}
	for all $\mu \in \mathbb{R}$.
\end{lemma}
\begin{proof}
	From $f \in  \ker (Jq\cdot \nabla_q)$ it follows immediately that $f \in \ker \mathcal{A}^{*}\mathcal{A}$. Then the claim follows from the expression \eqref{eq:asymvar_spectral}.
	\qed
\end{proof}
\begin{example}
	\label{ex:commutation quadratic observables}
	Recall the case of observables of the form $f(q)=q\cdot Kq+l\cdot q+C$
	with $K\in\mathbb{R}_{sym}^{d\times d}$, $l\in\mathbb{R}^{d}$ and
	$C\in\mathbb{R}$ from Section \ref{sec:small perturbations}. If $[J,K]=0$
	and $l\in\ker J$, then $f\in\ker (Jq\cdot \nabla_q)$ as 
	\[
	Jq\cdot\nabla_{q}(q\cdot Kq+l\cdot q+C)=2Jq\cdot Kq+Jq\cdot l=q\cdot(KJ-JK)q-q\cdot Jl=0.
	\]
	From the preceding lemma it follows that $\sigma_{f}^{2}(\mu)=\sigma_{f}^{2}(0)$
	for all $\mu\in\mathbb{R},$ showing that the assumption in Theorem
	\ref{cor:small pert unit var} does not exclude nontrivial cases.
\end{example}
The following result shows that the dynamics (\ref{eq: unit covariance perfect perturbation})
is particularly effective for antisymmetric observables (at least
in the limit of large perturbations):
\begin{proposition}
	\label{prop:antisymmetric observables}Let $f\in L_{0}^{2}(\pi)$
	satisfy $f(-q)=-f(q)$ and assume that $\ker J=\{0\}$.
	Furthermore, assume that the eigenvalues of $J$ are rationally independent, i.e. 
	\begin{equation}
	\label{eq:rat_indp_spectrum}
	\sigma(J)=\{\pm i\lambda_{1},\pm i\lambda_{2},\ldots,\pm i\lambda_d\}
	\end{equation}
	with $\lambda_{i}\in\mathbb{R}_{>0}$ and  $\sum_i k_i \lambda_i \neq 0$ for all $(k_1,\ldots,k_d)\in\mathbb{Z}^d\setminus(0,\ldots,0)$. Then $\lim_{\mu\rightarrow\infty}\sigma_{f}^{2}(\mu)=0$.
\end{proposition} 
\begin{proof}
	[of Proposition \ref{prop:antisymmetric observables}]
	The claim would
	immediately follow from $f\in\ker(Jq\cdot\nabla)^{\perp}$ according to Theorem \ref{prop:large pert}, but that does not seem to be so easy to prove directly. Instead, we again make
	use of the Hermite polynomials.
	
	Recall from the proof of Proposition \ref{prop:asymvar_op_formula} that $\gen$ is invertible on $L_{0}^{2}(\widehat{\pi})$ and its inverse leaves the Hermite spaces $H_m$ invariant. Consequently, the
	asymptotic variance of an observable $f\in L_{0}^{2}(\widehat{\pi})$ can be written as  
	\begin{subequations}
	\begin{eqnarray}
	\sigma_{f}^{2} & = & \langle f,(-\mathcal{L})^{-1}f\rangle_{L^{2}(\widehat{\pi})} \\
	& = & \sum_{m=1}^{\infty}\langle\Pi_{m}f,(-\mathcal{L}\vert_{H_{m}})^{-1}\Pi_{m}f\rangle_{L^2(\widehat{\pi})},\label{eq:asymvar decomposition} 
	\end{eqnarray}
	\end{subequations}
	where $\Pi_{m}:L_{0}^{2}(\widehat{\pi})\rightarrow H_{m}$ denotes the orthogonal
	projection onto $H_{m}$. From (\ref{eq: Hermite polynomials}) it
	is clear that $g_{a}$ is symmetric for $\vert\alpha\vert$ even and
	antisymmetric for $\vert\alpha\vert$ odd. Therefore, from $f$ being
	antisymmetric it follows that 
	\[
	f\in\bigoplus_{m\ge1,m\,\text{odd}}H_{m}.
	\]
	In view of (\ref{eq:spectrum of B}), (\ref{eq:spectrum on subspaces}) and (\ref{eq:rat_indp_spectrum})
	the spectrum of $\mathcal{L}_{\vert H_{m}}$ can be written as 
	\begin{subequations}
	\begin{eqnarray}
	\sigma(\mathcal{L}\vert_{H_{m}}) & = &\left\lbrace \mu\sum_{j=1}^{2d}\alpha_{j}\beta_{j}+C_{\alpha,\gamma}:\,\vert\alpha\vert=m,\,\beta_{j}\in\sigma(J)\right\rbrace  \nonumber \\
	& = & \left\lbrace i\mu\sum_{j=1}^{d}(\alpha_{j}-\alpha_{j+d})\lambda_{j}+C_{\alpha,\gamma}:\,\vert\alpha\vert=m\right\rbrace  \label{eq:spec_grow}
	\end{eqnarray}
	\end{subequations}
	with appropriate real constants $C_{\alpha,\gamma}\in\mathbb{R}$ that depend
	on $\alpha$ and $\gamma$, but not on $\mu$. For $\vert\alpha\vert=\sum_{j=1}^{2d} \alpha_j=m$ odd, we have that
	\begin{equation}
	\label{eq:nonzero}
	\sum_{j=1}^{d}(\alpha_{j}-\alpha_{j+d})\lambda_{j} \neq 0.
	\end{equation}
	Indeed, assume to the contrary that the above expression is zero. Then it follows that $\alpha_j = \alpha_{j+d}$ for all $j=1,\ldots,d$ by rational independence of $\lambda_1,\ldots,\lambda_d$.
	From \eqref{eq:spec_grow} and \eqref{eq:nonzero} it is clear that
	\begin{equation*}
	\sup \left\lbrace r>0 : B(0,r) \cap \sigma(\mathcal{L}\vert_{H_m}) = \emptyset \right\rbrace \xrightarrow{\mu \rightarrow \infty} \infty,
	\end{equation*}
	where $B(0,r)$ denotes the ball of radius $r$ centered at the origin in $\mathbb{C}$.
	Consequently, the spectral radius of $(-\mathcal{L}\vert_{H_m})^{-1}$ and hence $(-\mathcal{L}\vert_{H_m})^{-1}$ itself converge to zero as $\mu \rightarrow \infty$. The result then follows from (\ref{eq:asymvar decomposition}). \qed\end{proof}
\begin{remark}
	The idea of the preceding proof can be explained using Figure \ref{fig:good_spectrum} and Remark \ref{rem:projection}. Since the real eigenvalues correspond to Hermite polynomials of even order, antisymmetric observables are orthogonal to the associated subspaces. The rational independence condition on the eigenvalues of $J$ prevents cancellations  that would lead to further eigenvalues on the real axis.
\end{remark}
The following corollary gives a version of the converse of Proposition \ref{prop:antisymmetric observables} and provides further intuition into the mechanics of the variance reduction achieved by the perturbation.
\begin{corollary}
	Let $f\in L_{0}^{2}(\pi)$ and assume that $lim_{\mu\rightarrow\infty}\sigma_{f}^{2}(\mu)=0$. Then 
	\[
	\int_{B(0,r)}f\mathrm{dq=0}
	\]
	for all $r\in(0,\infty)$, where $B(0,r)$ denotes the ball centered at $0$ with radius $r$.
\end{corollary}
\begin{proof}
	According to Theorem \ref{prop:large pert},  $\lim_{\mu\rightarrow\infty}\sigma_{f}^{2}(\mu)=0$  implies $\sigma_{\Pi f}^{2}(0)=0$. We can write 
	\begin{subequations}
	\begin{eqnarray}
	\sigma_{\Pi f}^{2}(0) & = & \langle \Pi f, (-\mathcal{L}_0)^{-1}\Pi f \rangle_{ L^{2}(\widehat{\pi})} \nonumber \\
	& = & \frac{1}{2}\langle \Pi f, \left((-\mathcal{L}_0)^{-1}+(-\mathcal{L}^{*}_0)^{-1}\right)\Pi f \rangle_{ L^{2}(\widehat{\pi})} \nonumber
	\end{eqnarray}
	\end{subequations}
	and recall from the proof of Proposition \ref{prop:asymvar_op_formula} that $(-\mathcal{L}_0)^{-1}$ and $(-\mathcal{L}^{*}_0)^{-1}$ leave the Hermite spaces $H_m$ invariant. Therefore  
	\begin{equation}
	\ker \left((-\mathcal{L}_0)^{-1}+(-\mathcal{L}^{*}_0)^{-1}\right) = {0}
	\end{equation}
	in $L^2_0(\widehat{\pi})$, and in particular $\sigma_{\Pi f}^{2}(0)=0$ implies $\Pi f = 0$, which in turn shows that
	  $f\in\ker(Jq\cdot\nabla)^{\perp}$. Using $\ker(Jq\cdot\nabla)^{\perp}=\overline{\im(Jq\cdot\nabla)}$,
	it follows that there exists a sequence $(\phi_n)_n\in C_c^{\infty}(\mathbb{R}^d)$ such that $Jq\cdot\nabla\phi_n \rightarrow f$ in $L^2(\pi)$. Taking a subsequence if necessary, we can assume that the convergence is pointwise $\pi$-almost everywhere and that the sequence is pointwise bounded by a function in $L^1(\pi)$. 
	Since $J$ is antisymmetric, we have that $Jq\cdot\nabla\phi_n=\nabla\cdot(\phi_n Jq)$.
	Now Gauss's theorem yields
	\[
	\int_{B(0,r)}f\mathrm{d}q=\int_{B(0,r)}\nabla\cdot(\phi Jq)\mathrm{d}q=\int_{\partial B(0,r)}\phi Jq\cdot\mathrm{d}n,
	\]
	where $n$ denotes
	the outward normal to the sphere $\partial B(0,r)$. This quantity
	is zero due to the orthogonality of $Jq$ and $n$, and so the result
	follows from Lebesgue's dominated convergence theorem.\qed
\end{proof}
\subsection{Optimal Choices of $J$ for Quadratic Observables}

Assume $f\in L_{0}^{2}(\pi)$ is given by $f(q)=q\cdot Kq+l\cdot q -\Tr K$,  with $K\in\mathbb{R}_{sym}^{d\times d}$ and $l\in\mathbb{R}^{d}$ (note that the constant term is chosen such that $ \pi(f)=0 $). Our objective is to choose $J$ in such a way that $\lim_{\mu\rightarrow\infty}\sigma_{f}^{2}(\mu)$ becomes as small as possible. To stress the dependence on the choice
of $J$, we introduce the notation $\sigma_{f}^{2}(\mu,J)$. Also, we denote the orthogonal projection onto $(\ker J)^{\perp}$ by $\Pi^{\perp}_{\ker J}$.
\begin{lemma}
	\label{lem:lin_observables}{(Zero variance limit for linear observables).} Assume $K=0$ and $\Pi^{\perp}_{\ker J}l=0$. Then 
	\[
	\lim_{\mu\rightarrow\infty}\sigma_{f}^{2}(\mu,J)=0.
	\]
\end{lemma}
\begin{proof}
	According to Proposition \ref{prop:large pert}, we have to show that
	$\Pi f=0$, where $\Pi$ is the $L^{2}(\pi)$-orthogonal projection
	onto $\ker(Jq\cdot\nabla)$. Let us thus prove that 
	\[
	f\in\ker(Jq\cdot\nabla)^{\perp}=\overline{\im(Jq\cdot\nabla)^{*}}=\overline{\im(Jq\cdot\nabla)},
	\]
	where the second identity uses the fact that $(Jq\cdot\nabla)^{*}=-Jq\cdot\nabla$.
	Indeed, since $\Pi^{\perp}_{\ker J}=0$, by Fredholm's alternative there exists $u \in \mathbb{R}^d$ such that $Ju=l$. Now define $\phi\in L_{0}^{2}(\pi)$
	by $\phi(q)=-u\cdot q,$ leading to 
	\[
	f=Jq\cdot\nabla\phi,
	\]
	so the result follows.\qed
\end{proof}

\begin{lemma}
	\label{lem:optimal_perturbation}{(Zero variance limit for purely quadratic observables.)} Let $l=0$ and consider the decomposition $K=K_{0}+K_{1}$ into the traceless part $K_{0}=K-\frac{\Tr K}{d}\cdot I$ and the
	trace-part $K_{1}=\frac{\Tr K}{d}\cdot I.$ For the corresponding
	decomposition of the observable 
	\[
	f(q)=f_{0}(q)+f_{1}(q)=q\cdot K_{0}q+q\cdot K_{1}q-\Tr K
	\]
	the following holds:
	\begin{enumerate}[label=(\alph*)]
		\item There exists an antisymmetric matrix $J$ such that  $\lim_{\mu\rightarrow\infty}\sigma_{f_{0}}^{2}(\mu,J)=0,$
		and there is an algorithmic way (see Algorithm \ref{alg:optimal J}) to compute an appropriate $J$ in terms
		of $K$.
		\item The trace-part is not effected by the perturbation, i.e. $\sigma_{f_{1}}^{2}(\mu,J)=\sigma_{f_{1}}^{2}(0)$ for all $\mu\in\mathbb{R}$.
	\end{enumerate}
\end{lemma}
\begin{proof}
	To prove the first claim, according to Theorem \ref{prop:large pert}
	it is sufficient to show that $f_{0}\in\ker(Jq\cdot\nabla)^{\perp}=\overline{\im(Jq\cdot\nabla)}$.
	Let us consider the function $\phi(q)=q\cdot Aq$, with $A\in\mathbb{R}_{sym}^{d\times d}$. It holds that
	\begin{equation*}
	Jq\cdot\nabla\phi=q\cdot(J^{T}Aq)=q\cdot[A,J]q.
	\end{equation*}
	The task of finding an antisymmetric matrix $J$ such that 
	\begin{equation}
	\label{eq:quad_var_reduction}
	\lim_{\mu\rightarrow\infty}\sigma_{f_{0}}^{2}(\mu,J)=0
	\end{equation}
	can therefore be accomplished by constructing an antisymmetric matrix
	$J$ such that there exists a symmetric matrix $A$ with the property
	$K_{0}=[A,J]$.  Given any traceless matrix $K_{0}$ there exists
	an orthogonal matrix $U\in O(\mathbb{R}^{d})$ such that $UK_{0}U^{T}$
	has zero entries on the diagonal, and that $U$ can be obtained in
	an algorithmic manner (see for example \cite{alg_zero_diag} or \cite[Chapter 2, Section 2, Problem 3]{Horn_Johnson_Matrix_Analysis}; for the reader's convenience we have summarised the algorithm in Appendix \ref{tracefree}.) Assume
	thus that such a matrix $U\in O(\mathbb{R}^{d})$ has been found and choose real numbers $a_1,\ldots,a_d \in \mathbb{R}$ such that $a_{i}\neq a_{j}$ if $i\neq j$.
	We now set
	\begin{equation}
	\bar{A}=\diag(a_{1},\ldots,a_{n}),
	\end{equation}
	and 
	\begin{equation}
	\bar{J}_{ij}= 
	\begin{cases}
	\frac{(UK_{0}U^{T})_{ij}}{a_{i}-a_{j}} & \text{if } i\neq j, \\
	0 & \text{if } i=j. \\ 
	\end{cases}
	\end{equation}
	Observe that since $UK_{0}U^{T}$ is symmetric, $\bar{J}$ is antisymmetric. 
	A short calculation shows that $[\bar{A},\bar{J}]= UK_{0}U^{T}$.
	We can thus define $A=U^{T}\bar{A}U$ and $J=U^{T}\bar{J}U$ to obtain $[A,J]=K_0$. Therefore, the $J$ constructed in this way indeed satisfies \eqref{eq:quad_var_reduction}.  For the second claim, note that $f_{1}\in\ker(Jq\cdot\nabla)$, since
	\begin{equation}
	\label{eq:constant_trace}
	Jq\cdot\nabla\left(q\cdot\frac{\Tr K}{d}q\right)=2\frac{\Tr K}{d}q\cdot Jq=0
	\end{equation}
	because of the antisymmetry of $J$. The result then follows from
	Lemma \ref{lem:invariant observables}.\qed
\end{proof}
We would like to stress that the perturbation $J$ constructed in the previous lemma is far from unique due to the freedom of choice of $U$ and $a_1,\ldots,a_d \in \mathbb{R}$ in its proof. However, it is asymptotically optimal:
\begin{corollary}
	\label{cor:optimality}
	In the setting of Lemma \ref{lem:optimal_perturbation} the following holds:
	\[
	\min_{J^T=-J}\left(\lim_{\mu\rightarrow\infty} \sigma^2_{f}(\mu,J)\right)=\sigma^2_{f_1}(0).
	\]
\end{corollary}
\begin{proof}
	The claim follows immediately since $f_{1}\in\ker(Jq\cdot\nabla)$ for arbitrary antisymmetric $J$ as shown in \eqref{eq:constant_trace}, and therefore the contribution of the trace part $f_1$ to the asymptotic variance cannot be reduced by any choice of $J$ according to Lemma \ref{lem:invariant observables}.	
\end{proof}
As the proof of Lemma \ref{lem:optimal_perturbation} is constructive, we obtain the following algorithm for determining optimal perturbations for quadratic observables:
\begin{algorithm}
	\label{alg:optimal J}
	Given $K\in\mathbb{R}_{sym}^{d\times d}$,
	determine an optimal antisymmetric perturbation $J$ as follows:
	\begin{enumerate}
		\item Set $K_{0}=K-\frac{\Tr K}{d}\cdot I.$
		\item Find $U\in O(\mathbb{R}^{d})$ such that $UK_{0}U^{T}$ has zero entries
		on the diagonal (see Appendix \ref{tracefree}).
		\item Choose $a_{i}\in\mathbb{R},$ $i=1,\ldots d$ such that $a_{i}\neq a_{j}$
		for $i\neq j$ and set 
		\[
		\bar{J}_{ij}=\frac{(UK_{0}U^{T})_{ij}}{a_{i}-a_{j}}
		\]
		for $i\ne j$ and $\bar{J}_{ii}=0$ otherwise.
		\item
		Set $J=U^{T}\bar{J}U$.
	\end{enumerate}
\end{algorithm}
\begin{remark}
	In \cite{duncan2016variance}, the authors consider the task of finding optimal perturbations $J$ for the nonreversible overdamped Langevin dynamics given in \eqref{eq:nonrev_overdamped_J}. In the Gaussian case this optimization problem turns out be equivalent to the one considered in this section. Indeed, equation (39) of \cite{duncan2016variance} can be rephrased as 
	\begin{equation*}
	f \in \ker(Jq\cdot \nabla)^{\perp}.
	\end{equation*}
	Therefore, Algorithm \ref{alg:optimal J} and its generalization Algorithm \ref{alg:optimal J general} (described in Section~\ref{sec:arbitrary covariance})  can be used without modifications to find optimal perturbations of overdamped Langevin dynamics.
\end{remark}

\subsection{Gaussians with Arbitrary Covariance and Preconditioning}
\label{sec:arbitrary covariance}

In this section we  extend the results of the preceding sections to the case
when the target measure $\pi$ is given by a Gaussian with arbitrary
covariance, i.e. $V(q)=\frac{1}{2}q\cdot Sq$ with $S\in\mathbb{R}_{sym}^{d\times d}$ symmetric and positive definite. The
dynamics (\ref{eq:perturbed_underdamped}) then takes the
form 

\begin{align}
\mathrm{d}q_{t} & =M^{-1}p_{t}\mathrm{d}t-\mu J_{1}Sq_{t}\mathrm{d}t\nonumber, \\
\mathrm{d}p_{t} & =-Sq_{t}\mathrm{d}t-\nu J_{2}M^{-1}p_{t}\mathrm{d}t-\Gamma M^{-1}p_{t}\mathrm{d}t+\sqrt{2\Gamma}\mathrm{d}W_{t}.\label{eq:Underdamped Langevin Gaussian}
\end{align}
The key observation is now that the choices $M=S$ and $\Gamma=\gamma S$
together with the transformation $\widetilde{q}=S^{1/2}q$ and $\widetilde{p}=S^{-1/2}p$
lead to the dynamics
\begin{align}
\mathrm{d}\widetilde{q}_{t} & =\widetilde{p}_{t}\mathrm{d}t-\mu S^{1/2}J_{1}S^{1/2}\widetilde{q}_{t}\mathrm{d}t,\nonumber \\
\mathrm{d}\widetilde{p}_{t} & =-\widetilde{q}_{t}\mathrm{d}t-\mu S^{-1/2}J_{2}S^{-1/2}\widetilde{p}_{t}\mathrm{d}t-\gamma\widetilde{p}_{t}\mathrm{d}t+\sqrt{2\gamma}\mathrm{d}W_{t},\label{eq:Underdamped Langevin transformed}
\end{align}
which is of the form (\ref{eq:unit covariance}) if $J_{1}$ and
$J_{2}$ obey the condition $SJ_{1}S=J_{2}$ (note that both $S^{1/2}J_{1}S^{1/2}$
and $S^{-1/2}J_{2}S^{-1/2}$ are of course antisymmetric). Clearly
the dynamics (\ref{eq:Underdamped Langevin transformed}) is ergodic
with respect to a Gaussian measure with unit covariance, in the following
denoted by $\widetilde{\pi}$. The connection between the asymptotic variances
associated to (\ref{eq:Underdamped Langevin Gaussian}) and (\ref{eq:Underdamped Langevin transformed})
is as follows: 
\\\\
For an observable $f\in L_{0}^{2}(\pi)$ we can write 
\[
\sqrt{T}\bigg(\frac{1}{T}\int_{0}^{T}f(q_{s})\mathrm{d}s-\pi(f)\bigg)=\sqrt{T}\bigg(\frac{1}{T}\int_{0}^{T}\widetilde{f}(\widetilde{q}_{s})\mathrm{d}s-\widetilde{\pi}(\widetilde{f})\bigg),
\]
where $\widetilde{f}(q)=f(S^{-1/2}q)$. Therefore, the asymptotic variances
satisfy
\begin{equation}
\sigma_{f}^{2}=\widetilde{\sigma}_{\widetilde{f}}^{2},\label{eq:asymvar transform}
\end{equation}
where $\widetilde{\sigma}_{\widetilde{f}}^{2}$ denotes the asymptotic variance
of the process $(\widetilde{q}_{t})_{t\ge0}$. Because of this, the results
from the previous sections generalise to (\ref{eq:Underdamped Langevin Gaussian}),
subject to the condition that the choices $M=S$, $\Gamma=\gamma S$
and $SJ_{1}S=J_{2}$ are made. We formulate our results in this general
setting as corollaries:
\begin{corollary}
	\label{cor:small_pert_general}
	Consider the dynamics 
	\begin{align}
	\mathrm{d}q_{t} & =M^{-1}p_{t}\mathrm{d}t-\mu J_{1}\nabla V(q_{t})\mathrm{d}t,\nonumber \\
	\mathrm{d}p_{t} & =-\nabla V(q_{t})\mathrm{d}t-\mu J_{2}M^{-1}p_{t}\mathrm{d}t-\Gamma M^{-1}p_{t}\mathrm{d}t+\sqrt{2\Gamma}\mathrm{d}W_{t},\label{eq: perturbed Langevin corollary}
	\end{align}
	with $V(q)=\frac{1}{2}q\cdot Sq$. Assume that $M=S$, $\Gamma=\gamma S$
	with $\gamma > \sqrt{2}$ and $SJ_{1}S=J_{2}$. Let $f\in L^{2}(\pi)$ be an observable of the form 
	\begin{equation}
	f(q)=q\cdot Kq+l\cdot q+C\label{eq:quadratic observable}
	\end{equation}
	with $K\in\mathbb{R}_{sym}^{d\times d}$, $l\in\mathbb{R}^{d}$ and
	$C\in\mathbb{R}$. If at least one of the conditions $KJ_{1}S\neq SJ_{1}K$
	and $l \notin \ker J$ is satisfied, then the asymptotic variance is at a local maximum for the unperturbed sampler, i.e.
	\[
	\left. \partial_{\mu}\sigma_{f}^{2}\right\rvert_{\mu=0}=0\qquad \mbox{ and } \qquad 	\left. \partial_{\mu}^{2}\sigma_{f}^{2}\right\rvert_{\mu=0}<0.
	\]
\end{corollary}
\begin{proof}
	Note that 
	\[
	\widetilde{f}(q)=f(S^{-1/2}q)=q\cdot S^{-1/2}KS^{-1/2}q+S^{-1/2}l\cdot q+C=q\cdot\widetilde{K}q+\widetilde{l}\cdot q+C
	\]
	is again of the form (\ref{eq:quadratic observable}) (where in the
	last equality, $\widetilde{K}=S^{-1/2}KS^{-1/2}$ and $\widetilde{l}=S^{-1/2}l$
	have been defined). From (\ref{eq:Underdamped Langevin transformed}),
	(\ref{eq:asymvar transform}) and Theorem \ref{cor:small pert unit var}
	the claim follows if at least one of the conditions $[\widetilde{K},S^{1/2}J_{1}S^{1/2}]\neq0$
	and $\widetilde{l}\notin\ker(S^{1/2}J_{1}S^{1/2})$ is satisfied. The
	first of those can easily seen to be equivalent to 
	\[
	S^{-1/2}(KJS-SJK)S^{-1/2}\neq0,
	\]
	which is equivalent to $KJ_{1}S\neq SJ_{1}K$ since $S$ is nondegenerate.
	The second condition is equivalent to 
	\[
	S^{1/2}J_{1}l\neq0,
	\]
	which is equivalent to $J_{1}l\neq0,$ again by nondegeneracy of $S$. \qed \end{proof}
\begin{corollary}
	\label{cor:limit_asym_var}
	Assume the setting from the previous corollary and denote by $\Pi$
	the orthogonal projection onto $\ker(J_{1}Sq\cdot\nabla)$. For $f\in L^{2}(\pi)$
	it holds that
	\[
	\lim_{\mu\rightarrow\infty}\sigma_{f}^{2}(\mu)=\sigma_{\Pi f}^{2}(0)\le\sigma_{f}^{2}(0).
	\]
\end{corollary}
\begin{proof}
	Theorem \ref{prop:large pert} implies 
	\[
	\lim_{\mu\rightarrow\infty}\widetilde{\sigma}_{\widetilde{f}}^{2}(\mu)=\widetilde{\sigma}_{\widetilde{\Pi}\widetilde{f}}^{2}(0)\le\widetilde{\sigma}_{\widetilde{f}}^{2}(0)
	\]
	for the transformed system (\ref{eq:Underdamped Langevin transformed}).
	Here $\widetilde{f}(q)=f(S^{-1/2}q)$ is the transformed observable and
	$\widetilde{\Pi}$ denotes $L^{2}(\pi)$-orthogonal projection onto $\ker(S^{1/2}J_{1}S^{1/2}q\cdot\nabla)$.
	According to (\ref{eq:asymvar transform}), it is sufficient to show
	that $(\Pi f)\circ S^{-1/2}=\widetilde{\Pi}\widetilde{f}$. This however follows
	directly from the fact that the linear transformation $\phi\mapsto\phi\circ S^{1/2}$
	maps $\ker(S^{1/2}J_{1}S^{1/2}q\cdot\nabla)$ bijectively onto $\ker(J_{1}Sq\cdot\nabla)$.\qed
\end{proof}
Let us also reformulate Algorithm \ref{alg:optimal J} for the case of a Gaussian with arbitrary covariance.
\begin{algorithm}
	\label{alg:optimal J general}Given $K,S\in\mathbb{R}_{sym}^{d\times d}$
	with $f(q)=q\cdot Kq$ and $V(q)=\frac{1}{2}q\cdot Sq$ (assuming $S$ is nondegenerate), determine optimal perturbations $J_{1}$
	and $J_{2}$ as follows:
	\begin{enumerate}
		\item Set $\widetilde{K}=S^{-1/2}KS^{-1/2}$ and $\widetilde{K}_{0}=\widetilde{K}-\frac{\Tr\widetilde{K}}{d}\cdot I$.
		\item Find $U\in O(\mathbb{R}^{d})$ such that $U\widetilde{K}_{0}U^{T}$ has
		zero entries on the diagonal (see Appendix \ref{tracefree}).
		\item Choose $a_{i}\in\mathbb{R}$, $i=1,\ldots,d$ such that $a_{i}\ne a_{j}$
		for $i\ne j$ and set 
		\[
		\bar{J}_{ij}=\frac{(U\widetilde{K}_{0}U^{T})_{ij}}{a_{i}-a_{j}}.
		\]
		\item
		Set $\widetilde{J}=U^{T}\bar{J}U$.
		\item Put $J_{1}=S^{-1/2}\widetilde{J}S^{-1/2}$ and $J_{2}=S^{1/2}JS^{1/2}$.
	\end{enumerate}
\end{algorithm}
Finally, we obtain the following optimality result from Lemma \ref{lem:lin_observables} and Corollary \ref{cor:optimality}.
\begin{corollary}
	Let $f(q)=q\cdot Kq+l\cdot q-\Tr K$ and assume that $\Pi^{\perp}_{\ker J}l=0$.
	Then 
	\[
	\min_{J_1^T=-J_1,\, J_2=SJ_1 S}\left(\lim_{\mu\rightarrow\infty} \sigma^2_{f}(\mu,J_1,J_2)\right)=\sigma^2_{f_1}(0),
	\]
	where $f_{1}(q)=q\cdot K_{1}q$, $K_{1}=\frac{\Tr(S^{-1}K)}{d}S$.
	Optimal choices for $J_{1}$ and $J_{2}$ can be obtained using Algorithm \ref{alg:optimal J general}.
\end{corollary}
\begin{remark}
	Since in Section \ref{sec:small perturbations} we analysed the case
	where $J_{1}$ and $J_{2}$ are proportional, we are not able to drop
	the restriction $J_{2}=SJ_{1}S$ from the above optimality
	result. Analysis of completely arbitrary perturbations will be the
	subject of future work. 
\end{remark}

\begin{remark}
	The choices $M=S$ and $\Gamma=\gamma S$ have been introduced to
	make the perturbations considered in this article lead to samplers that perform well in terms of reducing the asymptotic variance. However, adjusting
	the mass and friction matrices according to the target covariance
	in this way (i.e. $M=S$ and $\Gamma=\gamma S$) is a popular way of preconditioning the dynamics, see for instance \cite{GirolamiCalderhead2011} and, in particular mass-tensor molecular dynamics~\cite{Bennett1975267}. Here we will present an argument why such a preconditioning
	is indeed beneficial in terms of the convergence rate of the dynamics.
	Let us first assume that $S$ is diagonal, i.e. $S=\diag(s^{(1)},\ldots,s^{(d)})$
	and that $M=\diag(m^{(d)},\ldots,m^{(d)})$ and $\Gamma=\diag(\gamma^{(d)},\ldots,\gamma^{(d)})$
	are chosen diagonally as well. Then (\ref{eq:Underdamped Langevin Gaussian})
	decouples into one-dimensional SDEs of the following form: 
	\begin{align}
	\mathrm{d}q_{t}^{(i)} & =\frac{1}{m^{(i)}}p_{t}^{(i)}\mathrm{d}t,\nonumber \\
	\mathrm{d}p_{t}^{(i)} & =-s^{(i)}q_{t}^{(i)}\mathrm{d}t-\frac{\gamma^{(i)}}{m^{(i)}}p_{t}^{(i)}\mathrm{d}t+\sqrt{2\gamma^{(i)}}\mathrm{d}W_{t},\quad i=1,\ldots,d.\label{eq:decoupled Langevin}
	\end{align}
	Let us write those Ornstein-Uhlenbeck processes as 
	\begin{equation}
	\mathrm{d}X_{t}^{(i)}=-B^{(i)}X_{t}^{(i)}\mathrm{d}t+\sqrt{2Q^{(i)}}\mathrm{d}W_{t}^{(i)}\label{eq: decoupled OU process}
	\end{equation}
	with 
	\[
	B^{(i)}=\left(\begin{array}{cc}
	0 & -\frac{1}{m^{(i)}}\\
	s^{(i)} & \frac{\gamma^{(i)}}{m^{(i)}}
	\end{array}\right)\,\text{and }\,Q^{(i)}=\left(\begin{array}{cc}
	0 & 0\\
	0 & \gamma^{(i)}
	\end{array}\right).
	\]
	As in Section \ref{sec:exp_decay}, the rate of the exponential decay of (\ref{eq: decoupled OU process}) is equal to $\min\text{Re}\,\sigma(B^{(i)})$. A short calculation shows that the eigenvalues of $B^{(i)}$ are given by  
	\[
	\lambda_{1,2}^{(i)}=\frac{\gamma^{(i)}}{2m^{(i)}}\pm\sqrt{\bigg(\frac{\gamma^{(i)}}{2m^{(i)}}\bigg)^{2}-\frac{s^{(i)}}{m^{(i)}}}.
	\]
	Therefore, the rate of exponential decay is maximal when 
	\begin{equation}
	\bigg(\frac{\gamma^{(i)}}{2m^{(i)}}\bigg)^{2}-\frac{s^{(i)}}{m^{(i)}}=0,\label{eq:gm constraint}
	\end{equation}
	in which case it is given by 
	\[
	(\lambda^{(i)})^{*}=\sqrt{\frac{s^{(i)}}{m^{(i)}}}.
	\]
	Naturally, it is reasonable to choose $m^{(i)}$ in such a way that
	the exponential rate $(\lambda^{(i)})^{*}$ is the same for all $i$, leading
	to the restriction $M=cS$ with $c>0$. Choosing $c$ small will result in fast convergence to equilibrium,
	but also make the dynamics (\ref{eq:decoupled Langevin}) quite stiff,
	requiring a very small timestep $\Delta t$ in a discretisation scheme.
	The choice of $c$ will therefore need to strike a balance between
	those two competing effects. The constraint (\ref{eq:gm constraint})
	then implies $\Gamma=2cS$.	 By a coordinate transformation, the preceding argument also applies if $S$, $M$ and $\Gamma$ are diagonal in the same basis, and of course $M$ and $\Gamma$ can always be chosen that way.
	Numerical experiments show that it is possible to increase the rate of convergence to equilibrium even further by choosing $M$ and $\Gamma$ nondiagonally with respect to $S$ (although
	only by a small margin). A clearer understanding of this is a topic of further investigation.
\end{remark}


%\section{Numerical implementation}
%\label{sec:implementation}
%\newcommand{\anoise}{{\mathcal{AN}}}
\newcommand{\pnoise}{{\mathcal{PN}}}
\section{Stochastic Games for V-Formation}
\label{sec:sgv}

We describe the specialization of the stochastic-game verification problem to
V-formation.  In particular, we present the AMPC-based control strategy for reaching a V-formation, and the various attacker strategies against which we evaluate the resilience of our controller.

The MDP $\M$ for V-formation was presented in Section~\ref{sec:background}. The state variables of the MDP are the positions and velocities of the birds, and the control variables (defining the actions) are the accelerations and displacements. In the transition relation given in equation~(\ref{eq:v}), the attacker chooses the displacement $\vec{d}(t)$ it needs to manipulate the position of the birds,
whereas the controller chooses the acceleration $\vec{a}(t)$ to apply. Together, the pair $(\vec{a}(t),\vec{d}(t))$ defines the action that transforms one MDP state to another. We now define the controller's and attacker's strategies.

\subsection{Controller's Adaptive Strategies}

Given current state $(\vec{x}(t),\vec{v}(t))$, the controller's strategy $\sigma_C$ returns a probability distribution on the space of all possible accelerations (for all birds).  As mentioned above, this probability distribution is specified implicitly via a randomized algorithm that returns an actual acceleration (again for all birds).  This randomized algorithm is the AMPC algorithm, which inherits its randomization from the randomized PSO procedure it deploys.  

When the controller computes an acceleration, it assumes that the attacker does {\em{not}} introduce any disturbances; i.e., the controller uses the following model:
\vspace*{-4mm}\begin{eqnarray}
 \xv_i(t + 1) &=& \xv_i(t) + \vv_i(t+1) \qquad \forall~i\,{\in}\,\{1,\ldots,B\}, \nonumber \\
 \vv_i(t + 1) &=& \vv_i(t) + \va_i(t), \label{eq:noattack} %\\[-6mm]
\end{eqnarray}
where $\va(t)$ is the only control variable. Note that the controller chooses its next action $\va(t)$ based on the current configuration $(\xv(t),\vv(t))$ of the flock using MPC. The current configuration may have been influenced by the disturbance $\vec{d}(t-1)$ introduced by the attacker in the previous time step.  Hence, the current state need not be the state predicted by the controller when performing MPC in step $t-1$. Moreover, depending on the severity of the attacker action $\vec{d}(t-1)$, the AMPC procedure dynamically adapts its behavior, i.e.\ the choice of horizon $h$, in order to enable the controller to pick the best control action $\vec{a}(t)$ in response.

\subsection{Attacker's Strategies}

We are interested in evaluating the resilience of our V-formation controller when it is threatened by an attacker that can remove a certain number of birds from the flock, or manipulate a certain number of birds by taking control of their actuators (modeled by the displacement term in equation~(\ref{eq:trans})).
We assume that the attack lasts for a limited amount of time, after which the controller attempts to bring the system back into the good set of states. When there is no attack, the system behavior is the one given by equation~(\ref{eq:noattack}).

Note that there can be many different criteria for evaluating the success of an attack,  %(see Remark~\ref{remark:criteria})
but in our experiments, the controller is declared the winner if it can bring the flock to V-formation.
We consider three attack strategies (but see the future work discussion in Section~\ref{sec:conclusion}), each of which defines a V-formation game.

\vspace*{-0.5mm}\paragraph{\bf Remove Birds Game.}
In an RBG, the attacker selects a subset of $R$ birds, where $R\,{\ll}\,B$, and removes them from the flock.  The removal of bird $i$ from the flock at time $t\,{=}\,0$ can be simulated in our framework by allowing the attacker to set the displacement $\vd_i(0)$ for bird $i$ to $\infty$.  We assume that the flock is in a V-formation at time $t\,{=}\,0$.  
Thus, the goal of the controller is to bring the flock back into a V-formation consisting of $B\,{-}\,R$ birds.
%he controller needs to find the best adjustments in velocity $a_i$ for all remaining birds $i \in N - R$ during its turn. %$i \in N \wedge i \notin R$.
%Essentially, this results in a single-move game for the adversary. 
In an RBG, the attacker plays only one move.
When picking birds, the attacker is able to decide which birds will have the greatest negative impact on the flock's fitness when removed from the flock. Apart from seeing if the controller can bring the flock back to a V-formation, we also analyze the time it takes the controller to do so. 
%return to a v-formation for $R \leq \lceil\log(N)\rceil$ and 

% \todo[inline]{SAS: I would only suggest that the size R of the subset of
% birds removed from the flock (of size N) be such that R << N.  O/w I am
% not sure how interesting this game is.  Jesse has simulation results for
% R=1 and N=7.  Also, we should consider this game with and without process
% noise (PN), as Jesse has shown that the resiliency of the flock to remain
% in a V is highly dependent on the magnitude of PN.  It does very well with
% no PN or small PN, but resilience seems to degrade with increasing PN.}
%
%\begin{theorem}
%For any birds picked by the attacker, where $\left\vert{N - R}\right\vert \geq 3$, the planner can find 
%accelerations for each remaining bird in $N$ that will finally lead to a state $s^{*}$ such that cost 
%$J(s^{*})\{\leqslant}\,\varphi$.
%\end{theorem}

\vspace*{-0.5mm}\paragraph{\bf Random Displacement Game.}
In an RDG, the attacker chooses the displacement vector for a fixed number $R$ of birds uniformly from the space $[0,M]\times[0,2\pi]$. This means that the magnitude of the displacement vector is picked from the interval $[0,M]$, and the direction of the displacement vector is picked from the interval $[0,2\pi]$. We vary $M$ in our experiments. The $R$ birds that are picked in different steps are not necessarily the same, as the attacker makes this choice uniformly at random at runtime as well.
%In our second game, each player has control over all birds in the flock. The flock starts in a V-formation. However, both players have different goals and strategies. While the controller wants to keep the flock in a V-formation, the adversarial player tries to disrupt the V. Both players use the same planning approach but the controller tries to minimize the fitness function while the adversary tries to maximize the fitness in each step.
%In our second game, the adversarial player introduces malicious birds into the flock. These birds are controlled by the other player and hence can perturb the flock. To do so, the adversary adds small amounts of noise to this bird to distract the flock and disturb the v-formation. If this alone is not successful, the adversary can use a greater amount of noise to achieve the goal. However, this allows the controller to identify the adversary and henceforth ignore the malicious bird. 
The game starts from an initial V-formation. The attacker is allowed a fixed number of moves, say $20$, after which the displacement vector is identically $0$ for all birds.  The controller, which has been running in parallel with the attacker, is then tasked with moving the flock back to a V-formation, if necessary.
%
\vspace*{-0.5mm}\paragraph{\bf{AMPC Game.}}
An AMPC game is similar to an RDG except that the attacker does not use a uniform distribution to determine the displacement vector. The attacker is advanced and calculates the displacement (that will be the worst for the controller) using the AMPC procedure. See Figure~\ref{fig:ampc}.  In detail, the attacker applies AMPC, but assumes the controller applies zero acceleration. Thus, the attacker uses the following model of the flock dynamics:
\vspace*{-1mm}\begin{eqnarray}
 \xv_i(t + 1) &=& \xv_i(t) + \vv_i(t+1) + \vd_i(t) \qquad \forall~i\,{\in}\,\{1,\ldots,B\}, \nonumber \\
 \vv_i(t + 1) &=& \vv_i(t). \label{eq:attack} %\\[-6mm]
\end{eqnarray}
Note that the attacker is still allowed to have $\vd_i(t)$ be nonzero for a small number $R$ of birds. However, it can choose which $R$ birds it picks in each step.  It uses the AMPC procedure to simultaneously pick the $R$ birds and their displacements.
%Being a fair game, both players have the same capabilities. This means the controller as well as the adversary are able to use receding horizons to try to predict the best moves for their individual birds.

%\begin{theorem}
%
%\end{theorem}

%\paragraph{\bf Game 3.}%: Interior Lines}
% In our third game the adversary has only access to a specific subset of the birds. One could consider the attacker to add a set of malicious birds $M$ to the existing flock $N$.  Additionally we assume the controller is able to detect the attacker and hence the adversarial player needs to wait for the opportune moment to perform the actual attack. This means, the adversarial player can disrupt the V-formation slightly but only has one single move to interrupt and perturb the V-formation permanently. 
% \todo[inline] {Lukas: some important questions: the ATTACKER-ARES only controls the malicious birds and the CTL-ARES only the 'good' birds. however, does the CTL-ARES consider the malicious birds in its planning as 'good' birds? same for the ATTACKER-ARES. To me it would make sense, that the ATTACKER-ARES knows which ones are malicious birds and which ones are 'good' birds, but the CTL-ARES does not. So the CTL-ARES would consider ALL birds ($M \cup N$) but only controls the 'good' ones ($N$) -- i hope this makes any sense.}
%The third game is very similar to the second. However, when performing the final move, the attacker can decide whether it is more beneficial to introduce noise with a great magnitude to the flock or simply remove a specific number of birds from the flock. Again, we consider this a fair game where both players are able to use receding horizons do identify potential moves. Furthermore, we allow the adversary to remove up to $\log(N)$ birds from the flock.
%\subsection{Implementation: the Game is on}
%\label{sec:implementation}
%
%\todo[inline]{The following section would be the new implementation of our algorithm that deals with stochastic MDP and two-player games.}
%
% For this work, we extended the original \emph{deterministic Markov Decision Process} presented by Lukina et al.~\cite{lukina2016arxiv} to a \emph{classical MDP}~\cite{russellnorvig} by adding noise to the transition relation of the MDP. By doing so, we improved the original model and made it more realistic.
%
%We added and analyzed two different types of noises, processing noise ($\pnoise$) and actuator noise ($\anoise$). $\pnoise$ is applied to the position of each bird in our flock and changes the transition relation as follows
%\vspace*{-1mm}\begin{eqnarray*}
%\label{eq:pnoise_model}
% \xv_i(t + 1) &=& \xv_i(t) + \vv_i(t+1) + \pnoise %\label{eq:x_anoise},\\
% \vv_i(t + 1) &=& \vv_i(t) + \va_i(t) \label{eq:v_anoise},\\[-6mm]
%\end{eqnarray*}
%where $\pnoise \sim \mathcal{N}(0, \sigma^2)$. Here, $\sigma$ 

%In contrast, actuator noise is added to the acceleration action of the transition relation.
%\vspace*{-1mm}\begin{eqnarray*}
%\label{eq:model}
 %\xv_i(t + 1) &=& \xv_i(t) + \vv_i(t+1)\label{eq:x_anoise},\\
 %\vv_i(t + 1) &=& \vv_i(t) + \va_i(t) + \anoise\label{eq:v_anoise},\\[-6mm]
%\end{eqnarray*}

%\noindent where $\anoise \sim \mathcal{N}(0, \sigma^2)$. For our experiments we tried different $\sigma$, i.e. $\sigma = 0.05, 0.1, 0.2, 0.25$ and $0.3$.

%\begin{remark}\label{remark:criteria}
%Even though we use reaching V-formation as our success criterion (for the controller), we could have also used other criteria to decide if the attacker has been successful. For example, one could have used following criteria.
%
%\begin{itemize}
%\item \emph{Energy attack} is considered successful when a flock is not traveling in a V-formation for a certain amount of time. 
%
%\vspace*{1mm}\item \emph{Collisions} occur when two birds are in dangerous proximity from each other. This may happen through spoofing of existing birds or adversarial birds deliberately trying to lead to collisions with the others.
%
%\vspace*{1mm}\item \emph{Heading change} brings success, when the entire flock is diverged from its original direction (mission target) by a certain degree. 
%\end{itemize}
%\end{remark}

\begin{theorem}[AMPC resilience in a C-A game]
\label{thm:resilience}
Given a controller-attacker game, there is a finite maximum horizon $h_{\mathit{max}}$ and a finite maximum number of game-execution steps $m$ such that AMPC controller will win the controller-attacker game in $m$ steps with probability one.
\end{theorem}

\begin{proof}
Since the flock MDP (defined by Equation~6) is controllable, the PSO algorithm we use is fair, and the attack has a bounded duration, the proof of the theorem follows from Theorem~\ref{thm:ampc}. 
\end{proof}

\begin{remark}
While Theorem~\ref{thm:resilience} states that the controller is expected to win with probability one, we expect winning probability to be possibly lower than one in many cases because: (1)~the maximum horizon $h_{\mathit{max}}$ is fixed in advance, and so is (2) the maximum number of execution steps $m$; (3) the underlying PSO algorithm is also run with bounded number of particles and time.
\end{remark}


\section{Numerical Experiments: Diffusion Bridge Sampling}
\label{sec:numerics}
% !TEX root = main.tex

\begin{figure*}[t!]
  \centering
\includegraphics[width=0.325\textwidth]{./figs/0_simp_c1c2.pdf}
\includegraphics[width=0.315\textwidth]{./figs/0_simp_c3c4.pdf}
\includegraphics[width=0.26\textwidth]{./figs/0_simp_quad_bnds.pdf}
  \caption{We verify that the performative risk bounds in Assumption~\ref{ass:exist_V} are satisfied in the example discussed in Section~\ref{sec:example}. (a) As a function of $r$ (the radius of the domain where the inequalities hold), we show the tightest constants $c_1$ and $c_2$ for the bound. We also plot $\sqrt{c_1/c_2}r$, which is the radius of a neighborhood of $x=0$ to which Theorem~\ref{th:perturb1} can be applied. (b) As a function of $r$, we show the tightest constants for $c_3$ and $c_4$. (c) Choosing the $c_1$ and $c_2$ constants for $r = 0.5$, we visualize how the quadratic bounds hold for the performative risk locally.}
  \label{fig:simp_ex_demo}
\end{figure*}
In this section, we revisit the models introduced in Section~\ref{sec:example}. We demonstrate how the results of Sections~\ref{sec:analysis_prm} and~\ref{sec:analysis_RGD} can be applied. First, we show that the example satisfies Assumption~\ref{ass:exist_V} and we calculate its corresponding constants. Second, we apply Theorem~\ref{th:perturb1} and show the theoretical convergence rates match simulated trajectories. 
Finally, we also apply Theorem~\ref{th:perf_align} to the example from Section~\ref{sec:simple_ex} and characterize the class of distribution shifts satisfy the performative alignment condition.

\subsection{Checking the curvature of the performative risk and region of convergence}

Recall the example from Section~\ref{sec:simple_ex}, where $x$ was a scalar, the loss function was the squared error, and the decision-dependent distribution was a Bernoulli random variable whose distribution was determined by $p(\cdot)$. In this section, we consider the specific decision-dependent distribution shift $p = \varphi$, which is defined in Equation~\eqref{eq:varphi_def}.

When we consider this example, we can see that the bounds on Assumption~\ref{ass:exist_V} cannot hold globally, which matches our previous observation that there are multiple isolated performative risk minimizers. However, these bounds may hold locally: we can view the constants $(c_i)_{i=1}^4$ from Assumption~\ref{ass:exist_V} as a function of the size of the domain $r$.

For concreteness, let us focus on the equilibrium point $x = 0$. Recall that Assumption~\ref{ass:exist_V} must hold locally, on the domain $\{ x : |x-x^*| \le r\}$. As we increase $r$, the constants will worsen; we visualize this in Figure~\ref{fig:simp_ex_demo}(a)--(b). Note that these bounds only have to hold locally around the equilibria, as visualized in Figure~\ref{fig:simp_ex_demo}(c). Furthermore, the gradient bounds in Assumption~\ref{ass:exist_V} cannot hold beyond $r > 0.40$, since $\nabla PR(x) = 0$ at that point.

Recall that the convergence results of Theorem~\ref{th:perturb1} can only apply to all initial conditions satisfying $|x_0 - x^*| < \sqrt{c_1/c_2}r$; we visualize this as well in Figure~\ref{fig:simp_ex_demo}(a). 
On the set $(0,0.40]$, we can see the quantity $\sqrt{c_1/c_2}r$ is the largest at $r = 0.4$, with constants $c_1 = 0.50$ and $c_2 = 1.78$. 
Thus, around the equilibrium $x = 0$, the theorem can be applied to all points in the set $\{ x : |x| \le 0.21 \}$, with $\delta = 0$. Thus, our theorem shows that all points in this neighborhood of $x = 0$ will converge. This under-approximates the true region of attraction, which we numerically saw to be $\{ x : x < 0.23 \}$.



\subsection{Performative alignment with squared error and Bernoulli distributions}
\label{sec:perf_align_ex}

We again consider the example from Section~\ref{sec:simple_ex}. However, in this section, we consider a general decision-dependent distribution shift $p(\cdot)$. 
% Recall the example from Section~\ref{sec:simple_ex}, where $x$ was a scalar, the loss function was the squared error, and the decision-dependent distribution was a Bernoulli random variable whose distribution was determined by $p(\cdot)$.  
We suppose that $p(0) = 0$ and $p(1) = 1$, so we have two performative risk minimizers as in our previous example. We have $\nabla_{x_1}R(x,x) = x - p(x)$ and $\nabla_{x_2}R(x,x) = (1/2 - x) p'(x)$. 
The performative alignment condition becomes:
\begin{equation}
    \label{eq:perf_align_ex}
    |1/2 - x|^2 |p'(x)|^2 \le (p(x)-x)(1/2 - x)p'(x)
\end{equation}
Theorem~\ref{th:perf_align} states that if this condition holds for all $x \in (0,c)$, then any initial conditions $x_0 \in (0,c)$ will converge to $x = 0$. Similarly, if this condition holds for all $x \in (c,1)$, then all initial conditions in $(c,1)$ will converge to $x = 1$. Theorem~\ref{th:perf_align} also implies that this condition cannot be satisfied for all $x \in (0,1)$, as then these initial conditions would converge to \textit{both} $x = 0$ and $x = 1$.

If we suppose that $p(\cdot)$ is monotonic on $(0,1)$, i.e. $p'(x) \ge 0$, we can also interpret the performative alignment condition as follows. For $x \in (1/2,1)$, the performative alignment condition becomes $p(x) - x \ge (1/2 - x)p'(x)$. In this regime, $(1/2 - x)p'(x) \le 0$. In this setting, if $p(x) - x$ is too negative, the RGD flow will push $x$ away from the nearby minimizer $x = 1$. Similarly, for $x \in (0,1/2)$, the condition becomes $p(x) - x \le (1/2 - x)p'(x)$. In this regime, $(1/2 - x)p'(x) \ge 0$, and the condition states that $p(x) - x$ cannot be too large, or the RGD flow will push $x$ away from the minimizer $x = 0$.

In this section, we used Theorem~\ref{th:perf_align} to identify conditions on the decision-dependent distribution shift $p(\cdot)$ which ensure that the performative risk does not increase even when the dynamics follow repeated gradient descent.
For this example, the condition is that $p$ satisfies Equation~\eqref{eq:perf_align_ex} for all $x \in (0,c)$. 
More generally, the performative alignment condition allow us to specify a class of distribution shifts which behave well with respect to performative risk minimization.




\section{Outlook and Future Work}
\label{sec:outlook}
\section{Conclusion and outlook}
\label{sec:outlook}
In the previous sections we presented a novel variational model for the reconstruction of highly subsampled dynamic MRI data where an anatomical scan (at high spatial resolution) has been acquired prior to the dynamic sequence. 
Combining radial golden angle sampling with a suitable time regularization, spatial TV regularization and with the infimal convolution of TV Bregman distances allowing to incorporate the structural information of the anatomical prior, we obtained spatially highly resolved reconstructions at a high temporal resolution. 

Summing up the results of tests on a simulated data set based on fMRI as well as on experimental small animal DCE-MRI data, we draw the following conclusions:
naturally, a simple least squares (LS) reconstruction of each individual frame could not provide meaningful results due to the severe undersampling. 
Adding a spatial TV regularization did not significantly increase the quality of the reconstructed images. 
As expected, this approach yielded piecewise constant images, but the ratio of sampled Fourier coefficients in $k$-space in comparison to the desired spatial resolution of the reconstructions was too small to obtain reasonable results. 
Remarkably, adding only Tikhonov regularization on the time derivative without any additional spatial regularization already resulted in by far more meaningful reconstructions than the LS approach, while the obtained images were still corrupted by heavy noise.
Integrating spatial TV regularization to the aforementioned model removed most of the noise and indeed provided high quality reconstructions. 
Incorporating structural information from the anatomical prior, we could then obtain very detailed results despite the severe subsampling enabling high temporal resolution.

In view of these promising results, we state some open questions and sketch additional ideas whose detailed study is left to future research.

We used the infimal convolution of TV Bregman distances to incorporate structural information from the anatomical prescan. 
Naturally, this gives rise to the question whether alternative means of incorporating structural prior information such as the concepts of weighted total variation (wTV) or directional total variation (dTV), respectively, both proposed in \cite{Ehrhardt2016}, yield significantly different results. 
In any case, it would be interesting to see how such a modified approach compares to the method proposed in this paper concerning quality of the reconstructed images, but also regarding computational complexity of solving the respective minimization problem. 

Moreover, the temporal coupling of time frames serves as a further starting point for future research. 
Here, we decided to apply Tikhonov regularization of the time derivative, however, one could also argue in favor of other concepts: 
since in the areas of application we considered in this paper the dynamic changes happen to take place in only a small portion of the entire image domain, decomposition of the dynamic sequence into a low rank part $L$ and a part $S$ which is sparse in some transform domain \cite{Tremoulheac:lowRankPlusSparsePrior,Otazo:lowRankPlusSparseMatrixDecomposition} could be an interesting alternative. 
Assuming that the dynamic changes mainly are contained in $S$, while $L$ ideally comprises the part staying constant over time, it seems particularly reasonable that the structures of the constant part of every time frame bear close resemblance to the structure of the anatomical prior image. 
Hence it would stand to reason to apply the infimal convolution of TV Bregman distances only to the low rank part leaving the sparse part untouched. 
However, against the backdrop of different dimensions of the low rank part of the dynamic sequence $L$ and the anatomical prior image $u_0$ it is not yet clear what would be the most suitable way of solving the corresponding optimization problem. 

Finally, with respect to experimental data, a more careful correction of artifacts due to different acquisition protocols between anatomical prescan and the dynamic sequence might be an interesting aspect.


\section*{Acknowledgments}
 AD was supported by the EPSRC under grant No. EP/J009636/1. NN is supported by EPSRC through a Roth Departmental Scholarship. GP is partially supported by the EPSRC under grants No. EP/J009636/1, EP/L024926/1, EP/L020564/1 and EP/L025159/1. Part of the work reported in this paper was done while NN and GP were visiting the Institut Henri Poincar\'{e} during the Trimester Program "Stochastic Dynamics Out of Equilibrium". The hospitality of the Institute and of the organizers of the program is greatly acknowledged. 

\appendix
\section{Estimates for the Bias and Variance}
\label{app:proofs}

\begin{proof}[of Lemma \ref{lemma:bias}]
Suppose that $(P_t)_{t \ge 0}$ satisfies \eqref{eq:hypocoercive estimate}.  Let $\pi_0$ be an initial distribution of $(X_t)_{t \ge 0}$ such that $\pi_0 \ll \pi$ and $h = \frac{d\pi_0}{d\pi} \in L^2(\pi)$.  Slightly abusing notation, we denote by $\pi_0 P_t$ the law of $X_t$ given $X_0 \sim \pi$.  Then
\begin{align*}
	\lVert \pi_0 P_t - \pi \rVert_{TV} = \left\lVert P_t^* h - 1 \right\rVert_{L^1(\pi)} \leq \left\lVert P_t^* \right\rVert_{L^2(\pi)\rightarrow L^2(\pi)} \left\lVert h - 1 \right\rVert_{L^2(\pi)} \leq Ce^{-\lambda t}\left\lVert h - 1 \right\rVert_{L^2(\pi)},
\end{align*}
where $P_t^*$ denotes the $L^2(\pi)$-adjoint of $P_t$.  Since $f$ is assumed to be bounded, we immediately obtain
$$
  \norm{\mathbb{E}[f(X_t) | X_0 \sim \pi_0] - \pi(f)} \leq C\Norm{f}_{L^\infty}e^{-\lambda t}\left(\mbox{Var}_{\pi}\left[\frac{d\pi_0}{d\pi}\right]\right)^{1/2},
$$
and so, for $X_0 \sim \pi_0$,
$$
	\norm{\pi_T(f) - \pi(f)} \leq \frac{C}{\lambda T}{\left(1 - e^{-\lambda t}\right)}\lVert f \rVert_{L^\infty}\left(\mbox{Var}_{\pi}\left[\frac{d\pi_0}{d\pi}\right]\right)^{1/2},
$$
as required.
  \qed 
\end{proof}

\begin{proof}[of Lemma \ref{lemma:variance}]
Given $f \in L^2(\pi)$, for fixed $T > 0$, 
\begin{equation}
  \chi_T(x) := \int_{0}^{T} \left(\pi(f) - P_t f(x)\right)\mathrm{d}t.
\end{equation}
Then we have that $\chi_T \in \mathcal{D}(\mathcal{L})$ and $\mathcal{L}\chi_T  = f - P_T f$, moreover
\begin{align*}
  \Norm{\chi_{T} - \chi_{T'}}_{L^2(\pi)} &= \Norm{\int^{T'}_{T} P_t(f)-\pi(f)\,\mathrm{d}t}_{L^2(\pi)}\\
  &\leq C\Norm{f}_{L^2(\pi)}\int_{T}^{T'}e^{-\lambda t}\,\mathrm{d}t,
\end{align*}
so that $\lbrace \chi_T \rbrace_{T \geq 0}$ is a Cauchy sequence in $L^2(\pi)$ converging to $\chi = \int_0^\infty \left(\pi(f) - P_tf\right)\mathrm{d}t$.  Since $\mathcal{L}$ is closed and
$$
  (\mathcal{L}\chi_T, \chi_T) \rightarrow (f-\pi(f), \chi),\quad T \rightarrow \infty,
$$
in $L^2(\pi)$, it follows that $\chi\in\mathcal{D}(\mathcal{L})$ and $\mathcal{L}\chi = f - \pi(f)$.  Moreover,
$$
  \Norm{\chi}_{L^{2}(\pi)} \leq \int_0^\infty \Norm{P_t(f) - \pi(f)}_{L^2(\pi)}\,\mathrm{d}t \leq K_{\lambda}\Norm{f-\pi(f)}_{L^2(\pi)},\quad 
$$
where $K_{\lambda} = C\int_0^\infty e^{-\lambda t}\,\mathrm{d}t$. 
Since we assume that $f$ is smooth, the coefficients are smooth and $\gen$ is hypoelliptic, then $\mathcal{L}\chi = f-\pi(f)$ implies that  $\chi \in C^{\infty}(\R^d)$, and thus we can apply It\^{o}'s formula to $\chi(X_t)$ to obtain:
$$
\frac{1}{T}\int_0^T \left[f(X_t) - \pi(f)\right]\,\mathrm{d}t = \frac{1}{T}\left[\chi(X_0) - \chi(X_T)\right] + \frac{1}{T}\int_0^T \nabla\chi(X_t)\sigma(X_t)\,\mathrm{d}W_t.
$$
One can check that the conditions of \cite[Theorem 7.1.4]{EthierKu86} hold.  In particular, the following central limit theorem follows
$$
	\frac{1}{\sqrt{T}}\int_0^T \nabla\chi(X_t)\sigma(X_t)\,\mathrm{d}W_t \xrightarrow{d} \mathcal{N}(0,2\sigma^2_f),\quad \mbox{ as } T \rightarrow \infty.
$$
By Theorem \ref{theorem:invariance_theorem}, the generator $\gen$ has the form
$$
	\gen = \pi^{-1}\nabla\cdot\left(\pi\Sigma \nabla \cdot \right) + \gamma\cdot\nabla,
$$
where $\nabla\cdot(\pi \gamma) = 0$.  It follows that
\begin{equation}
\label{eq:variance_equation}
\sigma^2_f = \inner{\Sigma \nabla\chi}{\nabla\chi}_{L^2(\pi)} = -\inner{\gen \chi}{\chi}_{L^2(\pi)}  = \inner{\chi}{f}_{L^2(\pi)} < \infty.
\end{equation}
First suppose that $X_0 \sim \pi$.  Then $(\chi(X_t))_{t\geq 0}$ is a stationary process, and so 
$$
	\frac{1}{\sqrt{T}}\left(\chi(X_0) - \chi(X_T)\right) \rightarrow 0,\quad \mbox{a.s as } T \rightarrow \infty.
$$
From which \eqref{eq:CLT} follows.  More generally, suppose that $X_0 \sim \pi_0$, where $\pi_0(x) = h(x)\pi(x)$ for $h \in L^2(\pi)$.  If $f \in L^\infty(\pi)$, then by \eqref{lemma:bias},
\begin{align*}
	|\chi(x)| &\leq \int_0^\infty |\pi(f) - P_t f(x)|\,dt \\
			  &\leq \int_0^\infty \lVert f\rVert_{L^\infty}\lVert\pi - \pi_0 P_t\rVert_{TV}\,dt \\
			  & \leq \frac{C}{\lambda }\lVert f\rVert_{L^\infty}\left(\mbox{Var}_{\pi}\left[\frac{d\pi_0}{d\pi}\right]\right)^{1/2},
\end{align*} 
so that $\chi \in L^\infty(\pi)$.  Therefore $\frac{1}{\sqrt{T}}(\chi(X_0)- \chi(X_T)) \xrightarrow{p} 0$ as $T \rightarrow \infty$, and so \eqref{eq:CLT} holds in this case, similarly.
\end{proof}



\section{Proofs of Section \ref{sec:perturbed_langevin}}
\label{app:hypocoercivity}

\begin{proof} of Lemma \ref{lem:hypoellipticity}
	We first note that $\gen$ in \eqref{eq:generator} can be written in the ``sum of squares'' form:
	$$
	\mathcal{L}=A_{0}+\frac{1}{2}\sum_{k=1}^{d}A_{k}^{2},
	$$
	where 
	$$
	A_{0}=M^{-1}p\cdot\nabla_{q}-\nabla_{q}V\cdot\nabla_{p}-\mu J_{1}\nabla_{q}V\cdot\nabla_{q}-\nu J_{2}M^{-1}p\cdot\nabla_{p}-\Gamma M^{-1}p\cdot\nabla_{p}
	$$
	and
	$$
	A_{k}=e_{k}\cdot\Gamma^{1/2}\nabla_{p}, \quad k = 1,\ldots, d.
	$$
	Here $\lbrace e_k \rbrace_{k=1,\ldots, d}$ denotes the standard Euclidean basis and $\Gamma^{1/2}$ is the unique positive definite square root of the matrix $\Gamma$.   The relevant commutators turn out to be
	\[
	[A_{0},A_{k}]=e_{k}\cdot\Gamma^{1/2} M^{-1}(\Gamma\nabla_{p}-\nabla_{q}-\nu J_{2}\nabla_{p}), \quad k = 1,\ldots, k.
	\]
	Because $\Gamma$ has full rank on $\R^d$, it follows that 
	\[
	\Span\{A_{k}:k=1,\ldots d\}=\Span\{\partial_{p_{k}}:k=1,\ldots,d\}.
	\]
	Since
	\[
	e_{k}\cdot\Gamma^{1/2} M^{-1}(\Gamma\nabla_{p}-\nu J_{2}\nabla_{p})\in\Span\{A_{j}:j=1,\ldots d\},\quad k=1,\ldots,d,
	\]
	and $\Span(\lbrace \Gamma^{1/2}M^{-1}\nabla_q \,:\, k=1,\ldots, d \rbrace) = \Span\{\partial_{q_{k}}:k=1,\ldots,d\}$, it follows that
	\[
	\Span(\lbrace A_{k}:k=0,1,\ldots,d\rbrace\cup\{[A_{0},A_{k}]:k=1,\ldots,d\}) = \R,
	\]
	so the assumptions of H\"{o}rmander's theorem hold.\qed
\end{proof}

\subsection{The overdamped limit}

The following is a technical lemma required for the proof of Proposition \ref{prop: overdamped limit}:
\begin{lemma}
	\label{lem:bounded p}Assume the conditions from Proposition \ref{prop: overdamped limit}.
	Then for every $T>0$ there exists $C>0$ such that 
	\[
	\mathbb{E} \left( \sup_{0\le t\le T}\vert p_{t}^{\epsilon}\vert^{2} \right) \le C.
	\]
	
\end{lemma}
\begin{proof}
	Using variation of constants, we can write the second line of (\ref{eq:rescaling})
	as 
	\[
	p_{t}^{\epsilon}=e^{-\frac{t}{\epsilon^{2}}(\nu J_{2}+\Gamma)M^{-1}}p_{0}-\frac{1}{\epsilon}\int_{0}^{t}e^{-\frac{(t-s)}{\epsilon^{2}}(\nu J_{2}+\Gamma)M^{-1}}\nabla_{q}V(q_{s}^{\epsilon})\mathrm{d}s+\frac{1}{\epsilon}\sqrt{2\Gamma}\int_{0}^{t}e^{-\frac{(t-s)}{\epsilon^{2}}(\nu J_{2}+\Gamma)M^{-1}}\mathrm{d}W_{s}.
	\]
	We then compute 
	\begin{align}
	\mathbb{E}\sup_{0\le t\le T}\vert p_{t}^{\epsilon}\vert^{2} & =\sup_{0\le t\le T}\left\lvert e^{-\frac{t}{\epsilon^{2}}(\nu J_{2}+\Gamma)M^{-1}}p_{0}\right\rvert^{2}+\frac{1}{\epsilon^{2}}\mathbb{E}\sup_{0\le t\le T}\left\lvert\int_{0}^{t}e^{-\frac{(t-s)}{\epsilon^{2}}(\nu J_{2}+\Gamma)M^{-1}}\nabla_{q}V(q_{s}^{\epsilon})\mathrm{d}s\right\rvert^{2}\nonumber \\
	& +\frac{1}{\epsilon^{2}}\mathbb{E}\sup_{0\le t\le T}\left\lvert\sqrt{2\Gamma}\int_{0}^{t}e^{-\frac{(t-s)}{\epsilon^{2}}(\nu J_{2}+\Gamma)M^{-1}}\mathrm{d}W_{s}\right\rvert^{2}\nonumber \\
	& -\frac{1}{\epsilon}\mathbb{E}\sup_{0\le t\le T}\left(e^{-\frac{t}{\epsilon^{2}}(\nu J_{2}+\Gamma)M^{-1}}p_{0}\cdot\int_{0}^{t}e^{-\frac{(t-s)}{\epsilon^{2}}(\nu J_{2}+\Gamma)M^{-1}}\nabla_{q}V(q_{s}^{\epsilon})\mathrm{d}s\right)\label{eq:P^2}\\
	& +\frac{1}{\epsilon}\mathbb{E}\sup_{0\le t\le T}\left(e^{-\frac{t}{\epsilon^{2}}(\nu J_{2}+\Gamma)M^{-1}}p_{0}\cdot\frac{1}{\epsilon}\sqrt{2\Gamma}\int_{0}^{t}e^{-\frac{(t-s)}{\epsilon^{2}}(\nu J_{2}+\Gamma)M^{-1}}\mathrm{d}W_{s}\right)\nonumber \\
	& -\frac{1}{\epsilon^{2}}\mathbb{E}\sup_{0\le t\le T}\left(\int_{0}^{t}e^{-\frac{(t-s)}{\epsilon^{2}}(\nu J_{2}+\Gamma)M^{-1}}\nabla_{q}V(q_{s}^{\epsilon})\mathrm{d}s\cdot\sqrt{2\Gamma}\int_{0}^{t}e^{-\frac{(t-s)}{\epsilon^{2}}(\nu J_{2}+\Gamma)M^{-1}}\mathrm{d}W_{s}\right).\nonumber 
	\end{align}
	Clearly, the first term on the right hand side of (\ref{eq:P^2})
	is bounded. For the second term, observe that 
	\begin{equation}
	\frac{1}{\epsilon^{2}}\mathbb{E}\sup_{0\le t\le T}\left\lvert\int_{0}^{t}e^{-\frac{(t-s)}{\epsilon^{2}}(\nu J_{2}+\Gamma)M^{-1}}\nabla_{q}V(q_{s}^{\epsilon})\mathrm{d}s\right\rvert^{2}\le\frac{1}{\epsilon^{2}}\sup_{0\le t\le T}\int_{0}^{t}\left\lVert e^{-\frac{(t-s)}{\epsilon^{2}}(\nu J_{2}+\Gamma)M^{-1}}\right\rVert^{2}\mathrm{d}s\label{eq:estimate1}
	\end{equation}
	since $V \in C^1(\mathbb{T}^d)$ and therefore $\nabla_{q}V$ is bounded. By the basic matrix exponential estimate
	$\Vert e^{-t(\nu J_{2}+\Gamma)M^{-1}}\Vert\le Ce^{-\omega t}$ for
	suitable $C$ and $\omega$, we see that (\ref{eq:estimate1}) can
	further be bounded by 
	\[
	\frac{1}{\epsilon^{2}}C\sup_{0\le t\le T}\int_{0}^{t}e^{-2\omega\frac{(t-s)}{\epsilon^{2}}}\mathrm{d}s=\frac{C}{2\omega}\left(1-e^{-2\omega\frac{T}{\epsilon^{2}}}\right),
	\]
	so this term is bounded as well. The third term is bounded by the
	Burkholder\textendash Davis\textendash Gundy inequality and a similar
	argument to the one used for the second term applies. The cross terms can
	be bounded by the previous ones, using the Cauchy-Schwarz inequality
	and the elementary fact that $\sup(ab)\le\sup a\cdot\sup b$ for $a,b>0$, so the
	result follows. \qed
\end{proof}

\begin{proof}
	[of Proposition \ref{prop: overdamped limit}] Equations (\ref{eq:rescaling})
	can be written in integral form as
	\[
	(\nu J_{2}+\Gamma)q_{t}^{\epsilon}=(\nu J_{2}+\Gamma)q_{0}^{\epsilon}+\frac{1}{\epsilon}\int_{0}^{t}(\nu J_{2}+\Gamma)M^{-1}p_{s}^{\epsilon}\mathrm{d}s-\mu\int_{0}^{t}(\nu J_{2}+\Gamma)J_{1}\nabla_{q}V(q_{s}^{\epsilon})\mathrm{d}s
	\]
	and 
	\begin{equation}
	-\int_{0}^{t}\nabla V(q_{s}^{\epsilon})\mathrm{d}s-\frac{1}{\epsilon}\int_{0}^{t}(\nu J_{2}+\Gamma)M^{-1}p_{s}^{\epsilon}\mathrm{d}s+\sqrt{2\Gamma}W(t)=\epsilon(p_{t}^{\epsilon}-p_{0}),\label{eq:rescaled p equation-1}
	\end{equation}
	where the first line has been multiplied by the matrix $\nu J_{2}+\Gamma$.
	Combining both equations yields
	\[
	q_{t}^{\epsilon}=q_{0}^{\epsilon}-\int_{0}^{t}(\nu J_{2}+\Gamma)\nabla_{q}V(q_{s}^{\epsilon})\mathrm{d}s-\epsilon(\nu J_{2}+\Gamma)^{-1}(p_{t}^{\epsilon}-p_{0})-\mu\int_{0}^{t}J_{1}\nabla_{q}V(q_{s})\mathrm{d}s+(\nu J_{2}+\Gamma)^{-1}\sqrt{2\Gamma}W_{t}.
	\]
	Now applying Lemma \ref{lem:bounded p} gives the desired result,
	since the above equation differs from the integral version of (\ref{eq:overdamped limit})
	only by the term $\epsilon(\nu J_{2}+\Gamma)^{-1}(p_{t}^{\epsilon}-p_{0})$
	which vanishes in the limit as $\epsilon\rightarrow0$. \qed
\end{proof}

\subsection{Hypocoercivity}

The objective of this section is to prove that the perturbed dynamics
(\ref{eq:perturbed_underdamped}) converges to equilibrium
exponentially fast, i.e. that the associated semigroup $(P_t)_{t\ge0}$ satisfies the estimate \eqref{eq:hypocoercive estimate}. We we will be using the theory of hypocoercivity outlined in
\cite{villani2009hypocoercivity} (see also the exposition in \cite[Section 6.2]{pavliotis2014stochastic}).
We provide a brief review of the theory of hypocoercivity.
\\\\
Let $(\mathcal{H},\langle\cdot,\cdot\rangle)$ be a real separable
Hilbert space and consider two unbounded operators $A$ and $B$ with
domains $D(A)$ and $D(B)$ respectively, $B$ antisymmetric. Let
$S\subset\mathcal{H}$ be a dense vectorspace such that $S\subset D(A)\cap D(B)$,
i.e. the operations of $A$ and $B$ are authorised on $S$. The theory
of hypocoercivity is concerned with equations of the form 
\begin{equation}
\label{eq:abstract fp equation}
\partial_{t}h+Lh=0,
\end{equation}
and the associated semigroup $(P_t)_{t\ge0}$ generated by $L=A^{*}A-B$. Let
us also introduce the notation $K=\ker L$. With the choices $\mathcal{H}=L^{2}(\widehat{\pi})$,
$A=\sigma\nabla_{p}$ and $B=M^{-1}p\cdot\nabla_{q}-\nabla_{q}V\cdot\nabla_{p}-\mu J_{1}\nabla_{q}V\cdot\nabla_{q}-\nu J_{2}M^{-1}p\cdot\nabla_{p},$
it turns out that $L$ is the (flat) $L^2(\mathbb{R}^{2d})$-adjoint of the generator $\mathcal{L}$ given in \eqref{eq:generator} and therefore equation \eqref{eq:abstract fp equation} is the Fokker-Planck equation associated to the dynamics \eqref{eq:perturbed_underdamped}. 
	In many situations of practical interest, the operator $A^{*}A$ is
	coercive only in certain directions of the state space, and therefore
	exponential return to equilibrium does not follow in general. In our
	case for instance, the noise acts only in the $p$-variables and therefore
	relaxation in the $q$-variables cannot be concluded a priori. However,
	intuitively speaking, the noise gets transported through the equations
	by the Hamiltonian part of the dynamics. This is what the theory of
	hypocoercivity makes precise. Under some conditions on the interactions
	between $A$ and $B$ (encoded in their iterated commutators), exponential
	return to equilibrium can be proved.
To state the main abstract theorem, we need the following definitions: 
\begin{definition}
	(Coercivity) Let $T$ be an unbounded operator on $\mathcal{H}$ with
	domain $D(T)$ and kernel $K$. Assume that there exists another Hilbert
	space $(\tilde{\mathcal{H}},\langle\cdot,\cdot\rangle_{\tilde{\mathcal{H}}})$,
	continuously and densely embedded in $K^{\perp}$. The operator is
	said to be $\lambda$-coercive if 
	\[
	\langle Th,h\rangle_{\tilde{\mathcal{H}}}\ge\lambda\Vert h\Vert_{\tilde{\mathcal{H}}}^{2}
	\]
	for all $h\in K^{\perp}\cap D(T)$. 
\end{definition}

\begin{definition}
	An operator $T$ on $\mathcal{H}$ is said to be relatively bounded
	with respect to the operators $T_{1},\ldots,T_{n}$ if the intersection
	of the domains $\cap D(T_{j})$ is contained in $D(T)$ and there
	exists a constant $\alpha>0$ such that 
	\[
	\Vert Th\Vert\le\alpha(\Vert T_{1}h\Vert+\ldots+\Vert T_{n}h\Vert)
	\]
	holds for all $h\in D(T)$. 
\end{definition}
We can now proceed to the main result of the theory.
\begin{theorem}{\cite[Theorem 24]{villani2009hypocoercivity}}
	\label{thm: hypocoercivity abstract}Assume there exists $N\in\mathbb{N}$
	and possibly unbounded operators $$C_{0},C_{1},\ldots,C_{N+1},R_{1},\ldots,R_{N+1},Z_{1},\ldots,Z_{N+1},$$
	such that $C_{0}=A$, 
	\begin{equation}
	[C_{j},B]=Z_{j+1}C_{j+1}+R_{j+1}\quad(0\le j\le N),\quad C_{N+1}=0,\label{eq:iterated commutators}
	\end{equation}
	and for all $k=0,1,\ldots,N$ 
	\begin{enumerate}[label=(\alph*)]
		\item \label{it:hypo1} $[A,C_{k}]$ is relatively bounded with respect to $\{C_{j}\}_{0\le j\le k}$
		and $\{C_{j}A\}_{0\le j\le k-1}$, 
		\item \label{it:hypo2} $[C_{k},A^{*}]$ is relatively bounded with respect to $I$ and $\{C_{j}\}_{0\le j\le k}$
		,
		\item \label{it:hypo3} $R_{k}$ is relatively bounded with respect to $\{C_{j}\}_{0\le j\le k-1}$
		and $\{C_{j}A\}_{0\le j\le k-1}$ and
		\item \label{it:hypo4} there are positive constants $\lambda_{i}$, $\Lambda_{i}$ such that
		$\lambda_{j}I\le Z_{j}\le\Lambda_{j}I$.
	\end{enumerate}
	Furthermore, assume that $\sum_{j=0}^{N}C_{j}^{*}C_{j}$ is $\kappa$-coercive
	for some $\kappa>0$.  Then, there exists $C\ge0$ and $\lambda>0$ such that 
	\begin{equation}
	\Vert P_t\Vert_{\mathcal{H}^{1}/K\rightarrow \mathcal{H}^{1}/K}\le Ce^{-\lambda t},\label{eq:hypocoercivity estimate}
	\end{equation}
	where $\mathcal{H}^{1}\subset\mathcal{H}$ is the subspace associated
	to the norm
	\begin{equation}
	\label{eq:abstractH1_norm}
	\Vert h\Vert_{\mathcal{H}^{1}}=\sqrt{\Vert h\Vert^{2}+\sum_{k=0}^{N}\Vert C_{k}h\Vert^{2}}
	\end{equation}
	and $K=\ker(A^{*}A-B)$. \end{theorem}
\begin{remark}
	Property (\ref{eq:hypocoercivity estimate}) is called \emph{hypocoercivity
		of $L$ on $\mathcal{H}^{1}:=(K^{\perp},\Vert\cdot\Vert_{\mathcal{H}^{1}})$.}
\end{remark}
If the conditions of the above theorem hold, we also get a regularization
result for the semigroup $e^{-tL}$ (see  \cite[Theorem A.12]{villani2009hypocoercivity}):
\begin{theorem}
	\label{thm:hypocoercive regularisation}Assume the setting and notation
	of Theorem \ref{thm: hypocoercivity abstract}. Then there exists
	a constant $C>0$ such that for all $k=0,1,\ldots,N$ and $t \in (0,1]$ the following
	holds:
	\[
	\Vert C_{k}P_{t}h\Vert\le C\frac{\Vert h\Vert}{t^{k+\frac{1}{2}}},\quad h\in\mathcal{H}.
	\]
\end{theorem}
\begin{proof}
	[of Theorem \ref{theorem:Hypocoercivity}]. We pove the claim by verifying
	the conditions of Theorem \ref{thm: hypocoercivity abstract}. Recall
	that $C_{0}=A=\sigma\nabla_{p}$ and 
	\[
	B=M^{-1}p\cdot\nabla_{q}-\nabla_{q}V\cdot\nabla_{p}-\mu J_{1}\nabla_{q}V\cdot\nabla_{q}-\nu J_{2}M^{-1}p\cdot\nabla_{p}.
	\]
	A quick calculation shows that 
	\[
	A^{*}=\sigma M^{-1}p-\sigma\nabla_{p},
	\]
	so that indeed 
	\[
	A^{*}A=\Gamma M^{-1}p\cdot\nabla_{p}-\nabla^{T}\Gamma\nabla=\mathcal{L}_{therm}
	\]
	and 
	\[
	A^{*}A-B=-\mathcal{L}^{*}.
	\]
	We make the choice $N=1$ and calculate the commutator 
	\[
	[A,B]=\sigma M^{-1}(\nabla_{q}+\nu J_{2}\nabla_{p}).
	\]
	Let us now set $C_{1}=\sigma M^{-1}\nabla_{q}$, $Z_{1}=1$ and $R_{1}=\nu\sigma M^{-1}J_{2}\nabla_{p}$,
	such that (\ref{eq:iterated commutators}) holds for $j=0$. Note
	that $[A,A]=0$\footnote{This is not true automatically, since $[A,A]$ stands for the array
		$([A_{j},A_{k}])_{jk}$.}, $[A,C_{1}]=0$ and $[A^{*},C_{1}]=0$. Furthermore, we have that
	\[
	[A,A^{*}]=\sigma M^{-1}\sigma.
	\]
	We now compute 
	\[
	[C_{1},B]=-\sigma M^{-1}\nabla^{2}V\nabla_{p}+\mu\sigma M^{-1}\nabla^{2}VJ_{1}\nabla_{q}
	\]
	and choose $R_{2}=[C_{1},B]$, $Z_{2}=1$ and recall that $C_{2}=0$
	by assumption (of Theorem \ref{thm: hypocoercivity abstract}). With those choices, assumptions \ref{it:hypo1}-\ref{it:hypo4} of Theorem \ref{thm: hypocoercivity abstract} are fulfilled. Indeed, assumption \ref{it:hypo1} holds trivially 
	since all relevant commutators are zero. Assumption \ref{it:hypo2} follows from the fact that $[A,A^{*}]=\sigma M^{-1}\sigma$ is clearly bounded
	relative to $I$. To verify assumption \ref{it:hypo3}, let us start with the
	case $k=1$. It is necessary to show that $R_{1}=\nu\sigma M^{-1}J_{2}\nabla_{p}$
	is bounded relatively to $A=\sigma\nabla_{p}$ and $A^{2}$.
	This is obvious since the $p$-derivatives appearing in $R_{1}$ can
	be controlled by the $p$-derivatives appearing in $A$. For $k=2,$
	a similar argument shows that $R_{2}=-\sigma M^{-1}\nabla^{2}V\nabla_{p}+\mu\sigma M^{-1}\nabla^{2}VJ_{1}\nabla_{q}$
	is bounded relatively to $A=\sigma\nabla_{p}$ and $C_{1}=\sigma M^{-1}\nabla_{q}$
	because of the assumption that $\nabla^{2}V$ is bounded. Note that it
	is crucial for the preceding arguments to assume that the matrices
	$\sigma$ and $M$ have full rank. Assumption \ref{it:hypo4} is trivially satisfied,
	since $Z_{1}$ and $Z_{2}$ are equal to the identity.  It remains to show that 
	\[
	T:=\sum_{j=0}^{N}C_{j}^{*}C_{j}
	\]
	is $\kappa$-coercive for some $\kappa>0$.  It is straightforward
	to see that the kernel of $T$ consists of constant functions and
	therefore 
	\[
	(\ker T)^{\perp}=\{\phi\in L^{2}(\mathbb{R}^{2d},\widehat{\pi}):\quad\widehat{\pi}(\phi)=0\}.
	\]
	Hence, $\kappa$-coercivity of $T$ amounts to the functional inequality
	\[
	\int_{\mathbb{R}^{2d}}\big(\vert\sigma M^{-1}\nabla_{q}\phi\vert^{2}+\vert\sigma\nabla_{p}\phi\vert^{2}\big)\mathrm{d}\widehat{\pi}\ge\kappa\bigg(\int_{\mathbb{R}^{2d}}\phi^{2}\mathrm{d}\widehat{\pi}-\left(\int_{\mathbb{R}^{2d}}\phi\mathrm{d}\widehat{\pi}\right)^{2}\bigg),\quad\phi\in H^{1}(\widehat{\pi}).
	\]
	Since the transformation $\phi\mapsto\psi$, $\psi(q,p)=\phi(\sigma^{-1}Mq,\sigma^{-1}p)$ is bijective on $H^{1}(\mathbb{R}^{2d},\widehat{\pi})$, the above is equivalent to 
	\[
	\int_{\mathbb{R}^{2d}}\big(\vert\nabla_{q}\psi\vert^{2}+\vert\nabla_{p}\psi\vert^{2}\big)\mathrm{d}\widehat{\pi}\ge\kappa\bigg(\int_{\mathbb{R}^{2d}}\psi^{2}\mathrm{d}\widehat{\pi}-\left(\int_{\mathbb{R}^{2d}}\psi\mathrm{d}\widehat{\pi}\right)^{2}\bigg),\quad\psi\in H^{1}(\widehat{\pi}),
	\]
	i.e. a Poincar\'{e} inequality for $\widehat{\pi}$. Since $\widehat{\pi}=\pi\otimes\mathcal{N}(0,M),$
	coercivity of $T$ boils down to a Poincar\'{e} inequality for $\pi$
	as in Assumption \ref{ass:bounded+Poincare}. This concludes the proof of the hypocoercive decay estimate
	(\ref{eq:hypocoercivity estimate}). Clearly, the abstract $\mathcal{H}^{1}$-norm from \eqref{eq:abstractH1_norm}
	is equivalent to the Sobolev norm $H^{1}(\widehat{\pi})$, and therefore it follows that there exist constants $C\ge0$ and $\lambda\ge0$ such that 
	\begin{equation}
	\label{eq:H1_decay}
	\Vert P_{t} f \Vert_{H^1(\widehat{\pi})}  \le C e^{-\lambda t} \Vert f \Vert _{H^1(\widehat{\pi})},
	\end{equation}
	for all $f \in H^1(\widehat{\pi})\setminus K$, where $K=\ker T$ consists of constant functions.  Let us now lift this estimate to $L^2(\widehat{\pi})$. There exist a constant $\tilde{C}\ge0$ such that 
	\begin{equation}
	\Vert h \Vert_{H^1(\widehat{\pi})} \le \tilde{C} \sum_{k=0}^2 \Vert C_k h \Vert_{L^2(\widehat{\pi})},
	\quad f \in H^1(\widehat{\pi}).
	\end{equation}
	Therefore, Theorem \ref{thm:hypocoercive regularisation} implies 
	\begin{equation}
	\label{eq:H1L2_reg}
	\Vert P_{1} f \Vert_{H^1(\widehat{\pi})} \le \tilde{C} \Vert f \Vert_{L^2(\widehat{\pi})},
	\quad f \in L^2(\widehat{\pi}), 
	\end{equation}
	for $t=1$ and a possibly different constant $\tilde{C}$. Let us now assume that $t\ge1$ and $f \in L^2(\widehat{\pi})\setminus K$.	It holds that
	\begin{equation}
	\Vert P_t f \Vert_{L^2(\widehat{\pi})} \le \Vert P_t f \Vert_{H^1(\widehat{\pi})} = \Vert P_{t-1}P_{1} f \Vert_{H^1(\widehat{\pi})}
	\le C e^{-\lambda (t-1)} \Vert P_{1} f \Vert_{H^1(\widehat{\pi})}, 
	\end{equation}
	where the last inequality follows from \eqref{eq:H1_decay}. Now applying \eqref{eq:H1L2_reg} and gathering constants results in 
	\begin{equation}
	\Vert P_t f\Vert_{L^2(\widehat{\pi})} \le C e^{-\lambda t}\Vert f \Vert_{L^2(\widehat{\pi})}, \quad f \in L^2(\widehat{\pi})\setminus K.
	\end{equation} 
	Note that although we assumed $t\ge1$, the above estimate also holds for $t\ge0$ (although possibly with a different constant $C$) since $\Vert P_t \Vert_{L^2(\widehat{\pi})\rightarrow L^2(\widehat{\pi})}$ is bounded on $[0,1]$. 
	\qed  
\end{proof}


\section{Asymptotic Variance of Linear and Quadratic Observables in the Gaussian Case}
\label{app:Gaussian_proofs}

We begin by deriving a formula for the asymptotic variance of observables
of the form 
\[
f(q)=q\cdot Kq+l\cdot q-\Tr K,
\]
with $K\in\mathbb{R}_{sym}^{d\times d}$ and $l\in\mathbb{R}^{d}$.
Note that the constant term is chosen such that $\widehat{\pi}(f)=0$.
The following calculations are very much along the lines of \cite[Section  4]{duncan2016variance}. Since the Hessian of $V$ is bounded and the target measure $\pi$ is Gaussian, Assumption \ref{ass:bounded+Poincare} is satisfied and exponential decay of the semigroup $(P_t)_{t\ge0}$ as in \eqref{eq:hypocoercive estimate} follows by  Theorem \ref{theorem:Hypocoercivity}. According to Lemma \ref{lemma:variance}, the
asymptotic variance is then given by 
\begin{equation}
\sigma_{f}^{2}=\langle \chi,f\rangle_{L^{2}(\widehat{\pi})},
\end{equation}
where $\chi$ is the solution to the Poisson equation 
\begin{equation}
\label{eq:Poisson equation Gauss}
-\mathcal{L}\chi=f,\quad\widehat{\pi}(\chi)=0.
\end{equation}
Recall that 
\[
\mathcal{L}=-Bx\cdot\nabla+\nabla^{T}Q\nabla=-x\cdot A\nabla+\nabla^{T}Q\nabla
\]
is the generator as in (\ref{eq:OU generator}), where for later convenience
we have defined $A=B^{T}$, i.e.
\begin{equation}
A=\left(\begin{array}{cc}
-\mu J & I\\
-I & \gamma I -\nu J
\end{array}\right) \in \mathbb{R}^{2d \times 2d}.\label{eq:Amatrix}
\end{equation}
In the sequel we will solve (\ref{eq:Poisson equation Gauss}) analytically.  First, we introduce the notation 
\[
\bar{K}=\left(\begin{array}{cc}
K & \boldsymbol{0}\\
\boldsymbol{0} & \boldsymbol{0}
\end{array}\right)\in\mathbb{R}^{2d\times2d}
\]
and 
\[
\bar{l}=\left(\begin{array}{c}
l\\
\boldsymbol{0}
\end{array}\right)\in\mathbb{R}^{2d},
\]
such that by slight abuse of notation $f$ is given by 
\[
f(x)=x\cdot\bar{K}x+\bar{l}\cdot x-\Tr\bar{K}.
\]
By uniqueness (up to a constant) of the solution to the Poisson equation \eqref{eq:Poisson equation Gauss} and
linearity of $\mathcal{L}$, $g$ has to be a quadratic polynomial,
so we can write 
\[
g(x)=x\cdot Cx+D\cdot x-\Tr C,
\]
where $C\in\mathbb{R}_{sym}^{2d\times2d}$ and $D\in\mathbb{R}^{2d}$ (notice that $C$ can be chosen to be symmetrical since $x \cdot C x$ does not depend on the antisymmetric part of $C$).
Plugging this ansatz into (\ref{eq:Poisson equation Gauss}) yields
\[
-\mathcal{L}g(x)=x\cdot A\big(2Cx+D\big)-\gamma\Tr_{p}C=x\cdot\bar{K}x+\bar{l}\cdot x-\Tr\bar{K},
\]
where 
\[
\Tr_{p}C=\sum_{i=n+1}^{2n}C_{ii}
\]
denotes the trace of the momentum component of $C$. Comparing different
powers of $x$, this leads to the conditions
\begin{subequations}
\begin{eqnarray}
AC+CA^{T} & =\bar{K},
\label{eq:Lyapunov equation}\\
AD & =\bar{l},
\label{eq:linear condition}\\
\gamma\Tr_{p}C & =\Tr\bar{K}.
\label{eq:trace condition}
\end{eqnarray}
\end{subequations}
Note that (\ref{eq:trace condition}) will be satisfied eventually
by existence and uniqueness of the solution to (\ref{eq:Poisson equation Gauss}).
Then, by the calculations in \cite{duncan2016variance}, the asymptotic variance is given by 
\begin{equation}
\sigma_{f}^{2}=2\Tr(C\bar{K})+D\cdot\bar{l}.
\label{eq:Gaussian asymvar}
\end{equation}

\begin{proof}[of Proposition \ref{thm: local quadratic observable}]. According to
	(\ref{eq:Gaussian asymvar}) and (\ref{eq:Lyapunov equation}), the
	asymptotic variance satisfies 
	\[
	\sigma_{f}^{2}=2\Tr(C\bar{K}),
	\]
	where the matrix $C$ solves 
	\begin{equation}
	AC+CA^{T}=\bar{K}\label{eq: lyap equation}
	\end{equation}
	and $A$ is given as in (\ref{eq:Amatrix}). We will use the notation
	\[
	C(\mu,\nu)=\left(\begin{array}{cc}
	C_{1}(\mu,\nu) & C_{2}(\mu.\nu)\\
	C_{2}^{T}(\mu.\nu) & C_{3}(\mu,\nu)
	\end{array}\right)
	\]
	and the abbreviations $C(0):=C(0,0)$, $C^{\mu}(0):=\partial_{\mu}C\vert_{\mu,\nu=0}$
	and $C^{\nu}(0):=\partial_{\nu}C\vert_{\mu,\nu=0}$.  Let us first determine $C(0)$, i.e. the solution to the equation
	\[
	\left(\begin{array}{cc}
	\boldsymbol{0} & I\\
	-I & \gamma I
	\end{array}\right)C(0)+C(0)\left(\begin{array}{cc}
	\boldsymbol{0} & I\\
	-I & \gamma I
	\end{array}\right)^{T}=\left(\begin{array}{cc}
	K & \boldsymbol{0}\\
	\boldsymbol{0} & \boldsymbol{0}
	\end{array}\right).
	\]
	This leads to the following system of equations,
	\begin{subequations}
	\begin{eqnarray}
	C_{2}(0)+C_{2}(0)^{T} & =K,
	\label{eq:C(0) 1}\\
	-C_{1}(0)+\gamma C_{2}(0)+C_{3}(0) & =\boldsymbol{0},
	\label{eq:C(0) 2}\\
	-C_{1}(0)+\gamma C_{2}(0)^{T}+C_{3}(0) & =\boldsymbol{0},
	\label{eq:C(0) 3}\\
	-C_{2}(0)-C_{2}(0)^{T}+2\gamma C_{3}(0) & =\boldsymbol{0}.\\ \label{eq:C(0) 4}
	\end{eqnarray}
	\end{subequations}
	Note that equations (\ref{eq:C(0) 2}) and (\ref{eq:C(0) 3}) are
	equivalent by taking the transpose. Plugging (\ref{eq:C(0) 1}) into
	(\ref{eq:C(0) 4}) yields 
	\begin{equation}
	C_{3}(0)=\frac{1}{2\gamma}K.\label{eq:C3(0) result}
	\end{equation}
	Adding (\ref{eq:C(0) 2}) and (\ref{eq:C(0) 3}), together with (\ref{eq:C(0) 1})
	and (\ref{eq:C3(0) result}) leads to 
	\[
	C_{1}(0)=\frac{1}{2\gamma}K+\frac{\gamma}{2}K.
	\]
	Solving (\ref{eq:C(0) 2}) we obtain, 
	\[
	C_{2}(0)=\frac{1}{2}K,
	\]
	so that
	\begin{equation}
	C(0)=\left(\begin{array}{cc}
	\frac{1}{2\gamma}K+\frac{\gamma}{2}K & \frac{1}{2}K\\
	\frac{1}{2}K & \frac{1}{2\gamma}K
	\end{array}\right).\label{eq:C(0) result}
	\end{equation}
	Taking the $\mu$-derivative of (\ref{eq: lyap equation}) and setting
	$\mu=\nu=0$ yields 
	\begin{equation}
	A^{\mu}(0)C(0)+A(0)C^{\mu}(0)+C^{\mu}(0)A(0)^{T}+C(0)A^{\mu}(0)^{T}=\boldsymbol{0}.\label{eq:mu lyap}
	\end{equation}
	Notice that 
	\begin{align*}
	A^{\mu}(0)C(0)+C(0)A^{\mu}(0)^{T}\\
	=\left(\begin{array}{cc}
	-J & \boldsymbol{0}\\
	\boldsymbol{0} & \boldsymbol{0}
	\end{array}\right)C(0)+C(0)\left(\begin{array}{cc}
	J & \boldsymbol{0}\\
	\boldsymbol{0} & \boldsymbol{0}
	\end{array}\right)\\
	=\left(\begin{array}{cc}
	\big(\frac{1}{2\gamma}+\frac{\gamma}{2}\big)[K,J] & -\frac{1}{2}JK\\
	\frac{1}{2}KJ & \boldsymbol{0}
	\end{array}\right).
	\end{align*}
	With computations similar to those in the derivation of (\ref{eq:C(0) result})
	(or by simple substitution), equation (\ref{eq:mu lyap}) can be solved
	by 
	\begin{equation}
	C^{\mu}(0)=\left(\begin{array}{cc}
	-\big(\frac{\gamma^{2}}{4}+\frac{1}{4\gamma^{2}}+\frac{1}{4}\big)[K,J] & \frac{1}{2\gamma}JK-\frac{\gamma}{4}[K,J]\\
	-\frac{1}{2\gamma}KJ-\frac{\gamma}{4}[K,J] & -\big(\frac{1}{4\gamma^{2}}+\frac{1}{4}\big)[K,J]
	\end{array}\right).\label{eq:C^mu}
	\end{equation}
	We employ a similar strategy to determine $C^{\nu}(0)$: Taking the
	$\nu$-derivative in equation (\ref{eq: lyap equation}), setting
	$\mu=\nu=0$ and inserting $C(0)$ and $A(0)$ as in (\ref{eq:C(0) result})
	and (\ref{eq:Amatrix}) leads to the equation
	\[
	\left(\begin{array}{cc}
	\boldsymbol{0} & I\\
	-I & \gamma I
	\end{array}\right)C^{\nu}(0)+C^{\nu}(0)\left(\begin{array}{cc}
	\boldsymbol{0} & I\\
	-I & \gamma I
	\end{array}\right)=\left(\begin{array}{cc}
	\boldsymbol{0} & -\frac{1}{2}KJ\\
	\frac{1}{2}JK & -\frac{1}{2\gamma}[K,J]
	\end{array}\right),
	\]
	which can be solved by 
	\begin{equation}
	C^{\nu}(0)=\left(\begin{array}{cc}
	\big(-\frac{1}{4\gamma^{2}}+\frac{1}{4}\big)[K,J] & \frac{1}{\gamma}\big(-\frac{1}{2}KJ+\frac{1}{4}[K,J]\big)\\
	\frac{1}{\gamma}\big(\frac{1}{2}KJ-\frac{1}{4}[K,J]\big) & -\frac{1}{4\gamma^{2}}[K,J]
	\end{array}\right).\label{eq:C^nu}
	\end{equation}
	Note that $\Tr (C\bar{K})=\Tr(C_{1}K)$, and so 
	\begin{alignat*}{1}
	\partial_{\mu}\Theta\vert_{\mu,\nu=0} & =2\Tr(C_{1}^{\mu}(0)K)=\\
	& =-\big(\frac{\gamma^{2}}{4}+\frac{1}{4\gamma^{2}}+\frac{1}{4}\big)\cdot\Tr([K,J]K)=0,
	\end{alignat*}
	since clearly $\Tr([K,J],K)=\Tr(KJK)-\Tr(JK^{2})=0$. In the same
	way it follows that 
	\[
	\partial_{\nu}\Theta\vert_{\mu,\nu=0}=0,
	\]
	proving (\ref{eq:gradTheta}). 
	\\\\
	Taking the second $\mu$-derivative of (\ref{eq: lyap equation})
	and setting $\mu=\nu=0$ yields
	\[
	2A^{\mu}(0)C^{\mu}(0)+A(0)C^{\mu\mu}(0)+C^{\mu\mu}(0)A(0)^{T}+2C^{\mu}(0)A^{\mu}(0)^{T}=\boldsymbol{0},
	\]
	employing the notation $C^{\mu\mu}(0)=\partial_{\mu}^{2}C\vert_{\mu,\nu=0}$
	and noticing that $\partial_{\mu}^{2}A=0$. Using (\ref{eq:C^mu})
	we calculate
	\[
	A^{\mu}(0)C^{\mu}(0)+C^{\mu}(0)A^{\mu}(0)^{T}=\left(\begin{array}{cc}
	\big(\frac{\gamma^{2}}{4}+\frac{1}{4\gamma^{2}}+\frac{1}{4}\big)[J,[K,J]] & -\frac{1}{2\gamma}J^{2}K+\frac{\gamma}{4}J[K,J]\\
	-\frac{1}{2\gamma}KJ^{2}-\frac{\gamma}{4}[K,J]J & \boldsymbol{0}
	\end{array}\right).
	\]
	As before, we make the ansatz
	\[
	C^{\mu\mu}(0)=\left(\begin{array}{cc}
	C_{1}^{\mu\mu}(0) & C_{2}^{\mu\mu}(0)\\
	\left(C_{2}^{\mu\mu}(0)\right)^T & C_{3}^{\mu\mu}(0)
	\end{array}\right),
	\]
	leading to the equations
	\begin{subequations}
	\begin{eqnarray}
	C_{2}^{\mu\mu}(0)+C_{2}^{\mu\mu}(0)^{T} & = &-\big(\frac{\gamma^{2}}{4}+\frac{1}{4\gamma^{2}}+\frac{1}{4}\big)[J,[K,J]]\label{eq:C''1}\\
	-C_{1}^{\mu\mu}(0)+\gamma C_{2}^{\mu\mu}(0)+C_{3}^{\mu}(0) & =&\frac{1}{\gamma}J^{2}K-\frac{\gamma}{2}J[K,J]\label{eq:C''2}\\
	-C_{1}^{\mu\mu}(0)+\gamma C_{2}^{\mu\mu}(0)^{T}+C_{3}^{\mu}(0) & =&\frac{1}{\gamma}KJ^{2}+\frac{\gamma}{2}[K,J]J\label{eq:C''3}\\
	-C_{2}^{\mu\mu}(0)-C_{2}^{\mu\mu}(0)^{T}+2\gamma C_{3}^{\mu\mu}(0) & =&\boldsymbol{0}.
	\label{eq:C''4}
	\end{eqnarray}
	\end{subequations}
	Again, (\ref{eq:C''2}) and (\ref{eq:C''3}) are equivalent by taking
	the transpose. Plugging (\ref{eq:C''1}) into (\ref{eq:C''4}) and
	combing with (\ref{eq:C''2}) or (\ref{eq:C''3}) gives
	\[
	C_{1}^{\mu\mu}(0)=\big(\frac{\gamma}{4}+\frac{1}{4\gamma^{3}}+\frac{\gamma^{3}}{4}\big)(2JKJ-J^{2}K-KJ^{2})-\frac{1}{\gamma}JKJ.
	\]
	Now 
	\[
	\partial_{\mu}^{2}\Theta\vert_{\mu,\nu=0}=2\Tr(C_{1}^{\mu\mu}(0)K)=-(\gamma+\frac{1}{\gamma^{3}}+\gamma^{3})\big(\Tr(JKJK)-\Tr(J^{2}K^{2})\big)-\frac{2}{\gamma}\Tr(JKJK)
	\]
	gives the first part of (\ref{eq:HessTheta}). We proceed in the same
	way to determine $C_{1}^{\nu\nu}(0)$. Analogously, we get 
	\[
	A^{\nu}(0)C^{\nu}(0)+C^{\nu}(0)A^{\nu}(0)^{T}=\left(\begin{array}{cc}
	\boldsymbol{0} & \frac{1}{\gamma}(KJ^{2}-\frac{1}{2}[K,J]J)\\
	\frac{1}{\gamma}(JKJ-\frac{1}{2}J[K,J]) & \frac{1}{2\gamma^{2}}([K,J]J-J[K,J])
	\end{array}\right).
	\]
	Solving the resulting linear matrix system (similar to (\ref{eq:C''1})-(\ref{eq:C''4}))
	results in
	\[
	C_{1}^{\nu\nu}(0)=\big(\frac{1}{4\gamma^{3}}-\frac{1}{4\gamma}\big)(KJ^{2}+J^{2}K)-\big(\frac{1}{2\gamma^{3}}+\frac{1}{2\gamma}\big)JKJ,
	\]
	leading to 
	\[
	\partial_{\nu}^{2}\Theta\vert_{\mu,\nu=0}=2\Tr(C_{1}^{\nu\nu}(0)K)=\big(\frac{1}{\gamma^{3}}-\frac{1}{2\gamma}\big)\Tr(J^{2}K^{2})\big)-\big(\frac{1}{2\gamma^{3}}+\frac{1}{2\gamma}\big)\Tr(JKJK).
	\]
	To compute the cross term $C_{1}^{\mu\nu}(0)$ we take the mixed derivative
	$\partial_{\mu\nu}^{2}$ of (\ref{eq: lyap equation}) and set $\mu=\nu=0$
	to arrive at 
	\[
	A^{\mu}(0)C^{\nu}(0)+A^{\nu}(0)C^{\mu}(0)+A(0)C^{\mu\nu}(0)+C^{\mu\nu}(0)A(0)^{T}+C^{\mu}(0)A^{\nu}(0)^{T}+C^{\nu}(0)A^{\mu}(0)^{T}=\boldsymbol{0}.
	\]
	Using $\eqref{eq:C^mu}$ and (\ref{eq:C^nu}) we see that 
	\begin{multline*}
	A^{\mu}(0)C^{\nu}(0)+A^{\nu}(0)C^{\mu}(0)+C^{\mu}(0)A^{\nu}(0)^{T}+C^{\nu}(0)A^{\mu}(0)^{T}\\
	=\left(\begin{array}{cc}
	\big(\frac{1}{4\gamma^{2}}-\frac{1}{4}\big)[J,[K,J]] & \frac{1}{\gamma}JKJ-\frac{1}{4\gamma}J[K,J]-\frac{\gamma}{4}[K,J]J\\
	\frac{1}{2\gamma}JKJ+\frac{1}{2\gamma}KJ^{2}+\frac{\gamma}{4}J[K,J]-\frac{1}{4\gamma}[K,J]J & \big(\frac{1}{4\gamma^{2}}+\frac{1}{4}\big)[J,[K,J]]
	\end{array}\right).
	\end{multline*}
	The ensuing linear matrix system yields the solution
	\[
	C_{1}^{\mu\nu}(0)=\big(-\frac{1}{4\gamma^{3}}+\frac{\gamma}{4}-\frac{1}{4\gamma}\big)[J,[K,J]]+\frac{1}{\gamma}JKJ,
	\]
	leading to 
	\begin{equation}
	\partial_{\mu\nu}^{2}\Theta\vert_{\mu,\nu=0}=2\Tr(C_{1}^{\mu\nu}(0)K)=\big(\frac{1}{\gamma^{3}}+\frac{1}{\gamma}-\gamma\big)\Tr(J^{2}K^{2})+\big(-\frac{1}{\gamma^{3}}+\frac{1}{\gamma}+\gamma\big)\Tr(JKJK).
	\end{equation}
	This completes the proof.
	\qed 
    \end{proof}
\begin{proof}
	[Proof of Proposition \ref{thm:linear_full_J}] By (\ref{eq:linear condition})
	and (\ref{eq:Gaussian asymvar}) the function $\Theta$ satisfies
	\[
	\Theta(\mu,\nu)=\bar{l}\cdot A^{-1}\bar{l}.
	\]
	Recall the following formula for blockwise inversion of matrices using the Schur complement:
	\begin{equation}
	\left(\begin{array}{cc}
	U & V\\
	W & X
	\end{array}\right)^{-1}=\left(\begin{array}{cc}
	(U-VX^{-1}W)^{-1} & \ldots\\
	\ldots & \ldots
	\end{array}\right),\label{eq: blockwise inversion}
	\end{equation}
	provided that $X$ and $U-VX^{-1}W$ are invertible. Using this, we obtain 
	\[
	\Theta(\mu,\nu)=l\cdot\big(-\mu J+(\gamma-\nu J)^{-1}\big)l.
	\]
	Taking derivatives, setting $\mu=\nu=0$ and using the fact that $J^{T}=-J$
	leads to the desired result.
	\qed
\end{proof}

\begin{lemma}
	\label{lem:basic_inequalities}
	The following holds: 
	\begin{enumerate}[label=(\alph*)]
		\item \label{it:gaussian_lem1}$\gamma-\frac{4}{\gamma^{3}}-\gamma^{3}-\frac{1}{\gamma}<0$
		for $\gamma\in(0,\infty)$. 
		\item \label{it:gaussian_lem2} Let $J=-J^{T}$ and $K=K^{T}$. Then $\Tr(JKJK)-\Tr(J^{2}K^{2})\ge0$.
		Furthermore, equality holds if and only if $[J,K]=0.$ 
	\end{enumerate}
\end{lemma}
\begin{proof}
	To show \ref{it:gaussian_lem1} we note that $\gamma-\frac{4}{\gamma^{3}}-\gamma^{3}-\frac{1}{\gamma}<\gamma-\frac{4}{\gamma^{3}}-\gamma^{3}=\gamma(1 - \frac{4}{\gamma^4}-\gamma^2)$.  The function $f(\gamma):=1 - \frac{4}{\gamma^4}-\gamma^2$
	has a unique global maximum on $(0,\infty)$ at $\gamma_{min}=8^{1/6}$
	with $f(\gamma_{min})=-2$, so the result follows.
	\\\\
	For \ref{it:gaussian_lem2} we note that $[J,K]^{T}=[J,K],$ and that $[J,K]^{2}$ is symmetric
	and nonnegative definite. We can write 
	\[
	\Tr([J,K]^{2})=\sum_{i}\lambda_{i}^{2},
	\]
	with $\lambda_{i}$ denoting the (real) eigenvalues of $[J,K]$. From
	this it follows that $\Tr([J,K]^{2})\ge0$ with equality if and only
	if $[J,K]=0$. Now expand 
	\[
	\Tr([J,K]^{2})=2\big(\Tr(JKJK)-\Tr(J^{2}K^{2}),
	\]
	which implies the advertised claim. \qed
\end{proof}


\section{Orthogonal Transformation of Tracefree Symmetric Matrices into a Matrix with Zeros on the Diagonal}
\label{tracefree}

Given a symmetric matrix $K\in\mathbb{R}_{sym}^{d\times d}$ with
$\Tr K=0$, we seek to find an orthogonal matrix $U\in O(\mathbb{R}^{d})$
such that $UKU^{T}$ has zeros on the diagonal. This is a crucial
step in Algorithms \ref{alg:optimal J} and \ref{alg:optimal J general}
and has been addressed in various places in the literature (see for
instance \cite{alg_zero_diag} or \cite{Bhatia1997}, Chapter 2,
Section 2). For the convenience of the reader, in the following we
summarize an algorithm very similar to the one in \cite{alg_zero_diag}.
\\\\
Since $K$ is symmetric, there exists an orthogonal matrix $U_{0}\in O(\mathbb{R}^{d})$
such that $U_{0}KU_{0}^{T}=\diag(\lambda_{1},\ldots,\lambda_{d})$.
Now the algorithm proceeds iteratively, orthogonally transforming
this matrix into one with the first diagonal entry vanishing, then
the first two diagonal entries vanishing, etc, until after $d$ steps
we are left with a matrix with zeros on the diagonal. Starting with
$\lambda_{1}$, assume that $\lambda_{1}\neq0$ (otherwise proceed
with $\lambda_{2}$). Since $\Tr(K)=\Tr(U_{0}KU_{0}^{T})=\sum\lambda_{i}=0$,
there exists $\lambda_{j}$, $j\in\{2,\ldots,d\}$ such that $\lambda_{1}\lambda_{j}<0$
(i.e. $\lambda_{1}$ and $\lambda_{j}$ have opposing signs). We now
apply a rotation in the $1j$-plane to transform the first diagonal
entry into zero. More specifically, let 
\[
U_{1}=\begin{blockarray}{ccccccccc}
 ~ & ~ & ~ & ~ & j & ~ & ~ & ~ & ~\\
\begin{block}{(cccccccc)c}
\cos\alpha & 0 & \ldots & 0 & -\sin\alpha & 0 & \ldots & 0 & ~\\
0 & 1 & ~ & ~ & 0 & ~ & ~ & \vdots & ~\\
\vdots & ~ & \ddots & ~ & ~ & ~ & ~ & ~ & ~\\
0 & ~ & ~ & 1 & 0 & ~ & ~ & ~ & ~\\
\sin\alpha & 0 & ~ & 0 & \cos\alpha & 0 & ~ & ~ & j\\
0 & ~ & ~ & ~ & 0 & 1 & ~ & \vdots & ~\\
\vdots & ~ & ~ & ~ & ~ & ~ & \ddots & 0 & ~\\
0 & 0 & \hdots & ~ & ~ & \hdots & 0 & 1 & ~ \\
\end{block}\end{blockarray} \in O(\mathbb{R}^{d})
\]
with $\alpha=\arctan\sqrt{-\frac{\lambda_{1}}{\lambda_{j}}}.$ We
then have $(U_{1}U_{0}KU_{0}^{T}U_{1}^{T})_{11}=0$. Now the same
procedure can be applied to the second diagonal entry $\text{\ensuremath{\lambda}}_{2}$,
leading to the matrix $U_{2}U_{1}U_{0}KU_{0}^{T}U_{1}^{T}U_{2}^{T}$
with 
\[
(U_{2}U_{1}U_{0}KU_{0}^{T}U_{1}^{T}U_{2}^{T})_{11}=(U_{2}U_{1}U_{0}KU_{0}^{T}U_{1}^{T}U_{2}^{T})_{22}=0
\]
Iterating this process, we obtain that $U_{d}\ldots U_{1}U_{0}KU_{0}^{T}U_{1}^{T}\ldots U_{d}^{T}$
has zeros on the diagonal, so $U_{d}\ldots U_{1}U_{0}\in O(\mathbb{R}^{d})$
is the required orthogonal transformation. 


\bibliographystyle{alpha}
%%%%%%%%%%%%%%%%%%%% author.tex %%%%%%%%%%%%%%%%%%%%%%%%%%%%%%%%%%%
%
% sample root file for your "contribution" to a contributed volume
%
% Use this file as a template for your own input.
%
%%%%%%%%%%%%%%%% Springer %%%%%%%%%%%%%%%%%%%%%%%%%%%%%%%%%%


% RECOMMENDED %%%%%%%%%%%%%%%%%%%%%%%%%%%%%%%%%%%%%%%%%%%%%%%%%%%
\documentclass[graybox]{svmult}

% choose options for [] as required from the list
% in the Reference Guide

\usepackage{mathptmx}       % selects Times Roman as basic font
\usepackage{helvet}         % selects Helvetica as sans-serif font
\usepackage{courier}        % selects Courier as typewriter font
\usepackage{type1cm}        % activate if the above 3 fonts are
% not available on your system
%
\usepackage{makeidx}         % allows index generation
\usepackage{graphicx}        % standard LaTeX graphics tool
% when including figure files
\usepackage{multicol}        % used for the two-column index
\usepackage[bottom]{footmisc}% places footnotes at page bottom

% OLD PREAMBLE:

% \usepackage{jsen}
% \usepackage{cite}
% \usepackage{amsmath,amssymb,amsfonts, bbm, mathtools}
% \usepackage{algorithm,algorithmic}
% \usepackage{graphicx}
% \usepackage{textcomp}
% \usepackage{wrapfig}
% \usepackage{xfrac}
% \usepackage{stackengine}
% \usepackage{subfigure}
% \def\delequal{\mathrel{\ensurestackMath{\stackon[1pt]{=}{\scriptstyle\Delta}}}}



% \usepackage{color, soul}
% \newcommand{\hlt}[1]{\hl{#1}}
% \newcommand{\red}[1]{\textcolor{red}{#1}}

% \def\BibTeX{{\rm B\kern-.05em{\sc i\kern-.025em b}\kern-.08em
%     T\kern-.1667em\lower.7ex\hbox{E}\kern-.125emX}}
% \markboth{\journalname, VOL. XX, NO. XX, XXXX 2017}
% {Author \MakeLowercase{\textit{et al.}}: Preparation of Papers for IEEE TRANSACTIONS and JOURNALS (February 2017)}
% \definecolor{abstractbg}{rgb}{0.89804,0.94510,0.83137}
% \setlength{\fboxrule}{0pt}
% \setlength{\fboxsep}{0pt}

% NEW PREAMBLE:


\usepackage{amsmath,amsfonts,amssymb,bbm, amsthm, xfrac}
\usepackage{algorithmic}
\usepackage{algorithm}
\usepackage{array, multirow}
% \usepackage[caption=false,font=normalsize,labelfont=sf,textfont=sf]{subfig}
\usepackage{caption, subcaption}
\usepackage{textcomp}
\usepackage{stfloats}
\usepackage{url}
\usepackage{verbatim}
\usepackage{graphicx}
\usepackage{cite}
\usepackage{caption}
\usepackage{subcaption}
\hyphenation{}

\theoremstyle{plain}
\newtheorem{theorem}{Theorem}

\usepackage{color, soul}
\newcommand{\hlt}[1]{\hl{#1}}
\newcommand{\red}[1]{\textcolor{red}{#1}}


% see the list of further useful packages
% in the Reference Guide

\makeindex             % used for the subject index
% please use the style svind.ist with
% your makeindex program

%%%%%%%%%%%%%%%%%%%%%%%%%%%%%%%%%%%%%%%%%%%%%%%%%%%%%%%%%%%%%%%%%%%%%%%%%%%%%%%%

\begin{document}

\title*{A MultiMesh Finite Element Method for the Stokes Problem}

  %% High order methods on arbitrarily many intersecting meshes:
  %% Multimesh}
%\titlerunning{Multimesh}
% Use \titlerunning{Short Title} for an abbreviated version of
% your contribution title if the original one is too long
%\author{Name of First Author and Name of Second Author}
\author{August Johansson \and
  %Benjamin Kehlet \and
  Mats G. Larson \and
  Anders Logg}
% Use \authorrunning{Short Title} for an abbreviated version of
% your contribution title if the original one is too long
%\institute{Name of First Author \at Name, Address of Institute, \email{name@email.address}
%\and Name of Second Author \at Name, Address of Institute \email{name@email.address}}
\institute{August Johansson \at Simula Research Laboratory, P.O.\ Box 134, 1325 Lysaker, Norway, \email{august.johansson@gmail.com}
 % \and
 % Benjamin Kehlet \at  Simula Research Laboratory, P.O.\ Box 134, 1325 Lysaker, Norway, \email{benjamik@simula.no}
  \and
  Mats G. Larson \at Department of Mathematics, Ume{\aa} University, 90187, Ume{\aa}, Sweden, \email{mats.larson@math.umu.se}
  \and
  Anders Logg \at Department of Mathematical Sciences, Chalmers University of Technology and University of Gothenburg, 41296 G\"oteborg, Sweden, \email{logg@chalmers.se}.}

\maketitle

%%%%%%%%%%%%%%%%%%%%%%%%%%%%%%%%%%%%%%%%%%%%%%%%%%%%%%%%%%%%%%%%%%%%%%%%%%%%%%%%
\abstract{The multimesh finite element method enables the solution of partial differential equations on a computational mesh composed by multiple arbitrarily overlapping meshes. The discretization is based on a continuous--discontinuous function space with interface conditions enforced by means of Nitsche's method.
In this contribution, we consider the Stokes problem as a first step towards flow applications. The multimesh formulation leads to so called cut elements in the underlying meshes close to overlaps. These demand stabilization to ensure coercivity and stability of the stiffness matrix. We employ a consistent least-squares term on the overlap to ensure that the inf-sup condition holds. We here present the method for the Stokes problem, discuss the implementation, and verify that we have optimal convergence.
%Each chapter should be preceded by an abstract (10--15 lines long) that summarizes the content. The abstract will appear \textit{online} at \url{www.SpringerLink.com} and be available with unrestricted access. This allows unregistered users to read the abstract as a teaser for the complete chapter. As a general rule the abstracts will not appear in the printed version of your book unless it is the style of your particular book or that of the series to which your book belongs.\newline\indent
%  Please use the 'starred' version of the new Springer \texttt{abstract} command for typesetting the text of the online abstracts (cf. source file of this chapter template \texttt{abstract}) and include them with the source files of your manuscript. Use the plain \texttt{abstract} command if the abstract is also to appear in the printed version of the book.
  }
%%%%%%%%%%%%%%%%%%%%%%%%%%%%%%%%%%%%%%%%%%%%%%%%%%%%%%%%%%%%%%%%%%%%%%%%%%%%%%%%
\section{Introduction}
\label{sec:1}

Consider the Stokes problem
\begin{alignat}{2}
  \label{eq:stokes1}
  -\Delta \bfu + \nabla p &= \bff \qquad &&\text{in $\Omega$},
  \\
  \label{eq:stokes2}
  \divv \bfu &= 0 \qquad &&\text{in $\Omega$},
  \\
  \bfu &= \bfzero \qquad &&\text{on $\partial \Omega$},
\end{alignat}
for the velocity $\bfu : \Omega \rightarrow \R^d$ and pressure $\bfp : \Omega \rightarrow \R$ in a polytopic domain $\Omega\subset\R^d$, $d = 2,3$.

The Stokes problem is considered here as a first step towards a multimesh formulation for multi-body flow problems, and ultimately fluid--structure interaction problems, in which each body is discretized by an individual boundary-fitted mesh and the boundary-fitted meshes move freely on top of a fixed background mesh. The applications for such a formulation are many, e.g., the simulation of blood platelets in a blood stream, the optimization of the configuration of an array of wind turbines, or the investigation of the effect of building locations in a simulation of urban wind conditions and pollution. Common to these applications is that the multimesh method removes the need for costly mesh (re)generation and allows the platelets, wind turbines or buildings to be moved around freely in the domain, either in each timestep as a part of a dynamic simulation, or in each iteration as part of an optimization problem.

The multimesh formulation presented here is a generalization of the formulation presented and analyzed in~\cite{Johansson:2015aa} for two domains. For comparison, the multimesh discretization of the Poisson problem for arbitrarily many intersecting meshes is presented in~\cite{mmfem-1} and analyzed in~\cite{mmfem-2}.

%%%%%%%%%%%%%%%%%%%%%%%%%%%%%%%%%%%%%%%%%%%%%%%%%%%%%%%%%%%%%%%%%%%%%%%%%%%%%%%%
\section{Notation}
\label{sec:notation}

We first review the notation for domains, interfaces, meshes and overlaps used to formulate the multimesh finite element method. For a more detailed exposition, we refer to~\cite{mmfem-1}.

\begin{mytcolorbox}
\emph{Notation for domains}
\tcblower

Let $\Omega = \hatOmega_0 \subset \R^d$, $d = 2,3$, be a domain with polytopic boundary (the background domain).

Let $\hatOmega_i \subset \hatOmega_0$, $i=1,\ldots, N$ be the so-called \emph{predomains} with polytopic boundaries (see Figure~\ref{fig:three_domains}).

Let $\Omega_i = \hatOmega_i \setminus \bigcup_{j=i+1}^{N} \hatOmega_j$,  $i=0, \ldots, N$ be a partition of $\Omega$ (see Figure~\ref{fig:three_domains_partition}).
\end{mytcolorbox}

\begin{figure}
  \centering
  \subfloat[]{\label{fig:three_domains_a}\includegraphics[height=0.2\linewidth]{figs/domains.pdf}}\qquad\qquad\qquad
  \subfloat[]{\label{fig:three_domains_b}\includegraphics[height=0.2\linewidth]{figs/ordering.pdf}}
  \caption{(a) Three polygonal predomains. (b) The predomains are placed on top of each other in an ordering such that
    $\hatOmega_0$ is placed lowest, $\hatOmega_1$ is in the middle and $\hatOmega_2$ is on top.}
  \label{fig:three_domains}
\end{figure}

\begin{remark}
  \label{rem:boundary-overlap}
To simplify the presentation, the domains $\Omega_1, \ldots, \Omega_N$ are not allowed to intersect the boundary of $\Omega$.
\end{remark}

\begin{figure}
  \begin{center}
    \includegraphics[height=0.2\linewidth]{figs/partition_of_Omega.pdf}
    \caption{Partition of $\Omega = \Omega_0 \cup \Omega_1 \cup \Omega_2$. Note that $\Omega_2 = \hatOmega_2$.}
    \label{fig:three_domains_partition}
  \end{center}
\end{figure}

\begin{mytcolorbox}
\emph{Notation for interfaces}
\tcblower

Let the \emph{interface} $\Gamma_i$ be defined by $\Gamma_i = \partial \hatOmega_i \setminus \bigcup_{j=i+1}^N \hatOmega_j$, $i=1, \ldots, N-1$ (see Figure~\ref{fig:two_interfaces_a}).

Let $\Gamma_{ij} = \Gamma_i \cap \Omega_j$, $i > j$ be a partition of $\Gamma_i$ (see Figure~\ref{fig:two_interfaces_b}).
\end{mytcolorbox}

\begin{figure}
  \centering
  \subfloat[]{\label{fig:two_interfaces_a}\includegraphics[width=0.2\linewidth]{figs/interfaces.pdf}}\qquad\qquad\qquad
  \subfloat[]{\label{fig:two_interfaces_b}\includegraphics[width=0.2\linewidth]{figs/partition_of_Gamma.pdf}}
  \caption{(a) The two interfaces of the domains in Figure~\ref{fig:three_domains}: $\Gamma_1 = \partial \hatOmega_1 \setminus \hatOmega_2$ (dashed line) and  $\Gamma_2 = \partial \hatOmega_2$ (filled line). Note that $\Gamma_1$ is not a closed curve. (b) Partition of $\Gamma_2 = \Gamma_{20} \cup \Gamma_{21}$.}
\end{figure}

\begin{mytcolorbox}
\emph{Notation for meshes}
\tcblower

Let $\hatmcK_{h,i}$ be a quasi-uniform \cite{BreSco08} \emph{premesh} on $\hatOmega_i$ with mesh parameter $h_i = \max_{K\in \hatmcK_{h,i}} \diam(K)$, $i=0,\ldots,N$ (see Figure~\ref{fig:three_meshes_a}).

Let $\mcK_{h,i} = \{ K \in \hatmcK_{h,i} : K \cap \Omega_i \neq \emptyset \}$, $i=0,\ldots,N$ be the \emph{active meshes} (see Figure~\ref{fig:three_meshes_b}).

The \emph{multimesh} is formed by the active meshes placed in the given ordering (see Figure~\ref{fig:multimesh}).

Let $\Omega_{h,i} = \bigcup_{K\in\mcK_{h,i}} K$, $i=0,\ldots,N$ be the \emph{active domains}.
\end{mytcolorbox}

\begin{mytcolorbox}
\emph{Notation for overlaps}
\tcblower

Let $\OO_i$ denote the \emph{overlap} defined by  $\OO_i = \Omega_{h,i} \setminus \Omega_i$, $i=0,\ldots,N-1$.

Let $\OO_{ij} = \OO_i \cap \Omega_j = \Omega_{h,i} \cap \Omega_j$, $i < j$ be a partition of $\OO_i$.

% For $i < j$, let
%   \begin{align}
%     \label{eq:indicatorfunction}
%     \delta_{ij} =
%     \begin{cases}
%       1, \quad \OO_{ij} \neq \emptyset,
%       \\
%       0, \quad \text{otherwise},
%     \end{cases}
%   \end{align}
% be a function indicating which overlaps are non-empty.
% For ease of notation, we further let $\delta_{ii} = 1$ for $i = 0, \ldots,N$.

% Let $N_{\OO} = \max(\max_i \sum_j \delta_{ij}, \max_j \sum_i \delta_{ij})$ be the maximum number of overlaps. Note that $N_\OO$ is bounded by $N$ but is usually much smaller.
\end{mytcolorbox}

\begin{figure}
  \centering
  \subfloat[]{\label{fig:three_meshes_a}\includegraphics[height=0.2\linewidth]{figs/meshes.pdf}}\qquad\qquad\qquad
  \subfloat[]{\label{fig:three_meshes_b}\includegraphics[height=0.2\linewidth]{figs/active_meshes.pdf}}
  \caption{(a) The three premeshes. (b) The corresponding active meshes (cf.\ Figure~\ref{fig:three_domains}).}
\end{figure}

\begin{figure}
  \centering
  \subfloat[]{\label{fig:overlap}\includegraphics[height=0.2\linewidth]{figs/overlap.pdf}}\qquad\qquad\qquad
  \subfloat[]{\label{fig:multimesh}\includegraphics[height=0.2\linewidth]{figs/multimesh.pdf}}
  \caption{(a) Given three ordered triangles $K_0$, $K_1$ and $K_2$, the overlaps are $\mcO_{01}$ in green, $\mcO_{02}$ in red and $\mcO_{12}$ in blue. (b) The multimesh of the domains in Figure~\ref{fig:three_domains_b} consists of the active meshes in Figure~\ref{fig:three_meshes_b}.}
\end{figure}

%%%%%%%%%%%%%%%%%%%%%%%%%%%%%%%%%%%%%%%%%%%%%%%%%%%%%%%%%%%%%%%%%%%%%%%%%%%%%%%%
\section{MultiMesh Finite Element Method}

To formulate the multimesh finite element for the Stokes problem~\eqref{eq:stokes1}--\eqref{eq:stokes2}, we assume for each (active) mesh $\mcK_{h,i}$ the existence of a pair of inf-sub stable spaces $\bfV_{h,i}\times Q_{h,i}$, $i=0,1, \ldots, N$, away from the interface. To be precise, we assume inf-sup stability in $\omega_{h,i} \subset \Omega_{h,i}$, where $\omega_{h,i}$ is close to $\Omega_i$ in the sense that $\Omega_{h,i} \setminus \omega_{h,i} \subset U_\delta(\Gamma_i)$, where
\begin{align}
  U_\delta(\Gamma_i) = \bigcup_{\bfx \in \Gamma_i} B_\delta(\bfx)
\end{align}
and $B_\delta(\bfx)$ is a ball of radius $\delta$ centered at $\bfx$. In other words, $U_\delta(\Gamma_i)$ is the tubular neighborhood of $\Gamma$ with thickness $\delta$. In the numerical examples, we let $\omega_{h,i}$ be the union of elements in $\mcK_{h,i}$ with empty intersection with $\Gamma_{ij}$, $j > i$.

The inf-sup condition may expressed on each submesh $\omega_{h,i}$ by
\begin{align}
  \label{assum:infsup}
  \| p_i - \lambda_{\omega_{h,i}}(p) \|_{\omega_{h,i}}
  \lesssim
  \sup_{\bfv \in \bfW_{h,i}} \frac{(\divv \bfv,p)_{\omega_{h,i}}}{\|D \bfv \|_{\omega_{h,i}}},
\end{align}
where $\lambda_{\omega_{h,i}}(p)$ is the average of $p$ over $\omega_{h,i}$ and $\bfW_{h,i}$ is the subspace of $\bfV_{h,i}$ defined by
\begin{align}
  \label{eq:bfW}
  \bfW_{h,i}
  &=
  \{\bfv \in \bfV_{h,i}: \text{$\bfv=\bfzero$ \text{on} $\overline{\Omega_{h,i} \setminus \omega_{h,i}}$}\}.
\end{align}

We now define the multimesh finite element space as the direct sum
\begin{align}
  \bfV_h \times Q_h = \bigoplus_{i=0}^N \bfV_{h,i}\times Q_{h,i},
\end{align}
where $\bfV_h$ and $Q_h$ consist of piecewise polynomial of degree $k$ and $l$, respectively. This means that an element $\bfv \in \bfV_h$ is a tuple $(\bfv_0, \ldots, \bfv_N)$, and the inclusion $\bfV_h \hookrightarrow L^2(\Omega)$ is defined by $\bfv(\bfx) = \bfv_i(\bfx)$ for $\bfx\in\Omega_i$. A similar interpretation is done for $q \in Q_h$.

We now consider the following asymmetric finite element method: Find $(\bfu_h,p_h) \in \bfV_h \times Q_h$ such that
$A_h((\bfu_h, p_h), (\bfv, q)) = l_h(\bfv)$ for all
$(\bfv,q) \in \bfV_h \times Q_h$,
where
\begin{align}
  \label{eq:Ah}
  A_h((\bfu, p), (\bfv, q))
  &=
  a_h(\bfu,\bfv) + s_h(\bfu, \bfv) +  b_h(\bfu,q)
  + b_h(\bfv,p) + d_h((\bfu,p),(\bfv,q)),
  \\
  \label{eq:ah}
  a_{h}(\bfu,\bfv) &= \sum_{i=0}^N (D \bfu_i, D \bfv_i)_{\Omega_i}
  \\ \nonumber
  &\qquad
  - \sum_{i=1}^N \sum_{j=0}^{i-1} \big( (\langle (D \bfu) \cdot \bfn_i \rangle,[ \bfv])_{\Gamma_{ij}}
  + ([ \bfu], \langle (D \bfv) \cdot \bfn_i \rangle)_{\Gamma_{ij}} \big)
  \\ \nonumber
  &\qquad
  +  \sum_{i=1}^N \sum_{j=0}^{i-1} \beta_0 h^{-1}([\bfu],[ \bfv])_{\Gamma_{ij}},
  \\
  s_{h}(\bfu,\bfv)&=
  \sum_{i=0}^{N-1} \sum_{j=i+1}^N  \beta_1 ([D\bfu_i], [D\bfv_i])_{\OO_{ij}},
  \\
  \label{eq:bh}
  b_h(\bfu,q) &= -  \sum_{i=0}^N (\divv \bfu_i,q_i)_{\Omega_i}
  +  \sum_{i=1}^N \sum_{j=0}^{i-1} ([\bfn_i \cdot \bfu],\langle q \rangle )_{\Gamma_{ij}},
  \\
  \label{eq:dh}
  d_h((\bfu,p),(\bfv,q))&=
  \sum_{i=0}^N \delta h^2(\Delta \bfu_i - \nabla p_i,\Delta \bfv_i + \nabla q_i)_{\Omega_{h,i}\setminus \omega_{h,i}},
  \\
  l_h(\bfv) &=  \sum_{i=0}^N (\bff,\bfv_i)_{\Omega_i}
  -  \sum_{i=0}^N \delta h^2(\bff,\Delta \bfv_i + \nabla q_i)_{\Omega_{h,i}\setminus \omega_{h,i}}.
\end{align}
Here, $\beta_0$ and $\beta_1$ are stabilization parameters that must be sufficiently large to ensure that the bilinear form $A_h$ is coercive; cf.~\cite{Johansson:2015aa} for an analysis of the two-domain case.

For simplicity, we use the global mesh size $h$ here and throughout the presentation. If the meshes are of substantially different sizes, it may be beneficial to introduce the individual mesh sizes $h_i$ in~\eqref{eq:dh} and the average $h_i^{-1} + h_j^{-1}$ in~\eqref{eq:ah}.

Note that since $\Gamma_i$ is partitioned into interfaces $\Gamma_{ij}$ relative to underlying meshes, the sums of the interface terms are over  $0 \leq j < i \leq N$. In contrast, the sums of the overlap terms are over $0 \leq i < j \leq N$ since the overlap $\OO_i$ is partitioned into overlaps $\OO_{ij}$ relative to overlapping meshes.

The jump terms on $\OO_{ij}$ and $\Gamma_{ij}$ are defined by $[\bfv] = \bfv_i - \bfv_j$, where $\bfv_i$ and $\bfv_j$ are the finite element solutions (components) on the active meshes $\mcK_{h,i}$ and $\mcK_{h,j}$. The average normal flux is defined on $\Gamma_{ij}$ by
\begin{equation}\label{eq:average}
  \langle \bfn_i\cdot \nabla \bfv \rangle = (\bfn_i \cdot \nabla \bfv_{i} + \bfn_{i} \cdot \nabla \bfv_{j})/2.
\end{equation}
Here, any convex combination is valid~\cite{Hansbo:2003aa}.

The proposed formulation~\eqref{eq:Ah} is identical to the one proposed in \cite{Johansson:2015aa} with sums over all domains and interfaces. Also note the similarity with the multimesh formulation for the Poisson problem presented in~\cite{mmfem-1}, the difference being the additional least-squares term $d_h$ (and the corresponding term in $l_h$) since we only assume inf-sup stability in $\omega_{h,i}$. If we do not assume inf-sup stability anywhere (e.g.\ if we would use a velocity-pressure element of equal order), the least-squares term should be applied over the whole domain as in \cite{Massing:2014aa}. Please cf.\ \cite{Massing:2014aa} for the use of a symmetric $d_h$.

Other stabilization terms may be considered. By norm equivalence, the stabilization term $s_h(\bfu, \bfv)$ may alternatively be formulated as
\begin{equation}\label{eq:sh-L2}
  s_h (\bfu,\bfv) = \sum_{i=0}^{N-1} \sum_{j=i+1}^N \beta_2 h^{-2} ([ \bfu ], [\bfv ])_{\OO_{ij}}.
\end{equation}
where $\beta_2$ is a stabilization parameter; see~\cite{mmfem-2}.

Note that the finite element method weakly approximates continuity in the sense that $[\bfu_h] = \bfzero$ and $[\bfn_i \cdot \nabla \bfu_h] = 0$ on all interfaces.

%% Stabilization of the pressure may also be desired since the method only guarantees $p_h \in L^2(\Omega)$, and thus one may consider using
%% \begin{align}
%%   \sum_{i=0}^{N-1} \sum_{j=i+1}^N \beta_3 h^{-2} ([p],[q])_{\OO_{ij}},
%% \end{align}
%% where $\beta_3$ is a stabilization parameter.

%%%%%%%%%%%%%%%%%%%%%%%%%%%%%%%%%%%%%%%%%%%%%%%%%%%%%%%%%%%%%%%%%%%%%%%%%%%%%%%%
\section{Implementation}

We have implemented the multimesh finite element method as part of the software framework FEniCS~\cite{Logg:2012aa,Alnaes:2015aa}. One of the main features of FEniCS is the form language UFL~\cite{Alnaes:2014aa} which allows variational forms to be expressed in near-mathematical notation. However, to express the multimesh finite element method~\eqref{eq:Ah}, a number of custom measures must be introduced. In particular, new measures must be introduced for integrals over cut cells, interfaces and overlaps.
These measures are then mapped to quadrature rules that are computed at runtime. An overview of these algorithms algorithms and the implementation is given in~\cite{Johansson:2017ab}.

To express the multimesh finite element method, we let \texttt{dX} denote the integration over domains $\Omega_i$, $i=0, \ldots, N$, including cut cells. Integration over $\Gamma_{ij}$ and $\OO_{ij}$ are expressed using the measures \texttt{dI} and \texttt{dO}, respectively. We let \texttt{dC} denote integration over $\Omega_{h,i} \setminus \omega_{h,i}$. Now the multimesh finite element method for the Stokes problem may be expressed as
\begin{lstlisting}[language=Python,numbers=none]
  a_h = inner(grad(u), grad(v))*dX \
      - inner(avg(grad(u)), tensor_jump(v, n))*dI \
      - inner(avg(grad(v)), tensor_jump(u, n))*dI \
      + beta_0/h * inner(jump(u), jump(v))*dI
  s_h = beta_1 * inner(jump(grad(u)), jump(grad(v)))*dO
  b_h = lambda v, q: inner(-div(v), q)*dX \
                   + inner(jump(v, n), avg(q))*dI
  d_h = delta*h**2 * inner(-div(grad(u)) + grad(p), \
                           -div(grad(v)) - grad(q))*dC
\end{lstlisting}
This makes it easy to implement the somewhat lengthy form~\eqref{eq:Ah}, as well as investigate the effect of different stabilization terms.

%%%%%%%%%%%%%%%%%%%%%%%%%%%%%%%%%%%%%%%%%%%%%%%%%%%%%%%%%%%%%%%%%%%%%%%%%%%%%%%%
\section{Numerical Results}

To investigate the convergence of the multimesh finite element method, we solve the Stokes problem in the unit square with the following exact solution
\begin{align}
  \label{eq:exactu}
  \bfu(x, y) &= 2 \pi \sin(\pi x) \sin(\pi y) \cdot
  ( \cos(\pi y) \sin(\pi x) , -\cos(\pi x) \sin(\pi y) ), \\
  \label{eq:exactp}
  p(x, y) &= \sin(2 \pi x) \sin(2 \pi y),
\end{align}
and corresponding right hand side. We use $P_kP_{k-1}$ Taylor--Hood elements with $k\in \{2,3,4\}$ and we use $N\in \{1,2,4,8,16,32\}$ randomly placed domains as in \cite{mmfem-1} (see Figure~\ref{fig:poisson_meshes}). Due to the random placement of domains, some domains are completely hidden and will not contribute to the solution. For $N=8$, this is the case for one domain, for $N=16$, three domains and for $N=32$, four domains are completely hidden. This is automatically handled by the computational geometry routines. Convergence results are presented in Figure~\ref{fig:convplots} as well as in Table~\ref{table:rates}.

\begin{figure}
  \begin{center}
    %\includegraphics[width=0.320\linewidth]{figs/multimesh_2-crop.pdf}\ \
    \includegraphics[width=0.320\linewidth]{figs/multimesh_3-crop.pdf}\ \
    \includegraphics[width=0.320\linewidth]{figs/multimesh_5-crop.pdf}\ \ %\\ \ \\
    %\includegraphics[width=0.320\linewidth]{figs/multimesh_9-crop.pdf}\ \
    %\includegraphics[width=0.320\linewidth]{figs/multimesh_17-crop.pdf}\ \
    \includegraphics[width=0.320\linewidth]{figs/multimesh_33-crop.pdf}
  \end{center}
  \caption{A sequence of $N$ meshes are randomly placed on top of a fixed background mesh of the unit square shown here for $N=2, 4$ and $32$ using the coarsest refinement level.}
  \label{fig:poisson_meshes}
\end{figure}

\begin{figure}
  \centering

  \includegraphics[width=0.32\textwidth]{figs/velocity_L2_p2-crop.pdf}
  \includegraphics[width=0.32\textwidth]{figs/velocity_L2_p3-crop.pdf}
  \includegraphics[width=0.32\textwidth]{figs/velocity_L2_p4-crop.pdf}

  \vspace{0.5cm}

  \includegraphics[width=0.32\textwidth]{figs/velocity_H10_p2-crop.pdf}
  \includegraphics[width=0.32\textwidth]{figs/velocity_H10_p3-crop.pdf}
  \includegraphics[width=0.32\textwidth]{figs/velocity_H10_p4-crop.pdf}

  \vspace{0.5cm}

  \includegraphics[width=0.32\textwidth]{figs/pressure_L2_p2-crop.pdf}
  \includegraphics[width=0.32\textwidth]{figs/pressure_L2_p3-crop.pdf}
  \includegraphics[width=0.32\textwidth]{figs/pressure_L2_p4-crop.pdf}
  \caption{Convergence results for $k=2$, $3$ and $4$ (left to right) using up to $32$ meshes (single mesh results are $N=0$). From top to bottom we have the velocity error in the $L^2(\Omega)$ norm, the velocity error in the $H^1_0(\Omega)$ norm, and the pressure error in the $L^2(\Omega)$ norm. Results less than $10^{-8}$ are not included in the convergence lines due to limits in floating point precision. }
  \label{fig:convplots}
\end{figure}

\begin{table}
  \centering
  \caption{Error rates for $\bfe_{L^2} = \|\bfu - \bfu_h\|_{L^2(\Omega)}$, $\bfe_{H^1_0} = \|\bfu - \bfu_h\|_{H^1_0(\Omega)}$ and $e_{L^2} = \|p - p_h\|_{L^2(\Omega)}$.}
  \label{table:rates}
  \begin{tabular}{|c | c c c | c c c | c c c|}
    \toprule
    & & $k=2$  & & & $k=3$  &  & & $k=4$  &   \\
    \midrule
    $N$&
    $\bfe_{L^2}$ & $\bfe_{H^1_0}$ & $e_{L^2}$ &
    $\bfe_{L^2}$ & $\bfe_{H^1_0}$ & $e_{L^2}$ &
    $\bfe_{L^2}$ & $\bfe_{H^1_0}$ & $e_{L^2}$ \\
    \midrule
    0 & 2.9952 & 1.9709 & 2.1466 & 4.0289 & 2.9966 & 3.0028 & 4.9508 & 3.9844 & 4.2286 \\
    1 & 2.9750 & 1.9658 & 1.9291 & 4.1153 & 3.0912 & 3.1932 & 4.8861 & 4.0006 & 4.0587 \\
    2 & 3.2764 & 2.1472 & 2.5036 & 3.9087 & 2.9021 & 2.8489 & 4.8677 & 4.0416 & 4.0832 \\
    4 & 3.6666 & 2.5971 & 2.8597 & 4.3996 & 3.3125 & 3.3957 & 5.2741 & 4.0966 & 4.1609 \\
    8 & 3.0359 & 1.9697 & 2.1163 & 4.3412 & 3.2258 & 3.4213 & 4.8840 & 3.9409 & 4.0169 \\
    16 & 3.4131 & 2.3298 & 2.4794 & 4.5033 & 3.3907 & 3.5910 & 5.4702 & 3.9664 & 4.0729 \\
    32 & 3.2832 & 2.1505 & 2.3255 & 4.4196 & 3.2922 & 3.4362 & 5.7538 & 4.3191 & 4.2848 \\
    \bottomrule
  \end{tabular}
\end{table}

%%%%%%%%%%%%%%%%%%%%%%%%%%%%%%%%%%%%%%%%%%%%%%%%%%%%%%%%%%%%%%%%%%%%%%%%%%%%%%%%
\section{Discussion}

The results presented in Table~\ref{table:rates} and Figure~\ref{fig:convplots} show the expected order of convergence for the velocity in the $L^2(\Omega)$ norm ($k + 1$), for the velocity in the $H^1_0(\Omega)$ norm ($k$), and for the pressure in the $L^2(\Omega)$ norm ($k$).

A detailed inspection of Figure~\ref{fig:convplots} reveals that, as expected, the multimesh discretization yields larger errors than the single mesh discretization (standard Taylor--Hood on one single mesh). The errors introduced by the multimesh discretization are one to two orders of magnitude larger than the single mesh error. However, the convergence rate is optimal and it should be noted that the results presented here are for an extreme scenario where a large number of meshes are simultaneously overlapping; see Figure~\ref{fig:poisson_meshes}. For a normal application, such as the simulation of flow around a collection of objects, each object would be embedded in a boundary-fitted mesh and only a small number of meshes would simultaneously overlap (in addition to each mesh overlapping the fixed background mesh), corresponding to the situation when two or more objects are close.

The presented method and implementation demonstrate the viability of the multimesh method as an attractive alternative to existing methods for discretization of PDEs on domains undergoing large deformations. In particular, the discretization and the implementation are robust to thin intersections and rounding errors, both of which are bound to appear in a simulation involving a large number of meshes, timesteps or configurations.

%%%%%%%%%%%%%%%%%%%%%%%%%%%%%%%%%%%%%%%%%%%%%%%%%%%%%%%%%%%%%%%%%%%%%%%%%%%%%%%%
\begin{acknowledgement}
  August Johansson was supported by the Research Council of Norway through the FRIPRO Program at Simula Research Laboratory, project number 25123. Mats G.\ Larson was supported in part by the Swedish Foundation for Strategic Research Grant No.\ AM13-0029, the Swedish Research Council Grants Nos.\  2013-4708, 2017-03911, and the Swedish Research Programme Essence. Anders Logg was supported by the Swedish Research Council Grant No.\ 2014-6093.
\end{acknowledgement}


%%%%%%%%%%%%%%%%%%%%%%%%%%%%%%%%%%%%%%%%%%%%%%%%%%%%%%%%%%%%%%%%%%%%%%%%%%%%%%%%
\bibliographystyle{spmpsci}
\bibliography{bibliography}
\end{document}


\end{document}
