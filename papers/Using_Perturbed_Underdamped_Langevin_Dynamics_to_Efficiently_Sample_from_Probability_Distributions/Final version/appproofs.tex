
\begin{proof}[of Lemma \ref{lemma:bias}]
Suppose that $(P_t)_{t \ge 0}$ satisfies \eqref{eq:hypocoercive estimate}.  Let $\pi_0$ be an initial distribution of $(X_t)_{t \ge 0}$ such that $\pi_0 \ll \pi$ and $h = \frac{d\pi_0}{d\pi} \in L^2(\pi)$.  Slightly abusing notation, we denote by $\pi_0 P_t$ the law of $X_t$ given $X_0 \sim \pi$.  Then
\begin{align*}
	\lVert \pi_0 P_t - \pi \rVert_{TV} = \left\lVert P_t^* h - 1 \right\rVert_{L^1(\pi)} \leq \left\lVert P_t^* \right\rVert_{L^2(\pi)\rightarrow L^2(\pi)} \left\lVert h - 1 \right\rVert_{L^2(\pi)} \leq Ce^{-\lambda t}\left\lVert h - 1 \right\rVert_{L^2(\pi)},
\end{align*}
where $P_t^*$ denotes the $L^2(\pi)$-adjoint of $P_t$.  Since $f$ is assumed to be bounded, we immediately obtain
$$
  \norm{\mathbb{E}[f(X_t) | X_0 \sim \pi_0] - \pi(f)} \leq C\Norm{f}_{L^\infty}e^{-\lambda t}\left(\mbox{Var}_{\pi}\left[\frac{d\pi_0}{d\pi}\right]\right)^{1/2},
$$
and so, for $X_0 \sim \pi_0$,
$$
	\norm{\pi_T(f) - \pi(f)} \leq \frac{C}{\lambda T}{\left(1 - e^{-\lambda t}\right)}\lVert f \rVert_{L^\infty}\left(\mbox{Var}_{\pi}\left[\frac{d\pi_0}{d\pi}\right]\right)^{1/2},
$$
as required.
  \qed 
\end{proof}

\begin{proof}[of Lemma \ref{lemma:variance}]
Given $f \in L^2(\pi)$, for fixed $T > 0$, 
\begin{equation}
  \chi_T(x) := \int_{0}^{T} \left(\pi(f) - P_t f(x)\right)\mathrm{d}t.
\end{equation}
Then we have that $\chi_T \in \mathcal{D}(\mathcal{L})$ and $\mathcal{L}\chi_T  = f - P_T f$, moreover
\begin{align*}
  \Norm{\chi_{T} - \chi_{T'}}_{L^2(\pi)} &= \Norm{\int^{T'}_{T} P_t(f)-\pi(f)\,\mathrm{d}t}_{L^2(\pi)}\\
  &\leq C\Norm{f}_{L^2(\pi)}\int_{T}^{T'}e^{-\lambda t}\,\mathrm{d}t,
\end{align*}
so that $\lbrace \chi_T \rbrace_{T \geq 0}$ is a Cauchy sequence in $L^2(\pi)$ converging to $\chi = \int_0^\infty \left(\pi(f) - P_tf\right)\mathrm{d}t$.  Since $\mathcal{L}$ is closed and
$$
  (\mathcal{L}\chi_T, \chi_T) \rightarrow (f-\pi(f), \chi),\quad T \rightarrow \infty,
$$
in $L^2(\pi)$, it follows that $\chi\in\mathcal{D}(\mathcal{L})$ and $\mathcal{L}\chi = f - \pi(f)$.  Moreover,
$$
  \Norm{\chi}_{L^{2}(\pi)} \leq \int_0^\infty \Norm{P_t(f) - \pi(f)}_{L^2(\pi)}\,\mathrm{d}t \leq K_{\lambda}\Norm{f-\pi(f)}_{L^2(\pi)},\quad 
$$
where $K_{\lambda} = C\int_0^\infty e^{-\lambda t}\,\mathrm{d}t$. 
Since we assume that $f$ is smooth, the coefficients are smooth and $\gen$ is hypoelliptic, then $\mathcal{L}\chi = f-\pi(f)$ implies that  $\chi \in C^{\infty}(\R^d)$, and thus we can apply It\^{o}'s formula to $\chi(X_t)$ to obtain:
$$
\frac{1}{T}\int_0^T \left[f(X_t) - \pi(f)\right]\,\mathrm{d}t = \frac{1}{T}\left[\chi(X_0) - \chi(X_T)\right] + \frac{1}{T}\int_0^T \nabla\chi(X_t)\sigma(X_t)\,\mathrm{d}W_t.
$$
One can check that the conditions of \cite[Theorem 7.1.4]{EthierKu86} hold.  In particular, the following central limit theorem follows
$$
	\frac{1}{\sqrt{T}}\int_0^T \nabla\chi(X_t)\sigma(X_t)\,\mathrm{d}W_t \xrightarrow{d} \mathcal{N}(0,2\sigma^2_f),\quad \mbox{ as } T \rightarrow \infty.
$$
By Theorem \ref{theorem:invariance_theorem}, the generator $\gen$ has the form
$$
	\gen = \pi^{-1}\nabla\cdot\left(\pi\Sigma \nabla \cdot \right) + \gamma\cdot\nabla,
$$
where $\nabla\cdot(\pi \gamma) = 0$.  It follows that
\begin{equation}
\label{eq:variance_equation}
\sigma^2_f = \inner{\Sigma \nabla\chi}{\nabla\chi}_{L^2(\pi)} = -\inner{\gen \chi}{\chi}_{L^2(\pi)}  = \inner{\chi}{f}_{L^2(\pi)} < \infty.
\end{equation}
First suppose that $X_0 \sim \pi$.  Then $(\chi(X_t))_{t\geq 0}$ is a stationary process, and so 
$$
	\frac{1}{\sqrt{T}}\left(\chi(X_0) - \chi(X_T)\right) \rightarrow 0,\quad \mbox{a.s as } T \rightarrow \infty.
$$
From which \eqref{eq:CLT} follows.  More generally, suppose that $X_0 \sim \pi_0$, where $\pi_0(x) = h(x)\pi(x)$ for $h \in L^2(\pi)$.  If $f \in L^\infty(\pi)$, then by \eqref{lemma:bias},
\begin{align*}
	|\chi(x)| &\leq \int_0^\infty |\pi(f) - P_t f(x)|\,dt \\
			  &\leq \int_0^\infty \lVert f\rVert_{L^\infty}\lVert\pi - \pi_0 P_t\rVert_{TV}\,dt \\
			  & \leq \frac{C}{\lambda }\lVert f\rVert_{L^\infty}\left(\mbox{Var}_{\pi}\left[\frac{d\pi_0}{d\pi}\right]\right)^{1/2},
\end{align*} 
so that $\chi \in L^\infty(\pi)$.  Therefore $\frac{1}{\sqrt{T}}(\chi(X_0)- \chi(X_T)) \xrightarrow{p} 0$ as $T \rightarrow \infty$, and so \eqref{eq:CLT} holds in this case, similarly.
\end{proof}
