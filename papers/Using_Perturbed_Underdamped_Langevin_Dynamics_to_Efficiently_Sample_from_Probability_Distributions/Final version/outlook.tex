
A new family of Langevin samplers was introduced in this paper. These new SDE samplers consist of perturbations of the underdamped Langevin dynamics (that is known to be ergodic with respect to the canonical measure), where auxiliary drift terms in the equations for both the position and the momentum are added, in a way that the perturbed family of dynamics is ergodic with respect to the same (canonical) distribution. These new Langevin samplers were studied in detail for Gaussian target distributions where it was shown, using tools from spectral theory for differential operators, that an appropriate choice of the perturbations in the equations for the position and momentum can improve the performance of the Langvin sampler, at least in terms of reducing the asymptotic variance. The performance of the perturbed Langevin sampler to non-Gaussian target densities was tested numerically on the problem of diffusion bridge sampling.

The work presented in this paper can be improved and extended in several directions. First, a rigorous analysis of the new family of Langevin samplers for non-Gaussian target densities is needed. The analytical tools developed in~\cite{duncan2016variance} can be used as a starting point. Furthermore, the study of the actual computational cost and its minimization by an appropriate choice of the numerical scheme and of the perturbations in position and momentum would be of interest to practitioners. In addition, the analysis of our proposed samplers can be facilitated by using tools from symplectic and differential geometry. Finally, combining the new Langevin samplers with existing variance reduction techniques such as zero variance MCMC, preconditioning/Riemannian manifold MCMC can lead to sampling schemes that can be of interest to practitioners, in particular in molecular dynamics simulations. All these topics are currently under investigation.


\begin{comment}
\notate{There needs to be a conclusion to the paper}
\subsection{'Symplectic manifold MCMC'}

The generator of the unperturbed Langevin dynamics (\ref{eq:langevin})
given by 
\[
\mathcal{L}_{0}=M^{-1}p\cdot\nabla_{q}-\nabla V(q)\cdot\nabla_{p}-\Gamma M^{-1}p\cdot\nabla_{p}+\nabla\Gamma\nabla
\]
is often written as 
\[
\mathcal{L}_{0}=\{H,\cdot\}-\Gamma M^{-1}p\cdot\nabla_{p}+\nabla\Gamma\nabla,
\]
where the Hamiltonian is expressed in terms of the \emph{Hamiltonian
	\[
	H(q,p)=V(q)+\frac{1}{2}p^{T}M^{-1}p
	\]
}and the \emph{Poisson bracket
	\[
	\{A,B\}=\big((\nabla_{q}A)^{T}(\nabla_{p}A)^{T}\Pi_{0}\left(\begin{array}{c}
	\nabla_{q}B\\
	\nabla_{p}B
	\end{array}\right).
	\]
}Here $\Pi_{0}$ denotes the $2d\times2d$-matrix 
\[
\Pi_{0}=\left(\begin{array}{cc}
\boldsymbol{0} & -I\\
I & \boldsymbol{0}
\end{array}\right)
\]
and $A,B:\mathbb{R}^{2d}\rightarrow\mathbb{R}^{2d}$ are sufficiently
regular functions. Now observe that the generator (\ref{eq:generator})
can be expressed in the same way if $\Pi_{0}$ is replaced by 
\[
\tilde{\Pi}=\left(\begin{array}{cc}
\mu J_{1} & -I\\
I & \nu J_{2}
\end{array}\right).
\]
Therefore, the perturbations under investigation in this paper can
be interpreted more abstractly as a change of Poisson structure. In
this framework, the unperturbed Langevin dynamics (\ref{eq:langevin})
should be thought of as evolving in $\mathbb{R}^{2d}$ equipped with
the canonical symplectic structure associated to $\Pi_{0}$. The perturbed
dynamics (\ref{eq:perturbed_underdamped}) then represent
a process in $\mathbb{R}^{2d}$ equipped with the symplectic structure
given rise to $\tilde{\Pi}$. This alternative viewpoint has multiple
advantages. For instance, the underlying symplectic structure suggests
efficient numerical integrators for the perturbed dynamics. Moreover,
this formulation naturally allows for the possibility of introducing
point-dependent Poisson structures connented to perturbations $J_{1}(q,p)$
and $J_{2}(q,p)$ that depend on $q$ and $p$. In this way, it might
be possible to extend the results of this paper to target measures
with locally different covariance structures or targets that are far
away from Gaussian. Lastly, this formulation interacts nicely with
\emph{Riemannian manifold Monte Carlo }approaches (see \cite{GirolamiCalderhead2011}),
where the metric structure of the underlying manifold instead of the
symplectic structure is changed. All of those connections will be
explored further in a subsequent publication.


\end{comment}
