
Sampling from probability measures in high-dimensional spaces is a problem that appears frequently in applications, e.g. in computational statistical mechanics and in Bayesian statistics. In particular, we are faced with the problem of computing expectations with respect to a probability measure $\pi$ on $\mathbb{R}^{d}$, i.e. we wish to evaluate integrals of the form:
\begin{equation}
\label{eq:expectation}
\pi(f):=\int_{\mathbb{R}^{d}}f(x)\pi(\mathrm{d}x).
\end{equation}
As is typical in many applications, particularly in molecular dynamics
and Bayesian inference, the density (for convenience denoted by the same symbol $\pi$) is known only up to a normalization constant; furthermore, the dimension of the underlying space is quite often large enough to render deterministic quadrature schemes computationally
infeasible. 

A standard approach to approximating such integrals is
Markov Chain Monte Carlo (MCMC) techniques \cite{GelmanCaStDuVeRu2014,liu2008monte,robert2013monte}, where a Markov process $(X_{t})_{t\geq0}$ is constructed which is ergodic with respect to the probability measure $\pi$. Then, defining the long-time average
\begin{equation}
\label{eq:estimator}
\pi_{T}(f):=\frac{1}{T}\int_{0}^{T}f(X_{s})\mathrm{d}s
\end{equation}
for $f\in L^{1}(\pi)$, the ergodic theorem guarantees almost sure
convergence of the long-time average $\pi_{T}(f)$ to $\pi(f)$.\\

There are infinitely many Markov, and, for the purposes of this paper diffusion, processes that can be constructed in such a way that they are ergodic with respect to the target distribution. A natural question is then how to choose the ergodic diffusion process $(X_{t})_{t\geq0}$. Naturally the choice should be dictated by the requirement that the computational cost of (approximately) calculating~\eqref{eq:expectation} is minimized. A standard example is given by the \emph{overdamped Langevin dynamics} defined to be
the unique (strong) solution $(X_{t})_{t\ge0}$ of the following stochastic differential equation
(SDE):
\begin{equation}
\mathrm{d}X_{t}=-\nabla V(X_{t})\mathrm{d}t+\sqrt{2}\mathrm{d}W_{t},\label{eq:overdamped}
\end{equation}
where $V=-\log\pi$ is the potential associated with the smooth positive
density $\pi$. Under appropriate assumptions on $V$, i.e. on the measure $\pi(\mathrm{d}x)$, the process $(X_{t})_{t\ge0}$ is ergodic and in fact reversible with respect to the target distribution.
\\
Another well-known example is the \emph{underdamped Langevin
dynamics} given by $(X_t)_{t\ge0} = (q_t, p_t)_{t\ge0}$ defined on the extended space (phase space) $\mathbb{R}^{d}\times\mathbb{\mathbb{R}}^{d}$ by the following pair of coupled SDEs:
\begin{equation}
\begin{aligned}\mathrm{d}q_{t} & =M^{-1}p_{t}\mathrm{d}t,\\
\mathrm{d}p_{t} & =-\nabla V(q_{t})\mathrm{d}t-\Gamma M^{-1}p_{t}\mathrm{d}t+\sqrt{2\Gamma}\mathrm{d}W_{t},
\end{aligned}
\label{eq:langevin}
\end{equation}
where mass and friction tensors $M$ and $\Gamma$, respectively, are assumed to be symmetric positive definite matrices. It is well-known~\cite{pavliotis2014stochastic,LS2016} that $(q_{t},p_{t})_{t\ge0}$ is ergodic with respect to the measure $\widehat{\pi}:=\pi\otimes\mathcal{N}(0,M)$, having density with respect to the Lebesgue measure on $\mathbb{R}^{2d}$ given by
\begin{equation}
\label{eq:augmented target}
\widehat{\pi}(q,p)=\frac{1}{\widehat{Z}}\exp\left(-V(q)-\frac{1}{2}p\cdot M^{-1}p\right),
\end{equation}
where $\widehat{Z}$ is a normalization constant. Note that $\widehat{\pi}$ has marginal $\pi$ with respect to $p$ and thus for functions $f\in L^{1}(\pi)$, we have that $\frac{1}{t}\int_0^t f(q_{t})\,\mathrm{d}t\rightarrow\pi(f)$ almost surely. Notice also that the dynamics restricted to the $q$-variables is no longer Markovian. The $p$-variables can thus be interpreted as giving some instantaneous memory to the system, facilitating efficient exploration of the state space. Higher order Markovian models, based on a finite dimensional (Markovian) approximation of the generalized Langevin equation can also be used~\cite{ceriotti_al09}.
\\\\
As there is a lot of freedom in choosing the dynamics in~\eqref{eq:estimator}, see the discussion in Section~\ref{sec:background}, it is desirable to choose the diffusion process $(X_t)_{t\ge0}$ in such a way that $\pi_T(f)$ can provide a good estimation of $\pi(f)$. The performance of the estimator~\eqref{eq:estimator} can be quantified in various manners. The ultimate goal, of course, is to choose the dynamics as well as the numerical discretization in such a way that the computational cost of the longtime-average estimator is minimized, for a given tolerance. The minimization of the computational cost consists of three steps: bias correction, variance reduction and choice of an appropriate discretization scheme. For the latter step see Section~\ref{sec:numerics} and~\cite[Sec. 6]{duncan2016variance}. 

Under appropriate conditions on the potential $V$ it can be shown that both \eqref{eq:overdamped} and \eqref{eq:langevin} converge to equilibrium exponentially fast, e.g. in relative entropy.  One performance objective would then be to choose the process $(X_t)_{t\ge0}$ so that this rate of convergence is maximised.
Conditions on the potential $V$ which guarantee exponential convergence to equilibrium, both in $L^{2}(\pi)$ and in relative entropy can be found in \cite{markowich2000trend,bakry2013analysis}. A powerful technique for proving exponentially fast convergence to equilibrium that will be used in this paper is C. Villani's theory of hypocoercivity~\cite{villani2009hypocoercivity}. In the case when the target measure $\pi$ is Gaussian, both the overdamped ~\eqref{eq:overdamped} and the underdamped~\eqref{eq:langevin} dynamics become generalized Ornstein-Uhlenbeck processes. For such processes the entire spectrum of the generator -- or, equivalently, the Fokker-Planck operator -- can be computed analytically and, in particular, an explicit formula for the $L^2$-spectral gap can be obtained~\cite{Metafune_formula,OPP12,OPP2015}.  A detailed analysis of the convergence to equilibrium in relative entropy for stochastic differential equations with linear drift, i.e. generalized Ornstein-Uhlenbeck processes, has been carried out in \cite{Arnold2014}. 
% The long-time behaviour of the underdamped Langevin process \eqref{eq:langevin} and the corresponding Fokker-Planck equation has been analysed using various methodologies.  By following a Meyn and Tweedie approach, in \cite{talay1999approximation,talay2002stochastic} it was shown that, for a smooth function $f$ with polynomially growing dervatives, the semigroup ${P}_t f(q,p) = \mathbb{E}[f(q_t,p_t)\, | (q_0, p_0)=(q,p)]$ converges exponentially fast to $\mathbb{E}_{\widehat{\pi}}[f]$, along with all its derivatives.  In \cite{desvillettes2001trend}, the authors apply logarithmic Sobolev inequalities to obtain explicit polynomial decay estimates for perturbed quadratic potentials, see also \cite{bakry1994hypercontractivite,malrieu2001inegalites}.  In \cite{herau2002isotropic} resolvent and spectral estimates are used to establish exponential convergence to equilibrium in relative entropy, following the approach of Bakry and Emery \cite{bakry1984hypercontractivite}, see also \cite{HelfferNier2005}.  
\\\\
In addition to speeding up convergence to equilibrium, i.e. reducing the bias of the estimator~\eqref{eq:estimator}, one is naturally also interested in reducing the asymptotic variance. Under appropriate conditions on the target measure $\pi$ and the observable $f$, the estimator $\pi_T(f)$ satisfies a central limit theorem (CLT)~\cite{KomorowskiLandimOlla2012}, that is,
$$
	\frac{1}{\sqrt{T}}\left(\pi_T(f) - \pi(f)\right) \xrightarrow[T\rightarrow\infty]{d} \mathcal{N}(0, 2\,\sigma^2_f),
$$
where $\sigma^2_f < \infty$ is the \emph{asymptotic variance} of the estimator $\pi_T(f)$. The asymptotic variance characterises how quickly fluctuations of $\pi_T(f)$ around $\pi(f)$ contract to $0$. Consequently, another natural objective is to choose the process $(X_t)_{t\ge0}$ such that $\sigma^2_f$ is as small as possible. It is well known that the asymptotic variance can be expressed in terms of the solution to an appropriate Poisson equation for the generator of the dynamics~\cite{KomorowskiLandimOlla2012}
\begin{equation}\label{e:poisson}
- \cL \phi = f - \pi (f), \quad \sigma^2_f = \int_{\mathbb{R}^d} \phi (- \cL \phi) \, \pi(\mathrm{d}x).
\end{equation}
Techniques from the theory of partial differential equations can then be used in order to study the problem of minimizing the asymptotic variance. This is the approach that was taken in~\cite{duncan2016variance}, see also~\cite{asvar_Hwang}, and it will also be used in this paper.  

Other measures of performance have also been considered.  For example, in~\cite{LDgraphs,LargeDeviations}, performance of the estimator is quantified in terms of the rate functional of the ensemble measure $\frac{1}{t}\int_0^t \delta_{X(t)}(dx)$. See also~\cite{JoulinOllivier2010} for a study of the nonasymptotic behaviour of MCMC techniques, including the case of overdamped Langevin dynamics.  

Similar analyses have been carried out for various modifications of \eqref{eq:overdamped}. Of particular interest to us are the {\it Riemannian manifold MCMC}~\cite{GirolamiCalderhead2011} (see the discussion in Section~\ref{sec:background}) and the {\it nonreversible Langevin samplers}~\cite{Hwang1993,Hwang2005}. As a particular example of the general framework that was introduced in~\cite{GirolamiCalderhead2011}, we mention the preconditioned overdamped Langevin dynamics that was presented in~\cite{alrachid2016some} 
\begin{equation}\label{e:precond-mala}
	dX_t = -P \nabla V(X_t)\,dt + \sqrt{2P}\,dW_t.
\end{equation}
In this paper, the long-time behaviour of as well as the asymptotic variance of the corresponding estimator $\pi_T(f)$ are studied and applied to equilibrium sampling in molecular dynamics. A variant of the standard underdamped Langevin dynamics that can be thought of as a form of preconditioning and that has been used by practitioners is the {\it mass-tensor molecular dynamics}~\cite{Bennett1975267}.

The nonreversible overdamped Langevin dynamics 
\begin{equation}
\label{eq:nonreversible_overdamped}
	dX_t = -\left(\nabla V(X_t) - \gamma(X_t)\right)\,dt + \sqrt{2}\,dW_t,
\end{equation}
where the vector field $\gamma$ satisfies $\nabla\cdot(\pi \gamma) = 0$ is ergodic (but not reversible) with respect to the target measure $\pi$ for all choices of the divergence-free vector field $\gamma$.  The asymptotic behaviour of this process was considered for Gaussian diffusions in~\cite{Hwang1993}, where the rate of convergence of the covariance to equilibrium was quantified in terms of the choice of $\gamma$. This work was extended to the case of non-Gaussian target densities, and consequently for nonlinear SDEs of the form~\eqref{eq:nonreversible_overdamped} in~\cite{Hwang2005}. The problem of constructing the optimal nonreversible perturbation, in terms of the $L^2(\pi)$ spectral gap for Gaussian target densities was studied in~ \cite{LelievreNierPavliotis2013} see also~\cite{hwang2014}. Optimal nonreversible perturbations with respect to miniziming the asymptotic variance were studied in~\cite{duncan2016variance,asvar_Hwang}. In all these works it was shown that, in theory (i.e. without taking into account the computational cost of the discretization of the dynamics~\eqref{eq:nonreversible_overdamped}), the nonreversible Langevin sampler~\eqref{eq:nonreversible_overdamped} always outperforms the reversible one~\eqref{eq:overdamped}, both in terms of converging faster to the target distribution as well as in terms of having a lower asymptotic variance. It should be emphasized that the two optimality criteria, maximizing the spectral gap and minimizing the asymptotic variance, lead to different choices for the nonreversible drift $\gamma(x)$. 


%Various extensions of  \eqref{eq:langevin} have also been studied.  In \cite{eckmann2000non,eckmann1999non} hypoellipticity estimates are used to establish the existence of a unique invariant measure for a chain of anharmonic oscillators coupled to heat baths.   In \cite{jakivsc1997ergodic} the ergodicity of the generalised Langevin enquation (GLE) is established, where the term $-\Gamma M^{-1}p_t\,dt$ in \eqref{eq:langevin} is replaced by a non-Markovian friction term $\int_0^t \gamma(t-s)p_s\,ds$.  In \cite{ottobre2011asymptotic} for a particular class of GLEs, techniques from the theory of hypocoercivity \cite{villani2009hypocoercivity} were used to prove exponentially fast convergence to equilibrium in relative entropy.  More recently, in \cite{redon2016error} a similar analysis is carried out for the \emph{modified Langevin dynamics} where the kinetic energy $\frac{1}{2}|p|^2$ is replaced by a general kinetic energy function $U(p)$.   

The goal of this paper is to extend the analysis presented in~\cite{duncan2016variance,LelievreNierPavliotis2013} by introducing the following modification of the standard underdamped Langevin dynamics:
\begin{equation}
\label{eq:perturbed_underdamped}
\begin{aligned}
\mathrm{d}q_{t} & =M^{-1}p_{t}\mathrm{d}t-\mu J_{1}\nabla V(q_{t})\mathrm{d}t, \\
\mathrm{d}p_{t} & =-\nabla V(q_{t})\mathrm{d}t-\nu J_{2}M^{-1}p_{t}\mathrm{d}t-\Gamma M^{-1}p_{t}\mathrm{d}t+\sqrt{2\Gamma}\mathrm{d}W_{t},
\end{aligned}
\end{equation}
where $M,\Gamma\in\mathbb{R}^{d\times d}$ are constant strictly positive definite matrices, $\mu$ and $\nu$ are scalar constants and $J_1, J_2 \in \mathbb{R}^{d\times d}$ are constant skew-symmetric matrices. 
%The dynamics~\eqref{eq:perturbed_underdamped} can be thought of as a hybrid between the overdamped and the underdamped Langevin dynamics. 
As demonstrated in Section~\ref{sec:perturbed_langevin}, the process defined by \eqref{eq:perturbed_underdamped} will be ergodic with respect to the Gibbs measure $\widehat{\pi}$ defined in \eqref{eq:augmented target}.  
\\\\
Our objective is to investigate the use of these dynamics for computing ergodic averages of the form \eqref{eq:estimator}. To this end, we study the long time behaviour of \eqref{eq:perturbed_underdamped} and, using hypocoercivity techniques, prove that the process converges exponentially fast to equilibrium.  This perturbed underdamped Langevin process introduces a number of parameters in addition to the mass and friction tensors which must be tuned to ensure that the process is an efficient sampler.  For Gaussian target densities, we derive estimates for the spectral gap and the asymptotic variance, valid in certain parameter regimes.  Moreover, for certain classes of observables, we are able to identify the choices of parameters which lead to the optimal performance in terms of asymptotic variance. While these results are valid for Gaussian target densities, we advocate these particular parameter choices also for more complex target densities.  To demonstrate their efficacy, we perform a number of numerical experiments on more complex, multimodal distributions. In particular, we use the Langevin sampler~\eqref{eq:perturbed_underdamped} in order to study the problem of diffusion bridge sampling. 
\\\\
The rest of the paper is organized as follows.  In Section \ref{sec:background} we present some background material on Langevin dynamics, we construct general classes of Langevin samplers and we introduce criteria for assessing the performance of the samplers.  In Section \ref{sec:perturbed_langevin} we study qualitative properties of the perturbed underdamped Langevin dynamics \eqref{eq:perturbed_underdamped} including exponentially fast convergence to equilibrium and the overdamped limit. In Section~\ref{sec:Gaussian} we study in detail the performance of the Langevin sampler~\eqref{eq:perturbed_underdamped} for the case of Gaussian target distributions. In Section~\ref{sec:numerics} we introduce a numerical scheme for simulating the perturbed dynamics~\eqref{eq:perturbed_underdamped} and we present numerical experiments on the implementation of the proposed samplers for the problem of diffusion bridge sampling. Section~\ref{sec:outlook} is reserved for conclusions and suggestions for further work. Finally, the appendices contain the proofs of the main results presented in this paper and of several technical results.

















