
\subsection{Background and Preliminaries}




In this section we consider estimators of the form \eqref{eq:estimator} where $(X_t)_{t\ge0}$ is a diffusion process given by the solution of the following It\^{o} SDE:
\begin{equation}
\label{eq:sde_general}
	\mathrm{d}X_t = a(X_t)\,\mathrm{d}t + \sqrt{2}b(X_t)\,\mathrm{d}W_t,
\end{equation}
with drift coefficient $a: \R^d \rightarrow \R^d$ and  diffusion coefficient $b:\R^d \rightarrow \R^{d\times m}$ both having smooth components, and where $(W_t)_{t\ge0}$ is a standard $\R^m$--valued Brownian motion.  Associated with (\ref{eq:sde_general}) is the infinitesimal generator $\gen$ formally given by
\begin{equation}
\label{eq:generator_general}
\gen f = a\cdot\nabla f  + \Sigma : D^2 f, \quad f \in C^2_c(\R^d)
\end{equation}
where $\Sigma = bb^\top$, $D^2 f$ denotes the Hessian of the function $f$ and $\, : \,$ denotes the Frobenius inner product.  In general, $\Sigma$ is nonnegative definite, and could possibly be degenerate.   In particular, the infinitesimal generator \eqref{eq:generator_general} need not be uniformly elliptic.  To ensure that the corresponding semigroup exhibits sufficient smoothing behaviour,  we shall require that  the process \eqref{eq:sde_general} is hypoelliptic in the sense of H{\"o}rmander.   If this condition holds, then irreducibility of the process $(X_t)_{t\ge0}$ will be an immediate consequence of the existence of a strictly positive invariant distribution $\pi(x)\mathrm{d}x$, see \cite{kliemann1987recurrence}.
\\\\
Suppose that $(X_t)_{t\geq 0}$ is nonexplosive.  It follows from the hypoellipticity assumption that the process $(X_t)_{t\geq 0}$ possesses a smooth transition density $p(t,x, y)$ which is defined for all $t \geq 0$ and $x, y \in \R^d$, \cite[Theorem VII.5.6]{bass1998diffusions}.   The associated strongly continuous Markov semigroup $(P_t)_{t\geq 0}$ is defined by 
\begin{equation}
	P_t f(x) = \int_{\R^d} p(t, x, y)f(y)\,\mathrm{d}y,\quad t \geq 0.
\end{equation}  
Suppose that $(P_t)_{t\ge0}$ is invariant with respect to the target distribution $\pi(x)\,\mathrm{d}x$, i.e.
\begin{equation*}
	\int_{\mathbb{R}^d} P_t f(x)\pi(x)\,\mathrm{d}x = \int_{\mathbb{R}^d} f(x)\pi(x)\,\mathrm{d}x,\quad t \geq 0,
\end{equation*}
for all bounded continuous functions $f$. Then $(P_t)_{t\ge0}$ can be extended to a positivity preserving contraction semigroup on $L^2(\pi)$ which is strongly continuous.   Moreover, the infinitesimal generator corresponding to $(P_t)_{t\ge0}$ is given by an extension of $(\gen, C^{2}_c(\R^d))$, also denoted by $\gen$.
\\\\
Due to hypoellipticity, the probability measure $\pi$ on $\mathbb{R}^d$ has a smooth and positive density with respect to the Lebesgue measure, and (slightly abusing the notation) we will denote this density also by $\pi$. Let $L^2(\pi)$ be the Hilbert space of $\pi$-square integrable functions equipped with inner product $\inner{\cdot}{\cdot}_{L^2(\pi)}$ and norm $\Norm{\cdot}_{L^2(\pi)}$. We will also make use of the \emph{Sobolev space} 
\begin{equation}
H^1(\pi)=\{f \in L^2(\pi):\quad \Vert\nabla f\Vert^2_{L^2(\pi)}<\infty\}
\end{equation} of $L^2(\pi)$-functions with weak derivatives in $L^2(\pi)$, equipped with norm
\begin{equation*}
\Vert f \Vert ^2_{H^1(\pi)} = \Vert f \Vert ^2_{L^2(\pi)} + \Vert \nabla f \Vert ^2_{L^2(\pi).}
\end{equation*}

\subsection{A General Characterisation of Ergodic Diffusions}
\label{sec:characterisation}

A natural question is what conditions on the coefficients $a$ and $b$ of \eqref{eq:sde_general} are required to ensure that $(X_t)_{t\ge0}$ is invariant with respect to the distribution $\pi(x)\,\mathrm{d}x$.    The following result provides a necessary and sufficient condition for a diffusion process to be invariant with respect to a given target distribution.

\begin{theorem}
\label{theorem:invariance_theorem}
Consider a diffusion process $(X_t)_{t\ge0}$ on $\mathbb{R}^{d}$ defined by the unique, non-explosive solution to the It\^{o} SDE \eqref{eq:sde_general} with drift $a \in C^1(\mathbb{R}^{d}; \mathbb{R}^d)$ and  diffusion coefficient $b\in C^1(\mathbb{R}^d; \mathbb{R}^{d\times m})$.  Then $(X_t)_{t\ge0}$ is invariant with respect to ${\pi}$ if and only if  
\begin{equation}
\label{eq:invariant_drift}
a = \Sigma \nabla \log \pi + \nabla\cdot \Sigma + \gamma,
\end{equation} 
where $\Sigma = bb^\top$ and $\gamma: \mathbb{R}^D\rightarrow \mathbb{R}^D$  is a continuously differentiable vector field satisfying 
\begin{equation}
\label{eq:invariance_condition}
\nabla\cdot\left(\pi \gamma \right) = 0.
\end{equation}  
If additionally $\gamma \in L^1({\pi})$, then there exists a skew-symmetric matrix function $C:\mathbb{R}^d \rightarrow \mathbb{R}^{d\times d}$ such that
$$
  \gamma = \frac{1}{{\pi}} \nabla\cdot\left({\pi} C \right).
$$
In this case the infinitesimal generator can be written as an $L^2(\pi)$-extension of
$$
\mathcal{L}f = \frac{1}{{\pi}}\nabla\cdot\left((\Sigma + C){\pi}\nabla f\right),\quad f\in C^2_c(\mathbb{R}^d).
$$ 
\end{theorem}

The proof of this result can be found in~\cite[Ch. 4]{pavliotis2014stochastic}; similar versions of this characterisation can be found in~\cite{villani2009hypocoercivity} and \cite{Hwang2005}. See also~\cite{ma2015complete}.

\begin{remark}
If \eqref{eq:invariant_drift} holds and $\mathcal{L}$ is hypoelliptic it follows immediately that $(X_t)_{t\ge0}$ is ergodic with unique invariant distribution $\pi(x)\,\mathrm{d}x$.
\end{remark}

More generally, we can consider It\^{o} diffusions in an extended phase space:

\begin{equation}\label{e:SDE}
\mathrm{d} Z_{t} = b(Z_t) \, \mathrm{d}t + \sqrt{2}\sigma(Z_{t}) \, \mathrm{d}W_{t}, 
\end{equation}
where $(W_{t})_{t\ge0}$ is a standard Brownian motion in $\R^{N}$, $N \geq d$. This is a Markov process with generator

\begin{equation}\label{e:gen}
\mathcal{L} = b(z) \cdot \nabla_z +  \Sigma(z) : D^2_z , 
\end{equation}
where $\Sigma(z) = \big( \sigma \sigma^{T} \big)(z)$. We will consider dynamics $(Z_t)_{t\ge0}$ that is ergodic with respect to $\pi_z(z) \, \mathrm{d}z$ such that 
\begin{equation}\label{e:marginal}
\int_{\R^{m}} \pi_z (x, \, y) \, \mathrm{d}y = \pi(x).
\end{equation}
where $z = (x, \, y), \; x \in \R^d, \, y \in \R^m, \; d+m = N$.

There are various well-known choices of dynamics which are invariant (and indeed ergodic) with respect to the target distribution $\pi(x)\mathrm{d}x$.
\begin{enumerate}
  \item Choosing  $b = I$ and $\gamma = 0$ we immediately recover the overdamped Langevin dynamics (\ref{eq:overdamped}).
  \item Choosing $b = I$, and $\gamma \neq 0$ such that \eqref{eq:invariance_condition} holds gives rise to the nonreversible overdamped equation defined by \eqref{eq:nonreversible_overdamped}.  As it  satisfies the conditions of Theorem \ref{theorem:invariance_theorem}, it is ergodic with respect to $\pi$.  In particular choosing $\gamma(x) = J\nabla V(x)$ for a constant skew-symmetric matrix $J$ we obtain
  \begin{equation}
  \label{eq:nonrev_overdamped_J}
    \mathrm{d}X_t = -(I + J)\nabla V(X_t)\,\mathrm{d}t + \sqrt{2}\,\mathrm{d}W_t,
  \end{equation}
  which has been studied in previous works.  

  \item Given a target density $\pi > 0$ on $\R^d$, if we consider the augmented target density $\widehat{\pi}$ on $\R^{2d}$ given in \eqref{eq:augmented target},
  then choosing
  \begin{equation}
  \label{eq:underdamped_gamma}
    \gamma((q,p)) = \left(\begin{array}{c} M^{-1}p \\ -\nabla V(q)\end{array}\right)
  \end{equation}
  and 
  \begin{equation}
    \label{eq:underdamped_sigma}
    b = \left(\begin{array}{c}\boldsymbol{0} \\ \sqrt{\Gamma}\end{array}\right) \in \mathbb{R}^{2d \times d},
  \end{equation}
  where $M$ and $\Gamma$ are positive definite symmetric matrices, the conditions of Theorem \ref{theorem:invariance_theorem} are satisfied for the target density $\widehat{\pi}$.  The resulting dynamics $(q_t, p_t)_{t\ge0}$ is determined by the underdamped Langevin equation (\ref{eq:langevin}). It is straightforward to verify that the generator is hypoelliptic, \cite[Sec 2.2.3.1]{Free_energy_computations}, and thus $(q_t, p_t)_{t\ge0}$ is ergodic. 

  \item More generally, consider the augmented target density $\widehat{\pi}$ on $\mathbb{R}^{2d}$ as above, and choose 
  \begin{equation}
  \label{eq:underdamped_gamma_perturbed}
    \gamma((q,p)) = \left(\begin{array}{c} M^{-1}p - \mu J_1\nabla V(q) \\ -\nabla V(q) - \nu J_2 M^{-1}p\end{array}\right)
  \end{equation}
  and 
  \begin{equation}
    \label{eq:underdamped_sigma_pertured}
    b = \left(\begin{array}{c}\boldsymbol{0} \\ \sqrt{\Gamma}\end{array}\right) \in \mathbb{R}^{2d \times d},
  \end{equation}
  where $\mu$ and $\nu$ are scalar constants and $J_1, J_2 \in \mathbb{R}^{d\times d}$ are constant skew-symmetric matrices.  With this choice we recover the perturbed Langevin dynamics \eqref{eq:perturbed_underdamped}.  It is straightforward to check that \eqref{eq:underdamped_gamma_perturbed}  satisfies the invariance condition (\ref{eq:invariance_condition}), and thus Theorem \ref{theorem:invariance_theorem} guarantees that (\ref{eq:perturbed_underdamped}) is invariant with respect to $\widehat{\pi}$. 

  \item In a similar fashion, one can introduce an  augmented target density on $\mathbb{R}^{(m+2)d}$, with
  \begin{align*}
   \widehat{\widehat{\pi}}(q, p, u_1,\ldots, u_m) \propto e^{-\frac{|p|^2}{2} - \frac{u_1^2 + \ldots + u_m^2}{2}-V(q)},
  \end{align*}
  where $p, q, u_i \in \mathbb{R}^d$, for $i=1,\ldots, m$.  Clearly $\int_{\mathbb{R}^{d}\times \mathbb{R}^{md}} \widehat{\widehat{\pi}}(q, p, u_1,\ldots,u_m)\,\mathrm{d}p\,\mathrm{d}u_1\,\ldots \mathrm{d}u_m = \pi(q)$. We now define $\gamma:\mathbb{R}^{(m+2)d}\rightarrow \mathbb{R}^{(m+2)d}$ by
  $$
  \gamma(q,p, u_1,\ldots,u_m) = \left(\begin{array}{c}p \\ -\nabla_q V(q) + \sum_{j=1}^{m} \lambda_j u_j \\ -\lambda_1 p \\ \vdots \\ -\lambda_m p \end{array}\right) 
  $$
  and $b: \mathbb{R}^{(m+2)d}\rightarrow \mathbb{R}^{(m+2)d\times (m+2)d}$ by 
  $$
  b(q,p,u_1,\ldots, u_m) = \left(\begin{array}{cccccc}\boldsymbol{0} & \boldsymbol{0} & \boldsymbol{0} & \boldsymbol{0} & \ldots & \boldsymbol{0}\\ \boldsymbol{0} & \boldsymbol{0} & \boldsymbol{0} & \boldsymbol{0}& \ldots & \boldsymbol{0} \\ \boldsymbol{0} & \boldsymbol{0} & \sqrt{\alpha_1}I_{d\times d} & \boldsymbol{0} & \ldots & \boldsymbol{0} \\ \boldsymbol{0} & \boldsymbol{0} & \boldsymbol{0} & \sqrt{\alpha_2}I_{d\times d} & \ldots & \boldsymbol{0} \\ 
   \vdots & \vdots & \vdots & \vdots & \ddots & \vdots \\ \boldsymbol{0} & \boldsymbol{0} & \boldsymbol{0} &\boldsymbol{0} & \ldots & \sqrt{\alpha_m}I_{d\times d}\end{array}\right),
  $$
  where $\lambda_i \in \mathbb{R}$ and $\alpha_i > 0$, for $i=1,\ldots, m$.  The resulting process \eqref{eq:sde_general} is given by 
  \begin{equation}
  \label{eq:gle_markov}
  \begin{aligned}
    \mathrm{d}q_t &= p_t \,\mathrm{d}t \\ 
    \mathrm{d}p_t &= -\nabla_q V(q_t)\,\mathrm{d}t + \sum_{j=1}^{d}\lambda_j u^{j}(t)\,\mathrm{d}t \\
    \mathrm{d}u^{1}_t &= -\lambda_1 p_t\,\mathrm{d}t -\alpha_1 u^{1}_t\,\mathrm{d}t + \sqrt{2\alpha_1 }\,\mathrm{d}W^{1}_t\\
    \vdots &  \\
    \mathrm{d}u^{m}_t &= -\lambda_m p_t\,\mathrm{d}t -\alpha_m u^{m}_t\,\mathrm{d}t + \sqrt{2\alpha_m }\,\mathrm{d}W^{m}_t,
  \end{aligned}
  \end{equation}
  where $(W^1_t)_{t\ge0}, \ldots (W^m_t)_{t\ge0}$ are independent $\mathbb{R}^d$--valued Brownian motions.   This process is ergodic with unique invariant distribution $\widehat{\widehat{\pi}}$, and under appropriate conditions on $V$, converges exponentially fast to equilibrium in relative entropy \cite{ottobre2011asymptotic}.  Equation \eqref{eq:gle_markov} is a Markovian representation of a generalised Langevin equation of the form
  \begin{align*}
  \mathrm{d}q_t &= p_t \,\mathrm{d}t \\
  \mathrm{d}p_t &= -\nabla_{q}V(q_t) \,\mathrm{d}t - \int_0^t F(t-s)p_s\,\mathrm{d}s + N(t),
  \end{align*}
  where $N(t)$ is a mean-zero stationary Gaussian process with autocorrelation function $F(t)$, i.e.
  $$
    \mathbb{E}\left[ N(t) \otimes N(s) \right] = F(t-s)I_{d\times d},
  $$
  and 
  $$
    F(t) = \sum_{i=1}^{m} \lambda_i^2 e^{-\alpha_i|t|}.
  $$

  \item Let $\widetilde{\pi}(z) \propto \exp(-\Phi(z))$ be a positive density on $\mathbb{R}^N$ where $N > d$ such that 
  $$
    \pi(x) = \int_{\mathbb{R}^{N-d}}\widetilde{\pi}(x,z)\,\mathrm{d}z,
  $$
  where $(x, y)\in \mathbb{R}^d\times \mathbb{R}^{N-d}$. Then choosing $b = I_{D\times D}$ and $\gamma = 0$ we obtain the dynamics
  \begin{align*}
      \mathrm{d}X_t &= -\nabla_x \Phi(X_t, Y_t)\,\mathrm{d}t + \sqrt{2}\,\mathrm{d}W^{1}_t \\ 
      \mathrm{d}Y_t &= -\nabla_y \Phi(X_t, Y_t)\,\mathrm{d}t + \sqrt{2}\,\mathrm{d}W^{2}_t,
  \end{align*}
  then  $(X_t, Y_t)_{t\ge0}$ is immediately ergodic with respect to $\widetilde{\pi}$.
\end{enumerate}

\subsection{Comparison Criteria}
\label{sec:comparison}

For a fixed observable $f$, a natural measure of accuracy of the estimator $\pi_T(f) = t^{-1}\int_0^{t}f(X_s)\,\mathrm{d}s$ is the \emph{mean square error} (MSE) defined by
\begin{equation}
\label{eq:mse}
MSE(f, T) := \mathbb{E}_{x}\norm{\pi_T(f) - \pi(f)}^2,
\end{equation}
where $\mathbb{E}_{x}$ denotes the expectation conditioned on the process $(X_t)_{t\ge0}$ starting at $x$.  
It is instructive to introduce the decomposition $MSE(f, T) = \mu^2(f, T) + \sigma^2(f, T)$, where
\begin{equation}
\label{eq:bias_variance_decomposition}
  \mu(f, T) = \norm{\mathbb{E}_{x}[\pi_T(f)] - \pi(f)}\quad\mbox{ and }\quad \sigma^2(f, T) = \mathbb{E}_{x}\norm{\pi_T(f) - \pi(f)}^2 = \mbox{Var}[\pi_T(f)].
\end{equation}
Here $\mu(f, T)$ measures the bias of the estimator $\pi_T(f)$ and $\sigma^2(f, T)$ measures the variance of fluctuations of $\pi_T(f)$ around the mean.   
\\\\
The speed of convergence to equilibrium of the process $(X_t)_{t\ge0}$ will control both the bias term $\mu(f, T)$ and the variance $\sigma^2(f, T)$.  To make this claim more precise, suppose that the semigroup $(P_t)_{t\ge0}$ associated with $(X_t)_{t\ge0}$ decays exponentially fast in $L^2(\pi)$, i.e. there exist constants  $\lambda > 0$ and $C\ge1$ such that
\begin{equation}
\label{eq:hypocoercive estimate}
 \left\lVert P_t g - \pi(g) \right\rVert_{L^2(\pi)} \leq C e^{- \lambda t} \left\lVert g-\pi(g) \right\rVert_{L^2(\pi)},\quad g\in L^2(\pi).
\end{equation}
\begin{remark}
	If \eqref{eq:hypocoercive estimate} holds with $C=1$, this estimate is equivalent to $-\gen$ having a spectral gap in $L^2(\pi)$. Allowing for a constant $C>1$ is essential for our purposes though in order to treat nonreversible and degenerate diffusion processes by the theory of \emph{hypocoercivity} as outlined in \cite{villani2009hypocoercivity}.  
\end{remark}
The following lemma characterises the decay of the bias $\mu(f,T)$ as $T\rightarrow \infty$ in terms of $\lambda$ and $C$.  The proof can be found in Appendix \ref{app:proofs}.
\begin{lemma}
\label{lemma:bias}
Let $(X_t)_{t\geq 0}$ be the unique, non-explosive solution of \eqref{eq:sde_general}, such that $X_0 \sim \pi_0 \ll \pi$ and $\frac{d\pi_0}{d\pi} \in L^2(\pi)$, where $\frac{d\pi_0}{d\pi}$ denotes the Radon-Nikodym derivative of $\pi_0$ with respect to $\pi$.  Suppose that the process is ergodic with respect to $\pi$ such that the Markov semigroup $(P_t)_{t\geq 0}$ satisfies (\ref{eq:hypocoercive estimate}).  Then for $f \in L^\infty(\pi)$,
\begin{equation*}
\mu(f, T)  \leq \frac{C}{\lambda T}\left({1 - e^{-\lambda T}}\right)\lVert f \rVert_{L^\infty}\mbox{Var}_{\pi}\left[\frac{d\pi_0}{d\pi}\right]^{\frac{1}{2}}.
\end{equation*}
\end{lemma}

The study of the behaviour of the variance $\sigma^2(f, T)$ involves deriving a central limit theorem for the additive functional $\int_0^t f(X_t)-\pi(f)\,\mathrm{d}t$.  As discussed in \cite{cattiaux2012central}, we reduce this problem to proving well-posedness of the Poisson equation 
\begin{equation}
\label{eq:poisson_general}
-\gen \chi = f - \pi(f),\quad \pi(\chi) = 0.
\end{equation}
The only complications in this approach arise from the fact that the generator $\gen$ need not be symmetric in $L^2(\pi)$ nor uniformly elliptic. The following result summarises conditions for the well-posedness of the Poisson equation and it also provides with  us with a formula for the asymptotic variance.  The proof can be found in Appendix \ref{app:proofs}.

\begin{lemma}
\label{lemma:variance}
Let $(X_t)_{t\geq 0}$ be the unique, non-explosive solution of \eqref{eq:sde_general}  with smooth drift and diffusion coefficients, such that the corresponding infinitesimal generator is hypoelliptic.  Syppose that $(X_t)_{t\ge0}$ is ergodic with respect to $\pi$ and moreover, $(P_t)_{t\ge0}$ decays exponentially fast in $L^2(\pi)$ as in \eqref{eq:hypocoercive estimate}.  Then for all $f\in L^2(\pi)$, there exists a unique mean zero solution $\chi$ to the Poisson equation \eqref{eq:poisson_general}.  If $X_0 \sim \pi$, then for all $f \in C^\infty(\mathbb{R}^d) \cap L^2(\pi)$
\begin{equation}
\label{eq:CLT}
  \sqrt{T}\left(\pi_T(f) - \pi(f)\right) \xrightarrow[T\rightarrow\infty]{d} \mathcal{N}(0, 2\sigma^2_f),
\end{equation}
where $\sigma^2_f$ is the asymptotic variance defined by
\begin{equation}
\label{eq:asymptoticvariance}
\sigma^2_{f} = \inner{\chi}{(-\gen)\chi}_{L^2(\pi)} = \inner{\nabla \chi}{\Sigma\nabla\chi}_{L^2(\pi)}.
\end{equation}
Moreover, if $X_0 \sim \pi_0$ where $\pi_0 \ll \pi$ and $\frac{d\pi_0}{d\pi}\in L^2(\pi)$ then \eqref{eq:CLT} holds for all $f \in C^\infty(\mathbb{R}^d) \cap L^\infty(\pi)$.
\end{lemma}
%
Clearly, observables that only differ by a constant have the same asymptotic variance. 
%Indeed, consider two observables $f\in L^{2}(\pi)$ and $f_{C}:=f+C\in L^{2}(\pi)$
%	with $C\in\mathbb{R}$. Then
%	\[
%	\pi(f_{C})=\pi(f)+C
%	\]
%	and 
%	\[
%	\frac{1}{t}\int_{0}^{t}f_{C}(X_{s})\mathrm{d}s=\frac{1}{t}\int_{0}^{t}f(X_{s})\mathrm{d}s+C,
%	\]
%	therefore
%	\[
%	\frac{1}{t}\int_{0}^{t}f(X_{s})\mathrm{d}s-\pi(f)=\frac{1}{t}\int_{0}^{t}f_{C}(X_{s})\mathrm{d}s-\pi(f_{C}).
%	\]
%	It thus follows that $\sigma_{f}^{2}=\sigma_{f_{C}}^{2}$. 
In the	sequel, we will hence restrict our attention to observables $f\in L^{2}(\pi)$	satisfying $\pi(f)=0$, simplifying expressions~\eqref{eq:poisson_general}
	and~\eqref{eq:CLT}. The corresponding subspace of $L^2(\pi)$ will be denoted by 
\begin{equation}
L_{0}^2(\pi):=\{f \in L^2(\pi): \pi(f)=0\}.
\end{equation}
If the exponential decay estimate \eqref{eq:hypocoercive estimate} is satisfied, then Lemma \ref{lemma:variance} shows that $-\gen$ is invertible on $L^2_{0}(\pi)$, so we can express the asymptoptic variance as 
\begin{equation}
\label{eq:asym variance inverse}
\sigma_{f}^2=\langle f, (-\gen)^{-1} f \rangle_{L^2(\pi)}, \quad f \in L^2_{0}(\pi).
\end{equation}
Let us also remark that from the proof of Lemma \ref{lemma:variance} it follows that the inverse of $\mathcal{L}$ is given by
\begin{equation}
\mathcal{L}^{-1}=\int_0^{\infty}P_t \,\mathrm{d}t.
\end{equation}
We note that the constants $C$ and $\lambda$ appearing in the exponential decay estimate \eqref{eq:hypocoercive estimate} also control the speed of convergence of $\sigma^2(f, T)$ to zero.  Indeed, it is straightforward to show that if \eqref{eq:hypocoercive estimate} is satisfied, then the solution $\chi$ of \eqref{eq:poisson_general} satisfies
\begin{equation}\label{e:estim-sigma}
  \sigma^2_{f} = \inner{\chi}{f-\pi(f)}_{L^2(\pi)} \leq \frac{C}{\lambda}\Norm{f}^2_{L^2(\pi)}.
\end{equation}

Lemmas \ref{lemma:bias} and \ref{lemma:variance} would suggest that choosing the coefficients $\Sigma$ and $\gamma$ to optimize the constants $C$ and $\lambda$ in~\eqref{e:estim-sigma} would be an effective means of improving the performance of the estimator $\pi_T(f)$, especially since the improvement in performance would be uniform over an entire class of observables.  When this is possible, this is indeed the case. However, as has been observed in \cite{LelievreNierPavliotis2013,Hwang1993,Hwang2005}, maximising the speed of convergence to equilibrium is a delicate task.  As the leading order term in $MSE(f, T)$, it is typically sufficient to focus specifically on the asymptotic variance $\sigma^2_{f}$ and study how the parameters of the SDE \eqref{eq:sde_general} can be chosen to minimise $\sigma^2_{f}$. This study was undertaken in \cite{duncan2016variance} for processes of the form \eqref{eq:nonreversible_overdamped}.
