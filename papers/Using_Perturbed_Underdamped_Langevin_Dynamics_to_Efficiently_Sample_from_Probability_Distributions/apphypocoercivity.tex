
\begin{proof} of Lemma \ref{lem:hypoellipticity}
	We first note that $\gen$ in \eqref{eq:generator} can be written in the ``sum of squares'' form:
	$$
	\mathcal{L}=A_{0}+\frac{1}{2}\sum_{k=1}^{d}A_{k}^{2},
	$$
	where 
	$$
	A_{0}=M^{-1}p\cdot\nabla_{q}-\nabla_{q}V\cdot\nabla_{p}-\mu J_{1}\nabla_{q}V\cdot\nabla_{q}-\nu J_{2}M^{-1}p\cdot\nabla_{p}-\Gamma M^{-1}p\cdot\nabla_{p}
	$$
	and
	$$
	A_{k}=e_{k}\cdot\Gamma^{1/2}\nabla_{p}, \quad k = 1,\ldots, d.
	$$
	Here $\lbrace e_k \rbrace_{k=1,\ldots, d}$ denotes the standard Euclidean basis and $\Gamma^{1/2}$ is the unique positive definite square root of the matrix $\Gamma$.   The relevant commutators turn out to be
	\[
	[A_{0},A_{k}]=e_{k}\cdot\Gamma^{1/2} M^{-1}(\Gamma\nabla_{p}-\nabla_{q}-\nu J_{2}\nabla_{p}), \quad k = 1,\ldots, k.
	\]
	Because $\Gamma$ has full rank on $\R^d$, it follows that 
	\[
	\Span\{A_{k}:k=1,\ldots d\}=\Span\{\partial_{p_{k}}:k=1,\ldots,d\}.
	\]
	Since
	\[
	e_{k}\cdot\Gamma^{1/2} M^{-1}(\Gamma\nabla_{p}-\nu J_{2}\nabla_{p})\in\Span\{A_{j}:j=1,\ldots d\},\quad k=1,\ldots,d,
	\]
	and $\Span(\lbrace \Gamma^{1/2}M^{-1}\nabla_q \,:\, k=1,\ldots, d \rbrace) = \Span\{\partial_{q_{k}}:k=1,\ldots,d\}$, it follows that
	\[
	\Span(\lbrace A_{k}:k=0,1,\ldots,d\rbrace\cup\{[A_{0},A_{k}]:k=1,\ldots,d\}) = \R,
	\]
	so the assumptions of H\"{o}rmander's theorem hold.\qed
\end{proof}

\subsection{The overdamped limit}

The following is a technical lemma required for the proof of Proposition \ref{prop: overdamped limit}:
\begin{lemma}
	\label{lem:bounded p}Assume the conditions from Proposition \ref{prop: overdamped limit}.
	Then for every $T>0$ there exists $C>0$ such that 
	\[
	\mathbb{E} \left( \sup_{0\le t\le T}\vert p_{t}^{\epsilon}\vert^{2} \right) \le C.
	\]
	
\end{lemma}
\begin{proof}
	Using variation of constants, we can write the second line of (\ref{eq:rescaling})
	as 
	\[
	p_{t}^{\epsilon}=e^{-\frac{t}{\epsilon^{2}}(\nu J_{2}+\Gamma)M^{-1}}p_{0}-\frac{1}{\epsilon}\int_{0}^{t}e^{-\frac{(t-s)}{\epsilon^{2}}(\nu J_{2}+\Gamma)M^{-1}}\nabla_{q}V(q_{s}^{\epsilon})\mathrm{d}s+\frac{1}{\epsilon}\sqrt{2\Gamma}\int_{0}^{t}e^{-\frac{(t-s)}{\epsilon^{2}}(\nu J_{2}+\Gamma)M^{-1}}\mathrm{d}W_{s}.
	\]
	We then compute 
	\begin{align}
	\mathbb{E}\sup_{0\le t\le T}\vert p_{t}^{\epsilon}\vert^{2} & =\sup_{0\le t\le T}\left\lvert e^{-\frac{t}{\epsilon^{2}}(\nu J_{2}+\Gamma)M^{-1}}p_{0}\right\rvert^{2}+\frac{1}{\epsilon^{2}}\mathbb{E}\sup_{0\le t\le T}\left\lvert\int_{0}^{t}e^{-\frac{(t-s)}{\epsilon^{2}}(\nu J_{2}+\Gamma)M^{-1}}\nabla_{q}V(q_{s}^{\epsilon})\mathrm{d}s\right\rvert^{2}\nonumber \\
	& +\frac{1}{\epsilon^{2}}\mathbb{E}\sup_{0\le t\le T}\left\lvert\sqrt{2\Gamma}\int_{0}^{t}e^{-\frac{(t-s)}{\epsilon^{2}}(\nu J_{2}+\Gamma)M^{-1}}\mathrm{d}W_{s}\right\rvert^{2}\nonumber \\
	& -\frac{1}{\epsilon}\mathbb{E}\sup_{0\le t\le T}\left(e^{-\frac{t}{\epsilon^{2}}(\nu J_{2}+\Gamma)M^{-1}}p_{0}\cdot\int_{0}^{t}e^{-\frac{(t-s)}{\epsilon^{2}}(\nu J_{2}+\Gamma)M^{-1}}\nabla_{q}V(q_{s}^{\epsilon})\mathrm{d}s\right)\label{eq:P^2}\\
	& +\frac{1}{\epsilon}\mathbb{E}\sup_{0\le t\le T}\left(e^{-\frac{t}{\epsilon^{2}}(\nu J_{2}+\Gamma)M^{-1}}p_{0}\cdot\frac{1}{\epsilon}\sqrt{2\Gamma}\int_{0}^{t}e^{-\frac{(t-s)}{\epsilon^{2}}(\nu J_{2}+\Gamma)M^{-1}}\mathrm{d}W_{s}\right)\nonumber \\
	& -\frac{1}{\epsilon^{2}}\mathbb{E}\sup_{0\le t\le T}\left(\int_{0}^{t}e^{-\frac{(t-s)}{\epsilon^{2}}(\nu J_{2}+\Gamma)M^{-1}}\nabla_{q}V(q_{s}^{\epsilon})\mathrm{d}s\cdot\sqrt{2\Gamma}\int_{0}^{t}e^{-\frac{(t-s)}{\epsilon^{2}}(\nu J_{2}+\Gamma)M^{-1}}\mathrm{d}W_{s}\right).\nonumber 
	\end{align}
	Clearly, the first term on the right hand side of (\ref{eq:P^2})
	is bounded. For the second term, observe that 
	\begin{equation}
	\frac{1}{\epsilon^{2}}\mathbb{E}\sup_{0\le t\le T}\left\lvert\int_{0}^{t}e^{-\frac{(t-s)}{\epsilon^{2}}(\nu J_{2}+\Gamma)M^{-1}}\nabla_{q}V(q_{s}^{\epsilon})\mathrm{d}s\right\rvert^{2}\le\frac{1}{\epsilon^{2}}\sup_{0\le t\le T}\int_{0}^{t}\left\lVert e^{-\frac{(t-s)}{\epsilon^{2}}(\nu J_{2}+\Gamma)M^{-1}}\right\rVert^{2}\mathrm{d}s\label{eq:estimate1}
	\end{equation}
	since $V \in C^1(\mathbb{T}^d)$ and therefore $\nabla_{q}V$ is bounded. By the basic matrix exponential estimate
	$\Vert e^{-t(\nu J_{2}+\Gamma)M^{-1}}\Vert\le Ce^{-\omega t}$ for
	suitable $C$ and $\omega$, we see that (\ref{eq:estimate1}) can
	further be bounded by 
	\[
	\frac{1}{\epsilon^{2}}C\sup_{0\le t\le T}\int_{0}^{t}e^{-2\omega\frac{(t-s)}{\epsilon^{2}}}\mathrm{d}s=\frac{C}{2\omega}\left(1-e^{-2\omega\frac{T}{\epsilon^{2}}}\right),
	\]
	so this term is bounded as well. The third term is bounded by the
	Burkholder\textendash Davis\textendash Gundy inequality and a similar
	argument to the one used for the second term applies. The cross terms can
	be bounded by the previous ones, using the Cauchy-Schwarz inequality
	and the elementary fact that $\sup(ab)\le\sup a\cdot\sup b$ for $a,b>0$, so the
	result follows. \qed
\end{proof}

\begin{proof}
	[of Proposition \ref{prop: overdamped limit}] Equations (\ref{eq:rescaling})
	can be written in integral form as
	\[
	(\nu J_{2}+\Gamma)q_{t}^{\epsilon}=(\nu J_{2}+\Gamma)q_{0}^{\epsilon}+\frac{1}{\epsilon}\int_{0}^{t}(\nu J_{2}+\Gamma)M^{-1}p_{s}^{\epsilon}\mathrm{d}s-\mu\int_{0}^{t}(\nu J_{2}+\Gamma)J_{1}\nabla_{q}V(q_{s}^{\epsilon})\mathrm{d}s
	\]
	and 
	\begin{equation}
	-\int_{0}^{t}\nabla V(q_{s}^{\epsilon})\mathrm{d}s-\frac{1}{\epsilon}\int_{0}^{t}(\nu J_{2}+\Gamma)M^{-1}p_{s}^{\epsilon}\mathrm{d}s+\sqrt{2\Gamma}W(t)=\epsilon(p_{t}^{\epsilon}-p_{0}),\label{eq:rescaled p equation-1}
	\end{equation}
	where the first line has been multiplied by the matrix $\nu J_{2}+\Gamma$.
	Combining both equations yields
	\[
	q_{t}^{\epsilon}=q_{0}^{\epsilon}-\int_{0}^{t}(\nu J_{2}+\Gamma)\nabla_{q}V(q_{s}^{\epsilon})\mathrm{d}s-\epsilon(\nu J_{2}+\Gamma)^{-1}(p_{t}^{\epsilon}-p_{0})-\mu\int_{0}^{t}J_{1}\nabla_{q}V(q_{s})\mathrm{d}s+(\nu J_{2}+\Gamma)^{-1}\sqrt{2\Gamma}W_{t}.
	\]
	Now applying Lemma \ref{lem:bounded p} gives the desired result,
	since the above equation differs from the integral version of (\ref{eq:overdamped limit})
	only by the term $\epsilon(\nu J_{2}+\Gamma)^{-1}(p_{t}^{\epsilon}-p_{0})$
	which vanishes in the limit as $\epsilon\rightarrow0$. \qed
\end{proof}

\subsection{Hypocoercivity}

The objective of this section is to prove that the perturbed dynamics
(\ref{eq:perturbed_underdamped}) converges to equilibrium
exponentially fast, i.e. that the associated semigroup $(P_t)_{t\ge0}$ satisfies the estimate \eqref{eq:hypocoercive estimate}. We we will be using the theory of hypocoercivity outlined in
\cite{villani2009hypocoercivity} (see also the exposition in \cite[Section 6.2]{pavliotis2014stochastic}).
We provide a brief review of the theory of hypocoercivity.
\\\\
Let $(\mathcal{H},\langle\cdot,\cdot\rangle)$ be a real separable
Hilbert space and consider two unbounded operators $A$ and $B$ with
domains $D(A)$ and $D(B)$ respectively, $B$ antisymmetric. Let
$S\subset\mathcal{H}$ be a dense vectorspace such that $S\subset D(A)\cap D(B)$,
i.e. the operations of $A$ and $B$ are authorised on $S$. The theory
of hypocoercivity is concerned with equations of the form 
\begin{equation}
\label{eq:abstract fp equation}
\partial_{t}h+Lh=0,
\end{equation}
and the associated semigroup $(P_t)_{t\ge0}$ generated by $L=A^{*}A-B$. Let
us also introduce the notation $K=\ker L$. With the choices $\mathcal{H}=L^{2}(\widehat{\pi})$,
$A=\sigma\nabla_{p}$ and $B=M^{-1}p\cdot\nabla_{q}-\nabla_{q}V\cdot\nabla_{p}-\mu J_{1}\nabla_{q}V\cdot\nabla_{q}-\nu J_{2}M^{-1}p\cdot\nabla_{p},$
it turns out that $L$ is the (flat) $L^2(\mathbb{R}^{2d})$-adjoint of the generator $\mathcal{L}$ given in \eqref{eq:generator} and therefore equation \eqref{eq:abstract fp equation} is the Fokker-Planck equation associated to the dynamics \eqref{eq:perturbed_underdamped}. 
	In many situations of practical interest, the operator $A^{*}A$ is
	coercive only in certain directions of the state space, and therefore
	exponential return to equilibrium does not follow in general. In our
	case for instance, the noise acts only in the $p$-variables and therefore
	relaxation in the $q$-variables cannot be concluded a priori. However,
	intuitively speaking, the noise gets transported through the equations
	by the Hamiltonian part of the dynamics. This is what the theory of
	hypocoercivity makes precise. Under some conditions on the interactions
	between $A$ and $B$ (encoded in their iterated commutators), exponential
	return to equilibrium can be proved.
To state the main abstract theorem, we need the following definitions: 
\begin{definition}
	(Coercivity) Let $T$ be an unbounded operator on $\mathcal{H}$ with
	domain $D(T)$ and kernel $K$. Assume that there exists another Hilbert
	space $(\tilde{\mathcal{H}},\langle\cdot,\cdot\rangle_{\tilde{\mathcal{H}}})$,
	continuously and densely embedded in $K^{\perp}$. The operator is
	said to be $\lambda$-coercive if 
	\[
	\langle Th,h\rangle_{\tilde{\mathcal{H}}}\ge\lambda\Vert h\Vert_{\tilde{\mathcal{H}}}^{2}
	\]
	for all $h\in K^{\perp}\cap D(T)$. 
\end{definition}

\begin{definition}
	An operator $T$ on $\mathcal{H}$ is said to be relatively bounded
	with respect to the operators $T_{1},\ldots,T_{n}$ if the intersection
	of the domains $\cap D(T_{j})$ is contained in $D(T)$ and there
	exists a constant $\alpha>0$ such that 
	\[
	\Vert Th\Vert\le\alpha(\Vert T_{1}h\Vert+\ldots+\Vert T_{n}h\Vert)
	\]
	holds for all $h\in D(T)$. 
\end{definition}
We can now proceed to the main result of the theory.
\begin{theorem}{\cite[Theorem 24]{villani2009hypocoercivity}}
	\label{thm: hypocoercivity abstract}Assume there exists $N\in\mathbb{N}$
	and possibly unbounded operators $$C_{0},C_{1},\ldots,C_{N+1},R_{1},\ldots,R_{N+1},Z_{1},\ldots,Z_{N+1},$$
	such that $C_{0}=A$, 
	\begin{equation}
	[C_{j},B]=Z_{j+1}C_{j+1}+R_{j+1}\quad(0\le j\le N),\quad C_{N+1}=0,\label{eq:iterated commutators}
	\end{equation}
	and for all $k=0,1,\ldots,N$ 
	\begin{enumerate}[label=(\alph*)]
		\item \label{it:hypo1} $[A,C_{k}]$ is relatively bounded with respect to $\{C_{j}\}_{0\le j\le k}$
		and $\{C_{j}A\}_{0\le j\le k-1}$, 
		\item \label{it:hypo2} $[C_{k},A^{*}]$ is relatively bounded with respect to $I$ and $\{C_{j}\}_{0\le j\le k}$
		,
		\item \label{it:hypo3} $R_{k}$ is relatively bounded with respect to $\{C_{j}\}_{0\le j\le k-1}$
		and $\{C_{j}A\}_{0\le j\le k-1}$ and
		\item \label{it:hypo4} there are positive constants $\lambda_{i}$, $\Lambda_{i}$ such that
		$\lambda_{j}I\le Z_{j}\le\Lambda_{j}I$.
	\end{enumerate}
	Furthermore, assume that $\sum_{j=0}^{N}C_{j}^{*}C_{j}$ is $\kappa$-coercive
	for some $\kappa>0$.  Then, there exists $C\ge0$ and $\lambda>0$ such that 
	\begin{equation}
	\Vert P_t\Vert_{\mathcal{H}^{1}/K\rightarrow \mathcal{H}^{1}/K}\le Ce^{-\lambda t},\label{eq:hypocoercivity estimate}
	\end{equation}
	where $\mathcal{H}^{1}\subset\mathcal{H}$ is the subspace associated
	to the norm
	\begin{equation}
	\label{eq:abstractH1_norm}
	\Vert h\Vert_{\mathcal{H}^{1}}=\sqrt{\Vert h\Vert^{2}+\sum_{k=0}^{N}\Vert C_{k}h\Vert^{2}}
	\end{equation}
	and $K=\ker(A^{*}A-B)$. \end{theorem}
\begin{remark}
	Property (\ref{eq:hypocoercivity estimate}) is called \emph{hypocoercivity
		of $L$ on $\mathcal{H}^{1}:=(K^{\perp},\Vert\cdot\Vert_{\mathcal{H}^{1}})$.}
\end{remark}
If the conditions of the above theorem hold, we also get a regularization
result for the semigroup $e^{-tL}$ (see  \cite[Theorem A.12]{villani2009hypocoercivity}):
\begin{theorem}
	\label{thm:hypocoercive regularisation}Assume the setting and notation
	of Theorem \ref{thm: hypocoercivity abstract}. Then there exists
	a constant $C>0$ such that for all $k=0,1,\ldots,N$ and $t \in (0,1]$ the following
	holds:
	\[
	\Vert C_{k}P_{t}h\Vert\le C\frac{\Vert h\Vert}{t^{k+\frac{1}{2}}},\quad h\in\mathcal{H}.
	\]
\end{theorem}
\begin{proof}
	[of Theorem \ref{theorem:Hypocoercivity}]. We pove the claim by verifying
	the conditions of Theorem \ref{thm: hypocoercivity abstract}. Recall
	that $C_{0}=A=\sigma\nabla_{p}$ and 
	\[
	B=M^{-1}p\cdot\nabla_{q}-\nabla_{q}V\cdot\nabla_{p}-\mu J_{1}\nabla_{q}V\cdot\nabla_{q}-\nu J_{2}M^{-1}p\cdot\nabla_{p}.
	\]
	A quick calculation shows that 
	\[
	A^{*}=\sigma M^{-1}p-\sigma\nabla_{p},
	\]
	so that indeed 
	\[
	A^{*}A=\Gamma M^{-1}p\cdot\nabla_{p}-\nabla^{T}\Gamma\nabla=\mathcal{L}_{therm}
	\]
	and 
	\[
	A^{*}A-B=-\mathcal{L}^{*}.
	\]
	We make the choice $N=1$ and calculate the commutator 
	\[
	[A,B]=\sigma M^{-1}(\nabla_{q}+\nu J_{2}\nabla_{p}).
	\]
	Let us now set $C_{1}=\sigma M^{-1}\nabla_{q}$, $Z_{1}=1$ and $R_{1}=\nu\sigma M^{-1}J_{2}\nabla_{p}$,
	such that (\ref{eq:iterated commutators}) holds for $j=0$. Note
	that $[A,A]=0$\footnote{This is not true automatically, since $[A,A]$ stands for the array
		$([A_{j},A_{k}])_{jk}$.}, $[A,C_{1}]=0$ and $[A^{*},C_{1}]=0$. Furthermore, we have that
	\[
	[A,A^{*}]=\sigma M^{-1}\sigma.
	\]
	We now compute 
	\[
	[C_{1},B]=-\sigma M^{-1}\nabla^{2}V\nabla_{p}+\mu\sigma M^{-1}\nabla^{2}VJ_{1}\nabla_{q}
	\]
	and choose $R_{2}=[C_{1},B]$, $Z_{2}=1$ and recall that $C_{2}=0$
	by assumption (of Theorem \ref{thm: hypocoercivity abstract}). With those choices, assumptions \ref{it:hypo1}-\ref{it:hypo4} of Theorem \ref{thm: hypocoercivity abstract} are fulfilled. Indeed, assumption \ref{it:hypo1} holds trivially 
	since all relevant commutators are zero. Assumption \ref{it:hypo2} follows from the fact that $[A,A^{*}]=\sigma M^{-1}\sigma$ is clearly bounded
	relative to $I$. To verify assumption \ref{it:hypo3}, let us start with the
	case $k=1$. It is necessary to show that $R_{1}=\nu\sigma M^{-1}J_{2}\nabla_{p}$
	is bounded relatively to $A=\sigma\nabla_{p}$ and $A^{2}$.
	This is obvious since the $p$-derivatives appearing in $R_{1}$ can
	be controlled by the $p$-derivatives appearing in $A$. For $k=2,$
	a similar argument shows that $R_{2}=-\sigma M^{-1}\nabla^{2}V\nabla_{p}+\mu\sigma M^{-1}\nabla^{2}VJ_{1}\nabla_{q}$
	is bounded relatively to $A=\sigma\nabla_{p}$ and $C_{1}=\sigma M^{-1}\nabla_{q}$
	because of the assumption that $\nabla^{2}V$ is bounded. Note that it
	is crucial for the preceding arguments to assume that the matrices
	$\sigma$ and $M$ have full rank. Assumption \ref{it:hypo4} is trivially satisfied,
	since $Z_{1}$ and $Z_{2}$ are equal to the identity.  It remains to show that 
	\[
	T:=\sum_{j=0}^{N}C_{j}^{*}C_{j}
	\]
	is $\kappa$-coercive for some $\kappa>0$.  It is straightforward
	to see that the kernel of $T$ consists of constant functions and
	therefore 
	\[
	(\ker T)^{\perp}=\{\phi\in L^{2}(\mathbb{R}^{2d},\widehat{\pi}):\quad\widehat{\pi}(\phi)=0\}.
	\]
	Hence, $\kappa$-coercivity of $T$ amounts to the functional inequality
	\[
	\int_{\mathbb{R}^{2d}}\big(\vert\sigma M^{-1}\nabla_{q}\phi\vert^{2}+\vert\sigma\nabla_{p}\phi\vert^{2}\big)\mathrm{d}\widehat{\pi}\ge\kappa\bigg(\int_{\mathbb{R}^{2d}}\phi^{2}\mathrm{d}\widehat{\pi}-\left(\int_{\mathbb{R}^{2d}}\phi\mathrm{d}\widehat{\pi}\right)^{2}\bigg),\quad\phi\in H^{1}(\widehat{\pi}).
	\]
	Since the transformation $\phi\mapsto\psi$, $\psi(q,p)=\phi(\sigma^{-1}Mq,\sigma^{-1}p)$ is bijective on $H^{1}(\mathbb{R}^{2d},\widehat{\pi})$, the above is equivalent to 
	\[
	\int_{\mathbb{R}^{2d}}\big(\vert\nabla_{q}\psi\vert^{2}+\vert\nabla_{p}\psi\vert^{2}\big)\mathrm{d}\widehat{\pi}\ge\kappa\bigg(\int_{\mathbb{R}^{2d}}\psi^{2}\mathrm{d}\widehat{\pi}-\left(\int_{\mathbb{R}^{2d}}\psi\mathrm{d}\widehat{\pi}\right)^{2}\bigg),\quad\psi\in H^{1}(\widehat{\pi}),
	\]
	i.e. a Poincar\'{e} inequality for $\widehat{\pi}$. Since $\widehat{\pi}=\pi\otimes\mathcal{N}(0,M),$
	coercivity of $T$ boils down to a Poincar\'{e} inequality for $\pi$
	as in Assumption \ref{ass:bounded+Poincare}. This concludes the proof of the hypocoercive decay estimate
	(\ref{eq:hypocoercivity estimate}). Clearly, the abstract $\mathcal{H}^{1}$-norm from \eqref{eq:abstractH1_norm}
	is equivalent to the Sobolev norm $H^{1}(\widehat{\pi})$, and therefore it follows that there exist constants $C\ge0$ and $\lambda\ge0$ such that 
	\begin{equation}
	\label{eq:H1_decay}
	\Vert P_{t} f \Vert_{H^1(\widehat{\pi})}  \le C e^{-\lambda t} \Vert f \Vert _{H^1(\widehat{\pi})},
	\end{equation}
	for all $f \in H^1(\widehat{\pi})\setminus K$, where $K=\ker T$ consists of constant functions.  Let us now lift this estimate to $L^2(\widehat{\pi})$. There exist a constant $\tilde{C}\ge0$ such that 
	\begin{equation}
	\Vert h \Vert_{H^1(\widehat{\pi})} \le \tilde{C} \sum_{k=0}^2 \Vert C_k h \Vert_{L^2(\widehat{\pi})},
	\quad f \in H^1(\widehat{\pi}).
	\end{equation}
	Therefore, Theorem \ref{thm:hypocoercive regularisation} implies 
	\begin{equation}
	\label{eq:H1L2_reg}
	\Vert P_{1} f \Vert_{H^1(\widehat{\pi})} \le \tilde{C} \Vert f \Vert_{L^2(\widehat{\pi})},
	\quad f \in L^2(\widehat{\pi}), 
	\end{equation}
	for $t=1$ and a possibly different constant $\tilde{C}$. Let us now assume that $t\ge1$ and $f \in L^2(\widehat{\pi})\setminus K$.	It holds that
	\begin{equation}
	\Vert P_t f \Vert_{L^2(\widehat{\pi})} \le \Vert P_t f \Vert_{H^1(\widehat{\pi})} = \Vert P_{t-1}P_{1} f \Vert_{H^1(\widehat{\pi})}
	\le C e^{-\lambda (t-1)} \Vert P_{1} f \Vert_{H^1(\widehat{\pi})}, 
	\end{equation}
	where the last inequality follows from \eqref{eq:H1_decay}. Now applying \eqref{eq:H1L2_reg} and gathering constants results in 
	\begin{equation}
	\Vert P_t f\Vert_{L^2(\widehat{\pi})} \le C e^{-\lambda t}\Vert f \Vert_{L^2(\widehat{\pi})}, \quad f \in L^2(\widehat{\pi})\setminus K.
	\end{equation} 
	Note that although we assumed $t\ge1$, the above estimate also holds for $t\ge0$ (although possibly with a different constant $C$) since $\Vert P_t \Vert_{L^2(\widehat{\pi})\rightarrow L^2(\widehat{\pi})}$ is bounded on $[0,1]$. 
	\qed  
\end{proof}
