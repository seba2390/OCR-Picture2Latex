\documentclass[11pt]{amsart}

\usepackage{amssymb}
\usepackage{overpic}
\usepackage{enumitem}   
\usepackage{graphicx} 
\usepackage{multicol}
\usepackage{xcolor}
\usepackage{amsrefs}
\usepackage[a4paper,margin=1.5cm]{geometry}
\usepackage[hidelinks]{hyperref}

\graphicspath{{Figures/}}

\newtheorem{theorem}{Theorem}
\newtheorem{question}{Question}
\newtheorem{proposition}{Proposition} 
\newtheorem{corollary}{Corollary} 
\newtheorem{remark}{Remark} 
\newtheorem{lemma}{Lemma} 
\newtheorem{definition}{Definition} 
\newtheorem{example}{Example} 
\newtheorem{quest}{Question} 

\setlength{\parskip}{0.2cm} 

\begin{document}
	
\title[Non-hyperbolic limit cycles on planar polynomial vector fields]{On the structural instability of non-hyperbolic limit cycles on planar polynomial vector fields}

\author[Paulo Santana]
{Paulo Santana$^1$}

\dedicatory{To Jorge Sotomayor, in memorian.}

\address{$^1$ IBILCE-UNESP, CEP 15054--000, S\~ao Jos\'e do Rio Preto, S\~ao Paulo, Brazil}
\email{paulo.santana@unesp.br}

\subjclass[2020]{34A34; 34C07; 34C25; 37G15}

\keywords{Structural Stability; Limit cycles; Polynomial vector fields}

\begin{abstract}
	It is well known that non-hyperbolic limit cycles are structurally stable in the set of planar smooth and analytical vector fields. In the case of planar polynomial vector fields, it is known that non-hyperbolic limit cycle of even degree are structurally unstable. However, it is not known if those of odd degree are also structurally unstable. Here, we prove that such limit cycles are structurally unstable if we consider the compact-open topology on the space of polynomial vector fields.
\end{abstract}

\maketitle

\section{Introduction}

Roughly speaking, a vector field $X$ is structurally stable if small perturbations does not change the topological character of its orbits. Let $B^2=\{x\in\mathbb{R}^2\colon ||x||\leqslant1\}$ denote the closed unit disk and let $\partial B^2$ denote its topological boundary. The notion of \emph{structural stability} (first known as \emph{robustness}) is due to Andronov and Pontrjagin \cite{AndPon1937}, whose in 1937 enunciated (without proofs) sufficient and necessary conditions for an analytical vector field $X$, traversal to $\partial B^2$, to be structurally stable in $B^2$. In 1952 DeBaggis \cite{Bag1952} provided the omitted proofs under the less restrictive hypothesis of $X$ being of class $C^1$. On January of 1959 M. M. Peixoto \cite{Pei1959} provided an equivalent definition of structural stability and extend some of the previous results to $B^n=\{x\in\mathbb{R}^n\colon ||x||\leqslant1\}$. On June of 1959 M. M. Peixoto and his wife, M. C. Peixoto \cite{PeiPei1959}, extended the notion of structural stability to vector fields of class $C^1$ defined on a two-dimensional manifold $M\subset\mathbb{R}^2$ with boundary and corners, allowing contact between $X$ and $\partial M$. In 1962 M. M. Peixoto \cite{Pei1962} (from now on referred only as Peixoto) provided sufficient and necessary conditions for structural stability on vector fields of class $C^1$ defined on a two-dimensional closed manifold $M$ (i.e. compact and without boundary). Such characterization is known as \emph{Peixoto's Theorem}. In 1973 Peixoto \cite{Pei1973} also derived a relation between structural stability and Graph Theory. Given a two-dimensional closed manifold $M$, let $\mathfrak{X}$ be the family of all $C^1$-vector fields defined over $M$. Let also $\Sigma\subset\mathfrak{X}$ be the set of all the structural stable vector fields over $M$ and denote $\mathfrak{X}_1=\mathfrak{X}\backslash\Sigma$. In 1974 Sotomayor \cite{Soto1974} provided the complete characterization of all the structurally stable vector fields of $\mathfrak{X}_1$, giving rise to the notion of \emph{structural stability of first order} (or vector fields of \emph{codimension one}). In 1977 Teixeira \cite{Tei1977} extended the work of Sotomayor, allowing $M$ to be a manifold with boundary. From there on, the notion of structural stability flourished in many ways, with the characterization of structural stability of $C^1$-vector fields on open surfaces \cite{Kot1982}, of polynomial vector fields on compact \cites{Soto1985,Cai1979,San1977,Vel} and non-compact \cites{Sha1987,DumSha1990} two dimensional manifolds, $C^1$-vector fields on compact $n$-dimensional manifolds \cite{Mar1961}, gradient flows \cite{Sha1990}, polynomial foliations \cite{JarLliSha2005}, quasihomogenous vector fields \cites{OliZha2014,AlgFueGamGar2018}, reversible vector fields \cite{BuzRobTei2021} and piecewise smooth vector fields \cites{BuzRodTei2022,PesSot2012}. Moreover, there is a great effort of Art\'es and coauthors for the classification of the quadratic vector fields with low codimension. For the classification of quadratic vector fields of codimension zero and one we refer to \cite{ArtKooLli} and \cite{ArtRezLli}. For the ongoing classification of those with codimension two, we refer to \cites{ArtOliRez,ArtMotRez,Art2024}. For the history of Peixoto and the develop of the theory of structural stability, we refer to \cite{SotGarMel2020,Sot2020}. In this paper, we work on the structural instability of non-hyperbolic limit cycles in the set of planar polynomial vector fields.

\section{Statement and discussion of the main results}

As stated in this previous section, the notion of \emph{robustness} of a vector field was first provided by Andronov and Pontrjagin \cite{AndPon1937}. Let $B=\{x\in\mathbb{R}^2\colon ||x||\leqslant1\}$, where $||\cdot||$ denotes the standard norm of $\mathbb{R}^2$, and let $\mathfrak{X}$ be the set of $C^1$-vector fields defined over $B$ and without contact with $\partial B^2$, where $\partial B^2$ denotes the topological boundary of $B^2$. Let also $\mathfrak{X}$ be endowed with the $C^1$-topology (i.e. to vector fields are close if their components and its first order partial derivatives are close).

\begin{definition}[Robustness in the sense of Andronov and Pontrjagin]
	Let $X\in\mathfrak{X}$. We say that $X$ is \emph{robust} if for every $\varepsilon>0$ there is a neighborhood $N\subset\mathfrak{X}$ of $X$ and a family of homeomorphisms $h\colon N\to\text{Hom}(B^2,B^2)$, such that for every $Y\in N$ the homeomorphism $h_Y\colon B^2\to B^2$ sends orbits of $X$ to orbits of $Y$, either preserving of reversing the direction of all orbits, and such that the following statements hold.
	\begin{enumerate}[label=(\alph*)]
		\item $h_X=\text{Id}_{B^2}$;
		\item For every $Y\in N$ and $x\in B^2$, $||h_Y(x)-x||<\varepsilon$.
	\end{enumerate}
\end{definition}

One of the first and main contributions of Peixoto \cite{Pei1959} was to prove the above statements $(a)$ and $(b)$ are redundant (even in the case of analytical vector fields). That is, the homeomorphism $h_Y$ does not need to be in a $\varepsilon$-neighborhood of the identity map. Hence, Peixoto provided a more general, and yet equivalent, notion of robustness. He called such notion \emph{structural stability}.

\begin{definition}[Structural stability in the sense of Peixoto]
	Let $X\in\mathfrak{X}$. We say that $X$ is \emph{structural stable} if there is a neighborhood $N\subset\mathfrak{X}$ of $X$ such that for each $Y\in N$ there is an homeomorphism $h_Y\colon B^2\to B^2$ sending orbits of $X$ to orbits of $Y$, either preserving of reversing the direction of all orbits.
\end{definition}

After that, Peixoto also extended his work by replacing $B^2$ for a two-dimensional manifold $M\subset\mathbb{R}^2$ with boundary and conners \cite{PeiPei1959} and then extended it again allowing $M$ to be any two-dimensional closed manifold \cite{Pei1962}. Similar results also hold for the case in which $M\subset\mathbb{R}^2$ is an open surface (e.g. $M=\mathbb{R}^2$). See \cite{Kot1982}. However, when dealing with structural stability in \emph{polynomial vector fields} of some maximum degree $n$, things become more difficult to work with. This is the case because to proof that some object \emph{is not} structurally stable (e.g. a non-hyperbolic limit cycle) one need to \emph{construct} some suitable perturbation that ``breaks'' this object (e.g. slip the non-hyperbolic limit cycle in two or more limit cycles). Therefore, when working with a more restrict space of vector fields, one may obtain a more broader set of structurally stable objects. Concerns about this go back to Aldronov et al \cite[$\mathsection6.3$]{And1971}. Still, there are great approaches of the structural stability of polynomial vector fields. For this, we refer the works of Sotomayor \cite{Soto1985} and Shafer \cite{Sha1987}. On one hand, Sotomayor \cite{Soto1985} defined the structural stability of a planar polynomial vector field $X$ as the structural stability of its Poincar\'e compactification $p(X)$ (see \cite[Chapter $5$]{DumLliArt2006}) and endowed the space of vector fields with the coefficients topology. On the other hand, Shafer \cite{Sha1987} approachs $X$ as a vector field defined on the open surface $M=\mathbb{R}^2$ and endows the space of vector fields with either the Whitney's topology $C^r$-topology, $r\geqslant 1$, or the coefficients topology. However, in the case of coefficients topology, there is an open question that kept both Sotomayor and Shafer from obtaining necessary \emph{and} sufficient conditions for structural stability.

\begin{question}\label{Q1}
	Let $\mathcal{X}_n$ be the set of the planar polynomial vector fields of degree at most $n$, endowed with the coefficients topology. If $X\in\mathcal{X}_n$ has a non-hyperbolic limit cycle of odd degree, then is $X$ structurally unstable in $\mathcal{X}_n$?
\end{question}

Here we recall that the degree of a limit cycle (also known as \emph{multiplicity}) is the order of the first non-zero derivative of its displacement map. Question~\ref{Q1} was explicitly raised by both Sotomayor \cite[Problem $1.1$]{Soto1985} and Shafer \cite[Question $3.4$]{Sha1987}. In the case of non-hyperbolic limit cycles of even degree, the structural instability follows from the theory of rotated vector fields. More precisely, if $X=(P,Q)$ has an non-hyperbolic limit cycle $\gamma$ of even degree, then consider $Y_\lambda=X+\lambda X^\perp$, where $X^\perp=(-Q,P)$. It follows from \cite[Theorem $2$, p. $387$]{Perko2001} that for $|\lambda|>0$ small enough, $\gamma$ had either vanished or slip in two or more limit cycles. In the case of smooth and analytical vector fields it is well known that non-hyperbolic limit cycles (in particular, those with odd degree) are structurally unstable. Briefly, the proof work as follows. Let $X=(P,Q)$ be a planar vector field of class $C^r$, $r\geqslant 1$, and let $\gamma(t)$ be a limit cycle of $X$, with period $T>0$. Andronov et al \cite[p. 124]{And1971}, proved that there are a neighborhood $G\subset\mathbb{R}^2$ of $\gamma$ and a function $F\colon G\to\mathbb{R}$ of class $C^{r+1}$, such that 
	\[F(\gamma(t))=0, \quad \frac{\partial F}{\partial x}(\gamma(t))^2+\frac{\partial F}{\partial y}(\gamma(t))^2>0,\]
for all $t\in[0,T]$. Peixoto \cite[Lemma~$6$]{Pei1959}, proved that if $X$ is analytical, then $F$ is also analytical. With such function, we consider the vector field $Y_\lambda=(R_\lambda,S_\lambda)$ given by
	\[R_\lambda(x,y)=P(x,y)+\lambda F(x,y)\frac{\partial F}{\partial x}(x,y), \quad S_\lambda(x,y)=Q(x,y)+\lambda F(x,y)\frac{\partial F}{\partial y}(x,y),\]
with $\lambda\in\mathbb{R}$. From the Poincar\'e-Bendixson Theorem (see \cite[Section $1.7$]{DumLliArt2006} it can be proven that for $|\lambda|>0$ small enough, $\gamma$ had slip in at least two limit cycles. For more details, see \cite[$\mathsection15$]{And1971} and \cite[Lemma~$6$]{Pei1959}. Observe that if $F$ is polynomial, then $\gamma$ is an algebraic limit cycle. Since not every limit cycle is algebraic (see \cite{GasGiaTor2007}), it follows that even if $X$ is polynomial, $F$ may not be. Hence, for the polynomial case the usual tools breaks down. We now state our main results concretely.

Given two topological spaces $A$ and $B$, let $C(A,B)$ be the set of the continuous functions $f\colon A\to B$. Given a compact set $K\subset A$ and an open set $U\subset B$, consider $V(K,U)\subset C(A,B)$ the set of continuous functions $f\colon A\to B$ such that $f(K)\subset U$. The \emph{compact-open topology} is the topology on $C(A,B)$ having the collection of all such $V(K,U)$ as a sub-base. In other words, the compact-open topology is the smaller topology that contain all such $V(K,U)$. Consider $A=B=\mathbb{R}^2$ and let $C(A,B)=C(\mathbb{R}^2,\mathbb{R}^2)$ be endowed with the compact-open topology. Let $\mathcal{X}$ be the set of all planar polynomial vector fields \emph{of any degree}. Since $\mathcal{X}\subset C(\mathbb{R}^2,\mathbb{R}^2)$, we can endows $\mathcal{X}$ with the \emph{subspace topology}, inherited from the compact open topology on $C(\mathbb{R}^2,\mathbb{R}^2)$. That is, two vector fields $X$ $Y\in\mathcal{X}$ are close if where is a compact $K\subset\mathbb{R}^2$ and a small $\varepsilon>0$ such that
	\[||X(x,y)-Y(x,y)||<\varepsilon,\]
for every $(x,y)\in K$. For the sake of simplicity, from now on we will say that $\mathcal{X}$ is endowed with the compact-open topology. Our main result state that non-hyperbolic limit cycles of odd degree are structurally in $\mathcal{X}$.

\begin{theorem}\label{T1}
	Let $X$ be a planar polynomial vector field with a non-hyperbolic limit cycle of odd degree. Then $X$ is structurally unstable in $\mathcal{X}$, endowed with the compact-open topology. Moreover, if $\deg X=N$, then for every $n\geqslant N$ the perturbation $X_\lambda$ of $X$ can be choose such that $||X-X_\lambda||=o(||(x,y)||^n)$.
\end{theorem}

We observe that the perturbation that we will construct to prove Theorem~\ref{T1} is still within $\mathcal{X}$, i.e. the perturbation that breaks the limit-cycle is polynomial. We regard this because although the author in \cite{Sha1987} works with the Whitney's topology, his perturbations are not polynomial. 

Given $n\in\mathbb{N}$, let $\mathcal{X}_n$ be the space of planar polynomial vector fields of degree at most $n$, endowed with the coefficients topology. In this paper we also provide a different proof of the following already-known fact.

\begin{theorem}\label{T2}
	Let $X$ be a planar polynomial vector field with a non-hyperbolic limit cycle of even degree. Then $X$ is structurally unstable in $\mathcal{X}_n$, endowed with the coefficients topology. 
\end{theorem}

As commented before, Theorem~\ref{T2} follows as a corollary of the theory of rotated vector fields (see \cite[Section $4.6$]{Perko2001}). However, we have a different proof that may be extended to a general proof that also extends to the non-hyperbolic limit cycles of odd degree and thus providing a definitive answer to Question~\ref{Q1}. More precisely, let $X\in\mathcal{X}_n$ have a non-hyperbolic limit cycle $\gamma$, of whether even or odd degree. In our proof of Theorem~\ref{T2} we were able to prove that there are a neighborhood $N\subset\mathcal{X}_n$ of $X$, in relation to the coefficients topology, and a non-constant analytical map $\Phi\colon N\to\mathbb{R}$ satisfying the following statements.
\begin{enumerate}[label=(\alph*)]
	\item $0$ is not a regular value of $\Phi$;
	\item $X\in \Phi^{-1}(0)$;
	\item If $Y\in N$ has a non-hyperbolic limit cycle near $\gamma$, then $Y\in \Phi^{-1}(0)$;
	\item $\Phi^{-1}(0)$ has zero Lebesgue measure on $N$.
\end{enumerate}
With this map, the following statements hold (see Remark~\ref{Remark1}).
\begin{enumerate}[label=(\roman*)]
	\item If there is $Y\in N$ such that $\Phi(Y)\neq0$, then non-hyperbolic limit cycles of \emph{even} degree are not structurally stable;
	\item If there is $Y\in N$ such that $\Phi(Y)<0$, then non-hyperbolic limit cycles of \emph{odd} degree are not structurally stable.
\end{enumerate}
It is clear that statement $(i)$ follows directly from statement $(d)$ and thus we have a proof of Theorem~\ref{T2}. However, we were not able to prove that there is $Y\in N$ such that $\Phi(Y)<0$. Therefore, statement $(ii)$ remains an open sufficient condition for a positive answer for Question~\ref{Q1}. Since as far as we known this sufficient condition is new, we find it useful to state it in this paper.

\begin{question}\label{Q2}
	Let $\Phi\colon N\to\mathbb{R}$ be given as in this proof of Theorem~\ref{T2}. There is $Y\in N$ such that $\Phi(Y)<0$?
\end{question}

The paper is organized as follows. In Section~\ref{Sec3} we have tome preliminaries results. Theorems~\ref{T1} and \ref{T2} are proved in Sections~\ref{Sec4} and \ref{Sec5}, respectively. 

\section{Preliminaries}\label{Sec3}

\subsection{Whitney's stratification}\label{Sub3.1}

Let $Z\subset\mathbb{R}^n$ be a closed set. An analytical \emph{stratification} of $Z$ is a filtration of $Z$ by closed sets
\[Z=Z_d\supset Z_{d-1}\supset\dots\supset Z_1\supset Z_0,\]
such that $Z_i\backslash Z_{i-1}$ is either empty or an analytical manifold of dimension $i$. Each connected component of $Z_i\backslash Z_{i-1}$ is called a \emph{stratum} of dimension $i$. Thus, $Z$ is the disjoint union of the strata. An analytical \emph{Whitney stratification} is, among other things, a locally finite analytical stratification. That is, given $p\in Z$ there is a neighborhood $U\subset\mathbb{R}^n$ of $p$ such that at most a finite number of strata intersects $U$. A set $Z\subset\mathbb{R}^n$ is \emph{analytic} if there are a finite number of analytical functions $f_1,\dots,f_k\colon\mathbb{R}^n\to\mathbb{R}$, such that
\[Z=\{x\in\mathbb{R}^n\colon f_1(x)=\dots=f_k(x)=0\}.\]

\begin{theorem}[Theorem~$1.2.10$ of \cite{Trot}]\label{T5}
	Every analytic subset of $\mathbb{R}^n$ admits an analytical Whitney stratification.
\end{theorem}

Let $f\colon\mathbb{R}^n\to\mathbb{R}$ be an analytical non-constant function. If $0\in\mathbb{R}$ is a regular value of $f$, then it follows from the Implicit Function Theorem that $f^{-1}(0)\subset\mathbb{R}^n$ is a analytical manifold of codimension $1$. Hence, Theorem~\ref{T5} is stating that if $0$ is not a regular value of $f$, then $f^{-1}(0)$ is yet endowed with some regularity. More precisely, in this case it follows from Theorem~\ref{T5} that
\[f^{-1}(0)=B_1\cup B_2\cup\dots\cup B_n,\]
where the union is disjoint and $B_i$ is an analytical manifold of codimension $i$. Moreover, if we are interested in a particular point $p\in f^{-1}(0)$, then it follows from the locally finite property that we can restrict the domain of $f$ to a neighborhood of $p$ and thus assume that each $B_i$ has at most a finite number os connected components. In particular, we conclude that $f^{-1}(0)$ has zero Lebesgue measure on that neighborhood. For more details in stratification theory, we refer to \cite{Trot}.

\subsection{Bernstein Polynomials}\label{Sub3.2}

Let $f\colon[0,1]\to\mathbb{R}$ be a continuous function. The \emph{Bernstein polynomial} of degree $n$ associated to $f$ is given by,
\begin{equation}\label{8}
	B_n^f(x)=\sum_{k=0}^{n}f\left(\frac{k}{n}\right)\binom{n}{k}x^k(1-x)^{n-k},
\end{equation}
where,
	\[\binom{n}{k}=\frac{n!}{k!\,(n-k)!}.\]
One of the properties of the Bernstein polynomials is that $B_n\to f$ uniformly in $[0,1]$. In fact, the original proof of the Weierstrass Approximation Theorem is precisely given by the proof that $B_n\to f$ uniformly. See \cite[Section $1.1$]{Lorentz}. Opening the binomials in \eqref{8} we obtain,
\begin{equation}\label{14}
	B_n^f(x)=\sum_{k=0}^{n}f\left(\frac{k}{n}\right)\binom{n}{k}\left[\sum_{i=0}^{n-k}(-1)^i\binom{n-k}{i}\right]x^{k+i}.
\end{equation}
Replacing the change of index $i\mapsto i-k$ in \eqref{14} we obtain,
\begin{equation}\label{15}
	B_n^f(x)=\sum_{k=0}^{n}f\left(\frac{k}{n}\right)\binom{n}{k}\left[\sum_{i=k}^{n}(-1)^{i-k}\binom{n-k}{i-k}\right]x^{i}.
\end{equation}
Finally, observe that $\binom{n}{k}\binom{n-k}{i-k}=\binom{n}{i}\binom{i}{k}$. Replacing this at \eqref {15} we obtain,
\begin{equation}\label{16}
	B_n^f(x)=\sum_{k=0}^{n}f\left(\frac{k}{n}\right)\left[\sum_{i=k}^{n}(-1)^{i-k}\binom{n}{i}\binom{i}{k}\right]x^{i}.
\end{equation}
Hence, if we write
	\[B_n^f(x)=a_n(n)x^n+a_{n-1}(n)x^{n-1}+\dots+a_1(n)x+a_0(n),\]
then it follows from \eqref{16} that
\begin{equation}\label{17}
	a_j(n)=\sum_{k=0}^{j}f\left(\frac{k}{n}\right)(-1)^{j-k}\binom{n}{j}\binom{j}{k},
\end{equation}
for every $j\in\{0,\dots,n\}$ and $n\in\mathbb{N}$. In the following result we prove that if $f\equiv0$ in a neighborhood of $0$, then every coefficient $a_j$ is eventually zero.

\begin{proposition}\label{P1}
	Let $f\colon[0,1]\to\mathbb{R}$ be a continuous function. If there is $\varepsilon>0$ such that $f(x)=0$ for every $x\in[0,\varepsilon)$, then for every $j\in\mathbb{N}$ there is $n_j\in\mathbb{N}$ such that $a_j(n)=0$, for every $n\geqslant n_j$.
\end{proposition}

\begin{proof} It follows from \eqref{17} that we need only to consider $n_j$ large enough such that $j/n_j<\varepsilon$. \end{proof}

In the two-dimensional case the Weierstrass Approximation Theorem follows similarly by a generalization of the Bernstein polynomials. More precisely, let $R=[0,1]^2\subset\mathbb{R}^2$ and consider a continuous map $F\colon R\to\mathbb{R}$. In this case, the Bernstein polynomial associated to $F$ can be generalized by,
\begin{equation}\label{18}
	B_{m,n}^F(x,y)=\sum_{r=0}^{m}\sum_{s=0}^{n}F\left(\frac{r}{m},\frac{s}{n}\right)\binom{m}{r}\binom{n}{s}x^ry^s(1-x)^{m-r}(1-y)^{n-s}.
\end{equation}
Similarly to the one-dimensional case, we have that $B_{m,n}\to F$ uniformly in $R$. See \cite{HilSch1933}. Moreover, similarly to the previous case, observe that we can write \eqref{18} as,
\begin{equation}\label{19}
	B_{m,n}^F(x,y)=\sum_{r=0}^{m}\sum_{s=0}^{n}F\left(\frac{r}{m},\frac{s}{n}\right)\left[\sum_{k_1=r}^{m}\sum_{k_2=s}^{n}(-1)^{k_1+k_2-r-s}\binom{m}{k_1}\binom{k_1}{r}\binom{n}{k_2}\binom{k_2}{s}\right]x^{k_1}y^{k_2}.
\end{equation}	
Thus, it follows from \eqref{19} that the coefficient $a_{ij}(m,n)$ of the monomial $x^iy^j$ of $B_{m,n}^F$ is given by,
\begin{equation}\label{20}
	a_{ij}(m,n)=\sum_{r=0}^{i}\sum_{s=0}^{j}F\left(\frac{r}{m},\frac{s}{n}\right)(-1)^{i+j-r-s}\binom{m}{i}\binom{i}{r}\binom{n}{j}\binom{j}{s}.
\end{equation}
Hence, it follows that if $F\equiv0$ in a neighborhood of the origin, then every coefficient $a_{ij}$ is eventually zero. More precisely, we have the following proposition.

\begin{proposition}\label{P2}
	Let $F\colon R\to\mathbb{R}$ be a continuous map. If there is $\varepsilon>0$ such that $F(x,y)=0$ for every $(x,y)\in[0,\varepsilon)^2$, then for every $(i,j)\in\mathbb{N}^2$ there is $n_{ij}\in\mathbb{N}$ such that $a_{ij}(m,n)=0$, for every $m$, $n\geqslant n_j$.
\end{proposition}

For more details about the Bernstein polynomials, we refer to \cite{Lorentz}.

\subsection{Approximating a planar smooth function by a polynomial}\label{Sub3.3}

\begin{proposition}\label{P3}
	Let $\varphi\colon\mathbb{R}^2\to\mathbb{R}$ be function function of class $C^2$ and $B\subset\mathbb{R}^2$ a closed ball centered at the origin. Then given $\varepsilon>0$, there are polynomials $p$, $q\colon\mathbb{R}^2\to\mathbb{R}$ such that
		\[|\varphi(x,y)-p(x,y)|<\varepsilon, \quad \left|\frac{\partial\varphi}{\partial x}(x,y)-\frac{\partial p}{\partial x}(x,y)\right|<\varepsilon, \quad \left|\frac{\partial^2\varphi}{\partial x^2}(x,y)-\frac{\partial^2 p}{\partial x^2}(x,y)\right|<\varepsilon,\]
	and
		\[|\varphi(x,y)-q(x,y)|<\varepsilon, \quad \left|\frac{\partial\varphi}{\partial y}(x,y)-\frac{\partial q}{\partial y}(x,y)\right|<\varepsilon, \quad \left|\frac{\partial^2\varphi}{\partial y^2}(x,y)-\frac{\partial^2 q}{\partial y^2}(x,y)\right|<\varepsilon,\]		
	for all $(x,y)\in B$.
\end{proposition}

\begin{proof} Let $\psi_1$, $\psi_2\colon\mathbb{R}\to\mathbb{R}$ be given by $\psi_1(y)=\varphi(0,y)$, $\psi_2(y)=\frac{\partial\varphi}{\partial x}(0,y)$. Given $\delta_1>0$ and $\delta_2>0$, it follows from the Weierstrass Approximation Theorem that there are polynomials $r_1$, $r_2\colon\mathbb{R}\to\mathbb{R}$ such that
	\[|\psi_1(y)-r_1(y)|<\delta_1, \quad |\psi_2(y)-r_2(y)|<\delta_2,\]
for all $y\in\mathbb{R}$ such that $(0,y)\in B$. Given $\delta_3>0$, it also follows from the Weierstrass Approximation Theorem that there is a polynomial $q_2\colon\mathbb{R}^2\to\mathbb{R}$ such that
\begin{equation}\label{10}
	\left|\frac{\partial^2\varphi}{\partial x^2}(x,y)-q_2(x,y)\right|<\delta_3,
\end{equation}
for all $(x,y)\in B$. Therefore, let $q_1\colon\mathbb{R}^2\to\mathbb{R}^2$ be the polynomial given by,
\begin{equation}\label{2}
	q_1(x,y)=r_2(y)+\int_{0}^{x}q_2(t,y)\;dt.
\end{equation}
For all $(x,y)\in B$, observe that
\begin{equation}\label{11}
	\begin{array}{rl}
		\displaystyle \left|\frac{\partial\varphi}{\partial x}(x,y)-q_1(x,y)\right| &\displaystyle= \left|\left(\frac{\partial\varphi}{\partial x}(0,y)+\int_{0}^{x}\frac{\partial^2\varphi}{\partial x^2}(t,y)\;dt\right)-\left(r_2(y)+\int_{0}^{x}q_2(t,y)\;dt\right)\right| \vspace{0.2cm} \\
		&\displaystyle\leqslant |\psi_2(y)-r_2(y)|+\int_{0}^{x}\left|\frac{\partial^2\varphi}{\partial x^2}(t,y)\;dt-q_2(t,y)\right|dt \vspace{0.2cm} \\
		&\displaystyle\leqslant \delta_2+\int_{0}^{x}\delta_3\;dt \leqslant\delta_2+|x|\delta_3.
	\end{array}
\end{equation}
Since $B$ is bounded, it follows that $|x|$ is bounded. Therefore, giving $\delta_4>0$, it follows that there are $\delta_2>0$ and $\delta_3>0$ small enough such that,
\begin{equation}\label{9}
	\left|\frac{\partial\varphi}{\partial x}(x,y)-q_1(x,y)\right|<\delta_4,
\end{equation}
for all $(x,y)\in B$. Let now $p\colon\mathbb{R}\to\mathbb{R}$ be the polynomial given by,
\begin{equation}\label{13}
	p(x,y)=r_1(y)+\int_{0}^{x}q_1(t,y)\;dt.
\end{equation}
It follows similarly to \eqref{11} that,
	\[|\varphi(x,y)-p(x,y)|\leqslant|\psi_1(y)-r_1(y)|+\int_{0}^{x}\left|\frac{\partial\varphi}{\partial x}(t,y)-q_1(t,y)\right|dt\leqslant\delta_1+|x|\delta_4.\]
Hence, there are $\delta_1>0$ and $\delta_4>0$ small enough such that,
\begin{equation}\label{12}
	|\varphi(x,y)-p(x,y)|<\varepsilon.
\end{equation}
It follows from \eqref{13} and \eqref{2} that,
	\[\frac{\partial p}{\partial x}=q_1, \quad \frac{\partial^2p}{\partial x^2}=\frac{\partial q_1}{\partial x}=q_2.\]
Hence, taking $\delta_3<\varepsilon$ and $\delta_4<\varepsilon$, it follows from \eqref{10}, \eqref{9} and \eqref{12} that $p$ has the desired properties. Similarly, one can obtain $q$. \end{proof}

\section{Proof of Theorem~\ref{T1}}\label{Sec4}

\begin{proof} Let $\gamma(t)$ be the parametrization of $\gamma$, given by the flow of $X$, and let $T>0$ be its period. Reversing the time variable if necessary, we can assume that $\gamma$ is stable. It follows from \cite[p. 124]{And1971} that there is a neighborhood $G\subset\mathbb{R}^2$ of $\gamma$ and a function $F\colon G\to\mathbb{R}$ of class $C^2$ such that
\begin{equation}\label{4}
	F(\gamma(t))=0, \quad \frac{\partial F}{\partial x}(\gamma(t))^2+\frac{\partial F}{\partial y}(\gamma(t))^2>0,
\end{equation}
for all $t\in[0,T]$. Using bump-functions we can assume that $F$ is defined on the whole plane. Let $B\subset\mathbb{R}$ be a closed ball such that $\gamma\subset\text{Int}(B)$. It follows from Proposition~\ref{P3} that for each $n\in\mathbb{N}$ there are polynomials $p_n$, $q_n\colon\mathbb{R}^2\to\mathbb{R}$ satisfying
	\[|F(x,y)-p_n(x,y)|<\frac{1}{n}, \quad \left|\frac{\partial F}{\partial x}(x,y)-\frac{\partial p_n}{\partial x}(x,y)\right|<\frac{1}{n}, \quad \left|\frac{\partial^2 F}{\partial x^2}(x,y)-\frac{\partial^2 p_n}{\partial x^2}(x,y)\right|<\frac{1}{n},\]
and
	\[|F(x,y)-q_n(x,y)|<\frac{1}{n}, \quad \left|\frac{\partial F}{\partial y}(x,y)-\frac{\partial q_n}{\partial y}(x,y)\right|<\frac{1}{n}, \quad \left|\frac{\partial^2 F}{\partial y^2}(x,y)-\frac{\partial^2 q_n}{\partial y^2}(x,y)\right|<\frac{1}{n},\]
for all $(x,y)\in B$. Let $X_{\lambda,n}=(P_{\lambda,n},Q_{\lambda,n})$ be the two-parameter family of planar polynomial vector field given by,
	\[P_{\lambda,n}(x,y)=P(x,y)+\lambda p_n(x,y)\frac{\partial p_n}{\partial x}(x,y), \quad Q_{\lambda,n}(x,y)=Q(x,y)+\lambda q_n(x,y)\frac{\partial q_n}{\partial y}(x,y),\]
for $\lambda>0$ small. Since $p_n\to F$, $q_n\to F$ and
	\[\frac{\partial p_n}{\partial x}\to\frac{\partial F}{\partial x}, \quad \frac{\partial q_n}{\partial y}\to\frac{\partial F}{\partial y},\]
uniformly in $B$ as $n\to\infty$, we define $X_{\lambda,\infty}=(P_{\lambda,\infty},Q_{\lambda,\infty})$ as
\begin{equation}\label{21}
	P_{\lambda,\infty}=P(x,y)+\lambda F(x,y)\frac{\partial F}{\partial x}(x,y), \quad Q_{\lambda,\infty}=Q(x,y)+\lambda F(x,y)\frac{\partial F}{\partial y}(x,y).
\end{equation}
Since $\gamma$ is not semi-stable, it follows that the perturbation $\gamma_{\lambda,n}$ of $\gamma$ is well defined and still a periodic orbit. Observe that,
\begin{equation}\label{6}
	\int_{\gamma_{\lambda,n}}\frac{\partial P_{\lambda,n}}{\partial x}+\frac{\partial Q_{\lambda,n}}{\partial y}=\int_{\gamma_{\lambda,n}}\frac{\partial P}{\partial x}+\frac{\partial Q}{\partial y}+\lambda\int_{\gamma_{\lambda,n}}\left(\frac{\partial p_n}{\partial x}\right)^2+\left(\frac{\partial q_n}{\partial y}\right)^2+\lambda\int_{\gamma_{\lambda,n}} p_n\frac{\partial^2p_n}{\partial x^2}+q_n\frac{\partial^2 q_n}{\partial y^2}.
\end{equation}
In what follows we fix $\lambda>0$. Since $\gamma_{\lambda,0}\to\gamma$ uniformly as $n\to\infty$, we claim that,
	\[\lim\limits_{n\to\infty}\int_{\gamma_{\lambda,n}}\frac{\partial P}{\partial x}+\frac{\partial Q}{\partial y}=\int_{\gamma}\frac{\partial P}{\partial x}+\frac{\partial Q}{\partial y}=0.\]
Indeed, let $T_{\lambda,n}$ be the period of $\gamma_{\lambda,n}$. Since $F(\gamma)=0$, it follows that $\gamma$ is also a periodic orbit of $X_{\lambda,\infty}$ and thus $T_{\lambda,n}\to T$ as $n\to\infty$. Hence,
	\[\begin{array}{rl}
		\displaystyle \lim\limits_{n\to\infty}\int_{\gamma_{\lambda,n}}\frac{\partial P}{\partial x}+\frac{\partial Q}{\partial y} &\displaystyle= \lim\limits_{n\to\infty}\int_{0}^{T_{\lambda,n}}\left(\frac{\partial P}{\partial x}+\frac{\partial Q}{\partial x}\right)(\gamma_{\lambda,n}(t))\;dt \vspace{0.2cm} \\
		&\displaystyle= \lim\limits_{n\to\infty}\int_{0}^{T}\left(\frac{\partial P}{\partial x}+\frac{\partial Q}{\partial x}\right)(\gamma_{\lambda,n}(t))\;dt+\lim\limits_{n\to0}\int_{T}^{T_{\lambda,n}}\left(\frac{\partial P}{\partial x}+\frac{\partial Q}{\partial x}\right)(\gamma_{\lambda,n}(t))\;dt\vspace{0.2cm} \\
		&\displaystyle=\int_{0}^{T}\left(\frac{\partial P}{\partial x}+\frac{\partial Q}{\partial x}\right)(\gamma(t))\;dt=\int_{\gamma}\frac{\partial P}{\partial x}+\frac{\partial Q}{\partial x}=0,
	\end{array}\]
with the last equality following from the fact that $\gamma$ is a non-hyperbolic limit cycle of $X$ and the equality before that following from the uniform convergence $\gamma_{\lambda,n}\to\gamma$. Similarly, since $p_n\to F$, $q_n\to F$, $\frac{\partial^2p_n}{\partial x^2}\to\frac{\partial^2F}{\partial x^2}$ and $\frac{\partial^2q_n}{\partial y^2}\to\frac{\partial^2F}{\partial y^2}$ uniformly in $B$ as $n\to\infty$, it follows that
	\[\lim\limits_{n\to\infty}\int_{\gamma_{\lambda,n}} p_n\frac{\partial^2p_n}{\partial x^2}+q_n\frac{\partial^2 q_n}{\partial y^2}=\int_{\gamma}F\frac{\partial^2 F}{\partial x^2}+F\frac{\partial^2 F}{\partial y^2}=0,\]
with the last equality following from $F(\gamma)=0$. Moreover, it also follows from \eqref{4} that,
	\[\lim\limits_{n\to\infty}\int_{\gamma_{\lambda,n}}\left(\frac{\partial p_n}{\partial x}\right)^2+\left(\frac{\partial q_n}{\partial y}\right)^2=\int_{\gamma}\left(\frac{\partial F}{\partial x}\right)^2+\left(\frac{\partial F}{\partial y}\right)^2>0.\]
Therefore, given $\lambda>0$ small, it follows from \eqref{6} that
\begin{equation}\label{7}
	\int_{\gamma_{\lambda,n}}\frac{\partial P_{\lambda,n}}{\partial x}+\frac{\partial Q_{\lambda,n}}{\partial y}>0,
\end{equation}
for $n\in\mathbb{N}$ big enough. Hence, we conclude that for every $\lambda>0$ there is $n_\lambda\in\mathbb{N}$ such that if $n\geqslant n_\lambda$, then $\gamma_{\lambda,n}$ is a unstable hyperbolic limit cycle of $X_{\lambda,n}$. We claim that $X_{\lambda,n}$ has at least two others limit cycles near $\gamma_{\lambda,n}$. Indeed, let $\ell$ be a transversal section of $\gamma$, endowed with a metric $s$ such that $s=0$ at $\gamma$, $s<0$ in the bounded component of $\mathbb{R}^2\backslash\gamma$ and $s>0$ in the unbounded component of $\mathbb{R}^2\backslash\gamma$. Let $\pi\colon\ell\to\ell$ be the first return map associated to $\gamma$. Let $s_1$, $s_2\in\ell$ be such that $s_1<0<s_2$. Since $\gamma$ is a stable limit cycle of $X$, it follows that $s_1<\pi(s_1)<0<\pi(s_2)<s_2$. See Figure~\ref{Fig1}.
\begin{figure}[ht]
	\begin{center}
		\begin{overpic}[height=5cm]{Fig1.eps} 
		%\begin{overpic}[height=5cm,grid,tics=5]{Fig1.eps} 
			\put(36,37){$s_1$}
			\put(35.5,25.5){$\pi(s_1)$}
			\put(83,69){$s_2$}
			\put(94,60.5){$\pi(s_2)$}
			\put(87,25){$\gamma$}
		\end{overpic}
	\end{center}
	\caption{Illustration of $s_1$, $s_2$, $\overline{s_1}$ and $\overline{s_2}$.}\label{Fig1}
\end{figure}
Let $\gamma_i$ be the orbit of $X$ from $s_i$ to $\pi(s_i)$, $\ell_i\subset\ell$ be the segment between $s_i$ and $\pi(s_i)$ and $C_i=\gamma_i\cup\ell_i$, $i\in\{1,2\}$. Observe that $C_1$ and $C_2$ bounds a region $A$ invariant by the flow of $X$. Observe that $X_{\lambda,\infty}\to X$ in the $C^1$-topology. Hence, in $B$, we have also $X_{\lambda,n}\to X$ in the $C^1$-topology. Therefore, it follows from the continuous dependence of initial conditions (see \cite[Theorem $8$, p. $25$]{And1971}) that for $\lambda>0$ small enough and $n\geqslant n_\lambda$, the orbit of $X_{\lambda,n}$ through $s_1$ and $s_2$ also bounds a region $A_{\lambda,n}$, with $\gamma_{\lambda,n}\subset\text{Int}(A_{\lambda,n})$, invariant by the flow of $X_{\lambda,n}$. Since $\gamma_{\lambda,n}$ is now a unstable limit cycle, it follows from the Poincare-Bendixson Theorem that $X_{\lambda,n}$ have at least two others limit cycles in $A_{\lambda,n}$. Since the original invariant region $A$ can be taken arbitrarily small, it follows that these two limit cycles are arbitrarily close to $\gamma_{\lambda,n}$. 

We now proof that we can take $X_{\lambda,n}$ arbitrarily close to $X$ in the compact-open topology. Indeed, first we recall that $F\colon G\to\mathbb{R}$ is initially defined in a neighborhood $G\subset\mathbb{R}^2$ of $\gamma$. Since $F(\gamma)=0$, it follows that given $\varepsilon>0$, we can restrict $G$ such that $|F(x,y)|<\varepsilon$, for every $(x,y)\in G$. Then we can use bump-functions to extend $F$ to the whole plane. Hence, it follows that $|F(x,y)|<\varepsilon$ for every $(x,y)\in\mathbb{R}^2$. Therefore we have it follows from \eqref{21} that $||X-X_{\lambda,\infty}||<\lambda\varepsilon||\nabla F||$
for every $(x,y)\in B$, where $\nabla F$ denotes the gradient vector of $F$. Since $\nabla F$ is bounded and $\lambda>0$ and $\varepsilon>0$ can be taken arbitrarily small, it follows that $X_{\lambda,0}$ is arbitrarily close to $X$ in $\mathbb{R}^2$. The proof now follows from the fact that $X_{\lambda,n}\to X_{\lambda,\infty}$ in $B$. 

Finally, it follows from the proof of Proposition~\ref{P3} that the polynomials $p_n$ and $q_n$ are given by linear combinations of primitives of Bernstein polynomials. Except by translation followed by a linear change of coordinates, observe that we can suppose $\gamma\subset R$, where $R=[0,1]^2$. Hence, it follows from Propositions~\ref{P1} and \ref{P2} that for every $k\in\mathbb{N}$, there is $n_k\in\mathbb{N}$ such that $p_n$, $q_n\in o(||(x,y)||^k)$, for every $n\geqslant n_k$. Hence, $o(X-X_{\lambda,n})$ is arbitrarily high. \end{proof}

\section{Proof of Theorem~\ref{T2}}\label{Sec5}

\begin{proof} Let $\gamma$ be a non-hyperbolic limit cycle for $X\in\mathcal{X}_n$. Let also $N\subset\mathfrak{X}_n$ be a small enough neighborhood of $X$ such that the displacement map $D\colon N\times\ell\to\mathbb{R}$ is well defined, where $\ell$ is a transversal section of $\gamma$. Let $d$ be the degree of $\gamma$. That is, $d\geqslant2$ is the first integer such that,
	\[\frac{\partial^d D}{\partial x^d}(X,0)\neq0.\]
It follows from the Weierstrass Preparation Theorem (see \cite[Chapter $4$]{GolGui1973}) that $D(Y,x)=U(Y,x)P(Y,x)$, where $U$ is a strictly positive analytical function and
	\[P(Y,x)=x^d+a_{n-1}(Y)x^{d-1}+\dots+a_1(Y)x+a_0(Y),\]
where $a_i$ is analytical and $a_i(X)=0$, $i\in\{1,\dots,n-1\}$. Given $\varepsilon>0$, let $X_\lambda$ be the $1$-parameter family given by
\begin{equation}\label{1}
	P_\lambda(x,y)=P(x,y)-\lambda\varepsilon Q(x,y) \quad Q_\lambda(x,y)=Q(x,y)+\lambda\varepsilon P(x,y),
\end{equation}
with $\lambda\in(-1,1)$. Observe that if $\varepsilon>0$ is small enough, then $X_\lambda\in N$, for all $\lambda\in(-1,1)$. Observe also that $X_0=X$. Let $\gamma(t)$ be a parametrization of $\gamma$ such that $\gamma(0)=p$, where $\{p\}=\ell\cap\gamma$. Let $T>0$ be the period of $\gamma$. It follows from \cite[Lemma $2$]{Perko1992} that,
\begin{equation}\label{5}
	\frac{\partial D}{\partial\lambda}(X_0,0)=C\int_{0}^{T}\left(e^{-\int_{0}^{t}Div(\gamma(s))\;ds}\right)X_0(\gamma(t))\land\dfrac{\partial X_0}{\partial \lambda}(\gamma(t))\;dt,
\end{equation}
with $C\in\mathbb{R}\backslash\{0\}$ and $(x_1,x_2)\land(y_1,y_2)=x_1y_2-x_2y_1$. Observe that, 
	\[a_0(Y)=\frac{D(Y,0)}{U(Y,0)},\]
for any $Y\in N$. To simplify the notation, let $V(Y,x)=U(Y,x)^{-1}$. Hence,
	\[\frac{\partial a_0}{\partial\lambda}(X_0)=\frac{\partial V}{\partial\lambda}(X_0,0)\underbrace{D(X_0,0)}_{0}+V(X_0,0)\frac{\partial D}{\partial\lambda}(X_0,0).\]
Thus, it follows from \eqref{5} that,
\begin{equation}\label{3}
	\frac{\partial a_0}{\partial\lambda}(X_0)=V(X_0,0)C\int_{0}^{T}\left(e^{-\int_{0}^{t}Div(\gamma(s))\;ds}\right)X_0(\gamma(t))\land\dfrac{\partial X_0}{\partial \lambda}(\gamma(t))\;dt.
\end{equation}
It follows from \eqref{1} that,
	\[X_0(x,y)\land\dfrac{\partial X_0}{\partial \lambda}(x,y)=\varepsilon\bigl(P(x,y)^2+Q(x,y)^2\bigr).\]
Hence
	\[X_0(\gamma(t))\land\dfrac{\partial X_0}{\partial \lambda}(\gamma(t))>0,\]
for all $t\in[0,T]$. Thus, it follows from \eqref{3} that,
	\[\frac{\partial a_0}{\partial\lambda}(X)\neq0.\]
Therefore, the function $a_0\colon N\to\mathbb{R}$ has a non-zero directional derivative and thus it is not constant. In particular, the polynomial
	\[P(Y,x)=x^d+a_{n-1}(Y)x^{d-1}+\dots+a_1(Y)x+a_0(Y),\]
is not constant in $Y$. From now on, we denote $P(Y)=P(Y,\cdot)$. Let $\Delta\colon\mathbb{R}^{d+1}\to\mathbb{R}$ denote the discriminant of polynomials of degree $d$ (see \cite[Chapter $12$]{GelKapZel1994}). Observe if $\gamma(Y)$ is a non-hyperbolic limit cycle of $Y$, then $\Delta(P(Y))=0$. Since $\Delta\circ P$ is non-constant, it follows from the stratification theory (recall Section~\ref{Sub3.1}) that the set 
	\[\Omega=\{Y\in N\colon \Delta(P(Y))=0\},\]
is given by the disjoint union of analytical manifolds of codimension at least one, each of them having at most a finite number of connected components. In particular, $\Omega$ has zero Lebesgue measure on $N$ and thus we can take $Y$ arbitrarily close to $X$ such that $\Delta(P(Y))\neq0$. Therefore, if $d$ is even, then $P(Y)$ has either zero or at least two real roots. In the former, $Y$ has no limit cycles near $\gamma$. In the latter, $Y$ has at least two hyperbolic limit cycles near $\gamma$. In either case, we have proved that $\gamma$ is structurally unstable. \end{proof}

\begin{remark}\label{Remark1}
	In the context of the proof of Theorem~\ref{T2}, if $d$ is odd, then it follows from \cite[Section $4$]{NickDye} that the following statements hold.
	\begin{enumerate}[label=(\roman*)]
		\item If $\Delta(P(Y))>0$, then the number of real roots of $P$ is congruent to $1$ modulo $4$;
		\item If $\Delta(P(Y))<0$, then the number of real roots of $P$ is congruent to $3$ modulo $4$.
	\end{enumerate}
	Therefore, if there is $Y\in N$ such that $\Delta(P(Y))<0$, then $Y$ has at least three hyperbolic limit cycles near $\gamma$ and thus we conclude that non-hyperbolic limit cycles of odd degree are also structurally unstable in $\mathcal{X}_n$. Hence, by taking $\Phi=\Delta\circ P$ we obtain Question~\ref{Q2}.
\end{remark}

\section*{Acknowledgments}

The author is supported by S\~ao Paulo Research Foundation (FAPESP), grants 2019/10269-3, 2021/01799-9 and 2022/14353-1.

\section*{Declarations}

\noindent\textbf{Competing interests:} On behalf of all authors, the corresponding author states that there is no conflict of interest.

\begin{thebibliography}{99}
	
\bibitem{AlgFueGamGar2018}
{\sc A. Algaba et al}, 
{\it Structural stability of planar quasi-homogeneous vector fields},
Journal of Mathematical Analysis and Applications, \textbf{468} (2018), p. 212-226.
	
\bibitem{AndPon1937}
{\sc A. Andronov and L. Pontrjagin}, 
{\it Systemes Grossiers},
Doklady Akademii Nauk SSSR. \textbf{14} (1937), p 247–250.
	
\bibitem{And1971}
{\sc A. Andronov et al}, 
{\it Theory of bifurcations of dynamic systems on a plane},
Israel Program for Scientific Translations;[available from the US Department of Commerce, National Technical Information Service, Springfield, Va.], 1971.

\bibitem{Art2024}
{\sc J. Artés}, 
{\it Structurally Unstable Quadratic Vector Fields of Codimension Two: Families Possessing One Finite Saddle-Node and a Separatrix Connection},
Qual. Theory Dyn. Syst., \textbf{23} (2024).

\bibitem{ArtKooLli}
{\sc J. Artés, R. Kooij and J. Llibre}, 
{\it Structurally stable quadratic vector fields},
American Mathematical Soc., \textbf{639} (1998).

\bibitem{ArtRezLli}
{\sc J. Artés, A. Rezende and J. Llibre}, 
{\it Structurally unstable quadratic vector fields of codimension one},
Cham: Birkhäuser (2018).

\bibitem{ArtOliRez}
{\sc J. Artés, R. Oliveira and A. Rezende}, 
{\it Structurally Unstable Quadratic Vector Fields of Codimension Two: Families Possessing Either a Cusp Point or Two Finite Saddle-Nodes},
Journal of Dynamics and Differential Equations, \textbf{33} (2021), p. 1779–1821.

\bibitem{ArtMotRez}
{\sc J. Artés, M. Mota and A. Rezende}, 
{\it Structurally unstable quadratic vector fields of codimension two: families possessing a finite	saddle-node and an infinite saddle-node},
Electronic Journal of Qualitative Theory of Differential Equations, \textbf{35} (2021), p. 1-89.	

\bibitem{ArtMotRez}
{\sc J. Artés, M. Mota and A. Rezende}, 
{\it Structurally unstable quadratic vector fields of codimension two: families possessing a finite	saddle-node and an infinite saddle-node},
Electronic Journal of Qualitative Theory of Differential Equations, \textbf{35} (2021), p. 1-89.	

\bibitem{BuzRobTei2021}
{\sc C. Buzzi, L. Roberto and M. Teixeira}, 
{\it Stability conditions for reversible and partially integrable systems},
Port. Math. \textbf{78} (2021), p. 43-63.

\bibitem{BuzRodTei2022}
{\sc C. Buzzi, A. Rodero and M. Teixeira}, 
{\it Stability conditions for refractive partially integrable piecewise smooth vector fields},
Physica D: Nonlinear Phenomena \textbf{440} (2022), p. 133462.

\bibitem{Cai1979}
{\sc S. Cai}, 
{\it A note on: ``Classification of generic quadratic vector fields with no limit cycles''},
JZhejiang Daxue Xuebao, \textbf{4} (1979), p. 105-113.

\bibitem{Bag1952}
{\sc F. DeBaggis}, 
{\it Dynamical systems with stable structures},
Contributions to the theory of nonlinear oscillations (1952), p. 37-59.

\bibitem{DumLliArt2006}
{\sc F. Dumortier, J. Llibre and J. C. Art\'es}, 
{\it Qualitative theory of planar differential systems},
Universitext, Springer-Verlag, Berlim, 2006.
	
\bibitem{DumSha1990}
{\sc F. Dumortier and D. Shafer}, 
{\it Restrictions on the equivalence homeomorphism in stability of polynomial vector fields},
Journal of the London Mathematical Society, \textbf{41} (1990), p. 100-108.

\bibitem{GasGiaTor2007}
{\sc A. Gasull, H. Giacomini and J. Torregrosa}, 
{\it Explicit non-algebraic limit cycles for polynomial systems},
Journal of Computational and Applied Mathematics, \textbf{200} (2007).

\bibitem{GelKapZel1994}
{\sc I. Gelfand, M. Kapranov and A. Zelevinsky}, 
{\it Discriminants, Resultants, and Multidimensional Determinants},
Birkhäuser Boston, MA, 1994.

\bibitem{GolGui1973}
{\sc M. Golubitsky and V. Guillemin}, 
{\it Stable mappings and their singularities},
Graduate Texts in Mathematics 14, (1973).

\bibitem{HilSch1933}
{\sc T. Hildebrandt and I. Schoenberg}, 
{\it On Linear Functional Operations and the Moment Problem for a Finite Interval in One or Several Dimensions},
Annals of Mathematics, \textbf{34} (1933), p. 317-328.

\bibitem{JarLliSha2005}
{\sc X. Jarque, J. Llibre and D. Shafer}, 
{\it Structural stability of planar polynomial foliations},
Journal of Dynamics and Differential Equations, \textbf{17} (2005), p. 573-587.

\bibitem{Kot1982}
{\sc J. Kotus, M. Krych and Z. Nitecki}, 
{\it Global structural stability of flows on open surfaces},
Memoirs of the American Mathematical Society, \textbf{37} (1982).

\bibitem{Lorentz}
{\sc G. Lorentz}, 
{\it Bernstein Polynomials},
American Mathematical Society, Chelsea Publishing (1997).
	
\bibitem{Mar1961}
{\sc L. Markus}, 
{\it Structurally stable differential systems},
Annals of Mathematics, \textbf{73} (1961),  p. 1-19.

\bibitem{NickDye}
{\sc R. Nickalls and R. Dye}, 
{\it The Geometry of the Discriminant of a Polynomial},
The Mathematical Gazette JSTOR, \textbf{80} (1996), pp 279-285.

\bibitem{OliZha2014}
{\sc R. Oliveira and Y. Zhao}, 
{\it Structural Stability of Planar Quasihomogeneous Vector Fields},
Qual. Theory Dyn. Syst., \textbf{13} (2014),  p. 39-72.
	
\bibitem{PeiPei1959}
{\sc M. C. Peixoto and M. M. Peixoto}, 
{\it Structural stability in the plane with enlarged boundary conditions},
An. Acad. Brasil. Ci, \textbf{31} (1959), p. 135-160.
	
\bibitem{Pei1959}
{\sc M. M. Peixoto}, 
{\it On Structural Stability},
Annals of Mathematics, \textbf{69} (1959), p. 199-222.
	
\bibitem{Pei1962}
{\sc M. M. Peixoto}, 
{\it Structural stability on two-dimensional manifolds},
Topology, \textbf{1} (1962), p. 101-120.
	
\bibitem{Pei1973}
{\sc M. M. Peixoto}, 
{\it On the classification of flows on 2-manifolds},
Dynamical systems, 1973, pp. 389-419.
	
\bibitem{Perko2001}
{\sc L. Perko}, 
{\it Differential equations and dynamical systems},
vol. 7 of Texts in Applied Mathematics, Springer-Verlag, New York, third ed, 2001.

\bibitem{Perko1992}
{\sc L. Perko}, 
{\it Bifurcation of Limit Cycles: Geometric Theory},
Proceedings of the American Mathematical Society, \textbf{114} (1992), p. 225-236.

\bibitem{PesSot2012}
{\sc C. Pessoa and J. Sotomayor}, 
{\it Stable piecewise polynomial vector fields},
Electronic Journal of Differential Equations \textbf{2012} (2012).

\bibitem{San1977}
{\sc T. Santos}, 
{\it Classification of generic quadratic vector fields with no limit cycles},
Geometry and topology (1977), p. 605-640.
	
\bibitem{Sha1987}
{\sc D. Shafer}, 
{\it Structural stability and generic properties of planar polynomial vector fields},
Rev. Mat. Iberoamericana, \textbf{3} (1987), p. 337-355.
	
\bibitem{Sha1990}
{\sc D. Shafer}, 
{\it Structure and stability of gradient polynomial vector fields},
Journal of the London Mathematical Society, \textbf{1} (1990), p. 109-121.

\bibitem{Sot2020}
{\sc J. Sotomayor}, 
{\it On Maurício M. Peixoto and the arrival of structural stability to Rio de Janeiro, 1955},
Anais da Academia Brasileira de Ciencias, \textbf{92} (2020).
	
\bibitem{Soto1974}
{\sc J. Sotomayor}, 
{\it Generic one-parameter families of vector fields on two-dimensional manifolds},
Publications Mathématiques de l'Institut des Hautes Études Scientifiques, \textbf{43} (1974), p. 5-46.
	
\bibitem{Soto1985}
{\sc J. Sotomayor}, 
{\it Stable planar polynomial vector fields},
Revista Matemática Iberoamericana, \textbf{1:2} (1985), p. 15-23.
	
\bibitem{SotGarMel2020}
{\sc J. Sotomayor, R. Gargia and L. Mello}, 
{\it Mauricio Matos Peixoto},
Revista Matemática Universitária, \textbf{1} (2020), p. 1-22.
	
\bibitem{Tei1977}
{\sc M. Teixeira}, 
{\it Generic bifurcations in manifolds with boundary},
Journal of Differential Equations, \textbf{25} (1977), p. 65-89.
	
\bibitem{Trot}
{\sc D. Trotman}, 
{\it Stratification theory},
Handbook of geometry and topology of singularities I. Springer, Cham, 2020. 243-273.
	
\bibitem{Vel}
{\sc E. Velasco}, 
{\it Generic Properties of Polynomial Vector Fields at Infinity},
Transactions of the American Mathematical Society, \textbf{143} (1969), p. 201-222.
	
\end{thebibliography}

\end{document}
