\section{\dataset{} Task}
% Given a base passage ($x_b = x_{b_1}, x_{b_2}.. x_{b_N}$) consisting of $N$ sentences and the target passage ($x_t=x_{t_1}, x_{t_2}...x_{t_M}$) consisting of $M$ sentences, signifying different versions of the same article, we aim to generate a set of (discriminating question ($q$), an extractive answer to the question according to the base passage ($a_{{b}})$, and an extractive answer to the question according to the target passage ($a_{{t}})$), where $a_{{b}}!=a_{{t}}$. We limit that $a_b$ to be substring of $x_b$ and $a_t$ to be a substring of $x_t$. 
% \pj{We might not always have a base-answer. Maybe we can talk only with reference to xt and at.}

% These discriminating questions have several requirements. They should be answerable by at least one of the two passages, \pj{answerable by one but not the other. The current definition does not explain this.} seeking factual information and stand-alone~\cite{choi-etal-2021-decontextualization} (i.e., interpretable when presented by itself without the passage). The generated ($q$, $a_b$, $a_t$) set do not cover comprehensive factual changes between the pair, as some difference can be summarized succinctly in natural question answer pairs.\pj{Not sure I follow this}\ec{we need an example} Instead, annotators only mark that there is no factual change when they are fairly confident that there is no new information about the answer span in the target passage. \pj{base passage}

% \pj{I'm a bit confused by the above definition. Added an alternative in the comments here.}
The goal of \dataset{} is to capture how two similar passages differ from each other using question-answer pairs. 
In particular, given a base passage $x_b$ and a target passage $x_t$, where $x_t$ and $x_b$ are different versions of the same article, we aim  to generate discriminating questions $Q_t$. For each $q_t \in Q_t$, the information to deduce the corresponding answer span $a_t \in A_t$ must be missing in $x_b$. To limit the scope of possible questions, each answer span $a_t \in A_t$ must be a substring of $x_t$. While $a_t$ could also be a substring of $x_b$, $x_b$ must be missing the required information to deduce $a_t$ is the correct answer. Alternatively, there could be a corresponding answer span $a_b$, which is the answer resulting from answering $q_t$ with $x_b$. Note that we consider paraphases of $a_t$, such as lexicalizing numbers and using alternate entity names, as equivalent answers. 
%either be unanswerable with the base passage or produce a different answer than $a_t$.

This discriminating question has certain additional requirements: it should be  seeking factual information and stand-alone~\cite{choi-etal-2021-decontextualization} (i.e., interpretable when presented by itself without the passage). It is possible that no such discriminating question can be written. The annotators only mark that there is no factual change when they are fairly confident that there is no new information about the answer span in the target passage.

Consider the following example:

% We provide an example instance below:
\begin{itemize}
 \item $x_b $ = John Doe won two gold medals at the Olympics in 2012.
 \item $x_t $ = John Doe won a gold medal at the Olympics in 2012.
\end{itemize}

% The space of these questions is possibly very large and can rely on very minute differences in wording. In order to narrow the space of possible questions, we focus on whether or not there’s a difference between the two passages given a noun phrase that would serve as the answer to the question. For instance, consider the following two sentences:

%Now, recall our goal is to write a question that contains new information in the first passage that is not present in the second passage. Which answer spans could serve as good answers for discriminating questions? Are there any negative answer spans in this pair of passages?


Annotators are informed that the goal of the process is to collect \textit{disambiguated} and \textit{information-seeking} queries that can be answered with one passage but not with the other. By \textit{disambiguated} queries, we mean queries that refer to roughly a single answer without any context. For instance, ``Who won two gold medals in the 2012 Olympics?'' could refer to several different people, and questions of the form ``How many medals did he win in the 2012 Olympics?'' are not answerable at all without the presence of the John Doe passage.

Information-seeking queries are ones where the questioner would not need to know the answer in advance for the question to make sense. This is related to the original goals of Natural Questions \citep{kwiatkowski-etal-2019-natural} and corresponds to the \emph{Cranfield}-style questions described by \citet{rodriguez-boyd-graber-2021-evaluation}. As an example, ``What did Al Capone's mother do for a living?'' seems like an information-seeking query. On the other hand, ``Which Italian-American gangster's mother was a seamstress?'' does not: why would the questioner assume that such a person even exists unless they already knew the answer?
%
We describe the annotation process to acquire such discriminating question set in the next section. 





%Where \textsc{ans\_base} is the answer to the   These edits may be just stylistic or may refer to factual changes, and they are collected so that they differ in between five and twenty words.


% Each example includes a non exhaustive list of spans in the target passage together with expert annotations with a discriminating question which can be answered with that span.

% We use the term \emph{discriminating question} for two passages, to refer to an information-seeking question that has two non equivalent answers when using these passages as context, or is only answerable for one of the two passages. A discriminating question is then able to tell apart the two passages in a minimal way.


%These should then be answerable out of context of either passage. If such a question is writable, we say that the example has a \textit{factual change} \pj{factual change wrt to the answer span?}. In certain cases, there may be a question in theory but the question is overly \je{Maybe instead we can distinguish between information seeking questions.}difficult to write: we skip those cases. Instead, an annotator will say there is no factual change when they are fairly confident that there is no new information about the noun phrase in the example.

% The goal is to precisely capture how the target passage differs from the base using question-answer pairs. In particular, the task for models and human annotators is to, given the passages and a candidate span, generate a discriminating question if possible, or decide it is not possible.

% that is only answerable correctly given the target passage: the information must be missing from the base passage. To limit the scope of possible questions, each example also has an answer span given. Thus, the question could either be unanswerable with the base passage or produce a different answer.
% because the answer span is not mentioned at all as new information is added to the passage
% Alternatively, it could be that the answer has changed between the two passages. In other words, if the question is answerable with the base passage, the answer should have changed. 



