\section{Motivation}
In the previous section, we described \dataset{} and its annotation procedure. As mentioned, the purpose of \dataset{} is to detect and describe factual changes. In particular, \dataset{} is a rough measurement of a model's ability to automatically construct a database of question-answer pairs that encapsulate the changes. There are many possible formats that could be used as an alternative to summarize factual changes, such as paragraphs, knowledge base triples, or individual claims.

While paragraphs can contain nuance, they lack atomicity. It is thus difficult to tell what exactly changed or otherwise compare two changes to each other. This makes them less useful as a database. 

On the other hand, knowledge base triples are limiting in the types of factual changes that can be described: regardless of the exact setup, the nodes and relations come from some form of fixed vocabulary that may require discarding interesting changes. For instance, changes related to a set of entities, date ranges, various numbers, or abstract information may all be challenging.

Another alternative method would be a list of claims, similar to Vitamin-C \citep{schuster-etal-2021-get}. This method is also atomic and more flexible than knowledge base triples. However, question-answer pairs have a few advantages. First, question-answer pairs are semi-structured information, forming a loose key-value pair. Factual edits may change the answer to an existing question or add information corresponding to an entirely new information, requiring a new question. Conversely, claims are more difficult to relate to each other.

Finally, question-answer pairs are interesting because question answering is interesting. Previous work has seen the use of a database of question-answer pairs as a method to improve question answering performance \citep{lewis-etal-2021-paq}. A good method for automatically creating and updating such a database thus seems quite useful. As factual corpora change over time, we envision constructing such a database to require iterative updates. 
