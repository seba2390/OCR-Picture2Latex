%% Abstract

In many applications of clustering
(for example, ontologies or clusterings of animal or plant species),
hierarchical clusterings are more descriptive
than a flat clustering.
A hierarchical clustering over $n$ \elements
is represented by a rooted binary tree
with $n$ leaves, each corresponding to one \element.
The subtrees rooted at interior nodes capture the clusters.
In this paper, we study active learning
of a hierarchical clustering using only ordinal queries.
An ordinal query consists of a set of three \elements,
and the response to a query reveals the two \elements
(among the three \elements in the query)
which are ``closer'' to each other than to the third one.
We say that \elements \ElS and \ElSP
are closer to each other than \ElSPP
if there exists a cluster containing \ElS and \ElSP,
but not \ElSPP.

When all the query responses are correct,
there is a deterministic algorithm that 
learns the underlying hierarchical clustering
using at most $n \log_2 n$ adaptive ordinal queries.
We generalize this algorithm to be robust in a model in which
each query response is correct independently with probability $p > \Half$,
and adversarially incorrect with probability $1 - p$.
We show that in the presence of noise,
our algorithm outputs the correct hierarchical clustering
with probability at least $1 - \Err$,
using $O(n \log n + n \log(1/\Err))$ adaptive ordinal queries.
For our results, adaptivity is crucial:
we prove that even in the absence of noise,
every non-adaptive algorithm requires $\Omega(n^3)$ ordinal queries
in the worst case.
