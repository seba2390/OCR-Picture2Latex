\documentclass[12pt]{article}
\pdfoutput=1
\usepackage{subfigure}
\usepackage{amssymb,amsmath}
\usepackage{graphicx}
\usepackage{color}
\usepackage[colorlinks=true
,urlcolor=blue
,citecolor=blue
,linkcolor=blue
,pagecolor=blue
,linktocpage=true
,pdfproducer=medialab
]{hyperref}
%\usepackage[a4paper,width=16cm]{geometry}
\usepackage[a4paper]{geometry}
%\linespread{2}

\usepackage{placeins}
\usepackage[utf8]{inputenc}


\makeatletter \renewcommand{\@dotsep}{10000} \makeatother

%\renewcommand{\textfraction}{0.00}


%%%%%%%%%%%%%%%%%%%%%%%%%%%%%%%%%%%%%%%%%%%%%%%%
\def\te{\tilde e}
\def\tl{\tilde l}
\def\tu{\tilde u}
\def\ts{\tilde s}
\def\tb{\tilde b}
\def\tf{\tilde f}
\def\td{\tilde d}
\def\tst{\tilde t}
\def\ttau{\tilde \tau}
\def\tmu{\tilde \mu}
\def\tg{\tilde g}
\def\tw{\widetilde \chi^{\pm}}
\def\tz{\widetilde \chi^0}
\def\tnu{\tilde\nu}
\def\tell{\tilde\ell}
\def\tq{\tilde q}
\def\fp{HB/FP}
\def\mmess{M_{\rm mess}}
\mathchardef\mhyphen="2D
\def\tbta{t\mhyphen b\mhyphen\tau}
\def\btau{b\mhyphen\tau}
%%%%%%%%%%%%%%%%%%%%%%%%%%%%%%%%%%%%%%%%%%%%%%%%%
\newcommand{\lt}{\left}
\newcommand{\rt}{\right}
\newcommand{\beq}{\begin{equation}}
\newcommand{\eeq}{\end{equation}}
\newcommand{\bea}{\begin{eqnarray}}
\newcommand{\eea}{\end{eqnarray}}
\newcommand{\un}{\underline}
\newcommand{\ov}{\overline}
\newcommand{\la}{\langle}
\newcommand{\ra}{\rangle}
\newcommand{\vev}[1]{\left\langle #1\right\rangle}
\newcommand{\mgut}{M_{{\rm GUT}}}
\newcommand{\PS}{SU(4)_{C}\times SU(2)_{L}\times SU(2)_{R}}
%%%%%%%%%%%%%%%%%%%%%%%%%%%%%%%%%%%%%%%%%%%%%%%%%%%%%%%%%%%%%%%%%%%%%%%%%%%%%%%%

\setlength\textheight{24cm}
\setlength\topmargin{-1cm}
\setlength\oddsidemargin{0cm} \setlength\evensidemargin{0cm}
\setlength\textwidth{16.2cm}

\usepackage[nodisplayskipstretch]{setspace}

\begin{document}

\begin{titlepage}
\pagestyle{empty}


\vspace*{0.2in}
\begin{center}
{\Large \bf  Muon $g-2$ in an Alternative Quasi-Yukawa Unification with Low Fine-Tuned Inverse SeeSaw Mechanism
  }\\  
\vspace{1cm}
{\bf  Zafer Alt\i n$^{a,}$\footnote{E-mail: 501407009@ogr.uludag.edu.tr}, \"{O}zer \"{O}zdal $^{b,}$\footnote{E-mail: ozer.ozdal@concordia.ca} and
Cem Salih \"{U}n$^{a,}$\footnote{E-mail: cemsalihun@uludag.edu.tr}}
\vspace{0.5cm}

{\it $^a$Department of Physics, Uluda\~{g} University, TR16059 Bursa, Turkey \\
{\it $^b$Department of Physics, Concordia University 7141 Sherbrooke St. West, Montreal, Quebec, Canada H4B 1R6}
}

\end{center}



\vspace{0.5cm}
\begin{abstract}
\noindent
%\baselineskip 36pt
We explore the low scale implications of the Pati-Salam Model including the TeV scale right-handed neutrinos interacting and mixing with the MSSM fields through the inverse seesaw (IS) mechanism in the light of the muon anomalous magnetic moment (muon $g-2$) resolution, and highlight the solutions which are compatible with the Quasi-Yukawa Unification condition (QYU). We find that the presence of the right-handed neutrinos causes heavy smuons as $m_{\tilde{\mu}} \gtrsim 800$ GeV in order to avoid tachyonic staus at the low scale. On the other hand, the sneutrinos can be as light as about 100 GeV along with the light charginos of mass $\lesssim 400$ GeV, they can yield so large contributions to muon $g-2$ that the discrepancy between the experiment and the theory can be resolved. In addition, the model predicts relatively light Higgsinos ($\mu \lesssim 700$ GeV); and hence the second chargino mass is also light enough ($\lesssim 700$ GeV) to contribute to muon $g-2$. Light Higgsinos also yield less fine-tuning at the electroweak scale, and the regions compatible with muon $g-2$ restricts $\Delta_{EW}\lesssim 100$ strictly, and this region also satisfies the QYU condition. In addition, the ratios among the Yukawa couplings should be $1.8 \lesssim y_{t}/y_{b} \lesssim 2.6$, $y_{\tau}/y_{b}\sim 1.3 $ to yield correct fermion masses. Even though the right-handed neutrino Yukawa coupling can be varied freely, the solutions bound its range to $0.8\lesssim y_{\nu}/y_{b} \lesssim 1.7$. 

\end{abstract}

\end{titlepage}

%\baselineskip 36pt

%%%%%%%%%%%%%%%%%%%%%%%%%%s
% Main body
%%%%%%%%%%%%%%%%%%%%%%%%%%


\section{Introduction}
\label{sec:Intro}

Supersymmetry (SUSY) is one of the forefront candidates for physics beyond the Standard Model (SM). Resolving the gauge hierarchy problem the Higgs boson mass is not too much sensitive to the ultraviolet scale. In addition, minimal supersymmetric version of the SM (MSSM) nicely unifies the three gauge couplings of the SM, and hence, one can identify the unification scale as $\mgut \sim 2\times 10^{16}$ GeV. In this context, SUSY models can study the high energy origins of physics, which is strictly be tested at the low scale experiments by connecting $\mgut$ to the electroweak scale through the renormalization group equations (RGEs). RGEs allow one to build high scale models, and these models can significantly reduce the number of free parameters in comparison to the low scale MSSM models with free parameters more than a hundred. In this approach, minimal SUSY models have been built such as constrained MSSM (CMSSM) and non-universal Higgs models (NUHM), and their phenomenological implications have been excessively explored. These minimal models have been built with the inspiration from $SO(10)$ grand unified theories (GUTs). These GUT models do not only unify the gauge couplings, but the matter fields are also unified into a single representation, since the spinor representation is $16-$dimensional. All the matter fields of a family in MSSM can be resided into such a large representation. In addition, there is still one more space left out, which can be filled naturally by the right-handed neutrinos. In this sense, $SO(10)$ GUTs provide a natural framework to implement the SeeSaw mechanisms through which the neutrinos mix each other and receive non-zero masses favored by the current experiments \cite{Wendell:2010md}.

In addition to the unifications of the gauge couplings and the matter fields, another interesting feature in the GUT models based on the $SO(10)$ gauge symmetry imposed at $\mgut$ is the Yukawa coupling unification (YU) \cite{big-422}. In addition to $SO(10)$ GUT, also the high scale models with Pati-Salam gauge group ($G_{{\rm PS}}=\PS$, hereafter $4-2-2$ for short) \cite{pati-salam}, preserves YU, since it is the maximal subgroup of $SO(10)$. Even though it does not provide a GUT model ($g_{4}\neq g_{L}\neq g_{R}$ in principle), if it breaks into the MSSM gauge group at a scale near by $\mgut$, the gauge couplings receive negligible threshold corrections; and hence, the gauge coupling unification can be maintained in $4-2-2$ as well. In addition,  imposing left-right (LR) symmetry requires $g_{L}=g_{R}\equiv g_{2}$, and consequently $M_{L}=M_{R}\equiv M_{2}$. Even though the hypercharge is not a direct symmetry in $4-2-2$, the hypercharge jenerator can be expressed as 

\begin{equation}
Y=\sqrt{\frac{3}{5}}I_{3R}+\sqrt{\frac{2}{5}}(B-L)
\end{equation}
where $I_{3R}$ and $B-L$ are diagonal generators of $SU(2)_{R}$ and $SU(4)_{C}$ symmetry groups respectively. This relations for the hypercharge generator also yields non-universal gaugino mass terms for the MSSM gaugino fields as

\begin{equation}
M_{1}=\frac{3}{5}M_{2}+\frac{2}{5}M_{3}
\label{422gauginos}
\end{equation}
with $M_{1}$, $M_{2}$, and $M_{3}$ being soft supersymmetry breaking (SSB) mass terms for the MSSM gauginos associated with the $U(1)_{Y}$, $SU(2)_{L}$, and $SU(3)_{C}$ gauge groups respectively.

YU shared with the SUSY high scale models mentioned above provides an exclusive impact on the low scale phenomenology, even though YU is imposed at $\mgut$. This impact is mostly based on the fact that the bottom Yukawa coupling needs to receive the largest negative threshold corrections in order to yield consistent masses for the top and bottom quarks \cite{Gogoladze:2010fu}. Even though it is a very effective condition from $\mgut$ to shape the low scale parameter space, YU fails to yield consistent fermion masses for the first two families, since it predicts $N=U\propto D=L$, where $N,U,D,L$ are Dirac mass matrices for right-handed neutrinos, up and down quarks, and charged leptons respectively. In addition, the proportionality between the up and down quarks results in vanishing flavor (Cabibbo-Kobayashi-Maskawa) mixing \cite{Langacker:1980js}. Also the mass relations resulted from YU $m_{c}^{0}/m_{t}^{0}=m_{s}^{0}/m_{b}^{0}$ are clearly contradicted with the experimental results, where the superscript zero refers the parameters evaluated at $\mgut$. In order to correct these mass relations of the fermions, one can add new vector-like matter multiplets, which can mix with the fermions \cite{Witten:1979nr}. Another approach is to extend the content with new Higgs fields from another representations \cite{Babu:1992ia}. In this case, one can assume that the extra Higgs fields negligibly interact with the third family matter fields, and the MSSM Higgs doublets reside solely in $10-$plet of $SO(10)$ to maintain YU for the third family fermions, while the mass relations for the first two family fermions are corrected with these extra Higgs fields \cite{Joshipura:2012sr}. 

On the other hand, if we do not follow the assumptions mentioned above, the two approaches break YU. In this case, if we restrict the deviations in the Yukawa couplings of the third family up to, say  $20\%$; then, this modified unification scheme is called QYU \cite{Gomez:2002tj}. Even though the deviation is restricted to small amounts, QYU yields drastically different phenomenology at the low scale. For instance, $4-2-2$ is the only model, as to our knowledge, which yields light gluino ($m_{\tilde{g}}\lesssim 1$ TeV) to be next to lightest supersymmetric particle (NLSP), when YU is imposed at the GUT scale \cite{Gogoladze:2009ug}.  Relaxing it to $b-\tau$ YU allows stop NLSP solutions in addition to gluino \cite{Raza:2014upa}. On the other hand, QYU in $4-2-2$ allows a variety of NLSP species, while stop and gluino NLSP solutions are not compatible with QYU \cite{Dar:2011sj}. In addition, the parameter space compatible with YU yields large fine-tuning. While QYU can be realized with acceptable fine-tuning \cite{Dar:2011sj}.   

Based on the discussion above, YU provides a strict framework, in which the representations from a possible GUT gauge group is rather required to be minimal, since the MSSM Higgs fields are allowed to reside in $SO(10)$'s $10-$plet. Even though the framework can be extended in the QYU case, it is still minimalistic since only one extra representation for the Higgs fields (with those from $(15,1,3)$ \cite{Gomez:2002tj}) is allowed. However, if the framework is extended to include other possible Higgs representations, the MSSM Higgs fields become linear superpositions of those from these representations, and the  Yukawa couplings can receive different contributions depending on the vertices between the relevant matter and Higgs fields \cite{Bajc:2004xe}. In addition to the extra Higgs fields, the presence of higher dimensional operators also contribute to the Yukawa coupling such that the top quark Yukawa coupling can receive a significant correction from such operators \cite{Antusch:2013rxa}, and its deviation from YU cannot be restricted within $t-b-\tau$ QYU. In this context, the unification scheme can be identified as $b-\tau$ QYU \cite{Hebbar:2017olk}. 

The discussion followed so far does not consider the right-handed neutrinos. If LR symmetry is imposed in $4-2-2$, it requires the existence of the right-handed neutrinos, which actively participate in interactions through $SU(2)_{R}$ gauge group. Usually the effects from the right-handed neutrinos can be neglected safely due to the smallness of neutrino masses established by the experiments \cite{Wendell:2010md}, which stringently restricts the neutrino Yukawa coupling as $y_{\nu}\lesssim 10^{-7}$ \cite{Coriano:2014wxa}. On the other hand, this result does not hold when the IS mechanism is implemented, in which a large neutrino Yukawa coupling ($y_{\nu}\sim y_{t}$) can still be consistent with the smallness of neutrino masses \cite{Khalil:2010iu}.  With the presence of the right-handed neutrinos with a large Yukawa coupling, the unification scheme discussed above should be modified to include the right-handed neutrinos. In this case YU should be imposed as $y_{t}=y_{b}=y_{\tau}=y_{\nu} \equiv y$ at $\mgut$. In SUSY models, the right-handed neutrinos, in contrast to the charged leptons, interact with $H_{u}$, and the deviation in $y_{\nu}$ from YU should be proportional to those which deviate $y_{t}$ from YU. In the case of $b-\tau$ QYU, also one should impose another QYU scenario between $y_{t}$ and $y_{\nu}$ simultaneously. Following Refs. \cite{Gomez:2002tj,Dar:2011sj}, the deviations in Yukawa couplings can be formulated as 

\begin{equation}
\setstretch{1.5}
\begin{array}{l}
y_{b}:y_{\tau}=\mid 1- C_{b\tau} \mid:\mid 1+3C_{b\tau}\mid \\ 
y_{t}:y_{\nu}=\mid 1+C_{t\nu}\mid:\mid 1- 3C_{t\nu}\mid,
\label{QYUCondition}
\end{array}
\end{equation}
where $C_{b \tau}$ and $C_{t \nu}$ measure the deviation in Yukawa couplings from YU. Note that  since YU is broken first by the higher order operators as $y_{t}=y_{\nu}$ and $y_{b}=y_{\tau}$, $C_{b \tau}$ and $C_{t \nu}$ are not related to each other.

Previous studies of QYU (see for instance Refs. \cite{Gomez:2002tj,Dar:2011sj}) have revealed that the general QYU scenarios are mostly compatible in the regions with large $\tan\beta$. Such regions, depending on the mass spectrum of the supersymmetric particles, can also yield large SUSY contributions to muon $g-2$. The SM predictions exhibit about $3\sigma$ deviation from the experimental results, and this situation can be expressed as \cite{Bennett:2006fi}

\begin{equation}
 \Delta a_{\mu}\equiv a_{\mu}^{{\rm exp}}-a_{\mu}^{{\rm SM}}=(28.7\pm 8.0)\times 10^{-10} ~(1\sigma)~.
\end{equation} 

This discrepancy has been survived even after highly accurate calculations over the SM predictions were performed \cite{Davier:2010nc}; therefore, the discrepancy can be interpreted as the effect of the new physics beyond the SM. In our work, we will explore the low scale implications of $4-2-2$ including the TeV scale right-handed neutrinos, which interact and mix with the MSSM fields through the IS mechanism in the light of muon $g-2$ resolution, and highlight the solutions which are compatible with the QYU condition given in Eq. (\ref{QYUCondition}). The rest of the paper is organized as follows: We will briefly discuss the effect of the presence of the right-handed neutrinos on muon $g-2$ along with the sparticle mass spectrum in Section \ref{sec:MSSMIS}. We describe our scanning procedure and the experimental constraints employed in our data generation and analyses are summarized in Section \ref{sec:scan}. Then, we first present our results for muon $g-2$ and the sparticle spectrum in Section \ref{sec:res}. Section \ref{sec:FTg2} discusses muon $g-2$ resolution in respect of the fine-tuning, which is required to have correct electroweak symmetry breaking scale. Finally we will summarize and conclude in Section \ref{sec:conc}.




\section{Muon $g-2$ in MSSM with Inverse Seesaw}
\label{sec:MSSMIS}

\begin{figure}[h!]
\centering
\includegraphics[scale=0.3]{MSSMISS_Feynmann.png}
\caption{The SUSY contributions to muon $g-2$ involving with the sneutrinos and charginos.}
\label{fig1}
\end{figure}


We will discuss the SUSY contributions to muon $g-2$ in MSSM, when the right-handed neutrinos are present and they mix through the IS mechanism. In addition to the SUSY contributions to muon $g-2$ in MSSM \cite{Moroi:1995yh}, we also present two diagrams involving with the neutrinos to illustrate the contributions arising because of the IS mechanism, in which also the charginos are running in the loop due to the charge conservation. The behavior of the SUSY contributions can be understood by calculating these diagrams with the mass insertion method, which is represented with dots in the diagrams. The approximate contributions can be obtained as follows \cite{Khalil:2015wua}: 


\begin{equation}\hspace{-3.2cm}
(\Delta a_{\mu})_{C1}\approx \frac{m_{\mu}^{2}\mu^{2}\cot\beta}{m^{2}_{\tilde{N}}-m^{2}_{\tilde{\nu}}} \left[  \frac{f_{\chi}(\mu^{2}/m^{2}_{\tilde{N}})}{m^{2}_{\tilde{N}}} -\frac{f_{\chi}(\mu^{2}/m^{2}_{\tilde{\nu}})}{m^{2}_{\tilde{\nu}}}  \right]
\label{aC1}
\end{equation}

\begin{equation}\hspace{-2.8cm}
(\Delta a_{\mu})_{C2}\approx \frac{m^{2}_{\mu}M_{2}\mu\tan\beta}{m^{2}_{\tilde{N}}}  \left[ \frac{f_{\chi}(M_{2}^{2}/m^{2}_{\tilde{N}})-f_{\chi}(\mu^{2}/m^{2}_{\tilde{N}})}{M_{2}^{2}-\mu^{2}}   \right]
\label{aC2}
\end{equation}

The $\tan\beta$ dependence of muon $g-2$ contributions can be seen from these equations. Note that there are other terms, which do not depend on $\tan\beta$, but these terms are rather negligible, unless the supersymmetric particles are so light that they are excluded by the current mass bounds. The two diagrams shown in Figure \ref{fig1} exhibits different behavior in respect of $\tan\beta$. The first diagram is expected to be effective, when $\tan\beta \ll 1$. Recall that if the charged sleptons ran in the loop, the contributions would be enhanced with $\tan\beta$. The suppression in the sneutrino case is because the sneutrinos interact with $H_{u}$, while charged sleptons interact with $H_{d}$. On the other hand, the contributions represented in the second diagram exhibits an enhancement as $\tan\beta$ increases. In this processes, the $\tan\beta$ enhancement arises from the mixing between two charginos; i.e. the Wino and Higgsino. In this sense, one can expect that the second diagram illustrates the dominant processes in the SUSY contributions to muon $g-2$. 

We should note here that the second diagram is already present in the usual MSSM framework, in which the right-handed neutrinos, and hence the IS mechanism, are absent. In this context, one can conclude that the resolution to muon $g-2$ discrepancy is not improved too much when the right-handed neutrinos are present and they mix through the IS mechanism. However, a recent study shows that muon $g-2$ discrepancy can be significantly resolved in MSSM extended with a $U(1)_{B-L}$ symmetry (BLSSM) \cite{Khalil:2015wua} when the IS mechanism is also implemented. The resolution happens even the universal boundary conditions are imposed at $\mgut$, which is not possible in the MSSM framework without the IS mechanism. Thus, the effects of the right-handed sneutrinos could be indirect, while their direct contributions to muon $g-2$ is significantly suppressed by $\tan\beta$. Such indirect effects can be understood by considering the following RGEs for the relevant parameters, which run the parameters from $\mgut$ to the low scale


\begin{equation}
\setstretch{2.5}
\begin{array}{ll}
\dfrac{dm_{\tilde{L}}^{2}}{dt} & =\left( \dfrac{dm_{\tilde{L}}^{2}}{dt} \right)_{MSSM}-2m_{H_{u}}^{2}Y_{\nu}^{\dagger}Y_{\nu}-2T_{\nu}^{\dagger}T_{\nu}-2m_{\tilde{l}}^{2}Y_{\nu}^{\dagger}Y_{\nu} -2Y_{\nu}^{\dagger}m_{\tilde{\nu}}^{2}Y_{\nu} \\
\dfrac{dm_{\tilde{e}}^{2}}{dt} & =\left( \dfrac{dm_{\tilde{e}}^{2}}{dt} \right)_{MSSM} \\
\dfrac{dm_{\tilde{N}}^{2}}{dt} & =-2 \left( 2m_{H_{u}}^{2} Y_{\nu}Y_{\nu}^{\dagger}+2 T_{\nu}T_{\nu}^{\dagger}+2 Y_{\nu}m_{\tilde{l}}^{2}Y_{\nu}^{\dagger} +2m_{\tilde{\nu}}^{2}Y_{\nu}Y_{\nu}^{\dagger} \right) \\
\dfrac{dm_{H_{u}}^{2}}{dt}& =\left( \dfrac{dm_{H_{u}}^{2}}{dt} \right)_{MSSM}-2m_{H_{u}}^{2}Tr\left(Y_{\nu}Y_{\nu}^{\dagger}\right)-2Tr\left(T_{\nu}^{\ast}T_{\nu}^{T}\right) \\
& \hspace{3.4cm}-2Tr\left(m_{\tilde{l}}^{2}Y_{\nu}^{\dagger}Y_{\nu}\right)-2Tr\left( m_{\tilde{\nu}}^{2}Y_{\nu}Y_{\nu}^{\dagger} \right) \\
\dfrac{d\mu}{dt} & =\left( \dfrac{d\mu}{dt} \right)_{MSSM}-\mu Tr \left( Y_{\nu}Y_{\nu}^{\dagger} \right)
\end{array}
\label{RGEs}
\end{equation}
where we have used the usual notation for the MSSM fields. In addition, $\tilde{N}$ denotes the right-handed sneutrino field. The first terms in the equations with the subscript MSSM represent the RGEs for these parameters within the MSSM framework without the right-handed neutrino. The other terms are relevant to the presence of the right-handed neutrinos. As is seen from the RGEs given above, the neutrino Yukawa couplings, $Y_{\nu}$, and its trilinear interaction term $T_{\nu}$ are effective in lowering the SSB masses of the charged sleptons, and as a result they could be much lighter at the low scale than those in the usual MSSM without the right-handed neutrinos. If the texture of the neutrino Yukawa couplings are similar to the up-type quarks Yukawa couplings ($Y_{\nu}\sim Y_{u}$) \cite{Khalil:2010iu}, then neutrino Yukawa couplings and trilinear interaction term can result in tachyonic states ($m_{L,e,N}^{2} < 0$) especially for staus. In this context, the smuon masses can be found slightly heavier in order to avoid tachyonic stau mass eigenstates \cite{Gogoladze:2014vea}. However, the SUSY contributions to muon $g-2$ from smuon-neutralino loop can be suppressed, if $m_{\tilde{\mu}} \gtrsim 800$ \cite{Babu:2014lwa}.


Similar discussion holds for the right-handed sneutrinos. Its SSB mass parameter is determined with the common mass scale for the scalars, $m_{0}$; and hence, $m_{0}$ cannot be lower than certain scales not to have tachyonic sneutrinos at the low scale. The RGE for $m_{H_{u}}$ reveals an interesting feature for the IS mechanism that $Y_{\nu}$ and $T_{\nu}$ lower its value from $\mgut$ to the low scale as $Y_{t}$ gives the same impact as MSSM. The electroweak symmetry breaking requires $m_{H_{u}}^{2} < 0$, and MSSM can have only stops to have negative $m_{H_{u}}^{2}$ at the low scale. This fact leads to heavy stops and/or large mixing between left and right-handed stops in the MSSM. On the other hand, when the IS mechanism is implemented in the MSSM framework, the sneutrinos, together with the stops, yield $m_{H_{u}}^{2} < 0$, which loose the pressure on the stop sector. In this case, even if the mixing between the left and right-handed stops are small, it is still possible to have stops at around TeV scale in the mass spectra. 

Before concluding this section, the last RGE for the $\mu-$term in Eq.(\ref{RGEs}) is also interesting in the naturalness point of view. As shown in previous studies \cite{Baer:2012mv}, the required fine-tuning at the electroweak scale is mostly determined by $\mu$. Its RGE in the case with the IS mechanism shows that the $\mu-$term is lowered further than that in the usual MSSM by the neutrino Yukawa couplings and trilinear scalar interaction terms. Hence, one can expect that the MSSM with IS can yield significantly low fine-tuned solutions at the low scale. 

Note that even though the RGEs are more or less the same in the case with Type I seesaw, the terms with neutrino Yukawa couplings and trilinear scalar interaction term are quite negligible, since $Y_{\nu}\lesssim 10^{-7}$. In this sense, despite the presence of the right-handed neutrinos, the low scale implications of SUSY Type-I Seesaw are almost the same as those in the usual MSSM models.




\section{Scanning Procedure and Experimental Constraints}
 \label{sec:scan}

In scanning the fundamental parameter space, we have employed SPheno 3.3.3 package \cite{Porod:2003um} obtained with SARAH 4.6.0 \cite{Staub:2008uz}. This package evolves the weak scale values of gauge and Yukawa couplings to $\mgut$ via the MSSM RGEs, which are modified to include the IS mechanism. $\mgut$ is dynamically determined with the gauge coupling unification condition. Note that we do not strictly  enforce the unification condition at $\mgut$, since a few percent deviation from the unification can be assigned to unknown GUT-scale threshold corrections \cite{Hisano:1992jj}, which modify the unification condition as $g_{1}=g_{2}\approx g_{3}$. With the boundary conditions given at $M_{{\rm GUT}}$, all the SSB parameters along with the gauge and Yukawa couplings are evolved back to the weak scale.

We have performed random scans over the following parameter space 

\begin{equation}
\setstretch{1.5}
\begin{array}{ccc}
0 \leq &  ~ m_{0},m_{H_{d}},m_{H_{u}} ~  &  \leq ~ 5~{\rm TeV}, \\
-5 \leq &  ~ M_{2}~&\leq ~ 0~{\rm TeV}, \\
0 \leq &  ~ 	M_{3}~&\leq ~ 5~{\rm TeV}, \\
-3  \leq & A_{0}/m_{0} &\leq 3  \\
 35  \leq & \tan\beta & \leq 60
 \end{array}
\label{paramSpace}
\end{equation}
\begin{equation*}
\mu < 0~,\hspace{0.3cm} m_{t}=173.3~{\rm GeV}
\end{equation*}
where $m_{0}$ symbolizes the SSB mass term for the matter scalars, while $m_{H_{d}}$ and $m_{H_{u}}$ denote the SSB mass terms for the MSSM Higgs doublets. $M_2$ and $M_3$  stand for the gauginos associated with the $SU(2)_{L}$ and $SU(3)_{C}$ respectively. The SSB mass term, $M_{1}$ for the $U(1)_{Y}$ gaugino is determined in terms of $M_{2}$ and $M_{3}$ as given in Eq.(\ref{422gauginos}). $A_0$ is the SSB trilinear coupling, and $\tan\beta$ is the ratio of vacuum expectation values (VEVs) of the MSSM Higgs doublets. The value of $\mu$-term in MSSM is determined by the radiative electroweak symmetry breaking (REWSB) condition but not its sign; thus, its sign is one of the free parameter in MSSM and it is set negative in our scans. In addition, we have employed the current central value for the top quark mass as $m_t=173.3$ GeV \cite{Group:2009ad}. Note that the sparticle spectrum is not too sensitive in one or two sigma variation in the top quark mass \cite{Gogoladze:2011db}, but it can shift the Higgs boson mass by 1-2 GeV \cite{Gogoladze:2011aa}.  In addition to these free parameters, the experiments do not provide any value for the neutrinos Yukawa coupling at the low scale, in contrast to those associated with the charged leptons. Hence, they need to be provided as an input at the low scale. In our scans, we vary Yukawa coupling $Y_{\nu}$ within perturbative level.

In adjusting the ranges of the free parameters, we restrict the scalar and gaugino SSB mass terms not to exceed 5 TeV in order to remain in the regions which yield acceptable fine-tuning at the low scale. The range of the trilinear scalar coupling is set to avoid charge and/or color breaking minima, which requires $|A_{0}|\lesssim 3m_{0}$. Among these parameters, we bound $\tan\beta$ at 35 from below. Even though the general MSSM framework can be consistent with the current experimental results including the Higgs boson mass, Yukawa unification requires rather large $\tan\beta$ to satisfy the correct masses for quarks and charged leptons \cite{bigger-422}. Even in the case of QYU, the unification scheme needs $\tan\beta \gtrsim 40$ \cite{Dar:2011sj}.

The REWSB condition puts crucial theoretical constraint \cite{Ibanez:Ross} on the parameter space given in Eq.(\ref{paramSpace}). According to this constraint, the SSB mass-squared terms for the Higgs doublets are required to be negative at the low scale, though they are positive-defined at $\mgut$. In this context, the relevant parameters in the RGE evolutions of these mass parameters should be tuned in a way that, $m_{H_{u}}^{2}$ and/or $m_{H_{d}}^{2}$ have to be turn negative. Another constraint is dark matter observations and it restricts the parameter space which requires the lightest supersymmetric particle (LSP) stable and no electric and color charge, which excludes the regions leading to stau or stop LSP solutions \cite{Nakamura:2010zzi}. On the other hand, even if a solution does not satisfy the dark matter observations, it can still survive in conjunction with other form(s) of the dark matter formation \cite{Baer:2012by}. Based on this discussion, we accept only the solutions which yield neutralino LSP at the low scale, but we do not apply any constraint from the dark matter experiments.

In scanning the parameter space, we use our interface which employs Metropolis-Hasting algorithm described in \cite{Belanger:2009ti}. All collected data points satisfy the requirement of REWSB and neutralino LSP. After collecting data, we subsequently impose the mass bounds on all 	the sparticles \cite{Agashe:2014kda} and the constraints from rare decay processes $B_s \rightarrow \mu^+ \mu^-$ \cite{Aaij:2012nna} and $b \rightarrow s \gamma$ \cite{Amhis:2012bh}. In addition those bounds we have imposed Higgs boson \cite{Aad:2012tfa} and gluino masses \cite{TheATLAScollaboration:2015aaa}. The experimental constraints mentioned above can be summarized below:
\begin{equation}
\setstretch{1.5}
\begin{array}{rr}
m_{\tilde{\chi}^{\pm}_{1}} & \geq  103.5~{\rm GeV}, \\
 123 \leq m_h & \leq ~ 127~~{\rm GeV}, \\
m_{\tilde \tau} & \geq ~ 105~{\rm GeV}, \\
m_{\tilde g} & \geq  1800~{\rm GeV}, \\
 \end{array}
\label{experimentalcons}
\end{equation}
\begin{equation*}
\begin{split}
0.8\times 10^{-9} \leq BR(B_s \rightarrow \mu^+ \mu^-)~~ & \leq \quad 6.2 \times 10^{-9} (2\sigma), \\
2.99\times 10^{-4} \leq BR(b \rightarrow s \gamma) ~~ & \leq \quad 3.87 \times 10^{-4} (2\sigma) \\
\end{split}
\end{equation*}

Finally we identify the regions compatible with QYU by restricting the deviations in the Yukawa couplings within to $20\%$ by applying $C_{t\nu} \leq 0.2$ and $C_{b\tau}\leq 0.2$, which refers to the QYU condition.

\section{Fundamental Parameter Space of Muon $g-2$ and Sparticle Spectrum}
\label{sec:res}

\begin{figure}[ht!]
\centering
\subfigure{\includegraphics[scale=1]{ISS_NUGM_Rtvbtau_DAMU_C2.png}}
\subfigure{\includegraphics[scale=1]{ISS_NUGM_Rtvbtau_DAMU_C1.png}}
\caption{Plots in the $\Delta a_{\mu}-C_{b\tau}$ and $\Delta a_{\mu}-C_{t\nu}$ planes. All points are consistent with REWSB and neutralino LSP. Green points satisfy the experimental constraints mentioned in Section \ref{sec:scan}. Yellow band is an independent subset of gray points, and they indicate the values of $\Delta a_{\mu}$ which would bring the theory and the experiment within $1\sigma$.}
\label{fig2}
\end{figure}

\begin{figure}[ht!]
\centering
\subfigure{\includegraphics[scale=1]{ISS_NUGM_Rtvbtau_DAMU_m0.png}}
\subfigure{\includegraphics[scale=1]{ISS_NUGM_Rtvbtau_DAMU_M2.png}}
\subfigure{\includegraphics[scale=1]{ISS_NUGM_Rtvbtau_DAMU_M1.png}}
\subfigure{\includegraphics[scale=1]{ISS_NUGM_Rtvbtau_DAMU_tanb.png}}
\caption{Plots in the $\Delta a_{\mu}-m_{0}$, $\Delta a_{\mu}-M_{2}$, $\Delta a_{\mu}-M_{1}$ and $\Delta a_{\mu}-\tan\beta$ planes. The color coding is the same as Figure \ref{fig2}. In addition, blue points form a subset of green and they represent the solutions compatible with the QYU condition.}
\label{fig3}
\end{figure}

\begin{figure}[ht!]
\centering
\subfigure{\includegraphics[scale=1]{ISS_NUGM_Rtvbtau_DAMU_Mchi1.png}}
\subfigure{\includegraphics[scale=1]{ISS_NUGM_Rtvbtau_DAMU_Mcha1.png}}
\subfigure{\includegraphics[scale=1]{ISS_NUGM_Rtvbtau_DAMU_Massmu1.png}}
\subfigure{\includegraphics[scale=1]{ISS_NUGM_Rtvbtau_DAMU_MSvRe1.png}}
\caption{Plots in the $\Delta a_{\mu}-m_{\tilde{\chi}_{1}^{0}}$, $\Delta a_{\mu}-m_{\tilde{\chi}_{1}^{\pm}}$, $\Delta a_{\mu}-m_{\tilde{\mu}_{1}}$ and $\Delta a_{\mu}-m_{\tilde{\nu}_{1}}$ planes. The color coding is the same as Figure \ref{fig3}.}
\label{fig4}
\end{figure}

In this section, we discuss muon $g-2$ results and highlight the solutions compatible with the QYU condition. We start first with Figure \ref{fig2} displaying the deviations of the Yukawa couplings from the unification with plots in the $\Delta a_{\mu}-C_{b\tau}$ and $\Delta a_{\mu}-C_{t\nu}$ planes. All points are consistent with REWSB and neutralino LSP. Green points satisfy the experimental constraints mention in Section \ref{sec:scan}. Yellow band is an independent subset of gray points, and they indicate the values of $\Delta a_{\mu}$ which would bring the theory and the experiment within $1\sigma$. As seen from the $\Delta a_{\mu}-C_{b\tau}$, $C_{b\tau}$ measuring the deviation in $y_{b}$ and $y_{\tau}$ can barely reach to $20\%$, and the experimental constraints restrict it to $C_{b\tau} \lesssim 0.1$ (green). In addition, the region compatible with resolution to muon $g-2$ discrepancy (yellow band) bounds it further to $C_{b\tau} \lesssim 0.07$. On the other hand, $C_{t\nu}$ can be as large as about $0.3$ as shown in the $\Delta a_{\mu}-C_{t\nu}$. This is not surprising, since $y_{t}$, and consequently $y_{\nu}$, can receive large corrections from the extra Higgs fields and also the higher dimensional operators. However, it is still possible to restrict it within to $20\%$. Besides, imposing the QYU condition will exclude the solutions with $C_{t\nu} > 0.2$.

Figure \ref{fig3} represents the correlations between muon $g-2$ and the relevant fundamental parameters with plots in the $\Delta a_{\mu}-m_{0}$, $\Delta a_{\mu}-M_{2}$, $\Delta a_{\mu}-M_{1}$ and $\Delta a_{\mu}-\tan\beta$ planes. The color coding is the same as Figure \ref{fig2}. In addition, blue points form a subset of green and they represent the solutions compatible with the QYU condition. The $\Delta a_{\mu}-m_{0}$ shows that $m_{0}$ cannot be greater than about 1.2 TeV in order for the resolution to muon $g-2$ discrepancy. Since $m_{0}$ controls the scalar masses, it is understandable with the need of light scalars, which run in the loops contributing to muon $g-2$, at the low scale. However, the regions with $m_{0} \lesssim 600$ GeV cannot provide a resolution to muon $g-2$ consistently accommodated with the current experimental constraints. This result arises from the effects of the right-handed neutrino sector discussed along with the RGEs in the previous section. The gray region coinciding with the yellow band yield inconsistently light charged sleptons ($m_{\tilde{l}} < 100 $ GeV) especially for those from the third family. In addition, the Higgs boson mass is problematic in these regions, since most of the solutions predict $m_{h} < 125$ GeV. Similarly, muon $g-2$ condition requires light weakinos (Bino and Wino), and $|M_{2}| \lesssim 500$ GeV as seen from the $\Delta a_{\mu}-M_{2}$ plane. This parameter controls the wino mass at the low scale as $m_{\tilde{W}}\approx |M_{2}|$; hence, muon $g-2$ condition necessitates light charginos at the low scale. Similarly, $M_{1}$, which controls the Bino mass as $m_{\tilde{B}} \approx 0.5 |M_{1}|$ \cite{Gogoladze:2009bd}, needs to be light ($\lesssim 500$ GeV). Since we set in $\mu < 0$, one can expect to have large SUSY contributions to muon $g-2$ when $M_{1}$ is negative in contrast to the results shown in the $\Delta a_{\mu}-M_{1}$ plane, where the SUSY contributions seem suppressed when $M_{1} < 0$. 

We consider the low scale mass spectrum for the supersymmetric particles in Figure \ref{fig4} with plots in the $\Delta a_{\mu}-m_{\tilde{\chi}_{1}^{0}}$, $\Delta a_{\mu}-m_{\tilde{\chi}_{1}^{\pm}}$, $\Delta a_{\mu}-m_{\tilde{\mu}_{1}}$ and $\Delta a_{\mu}-m_{\tilde{\nu}_{1}}$ planes. The color coding is the same as Figure \ref{fig3}. The neutralino LSP mass cannot exceed about $200$ GeV in order for maintaining the resolution to muon $g-2$, while the lightest chargino can be as heavy as about $400$ GeV as seen from the $\Delta a_{\mu}-m_{\tilde{\chi}_{1}^{\pm}}$. In contrast to the usual MSSM implications, the IS mechanism yields rather heavy smuons ($m_{\tilde{\mu}} \gtrsim 800$ GeV), which can significantly suppress the SUSY contributions from the smuon-neutralino loop processes. As discussed above, the smuon masses are mostly bounded from below by the stau mass, which can turn to be tachyonic when the other charged sleptons are light due to its large trilinear SSB term ($T_{\tau}\equiv A_{\tau}y_{\tau}$). The solutions in such regions are required to yield $m_{\tilde{\tau}} \geq 105$ GeV to be consistent the mass bounds on sparticles. While the stau mass bound has a strong impact on the smuon masses, the sneutrinos, on the other hand, can be as light as about $100$ GeV, which yield significant SUSY contributions to muon $g-2$ along with light charginos. Consequently, the main contributions to muon $g-2$ are provided from the sneutrino-chargino loop processes, while those from smuon and neutralino are highly suppressed due to the heavy smuon masses. This also explains why there is no significant muon $g-2$ contributions when $M_{1}$ is negative. In this region, $M_{2}$ needs to be larger than $M_{3}$, which yields relatively heavy charginos at the low scale, so the SUSY contributions from chargino and sneutrino are also suppressed. In addition, the reason why the neutralino mass is bounded from above as $m_{\tilde{\chi}_{1}^{0}} \lesssim 200$ GeV is only the condition which requires neutralino to be LSP for all the solutions.




\section{Fine-Tuning and Muon $g-2$ in MSSM with IS}
\label{sec:FTg2}

\begin{figure}[ht!]
\centering
\subfigure{\includegraphics[scale=1]{ISS_NUGM_Rtvbtau_DAMU_MuSUSY.png}}
\subfigure{\includegraphics[scale=1]{ISS_NUGM_Rtvbtau_DAMU_M2_MuSUSY.png}}
\subfigure{\includegraphics[scale=1]{ISS_NUGM_Rtvbtau_DAMU_Mcha2.png}}
\subfigure{\includegraphics[scale=1]{ISS_NUGM_Rtvbtau_DAMU_M1_MuSUSY.png}}
\caption{Plots in the $\Delta a_{\mu}-\mu$, $\Delta a_{\mu}-M_{2}/\mu$, $\Delta a_{\mu}-m_{\tilde{\chi}_{2}^{\pm}}$, and $\Delta a_{\mu}-M_{1}/\mu$ planes. The color coding is the same as Figure \ref{fig3}.}
\label{fig5}
\end{figure}


%\begin{figure}[ht!]
%\centering
%\subfigure{\includegraphics[scale=1]{ISS_NUGM_Rtvbtau_DAMU_Mchi2.png}}
%\subfigure{\includegraphics[scale=1]{ISS_NUGM_Rtvbtau_DAMU_Mchi3.png}}
%\subfigure{\includegraphics[scale=1]{ISS_NUGM_Rtvbtau_DAMU_Mchi4.png}}
%\subfigure{\includegraphics[scale=1]{ISS_NUGM_Rtvbtau_DAMU_Mcha2.png}}
%\caption{Plots in the $\Delta a_{\mu}-m_{\tilde{\chi}_{2}^{0}}$, $\Delta a_{\mu}-m_{\tilde{\chi}_{3}^{0}}$, $\Delta a_{\mu}-m_{\tilde{\chi}_{4}^{0}}$ and $\Delta a_{\mu}-m_{\tilde{\chi}_{2}^{\pm}}$ planes. The color coding is the same as Figure \ref{fig3}.}
%\label{fig6}
%\end{figure}


\begin{figure}[ht!]
\centering
\subfigure{\includegraphics[scale=1]{ISS_NUGM_Rtvbtau_DAMU_FT.png}}
\subfigure{\includegraphics[scale=1]{ISS_NUGM_Rtvbtau_YtYb_FT.png}}
\subfigure{\includegraphics[scale=1]{ISS_NUGM_Rtvbtau_YtauYb_FT.png}}
\subfigure{\includegraphics[scale=1]{ISS_NUGM_Rtvbtau_YnuYb_FT.png}}
\caption{Plots in the $\Delta a_{\mu}-\Delta_{EW}$, $y_{t}/y_{b}-\Delta_{EW}$, $y_{\tau}/y_{b}-\Delta_{EW}$ and $y_{\nu}/y_{b}-\Delta_{EW}$ planes. The color coding is the same as Figure \ref{fig3}.}
\label{fig6}
\end{figure}

%\begin{figure}[ht!]
%\centering
%\subfigure{\includegraphics[scale=1]{ISS_NUGM_Rtvbtau_YtauYb_YtYb.png}}
%\caption{Plots in the $y_{\tau}/y_{b}-y_{t}/y_{b}$ plane. The color coding is the same as Figure \ref{fig3}.}
%\label{fig7}
%\end{figure}

As we discussed in the previous section, the dominant contribution to muon $g-2$ comes from the sneutrino-chargino loop processes. In these processes, the chargino can be either Wino or Higgsino, each of which corresponds to different nature of the SUSY contributions. If the chargino is Wino-like, then the contributions are generated through $SU(2)$ interactions, while the Yukawa interactions take part when the chargino is Higgsino like. Depending on the ratios of their masses, the chargino could also be a mixture of these two particles. Figure \ref{fig5} represents the result for the Higgsino mass and its mass ratio to the chargino mass with plots in the $\Delta a_{\mu}-\mu$, $\Delta a_{\mu}-M_{2}/\mu$, $\Delta a_{\mu}-m_{\tilde{\chi}_{2}^{\pm}}$, and $\Delta a_{\mu}-M_{1}/\mu$ planes. The color coding is the same as Figure \ref{fig3}. According to the $\Delta a_{\mu}-\mu$ plane, the Higgsinos, whose masses are equal to $\mu$, can be as light as about 500 GeV, while muon $g-2$ condition bounds its mass from above at about 700 GeV. In this sense, the model predicts relatively light Higgsinos at the low scale compatible with the QYU condition. The $\Delta a_{\mu}-M_{2}/\mu$ plane compares the Wino and Higgsino masses to each other by considering their mass ratio. The results in this plane show that, despite the light Higgsinos, the Wino is mostly lighter than the Higgsino over all the parameter space, when the solutions yield muon $g-2$ results that would bring the experiment and theory within to $1\sigma$, since $M_{2}/\mu \lesssim 1$. On the other hand, muon $g-2$ resolution also bounds this mass ratio from below at about 0.5, which leads to a comparable mixing between the Wino and the Higgsino in formation of the lightest chargino. If the low scale spectrum includes two charginos lighter than about a TeV, then the processes can contribute to muon $g-2$, even when the heaviest chargino runs in the loop. Even though the heaviest chargino contribution can only be minor in comparison to the lightest chargino contribution, its mass cannot be heavier than about 800 GeV for the resolution to muon $g-2$ discrepancy, as seen from the  $\Delta a_{\mu}-m_{\tilde{\chi}_{2}^{\pm}}$. Finally, we also present the mass ratio of $M_{1}$ and $\mu$ in the $\Delta a_{\mu}-M_{1}/\mu$ plane. Even though $M_{1}$ does not interfere  in the SUSY contributions to muon $g-2$ due to the heavy smuons, the results in this plane reveal nature of LSP neutralino. Since $|M_{1}/\mu| \lesssim 1$, the Bino mixes in the LSP neutralino formation more than the Higgsinos. Comparing this plane with the results shown in Figure \ref{fig3}, one can easily see $M_{2} \lesssim M_{1}$, hence the model yields Wino-like LSP neutralino at the low scale. In addition, many of the solutions yield significant mixture of the neutralinos.

The light Higgsinos are also interesting from the naturalness point of view. Since the mass bounds on the supersymmetric particles become severe after the latest results from the LHC experiments, the solutions can barely be placed in the natural region characterized with $m_{\tilde{t}_{1}},m_{\tilde{t}_{2}}, m_{\tilde{b}_{1}} \lesssim 500$ GeV. Especially the Higgs boson mass constraint requires at least one stop to have mass above TeV scale. On the other hand, deviation from the natural region can be measured with $\Delta_{EW}$, the fine-tuning parameter as defined in Ref. \cite{Baer:2012mv}. $\Delta_{EW}$ is a function of $\mu$, $m_{H_{d}}$, $m_{H_{u}}$, and $\tan\beta$, in principal; however, the terms proportional to $m_{H_{d}}$ are suppressed by $\tan\beta$, and the correct electroweak symmetry breaking scale requires $\mu \approx m_{H_{u}}$ over most of the fundamental parameter space. In this sense, the Higgsino masses can also indicate the fine-tuning amount required to have consistent electroweak symmetry breaking. Since our model predicts relatively light Higgsinos ($\lesssim 800$ GeV) compatible with the resolution to muon $g-2$ discrepancy, such solutions also need low fine-tuning. In general fashion, the acceptable fine-tuning is identified with the condition $\Delta_{EW} \leq 1000$. Figure \ref{fig6} investigates our discussion about the fine-tuning with plots in the $\Delta a_{\mu}-\Delta_{EW}$, $y_{t}/y_{b}-\Delta_{EW}$, $y_{\tau}/y_{b}-\Delta_{EW}$ and $y_{\nu}/y_{b}-\Delta_{EW}$ planes. The color coding is the same as Figure \ref{fig3}. As seen from the  $\Delta a_{\mu}-\Delta_{EW}$ plane, $\Delta_{EW}$ can be as low as 20 compatible with muon $g-2$ condition. Indeed, muon $g-2$ condition restricts $\Delta_{EW} \lesssim 100$, and the discrepancy cannot be solved within $1\sigma$ when $\Delta_{EW} > 100$. In addition to muon $g-2$ resolution, we also discuss the Yukawa couplings, whose deviations are also restricted by the QYU condition in the $y_{t}/y_{b}-\Delta_{EW}$, $y_{\tau}/y_{b}-\Delta_{EW}$ and $y_{\nu}/y_{b}-\Delta_{EW}$ planes. According to the results represented in these planes, $y_{t}/y_{b} \gtrsim 2$, $y_{\tau}/y_{b}\gtrsim 1.2$, and $y_{\nu}/y_{b} \gtrsim 0.8$ over the region with the acceptable fine-tuning.
 






\section{Conclusion}
\label{sec:conc}

We explore the low scale implications of $4-2-2$ including the TeV scale right-handed neutrinos interacting and mixing with the MSSM fields through the IS mechanism, in light of muon $g-2$ resolution and highlight the solutions which are compatible with the QYU condition. We found that the presence of the right-handed neutrinos cause the smuons are rather heavy as $m_{\tilde{\mu}} \gtrsim 800$ GeV in order to avoid tachyonic staus at the low scale. In this context, the usual MSSM contributions to muon $g-2$, which are provided from smuon-neutralino loop, is strongly suppressed. On the other hand, the sneutrinos can be as light as about 100 GeV along with the light charginos of mass $\lesssim 400$ GeV can yield so large contributions to muon $g-2$ that the discrepancy between the experiment and the theory can be resolved. In addition, the model predicts relatively light Higgsinos ($\mu \lesssim 700$ GeV); and hence the second chargino mass is also light enough ($\lesssim 700$ GeV) to contribute to muon $g-2$. Despite the Higgsino mixing in the lightest neutralino and chargino is limited, the light Higgsinos are interesting from the naturalness point of view, since such solutions of the light Higgsinos need to be fine-tuned much less than the other solutions. We found that such solutions can be also compatible with the QYU, since $\Delta_{EW}$ can be as low as about 100. The acceptable fine-tuning can also have a strong impact on the Yukawa couplings in terms of their ratios, and this impact also shapes the fundamental parameter space of QYU, since it is rather related to the corrections in the Yukawa couplings. In the regions with acceptable fine-tuning and compatible with muon $g-2$ resolution and the QYU condition, the ratios among the Yukawa couplings can be summarized as $1.8 \lesssim y_{t}/y_{b} \lesssim 2.6$, $y_{\tau}/y_{b}\sim 1.3 $. In addition, even though the right-handed neutrino Yukawa coupling can be varied freely, the solutions restrict its range as $0.8\lesssim y_{\nu}/y_{b} \lesssim 1.7$. 


\vspace{0.3cm}
\noindent {\bf Acknowledgments}

We would like to thank Zerrin K\i rca, B\"{u}\c{s}ra Ni\c{s} and Ali \c{C}i\c{c}i for discussions and complementary suggestions. This work is supported by the Scientific and Technological Research Council of Turkey (TUBITAK) Grant no. MFAG-114F461. Part of numerical calculations reported in this paper were performed at the National Academic Network and Information Center (ULAKBIM) of TUBITAK, High Performance anf Grid Computing Center (TRUBA Resources).


%%merlin.mbs apsrmp4-1.bst 2010-07-25 4.21a (PWD, AO, DPC) hacked
%Control: key (0)
%Control: author (3) reversed first dotless
%Control: editor formatted (0) differently from author
%Control: production of article title (0) allowed
%Control: page (1) range
%Control: year (0) verbatim
%Control: production of eprint (0) enabled
\begin{thebibliography}{312}%
\makeatletter
\providecommand \@ifxundefined [1]{%
 \@ifx{#1\undefined}
}%
\providecommand \@ifnum [1]{%
 \ifnum #1\expandafter \@firstoftwo
 \else \expandafter \@secondoftwo
 \fi
}%
\providecommand \@ifx [1]{%
 \ifx #1\expandafter \@firstoftwo
 \else \expandafter \@secondoftwo
 \fi
}%
\providecommand \natexlab [1]{#1}%
\providecommand \enquote  [1]{``#1''}%
\providecommand \bibnamefont  [1]{#1}%
\providecommand \bibfnamefont [1]{#1}%
\providecommand \citenamefont [1]{#1}%
\providecommand \href@noop [0]{\@secondoftwo}%
\providecommand \href [0]{\begingroup \@sanitize@url \@href}%
\providecommand \@href[1]{\@@startlink{#1}\@@href}%
\providecommand \@@href[1]{\endgroup#1\@@endlink}%
\providecommand \@sanitize@url [0]{\catcode `\\12\catcode `\$12\catcode
  `\&12\catcode `\#12\catcode `\^12\catcode `\_12\catcode `\%12\relax}%
\providecommand \@@startlink[1]{}%
\providecommand \@@endlink[0]{}%
\providecommand \url  [0]{\begingroup\@sanitize@url \@url }%
\providecommand \@url [1]{\endgroup\@href {#1}{\urlprefix }}%
\providecommand \urlprefix  [0]{URL }%
\providecommand \Eprint [0]{\href }%
\providecommand \doibase [0]{http://dx.doi.org/}%
\providecommand \selectlanguage [0]{\@gobble}%
\providecommand \bibinfo  [0]{\@secondoftwo}%
\providecommand \bibfield  [0]{\@secondoftwo}%
\providecommand \translation [1]{[#1]}%
\providecommand \BibitemOpen [0]{}%
\providecommand \bibitemStop [0]{}%
\providecommand \bibitemNoStop [0]{.\EOS\space}%
\providecommand \EOS [0]{\spacefactor3000\relax}%
\providecommand \BibitemShut  [1]{\csname bibitem#1\endcsname}%
\let\auto@bib@innerbib\@empty
%</preamble>
\bibitem [{\citenamefont {Aaronson}\ \emph {et~al.}(2019)\citenamefont
  {Aaronson}, \citenamefont {Cojocaru}, \citenamefont {Gheorghiu},\ and\
  \citenamefont {Kashefi}}]{ACGK19}%
  \BibitemOpen
  \bibfield  {author} {\bibinfo {author} {\bibnamefont {Aaronson},
  \bibfnamefont {Scott}}, \bibinfo {author} {\bibfnamefont {Alexandru}\
  \bibnamefont {Cojocaru}}, \bibinfo {author} {\bibfnamefont {Alexandru}\
  \bibnamefont {Gheorghiu}}, \ and\ \bibinfo {author} {\bibfnamefont {Elham}\
  \bibnamefont {Kashefi}}} (\bibinfo {year} {2019}),\ \bibfield  {title}
  {\enquote {\bibinfo {title} {Complexity-theoretic limitations on blind
  delegated quantum computation},}\ }in\ \href {\doibase
  10.4230/LIPIcs.ICALP.2019.6} {\emph {\bibinfo {booktitle} {46th International
  Colloquium on Automata, Languages, and Programming (ICALP 2019)}}},\ \bibinfo
  {series} {LIPIcs}, Vol.\ \bibinfo {volume} {132},\ \bibinfo {editor} {edited
  by\ \bibinfo {editor} {\bibfnamefont {Christel}\ \bibnamefont {Baier}},
  \bibinfo {editor} {\bibfnamefont {Ioannis}\ \bibnamefont {Chatzigiannakis}},
  \bibinfo {editor} {\bibfnamefont {Paola}\ \bibnamefont {Flocchini}}, \ and\
  \bibinfo {editor} {\bibfnamefont {Stefano}\ \bibnamefont {Leonardi}}}\
  (\bibinfo  {publisher} {Schloss Dagstuhl})\ pp.\ \bibinfo {pages}
  {6:1--6:13},\ \Eprint {http://arxiv.org/abs/arXiv:1704.08482}
  {arXiv:1704.08482} \BibitemShut {NoStop}%
\bibitem [{\citenamefont {Ac\'{\i}n}\ \emph {et~al.}(2007)\citenamefont
  {Ac\'{\i}n}, \citenamefont {Brunner}, \citenamefont {Gisin}, \citenamefont
  {Massar}, \citenamefont {Pironio},\ and\ \citenamefont {Scarani}}]{ABGMPS07}%
  \BibitemOpen
  \bibfield  {author} {\bibinfo {author} {\bibnamefont {Ac\'{\i}n},
  \bibfnamefont {Antonio}}, \bibinfo {author} {\bibfnamefont {Nicolas}\
  \bibnamefont {Brunner}}, \bibinfo {author} {\bibfnamefont {Nicolas}\
  \bibnamefont {Gisin}}, \bibinfo {author} {\bibfnamefont {Serge}\ \bibnamefont
  {Massar}}, \bibinfo {author} {\bibfnamefont {Stefano}\ \bibnamefont
  {Pironio}}, \ and\ \bibinfo {author} {\bibfnamefont {Valerio}\ \bibnamefont
  {Scarani}}} (\bibinfo {year} {2007}),\ \bibfield  {title} {\enquote {\bibinfo
  {title} {Device-independent security of quantum cryptography against
  collective attacks},}\ }\href {\doibase 10.1103/PhysRevLett.98.230501}
  {\bibfield  {journal} {\bibinfo  {journal} {Phys. Rev. Lett.}\ }\textbf
  {\bibinfo {volume} {98}},\ \bibinfo {pages} {230501}}\BibitemShut {NoStop}%
\bibitem [{\citenamefont {Ac\'{\i}n}\ \emph {et~al.}(2012)\citenamefont
  {Ac\'{\i}n}, \citenamefont {Massar},\ and\ \citenamefont {Pironio}}]{AMP12}%
  \BibitemOpen
  \bibfield  {author} {\bibinfo {author} {\bibnamefont {Ac\'{\i}n},
  \bibfnamefont {Antonio}}, \bibinfo {author} {\bibfnamefont {Serge}\
  \bibnamefont {Massar}}, \ and\ \bibinfo {author} {\bibfnamefont {Stefano}\
  \bibnamefont {Pironio}}} (\bibinfo {year} {2012}),\ \bibfield  {title}
  {\enquote {\bibinfo {title} {Randomness versus nonlocality and
  entanglement},}\ }\href {\doibase 10.1103/PhysRevLett.108.100402} {\bibfield
  {journal} {\bibinfo  {journal} {Phys. Rev. Lett.}\ }\textbf {\bibinfo
  {volume} {108}},\ \bibinfo {pages} {100402}},\ \Eprint
  {http://arxiv.org/abs/arXiv:1107.2754} {arXiv:1107.2754} \BibitemShut
  {NoStop}%
\bibitem [{\citenamefont {Aggarwal}\ \emph {et~al.}(2019)\citenamefont
  {Aggarwal}, \citenamefont {Chung}, \citenamefont {Lin},\ and\ \citenamefont
  {Vidick}}]{ACLV19}%
  \BibitemOpen
  \bibfield  {author} {\bibinfo {author} {\bibnamefont {Aggarwal},
  \bibfnamefont {Divesh}}, \bibinfo {author} {\bibfnamefont {Kai-Min}\
  \bibnamefont {Chung}}, \bibinfo {author} {\bibfnamefont {Han-Hsuan}\
  \bibnamefont {Lin}}, \ and\ \bibinfo {author} {\bibfnamefont {Thomas}\
  \bibnamefont {Vidick}}} (\bibinfo {year} {2019}),\ \bibfield  {title}
  {\enquote {\bibinfo {title} {A quantum-proof non-malleable extractor},}\ }in\
  \href {\doibase 10.1007/978-3-030-17656-3_16} {\emph {\bibinfo {booktitle}
  {Advances in Cryptology -- EUROCRYPT 2019}}},\ \bibinfo {editor} {edited by\
  \bibinfo {editor} {\bibfnamefont {Yuval}\ \bibnamefont {Ishai}}\ and\
  \bibinfo {editor} {\bibfnamefont {Vincent}\ \bibnamefont {Rijmen}}}\
  (\bibinfo  {publisher} {Springer})\ pp.\ \bibinfo {pages} {442--469},\
  \Eprint {http://arxiv.org/abs/arXiv:1710.00557} {arXiv:1710.00557}
  \BibitemShut {NoStop}%
\bibitem [{\citenamefont {Aharonov}\ \emph {et~al.}(2010)\citenamefont
  {Aharonov}, \citenamefont {{Ben-Or}},\ and\ \citenamefont {Eban}}]{ABE10}%
  \BibitemOpen
  \bibfield  {author} {\bibinfo {author} {\bibnamefont {Aharonov},
  \bibfnamefont {Dorit}}, \bibinfo {author} {\bibfnamefont {Michael}\
  \bibnamefont {{Ben-Or}}}, \ and\ \bibinfo {author} {\bibfnamefont {Elad}\
  \bibnamefont {Eban}}} (\bibinfo {year} {2010}),\ \bibfield  {title} {\enquote
  {\bibinfo {title} {Interactive proofs for quantum computations},}\ }in\
  \href@noop {} {\emph {\bibinfo {booktitle} {Proceedings of Innovations in
  Computer Science, ICS 2010}}}\ (\bibinfo  {publisher} {Tsinghua University
  Press})\ pp.\ \bibinfo {pages} {453--469},\ \Eprint
  {http://arxiv.org/abs/arXiv:0810.5375} {arXiv:0810.5375} \BibitemShut
  {NoStop}%
\bibitem [{\citenamefont {Ahlswede}\ and\ \citenamefont
  {Csisz\'ar}(1993)}]{AC93}%
  \BibitemOpen
  \bibfield  {author} {\bibinfo {author} {\bibnamefont {Ahlswede},
  \bibfnamefont {Rudolph}}, \ and\ \bibinfo {author} {\bibfnamefont {Imre}\
  \bibnamefont {Csisz\'ar}}} (\bibinfo {year} {1993}),\ \bibfield  {title}
  {\enquote {\bibinfo {title} {Common randomness in information theory and
  cryptography---{Part I}: Secret sharing},}\ }\href {\doibase
  10.1109/18.243431} {\bibfield  {journal} {\bibinfo  {journal} {IEEE Trans.
  Inf. Theory}\ }\textbf {\bibinfo {volume} {39}}~(\bibinfo {number} {4}),\
  \bibinfo {pages} {1121--1132}}\BibitemShut {NoStop}%
\bibitem [{\citenamefont {Alagic}\ \emph {et~al.}(2016)\citenamefont {Alagic},
  \citenamefont {Broadbent}, \citenamefont {Fefferman}, \citenamefont
  {Gagliardoni}, \citenamefont {Schaffner},\ and\ \citenamefont
  {St.~Jules}}]{ABFGSSJ16}%
  \BibitemOpen
  \bibfield  {author} {\bibinfo {author} {\bibnamefont {Alagic}, \bibfnamefont
  {Gorjan}}, \bibinfo {author} {\bibfnamefont {Anne}\ \bibnamefont
  {Broadbent}}, \bibinfo {author} {\bibfnamefont {Bill}\ \bibnamefont
  {Fefferman}}, \bibinfo {author} {\bibfnamefont {Tommaso}\ \bibnamefont
  {Gagliardoni}}, \bibinfo {author} {\bibfnamefont {Christian}\ \bibnamefont
  {Schaffner}}, \ and\ \bibinfo {author} {\bibfnamefont {Michael}\ \bibnamefont
  {St.~Jules}}} (\bibinfo {year} {2016}),\ \bibfield  {title} {\enquote
  {\bibinfo {title} {Computational security of quantum encryption},}\ }in\
  \href {\doibase 10.1007/978-3-319-49175-2_3} {\emph {\bibinfo {booktitle}
  {Proceedings of the 9th International Conference on Information Theoretic
  Security, ICITS 2016}}},\ \bibinfo {editor} {edited by\ \bibinfo {editor}
  {\bibfnamefont {Anderson~C.A.}\ \bibnamefont {Nascimento}}\ and\ \bibinfo
  {editor} {\bibfnamefont {Paulo}\ \bibnamefont {Barreto}}}\ (\bibinfo
  {publisher} {Springer})\ pp.\ \bibinfo {pages} {47--71},\ \Eprint
  {http://arxiv.org/abs/arXiv:1602.01441} {arXiv:1602.01441} \BibitemShut
  {NoStop}%
\bibitem [{\citenamefont {Alagic}\ \emph {et~al.}(2018)\citenamefont {Alagic},
  \citenamefont {Gagliardoni},\ and\ \citenamefont {Majenz}}]{AGM18}%
  \BibitemOpen
  \bibfield  {author} {\bibinfo {author} {\bibnamefont {Alagic}, \bibfnamefont
  {Gorjan}}, \bibinfo {author} {\bibfnamefont {Tommaso}\ \bibnamefont
  {Gagliardoni}}, \ and\ \bibinfo {author} {\bibfnamefont {Christian}\
  \bibnamefont {Majenz}}} (\bibinfo {year} {2018}),\ \bibfield  {title}
  {\enquote {\bibinfo {title} {Unforgeable quantum encryption},}\ }in\ \href
  {\doibase 10.1007/978-3-319-78372-7_16} {\emph {\bibinfo {booktitle}
  {Advances in Cryptology -- {EUROCRYPT} 2018, Proceedings, Part {III}}}},\
  \bibinfo {series} {LNCS}, Vol.\ \bibinfo {volume} {10822},\ \bibinfo {editor}
  {edited by\ \bibinfo {editor} {\bibfnamefont {Jesper~B.}\ \bibnamefont
  {Nielsen}}\ and\ \bibinfo {editor} {\bibfnamefont {Vincent}\ \bibnamefont
  {Rijmen}}}\ (\bibinfo  {publisher} {Springer})\ pp.\ \bibinfo {pages}
  {489--519},\ \Eprint {http://arxiv.org/abs/arXiv:1709.06539}
  {arXiv:1709.06539} \BibitemShut {NoStop}%
\bibitem [{\citenamefont {Alagic}\ and\ \citenamefont {Majenz}(2017)}]{AM17}%
  \BibitemOpen
  \bibfield  {author} {\bibinfo {author} {\bibnamefont {Alagic}, \bibfnamefont
  {Gorjan}}, \ and\ \bibinfo {author} {\bibfnamefont {Christian}\ \bibnamefont
  {Majenz}}} (\bibinfo {year} {2017}),\ \bibfield  {title} {\enquote {\bibinfo
  {title} {Quantum non-malleability and authentication},}\ }in\ \href {\doibase
  10.1007/978-3-319-63715-0_11} {\emph {\bibinfo {booktitle} {Advances in
  Cryptology -- CRYPTO 2017, Proceedings, Part II}}},\ \bibinfo {series}
  {LNCS}, Vol.\ \bibinfo {volume} {10402},\ \bibinfo {editor} {edited by\
  \bibinfo {editor} {\bibfnamefont {Jonathan}\ \bibnamefont {Katz}}\ and\
  \bibinfo {editor} {\bibfnamefont {Hovav}\ \bibnamefont {Shacham}}}\ (\bibinfo
   {publisher} {Springer})\ pp.\ \bibinfo {pages} {310--341},\ \Eprint
  {http://arxiv.org/abs/arXiv:1610.04214} {arXiv:1610.04214} \BibitemShut
  {NoStop}%
\bibitem [{\citenamefont {Alicki}\ and\ \citenamefont {Fannes}(2004)}]{AF04}%
  \BibitemOpen
  \bibfield  {author} {\bibinfo {author} {\bibnamefont {Alicki}, \bibfnamefont
  {Robert}}, \ and\ \bibinfo {author} {\bibfnamefont {Mark}\ \bibnamefont
  {Fannes}}} (\bibinfo {year} {2004}),\ \bibfield  {title} {\enquote {\bibinfo
  {title} {Continuity of quantum conditional information},}\ }\href {\doibase
  10.1088/0305-4470/37/5/L01} {\bibfield  {journal} {\bibinfo  {journal} {J.
  Phys. A}\ }\textbf {\bibinfo {volume} {37}},\ \bibinfo {pages}
  {L55--L57}}\BibitemShut {NoStop}%
\bibitem [{\citenamefont {Alon}\ \emph {et~al.}(2020)\citenamefont {Alon},
  \citenamefont {Chung}, \citenamefont {Chung}, \citenamefont {Huang},
  \citenamefont {Lee},\ and\ \citenamefont {Shen}}]{ACCHLS21}%
  \BibitemOpen
  \bibfield  {author} {\bibinfo {author} {\bibnamefont {Alon}, \bibfnamefont
  {Bar}}, \bibinfo {author} {\bibfnamefont {Hao}\ \bibnamefont {Chung}},
  \bibinfo {author} {\bibfnamefont {Kai-Min}\ \bibnamefont {Chung}}, \bibinfo
  {author} {\bibfnamefont {Mi-Ying}\ \bibnamefont {Huang}}, \bibinfo {author}
  {\bibfnamefont {Yi}~\bibnamefont {Lee}}, \ and\ \bibinfo {author}
  {\bibfnamefont {Yu-Ching}\ \bibnamefont {Shen}}} (\bibinfo {year} {2020}),\
  \href@noop {} {\enquote {\bibinfo {title} {Round efficient secure multiparty
  quantum computation with identifiable abort},}\ }\bibinfo {howpublished} {to
  appear at CRYPTO 2021},\ \bibinfo {note} {e-Print
  \href{http://eprint.iacr.org/2020/1464}{IACR 2020/1464}}\BibitemShut
  {NoStop}%
\bibitem [{\citenamefont {Ambainis}\ \emph {et~al.}(2009)\citenamefont
  {Ambainis}, \citenamefont {Bouda},\ and\ \citenamefont {Winter}}]{ABW09}%
  \BibitemOpen
  \bibfield  {author} {\bibinfo {author} {\bibnamefont {Ambainis},
  \bibfnamefont {Andris}}, \bibinfo {author} {\bibfnamefont {Jan}\ \bibnamefont
  {Bouda}}, \ and\ \bibinfo {author} {\bibfnamefont {Andreas}\ \bibnamefont
  {Winter}}} (\bibinfo {year} {2009}),\ \bibfield  {title} {\enquote {\bibinfo
  {title} {Non-malleable encryption of quantum information},}\ }\href {\doibase
  10.1063/1.3094756} {\bibfield  {journal} {\bibinfo  {journal} {J. Math.
  Phys.}\ }\textbf {\bibinfo {volume} {50}}~(\bibinfo {number} {4}),\ \bibinfo
  {pages} {042106}},\ \Eprint {http://arxiv.org/abs/arXiv:0808.0353}
  {arXiv:0808.0353} \BibitemShut {NoStop}%
\bibitem [{\citenamefont {Ambainis}\ \emph {et~al.}(2000)\citenamefont
  {Ambainis}, \citenamefont {Mosca}, \citenamefont {Tapp},\ and\ \citenamefont
  {de~Wolf}}]{AMTW00}%
  \BibitemOpen
  \bibfield  {author} {\bibinfo {author} {\bibnamefont {Ambainis},
  \bibfnamefont {Andris}}, \bibinfo {author} {\bibfnamefont {Michele}\
  \bibnamefont {Mosca}}, \bibinfo {author} {\bibfnamefont {Alain}\ \bibnamefont
  {Tapp}}, \ and\ \bibinfo {author} {\bibfnamefont {Ronald}\ \bibnamefont
  {de~Wolf}}} (\bibinfo {year} {2000}),\ \bibfield  {title} {\enquote {\bibinfo
  {title} {Private quantum channels},}\ }in\ \href@noop {} {\emph {\bibinfo
  {booktitle} {Proceedings of the 41st Symposium on Foundations of Computer
  Science, FOCS~'00}}}\ (\bibinfo  {publisher} {IEEE})\ p.\ \bibinfo {pages}
  {547},\ \Eprint {http://arxiv.org/abs/arXiv:quant-ph/0003101}
  {arXiv:quant-ph/0003101} \BibitemShut {NoStop}%
\bibitem [{\citenamefont {Ambainis}\ and\ \citenamefont {Smith}(2004)}]{AS04}%
  \BibitemOpen
  \bibfield  {author} {\bibinfo {author} {\bibnamefont {Ambainis},
  \bibfnamefont {Andris}}, \ and\ \bibinfo {author} {\bibfnamefont {Adam}\
  \bibnamefont {Smith}}} (\bibinfo {year} {2004}),\ \bibfield  {title}
  {\enquote {\bibinfo {title} {Small pseudo-random families of matrices:
  Derandomizing approximate quantum encryption},}\ }in\ \href {\doibase
  10.1007/978-3-540-27821-4_23} {\emph {\bibinfo {booktitle} {Proceedings of
  the 8th International Workshop on Randomization and Computation, RANDOM
  2004}}}\ (\bibinfo  {publisher} {Springer})\ pp.\ \bibinfo {pages}
  {249--260},\ \Eprint {http://arxiv.org/abs/arXiv:quant-ph/0404075}
  {arXiv:quant-ph/0404075} \BibitemShut {NoStop}%
\bibitem [{\citenamefont {Arnon-Friedman}(2018)}]{ArnonThesis}%
  \BibitemOpen
  \bibfield  {author} {\bibinfo {author} {\bibnamefont {Arnon-Friedman},
  \bibfnamefont {Rotem}}} (\bibinfo {year} {2018}),\ \emph {\bibinfo {title}
  {Reductions to IID in Device-independent Quantum Information Processing}},\
  \href@noop {} {Ph.D. thesis}\ (\bibinfo  {school} {Swiss Federal Institute of
  Technology (ETH) Zurich}),\ \Eprint {http://arxiv.org/abs/arXiv:1812.10922}
  {arXiv:1812.10922} \BibitemShut {NoStop}%
\bibitem [{\citenamefont {Arnon-Friedman}\ \emph {et~al.}(2018)\citenamefont
  {Arnon-Friedman}, \citenamefont {Dupuis}, \citenamefont {Fawzi},
  \citenamefont {Renner},\ and\ \citenamefont {Vidick}}]{ADFRV18}%
  \BibitemOpen
  \bibfield  {author} {\bibinfo {author} {\bibnamefont {Arnon-Friedman},
  \bibfnamefont {Rotem}}, \bibinfo {author} {\bibfnamefont {Fr{\'e}d{\'e}ric}\
  \bibnamefont {Dupuis}}, \bibinfo {author} {\bibfnamefont {Omar}\ \bibnamefont
  {Fawzi}}, \bibinfo {author} {\bibfnamefont {Renato}\ \bibnamefont {Renner}},
  \ and\ \bibinfo {author} {\bibfnamefont {Thomas}\ \bibnamefont {Vidick}}}
  (\bibinfo {year} {2018}),\ \bibfield  {title} {\enquote {\bibinfo {title}
  {Practical device-independent quantum cryptography via entropy
  accumulation},}\ }\href@noop {} {\bibfield  {journal} {\bibinfo  {journal}
  {Nat. Commun.}\ }\textbf {\bibinfo {volume} {9}}~(\bibinfo {number} {1}),\
  \bibinfo {pages} {1--11}}\BibitemShut {NoStop}%
\bibitem [{\citenamefont {Arnon-Friedman}\ \emph {et~al.}(2016)\citenamefont
  {Arnon-Friedman}, \citenamefont {Portmann},\ and\ \citenamefont
  {Scholz}}]{AFPS16}%
  \BibitemOpen
  \bibfield  {author} {\bibinfo {author} {\bibnamefont {Arnon-Friedman},
  \bibfnamefont {Rotem}}, \bibinfo {author} {\bibfnamefont {Christopher}\
  \bibnamefont {Portmann}}, \ and\ \bibinfo {author} {\bibfnamefont
  {Volkher~B.}\ \bibnamefont {Scholz}}} (\bibinfo {year} {2016}),\ \bibfield
  {title} {\enquote {\bibinfo {title} {Quantum-proof multi-source randomness
  extractors in the {Markov} model},}\ }in\ \href {\doibase
  10.4230/LIPIcs.TQC.2016.2} {\emph {\bibinfo {booktitle} {11th Conference on
  the Theory of Quantum Computation, Communication and Cryptography (TQC
  2016)}}},\ \bibinfo {series} {LIPIcs}, Vol.~\bibinfo {volume} {61}\ (\bibinfo
   {publisher} {Schloss Dagstuhl})\ pp.\ \bibinfo {pages} {2:1--2:34},\ \Eprint
  {http://arxiv.org/abs/arXiv:1510.06743} {arXiv:1510.06743} \BibitemShut
  {NoStop}%
\bibitem [{\citenamefont {Arnon-Friedman}\ \emph {et~al.}(2019)\citenamefont
  {Arnon-Friedman}, \citenamefont {Renner},\ and\ \citenamefont
  {Vidick}}]{AFRV19}%
  \BibitemOpen
  \bibfield  {author} {\bibinfo {author} {\bibnamefont {Arnon-Friedman},
  \bibfnamefont {Rotem}}, \bibinfo {author} {\bibfnamefont {Renato}\
  \bibnamefont {Renner}}, \ and\ \bibinfo {author} {\bibfnamefont {Thomas}\
  \bibnamefont {Vidick}}} (\bibinfo {year} {2019}),\ \bibfield  {title}
  {\enquote {\bibinfo {title} {Simple and tight device-independent security
  proofs},}\ }\href {\doibase 10.1137/18M1174726} {\bibfield  {journal}
  {\bibinfo  {journal} {SIAM J. Comput.}\ }\textbf {\bibinfo {volume}
  {48}}~(\bibinfo {number} {1}),\ \bibinfo {pages} {181--225}},\ \Eprint
  {http://arxiv.org/abs/arXiv:1607.01797} {arXiv:1607.01797} \BibitemShut
  {NoStop}%
\bibitem [{\citenamefont {Aspect}\ \emph {et~al.}(1982)\citenamefont {Aspect},
  \citenamefont {Dalibard},\ and\ \citenamefont {Roger}}]{Aspect82}%
  \BibitemOpen
  \bibfield  {author} {\bibinfo {author} {\bibnamefont {Aspect}, \bibfnamefont
  {Alain}}, \bibinfo {author} {\bibfnamefont {Jean}\ \bibnamefont {Dalibard}},
  \ and\ \bibinfo {author} {\bibfnamefont {G\'erard}\ \bibnamefont {Roger}}}
  (\bibinfo {year} {1982}),\ \bibfield  {title} {\enquote {\bibinfo {title}
  {Experimental test of {Bell}'s inequalities using time-varying analyzers},}\
  }\href {\doibase 10.1103/PhysRevLett.49.1804} {\bibfield  {journal} {\bibinfo
   {journal} {Phys. Rev. Lett.}\ }\textbf {\bibinfo {volume} {49}},\ \bibinfo
  {pages} {1804--1807}}\BibitemShut {NoStop}%
\bibitem [{\citenamefont {Aspect}\ \emph {et~al.}(1981)\citenamefont {Aspect},
  \citenamefont {Grangier},\ and\ \citenamefont {Roger}}]{Aspect81}%
  \BibitemOpen
  \bibfield  {author} {\bibinfo {author} {\bibnamefont {Aspect}, \bibfnamefont
  {Alain}}, \bibinfo {author} {\bibfnamefont {Philippe}\ \bibnamefont
  {Grangier}}, \ and\ \bibinfo {author} {\bibfnamefont {G\'erard}\ \bibnamefont
  {Roger}}} (\bibinfo {year} {1981}),\ \bibfield  {title} {\enquote {\bibinfo
  {title} {Experimental tests of realistic local theories via {Bell}'s
  theorem},}\ }\href {\doibase 10.1103/PhysRevLett.47.460} {\bibfield
  {journal} {\bibinfo  {journal} {Phys. Rev. Lett.}\ }\textbf {\bibinfo
  {volume} {47}},\ \bibinfo {pages} {460--463}}\BibitemShut {NoStop}%
\bibitem [{\citenamefont {Backes}\ \emph {et~al.}(2004)\citenamefont {Backes},
  \citenamefont {Pfitzmann},\ and\ \citenamefont {Waidner}}]{BPW04}%
  \BibitemOpen
  \bibfield  {author} {\bibinfo {author} {\bibnamefont {Backes}, \bibfnamefont
  {Michael}}, \bibinfo {author} {\bibfnamefont {Birgit}\ \bibnamefont
  {Pfitzmann}}, \ and\ \bibinfo {author} {\bibfnamefont {Michael}\ \bibnamefont
  {Waidner}}} (\bibinfo {year} {2004}),\ \bibfield  {title} {\enquote {\bibinfo
  {title} {A general composition theorem for secure reactive systems},}\ }in\
  \href {\doibase 10.1007/978-3-540-24638-1_19} {\emph {\bibinfo {booktitle}
  {Theory of Cryptography, Proceedings of TCC 2004}}},\ \bibinfo {series}
  {LNCS}, Vol.\ \bibinfo {volume} {2951}\ (\bibinfo  {publisher} {Springer})\
  pp.\ \bibinfo {pages} {336--354}\BibitemShut {NoStop}%
\bibitem [{\citenamefont {Backes}\ \emph {et~al.}(2007)\citenamefont {Backes},
  \citenamefont {Pfitzmann},\ and\ \citenamefont {Waidner}}]{BPW07}%
  \BibitemOpen
  \bibfield  {author} {\bibinfo {author} {\bibnamefont {Backes}, \bibfnamefont
  {Michael}}, \bibinfo {author} {\bibfnamefont {Birgit}\ \bibnamefont
  {Pfitzmann}}, \ and\ \bibinfo {author} {\bibfnamefont {Michael}\ \bibnamefont
  {Waidner}}} (\bibinfo {year} {2007}),\ \bibfield  {title} {\enquote {\bibinfo
  {title} {The reactive simulatability ({RSIM}) framework for asynchronous
  systems},}\ }\href {\doibase 10.1016/j.ic.2007.05.002} {\bibfield  {journal}
  {\bibinfo  {journal} {Inform. and Comput.}\ }\textbf {\bibinfo {volume}
  {205}}~(\bibinfo {number} {12}),\ \bibinfo {pages} {1685--1720}},\ \bibinfo
  {note} {extended version of~\textcite{PW01}, e-Print
  \href{http://eprint.iacr.org/2004/082}{IACR 2004/082}}\BibitemShut {NoStop}%
\bibitem [{\citenamefont {Badertscher}\ \emph {et~al.}(2020)\citenamefont
  {Badertscher}, \citenamefont {Cojocaru}, \citenamefont {Colisson},
  \citenamefont {Kashefi}, \citenamefont {Leichtle}, \citenamefont {Mantri},\
  and\ \citenamefont {Wallden}}]{BCCKLMW20}%
  \BibitemOpen
  \bibfield  {author} {\bibinfo {author} {\bibnamefont {Badertscher},
  \bibfnamefont {Christian}}, \bibinfo {author} {\bibfnamefont {Alexandru}\
  \bibnamefont {Cojocaru}}, \bibinfo {author} {\bibfnamefont {L{\'e}o}\
  \bibnamefont {Colisson}}, \bibinfo {author} {\bibfnamefont {Elham}\
  \bibnamefont {Kashefi}}, \bibinfo {author} {\bibfnamefont {Dominik}\
  \bibnamefont {Leichtle}}, \bibinfo {author} {\bibfnamefont {Atul}\
  \bibnamefont {Mantri}}, \ and\ \bibinfo {author} {\bibfnamefont {Petros}\
  \bibnamefont {Wallden}}} (\bibinfo {year} {2020}),\ \bibfield  {title}
  {\enquote {\bibinfo {title} {Security limitations of classical-client
  delegated quantum computing},}\ }in\ \href {\doibase
  10.1007/978-3-030-64834-3_23} {\emph {\bibinfo {booktitle} {Advances in
  Cryptology -- ASIACRYPT 2020, Proceedings, Part {II}}}},\ \bibinfo {series}
  {LNCS}, Vol.\ \bibinfo {volume} {12492},\ \bibinfo {editor} {edited by\
  \bibinfo {editor} {\bibfnamefont {Shiho}\ \bibnamefont {Moriai}}\ and\
  \bibinfo {editor} {\bibfnamefont {Huaxiong}\ \bibnamefont {Wang}}}\ (\bibinfo
   {publisher} {Springer},\ \bibinfo {address} {Cham})\ pp.\ \bibinfo {pages}
  {667--696},\ \Eprint {http://arxiv.org/abs/arXiv:2007.01668}
  {arXiv:2007.01668} \BibitemShut {NoStop}%
\bibitem [{\citenamefont {Banfi}\ \emph {et~al.}(2019)\citenamefont {Banfi},
  \citenamefont {Maurer}, \citenamefont {Portmann},\ and\ \citenamefont
  {Zhu}}]{BMPZ19}%
  \BibitemOpen
  \bibfield  {author} {\bibinfo {author} {\bibnamefont {Banfi}, \bibfnamefont
  {Fabio}}, \bibinfo {author} {\bibfnamefont {Ueli}\ \bibnamefont {Maurer}},
  \bibinfo {author} {\bibfnamefont {Christopher}\ \bibnamefont {Portmann}}, \
  and\ \bibinfo {author} {\bibfnamefont {Jiamin}\ \bibnamefont {Zhu}}}
  (\bibinfo {year} {2019}),\ \bibfield  {title} {\enquote {\bibinfo {title}
  {Composable and finite computational security of quantum message
  transmission},}\ }in\ \href {\doibase 10.1007/978-3-030-36030-6_12} {\emph
  {\bibinfo {booktitle} {Theory of Cryptography, Proceedings of {TCC} 2019,
  Part {I}}}},\ \bibinfo {series} {LNCS}, Vol.\ \bibinfo {volume} {11891}\
  (\bibinfo  {publisher} {Springer})\ pp.\ \bibinfo {pages} {282--311},\
  \Eprint {http://arxiv.org/abs/arXiv:1908.03436} {arXiv:1908.03436}
  \BibitemShut {NoStop}%
\bibitem [{\citenamefont {Barnum}\ \emph {et~al.}(2002)\citenamefont {Barnum},
  \citenamefont {Cr{\'e}peau}, \citenamefont {Gottesman}, \citenamefont
  {Smith},\ and\ \citenamefont {Tapp}}]{BCGST02}%
  \BibitemOpen
  \bibfield  {author} {\bibinfo {author} {\bibnamefont {Barnum}, \bibfnamefont
  {Howard}}, \bibinfo {author} {\bibfnamefont {Claude}\ \bibnamefont
  {Cr{\'e}peau}}, \bibinfo {author} {\bibfnamefont {Daniel}\ \bibnamefont
  {Gottesman}}, \bibinfo {author} {\bibfnamefont {Adam}\ \bibnamefont {Smith}},
  \ and\ \bibinfo {author} {\bibfnamefont {Alain}\ \bibnamefont {Tapp}}}
  (\bibinfo {year} {2002}),\ \bibfield  {title} {\enquote {\bibinfo {title}
  {Authentication of quantum messages},}\ }in\ \href {\doibase
  10.1109/SFCS.2002.1181969} {\emph {\bibinfo {booktitle} {Proceedings of the
  43rd Symposium on Foundations of Computer Science, FOCS~'02}}}\ (\bibinfo
  {publisher} {IEEE})\ pp.\ \bibinfo {pages} {449--458},\ \Eprint
  {http://arxiv.org/abs/arXiv:quant-ph/0205128} {arXiv:quant-ph/0205128}
  \BibitemShut {NoStop}%
\bibitem [{\citenamefont {Barrett}\ \emph {et~al.}(2013)\citenamefont
  {Barrett}, \citenamefont {Colbeck},\ and\ \citenamefont {Kent}}]{BCK13}%
  \BibitemOpen
  \bibfield  {author} {\bibinfo {author} {\bibnamefont {Barrett}, \bibfnamefont
  {Jonathan}}, \bibinfo {author} {\bibfnamefont {Roger}\ \bibnamefont
  {Colbeck}}, \ and\ \bibinfo {author} {\bibfnamefont {Adrian}\ \bibnamefont
  {Kent}}} (\bibinfo {year} {2013}),\ \bibfield  {title} {\enquote {\bibinfo
  {title} {Memory attacks on device-independent quantum cryptography},}\ }\href
  {\doibase 10.1103/PhysRevLett.110.010503} {\bibfield  {journal} {\bibinfo
  {journal} {Phys. Rev. Lett.}\ }\textbf {\bibinfo {volume} {110}},\ \bibinfo
  {pages} {010503}},\ \Eprint {http://arxiv.org/abs/arXiv:1201.4407}
  {arXiv:1201.4407} \BibitemShut {NoStop}%
\bibitem [{\citenamefont {Barrett}\ \emph {et~al.}(2005)\citenamefont
  {Barrett}, \citenamefont {Hardy},\ and\ \citenamefont {Kent}}]{BHK05}%
  \BibitemOpen
  \bibfield  {author} {\bibinfo {author} {\bibnamefont {Barrett}, \bibfnamefont
  {Jonathan}}, \bibinfo {author} {\bibfnamefont {Lucien}\ \bibnamefont
  {Hardy}}, \ and\ \bibinfo {author} {\bibfnamefont {Adrian}\ \bibnamefont
  {Kent}}} (\bibinfo {year} {2005}),\ \bibfield  {title} {\enquote {\bibinfo
  {title} {No signaling and quantum key distribution},}\ }\href {\doibase
  10.1103/PhysRevLett.95.010503} {\bibfield  {journal} {\bibinfo  {journal}
  {Phys. Rev. Lett.}\ }\textbf {\bibinfo {volume} {95}}~(\bibinfo {number}
  {1}),\ \bibinfo {pages} {1--4}}\BibitemShut {NoStop}%
\bibitem [{\citenamefont {Baumgratz}\ \emph {et~al.}(2014)\citenamefont
  {Baumgratz}, \citenamefont {Cramer},\ and\ \citenamefont {Plenio}}]{BCP14}%
  \BibitemOpen
  \bibfield  {author} {\bibinfo {author} {\bibnamefont {Baumgratz},
  \bibfnamefont {Tillmann}}, \bibinfo {author} {\bibfnamefont {Marcus}\
  \bibnamefont {Cramer}}, \ and\ \bibinfo {author} {\bibfnamefont {Martin~B.}\
  \bibnamefont {Plenio}}} (\bibinfo {year} {2014}),\ \bibfield  {title}
  {\enquote {\bibinfo {title} {Quantifying coherence},}\ }\href {\doibase
  10.1103/PhysRevLett.113.140401} {\bibfield  {journal} {\bibinfo  {journal}
  {Phys. Rev. Lett.}\ }\textbf {\bibinfo {volume} {113}},\ \bibinfo {pages}
  {140401}},\ \Eprint {http://arxiv.org/abs/arxiv:1311.0275} {arxiv:1311.0275}
  \BibitemShut {NoStop}%
\bibitem [{\citenamefont {Beaver}(1992)}]{Bea92}%
  \BibitemOpen
  \bibfield  {author} {\bibinfo {author} {\bibnamefont {Beaver}, \bibfnamefont
  {Donald}}} (\bibinfo {year} {1992}),\ \bibfield  {title} {\enquote {\bibinfo
  {title} {Foundations of secure interactive computing},}\ }in\ \href {\doibase
  10.1007/3-540-46766-1_31} {\emph {\bibinfo {booktitle} {Advances in
  Cryptology -- CRYPTO~'91}}},\ \bibinfo {series} {LNCS}, Vol.\ \bibinfo
  {volume} {576}\ (\bibinfo  {publisher} {Springer})\ pp.\ \bibinfo {pages}
  {377--391}\BibitemShut {NoStop}%
\bibitem [{\citenamefont {Bell}(1964)}]{Bell64}%
  \BibitemOpen
  \bibfield  {author} {\bibinfo {author} {\bibnamefont {Bell}, \bibfnamefont
  {John~Stewart}}} (\bibinfo {year} {1964}),\ \bibfield  {title} {\enquote
  {\bibinfo {title} {On the {E}instein-{P}odolsky-{R}osen paradox},}\
  }\href@noop {} {\bibfield  {journal} {\bibinfo  {journal} {Physics}\ }\textbf
  {\bibinfo {volume} {1}}~(\bibinfo {number} {3}),\ \bibinfo {pages}
  {195--200}}\BibitemShut {NoStop}%
\bibitem [{\citenamefont {Bell}(1966)}]{Bell66}%
  \BibitemOpen
  \bibfield  {author} {\bibinfo {author} {\bibnamefont {Bell}, \bibfnamefont
  {John~Stewart}}} (\bibinfo {year} {1966}),\ \bibfield  {title} {\enquote
  {\bibinfo {title} {On the problem of hidden variables in quantum
  mechanics},}\ }\href {\doibase 10.1103/RevModPhys.38.447} {\bibfield
  {journal} {\bibinfo  {journal} {Rev. Mod. Phys.}\ }\textbf {\bibinfo {volume}
  {38}},\ \bibinfo {pages} {447--452}}\BibitemShut {NoStop}%
\bibitem [{\citenamefont {Bell}\ and\ \citenamefont {Aspect}(2004)}]{BellFree}%
  \BibitemOpen
  \bibfield  {author} {\bibinfo {author} {\bibnamefont {Bell}, \bibfnamefont
  {John~Stewart}}, \ and\ \bibinfo {author} {\bibfnamefont {Alain}\
  \bibnamefont {Aspect}}} (\bibinfo {year} {2004}),\ \enquote {\bibinfo {title}
  {Free variables and local causality},}\ in\ \href {\doibase
  10.1017/CBO9780511815676.014} {\emph {\bibinfo {booktitle} {Speakable and
  Unspeakable in Quantum Mechanics: Collected Papers on Quantum Philosophy}}},\
  Chap.~\bibinfo {chapter} {12},\ \bibinfo {edition} {2nd}\ ed.\ (\bibinfo
  {publisher} {Cambridge University Press})\ pp.\ \bibinfo {pages}
  {100--104}\BibitemShut {NoStop}%
\bibitem [{\citenamefont {Bellare}\ \emph {et~al.}(1997)\citenamefont
  {Bellare}, \citenamefont {Desai}, \citenamefont {Jokipii},\ and\
  \citenamefont {Rogaway}}]{BDJR97}%
  \BibitemOpen
  \bibfield  {author} {\bibinfo {author} {\bibnamefont {Bellare}, \bibfnamefont
  {Mihir}}, \bibinfo {author} {\bibfnamefont {Anand}\ \bibnamefont {Desai}},
  \bibinfo {author} {\bibfnamefont {Eron}\ \bibnamefont {Jokipii}}, \ and\
  \bibinfo {author} {\bibfnamefont {Phillip}\ \bibnamefont {Rogaway}}}
  (\bibinfo {year} {1997}),\ \bibfield  {title} {\enquote {\bibinfo {title} {A
  concrete security treatment of symmetric encryption},}\ }in\ \href {\doibase
  10.1109/SFCS.1997.646128} {\emph {\bibinfo {booktitle} {Proceedings of the
  38th Annual Symposium on Foundations of Computer Science, FOCS~'97}}}\
  (\bibinfo  {publisher} {IEEE})\ pp.\ \bibinfo {pages} {394--403}\BibitemShut
  {NoStop}%
\bibitem [{\citenamefont {Bellare}\ \emph {et~al.}(1998)\citenamefont
  {Bellare}, \citenamefont {Desai}, \citenamefont {Pointcheval},\ and\
  \citenamefont {Rogaway}}]{BDPR98}%
  \BibitemOpen
  \bibfield  {author} {\bibinfo {author} {\bibnamefont {Bellare}, \bibfnamefont
  {Mihir}}, \bibinfo {author} {\bibfnamefont {Anand}\ \bibnamefont {Desai}},
  \bibinfo {author} {\bibfnamefont {David}\ \bibnamefont {Pointcheval}}, \ and\
  \bibinfo {author} {\bibfnamefont {Phillip}\ \bibnamefont {Rogaway}}}
  (\bibinfo {year} {1998}),\ \bibfield  {title} {\enquote {\bibinfo {title}
  {Relations among notions of security for public-key encryption schemes},}\
  }in\ \href {\doibase 10.1007/BFb0055718} {\emph {\bibinfo {booktitle}
  {Advances in Cryptology -- CRYPTO~'98}}}\ (\bibinfo  {publisher} {Springer})\
  pp.\ \bibinfo {pages} {26--45}\BibitemShut {NoStop}%
\bibitem [{\citenamefont {Bellare}\ and\ \citenamefont {Rogaway}(2006)}]{BR06}%
  \BibitemOpen
  \bibfield  {author} {\bibinfo {author} {\bibnamefont {Bellare}, \bibfnamefont
  {Mihir}}, \ and\ \bibinfo {author} {\bibfnamefont {Phillip}\ \bibnamefont
  {Rogaway}}} (\bibinfo {year} {2006}),\ \bibfield  {title} {\enquote {\bibinfo
  {title} {The security of triple encryption and a framework for code-based
  game-playing proofs},}\ }in\ \href {\doibase 10.1007/11761679_25} {\emph
  {\bibinfo {booktitle} {Advances in Cryptology -- EUROCRYPT 2006}}},\ \bibinfo
  {series} {LNCS}, Vol.\ \bibinfo {volume} {4004},\ \bibinfo {editor} {edited
  by\ \bibinfo {editor} {\bibfnamefont {Serge}\ \bibnamefont {Vaudenay}}}\
  (\bibinfo  {publisher} {Springer})\ pp.\ \bibinfo {pages} {409--426},\
  \bibinfo {note} {e-Print \href{http://eprint.iacr.org/2004/331}{IACR
  2004/331}}\BibitemShut {NoStop}%
\bibitem [{\citenamefont {{Ben-Aroya}}\ and\ \citenamefont
  {{Ta-Shma}}(2012)}]{BT12}%
  \BibitemOpen
  \bibfield  {author} {\bibinfo {author} {\bibnamefont {{Ben-Aroya}},
  \bibfnamefont {Avraham}}, \ and\ \bibinfo {author} {\bibfnamefont {Amnon}\
  \bibnamefont {{Ta-Shma}}}} (\bibinfo {year} {2012}),\ \bibfield  {title}
  {\enquote {\bibinfo {title} {Better short-seed quantum-proof extractors},}\
  }\href {\doibase 10.1016/j.tcs.2011.11.036} {\bibfield  {journal} {\bibinfo
  {journal} {Theoretical Computer Science}\ }\textbf {\bibinfo {volume}
  {419}},\ \bibinfo {pages} {17--25}},\ \Eprint
  {http://arxiv.org/abs/arXiv:1004.3737} {arXiv:1004.3737} \BibitemShut
  {NoStop}%
\bibitem [{\citenamefont {{Ben-Or}}\ \emph {et~al.}(2006)\citenamefont
  {{Ben-Or}}, \citenamefont {Cr\'epeau}, \citenamefont {Gottesman},
  \citenamefont {Hassidim},\ and\ \citenamefont {Smith}}]{BCGHS06}%
  \BibitemOpen
  \bibfield  {author} {\bibinfo {author} {\bibnamefont {{Ben-Or}},
  \bibfnamefont {Michael}}, \bibinfo {author} {\bibfnamefont {Claude}\
  \bibnamefont {Cr\'epeau}}, \bibinfo {author} {\bibfnamefont {Daniel}\
  \bibnamefont {Gottesman}}, \bibinfo {author} {\bibfnamefont {Avinatan}\
  \bibnamefont {Hassidim}}, \ and\ \bibinfo {author} {\bibfnamefont {Adam}\
  \bibnamefont {Smith}}} (\bibinfo {year} {2006}),\ \bibfield  {title}
  {\enquote {\bibinfo {title} {Secure multiparty quantum computation with
  (only) a strict honest majority},}\ }in\ \href {\doibase
  10.1109/FOCS.2006.68} {\emph {\bibinfo {booktitle} {Proceedings of the 47th
  Symposium on Foundations of Computer Science, FOCS~'06}}},\ pp.\ \bibinfo
  {pages} {249--260},\ \Eprint {http://arxiv.org/abs/arXiv:0801.1544}
  {arXiv:0801.1544} \BibitemShut {NoStop}%
\bibitem [{\citenamefont {{Ben-Or}}\ \emph {et~al.}(2005)\citenamefont
  {{Ben-Or}}, \citenamefont {Horodecki}, \citenamefont {Leung}, \citenamefont
  {Mayers},\ and\ \citenamefont {Oppenheim}}]{BHLMO05}%
  \BibitemOpen
  \bibfield  {author} {\bibinfo {author} {\bibnamefont {{Ben-Or}},
  \bibfnamefont {Michael}}, \bibinfo {author} {\bibfnamefont {Micha\l{}}\
  \bibnamefont {Horodecki}}, \bibinfo {author} {\bibfnamefont {Debbie}\
  \bibnamefont {Leung}}, \bibinfo {author} {\bibfnamefont {Dominic}\
  \bibnamefont {Mayers}}, \ and\ \bibinfo {author} {\bibfnamefont {Jonathan}\
  \bibnamefont {Oppenheim}}} (\bibinfo {year} {2005}),\ \bibfield  {title}
  {\enquote {\bibinfo {title} {The universal composable security of quantum key
  distribution},}\ }in\ \href {\doibase 10.1007/978-3-540-30576-7_21} {\emph
  {\bibinfo {booktitle} {Theory of Cryptography, Proceedings of TCC 2005}}},\
  \bibinfo {series} {LNCS}, Vol.\ \bibinfo {volume} {3378}\ (\bibinfo
  {publisher} {Springer})\ pp.\ \bibinfo {pages} {386--406},\ \Eprint
  {http://arxiv.org/abs/arXiv:quant-ph/0409078} {arXiv:quant-ph/0409078}
  \BibitemShut {NoStop}%
\bibitem [{\citenamefont {{Ben-Or}}\ and\ \citenamefont {Mayers}(2004)}]{BM04}%
  \BibitemOpen
  \bibfield  {author} {\bibinfo {author} {\bibnamefont {{Ben-Or}},
  \bibfnamefont {Michael}}, \ and\ \bibinfo {author} {\bibfnamefont {Dominic}\
  \bibnamefont {Mayers}}} (\bibinfo {year} {2004}),\ \href@noop {} {\enquote
  {\bibinfo {title} {General security definition and composability for quantum
  \& classical protocols},}\ }\bibinfo {howpublished} {e-Print},\ \Eprint
  {http://arxiv.org/abs/arXiv:quant-ph/0409062} {arXiv:quant-ph/0409062}
  \BibitemShut {NoStop}%
\bibitem [{\citenamefont {Bennett}\ \emph
  {et~al.}(1996{\natexlab{a}})\citenamefont {Bennett}, \citenamefont
  {Bernstein}, \citenamefont {Popescu},\ and\ \citenamefont
  {Schumacher}}]{Bennett96}%
  \BibitemOpen
  \bibfield  {author} {\bibinfo {author} {\bibnamefont {Bennett}, \bibfnamefont
  {Charles~H}}, \bibinfo {author} {\bibfnamefont {Herbert~J.}\ \bibnamefont
  {Bernstein}}, \bibinfo {author} {\bibfnamefont {Sandu}\ \bibnamefont
  {Popescu}}, \ and\ \bibinfo {author} {\bibfnamefont {Benjamin}\ \bibnamefont
  {Schumacher}}} (\bibinfo {year} {1996}{\natexlab{a}}),\ \bibfield  {title}
  {\enquote {\bibinfo {title} {Concentrating partial entanglement by local
  operations},}\ }\href {\doibase 10.1103/PhysRevA.53.2046} {\bibfield
  {journal} {\bibinfo  {journal} {Phys. Rev. A}\ }\textbf {\bibinfo {volume}
  {53}},\ \bibinfo {pages} {2046--2052}}\BibitemShut {NoStop}%
\bibitem [{\citenamefont {Bennett}\ \emph
  {et~al.}(1992{\natexlab{a}})\citenamefont {Bennett}, \citenamefont
  {Bessette}, \citenamefont {Brassard}, \citenamefont {Salvail},\ and\
  \citenamefont {Smolin}}]{BBBSS92}%
  \BibitemOpen
  \bibfield  {author} {\bibinfo {author} {\bibnamefont {Bennett}, \bibfnamefont
  {Charles~H}}, \bibinfo {author} {\bibfnamefont {Fran{\c{c}}ois}\ \bibnamefont
  {Bessette}}, \bibinfo {author} {\bibfnamefont {Gilles}\ \bibnamefont
  {Brassard}}, \bibinfo {author} {\bibfnamefont {Louis}\ \bibnamefont
  {Salvail}}, \ and\ \bibinfo {author} {\bibfnamefont {John}\ \bibnamefont
  {Smolin}}} (\bibinfo {year} {1992}{\natexlab{a}}),\ \bibfield  {title}
  {\enquote {\bibinfo {title} {Experimental quantum cryptography},}\ }\href
  {\doibase 10.1007/BF00191318} {\bibfield  {journal} {\bibinfo  {journal} {J.
  Crypt.}\ }\textbf {\bibinfo {volume} {5}}~(\bibinfo {number} {1}),\ \bibinfo
  {pages} {3--28}}\BibitemShut {NoStop}%
\bibitem [{\citenamefont {Bennett}\ and\ \citenamefont
  {Brassard}(1984)}]{BB84}%
  \BibitemOpen
  \bibfield  {author} {\bibinfo {author} {\bibnamefont {Bennett}, \bibfnamefont
  {Charles~H}}, \ and\ \bibinfo {author} {\bibfnamefont {Gilles}\ \bibnamefont
  {Brassard}}} (\bibinfo {year} {1984}),\ \bibfield  {title} {\enquote
  {\bibinfo {title} {Quantum cryptography: Public key distribution and coin
  tossing},}\ }in\ \href@noop {} {\emph {\bibinfo {booktitle} {Proceedings of
  IEEE International Conference on Computers, Systems, and Signal
  Processing}}},\ pp.\ \bibinfo {pages} {175--179}\BibitemShut {NoStop}%
\bibitem [{\citenamefont {Bennett}\ \emph {et~al.}(1995)\citenamefont
  {Bennett}, \citenamefont {Brassard}, \citenamefont {Cr{\'e}peau},\ and\
  \citenamefont {Maurer}}]{BBCM95}%
  \BibitemOpen
  \bibfield  {author} {\bibinfo {author} {\bibnamefont {Bennett}, \bibfnamefont
  {Charles~H}}, \bibinfo {author} {\bibfnamefont {Gilles}\ \bibnamefont
  {Brassard}}, \bibinfo {author} {\bibfnamefont {Claude}\ \bibnamefont
  {Cr{\'e}peau}}, \ and\ \bibinfo {author} {\bibfnamefont {Ueli}\ \bibnamefont
  {Maurer}}} (\bibinfo {year} {1995}),\ \bibfield  {title} {\enquote {\bibinfo
  {title} {Generalized privacy amplification},}\ }\href {\doibase
  10.1109/18.476316} {\bibfield  {journal} {\bibinfo  {journal} {IEEE Trans.
  Inf. Theory}\ }\textbf {\bibinfo {volume} {41}}~(\bibinfo {number} {6}),\
  \bibinfo {pages} {1915--1923}}\BibitemShut {NoStop}%
\bibitem [{\citenamefont {Bennett}\ \emph
  {et~al.}(1992{\natexlab{b}})\citenamefont {Bennett}, \citenamefont
  {Brassard}, \citenamefont {Cr{\'{e}}peau},\ and\ \citenamefont
  {Skubiszewska}}]{BBCS92}%
  \BibitemOpen
  \bibfield  {author} {\bibinfo {author} {\bibnamefont {Bennett}, \bibfnamefont
  {Charles~H}}, \bibinfo {author} {\bibfnamefont {Gilles}\ \bibnamefont
  {Brassard}}, \bibinfo {author} {\bibfnamefont {Claude}\ \bibnamefont
  {Cr{\'{e}}peau}}, \ and\ \bibinfo {author} {\bibfnamefont
  {Marie{-}H{\'{e}}l{\`{e}}ne}\ \bibnamefont {Skubiszewska}}} (\bibinfo {year}
  {1992}{\natexlab{b}}),\ \bibfield  {title} {\enquote {\bibinfo {title}
  {Practical quantum oblivious transfer},}\ }in\ \href {\doibase
  10.1007/3-540-46766-1_29} {\emph {\bibinfo {booktitle} {Advances in
  Cryptology -- CRYPTO~'91}}},\ \bibinfo {series} {LNCS}, Vol.\ \bibinfo
  {volume} {576}\ (\bibinfo  {publisher} {Springer})\ pp.\ \bibinfo {pages}
  {351--366}\BibitemShut {NoStop}%
\bibitem [{\citenamefont {Bennett}\ \emph
  {et~al.}(1992{\natexlab{c}})\citenamefont {Bennett}, \citenamefont
  {Brassard},\ and\ \citenamefont {Mermin}}]{BBM92}%
  \BibitemOpen
  \bibfield  {author} {\bibinfo {author} {\bibnamefont {Bennett}, \bibfnamefont
  {Charles~H}}, \bibinfo {author} {\bibfnamefont {Gilles}\ \bibnamefont
  {Brassard}}, \ and\ \bibinfo {author} {\bibfnamefont {N.~David}\ \bibnamefont
  {Mermin}}} (\bibinfo {year} {1992}{\natexlab{c}}),\ \bibfield  {title}
  {\enquote {\bibinfo {title} {Quantum cryptography without {Bell}'s
  theorem},}\ }\href {\doibase 10.1103/PhysRevLett.68.557} {\bibfield
  {journal} {\bibinfo  {journal} {Phys. Rev. Lett.}\ }\textbf {\bibinfo
  {volume} {68}},\ \bibinfo {pages} {557--559}}\BibitemShut {NoStop}%
\bibitem [{\citenamefont {Bennett}\ \emph
  {et~al.}(1996{\natexlab{b}})\citenamefont {Bennett}, \citenamefont
  {Brassard}, \citenamefont {Popescu}, \citenamefont {Schumacher},
  \citenamefont {Smolin},\ and\ \citenamefont {Wootters}}]{Benett96b}%
  \BibitemOpen
  \bibfield  {author} {\bibinfo {author} {\bibnamefont {Bennett}, \bibfnamefont
  {Charles~H}}, \bibinfo {author} {\bibfnamefont {Gilles}\ \bibnamefont
  {Brassard}}, \bibinfo {author} {\bibfnamefont {Sandu}\ \bibnamefont
  {Popescu}}, \bibinfo {author} {\bibfnamefont {Benjamin}\ \bibnamefont
  {Schumacher}}, \bibinfo {author} {\bibfnamefont {John~A.}\ \bibnamefont
  {Smolin}}, \ and\ \bibinfo {author} {\bibfnamefont {William~K.}\ \bibnamefont
  {Wootters}}} (\bibinfo {year} {1996}{\natexlab{b}}),\ \bibfield  {title}
  {\enquote {\bibinfo {title} {Purification of noisy entanglement and faithful
  teleportation via noisy channels},}\ }\href {\doibase
  10.1103/PhysRevLett.76.722} {\bibfield  {journal} {\bibinfo  {journal} {Phys.
  Rev. Lett.}\ }\textbf {\bibinfo {volume} {76}},\ \bibinfo {pages}
  {722--725}}\BibitemShut {NoStop}%
\bibitem [{\citenamefont {Bennett}\ \emph {et~al.}(1988)\citenamefont
  {Bennett}, \citenamefont {Brassard},\ and\ \citenamefont {Robert}}]{BBR88}%
  \BibitemOpen
  \bibfield  {author} {\bibinfo {author} {\bibnamefont {Bennett}, \bibfnamefont
  {Charles~H}}, \bibinfo {author} {\bibfnamefont {Gilles}\ \bibnamefont
  {Brassard}}, \ and\ \bibinfo {author} {\bibfnamefont {Jean-Marc}\
  \bibnamefont {Robert}}} (\bibinfo {year} {1988}),\ \bibfield  {title}
  {\enquote {\bibinfo {title} {Privacy amplification by public discussion},}\
  }\href {\doibase 10.1137/0217014} {\bibfield  {journal} {\bibinfo  {journal}
  {SIAM J. Comput.}\ }\textbf {\bibinfo {volume} {17}}~(\bibinfo {number}
  {2}),\ \bibinfo {pages} {210--229}}\BibitemShut {NoStop}%
\bibitem [{\citenamefont {Berta}\ \emph {et~al.}(2010)\citenamefont {Berta},
  \citenamefont {Christandl}, \citenamefont {Colbeck}, \citenamefont {Renes},\
  and\ \citenamefont {Renner}}]{Berta10}%
  \BibitemOpen
  \bibfield  {author} {\bibinfo {author} {\bibnamefont {Berta}, \bibfnamefont
  {Mario}}, \bibinfo {author} {\bibfnamefont {Matthias}\ \bibnamefont
  {Christandl}}, \bibinfo {author} {\bibfnamefont {Roger}\ \bibnamefont
  {Colbeck}}, \bibinfo {author} {\bibfnamefont {Joseph~M.}\ \bibnamefont
  {Renes}}, \ and\ \bibinfo {author} {\bibfnamefont {Renato}\ \bibnamefont
  {Renner}}} (\bibinfo {year} {2010}),\ \bibfield  {title} {\enquote {\bibinfo
  {title} {The uncertainty principle in the presence of quantum memory},}\
  }\href {\doibase 10.1038/nphys1734} {\bibfield  {journal} {\bibinfo
  {journal} {Nat. Phys.}\ }\textbf {\bibinfo {volume} {6}}~(\bibinfo {number}
  {9}),\ \bibinfo {pages} {659--662}}\BibitemShut {NoStop}%
\bibitem [{\citenamefont {Berta}\ \emph {et~al.}(2017)\citenamefont {Berta},
  \citenamefont {Fawzi},\ and\ \citenamefont {Scholz}}]{BFS17}%
  \BibitemOpen
  \bibfield  {author} {\bibinfo {author} {\bibnamefont {Berta}, \bibfnamefont
  {Mario}}, \bibinfo {author} {\bibfnamefont {Omar}\ \bibnamefont {Fawzi}}, \
  and\ \bibinfo {author} {\bibfnamefont {Volkher~B.}\ \bibnamefont {Scholz}}}
  (\bibinfo {year} {2017}),\ \bibfield  {title} {\enquote {\bibinfo {title}
  {Quantum-proof randomness extractors via operator space theory},}\ }\href
  {\doibase 10.1109/TIT.2016.2627531} {\bibfield  {journal} {\bibinfo
  {journal} {IEEE Trans. Inf. Theory}\ }\textbf {\bibinfo {volume}
  {63}}~(\bibinfo {number} {4}),\ \bibinfo {pages} {2480--2503}},\ \Eprint
  {http://arxiv.org/abs/arxiv:1409.3563} {arxiv:1409.3563} \BibitemShut
  {NoStop}%
\bibitem [{\citenamefont {Biham}\ \emph {et~al.}(2000)\citenamefont {Biham},
  \citenamefont {Boyer}, \citenamefont {Boykin}, \citenamefont {Mor},\ and\
  \citenamefont {Roychowdhury}}]{BBBMR00}%
  \BibitemOpen
  \bibfield  {author} {\bibinfo {author} {\bibnamefont {Biham}, \bibfnamefont
  {Eli}}, \bibinfo {author} {\bibfnamefont {Michel}\ \bibnamefont {Boyer}},
  \bibinfo {author} {\bibfnamefont {P.~Oscar}\ \bibnamefont {Boykin}}, \bibinfo
  {author} {\bibfnamefont {Tal}\ \bibnamefont {Mor}}, \ and\ \bibinfo {author}
  {\bibfnamefont {Vwani}\ \bibnamefont {Roychowdhury}}} (\bibinfo {year}
  {2000}),\ \bibfield  {title} {\enquote {\bibinfo {title} {A proof of the
  security of quantum key distribution (extended abstract)},}\ }in\ \href
  {\doibase 10.1145/335305.335406} {\emph {\bibinfo {booktitle} {Proceedings of
  the 32nd Symposium on Theory of Computing, STOC~'00}}}\ (\bibinfo
  {publisher} {ACM})\ pp.\ \bibinfo {pages} {715--724},\ \Eprint
  {http://arxiv.org/abs/arXiv:quant-ph/9912053} {arXiv:quant-ph/9912053}
  \BibitemShut {NoStop}%
\bibitem [{\citenamefont {Biham}\ \emph {et~al.}(2006)\citenamefont {Biham},
  \citenamefont {Boyer}, \citenamefont {Boykin}, \citenamefont {Mor},\ and\
  \citenamefont {Roychowdhury}}]{BBBMR06}%
  \BibitemOpen
  \bibfield  {author} {\bibinfo {author} {\bibnamefont {Biham}, \bibfnamefont
  {Eli}}, \bibinfo {author} {\bibfnamefont {Michel}\ \bibnamefont {Boyer}},
  \bibinfo {author} {\bibfnamefont {P.~Oscar}\ \bibnamefont {Boykin}}, \bibinfo
  {author} {\bibfnamefont {Tal}\ \bibnamefont {Mor}}, \ and\ \bibinfo {author}
  {\bibfnamefont {Vwani}\ \bibnamefont {Roychowdhury}}} (\bibinfo {year}
  {2006}),\ \bibfield  {title} {\enquote {\bibinfo {title} {A proof of the
  security of quantum key distribution},}\ }\href {\doibase
  10.1007/s00145-005-0011-3} {\bibfield  {journal} {\bibinfo  {journal} {J.
  Crypt.}\ }\textbf {\bibinfo {volume} {19}}~(\bibinfo {number} {4}),\ \bibinfo
  {pages} {381--439}},\ \bibinfo {note} {full version of \textcite{BBBMR00}},\
  \Eprint {http://arxiv.org/abs/arXiv:quant-ph/0511175}
  {arXiv:quant-ph/0511175} \BibitemShut {NoStop}%
\bibitem [{\citenamefont {Biham}\ \emph {et~al.}(2002)\citenamefont {Biham},
  \citenamefont {Boyer}, \citenamefont {Brassard}, \citenamefont {van~de
  Graaf},\ and\ \citenamefont {Mor}}]{BBBvdGM02}%
  \BibitemOpen
  \bibfield  {author} {\bibinfo {author} {\bibnamefont {Biham}, \bibfnamefont
  {Eli}}, \bibinfo {author} {\bibfnamefont {Michel}\ \bibnamefont {Boyer}},
  \bibinfo {author} {\bibfnamefont {Gilles}\ \bibnamefont {Brassard}}, \bibinfo
  {author} {\bibfnamefont {Jeroen}\ \bibnamefont {van~de Graaf}}, \ and\
  \bibinfo {author} {\bibfnamefont {Tal}\ \bibnamefont {Mor}}} (\bibinfo {year}
  {2002}),\ \bibfield  {title} {\enquote {\bibinfo {title} {Security of quantum
  key distribution against all collective attacks},}\ }\href {\doibase
  10.1007/s00453-002-0973-6} {\bibfield  {journal} {\bibinfo  {journal}
  {Algorithmica}\ }\textbf {\bibinfo {volume} {34}}~(\bibinfo {number} {4}),\
  \bibinfo {pages} {372--388}},\ \Eprint
  {http://arxiv.org/abs/quant-ph/9801022} {quant-ph/9801022} \BibitemShut
  {NoStop}%
\bibitem [{\citenamefont {Biham}\ and\ \citenamefont {Mor}(1997)}]{BM97b}%
  \BibitemOpen
  \bibfield  {author} {\bibinfo {author} {\bibnamefont {Biham}, \bibfnamefont
  {Eli}}, \ and\ \bibinfo {author} {\bibfnamefont {Tal}\ \bibnamefont {Mor}}}
  (\bibinfo {year} {1997}),\ \bibfield  {title} {\enquote {\bibinfo {title}
  {Security of quantum cryptography against collective attacks},}\ }\href
  {\doibase 10.1103/PhysRevLett.78.2256} {\bibfield  {journal} {\bibinfo
  {journal} {Phys. Rev. Lett.}\ }\textbf {\bibinfo {volume} {78}},\ \bibinfo
  {pages} {2256--2259}},\ \Eprint {http://arxiv.org/abs/arXiv:quant-ph/9605007}
  {arXiv:quant-ph/9605007} \BibitemShut {NoStop}%
\bibitem [{\citenamefont {Blum}(1983)}]{Blu83}%
  \BibitemOpen
  \bibfield  {author} {\bibinfo {author} {\bibnamefont {Blum}, \bibfnamefont
  {Manuel}}} (\bibinfo {year} {1983}),\ \bibfield  {title} {\enquote {\bibinfo
  {title} {Coin flipping by telephone a protocol for solving impossible
  problems},}\ }\href {\doibase 10.1145/1008908.1008911} {\bibfield  {journal}
  {\bibinfo  {journal} {ACM SIGACT News}\ }\textbf {\bibinfo {volume}
  {15}}~(\bibinfo {number} {1}),\ \bibinfo {pages} {23--27}}\BibitemShut
  {NoStop}%
\bibitem [{\citenamefont {Boileau}\ \emph {et~al.}(2005)\citenamefont
  {Boileau}, \citenamefont {Tamaki}, \citenamefont {Batuwantudawe},
  \citenamefont {Laflamme},\ and\ \citenamefont {Renes}}]{Boileau}%
  \BibitemOpen
  \bibfield  {author} {\bibinfo {author} {\bibnamefont {Boileau}, \bibfnamefont
  {J-C}}, \bibinfo {author} {\bibfnamefont {Kiyoshi}\ \bibnamefont {Tamaki}},
  \bibinfo {author} {\bibfnamefont {Jamie}\ \bibnamefont {Batuwantudawe}},
  \bibinfo {author} {\bibfnamefont {Raymond}\ \bibnamefont {Laflamme}}, \ and\
  \bibinfo {author} {\bibfnamefont {Joseph~M.}\ \bibnamefont {Renes}}}
  (\bibinfo {year} {2005}),\ \bibfield  {title} {\enquote {\bibinfo {title}
  {Unconditional security of a three state quantum key distribution
  protocol},}\ }\href {\doibase 10.1103/PhysRevLett.94.040503} {\bibfield
  {journal} {\bibinfo  {journal} {Phys. Rev. Lett.}\ }\textbf {\bibinfo
  {volume} {94}},\ \bibinfo {pages} {040503}}\BibitemShut {NoStop}%
\bibitem [{\citenamefont {Born}(1926)}]{Born26}%
  \BibitemOpen
  \bibfield  {author} {\bibinfo {author} {\bibnamefont {Born}, \bibfnamefont
  {Max}}} (\bibinfo {year} {1926}),\ \bibfield  {title} {\enquote {\bibinfo
  {title} {Zur {Q}uantenmechanik der {S}to{\ss}vorg{\"a}nge},}\ }\href@noop {}
  {\bibfield  {journal} {\bibinfo  {journal} {Zeitschrift f{\"u}r Physik}\
  }\textbf {\bibinfo {volume} {37}}~(\bibinfo {number} {12}),\ \bibinfo {pages}
  {863--867}}\BibitemShut {NoStop}%
\bibitem [{\citenamefont {Boykin}\ and\ \citenamefont
  {Roychowdhury}(2003)}]{BR03}%
  \BibitemOpen
  \bibfield  {author} {\bibinfo {author} {\bibnamefont {Boykin}, \bibfnamefont
  {P~Oscar}}, \ and\ \bibinfo {author} {\bibfnamefont {Vwani}\ \bibnamefont
  {Roychowdhury}}} (\bibinfo {year} {2003}),\ \bibfield  {title} {\enquote
  {\bibinfo {title} {Optimal encryption of quantum bits},}\ }\href {\doibase
  10.1103/PhysRevA.67.042317} {\bibfield  {journal} {\bibinfo  {journal} {Phys.
  Rev. A}\ }\textbf {\bibinfo {volume} {67}},\ \bibinfo {pages} {042317}},\
  \Eprint {http://arxiv.org/abs/arXiv:quant-ph/0003059}
  {arXiv:quant-ph/0003059} \BibitemShut {NoStop}%
\bibitem [{\citenamefont {Bozzio}\ \emph {et~al.}(2019)\citenamefont {Bozzio},
  \citenamefont {Diamanti},\ and\ \citenamefont {Grosshans}}]{BDG19}%
  \BibitemOpen
  \bibfield  {author} {\bibinfo {author} {\bibnamefont {Bozzio}, \bibfnamefont
  {Mathieu}}, \bibinfo {author} {\bibfnamefont {Eleni}\ \bibnamefont
  {Diamanti}}, \ and\ \bibinfo {author} {\bibfnamefont {Fr\'ed\'eric}\
  \bibnamefont {Grosshans}}} (\bibinfo {year} {2019}),\ \bibfield  {title}
  {\enquote {\bibinfo {title} {Semi-device-independent quantum money with
  coherent states},}\ }\href {\doibase 10.1103/PhysRevA.99.022336} {\bibfield
  {journal} {\bibinfo  {journal} {Phys. Rev. A}\ }\textbf {\bibinfo {volume}
  {99}},\ \bibinfo {pages} {022336}},\ \Eprint
  {http://arxiv.org/abs/arXiv:1812.09256} {arXiv:1812.09256} \BibitemShut
  {NoStop}%
\bibitem [{\citenamefont {Branciard}\ \emph {et~al.}(2012)\citenamefont
  {Branciard}, \citenamefont {Cavalcanti}, \citenamefont {Walborn},
  \citenamefont {Scarani},\ and\ \citenamefont {Wiseman}}]{BCWSW12}%
  \BibitemOpen
  \bibfield  {author} {\bibinfo {author} {\bibnamefont {Branciard},
  \bibfnamefont {Cyril}}, \bibinfo {author} {\bibfnamefont {Eric~G.}\
  \bibnamefont {Cavalcanti}}, \bibinfo {author} {\bibfnamefont {Stephen~P.}\
  \bibnamefont {Walborn}}, \bibinfo {author} {\bibfnamefont {Valerio}\
  \bibnamefont {Scarani}}, \ and\ \bibinfo {author} {\bibfnamefont {Howard~M.}\
  \bibnamefont {Wiseman}}} (\bibinfo {year} {2012}),\ \bibfield  {title}
  {\enquote {\bibinfo {title} {One-sided device-independent quantum key
  distribution: Security, feasibility, and the connection with steering},}\
  }\href {\doibase 10.1103/PhysRevA.85.010301} {\bibfield  {journal} {\bibinfo
  {journal} {Phys. Rev. A}\ }\textbf {\bibinfo {volume} {85}},\ \bibinfo
  {pages} {010301}}\BibitemShut {NoStop}%
\bibitem [{\citenamefont {Brand{\~a}o}\ \emph {et~al.}(2016)\citenamefont
  {Brand{\~a}o}, \citenamefont {Ramanathan}, \citenamefont {Grudka},
  \citenamefont {Horodecki}, \citenamefont {Horodecki}, \citenamefont
  {Horodecki}, \citenamefont {Szarek},\ and\ \citenamefont
  {Wojew{\'o}dka}}]{BRGHHHSW16}%
  \BibitemOpen
  \bibfield  {author} {\bibinfo {author} {\bibnamefont {Brand{\~a}o},
  \bibfnamefont {Fernando G S~L}}, \bibinfo {author} {\bibfnamefont
  {Ravishankar}\ \bibnamefont {Ramanathan}}, \bibinfo {author} {\bibfnamefont
  {Andrzej}\ \bibnamefont {Grudka}}, \bibinfo {author} {\bibfnamefont {Karol}\
  \bibnamefont {Horodecki}}, \bibinfo {author} {\bibfnamefont {Micha\l{}}\
  \bibnamefont {Horodecki}}, \bibinfo {author} {\bibfnamefont {Pawe\l{}}\
  \bibnamefont {Horodecki}}, \bibinfo {author} {\bibfnamefont {Tomasz}\
  \bibnamefont {Szarek}}, \ and\ \bibinfo {author} {\bibfnamefont {Hanna}\
  \bibnamefont {Wojew{\'o}dka}}} (\bibinfo {year} {2016}),\ \bibfield  {title}
  {\enquote {\bibinfo {title} {Realistic noise-tolerant randomness
  amplification using finite number of devices},}\ }\href {\doibase
  10.1038/ncomms11345} {\bibfield  {journal} {\bibinfo  {journal} {Nat.
  Commun.}\ }\textbf {\bibinfo {volume} {7}},\ \bibinfo {pages} {11345}},\
  \Eprint {http://arxiv.org/abs/arXiv:1310.4544} {arXiv:1310.4544} \BibitemShut
  {NoStop}%
\bibitem [{\citenamefont {Brassard}\ \emph {et~al.}(1998)\citenamefont
  {Brassard}, \citenamefont {Cr\'epeau}, \citenamefont {Mayers},\ and\
  \citenamefont {Salvail}}]{BCMS98}%
  \BibitemOpen
  \bibfield  {author} {\bibinfo {author} {\bibnamefont {Brassard},
  \bibfnamefont {Gilles}}, \bibinfo {author} {\bibfnamefont {Claude}\
  \bibnamefont {Cr\'epeau}}, \bibinfo {author} {\bibfnamefont {Dominic}\
  \bibnamefont {Mayers}}, \ and\ \bibinfo {author} {\bibfnamefont {Louis}\
  \bibnamefont {Salvail}}} (\bibinfo {year} {1998}),\ \href@noop {} {\enquote
  {\bibinfo {title} {Defeating classical bit commitments with a quantum
  computer},}\ }\bibinfo {howpublished} {e-print},\ \Eprint
  {http://arxiv.org/abs/arXiv:quant-ph/9806031} {arXiv:quant-ph/9806031}
  \BibitemShut {NoStop}%
\bibitem [{\citenamefont {Brassard}\ \emph {et~al.}(2000)\citenamefont
  {Brassard}, \citenamefont {L\"utkenhaus}, \citenamefont {Mor},\ and\
  \citenamefont {Sanders}}]{Brassardetal2000}%
  \BibitemOpen
  \bibfield  {author} {\bibinfo {author} {\bibnamefont {Brassard},
  \bibfnamefont {Gilles}}, \bibinfo {author} {\bibfnamefont {Norbert}\
  \bibnamefont {L\"utkenhaus}}, \bibinfo {author} {\bibfnamefont {Tal}\
  \bibnamefont {Mor}}, \ and\ \bibinfo {author} {\bibfnamefont {Barry~C.}\
  \bibnamefont {Sanders}}} (\bibinfo {year} {2000}),\ \bibfield  {title}
  {\enquote {\bibinfo {title} {Limitations on practical quantum
  cryptography},}\ }\href {\doibase 10.1103/PhysRevLett.85.1330} {\bibfield
  {journal} {\bibinfo  {journal} {Phys. Rev. Lett.}\ }\textbf {\bibinfo
  {volume} {85}},\ \bibinfo {pages} {1330--1333}}\BibitemShut {NoStop}%
\bibitem [{\citenamefont {Braunstein}\ and\ \citenamefont
  {Pirandola}(2012)}]{BP12}%
  \BibitemOpen
  \bibfield  {author} {\bibinfo {author} {\bibnamefont {Braunstein},
  \bibfnamefont {Samuel~L}}, \ and\ \bibinfo {author} {\bibfnamefont {Stefano}\
  \bibnamefont {Pirandola}}} (\bibinfo {year} {2012}),\ \bibfield  {title}
  {\enquote {\bibinfo {title} {Side-channel-free quantum key distribution},}\
  }\href {\doibase 10.1103/PhysRevLett.108.130502} {\bibfield  {journal}
  {\bibinfo  {journal} {Phys. Rev. Lett.}\ }\textbf {\bibinfo {volume} {108}},\
  \bibinfo {pages} {130502}}\BibitemShut {NoStop}%
\bibitem [{\citenamefont {Broadbent}\ \emph {et~al.}(2009)\citenamefont
  {Broadbent}, \citenamefont {Fitzsimons},\ and\ \citenamefont
  {Kashefi}}]{BFK09}%
  \BibitemOpen
  \bibfield  {author} {\bibinfo {author} {\bibnamefont {Broadbent},
  \bibfnamefont {Anne}}, \bibinfo {author} {\bibfnamefont {Joseph}\
  \bibnamefont {Fitzsimons}}, \ and\ \bibinfo {author} {\bibfnamefont {Elham}\
  \bibnamefont {Kashefi}}} (\bibinfo {year} {2009}),\ \bibfield  {title}
  {\enquote {\bibinfo {title} {Universal blind quantum computation},}\ }in\
  \href {\doibase 10.1109/FOCS.2009.36} {\emph {\bibinfo {booktitle}
  {Proceedings of the 50th Symposium on Foundations of Computer Science,
  FOCS~'09}}}\ (\bibinfo  {publisher} {IEEE Computer Society})\ pp.\ \bibinfo
  {pages} {517--526},\ \Eprint {http://arxiv.org/abs/arXiv:0807.4154}
  {arXiv:0807.4154} \BibitemShut {NoStop}%
\bibitem [{\citenamefont {Broadbent}\ \emph {et~al.}(2013)\citenamefont
  {Broadbent}, \citenamefont {Gutoski},\ and\ \citenamefont {Stebila}}]{BGS13}%
  \BibitemOpen
  \bibfield  {author} {\bibinfo {author} {\bibnamefont {Broadbent},
  \bibfnamefont {Anne}}, \bibinfo {author} {\bibfnamefont {Gus}\ \bibnamefont
  {Gutoski}}, \ and\ \bibinfo {author} {\bibfnamefont {Douglas}\ \bibnamefont
  {Stebila}}} (\bibinfo {year} {2013}),\ \bibfield  {title} {\enquote {\bibinfo
  {title} {Quantum one-time programs},}\ }in\ \href {\doibase
  10.1007/978-3-642-40084-1_20} {\emph {\bibinfo {booktitle} {Advances in
  Cryptology -- CRYPTO 2013}}},\ \bibinfo {series} {LNCS}, Vol.\ \bibinfo
  {volume} {8043}\ (\bibinfo  {publisher} {Springer})\ pp.\ \bibinfo {pages}
  {344--360},\ \Eprint {http://arxiv.org/abs/arXiv:1211.1080} {arXiv:1211.1080}
  \BibitemShut {NoStop}%
\bibitem [{\citenamefont {Broadbent}\ and\ \citenamefont
  {Jeffery}(2015)}]{BJ15}%
  \BibitemOpen
  \bibfield  {author} {\bibinfo {author} {\bibnamefont {Broadbent},
  \bibfnamefont {Anne}}, \ and\ \bibinfo {author} {\bibfnamefont {Stacey}\
  \bibnamefont {Jeffery}}} (\bibinfo {year} {2015}),\ \bibfield  {title}
  {\enquote {\bibinfo {title} {Quantum homomorphic encryption for circuits of
  low t-gate complexity},}\ }in\ \href {\doibase 10.1007/978-3-662-48000-7_30}
  {\emph {\bibinfo {booktitle} {Advances in Cryptology -- CRYPTO 2015}}},\
  \bibinfo {editor} {edited by\ \bibinfo {editor} {\bibfnamefont {Rosario}\
  \bibnamefont {Gennaro}}\ and\ \bibinfo {editor} {\bibfnamefont {Matthew}\
  \bibnamefont {Robshaw}}}\ (\bibinfo  {publisher} {Springer})\ pp.\ \bibinfo
  {pages} {609--629},\ \Eprint {http://arxiv.org/abs/arXiv:1412.8766}
  {arXiv:1412.8766} \BibitemShut {NoStop}%
\bibitem [{\citenamefont {Broadbent}\ and\ \citenamefont
  {Schaffner}(2016)}]{BS16}%
  \BibitemOpen
  \bibfield  {author} {\bibinfo {author} {\bibnamefont {Broadbent},
  \bibfnamefont {Anne}}, \ and\ \bibinfo {author} {\bibfnamefont {Christian}\
  \bibnamefont {Schaffner}}} (\bibinfo {year} {2016}),\ \bibfield  {title}
  {\enquote {\bibinfo {title} {Quantum cryptography beyond quantum key
  distribution},}\ }\href {\doibase 10.1007/s10623-015-0157-4} {\bibfield
  {journal} {\bibinfo  {journal} {Des. Codes Cryptogr.}\ }\textbf {\bibinfo
  {volume} {78}}~(\bibinfo {number} {1}),\ \bibinfo {pages} {351--382}},\
  \Eprint {http://arxiv.org/abs/arXiv:1510.06120} {arXiv:1510.06120}
  \BibitemShut {NoStop}%
\bibitem [{\citenamefont {Broadbent}\ and\ \citenamefont
  {Wainewright}(2016)}]{BW16}%
  \BibitemOpen
  \bibfield  {author} {\bibinfo {author} {\bibnamefont {Broadbent},
  \bibfnamefont {Anne}}, \ and\ \bibinfo {author} {\bibfnamefont {Evelyn}\
  \bibnamefont {Wainewright}}} (\bibinfo {year} {2016}),\ \bibfield  {title}
  {\enquote {\bibinfo {title} {Efficient simulation for quantum message
  authentication},}\ }in\ \href {\doibase 10.1007/978-3-319-49175-2_4} {\emph
  {\bibinfo {booktitle} {Proceedings of the 9th International Conference on
  Information Theoretic Security, ICITS 2016}}}\ (\bibinfo  {publisher}
  {Springer})\ pp.\ \bibinfo {pages} {72--91},\ \Eprint
  {http://arxiv.org/abs/arXiv:1607.03075} {arXiv:1607.03075} \BibitemShut
  {NoStop}%
\bibitem [{\citenamefont {Brunner}\ \emph {et~al.}(2014)\citenamefont
  {Brunner}, \citenamefont {Cavalcanti}, \citenamefont {Pironio}, \citenamefont
  {Scarani},\ and\ \citenamefont {Wehner}}]{BCPSW14}%
  \BibitemOpen
  \bibfield  {author} {\bibinfo {author} {\bibnamefont {Brunner}, \bibfnamefont
  {Nicolas}}, \bibinfo {author} {\bibfnamefont {Daniel}\ \bibnamefont
  {Cavalcanti}}, \bibinfo {author} {\bibfnamefont {Stefano}\ \bibnamefont
  {Pironio}}, \bibinfo {author} {\bibfnamefont {Valerio}\ \bibnamefont
  {Scarani}}, \ and\ \bibinfo {author} {\bibfnamefont {Stephanie}\ \bibnamefont
  {Wehner}}} (\bibinfo {year} {2014}),\ \bibfield  {title} {\enquote {\bibinfo
  {title} {Bell nonlocality},}\ }\href {\doibase 10.1103/RevModPhys.86.419}
  {\bibfield  {journal} {\bibinfo  {journal} {Rev. Mod. Phys.}\ }\textbf
  {\bibinfo {volume} {86}},\ \bibinfo {pages} {419--478}},\ \Eprint
  {http://arxiv.org/abs/arXiv:1303.2849} {arXiv:1303.2849} \BibitemShut
  {NoStop}%
\bibitem [{\citenamefont {Buhrman}\ \emph {et~al.}(2014)\citenamefont
  {Buhrman}, \citenamefont {Chandran}, \citenamefont {Fehr}, \citenamefont
  {Gelles}, \citenamefont {Goyal}, \citenamefont {Ostrovsky},\ and\
  \citenamefont {Schaffner}}]{BCFGGOS14}%
  \BibitemOpen
  \bibfield  {author} {\bibinfo {author} {\bibnamefont {Buhrman}, \bibfnamefont
  {Harry}}, \bibinfo {author} {\bibfnamefont {Nishanth}\ \bibnamefont
  {Chandran}}, \bibinfo {author} {\bibfnamefont {Serge}\ \bibnamefont {Fehr}},
  \bibinfo {author} {\bibfnamefont {Ran}\ \bibnamefont {Gelles}}, \bibinfo
  {author} {\bibfnamefont {Vipul}\ \bibnamefont {Goyal}}, \bibinfo {author}
  {\bibfnamefont {Rafail}\ \bibnamefont {Ostrovsky}}, \ and\ \bibinfo {author}
  {\bibfnamefont {Christian}\ \bibnamefont {Schaffner}}} (\bibinfo {year}
  {2014}),\ \bibfield  {title} {\enquote {\bibinfo {title} {Position-based
  quantum cryptography: Impossibility and constructions},}\ }\href {\doibase
  10.1137/130913687} {\bibfield  {journal} {\bibinfo  {journal} {SIAM J.
  Comput.}\ }\textbf {\bibinfo {volume} {43}}~(\bibinfo {number} {1}),\
  \bibinfo {pages} {150--178}},\ \bibinfo {note} {a preliminary version
  appeared at CRYPTO 2011},\ \Eprint {http://arxiv.org/abs/arXiv:1009.2490}
  {arXiv:1009.2490} \BibitemShut {NoStop}%
\bibitem [{\citenamefont {Calderbank}\ and\ \citenamefont {Shor}(1996)}]{CS96}%
  \BibitemOpen
  \bibfield  {author} {\bibinfo {author} {\bibnamefont {Calderbank},
  \bibfnamefont {A~R}}, \ and\ \bibinfo {author} {\bibfnamefont {Peter~W.}\
  \bibnamefont {Shor}}} (\bibinfo {year} {1996}),\ \bibfield  {title} {\enquote
  {\bibinfo {title} {Good quantum error-correcting codes exist},}\ }\href
  {\doibase 10.1103/PhysRevA.54.1098} {\bibfield  {journal} {\bibinfo
  {journal} {Phys. Rev. A}\ }\textbf {\bibinfo {volume} {54}},\ \bibinfo
  {pages} {1098--1105}}\BibitemShut {NoStop}%
\bibitem [{\citenamefont {Canetti}(2000)}]{Can00}%
  \BibitemOpen
  \bibfield  {author} {\bibinfo {author} {\bibnamefont {Canetti}, \bibfnamefont
  {Ran}}} (\bibinfo {year} {2000}),\ \bibfield  {title} {\enquote {\bibinfo
  {title} {Security and composition of multiparty cryptographic protocols},}\
  }\href {\doibase 10.1007/s001459910006} {\bibfield  {journal} {\bibinfo
  {journal} {J. Crypt.}\ }\textbf {\bibinfo {volume} {13}}~(\bibinfo {number}
  {1}),\ \bibinfo {pages} {143--202}},\ \bibinfo {note} {e-Print
  \href{http://eprint.iacr.org/1998/018}{IACR 1998/018}}\BibitemShut {NoStop}%
\bibitem [{\citenamefont {Canetti}(2001)}]{Can01}%
  \BibitemOpen
  \bibfield  {author} {\bibinfo {author} {\bibnamefont {Canetti}, \bibfnamefont
  {Ran}}} (\bibinfo {year} {2001}),\ \bibfield  {title} {\enquote {\bibinfo
  {title} {Universally composable security: A new paradigm for cryptographic
  protocols},}\ }in\ \href {\doibase 10.1109/SFCS.2001.959888} {\emph {\bibinfo
  {booktitle} {Proceedings of the 42nd Symposium on Foundations of Computer
  Science, FOCS~'01}}}\ (\bibinfo  {publisher} {IEEE})\ pp.\ \bibinfo {pages}
  {136--145}\BibitemShut {NoStop}%
\bibitem [{\citenamefont {Canetti}(2020)}]{Can20}%
  \BibitemOpen
  \bibfield  {author} {\bibinfo {author} {\bibnamefont {Canetti}, \bibfnamefont
  {Ran}}} (\bibinfo {year} {2020}),\ \href@noop {} {\enquote {\bibinfo {title}
  {Universally composable security: A new paradigm for cryptographic
  protocols},}\ }\bibinfo {howpublished} {e-Print
  \href{http://eprint.iacr.org/2000/067}{IACR 2000/067}},\ \bibinfo {note}
  {updated version of~\textcite{Can01}}\BibitemShut {NoStop}%
\bibitem [{\citenamefont {Canetti}\ \emph
  {et~al.}(2006{\natexlab{a}})\citenamefont {Canetti}, \citenamefont {Cheung},
  \citenamefont {Kaynar}, \citenamefont {Liskov}, \citenamefont {Lynch},
  \citenamefont {Pereira},\ and\ \citenamefont {Segala}}]{CCKLLPS06a}%
  \BibitemOpen
  \bibfield  {author} {\bibinfo {author} {\bibnamefont {Canetti}, \bibfnamefont
  {Ran}}, \bibinfo {author} {\bibfnamefont {Ling}\ \bibnamefont {Cheung}},
  \bibinfo {author} {\bibfnamefont {Dilsun~Kirli}\ \bibnamefont {Kaynar}},
  \bibinfo {author} {\bibfnamefont {Moses}\ \bibnamefont {Liskov}}, \bibinfo
  {author} {\bibfnamefont {Nancy~A.}\ \bibnamefont {Lynch}}, \bibinfo {author}
  {\bibfnamefont {Olivier}\ \bibnamefont {Pereira}}, \ and\ \bibinfo {author}
  {\bibfnamefont {Roberto}\ \bibnamefont {Segala}}} (\bibinfo {year}
  {2006}{\natexlab{a}}),\ \bibfield  {title} {\enquote {\bibinfo {title}
  {Task-structured probabilistic {I/O} automata},}\ }in\ \href {\doibase
  10.1109/WODES.2006.1678432} {\emph {\bibinfo {booktitle} {Proceedings of the
  8th International Workshop on Discrete Event Systems, {WODES} 2006}}}\
  (\bibinfo  {publisher} {IEEE})\ pp.\ \bibinfo {pages} {207--214},\ \bibinfo
  {note} {extended version available at
  \url{http://theory.csail.mit.edu/~lcheung/papers/task-PIOA-TR.pdf}}\BibitemShut
  {NoStop}%
\bibitem [{\citenamefont {Canetti}\ \emph
  {et~al.}(2006{\natexlab{b}})\citenamefont {Canetti}, \citenamefont {Cheung},
  \citenamefont {Kaynar}, \citenamefont {Liskov}, \citenamefont {Lynch},
  \citenamefont {Pereira},\ and\ \citenamefont {Segala}}]{CCKLLPS06b}%
  \BibitemOpen
  \bibfield  {author} {\bibinfo {author} {\bibnamefont {Canetti}, \bibfnamefont
  {Ran}}, \bibinfo {author} {\bibfnamefont {Ling}\ \bibnamefont {Cheung}},
  \bibinfo {author} {\bibfnamefont {Dilsun~Kirli}\ \bibnamefont {Kaynar}},
  \bibinfo {author} {\bibfnamefont {Moses}\ \bibnamefont {Liskov}}, \bibinfo
  {author} {\bibfnamefont {Nancy~A.}\ \bibnamefont {Lynch}}, \bibinfo {author}
  {\bibfnamefont {Olivier}\ \bibnamefont {Pereira}}, \ and\ \bibinfo {author}
  {\bibfnamefont {Roberto}\ \bibnamefont {Segala}}} (\bibinfo {year}
  {2006}{\natexlab{b}}),\ \bibfield  {title} {\enquote {\bibinfo {title}
  {Time-bounded task-{PIOAs}: {A} framework for analyzing security
  protocols},}\ }in\ \href {\doibase 10.1007/11864219_17} {\emph {\bibinfo
  {booktitle} {Proceedings of the 20th International Symposium on Distributed
  Computing, {DISC} 2006}}},\ pp.\ \bibinfo {pages} {238--253}\BibitemShut
  {NoStop}%
\bibitem [{\citenamefont {Canetti}\ \emph {et~al.}(2007)\citenamefont
  {Canetti}, \citenamefont {Dodis}, \citenamefont {Pass},\ and\ \citenamefont
  {Walfish}}]{CDPW07}%
  \BibitemOpen
  \bibfield  {author} {\bibinfo {author} {\bibnamefont {Canetti}, \bibfnamefont
  {Ran}}, \bibinfo {author} {\bibfnamefont {Yevgeniy}\ \bibnamefont {Dodis}},
  \bibinfo {author} {\bibfnamefont {Rafael}\ \bibnamefont {Pass}}, \ and\
  \bibinfo {author} {\bibfnamefont {Shabsi}\ \bibnamefont {Walfish}}} (\bibinfo
  {year} {2007}),\ \bibfield  {title} {\enquote {\bibinfo {title} {Universally
  composable security with global setup},}\ }in\ \href {\doibase
  10.1007/978-3-540-70936-7_4} {\emph {\bibinfo {booktitle} {Theory of
  Cryptography, Proceedings of TCC 2007}}},\ \bibinfo {series} {LNCS}, Vol.\
  \bibinfo {volume} {4392}\ (\bibinfo  {publisher} {Springer})\ pp.\ \bibinfo
  {pages} {61--85},\ \bibinfo {note} {e-Print
  \href{http://eprint.iacr.org/2006/432}{IACR 2006/432}}\BibitemShut {NoStop}%
\bibitem [{\citenamefont {Canetti}\ and\ \citenamefont
  {Fischlin}(2001)}]{CF01}%
  \BibitemOpen
  \bibfield  {author} {\bibinfo {author} {\bibnamefont {Canetti}, \bibfnamefont
  {Ran}}, \ and\ \bibinfo {author} {\bibfnamefont {Marc}\ \bibnamefont
  {Fischlin}}} (\bibinfo {year} {2001}),\ \bibfield  {title} {\enquote
  {\bibinfo {title} {Universally composable commitments},}\ }in\ \href
  {\doibase 10.1007/3-540-44647-8_2} {\emph {\bibinfo {booktitle} {Advances in
  Cryptology --- CRYPTO 2001}}},\ \bibinfo {editor} {edited by\ \bibinfo
  {editor} {\bibfnamefont {Joe}\ \bibnamefont {Kilian}}}\ (\bibinfo
  {publisher} {Springer})\ pp.\ \bibinfo {pages} {19--40},\ \bibinfo {note}
  {e-Print \href{http://eprint.iacr.org/2001/055}{IACR 2001/055}}\BibitemShut
  {NoStop}%
\bibitem [{\citenamefont {Canetti}\ \emph {et~al.}(2003)\citenamefont
  {Canetti}, \citenamefont {Krawczyk},\ and\ \citenamefont {Nielsen}}]{CKN03}%
  \BibitemOpen
  \bibfield  {author} {\bibinfo {author} {\bibnamefont {Canetti}, \bibfnamefont
  {Ran}}, \bibinfo {author} {\bibfnamefont {Hugo}\ \bibnamefont {Krawczyk}}, \
  and\ \bibinfo {author} {\bibfnamefont {Jesper~B.}\ \bibnamefont {Nielsen}}}
  (\bibinfo {year} {2003}),\ \bibfield  {title} {\enquote {\bibinfo {title}
  {Relaxing chosen-ciphertext security},}\ }in\ \href {\doibase
  10.1007/978-3-540-45146-4_33} {\emph {\bibinfo {booktitle} {Advances in
  Cryptology -- CRYPTO 2003}}},\ \bibinfo {editor} {edited by\ \bibinfo
  {editor} {\bibfnamefont {Dan}\ \bibnamefont {Boneh}}}\ (\bibinfo  {publisher}
  {Springer})\ pp.\ \bibinfo {pages} {565--582}\BibitemShut {NoStop}%
\bibitem [{\citenamefont {Canetti}\ \emph {et~al.}(2002)\citenamefont
  {Canetti}, \citenamefont {Lindell}, \citenamefont {Ostrovsky},\ and\
  \citenamefont {Sahai}}]{CLOS02}%
  \BibitemOpen
  \bibfield  {author} {\bibinfo {author} {\bibnamefont {Canetti}, \bibfnamefont
  {Ran}}, \bibinfo {author} {\bibfnamefont {Yehuda}\ \bibnamefont {Lindell}},
  \bibinfo {author} {\bibfnamefont {Rafail}\ \bibnamefont {Ostrovsky}}, \ and\
  \bibinfo {author} {\bibfnamefont {Amit}\ \bibnamefont {Sahai}}} (\bibinfo
  {year} {2002}),\ \bibfield  {title} {\enquote {\bibinfo {title} {Universally
  composable two-party and multi-party secure computation},}\ }in\ \href
  {\doibase 10.1145/509907.509980} {\emph {\bibinfo {booktitle} {Proceedings of
  the 34th Symposium on Theory of Computing, STOC~'02}}}\ (\bibinfo
  {publisher} {ACM})\ p.\ \bibinfo {pages} {494–503},\ \bibinfo {note}
  {e-Print \href{http://eprint.iacr.org/2002/140}{IACR 2002/140}}\BibitemShut
  {NoStop}%
\bibitem [{\citenamefont {Carter}\ and\ \citenamefont {Wegman}(1979)}]{CW79}%
  \BibitemOpen
  \bibfield  {author} {\bibinfo {author} {\bibnamefont {Carter}, \bibfnamefont
  {Larry}}, \ and\ \bibinfo {author} {\bibfnamefont {Mark~N.}\ \bibnamefont
  {Wegman}}} (\bibinfo {year} {1979}),\ \bibfield  {title} {\enquote {\bibinfo
  {title} {Universal classes of hash functions},}\ }\href {\doibase
  10.1016/0022-0000(79)90044-8} {\bibfield  {journal} {\bibinfo  {journal} {J.
  Comput. Syst. Sci.}\ }\textbf {\bibinfo {volume} {18}}~(\bibinfo {number}
  {2}),\ \bibinfo {pages} {143--154}}\BibitemShut {NoStop}%
\bibitem [{\citenamefont {Chandran}\ \emph {et~al.}(2009)\citenamefont
  {Chandran}, \citenamefont {Goyal}, \citenamefont {Moriarty},\ and\
  \citenamefont {Ostrovsky}}]{CGMO09}%
  \BibitemOpen
  \bibfield  {author} {\bibinfo {author} {\bibnamefont {Chandran},
  \bibfnamefont {Nishanth}}, \bibinfo {author} {\bibfnamefont {Vipul}\
  \bibnamefont {Goyal}}, \bibinfo {author} {\bibfnamefont {Ryan}\ \bibnamefont
  {Moriarty}}, \ and\ \bibinfo {author} {\bibfnamefont {Rafail}\ \bibnamefont
  {Ostrovsky}}} (\bibinfo {year} {2009}),\ \bibfield  {title} {\enquote
  {\bibinfo {title} {Position based cryptography},}\ }in\ \href {\doibase
  10.1007/978-3-642-03356-8_23} {\emph {\bibinfo {booktitle} {Advances in
  Cryptology -- CRYPTO 2009}}},\ \bibinfo {editor} {edited by\ \bibinfo
  {editor} {\bibfnamefont {Shai}\ \bibnamefont {Halevi}}}\ (\bibinfo
  {publisher} {Springer})\ pp.\ \bibinfo {pages} {391--407}\BibitemShut
  {NoStop}%
\bibitem [{\citenamefont {Chen}\ \emph {et~al.}(2017)\citenamefont {Chen},
  \citenamefont {Chung}, \citenamefont {Lai}, \citenamefont {Vadhan},\ and\
  \citenamefont {Wu}}]{CCLVW17}%
  \BibitemOpen
  \bibfield  {author} {\bibinfo {author} {\bibnamefont {Chen}, \bibfnamefont
  {Yi-Hsiu}}, \bibinfo {author} {\bibfnamefont {Kai-Min}\ \bibnamefont
  {Chung}}, \bibinfo {author} {\bibfnamefont {Ching-Yi}\ \bibnamefont {Lai}},
  \bibinfo {author} {\bibfnamefont {Salil~P.}\ \bibnamefont {Vadhan}}, \ and\
  \bibinfo {author} {\bibfnamefont {Xiaodi}\ \bibnamefont {Wu}}} (\bibinfo
  {year} {2017}),\ \href@noop {} {\enquote {\bibinfo {title} {Computational
  notions of quantum min-entropy},}\ }\bibinfo {howpublished} {e-print},\
  \Eprint {http://arxiv.org/abs/arXiv:1704.07309} {arXiv:1704.07309}
  \BibitemShut {NoStop}%
\bibitem [{\citenamefont {Childs}(2005)}]{Chi05}%
  \BibitemOpen
  \bibfield  {author} {\bibinfo {author} {\bibnamefont {Childs}, \bibfnamefont
  {Andrew~M}}} (\bibinfo {year} {2005}),\ \bibfield  {title} {\enquote
  {\bibinfo {title} {Secure assisted quantum computation},}\ }\href@noop {}
  {\bibfield  {journal} {\bibinfo  {journal} {Quantum Inf. Comput.}\ }\textbf
  {\bibinfo {volume} {5}}~(\bibinfo {number} {6}),\ \bibinfo {pages}
  {456--466}},\ \Eprint {http://arxiv.org/abs/arXiv:quant-ph/0111046}
  {arXiv:quant-ph/0111046} \BibitemShut {NoStop}%
\bibitem [{\citenamefont {Chiribella}\ \emph {et~al.}(2009)\citenamefont
  {Chiribella}, \citenamefont {D'Ariano},\ and\ \citenamefont
  {Perinotti}}]{CDP09}%
  \BibitemOpen
  \bibfield  {author} {\bibinfo {author} {\bibnamefont {Chiribella},
  \bibfnamefont {Giulio}}, \bibinfo {author} {\bibfnamefont {Giacomo~Mauro}\
  \bibnamefont {D'Ariano}}, \ and\ \bibinfo {author} {\bibfnamefont {Paolo}\
  \bibnamefont {Perinotti}}} (\bibinfo {year} {2009}),\ \bibfield  {title}
  {\enquote {\bibinfo {title} {Theoretical framework for quantum networks},}\
  }\href {\doibase 10.1103/PhysRevA.80.022339} {\bibfield  {journal} {\bibinfo
  {journal} {Phys. Rev. A}\ }\textbf {\bibinfo {volume} {80}},\ \bibinfo
  {pages} {022339}},\ \Eprint {http://arxiv.org/abs/arXiv:0904.4483}
  {arXiv:0904.4483} \BibitemShut {NoStop}%
\bibitem [{\citenamefont {Chitambar}\ and\ \citenamefont {Gour}(2019)}]{CG19}%
  \BibitemOpen
  \bibfield  {author} {\bibinfo {author} {\bibnamefont {Chitambar},
  \bibfnamefont {Eric}}, \ and\ \bibinfo {author} {\bibfnamefont {Gilad}\
  \bibnamefont {Gour}}} (\bibinfo {year} {2019}),\ \bibfield  {title} {\enquote
  {\bibinfo {title} {Quantum resource theories},}\ }\href {\doibase
  10.1103/RevModPhys.91.025001} {\bibfield  {journal} {\bibinfo  {journal}
  {Rev. Mod. Phys.}\ }\textbf {\bibinfo {volume} {91}},\ \bibinfo {pages}
  {025001}}\BibitemShut {NoStop}%
\bibitem [{\citenamefont {Christandl}\ \emph {et~al.}(2007)\citenamefont
  {Christandl}, \citenamefont {Ekert}, \citenamefont {Horodecki}, \citenamefont
  {Horodecki}, \citenamefont {Oppenheim},\ and\ \citenamefont
  {Renner}}]{christandl2007unifying}%
  \BibitemOpen
  \bibfield  {author} {\bibinfo {author} {\bibnamefont {Christandl},
  \bibfnamefont {Matthias}}, \bibinfo {author} {\bibfnamefont {Artur}\
  \bibnamefont {Ekert}}, \bibinfo {author} {\bibfnamefont {Micha{\l}}\
  \bibnamefont {Horodecki}}, \bibinfo {author} {\bibfnamefont {Pawe{\l}}\
  \bibnamefont {Horodecki}}, \bibinfo {author} {\bibfnamefont {Jonathan}\
  \bibnamefont {Oppenheim}}, \ and\ \bibinfo {author} {\bibfnamefont {Renato}\
  \bibnamefont {Renner}}} (\bibinfo {year} {2007}),\ \bibfield  {title}
  {\enquote {\bibinfo {title} {Unifying classical and quantum key
  distillation},}\ }in\ \href {\doibase 10.1007/978-3-540-70936-7_25} {\emph
  {\bibinfo {booktitle} {Theory of Cryptography Conference, Proceedings of
  {TCC} 2007}}},\ \bibinfo {series} {LNCS}, Vol.\ \bibinfo {volume} {4392},\
  \bibinfo {editor} {edited by\ \bibinfo {editor} {\bibfnamefont {Salil~P.}\
  \bibnamefont {Vadhan}}}\ (\bibinfo  {publisher} {Springer})\ pp.\ \bibinfo
  {pages} {456--478},\ \Eprint {http://arxiv.org/abs/arXiv:quant-ph/0608199}
  {arXiv:quant-ph/0608199} \BibitemShut {NoStop}%
\bibitem [{\citenamefont {Christandl}\ \emph {et~al.}(2009)\citenamefont
  {Christandl}, \citenamefont {K\"onig},\ and\ \citenamefont {Renner}}]{CKR09}%
  \BibitemOpen
  \bibfield  {author} {\bibinfo {author} {\bibnamefont {Christandl},
  \bibfnamefont {Matthias}}, \bibinfo {author} {\bibfnamefont {Robert}\
  \bibnamefont {K\"onig}}, \ and\ \bibinfo {author} {\bibfnamefont {Renato}\
  \bibnamefont {Renner}}} (\bibinfo {year} {2009}),\ \bibfield  {title}
  {\enquote {\bibinfo {title} {Postselection technique for quantum channels
  with applications to quantum cryptography},}\ }\href {\doibase
  10.1103/PhysRevLett.102.020504} {\bibfield  {journal} {\bibinfo  {journal}
  {Phys. Rev. Lett.}\ }\textbf {\bibinfo {volume} {102}},\ \bibinfo {pages}
  {020504}},\ \Eprint {http://arxiv.org/abs/arXiv:0809.3019} {arXiv:0809.3019}
  \BibitemShut {NoStop}%
\bibitem [{\citenamefont {Christandl}\ \emph {et~al.}(2004)\citenamefont
  {Christandl}, \citenamefont {Renner},\ and\ \citenamefont {Ekert}}]{CRE04}%
  \BibitemOpen
  \bibfield  {author} {\bibinfo {author} {\bibnamefont {Christandl},
  \bibfnamefont {Matthias}}, \bibinfo {author} {\bibfnamefont {Renato}\
  \bibnamefont {Renner}}, \ and\ \bibinfo {author} {\bibfnamefont {Artur}\
  \bibnamefont {Ekert}}} (\bibinfo {year} {2004}),\ \href@noop {} {\enquote
  {\bibinfo {title} {A generic security proof for quantum key distribution},}\
  }\bibinfo {howpublished} {e-Print},\ \Eprint
  {http://arxiv.org/abs/arXiv:quant-ph/0402131} {arXiv:quant-ph/0402131}
  \BibitemShut {NoStop}%
\bibitem [{\citenamefont {Christensen}\ \emph {et~al.}(2013)\citenamefont
  {Christensen}, \citenamefont {McCusker}, \citenamefont {Altepeter},
  \citenamefont {Calkins}, \citenamefont {Gerrits}, \citenamefont {Lita},
  \citenamefont {Miller}, \citenamefont {Shalm}, \citenamefont {Zhang},
  \citenamefont {Nam}, \citenamefont {Brunner}, \citenamefont {Lim},
  \citenamefont {Gisin},\ and\ \citenamefont {Kwiat}}]{Christensen}%
  \BibitemOpen
  \bibfield  {author} {\bibinfo {author} {\bibnamefont {Christensen},
  \bibfnamefont {Bradley~G}}, \bibinfo {author} {\bibfnamefont {Kevin~T.}\
  \bibnamefont {McCusker}}, \bibinfo {author} {\bibfnamefont {J.~B.}\
  \bibnamefont {Altepeter}}, \bibinfo {author} {\bibfnamefont {Brice}\
  \bibnamefont {Calkins}}, \bibinfo {author} {\bibfnamefont {Thomas}\
  \bibnamefont {Gerrits}}, \bibinfo {author} {\bibfnamefont {Adriana~E.}\
  \bibnamefont {Lita}}, \bibinfo {author} {\bibfnamefont {Aaron}\ \bibnamefont
  {Miller}}, \bibinfo {author} {\bibfnamefont {L.~K.}\ \bibnamefont {Shalm}},
  \bibinfo {author} {\bibfnamefont {Y.}~\bibnamefont {Zhang}}, \bibinfo
  {author} {\bibfnamefont {S.~W.}\ \bibnamefont {Nam}}, \bibinfo {author}
  {\bibfnamefont {Nicolas}\ \bibnamefont {Brunner}}, \bibinfo {author}
  {\bibfnamefont {Charles Ci~Wen}\ \bibnamefont {Lim}}, \bibinfo {author}
  {\bibfnamefont {Nicolas}\ \bibnamefont {Gisin}}, \ and\ \bibinfo {author}
  {\bibfnamefont {Paul~G.}\ \bibnamefont {Kwiat}}} (\bibinfo {year} {2013}),\
  \bibfield  {title} {\enquote {\bibinfo {title} {Detection-loophole-free test
  of quantum nonlocality, and applications},}\ }\href {\doibase
  10.1103/PhysRevLett.111.130406} {\bibfield  {journal} {\bibinfo  {journal}
  {Phys. Rev. Lett.}\ }\textbf {\bibinfo {volume} {111}},\ \bibinfo {pages}
  {130406}}\BibitemShut {NoStop}%
\bibitem [{\citenamefont {Chung}\ \emph
  {et~al.}(2014{\natexlab{a}})\citenamefont {Chung}, \citenamefont {Li},\ and\
  \citenamefont {Wu}}]{CLW14}%
  \BibitemOpen
  \bibfield  {author} {\bibinfo {author} {\bibnamefont {Chung}, \bibfnamefont
  {Kai-Min}}, \bibinfo {author} {\bibfnamefont {Xin}\ \bibnamefont {Li}}, \
  and\ \bibinfo {author} {\bibfnamefont {Xiaodi}\ \bibnamefont {Wu}}} (\bibinfo
  {year} {2014}{\natexlab{a}}),\ \href@noop {} {\enquote {\bibinfo {title}
  {Multi-source randomness extractors against quantum side information, and
  their applications},}\ }\bibinfo {howpublished} {e-Print},\ \Eprint
  {http://arxiv.org/abs/arXiv:1411.2315} {arXiv:1411.2315} \BibitemShut
  {NoStop}%
\bibitem [{\citenamefont {Chung}\ \emph
  {et~al.}(2014{\natexlab{b}})\citenamefont {Chung}, \citenamefont {Shi},\ and\
  \citenamefont {Wu}}]{CSW14}%
  \BibitemOpen
  \bibfield  {author} {\bibinfo {author} {\bibnamefont {Chung}, \bibfnamefont
  {Kai-Min}}, \bibinfo {author} {\bibfnamefont {Yaoyun}\ \bibnamefont {Shi}}, \
  and\ \bibinfo {author} {\bibfnamefont {Xiaodi}\ \bibnamefont {Wu}}} (\bibinfo
  {year} {2014}{\natexlab{b}}),\ \href@noop {} {\enquote {\bibinfo {title}
  {Physical randomness extractors: Generating random numbers with minimal
  assumptions},}\ }\bibinfo {howpublished} {e-Print},\ \Eprint
  {http://arxiv.org/abs/arXiv:1402.4797} {arXiv:1402.4797} \BibitemShut
  {NoStop}%
\bibitem [{\citenamefont {Clauser}\ \emph {et~al.}(1969)\citenamefont
  {Clauser}, \citenamefont {Horne}, \citenamefont {Shimony},\ and\
  \citenamefont {Holt}}]{CHSH69}%
  \BibitemOpen
  \bibfield  {author} {\bibinfo {author} {\bibnamefont {Clauser}, \bibfnamefont
  {John}}, \bibinfo {author} {\bibfnamefont {Michael}\ \bibnamefont {Horne}},
  \bibinfo {author} {\bibfnamefont {Abner}\ \bibnamefont {Shimony}}, \ and\
  \bibinfo {author} {\bibfnamefont {Richard}\ \bibnamefont {Holt}}} (\bibinfo
  {year} {1969}),\ \bibfield  {title} {\enquote {\bibinfo {title} {Proposed
  experiment to test local hidden-variable theories},}\ }\href {\doibase
  10.1103/PhysRevLett.23.880} {\bibfield  {journal} {\bibinfo  {journal} {Phys.
  Rev. Lett.}\ }\textbf {\bibinfo {volume} {23}}~(\bibinfo {number} {15}),\
  \bibinfo {pages} {880--884}}\BibitemShut {NoStop}%
\bibitem [{\citenamefont {Coffman}\ \emph {et~al.}(2000)\citenamefont
  {Coffman}, \citenamefont {Kundu},\ and\ \citenamefont
  {Wootters}}]{Coffman00}%
  \BibitemOpen
  \bibfield  {author} {\bibinfo {author} {\bibnamefont {Coffman}, \bibfnamefont
  {Valerie}}, \bibinfo {author} {\bibfnamefont {Joydip}\ \bibnamefont {Kundu}},
  \ and\ \bibinfo {author} {\bibfnamefont {William~K.}\ \bibnamefont
  {Wootters}}} (\bibinfo {year} {2000}),\ \bibfield  {title} {\enquote
  {\bibinfo {title} {Distributed entanglement},}\ }\href {\doibase
  10.1103/PhysRevA.61.052306} {\bibfield  {journal} {\bibinfo  {journal} {Phys.
  Rev. A}\ }\textbf {\bibinfo {volume} {61}},\ \bibinfo {pages}
  {052306}}\BibitemShut {NoStop}%
\bibitem [{\citenamefont {Colbeck}(2006)}]{Col06}%
  \BibitemOpen
  \bibfield  {author} {\bibinfo {author} {\bibnamefont {Colbeck}, \bibfnamefont
  {Roger}}} (\bibinfo {year} {2006}),\ \emph {\bibinfo {title} {Quantum And
  Relativistic Protocols For Secure Multi-Party Computation}},\ \href@noop {}
  {Ph.D. thesis}\ (\bibinfo  {school} {University of Cambridge}),\ \Eprint
  {http://arxiv.org/abs/arXiv:0911.3814} {arXiv:0911.3814} \BibitemShut
  {NoStop}%
\bibitem [{\citenamefont {Colbeck}\ and\ \citenamefont {Renner}(2011)}]{CR11}%
  \BibitemOpen
  \bibfield  {author} {\bibinfo {author} {\bibnamefont {Colbeck}, \bibfnamefont
  {Roger}}, \ and\ \bibinfo {author} {\bibfnamefont {Renato}\ \bibnamefont
  {Renner}}} (\bibinfo {year} {2011}),\ \bibfield  {title} {\enquote {\bibinfo
  {title} {No extension of quantum theory can have improved predictive
  power},}\ }\href {\doibase 10.1038/ncomms1416} {\bibfield  {journal}
  {\bibinfo  {journal} {Nat. Commun.}\ }\textbf {\bibinfo {volume} {2}},\
  \bibinfo {pages} {411}},\ \Eprint {http://arxiv.org/abs/arXiv:1005.5173}
  {arXiv:1005.5173} \BibitemShut {NoStop}%
\bibitem [{\citenamefont {Colbeck}\ and\ \citenamefont {Renner}(2012)}]{CR12}%
  \BibitemOpen
  \bibfield  {author} {\bibinfo {author} {\bibnamefont {Colbeck}, \bibfnamefont
  {Roger}}, \ and\ \bibinfo {author} {\bibfnamefont {Renato}\ \bibnamefont
  {Renner}}} (\bibinfo {year} {2012}),\ \bibfield  {title} {\enquote {\bibinfo
  {title} {Free randomness can be amplified},}\ }\href {\doibase
  10.1038/nphys2300} {\bibfield  {journal} {\bibinfo  {journal} {Nat. Phys.}\
  }\textbf {\bibinfo {volume} {8}}~(\bibinfo {number} {6}),\ \bibinfo {pages}
  {450--454}},\ \Eprint {http://arxiv.org/abs/arXiv:1105.3195}
  {arXiv:1105.3195} \BibitemShut {NoStop}%
\bibitem [{\citenamefont {Coles}\ \emph {et~al.}(2017)\citenamefont {Coles},
  \citenamefont {Berta}, \citenamefont {Tomamichel},\ and\ \citenamefont
  {Wehner}}]{Coles}%
  \BibitemOpen
  \bibfield  {author} {\bibinfo {author} {\bibnamefont {Coles}, \bibfnamefont
  {Patrick~J}}, \bibinfo {author} {\bibfnamefont {Mario}\ \bibnamefont
  {Berta}}, \bibinfo {author} {\bibfnamefont {Marco}\ \bibnamefont
  {Tomamichel}}, \ and\ \bibinfo {author} {\bibfnamefont {Stephanie}\
  \bibnamefont {Wehner}}} (\bibinfo {year} {2017}),\ \bibfield  {title}
  {\enquote {\bibinfo {title} {Entropic uncertainty relations and their
  applications},}\ }\href {\doibase 10.1103/RevModPhys.89.015002} {\bibfield
  {journal} {\bibinfo  {journal} {Rev. Mod. Phys.}\ }\textbf {\bibinfo {volume}
  {89}},\ \bibinfo {pages} {015002}}\BibitemShut {NoStop}%
\bibitem [{\citenamefont {Conway}\ and\ \citenamefont
  {Kochen}(2006)}]{Conway2006}%
  \BibitemOpen
  \bibfield  {author} {\bibinfo {author} {\bibnamefont {Conway}, \bibfnamefont
  {John}}, \ and\ \bibinfo {author} {\bibfnamefont {Simon}\ \bibnamefont
  {Kochen}}} (\bibinfo {year} {2006}),\ \bibfield  {title} {\enquote {\bibinfo
  {title} {The free will theorem},}\ }\href {\doibase
  10.1007/s10701-006-9068-6} {\bibfield  {journal} {\bibinfo  {journal} {Found.
  Phys.}\ }\textbf {\bibinfo {volume} {36}}~(\bibinfo {number} {10}),\ \bibinfo
  {pages} {1441--1473}}\BibitemShut {NoStop}%
\bibitem [{\citenamefont {Coretti}\ \emph {et~al.}(2013)\citenamefont
  {Coretti}, \citenamefont {Maurer},\ and\ \citenamefont {Tackmann}}]{CMT13}%
  \BibitemOpen
  \bibfield  {author} {\bibinfo {author} {\bibnamefont {Coretti}, \bibfnamefont
  {Sandro}}, \bibinfo {author} {\bibfnamefont {Ueli}\ \bibnamefont {Maurer}}, \
  and\ \bibinfo {author} {\bibfnamefont {Bj\"orn}\ \bibnamefont {Tackmann}}}
  (\bibinfo {year} {2013}),\ \bibfield  {title} {\enquote {\bibinfo {title}
  {Constructing confidential channels from authenticated channels---public-key
  encryption revisited},}\ }in\ \href {\doibase 10.1007/978-3-642-42033-7_8}
  {\emph {\bibinfo {booktitle} {Advances in Cryptology -- ASIACRYPT 2013}}},\
  \bibinfo {series} {LNCS}, Vol.\ \bibinfo {volume} {8269}\ (\bibinfo
  {publisher} {Springer})\ pp.\ \bibinfo {pages} {134--153},\ \bibinfo {note}
  {e-Print \href{http://eprint.iacr.org/2013/719}{IACR 2013/719}}\BibitemShut
  {NoStop}%
\bibitem [{\citenamefont {Cover}\ and\ \citenamefont {Thomas}(2012)}]{CT12}%
  \BibitemOpen
  \bibfield  {author} {\bibinfo {author} {\bibnamefont {Cover}, \bibfnamefont
  {Thomas~M}}, \ and\ \bibinfo {author} {\bibfnamefont {Joy~A.}\ \bibnamefont
  {Thomas}}} (\bibinfo {year} {2012}),\ \href@noop {} {\emph {\bibinfo {title}
  {Elements of information theory}}}\ (\bibinfo  {publisher} {John Wiley \&
  Sons})\BibitemShut {NoStop}%
\bibitem [{\citenamefont {Cramer}\ \emph {et~al.}(2015)\citenamefont {Cramer},
  \citenamefont {Damg{\aa}rd},\ and\ \citenamefont {Nielsen}}]{CDN15}%
  \BibitemOpen
  \bibfield  {author} {\bibinfo {author} {\bibnamefont {Cramer}, \bibfnamefont
  {Ronald}}, \bibinfo {author} {\bibfnamefont {Ivan~B.}\ \bibnamefont
  {Damg{\aa}rd}}, \ and\ \bibinfo {author} {\bibfnamefont {Jesper~B.}\
  \bibnamefont {Nielsen}}} (\bibinfo {year} {2015}),\ \href {\doibase
  10.1017/CBO9781107337756} {\emph {\bibinfo {title} {Secure Multiparty
  Computation and Secret Sharing}}}\ (\bibinfo  {publisher} {Cambridge
  University Press})\BibitemShut {NoStop}%
\bibitem [{\citenamefont {Cr\'{e}peau}\ \emph {et~al.}(2002)\citenamefont
  {Cr\'{e}peau}, \citenamefont {Gottesman},\ and\ \citenamefont
  {Smith}}]{CGS02}%
  \BibitemOpen
  \bibfield  {author} {\bibinfo {author} {\bibnamefont {Cr\'{e}peau},
  \bibfnamefont {Claude}}, \bibinfo {author} {\bibfnamefont {Daniel}\
  \bibnamefont {Gottesman}}, \ and\ \bibinfo {author} {\bibfnamefont {Adam}\
  \bibnamefont {Smith}}} (\bibinfo {year} {2002}),\ \bibfield  {title}
  {\enquote {\bibinfo {title} {Secure multi-party quantum computation},}\ }in\
  \href {\doibase 10.1145/509907.510000} {\emph {\bibinfo {booktitle}
  {Proceedings of the 34th Symposium on Theory of Computing, STOC~'02}}}\
  (\bibinfo  {publisher} {ACM})\ pp.\ \bibinfo {pages} {643--652},\ \Eprint
  {http://arxiv.org/abs/arXiv:quant-ph/0206138} {arXiv:quant-ph/0206138}
  \BibitemShut {NoStop}%
\bibitem [{\citenamefont {Cr\'e{}peau}\ and\ \citenamefont
  {Kilian}(1988)}]{CK88}%
  \BibitemOpen
  \bibfield  {author} {\bibinfo {author} {\bibnamefont {Cr\'e{}peau},
  \bibfnamefont {Claude}}, \ and\ \bibinfo {author} {\bibfnamefont {Joe}\
  \bibnamefont {Kilian}}} (\bibinfo {year} {1988}),\ \bibfield  {title}
  {\enquote {\bibinfo {title} {Achieving oblivious transfer using weakened
  security assumptions},}\ }in\ \href {\doibase 10.1109/SFCS.1988.21920} {\emph
  {\bibinfo {booktitle} {Proceedings of the 29th Symposium on Foundations of
  Computer Science, FOCS~'88}}},\ pp.\ \bibinfo {pages} {42--52}\BibitemShut
  {NoStop}%
\bibitem [{\citenamefont {Curty}\ \emph {et~al.}(2014)\citenamefont {Curty},
  \citenamefont {Xu}, \citenamefont {Cui}, \citenamefont {Lim}, \citenamefont
  {Tamaki},\ and\ \citenamefont {Lo}}]{CXCLTL14}%
  \BibitemOpen
  \bibfield  {author} {\bibinfo {author} {\bibnamefont {Curty}, \bibfnamefont
  {Marcos}}, \bibinfo {author} {\bibfnamefont {Feihu}\ \bibnamefont {Xu}},
  \bibinfo {author} {\bibfnamefont {Wei}\ \bibnamefont {Cui}}, \bibinfo
  {author} {\bibfnamefont {Charles Ci~Wen}\ \bibnamefont {Lim}}, \bibinfo
  {author} {\bibfnamefont {Kiyoshi}\ \bibnamefont {Tamaki}}, \ and\ \bibinfo
  {author} {\bibfnamefont {Hoi-Kwong}\ \bibnamefont {Lo}}} (\bibinfo {year}
  {2014}),\ \bibfield  {title} {\enquote {\bibinfo {title} {Finite-key analysis
  for measurement-device-independent quantum key distribution},}\ }\href
  {\doibase 10.1038/ncomms4732} {\bibfield  {journal} {\bibinfo  {journal}
  {Nat. Commun.}\ }\textbf {\bibinfo {volume} {5}},\ \bibinfo {pages} {3732}},\
  \Eprint {http://arxiv.org/abs/arXiv:1307.1081} {arXiv:1307.1081} \BibitemShut
  {NoStop}%
\bibitem [{\citenamefont {Damg{\aa}rd}\ \emph {et~al.}(2007)\citenamefont
  {Damg{\aa}rd}, \citenamefont {Fehr}, \citenamefont {Salvail},\ and\
  \citenamefont {Schaffner}}]{DFSS07}%
  \BibitemOpen
  \bibfield  {author} {\bibinfo {author} {\bibnamefont {Damg{\aa}rd},
  \bibfnamefont {Ivan~B}}, \bibinfo {author} {\bibfnamefont {Serge}\
  \bibnamefont {Fehr}}, \bibinfo {author} {\bibfnamefont {Louis}\ \bibnamefont
  {Salvail}}, \ and\ \bibinfo {author} {\bibfnamefont {Christian}\ \bibnamefont
  {Schaffner}}} (\bibinfo {year} {2007}),\ \bibfield  {title} {\enquote
  {\bibinfo {title} {Secure identification and {QKD} in the
  bounded-quantum-storage model},}\ }in\ \href {\doibase
  10.1007/978-3-540-74143-5_19} {\emph {\bibinfo {booktitle} {Advances in
  Cryptology -- CRYPTO 2007}}},\ \bibinfo {editor} {edited by\ \bibinfo
  {editor} {\bibfnamefont {Alfred}\ \bibnamefont {Menezes}}}\ (\bibinfo
  {publisher} {Springer})\ pp.\ \bibinfo {pages} {342--359}\BibitemShut
  {NoStop}%
\bibitem [{\citenamefont {Damg{\aa}rd}\ \emph {et~al.}(2008)\citenamefont
  {Damg{\aa}rd}, \citenamefont {Fehr}, \citenamefont {Salvail},\ and\
  \citenamefont {Schaffner}}]{DFSS08}%
  \BibitemOpen
  \bibfield  {author} {\bibinfo {author} {\bibnamefont {Damg{\aa}rd},
  \bibfnamefont {Ivan~B}}, \bibinfo {author} {\bibfnamefont {Serge}\
  \bibnamefont {Fehr}}, \bibinfo {author} {\bibfnamefont {Louis}\ \bibnamefont
  {Salvail}}, \ and\ \bibinfo {author} {\bibfnamefont {Christian}\ \bibnamefont
  {Schaffner}}} (\bibinfo {year} {2008}),\ \bibfield  {title} {\enquote
  {\bibinfo {title} {Cryptography in the bounded-quantum-storage model},}\
  }\href {\doibase 10.1137/060651343} {\bibfield  {journal} {\bibinfo
  {journal} {SIAM J. Comput.}\ }\textbf {\bibinfo {volume} {37}}~(\bibinfo
  {number} {6}),\ \bibinfo {pages} {1865--1890}},\ \bibinfo {note} {a
  preliminary version appeared at FOCS '05},\ \Eprint
  {http://arxiv.org/abs/arXiv:quant-ph/0508222} {arXiv:quant-ph/0508222}
  \BibitemShut {NoStop}%
\bibitem [{\citenamefont {De}\ \emph {et~al.}(2012)\citenamefont {De},
  \citenamefont {Portmann}, \citenamefont {Vidick},\ and\ \citenamefont
  {Renner}}]{DPVR12}%
  \BibitemOpen
  \bibfield  {author} {\bibinfo {author} {\bibnamefont {De}, \bibfnamefont
  {Anindya}}, \bibinfo {author} {\bibfnamefont {Christopher}\ \bibnamefont
  {Portmann}}, \bibinfo {author} {\bibfnamefont {Thomas}\ \bibnamefont
  {Vidick}}, \ and\ \bibinfo {author} {\bibfnamefont {Renato}\ \bibnamefont
  {Renner}}} (\bibinfo {year} {2012}),\ \bibfield  {title} {\enquote {\bibinfo
  {title} {Trevisan's extractor in the presence of quantum side information},}\
  }\href {\doibase 10.1137/100813683} {\bibfield  {journal} {\bibinfo
  {journal} {SIAM J. Comput.}\ }\textbf {\bibinfo {volume} {41}}~(\bibinfo
  {number} {4}),\ \bibinfo {pages} {915--940}},\ \Eprint
  {http://arxiv.org/abs/arXiv:0912.5514} {arXiv:0912.5514} \BibitemShut
  {NoStop}%
\bibitem [{\citenamefont {Demay}\ and\ \citenamefont {Maurer}(2013)}]{DM13}%
  \BibitemOpen
  \bibfield  {author} {\bibinfo {author} {\bibnamefont {Demay}, \bibfnamefont
  {Gregory}}, \ and\ \bibinfo {author} {\bibfnamefont {Ueli}\ \bibnamefont
  {Maurer}}} (\bibinfo {year} {2013}),\ \bibfield  {title} {\enquote {\bibinfo
  {title} {Unfair coin tossing},}\ }in\ \href {\doibase
  10.1109/ISIT.2013.6620488} {\emph {\bibinfo {booktitle} {Proceedings of the
  2013 IEEE International Symposium on Information Theory, ISIT 2013}}}\
  (\bibinfo  {publisher} {IEEE})\ pp.\ \bibinfo {pages}
  {1556--1560}\BibitemShut {NoStop}%
\bibitem [{\citenamefont {Devetak}\ and\ \citenamefont {Winter}(2005)}]{DW05}%
  \BibitemOpen
  \bibfield  {author} {\bibinfo {author} {\bibnamefont {Devetak}, \bibfnamefont
  {Igor}}, \ and\ \bibinfo {author} {\bibfnamefont {Andreas}\ \bibnamefont
  {Winter}}} (\bibinfo {year} {2005}),\ \bibfield  {title} {\enquote {\bibinfo
  {title} {Distillation of secret key and entanglement from quantum states},}\
  }\href {\doibase 10.1098/rspa.2004.1372} {\bibfield  {journal} {\bibinfo
  {journal} {Proc. R. Soc. London, Ser. A}\ }\textbf {\bibinfo {volume}
  {461}}~(\bibinfo {number} {2053}),\ \bibinfo {pages} {207--235}},\ \Eprint
  {http://arxiv.org/abs/arXiv:quant-ph/0306078} {arXiv:quant-ph/0306078}
  \BibitemShut {NoStop}%
\bibitem [{\citenamefont {Dickinson}\ and\ \citenamefont {Nayak}(2006)}]{DN06}%
  \BibitemOpen
  \bibfield  {author} {\bibinfo {author} {\bibnamefont {Dickinson},
  \bibfnamefont {Paul}}, \ and\ \bibinfo {author} {\bibfnamefont {Ashwin}\
  \bibnamefont {Nayak}}} (\bibinfo {year} {2006}),\ \bibfield  {title}
  {\enquote {\bibinfo {title} {Approximate randomization of quantum states with
  fewer bits of key},}\ }in\ \href {\doibase 10.1063/1.2400876} {\emph
  {\bibinfo {booktitle} {AIP Conference Proceedings}}},\ Vol.\ \bibinfo
  {volume} {864},\ pp.\ \bibinfo {pages} {18--36},\ \Eprint
  {http://arxiv.org/abs/arXiv:quant-ph/0611033} {arXiv:quant-ph/0611033}
  \BibitemShut {NoStop}%
\bibitem [{\citenamefont {DiVincenzo}\ \emph {et~al.}(2004)\citenamefont
  {DiVincenzo}, \citenamefont {Horodecki}, \citenamefont {Leung}, \citenamefont
  {Smolin},\ and\ \citenamefont {Terhal}}]{DHLST04}%
  \BibitemOpen
  \bibfield  {author} {\bibinfo {author} {\bibnamefont {DiVincenzo},
  \bibfnamefont {David}}, \bibinfo {author} {\bibfnamefont {Micha\l{}}\
  \bibnamefont {Horodecki}}, \bibinfo {author} {\bibfnamefont {Debbie}\
  \bibnamefont {Leung}}, \bibinfo {author} {\bibfnamefont {John}\ \bibnamefont
  {Smolin}}, \ and\ \bibinfo {author} {\bibfnamefont {Barbara}\ \bibnamefont
  {Terhal}}} (\bibinfo {year} {2004}),\ \bibfield  {title} {\enquote {\bibinfo
  {title} {Locking classical correlation in quantum states},}\ }\href@noop {}
  {\bibfield  {journal} {\bibinfo  {journal} {Phys. Rev. Lett.}\ }\textbf
  {\bibinfo {volume} {92}},\ \bibinfo {pages} {067902}},\ \Eprint
  {http://arxiv.org/abs/arXiv:quant-ph/0303088} {arXiv:quant-ph/0303088}
  \BibitemShut {NoStop}%
\bibitem [{\citenamefont {Dodis}\ and\ \citenamefont {Wichs}(2009)}]{DW09}%
  \BibitemOpen
  \bibfield  {author} {\bibinfo {author} {\bibnamefont {Dodis}, \bibfnamefont
  {Yevgeniy}}, \ and\ \bibinfo {author} {\bibfnamefont {Daniel}\ \bibnamefont
  {Wichs}}} (\bibinfo {year} {2009}),\ \bibfield  {title} {\enquote {\bibinfo
  {title} {Non-malleable extractors and symmetric key cryptography from weak
  secrets},}\ }in\ \href {\doibase 10.1145/1536414.1536496} {\emph {\bibinfo
  {booktitle} {Proceedings of the 41st Symposium on Theory of Computing,
  STOC~'09}}}\ (\bibinfo  {publisher} {ACM})\ pp.\ \bibinfo {pages}
  {601--610},\ \bibinfo {note} {e-Print
  \href{http://eprint.iacr.org/2008/503}{IACR 2008/503}}\BibitemShut {NoStop}%
\bibitem [{\citenamefont {Dulek}\ \emph {et~al.}(2020)\citenamefont {Dulek},
  \citenamefont {Grilo}, \citenamefont {Jeffery}, \citenamefont {Majenz},\ and\
  \citenamefont {Schaffner}}]{DGJMS20}%
  \BibitemOpen
  \bibfield  {author} {\bibinfo {author} {\bibnamefont {Dulek}, \bibfnamefont
  {Yfke}}, \bibinfo {author} {\bibfnamefont {Alex~B.}\ \bibnamefont {Grilo}},
  \bibinfo {author} {\bibfnamefont {Stacey}\ \bibnamefont {Jeffery}}, \bibinfo
  {author} {\bibfnamefont {Christian}\ \bibnamefont {Majenz}}, \ and\ \bibinfo
  {author} {\bibfnamefont {Christian}\ \bibnamefont {Schaffner}}} (\bibinfo
  {year} {2020}),\ \bibfield  {title} {\enquote {\bibinfo {title} {Secure
  multi-party quantum computation with a dishonest majority},}\ }in\ \href
  {\doibase 10.1007/978-3-030-45727-3_25} {\emph {\bibinfo {booktitle}
  {Advances in Cryptology -- EUROCRYPT 2020}}},\ \bibinfo {editor} {edited by\
  \bibinfo {editor} {\bibfnamefont {Anne}\ \bibnamefont {Canteaut}}\ and\
  \bibinfo {editor} {\bibfnamefont {Yuval}\ \bibnamefont {Ishai}}}\ (\bibinfo
  {publisher} {Springer})\ pp.\ \bibinfo {pages} {729--758},\ \Eprint
  {http://arxiv.org/abs/arXiv:1909.13770} {arXiv:1909.13770} \BibitemShut
  {NoStop}%
\bibitem [{\citenamefont {Dunjko}\ \emph {et~al.}(2014)\citenamefont {Dunjko},
  \citenamefont {Fitzsimons}, \citenamefont {Portmann},\ and\ \citenamefont
  {Renner}}]{DFPR14}%
  \BibitemOpen
  \bibfield  {author} {\bibinfo {author} {\bibnamefont {Dunjko}, \bibfnamefont
  {Vedran}}, \bibinfo {author} {\bibfnamefont {Joseph}\ \bibnamefont
  {Fitzsimons}}, \bibinfo {author} {\bibfnamefont {Christopher}\ \bibnamefont
  {Portmann}}, \ and\ \bibinfo {author} {\bibfnamefont {Renato}\ \bibnamefont
  {Renner}}} (\bibinfo {year} {2014}),\ \bibfield  {title} {\enquote {\bibinfo
  {title} {Composable security of delegated quantum computation},}\ }in\ \href
  {\doibase 10.1007/978-3-662-45608-8_22} {\emph {\bibinfo {booktitle}
  {Advances in Cryptology -- ASIACRYPT 2014, Proceedings, Part II}}},\ \bibinfo
  {series} {LNCS}, Vol.\ \bibinfo {volume} {8874}\ (\bibinfo  {publisher}
  {Springer})\ pp.\ \bibinfo {pages} {406--425},\ \Eprint
  {http://arxiv.org/abs/arXiv:1301.3662} {arXiv:1301.3662} \BibitemShut
  {NoStop}%
\bibitem [{\citenamefont {Dunjko}\ and\ \citenamefont {Kashefi}(2016)}]{DK16}%
  \BibitemOpen
  \bibfield  {author} {\bibinfo {author} {\bibnamefont {Dunjko}, \bibfnamefont
  {Vedran}}, \ and\ \bibinfo {author} {\bibfnamefont {Elham}\ \bibnamefont
  {Kashefi}}} (\bibinfo {year} {2016}),\ \href@noop {} {\enquote {\bibinfo
  {title} {Blind quantum computing with two almost identical states},}\
  }\bibinfo {howpublished} {e-Print},\ \Eprint
  {http://arxiv.org/abs/arXiv:1604.01586} {arXiv:1604.01586} \BibitemShut
  {NoStop}%
\bibitem [{\citenamefont {Dupuis}\ \emph {et~al.}(2020)\citenamefont {Dupuis},
  \citenamefont {Fawzi},\ and\ \citenamefont {Renner}}]{DFR20}%
  \BibitemOpen
  \bibfield  {author} {\bibinfo {author} {\bibnamefont {Dupuis}, \bibfnamefont
  {Fr{\'e}d{\'e}ric}}, \bibinfo {author} {\bibfnamefont {Omar}\ \bibnamefont
  {Fawzi}}, \ and\ \bibinfo {author} {\bibfnamefont {Renato}\ \bibnamefont
  {Renner}}} (\bibinfo {year} {2020}),\ \bibfield  {title} {\enquote {\bibinfo
  {title} {Entropy accumulation},}\ }\href {\doibase
  10.1007/s00220-020-03839-5} {\bibfield  {journal} {\bibinfo  {journal}
  {Commun. Math. Phys.}\ }\textbf {\bibinfo {volume} {379}}~(\bibinfo {number}
  {3}),\ \bibinfo {pages} {867--913}},\ \Eprint
  {http://arxiv.org/abs/arXiv:1607.01796} {arXiv:1607.01796} \BibitemShut
  {NoStop}%
\bibitem [{\citenamefont {Dupuis}\ \emph {et~al.}(2012)\citenamefont {Dupuis},
  \citenamefont {Nielsen},\ and\ \citenamefont {Salvail}}]{DNS12}%
  \BibitemOpen
  \bibfield  {author} {\bibinfo {author} {\bibnamefont {Dupuis}, \bibfnamefont
  {Fr{\'e}d{\'e}ric}}, \bibinfo {author} {\bibfnamefont {Jesper~B.}\
  \bibnamefont {Nielsen}}, \ and\ \bibinfo {author} {\bibfnamefont {Louis}\
  \bibnamefont {Salvail}}} (\bibinfo {year} {2012}),\ \bibfield  {title}
  {\enquote {\bibinfo {title} {Actively secure two-party evaluation of any
  quantum operation},}\ }in\ \href {\doibase 10.1007/978-3-642-32009-5_46}
  {\emph {\bibinfo {booktitle} {Advances in Cryptology -- CRYPTO 2012}}},\
  \bibinfo {series} {LNCS}, Vol.\ \bibinfo {volume} {7417},\ \bibinfo {editor}
  {edited by\ \bibinfo {editor} {\bibfnamefont {Reihaneh}\ \bibnamefont
  {Safavi-Naini}}\ and\ \bibinfo {editor} {\bibfnamefont {Ran}\ \bibnamefont
  {Canetti}}}\ (\bibinfo  {publisher} {Springer})\ pp.\ \bibinfo {pages}
  {794--811},\ \bibinfo {note} {e-Print
  \href{http://eprint.iacr.org/2012/304}{IACR 2012/304}}\BibitemShut {NoStop}%
\bibitem [{\citenamefont {Einstein}\ \emph {et~al.}(1935)\citenamefont
  {Einstein}, \citenamefont {Podolsky},\ and\ \citenamefont {Rosen}}]{EPR35}%
  \BibitemOpen
  \bibfield  {author} {\bibinfo {author} {\bibnamefont {Einstein},
  \bibfnamefont {Albert}}, \bibinfo {author} {\bibfnamefont {Boris}\
  \bibnamefont {Podolsky}}, \ and\ \bibinfo {author} {\bibfnamefont {Nathan}\
  \bibnamefont {Rosen}}} (\bibinfo {year} {1935}),\ \bibfield  {title}
  {\enquote {\bibinfo {title} {Can quantum-mechanical description of physical
  reality be considered complete?}}\ }\href {\doibase 10.1103/PhysRev.47.777}
  {\bibfield  {journal} {\bibinfo  {journal} {Phys. Rev.}\ }\textbf {\bibinfo
  {volume} {47}},\ \bibinfo {pages} {777--780}}\BibitemShut {NoStop}%
\bibitem [{\citenamefont {Ekert}(1991)}]{Eke91}%
  \BibitemOpen
  \bibfield  {author} {\bibinfo {author} {\bibnamefont {Ekert}, \bibfnamefont
  {Artur}}} (\bibinfo {year} {1991}),\ \bibfield  {title} {\enquote {\bibinfo
  {title} {Quantum cryptography based on {Bell}'s theorem},}\ }\href {\doibase
  10.1103/PhysRevLett.67.661} {\bibfield  {journal} {\bibinfo  {journal} {Phys.
  Rev. Lett.}\ }\textbf {\bibinfo {volume} {67}},\ \bibinfo {pages}
  {661--663}}\BibitemShut {NoStop}%
\bibitem [{\citenamefont {Ekert}\ and\ \citenamefont {Renner}(2014)}]{ER14}%
  \BibitemOpen
  \bibfield  {author} {\bibinfo {author} {\bibnamefont {Ekert}, \bibfnamefont
  {Artur}}, \ and\ \bibinfo {author} {\bibfnamefont {Renato}\ \bibnamefont
  {Renner}}} (\bibinfo {year} {2014}),\ \bibfield  {title} {\enquote {\bibinfo
  {title} {The ultimate physical limits of privacy},}\ }\href {\doibase
  10.1038/nature13132} {\bibfield  {journal} {\bibinfo  {journal} {Nature}\
  }\textbf {\bibinfo {volume} {507}}~(\bibinfo {number} {7493}),\ \bibinfo
  {pages} {443--447}},\ \bibinfo {note} {perspectives}\BibitemShut {NoStop}%
\bibitem [{\citenamefont {Elkouss}\ \emph {et~al.}(2009)\citenamefont
  {Elkouss}, \citenamefont {Leverrier}, \citenamefont {Alleaume},\ and\
  \citenamefont {Boutros}}]{ELAB09}%
  \BibitemOpen
  \bibfield  {author} {\bibinfo {author} {\bibnamefont {Elkouss}, \bibfnamefont
  {David}}, \bibinfo {author} {\bibfnamefont {Anthony}\ \bibnamefont
  {Leverrier}}, \bibinfo {author} {\bibfnamefont {Romain}\ \bibnamefont
  {Alleaume}}, \ and\ \bibinfo {author} {\bibfnamefont {Joseph~J.}\
  \bibnamefont {Boutros}}} (\bibinfo {year} {2009}),\ \bibfield  {title}
  {\enquote {\bibinfo {title} {Efficient reconciliation protocol for
  discrete-variable quantum key distribution},}\ }in\ \href {\doibase
  10.1109/ISIT.2009.5205475} {\emph {\bibinfo {booktitle} {Proceedings of the
  2009 IEEE International Symposium on Information Theory, ISIT 2009}}}\
  (\bibinfo  {publisher} {IEEE})\ pp.\ \bibinfo {pages}
  {1879--1883}\BibitemShut {NoStop}%
\bibitem [{\citenamefont {Elkouss}\ \emph {et~al.}(2011)\citenamefont
  {Elkouss}, \citenamefont {Martinez-Mateo},\ and\ \citenamefont
  {Martin}}]{EMM11}%
  \BibitemOpen
  \bibfield  {author} {\bibinfo {author} {\bibnamefont {Elkouss}, \bibfnamefont
  {David}}, \bibinfo {author} {\bibfnamefont {Jesus}\ \bibnamefont
  {Martinez-Mateo}}, \ and\ \bibinfo {author} {\bibfnamefont {Vicente}\
  \bibnamefont {Martin}}} (\bibinfo {year} {2011}),\ \bibfield  {title}
  {\enquote {\bibinfo {title} {Information reconciliation for quantum key
  distribution},}\ }\href@noop {} {\bibfield  {journal} {\bibinfo  {journal}
  {Quantum Inf. Comput.}\ }\textbf {\bibinfo {volume} {11}}~(\bibinfo {number}
  {3}),\ \bibinfo {pages} {226--238}}\BibitemShut {NoStop}%
\bibitem [{\citenamefont {Fehr}\ and\ \citenamefont {Schaffner}(2008)}]{FS08}%
  \BibitemOpen
  \bibfield  {author} {\bibinfo {author} {\bibnamefont {Fehr}, \bibfnamefont
  {Serge}}, \ and\ \bibinfo {author} {\bibfnamefont {Christian}\ \bibnamefont
  {Schaffner}}} (\bibinfo {year} {2008}),\ \bibfield  {title} {\enquote
  {\bibinfo {title} {Randomness extraction via $\delta$-biased masking in the
  presence of a quantum attacker},}\ }in\ \href {\doibase
  10.1007/978-3-540-78524-8_26} {\emph {\bibinfo {booktitle} {Theory of
  Cryptography, Proceedings of TCC 2008}}},\ \bibinfo {series} {LNCS}, Vol.\
  \bibinfo {volume} {4948}\ (\bibinfo  {publisher} {Springer})\ pp.\ \bibinfo
  {pages} {465--481},\ \Eprint {http://arxiv.org/abs/arXiv:0706.2606}
  {arXiv:0706.2606} \BibitemShut {NoStop}%
\bibitem [{\citenamefont {Fitzsimons}\ and\ \citenamefont
  {Kashefi}(2017)}]{FK17}%
  \BibitemOpen
  \bibfield  {author} {\bibinfo {author} {\bibnamefont {Fitzsimons},
  \bibfnamefont {Joseph~F}}, \ and\ \bibinfo {author} {\bibfnamefont {Elham}\
  \bibnamefont {Kashefi}}} (\bibinfo {year} {2017}),\ \bibfield  {title}
  {\enquote {\bibinfo {title} {Unconditionally verifiable blind computation},}\
  }\href {\doibase 10.1103/PhysRevA.96.012303} {\bibfield  {journal} {\bibinfo
  {journal} {Phys. Rev. A}\ }\textbf {\bibinfo {volume} {96}},\ \bibinfo
  {pages} {012303}},\ \Eprint {http://arxiv.org/abs/arXiv:1203.5217}
  {arXiv:1203.5217} \BibitemShut {NoStop}%
\bibitem [{\citenamefont {Freedman}\ and\ \citenamefont
  {Clauser}(1972)}]{FreedmanClauser}%
  \BibitemOpen
  \bibfield  {author} {\bibinfo {author} {\bibnamefont {Freedman},
  \bibfnamefont {Stuart~J}}, \ and\ \bibinfo {author} {\bibfnamefont {John~F.}\
  \bibnamefont {Clauser}}} (\bibinfo {year} {1972}),\ \bibfield  {title}
  {\enquote {\bibinfo {title} {Experimental test of local hidden-variable
  theories},}\ }\href {\doibase 10.1103/PhysRevLett.28.938} {\bibfield
  {journal} {\bibinfo  {journal} {Phys. Rev. Lett.}\ }\textbf {\bibinfo
  {volume} {28}},\ \bibinfo {pages} {938--941}}\BibitemShut {NoStop}%
\bibitem [{\citenamefont {Fuchs}(1998)}]{Fuchs98}%
  \BibitemOpen
  \bibfield  {author} {\bibinfo {author} {\bibnamefont {Fuchs}, \bibfnamefont
  {Christopher~A}}} (\bibinfo {year} {1998}),\ \bibfield  {title} {\enquote
  {\bibinfo {title} {Information gain vs.\ state disturbance in quantum
  theory},}\ }\href@noop {} {\bibfield  {journal} {\bibinfo  {journal}
  {Fortschritte der Physik: Progress of Physics}\ }\textbf {\bibinfo {volume}
  {46}}~(\bibinfo {number} {4-5}),\ \bibinfo {pages} {535--565}},\ \Eprint
  {http://arxiv.org/abs/arXiv:quant-ph/9611010} {arXiv:quant-ph/9611010}
  \BibitemShut {NoStop}%
\bibitem [{\citenamefont {Fuchs}\ \emph {et~al.}(1997)\citenamefont {Fuchs},
  \citenamefont {Gisin}, \citenamefont {Griffiths}, \citenamefont {Niu},\ and\
  \citenamefont {Peres}}]{Fuchsetal1997}%
  \BibitemOpen
  \bibfield  {author} {\bibinfo {author} {\bibnamefont {Fuchs}, \bibfnamefont
  {Christopher~A}}, \bibinfo {author} {\bibfnamefont {Nicolas}\ \bibnamefont
  {Gisin}}, \bibinfo {author} {\bibfnamefont {Robert~B.}\ \bibnamefont
  {Griffiths}}, \bibinfo {author} {\bibfnamefont {Chi-Sheng}\ \bibnamefont
  {Niu}}, \ and\ \bibinfo {author} {\bibfnamefont {Asher}\ \bibnamefont
  {Peres}}} (\bibinfo {year} {1997}),\ \bibfield  {title} {\enquote {\bibinfo
  {title} {Optimal eavesdropping in quantum cryptography. i. information bound
  and optimal strategy},}\ }\href {\doibase 10.1103/PhysRevA.56.1163}
  {\bibfield  {journal} {\bibinfo  {journal} {Phys. Rev. A}\ }\textbf {\bibinfo
  {volume} {56}},\ \bibinfo {pages} {1163--1172}}\BibitemShut {NoStop}%
\bibitem [{\citenamefont {Fuchs}\ and\ \citenamefont {Van
  De~Graaf}(1999)}]{FuchsvanGraaf}%
  \BibitemOpen
  \bibfield  {author} {\bibinfo {author} {\bibnamefont {Fuchs}, \bibfnamefont
  {Christopher~A}}, \ and\ \bibinfo {author} {\bibfnamefont {Jeroen}\
  \bibnamefont {Van De~Graaf}}} (\bibinfo {year} {1999}),\ \bibfield  {title}
  {\enquote {\bibinfo {title} {Cryptographic distinguishability measures for
  quantum-mechanical states},}\ }\href@noop {} {\bibfield  {journal} {\bibinfo
  {journal} {IEEE Trans. Inf. Theory}\ }\textbf {\bibinfo {volume}
  {45}}~(\bibinfo {number} {4}),\ \bibinfo {pages} {1216--1227}}\BibitemShut
  {NoStop}%
\bibitem [{\citenamefont {Fung}\ \emph {et~al.}(2007)\citenamefont {Fung},
  \citenamefont {Qi}, \citenamefont {Tamaki},\ and\ \citenamefont
  {Lo}}]{FQTL07}%
  \BibitemOpen
  \bibfield  {author} {\bibinfo {author} {\bibnamefont {Fung}, \bibfnamefont
  {Chi-Hang~Fred}}, \bibinfo {author} {\bibfnamefont {Bing}\ \bibnamefont
  {Qi}}, \bibinfo {author} {\bibfnamefont {Kiyoshi}\ \bibnamefont {Tamaki}}, \
  and\ \bibinfo {author} {\bibfnamefont {Hoi-Kwong}\ \bibnamefont {Lo}}}
  (\bibinfo {year} {2007}),\ \bibfield  {title} {\enquote {\bibinfo {title}
  {Phase-remapping attack in practical quantum-key-distribution systems},}\
  }\href {\doibase 10.1103/PhysRevA.75.032314} {\bibfield  {journal} {\bibinfo
  {journal} {Phys. Rev. A}\ }\textbf {\bibinfo {volume} {75}}~(\bibinfo
  {number} {3}),\ \bibinfo {pages} {032314}},\ \Eprint
  {http://arxiv.org/abs/arXiv:quant-ph/0601115} {arXiv:quant-ph/0601115}
  \BibitemShut {NoStop}%
\bibitem [{\citenamefont {Garg}\ \emph {et~al.}(2017)\citenamefont {Garg},
  \citenamefont {Yuen},\ and\ \citenamefont {Zhandry}}]{GYZ17}%
  \BibitemOpen
  \bibfield  {author} {\bibinfo {author} {\bibnamefont {Garg}, \bibfnamefont
  {Sumegha}}, \bibinfo {author} {\bibfnamefont {Henry}\ \bibnamefont {Yuen}}, \
  and\ \bibinfo {author} {\bibfnamefont {Mark}\ \bibnamefont {Zhandry}}}
  (\bibinfo {year} {2017}),\ \bibfield  {title} {\enquote {\bibinfo {title}
  {New security notions and feasibility results for authentication of quantum
  data},}\ }in\ \href {\doibase 10.1007/978-3-319-63715-0_12} {\emph {\bibinfo
  {booktitle} {Advances in Cryptology -- CRYPTO 2017}}},\ \bibinfo {series}
  {LNCS}, Vol.\ \bibinfo {volume} {10402},\ \bibinfo {editor} {edited by\
  \bibinfo {editor} {\bibfnamefont {Jonathan}\ \bibnamefont {Katz}}\ and\
  \bibinfo {editor} {\bibfnamefont {Hovav}\ \bibnamefont {Shacham}}}\ (\bibinfo
   {publisher} {Springer})\ pp.\ \bibinfo {pages} {342--371},\ \Eprint
  {http://arxiv.org/abs/arXiv:1607.07759} {arXiv:1607.07759} \BibitemShut
  {NoStop}%
\bibitem [{\citenamefont {Gavinsky}\ \emph {et~al.}(2007)\citenamefont
  {Gavinsky}, \citenamefont {Kempe}, \citenamefont {Kerenidis}, \citenamefont
  {Raz},\ and\ \citenamefont {de~Wolf}}]{GKKRD07}%
  \BibitemOpen
  \bibfield  {author} {\bibinfo {author} {\bibnamefont {Gavinsky},
  \bibfnamefont {Dmitry}}, \bibinfo {author} {\bibfnamefont {Julia}\
  \bibnamefont {Kempe}}, \bibinfo {author} {\bibfnamefont {Iordanis}\
  \bibnamefont {Kerenidis}}, \bibinfo {author} {\bibfnamefont {Ran}\
  \bibnamefont {Raz}}, \ and\ \bibinfo {author} {\bibfnamefont {Ronald}\
  \bibnamefont {de~Wolf}}} (\bibinfo {year} {2007}),\ \bibfield  {title}
  {\enquote {\bibinfo {title} {Exponential separations for one-way quantum
  communication complexity, with applications to cryptography},}\ }in\ \href
  {\doibase 10.1145/1250790.1250866} {\emph {\bibinfo {booktitle} {Proceedings
  of the 39th Symposium on Theory of Computing, STOC~'07}}}\ (\bibinfo
  {publisher} {ACM})\ pp.\ \bibinfo {pages} {516--525},\ \Eprint
  {http://arxiv.org/abs/arXiv:quant-ph/0611209} {arXiv:quant-ph/0611209}
  \BibitemShut {NoStop}%
\bibitem [{\citenamefont {Gerhardt}\ \emph {et~al.}(2011)\citenamefont
  {Gerhardt}, \citenamefont {Liu}, \citenamefont {Lamas-Linares}, \citenamefont
  {Skaar}, \citenamefont {Kurtsiefer},\ and\ \citenamefont
  {Makarov}}]{GLLSKM11}%
  \BibitemOpen
  \bibfield  {author} {\bibinfo {author} {\bibnamefont {Gerhardt},
  \bibfnamefont {Ilja}}, \bibinfo {author} {\bibfnamefont {Qin}\ \bibnamefont
  {Liu}}, \bibinfo {author} {\bibfnamefont {Ant{\'\i}a}\ \bibnamefont
  {Lamas-Linares}}, \bibinfo {author} {\bibfnamefont {Johannes}\ \bibnamefont
  {Skaar}}, \bibinfo {author} {\bibfnamefont {Christian}\ \bibnamefont
  {Kurtsiefer}}, \ and\ \bibinfo {author} {\bibfnamefont {Vadim}\ \bibnamefont
  {Makarov}}} (\bibinfo {year} {2011}),\ \bibfield  {title} {\enquote {\bibinfo
  {title} {Full-field implementation of a perfect eavesdropper on a quantum
  cryptography system},}\ }\href {\doibase 10.1038/ncomms1348} {\bibfield
  {journal} {\bibinfo  {journal} {Nat. Commun.}\ }\textbf {\bibinfo {volume}
  {2}},\ \bibinfo {pages} {349}},\ \Eprint
  {http://arxiv.org/abs/arXiv:1011.0105} {arXiv:1011.0105} \BibitemShut
  {NoStop}%
\bibitem [{\citenamefont {Gheorghiu}\ and\ \citenamefont
  {Vidick}(2019)}]{GV19}%
  \BibitemOpen
  \bibfield  {author} {\bibinfo {author} {\bibnamefont {Gheorghiu},
  \bibfnamefont {Alexandru}}, \ and\ \bibinfo {author} {\bibfnamefont {Thomas}\
  \bibnamefont {Vidick}}} (\bibinfo {year} {2019}),\ \bibfield  {title}
  {\enquote {\bibinfo {title} {Computationally-secure and composable remote
  state preparation},}\ }in\ \href {\doibase 10.1109/FOCS.2019.00066} {\emph
  {\bibinfo {booktitle} {Proceedings of the 60th Symposium on Foundations of
  Computer Science, FOCS~'19}}},\ pp.\ \bibinfo {pages} {1024--1033},\ \Eprint
  {http://arxiv.org/abs/arXiv:1904.06320} {arXiv:1904.06320} \BibitemShut
  {NoStop}%
\bibitem [{\citenamefont {Gisin}\ \emph {et~al.}(2006)\citenamefont {Gisin},
  \citenamefont {Fasel}, \citenamefont {Kraus}, \citenamefont {Zbinden},\ and\
  \citenamefont {Ribordy}}]{GisinFaselKraus2006}%
  \BibitemOpen
  \bibfield  {author} {\bibinfo {author} {\bibnamefont {Gisin}, \bibfnamefont
  {Nicolas}}, \bibinfo {author} {\bibfnamefont {Sylvain}\ \bibnamefont
  {Fasel}}, \bibinfo {author} {\bibfnamefont {Barbara}\ \bibnamefont {Kraus}},
  \bibinfo {author} {\bibfnamefont {Hugo}\ \bibnamefont {Zbinden}}, \ and\
  \bibinfo {author} {\bibfnamefont {Gr\'egoire}\ \bibnamefont {Ribordy}}}
  (\bibinfo {year} {2006}),\ \bibfield  {title} {\enquote {\bibinfo {title}
  {Trojan-horse attacks on quantum-key-distribution systems},}\ }\href
  {\doibase 10.1103/PhysRevA.73.022320} {\bibfield  {journal} {\bibinfo
  {journal} {Phys. Rev. A}\ }\textbf {\bibinfo {volume} {73}},\ \bibinfo
  {pages} {022320}},\ \Eprint {http://arxiv.org/abs/arXiv:quant-ph/0507063}
  {arXiv:quant-ph/0507063} \BibitemShut {NoStop}%
\bibitem [{\citenamefont {Giustina}\ \emph {et~al.}(2013)\citenamefont
  {Giustina}, \citenamefont {Mech}, \citenamefont {Ramelow}, \citenamefont
  {Wittmann}, \citenamefont {Kofler}, \citenamefont {Beyer}, \citenamefont
  {Lita}, \citenamefont {Calkins}, \citenamefont {Gerrits}, \citenamefont
  {Nam}, \citenamefont {Ursin},\ and\ \citenamefont {Zeilinger}}]{Giustina13}%
  \BibitemOpen
  \bibfield  {author} {\bibinfo {author} {\bibnamefont {Giustina},
  \bibfnamefont {Marissa}}, \bibinfo {author} {\bibfnamefont {Alexandra}\
  \bibnamefont {Mech}}, \bibinfo {author} {\bibfnamefont {Sven}\ \bibnamefont
  {Ramelow}}, \bibinfo {author} {\bibfnamefont {Bernhard}\ \bibnamefont
  {Wittmann}}, \bibinfo {author} {\bibfnamefont {Johannes}\ \bibnamefont
  {Kofler}}, \bibinfo {author} {\bibfnamefont {J{\"o}rn}\ \bibnamefont
  {Beyer}}, \bibinfo {author} {\bibfnamefont {Adriana}\ \bibnamefont {Lita}},
  \bibinfo {author} {\bibfnamefont {Brice}\ \bibnamefont {Calkins}}, \bibinfo
  {author} {\bibfnamefont {Thomas}\ \bibnamefont {Gerrits}}, \bibinfo {author}
  {\bibfnamefont {Sae~Woo}\ \bibnamefont {Nam}}, \bibinfo {author}
  {\bibfnamefont {Rupert}\ \bibnamefont {Ursin}}, \ and\ \bibinfo {author}
  {\bibfnamefont {Anton}\ \bibnamefont {Zeilinger}}} (\bibinfo {year} {2013}),\
  \bibfield  {title} {\enquote {\bibinfo {title} {Bell violation using
  entangled photons without the fair-sampling assumption},}\ }\href {\doibase
  10.1038/nature12012} {\bibfield  {journal} {\bibinfo  {journal} {Nature}\
  }\textbf {\bibinfo {volume} {497}}~(\bibinfo {number} {7448}),\ \bibinfo
  {pages} {227--230}}\BibitemShut {NoStop}%
\bibitem [{\citenamefont {Giustina}\ \emph {et~al.}(2015)\citenamefont
  {Giustina}, \citenamefont {Versteegh}, \citenamefont {Wengerowsky},
  \citenamefont {Handsteiner}, \citenamefont {Hochrainer}, \citenamefont
  {Phelan}, \citenamefont {Steinlechner}, \citenamefont {Kofler}, \citenamefont
  {Larsson}, \citenamefont {Abell\'an}, \citenamefont {Amaya}, \citenamefont
  {Pruneri}, \citenamefont {Mitchell}, \citenamefont {Beyer}, \citenamefont
  {Gerrits}, \citenamefont {Lita}, \citenamefont {Shalm}, \citenamefont {Nam},
  \citenamefont {Scheidl}, \citenamefont {Ursin}, \citenamefont {Wittmann},\
  and\ \citenamefont {Zeilinger}}]{Giustina15}%
  \BibitemOpen
  \bibfield  {author} {\bibinfo {author} {\bibnamefont {Giustina},
  \bibfnamefont {Marissa}}, \bibinfo {author} {\bibfnamefont {Marijn A.~M.}\
  \bibnamefont {Versteegh}}, \bibinfo {author} {\bibfnamefont {S\"oren}\
  \bibnamefont {Wengerowsky}}, \bibinfo {author} {\bibfnamefont {Johannes}\
  \bibnamefont {Handsteiner}}, \bibinfo {author} {\bibfnamefont {Armin}\
  \bibnamefont {Hochrainer}}, \bibinfo {author} {\bibfnamefont {Kevin}\
  \bibnamefont {Phelan}}, \bibinfo {author} {\bibfnamefont {Fabian}\
  \bibnamefont {Steinlechner}}, \bibinfo {author} {\bibfnamefont {Johannes}\
  \bibnamefont {Kofler}}, \bibinfo {author} {\bibfnamefont {Jan-\AA{}ke}\
  \bibnamefont {Larsson}}, \bibinfo {author} {\bibfnamefont {Carlos}\
  \bibnamefont {Abell\'an}}, \bibinfo {author} {\bibfnamefont {Waldimar}\
  \bibnamefont {Amaya}}, \bibinfo {author} {\bibfnamefont {Valerio}\
  \bibnamefont {Pruneri}}, \bibinfo {author} {\bibfnamefont {Morgan~W.}\
  \bibnamefont {Mitchell}}, \bibinfo {author} {\bibfnamefont {J\"orn}\
  \bibnamefont {Beyer}}, \bibinfo {author} {\bibfnamefont {Thomas}\
  \bibnamefont {Gerrits}}, \bibinfo {author} {\bibfnamefont {Adriana~E.}\
  \bibnamefont {Lita}}, \bibinfo {author} {\bibfnamefont {Lynden~K.}\
  \bibnamefont {Shalm}}, \bibinfo {author} {\bibfnamefont {Sae~Woo}\
  \bibnamefont {Nam}}, \bibinfo {author} {\bibfnamefont {Thomas}\ \bibnamefont
  {Scheidl}}, \bibinfo {author} {\bibfnamefont {Rupert}\ \bibnamefont {Ursin}},
  \bibinfo {author} {\bibfnamefont {Bernhard}\ \bibnamefont {Wittmann}}, \ and\
  \bibinfo {author} {\bibfnamefont {Anton}\ \bibnamefont {Zeilinger}}}
  (\bibinfo {year} {2015}),\ \bibfield  {title} {\enquote {\bibinfo {title}
  {Significant-loophole-free test of {Bell}'s theorem with entangled
  photons},}\ }\href {\doibase 10.1103/PhysRevLett.115.250401} {\bibfield
  {journal} {\bibinfo  {journal} {Phys. Rev. Lett.}\ }\textbf {\bibinfo
  {volume} {115}},\ \bibinfo {pages} {250401}}\BibitemShut {NoStop}%
\bibitem [{\citenamefont {Goldreich}(2004)}]{Gol04}%
  \BibitemOpen
  \bibfield  {author} {\bibinfo {author} {\bibnamefont {Goldreich},
  \bibfnamefont {Oded}}} (\bibinfo {year} {2004}),\ \href@noop {} {\emph
  {\bibinfo {title} {Foundations of Cryptography: Volume 2, Basic
  Applications}}}\ (\bibinfo  {publisher} {Cambridge University Press},\
  \bibinfo {address} {New York, NY, USA})\BibitemShut {NoStop}%
\bibitem [{\citenamefont {Goldreich}\ \emph {et~al.}(1987)\citenamefont
  {Goldreich}, \citenamefont {Micali},\ and\ \citenamefont
  {Wigderson}}]{GMW87}%
  \BibitemOpen
  \bibfield  {author} {\bibinfo {author} {\bibnamefont {Goldreich},
  \bibfnamefont {Oded}}, \bibinfo {author} {\bibfnamefont {Silvia}\
  \bibnamefont {Micali}}, \ and\ \bibinfo {author} {\bibfnamefont {Avi}\
  \bibnamefont {Wigderson}}} (\bibinfo {year} {1987}),\ \bibfield  {title}
  {\enquote {\bibinfo {title} {How to play any mental game},}\ }in\ \href
  {\doibase 10.1145/28395.28420} {\emph {\bibinfo {booktitle} {Proceedings of
  the 19th Symposium on Theory of Computing, STOC~'87}}}\ (\bibinfo
  {publisher} {ACM})\ pp.\ \bibinfo {pages} {218--–229}\BibitemShut {NoStop}%
\bibitem [{\citenamefont {Goldreich}\ \emph {et~al.}(1986)\citenamefont
  {Goldreich}, \citenamefont {Micali},\ and\ \citenamefont
  {Wigderson}}]{GMW86}%
  \BibitemOpen
  \bibfield  {author} {\bibinfo {author} {\bibnamefont {Goldreich},
  \bibfnamefont {Oded}}, \bibinfo {author} {\bibfnamefont {Silvio}\
  \bibnamefont {Micali}}, \ and\ \bibinfo {author} {\bibfnamefont {Avi}\
  \bibnamefont {Wigderson}}} (\bibinfo {year} {1986}),\ \bibfield  {title}
  {\enquote {\bibinfo {title} {Proofs that yield nothing but their validity and
  a methodology of cryptographic protocol design},}\ }in\ \href {\doibase
  10.1109/SFCS.1986.47} {\emph {\bibinfo {booktitle} {Proceedings of the 27th
  Symposium on Foundations of Computer Science, FOCS~'86}}}\ (\bibinfo
  {publisher} {IEEE})\ pp.\ \bibinfo {pages} {174--187}\BibitemShut {NoStop}%
\bibitem [{\citenamefont {Gottesman}\ and\ \citenamefont {Lo}(2003)}]{GL03}%
  \BibitemOpen
  \bibfield  {author} {\bibinfo {author} {\bibnamefont {Gottesman},
  \bibfnamefont {Daniel}}, \ and\ \bibinfo {author} {\bibfnamefont {Hoi-Kwong}\
  \bibnamefont {Lo}}} (\bibinfo {year} {2003}),\ \bibfield  {title} {\enquote
  {\bibinfo {title} {Proof of security of quantum key distribution with two-way
  classical communications},}\ }\href {\doibase 10.1109/TIT.2002.807289}
  {\bibfield  {journal} {\bibinfo  {journal} {IEEE Trans. Inf. Theory}\
  }\textbf {\bibinfo {volume} {49}}~(\bibinfo {number} {2}),\ \bibinfo {pages}
  {457--475}},\ \Eprint {http://arxiv.org/abs/arXiv:quant-ph/0105121}
  {arXiv:quant-ph/0105121} \BibitemShut {NoStop}%
\bibitem [{\citenamefont {Gottesman}\ \emph {et~al.}(2004)\citenamefont
  {Gottesman}, \citenamefont {Lo}, \citenamefont {L\"{u}tkenhaus},\ and\
  \citenamefont {Preskill}}]{GLLP04}%
  \BibitemOpen
  \bibfield  {author} {\bibinfo {author} {\bibnamefont {Gottesman},
  \bibfnamefont {Daniel}}, \bibinfo {author} {\bibfnamefont {Hoi-Kwong}\
  \bibnamefont {Lo}}, \bibinfo {author} {\bibfnamefont {Norbert}\ \bibnamefont
  {L\"{u}tkenhaus}}, \ and\ \bibinfo {author} {\bibfnamefont {John}\
  \bibnamefont {Preskill}}} (\bibinfo {year} {2004}),\ \bibfield  {title}
  {\enquote {\bibinfo {title} {Security of quantum key distribution with
  imperfect devices},}\ }\href@noop {} {\bibfield  {journal} {\bibinfo
  {journal} {Quantum Inf. Comput.}\ }\textbf {\bibinfo {volume} {4}}~(\bibinfo
  {number} {5}),\ \bibinfo {pages} {325--360}},\ \Eprint
  {http://arxiv.org/abs/arXiv:quant-ph/0212066} {arXiv:quant-ph/0212066}
  \BibitemShut {NoStop}%
\bibitem [{\citenamefont {Goyal}\ \emph {et~al.}(2010)\citenamefont {Goyal},
  \citenamefont {Ishai}, \citenamefont {Sahai}, \citenamefont {Venkatesan},\
  and\ \citenamefont {Wadia}}]{GISVW10}%
  \BibitemOpen
  \bibfield  {author} {\bibinfo {author} {\bibnamefont {Goyal}, \bibfnamefont
  {Vipul}}, \bibinfo {author} {\bibfnamefont {Yuval}\ \bibnamefont {Ishai}},
  \bibinfo {author} {\bibfnamefont {Amit}\ \bibnamefont {Sahai}}, \bibinfo
  {author} {\bibfnamefont {Ramarathnam}\ \bibnamefont {Venkatesan}}, \ and\
  \bibinfo {author} {\bibfnamefont {Akshay}\ \bibnamefont {Wadia}}} (\bibinfo
  {year} {2010}),\ \bibfield  {title} {\enquote {\bibinfo {title} {Founding
  cryptography on tamper-proof hardware tokens},}\ }in\ \href {\doibase
  10.1007/978-3-642-11799-2_19} {\emph {\bibinfo {booktitle} {Theory of
  Cryptography, Proceedings of TCC 2010}}},\ \bibinfo {series} {LNCS}, Vol.\
  \bibinfo {volume} {5978}\ (\bibinfo  {publisher} {Springer})\ pp.\ \bibinfo
  {pages} {308--326},\ \bibinfo {note} {e-Print
  \href{http://eprint.iacr.org/2010/153}{IACR 2010/153}}\BibitemShut {NoStop}%
\bibitem [{\citenamefont {Gutoski}(2012)}]{Gut12}%
  \BibitemOpen
  \bibfield  {author} {\bibinfo {author} {\bibnamefont {Gutoski}, \bibfnamefont
  {Gus}}} (\bibinfo {year} {2012}),\ \bibfield  {title} {\enquote {\bibinfo
  {title} {On a measure of distance for quantum strategies},}\ }\href {\doibase
  10.1063/1.3693621} {\bibfield  {journal} {\bibinfo  {journal} {J. Math.
  Phys.}\ }\textbf {\bibinfo {volume} {53}}~(\bibinfo {number} {3}),\ \bibinfo
  {pages} {032202}},\ \Eprint {http://arxiv.org/abs/arXiv:1008.4636}
  {arXiv:1008.4636} \BibitemShut {NoStop}%
\bibitem [{\citenamefont {Gutoski}\ and\ \citenamefont {Watrous}(2007)}]{GW07}%
  \BibitemOpen
  \bibfield  {author} {\bibinfo {author} {\bibnamefont {Gutoski}, \bibfnamefont
  {Gus}}, \ and\ \bibinfo {author} {\bibfnamefont {John}\ \bibnamefont
  {Watrous}}} (\bibinfo {year} {2007}),\ \bibfield  {title} {\enquote {\bibinfo
  {title} {Toward a general theory of quantum games},}\ }in\ \href {\doibase
  10.1145/1250790.1250873} {\emph {\bibinfo {booktitle} {Proceedings of the
  39th Symposium on Theory of Computing, STOC~'07}}}\ (\bibinfo  {publisher}
  {ACM})\ pp.\ \bibinfo {pages} {565--574},\ \Eprint
  {http://arxiv.org/abs/arXiv:quant-ph/0611234} {arXiv:quant-ph/0611234}
  \BibitemShut {NoStop}%
\bibitem [{\citenamefont {Hardy}(2005)}]{Har05}%
  \BibitemOpen
  \bibfield  {author} {\bibinfo {author} {\bibnamefont {Hardy}, \bibfnamefont
  {Lucien}}} (\bibinfo {year} {2005}),\ \href@noop {} {\enquote {\bibinfo
  {title} {Probability theories with dynamic causal structure: A new framework
  for quantum gravity},}\ }\bibinfo {howpublished} {e-Print},\ \Eprint
  {http://arxiv.org/abs/arXiv:gr-qc/0509120} {arXiv:gr-qc/0509120} \BibitemShut
  {NoStop}%
\bibitem [{\citenamefont {Hardy}(2007)}]{Har07}%
  \BibitemOpen
  \bibfield  {author} {\bibinfo {author} {\bibnamefont {Hardy}, \bibfnamefont
  {Lucien}}} (\bibinfo {year} {2007}),\ \bibfield  {title} {\enquote {\bibinfo
  {title} {Towards quantum gravity: a framework for probabilistic theories with
  non-fixed causal structure},}\ }\href {\doibase 10.1088/1751-8113/40/12/S12}
  {\bibfield  {journal} {\bibinfo  {journal} {J. Phys. A}\ }\textbf {\bibinfo
  {volume} {40}}~(\bibinfo {number} {12}),\ \bibinfo {pages} {3081}},\ \Eprint
  {http://arxiv.org/abs/arXiv:gr-qc/0608043} {arXiv:gr-qc/0608043} \BibitemShut
  {NoStop}%
\bibitem [{\citenamefont {Hardy}(2011)}]{Har11}%
  \BibitemOpen
  \bibfield  {author} {\bibinfo {author} {\bibnamefont {Hardy}, \bibfnamefont
  {Lucien}}} (\bibinfo {year} {2011}),\ \href@noop {} {\enquote {\bibinfo
  {title} {Reformulating and reconstructing quantum theory},}\ }\bibinfo
  {howpublished} {e-print},\ \Eprint {http://arxiv.org/abs/arXiv:1104.2066}
  {arXiv:1104.2066} \BibitemShut {NoStop}%
\bibitem [{\citenamefont {Hardy}(2012)}]{Har12}%
  \BibitemOpen
  \bibfield  {author} {\bibinfo {author} {\bibnamefont {Hardy}, \bibfnamefont
  {Lucien}}} (\bibinfo {year} {2012}),\ \bibfield  {title} {\enquote {\bibinfo
  {title} {The operator tensor formulation of quantum theory},}\ }\href
  {\doibase 10.1098/rsta.2011.0326} {\bibfield  {journal} {\bibinfo  {journal}
  {Philos. Trans. R. Soc. London, Ser. A}\ }\textbf {\bibinfo {volume}
  {370}}~(\bibinfo {number} {1971}),\ \bibinfo {pages} {3385--3417}},\ \Eprint
  {http://arxiv.org/abs/arXiv:1201.4390} {arXiv:1201.4390} \BibitemShut
  {NoStop}%
\bibitem [{\citenamefont {Hardy}(2015)}]{Har15}%
  \BibitemOpen
  \bibfield  {author} {\bibinfo {author} {\bibnamefont {Hardy}, \bibfnamefont
  {Lucien}}} (\bibinfo {year} {2015}),\ \bibfield  {title} {\enquote {\bibinfo
  {title} {Quantum theory with bold operator tensors},}\ }\href {\doibase
  10.1098/rsta.2014.0239} {\bibfield  {journal} {\bibinfo  {journal} {Philos.
  Trans. R. Soc. London, Ser. A}\ }\textbf {\bibinfo {volume} {373}}~(\bibinfo
  {number} {2047}),\ 10.1098/rsta.2014.0239}\BibitemShut {NoStop}%
\bibitem [{\citenamefont {Hayashi}\ and\ \citenamefont
  {Tsurumaru}(2012)}]{HT12}%
  \BibitemOpen
  \bibfield  {author} {\bibinfo {author} {\bibnamefont {Hayashi}, \bibfnamefont
  {Masahito}}, \ and\ \bibinfo {author} {\bibfnamefont {Toyohiro}\ \bibnamefont
  {Tsurumaru}}} (\bibinfo {year} {2012}),\ \bibfield  {title} {\enquote
  {\bibinfo {title} {Concise and tight security analysis of the
  {Bennett}{\textendash}{Brassard} 1984 protocol with finite key lengths},}\
  }\href {\doibase 10.1088/1367-2630/14/9/093014} {\bibfield  {journal}
  {\bibinfo  {journal} {New J. Phys.}\ }\textbf {\bibinfo {volume}
  {14}}~(\bibinfo {number} {9}),\ \bibinfo {pages} {093014}},\ \Eprint
  {http://arxiv.org/abs/arXiv:1107.0589} {arXiv:1107.0589} \BibitemShut
  {NoStop}%
\bibitem [{\citenamefont {Hayden}\ \emph {et~al.}(2011)\citenamefont {Hayden},
  \citenamefont {Leung},\ and\ \citenamefont {Mayers}}]{HLM11}%
  \BibitemOpen
  \bibfield  {author} {\bibinfo {author} {\bibnamefont {Hayden}, \bibfnamefont
  {Patrick}}, \bibinfo {author} {\bibfnamefont {Debbie}\ \bibnamefont {Leung}},
  \ and\ \bibinfo {author} {\bibfnamefont {Dominic}\ \bibnamefont {Mayers}}}
  (\bibinfo {year} {2011}),\ \href@noop {} {\enquote {\bibinfo {title} {The
  universal composable security of quantum message authentication with key
  recycling},}\ }\bibinfo {howpublished} {presented at QCrypt 2011, e-Print},\
  \Eprint {http://arxiv.org/abs/arXiv:1610.09434} {arXiv:1610.09434}
  \BibitemShut {NoStop}%
\bibitem [{\citenamefont {Hayden}\ \emph {et~al.}(2004)\citenamefont {Hayden},
  \citenamefont {Leung}, \citenamefont {Shor},\ and\ \citenamefont
  {Winter}}]{HLSW04}%
  \BibitemOpen
  \bibfield  {author} {\bibinfo {author} {\bibnamefont {Hayden}, \bibfnamefont
  {Patrick}}, \bibinfo {author} {\bibfnamefont {Debbie}\ \bibnamefont {Leung}},
  \bibinfo {author} {\bibfnamefont {Peter~W.}\ \bibnamefont {Shor}}, \ and\
  \bibinfo {author} {\bibfnamefont {Andreas}\ \bibnamefont {Winter}}} (\bibinfo
  {year} {2004}),\ \bibfield  {title} {\enquote {\bibinfo {title} {Randomizing
  quantum states: Constructions and applications},}\ }\href {\doibase
  10.1007/s00220-004-1087-6} {\bibfield  {journal} {\bibinfo  {journal}
  {Commun. Math. Phys.}\ }\textbf {\bibinfo {volume} {250}},\ \bibinfo {pages}
  {371--391}},\ \Eprint {http://arxiv.org/abs/arXiv:quant-ph/0307104v3}
  {arXiv:quant-ph/0307104v3} \BibitemShut {NoStop}%
\bibitem [{\citenamefont {Helstrom}(1976)}]{Hel76}%
  \BibitemOpen
  \bibfield  {author} {\bibinfo {author} {\bibnamefont {Helstrom},
  \bibfnamefont {Carl~W}}} (\bibinfo {year} {1976}),\ \href@noop {} {\emph
  {\bibinfo {title} {Quantum Detection and Estimation Theory}}},\ \bibinfo
  {series} {Mathematics in science and engineering}, Vol.\ \bibinfo {volume}
  {123}\ (\bibinfo  {publisher} {Academic Press})\BibitemShut {NoStop}%
\bibitem [{\citenamefont {Hensen}\ \emph {et~al.}(2015)\citenamefont {Hensen},
  \citenamefont {Bernien}, \citenamefont {Dr{\'e}au}, \citenamefont {Reiserer},
  \citenamefont {Kalb}, \citenamefont {Blok}, \citenamefont {Ruitenberg},
  \citenamefont {Vermeulen}, \citenamefont {Schouten}, \citenamefont
  {Abell{\'a}n}, \citenamefont {Amaya}, \citenamefont {Pruneri}, \citenamefont
  {Mitchell}, \citenamefont {Markham}, \citenamefont {Twitchen}, \citenamefont
  {Elkouss}, \citenamefont {Wehner}, \citenamefont {Taminiau},\ and\
  \citenamefont {Hanson}}]{Hensen}%
  \BibitemOpen
  \bibfield  {author} {\bibinfo {author} {\bibnamefont {Hensen}, \bibfnamefont
  {B}}, \bibinfo {author} {\bibfnamefont {H.}~\bibnamefont {Bernien}}, \bibinfo
  {author} {\bibfnamefont {A.~E.}\ \bibnamefont {Dr{\'e}au}}, \bibinfo {author}
  {\bibfnamefont {A.}~\bibnamefont {Reiserer}}, \bibinfo {author}
  {\bibfnamefont {N.}~\bibnamefont {Kalb}}, \bibinfo {author} {\bibfnamefont
  {M.~S.}\ \bibnamefont {Blok}}, \bibinfo {author} {\bibfnamefont
  {J.}~\bibnamefont {Ruitenberg}}, \bibinfo {author} {\bibfnamefont {R.~F.~L.}\
  \bibnamefont {Vermeulen}}, \bibinfo {author} {\bibfnamefont {R.~N.}\
  \bibnamefont {Schouten}}, \bibinfo {author} {\bibfnamefont {C.}~\bibnamefont
  {Abell{\'a}n}}, \bibinfo {author} {\bibfnamefont {W.}~\bibnamefont {Amaya}},
  \bibinfo {author} {\bibfnamefont {V.}~\bibnamefont {Pruneri}}, \bibinfo
  {author} {\bibfnamefont {M.~W.}\ \bibnamefont {Mitchell}}, \bibinfo {author}
  {\bibfnamefont {M.}~\bibnamefont {Markham}}, \bibinfo {author} {\bibfnamefont
  {D.~J.}\ \bibnamefont {Twitchen}}, \bibinfo {author} {\bibfnamefont
  {D.}~\bibnamefont {Elkouss}}, \bibinfo {author} {\bibfnamefont
  {S.}~\bibnamefont {Wehner}}, \bibinfo {author} {\bibfnamefont {T.~H.}\
  \bibnamefont {Taminiau}}, \ and\ \bibinfo {author} {\bibfnamefont
  {R.}~\bibnamefont {Hanson}}} (\bibinfo {year} {2015}),\ \bibfield  {title}
  {\enquote {\bibinfo {title} {Loophole-free bell inequality violation using
  electron spins separated by 1.3 kilometres},}\ }\href {\doibase
  10.1038/nature15759} {\bibfield  {journal} {\bibinfo  {journal} {Nature}\
  }\textbf {\bibinfo {volume} {526}}~(\bibinfo {number} {7575}),\ \bibinfo
  {pages} {682--686}}\BibitemShut {NoStop}%
\bibitem [{\citenamefont {Hofheinz}\ \emph {et~al.}(2006)\citenamefont
  {Hofheinz}, \citenamefont {M\"{u}ller-Quade},\ and\ \citenamefont
  {Unruh}}]{HMU06}%
  \BibitemOpen
  \bibfield  {author} {\bibinfo {author} {\bibnamefont {Hofheinz},
  \bibfnamefont {Dennis}}, \bibinfo {author} {\bibfnamefont {J\"{o}rn}\
  \bibnamefont {M\"{u}ller-Quade}}, \ and\ \bibinfo {author} {\bibfnamefont
  {Dominique}\ \bibnamefont {Unruh}}} (\bibinfo {year} {2006}),\ \bibfield
  {title} {\enquote {\bibinfo {title} {On the (im)possibility of extending coin
  toss},}\ }in\ \href@noop {} {\emph {\bibinfo {booktitle} {Advances in
  Cryptology -- EUROCRYPT 2006}}},\ \bibinfo {series} {LNCS}, Vol.\ \bibinfo
  {volume} {4004}\ (\bibinfo  {publisher} {Springer})\ pp.\ \bibinfo {pages}
  {504--521},\ \bibinfo {note} {e-Print
  \href{http://eprint.iacr.org/2006/177}{IACR 2006/177}}\BibitemShut {NoStop}%
\bibitem [{\citenamefont {Hofheinz}\ and\ \citenamefont {Shoup}(2013)}]{HS13}%
  \BibitemOpen
  \bibfield  {author} {\bibinfo {author} {\bibnamefont {Hofheinz},
  \bibfnamefont {Dennis}}, \ and\ \bibinfo {author} {\bibfnamefont {Victor}\
  \bibnamefont {Shoup}}} (\bibinfo {year} {2013}),\ \bibfield  {title}
  {\enquote {\bibinfo {title} {{GNUC}: A new universal composability
  framework},}\ }\href {\doibase 10.1007/s00145-013-9160-y} {\bibfield
  {journal} {\bibinfo  {journal} {J. Crypt.}\ ,\ \bibinfo {pages}
  {1--86}}}\bibinfo {note} {E-Print \href{http://eprint.iacr.org/2011/303}{IACR
  2011/303}}\BibitemShut {NoStop}%
\bibitem [{\citenamefont {Horodecki}\ \emph {et~al.}(2008)\citenamefont
  {Horodecki}, \citenamefont {Horodecki}, \citenamefont {Horodecki},
  \citenamefont {Leung},\ and\ \citenamefont {Oppenheim}}]{HHHLO08}%
  \BibitemOpen
  \bibfield  {author} {\bibinfo {author} {\bibnamefont {Horodecki},
  \bibfnamefont {Karol}}, \bibinfo {author} {\bibfnamefont {Micha\l{}}\
  \bibnamefont {Horodecki}}, \bibinfo {author} {\bibfnamefont {Pave\l{}}\
  \bibnamefont {Horodecki}}, \bibinfo {author} {\bibfnamefont {Debbie}\
  \bibnamefont {Leung}}, \ and\ \bibinfo {author} {\bibfnamefont {Jonathan}\
  \bibnamefont {Oppenheim}}} (\bibinfo {year} {2008}),\ \bibfield  {title}
  {\enquote {\bibinfo {title} {Quantum key distribution based on private
  states: Unconditional security over untrusted channels with zero quantum
  capacity},}\ }\href {\doibase 10.1109/TIT.2008.921870} {\bibfield  {journal}
  {\bibinfo  {journal} {IEEE Trans. Inf. Theory}\ }\textbf {\bibinfo {volume}
  {54}}~(\bibinfo {number} {6}),\ \bibinfo {pages} {2604--2620}}\BibitemShut
  {NoStop}%
\bibitem [{\citenamefont {Horodecki}\ and\ \citenamefont
  {Stankiewicz}(2020)}]{HS20}%
  \BibitemOpen
  \bibfield  {author} {\bibinfo {author} {\bibnamefont {Horodecki},
  \bibfnamefont {Karol}}, \ and\ \bibinfo {author} {\bibfnamefont {Maciej}\
  \bibnamefont {Stankiewicz}}} (\bibinfo {year} {2020}),\ \bibfield  {title}
  {\enquote {\bibinfo {title} {Semi-device-independent quantum money},}\ }\href
  {\doibase 10.1088/1367-2630/ab6872} {\bibfield  {journal} {\bibinfo
  {journal} {New J. Phys.}\ }\textbf {\bibinfo {volume} {22}}~(\bibinfo
  {number} {2}),\ \bibinfo {pages} {023007}},\ \Eprint
  {http://arxiv.org/abs/arxiv:1811.10552} {arxiv:1811.10552} \BibitemShut
  {NoStop}%
\bibitem [{\citenamefont {Horodecki}\ \emph {et~al.}(1998)\citenamefont
  {Horodecki}, \citenamefont {Horodecki},\ and\ \citenamefont
  {Horodecki}}]{HHH98}%
  \BibitemOpen
  \bibfield  {author} {\bibinfo {author} {\bibnamefont {Horodecki},
  \bibfnamefont {Micha\l{}}}, \bibinfo {author} {\bibfnamefont {Pawe\l{}}\
  \bibnamefont {Horodecki}}, \ and\ \bibinfo {author} {\bibfnamefont {Ryszard}\
  \bibnamefont {Horodecki}}} (\bibinfo {year} {1998}),\ \bibfield  {title}
  {\enquote {\bibinfo {title} {Mixed-state entanglement and distillation: Is
  there a ``bound'' entanglement in nature?}}\ }\href {\doibase
  10.1103/PhysRevLett.80.5239} {\bibfield  {journal} {\bibinfo  {journal}
  {Phys. Rev. Lett.}\ }\textbf {\bibinfo {volume} {80}},\ \bibinfo {pages}
  {5239--5242}}\BibitemShut {NoStop}%
\bibitem [{\citenamefont {Hwang}(2003)}]{Hwang2003}%
  \BibitemOpen
  \bibfield  {author} {\bibinfo {author} {\bibnamefont {Hwang}, \bibfnamefont
  {Won-Young}}} (\bibinfo {year} {2003}),\ \bibfield  {title} {\enquote
  {\bibinfo {title} {Quantum key distribution with high loss: Toward global
  secure communication},}\ }\href {\doibase 10.1103/PhysRevLett.91.057901}
  {\bibfield  {journal} {\bibinfo  {journal} {Phys. Rev. Lett.}\ }\textbf
  {\bibinfo {volume} {91}},\ \bibinfo {pages} {057901}}\BibitemShut {NoStop}%
\bibitem [{\citenamefont {Inamori}\ \emph {et~al.}(2007)\citenamefont
  {Inamori}, \citenamefont {L{\"u}tkenhaus},\ and\ \citenamefont
  {Mayers}}]{ILM07}%
  \BibitemOpen
  \bibfield  {author} {\bibinfo {author} {\bibnamefont {Inamori}, \bibfnamefont
  {Hitoshi}}, \bibinfo {author} {\bibfnamefont {Norbert}\ \bibnamefont
  {L{\"u}tkenhaus}}, \ and\ \bibinfo {author} {\bibfnamefont {Dominic}\
  \bibnamefont {Mayers}}} (\bibinfo {year} {2007}),\ \bibfield  {title}
  {\enquote {\bibinfo {title} {Unconditional security of practical quantum key
  distribution},}\ }\href {\doibase 10.1140/epjd/e2007-00010-4} {\bibfield
  {journal} {\bibinfo  {journal} {Eur. Phys. J. D}\ }\textbf {\bibinfo {volume}
  {41}}~(\bibinfo {number} {3}),\ \bibinfo {pages} {599--627}},\ \Eprint
  {http://arxiv.org/abs/arXiv:quant-ph/0107017} {arXiv:quant-ph/0107017}
  \BibitemShut {NoStop}%
\bibitem [{\citenamefont {Inoue}\ \emph {et~al.}(2002)\citenamefont {Inoue},
  \citenamefont {Waks},\ and\ \citenamefont {Yamamoto}}]{IWY02}%
  \BibitemOpen
  \bibfield  {author} {\bibinfo {author} {\bibnamefont {Inoue}, \bibfnamefont
  {Kyo}}, \bibinfo {author} {\bibfnamefont {Edo}\ \bibnamefont {Waks}}, \ and\
  \bibinfo {author} {\bibfnamefont {Yoshihisa}\ \bibnamefont {Yamamoto}}}
  (\bibinfo {year} {2002}),\ \bibfield  {title} {\enquote {\bibinfo {title}
  {Differential phase shift quantum key distribution},}\ }\href {\doibase
  10.1103/PhysRevLett.89.037902} {\bibfield  {journal} {\bibinfo  {journal}
  {Phys. Rev. Lett.}\ }\textbf {\bibinfo {volume} {89}},\ \bibinfo {pages}
  {037902}}\BibitemShut {NoStop}%
\bibitem [{\citenamefont {Ishai}\ \emph {et~al.}(2014)\citenamefont {Ishai},
  \citenamefont {Ostrovsky},\ and\ \citenamefont {Zikas}}]{IOZ14}%
  \BibitemOpen
  \bibfield  {author} {\bibinfo {author} {\bibnamefont {Ishai}, \bibfnamefont
  {Yuval}}, \bibinfo {author} {\bibfnamefont {Rafail}\ \bibnamefont
  {Ostrovsky}}, \ and\ \bibinfo {author} {\bibfnamefont {Vassilis}\
  \bibnamefont {Zikas}}} (\bibinfo {year} {2014}),\ \bibfield  {title}
  {\enquote {\bibinfo {title} {Secure multi-party computation with identifiable
  abort},}\ }in\ \href {\doibase 10.1007/978-3-662-44381-1_21} {\emph {\bibinfo
  {booktitle} {Advances in Cryptology -- CRYPTO 2014}}},\ \bibinfo {editor}
  {edited by\ \bibinfo {editor} {\bibfnamefont {Juan~A.}\ \bibnamefont
  {Garay}}\ and\ \bibinfo {editor} {\bibfnamefont {Rosario}\ \bibnamefont
  {Gennaro}}}\ (\bibinfo  {publisher} {Springer})\ pp.\ \bibinfo {pages}
  {369--386},\ \bibinfo {note} {e-Print
  \href{http://eprint.iacr.org/2015/325}{IACR 2015/325}}\BibitemShut {NoStop}%
\bibitem [{\citenamefont {Ishai}\ \emph {et~al.}(2008)\citenamefont {Ishai},
  \citenamefont {Prabhakaran},\ and\ \citenamefont {Sahai}}]{IPS08}%
  \BibitemOpen
  \bibfield  {author} {\bibinfo {author} {\bibnamefont {Ishai}, \bibfnamefont
  {Yuval}}, \bibinfo {author} {\bibfnamefont {Manoj}\ \bibnamefont
  {Prabhakaran}}, \ and\ \bibinfo {author} {\bibfnamefont {Amit}\ \bibnamefont
  {Sahai}}} (\bibinfo {year} {2008}),\ \bibfield  {title} {\enquote {\bibinfo
  {title} {Founding cryptography on oblivious transfer - efficiently},}\ }in\
  \href {\doibase 10.1007/978-3-540-85174-5_32} {\emph {\bibinfo {booktitle}
  {Advances in Cryptology -- CRYPTO 2008}}},\ \bibinfo {series} {LNCS}, Vol.\
  \bibinfo {volume} {5157}\ (\bibinfo  {publisher} {Springer})\ pp.\ \bibinfo
  {pages} {572--591}\BibitemShut {NoStop}%
\bibitem [{\citenamefont {Jost}\ and\ \citenamefont {Maurer}(2018)}]{JM18}%
  \BibitemOpen
  \bibfield  {author} {\bibinfo {author} {\bibnamefont {Jost}, \bibfnamefont
  {Daniel}}, \ and\ \bibinfo {author} {\bibfnamefont {Ueli}\ \bibnamefont
  {Maurer}}} (\bibinfo {year} {2018}),\ \bibfield  {title} {\enquote {\bibinfo
  {title} {Security definitions for hash functions: Combining {UCE} and
  {I}ndifferentiability},}\ }in\ \href@noop {} {\emph {\bibinfo {booktitle}
  {International Conference on Security and Cryptography for Networks -- SCN
  2018}}},\ \bibinfo {series} {LNCS}, Vol.\ \bibinfo {volume} {11035},\
  \bibinfo {editor} {edited by\ \bibinfo {editor} {\bibfnamefont {Dario}\
  \bibnamefont {Catalano}}\ and\ \bibinfo {editor} {\bibfnamefont {Roberto}\
  \bibnamefont {De~Prisco}}}\ (\bibinfo  {publisher} {Springer})\ pp.\ \bibinfo
  {pages} {83--101},\ \bibinfo {note} {e-Print
  \href{http://eprint.iacr.org/2006/281}{IACR 2006/281}}\BibitemShut {NoStop}%
\bibitem [{\citenamefont {Jouguet}\ and\ \citenamefont
  {Kunz-Jacques}(2014)}]{JK14}%
  \BibitemOpen
  \bibfield  {author} {\bibinfo {author} {\bibnamefont {Jouguet}, \bibfnamefont
  {Paul}}, \ and\ \bibinfo {author} {\bibfnamefont {Sebastien}\ \bibnamefont
  {Kunz-Jacques}}} (\bibinfo {year} {2014}),\ \bibfield  {title} {\enquote
  {\bibinfo {title} {High performance error correction for quantum key
  distribution using polar codes},}\ }\href@noop {} {\bibfield  {journal}
  {\bibinfo  {journal} {Quantum Inf. Comput.}\ }\textbf {\bibinfo {volume}
  {14}}~(\bibinfo {number} {3-4}),\ \bibinfo {pages} {329--338}}\BibitemShut
  {NoStop}%
\bibitem [{\citenamefont {Kaniewski}(2015)}]{Kan15}%
  \BibitemOpen
  \bibfield  {author} {\bibinfo {author} {\bibnamefont {Kaniewski},
  \bibfnamefont {J\c{e}drzej}}} (\bibinfo {year} {2015}),\ \emph {\bibinfo
  {title} {Relativistic quantum cryptography}},\ \href@noop {} {Ph.D. thesis}\
  (\bibinfo  {school} {National University of Singapore}),\ \Eprint
  {http://arxiv.org/abs/arXiv:1512.00602} {arXiv:1512.00602} \BibitemShut
  {NoStop}%
\bibitem [{\citenamefont {Kaniewski}\ \emph {et~al.}(2013)\citenamefont
  {Kaniewski}, \citenamefont {Tomamichel}, \citenamefont {H\"anggi},\ and\
  \citenamefont {Wehner}}]{KTHW13}%
  \BibitemOpen
  \bibfield  {author} {\bibinfo {author} {\bibnamefont {Kaniewski},
  \bibfnamefont {J\c{e}drzej}}, \bibinfo {author} {\bibfnamefont {Marco}\
  \bibnamefont {Tomamichel}}, \bibinfo {author} {\bibfnamefont {Esther}\
  \bibnamefont {H\"anggi}}, \ and\ \bibinfo {author} {\bibfnamefont
  {Stephanie}\ \bibnamefont {Wehner}}} (\bibinfo {year} {2013}),\ \bibfield
  {title} {\enquote {\bibinfo {title} {Secure bit commitment from relativistic
  constraints},}\ }\href {\doibase 10.1109/TIT.2013.2247463} {\bibfield
  {journal} {\bibinfo  {journal} {IEEE Transactions on Information Theory}\
  }\textbf {\bibinfo {volume} {59}}~(\bibinfo {number} {7}),\ \bibinfo {pages}
  {4687--4699}},\ \Eprint {http://arxiv.org/abs/arXiv:1206.1740}
  {arXiv:1206.1740} \BibitemShut {NoStop}%
\bibitem [{\citenamefont {Katz}\ and\ \citenamefont {Yung}(2006)}]{KY06}%
  \BibitemOpen
  \bibfield  {author} {\bibinfo {author} {\bibnamefont {Katz}, \bibfnamefont
  {Jonathan}}, \ and\ \bibinfo {author} {\bibfnamefont {Moti}\ \bibnamefont
  {Yung}}} (\bibinfo {year} {2006}),\ \bibfield  {title} {\enquote {\bibinfo
  {title} {Characterization of security notions for probabilistic private-key
  encryption},}\ }\href {\doibase 10.1007/s00145-005-0310-8} {\bibfield
  {journal} {\bibinfo  {journal} {J. Crypt.}\ }\textbf {\bibinfo {volume}
  {19}}~(\bibinfo {number} {1}),\ \bibinfo {pages} {67--95}}\BibitemShut
  {NoStop}%
\bibitem [{\citenamefont {Kent}(1999)}]{Ken99}%
  \BibitemOpen
  \bibfield  {author} {\bibinfo {author} {\bibnamefont {Kent}, \bibfnamefont
  {Adrian}}} (\bibinfo {year} {1999}),\ \bibfield  {title} {\enquote {\bibinfo
  {title} {Unconditionally secure bit commitment},}\ }\href {\doibase
  10.1103/PhysRevLett.83.1447} {\bibfield  {journal} {\bibinfo  {journal}
  {Phys. Rev. Lett.}\ }\textbf {\bibinfo {volume} {83}},\ \bibinfo {pages}
  {1447--1450}},\ \Eprint {http://arxiv.org/abs/arXiv:quant-ph/9810068}
  {arXiv:quant-ph/9810068} \BibitemShut {NoStop}%
\bibitem [{\citenamefont {Kent}(2012)}]{Ken12}%
  \BibitemOpen
  \bibfield  {author} {\bibinfo {author} {\bibnamefont {Kent}, \bibfnamefont
  {Adrian}}} (\bibinfo {year} {2012}),\ \bibfield  {title} {\enquote {\bibinfo
  {title} {Unconditionally secure bit commitment by transmitting measurement
  outcomes},}\ }\href {\doibase 10.1103/PhysRevLett.109.130501} {\bibfield
  {journal} {\bibinfo  {journal} {Phys. Rev. Lett.}\ }\textbf {\bibinfo
  {volume} {109}},\ \bibinfo {pages} {130501}},\ \Eprint
  {http://arxiv.org/abs/arXiv:1108.2879} {arXiv:1108.2879} \BibitemShut
  {NoStop}%
\bibitem [{\citenamefont {Kessler}\ and\ \citenamefont
  {Arnon-Friedman}(2020)}]{KAF20}%
  \BibitemOpen
  \bibfield  {author} {\bibinfo {author} {\bibnamefont {Kessler}, \bibfnamefont
  {Max}}, \ and\ \bibinfo {author} {\bibfnamefont {Rotem}\ \bibnamefont
  {Arnon-Friedman}}} (\bibinfo {year} {2020}),\ \bibfield  {title} {\enquote
  {\bibinfo {title} {Device-independent randomness amplification and
  privatization},}\ }\href {\doibase 10.1109/JSAIT.2020.3012498} {\bibfield
  {journal} {\bibinfo  {journal} {IEEE J. Sel. Areas Inf. Theory}\ }\textbf
  {\bibinfo {volume} {1}}~(\bibinfo {number} {2}),\ \bibinfo {pages}
  {568--584}},\ \Eprint {http://arxiv.org/abs/arXiv:1705.04148}
  {arXiv:1705.04148} \BibitemShut {NoStop}%
\bibitem [{\citenamefont {Koashi}(2004)}]{Koashi04}%
  \BibitemOpen
  \bibfield  {author} {\bibinfo {author} {\bibnamefont {Koashi}, \bibfnamefont
  {Masato}}} (\bibinfo {year} {2004}),\ \bibfield  {title} {\enquote {\bibinfo
  {title} {Unconditional security of coherent-state quantum key distribution
  with a strong phase-reference pulse},}\ }\href {\doibase
  10.1103/PhysRevLett.93.120501} {\bibfield  {journal} {\bibinfo  {journal}
  {Phys. Rev. Lett.}\ }\textbf {\bibinfo {volume} {93}},\ \bibinfo {pages}
  {120501}}\BibitemShut {NoStop}%
\bibitem [{\citenamefont {Koashi}(2009)}]{Koa09}%
  \BibitemOpen
  \bibfield  {author} {\bibinfo {author} {\bibnamefont {Koashi}, \bibfnamefont
  {Masato}}} (\bibinfo {year} {2009}),\ \bibfield  {title} {\enquote {\bibinfo
  {title} {Simple security proof of quantum key distribution based on
  complementarity},}\ }\href {\doibase 10.1088/1367-2630/11/4/045018}
  {\bibfield  {journal} {\bibinfo  {journal} {New J. Phys.}\ }\textbf {\bibinfo
  {volume} {11}}~(\bibinfo {number} {4}),\ \bibinfo {pages}
  {045018}}\BibitemShut {NoStop}%
\bibitem [{\citenamefont {Koashi}\ and\ \citenamefont
  {Winter}(2004)}]{KoashiWinter04}%
  \BibitemOpen
  \bibfield  {author} {\bibinfo {author} {\bibnamefont {Koashi}, \bibfnamefont
  {Masato}}, \ and\ \bibinfo {author} {\bibfnamefont {Andreas}\ \bibnamefont
  {Winter}}} (\bibinfo {year} {2004}),\ \bibfield  {title} {\enquote {\bibinfo
  {title} {Monogamy of quantum entanglement and other correlations},}\ }\href
  {\doibase 10.1103/PhysRevA.69.022309} {\bibfield  {journal} {\bibinfo
  {journal} {Phys. Rev. A}\ }\textbf {\bibinfo {volume} {69}},\ \bibinfo
  {pages} {022309}}\BibitemShut {NoStop}%
\bibitem [{\citenamefont {Kochen}\ and\ \citenamefont
  {Specker}(1967)}]{KocSpe67}%
  \BibitemOpen
  \bibfield  {author} {\bibinfo {author} {\bibnamefont {Kochen}, \bibfnamefont
  {Simon~B}}, \ and\ \bibinfo {author} {\bibfnamefont {Ernst~P.}\ \bibnamefont
  {Specker}}} (\bibinfo {year} {1967}),\ \bibfield  {title} {\enquote {\bibinfo
  {title} {The problem of hidden variables in quantum mechanics},}\ }\href@noop
  {} {\bibfield  {journal} {\bibinfo  {journal} {J. Math. Mech.}\ }\textbf
  {\bibinfo {volume} {17}},\ \bibinfo {pages} {59--87}}\BibitemShut {NoStop}%
\bibitem [{\citenamefont {K\"onig}\ \emph {et~al.}(2005)\citenamefont
  {K\"onig}, \citenamefont {Maurer},\ and\ \citenamefont {Renner}}]{KMR05}%
  \BibitemOpen
  \bibfield  {author} {\bibinfo {author} {\bibnamefont {K\"onig}, \bibfnamefont
  {Robert}}, \bibinfo {author} {\bibfnamefont {Ueli}\ \bibnamefont {Maurer}}, \
  and\ \bibinfo {author} {\bibfnamefont {Renato}\ \bibnamefont {Renner}}}
  (\bibinfo {year} {2005}),\ \bibfield  {title} {\enquote {\bibinfo {title} {On
  the power of quantum memory},}\ }\href {\doibase 10.1109/TIT.2005.850087}
  {\bibfield  {journal} {\bibinfo  {journal} {IEEE Trans. Inf. Theory}\
  }\textbf {\bibinfo {volume} {51}}~(\bibinfo {number} {7}),\ \bibinfo {pages}
  {2391--2401}},\ \Eprint {http://arxiv.org/abs/quant-ph/0305154}
  {quant-ph/0305154} \BibitemShut {NoStop}%
\bibitem [{\citenamefont {K\"onig}\ \emph {et~al.}(2007)\citenamefont
  {K\"onig}, \citenamefont {Renner}, \citenamefont {Bariska},\ and\
  \citenamefont {Maurer}}]{KRBM07}%
  \BibitemOpen
  \bibfield  {author} {\bibinfo {author} {\bibnamefont {K\"onig}, \bibfnamefont
  {Robert}}, \bibinfo {author} {\bibfnamefont {Renato}\ \bibnamefont {Renner}},
  \bibinfo {author} {\bibfnamefont {Andor}\ \bibnamefont {Bariska}}, \ and\
  \bibinfo {author} {\bibfnamefont {Ueli}\ \bibnamefont {Maurer}}} (\bibinfo
  {year} {2007}),\ \bibfield  {title} {\enquote {\bibinfo {title} {Small
  accessible quantum information does not imply security},}\ }\href {\doibase
  10.1103/PhysRevLett.98.140502} {\bibfield  {journal} {\bibinfo  {journal}
  {Phys. Rev. Lett.}\ }\textbf {\bibinfo {volume} {98}},\ \bibinfo {pages}
  {140502}},\ \Eprint {http://arxiv.org/abs/arXiv:quant-ph/0512021}
  {arXiv:quant-ph/0512021} \BibitemShut {NoStop}%
\bibitem [{\citenamefont {K\"onig}\ and\ \citenamefont {Terhal}(2008)}]{KT08}%
  \BibitemOpen
  \bibfield  {author} {\bibinfo {author} {\bibnamefont {K\"onig}, \bibfnamefont
  {Robert}}, \ and\ \bibinfo {author} {\bibfnamefont {Barbara~M.}\ \bibnamefont
  {Terhal}}} (\bibinfo {year} {2008}),\ \bibfield  {title} {\enquote {\bibinfo
  {title} {The bounded-storage model in the presence of a quantum adversary},}\
  }\href {\doibase 10.1109/TIT.2007.913245} {\bibfield  {journal} {\bibinfo
  {journal} {IEEE Trans. Inf. Theory}\ }\textbf {\bibinfo {volume}
  {54}}~(\bibinfo {number} {2}),\ \bibinfo {pages} {749--762}},\ \Eprint
  {http://arxiv.org/abs/arXiv:quant-ph/0608101} {arXiv:quant-ph/0608101}
  \BibitemShut {NoStop}%
\bibitem [{\citenamefont {K\"onig}\ \emph {et~al.}(2012)\citenamefont
  {K\"onig}, \citenamefont {Wehner},\ and\ \citenamefont
  {Wullschleger}}]{KWW12}%
  \BibitemOpen
  \bibfield  {author} {\bibinfo {author} {\bibnamefont {K\"onig}, \bibfnamefont
  {Robert}}, \bibinfo {author} {\bibfnamefont {Stephanie}\ \bibnamefont
  {Wehner}}, \ and\ \bibinfo {author} {\bibfnamefont {J\"urg}\ \bibnamefont
  {Wullschleger}}} (\bibinfo {year} {2012}),\ \bibfield  {title} {\enquote
  {\bibinfo {title} {Unconditional security from noisy quantum storage},}\
  }\href {\doibase 10.1109/TIT.2011.2177772} {\bibfield  {journal} {\bibinfo
  {journal} {IEEE Transactions on Information Theory}\ }\textbf {\bibinfo
  {volume} {58}}~(\bibinfo {number} {3}),\ \bibinfo {pages} {1962--1984}},\
  \Eprint {http://arxiv.org/abs/arXiv:0906.1030} {arXiv:0906.1030} \BibitemShut
  {NoStop}%
\bibitem [{\citenamefont {Kraus}\ \emph {et~al.}(2005)\citenamefont {Kraus},
  \citenamefont {Gisin},\ and\ \citenamefont {Renner}}]{PhysRevLett.95.080501}%
  \BibitemOpen
  \bibfield  {author} {\bibinfo {author} {\bibnamefont {Kraus}, \bibfnamefont
  {Barbara}}, \bibinfo {author} {\bibfnamefont {Nicolas}\ \bibnamefont
  {Gisin}}, \ and\ \bibinfo {author} {\bibfnamefont {Renner}\ \bibnamefont
  {Renner}}} (\bibinfo {year} {2005}),\ \bibfield  {title} {\enquote {\bibinfo
  {title} {Lower and upper bounds on the secret-key rate for quantum key
  distribution protocols using one-way classical communication},}\ }\href
  {\doibase 10.1103/PhysRevLett.95.080501} {\bibfield  {journal} {\bibinfo
  {journal} {Phys. Rev. Lett.}\ }\textbf {\bibinfo {volume} {95}},\ \bibinfo
  {pages} {080501}}\BibitemShut {NoStop}%
\bibitem [{\citenamefont {K{\"u}sters}(2006)}]{Kus06}%
  \BibitemOpen
  \bibfield  {author} {\bibinfo {author} {\bibnamefont {K{\"u}sters},
  \bibfnamefont {Ralf}}} (\bibinfo {year} {2006}),\ \bibfield  {title}
  {\enquote {\bibinfo {title} {Simulation-based security with inexhaustible
  interactive turing machines},}\ }in\ \href {\doibase 10.1109/CSFW.2006.30}
  {\emph {\bibinfo {booktitle} {Proceedings of the 19th IEEE workshop on
  Computer Security Foundations, CSFW~'06}}}\ (\bibinfo  {publisher} {IEEE})\
  pp.\ \bibinfo {pages} {309--320}\BibitemShut {NoStop}%
\bibitem [{\citenamefont {Laneve}\ and\ \citenamefont {del Rio}(2021)}]{LdR21}%
  \BibitemOpen
  \bibfield  {author} {\bibinfo {author} {\bibnamefont {Laneve}, \bibfnamefont
  {Lorenzo}}, \ and\ \bibinfo {author} {\bibfnamefont {L\'idia}\ \bibnamefont
  {del Rio}}} (\bibinfo {year} {2021}),\ \href@noop {} {\enquote {\bibinfo
  {title} {Impossibility of composable oblivious transfer in relativistic
  quantum cryptography},}\ }\bibinfo {howpublished} {e-Print},\ \Eprint
  {http://arxiv.org/abs/arXiv:2106.11200} {arXiv:2106.11200} \BibitemShut
  {NoStop}%
\bibitem [{\citenamefont {Leverrier}\ \emph {et~al.}(2008)\citenamefont
  {Leverrier}, \citenamefont {All\'eaume}, \citenamefont {Boutros},
  \citenamefont {Z\'emor},\ and\ \citenamefont {Grangier}}]{LABZG08}%
  \BibitemOpen
  \bibfield  {author} {\bibinfo {author} {\bibnamefont {Leverrier},
  \bibfnamefont {Anthony}}, \bibinfo {author} {\bibfnamefont {Romain}\
  \bibnamefont {All\'eaume}}, \bibinfo {author} {\bibfnamefont {Joseph}\
  \bibnamefont {Boutros}}, \bibinfo {author} {\bibfnamefont {Gilles}\
  \bibnamefont {Z\'emor}}, \ and\ \bibinfo {author} {\bibfnamefont {Philippe}\
  \bibnamefont {Grangier}}} (\bibinfo {year} {2008}),\ \bibfield  {title}
  {\enquote {\bibinfo {title} {Multidimensional reconciliation for a
  continuous-variable quantum key distribution},}\ }\href {\doibase
  10.1103/PhysRevA.77.042325} {\bibfield  {journal} {\bibinfo  {journal} {Phys.
  Rev. A}\ }\textbf {\bibinfo {volume} {77}},\ \bibinfo {pages}
  {042325}}\BibitemShut {NoStop}%
\bibitem [{\citenamefont {Lim}\ \emph {et~al.}(2014)\citenamefont {Lim},
  \citenamefont {Curty}, \citenamefont {Walenta}, \citenamefont {Xu},\ and\
  \citenamefont {Zbinden}}]{LCWXZ14}%
  \BibitemOpen
  \bibfield  {author} {\bibinfo {author} {\bibnamefont {Lim}, \bibfnamefont
  {Charles Ci~Wen}}, \bibinfo {author} {\bibfnamefont {Marcos}\ \bibnamefont
  {Curty}}, \bibinfo {author} {\bibfnamefont {Nino}\ \bibnamefont {Walenta}},
  \bibinfo {author} {\bibfnamefont {Feihu}\ \bibnamefont {Xu}}, \ and\ \bibinfo
  {author} {\bibfnamefont {Hugo}\ \bibnamefont {Zbinden}}} (\bibinfo {year}
  {2014}),\ \bibfield  {title} {\enquote {\bibinfo {title} {Concise security
  bounds for practical decoy-state quantum key distribution},}\ }\href
  {\doibase 10.1103/PhysRevA.89.022307} {\bibfield  {journal} {\bibinfo
  {journal} {Phys. Rev. A}\ }\textbf {\bibinfo {volume} {89}},\ \bibinfo
  {pages} {022307}}\BibitemShut {NoStop}%
\bibitem [{\citenamefont {Lim}\ \emph {et~al.}(2013)\citenamefont {Lim},
  \citenamefont {Portmann}, \citenamefont {Tomamichel}, \citenamefont
  {Renner},\ and\ \citenamefont {Gisin}}]{LPTRG13}%
  \BibitemOpen
  \bibfield  {author} {\bibinfo {author} {\bibnamefont {Lim}, \bibfnamefont
  {Charles Ci~Wen}}, \bibinfo {author} {\bibfnamefont {Christopher}\
  \bibnamefont {Portmann}}, \bibinfo {author} {\bibfnamefont {Marco}\
  \bibnamefont {Tomamichel}}, \bibinfo {author} {\bibfnamefont {Renato}\
  \bibnamefont {Renner}}, \ and\ \bibinfo {author} {\bibfnamefont {Nicolas}\
  \bibnamefont {Gisin}}} (\bibinfo {year} {2013}),\ \bibfield  {title}
  {\enquote {\bibinfo {title} {Device-independent quantum key distribution with
  local {Bell} test},}\ }\href {\doibase 10.1103/PhysRevX.3.031006} {\bibfield
  {journal} {\bibinfo  {journal} {Phys. Rev. X}\ }\textbf {\bibinfo {volume}
  {3}},\ \bibinfo {pages} {031006}},\ \Eprint
  {http://arxiv.org/abs/arXiv:1208.0023} {arXiv:1208.0023} \BibitemShut
  {NoStop}%
\bibitem [{\citenamefont {Lipinska}\ \emph {et~al.}(2020)\citenamefont
  {Lipinska}, \citenamefont {Ribeiro},\ and\ \citenamefont {Wehner}}]{LRW20}%
  \BibitemOpen
  \bibfield  {author} {\bibinfo {author} {\bibnamefont {Lipinska},
  \bibfnamefont {Victoria}}, \bibinfo {author} {\bibfnamefont {J\'er\'emy}\
  \bibnamefont {Ribeiro}}, \ and\ \bibinfo {author} {\bibfnamefont {Stephanie}\
  \bibnamefont {Wehner}}} (\bibinfo {year} {2020}),\ \bibfield  {title}
  {\enquote {\bibinfo {title} {Secure multiparty quantum computation with few
  qubits},}\ }\href {\doibase 10.1103/PhysRevA.102.022405} {\bibfield
  {journal} {\bibinfo  {journal} {Phys. Rev. A}\ }\textbf {\bibinfo {volume}
  {102}},\ \bibinfo {pages} {022405}},\ \Eprint
  {http://arxiv.org/abs/arXiv:2004.10486} {arXiv:2004.10486} \BibitemShut
  {NoStop}%
\bibitem [{\citenamefont {Liu}\ \emph {et~al.}(2013)\citenamefont {Liu},
  \citenamefont {Chen}, \citenamefont {Wang}, \citenamefont {Liang},
  \citenamefont {Shentu}, \citenamefont {Wang}, \citenamefont {Cui},
  \citenamefont {Yin}, \citenamefont {Liu}, \citenamefont {Li}, \citenamefont
  {Ma}, \citenamefont {Pelc}, \citenamefont {Fejer}, \citenamefont {Peng},
  \citenamefont {Zhang},\ and\ \citenamefont {Pan}}]{Liu13}%
  \BibitemOpen
  \bibfield  {author} {\bibinfo {author} {\bibnamefont {Liu}, \bibfnamefont
  {Yang}}, \bibinfo {author} {\bibfnamefont {Teng-Yun}\ \bibnamefont {Chen}},
  \bibinfo {author} {\bibfnamefont {Liu-Jun}\ \bibnamefont {Wang}}, \bibinfo
  {author} {\bibfnamefont {Hao}\ \bibnamefont {Liang}}, \bibinfo {author}
  {\bibfnamefont {Guo-Liang}\ \bibnamefont {Shentu}}, \bibinfo {author}
  {\bibfnamefont {Jian}\ \bibnamefont {Wang}}, \bibinfo {author} {\bibfnamefont
  {Ke}~\bibnamefont {Cui}}, \bibinfo {author} {\bibfnamefont {Hua-Lei}\
  \bibnamefont {Yin}}, \bibinfo {author} {\bibfnamefont {Nai-Le}\ \bibnamefont
  {Liu}}, \bibinfo {author} {\bibfnamefont {Li}~\bibnamefont {Li}}, \bibinfo
  {author} {\bibfnamefont {Xiongfeng}\ \bibnamefont {Ma}}, \bibinfo {author}
  {\bibfnamefont {Jason~S.}\ \bibnamefont {Pelc}}, \bibinfo {author}
  {\bibfnamefont {M.~M.}\ \bibnamefont {Fejer}}, \bibinfo {author}
  {\bibfnamefont {Cheng-Zhi}\ \bibnamefont {Peng}}, \bibinfo {author}
  {\bibfnamefont {Qiang}\ \bibnamefont {Zhang}}, \ and\ \bibinfo {author}
  {\bibfnamefont {Jian-Wei}\ \bibnamefont {Pan}}} (\bibinfo {year} {2013}),\
  \bibfield  {title} {\enquote {\bibinfo {title} {Experimental
  measurement-device-independent quantum key distribution},}\ }\href {\doibase
  10.1103/PhysRevLett.111.130502} {\bibfield  {journal} {\bibinfo  {journal}
  {Phys. Rev. Lett.}\ }\textbf {\bibinfo {volume} {111}},\ \bibinfo {pages}
  {130502}},\ \Eprint {http://arxiv.org/abs/arXiv:1209.6178} {arXiv:1209.6178}
  \BibitemShut {NoStop}%
\bibitem [{\citenamefont {Liu}(2014)}]{Liu14}%
  \BibitemOpen
  \bibfield  {author} {\bibinfo {author} {\bibnamefont {Liu}, \bibfnamefont
  {Yi-Kai}}} (\bibinfo {year} {2014}),\ \bibfield  {title} {\enquote {\bibinfo
  {title} {Single-shot security for one-time memories in the isolated qubits
  model},}\ }in\ \href {\doibase 10.1007/978-3-662-44381-1_2} {\emph {\bibinfo
  {booktitle} {Advances in Cryptology -- CRYPTO 2014}}},\ \bibinfo {editor}
  {edited by\ \bibinfo {editor} {\bibfnamefont {Juan~A.}\ \bibnamefont
  {Garay}}\ and\ \bibinfo {editor} {\bibfnamefont {Rosario}\ \bibnamefont
  {Gennaro}}}\ (\bibinfo  {publisher} {Springer})\ pp.\ \bibinfo {pages}
  {19--36},\ \Eprint {http://arxiv.org/abs/arXiv:1402.0049} {arXiv:1402.0049}
  \BibitemShut {NoStop}%
\bibitem [{\citenamefont {Liu}(2015)}]{Liu15}%
  \BibitemOpen
  \bibfield  {author} {\bibinfo {author} {\bibnamefont {Liu}, \bibfnamefont
  {Yi-Kai}}} (\bibinfo {year} {2015}),\ \bibfield  {title} {\enquote {\bibinfo
  {title} {Privacy amplification in the isolated qubits model},}\ }in\ \href
  {\doibase 10.1007/978-3-662-46803-6_26} {\emph {\bibinfo {booktitle}
  {Advances in Cryptology -- EUROCRYPT 2015}}},\ \bibinfo {editor} {edited by\
  \bibinfo {editor} {\bibfnamefont {Elisabeth}\ \bibnamefont {Oswald}}\ and\
  \bibinfo {editor} {\bibfnamefont {Marc}\ \bibnamefont {Fischlin}}}\ (\bibinfo
   {publisher} {Springer})\ pp.\ \bibinfo {pages} {785--814},\ \Eprint
  {http://arxiv.org/abs/arxiv:1410.3918} {arxiv:1410.3918} \BibitemShut
  {NoStop}%
\bibitem [{\citenamefont {Lo}(2003)}]{Lo03}%
  \BibitemOpen
  \bibfield  {author} {\bibinfo {author} {\bibnamefont {Lo}, \bibfnamefont
  {Hoi-Kwong}}} (\bibinfo {year} {2003}),\ \bibfield  {title} {\enquote
  {\bibinfo {title} {Method for decoupling error correction from privacy
  amplification},}\ }\href {\doibase 10.1088/1367-2630/5/1/336} {\bibfield
  {journal} {\bibinfo  {journal} {New J. Phys.}\ }\textbf {\bibinfo {volume}
  {5}},\ \bibinfo {pages} {36--36}}\BibitemShut {NoStop}%
\bibitem [{\citenamefont {Lo}\ and\ \citenamefont {Chau}(1999)}]{LoChau99}%
  \BibitemOpen
  \bibfield  {author} {\bibinfo {author} {\bibnamefont {Lo}, \bibfnamefont
  {Hoi-Kwong}}, \ and\ \bibinfo {author} {\bibfnamefont {Hoi~Fung}\
  \bibnamefont {Chau}}} (\bibinfo {year} {1999}),\ \bibfield  {title} {\enquote
  {\bibinfo {title} {Unconditional security of quantum key distribution over
  arbitrarily long distances},}\ }\href {\doibase
  10.1126/science.283.5410.2050} {\bibfield  {journal} {\bibinfo  {journal}
  {Science}\ }\textbf {\bibinfo {volume} {283}}~(\bibinfo {number} {5410}),\
  \bibinfo {pages} {2050--2056}}\BibitemShut {NoStop}%
\bibitem [{\citenamefont {Lo}\ \emph {et~al.}(2012)\citenamefont {Lo},
  \citenamefont {Curty},\ and\ \citenamefont {Qi}}]{LCQ12}%
  \BibitemOpen
  \bibfield  {author} {\bibinfo {author} {\bibnamefont {Lo}, \bibfnamefont
  {Hoi-Kwong}}, \bibinfo {author} {\bibfnamefont {Marcos}\ \bibnamefont
  {Curty}}, \ and\ \bibinfo {author} {\bibfnamefont {Bing}\ \bibnamefont {Qi}}}
  (\bibinfo {year} {2012}),\ \bibfield  {title} {\enquote {\bibinfo {title}
  {Measurement-device-independent quantum key distribution},}\ }\href {\doibase
  10.1103/PhysRevLett.108.130503} {\bibfield  {journal} {\bibinfo  {journal}
  {Phys. Rev. Lett.}\ }\textbf {\bibinfo {volume} {108}},\ \bibinfo {pages}
  {130503}},\ \Eprint {http://arxiv.org/abs/arXiv:1109.1473} {arXiv:1109.1473}
  \BibitemShut {NoStop}%
\bibitem [{\citenamefont {Lo}\ \emph {et~al.}(2005)\citenamefont {Lo},
  \citenamefont {Ma},\ and\ \citenamefont {Chen}}]{Loetal2005}%
  \BibitemOpen
  \bibfield  {author} {\bibinfo {author} {\bibnamefont {Lo}, \bibfnamefont
  {Hoi-Kwong}}, \bibinfo {author} {\bibfnamefont {Xiongfeng}\ \bibnamefont
  {Ma}}, \ and\ \bibinfo {author} {\bibfnamefont {Kai}\ \bibnamefont {Chen}}}
  (\bibinfo {year} {2005}),\ \bibfield  {title} {\enquote {\bibinfo {title}
  {Decoy state quantum key distribution},}\ }\href {\doibase
  10.1103/PhysRevLett.94.230504} {\bibfield  {journal} {\bibinfo  {journal}
  {Phys. Rev. Lett.}\ }\textbf {\bibinfo {volume} {94}},\ \bibinfo {pages}
  {230504}}\BibitemShut {NoStop}%
\bibitem [{\citenamefont {Lucamarini}\ \emph {et~al.}(2018)\citenamefont
  {Lucamarini}, \citenamefont {Yuan}, \citenamefont {Dynes},\ and\
  \citenamefont {Shields}}]{Lucamarinietal}%
  \BibitemOpen
  \bibfield  {author} {\bibinfo {author} {\bibnamefont {Lucamarini},
  \bibfnamefont {Marco}}, \bibinfo {author} {\bibfnamefont {Zhiliang~L.}\
  \bibnamefont {Yuan}}, \bibinfo {author} {\bibfnamefont {James~F.}\
  \bibnamefont {Dynes}}, \ and\ \bibinfo {author} {\bibfnamefont {Andrew~J.}\
  \bibnamefont {Shields}}} (\bibinfo {year} {2018}),\ \bibfield  {title}
  {\enquote {\bibinfo {title} {Overcoming the rate--distance limit of quantum
  key distribution without quantum repeaters},}\ }\href@noop {} {\bibfield
  {journal} {\bibinfo  {journal} {Nature}\ }\textbf {\bibinfo {volume}
  {557}}~(\bibinfo {number} {7705}),\ \bibinfo {pages} {400--403}}\BibitemShut
  {NoStop}%
\bibitem [{\citenamefont {L\"utkenhaus}(2000)}]{Lutkenhaus2000}%
  \BibitemOpen
  \bibfield  {author} {\bibinfo {author} {\bibnamefont {L\"utkenhaus},
  \bibfnamefont {Norbert}}} (\bibinfo {year} {2000}),\ \bibfield  {title}
  {\enquote {\bibinfo {title} {Security against individual attacks for
  realistic quantum key distribution},}\ }\href {\doibase
  10.1103/PhysRevA.61.052304} {\bibfield  {journal} {\bibinfo  {journal} {Phys.
  Rev. A}\ }\textbf {\bibinfo {volume} {61}},\ \bibinfo {pages}
  {052304}}\BibitemShut {NoStop}%
\bibitem [{\citenamefont {Lydersen}\ \emph {et~al.}(2010)\citenamefont
  {Lydersen}, \citenamefont {Wiechers}, \citenamefont {Wittmann}, \citenamefont
  {Elser}, \citenamefont {Skaar},\ and\ \citenamefont {Makarov}}]{LWWESM10}%
  \BibitemOpen
  \bibfield  {author} {\bibinfo {author} {\bibnamefont {Lydersen},
  \bibfnamefont {Lars}}, \bibinfo {author} {\bibfnamefont {Carlos}\
  \bibnamefont {Wiechers}}, \bibinfo {author} {\bibfnamefont {Christoffer}\
  \bibnamefont {Wittmann}}, \bibinfo {author} {\bibfnamefont {Dominique}\
  \bibnamefont {Elser}}, \bibinfo {author} {\bibfnamefont {Johannes}\
  \bibnamefont {Skaar}}, \ and\ \bibinfo {author} {\bibfnamefont {Vadim}\
  \bibnamefont {Makarov}}} (\bibinfo {year} {2010}),\ \bibfield  {title}
  {\enquote {\bibinfo {title} {Hacking commercial quantum cryptography systems
  by tailored bright illumination},}\ }\href {\doibase
  10.1038/nphoton.2010.214} {\bibfield  {journal} {\bibinfo  {journal} {Nat.
  Photonics}\ }\textbf {\bibinfo {volume} {4}}~(\bibinfo {number} {10}),\
  \bibinfo {pages} {686--689}},\ \Eprint {http://arxiv.org/abs/arXiv:1008.4593}
  {arXiv:1008.4593} \BibitemShut {NoStop}%
\bibitem [{\citenamefont {Ma}\ \emph {et~al.}(2019)\citenamefont {Ma},
  \citenamefont {Zhou}, \citenamefont {Yuan},\ and\ \citenamefont
  {Ma}}]{MZYM19}%
  \BibitemOpen
  \bibfield  {author} {\bibinfo {author} {\bibnamefont {Ma}, \bibfnamefont
  {Jiajun}}, \bibinfo {author} {\bibfnamefont {You}\ \bibnamefont {Zhou}},
  \bibinfo {author} {\bibfnamefont {Xiao}\ \bibnamefont {Yuan}}, \ and\
  \bibinfo {author} {\bibfnamefont {Xiongfeng}\ \bibnamefont {Ma}}} (\bibinfo
  {year} {2019}),\ \bibfield  {title} {\enquote {\bibinfo {title} {Operational
  interpretation of coherence in quantum key distribution},}\ }\href {\doibase
  10.1103/PhysRevA.99.062325} {\bibfield  {journal} {\bibinfo  {journal} {Phys.
  Rev. A}\ }\textbf {\bibinfo {volume} {99}},\ \bibinfo {pages} {062325}},\
  \Eprint {http://arxiv.org/abs/arXiv:1810.03267} {arXiv:1810.03267}
  \BibitemShut {NoStop}%
\bibitem [{\citenamefont {Ma}\ and\ \citenamefont {Razavi}(2012)}]{MR12}%
  \BibitemOpen
  \bibfield  {author} {\bibinfo {author} {\bibnamefont {Ma}, \bibfnamefont
  {Xiongfeng}}, \ and\ \bibinfo {author} {\bibfnamefont {Mohsen}\ \bibnamefont
  {Razavi}}} (\bibinfo {year} {2012}),\ \bibfield  {title} {\enquote {\bibinfo
  {title} {Alternative schemes for measurement-device-independent quantum key
  distribution},}\ }\href {\doibase 10.1103/PhysRevA.86.062319} {\bibfield
  {journal} {\bibinfo  {journal} {Phys. Rev. A}\ }\textbf {\bibinfo {volume}
  {86}},\ \bibinfo {pages} {062319}},\ \Eprint
  {http://arxiv.org/abs/arxiv:1204.4856} {arxiv:1204.4856} \BibitemShut
  {NoStop}%
\bibitem [{\citenamefont {Makarov}(2009)}]{Makarov2009}%
  \BibitemOpen
  \bibfield  {author} {\bibinfo {author} {\bibnamefont {Makarov}, \bibfnamefont
  {Vadim}}} (\bibinfo {year} {2009}),\ \bibfield  {title} {\enquote {\bibinfo
  {title} {Controlling passively quenched single photon detectors by bright
  light},}\ }\href {\doibase 10.1088/1367-2630/11/6/065003} {\bibfield
  {journal} {\bibinfo  {journal} {New J. Phys.}\ }\textbf {\bibinfo {volume}
  {11}}~(\bibinfo {number} {6}),\ \bibinfo {pages} {065003}}\BibitemShut
  {NoStop}%
\bibitem [{\citenamefont {Makarov}\ \emph {et~al.}(2006)\citenamefont
  {Makarov}, \citenamefont {Anisimov},\ and\ \citenamefont
  {Skaar}}]{Makarovetal2006}%
  \BibitemOpen
  \bibfield  {author} {\bibinfo {author} {\bibnamefont {Makarov}, \bibfnamefont
  {Vadim}}, \bibinfo {author} {\bibfnamefont {Andrey}\ \bibnamefont
  {Anisimov}}, \ and\ \bibinfo {author} {\bibfnamefont {Johannes}\ \bibnamefont
  {Skaar}}} (\bibinfo {year} {2006}),\ \bibfield  {title} {\enquote {\bibinfo
  {title} {Effects of detector efficiency mismatch on security of quantum
  cryptosystems},}\ }\href {\doibase 10.1103/PhysRevA.74.022313} {\bibfield
  {journal} {\bibinfo  {journal} {Phys. Rev. A}\ }\textbf {\bibinfo {volume}
  {74}},\ \bibinfo {pages} {022313}}\BibitemShut {NoStop}%
\bibitem [{\citenamefont {Mateus}\ \emph {et~al.}(2003)\citenamefont {Mateus},
  \citenamefont {Mitchell},\ and\ \citenamefont {Scedrov}}]{MMS03}%
  \BibitemOpen
  \bibfield  {author} {\bibinfo {author} {\bibnamefont {Mateus}, \bibfnamefont
  {Paulo}}, \bibinfo {author} {\bibfnamefont {John~C.}\ \bibnamefont
  {Mitchell}}, \ and\ \bibinfo {author} {\bibfnamefont {Andre}\ \bibnamefont
  {Scedrov}}} (\bibinfo {year} {2003}),\ \bibfield  {title} {\enquote {\bibinfo
  {title} {Composition of cryptographic protocols in a probabilistic
  polynomial-time process calculus},}\ }in\ \href {\doibase
  10.1007/978-3-540-45187-7_22} {\emph {\bibinfo {booktitle} {{CONCUR} 2003 --
  Concurrency Theory}}},\ \bibinfo {series} {LNCS}, Vol.\ \bibinfo {volume}
  {2761}\ (\bibinfo  {publisher} {Springer})\ pp.\ \bibinfo {pages}
  {327--349}\BibitemShut {NoStop}%
\bibitem [{\citenamefont {Mauerer}\ \emph {et~al.}(2012)\citenamefont
  {Mauerer}, \citenamefont {Portmann},\ and\ \citenamefont {Scholz}}]{MPS12}%
  \BibitemOpen
  \bibfield  {author} {\bibinfo {author} {\bibnamefont {Mauerer}, \bibfnamefont
  {Wolfgang}}, \bibinfo {author} {\bibfnamefont {Christopher}\ \bibnamefont
  {Portmann}}, \ and\ \bibinfo {author} {\bibfnamefont {Volkher~B.}\
  \bibnamefont {Scholz}}} (\bibinfo {year} {2012}),\ \href@noop {} {\enquote
  {\bibinfo {title} {A modular framework for randomness extraction based on
  trevisan's construction},}\ }\bibinfo {howpublished} {e-Print},\ \Eprint
  {http://arxiv.org/abs/arXiv:1212.0520} {arXiv:1212.0520} \BibitemShut
  {NoStop}%
\bibitem [{\citenamefont {Maurer}(1993)}]{Mau93}%
  \BibitemOpen
  \bibfield  {author} {\bibinfo {author} {\bibnamefont {Maurer}, \bibfnamefont
  {Ueli}}} (\bibinfo {year} {1993}),\ \bibfield  {title} {\enquote {\bibinfo
  {title} {Secret key agreement by public discussion},}\ }\href {\doibase
  10.1109/18.256484} {\bibfield  {journal} {\bibinfo  {journal} {IEEE Trans.
  Inf. Theory}\ }\textbf {\bibinfo {volume} {39}}~(\bibinfo {number} {3}),\
  \bibinfo {pages} {733--742}},\ \bibinfo {note} {a preliminary version
  appeared at CRYPTO~'92}\BibitemShut {NoStop}%
\bibitem [{\citenamefont {Maurer}(1994)}]{Mau94}%
  \BibitemOpen
  \bibfield  {author} {\bibinfo {author} {\bibnamefont {Maurer}, \bibfnamefont
  {Ueli}}} (\bibinfo {year} {1994}),\ \enquote {\bibinfo {title} {The strong
  secret key rate of discrete random triples},}\ in\ \href {\doibase
  10.1007/978-1-4615-2694-0_27} {\emph {\bibinfo {booktitle} {Communications
  and Cryptography: Two Sides of One Tapestry}}},\ \bibinfo {series} {The
  Springer International Series in Engineering and Computer Science}, Vol.\
  \bibinfo {volume} {276}\ (\bibinfo  {publisher} {Springer})\ pp.\ \bibinfo
  {pages} {271--285}\BibitemShut {NoStop}%
\bibitem [{\citenamefont {Maurer}(2002)}]{Mau02}%
  \BibitemOpen
  \bibfield  {author} {\bibinfo {author} {\bibnamefont {Maurer}, \bibfnamefont
  {Ueli}}} (\bibinfo {year} {2002}),\ \bibfield  {title} {\enquote {\bibinfo
  {title} {Indistinguishability of random systems},}\ }in\ \href {\doibase
  10.1007/3-540-46035-7_8} {\emph {\bibinfo {booktitle} {Advances in Cryptology
  -- EUROCRYPT 2002}}},\ \bibinfo {series} {LNCS}, Vol.\ \bibinfo {volume}
  {2332}\ (\bibinfo  {publisher} {Springer})\ pp.\ \bibinfo {pages}
  {110--132}\BibitemShut {NoStop}%
\bibitem [{\citenamefont {Maurer}(2012)}]{Mau12}%
  \BibitemOpen
  \bibfield  {author} {\bibinfo {author} {\bibnamefont {Maurer}, \bibfnamefont
  {Ueli}}} (\bibinfo {year} {2012}),\ \bibfield  {title} {\enquote {\bibinfo
  {title} {Constructive cryptography---a new paradigm for security definitions
  and proofs},}\ }in\ \href {\doibase 10.1007/978-3-642-27375-9_3} {\emph
  {\bibinfo {booktitle} {Proceedings of Theory of Security and Applications,
  TOSCA 2011}}},\ \bibinfo {series} {LNCS}, Vol.\ \bibinfo {volume} {6993}\
  (\bibinfo  {publisher} {Springer})\ pp.\ \bibinfo {pages}
  {33--56}\BibitemShut {NoStop}%
\bibitem [{\citenamefont {Maurer}\ \emph {et~al.}(2007)\citenamefont {Maurer},
  \citenamefont {Pietrzak},\ and\ \citenamefont {Renner}}]{MPR07}%
  \BibitemOpen
  \bibfield  {author} {\bibinfo {author} {\bibnamefont {Maurer}, \bibfnamefont
  {Ueli}}, \bibinfo {author} {\bibfnamefont {Krzysztof}\ \bibnamefont
  {Pietrzak}}, \ and\ \bibinfo {author} {\bibfnamefont {Renato}\ \bibnamefont
  {Renner}}} (\bibinfo {year} {2007}),\ \bibfield  {title} {\enquote {\bibinfo
  {title} {Indistinguishability amplification},}\ }in\ \href {\doibase
  10.1007/978-3-540-74143-5_8} {\emph {\bibinfo {booktitle} {Advances in
  Cryptology -- CRYPTO 2007}}},\ \bibinfo {series} {LNCS}, Vol.\ \bibinfo
  {volume} {4622}\ (\bibinfo  {publisher} {Springer})\ pp.\ \bibinfo {pages}
  {130--149}\BibitemShut {NoStop}%
\bibitem [{\citenamefont {Maurer}\ and\ \citenamefont {Renner}(2011)}]{MR11}%
  \BibitemOpen
  \bibfield  {author} {\bibinfo {author} {\bibnamefont {Maurer}, \bibfnamefont
  {Ueli}}, \ and\ \bibinfo {author} {\bibfnamefont {Renato}\ \bibnamefont
  {Renner}}} (\bibinfo {year} {2011}),\ \bibfield  {title} {\enquote {\bibinfo
  {title} {Abstract cryptography},}\ }in\ \href@noop {} {\emph {\bibinfo
  {booktitle} {Proceedings of Innovations in Computer Science, ICS 2011}}}\
  (\bibinfo  {publisher} {Tsinghua University Press})\ pp.\ \bibinfo {pages}
  {1--21}\BibitemShut {NoStop}%
\bibitem [{\citenamefont {Maurer}\ and\ \citenamefont {Renner}(2016)}]{MR16}%
  \BibitemOpen
  \bibfield  {author} {\bibinfo {author} {\bibnamefont {Maurer}, \bibfnamefont
  {Ueli}}, \ and\ \bibinfo {author} {\bibfnamefont {Renato}\ \bibnamefont
  {Renner}}} (\bibinfo {year} {2016}),\ \bibfield  {title} {\enquote {\bibinfo
  {title} {From indifferentiability to constructive cryptography (and back)},}\
  }in\ \href {\doibase 10.1007/978-3-662-53641-4_1} {\emph {\bibinfo
  {booktitle} {Theory of Cryptography, Proceedings of {TCC} 2016-B, Part
  {I}}}},\ \bibinfo {series} {LNCS}, Vol.\ \bibinfo {volume} {9985}\ (\bibinfo
  {publisher} {Springer})\ pp.\ \bibinfo {pages} {3--24},\ \bibinfo {note}
  {e-Print \href{http://eprint.iacr.org/2016/903}{IACR 2016/903}}\BibitemShut
  {NoStop}%
\bibitem [{\citenamefont {Maurer}\ \emph {et~al.}(2012)\citenamefont {Maurer},
  \citenamefont {R\"uedlinger},\ and\ \citenamefont {Tackmann}}]{MRT12}%
  \BibitemOpen
  \bibfield  {author} {\bibinfo {author} {\bibnamefont {Maurer}, \bibfnamefont
  {Ueli}}, \bibinfo {author} {\bibfnamefont {Andreas}\ \bibnamefont
  {R\"uedlinger}}, \ and\ \bibinfo {author} {\bibfnamefont {Bj\"orn}\
  \bibnamefont {Tackmann}}} (\bibinfo {year} {2012}),\ \bibfield  {title}
  {\enquote {\bibinfo {title} {Confidentiality and integrity: A constructive
  perspective},}\ }in\ \href {\doibase 10.1007/978-3-642-28914-9_12} {\emph
  {\bibinfo {booktitle} {Theory of Cryptography, Proceedings of TCC 2012}}},\
  \bibinfo {series} {LNCS}, Vol.\ \bibinfo {volume} {7194},\ \bibinfo {editor}
  {edited by\ \bibinfo {editor} {\bibfnamefont {Ronald}\ \bibnamefont
  {Cramer}}}\ (\bibinfo  {publisher} {Springer})\ pp.\ \bibinfo {pages}
  {209--229}\BibitemShut {NoStop}%
\bibitem [{\citenamefont {Maurer}\ and\ \citenamefont {Wolf}(2000)}]{MW00}%
  \BibitemOpen
  \bibfield  {author} {\bibinfo {author} {\bibnamefont {Maurer}, \bibfnamefont
  {Ueli}}, \ and\ \bibinfo {author} {\bibfnamefont {Stefan}\ \bibnamefont
  {Wolf}}} (\bibinfo {year} {2000}),\ \bibfield  {title} {\enquote {\bibinfo
  {title} {Information-theoretic key agreement: From weak to strong secrecy for
  free},}\ }in\ \href {\doibase 10.1007/3-540-45539-6_24} {\emph {\bibinfo
  {booktitle} {Advances in Cryptology -- EUROCRYPT 2000}}},\ \bibinfo {series}
  {LNCS}, Vol.\ \bibinfo {volume} {1807}\ (\bibinfo  {publisher} {Springer})\
  pp.\ \bibinfo {pages} {351--368}\BibitemShut {NoStop}%
\bibitem [{\citenamefont {Mayers}(1996)}]{May96}%
  \BibitemOpen
  \bibfield  {author} {\bibinfo {author} {\bibnamefont {Mayers}, \bibfnamefont
  {Dominic}}} (\bibinfo {year} {1996}),\ \bibfield  {title} {\enquote {\bibinfo
  {title} {Quantum key distribution and string oblivious transfer in noisy
  channels},}\ }in\ \href {\doibase 10.1007/3-540-68697-5_26} {\emph {\bibinfo
  {booktitle} {Advances in Cryptology -- CRYPTO~'96}}},\ \bibinfo {series}
  {LNCS}, Vol.\ \bibinfo {volume} {1109}\ (\bibinfo  {publisher} {Springer})\
  pp.\ \bibinfo {pages} {343--357},\ \Eprint
  {http://arxiv.org/abs/arXiv:quant-ph/9606003} {arXiv:quant-ph/9606003}
  \BibitemShut {NoStop}%
\bibitem [{\citenamefont {Mayers}(2001)}]{May01}%
  \BibitemOpen
  \bibfield  {author} {\bibinfo {author} {\bibnamefont {Mayers}, \bibfnamefont
  {Dominic}}} (\bibinfo {year} {2001}),\ \bibfield  {title} {\enquote {\bibinfo
  {title} {Unconditional security in quantum cryptography},}\ }\href {\doibase
  10.1145/382780.382781} {\bibfield  {journal} {\bibinfo  {journal} {J. ACM}\
  }\textbf {\bibinfo {volume} {48}}~(\bibinfo {number} {3}),\ \bibinfo {pages}
  {351--406}},\ \Eprint {http://arxiv.org/abs/arXiv:quant-ph/9802025}
  {arXiv:quant-ph/9802025} \BibitemShut {NoStop}%
\bibitem [{\citenamefont {Micali}\ and\ \citenamefont {Rogaway}(1992)}]{MR92}%
  \BibitemOpen
  \bibfield  {author} {\bibinfo {author} {\bibnamefont {Micali}, \bibfnamefont
  {Silvio}}, \ and\ \bibinfo {author} {\bibfnamefont {Phillip}\ \bibnamefont
  {Rogaway}}} (\bibinfo {year} {1992}),\ \bibfield  {title} {\enquote {\bibinfo
  {title} {Secure computation (abstract)},}\ }in\ \href {\doibase
  10.1007/3-540-46766-1_32} {\emph {\bibinfo {booktitle} {Advances in
  Cryptology -- CRYPTO~'91}}},\ \bibinfo {series} {LNCS}, Vol.\ \bibinfo
  {volume} {576}\ (\bibinfo  {publisher} {Springer})\ pp.\ \bibinfo {pages}
  {392--404}\BibitemShut {NoStop}%
\bibitem [{\citenamefont {Miller}\ and\ \citenamefont {Shi}(2014)}]{MS14}%
  \BibitemOpen
  \bibfield  {author} {\bibinfo {author} {\bibnamefont {Miller}, \bibfnamefont
  {Carl}}, \ and\ \bibinfo {author} {\bibfnamefont {Yaoyun}\ \bibnamefont
  {Shi}}} (\bibinfo {year} {2014}),\ \bibfield  {title} {\enquote {\bibinfo
  {title} {Robust protocols for securely expanding randomness and distributing
  keys using untrusted quantum devices},}\ }in\ \href {\doibase
  10.1145/2591796.2591843} {\emph {\bibinfo {booktitle} {Proceedings of the
  46th Symposium on Theory of Computing, STOC~'14}}}\ (\bibinfo  {publisher}
  {ACM})\ pp.\ \bibinfo {pages} {417--426},\ \Eprint
  {http://arxiv.org/abs/arXiv:1402.0489} {arXiv:1402.0489} \BibitemShut
  {NoStop}%
\bibitem [{\citenamefont {Mitchell}\ \emph {et~al.}(2006)\citenamefont
  {Mitchell}, \citenamefont {Ramanathan}, \citenamefont {Scedrov},\ and\
  \citenamefont {Teague}}]{MRST06}%
  \BibitemOpen
  \bibfield  {author} {\bibinfo {author} {\bibnamefont {Mitchell},
  \bibfnamefont {John~C}}, \bibinfo {author} {\bibfnamefont {Ajith}\
  \bibnamefont {Ramanathan}}, \bibinfo {author} {\bibfnamefont {Andre}\
  \bibnamefont {Scedrov}}, \ and\ \bibinfo {author} {\bibfnamefont {Vanessa}\
  \bibnamefont {Teague}}} (\bibinfo {year} {2006}),\ \bibfield  {title}
  {\enquote {\bibinfo {title} {A probabilistic polynomial-time process calculus
  for the analysis of cryptographic protocols},}\ }\href {\doibase
  10.1016/j.tcs.2005.10.044} {\bibfield  {journal} {\bibinfo  {journal} {Theor.
  Comput. Sci.}\ }\textbf {\bibinfo {volume} {353}}~(\bibinfo {number}
  {1–3}),\ \bibinfo {pages} {118--164}}\BibitemShut {NoStop}%
\bibitem [{\citenamefont {Muller}\ \emph {et~al.}(1997)\citenamefont {Muller},
  \citenamefont {Herzog}, \citenamefont {Huttner}, \citenamefont {Tittel},
  \citenamefont {Zbinden},\ and\ \citenamefont {Gisin}}]{Muller97}%
  \BibitemOpen
  \bibfield  {author} {\bibinfo {author} {\bibnamefont {Muller}, \bibfnamefont
  {Antoine}}, \bibinfo {author} {\bibfnamefont {Thomas}\ \bibnamefont
  {Herzog}}, \bibinfo {author} {\bibfnamefont {Bruno}\ \bibnamefont {Huttner}},
  \bibinfo {author} {\bibfnamefont {Woflgang}\ \bibnamefont {Tittel}}, \bibinfo
  {author} {\bibfnamefont {Hugo}\ \bibnamefont {Zbinden}}, \ and\ \bibinfo
  {author} {\bibfnamefont {Nicolas}\ \bibnamefont {Gisin}}} (\bibinfo {year}
  {1997}),\ \bibfield  {title} {\enquote {\bibinfo {title} {``plug and play''
  systems for quantum cryptography},}\ }\href {\doibase
  https://doi.org/10.1063/1.118224} {\bibfield  {journal} {\bibinfo  {journal}
  {Appl. Phys. Lett.}\ }\textbf {\bibinfo {volume} {70}}~(\bibinfo {number}
  {7}),\ \bibinfo {pages} {793--795}}\BibitemShut {NoStop}%
\bibitem [{\citenamefont {M\"uller-Quade}\ and\ \citenamefont
  {Renner}(2009)}]{MR09}%
  \BibitemOpen
  \bibfield  {author} {\bibinfo {author} {\bibnamefont {M\"uller-Quade},
  \bibfnamefont {J\"orn}}, \ and\ \bibinfo {author} {\bibfnamefont {Renato}\
  \bibnamefont {Renner}}} (\bibinfo {year} {2009}),\ \bibfield  {title}
  {\enquote {\bibinfo {title} {Composability in quantum cryptography},}\ }\href
  {\doibase 10.1088/1367-2630/11/8/085006} {\bibfield  {journal} {\bibinfo
  {journal} {New J. Phys.}\ }\textbf {\bibinfo {volume} {11}}~(\bibinfo
  {number} {8}),\ \bibinfo {pages} {085006}},\ \Eprint
  {http://arxiv.org/abs/arXiv:1006.2215} {arXiv:1006.2215} \BibitemShut
  {NoStop}%
\bibitem [{\citenamefont {Nielsen}\ and\ \citenamefont
  {Chuang}(2010)}]{nielsen2010quantum}%
  \BibitemOpen
  \bibfield  {author} {\bibinfo {author} {\bibnamefont {Nielsen}, \bibfnamefont
  {Michael~A}}, \ and\ \bibinfo {author} {\bibfnamefont {Isaac~L}\ \bibnamefont
  {Chuang}}} (\bibinfo {year} {2010}),\ \href@noop {} {\emph {\bibinfo {title}
  {Quantum Computation and Quantum Information}}}\ (\bibinfo  {publisher}
  {Cambridge University Press})\BibitemShut {NoStop}%
\bibitem [{\citenamefont {Nishioka}\ \emph {et~al.}(2002)\citenamefont
  {Nishioka}, \citenamefont {Ishizuka}, \citenamefont {Toshio},\ and\
  \citenamefont {Abe}}]{Nishioka2002}%
  \BibitemOpen
  \bibfield  {author} {\bibinfo {author} {\bibnamefont {Nishioka},
  \bibfnamefont {Tsuyoshi}}, \bibinfo {author} {\bibfnamefont {Hirokazu}\
  \bibnamefont {Ishizuka}}, \bibinfo {author} {\bibnamefont {Toshio}}, \ and\
  \bibinfo {author} {\bibfnamefont {Junichi}\ \bibnamefont {Abe}}} (\bibinfo
  {year} {2002}),\ \bibfield  {title} {\enquote {\bibinfo {title} {``circular
  type'' quantum key distribution},}\ }\href {\doibase 10.1109/68.992616}
  {\bibfield  {journal} {\bibinfo  {journal} {IEEE Photonics Technol. Lett.}\
  }\textbf {\bibinfo {volume} {14}}~(\bibinfo {number} {4}),\ \bibinfo {pages}
  {576--578}},\ \Eprint {http://arxiv.org/abs/arXiv:quant-ph/0106083}
  {arXiv:quant-ph/0106083} \BibitemShut {NoStop}%
\bibitem [{\citenamefont {Ohya}\ and\ \citenamefont {Petz}(1993)}]{OP93}%
  \BibitemOpen
  \bibfield  {author} {\bibinfo {author} {\bibnamefont {Ohya}, \bibfnamefont
  {Masanori}}, \ and\ \bibinfo {author} {\bibfnamefont {D\'enes}\ \bibnamefont
  {Petz}}} (\bibinfo {year} {1993}),\ \href@noop {} {\emph {\bibinfo {title}
  {Quantum Entropy and Its Use}}}\ (\bibinfo  {publisher}
  {Springer})\BibitemShut {NoStop}%
\bibitem [{\citenamefont {Paw\l{}owski}\ and\ \citenamefont
  {Brunner}(2011)}]{PB11}%
  \BibitemOpen
  \bibfield  {author} {\bibinfo {author} {\bibnamefont {Paw\l{}owski},
  \bibfnamefont {Marcin}}, \ and\ \bibinfo {author} {\bibfnamefont {Nicolas}\
  \bibnamefont {Brunner}}} (\bibinfo {year} {2011}),\ \bibfield  {title}
  {\enquote {\bibinfo {title} {Semi-device-independent security of one-way
  quantum key distribution},}\ }\href {\doibase 10.1103/PhysRevA.84.010302}
  {\bibfield  {journal} {\bibinfo  {journal} {Phys. Rev. A}\ }\textbf {\bibinfo
  {volume} {84}},\ \bibinfo {pages} {010302}},\ \Eprint
  {http://arxiv.org/abs/arXiv:1103.4105} {arXiv:1103.4105} \BibitemShut
  {NoStop}%
\bibitem [{\citenamefont {Peres}\ and\ \citenamefont
  {Terno}(2004)}]{Peres_Terno_RMP}%
  \BibitemOpen
  \bibfield  {author} {\bibinfo {author} {\bibnamefont {Peres}, \bibfnamefont
  {Asher}}, \ and\ \bibinfo {author} {\bibfnamefont {Daniel~R.}\ \bibnamefont
  {Terno}}} (\bibinfo {year} {2004}),\ \bibfield  {title} {\enquote {\bibinfo
  {title} {Quantum information and relativity theory},}\ }\href {\doibase
  10.1103/RevModPhys.76.93} {\bibfield  {journal} {\bibinfo  {journal} {Rev.
  Mod. Phys.}\ }\textbf {\bibinfo {volume} {76}},\ \bibinfo {pages}
  {93--123}}\BibitemShut {NoStop}%
\bibitem [{\citenamefont {Pfitzmann}\ and\ \citenamefont
  {Waidner}(2000)}]{PW00}%
  \BibitemOpen
  \bibfield  {author} {\bibinfo {author} {\bibnamefont {Pfitzmann},
  \bibfnamefont {Birgit}}, \ and\ \bibinfo {author} {\bibfnamefont {Michael}\
  \bibnamefont {Waidner}}} (\bibinfo {year} {2000}),\ \bibfield  {title}
  {\enquote {\bibinfo {title} {Composition and integrity preservation of secure
  reactive systems},}\ }in\ \href {\doibase 10.1145/352600.352639} {\emph
  {\bibinfo {booktitle} {Proceedings of the 7th ACM Conference on Computer and
  Communications Security, CSS~'00}}}\ (\bibinfo  {publisher} {ACM})\ pp.\
  \bibinfo {pages} {245--254}\BibitemShut {NoStop}%
\bibitem [{\citenamefont {Pfitzmann}\ and\ \citenamefont
  {Waidner}(2001)}]{PW01}%
  \BibitemOpen
  \bibfield  {author} {\bibinfo {author} {\bibnamefont {Pfitzmann},
  \bibfnamefont {Birgit}}, \ and\ \bibinfo {author} {\bibfnamefont {Michael}\
  \bibnamefont {Waidner}}} (\bibinfo {year} {2001}),\ \bibfield  {title}
  {\enquote {\bibinfo {title} {A model for asynchronous reactive systems and
  its application to secure message transmission},}\ }in\ \href {\doibase
  10.1109/SECPRI.2001.924298} {\emph {\bibinfo {booktitle} {IEEE Symposium on
  Security and Privacy}}}\ (\bibinfo  {publisher} {IEEE})\ pp.\ \bibinfo
  {pages} {184--200}\BibitemShut {NoStop}%
\bibitem [{\citenamefont {Pirandola}\ \emph {et~al.}(2015)\citenamefont
  {Pirandola}, \citenamefont {Ottaviani}, \citenamefont {Spedalieri},
  \citenamefont {Weedbrook}, \citenamefont {Braunstein}, \citenamefont {Lloyd},
  \citenamefont {Gehring}, \citenamefont {Jacobsen},\ and\ \citenamefont
  {Andersen}}]{Pirandola2015}%
  \BibitemOpen
  \bibfield  {author} {\bibinfo {author} {\bibnamefont {Pirandola},
  \bibfnamefont {Stefano}}, \bibinfo {author} {\bibfnamefont {Carlo}\
  \bibnamefont {Ottaviani}}, \bibinfo {author} {\bibfnamefont {Gaetana}\
  \bibnamefont {Spedalieri}}, \bibinfo {author} {\bibfnamefont {Christian}\
  \bibnamefont {Weedbrook}}, \bibinfo {author} {\bibfnamefont {Samuel~L.}\
  \bibnamefont {Braunstein}}, \bibinfo {author} {\bibfnamefont {Seth}\
  \bibnamefont {Lloyd}}, \bibinfo {author} {\bibfnamefont {Tobias}\
  \bibnamefont {Gehring}}, \bibinfo {author} {\bibfnamefont {Christian~S.}\
  \bibnamefont {Jacobsen}}, \ and\ \bibinfo {author} {\bibfnamefont {Ulrik~L.}\
  \bibnamefont {Andersen}}} (\bibinfo {year} {2015}),\ \bibfield  {title}
  {\enquote {\bibinfo {title} {High-rate measurement-device-independent quantum
  cryptography},}\ }\href {\doibase 10.1038/nphoton.2015.83} {\bibfield
  {journal} {\bibinfo  {journal} {Nat. Photonics}\ }\textbf {\bibinfo {volume}
  {9}},\ \bibinfo {pages} {397}}\BibitemShut {NoStop}%
\bibitem [{\citenamefont {Pironio}\ \emph {et~al.}(2009)\citenamefont
  {Pironio}, \citenamefont {Ac\'in}, \citenamefont {Brunner}, \citenamefont
  {Gisin}, \citenamefont {Massar},\ and\ \citenamefont {Scarani}}]{PABGMS09}%
  \BibitemOpen
  \bibfield  {author} {\bibinfo {author} {\bibnamefont {Pironio}, \bibfnamefont
  {Stefano}}, \bibinfo {author} {\bibfnamefont {Antonio}\ \bibnamefont
  {Ac\'in}}, \bibinfo {author} {\bibfnamefont {Nicolas}\ \bibnamefont
  {Brunner}}, \bibinfo {author} {\bibfnamefont {Nicolas}\ \bibnamefont
  {Gisin}}, \bibinfo {author} {\bibfnamefont {Serge}\ \bibnamefont {Massar}}, \
  and\ \bibinfo {author} {\bibfnamefont {Valerio}\ \bibnamefont {Scarani}}}
  (\bibinfo {year} {2009}),\ \bibfield  {title} {\enquote {\bibinfo {title}
  {Device-independent quantum key distribution secure against collective
  attacks},}\ }\href {\doibase 10.1088/1367-2630/11/4/045021} {\bibfield
  {journal} {\bibinfo  {journal} {New J. Phys.}\ }\textbf {\bibinfo {volume}
  {11}}~(\bibinfo {number} {4}),\ \bibinfo {pages} {045021}},\ \Eprint
  {http://arxiv.org/abs/arXiv:0903.4460} {arXiv:0903.4460} \BibitemShut
  {NoStop}%
\bibitem [{\citenamefont {Pironio}\ \emph {et~al.}(2010)\citenamefont
  {Pironio}, \citenamefont {Ac\'in}, \citenamefont {Massar}, \citenamefont
  {de~La~Giroday}, \citenamefont {Matsukevich}, \citenamefont {Maunz},
  \citenamefont {Olmschenk}, \citenamefont {Hayes}, \citenamefont {Luo},\ and\
  \citenamefont {Manning}}]{PAM10}%
  \BibitemOpen
  \bibfield  {author} {\bibinfo {author} {\bibnamefont {Pironio}, \bibfnamefont
  {Stefano}}, \bibinfo {author} {\bibfnamefont {Antonio}\ \bibnamefont
  {Ac\'in}}, \bibinfo {author} {\bibfnamefont {Serge}\ \bibnamefont {Massar}},
  \bibinfo {author} {\bibfnamefont {A~Boyer}\ \bibnamefont {de~La~Giroday}},
  \bibinfo {author} {\bibfnamefont {Dzimitry~N}\ \bibnamefont {Matsukevich}},
  \bibinfo {author} {\bibfnamefont {Peter}\ \bibnamefont {Maunz}}, \bibinfo
  {author} {\bibfnamefont {Steven}\ \bibnamefont {Olmschenk}}, \bibinfo
  {author} {\bibfnamefont {David}\ \bibnamefont {Hayes}}, \bibinfo {author}
  {\bibfnamefont {Le}~\bibnamefont {Luo}}, \ and\ \bibinfo {author}
  {\bibfnamefont {T~Andrew}\ \bibnamefont {Manning}}} (\bibinfo {year}
  {2010}),\ \bibfield  {title} {\enquote {\bibinfo {title} {Random numbers
  certified by {Bell}'s theorem},}\ }\href {\doibase 10.1038/nature09008}
  {\bibfield  {journal} {\bibinfo  {journal} {Nature}\ }\textbf {\bibinfo
  {volume} {464}}~(\bibinfo {number} {7291}),\ \bibinfo {pages} {1021--1024}},\
  \Eprint {http://arxiv.org/abs/arXiv:0911.3427} {arXiv:0911.3427} \BibitemShut
  {NoStop}%
\bibitem [{\citenamefont {Portmann}(2014)}]{Por14}%
  \BibitemOpen
  \bibfield  {author} {\bibinfo {author} {\bibnamefont {Portmann},
  \bibfnamefont {Christopher}}} (\bibinfo {year} {2014}),\ \bibfield  {title}
  {\enquote {\bibinfo {title} {Key recycling in authentication},}\ }\href
  {\doibase 10.1109/TIT.2014.2317312} {\bibfield  {journal} {\bibinfo
  {journal} {IEEE Trans. Inf. Theory}\ }\textbf {\bibinfo {volume}
  {60}}~(\bibinfo {number} {7}),\ \bibinfo {pages} {4383--4396}},\ \Eprint
  {http://arxiv.org/abs/arXiv:1202.1229} {arXiv:1202.1229} \BibitemShut
  {NoStop}%
\bibitem [{\citenamefont {Portmann}(2017{\natexlab{a}})}]{Por17}%
  \BibitemOpen
  \bibfield  {author} {\bibinfo {author} {\bibnamefont {Portmann},
  \bibfnamefont {Christopher}}} (\bibinfo {year} {2017}{\natexlab{a}}),\
  \bibfield  {title} {\enquote {\bibinfo {title} {Quantum authentication with
  key recycling},}\ }in\ \href {\doibase 10.1007/978-3-319-56617-7_12} {\emph
  {\bibinfo {booktitle} {Advances in Cryptology -- {EUROCRYPT} 2017,
  Proceedings, Part {III}}}},\ \bibinfo {series} {LNCS}, Vol.\ \bibinfo
  {volume} {10212}\ (\bibinfo  {publisher} {Springer})\ pp.\ \bibinfo {pages}
  {339--368},\ \Eprint {http://arxiv.org/abs/arXiv:1610.03422}
  {arXiv:1610.03422} \BibitemShut {NoStop}%
\bibitem [{\citenamefont {Portmann}(2017{\natexlab{b}})}]{Portmann2017}%
  \BibitemOpen
  \bibfield  {author} {\bibinfo {author} {\bibnamefont {Portmann},
  \bibfnamefont {Christopher}}} (\bibinfo {year} {2017}{\natexlab{b}}),\
  \href@noop {} {\enquote {\bibinfo {title} {({Quantum}) {Min}-entropy
  resources},}\ }\bibinfo {howpublished} {e-Print},\ \Eprint
  {http://arxiv.org/abs/arXiv:1705.10595} {arXiv:1705.10595} \BibitemShut
  {NoStop}%
\bibitem [{\citenamefont {Portmann}\ \emph {et~al.}(2017)\citenamefont
  {Portmann}, \citenamefont {Matt}, \citenamefont {Maurer}, \citenamefont
  {Renner},\ and\ \citenamefont {Tackmann}}]{PMMRT17}%
  \BibitemOpen
  \bibfield  {author} {\bibinfo {author} {\bibnamefont {Portmann},
  \bibfnamefont {Christopher}}, \bibinfo {author} {\bibfnamefont {Christian}\
  \bibnamefont {Matt}}, \bibinfo {author} {\bibfnamefont {Ueli}\ \bibnamefont
  {Maurer}}, \bibinfo {author} {\bibfnamefont {Renato}\ \bibnamefont {Renner}},
  \ and\ \bibinfo {author} {\bibfnamefont {Bj\"orn}\ \bibnamefont {Tackmann}}}
  (\bibinfo {year} {2017}),\ \bibfield  {title} {\enquote {\bibinfo {title}
  {Causal boxes: Quantum information-processing systems closed under
  composition},}\ }\href {\doibase 10.1109/TIT.2017.2676805} {\bibfield
  {journal} {\bibinfo  {journal} {IEEE Transactions on Information Theory}\
  }\textbf {\bibinfo {volume} {63}}~(\bibinfo {number} {5}),\ \bibinfo {pages}
  {3277--3305}},\ \Eprint {http://arxiv.org/abs/arXiv:1512.02240}
  {arXiv:1512.02240} \BibitemShut {NoStop}%
\bibitem [{\citenamefont {Prokop}(2020)}]{Pro20}%
  \BibitemOpen
  \bibfield  {author} {\bibinfo {author} {\bibnamefont {Prokop}, \bibfnamefont
  {Mil\v{o}s}}} (\bibinfo {year} {2020}),\ \href
  {https://project-archive.inf.ed.ac.uk/ug4/20201685/ug4_proj.pdf} {\enquote
  {\bibinfo {title} {Composable security of quantum bit commitment protocol},}\
  }\bibinfo {howpublished} {e-print}\BibitemShut {NoStop}%
\bibitem [{\citenamefont {Qi}\ \emph {et~al.}(2007)\citenamefont {Qi},
  \citenamefont {Fung}, \citenamefont {Lo},\ and\ \citenamefont
  {Ma}}]{qi2007time}%
  \BibitemOpen
  \bibfield  {author} {\bibinfo {author} {\bibnamefont {Qi}, \bibfnamefont
  {Bing}}, \bibinfo {author} {\bibfnamefont {Chi-Hang~Fred}\ \bibnamefont
  {Fung}}, \bibinfo {author} {\bibfnamefont {Hoi-Kwong}\ \bibnamefont {Lo}}, \
  and\ \bibinfo {author} {\bibfnamefont {Xiongfeng}\ \bibnamefont {Ma}}}
  (\bibinfo {year} {2007}),\ \bibfield  {title} {\enquote {\bibinfo {title}
  {Time-shift attack in practical quantum cryptosystems},}\ }\href@noop {}
  {\bibfield  {journal} {\bibinfo  {journal} {Quantum Inf. Comput.}\ }\textbf
  {\bibinfo {volume} {7}}~(\bibinfo {number} {1}),\ \bibinfo {pages}
  {73--82}}\BibitemShut {NoStop}%
\bibitem [{\citenamefont {Reichardt}\ \emph {et~al.}(2013)\citenamefont
  {Reichardt}, \citenamefont {Unger},\ and\ \citenamefont {Vazirani}}]{RUV13}%
  \BibitemOpen
  \bibfield  {author} {\bibinfo {author} {\bibnamefont {Reichardt},
  \bibfnamefont {Ben~W}}, \bibinfo {author} {\bibfnamefont {Falk}\ \bibnamefont
  {Unger}}, \ and\ \bibinfo {author} {\bibfnamefont {Umesh}\ \bibnamefont
  {Vazirani}}} (\bibinfo {year} {2013}),\ \bibfield  {title} {\enquote
  {\bibinfo {title} {Classical command of quantum systems},}\ }\href {\doibase
  10.1038/nature12035} {\bibfield  {journal} {\bibinfo  {journal} {Nature}\
  }\textbf {\bibinfo {volume} {496}},\ \bibinfo {pages} {456--460}},\ \bibinfo
  {note} {full version available on arXiv},\ \Eprint
  {http://arxiv.org/abs/arXiv:1209.0448} {arXiv:1209.0448} \BibitemShut
  {NoStop}%
\bibitem [{\citenamefont {Renes}(2013)}]{Renes2013}%
  \BibitemOpen
  \bibfield  {author} {\bibinfo {author} {\bibnamefont {Renes}, \bibfnamefont
  {Joseph~M}}} (\bibinfo {year} {2013}),\ \bibfield  {title} {\enquote
  {\bibinfo {title} {The physics of quantum information: Complementarity,
  uncertainty, and entanglement},}\ }\href@noop {} {\bibfield  {journal}
  {\bibinfo  {journal} {Int. J. Quantum Inf.}\ }\textbf {\bibinfo {volume}
  {11}}~(\bibinfo {number} {08}),\ \bibinfo {pages} {1330002}}\BibitemShut
  {NoStop}%
\bibitem [{\citenamefont {Renes}\ and\ \citenamefont
  {Renner}(2012)}]{RenRen12}%
  \BibitemOpen
  \bibfield  {author} {\bibinfo {author} {\bibnamefont {Renes}, \bibfnamefont
  {Joseph~M}}, \ and\ \bibinfo {author} {\bibfnamefont {Renato}\ \bibnamefont
  {Renner}}} (\bibinfo {year} {2012}),\ \bibfield  {title} {\enquote {\bibinfo
  {title} {One-shot classical data compression with quantum side information
  and the distillation of common randomness or secret keys},}\ }\href {\doibase
  10.1109/TIT.2011.2177589} {\bibfield  {journal} {\bibinfo  {journal} {IEEE
  Trans. Inf. Theory}\ }\textbf {\bibinfo {volume} {58}}~(\bibinfo {number}
  {3}),\ \bibinfo {pages} {1985--1991}}\BibitemShut {NoStop}%
\bibitem [{\citenamefont {Renes}\ and\ \citenamefont
  {Renner}(2020)}]{RenesRenner2020}%
  \BibitemOpen
  \bibfield  {author} {\bibinfo {author} {\bibnamefont {Renes}, \bibfnamefont
  {Joseph~M}}, \ and\ \bibinfo {author} {\bibfnamefont {Renato}\ \bibnamefont
  {Renner}}} (\bibinfo {year} {2020}),\ \href@noop {} {\enquote {\bibinfo
  {title} {Are quantum cryptographic security claims vacuous?}}\ }\bibinfo
  {howpublished} {e-Print},\ \Eprint {http://arxiv.org/abs/arXiv:2010.11961}
  {arXiv:2010.11961} \BibitemShut {NoStop}%
\bibitem [{\citenamefont {Renner}(2005)}]{Ren05}%
  \BibitemOpen
  \bibfield  {author} {\bibinfo {author} {\bibnamefont {Renner}, \bibfnamefont
  {Renato}}} (\bibinfo {year} {2005}),\ \emph {\bibinfo {title} {Security of
  Quantum Key Distribution}},\ \href@noop {} {Ph.D. thesis}\ (\bibinfo
  {school} {Swiss Federal Institute of Technology (ETH) Zurich}),\ \Eprint
  {http://arxiv.org/abs/arXiv:quant-ph/0512258} {arXiv:quant-ph/0512258}
  \BibitemShut {NoStop}%
\bibitem [{\citenamefont {Renner}(2007)}]{Ren07}%
  \BibitemOpen
  \bibfield  {author} {\bibinfo {author} {\bibnamefont {Renner}, \bibfnamefont
  {Renato}}} (\bibinfo {year} {2007}),\ \bibfield  {title} {\enquote {\bibinfo
  {title} {Symmetry of large physical systems implies independence of
  subsystems},}\ }\href {\doibase 10.1038/nphys684} {\bibfield  {journal}
  {\bibinfo  {journal} {Nat. Phys.}\ }\textbf {\bibinfo {volume} {3}}~(\bibinfo
  {number} {9}),\ \bibinfo {pages} {645--649}},\ \Eprint
  {http://arxiv.org/abs/arXiv:quant-ph/0703069} {arXiv:quant-ph/0703069}
  \BibitemShut {NoStop}%
\bibitem [{\citenamefont {Renner}\ \emph {et~al.}(2005)\citenamefont {Renner},
  \citenamefont {Gisin},\ and\ \citenamefont {Kraus}}]{RGK05}%
  \BibitemOpen
  \bibfield  {author} {\bibinfo {author} {\bibnamefont {Renner}, \bibfnamefont
  {Renato}}, \bibinfo {author} {\bibfnamefont {Nicolas}\ \bibnamefont {Gisin}},
  \ and\ \bibinfo {author} {\bibfnamefont {Barbara}\ \bibnamefont {Kraus}}}
  (\bibinfo {year} {2005}),\ \bibfield  {title} {\enquote {\bibinfo {title}
  {Information-theoretic security proof for quantum-key-distribution
  protocols},}\ }\href {\doibase 10.1103/PhysRevA.72.012332} {\bibfield
  {journal} {\bibinfo  {journal} {Phys. Rev. A}\ }\textbf {\bibinfo {volume}
  {72}},\ \bibinfo {pages} {012332}},\ \Eprint
  {http://arxiv.org/abs/arXiv:quant-ph/0502064} {arXiv:quant-ph/0502064}
  \BibitemShut {NoStop}%
\bibitem [{\citenamefont {Renner}\ and\ \citenamefont {K\"onig}(2005)}]{RK05}%
  \BibitemOpen
  \bibfield  {author} {\bibinfo {author} {\bibnamefont {Renner}, \bibfnamefont
  {Renato}}, \ and\ \bibinfo {author} {\bibfnamefont {Robert}\ \bibnamefont
  {K\"onig}}} (\bibinfo {year} {2005}),\ \bibfield  {title} {\enquote {\bibinfo
  {title} {Universally composable privacy amplification against quantum
  adversaries},}\ }in\ \href {\doibase 10.1007/978-3-540-30576-7_22} {\emph
  {\bibinfo {booktitle} {Theory of Cryptography, Proceedings of TCC 2005}}},\
  \bibinfo {series} {LNCS}, Vol.\ \bibinfo {volume} {3378},\ \bibinfo {editor}
  {edited by\ \bibinfo {editor} {\bibfnamefont {Joe}\ \bibnamefont {Kilian}}}\
  (\bibinfo  {publisher} {Springer})\ pp.\ \bibinfo {pages} {407--425},\
  \Eprint {http://arxiv.org/abs/arXiv:quant-ph/0403133}
  {arXiv:quant-ph/0403133} \BibitemShut {NoStop}%
\bibitem [{\citenamefont {Renner}\ and\ \citenamefont {Wolf}(2003)}]{RW03}%
  \BibitemOpen
  \bibfield  {author} {\bibinfo {author} {\bibnamefont {Renner}, \bibfnamefont
  {Renato}}, \ and\ \bibinfo {author} {\bibfnamefont {Stefan}\ \bibnamefont
  {Wolf}}} (\bibinfo {year} {2003}),\ \bibfield  {title} {\enquote {\bibinfo
  {title} {Unconditional authenticity and privacy from an arbitrarily weak
  secret},}\ }in\ \href {\doibase 10.1007/978-3-540-45146-4_5} {\emph {\bibinfo
  {booktitle} {Advances in Cryptology -- CRYPTO 2003}}},\ \bibinfo {series}
  {LNCS}, Vol.\ \bibinfo {volume} {2729}\ (\bibinfo  {publisher} {Springer})\
  pp.\ \bibinfo {pages} {78--95}\BibitemShut {NoStop}%
\bibitem [{\citenamefont {Renner}\ and\ \citenamefont {Wolf}(2005)}]{RW05}%
  \BibitemOpen
  \bibfield  {author} {\bibinfo {author} {\bibnamefont {Renner}, \bibfnamefont
  {Renato}}, \ and\ \bibinfo {author} {\bibfnamefont {Stefan}\ \bibnamefont
  {Wolf}}} (\bibinfo {year} {2005}),\ \bibfield  {title} {\enquote {\bibinfo
  {title} {Simple and tight bounds for information reconciliation and privacy
  amplification},}\ }in\ \href {\doibase 10.1007/11593447_11} {\emph {\bibinfo
  {booktitle} {Advances in Cryptology -- ASIACRYPT 2005}}},\ \bibinfo {series}
  {LNCS}, Vol.\ \bibinfo {volume} {3788},\ \bibinfo {editor} {edited by\
  \bibinfo {editor} {\bibfnamefont {Bimal}\ \bibnamefont {Roy}}}\ (\bibinfo
  {publisher} {Springer})\ pp.\ \bibinfo {pages} {199--216}\BibitemShut
  {NoStop}%
\bibitem [{\citenamefont {Renner}\ and\ \citenamefont {Cirac}(2009)}]{RC09}%
  \BibitemOpen
  \bibfield  {author} {\bibinfo {author} {\bibnamefont {Renner}, \bibfnamefont
  {Renner}}, \ and\ \bibinfo {author} {\bibfnamefont {J.~Ignacio}\ \bibnamefont
  {Cirac}}} (\bibinfo {year} {2009}),\ \bibfield  {title} {\enquote {\bibinfo
  {title} {{de Finetti} representation theorem for infinite-dimensional quantum
  systems and applications to quantum cryptography},}\ }\href {\doibase
  10.1103/PhysRevLett.102.110504} {\bibfield  {journal} {\bibinfo  {journal}
  {Phys. Rev. Lett.}\ }\textbf {\bibinfo {volume} {102}},\ \bibinfo {pages}
  {110504}},\ \Eprint {http://arxiv.org/abs/arXiv:0809.2243} {arXiv:0809.2243}
  \BibitemShut {NoStop}%
\bibitem [{\citenamefont {Rivest}\ \emph {et~al.}(1978)\citenamefont {Rivest},
  \citenamefont {Shamir},\ and\ \citenamefont {Adleman}}]{RSA78}%
  \BibitemOpen
  \bibfield  {author} {\bibinfo {author} {\bibnamefont {Rivest}, \bibfnamefont
  {Ronald~L}}, \bibinfo {author} {\bibfnamefont {Adi}\ \bibnamefont {Shamir}},
  \ and\ \bibinfo {author} {\bibfnamefont {Leonard}\ \bibnamefont {Adleman}}}
  (\bibinfo {year} {1978}),\ \bibfield  {title} {\enquote {\bibinfo {title} {A
  method for obtaining digital signatures and public-key cryptosystems},}\
  }\href@noop {} {\bibfield  {journal} {\bibinfo  {journal} {Commun. ACM}\
  }\textbf {\bibinfo {volume} {21}}~(\bibinfo {number} {2}),\ \bibinfo {pages}
  {120--126}}\BibitemShut {NoStop}%
\bibitem [{\citenamefont {Rogaway}(2006)}]{Rog06}%
  \BibitemOpen
  \bibfield  {author} {\bibinfo {author} {\bibnamefont {Rogaway}, \bibfnamefont
  {Phillip}}} (\bibinfo {year} {2006}),\ \bibfield  {title} {\enquote {\bibinfo
  {title} {Formalizing human ignorance},}\ }in\ \href {\doibase
  10.1007/11958239_14} {\emph {\bibinfo {booktitle} {Progress in Cryptology --
  VIETCRYPT 2006}}},\ \bibinfo {series} {LNCS}, Vol.\ \bibinfo {volume} {4341}\
  (\bibinfo  {publisher} {Springer})\ pp.\ \bibinfo {pages} {211--228},\
  \bibinfo {note} {e-Print \href{http://eprint.iacr.org/2006/281}{IACR
  2006/281}}\BibitemShut {NoStop}%
\bibitem [{\citenamefont {Rosenfeld}\ \emph {et~al.}(2017)\citenamefont
  {Rosenfeld}, \citenamefont {Burchardt}, \citenamefont {Garthoff},
  \citenamefont {Redeker}, \citenamefont {Ortegel}, \citenamefont {Rau},\ and\
  \citenamefont {Weinfurter}}]{Rosenfeld}%
  \BibitemOpen
  \bibfield  {author} {\bibinfo {author} {\bibnamefont {Rosenfeld},
  \bibfnamefont {Wenjamin}}, \bibinfo {author} {\bibfnamefont {Daniel}\
  \bibnamefont {Burchardt}}, \bibinfo {author} {\bibfnamefont {Robert}\
  \bibnamefont {Garthoff}}, \bibinfo {author} {\bibfnamefont {Kai}\
  \bibnamefont {Redeker}}, \bibinfo {author} {\bibfnamefont {Norbert}\
  \bibnamefont {Ortegel}}, \bibinfo {author} {\bibfnamefont {Markus}\
  \bibnamefont {Rau}}, \ and\ \bibinfo {author} {\bibfnamefont {Harald}\
  \bibnamefont {Weinfurter}}} (\bibinfo {year} {2017}),\ \bibfield  {title}
  {\enquote {\bibinfo {title} {Event-ready bell test using entangled atoms
  simultaneously closing detection and locality loopholes},}\ }\href {\doibase
  10.1103/PhysRevLett.119.010402} {\bibfield  {journal} {\bibinfo  {journal}
  {Phys. Rev. Lett.}\ }\textbf {\bibinfo {volume} {119}},\ \bibinfo {pages}
  {010402}}\BibitemShut {NoStop}%
\bibitem [{\citenamefont {Rowe}\ \emph {et~al.}(2001)\citenamefont {Rowe},
  \citenamefont {Kielpinski}, \citenamefont {Meyer}, \citenamefont {Sackett},
  \citenamefont {Itano}, \citenamefont {Monroe},\ and\ \citenamefont
  {Wineland}}]{Rowe}%
  \BibitemOpen
  \bibfield  {author} {\bibinfo {author} {\bibnamefont {Rowe}, \bibfnamefont
  {M~A}}, \bibinfo {author} {\bibfnamefont {David}\ \bibnamefont {Kielpinski}},
  \bibinfo {author} {\bibfnamefont {V.}~\bibnamefont {Meyer}}, \bibinfo
  {author} {\bibfnamefont {Charles~A.}\ \bibnamefont {Sackett}}, \bibinfo
  {author} {\bibfnamefont {Wayne~M.}\ \bibnamefont {Itano}}, \bibinfo {author}
  {\bibfnamefont {C.}~\bibnamefont {Monroe}}, \ and\ \bibinfo {author}
  {\bibfnamefont {D.~J.}\ \bibnamefont {Wineland}}} (\bibinfo {year} {2001}),\
  \bibfield  {title} {\enquote {\bibinfo {title} {Experimental violation of a
  bell's inequality with efficient detection},}\ }\href {\doibase
  10.1038/35057215} {\bibfield  {journal} {\bibinfo  {journal} {Nature}\
  }\textbf {\bibinfo {volume} {409}}~(\bibinfo {number} {6822}),\ \bibinfo
  {pages} {791--794}}\BibitemShut {NoStop}%
\bibitem [{\citenamefont {Sasaki}\ \emph {et~al.}(2014)\citenamefont {Sasaki},
  \citenamefont {Yamamoto},\ and\ \citenamefont {Koashi}}]{SYK14}%
  \BibitemOpen
  \bibfield  {author} {\bibinfo {author} {\bibnamefont {Sasaki}, \bibfnamefont
  {Toshihiko}}, \bibinfo {author} {\bibfnamefont {Yoshihisa}\ \bibnamefont
  {Yamamoto}}, \ and\ \bibinfo {author} {\bibfnamefont {Masato}\ \bibnamefont
  {Koashi}}} (\bibinfo {year} {2014}),\ \bibfield  {title} {\enquote {\bibinfo
  {title} {Practical quantum key distribution protocol without monitoring
  signal disturbance},}\ }\href {\doibase 10.1038/nature13303} {\bibfield
  {journal} {\bibinfo  {journal} {Nature}\ }\textbf {\bibinfo {volume} {509}},\
  \bibinfo {pages} {475}}\BibitemShut {NoStop}%
\bibitem [{\citenamefont {Scarani}(2013)}]{Sca13}%
  \BibitemOpen
  \bibfield  {author} {\bibinfo {author} {\bibnamefont {Scarani}, \bibfnamefont
  {Valerio}}} (\bibinfo {year} {2013}),\ \href@noop {} {\enquote {\bibinfo
  {title} {The device-independent outlook on quantum physics (lecture notes on
  the power of {Bell}'s theorem)},}\ }\bibinfo {howpublished} {e-Print},\
  \Eprint {http://arxiv.org/abs/arXiv:1303.3081} {arXiv:1303.3081} \BibitemShut
  {NoStop}%
\bibitem [{\citenamefont {Scarani}\ \emph {et~al.}(2004)\citenamefont
  {Scarani}, \citenamefont {Ac\'{\i}n}, \citenamefont {Ribordy},\ and\
  \citenamefont {Gisin}}]{SARG}%
  \BibitemOpen
  \bibfield  {author} {\bibinfo {author} {\bibnamefont {Scarani}, \bibfnamefont
  {Valerio}}, \bibinfo {author} {\bibfnamefont {Antonio}\ \bibnamefont
  {Ac\'{\i}n}}, \bibinfo {author} {\bibfnamefont {Gr\'egoire}\ \bibnamefont
  {Ribordy}}, \ and\ \bibinfo {author} {\bibfnamefont {Nicolas}\ \bibnamefont
  {Gisin}}} (\bibinfo {year} {2004}),\ \bibfield  {title} {\enquote {\bibinfo
  {title} {Quantum cryptography protocols robust against photon number
  splitting attacks for weak laser pulse implementations},}\ }\href {\doibase
  10.1103/PhysRevLett.92.057901} {\bibfield  {journal} {\bibinfo  {journal}
  {Phys. Rev. Lett.}\ }\textbf {\bibinfo {volume} {92}},\ \bibinfo {pages}
  {057901}}\BibitemShut {NoStop}%
\bibitem [{\citenamefont {Scarani}\ \emph {et~al.}(2009)\citenamefont
  {Scarani}, \citenamefont {Bechmann-Pasquinucci}, \citenamefont {Cerf},
  \citenamefont {Du\ifmmode~\check{s}\else \v{s}\fi{}ek}, \citenamefont
  {L\"utkenhaus},\ and\ \citenamefont {Peev}}]{SBCDLP09}%
  \BibitemOpen
  \bibfield  {author} {\bibinfo {author} {\bibnamefont {Scarani}, \bibfnamefont
  {Valerio}}, \bibinfo {author} {\bibfnamefont {Helle}\ \bibnamefont
  {Bechmann-Pasquinucci}}, \bibinfo {author} {\bibfnamefont {Nicolas~J.}\
  \bibnamefont {Cerf}}, \bibinfo {author} {\bibfnamefont {Miloslav}\
  \bibnamefont {Du\ifmmode~\check{s}\else \v{s}\fi{}ek}}, \bibinfo {author}
  {\bibfnamefont {Norbert}\ \bibnamefont {L\"utkenhaus}}, \ and\ \bibinfo
  {author} {\bibfnamefont {Momtchil}\ \bibnamefont {Peev}}} (\bibinfo {year}
  {2009}),\ \bibfield  {title} {\enquote {\bibinfo {title} {The security of
  practical quantum key distribution},}\ }\href {\doibase
  10.1103/RevModPhys.81.1301} {\bibfield  {journal} {\bibinfo  {journal} {Rev.
  Mod. Phys.}\ }\textbf {\bibinfo {volume} {81}},\ \bibinfo {pages}
  {1301--1350}},\ \Eprint {http://arxiv.org/abs/arXiv:0802.4155}
  {arXiv:0802.4155} \BibitemShut {NoStop}%
\bibitem [{\citenamefont {Scarani}\ and\ \citenamefont {Renner}(2008)}]{SR08}%
  \BibitemOpen
  \bibfield  {author} {\bibinfo {author} {\bibnamefont {Scarani}, \bibfnamefont
  {Valerio}}, \ and\ \bibinfo {author} {\bibfnamefont {Renato}\ \bibnamefont
  {Renner}}} (\bibinfo {year} {2008}),\ \bibfield  {title} {\enquote {\bibinfo
  {title} {Quantum cryptography with finite resources: Unconditional security
  bound for discrete-variable protocols with one-way postprocessing},}\ }\href
  {\doibase 10.1103/PhysRevLett.100.200501} {\bibfield  {journal} {\bibinfo
  {journal} {Phys. Rev. Lett.}\ }\textbf {\bibinfo {volume} {100}},\ \bibinfo
  {pages} {200501}},\ \Eprint {http://arxiv.org/abs/arXiv:0708.0709}
  {arXiv:0708.0709} \BibitemShut {NoStop}%
\bibitem [{\citenamefont {Schaffner}\ \emph {et~al.}(2009)\citenamefont
  {Schaffner}, \citenamefont {Terhal},\ and\ \citenamefont {Wehner}}]{STW09}%
  \BibitemOpen
  \bibfield  {author} {\bibinfo {author} {\bibnamefont {Schaffner},
  \bibfnamefont {Christian}}, \bibinfo {author} {\bibfnamefont {Barbara}\
  \bibnamefont {Terhal}}, \ and\ \bibinfo {author} {\bibfnamefont {Stephanie}\
  \bibnamefont {Wehner}}} (\bibinfo {year} {2009}),\ \bibfield  {title}
  {\enquote {\bibinfo {title} {Robust cryptography in the noisy-quantum-storage
  model},}\ }\href@noop {} {\bibfield  {journal} {\bibinfo  {journal} {Quantum
  Inf. Comput.}\ }\textbf {\bibinfo {volume} {9}}~(\bibinfo {number} {11}),\
  \bibinfo {pages} {963--996}},\ \Eprint {http://arxiv.org/abs/arXiv:0807.1333}
  {arXiv:0807.1333} \BibitemShut {NoStop}%
\bibitem [{\citenamefont {Seiler}\ and\ \citenamefont {Maurer}(2016)}]{SM16}%
  \BibitemOpen
  \bibfield  {author} {\bibinfo {author} {\bibnamefont {Seiler}, \bibfnamefont
  {Gregor}}, \ and\ \bibinfo {author} {\bibfnamefont {Ueli}\ \bibnamefont
  {Maurer}}} (\bibinfo {year} {2016}),\ \bibfield  {title} {\enquote {\bibinfo
  {title} {On the impossibility of information-theoretic composable coin toss
  extension},}\ }in\ \href {\doibase 10.1109/ISIT.2016.7541861} {\emph
  {\bibinfo {booktitle} {Proceedings of the 2016 IEEE International Symposium
  on Information Theory, ISIT 2016}}}\ (\bibinfo  {publisher} {IEEE})\ pp.\
  \bibinfo {pages} {3058--3061}\BibitemShut {NoStop}%
\bibitem [{\citenamefont {Shalm}\ \emph {et~al.}(2015)\citenamefont {Shalm},
  \citenamefont {Meyer-Scott}, \citenamefont {Christensen}, \citenamefont
  {Bierhorst}, \citenamefont {Wayne}, \citenamefont {Stevens}, \citenamefont
  {Gerrits}, \citenamefont {Glancy}, \citenamefont {Hamel}, \citenamefont
  {Allman}, \citenamefont {Coakley}, \citenamefont {Dyer}, \citenamefont
  {Hodge}, \citenamefont {Lita}, \citenamefont {Verma}, \citenamefont
  {Lambrocco}, \citenamefont {Tortorici}, \citenamefont {Migdall},
  \citenamefont {Zhang}, \citenamefont {Kumor}, \citenamefont {Farr},
  \citenamefont {Marsili}, \citenamefont {Shaw}, \citenamefont {Stern},
  \citenamefont {Abell\'an}, \citenamefont {Amaya}, \citenamefont {Pruneri},
  \citenamefont {Jennewein}, \citenamefont {Mitchell}, \citenamefont {Kwiat},
  \citenamefont {Bienfang}, \citenamefont {Mirin}, \citenamefont {Knill},\ and\
  \citenamefont {Nam}}]{Shalm}%
  \BibitemOpen
  \bibfield  {author} {\bibinfo {author} {\bibnamefont {Shalm}, \bibfnamefont
  {Lynden~K}}, \bibinfo {author} {\bibfnamefont {Evan}\ \bibnamefont
  {Meyer-Scott}}, \bibinfo {author} {\bibfnamefont {Bradley~G.}\ \bibnamefont
  {Christensen}}, \bibinfo {author} {\bibfnamefont {Peter}\ \bibnamefont
  {Bierhorst}}, \bibinfo {author} {\bibfnamefont {Michael~A.}\ \bibnamefont
  {Wayne}}, \bibinfo {author} {\bibfnamefont {Martin~J.}\ \bibnamefont
  {Stevens}}, \bibinfo {author} {\bibfnamefont {Thomas}\ \bibnamefont
  {Gerrits}}, \bibinfo {author} {\bibfnamefont {Scott}\ \bibnamefont {Glancy}},
  \bibinfo {author} {\bibfnamefont {Deny~R.}\ \bibnamefont {Hamel}}, \bibinfo
  {author} {\bibfnamefont {Michael~S.}\ \bibnamefont {Allman}}, \bibinfo
  {author} {\bibfnamefont {Kevin~J.}\ \bibnamefont {Coakley}}, \bibinfo
  {author} {\bibfnamefont {Shellee~D.}\ \bibnamefont {Dyer}}, \bibinfo {author}
  {\bibfnamefont {Carson}\ \bibnamefont {Hodge}}, \bibinfo {author}
  {\bibfnamefont {Adriana~E.}\ \bibnamefont {Lita}}, \bibinfo {author}
  {\bibfnamefont {Varun~B.}\ \bibnamefont {Verma}}, \bibinfo {author}
  {\bibfnamefont {Camilla}\ \bibnamefont {Lambrocco}}, \bibinfo {author}
  {\bibfnamefont {Edward}\ \bibnamefont {Tortorici}}, \bibinfo {author}
  {\bibfnamefont {Alan~L.}\ \bibnamefont {Migdall}}, \bibinfo {author}
  {\bibfnamefont {Yanbao}\ \bibnamefont {Zhang}}, \bibinfo {author}
  {\bibfnamefont {Daniel~R.}\ \bibnamefont {Kumor}}, \bibinfo {author}
  {\bibfnamefont {William~H.}\ \bibnamefont {Farr}}, \bibinfo {author}
  {\bibfnamefont {Francesco}\ \bibnamefont {Marsili}}, \bibinfo {author}
  {\bibfnamefont {Matthew~D.}\ \bibnamefont {Shaw}}, \bibinfo {author}
  {\bibfnamefont {Jeffrey~A.}\ \bibnamefont {Stern}}, \bibinfo {author}
  {\bibfnamefont {Carlos}\ \bibnamefont {Abell\'an}}, \bibinfo {author}
  {\bibfnamefont {Waldimar}\ \bibnamefont {Amaya}}, \bibinfo {author}
  {\bibfnamefont {Valerio}\ \bibnamefont {Pruneri}}, \bibinfo {author}
  {\bibfnamefont {Thomas}\ \bibnamefont {Jennewein}}, \bibinfo {author}
  {\bibfnamefont {Morgan~W.}\ \bibnamefont {Mitchell}}, \bibinfo {author}
  {\bibfnamefont {Paul~G.}\ \bibnamefont {Kwiat}}, \bibinfo {author}
  {\bibfnamefont {Joshua~C.}\ \bibnamefont {Bienfang}}, \bibinfo {author}
  {\bibfnamefont {Richard~P.}\ \bibnamefont {Mirin}}, \bibinfo {author}
  {\bibfnamefont {Emanuel}\ \bibnamefont {Knill}}, \ and\ \bibinfo {author}
  {\bibfnamefont {Sae~Woo}\ \bibnamefont {Nam}}} (\bibinfo {year} {2015}),\
  \bibfield  {title} {\enquote {\bibinfo {title} {Strong loophole-free test of
  local realism},}\ }\href {\doibase 10.1103/PhysRevLett.115.250402} {\bibfield
   {journal} {\bibinfo  {journal} {Phys. Rev. Lett.}\ }\textbf {\bibinfo
  {volume} {115}},\ \bibinfo {pages} {250402}}\BibitemShut {NoStop}%
\bibitem [{\citenamefont {Shaltiel}(2004)}]{Shaltiel04}%
  \BibitemOpen
  \bibfield  {author} {\bibinfo {author} {\bibnamefont {Shaltiel},
  \bibfnamefont {Ronen}}} (\bibinfo {year} {2004}),\ \bibfield  {title}
  {\enquote {\bibinfo {title} {Recent developments in explicit constructions of
  extractors},}\ }in\ \href {\doibase 10.1142/9789812562494_0013} {\emph
  {\bibinfo {booktitle} {Current Trends in Theoretical Computer Science: The
  Challenge of the New Century, Vol 1: Algorithms and Complexity}}}\ (\bibinfo
  {publisher} {World Scientific})\ pp.\ \bibinfo {pages} {189--228}\BibitemShut
  {NoStop}%
\bibitem [{\citenamefont {Shannon}(1949)}]{Shannon49}%
  \BibitemOpen
  \bibfield  {author} {\bibinfo {author} {\bibnamefont {Shannon}, \bibfnamefont
  {Claude~E}}} (\bibinfo {year} {1949}),\ \bibfield  {title} {\enquote
  {\bibinfo {title} {Communication theory of secrecy systems},}\ }\href@noop {}
  {\bibfield  {journal} {\bibinfo  {journal} {Bell system technical journal}\
  }\textbf {\bibinfo {volume} {28}}~(\bibinfo {number} {4}),\ \bibinfo {pages}
  {656--715}}\BibitemShut {NoStop}%
\bibitem [{\citenamefont {Sheridan}\ \emph {et~al.}(2010)\citenamefont
  {Sheridan}, \citenamefont {Thinh},\ and\ \citenamefont {Scarani}}]{STS10}%
  \BibitemOpen
  \bibfield  {author} {\bibinfo {author} {\bibnamefont {Sheridan},
  \bibfnamefont {Lana}}, \bibinfo {author} {\bibfnamefont {Phuc~Le}\
  \bibnamefont {Thinh}}, \ and\ \bibinfo {author} {\bibfnamefont {Valerio}\
  \bibnamefont {Scarani}}} (\bibinfo {year} {2010}),\ \bibfield  {title}
  {\enquote {\bibinfo {title} {Finite-key security against coherent attacks in
  quantum key distribution},}\ }\href {\doibase 10.1088/1367-2630/12/12/123019}
  {\bibfield  {journal} {\bibinfo  {journal} {New J. Phys.}\ }\textbf {\bibinfo
  {volume} {12}}~(\bibinfo {number} {12}),\ \bibinfo {pages} {123019}},\
  \Eprint {http://arxiv.org/abs/arXiv:1008.2596} {arXiv:1008.2596} \BibitemShut
  {NoStop}%
\bibitem [{\citenamefont {Shor}(1997)}]{Shor97}%
  \BibitemOpen
  \bibfield  {author} {\bibinfo {author} {\bibnamefont {Shor}, \bibfnamefont
  {Peter~W}}} (\bibinfo {year} {1997}),\ \bibfield  {title} {\enquote {\bibinfo
  {title} {Polynomial-time algorithms for prime factorization and discrete
  logarithms on a quantum computer},}\ }\href {\doibase
  10.1137/S0097539795293172} {\bibfield  {journal} {\bibinfo  {journal} {SIAM
  J. Comput.}\ }\textbf {\bibinfo {volume} {26}}~(\bibinfo {number} {5}),\
  \bibinfo {pages} {1484--1509}}\BibitemShut {NoStop}%
\bibitem [{\citenamefont {Shor}\ and\ \citenamefont {Preskill}(2000)}]{SP00}%
  \BibitemOpen
  \bibfield  {author} {\bibinfo {author} {\bibnamefont {Shor}, \bibfnamefont
  {Peter~W}}, \ and\ \bibinfo {author} {\bibfnamefont {John}\ \bibnamefont
  {Preskill}}} (\bibinfo {year} {2000}),\ \bibfield  {title} {\enquote
  {\bibinfo {title} {Simple proof of security of the {BB84} quantum key
  distribution protocol},}\ }\href {\doibase 10.1103/PhysRevLett.85.441}
  {\bibfield  {journal} {\bibinfo  {journal} {Phys. Rev. Lett.}\ }\textbf
  {\bibinfo {volume} {85}},\ \bibinfo {pages} {441--444}},\ \Eprint
  {http://arxiv.org/abs/arXiv:quant-ph/0003004} {arXiv:quant-ph/0003004}
  \BibitemShut {NoStop}%
\bibitem [{\citenamefont {Simmons}(1985)}]{Sim85}%
  \BibitemOpen
  \bibfield  {author} {\bibinfo {author} {\bibnamefont {Simmons}, \bibfnamefont
  {Gustavus~J}}} (\bibinfo {year} {1985}),\ \bibfield  {title} {\enquote
  {\bibinfo {title} {Authentication theory/coding theory},}\ }in\ \href
  {\doibase 10.1007/3-540-39568-7_32} {\emph {\bibinfo {booktitle} {Advances in
  Cryptology -- CRYPTO~'84}}},\ \bibinfo {series} {LNCS}, Vol.\ \bibinfo
  {volume} {196}\ (\bibinfo  {publisher} {Springer})\ pp.\ \bibinfo {pages}
  {411--431}\BibitemShut {NoStop}%
\bibitem [{\citenamefont {Simmons}(1988)}]{Sim88}%
  \BibitemOpen
  \bibfield  {author} {\bibinfo {author} {\bibnamefont {Simmons}, \bibfnamefont
  {Gustavus~J}}} (\bibinfo {year} {1988}),\ \bibfield  {title} {\enquote
  {\bibinfo {title} {A survey of information authentication},}\ }\href
  {\doibase 10.1109/5.4445} {\bibfield  {journal} {\bibinfo  {journal} {Proc.
  IEEE}\ }\textbf {\bibinfo {volume} {76}}~(\bibinfo {number} {5}),\ \bibinfo
  {pages} {603--620}}\BibitemShut {NoStop}%
\bibitem [{\citenamefont {Steane}(1996)}]{Steane96}%
  \BibitemOpen
  \bibfield  {author} {\bibinfo {author} {\bibnamefont {Steane}, \bibfnamefont
  {Andrew}}} (\bibinfo {year} {1996}),\ \bibfield  {title} {\enquote {\bibinfo
  {title} {Multiple-particle interference and quantum error correction},}\
  }\href@noop {} {\bibfield  {journal} {\bibinfo  {journal} {Proc. R. Soc.
  London, Ser. A}\ }\textbf {\bibinfo {volume} {452}}~(\bibinfo {number}
  {1954}),\ \bibinfo {pages} {2551--2577}}\BibitemShut {NoStop}%
\bibitem [{\citenamefont {Stinson}(1990)}]{Sti90}%
  \BibitemOpen
  \bibfield  {author} {\bibinfo {author} {\bibnamefont {Stinson}, \bibfnamefont
  {Douglas~R}}} (\bibinfo {year} {1990}),\ \bibfield  {title} {\enquote
  {\bibinfo {title} {The combinatorics of authentication and secrecy codes},}\
  }\href {\doibase 10.1007/BF02252868} {\bibfield  {journal} {\bibinfo
  {journal} {J. Crypt.}\ }\textbf {\bibinfo {volume} {2}}~(\bibinfo {number}
  {1}),\ \bibinfo {pages} {23--49}}\BibitemShut {NoStop}%
\bibitem [{\citenamefont {Stinson}(1994)}]{Sti94}%
  \BibitemOpen
  \bibfield  {author} {\bibinfo {author} {\bibnamefont {Stinson}, \bibfnamefont
  {Douglas~R}}} (\bibinfo {year} {1994}),\ \bibfield  {title} {\enquote
  {\bibinfo {title} {Universal hashing and authentication codes},}\ }\href
  {\doibase 10.1007/BF01388651} {\bibfield  {journal} {\bibinfo  {journal}
  {Des. Codes Cryptogr.}\ }\textbf {\bibinfo {volume} {4}}~(\bibinfo {number}
  {3}),\ \bibinfo {pages} {369--380}},\ \bibinfo {note} {a preliminary version
  appeared at CRYPTO~'91}\BibitemShut {NoStop}%
\bibitem [{\citenamefont {Streltsov}\ \emph {et~al.}(2017)\citenamefont
  {Streltsov}, \citenamefont {Adesso},\ and\ \citenamefont {Plenio}}]{SAP17}%
  \BibitemOpen
  \bibfield  {author} {\bibinfo {author} {\bibnamefont {Streltsov},
  \bibfnamefont {Alexander}}, \bibinfo {author} {\bibfnamefont {Gerardo}\
  \bibnamefont {Adesso}}, \ and\ \bibinfo {author} {\bibfnamefont {Martin~B.}\
  \bibnamefont {Plenio}}} (\bibinfo {year} {2017}),\ \bibfield  {title}
  {\enquote {\bibinfo {title} {Colloquium: Quantum coherence as a resource},}\
  }\href {\doibase 10.1103/RevModPhys.89.041003} {\bibfield  {journal}
  {\bibinfo  {journal} {Rev. Mod. Phys.}\ }\textbf {\bibinfo {volume} {89}},\
  \bibinfo {pages} {041003}}\BibitemShut {NoStop}%
\bibitem [{\citenamefont {Stucki}\ \emph {et~al.}(2005)\citenamefont {Stucki},
  \citenamefont {Brunner}, \citenamefont {Gisin}, \citenamefont {Scarani},\
  and\ \citenamefont {Zbinden}}]{SBGSZ05}%
  \BibitemOpen
  \bibfield  {author} {\bibinfo {author} {\bibnamefont {Stucki}, \bibfnamefont
  {Damien}}, \bibinfo {author} {\bibfnamefont {Nicolas}\ \bibnamefont
  {Brunner}}, \bibinfo {author} {\bibfnamefont {Nicolas}\ \bibnamefont
  {Gisin}}, \bibinfo {author} {\bibfnamefont {Valerio}\ \bibnamefont
  {Scarani}}, \ and\ \bibinfo {author} {\bibfnamefont {Hugo}\ \bibnamefont
  {Zbinden}}} (\bibinfo {year} {2005}),\ \bibfield  {title} {\enquote {\bibinfo
  {title} {Fast and simple one-way quantum key distribution},}\ }\href
  {\doibase 10.1063/1.2126792} {\bibfield  {journal} {\bibinfo  {journal}
  {Appl. Phys. Lett.}\ }\textbf {\bibinfo {volume} {87}}~(\bibinfo {number}
  {19}),\ \bibinfo {pages} {194108}},\ \Eprint
  {http://arxiv.org/abs/arXiv:quant-ph/0506097} {arXiv:quant-ph/0506097}
  \BibitemShut {NoStop}%
\bibitem [{\citenamefont {Tamaki}\ \emph {et~al.}(2003)\citenamefont {Tamaki},
  \citenamefont {Koashi},\ and\ \citenamefont {Imoto}}]{Tamaki03}%
  \BibitemOpen
  \bibfield  {author} {\bibinfo {author} {\bibnamefont {Tamaki}, \bibfnamefont
  {Kiyoshi}}, \bibinfo {author} {\bibfnamefont {Masato}\ \bibnamefont
  {Koashi}}, \ and\ \bibinfo {author} {\bibfnamefont {Nobuyuki}\ \bibnamefont
  {Imoto}}} (\bibinfo {year} {2003}),\ \bibfield  {title} {\enquote {\bibinfo
  {title} {Unconditionally secure key distribution based on two nonorthogonal
  states},}\ }\href {\doibase 10.1103/PhysRevLett.90.167904} {\bibfield
  {journal} {\bibinfo  {journal} {Phys. Rev. Lett.}\ }\textbf {\bibinfo
  {volume} {90}},\ \bibinfo {pages} {167904}}\BibitemShut {NoStop}%
\bibitem [{\citenamefont {Tamaki}\ and\ \citenamefont {Lo}(2006)}]{TamakiLo}%
  \BibitemOpen
  \bibfield  {author} {\bibinfo {author} {\bibnamefont {Tamaki}, \bibfnamefont
  {Kiyoshi}}, \ and\ \bibinfo {author} {\bibfnamefont {Hoi-Kwong}\ \bibnamefont
  {Lo}}} (\bibinfo {year} {2006}),\ \bibfield  {title} {\enquote {\bibinfo
  {title} {Unconditionally secure key distillation from multiphotons},}\ }\href
  {\doibase 10.1103/PhysRevA.73.010302} {\bibfield  {journal} {\bibinfo
  {journal} {Phys. Rev. A}\ }\textbf {\bibinfo {volume} {73}},\ \bibinfo
  {pages} {010302}}\BibitemShut {NoStop}%
\bibitem [{\citenamefont {Tamaki}\ \emph {et~al.}(2012)\citenamefont {Tamaki},
  \citenamefont {Lo}, \citenamefont {Fung},\ and\ \citenamefont
  {Qi}}]{TamakiLo12}%
  \BibitemOpen
  \bibfield  {author} {\bibinfo {author} {\bibnamefont {Tamaki}, \bibfnamefont
  {Kiyoshi}}, \bibinfo {author} {\bibfnamefont {Hoi-Kwong}\ \bibnamefont {Lo}},
  \bibinfo {author} {\bibfnamefont {Chi-Hang~Fred}\ \bibnamefont {Fung}}, \
  and\ \bibinfo {author} {\bibfnamefont {Bing}\ \bibnamefont {Qi}}} (\bibinfo
  {year} {2012}),\ \bibfield  {title} {\enquote {\bibinfo {title} {Phase
  encoding schemes for measurement-device-independent quantum key distribution
  with basis-dependent flaw},}\ }\href {\doibase 10.1103/PhysRevA.85.042307}
  {\bibfield  {journal} {\bibinfo  {journal} {Phys. Rev. A}\ }\textbf {\bibinfo
  {volume} {85}},\ \bibinfo {pages} {042307}}\BibitemShut {NoStop}%
\bibitem [{\citenamefont {Tan}\ \emph {et~al.}(2020)\citenamefont {Tan},
  \citenamefont {Lim},\ and\ \citenamefont {Renner}}]{TLR20}%
  \BibitemOpen
  \bibfield  {author} {\bibinfo {author} {\bibnamefont {Tan}, \bibfnamefont
  {Ernest Y-Z}}, \bibinfo {author} {\bibfnamefont {Charles Ci~Wen}\
  \bibnamefont {Lim}}, \ and\ \bibinfo {author} {\bibfnamefont {Renato}\
  \bibnamefont {Renner}}} (\bibinfo {year} {2020}),\ \bibfield  {title}
  {\enquote {\bibinfo {title} {Advantage distillation for device-independent
  quantum key distribution},}\ }\href {\doibase 10.1103/PhysRevLett.124.020502}
  {\bibfield  {journal} {\bibinfo  {journal} {Phys. Rev. Lett.}\ }\textbf
  {\bibinfo {volume} {124}},\ \bibinfo {pages} {020502}},\ \Eprint
  {http://arxiv.org/abs/arXiv:1903.10535} {arXiv:1903.10535} \BibitemShut
  {NoStop}%
\bibitem [{\citenamefont {Tang}\ \emph {et~al.}(2014)\citenamefont {Tang},
  \citenamefont {Yin}, \citenamefont {Chen}, \citenamefont {Liu}, \citenamefont
  {Zhang}, \citenamefont {Jiang}, \citenamefont {Zhang}, \citenamefont {Wang},
  \citenamefont {You}, \citenamefont {Guan}, \citenamefont {Yang},
  \citenamefont {Wang}, \citenamefont {Liang}, \citenamefont {Zhang},
  \citenamefont {Zhou}, \citenamefont {Ma}, \citenamefont {Chen}, \citenamefont
  {Zhang},\ and\ \citenamefont {Pan}}]{Tang2014}%
  \BibitemOpen
  \bibfield  {author} {\bibinfo {author} {\bibnamefont {Tang}, \bibfnamefont
  {Yan-Lin}}, \bibinfo {author} {\bibfnamefont {Hua-Lei}\ \bibnamefont {Yin}},
  \bibinfo {author} {\bibfnamefont {Si-Jing}\ \bibnamefont {Chen}}, \bibinfo
  {author} {\bibfnamefont {Yang}\ \bibnamefont {Liu}}, \bibinfo {author}
  {\bibfnamefont {Wei-Jun}\ \bibnamefont {Zhang}}, \bibinfo {author}
  {\bibfnamefont {Xiao}\ \bibnamefont {Jiang}}, \bibinfo {author}
  {\bibfnamefont {Lu}~\bibnamefont {Zhang}}, \bibinfo {author} {\bibfnamefont
  {Jian}\ \bibnamefont {Wang}}, \bibinfo {author} {\bibfnamefont {Li-Xing}\
  \bibnamefont {You}}, \bibinfo {author} {\bibfnamefont {Jian-Yu}\ \bibnamefont
  {Guan}}, \bibinfo {author} {\bibfnamefont {Dong-Xu}\ \bibnamefont {Yang}},
  \bibinfo {author} {\bibfnamefont {Zhen}\ \bibnamefont {Wang}}, \bibinfo
  {author} {\bibfnamefont {Hao}\ \bibnamefont {Liang}}, \bibinfo {author}
  {\bibfnamefont {Zhen}\ \bibnamefont {Zhang}}, \bibinfo {author}
  {\bibfnamefont {Nan}\ \bibnamefont {Zhou}}, \bibinfo {author} {\bibfnamefont
  {Xiongfeng}\ \bibnamefont {Ma}}, \bibinfo {author} {\bibfnamefont {Teng-Yun}\
  \bibnamefont {Chen}}, \bibinfo {author} {\bibfnamefont {Qiang}\ \bibnamefont
  {Zhang}}, \ and\ \bibinfo {author} {\bibfnamefont {Jian-Wei}\ \bibnamefont
  {Pan}}} (\bibinfo {year} {2014}),\ \bibfield  {title} {\enquote {\bibinfo
  {title} {Measurement-device-independent quantum key distribution over 200
  km},}\ }\href {\doibase 10.1103/PhysRevLett.113.190501} {\bibfield  {journal}
  {\bibinfo  {journal} {Phys. Rev. Lett.}\ }\textbf {\bibinfo {volume} {113}},\
  \bibinfo {pages} {190501}}\BibitemShut {NoStop}%
\bibitem [{\citenamefont {Terhal}(2004)}]{Terhal04}%
  \BibitemOpen
  \bibfield  {author} {\bibinfo {author} {\bibnamefont {Terhal}, \bibfnamefont
  {Barbara~M}}} (\bibinfo {year} {2004}),\ \bibfield  {title} {\enquote
  {\bibinfo {title} {Is entanglement monogamous?}}\ }\href {\doibase
  10.1147/rd.481.0071} {\bibfield  {journal} {\bibinfo  {journal} {IBM J. Res.
  Dev.}\ }\textbf {\bibinfo {volume} {48}}~(\bibinfo {number} {1}),\ \bibinfo
  {pages} {71--78}}\BibitemShut {NoStop}%
\bibitem [{\citenamefont {Thorisson}(2000)}]{Tho00}%
  \BibitemOpen
  \bibfield  {author} {\bibinfo {author} {\bibnamefont {Thorisson},
  \bibfnamefont {Hermann}}} (\bibinfo {year} {2000}),\ \href@noop {} {\emph
  {\bibinfo {title} {Coupling, Stationarity, and Regeneration}}},\ Probability
  and its Applications (New York)\ (\bibinfo  {publisher}
  {Springer})\BibitemShut {NoStop}%
\bibitem [{\citenamefont {Tittel}\ \emph {et~al.}(1998)\citenamefont {Tittel},
  \citenamefont {Brendel}, \citenamefont {Gisin}, \citenamefont {Herzog},
  \citenamefont {Zbinden},\ and\ \citenamefont {Gisin}}]{Tittel}%
  \BibitemOpen
  \bibfield  {author} {\bibinfo {author} {\bibnamefont {Tittel}, \bibfnamefont
  {Wolfgang}}, \bibinfo {author} {\bibfnamefont {Jurgen}\ \bibnamefont
  {Brendel}}, \bibinfo {author} {\bibfnamefont {Bernard}\ \bibnamefont
  {Gisin}}, \bibinfo {author} {\bibfnamefont {Thomas}\ \bibnamefont {Herzog}},
  \bibinfo {author} {\bibfnamefont {Hugo}\ \bibnamefont {Zbinden}}, \ and\
  \bibinfo {author} {\bibfnamefont {Nicolas}\ \bibnamefont {Gisin}}} (\bibinfo
  {year} {1998}),\ \bibfield  {title} {\enquote {\bibinfo {title} {Experimental
  demonstration of quantum correlations over more than 10 km},}\ }\href
  {\doibase 10.1103/PhysRevA.57.3229} {\bibfield  {journal} {\bibinfo
  {journal} {Phys. Rev. A}\ }\textbf {\bibinfo {volume} {57}},\ \bibinfo
  {pages} {3229--3232}}\BibitemShut {NoStop}%
\bibitem [{\citenamefont {Tomamichel}\ and\ \citenamefont
  {Leverrier}(2017)}]{TL17}%
  \BibitemOpen
  \bibfield  {author} {\bibinfo {author} {\bibnamefont {Tomamichel},
  \bibfnamefont {Marco}}, \ and\ \bibinfo {author} {\bibfnamefont {Anthony}\
  \bibnamefont {Leverrier}}} (\bibinfo {year} {2017}),\ \bibfield  {title}
  {\enquote {\bibinfo {title} {A largely self-contained and complete security
  proof for quantum key distribution},}\ }\href {\doibase
  10.22331/q-2017-07-14-14} {\bibfield  {journal} {\bibinfo  {journal}
  {Quantum}\ }\textbf {\bibinfo {volume} {1}},\ \bibinfo {pages} {14}},\
  \Eprint {http://arxiv.org/abs/arXiv:1506.08458} {arXiv:1506.08458}
  \BibitemShut {NoStop}%
\bibitem [{\citenamefont {Tomamichel}\ \emph {et~al.}(2012)\citenamefont
  {Tomamichel}, \citenamefont {Lim}, \citenamefont {Gisin},\ and\ \citenamefont
  {Renner}}]{TLGR12}%
  \BibitemOpen
  \bibfield  {author} {\bibinfo {author} {\bibnamefont {Tomamichel},
  \bibfnamefont {Marco}}, \bibinfo {author} {\bibfnamefont {Charles Ci~Wen}\
  \bibnamefont {Lim}}, \bibinfo {author} {\bibfnamefont {Nicolas}\ \bibnamefont
  {Gisin}}, \ and\ \bibinfo {author} {\bibfnamefont {Renato}\ \bibnamefont
  {Renner}}} (\bibinfo {year} {2012}),\ \bibfield  {title} {\enquote {\bibinfo
  {title} {Tight finite-key analysis for quantum cryptography},}\ }\href
  {\doibase 10.1038/ncomms1631} {\bibfield  {journal} {\bibinfo  {journal}
  {Nat. Commun.}\ }\textbf {\bibinfo {volume} {3}},\ \bibinfo {pages} {634}},\
  \Eprint {http://arxiv.org/abs/arXiv:1103.4130} {arXiv:1103.4130} \BibitemShut
  {NoStop}%
\bibitem [{\citenamefont {Tomamichel}\ and\ \citenamefont
  {Renner}(2011)}]{TR11}%
  \BibitemOpen
  \bibfield  {author} {\bibinfo {author} {\bibnamefont {Tomamichel},
  \bibfnamefont {Marco}}, \ and\ \bibinfo {author} {\bibfnamefont {Renato}\
  \bibnamefont {Renner}}} (\bibinfo {year} {2011}),\ \bibfield  {title}
  {\enquote {\bibinfo {title} {Uncertainty relation for smooth entropies},}\
  }\href {\doibase 10.1103/PhysRevLett.106.110506} {\bibfield  {journal}
  {\bibinfo  {journal} {Phys. Rev. Lett.}\ }\textbf {\bibinfo {volume} {106}},\
  \bibinfo {pages} {110506}},\ \Eprint {http://arxiv.org/abs/arXiv:1009.2015}
  {arXiv:1009.2015} \BibitemShut {NoStop}%
\bibitem [{\citenamefont {Tomamichel}\ \emph {et~al.}(2010)\citenamefont
  {Tomamichel}, \citenamefont {Schaffner}, \citenamefont {Smith},\ and\
  \citenamefont {Renner}}]{TSSR10}%
  \BibitemOpen
  \bibfield  {author} {\bibinfo {author} {\bibnamefont {Tomamichel},
  \bibfnamefont {Marco}}, \bibinfo {author} {\bibfnamefont {Christian}\
  \bibnamefont {Schaffner}}, \bibinfo {author} {\bibfnamefont {Adam}\
  \bibnamefont {Smith}}, \ and\ \bibinfo {author} {\bibfnamefont {Renato}\
  \bibnamefont {Renner}}} (\bibinfo {year} {2010}),\ \bibfield  {title}
  {\enquote {\bibinfo {title} {Leftover hashing against quantum side
  information},}\ }in\ \href {\doibase 10.1109/ISIT.2010.5513652} {\emph
  {\bibinfo {booktitle} {Proceedings of the 2010 IEEE International Symposium
  on Information Theory, ISIT 2010}}}\ (\bibinfo  {publisher} {IEEE})\ pp.\
  \bibinfo {pages} {2703--2707},\ \Eprint
  {http://arxiv.org/abs/arXiv:1002.2436} {arXiv:1002.2436} \BibitemShut
  {NoStop}%
\bibitem [{\citenamefont {Unruh}(2004)}]{Unr04}%
  \BibitemOpen
  \bibfield  {author} {\bibinfo {author} {\bibnamefont {Unruh}, \bibfnamefont
  {Dominique}}} (\bibinfo {year} {2004}),\ \href@noop {} {\enquote {\bibinfo
  {title} {Simulatable security for quantum protocols},}\ }\bibinfo
  {howpublished} {e-Print},\ \Eprint
  {http://arxiv.org/abs/arXiv:quant-ph/0409125} {arXiv:quant-ph/0409125}
  \BibitemShut {NoStop}%
\bibitem [{\citenamefont {Unruh}(2010)}]{Unr10}%
  \BibitemOpen
  \bibfield  {author} {\bibinfo {author} {\bibnamefont {Unruh}, \bibfnamefont
  {Dominique}}} (\bibinfo {year} {2010}),\ \bibfield  {title} {\enquote
  {\bibinfo {title} {Universally composable quantum multi-party computation},}\
  }in\ \href {\doibase 10.1007/978-3-642-13190-5_25} {\emph {\bibinfo
  {booktitle} {Advances in Cryptology -- EUROCRYPT 2010}}},\ \bibinfo {series}
  {LNCS}, Vol.\ \bibinfo {volume} {6110}\ (\bibinfo  {publisher} {Springer})\
  pp.\ \bibinfo {pages} {486--505},\ \Eprint
  {http://arxiv.org/abs/arXiv:0910.2912} {arXiv:0910.2912} \BibitemShut
  {NoStop}%
\bibitem [{\citenamefont {Unruh}(2011)}]{Unr11}%
  \BibitemOpen
  \bibfield  {author} {\bibinfo {author} {\bibnamefont {Unruh}, \bibfnamefont
  {Dominique}}} (\bibinfo {year} {2011}),\ \bibfield  {title} {\enquote
  {\bibinfo {title} {Concurrent composition in the bounded quantum storage
  model},}\ }in\ \href {\doibase 10.1007/978-3-642-20465-4_26} {\emph {\bibinfo
  {booktitle} {Advances in Cryptology -- EUROCRYPT 2011}}},\ \bibinfo {series}
  {LNCS}, Vol.\ \bibinfo {volume} {6632}\ (\bibinfo  {publisher} {Springer})\
  pp.\ \bibinfo {pages} {467--486},\ \bibinfo {note} {e-Print
  \href{http://eprint.iacr.org/2010/229}{IACR 2010/229}}\BibitemShut {NoStop}%
\bibitem [{\citenamefont {Unruh}(2013)}]{Unr13}%
  \BibitemOpen
  \bibfield  {author} {\bibinfo {author} {\bibnamefont {Unruh}, \bibfnamefont
  {Dominique}}} (\bibinfo {year} {2013}),\ \bibfield  {title} {\enquote
  {\bibinfo {title} {Everlasting multi-party computation},}\ }in\ \href
  {\doibase 10.1007/978-3-642-40084-1_22} {\emph {\bibinfo {booktitle}
  {Advances in Cryptology -- CRYPTO 2013}}},\ \bibinfo {series} {LNCS}, Vol.\
  \bibinfo {volume} {8043}\ (\bibinfo  {publisher} {Springer})\ pp.\ \bibinfo
  {pages} {380--397},\ \bibinfo {note} {e-Print
  \href{http://eprint.iacr.org/2012/177}{IACR 2012/177}}\BibitemShut {NoStop}%
\bibitem [{\citenamefont {Unruh}(2014)}]{Unr14}%
  \BibitemOpen
  \bibfield  {author} {\bibinfo {author} {\bibnamefont {Unruh}, \bibfnamefont
  {Dominique}}} (\bibinfo {year} {2014}),\ \bibfield  {title} {\enquote
  {\bibinfo {title} {Quantum position verification in the random oracle
  model},}\ }in\ \href {\doibase 10.1007/978-3-662-44381-1_1} {\emph {\bibinfo
  {booktitle} {Advances in Cryptology -- CRYPTO 2014}}},\ \bibinfo {series}
  {LNCS}, Vol.\ \bibinfo {volume} {8617}\ (\bibinfo  {publisher} {Springer})\
  pp.\ \bibinfo {pages} {1--18},\ \bibinfo {note} {e-Print
  \href{http://eprint.iacr.org/2014/118}{IACR 2014/118}}\BibitemShut {NoStop}%
\bibitem [{\citenamefont {Vakhitov}\ \emph {et~al.}(2001)\citenamefont
  {Vakhitov}, \citenamefont {Makarov},\ and\ \citenamefont
  {Hjelme}}]{Vakhitov2001}%
  \BibitemOpen
  \bibfield  {author} {\bibinfo {author} {\bibnamefont {Vakhitov},
  \bibfnamefont {Artem}}, \bibinfo {author} {\bibfnamefont {Vadim}\
  \bibnamefont {Makarov}}, \ and\ \bibinfo {author} {\bibfnamefont {Dag~R.}\
  \bibnamefont {Hjelme}}} (\bibinfo {year} {2001}),\ \bibfield  {title}
  {\enquote {\bibinfo {title} {Large pulse attack as a method of conventional
  optical eavesdropping in quantum cryptography},}\ }\href {\doibase
  10.1080/09500340108240904} {\bibfield  {journal} {\bibinfo  {journal} {J.
  Mod. Opt.}\ }\textbf {\bibinfo {volume} {48}}~(\bibinfo {number} {13}),\
  \bibinfo {pages} {2023--2038}}\BibitemShut {NoStop}%
\bibitem [{\citenamefont {Vazirani}\ and\ \citenamefont {Vidick}(2012)}]{VV12}%
  \BibitemOpen
  \bibfield  {author} {\bibinfo {author} {\bibnamefont {Vazirani},
  \bibfnamefont {Umesh}}, \ and\ \bibinfo {author} {\bibfnamefont {Thomas}\
  \bibnamefont {Vidick}}} (\bibinfo {year} {2012}),\ \bibfield  {title}
  {\enquote {\bibinfo {title} {Certifiable quantum dice: or, true random number
  generation secure against quantum adversaries},}\ }in\ \href {\doibase
  10.1145/2213977.2213984} {\emph {\bibinfo {booktitle} {Proceedings of the
  44th Symposium on Theory of Computing, STOC~'12}}}\ (\bibinfo  {publisher}
  {ACM})\ pp.\ \bibinfo {pages} {61--76},\ \Eprint
  {http://arxiv.org/abs/arXiv:1111.6054} {arXiv:1111.6054} \BibitemShut
  {NoStop}%
\bibitem [{\citenamefont {Vazirani}\ and\ \citenamefont {Vidick}(2014)}]{VV14}%
  \BibitemOpen
  \bibfield  {author} {\bibinfo {author} {\bibnamefont {Vazirani},
  \bibfnamefont {Umesh}}, \ and\ \bibinfo {author} {\bibfnamefont {Thomas}\
  \bibnamefont {Vidick}}} (\bibinfo {year} {2014}),\ \bibfield  {title}
  {\enquote {\bibinfo {title} {Fully device-independent quantum key
  distribution},}\ }\href {\doibase 10.1103/PhysRevLett.113.140501} {\bibfield
  {journal} {\bibinfo  {journal} {Phys. Rev. Lett.}\ }\textbf {\bibinfo
  {volume} {113}},\ \bibinfo {pages} {140501}},\ \Eprint
  {http://arxiv.org/abs/arXiv:1210.1810} {arXiv:1210.1810} \BibitemShut
  {NoStop}%
\bibitem [{\citenamefont {Vernam}(1926)}]{Vernam26}%
  \BibitemOpen
  \bibfield  {author} {\bibinfo {author} {\bibnamefont {Vernam}, \bibfnamefont
  {Gilbert~S}}} (\bibinfo {year} {1926}),\ \bibfield  {title} {\enquote
  {\bibinfo {title} {Cipher printing telegraph systems for secret wire and
  radio telegraphic communications},}\ }\href@noop {} {\bibfield  {journal}
  {\bibinfo  {journal} {Trans. Am. Inst. Electr. Eng.}\ }\textbf {\bibinfo
  {volume} {XLV}},\ \bibinfo {pages} {295--301}}\BibitemShut {NoStop}%
\bibitem [{\citenamefont {Vilasini}\ \emph {et~al.}(2019)\citenamefont
  {Vilasini}, \citenamefont {Portmann},\ and\ \citenamefont {del
  Rio}}]{VPdR19}%
  \BibitemOpen
  \bibfield  {author} {\bibinfo {author} {\bibnamefont {Vilasini},
  \bibfnamefont {V}}, \bibinfo {author} {\bibfnamefont {Christopher}\
  \bibnamefont {Portmann}}, \ and\ \bibinfo {author} {\bibfnamefont {L\'idia}\
  \bibnamefont {del Rio}}} (\bibinfo {year} {2019}),\ \bibfield  {title}
  {\enquote {\bibinfo {title} {Composable security in relativistic quantum
  cryptography},}\ }\href {\doibase 10.1088/1367-2630/ab0e3b} {\bibfield
  {journal} {\bibinfo  {journal} {New J. Phys.}\ }\textbf {\bibinfo {volume}
  {21}}~(\bibinfo {number} {4}),\ \bibinfo {pages} {043057}},\ \Eprint
  {http://arxiv.org/abs/arXiv:1708.00433} {arXiv:1708.00433} \BibitemShut
  {NoStop}%
\bibitem [{\citenamefont {Wang}(2005)}]{Wang2005}%
  \BibitemOpen
  \bibfield  {author} {\bibinfo {author} {\bibnamefont {Wang}, \bibfnamefont
  {Xiang-Bin}}} (\bibinfo {year} {2005}),\ \bibfield  {title} {\enquote
  {\bibinfo {title} {Beating the photon-number-splitting attack in practical
  quantum cryptography},}\ }\href {\doibase 10.1103/PhysRevLett.94.230503}
  {\bibfield  {journal} {\bibinfo  {journal} {Phys. Rev. Lett.}\ }\textbf
  {\bibinfo {volume} {94}},\ \bibinfo {pages} {230503}},\ \Eprint
  {http://arxiv.org/abs/arxiv:quant-ph/0410075} {arxiv:quant-ph/0410075}
  \BibitemShut {NoStop}%
\bibitem [{\citenamefont {Watrous}(2018)}]{Wat18}%
  \BibitemOpen
  \bibfield  {author} {\bibinfo {author} {\bibnamefont {Watrous}, \bibfnamefont
  {John}}} (\bibinfo {year} {2018}),\ \href {\doibase 10.1017/9781316848142}
  {\emph {\bibinfo {title} {The Theory of Quantum Information}}}\ (\bibinfo
  {publisher} {Cambridge University Press})\ \bibinfo {note} {available at
  \url{http://cs.uwaterloo.ca/~watrous/TQI/}}\BibitemShut {NoStop}%
\bibitem [{\citenamefont {Webb}(2015)}]{Web15}%
  \BibitemOpen
  \bibfield  {author} {\bibinfo {author} {\bibnamefont {Webb}, \bibfnamefont
  {Zak}}} (\bibinfo {year} {2015}),\ \bibfield  {title} {\enquote {\bibinfo
  {title} {The {Clifford} group forms a unitary 3-design},}\ }\href@noop {}
  {\bibfield  {journal} {\bibinfo  {journal} {Quantum Inf. Comput.}\ }\textbf
  {\bibinfo {volume} {16}}~(\bibinfo {number} {15{\&}16}),\ \bibinfo {pages}
  {1379--1400}},\ \Eprint {http://arxiv.org/abs/arXiv:1510.02769}
  {arXiv:1510.02769} \BibitemShut {NoStop}%
\bibitem [{\citenamefont {Wegman}\ and\ \citenamefont {Carter}(1981)}]{WC81}%
  \BibitemOpen
  \bibfield  {author} {\bibinfo {author} {\bibnamefont {Wegman}, \bibfnamefont
  {Mark~N}}, \ and\ \bibinfo {author} {\bibfnamefont {Larry}\ \bibnamefont
  {Carter}}} (\bibinfo {year} {1981}),\ \bibfield  {title} {\enquote {\bibinfo
  {title} {New hash functions and their use in authentication and set
  equality},}\ }\href@noop {} {\bibfield  {journal} {\bibinfo  {journal} {J.
  Comput. Syst. Sci.}\ }\textbf {\bibinfo {volume} {22}}~(\bibinfo {number}
  {3}),\ \bibinfo {pages} {265--279}}\BibitemShut {NoStop}%
\bibitem [{\citenamefont {Wehner}\ \emph {et~al.}(2008)\citenamefont {Wehner},
  \citenamefont {Schaffner},\ and\ \citenamefont {Terhal}}]{WST08}%
  \BibitemOpen
  \bibfield  {author} {\bibinfo {author} {\bibnamefont {Wehner}, \bibfnamefont
  {Stephanie}}, \bibinfo {author} {\bibfnamefont {Christian}\ \bibnamefont
  {Schaffner}}, \ and\ \bibinfo {author} {\bibfnamefont {Barbara~M.}\
  \bibnamefont {Terhal}}} (\bibinfo {year} {2008}),\ \bibfield  {title}
  {\enquote {\bibinfo {title} {Cryptography from noisy storage},}\ }\href
  {\doibase 10.1103/PhysRevLett.100.220502} {\bibfield  {journal} {\bibinfo
  {journal} {Phys. Rev. Lett.}\ }\textbf {\bibinfo {volume} {100}},\ \bibinfo
  {pages} {220502}},\ \Eprint {http://arxiv.org/abs/arXiv:0711.2895}
  {arXiv:0711.2895} \BibitemShut {NoStop}%
\bibitem [{\citenamefont {Weier}\ \emph {et~al.}(2011)\citenamefont {Weier},
  \citenamefont {Krauss}, \citenamefont {Rau}, \citenamefont {F\"{u}rst},
  \citenamefont {Nauerth},\ and\ \citenamefont {Weinfurter}}]{WKRFNW11}%
  \BibitemOpen
  \bibfield  {author} {\bibinfo {author} {\bibnamefont {Weier}, \bibfnamefont
  {Henning}}, \bibinfo {author} {\bibfnamefont {Harald}\ \bibnamefont
  {Krauss}}, \bibinfo {author} {\bibfnamefont {Markus}\ \bibnamefont {Rau}},
  \bibinfo {author} {\bibfnamefont {Martin}\ \bibnamefont {F\"{u}rst}},
  \bibinfo {author} {\bibfnamefont {Sebastian}\ \bibnamefont {Nauerth}}, \ and\
  \bibinfo {author} {\bibfnamefont {Harald}\ \bibnamefont {Weinfurter}}}
  (\bibinfo {year} {2011}),\ \bibfield  {title} {\enquote {\bibinfo {title}
  {Quantum eavesdropping without interception: an attack exploiting the dead
  time of single-photon detectors},}\ }\href {\doibase
  10.1088/1367-2630/13/7/073024} {\bibfield  {journal} {\bibinfo  {journal}
  {New J. Phys.}\ }\textbf {\bibinfo {volume} {13}}~(\bibinfo {number} {7}),\
  \bibinfo {pages} {073024}},\ \Eprint {http://arxiv.org/abs/arXiv:1101.5289}
  {arXiv:1101.5289} \BibitemShut {NoStop}%
\bibitem [{\citenamefont {Weihs}\ \emph {et~al.}(1998)\citenamefont {Weihs},
  \citenamefont {Jennewein}, \citenamefont {Simon}, \citenamefont
  {Weinfurter},\ and\ \citenamefont {Zeilinger}}]{Weihs}%
  \BibitemOpen
  \bibfield  {author} {\bibinfo {author} {\bibnamefont {Weihs}, \bibfnamefont
  {Gregor}}, \bibinfo {author} {\bibfnamefont {Thomas}\ \bibnamefont
  {Jennewein}}, \bibinfo {author} {\bibfnamefont {Christoph}\ \bibnamefont
  {Simon}}, \bibinfo {author} {\bibfnamefont {Harald}\ \bibnamefont
  {Weinfurter}}, \ and\ \bibinfo {author} {\bibfnamefont {Anton}\ \bibnamefont
  {Zeilinger}}} (\bibinfo {year} {1998}),\ \bibfield  {title} {\enquote
  {\bibinfo {title} {Violation of {Bell}'s inequality under strict einstein
  locality conditions},}\ }\href {\doibase 10.1103/PhysRevLett.81.5039}
  {\bibfield  {journal} {\bibinfo  {journal} {Phys. Rev. Lett.}\ }\textbf
  {\bibinfo {volume} {81}},\ \bibinfo {pages} {5039--5043}}\BibitemShut
  {NoStop}%
\bibitem [{\citenamefont {Wiesner}(1983)}]{Wie83}%
  \BibitemOpen
  \bibfield  {author} {\bibinfo {author} {\bibnamefont {Wiesner}, \bibfnamefont
  {Stephen}}} (\bibinfo {year} {1983}),\ \bibfield  {title} {\enquote {\bibinfo
  {title} {Conjugate coding},}\ }\href@noop {} {\bibfield  {journal} {\bibinfo
  {journal} {SIGACT news}\ }\textbf {\bibinfo {volume} {15}}~(\bibinfo {number}
  {1}),\ \bibinfo {pages} {78--88}},\ \bibinfo {note} {original manuscript
  written circa 1969}\BibitemShut {NoStop}%
\bibitem [{\citenamefont {Winter}(2017)}]{Win17}%
  \BibitemOpen
  \bibfield  {author} {\bibinfo {author} {\bibnamefont {Winter}, \bibfnamefont
  {Andreas}}} (\bibinfo {year} {2017}),\ \bibfield  {title} {\enquote {\bibinfo
  {title} {Weak locking capacity of quantum channels can be much larger than
  private capacity},}\ }\href {\doibase 10.1007/s00145-015-9215-3} {\bibfield
  {journal} {\bibinfo  {journal} {J. Crypt.}\ }\textbf {\bibinfo {volume}
  {30}}~(\bibinfo {number} {1}),\ \bibinfo {pages} {1--21}},\ \Eprint
  {http://arxiv.org/abs/arXiv:1403.6361} {arXiv:1403.6361} \BibitemShut
  {NoStop}%
\bibitem [{\citenamefont {Wolf}(1999)}]{Wol99}%
  \BibitemOpen
  \bibfield  {author} {\bibinfo {author} {\bibnamefont {Wolf}, \bibfnamefont
  {Stefan}}} (\bibinfo {year} {1999}),\ \emph {\bibinfo {title}
  {Information-Theoretically and Computationally Secure Key Agreement in
  Cryptography}},\ \href@noop {} {Ph.D. thesis}\ (\bibinfo  {school} {Swiss
  Federal Institute of Technology (ETH) Zurich})\BibitemShut {NoStop}%
\bibitem [{\citenamefont {Wootters}\ and\ \citenamefont
  {Zurek}(1982)}]{Wootters82}%
  \BibitemOpen
  \bibfield  {author} {\bibinfo {author} {\bibnamefont {Wootters},
  \bibfnamefont {William~K}}, \ and\ \bibinfo {author} {\bibfnamefont
  {Wojciech~H.}\ \bibnamefont {Zurek}}} (\bibinfo {year} {1982}),\ \bibfield
  {title} {\enquote {\bibinfo {title} {A single quantum cannot be cloned},}\
  }\href {\doibase 10.1038/299802a0} {\bibfield  {journal} {\bibinfo  {journal}
  {Nature}\ }\textbf {\bibinfo {volume} {299}}~(\bibinfo {number} {5886}),\
  \bibinfo {pages} {802--803}}\BibitemShut {NoStop}%
\bibitem [{\citenamefont {Xu}\ \emph {et~al.}(2010)\citenamefont {Xu},
  \citenamefont {Qi},\ and\ \citenamefont {Lo}}]{XQL10}%
  \BibitemOpen
  \bibfield  {author} {\bibinfo {author} {\bibnamefont {Xu}, \bibfnamefont
  {Feihu}}, \bibinfo {author} {\bibfnamefont {Bing}\ \bibnamefont {Qi}}, \ and\
  \bibinfo {author} {\bibfnamefont {Hoi-Kwong}\ \bibnamefont {Lo}}} (\bibinfo
  {year} {2010}),\ \bibfield  {title} {\enquote {\bibinfo {title} {Experimental
  demonstration of phase-remapping attack in a practical quantum key
  distribution system},}\ }\href {\doibase 10.1088/1367-2630/12/11/113026}
  {\bibfield  {journal} {\bibinfo  {journal} {New J. Phys.}\ }\textbf {\bibinfo
  {volume} {12}}~(\bibinfo {number} {11}),\ \bibinfo {pages}
  {113026}}\BibitemShut {NoStop}%
\bibitem [{\citenamefont {Yin}\ \emph {et~al.}(2016)\citenamefont {Yin},
  \citenamefont {Chen}, \citenamefont {Yu}, \citenamefont {Liu}, \citenamefont
  {You}, \citenamefont {Zhou}, \citenamefont {Chen}, \citenamefont {Mao},
  \citenamefont {Huang}, \citenamefont {Zhang}, \citenamefont {Chen},
  \citenamefont {Li}, \citenamefont {Nolan}, \citenamefont {Zhou},
  \citenamefont {Jiang}, \citenamefont {Wang}, \citenamefont {Zhang},
  \citenamefont {Wang},\ and\ \citenamefont {Pan}}]{Yin2016}%
  \BibitemOpen
  \bibfield  {author} {\bibinfo {author} {\bibnamefont {Yin}, \bibfnamefont
  {Hua-Lei}}, \bibinfo {author} {\bibfnamefont {Teng-Yun}\ \bibnamefont
  {Chen}}, \bibinfo {author} {\bibfnamefont {Zong-Wen}\ \bibnamefont {Yu}},
  \bibinfo {author} {\bibfnamefont {Hui}\ \bibnamefont {Liu}}, \bibinfo
  {author} {\bibfnamefont {Li-Xing}\ \bibnamefont {You}}, \bibinfo {author}
  {\bibfnamefont {Yi-Heng}\ \bibnamefont {Zhou}}, \bibinfo {author}
  {\bibfnamefont {Si-Jing}\ \bibnamefont {Chen}}, \bibinfo {author}
  {\bibfnamefont {Yingqiu}\ \bibnamefont {Mao}}, \bibinfo {author}
  {\bibfnamefont {Ming-Qi}\ \bibnamefont {Huang}}, \bibinfo {author}
  {\bibfnamefont {Wei-Jun}\ \bibnamefont {Zhang}}, \bibinfo {author}
  {\bibfnamefont {Hao}\ \bibnamefont {Chen}}, \bibinfo {author} {\bibfnamefont
  {Ming~Jun}\ \bibnamefont {Li}}, \bibinfo {author} {\bibfnamefont {Daniel}\
  \bibnamefont {Nolan}}, \bibinfo {author} {\bibfnamefont {Fei}\ \bibnamefont
  {Zhou}}, \bibinfo {author} {\bibfnamefont {Xiao}\ \bibnamefont {Jiang}},
  \bibinfo {author} {\bibfnamefont {Zhen}\ \bibnamefont {Wang}}, \bibinfo
  {author} {\bibfnamefont {Qiang}\ \bibnamefont {Zhang}}, \bibinfo {author}
  {\bibfnamefont {Xiang-Bin}\ \bibnamefont {Wang}}, \ and\ \bibinfo {author}
  {\bibfnamefont {Jian-Wei}\ \bibnamefont {Pan}}} (\bibinfo {year} {2016}),\
  \bibfield  {title} {\enquote {\bibinfo {title}
  {Measurement-device-independent quantum key distribution over a 404 km
  optical fiber},}\ }\href {\doibase 10.1103/PhysRevLett.117.190501} {\bibfield
   {journal} {\bibinfo  {journal} {Phys. Rev. Lett.}\ }\textbf {\bibinfo
  {volume} {117}},\ \bibinfo {pages} {190501}},\ \Eprint
  {http://arxiv.org/abs/arXiv:1606.06821} {arXiv:1606.06821} \BibitemShut
  {NoStop}%
\bibitem [{\citenamefont {Yin}\ and\ \citenamefont {Chen}(2019)}]{YinChen}%
  \BibitemOpen
  \bibfield  {author} {\bibinfo {author} {\bibnamefont {Yin}, \bibfnamefont
  {Hua-Lei}}, \ and\ \bibinfo {author} {\bibfnamefont {Zeng-Bing}\ \bibnamefont
  {Chen}}} (\bibinfo {year} {2019}),\ \bibfield  {title} {\enquote {\bibinfo
  {title} {Finite-key analysis for twin-field quantum key distribution with
  composable security},}\ }\href@noop {} {\bibfield  {journal} {\bibinfo
  {journal} {Sci. Rep.}\ }\textbf {\bibinfo {volume} {9}}~(\bibinfo {number}
  {1}),\ \bibinfo {pages} {1--9}}\BibitemShut {NoStop}%
\bibitem [{\citenamefont {Yuan}\ \emph {et~al.}(2010)\citenamefont {Yuan},
  \citenamefont {Dynes},\ and\ \citenamefont {Shields}}]{Yuanetal2010}%
  \BibitemOpen
  \bibfield  {author} {\bibinfo {author} {\bibnamefont {Yuan}, \bibfnamefont
  {Zhiliang}}, \bibinfo {author} {\bibfnamefont {James~F.}\ \bibnamefont
  {Dynes}}, \ and\ \bibinfo {author} {\bibfnamefont {Andrew~J.}\ \bibnamefont
  {Shields}}} (\bibinfo {year} {2010}),\ \bibfield  {title} {\enquote {\bibinfo
  {title} {Avoiding the blinding attack in {QKD}},}\ }\href {\doibase
  10.1038/nphoton.2010.269} {\bibfield  {journal} {\bibinfo  {journal} {Nat.
  Photonics}\ }\textbf {\bibinfo {volume} {4}},\ \bibinfo {pages}
  {800}}\BibitemShut {NoStop}%
\bibitem [{\citenamefont {Zhandry}(2012)}]{Zha12}%
  \BibitemOpen
  \bibfield  {author} {\bibinfo {author} {\bibnamefont {Zhandry}, \bibfnamefont
  {Mark}}} (\bibinfo {year} {2012}),\ \bibfield  {title} {\enquote {\bibinfo
  {title} {How to construct quantum random functions},}\ }in\ \href {\doibase
  10.1109/FOCS.2012.37} {\emph {\bibinfo {booktitle} {Proceedings of the 53rd
  Symposium on Foundations of Computer Science, FOCS~'12}}}\ (\bibinfo
  {publisher} {IEEE})\ pp.\ \bibinfo {pages} {679--687},\ \bibinfo {note}
  {e-Print \href{http://eprint.iacr.org/2012/182}{IACR 2012/182}}\BibitemShut
  {NoStop}%
\bibitem [{\citenamefont {Zhao}\ \emph {et~al.}(2008)\citenamefont {Zhao},
  \citenamefont {Fung}, \citenamefont {Qi}, \citenamefont {Chen},\ and\
  \citenamefont {Lo}}]{Zhaoetal2008}%
  \BibitemOpen
  \bibfield  {author} {\bibinfo {author} {\bibnamefont {Zhao}, \bibfnamefont
  {Yi}}, \bibinfo {author} {\bibfnamefont {Chi-Hang~Fred}\ \bibnamefont
  {Fung}}, \bibinfo {author} {\bibfnamefont {Bing}\ \bibnamefont {Qi}},
  \bibinfo {author} {\bibfnamefont {Christine}\ \bibnamefont {Chen}}, \ and\
  \bibinfo {author} {\bibfnamefont {Hoi-Kwong}\ \bibnamefont {Lo}}} (\bibinfo
  {year} {2008}),\ \bibfield  {title} {\enquote {\bibinfo {title} {Quantum
  hacking: Experimental demonstration of time-shift attack against practical
  quantum-key-distribution systems},}\ }\href {\doibase
  10.1103/PhysRevA.78.042333} {\bibfield  {journal} {\bibinfo  {journal} {Phys.
  Rev. A}\ }\textbf {\bibinfo {volume} {78}},\ \bibinfo {pages}
  {042333}}\BibitemShut {NoStop}%
\bibitem [{\citenamefont {Zhu}(2017)}]{Zhu17}%
  \BibitemOpen
  \bibfield  {author} {\bibinfo {author} {\bibnamefont {Zhu}, \bibfnamefont
  {Huangjun}}} (\bibinfo {year} {2017}),\ \bibfield  {title} {\enquote
  {\bibinfo {title} {Multiqubit {Clifford} groups are unitary 3-designs},}\
  }\href {\doibase 10.1103/PhysRevA.96.062336} {\bibfield  {journal} {\bibinfo
  {journal} {Phys. Rev. A}\ }\textbf {\bibinfo {volume} {96}},\ \bibinfo
  {pages} {062336}},\ \Eprint {http://arxiv.org/abs/arXiv:1510.02619}
  {arXiv:1510.02619} \BibitemShut {NoStop}%
\bibitem [{\citenamefont {Zuckerman}(1990)}]{Zuc90}%
  \BibitemOpen
  \bibfield  {author} {\bibinfo {author} {\bibnamefont {Zuckerman},
  \bibfnamefont {David}}} (\bibinfo {year} {1990}),\ \bibfield  {title}
  {\enquote {\bibinfo {title} {General weak random sources},}\ }in\ \href
  {\doibase 10.1109/FSCS.1990.89574} {\emph {\bibinfo {booktitle} {Proceedings
  of the 31st Symposium on Foundations of Computer Science, FOCS~'90}}}\
  (\bibinfo  {publisher} {IEEE})\ pp.\ \bibinfo {pages} {534--543}\BibitemShut
  {NoStop}%
\end{thebibliography}%

\begin{thebibliography}{99}

%\cite{Wendell:2010md}
\bibitem{Wendell:2010md} 
  R.~Wendell {\it et al.} [Super-Kamiokande Collaboration],
  %``Atmospheric neutrino oscillation analysis with sub-leading effects in Super-Kamiokande I, II, and III,''
  Phys.\ Rev.\ D {\bf 81}, 092004 (2010)
  doi:10.1103/PhysRevD.81.092004
  [arXiv:1002.3471 [hep-ex]].
  %%CITATION = doi:10.1103/PhysRevD.81.092004;%%
  %278 citations counted in INSPIRE as of 22 Nov 2016
  
  
\bibitem{big-422}
B. Ananthanarayan, G. Lazarides and Q. Shafi, Phys. Rev. D {\bf 44},
1613 (1991; Phys. Lett. B {\bf 300}, 24 (1993)5; Q.~Shafi and
B.~Ananthanarayan, Trieste HEP Cosmol.1991:233-244;
%%%%%%%5
%\bibitem{pati-salam}
J.~C.~Pati and A.~Salam,
  %``Lepton Number As The Fourth Color,''
  Phys.\ Rev.\  D {\bf 10}, 275 (1974).  
  
\bibitem{pati-salam}
J.~C.~Pati and A.~Salam,
  %``Lepton Number As The Fourth Color,''
  Phys.\ Rev.\  D {\bf 10}, 275 (1974).
  
  
  
\bibitem{bigger-422}See, incomplete list of references,
L.~J.~Hall, R.~Rattazzi and U.~Sarid, Phys.\ Rev.\  D {\bf 50}, 7048 (1994);
%V. Barger, M. Berger and P. Ohmann, Phys. Rev. D {\bf 49}, (1994) 4908;
 B. Ananthanarayan, Q. Shafi and X.
Wang, Phys. Rev. D {\bf 50}, 5980 (1994);
%G. Anderson et al. Phys. Rev. D {\bf 47}, (1993) 3702 and Phys. Rev. D {\bf 49},  3660 (1994);
R. Rattazzi and U. Sarid, Phys. Rev. D {\bf 53}, 1553
(1996); T. Blazek, M. Carena, S. Raby and C. Wagner, Phys. Rev. D
{\bf 56}, 6919 (1997);
J.~L.~Chkareuli and I.~G.~Gogoladze,
  %``Unification picture in minimal supersymmetric SU(5) model with string
  %remnants,''
  Phys.\ Rev.\  D {\bf 58}, 055011 (1998);
T. Blazek, S. Raby and K. Tobe, Phys. Rev. D
{\bf 62}, 055001 (2000); H. Baer, M. Brhlik, M. Diaz,
J. Ferrandis, P. Mercadante, P. Quintana and X. Tata, Phys. Rev. D
{\bf 63}, 015007(2001);  C.~Balazs and R.~Dermisek, JHEP {\bf 0306}, 024 (2003);
 U. Chattopadhyay, A. Corsetti and P.
Nath, Phys. Rev. D {\bf 66} 035003, (2002); T.~Blazek, R.~Dermisek
and S.~Raby, Phys.\ Rev.\ Lett.\  {\bf 88}, 111804 (2002); M. Gomez, T. Ibrahim, P. Nath and
S. Skadhauge, Phys. Rev. D {\bf 72}, 095008 (2005); K. Tobe and J.
D. Wells, Nucl. Phys. B {\bf 663}, 123 (2003);
I.~Gogoladze, Y.~Mimura, S.~Nandi,
  %``Unification of gauge, Higgs and matter in extra dimensions,''
  Phys.\ Lett.\  {\bf B562}, 307 (2003);
W.~Altmannshofer,
D.~Guadagnoli, S.~Raby and D.~M.~Straub, Phys.\ Lett.\  B {\bf 668},
385 (2008);
S.~Antusch and M.~Spinrath,
 Phys.\ Rev.\  D {\bf 78}, 075020 (2008);
 H.~Baer, S.~Kraml and S.~Sekmen, JHEP {\bf 0909}, 005 (2009);
S.~Antusch and M.~Spinrath,
Phys.\ Rev.\  D {\bf 79}, 095004 (2009);
%D.~Guadagnoli, S.~Raby and D.~M.~Straub, JHEP {\bf 0910}, 059 (2009);
K.~Choi, D.~Guadagnoli, S.~H.~Im and C.~B.~Park,
  JHEP {\bf 1010}, 025 (2010);
%arXiv:1005.0618 [hep-ph].
 %\cite{arXiv:1105.5122}
%\bibitem{arXiv:1107.1228}
 % I.~Gogoladze, Q.~Shafi and C.~S.~Un,
  %``SO(10) Yukawa Unification with mu < 0,''
 % Phys.\ Lett.\ B\ {\bf 704}, 201  (2011);
 % [arXiv:1107.1228 [hep-ph]].
  %%CITATION = PHLTA,B704,201;
%\cite{arXiv:1107.2764}
%\bibitem{arXiv:1107.2764}
  M.~Badziak, M.~Olechowski and S.~Pokorski,
  %``Yukawa unification in SO(10) with light sparticle spectrum,''
  JHEP\ {\bf 1108}, 147  (2011);
  %[arXiv:1107.2764 [hep-ph]].
  %%CITATION = JHEPA,1108,147;%%
%\cite{arXiv:1111.6547}
%\bibitem{arXiv:1111.6547}
  S.~Antusch, L.~Calibbi, V.~Maurer, M.~Monaco and M.~Spinrath,
  %``Naturalness and GUT Scale Yukawa Coupling Ratios in the CMSSM,''
%  arXiv:1111.6547 [hep-ph];
   Phys.\ Rev.\ D {\bf 85}, 035025 (2012).
  %%CITATION = ARXIV:1111.6547;%%
%\cite{arXiv:1111.3639}
%\bibitem{arXiv:1111.3639}
  J.~S.~Gainer, R.~Huo and C.~E.~M.~Wagner,
  %``An Alternative Yukawa Unified SUSY Scenario,''
  %arXiv:1111.3639 [hep-ph].
    JHEP {\bf 1203}, 097 (2012);
H.~Baer, S.~Raza and Q.~Shafi,
  %``A Heavier gluino from $t-b-\tau$ Yukawa-unified SUSY,''
  Phys.\ Lett.\ B {\bf 712}, 250 (2012);
%  arXiv:1201.5668 [hep-ph];
  %%CITATION = ARXIV:1111.3639;%
  %\cite{Gogoladze:2012ii}
%\bibitem{Gogoladze:2012ii}
  I.~Gogoladze, Q.~Shafi, C.~S.~Un and ,
  %``125 GeV Higgs Boson from t-b-tau Yukawa Unification,''
  JHEP {\bf 1207}, 055 (2012);
 % [arXiv:1203.6082 [hep-ph]].
  %%CITATION = ARXIV:1203.6082;%%
  %8 citations counted in INSPIRE as of 25 Mar 2013
%\cite{Badziak:2012mm}
%\bibitem{Badziak:2012mm}
  M.~Badziak,
  %``Yukawa unification in SUSY SO(10) in light of the LHC Higgs data,''
  Mod.\ Phys.\ Lett.\ A {\bf 27}, 1230020 (2012);
 % [arXiv:1205.6232 [hep-ph]].
  %%CITATION = ARXIV:1205.6232;%%
  %4 citations counted in INSPIRE as of 14 Mar 2013
%\cite{Elor:2012ig}
%\bibitem{Elor:2012ig}
  G.~Elor, L.~J.~Hall, D.~Pinner and  J.~T.~Ruderman,
  %``Yukawa Unification and the Superpartner Mass Scale,''
  JHEP {\bf 1210}, 111 (2012).
%  [arXiv:1206.5301 [hep-ph]].
  %%CITATION = ARXIV:1206.5301;%%
  %5 citations counted in INSPIRE as of 25 Mar 2013
%\cite{Gogoladze:2011ce}
%\bibitem{Gogoladze:2011ce}
  I.~Gogoladze, Q.~Shafi and C.~S.~Un,
  %``SO(10) Yukawa Unification with mu < 0,''
  Phys.\ Lett.\ B {\bf 704}, 201 (2011)
%  [arXiv:1107.1228 [hep-ph]].
  %%CITATION = ARXIV:1107.1228;%%
  %15 citations counted in INSPIRE as of 14 Nov 2013
%\cite{Gogoladze:2011aa}
%\bibitem{Gogoladze:2011aa}
  I.~Gogoladze, Q.~Shafi and C.~S.~Un,
  %``Higgs Boson Mass from t-b-$\tau$ Yukawa Unification,''
  JHEP {\bf 1208}, 028 (2012)
%  [arXiv:1112.2206 [hep-ph]].
  %%CITATION = ARXIV:1112.2206;%%
  %41 citations counted in INSPIRE as of 14 Nov 2013
%\cite{Gogoladze:2012ii}
%\bibitem{Gogoladze:2012ii}
%  I.~Gogoladze, Q.~Shafi and C.~S.~Un,
  %``125 GeV Higgs Boson from t-b-tau Yukawa Unification,''
%  JHEP {\bf 1207}, 055 (2012)
%  [arXiv:1203.6082 [hep-ph]].
  %%CITATION = ARXIV:1203.6082;%%
  %15 citations counted in INSPIRE as of 14 Nov 2013
%\cite{Ajaib:2013zha}
%\cite{Ajaib:2013kka}
%\bibitem{Ajaib:2013kka}
  M.~A.~Ajaib, I.~Gogoladze and Q.~Shafi,
  %``Sparticle Spectroscopy from SO(10) GUT with a Unified Higgs Sector,''
  arXiv:1307.4882 [hep-ph].
  %%CITATION = ARXIV:1307.4882;%%
  %1 citations counted in INSPIRE as of 14 Nov 2013
%\bibitem{Ajaib:2013zha}
  M.~Adeel Ajaib, I.~Gogoladze, Q.~Shafi and C.~S.~Un,
  %``A Predictive Yukawa Unified SO(10) Model: Higgs and Sparticle Masses,''
  JHEP {\bf 1307}, 139 (2013)
%  [arXiv:1303.6964 [hep-ph]].
  %%CITATION = ARXIV:1303.6964;%%
  %7 citations counted in INSPIRE as of 14 Nov 2013
%\cite{Ajaib:2013uda}
%\bibitem{Ajaib:2013uda}
  M.~A.~Ajaib, I.~Gogoladze, Q.~Shafi and C.~S.~Un,
  %``Higgs and Sparticle Masses from Yukawa Unified SO(10): A Snowmass White Paper,''
  arXiv:1308.4652 [hep-ph].
  %%CITATION = ARXIV:1308.4652;%%
  %1 citations counted in INSPIRE as of 14 Nov 2013  
  
%\cite{Gogoladze:2010fu}
\bibitem{Gogoladze:2010fu} 
  I.~Gogoladze, R.~Khalid, S.~Raza and Q.~Shafi,
  %``$t-b-\tau$ Yukawa unification for $\mu < 0$ with a sub-TeV sparticle spectrum,''
  JHEP {\bf 1012}, 055 (2010);
  %%CITATION = doi:10.1007/JHEP12(2010)055;%%
  %47 citations counted in INSPIRE as of 23 Nov 2016  
%\cite{Pierce:1996zz}
%
  D.~M.~Pierce, J.~A.~Bagger, K.~T.~Matchev and R.~j.~Zhang,
  %``Precision corrections in the minimal supersymmetric standard model,''
  Nucl.\ Phys.\ B {\bf 491}, 3 (1997).
  %%CITATION = doi:10.1016/S0550-3213(96)00683-9;%%
  %953 citations counted in INSPIRE as of 23 Nov 2016

%\cite{Langacker:1980js}
\bibitem{Langacker:1980js} 
  P.~Langacker,
  %``Grand Unified Theories and Proton Decay,''
  Phys.\ Rept.\  {\bf 72}, 185 (1981).
  %%CITATION = doi:10.1016/0370-1573(81)90059-4;%%
  %1269 citations counted in INSPIRE as of 23 Nov 2016

\bibitem{Witten:1979nr}
See, for instance, E.~Witten,
%``Neutrino Masses in the Minimal O(10) Theory,''
Phys.\ Lett.\  {\bf B91}, 81 (1980);
%\bibitem{10m}
  S.~M.~Barr,
  %``Light Fermion Mass Hierarchy And Grand Unification,''
  Phys.\ Rev.\ D {\bf 21}, 1424 (1980);
  %%CITATION = PHRVA,D21,1424;%%
%\cite{Nomura:1998gm}
%\bibitem{Nomura:1998gm}
  Y.~Nomura and T.~Yanagida,
  %``Bimaximal neutrino mixing in SO(10)(GUT),''
  Phys.\ Rev.\ D {\bf 59}, 017303 (1999);
 % [hep-ph/9807325];
  %%CITATION = HEP-PH/9807325;%%
%\cite{Frigerio:2008ai}
%\bibitem{Frigerio:2008ai}
  M.~Frigerio, P.~Hosteins, S.~Lavignac and A.~Romanino,
  %``A New, direct link between the baryon asymmetry and neutrino masses,''
  Nucl.\ Phys.\ B {\bf 806}, 84 (2009);
 % [arXiv:0804.0801 [hep-ph]];
  %%CITATION = ARXIV:0804.0801;%%
%\cite{Barr:2007ma}
%\bibitem{Barr:2007ma}
  S.~M.~Barr,
  %``Radiative fermion mass hierarchy in a non-supersymmetric unified theory,''
  Phys.\ Rev.\ D {\bf 76}, 105024 (2007);
%  [arXiv:0706.1490 [hep-ph]].
  %%CITATION = ARXIV:0706.1490;%%
  %\cite{Malinsky:2007qy}
%\bibitem{malinsky}
  M.~Malinsky,
  %``Quark and lepton masses and mixing in SO(10) with a GUT-scale vector matter,''
  Phys.\ Rev.\ D {\bf 77}, 055016 (2008);
  %[arXiv:0710.0581 [hep-ph]];
  %%CITATION = ARXIV:0710.0581;%%
  %\cite{Heinze:2010du}
%\bibitem{Heinze:2010du}
  M.~Heinze and M.~Malinsky,
  %``Flavour structure of supersymmetric SO(10) GUTs with extended matter sector,''
  Phys.\ Rev.\ D {\bf 83}, 035018 (2011);
  %  [arXiv:1008.4813 [hep-ph]].
  %%CITATION = ARXIV:1008.4813;%%
  %\cite{Babu:2012pb}
%\bibitem{Babu:2012pb}
  K.~S.~Babu, B.~Bajc and Z.~Tavartkiladze,
  %``Realistic Fermion Masses and Nucleon Decay Rates in SUSY SU(5) with Vector-Like Matter,''
  Phys.\ Rev.\ D {\bf 86}, 075005 (2012)
 % [arXiv:1207.6388 [hep-ph]].
  %%CITATION = ARXIV:1207.6388;%%
   and references therein.

  %\cite{Babu:1992ia}
\bibitem{Babu:1992ia}
  K.~S.~Babu and R.~N.~Mohapatra,
  %``Predictive neutrino spectrum in minimal SO(10) grand unification,''
  Phys.\ Rev.\ Lett.\  {\bf 70}, 2845 (1993).
  %[hep-ph/9209215].
  %%CITATION = HEP-PH/9209215;%%

%\cite{Joshipura:2012sr}
\bibitem{Joshipura:2012sr}
 For a brief review, see  A.~S.~Joshipura and K.~M.~Patel,
  %``Yukawa coupling unification in SO(10) with positive \mu\ and a heavier gluino,''
  Phys.\ Rev.\ D {\bf 86}, 035019 (2012) and references therein.
%  [arXiv:1206.3910 [hep-ph]].
  %%CITATION = ARXIV:1206.3910;%%
  %1 citations counted in INSPIRE as of 14 Mar 2013

  %\cite{Gomez:2002tj}
\bibitem{Gomez:2002tj}
  M.~E.~Gomez, G.~Lazarides and C.~Pallis,
  %``Yukawa quasi-unification,''
  Nucl.\ Phys.\ B {\bf 638}, 165 (2002)
  [hep-ph/0203131];
  %%CITATION = HEP-PH/0203131;%%
  %94 citations counted in INSPIRE as of 02 Oct 2014
%\cite{Gomez:2003cu}
%\bibitem{Gomez:2003cu}
  M.~E.~Gomez, G.~Lazarides and C.~Pallis,
  %``On Yukawa quasiunification with mu less than 0,''
  Phys.\ Rev.\ D {\bf 67}, 097701 (2003)
  [hep-ph/0301064];
  %%CITATION = HEP-PH/0301064;%%
  %43 citations counted in INSPIRE as of 02 Oct 2014
%\cite{Pallis:2003jc}
%\bibitem{Pallis:2003jc}
  C.~Pallis and M.~E.~Gomez,
  %``Yukawa quasiunification and neutralino relic density,''
  hep-ph/0303098;
  %%CITATION = HEP-PH/0303098;%%
  %17 citations counted in INSPIRE as of 02 Oct 2014

%\cite{Gogoladze:2009ug}
\bibitem{Gogoladze:2009ug} 
  I.~Gogoladze, R.~Khalid and Q.~Shafi,
  %``Yukawa Unification and Neutralino Dark Matter in SU(4)(c) x SU(2)(L) x SU(2)(R),''
  Phys.\ Rev.\ D {\bf 79}, 115004 (2009)
  doi:10.1103/PhysRevD.79.115004
  [arXiv:0903.5204 [hep-ph]].
  %%CITATION = doi:10.1103/PhysRevD.79.115004;%%
  %59 citations counted in INSPIRE as of 23 Nov 2016

%\cite{Raza:2014upa}
\bibitem{Raza:2014upa} 
  S.~Raza, Q.~Shafi and C.~S.~Ün,
  %``NLSP gluino and NLSP stop scenarios from $b-\tau$ Yukawa unification,''
  Phys.\ Rev.\ D {\bf 92}, no. 5, 055010 (2015)
  doi:10.1103/PhysRevD.92.055010
  [arXiv:1412.7672 [hep-ph]].
  %%CITATION = doi:10.1103/PhysRevD.92.055010;%%
  %10 citations counted in INSPIRE as of 23 Nov 2016

%\cite{Dar:2011sj}
\bibitem{Dar:2011sj} 
  S.~Dar, I.~Gogoladze, Q.~Shafi and C.~S.~Un,
  %``Sparticle Spectroscopy with Neutralino Dark matter from t-b-tau Quasi-Yukawa Unification,''
  Phys.\ Rev.\ D {\bf 84}, 085015 (2011);
  %%CITATION = doi:10.1103/PhysRevD.84.085015;%%
  %20 citations counted in INSPIRE as of 23 Nov 2016
%\cite{Shafi:2015lfa}
%\bibitem{Shafi:2015lfa} 
  Q.~Shafi, Ş.~H.~Tanyıldızı and C.~S.~Un,
  %``Neutralino Dark Matter and Other LHC Predictions from Quasi Yukawa Unification,''
  Nucl.\ Phys.\ B {\bf 900}, 400 (2015);
  %%CITATION = doi:10.1016/j.nuclphysb.2015.09.019;%%
  %7 citations counted in INSPIRE as of 23 Nov 2016
%\cite{Hicyilmaz:2016kty}
%\bibitem{Hicyilmaz:2016kty} 
  Y.~Hiçyılmaz, M.~Ceylan, A.~Altas, L.~Solmaz and C.~S.~Un,
  %``Quasi Yukawa Unification and Fine-Tuning in U(1) Extended SSM,''
  Phys.\ Rev.\ D {\bf 94}, no. 9, 095001 (2016).
  %%CITATION = doi:10.1103/PhysRevD.94.095001;%%
  %2 citations counted in INSPIRE as of 30 Nov 2016


%\cite{Bajc:2004xe}
\bibitem{Bajc:2004xe}
  B.~Bajc, A.~Melfo, G.~Senjanovic and F.~Vissani,
  %``The Minimal supersymmetric grand unified theory. 1. Symmetry breaking and
  %the particle spectrum,''
  Phys.\ Rev.\  D {\bf 70}, 035007 (2004);
  S.~Bertolini, M.~Frigerio, M.~Malinsky,
  %``Fermion masses in SUSY SO(10) with type II seesaw: A Non-minimal predictive scenario,''
  Phys.\ Rev.\  {\bf D70}, 095002 (2004);
%  [hep-ph/0406117].
  %[arXiv:hep-ph/0402122].
  %%CITATION = PHRVA,D70,035007;%%
  B.~Dutta, Y.~Mimura, R.~N.~Mohapatra,
  %``CKM CP violation in a minimal SO(10) model for neutrinos and its implications,''
  Phys.\ Rev.\  {\bf D69}, 115014 (2004);
%\bibitem{Fukuyama:2004ps}
  T.~Fukuyama, A.~Ilakovac, T.~Kikuchi, S.~Meljanac and N.~Okada,
  %``SO(10) group theory for the unified model building,''
  J.\ Math.\ Phys.\  {\bf 46}, 033505 (2005).
%  [arXiv:hep-ph/0405300].

%\cite{Antusch:2013rxa}
\bibitem{Antusch:2013rxa} 
  S.~Antusch, S.~F.~King and M.~Spinrath,
  %``GUT predictions for quark-lepton Yukawa coupling ratios with messenger masses from non-singlets,''
  Phys.\ Rev.\ D {\bf 89}, no. 5, 055027 (2014);
  %%CITATION = doi:10.1103/PhysRevD.89.055027;%%
  %26 citations counted in INSPIRE as of 01 Dec 2016
%\cite{Antusch:2011xz}
%\bibitem{Antusch:2011xz} 
  S.~Antusch, L.~Calibbi, V.~Maurer, M.~Monaco and M.~Spinrath,
  %``Naturalness and GUT Scale Yukawa Coupling Ratios in the CMSSM,''
  Phys.\ Rev.\ D {\bf 85}, 035025 (2012);
  %%CITATION = doi:10.1103/PhysRevD.85.035025;%%
  %23 citations counted in INSPIRE as of 01 Dec 2016
%\cite{Antusch:2009gu}
%\bibitem{Antusch:2009gu} 
  S.~Antusch and M.~Spinrath,
  %``New GUT predictions for quark and lepton mass ratios confronted with phenomenology,''
  Phys.\ Rev.\ D {\bf 79}, 095004 (2009);
  %%CITATION = doi:10.1103/PhysRevD.79.095004;%%
  %80 citations counted in INSPIRE as of 01 Dec 2016
%\cite{Trine:2009ns}
%\bibitem{Trine:2009ns} 
  S.~Trine, S.~Westhoff and S.~Wiesenfeldt,
  %``Probing Yukawa Unification with K and B Mixing,''
  JHEP {\bf 0908}, 002 (2009).
  %%CITATION = doi:10.1088/1126-6708/2009/08/002;%%
  %17 citations counted in INSPIRE as of 01 Dec 2016
%\cite{Hebbar:2017olk}
\bibitem{Hebbar:2017olk} 
  A.~Hebbar, Q.~Shafi and C.~S.~Un,
  %``Light Higgsinos, Heavy Gluino and $b-\tau$ Quasi-Yukawa Unification: Will the LHC find the Gluino?,''
  arXiv:1702.05431 [hep-ph], and references therein.
  %%CITATION = ARXIV:1702.05431;%%
  %1 citations counted in INSPIRE as of 01 Mar 2017


%\cite{Coriano:2014wxa}
\bibitem{Coriano:2014wxa} 
  C.~Coriano, L.~Delle Rose and C.~Marzo,
  %``Stability constraints of the scalar potential in extensions of the Standard Model with TeV scale right handed neutrinos,''
  Nucl.\ Part.\ Phys.\ Proc.\  {\bf 265-266}, 311 (2015);
  %%CITATION = doi:10.1016/j.nuclphysbps.2015.06.078;%%
%\cite{Khalil:2010zza}
%\bibitem{Khalil:2010zza} 
  S.~Khalil and H.~Okada,
  %``TeV scale B-L extension of the standard model,''
  Prog.\ Theor.\ Phys.\ Suppl.\  {\bf 180}, 35 (2010);
  %%CITATION = doi:10.1143/PTPS.180.35;%%
  %1 citations counted in INSPIRE as of 23 Nov 2016
%\cite{Abbas:2007ag}
%\bibitem{Abbas:2007ag} 
  M.~Abbas and S.~Khalil,
  %``Neutrino masses, mixing and leptogenesis in TeV scale $B$ - L extension of the standard model,''
  JHEP {\bf 0804}, 056 (2008), and references therein.
  %%CITATION = doi:10.1088/1126-6708/2008/04/056;%%
  %27 citations counted in INSPIRE as of 23 Nov 2016

%\cite{Khalil:2010iu}
\bibitem{Khalil:2010iu} 
  S.~Khalil,
  %``TeV-scale gauged B-L symmetry with inverse seesaw mechanism,''
  Phys.\ Rev.\ D {\bf 82}, 077702 (2010);
  %%CITATION = doi:10.1103/PhysRevD.82.077702;%%
  %46 citations counted in INSPIRE as of 23 Nov 2016
%\cite{Masiero:2004hg}
%\bibitem{Masiero:2004hg} 
  A.~Masiero, S.~K.~Vempati and O.~Vives,
  %``SUSY seesaw and FCNC,''
  Nucl.\ Phys.\ Proc.\ Suppl.\  {\bf 137}, 156 (2004).
  %%CITATION = doi:10.1016/j.nuclphysbps.2004.10.058;%%
  %14 citations counted in INSPIRE as of 23 Nov 2016

%\cite{Bennett:2006fi}
\bibitem{Bennett:2006fi} 
  G.~W.~Bennett {\it et al.} [Muon g-2 Collaboration],
  %``Final Report of the Muon E821 Anomalous Magnetic Moment Measurement at BNL,''
  Phys.\ Rev.\ D {\bf 73}, 072003 (2006);
  %%CITATION = doi:10.1103/PhysRevD.73.072003;%%
  %1287 citations counted in INSPIRE as of 02 Dec 2016
%\cite{Bennett:2008dy}
%\bibitem{Bennett:2008dy} 
  G.~W.~Bennett {\it et al.} [Muon (g-2) Collaboration],
  %``An Improved Limit on the Muon Electric Dipole Moment,''
  Phys.\ Rev.\ D {\bf 80}, 052008 (2009).
  %%CITATION = doi:10.1103/PhysRevD.80.052008;%%
  %137 citations counted in INSPIRE as of 02 Dec 2016

\bibitem{Davier:2010nc}
  M.~Davier, A.~Hoecker, B.~Malaescu and Z.~Zhang,
  %``Reevaluation of the Hadronic Contributions to the Muon g-2 and to alpha(MZ),''
  Eur.\ Phys.\ J.\ C {\bf 71}, 1515 (2011)
  [Eur.\ Phys.\ J.\ C {\bf 72}, 1874 (2012)]
  [arXiv:1010.4180 [hep-ph]];
  %\cite{Hagiwara:2011af}
%\bibitem{Hagiwara:2011af}
  K.~Hagiwara, R.~Liao, A.~D.~Martin, D.~Nomura and T.~Teubner,
  %``(g-2)_mu and alpha(M_Z^2) re-evaluated using new precise data,''
  J.\ Phys.\ G {\bf 38}, 085003 (2011)
  [arXiv:1105.3149 [hep-ph]].
  %%CITATION = ARXIV:1105.3149;%%
  %330 citations counted in INSPIRE as of 29 juil. 2015
  %%CITATION = ARXIV:1010.4180;%%
  %430 citations counted in INSPIRE as of 29 juil. 2015


%\cite{Moroi:1995yh}
\bibitem{Moroi:1995yh}
  T.~Moroi,
  %``The Muon anomalous magnetic dipole moment in the minimal supersymmetric standard model,''
  Phys.\ Rev.\ D {\bf 53}, 6565 (1996)
  [Phys.\ Rev.\ D {\bf 56}, 4424 (1997)]
  [hep-ph/9512396];
  %%CITATION = HEP-PH/9512396;%%
  %430 citations counted in INSPIRE as of 29 juil. 2015
%\cite{Martin:2001st}
%\bibitem{Martin:2001st}
  S.~P.~Martin and J.~D.~Wells,
  %``Muon anomalous magnetic dipole moment in supersymmetric theories,''
  Phys.\ Rev.\ D {\bf 64}, 035003 (2001)
  [hep-ph/0103067];
  %%CITATION = HEP-PH/0103067;%%
  %224 citations counted in INSPIRE as of 29 Jul 2015
%\cite{Giudice:2012pf}
%\bibitem{Giudice:2012pf}
  G.~F.~Giudice, P.~Paradisi, A.~Strumia and A.~Strumia,
  %``Correlation between the Higgs Decay Rate to Two Photons and the Muon g - 2,''
  JHEP {\bf 1210}, 186 (2012)
  [arXiv:1207.6393 [hep-ph]].
  %%CITATION = ARXIV:1207.6393;%%
  %59 citations counted in INSPIRE as of 29 Jul 2015

%\cite{Khalil:2015wua}
\bibitem{Khalil:2015wua} 
  S.~Khalil and C.~S.~Un,
  %``Muon Anomalous Magnetic Moment in SUSY B-L Model with Inverse Seesaw,''
  Phys.\ Lett.\ B {\bf 763}, 164 (2016)
  %%CITATION = doi:10.1016/j.physletb.2016.10.035;%%
  %6 citations counted in INSPIRE as of 25 Nov 2016

%\cite{Gogoladze:2014vea}
\bibitem{Gogoladze:2014vea} 
  I.~Gogoladze, B.~He, A.~Mustafayev, S.~Raza and Q.~Shafi,
  %``Effects of Neutrino Inverse Seesaw Mechanism on the Sparticle Spectrum in CMSSM and NUHM2,''
  JHEP {\bf 1405}, 078 (2014). 
   %%CITATION = doi:10.1007/JHEP05(2014)078;%%
  %4 citations counted in INSPIRE as of 29 Nov 2016

%\cite{Babu:2014lwa}
\bibitem{Babu:2014lwa} 
  K.~S.~Babu, I.~Gogoladze, Q.~Shafi and C.~S.~Ün,
  %``Muon g-2, 125 GeV Higgs boson, and neutralino dark matter in a flavor symmetry-based MSSM,''
  Phys.\ Rev.\ D {\bf 90}, no. 11, 116002 (2014);
  %%CITATION = doi:10.1103/PhysRevD.90.116002;%%
  %14 citations counted in INSPIRE as of 02 Dec 2016
%\cite{Ajaib:2014ana}
%\bibitem{Ajaib:2014ana} 
  M.~A.~Ajaib, I.~Gogoladze, Q.~Shafi and C.~S.~Ün,
  %``Split sfermion families, Yukawa unification and muon $g - 2$,''
  JHEP {\bf 1405}, 079 (2014);
  %%CITATION = doi:10.1007/JHEP05(2014)079;%%
  %23 citations counted in INSPIRE as of 02 Dec 2016
%\cite{Gogoladze:2014cha}
%\bibitem{Gogoladze:2014cha} 
  I.~Gogoladze, F.~Nasir, Q.~Shafi and C.~S.~Un,
  %``Nonuniversal Gaugino Masses and Muon g-2,''
  Phys.\ Rev.\ D {\bf 90}, no. 3, 035008 (2014);
  %%CITATION = doi:10.1103/PhysRevD.90.035008;%%
  %23 citations counted in INSPIRE as of 02 Dec 2016
%\cite{Gogoladze:2015jua}
%\bibitem{Gogoladze:2015jua} 
  I.~Gogoladze, Q.~Shafi and C.~S.~Ün,
  %``Reconciling the muon g−2 , a 125 GeV Higgs boson, and dark matter in gauge mediation models,''
  Phys.\ Rev.\ D {\bf 92}, no. 11, 115014 (2015);
  %%CITATION = doi:10.1103/PhysRevD.92.115014;%%
  %6 citations counted in INSPIRE as of 02 Dec 2016
%\cite{Padley:2015uma}
%\bibitem{Padley:2015uma} 
  B.~P.~Padley, K.~Sinha and K.~Wang,
  %``Natural Supersymmetry, Muon $g-2$, and the Last Crevices for the Top Squark,''
  Phys.\ Rev.\ D {\bf 92}, no. 5, 055025 (2015); and references therein.
  %%CITATION = doi:10.1103/PhysRevD.92.055025;%%
  %10 citations counted in INSPIRE as of 02 Dec 2016



%\cite{Baer:2012mv}
\bibitem{Baer:2012mv} See for instance, \\
  H.~Baer, V.~Barger, P.~Huang, D.~Mickelson, A.~Mustafayev and X.~Tata,
  %``Post-LHC7 fine-tuning in the minimal supergravity/CMSSM model with a 125 GeV Higgs boson,''
  Phys.\ Rev.\ D {\bf 87}, no. 3, 035017 (2013);
  %%CITATION = doi:10.1103/PhysRevD.87.035017;%%
  %102 citations counted in INSPIRE as of 29 Nov 2016
%\cite{Demir:2014jqa}
%\bibitem{Demir:2014jqa} 
  D.~A.~Demir and C.~S.~Ün,
  %``Stop on Top: SUSY Parameter Regions, Fine-Tuning Constraints,''
  Phys.\ Rev.\ D {\bf 90}, 095015 (2014);
  %%CITATION = doi:10.1103/PhysRevD.90.095015;%%
  %7 citations counted in INSPIRE as of 29 Nov 2016
%\cite{Cici:2016oqr}
%\bibitem{Cici:2016oqr} 
  A.~Cici, Z.~Kirca and C.~S.~Un,
  %``Light Stops and Fine-Tuning in MSSM,''
  arXiv:1611.05270 [hep-ph]; and references therein.
  %%CITATION = ARXIV:1611.05270;%%
  %1 citations counted in INSPIRE as of 29 Nov 2016

\bibitem{Porod:2003um}
  Porod, W.
  %``SPheno, a program for calculating supersymmetric spectra, SUSY particle decays and SUSY particle production at e+ e- colliders,''
  {\it Comput.\ Phys.\ Commun.\ } {\bf 2003}, {\it 153}, 275;  
 % \bibitem{Porod2}
  Porod, W. and Staub, F.
  %``SPheno 3.1: Extensions including flavour, CP-phases and models beyond the MSSM,''
  {\it Comput.\ Phys.\ Commun.\ } {\bf 2012} {\it 183}, 2458.
  %%CITATION = doi:10.1016/j.cpc.2012.05.021;%%
  %299 citations counted in INSPIRE as of 01 Nov 2016
  %%CITATION = doi:10.1016/S0010-4655(03)00222-4;%%
  %666 citations counted in INSPIRE as of 01 Nov 2016

\bibitem{Staub:2008uz}
  Staub, F.
  %``Sarah,''
{\bf 2008}, {\it Preprint  arXiv:0806.0538}.
  %%CITATION = ARXIV:0806.0538;%%
  %178 citations counted in INSPIRE as of 01 Nov 2016
  
  \bibitem{Hisano:1992jj}
  Hisano, J.; Murayama, H.; and Yanagida, T.
  %``Nucleon decay in the minimal supersymmetric SU(5) grand unification,''
 {\it Nucl.\ Phys.\ B } {\bf 1993} {\it 402}, 46;
%\bibitem{GUTth}
  Chkareuli, J. L.; and Gogoladze, I. G.
  %``Unification picture in minimal supersymmetric SU(5) model with string remnants,''
 {\it Phys.\ Rev.\ D }{\bf 1998} {\it 58}, 055011.
  %%CITATION = doi:10.1103/PhysRevD.58.055011;%%
  %71 citations counted in INSPIRE as of 01 Nov 2016
  %%CITATION = doi:10.1016/0550-3213(93)90636-4;%%
  %410 citations counted in INSPIRE as of 01 Nov 2016
  
  \bibitem{Group:2009ad}
  T.~E.~W.~Group [CDF and D0 Collaborations], {\bf 2009}, {\it Preprint
  %``Combination of CDF and D0 Results on the Mass of the Top Quark,''
  arXiv:0903.2503}.
  %%CITATION = ARXIV:0903.2503;%%
  %282 citations counted in INSPIRE as of 01 Nov 2016
  
  \bibitem{Gogoladze:2011db}
  Gogoladze, I.; Khalid, R.; Raza S.; and Shafi Q.
  %``Higgs and Sparticle Spectroscopy with Gauge-Yukawa Unification,''
  {\it JHEP } {\bf 2011}, {\it 1106}, 117.
  %%CITATION = doi:10.1007/JHEP06(2011)117;%%
  %35 citations counted in INSPIRE as of 01 Nov 2016
  %\cite{Gogoladze:2011aa}
\bibitem{Gogoladze:2011aa}
  Gogoladze, I.; Shafi, Q.; and Un, C. S.
  %``Higgs Boson Mass from t-b-$\tau$ Yukawa Unification,''
 {\it JHEP} {\bf 2012} {\it 1208}, 028;
  %\cite{Ajaib:2013zha}
%\bibitem{Ajaib:2013zha}
  Adeel Ajaib, M.; Gogoladze, I.; Shafi, Q.; and Un, C. S.
  %``A Predictive Yukawa Unified SO(10) Model: Higgs and Sparticle Masses,''
  {\it JHEP } {\bf 2013} {\it 1307}, 139.
  %%CITATION = doi:10.1007/JHEP07(2013)139;%%
  %29 citations counted in INSPIRE as of 01 Nov 2016
  
\bibitem{Ibanez:Ross}
  Ibanez, L. E.; and Ross, G. G.
    %``SU(2)-L x U(1) Symmetry Breaking as a Radiative Effect of Supersymmetry Breaking in Guts,''
  {\it Phys.\ Lett.} {\it 110B} {\bf 1982} 215;
%\bibitem{REWSB2}
  Inoue, K.; Kakuto, A.; Komatsu, H.; and Takeshita S.,
  %``Aspects of Grand Unified Models with Softly Broken Supersymmetry,''
  {\it Prog.\ Theor.\ Phys.\ }  {\bf 1982} {\it 68}, 927;
%\bibitem{REWSB3}
  Ibanez, L. E.
  %``Locally Supersymmetric SU(5) Grand Unification,''
  {\it Phys.\ Lett.\ } {\bf 1982} {\it 118B}, 73;
%\bibitem{REWSB4}  
  Ellis, J. R.; Nanopoulos D. V.; and Tamvakis, K.
  %``Grand Unification in Simple Supergravity,''
 {\it Phys.\ Lett.\ } {\bf 1983} {\it 121B}, 123;
% \bibitem{REWSB5}
   Alvarez-Gaume, L.; Polchinski, J.; and Wise, M. B.
  %``Minimal Low-Energy Supergravity,''
 {\it Nucl.\ Phys.\ B} {\bf 1983} {\it 221}, 495.  
 
 %\cite{Nakamura:2010zzi}
\bibitem{Nakamura:2010zzi}
  Nakamura, K. {\it et al.} [Particle Data Group Collaboration],
  %``Review of particle physics,''
  {\it J.\ Phys.\ G} {\bf 2010} {\it 37}, 075021.
  %%CITATION = doi:10.1088/0954-3899/37/7A/075021;%%
  %5619 citations counted in INSPIRE as of 01 Nov 2016
  
 %\cite{Baer:2012by}
\bibitem{Baer:2012by} See for instance;

  H.~Baer, I.~Gogoladze, A.~Mustafayev, S.~Raza and Q.~Shafi,
  %``Sparticle mass spectra from SU(5) SUSY GUT models with $b-\tau$ Yukawa coupling unification,''
  JHEP {\bf 1203}, 047 (2012)
  doi:10.1007/JHEP03(2012)047
  [arXiv:1201.4412 [hep-ph]];
  %%CITATION = doi:10.1007/JHEP03(2012)047;%%
  %13 citations counted in INSPIRE as of 20 Nov 2015
 %\cite{Li:2014xqa}
%\bibitem{Li:2014xqa}
  T.~Li, D.~V.~Nanopoulos, S.~Raza and X.~C.~Wang,
  %``A Realistic Intersecting D6-Brane Model after the First LHC Run,''
  JHEP {\bf 1408}, 128 (2014)
  doi:10.1007/JHEP08(2014)128
  [arXiv:1406.5574 [hep-ph]].
  %%CITATION = doi:10.1007/JHEP08(2014)128;%%
  %2 citations counted in INSPIRE as of 20 Nov 2015  
  
  
  
\bibitem{Belanger:2009ti}
  Belanger, G.; Boudjema, F.; Pukhov, A.; and Singh, R. K.
  %``Constraining the MSSM with universal gaugino masses and implication for searches at the LHC,''
  {\it JHEP } {\bf 2009} {\it 0911}, 026;
 %  \bibitem{SekmenMH}
  Baer, H.; Kraml, S.; Sekmen, S.; and Summy, H.
  {\it JHEP } {\bf 2008} {\it 0803}, 056.

\bibitem{Agashe:2014kda}
  Olive, K. A. {\it et al.} [Particle Data Group Collaboration],
  %``Review of Particle Physics,''
  {\it Chin.\ Phys.\ C } {\bf 2014} {\it 38}, 090001.

%\cite{Aaij:2012nna}
\bibitem{Aaij:2012nna}
  Aaij, R. {\it et al.} [LHCb Collaboration],
  %``First Evidence for the Decay $B_s^0 \to \mu^+ \mu^-$,''
  {\it Phys.\ Rev.\ Lett.\ } {\bf 2013}  {\it 110}, no. 2, 021801.
  %%CITATION = doi:10.1103/PhysRevLett.110.021801;%%
  %401 citations counted in INSPIRE as of 01 Nov 2016

%\cite{Amhis:2012bh}
\bibitem{Amhis:2012bh} 
  Y.~Amhis {\it et al.} [Heavy Flavor Averaging Group Collaboration],
  %``Averages of B-Hadron, C-Hadron, and tau-lepton properties as of early 2012,''
  arXiv:1207.1158 [hep-ex].
  %%CITATION = ARXIV:1207.1158;%%
  %743 citations counted in INSPIRE as of 23 Nov 2016

%\cite{Aad:2012tfa}
\bibitem{Aad:2012tfa} 
  Aad, G.; {\it et al.} [ATLAS Collaboration],
  %``Observation of a new particle in the search for the Standard Model Higgs boson with the ATLAS detector at the LHC,''
  {\it Phys.\ Lett.\ B} {\bf 2012}, {\it 716}, 1-29; 
  %%CITATION = doi:10.1016/j.physletb.2012.08.020;%%
  %6598 citations counted in INSPIRE as of 31 Oct 2016
%\cite{Chatrchyan:2013lba}
%\bibitem{Chatrchyan:2013lba} 
 Chatrchyan, S.; {\it et al.} [CMS Collaboration],
  %``Observation of a new boson with mass near 125 GeV in pp collisions at $\sqrt{s}$ = 7 and 8 TeV,''
  {\it JHEP} {\bf 2013}, {\it 1306}, 081.
  %%CITATION = doi:10.1007/JHEP06(2013)081;%%
  %459 citations counted in INSPIRE as of 31 Oct 2016
  
  \bibitem{TheATLAScollaboration:2015aaa}
  The ATLAS collaboration,
 {\it  ATLAS-CONF-2015-067} {\bf 2015}.
  %%CITATION = ATLAS-CONF-2013-068;%%
  %62 citations counted in INSPIRE as of 01 Nov 2016

%\cite{Gogoladze:2009bd}
\bibitem{Gogoladze:2009bd} 
  I.~Gogoladze, M.~U.~Rehman and Q.~Shafi,
  %``Amelioration of Little Hierarchy Problem in SU(4)(c) x SU(2)(L) x SU(2)(R),''
  Phys.\ Rev.\ D {\bf 80}, 105002 (2009)
  doi:10.1103/PhysRevD.80.105002
  [arXiv:0907.0728 [hep-ph]].
  %%CITATION = doi:10.1103/PhysRevD.80.105002;%%
  %27 citations counted in INSPIRE as of 01 Dec 2016

%\cite{Antusch:2011xz}
\bibitem{Antusch:2011xz} 
  S.~Antusch, L.~Calibbi, V.~Maurer, M.~Monaco and M.~Spinrath,
  %``Naturalness and GUT Scale Yukawa Coupling Ratios in the CMSSM,''
  Phys.\ Rev.\ D {\bf 85}, 035025 (2012)
  doi:10.1103/PhysRevD.85.035025
  [arXiv:1111.6547 [hep-ph]].
  %%CITATION = doi:10.1103/PhysRevD.85.035025;%%
  %23 citations counted in INSPIRE as of 01 Dec 2016



\end{thebibliography}	



\end{document}
