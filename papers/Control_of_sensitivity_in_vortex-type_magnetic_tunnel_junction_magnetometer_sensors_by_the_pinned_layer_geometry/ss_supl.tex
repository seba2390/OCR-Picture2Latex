\section{Magnetic properties and TMR characteristics of the sensor films}
We show the magnetic properties of the blanket film in Fig.~\ref{fig:film_vsm}(a). We optimized the thickness of CoFe in the pinned SAF for zero remnant magnetization, to reduce the effect of stray fields on the vortex core in the free layer. The magnetization reversal process of the full stack is indicated by the colored arrows. Although NiFe is rather thick, the coupling in the free layer between NiFe and CoFeB is antiferromagnetic through the Ru spacer, indicating the uniformity of the Ru ultra-thin layer. Fig.~\ref{fig:film_vsm}(b) shows the TMR characteristics of large pillars after the two-step field annealing process. The MTJ pillars have a 2:1 elliptical cross section, with a major axis of 48 $\mu$m in length. After the first annealing process, we obtained a relatively high TMR ratio of 140--150 $\%$, and the transfer curve shows switching character. After the second annealing step, the pinned layer is rotated orthogonal to free layer easy axis, and a linearized transfer curve is obtained.


\begin{figure}[hb]
 \begin{center}
     \includegraphics[width=0.8\textwidth]{figS1_film_vsm_mr.pdf}
    \caption{(a) Magnetic characteristics of the top pinned SAF (top), and MTJ films (bottom). (b) TMR characteristics after the first (top) and second (bottom) annealing steps.}
    \label{fig:film_vsm}
 \end{center}
\end{figure}

\newpage

\section{Induced magnetic anistropy in NiFe}

The NiFe exhibit an induced magnetic anisotropy due to Ni-Fe pair ordering \cite{sup_chikazumi_1955}. The pair ordering occurs during the deposition in a magnetic field \cite{sup_katada_2000}. In Fig.~\ref{fig:sup_aniso}, we show the results of magnetization loops of blanket films of: thermally-oxidized silicon substrate/Ta 5/Ru 10/Ta 5/Ni$_{80}$Fe$_{20}$ 100/MgO $1.5$/Ta 1.0, where the numbers are the nominal thicknesses in nanometers. We measured the magnetiztion loops along the easy and hard axes in the as-deposited state, and after the first pin annealing with the same conditions as the main text, \emph{i.e.}~$350^\circ\mathrm{C}$ along the easy axis of induced anisotropy in NiFe ($Y$ direction). We obtain {a saturation magnetization of 800 emu/cm$^3$, an anisotropy field of 4 Oe, and an anisotropy energy of $1.6\times 10^3$ erg/cm$^3$,} similar to the literature values \cite{sup_katada_2000}. The anisotropy field is not affected by pin annealing. However, coercivity is reduced and linearity is improved in the hard axis loop.


\begin{figure}[hb]
 \begin{center}
     \includegraphics[width=0.6\textwidth]{figS2_MH_asdepo_ann.pdf}
    \caption{The magnetization loops along the easy ($Y$) and hard ($X$) axes of NiFe films in the as deposited state, and after the first annealing.}
    \label{fig:sup_aniso}
 \end{center}
\end{figure}

\newpage

\section{Critical size for vortex stability}

We show the effect of induced anisotropy on the domain structure in Fig.~\ref{fig:sup_domain}. We fabricated circular disks with varying diameters from the films in Sec.~S2, after the first pin annealing. We used longitudinal MOKE to image the domain structure during magnetization loops with field applied along the easy or hard axes. In large diameter disks, a reversal domain forms at zero field [Fig.~\ref{fig:sup_domain}(a)]. The initial nucleation domains form parallel to easy axis direction. When the anisotropy axis is transverse to the applied field, multiple vortices are formed [indicated by {black} arrows in Fig.~\ref{fig:sup_domain}(a)]. At a critical diameter of 30 $\mu$m, the nucleation starts from a reversal domain in the easy-axis loop, whereas the vortex state nucleates for the hard-axis loop [indicated by an arrow in Fig.~\ref{fig:sup_domain}(b)]. Below that diameter, the vortex state is the stable reversal configuration, regardless of the field direction [Fig.~\ref{fig:sup_domain}(c)].


\begin{figure}[hb]
 \begin{center}
     \includegraphics[width=0.7\textwidth]{figS3_domain.pdf}
    \caption{Longitudinal MOKE domain images of NiFe disks during field scans along the easy and hard axes directions. The disk diameters are: (a) 60 $\mu$m, (b) 30 $\mu$m, and (c) 24 $\mu$m.}
    \label{fig:sup_domain}
 \end{center}
\end{figure}

\newpage
\section{Nucleation features in the TMR-$H$ curves}
{In Figs.~3(a,d), and 4(a--d) of the main text, there are dips that appear near $\pm$50 Oe for small $r_t/R_b$ ratios. They coincide with the nucleation field of the vortex state estimated from the VSM measurements of Fig.~2(a). We show the explanation from simulation results in Fig.~\ref{fig:sup_nuc}. Below the saturation field, and before the nucleation event, a curly domain appears [Fig.~\ref{fig:sup_nuc}(a)]. The magnetization vector becomes flat at the poles of the disk to minimize the magnetostatic energy. This causes a rotation of the magnetization at the center of the disk towards a $\approx 90^\circ$ direction. After vortex nucleation, the magnetization at the center region returns back to $0^\circ$ [Fig.~\ref{fig:sup_nuc}(b)]. If the area enclosed by the pinned layer is small  [orange curve in Fig.~\ref{fig:sup_nuc}(c)], then before nucleation there is a dip in $\Delta m_x$. After nucleation, $\Delta m_x$ increases to 1, until the vortex position is close to the pinned layer edge. If the area enclosed by the pinned layer is large, then the average $\Delta m_x$ is smaller in the vortex state compared to pre-nucleation state [blue curve in Fig.~\ref{fig:sup_nuc}(c)]. }

\begin{figure}[hb]
 \begin{center}
     \includegraphics[width=0.8\textwidth]{figS4_nucleation.pdf}
    \caption{Simulations of the domain state before and after the vortex nucleation. (a) Before vortex nucleation, and (b) after the vortex nucleation. In (a), we show a small area to be enclosed by the pinned layer, equivalent to a projection along the $X$ direction (blue arrow). (c) The $\Delta m_x$--$H_x$ loops for a small enclosed area ($r_t/R_b = 0.2$), or a large one ($r_t/R_b = 1.0$). The states in (a) and (b) are indicated on the curves. }
    \label{fig:sup_nuc}
 \end{center}
\end{figure}

% \clearpage
% \newpage

\begin{thebibliography}{1}
   \bibitem{sup_chikazumi_1955}
    S.~Chikazumi and T.~Oomura, ``On the {{Origin}} of {{Magnetic Anisotropy
      Induced}} by {{Magnetic Annealing}},'' \emph{Journal of the Physical Society
      of Japan}, vol.~10, no.~10, pp. 842--849, Oct. 1955.
    
    \bibitem{sup_katada_2000}
    H.~Katada, T.~Shimatsu, I.~Watanabe, H.~Muraoka, Y.~Sugita, and Y.~Nakamura,
      ``Induced uniaxial magnetic anisotropy field in very thin {{NiFe}} and
      {{CoZrNb}} films,'' \emph{IEEE Transactions on Magnetics}, vol.~36, no.~5,
      pp. 2905--2908, Sep. 2000.

\end{thebibliography}