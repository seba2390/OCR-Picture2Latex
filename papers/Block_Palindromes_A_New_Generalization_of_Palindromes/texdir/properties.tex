\section{Properties of Block Palindromes}
In this section, we investigate the properties of block palindromes.
We assume that $\T$ is an input string of length $\NN$ in the rest of the paper.

Since there are $O(2^\NN)$ factorization of $\T$ and block palindromes are symmetric, there are $O(2^{\NN/2})$ block palindromes of $\T$.
Moreover, there is a tight example that $\T$ consists of only the same characters.

Although there are a huge number of block palindromes of a string, they are very redundant.
To look for more essential properties of block palindromes, we define
the \emph{largest block palindrome} which is a representative of other block palindromes.
A block palindrome $\ff=\ff_{-\nn} \cdots \ff_{\nn}$ of $\T$ that has
the largest number of blocks among all block palindromes of $\T$ is called the largest block palindrome.
Note that each block $\ff_\ii$ for $0 \leq \ii \leq \nn$ is an unbordered substring and $\ff_\ii$ for $0 < \ii \leq \nn$ is  the shortest border of $\T[\kk + 1 \ldots \NN - \kk]$, where $\kk=0$ if $\ii=\nn$ and $\kk=|\ff_{\ii+1} \cdots \ff_\nn|$ otherwise.
So, the largest block palindrome of $\T$ is unique.
The largest block palindrome is a representative of all block palindromes in the sense that all block palindromes can be represented as block palindromes of $\ff$.

A natural and prompt question would be about how to efficiently compute the largest block palindrome of $\T$.
The following theorem answers this question.

\begin{theorem}\label{theorem:largest}
	The largest block palindrome $\ff_{-\nn} \cdots \ff_{\nn}$ of $\T$ can be computed in $O(\NN)$ time.
\end{theorem}
\begin{proof}
	We construct a data structure in $O(\NN)$ time that can answer any LCE query in constant time.

	We greedily compute the blocks from outside $\ff_{\nn}$ to inner $\ff_{1}$ by LCE queries.
	We assume that we compute the shortest border $\ff_{\ii}$ of $\T[\bb \ldots \ee]$.
	For $\kk=1$ to $\floor{(\ee - \bb + 1)/2}$, we check whether $\T[\bb \ldots \bb + \kk - 1]$ is the border of $\T[\bb \ldots \ee]$ or not by checking whether $\LCE(\bb, \ee-\kk+1) \geq \kk$ or not.
	If $\T[\bb \ldots \ee]$ does not have any border, we obtain $\ff_0 = \T[\bb \ldots \ee]$.
	Otherwise, we obtain the shortest border $\ff_\ii=\T[\bb \ldots \bb + \kk-1]$ of $\T[\bb \ldots \ee]$, and compute the more inner blocks for $\T[\bb + \kk \ldots \ee - \kk]$.
	Since the number of LCE queries is $O(\NN)$ and each LCE query takes constant time, the largest block palindrome of $\T$ can be computed in $O(\NN)$ time.
\qed
\end{proof}

So far, we have considered only block palindromes that are equal to $\T$ itself.
Next, we consider block palindromes that appear as substrings in $\T$.
We define a \emph{maximal block palindrome} which is a representative of some block palindromes in $\T$, and study how many maximal block palindromes can appear in $\T$.

For a half-position $1 \leq \cc \leq \NN$ and an integer $1 \leq \dd \leq \NN/2$, let $\FF_\T(\cc, \dd)=\{\ff | \ff=\ff_{-\nn} \cdots \ff_0 \cdots \ff_{\nn} \mbox{ is the largest block palindrome}, \ff_0=\T[\cc - \dd + 1 \ldots \cc + \dd-1], \ff=\T[\cc - \dd - \kk+1 \ldots \cc + \dd + \kk-1], \kk = |\ff_1 \cdots \ff_\nn| \}$ be the set of largest block palindromes whose center positions are the same and whose center blocks appear at $\T[\cc-\dd+1 \ldots \cc+\dd-1]$.
When context is clear, we denote $\FF_{\T}$ by $\FF$.
For a string $\T$, a largest block palindrome $\ff \in \FF(\cc, \dd)$ such that $|\ff|$ is the longest, namely the number of blocks are maximal among all largest block palindromes of $\FF(\cc, \dd)$, is called a \textit{maximal block palindrome}.


We remark that the maximal block palindrome of $\FF(\cc, \dd)$ is a representative of all the largest block palindromes of $\FF(\cc, \dd)$.

\begin{remarkx}
	\label{lem:maximal-palindrome-is-representative}
  For a half-position $1 \leq \cc \leq \NN$ and an integer $1 \leq \dd \leq \NN/2$, any largest block palindrome $\ff=\ff_{-\nn} \cdots \ff_{\nn} \in \FF(\cc, \dd)$ is a sub-factorization of the maximal block palindrome $\vg = \vg_{-\mm} \cdots \vg_{\mm} \in \FF(\cc, \dd)$, that is, $\nn \leq \mm$ and $\ff_{\ii}=\vg_{\ii}$ for $0 \leq \ii \leq \nn$.
\end{remarkx}
\begin{proof}
	We assume that the statement does not hold.
	Let $\ff_\jj$ be a block that $\ff_{\jj} \neq \vg_{\jj}$, and $\jj=0$ or $\ff_{\ii}=\vg_{\ii}$ for $0 \leq \ii < \jj \leq \nn$.
	If $|\ff_{\jj}| < |\vg_{\jj}|$, $\ff_{\jj}$ is a border of $\vg_{\jj}$ and it contradicts that $\vg_{\jj}$ is the largest block palindrome.
	We can say the same things for the case $|\ff_{\jj}| > |\vg_{\jj}|$.
	Therefore, such $\ff_\jj$ and $\vg_\jj$ do not exist and this statement holds.
	\qed
\end{proof}

We are interested in how many maximal block palindromes can appear in $\T$.
It is trivially upper bounded by $O(\NN^2)$ since there are $O(\NN^2)$ substrings which can be center substrings.
If there is no limitation on the size of maximal block palindromes, we can easily see that it is tight.
For a string $\T$ of length $\NN$ in which the characters are all distinct, any substring $\ww$ is unbordered, and there is at least one maximal block palindrome that contains $\ww$ as a center block.
Thus, $\T$ can contain $\Theta(\NN^2)$ maximal block palindromes.
The following example says that the number of maximal block palindromes having three blocks has also the same tight upper bound.

\begin{examplex}
	\label{lem:bound-of-num-maximal-palindromes}
	The number of maximal block palindromes in $\T=\mathtt{a}^n\mathtt{b}^n\mathtt{a}\mathtt{b}\mathtt{a}^n\mathtt{b}^n$ that have at least three blocks is $\Theta(\NN^2)$, where $\cc^\nn$ for a character $\cc$ denotes run of $\cc$ of length $\nn$ , and $\nn=(\NN-2)/4$.
\end{examplex}
For convenience, we denote $\T$ by $\T=\vA_0 \vB_1 \vA_1 \vB_2 \vA_2 \vB_3$, where $\vA_0$, $\vB_1$, $\vA_1$, $\vB_2$, $\vA_2$, and $\vB_3$ are strings $\mathtt{a}^n$, $\mathtt{b}^n$, $\mathtt{a}$, $\mathtt{b}$, $\mathtt{a}^n$, and $\mathtt{b}^n$, respectively.
There are maximal block palindromes of size three that, for $1<\ii \leq \nn$, $1<\jj \leq \nn$, $\T[\nn-\jj+1 \ldots \NN-\nn+\ii-1]$=$(\vA_0[\nn-\jj+1 \ldots \nn] \vB_1[1..\ii-1])(\vB_1[\ii \ldots \nn]\vA_1 \vB_2 \vA_2[1 \ldots \jj])(\vA_2[\nn-\jj+1 \ldots \nn]\vB_3[1 \ldots \ii-1])$ and they are unbordered, where the parentheses indicate blocks.


%% If we consider only maximal block palindromes of even size, whose center block must be empty,
%% the number of occurrences of center blocks are at most $\NN-1$,
%% and thus, $\T$ can contain $\Theta(\NN)$ maximal block palindromes of even size.

Remark that the upper bound is reduced to $O(\NN)$ if we impose a limitation on the lengths of center blocks.
\begin{remarkx}
	\label{lem:limited-center-block}
  For any constant $\kk \ge 0$, a string of length $\NN$ can contain $\Theta(\NN)$
  maximal block palindromes whose center blocks are of length $\le \kk$ because there are $O(\NN)$ possible center blocks.
  In particular, a string contains at most $\NN - 1$ maximal block palindromes of even size (i.e., the center blocks must be empty)
  because the number of occurrences of center blocks are at most $\NN-1$.
\end{remarkx}



The following lemma shows an interesting property of maximal block palindromes, and this property can be used for the proof of Lemma~\ref{lem:size-maximal-block-palindrome}.
\begin{lemma}
	\label{lem:there-is-no-same-starting-positions-of-factors}
  For a half-position $1 \leq \cc \leq \NN$ and two integers $1 \leq \dd<\dd^\prime \leq \NN/2$, two largest block palindromes $\ff=\ff_{-\nn} \cdots \ff_{\nn} \in \FF(\cc, \dd)$ and $\vg = \vg_{-\mm} \cdots \vg_{\mm} \in \FF(\cc, \dd^\prime)$ do not share the block boundaries, namely, the ending positions of blocks $\kk_\ii$ and $\kk^\prime_\ii$ such that $\kk_\ii=\cc + \dd -1 + |\ff_1 \cdots \ff_\ii|$ and $\kk^\prime_\ii=\cc + \dd^\prime -1 + |\vg_1 \cdots \vg_\jj|$ do not equal for any $1 \leq \ii \leq \nn$ and $1 \leq \jj \leq \mm$.
\end{lemma}
\begin{proof}
  Similar to Remark~\ref{lem:maximal-palindrome-is-representative}, if we assume that this lemma does not hold, a block of $\ff$ or $\vg$ must have a border and it contradicts that $\ff$ and $\vg$ are the largest block palindromes.
	\qed
\end{proof}

Let $\|\MBP(\T)\|$ denote the sum of the sizes of all maximal block palindromes in $\T$.
\begin{lemma}
  \label{lem:size-maximal-block-palindrome}
  For any string $\T$ of length $\NN$, $\|\MBP(\T)\| \le \NN(2\NN-1)$.
\end{lemma}
\begin{proof}
  From Lemma~\ref{lem:there-is-no-same-starting-positions-of-factors}, any two largest block palindromes, whose center positions
  are same but center blocks are different, do not share the block boundaries.
  This implies that, for a half-position $\cc$, the number of blocks of maximal block palindromes whose center position is $\cc$ is up to $\NN$.
  Since there are $2\NN-1$ center positions, we have $\|\MBP(\T)\| \le \NN(2\NN-1)$.
	\qed
\end{proof}
