\section{Introduction}
A palindrome is a string that is equal to its reverse, e.g., ``\texttt{Able_was_I_ere_I_saw_Elba}'' (we treat upper and lower characters are the same for simple explanations).
Palindromes have been studied in combinatorics on words and stringology.

Many research focused on finding palindromic structure of a string.
Manacher~\cite{Manacher75} proposed a beautiful algorithm that enumerates all maximal palindromes of a string.
Kosolobov et al.~\cite{KosolobovRS15} proved that, a language $\PP^\kk$ can be recognizable in $O(\kk \NN)$ time, where $\PP$ is the language of all nonempty palindromes and $\NN$ is the length of an input string.
Alatabbi et al.~\cite{Alatabbi13} considered maximal palindromic factorization in which all factors are maximal palindromes.
They also consider a problem of computing the fewest palindromic factorization, and proposed off-line linear-time algorithms.
Later, I et al.~\cite{ISIBT14} and Fici et al.~\cite{FiciGKK14} independently proposed on-line $O(\NN \log \NN)$-time algorithms, where $\NN$ is the length of an input string.
Similar problems were also considered, such as, computing palindromic length~\cite{BorozdinKRS17}, computing palindromic covers~\cite{ISIBT14}, computing palindromic pattern matching~\cite{I2013}.

A gapped palindrome is a generalization of a palindrome that becomes a palindrome when a center substring is replaced by a character, where the center substring is a substring whose beginning and ending positions are equally far from the beginning and ending positions of the input string, respectively.
For example, ``\texttt{Madam,_he_is_Adam}'' is a gapped palindrome, and it becomes a palindrome if the center substring ``\texttt{m,_he_is_}'' is replaced by a character.
Gapped palindromes play an important role in molecular biology since they model a hairpin data structure of DNA and RNA sequences, see e.g.~\cite{Gerald08}.
Several problems were considered such as, enumeration of exact gapped palindromes of a string~\cite{KolpakovK09} and also enumeration of approximate gapped palindromes~\cite{NarisadaDNIS17,HsuCC10}, finding maximal length of long armed or and constrained length gapped palindrome~\cite{Gupta2016}.

In this paper, we consider the notion of block palindromes~\cite{BIO2015}, which is
a new generalization of palindromes and also gapped palindromes~\footnote{Block palindromes were firstly introduced in a problem of 2015 British Informatics Olympiad~\cite{BIO2015}, but we did not know the existence at the first version of this paper.}.
A block palindrome is a string that becomes a palindrome when identical substrings are replaced with a distinct character.
More precisely, a block palindrome is a ``symmetric'' factorization $\ff=\ff_{-\nn} \cdots \ff_{-1} \ff_0 \ff_1 \cdots \ff_\nn$ of a string with the center factor $\ff_0$ is a string (which may be empty) and each of other factor $\ff_{-\ii}=\ff_{\ii}$ is a non-empty string for any $1 \le \ii \le \nn$.
We also call a factor a block.
For convenience, let $\ff=\ff_0$ when $\nn=0$.
For example, a factorization ``\texttt{To|kyo|_|and|_|Kyo|to}'' is a block palindrome, where ``\texttt{|}'' is a mark to distinguish adjacent blocks.
Palindromes and gapped palindromes are special cases of block palindromes:
For a palindrome, all blocks are characters, and for a gapped palindrome, the center block $\ff_0$ is a string and the other blocks are characters.

We investigate several properties of block palindromes.
We introduce the notion of maximal block palindromes to concisely represent all block palindromes in a string,
and propose an algorithm which enumerates all maximal block palindromes in a string $\T$ in $O(|\T| + \|\MBP(\T)\|)$ time, where $\|\MBP(\T)\|$ is the output size. This is optimal unless all the maximal block palindromes can be represented in a more compact way.
