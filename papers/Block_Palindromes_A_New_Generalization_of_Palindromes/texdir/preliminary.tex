\section{Preliminaries}

Let $\Sigma$ be an integer alphabet.
An element of $\Sigma^*$ is called a \emph{string}.
The string of length 0 is called the \emph{empty} string,
and is denoted by $\varepsilon$.
Although $\varepsilon$ is not contained in $\Sigma$,
we sometimes call $\varepsilon$ the empty character for convenience.
For a string $\T=\xx \yy \zz$, $\xx$, $\yy$ and $\zz$ are called a \emph{prefix}, \emph{substring}, and \emph{suffix} of $\T$, respectively.
In particular, a prefix (resp. suffix) $\xx$ of $\T$ is called a \emph{proper} prefix (resp. suffix) iff $\xx \neq \T$.
A non-empty string that is a proper prefix and also a proper suffix of $\T$ is called a \emph{border} of $\T$.
Hence, a string of length $\NN$ can have
at most $\NN-1$ borders of length ranging from 1 to $\NN-1$.
A string which does not have any borders is called an \emph{unbordered} string.
%
For $1 \leq \ii \leq \jj \leq |\T|$,
a substring of $\T$ which begins at position $\ii$ and ends at
position $\jj$ is denoted by $\T[\ii \ldots \jj]$.
For convenience, let $\T[\ii \ldots \jj] = \varepsilon$ if $\jj < \ii$.

In this paper, we also consider \emph{half-positions} $\kk+1/2$ for integers $0 \leq \kk \leq |\T|$.
For convenience, for a half-position $\ii$ and an integer $\rr$ such that $1/2 \leq \ii-\rr \leq \ii + \rr \leq |\T|+1/2$, let $\T[\ii-\rr \ldots \ii+\rr] = \T[\ceil{\ii-\rr} \ldots \floor{\ii+\rr}]$.
Note that $\T[\ii]$ for a half-position $\ii$ is the empty character.
The position $\cc=(|\T|+1)/2$ is called the \emph{center position} of $\T$,  $\T[\cc]$ is called the \emph{center character} of $\T$, and $\T[\cc -\dd \ldots \cc+\dd]$ for an integer $\dd$ is  called a \emph{center substring} of $\T$.

For a string $\T$ and integers $1 \leq \ii, \jj \leq |\T|$, a \emph{longest common extension} (LCE) query $\LCE_\T(\ii, \jj)$ asks the length of the longest common prefix of the two suffixes $\T[\ii \ldots |\T|]$ and $\T[\jj \ldots |\T|]$ of $\T$.
When clear from the context, $\LCE_\T(\ii, \jj)$ is abbreviated as $\LCE(\ii, \jj)$.
It is well known that if $\T$ is drawn from an integer alphabet of size polynomially bounded in $|\T|$, then 
LCE queries for $\T$ can be answered in constant time
after an $O(|\T|)$-time preprocessing, e.g.,
by constructing the suffix tree of $\T$ and a data structure for lowest common ancestor queries on the tree~\cite{Gusfield1997AST}.

For a block palindrome $\ff=\ff_{-\nn} \cdots \ff_{-1} \ff_0 \ff_1 \cdots \ff_\nn$, the length of $\ff$ denoted by $|\ff|$ is the total length of blocks, and the size of $\ff$ denoted by $\|\ff\|$ is the number of non-empty blocks.
A block palindrome is \emph{even} if its size is even
(that is, the center block $\ff_0$ is the empty string),
and otherwise \emph{odd} (that is, the center block $\ff_0$ is non-empty).
