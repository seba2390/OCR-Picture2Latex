
% \begin{figure*}[htp]

% \centering
% \includegraphics[width=.5\textwidth]{figures/binary.pdf}\hfill
% \includegraphics[width=.5\textwidth]{figures/multi-class.pdf}\hfill
% \includegraphics[width=.5\textwidth]{figures/regression.pdf}

% \caption{default}
% \label{fig:figure3}

% \end{figure*}



\begin{figure*}
\ncp\ncp
\centering
  \includegraphics[width=0.95\textwidth]{figures/1h_77_datasets_all.pdf}
  \ncp\ncp\ncp\ncp\ncp\ncp\ncp\ncp
  \caption{A radar diagram of the performance of {\sysname} vs. existing systems on multiple tasks (77 datasets) with a time budget of 1 hour for all systems. The outer numbers indicate  different dataset IDs and the ticks inside the figure denote performance ranges of respective metrics; e.g., 0.2, 0.4, ..., etc. for F1 in binary classification. For any dataset, the system with the out most curve has the best performance. As an example, KGpipAutoSklearn and KGpipFLAML achieved 100\% and 97\% F1 on dataset \#23 (multi-class classification) compared to 65\% and 26\% for AutoSklearn and FLAML, respectively. }
  \label{1hr_exps}
  \ncp\ncp\ncp\ncp
\end{figure*}


%%%%%%%%%%% The figure below is for AL %%%%%%%%%%%%%%

% \begin{figure}
% \ncp\ncp
% %  \centering
%   \includegraphics[width=\columnwidth]{figures_raw/avg_1h_updated_AL_only.pdf}
%   \ncp\ncp\ncp\ncp\ncp\ncp\ncp\ncp
%   \caption{Performance over 1 hour for {\sysname} vs. Auto-Sklearn,  FLAML and AL \textit{on the datasets on which AL worked}.}
%   \label{1hr_AL_exps}
%   \ncp\ncp\ncp\ncp\ncp\ncp\ncp
% \end{figure}