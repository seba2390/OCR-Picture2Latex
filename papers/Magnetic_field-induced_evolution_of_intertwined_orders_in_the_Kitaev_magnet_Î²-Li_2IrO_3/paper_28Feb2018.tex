\documentclass[prx,aps,groupedaddress,twocolumn]{revtex4-1}
%\documentclass[prb,aps,superscriptaddress,twocolumn,longbibliography]{revtex4-1}
\usepackage{graphicx}% Include figure files
\usepackage{dcolumn}% Align table columns on decimal point
\usepackage{bm}% bold math
\usepackage{color}

\usepackage{epstopdf}

\usepackage{amsmath, amssymb, graphics, setspace}
\DeclareMathOperator{\Tr}{Tr}

\newcommand{\mathsym}[1]{{}}
\newcommand{\unicode}[1]{{}}

\newcounter{mathematicapage}
%\usepackage{setspace}%Double space

\usepackage{titlesec}
\usepackage{bm}
\usepackage{comment}
\usepackage{verbatim}

\usepackage{graphicx}
\usepackage{subfigure}
\usepackage{tabularx}

\usepackage{color}
\usepackage[colorlinks,bookmarks=false,citecolor=blue,linkcolor=red,urlcolor=blue]{hyperref}
\def\cred{\color{red}}
\definecolor{darkred}{rgb}{0.7,0.0,0.0}
%\definecolor{darkred}{rgb}{0.45,0.02,0} 
\def\cbl{\color{blue}}
\definecolor{darkblue}{rgb}{0,0.02,0.45}
\def\cdbl{\color{darkblue}}
\definecolor{darkgreen}{rgb}{0.02,0.45,0.0} 
\def\cgr{\color{darkgreen}}
\def\magen{\color{magenta}}
\definecolor{violet}{rgb}{0.8,0.2,0.6}
\def\vio{\color{violet}}

\def\ua{\uparrow}
\def\da{\downarrow}
\def\uua{\Uparrow}
\def\dda{\Downarrow}

\def\be{\begin{equation}}
\def\ee{\end{equation}}
\def\bea{\begin{eqnarray}}
\def\eea{\end{eqnarray}}

\def\vare{\varepsilon}
\def\vec{\mathbf}
\def\bs{\boldsymbol}
\def\mc{\mathcal}

%\nofiles
%\doublespacing

\usepackage{times}

\begin{document}
\title{Magnetic field-induced evolution of intertwined orders in the  Kitaev magnet $\beta$-Li$_2$IrO$_3$}
%\title{Field-induced evolution of intertwined orders in the hyperhoneycomb Kitaev magnet $\beta$-Li$_2$IrO$_3$}
%\title{Magnetic field-induced instability in the hyperhoneycomb Kitaev magnet $\beta$-Li$_2$IrO$_3$}

\author{Ioannis Rousochatzakis}
\affiliation{School of Physics and Astronomy, University of Minnesota, Minneapolis,
MN 55116, USA}

\author{Natalia  B. Perkins}
\affiliation{School of Physics and Astronomy, University of Minnesota, Minneapolis,
MN 55116, USA}

\begin{abstract}
Recent scattering experiments in the 3D Kitaev magnet $\beta$-Li$_2$IrO$_3$ have shown that a relatively weak magnetic field along the crystallographic ${\bf b}$-axis drives the system from its incommensurate counter-rotating order to a correlated paramagnet, with a significant uniform `zigzag' component superimposing the magnetization along the field. 
%
Here it is shown that the zigzag order is not emerging from its linear coupling to the field (via a staggered, off-diagonal element of the ${\bf g}$-tensor), but from its intertwining with the incommensurate order and the longitudinal magnetization. 
%
The emerging picture explains all qualitative experimental findings at zero and finite fields, including the rapid decline of the incommensurate order with field and the so-called intensity sum rule. 
%
The latter are shown to be independent signatures of the smallness of the Heisenberg exchange $J$, compared to the Kitaev coupling $K$ and the off-diagonal  anisotropy $\Gamma$. 
%
Remarkably, in the regime of interest, the field $H^\ast$ at which the incommensurate component vanishes, depends essentially only on $J$, which allows to extract an estimate of $J\!\simeq\!4K$ from reported measurements of $H^\ast$.
%
We also comment on recent experiments in pressurized $\beta$-Li$_2$IrO$_3$ and conclude that $J$ decreases with pressure.
\end{abstract}
\maketitle

%\vspace*{-0.25cm} 
\section{Introduction}
\vspace*{-0.3cm}
%{\cdbl {\it Introduction}} -- 
The realization~\cite{Jackeli2009,Jackeli2010} that certain correlated materials based on 4d and 5d transition metals, like Ir$^{4+}$ or Ru$^{3+}$, host the key microscopic ingredients for the so-called Kitaev spin liquid~\cite{Kitaev2006} has spurred tremendous experimental and theoretical interest in the last decade~\cite{BookCao,Balents2014,Rau2016,Trebst2017,Hermanns2017,Winter2017}. 
%
A recurring theme in this research is that the predicted quantum spin liquids~\cite{Kitaev2006,Mandal2009,Kimchi2014,Hermanns2016} are fragile against various realistic perturbations~\cite{Jackeli2010,Schaffer2012,Jackeli2013,Lee2014,Katukuri2014,Katukuri2015,IoannisK1K2,Satoshi2016} and are preempted by magnetic order at low enough temperatures~\cite{
%Na213
Singh2010, Singh2012,Liu2011,
%RuCl3
Sears2015,Johnson2015,
%a-Li13
Williams2016,
%
%gamma-213
Biffin2014b, 
%beta-213
Modic2014,Biffin2014a,Takayama2015}. 
%
Nevertheless, there is overwhelming evidence that external perturbations, such as magnetic field~\cite{
Baek2017,Yadav2016,Sears2017,Zheng2017,Hentrich2017,Kataev2017,%a-RuCl3
Ruiz2017,Tsirlin2018,%beta-213
Modic2017,%gamma-213
Chern2017,Janssen2016,Vojta2017,Winter2018%theory
}, chemical substitution~\cite{Takagi2018}, or pressure~\cite{
Takayama2015,Veiga2017,Tsirlin2018,%beta-213
Breznay2017,%gamma-213
Bastien2018,Biesner2018,%a-RuCl3
KimKimKee2016%theory - ab initio
}, can drive these materials to various types of correlated phases, including spin liquids. 
%
To go forward, it is therefore crucial to map out the most relevant instabilities, and identify their distinctive experimental fingerprints.



In this vein, we study the enigmatic magnetic-field induced instability reported recently by Ruiz {\it et al}~\cite{Ruiz2017} in the 3D hyperhoneycomb iridate $\beta$-Li$_2$IrO$_3$~\cite{Takayama2015,Biffin2014a}. 
%
At zero field, this magnet was known~\cite{Biffin2014a} to develop (below $T_N\!=\!37$ K) a counter-rotating incommensurate (IC) modulation, similar with those in the 3D stripy-honeycomb $\gamma$-Li$_2$IrO$_3$~\cite{Biffin2014b,Modic2014} and the layered honeycomb $\alpha$-Li$_2$IrO$_3$~\cite{Williams2016}.
%
The new findings from the magnetic resonant X-ray scattering data under a finite field are the following~\cite{Ruiz2017}:
%
(i) The IC order of $\beta$-Li$_2$IrO$_3$ is very fragile against a magnetic field along the crystallographic ${\bf b}$-axis, and disappears completely at a characteristic field $H^\ast$.  
%
(ii) The system develops a significant uniform `zigzag' component along ${\bf a}$ (superimposing the magnetization along ${\bf b}$), similar to the zigzag order of Na$_2$IrO$_3$~\cite{Singh2010,Singh2012,Liu2011,Choi2012,Ye2012,Chun2015} and $\alpha$-RuCl$_3$~\cite{Plumb2014,Sears2015,Kubota2015,Majumder2015,Banerjee2016, Johnson2015}.
%
(iii) The zigzag component grows linearly with field until it shows a kink at $H^\ast$, but is otherwise undetectable at zero field, consistent with the experiments of Biffin {\it et al}~\cite{Biffin2014a}. 
%
(iv) Quite surprisingly, the sum of the intensities of the Bragg peaks associated with the IC and the uniform components (rescaled by some factor) remains constant up to a field slightly larger than $H^\ast$.

One way to rationalize the appearance of the zigzag component is to build on the insight that a field along ${\bf b}$ couples linearly not only to the uniform magnetization along ${\bf b}$, but also to the zigzag component along ${\bf a}$, by virtue of the off-diagonal element $g_{ab}$ of the ${\bf g}$-tensor.~\cite{Ruiz2017}  This would explain the growth of the zigzag component (besides the magnetization along ${\bf b}$) at the expense of the IC order. However, this picture of a field-induced zigzag order cannot readily explain the significant zigzag amplitude at $H^\ast$, the intensity sum rule (point (iv) above), as well as the $T$-dependence of the intensity of the zigzag component, which is very similar to that of an order parameter~\cite{Ruiz2017}. 


The results presented below reveal that a more consistent scenario is that the IC counter-rotating component, the zigzag component along ${\bf a}$ and the magnetization along ${\bf b}$ are intertwined components of the same order.  
%
It is shown in particular, that the significant growth of the zigzag component with field occurs even in the absence of the off-diagonal element $g_{ab}$. This demonstrates that the zigzag component does not originate in its linear coupling to the field, but rather in an intrinsic coupling with the IC order and the magnetization along ${\bf b}$. 
%
In fact, as shown in Ref.~[\onlinecite{Sam2018}], both the zigzag component and the magnetization along ${\bf b}$ are already present at zero field, albeit with an amplitude that is too weak to be detected. 

The emerging picture explains all the qualitative experimental  results at both zero~\cite{Biffin2014a} and finite fields~\cite{Ruiz2017}. First, the weak zigzag amplitude at zero-field, the rapid decline of the IC order with field, and the intensity sum rule are all facets of the same fact. Namely, that the Heisenberg coupling $J$ is much weaker than both the Kitaev interaction $K$ and the off-diagonal exchange anisotropy $\Gamma$. Second, the characteristic field $H^\ast$ is essentially independent of the dominant interactions $K$ and $\Gamma$ and scales linearly with $J$. This allows to deduce an estimate of $J\!\simeq\!4$\,K from reported data of $H^\ast$. Furthermore, a comparison with recent experiments under pressure~\cite{Tsirlin2018} suggests that $J$ decreases (below 4\,K) with pressure. 



The present work builds on the recent study by Ducatman {\it et al}~\cite{Sam2018}, which is based on the intuitive idea that the observed IC order~\cite{Biffin2014a} can be thought of as a long-wavelength twisting of a nearby commensurate order, called the `$K$-state' [see Fig.~\ref{fig:PT}\,(a)], with the same qualitative features. Namely, the same propagation vector (with periodicity very close to the experimental value $0.57$), the same irreducible representation, the counter-rotating moments, and non-coplanarity. The present study essentially addresses the fate of this $K$-state under a field along the ${\bf b}$-axis. But first, let us repeat the most important aspects of the system and the zero-field $K$-state.~\cite{Sam2018}


%%%%%%%%%%%%%%%%%%%%%%%%%%%%%%%%%%%%%
\vspace*{-0.5cm} 
\section{Lattice structure and microscopic spin model}
%\section{Main aspects of $\beta$-Li$_2$IrO$_3$}
\vspace*{-0.3cm}
%{\cdbl {\it Lattice structure and microscopic spin model}} --
The lattice structure of $\beta$-Li$_2$IrO$_3$ has been discussed in detail elsewhere.~\cite{Biffin2014a,Lee2015,Lee2016,Ruiz2017,Sam2018}
%
The Ir ions form interwoven networks of two types of `zig-zag' chains, one propagating along $\hat{{\bf a}}$+$\hat{{\bf b}}$ and the other along $\hat{{\bf a}}$-$\hat{{\bf b}}$. 
%
The first type of chains comprise the nearest-neighbor (NN) bonds denoted by $x$ and $y$ in Fig.~\ref{fig:PT}\,(b), while the second type comprise the NN bonds denoted by $x'$ and $y'$. We shall refer to these as $xy$- and $x'y'$-chains, respectively.
%
Finally, the two chain types are connected via NN bonds that are oriented along the $\hat{{\bf c}}$-axis and are denoted by $z$ in Fig.~\ref{fig:PT}\,(b).


The microscopic $J$-$K$-$\Gamma$ model, introduced by Lee {\it et al},~\cite{Lee2015,Lee2016} features three types of NN interactions, the Heisenberg exchange $J$, the Kitaev anisotropy $K$ and the off-diagonal symmetric anisotropy $\Gamma$. For a given bond of type $t$, between NN sites $i$ and $j$, the total interaction takes the form
\be
\mc{H}_{ij}^{(t)} = J \vec{S}_i\cdot \vec{S}_j
+ K S_i^{\alpha_t} S_j^{\alpha_t} 
+\sigma_t \Gamma (S_i^{\beta_t} S_j^{\gamma_t}+S_i^{\beta_t} S_j^{\gamma_t})\,,
\ee
%
where $(\alpha_t,\beta_t,\gamma_t)\!=\!(x,y,z)$ for $t\!\in\!\{x,x'\}$, 
$(y,z,x)$ for $t\!\in\!\{y,y'\}$, and $(z,x,y)$ for $t\!=\!z$. 
%
The prefactor $\sigma_t$ of the $\Gamma$ terms is $+1$ for $t\!\in\!\{x, z, y'\}$ and $-1$ for $t\!\in\!\{y,x'\}$.
%
This modulation of the prefactors is tied to the following convention for the relation between the crystallographic axes $\{\hat{{\bf a}}, \hat{{\bf b}},\hat{{\bf c}}\}$ and the Cartesian axes $\{\hat{{\bf x}}, \hat{{\bf y}}, \hat{{\bf z}}\}$~\footnote{From these relations it follows, in particular, that the $\hat{\bf b}$-axis is special because the Kitaev exchange on the $z$-type of bonds couples the spin projections along precisely this axis.~\cite{Ruiz2017}}:
\be
\label{eq:abc}
\begin{array}{c}
\hat{{\bf x}}=\frac{\hat{{\bf a}}+\hat{{\bf c}}}{\sqrt{2}}, ~
\hat{{\bf y}}=\frac{\hat{{\bf c}}-\hat{{\bf a}}}{\sqrt{2}}, ~
\hat{{\bf z}}=-\hat{{\bf b}}.
\end{array}
\ee
The total Hamiltonian in a field ${\bf H}$ takes the form
\be
\begin{array}{c}
\mc{H}=\sum_t \sum_{\langle ij\rangle \in t} \mc{H}_{ij}^{(t)} - \mu_B {\bf H} \cdot \sum_{i} {\bf g}_i \cdot {\bf S}_i\,,
\end{array}
\ee 
where $\mu_B$ is the Bohr magneton and ${\bf g}_i$ is the electronic ${\bf g}$-tensor at site $i$.
%
In the following, we work in units of $\sqrt{J^2\!+\!K^2\!+\!\Gamma^2}\!=\!1$ and use the parametrization~\cite{Lee2015,Lee2016,Sam2018}:
\be\label{eq:rphi}
\begin{array}{c}
J=\sin r \cos\phi, ~
K=\sin r \sin\phi, ~
\Gamma = -\cos r\,,
\end{array}\ee
where $\phi\!\in\![0,2\pi)$, $r\!\in\![0,\pi/2]$. 
%
In particular, we shall focus on the `$K$-region' of Fig.~\ref{fig:PT}\,(a). As argued in Ref.~\cite{Sam2018}, the actual IC order inside this region can be thought of as a long-wavelength twisting of a nearby commensurate state, called the `$K$-state'. This state is a local minimum of the energy, except at $\phi\!=\!3\pi/2$ where it is one of the global minima (which form an $\mc{S}^2$ manifold), and probably survives as such in a small finite window of $\phi$ above $3\pi/2$ due to the lattice cut-off.~\cite{Sam2018} Nevertheless, this state has all qualitative features of the experimentally observed IC order at zero field~\cite{Sam2018}.


\begin{figure}[!t]
\includegraphics[width=1\columnwidth]{Fig1}
\vspace*{-0.25cm}
\caption{
(a) The relevant parameter space in $(r,\phi)$ [see Eq.~(\ref{eq:rphi})], along with the $K$- and $\Gamma$-regions proposed in Ref.~[\onlinecite{Sam2018}]. 
%
The points $A$ and $B$, referred to in subsequent figures, correspond to $(r,\phi)\!=\!(0.32\pi,1.52\pi)$ and $(0.32\pi,1.625\pi)$. 
%
(b-c) The structure of the $K$-state at low ($H\!\le\!H^\ast$) and high ($H\!\ge\!H^\ast$) fields. The solid (dashed) green and red bonds depict the $xy$ ($x'y'$) chains running along $\hat{{\bf a}}$+$\hat{{\bf b}}$ ($\hat{{\bf a}}$-$\hat{{\bf b}}$). The blue vertical segments point along the ${\bf c}$-axis and depict the $z$-bonds.
%
The Cartesian components of the spins are shown in the side panels. 
%
Both states respect the combined operation $\Theta C_{2{\bf c}}$, where $\Theta$ is time reversal and $C_{2{\bf c}}$ is the two-fold rotation around ${\bf c}$. 
%
The high-field K-state is qualitatively similar to the state `FM-SZ$_{\text{FM}}$' of Ref.~[\onlinecite{Lee2015}], which is stabilized in a finite region with $\phi\!<\!3\pi/2$.
}
\label{fig:PT}
\end{figure}  





\vspace*{-0.25cm} 
\section{Main aspects of the zero-field $K$-state}
\vspace*{-0.3cm}
%{\cdbl {\it Structure of the zero-field $K$-state}} --
The detailed structure of the $K$-state is discussed in Ref.~\cite{Sam2018}.  A schematic representation is shown in Fig.~\ref{fig:PT}\,(b). 
%
Let us repeat here the main features that are needed for our purposes.
%
The $K$-state features six spin sublattices, three (${\bf A}$, ${\bf B}$, ${\bf C}$) along the $xy$-chains and three (${\bf A}'$, ${\bf B}'$, ${\bf C}'$) along the $x'y'$-chains. The corresponding Cartesian components are given in the side panels of Fig.~\ref{fig:PT}\,(b), where $S=1/2$ is the classical spin length, while the numbers $x_1$, $y_1$, $z_1$, $x_2$ and $z_2$ are all positive (at zero field), and obey the constraints $x_1^2+y_1^2+z_1^2\!=\!1$ and $2x_2^2+z_2^2\!=\!1$. 

The sublattices $\{{\bf A}, {\bf B}, {\bf C}\}$ (and likewise the sublattices  $\{{\bf A}', {\bf B}', {\bf C}'\}$) form an almost ideal 120$^\circ$-pattern. 
%
The counter-rotating modulation of the moments can be seen in Fig.~\ref{fig:PT}\,(b) by noticing that, along the $xy$-chains (similarly for the $x'y'$-chains), the odd sites (gray circles) modulate in a ${\bf A B C A B} \cdots$ pattern, while the even sites (white circles) show a ${\bf C B A C B}\cdots$ pattern.
%
This modulation shows up in the Fourier component of the static structure factor at ${\bf Q}\!=\!2\hat{\bf a}/3$, which takes the characteristic form ${\bf M}_{{\bf Q}={2\hat{{\bf a}}/3}}\!=\!(i M_a A, i M_b C, M_c F)$, along ${\bf a}$, ${\bf b}$ and ${\bf c}$. Here, $A$, $C$ and $F$ denote, respectively, the N\'eel, stripy and FM basis vectors of the 4-site unit cell~\cite{Biffin2014a,Sam2018}. The amplitudes $M_a$, $M_b$ and $M_c$ are given by~\cite{Sam2018}
\be\label{eq:SofQ2o3}
\begin{array}{c}
M_a=i 2S (x_1+2x_2-y_1),~~M_b=-i2S (z_1+z_2),\\
M_c=i 2S \sqrt{3}(x_1+y_1)\,.
\end{array}
\ee
The counter-rotating modulation is however not the only component of the $K$-state, because there are two types of deviations from the ideal 120$^\circ$-pattern when $\phi\!>\!3\pi/2$: 
%
i) an in-plane canting of the zigzag type, whose direction alternates between ${\bf a}$ and $-{\bf a}$ for $xy$- and $x'y'$-chains, respectively, and ii) an out-of-plane ferromagnetic (FM) canting along ${\bf b}$. 
%
Both of these cantings are uniform from one unit cell to another, and show up directly in the ${\bf Q}=0$ Fourier component of the static spin structure factor, along ${\bf a}$ and ${\bf b}$. Specifically, ${\bf M}_{{\bf Q}=0}\!=\!(M_a' G, M_b' F, 0)$, where $G$ and $F$ denote, respectively, the zigzag and FM basis vectors of the 4-site primitive unit cell~\cite{Biffin2014a,Sam2018}. 
%
The amplitudes $M_a'$ and $M_b'$ are~\footnote{Here and in Eq.~(\ref{eq:SofQ2o3}) we have multiplied by a factor of $2S$ compared to the expressions given in Ref.~[\onlinecite{Sam2018}].}  
\be\label{eq:SofQ0}
M_a' = -4S(x_1-y_1-x_2), ~~ M_b' = 2S (2z_1-z_2)\,.
\ee
These amplitudes vanish when $J\to 0^+$, because in this limit $\{{\bf A}, {\bf B}, {\bf C}\}$ and $\{{\bf A}', {\bf B}', {\bf C}'\}$ reach their ideal 120$^\circ$-patterns. 





%%%%%%%%%%%%%%%%%%%%%%%%%%%%%%%%%%%%%
\vspace*{-0.25cm} 
\section{Total Energy of the $K$-state in a field}
\vspace*{-0.3cm}
%{\cdbl {\it Total Energy of the $K$-state in a field}} --
We now move to the main part of this study and analyze the fate of the $K$-state under a field $H$ along the ${\bf b}$-axis. 
%
As discussed in detail in Ref.~\cite{Ruiz2017}, the total polarization along ${\bf b}$ couples to the field via the diagonal element $g_{bb}$ of the ${\bf g}$-tensor, while the uniform zigzag canting along ${\bf a}$ (i.e., the staggered magnetization from $xy$- to $x'y'$-chains), couples to the field via the off-diagonal element $g_{ab}$, whose sign alternates between $xy$ and $x'y'$ chains due to the two-fold symmetry $C_{2{\bf a}}$ that passes through the middle of the $z$ bonds.  
%
So, a magnetic field along ${\bf b}$ couples linearly to both $M_a'$ and $M_b'$.  
%
Furthermore, such a field does not break the symmetry $\Theta C_{2{\bf c}}$ obeyed by the $K$-state, see Fig.~\ref{fig:PT}\,(b). 
%
These arguments suggest that we can use the $K$-state ansatz [side panel of Fig.~\ref{fig:PT}\,(b)], but now the coefficients $x_1$, $y_1$, etc will change with the field. 
%
To find these coefficients we must minimize the total energy,
\bea\label{eq:En}
\begin{array}{l}
E/N \!=\! S^2\Big\{
K \left[ 3-2(y_1-x_2)^2 \right] \\
~~~\!+\!2 \Gamma \left[1-z_1^2 + x_2^2  + 2 (y_1+x_2) z_1 + 2 x_1 z_2 \right] \\ 
~~~\!+\! J \left[1+2 (z_1-z_2)^2- 4 x_1 x_2  + 4 (x_1+x_2) y_1 \right] \Big\}/6 \\
~~~
\!-\! \mu_B H S \left[\sqrt{2} g_{ab} (x_1 - x_2 - y_1) + g_{bb} (-2 z_1 + z_2)\right]/3\,,
\end{array}
\eea
($N$ is the total number of sites) for given $H$, $J$, $K$, $\Gamma$, $g_{bb}$, and $g_{ab}$. From $x_1$, $y_1$, etc we can then deduce the various components of the structure factor using Eqs.~(\ref{eq:SofQ2o3}) and (\ref{eq:SofQ0}). 



%%%%%%%%%%%%%%%%%%%%%%%%%%%%%%%%%%%%%

\begin{figure}[!t]
\includegraphics[width=0.9\columnwidth]{SFcomponents_vs_H_r0p32pi_phi1p52pi}
\includegraphics[width=0.9\columnwidth]{SFcomponents_vs_H_r0p32pi_phi1p625pi}
\vspace*{-0.25cm}
\caption{Evolution of the various Fourier components of the static structure factor of the $K$-state with a magnetic field $H$ along ${\bf b}$, for the points $A$ (panel a) and $B$ (panel b) in Fig.~\ref{fig:PT}\,(a).}
\vspace*{-0.5cm}
\label{fig:SofQvsH}
\end{figure}  


\vspace*{-0.25cm} 
\section{Main results}
\vspace*{-0.3cm}
%{\cdbl {\it Main results}} -- 
Let us first highlight the results that are directly related to the reported experiments.
%
For demonstration, we have taken $g_{bb}\!=\!2$ and $g_{ab}\!=\!0.1$ (we shall address the role of $g_{ab}$ separately below).
%
Figs.~\ref{fig:SofQvsH}\,(a) and (b) show the magnitudes of the various Fourier components, at the points $A$ and $B$ of Fig.~\ref{fig:PT}\,(a). 
%
The results show that the ${\bf Q}\!=\!2\hat{{\bf a}}/3$ components, $M_a$, $M_b$ and $M_c$, decline with the field, and vanish completely at (and above) a characteristic field $H^\ast$. 
%
At the same time, the ${\bf Q}\!=\!0$ components, $M_a'$ and $M_b'$, grow linearly with the field, and they show a kink at $H^\ast$. These results are consistent with the data reported in Ref.~[\onlinecite{Ruiz2017}].

We can also see an important feature of the ${\bf Q}\!=\!0$ components, which is related to the zero-field scattering experiments of Biffin {\it et al}~\cite{Biffin2014a}.
%
Namely, that both $|M_a'(0)|$ and $|M_b'(0)|$ are very small when $\phi$ is close to $3\pi/2$ [see Fig.~\ref{fig:SofQvsH}\,(a)], where both the zigzag and the FM canting become small. 
%
As we move away from the line $\phi\!=\!3\pi/2$, the zigzag amplitude $M_a'(0)$, in particular, is not small any longer, see Fig.~\ref{fig:SofQvsH}\,(b). 
%
So, the absence of the ${\bf Q}\!=\!0$ Bragg peaks from the zero-field scattering experiments of Biffin {\it et al}.~\cite{Biffin2014a} is the first evidence that $\phi$ is close to $3\pi/2$, i.e., that $J$ is much weaker than both $K$ and $\Gamma$. 
 

The next qualitative experimental result that we would like to address is the intensity sum rule of Ref.~[\onlinecite{Ruiz2017}], i.e. the finding that the sum of a certain combination of the intensities corresponding to zero- and finite-$Q$ Bragg peaks, remains almost constant even slightly above $H^\ast$.
%
To this end, we will need to understand the behavior of the coefficients $x_1$, $y_1$, etc first. 

From Eqs.~(\ref{eq:SofQ2o3}-\ref{eq:SofQ0}) we find that the simultaneous vanishing of $M_a$, $M_b$ and $M_c$ for $H\!\ge\!H^\ast$ imply that 
\be\label{eq:HgeHc}
H\ge H^\ast:~~x_1=-y_1=-x_2, ~~
z_2=-z_1\,.
\ee
These relations are indeed satisfied as we see in Fig.~\ref{fig:x1y1etcvsH}. In particular, while all coefficients are positive at zero field, the coefficients $y_1$, $x_2$ and $z_1$ change sign at some intermediate field, and eventually satisfy (\ref{eq:HgeHc}) for $H\ge H^\ast$. 


The explanation of the intensity sum rule reported in Ref.~[\onlinecite{Ruiz2017}] stems from another aspect of Fig.~\ref{fig:x1y1etcvsH}\,(a), namely, that 
\be\label{eq:Approx}
\begin{array}{l}
H\!\le\!H^\ast,\\
 \phi\to\left(\frac{3\pi}{2}\right)^+
\end{array}\!\!:
\left\{\!\!
\begin{array}{l}
z_2\simeq x_1,~~x_2\simeq z_1\simeq y_1,\\
x_1(0)\simeq\sqrt{2/3},~~y_1(0)\simeq1/\sqrt{6}.
\end{array}\right.
\ee
Remarkably, these approximate relations hold all the way down to zero field, where they stem from the special structure of the ground state manifold along $\phi\!=\!3\pi/2$, and the concomitant lifting of the degeneracy  by an infinitesimal positive $J$.~\cite{Sam2018}  
%
For larger $J$, the approximations in Eq.~(\ref{eq:Approx}) become progressively worse as shown in Fig.~\ref{fig:x1y1etcvsH}\,(b). 


 
\begin{figure}[!t]
\includegraphics[width=0.95\columnwidth]{x1y1etc_vs_H_r0p32pi_phi1p52pi}
\includegraphics[width=0.95\columnwidth]{x1y1etc_vs_H_r0p32pi_phi1p625pi}
\caption{Evolution of the coefficients $x_1$, $y_1$, etc with a magnetif field $H$ along ${\bf b}$, for the points $A$ (panel a) and $B$ (panel b) in Fig.~\ref{fig:PT}\,(a)}
\vspace*{-0.25cm}
\label{fig:x1y1etcvsH}
\end{figure}  

Now, to see how Eq.~(\ref{eq:Approx}) leads to the intensity sum rule, we take the following combinations of the Bragg peak intensities,  
\be\label{eq:SumRule}
\begin{array}{l}
I_{I}\!=\!|M_a|^2\!+\!|M_b|^2\!+\!|M_c|^2, ~~
I_{V}\!=\!|M_a'|^2\!+\!|M_b'|^2\,.
\end{array}
\ee
%
Fig.~\ref{fig:SumRule} shows the behavior of the quantities $I_I/I_I(0)$, $\alpha I_V/I_I(0)$ and $I_{\text{tot}}\!\equiv\!(I_I+\alpha I_V)/I_I(0)$, where the constant $\alpha\!\equiv\!I_I(0)/I_V(H^\ast)$ fixes $I_{\text{tot}}(H^\ast)\!=\!1$. 
%
We first discuss the results shown in Fig.~\ref{fig:SumRule}\,(a), which are obtained at the point $A$ of Fig.~\ref{fig:PT}\,(a). 
%
As expected, the intensity associated with the counter-rotating component of the order, $I_I$, declines quickly with field, while the intensity associated with the uniform components, $I_V$, grows quadratically with field up to $H^\ast$. 
%
At the same time, the total intensity $I_{\text{tot}}$ remains extremely close to $1$ from zero field all the way up to $H^\ast$. This behavior is fully consistent with the intensity sum rule of Fig.~4\,(a) of Ref.~[\onlinecite{Ruiz2017}].

Importantly, while it is not clear which exact combinations of the zero- and finite-${\bf Q}$ intensities are involved in the sum rule reported in Ref.~[\onlinecite{Ruiz2017}], we can show that this does not matter as long as we are close to the line $\phi\!=\!3\pi/2$, where the approximate relations (\ref{eq:Approx}) hold. Indeed, these relations give:
\be\label{eq:HleHc}
\begin{array}{l}
H\!\le\!H^\ast,\\
 \phi\to\left(\frac{3\pi}{2}\right)^+
 \end{array}\!\!:
\!\left\{\!\!
\begin{array}{l}
|M_a|^2 \!\simeq\! |M_b|^2 \!\simeq\! \frac{1}{3}|M_c|^2 \!\simeq\! (x_1\!+\!y_1)^2,\\
%
|M_b'|^2 \!\simeq\! \frac{1}{4}|M_a'|^2 \!\simeq\! (x_1\!-\!2y_1)^2.
\end{array}
\right.
\ee
This implies that any linear combination among $\{|M_a|^2$, $|M_b|^2$, $|M_c|^2$, $|M_a'|^2$, $|M_b'|^2\}$ will always be of the form $ (x_1\!+\!y_1)^2$+$\beta(x_1\!-\!2y_1)^2$, which, in turn, becomes independent of $H$ if we choose $\beta\!=\!1/2$ (using the spin length constraint $1\!=\!x_1^2\!+\!y_1^2\!+\!z_1^2\!\simeq \!x_1^2\!+\!2y_1^2$). This value is consistent with the limiting value of $\alpha$ defined above when $\phi\!\to\!(3\pi/2)^+$. 


The intensity sum rule is no longer satisfied as we go further away from $\phi\!=\!3\pi/2$. This can be seen in Fig.~\ref{fig:SumRule}\,(b), which shows $I_I$, $I_V$ and $I_{\text{tot}}$ computed at the point $B$ of Fig.~\ref{fig:PT}\,(a). The total intensity $I_{\text{tot}}$ deviates from the value $1$ in the entire region between $H\!=\!0$ and $H\!=\!H^\ast$, except at $H^\ast$ where it is fixed to $1$ by definition. 
%
We also see that $I_{\text{tot}}$ shows a fast drop below $1$ at $H\!>\!H^\ast$, in contrast to the experimental data,~\cite{Ruiz2017} but also in contrast to the data of Fig.~\ref{fig:SumRule}\,(a), where the corresponding drop at $H\!>\!H^\ast$ is much slower. 
%
Note also that the overall deviation from the sum rule could  be even worse for linear combinations of $\{|M_a|^2$, $|M_b|^2$, $|M_c|^2$, $|M_a'|^2$, $|M_b'|^2\}$ other that the ones involved in $I_I$ and $I_V$.  
%
We can therefore safely conclude that the experimental observation of the sum rule is an independent signature of the smallness of $J$.

\begin{figure}[!t]
\includegraphics[width=0.9\columnwidth]{SumRule_vs_H_r0p32pi_phi1p52pi}
\includegraphics[width=0.9\columnwidth]{SumRule_vs_H_r0p32pi_phi1p625pi}
\caption{Evolution of the combinations $I_I$, $I_V$ and $I_{\text{tot}}$ [see Eq.~(\ref{eq:SumRule})] with a magnetic field $H$ along ${\bf b}$, for the points $A$ (panel a) and $B$ (panel b) of Fig.~\ref{fig:PT}\,(a). 
%
In (a), $\alpha\!\simeq\!0.503$ [very close to the expected value of $1/2$ when $\phi\!\to\!(3\pi/2)^+$, see text], while in (b), $\alpha\!\simeq\!0.535$.}
\vspace*{-0.25cm}
\label{fig:SumRule}
\end{figure}  


%%%%%%%%%%%%%%%%%%%%%%%%%%%%%%%%%%%%%
\vspace*{-0.25cm}
\section{Nature of the high-field state ($H\ge H^\ast$)}
\vspace*{-0.3cm}
%{\cdbl {\it Nature of the high-field state ($H\ge H^\ast$)}} -- 
Based on Eq.~(\ref{eq:HgeHc}), the Cartesian components of the various spin sublattices for $H\!\ge\!H^\ast$ are:
\be\label{eq:StateHc}
H\ge H^\ast:
\left\{\!\!
\begin{array}{l}
{\bf A}={\bf B}={\bf C}=S [x_1,-x_1,z_1],\\
{\bf A}'={\bf B}'={\bf C}'= S [-x_1,x_1,z_1],
\end{array}
\right.
\ee
see Fig.~\ref{fig:PT}\,(c).  
%
So the $xy$- and $x'y'$-chains form two separate FM subsystems, which give a total FM moment along $-{\bf z}\!=\!{\bf b}$ [$z_1(H^\ast)$ is negative] and a staggered, zigzag moment along $\frac{{\bf x}-{\bf y}}{\sqrt{2}}\!=\!{\bf a}$ [see Eq.~(\ref{eq:abc})].  
%
This state is qualitatively the same with the state `FM-SZ$_{\text{FM}}$' reported by Lee {\it et al}~\cite{Lee2015,Lee2016} for $\phi\!<\!3\pi/2$, see Fig.~\ref{fig:PT}\,(a). 
%
The only difference is that in that state $x_1\!=\!\frac{1}{\sqrt{3}}$, while here $x_1$ in general deviates from this value and depends on the field. 
%
However, $x_1$ becomes very close to $\frac{1}{\sqrt{3}}$ precisely at $H\!=\!H^\ast$ when $\phi\!\to\!(3\pi/2)^+$, see Fig.~\ref{fig:x1y1etcvsH}\,(a).  
%
So, the effect of the field is to suppress the counter-rotating component of the state  and, at the same time, effectively drive the system towards the state that is stabilized for negative $J$. Intuitively then, the field plays the role of a FM Heisenberg coupling that counteracts the effect of $J$ (which is positive). 
%
This also tells us that the field $H^\ast$ required to achieve this must be proportional to $J$, and we will confirm this below. 

Now, the approximate relation $x_1(H^\ast)\!\approx\!\frac{1}{\sqrt{3}}$ for $\phi\!\to\!(2\pi/3)^+$ gives, based on Eq.~(\ref{eq:SofQ0}), $M_a'(H^\ast)\approx -2\sqrt{3}$ and $M_b'(H^\ast)\approx -\sqrt{3}$, see Fig.~\ref{fig:SofQvsH}\,(a). Remarkably, these values are independent of the ratio $K/\Gamma$, as long as we are inside the $K$-region of Fig.~\ref{fig:PT}\,(a) and close to the line $\phi\!=\!3\pi/2$. 
%
Furtermore, imposing (\ref{eq:HgeHc}) to (\ref{eq:En}) and minimizing the resulting expression for $E/N$ gives an implicit relation for $x_1(H)$:
\be\label{eq:Hvsx1}
\begin{array}{c}
H\ge H^\ast:~~\mu_B H = 2S \frac{\Gamma(4x_1^2-1)+(2J-\Gamma)x_1 \sqrt{1-2x_1^2}}{2g_{bb} x_1-\sqrt{2} g_{ab}\sqrt{1-2x_1^2}}\,.
\end{array}
\ee
Note that $K$ does not appear explicitly in this relation, which can give the wrong impression that $H^\ast$ does not depend on $K$. This is however not true, because %both 
$H^\ast$ and $x_1(H^\ast)$ % are unknown a-priori [i.e., 
cannot be both determined solely by Eq.~(\ref{eq:Hvsx1}).


Eq.~(\ref{eq:Hvsx1}) gives the large-field behavior of $M_a'$ and $M_b'$,
\be
H\ge H^\ast: ~~ M_a'=-12S x_1, ~~M_b' = - 6S \sqrt{1-2x_1^2}\,.
\ee
The large-field behavior of $x_1$, $|M_a'|$ and $|M_b'|$ are shown in Figs.~\ref{fig:x1y1etcvsH_WholeRegion} and \ref{fig:SofQvsH_WholeRegion}.
%
In the infinite-field limit, in particular, 
\be
\begin{array}{c}
H\!\to\!\infty:
~
M_a' \!\to\! -\frac{6g_{ab}}{\sqrt{2}\sqrt{g_{ab}^2+g_{bb}^2}},
~
M_b ' \!\to\! -\frac{3g_{bb}}{\sqrt{g_{ab}^2+g_{bb}^2}}\,.
\end{array}
\ee
With $g_{ab}\!\ll\!g_{bb}$ we get $M_a'\!\simeq\!0$,  $M_b'\!\simeq\!-3$, $x_1\!\simeq\!0$, $z_1\!\simeq\!-1$, and ${\bf A}\!=\!{\bf B}\!=\!{\bf C}\!=\!{\bf A}'\!=\!{\bf B}'\!=\!{\bf C}'\!=\!S \hat{{\bf b}}$, as expected. 


%%%%%%%%%%%%%%%%%%%%%%%%%%%%%%%%%%%%%
\vspace*{-0.25cm}
\section{The role of $g_{ab}$ and the origin of the zigzag component}
\vspace*{-0.3cm}
%{\cdbl {\it Origin of the zigzag component \& the role of $g_{ab}$}} --
As announced in the Introduction, the significant growth of the zigzag component under the field along ${\bf b}$ does not originate in the linear coupling between the zigzag component and the field, via $g_{ab}$. 
%
To show this we compare the response shown already in Fig.~\ref{fig:SofQvsH}\,(a) for $g_{ab}\!=\!0.1$, with the response for $g_{ab}\!=\!0$. 
%
This comparison shows that, while $H^\ast$ shifts to slightly higher value, the large size of the zigzag component at $H^\ast$ remains robust. 
%
A further comparison to the case of $g_{ab}\!=\!-0.1$ shows that even the choice of the sign of $g_{ab}$, which has been arbitrarily considered to be positive so far [i.e., positive (negative) on the $xy$ ($x'y'$) bonds], does not alter the large magnitude of $|M_a'(H^\ast)|$. 
%
Taken together, these results show that, although $M_a'$ couples linearly to the field via $g_{ab}$, its significant growth is not related to $g_{ab}$ but to an inherent tendency of the system to reach the state described by Eq.~(\ref{eq:StateHc}). This state has already a finite component at zero-field, although its amplitude is undetectable for weak $J$. 

\begin{figure}[!t]
\includegraphics[width=0.49\textwidth]{SofQ_r0p32pi_phi1p52pi_gab0p1_0_m0p1}
\caption{
Effect of $g_{ab}$ on the evolution of the various components of the structure factor. 
%
Data are taken at the point $A$ of Fig.~\ref{fig:PT}\,(a), with   $g_{bb}\!=\!2$ and $g_{ab}\!=\!0.1$ [dotted lines, same as in Fig.~\ref{fig:SofQvsH}\,(a)], 0 (solid) and $-0.1$ (long-dashed).}
\vspace*{-0.25cm}
\label{fig:gab}
\end{figure}




%%%%%%%%%%%%%%%%%%%%%%%%%%%%%%%%%%%%%
\vspace*{-0.3cm}
\section{Dependence of $H^\ast$ on model parameters}
\vspace*{-0.3cm}
%{\cdbl {\it Dependence of $H^\ast$ on model parameters}} -- 
We now turn to the important question of the dependence of the characteristic field $H^\ast$ on the coupling parameters. To address this question we have calculated $H^\ast$ for various paths and the results are shown in Fig.~\ref{fig:HcFixed}. Panel (a) shows the evolution of $H^\ast$ with $K$, for fixed $\Gamma\!=\!-1$ and various values of $J$, while panel (b) shows the evolution of $H^\ast$ with $\Gamma$ for fixed $K\!=\!-1$ and the same set of $J$ values as in panel (a). 
%
The results from panels (a) and (b) can be summarized as follows. 
%
(i) $H^\ast$ tends to decrease with increasing $|K|$ and $|\Gamma|$. 
%
(ii) The rate of this decrease is almost zero at small $J$ and then increases with increasing $J$.  
%
(iii) $H^\ast$ has a much stronger dependence on $J$ compared to its dependence on $K$ and $\Gamma$. In particular, $H^\ast$ grows with increasing $J$, with an almost constant rate. 

These features can also be seen in panel (c), which shows the data collected from (a) and (b), with $J$ on the horizontal axis. 
%
This figure shows explicitly that for small enough $J$, $H^\ast$ is essentially independent of $K$ and $\Gamma$ [point (ii) above] and grows linearly with $J$ [point (iii) above]. 
%
It also shows that the deviation from this linear growth becomes larger for larger $J$, where a dependence on $K$ and $\Gamma$ shows up.



\begin{figure}[!t]
\includegraphics[width=0.49\textwidth]{FigHc}
\caption{
(a) Evolution of $H^\ast$ as a function of $K$ for fixed $\Gamma\!=\!-1$ and various values of $J$. 
%
(b) Evolution of $H^\ast$ as a function of $\Gamma$ for fixed $K\!=\!-1$ and various values of $J$. 
%
(c) Collected data shown in panels (a) and (b), with $J$ on the horizontal axis.  The line shown in a guide to the eye. 
%
For all cases we have taken $g_{ab}\!=\!0.1$ and $g_{bb}\!=\!2$.}
\vspace*{-0.25cm}
\label{fig:HcFixed}
\end{figure}
 


%%%%%%%%%%%%%%%%%%%%%%%%%%%%%%%%%%%%%
\vspace*{-0.25cm}
\section{Discussion}
\vspace*{-0.25cm}
%{\cdbl {\it Discussion}} --
The results presented here provide a consistent interpretation of the recent scattering experiments by Ruiz {\it et al}~\cite{Ruiz2017}. 
%
First, the analysis confirms the linear growth of a uniform zigzag component along ${\bf a}$, the rapid decline of the IC order, and the intensity sum rule. 
%
The latter two observations are signatures of the smallness of $J$ compared to $K$ and $\Gamma$.


Second, the significant growth of the zigzag component under the field does {\it not} stem from their linear coupling, as the zigzag amplitude at $H^\ast$ remains almost the same in the absence of this coupling (i.e., for $g_{ab}\!=\!0$). 
%
This shows a strong intertwining of the zigzag component to both the IC counter-rotating component and the longitudinal magnetization, which is the one most directly driven by the field.
%
Moreover, the zigzag component is already present at zero field (the same is true for the magnetization along ${\bf b}$) but, for weak $J$, its amplitude is too weak to be observed, consistent with Ref.~[\onlinecite{Biffin2014a}]. 


\begin{figure}[!t]
\includegraphics[width=0.9\columnwidth]{x1y1etc_vs_H_r0p32pi_phi1p52pi_WholeRegion}
\includegraphics[width=0.9\columnwidth]{x1y1etc_vs_H_r0p32pi_phi1p625pi_WholeRegion}
\caption{Same as in Fig.~\ref{fig:x1y1etcvsH} but now we show the behavior up to fields much higher than $H^\ast$. Panels (a) and (b) correspond again to the parameter points $A$ and $B$, respectively, of Fig.~\ref{fig:PT}\,(a).}
\vspace*{-0.25cm}
\label{fig:x1y1etcvsH_WholeRegion}
\end{figure}  


\begin{figure}[!t]
\includegraphics[width=0.9\columnwidth]{SFcomponents_vs_H_r0p32pi_phi1p52pi_WholeRegion}
\includegraphics[width=0.9\columnwidth]{SFcomponents_vs_H_r0p32pi_phi1p625pi_WholeRegion}
\caption{Same as in Fig.~\ref{fig:SofQvsH} but now we show the behavior up to fields much higher than $H^\ast$. Panels (a) and (b) correspond again to the parameter points $A$ and $B$, respectively, of Fig.~\ref{fig:PT}\,(a).}
\vspace*{-0.25cm}
\label{fig:SofQvsH_WholeRegion}
\end{figure}  




Third, the field acts effectively as a FM Heisenberg coupling that counteracts the effect of $J$, and the state reached at $H^\ast$ is qualitatively the same as the state stabilized by a negative $J$ at zero field.  An immediate ramification is that $H^\ast$ grows linearly with $J$, which is demonstrated numerically in the relevant regime of interest. 
%
Accordingly, the rapid decline of the IC component is another  signature of the smallness of $J$. 
%
In particular, the curve shown in Fig.~\ref{fig:HcFixed}\,(c) gives $J\!\simeq\!4$\,K for the experimental value of $H^\ast\!=\!2.8$\,T~\cite{Ruiz2017} (assuming $g_{bb}\!=\!2$ and $g_{ab}\!=\!0.1$; the latter does not affect $H^\ast$ as strongly as $g_{bb}$), which is indeed much smaller~\footnote{We should note that these {\it ab initio} calculations give a negative $J$, which on the basis of the $J$-$K$-$\Gamma$ model cannot deliver the IC order reported experimentally.} than the reported {\it ab initio} values of $K$ and $\Gamma$~\cite{Katukuri2016,KimKimKee2016, Tsirlin2018}.
%
%H=1 Tesla ?> muB*H = 1.3376/2 Kelvin


Given this, the recent report~\cite{Tsirlin2018} that $H^\ast$ decreases with pressure is direct evidence (within the $J$-$K$-$\Gamma$ framework) that $J$ decreases under pressure. 
%
However, the behavior of $H^\ast$ alone cannot tell us what happens to $K$ and $\Gamma$, because $H^\ast$ shows a weak dependence on these couplings only at larger $J$; There is however independent evidence from $\mu$SR~\cite{Tsirlin2018} and {\it ab initio} calculations~\cite{KimKimKee2016,Tsirlin2018} showing that pressure increases the ratio $|\Gamma/K|$, with the system approaching the correlated classical spin liquid regime of the large-$\Gamma$ model~\cite{IoannisGamma}.

An experimental result that is not readily captured by the present classical description is the reported value of the magnetization at $H^\ast$ per Ir site, $m/\mu_B\!\simeq\!0.31$~\cite{Ruiz2017}, which is about two times smaller than the value $z_1(H^\ast)g_{bb} S\!\simeq\!\frac{1}{\sqrt{3}}$ (for weak $J$ and $g_{ab}\!=\!2$), deduced from the above analysis. 
%
Given that the IC component disappears at $H^\ast$, the discommensurations that are missed by the $K$-state ansatz at zero field~\cite{Sam2018} are not present any longer. 
%
Therefore, the above large disagreement in the magnetic moment $m$ should be attributed to the reduction of the spin length by quantum fluctuations. 
%
Such an unusually~\footnote{We know from other anisotropic models of this type (see e.g. Ref.~[\onlinecite{IoannisK1K2}]) that the spin length reduction by quantum fluctuations is otherwise very small due to the anisotropy spin gap.} large reduction can only be expected close to the frustrated region $\phi\!\sim\!3\pi/2$, which is in line with the remaining evidence above. A standard $1/S$ expansion around the $K$-state at $H^\ast$ should confirm this picture, but this is out of the scope of the present study.  

The broader emerging picture, taken together with the evidence from the zero-field study of Ref.~[\onlinecite{Sam2018}], gives strong confidence to the intuitive hypothesis made in \cite{Sam2018}, that the actual zero-field IC phase of $\beta$-Li$_2$IrO$_3$ is a long-wavelength twisting of a nearby commensurate state (the $K$-state). 
%
This route can be useful in analyzing related materials with analogous IC orders, such as $\alpha$-Li$_2$IrO$_3$ and $\gamma$-Li$_2$IrO$_3$~\cite{Singh2012,Biffin2014b,Modic2014,Williams2016,Satoshi2016}, and thus shed light to the delicate interplay between the various microscopic interactions, and map out the most relevant instabilities and their distinctive experimental fingerprints.


%%%%%%%%%%%%%%%%%%%%%%%%%%%%%%%%%%%%%

\vspace*{-0.3cm}
\section{Acknowledgments}
\vspace*{-0.3cm}
We are grateful to Samuel Ducatman for collaboration on related work, and to Cristian Batista, Radu Coldea and Alexander Tsirlin for valuable discussions. 
%
This work was supported by the U.S. Department of Energy,  Office of Science, Basic Energy Sciences under Award \# DE-SC0018056.



\bibliography{refs}\end{document}


\end{document}
