\subsection{Analyses}

To investigate the source of our model's improvement over the baselines, we computed two selection accuracy metrics on the SRMS dataset, \textit{WhenToCom} and \textit{Grouping}. \textit{WhenToCom} accuracy measures how often communication between a requester and a supporter(s) is established and when it is needed; and \textit{Grouping} measures how often the correct group is created when there is indeed communication. We also comment on the trade-off between bandwidth and performance of communication by conducting a controlled experiment on the size of query and key on the 3D recognition dataset. 

\noindent
\textbf{Effect of handshake communication}
As demonstrated in Figure~\ref{fig:hcom_effect}, we conduct an ablation study on the proposed handshake communication. In the \textit{Ours (w/o H-Com)} model, we remove the handshake communication module, so that each agent only uses its local observations to compute both (1) the communication score and (2) its communication group.

We additionally provide the result of \textit{RandCom}.  We observed that our model with the proposed handshake communication offers a significant improvement over both \textit{RandCom} and our model without handshake communication. This finding demonstrates the necessity of communication for deciding when to communicate and who to communicate with. That is, an agent without communication cannot decide what information it needs and which supporter has the relevant information to help better perform on perception tasks.

Figure~\ref{fig:vis_graph} visualizes three examples from the 3D shape recognition task. Each agent clearly knows when communication is needed based on information provided by the supporters and its own observation. For example, in the first three box examples, the degraded agent on the left knows to select an informative view from the other agents; the non-degraded agent in the middle decides to select a more informative view even though it possesses sufficient information; and the third agent decides that communication is not needed because it has the most informative view among all. It is worth mentioning that all 9 views are provided to every agent and the agent needs to identify informative views and detrimental views based on the matching scores. 
\noindent
\textbf{Query and key size}
We further analyze the effect of query and key size on \textit{Grouping} accuracy and classification accuracy on the 3D shape classification task. 
We vary the query size from $1$ to $128$ with a fixed key size of $16$ as shown in Figure~\ref{fig:abl_msg}. We observe that both selection and classification accuracy improve as the message size increases. Our model can perform favorably with a message size of $4$. The same trend is also observed for key sizes. Most noticeably, we find that there exists asymmetry in query-key size. While the selection accuracy saturates at 16-dimensional query, selection accuracy consistently improves with increasing key size until 1024-dimensional key. Our model exploits this asymmetry to save bandwidth in communication while maintaining high performance.  
