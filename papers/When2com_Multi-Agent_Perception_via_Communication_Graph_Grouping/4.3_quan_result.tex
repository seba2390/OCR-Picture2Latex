\subsection{Quantitative results}

\noindent
\textbf{Single-Request Multiple-Support (SRMS)}
The goal of this case is to examine if our model is able to learn when to communicate and learn who to communicate with for a single requesting agent. 
Figure~\ref{fig:when2com_exp} shows the performance of our proposed model and several baseline models. 
Although most fully-connected methods can improve the prediction mIoU compared with \textit{NoCom}, they need to propagate all information in a fully-connected manner and thus require high bandwidth consumption. 
In contrast, our model reports higher prediction accuracy yet smaller bandwidth usage (Who2com~\cite{liu2020who2com}: $2$ MBpf; ours: $0.98$ MBpf).
Another observation is that our model is able to further improve compared with \textit{Who2com}~\cite{liu2020who2com}. 
This demonstrates the benefit of learning when to communicate, which reduces the waste of bandwidth and prevent detrimental message when the requesting agent has sufficient information and communication is not required. 

\noindent
\textbf{Multiple-Request Multiple-Support (MRMS)}
In this case, we further address a more challenging problem, where multiple agents suffer degradation. 
Each agent should (1) identify when it needs to communicate, (2) decide who to communicate with when it needs to, and (3) avoid the selection of noisy views from the supporting agents. 
We list the experiment results in the Table~\ref{tab:mimo_case1}\zk{Should move tables so that this is Table 1}. 
It can be seen that, when the requesting agents cannot prevent the selection of noisy supporting agents, both \textit{CatAll} and \textit{RandCom} perform even worse than \textit{NoCom}. 
This verifies our intuition that the information from the supporting agents is not always beneficial for the requesting agents, and selection of incorrect information may even hinder the prediction of the requesters.

With the use of attention mechanisms for weighting the feature maps from the supporting agents, both \textit{AuxAttend} and \textit{Who2com}~\cite{liu2020who2com} are able to prevent incorrect views from deteriorating performance and thus improve with respect to \textit{NoCom}, \textit{CatAll}, and \textit{RandCom}. 
However, without learning when to communicate, those models are forced to always request information from at least one supporting agent resulting in both poorer performance and unnecessary bandwidth usage. 

In addition to the above baseline methods, we also consider CommNet~\cite{sukhbaatar2016learning} and TarMac~\cite{das2019tarmac}. 
Even though CommNet integrates the information from other agents by using an average pooling mechanism, it does not improve the prediction of either degraded or non-degraded requesting agents because it indiscriminately incorporates all views. 

On the other hand, TarMac~\cite{das2019tarmac} is able to provide better results compared with the baseline models. 
However, TarMac uses one-way communication and results in large bandwidth usage which presents difficulty in the real scenario. 
On the contrary, our model is not only able to outperform it on both degraded and non-degraded samples, but also consumes less bandwidth by using our asymmetric query mechanism and pruning redundant connections within the network with the activation function. 


\noindent
\textbf{Multiple-Request Multiple-Partial-Support (MRMPS)}
In this case, there is less chance to have completely overlapped observations between any two agents. 
This presents an inherent difficulty in the perception task because only incomplete information is available for the prediction. 
As shown in the right part of Table~\ref{tab:mimo_case1}, the performance improvement margin of all FC and DistCom models is smaller with respect to \textit{NoCom}, in comparison to more significant improvement observed in the previous scenario. 

Nonetheless, we observe that all methods exhibit a similar trend as the previous scenario. Our model is still able to maintain a similar prediction accuracy as fully-connected models, while we only use one-fourth bandwidth for communication across agents. This demonstrates the superior bandwidth-efficiency of our model.

\noindent
\textbf{Multi-agent 3D shape Recognition}
In order to demonstrate the generalization of our model, here we apply our model to the task of multi-agent 3D shape classification.
Table~\ref{tab:3d_shape} provides the quantitative evaluation on this task using our proposed model and other baselines, including \textit{VAIN}~\cite{hoshen2017vain}, \textit{CommNet}~\cite{sukhbaatar2016learning}, and \textit{TarMac}~\cite{das2019tarmac}. Our model is able to perform competitively compared with \textit{TarMac}~\cite{das2019tarmac} with only approximately one-eighth bandwidth usage. We also provide qualitative results in Figure~\ref{fig:vis_graph} to demonstrate the effectiveness of our model, which allows agents to communicate with the correct and informative agents.

