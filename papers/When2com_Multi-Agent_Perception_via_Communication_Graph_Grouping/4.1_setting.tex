\subsection{Experimental Cases and Datasets}
\section{Dataset}
\label{sec:dataset}
%\sarah{add statistics about distribution of merge patterns}
%\alexey{I added some numbers in the section 4 (around line 270). Detailed numbers are in Appendix. We can move it up here if needed...}
%To create a dataset for self-supervised pretraining, we clone all non-fork repositories with more than 20 stars in GitHub that have C, C++, C\#, Python, Java, JavaScript, TypeScript, PHP, Go, and Ruby as their top language. The resulting dataset comprises over 64 million source code files. 
%\chris{why do we list languages here that we don't ever evaluate on?  A reviewer will find this confusing and ask about it.  We found that language specific models work better than multi-lingual models, right?}

The finetuning dataset is mined from over 100,000 open source software repositories in multiple programming languages with merge conflicts. It contains commits from git histories with exactly two parents, which resulted in a merge conflict.  We replay \texttt{git merge} on the two parents to see if it generates any conflicts. Otherwise, we ignore the merge from our dataset. We use the approach introduced by~\citet{Dinella2021} to extract resolution regions---however, we do not restrict ourselves to conflicts with less than 30 lines only.  Lastly, we extract token-level conflicts and conflict resolution classification labels (introduced in Section \ref{formulation}) from line-level conflicts and resolutions. Tab.~\ref{tab:fintuning_dataset} provides a summary of the finetuning dataset.

\begin{table}[htb]
\centering
\caption{Number of merge conflicts in the dataset.}
\begin{tabular}{llllllllllll} \toprule
\textbf{Programming language} & \textbf{Development set}  & \textbf{Test set} \\ \midrule
C\# & 27874 & 6969 \\ 
JavaScript & 66573 & 16644\\ 
TypeScript & 22422 & 5606\\ 
Java & 103065 & 25767 \\ 
\bottomrule
\end{tabular}
\label{tab:fintuning_dataset}
\end{table}
The finetuning dataset is split into development and test sets in the proportion 80/20 at random at the file-level. The development set is further split into training and validation sets in 80/20 proportion at the merge conflict level.    

\subsubsection{Collaborative Semantic Segmentation}

Our first task is collaborative 2D semantic segmentation of a 3D scene. Given observations (an RGB image, aligned dense depth map, and pose) from several mobile agents, the objective of this task is to produce an accurate 2D semantic segmentation mask for each agent.

Since current semantic segmentation datasets~\cite{geiger2013vision,cordts2016cityscapes,RobotCarDatasetIJRR,hecker2018end} only provide RGB images and labels captured from the perspective of single agent, we thus use AirSim simulator~\cite{airsim2017fsr} to collect our AirSim-MAP (Multi-Agent Perception) dataset.
For this dataset, we fly a swarm of five to six drones with different yaw rates through a series of waypoints in the AirSim ``CityEnviron'' environment. 
We record pose, depth maps, and RGB images for each agent. Note that we also provide semantic segmentation masks for \textbf{all} drones.  

We consider three experimental cases within this task.  We refer to the agent attempting segmentation as the \textbf{requesting} agent, and all other agents as the \textbf{supporting} agents.  Details for each case are listed as follows: 

\noindent
\textbf{Single-Request Multiple-Support (SRMS)} 
This first case examines the effectiveness of communication for a single requesting agent under the assumption that if an agent is degraded, then its original, non-degraded information will be present in one of the supporting agents.  We include a total of five agents, of which only one is selected for possible degradation.  We add noise to a random selection of $50\%$ of this agent's frames, and we randomly replace one of the remaining agents with the \textit{non-degraded} frame of the original agent.  Note that only the segmentation mask of the original agent is used as supervision.  

\noindent
\textbf{Multiple-Request Multiple-Support (MRMS)}
The second case considers a more challenging problem, where multiple agents can suffer degradation.  Instead of requiring a single segmentation output, this case requires segmentation outputs for all agents, degraded and non-degraded.  We follow the setup of the previous case, and we ensure that each of the several degraded requesting agents has a corresponding non-degraded image among its supporting agents.  

\noindent
\textbf{Multiple-Request Multiple-Partial-Support (MRMPS)}
The third case removes the assumption that there exists a clean version of the degraded view among the supporting agents.  Instead, the degraded agent must select the most informative view(s) from the other agents, and these views might have a variable degree of relevance.  Specifically, as the drone group moves through the environment, the images from each drone periodically and partially overlap with those of other drones.  Intuitively, the segmentation output of the requesting drone is only aided from the supporting drones that have overlapping views. 

\begin{table*}
\caption{\textbf{Experimental results on Multiple-Request Multiple-Support and Multiple-Request Multiple-Partial-Support.} Note that we evaluate these models with the metric of mean intersection of union (mIoU) and use MBytes per frame (Mbpf) and the averaged number of links per agent for measuring bandwidth.}
\label{tab:mimo_case1}
\centering
\resizebox{\linewidth}{!}{%
\begin{tabular}{ccccccccccc}
\toprule
& &  \multicolumn{4}{c}{Multiple-Request Multiple-Support} & & \multicolumn{4}{c}{Multiple-Request Multiple-Partial-Support} \\
 \cmidrule(lr){3-6}  \cmidrule(lr){8-11} 
& Models & Bandwidth (Mbpf / $\#$ of links) & Noisy & Normal & Avg.&  & Bandwidth  (Mbpf / $\#$ of links)& Noisy & Normal & Avg.  \\
\midrule
 & AllNorm & - & 57.85 & 57.74 & 57.80&  & - & 47.9 & 48.37 & 48.14  \\
\midrule

\multirow{5}{*}{ Fully-Connect. } & CatAll & 2.5 / 5 & 29.07 & 51.83 & 40.45 & & 2.0 / 4 & 26.86 & 45.27 & 36.07 \\
& AuxAttend & 2.5 / 5 & 33.69 & 56.27 & 44.98  & & 2.0 / 4 & 26.97 & 51.03 & 39.00\\
& CommNet~\cite{sukhbaatar2016learning} & 2.5 / 5 & 23.68 & 52.67 & 38.18 & & 2.0 / 4 & 26.56 & 49.07 & 37.82 \\
& TarMac~\cite{das2019tarmac} & 2.5 / 5 & 51.09 & 56.74 & 53.92 & & 2.0 / 4 & 29.78 & \textbf{51.39} & 40.59 \\
\midrule
\multirow{3}{*}{ Distri. } & RandCom & 0.5 / 1 & 21.22 & 52.74 & 36.98 & & 0.5 / 1 & 24.13 & 45.19 & 34.66 \\
& Who2com~\cite{liu2020who2com} & 0.5 / 1 & 31.96 & 56.11 & 44.04 &  & 0.5 / 1 & 26.97 & 50.71 & 38.84 \\
& Ours & \textbf{0.385 / 0.77} & \textbf{56.52} & \textbf{58.04} & \textbf{57.28} & & \textbf{0.55} / 1.08 & \textbf{30.38} & 51.26 & \textbf{40.82} \\
\midrule
 & OccDeg & - & 30.06 & 56.31 & 43.19&  & - & 25.2 & 46.74 & 35.97 \\
\bottomrule
\end{tabular}}
\end{table*}



\subsubsection{Multi-Agent 3D Shape Classification}
In addition to the semantic segmentation task, we also consider a multi-agent 3D shape classification task.  
For this experimental case, we construct a multi-agent variant of the \textbf{ModelNet 40} dataset~\cite{wu20153d}. 
The original dataset contains 40 common object categories from ModelNet with 100 unique CAD models per category and 12 different views of each model.  
However, our variant adds a communication group structure to the original dataset.  Specifically, we sample three sets of class-based image triplets.  
Each triplet corresponds to a randomly selected 3D object model and each triplet contains three randomly selected 2D views of its corresponding object model.  
To make this problem setting more challenging, we further degrade one image from each triplet. 
The objective of this task is to predict the corresponding object class for each agent by leveraging the information from all agents. 
Figure~\ref{fig:vis_graph} shows an example of the dataset in one trial with 9 agents.
This modified task is essentially a distributed version of the multi-view classification task~\cite{wu20153d}.

