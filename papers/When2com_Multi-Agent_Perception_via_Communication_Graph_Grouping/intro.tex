%%%%%%%%% BODY TEXT
\section{Introduction}

Remarkable progress has been achieved for single-agent perception and recognition, where one or more sensor modalities are used to perform object detection~\cite{redmon2016you,Redmon_2017_CVPR,Lin_2017_ICCV} and segmentation~\cite{chen2018encoder,He_2017_ICCV,kirillov2019panoptic}, depth estimation~\cite{godard2017unsupervised,Zhou_2017_CVPR,godard2019digging}, and various other scene understanding tasks. 
However, in many applications, such as robotics, there may be multiple agents distributed in the environment, each of which has local sensors. 
Such multi-agent systems are advantageous in many cases, for example, to increase coverage across the environment or to improve robustness to failures. 
\section{System Overview} \label{sec:overview}

In this section, we give an overview of our keyword search system. Our approach can be summarized as a two-phase framework, as illustrated in Figure \ref{fig:overview}. 

\begin{figure} [h]
	\centering
	\scalebox{1.0}[1.0]
	{
		\resizebox{\linewidth}{!}
		{
			\includegraphics[scale=0.5]{visio_pics/approach_overview_simp.pdf}
		}
	}
	\vspace{-0.3in}
	\caption{An Overview of Our Approach.}
	\label{fig:overview}
	\vspace{-0.2in}
\end{figure}

\subsection{Phase-I: Segmentation and Annotation} \label{sec:phase1}
The first phase is to segment the keyword token sequence $RQ = \{k_1, k_2, ..., k_m\}$ into several \emph{terms} and each term is annotated with one of the three characters $\{$entity, class, relation$\}$. The converted query is called \emph{annotated query}. Formally, we denote an \textit{annotated query} as $AQ= \{t_1:c_1, t_2:c_2, ..., t_l:c_l\}$, where each $t_i$ is a term and  $c_i \in \{$entity, class, relation$\}$. Note that the first phase (i.e., segmentation and annotation) is not the focus of this paper, as it has been studied extensively \cite{hua2015short,han2011generative, li2011faerie, nakashole2012patty,cai2013large}. We briefly describe the implementation of the first phase as follows. 

For each continuous subsequence $s$ in $RQ$, we check whether it could be matched to an entity, a class, or a relation of the RDF dataset, by employing the existing techniques of entity linking \cite{han2011generative,li2011faerie, ratinov2011local} and relation paraphrasing \cite{nakashole2012patty,cai2013large}. If $s$ is matched, we regard $s$ as a \emph{candidate term} $t_i$, and annotate it with the corresponding character (entity, class, or relation).
We may find that two candidate terms $t_i$ and $t_j$ \emph{overlap} with each other. We say $t_i$ overlaps with $t_j$ if and only if they have at least one common token. Obviously, if two terms overlap, they cannot occur at the same segmentation result. For example, ``university'' and ``university locate USA'' cannot occur in the same segmentation result. We build a \emph{candidate term graph} to describe the mutually exclusive relations: (1) each candidate term $t_i$ is represented a vertex; (2) there is an edge between $t_i$ and $t_j$ if and only of there is \emph{no} overlapping tokens between $t_i$ and $t_j$. Thus each maximal clique in the candidate term graph stands for a possible segmentation result. To obtain top-$N$ best $AQ$, we employ the maximal clique algorithm \cite{bron1973algorithm}, and adopt the pairwise metrics in \cite{hua2015short} to rank the segmentation result.
In out example, we get the top-2 $AQ$ as shown in Figure \ref{fig:overview}.

In the first phase, we have converted keyword token sequence $RQ$ into top-$N$ $AQ$ by some off the shelf techniques. Furthermore, these terms in $AQ$ have been matched to some elementary query graph building blocks (i.e., entity/class vertices and predicate edges). Specifically, if a term $t_i$ is annotated with ``entity'' or ``class'', it will be matched to candidate entity/class vertices in RDF graph.
%We do not distinguish entity vertex or class vertex until generating SPARQL, since we have the same operation on both of them.
If a term $t_i$ is annotated with ``relation'', it will be matched to a set of candidate predicates.

\vspace{-0.05in}
\begin{example}\label{example:aq}
	Given a keyword token sequence $RQ = \{$scientist, graduate, from, university, locate, USA$\}$, we obtain the annotated query $AQ=\{$``scientist'':class, ``graduate from'':relation, ``university'':class, ``locate'':relation, ``USA'':entity $\}$, where ``scientist'' is matched to $\{$dbo:Scientist$\}$, ``university'' is matched to $\{$dbo:University$\}$, ``USA'' is matched to two possible entities $\{$res:USA$\_$Today, res:United$\_$\\States$\}$ due to the ambiguity. Also, the relations ``graduate from'' and ``locate'' also match to two candidate predicates $\{$dbo:almaMater, dbo:education$\}$ and $\{$dbo:country, dbo:location$\}$
\end{example}

\vspace{-0.1in}
\subsection{Phase-II: Query Graph Assembly}
In the second phase, we concentrate on how to assemble a query graph $Q$ based on these elementary building blocks. Formally, the query graph assembly problem is defined as follows:

\vspace{-0.05in}
\begin{definition} \textbf{(Query Graph Assembly Problem)}\label{def:querygraphassembling}
	Given $n$ terms $t^{v}_i$ ($i=1,...,n$) annotated with ``entity'' or ``class'', and $m$ terms $t_j^{e}$ ($j=1,...,m$) annotated with ``relation'', each term $t^{v}_i$ is matched to a set $V_i$ of candidate entity/class vertices and each $t^{e}_j$ is matched to a set $E_j$ of candidate predicate edges. Let $\Upsilon=\{V_1, ..., V_n\}$ and $\Gamma=\{E_1, ..., E_m\}$. A valid assembly query graph is denoted as $Q (V_Q,E_Q)$, which satisfies the following constraints:
	\begin{enumerate}
		\item $|V_Q|=n$, and $\forall V_i \in \Upsilon, V_Q \cap V_i \not= \phi$; 
		\emph{/*each entity or class vertex set $V_i$ has exactly one vertex in  $Q$*/}
		\item $|E_Q|=m$, and $\forall E_j \in \Gamma, E_Q \cap E_j \not= \phi$. 
		\emph{/*each predicate edge set $E_j$ has exactly one edge in $Q$*/}
	\end{enumerate}
	Each edge $e(\langle v_1, v_2\rangle,p) \in E_Q$ connects a pair of vertices $\langle v_1, v_2\rangle \in V_Q$ by a predicate $p$.
	The assembly cost of $Q$ is
	\begin{equation} \label{equ:cost}
	cost(Q)=\sum_{e(\langle v_1, v_2\rangle,p) \in E_Q}{w(\langle v_1, v_2\rangle,p)}
	\end{equation}
	where  $w(\langle v_1, v_2\rangle,p)$ denotes the triple assembly cost. 
	
	
	The \emph{query graph assembly} (QGA for short) problem is to construct a valid graph $Q$ with the minimum assembly cost. 
\end{definition} 

%\vspace{-0.06in}

There are two aspects that should be explained for QGA.

\textbf{\emph{1. Constraints:}}
The two constraints in Definition \ref{def:querygraphassembling} mean that each term $t^{v}_i$ ($1\leq i \leq n$) only corresponds to a single entity/class vertex in $Q$. For example, although ``USA'' may match two candidate entities dbo:USA$\_$Today and dbo:United$\_$States, in the final query graph $Q$, ``USA'' only matches a single entity (dbo:United$\_$States). It is analogue for the relation term $t_j^{e}$ ($1\leq j \leq m$). 

\textbf{\emph{2. Disengaged Edges:}}
A \emph{predicate edge} $e(\langle \cdot,\cdot \rangle,p)$ (in $E_j$) does not have two fixed endpoints but its edge label is fixed to predicate $p$. Thus, a predicate edge can be also called a \emph{disengaged} edge. The triple assembly cost $w(\langle v_1, v_2\rangle,p)$ measures the goodness of assembling $\langle v_1, v_2\rangle$ and $p$ into an edge in $Q$. Then the goal of the QGA problem is to determine the endpoints of $e(\langle \cdot,\cdot \rangle,p)$ to minimize the overall $cost(Q)$.

After finding the optimal $Q$ with minimum $cost(Q)$, we can translate it to SPARQL statements naturally, as illustrated in Figure \ref{fig:overview}.

\subsection{Graph Embedding Cost Model}
Note that the triple assembly cost $w(\langle v_1, v_2\rangle,p)$ can be any positive cost function, which does not affect the hardness of QGA. In other words, the QGA problem is a general computing framework to interpret the input keywords as SPARQL, which does not depend on any specific triple assembly cost function.

\begin{figure} [b]
	\centering
	\scalebox{0.55} [0.50]
	{
		\resizebox{\linewidth}{!}
		{
			\includegraphics[scale=1.0]{visio_pics/transe_visualizing.pdf}
		}
	}
	\caption{Visualizing the Intuition of Graph Embedding.}
	\label{fig:transe_visualizing}
	\vspace{-0.2in}
\end{figure}

The only thing affected by the selection of triple assembly cost function is the system's accuracy. A good cost function can guide to assemble correct query graph $Q$ that implies users' query intention. The process of assembling $\langle v_1, v_2\rangle$ and $p$ into a triple is analogue to ``link prediction'' problem in the RDF knowledge graph \cite{miller2009nonparametric}. Given two entity/class vertices $v_1$ and $v_2$, the link prediction is to ``predict'' the predicate $p$ between $v_1$ and $v_2$, and $w(\langle v_1,v_2\rangle, p)$ is a \emph{measure} of the prediction. Recent research show that the graph embedding technique is superior to other traditional approaches, such as \cite{miller2009nonparametric,nickel2011three,jenatton2012latent}. In the graph embedding model, all subjects (s), objects (o) and predicates (p) are encoded as multi-dimensional vectors $\overrightarrow{s}$, $\overrightarrow{p}$ and $\overrightarrow{o}$ such that $\overrightarrow s  + \overrightarrow p  \approx \overrightarrow o $ if $\langle s,p,o \rangle \in G$ (i.e., $\langle s,p,o\rangle$ is a triple in RDF graph); while $\overrightarrow s  + \overrightarrow p$ should be far away from $\overrightarrow o$ otherwise. Figure \ref{fig:transe_visualizing} visualizes the intuition. From the intuition, the structural information among entities, classes and relations in RDF graph is embedded into vectors. Therefore, we define the triple assembly cost based on graph embedding vectors as follows.


\begin{definition}\textbf{ (Triple Assembly Cost) }. \label{def:tripleassemblycost}
	Given two entity/class vertices $v_1$ and $v_2$ and a predicate edge $p$, the \emph{cost} of assembly triple $(v_1,p,v_2)$ is denoted as follows:
	\begin{equation}\label{equ:tripleassembly}
	w(\langle v_1, v_2\rangle, p) = MIN(| \overrightarrow{v_1} + \overrightarrow p  - \overrightarrow {v_2 } |, | \overrightarrow{v_2} + \overrightarrow p  - \overrightarrow {v_1 } |)
	\end{equation}
	where $\overrightarrow{v_1}$, $\overrightarrow {v_2}$ and $\overrightarrow {p}$ are the encoded multi-dimensional vectors  of $v_1$, $v_2$ and $p$, respectively. 
	%where $|x|_+$ denotes the positive part of $x$. 
\end{definition}  

\begin{figure} [h]
	\centering
	\scalebox{1.0}
	{
		\resizebox{\linewidth}{!}
		{
			\includegraphics[scale=1.0]{visio_pics/query_graph_elements_example2.pdf}
		}
	}
%	\caption{Candidate Entity/Class Vertices and Predicate Edges.}
	\caption{Elementary Query Graph Building Blocks.}
	\label{fig:graph_elements_exp2}
	\vspace{-0.3in}
\end{figure}

\begin{figure} [h]
	\newcommand{\mywidth}{0.23\textwidth}
	\centering
	\begin{subfigure}[t]{\mywidth}
		\centering
		\resizebox{\linewidth}{!}
		{
			\includegraphics{visio_pics/query_assembly_graph_q1.pdf}
			
		}
		\caption{$cost(Q_1)=1.76$}
		\label{fig:assembly_query_graph_q1}
	\end{subfigure}
	\begin{subfigure}[t]{\mywidth}
		\centering
		\resizebox{1.0\linewidth}{!}
		{
			\includegraphics{visio_pics/query_assembly_graph_q2.pdf}
		}
		\caption[font=\small]{$cost(Q_2)=2.46$}
		%        \vspace{0.1in}
		\label{fig:assembly_query_graph_q2}
	\end{subfigure}
	\caption{Possible Assembly Query Graphs.}
	%    \vspace{-0.1in}
	\label{fig:assembly_query_graph}
	\vspace{-0.1in}
\end{figure}

\begin{example} In our example, there are three entity/class terms ``scientist'', ``university'', ``USA'' and two relation terms ``graduate from''  and ``locate''. Their corresponding entity/class vertices and predicate edges are shown in Figure \ref{fig:graph_elements_exp2}. There are two different assembly query graph $Q_1$ and $Q_2$ in Figure \ref{fig:assembly_query_graph}, among which $cost(Q_1)<cost(Q_2)$. Thus, the QGA problem result is $Q_1$ (Figure \ref{fig:assembly_query_graph_q1}).
%It means that we interpret the keywords as a query graph $Q$ and evaluate $Q$ using SPARQL query engine to return answers to users. 
\end{example}

Thus, we tackle the problem of \textit{multi-agent collaborative perception}, an under-studied topic in the literature, where multiple agents are able to exchange information to improve overall accuracy towards perception tasks (\textit{e.g.}, semantic segmentation or object recognition).
One major challenge for multi-agent collaborative perception is the transmission bandwidth, as high bandwidth results in network congestion and latency in the agent network.
We therefore investigate the scenario where information across all agents (and hence sensors) is not available in a centralized manner, and agents can only communicate through bandwidth-limited channels. 
We also consider several challenging scenarios where some sensor data may be uninformative or degraded. 

Prior works on learning to communicate~\cite{sukhbaatar2016learning,foerster2016learning} mainly address decision-making tasks (rather than for improving perception) under simple perceptual environments. 
In addition, these methods also do not consider bandwidth limitations: They learn to communicate across a fully-connected graph (i.e. all agents communicate with each other through broadcasts).
Such methods cannot scale as the number of agents increases. 
Similarly, since all information is broadcast there is no decision of \textit{when} to communicate conditioned on the need. 
An agent does not need to consume bandwidth when the local observation is sufficient for the prediction. 
When messages sent by other agents are degraded or irrelevant, communication thus could be detrimental to the perception task.   

In this paper, we propose a learning-based communication model for collaborative perception.
We specifically view the problem as learning to construct the communication group (i.e. each agent decides what to transmit and which agent(s) to communicate with) and to decide when to communicate without explicit supervision for such decisions during training. 
In contrast to broadcast-based methods (e.g. TarMac~\cite{das2019tarmac}) and inspired by the general attention mechanisms, our method decouples the stages of communication and this allows for \textit{asymmetric message and key sizes}, reducing the amount of transmitted data.

Our method can be generalized to several downstream vision tasks, including multi-agent collaborative semantic segmentation (dense prediction) and multi-agent 3D shape recognition (global prediction). 
Our model is able to be trained in an end-to-end manner with only supervision from downstream tasks (e.g. ground-truth masks for segmentation and class labels for image recognition) and without the need for explicit ground-truth communication labels. 

We demonstrate across different tasks that our method can perform favorably against previous works on learning to communicate while using less bandwidth. 
We provide extensive analyses, including trade-offs between message and query sizes, the correlation between ground-truth key and predicted message, and visualization of the learned communication groups.

Our contributions are listed as follows:
\begin{itemize}[topsep=0pt,itemsep=-1ex,partopsep=1ex,parsep=1ex,labelindent=0.0em,labelsep=0.2cm,leftmargin=*]
\item We address the under-explored area of collaborative perception, which is at the intersection of perception, multi-agent systems, and communication.
\item We propose a unified framework that learns both how to construct communication groups and when to communicate. It does not require ground truth communication labels during training, and it can dynamically reduce bandwidth during inference.
\item Our model can be generalized to several down-stream tasks, and we show through rigorous experimentation that it can perform better when compared with previous works investigating learned communication.
\item  We provide a collaborative multi-agent semantic segmentation dataset, AirSim-MAP, where each agent has its own depth, pose, RBG images, and semantic segmentation masks. This dataset allows researchers to further investigate solutions to multi-agent perception. 
\end{itemize}