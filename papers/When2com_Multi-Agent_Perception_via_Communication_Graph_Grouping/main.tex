\documentclass[10pt,twocolumn,letterpaper]{article}

\usepackage{cvpr}
\usepackage{times}
\usepackage{epsfig}
\usepackage{graphicx}
\usepackage{amsmath}
\usepackage{amssymb}
%%%%Update on 4 March 2021%%%
\usepackage[normalem]{ulem} % For using the command \uuline{} in the \abstract 
\usepackage{amsfonts} % Use for some special math mark
%%\usepackage{pxfonts} % Use for some special math mark 
\usepackage{wasysym} % Use for some special mark
%\usepackage{subfigure} % For some paper the use \subfigure
\usepackage{mathcomp} % For permille mark
\usepackage{CJKutf8} % For Chinese font
\usepackage{pifont} % For some special mark 
\usepackage{bm} % Forr math enviroment bold format
\graphicspath{{./Definitions/}} % Use for import path the figures paper used

% For advanced cross-references for floats for XML2PDF
\usepackage{cleveref}
\crefname{figure}{Figure}{Figures}
\crefname{specialtable}{Table}{Tables}
\crefrangelabelformat{figure}{#3#1#4--#5#2#6}
\crefrangelabelformat{specialtable}{#3#1#4--#5#2#6}
\crefname{section}{Section}{Sections}
\crefname{paragraph}{Section}{Sections}
\crefrangelabelformat{section}{#3#1#4--#5#2#6}
\crefname{appendix}{Appendix}{Appendices}
\crefrangelabelformat{appendix}{#3#1#4--#5#2#6}
\crefname{scheme}{Scheme}{Schemes}
\crefrangelabelformat{scheme}{#3#1#4--#5\crefstripprefix{#1}{#2}#6}
\crefname{chart}{Chart}{Charts}
\crefrangelabelformat{chart}{#3#1#4--#5\crefstripprefix{#1}{#2}#6}
\newcommand{\fig}[1]{Figure~\ref{#1}}
\newcommand{\sect}[1]{Section~\ref{#1}}
\newcommand{\tabref}[1]{Table~\ref{#1}}
\newcommand{\sche}[1]{Scheme~\ref{#1}}
\newcommand{\equ}[1]{\ref{#1}}
\newcommand{\boxref}[1]{Box~\ref{#1}}
\newcommand{\app}[1]{Appendix~\ref{#1}}
\newcommand{\cchart}[1]{Chart~\ref{#1}}

% I don't what the below codes are for?
\makeatletter
\def\T@n@@nc@d@ngM@cr@M@d{}
\def\LY@n@@nc@d@ngM@cr@M@d{}
\makeatother

\let\orignewcommand\newcommand  % Store the original \newcommand
\let\newcommand\providecommand  % Make \newcommand behave like \providecommand
\usepackage{verse}
\let\newcommand\orignewcommand  % Use the original `\newcommand` in future
\makeatletter
\renewcommand*{\theHpoemline}{\arabic{verse@envctr}.\arabic{poemline}} % Use the original definition from verse.sty
\makeatother

% Define a matrix envrioment
\newsavebox\foobox
\newcommand{\slantbox}[2][.2]{\mbox{%
        \sbox{\foobox}{#2}%
        \hskip\wd\foobox
        \pdfsave
        \pdfsetmatrix{1 0 #1 1}%
        \llap{\usebox{\foobox}}%
        \pdfrestore
}}

%\setlength{\fboxsep}{0cm}

% Define triangledown mark
\let\oldblacktriangledown\blacktriangledown

% Define \dagger using unicode
\renewcommand{\dagger}{\mathchar"2279}
\renewcommand{\ddagger}{\mathchar"227A}

% Define italic In, Max
\newcommand{\mmathit}[1]{
  \ifthenelse{\equal{#1}{\ln}}{\mathit{ln}}{
    \ifthenelse{\equal{#1}{\max}}{\mathit{max}}{\mathit{#1}}
  }
}
\makeatother
\robustify{\footnote} % I don't know what it is for? Maybe for Chicago journal footnote?

\pdfstringdefDisableCommands{% % I don't know what it is for?
\def\footnote#1{}%
}

%TODO
\newcommand{\todo}[1]{{\color{red}{\bf [TODO]:~{#1}}}}

%THEOREMS
\newtheorem{theorem}{Theorem}
\newtheorem{corollary}{Corollary}
\newtheorem{lemma}{Lemma}
\newtheorem{proposition}{Proposition}
\newtheorem{problem}{Problem}
\newtheorem{definition}{Definition}
\newtheorem{remark}{Remark}
\newtheorem{example}{Example}
\newtheorem{assumption}{Assumption}

%HANS' CONVENIENCES
\newcommand{\define}[1]{\textit{#1}}
\newcommand{\join}{\vee}
\newcommand{\meet}{\wedge}
\newcommand{\bigjoin}{\bigvee}
\newcommand{\bigmeet}{\bigwedge}
\newcommand{\jointimes}{\boxplus}
\newcommand{\meettimes}{\boxplus'}
\newcommand{\bigjoinplus}{\bigjoin}
\newcommand{\bigmeetplus}{\bigmeet}
\newcommand{\joinplus}{\join}
\newcommand{\meetplus}{\meet}
\newcommand{\lattice}[1]{\mathbf{#1}}
\newcommand{\semimod}{\mathcal{S}}
\newcommand{\graph}{\mathcal{G}}
\newcommand{\nodes}{\mathcal{V}}
\newcommand{\agents}{\{1,2,\dots,N\}}
\newcommand{\edges}{\mathcal{E}}
\newcommand{\neighbors}{\mathcal{N}}
\newcommand{\Weights}{\mathcal{A}}
\renewcommand{\leq}{\leqslant}
\renewcommand{\geq}{\geqslant}
\renewcommand{\preceq}{\preccurlyeq}
\renewcommand{\succeq}{\succcurlyeq}
\newcommand{\Rmax}{\mathbb{R}_{\mathrm{max}}}
\newcommand{\Rmin}{\mathbb{R}_{\mathrm{min}}}
\newcommand{\Rext}{\overline{\mathbb{R}}}
\newcommand{\R}{\mathbb{R}}
\newcommand{\N}{\mathbb{N}}
\newcommand{\A}{\mathbf{A}}
\newcommand{\B}{\mathbf{B}}
\newcommand{\x}{\mathbf{x}}
\newcommand{\e}{\mathbf{e}}
\newcommand{\X}{\mathbf{X}}
\newcommand{\W}{\mathbf{W}}
\newcommand{\weights}{\mathcal{W}}
\newcommand{\alternatives}{\mathcal{X}}
\newcommand{\xsol}{\bar{\mathbf{x}}}
\newcommand{\y}{\mathbf{y}}
\newcommand{\Y}{\mathbf{Y}}
\newcommand{\z}{\mathbf{z}}
\newcommand{\Z}{\mathbf{Z}}
\renewcommand{\a}{\mathbf{a}}
\renewcommand{\b}{\mathbf{b}}
\newcommand{\I}{\mathbf{I}}
\DeclareMathOperator{\supp}{supp}
\newcommand{\Par}[2]{\mathcal{P}_{{#1} \to {#2}}}
\newcommand{\Laplacian}{\mathcal{L}}
\newcommand{\F}{\mathcal{F}}
\newcommand{\inv}[1]{{#1}^{\sharp}}
\newcommand{\energy}{Q}
\newcommand{\err}{\mathrm{err}}
\newcommand{\argmin}{\mathrm{argmin}}
\newcommand{\argmax}{\mathrm{argmax}}

% Include other packages here, before hyperref.

% If you comment hyperref and then uncomment it, you should delete
% egpaper.aux before re-running latex.  (Or just hit 'q' on the first latex
% run, let it finish, and you should be clear).
\usepackage[pagebackref=true,breaklinks=true,letterpaper=true,colorlinks,bookmarks=false]{hyperref}

\cvprfinalcopy % *** Uncomment this line for the final submission
\def\cvprPaperID{1359} % *** Enter the CVPR Paper ID here
\def\httilde{\mbox{\tt\raisebox{-.5ex}{\symbol{126}}}}

% Pages are numbered in submission mode, and unnumbered in camera-ready
\ifcvprfinal\pagestyle{empty}\fi

\begin{document}

%%%%%%%%% TITLE
\title{When2com: Multi-Agent Perception via Communication Graph Grouping}
\author{Yen-Cheng Liu, Junjiao Tian\thanks{equal contribution}, Nathaniel Glaser\footnotemark[1],  Zsolt Kira\\
Georgia Institute of Technology\\
{\tt\small \{ycliu, jtian73, nglaser3, zkira\}@gatech.edu}
% For a paper whose authors are all at the same institution,
% omit the following lines up until the closing ``}''.
% Additional authors and addresses can be added with ``\and'',
% just like the second author.
% To save space, use either the email address or home page, not both
}

\maketitle

%%%%%%%%% ABSTRACT
\begin{abstract}

While significant advances have been made for single-agent perception, many applications require multiple sensing agents and cross-agent communication due to benefits such as coverage and robustness. 
It is therefore critical to develop frameworks which support multi-agent collaborative perception in a distributed and bandwidth-efficient manner. 
In this paper, we address the collaborative perception problem, where one agent is required to perform a perception task and can communicate and share information with other agents on the same task. 
Specifically, we propose a communication framework by learning both to construct communication groups and decide when to communicate. 
We demonstrate the generalizability of our framework on two different perception tasks and show that it significantly reduces communication bandwidth while maintaining superior performance. 

\end{abstract}

%%%%%%%%% BODY TEXT
% !TEX root = ../arxiv.tex

Unsupervised domain adaptation (UDA) is a variant of semi-supervised learning \cite{blum1998combining}, where the available unlabelled data comes from a different distribution than the annotated dataset \cite{Ben-DavidBCP06}.
A case in point is to exploit synthetic data, where annotation is more accessible compared to the costly labelling of real-world images \cite{RichterVRK16,RosSMVL16}.
Along with some success in addressing UDA for semantic segmentation \cite{TsaiHSS0C18,VuJBCP19,0001S20,ZouYKW18}, the developed methods are growing increasingly sophisticated and often combine style transfer networks, adversarial training or network ensembles \cite{KimB20a,LiYV19,TsaiSSC19,Yang_2020_ECCV}.
This increase in model complexity impedes reproducibility, potentially slowing further progress.

In this work, we propose a UDA framework reaching state-of-the-art segmentation accuracy (measured by the Intersection-over-Union, IoU) without incurring substantial training efforts.
Toward this goal, we adopt a simple semi-supervised approach, \emph{self-training} \cite{ChenWB11,lee2013pseudo,ZouYKW18}, used in recent works only in conjunction with adversarial training or network ensembles \cite{ChoiKK19,KimB20a,Mei_2020_ECCV,Wang_2020_ECCV,0001S20,Zheng_2020_IJCV,ZhengY20}.
By contrast, we use self-training \emph{standalone}.
Compared to previous self-training methods \cite{ChenLCCCZAS20,Li_2020_ECCV,subhani2020learning,ZouYKW18,ZouYLKW19}, our approach also sidesteps the inconvenience of multiple training rounds, as they often require expert intervention between consecutive rounds.
We train our model using co-evolving pseudo labels end-to-end without such need.

\begin{figure}[t]%
    \centering
    \def\svgwidth{\linewidth}
    \input{figures/preview/bars.pdf_tex}
    \caption{\textbf{Results preview.} Unlike much recent work that combines multiple training paradigms, such as adversarial training and style transfer, our approach retains the modest single-round training complexity of self-training, yet improves the state of the art for adapting semantic segmentation by a significant margin.}
    \label{fig:preview}
\end{figure}

Our method leverages the ubiquitous \emph{data augmentation} techniques from fully supervised learning \cite{deeplabv3plus2018,ZhaoSQWJ17}: photometric jitter, flipping and multi-scale cropping.
We enforce \emph{consistency} of the semantic maps produced by the model across these image perturbations.
The following assumption formalises the key premise:

\myparagraph{Assumption 1.}
Let $f: \mathcal{I} \rightarrow \mathcal{M}$ represent a pixelwise mapping from images $\mathcal{I}$ to semantic output $\mathcal{M}$.
Denote $\rho_{\bm{\epsilon}}: \mathcal{I} \rightarrow \mathcal{I}$ a photometric image transform and, similarly, $\tau_{\bm{\epsilon}'}: \mathcal{I} \rightarrow \mathcal{I}$ a spatial similarity transformation, where $\bm{\epsilon},\bm{\epsilon}'\sim p(\cdot)$ are control variables following some pre-defined density (\eg, $p \equiv \mathcal{N}(0, 1)$).
Then, for any image $I \in \mathcal{I}$, $f$ is \emph{invariant} under $\rho_{\bm{\epsilon}}$ and \emph{equivariant} under $\tau_{\bm{\epsilon}'}$, \ie~$f(\rho_{\bm{\epsilon}}(I)) = f(I)$ and $f(\tau_{\bm{\epsilon}'}(I)) = \tau_{\bm{\epsilon}'}(f(I))$.

\smallskip
\noindent Next, we introduce a training framework using a \emph{momentum network} -- a slowly advancing copy of the original model.
The momentum network provides stable, yet recent targets for model updates, as opposed to the fixed supervision in model distillation \cite{Chen0G18,Zheng_2020_IJCV,ZhengY20}.
We also re-visit the problem of long-tail recognition in the context of generating pseudo labels for self-supervision.
In particular, we maintain an \emph{exponentially moving class prior} used to discount the confidence thresholds for those classes with few samples and increase their relative contribution to the training loss.
Our framework is simple to train, adds moderate computational overhead compared to a fully supervised setup, yet sets a new state of the art on established benchmarks (\cf \cref{fig:preview}).


\section{Related Work}\label{sec:related}
 
The authors in \cite{humphreys2007noncontact} showed that it is possible to extract the PPG signal from the video using a complementary metal-oxide semiconductor camera by illuminating a region of tissue using through external light-emitting diodes at dual-wavelength (760nm and 880nm).  Further, the authors of  \cite{verkruysse2008remote} demonstrated that the PPG signal can be estimated by just using ambient light as a source of illumination along with a simple digital camera.  Further in \cite{poh2011advancements}, the PPG waveform was estimated from the videos recorded using a low-cost webcam. The red, green, and blue channels of the images were decomposed into independent sources using independent component analysis. One of the independent sources was selected to estimate PPG and further calculate HR, and HRV. All these works showed the possibility of extracting PPG signals from the videos and proved the similarity of this signal with the one obtained using a contact device. Further, the authors in \cite{10.1109/CVPR.2013.440} showed that heart rate can be extracted from features from the head as well by capturing the subtle head movements that happen due to blood flow.

%
The authors of \cite{kumar2015distanceppg} proposed a methodology that overcomes a challenge in extracting PPG for people with darker skin tones. The challenge due to slight movement and low lighting conditions during recording a video was also addressed. They implemented the method where PPG signal is extracted from different regions of the face and signal from each region is combined using their weighted average making weights different for different people depending on their skin color. 
%

There are other attempts where authors of \cite{6523142,6909939, 7410772, 7412627} have introduced different methodologies to make algorithms for estimating pulse rate robust to illumination variation and motion of the subjects. The paper \cite{6523142} introduces a chrominance-based method to reduce the effect of motion in estimating pulse rate. The authors of \cite{6909939} used a technique in which face tracking and normalized least square adaptive filtering is used to counter the effects of variations due to illumination and subject movement. 
The paper \cite{7410772} resolves the issue of subject movement by choosing the rectangular ROI's on the face relative to the facial landmarks and facial landmarks are tracked in the video using pose-free facial landmark fitting tracker discussed in \cite{yu2016face} followed by the removal of noise due to illumination to extract noise-free PPG signal for estimating pulse rate. 

Recently, the use of machine learning in the prediction of health parameters have gained attention. The paper \cite{osman2015supervised} used a supervised learning methodology to predict the pulse rate from the videos taken from any off-the-shelf camera. Their model showed the possibility of using machine learning methods to estimate the pulse rate. However, our method outperforms their results when the root mean squared error of the predicted pulse rate is compared. The authors in \cite{hsu2017deep} proposed a deep learning methodology to predict the pulse rate from the facial videos. The researchers trained a convolutional neural network (CNN) on the images generated using Short-Time Fourier Transform (STFT) applied on the R, G, \& B channels from the facial region of interests.
The authors of \cite{osman2015supervised, hsu2017deep} only predicted pulse rate, and we extended our work in predicting variance in the pulse rate measurements as well.

All the related work discussed above utilizes filtering and digital signal processing to extract PPG signals from the video which is further used to estimate the PR and PRV.  %
The method proposed in \cite{kumar2015distanceppg} is person dependent since the weights will be different for people with different skin tone. In contrast, we propose a deep learning model to predict the PR which is independent of the person who is being trained. Thus, the model would work even if there is no prior training model built for that individual and hence, making our model robust. 

%









\section{Proposed Approach} \label{sec:method}

Our goal is to create a unified model that maps task representations (e.g., obtained using task2vec~\cite{achille2019task2vec}) to simulation parameters, which are in turn used to render synthetic pre-training datasets for not only tasks that are seen during training, but also novel tasks.
This is a challenging problem, as the number of possible simulation parameter configurations is combinatorially large, making a brute-force approach infeasible when the number of parameters grows. 

\subsection{Overview} 

\cref{fig:controller-approach} shows an overview of our approach. During training, a batch of ``seen'' tasks is provided as input. Their task2vec vector representations are fed as input to \ours, which is a parametric model (shared across all tasks) mapping these downstream task2vecs to simulation parameters, such as lighting direction, amount of blur, background variability, etc.  These parameters are then used by a data generator (in our implementation, built using the Three-D-World platform~\cite{gan2020threedworld}) to generate a dataset of synthetic images. A classifier model then gets pre-trained on these synthetic images, and the backbone is subsequently used for evaluation on specific downstream task. The classifier's accuracy on this task is used as a reward to update \ours's parameters. 
Once trained, \ours can also be used to efficiently predict simulation parameters in {\em one-shot} for ``unseen'' tasks that it has not encountered during training. 


\subsection{\ours Model} 


Let us denote \ours's parameters with $\theta$. Given the task2vec representation of a downstream task $\bs{x} \in \mc{X}$ as input, \ours outputs simulation parameters $a \in \Omega$. The model consists of $M$ output heads, one for each simulation parameter. In the following discussion, just as in our experiments, each simulation parameter is discretized to a few levels to limit the space of possible outputs. Each head outputs a categorical distribution $\pi_i(\bs{x}, \theta) \in \Delta^{k_i}$, where $k_i$ is the number of discrete values for parameter $i \in [M]$, and $\Delta^{k_i}$, a standard $k_i$-simplex. The set of argmax outputs $\nu(\bs{x}, \theta) = \{\nu_i | \nu_i = \argmax_{j \in [k_i]} \pi_{i, j} ~\forall i \in [M]\}$ is the set of simulation parameter values used for synthetic data generation. Subsequently, we drop annotating the dependence of $\pi$ and $\nu$ on $\theta$ and $\bs{x}$ when clear.

\subsection{\ours Training} 


Since Task2Sim aims to maximize downstream accuracy after pre-training, we use this accuracy as the reward in our training optimization\footnote{Note that our rewards depend only on the task2vec input and the output action and do not involve any states, and thus our problem can be considered similar to a stateless-RL or contextual bandits problem \cite{langford2007epoch}.}.
Note that this downstream accuracy is a non-differentiable function of the output simulation parameters (assuming any simulation engine can be used as a black box) and hence direct gradient-based optimization cannot be used to train \ours. Instead, we use REINFORCE~\cite{williams1992simple}, to approximate gradients of downstream task performance with respect to model parameters $\theta$. 

\ours's outputs represent a distribution over ``actions'' corresponding to different values of the set of $M$ simulation parameters. $P(a) = \prod_{i \in [M]} \pi_i(a_i)$ is the probability of picking action $a = [a_i]_{i \in [M]}$, under policy $\pi = [\pi_i]_{i \in [M]}$. Remember that the output $\pi$ is a function of the parameters $\theta$ and the task representation $\bs{x}$. To train the model, we maximize the expected reward under its policy, defined as
\begin{align}
    R = \E_{a \in \Omega}[R(a)] = \sum_{a \in \Omega} P(a) R(a)
\end{align}
where $\Omega$ is the space of all outputs $a$ and $R(a)$ is the reward when parameter values corresponding to action $a$ are chosen. Since reward is the downstream accuracy, $R(a) \in [0, 100]$.  
Using the REINFORCE rule, we have
\begin{align}
    \nabla_{\theta} R 
    &= \E_{a \in \Omega} \left[ (\nabla_{\theta} \log P(a)) R(a) \right] \\
    &= \E_{a \in \Omega} \left[ \left(\sum_{i \in [M]} \nabla_{\theta} \log \pi_i(a_i) \right) R(a) \right]
\end{align}
where the 2nd step comes from linearity of the derivative. In practice, we use a point estimate of the above expectation at a sample $a \sim (\pi + \epsilon)$ ($\epsilon$ being some exploration noise added to the Task2Sim output distribution) with a self-critical baseline following \cite{rennie2017self}:
\begin{align} \label{eq:grad-pt-est}
    \nabla_{\theta} R \approx \left(\sum_{i \in [M]} \nabla_{\theta} \log \pi_i(a_i) \right) \left( R(a) - R(\nu) \right) 
\end{align}
where, as a reminder $\nu$ is the set of the distribution argmax parameter values from the \name{} model heads.

A pseudo-code of our approach is shown in \cref{alg:train}.  Specifically, we update the model parameters $\theta$ using minibatches of tasks sampled from a set of ``seen'' tasks. Similar to \cite{oh2018self}, we also employ self-imitation learning biased towards actions found to have better rewards. This is done by keeping track of the best action encountered in the learning process and using it for additional updates to the model, besides the ones in \cref{ln:update} of \cref{alg:train}. 
Furthermore, we use the test accuracy of a 5-nearest neighbors classifier operating on features generated by the pretrained backbone as a proxy for downstream task performance since it is computationally much faster than other common evaluation criteria used in transfer learning, e.g., linear probing or full-network finetuning. Our experiments demonstrate that this proxy evaluation measure indeed correlates with, and thus, helps in final downstream performance with linear probing or full-network finetuning. 






\begin{algorithm}
\DontPrintSemicolon
 \textbf{Input:} Set of $N$ ``seen'' downstream tasks represented by task2vecs $\mc{T} = \{\bs{x}_i | i \in [N]\}$. \\
 Given initial Task2Sim parameters $\theta_0$ and initial noise level $\epsilon_0$\\
 Initialize $a_{max}^{(i)} | i \in [N]$ the maximum reward action for each seen task \\
 \For{$t \in [T]$}{
 Set noise level $\epsilon = \frac{\epsilon_0}{t} $ \\
 Sample minibatch $\tau$ of size $n$ from $\mc{T}$  \\
 Get \ours output distributions $\pi^{(i)} | i \in [n]$ \\
 Sample outputs $a^{(i)} \sim \pi^{(i)} + \epsilon$ \\
 Get Rewards $R(a^{(i)})$ by generating a synthetic dataset with parameters $a^{(i)}$, pre-training a backbone on it, and getting the 5-NN downstream accuracy using this backbone \\
 Update $a_{max}^{(i)}$ if $R(a^{(i)}) > R(a_{max}^{(i)})$ \\
 Get point estimates of reward gradients $dr^{(i)}$ for each task in minibatch using \cref{eq:grad-pt-est} \\
 $\theta_{t,0} \leftarrow \theta_{t-1} + \frac{\sum_{i \in [n]} dr^{(i)}}{n}$ \label{ln:update} \\
 \For{$j \in [T_{si}]$}{ 
    \tcp{Self Imitation}
    Get reward gradient estimates $dr_{si}^{(i)}$ from \cref{eq:grad-pt-est} for $a \leftarrow a_{max}^{(i)}$ \\
    $\theta_{t, j}  \leftarrow \theta_{t, j-1} + \frac{\sum_{i \in [n]} dr_{si}^{(i)}}{n}$
 }
 $\theta_{t} \leftarrow \theta_{t, T_{si}}$
 }
 \textbf{Output}: Trained model with parameters $\theta_T$. 
 \caption{Training Task2Sim}
 \label{alg:train}  
\end{algorithm}

\section{Experimental Evaluation}
\label{sec:experiment}
To demonstrate the viability of our modeling methodology, we show experimentally how through the deliberate combination and configuration of parallel FREEs, full control over 2DOF spacial forces can be achieved by using only the minimum combination of three FREEs.
To this end, we carefully chose the fiber angle $\Gamma$ of each of these actuators to achieve a well-balanced force zonotope (Fig.~\ref{fig:rigDiagram}).
We combined a contracting and counterclockwise twisting FREE with a fiber angle of $\Gamma = 48^\circ$, a contracting and clockwise twisting FREE with $\Gamma = -48^\circ$, and an extending FREE with $\Gamma = -85^\circ$.
All three FREEs were designed with a nominal radius of $R$ = \unit[5]{mm} and a length of $L$ = \unit[100]{mm}.
%
\begin{figure}
    \centering
    \includegraphics[width=0.75\linewidth]{figures/rigDiagram_wlabels10.pdf}
    \caption{In the experimental evaluation, we employed a parallel combination of three FREEs (top) to yield forces along and moments about the $z$-axis of an end effector.
    The FREEs were carefully selected to yield a well-balanced force zonotope (bottom) to gain full control authority over $F^{\hat{z}_e}$ and $M^{\hat{z}_e}$.
    To this end, we used one extending FREE, and two contracting FREEs which generate antagonistic moments about the end effector $z$-axis.}
    \label{fig:rigDiagram}
\end{figure}


\subsection{Experimental Setup}
To measure the forces generated by this actuator combination under a varying state $\vec{x}$ and pressure input $\vec{p}$, we developed a custom built test platform (Fig.~\ref{fig:rig}). 
%
\begin{figure}
    \centering
    \includegraphics[width=0.9\linewidth]{figures/photos/rig_labeled.pdf}
    \caption{\revcomment{1.3}{This experimental platform is used to generate a targeted displacement (extension and twist) of the end effector and to measure the forces and torques created by a parallel combination of three FREEs. A linear actuator and servomotor impose an extension and a twist, respectively, while the net force and moment generated by the FREEs is measured with a force load cell and moment load cell mounted in series.}}
    \label{fig:rig}
\end{figure}
%
In the test platform, a linear actuator (ServoCity HDA 6-50) and a rotational servomotor (Hitec HS-645mg) were used to impose a 2-dimensional displacement on the end effector. 
A force load cell (LoadStar  RAS1-25lb) and a moment load cell (LoadStar RST1-6Nm) measured the end-effector forces $F^{\hat{z_e}}$ and moments $M^{\hat{z_e}}$, respectively.
During the experiments, the pressures inside the FREEs were varied using pneumatic pressure regulators (Enfield TR-010-g10-s). 

The FREE attachment points (measured from the end effector origin) were measured to be:
\begin{align}
    \vec{d}_1 &= \bmx 0.013 & 0 & 0 \emx^T  \text{m}\\
    \vec{d}_2 &= \bmx -0.006 & 0.011 & 0 \emx^T  \text{m}\\
    \vec{d}_3 &= \bmx -0.006 & -0.011 & 0 \emx^T \text{m}
%    \vec{d}_i &= \bmx 0 & 0 & 0 \emx^T , && \text{for } i = 1,2,3
\end{align}
All three FREEs were oriented parallel to the end effector $z$-axis:
\begin{align}
    \hat{a}_i &= \bmx 0 & 0 & 1 \emx^T, \hspace{20pt} \text{for } i = 1,2,3
\end{align}
Based on this geometry, the transformation matrices $\bar{\mathcal{D}}_i$ were given by:
\begin{align}
    \bar{\mathcal{D}}_1 &= \bmx 0 & 0 & 1 & 0 & -0.013 & 0 \\ 0 & 0 & 0 & 0 & 0 & 1 \emx^T  \\
    \bar{\mathcal{D}}_2 &= \bmx 0 & 0 & 1 & 0.011 & 0.006 & 0 \\ 0 & 0 & 0 & 0 & 0 & 1 \emx^T  \\
    \bar{\mathcal{D}}_3 &= \bmx 0 & 0 & 1 & -0.011 & 0.006 & 0 \\ 0 & 0 & 0 & 0 & 0 & 1 \emx^T 
%    \bar{\mathcal{D}}_i &= \bmx 0 & 0 & 1 & 0 & 0 & 0 \\ 0 & 0 & 0 & 0 & 0 & 1 \emx^T , && \text{for } i = 1,2,3
\end{align}
These matrices were used in equation \eqref{eq:zeta} to yield the state-dependent fluid Jacobian $\bar{J}_x$ and to compute the resulting force zontopes.
%while using measured values of $\vec{\zeta}^{\,\text{meas}} (\vec{q}, \vec{P})$ and $\vec{\zeta}^{\,\text{meas}} (\vec{q}, 0)$ in \eqref{eq:fiberIso} yields the empirical measurements of the active force.



\subsection{Isolating the Active Force}
To compare our model force predictions (which focus only on the active forces induced by the fibers)
to those measured empirically on a physical system, we had to remove the elastic force components attributed to the elastomer. 
Under the assumption that the elastomer force is merely a function of the displacement $\vec{x}$ and independent of pressure $\vec{p}$ \cite{bruder2017model}, this force component can be approximated by the measured force at a pressure of $\vec{p}=0$. 
That is: 
\begin{align}
    \vec{f}_{\text{elast}} (\vec{x}) = \vec{f}_{\text{\,meas}} (\vec{x}, 0)
\end{align}
With this, the active generalized forces were measured indirectly by subtracting off the force generated at zero pressure:
\begin{align}
    \vec{f} (\vec{x}, \vec{p})  &= \vec{f}_{\text{meas}} (\vec{x}, \vec{p}) - \vec{f}_{\text{meas}} (\vec{x}, 0)     \label{eq:fiberIso}
\end{align}


%To validate our parallel force model, we compare its force predictions, $\vec{\zeta}_{\text{pred}}$, to those measured empirically on a physical system, $\vec{\zeta}_\text{meas}$. 
%From \eqref{eq:Z} and \eqref{eq:zeta}, the force at the end effector is given by:
%\begin{align}
%    \vec{\zeta}(\vec{q}, \vec{P}) &= \sum_{i=1}^n \bar{\mathcal{D}}_i \left( {\bar{J}_V}_i^T(\vec{q_i}) P_i + \vec{Z}_i^{\text{elast}} (\vec{q_i}) \right) \\
%    &= \underbrace{\sum_{i=1}^n \bar{\mathcal{D}}_i {\bar{J}_V}_i^T(\vec{q_i}) P_i}_{\vec{\zeta}^{\,\text{fiber}} (\vec{q}, \vec{P})} + \underbrace{\sum_{i=1}^n \bar{\mathcal{D}}_i \vec{Z}_i^{\text{elast}} (\vec{q_i})}_{\vec{\zeta}^{\text{elast}} (\vec{q})}   \label{eq:zetaSplit}
%     &= \vec{\zeta}^{\,\text{fiber}} (\vec{q}, \vec{P}) + \vec{\zeta}^{\text{elast}} (\vec{q})
%\end{align}
%\Dan{These will need to reflect changes made to previous section.}
%The model presented in this paper does not specify the elastomer forces, $\vec{\zeta}^{\text{elast}}$, therefore we only validate its predictions %of the fiber forces, $\vec{\zeta}^{\,\text{fiber}}$. 
%We isolate the fiber forces by noting that $\vec{\zeta}^{\text{elast}} (\vec{q}) = \vec{\zeta}(\vec{q}, 0)$ and rearranging \eqref{eq:zetaSplit}
%\begin{align}
%    \vec{\zeta}^{\,\text{fiber}} (\vec{q}, \vec{P})  &= \vec{\zeta}(\vec{q}, \vec{P}) - \vec{\zeta}(\vec{q}, 0)     \label{eq:fiberIso}
%%    \vec{\zeta}^{\,\text{fiber}}_{\text{emp}} (\vec{q}, \vec{P})  &= \vec{\zeta}_{\text{emp}}(\vec{q}, \vec{P}) - %\vec{\zeta}_{\text{emp}}(\vec{q}, 0)
%\end{align}
%Thus we measure the fiber forces indirectly by subtracting off the forces generated at zero pressure.  


\subsection{Experimental Protocol}
The force and moment generated by the parallel combination of FREEs about the end effector $z$-axis  was measured in four different geometric configurations: neutral, extended, twisted, and simultaneously extended and twisted (see Table \ref{table:RMSE} for the exact deformation amounts). 
At each of these configurations, the forces were measured at all pressure combinations in the set
\begin{align}
    \mathcal{P} &= \left\{ \bmx \alpha_1 & \alpha_2 & \alpha_3 \emx^T p^{\text{max}} \, : \, \alpha_i = \left\{ 0, \frac{1}{4}, \frac{1}{2}, \frac{3}{4}, 1 \right\} \right\}
\end{align}
with $p^{\text{max}}$ = \unit[103.4]{kPa}. 
\revcomment{3.2}{The experiment was performed twice using two different sets of FREEs to observe how fabrication variability might affect performance. The results from Trial 1 are displayed in Fig.~\ref{fig:results} and the error for both trials is given in Table \ref{table:RMSE}.}



\subsection{Results}

\begin{figure*}[ht]
\centering

\def\picScale{0.08}    % define variable for scaling all pictures evenly
\def\plotScale{0.2}    % define variable for scaling all plots evenly
\def\colWidth{0.22\linewidth}

\begin{tikzpicture} %[every node/.style={draw=black}]
% \draw[help lines] (0,0) grid (4,2);
\matrix [row sep=0cm, column sep=0cm, style={align=center}] (my matrix) at (0,0) %(2,1)
{
& \node (q1) {(a) $\Delta l = 0, \Delta \phi = 0$}; & \node (q2) {(b) $\Delta l = 5\text{mm}, \Delta \phi = 0$}; & \node (q3) {(c) $\Delta l = 0, \Delta \phi = 20^\circ$}; & \node (q4) {(d) $\Delta l = 5\text{mm}, \Delta \phi = 20^\circ$};

\\

&
\node[style={anchor=center}] {\includegraphics[width=\colWidth]{figures/photos/s0w0pic_colored.pdf}}; %\fill[blue] (0,0) circle (2pt);
&
\node[style={anchor=center}] {\includegraphics[width=\colWidth]{figures/photos/s5w0pic_colored.pdf}}; %\fill[blue] (0,0) circle (2pt);
&
\node[style={anchor=center}] {\includegraphics[width=\colWidth]{figures/photos/s0w20pic_colored.pdf}}; %\fill[blue] (0,0) circle (2pt);
&
\node[style={anchor=center}] {\includegraphics[width=\colWidth]{figures/photos/s5w20pic_colored.pdf}}; %\fill[blue] (0,0) circle (2pt);

\\

\node[rotate=90] (ylabel) {Moment, $M^{\hat{z}_e}$ (N-m)};
&
\node[style={anchor=center}] {\includegraphics[width=\colWidth]{figures/plots3/s0w0.pdf}}; %\fill[blue] (0,0) circle (2pt);
&
\node[style={anchor=center}] {\includegraphics[width=\colWidth]{figures/plots3/s5w0.pdf}}; %\fill[blue] (0,0) circle (2pt);
&
\node[style={anchor=center}] {\includegraphics[width=\colWidth]{figures/plots3/s0w20.pdf}}; %\fill[blue] (0,0) circle (2pt);
&
\node[style={anchor=center}] {\includegraphics[width=\colWidth]{figures/plots3/s5w20.pdf}}; %\fill[blue] (0,0) circle (2pt);

\\

& \node (xlabel1) {Force, $F^{\hat{z}_e}$ (N)}; & \node (xlabel2) {Force, $F^{\hat{z}_e}$ (N)}; & \node (xlabel3) {Force, $F^{\hat{z}_e}$ (N)}; & \node (xlabel4) {Force, $F^{\hat{z}_e}$ (N)};

\\
};
\end{tikzpicture}

\caption{For four different deformed configurations (top row), we compare the predicted and the measured forces for the parallel combination of three FREEs (bottom row). 
\revcomment{2.6}{Data points and predictions corresponding to the same input pressures are connected by a thin line, and the convex hull of the measured data points is outlined in black.}
The Trial 1 data is overlaid on top of the theoretical force zonotopes (grey areas) for each of the four configurations.
Identical colors indicate correspondence between a FREE and its resulting force/torque direction.}
\label{fig:results}
\end{figure*}






% & \node (a) {(a)}; & \node (b) {(b)}; & \node (c) {(c)}; & \node (d) {(d)};


For comparison, the measured forces are superimposed over the force zonotope generated by our model in Fig.~\ref{fig:results}a-~\ref{fig:results}d.
To quantify the accuracy of the model, we defined the error at the $j^{th}$ evaluation point as the difference between the modeled and measured forces
\begin{align}
%    \vec{e}_j &= \left( {\vec{\zeta}_{\,\text{mod}}} - {\vec{\zeta}_{\,\text{emp}}} \right)_j
%    e_j &= \left( F/M_{\,\text{mod}} - F/M_{\,\text{emp}} \right)_j
    e^F_j &= \left( F^{\hat{z}_e}_{\text{pred}, j} - F^{\hat{z}_e}_{\text{meas}, j} \right) \\
    e^M_j &= \left( M^{\hat{z}_e}_{\text{pred}, j} - M^{\hat{z}_e}_{\text{meas}, j} \right)
\end{align}
and evaluated the error across all $N = 125$ trials of a given end effector configuration.
% using the following metrics:
% \begin{align}
%     \text{RMSE} &= \sqrt{ \frac{\sum_{j=1}^{N} e_j^2}{N} } \\
%     \text{Max Error} &= \max \{ \left| e_j \right| : j = 1, ... , N \}
% \end{align}
As shown in Table \ref{table:RMSE}, the root-mean-square error (RMSE) is less than \unit[1.5]{N} (\unit[${8 \times 10^{-3}}$]{Nm}), and the maximum error is less than \unit[3]{N}  (\unit[${19 \times 10^{-3}}$]{Nm}) across all trials and configurations.

\begin{table}[H]
\centering
\caption{Root-mean-square error and maximum error}
\begin{tabular}{| c | c || c | c | c | c|}
    \hline
     & \rule{0pt}{2ex} \textbf{Disp.} & \multicolumn{2}{c |}{\textbf{RMSE}} & \multicolumn{2}{c |}{\textbf{Max Error}} \\ 
     \cline{2-6}
     & \rule{0pt}{2ex} (mm, $^\circ$) & F (N) & M (Nm) & F (N) & M (Nm) \\
     \hline
     \multirow{4}{*}{\rotatebox[origin=c]{90}{\textbf{Trial 1}}}
     & 0, 0 & 1.13 & $3.8 \times 10^{-3}$ & 2.96 & $7.8 \times 10^{-3}$ \\
     & 5, 0 & 0.74 & $3.2 \times 10^{-3}$ & 2.31 & $7.4 \times 10^{-3}$ \\
     & 0, 20 & 1.47 & $6.3 \times 10^{-3}$ & 2.52 & $15.6 \times 10^{-3}$\\
     & 5, 20 & 1.18 & $4.6 \times 10^{-3}$ & 2.85 & $12.4 \times 10^{-3}$ \\  
     \hline
     \multirow{4}{*}{\rotatebox[origin=c]{90}{\textbf{Trial 2}}}
     & 0, 0 & 0.93 & $6.0 \times 10^{-3}$ & 1.90 & $13.3 \times 10^{-3}$ \\
     & 5, 0 & 1.00 & $7.7 \times 10^{-3}$ & 2.97 & $19.0 \times 10^{-3}$ \\
     & 0, 20 & 0.77 & $6.9 \times 10^{-3}$ & 2.89 & $15.7 \times 10^{-3}$\\
     & 5, 20 & 0.95 & $5.3 \times 10^{-3}$ & 2.22 & $13.3 \times 10^{-3}$ \\  
     \hline
\end{tabular}
\label{table:RMSE}
\end{table}

\begin{figure}
    \centering
    \includegraphics[width=\linewidth]{figures/photos/buckling.pdf}
    \caption{At high fluid pressure the FREE with fiber angle of $-85^\circ$ started to buckle.  This effect was less pronounced when the system was extended along the $z$-axis.}
    \label{fig:buckling}
\end{figure}

%Experimental precision was limited by unmodeled material defects in the FREEs, as well as sensor inaccuracy. While the commercial force and moment sensors used have a quoted accuracy of 0.02\% for the force sensor and 0.2\% for the moment sensor (LoadStar Sensors, 2015), a drifting of up to 0.5 N away from zero was noticed on the force sensor during testing.

It should be noted, that throughout the experiments, the FREE with a fiber angle of $-85^\circ$ exhibited noticeable buckling behavior at pressures above $\approx$ \unit[50]{kPa} (Fig.~\ref{fig:buckling}). 
This behavior was more pronounced during testing in the non-extended configurations (Fig.~\ref{fig:results}a~and~\ref{fig:results}c). 
The buckling might explain the noticeable leftward offset of many of the points in Fig.~\ref{fig:results}a and Fig.~\ref{fig:results}c, since it is reasonable to assume that buckling reduces the efficacy of of the FREE to exert force in the direction normal to the force sensor. 

\begin{figure}
    \centering
    \includegraphics[width=\linewidth]{figures/zntp_vs_x4.pdf}
    \caption{A visualization of how the \emph{force zonotope} of the parallel combination of three FREEs (see Fig.~\ref{fig:rig}) changes as a function of the end effector state $x$. One can observe that the change in the zonotope ultimately limits the work-space of such a system.  In particular the zonotope will collapse for compressions of more than \unit[-10]{mm}.  For \revcomment{2.5}{scale and comparison, the convex hulls of the measured points from Fig.~\ref{fig:results}} are superimposed over their corresponding zonotope at the configurations that were evaluated experimentally.}
    % \marginnote{\#2.5}
    \label{fig:zntp_vs_x}
\end{figure}
% \vspace{-0.5em}
\section{Conclusion}
% \vspace{-0.5em}
Recent advances in multimodal single-cell technology have enabled the simultaneous profiling of the transcriptome alongside other cellular modalities, leading to an increase in the availability of multimodal single-cell data. In this paper, we present \method{}, a multimodal transformer model for single-cell surface protein abundance from gene expression measurements. We combined the data with prior biological interaction knowledge from the STRING database into a richly connected heterogeneous graph and leveraged the transformer architectures to learn an accurate mapping between gene expression and surface protein abundance. Remarkably, \method{} achieves superior and more stable performance than other baselines on both 2021 and 2022 NeurIPS single-cell datasets.

\noindent\textbf{Future Work.}
% Our work is an extension of the model we implemented in the NeurIPS 2022 competition. 
Our framework of multimodal transformers with the cross-modality heterogeneous graph goes far beyond the specific downstream task of modality prediction, and there are lots of potentials to be further explored. Our graph contains three types of nodes. While the cell embeddings are used for predictions, the remaining protein embeddings and gene embeddings may be further interpreted for other tasks. The similarities between proteins may show data-specific protein-protein relationships, while the attention matrix of the gene transformer may help to identify marker genes of each cell type. Additionally, we may achieve gene interaction prediction using the attention mechanism.
% under adequate regulations. 
% We expect \method{} to be capable of much more than just modality prediction. Note that currently, we fuse information from different transformers with message-passing GNNs. 
To extend more on transformers, a potential next step is implementing cross-attention cross-modalities. Ideally, all three types of nodes, namely genes, proteins, and cells, would be jointly modeled using a large transformer that includes specific regulations for each modality. 

% insight of protein and gene embedding (diff task)

% all in one transformer

% \noindent\textbf{Limitations and future work}
% Despite the noticeable performance improvement by utilizing transformers with the cross-modality heterogeneous graph, there are still bottlenecks in the current settings. To begin with, we noticed that the performance variations of all methods are consistently higher in the ``CITE'' dataset compared to the ``GEX2ADT'' dataset. We hypothesized that the increased variability in ``CITE'' was due to both less number of training samples (43k vs. 66k cells) and a significantly more number of testing samples used (28k vs. 1k cells). One straightforward solution to alleviate the high variation issue is to include more training samples, which is not always possible given the training data availability. Nevertheless, publicly available single-cell datasets have been accumulated over the past decades and are still being collected on an ever-increasing scale. Taking advantage of these large-scale atlases is the key to a more stable and well-performing model, as some of the intra-cell variations could be common across different datasets. For example, reference-based methods are commonly used to identify the cell identity of a single cell, or cell-type compositions of a mixture of cells. (other examples for pretrained, e.g., scbert)


%\noindent\textbf{Future work.}
% Our work is an extension of the model we implemented in the NeurIPS 2022 competition. Now our framework of multimodal transformers with the cross-modality heterogeneous graph goes far beyond the specific downstream task of modality prediction, and there are lots of potentials to be further explored. Our graph contains three types of nodes. while the cell embeddings are used for predictions, the remaining protein embeddings and gene embeddings may be further interpreted for other tasks. The similarities between proteins may show data-specific protein-protein relationships, while the attention matrix of the gene transformer may help to identify marker genes of each cell type. Additionally, we may achieve gene interaction prediction using the attention mechanism under adequate regulations. We expect \method{} to be capable of much more than just modality prediction. Note that currently, we fuse information from different transformers with message-passing GNNs. To extend more on transformers, a potential next step is implementing cross-attention cross-modalities. Ideally, all three types of nodes, namely genes, proteins, and cells, would be jointly modeled using a large transformer that includes specific regulations for each modality. The self-attention within each modality would reconstruct the prior interaction network, while the cross-attention between modalities would be supervised by the data observations. Then, The attention matrix will provide insights into all the internal interactions and cross-relationships. With the linearized transformer, this idea would be both practical and versatile.

% \begin{acks}
% This research is supported by the National Science Foundation (NSF) and Johnson \& Johnson.
% \end{acks}
%-------------------------------------------------------------------------
\section{Acknowledgement}
\label{sec:acknowledgement}
This work was supported by ONR grant N00014-18-1-2829.

{\small
\bibliographystyle{ieee_fullname}
\bibliography{egbib}
}

\end{document}
