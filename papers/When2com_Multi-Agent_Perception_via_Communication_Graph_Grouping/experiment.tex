\section{Experiment}
We evaluate the performance of our proposed framework on two distinct perception tasks: collaborative semantic segmentation and multi-view 3D shape recognition.  
% The first task is divided into three cases.  All experimental cases are summarized in Figure ~\ref{fig:experimental_cases}.
% We will first introduce the datasets for both tasks and then briefly describe the baseline methods.  


\subsection{Experimental Cases and Datasets}
\section{Dataset}
\label{sec:dataset}
%\sarah{add statistics about distribution of merge patterns}
%\alexey{I added some numbers in the section 4 (around line 270). Detailed numbers are in Appendix. We can move it up here if needed...}
%To create a dataset for self-supervised pretraining, we clone all non-fork repositories with more than 20 stars in GitHub that have C, C++, C\#, Python, Java, JavaScript, TypeScript, PHP, Go, and Ruby as their top language. The resulting dataset comprises over 64 million source code files. 
%\chris{why do we list languages here that we don't ever evaluate on?  A reviewer will find this confusing and ask about it.  We found that language specific models work better than multi-lingual models, right?}

The finetuning dataset is mined from over 100,000 open source software repositories in multiple programming languages with merge conflicts. It contains commits from git histories with exactly two parents, which resulted in a merge conflict.  We replay \texttt{git merge} on the two parents to see if it generates any conflicts. Otherwise, we ignore the merge from our dataset. We use the approach introduced by~\citet{Dinella2021} to extract resolution regions---however, we do not restrict ourselves to conflicts with less than 30 lines only.  Lastly, we extract token-level conflicts and conflict resolution classification labels (introduced in Section \ref{formulation}) from line-level conflicts and resolutions. Tab.~\ref{tab:fintuning_dataset} provides a summary of the finetuning dataset.

\begin{table}[htb]
\centering
\caption{Number of merge conflicts in the dataset.}
\begin{tabular}{llllllllllll} \toprule
\textbf{Programming language} & \textbf{Development set}  & \textbf{Test set} \\ \midrule
C\# & 27874 & 6969 \\ 
JavaScript & 66573 & 16644\\ 
TypeScript & 22422 & 5606\\ 
Java & 103065 & 25767 \\ 
\bottomrule
\end{tabular}
\label{tab:fintuning_dataset}
\end{table}
The finetuning dataset is split into development and test sets in the proportion 80/20 at random at the file-level. The development set is further split into training and validation sets in 80/20 proportion at the merge conflict level.    

\subsubsection{Collaborative Semantic Segmentation}

Our first task is collaborative 2D semantic segmentation of a 3D scene. Given observations (an RGB image, aligned dense depth map, and pose) from several mobile agents, the objective of this task is to produce an accurate 2D semantic segmentation mask for each agent.

Since current semantic segmentation datasets~\cite{geiger2013vision,cordts2016cityscapes,RobotCarDatasetIJRR,hecker2018end} only provide RGB images and labels captured from the perspective of single agent, we thus use AirSim simulator~\cite{airsim2017fsr} to collect our AirSim-MAP (Multi-Agent Perception) dataset.
For this dataset, we fly a swarm of five to six drones with different yaw rates through a series of waypoints in the AirSim ``CityEnviron'' environment. 
We record pose, depth maps, and RGB images for each agent. Note that we also provide semantic segmentation masks for \textbf{all} drones.  

We consider three experimental cases within this task.  We refer to the agent attempting segmentation as the \textbf{requesting} agent, and all other agents as the \textbf{supporting} agents.  Details for each case are listed as follows: 

\noindent
\textbf{Single-Request Multiple-Support (SRMS)} 
This first case examines the effectiveness of communication for a single requesting agent under the assumption that if an agent is degraded, then its original, non-degraded information will be present in one of the supporting agents.  We include a total of five agents, of which only one is selected for possible degradation.  We add noise to a random selection of $50\%$ of this agent's frames, and we randomly replace one of the remaining agents with the \textit{non-degraded} frame of the original agent.  Note that only the segmentation mask of the original agent is used as supervision.  

\noindent
\textbf{Multiple-Request Multiple-Support (MRMS)}
The second case considers a more challenging problem, where multiple agents can suffer degradation.  Instead of requiring a single segmentation output, this case requires segmentation outputs for all agents, degraded and non-degraded.  We follow the setup of the previous case, and we ensure that each of the several degraded requesting agents has a corresponding non-degraded image among its supporting agents.  

\noindent
\textbf{Multiple-Request Multiple-Partial-Support (MRMPS)}
The third case removes the assumption that there exists a clean version of the degraded view among the supporting agents.  Instead, the degraded agent must select the most informative view(s) from the other agents, and these views might have a variable degree of relevance.  Specifically, as the drone group moves through the environment, the images from each drone periodically and partially overlap with those of other drones.  Intuitively, the segmentation output of the requesting drone is only aided from the supporting drones that have overlapping views. 

\begin{table*}
\caption{\textbf{Experimental results on Multiple-Request Multiple-Support and Multiple-Request Multiple-Partial-Support.} Note that we evaluate these models with the metric of mean intersection of union (mIoU) and use MBytes per frame (Mbpf) and the averaged number of links per agent for measuring bandwidth.}
\label{tab:mimo_case1}
\centering
\resizebox{\linewidth}{!}{%
\begin{tabular}{ccccccccccc}
\toprule
& &  \multicolumn{4}{c}{Multiple-Request Multiple-Support} & & \multicolumn{4}{c}{Multiple-Request Multiple-Partial-Support} \\
 \cmidrule(lr){3-6}  \cmidrule(lr){8-11} 
& Models & Bandwidth (Mbpf / $\#$ of links) & Noisy & Normal & Avg.&  & Bandwidth  (Mbpf / $\#$ of links)& Noisy & Normal & Avg.  \\
\midrule
 & AllNorm & - & 57.85 & 57.74 & 57.80&  & - & 47.9 & 48.37 & 48.14  \\
\midrule

\multirow{5}{*}{ Fully-Connect. } & CatAll & 2.5 / 5 & 29.07 & 51.83 & 40.45 & & 2.0 / 4 & 26.86 & 45.27 & 36.07 \\
& AuxAttend & 2.5 / 5 & 33.69 & 56.27 & 44.98  & & 2.0 / 4 & 26.97 & 51.03 & 39.00\\
& CommNet~\cite{sukhbaatar2016learning} & 2.5 / 5 & 23.68 & 52.67 & 38.18 & & 2.0 / 4 & 26.56 & 49.07 & 37.82 \\
& TarMac~\cite{das2019tarmac} & 2.5 / 5 & 51.09 & 56.74 & 53.92 & & 2.0 / 4 & 29.78 & \textbf{51.39} & 40.59 \\
\midrule
\multirow{3}{*}{ Distri. } & RandCom & 0.5 / 1 & 21.22 & 52.74 & 36.98 & & 0.5 / 1 & 24.13 & 45.19 & 34.66 \\
& Who2com~\cite{liu2020who2com} & 0.5 / 1 & 31.96 & 56.11 & 44.04 &  & 0.5 / 1 & 26.97 & 50.71 & 38.84 \\
& Ours & \textbf{0.385 / 0.77} & \textbf{56.52} & \textbf{58.04} & \textbf{57.28} & & \textbf{0.55} / 1.08 & \textbf{30.38} & 51.26 & \textbf{40.82} \\
\midrule
 & OccDeg & - & 30.06 & 56.31 & 43.19&  & - & 25.2 & 46.74 & 35.97 \\
\bottomrule
\end{tabular}}
\end{table*}



\subsubsection{Multi-Agent 3D Shape Classification}
In addition to the semantic segmentation task, we also consider a multi-agent 3D shape classification task.  
For this experimental case, we construct a multi-agent variant of the \textbf{ModelNet 40} dataset~\cite{wu20153d}. 
The original dataset contains 40 common object categories from ModelNet with 100 unique CAD models per category and 12 different views of each model.  
However, our variant adds a communication group structure to the original dataset.  Specifically, we sample three sets of class-based image triplets.  
Each triplet corresponds to a randomly selected 3D object model and each triplet contains three randomly selected 2D views of its corresponding object model.  
To make this problem setting more challenging, we further degrade one image from each triplet. 
The objective of this task is to predict the corresponding object class for each agent by leveraging the information from all agents. 
Figure~\ref{fig:vis_graph} shows an example of the dataset in one trial with 9 agents.
This modified task is essentially a distributed version of the multi-view classification task~\cite{wu20153d}.


\begin{figure}[t]
    \vspace{1mm}
    \begin{center}
    \centerline{\includegraphics[width=\linewidth]{./figure/when2com.pdf}}
    \caption{
        \textbf{Experimental results of Single-Request Multiple-Support.}} 
        \label{fig:when2com_exp}
    \end{center}
    \vspace{-8mm}
\end{figure}


\subsection{Baselines and Evaluation Metrics}
Here we consider several fully-connected (FC) and distributed communication (DistCom) models as our baselines. FC models fuse all of the agents' observations (either weighted or unweighted) whereas DistCom models only fuse a subset of those observations.

\begin{itemize}[topsep=0pt,itemsep=-1ex,partopsep=1ex,parsep=1ex,labelindent=0.0em,labelsep=0.2cm,leftmargin=*]

\item \textit{CatAll} \textbf{(FC)} is a naive FC model baseline which concatenates the encoded image features of all agents prior to subsequent network stages.

\item \textit{Auxiliary-View Attention (AuxAttend;\textbf{FC})} uses an attention mechanism to weight auxiliary views from the supporting agents. 

\item  \textit{RandCom} \textbf{(DistCom)} is a naive distributed baseline which randomly selects one of other agents as a supporting agent. 

\item \textit{Who2com}~\cite{liu2020who2com} \textbf{(DistCom)} excludes self-attention mechanism such that it always communicates with one of the supporting agents. 

\item \textit{OccDeg and AllNorm} are baselines that employ no communication, i.e. each agent (view) independently computes the output for itself. For \textit{OccDeg} the data is degraded similarly as before, while in \textit{AllNorm} we use clean images for all views. These two serve as an upper and lower reference for comparison. 
\end{itemize}

We also consider communication modules of \textit{CommNet}~\cite{sukhbaatar2016learning}, \textit{VAIN}~\cite{hoshen2017vain}, and \textit{TarMac}~\cite{das2019tarmac} as our baseline methods for all multiple-outputs tasks. 
For a fair comparison, we use ResNet18~\cite{he2016deep} as the feature backbone for our and all mentioned  baseline models. For the 3D recognition task, we also add MVCNN~\cite{wu20153d} as a baseline.

We evaluate the performance of all the models with mean IoU on the segmentation task and prediction accuracy on the 3D shape recognition task. In addition, we report Bandwidth of all FC and DistCom models in Megabyte per frame (MBpf). To obtain MBpf, We add the size of the feature vectors which need to be transmitted to the requesters and size of keys broadcast to all supporters and multiply by the number of bytes required for storage.


\begin{table*}
\caption{\textbf{Experimental results on Multi-agent 3D Shape recognition.} We report the accuracy of the degraded split, and all methods perform similar results for the normal split ($\approx 83\%$). }
\label{tab:3d_shape}
\centering
\resizebox{\linewidth}{!}{%
\begin{tabular}{cccc|cccccc}
\toprule
 & OccDeg & AllNorm & RandCom & CatAll & MVCNN~\cite{wu20153d} & CommNet~\cite{sukhbaatar2016learning} & VAIN~\cite{hoshen2017vain} & TarMac~\cite{das2019tarmac} & Ours \\
\midrule
Degraded Split Accuracy ($\%$)&55.02 & 83.66 & 54.28 & 73.82 & 31.80 & 71.52 & 75.09 & 78.73 &\textbf{80.72}\\
Bandwidth (links/MBpf) &  - & - & 0.11 / 0.89 & 1 / 8 & 1 / 8 & 1 / 8 & 1 / 8 & 1 / 8 &  \textbf{0.176} / \textbf{1.32} \\
\bottomrule
\end{tabular}}
\end{table*}
\begin{figure*}[t]
    \vspace{1mm}
    \begin{center}
    \centerline{\includegraphics[width=0.7\linewidth]{./figure/vis_graph_pic.pdf}}
    \caption{
        \textbf{Bipartite communication graph between supporting and requesting agents.}  During the query phase, each requesting agent sends a low-dimensional query vector to all other agents (including itself) to establish communication.  Then during the transmission phase, supporting agents transmit their high-dimensional feature representations.  We visualize the flow of data during the transmission phase, where blue and red arrows refer to internal and external communication, respectively.  More prominent colors and larger numerical values indicate stronger feature weightings, whereas missing arrows represent the pruned links in the communication graph.  Note that these images are ordered for visualization purposes; the actual dataset is unordered, and each agent observes a random class with a random chance of degradation.}
        \label{fig:vis_graph}
    \end{center}
    \vspace{-8mm}
\end{figure*}

\subsection{Quantitative results}

\noindent
\textbf{Single-Request Multiple-Support (SRMS)}
The goal of this case is to examine if our model is able to learn when to communicate and learn who to communicate with for a single requesting agent. 
Figure~\ref{fig:when2com_exp} shows the performance of our proposed model and several baseline models. 
Although most fully-connected methods can improve the prediction mIoU compared with \textit{NoCom}, they need to propagate all information in a fully-connected manner and thus require high bandwidth consumption. 
In contrast, our model reports higher prediction accuracy yet smaller bandwidth usage (Who2com~\cite{liu2020who2com}: $2$ MBpf; ours: $0.98$ MBpf).
Another observation is that our model is able to further improve compared with \textit{Who2com}~\cite{liu2020who2com}. 
This demonstrates the benefit of learning when to communicate, which reduces the waste of bandwidth and prevent detrimental message when the requesting agent has sufficient information and communication is not required. 

\noindent
\textbf{Multiple-Request Multiple-Support (MRMS)}
In this case, we further address a more challenging problem, where multiple agents suffer degradation. 
Each agent should (1) identify when it needs to communicate, (2) decide who to communicate with when it needs to, and (3) avoid the selection of noisy views from the supporting agents. 
We list the experiment results in the Table~\ref{tab:mimo_case1}\zk{Should move tables so that this is Table 1}. 
It can be seen that, when the requesting agents cannot prevent the selection of noisy supporting agents, both \textit{CatAll} and \textit{RandCom} perform even worse than \textit{NoCom}. 
This verifies our intuition that the information from the supporting agents is not always beneficial for the requesting agents, and selection of incorrect information may even hinder the prediction of the requesters.

With the use of attention mechanisms for weighting the feature maps from the supporting agents, both \textit{AuxAttend} and \textit{Who2com}~\cite{liu2020who2com} are able to prevent incorrect views from deteriorating performance and thus improve with respect to \textit{NoCom}, \textit{CatAll}, and \textit{RandCom}. 
However, without learning when to communicate, those models are forced to always request information from at least one supporting agent resulting in both poorer performance and unnecessary bandwidth usage. 

In addition to the above baseline methods, we also consider CommNet~\cite{sukhbaatar2016learning} and TarMac~\cite{das2019tarmac}. 
Even though CommNet integrates the information from other agents by using an average pooling mechanism, it does not improve the prediction of either degraded or non-degraded requesting agents because it indiscriminately incorporates all views. 

On the other hand, TarMac~\cite{das2019tarmac} is able to provide better results compared with the baseline models. 
However, TarMac uses one-way communication and results in large bandwidth usage which presents difficulty in the real scenario. 
On the contrary, our model is not only able to outperform it on both degraded and non-degraded samples, but also consumes less bandwidth by using our asymmetric query mechanism and pruning redundant connections within the network with the activation function. 


\noindent
\textbf{Multiple-Request Multiple-Partial-Support (MRMPS)}
In this case, there is less chance to have completely overlapped observations between any two agents. 
This presents an inherent difficulty in the perception task because only incomplete information is available for the prediction. 
As shown in the right part of Table~\ref{tab:mimo_case1}, the performance improvement margin of all FC and DistCom models is smaller with respect to \textit{NoCom}, in comparison to more significant improvement observed in the previous scenario. 

Nonetheless, we observe that all methods exhibit a similar trend as the previous scenario. Our model is still able to maintain a similar prediction accuracy as fully-connected models, while we only use one-fourth bandwidth for communication across agents. This demonstrates the superior bandwidth-efficiency of our model.

\noindent
\textbf{Multi-agent 3D shape Recognition}
In order to demonstrate the generalization of our model, here we apply our model to the task of multi-agent 3D shape classification.
Table~\ref{tab:3d_shape} provides the quantitative evaluation on this task using our proposed model and other baselines, including \textit{VAIN}~\cite{hoshen2017vain}, \textit{CommNet}~\cite{sukhbaatar2016learning}, and \textit{TarMac}~\cite{das2019tarmac}. Our model is able to perform competitively compared with \textit{TarMac}~\cite{das2019tarmac} with only approximately one-eighth bandwidth usage. We also provide qualitative results in Figure~\ref{fig:vis_graph} to demonstrate the effectiveness of our model, which allows agents to communicate with the correct and informative agents.


\begin{figure}[t]
    \vspace{-6mm}
    \begin{center}
    \centerline{\includegraphics[width=70mm]{./figure/msg_size_pic.pdf}}
    \caption{
        \textbf{Ablation study on varying query size.}} 
        \label{fig:abl_msg}
    \end{center}
        \vspace{-10mm}
\end{figure}


\subsection{Analyses}

To investigate the source of our model's improvement over the baselines, we computed two selection accuracy metrics on the SRMS dataset, \textit{WhenToCom} and \textit{Grouping}. \textit{WhenToCom} accuracy measures how often communication between a requester and a supporter(s) is established and when it is needed; and \textit{Grouping} measures how often the correct group is created when there is indeed communication. We also comment on the trade-off between bandwidth and performance of communication by conducting a controlled experiment on the size of query and key on the 3D recognition dataset. 

\noindent
\textbf{Effect of handshake communication}
As demonstrated in Figure~\ref{fig:hcom_effect}, we conduct an ablation study on the proposed handshake communication. In the \textit{Ours (w/o H-Com)} model, we remove the handshake communication module, so that each agent only uses its local observations to compute both (1) the communication score and (2) its communication group.

We additionally provide the result of \textit{RandCom}.  We observed that our model with the proposed handshake communication offers a significant improvement over both \textit{RandCom} and our model without handshake communication. This finding demonstrates the necessity of communication for deciding when to communicate and who to communicate with. That is, an agent without communication cannot decide what information it needs and which supporter has the relevant information to help better perform on perception tasks.

Figure~\ref{fig:vis_graph} visualizes three examples from the 3D shape recognition task. Each agent clearly knows when communication is needed based on information provided by the supporters and its own observation. For example, in the first three box examples, the degraded agent on the left knows to select an informative view from the other agents; the non-degraded agent in the middle decides to select a more informative view even though it possesses sufficient information; and the third agent decides that communication is not needed because it has the most informative view among all. It is worth mentioning that all 9 views are provided to every agent and the agent needs to identify informative views and detrimental views based on the matching scores. 
\noindent
\textbf{Query and key size}
We further analyze the effect of query and key size on \textit{Grouping} accuracy and classification accuracy on the 3D shape classification task. 
We vary the query size from $1$ to $128$ with a fixed key size of $16$ as shown in Figure~\ref{fig:abl_msg}. We observe that both selection and classification accuracy improve as the message size increases. Our model can perform favorably with a message size of $4$. The same trend is also observed for key sizes. Most noticeably, we find that there exists asymmetry in query-key size. While the selection accuracy saturates at 16-dimensional query, selection accuracy consistently improves with increasing key size until 1024-dimensional key. Our model exploits this asymmetry to save bandwidth in communication while maintaining high performance.  

\begin{figure}[t]
    \vspace{-6mm}
    \begin{center}
    \centerline{\includegraphics[width=0.8\linewidth]{./figure/effect_hcom_pic.pdf}}
    \caption{
        \textbf{Effect of our proposed H-Com.} Handshake communication significantly improves the communication accuracy.} 
        \label{fig:hcom_effect}
    \vspace{-14mm}
    \end{center}
\end{figure}



% \textbf{CommNet}~\cite{sukhbaatar2016learning}  uses a simple average pooling mechanism on every transmitted features collected from other agents. This approach assumes that all features from other agents are equally important for the final prediction. 

% \textbf{VAIN}~\cite{hoshen2017vain} learns a identity vector for each agent and uses Euclidean distance to compute the similarity scores across agents. It further weight the feature based on the similarity scores for prediction. Note that self-attention is intentionally left out in the original paper.

% \textbf{TarMac}~\cite{das2019tarmac} uses a dot-product attention mechanism which requires the size of key and message to be the identical. We experimentally found that this is a bottleneck in band-width limited communication. In fact, we will show that there is a asymmetric relationship between the key and message size and it is important to take this into account to achieve good performance while maintain low transmission bandwidth. 


% . It does not have self-attention, i.e., the model selectively appends weighted features from the other agents to its own feature based on the proposed 3-way communication. 
% Therefore, it lacks the ability to suppress noisy degraded view and the flexibility to not append any other features if they would introduce additional noise. 



















% - Always communication model (cite VAIN, ComNet) does not perform better than the when2com model. 

% - Knowing how to properly use own observation for prediction or use the information for prediction is important for prediction. 

% - Random selection is even worse than single noise model
% This is because the message is not always beneficial and sometimes will distract the model for its prediction. 

% - A naive way to determine when2com is to based on own observations to determine whether to communicate. 



% - CatAll(cite two body)
%   *centralized: waste the bandwidth
%   *cannot scale up as the number of agents increases
  
 
  
% \subsection{Analyses}
% - Always com v.s. when2com
% - Message size 
% - Key size
% - Attention masks


% \subsection{Supervision reduction}

% train with only one ground-truth segmentation mask