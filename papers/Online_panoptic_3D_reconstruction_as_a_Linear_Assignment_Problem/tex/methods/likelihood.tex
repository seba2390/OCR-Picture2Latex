
The likelihood of a detection matching a target can be evaluated in many ways. PanopticFusion \cite{panopticfusion} applies the Intersection over Union (IoU) metric, popular in both evaluation of object detection and image segmentation works \cite{p_voc,panoptic_segmentation}, as well as estimating overlap in object tracking methods \cite{ab3dmot}. On the other hand, Voxblox++ \cite{voxblox++} and it's recent follower \cite{interactive_3d_scenes} simply count intersecting voxels. Applying statistical distance metrics -- \textit{e.g.} the Mahalanobis distance \cite{mahalanobis} -- have also been proposed, increasing accuracy when tracking dynamic objects. \cite{probabilistic_3d_mot,two_stage_data_association}

We found IoU over visible segments to work best in our case. It is computed by dividing the intersection -- in this case, number of intersecting voxels between two segments -- by union -- the total number of voxels between the two. Only parts of the target segments seen in the current camera view are considered, thus objects being only partially visible should not affect the metric as much. The IoU scores are normalised across detection-target pairs to estimate a probability distribution. To avoid setting a fixed threshold for generating new targets, we instead chose to set the threshold as $1/n$, where $n$ is the number of possible matches. In our tests, this strategy seemed to provide better results than any single fixed threshold. The association algorithm's precision and recall could be further tuned by multiplying the threshold with some constant, however we found that to not be necessary in our case.

The same method could be applied with other likelihood metrics as well. Bhattacharyya distance \cite{bhattacharyya} -- a divergence metric between two probability distributions -- was also considered as a likelihood metric. By representing voxel clusters as multivariate Gaussian distributions, we could also take into account the object's shape and size in addition to overlap. However, we found the metric to be less consistent than IoU with the current system and dataset, most likely because many objects are only partially visible.