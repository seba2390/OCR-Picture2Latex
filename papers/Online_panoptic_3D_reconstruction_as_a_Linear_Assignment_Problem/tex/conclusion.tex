
% Future work:
% - implement a system utilising the online panoptic segmentation in real time
% - extend to tracking of dynamic objects
% - other sensor modalities: multiple cameras / LiDAR for increased situational awareness

% Demos and examples at project home page since we don't have room here? (e.g. github)
% - or as an appendix

In this work, we revisited the idea of sequentially integrating panoptic image segmentation to 3D reconstruction introduced in \cite{panopticfusion}. We formulated the task as a Linear Assignment Problem and studied a way of solving it optimally fast enough for real-time applications. Our method seems to outperform earlier works when operating strictly online.

In the future, we aim to research real-time applications that benefit from the more sophisticated scene understanding that panoptic reconstruction offers. Other possible research topics also include extending this method for tracking dynamic objects in 3D simultaneously and applying the method for different sensor modalities. Because input data is only required to be segmented point clouds and pose information, the system could also be adapted for LiDARs and multi-camera setups relatively easily. These would provide increased real-time situational awareness compared to a single camera, albeit at the cost of increased computational requirements. Because the method only processes data seen in the current camera view, it should also be scalable to larger environments as well.
