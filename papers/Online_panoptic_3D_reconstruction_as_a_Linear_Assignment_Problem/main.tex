% This is samplepaper.tex, a sample chapter demonstrating the
% LLNCS macro package for Springer Computer Science proceedings;
% Version 2.20 of 2017/10/04
%
\documentclass[runningheads]{llncs}
%
\usepackage{graphicx}
\usepackage{subfiles}
\usepackage{threeparttable}
\usepackage{subcaption}
\usepackage{amsmath}

\DeclareMathOperator*{\argmax}{argmax}

\setlength{\textfloatsep}{1cm}

\begin{document}
%
\title{Online panoptic 3D reconstruction as a Linear Assignment Problem}
%
%\titlerunning{Abbreviated paper title}
% If the paper title is too long for the running head, you can set
% an abbreviated paper title here
%
\author{Leevi Raivio\inst{1}\orcidID{0000-0002-6902-5201} \and
Esa Rahtu\inst{1}\orcidID{0000-0001-8767-0864}}
%
\authorrunning{L. Raivio and E. Rahtu}
% First names are abbreviated in the running head.
% If there are more than two authors, 'et al.' is used.
%
\institute{Tampere University, Korkeakoulunkatu 7, 33720 Tampere, Finland \\ \email{firstname.lastname@tuni.fi}}
%
\maketitle              % typeset the header of the contribution
%
\begin{abstract}
\subfile{tex/abstract}

\keywords{3D Reconstruction \and Panoptic Segmentation \and Real Time}
\end{abstract}
%
%
%
\section{Introduction}
    \label{sec:intro}
    \subfile{tex/introduction}

\section{Related work}
    \label{sec:background}
    \subfile{tex/background}

\section{Methods}
    \label{sec:methods}
    \subfile{tex/methods/intro}
        
    \subsection{Instance association as a Linear Assignment Problem}
        \label{subsec:association}
        \subfile{tex/methods/association}
      
    \subsection{Integrating panoptic labels into a voxel grid}
        \label{subsec:integration}
        \subfile{tex/methods/integration}
        
    \subsection{Association likelihood estimation}
        \label{subsec:likelihood}
        \subfile{tex/methods/likelihood}

\section{Evaluation}
    \label{sec:evaluation}
    \subfile{tex/evaluation/intro}
     
    \subsection{Data}
        \label{subsec:data}
        \subfile{tex/evaluation/data}
        
    \subsection{Panoptic Quality on ScanNet with an open validation set}
        \label{subsec:scannet_open_test}
        \subfile{tex/evaluation/scannet_open_test}

\section{Conclusion}
    \label{sec:conclusion}
    \subfile{tex/conclusion}
    
\bibliographystyle{splncs04}
\bibliography{refs}

\end{document}
