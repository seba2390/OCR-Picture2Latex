\documentclass[twoside,11pt]{article}

\usepackage{blindtext}

% Any additional packages needed should be included after jmlr2e.
% Note that jmlr2e.sty includes epsfig, amssymb, natbib and graphicx,
% and defines many common macros, such as 'proof' and 'example'.
%
% It also sets the bibliographystyle to plainnat; for more information on
% natbib citation styles, see the natbib documentation, a copy of which
% is archived at http://www.jmlr.org/format/natbib.pdf

% Available options for package jmlr2e are:
%
%   - abbrvbib : use abbrvnat for the bibliography style
%   - nohyperref : do not load the hyperref package
%   - preprint : remove JMLR specific information from the template,
%         useful for example for posting to preprint servers.
%
% Example of using the package with custom options:
%
% \usepackage[abbrvbib, preprint]{jmlr2e}

\usepackage[preprint]{jmlr2e}
\usepackage{hyperref}
\usepackage{url}
\usepackage{mathrsfs, graphicx, multirow}
\usepackage{caption}
\usepackage{subfigure}
\usepackage{amsmath,amsfonts,bm}
\usepackage{color}
\usepackage{amsmath}
\allowdisplaybreaks[4]

% Definitions of handy macros can go here

\newcommand{\dataset}{{\cal D}}
\newcommand{\fracpartial}[2]{\frac{\partial #1}{\partial  #2}}

% Heading arguments are {volume}{year}{pages}{date submitted}{date published}{paper id}{author-full-names}

\usepackage{lastpage}
\jmlrheading{23}{2022}{1-\pageref{LastPage}}{1/21; Revised 5/22}{9/22}{21-0000}{Jiashuo Liu, Jiayun Wu, Bo Li and Peng Cui}

% Short headings should be running head and authors last names

\ShortHeadings{Predictive Heterogeneity: Measures and Applications}{Jiashuo Liu, Jiayun Wu, Bo Li and Peng Cui}

\newcommand{\fix}{\marginpar{FIX}}
\newcommand{\new}{\marginpar{NEW}}
%\newtheorem{theorem}{Theorem}
%\newtheorem{definition}{Definition}
%\newtheorem{proposition}{Proposition}
%\newtheorem{example}{Example}
%\newtheorem{remark}{Remark}



\firstpageno{1}

\begin{document}

\title{Predictive Heterogeneity: Measures and Applications}

\author{\name Jiashuo Liu\thanks{Equal Contributions.} \email liujiashuo77@gmail.com \\
       \addr Department of Computer Science and Technology\\
       Tsinghua University
       \AND
       \name Jiayun Wu$^*$ \email jiayun.wu.work@gmail.com \\
       \addr Department of Computer Science and Technology\\
		Tsinghua University
		\AND 
	   \name Bo Li \email libo@sem.tsinghua.edu.cn\\
	   \addr School of Economics and Management\\
	   Tsinghua University
	   \AND 
	   \name Peng Cui\thanks{Corresponding Author.} \email cuip@tsinghua.edu.cn\\
	   \addr Department of Computer Science and Technology\\
	   Tsinghua University
}

\editor{My editor}

\maketitle

\begin{abstract}%   <- trailing '%' for backward compatibility of .sty file
As an intrinsic and fundamental property of big data, data heterogeneity exists in a variety of real-world applications, such as precision medicine, autonomous driving, financial applications, etc.
For machine learning algorithms, the ignorance of data heterogeneity will greatly hurt the generalization performance and the algorithmic fairness, since the prediction mechanisms among different sub-populations are likely to differ from each other.
In this work, we focus on the data heterogeneity that affects the prediction of machine learning models, and firstly propose the \emph{usable predictive heterogeneity}, which takes into account the model capacity and computational constraints.
We prove that it can be reliably estimated from finite data with probably approximately correct (PAC) bounds.
Additionally, we design a bi-level optimization algorithm to explore the usable predictive heterogeneity from data.
Empirically, the explored heterogeneity provides insights for sub-population divisions in income prediction, crop yield prediction and image classification tasks, and leveraging such heterogeneity benefits the out-of-distribution generalization performance.
\end{abstract}

\begin{keywords}
  Predictive Heterogeneity, Out-of-Distribution Generalization, Computation Constraints
\end{keywords}

\section{Introduction}

Capturing the motion of the human body pose has great values in widespread applications, such as movement analysis, human-computer interaction, films making, digital avatar animation, and virtual reality. 
Traditional marker-based motion capture system can acquire accurate movement information of humans, but is only applicable to limited scenes due to the time-consuming fitting process and prohibitively expensive costs. 
In contrast, markerless motion capture based on RGB image and video processing algorithms is a promising alternative that has attracted numerous research in the fields of deep learning and computer vision. 
%encoded with prior knowledge of realistic human pose and shape deforming
Especially, thanks to the parameteric SMPL model~\citep{smpl:loper2015smpl} and various diverse datasets with 3D annotations~\citep{h36m:ionescu2013human3, mpii3d:mehta2017monocular, 3dpw:von2018recovering}, remarkable progress has been made on monocular 3D human pose and shape estimation and motion capture. 
%such as~\citep{hmr:kanazawa2018end, vibe:kocabas2020vibe,  maed:wan2021encoder, romp:sun2021monocular, spin:kolotouros2019learning, pare:kocabas2021pare}.




Existing regression-based human mesh recovery methods are actually implicitly based on an assumption that predicting 3d body joint rotations and human shape strongly depends on the given image features. 
The pose and shape parameters are directly estimated from the image feature using MLP regressors. 
Nevertheless, due to the inherent ambiguity, the mapping from the 2D image feature to 3D pose and shape is an ill-posed problem.
To achieve accurate pose and shape estimation, it is necessary to initialize the mean pose and shape parameters and use an \textit{iterative residual regression} manner to reduce error. 
Such an end-to-end learning and inference scheme~\citep{hmr:kanazawa2018end} has been proven to be effective in practice, but ignores the temporal information and produces implausible human motions and unsatisfactory pose jitters for video streaming data. 
Video-based methods such as~\citep{hmmr:kanazawa2019learning,vibe:kocabas2020vibe, tcmr:choi2021beyond, mps-net:wei2022capturing} may leverage large-scale motion capture data as priors and exploit temporal information among different frames to penalize implausible motions. 
 They usually enhance singe-frame feature using a temporal encoder and then still use a deep regressor to predict SMPL parameters based on the temporally enhanced image feature,  as shown in the left subfigure of Fig.\ref{fig:temporal_contrast}. 
This scheme, however, is unable to focus on joint-level rotational motion specific to each joint, failing to ensure the temporal consistency of local joints. 
To address these problems, we attempt to understand the human 3D reconstruction from a causal perspective. 
We argue that assuming a still background, the primary causes behind the image pixel changes and human body appearance changes are 1) the motions of 3D joint rotations in human skeletal dynamics and 2) the viewpoint changes of the observer (camera). 
In fact, a prior human body model exists independently of a given specific image. And the 3D relative rotations of all joints (relative to the parent joint) and body shape can be abstracted beyond image pixels and independent of the image contents and observer views. In other words, the joint rotations cannot be ``seen'' and they are image-independent and viewpoint-independent concepts.  %Additionally, the motion of the combined pose is huge and highly complicated, while the rotation motion of each separate joint is more tracable and periodic.

\begin{figure}
	\centering
	\includegraphics[width=0.95\linewidth]{figures/simple_contrast.pdf}
	\caption{\textbf{Left:} Mainstream temporal-based human mesh methods, e.g.~\citep{hmmr:kanazawa2019learning,vibe:kocabas2020vibe, tcmr:choi2021beyond}, adopt a temporal encoder to mix temporal information from past and future frames and then regress the SMPL parameters from the \textit{temporally enhanced feature} for each frame. \textbf{Right:} Our method first acquires tokens of each joint in the time dimension and then separately capture the motion of each joint using a shared temporal encoder.
	}
	\label{fig:temporal_contrast}\vspace{-0.2in}
\end{figure}

Based on the considerations above, we propose a novel 3D human pose and shape estimation model based on \textbf{in}dependent \textbf{t}okens (INT). 
The core idea of the model is to introduce three types of independent tokens that specifically encode the 3D rotation information of every joint, the shape of human body and the information about camera. 
These initialized tokens learn prior knowledge and mutual relationships from large-scale training data, requiring neither an iterative regressor to take mean shape and pose as initial estimate~\citep{hmr:kanazawa2018end, spin:kolotouros2019learning, vibe:kocabas2020vibe, tcmr:choi2021beyond}, nor a kinematic topology decoder defined by human prior knowledge~\citep{maed:wan2021encoder} . 
Given an image as a conditional observation, these tokens are repeatedly updated by interacting with 2D image evidence using a Transformer~\citep{transformer:vaswani2017attention}. 
Finally, they are transformed into the posterior estimates of pose, shape and camera parameters. 
As a consequence, this method of abstracting joint rotation tokens from image pixels can represent the motion state of each joint and establish correlations in time dimension.
 Benefiting from this, we can separately capture the temporal rotational motion of every joint by sending the tokens of each joint at different timestamps to a temporal model.
In comparison to capturing the overall temporal changes in image features and the whole pose, this modeling scheme focuses on capturing separate rotational motions of all joints, which is conducive to maintaining the temporal coherence and rationality of each joint rotation.

We evaluate our model on the challenging 3DPW~\citep{3dpw:von2018recovering} benchmark and Human3.6m~\citep{h36m:ionescu2013human3}. 
Using vanilla ResNet-50 and Transformer architectures, our model obtains 42.0 mm error in PA-MPJPE metric for 3DPW, outperforming all state-of-the-art counterparts with a large margin. The same model obtains 38.4 mm error in PA-MPJPE metric for Human3.6m, which is on par with the state-of-the-art methods. 
Also, the qualitative results show that our model produces accurate pixel-alignment human mesh reconstructions for indoor or in-the-wild images, and shows fewer motion jitters in local joints when processing video data. 
We strongly encourage the readers to see the video results in the supplementary materials for reference and comparison.

 %simple and straightforward, We first and then extend it to capture separate rotational motion of each joint.

% 可视化prior learned pose and shape


\section{Preliminaries on Mutual Information and Predictive $\mathcal V$-Information}
In this section, we briefly introduce the mutual information and predictive $\mathcal V$-information \citep{DBLP:conf/iclr/XuZSSE20} which are the preliminaries of our proposed predictive heterogeneity.\\\\
\textbf{Notations.} For a probability triple $(\mathbb S,  \mathcal F, \mathbb P)$, define random variables $X: \mathbb S\rightarrow \mathcal X$ and $Y: \mathbb S\rightarrow \mathcal Y$ where $\mathcal X$ is the covariate space and $\mathcal Y$ is the target space. Accordingly. $x \in \mathcal X$ denotes the covariates, and $y\in\mathcal{Y}$ denotes the target. Denote the set of random categorical variables as $\mathcal C = \{ C: \mathbb S \rightarrow \mathbb N| \text{supp}(C) \;\text{is finite} \}$. Additionally, $\mathcal{P}(\mathcal{X}), \mathcal{P}(\mathcal Y)$ denote the set of all probability measures over the Borel algebra on the spaces $\mathcal{X}, \mathcal{Y}$ respectively. 
$H(\cdot)$ denotes the Shannon entropy of a discrete random variable and the differential entropy of a continuous variable, and $H(\cdot|\cdot)$ denotes the conditional entropy of two random variables.




In information theory, the mutual information of two random variables $X$, $Y$ measures the dependence between the two variables, which quantifies the reduction of entropy for one variable when observing the other:
\begin{small}
\begin{equation}
	\mathbb{I}(X;Y) = H(Y) - H(Y|X).
\end{equation}	
\end{small}
It is known that the mutual information is associated with the predictability of $Y$ \citep{cover1991infomationtheory}. While the standard definition of mutual information unrealistically assumes the unbounded computational capacity of the predictor, rendering it hard to estimate especially in high dimensions.
To mitigate this problem, \cite{DBLP:conf/iclr/XuZSSE20} raise the predictive $\mathcal V$-information under realistic computational constraints, where the predictor is only allowed to use models in the predictive family $\mathcal V$ to predict the target variable $Y$.

\begin{definition}[Predictive Family \citep{DBLP:conf/iclr/XuZSSE20}]
	Let $\Omega=\{f:\mathcal{X}\cup\{\emptyset\}\rightarrow \mathcal{P}(\mathcal Y)\}$. We say that $\mathcal V \subseteq \Omega$ is a predictive family if it satisfies:
	\begin{small}
	\begin{equation}
	\label{equ:condition}
		\forall f\in\mathcal{V},\ \  \forall P\in \mathrm{range}(f),\ \  \exists f'\in\mathcal{V}, \quad\text{s.t. }\forall x\in\mathcal{X}, f'[x]=P, f'[\emptyset]=P.
	\end{equation}
	\end{small}
\end{definition}
A predictive family contains all predictive models that are allowed to use, which forms computational or statistical constraints.
The additional condition in Equation \ref{equ:condition} means that the predictor can always ignore the input covariates ($x$) if it chooses to (only use $\emptyset$).

\begin{definition}[Predictive $\mathcal V$-information \citep{DBLP:conf/iclr/XuZSSE20}]
\label{def:predictive_v_information}
	Let $X, Y$ be two random variables taking values in $\mathcal{X}\times\mathcal{Y}$ and $\mathcal V$ be a predictive family. The predictive $\mathcal V$-information from $X$ to $Y$ is defined as:
	\begin{small}
	\begin{equation}
		\mathbb{I}_{\mathcal V}(X\rightarrow Y) = H_{\mathcal V}(Y|\emptyset)-H_{\mathcal V}(Y|X),
	\end{equation}	
	\end{small}
	where $H_{\mathcal V}(Y|\emptyset)$, $H_{\mathcal V}(Y|X)$ are the predictive conditional $\mathcal V$-entropy defined as:
	\begin{small}
	\begin{align}
		H_{\mathcal V}(Y|X) &= \inf\limits_{f\in\mathcal{V}}\mathbb{E}_{x,y\sim X,Y}[-\log f[x](y)]. \\
		H_{\mathcal V}(Y|\emptyset) &= \inf\limits_{f\in\mathcal{V}}\mathbb{E}_{y\sim Y}[-\log f[\emptyset](y)].
	\end{align}	
	\end{small}
	Notably that $f\in\mathcal V$ is a mapping: $\mathcal{X}\cup\{\emptyset\}\rightarrow \mathcal{P}(\mathcal Y)$, so $f[x]\in\mathcal{P}(\mathcal{Y})$ is a probability measure on $\mathcal{Y}$, and $f[x](y)\in\mathbb{R}$ is the density evaluated on $y\in\mathcal Y$. $H_{\mathcal V}(Y|\emptyset)$ is also denoted as $H_{\mathcal V}(Y)$.
\end{definition}


Compared with the mutual information, the predictive $\mathcal V$-information restricts the computational power and is much easier to estimate in high-dimensional cases.
When the predictive family $\mathcal V$ contains all possible models, i.e. $\mathcal V = \Omega$, it is proved that $\mathbb{I}_{\mathcal V}(X\rightarrow Y)=\mathbb{I}(X;Y)$ \citep{DBLP:conf/iclr/XuZSSE20}.



























\section{Method}


Our goal is to build a model that represents joint rotations, shape and camera information using tokens independent of image feature and further captures the rotational motion information of each joint from video data.
In this section, we first revisit prior SMPL-based human mesh recovery methods and then describe our model design.

\begin{figure}[t]
	\centering
	\includegraphics[width=0.8\linewidth]{figures/basemodel.pdf}
	\caption{Base model for singe-frame input. We first use an image encoder to extract feature maps from a given cropped image, then we flatten the feature map into a sequence and add a learnable position embedding. The learnable joint rotation tokens, shape and camera tokens are appended to the sequence and sent to the transformer. Finally, we use three linear heads, called rotation head, shape head and camera head, to convert the joint rotation tokens, shape token and camera token to the SMPL parameters, for 3D mesh reconstruction and 2D reprojection on the image plane.}
	\label{fig:base_model}\vspace{-0.1in}
\end{figure}



%In this section, we first revisit the representative SMPL based 3D human pose estimation and mesh recovery method, including the image-based and video-based estimation. Second, we introduce our token representation for estimating the pose, shape and camera parameters, designed for single-frame images using transformer. Then, we further model the temporal 

\subsection{Revisiting SMPL-based human mesh recovery}

The classic human mesh recovery (HMR) methods~\citep{hmr:kanazawa2018end, hmmr:kanazawa2019learning, vibe:kocabas2020vibe} represent human body as a mesh using the parameteric SMPL~\citep{smpl:loper2015smpl} model. The SMPL mesh model is a differentiable function that output 6890 surface vertices $\mathcal{M}(\boldsymbol{\theta}, \boldsymbol{\beta}) \in \mathbb{R}^{6890 \times 3}$, which are deformed with linear blend skinning driven by the pose $\boldsymbol{\theta} \in \mathbb{R}^{72}$ and shape $\boldsymbol{\beta} \in \mathbb{R}^{10}$ parameters. The pose $\boldsymbol{\theta}$ parameter include the global rotation $R$ and 23 relative joint rotations in axis-angle format. To obtain the 3D positions of body joints, a pretrained linear regressor $W$ is used to achieve $J_{3d}=W\mathcal{M}(\boldsymbol{\theta}, \boldsymbol{\beta})$. To leverage 2D joint supervision, a weak-perspective camera model is usually used to project 3D joint positions into the 2D image plane, i.e., $J_{2d}=s\Pi(RJ_{3d})+\boldsymbol{t} $, where $\Pi$ is an orthographic projection, the scale value $s$ and translation $\boldsymbol{t}\in \mathbb{R}^{2}$ are camera related parameters.

For the image-based HMR methods like~\citep{hmr:kanazawa2018end,spin:kolotouros2019learning}, an image encoder $f(\cdot)$ and a MLP regressor are used to estimate the set of reconstruction parameters $\boldsymbol{\Theta}=\{\boldsymbol{\theta}, \boldsymbol{\beta}, s, t\}$, which constitutes an 85-dim vector to regress. These parameters are iteratively regressed from the encoded image feature vector $\boldsymbol{f}$ by the regressor. 
For the video-based HMR methods like~\citep{vibe:kocabas2020vibe,tcmr:choi2021beyond,hmmr:kanazawa2019learning,maed:wan2021encoder,mps-net:wei2022capturing}, temporal models based on 1D convolution~\citep{hmmr:kanazawa2019learning}, GRU~\citep{tcmr:choi2021beyond} and self-attention models~\citep{vibe:kocabas2020vibe,maed:wan2021encoder} are introduced to capture the motion information in consecutive video frames. A temporal encoder $g(\cdot)$ is exploited to achieve temporally encoded feature vector, formulated as a process like: $\boldsymbol{m}_t=g(\boldsymbol{f}_{t-T/2}, ..., \boldsymbol{f}_{t},...,\boldsymbol{f}_{t+T/2})$. Also, a regressor is used to estimate the $\boldsymbol{\Theta}_t$ from the current frame's feature $\boldsymbol{m}_t$ encoded with temporal information. 
%Unlike the classic human mesh recovery methods, which estimates the pose, shape and camera parameters from the encoded

%First, our model estimates 3D human pose and reconstructs the body mesh from a single RGB image containing centered single person. 
%Unlike the classic human mesh recovery (HMR) methods that use an iterative regressor to estimate the SMPL pose $\theta \in \mathbb{R}^{72}$, shape $\beta \in \mathbb{R}^{10}$ and camera (including translation $t\in R^2$ and scale $s\in R$) paramters from the encoded image feature and initial pose & shape parameters. Besides extracted feature representation, we introduce three types of addition token representations: joint tokens, shape token, and camera token. We represent the 3D pose information as 24 joint rotation token vectors denoted as $J \in \mathbb{R}^{24\times d}$, each of which has $d$-length dimension. 

\subsection{Estimating SMPL parameters based on independent tokens}

%
% A common design of aforementioned methods is that the pose, shape and camera parameters are directly regressed from the extracted feature vector using a MLP regressor. 
%In our framework, we instead hypothesize that, the 3D information about joint rotations, body shape and camera parameters are independent of 2D image pixels, though they are strongly related with image pixels. To embody this, we encode the 3D joint rotations, shape and camera information into learnable token embeddings, the source of which is not image feature but can interact with image feature. 


In this section, we first introduce the token based semantic representation and then describe our model design for single frame input, mainly including the \verb|Base model| and \verb|SMPL heads|.



 {\bf Joint rotation tokens, shape token \& camera token.} We introduce three types of token representations: (1) joint rotation tokens consist of 24 tokens, each of which encodes the joint 3D relative rotation information (including the global rotation) ${\boldsymbol{r}_i\in \mathbb{R}^d}, i=1,..,24$; (2) shape token is a token vector $\boldsymbol{s}\in \mathbb{R}^d$ encoding the body shape information; (3) camera token is also a token vector $\boldsymbol{c}\in \mathbb{R}^d$ encoding the translation and scale information. $d$ is the vector dimension for all tokens.
 %\verb|[joint]|

{\bf Base model.} Inspired by ViT ~\citep{vit:dosovitskiy2020image} and TokenPose~\citep{tokenpose:li2021tokenpose}, we embody our scheme into a Transformer-based architecture design (Fig.~\ref{fig:base_model}). We adopt a CNN to extract image feature map $\boldsymbol{f} \in \mathbb{R}^{c\times h \times w}$ from a given RGB image $I$ cropped with a human body. We reshape the extracted feature map into a sequence of flattened patches and apply patch embedding $\mathbf{E}$ (linear transformation) to each patch to achieve the sequence $\boldsymbol{f}_p \in \mathbb{R}^{S\times d}$ where $S=h\times w$. We append the totally learnable joint rotation tokens, shape token and camera token to the sequence $\boldsymbol{f}_p$, namely \textit{prior tokens}.
We only inject the learnable position embedding $\mathbf{PE} \in \mathbb{R}^{S\times d}$ into the $\boldsymbol{f}_p$ to preserve the 2D structure position information. Then we send the whole sequence $\boldsymbol{S}_0 \in \mathbb{R}^{(S+24+1+1)\times d}$ to a standard Transformer encoder with $L$ layers and achieve these corresponding tokens from the final layer. 
%We denote the process above as the \verb|Base model|

%in the sequence $S_{L}$

{\bf SMPL heads.} To achieve the estimated SMPL parameters $\Theta=\{\boldsymbol{\theta}, \boldsymbol{\beta}, s, t\}$ for 3D human mesh reconstruction, we use three \textit{linear} SMPL heads -- \textit{rotation head}, \textit{shape head} and \textit{camera head} -- to transform the corresponding tokens outputted from the final transformer layer. Particularly, the rotation head (a shared linear layer) transforms each joint token into a 6D rotation representation~\citep{rot6d:zhou2019continuity}. %It is empirically adopted in common methods like~\citep{spin:kolotouros2019learning, maed:wan2021encoder}. 
The shape and camera heads (two linear layers) separately transform the shape token and camera token into the shape parameters (10-dim vector) and camera parameters (3-dim vector). Finally, we convert the 6D rotation representations to SMPL pose (in axis-angle format) and use these parameters to generate the human mesh. Further, we obtain the predicted joint 3D locations $\boldsymbol{J}_{3d}$ and then project them into 2D locations $\boldsymbol{J}_{2d}$ using a weak-perspective camera model.


%, enabling the joint tokens to directly reflect the real rotation state. 



% In fact, our multi-layer transformer architecture still reserve the spirit of iteration, since each transformer layer process the tokens in a fixed compute manner. But the learnable token embeddings can capture the data-driven geometric priors between joint tokens, rather than only dependent on image feature and initialized SMPL parameters

% {\bf ViT-like spatial Transformer.}


\subsection{Rotational motion capturing using a temporal Transformer}


\begin{figure}
	\centering
	\includegraphics[width=.9\linewidth]{figures/temporal.pdf}
	\caption{The overall temporal model framework. \textbf{\uppercase\expandafter{\romannumeral1}. The base model}. We feed the frames of a given video clip to the same base model and achieve the tokens for each frame. \textbf{\uppercase\expandafter{\romannumeral2}. The temporal model}. We use a transformer as the rotation motion encoder to capture the motion of each joint. 
	\textbf{\uppercase\expandafter{\romannumeral3}. The SMPL heads}. We feed the updated joint tokens, shape token and camera token of each frame to the SMPL heads shared with image-based model, to achieve the final SMPL parameters.}
	\label{fig:overall}\vspace{-0.2in}
\end{figure}
We aim to capture the rotational motion at the joint level. Given a video clip $V = \{I_t\}_{t=1}^T$ of length $T$, we feed these $T$ frames to the Base model and acquire the estimated $N$ joint rotation tokens for each frame: $\{\hat{\boldsymbol{r}}_1^t,...,\hat{\boldsymbol{r}}_N^t\}_{t=1}^T \in \mathbb{R}^{T\times N \times d}$, from the final transformer layer. 

We use another standard Transformer as the temporal model to capture the motion of each joint; we denote it as \verb|Temporal Transformer|. For the $n$-th type joint token $\boldsymbol{r}_n$, such as the left knee, the token sequence formed in the time axis is $X_n=\{\hat{\boldsymbol{r}}_n^1, \hat{\boldsymbol{r}}_n^2, ..., \hat{\boldsymbol{r}}_n^T\} \in \mathbb{R}^{T \times d}$, where $n\in\{1,...,N\}$. We feed each sequence $X_n$ to the Temporal Transformer and achieve a new sequence $X_n'$, so that each updated joint token from a particular moment is mixed with the joint rotation information from past and future frames. 
%For $N$ such sequences, i.e., $\{X_1, ..., X_n\}$, they are send to the same Temporal Transformer. 
Then, we  reshape the temporally updated tokens $\{X_1', ..., X_n'\}$ from $N \times T \times d$ to $T \times N \times d$. 
 Note that the temporal information in camera and shape tokens is not taken into consideration in our model since we hope to capture the pure rotational motion information of joints. 
 Finally, for timestamp $t$, the updated $N$ joint tokens are then fed into the rotation head to achieve the joint rotations $\boldsymbol{\theta}_t$; the shape token and camera token outputted from the Base model are fed to the shape head and camera head to achieve the $\boldsymbol{\beta}_t, s_t, \boldsymbol{t}_t$. The overall framework is shown in Fig.~\ref{fig:overall}.

%The Temporal Transformer can be viewed as a model to update the joint tokens using the rotational information from future and past. 

%
%{\bf Masked modeling.} We also conduct masking some tokens in the time sequence to improve the robustness in temporal consistency. Inspired by PoseBERT and MAE, we conduce a more simple masked modeling by reconstructing the joint tokens in time axis. We randomly mask $m\%$ tokens, such as 12.5\%, 

\subsection{Loss function}
Leveraging full supervision from different formats of annotations is critical to train the model well and attain the generalization in different cases.
Following common human mesh recovery methods, we use SMPL parameters loss, L2 normalization, 3D joint location loss and projected 2D joint location loss, when the corresponding SMPL, 3D/2D location supervision signals are available.
$$\mathcal{L}_{smpl}=w_{\theta} \cdot \left\|\boldsymbol{\theta}-\boldsymbol{\theta}_{gt}\right\|_2 +   w_{\beta} \cdot \left\|\boldsymbol{\beta}-\boldsymbol{\beta}_{gt}\right\|_2,$$
$$\mathcal{L}_{norm}= \left\| \boldsymbol{\theta}\right\|_2 + \left\| \boldsymbol{\beta}\right\|_2,$$
$$\mathcal{L}_{3D}=\left\| \boldsymbol{J}_{3d} - \boldsymbol{J}_{3dgt}\right\|_2 , \mathcal{L}_{2D}=\left\| \boldsymbol{J}_{2d} - \boldsymbol{J}_{2dgt}\right\|_2, $$
$$\mathcal{L}_{temp}=\left\| (\boldsymbol{J}_{3d}^{t+1}- \boldsymbol{J}_{3d}^{t}) - (\boldsymbol{J}_{3dgt}^{t+1}- \boldsymbol{J}_{3dgt}^{t}) \right\|_2,$$
$$\mathcal{L}=\mathcal{L}_{smpl} + w_{norm} \cdot \mathcal{L}_{norm} + w_{3D} \cdot \mathcal{L}_{3D} + w_{2D} \cdot \mathcal{L}_{2D} + w_{temp} \cdot \mathcal{L}_{temp}.$$

$\boldsymbol{\theta}_{gt}, \boldsymbol{\beta}_{gt}$ are the groundtruth SMPL parameters. $\boldsymbol{J}_{3D}, \boldsymbol{J}_{2D}$ are the groundtruth joint 3D and 2D locations. The $\mathcal{L}_{temp}$ is a temporal loss for video data, which supervises the velocity of the joint temporal movement in 3D space. The $w_{\theta}, w_{\beta}, w_{norm}, w_{3D}, w_{2D}, w_{temp}$ are the weights to balance all of loss functions (see more details in Appendix~\ref{appendix:training}).

\section{Algorithm}

To empirically estimate the predictive heterogeneity in Definition \ref{def:empirical_predictive_heterogeneity}, we derive the Information Maximization (IM) algorithm from the formal definition in Equation \ref{equ:appendix-formal-empirical} to infer the distribution of $\mathcal{E}$ that maximizes the empirical predictive heterogeneity $\hat{\mathcal{H}}^{\mathscr{E}_K}_{\mathcal V}(X\rightarrow Y; \mathcal D)$.
% Actually, our objective function in Equation \label{equ:bi-level} is derived from the formal definition of $\hat{\mathcal{H}}^{\mathscr{E}_K}_{\mathcal V}(X\rightarrow Y; \mathcal D)$ in Equation \ref{equ:appendix-formal-empirical}.




\textbf{Objective Function.}\quad Given dataset $\mathcal D=\{X_N,Y_N\} = \{(x_i,y_i)\}_{i=1}^N$, denote $\text{supp}(\mathcal{E})=\{e_1, \dots, e_K\}$, we parameterize the distribution of $\mathcal{E}|(X_N,Y_N)$ with weight matrix $W\in\mathcal{W}_K$, where $K$ is the pre-defined number of environments and $\mathcal{W}_K=\{W: W\in\mathbb{R}_+^{N\times K}\text{ and }W\textbf{1}_K=\textbf{1}_N\}$ is the allowed weight space.
Each element $w_{ij}$ in $W$ represents $P(\mathcal{E}=e_j|x_i,y_i)$ (the probability of the $i$-th data point belonging to the $j$-th sub-population).
% The goal of our information maximization algorithm is to learn the best sub-population assignment matrix $W$ that maximizes the empirical predictive heterogeneity.
For a predictive family $\mathcal V$, the solution to the supremum problem in the Definition \ref{def:empirical_predictive_heterogeneity} is equivalent to the following objective function:
\begin{small}
\begin{equation}
\label{equ:bi-level}
\begin{aligned}
	\min\limits_{W\in\mathcal{W}_K} &\mathcal{R}_{\mathcal V}(W,\theta_1^*(W),\dots,\theta^*_K(W))=\left\{\frac{1}{N}\sum_{i=1}^N\sum_{j=1}^K w_{ij}\ell_{\mathcal V}(f_{\theta_j^*}(x_i),y_i) + U_{\mathcal V}(W,Y_N)
	\right\},\\
	&\text{s.t.}\quad  \theta^*_j(W) \in \arg\min_{\theta} \left\{\mathcal{L}_{\mathcal V}^j(W,\theta)=\sum_{i=1}^Nw_{ij}\ell_{\mathcal V}(f_{\theta}(x_i),y_i)\right\},\quad \text{for }j=1,\dots,K,
\end{aligned}		
\end{equation}
\end{small}
where $f_{\theta}:\mathcal{X}\rightarrow \mathcal{Y}$ denotes a predicting function parameterized by $\theta$, $\ell_{\mathcal{V}}(\cdot,\cdot):\mathcal{Y}\times\mathcal{Y}\rightarrow \mathbb{R}$ represents a loss function and $U_{\mathcal V}(W,Y_N)$ is a regularizer.
Specifically, $f_\theta$, $\ell_{\mathcal V}$ and $U_{\mathcal V}$ are determined by the predictive family $\mathcal{V}$. 
% \xrz{Since $U(W,Y_N)$ is determined by the predictive family, it would be better to write it as $U_{\mathcal{V}}(W,Y_N)$ and $\mathcal{R}_{\mathcal{V}}(W,\theta_1^*(W),\dots,\theta^*_K(W))$?}
Here we provide implementations for two typical and general machine learning tasks, regression and classification.

\subsection{Regression}
For the \emph{regression task}, the predictive family is typically modeled as:
\begin{small}
\begin{equation}
	\mathcal{V}_1 = \{g: g[x]=\mathcal{N}(f_{\theta}(x), \sigma^2), f\text{ is the predicting function and }\theta\text{ is learnable, }\sigma \text{ is a constant}\}.
\end{equation}	
\end{small}
The corresponding loss function is $\ell_{\mathcal{V}_1}(f_\theta(X),Y)=(f_\theta(X)-Y)^2$, and $U_{\mathcal{V}_1}(W,Y_N)$ becomes 
\begin{small}
\begin{equation}
\label{equ:regularizer-regression}
	U_{\mathcal{V}_1}(W,Y_N) = \text{Var}_{j\in [K]}(\overline{Y_N^j})= \sum_{j=1}^K\left(\sum_{i=1}^N w_{ij}y_i\right)^2\frac{1}{N\sum_{i=1}^Nw_{ij}}-\left(\frac{1}{N}\sum_{i=1}^Ny_i\right)^2
\end{equation}	
\end{small}
where $\overline{Y^j_N}$ denotes the mean value of the label $Y$ given $\mathcal{E}=e_j$ and $U(W,Y_N)$ calculates the variance of $\overline{Y^j_N}$ among sub-populations $e_1\sim e_K$.


\subsection{Classification}
For the \emph{classification task}, the predictive family is typically modeled as:
\begin{small}
\begin{equation}
	\mathcal{V}_2 = \{g: g[x]=f_\theta(x)\in\Delta_c, f\text{ is the classification model and }\theta\text{ is learnable}\},
\end{equation}	
\end{small}
where $c$ is the class number and $\Delta_c$ denotes the $c$-dimensional simplex.
Here each model in the predictive family $\mathcal V_2$ outputs a discrete distribution in the form of a $c$-dimensional simplex.
In this case, the corresponding loss function $\ell_{\mathcal V_2}(\cdot,\cdot)$ is the cross entropy loss and the regularizer becomes $U_{\mathcal V_2}(W,Y_N) = -\sum_{j=1}^K \frac{1}{N}(\sum_{i=1}^Nw_{ij}) H(Y_N^j)$, where $H(Y_N^j)$ is the entropy of $Y$ given $\mathcal{E}=e_j$.

% \xrz{Why choose $\mathcal{V}_1$ and $\mathcal{V}_2$ in this two cases?}

\subsection{Optimization.}
The bi-level optimization in Equation \ref{equ:bi-level} can be solved by performing projected gradient descent w.r.t. $W$.
The gradient of $W$ can be calculated by: (we omit the subscript $\mathcal{V}$ here)
\begin{small}
\begin{align}
	\nabla_W\mathcal R &=\nabla_W U + \left[\ell(f_{\theta_j}(x_i),y_i)\right]_{i,j}^{N\times K} + \sum_{j=1}^K \boxed{\nabla_{\theta_j}\mathcal{R}|_{\theta^*_j}\nabla_W\theta_j^*},\\
	\label{equ:tstep}
	\text{where }\boxed{\left.\nabla_{\theta_j}\mathcal{R}\right\vert_{\theta^*_j}\nabla_W\theta_j^*} & \approx \left.\nabla_{\theta_j}\mathcal{R}\right\vert_{\theta_j^t}\sum_{h\leq t}\left [ \prod_{k<h}(I - \left.\frac{\partial^2\mathcal{L}^j}{\partial\theta_j \partial \theta_j^{\mathrm{T}}}\right\vert_{\theta_j^{t-k-1}})\right ]\left.\frac{\partial^2\mathcal{L}^j}{\partial\theta_j\partial W^{\mathrm{T}}}\right\vert_{\theta_j^{t-h-1}} \\
	\label{equ:1step}
	&\approx  \left.\nabla_{\theta_j}\mathcal{R}\right\vert_{\theta_j^t}\left.\frac{\partial^2\mathcal{L}^j}{\partial \theta_j\partial W^{\mathrm{T}}}\right \vert_{\theta_j^{t-1}}\quad\quad\text{  , for }j=1,\dots,K.
\end{align}
\end{small}
where $\ell(f_{\theta_j}(x_i),y_i)]_{i,j}^{N\times K}$ is an $N\times K$ matrix in which the $(i,j)$-th element is $\ell(f_{\theta_j}(x_i),y_i)$.
Here Equation \ref{equ:tstep} approximates $\theta_j^*$ by $\theta_j^t$ from $t$ steps of inner loop gradient descent and Equation \ref{equ:1step} takes $t=1$ and performs \emph{1-step truncated backpropagation} \citep{shaban2019truncated,zhou2022model}.
Our information maximization algorithm updates $W$ by projected gradient descent as:
\begin{small}
\begin{equation}
	W \leftarrow \text{Proj}_{\mathcal{W}_K}\left(W-\eta\nabla_W\mathcal R\right),\quad \eta\text{ is the learning rate of }W.
\end{equation}	
\end{small}

Then we prove that minimizing Equation \ref{equ:bi-level} exactly finds the supremum w.r.t. $\mathcal{E}$ in the Definition \ref{def:empirical_predictive_heterogeneity} (formal) of the empirical predictive heterogeneity.
%And we also demonstrate its relationship with the expectation maximization (EM) algorithm.

\begin{theorem}[Justification of the IM Algorithm]
\label{theorem:IM}
	For the regression task with predictive family $\mathcal{V}_1$ and classification task with $\mathcal{V}_2$, the optimization of Equation \ref{equ:bi-level} is equivalent to the supremum problem of the empirical predictive heterogeneity $\hat{\mathcal{H}}^{\mathscr{E}_K}_{\mathcal V_1}(X\rightarrow Y; \mathcal D)$, $\hat{\mathcal{H}}^{\mathscr{E}_K}_{\mathcal V_2}(X\rightarrow Y; \mathcal D)$ respectively in Equation \ref{equ:appendix-formal-empirical}  with the pre-defined environment number $K$ (i.e. $|\text{supp}(\mathcal E)|=K$).
	Proofs can be found at Appendix \ref{proof: IM}.
\end{theorem}


\begin{remark}[Difference from Expectation Maximization]
	The expectation maximization (EM) algorithm is to infer latent variables of a statistic model to achieve the \textbf{maximum likelihood}.
	Our proposed information maximization (IM) algorithm is to infer the latent variables $W$ which brings the \textbf{maximal predictive heterogeneity} associated with the maximal information.
	Due to the regularizer $U_{\mathcal V}$ in our objective function, the EM algorithm cannot efficiently solve our problem, and therefore we adopt bi-level optimization techniques. 
\end{remark}

% \textbf{Approximation Accuracy.}\quad 
% To demonstrate the approximation accuracy of our IM algorithm, for the three toy cases in Section \ref{sec:linear}, we plot the estimated $\hat{\mathcal{H}}^u_{\mathcal V}$ of our algorithm as well as the precise values of $\mathcal{H}^u_{\mathcal V}$ in Figure \ref{fig:toy}.
% From the results, we can see that the approximated estimation of our IM algorithm stays closely to the precise values in all the three cases.


\subsection{Approximation Accuracy}
\begin{figure}[t]
    \centering
    \includegraphics[width=\textwidth]{./fig/toy.png}
    \caption{Numerical results of the toy examples in Section \ref{sec:linear}. The left sub-figure plots the estimated predictive heterogeneity under the setting of Theorem \ref{theorem: homogeneous}, the middle sub-figure plots the theoretical approximation, empirical approximation and our results under the setting of Theorem \ref{theorem: selection-bias}, and the right one is under the setting of Theorem \ref{theorem: omitted variable}.}
    \label{fig:appendix-toy}
\end{figure}

Here we provide some additional numerical results of our linear examples in Section \ref{sec:linear}.
In the left sub-figure of Figure \ref{fig:appendix-toy}, we plot the estimated predictive heterogeneity under the setting of Theorem \ref{theorem: homogeneous} where the analytical solution is equal to 0.
From the results, we can see that with the growing of sample size, the estimated value of our IM algorithm is approaching to 0 (note that the $y$-axis is $\ln(\text{estimated value})$).
In the middle sub-figure, for the setting in Theorem \ref{theorem: selection-bias}, we plot the theoretical approximation, empirical approximation (finite sample case) and the estimated value of the predictive heterogeneity under different ratios between the majority and the minority (which controls the $\text{Var}[r(\mathcal{E}^*)]$ in Equation \ref{equ:approximation1}).
And the right sub-figure plots the same values under the setting in Theorem \ref{theorem: omitted variable}.
From these two figures, we can see that (1) the empirical approximation under finite samples lies closely to the theoretical approximation, which is supported by our generalization bounds in Theorem \ref{theorem:pac}; (2) the estimated value of our IM algorithm is closely to the theoretical approximation,, which demonstrates the accuracy of our approximation algorithm in Equation \ref{equ:tstep} and \ref{equ:1step}. 
Also, as the ratio changes from $4:1$ to $1:1$, the data heterogeneity is increasing, and our predictive heterogeneity is also increasing, which is controlled by the term $\text{Var}(r(\mathcal{E}^*))$ in Equation \ref{equ:approximation1} and \ref{equ:approximation2}.







\section{Experiments}
\label{section:exp}

\begin{figure}[b]
  \centering
  \includegraphics[width=\textwidth,height=4.5cm]{./fig/climate.png}
  \vskip -0.1in
  \caption{Results on the crop yield data. We color each region according to its main crop type, and the shade represents the proportion of the main crop type after smoothing via $k$-means ($k=3$).}
  \label{fig:climate}
  \vskip -0.2in
\end{figure}

\begin{figure}[t]
\centering
\begin{minipage}{.48\textwidth}
  \centering
  \includegraphics[width=\linewidth]{./fig/adults.png}
  \caption{Results on the Adults data. Here we show the average of features and the feature coefficients of the two learned sub-populations.}
  \label{fig:test1}
\end{minipage}%
\hfill
\begin{minipage}{.48\textwidth}
  \centering
  \includegraphics[width=\linewidth]{./fig/new_birds.png}
  \caption{Results on the Waterbird data. Here we \emph{randomly sample} 50 images for each class and each learned sub-population.}
  \label{fig:test2}
\end{minipage}
\vskip -0.2in
\end{figure}



\subsection{Reveal Explainable Sub-population Structures}
The predictive heterogeneity could provide valuable insights for the sub-population division and support decision-making across various fields, including agricultural and sociological research, as well as object recognition.
Our illustrative examples below reveal that the learned sub-population divisions are highly explainable and relevant for decision-making purposes.

\textbf{Example: Agriculture}\quad  It is known that the climate affects crop yields and crop suitability \citep{lobell2008prioritizing}.
We utilize the data from the NOAA database which contains daily weather from weather stations around the world.
Following \cite{zhao2021comparing}, we extracted summary statistics from the weather sequence of the year 2018, including the average yearly temperature, humidity, wind speed and rainy days.
The task is to predict the \emph{crop yield} in each place with \emph{weather summary statistics} and \emph{location covariates (i.e. longitude and latitude)} of the place.
For easy illustration, we focus on the places with crop types of wheat or rice.
Notably, our input covariates do \emph{not} contain the crop type information. 
We use MLP models in this task and set $K=2$ for our IM algorithm.

Given that crop yield prediction mechanisms are closely related to crop type, which is unknown in the prediction task, we believe this causes data heterogeneity in the entire dataset, and the recognized predictive heterogeneity should relate to it. 
To demonstrate the rationality of our measure, we plot the real distribution map of wheat and rice planting areas in Figure \ref{fig:climate}(a) and the learned two sub-populations of our IM algorithm in Figure \ref{fig:climate}(b). 
The division given by our algorithm is quite similar to the real division of the two crops, indicating the rationality of our measure. 
We observe some discrepancies in areas such as the Tibet Plateau in Asia, which we attribute to the absence of significant features such as population density and altitude that significantly affect crop yields.


\textbf{Example: Sociology}\quad 
We use the UCI Adult dataset \citep{misc_adult_2}, which is widely used in the study of algorithmic fairness and derived from the 1994 Current Population Survey conducted by the US Census Bureau.
The task is to predict whether the income of a person is greater or less than 50k US dollars based on personal features.
We use linear models in this task and set $K=2$.
In this example, we aim to investigate whether \emph{sub-population structures} within data affect the learning of machine learning models.

In Figure \ref{fig:test1} (a), we plot summary statistics for the two sub-populations, revealing a key difference in capital gain.
In Figure \ref{fig:test1} (b), we present the feature importance given by linear models for the two sub-populations, and find that for individuals with high capital gain, the prediction model mainly relies on capital gain, which is fair.
However, for individuals with low capital gain, models also consider sensitive attributes such as sex and marital status, which have been known to cause discrimination.
Our results are consistent with those found in \citep{zhao2021comparing} and can help identify potential inequalities in decision-making.
For example, our findings suggest potential discrimination towards individuals with low capital gain, which could motivate algorithmic design and improve policy fairness.

\textbf{Example: Object Recognition}\quad Finally, we utilize the Waterbird dataset \citep{sagawa2019distributionally}, which is widely used as a benchmark in the field of robust learning, to investigate the impact of spurious correlations on machine learning models.
The task is to recognize waterbirds or landbirds, but the images contain \emph{spurious correlations} between the background and the target label. 
For the majority of images, waterbirds are located on water and landbirds on land, whereas for a minority of images, this correlation is reversed. 
Therefore, the spurious correlation leads to predictive heterogeneity in this dataset, which could significantly affect the performance of machine learning models.
In this example, we use the ResNet18 and set $K=2$ in our IM algorithm.

Our method successfully captures the spurious correlation and identifies two sub-populations of images with inverse correlations between the object and the background.
To demonstrate the effectiveness of our method, we randomly sample 50 images for each class and each learned sub-population and plot them in Figure \ref{fig:test2}. 
In sub-population 1, the majority of landbirds are on the ground and waterbirds are in the water, while in sub-population 2, the majority of landbirds are in the water and waterbirds are on the ground.
Our findings suggest that the proposed approach can be leveraged by robust learning methods \citep{sagawa2019distributionally, koyama2020out} to improve the generalization ability of machine learning models. 
By eliminating the influence of spurious correlations, our method could significantly enhance the robustness and reliability of machine learning models. 
Overall, our study highlights the importance of addressing predictive heterogeneity in image classification tasks and provides a practical solution for achieving robust learning performance.


\begin{figure}[htbp]
\centering
\begin{minipage}{.48\textwidth}
  \centering
  \includegraphics[width=\linewidth]{./fig/age.png}
  \caption{Results on the COVID-19 data. We plot the age distributions of dead people ($Y=1$) in each learned subgroup.}
  \label{fig:COVID}
\end{minipage}%
\hfill
\begin{minipage}{.48\textwidth}
  \centering
  \includegraphics[width=\linewidth]{./fig/mean_features.png}
  \caption{Results on the COVID-19 data. We show the averages of typical features of dead people ($Y=1$) in each learned subgroup.}
  \label{fig:COVID2}
\end{minipage}
\vskip -0.2in
\end{figure}



\subsection{Assist Scientific Discovery: Uncover Factors Related to Mortality}
In order to fully demonstrate the efficacy of our predictive heterogeneity, we focus on the application of healthcare, utilizing the COVID-19 dataset of Brazilian patients. 
This dataset comprises 6882 COVID-positive patients from Brazil, whose data was recorded between February 27th and May 4th, 2020. 
The dataset includes a wide range of risk factors, including comorbidities, symptoms, and demographic characteristics. 
The binary label corresponds to mortality caused by COVID-19. 
Our aim is to validate the sub-populations learned through our methodology on this dataset, by thoroughly \emph{explaining each group} and showcasing how our predictive heterogeneity can be employed to \emph{uncover features related to mortality that are otherwise difficult to detect}.

\subsubsection{Learned Sub-populations.}
When predicting mortality based on risk factors, it is important to consider that patients with various underlying diseases and demographic characteristics, such as age and sex, may exhibit different probabilities of mortality. 
Furthermore, it is plausible that the mortality of different individuals can be attributed to distinct factors. 
In light of these considerations, the predictive heterogeneity for this dataset is caused by the diversity of mechanisms that contribute to mortality among various sub-populations.

In this experiment, we use linear models and the loss function is binary cross-entropy loss. 
We select the sub-population number $K\in \{2,3,4,5,6\}$ that exhibit the maximal empirical predictive heterogeneity$\hat{\mathcal{H}}_{\mathcal V}^{\mathscr E_K}(X\rightarrow Y;\mathcal D)$, which results in three distinct subgroups (the optimal $K=3$).
Besides, we empirically observe that when $K>3$, the learned sub-populations will shrink to 3 sub-populations.
In Figure \ref{fig:COVID} and \ref{fig:COVID2}, to conduct a more thorough examination of the learned subgroups, we analyze the age distribution of each group, as well as the average value of their corresponding risk factors. 
Our analysis reveals several noteworthy findings:
\begin{itemize}
	\item[1.] We observe a distinct difference in the age distribution of the learned subgroups. Specifically, Group 0 is primarily composed of individuals over the age of 70, while Group 1 consists of individuals around 60 years old. 
Group 2, on the other hand, is comprised of middle-aged individuals spanning multiple age groups.
	\item[2.]  The average values of the risk factors reveal notable differences among the various subgroups, indicative of distinct causes of mortality. More specifically, Group 0 exhibits a considerably higher prevalence of underlying diseases, such as renal, neurologic, liver, and immunosuppression, when compared to the other groups. In contrast, Group 1 shows a substantially lower level of underlying diseases in comparison. Interestingly, Group 2 does not exhibit any underlying diseases, yet has a markedly higher level of diarrhea and vomiting. These findings suggest that the learned subgroups may be used to identify specific risk factors associated with mortality, which can inform targeted interventions for individuals with distinct risk profiles.
\end{itemize}
Having identified distinct patterns among the subgroups, we seek to identify the specific risk factors associated with mortality. 
To further validate our findings, we incorporate the expertise of domain experts. 
By leveraging their insights, we are able to confirm the reliability of the identified risk factors and the importance of our subgroup analysis.


\subsubsection{Scientific Findings}
Based on the learned group, we fit a logistic regression model on each group and pick the top-6 features with the largest coefficients, which are shown in Table \ref{tab:top-features}.

Firstly, our analysis reveals that in Group 0 and 1, the top features associated with mortality are primarily SPO2 and underlying diseases, which align with the common risk factors of older individuals. 
In contrast, Group 2 exhibits a distinct set of top features, including symptoms of COVID-19 such as fever, cough, and vomiting. 
Notably, Group 2 is composed of middle-aged individuals spanning multiple age groups. 
Our findings suggest that severe COVID-19 symptoms can lead to mortality regardless of age.
%, further emphasizing the need for targeted interventions that account for individual risk factors.

Secondly, to further our analysis, we fit a model for the entire dataset and observe that the top features remain SPO2 and underlying diseases, consistent with the top features found for older individuals. 
However, this may not be beneficial or could even lead to harm for interventions targeted towards younger or middle-aged individuals who generally do not have severe underlying diseases. 
For instance, doctors may tend to treat younger patients with severe COVID-19 symptoms optimistically and underestimate their mortality risk because they typically do not have underlying diseases.
Thus, exploring and leveraging the predictive heterogeneity within the data can lead to more reliable scientific discoveries while avoiding potential harm caused by latent heterogeneity.

%Thirdly, from our analysis, we find two features on Group 2, i.e. vomiting and diarrhoea, which rarely appear in traditional analysis.
%We investigate the related literatures on COVID-19 and find that in various studies, these two features are recognized as important indicators of higher risk of mortality caused by COVID-19.
%\citet{2020COVID} highlighted the manifestations and potential mechanisms of gastrointestinal and hepatic injuries in COVID-19 to raise awareness of digestive system injury in COVID-19.
%\citet{Liu442} analyzed 29,393 laboratory-confirmed COVID-19 patients diagnosed before 21 March 2020 in cities outside of Wuhan in mainland China and found that  patients with fever and no GI symptoms and patients with both GI symptoms and fever all had significantly higher risk of death, where GI symptoms refer to one of the following symptoms (nausea, vomiting, diarrhoea or abdominal pain). 
%\citet{2021Gastrointestinal} also found that gastrointestinal symptoms are associated with severity of COVID-19, and the severe rate was more than 40\% in COVID-19 patients with gastrointestinal symptoms.
%\citet{0Diarrhea} demonstrated that the presence of diarrhea as a presenting symptom is associated with increased disease severity and likely worse prognosis.
%And \citet{2022COVID} called that COVID-19 should be considered in the differential diagnosis for patients who present with abdominal pain and gastrointestinal symptoms typical of gastroenteritis or surgical abdomen, even if they lack respiratory symptoms of COVID-19. 
%These studies validate the reliability of our findings, which also shows that our predictive heterogeneity could help to discover unusual risk factors that do not appear in the analysis on the overall dataset.


Thirdly, our analysis reveals two important features in Group 2, namely vomiting and diarrhea, which are rarely considered in traditional analysis. 
We have reviewed relevant literature on COVID-19 and discovered that various studies have recognized these two symptoms as important indicators of higher risk of mortality caused by COVID-19. 
\citet{2020COVID} highlighted the potential mechanisms of gastrointestinal and hepatic injuries in COVID-19 to raise awareness of digestive system injury in COVID-19. 
\citet{Liu442} analyzed 29,393 laboratory-confirmed COVID-19 patients diagnosed before March 21, 2020, in cities outside of Wuhan in mainland China and found that patients with both GI symptoms and fever and patients with fever alone had a significantly higher risk of death, where GI symptoms refer to one of the following symptoms: nausea, vomiting, diarrhea, or abdominal pain. 
\citet{2021Gastrointestinal} also found that gastrointestinal symptoms are associated with the severity of COVID-19, and the severe rate was more than 40\% in COVID-19 patients with gastrointestinal symptoms. 
\citet{0Diarrhea} demonstrated that the presence of diarrhea as a presenting symptom is associated with increased disease severity and likely worse prognosis. 
\citet{2022COVID} have called for the consideration of COVID-19 in the differential diagnosis for patients who present with abdominal pain and gastrointestinal symptoms typical of gastroenteritis or surgical abdomen, even if they lack respiratory symptoms of COVID-19. 
These studies validate the reliability of our findings and demonstrate that studies utilizing the proposed predictive heterogeneity can uncover unusual risk factors that do not appear in analysis of the overall dataset.

This example serves as an illustration of the potential benefits that our predictive heterogeneity can offer to a wide range of scientific fields.
By exploiting the heterogeneity within a dataset, our approach can reveal novel patterns and relationships that may be overlooked in traditional analyses, leading to more reliable and comprehensive scientific discoveries



\begin{table}[htbp]
\centering
\caption{Top features of each learned subgroup and overall data on the COVID-19 dataset.}
\label{tab:top-features}
\resizebox{\textwidth}{1.2cm}{
\begin{tabular}{c|llllll}
\hline
Group ID & \multicolumn{6}{c}{Top Features} \\ \hline
0        &    SPO2 &   Diabetes  &   Renal  & Neurologic    &  Pulmonary   &  Cardiovascular  \\ \hline
1        &    Diabetes &  SPO2   &  Neurologic   & Cardiovascular    & Pulmonary     &  Renal  \\ \hline
2        &    \bf Fever & \bf Cough   & Renal    & \bf Vomitting    & \bf Shortness of breath    &  \bf Dihareea  \\ \hline
All      &    SPO2 &    Renal &   Neurologic  &   Diabetes  & Pulmonary    &  Cardiovascular  \\ \hline
\end{tabular}}
\end{table}















\subsection{Benefit Generalization}
\label{section:ood}

\begin{table}[b]
\vskip -0.1in
\caption{Results of the experiments on out-of-distribution generalization, including the simulated data and colored MNIST data.}
\label{table:results}
%\vskip 0.005in
\centering\resizebox{\textwidth}{1.9cm}{
\begin{tabular}{|cc|cccc|||cc|}
\hline
\multicolumn{2}{|c|}{\multirow{3}{*}{\large Method}} & \multicolumn{4}{c|||}{\textbf{\large 1. Simulated Data}} & \multicolumn{2}{c|}{\textbf{\large 2. Colored MNIST}}\\ 
\multicolumn{2}{|c|}{}     &  \multicolumn{2}{c}{\bf Training Sub-population Error} & \multicolumn{2}{c|||}{\bf Test Error} & \multirow{2}{*}{\bf Train Accuracy} & \multicolumn{1}{c|}{\multirow{2}{*}{\bf Test Accuracy}}                                                                                                                  \\  
\multicolumn{2}{|l|}{}                                                                  & \multicolumn{1}{c}{Major ($r=1.9$)}            & \multicolumn{1}{c}{Minor ($r=-1.9$)}           & \multicolumn{1}{c}{$r=-2.3$}           & $r=-2.7$    &  &        \\ \hline
\multicolumn{2}{|c|}{ERM}                                 & \multicolumn{1}{c}{0.255{(\scriptsize$\pm 0.024$)}} & \multicolumn{1}{c|}{0.740{\scriptsize($\pm 0.022$)}} & \multicolumn{1}{c}{0.738{\scriptsize($\pm 0.035$)}} & 0.737{\scriptsize($\pm 0.023$)} & 0.998{\scriptsize($\pm 0.001$)} & 0.406{\scriptsize($\pm 0.019$)}  \\ 
\multicolumn{2}{|c|}{EIIL}                                   & \multicolumn{1}{c}{\bf 0.164{\scriptsize($\pm 0.014$)}} & \multicolumn{1}{c|}{1.428{\scriptsize($\pm 0.035$)}} & \multicolumn{1}{c}{1.431{\scriptsize($\pm 0.061$)}} & 1.431{\scriptsize($\pm 0.046$)} & 0.812{\scriptsize($\pm 0.006$)} & 0.610{\scriptsize($\pm 0.016$)}\\ \cline{1-2}
\multicolumn{1}{|c}{\multirow{3}{*}{\large KMeans}} & Balance   & \multicolumn{1}{c}{0.231{\scriptsize($\pm 0.022$)}} & \multicolumn{1}{c|}{0.847{\scriptsize($\pm 0.024$)}} & \multicolumn{1}{c}{0.846{\scriptsize($\pm 0.039$)}} & 0.845{\scriptsize($\pm 0.026$)}& \bf 0.999{\scriptsize($\pm 0.001$)} & 0.328{\scriptsize($\pm 0.021$)} \\ 
\multicolumn{1}{|c}{}                        & IRM        & \multicolumn{1}{c}{0.231{\scriptsize($\pm 0.022$)}} & \multicolumn{1}{c|}{0.845{\scriptsize($\pm 0.024$)}} & \multicolumn{1}{c}{0.844{\scriptsize($\pm 0.039$)}} & 0.843{\scriptsize($\pm 0.026$)} & 0.947{\scriptsize($\pm 0.004$)} & 0.259{\scriptsize($\pm 0.021$)}\\
\multicolumn{1}{|c}{}                        & IGA         & \multicolumn{1}{c}{0.235{\scriptsize($\pm 0.022$)}} & \multicolumn{1}{c|}{0.840{\scriptsize($\pm 0.023$)}} & \multicolumn{1}{c}{0.839{\scriptsize($\pm 0.038$)}} & 0.838{\scriptsize($\pm 0.027$)} & 0.997{\scriptsize($\pm 0.001$)} & 0.302{\scriptsize($\pm 0.021$)}\\ \cline{1-2}
\multicolumn{1}{|c}{\multirow{3}{*}{\large Ours}}   & Balance   & \multicolumn{1}{c}{0.403{\scriptsize($\pm 0.041$)}} & \multicolumn{1}{c|}{\bf 0.423{\scriptsize($\pm 0.016$)}} & \multicolumn{1}{c}{\bf 0.416{\scriptsize($\pm 0.022$)}} & \bf 0.416{\scriptsize($\pm 0.014$)} & 0.749{\scriptsize($\pm 0.012$)} & \bf 0.692{\scriptsize($\pm 0.039$)} \\ 
\multicolumn{1}{|c}{}                        & IRM        & \multicolumn{1}{c}{0.391{\scriptsize($\pm 0.039$)}} & \multicolumn{1}{c|}{\bf 0.432{\scriptsize($\pm 0.016$)}} & \multicolumn{1}{c}{\bf 0.430{\scriptsize($\pm 0.022$)}} &\bf 0.430{\scriptsize($\pm 0.014$)}  & 0.759{\scriptsize($\pm 0.014$)} & \bf 0.727{\scriptsize($\pm 0.047$)}\\
\multicolumn{1}{|c}{}                        & IGA        & \multicolumn{1}{c}{0.449{\scriptsize($\pm 0.037$)}} & \multicolumn{1}{c|}{\bf 0.426{\scriptsize($\pm 0.017$)}} & \multicolumn{1}{c}{\bf 0.417{\scriptsize($\pm 0.022$)}} &\bf 0.417{\scriptsize($\pm 0.014$)}  & 0.759{\scriptsize($\pm 0.012$)} & \bf  0.713{\scriptsize($\pm 0.034$)}\\ \hline
\end{tabular}
}
\end{table}


%\begin{figure}
%\centering
%\begin{minipage}{.69\textwidth}
%  \centering
%\subfigure[KMeans.] {
% \label{fig:a}     
%\includegraphics[width=0.3\linewidth]{./fig/kmeans.png}  
%}      
%\subfigure[EIIL.] {
% \label{fig:a}     
%\includegraphics[width=0.3\linewidth]{./fig/eiil.png}  
%}   
%\subfigure[Our IM.] {
% \label{fig:a}     
%	\includegraphics[width=0.3\linewidth]{./fig/ours.png}  
%	}
%	\vskip -0.1in
%  \caption{Sub-population division on the simulated data of three methods, where two colors denote two sub-populations.}
%  \label{fig:test3}
%\end{minipage}
%\hfill
%\begin{minipage}{.3\textwidth}
%  \centering
%  \includegraphics[width=\linewidth]{./fig/mnist.png}
%  \caption{Sub-population division on the MNIST data of our IM algorithm.}
%  \label{fig:test4}
%\end{minipage}
%\vskip -0.25in
%\end{figure}



\begin{figure}
\centering
\begin{minipage}{1.0\textwidth}
  \centering
\subfigure[KMeans.] {
 \label{fig:a}     
\includegraphics[width=0.3\linewidth]{./fig/kmeans.png}  
}      
\subfigure[EIIL.] {
 \label{fig:a}     
\includegraphics[width=0.3\linewidth]{./fig/eiil.png}  
}   
\subfigure[Our IM.] {
 \label{fig:a}     
	\includegraphics[width=0.3\linewidth]{./fig/ours.png}  
	}
	\vskip -0.1in
  \caption{Sub-population division on the simulated data of three methods, where two colors denote two sub-populations.}
  \label{fig:test3}
\end{minipage}
\end{figure}


 In this section, we aim to evaluate the efficacy of our IM algorithm in enhancing the out-of-distribution (OOD) generalization performance of machine learning models. 
 To this end, we conduct experiments on both simulated data and real-world colored MNIST data. 
 Our results suggest that the learned sub-population structures by our IM algorithm could significantly benefit the OOD generalization of machine learning models.

\textbf{Baselines}\quad First, we compare with \emph{empirical risk minimization} (ERM) and \emph{environment inference for invariant learning} (EIIL, \citep{creager2021environment}) which infers the environments for learning invariance.
Then we compare with the well-known \emph{KMeans} algorithm, which is the most popular clustering algorithm.
For our IM algorithm and KMeans, we involve three algorithms as backbones to leverage the learned sub-populations, including sub-population balancing and invariant learning methods.
The sub-population balancing simply equally weighs the learned sub-populations.
\emph{invariant risk minimization} (IRM, \citep{arjovsky2019invariant}) and \emph{inter-environment gradient alignment} (IGA, \citep{koyama2020out}) are typical methods in OOD generalization, which take the sub-populations as input environments to learn the invariant models.

\subsubsection{Simulation Data of Sample Selection Bias}
The input features $X=[S,T,V]^T\in\mathbb{R}^{10}$ consist of stable features $S\in\mathbb{R}^5$, noisy features $T\in\mathbb{R}^4$ and the spurious feature $V\in\mathbb{R}$:
\begin{small}
\begin{equation}
	S\sim\mathcal{N}(0,2\textbf{I}_5), T\sim\mathcal{N}(0,2\textbf{I}_4), Y=\theta_S^TS + h(S)+\mathcal{N}(0,0.5), V\sim\text{Laplace}(\text{sign}(r)\cdot Y, 1/(5\ln |r|))
\end{equation}	
\end{small}
where $\theta_S\in\mathbb{R}^5$ is the coefficient and $h(S)=S_1S_2S_3$ is the nonlinear term.
$|r|>1$ is a factor for each sub-population, and here the data heterogeneity is brought by the \emph{endogeneity with hidden variable} \citep{fan2014challenges}.
$V$ is the \emph{spurious feature} whose relationship with $Y$ is unstable across sub-populations and is controlled by the factor $r$.
Intuitively, $\text{sign}(r)$ controls whether the spurious correlation between $V$ and $Y$ is positive or negative. 
And $|r|$ controls the strength of the spurious correlation, i.e. the larger $|r|$ means the stronger spurious correlation.
In \emph{training}, we generate 10000 points, where the major group contains 80\% data with $r=1.9$ (i.e. strong \emph{positive} spurious correlation) and the minor group contains 20\% data with $r=-1.9$ (i.e. strong \emph{negative} spurious correlation).
In \emph{testing}, we test the performances of the two groups respectively, and we also set $r=-2.3$ and $r=-2.7$ to simulate stronger distributional shifts.
We use linear regression and set $K=2$ for all methods, and we report the mean-square errors (MSE) of all methods.

The results over 10 runs are shown in Table \ref{table:results}.
From the results in Table \ref{table:results}, for both the simulated and colored MNIST data, the two backbones with our IM algorithm achieve \emph{the best OOD generalization performances}.
Also, for the simulated data, the learned predictive heterogeneity enables backbone algorithms to equally treat the majority and minority inside data (i.e. low-performance gap between 'Major' and 'Minor'), and significantly benefits the OOD generalization.
Further, we plot the learned sub-populations of our IM algorithm in Figure \ref{fig:test3}.
From Figure \ref{fig:test3}, compared with KMeans and EIIL, our predictive heterogeneity exploits the spurious correlation between $V$ and $Y$, and enables the backbone algorithms to eliminate it.


%\textbf{Sensitivity of $K$}\quad We add more results of choosing different $K$s for this simulated experiment to show that the \emph{OOD generalization performances} of some typical algorithms plus our proposed method are not sensitive to the choices of $K$.
%\begin{figure}[h]
%    \centering
%    \includegraphics[width=0.7\textwidth]{./fig/appendix.png}
%    \caption{The out-of-distribution generalization error of our methods with Sub-population Balancing, IRM and IGA as backbones. Here we plot the errors of different backbones under $r=-2.7$, which introduces strong distributional shifts with training data.}
%    \label{fig:appendix-k}
%\end{figure}
%
%In Figure \ref{fig:appendix-k}, we show the out-of-distribution generalization error of our methods with Sub-population Balancing, IRM and IGA as backbones.
%We plot the OOD testing performances under $r=-2.7$, which has strong distributional shift with the training distribution.
%From the results, we can see that the performances of three OOD generalization methods \emph{do not be affected much} by the choice of $K$, and from Table \ref{table:results} , ours significantly outperforms all the baselines.

\subsubsection{Simulation Data of Hidden Variables}
\quad\textcolor{black}{Also, we add one more experiment to show that (1) when the chosen $K$ is smaller than the ground-truth, the performances of our methods will drop but are still better than ERM (2) when the chosen $K$ is larger, the performances are not affected much.}

\textcolor{black}{The input features $X=[S,T,V]\in\mathbb{R}^{10}$ consist of stable features $S\in\mathbb{R}^5$, noisy features $T\in\mathbb{R}^4$ and the spurious feature $V\in\mathbb R$:
$$
S\sim \mathcal{N}(2,2\mathbb I_5),\quad  T\sim \mathcal{N}(0, 2\mathbb I_4), \quad Y=\theta_S^TS + S_1S_2S_3+\mathcal{N}(0,0.5),
$$
and we generate the spurious feature via:
$$
V = \theta_V^e Y + \mathcal{N}(0, 0.3),
$$
where $\theta_V^e$ varies across sub-populations and is dependent on which sub-population the data point belongs to.
In training, we sample 8000 data points from $e_1$ with $\theta_V^1=3.0$, 1000 points from $e_2$ with $\theta_V^2=-1.0$, 1000 points from $e_3$ with $\theta_V^3=-2.0$ and 1000 points from $e_4$ with $\theta_V^4=-3.0$.
Therefore, the ground-truth number of sub-populations is 4.
In testing, we test the performances on $e_4$ with $\theta_V^4=-3.0$, which has strong distributional shifts from training data.
The average MSE over 10 runs are shown in Figure \ref{fig:appendix-rebuttal}.}


%\begin{figure}[h]
%    \centering
%    \includegraphics[width=0.6\textwidth]{./fig/appendix_rebuttal.png}
%    \caption{\textcolor{black}{The out-of-distribution generalization error of our methods with Sub-population Balancing, IRM and IGA as backbones for the added experiments. The ground-truth sub-population number is 4.}}
%    \label{fig:appendix-rebuttal}
%\end{figure}

\textcolor{black}{From the results, we can see that when $K$ is smaller than the ground-truth, increasing $K$ benefits the OOD generalization performance, and when $K$ is larger, the performances are not affected much.}

\textcolor{black}{For our IM algorithm, we think there are mainly two ways to choose $K$:}

\begin{itemize}
    \item \textcolor{black}{According to the predictive heterogeneity index: When the chosen $K$ is smaller than the ground-truth, our measure tends to increase quickly when increasing $K$; and when $K$ is larger than the ground-truth, the increasing speed will slow down, which could direct people to choose an appropriate $K$.}
    \item \textcolor{black}{According to the prediction model: Since our IM algorithm aims to learn sub-populations with different prediction mechanisms, one could compare the learned model parameters $\theta_1, \dots, \theta_K$ to judge whether $K$ is much larger than the ground-truth, i.e., if two resultant models are quite similar, $K$ may be too large (divide one sub-population into two). For linear models, one can directly compare the coefficients. For deep models, we think one can calculate the transfer losses across sub-populations.}
\end{itemize}


%\begin{figure}[b]
%\centering
%%\begin{minipage}{.5\textwidth}
%  \centering
%  \includegraphics[width=0.6\linewidth]{./fig/mnist.png}
%  \caption{Sub-population division on the MNIST data of our IM algorithm.}
%  \label{fig:test4}
%%\end{minipage}
%\vskip -0.25in
%\end{figure}


\begin{figure}[htbp]
\centering
\begin{minipage}{.55\textwidth}
  \centering
  \includegraphics[width=\linewidth]{./fig/appendix_rebuttal.png}
  \caption{The OOD generalization error of our methods with Sub-population Balancing, IRM and IGA as backbones for the added experiments. The ground-truth sub-population number is 4.}
  \label{fig:appendix-rebuttal}
\end{minipage}%
\hfill
\begin{minipage}{.42\textwidth}
  \centering
  \includegraphics[width=\linewidth]{./fig/mnist.png}
  \caption{Sub-population division on the MNIST data of our IM algorithm.}
  \label{fig:test4}
\end{minipage}
\vskip -0.2in
\end{figure}


\subsubsection{Colored MNIST}
Following \cite{arjovsky2019invariant}, we design a binary classification task constructed on the MNIST dataset.
Firstly, digits $0\sim4$ are labeled $Y=0$ and digits $5\sim 9$ are labeled $Y=1$. 
Secondly, noisy labels $\tilde{Y}$ are induced by randomly flipping the label $Y$ with a probability of 0.2.
Then we sample the colored id $V$ spurious correlated with $\tilde{Y}$ as 
\begin{equation}
	V=\Big\{\begin{array}{ll}
     +\tilde{Y}, &\text{with probability }r,  \\
     -\tilde{Y}, &\text{with probability }1-r.
\end{array}
\end{equation}

In fact, $r$ controls the spurious correlation between $\tilde{Y}$ and $V$. 
In \emph{training}, we randomly sample 10000 data points and set $r=0.85$, meaning that for 85\% of the data, $V$ is positively correlated with $\tilde{Y}$ and for the rest 15\%, the spurious correlation becomes negative, which causes data heterogeneity w.r.t. $V$ and $\tilde{Y}$.
In \emph{testing}, we set $r=0$ (\emph{strong negative spurious correlation}), bringing strong shifts between training and testing.

From the results in Table \ref{table:results}, for both the simulated and colored MNIST data, the two backbones with our IM algorithm achieve \emph{the best OOD generalization performances}.
We plot the learned sub-populations of our IM algorithm in Figure \ref{fig:test4}.
From Figure \ref{fig:test4}, the learned sub-populations of our method also reflect the different directions of the spurious correlation between digit labels $Y$ and colors (red or green), which helps backbone methods to avoid using colors to predict digits.























\section{Related Work}
To the best of our knowledge, data heterogeneity has not converged to a uniform formulation so far, and has different meanings among different fields.
\cite{li1995definition} define the heterogeneity in \emph{ecology} based on the system property and complexity or variability.
\cite{rosenbaum2005heterogeneity} views the uncertainty of the potential outcome as unit heterogeneity in observational studies in \emph{economics}.
For \emph{graph} data, the heterogeneity refers to various types of nodes and edges (\cite{wang2019heterogeneous}).
More recently, in machine learning, several works of \emph{causal learning} \citep{peters2016causal, arjovsky2019invariant, koyama2020out, creager2021environment} and \emph{robust learning} \citep{sagawa2019distributionally} leverage heterogeneous data from multiple environments to improve the out-of-distribution generalization ability.
Specifically, invariant learning methods \citep{arjovsky2019invariant, koyama2020out, creager2021environment, zhou2022model} leverage the heterogeneous environment to learn the invariant predictors that have uniform performances across environments.
And in distributionally robust optimization field, \cite{sagawa2019distributionally, duchi2022distributionally} propose to optimize the worst-group prediction error to guarantee the OOD generalization performance.
However, in machine learning, previous works have not provided a precise definition or sound quantification of data heterogeneity, which makes it confusing and hard to leverage to develop more rational machine learning algorithms.

As for clustering algorithms, most algorithms only focus on the covariates $X$, typified by KMeans and Gaussian Mixture Model (GMM, \citep{reynolds2009gaussian}).
However, the learned clusters by KMeans can only reflect heterogeneous structures in $P(X)$, which is shown by our experiments.
Notably that our predictive heterogeneity could reflect the heterogeneity in $P(Y|X)$.
And the expectation maximization (EM, \citep{moon1996expectation}) can also be used for clustering.
However, our IM algorithm has essential differences from EM, for our IM algorithm infers latent variables that maximizes the predictive heterogeneity but EM maximizes the likelihood.
Also, there are methods \citep{creager2021environment} from the invariant learning field to infer environments.
Though it could benefit the OOD generalization, it lacks the theoretical foundation and only works in some settings.

\section{Discussion on differences with sub-group discovery}
\textcolor{black}{Subgroup discovery (SD, \citep{helal2016subgroup}) is aimed at extracting "interesting" relations among different variables ($X$) with respect to a target variable $Y$. Coverage and precision of each discovered group is the focus of such method. To be specific, it learns a partition on $P(X)$ such that some target label $y$ dominates within each group. The most siginficant gap between subgroup discovery and our predictive heterogeneity lies in the pattern of distributional shift among clusters: for subgroup discovery, $P(X)$ and $P(Y)$ varies across subgroups but there is a universal $P(Y|X)$. While for predictive heterogeneity $P(Y|X)$ differs across sub-population, which indicates diversified prediction mechanism. It is such disparity of prediction mechanism that inhibits   the performance of a universal predictive model on a heterogeneous dataset, which is the emphasis of OOD problem and group fairness.} 

\textcolor{black}{We think sub-group discovery is more applicable for settings where the distributional shift is minor while high explainability is required, since it generates simplified rules that people can understand. Also, sub-group discovery methods is suitable for the settings that only involve tabular data (typlically from a relational database), where the input features have clear semantics. 
And our proposed method could deal with general machine learning settings, including complicated data (e.g., image data) that involves representation learning.
Also, when people have to handle settings where data heterogeneity w.r.t. prediciton mechanism exists inside data, our method is more applicable.
However, both kinds of methods can be used to help people understand data and make more reasonable decisions.}

\section{Discussion on the Potential for fairness}
\textcolor{black}{We find combining our measure with algorithmic fairness is an interesting and promising direction and we think our measure has the potential to deal with algorithmic bias. 
Our method could generate sub-populations with possibly different prediction mechanisms, which could do some help in the following aspects:}

\textcolor{black}{\textbf{Risk feature selection}: we could select features according to our predictive heterogeneity measure to see what features bring the largest heterogeneity. If they are sensitive features, people should avoid their effects, and if they are not, they could direct people to build better machine learning models.}

\textcolor{black}{\textbf{Examine the algorithmic fairness}: we could use the learned sub-populations to examine whether a given algorithm is fair by calculating the performance gap across the sub-populations.}
\section{Conclusion}
We define the predictive heterogeneity, as the first quantitative formulation of the data heterogeneity that affects the prediction of machine learning models.
We demonstrate its theoretical properties and show that it benefits the out-of-distribution generalization performances.



% Acknowledgements and Disclosure of Funding should go at the end, before appendices and references

%\acks{All acknowledgements go at the end of the paper before appendices and references.
%Moreover, you are required to declare funding (financial activities supporting the
%submitted work) and competing interests (related financial activities outside the submitted work).
%More information about this disclosure can be found on the JMLR website.}


% Manual newpage inserted to improve layout of sample file - not
% needed in general before appendices/bibliography.

\newpage


%\bibliography{sample}
%\bibliographystyle{sample}
\appendix
\section{Proof of Proposition \ref{proposition1}}
\label{proof: prop1}
\begin{proof}[Proof of Proposition \ref{proposition1}]
\\\\
1. \emph{Monotonicity}:

Because of $\mathscr E_1 \subseteq \mathscr E_2$,
\begin{small}
\begin{align}
   \mathcal{H}^{\mathscr E_1}_{\mathcal V}(X \rightarrow Y) &= \sup_{\mathcal{E} \in \mathscr E_1}\mathbb{I}_{\mathcal{V}}(X\rightarrow Y|\mathcal{E})-\mathbb{I}_{\mathcal{V}}(X\rightarrow Y) \\
    &\leq \sup_{\mathcal{E} \in \mathscr E_2}\mathbb{I}_{\mathcal{V}}(X\rightarrow Y|\mathcal{E})-\mathbb{I}_{\mathcal{V}}(X\rightarrow Y) \\
    &= \mathcal{H}^{\mathscr E_2}_{\mathcal V}(X \rightarrow Y).
\end{align}
\end{small}
2. \emph{Nonnegativity}:

According to the definition of the environment set, there exists $\mathcal E_0 \in \mathscr E$ such that for any $e \in \text{supp}(\mathcal E)$, $X,Y|\mathcal E=e$ is identically distributed as $X,Y$. Thus, we have
\begin{small}
\begin{align}
    \mathcal{H}^{\mathscr E}_{\mathcal V}(X \rightarrow Y) &=
    \sup_{\mathcal{E} \in \mathscr E} \left[H_\mathcal V(Y|\emptyset,\mathcal E) - H_\mathcal V(Y|X,\mathcal E)\right] - \left[H_\mathcal V(Y|\emptyset) - H_\mathcal V(Y|X)\right] \\
    &\geq \left[H_\mathcal V(Y|\emptyset,\mathcal E_0) - H_\mathcal V(Y|X,\mathcal E_0)\right] - \left[H_\mathcal V(Y|\emptyset) - H_\mathcal V(Y|X)\right].
\end{align}
\end{small}
Specifically, 
\begin{small}
\begin{align}
    H_\mathcal V(Y|X,\mathcal E_0) &= \mathbb E_{e \sim \mathcal E_0} \left[ \inf\limits_{f\in\mathcal{V}}\mathbb{E}_{x,y\sim X,Y | \mathcal E=e}[-\log f[x](y)] \right] \\
    &= \mathbb E_{e \sim \mathcal E_0} \left[ \inf\limits_{f\in\mathcal{V}}\mathbb{E}_{x,y\sim X,Y}[-\log f[x](y)] \right] \\
    &= H_\mathcal V(Y|X).
\end{align}
\end{small}
Similarly, $H_\mathcal V(Y|\emptyset,\mathcal E_0) = H_\mathcal V(Y|\emptyset)$.
Thus, $\mathcal{H}^{\mathscr E}_{\mathcal V}(X \rightarrow Y) \geq 0$.\\\\
3. \emph{Boundedness}:

First, we have 
\begin{small}
\begin{align}
    H_{\mathcal V}(Y|X,\mathcal E) &= \mathbb E_{e \sim \mathcal E} \left[ \inf\limits_{f\in\mathcal{V}}\mathbb{E}_{x,y\sim X,Y|\mathcal E=e}[-\log f[x](y)] \right] \\
    &= \mathbb E_{e \sim \mathcal E} \left[ \inf\limits_{f\in\mathcal{V}}\mathbb{E}_{x\sim X|\mathcal E=e} \left[\mathbb E_{y \sim Y|x,e}[-\log f[x](y)] \right] \right]  \\
    &\geq 0,
\end{align}
\end{small}
by noticing that $\mathbb E_{y \sim Y|x}[-\log f[x](y)]$ is the cross entropy between $Y|x,e$ and $f[x]$.

Next,
\begin{small}
\begin{align}
    H_{\mathcal V}(Y|\emptyset,\mathcal E) &= \mathbb E_{e \sim \mathcal E} \left[ \inf\limits_{f\in\mathcal{V}}\mathbb{E}_{y\sim Y|\mathcal E=e}[-\log f[\emptyset](y)] \right] \\
    \label{equ:proposition1_3_1}
    &\leq \inf\limits_{f\in\mathcal{V}} \mathbb E_{e \sim \mathcal E} \left[ \mathbb{E}_{y\sim Y|\mathcal E=e}[-\log f[\emptyset](y)] \right] \\
    &= \inf\limits_{f\in\mathcal{V}} \mathbb{E}_{y\sim Y}[-\log f[\emptyset](y)] \\
    &= H_{\mathcal V}(Y|\emptyset),
\end{align}
\end{small}
where Equation \ref{equ:proposition1_3_1} is due to Jensen's inequality.

Combing the above inequalities,
\begin{align}
    \mathcal{H}^{\mathscr E}_{\mathcal V}(X \rightarrow Y) &=
    \sup_{\mathcal{E} \in \mathscr E} \left[H_\mathcal V(Y|\emptyset,\mathcal E) - H_\mathcal V(Y|X,\mathcal E)\right] - \left[H_\mathcal V(Y|\emptyset) - H_\mathcal V(Y|X)\right] \\
    &\leq  \sup_{\mathcal{E} \in \mathscr E} H_\mathcal V(Y|\emptyset,\mathcal E)  - \left[H_\mathcal V(Y|\emptyset) - H_\mathcal V(Y|X)\right] \\
    &\leq H_\mathcal V(Y|\emptyset) - \left[H_\mathcal V(Y|\emptyset) - H_\mathcal V(Y|X)\right] \\
    &= H_\mathcal V(Y|X).
\end{align}
4. \emph{Corner Case}:

According to Proposition 2 in \cite{DBLP:conf/iclr/XuZSSE20}, 
\begin{align}
    H_\Omega(Y|\emptyset) &= H(Y). \\
    H_\Omega(Y|X) &= H(Y|X).
\end{align}
By taking random variables $R,S$ identically distributed as $X,Y|\mathcal E=e$ for $e \in \text{supp}(\mathcal E)$, we have
\begin{align}
    H_{\Omega}(Y|X,\mathcal E=e) = H_{\Omega}(S|R) = H(S|R) = H(Y|X,\mathcal E=e).
\end{align}
Thus, 
\begin{align}
    H_\Omega(Y|X,\mathcal E) = \mathbb E_{e\sim \mathcal E}[H_\Omega(Y|X,\mathcal E=e)] = \mathbb E_{e\sim \mathcal E}[H(Y|X,\mathcal E=e)] = H(Y|X,\mathcal E).
\end{align}
Similarly, we have $ H_\Omega(Y|\emptyset,\mathcal E) = H(Y|\mathcal E)$.
Thus,
\begin{align}
    \mathcal{H}^{\mathscr E}_{\Omega}(X \rightarrow Y) &= 
    \sup_{\mathcal{E} \in \mathscr E} \left[H_\Omega(Y|\emptyset,\mathcal E) - H_\Omega(Y|X,\mathcal E)\right] - \left[H_\Omega(Y|\emptyset) - H_\Omega(Y|X)\right] \\
    &=  \sup_{\mathcal{E} \in \mathscr E} \left[H(Y|\mathcal E) - H(Y|X,\mathcal E)\right] - \left[H(Y) - H(Y|X)\right] \\
    &= \sup_{\mathcal{E} \in \mathscr E}\mathbb{I}(Y;X|\mathcal{E})-\mathbb{I}(Y;X) \\
    &= \mathcal{H}^{\mathscr E}(X, Y).
\end{align}
\end{proof}




\section{Proof of Theorem \ref{theorem: homogeneous}}
\label{proof:homogeneous}
\begin{proof}[Proof of Theorem \ref{theorem: homogeneous}]
\quad

1)

\begin{align}
    H_{\mathcal V_{\mathcal G}}(Y|X) &= \inf\limits_{f\in \mathcal V_{\mathcal G}}\mathbb{E}_{x\sim X} \left[\mathbb E_{y \sim Y|x}[-\log f[x](y)] \right] \\
    \label{equ:theorem1_1_1}
    &\leq \mathbb{E}_{x\sim X} \left[\mathbb E_{y \sim Y|x}[-\log \frac{1}{\sqrt{2\pi} \cdot \frac{1}{\sqrt{2\pi}}} \exp{ \left[-\frac{(y-g(x))^2}{2\cdot \frac{1}{2\pi} } \right] } \right] \\
    &= \mathbb{E}_{x\sim X} \left[\mathbb E_{y \sim Y|x}[\pi (y-g(x))^2 ] \right] = \pi\sigma^2.
\end{align}
Equation \ref{equ:theorem1_1_1} holds by taking $f[x] = \mathcal N(g(x), \frac{1}{2\pi})$.

2) 

Given the function family $\mathcal{V}_\sigma=\{f | f[x]=\mathcal{N}(\theta x,\sigma^2), \theta \in \mathbb R, \sigma \text{ fixed }\}$, by expanding the Gaussian probability density function in the definition of predictive $\mathcal V$-information, it could be shown that 
\begin{align}
\label{equ:Iv(X->Y)}
    \mathbb{I}_{\mathcal{V}_\sigma}(X\rightarrow Y) &\propto \min_{k\in \mathbb R} -\mathbb{E}[(Y-kX)^2] + \text{Var}(Y),
\end{align}
where the predictive $\mathcal V$-information is proportional to Mean Square Error subtracted by the variance of target, by a coefficient completely dependent on $\sigma$.

The minimization problem is solved by 
\begin{equation}
    k = \frac{\mathbb E[XY]}{\mathbb E[X^2]} = 1.
\end{equation}
Substituting $k$ into eq.\ref{equ:Iv(X->Y)},
\begin{align}
    \mathbb{I}_{\mathcal{V}_\sigma}(X\rightarrow Y) &\propto (-\mathbb E[\epsilon^2] + \text{Var}(X+\epsilon))=\text{Var}(X) = \mathbb E[X^2].
\end{align}
Denote $\text{supp}(\mathcal{E})=\{\mathcal{E}_1,\mathcal{E}_2\}$. Let $Q$ be the joint distribution of $(X,\epsilon,\mathcal E)$. 
% and $P(X,\epsilon) = Q(X,\epsilon,\mathcal E_1) + Q(X,\epsilon,\mathcal E_2)$. 
Let $Q(\mathcal E_1)=\alpha$ and $ Q(\mathcal E_2)=1-\alpha $ be the marginal of $\mathcal E$. Abbreviate $Q(X,\epsilon|\mathcal E=\mathcal E_1)$ by $P_1(X,\epsilon)$ and $Q(X,\epsilon|\mathcal E=\mathcal E_2)$ by $P_2(X,\epsilon)$.

Similar to \ref{equ:Iv(X->Y)},
\begin{align}
\label{equ:Iv(X->Y|e)}
    \mathbb{I}_{\mathcal{V}_\sigma}(X\rightarrow Y|\mathcal E) &\propto \min_k -\mathbb{E}[(Y-kX)^2|\mathcal E] + \text{Var}(Y|\mathcal E).
\end{align}
For $\mathcal E=\mathcal E_1$, the minimization problem is solved by
\begin{align}
    k = \frac{\mathbb E_{P_1}[XY]}{\mathbb E_{P_1}[X^2]}.
\end{align}
Thus,
\begin{small}
\begin{align}
    \mathbb{I}_{\mathcal{V}_\sigma}(X\rightarrow Y|\mathcal E=\mathcal E_1) &\propto -\mathbb E_{P_1}\left[\left(Y-\frac{\mathbb E_{P_1}[XY]}{\mathbb E_{P_1}[X^2]}X\right)^2\right] + \text{Var}_{P_1}(Y) \\
    &= -\mathbb E_{P_1}[Y^2] + \frac{\mathbb E_{P_1}^2[XY]}{\mathbb E_{P_1}[X^2]} + (\mathbb E_{P_1}[Y^2] - \mathbb E_{P_1}^2[Y]) 
    = -\mathbb E_{P_1}^2[Y] + \frac{\mathbb E_{P_1}^2[XY]}{\mathbb E_{P_1}[X^2]}.
\end{align}
\end{small}
Similarly, we have
\begin{align}
\label{equ:Iv(X->Y|e_2)}
    \mathbb{I}_{\mathcal{V}_\sigma}(X\rightarrow Y|\mathcal E=\mathcal E_2) &\propto -\mathbb E_{P_2}^2[Y] + \frac{\mathbb E_{P_2}^2[XY]}{\mathbb E_{P_2}[X^2]}.
\end{align}
Notably, $\mathbb E_{P_1}[X^2]$ and $\mathbb E_{P_2}[X^2]$ are constrained by $\alpha$ and $\mathbb E[X^2]$.
\begin{equation}
    \mathbb E[X^2] = \mathbb E[\mathbb E[X^2|\mathcal E]] = \alpha \mathbb E_{P_1}[X^2] + (1-\alpha)\mathbb E_{P_2}[X^2].
\end{equation}
Similarly,
\begin{equation}
    \mathbb E[X^2] = \mathbb E[XY] = \alpha \mathbb E_{P_1}[XY] + (1-\alpha)\mathbb E_{P_2}[XY].
\end{equation}
\begin{equation}
    0 =\mathbb E[Y] = \alpha \mathbb E_{P_1}[Y] + (1-\alpha)\mathbb E_{P_2}[Y].
\end{equation}
The moments of $P_2$ could thereafter be represented by those of $P_1$.
\begin{small}
\begin{equation}
    \mathbb E_{P_2}[X^2] = \frac{\mathbb E[X^2] - \alpha \mathbb E_{P_1}[X^2]}{1-\alpha},
    \mathbb E_{P_2}[XY] = \frac{\mathbb E[X^2] - \alpha \mathbb E_{P_1}[XY]}{1-\alpha},
    \mathbb  E_{P_2}[Y] = - \frac{\alpha \mathbb E_{P_1}[Y]}{1-\alpha}.
\end{equation}
\end{small}
Substituting to eq.\ref{equ:Iv(X->Y|e_2)},
\begin{align}
    \mathbb{I}_{\mathcal{V}_\sigma}(X\rightarrow Y|\mathcal E=\mathcal E_2) &\propto -\frac{\alpha^2}{(1-\alpha)^2}E_{P_1}^2[Y] + \frac{1}{1-\alpha}\frac{\left(\mathbb E[X^2] - \alpha \mathbb E_{P_1}[XY]\right)^2}{\mathbb E[X^2] - \alpha \mathbb E_{P_1}[X^2]}.
\end{align}
Thus,
\begin{small}
\begin{align}
    \mathcal H_{\mathcal V_\sigma}^{\mathscr E}(X \rightarrow Y) &= \sup_{\mathcal E \in \mathscr E} -\mathbb{I}_{\mathcal{V}_\sigma}(X\rightarrow Y) + \alpha \mathbb{I}_{\mathcal{V}_\sigma}(X\rightarrow Y|\mathcal E=\mathcal E_1) + (1-\alpha) \mathbb{I}_{\mathcal{V}_\sigma}(X\rightarrow Y|\mathcal E=\mathcal E_2) \\
    &\propto \sup_{\mathcal E \in \mathscr E} -\mathbb E[X^2] - \alpha \mathbb E_{P_1}^2[Y] + \alpha \frac{\mathbb E_{P_1}^2[XY]}{\mathbb E_{P_1}[X^2]} - \frac{\alpha^2}{1-\alpha} \mathbb E_{P_1}^2[Y] + \frac{\left(\mathbb E[X^2] - \alpha \mathbb E_{P_1}[XY]\right)^2}{\mathbb E[X^2] - \alpha \mathbb E_{P_1}[X^2]} \\
%    &= \sup_{\mathcal E \in \mathscr E} -\frac{\alpha}{1-\alpha}\mathbb E_{P_1}^2[Y] + \alpha \frac{\left(\mathbb E_{P_1}[X^2]-\mathbb E_{P_1}[XY]\right)^2}{\mathbb E_{P_1}[X^2]\left(\mathbb E[X^2] - \alpha \mathbb E_{P_1}[X^2]\right)} \mathbb E[X^2] \\
    &= \sup_{\mathcal E \in \mathscr E} -\frac{\alpha}{1-\alpha}\mathbb E_{P_1}^2[X+\epsilon] + \alpha \frac{\mathbb E_{P_1}^2[X\epsilon]}{\mathbb E_{P_1}[X^2]\left(\mathbb E[X^2] - \alpha \mathbb E_{P_1}[X^2]\right)} \mathbb E[X^2].
\end{align}
\end{small}
Assuming $X \perp \epsilon \;|\; \mathcal E$, 
\begin{align}
    \mathcal H_{\mathcal V_\sigma}^{\mathscr E}(X \rightarrow Y) \propto \sup_{\mathcal E \in \mathscr E} -\frac{\alpha}{1-\alpha}\mathbb E_{P_1}^2[X+\epsilon] \leq 0.
\end{align}
From Proposition \ref{proposition1}, we have $\mathcal H_{\mathcal V_\sigma}^{\mathscr E}(X \rightarrow Y) \geq 0$. Thus, $\mathcal H_{\mathcal V_\sigma}^{\mathscr E}(X \rightarrow Y) = 0$.
\end{proof}


\section{Proof of Linear Cases (Theorem \ref{theorem: selection-bias} and \ref{theorem: omitted variable})}
\label{proof: linear}
\begin{proof}[Proof of Theorem \ref{theorem: selection-bias}]

For the ease of notion, we denote the $r(\mathcal{E}^*)$ as $r_e$, $\sigma(\mathcal{E}^*)$ as $\sigma_e$, and $\sigma(\mathcal{E}^*)\cdot\epsilon_v$ as $\epsilon_e$. 
And we omit the superscript $\mathcal C$ of $\mathcal{H}_{\mathcal V}^{\mathcal C}$.
	Firstly, we calculate the $H_{\mathcal{V}}[Y|\emptyset]$ as:
	\begin{align}
		H_{\mathcal{V}}[Y|\emptyset] &= \frac{1}{2\sigma^2}\text{Var}(Y) + \log\sigma + \frac{1}{2}\log 2\pi,\\
		H_{\mathcal{V}}[Y|\emptyset,\mathcal{E}^*] &= \frac{1}{2\sigma^2} \mathbb{E}_{\mathcal{E}^*}[\text{Var}(Y|\mathcal{E}^*)]+ \log\sigma + \frac{1}{2}\log 2\pi.
	\end{align}
	Therefore, we have
	\begin{equation}
		H_{\mathcal{V}}[Y|\emptyset,\mathcal{E}^*] - H_{\mathcal{V}}[Y|\emptyset] = -\frac{1}{2\sigma^2}\text{Var}(\mathbb{E}[Y|\mathcal{E}^*])\leq 0.
	\end{equation}
	As for $H_{\mathcal{V}}[Y|X]$, we have
	\begin{align}
		H_{\mathcal{V}}[Y|X] &= \inf_{h_S,h_V}\mathbb{E}_{X,Y}\left[\|Y-(h_SS+h_VV)\|^2\right]\frac{1}{2\sigma^2}\\
%		&= \inf_{h_S,h_V}\mathbb{E}_{X,Y}\left[\|f(S)+\epsilon_Y-(h_SS+h_VV)\|^2\right]\frac{1}{2\sigma^2}\\
		&= \inf_{h_S,h_V}\mathbb{E}_{\mathcal{E}^*}\left[\mathbb{E}[\|f(S)+\epsilon_Y-(h_SS+h_V(r_ef(S)+\epsilon_e))\|^2|\mathcal{E}^*]\right]\frac{1}{2\sigma^2},
	\end{align}
	where we let $h_S=h_S-\beta$ here.
	Then we have
	\begin{align}
	2\sigma^2 H_{\mathcal{V}}[Y|X] &= \inf_{h_S,h_V}\mathbb{E}_{\mathcal{E}^*}\left[\mathbb{E}[\|(1-h_Vr_e)f(S)+\epsilon_Y-h_SS-h_V\epsilon_e\|^2|\mathcal{E}^*]\right]\\
		&= \inf_{h_S,h_V}\mathbb{E}_{\mathcal{E}^*}\left[\mathbb{E}[\|(1-h_Vr_e)f(S)-h_SS\|^2|\mathcal{E}^*]\right] + \sigma_Y^2 + h_V^2\mathbb{E}_{\mathcal{E}^*}[\sigma_e^2],
	\end{align}
	notably that here for $e_i,e_j\in\text{supp}(\mathcal{E}^*)$, we assume $P^{e_i}(S,Y)=P^{e_j}(S,Y)$ (we choose such $\mathcal{E}^*$ as one possible split).
	And the solution of $h_S,h_V$ is
	\begin{align}
		h_S &= \frac{\text{Var}(r_e)\mathbb{E}[f^2(S)]\mathbb{E}[f(S)S]+\mathbb{E}[\sigma_e^2]\mathbb{E}[f(S)S]}{\mathbb{E}[r_e^2]\mathbb{E}[f^2(S)]\mathbb{E}[S^2] + \mathbb{E}[\sigma_e^2]\mathbb{E}[S^2] - \mathbb{E}^2[r_e]\mathbb{E}^2[f(S)S]},\\
		h_V &= \frac{\mathbb{E}[r_e](\mathbb{E}[f^2(S)]\mathbb{E}[S^2]-\mathbb{E}^2[f(S)S])}{\mathbb{E}[r_e^2]\mathbb{E}[f^2(S)]\mathbb{E}[S^2] + \mathbb{E}[\sigma_e^2]\mathbb{E}[S^2] - \mathbb{E}^2[r_e]\mathbb{E}^2[f(S)S]}.
	\end{align}
	According to the assumption that $\mathbb{E}[f(S)S]=0$, we have
	\begin{align}
		h_S = 0,\quad
		h_V = \frac{\mathbb{E}[r(\mathcal E^*)]\mathbb{E}[f^2]}{\mathbb{E}[r^2(\mathcal E^*)]\mathbb{E}[f^2]+\mathbb{E}[\sigma^2(\mathcal E^*)]}.
	\end{align}
	Therefore, we have
	\begin{align}
		2\sigma^2 H_{\mathcal{V}}[Y|X] &= \mathbb{E}_{\mathcal{E}^*}[\mathbb{E}[\|(1-h_Vr_e)f(S)\|^2|\mathcal{E}^*]] + \sigma_Y^2+h_V^2\mathbb{E}_{\mathcal{E}^*}[\sigma_e^2]\\
		&= \frac{\text{Var}(r_e)\mathbb{E}[f^2]+\mathbb{E}[\sigma^2(\mathcal E^*)]}{\mathbb{E}[r_e^2]\mathbb{E}[f^2]+\mathbb{E}[\sigma^2(\mathcal E^*)]}\mathbb{E}[f^2(S)]+ \sigma_Y^2,\\
		2\sigma^2 H_{\mathcal{V}}[Y|X,\mathcal{E}^*] &= \sigma_Y^2+ \mathbb{E}[(\frac{1}{\frac{r_e^2\mathbb{E}[f^2]}{\sigma_e^2}+1})^2]\mathbb{E}[f^2]+ \mathbb{E}_{\mathcal{E}^*}[(\frac{1}{\frac{r_e}{\sigma_e}+\frac{\sigma_e}{r_e\mathbb{E}[f^2]}})^2].
	\end{align}
	Note that here we simply set $\sigma=1$ in the main body.
    And we have:
    \begin{equation}
        \mathcal{H}_{\mathcal V}(X\rightarrow Y)\approx \frac{\text{Var}(r_e)\mathbb{E}[f^2]+\mathbb{E}[\sigma^2(\mathcal E^*)]}{\mathbb{E}[r_e^2]\mathbb{E}[f^2]+\mathbb{E}[\sigma^2(\mathcal E^*)]}\mathbb{E}[f^2(S)]
    \end{equation}
    The approximation error is bounded by $\frac{1}{2}\max(\sigma_Y^2, R(r(\mathcal E^*), \sigma(\mathcal E^*), \mathbb{E}[f^2]))$, and $R(r(\mathcal E^*), \sigma(\mathcal E^*), \mathbb{E}[f^2])$ is defined as:
    \begin{equation}
        R(r(\mathcal E^*), \sigma(\mathcal E^*), \mathbb{E}[f^2]) = \mathbb{E}[(\frac{1}{\frac{r_e^2\mathbb{E}[f^2]}{\sigma_e^2}+1})^2]\mathbb{E}[f^2]+ \mathbb{E}_{\mathcal{E}^*}[(\frac{1}{\frac{r_e}{\sigma_e}+\frac{\sigma_e}{r_e\mathbb{E}[f^2]}})^2]
    \end{equation}
\end{proof}

\begin{proof}[Proof of Theorem \ref{theorem: omitted variable}]
	Similar as the above proof.	
\end{proof}



\section{Proof of the Error Bound for Finite Sample Estimation (Theorem \ref{theorem:pac})}
\label{proof: pac}

In this section, we will prove the error bound of estimating the predictive heterogeneity with the empirical predictive heterogeneity. Before the proof of Theorem \ref{theorem:pac} which is inspired by \cite{DBLP:conf/iclr/XuZSSE20}, we will introduce three lemmas.

\begin{lemma}
\label{lemma:err_1}
Assume $\forall x \in \mathcal X$,$\forall y \in \mathcal Y$,$\forall f \in \mathcal V$, $\log f[x](y) \in [-B,B]$ where $B > 0$. Define a function class $\mathcal G_{\mathcal V}^k = \{g|g(x,y) = \log f[x](y)q(\mathcal E=e_k|x,y), f\in \mathcal V, q \in \mathcal Q  \}$. Denote the Rademacher complexity of $\mathcal G$ with $N$ samples by $\mathscr R_{N}(\mathcal G)$. Define 
\begin{equation}
\hat f_k = \arg \inf_f \frac{1}{|\mathcal D|}  \sum_{x_i,y_i \in \mathcal D} -\log f[x_i](y_i) q(\mathcal E=e_k|x_i,y_i).
\end{equation}

Then for any $q \in \mathcal Q$,  any $\delta \in (0,1)$, with a probability over $1 - \delta$,  we have
\begin{small}
\begin{align}
    &\quad\; \left|q(\mathcal E=e_k)H_{\mathcal V}(Y|X,\mathcal E=e_k)  - \frac{1}{|\mathcal D|} \sum_{x_i,y_i \in \mathcal D} -\log \hat{f_k}[x_i](y_i) q(\mathcal E=e_k|x_i,y_i) \right| \\
    &\leq 2\mathscr R_{|\mathcal D|}(\mathcal G_{\mathcal V}^k) + B\sqrt{\frac{2\log{\frac{1}{\delta}}}{|\mathcal D|}}.
\end{align}
\end{small}
\end{lemma}
\begin{proof}
Apply McDiarmid's inequality to the function $\Phi(\mathcal D)$ which is defined as:
\begin{small}
\begin{align}
    \Phi(\mathcal D)  
    &= \sup_{f\in \mathcal V, q \in \mathcal Q} \left| q(\mathcal E=e_k)\mathbb E_{q} \left[ -\log f[x](y)|\mathcal E=e_k \right]  - \frac{1}{|\mathcal D|} \sum_{x_i,y_i \in \mathcal D} -\log f[x_i](y_i) q(\mathcal E=e_k|x_i,y_i)   \right|.
\end{align}
\end{small}
Let $\mathcal D$ and $\mathcal D'$ be two identical datasets except for one data point $x_j \neq x_j'$. We have:
\begin{small}
\begin{align}
    & \quad\; \Phi(\mathcal D) - \Phi(\mathcal D') \\
    & \leq \sup_{f\in \mathcal V, q \in \mathcal Q} \left[ \left| q(\mathcal E=e_k)\mathbb E_{q} \left[ -\log f[x](y)|\mathcal E=e_k \right]  - \frac{1}{|\mathcal D|} \sum_{x_i,y_i \in \mathcal D} -\log f[x_i](y_i) q(\mathcal E=e_k|x_i,y_i)   \right| \right.\\ 
    &\left. \quad\quad\quad\quad - \left| q(\mathcal E=e_k)\mathbb E_{q} \left[ -\log f[x](y)|\mathcal E=e_k \right]  - \frac{1}{|\mathcal D'|} \sum_{x_i',y_i' \in \mathcal D'} -\log f[x_i'](y_i') q(\mathcal E=e_k|x_i',y_i')   \right|\right] \\
    &\leq \sup_{f\in \mathcal V, q \in \mathcal Q} \left| \frac{1}{|\mathcal D|} \sum_{x_i,y_i \in \mathcal D} -\log f[x_i](y_i) q(\mathcal E=e_k|x_i,y_i) - \frac{1}{|\mathcal D'|} \sum_{x_i',y_i' \in \mathcal D'} -\log f[x_i'](y_i') q(\mathcal E=e_k|x_i',y_i')  \right| \\
    &=  \sup_{f\in \mathcal V, q \in \mathcal Q}\frac{1}{|\mathcal D|}  \left| \log f[x_j](y_j) q(\mathcal E=e_k|x_j,y_j) - \log f[x_j'](y_j') q(\mathcal E=e_k|x_j',y_j') \right| \\
    &\leq \frac{2B}{|\mathcal D|}.
\end{align}
\end{small}
According to McDiarmid's inequality, for any $\delta \in (0,1)$, with a probability over $1 - \delta$, we have:
\begin{align}
    \label{equ:err_7}
    \Phi(\mathcal D) \leq \mathbb E_{\mathcal D} [\Phi(\mathcal D)] + B\sqrt{\frac{2\log{\frac{1}{\delta}}}{|\mathcal D|}}.
\end{align}
Next we derive a bound for $\mathbb E_{\mathcal D}[\Phi(\mathcal D)]$.
Consider a dataset $\mathcal D'$ independently and identically drawn from $q(X,Y) = P(X,Y)$ with the same size as $\mathcal D$. We notice that
\begin{small}
\begin{align}
    q(\mathcal E=e_k)\mathbb E_{q} \left[ -\log f[x](y)|\mathcal E=e_k \right]
%    &= q(\mathcal E=e_k)\mathbb E_{q} \left[ -\log f[x](y) q(\mathcal E=e_k|x,y)|\mathcal E=e_k \right] \\
%    &= \mathbb E_q\left[ \mathbb E_q\left[ -\log f[x](y) q(\mathcal E=e_k|x,y)|\mathcal E=e_k \right] \right] \\
%    &= \mathbb E_q\left[ -\log f[x](y) q(\mathcal E=e_k|x,y) \right] \\
    = \mathbb E_{\mathcal D'} \left[ -\frac{1}{|\mathcal D'|} \sum_{x_i',y_i' \in \mathcal D'} -\log f[x_i'](y_i') q(\mathcal E=e_k|x_i',y_i') \right].
\end{align}
\end{small}
Thus, $\mathbb E_{\mathcal D}[\Phi(\mathcal D)]$ could be reformulated as:
\begin{small}
\begin{align}
    \mathbb E_{\mathcal D}[\Phi(\mathcal D)] &= \mathbb E_{\mathcal D}\left[ \sup_{f\in \mathcal V, q \in \mathcal Q}  \left| \mathbb E_{\mathcal D'} \left[ -\frac{1}{|\mathcal D'|} \sum_{x_i',y_i' \in \mathcal D'} -\log f[x_i'](y_i') q(\mathcal E=e_k|x_i',y_i') \right] 
    \right.\right.\\
    &\left.\left. \quad\quad\quad\quad\quad\quad\quad - \frac{1}{|\mathcal D|} \sum_{x_i,y_i \in \mathcal D} -\log f[x_i](y_i) q(\mathcal E=e_k|x_i,y_i) \right| \right] \\
    &\leq \mathbb E_{\mathcal D}\left[ \sup_{f\in \mathcal V, q \in \mathcal Q} \mathbb E_{\mathcal D'} \left| -\frac{1}{|\mathcal D'|} \sum_{x_i',y_i' \in \mathcal D'} -\log f[x_i'](y_i') q(\mathcal E=e_k|x_i',y_i') \right.\right. \\
    &\left.\left.\quad\quad\quad\quad\quad\quad\quad\quad\quad - \frac{1}{|\mathcal D|} \sum_{x_i,y_i \in \mathcal D} -\log f[x_i](y_i) q(\mathcal E=e_k|x_i,y_i) \right| \right] \\
    \label{equ:err_1}
    &\leq \mathbb E_{\mathcal D, \mathcal D'} \left[  \sup_{f\in \mathcal V, q \in \mathcal Q} \frac{1}{|\mathcal D|} \left| \sum_{x_i,y_i \in \mathcal D} \log f[x_i](y_i) q(\mathcal E=e_k|x_i,y_i) \right.\right.\\
    &\left.\left.\quad\quad\quad\quad\quad\quad\quad\quad\quad - \sum_{x_i',y_i' \in \mathcal D'} \log f[x_i'](y_i') q(\mathcal E=e_k|x_i',y_i') \right| \right] \\
    \label{equ:err_2}
%    &= \mathbb E_{\mathcal D, \mathcal D', \sigma} \left[  \sup_{f\in \mathcal V, q \in \mathcal Q} \frac{1}{|\mathcal D|} \left| \sum_{x_i,y_i \in \mathcal D} \sigma_i \log f[x_i](y_i) q(\mathcal E=e_k|x_i,y_i) \right.\right.\\
    &\left.\left.\quad\quad\quad\quad\quad\quad\quad\quad\quad - \sum_{x_i',y_i' \in \mathcal D'} \sigma_i \log f[x_i'](y_i') q(\mathcal E=e_k|x_i',y_i') \right| \right] \\
    &\leq \mathbb E_{\mathcal D, \sigma} \left[ \sup_{f\in \mathcal V, q \in \mathcal Q} \frac{1}{|\mathcal D|} \left| \sum_{x_i,y_i \in \mathcal D} \sigma_i \log f[x_i](y_i) q(\mathcal E=e_k|x_i,y_i) \right| \right] \\
    &\quad\; + \mathbb E_{\mathcal D', \sigma} \left[ \sup_{f\in \mathcal V, q \in \mathcal Q} \frac{1}{|\mathcal D'|} \left| \sum_{x_i',y_i' \in \mathcal D'} \sigma_i \log f[x_i'](y_i') q(\mathcal E=e_k|x_i',y_i') \right| \right] \\
    \label{equ:err_6}
    &= 2\mathscr R_{|\mathcal D|}(\mathcal G_{\mathcal V}^k),
\end{align}
\end{small}
where $\sigma_i$ are independent Rademacher variables. Equation \ref{equ:err_1} follows from Jensen's inequality and the convexity of $\sup$. Equation \ref{equ:err_2} holds due to the symmetry of $\log f[x_i](y_i) q(\mathcal E=e_k|x_i,y_i) - \log f[x_i'](y_i') q(\mathcal E=e_k|x_i',y_i')$ and the argument that Radamacher variables preserve the expected sum of symmetric random variables with a convex mapping (\cite{banach_probability}, Lemma 6.3).

Substituting Equation \ref{equ:err_6} to Equation \ref{equ:err_7}, we have for any $\delta \in (0,1)$, with a probability over $1 - \delta$, $\forall f \in \mathcal V$, $\forall q \in \mathcal Q$, the following holds:
\begin{align}
\label{equ:err_3}
    &\quad\; \left| q(\mathcal E=e_k)\mathbb E_{q} \left[ -\log f[x](y)|\mathcal E=e_k \right]  - \frac{1}{|\mathcal D|} \sum_{x_i,y_i \in \mathcal D} -\log f[x_i](y_i) q(\mathcal E=e_k|x_i,y_i)   \right|\\
    &\leq 2\mathscr R_{|\mathcal D|}(\mathcal G_{\mathcal V}^k) + B\sqrt{\frac{2\log{\frac{1}{\delta}}}{|\mathcal D|}}.
\end{align}
Let $\Tilde{f_k} = \arg \inf_f \{q(\mathcal E=e_k)\mathbb E_{q} \left[ -\log f[x](y)|\mathcal E=e_k \right]\}$. 

Let $\hat{f_k} = \arg \inf_f \{\frac{1}{|\mathcal D|} \sum_{x_i,y_i \in \mathcal D} -\log f[x_i](y_i)  q(\mathcal E=e_k|x_i,y_i)\}$.

Now we have
\begin{align}
    \label{equ:err_4}
    &\quad\; q(\mathcal E=e_k)\mathbb E_{q} \left[ -\log \Tilde{f_k}[x](y)|\mathcal E=e_k \right]  - \frac{1}{|\mathcal D|} \sum_{x_i,y_i \in \mathcal D} -\log \Tilde{f_k}[x_i](y_i) q(\mathcal E=e_k|x_i,y_i) \\
    &\leq q(\mathcal E=e_k)H_{\mathcal V}(Y|X,\mathcal E=e_k)  - \frac{1}{|\mathcal D|} \sum_{x_i,y_i \in \mathcal D} -\log \hat{f_k}[x_i](y_i) q(\mathcal E=e_k|x_i,y_i) \\
    \label{equ:err_5}
    &\leq q(\mathcal E=e_k)\mathbb E_{q} \left[ -\log \hat{f_k}[x](y)|\mathcal E=e_k \right]  - \frac{1}{|\mathcal D|} \sum_{x_i,y_i \in \mathcal D} -\log \hat{f_k}[x_i](y_i) q(\mathcal E=e_k|x_i,y_i).
\end{align}
Combining Equation \ref{equ:err_3} and Equation \ref{equ:err_4}-\ref{equ:err_5}, the lemma is proved.
\end{proof}

\begin{lemma}
\label{lemma:err_2}
Assume $\forall x \in \mathcal X$,$\forall y \in \mathcal Y$,$\forall f \in \mathcal V$, $\log f[\emptyset](y) \in [-B,B]$ where $B > 0$. The definition of $\mathcal G_{\mathcal V}^k$ and $\mathscr R_{N}(\mathcal G)$ follows from Lemma \ref{lemma:err_1}. Define $\hat f_k = \arg \inf_f \frac{1}{|\mathcal D|}  \sum_{x_i,y_i \in \mathcal D} -\log f[\emptyset](y_i) q(\mathcal E=e_k|x_i,y_i)$.

Then for any $q \in \mathcal Q$,  any $\delta \in (0,1)$, with a probability over $1 - \delta$,  we have
\begin{align}
    &\quad\; \left|q(\mathcal E=e_k)H_{\mathcal V}(Y|\mathcal E=e_k)  - \frac{1}{|\mathcal D|} \sum_{x_i,y_i \in \mathcal D} -\log \hat{f_k}[\emptyset](y_i) q(\mathcal E=e_k|x_i,y_i) \right| \\
    &\leq 2\mathscr R_{|\mathcal D|}(\mathcal G_{\mathcal V}^k) + B\sqrt{\frac{2\log{\frac{1}{\delta}}}{|\mathcal D|}}.
\end{align}
\end{lemma}
\begin{proof}
Similar to  Lemma \ref{lemma:err_1}, we could prove that
\begin{align}
    \label{equ:err_8}
    &\quad\; \left|q(\mathcal E=e_k)H_{\mathcal V}(Y|\mathcal E=e_k)  - \frac{1}{|\mathcal D|} \sum_{x_i,y_i \in \mathcal D} -\log \hat{f_k}[\emptyset](y_i) q(\mathcal E=e_k|x_i,y_i) \right| \\
    &\leq 2\mathscr R_{|\mathcal D|}(\mathcal G_{\mathcal V^\emptyset}^k) + B\sqrt{\frac{2\log{\frac{1}{\delta}}}{|\mathcal D|}},
\end{align}
where $\mathcal G_{\mathcal V^\emptyset}^k = \{g|g(x,y) = \log f[\emptyset](y)q(\mathcal E=e_k|x,y), f\in \mathcal V, q \in \mathcal Q  \}$.

According to the definition for the predictive family $\mathcal V$ (\cite{DBLP:conf/iclr/XuZSSE20}, Definition 1), $\forall f \in \mathcal V$, there exists $f' \in \mathcal V$ such that $\forall x \in \mathcal X$, $f[\emptyset] = f'[x]$. Thus, $\mathcal G_{\mathcal V^\emptyset}^k \subset \mathcal G_{\mathcal V}^k$, and therefore $\mathscr R_{|\mathcal D|}(\mathcal G_{\mathcal V^\emptyset}^k) \leq \mathscr R_{|\mathcal D|}(\mathcal G_{\mathcal V}^k)$. Substituting into Equation \ref{equ:err_8}, the lemma is proved.
\end{proof}

\begin{lemma}[\citep{DBLP:conf/iclr/XuZSSE20}, Theorem 1]
\label{lemma:err_3}
Assume $\forall x \in \mathcal X$,$\forall y \in \mathcal Y$,$\forall f \in \mathcal V$, $\log f[x](y) \in [-B,B]$ where $B > 0$. Define a function class $\mathcal G_{\mathcal V}^* = \{g|g(x,y) = \log f[x](y), f\in \mathcal V\}$. The definition of $\mathscr R_{N}(\mathcal G)$ follows from Lemma \ref{lemma:err_1}. 

Then for any $\delta \in (0,0.5)$, with a probability over $1 - 2\delta$,  we have
\begin{align}
    \left|\mathbb I_{\mathcal V}(X\rightarrow Y)  -  \hat{\mathbb I}_{\mathcal V}(X\rightarrow Y) \right| 
    \leq 4\mathscr R_{|\mathcal D|}(\mathcal G_{\mathcal V}^*) + 2B\sqrt{\frac{2\log{\frac{1}{\delta}}}{|\mathcal D|}}.
\end{align}
\end{lemma}

Finally we are prepared to prove Theorem \ref{theorem:pac}.

% \begin{theorem}
% Assume $\forall x \in \mathcal X$, $\forall y \in \mathcal Y$, $\forall f \in \mathcal V$, $\log f[x](y) \in [-B,B]$ where $B > 0$. Given $e \in \mathrm{supp}(\mathcal E)$, define a function class $\mathcal G_{\mathcal V} = \{g|g(x,y) = \log f[x](y)q(\mathcal E=e|x,y), f\in \mathcal V, q \in \mathcal Q  \}$. Denote the Rademacher complexity of $\mathcal G$ with $N$ samples by $\mathscr R_{N}(\mathcal G)$. Let $K = | \mathrm{supp}(\mathcal E)|$.

% Then for any $\delta \in \left(0,\frac{1}{2(K+1)}\right)$, with a probability over $1 - 2(K+1)\delta$, for dataset $\mathcal{D}$, we have
% \begin{align}
%     |\mathcal H_K^\mathcal V - \hat{\mathcal H}_K^\mathcal V(\mathcal D)| \leq 4(K+1)\mathscr R_{|\mathcal D|}(\mathcal G_{\mathcal V}) + 2(K+1)B\sqrt{\frac{2\log{\frac{1}{\delta}}}{|\mathcal D|}}.
% \end{align}
% \end{theorem}

\begin{proof}[Proof of Theorem \ref{theorem:pac}]
We first bound the error of empirical estimation with the sum of items in Lemma \ref{lemma:err_1},\ref{lemma:err_2},\ref{lemma:err_3}.
\begin{small}
\begin{align}
    &\quad\; |\mathcal H_\mathcal V^{\mathscr E_K}(X\rightarrow Y) - \hat{H}_\mathcal V^{\mathscr E_K}(X\rightarrow Y;\mathcal D)| \\
%    &= \left|\left[\sup_{\mathcal E \in \mathscr E_K}\mathbb{I}_{\mathcal{V}}(X\rightarrow Y|\mathcal{E})- \mathbb{I}_{\mathcal{V}}(X\rightarrow Y)\right]
%    - \left[\sup_{\mathcal E \in \mathscr E_K}\hat{\mathbb{I}}_{\mathcal{V}}(X\rightarrow Y|\mathcal{E};\mathcal D)-  \hat {\mathbb{I}}_{\mathcal{V}}(X\rightarrow Y;\mathcal D)\right]\right| \\
    &\leq \left|\sup_{\mathcal E \in \mathscr E_K} \mathbb{I}_{\mathcal{V}}(X\rightarrow Y| {\mathcal{E}}) - \sup_{\mathcal E \in \mathscr E_K} \hat{\mathbb{I}}_{\mathcal{V}}(X\rightarrow Y| {\mathcal{E}};\mathcal D)  \right| 
    + \left| \mathbb{I}_{\mathcal{V}}(X\rightarrow Y) - \hat{\mathbb{I}}_{\mathcal{V}}(X\rightarrow Y;\mathcal D)  \right| \\
    &\leq \sup_{\mathcal E \in \mathscr E_K} \left| \mathbb{I}_{\mathcal{V}}(X\rightarrow Y| {\mathcal{E}}) - \hat{\mathbb{I}}_{\mathcal{V}}(X\rightarrow Y| {\mathcal{E}};\mathcal D) \right| 
    + \left| \mathbb{I}_{\mathcal{V}}(X\rightarrow Y) - \hat{\mathbb{I}}_{\mathcal{V}}(X\rightarrow Y;\mathcal D)  \right| \\
    &= \sup_{q\in \mathcal Q}\left| \sum_{k=1}^K \left[{q}(\mathcal E=e_k)H_{\mathcal V}(Y|\mathcal E=e_k) - {q}(\mathcal E=e_k)H_{\mathcal V}(Y|X, \mathcal E=e_k)\right] \right. \\
    &\left. \quad\quad - \sum_{k=1}^K \left[ q(\mathcal E=e_k)\hat H_{\mathcal V}(Y|\mathcal E=e_k;\mathcal D) - q(\mathcal E=e_k)\hat H_{\mathcal V}(Y|X, \mathcal E=e_k;\mathcal D)\right] \right| \\
    &\quad + \left| \mathbb{I}_{\mathcal{V}}(X\rightarrow Y) - \hat{\mathbb{I}}_{\mathcal{V}}(X\rightarrow Y;\mathcal D)  \right| \\ 
    &\leq \sum_{k=1}^K \sup_{q\in \mathcal Q}\left| {q}(\mathcal E=e_k)H_{\mathcal V}(Y|\mathcal E=e_k) -  q(\mathcal E=e_k)\hat H_{\mathcal V}(Y|\mathcal E=e_k;\mathcal D) \right| \\
    &\quad + \sum_{k=1}^K \sup_{q \in \mathcal Q}\left| {q}(\mathcal E=e_k)H_{\mathcal V}(Y|X, \mathcal E=e_k) -  q(\mathcal E=e_k)\hat H_{\mathcal V}(Y|X, \mathcal E=e_k;\mathcal D) \right| \\
    &\quad + \left| \mathbb{I}_{\mathcal{V}}(X\rightarrow Y) - \hat{\mathbb{I}}_{\mathcal{V}}(X\rightarrow Y;\mathcal D)  \right| \\ 
    &= \sum_{k=1}^K \sup_{q\in \mathcal Q}\left| {q}(\mathcal E=e_k)H_{\mathcal V}(Y|\mathcal E=e_k) - \frac{1}{|\mathcal D|} \sum_{x_i,y_i \in \mathcal D} -\log \hat{f}_k[x_i](y_i) q(\mathcal E=e_k|x_i,y_i) \right| \\
    &\quad + \sum_{k=1}^K \sup_{q \in \mathcal Q}\left| {q}(\mathcal E=e_k)H_{\mathcal V}(Y|X, \mathcal E=e_k) 
    - \frac{1}{|\mathcal D|} \sum_{x_i,y_i \in \mathcal D} -\log \hat{f}_k'[\emptyset](y_i) q(\mathcal E=e_k|x_i,y_i) \right| \\
    &\quad + \left| \mathbb{I}_{\mathcal{V}}(X\rightarrow Y) - \hat{\mathbb{I}}_{\mathcal{V}}(X\rightarrow Y;\mathcal D)  \right|,
\end{align}
\end{small}
where $\hat f_k = \arg \inf_f \frac{1}{|\mathcal D|}  \sum_{x_i,y_i \in \mathcal D} -\log f[x_i](y_i) q(\mathcal E=e_k|x_i,y_i)$,

and $\hat f_k' = \arg \inf_f \frac{1}{|\mathcal D|}  \sum_{x_i,y_i \in \mathcal D} -\log f[\emptyset](y_i) q(\mathcal E=e_k|x_i,y_i)$, for any $q\in \mathcal Q$ and $1\leq k \leq K$. 

For simplicity, let
\begin{small}
\begin{align}
    \mathrm {Err}_k &= \sup_{q \in \mathcal Q}\left| {q}(\mathcal E=e_k)H_{\mathcal V}(Y|X, \mathcal E=e_k) 
    - \frac{1}{|\mathcal D|} \sum_{x_i,y_i \in \mathcal D} -\log \hat{f}_k[x_i](y_i) q(\mathcal E=e_k|x_i,y_i) \right|. \\
    \mathrm {Err}_k' &= \sup_{q \in \mathcal Q}\left| {q}(\mathcal E=e_k)H_{\mathcal V}(Y|X, \mathcal E=e_k) 
    - \frac{1}{|\mathcal D|} \sum_{x_i,y_i \in \mathcal D} -\log \hat{f}_k'[\emptyset](y_i) q(\mathcal E=e_k|x_i,y_i) \right|. \\
    \mathrm {Err}^* &= \left| \mathbb{I}_{\mathcal{V}}(X\rightarrow Y) - \hat{\mathbb{I}}_{\mathcal{V}}(X\rightarrow Y;\mathcal D)  \right|.
\end{align}
\end{small}

Then, by Lemma \ref{lemma:err_1},\ref{lemma:err_2},\ref{lemma:err_3}, 
\begin{small}
\begin{align}
    &\quad\; \mathrm{Pr}\left[|\mathcal H_K^\mathcal V - \hat{\mathcal H}_K^\mathcal V(\mathcal D)| > 4(K+1)\mathscr R_{|\mathcal D|}(\mathcal G_{\mathcal V}) + 2(K+1)B\sqrt{\frac{2\log{\frac{1}{\delta}}}{|\mathcal D|}}\right] \\
    &\leq \mathrm{Pr}\left[\sum_{i=1}^K \mathrm {Err}_k + \sum_{i=1}^K \mathrm {Err}_k' + \mathrm {Err}^* > 4(K+1)\mathscr R_{|\mathcal D|}(\mathcal G_{\mathcal V}) + 2(K+1)B\sqrt{\frac{2\log{\frac{1}{\delta}}}{|\mathcal D|}}\right] \\
    \label{equ:err_9}
    &\leq  \mathrm{Pr}\left[\sum_{i=1}^K \mathrm {Err}_k + \sum_{i=1}^K \mathrm {Err}_k' + \mathrm {Err}^* > \sum_{k=1}^K 4\mathscr R_{|\mathcal D|}(\mathcal G_{\mathcal V}^k) + 4\mathscr R_{|\mathcal D|}(\mathcal G_{\mathcal V}^*) + 2(K+1)B\sqrt{\frac{2\log{\frac{1}{\delta}}}{|\mathcal D|}}\right] \\
%    &\leq \mathrm{Pr}\left[ \bigcup_{k=1}^K \left(\mathrm{Err_k > }2\mathscr R_{|\mathcal D|}(\mathcal G_{\mathcal V}^k) + B\sqrt{\frac{2\log{\frac{1}{\delta}}}{|\mathcal D)|}}\right) + \bigcup_{k=1}^K \left(\mathrm{Err_k' > }2\mathscr R_{|\mathcal D|}(\mathcal G_{\mathcal V}^k) + B\sqrt{\frac{2\log{\frac{1}{\delta}}}{|\mathcal D)|}}\right) \right. \\
    &\left. \quad\quad\quad + \left(\mathrm{Err}^* > 4\mathscr R_{|\mathcal D|}(\mathcal G_{\mathcal V}^*) + 2B\sqrt{\frac{2\log{\frac{1}{\delta}}}{|\mathcal D|}} \right) \right] \\
%    &\leq \sum_{k=1}^K \mathrm{Pr}\left[ \mathrm{Err_k > }2\mathscr R_{|\mathcal D|}(\mathcal G_{\mathcal V}^k) + B\sqrt{\frac{2\log{\frac{1}{\delta}}}{|\mathcal D)|}} \right] + \sum_{k=1}^K \mathrm{Pr}\left[ \mathrm{Err_k' > }2\mathscr R_{|\mathcal D|}(\mathcal G_{\mathcal V}^k) + B\sqrt{\frac{2\log{\frac{1}{\delta}}}{|\mathcal D)|}} \right] \\
    &\quad\; + \mathrm{Pr}\left[ \mathrm{Err}^* > 4\mathscr R_{|\mathcal D|}(\mathcal G_{\mathcal V}^*) + 2B\sqrt{\frac{2\log{\frac{1}{\delta}}}{|\mathcal D|}}  \right] \\
    &\leq 2(K+1)\delta.
\end{align}
\end{small}
Equation \ref{equ:err_9} is because of $\mathcal G_{\mathcal V}^k = \mathcal G_{\mathcal V}$, $\mathcal G_{\mathcal V}^* \subset \mathcal G_{\mathcal V}$ and therefore $R_{|\mathcal D|}(\mathcal G_{\mathcal V}^k) \leq R_{|\mathcal D|}(\mathcal G_{\mathcal V})$, $R_{|\mathcal D|}(\mathcal G_{\mathcal V}^*) \leq R_{|\mathcal D|}(\mathcal G_{\mathcal V})$.
Hence,
\begin{small}
\begin{align}
    &\quad \mathrm{Pr}\left[|\mathcal H_\mathcal V^{\mathscr E_K}(X\rightarrow Y) - \hat{H}_\mathcal V^{\mathscr E_K}(X\rightarrow Y;\mathcal D)| \leq 4(K+1)\mathscr R_{|\mathcal D|}(\mathcal G_{\mathcal V}) + 2(K+1)B\sqrt{\frac{2\log{\frac{1}{\delta}}}{|\mathcal D|}}\right] \\
    &\geq 1 - 2(K+1)\delta.
\end{align}
\end{small}
\end{proof}



\section{Proof of Theorem \ref{theorem:IM}}
\label{proof: IM}
\begin{proof}[Proof of Theorem \ref{theorem:IM}]
    The objective function of our IM algorithm is directly derived from the definition of empirical predictive heterogeneity in Definition \ref{def:empirical_predictive_heterogeneity}.
    For the regression task, we assume the predictive family as 
    \begin{small}
\begin{equation}
	\mathcal{V}_1 = \{g: g[x]=\mathcal{N}(f_{\theta}(x), \sigma^2), f\text{ is the regression model and }\theta\text{ is learnable, }\sigma=1.0 (\text{fixed})\},
\end{equation}	
\end{small}
where we only care about the output of the model and the noise scale of the Gaussian distribution is often ignored, for which we simply set $\sigma=1.0$ as a fixed term.
Then for each environment $e\in\text{supp}(\mathcal{E}^*)$, the $\mathbb{I}_{\mathcal{V}}(X\rightarrow Y|\mathcal{E}^*=e)$ becomes
\begin{equation}
	\mathbb{I}_{\mathcal{V}}(X\rightarrow Y|\mathcal{E}^*=e)\propto \min_\theta \mathbb{E}^[\|Y-f_\theta(X)\|^2|\mathcal{E}^*=e] - \text{Var}(Y|\mathcal{E}^*),
\end{equation}
which corresponds with the MSE loss and the proposed regularizer in Equation \ref{equ:regularizer-regression}.
For the classification task, the derivation is similar, and the regularizer becomes the entropy of $Y$ in sub-population $e$ and the loss function becomes the cross-entropy loss.
\end{proof}

















\vskip 0.2in
\bibliography{sample}

\end{document}
