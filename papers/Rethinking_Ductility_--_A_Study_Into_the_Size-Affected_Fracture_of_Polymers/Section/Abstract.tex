Two-photon polymerization direct laser writing (TPP-DLW) is an emergent additive manufacturing technique which enables the creation of parts with nanoscale features, enabling the utilization of enhanced properties of nanomaterials.
This study explores the impact of print parameters on the fracture toughness of structures fabricated using two-photon polymerization direct laser writing (TPP-DLW). 
The variation of laser power, scanning speed, and layer thickness ultimately changes the polymer degree of conversion, and we analyze the impact of DC on the fracture toughness of the printed structures. 
Our results demonstrate that varying the DC can significantly alter the fracture toughness of the printed structures, with significant reduction in toughness with increasing DC.
By means of a systematic Linear elastic fracture mechanics (LEFM) and Size-effect law (SEL) fracture analysis, we further show that the ductile to brittle transition in fracture behavior with increasing DC occurs due to 1 order of magnitude reduction in the process zone size from 100 $\mu$m  to 3 $\mu$m.
% We also show that the incorporation of the heterogeneity principle can improve the fracture toughness of TPP-DLW structures, which is essential for their use in various fields such as microfluidics, tissue engineering, and microelectromechanical systems. 
Our findings provide important insights into the relationship between print parameters and fracture toughness in TPP-DLW structures, which can guide the development of improved printing techniques and materials for advanced applications.