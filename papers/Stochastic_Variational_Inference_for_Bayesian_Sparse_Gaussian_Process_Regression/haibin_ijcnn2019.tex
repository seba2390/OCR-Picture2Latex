\documentclass[conference]{IEEEtran}
\IEEEoverridecommandlockouts
% The preceding line is only needed to identify funding in the first footnote. If that is unneeded, please comment it out.
\usepackage{cite}
%\usepackage{cancel}
\usepackage{amsmath,amssymb,amsfonts}
\usepackage{algorithmic}
\usepackage{graphicx}
\usepackage{textcomp}
\usepackage{xcolor}
\def\BibTeX{{\rm B\kern-.05em{\sc i\kern-.025em b}\kern-.08em
    T\kern-.1667em\lower.7ex\hbox{E}\kern-.125emX}}
%\usepackage{soul}
\newtheorem{theorem}{Theorem}
\newtheorem{corollary}{Corollary}
% For figures
\usepackage{graphicx} % more modern
%\usepackage{epsfig} % less modern
%\usepackage{subfigure} 

% For citations
%\usepackage[numbers]{natbib}

% For algorithms
%\usepackage{algorithm}
% \usepackage[noend]{algorithmic}
% \newcommand{\theHalgorithm}{\arabic{algorithm}}
% \def\BibTeX{{\rm B\kern-.05em{\sc i\kern-.025em b}\kern-.08em
%     T\kern-.1667em\lower.7ex\hbox{E}\kern-.125emX}}
\begin{document}

\title{Stochastic Variational Inference for Bayesian Sparse Gaussian Process Regression\thanks{This research is supported by the National Research Foundation, Prime Minister’s Office, Singapore under its Campus for Research Excellence and Technological Enterprise (CREATE) programme and the Singapore Ministry of Education Academic Research Fund Tier $2$, MOE$2016$-T$2$-$2$-$156$.}
}

\author{\IEEEauthorblockN{Haibin Yu} 
\IEEEauthorblockA{%\textit{Department of Computer Science} \\
\textit{National University of Singapore}\\
Republic of Singapore \\
haibin@u.nus.edu}
\and
\IEEEauthorblockN{Trong Nghia Hoang}
\IEEEauthorblockA{%\textit{Department of Computer Science} \\
\textit{MIT-IBM Watson AI Lab}\\
Cambridge, MA, USA\\
nghiaht@ibm.com}
\and
\IEEEauthorblockN{Bryan Kian Hsiang Low}
\IEEEauthorblockA{%\textit{Department of Computer Science} \\
\textit{National University of Singapore}\\
Republic of Singapore \\
lowkh@comp.nus.edu.sg}
\and
\IEEEauthorblockN{Patrick Jaillet}
\IEEEauthorblockA{%\textit{Department of EECS} \\
\textit{MIT}\\
Cambridge, MA, USA\\
jaillet@mit.edu}
}

\maketitle

\begin{abstract}
This paper presents a novel  variational inference framework for deriving a family of  Bayesian \emph{sparse Gaussian process regression} (SGPR) models whose approximations are variationally optimal with respect to the full-rank GPR model enriched with various corresponding correlation structures of the observation noises.
Our \emph{variational Bayesian SGPR} (VBSGPR) models jointly treat both the distributions of the inducing variables and hyperparameters as variational parameters, which enables the decomposability of the variational lower bound that in turn can be exploited for stochastic optimization.  
Such a stochastic optimization involves iteratively following the stochastic gradient of the variational lower bound to improve its estimates of the optimal variational distributions of the inducing variables and hyperparameters (and hence the predictive distribution) of our VBSGPR models and is guaranteed to achieve asymptotic convergence to them.
We show that the  stochastic gradient is an unbiased estimator of the exact gradient and can be computed in constant time per iteration, hence achieving scalability to big data.
We empirically evaluate the performance of our proposed framework on two real-world, massive datasets.
\end{abstract}

%\begin{IEEEkeywords}
%Gaussian process, variational inference, stochastic optimization. 
%\end{IEEEkeywords}

\section{Introduction}
\label{sect:intro}
A \emph{Gaussian process regression} (GPR) model is a rich class of Bayesian non-parametric models that can exploit correlation of the data/observations for performing probabilistic non-linear regression by providing a Gaussian predictive distribution with formal measures of predictive uncertainty.
	Such a \emph{full-rank GPR} (FGPR) model, though highly expressive, incurs cubic time in the data size to compute the predictive distribution and 
	learn the hyperparameters (i.e., defining its correlation structure) via maximum likelihood estimation, 
% \cite{Rasmussen06}, 
specifically, in each iteration of gradient ascent to refine the  hyperparameter estimates to improve the log-marginal likelihood.
	%Introduction Second Edition
	So, to learn the hyperparameters in reasonable time, only a very small subset of the data can be considered, which compromises the estimation accuracy:
	It is typically not representative of all the data in describing the underlying correlation structure due to its sparsity over the input space. 
	
	To improve its time efficiency, a number of \emph{sparse GPR} (SGPR) models exploiting low-rank covariance matrix approximations \cite{candela10,candela05} have been proposed,
	%,Snelson07a To boost its scalability, a number of \emph{sparse GPR} (SGPR) models exploiting low-rank covariance matrix approximation \cite{candela10,LowAAAI15,candela05,Candela07,Tresp03,Seeger03,Smola01,Snelson06,Snelson07a} have been proposed, 
	many of which impose a common structural assumption of conditional independence (but of varying degrees) on the FGPR model based on the notion of \emph{inducing variables} and can therefore be encompassed under a unifying view presented in~\cite{candela05}.
	% which include the \emph{subset of regressors} (SoR) \cite{Smola01}, \emph{deterministic training conditional} (DTC) \cite{Seeger03}, \emph{fully independent training conditional} (FITC) \cite{Snelson06}, \emph{fully independent conditional, partially independent training conditional} (PITC) \cite{Tresp03} and \emph{partially independent conditional} (PIC) \cite{Snelson07a} approximations.
	As a result, they incur linear time in the data size that is still prohibitively expensive for training with big data (i.e., million-sized datasets).
	To scale up to big data, parallel \cite{LowUAI13,LowAAAI15,LowDyDESS15} and online \cite{Csato02,LowAAAI14} variants of several of these SGPR models have been developed for prediction (by assuming known hyperparameters) but not hyperparameter learning.
	%but assumed the hyperparameters to be known.
	
	The chief concern with the unifying view of~\cite{candela05} is that it does not rigorously quantify the approximation quality of a SGPR model \cite{Titsias09a}.
	%it lacks theoretical rigor in approximating the FGPR model  because it does not optimize some 
	%distance between the distributions of some latent variables induced by the FGPR and SGPR models.
	%The work of \cite{Titsias09a} criticized that the existing SGPR models lack the theoretical rigor in approximating the FGPR model since they do not include some criterion to measure the distance/divergence between the FGPR and SGPR models. 
	%Therefore, \citeauthor{Titsias09} (\citeyear{Titsias09}) proposed an alternative framework of variational inference by maximizing a lower bound of the log-marginal likelihood. Interestingly, it could result in the DTC approximation of the FGPR model .
	%A major criticism \cite{Titsias09} of the above unifying view of SGPR models is the lack of theoretical rigor in its approximation because it does not involve minimizing some distance/divergence between the distributions induced by the FGPR and SGPR models.
	To address this concern, the work of~\cite{Titsias09} has proposed a principled variational inference framework that involves minimizing the \emph{Kullback-Leibler} (KL) distance between distributions of some latent variables (including the inducing variables) induced by the variational SGPR approximation and
 the FGPR model given the data/observations or, equivalently, maximizing a lower bound of the log-marginal likelihood to yield the \emph{deterministic training conditional} (DTC) approximation \cite{Seeger03}. Hyperparameter learning is then achieved by maximizing this variational lower bound with respect to the hyperparameters via gradient ascent, which still incurs linear time in the data size per iteration but can be substantially reduced by means of parallelization \cite{Yarin14} or stochastic optimization \cite{Lawrence13,cheng2016incremental}. 
 %(\textcolor{red}{Byron Boots paper is included here as ref. 13}) 
%
	%Hyperparamters can be determined by maximizing the lower bound which is mathematically equal to minimizing the \emph{Kullback-Leibler} (KL) distance of the resulting variational DTC approximation to the FGPR model. Its computational complexity is still linear with respect to the size of data per iteration using gradient ascent method. However, the work of \cite{Yarin14} shows the incurred time can be significantly reduced by means of parallelization on multiple cores/machines.
	Unifying frameworks of variational SGPR models and their stochastic and distributed variants are subsequently proposed in \cite{NghiaICML15,HoangICML16} to, respectively, perform stochastic and distributed variational inference 
	%\cite{hoffman2013stochastic} 
for any SGPR model (including DTC) spanned by the unifying view of~\cite{candela05}.
%\textcolor{black}{
The work of~\cite{bui2017streaming} has extended two SGPR models (i.e., DTC and \emph{fully independent training conditional} (FITC) approximation \cite{Snelson06}) to handle streaming data.
%which in the meantime updating the \emph{inducing variables} along the training procedure.}

	%the extension of this framework to include hyperparameter learning has been reported to be highly non-trivial and sidestepped by the authors.
	However, all the above-mentioned variational SGPR models and their stochastic and distributed variants suffer from the following critical issues: 
	(a) The above equivalence only holds for the case of fixed hyperparameters;
	otherwise, since the log-marginal likelihood 
	%(i.e., comprising a sum of its lower bound and the KL distance) 
	also depends on the same hyperparameters that are optimized to maximize its variational lower bound, the resulting KL distance, which quantifies the gap between the log-marginal likelihood and its lower bound, may not be minimized;
	%, the above equivalence no longer holds and hence the . 
	(b) similar to variational expectation-maximization \cite{Jordan08}, the log-marginal likelihood does not necessarily increase in each iteration of gradient ascent to refine the hyperparameter estimates to improve its variational lower bound; and
	(c) they all find point estimates of the hyperparameters, which risk overfitting, especially when the number of hyperparameters is all but small.
% \cite{Rasmussen06}. 
	
	To resolve these issues, the notable work of~\cite{Titsias13} has introduced a \emph{variational Bayesian DTC} (VBDTC) approximation (Section~\ref{Variational Inference of the Bayesian DTC}) capable of learning a variational distribution of the hyperparameters.
	%model selection \cite{DuvenaudICML13}, 
	This learned distribution of hyperparameters is particularly desirable in conveying the uncertainty/confidence of the hyperparameter estimates and for use in Bayesian GP regression (Section~\ref{predict}), active learning \cite{LowAAMAS13,NghiaICML14,LowAAMAS08,LowICAPS09,LowAAMAS11,LowAAMAS14,YehongAAAI16}, Bayesian optimization~\cite{Erik17,Ghahramani14,NghiaAAAI18,ling16}, among others.
Unfortunately, such a VBDTC approximation cannot handle big data (e.g., million-sized datasets) because it incurs linear time in the data size per iteration of gradient ascent. The  \emph{variational Bayesian sparse spectrum GPR} (VSSGPR) model~\cite{Gal2015Improving} overcomes this scalability issue by achieving constant time per iteration of stochastic gradient ascent. But, like VBDTC, VSSGPR imposes a highly restrictive assumption of conditional independence between the test outputs and the training data given the learned hyperparameters (i.e., in its test conditional in equation $4$ therein), thus compromising its predictive performance as shown in our experiments (Section~\ref{Experiments and Discussion}). 
This assumption is later relaxed in the work of~\cite{MinhAAAI17}.
%\cite{Tresp03}
It remains an open question whether more refined SGPR models as well as those others spanned by the unifying view of~\cite{candela05} (e.g., FITC, \emph{partially independent training conditional} (PITC), \emph{partially independent conditional} (PIC) \cite{Snelson07a} approximations) are amenable to the variational Bayesian treatment and achieve scalability through stochastic optimization.%\footnote{A later work of~\citet{Titsias2014doubly} has proposed a doubly stochastic variational inference framework that can learn a variational distribution of the FGPR hyperparameters via gradient ascent but in cubic time in the data size per iteration, as detailed in its supplementary materials and demonstrated in its experiments. To improve its scalability, one may be tempted to resort to a factorized likelihood assumed by the authors in the other problems: For example, the FGPR likelihood term in the gradient ascent updates can simply be replaced by some factorized SGPR one. This, however, yields a variational approximation to the SGPR model instead of the FGPR model, which likely explains why they have not made such an assumption in the context of their problem.}
		
To address this question, this paper presents a novel variational inference framework for deriving a family of Bayesian SGPR models (e.g., VBDTC, VBFITC, VBPIC) whose approximations are, interestingly, variationally optimal with respect to the FGPR model enriched with various corresponding correlation structures of the observation noises (Section~\ref{Variational Inference of the Bayesian DTC}).
	%for an anytime fully Bayesian SGPR model (e.g, variational Bayesian DTC (B-DTC), Bayesian PITC (B-FITC), Bayesian PIC (B-PIC) approximation) that, in particular, can produce good predictive performance fast and improve its predictive performance over time.
Our framework introduces a novel reparameterization of the GP model (Section~\ref{Seperation of Gaussian Process}) for enabling a variational treatment of the distribution of hyperparameters.
	% (Sections~\ref{Bayesian SGP} and~\ref{Seperation of Gaussian Process}). 
	Unlike VBDTC, our framework does not need to assume independently distributed observation noises with constant variance and is thus more robust to different noise correlation structures, hence catering to more realistic applications of GP.
%	 \cite{Huizenga95,Ali15}.
	Furthermore, instead of just considering the distribution of hyperparameters as variational parameters \cite{Titsias13,Gal2015Improving}, we jointly treat both the distributions of the inducing variables and hyperparameters as variational parameters, which  enables the decomposability of the variational lower bound that in turn can be exploited for stochastic optimization (Section~\ref{Stochastic Variational Inference for GPR}).  
	Such a stochastic optimization involves iteratively following the stochastic gradient of the variational lower bound to improve its estimates of the optimal variational distributions of the inducing variables and hyperparameters (and hence the predictive distribution (Section~\ref{predict})) of our \emph{variational Bayesian SGPR} (VBSGPR) models and is guaranteed to achieve asymptotic convergence to them.
	We show that the derived stochastic gradient is an unbiased estimator of the exact gradient and can be computed in constant time (i.e., independent of data size) per iteration, thus achieving scalability to big data.
	We empirically evaluate the performance of the stochastic variants of our VBSGPR models on two real-world  datasets (Section~\ref{Experiments and Discussion}).
%
%%%%%%%%%%%%%%%%%%%%%%%%%%%%%%%%%%%%%%%%%%%%%%%%%%%%%%%%%%%%%%%%%%%%%%%%%%%%%%%%%%%%%%%%%%%%%%%%
%
\section{Background and Notations}%\vspace{-1mm}
\subsection{Full-Rank GP Regression (FGPR) with Correlated  Noises}
\label{full}%\vspace{-0.5mm}
Let $\mathcal{X}$ denote a $d$-dimensional input feature space such that each input vector $\mathbf{x}\in\mathcal{X}$ is associated with a latent output variable $f_{\mathbf{x}}$.
% and its corresponding noisy output $y_{\mathbf{x}}\triangleq f_{\mathbf{x}}+\varepsilon_\mathbf{x}$ differing by an additive noise $\varepsilon_\mathbf{x}$. 
		Let $\{f_{\mathbf{x}}\}_{\mathbf{x}\in\mathcal{X}}$ denote a \emph{Gaussian process} (GP), that is, every finite subset of $\{f_{\mathbf{x}}\}_{\mathbf{x}\in\mathcal{X}}$ follows a multivariate Gaussian distribution. Then, the GP is fully specified by its \emph{prior} mean $\mathbb{E}[f_\mathbf{x}]$  (i.e., assumed to be zero to ease notations) and covariance $k_{\mathbf{x}\mathbf{x}'}\triangleq\mathrm{cov}[f_\mathbf{x}, f_{\mathbf{x}'}]$ for all $\mathbf{x}, \mathbf{x}' \in \mathcal{X}$, the latter of which can be defined, for example, by the widely-used squared exponential covariance function $k_{\mathbf{x}\mathbf{x}'} \triangleq 
%\sigma_f^2\exp(-0.5(\mathbf{x}-\mathbf{x}')^\top\mathbf{\Lambda}^\top\mathbf{\Lambda}(\mathbf{x}-\mathbf{x}')) = 
\sigma_f^2\exp(-0.5\|\mathbf{\Lambda}\mathbf{x} - \mathbf{\Lambda}\mathbf{x}'\|^2_2)$ where 
%\textcolor{red}{$\boldsymbol{\theta} \triangleq \{\mathbf{\Lambda},  \sigma_f\}$}, 
$\mathbf{\Lambda} = \mathrm{diag}[\lambda_1,\ldots,\lambda_d]$ and $\sigma_f^2$ are its \emph{inverted} length-scale and signal variance hyperparameters, respectively. 
Suppose that a column vector $\mathbf{y}_\mathcal{D}\triangleq (y_{\mathbf{x}})^\top_{\mathbf{x}\in\mathcal{D}}$ of noisy observed outputs $y_{\mathbf{x}}\triangleq f_{\mathbf{x}}+\varepsilon_{\mathbf{x}}$ (i.e., corrupted by an additive noise $\varepsilon_{\mathbf{x}}$) is available for some set $\mathcal{D}\subset\mathcal{X}$ of training inputs such that $\boldsymbol{\varepsilon}_{\mathcal{D}}\triangleq(\varepsilon_{\mathbf{x}})^\top_{\mathbf{x} \in \mathcal{D}}$ follows a multivariate Gaussian distribution $p(\boldsymbol{\varepsilon}_{\mathcal{D}}) \triangleq \mathcal{N}(\mathbf{0},\mathbf{C}_{\mathcal{DD}})$ where $\mathbf{C}_{\mathcal{DD}}$   
 is a covariance matrix representing the correlation of observation noises $\boldsymbol{\varepsilon}_\mathcal{D}$. 	It follows that $p(\mathbf{y}_\mathcal{D}|\mathbf{f}_\mathcal{D})=\mathcal{N}(\mathbf{f}_\mathcal{D},\mathbf{C}_{\mathcal{DD}})$ where $\mathbf{f}_\mathcal{D}\triangleq (f_{\mathbf{x}})^\top_{\mathbf{x}\in\mathcal{D}}$.
 %\triangleq\left(\varepsilon_\mathbf{x}\right)_{\mathbf{x}\in\mathcal{D}}^\top\sim\mathcal{N}(\mathbf{0},\mathbf{C}_{\mathcal{D}\mathcal{D}})$.
Then, a FGPR model with correlated observation noises 
%\cite{Murray-Smith01,Rasmussen06} 
can perform probabilistic regression by providing a GP \emph{posterior}/predictive distribution $p(f_{\mathbf{x}^*} | \mathbf{y}_\mathcal{D}) = \mathcal{N}(\mathbf{K}_{\mathbf{x}^*\mathcal{D}}(\mathbf{K}_{\mathcal{D}\mathcal{D}} + \mathbf{C}_{\mathcal{D}\mathcal{D}})^{-1}\mathbf{y}_\mathcal{D}, k_{\mathbf{x}^*\mathbf{x}^*} - \mathbf{K}_{\mathbf{x}^*\mathcal{D}}(\mathbf{K}_{\mathcal{D}\mathcal{D}} + \mathbf{C}_{\mathcal{D}\mathcal{D}})^{-1}\mathbf{K}_{\mathcal{D}\mathbf{x}^*})$ of the latent output $f_{\mathbf{x}^*}$ for any test input $\mathbf{x}^*\in\mathcal{X}$ where $\mathbf{K}_{\mathbf{x}^*\mathcal{D}} \triangleq (k_{\mathbf{x}^*\mathbf{x}})_{\mathbf{x}\in \mathcal{D}}$, $\mathbf{K}_\mathcal{DD} \triangleq (k_{\mathbf{x}\mathbf{x}'})_{\mathbf{x},\mathbf{x}' \in \mathcal{D}}$, and $\mathbf{K}_{\mathcal{D}\mathbf{x}^*} \triangleq \mathbf{K}^\top_{\mathbf{x}^*\mathcal{D}}$.
Computing the GP predictive distribution incurs $\mathcal{O}(|\mathcal{D}|^3)$ time due to inversion of $\mathbf{K}_\mathcal{DD} + \mathbf{C}_{\mathcal{D}\mathcal{D}}$. 
		%which is prohibitively expensive for big data.
		%hence rendering the FGPR model impractical for scaling to big data.`
The FGPR hyperparameters \textcolor{black}{$\boldsymbol{\theta}\triangleq (\lambda_1,\ldots,\lambda_d, \sigma_f)^\top$} 
%\st{$\mathbf{\Lambda}$ and $\sigma^2_f$} 
can be learned using \emph{maximum likelihood estimation} (MLE) by maximizing the log-marginal likelihood $\log p(\mathbf{y}_\mathcal{D}) = \log \mathcal{N}(\mathbf{0}, \mathbf{K}_\mathcal{DD}\hspace{-0mm} +\hspace{-0mm}  \mathbf{C}_{\mathcal{D}\mathcal{D}})$ with respect to \textcolor{black}{$\boldsymbol{\theta}$} 
%\st{$\mathbf{\Lambda}$ and $\sigma^2_f$} 
via gradient ascent, which incurs $\mathcal{O}(|\mathcal{D}|^3)$ time per iteration. 	
So, the FGPR model with correlated noises scales poorly in data size $|\mathcal{D}|$. 
To improve its scalability, our key idea is to impose different sparsity structures on $\mathbf{C}_{\mathcal{D}\mathcal{D}}$ to yield a family of VBSGPR models, as shown  in Section~\ref{Variational Inference of the Bayesian DTC}.%\vspace{-0.4mm}
	%
	\subsection{Sparse Gaussian Process Regression (SGPR)}
	\label{sgpr}%\vspace{-0.7mm}
	To reduce the cubic time cost of the FGPR model, 		
	the SGPR models spanned by the unifying view of~\cite{candela05} exploit a vector $\mathbf{f}_\mathcal{U}\triangleq (f_{\mathbf{x}})_{\mathbf{x}\in\mathcal{U}}^\top$ of inducing output variables for some small set $\mathcal{U}\subset\mathcal{X}$ of inducing inputs (i.e., $|\mathcal{U}|\ll|\mathcal{D}|$) for approximating the GP predictive distribution $p(f_{\mathbf{x}^*}|\mathbf{y}_\mathcal{D})$. In particular, they utilize a common structural assumption \cite{Snelson07a} that the joint distribution of $f_{\mathbf{x}^*}$ and $\mathbf{f}_\mathcal{D} \triangleq (f_{\mathbf{x}})_{\mathbf{x}\in\mathcal{D}}^\top$ given $\mathbf{f}_\mathcal{U}$ factorizes over a pre-defined partition of the input space $\mathcal{X}$ into $B$ disjoint subsets $\mathcal{X}_1,\ldots,\mathcal{X}_{B}$ (i.e., $\mathcal{X} = \mathcal{X}_1 \cup \mathcal{X}_2 \cup \ldots \cup \mathcal{X}_{B}$): 
%	
Formally, without loss of generality, supposing $\mathbf{x}^* \in \mathcal{X}_B$, %\vspace{-1mm}
%	\begin{equation} 
%		\begin{array}{c}
then $p(f_{\mathbf{x}^*},\mathbf{f}_\mathcal{D}|\mathbf{f}_\mathcal{U}) = p(f_{\mathbf{x}^*}|\mathbf{f}_{\mathcal{D}_B},\mathbf{f}_\mathcal{U}) \prod_{i=1}^{B}p(\mathbf{f}_{\mathcal{D}_i}|\mathbf{f}_\mathcal{U})$
%		\end{array}
%		\label{E3}\vspace{-1mm}
%	\end{equation} 
	where $\mathbf{f}_{\mathcal{D}_i} \triangleq (f_{\mathbf{x}})_{\mathbf{x}\in\mathcal{D}_i}^\top$ is a column vector of latent outputs for the disjoint subset $\mathcal{D}_i \triangleq (\mathcal{X}_i \cap \mathcal{D}) \subset \mathcal{D}$ for $i = 1, 2, \ldots, B$. Using this factorization,   
$p(f_{\mathbf{x}^*}|\mathbf{y}_\mathcal{D}) = \int p(f_{\mathbf{x}^*}|\mathbf{y}_{\mathcal{D}_B},\mathbf{f}_\mathcal{U})\ p(\mathbf{f}_\mathcal{U}|\mathbf{y}_\mathcal{D})\ \mathrm{d}\mathbf{f}_\mathcal{U} \simeq \int q(f_{\mathbf{x}^*}|\mathbf{y}_{\mathcal{D}_B},\mathbf{f}_\mathcal{U})\ q(\mathbf{f}_\mathcal{U})\ \mathrm{d}\mathbf{f}_\mathcal{U}$	
% in~\eqref{E3}	
%the GP predictive distribution reduces to
%$p(f_{\mathbf{x}^*} |\mathbf{y}_\mathcal{D})$ 
%\vspace{-1mm}% \cite{NghiaICML15}:
%	\begin{equation}
%		\begin{array}{rcl}
%			p(f_{\mathbf{x}^*}|\mathbf{y}_\mathcal{D})&\hspace{-2.4mm}=&\hspace{-2.4mm}
			%\int p(f_{\mathbf{x}^*}|\mathbf{y}_{\mathcal{D}},\mathbf{f}_\mathcal{U})p(\mathbf{f}_\mathcal{U}|\mathbf{y}_\mathcal{D})\mathrm{d}\mathbf{f}_\mathcal{U}\label{E4}\\&=&
%			\displaystyle\int p(f_{\mathbf{x}^*}|\mathbf{y}_{\mathcal{D}_B},\mathbf{f}_\mathcal{U})\ p(\mathbf{f}_\mathcal{U}|\mathbf{y}_\mathcal{D})\ \mathrm{d}\mathbf{f}_\mathcal{U}\\
%			&\hspace{-2.4mm}\simeq&\hspace{-2.4mm}\displaystyle\int q(f_{\mathbf{x}^*}|\mathbf{y}_{\mathcal{D}_B},\mathbf{f}_\mathcal{U})\ q(\mathbf{f}_\mathcal{U})\ \mathrm{d}\mathbf{f}_\mathcal{U}\vspace{-1.5mm}
%		\end{array}
%		\label{E6}
%	\end{equation}
	where $\mathbf{y}_{\mathcal{D}_B} \triangleq (y_{\mathbf{x}})^\top_{\mathbf{x}\in\mathcal{D}_B}$ is a vector of noisy observed outputs for the subset $\mathcal{D}_B$ of training inputs, the equality is derived in Appendix C.$1$ of \cite{NghiaICML15}, and
	% \ref{Derivation of Equation (3)}
	$p(f_{\mathbf{x}^*}|\mathbf{y}_{\mathcal{D}_B},\mathbf{f}_\mathcal{U})$ and $p(\mathbf{f}_\mathcal{U}|\mathbf{y}_\mathcal{D})$ are, respectively, approximated by $q(f_{\mathbf{x}^*}|\mathbf{y}_{\mathcal{D}_B},\mathbf{f}_\mathcal{U})$ and $q(\mathbf{f}_\mathcal{U})$ 
% in~\eqref{E6}	
that can be appropriately defined to reproduce the predictive distribution of any SGPR model \cite{NghiaICML15} spanned by the unifying view of \cite{candela05}, which can be computed in $\mathcal{O}(|\mathcal{D}||\mathcal{U}|^2)$ time.   
	The SGPR hyperparameters can be learned using MLE by maximizing its corresponding log-marginal likelihood via gradient ascent, which incurs $\mathcal{O}(|\mathcal{D}||\mathcal{U}|^2)$ time per iteration.
	To scale up to big data, these linear time complexities can be significantly reduced  using parallelization or stochastic optimization (Section~\ref{sect:intro}).%\vspace{-0.4mm}
	%
	\subsection{Bayesian SGPR Models}	
	\label{Bayesian SGP}\vspace{1mm}
	For the FGPR and SGPR models described above, % described in Sections~\ref{full} and~\ref{sgpr}, 
	point estimates of their hyperparameters are learned, which is vulnerable to overfitting, especially when the number of hyperparameters is all but small (Section~\ref{sect:intro}). %\cite{Rasmussen06,Titsias13}.
	%In the above FGPR and SGPR models, point estimations of the hyper-parameters are typically obtained by maximizing the marginal likelihood, which usually has a potential vulnerability to overfitting. 
	To mitigate this issue of overfitting, a Bayesian approach to sparse GP regression can be employed by introducing priors $p(\boldsymbol{\theta}) \triangleq p(\mathbf{\Lambda})\ p(\sigma_f)$ over  hyperparameters $\boldsymbol{\theta}$,	
%	hyperparameters $\mathbf{\Lambda}$
%	\footnote{For simplicity, we do not assign any prior over $\theta$ or $\sigma^2_f$: A Bayesian treatment of $\theta$ and $\sigma^2_f$ can be performed using standard variational inference with conjugate hyperpriors \cite{bishop2006pattern}.} 
thus yielding the predictive distribution:\vspace{-0.3mm}
	\begin{equation}
		\hspace{0mm}
		\begin{array}{rcl}
			%&=\int p(f_{\mathbf{x}^*}|\mathbf{y}_{\mathcal{D}_b},\mathbf{f}_\mathcal{U},\mathbf{\Lambda})p(\mathbf{f}_\mathcal{U}|\mathbf{y}_\mathcal{D},\mathbf{\Lambda})p(\mathbf{\Lambda}|\mathbf{y}_\mathcal{D})\mathrm{d}\mathbf{f}_\mathcal{U}\mathrm{d}\mathbf{\Lambda} \\
			p(f_{\mathbf{x}^*}|\mathbf{y}_\mathcal{D}) &\hspace{-2.4mm}=&\hspace{-2.4mm} \displaystyle\int\hspace{-0mm} p(f_{\mathbf{x}^*}|\mathbf{y}_{\mathcal{D}_B},\mathbf{f}_\mathcal{U},\boldsymbol{\theta})\   p(\mathbf{f}_\mathcal{U},\boldsymbol{\theta}|\mathbf{y}_\mathcal{D}) \ \mathrm{d}\mathbf{f}_\mathcal{U}\ \mathrm{d}\boldsymbol{\theta} \\
			&\hspace{-2.4mm}\simeq&\hspace{-2.4mm}\displaystyle\int\hspace{-0mm} q(f_{\mathbf{x}^*}|\mathbf{y}_{\mathcal{D}_B},\mathbf{f}_\mathcal{U},\boldsymbol{\theta})\ q(\mathbf{f}_\mathcal{U},\boldsymbol{\theta})\ \mathrm{d}\mathbf{f}_\mathcal{U}\ \mathrm{d}\boldsymbol{\theta} \vspace{-5.5mm}
		\end{array}
		\label{E7}\vspace{3mm}
	\end{equation}
	where 
	$p(\mathbf{f}_\mathcal{U}, \boldsymbol{\theta} | \mathbf{y}_\mathcal{D})$ is approximated by $q(\mathbf{f}_\mathcal{U},\boldsymbol{\theta})$ which generalizes $q(\mathbf{f}_\mathcal{U})$ above
% in~\eqref{E6} 
by additionally and jointly considering the hyperparameters $\boldsymbol{\theta}$ as variational variables.
	%This is then approximated with a variational distribution $q(\mathbf{f}_\mathcal{U},\mathbf{\Lambda})$ which generalizes that of~\eqref{E6} above: Both the hyper-parameters and inducing outputs are now treated as variational variables jointly distributed by $q(\mathbf{f}_\mathcal{U}, \mathbf{\Lambda})$. 
	Though~\eqref{E7}, in principle, allows a Bayesian treatment of $\boldsymbol{\theta}$ to be incorporated into the existing SGPR models, computing the resulting predictive distribution is intractable because
	%allows us to incorporate Bayesian learning of the hyper-parameters into the existing SGPR frameworks, it also makes the predictive distribution intractable \cite{Titsias13}: This is because 
	it involves integrating, over $\boldsymbol{\Lambda}$,
	probability terms in~\eqref{E7} containing the inverse of $\mathbf{K}_\mathcal{UU} \triangleq (k_{\mathbf{x}\mathbf{x}'})_{\mathbf{x},\mathbf{x}'\in\mathcal{U}}$ that depends on $\boldsymbol{\Lambda}$ but without an analytical form with respect to $\boldsymbol{\Lambda}$. %\cite{Titsias13}.
	%the evaluation of~\eqref{E7} (see Section~\ref{Variational Inference of the Bayesian DTC}) requires integrating over $\mathbf{\Lambda}$ for some term that involves the inversion of $\mathbf{K}_\mathcal{UU}$ where $\mathbf{K}_\mathcal{UU} \triangleq [k(x,x')]$ for all pairs of $x,x'\in \mathcal{U}$ (i.e., the prior covariance of $p(\mathbf{f}_\mathcal{U})$), which is intractable since we do not have the analytical form of $\mathbf{K}^{-1}_\mathcal{UU}$ with respect to $\mathbf{\Lambda}$. 
%
	To resolve this, we introduce a reparameterization trick	
%novel augmented reparameterization of the above Bayesian SGPR model which interestingly 
to make the prior distribution of inducing outputs independent of the hyperparameters $\boldsymbol{\theta}$, as discussed next. %(Section~\ref{Seperation of Gaussian Process}).

%utilize the standardized representation of the GP model \cite{Titsias13} that involves a reparameterization to make the prior distribution of inducing outputs independent of $\mathbf{\Lambda}$, as discussed next.\vspace{-0.5mm}
	% and  are independent, thus allowing us to overcome the above intractability in computing Eq.~\eqref{E7}. This is discussed in the next section.
	%
% \vspace{2mm}
\section{Reparameterizing Bayesian SGPR Models}
\label{Seperation of Gaussian Process}%\vspace{2mm}
Let $\phi: \mathbb{R}^d \rightarrow \mathcal{H}$ denote a non-linear feature map from the input space $\mathbb{R}^d$ into a \emph{reproducing kernel Hilbert space} (RKHS) $\mathcal{H}$ whose inner product is defined as $\langle\phi(\mathbf{x}),\phi(\mathbf{x}')\rangle_\mathcal{H} \triangleq \exp(-0.5\|\mathbf{x} - \mathbf{x}'\|_2^2)$. Given $\phi$, the GP covariance/kernel function can be interpreted as $k_{\mathbf{xx'}} \triangleq \left\langle\sigma(\mathbf{x})\phi(\boldsymbol{\Lambda}\mathbf{x}),\sigma(\mathbf{x}')\phi(\mathbf{\Lambda}\mathbf{x}')\right\rangle_\mathcal{H} = \sigma(\mathbf{x})\sigma(\mathbf{x}')\exp(-0.5\|\mathbf{\Lambda x} - \mathbf{\Lambda x}'\|_2^2)$ where $\sigma$ is an arbitrary function mapping from $\mathbb{R}^d$ to $\mathbb{R}$. This implies $k_\mathbf{xx} = \sigma^2(\mathbf{x})$ which allows $\sigma^2(\mathbf{x})$ to be interpreted as the prior variance $k_\mathbf{xx}$ of $f_\mathbf{x}$ (Section~\ref{full}). 
%If we assume that the prior covariances of latent outputs $\{f_\mathbf{x}\}_{\mathbf{x} \in \mathbb{R}^d}$ are identical across the input space (that is, $\forall \mathbf{x} \in \mathbb{R}^d, \sigma^2(\mathbf{x}) = \sigma^2_f$) which, consequently, reproduces the kernel function in Section~\ref{full}.

%\vspace{1mm}	
We will now describe the reparameterization trick: 
Let $\mathcal{I}\triangleq\{\mathbf{\Lambda}\mathbf{x}\}_{\mathbf{x} \in \mathcal{U}}$.
%; recall that $\mathcal{U} \subset \mathcal{X}$ is a small set of inducing inputs.
% such that $|\mathcal{U}| \ll |\mathcal{D}|$. 
Intuitively, $\mathcal{I}$ can be interpreted as a set of \emph{rotated inducing inputs} with the diagonal matrix $\mathbf{\Lambda}$ of inverted length-scales being the rotation matrix. 
%We will now discuss how $\mathcal{I}$ (instead of $\mathcal{U}$)
%
%As shall be seen in this section, by exploiting this small set $\mathcal{I}$ of \emph{rotated inducing inputs} instead of its unorthodox counterpart $\mathcal{U}$, we can interestingly eliminate the dependency on $\boldsymbol{\Lambda}$ of the inducing covariance matrix, which is crucial in resolving the intractability issue of the original Bayesian SGPR model in Section~\ref{Bayesian SGP}. 
%
%To understand this, 
Let each rotated inducing input $\mathbf{z} \in \mathcal{I}$ be associated with a latent output variable $s_{\mathbf{z}}$. 
%whose prior variance is assumed identical and equal to $\nu^2 > 0$.
Then, for all $\mathbf{z},\mathbf{z}' \in \mathcal{I}$,  
%such that \textcolor{black}{$\mathbf{z} = \mathbf{\Lambda\mathbf{x}}$} and \textcolor{black}{$\mathbf{z}' = \mathbf{\Lambda \mathbf{x}^\prime}$}, 
$\text{cov}[s_{\mathbf{z}}, s_{\mathbf{z}'}] \triangleq \langle \sigma(\mathbf{z})\phi(\mathbf{z}), \sigma(\mathbf{z}^\prime)\phi(\mathbf{z}^\prime) \rangle_{\mathcal{H}} = \sigma(\mathbf{z})\sigma(\mathbf{z}')\exp(-0.5\|\mathbf{z} - \mathbf{z}'\|_2^2)$, by definition of RKHS. 
%It then follows, by definition, that $\forall \mathbf{z},\mathbf{z}' \in \mathcal{I}$ such that \textcolor{red}{$\mathbf{z} = \mathbf{\Lambda\mathbf{x}}$} and \textcolor{red}{$\mathbf{z}' = \mathbf{\Lambda \mathbf{x}^\prime}$, $k_{\mathbf{z}\mathbf{z}'} = \sigma(\mathbf{z})\sigma(\mathbf{z}')\mathrm{exp}\left(-0.5\|\mathbf{z} - \mathbf{z}'\|_2^2\right)$}. 
By assuming that the prior variances of $s_\mathbf{z}$ for all $\mathbf{z}\in\mathcal{I}$ are identical 
%that the prior variances of $\{s_\mathbf{z}\}_{\mathbf{z}\in\mathcal{I}}$ are identical 
and equal to some constant $\zeta^2$ (i.e., $\sigma(\mathbf{z}) = \zeta > 0$), \textcolor{black}{$\text{cov}[s_{\mathbf{z}}, s_{\mathbf{z}'}] = \zeta^2\exp(-0.5\|\mathbf{z} - \mathbf{z}'\|_2^2)$} which is independent of $\boldsymbol{\theta}$. Consequently, the prior covariance matrix $\mathbf{\Sigma}_\mathcal{II} \triangleq (\text{cov}[s_{\mathbf{z}}, s_{\mathbf{z}'}])_{\mathbf{z},\mathbf{z}' \in \mathcal{I}}$ of the inducing output variables $\mathbf{s}_\mathcal{I} \triangleq (s_{\mathbf{z}})^\top_{\mathbf{z}\in\mathcal{I}}$ is independent of $\boldsymbol{\theta}$. %According to the definition of RKHS, we can rewrite $k_{\mathbf{z}\mathbf{z}'} = \langle \sigma(\mathbf{z})\phi(\mathbf{z}), \sigma(\mathbf{z}^\prime)\phi(\mathbf{z}^\prime) \rangle_{\mathcal{H}}$. 
%
% denote the latent outputs evaluated at $\mathbf{z} \in \mathcal{I}$. 
%
% Specifically, the joint distribution between latent outputs $\mathbf{f}_{\mathcal{D}}$ and inducing variables $\mathbf{s}_{\mathcal{I}}$ can be written as:
%\vspace{1mm}
Furthermore, the cross-covariance matrix $\mathbf{K}_{\mathcal{D}\mathcal{I}} \triangleq (\text{cov}[f_{\mathbf{x}}, s_{\mathbf{z}}])_{\mathbf{x} \in \mathcal{D},\mathbf{z} \in \mathcal{I}}$ between the latent outputs $\mathbf{f}_\mathcal{D}$ for some set $\mathcal{D}$ of training inputs and the inducing outputs $\mathbf{s}_\mathcal{I}$ can be computed analytically using 
%are also analytically tractable following 
the definition of RKHS: $\text{cov}[f_{\mathbf{x}}, s_{\mathbf{z}}] = \langle \sigma(\mathbf{x})\phi(\mathbf{\Lambda}\mathbf{x}), \sigma(\mathbf{z})\phi(\mathbf{z})\rangle_{\mathcal{H}} = \zeta\sigma(\mathbf{x})\exp(-0.5\|\mathbf{\Lambda}\mathbf{x}-\mathbf{z}\|^2_2)$. 
%following from our kernel definition. 
% and independent of $\boldsymbol{\Lambda}$ since $k_{\mathbf{x}\mathbf{z}} = \nu\sigma(\mathbf{x})\exp(-0.5\|\mathbf{\Lambda}\mathbf{x}-\mathbf{u}\|^2_2)$ which trivially follows from our kernel definition. 
Like many existing GP models, the prior variances of 
%latent outputs
$f_\mathbf{x}$ for all $\mathbf{x}\in\mathcal{X}$
% $\{f_\mathbf{x}\}_{\mathbf{x} \in \mathbb{R}^d\setminus\mathcal{I}}$ 
are assumed to be identical and equal to a signal variance hyperparameter $\sigma_f^2$ (i.e., $\sigma(\mathbf{x}) = \sigma_f$) for tractable learning, hence circumventing the need to learn 
%This is necessary to tractably learn the prior variance of $\{f_\mathbf{x}\}_{\mathbf{x}\in \mathcal{D}}$ because otherwise, we would have to learn 
an infinite number of prior variance hyperparameters.
% which is impossible. 
The resulting representation of the GP model from the reparameterization trick will allow the optimal variational distributions of inducing outputs $\mathbf{s}_\mathcal{I}$ and hyperparameters $\boldsymbol{\theta}$ (hence the predictive distribution) to be tractably derived for a family of VBSGPR models, as discussed in Section~\ref{Variational Inference of the Bayesian DTC}.\vspace{1mm}

%\noindent	
\emph{Remark 1:} The definition of $\mathcal{I}$ seems to suggest its construction by first selecting the inducing inputs $\mathcal{U}$ and then rotating them via $\mathbf{\Lambda}$, which is not possible since $\mathbf{\Lambda}$ is not known \emph{a priori}.
However, as shall be discussed in Section~\ref{Variational Inference of the Bayesian DTC},
it is possible to first select $\mathcal{I}$ and then optimize the variational distribution of $\mathbf{\Lambda}$, which has an effect of optimizing the distribution of inducing inputs $\mathcal{U}$ in original input space $\mathcal{X}$.\vspace{1mm}

%In practice, $\mathcal{U}$ cannot be selected beforehand for constructing $\mathcal{I}$


%we cannot, however, select in advance a subset of inducing inputs $\mathcal{U}$ and construct $\mathcal{I} = \{\mathbf{z} = \mathbf{\Lambda}\mathbf{x} | \mathbf{x} \in \mathcal{U}\}$ following our definition because we do not know $\mathbf{\Lambda}$ a priori. 
%Instead, we have to select $\mathcal{I}$ directly to learn the inverse length-scales $\mathbf{\Lambda}$ first which then dictates the inverse mapping to the set of unorthodox inducing inputs. This formulation, perhaps surprisingly, appears to have an advantage over the other existing inducing methods in the GP literature as it can implicitly optimize the distribution of inducing inputs as an effect of optimizing the variational distribution of $\mathbf{\Lambda}$, as detailed in Section~\ref{Variational Inference of the Bayesian DTC}.

%\noindent
\emph{Remark 2:} Let $\mathcal{Z}\triangleq\{\mathbf{\Lambda}\mathbf{x}\}_{\mathbf{x} \in \mathcal{X}}$.
%Let $\mathcal{Z}$ denote a $d$-dimensional input feature space such that $\mathcal{I} \subset\mathcal{Z}$.
By setting the (identical) prior variances of $s_\mathbf{z}$ for all $\mathbf{z}\in\mathcal{Z}$ to unity
%When the (identical) prior variance of our \emph{rotated inducing outputs} is set to unity 
(i.e., $\zeta = 1$), $\{s_\mathbf{z}\}_{\mathbf{z}\in\mathcal{Z}}$ denote a \emph{standard} GP with unit signal variance and length-scales~\cite{Titsias13},
which is a special case of our representation of the GP model here.
Then, $f_\mathbf{x} = \sigma_f s_{\mathbf{\Lambda x}}$ for all $\mathbf{x} \in \mathcal{X}$.
%
%our work interestingly induces a deterministic relationship between our GP-distributed random function $f_\mathbf{x}$ and another random function $g_\mathbf{x}$ distributed by a \emph{normalized} GP assuming that $\mathbf{\Lambda}$ and $\sigma_f$ are fixed. That is, for every $\mathbf{x} \in \mathbb{R}^d$, we have $f_\mathbf{x} = \sigma_f g_{\mathbf{\Lambda x}}$. This is known as the standard GP model \cite{Titsias13} which is a special case of our work. 
%
% In Section~\ref{Experiments and Discussion}, we demonstrate performances with respect to various $\nu$ values.  
%
\section{Variational Bayesian SGPR Models}
\label{Variational Inference of the Bayesian DTC}%\vspace{-0.5mm}
Using our representation
%the \emph{rotated inducing representation} 
of the GP model defined above (Section~\ref{Seperation of Gaussian Process}), the predictive distribution~\eqref{E7} of a Bayesian SGPR model can be slightly modified to $p(f_{\mathbf{x}^*}|\mathbf{y}_\mathcal{D})=\int p(f_{\mathbf{x}^*}|\mathbf{y}_{\mathcal{D}_B},\mathbf{s}_\mathcal{I},\boldsymbol{\theta})\ p(\mathbf{s}_\mathcal{I},\boldsymbol{\theta}|\mathbf{y}_\mathcal{D})\ \mathrm{d}\mathbf{s}_\mathcal{I}\ \mathrm{d}\boldsymbol{\theta}$ such that deriving the posterior $p(\mathbf{s}_\mathcal{I},\boldsymbol{\theta}|\mathbf{y}_\mathcal{D})=p(\mathbf{y}_\mathcal{D},\mathbf{s}_\mathcal{I},\boldsymbol{\theta})/p(\mathbf{y}_\mathcal{D})$ requires computing the likelihood:\vspace{-0.6mm}
\begin{equation}	
\hspace{0mm}
\begin{array}{c}
p(\mathbf{y}_\mathcal{D}) \hspace{-0mm}=\hspace{-0mm} \displaystyle\mathbb{E}_{\boldsymbol{\theta}}\hspace{-0mm}\left[\int \hspace{-0mm}p(\mathbf{y}_\mathcal{D}|\mathbf{f}_\mathcal{D}) p(\mathbf{f}_\mathcal{D}|\mathbf{s}_\mathcal{I},\boldsymbol{\theta}) p(\mathbf{s}_\mathcal{I})\mathrm{d}\mathbf{f}_\mathcal{D}\mathrm{d}\mathbf{s}_\mathcal{I}\right]\vspace{-0mm}
\end{array}
\label{E13}\vspace{-0.6mm}
\end{equation} 	
\textcolor{black}{where $p(\boldsymbol{\theta}) \triangleq \mathcal{N}(\mathbf{1}, \text{diag}[0.1])$, $p(\mathbf{s}_\mathcal{I})=\mathcal{N}(\mathbf{0},\mathbf{\Sigma}_\mathcal{II})$, $p(\mathbf{y}_\mathcal{D}|\mathbf{f}_\mathcal{D})=\mathcal{N}(\mathbf{f}_\mathcal{D},\mathbf{C}_{\mathcal{DD}})$}, and\vspace{-0.0mm}
\begin{equation}
\hspace{0mm}p(\mathbf{f}_\mathcal{D}|\mathbf{s}_\mathcal{I},\boldsymbol{\theta})=\displaystyle\mathcal{N}(\mathbf{K}_\mathcal{DI}\mathbf{\Sigma}_\mathcal{II}^{-1}\mathbf{s}_\mathcal{I},\mathbf{K}_\mathcal{DD} -\mathbf{K}_\mathcal{DI}\mathbf{\Sigma}_\mathcal{II}^{-1}\mathbf{K}_\mathcal{ID})
\label{Modified Gaussian Process Model}\vspace{-0.0mm}
\end{equation}
such that $\mathbf{K}_\mathcal{DI}$ is previously defined in Section~\ref{Seperation of Gaussian Process} and
$\mathbf{K}_\mathcal{ID} = \mathbf{K}^{\top}_\mathcal{DI}$. However, the integration in~\eqref{E13} (and hence $p(\mathbf{s}_\mathcal{I},\boldsymbol{\theta}|\mathbf{y}_\mathcal{D})$) cannot be evaluated in closed form. To resolve this, instead of using exact inference, we adopt variational inference to approximate the posterior distribution $p(\mathbf{f}_\mathcal{D},\mathbf{s}_\mathcal{I},\boldsymbol{\theta}|\mathbf{y}_\mathcal{D})=p(\mathbf{f}_\mathcal{D}|\mathbf{s}_\mathcal{I},\boldsymbol{\theta})\ p(\mathbf{s}_\mathcal{I},\boldsymbol{\theta}|\mathbf{y}_\mathcal{D})$ with a factorized variational distribution %\vspace{-2mm}
%	\begin{equation*}
		$q(\mathbf{f}_\mathcal{D},\mathbf{s}_\mathcal{I},\boldsymbol{\theta})\triangleq p(\mathbf{f}_\mathcal{D}|\mathbf{s}_\mathcal{I},\boldsymbol{\theta})\ q(\mathbf{s}_\mathcal{I})\ q(\boldsymbol{\theta})$ %\vspace{-2mm}
%	\end{equation*} 
where $p(\mathbf{f}_\mathcal{D}|\mathbf{s}_\mathcal{I},\boldsymbol{\theta})$ is the exact training conditional~\eqref{Modified Gaussian Process Model}, $	q(\mathbf{s}_\mathcal{I})\triangleq\mathcal{N}(\mathbf{m},\mathbf{S})$, 
$q(\boldsymbol{\theta}) \triangleq q(\mathbf{\Lambda})\ q(\sigma_f)$, 
$q(\mathbf{\Lambda})\triangleq\prod_{i=1}^{d}\mathcal{N}(\lambda_i|\nu_i,\xi_i)$ with $\boldsymbol{\nu} \triangleq (\nu_1,\ldots,\nu_d)^\top$ and $\mathbf{\Xi} \triangleq \mathrm{diag}[\xi_1,\ldots,\xi_d]$, and $q(\sigma_f)\triangleq\mathcal{N}(\alpha, \beta)$. Then, the log-marginal likelihood $\log p(\mathbf{y}_\mathcal{D})$ can be decomposed into a sum of its variational lower bound $\mathcal{L}(q)$ and the nonnegative KL distance between the variational distribution $q(\mathbf{f}_\mathcal{D},\mathbf{s}_\mathcal{I},\boldsymbol{\theta})$ and the posterior distribution $p(\mathbf{f}_\mathcal{D},\mathbf{s}_\mathcal{I},\boldsymbol{\theta}|\mathbf{y}_\mathcal{D})$, the latter of which quantifies the gap between $\log p(\mathbf{y}_\mathcal{D})$ and $\mathcal{L}(q)$, that is, 
\begin{equation}
\hspace{0mm}\log p(\mathbf{y}_\mathcal{D})\hspace{-0mm}=\hspace{-0mm}\mathcal{L}(q)+\mathrm{KL}(q(\mathbf{f}_\mathcal{D},\mathbf{s}_\mathcal{I},\boldsymbol{\theta})|| p(\mathbf{f}_\mathcal{D},\mathbf{s}_\mathcal{I},\boldsymbol{\theta}|\mathbf{y}_\mathcal{D})), \label{log likelihood}\vspace{-0mm}
\end{equation}
as derived in Appendix~\ref{Derivation of Equation (15)} of~\cite{HaibinAPP} where\vspace{-0.6mm} 
\begin{equation}
\hspace{-0mm}\mathcal{L}(q) \triangleq\hspace{-0mm} \int \hspace{-0mm}q(\mathbf{f}_\mathcal{D},\mathbf{s}_\mathcal{I},\boldsymbol{\theta})\log\frac{p(\mathbf{y}_\mathcal{D},\mathbf{f}_\mathcal{D},\mathbf{s}_\mathcal{I},\boldsymbol{\theta})}{q(\mathbf{f}_\mathcal{D},\mathbf{s}_\mathcal{I},\boldsymbol{\theta})}\ \mathrm{d}\mathbf{f}_\mathcal{D}\ \mathrm{d}\mathbf{s}_\mathcal{I}\ \mathrm{d}\boldsymbol{\theta}.\vspace{-0.6mm}
\end{equation}

\emph{Remark 3:} The likelihood term $p(\mathbf{y}_{\mathcal{D}})$~\eqref{E13} in~\eqref{log likelihood} is a constant with respect to $q(\mathbf{s}_\mathcal{I})$ and $q(\boldsymbol{\theta})$ (specifically, their parameters $\mathbf{m},\mathbf{S},\boldsymbol{\nu},\mathbf{\Xi},\alpha, \beta$). Consequently, maximizing $\mathcal{L}(q)$ with respect to $q(\mathbf{f}_\mathcal{D},\mathbf{s}_\mathcal{I},\boldsymbol{\theta})$ is equivalent to minimizing $\mathrm{KL}(q(\mathbf{f}_\mathcal{D},\mathbf{s}_\mathcal{I},\boldsymbol{\theta})|| p(\mathbf{f}_\mathcal{D},\mathbf{s}_\mathcal{I},\boldsymbol{\theta}|\mathbf{y}_\mathcal{D}))$. This equivalence, however, does not hold for existing variational SGPR models and their stochastic and distributed variants optimizing point estimates of all hyperparameters, as discussed in Section~\ref{sect:intro}.\vspace{1mm}

The variational inference framework of \cite{Titsias13} is similar in spirit to the above. However, the framework of \cite{Titsias13} assumes i.i.d. observation noises (i.e., $\mathbf{C}_{\mathcal{DD}} = \sigma_n^2\mathbf{I}$ and $\zeta = 1$) and ignores their correlation, 
which consequently yields the VBDTC approximation (see Remark $4$ later).
The challenge remains in investigating whether the other more refined SGPR models spanned by the unifying view of~\cite{candela05} (e.g., FITC, PITC, PIC) are amenable to such a variational Bayesian treatment since they have been empirically demonstrated~\cite{NghiaICML15,HoangICML16} to give better predictive performance than DTC.
%	
%This is similar in spirit to the previous work of \cite{Titsias13} which, however, 
% This consequently yields a variational Bayesian variant of DTC~\cite{Seeger03} (VBDTC) which completely ignores the correlation structure of observation noises. 
% 
%In contrast, our proposed variational framework is capable of relaxing 

To address this challenge, our key idea is to relax the strong assumption of i.i.d. observation noises with constant variance $\sigma^2_n$ imposed by VBDTC and allow observation noises to be correlated with some structure across the input space, hence being robust to different noise correlation structures and in turn catering to more realistic applications of GP. 
%More interestingly, as detailed in the remaining of this section, our work reveals that by relaxing this i.i.d. observation noise assumption of VBDTC, the other more refined SGPR models spanned by the unifying view of \cite{candela05} (e.g., FITC, PITC, PIC) are also amenable to a similar variational Bayesian treatment. This is important since these refined SGPR models have been empirically demonstrated \cite{NghiaICML15,HoangICML16} to give better predictive performance than DTC. 
Interestingly, this results in a noise-robust family of \emph{variational Bayesian SGPR} (VBSGPR) models (e.g., VBDTC, VBFITC, VBPIC), which we will describe below.

Let $\mathbf{C}_{\mathcal{DD}}\triangleq\mathrm{blkdiag}[\mathbf{K}^{\varepsilon}_{\mathcal{D}\mathcal{D}}-\mathbf{K}^{\varepsilon}_{\mathcal{D}\mathcal{U}}\mathbf{K}^{\varepsilon^{-1}}_{\mathcal{U}\mathcal{U}}\mathbf{K}^{\varepsilon}_{\mathcal{U}\mathcal{D}}]+\sigma^2_n\mathbf{I}$ be a block-diagonal noise covariance matrix constructed from the $B$ diagonal blocks of $\mathbf{K}^{\varepsilon}_{\mathcal{D}\mathcal{D}}-\mathbf{K}^{\varepsilon}_{\mathcal{D}\mathcal{U}}\mathbf{K}^{\varepsilon^{-1}}_{\mathcal{U}\mathcal{U}}\mathbf{K}^{\varepsilon}_{\mathcal{U}\mathcal{D}}+\sigma^2_n\mathbf{I}$, each of which is a matrix $\mathbf{C}_{\mathcal{D}_i\mathcal{D}_i}\triangleq\mathbf{K}^{\varepsilon}_{\mathcal{D}_i\mathcal{D}_i}-\mathbf{K}^{\varepsilon}_{\mathcal{D}_i\mathcal{U}}\mathbf{K}^{\varepsilon^{-1}}_{\mathcal{U}\mathcal{U}}\mathbf{K}^{\varepsilon}_{\mathcal{U}\mathcal{D}_i}+\sigma^2_n\mathbf{I}$ for $i=1,\ldots,B$, and $\mathbf{K}^{\varepsilon}_{\mathcal{D}\mathcal{D}} \triangleq (k^{\varepsilon}_{\mathbf{x}\mathbf{x}'})_{\mathbf{x},\mathbf{x}' \in \mathcal{D}}$, $\mathbf{K}^{\varepsilon}_{\mathcal{D}\mathcal{U}} \triangleq (k^{\varepsilon}_{\mathbf{x}\mathbf{x}'})_{\mathbf{x} \in \mathcal{D},\mathbf{x}' \in \mathcal{U}}$, $\mathbf{K}^{\varepsilon}_{\mathcal{U}\mathcal{U}}\triangleq (k^{\varepsilon}_{\mathbf{x}\mathbf{x}'})_{\mathbf{x},\mathbf{x}' \in \mathcal{U}}$, and $\mathbf{K}^{\varepsilon}_{\mathcal{U}\mathcal{D}} \triangleq \mathbf{K}^{\varepsilon^\top}_{\mathcal{D}\mathcal{U}}$ are matrices with components $k^{\varepsilon}_{\mathbf{x}\mathbf{x}'}$ defined by a covariance function similar to that used for $k_{\mathbf{x}\mathbf{x}'}$ (Section~\ref{full}) but with different hyperparameter values\footnote{We do not assign any prior over the hyperparameters of $k^{\varepsilon}_{\mathbf{x}\mathbf{x}'}$ and the noise variance $\sigma^2_n$. 
Instead, they are treated as parameters optimized to maximize $\mathcal{L}(q)$ via stochastic gradient ascent \cite{Lawrence13}. 
In our experiments, we observe that even if we set the hyperparameters of $k^{\varepsilon}_{\mathbf{x}\mathbf{x}'}$ by hand, the predictive performance does not vary much and our VBPIC approximation can significantly outperform the state-of-the-art variational SGPR models and their stochastic and distributed variants. A Bayesian treatment of these hyperparameters is highly non-trivial due to a complication similar to that discussed in Section~\ref{Bayesian SGP} 
%under `Bayesian SGPR Models' 
and will be investigated in our future work.}. Our first major result ensues:\vspace{1mm}
%			
\begin{theorem}
$\mathcal{L}(q)$ in~\eqref{log likelihood} can be analytically evaluated as \vspace{-0mm} 
%(\mathbf{\Sigma}_\mathcal{II}^{-1}\mathbf{\Psi}_\mathcal{II}\mathbf{\Sigma}_\mathcal{II}^{-1}\hspace{-0.7mm}+\hspace{-0.7mm}\mathbf{\Sigma}_\mathcal{II}^{-1})
\begin{equation}
%\vspace{-0mm}
\hspace{-0mm}
\begin{array}{rcl}
\mathcal{L}(q) &\hspace{-2.4mm}= &\hspace{-2.4mm}\displaystyle\frac{1}{2}\Big(2\mathbf{m}^{\hspace{-0.5mm}\top}\hspace{-0.5mm}\mathbf{\Sigma}_\mathcal{II}^{-1}\mathbf{\Omega}_\mathcal{ID}\mathbf{C}_\mathcal{DD}^{-1}\mathbf{y}_\mathcal{D}\hspace{-0.5mm}-\hspace{-0.5mm}\mathbf{m}^{\hspace{-0.5mm}\top}\hspace{-0.5mm}\mathbf{Q}\mathbf{m} -\mathrm{Tr}[\mathbf{S}\mathbf{Q}]\vspace{0.5mm}
\\
%\hspace{10mm}
&&\hspace{-2.4mm}-\hspace{0mm}\mathrm{Tr}[\mathbf{C}_\mathcal{DD}^{-1}\mathbf{\Upsilon}_\mathcal{DD}]\hspace{-0.7mm}+\hspace{-0.7mm}\mathrm{Tr}[\mathbf{\Sigma}_\mathcal{II}^{-1}\mathbf{\Psi}_\mathcal{II}] +\hspace{-0.5mm}\log|\mathbf{S}|\vspace{0.5mm}
\\
&&\hspace{-2.4mm}-\boldsymbol{\nu}^\top\boldsymbol{\nu}\hspace{-0.5mm}-\hspace{-0.5mm}\mathrm{Tr}[\mathbf{\Xi}]\hspace{-0.5mm}+\hspace{-0.5mm}\log|\mathbf{\Xi}|\hspace{-0.5mm}-\hspace{-0.5mm}\alpha^2\hspace{-0.5mm} -\hspace{-0.5mm} \beta\hspace{-0.5mm} +\hspace{-0.5mm} \log\beta\Big)\hspace{-1mm}+\hspace{-0.5mm}\mathrm{const}\vspace{0mm} 
\vspace{0mm}
\end{array}\hspace{-4.8mm}
\label{L(q)}
\end{equation} 
where $\mathbf{Q} \triangleq \mathbf{\Sigma}_\mathcal{II}^{-1}\mathbf{\Psi}_\mathcal{II}\mathbf{\Sigma}_\mathcal{II}^{-1}\hspace{-0mm}+\hspace{-0mm}\mathbf{\Sigma}_\mathcal{II}^{-1}$. More interestingly, using the above expression, it can be shown that $\mathcal{L}(q)$ is maximized at $q^*(\mathbf{s}_\mathcal{I})=\mathcal{N}(\mathbf{m}^*,\mathbf{S}^*)$ where
\begin{equation}
\begin{array}{rcl}
\mathbf{m}^*&\hspace{-2.4mm}\triangleq&\hspace{-2.4mm}\displaystyle\mathbf{\Sigma}_\mathcal{II}(\mathbf{\Sigma}_\mathcal{II}+\mathbf{\Psi}_\mathcal{II})^{-1}\mathbf{\Omega}_\mathcal{ID}\mathbf{C}^{-1}_\mathcal{DD}\mathbf{y}_\mathcal{D}\ ,\vspace{0.5mm}\\
\mathbf{S}^*&\hspace{-2.4mm}\triangleq&\hspace{-2.4mm}\displaystyle\mathbf{\Sigma}_\mathcal{II}(\mathbf{\Sigma}_\mathcal{II}+\mathbf{\Psi}_\mathcal{II})^{-1}\mathbf{\Sigma}_\mathcal{II}
\end{array}
\label{q(u)}\vspace{-0mm}
\end{equation} 
such that $\mathbf{\Omega}_\mathcal{ID}\triangleq\mathbb{E}_{q(\boldsymbol{\theta})}[\mathbf{K}_{\mathcal{ID}}]$, $\mathbf{\Upsilon}_\mathcal{DD}\triangleq\mathbb{E}_{q(\boldsymbol{\theta})}[\mathbf{K}_{\mathcal{DD}}]$, $\mathbf{\Psi}_\mathcal{II}\triangleq\mathbb{E}_{q(\boldsymbol{\theta})}[\mathbf{K}_{\mathcal{ID}}\mathbf{C}^{-1}_\mathcal{DD}\mathbf{K}_{\mathcal{DI}}]$, 
%$q(\mathbf{\Lambda},{\sigma_f})\triangleq q(\mathbf{\Lambda})\ q({\sigma_f})$, 
and $\mathrm{const}$ absorbs all terms indep. of $\mathbf{m}$, $\mathbf{S}$, $\boldsymbol{\nu}$, $\mathbf{\Xi}, \alpha, \beta$.\vspace{1mm}
\label{thm1}
\end{theorem}
Its proof is in Appendix~\ref{Derivation of the Lower Bound} of~\cite{HaibinAPP}. Appendix~\ref{Derivation of Omega, Psi and Upsilon} of~\cite{HaibinAPP} gives the closed-form expressions of $\mathbf{\Omega}_\mathcal{ID}$, $\mathbf{\Upsilon}_\mathcal{DD}$, and $\mathbf{\Psi}_\mathcal{II}$.\vspace{1mm}

\emph{Remark 4:} Note that $q^*(\mathbf{s}_\mathcal{I})$ in Theorem \ref{thm1} closely resembles that of PIC and PITC (see eqs.~$64$ and~$65$ in Appendix D.$1$.$1$ of~\cite{NghiaICML15}) except for the expectation over hyperparameters $\boldsymbol{\theta}$ due to the variational Bayesian treatment. So, we call them VBPIC and VBPITC, respectively. By setting $B = |\mathcal{D}|$, $\mathbf{C}_{\mathcal{DD}}$ becomes a diagonal matrix to give VBFIC and VBFITC. When $\mathbf{C}_{\mathcal{DD}}=\sigma^2_n\mathbf{I}$, $q^*(\mathbf{s}_\mathcal{I})$~\eqref{q(u)} resembles that of DTC (see eqs.~$68$ and~$69$ in Appendix D.$1$.$3$ of \cite{NghiaICML15}) except for the expectation over $\boldsymbol{\theta}$ due to the variational Bayesian treatment and coincides with that in Appendix B.$1$ of \cite{Titsias13}. So, we refer to it as VBDTC.\vspace{1mm} 

\emph{Remark 5:} In the non-Bayesian setting of the hyperparameters, it has been previously established that the predictive distribution of FITC can be reproduced as a direct result of applying either variational inference \cite{Titsias09a} with $\mathbf{C}_{\mathcal{DD}}=\mathrm{diag}[\mathbf{K}_{\mathcal{D}\mathcal{D}}-\mathbf{K}_{\mathcal{D}\mathcal{U}}\mathbf{K}^{-1}_{\mathcal{U}\mathcal{U}}\mathbf{K}_{\mathcal{U}\mathcal{D}}]+\sigma^2_n\mathbf{I}$ or expectation propagation \cite{Matthias2016Understanding} on the FGPR model. Our derivation of VBFITC is in fact similar in spirit to that of \cite{Titsias09a} except for our variational Bayesian treatment of its hyperparameters. On the other hand, it is unclear whether FITC's equivalent EP derivation in \cite{Matthias2016Understanding} can be easily extended to incorporate a Bayesian treatment of its hyperparameters.%\vspace{-2mm}
%
%When $\mathbf{C}_{\mathcal{DD}}=\sigma^2_n\mathbf{I}$, \citet{Matthias2016Understanding} have showed that DTC can be derived based on variational inference method while FITC can be derived based on expectation propagation (EP). However, by varying the structure of $\mathbf{C}_{\mathcal{DD}}$ with correlated noise structure, this work results in a family of VBSGPR models with the variational inference method.
%	
\section{Stochastic Optimization}	
%	\textcolor{red}{I eliminate the section Stochastic Variational Inference section, too trivial to discuss about the Robbins and Monro Methods.}
	%\label{section3}
	\label{Stochastic Variational Inference for GPR}%\vspace{-1mm}
	The VBDTC approximation~\cite{Titsias13} has explicitly plugged the optimal $q^*(\mathbf{s}_\mathcal{I})$ (see Theorem~\ref{thm1}) into $\mathcal{L}(q)$~\eqref{L(q)} and reduced it to $\mathcal{L}(q)$~\eqref{eq:TitsiasL(q)} in Appendix~\ref{Derivation of the Lower Bound} of~\cite{HaibinAPP}. Given $\mathcal{L}(q)$~\eqref{eq:TitsiasL(q)}, the parameters $\boldsymbol{\nu}$ and $\mathbf{\Xi}$ of $q(\mathbf{\Lambda})$ and $\alpha$ and $\beta$ of $q(\sigma_f)$ can be optimized via gradient ascent. However, evaluating the exact gradients
	$\partial\mathcal{L}/\partial\boldsymbol{\nu}$, $\partial\mathcal{L}/\partial\mathbf{\Xi}$, $\partial\mathcal{L}/\partial \alpha$ and $\partial\mathcal{L}/\partial \beta$  
	% with respect to all of the variables requires the whole dataset 
	incur $\mathcal{O}(|\mathcal{D}||\mathcal{I}|^2)$ time,
	which scales poorly in the data size $|\mathcal{D}|$. To overcome the above issue of scalability, we utilize stochastic gradient ascent updates instead of exact ones, which requires the stochastic gradients to be unbiased estimators of the exact gradients to guarantee convergence.
	The key idea is to iteratively compute the stochastic gradients by randomly sampling a single or few mini-batches of data from the dataset (i.e., comprising $B$ disjoint mini-batches) whose incurred time per iteration is independent of data size $|\mathcal{D}|$.
	To achieve this, an important requirement is the decomposability of $\mathcal{L}(q)$~\eqref{eq:TitsiasL(q)} into a summation of $B$ terms, each of which is associated with a mini-batch $(\mathcal{D}_i, \mathbf{y}_{\mathcal{D}_i})$ of data of size $|\mathcal{D}_i|=\mathcal{O}(|\mathcal{I}|)$	
%(i.e., comprising a disjoint subset $\mathcal{D}_i$ of training inputs and its corresponding noisy observed outputs $\mathbf{y}_{\mathcal{D}_i}$) 
that can be exploited for computing the stochastic gradients.
	%used to perform stochastic gradient ascent with respect to $\mathbf{m}$, $\mathbf{S}$, $\boldsymbol{\nu}$, and $\mathbf{\Xi}$. 
	Unfortunately, $\mathcal{L}(q)$~\eqref{eq:TitsiasL(q)} is not decomposable due to its $(\mathbf{\Sigma}_\mathcal{II}+\mathbf{\Psi}_\mathcal{II})^{-1}$ term.			
	%		Due to the item  $(\mathbf{\xi}_\mathcal{II}+\mathbf{\Psi}_\mathcal{II})^{-1}$ inside the lower bound $\mathcal{L}$, $\mathcal{L}$ can not be decomposed into summation of the observations which incurs to use the whole dataset to determine the variational parameters in the model; thus makes it unable to apply the stochastic variational inference (SVI) which is suitable for massive volume data sets. 
	To remedy this, we do not  plug $q^*(\mathbf{s}_\mathcal{I})$~\eqref{q(u)} into $\mathcal{L}(q)$~\eqref{L(q)} to yield~\eqref{eq:TitsiasL(q)} but
	instead jointly treat $q(\mathbf{s}_\mathcal{I})$, $q(\mathbf{\Lambda})$, and $q(\sigma_f)$ as variational parameters, which enables the decomposability of $\mathcal{L}(q)$~\eqref{L(q)}:\vspace{1mm}
%
\begin{corollary} $\mathcal{L}(q)$~\eqref{L(q)} (Theorem~\ref{thm1}) can be decomposed into\vspace{-0mm}	
	\begin{equation*}
		\hspace{-1.7mm}
		\begin{array}{l}
			\mathcal{L}(q)=\displaystyle\sum_{i=1}^B\mathcal{L}_i(q) \hspace{-0.5mm}+\hspace{-0.5mm} \frac{1}{2}\Big( \hspace{-1mm}-\hspace{-0.5mm}\mathbf{m}^\top\mathbf{\Sigma}_\mathcal{II}^{-1}\mathbf{m}\hspace{-0.5mm}- \hspace{-0.5mm}\mathrm{Tr}[\mathbf{S}\mathbf{\Sigma}_\mathcal{II}^{-1}]\hspace{-0.5mm}+\hspace{-0.5mm}\log|\mathbf{S}| \vspace{-0.5mm}\\
			 \quad\qquad\ \displaystyle-\boldsymbol{\nu}^\top\boldsymbol{\nu}\hspace{-0.5mm} -\hspace{-0.5mm}\mathrm{Tr}[\mathbf{\Xi}]\hspace{-0.5mm}+\hspace{-0.5mm}\log|\mathbf{\Xi}|\hspace{-0.5mm}-\hspace{-0.5mm}\alpha^2 \hspace{-0.5mm}-\hspace{-0.5mm} \beta \hspace{-0.5mm}+\hspace{-0.5mm} \log \beta\Big)\hspace{-1mm} +\hspace{-0.5mm}\mathrm{const}\ ,\\
			\mathcal{L}_i(q)\hspace{-0.5mm}\triangleq\hspace{-0.5mm}\displaystyle \frac{1}{2}\Big(2\mathbf{m}^{\hspace{-0.5mm}\top}\mathbf{\Sigma}_\mathcal{II}^{-1}\mathbf{\Omega}_{\mathcal{I}\mathcal{D}_i}\mathbf{C}_{\mathcal{D}_i\mathcal{D}_i}^{-1}\mathbf{y}_{\mathcal{D}_i}\hspace{-1mm}-\hspace{-0.5mm}\mathbf{m}^{\hspace{-0.5mm}\top}\mathbf{\Sigma}_\mathcal{II}^{-1}\mathbf{\Psi}_\mathcal{II}^{i}\mathbf{\Sigma}_\mathcal{II}^{-1}\mathbf{m}\\
			\quad\ \displaystyle -\mathrm{Tr}[\mathbf{S}\mathbf{\Sigma}_\mathcal{II}^{-1}\mathbf{\Psi}_\mathcal{II}^{i}\mathbf{\Sigma}_\mathcal{II}^{-1}]\hspace{-0.5mm}-\hspace{-0.5mm}\mathrm{Tr}[\mathbf{C}_{\mathcal{D}_i\mathcal{D}_i}^{-1}\mathbf{\Upsilon}_{\mathcal{D}_i\mathcal{D}_i}] \hspace{-0.5mm}+\hspace{-0.5mm}\mathrm{Tr}[\mathbf{\Sigma}_\mathcal{II}^{-1}\mathbf{\Psi}_\mathcal{II}^{i}]\Big)%\vspace{-2mm}
		\end{array}
		\label{Lower Bound}		
	\end{equation*}
	where 
	$\mathbf{\Psi}_\mathcal{II}^{i}\triangleq\mathbb{E}_{q(\boldsymbol{\theta})}[\mathbf{K}_{\mathcal{I}\mathcal{D}_i}\mathbf{C}^{-1}_{\mathcal{D}_i\mathcal{D}_i}\mathbf{K}_{\mathcal{D}_i\mathcal{I}}]$.\vspace{1mm}
	\label{decompo}
\end{corollary}	 
	
	Our main result below exploits the decomposability of $\mathcal{L}(q)$ in Corollary~\ref{decompo} to derive  stochastic gradients $\partial\widetilde{\mathcal{L}}/\partial\mathbf{m}$, $\partial\widetilde{\mathcal{L}}/\partial\mathbf{S}$, $\partial\widetilde{\mathcal{L}}/\partial\boldsymbol{\nu}$, $\partial\widetilde{\mathcal{L}}/\partial\mathbf{\Xi}$, $\partial\widetilde{\mathcal{L}}/\partial \alpha$, and $\partial\widetilde{\mathcal{L}}/\partial \beta$ that are unbiased estimators of their respective exact gradients,
	which is the key contribution of our work in this paper:\vspace{1mm}
	%
	%
	\begin{theorem}
		Let $\mathcal{S}$ be a set of i.i.d. samples drawn from a uniform distribution over $\{1,2,\dots,B\}$. %\vspace{-0.5mm}
		Construct the stochastic gradients $\partial\widetilde{\mathcal{L}}/\partial\mathbf{m}$, $\partial\widetilde{\mathcal{L}}/\partial\mathbf{S}$, $\partial\widetilde{\mathcal{L}}/\partial\boldsymbol{\nu}$, $\partial\widetilde{\mathcal{L}}/\partial\mathbf{\Xi}$, $\partial\widetilde{\mathcal{L}}/\partial \alpha$, and $\partial\widetilde{\mathcal{L}}/\partial\beta$ 
using the mini-batches $(\mathcal{D}_s, \mathbf{y}_{\mathcal{D}_s})$ for $s\in\mathcal{S}$ and current estimates of $(\mathbf{m}, \mathbf{S}, \boldsymbol{\nu}, \mathbf{\Xi}, \alpha, \beta)$ according to~\eqref{zoo} in Appendix~\ref{A.3} of~\cite{HaibinAPP}. 
		Then, $\mathbb{E}[\partial\widetilde{\mathcal{L}}/\partial\mathbf{m}]=\partial{\mathcal{L}}/\partial\mathbf{m}$, 
		$\mathbb{E}[\partial\widetilde{\mathcal{L}}/\partial\mathbf{S}]=\partial{\mathcal{L}}/\partial\mathbf{S}$, 
		$\mathbb{E}[\partial\widetilde{\mathcal{L}}/\partial\boldsymbol{\nu}]=\partial{\mathcal{L}}/\partial\boldsymbol{\nu}$,
		$\mathbb{E}[\partial\widetilde{\mathcal{L}}/\partial\mathbf{\Xi}]=\partial{\mathcal{L}}/\partial\mathbf{\Xi}$, $\mathbb{E}[\partial\widetilde{\mathcal{L}}/\partial \alpha]=\partial{\mathcal{L}}/\partial \alpha$, and \ $\mathbb{E}[\partial\widetilde{\mathcal{L}}/\partial \beta]=\partial{\mathcal{L}}/\partial \beta$.
		\label{thm2}\vspace{1mm}
	\end{theorem}	
	%		We prove that the stochastic gradient is the unbiased estimate of the full gradient, denoted as $\mathbb{E}(\partial\check{\mathcal{L}})=\partial\mathcal{L}$, 
Its proof is in Appendix~\ref{A.3} of~\cite{HaibinAPP}.\vspace{1mm} 

\emph{Remark 6:} The stochastic gradients (Theorem~\ref{thm2}) can be computed in closed form in $\mathcal{O}(|\mathcal{S}||\mathcal{I}|^3)$ time per iteration that reduces to $\mathcal{O}(|\mathcal{I}|^3)$ time by setting $|\mathcal{S}|=1$ in our experiments.
So, if the number of iterations of stochastic gradient ascent needed for convergence is  much smaller than $t \min(|\mathcal{D}|/|\mathcal{I}|,B)$ where $t$ is the required number of iterations of exact gradient ascent, then our stochastic variants achieve a huge speedup over the corresponding VBSGPR models.\vspace{0mm}
	%
	%		
\section{Bayesian Prediction with VBSGPR Models}
\label{predict}
Recall that the predictive distribution $p(f_{\mathbf{x}^*}|\mathbf{y}_\mathcal{D})$ is computationally intractable. We thus approximate it by $q(f_{\mathbf{x}^*}|\mathbf{y}_\mathcal{D})=\int q(f_{\mathbf{x}^*}|\mathbf{y}_{\mathcal{D}_B},\mathbf{s}_\mathcal{I},\mathbf{\Lambda},\sigma_f)\ q^+(\mathbf{s}_\mathcal{I})\ q^+(\mathbf{\Lambda})\ q^+(\sigma_f)\ \mathrm{d}\mathbf{s}_\mathcal{I}\ \mathrm{d}\mathbf{\Lambda}\ \mathrm{d}\mathbf{\sigma_f}$ where $q^+(\mathbf{s}_\mathcal{I})\triangleq\mathcal{N}(\mathbf{m}^+,\mathbf{S}^+)$,  $q^+(\mathbf{\Lambda})\triangleq\prod_{i=1}^{d}\mathcal{N}(\nu_i^+,\xi_i^+)$ with $\boldsymbol{\nu}^+ \triangleq (\nu_1^+,\ldots,\nu_d^+)^\top$ and $\mathbf{\Xi}^+ \triangleq \mathrm{diag}[\xi_1^+,\ldots,\xi_d^+]$, and $q(\sigma_f)\triangleq\mathcal{N}(\alpha^+,\beta^+)$ are obtained from the stochastic gradient ascent updates (Section~\ref{Stochastic Variational Inference for GPR}). Note that $q(f_{\mathbf{x}^*}|\mathbf{y}_{\mathcal{D}_B}, \mathbf{s}_\mathcal{I},\mathbf{\Lambda},\sigma_f)$ is set to $p(f_{\mathbf{x}^*}|\mathbf{s}_\mathcal{I},\mathbf{\Lambda},\sigma_f)$ for the VBPITC, VBFIC, VBFITC, and VBDTC models and to  $p(f_{\mathbf{x}^*}|\mathbf{y}_{\mathcal{D}_B}, \mathbf{s}_\mathcal{I},\mathbf{\Lambda},\sigma_f)$ for the VBPIC model. Although the predictive distribution $q(f_{\mathbf{x}^*}|\mathbf{y}_\mathcal{D})$ is not Gaussian, its predictive mean $\mu_{\mathbf{x}^*|\mathcal{D}}$ and variance $\sigma^2_{\mathbf{x}^*|\mathcal{D}}$ can be computed analytically for VBPITC, VBFIC, VBFITC, and VBDTC and via sampling for VBPIC, as derived in Appendix~\ref{q(y^*)} of~\cite{HaibinAPP}.
Their respective predictive means $\mu_{\mathbf{x}^*|\mathcal{D}}$ closely resemble that of PITC, FIC, FITC, DTC, and PIC (see eqs.~$84$ and~$86$ in Appendix D.$4$ of~\cite{NghiaICML15}) except for the expectations over $\mathbf{\Lambda}$ and $\sigma_f$ due to the variational Bayesian treatment. Their predictive variances $\sigma^2_{\mathbf{x}^*|\mathcal{D}}$ are also similar except for the expectations over $\mathbf{\Lambda}$ and $\sigma_f$ and an additional positive term arising from the uncertainty of $\mathbf{\Lambda}$ and $\sigma_f$.
%	
\section{Experiments and Discussion}
	\label{Experiments and Discussion}%\vspace{-0.5mm}
	This section empirically evaluates the predictive performance and time efficiency of the stochastic variants, denoted by VBDTC$+$, VBFITC$+$, and VBPIC$+$, of our VBSGPR models (respectively, VBDTC, VBFITC, and VBPIC).
We will first use the small AIMPEAK dataset \cite{LowUAI13} on traffic speeds of size $41850$ to  evaluate the convergence of the variational distributions $q^+(\mathbf{s}_\mathcal{I})$ and $q^+(\mathbf{\Lambda},\sigma_f)$ induced by our stochastic variants VBDTC$+$, VBFITC$+$, and VBPIC$+$ to, respectively, $q(\mathbf{s}_\mathcal{I})$ and $q(\mathbf{\Lambda},\sigma_f)$ induced by VBDTC \cite{Titsias13}, VBFITC, and VBPIC performing exact gradient ascent updates via \emph{scaled conjugate gradient} (SCG).
To do this, we use the KL distance metric to measure the distance between the variational distributions obtained from the stochastic vs. exact gradient ascent.	

Then, using the real-world TWITTER dataset 
%~\cite{Twitter13} 
on buzz events of size $583250$ and AIRLINE dataset~\cite{Lawrence13} on flight delays of size $2055733$, we will compare the performance of the stochastic variants of our VBSGPR models with that of the state-of-the-art GP models such as the stochastic variants of variational DTC (SVIGP) \cite{Lawrence13} and variational PIC (PIC$+$) \cite{NghiaICML15}, distributed variational DTC (Dist-VGP) \cite{Yarin14}, and rBCM \cite{deisenroth2015distributed}, all of which find point estimates of hyperparameters.
Such a comparison will demonstrate the benefits of adopting a variational Bayesian treatment of the hyperparameters by our VBSGPR models.
We will also compare the performance of our stochastic VBSGPR models with that of the stochastic variant of \emph{variational Bayesian sparse spectrum GPR} (VSSGPR) model \cite{Gal2015Improving}. 
To evaluate their predictive performance, we use the \emph{root mean square error} (RMSE) metric: $\sqrt{\sum_{\mathbf{x}^*\in\mathcal{T}}(y_{\mathbf{x}^*}-\mu_{\mathbf{x}^*|\mathcal{D}})^2/|\mathcal{T}|}$ and 
the \emph{mean negative log probability} (MNLP) metric: $0.5\sum_{\mathbf{x}^*\in\mathcal{T}}\{(y_{\mathbf{x}^*}-\mu_{\mathbf{x}^*|\mathcal{D}})^2 / \sigma^2_{x^*|\mathcal{D}} + \log(2\pi\sigma^2_{x^*|\mathcal{D}})\} / |\mathcal{T}|$ 
%(c) incurred time, and (d) time efficiency (TE) vs. predictive efficiency (PE). Formally, TE (PE) is defined as the incurred time (RMSE) of our VBSGPR models.
where $\mathcal{T}$ denotes a set of test inputs.

All datasets are modeled using GPs whose prior covariance is defined by the squared exponential covariance function defined in Section~\ref{full}. 
All experiments are run on a Linux system with Intel$\circledR$ Xeon$\circledR$ E$5$-$2683$ CPU at $2.1$GHz with $256$GB memory.
%\vspace{-1mm}

\subsection{Empirical Convergence of Stochastic VBSGPR Models}
	\label{app:AIMPEAK}
The AIMPEAK dataset \cite{LowUAI13} of size $41850$ comprises traffic speeds (km/h) along $775$ road segments of an urban road network during morning peak hours on April $20$, $2011$. Each input (i.e., road segment) denotes a $5$D feature vector of length, number of lanes, speed limit, direction, and time, the last of which comprises $54$ five-minute time slots. The output corresponds to  traffic speed. We randomly select training data of size $1000$, \textcolor{black}{which is partitioned into $B=10$ mini-batches}, and $50$ inducing inputs from the inputs of the training data.	
		\begin{figure}
			\begin{tabular}{ccc}
				\hspace{-2.5mm}\includegraphics[height=2.5cm]{KLqU_DTC_new.pdf} & \hspace{-4mm}\includegraphics[height=2.55cm]{KLqU_FITC_new.pdf} & \hspace{-4mm}\includegraphics[height=2.5cm]{KLqU_PIC_new.pdf}\vspace{-1mm}\\
				\hspace{-2.5mm}(a) & \hspace{-4mm}(b) & \hspace{-4mm}(c) \vspace{0mm}\\
				\hspace{-2.5mm}\includegraphics[height=2.5cm]{KLqHyper_DTC_new.pdf} & \hspace{-4mm}\includegraphics[height=2.52cm]{KLqHyper_FITC_new.pdf} & \hspace{-4mm}\includegraphics[height=2.5cm]{KLqHyper_PIC_new.pdf} \vspace{-1mm}\\
				\hspace{-2.5mm}(d) & \hspace{-4mm}(e) & \hspace{-4mm}(f)%\vspace{-1mm}
			\end{tabular}
			\caption{Graphs of KL distance $\mathrm{KL}(q(\mathbf{s}_\mathcal{I})||q^+(\mathbf{s}_\mathcal{I}))$ of (a) VBDTC$+$ to VBDTC, (b) VBFITC$+$ to VBFITC, (c) VBPIC$+$ to VBPIC, and $\mathrm{KL}((q(\mathbf{\Lambda},\sigma_f)||q^+(\mathbf{\Lambda},\sigma_f))$ of (d) VBDTC$+$ to VBDTC, (e) VBFITC$+$ to VBFITC, (f) VBPIC$+$ to VBPIC vs. no. $t$ of iterations for AIMPEAK dataset.}
			\label{fig1}%\vspace{-4mm}
		\end{figure}
				
		Figs.~\ref{fig1}a-\ref{fig1}c (Figs.~\ref{fig1}d-\ref{fig1}f) shows results of the KL distance $\mathrm{KL}(q(\mathbf{s}_\mathcal{I})||q^+(\mathbf{s}_\mathcal{I}))$ ($\mathrm{KL}(q(\mathbf{\Lambda},\sigma_f)||q^+(\mathbf{\Lambda},\sigma_f))$) of $q^+(\mathbf{s}_\mathcal{I})$ to $q(\mathbf{s}_\mathcal{I})$ ($q^+(\mathbf{\Lambda},\sigma_f)$ to $q(\mathbf{\Lambda},\sigma_f)$) averaged over $5$ random selections of training data and mini-batch sequences with an increasing number $t$ of iterations. It can be observed that the variational distributions $q^+(\mathbf{s}_\mathcal{I})$ and $q^+(\mathbf{\Lambda},\sigma_f)$ induced by VBDTC$+$, VBFITC$+$, and VBPIC$+$ converge rapidly to, respectively, $q(\mathbf{s}_\mathcal{I})$ and $q(\mathbf{\Lambda},\sigma_f)$ induced by VBDTC, VBFITC, and VBPIC, thus corroborating our theoretical results in Section~\ref{Stochastic Variational Inference for GPR}. From Figs.~\ref{fig1}a-\ref{fig1}c, it can also be observed that $q^+(\mathbf{s}_\mathcal{I})$ induced by  VBDTC$+$ converges faster to $q(\mathbf{s}_\mathcal{I})$ than that by VBFITC$+$ and VBPIC$+$, which can be explained by its much simpler noise structure by assuming i.i.d. observation noises with constant variance $ \sigma_n^2 $.%\vspace{-1mm}
%
\subsection{Empirical Evaluation on AIRLINE and TWITTER Datasets}%\vspace{-1mm}	
%	 on two real-world datasets:
The TWITTER dataset 
%~\cite{Twitter13} 
contains $583250$ instances of buzz events on Twitter.  The input denotes a relatively large $77$D feature vector described at 
http://ama.liglab.fr/datasets/buzz/, which makes this dataset suitable for evaluating  robustness to overfitting.
The output is the popularity of the instance's topic. 		
%		\item[(a)] The \emph{EMULATE mean sea level pressure} (EMSLP) dataset \cite{ansell2006daily} of size $1278250$ spans a $5^\circ$ lat.-lon. grid bounded within latitude $25$-$70$N and longitude $70$W-$50$E from $1900$ to $2003$. Each input denotes a $6$-dimensional feature vector of latitude, longitude, year, month, day, and incremental day count (starting from $0$ on first day). The output is the mean sea level pressure (Pa). 
%
The massive benchmark AIRLINE dataset~\cite{Lawrence13} contains $2055733$ records of information about every commercial flight in the USA from January to April $2008$. The input denotes an $8$D feature vector of age of the aircraft (no. of years in service), travel distance (km), airtime, departure and arrival time (min.) as well as day of the week, day of month, and month. The output is the delay time (min.) of the flight. 
	%We follow the same experimental setting as described in \cite{NghiaICML15}. 
For each dataset, $5\%$ is randomly selected and set aside as test data. The remaining data is used as training data and \textcolor{black}{partitioned into $B=1000$ mini-batches} using $k$-means (i.e., $k = B$). We randomly select $100$ inducing inputs from the inputs of the training data. 

	Figs.~\ref{fig4}a and~\ref{fig4}b show results of RMSE and MNLP achieved by the stochastic variants of our VBSGPR models averaged over $5$ random selections of $5\%$ test data and mini-batch sequences with an increasing number $t$ of iterations for the AIRLINE dataset.
It can be observed that VBPIC$+$ (RMSE of $21.87$~min. and MNLP of $4.53$) achieves considerably better predictive performance than VBFITC$+$ (RMSE of $37.05$~min. and \textcolor{black}{MNLP of $7.84$}) and VBDTC$+$ (RMSE of $37.55$~min. and \textcolor{black}{MNLP of $8.06$}). 
%\textcolor{red}{The results of their MNLPs show similar performance as  VBDTC$+$ (MNLP of $4.53$), VBFITC$+$ (MNLP of $7.87$) and VBDTC$+$ (MNLP of $7.87$)}.
		To explain this, VBFITC$+$ and VBDTC$+$ have both imposed a strong assumption of independently distributed observation noises. 
		In contrast, VBPIC$+$ caters to correlation of observation noises within each mini-batch of data (Sections~\ref{Variational Inference of the Bayesian DTC} and~\ref{Stochastic Variational Inference for GPR}), hence modeling and predicting real-world datasets with correlated noises better.
		% that the superior predictive performance (lower RMSE) of VBPIC+ over VBFITC+ and VBDTC+, which is expected: 
		Furthermore, unlike VBFITC$+$ and VBDTC$+$, VBPIC$+$ does not assume conditional independence between the training and test outputs given the inducing outputs in its test conditional.
	\begin{figure}
		\begin{tabular}{cc}
			\hspace{-2mm}\includegraphics[scale=0.16]{airline_rmse} &
			\hspace{-4mm}\includegraphics[scale=0.16]{airline_mnlp} 
			\vspace{-1mm}\\
			\hspace{-2mm}{(a)} & \hspace{-4mm}{(b)}
			\\
			\hspace{-2mm}\includegraphics[scale=0.16]{airline_time} &
			\hspace{-4mm}\includegraphics[scale=0.16]{airline_rmse_time}
			\vspace{-1mm}\\
			\hspace{-2mm}{(c)} & \hspace{-4mm}{(d)}\vspace{-0.7mm}
		\end{tabular}
		\caption{Graphs of (a) RMSE, (b) MNLP, and (c) total incurred time vs. number $t$ of iterations, and (d) graphs of RMSE vs. total incurred time of VBDTC$+$, VBFITC$+$, and VBPIC$+$ for the AIRLINE dataset.}
		\label{fig4}
	\end{figure}

Fig.~\ref{fig4}c exhibits a near-linear increase in total incurred time with an increasing number $t$ of iterations for VBDTC$+$, VBFITC$+$, and VBPIC$+$. Our experiments reveal that VBDTC$+$, VBFITC$+$, and VBPIC$+$ incur, respectively, an average of $0.0122$, $0.0132$, and $0.038$ seconds per iteration of stochastic gradient ascent update. 
Fig.~\ref{fig4}d shows that VBPIC$+$ can achieve a more superior trade-off between predictive performance vs. time efficiency than VBDTC$+$ and VBFITC$+$. 
%
%It can be elucidated from  that we can achieve better predictive performance without compromising much time efficiency. 

		Figs.~\ref{fig3}a and~\ref{fig3}b show results of RMSE and MNLP achieved by the stochastic variants of our VBSGPR models averaged over $5$ random selections of $5\%$ test data and mini-batch sequences with an increasing number $t$ of iterations for the TWITTER dataset. 
		The observations are similar to that for the AIRLINE dataset:
		It can be observed that \textcolor{black}{VBPIC$+$ (RMSE of $131.46$ and MNLP of $6.45$) achieves significantly better predictive performance than VBFITC$+$ (RMSE of $212.67$ and MNLP of $7.21$) and VBDTC$+$ (RMSE of $247.38$ and MNLP of $7.69$)}; 		
%		\textcolor{red}{The results of their MNLPs show similar performance as  VBDTC$+$ (), VBFITC$+$ () and VBDTC$+$ ()}; 
this can be explained by the same reasons as that discussed previously for the AIRLINE dataset.
	\begin{figure}
		\begin{tabular}{cc}
			\hspace{-2mm}\includegraphics[scale=0.16]{twitter_rmse} &
			\hspace{-4mm}\includegraphics[scale=0.16]{twitter_mnlp} 
			\vspace{-1mm}\\
			\hspace{-2mm}{(a)} & \hspace{-4mm}{(b)}
			\\
			\hspace{-2mm}\includegraphics[scale=0.16]{twitter_time} &
			\hspace{-4mm}\includegraphics[scale=0.16]{twitter_rmse_time}
			\vspace{-1mm}\\
			\hspace{-2mm}{(c)} & \hspace{-4mm}{(d)}\vspace{-1mm}
		\end{tabular}
		\caption{Graphs of (a) RMSE, (b) MNLP, and (c) total incurred time vs. number $t$ of iterations, and (d) graphs of RMSE vs. total incurred time of VBDTC$+$, VBFITC$+$, and VBPIC$+$ for the TWITTER dataset.}
		\label{fig3}
	\end{figure}
		
		Fig.~\ref{fig3}c also exhibits a linear increase in total incurred time with an increasing number $t$ of iterations for VBDTC$+$, VBFITC$+$, and VBPIC$+$. Our experiments reveal that VBDTC$+$, VBFITC$+$, and VBPIC$+$ incur, respectively, \textcolor{black}{an average of $0.0073$, $0.0075$, and $0.0087$ seconds per iteration of stochastic gradient ascent update, which are shorter than that for the AIRLINE dataset} 
due to a smaller mini-batch size.		
%(Section~\ref{Experiments and Discussion}) 
%due to a much larger input dimension of $77$, hence needing to compute expectations and derivatives with respect to many more $\nu^+_i$'s and $\xi^+_i$'s (Section~\ref{Variational Inference of the Bayesian DTC}). 
Fig.~\ref{fig3}d reveals that VBPIC$+$ can similarly achieve the best trade-off between predictive performance vs. time efficiency.
		
Table~\ref{tab:exp} compares the predictive performance (RMSEs) achieved by state-of-the-art GP models for the AIRLINE and TWITTER datasets. 
It can be observed that our VBPIC$+$ significantly outperforms state-of-the-art SVIGP, Dist-VGP, rBCM, and PIC$+$, which find point estimates of hyperparameters, and VSSGPR due to its restrictive assumption, as discussed in Section~\ref{sect:intro}. In contrast, our VBPIC$+$ assumes a variational Bayesian treatment of its hyperparameters, thus achieving robustness to overfitting due to Bayesian model selection, as demonstrated later. Unlike VSSGPR, VBPIC$+$ does not assume conditional independence between the training and test outputs in its test conditional.
\begin{table}
%\hspace{-2.3mm}
\begin{small}	
\begin{tabular}{l|cccccc}
\hline
\hspace{-2.7mm} Dataset & \hspace{-2mm} SVIGP & \hspace{-3mm} Dist-VGP & \hspace{-3mm} rBCM & \hspace{-3mm} PIC$+$ & \hspace{-3.5mm} VBPIC$+$ & \hspace{-3.5mm} VSSGPR \hspace{-2.7mm} \\ 
\hline
\hspace{-2mm}AIRLINE & \hspace{-2mm} $39.53$ & \hspace{-3mm} $35.30$ & \hspace{-3.5mm} $34.40$ & \hspace{-3.5mm} $24.9$ & \hspace{-4mm} {\bf 21.87} & \hspace{-3.5mm} $38.95$\\ 
\hspace{-2mm}TWITTER\hspace{-2mm} & \hspace{-2mm} $-$ & \hspace{-3mm} $-$ & \hspace{-3.5mm} $-$ & \hspace{-3.5mm} $190.2$ & \hspace{-4mm} {\bf 131.4} & \hspace{-3.5mm} $585.9$\\
\hline
%\vspace{-1mm}
\end{tabular}
\end{small}
\caption{RMSE achieved by VBPIC$+$ and state-of-the-art GP models for AIRLINE and TWITTER datasets. The results of PIC$+$ and VSSGPR are obtained using their GitHub codes. The results of Dist-VGP and rBCM are taken from their respective papers and that of SVIGP is reported in~\cite{NghiaICML15}. They are all based on the same settings of training/test data sizes $= 2$M/$100$K ($554$K/$29$K) for the AIRLINE (TWITTER) dataset.}
\label{tab:exp}\vspace{-1mm}
\end{table}
		\begin{figure}
			\begin{tabular}{cc}
				\hspace{-2.5mm}\includegraphics[scale=0.16]{twitter_sample} &
				\hspace{-4mm}\includegraphics[scale=0.16]{airline_sample_range20_40}\vspace{-1mm}\\
				\hspace{-2.5mm}{(a)} & \hspace{-4mm}{(b)}\vspace{-1mm}
			\end{tabular}
			%	\hfill
			%	\begin{minipage}[h]{0.49\linewidth}
			%		\centerline{\includegraphics[scale=0.31]{MyFlightDelay_FITC.pdf}}
			%		\centerline{(a)}
			%	\end{minipage}
			%	\hfill
			%	\begin{minipage}[h]{0.49\linewidth}
			%		\centerline{\includegraphics[scale=0.30]{MyFlightDelay_PIC.pdf}}
			%		\centerline{(a)}
			%	\end{minipage}
			\caption{Graphs of RMSEs of VBPIC$+$ vs. number $t$ of iterations with varying sampling sizes for computing its predictive mean for the (a) TWITTER and (b) AIRLINE datasets.}
			\label{fig5}%\vspace{-3mm}
		\end{figure}

Fig.~\ref{fig5} shows results of RMSEs achieved by our VBPIC$+$ with an increasing number $t$ of iterations and varying sample sizes for computing its predictive mean (Section~\ref{predict}).
%compares the predictive performance among different sample sizes for . 
Note that a sample size of $1$ reduces VBPIC to PIC that treats its sampled hyperparameters as a point estimate. By increasing the sample size, it can be observed that VBPIC$+$ converges faster to a lower RMSE using less iterations
%, which is highly desirable in practice when a greater data usage costs more. 
%Furthermore, under both circumstances, we can observe from Fig.~\ref{fig5}a that the larger the sample size, the better the predictive performance.
%VBPIC$+$ achieves lower RMSE 
due to its Bayesian model selection/averaging, thus demonstrating its increasing robustness to overfitting.

Fig.~\ref{fig6} displays the $95\%$ confidence intervals (mean $\nu^+_i$ $\pm$ 2$\ \times\ $standard deviation $(\xi^+_i)^{1/2}$) for \textcolor{black}{inverted length-scale} hyperparameters $\lambda_i$ for $i=1,\ldots,d$ \textcolor{black}{after $t=10000$ iterations} for the TWITTER ($d=77$ normalized input dimensions) and AIRLINE ($d=8$ normalized input dimensions) datasets. It can be observed that the confidence intervals  are generally wider (i.e., larger uncertainty of $\lambda_1,\ldots,\lambda_d$) for the TWITTER dataset than for the AIRLINE dataset. To confirm this, we measure the \emph{mean log variance} (MLV) $\sum^d_{i=1}\log\xi^+_i/d$ of $\lambda_1,\ldots,\lambda_d$ and notice that the TWITTER dataset gives a higher MLV of $-4.09$ than that for the AIRLINE dataset (i.e., MLV $= -6.55$).
So, with a larger uncertainty of $\lambda_1,\ldots,\lambda_d$, their point estimates have a greater tendency to overfit and hence yield a poorer predictive performance, as observed in Fig.~\ref{fig5} (compare the performance gap between sample sizes of $1$ vs. $256$).
%illustrates the distributions of hyperparameters (length-scales) for TWITTER and AIRLINE datasets. 
%We proposed an interesting metric, mean of the log determinant (MLD), to measure the variance scale of the hyperparameters (length-scales). The MLD for twitter dataset is $-5.80$ compared to the MLD for airline dataset which is $-6.68$. It can be discovered from the MLD that the variance scale for TWITTER dataset is wider than that of AIRLINE dataset. 
%
%This can also be deduced from Fig.~\ref{fig5} that we achieve much larger RMSE gain when increasing the sample size for TWITTER dataset than that of AIRLINE dataset.
%
% Note that a sample size of $1$ reduces VBPIC to PIC that treats its sampled hyperparameters as a point estimate.	
% By increasing the sample size, it can be observed that VBPIC$+$ converges faster to a lower RMSE using less data, which is highly desirable in practice when it costs more for a greater data usage.
%Under the circumstances that a dataset is extremely large and we would like to extract certain pattern from the dataset, however, each time we have to pay to get the data (i.e. money or memory), by incorporating the Bayesian anytime SGPR model, it could allow us to run less iterations to find the certain pattern, which is of highly practical use. 
% Furthermore, when input dimension is relatively large (i.e., $77$ for  TWITTER dataset), utilizing a point estimate (i.e., sample size of $1$) yields poor predictive performance that does not improve even after $56$ iterations (Fig.~\ref{fig5}a), hence revealing its vulnerability to overfitting.
% In contrast, when the sample size is beyond $20$, VBPIC$+$ achieves much lower RMSE due to Bayesian model selection, thus demonstrating its robustness to overfitting.\vspace{-2mm}
		\begin{figure}
			\begin{tabular}{cc}
				\hspace{-2mm}\includegraphics[scale=0.16]{twitter_ls} &
				\hspace{-3mm}\includegraphics[scale=0.16]{airline_ls}\vspace{-1mm}\\
				\hspace{-2mm}{(a)} & \hspace{-3mm}{(b)}\vspace{-2mm}
			\end{tabular}
			%	\hfill
			%	\begin{minipage}[h]{0.49\linewidth}
			%		\centerline{\includegraphics[scale=0.31]{MyFlightDelay_FITC.pdf}}
			%		\centerline{(a)}
			%	\end{minipage}
			%	\hfill
			%	\begin{minipage}[h]{0.49\linewidth}
			%		\centerline{\includegraphics[scale=0.30]{MyFlightDelay_PIC.pdf}}
			%		\centerline{(a)}
			%	\end{minipage}
			\caption{$95\%$ confidence intervals (mean $\nu^+_i$ $\pm$ 2$\ \times\ $standard deviation $(\xi^+_i)^{1/2}$) for \textcolor{black}{inverted length-scale} hyperparameters $\lambda_i$ for $i=1,\ldots,d$ \textcolor{black}{after $t=10000$ iterations} for the (a) TWITTER ($d=77$ normalized input dimensions) and (b) AIRLINE ($d=8$ normalized input dimensions) datasets.}
			\label{fig6}%\vspace{-3mm}
		\end{figure}
	%Due to the strong assumption of the current framework, the performance is not as good as the PIC+ in this instance. One possible explanation for this would be that the hyperparameters for this dataset learned by \cite{NghiaICML15} is more fit-table. 
	%
	%
	%
	\section{Conclusion}
	%\vspace{-1mm}
	This paper describes a novel variational inference framework for a family of VBSGPR models (e.g., VBDTC, VBFITC, VBPIC) whose approximations are variationally optimal with respect to the FGPR model enriched with various corresponding correlation structures of the observation noises.
	Our variational Bayesian treatment of hyperparameters enables our VBSGPR models to mitigate critical issues (e.g., overfitting) which plague existing variational SGPR models that optimize point estimates of hyperparameters (Section~\ref{sect:intro}).	
%	Our proposed stochastic gradient ascent method allows 
The stochastic variants of our VBSGPR models can yield good predictive performance fast and improve their predictive performance over time, thus achieving scalability to big data.
	%	of anytime fully Bayesian SGPR models (e.g. B-DTC, B-FITC and B-PIC) which can produce good predictive performance fast and a stochastic gradient ascent method can be derived that is guaranteed to achieve asymptotic convergence to the predictive distribution of any Bayesian SGPR model if the heteroscedastic noise satisfies certain structure. We also prove that the lower bound satisfies the decomposition conditions, then the stochastic gradient ascent method is an unbiased estimator of the exact gradient and can be computed in constant time per iteration. 
	Empirical evaluation on two real-world datasets reveals that the stochastic variant of our VBPIC can significantly outperform existing state-of-the-art GP models, thus demonstrating its robustness to overfitting due to Bayesian model selection while preserving scalability to big data through stochastic optimization.	
	For our future work, we plan to integrate our proposed framework with that of decentralized/distributed data/model fusion~\cite{chen2013gaussian,LowTASE15,LowUAI12,NghiaAAAI19,Ruofei18} for collective online learning of a massive number of VBSGPR models. 
	%	We do experiments on two real-world million-size datasets to show that the Bayesian SGPR (specifically, B-DTC, B-FITC and B-PIC) outperforms the existing anytime framework of point estimate SGPR models \cite{NghiaICML15}. 
	%In our future work, we plan to apply the variational Bayesian treatment and stochastic optimization to the hyperparameters defining the covariance function representing the observation noises as well as to recently developed variational SGPR models of \citet{HoangICML16}.
\bibliographystyle{IEEEtran}
\bibliography{Reference}

\clearpage
\appendix

	% \subsection{MNLP Performance of VBSGPR models AIRLINE and TWITTER Dataset}
	% \label{MNLP Performance}
	% Similar to Fig.~\ref{fig4} and Fig.\ref{fig3} that show the convergence of the RMSEs of VBPIC+, VBFITC+ and VBDTC+. It can be observed from Fig.~\ref{fig6}(a) that for the AIRLINE dataset, 
	% \begin{figure}[h]
	% 	\begin{tabular}{cc}
	% 		\hspace{-10mm}\includegraphics[scale=0.17]{airline_rmse.pdf} &
	% 		\hspace{-6mm}\includegraphics[scale=0.17]{twitter_rmse.pdf}
	% 		\vspace{-0.5mm}\\
	% 		\hspace{-10mm}{(a)} & \hspace{-6mm}{(b)}
	% 	\end{tabular}
	% 	\caption{Graphs of MNLPs of VBDTC$+$, VBFITC$+$, and VBPIC$+$ vs. number $t$ of iterations for the (a) AIRLINE  and (b) TWITTER dataset.}
	% 	\label{fig6}
	% \end{figure}

	\subsection{Derivation of~\eqref{log likelihood}}
	\label{Derivation of Equation (15)}
	For all $\mathbf{f}_\mathcal{D}$, $\mathbf{s}_\mathcal{I}$, $\mathbf{\Lambda}$, and $\sigma_f$,\vspace{-1.5mm}
	$$
	\vspace{-4.5mm}
	p(\mathbf{y}_\mathcal{D}) =\frac{p(\mathbf{y}_\mathcal{D},\mathbf{f}_\mathcal{D},\mathbf{s}_\mathcal{I},\mathbf{\Lambda},\sigma_f)}{p(\mathbf{f}_\mathcal{D},\mathbf{s}_\mathcal{I},\mathbf{\Lambda},\sigma_f|\mathbf{y}_\mathcal{D})}\ .
	\vspace{-0mm}
	$$
	So,
	$$	
	\vspace{-0mm}	
	\log p(\mathbf{y}_\mathcal{D})=\log \frac{p(\mathbf{y}_\mathcal{D},\mathbf{f}_\mathcal{D},\mathbf{s}_\mathcal{I},\mathbf{\Lambda},\sigma_f)}{p(\mathbf{f}_\mathcal{D},\mathbf{s}_\mathcal{I},\mathbf{\Lambda},\sigma_f|\mathbf{y}_\mathcal{D})}\ .
	\vspace{1mm}
	$$
	Let $q(\mathbf{f}_\mathcal{D},\mathbf{s}_\mathcal{I},\mathbf{\Lambda},\sigma_f)$ be an arbitrary probability density function that is independent of $\mathbf{y}_\mathcal{D}$. Integrating both sides of the above equation with respect to $q(\mathbf{f}_\mathcal{D},\mathbf{s}_\mathcal{I},\mathbf{\Lambda},\sigma_f)$ yields\vspace{-1mm}
	\begin{equation}\vspace{-3mm}
	\hspace{-1.7mm}
	\begin{array}{l}
	\log p(\mathbf{y}_\mathcal{D})\\
	\displaystyle=\hspace{-2mm}\int\hspace{-1mm} q(\mathbf{f}_\mathcal{D},\mathbf{s}_\mathcal{I},\mathbf{\Lambda},\sigma_f)\log \frac{p(\mathbf{y}_\mathcal{D},\mathbf{f}_\mathcal{D},\mathbf{s}_\mathcal{I},\mathbf{\Lambda},\sigma_f)}{p(\mathbf{f}_\mathcal{D},\mathbf{s}_\mathcal{I},\mathbf{\Lambda},\sigma_f|\mathbf{y}_\mathcal{D})} \mathrm{d}\mathbf{f}_\mathcal{D}\mathrm{d}\mathbf{s}_\mathcal{I}\mathrm{d}\mathbf{\Lambda} \mathrm{d}\sigma_f
	\end{array}
	\label{crab}\vspace{0mm}
	\end{equation}
	Using $\log(ab)=\log(a)+\log(b)$, 
	%\vspace{-2mm}
	\begin{equation*}
		\begin{array}{rcl}
			\displaystyle	\log \frac{p(\mathbf{y}_\mathcal{D},\mathbf{f}_\mathcal{D},\mathbf{s}_\mathcal{I},\mathbf{\Lambda},\sigma_f)}{p(\mathbf{f}_\mathcal{D},\mathbf{s}_\mathcal{I},\mathbf{\Lambda},\sigma_f|\mathbf{y}_\mathcal{D})}&\hspace{-2.4mm}=&\hspace{-2.4mm}\displaystyle\log\dfrac{p(\mathbf{y}_\mathcal{D},\mathbf{f}_\mathcal{D},\mathbf{s}_\mathcal{I},\mathbf{\Lambda},\sigma_f)}{q(\mathbf{f}_\mathcal{D},\mathbf{s}_\mathcal{I},\mathbf{\Lambda},\sigma_f)}\vspace{1mm}\displaystyle\vspace{1mm}\\
			&&\hspace{-2.4mm}\displaystyle +\log\dfrac{q(\mathbf{f}_\mathcal{D},\mathbf{s}_\mathcal{I},\mathbf{\Lambda},\sigma_f)}{p(\mathbf{f}_\mathcal{D},\mathbf{s}_\mathcal{I},\mathbf{\Lambda},\sigma_f|\mathbf{y}_\mathcal{D})}
		\end{array}
		\vspace{-2mm}
	\end{equation*}
	\vspace{-1mm}
	which is substituted into~\eqref{crab} to give
	\vspace{-1mm}
	\begin{equation}
	\hspace{-1.7mm}
		\begin{array}{l}
			\log p(\mathbf{y}_\mathcal{D})=\vspace{1mm}\\
			\displaystyle\int q(\mathbf{f}_\mathcal{D},\mathbf{s}_\mathcal{I},\mathbf{\Lambda},\sigma_f)\log\dfrac{p(\mathbf{y}_\mathcal{D},\mathbf{f}_\mathcal{D},\mathbf{s}_\mathcal{I},\mathbf{\Lambda},\sigma_f)}{q(\mathbf{f}_\mathcal{D},\mathbf{s}_\mathcal{I},\mathbf{\Lambda},\sigma_f)} \mathrm{d}\mathbf{f}_\mathcal{D} \mathrm{d}\mathbf{s}_\mathcal{I} \mathrm{d}\mathbf{\Lambda} \mathrm{d}\sigma_f \vspace{1mm}\\
			\displaystyle+\hspace{-1mm}\int\hspace{-1mm} q(\mathbf{f}_\mathcal{D},\mathbf{s}_\mathcal{I},\mathbf{\Lambda},\sigma_f)\log\dfrac{q(\mathbf{f}_\mathcal{D},\mathbf{s}_\mathcal{I},\mathbf{\Lambda},\sigma_f)}{p(\mathbf{f}_\mathcal{D},\mathbf{s}_\mathcal{I},\mathbf{\Lambda},\sigma_f|\mathbf{y}_\mathcal{D})} \mathrm{d}\mathbf{f}_\mathcal{D}\mathrm{d}\mathbf{s}_\mathcal{I} \mathrm{d}\mathbf{\Lambda} \mathrm{d}\sigma_f .
		\end{array}
		\label{crab2}\vspace{1mm}
	\end{equation}\vspace{0mm}
	The first and second terms on the RHS of~\eqref{crab2} correspond to the variational lower bound $\mathcal{L}(q)$ and $\mathrm{KL}(q(\mathbf{f}_\mathcal{D},\mathbf{s}_\mathcal{I},\mathbf{\Lambda},\sigma_f)|| p(\mathbf{f}_\mathcal{D},\mathbf{s}_\mathcal{I},\mathbf{\Lambda},\sigma_f|\mathbf{y}_\mathcal{D}))$, respectively.
% 
% 
% 
\vspace{-3mm}
\subsection{Proof of Theorem~\ref{thm1}}
	\label{Derivation of the Lower Bound}
	\label{Derivation of Lower Bound L}
	%
	\vspace{-3mm}
	Given that 	\vspace{-2mm}
	\begin{equation*}
		\begin{array}{l}	\vspace{-0mm}
	p(\mathbf{y}_\mathcal{D},\mathbf{f}_\mathcal{D},\mathbf{s}_\mathcal{I},\mathbf{\Lambda},\sigma_f)=
				\\
	p(\mathbf{y}_\mathcal{D}|\mathbf{f}_\mathcal{D}) p(\mathbf{f}_\mathcal{D}|\mathbf{s}_\mathcal{I},\mathbf{\Lambda},\sigma_f)\ p(\mathbf{s}_\mathcal{I})\ p(\mathbf{\Lambda})\ p(\sigma_f)
		\end{array}
	\end{equation*}
	\vspace{-3mm}
	\begin{equation*}
		\hspace{-1.7mm}
		\begin{array}{l}
			\mathcal{L}(q)\\
			\displaystyle=\hspace{-2mm}\int \hspace{-1mm} q(\mathbf{f}_\mathcal{D},\mathbf{s}_\mathcal{I},\mathbf{\Lambda},\sigma_f)\log\dfrac{p(\mathbf{y}_\mathcal{D},\mathbf{f}_\mathcal{D},\mathbf{s}_\mathcal{I},\mathbf{\Lambda},\sigma_f)}{q(\mathbf{f}_\mathcal{D},\mathbf{s}_\mathcal{I},\mathbf{\Lambda},\sigma_f)} \mathrm{d}\mathbf{f}_\mathcal{D}\mathrm{d}\mathbf{s}_\mathcal{I} \mathrm{d}\mathbf{\Lambda}\mathrm{d}\sigma_f \vspace{-1mm}
			\\
			\displaystyle=\int q(\mathbf{f}_\mathcal{D},\mathbf{s}_\mathcal{I},\mathbf{\Lambda},\sigma_f) \\ 
			\displaystyle \hspace{-0.5mm}\log\hspace{-0.5mm}\dfrac{p(\mathbf{y}_\mathcal{D}|\mathbf{f}_\mathcal{D})p(\mathbf{f}_\mathcal{D}|\mathbf{s}_\mathcal{I},\mathbf{\Lambda},\sigma_f)p(\mathbf{s}_\mathcal{I})p(\mathbf{\Lambda})p(\sigma_f)}{p(\mathbf{f}_\mathcal{D}|\mathbf{s}_\mathcal{I},\mathbf{\Lambda},\sigma_f)q(\mathbf{s}_\mathcal{I})q(\mathbf{\Lambda})q(\sigma_f)} \mathrm{d}\mathbf{f}_\mathcal{D} \mathrm{d}\mathbf{s}_\mathcal{I} \mathrm{d}\mathbf{\Lambda} \mathrm{d}\sigma_f %\vspace{1mm}
			\\
			\displaystyle=\int q(\mathbf{f}_\mathcal{D},\mathbf{s}_\mathcal{I},\mathbf{\Lambda},\sigma_f)\\
			\quad\displaystyle\log \dfrac{p(\mathbf{y}_\mathcal{D}|\mathbf{f}_\mathcal{D})p(\mathbf{s}_\mathcal{I})p(\mathbf{\Lambda})p(\sigma_f)}{q(\mathbf{s}_\mathcal{I})q(\mathbf{\Lambda})q(\sigma_f)}\ \mathrm{d}\mathbf{f}_\mathcal{D}\ \mathrm{d}\mathbf{s}_\mathcal{I}\ \mathrm{d}\mathbf{\Lambda}\  \mathrm{d}\sigma_f\\
			\displaystyle=\int p(\mathbf{f}_\mathcal{D}|\mathbf{s}_\mathcal{I},\mathbf{\Lambda},\sigma_f)\ q(\mathbf{s}_\mathcal{I})\ q(\mathbf{\Lambda})\ q(\sigma_f)\Bigg(\log p(\mathbf{y}_\mathcal{D}|\mathbf{f}_\mathcal{D})\ +\\
			\quad\displaystyle\log\dfrac{p(\mathbf{s}_\mathcal{I})}{q(\mathbf{s}_\mathcal{I})}+\log\dfrac{p(\mathbf{\Lambda})}{q(\mathbf{\Lambda})}+\log\dfrac{p(\sigma_f)}{q(\sigma_f)}\Bigg) \mathrm{d}\mathbf{f}_\mathcal{D}\ \mathrm{d}\mathbf{s}_\mathcal{I}\ \mathrm{d}\mathbf{\Lambda}\ \mathrm{d}\sigma_f\vspace{0mm}\\
			\displaystyle= \mathcal{F}(q)+\int q(\mathbf{s}_\mathcal{I})\log\dfrac{p(\mathbf{s}_\mathcal{I})}{q(\mathbf{s}_\mathcal{I})}\ \mathrm{d}\mathbf{s}_\mathcal{I}+
			\int q(\mathbf{\Lambda})\log\dfrac{p(\mathbf{\Lambda})}{q(\mathbf{\Lambda})}\ \mathrm{d}\mathbf{\Lambda}\ \vspace{-0mm}\\
			\displaystyle\quad+\int q(\sigma_f)\log\dfrac{p(\sigma_f)}{q(\sigma_f)}\ \mathrm{d}\sigma_f
			%&=\mathcal{F}+\int q(\mathbf{\Lambda})\log\dfrac{p(\mathbf{\Lambda})}{q(\mathbf{\Lambda})}d\mathbf{\Lambda}
		\end{array}
	\end{equation*}
	where
	\begin{equation*}
		%\hspace{-1.7mm}
		\begin{array}{rcl}
	\mathcal{F}(q)&\hspace{-2.4mm}=&\hspace{-2.4mm}\displaystyle\int q(\mathbf{s}_\mathcal{I})\ \mathcal{G}(q,\mathbf{s}_\mathcal{I})\ \mathrm{d}\mathbf{s}_\mathcal{I}\\
	\mathcal{G}(q, \mathbf{s}_\mathcal{I})&\hspace{-2.4mm}=&\hspace{-2.4mm}\displaystyle\int q(\sigma_f)\ q(\mathbf{\Lambda})\ \mathcal{H}(\mathbf{s}_\mathcal{I},\mathbf{\Lambda},\sigma_f)\ \mathrm{d}\mathbf{\Lambda}\ \mathrm{d}\sigma_f \\
	%	\mathcal{H}(\mathbf{s}_\mathcal{I},\mathbf{\Lambda})\hspace{-2.4mm}&=&\hspace{-2.4mm}\int q(\sigma_f)\mathcal{J}(\mathbf{\Lambda},\sigma_f)\mathrm{d}\sigma_f\\
	\mathcal{H}(\mathbf{s}_\mathcal{I},\mathbf{\Lambda},\sigma_f)&\hspace{-2.4mm}=&\hspace{-2.4mm}\displaystyle\int p(\mathbf{f}_\mathcal{D}|\mathbf{s}_\mathcal{I},\mathbf{\Lambda},\sigma_f)\ \log p(\mathbf{y}_\mathcal{D}|\mathbf{f}_\mathcal{D})\ \mathrm{d}\mathbf{f}_\mathcal{D}\ .
		\end{array}
	\end{equation*}
	Let us first derive the closed-form expression of $H(\mathbf{s}_\mathcal{I},\mathbf{\Lambda},\sigma_f)$:
	\begin{equation*}
		\hspace{-1.7mm}
		\begin{array}{l}
			H(\mathbf{s}_\mathcal{I},\mathbf{\Lambda},\sigma_f)\\
			\displaystyle=\int p(\mathbf{f}_\mathcal{D}|\mathbf{s}_\mathcal{I},\mathbf{\Lambda},\sigma_f)\log p(\mathbf{y}_\mathcal{D}|\mathbf{f}_\mathcal{D})\ \mathrm{d}\mathbf{f}_\mathcal{D}  \\
			\displaystyle = \int p(\mathbf{f}_\mathcal{D}|\mathbf{s}_\mathcal{I},\mathbf{\Lambda},\sigma_f)
			\Bigg(-\frac{|\mathcal{D}|}{2}\log2\pi-\frac{1}{2}\log|\mathbf{C}_\mathcal{DD}|\\
			\displaystyle\quad\qquad\qquad\qquad\qquad-\frac{1}{2}(\mathbf{y}_\mathcal{D}-\mathbf{f}_\mathcal{D})^\top\mathbf{C}_\mathcal{DD}^{-1}(\mathbf{y}_\mathcal{D}-\mathbf{f}_\mathcal{D})\Bigg) \mathrm{d}\mathbf{f}_\mathcal{D} \vspace{1mm}\\
			\displaystyle = -\frac{|\mathcal{D}|}{2}\log2\pi-\frac{1}{2}\log|\mathbf{C}_\mathcal{DD}|\vspace{1mm}\\
			\displaystyle\quad-\mathbb{E}_{p(\mathbf{f}_\mathcal{D}|\mathbf{s}_\mathcal{I},\mathbf{\Lambda},\sigma_f)}\left[\frac{1}{2}(\mathbf{y}_\mathcal{D}-\mathbf{f}_\mathcal{D})^\top\mathbf{C}_\mathcal{DD}^{-1}(\mathbf{y}_\mathcal{D}-\mathbf{f}_\mathcal{D})\right] \vspace{1mm}\\
			\displaystyle = -\frac{|\mathcal{D}|}{2}\log2\pi-\frac{1}{2}\log|\mathbf{C}_\mathcal{DD}|\\
			\displaystyle\quad-\frac{1}{2}(\mathbf{y}_\mathcal{D}-\mathbf{K}_\mathcal{DI}\mathbf{\Sigma}_\mathcal{II}^{-1}\mathbf{s}_\mathcal{I})^\top\mathbf{C}_\mathcal{DD}^{-1}(\mathbf{y}_\mathcal{D}-\mathbf{K}_\mathcal{DI}\mathbf{\Sigma}_\mathcal{II}^{-1}\mathbf{s}_\mathcal{I})\vspace{1mm}\\
			\displaystyle\quad-\frac{1}{2}\mathrm{Tr}[\mathbf{C}_\mathcal{DD}^{-1}\mathbf{K}_\mathcal{DD}]+\frac{1}{2}\mathrm{Tr}[\mathbf{C}_\mathcal{DD}^{-1}\mathbf{K}_\mathcal{DI}\mathbf{\Sigma}_\mathcal{II}^{-1}\mathbf{K}_\mathcal{ID}]
		\end{array}			
	\end{equation*}
	where the last equality follows from eq.~$380$ of~\cite{IMM2012-03274} and~\eqref{Modified Gaussian Process Model}.
	%
	%the expectation $\mathbb{E}[\frac{1}{2}(\mathbf{y}_\mathcal{D}-\mathbf{f}_\mathcal{D})^\top\mathbf{C}_\mathcal{DD}^{-1}(\mathbf{y}_\mathcal{D}-\mathbf{f}_\mathcal{D})]_{p(\mathbf{f}_\mathcal{D}|\mathbf{s}_\mathcal{I},\mathbf{\Lambda})}$ follows the expectation of the square forms given that $		p(\mathbf{f}_\mathcal{D}|\mathbf{s}_\mathcal{I},\mathbf{\Lambda})=\mathcal{N}(\mathbf{f}_\mathcal{D}|\mathbf{\mathbf{K}}_\mathcal{DI}\mathbf{\Sigma}_\mathcal{II}^{-1}\mathbf{s}_\mathcal{I},\mathbf{K}_\mathcal{DD}-\mathbf{K}_\mathcal{DI}\mathbf{\Sigma}_\mathcal{II}^{-1}\mathbf{K}_\mathcal{ID})$\footnote{For details please refer to the equation (380) in \cite{IMM2012-03274}}.\\
	
	The closed-form expression of $\mathcal{G}(q, \mathbf{s}_\mathcal{I})$ can then be derived as follows:
	\begin{equation*}
		\hspace{-1.7mm}{}
		\begin{array}{l}
			\mathcal{G}(q, \mathbf{s}_\mathcal{I})\\
			\displaystyle=\int q(\sigma_f)\ q(\mathbf{\Lambda})\ \mathcal{H}(\mathbf{s}_\mathcal{I},\mathbf{\Lambda},\sigma_f)\ \mathrm{d}\mathbf{\Lambda}\ \mathrm{d}\sigma_f\\
			\displaystyle=-\frac{|\mathcal{D}|}{2}\log2\pi-\frac{1}{2}\log|\mathbf{C}_\mathcal{DD}|\vspace{1mm}\\
			\displaystyle\hspace{-2mm} -\frac{1}{2}\mathbb{E}_{q(\mathbf{\Lambda},\sigma_f)}\left[(\mathbf{y}_\mathcal{D}-\mathbf{K}_\mathcal{DI}\mathbf{\Sigma}_\mathcal{II}^{-1}\mathbf{s}_\mathcal{I})^\top\mathbf{C}_\mathcal{DD}^{-1}(\mathbf{y}_\mathcal{D}-\mathbf{K}_\mathcal{DI}\mathbf{\Sigma}_\mathcal{II}^{-1}\mathbf{s}_\mathcal{I})\right]\vspace{1mm}\\
			\displaystyle \hspace{-2mm}-\frac{1}{2}\mathrm{Tr}[\mathbf{C}_\mathcal{DD}^{-1}\mathbb{E}_{q(\mathbf{\Lambda},\sigma_f)}[\mathbf{K}_\mathcal{DD}]]\hspace{-0.5mm} +\hspace{-0.5mm}\frac{1}{2}\mathrm{Tr}[\mathbf{\Sigma}_\mathcal{II}^{-1}\mathbb{E}_{q(\mathbf{\Lambda},\sigma_f)}[\mathbf{K}_\mathcal{ID}\mathbf{C}_\mathcal{DD}^{-1}\mathbf{K}_\mathcal{DI}]] \vspace{1mm}\\
			\displaystyle=-\frac{|\mathcal{D}|}{2}\log2\pi-\frac{1}{2}\log|\mathbf{C}_\mathcal{DD}|-\frac{1}{2}\mathbf{y}_\mathcal{D}^\top\mathbf{C}_\mathcal{DD}^{-1}\mathbf{y}_\mathcal{D}\vspace{1mm}\\
			\displaystyle\quad +\mathbf{s}_\mathcal{I}^\top\mathbf{\Sigma}_\mathcal{II}^{-1}\mathbf{\Omega}_\mathcal{ID}\mathbf{C}_\mathcal{DD}^{-1}\mathbf{y}_\mathcal{D}-\frac{1}{2}\mathbf{s}_\mathcal{I}^\top\mathbf{\Sigma}_\mathcal{II}^{-1}\mathbf{\Psi}_\mathcal{II}\mathbf{\Sigma}_\mathcal{II}^{-1}\mathbf{s}_\mathcal{I}\vspace{1mm}\\
			\displaystyle\quad -\frac{1}{2}\mathrm{Tr}[\mathbf{C}_\mathcal{DD}^{-1}\mathbf{\Upsilon}_\mathcal{DD}] +\frac{1}{2}\mathrm{Tr}[\mathbf{\Sigma}_\mathcal{II}^{-1}\mathbf{\Psi}_\mathcal{II}] 
		\end{array}
	\end{equation*}
	such that the last equality follows from
	\begin{equation*}
		\hspace{-1.7mm}
		\begin{array}{l}
			\displaystyle\mathbb{E}_{q(\mathbf{\Lambda},\sigma_f)}\left[(\mathbf{y}_\mathcal{D}-\mathbf{K}_\mathcal{DI}\mathbf{\Sigma}_\mathcal{II}^{-1}\mathbf{s}_\mathcal{I})^\top\mathbf{C}_\mathcal{DD}^{-1}(\mathbf{y}_\mathcal{D}-\mathbf{K}_\mathcal{DI}\mathbf{\Sigma}_\mathcal{II}^{-1}\mathbf{s}_\mathcal{I})\right] \vspace{1mm}\\
			\displaystyle=\mathbb{E}_{q(\mathbf{\Lambda},\sigma_f)}\left[(\mathbf{y}_\mathcal{D}^\top-\mathbf{s}_\mathcal{I}^\top\mathbf{\Sigma}_\mathcal{II}^{-1}\mathbf{K}_\mathcal{ID})\mathbf{C}_\mathcal{DD}^{-1}(\mathbf{y}_\mathcal{D}-\mathbf{K}_\mathcal{DI}\mathbf{\Sigma}_\mathcal{II}^{-1}\mathbf{s}_\mathcal{I})\right] \vspace{1mm}\\
			\displaystyle=\mathbf{y}_\mathcal{D}^\top\mathbf{C}_\mathcal{DD}^{-1}\mathbf{y}_\mathcal{D}\hspace{-0.5mm}-\hspace{-0.5mm}2\mathbf{s}_\mathcal{I}^\top\mathbf{\Sigma}_\mathcal{II}^{-1}\mathbf{\Omega}_\mathcal{ID}\mathbf{C}_\mathcal{DD}^{-1}\mathbf{y}_\mathcal{D}\hspace{-0.5mm}+\hspace{-0.5mm}\mathbf{s}_\mathcal{I}^\top\mathbf{\Sigma}_\mathcal{II}^{-1}\mathbf{\Psi}_\mathcal{II}\mathbf{\Sigma}_\mathcal{II}^{-1}\mathbf{s}_\mathcal{I}.	
			%&Tr\Big(\mathbb{E}_{\mathbf{\Lambda}}(\mathbf{K}_{nm }\mathbf{K}_{mm}^{-1}\mathbf{K}_{mn})\Big)=  Tr\Big(\mathbf{K}_{mm}^{-1}\mathbf{\Psi}\Big) \\
		\end{array}
	\end{equation*}
	The closed-form expression of $\mathcal{F}(q)$ is \vspace{-2mm}
	\begin{equation*}
		%\hspace{-1.7mm}
		\vspace{-2mm}
		\begin{array}{rcl}
				\mathcal{F}(q)&\hspace{-2.4mm}=&\hspace{-2.4mm}\displaystyle\int q(\mathbf{s}_\mathcal{I})\ \mathcal{G}(q,\mathbf{s}_\mathcal{I})\ \mathrm{d}\mathbf{s}_\mathcal{I}
				=\displaystyle\mathbb{E}_{q(\mathbf{s}_\mathcal{I})}[\mathcal{G}(q,\mathbf{s}_\mathcal{I})] 
		\end{array}
	\end{equation*}
	\vspace{-1mm}
	where, using $q(\mathbf{s}_\mathcal{I})=\mathcal{N}(\mathbf{m},\mathbf{S})$,%\vspace{-1mm}
	\begin{equation*}
		\hspace{-1.7mm}
		\begin{array}{l}
			\displaystyle\mathbb{E}_{q(\mathbf{s}_\mathcal{I})}[\mathcal{G}(q,\mathbf{s}_\mathcal{I})] \vspace{1mm}\\
			\displaystyle=-\frac{|\mathcal{D}|}{2}\log2\pi-\frac{1}{2}\log|\mathbf{C}_\mathcal{DD}|-\frac{1}{2}\mathbf{y}_\mathcal{D}^\top\mathbf{C}_\mathcal{DD}^{-1}\mathbf{y}_\mathcal{D}\vspace{1mm}\\
			\displaystyle\quad+\mathbf{m}^\top\mathbf{\Sigma}_\mathcal{II}^{-1}\mathbf{\Omega}_\mathcal{ID}\mathbf{C}_\mathcal{DD}^{-1}\mathbf{y}_\mathcal{D}-\frac{1}{2}\mathbf{m}^\top\mathbf{\Sigma}_\mathcal{II}^{-1}\mathbf{\Psi}_\mathcal{II}\mathbf{\Sigma}_\mathcal{II}^{-1}\mathbf{m}\vspace{1mm}\\
			\displaystyle\quad-\frac{1}{2}\mathrm{Tr}[\mathbf{S}\mathbf{\Sigma}_\mathcal{II}^{-1}\mathbf{\Psi}_\mathcal{II}\mathbf{\Sigma}_\mathcal{II}^{-1}]\vspace{1mm}\\
			\displaystyle\quad-\frac{1}{2}\mathrm{Tr}[\mathbf{C}_\mathcal{DD}^{-1}\mathbf{\Upsilon}_\mathcal{DD}]+\frac{1}{2}\mathrm{Tr}[\mathbf{\Sigma}_\mathcal{II}^{-1}\mathbf{\Psi}_\mathcal{II}] \vspace{1mm}\\
			
		\end{array}
	\end{equation*}
%	\vspace{-1mm}
	such that $\mathbb{E}_{q(\mathbf{s}_\mathcal{I})}[\mathcal{G}(q,\mathbf{s}_\mathcal{I})]$ is derived using eqs.~$374$ and~$380$ of~\cite{IMM2012-03274}. Since \vspace{-1mm} 
	\begin{equation*}
		\int q(\mathbf{s}_\mathcal{I})\log\dfrac{p(\mathbf{s}_\mathcal{I})}{q(\mathbf{s}_\mathcal{I})}\ \mathrm{d}\mathbf{s}_\mathcal{I}=\mathbb{E}_{q(\mathbf{s}_\mathcal{I})}[\log p(\mathbf{s}_\mathcal{I})]+\mathbb{H}[q(\mathbf{s}_\mathcal{I})]
	\end{equation*}
	\vspace{-1mm}
	where
%	\vspace{-1mm}
	\begin{equation*}
	\hspace{-1.7mm}
	\begin{array}{l}
		\displaystyle\mathbb{E}_{q(\mathbf{s}_\mathcal{I})}[\log p(\mathbf{s}_\mathcal{I})]\vspace{1mm}\\
		\displaystyle=-\frac{|\mathcal{I}|}{2}\log2\pi-\frac{1}{2}\log|\mathbf{\Sigma}_\mathcal{II}|-\frac{1}{2}\mathbf{m}^\top\mathbf{\Sigma}_\mathcal{II}^{-1}\mathbf{m}-\frac{1}{2}\mathrm{Tr}[\mathbf{S}\mathbf{\Sigma}_\mathcal{II}^{-1}]
	\end{array}
	\end{equation*}
%	\vspace{-1mm}
and 
%\vspace{-1mm}
$$
\displaystyle\mathbb{H}[q(\mathbf{s}_\mathcal{I})]=\frac{|\mathcal{I}|}{2}\log2\pi+\frac{|\mathcal{I}|}{2}+\frac{1}{2}\log|\mathbf{S}|
$$
\vspace{-1mm}
denotes a Gaussian entropy with respect to $q(\mathbf{s}_\mathcal{I})$,	
%\vspace{-1mm}
$$
\displaystyle\int q(\mathbf{\Lambda})\log\dfrac{p(\mathbf{\Lambda})}{q(\mathbf{\Lambda})}d\mathbf{\Lambda}=-\dfrac{1}{2}\boldsymbol{\nu}^\top\boldsymbol{\nu}-\dfrac{1}{2}\mathrm{Tr}[\mathbf{\Xi}]+\dfrac{1}{2}\log|\mathbf{\Xi}|+\dfrac{d}{2}\ ,
$$
%\vspace{-1mm}	
and
%\vspace{-1mm}
$$	
\displaystyle \int q(\sigma_f)\log\dfrac{p(\sigma_f)}{q(\sigma_f)}\ \mathrm{d}\sigma_f=-\dfrac{1}{2}\alpha^2-\dfrac{1}{2}\beta+\dfrac{1}{2}\log \beta+\dfrac{1}{2}\ ,
$$
	\begin{equation*}
		\hspace{-1.7mm}
		\begin{array}{l}
			\displaystyle\hspace{5mm}\mathcal{L}(q)\vspace{0mm}\\
			\displaystyle= \mathcal{F}(q)+\int q(\mathbf{s}_\mathcal{I})\log\dfrac{p(\mathbf{s}_\mathcal{I})}{q(\mathbf{s}_\mathcal{I})}\ \mathrm{d}\mathbf{s}_\mathcal{I}+
			\int q(\mathbf{\Lambda})\log\dfrac{p(\mathbf{\Lambda})}{q(\mathbf{\Lambda})}\ \mathrm{d}\mathbf{\Lambda}\ \vspace{-0.5mm}\\
			\displaystyle\quad+\int q(\sigma_f)\log\dfrac{p(\sigma_f)}{q(\sigma_f)}\ \mathrm{d}\sigma_f\vspace{0.5mm}\\
			\displaystyle=-\frac{|\mathcal{D}|}{2}\log2\pi-\frac{1}{2}\log|\mathbf{C}_\mathcal{DD}|-\frac{1}{2}\mathbf{y}_\mathcal{D}^\top\mathbf{C}_\mathcal{DD}^{-1}\mathbf{y}_\mathcal{D}\vspace{1mm}\\
			\displaystyle\quad+\mathbf{m}^\top\mathbf{\Sigma}_\mathcal{II}^{-1}\mathbf{\Omega}_\mathcal{ID}\mathbf{C}_\mathcal{DD}^{-1}\mathbf{y}_\mathcal{D}-\frac{1}{2}\mathbf{m}^\top(\mathbf{\Sigma}_\mathcal{II}^{-1}\mathbf{\Psi}_\mathcal{II}\mathbf{\Sigma}_\mathcal{II}^{-1}+\mathbf{\Sigma}_\mathcal{II}^{-1})\mathbf{m}\vspace{1mm}\\
			\displaystyle\quad-\frac{1}{2}\mathrm{Tr}[\mathbf{S}(\mathbf{\Sigma}_\mathcal{II}^{-1}\mathbf{\Psi}_\mathcal{II}\mathbf{\Sigma}_\mathcal{II}^{-1}+ \mathbf{\Sigma}_\mathcal{II}^{-1})]-\frac{1}{2}\mathrm{Tr}[\mathbf{C}_\mathcal{DD}^{-1}\mathbf{\Upsilon}_\mathcal{DD}]\vspace{1mm}\\
			\displaystyle\quad+\frac{1}{2}\mathrm{Tr}[\mathbf{\Sigma}_\mathcal{II}^{-1}\mathbf{\Psi}_\mathcal{II}]-\frac{1}{2}\log|\mathbf{\Sigma}_\mathcal{II}|+\frac{|\mathcal{I}|}{2}+\frac{1}{2}\log|\mathbf{S}|\vspace{1mm}\\
			\displaystyle \quad-\dfrac{1}{2}\boldsymbol{\nu}^\top\boldsymbol{\nu}-\dfrac{1}{2}\mathrm{Tr}[\mathbf{\Xi}]+\dfrac{1}{2}\log|\mathbf{\Xi}|+\dfrac{d}{2}-\dfrac{1}{2}\alpha^2-\dfrac{1}{2}\beta\vspace{0.5mm}\\
			\quad+\dfrac{1}{2}\log \beta+\dfrac{1}{2}\vspace{0.5mm}\\
			\displaystyle\hspace{-0.5mm}=\hspace{-0.5mm} \dfrac{1}{2}\hspace{-0.5mm}\Big(\hspace{-0.5mm}2\mathbf{m}^\top\mathbf{\Sigma}_\mathcal{II}^{-1}\mathbf{\Omega}_\mathcal{ID}\mathbf{C}_\mathcal{DD}^{-1}\mathbf{y}_\mathcal{D}\hspace{-0.5mm}-\hspace{-0.5mm}\mathbf{m}^\top(\mathbf{\Sigma}_\mathcal{II}^{-1}\mathbf{\Psi}_\mathcal{II}\mathbf{\Sigma}_\mathcal{II}^{-1}+\mathbf{\Sigma}_\mathcal{II}^{-1})\mathbf{m}\vspace{0.5mm}\\
			\displaystyle\quad-\mathrm{Tr}[\mathbf{S}(\mathbf{\Sigma}_\mathcal{II}^{-1}\mathbf{\Psi}_\mathcal{II}\mathbf{\Sigma}_\mathcal{II}^{-1}+ \mathbf{\Sigma}_\mathcal{II}^{-1})]-\mathrm{Tr}[\mathbf{C}_\mathcal{DD}^{-1}\mathbf{\Upsilon}_\mathcal{DD}]\vspace{1mm}\\
			\displaystyle\quad+\mathrm{Tr}[\mathbf{\Sigma}_\mathcal{II}^{-1}\mathbf{\Psi}_\mathcal{II}]+\log|\mathbf{S}|-\boldsymbol{\nu}^\top\boldsymbol{\nu}-\mathrm{Tr}[\mathbf{\Xi}]+\log|\mathbf{\Xi}|\vspace{1mm}\\
			\displaystyle\quad-\alpha^2-\beta
			+\log \beta\Big)+ \mathrm{const}
		\end{array}
	\end{equation*}
	where $\mathrm{const}$ absorbs all terms independent of $\mathbf{m}$, $\mathbf{S}$, $\boldsymbol{\nu}$, $\mathbf{\Xi},\alpha,\beta$.
	Then, by setting
	\vspace{-1mm}
	\begin{equation*}
		\begin{array}{rcl}
			\dfrac{\partial\mathcal{L}}{\partial\mathbf{m}}&\hspace{-2.4mm}=&\hspace{-2.4mm}\displaystyle\mathbf{\Sigma}_\mathcal{II}^{-1}\mathbf{\Omega}_\mathcal{ID}\mathbf{C}_\mathcal{DD}^{-1}\mathbf{y}_\mathcal{D}-(\mathbf{\Sigma}_\mathcal{II}^{-1}\mathbf{\Psi}_\mathcal{II}\mathbf{\Sigma}_\mathcal{II}^{-1}+\mathbf{\Sigma}_\mathcal{II}^{-1})\mathbf{m}\ ,\vspace{1mm} \\
			\dfrac{\partial\mathcal{L}}{\partial\mathbf{S}}&\hspace{-2.4mm}=&\hspace{-2.4mm}\displaystyle\dfrac{1}{2}\mathbf{S}^{-1}-\dfrac{1}{2}(\mathbf{\Sigma}_\mathcal{II}^{-1}\mathbf{\Psi}_\mathcal{II}\mathbf{\Sigma}_\mathcal{II}^{-1}+\mathbf{\Sigma}_\mathcal{II}^{-1})
		\end{array}\vspace{0mm}
	\end{equation*}
	to zero, it can be derived that $\mathcal{L}(q)$ is maximized at $q^*(\mathbf{s}_\mathcal{I})=\mathcal{N}(\mathbf{m}^*,\mathbf{S}^*)$ where
	%\vspace{-1mm}
	\begin{equation}
		\begin{array}{rcl}
			\mathbf{m}^*&\hspace{-2.4mm}=&\hspace{-2.4mm}\displaystyle(\mathbf{\Sigma}_\mathcal{II}^{-1}\mathbf{\Psi}_\mathcal{II}\mathbf{\Sigma}_\mathcal{II}^{-1}+\mathbf{\Sigma}_\mathcal{II}^{-1})^{-1}\mathbf{\Sigma}_\mathcal{II}^{-1}\mathbf{\Omega}_\mathcal{ID}\mathbf{C}_\mathcal{DD}^{-1}\mathbf{y}_\mathcal{D}\ , \vspace{1mm} \\
			\mathbf{S}^*&\hspace{-2.4mm}=&\hspace{-2.4mm}\displaystyle(\mathbf{\Sigma}_\mathcal{II}^{-1}\mathbf{\Psi}_\mathcal{II}\mathbf{\Sigma}_\mathcal{II}^{-1}+\mathbf{\Sigma}_\mathcal{II}^{-1})^{-1}\ .
		\end{array}
		\label{happy}
	\end{equation}
	%\vspace{-1mm}
	By substituting	
	%\vspace{-1mm}
	\begin{equation*}
		%\hspace{-1.7mm}
		\begin{array}{l}
			\displaystyle(\mathbf{\Sigma}_\mathcal{II}^{-1}\mathbf{\Psi}_\mathcal{II}\mathbf{\Sigma}_\mathcal{II}^{-1}+\mathbf{\Sigma}_\mathcal{II}^{-1})^{-1}\vspace{1mm}\\
			\displaystyle=((\mathbf{I}+\mathbf{\Sigma}_\mathcal{II}^{-1}\mathbf{\Psi}_\mathcal{II})\mathbf{\Sigma}_\mathcal{II}^{-1})^{-1}\vspace{1mm}\\
			\displaystyle=\mathbf{\Sigma}_\mathcal{II}(\mathbf{I}+\mathbf{\Sigma}_\mathcal{II}^{-1}\mathbf{\Psi}_\mathcal{II})^{-1}\vspace{1mm}\\
			\displaystyle=\mathbf{\Sigma}_\mathcal{II}(\mathbf{\Sigma}_\mathcal{II}^{-1}(\mathbf{\Sigma}_\mathcal{II}+\mathbf{\Psi}_\mathcal{II}))^{-1}\vspace{1mm}\\
			\displaystyle=\mathbf{\Sigma}_\mathcal{II}(\mathbf{\Sigma}_\mathcal{II}+\mathbf{\Psi}_\mathcal{II})^{-1}\mathbf{\Sigma}_\mathcal{II}
		\end{array}
	\end{equation*}
	%\vspace{-1mm}
	into~\eqref{happy},~\eqref{q(u)} in Theorem~\ref{thm1} results.
	%\vspace{-1mm}
	Using~\eqref{q(u)},
%	\vspace{-1mm}
	%
	\begin{equation*}
		\begin{array}{l}
			\displaystyle\mathbf{m}^{*\top}\mathbf{\Sigma}_\mathcal{II}^{-1}\mathbf{\Omega}_\mathcal{ID}\mathbf{C}_\mathcal{DD}^{-1}\mathbf{y}_\mathcal{D}\vspace{1mm}\\
			\displaystyle=\mathbf{y}_\mathcal{D}^\top\mathbf{C}_\mathcal{DD}^{-1}\mathbf{\Omega}_\mathcal{ID}^\top(\mathbf{\Sigma}_\mathcal{II}+\mathbf{\Psi}_\mathcal{II})^{-1}\mathbf{\Omega}_\mathcal{ID}\mathbf{C}_\mathcal{DD}^{-1}\mathbf{y}_\mathcal{D}\ , \vspace{2mm}\\
			\displaystyle\mathbf{m}^{*\top}\big(\mathbf{\Sigma}_\mathcal{II}^{-1}\mathbf{\Psi}_\mathcal{II}\mathbf{\Sigma}_\mathcal{II}^{-1}+\mathbf{\Sigma}_\mathcal{II}^{-1}\big)\mathbf{m}^*\vspace{1mm}\\
			\displaystyle=\mathbf{y}_\mathcal{D}^\top\mathbf{C}_\mathcal{DD}^{-1}\mathbf{\Omega}_\mathcal{ID}^\top(\mathbf{\Sigma}_\mathcal{II}+\mathbf{\Psi}_\mathcal{II})^{-1}\mathbf{\Omega}_\mathcal{ID}\mathbf{C}_\mathcal{DD}^{-1}\mathbf{y}_\mathcal{D}\ , \vspace{2mm}\\
			\displaystyle\mathrm{Tr}(\mathbf{S}^*(\mathbf{\Sigma}_\mathcal{II}^{-1}\mathbf{\Psi}_\mathcal{II}\mathbf{\Sigma}_\mathcal{II}^{-1}+\mathbf{\Sigma}_\mathcal{II}^{-1}))=|\mathcal{I}|
		\end{array}
	\end{equation*}
	\vspace{-1mm}
	which reduce $\mathcal{L}(q)$ to\vspace{-1mm}
	\begin{equation}
		%\hspace{-1.7mm}
		\begin{array}{l}
			\mathcal{L}(q)=
			\displaystyle\frac{1}{2}\Big(\mathbf{y}_\mathcal{D}^\top\mathbf{C}_\mathcal{DD}^{-1}\mathbf{\Omega}_\mathcal{ID}^\top(\mathbf{\Sigma}_\mathcal{II}+\mathbf{\Psi}_\mathcal{II})^{-1}\mathbf{\Omega}_\mathcal{ID}\mathbf{C}_\mathcal{DD}^{-1}\mathbf{y}_\mathcal{D}\vspace{1mm}\\
			\displaystyle-\mathrm{Tr}[\mathbf{C}_\mathcal{DD}^{-1}\mathbf{\Upsilon}_\mathcal{DD}]+\mathrm{Tr}[\mathbf{\Sigma}_\mathcal{II}^{-1}\mathbf{\Psi}_\mathcal{II}]-\log|\mathbf{\Sigma}_\mathcal{II}+\mathbf{\Psi}_\mathcal{II}|\vspace{1mm}\\
			\displaystyle-\boldsymbol{\nu}^\top\boldsymbol{\nu}-\mathrm{Tr}[\mathbf{\Xi}]+\log|\mathbf{\Xi}| -\alpha^2-\beta +\log \beta\Big)+\mathrm{const}\ .
		\end{array} \vspace{-1mm}
		\label{eq:TitsiasL(q)}
	\end{equation}


	\vspace{-5mm}
	\subsection{Derivation of $\mathbf{\Omega}_\mathcal{ID}$, $\mathbf{\Psi}_\mathcal{II}$, and $\mathbf{\Upsilon}_\mathcal{DD}$}
	\label{Derivation of Omega, Psi and Upsilon}	
	%
	Let $\mathbf{\Omega}_\mathcal{ID}\triangleq (\omega_{\mathbf{z}\mathbf{x}})_{\mathbf{z}\in\mathcal{I},\mathbf{x}\in\mathcal{D}}$, $\mathbf{z}\triangleq (z_1,\ldots,z_d)^\top$, and $\mathbf{x}\triangleq (x_1,\ldots,x_d)^\top$. Since $\mathbf{\Omega}_\mathcal{ID}\hspace{-0mm}\triangleq\hspace{-0mm}\mathbb{E}_{q(\mathbf{\Lambda},\sigma_f)}\hspace{-0mm}(\mathbf{K}_{\mathcal{ID}})$, 
	\begin{equation*}
		\hspace{-1.7mm}
		\begin{array}{l}
			\omega_{\mathbf{z}\mathbf{x}}\\
			\displaystyle=\int q(\sigma_f)\ q(\mathbf{\Lambda})\ \mathrm{cov}[s_\mathbf{z},f_\mathbf{x}]\ \mathrm{d}\mathbf{\Lambda}\ \mathrm{d}\sigma_f\\
			\displaystyle=\int q(\sigma_f)\left(\int q(\mathbf{\Lambda})\ \sigma_f \exp\hspace{-0.7mm}\left(\hspace{-0.7mm}-\dfrac{1}{2}\sum_{k=1}^{d}(\lambda_k x_{k}-{z}_{k})^2\hspace{-0.7mm}\right) \hspace{-0.7mm}\mathrm{d}\mathbf{\Lambda}\hspace{-0.7mm}\right)\hspace{-0.7mm}\mathrm{d}\sigma_f \vspace{1mm}\\
			\displaystyle=\int q(\sigma_f)\ \sigma_f\\
			\displaystyle\quad \prod_{k=1}^{d}\int \exp\hspace{-0.7mm}\left(-\dfrac{1}{2}\sum_{k=1}^{d}(\lambda_k x_{k}-{z}_{k})^2\right)\mathcal{N}(\lambda_k|\nu_k,\xi_k)\ \mathrm{d}\lambda_k\ \mathrm{d}\sigma_f \\
			\displaystyle =\int q(\sigma_f)\ \sigma_f \prod_{k=1}^{d}{(\xi_kx_{k}^2+1)^{-\frac{1}{2}}}\exp\left(-\frac{(x_{k}\nu_k-z_{k})^2}{2(\xi_kx_{k}^2+1)}\right)\mathrm{d}\sigma_f \\
			\displaystyle =\alpha \prod_{k=1}^{d}{(\xi_kx_{k}^2+1)^{-\frac{1}{2}}}\exp\left(-\frac{(x_{k}\nu_k-z_{k})^2}{2(\xi_kx_{k}^2+1)}\right).
		\end{array}
	\end{equation*}
%	
\noindent
	Since $\mathbf{C}_\mathcal{DD}$ is a block-diagonal matrix constructed using the $B$ blocks $\mathbf{C}_{\mathcal{D}_i\mathcal{D}_i}$ for $i=1,\ldots,B$,
	$\mathbf{C}^{-1}_\mathcal{DD}$ is also a block-diagonal matrix constructed using the $B$ blocks $\mathbf{C}^{-1}_{\mathcal{D}_i\mathcal{D}_i}$ for $i=1,\ldots,B$.
	Let $\mathbf{C}^{-1}_{\mathcal{D}_i\mathcal{D}_i}\triangleq (c^{i}_{\mathbf{x}\mathbf{x}'})_{\mathbf{x},\mathbf{x}'\in\mathcal{D}_i}$.
%	
	Let $\mathbf{\Psi}_\mathcal{II}\triangleq (\psi_{\mathbf{z}\mathbf{z}'})_{\mathbf{z},\mathbf{z}'\in\mathcal{I}}$, $\mathbf{z}'\triangleq (z'_1,\ldots,z'_d)^\top$, and $\mathbf{x}'\triangleq (x'_1,\ldots,x'_d)^\top$.
	Since $\mathbf{\Psi}_\mathcal{II}\triangleq\mathbb{E}_{q(\mathbf{\Lambda},\sigma_f)}(\mathbf{K}_{\mathcal{ID}}\mathbf{C}^{-1}_\mathcal{DD}\mathbf{K}_{\mathcal{DI}})$,
	%
	\vspace{-4mm}
	\begin{equation*}
		\hspace{-1.7mm}
		\begin{array}{l}
			\psi_{\mathbf{z}\mathbf{z}'} \vspace{-1mm}\\
			\displaystyle=\hspace{-1mm}\int \hspace{-1mm} q(\sigma_f)\hspace{-0.5mm} \sum_{i=1}^{B}\hspace{-0.5mm}\sum_{\mathbf{x},\mathbf{x}'\in\mathcal{D}_i}\hspace{-0.5mm}\mathbb{E}_{\mathbf{\Lambda}}\left[\mathrm{cov}[s_\mathbf{z},f_\mathbf{x}]\ c^{i}_{\mathbf{x}\mathbf{x}'}\ \mathrm{cov}[f_{\mathbf{x}'},s_{\mathbf{z}'}] \right]\hspace{-0.5mm} \mathrm{d}\sigma_f \\
			\displaystyle=\hspace{-1mm}\int\hspace{-1mm} q(\sigma_f) \sum_{i=1}^{B}\hspace{-0.5mm}\sum_{\mathbf{x},\mathbf{x}'\in\mathcal{D}_i}\hspace{-0.5mm}c^{i}_{\mathbf{x}\mathbf{x}'}\ \mathbb{E}_{\mathbf{\Lambda}}\left[\mathrm{cov}[s_\mathbf{z},f_\mathbf{x}]\ \mathrm{cov}[f_{\mathbf{x}'},s_{\mathbf{z}'}] \right]\hspace{-0.5mm} \mathrm{d}\sigma_f \\
			\displaystyle=\int q(\sigma_f)\sum_{i=1}^{B}\sum_{\mathbf{x},\mathbf{x}'\in\mathcal{D}_i}\sigma_f^2\ c^{i}_{\mathbf{x}\mathbf{x}'}\prod_{k=1}^{d}\Bigg\{(\xi_k(x_{k}^2+x_{k}^{\prime2})+1)^{-\frac{1}{2}} \vspace{-1mm}\\
			\quad\exp\left(-\frac{\xi_k(z'_{k}x_{k}-z_{k}x'_{k})^2+(x_{k}\nu_k-z_{k})^2+(x'_{k}\nu_k-z'_{k})^2}{2(\xi_k(x_{k}^2+x_{k}^{\prime 2})+1)}\right)\Bigg\}\ \mathrm{d}\sigma_f \\
			\displaystyle= \sum_{i=1}^{B}\sum_{\mathbf{x},\mathbf{x}'\in\mathcal{D}_i}(\beta+\alpha^2)\ c^{i}_{\mathbf{x}\mathbf{x}'}\prod_{k=1}^{d}\Bigg\{(\xi_k(x_{k}^2+x_{k}^{\prime2})+1)^{-\frac{1}{2}} \vspace{-1mm}\\
			\qquad\quad\exp\left(-\frac{\xi_k(z'_{k}x_{k}-z_{k}x'_{k})^2+(x_{k}\nu_k-z_{k})^2+(x'_{k}\nu_k-z'_{k})^2}{2(\xi_k(x_{k}^2+x_{k}^{\prime 2})+1)}\right)\Bigg\}\ . 
		\end{array}
	\end{equation*}
	\vspace{-0mm}
	Let $\mathbf{\Upsilon}_\mathcal{DD}\triangleq (\gamma_{\mathbf{x}\mathbf{x}'})_{\mathbf{x},\mathbf{x}'\in\mathcal{D}}$. Since $\mathbf{\Upsilon}_\mathcal{DD}\triangleq\mathbb{E}_{q(\mathbf{\Lambda},\sigma_f)}(\mathbf{K}_{\mathcal{DD}})$,
	\vspace{-2mm}
	\begin{equation*}
		\hspace{-1.7mm}
		\begin{array}{l}
			\gamma_{\mathbf{x}\mathbf{x}'}\\
			\displaystyle=\int q(\sigma_f)q(\mathbf{\Lambda})\ k_{\mathbf{x}\mathbf{x}'}\ \mathrm{d}\mathbf{\Lambda}\ \mathrm{d}\sigma_f \vspace{-1mm}\\
			\displaystyle=\hspace{-0.5mm}\int\hspace{-0.5mm} q(\sigma_f)\hspace{-0.5mm}\int \hspace{-0.5mm}q(\mathbf{\Lambda})\sigma_f^2 \exp\left(-\dfrac{1}{2}\sum_{k=1}^{d}\lambda_k^2(x_{k}-x'_{k})^2\right)\hspace{-0.5mm} \mathrm{d}\mathbf{\Lambda} \mathrm{d}\sigma_f \vspace{-1mm}\\
			\displaystyle=\int q(\sigma_f)\ \sigma_f^2\\
			\displaystyle\prod_{k=1}^{d}\int \exp\hspace{-1mm}\left(-\dfrac{1}{2}\sum_{k=1}^{d}\lambda_k^2(x_{k}-x'_{k})^2\right)\mathcal{N}(\lambda_k|\nu_k,\xi_k)\ \mathrm{d}\lambda_k\  \mathrm{d}\sigma_f   \vspace{1mm}\\
			\displaystyle = \int q(\sigma_f)\ \sigma_f^2\\
			\displaystyle\hspace{-0.5mm}\prod_{k=1}^{d}\hspace{-0.5mm}{(\xi_k(x_{k}-x'_{k})^2+1)^{-\frac{1}{2}}}\exp\hspace{-0.7mm}\left(\hspace{-0.7mm}-\frac{\nu_k^2(x_{k}-x'_{k})^2}{2(\xi_k(x_{k}-x'_{k})^2+1)}\right)\hspace{-0.5mm}\mathrm{d}\sigma_f \\
			=\displaystyle (\beta+\alpha^2)\\
			\displaystyle\prod_{k=1}^{d}{(\xi_k(x_{k}-x'_{k})^2+1)^{-\frac{1}{2}}}\exp\hspace{-0.7mm}\left(\hspace{-0.7mm}-\frac{\nu_k^2(x_{k}-x'_{k})^2}{2(\xi_k(x_{k}-x'_{k})^2+1)}\right).
		\end{array}
	\end{equation*}
	%
	%
%
%
%
%
%
%

%
\vspace{-5mm}
\subsection{Proof of Theorem~\ref{thm2}}
	\label{A.3}
	\vspace{-2mm}
	Let
	\vspace{-1mm}
		\begin{equation}
			\hspace{-2.7mm}
			\begin{array}{l}
				\displaystyle\dfrac{\partial\widetilde{\mathcal{L}}}{\partial\mathbf{m}}\hspace{-0.5mm}\triangleq\hspace{-0.5mm}\dfrac{B}{|\mathcal{S}|}\hspace{-0.5mm}\sum_{s\in\mathcal{S}}\hspace{-0.5mm}\dfrac{\partial{\mathcal{L}}_s}{\partial\mathbf{m}}\hspace{-0.5mm}-\hspace{-0.5mm}\mathbf{\Sigma}_\mathcal{II}^{-1}\mathbf{m}\ , 
				\displaystyle\dfrac{\partial\widetilde{\mathcal{L}}}{\partial\mathbf{\Xi}}\hspace{-0.5mm}\triangleq\hspace{-0.5mm}\dfrac{B}{|\mathcal{S}|}\hspace{-0.5mm}\sum_{s\in\mathcal{S}}\hspace{-0.5mm}\dfrac{\partial{\mathcal{L}}_s}{\partial\mathbf{\Xi}}\hspace{-1mm}-\hspace{-0.5mm}\dfrac{1}{2}\mathbf{I}\hspace{-0.5mm}+\hspace{-0.5mm}\dfrac{1}{2}\mathbf{\Xi}^{-1},\vspace{1mm}\\
				\displaystyle\dfrac{\partial{\widetilde{\mathcal{L}}}}{\partial\mathbf{S}}\hspace{-0.5mm}\triangleq\hspace{-0.5mm}\dfrac{B}{|\mathcal{S}|}\hspace{-0.5mm}\sum_{s\in\mathcal{S}}\hspace{-0.5mm}\dfrac{\partial{\mathcal{L}}_s}{\partial\mathbf{S}}\hspace{-0.5mm}+\hspace{-0.5mm}\dfrac{1}{2}\mathbf{S}^{-1}\hspace{-1mm}-\hspace{-0.5mm}\dfrac{1}{2}\mathbf{\Sigma}_\mathcal{II}^{-1}\ , \ 
				\dfrac{\partial\widetilde{\mathcal{L}}}{\partial\boldsymbol{\nu}}\hspace{-0.5mm}\triangleq\hspace{-0.5mm}\dfrac{B}{|\mathcal{S}|}\hspace{-0.5mm}\sum_{s\in\mathcal{S}}\hspace{-0.5mm}\dfrac{\partial{\mathcal{L}}_s}{\partial\boldsymbol{\nu}}\hspace{-0.5mm}-\hspace{-0.5mm}\boldsymbol{\nu},
				\vspace{1mm} \\
				\displaystyle\dfrac{\partial\widetilde{\mathcal{L}}}{\partial \alpha}\hspace{-0.5mm}\triangleq\hspace{-0.5mm}\dfrac{B}{|\mathcal{S}|}\hspace{-0.5mm}\sum_{s\in\mathcal{S}}\dfrac{\partial{\mathcal{L}}_s}{\partial \alpha}\hspace{-0.5mm}-\hspace{-0.5mm}\alpha\ , \ 
				\dfrac{\partial\widetilde{\mathcal{L}}}{\partial \beta}\hspace{-0.5mm}\triangleq\hspace{-0.5mm}\dfrac{B}{|\mathcal{S}|}\hspace{-0.5mm}\sum_{s\in\mathcal{S}}\hspace{-0.5mm}\dfrac{\partial{\mathcal{L}}_s}{\partial \beta}\hspace{-0.5mm}-\hspace{-0.5mm}\dfrac{\beta-1}{2\beta}
			\end{array}
			\label{zoo}
		\end{equation}
		where
		\begin{equation*}
			\hspace{-1.7mm}
			\begin{array}{l}
				\displaystyle\dfrac{\partial{\mathcal{L}}_s}{\partial\mathbf{m}}=\mathbf{\Sigma}_\mathcal{II}^{-1}\mathbf{\Omega}_{\mathcal{I}\mathcal{D}_s}\mathbf{C}_{\mathcal{D}_s\mathcal{D}_s}^{-1}\mathbf{y}_{\mathcal{D}_s}-\mathbf{\Sigma}_\mathcal{II}^{-1}\mathbf{\Psi}_\mathcal{II}^{s}\mathbf{\Sigma}_\mathcal{II}^{-1}\mathbf{m}\ ,\vspace{2mm}\\
				\displaystyle\dfrac{\partial{\mathcal{L}}_s}{\partial\mathbf{S}}=-\dfrac{1}{2}\mathbf{\Sigma}_\mathcal{II}^{-1}\mathbf{\Psi}_\mathcal{II}^{s}\mathbf{\Sigma}_\mathcal{II}^{-1}\ ,\vspace{2mm}\\
				\displaystyle\dfrac{\partial{\mathcal{L}}_s}{\partial\boldsymbol{\nu}}=\mathbf{m}^\top\mathbf{\Sigma}_\mathcal{II}^{-1}\dfrac{\partial\mathbf{\Omega}_{\mathcal{I}\mathcal{D}_s}}{\partial\boldsymbol{\nu}}\mathbf{C}_{\mathcal{D}_s\mathcal{D}_s}^{-1}\mathbf{y}_{\mathcal{D}_s}\hspace{-1mm}-\hspace{-0.5mm}\frac{1}{2}\mathbf{m}^\top\mathbf{\Sigma}_\mathcal{II}^{-1}\dfrac{\partial\mathbf{\Psi}_\mathcal{II}^{s}}{\partial\boldsymbol{\nu}}\mathbf{\Sigma}_\mathcal{II}^{-1}\mathbf{m}\vspace{1mm}\\
				\displaystyle\qquad\quad\ -\frac{1}{2}\mathrm{Tr}\Big[\mathbf{S}\mathbf{\Sigma}_\mathcal{II}^{-1}\dfrac{\partial\mathbf{\Psi}_\mathcal{II}^{s}}{\partial\boldsymbol{\nu}}\mathbf{\Sigma}_\mathcal{II}^{-1}\Big]-\frac{1}{2}\mathrm{Tr}\Big[\mathbf{C}_{\mathcal{D}_s\mathcal{D}_s}^{-1}\dfrac{\partial\mathbf{\Upsilon}_{\mathcal{D}_s\mathcal{D}_s}}{\partial\boldsymbol{\nu}}\Big]\vspace{1mm}\\
				\displaystyle\qquad\quad\ + \frac{1}{2}\mathrm{Tr}\Big[\mathbf{\Sigma}_\mathcal{II}^{-1}\dfrac{\partial\mathbf{\Psi}_\mathcal{II}^{s}}{\partial\boldsymbol{\nu}}\Big]\ ,\vspace{1mm}\\
				\displaystyle\dfrac{\partial{\mathcal{L}}_s}{\partial\mathbf{\Xi}}=\mathbf{m}^\top\mathbf{\Sigma}_\mathcal{II}^{-1}\dfrac{\partial\mathbf{\Omega}_{\mathcal{I}\mathcal{D}_s}}{\partial\mathbf{\Xi}}\mathbf{C}_{\mathcal{D}_s\mathcal{D}_s}^{-1}\mathbf{y}_{\mathcal{D}_s}\hspace{-1mm}-\hspace{-1mm}\frac{1}{2}\mathbf{m}^\top\mathbf{\Sigma}_\mathcal{II}^{-1}\dfrac{\partial\mathbf{\Psi}_\mathcal{II}^{s}}{\partial\mathbf{\Xi}}\mathbf{\Sigma}_\mathcal{II}^{-1}\mathbf{m}\vspace{1mm}\\
				\displaystyle\qquad\quad\ -\frac{1}{2}\mathrm{Tr}\Big[\mathbf{S}\mathbf{\Sigma}_\mathcal{II}^{-1}\dfrac{\partial\mathbf{\Psi}_\mathcal{II}^{s}}{\partial\mathbf{\Xi}}\mathbf{\Sigma}_\mathcal{II}^{-1}\Big]-\frac{1}{2}\mathrm{Tr}\Big[\mathbf{C}_{\mathcal{D}_s\mathcal{D}_s}^{-1}\dfrac{\partial\mathbf{\Upsilon}_{\mathcal{D}_s\mathcal{D}_s}}{\partial\mathbf{\Xi}}\Big]\vspace{1mm}\\
				\displaystyle\qquad\quad\ +\frac{1}{2}\mathrm{Tr}\Big[\mathbf{\Sigma}_\mathcal{II}^{-1}\dfrac{\partial\mathbf{\Psi}_\mathcal{II}^{s}}{\partial\mathbf{\Xi}}\Big]\ ,\vspace{2mm}\\
				%\end{aligned}
				%\end{equation}
				%Since $\mathcal{S}$ is a set of i.i.d samples drawn form a uniform distribution over the index set $\{1,2,\dots,B\}$, so the expectation of the SGA is:
				%\begin{equation}
				%\begin{aligned}
				%&\mathbb{E}\Big[	\dfrac{\partial\check{\mathcal{L}}^+}{\partial\mathbf{m}}\Big]=-\mathbf{K}_{mm}^{-1}\mathbf{m}\\
				%&+\dfrac{B}{|\mathcal{S}|}\mathbb{E}\Bigg[\sum_{i\in\mathcal{S}}\Big\{\mathbf{K}_{mm}^{-1}\mathbf{\Omega}_{\mathcal{D}_i}C_{\mathcal{D}_i}^{-1}\mathbf{y}_{\mathcal{D}_i}- \mathbf{K}_{mm}^{-1}\tilde{\mathbf{\Psi}}_{\mathcal{D}_i}\mathbf{K}_{mm}^{-1}\mathbf{m}\Big\}\Bigg]\\ 
				\displaystyle\dfrac{\partial{\mathcal{L}}_s}{\partial \alpha}=\mathbf{m}^\top\mathbf{\Sigma}_\mathcal{II}^{-1}\dfrac{\partial\mathbf{\Omega}_{\mathcal{I}\mathcal{D}_s}}{\partial \alpha}\mathbf{C}_{\mathcal{D}_s\mathcal{D}_s}^{-1}\mathbf{y}_{\mathcal{D}_s}\hspace{-1mm}-\hspace{-1mm}\frac{1}{2}\mathbf{m}^\top\mathbf{\Sigma}_\mathcal{II}^{-1}\dfrac{\partial\mathbf{\Psi}_\mathcal{II}^{s}}{\partial \alpha}\mathbf{\Sigma}_\mathcal{II}^{-1}\mathbf{m}\vspace{1mm}\\ \displaystyle\qquad\quad\;-\frac{1}{2}\mathrm{Tr}\Big[\mathbf{S}\mathbf{\Sigma}_\mathcal{II}^{-1}\dfrac{\partial\mathbf{\Psi}_\mathcal{II}^{s}}{\partial \alpha}\mathbf{\Sigma}_\mathcal{II}^{-1}\Big] 
				-\frac{1}{2}\mathrm{Tr}\Big[\mathbf{C}_{\mathcal{D}_s\mathcal{D}_s}^{-1}\dfrac{\partial\mathbf{\Upsilon}_{\mathcal{D}_s\mathcal{D}_s}}{\partial \alpha}\Big]\vspace{1mm} \\
				\displaystyle\qquad\quad\;+\frac{1}{2}\mathrm{Tr}\Big[\mathbf{\Sigma}_\mathcal{II}^{-1}\dfrac{\partial\mathbf{\Psi}_\mathcal{II}^{s}}{\partial \alpha}\Big]\vspace{2mm}\ ,\\
				\displaystyle\dfrac{\partial{\mathcal{L}}_s}{\partial \beta}=\mathbf{m}^\top\mathbf{\Sigma}_\mathcal{II}^{-1}\dfrac{\partial\mathbf{\Omega}_{\mathcal{I}\mathcal{D}_s}}{\partial \beta}\mathbf{C}_{\mathcal{D}_s\mathcal{D}_s}^{-1}\mathbf{y}_{\mathcal{D}_s}\hspace{-1mm}-\hspace{-1mm}\frac{1}{2}\mathbf{m}^\top\mathbf{\Sigma}_\mathcal{II}^{-1}\dfrac{\partial\mathbf{\Psi}_\mathcal{II}^{s}}{\partial \beta}\mathbf{\Sigma}_\mathcal{II}^{-1}\mathbf{m}\vspace{1mm} \\
				\displaystyle\qquad\quad\;-\frac{1}{2}\mathrm{Tr}\Big[\mathbf{S}\mathbf{\Sigma}_\mathcal{II}^{-1}\dfrac{\partial\mathbf{\Psi}_\mathcal{II}^{s}}{\partial \beta}\mathbf{\Sigma}_\mathcal{II}^{-1}\Big]- \frac{1}{2}\mathrm{Tr}\Big[\mathbf{C}_{\mathcal{D}_s\mathcal{D}_s}^{-1}\dfrac{\partial\mathbf{\Upsilon}_{\mathcal{D}_s\mathcal{D}_s}}{\partial \beta}\Big]\vspace{1mm}\\ \displaystyle\qquad\quad\;+\frac{1}{2}\mathrm{Tr}\Big[\mathbf{\Sigma}_\mathcal{II}^{-1}\dfrac{\partial\mathbf{\Psi}_\mathcal{II}^{s}}{\partial \beta}\Big]\ ,
			\end{array}
		\end{equation*}	
and the closed-form expressions of $\partial\mathbf{\Omega}_{\mathcal{I}\mathcal{D}_s}/\partial\boldsymbol{\nu}$, $\partial\mathbf{\Psi}_\mathcal{II}^{s}/\partial\boldsymbol{\nu}$, $\partial\mathbf{\Upsilon}_{\mathcal{D}_s\mathcal{D}_s}/\partial\boldsymbol{\nu}$, $\partial\mathbf{\Omega}_{\mathcal{I}\mathcal{D}_s}/\partial\mathbf{\Xi}$, $\partial\mathbf{\Psi}_\mathcal{II}^{s}/\partial\mathbf{\Xi}$,  $\partial\mathbf{\Upsilon}_{\mathcal{D}_s\mathcal{D}_s}/\partial\mathbf{\Xi}$ $\partial\mathbf{\Omega}_{\mathcal{I}\mathcal{D}_s}/\partial \alpha$, $\partial\mathbf{\Psi}_\mathcal{II}^{s}/\partial \alpha$, $\partial\mathbf{\Upsilon}_{\mathcal{D}_s\mathcal{D}_s}/\partial \alpha$, $\partial\mathbf{\Omega}_{\mathcal{I}\mathcal{D}_s}/\partial \beta$, $\partial\mathbf{\Psi}_\mathcal{II}^{s}/\partial \beta$, and $\partial\mathbf{\Upsilon}_{\mathcal{D}_s\mathcal{D}_s}/\partial \beta$ are given in Appendix~\ref{argghh}.
%	

	Then, since
	\begin{equation*}
		\begin{array}{l}
			\displaystyle\mathbb{E}[ \mathbf{\Sigma}_\mathcal{II}^{-1}\mathbf{\Omega}_{\mathcal{I}\mathcal{D}_s}\mathbf{C}_{\mathcal{D}_s\mathcal{D}_s}^{-1}\mathbf{y}_{\mathcal{D}_s}-\mathbf{\Sigma}_\mathcal{II}^{-1}\mathbf{\Psi}_\mathcal{II}^{s}\mathbf{\Sigma}_\mathcal{II}^{-1}\mathbf{m}] \\
			\displaystyle=\sum_{i=1}^{B}p(s=i)(\mathbf{\Sigma}_\mathcal{II}^{-1}\mathbf{\Omega}_{\mathcal{I}\mathcal{D}_i}\mathbf{C}_{\mathcal{D}_i\mathcal{D}_i}^{-1}\mathbf{y}_{\mathcal{D}_i}-\mathbf{\Sigma}_\mathcal{II}^{-1}\mathbf{\Psi}_\mathcal{II}^{i}\mathbf{\Sigma}_\mathcal{II}^{-1}\mathbf{m}) \\
			\displaystyle=\sum_{i=1}^{B}\dfrac{1}{B}(\mathbf{\Sigma}_\mathcal{II}^{-1}\mathbf{\Omega}_{\mathcal{I}\mathcal{D}_i}\mathbf{C}_{\mathcal{D}_i\mathcal{D}_i}^{-1}\mathbf{y}_{\mathcal{D}_i}-\mathbf{\Sigma}_\mathcal{II}^{-1}\mathbf{\Psi}_\mathcal{II}^{i}\mathbf{\Sigma}_\mathcal{II}^{-1}\mathbf{m}) \\
			\displaystyle=\dfrac{1}{B}\sum_{i=1}^{B}\mathbf{\Sigma}_\mathcal{II}^{-1}\mathbf{\Omega}_{\mathcal{I}\mathcal{D}_i}\mathbf{C}_{\mathcal{D}_i\mathcal{D}_i}^{-1}\mathbf{y}_{\mathcal{D}_i}-\mathbf{\Sigma}_\mathcal{II}^{-1}\mathbf{\Psi}_\mathcal{II}^{i}\mathbf{\Sigma}_\mathcal{II}^{-1}\mathbf{m}\ ,
		\end{array}
	\end{equation*}
	\begin{equation*}
		\begin{array}{l}
			\displaystyle\mathbb{E}\left[\sum_{s\in\mathcal{S}} \mathbf{\Sigma}_\mathcal{II}^{-1}\mathbf{\Omega}_{\mathcal{I}\mathcal{D}_s}\mathbf{C}_{\mathcal{D}_s\mathcal{D}_s}^{-1}\mathbf{y}_{\mathcal{D}_s}-\mathbf{\Sigma}_\mathcal{II}^{-1}\mathbf{\Psi}_\mathcal{II}^{s}\mathbf{\Sigma}_\mathcal{II}^{-1}\mathbf{m}\right]\\
			\displaystyle=\dfrac{|\mathcal{S}|}{B}\sum_{i=1}^{B}\mathbf{\Sigma}_\mathcal{II}^{-1}\mathbf{\Omega}_{\mathcal{I}\mathcal{D}_i}\mathbf{C}_{\mathcal{D}_i\mathcal{D}_i}^{-1}\mathbf{y}_{\mathcal{D}_i}-\mathbf{\Sigma}_\mathcal{II}^{-1}\mathbf{\Psi}_\mathcal{II}^{i}\mathbf{\Sigma}_\mathcal{II}^{-1}\mathbf{m}\ .
		\end{array}
	\end{equation*}
	It follows that $\mathbb{E}[\partial\widetilde{\mathcal{L}}/\partial\mathbf{m}]=\partial{\mathcal{L}}/\partial\mathbf{m}$. The proofs for $\mathbb{E}[\partial\widetilde{\mathcal{L}}/\partial\mathbf{S}]=\partial{\mathcal{L}}/\partial\mathbf{S}$, 
	$\mathbb{E}[\partial\widetilde{\mathcal{L}}/\partial\boldsymbol{\nu}]=\partial{\mathcal{L}}/\partial\boldsymbol{\nu}$, 
	$\mathbb{E}[\partial\widetilde{\mathcal{L}}/\partial\mathbf{\Xi}]=\partial{\mathcal{L}}/\partial\mathbf{\Xi}$, 	$\mathbb{E}[\partial\widetilde{\mathcal{L}}/\partial   \alpha]=\partial{\mathcal{L}}/\partial \alpha$, and $\mathbb{E}[\partial\widetilde{\mathcal{L}}/\partial \beta]=\partial{\mathcal{L}}/\partial \beta$ follow a similar procedure as the above.


	\subsection{Derivatives of $\mathbf{\Omega}_{\mathcal{I}\mathcal{D}_s}$, $\mathbf{\Psi}^s_\mathcal{II}$, and $\mathbf{\Upsilon}_{\mathcal{D}_s\mathcal{D}_s}$ with respect to $\boldsymbol{\nu}$, $\mathbf{\Xi}$, $\alpha$, and $\beta$}
	\label{argghh}
	\hspace{2mm}Note that $\boldsymbol{\nu}= (\nu_1,\ldots,\nu_d)^\top$ and $\mathbf{\Xi}=\mathrm{diag}[\xi_1,\ldots,\xi_d]^\top$, as defined previously in Section~\ref{Variational Inference of the Bayesian DTC}.
	
	From Appendix~\ref{Derivation of Omega, Psi and Upsilon},
	\begin{equation*}
		%	\begin{aligned}
		\omega_\mathbf{zx}=\alpha \prod_{k=1}^{d}{(\xi_kx_{k}^2+1)^{-\frac{1}{2}}}\exp\left(-\frac{(x_{k}\nu_k-z_{k})^2}{2(\xi_kx_{k}^2+1)}\right)
		%	\end{aligned}
	\end{equation*}
	where $\mathbf{z} = (z_1,\ldots,z_d)^\top$ and $\mathbf{x} = (x_1,\ldots,x_d)^\top$.	
	The partial derivative of $\omega_\mathbf{zx}$ with respect to $\boldsymbol{\nu}$, $\mathbf{\Xi}$, $\alpha$, and $\beta$ can be derived as follows:
	\begin{equation*}
		\begin{array}{rcl}
			\displaystyle\dfrac{\partial\omega_\mathbf{zx}}{\partial\nu_i}&\hspace{-2.4mm}=&\hspace{-2.4mm} \displaystyle\alpha \prod_{k=1}^{d}{(\xi_kx_{k}^2+1)^{-\frac{1}{2}}}\exp\left(-\frac{(x_{k}\nu_k-z_{k})^2}{2(\xi_kx_{k}^2+1)}\right)\\
			&&\displaystyle\quad\times\left(-\frac{(x_{i}\nu_i-z_{i})^2}{2(\xi_ix_{i}^2+1)}\right)^\prime\\
			&\hspace{-2.4mm}=&\hspace{-2.4mm} \displaystyle\alpha \prod_{k=1}^{d}{(\xi_kx_{k}^2+1)^{-\frac{1}{2}}}\exp\left(-\frac{(x_{k}\nu_k-z_{k})^2}{2(\xi_kx_{k}^2+1)}\right)\\
			&&\displaystyle\quad\times\left(\dfrac{-\nu_ix_{i}^2+z_{i}x_{i}}{\xi_ix_{i}^2+1}\right),
		\end{array}
	\end{equation*}
	%	Therefore, the derivatives with respect to the mean vector $\boldsymbol{\nu}$ could be denoted as:
	%	\begin{equation}
	%	\dfrac{\partial\omega_\mathbf{zx}}{\partial\boldsymbol{\nu}}=\omega_\mathbf{zx}
	%	\begin{pmatrix}
	%	\dfrac{-\nu_1x_{1}^2+z_{1}x_{1}}{\xi_1x_{1}^2+1} \\
	%	\dfrac{-\nu_2x_{2}^2+z_{2}x_{2}}{\xi_2x_{2}^2+1}  \\
	%	\dots \\
	%	\dfrac{-\nu_dx_{d}^2+z_{d}x_{d}}{\xi_dx_{d}^2+1}
	%	\end{pmatrix}	
	%	\end{equation}
	\begin{equation*}
		\begin{array}{rcl}
			\displaystyle\dfrac{\partial\omega_\mathbf{zx}}{\partial\xi_i}&\hspace{-2.4mm}=&\hspace{-2.4mm}
			\displaystyle\alpha \prod_{k\neq i}{(\xi_kx_{k}^2+1)^{-\frac{1}{2}}}\exp\left(-\frac{(x_{k}\nu_k-z_{k})^2}{2(\xi_kx_{k}^2+1)}\right)\\
			&&\displaystyle\quad\times\Bigg({(\xi_ix_{i}^2+1)^{-\frac{1}{2}}}\exp\left(-\frac{(x_{i}\nu_i-z_{i})^2}{2(\xi_ix_{i}^2+1)}\right)\Bigg)^\prime \\
			&\hspace{-2.4mm}=&\hspace{-2.4mm}\displaystyle\alpha \prod_{k\neq i}{(\xi_kx_{k}^2+1)^{-\frac{1}{2}}}\exp\left(-\frac{(x_{k}\nu_k-z_{k})^2}{2(\xi_kx_{k}^2+1)}\right)\\
			&&\displaystyle\quad\times\Bigg\{\Big({(\xi_ix_{i}^2+1)^{-\frac{1}{2}}}\Big)^\prime\exp\left(-\frac{(x_{i}\nu_i-z_{i})^2}{2(\xi_ix_{i}^2+1)}\right)\\
			&&\displaystyle\qquad+{(\xi_ix_{i}^2+1)^{-\frac{1}{2}}}\Bigg (\exp\left(-\frac{(x_{i}\nu_i-z_{i})^2}{2(\xi_ix_{i}^2+1)}\right)\Bigg)^\prime\Bigg\} \\
			&\hspace{-2.4mm}=&\hspace{-2.4mm}\displaystyle\alpha \prod^d_{k=1}{(\xi_kx_{k}^2+1)^{-\frac{1}{2}}}\exp\left(-\frac{(x_{k}\nu_k-z_{k})^2}{2(\xi_kx_{k}^2+1)}\right)\\
			&&\displaystyle\quad\times\left(-\dfrac{x_{i}^2}{2(\xi_ix_{i}^2+1)}+\dfrac{x_{i}^2(x_{i}\nu_i-z_{i})^2}{2(\xi_ix_{i}^2+1)^2}\right) ,
		\end{array}
	\end{equation*}
	\begin{equation*}
	\begin{array}{l}
	\displaystyle\dfrac{\partial\omega_\mathbf{zx}}{\partial \alpha}= \prod_{k=1}^{d}{(\xi_kx_{k}^2+1)^{-\frac{1}{2}}}\exp\left(-\frac{(x_{k}\nu_k-z_{k})^2}{2(\xi_kx_{k}^2+1)}\right) ,\\
	\displaystyle\dfrac{\partial\omega_\mathbf{zx}}{\partial \beta}=0\ .
	\end{array}
	\end{equation*}
	%	Therefore, the derivatives with respect to the covariance matrix $\mathbf{\Xi}$ could be denoted as:
	%	\begin{equation}
	%	\begin{aligned}
	%	\dfrac{\partial\Omega_{ij}}{\partial\mathbf{\Xi}}&=\Omega_{ij}
	%	\begin{pmatrix}
	%	\varOmega_{ij}^1	 & \dots & 0\\
	%	\vdots& \ddots& \vdots \\
	%	0 & \cdots & \varOmega_{ij}^d \\
	%	\end{pmatrix}
	%	\end{aligned}
	%	\end{equation}
	%	where $\varOmega_{ij}^1=	-\dfrac{x_{j1}^2}{2(\xi_1x_{j1}^2+1)}+\dfrac{x_{j1}^2(x_{j1}\nu_1-z_{i1})^2}{2(\xi_1x_{j1}^2+1)^2}$ and $\varOmega_{ij}^d=	-\dfrac{x_{jd}^2}{2(\xi_dx_{jd}^2+1)}+\dfrac{x_{jd}^2(x_{jd}\nu_d-z_{id})^2}{2(\xi_dx_{jd}^2+1)^2}$
	From Appendix~\ref{Derivation of Omega, Psi and Upsilon},
	\begin{equation*}
		\hspace{-1.7mm}
		\begin{array}{rcl}
			\psi_{\mathbf{z}\mathbf{z}^\prime}&\hspace{-2.4mm}=&\hspace{-2.4mm}
			%\Big[\mathbb{E}_{\mathbf{\Lambda}}\big(\mathbf{K}_{mn}\mathbf{C}^{-1}\mathbf{K}_{nm}\big)\Big]_{ij}\\
			%\displaystyle=\sum_{l=1}^{B}\mathbb{E}_{\mathbf{\Lambda}}\big(\mathbf{K}_{i\mathcal{D}_l}\mathbf{C}_{\mathcal{D}_l\mathcal{D}_l}^{-1}\mathbf{K}_{\mathcal{D}_lj}\big)\\
			%		\displaystyle= \sum_{i=1}^{B}\sum_{\mathbf{x},\mathbf{x}^\prime\in\mathcal{D}_i}\mathbb{E}_{\mathbf{\Lambda}}\left[\mathrm{cov}[s_\mathbf{z},f_\mathbf{x}]\ c^{i}_{\mathbf{x}\mathbf{x}^\prime}\ \mathrm{cov}[f_{\mathbf{x}^\prime},s_{\mathbf{z}^\prime}] \right] \\
			%		\displaystyle= \sum_{i=1}^{B}\sum_{\mathbf{x},\mathbf{x}^\prime\in\mathcal{D}_i}c^{i}_{\mathbf{x}\mathbf{x}^\prime}\ \mathbb{E}_{\mathbf{\Lambda}}\left[\mathrm{cov}[s_\mathbf{z},f_\mathbf{x}]\ \mathrm{cov}[f_{\mathbf{x}^\prime},s_{\mathbf{z}^\prime}] \right] \\
				\displaystyle\sum_{i=1}^{B}\sum_{\mathbf{x},\mathbf{x}'\in\mathcal{D}_i}(\beta+\alpha^2)\ c^{i}_{\mathbf{x}\mathbf{x}'}\prod_{k=1}^{d}\Bigg\{\hspace{-0.5mm}(\xi_k(x_{k}^2+x_{k}^{\prime2})+1)^{-\frac{1}{2}}\\
			&&\hspace{-2.4mm}\exp\left(-\frac{\xi_k(z'_{k}x_{k}-z_{k}x'_{k})^2+(x_{k}\nu_k-z_{k})^2+(x'_{k}\nu_k-z'_{k})^2}{2(\xi_k(x_{k}^2+x_{k}^{\prime 2})+1)}\right)\Bigg\}
		\end{array}
	\end{equation*}
	where $\mathbf{z}^\prime \triangleq (z^\prime_1,\ldots,z^\prime_d)^\top$ and $\mathbf{x}^\prime \triangleq (x^\prime_1,\ldots,x^\prime_d)^\top$.
	%		\begin{equation}
	%		\begin{aligned}
	%		&\Psi_{ij}=\\
	%		&\sum_{l=1}^{B}\sum_{p=1}^{|\mathcal{D}_l|}\sum_{q=1}^{|\mathcal{D}_l|}\sigma_f^2C_{pq}^{(\mathcal{D}_l)-1}\prod_{k=1}^{d}\dfrac{1}{(\xi_k(x_{pk}^2+x_{qk}^2)+1)^{\frac{1}{2}}}\mathcal{J}_k^{(pq)} \\
	%		\end{aligned}
	%		\end{equation}
	%		where $\mathcal{J}_k^{(pq)}= e^{-\frac{\xi_k(z_{jk}x_{pk}-z_{ik}x_{qk})^2+(x_{pk}\nu_k-z_{ik})^2+(x_{qk}\nu_k-z_{jk})^2}{2(\xi_k(x_{pk}^2+x_{qk}^2)+1)}}$
	The partial derivative of $\psi_\mathbf{xx^\prime}$ with respect to $\boldsymbol{\nu}$, $\mathbf{\Xi}$, $\alpha$, and $\beta$ can be derived as follows:
	%		
	%		Assume $\Delta_{pq}^{(\mathcal{D}_l)}=\sigma_f^2C_{pq}^{(\mathcal{D}_l)-1}\prod_{k=1}^{d}\dfrac{1}{(\xi_k(x_{pk}^2+x_{qk}^2)+1)^{\frac{1}{2}}}\times\\e^{-\frac{\xi_k(z_{jk}x_{pk}-z_{ik}x_{qk})^2+(x_{pk}\nu_k-z_{ik})^2+(x_{qk}\nu_k-z_{jk})^2}{2(\xi_k(x_{pk}^2+x_{qk}^2)+1)}}$
	\begin{equation*}
		\hspace{-1.7mm}
		\begin{array}{l}
			\displaystyle\dfrac{\partial\psi_\mathbf{zz^\prime}}{\partial\nu_i}\\
			=\displaystyle\sum_{j=1}^{B}\sum_{\mathbf{x},\mathbf{x}^\prime\in\mathcal{D}_j}\hspace{-0.5mm}\Biggl[\hspace{-0.5mm}\left(\beta+\alpha^2\right) c^{j}_{\mathbf{x}\mathbf{x}^\prime}\prod_{k=1}^{d}\Bigg\{(\xi_k(x_{k}^2+x_{k}^{\prime2})+1)^{-\frac{1}{2}}\\
			\quad\exp\left(-\frac{\xi_k(z^\prime_{k}x_{k}-z_{k}x^\prime_{k})^2+(x_{k}\nu_k-z_{k})^2+(x^\prime_{k}\nu_k-z^\prime_{k})^2}{2(\xi_k(x_{k}^2+x_{k}^{\prime 2})+1)}\right)\hspace{-0.5mm}\Bigg\}\\
			\quad\times\left(-\frac{\xi_i(z^\prime_{i}x_{i}-z_{i}x^\prime_{i})^2+(x_{i}\nu_i-z_{i})^2+(x^\prime_{i}\nu_i-z^\prime_{i})^2}{2(\xi_i(x_{i}^2+x_{i}^{\prime 2})+1)}\right)^\prime\Biggr]\\
			\displaystyle=\sum\limits_{j=1}^{B}\sum\limits_{\mathbf{x},\mathbf{x}^\prime\in\mathcal{D}_j}\hspace{-0.5mm}\Biggl[\hspace{-0.5mm}\left(\beta+\alpha^2\right) c^{j}_{\mathbf{x}\mathbf{x}^\prime}\prod_{k=1}^{d}\Bigg\{(\xi_k(x_{k}^2+x_{k}^{\prime2})+1)^{-\frac{1}{2}}\\
			\quad\exp\left(-\frac{\xi_k(z^\prime_{k}x_{k}-z_{k}x^\prime_{k})^2+(x_{k}\nu_k-z_{k})^2+(x^\prime_{k}\nu_k-z^\prime_{k})^2}{2(\xi_k(x_{k}^2+x_{k}^{\prime 2})+1)}\right)\hspace{-0.5mm}\Bigg\}\\
			\quad\times\left(-\dfrac{\nu_i(x_{i}^2+x_{i}^{\prime2})-(z_{i}x_{i}+z^\prime_{i}x_{i}^\prime)}{\xi_i(x_{i}^2+x_{i}^{\prime2})+1}\right)\Biggr],
			%		=&\sum_{l=1}^{B}\sum_{p=1}^{|\mathcal{D}_l|}\sum_{q=1}^{|\mathcal{D}_l|}\Delta_{pq}^{(l)}\dfrac{-\nu_1(x_{p1}^2+x_{q1}^2)+(z_{i1}x_{p1}+z_{j1}x_{q1})}{\xi_1(x_{p1}^2+x_{q1}^2)+1}
		\end{array}
	\end{equation*}
	%		\begin{equation}
	%		\dfrac{\partial\mathbf{\Psi}_{ij}}{\partial\boldsymbol{\nu}}=
	%		\begin{pmatrix}
	%		\sum_{l=1}^{B}\sum_{p=1}^{s}\sum_{q=1}^{s}\Delta_{pq}^{(l)}\times\\
	%		\dfrac{-\nu_1(x_{p1}^2+x_{q1}^2)+(z_{i1}x_{p1}+z_{j1}x_{q1})}{\xi_1(x_{p1}^2+x_{q1}^2)+1} \\
	%		\sum_{l=1}^{B}\sum_{p=1}^{s}\sum_{q=1}^{s}\Delta_{pq}^{(l)}\times\\
	%		\dfrac{-\nu_2(x_{p2}^2+x_{q2}^2)+(z_{i2}x_{p2}+z_{j2}x_{q2})}{\xi_2(x_{p2}^2+x_{q2}^2)+1} \\
	%		\vdots \\
	%		\sum_{l=1}^{B}\sum_{p=1}^{s}\sum_{q=1}^{s}\Delta_{pq}^{(l)}\times\\
	%		\dfrac{-\nu_K(x_{pK}^2+x_{qK}^2)+(z_{iK}x_{pK}+z_{jK}x_{qK})}{\xi_K(x_{pK}^2+x_{qK}^2)+1}
	%		\end{pmatrix}
	%		\end{equation}
	\begin{equation*}
		\hspace{-1.7mm}
		\begin{array}{l}
			\displaystyle\dfrac{\partial\psi_\mathbf{zz^\prime}}{\partial\xi_i}\\
			%\sum_{l=1}^{B}\sum_{p=1}^{s}\sum_{q=1}^{s}\sigma_f^2C_{pq}^{(l)-1}\prod_{k=2}^{d}\dfrac{1}{(\xi_k(x_{pk}^2+x_{qk}^2)+1)^{\frac{1}{2}}}\\
			%		& e^{-\frac{\xi_k(z_{jk}x_{pk}-z_{ik}x_{qk})^2+(x_{pk}\nu_k-z_{ik})^2+(x_{qk}\nu_k-z_{jk})^2}{2(\xi_k(x_{pk}^2+x_{qk}^2)+1)}}\times \\
			\displaystyle=\sum\limits_{j=1}^{B}\sum\limits_{\mathbf{x},\mathbf{x}^\prime\in\mathcal{D}_j}\hspace{-0.5mm}\Biggl[\hspace{-0.5mm}\left(\beta+\alpha^2\right) c^{j}_{\mathbf{x}\mathbf{x}^\prime}\prod_{k\neq i}\Bigg\{(\xi_k(x_{k}^2+x_{k}^{\prime2})+1)^{-\frac{1}{2}}\\
			\quad\exp\left(-\frac{\xi_k(z^\prime_{k}x_{k}-z_{k}x^\prime_{k})^2+(x_{k}\nu_k-z_{k})^2+(x^\prime_{k}\nu_k-z^\prime_{k})^2}{2(\xi_k(x_{k}^2+x_{k}^{\prime 2})+1)}\right)\hspace{-0.5mm}\Bigg\}\\
			\quad\times\Biggl((\xi_i(x_{i}^2+x_{i}^{\prime2})+1)^{-\frac{1}{2}}\times\\
			\quad \exp\left(-\frac{\xi_i(z^\prime_{i}x_{i}-z_{i}x^\prime_{i})^2+(x_{i}\nu_i-z_{i})^2+(x^\prime_{i}\nu_i-z^\prime_{i})^2}{2(\xi_i(x_{i}^2+x_{i}^{\prime 2})+1)}\right)\Biggr)^\prime\Biggr]\\
			\displaystyle=\sum\limits_{j=1}^{B}\sum\limits_{\mathbf{x},\mathbf{x}^\prime\in\mathcal{D}_j}\hspace{-0.5mm}\Biggl[\hspace{-0.5mm}\left(\beta+\alpha^2\right) c^{j}_{\mathbf{x}\mathbf{x}^\prime}\prod_{k=1}^{d}\Bigg\{(\xi_k(x_{k}^2+x_{k}^{\prime2})+1)^{-\frac{1}{2}}\\
			\quad\exp\left(-\frac{\xi_k(z^\prime_{k}x_{k}-z_{k}x^\prime_{k})^2+(x_{k}\nu_k-z_{k})^2+(x^\prime_{k}\nu_k-z^\prime_{k})^2}{2(\xi_k(x_{k}^2+x_{k}^{\prime 2})+1)}\right)\hspace{-0.5mm}\Bigg\}\\
			%		&=\sum_{l=1}^{B}\sum_{p=1}^{s}\sum_{q=1}^{s}\Delta_{pq}^{(l)}\cdot
			\quad\times\Bigg(-\dfrac{x_{i}^2+x_{i}^{\prime2}}{2\big(\xi_i(x_{i}^2+x_{i}^{\prime2})+1\big)}\\
			\qquad+\dfrac{\Big(z_{i}x_{i}+z_{i}^\prime x_{i}^\prime-\nu_i(x_{i}^2+x_{i}^{\prime2})\Big)^2}{2\big(\xi_i(x_{i}^2+x_{i}^{\prime2})+1\big)^2}\Bigg)\Biggr],
		\end{array}
	\end{equation*}
	\begin{equation*}
	\begin{array}{l}
	\displaystyle\dfrac{\partial\psi_\mathbf{zz^\prime}}{\partial \alpha}=\hspace{-0.5mm}\sum\limits_{i=1}^{B}\hspace{-0.5mm}\sum\limits_{\mathbf{x},\mathbf{x}^\prime\in\mathcal{D}_i}\hspace{-0.5mm}2\alpha\ c^{i}_{\mathbf{x}\mathbf{x}^\prime}\hspace{-0.5mm}\prod_{k=1}^{d}\hspace{-0.5mm}\Bigg\{\hspace{-0.5mm}(\xi_k(x_{k}^2+x_{k}^{\prime2})+1)^{-\frac{1}{2}} \\
	\displaystyle \hspace{-2mm}\exp\hspace{-1mm}\left(\hspace{-1mm}-\frac{\xi_k(z^\prime_{k}x_{k}-z_{k}x^\prime_{k})^2+(x_{k}\nu_k-z_{k})^2+(x^\prime_{k}\nu_k-z^\prime_{k})^2}{2(\xi_k(x_{k}^2+x_{k}^{\prime 2})+1)}\hspace{-1mm}\right)\hspace{-1mm}\Bigg\},\vspace{1mm}\\
	\displaystyle\dfrac{\partial\psi_\mathbf{zz^\prime}}{\partial \beta}=\hspace{-0.5mm}\sum\limits_{i=1}^{B}\hspace{-0.5mm}\sum\limits_{\mathbf{x},\mathbf{x}^\prime\in\mathcal{D}_i} c^{i}_{\mathbf{x}\mathbf{x}^\prime}\hspace{-0.5mm}\prod_{k=1}^{d}\hspace{-0.5mm}\Bigg\{\hspace{-0.5mm}(\xi_k(x_{k}^2+x_{k}^{\prime2})+1)^{-\frac{1}{2}}\\
	\displaystyle\hspace{-2mm}\exp\hspace{-1mm}\left(\hspace{-1mm}-\frac{\xi_k(z^\prime_{k}x_{k}-z_{k}x^\prime_{k})^2+(x_{k}\nu_k-z_{k})^2+(x^\prime_{k}\nu_k-z^\prime_{k})^2}{2(\xi_k(x_{k}^2+x_{k}^{\prime 2})+1)}\hspace{-1mm}\right)\hspace{-1mm}\Bigg\}.
	\end{array}
	\end{equation*}
	%		\begin{equation}
	%		\dfrac{\partial\mathbf{\Psi}_{ij}}{\partial\mathbf{\xi}}=
	%		\begin{pmatrix}
	%		\dfrac{\partial\mathbf{\Psi}_{ij}}{\partial\xi_1} & \dots & 0 \\
	%		\vdots & \ddots & \vdots \\
	%		0 & \dots & \dfrac{\partial\mathbf{\Psi}_{ij}}{\partial\xi_K}
	%		\end{pmatrix}
	%		\end{equation}
	From Appendix~\ref{Derivation of Omega, Psi and Upsilon},
	\begin{equation*}
		\begin{array}{rcl}
			\gamma_{\mathbf{x}\mathbf{x}^\prime}&\hspace{-2.4mm}=&\hspace{-2.4mm}\displaystyle\left(\beta+\alpha^2\right)\prod_{k=1}^{d}{(\xi_k(x_{k}-x^\prime_{k})^2+1)^{-\frac{1}{2}}}\\
			&&\hspace{-2.4mm}\displaystyle\exp\left(-\frac{\nu_k^2(x_{k}-x^\prime_{k})^2}{2(\xi_k(x_{k}-x^\prime_{k})^2+1)}\right).
		\end{array}
	\end{equation*}
	%	where $\mathbf{x} \triangleq [x_1,x_2,\dots,x_d]^\top$ and $\mathbf{x}^\prime \triangleq [x^\prime_1,x^\prime_2,\dots,x^\prime_d]^\top$
	%	
	The partial derivative of $\gamma_{\mathbf{x}\mathbf{x}^\prime}$ with respect to $\boldsymbol{\nu}$, $\mathbf{\Xi}$, $\alpha$, and $\beta$ can be derived as follows:
	\begin{equation*}
		\hspace{-1.7mm}
		\begin{array}{l}
			\displaystyle\dfrac{\partial\gamma_{\mathbf{x}\mathbf{x}^\prime}}{\partial\nu_i}\displaystyle=\left(\beta+\alpha^2\right)\\
			\displaystyle\prod_{k=1}^{d}{(\xi_k(x_{k}-x^\prime_{k})^2+1)^{-\frac{1}{2}}}\exp\hspace{-0.5mm}\left(-\frac{\nu_k^2(x_{k}-x^\prime_{k})^2}{2(\xi_k(x_{k}-x^\prime_{k})^2+1)}\right)\\
			\displaystyle\times\left(-\dfrac{\nu_i^2(x_{i}-x^\prime_{i})^2}{2\big(\xi_i(x_{i}-x^\prime_{i})^2+1\big)}\right)^\prime\vspace{1mm}\\
			\displaystyle=\left(\beta+\alpha^2\right)\\
			\displaystyle\prod_{k=1}^{d}{(\xi_k(x_{k}-x^\prime_{k})^2+1)^{-\frac{1}{2}}}\exp\hspace{-0.5mm}\left(-\frac{\nu_k^2(x_{k}-x^\prime_{k})^2}{2(\xi_k(x_{k}-x^\prime_{k})^2+1)}\right)\hspace{-1mm}\\
			\displaystyle\times\left(-\dfrac{\nu_i(x_{i}-x^\prime_{i})^2}{\xi_i(x_{i}-x^\prime_{i})^2+1}\right),
		\end{array}
	\end{equation*}
	%	Therefore, the derivatives with respect to the mean $\boldsymbol{\nu}$ could be denoted as:
	%	\begin{equation}
	%	\dfrac{\partial\Upsilon_{ij}}{\partial\boldsymbol{\nu}}=\Upsilon_{ij}
	%	\begin{pmatrix}
	%	-\dfrac{\nu_1(x_{i1}-x_{j1})^2}{\xi_1(x_{i1}-x_{j1})^2+1} \\
	%	-\dfrac{\nu_2(x_{i2}-x_{j2})^2}{\xi_2(x_{i2}-x_{j2})^2+1} \\
	%	\vdots \\
	%	-\dfrac{\nu_K(x_{id}-x_{jd})^2}{\xi_d(x_{id}-x_{jd})^2+1}
	%	\end{pmatrix}
	%	\end{equation}
	\begin{equation*}
		\hspace{-1.7mm}
		\begin{array}{l}
			\displaystyle\dfrac{\partial\gamma_{\mathbf{x}\mathbf{x}^\prime}}{\partial\xi_i}
			%	&\sigma_f^2\prod_{k\neq i}{(\xi_k(x_{k}-x^\prime_{k})^2+1)^{-\frac{1}{2}}}\exp\hspace{-1mm}\left(-\frac{\nu_k^2(x_{k}-x^\prime_{k})^2}{2(\xi_k(x_{k}-x^\prime_{k})^2+1)}\right)\hspace{-1mm}\times\\
			%	&\left({(\xi_i(x_{i}-x^\prime_{i})^2+1)^{-\frac{1}{2}}}\exp\hspace{-1mm}\left(-\frac{\nu_i^2(x_{i}-x^\prime_{i})^2}{2(\xi_i(x_{i}-x^\prime_{i})^2+1)}\right)\right)^\prime  \\
			\displaystyle=\left(\beta+\alpha^2\right)\\
			\displaystyle\prod_{k\neq i}{(\xi_k(x_{k}-x^\prime_{k})^2+1)^{-\frac{1}{2}}}\exp\hspace{-0.5mm}\left(-\frac{\nu_k^2(x_{k}-x^\prime_{k})^2}{2(\xi_k(x_{k}-x^\prime_{k})^2+1)}\right)\\
			\displaystyle\times\Bigg(\left({(\xi_i(x_{i}-x^\prime_{i})^2+1)^{-\frac{1}{2}}}\right)^\prime\exp\hspace{-0.5mm}\left(-\frac{\nu_i^2(x_{i}-x^\prime_{i})^2}{2(\xi_i(x_{i}-x^\prime_{i})^2+1)}\right)\\
			\displaystyle+{(\xi_i(x_{i}-x^\prime_{i})^2+1)^{-\frac{1}{2}}}\left(\exp\hspace{-0.5mm}\left(-\frac{\nu_i^2(x_{i}-x^\prime_{i})^2}{2(\xi_i(x_{i}-x^\prime_{i})^2+1)}\right)\right)^{\hspace{-0.5mm}\prime}\Bigg)\\
			\displaystyle=\left(\beta+\alpha^2\right)\\
			\displaystyle\prod_{k=1}^{d}{(\xi_k(x_{k}-x^\prime_{k})^2+1)^{-\frac{1}{2}}}\exp\hspace{-0.5mm}\left(-\frac{\nu_k^2(x_{k}-x^\prime_{k})^2}{2(\xi_k(x_{k}-x^\prime_{k})^2+1)}\right)\\
			\displaystyle\times\left(-\dfrac{(x_{i}-x^\prime_{i})^2}{2\big(\xi_i(x_{i}-x^\prime_{i})^2+1\big)}+\dfrac{\nu_i^2(x_{i}-x^\prime_{i})^4}{2\big(\xi_i(x_{i}-x^\prime_{i})^2+1\big)^2}\right) \\
			\displaystyle=\left(\beta+\alpha^2\right)\\
			\displaystyle\prod_{k=1}^{d}{(\xi_k(x_{k}-x^\prime_{k})^2+1)^{-\frac{1}{2}}}\exp\hspace{-0.5mm}\left(-\frac{\nu_k^2(x_{k}-x^\prime_{k})^2}{2(\xi_k(x_{k}-x^\prime_{k})^2+1)}\right)\\
			\displaystyle\times\dfrac{(\nu_i^2-\xi_i)(x_{i}-x^\prime_{i})^4-(x_{i}-x^\prime_{i})^2}{2\big(\xi_i(x_{i}-x^\prime_{i})^2+1\big)^2}\ ,
		\end{array}
	\end{equation*} 
		\begin{equation*}
		\begin{array}{l}
		\displaystyle\dfrac{\partial\gamma_{\mathbf{x}\mathbf{x}^\prime}}{\partial \alpha}=\\
		\displaystyle2\alpha\prod_{k=1}^{d}{(\xi_k(x_{k}-x^\prime_{k})^2+1)^{-\frac{1}{2}}}\exp\hspace{-0.5mm}\left(-\frac{\nu_k^2(x_{k}-x^\prime_{k})^2}{2(\xi_k(x_{k}-x^\prime_{k})^2+1)}\right),\vspace{1mm}\\
		\displaystyle\dfrac{\partial\gamma_{\mathbf{x}\mathbf{x}^\prime}}{\partial \beta}=\\
		\displaystyle\prod_{k=1}^{d}{(\xi_k(x_{k}-x^\prime_{k})^2+1)^{-\frac{1}{2}}}\exp\hspace{-0.5mm}\left(-\frac{\nu_k^2(x_{k}-x^\prime_{k})^2}{2(\xi_k(x_{k}-x^\prime_{k})^2+1)}\right).
		\end{array}
		\end{equation*}
% 
% 
%		 
\subsection{Derivation of $\mu_{\mathbf{x}^*|\mathcal{D}}$ and $\sigma^2_{\mathbf{x}^*|\mathcal{D}}$} 
	\label{q(y^*)}
	\subsubsection{VBPITC, VBFIC, VBFITC, and VBDTC}
	VBPITC, VBFIC, VBFITC, and VBDTC share the same approximated test conditional $q(f_{\mathbf{x}^*}|\mathbf{y}_{\mathcal{D}_B}, \mathbf{s}_\mathcal{I},\mathbf{\Lambda},\sigma_f)\triangleq p(f_{\mathbf{x}^*}|\mathbf{s}_\mathcal{I},\mathbf{\Lambda},\sigma_f)$ but differ in
	%\emph{Remark} The prediction framework of the Bayesian DTC, FITC, PITC are similar with each other except for the difference between 
	$q^+(\mathbf{s}_\mathcal{I})$, $q^+(\mathbf{\Lambda})$, and $q^+(\sigma_f)$ obtained from their stochastic gradient ascent updates.
	As a result,
	%The distribution of the $q(f_{\mathbf{x}^*}|\mathbf{y}_\mathcal{D})$ is denoted as:
	\begin{equation*}
	\hspace{-1.7mm}
	\begin{array}{l}
		q(f_{\mathbf{x}^*}|\mathbf{y}_\mathcal{D})\\
		\displaystyle=\int p(f_{\mathbf{x}^*}|\mathbf{s}_\mathcal{I},\mathbf{\Lambda},\sigma_f)\ q^+(\mathbf{s}_\mathcal{I})\ q^+(\mathbf{\Lambda})\ q^+(\sigma_f)\ \mathrm{d}\mathbf{s}_\mathcal{I}\ \mathrm{d}\mathbf{\Lambda}\ \mathrm{d}\sigma_f
	\end{array}
	\end{equation*}
	where
	\begin{equation*}
		\hspace{-1.7mm}
		\begin{array}{rcl}
			p(f_{\mathbf{x}^*}|\mathbf{s}_\mathcal{I},\mathbf{\Lambda},\sigma_f)&\hspace{-2.4mm}=&\hspace{-2.4mm}\displaystyle\mathcal{N}(\mathbf{K}_{\mathbf{x}^*\mathcal{I}}\mathbf{\Sigma}_\mathcal{II}^{-1}\mathbf{s}_\mathcal{I},\\
			&&\hspace{-2.4mm}\displaystyle\quad\ \ k_{\mathbf{x}^*\mathbf{x}^*} -\mathbf{\mathbf{K}}_{\mathbf{x}^*\mathcal{I}}\mathbf{\Sigma}_\mathcal{II}^{-1}\mathbf{\mathbf{K}}_{\mathcal{I}\mathbf{x}^*}\hspace{-0.5mm}) \ , \vspace{1mm}\\
			q^+(\mathbf{s}_\mathcal{I})&\hspace{-2.4mm}=&\hspace{-2.4mm}\displaystyle\mathcal{N}(\mathbf{m}^+,\mathbf{S}^+)\ , \\
			q^+(\mathbf{\Lambda})&\hspace{-2.4mm}=&\hspace{-2.4mm}\displaystyle\prod_{i=1}^{d}\mathcal{N}(\lambda_i|\nu_i^+,\xi_i^+)\ , \vspace{1mm}\\
			q^+(\sigma_f)&\hspace{-2.4mm}=&\hspace{-2.4mm}\mathcal{N}(\alpha^+,\beta^+)\ .	
		\end{array}
	\end{equation*}
	%such that $\mathbf{Q}_{\mathbf{x}^*\mathbf{x}^*}=$.		
	%Following the rule of conditional Gaussian calculus and after integrating out the variational parameters $\mathbf{s}_\mathcal{I}$ in close form:
	Then,
	\begin{equation*}
		\begin{array}{l}
			\displaystyle q(f_{\mathbf{x}^*}|\mathbf{y}_\mathcal{D},\mathbf{\Lambda},\sigma_f)\\
			\displaystyle=\int p(f_{\mathbf{x}^*}|\mathbf{s}_\mathcal{I},\mathbf{\Lambda},\sigma_f)\ q^+(\mathbf{s}_\mathcal{I})\ \mathrm{d}\mathbf{s}_\mathcal{I} \\
			\displaystyle =\mathcal{N}(\mathbf{K}_{\mathbf{x}^*\mathcal{I}}\mathbf{\Sigma}_\mathcal{II}^{-1}\mathbf{m}^+,k_{\mathbf{x}^*\mathbf{x}^*}-\mathbf{K}_{\mathbf{x}^*\mathcal{I}}\mathbf{\Sigma}_\mathcal{II}^{-1}\mathbf{K}_{\mathcal{I}\mathbf{x}^*}\\
			\displaystyle\qquad\qquad\qquad\qquad\ \ \  +\mathbf{K}_{\mathbf{x}^*\mathcal{I}}\mathbf{\Sigma}_\mathcal{II}^{-1}\mathbf{S}^+\mathbf{\Sigma}_\mathcal{II}^{-1}\mathbf{K}_{\mathcal{I}\mathbf{x}^*})\ .
		\end{array}
	\end{equation*}
	Finally, 	
	\begin{equation*}
		%\hspace{-13mm}
		\begin{array}{l}
			\displaystyle\mu_{\mathbf{x}^*|\mathcal{D}}\\
			\displaystyle\triangleq\mathbb{E}_{q(f_{\mathbf{x}^*}|\mathbf{y}_\mathcal{D})}[f_{\mathbf{x}^*}]\vspace{1mm}\\
			%\displaystyle=\int f_{\mathbf{x}^*}\ q(f_{\mathbf{x}^*}|\mathbf{y}_\mathcal{D})\ \mathrm{d}f_{\mathbf{x}^*} \\
			%\displaystyle=\int f_{\mathbf{x}^*}\int q(f_{\mathbf{x}^*}|\mathbf{y}_\mathcal{D},\mathbf{\Lambda})\ q^+(\mathbf{\Lambda})\ \mathrm{d}\mathbf{\Lambda}\ \mathrm{d}f_{\mathbf{x}^*} \\
			%\displaystyle=\int \left(\int f_{\mathbf{x}^*}\ q(f_{\mathbf{x}^*}|\mathbf{y}_\mathcal{D},\mathbf{\Lambda})\ \mathrm{d}f_{\mathbf{x}^*} \right)\ q^+(\mathbf{\Lambda})\ \mathrm{d}\mathbf{\Lambda} \\
			\displaystyle=\mathbb{E}_{q^+(\mathbf{\Lambda},\sigma_f)}[\mathbb{E}_{q(f_{\mathbf{x}^*}|\mathbf{y}_\mathcal{D},\mathbf{\Lambda}, \sigma_f)}[f_{\mathbf{x}^*}]] \vspace{1mm}\\
			\displaystyle=\mathbb{E}_{q^+(\mathbf{\Lambda},\sigma_f)}[ \mathbf{K}_{\mathbf{x}^*\mathcal{I}}\mathbf{\Sigma}_\mathcal{II}^{-1}\mathbf{m}^+ ] \vspace{1mm}\\
			\displaystyle=\mathbb{E}_{q^+(\mathbf{\Lambda},\sigma_f)}[\mathbf{\mathbf{K}}_{\mathbf{x}^*\mathcal{I}}]\mathbf{\Sigma}_\mathcal{II}^{-1}\mathbf{m}^+\ .
		\end{array}
	\end{equation*}		
	\begin{equation*}
		\hspace{-1.7mm}
		\begin{array}{l}
			\displaystyle\sigma^2_{\mathbf{x}^*|\mathcal{D}}\\
			\displaystyle\triangleq\mathbb{V}_{q(f_{\mathbf{x}^*}|\mathbf{y}_\mathcal{D})}[f_{\mathbf{x}^*}]\\
			%\displaystyle=\int (f_{\mathbf{x}^*}-\mathbb{E}_{q(f_{\mathbf{x}^*}|\mathbf{y}_\mathcal{D})}[f_{\mathbf{x}^*}])^2 q(f_{\mathbf{x}^*}|\mathbf{y}_\mathcal{D})\ \mathrm{d}f_{\mathbf{x}^*} \\
			%\displaystyle=\int \hspace{-1mm}(f_{\mathbf{x}^*}\hspace{-1mm}-\hspace{-0.5mm}\mathbb{E}_{q(f_{\mathbf{x}^*}|\mathbf{y}_\mathcal{D})}[f_{\mathbf{x}^*}])^2\hspace{-1mm}\int\hspace{-1mm} q(f_{\mathbf{x}^*}|\mathbf{y}_\mathcal{D},\mathbf{\Lambda}) q^+(\mathbf{\Lambda}) \mathrm{d}\mathbf{\Lambda} \mathrm{d}f_{\mathbf{x}^*} \\
			\displaystyle=\mathbb{E}_{q^+(\mathbf{\Lambda},\sigma_f)}[\mathbb{V}_{q(f_{\mathbf{x}^*}|\mathbf{y}_\mathcal{D},\mathbf{\Lambda},\sigma_f)}[f_{\mathbf{x}^*}]]\\
			\displaystyle +\  \mathbb{V}_{q^+(\mathbf{\Lambda},\sigma_f)}[\mathbb{E}_{q(f_{\mathbf{x}^*}|\mathbf{y}_\mathcal{D},\mathbf{\Lambda},\sigma_f)}[f_{\mathbf{x}^*}]]\vspace{1mm}\\
			\displaystyle\hspace{-0.5mm}=\hspace{-0.5mm}\mathbb{E}_{q^+(\mathbf{\Lambda},\sigma_f)}[k_{\mathbf{x}^*\mathbf{x}^*}\hspace{-1mm}-\hspace{-0.5mm}\mathbf{\mathbf{K}}_{\mathbf{x}^*\mathcal{I}}\mathbf{\Sigma}_\mathcal{II}^{-1}\mathbf{K}_{\mathcal{I}\mathbf{x}^*}\hspace{-1mm}+\hspace{-0.5mm}\mathbf{\mathbf{K}}_{\mathbf{x}^*\mathcal{I}}\mathbf{\Sigma}_\mathcal{II}^{-1}\mathbf{S}^+\mathbf{\Sigma}_\mathcal{II}^{-1}\mathbf{\mathbf{K}}_{\mathcal{I}\mathbf{x}^*}]\vspace{1mm}\\
			\displaystyle\quad+\mathbb{V}_{q^+(\mathbf{\Lambda},\sigma_f)}[\mathbf{m}^{+\top} \mathbf{\Sigma}_\mathcal{II}^{-1}\mathbf{K}_{\mathcal{I}\mathbf{x}^*} ]
		\end{array}
	\end{equation*}
	where
	\begin{equation*}
		%\hspace{-1.7mm}
		\begin{array}{l}
			\displaystyle\mathbb{V}_{q^+(\mathbf{\Lambda},\sigma_f)}[\mathbf{m}^{+\top} \mathbf{\Sigma}_\mathcal{II}^{-1}\mathbf{K}_{\mathcal{I}\mathbf{x}^*} ]\vspace{1mm}\\
			\displaystyle= \mathbf{m}^{+\top} \mathbf{\Sigma}_\mathcal{II}^{-1}\mathbb{V}_{q^+(\mathbf{\Lambda},\sigma_f)}[\mathbf{K}_{\mathcal{I}\mathbf{x}^*} ]\mathbf{\Sigma}_\mathcal{II}^{-1}\mathbf{m}^{+}\vspace{1mm}\\
			\displaystyle= \mathbf{m}^{+\top} \mathbf{\Sigma}_\mathcal{II}^{-1}\Big(\mathbb{E}_{q^+(\mathbf{\Lambda},\sigma_f)}[\mathbf{K}_{\mathcal{I}\mathbf{x}^*}\mathbf{K}_{\mathbf{x}^*\mathcal{I}} ]\vspace{1mm}\\
			\displaystyle\qquad\qquad -  \mathbb{E}_{q^+(\mathbf{\Lambda},\sigma_f)}[\mathbf{K}_{\mathcal{I}\mathbf{x}^*} ]\mathbb{E}_{q^+(\mathbf{\Lambda},\sigma_f)}[\mathbf{K}_{\mathbf{x}^*\mathcal{I}} ]\Big)\mathbf{\Sigma}_\mathcal{II}^{-1}\mathbf{m}^{+}.
		\end{array}
	\end{equation*}
	Note that the closed-form expressions of all the above expectation terms  with respect to 
%$q^+(\mathbf{\Lambda})$  and $q^+(\sigma_f)$ 
$q^+(\mathbf{\Lambda},\sigma_f)\triangleq q^+(\mathbf{\Lambda})q^+(\sigma_f)$ can be derived in a similar manner as that of $\mathbf{\Psi}_\mathcal{II}\triangleq\mathbb{E}_{q(\mathbf{\Lambda},\sigma_f)}[\mathbf{K}_{\mathcal{ID}}\mathbf{C}^{-1}_\mathcal{DD}\mathbf{K}_{\mathcal{DI}}]$, $\mathbf{\Omega}_\mathcal{ID}\triangleq\mathbb{E}_{q(\mathbf{\Lambda},\sigma_f)}[\mathbf{K}_{\mathcal{ID}}]$, and $\mathbf{\Upsilon}_\mathcal{DD}\triangleq\mathbb{E}_{q(\mathbf{\Lambda},\sigma_f)}[\mathbf{K}_{\mathcal{DD}}]$. Hence, $\mu_{\mathbf{x}^*|\mathcal{D}}$ and $\sigma^2_{\mathbf{x}^*|\mathcal{D}}$ can be derived in closed form.
	%
	\subsubsection{VBPIC}
	VBPIC uses the exact test conditional $q(f_{\mathbf{x}^*}|\mathbf{y}_{\mathcal{D}_B}, \mathbf{s}_\mathcal{I},\mathbf{\Lambda},\sigma_f)\triangleq p(f_{\mathbf{x}^*}|\mathbf{y}_{\mathcal{D}_B}, \mathbf{s}_\mathcal{I},\mathbf{\Lambda},\sigma_f)$.
	%Recall in the framework of Bayesian PIC, the conditional probability $p(f_{\mathbf{x}^*}|\mathbf{s}_\mathcal{I}^+,\mathbf{\Lambda}^+)$ enhances to $p(f_{\mathbf{x}^*}|\mathbf{s}_\mathcal{I}^+,\mathbf{\Lambda}^+,\mathbf{y}_{\mathcal{D}_B})$. where the $\mathbf{y}_{\mathcal{D}_B}$ is the local training observations which the test input $x^*$ belongs to the particular local block.		
	%		In the traditional structure, we assume that the testing output $y^*$ is conditional independent of $\mathbf{y}_\mathcal{D}$ given the inducing variables $\mathbf{s}_\mathcal{I}$. This will lead to the similar SGP models, such as DTC if we use simple Homoscedastic noise model; FITC, FIC, and PITC if we use Heteroscedastic noise. If we want to enhance the model into PIC framework. The testing output $y^*$ not only depends on the inducing variables but also depends on the local blocking observations. We could rewrite the following as: 
	%		\begin{equation}
	%		p(f_{\mathbf{x}^*}|\mathbf{s}_\mathcal{I}^+,\mathbf{\Lambda}^+)\Longrightarrow	p(f_{\mathbf{x}^*}|\mathbf{s}_\mathcal{I}^+,\mathbf{\Lambda}^+,\mathbf{y}_\mathcal{D}_{B_s})
	%		\end{equation}
	%		where the $\mathbf{y}_\mathcal{D}_{B_s}$ is the local training observations which the test input $x^*$ belongs to the particular local block. 
	To derive $p(f_{\mathbf{x}^*}|\mathbf{y}_{\mathcal{D}_B}, \mathbf{s}_\mathcal{I},\mathbf{\Lambda}, \sigma_f)$, we use the fundamental definition of GP to give the following expression for the Gaussian joint distribution $p(f_{\mathbf{x}^*},\mathbf{s}_\mathcal{I},\mathbf{y}_{\mathcal{D}_B}|\mathbf{\Lambda},\sigma_f)$:
	\begin{equation*}
		\mathcal{N}\Biggm(\mathbf{0},
		\begin{pmatrix}
			k_{\mathbf{x}^*\mathbf{x}^*} & \mathbf{K}_{\mathbf{x}^*\mathcal{I}} & \mathbf{K}_{\mathbf{x}^*\mathcal{D}_B}\\
			\mathbf{K}_{\mathcal{I}\mathbf{x}^*} & \mathbf{\Sigma}_{\mathcal{I}\mathcal{I}} & \mathbf{K}_{\mathcal{I}\mathcal{D}_B} \\
			\mathbf{K}_{\mathcal{D}_B\mathbf{x}^*} & \mathbf{K}_{\mathcal{D}_B\mathcal{I}} & \mathbf{K}_{\mathcal{D}_B\mathcal{D}_B}\hspace{-1mm} + \hspace{-0.5mm} \mathbf{C}_{\mathcal{D}_B\mathcal{D}_B}
		\end{pmatrix}\Bigg).
	\end{equation*} 
	Then, $p(f_{\mathbf{x}^*}|\mathbf{s}_\mathcal{I},\mathbf{y}_{\mathcal{D}_B},\mathbf{\Lambda},\sigma_f)
	=\mathcal{N}(\mathbb{E}_{p(f_{\mathbf{x}^*}|\mathbf{s}_\mathcal{I},\mathbf{y}_{\mathcal{D}_B},\mathbf{\Lambda},\sigma_f)}[f_{\mathbf{x}^*}],\mathbb{V}_{p(f_{\mathbf{x}^*}|\mathbf{s}_\mathcal{I},\mathbf{y}_{\mathcal{D}_B},\mathbf{\Lambda},\sigma_f)}[f_{\mathbf{x}^*}])$ 
	where
	\begin{equation*}
		\hspace{-1.7mm}
		\begin{array}{l}
			\displaystyle\mathbb{E}_{p(f_{\mathbf{x}^*}|\mathbf{s}_\mathcal{I},\mathbf{y}_{\mathcal{D}_B},\mathbf{\Lambda},\sigma_f)}[f_{\mathbf{x}^*}] \vspace{1mm}\\
			\hspace{-.5mm}=\hspace{-.5mm}\begin{pmatrix}
				\mathbf{K}_{\mathbf{x}^*\mathcal{I}} & \hspace{-2mm}\mathbf{K}_{\mathbf{x}^*\mathcal{D}_B}
			\end{pmatrix}
			\hspace{-1mm}
			\begin{pmatrix}
				\mathbf{\Sigma}_\mathcal{II} & \mathbf{K}_{\mathcal{I}\mathcal{D}_B} \\
				\mathbf{K}_{\mathcal{D}_B\mathcal{I}} & \mathbf{K}_{\mathcal{D}_B\mathcal{D}_B}\hspace{-3mm} + \hspace{-0.5mm} \mathbf{C}_{\mathcal{D}_B\mathcal{D}_B}
			\end{pmatrix}^{-1}
			\hspace{-1mm}
			\begin{pmatrix}
				\mathbf{s}_\mathcal{I} \\
				\mathbf{y}_{\mathcal{D}_B}
			\end{pmatrix}  ,\vspace{2mm}\\
			\displaystyle\mathbb{V}_{p(f_{\mathbf{x}^*}|\mathbf{s}_\mathcal{I},\mathbf{y}_{\mathcal{D}_B},\mathbf{\Lambda},\sigma_f)}[f_{\mathbf{x}^*}] =k_{\mathbf{x}^*\mathbf{x}^*}-\vspace{1mm}\\
			\hspace{-1mm}
			\begin{pmatrix}
				\mathbf{K}_{\mathbf{x}^*\mathcal{I}} & \hspace{-2mm}\mathbf{K}_{\mathbf{x}^*\mathcal{D}_B}
			\end{pmatrix}
			\hspace{-1mm}
			\begin{pmatrix}
				\mathbf{\Sigma}_\mathcal{II} & \mathbf{K}_{\mathcal{I}\mathcal{D}_B} \\
				\mathbf{K}_{\mathcal{D}_B\mathcal{I}} & \mathbf{K}_{\mathcal{D}_B\mathcal{D}_B}\hspace{-3mm} + \hspace{-0.5mm} \mathbf{C}_{\mathcal{D}_B\mathcal{D}_B}
			\end{pmatrix}^{-1}
			\hspace{-1mm}
			\begin{pmatrix}
				\mathbf{K}_{\mathcal{I}\mathbf{x}^*}\\
				\mathbf{K}_{\mathcal{D}_B\mathbf{x}^*}
			\end{pmatrix}.
		\end{array}
	\end{equation*}
	%
	To simplify the above expressions, let
	\begin{equation*}
		%\begin{aligned}
		\mathbf{J}\triangleq
		\begin{pmatrix}
			\mathbf{\Sigma}_{\mathcal{I}\mathcal{I}} & \mathbf{K}_{\mathcal{I}\mathcal{D}_B} \\
			\mathbf{K}_{\mathcal{D}_B\mathcal{I}} & \mathbf{K}_{\mathcal{D}_B\mathcal{D}_B}\hspace{-3mm} + \hspace{-0.5mm} \mathbf{C}_{\mathcal{D}_B\mathcal{D}_B}
		\end{pmatrix}^{-1}=
		\begin{pmatrix}
			\mathbf{J}_{\mathcal{I}\mathcal{I}} & \hspace{-1mm}\mathbf{J}_{\mathcal{I}\mathcal{D}_B} \\
			\mathbf{J}_{\mathcal{D}_B\mathcal{I}} & \mathbf{J}_{\mathcal{D}_B\mathcal{D}_B}
		\end{pmatrix}
		%\end{aligned}
	\end{equation*}
	where $\mathbf{J}_{\mathcal{I}\mathcal{I}}$, $\mathbf{J}_{\mathcal{I}\mathcal{D}_B}$,
	$\mathbf{J}_{\mathcal{D}_B\mathcal{I}}$, and $\mathbf{J}_{\mathcal{D}_B\mathcal{D}_B}$ can be derived by applying the matrix inversion lemma for partitioned matrices directly.
	%So we could rewrite $E^\prime$ and $V^\prime$ as:
	Then,
	\begin{equation*}
		\hspace{-1.7mm}
		\begin{array}{l}
			\displaystyle\mathbb{E}_{p(f_{\mathbf{x}^*}|\mathbf{s}_\mathcal{I},\mathbf{y}_{\mathcal{D}_B},\mathbf{\Lambda},\sigma_f)}[f_{\mathbf{x}^*}] \vspace{1mm}\\
			=\begin{pmatrix}
				\mathbf{K}_{\mathbf{x}^*\mathcal{I}} & \mathbf{K}_{\mathbf{x}^*\mathcal{D}_B}
			\end{pmatrix}
			\begin{pmatrix}
				\mathbf{J}_{\mathcal{I}\mathcal{I}} & \mathbf{J}_{\mathcal{I}\mathcal{D}_B} \\
				\mathbf{J}_{\mathcal{D}_B\mathcal{I}} & \mathbf{J}_{\mathcal{D}_B\mathcal{D}_B}
			\end{pmatrix}
			\begin{pmatrix}
				\mathbf{s}_\mathcal{I} \\
				\mathbf{y}_{\mathcal{D}_B}
			\end{pmatrix} \vspace{1mm}\\
			\displaystyle=\left(\mathbf{K}_{\mathbf{x}^*\mathcal{I}}\mathbf{J}_{\mathcal{I}\mathcal{I}}+\mathbf{K}_{\mathbf{x}^*\mathcal{D}_B}\mathbf{J}_{\mathcal{D}_B\mathcal{I}}\right)\mathbf{s}_\mathcal{I}\vspace{1mm}\\
			\displaystyle\quad+\left(\mathbf{K}_{\mathbf{x}^*\mathcal{I}}\mathbf{J}_{\mathcal{I}\mathcal{D}_B}+\mathbf{K}_{\mathbf{x}^*\mathcal{D}_B}\mathbf{J}_{\mathcal{D}_B\mathcal{D}_B}\right)\mathbf{y}_{\mathcal{D}_B}\ ,\vspace{2mm}\\
			\displaystyle\mathbb{V}_{p(f_{\mathbf{x}^*}|\mathbf{s}_\mathcal{I},\mathbf{y}_{\mathcal{D}_B},\mathbf{\Lambda},\sigma_f)}[f_{\mathbf{x}^*}] \vspace{1mm}\\
			=k_{\mathbf{x}^*\mathbf{x}^*}-
			\mathbf{K}_{\mathbf{x}^*(\mathcal{I}\cup\mathcal{D}_B)} 
			\mathbf{J}
			\mathbf{K}_{(\mathcal{I}\cup\mathcal{D}_B)\mathbf{x}^*}\ .
		\end{array}
	\end{equation*}
	Now,
	\begin{equation*}
	\hspace{-1.7mm}
		%\begin{aligned}
	\begin{array}{l}
		q(f_{\mathbf{x}^*}|\mathbf{y}_\mathcal{D})\\
		\displaystyle=\hspace{-1mm}\int\hspace{-0.5mm} p(f_{\mathbf{x}^*}|\mathbf{y}_{\mathcal{D}_B}, \mathbf{s}_\mathcal{I},\mathbf{\Lambda},\sigma_f)\ q^+(\mathbf{s}_\mathcal{I})\ q^+(\mathbf{\Lambda})\ q^+(\sigma_f)\mathrm{d}\mathbf{s}_\mathcal{I}\mathrm{d}\mathbf{\Lambda}\mathrm{d}\sigma_f
	\end{array}
		%&q(f_{\mathbf{x}^*}|\mathbf{y}_\mathcal{D})=\\
		%&\iint p(f_{\mathbf{x}^*}|\mathbf{s}_\mathcal{I}^+,\mathbf{\Lambda}^+,\mathbf{y}_{\mathcal{D}_B})q(\mathbf{s}_\mathcal{I}^+)q(\mathbf{\Lambda}^+)d\mathbf{s}_\mathcal{I}^+d\mathbf{\Lambda}^+ 
		%\end{aligned}
	\end{equation*}
	where
	\begin{equation*}
		\begin{array}{rcl}
			p(f_{\mathbf{x}^*}|\mathbf{y}_{\mathcal{D}_B}, \mathbf{s}_\mathcal{I},\mathbf{\Lambda},\sigma_f)&\hspace{-2.4mm}=&\hspace{-2.4mm}\displaystyle\mathcal{N}(f_{\mathbf{x}^*}|\mathbb{E}_{p(f_{\mathbf{x}^*}|\mathbf{s}_\mathcal{I},\mathbf{y}_{\mathcal{D}_B},\mathbf{\Lambda},\sigma_f)}[f_{\mathbf{x}^*}],\\
			&&\hspace{-2.4mm}\qquad\quad\mathbb{V}_{p(f_{\mathbf{x}^*}|\mathbf{s}_\mathcal{I},\mathbf{y}_{\mathcal{D}_B},\mathbf{\Lambda},\sigma_f)}[f_{\mathbf{x}^*}]) \ ,\vspace{1mm}\\
			q^+(\mathbf{s}_\mathcal{I})&\hspace{-2.4mm}=&\hspace{-2.4mm}\displaystyle\mathcal{N}(\mathbf{m}^+,\mathbf{S}^+)\ , \\
			q^+(\mathbf{\Lambda})&\hspace{-2.4mm}=&\hspace{-2.4mm}\displaystyle\prod_{i=1}^{d}\mathcal{N}(\lambda_i|\nu_i^+,\xi_i^+)\ , \vspace{1mm}\\
			q^+(\sigma_f)&\hspace{-2.4mm}=&\hspace{-2.4mm}\mathcal{N}(\alpha^+,\beta^+)\ .	
		\end{array}
	\end{equation*}
	%following the integration of conditional Gaussian distribution, we could get:
	%\begin{equation}
	%\begin{aligned}
	%p(y^*|\mathbf{s}_\mathcal{I}^+,\mathbf{\Lambda}^+,\mathbf{y}_{\mathcal{D}_B})&=\int p(y^{*}|f_{\mathbf{x}^*})p(f_{\mathbf{x}^*}|\mathbf{s}_\mathcal{I}^+,\mathbf{\Lambda}^+,\mathbf{y}_{\mathcal{D}_B})df_{\mathbf{x}^*} \\
	%&=\mathcal{N}(y^*|E^\prime,V^\prime+C_*)
	%\end{aligned}
	%\end{equation}
	%After integrating out the inducing variables $\mathbf{s}_\mathcal{I}^+$, we could get:
	Then,
	\begin{equation*}
		\begin{array}{l}
			\displaystyle q(f_{\mathbf{x}^*}|\mathbf{y}_\mathcal{D},\mathbf{\Lambda},\sigma_f)\\
			\displaystyle=\int p(f_{\mathbf{x}^*}|\mathbf{y}_{\mathcal{D}_B},\mathbf{s}_\mathcal{I},\mathbf{\Lambda},\sigma_f)\ q^+(\mathbf{s}_\mathcal{I})\ \mathrm{d}\mathbf{s}_\mathcal{I} \\
			\displaystyle =\mathcal{N}(\left(\mathbf{K}_{\mathbf{x}^*\mathcal{I}}\mathbf{J}_{\mathcal{I}\mathcal{I}}+\mathbf{K}_{\mathbf{x}^*\mathcal{D}_B}\mathbf{J}_{\mathcal{D}_B\mathcal{I}}\right)\mathbf{m}^+\\
			\displaystyle\qquad\ \ +\left(\mathbf{K}_{\mathbf{x}^*\mathcal{I}}\mathbf{J}_{\mathcal{I}\mathcal{D}_B}+\mathbf{K}_{\mathbf{x}^*\mathcal{D}_B}\mathbf{J}_{\mathcal{D}_B\mathcal{D}_B}\right)\mathbf{y}_{\mathcal{D}_B}\ ,\vspace{1mm}\\
			\displaystyle\qquad\ \ k_{\mathbf{x}^*\mathbf{x}^*}-
			\mathbf{K}_{\mathbf{x}^*(\mathcal{I}\cup\mathcal{D}_B)} 
			\mathbf{J}
			\mathbf{K}_{(\mathcal{I}\cup\mathcal{D}_B)\mathbf{x}^*}\\
			\displaystyle\qquad\ \ + \left(\mathbf{K}_{\mathbf{x}^*\mathcal{I}}\mathbf{J}_{\mathcal{I}\mathcal{I}}+\mathbf{K}_{\mathbf{x}^*\mathcal{D}_B}\mathbf{J}_{\mathcal{D}_B\mathcal{I}}\right)\\
			\displaystyle\qquad\qquad \mathbf{S}^+\left(\mathbf{K}_{\mathbf{x}^*\mathcal{I}}\mathbf{J}_{\mathcal{I}\mathcal{I}}+\mathbf{K}_{\mathbf{x}^*\mathcal{D}_B}\mathbf{J}_{\mathcal{D}_B\mathcal{I}}\right)^{\top} )\ .
		\end{array}
	\end{equation*}
	%
	Finally,
	\begin{equation*}
		\begin{array}{l}
			\displaystyle\mu_{\mathbf{x}^*|\mathcal{D}}\\
			\displaystyle\triangleq\mathbb{E}_{q(f_{\mathbf{x}^*}|\mathbf{y}_\mathcal{D})}[f_{\mathbf{x}^*}]\vspace{1mm}\\
			\displaystyle=\mathbb{E}_{q^+(\mathbf{\Lambda},\sigma_f)}[\mathbb{E}_{q(f_{\mathbf{x}^*}|\mathbf{y}_\mathcal{D},\mathbf{\Lambda},\sigma_f)}[f_{\mathbf{x}^*}]] \vspace{1mm}\\
			\displaystyle=\mathbb{E}_{q^+(\mathbf{\Lambda},\sigma_f)}[ \left(\mathbf{K}_{\mathbf{x}^*\mathcal{I}}\mathbf{J}_{\mathcal{I}\mathcal{I}}+\mathbf{K}_{\mathbf{x}^*\mathcal{D}_B}\mathbf{J}_{\mathcal{D}_B\mathcal{I}}\right)\mathbf{m}^+\\
			\displaystyle\qquad\qquad\quad\ \ +\left(\mathbf{K}_{\mathbf{x}^*\mathcal{I}}\mathbf{J}_{\mathcal{I}\mathcal{D}_B}+\mathbf{K}_{\mathbf{x}^*\mathcal{D}_B}\mathbf{J}_{\mathcal{D}_B\mathcal{D}_B}\right)\mathbf{y}_{\mathcal{D}_B} ] \vspace{1mm}\\
			\displaystyle=\mathbb{E}_{q^+(\mathbf{\Lambda},\sigma_f)}\left[\mathbf{K}_{\mathbf{x}^*\mathcal{I}}\mathbf{J}_{\mathcal{I}\mathcal{I}}+\mathbf{K}_{\mathbf{x}^*\mathcal{D}_B}\mathbf{J}_{\mathcal{D}_B\mathcal{I}}\right]\mathbf{m}^+\\
			\displaystyle\quad +\ \mathbb{E}_{q^+(\mathbf{\Lambda},\sigma_f)}\left[\mathbf{K}_{\mathbf{x}^*\mathcal{I}}\mathbf{J}_{\mathcal{I}\mathcal{D}_B}+\mathbf{K}_{\mathbf{x}^*\mathcal{D}_B}\mathbf{J}_{\mathcal{D}_B\mathcal{D}_B}\right]\mathbf{y}_{\mathcal{D}_B}\ .
		\end{array}
	\end{equation*}		
	\begin{equation*}
		\hspace{-5.7mm}
		\begin{array}{l}
			\displaystyle\sigma^2_{\mathbf{x}^*|\mathcal{D}}
			\displaystyle\triangleq\mathbb{V}_{q(f_{\mathbf{x}^*}|\mathbf{y}_\mathcal{D})}[f_{\mathbf{x}^*}] \vspace{1mm}\\
			\displaystyle=\mathbb{E}_{q^+(\mathbf{\Lambda},\sigma_f)}[\mathbb{V}_{q(f_{\mathbf{x}^*}|\mathbf{y}_\mathcal{D},\mathbf{\Lambda},\sigma_f)}[f_{\mathbf{x}^*}]] \\
			\ +\hspace{0.5mm} \mathbb{V}_{q^+(\mathbf{\Lambda},\sigma_f)}[\mathbb{E}_{q(f_{\mathbf{x}^*}|\mathbf{y}_\mathcal{D},\mathbf{\Lambda},\sigma_f)}[f_{\mathbf{x}^*}]]\vspace{1mm}\\
			\displaystyle=\mathbb{E}_{q^+(\mathbf{\Lambda},\sigma_f)}[k_{\mathbf{x}^*\mathbf{x}^*}-
			\mathbf{K}_{\mathbf{x}^*(\mathcal{I}\cup\mathcal{D}_B)} 
			\mathbf{J}
			\mathbf{K}_{(\mathcal{I}\cup\mathcal{D}_B)\mathbf{x}^*}\\
			\displaystyle\qquad\qquad\quad\ \  + \left(\mathbf{K}_{\mathbf{x}^*\mathcal{I}}\mathbf{J}_{\mathcal{I}\mathcal{I}}+\mathbf{K}_{\mathbf{x}^*\mathcal{D}_B}\mathbf{J}_{\mathcal{D}_B\mathcal{I}}\right)\\
			\displaystyle\qquad\qquad\quad\ \ \mathbf{S}^+\left(\mathbf{K}_{\mathbf{x}^*\mathcal{I}}\mathbf{J}_{\mathcal{I}\mathcal{I}}+\mathbf{K}_{\mathbf{x}^*\mathcal{D}_B}\mathbf{J}_{\mathcal{D}_B\mathcal{I}}\right)^{\top}]\vspace{1mm}\\
			\displaystyle\quad+\mathbb{V}_{q^+(\mathbf{\Lambda},\sigma_f)}\left[
			(\mathbf{m}^{+\top}\ \mathbf{y}^{\top}_{\mathcal{D}_B})
			\mathbf{J}
			\mathbf{K}_{(\mathcal{I}\cup\mathcal{D}_B)\mathbf{x}^*} \right]
		\end{array}
	\end{equation*}
	where
	\begin{equation*}
		%\hspace{-1.7mm}
		\begin{array}{l}
			\mathbb{V}_{q^+(\mathbf{\Lambda},\sigma_f)}\left[
			(\mathbf{m}^{+\top}\ \mathbf{y}^{\top}_{\mathcal{D}_B})
			\mathbf{J}
			\mathbf{K}_{(\mathcal{I}\cup\mathcal{D}_B)\mathbf{x}^*} \right]\\
			=(\mathbf{m}^{+\top}\ \mathbf{y}^{\top}_{\mathcal{D}_B})
			\mathbb{V}_{q^+(\mathbf{\Lambda},\sigma_f)}\left[
			\mathbf{J}
			\mathbf{K}_{(\mathcal{I}\cup\mathcal{D}_B)\mathbf{x}^*} \right]
			\begin{pmatrix}
				\mathbf{m}^{+} \\
				\mathbf{y}_{\mathcal{D}_B}
			\end{pmatrix}
			\vspace{1mm}\\
			=(\mathbf{m}^{+\top}\ \mathbf{y}^{\top}_{\mathcal{D}_B})
			\Big(\mathbb{E}_{q^+(\mathbf{\Lambda},\sigma_f)}\left[\mathbf{J}
			\mathbf{K}_{(\mathcal{I}\cup\mathcal{D}_B)\mathbf{x}^*}
			\mathbf{K}_{\mathbf{x}^*(\mathcal{I}\cup\mathcal{D}_B)}\mathbf{J} \right]\vspace{1mm}\\
			\displaystyle  -  \mathbb{E}_{q^+(\mathbf{\Lambda},\sigma_f)}[\mathbf{J}
			\mathbf{K}_{(\mathcal{I}\cup\mathcal{D}_B)\mathbf{x}^*} ]\mathbb{E}_{q^+(\mathbf{\Lambda},\sigma_f)}[\mathbf{K}_{\mathbf{x}^*(\mathcal{I}\cup\mathcal{D}_B)}\mathbf{J} ]\Big)\begin{pmatrix}
				\mathbf{m}^{+} \\
				\mathbf{y}_{\mathcal{D}_B}
			\end{pmatrix}\ .
		\end{array}
	\end{equation*}
	Unfortunately, the closed-form expressions of all the above expectation terms  with respect to $q^+(\mathbf{\Lambda},\sigma_f)\triangleq q^+(\mathbf{\Lambda})q^+(\sigma_f)$ cannot be obtained because it involves integrating, over $\mathbf{\Lambda}$, terms containing $\mathbf{J}$ that depends on $\mathbf{\Lambda}$ but without an analytical form with respect to $\mathbf{\Lambda}$.
	So, we approximate them via sampling.


\end{document}

Specifically, the joint distribution between latent output $\mathbf{f}_{\mathcal{D}}$ and inducing variable $\mathbf{s}_{\mathcal{I}}$ can be written as:
\begin{equation}
\hspace{0mm}
	\begin{pmatrix}
		\mathbf{f}_{\mathcal{D}} 
		\\
		\mathbf{s}_{\mathcal{I}}
	\end{pmatrix}
	\sim 
	\mathcal{N}\left( 
		\begin{pmatrix}
			\mathbf{0}
			\\
			\mathbf{0}
		\end{pmatrix}, 
		\\
		\\ 
		\begin{pmatrix}
			 \mathbf{K}_{\mathcal{D}\mathcal{D}} &  \mathbf{K}_{\mathcal{D}\mathcal{I}}
			 \\
			 \mathbf{K}_{\mathcal{D}\mathcal{I}}^\top & \mathbf{\Sigma}_{\mathcal{I}\mathcal{I}}
			 
		\end{pmatrix}
	\right)
\end{equation}
where $\mathbf{K}_{\mathcal{D}\mathcal{D}} =\left(\langle\sigma(\mathbf{x})\phi(\mathbf{\Lambda}\mathbf{x}), \sigma(\mathbf{x}')\phi(\mathbf{\Lambda}\mathbf{x}'\right)\rangle_{\mathcal{H}})_{\mathbf{x}, \mathbf{x}' \in \mathcal{D}}$, 
$\mathbf{K}_{\mathcal{D}\mathcal{I}} = (\langle\sigma(\mathbf{x})\phi(\mathbf{\Lambda}\mathbf{x}), \sigma(\mathbf{z})\phi(\mathbf{z})\rangle_{\mathcal{H}})_{\mathbf{x} \in \mathcal{D}, \mathbf{z} \in \mathcal{I}}$, 
$\mathbf{\Sigma}_{\mathcal{I}\mathcal{I}} = (\langle\sigma(\mathbf{z})\phi(\mathbf{z}), \sigma(\mathbf{z}')\phi(\mathbf{z}')\rangle_{\mathcal{H}})_{\mathbf{z}, \mathbf{z}' \in \mathcal{I}}$ (\textcolor{red}{reformat Eq. (2) in a nice way})