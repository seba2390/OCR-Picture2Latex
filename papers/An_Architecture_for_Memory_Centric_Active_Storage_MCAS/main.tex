\documentclass[letterpaper,twocolumn,10pt]{article}
\usepackage{usenix2019_v3}
\usepackage{comment}

\usepackage{tabularx}
\usepackage{graphicx}
\usepackage{xspace}
\usepackage{xcolor}
\usepackage{listings}
\usepackage{color}
\usepackage{subfig}
\usepackage{fancyhdr}

\pagestyle{fancy}
\lhead{}
\rhead{}
\rfoot{\thepage}
\cfoot{Revision 1.2}
\renewcommand{\headrulewidth}{0.4pt}
\renewcommand{\footrulewidth}{0.4pt}


\renewcommand{\headrulewidth}{0pt}

% Default fixed font does not support bold face
\DeclareFixedFont{\ttb}{T1}{txtt}{bx}{n}{9} % for bold
\DeclareFixedFont{\ttm}{T1}{txtt}{m}{n}{9}  % for normal

% code listing styles
% Custom colors

\definecolor{deepblue}{rgb}{0,0,0.5}
\definecolor{deepred}{rgb}{0.6,0,0}
\definecolor{deepgreen}{rgb}{0,0.5,0}

\lstset{numbers=left,xleftmargin=2em,frame=single,framexleftmargin=1.5em}

% Python style for highlighting
\newcommand\pythonstyle{\lstset{
language=Python,
basicstyle=\ttm,
numbers=left,
otherkeywords={self},             % Add keywords here
keywordstyle=\ttb\color{deepblue},
emph={MyClass,__init__},          % Custom highlighting
emphstyle=\ttb\color{deepred},    % Custom highlighting style
stringstyle=\color{deepgreen},
frame=tb,                         % Any extra options here
showstringspaces=false            % 
}}

\definecolor{delim}{RGB}{20,105,176}
\definecolor{numb}{RGB}{106, 109, 32}
\definecolor{string}{rgb}{0.64,0.08,0.08}

\lstdefinelanguage{json}{
    numbers=left,
    numberstyle=\scriptsize,
    frame=none,
    rulecolor=\color{black},
    showspaces=false,
    showtabs=false,
    breaklines=true,
    postbreak=\raisebox{0ex}[0ex][0ex]{\ensuremath{\color{gray}\hookrightarrow\space}},
    breakatwhitespace=true,
    basicstyle=\ttfamily\scriptsize,
    upquote=true,
    morestring=[b]",
    stringstyle=\color{string},
    literate=
     *{0}{{{\color{numb}0}}}{1}
      {1}{{{\color{numb}1}}}{1}
      {2}{{{\color{numb}2}}}{1}
      {3}{{{\color{numb}3}}}{1}
      {4}{{{\color{numb}4}}}{1}
      {5}{{{\color{numb}5}}}{1}
      {6}{{{\color{numb}6}}}{1}
      {7}{{{\color{numb}7}}}{1}
      {8}{{{\color{numb}8}}}{1}
      {9}{{{\color{numb}9}}}{1}
      {\{}{{{\color{delim}{\{}}}}{1}
      {\}}{{{\color{delim}{\}}}}}{1}
      {[}{{{\color{delim}{[}}}}{1}
      {]}{{{\color{delim}{]}}}}{1},
}



% Python environment
\lstnewenvironment{python}[1][]
{
\pythonstyle
\lstset{#1}
}
{}

% Python for external files
\newcommand\pythonexternal[2][]{{
\pythonstyle
\lstinputlisting[#1]{#2}}}

% Python for inline
\newcommand\py[1]{{\pythonstyle\lstinline!#1!}}

\newcommand{\todo}[1]{\textcolor{red}{{\it [TODO: #1]}}}
\newcommand{\ttcodefont}{\ttfamily\small}
\newcommand{\ttsmallcodefont}{\ttfamily\scriptsize}
\newcommand{\ttexcerptfont}{\ttfamily\scriptsize}


\newcommand{\hstore}{\code{hstore}\xspace}
\newcommand{\hstorecc}{\code{hstore-cc}\xspace}
\newcommand{\mapstore}{\code{mapstore}\xspace}
\newcommand{\pmem}{PM\xspace}
\newcommand{\currentversion}{v0.5.1\xspace}

\newcommand{\code}[1]{\begin{ttcodefont}#1\end{ttcodefont}}
\newcommand{\smallcode}[1]{\begin{ttsmallcodefont}#1\end{ttsmallcodefont}}

% color stuff
\newcommand{\mc}[2]{\multicolumn{#1}{c}{#2}}
\definecolor{Gray}{gray}{0.85}
\definecolor{LightCyan}{rgb}{0.88,1,1}
\newcolumntype{a}{>{\columncolor{Orange}}c}
\newcolumntype{b}{>{\columncolor{white}}c}

\lstset{language=C++,
  basicstyle=\ttfamily\scriptsize,
  keywordstyle=\color{blue}\ttfamily,
  stringstyle=\color{red}\ttfamily,
  commentstyle=\color{magenta}\ttfamily,
  escapechar=@,
  morecomment=[l][\color{magenta}]{\#}
}



%----------------------------------------------------------------------------------------
%	TITLE SECTION
%----------------------------------------------------------------------------------------

\date{}

% make title bold and 14 pt font (Latex default is non-bold, 16 pt)
\title{\Large \bf An Architecture for Memory Centric Active Storage (MCAS)}

%for single author (just remove % characters)

\author{
  {\rm Daniel G. Waddington}\\
  daniel.waddington@ibm.com \\
  IBM Research Almaden
\and
  {\rm Clem Dickey}\\
  dickeycl@us.ibm.com \\
  IBM Research Almaden
\and
  {\rm Moshik Hershcovitch}\\
  moshikh@il.ibm.com \\
  IBM Research Almaden
\and
  {\rm Sangeetha Seshadri}\\
  seshadrs@us.ibm.com \\
  IBM Research Almaden
} % end author




%----------------------------------------------------------------------------------------

\begin{document}

\date{} % Add a date here if you would like one to appear underneath the title block

\maketitle % Print the title

%% submission abstract
%%
%% The advent of CPU-attached persistent memory technology, such as Intels Optane Persistent Memory Modules (PMM), has brought with it new opportunities for storage. In 2018, IBM Research Almaden began investigating and developing a new enterprise-grade storage solution directly aimed at this emerging technology. MCAS  (Memory  Centric  Active  Storage)  defines  an evolved network-attached key-value store that offers both near-data compute and the ability to layer enterprise-grade data management services on shared persistent memory. As a converged memory-storage tier, MCAS moves towards eliminating the traditional separation of compute and storage, and thereby unifying the data space. This paper provides an in-depth review of the MCAS architecture and implementation, as well as general performance results.

%----------------------------------------------------------------------------------------
%	ABSTRACT
%----------------------------------------------------------------------------------------
\begin{abstract}
  
The advent of CPU-attached persistent memory technology, such as
Intel's Optane Persistent Memory Modules (PMM), has brought with it new
opportunities for storage.  In 2018, IBM Research Almaden
began investigating and developing a new enterprise-grade storage
solution directly aimed at this emerging technology.

MCAS (Memory Centric Active Storage) defines an ``evolved''
network-attached key-value store that offers both near-data compute
and the ability to layer enterprise-grade data management services on
shared persistent memory.  As a \textit{converged memory-storage
  tier}, MCAS moves towards eliminating the
traditional separation of compute and storage, and thereby unifying
the data space.

This paper provides an in-depth review of the MCAS architecture and
implementation, as well as general performance results.

\end{abstract}


%----------------------------------------------------------------------------------------
%	ARTICLE CONTENTS
%----------------------------------------------------------------------------------------

\section{Introduction}

% CLEMCOMMENT - Reads like "here are the neat features (and some
% drawbacks) of PM.  not a motivation to use PM to solve a
% problem. (And indeed, that's sort of what MCAS provides, neat
% features looking for a problem to solve. Still, I did not finish the
% intro thinking "Gee, how could one possibly solve problem XYZ? I
% have to read on to find out."
% DANCOMMENT - I agree, but not sure what to do.
Traditionally, the separation between volatile data in memory and
non-volatile data in storage devices (e.g., SSD) has been clear.  The
interface and semantics between the two domains is well defined; that
is, in the event of power-reset or power-fail events, data in memory
is lost and, in turn, is then retrieved from storage during the
recovery process.

With the advent of Persistent Memory (herein abbreviated to \pmem),
such as Intel's Optane DC Persistent Memory Modules (see
Figure~\ref{fig:aep}), this conventional separation of memory and
storage begins to blur.  Because \pmem behaves like memory, operations
on data held within \pmem can be performed in-place without having to
first load, and potentially de-serialize, from storage.  Likewise,
data written to \pmem need not be pushed down to storage to assure its
retention.  The result is that operations on durable data can be
performed at an order-of-magnitude lower latency than has been
previously possible.  Nevertheless, the Achilles' heel of \pmem is
that data management services traditionally realized by enterprise
storage systems (e.g., access control, encryption, replication,
compression, versioning, geo-distribution) cannot be easily realized
with additional software.

\begin{figure}[ht]
\centering
\includegraphics[width=0.8\columnwidth]{figs/intel-dimm.jpg}
\caption{Optane DC Persistent Memory Module}
\label{fig:aep}
\end{figure}

\pmem raises the data preservation boundary up
the stack into the main memory space.  It provides non-volatile memory
hardware that sits below existing volatile caches but that, unlike
existing DRAM-based memory, retains data in the event of failure
or reset.  The caveat is that data must be explicitly flushed from the
CPU cache (e.g. via \code{clflushopt}) for its persistence to be guaranteed.

It is also ``byte-addressable'' in that it is directly accessed via
load-store instructions provided by the CPU.  Intel Optane DC PMM,
which uses 3D XPoint (3DXP) technology, operates at a cache-line read-write
latency of around $300ns$ (see ~\cite{spectra2020,
  izraelevitz2019basic} for more detail).  Even though this is slower
than DRAM access latencies ($\sim100ns$) it is at least 30x faster
than state-of-the-art storage (e.g., NVMe SSD).  Capacity of \pmem is
also about 8x that of DRAM\footnote{For 3DXP, which is based on
  lattice-arranged Phase Change Memory (PCM).}.

Another consequence of \pmem being attached to the system as memory
is that it allows use of Direct Memory Access (DMA) and
Remote DMA (RDMA) to move data around.  For example, data can be
copied from \pmem to the network (via RDMA) or to another device such as
a GPU (via DMA), without requiring execution by the CPU.  This frees
the CPU to perform other tasks rather than executing \code{memcpy}
loops in order to move data.  Today, NVIDIA/Mellanox RDMA network
adapters can transfer data at near 400Gbps (50GiB/s) and therefore,
using multiple adapters, can even keep pace with the performance of \pmem.

\subsection{Current Limitations of Intel Optane \pmem}

While Intel Optane PMM provides many useful \pmem features as just
discussed a number of limitations are evident in the current generation.

\begin{itemize}

\item \textit{Endurance} - lifetime endurance of the hardware is
  significantly less than DRAM (3DXP at $10^6$ writes, DRAM at $10^{10}$)
  although orders-of-magnitude higher than NAND-flash. For intensive
  data write operations pushing through the cache this may be a
  significant limitation~\cite{227810}.

\item \textit{Asymmetric Performance Scaling} - write performance does
  not scale with increasing number of threads, while read performance
  scales at around 1.2\% degradation (from linear) per-core up to 28
  cores~\cite{9238605}.

\item \textit{64-bit Aligned Atomicity} - only aligned 64-bit writes
  can be guaranteed to happen atomically by the hardware.  There is
  currently no hardware support for multi-write atomicity/transactions and
  therefore this burden is left to the software.

\item \textit{Reliability \& Serviceability} - to provide maximum
  performance DIMMs must be configured to stripe data across 6
  devices.  In the event of a single DIMM failure, data on all of the
  DIMMs is effectively lost.
  
\item \textit{Cost} - although current 4Q2020 cost is $\sim0.5$x than
  that of DRAM, it is an order-of-magnitude higher than NAND-flash
  ($\sim 7\$/GB$ versus $\sim 1\$/GB$)

\end{itemize}

\section{Design Objectives and Solution Positioning}

With the previously discussed characteristics of \pmem in mind, the following
tenets in the design of MCAS were made.  The solution should:

\begin{enumerate}

\item Allow \pmem h/w resources to be shared as a network-attached
  capability using RDMA to provide maximum transfer speed.  Data
  sharing across independent nodes should be possible with appropriate
  locking/serialization provided by the MCAS system.
\item Maintain an immediate consistency model with guaranteed
  persistence (i.e. data is known to be flushed from volatile caches when a write is made).
\item Support zero-copy (RDMA-only) transfer of large data chunks
  enabling bulk data movement with CPU \code{memcpy} execution.
\item Minimize round-trip latency so that small reads/writes can be
  performed synchronously reducing s/w complexity in the client.
\item Scale through sharding to bound any performance degradation due
  to locking.
\item Provide flexible software-defined deployment for both
  on-premises and cloud.  Support both containerized and
  virtual-machine based deployment scenarios.
\item Enable safe, in-place user-defined operations directly on \pmem
  (both general and domain-specific).
\item Provide flexibility in the custom service layering (e.g.,
  combining replication and tiering).
  
\end{enumerate}

MCAS is positioned as a \textit{converged memory-storage tier}
providing high-performance random access to durable data.  Because
MCAS is based on \pmem it can provide fine-grained durability (per
write) as opposed to snap-shotting.  Even with synchronous, consistent
and guaranteed-persistent replication across multiple nodes, MCAS can
support still millions of updates per second.

%%-------------------------------------------------------------------------

\section{MCAS Core Architecture}

MCAS is implemented as a Linux process (known as the 'shard' process)
of which multiple instances can be concurrently deployed on the same
machine.  An MCAS process instance manages one or more network
end-points, each corresponds to a separate \textit{shard} (see
Figure~\ref{fig:hl_arch}).  

Shards are single-threaded and manage request handling for a
\textit{set} of pools.  They can be accessed concurrently by multiple
clients (from different nodes in the network) and can be
grouped into larger virtual data domains through client-side
clustering techniques such as consistent
hashing~\cite{TanenbaumSteen07}.

Each shard optionally maintains Active Data Object (ADO) processes
that provide custom functionality to the store.  The ADO processes
themselves may act as clients to other MCAS nodes (back-flow).  ADOs
are discussed in more detail in Section~\ref{sec:ado}.

\begin{figure}
\centering
\includegraphics[width=0.9\linewidth]{figs/high-level-arch.pdf}
\caption{MCAS High-Level Architecture}
\label{fig:hl_arch}
\end{figure}

Resources (memory, CPU cores) are statically allocated to
each shard through the MCAS configuration file. An example two-shard
configuration file is shown in Listing~\ref{lst:conf0}.

\begin{minipage}{\linewidth}
\begin{lstlisting}[language=json,
    caption={Example MCAS two-shard configuration},
    captionpos=b, label={lst:conf0}]
{
  "shards" :
  [
    {
      "core" : 0,
      "port" : 11911,
      "net"  : "mlx5_0",
      "default_backend" : "hstore",
      "dax_config" : [{
          "path": "/dev/dax0.0",
          "addr": "0x9000000000" }]
    },
    {
      "core" : 1,
      "port" : 11912,
      "net"  : "mlx5_0",
      "default_backend" : "hstore",
      "dax_config" : [{
          "path": "/dev/dax0.1",
          "addr": "0xA000000000" }]
    }
  ],
  "net_providers" : "verbs"
}
\end{lstlisting}
\end{minipage}

Each shard serves a single network end-point established using the
\textit{libfabric} library, which is part of the Open Fabric
Interfaces (OFI)
framework\footnote{https://ofiwg.github.io/libfabric/}.  This library
provides a common abstraction layer and services for high-performance
fabrics such as RDMA verbs, Intel TrueScale, and Cisco VIC.  It also
includes a provider for plain TCP/IP socket (TCP or UDP) but without
user-level and zero-copy capabilities.  MCAS primarily supports the
RDMA verbs and sockets providers.

Pools are they next level of data collection.  Each pool can only
belong to a single shard, which in turn means that the handling of
operations for a specific key-value pair is always performed by the
same shard and thus same thread.  Pools represent the \textit{security
  boundary} from a client perspective.  That is, access control and
memory resources are all bound to a pool.  If a client has
access-rights to a pool, then they also have access-rights to all
other key-value pairs in the pool.  

The overall data entity schema is given in Figure~\ref{fig:schema}.

\begin{figure}
\centering
\includegraphics[width=0.9\linewidth]{figs/schema.pdf}
\caption{MCAS Entity Relationships }
\label{fig:schema}
\end{figure}

%------------------------------------------------------------------------

\subsection{Client API} \label{subsec:Client API} 

Client applications interact with MCAS by linking to the client-API
library (\code{libcomponent-mcasclient.so}).  This library provides a
C++ based interface to MCAS.  The basic operations are very typical of
a traditional key-value store (see Table~\ref{tab:clientapi}); they
operate on \textit{opaque} values that are identified by a unique
key. Both keys and values are variable length and there is no
restriction on their size.

\begin{table}[t]
\begin{centering}
\begin{tabularx}{1.0\linewidth}{
>{\setlength{\hsize}{.3\hsize}\raggedright\footnotesize}X
>{\setlength{\hsize}{.7\hsize}\raggedright\arraybackslash\footnotesize}X }
  \hline
  \textbf{Function} & \textbf{Description} \\
  \hline
  \smallcode{create\_pool} & Create a new pool or open existing pool \\
  \smallcode{open\_pool} & Open existing pool (optional create on demand) \\
  \smallcode{close\_pool} & Release handle to pool \\
  \smallcode{delete\_pool} & Securely delete pool and release pool memory to shard \\
  \hline
  \smallcode{configure\_pool} & Configure pool (e.g., add secondary index) \\
  \hline
  \smallcode{put} & Write small ($<$ 2MiB) key-value pair.  Optionally allow overwrites \\
  \smallcode{get} & Read small ($<$ 2MiB) key-value pair \\
  \smallcode{async\_put} & Asynchronous version of put \\
  \smallcode{async\_get} & Asynchronous version of get \\

  \hline
  \smallcode{free\_memory} & Free memory allocated by get call \\
  \hline
  \smallcode{erase} & Erase key-value pair from pool \\
  \smallcode{async\_erase} & Asynchronous version of erase \\
  \hline
  \smallcode{get\_attributes} & Get attributes for pool or key/value pair \\
  \smallcode{get\_statistics} & Get shard statistics \\
  \smallcode{find} & Search key space in secondary index \\  
  \hline
\end{tabularx}
\caption{Basic Client API Summary}
\label{tab:clientapi}
\end{centering}
\end{table}

\subsubsection{Zero-copy Bulk Transfers}

MCAS also provides APIs for moving data to and from client-host memory
without a memory copy (memcpy) operation being performed under the
hood.  These \textit{direct transfer} APIs 
(see Table~\ref{tab:clientbulkapi}) are realized through the
underlying RDMA network hardware and allow data to be moved directly
from packet buffers into user-space memory (see
Figure~\ref{fig:direct}).  The memory for the direct APIs must be
allocated (e.g., via POSIX \code{alloc\_aligned}) and then registered
with the RDMA stack via the MCAS \code{register\_direct\_memory} call.
Under the hood, the direct APIs use RDMA read/write operations.
However, because the semantics of the MCAS protocol is
persistent-on-completion, two-sided operations (i.e. send/recv) are
still used to provide the acknowledgments (see Appendix A for detail).

The direct APIs can also be used with NVidia GPU-direct capabilities\footnote{https://docs.nvidia.com/cuda/gpudirect-rdma/index.html}.
This allows data to move from the MCAS server, across the network and
then directly from the NIC hardware into an application-defined region
of memory allocated inside the GPU.
In this scenario, 
the CPU
``host-code'' on the client must acquire the region of GPU memory
(e.g., via \code{cuMemAlloc}) and then register this memory with the
MCAS \code{register\_direct\_memory} call.  On completion of the
direct call on the CPU, movement of data into or out of the GPU is
known to be complete.

Depending on the PCIe arrangement and NIC hardware, direct transfers
are able to achieve transfer rates of tens of GiB/s.  Of course, these
transfers do not require CPU instruction execution and therefore the CPU is
free to perform other useful work.

\begin{figure}
\centering
\includegraphics[width=1.0\linewidth]{figs/direct_transfers.pdf}
\caption{MCAS Direct Transfers}
\label{fig:direct}
\end{figure}

\begin{table}[t]
\begin{centering}
\begin{tabularx}{1.0\linewidth}{
>{\setlength{\hsize}{.3\hsize}\raggedright\footnotesize}X
>{\setlength{\hsize}{.7\hsize}\raggedright\arraybackslash\footnotesize}X }
  \hline
  \textbf{Function} & \textbf{Description} \\
  \hline
  \smallcode{register\_direct\\\_memory} & Register client-allocated memory for direct API use\\
  \smallcode{unregister\_direct\\\_memory} & Unregister client-allocated memory for direct API use\\
  \smallcode{put\_direct} & Zero-copy large put operation using client provided memory \\
  \smallcode{get\_direct} & Zero-copy large get operation using client provided memory \\
  \smallcode{get\_direct\_offset} & Read sub-region of pool memory directly \\
  \smallcode{put\_direct\_offset} & Write sub-region of pool memory directly \\  
  \hline
  \smallcode{async\_put\_direct} & Asynchronous version of put\_direct \\
  \smallcode{async\_get\_direct} & Asynchronous version of get\_direct. \\
  \smallcode{async\_get\_direct\\\_offset} & Asynchronous version of get\_direct\_offset \\
  \smallcode{async\_put\_direct\\\_offset} & Asynchronous version of put\_direct\_offset \\
  \smallcode{check\_async\\\_completion} & Check for asynchronous operation completion \\
  \hline
\end{tabularx}
\caption{Advanced Bulk-transfer API}
\label{tab:clientbulkapi}
\end{centering}
\end{table}

\subsubsection{Direct Offset Operations}

For operations on large areas of memory the ability to perform
sub-region read/write operations is useful.  For this, MCAS provides
the \code{xxx\_direct\_offset} APIs (see Table~\ref{tab:clientbulkapi})
These functions allow direct read and write operations for a region of the value
space (associated with a key) that is defined by base offset and size pair.


\subsection{Primary Index Component}

MCAS uses a primary index to manage the mappings from key-to-value and
an optional secondary index for scanning of the key-space.

The primary index is provided as a \textit{storage engine} component
that provides broader services such as memory management.  It is a
pluggable component that implements are predefined interface
(\code{IKVStore}).  There are currently three storage engines included
in MCAS: \hstore, \hstorecc and \mapstore.

\hstore uses persistent memory for the main hash table and volatile
memory for the memory allocator itself.  Alternatively, \hstorecc uses
a persistent memory based memory allocator.  This allocator is slower
than its volatile memory counterpart, but it does not require
rebuilding after reset.

For DRAM-only scenarios, the \mapstore backend is available.  This is
based on a C++ STL \code{ordered\_set}.


\subsubsection{Memory Management}

Figure~\ref{fig:mem_arch} provides an overview of the \hstore memory
management architecture.  At the lowest level, MCAS uses either a
\textit{device-DAX} partition, which is a fixed-size partition of a
given interleave set or an \textit{fs-DAX} file. Device-DAX is used
when On-Demand Paging (ODP) hardware is not available.  A device-DAX
partition or fs-DAX file is configured for each shard.

To manage the shard memory resources, \hstore and \hstorecc use a
coarse-grained crash-consistent heap allocator, known as the
\textit{region allocator}. This allocates memory for individual pools.
Because shards are inherently single-threaded the region allocator
need not be thread-safe and is therefore lock-less.  Memory is managed
at a 32MiB granularity using an array of region descriptors (offset,
length, identifier) that are maintained in persistent memory.  Updates
to region descriptors are made write-atomic by using a simple undo-log
approach.

\begin{figure}
\centering
\includegraphics[width=1.0\linewidth]{figs/memory_arch.pdf}
\caption{Memory Management Architecture}
\label{fig:mem_arch}
\end{figure}

MCAS has no restrictions\footnote{Actually, the size is currently
  restricted by the RDMA verbs maximum frame size, which is 1GiB.} on
key and value lengths.  This means that a heap-allocator is necessary
to support variable-length region allocation.  To support high-rates
of key-value pair insertion and deletion, \hstore maintains a heap
allocator for key-value data in volatile (DRAM) memory.  However,
because the state of the allocator is neither crash-consistent or
power-fail durable, we must ``reconstitute'' its state after restart.
To achieve this, MCAS uses the key-value length information that is
stored in \pmem (and is crash-consistent) to rebuild the allocation
state.

% CLEMCOMMENT - the figure showsa minimum bucket size of 1 byte.
% DANCOMMENT - 2**0 = 1 ?
\begin{figure}
\centering
\includegraphics[width=1.0\linewidth]{figs/rcalloc.pdf}
\caption{Reconstituting Heap Allocator}
\label{fig:rcalloc}
\end{figure}

The reconstituting allocator manages a set of buckets that represent
different 2\textsuperscript{n} sizes (see Figure~\ref{fig:rcalloc}).
Objects (key or value data) up to a given size (4KiB) are allocated in
region lists.  Each region is 1MiB.  Objects are allocated by linear
scanning of the regions belonging to the corresponding bucket. If
there are no free slots in any region, a new region is allocated.
Regions and large-objects (> 4KiB) are allocated using an AVL-tree
based allocator.  If all slots in a region become free, the region can
be returned to the AVL-allocator.


\subsubsection{HStore Hash Table}

The core of \hstore is a hopscotch hash table~\cite{HerlihyST08}.  The
hash table is maintained in persistent memory. Expansion of the table
is achieved by adding successivevely larger \textit{segments}.  Each
additional segment doubles the table size.  Currently, shrinking of
the hash table is not supported.

To manage the mapping between the 64-bit hashed key, we adopt a
strategy also used by the Intel TBB hash table
implementation~\cite{Pheatt:2008:ITB:1352079.1352134}; see
Figure~\ref{fig:buckseg}.  The basic idea is to partition the hash
value into a left hand mask (where bits are ignored) and a right hand
set of bits representing the segment index and the segment offset
(i.e.  designating the bucket).  The high-order bit, outside of the
mask, is used to indicate the segment index.  The remaining bits are
used to define the bucket/segment offset.

\begin{figure}
\centering
\includegraphics[width=0.9\linewidth]{figs/buckseg.pdf}
\caption{Bucket-segment mapping}
\label{fig:buckseg}
\end{figure}

The hash table memory layout is given in \ref{fig:hopscotch}.  Each
entry in the table contains a \textit{hop information} bitmap of \textit{H}
bits (\textit{H}=63), that indicates which of the next \textit{H} entries contain
items that hashed to the current entry's virtual bucket.  In addition
to the hop information, each entry contains a state field, key-value
pointer-size pairs, and in-lined keys and values (when sufficiently
small).  Each entry fits a single cache line (64 bytes).

Segments are added by reducing the left hand mask, thus enabling
another position for the high-order bit.  The segment index is needed
because new segments cannot be assured to be allocated in contiguous
memory.  They are linked together to aid navigation of buckets beyond
the segment boundaries.

\begin{figure}[ht]
\centering
\includegraphics[width=1.0\linewidth]{figs/hopscotch.pdf}
\caption{Hash table arrangement}
\label{fig:hopscotch}
\end{figure}


\subsection{Secondary Index Component}

MCAS also supports the dynamic loading of a non-clustered ordered
index on the primary key, which we term the \textit{secondary
  index}\footnote{We use this term slightly different from the
  conventional database interpretation of forming an index on a
  different key from the primary key.}. This index, which is also
pluggable, implements a predefined interface (\code{IKVIndex}).  It
manages the key space only and its principal function is to provide an
index that can be efficiently scanned in key-order.  Scanning is based
on exact match, prefix or regular expression.  Currently, MCAS
provides only one (non-volatile) secondary index based on a volatile
red-black tree (C++ STL \code{map}).  Nevertheless, any alternative
secondary index can be easily developed and integrated into the
system.


\section{Active Data Objects (ADO)}
\label{sec:ado}

A key differentiator for MCAS is its ability to perform ``push-down''
operations in what are termed Active Data Objects (ADO).  The ADO
mechanism is based on an \textit{open protocol} layering approach (see
Figure~\ref{fig:adoopenproto}) in which a developer can implement a
client-side library (adapter) and server-side plugin that handle and interpret a
custom protocol.  Together these two components are referred to as the
\textit{personality}.

The ADO plugin is statically associated with a shard through a
parameter in the MCAS configuration file.

\begin{minipage}{\linewidth}
\begin{lstlisting}[language=json,
    caption={Example MCAS two-shard configuration},
    captionpos=b, label={lst:adoconf}]
{
  "shards" :
  [
    {
      "core" : 0,
      "port" : 11911,
      "net"  : "mlx5_0",
      "default_backend" : "hstore",
      "dax_config" : [{
          "path": "/dev/dax0.0",
          "addr": "0x9000000000" }],
      "ado_plugins" : [
        "libcomponent-adoplugin-rustexample.so",
        "libcomponent-adoplugin-passthru.so"
      ],
      "ado_cores" : "2",
      "ado_params" :  {
        "param1" : "some param",
        "param2" : "and another"
      }
    }
  ],
  "ado_path" : "/mcas/build/dist/bin/ado",
  "net_providers" : "verbs"
}
\end{lstlisting}
\end{minipage}

\begin{figure}
\centering
\includegraphics[width=0.8\linewidth]{figs/ADO_protostack.pdf}
\caption{ADO Open Protocol Architecture}
\label{fig:adoopenproto}
\end{figure}

Personalities can be used to create layered services and
functionality, both common and domain-specific.  Example common
services include replication, encryption, erasure-coding, versioning,
tiering, and snapshots.  Domain-specific services are centered around
data types, e.g., matrix operations, custom indices, and regular
expression matching.  Protocols are typically defined using Google
Flatbuffers, but other RPC frameworks can be used.

\begin{table}[ht]
\begin{centering}
\begin{tabularx}{1.0\linewidth}{
>{\setlength{\hsize}{.3\hsize}\raggedright\footnotesize}X
>{\setlength{\hsize}{.7\hsize}\raggedright\arraybackslash\footnotesize}X }
  \hline
  \textbf{Function} & \textbf{Description} \\
  \hline
  \smallcode{invoke\_ado} & Invoke Active Data Object \\
  \smallcode{async\_invoke\_ado} & Asynchronous version of invoke\_ado \\
  \smallcode{invoke\_put\_ado} & Invoke ADO with implicit put operation \\
  \smallcode{async\_invoke\_put\_ado} & Asynchronous version of invoke\_put\_ado \\
  \hline
\end{tabularx}
\caption{Advanced Client API}
\label{tab:clientadvapi}
\end{centering}
\end{table}

\subsection{ADO Invocation} \label{subsec:ADO Invocation}

To support the exchange of messages from the client to the ADO, MCAS
provides additional ``invocation'' APIs as summarized in
Table~\ref{tab:clientadvapi}.  These APIs are directed at a key-value
pair in an open pool.  In addition to the key, the parameters include
an opaque request, which encapsulates the protocol message:

\begin{figure}[h]
\begin{minipage}{\linewidth}
\begin{lstlisting}[frame=none]
status_t invoke_ado(
 const IMCAS::pool_t pool,
 const std::string& key,
 const void* request,
 const size_t request_len,
 const ado_flags_t flags,
 std::vector<ADO_response>& out_response,
 const size_t value_size = 0);
\end{lstlisting}
\end{minipage}
\end{figure}

The \code{invoke\_ado} invocation carries through to the ADO plugin on
Ethe server side.  Messages are passed from the main shard process to
the ADO process via a user-level IPC (UIPC) queue (see
Figure~\ref{fig:adoarch}).  UIPC is a user-level shared-memory region
that is instantiated with a lock-free FIFO.  Communications via UIPC
do not require a system call.  Before forwarding a message for an
\code{invoke\_ado} the shard thread locks the key-value pair so that
the ADO has the appropriate ownership.

\begin{figure}[ht]
\centering
\includegraphics[width=1.0\linewidth]{figs/threading_arch.pdf}
\caption{ADO Architecture}
\label{fig:adoarch}
\end{figure}

On receiving a message from the shard (via the UIPC queue)
the ADO process calls the plugin-implemented \code{do\_work} method. This
method is defined as follows:

\begin{figure}[h]
\begin{minipage}{\linewidth}
\begin{lstlisting}[frame=none]
status_t do_work(
  const uint64_t              work_id,
  const char*                 key,
  const size_t                key_len,
  IADO_plugin::value_space_t& values,
  const void*                 in_work_request,
  const size_t                in_work_request_len,
  const bool                  new_root,
  response_buffer_vector_t&   response_buffers) = 0;
\end{lstlisting}
\end{minipage}
\end{figure}

Note that the \code{do\_work} upcall\footnote{Moving up the stack}
provides both key-value and request information as well as an indication of
whether the key-value pair has been newly created.  If multiple ADO plugins are
defined (see Listing~\ref{lst:adoconf}), they are called on a
round-robin schedule.  Responses from the work is collected as a
vector, which is ultimately passed back to the client.

On returning from the \code{do\_work} call, the ADO container process
returns a message to the shard thread via the UIPC.  On receipt of the
completion notification the shard thread releases the corresponding
lock.

MCAS also provides the \code{invoke\_put\_ado} variation in the API.
This variation allows a value to be \code{put} immediately prior to the
invocation/upcall of the ADO plugin (avoiding the need for the client
to perform two consecutive calls).


\subsubsection{Experimental Signaling Hooks}

There are occasions when an ADO requires notification of
non-ADO invoke reads and writes (e.g., \code{put}/\code{get}) to
a pool.  As an experimental feature, MCAS supports the configuration
of \textit{ado signals} that relay notifications from non-ADO shard
operations.

\begin{lstlisting}[language=json,
    caption={Shard-level ADO Signal Configuration},
    captionpos=b, label={lst:adosignalconf}]
  ...
  "ado_signals" : ["post-put", "post-erase"],
  ...
\end{lstlisting}

Signals are dispatched to the ADO via the UIPC queue. Before the UIPC
message is sent, the corresponding key-value pair is 'read' locked.
Signals propagate to the ADO plugin via a \code{do\_work} invocation
% CLEMCOMMENT - Is "ADO::Signalpost-put" an example of a full message?
% Or maybe "ADO::Signal::post-put"
with a message prefix of `ADO::Signal'.  Responses to the clients are
postponed until completion of the ADO \code{do\_work} operation
(i.e. the client is stalled).

\subsection{ADO Execution Isolation}

The ADO plugin manipulates data that is
stored in persistent memory.  The shard process exposes the subject
pool memory to a separate ADO process.  This exposure is based either
on sharing of the fsdax file or performing a sub-space mapping with
devdax.  The latter requires a custom kernel module
(\code{mcasmod.ko}) to enable mapping between processes without a file
handle. Thus, the ADO process has complete visibility of the pool
memory, including the primary index; it cannot read or write memory
belonging to other pools.

ADO process instances effectively ``sandbox'' the compute performed by
the developer-supplied ADO plugin. The ADO process is launched on-demand and
remains operational while one or more clients has the corresponding
pool open.  Access to CPU cores and memory (persistent or DRAM) can be
specified via the MCAS shard configuration file. Operations performed
in the ADO generally cannot ``jam up'' or steal resources from the shard process.

Even though the ADO process has access to the primary index memory, it
does not manipulate this region of memory directly (avoiding thread
conflicts and the need for locking).  Instead, to perform operations
on the primary index, such as allocating a new key-value pair or
allocating memory from the pool, a callback interface is used.  The
callback functions are summarized in Table~\ref{tab:callbacks}.
Invocations on the callback interface by the plugin are passed back to
the shard thread via the UIPC queue.   

\begin{table}[ht]
\begin{centering}
\begin{tabularx}{1.0\linewidth}{
>{\setlength{\hsize}{.4\hsize}\raggedright\footnotesize}X
>{\setlength{\hsize}{.6\hsize}\raggedright\arraybackslash\footnotesize}X }
  \hline
  \textbf{Functions} & \textbf{Description} \\
  \hline
  \smallcode{create key/open key/erase key} & Key-value management \\
  \smallcode{resize value} & Resize existing value \\
  \smallcode{allocate/free memory} & Pool memory management \\
  \smallcode{get ref vector} & Get vector of key-value pairs \\
  \smallcode{iterate} & Iterate key-value pairs \\ 
  \smallcode{find key} & Scan for key through secondary index \\
  \smallcode{get pool info} & Retrieve memory utilization etc. \\
  \smallcode{unlock} & Explicitly unlock key-value pair \\
  \hline
\end{tabularx}
\caption{ADO plugin ``callback'' API}
\label{tab:callbacks}
\end{centering}
\end{table}

% CLEMCOMMENT - The release was not previously discussed, only the
% locking.
As previously discussed, locking for the ADO invoke target is explicitly
released when the call returns.  If the ADO operation creates or
opens other key-value pairs, then locks for these are also taken and
added to the deferred unlock list.

\subsubsection{Container-based Deployment}
MCAS is cloud-native ready.  Both the MCAS server process and ADO
processes can be deployed as Docker containers.  These containers are
built with provided Dockerfiles \code{Dockerfile.ado} and
\code{Dockerfile.mcas}.  Furthermore, MCAS can be deployed using the
Kubernates environment.  \code{mcas-server.yaml} is provided as a
reference pod configuration template.

\subsection{ADO Persistent Memory Operations}

Operations performed by the ADO that manipulate persistent memory must
be \textit{crash-consistent}.  That is, in the event of a
power-failure or reset, the data can be
``recovered'' to a known coherent state.  For example, given a list
insertion operation, in the event of recovery the list will either
have the element inserted or not at all; there will be no dangling
pointers or partially written data.

To support programming of crash-consistent ADO operations uses
memory-transactional programming through modified standard C++
templates libraries\footnote{We use EASTL from Electronic
Arts  because of its support to reset memory allocators.
(https://github.com/electronicarts/EASTL)}. 
The basic idea is to isolate the heap
memory for a data structure and instrument class methods with
\textit{undo logging}.  Each memory write is preceded by a copy-off
operation that saves the original state of the memory that will be
modified by the invocation. The copy-off operation is itself also
atomically transactional.  On completion of a program transaction,
which is made up of one or more method invocations on the data
structure, a \code{commit} call is made to clear the undo log.  In the
event of recovery from a ungraceful power-fail or reset event the undo
log is checked.  If the undo log is not empty, the copied-off regions
are restored to the heap and then the log is cleared.  This
effectively ``rewinds'' the data structure state to before the failed
transaction. More detail is given by the example listed in Appendix B.

Alternatively, PMDK or some other persistent programming methodology can
be used to write crash-consistent ADO operations. 

% not yet supported
%\subsubsection{Nested Indexing}

\section{Clustering}

To support coordination between MCAS processes at the ADO-level, MCAS
provides basic clustering support through the
\textit{libzyre}\footnote{https://github.com/zeromq/zyre} framework.
This library, based on ZeroMQ\footnote{https://zeromq.org/}, provides
proximity based peer-to-peer networking either through UDP beaconing
or a gossip protocol.  It uses reliable Dealer-Router pattern for
interconnections, assuring that messages are not lost unless a peer
application terminates.

Clustering is enabled by adding a \textit{cluster} section in the MCAS
configuration file, such as follows:

\begin{minipage}{\linewidth}
\begin{lstlisting}[language=json,caption={MCAS Process-level Cluster Configuration},captionpos=b, label={lst:clusterconf}]
{
    "cluster" :
    {
        "name" : "server-0",
        "group" : "MCAS-cluster-0",
        "addr" : "10.0.0.101",
        "port" : 11999
    },    
    "shards" :
    [
        {
          ...
\end{lstlisting}
\end{minipage}

Clustering is handled on a separate network port. Events, such
as a node joining or leaving a cluster, are propagated to each of
the shard threads and then to any active ADO processes. The plugin
method \code{cluster\_event} is up-called by the ADO thread receiving
notifications via the IPC queue:

\begin{minipage}{\linewidth}
\begin{lstlisting}[language=C++, frame=none]
void cluster_event(const std::string& sender,
                   const std::string& type,
                   const std::string& message);
\end{lstlisting}
\end{minipage}

Clustering can also be deployed directly in the ADO process directly
by using the cluster component (libcomponent-zyre.so).  This is a
useful approach when shard or pool activity needs to be associated to
a different cluster group name.

%% \section{Broader ADO-based Services}

%% The ``open protocol'' nature of the ADO design allows both general
%% and domain-specific services to be layered over the core key-value
%% functionality.

%% \begin{table}[ht]
%% \begin{centering}
%% \begin{tabularx}{1.0\linewidth}{
%% >{\setlength{\hsize}{.2\hsize}\raggedright\footnotesize}X
%% >{\setlength{\hsize}{.8\hsize}\raggedright\arraybackslash\footnotesize}X }
%%   \hline
%%   \textbf{Type} & \textbf{Description} \\
%%   \hline
%%   General & Replication, compression, encryption, erasure coding, snapshots, tiering, performance analysis, QoS management \\
%%   \hline
%%   Domain-Specific & Data manipulation, conversion, summarization, data derivation, domain operations (e.g., matrix multiply) \\
%%   \hline
%% \end{tabularx}
%% \caption{Example ADO services}
%% \label{tab:adoexamples}
%% \end{centering}
%% \end{table}

%% Table~\ref{tab:adoexamples} shows some example services that might be
%% implemented as ADO personalities in the MCAS system.

\section{Performance Evaluation}

In this section, we highlight the performance profile of MCAS through
experimental data. This results were collected from MCAS version v0.5.1
using the MPI-coordinated benchmark tool \code{mcas-mpi-bench}.

Our test system is based on the IBM Flashsystems platform,
which provides two PCIe-sharing canisters, each with two CPU sockets.
Table~\ref{tab:spec} gives the server system details.  To drive the
workload, four client systems are used. Even though the client systems are
of similar specification, they are not precisely the same.  This causes some
slight perturbation of results.

\begin{table}[ht]
\begin{centering}
\begin{tabularx}{1.0\linewidth}{
>{\setlength{\hsize}{.2\hsize}\raggedright\footnotesize}X
>{\setlength{\hsize}{.8\hsize}\raggedright\arraybackslash\footnotesize}X }
  \hline
  \textbf{Component} & \textbf{Description} \\
  \hline
  Processor & Intel Xeon Gold 5128 (Cascade Lake) 16-core 2.30GHz \\
  Cache & L1 (32KiB), L2 (1MiB), L3 (22MiB) \\ 
  DRAM & PC2400 DDR4 16GiB 12x DIMMs (192GiB) \\
  NVDIMM & Optane PMM 128GB 12x DIMMs (1.5TB) \\
  RDMA NIC & Mellanox ConnectX-5 (100GbE) \\
  OS & Linux Fedora 27 with Kernel 4.18.19 x86\_64 \\
  Compiler & GCC 7.3.1 \\
  NIC S/W & Mellanox OFED 4.5-1.0.1 \\   
  \hline
\end{tabularx}
\caption{Server system specification}
\label{tab:spec}
\end{centering}
\end{table}

The server and client nodes
are connected to a 100GiB Ethernet Switch (see
Figure~\ref{fig:net_top}).

\begin{figure}[]
\centering
\includegraphics[width=0.9\linewidth]{figs/net_topo.pdf}
\caption{Experimental Network Topology}
\label{fig:net_top}
\end{figure}

\subsection{Small Operations Throughput Scaling}

We examine the aggregate throughput of small \code{put} and \code{get}
operations (8-byte key, 16-byte value) with increasing number of
shards.  100\% read (\code{get}) and 100\% write (\code{put})
workloads are measured. Each shard is deployed as a separate process.
Five independent client threads, with separate pools and random keys,
drive the workload for each shard.  All client calls are synchronous
and the data is fully persistent and consistent on return.

\begin{figure}[ht]
\centering
\includegraphics[width=1.0\linewidth]{data/scaling-experiments/small-op-scaling.pdf}
\caption{Scaling of Small Operations}
\label{plot:smallop}
\end{figure}

The data shows that for 16 shards (i.e. a fully populated socket), the
system can support 2.72M puts/second and 7.69M gets/second.  At 16
shards, total degradation\footnote{Calculated from linear projection
  of observed single shard performance.} is ~39\% and ~29\%, for 100\%
read and 100\% write respectively.

\subsection{4KiB Operations Throughput Scaling}

We now examine throughput for \code{put} and \code{get} of larger 4KiB
values.  Key size remains at 8 bytes.  Note, this is not using the
zero-copy APIs (\code{put\_direct} and \code{get\_direct} that are
typically used for values above 128KiB.  The non-direct APIs must perform
memory copies on both client and server side.

\begin{figure}[ht!]
\centering
\includegraphics[width=1.0\linewidth]{data/scaling-experiments/4k-op-scaling.pdf}
\caption{Scaling of 4KiB Put/Get Operations}
\label{plot:4kop}
\end{figure}

Here read performance scales well up to around 12 shards.  Write performance
is considerably less. 
% MOSHIKHCOMMENT - The write is saturated at 8 threads and why this is happened?
% It is intersting it didn't happened at the previous graph.

\subsection{Aggregate Scaling at Different Value Sizes}

We now examine aggregate throughput for 16 shards on a single socket
with network-attached clients (see Figure~\ref{fig:net_top}).
Figure~\ref{plot:diffvalue-bw} shows data for 100\% read and 100\%
write workloads.  Values are doubled in size from 16B to 128KiB. Note that
the operations are non-direct (i.e. copy based).

\begin{figure}[ht!]
\centering
\includegraphics[width=1.0\linewidth]{data/diff-value-bw.pdf}
\caption{Aggregate-Scaling of Throughput for Changing Value Size}
\label{plot:diffvalue-bw}
\end{figure}

The results show that value sizes of 8KiB and above can saturate
over 90\% of the 100GbE bandwidth.


\subsection{128KiB Direct Operations Throughput Scaling}

We now examine performance scaling for the \code{put\_direct} and
\code{get\_direct} zero-copy operations.  Direct operations are not
used on smaller values due to overhead of performing scatter/gather
DMA.

For \code{get\_direct}, the 100GbE RDMA network connection becomes the
predominant limiter.  Network bandwidth is saturated at 6 shards.  For
\code{put\_direct}, the persistent memory is the limiter (at just over
10GiB/s).  This is congruent with findings reported
in~\cite{izraelevitz2019basic}.

\begin{figure}[ht!]
\centering
\includegraphics[width=1.0\linewidth]{data/scaling-experiments/128k-op-scaling.pdf}
\caption{Shard-Scaling of 128KiB Direct Put/Get Operations}
\label{plot:128kop}
\end{figure}

\goodbreak
\subsection{Latency}

A key differentiator for Persistent Memory is the ability to
achieve high throughput with low latencies.  The data is provided
as a histogram for 1 million samples across 40 bins.  Note that
the y-axes are presented with a logarithmic scale.

\begin{figure}[ht!]
  \centering

  \subfloat[Get (100\% read)]{
    \includegraphics[width=1.0\linewidth]{data/latency/get_16_lat.pdf}
  }

  \subfloat[Put (100\% write)]{
    \includegraphics[width=1.0\linewidth]{data/latency/put_16_lat.pdf}
  }

\caption{Latency Distribution for 16B Operations}
\end{figure}

\begin{figure}[ht!]
  \centering

  \subfloat[Get (100\% read)]{
    \includegraphics[width=1.0\linewidth]{data/latency/get_4k_lat.pdf}
  }

  \subfloat[Put (100\% write)]{
    \includegraphics[width=1.0\linewidth]{data/latency/put_4k_lat.pdf}
  }
\caption{Latency Distribution for 4KiB Operations}
\end{figure}

To provide the reader some comparison with SSD NVMe,
figure~\ref{fig:nvmef} shows the relationship between throughput and
latency for NVMe-over-fabric (same network).

\begin{figure}[ht!]

\centering
\includegraphics[width=1.0\linewidth]{data/nvme-f.pdf}
\caption{Throughput-Latency Relationship for NVMe-over-fabric}
\label{fig:nvmef}
\end{figure}


\newpage 
\subsection{Client Aggregation Fairness}

Here we look at how client performance is affected by ``competition''
on the same shard.  To do this, we collected throughput measures
(16-byte 100\% get operations) for an increasing number of threads.

\begin{figure}[ht!]
\centering
\includegraphics[width=1.0\linewidth]{data/scaling-experiments/client-agg.pdf}
\caption{Aggregation Behavior}
\label{plot:agg0}
\end{figure}

Saturation occurs at as little as four client threads.  Thereafter,
some nominal degradation occurs, but sharing is relatively fair
(i.e. each client gets its $n$-th share of performance where $n$ is
the number of clients).


\subsection{ADO Invocation Throughput Scaling}

To examine the throughput of ADO invocations in MCAS, we use the
\code{ado\_perf} tool. The test makes invocations using an 8-byte
message payload.  On the server, the shards are configured with the
``passthru'' plugin (\code{libcomponent-adoplugin-passthru.so}) which
performs no real compute.  The results shown are for a single-key
target, and a pool of keys target.  In the latter, 100K keys are used
that belong to the same pool.

\begin{figure}[ht!]
\centering
\includegraphics[width=1.0\linewidth]{data/scaling-experiments/ado-invoke-scaling.pdf}
\caption{Scaling of ADO Invocations}
\label{plot:adoscaling}
\end{figure}

The results show that at 16 shards, aggregate throughput is 7.5M IOPS
and 4.3M IOPS for same-key and key-set tests respectively.
Degradation from linear is 1.4\% for same-key and 18\% for key-set.


\section{Further Work}

Currently, MCAS is a research prototype. 
As we mature the platform to
a more robust production-grade solution, the following new features
will be considered:

\begin{itemize}
\item Additional language bindings (e.g., Rust, Go, Java).
\item Pool level authentication and admission control.
\item Improvements to crash-consistent ADO development (e.g., via h/w or compiler support).
\item Unified ADO protocol for enhanced key-value operations.
\item Boilerplate code for client-side replication.
\item Boilerplate code for FPGA accelerator integration.
\end{itemize}

\section{Availability}

The code for MCAS is available under an Apache 2.0 open source
license at https://github.com/IBM/mcas/.  Additional information is
also available at https://ibm.github.io/mcas/.

\section{Conclusion}

MCAS is a new type of system that aims to offer memory-storage convergence.  Based on
persistent memory and RDMA hardware, MCAS extends the traditional key-value paradigm
to support arbitrary (structured) value types with the potential for in-place operations.
Our initial experiments show that the ADO approach can result in a significant reduction
in network data movement and thus overall performance.


Going forward, our vision is to take MCAS beyond persistent memory and
position it for emerging Near-Memory Compute (NMC) and
Processing-in-Memory (PIM)
technologies~\cite{10.1145/2845084,10.1145/3299874.3317977,10.1145/3307650.3322237,
  10.1145/3036669.3038242, 10.1145/3400302.3415772, 10.1145/2997649}.
These technologies will be vital in addressing the memory-wall and
processor scale-up problems.
% CLEMCOMMENT - "make" did not generate the bibliography, for me.
% Comments on LaTeX forums seems to say that a "bibtex step ought to be run:
%  latex
%  bibtex
%  latex
%  latex
% But the make file doesn't latex, it runs latexpdf. And I don't see a bibtexpdf
% command.
% DANCOMMENT
% Just run 'bibtex main' 
\bibliographystyle{abbrv} 
\bibliography{references}

\chapter{Supplementary Material}
\label{appendix}

In this appendix, we present supplementary material for the techniques and
experiments presented in the main text.

\section{Baseline Results and Analysis for Informed Sampler}
\label{appendix:chap3}

Here, we give an in-depth
performance analysis of the various samplers and the effect of their
hyperparameters. We choose hyperparameters with the lowest PSRF value
after $10k$ iterations, for each sampler individually. If the
differences between PSRF are not significantly different among
multiple values, we choose the one that has the highest acceptance
rate.

\subsection{Experiment: Estimating Camera Extrinsics}
\label{appendix:chap3:room}

\subsubsection{Parameter Selection}
\paragraph{Metropolis Hastings (\MH)}

Figure~\ref{fig:exp1_MH} shows the median acceptance rates and PSRF
values corresponding to various proposal standard deviations of plain
\MH~sampling. Mixing gets better and the acceptance rate gets worse as
the standard deviation increases. The value $0.3$ is selected standard
deviation for this sampler.

\paragraph{Metropolis Hastings Within Gibbs (\MHWG)}

As mentioned in Section~\ref{sec:room}, the \MHWG~sampler with one-dimensional
updates did not converge for any value of proposal standard deviation.
This problem has high correlation of the camera parameters and is of
multi-modal nature, which this sampler has problems with.

\paragraph{Parallel Tempering (\PT)}

For \PT~sampling, we took the best performing \MH~sampler and used
different temperature chains to improve the mixing of the
sampler. Figure~\ref{fig:exp1_PT} shows the results corresponding to
different combination of temperature levels. The sampler with
temperature levels of $[1,3,27]$ performed best in terms of both
mixing and acceptance rate.

\paragraph{Effect of Mixture Coefficient in Informed Sampling (\MIXLMH)}

Figure~\ref{fig:exp1_alpha} shows the effect of mixture
coefficient ($\alpha$) on the informed sampling
\MIXLMH. Since there is no significant different in PSRF values for
$0 \le \alpha \le 0.7$, we chose $0.7$ due to its high acceptance
rate.


% \end{multicols}

\begin{figure}[h]
\centering
  \subfigure[MH]{%
    \includegraphics[width=.48\textwidth]{figures/supplementary/camPose_MH.pdf} \label{fig:exp1_MH}
  }
  \subfigure[PT]{%
    \includegraphics[width=.48\textwidth]{figures/supplementary/camPose_PT.pdf} \label{fig:exp1_PT}
  }
\\
  \subfigure[INF-MH]{%
    \includegraphics[width=.48\textwidth]{figures/supplementary/camPose_alpha.pdf} \label{fig:exp1_alpha}
  }
  \mycaption{Results of the `Estimating Camera Extrinsics' experiment}{PRSFs and Acceptance rates corresponding to (a) various standard deviations of \MH, (b) various temperature level combinations of \PT sampling and (c) various mixture coefficients of \MIXLMH sampling.}
\end{figure}



\begin{figure}[!t]
\centering
  \subfigure[\MH]{%
    \includegraphics[width=.48\textwidth]{figures/supplementary/occlusionExp_MH.pdf} \label{fig:exp2_MH}
  }
  \subfigure[\BMHWG]{%
    \includegraphics[width=.48\textwidth]{figures/supplementary/occlusionExp_BMHWG.pdf} \label{fig:exp2_BMHWG}
  }
\\
  \subfigure[\MHWG]{%
    \includegraphics[width=.48\textwidth]{figures/supplementary/occlusionExp_MHWG.pdf} \label{fig:exp2_MHWG}
  }
  \subfigure[\PT]{%
    \includegraphics[width=.48\textwidth]{figures/supplementary/occlusionExp_PT.pdf} \label{fig:exp2_PT}
  }
\\
  \subfigure[\INFBMHWG]{%
    \includegraphics[width=.5\textwidth]{figures/supplementary/occlusionExp_alpha.pdf} \label{fig:exp2_alpha}
  }
  \mycaption{Results of the `Occluding Tiles' experiment}{PRSF and
    Acceptance rates corresponding to various standard deviations of
    (a) \MH, (b) \BMHWG, (c) \MHWG, (d) various temperature level
    combinations of \PT~sampling and; (e) various mixture coefficients
    of our informed \INFBMHWG sampling.}
\end{figure}

%\onecolumn\newpage\twocolumn
\subsection{Experiment: Occluding Tiles}
\label{appendix:chap3:tiles}

\subsubsection{Parameter Selection}

\paragraph{Metropolis Hastings (\MH)}

Figure~\ref{fig:exp2_MH} shows the results of
\MH~sampling. Results show the poor convergence for all proposal
standard deviations and rapid decrease of AR with increasing standard
deviation. This is due to the high-dimensional nature of
the problem. We selected a standard deviation of $1.1$.

\paragraph{Blocked Metropolis Hastings Within Gibbs (\BMHWG)}

The results of \BMHWG are shown in Figure~\ref{fig:exp2_BMHWG}. In
this sampler we update only one block of tile variables (of dimension
four) in each sampling step. Results show much better performance
compared to plain \MH. The optimal proposal standard deviation for
this sampler is $0.7$.

\paragraph{Metropolis Hastings Within Gibbs (\MHWG)}

Figure~\ref{fig:exp2_MHWG} shows the result of \MHWG sampling. This
sampler is better than \BMHWG and converges much more quickly. Here
a standard deviation of $0.9$ is found to be best.

\paragraph{Parallel Tempering (\PT)}

Figure~\ref{fig:exp2_PT} shows the results of \PT sampling with various
temperature combinations. Results show no improvement in AR from plain
\MH sampling and again $[1,3,27]$ temperature levels are found to be optimal.

\paragraph{Effect of Mixture Coefficient in Informed Sampling (\INFBMHWG)}

Figure~\ref{fig:exp2_alpha} shows the effect of mixture
coefficient ($\alpha$) on the blocked informed sampling
\INFBMHWG. Since there is no significant different in PSRF values for
$0 \le \alpha \le 0.8$, we chose $0.8$ due to its high acceptance
rate.



\subsection{Experiment: Estimating Body Shape}
\label{appendix:chap3:body}

\subsubsection{Parameter Selection}
\paragraph{Metropolis Hastings (\MH)}

Figure~\ref{fig:exp3_MH} shows the result of \MH~sampling with various
proposal standard deviations. The value of $0.1$ is found to be
best.

\paragraph{Metropolis Hastings Within Gibbs (\MHWG)}

For \MHWG sampling we select $0.3$ proposal standard
deviation. Results are shown in Fig.~\ref{fig:exp3_MHWG}.


\paragraph{Parallel Tempering (\PT)}

As before, results in Fig.~\ref{fig:exp3_PT}, the temperature levels
were selected to be $[1,3,27]$ due its slightly higher AR.

\paragraph{Effect of Mixture Coefficient in Informed Sampling (\MIXLMH)}

Figure~\ref{fig:exp3_alpha} shows the effect of $\alpha$ on PSRF and
AR. Since there is no significant differences in PSRF values for $0 \le
\alpha \le 0.8$, we choose $0.8$.


\begin{figure}[t]
\centering
  \subfigure[\MH]{%
    \includegraphics[width=.48\textwidth]{figures/supplementary/bodyShape_MH.pdf} \label{fig:exp3_MH}
  }
  \subfigure[\MHWG]{%
    \includegraphics[width=.48\textwidth]{figures/supplementary/bodyShape_MHWG.pdf} \label{fig:exp3_MHWG}
  }
\\
  \subfigure[\PT]{%
    \includegraphics[width=.48\textwidth]{figures/supplementary/bodyShape_PT.pdf} \label{fig:exp3_PT}
  }
  \subfigure[\MIXLMH]{%
    \includegraphics[width=.48\textwidth]{figures/supplementary/bodyShape_alpha.pdf} \label{fig:exp3_alpha}
  }
\\
  \mycaption{Results of the `Body Shape Estimation' experiment}{PRSFs and
    Acceptance rates corresponding to various standard deviations of
    (a) \MH, (b) \MHWG; (c) various temperature level combinations
    of \PT sampling and; (d) various mixture coefficients of the
    informed \MIXLMH sampling.}
\end{figure}


\subsection{Results Overview}
Figure~\ref{fig:exp_summary} shows the summary results of the all the three
experimental studies related to informed sampler.
\begin{figure*}[h!]
\centering
  \subfigure[Results for: Estimating Camera Extrinsics]{%
    \includegraphics[width=0.9\textwidth]{figures/supplementary/camPose_ALL.pdf} \label{fig:exp1_all}
  }
  \subfigure[Results for: Occluding Tiles]{%
    \includegraphics[width=0.9\textwidth]{figures/supplementary/occlusionExp_ALL.pdf} \label{fig:exp2_all}
  }
  \subfigure[Results for: Estimating Body Shape]{%
    \includegraphics[width=0.9\textwidth]{figures/supplementary/bodyShape_ALL.pdf} \label{fig:exp3_all}
  }
  \label{fig:exp_summary}
  \mycaption{Summary of the statistics for the three experiments}{Shown are
    for several baseline methods and the informed samplers the
    acceptance rates (left), PSRFs (middle), and RMSE values
    (right). All results are median results over multiple test
    examples.}
\end{figure*}

\subsection{Additional Qualitative Results}

\subsubsection{Occluding Tiles}
In Figure~\ref{fig:exp2_visual_more} more qualitative results of the
occluding tiles experiment are shown. The informed sampling approach
(\INFBMHWG) is better than the best baseline (\MHWG). This still is a
very challenging problem since the parameters for occluded tiles are
flat over a large region. Some of the posterior variance of the
occluded tiles is already captured by the informed sampler.

\begin{figure*}[h!]
\begin{center}
\centerline{\includegraphics[width=0.95\textwidth]{figures/supplementary/occlusionExp_Visual.pdf}}
\mycaption{Additional qualitative results of the occluding tiles experiment}
  {From left to right: (a)
  Given image, (b) Ground truth tiles, (c) OpenCV heuristic and most probable estimates
  from 5000 samples obtained by (d) MHWG sampler (best baseline) and
  (e) our INF-BMHWG sampler. (f) Posterior expectation of the tiles
  boundaries obtained by INF-BMHWG sampling (First 2000 samples are
  discarded as burn-in).}
\label{fig:exp2_visual_more}
\end{center}
\end{figure*}

\subsubsection{Body Shape}
Figure~\ref{fig:exp3_bodyMeshes} shows some more results of 3D mesh
reconstruction using posterior samples obtained by our informed
sampling \MIXLMH.

\begin{figure*}[t]
\begin{center}
\centerline{\includegraphics[width=0.75\textwidth]{figures/supplementary/bodyMeshResults.pdf}}
\mycaption{Qualitative results for the body shape experiment}
  {Shown is the 3D mesh reconstruction results with first 1000 samples obtained
  using the \MIXLMH informed sampling method. (blue indicates small
  values and red indicates high values)}
\label{fig:exp3_bodyMeshes}
\end{center}
\end{figure*}

\clearpage



\section{Additional Results on the Face Problem with CMP}

Figure~\ref{fig:shading-qualitative-multiple-subjects-supp} shows inference results for reflectance maps, normal maps and lights for randomly chosen test images, and Fig.~\ref{fig:shading-qualitative-same-subject-supp} shows reflectance estimation results on multiple images of the same subject produced under different illumination conditions. CMP is able to produce estimates that are closer to the groundtruth across different subjects and illumination conditions.

\begin{figure*}[h]
  \begin{center}
  \centerline{\includegraphics[width=1.0\columnwidth]{figures/face_cmp_visual_results_supp.pdf}}
  \vspace{-1.2cm}
  \end{center}
	\mycaption{A visual comparison of inference results}{(a)~Observed images. (b)~Inferred reflectance maps. \textit{GT} is the photometric stereo groundtruth, \textit{BU} is the Biswas \etal (2009) reflectance estimate and \textit{Forest} is the consensus prediction. (c)~The variance of the inferred reflectance estimate produced by \MTD (normalized across rows).(d)~Visualization of inferred light directions. (e)~Inferred normal maps.}
	\label{fig:shading-qualitative-multiple-subjects-supp}
\end{figure*}


\begin{figure*}[h]
	\centering
	\setlength\fboxsep{0.2mm}
	\setlength\fboxrule{0pt}
	\begin{tikzpicture}

		\matrix at (0, 0) [matrix of nodes, nodes={anchor=east}, column sep=-0.05cm, row sep=-0.2cm]
		{
			\fbox{\includegraphics[width=1cm]{figures/sample_3_4_X.png}} &
			\fbox{\includegraphics[width=1cm]{figures/sample_3_4_GT.png}} &
			\fbox{\includegraphics[width=1cm]{figures/sample_3_4_BISWAS.png}}  &
			\fbox{\includegraphics[width=1cm]{figures/sample_3_4_VMP.png}}  &
			\fbox{\includegraphics[width=1cm]{figures/sample_3_4_FOREST.png}}  &
			\fbox{\includegraphics[width=1cm]{figures/sample_3_4_CMP.png}}  &
			\fbox{\includegraphics[width=1cm]{figures/sample_3_4_CMPVAR.png}}
			 \\

			\fbox{\includegraphics[width=1cm]{figures/sample_3_5_X.png}} &
			\fbox{\includegraphics[width=1cm]{figures/sample_3_5_GT.png}} &
			\fbox{\includegraphics[width=1cm]{figures/sample_3_5_BISWAS.png}}  &
			\fbox{\includegraphics[width=1cm]{figures/sample_3_5_VMP.png}}  &
			\fbox{\includegraphics[width=1cm]{figures/sample_3_5_FOREST.png}}  &
			\fbox{\includegraphics[width=1cm]{figures/sample_3_5_CMP.png}}  &
			\fbox{\includegraphics[width=1cm]{figures/sample_3_5_CMPVAR.png}}
			 \\

			\fbox{\includegraphics[width=1cm]{figures/sample_3_6_X.png}} &
			\fbox{\includegraphics[width=1cm]{figures/sample_3_6_GT.png}} &
			\fbox{\includegraphics[width=1cm]{figures/sample_3_6_BISWAS.png}}  &
			\fbox{\includegraphics[width=1cm]{figures/sample_3_6_VMP.png}}  &
			\fbox{\includegraphics[width=1cm]{figures/sample_3_6_FOREST.png}}  &
			\fbox{\includegraphics[width=1cm]{figures/sample_3_6_CMP.png}}  &
			\fbox{\includegraphics[width=1cm]{figures/sample_3_6_CMPVAR.png}}
			 \\
	     };

       \node at (-3.85, -2.0) {\small Observed};
       \node at (-2.55, -2.0) {\small `GT'};
       \node at (-1.27, -2.0) {\small BU};
       \node at (0.0, -2.0) {\small MP};
       \node at (1.27, -2.0) {\small Forest};
       \node at (2.55, -2.0) {\small \textbf{CMP}};
       \node at (3.85, -2.0) {\small Variance};

	\end{tikzpicture}
	\mycaption{Robustness to varying illumination}{Reflectance estimation on a subject images with varying illumination. Left to right: observed image, photometric stereo estimate (GT)
  which is used as a proxy for groundtruth, bottom-up estimate of \cite{Biswas2009}, VMP result, consensus forest estimate, CMP mean, and CMP variance.}
	\label{fig:shading-qualitative-same-subject-supp}
\end{figure*}

\clearpage

\section{Additional Material for Learning Sparse High Dimensional Filters}
\label{sec:appendix-bnn}

This part of supplementary material contains a more detailed overview of the permutohedral
lattice convolution in Section~\ref{sec:permconv}, more experiments in
Section~\ref{sec:addexps} and additional results with protocols for
the experiments presented in Chapter~\ref{chap:bnn} in Section~\ref{sec:addresults}.

\vspace{-0.2cm}
\subsection{General Permutohedral Convolutions}
\label{sec:permconv}

A core technical contribution of this work is the generalization of the Gaussian permutohedral lattice
convolution proposed in~\cite{adams2010fast} to the full non-separable case with the
ability to perform back-propagation. Although, conceptually, there are minor
differences between Gaussian and general parameterized filters, there are non-trivial practical
differences in terms of the algorithmic implementation. The Gauss filters belong to
the separable class and can thus be decomposed into multiple
sequential one dimensional convolutions. We are interested in the general filter
convolutions, which can not be decomposed. Thus, performing a general permutohedral
convolution at a lattice point requires the computation of the inner product with the
neighboring elements in all the directions in the high-dimensional space.

Here, we give more details of the implementation differences of separable
and non-separable filters. In the following, we will explain the scalar case first.
Recall, that the forward pass of general permutohedral convolution
involves 3 steps: \textit{splatting}, \textit{convolving} and \textit{slicing}.
We follow the same splatting and slicing strategies as in~\cite{adams2010fast}
since these operations do not depend on the filter kernel. The main difference
between our work and the existing implementation of~\cite{adams2010fast} is
the way that the convolution operation is executed. This proceeds by constructing
a \emph{blur neighbor} matrix $K$ that stores for every lattice point all
values of the lattice neighbors that are needed to compute the filter output.

\begin{figure}[t!]
  \centering
    \includegraphics[width=0.6\columnwidth]{figures/supplementary/lattice_construction}
  \mycaption{Illustration of 1D permutohedral lattice construction}
  {A $4\times 4$ $(x,y)$ grid lattice is projected onto the plane defined by the normal
  vector $(1,1)^{\top}$. This grid has $s+1=4$ and $d=2$ $(s+1)^{d}=4^2=16$ elements.
  In the projection, all points of the same color are projected onto the same points in the plane.
  The number of elements of the projected lattice is $t=(s+1)^d-s^d=4^2-3^2=7$, that is
  the $(4\times 4)$ grid minus the size of lattice that is $1$ smaller at each size, in this
  case a $(3\times 3)$ lattice (the upper right $(3\times 3)$ elements).
  }
\label{fig:latticeconstruction}
\end{figure}

The blur neighbor matrix is constructed by traversing through all the populated
lattice points and their neighboring elements.
% For efficiency, we do this matrix construction recursively with shared computations
% since $n^{th}$ neighbourhood elements are $1^{st}$ neighborhood elements of $n-1^{th}$ neighbourhood elements. \pg{do not understand}
This is done recursively to share computations. For any lattice point, the neighbors that are
$n$ hops away are the direct neighbors of the points that are $n-1$ hops away.
The size of a $d$ dimensional spatial filter with width $s+1$ is $(s+1)^{d}$ (\eg, a
$3\times 3$ filter, $s=2$ in $d=2$ has $3^2=9$ elements) and this size grows
exponentially in the number of dimensions $d$. The permutohedral lattice is constructed by
projecting a regular grid onto the plane spanned by the $d$ dimensional normal vector ${(1,\ldots,1)}^{\top}$. See
Fig.~\ref{fig:latticeconstruction} for an illustration of the 1D lattice construction.
Many corners of a grid filter are projected onto the same point, in total $t = {(s+1)}^{d} -
s^{d}$ elements remain in the permutohedral filter with $s$ neighborhood in $d-1$ dimensions.
If the lattice has $m$ populated elements, the
matrix $K$ has size $t\times m$. Note that, since the input signal is typically
sparse, only a few lattice corners are being populated in the \textit{slicing} step.
We use a hash-table to keep track of these points and traverse only through
the populated lattice points for this neighborhood matrix construction.

Once the blur neighbor matrix $K$ is constructed, we can perform the convolution
by the matrix vector multiplication
\begin{equation}
\ell' = BK,
\label{eq:conv}
\end{equation}
where $B$ is the $1 \times t$ filter kernel (whose values we will learn) and $\ell'\in\mathbb{R}^{1\times m}$
is the result of the filtering at the $m$ lattice points. In practice, we found that the
matrix $K$ is sometimes too large to fit into GPU memory and we divided the matrix $K$
into smaller pieces to compute Eq.~\ref{eq:conv} sequentially.

In the general multi-dimensional case, the signal $\ell$ is of $c$ dimensions. Then
the kernel $B$ is of size $c \times t$ and $K$ stores the $c$ dimensional vectors
accordingly. When the input and output points are different, we slice only the
input points and splat only at the output points.


\subsection{Additional Experiments}
\label{sec:addexps}
In this section, we discuss more use-cases for the learned bilateral filters, one
use-case of BNNs and two single filter applications for image and 3D mesh denoising.

\subsubsection{Recognition of subsampled MNIST}\label{sec:app_mnist}

One of the strengths of the proposed filter convolution is that it does not
require the input to lie on a regular grid. The only requirement is to define a distance
between features of the input signal.
We highlight this feature with the following experiment using the
classical MNIST ten class classification problem~\cite{lecun1998mnist}. We sample a
sparse set of $N$ points $(x,y)\in [0,1]\times [0,1]$
uniformly at random in the input image, use their interpolated values
as signal and the \emph{continuous} $(x,y)$ positions as features. This mimics
sub-sampling of a high-dimensional signal. To compare against a spatial convolution,
we interpolate the sparse set of values at the grid positions.

We take a reference implementation of LeNet~\cite{lecun1998gradient} that
is part of the Caffe project~\cite{jia2014caffe} and compare it
against the same architecture but replacing the first convolutional
layer with a bilateral convolution layer (BCL). The filter size
and numbers are adjusted to get a comparable number of parameters
($5\times 5$ for LeNet, $2$-neighborhood for BCL).

The results are shown in Table~\ref{tab:all-results}. We see that training
on the original MNIST data (column Original, LeNet vs. BNN) leads to a slight
decrease in performance of the BNN (99.03\%) compared to LeNet
(99.19\%). The BNN can be trained and evaluated on sparse
signals, and we resample the image as described above for $N=$ 100\%, 60\% and
20\% of the total number of pixels. The methods are also evaluated
on test images that are subsampled in the same way. Note that we can
train and test with different subsampling rates. We introduce an additional
bilinear interpolation layer for the LeNet architecture to train on the same
data. In essence, both models perform a spatial interpolation and thus we
expect them to yield a similar classification accuracy. Once the data is of
higher dimensions, the permutohedral convolution will be faster due to hashing
the sparse input points, as well as less memory demanding in comparison to
naive application of a spatial convolution with interpolated values.

\begin{table}[t]
  \begin{center}
    \footnotesize
    \centering
    \begin{tabular}[t]{lllll}
      \toprule
              &     & \multicolumn{3}{c}{Test Subsampling} \\
       Method  & Original & 100\% & 60\% & 20\%\\
      \midrule
       LeNet &  \textbf{0.9919} & 0.9660 & 0.9348 & \textbf{0.6434} \\
       BNN &  0.9903 & \textbf{0.9844} & \textbf{0.9534} & 0.5767 \\
      \hline
       LeNet 100\% & 0.9856 & 0.9809 & 0.9678 & \textbf{0.7386} \\
       BNN 100\% & \textbf{0.9900} & \textbf{0.9863} & \textbf{0.9699} & 0.6910 \\
      \hline
       LeNet 60\% & 0.9848 & 0.9821 & 0.9740 & 0.8151 \\
       BNN 60\% & \textbf{0.9885} & \textbf{0.9864} & \textbf{0.9771} & \textbf{0.8214}\\
      \hline
       LeNet 20\% & \textbf{0.9763} & \textbf{0.9754} & 0.9695 & 0.8928 \\
       BNN 20\% & 0.9728 & 0.9735 & \textbf{0.9701} & \textbf{0.9042}\\
      \bottomrule
    \end{tabular}
  \end{center}
\vspace{-.2cm}
\caption{Classification accuracy on MNIST. We compare the
    LeNet~\cite{lecun1998gradient} implementation that is part of
    Caffe~\cite{jia2014caffe} to the network with the first layer
    replaced by a bilateral convolution layer (BCL). Both are trained
    on the original image resolution (first two rows). Three more BNN
    and CNN models are trained with randomly subsampled images (100\%,
    60\% and 20\% of the pixels). An additional bilinear interpolation
    layer samples the input signal on a spatial grid for the CNN model.
  }
  \label{tab:all-results}
\vspace{-.5cm}
\end{table}

\subsubsection{Image Denoising}

The main application that inspired the development of the bilateral
filtering operation is image denoising~\cite{aurich1995non}, there
using a single Gaussian kernel. Our development allows to learn this
kernel function from data and we explore how to improve using a \emph{single}
but more general bilateral filter.

We use the Berkeley segmentation dataset
(BSDS500)~\cite{arbelaezi2011bsds500} as a test bed. The color
images in the dataset are converted to gray-scale,
and corrupted with Gaussian noise with a standard deviation of
$\frac {25} {255}$.

We compare the performance of four different filter models on a
denoising task.
The first baseline model (`Spatial' in Table \ref{tab:denoising}, $25$
weights) uses a single spatial filter with a kernel size of
$5$ and predicts the scalar gray-scale value at the center pixel. The next model
(`Gauss Bilateral') applies a bilateral \emph{Gaussian}
filter to the noisy input, using position and intensity features $\f=(x,y,v)^\top$.
The third setup (`Learned Bilateral', $65$ weights)
takes a Gauss kernel as initialization and
fits all filter weights on the train set to minimize the
mean squared error with respect to the clean images.
We run a combination
of spatial and permutohedral convolutions on spatial and bilateral
features (`Spatial + Bilateral (Learned)') to check for a complementary
performance of the two convolutions.

\label{sec:exp:denoising}
\begin{table}[!h]
\begin{center}
  \footnotesize
  \begin{tabular}[t]{lr}
    \toprule
    Method & PSNR \\
    \midrule
    Noisy Input & $20.17$ \\
    Spatial & $26.27$ \\
    Gauss Bilateral & $26.51$ \\
    Learned Bilateral & $26.58$ \\
    Spatial + Bilateral (Learned) & \textbf{$26.65$} \\
    \bottomrule
  \end{tabular}
\end{center}
\vspace{-0.5em}
\caption{PSNR results of a denoising task using the BSDS500
  dataset~\cite{arbelaezi2011bsds500}}
\vspace{-0.5em}
\label{tab:denoising}
\end{table}
\vspace{-0.2em}

The PSNR scores evaluated on full images of the test set are
shown in Table \ref{tab:denoising}. We find that an untrained bilateral
filter already performs better than a trained spatial convolution
($26.27$ to $26.51$). A learned convolution further improve the
performance slightly. We chose this simple one-kernel setup to
validate an advantage of the generalized bilateral filter. A competitive
denoising system would employ RGB color information and also
needs to be properly adjusted in network size. Multi-layer perceptrons
have obtained state-of-the-art denoising results~\cite{burger12cvpr}
and the permutohedral lattice layer can readily be used in such an
architecture, which is intended future work.

\subsection{Additional results}
\label{sec:addresults}

This section contains more qualitative results for the experiments presented in Chapter~\ref{chap:bnn}.

\begin{figure*}[th!]
  \centering
    \includegraphics[width=\columnwidth,trim={5cm 2.5cm 5cm 4.5cm},clip]{figures/supplementary/lattice_viz.pdf}
    \vspace{-0.7cm}
  \mycaption{Visualization of the Permutohedral Lattice}
  {Sample lattice visualizations for different feature spaces. All pixels falling in the same simplex cell are shown with
  the same color. $(x,y)$ features correspond to image pixel positions, and $(r,g,b) \in [0,255]$ correspond
  to the red, green and blue color values.}
\label{fig:latticeviz}
\end{figure*}

\subsubsection{Lattice Visualization}

Figure~\ref{fig:latticeviz} shows sample lattice visualizations for different feature spaces.

\newcolumntype{L}[1]{>{\raggedright\let\newline\\\arraybackslash\hspace{0pt}}b{#1}}
\newcolumntype{C}[1]{>{\centering\let\newline\\\arraybackslash\hspace{0pt}}b{#1}}
\newcolumntype{R}[1]{>{\raggedleft\let\newline\\\arraybackslash\hspace{0pt}}b{#1}}

\subsubsection{Color Upsampling}\label{sec:color_upsampling}
\label{sec:col_upsample_extra}

Some images of the upsampling for the Pascal
VOC12 dataset are shown in Fig.~\ref{fig:Colour_upsample_visuals}. It is
especially the low level image details that are better preserved with
a learned bilateral filter compared to the Gaussian case.

\begin{figure*}[t!]
  \centering
    \subfigure{%
   \raisebox{2.0em}{
    \includegraphics[width=.06\columnwidth]{figures/supplementary/2007_004969.jpg}
   }
  }
  \subfigure{%
    \includegraphics[width=.17\columnwidth]{figures/supplementary/2007_004969_gray.pdf}
  }
  \subfigure{%
    \includegraphics[width=.17\columnwidth]{figures/supplementary/2007_004969_gt.pdf}
  }
  \subfigure{%
    \includegraphics[width=.17\columnwidth]{figures/supplementary/2007_004969_bicubic.pdf}
  }
  \subfigure{%
    \includegraphics[width=.17\columnwidth]{figures/supplementary/2007_004969_gauss.pdf}
  }
  \subfigure{%
    \includegraphics[width=.17\columnwidth]{figures/supplementary/2007_004969_learnt.pdf}
  }\\
    \subfigure{%
   \raisebox{2.0em}{
    \includegraphics[width=.06\columnwidth]{figures/supplementary/2007_003106.jpg}
   }
  }
  \subfigure{%
    \includegraphics[width=.17\columnwidth]{figures/supplementary/2007_003106_gray.pdf}
  }
  \subfigure{%
    \includegraphics[width=.17\columnwidth]{figures/supplementary/2007_003106_gt.pdf}
  }
  \subfigure{%
    \includegraphics[width=.17\columnwidth]{figures/supplementary/2007_003106_bicubic.pdf}
  }
  \subfigure{%
    \includegraphics[width=.17\columnwidth]{figures/supplementary/2007_003106_gauss.pdf}
  }
  \subfigure{%
    \includegraphics[width=.17\columnwidth]{figures/supplementary/2007_003106_learnt.pdf}
  }\\
  \setcounter{subfigure}{0}
  \small{
  \subfigure[Inp.]{%
  \raisebox{2.0em}{
    \includegraphics[width=.06\columnwidth]{figures/supplementary/2007_006837.jpg}
   }
  }
  \subfigure[Guidance]{%
    \includegraphics[width=.17\columnwidth]{figures/supplementary/2007_006837_gray.pdf}
  }
   \subfigure[GT]{%
    \includegraphics[width=.17\columnwidth]{figures/supplementary/2007_006837_gt.pdf}
  }
  \subfigure[Bicubic]{%
    \includegraphics[width=.17\columnwidth]{figures/supplementary/2007_006837_bicubic.pdf}
  }
  \subfigure[Gauss-BF]{%
    \includegraphics[width=.17\columnwidth]{figures/supplementary/2007_006837_gauss.pdf}
  }
  \subfigure[Learned-BF]{%
    \includegraphics[width=.17\columnwidth]{figures/supplementary/2007_006837_learnt.pdf}
  }
  }
  \vspace{-0.5cm}
  \mycaption{Color Upsampling}{Color $8\times$ upsampling results
  using different methods, from left to right, (a)~Low-resolution input color image (Inp.),
  (b)~Gray scale guidance image, (c)~Ground-truth color image; Upsampled color images with
  (d)~Bicubic interpolation, (e) Gauss bilateral upsampling and, (f)~Learned bilateral
  updampgling (best viewed on screen).}

\label{fig:Colour_upsample_visuals}
\end{figure*}

\subsubsection{Depth Upsampling}
\label{sec:depth_upsample_extra}

Figure~\ref{fig:depth_upsample_visuals} presents some more qualitative results comparing bicubic interpolation, Gauss
bilateral and learned bilateral upsampling on NYU depth dataset image~\cite{silberman2012indoor}.

\subsubsection{Character Recognition}\label{sec:app_character}

 Figure~\ref{fig:nnrecognition} shows the schematic of different layers
 of the network architecture for LeNet-7~\cite{lecun1998mnist}
 and DeepCNet(5, 50)~\cite{ciresan2012multi,graham2014spatially}. For the BNN variants, the first layer filters are replaced
 with learned bilateral filters and are learned end-to-end.

\subsubsection{Semantic Segmentation}\label{sec:app_semantic_segmentation}
\label{sec:semantic_bnn_extra}

Some more visual results for semantic segmentation are shown in Figure~\ref{fig:semantic_visuals}.
These include the underlying DeepLab CNN\cite{chen2014semantic} result (DeepLab),
the 2 step mean-field result with Gaussian edge potentials (+2stepMF-GaussCRF)
and also corresponding results with learned edge potentials (+2stepMF-LearnedCRF).
In general, we observe that mean-field in learned CRF leads to slightly dilated
classification regions in comparison to using Gaussian CRF thereby filling-in the
false negative pixels and also correcting some mis-classified regions.

\begin{figure*}[t!]
  \centering
    \subfigure{%
   \raisebox{2.0em}{
    \includegraphics[width=.06\columnwidth]{figures/supplementary/2bicubic}
   }
  }
  \subfigure{%
    \includegraphics[width=.17\columnwidth]{figures/supplementary/2given_image}
  }
  \subfigure{%
    \includegraphics[width=.17\columnwidth]{figures/supplementary/2ground_truth}
  }
  \subfigure{%
    \includegraphics[width=.17\columnwidth]{figures/supplementary/2bicubic}
  }
  \subfigure{%
    \includegraphics[width=.17\columnwidth]{figures/supplementary/2gauss}
  }
  \subfigure{%
    \includegraphics[width=.17\columnwidth]{figures/supplementary/2learnt}
  }\\
    \subfigure{%
   \raisebox{2.0em}{
    \includegraphics[width=.06\columnwidth]{figures/supplementary/32bicubic}
   }
  }
  \subfigure{%
    \includegraphics[width=.17\columnwidth]{figures/supplementary/32given_image}
  }
  \subfigure{%
    \includegraphics[width=.17\columnwidth]{figures/supplementary/32ground_truth}
  }
  \subfigure{%
    \includegraphics[width=.17\columnwidth]{figures/supplementary/32bicubic}
  }
  \subfigure{%
    \includegraphics[width=.17\columnwidth]{figures/supplementary/32gauss}
  }
  \subfigure{%
    \includegraphics[width=.17\columnwidth]{figures/supplementary/32learnt}
  }\\
  \setcounter{subfigure}{0}
  \small{
  \subfigure[Inp.]{%
  \raisebox{2.0em}{
    \includegraphics[width=.06\columnwidth]{figures/supplementary/41bicubic}
   }
  }
  \subfigure[Guidance]{%
    \includegraphics[width=.17\columnwidth]{figures/supplementary/41given_image}
  }
   \subfigure[GT]{%
    \includegraphics[width=.17\columnwidth]{figures/supplementary/41ground_truth}
  }
  \subfigure[Bicubic]{%
    \includegraphics[width=.17\columnwidth]{figures/supplementary/41bicubic}
  }
  \subfigure[Gauss-BF]{%
    \includegraphics[width=.17\columnwidth]{figures/supplementary/41gauss}
  }
  \subfigure[Learned-BF]{%
    \includegraphics[width=.17\columnwidth]{figures/supplementary/41learnt}
  }
  }
  \mycaption{Depth Upsampling}{Depth $8\times$ upsampling results
  using different upsampling strategies, from left to right,
  (a)~Low-resolution input depth image (Inp.),
  (b)~High-resolution guidance image, (c)~Ground-truth depth; Upsampled depth images with
  (d)~Bicubic interpolation, (e) Gauss bilateral upsampling and, (f)~Learned bilateral
  updampgling (best viewed on screen).}

\label{fig:depth_upsample_visuals}
\end{figure*}

\subsubsection{Material Segmentation}\label{sec:app_material_segmentation}
\label{sec:material_bnn_extra}

In Fig.~\ref{fig:material_visuals-app2}, we present visual results comparing 2 step
mean-field inference with Gaussian and learned pairwise CRF potentials. In
general, we observe that the pixels belonging to dominant classes in the
training data are being more accurately classified with learned CRF. This leads to
a significant improvements in overall pixel accuracy. This also results
in a slight decrease of the accuracy from less frequent class pixels thereby
slightly reducing the average class accuracy with learning. We attribute this
to the type of annotation that is available for this dataset, which is not
for the entire image but for some segments in the image. We have very few
images of the infrequent classes to combat this behaviour during training.

\subsubsection{Experiment Protocols}
\label{sec:protocols}

Table~\ref{tbl:parameters} shows experiment protocols of different experiments.

 \begin{figure*}[t!]
  \centering
  \subfigure[LeNet-7]{
    \includegraphics[width=0.7\columnwidth]{figures/supplementary/lenet_cnn_network}
    }\\
    \subfigure[DeepCNet]{
    \includegraphics[width=\columnwidth]{figures/supplementary/deepcnet_cnn_network}
    }
  \mycaption{CNNs for Character Recognition}
  {Schematic of (top) LeNet-7~\cite{lecun1998mnist} and (bottom) DeepCNet(5,50)~\cite{ciresan2012multi,graham2014spatially} architectures used in Assamese
  character recognition experiments.}
\label{fig:nnrecognition}
\end{figure*}

\definecolor{voc_1}{RGB}{0, 0, 0}
\definecolor{voc_2}{RGB}{128, 0, 0}
\definecolor{voc_3}{RGB}{0, 128, 0}
\definecolor{voc_4}{RGB}{128, 128, 0}
\definecolor{voc_5}{RGB}{0, 0, 128}
\definecolor{voc_6}{RGB}{128, 0, 128}
\definecolor{voc_7}{RGB}{0, 128, 128}
\definecolor{voc_8}{RGB}{128, 128, 128}
\definecolor{voc_9}{RGB}{64, 0, 0}
\definecolor{voc_10}{RGB}{192, 0, 0}
\definecolor{voc_11}{RGB}{64, 128, 0}
\definecolor{voc_12}{RGB}{192, 128, 0}
\definecolor{voc_13}{RGB}{64, 0, 128}
\definecolor{voc_14}{RGB}{192, 0, 128}
\definecolor{voc_15}{RGB}{64, 128, 128}
\definecolor{voc_16}{RGB}{192, 128, 128}
\definecolor{voc_17}{RGB}{0, 64, 0}
\definecolor{voc_18}{RGB}{128, 64, 0}
\definecolor{voc_19}{RGB}{0, 192, 0}
\definecolor{voc_20}{RGB}{128, 192, 0}
\definecolor{voc_21}{RGB}{0, 64, 128}
\definecolor{voc_22}{RGB}{128, 64, 128}

\begin{figure*}[t]
  \centering
  \small{
  \fcolorbox{white}{voc_1}{\rule{0pt}{6pt}\rule{6pt}{0pt}} Background~~
  \fcolorbox{white}{voc_2}{\rule{0pt}{6pt}\rule{6pt}{0pt}} Aeroplane~~
  \fcolorbox{white}{voc_3}{\rule{0pt}{6pt}\rule{6pt}{0pt}} Bicycle~~
  \fcolorbox{white}{voc_4}{\rule{0pt}{6pt}\rule{6pt}{0pt}} Bird~~
  \fcolorbox{white}{voc_5}{\rule{0pt}{6pt}\rule{6pt}{0pt}} Boat~~
  \fcolorbox{white}{voc_6}{\rule{0pt}{6pt}\rule{6pt}{0pt}} Bottle~~
  \fcolorbox{white}{voc_7}{\rule{0pt}{6pt}\rule{6pt}{0pt}} Bus~~
  \fcolorbox{white}{voc_8}{\rule{0pt}{6pt}\rule{6pt}{0pt}} Car~~ \\
  \fcolorbox{white}{voc_9}{\rule{0pt}{6pt}\rule{6pt}{0pt}} Cat~~
  \fcolorbox{white}{voc_10}{\rule{0pt}{6pt}\rule{6pt}{0pt}} Chair~~
  \fcolorbox{white}{voc_11}{\rule{0pt}{6pt}\rule{6pt}{0pt}} Cow~~
  \fcolorbox{white}{voc_12}{\rule{0pt}{6pt}\rule{6pt}{0pt}} Dining Table~~
  \fcolorbox{white}{voc_13}{\rule{0pt}{6pt}\rule{6pt}{0pt}} Dog~~
  \fcolorbox{white}{voc_14}{\rule{0pt}{6pt}\rule{6pt}{0pt}} Horse~~
  \fcolorbox{white}{voc_15}{\rule{0pt}{6pt}\rule{6pt}{0pt}} Motorbike~~
  \fcolorbox{white}{voc_16}{\rule{0pt}{6pt}\rule{6pt}{0pt}} Person~~ \\
  \fcolorbox{white}{voc_17}{\rule{0pt}{6pt}\rule{6pt}{0pt}} Potted Plant~~
  \fcolorbox{white}{voc_18}{\rule{0pt}{6pt}\rule{6pt}{0pt}} Sheep~~
  \fcolorbox{white}{voc_19}{\rule{0pt}{6pt}\rule{6pt}{0pt}} Sofa~~
  \fcolorbox{white}{voc_20}{\rule{0pt}{6pt}\rule{6pt}{0pt}} Train~~
  \fcolorbox{white}{voc_21}{\rule{0pt}{6pt}\rule{6pt}{0pt}} TV monitor~~ \\
  }
  \subfigure{%
    \includegraphics[width=.18\columnwidth]{figures/supplementary/2007_001423_given.jpg}
  }
  \subfigure{%
    \includegraphics[width=.18\columnwidth]{figures/supplementary/2007_001423_gt.png}
  }
  \subfigure{%
    \includegraphics[width=.18\columnwidth]{figures/supplementary/2007_001423_cnn.png}
  }
  \subfigure{%
    \includegraphics[width=.18\columnwidth]{figures/supplementary/2007_001423_gauss.png}
  }
  \subfigure{%
    \includegraphics[width=.18\columnwidth]{figures/supplementary/2007_001423_learnt.png}
  }\\
  \subfigure{%
    \includegraphics[width=.18\columnwidth]{figures/supplementary/2007_001430_given.jpg}
  }
  \subfigure{%
    \includegraphics[width=.18\columnwidth]{figures/supplementary/2007_001430_gt.png}
  }
  \subfigure{%
    \includegraphics[width=.18\columnwidth]{figures/supplementary/2007_001430_cnn.png}
  }
  \subfigure{%
    \includegraphics[width=.18\columnwidth]{figures/supplementary/2007_001430_gauss.png}
  }
  \subfigure{%
    \includegraphics[width=.18\columnwidth]{figures/supplementary/2007_001430_learnt.png}
  }\\
    \subfigure{%
    \includegraphics[width=.18\columnwidth]{figures/supplementary/2007_007996_given.jpg}
  }
  \subfigure{%
    \includegraphics[width=.18\columnwidth]{figures/supplementary/2007_007996_gt.png}
  }
  \subfigure{%
    \includegraphics[width=.18\columnwidth]{figures/supplementary/2007_007996_cnn.png}
  }
  \subfigure{%
    \includegraphics[width=.18\columnwidth]{figures/supplementary/2007_007996_gauss.png}
  }
  \subfigure{%
    \includegraphics[width=.18\columnwidth]{figures/supplementary/2007_007996_learnt.png}
  }\\
   \subfigure{%
    \includegraphics[width=.18\columnwidth]{figures/supplementary/2010_002682_given.jpg}
  }
  \subfigure{%
    \includegraphics[width=.18\columnwidth]{figures/supplementary/2010_002682_gt.png}
  }
  \subfigure{%
    \includegraphics[width=.18\columnwidth]{figures/supplementary/2010_002682_cnn.png}
  }
  \subfigure{%
    \includegraphics[width=.18\columnwidth]{figures/supplementary/2010_002682_gauss.png}
  }
  \subfigure{%
    \includegraphics[width=.18\columnwidth]{figures/supplementary/2010_002682_learnt.png}
  }\\
     \subfigure{%
    \includegraphics[width=.18\columnwidth]{figures/supplementary/2010_004789_given.jpg}
  }
  \subfigure{%
    \includegraphics[width=.18\columnwidth]{figures/supplementary/2010_004789_gt.png}
  }
  \subfigure{%
    \includegraphics[width=.18\columnwidth]{figures/supplementary/2010_004789_cnn.png}
  }
  \subfigure{%
    \includegraphics[width=.18\columnwidth]{figures/supplementary/2010_004789_gauss.png}
  }
  \subfigure{%
    \includegraphics[width=.18\columnwidth]{figures/supplementary/2010_004789_learnt.png}
  }\\
       \subfigure{%
    \includegraphics[width=.18\columnwidth]{figures/supplementary/2007_001311_given.jpg}
  }
  \subfigure{%
    \includegraphics[width=.18\columnwidth]{figures/supplementary/2007_001311_gt.png}
  }
  \subfigure{%
    \includegraphics[width=.18\columnwidth]{figures/supplementary/2007_001311_cnn.png}
  }
  \subfigure{%
    \includegraphics[width=.18\columnwidth]{figures/supplementary/2007_001311_gauss.png}
  }
  \subfigure{%
    \includegraphics[width=.18\columnwidth]{figures/supplementary/2007_001311_learnt.png}
  }\\
  \setcounter{subfigure}{0}
  \subfigure[Input]{%
    \includegraphics[width=.18\columnwidth]{figures/supplementary/2010_003531_given.jpg}
  }
  \subfigure[Ground Truth]{%
    \includegraphics[width=.18\columnwidth]{figures/supplementary/2010_003531_gt.png}
  }
  \subfigure[DeepLab]{%
    \includegraphics[width=.18\columnwidth]{figures/supplementary/2010_003531_cnn.png}
  }
  \subfigure[+GaussCRF]{%
    \includegraphics[width=.18\columnwidth]{figures/supplementary/2010_003531_gauss.png}
  }
  \subfigure[+LearnedCRF]{%
    \includegraphics[width=.18\columnwidth]{figures/supplementary/2010_003531_learnt.png}
  }
  \vspace{-0.3cm}
  \mycaption{Semantic Segmentation}{Example results of semantic segmentation.
  (c)~depicts the unary results before application of MF, (d)~after two steps of MF with Gaussian edge CRF potentials, (e)~after
  two steps of MF with learned edge CRF potentials.}
    \label{fig:semantic_visuals}
\end{figure*}


\definecolor{minc_1}{HTML}{771111}
\definecolor{minc_2}{HTML}{CAC690}
\definecolor{minc_3}{HTML}{EEEEEE}
\definecolor{minc_4}{HTML}{7C8FA6}
\definecolor{minc_5}{HTML}{597D31}
\definecolor{minc_6}{HTML}{104410}
\definecolor{minc_7}{HTML}{BB819C}
\definecolor{minc_8}{HTML}{D0CE48}
\definecolor{minc_9}{HTML}{622745}
\definecolor{minc_10}{HTML}{666666}
\definecolor{minc_11}{HTML}{D54A31}
\definecolor{minc_12}{HTML}{101044}
\definecolor{minc_13}{HTML}{444126}
\definecolor{minc_14}{HTML}{75D646}
\definecolor{minc_15}{HTML}{DD4348}
\definecolor{minc_16}{HTML}{5C8577}
\definecolor{minc_17}{HTML}{C78472}
\definecolor{minc_18}{HTML}{75D6D0}
\definecolor{minc_19}{HTML}{5B4586}
\definecolor{minc_20}{HTML}{C04393}
\definecolor{minc_21}{HTML}{D69948}
\definecolor{minc_22}{HTML}{7370D8}
\definecolor{minc_23}{HTML}{7A3622}
\definecolor{minc_24}{HTML}{000000}

\begin{figure*}[t]
  \centering
  \small{
  \fcolorbox{white}{minc_1}{\rule{0pt}{6pt}\rule{6pt}{0pt}} Brick~~
  \fcolorbox{white}{minc_2}{\rule{0pt}{6pt}\rule{6pt}{0pt}} Carpet~~
  \fcolorbox{white}{minc_3}{\rule{0pt}{6pt}\rule{6pt}{0pt}} Ceramic~~
  \fcolorbox{white}{minc_4}{\rule{0pt}{6pt}\rule{6pt}{0pt}} Fabric~~
  \fcolorbox{white}{minc_5}{\rule{0pt}{6pt}\rule{6pt}{0pt}} Foliage~~
  \fcolorbox{white}{minc_6}{\rule{0pt}{6pt}\rule{6pt}{0pt}} Food~~
  \fcolorbox{white}{minc_7}{\rule{0pt}{6pt}\rule{6pt}{0pt}} Glass~~
  \fcolorbox{white}{minc_8}{\rule{0pt}{6pt}\rule{6pt}{0pt}} Hair~~ \\
  \fcolorbox{white}{minc_9}{\rule{0pt}{6pt}\rule{6pt}{0pt}} Leather~~
  \fcolorbox{white}{minc_10}{\rule{0pt}{6pt}\rule{6pt}{0pt}} Metal~~
  \fcolorbox{white}{minc_11}{\rule{0pt}{6pt}\rule{6pt}{0pt}} Mirror~~
  \fcolorbox{white}{minc_12}{\rule{0pt}{6pt}\rule{6pt}{0pt}} Other~~
  \fcolorbox{white}{minc_13}{\rule{0pt}{6pt}\rule{6pt}{0pt}} Painted~~
  \fcolorbox{white}{minc_14}{\rule{0pt}{6pt}\rule{6pt}{0pt}} Paper~~
  \fcolorbox{white}{minc_15}{\rule{0pt}{6pt}\rule{6pt}{0pt}} Plastic~~\\
  \fcolorbox{white}{minc_16}{\rule{0pt}{6pt}\rule{6pt}{0pt}} Polished Stone~~
  \fcolorbox{white}{minc_17}{\rule{0pt}{6pt}\rule{6pt}{0pt}} Skin~~
  \fcolorbox{white}{minc_18}{\rule{0pt}{6pt}\rule{6pt}{0pt}} Sky~~
  \fcolorbox{white}{minc_19}{\rule{0pt}{6pt}\rule{6pt}{0pt}} Stone~~
  \fcolorbox{white}{minc_20}{\rule{0pt}{6pt}\rule{6pt}{0pt}} Tile~~
  \fcolorbox{white}{minc_21}{\rule{0pt}{6pt}\rule{6pt}{0pt}} Wallpaper~~
  \fcolorbox{white}{minc_22}{\rule{0pt}{6pt}\rule{6pt}{0pt}} Water~~
  \fcolorbox{white}{minc_23}{\rule{0pt}{6pt}\rule{6pt}{0pt}} Wood~~ \\
  }
  \subfigure{%
    \includegraphics[width=.18\columnwidth]{figures/supplementary/000010868_given.jpg}
  }
  \subfigure{%
    \includegraphics[width=.18\columnwidth]{figures/supplementary/000010868_gt.png}
  }
  \subfigure{%
    \includegraphics[width=.18\columnwidth]{figures/supplementary/000010868_cnn.png}
  }
  \subfigure{%
    \includegraphics[width=.18\columnwidth]{figures/supplementary/000010868_gauss.png}
  }
  \subfigure{%
    \includegraphics[width=.18\columnwidth]{figures/supplementary/000010868_learnt.png}
  }\\[-2ex]
  \subfigure{%
    \includegraphics[width=.18\columnwidth]{figures/supplementary/000006011_given.jpg}
  }
  \subfigure{%
    \includegraphics[width=.18\columnwidth]{figures/supplementary/000006011_gt.png}
  }
  \subfigure{%
    \includegraphics[width=.18\columnwidth]{figures/supplementary/000006011_cnn.png}
  }
  \subfigure{%
    \includegraphics[width=.18\columnwidth]{figures/supplementary/000006011_gauss.png}
  }
  \subfigure{%
    \includegraphics[width=.18\columnwidth]{figures/supplementary/000006011_learnt.png}
  }\\[-2ex]
    \subfigure{%
    \includegraphics[width=.18\columnwidth]{figures/supplementary/000008553_given.jpg}
  }
  \subfigure{%
    \includegraphics[width=.18\columnwidth]{figures/supplementary/000008553_gt.png}
  }
  \subfigure{%
    \includegraphics[width=.18\columnwidth]{figures/supplementary/000008553_cnn.png}
  }
  \subfigure{%
    \includegraphics[width=.18\columnwidth]{figures/supplementary/000008553_gauss.png}
  }
  \subfigure{%
    \includegraphics[width=.18\columnwidth]{figures/supplementary/000008553_learnt.png}
  }\\[-2ex]
   \subfigure{%
    \includegraphics[width=.18\columnwidth]{figures/supplementary/000009188_given.jpg}
  }
  \subfigure{%
    \includegraphics[width=.18\columnwidth]{figures/supplementary/000009188_gt.png}
  }
  \subfigure{%
    \includegraphics[width=.18\columnwidth]{figures/supplementary/000009188_cnn.png}
  }
  \subfigure{%
    \includegraphics[width=.18\columnwidth]{figures/supplementary/000009188_gauss.png}
  }
  \subfigure{%
    \includegraphics[width=.18\columnwidth]{figures/supplementary/000009188_learnt.png}
  }\\[-2ex]
  \setcounter{subfigure}{0}
  \subfigure[Input]{%
    \includegraphics[width=.18\columnwidth]{figures/supplementary/000023570_given.jpg}
  }
  \subfigure[Ground Truth]{%
    \includegraphics[width=.18\columnwidth]{figures/supplementary/000023570_gt.png}
  }
  \subfigure[DeepLab]{%
    \includegraphics[width=.18\columnwidth]{figures/supplementary/000023570_cnn.png}
  }
  \subfigure[+GaussCRF]{%
    \includegraphics[width=.18\columnwidth]{figures/supplementary/000023570_gauss.png}
  }
  \subfigure[+LearnedCRF]{%
    \includegraphics[width=.18\columnwidth]{figures/supplementary/000023570_learnt.png}
  }
  \mycaption{Material Segmentation}{Example results of material segmentation.
  (c)~depicts the unary results before application of MF, (d)~after two steps of MF with Gaussian edge CRF potentials, (e)~after two steps of MF with learned edge CRF potentials.}
    \label{fig:material_visuals-app2}
\end{figure*}


\begin{table*}[h]
\tiny
  \centering
    \begin{tabular}{L{2.3cm} L{2.25cm} C{1.5cm} C{0.7cm} C{0.6cm} C{0.7cm} C{0.7cm} C{0.7cm} C{1.6cm} C{0.6cm} C{0.6cm} C{0.6cm}}
      \toprule
& & & & & \multicolumn{3}{c}{\textbf{Data Statistics}} & \multicolumn{4}{c}{\textbf{Training Protocol}} \\

\textbf{Experiment} & \textbf{Feature Types} & \textbf{Feature Scales} & \textbf{Filter Size} & \textbf{Filter Nbr.} & \textbf{Train}  & \textbf{Val.} & \textbf{Test} & \textbf{Loss Type} & \textbf{LR} & \textbf{Batch} & \textbf{Epochs} \\
      \midrule
      \multicolumn{2}{c}{\textbf{Single Bilateral Filter Applications}} & & & & & & & & & \\
      \textbf{2$\times$ Color Upsampling} & Position$_{1}$, Intensity (3D) & 0.13, 0.17 & 65 & 2 & 10581 & 1449 & 1456 & MSE & 1e-06 & 200 & 94.5\\
      \textbf{4$\times$ Color Upsampling} & Position$_{1}$, Intensity (3D) & 0.06, 0.17 & 65 & 2 & 10581 & 1449 & 1456 & MSE & 1e-06 & 200 & 94.5\\
      \textbf{8$\times$ Color Upsampling} & Position$_{1}$, Intensity (3D) & 0.03, 0.17 & 65 & 2 & 10581 & 1449 & 1456 & MSE & 1e-06 & 200 & 94.5\\
      \textbf{16$\times$ Color Upsampling} & Position$_{1}$, Intensity (3D) & 0.02, 0.17 & 65 & 2 & 10581 & 1449 & 1456 & MSE & 1e-06 & 200 & 94.5\\
      \textbf{Depth Upsampling} & Position$_{1}$, Color (5D) & 0.05, 0.02 & 665 & 2 & 795 & 100 & 654 & MSE & 1e-07 & 50 & 251.6\\
      \textbf{Mesh Denoising} & Isomap (4D) & 46.00 & 63 & 2 & 1000 & 200 & 500 & MSE & 100 & 10 & 100.0 \\
      \midrule
      \multicolumn{2}{c}{\textbf{DenseCRF Applications}} & & & & & & & & &\\
      \multicolumn{2}{l}{\textbf{Semantic Segmentation}} & & & & & & & & &\\
      \textbf{- 1step MF} & Position$_{1}$, Color (5D); Position$_{1}$ (2D) & 0.01, 0.34; 0.34  & 665; 19  & 2; 2 & 10581 & 1449 & 1456 & Logistic & 0.1 & 5 & 1.4 \\
      \textbf{- 2step MF} & Position$_{1}$, Color (5D); Position$_{1}$ (2D) & 0.01, 0.34; 0.34 & 665; 19 & 2; 2 & 10581 & 1449 & 1456 & Logistic & 0.1 & 5 & 1.4 \\
      \textbf{- \textit{loose} 2step MF} & Position$_{1}$, Color (5D); Position$_{1}$ (2D) & 0.01, 0.34; 0.34 & 665; 19 & 2; 2 &10581 & 1449 & 1456 & Logistic & 0.1 & 5 & +1.9  \\ \\
      \multicolumn{2}{l}{\textbf{Material Segmentation}} & & & & & & & & &\\
      \textbf{- 1step MF} & Position$_{2}$, Lab-Color (5D) & 5.00, 0.05, 0.30  & 665 & 2 & 928 & 150 & 1798 & Weighted Logistic & 1e-04 & 24 & 2.6 \\
      \textbf{- 2step MF} & Position$_{2}$, Lab-Color (5D) & 5.00, 0.05, 0.30 & 665 & 2 & 928 & 150 & 1798 & Weighted Logistic & 1e-04 & 12 & +0.7 \\
      \textbf{- \textit{loose} 2step MF} & Position$_{2}$, Lab-Color (5D) & 5.00, 0.05, 0.30 & 665 & 2 & 928 & 150 & 1798 & Weighted Logistic & 1e-04 & 12 & +0.2\\
      \midrule
      \multicolumn{2}{c}{\textbf{Neural Network Applications}} & & & & & & & & &\\
      \textbf{Tiles: CNN-9$\times$9} & - & - & 81 & 4 & 10000 & 1000 & 1000 & Logistic & 0.01 & 100 & 500.0 \\
      \textbf{Tiles: CNN-13$\times$13} & - & - & 169 & 6 & 10000 & 1000 & 1000 & Logistic & 0.01 & 100 & 500.0 \\
      \textbf{Tiles: CNN-17$\times$17} & - & - & 289 & 8 & 10000 & 1000 & 1000 & Logistic & 0.01 & 100 & 500.0 \\
      \textbf{Tiles: CNN-21$\times$21} & - & - & 441 & 10 & 10000 & 1000 & 1000 & Logistic & 0.01 & 100 & 500.0 \\
      \textbf{Tiles: BNN} & Position$_{1}$, Color (5D) & 0.05, 0.04 & 63 & 1 & 10000 & 1000 & 1000 & Logistic & 0.01 & 100 & 30.0 \\
      \textbf{LeNet} & - & - & 25 & 2 & 5490 & 1098 & 1647 & Logistic & 0.1 & 100 & 182.2 \\
      \textbf{Crop-LeNet} & - & - & 25 & 2 & 5490 & 1098 & 1647 & Logistic & 0.1 & 100 & 182.2 \\
      \textbf{BNN-LeNet} & Position$_{2}$ (2D) & 20.00 & 7 & 1 & 5490 & 1098 & 1647 & Logistic & 0.1 & 100 & 182.2 \\
      \textbf{DeepCNet} & - & - & 9 & 1 & 5490 & 1098 & 1647 & Logistic & 0.1 & 100 & 182.2 \\
      \textbf{Crop-DeepCNet} & - & - & 9 & 1 & 5490 & 1098 & 1647 & Logistic & 0.1 & 100 & 182.2 \\
      \textbf{BNN-DeepCNet} & Position$_{2}$ (2D) & 40.00  & 7 & 1 & 5490 & 1098 & 1647 & Logistic & 0.1 & 100 & 182.2 \\
      \bottomrule
      \\
    \end{tabular}
    \mycaption{Experiment Protocols} {Experiment protocols for the different experiments presented in this work. \textbf{Feature Types}:
    Feature spaces used for the bilateral convolutions. Position$_1$ corresponds to un-normalized pixel positions whereas Position$_2$ corresponds
    to pixel positions normalized to $[0,1]$ with respect to the given image. \textbf{Feature Scales}: Cross-validated scales for the features used.
     \textbf{Filter Size}: Number of elements in the filter that is being learned. \textbf{Filter Nbr.}: Half-width of the filter. \textbf{Train},
     \textbf{Val.} and \textbf{Test} corresponds to the number of train, validation and test images used in the experiment. \textbf{Loss Type}: Type
     of loss used for back-propagation. ``MSE'' corresponds to Euclidean mean squared error loss and ``Logistic'' corresponds to multinomial logistic
     loss. ``Weighted Logistic'' is the class-weighted multinomial logistic loss. We weighted the loss with inverse class probability for material
     segmentation task due to the small availability of training data with class imbalance. \textbf{LR}: Fixed learning rate used in stochastic gradient
     descent. \textbf{Batch}: Number of images used in one parameter update step. \textbf{Epochs}: Number of training epochs. In all the experiments,
     we used fixed momentum of 0.9 and weight decay of 0.0005 for stochastic gradient descent. ```Color Upsampling'' experiments in this Table corresponds
     to those performed on Pascal VOC12 dataset images. For all experiments using Pascal VOC12 images, we use extended
     training segmentation dataset available from~\cite{hariharan2011moredata}, and used standard validation and test splits
     from the main dataset~\cite{voc2012segmentation}.}
  \label{tbl:parameters}
\end{table*}

\clearpage

\section{Parameters and Additional Results for Video Propagation Networks}

In this Section, we present experiment protocols and additional qualitative results for experiments
on video object segmentation, semantic video segmentation and video color
propagation. Table~\ref{tbl:parameters_supp} shows the feature scales and other parameters used in different experiments.
Figures~\ref{fig:video_seg_pos_supp} show some qualitative results on video object segmentation
with some failure cases in Fig.~\ref{fig:video_seg_neg_supp}.
Figure~\ref{fig:semantic_visuals_supp} shows some qualitative results on semantic video segmentation and
Fig.~\ref{fig:color_visuals_supp} shows results on video color propagation.

\newcolumntype{L}[1]{>{\raggedright\let\newline\\\arraybackslash\hspace{0pt}}b{#1}}
\newcolumntype{C}[1]{>{\centering\let\newline\\\arraybackslash\hspace{0pt}}b{#1}}
\newcolumntype{R}[1]{>{\raggedleft\let\newline\\\arraybackslash\hspace{0pt}}b{#1}}

\begin{table*}[h]
\tiny
  \centering
    \begin{tabular}{L{3.0cm} L{2.4cm} L{2.8cm} L{2.8cm} C{0.5cm} C{1.0cm} L{1.2cm}}
      \toprule
\textbf{Experiment} & \textbf{Feature Type} & \textbf{Feature Scale-1, $\Lambda_a$} & \textbf{Feature Scale-2, $\Lambda_b$} & \textbf{$\alpha$} & \textbf{Input Frames} & \textbf{Loss Type} \\
      \midrule
      \textbf{Video Object Segmentation} & ($x,y,Y,Cb,Cr,t$) & (0.02,0.02,0.07,0.4,0.4,0.01) & (0.03,0.03,0.09,0.5,0.5,0.2) & 0.5 & 9 & Logistic\\
      \midrule
      \textbf{Semantic Video Segmentation} & & & & & \\
      \textbf{with CNN1~\cite{yu2015multi}-NoFlow} & ($x,y,R,G,B,t$) & (0.08,0.08,0.2,0.2,0.2,0.04) & (0.11,0.11,0.2,0.2,0.2,0.04) & 0.5 & 3 & Logistic \\
      \textbf{with CNN1~\cite{yu2015multi}-Flow} & ($x+u_x,y+u_y,R,G,B,t$) & (0.11,0.11,0.14,0.14,0.14,0.03) & (0.08,0.08,0.12,0.12,0.12,0.01) & 0.65 & 3 & Logistic\\
      \textbf{with CNN2~\cite{richter2016playing}-Flow} & ($x+u_x,y+u_y,R,G,B,t$) & (0.08,0.08,0.2,0.2,0.2,0.04) & (0.09,0.09,0.25,0.25,0.25,0.03) & 0.5 & 4 & Logistic\\
      \midrule
      \textbf{Video Color Propagation} & ($x,y,I,t$)  & (0.04,0.04,0.2,0.04) & No second kernel & 1 & 4 & MSE\\
      \bottomrule
      \\
    \end{tabular}
    \mycaption{Experiment Protocols} {Experiment protocols for the different experiments presented in this work. \textbf{Feature Types}:
    Feature spaces used for the bilateral convolutions, with position ($x,y$) and color
    ($R,G,B$ or $Y,Cb,Cr$) features $\in [0,255]$. $u_x$, $u_y$ denotes optical flow with respect
    to the present frame and $I$ denotes grayscale intensity.
    \textbf{Feature Scales ($\Lambda_a, \Lambda_b$)}: Cross-validated scales for the features used.
    \textbf{$\alpha$}: Exponential time decay for the input frames.
    \textbf{Input Frames}: Number of input frames for VPN.
    \textbf{Loss Type}: Type
     of loss used for back-propagation. ``MSE'' corresponds to Euclidean mean squared error loss and ``Logistic'' corresponds to multinomial logistic loss.}
  \label{tbl:parameters_supp}
\end{table*}

% \begin{figure}[th!]
% \begin{center}
%   \centerline{\includegraphics[width=\textwidth]{figures/video_seg_visuals_supp_small.pdf}}
%     \mycaption{Video Object Segmentation}
%     {Shown are the different frames in example videos with the corresponding
%     ground truth (GT) masks, predictions from BVS~\cite{marki2016bilateral},
%     OFL~\cite{tsaivideo}, VPN (VPN-Stage2) and VPN-DLab (VPN-DeepLab) models.}
%     \label{fig:video_seg_small_supp}
% \end{center}
% \vspace{-1.0cm}
% \end{figure}

\begin{figure}[th!]
\begin{center}
  \centerline{\includegraphics[width=0.7\textwidth]{figures/video_seg_visuals_supp_positive.pdf}}
    \mycaption{Video Object Segmentation}
    {Shown are the different frames in example videos with the corresponding
    ground truth (GT) masks, predictions from BVS~\cite{marki2016bilateral},
    OFL~\cite{tsaivideo}, VPN (VPN-Stage2) and VPN-DLab (VPN-DeepLab) models.}
    \label{fig:video_seg_pos_supp}
\end{center}
\vspace{-1.0cm}
\end{figure}

\begin{figure}[th!]
\begin{center}
  \centerline{\includegraphics[width=0.7\textwidth]{figures/video_seg_visuals_supp_negative.pdf}}
    \mycaption{Failure Cases for Video Object Segmentation}
    {Shown are the different frames in example videos with the corresponding
    ground truth (GT) masks, predictions from BVS~\cite{marki2016bilateral},
    OFL~\cite{tsaivideo}, VPN (VPN-Stage2) and VPN-DLab (VPN-DeepLab) models.}
    \label{fig:video_seg_neg_supp}
\end{center}
\vspace{-1.0cm}
\end{figure}

\begin{figure}[th!]
\begin{center}
  \centerline{\includegraphics[width=0.9\textwidth]{figures/supp_semantic_visual.pdf}}
    \mycaption{Semantic Video Segmentation}
    {Input video frames and the corresponding ground truth (GT)
    segmentation together with the predictions of CNN~\cite{yu2015multi} and with
    VPN-Flow.}
    \label{fig:semantic_visuals_supp}
\end{center}
\vspace{-0.7cm}
\end{figure}

\begin{figure}[th!]
\begin{center}
  \centerline{\includegraphics[width=\textwidth]{figures/colorization_visuals_supp.pdf}}
  \mycaption{Video Color Propagation}
  {Input grayscale video frames and corresponding ground-truth (GT) color images
  together with color predictions of Levin et al.~\cite{levin2004colorization} and VPN-Stage1 models.}
  \label{fig:color_visuals_supp}
\end{center}
\vspace{-0.7cm}
\end{figure}

\clearpage

\section{Additional Material for Bilateral Inception Networks}
\label{sec:binception-app}

In this section of the Appendix, we first discuss the use of approximate bilateral
filtering in BI modules (Sec.~\ref{sec:lattice}).
Later, we present some qualitative results using different models for the approach presented in
Chapter~\ref{chap:binception} (Sec.~\ref{sec:qualitative-app}).

\subsection{Approximate Bilateral Filtering}
\label{sec:lattice}

The bilateral inception module presented in Chapter~\ref{chap:binception} computes a matrix-vector
product between a Gaussian filter $K$ and a vector of activations $\bz_c$.
Bilateral filtering is an important operation and many algorithmic techniques have been
proposed to speed-up this operation~\cite{paris2006fast,adams2010fast,gastal2011domain}.
In the main paper we opted to implement what can be considered the
brute-force variant of explicitly constructing $K$ and then using BLAS to compute the
matrix-vector product. This resulted in a few millisecond operation.
The explicit way to compute is possible due to the
reduction to super-pixels, e.g., it would not work for DenseCRF variants
that operate on the full image resolution.

Here, we present experiments where we use the fast approximate bilateral filtering
algorithm of~\cite{adams2010fast}, which is also used in Chapter~\ref{chap:bnn}
for learning sparse high dimensional filters. This
choice allows for larger dimensions of matrix-vector multiplication. The reason for choosing
the explicit multiplication in Chapter~\ref{chap:binception} was that it was computationally faster.
For the small sizes of the involved matrices and vectors, the explicit computation is sufficient and we had no
GPU implementation of an approximate technique that matched this runtime. Also it
is conceptually easier and the gradient to the feature transformations ($\Lambda \mathbf{f}$) is
obtained using standard matrix calculus.

\subsubsection{Experiments}

We modified the existing segmentation architectures analogous to those in Chapter~\ref{chap:binception}.
The main difference is that, here, the inception modules use the lattice
approximation~\cite{adams2010fast} to compute the bilateral filtering.
Using the lattice approximation did not allow us to back-propagate through feature transformations ($\Lambda$)
and thus we used hand-specified feature scales as will be explained later.
Specifically, we take CNN architectures from the works
of~\cite{chen2014semantic,zheng2015conditional,bell2015minc} and insert the BI modules between
the spatial FC layers.
We use superpixels from~\cite{DollarICCV13edges}
for all the experiments with the lattice approximation. Experiments are
performed using Caffe neural network framework~\cite{jia2014caffe}.

\begin{table}
  \small
  \centering
  \begin{tabular}{p{5.5cm}>{\raggedright\arraybackslash}p{1.4cm}>{\centering\arraybackslash}p{2.2cm}}
    \toprule
		\textbf{Model} & \emph{IoU} & \emph{Runtime}(ms) \\
    \midrule

    %%%%%%%%%%%% Scores computed by us)%%%%%%%%%%%%
		\deeplablargefov & 68.9 & 145ms\\
    \midrule
    \bi{7}{2}-\bi{8}{10}& \textbf{73.8} & +600 \\
    \midrule
    \deeplablargefovcrf~\cite{chen2014semantic} & 72.7 & +830\\
    \deeplabmsclargefovcrf~\cite{chen2014semantic} & \textbf{73.6} & +880\\
    DeepLab-EdgeNet~\cite{chen2015semantic} & 71.7 & +30\\
    DeepLab-EdgeNet-CRF~\cite{chen2015semantic} & \textbf{73.6} & +860\\
  \bottomrule \\
  \end{tabular}
  \mycaption{Semantic Segmentation using the DeepLab model}
  {IoU scores on the Pascal VOC12 segmentation test dataset
  with different models and our modified inception model.
  Also shown are the corresponding runtimes in milliseconds. Runtimes
  also include superpixel computations (300 ms with Dollar superpixels~\cite{DollarICCV13edges})}
  \label{tab:largefovresults}
\end{table}

\paragraph{Semantic Segmentation}
The experiments in this section use the Pascal VOC12 segmentation dataset~\cite{voc2012segmentation} with 21 object classes and the images have a maximum resolution of 0.25 megapixels.
For all experiments on VOC12, we train using the extended training set of
10581 images collected by~\cite{hariharan2011moredata}.
We modified the \deeplab~network architecture of~\cite{chen2014semantic} and
the CRFasRNN architecture from~\cite{zheng2015conditional} which uses a CNN with
deconvolution layers followed by DenseCRF trained end-to-end.

\paragraph{DeepLab Model}\label{sec:deeplabmodel}
We experimented with the \bi{7}{2}-\bi{8}{10} inception model.
Results using the~\deeplab~model are summarized in Tab.~\ref{tab:largefovresults}.
Although we get similar improvements with inception modules as with the
explicit kernel computation, using lattice approximation is slower.

\begin{table}
  \small
  \centering
  \begin{tabular}{p{6.4cm}>{\raggedright\arraybackslash}p{1.8cm}>{\raggedright\arraybackslash}p{1.8cm}}
    \toprule
    \textbf{Model} & \emph{IoU (Val)} & \emph{IoU (Test)}\\
    \midrule
    %%%%%%%%%%%% Scores computed by us)%%%%%%%%%%%%
    CNN &  67.5 & - \\
    \deconv (CNN+Deconvolutions) & 69.8 & 72.0 \\
    \midrule
    \bi{3}{6}-\bi{4}{6}-\bi{7}{2}-\bi{8}{6}& 71.9 & - \\
    \bi{3}{6}-\bi{4}{6}-\bi{7}{2}-\bi{8}{6}-\gi{6}& 73.6 &  \href{http://host.robots.ox.ac.uk:8080/anonymous/VOTV5E.html}{\textbf{75.2}}\\
    \midrule
    \deconvcrf (CRF-RNN)~\cite{zheng2015conditional} & 73.0 & 74.7\\
    Context-CRF-RNN~\cite{yu2015multi} & ~~ - ~ & \textbf{75.3} \\
    \bottomrule \\
  \end{tabular}
  \mycaption{Semantic Segmentation using the CRFasRNN model}{IoU score corresponding to different models
  on Pascal VOC12 reduced validation / test segmentation dataset. The reduced validation set consists of 346 images
  as used in~\cite{zheng2015conditional} where we adapted the model from.}
  \label{tab:deconvresults-app}
\end{table}

\paragraph{CRFasRNN Model}\label{sec:deepinception}
We add BI modules after score-pool3, score-pool4, \fc{7} and \fc{8} $1\times1$ convolution layers
resulting in the \bi{3}{6}-\bi{4}{6}-\bi{7}{2}-\bi{8}{6}
model and also experimented with another variant where $BI_8$ is followed by another inception
module, G$(6)$, with 6 Gaussian kernels.
Note that here also we discarded both deconvolution and DenseCRF parts of the original model~\cite{zheng2015conditional}
and inserted the BI modules in the base CNN and found similar improvements compared to the inception modules with explicit
kernel computaion. See Tab.~\ref{tab:deconvresults-app} for results on the CRFasRNN model.

\paragraph{Material Segmentation}
Table~\ref{tab:mincresults-app} shows the results on the MINC dataset~\cite{bell2015minc}
obtained by modifying the AlexNet architecture with our inception modules. We observe
similar improvements as with explicit kernel construction.
For this model, we do not provide any learned setup due to very limited segment training
data. The weights to combine outputs in the bilateral inception layer are
found by validation on the validation set.

\begin{table}[t]
  \small
  \centering
  \begin{tabular}{p{3.5cm}>{\centering\arraybackslash}p{4.0cm}}
    \toprule
    \textbf{Model} & Class / Total accuracy\\
    \midrule

    %%%%%%%%%%%% Scores computed by us)%%%%%%%%%%%%
    AlexNet CNN & 55.3 / 58.9 \\
    \midrule
    \bi{7}{2}-\bi{8}{6}& 68.5 / 71.8 \\
    \bi{7}{2}-\bi{8}{6}-G$(6)$& 67.6 / 73.1 \\
    \midrule
    AlexNet-CRF & 65.5 / 71.0 \\
    \bottomrule \\
  \end{tabular}
  \mycaption{Material Segmentation using AlexNet}{Pixel accuracy of different models on
  the MINC material segmentation test dataset~\cite{bell2015minc}.}
  \label{tab:mincresults-app}
\end{table}

\paragraph{Scales of Bilateral Inception Modules}
\label{sec:scales}

Unlike the explicit kernel technique presented in the main text (Chapter~\ref{chap:binception}),
we didn't back-propagate through feature transformation ($\Lambda$)
using the approximate bilateral filter technique.
So, the feature scales are hand-specified and validated, which are as follows.
The optimal scale values for the \bi{7}{2}-\bi{8}{2} model are found by validation for the best performance which are
$\sigma_{xy}$ = (0.1, 0.1) for the spatial (XY) kernel and $\sigma_{rgbxy}$ = (0.1, 0.1, 0.1, 0.01, 0.01) for color and position (RGBXY)  kernel.
Next, as more kernels are added to \bi{8}{2}, we set scales to be $\alpha$*($\sigma_{xy}$, $\sigma_{rgbxy}$).
The value of $\alpha$ is chosen as  1, 0.5, 0.1, 0.05, 0.1, at uniform interval, for the \bi{8}{10} bilateral inception module.


\subsection{Qualitative Results}
\label{sec:qualitative-app}

In this section, we present more qualitative results obtained using the BI module with explicit
kernel computation technique presented in Chapter~\ref{chap:binception}. Results on the Pascal VOC12
dataset~\cite{voc2012segmentation} using the DeepLab-LargeFOV model are shown in Fig.~\ref{fig:semantic_visuals-app},
followed by the results on MINC dataset~\cite{bell2015minc}
in Fig.~\ref{fig:material_visuals-app} and on
Cityscapes dataset~\cite{Cordts2015Cvprw} in Fig.~\ref{fig:street_visuals-app}.


\definecolor{voc_1}{RGB}{0, 0, 0}
\definecolor{voc_2}{RGB}{128, 0, 0}
\definecolor{voc_3}{RGB}{0, 128, 0}
\definecolor{voc_4}{RGB}{128, 128, 0}
\definecolor{voc_5}{RGB}{0, 0, 128}
\definecolor{voc_6}{RGB}{128, 0, 128}
\definecolor{voc_7}{RGB}{0, 128, 128}
\definecolor{voc_8}{RGB}{128, 128, 128}
\definecolor{voc_9}{RGB}{64, 0, 0}
\definecolor{voc_10}{RGB}{192, 0, 0}
\definecolor{voc_11}{RGB}{64, 128, 0}
\definecolor{voc_12}{RGB}{192, 128, 0}
\definecolor{voc_13}{RGB}{64, 0, 128}
\definecolor{voc_14}{RGB}{192, 0, 128}
\definecolor{voc_15}{RGB}{64, 128, 128}
\definecolor{voc_16}{RGB}{192, 128, 128}
\definecolor{voc_17}{RGB}{0, 64, 0}
\definecolor{voc_18}{RGB}{128, 64, 0}
\definecolor{voc_19}{RGB}{0, 192, 0}
\definecolor{voc_20}{RGB}{128, 192, 0}
\definecolor{voc_21}{RGB}{0, 64, 128}
\definecolor{voc_22}{RGB}{128, 64, 128}

\begin{figure*}[!ht]
  \small
  \centering
  \fcolorbox{white}{voc_1}{\rule{0pt}{4pt}\rule{4pt}{0pt}} Background~~
  \fcolorbox{white}{voc_2}{\rule{0pt}{4pt}\rule{4pt}{0pt}} Aeroplane~~
  \fcolorbox{white}{voc_3}{\rule{0pt}{4pt}\rule{4pt}{0pt}} Bicycle~~
  \fcolorbox{white}{voc_4}{\rule{0pt}{4pt}\rule{4pt}{0pt}} Bird~~
  \fcolorbox{white}{voc_5}{\rule{0pt}{4pt}\rule{4pt}{0pt}} Boat~~
  \fcolorbox{white}{voc_6}{\rule{0pt}{4pt}\rule{4pt}{0pt}} Bottle~~
  \fcolorbox{white}{voc_7}{\rule{0pt}{4pt}\rule{4pt}{0pt}} Bus~~
  \fcolorbox{white}{voc_8}{\rule{0pt}{4pt}\rule{4pt}{0pt}} Car~~\\
  \fcolorbox{white}{voc_9}{\rule{0pt}{4pt}\rule{4pt}{0pt}} Cat~~
  \fcolorbox{white}{voc_10}{\rule{0pt}{4pt}\rule{4pt}{0pt}} Chair~~
  \fcolorbox{white}{voc_11}{\rule{0pt}{4pt}\rule{4pt}{0pt}} Cow~~
  \fcolorbox{white}{voc_12}{\rule{0pt}{4pt}\rule{4pt}{0pt}} Dining Table~~
  \fcolorbox{white}{voc_13}{\rule{0pt}{4pt}\rule{4pt}{0pt}} Dog~~
  \fcolorbox{white}{voc_14}{\rule{0pt}{4pt}\rule{4pt}{0pt}} Horse~~
  \fcolorbox{white}{voc_15}{\rule{0pt}{4pt}\rule{4pt}{0pt}} Motorbike~~
  \fcolorbox{white}{voc_16}{\rule{0pt}{4pt}\rule{4pt}{0pt}} Person~~\\
  \fcolorbox{white}{voc_17}{\rule{0pt}{4pt}\rule{4pt}{0pt}} Potted Plant~~
  \fcolorbox{white}{voc_18}{\rule{0pt}{4pt}\rule{4pt}{0pt}} Sheep~~
  \fcolorbox{white}{voc_19}{\rule{0pt}{4pt}\rule{4pt}{0pt}} Sofa~~
  \fcolorbox{white}{voc_20}{\rule{0pt}{4pt}\rule{4pt}{0pt}} Train~~
  \fcolorbox{white}{voc_21}{\rule{0pt}{4pt}\rule{4pt}{0pt}} TV monitor~~\\


  \subfigure{%
    \includegraphics[width=.15\columnwidth]{figures/supplementary/2008_001308_given.png}
  }
  \subfigure{%
    \includegraphics[width=.15\columnwidth]{figures/supplementary/2008_001308_sp.png}
  }
  \subfigure{%
    \includegraphics[width=.15\columnwidth]{figures/supplementary/2008_001308_gt.png}
  }
  \subfigure{%
    \includegraphics[width=.15\columnwidth]{figures/supplementary/2008_001308_cnn.png}
  }
  \subfigure{%
    \includegraphics[width=.15\columnwidth]{figures/supplementary/2008_001308_crf.png}
  }
  \subfigure{%
    \includegraphics[width=.15\columnwidth]{figures/supplementary/2008_001308_ours.png}
  }\\[-2ex]


  \subfigure{%
    \includegraphics[width=.15\columnwidth]{figures/supplementary/2008_001821_given.png}
  }
  \subfigure{%
    \includegraphics[width=.15\columnwidth]{figures/supplementary/2008_001821_sp.png}
  }
  \subfigure{%
    \includegraphics[width=.15\columnwidth]{figures/supplementary/2008_001821_gt.png}
  }
  \subfigure{%
    \includegraphics[width=.15\columnwidth]{figures/supplementary/2008_001821_cnn.png}
  }
  \subfigure{%
    \includegraphics[width=.15\columnwidth]{figures/supplementary/2008_001821_crf.png}
  }
  \subfigure{%
    \includegraphics[width=.15\columnwidth]{figures/supplementary/2008_001821_ours.png}
  }\\[-2ex]



  \subfigure{%
    \includegraphics[width=.15\columnwidth]{figures/supplementary/2008_004612_given.png}
  }
  \subfigure{%
    \includegraphics[width=.15\columnwidth]{figures/supplementary/2008_004612_sp.png}
  }
  \subfigure{%
    \includegraphics[width=.15\columnwidth]{figures/supplementary/2008_004612_gt.png}
  }
  \subfigure{%
    \includegraphics[width=.15\columnwidth]{figures/supplementary/2008_004612_cnn.png}
  }
  \subfigure{%
    \includegraphics[width=.15\columnwidth]{figures/supplementary/2008_004612_crf.png}
  }
  \subfigure{%
    \includegraphics[width=.15\columnwidth]{figures/supplementary/2008_004612_ours.png}
  }\\[-2ex]


  \subfigure{%
    \includegraphics[width=.15\columnwidth]{figures/supplementary/2009_001008_given.png}
  }
  \subfigure{%
    \includegraphics[width=.15\columnwidth]{figures/supplementary/2009_001008_sp.png}
  }
  \subfigure{%
    \includegraphics[width=.15\columnwidth]{figures/supplementary/2009_001008_gt.png}
  }
  \subfigure{%
    \includegraphics[width=.15\columnwidth]{figures/supplementary/2009_001008_cnn.png}
  }
  \subfigure{%
    \includegraphics[width=.15\columnwidth]{figures/supplementary/2009_001008_crf.png}
  }
  \subfigure{%
    \includegraphics[width=.15\columnwidth]{figures/supplementary/2009_001008_ours.png}
  }\\[-2ex]




  \subfigure{%
    \includegraphics[width=.15\columnwidth]{figures/supplementary/2009_004497_given.png}
  }
  \subfigure{%
    \includegraphics[width=.15\columnwidth]{figures/supplementary/2009_004497_sp.png}
  }
  \subfigure{%
    \includegraphics[width=.15\columnwidth]{figures/supplementary/2009_004497_gt.png}
  }
  \subfigure{%
    \includegraphics[width=.15\columnwidth]{figures/supplementary/2009_004497_cnn.png}
  }
  \subfigure{%
    \includegraphics[width=.15\columnwidth]{figures/supplementary/2009_004497_crf.png}
  }
  \subfigure{%
    \includegraphics[width=.15\columnwidth]{figures/supplementary/2009_004497_ours.png}
  }\\[-2ex]



  \setcounter{subfigure}{0}
  \subfigure[\scriptsize Input]{%
    \includegraphics[width=.15\columnwidth]{figures/supplementary/2010_001327_given.png}
  }
  \subfigure[\scriptsize Superpixels]{%
    \includegraphics[width=.15\columnwidth]{figures/supplementary/2010_001327_sp.png}
  }
  \subfigure[\scriptsize GT]{%
    \includegraphics[width=.15\columnwidth]{figures/supplementary/2010_001327_gt.png}
  }
  \subfigure[\scriptsize Deeplab]{%
    \includegraphics[width=.15\columnwidth]{figures/supplementary/2010_001327_cnn.png}
  }
  \subfigure[\scriptsize +DenseCRF]{%
    \includegraphics[width=.15\columnwidth]{figures/supplementary/2010_001327_crf.png}
  }
  \subfigure[\scriptsize Using BI]{%
    \includegraphics[width=.15\columnwidth]{figures/supplementary/2010_001327_ours.png}
  }
  \mycaption{Semantic Segmentation}{Example results of semantic segmentation
  on the Pascal VOC12 dataset.
  (d)~depicts the DeepLab CNN result, (e)~CNN + 10 steps of mean-field inference,
  (f~result obtained with bilateral inception (BI) modules (\bi{6}{2}+\bi{7}{6}) between \fc~layers.}
  \label{fig:semantic_visuals-app}
\end{figure*}


\definecolor{minc_1}{HTML}{771111}
\definecolor{minc_2}{HTML}{CAC690}
\definecolor{minc_3}{HTML}{EEEEEE}
\definecolor{minc_4}{HTML}{7C8FA6}
\definecolor{minc_5}{HTML}{597D31}
\definecolor{minc_6}{HTML}{104410}
\definecolor{minc_7}{HTML}{BB819C}
\definecolor{minc_8}{HTML}{D0CE48}
\definecolor{minc_9}{HTML}{622745}
\definecolor{minc_10}{HTML}{666666}
\definecolor{minc_11}{HTML}{D54A31}
\definecolor{minc_12}{HTML}{101044}
\definecolor{minc_13}{HTML}{444126}
\definecolor{minc_14}{HTML}{75D646}
\definecolor{minc_15}{HTML}{DD4348}
\definecolor{minc_16}{HTML}{5C8577}
\definecolor{minc_17}{HTML}{C78472}
\definecolor{minc_18}{HTML}{75D6D0}
\definecolor{minc_19}{HTML}{5B4586}
\definecolor{minc_20}{HTML}{C04393}
\definecolor{minc_21}{HTML}{D69948}
\definecolor{minc_22}{HTML}{7370D8}
\definecolor{minc_23}{HTML}{7A3622}
\definecolor{minc_24}{HTML}{000000}

\begin{figure*}[!ht]
  \small % scriptsize
  \centering
  \fcolorbox{white}{minc_1}{\rule{0pt}{4pt}\rule{4pt}{0pt}} Brick~~
  \fcolorbox{white}{minc_2}{\rule{0pt}{4pt}\rule{4pt}{0pt}} Carpet~~
  \fcolorbox{white}{minc_3}{\rule{0pt}{4pt}\rule{4pt}{0pt}} Ceramic~~
  \fcolorbox{white}{minc_4}{\rule{0pt}{4pt}\rule{4pt}{0pt}} Fabric~~
  \fcolorbox{white}{minc_5}{\rule{0pt}{4pt}\rule{4pt}{0pt}} Foliage~~
  \fcolorbox{white}{minc_6}{\rule{0pt}{4pt}\rule{4pt}{0pt}} Food~~
  \fcolorbox{white}{minc_7}{\rule{0pt}{4pt}\rule{4pt}{0pt}} Glass~~
  \fcolorbox{white}{minc_8}{\rule{0pt}{4pt}\rule{4pt}{0pt}} Hair~~\\
  \fcolorbox{white}{minc_9}{\rule{0pt}{4pt}\rule{4pt}{0pt}} Leather~~
  \fcolorbox{white}{minc_10}{\rule{0pt}{4pt}\rule{4pt}{0pt}} Metal~~
  \fcolorbox{white}{minc_11}{\rule{0pt}{4pt}\rule{4pt}{0pt}} Mirror~~
  \fcolorbox{white}{minc_12}{\rule{0pt}{4pt}\rule{4pt}{0pt}} Other~~
  \fcolorbox{white}{minc_13}{\rule{0pt}{4pt}\rule{4pt}{0pt}} Painted~~
  \fcolorbox{white}{minc_14}{\rule{0pt}{4pt}\rule{4pt}{0pt}} Paper~~
  \fcolorbox{white}{minc_15}{\rule{0pt}{4pt}\rule{4pt}{0pt}} Plastic~~\\
  \fcolorbox{white}{minc_16}{\rule{0pt}{4pt}\rule{4pt}{0pt}} Polished Stone~~
  \fcolorbox{white}{minc_17}{\rule{0pt}{4pt}\rule{4pt}{0pt}} Skin~~
  \fcolorbox{white}{minc_18}{\rule{0pt}{4pt}\rule{4pt}{0pt}} Sky~~
  \fcolorbox{white}{minc_19}{\rule{0pt}{4pt}\rule{4pt}{0pt}} Stone~~
  \fcolorbox{white}{minc_20}{\rule{0pt}{4pt}\rule{4pt}{0pt}} Tile~~
  \fcolorbox{white}{minc_21}{\rule{0pt}{4pt}\rule{4pt}{0pt}} Wallpaper~~
  \fcolorbox{white}{minc_22}{\rule{0pt}{4pt}\rule{4pt}{0pt}} Water~~
  \fcolorbox{white}{minc_23}{\rule{0pt}{4pt}\rule{4pt}{0pt}} Wood~~\\
  \subfigure{%
    \includegraphics[width=.15\columnwidth]{figures/supplementary/000008468_given.png}
  }
  \subfigure{%
    \includegraphics[width=.15\columnwidth]{figures/supplementary/000008468_sp.png}
  }
  \subfigure{%
    \includegraphics[width=.15\columnwidth]{figures/supplementary/000008468_gt.png}
  }
  \subfigure{%
    \includegraphics[width=.15\columnwidth]{figures/supplementary/000008468_cnn.png}
  }
  \subfigure{%
    \includegraphics[width=.15\columnwidth]{figures/supplementary/000008468_crf.png}
  }
  \subfigure{%
    \includegraphics[width=.15\columnwidth]{figures/supplementary/000008468_ours.png}
  }\\[-2ex]

  \subfigure{%
    \includegraphics[width=.15\columnwidth]{figures/supplementary/000009053_given.png}
  }
  \subfigure{%
    \includegraphics[width=.15\columnwidth]{figures/supplementary/000009053_sp.png}
  }
  \subfigure{%
    \includegraphics[width=.15\columnwidth]{figures/supplementary/000009053_gt.png}
  }
  \subfigure{%
    \includegraphics[width=.15\columnwidth]{figures/supplementary/000009053_cnn.png}
  }
  \subfigure{%
    \includegraphics[width=.15\columnwidth]{figures/supplementary/000009053_crf.png}
  }
  \subfigure{%
    \includegraphics[width=.15\columnwidth]{figures/supplementary/000009053_ours.png}
  }\\[-2ex]




  \subfigure{%
    \includegraphics[width=.15\columnwidth]{figures/supplementary/000014977_given.png}
  }
  \subfigure{%
    \includegraphics[width=.15\columnwidth]{figures/supplementary/000014977_sp.png}
  }
  \subfigure{%
    \includegraphics[width=.15\columnwidth]{figures/supplementary/000014977_gt.png}
  }
  \subfigure{%
    \includegraphics[width=.15\columnwidth]{figures/supplementary/000014977_cnn.png}
  }
  \subfigure{%
    \includegraphics[width=.15\columnwidth]{figures/supplementary/000014977_crf.png}
  }
  \subfigure{%
    \includegraphics[width=.15\columnwidth]{figures/supplementary/000014977_ours.png}
  }\\[-2ex]


  \subfigure{%
    \includegraphics[width=.15\columnwidth]{figures/supplementary/000022922_given.png}
  }
  \subfigure{%
    \includegraphics[width=.15\columnwidth]{figures/supplementary/000022922_sp.png}
  }
  \subfigure{%
    \includegraphics[width=.15\columnwidth]{figures/supplementary/000022922_gt.png}
  }
  \subfigure{%
    \includegraphics[width=.15\columnwidth]{figures/supplementary/000022922_cnn.png}
  }
  \subfigure{%
    \includegraphics[width=.15\columnwidth]{figures/supplementary/000022922_crf.png}
  }
  \subfigure{%
    \includegraphics[width=.15\columnwidth]{figures/supplementary/000022922_ours.png}
  }\\[-2ex]


  \subfigure{%
    \includegraphics[width=.15\columnwidth]{figures/supplementary/000025711_given.png}
  }
  \subfigure{%
    \includegraphics[width=.15\columnwidth]{figures/supplementary/000025711_sp.png}
  }
  \subfigure{%
    \includegraphics[width=.15\columnwidth]{figures/supplementary/000025711_gt.png}
  }
  \subfigure{%
    \includegraphics[width=.15\columnwidth]{figures/supplementary/000025711_cnn.png}
  }
  \subfigure{%
    \includegraphics[width=.15\columnwidth]{figures/supplementary/000025711_crf.png}
  }
  \subfigure{%
    \includegraphics[width=.15\columnwidth]{figures/supplementary/000025711_ours.png}
  }\\[-2ex]


  \subfigure{%
    \includegraphics[width=.15\columnwidth]{figures/supplementary/000034473_given.png}
  }
  \subfigure{%
    \includegraphics[width=.15\columnwidth]{figures/supplementary/000034473_sp.png}
  }
  \subfigure{%
    \includegraphics[width=.15\columnwidth]{figures/supplementary/000034473_gt.png}
  }
  \subfigure{%
    \includegraphics[width=.15\columnwidth]{figures/supplementary/000034473_cnn.png}
  }
  \subfigure{%
    \includegraphics[width=.15\columnwidth]{figures/supplementary/000034473_crf.png}
  }
  \subfigure{%
    \includegraphics[width=.15\columnwidth]{figures/supplementary/000034473_ours.png}
  }\\[-2ex]


  \subfigure{%
    \includegraphics[width=.15\columnwidth]{figures/supplementary/000035463_given.png}
  }
  \subfigure{%
    \includegraphics[width=.15\columnwidth]{figures/supplementary/000035463_sp.png}
  }
  \subfigure{%
    \includegraphics[width=.15\columnwidth]{figures/supplementary/000035463_gt.png}
  }
  \subfigure{%
    \includegraphics[width=.15\columnwidth]{figures/supplementary/000035463_cnn.png}
  }
  \subfigure{%
    \includegraphics[width=.15\columnwidth]{figures/supplementary/000035463_crf.png}
  }
  \subfigure{%
    \includegraphics[width=.15\columnwidth]{figures/supplementary/000035463_ours.png}
  }\\[-2ex]


  \setcounter{subfigure}{0}
  \subfigure[\scriptsize Input]{%
    \includegraphics[width=.15\columnwidth]{figures/supplementary/000035993_given.png}
  }
  \subfigure[\scriptsize Superpixels]{%
    \includegraphics[width=.15\columnwidth]{figures/supplementary/000035993_sp.png}
  }
  \subfigure[\scriptsize GT]{%
    \includegraphics[width=.15\columnwidth]{figures/supplementary/000035993_gt.png}
  }
  \subfigure[\scriptsize AlexNet]{%
    \includegraphics[width=.15\columnwidth]{figures/supplementary/000035993_cnn.png}
  }
  \subfigure[\scriptsize +DenseCRF]{%
    \includegraphics[width=.15\columnwidth]{figures/supplementary/000035993_crf.png}
  }
  \subfigure[\scriptsize Using BI]{%
    \includegraphics[width=.15\columnwidth]{figures/supplementary/000035993_ours.png}
  }
  \mycaption{Material Segmentation}{Example results of material segmentation.
  (d)~depicts the AlexNet CNN result, (e)~CNN + 10 steps of mean-field inference,
  (f)~result obtained with bilateral inception (BI) modules (\bi{7}{2}+\bi{8}{6}) between
  \fc~layers.}
\label{fig:material_visuals-app}
\end{figure*}


\definecolor{city_1}{RGB}{128, 64, 128}
\definecolor{city_2}{RGB}{244, 35, 232}
\definecolor{city_3}{RGB}{70, 70, 70}
\definecolor{city_4}{RGB}{102, 102, 156}
\definecolor{city_5}{RGB}{190, 153, 153}
\definecolor{city_6}{RGB}{153, 153, 153}
\definecolor{city_7}{RGB}{250, 170, 30}
\definecolor{city_8}{RGB}{220, 220, 0}
\definecolor{city_9}{RGB}{107, 142, 35}
\definecolor{city_10}{RGB}{152, 251, 152}
\definecolor{city_11}{RGB}{70, 130, 180}
\definecolor{city_12}{RGB}{220, 20, 60}
\definecolor{city_13}{RGB}{255, 0, 0}
\definecolor{city_14}{RGB}{0, 0, 142}
\definecolor{city_15}{RGB}{0, 0, 70}
\definecolor{city_16}{RGB}{0, 60, 100}
\definecolor{city_17}{RGB}{0, 80, 100}
\definecolor{city_18}{RGB}{0, 0, 230}
\definecolor{city_19}{RGB}{119, 11, 32}
\begin{figure*}[!ht]
  \small % scriptsize
  \centering


  \subfigure{%
    \includegraphics[width=.18\columnwidth]{figures/supplementary/frankfurt00000_016005_given.png}
  }
  \subfigure{%
    \includegraphics[width=.18\columnwidth]{figures/supplementary/frankfurt00000_016005_sp.png}
  }
  \subfigure{%
    \includegraphics[width=.18\columnwidth]{figures/supplementary/frankfurt00000_016005_gt.png}
  }
  \subfigure{%
    \includegraphics[width=.18\columnwidth]{figures/supplementary/frankfurt00000_016005_cnn.png}
  }
  \subfigure{%
    \includegraphics[width=.18\columnwidth]{figures/supplementary/frankfurt00000_016005_ours.png}
  }\\[-2ex]

  \subfigure{%
    \includegraphics[width=.18\columnwidth]{figures/supplementary/frankfurt00000_004617_given.png}
  }
  \subfigure{%
    \includegraphics[width=.18\columnwidth]{figures/supplementary/frankfurt00000_004617_sp.png}
  }
  \subfigure{%
    \includegraphics[width=.18\columnwidth]{figures/supplementary/frankfurt00000_004617_gt.png}
  }
  \subfigure{%
    \includegraphics[width=.18\columnwidth]{figures/supplementary/frankfurt00000_004617_cnn.png}
  }
  \subfigure{%
    \includegraphics[width=.18\columnwidth]{figures/supplementary/frankfurt00000_004617_ours.png}
  }\\[-2ex]

  \subfigure{%
    \includegraphics[width=.18\columnwidth]{figures/supplementary/frankfurt00000_020880_given.png}
  }
  \subfigure{%
    \includegraphics[width=.18\columnwidth]{figures/supplementary/frankfurt00000_020880_sp.png}
  }
  \subfigure{%
    \includegraphics[width=.18\columnwidth]{figures/supplementary/frankfurt00000_020880_gt.png}
  }
  \subfigure{%
    \includegraphics[width=.18\columnwidth]{figures/supplementary/frankfurt00000_020880_cnn.png}
  }
  \subfigure{%
    \includegraphics[width=.18\columnwidth]{figures/supplementary/frankfurt00000_020880_ours.png}
  }\\[-2ex]



  \subfigure{%
    \includegraphics[width=.18\columnwidth]{figures/supplementary/frankfurt00001_007285_given.png}
  }
  \subfigure{%
    \includegraphics[width=.18\columnwidth]{figures/supplementary/frankfurt00001_007285_sp.png}
  }
  \subfigure{%
    \includegraphics[width=.18\columnwidth]{figures/supplementary/frankfurt00001_007285_gt.png}
  }
  \subfigure{%
    \includegraphics[width=.18\columnwidth]{figures/supplementary/frankfurt00001_007285_cnn.png}
  }
  \subfigure{%
    \includegraphics[width=.18\columnwidth]{figures/supplementary/frankfurt00001_007285_ours.png}
  }\\[-2ex]


  \subfigure{%
    \includegraphics[width=.18\columnwidth]{figures/supplementary/frankfurt00001_059789_given.png}
  }
  \subfigure{%
    \includegraphics[width=.18\columnwidth]{figures/supplementary/frankfurt00001_059789_sp.png}
  }
  \subfigure{%
    \includegraphics[width=.18\columnwidth]{figures/supplementary/frankfurt00001_059789_gt.png}
  }
  \subfigure{%
    \includegraphics[width=.18\columnwidth]{figures/supplementary/frankfurt00001_059789_cnn.png}
  }
  \subfigure{%
    \includegraphics[width=.18\columnwidth]{figures/supplementary/frankfurt00001_059789_ours.png}
  }\\[-2ex]


  \subfigure{%
    \includegraphics[width=.18\columnwidth]{figures/supplementary/frankfurt00001_068208_given.png}
  }
  \subfigure{%
    \includegraphics[width=.18\columnwidth]{figures/supplementary/frankfurt00001_068208_sp.png}
  }
  \subfigure{%
    \includegraphics[width=.18\columnwidth]{figures/supplementary/frankfurt00001_068208_gt.png}
  }
  \subfigure{%
    \includegraphics[width=.18\columnwidth]{figures/supplementary/frankfurt00001_068208_cnn.png}
  }
  \subfigure{%
    \includegraphics[width=.18\columnwidth]{figures/supplementary/frankfurt00001_068208_ours.png}
  }\\[-2ex]

  \subfigure{%
    \includegraphics[width=.18\columnwidth]{figures/supplementary/frankfurt00001_082466_given.png}
  }
  \subfigure{%
    \includegraphics[width=.18\columnwidth]{figures/supplementary/frankfurt00001_082466_sp.png}
  }
  \subfigure{%
    \includegraphics[width=.18\columnwidth]{figures/supplementary/frankfurt00001_082466_gt.png}
  }
  \subfigure{%
    \includegraphics[width=.18\columnwidth]{figures/supplementary/frankfurt00001_082466_cnn.png}
  }
  \subfigure{%
    \includegraphics[width=.18\columnwidth]{figures/supplementary/frankfurt00001_082466_ours.png}
  }\\[-2ex]

  \subfigure{%
    \includegraphics[width=.18\columnwidth]{figures/supplementary/lindau00033_000019_given.png}
  }
  \subfigure{%
    \includegraphics[width=.18\columnwidth]{figures/supplementary/lindau00033_000019_sp.png}
  }
  \subfigure{%
    \includegraphics[width=.18\columnwidth]{figures/supplementary/lindau00033_000019_gt.png}
  }
  \subfigure{%
    \includegraphics[width=.18\columnwidth]{figures/supplementary/lindau00033_000019_cnn.png}
  }
  \subfigure{%
    \includegraphics[width=.18\columnwidth]{figures/supplementary/lindau00033_000019_ours.png}
  }\\[-2ex]

  \subfigure{%
    \includegraphics[width=.18\columnwidth]{figures/supplementary/lindau00052_000019_given.png}
  }
  \subfigure{%
    \includegraphics[width=.18\columnwidth]{figures/supplementary/lindau00052_000019_sp.png}
  }
  \subfigure{%
    \includegraphics[width=.18\columnwidth]{figures/supplementary/lindau00052_000019_gt.png}
  }
  \subfigure{%
    \includegraphics[width=.18\columnwidth]{figures/supplementary/lindau00052_000019_cnn.png}
  }
  \subfigure{%
    \includegraphics[width=.18\columnwidth]{figures/supplementary/lindau00052_000019_ours.png}
  }\\[-2ex]




  \subfigure{%
    \includegraphics[width=.18\columnwidth]{figures/supplementary/lindau00027_000019_given.png}
  }
  \subfigure{%
    \includegraphics[width=.18\columnwidth]{figures/supplementary/lindau00027_000019_sp.png}
  }
  \subfigure{%
    \includegraphics[width=.18\columnwidth]{figures/supplementary/lindau00027_000019_gt.png}
  }
  \subfigure{%
    \includegraphics[width=.18\columnwidth]{figures/supplementary/lindau00027_000019_cnn.png}
  }
  \subfigure{%
    \includegraphics[width=.18\columnwidth]{figures/supplementary/lindau00027_000019_ours.png}
  }\\[-2ex]



  \setcounter{subfigure}{0}
  \subfigure[\scriptsize Input]{%
    \includegraphics[width=.18\columnwidth]{figures/supplementary/lindau00029_000019_given.png}
  }
  \subfigure[\scriptsize Superpixels]{%
    \includegraphics[width=.18\columnwidth]{figures/supplementary/lindau00029_000019_sp.png}
  }
  \subfigure[\scriptsize GT]{%
    \includegraphics[width=.18\columnwidth]{figures/supplementary/lindau00029_000019_gt.png}
  }
  \subfigure[\scriptsize Deeplab]{%
    \includegraphics[width=.18\columnwidth]{figures/supplementary/lindau00029_000019_cnn.png}
  }
  \subfigure[\scriptsize Using BI]{%
    \includegraphics[width=.18\columnwidth]{figures/supplementary/lindau00029_000019_ours.png}
  }%\\[-2ex]

  \mycaption{Street Scene Segmentation}{Example results of street scene segmentation.
  (d)~depicts the DeepLab results, (e)~result obtained by adding bilateral inception (BI) modules (\bi{6}{2}+\bi{7}{6}) between \fc~layers.}
\label{fig:street_visuals-app}
\end{figure*}

\newpage
\subsubsection*{C. Example using ADO for multi-versioning}
\label{appendix:ado_versioning}
In this section, we describe a simple ADO example that adds versioning
to the basic key-value store.
\footnote{https://github.com/IBM/mcas/blob/master/examples/personalities/cpp\_versioning/}
The ADO creates an area in persistent memory that saves multiple
versions of values for a specific key and allows the client to
retrieve prior versions of a value.  In the ADO layer we ``raise''
the \code{get} and the \code{put} operations into the ADO handling
above the basic client API (see Table~\ref{tab:clientapi}). The code
is split into client-side library and server-side plugin. The message
protocol implementation is based on flatbuffers.

A client can invoke \code{put} and \code{get} operations (see Code
Listing~\ref{lst:client versioning}).  Under the hood, \code{put}
and \code{get} invocations result in calls to \code{invoke\_put\_ado}
and \code{invoke\_ado} respectively.

Corresponding messages are
constructed and sent as part of the ADO invoke payloads.  In both operations, the target is a
specific pool and key pair.  The messages are transmitted from the
client over the network to the main shard process and then transferred to
the ADO process via user-level IPC (see
Section~\ref{subsec:ADO Invocation} for more detail).
  
The ADO plugin handling starts with an up-call to the \code{do work}
 function (see Code Listing~\ref{lst:server_versioning}).  Here, the
 message is unpacked and then dispatched to the appropriate \code{put}
 or \code{get} handler.  The root pointer for the data structure that
 handles the different versions is provided as part of the \code{do
 work} invocation.  If it is the first-ever creation of the key the
 ADO plugin must initialize the root data structure.  In this example,
 the versioning metadata operations are made crash-consistent by using
 a basic undo log to ensure power-fail atomicity. In this example, we
 are using the \code{pmemlib} library and explicit 64-bit
 transactions\footnote{Building handcrafted crash-consistency can be
 complex but there are frameworks such as PMDK that can help to write
 crash-consistent code}.

During \code{put} invocation, the ADO handler creates (and persists)
an undo log that records the prior value and then perform a
transaction.  On successful completion of the transaction, the undo
log is cleared.  When the system starts the ADO checks for the need to
recover.  If the undo log is not clear, the logged data is copied back
to the original location in memory.  Finally, the result of \code{put}
and \code{get} operations are packed into a flatbuffer message and
returned to the client.

\newpage

\begin{lstlisting}[caption={Client-side for ADO versioning},
captionpos=b, label={lst:client versioning}, 
frame=none]
status_t Client::put(const pool_t pool,
                     const std::string& key,
                     const std::string& value)
{

  /* create request  message */
  ...
  s = _mcas->invoke_put_ado(pool,
                            key,
                            fbb.GetBufferPointer(),
                            fbb.GetSize(),
                            value.data(),
                            value.length() + 1, 
                            128, //root value length
                            component::IMCAS::ADO_FLAG_DETACHED,
                            response);
  return s;
}

status_t Client::get(const pool_t pool,
                     const std::string& key,
                     const int version_index,
                     std::string& out_value)
{

  /* create request  message */
  ...
  s = _mcas->invoke_ado(pool,
                        key,
                        fbb.GetBufferPointer(),
                        fbb.GetSize(),
                        0,
                        response);
  return s;
}

\end{lstlisting}


\begin{minipage}{\linewidth}
\begin{lstlisting}[caption={Server plugin for ADO versioning},
captionpos=b, label={lst:server_versioning}, 
frame=none]
status_t ADO_example_versioning_plugin::do_work(const uint64_t work_key,
                                        const char * key,
                                        size_t key_len,
                                        IADO_plugin::value_space_t& values,
                                        const void *in_work_request,
                                        const size_t in_work_request_len,
                                        bool new_root,
                                        response_buffer_vector_t& response_buffers)
{
  auto value = values[0].ptr;
  auto root = static_cast<ADO_example_versioning_plugin_root *>(value);
  if(new_root) {
    root->init();
  }
  else {
    root->check_recovery();
  }

  if(msg->element_as_PutRequest()) {
    ...
    //  Put
    auto value_to_free = root->add_version(detached_value, detached_value_len, value_to_free_len);
  /* create response message */
    return S_OK;
  }
  else if(msg->element_as_GetRequest()) {
    ...
    // Get
    root->get_version(pr->version_index(), return_value, return_value_len, timestamp);
  }
  /* create response message */
  ...
  return S_OK;
}

void init()
{
    pmem_memset_persist(this, 0, sizeof(ADO_example_versioning_plugin_root));
}

void check_recovery() {
    /* check for undo */
    if(_undo.mid_tx()) {
       /* recover from the undo log and then clear the undo log */
      _values[_current_slot] = _undo.value;
      _undo.clear();
    }
 }


void * add_version(void * value, size_t value_len, size_t& rv_len)
{
   void * rv = _values[_current_slot];
   rv_len = _value_lengths[_current_slot];

   /* create undo log for transaction */
   _undo = { _current_slot, _values[_current_slot], _value_lengths[_current_slot], _timestamps[_current_slot] };
   pmem_persist(&_undo, sizeof(_undo));

   /* perform transaction */
   _values[_current_slot] = value;

   pmem_persist(&_current_slot, sizeof(_current_slot));

   /* reset undo log */
   _undo.clear();
              
   return rv; /* return value to be deleted */
}

void get_version(int version_index, void*& out_value, size_t& out_value_len, cpu_time_t& out_time_stamp) const
{
   int slot = _current_slot - 1;
   while(version_index < 0) {
     slot--;
     if(slot == -1) slot = MAX_VERSIONS - 1;
     version_index++;
   }
   out_value = _values[slot];
}


\end{lstlisting}
\end{minipage}



\end{document}

%%  LocalWords:  Amongst
