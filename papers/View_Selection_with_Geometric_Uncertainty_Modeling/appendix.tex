\appendix
\label{sec:appendix}

\section{More Reconstruction Results}
\begin{figure*}
\centering
	\includegraphics[width=1\textwidth]{fig/dense-comparison-UAV30.pdf}
	\caption{Comparison of dense reconstruction of the orchard taken at 30 meters altitude. (a) Dense Reconstruction using 875 images, with closeup views of the trees. (b) Dense Reconstruction using 209 images extracted using our multi-resolution view selection method, with closeup views of the trees. }
\label{fig:denseComp30}
\end{figure*} 



\section{Lemmas and Theorems}
\subsection{Proof of Lemma~\ref{lem:diag1max}}
\begin{proof}
We prove the lemma by contradiction:
Suppose there exists a camera $s_k \in A-\{s_p,s_q\}$ such that
$Cone((s_k,\theta_k),g)$ intersects $\overline{v_1v_3}$ at point $u_1,u_2$, where $u_1 \leq v_1$ and $u_2 \geq v_3$ as shown in Fig~\ref{fig:campq} (b); 
Since $Cone((s_k,\theta_k),g)$ must contain target $g$, $u_1 = v_1$. 
We know that $u_1,u_2$ are on the vertical line passing through $g$, we can formulate $\overline{u_1v_2}$ using the law of sine of the triangle $\triangle(s_ku_1u_2)$.
\begin{align*}
\frac{\overline{u_1u_2}}{\sin(2\alpha)} &= \frac{h/\sin(\theta_k-\alpha)}{\sin(\pi/2-\theta_k-\alpha)} \\
\overline{u_1u_2} &= \frac{2h\sin(2\alpha)}{\sin(2\theta_k)-\sin(2\alpha)}
\end{align*}
Since $u_2 \geq v_3$, we want to find the minimum $\overline{u_1u_2}$ by choosing different $s_k \neq s_p,s_q$, which is equivalent to minimizing $\overline{u_1u_2}$ w.r.t. $\theta_k$. 
Thus, $\overline{u_1u_2}$ is minimized when $\sin(2\theta_k) = 1$, which results in $\theta_k = \pi/4$.
By substituting $\theta_k=\pi/4$, $\overline{v_1v_k} = \frac{2h\sin(2\alpha)}{1-\sin(2\alpha)} = diag_1$. It means that either $s_k = s_q$ or $\overline{u_1u_2} \geq \overline{v_1v_3}$, both of  which contradict with our assumption. 
%\vtxt{not really: length does not imply containment}
\end{proof}

\subsection{Proof of Lemma~\ref{lem:upperbound2camerasValue2d}}
\begin{proof}
Using small angle approximation, we get $\sin(\alpha) \approx \alpha$ and $\cos(\alpha)\approx 1$ and $\alpha^2 \approx 0$.
The angles are constrained such that $\theta_p, \theta_q \in [\pi/4-2\alpha, \pi/4]$.

\begin{align*}
diag_1 &= ||r_1^2+r_2^2-2\cdot r_1 \cdot r_2 \cdot \cos(\theta_p+\theta_q)||_2 \\
		&\approx ||2t^2\alpha^2 + 2t^2\alpha^2 - 4t^2\alpha^2 \cos(\theta_p+\theta_q)||_2 \\
		&= 2t\alpha||1-\cos(\theta_p+\theta_q)||_2
\end{align*}
$\max(diag_1) \leq 2t\alpha$ and $\min(diag_1) \geq \sqrt{1-4\alpha} \cdot 2t\alpha$

\begin{align*}
diag_2 &= ||r_1^2+r_4^2-2\cdot r_1 \cdot r_4 \cdot \cos(\pi - \theta_p-\theta_q + 2\alpha)||_2 \\
%		&\approx ||2t^2\alpha^2 + 2t^2\alpha^2 + 4t^2\alpha^2[\cos(\theta_p+\theta_q)-2\alpha\sin(\theta_p+\theta_q)]||_2 \\
		&\approx 2t\alpha||1+\cos(\theta_p+\theta_q)-2\alpha\sin(\theta_p+\theta_q)||_2
\end{align*}
$\max(diag_2) \leq \sqrt{1+2\alpha} \cdot 2t\alpha$ and $\min(diag_2) \geq \sqrt{1-4\alpha} \cdot 2t\alpha$.
Therefore,$diag_2 \leq \sqrt{\frac{1+2\alpha}{1-4\alpha}} diag_1$ and $1-4\alpha$ will not be negative since $\alpha$ must be less than $0.25$ to satisfy small angle approximation.
Given that $\varepsilon_2 = \max(diag1,diag2)$, we can conclude 
$$\varepsilon_2 \leq \frac{1+2\alpha}{1-4\alpha} \frac{2h\sin(2\alpha)}{1-\sin(2\alpha)}$$
\end{proof}

\subsection{Proof of Lemma~\ref{lem:grid2d1}}
\begin{proof}
We will add two more line segments $\overline{aa'}$ and $\overline{cc'}$ to generate a isosceles trapezoid $aa'cc'$ (Fig~\ref{fig:u3d2d}). When the angle $\angle{s_paa'} \geq \angle{s_pab}$, the diagonal $\overline{ac}$ will be the longest line segments in the trapezoid $aa'cc'$. Therefore, when $\angle{s_paa'} \geq \angle{s_pab}$, that is $\theta_p + \theta_q \geq \frac{\pi}{2} + \alpha$, is satisfied, $||diag_1|| > ||diag_2||$.
\end{proof}

\subsection{Proof of Lemma~\ref{lem:grid2d2}}
\begin{proof}
 First, when the inner half planes of both $Cone(s_p)$ and $Cone(s_q)$ intersect above $g$, it is clear that by moving the intersection down to $g$, $\theta_p + \theta_q$ is increased.
Now assume target $g$ is moving along the $x$ axis (Fig~\ref{fig:grid3dv2}) by some length $m$, where $m \leq \delta_d/2$. 
We can formulate $\theta_p + \theta_q$ as a function of $m$ and the distance between the cameras as 
\begin{align*}
f(m) = \theta_p + \theta_q = \tan^{-1}(\frac{h}{h/\tan(\pi/4-\alpha) - m}) + \\
 \tan^{-1}(\frac{h}{h/\tan(\pi/4-\alpha) + m}) + 2\alpha
\end{align*}
We can get the derivative $\frac{df(m)}{dm}$ as 
\begin{align*}
\frac{d}{dm} f(m) &= \{2m(2\cos(2\alpha)+2\cos(2\alpha)\sin(2\alpha)\} \cdot \\ &\{2m^2\sin(2\alpha)+2m^4\sin(2\alpha)+4m^2\sin^2(2\alpha) \\
&+m^4\sin^2(2\alpha)+m^4+4\}^{-1}\\
\end{align*}
Since $\frac{df(m)}{dm} \geq 0$, $\theta_p + \theta_q$ keeps increasing and is maximized  at target $g^* = g \pm \delta_d/2$.
%It is clear that the derivative is always positive. Therefore, $\theta_i + \theta_j$ is a monotonically increasing function with $m$, which concludes the proof. 
\end{proof}

\subsection{Proof of Theorem~\ref{thrm:grid3d1}}
\begin{proof}
The intersection length $x$ is obtained using the law of sines. 
\begin{align*}
\frac{x}{\sin(2\alpha)} &= \frac{h/\sin(\theta-\alpha)}{\sin(\frac{\pi}{2} - \theta - \alpha)}
		x &=\frac{2h\sin(2\alpha)}{\sin(2\theta)-\sin(2\alpha)}
\end{align*}
When the inner half-plane of $Cone(s_p)$ and $Cone(s_q)$ intersect $g \pm \delta_d/2$, $x$ is maximized.
We can now compute directly the worst case uncertainty when $\alpha \leq 0.1$ rad which gives the desired result.
\end{proof} 

\subsection{Proof of Lemma~\ref{lem:grid2dhori}}
\begin{figure}[h]
\centering
	\includegraphics[width=0.4\textwidth]{fig/allVariation.pdf}
	\caption{Variation in horizontal and vertical positions}
	\label{fig:variation}
\end{figure} 
First, we analyze the effects of horizontal variation $\lambda_h$. 

\begin{lemma}\label{lem:grid2dhori}
Let $s = (s_x,s_y)$ be a camera location in an optimal pair for target $g \in \overline{G}$. Let $\hat{s} = (s_x \pm \lambda_hh, s_y)$ obtained by perturbing $s$ in the horizontal direction. 
Let $\hat{x} = l \cap Cone(\hat{s})$ and $x = l \cap Cone(s)$. 
$$||\hat{x}|| \leq \frac{1}{1-\lambda_h} ||x||$$. 
\end{lemma}

\begin{proof}
From Lemma~\ref{lem:diag1max}, we can see that when sensor is at location $\hat{s} = (s_x + \lambda_hh,s_y)$, $||\hat{x}||$ is maximized. Therefore, $||\hat{x}|| \geq ||x||$. From Fig~\ref{fig:variation}, we can get the following relationship using similar triangles: 
$\frac{||x||}{b+c} = \frac{h}{\lambda_hh + a}$
 and $\frac{||\hat{x}||}{c} = \frac{h}{a+b}$. 
We can get the following result.
\begin{align*}
\frac{||\hat{x}||}{||x||} &= \frac{c(\lambda_hh+a)}{(a+b)(b+c)}  \leq \frac{\lambda_hh + a}{a+b}\\
& \leq \frac{h}{h-\lambda_hh} \leq \frac{1}{1-\lambda_h}
\end{align*}
\end{proof}

\subsection{Proof of Lemma~\ref{lem:grid2dvert}}

Then, we add vertical perturbation $\lambda_v h$ in between the viewing plane and the ground plane.
\begin{lemma}\label{lem:grid2dvert}
Let $s = (s_x,s_y)$ be a camera location in a optimal pair for target $g \in \overline{G}$. Let $\hat{s} = (s_x, s_y \pm \lambda_vh)$ obtained by perturbing $s$ in the vertical direction. 
Let $\hat{x} = l \cap Cone(\hat{s})$ and $x = l \cap Cone(s)$. 
$$||\hat{x}|| \leq (1+\lambda_v) ||x||$$
\end{lemma}

\begin{proof}
From Lemma~\ref{lem:diag1max}, we can see that when sensor is at location $\hat{s} = (s_x ,s_y+ \lambda_vh)$, $||\hat{x}||$ is maximized. Therefore, $||\hat{x}|| \geq ||x||$. From Fig~\ref{fig:variation}, we can get the following relationship using similar triangles:
$\frac{||x||}{b+c} = \frac{h}{a}$ and 
$\frac{||\hat{x}||}{c} = \frac{h + \lambda_vh}{a+b}$. 
We can get the following result.
\begin{align*}
\frac{||\hat{x}||}{||x||} &= \frac{ac(1+\lambda_v)}{(a+b)(b+c)}  \leq 1+\lambda_v
\end{align*}
\end{proof}

\section{Derivations}
\subsection{Wedge Intersection}
\label{sec:wedgeIntersection}
Using the law of sines over the triangle $s_pv_1v_2$, we get $\frac{r_1}{\sin(2\alpha)} = \frac{\overline{s_pv_1}}{\sin{\angle{s_pv_2s_q}}}$. We also have  $\angle{s_pv_2s_q} = \pi - 2\alpha-\angle{s_pv_1v_2} = \pi - 2\alpha - (\theta_p + \theta_q - 2\alpha) = \pi-\theta_p - \theta_q$. From  
$\triangle(s_pv_1s_q)$, we know that $\frac{\overline{s_pv_1}}{\sin(\theta_q-\alpha)} = \frac{t}{sin(\pi-\theta_p-\theta_q+2\alpha)}$. By combining both equations, we obtain:
$
r_1 = \frac{t\sin(\theta_q-\alpha)\sin(2\alpha)}{\sin(\theta_p+\theta_q-2\alpha)\sin(\theta_p+\theta_q)}
$
Using the same method, we have:
$
r_2 = \frac{t\sin(\theta_p+\alpha)\sin(2\alpha)}{\sin(\theta_p+\theta_q)\sin(\theta_p+\theta_q+2\alpha)}
$
From $\triangle(s_pv_3v_4)$, we get:
$
r_3 = \frac{t\sin(\theta_q+\alpha)\sin(2\alpha)}{\sin(\theta_p+\theta_q)\sin(\theta_p+\theta_q+2\alpha)}
$
Similarly, from $\triangle(s_qv_1v_4)$
$
r_4 =\frac{t\sin(\theta_p-\alpha)\sin(2\alpha)}{\sin(\theta_p+\theta_q-2\alpha)\sin(\theta_p+\theta_2)}
$

