\section{Two Sensors Bound}
In order to achieve $\varepsilon_\infty$ in 3-D, the number of cameras required is unbounded. 
In this section, we explore the uncertainty $\varepsilon_{pq}$ using only two sensors $s_p$ and $s_q$, which achieves $\varepsilon_{pq} \leq 1.5 \varepsilon_\infty$.

\subsection{Two Dimensions}
We start with the planar case. The cameras and target will appear on the lines $g \in G$ and $s \in S$ respectively and the distance between $G$ and $S$ remains to be $h$ as the height.

We set the coordinate system on target $g = [0,0]$ with $x$-axis being parallel to $G$. Let $s_p = [-h,h]$ and $s_q = [h,h]$ be two cameras and their angles are $\theta_p$,$\theta_q$ respectively. The vertices of the convex polygon $Q'$ are $\{v_1,v_2,v_3,v_4\}$. 

\begin{lemma}\label{lem:angle2d}
$\overline{t_1t_3} > \overline{t_2t_4}$ when $\theta_1 + \theta_2 - 2\alpha \geqslant \pi/2$ as shown in Fig~\ref{fig:acbd}. 
\end{lemma}

\begin{proof}
As shown in Fig~\ref{fig:acbd}, if $\angle eab \geqslant \pi/2$, then we can create $\overline{ab'} \perp \overline{ex}$ and $\overline{cd'} \perp \overline{fx}$. From the isosceles trapzoid $ab'cd'$, it is clear that $\overline{ac} > \overline{bd}$ since $\overline{ac}$ is the diagonal line. We conclude the proof by showing that $\angle eab = \theta_1+\theta_2-2\alpha$. 
\end{proof}
We will be focusing on the cases when $\theta_1 + \theta_2 -2\alpha \geqslant \pi/2$.
In practice, most of the reliable and trackable image features appear within $\pi/2$ of the camera's field of view while assuming that the camera do not rotate intentionally for any specific target point. Thus, the same target point will be successfully tracked when $\theta_{1,2} > \pi/4$, which leads to $\theta_1+ \theta_2-2\alpha \geqslant \pi/2$ the majority sensing configuration.     
Therefore, we define $u_\alpha(x,\{\theta_1, \theta_2, h \}) = min(\overline{ac})$. 

\begin{figure}[h]
\centering
	\includegraphics[width=0.7\columnwidth]{fig/acne.png}
	\caption{uncertainty model in 2D}
	\label{fig:acbd}
\end{figure} 

%Before analyzing the worse case, a closed form solution is omitted here. We define $u_\alpha(x,\{\theta_1,\theta_2,d\}) = u_\alpha(x,\{e,f\})$ as a function of $\alpha$, $\theta_{1,2}$, and $d$, where $d$ is the distance between the sensor $e$ and $f$.
%Using the law of sine by looking at the triangle $eba$, we can find that $\frac{\overline{ab}}{sin(2\alpha)} = \frac{\overline{ef}}{sin{\angle{ebf}}}$. We can also find out that $\angle{ebf} = \pi - 2\alpha-\angle{eab} = \pi - 2\alpha - (\theta_1 + \theta_2 - 2\alpha) = \pi-\theta_1 - \theta_2$. By looking at triangle 
%$abf$, we know that $\frac{\overline{ea}}{sin(\theta_2-\alpha)} = \frac{d}{sin(\pi-\theta_1-\theta_2+2\alpha)}$. By combining both equations, we can get the following:
%\begin{equation}\label{eq:A}
%l_1 = \overline{ab} = d\frac{sin(\theta_2-\alpha)sin(2\alpha)}{sin(\theta_1+\theta_2-2\alpha)sin(\theta_1+\theta_2)}
%\end{equation}
%Using the same method, we have:
%\begin{equation}\label{eq:B}
%l_2 = \overline{bc} = d\frac{sin(\theta_1+\alpha)sin(2\alpha)}{sin(\theta_1+\theta_2)sin(\theta_1+\theta_2+2\alpha)}
%\end{equation}
%Using and the law of cosine, the uncertainty diameter $u_\alpha(x,\{\theta_1,\theta_2,d\})$ can be calculated as 
%\begin{equation}\label{eq:C}
%u_\alpha(x,\{\theta_1,\theta_2,d\}) = \sqrt{l_1^2+l_2^2-2l_1l_2 cos(\theta_1+\theta_2)}
%\end{equation}
%
%Given Eq~\ref{eq:C}, we can analyze $max(u_\alpha(x,\{\theta_1,\theta_2,d\})$ for fixed sensors and target locations and varying $\theta_{1,2}$.




\subsection{Three Dimensions}

\begin{lemma}
$U_\alpha(x,\{v_1,v_2\}) = u_\alpha(x,\{v_1,v_2\})$
\end{lemma}
 We claim the maximum uncertainty diameter by intersecting two cones originated from $e,f$ is the same as the maximum uncertainty diameter by intersecting two wedges originated from $e,f$ in 2D. 
\begin{proof}
As shown in Fig~\ref{fig:u3d2d}, when we take the isosceles trapezoid $ab'cd'$ and create a truncated cone by rotating around the line $\overline{ex}$, the line $\overline{ac}$ remains the maximum line segments within.  This is because both $\overline{ab'}$ and $\overline{cd'}$ are the diameter created by rotation around $\overline{ex}$. We will call the actual polyhedra created by the two cones intersection as $K$. It is clear that $K$ is at least within one of the uncertainty cones. If other wise, $K$ will be truncated so that it is contained within both cones. Therefore, if we create a plane cut for cone $ecd'$ using two planes perpendicular to the line $\overline{ex}$ at $\overline{ab'},\overline{cd'}$, the truncated cone will include $K$. Thus, $U_\alpha(x,\{e,f\})  = \overline{ac}= u_\alpha(x,\{e,f\})$.
\end{proof}

\begin{figure}[h]
\centering
	\includegraphics[width=0.7\columnwidth]{fig/u3d2d.png}
	\caption{Two Cones intersection uncertainty in 3D}
	\label{fig:u3d2d}
\end{figure} 