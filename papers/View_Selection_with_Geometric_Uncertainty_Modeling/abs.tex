\begin{abstract}
  Estimating positions of world points from features observed in
  images is a key problem  in 3D reconstruction, image mosaicking,
  simultaneous localization and mapping and structure from motion. We
  consider a special instance in which there is a dominant ground plane $\mathcal{G}$ viewed from a parallel viewing plane $\mathcal{S}$ above it.
  Such instances commonly arise, for example, in aerial
  photography.
  
  Consider a world point $g \in \mathcal{G}$ and its worst case
  reconstruction uncertainty $\varepsilon(g,\mathcal{S})$ obtained by
  merging \emph{all} possible views of $g$ chosen from $\mathcal{S}$.  We
  first show that one can pick two views $s_p$ and $s_q$ such
  that the uncertainty $\varepsilon(g,\{s_p,s_q\})$ obtained using
  only these two views is almost as good as (i.e. within a small constant factor of)
  $\varepsilon(g,\mathcal{S})$. Next, we extend the result to the
  entire ground plane $\mathcal{G}$ and show that one can pick a small
  subset of $\mathcal{S'} \subseteq \mathcal{S}$ (which grows only
  linearly with the area of $\mathcal{G}$) and still obtain a constant
  factor approximation, for every point $g \in \mathcal{G}$, to the
  minimum worst case estimate obtained by merging all views in
  $\mathcal{S}$. 
  Finally, we present a multi-resolution view selection method which extends our techniques to non-planar scenes.
 We show that the method can
    produce rich and accurate dense reconstructions with a small number of views.

  Our results provide a view selection mechanism with provable
  performance guarantees which can drastically increase the speed of
  scene reconstruction algorithms. In addition to theoretical results,
  we demonstrate their effectiveness in an application where aerial
  imagery is used for monitoring farms and orchards.
\end{abstract}
