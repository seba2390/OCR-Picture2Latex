\section{Conclusion}

In this paper, we studied view selection for a specific but common
setting where a ground plane is viewed from above from a parallel
viewing plane.  We showed that for a given world point, two views can be
chosen so as to guarantee a reconstruction quality which is almost as
good as one that can be obtained by using all possible views. Next, by
fixing these two views and studying perturbations of the world point,
we showed that one can put a coarse grid on the viewing plane and
ensure good reconstructions everywhere. Even though the reconstruction
quality can be improved by increasing the grid resolution, we showed
that a grid resolution proportional to the scene depth suffices to
guarantee a constant factor deviation from the optimal reconstruction.
We then showed how to extend the bound in the presence of perturbations of the viewing or scene planes.
However, as the scene geometry gets more sophisticated, occlusions
must be addressed.  For this purpose, we presented a multi-resolution view selection mechanism.
We also presented an application of these results to image mosaicking and scene reconstruction 
from (low altitude) aerial imagery.  



Our results provide a foundation for multiple
avenues of future research.  An immediate extension is for scenes
which can be represented as surfaces composed of multiple planes.
Giving guarantees in the presence of occlusions 
 raises ``art gallery'' type research
problems~\cite{o1987art}.  Furthermore, rather than selecting views
apriori and in one shot, the view selection can be informed by the
reconstruction process as is commonly done in existing literature. Our
multi-resolution view selection method provides the starting point for a batch  scheme where a
coarse grid is used for reconstruction under the planar scene assumption
and further refined based on the intermediate reconstruction.

