\section{Problem Definition}
 In this section, we introduce the general sensor selection problem.
%We start with a basic version where the goal is to choose a small number 
Consider the world point $g \in \mathcal{G}$ and a camera $(s, \theta)$ where $s \in \mathbb{R}^3$ is the projection center and $\theta \in SO(3)$ is the  orientation.
%and a set of camera centers $\mathcal{S} = \{s_1,s_2,...,s_k\}$. Each camera $s_i \in \mathcal{S}$ has orientation $\theta_i \in SO(3)$. 
Suppose we have a set of measurements $\{p_1, \ldots, p_k\}$ where each $p_i$ is expressed as a unit vector pointing towards the observed pixel and anchored at the corresponding camera center.
%of the  unobserved true projection of $g$ onto camera $(s,\theta)$. 
We need a function $f(p_1,p_2,...,p_k) = \hat{g}$ that maps measurements to $\hat{g}$, the estimate of $g$. This way, we can define the estimation error to be $||g - \hat{g}||$ by choosing an error measure $||\cdot||$.

 \begin{figure}[h]
 \centering
 	\includegraphics[width=0.4\columnwidth]{fig/camModel.pdf}
 	\caption{Right circular cone for camera $(s,\theta)$ viewing target $g$}
 	\label{fig:cone}
 \end{figure}

In this paper, we will consider the following ``bounded uncertainty" characterization of the error:
 Consider the true measurement $p^* = Proj((s,\theta),g)$ given by the projection of $g$ onto camera $(s,\theta)$ which is also represented as a vector from $s$ pointing toward $g$. 
We make the assumption that the angle between the measurement $p$ and the true projection $p^*$ is bounded by a fixed threshold $\alpha$.
%In reality, we do not obtain $p^*$. 
%Instead, we obtained a corrupted measurement $p$ and $p^*$ is assumed to lie inside a bounded region around $p$. Geometrically, this expressed as a bound $\alpha$ on the angle between $p$ and $p^*$. 
For a given measurement $p$, the rays corresponding to all  possible $p^*$ formulate to a cone denoted as $Cone_\alpha((s,\theta),p)$ as shown in Fig~\ref{fig:cone}, which is a function of  both the camera parameters $s$ and $g$ as well as  the measurement $p$.
%$s,p,g,\alpha$, where $\alpha$ is the cone angle as shown in Fig~\ref{fig:cone}.
For the rest of the paper, we will assume a fixed $\alpha$ and drop the subscript.
%Let $Cone((s,\theta),g) = Cone_\alpha((s,\theta),g)$ for some fixed $\alpha$.  
%The estimation error then corresponds to $\varepsilon(g) = ||g-\hat{g}|| = 
By intersecting the cones from multiple measurements $p_i$ from views $(s_i, \theta_i)$, we can get an estimate of the true target location.
The uncertainty is given by  the diameter of the intersection given by
$||\cap Cone((s_i,\theta_i),p_i)||$.

%Let $M = \{p_1,p_2,...,p_k\}$ be a set of measurements obtained by projecting target $g$ such that $p_i = Proj((s_i,\theta_i),g)$.  Now we can formulate the first problem as to $\min|M'|$ for $M' \subseteq M$
%\begin{equation}\label{eq:singlexInitF}
%||g - f(M')|| \leq \rho ||g - f(M)||
%\end{equation}
%where $\rho \in \mathbb{R}$.
%
%The first problem mainly focuses on a single point $g$. For a set of points $g \in \mathcal{G}$, we can formulate the second problem as to $\min|M'|$ for $M' \subseteq M$ such that 
%
%\begin{equation}\label{eq:allxInitF}
%\max_{g \in \mathcal{G}}||g - f(M')|| \leq \phi \max_{g \in \mathcal{G}}||g - f(M)||
%\end{equation}
%where $\phi \in \mathbb{R}$. 
%
%

%In this paper, we will study the selection of good camera locations. 


For sensor selection purposes, rather than  a single cone, it is beneficial to associate a set of cones for each measurement.
This will allow us to replace the randomness in the measurement process with a deterministic {\em worst-case analysis.}
%In order to remove the dependency on the orientation $\theta$, we define the {\em worst-case estimation error} as the orientation varies. 
To do this, for a given true  target location $g$ and a camera pose $(s, \theta)$, we generate $p^* = Proj((s,\theta),g)$. Then for every possible measurement $p$ within  angle $\alpha$ of $p^*$, we define $Cone((s,\theta),p)$ and include it with the set $S(g, s, \theta)$ associated with this world point/camera pair. Note that each cone in the set includes the true location $g$.
We can further eliminate the dependency on camera orientation by taking the union of these sets for each allowable orientation.
That is, we define $S(g, s) = \bigcup_\theta S(g, s, \theta)$  with the additional requirement that $g \in Cone((s,\theta),p)$ for each cone included in the union.

%we generate all possible measurement cones  $\mathcal{C}_\alpha(s,g) = \{Cone((s,\theta),p^*):  \theta \in SO(3), g \in Cone((s,\theta),p^*) \}$ where $p^* = Proj((s,\theta),g)$ is the true projection.
%For the remainder of the paper we will use a shorthand notation $Cone((s,\theta),g) = Cone((s,\theta), Proj((s,\theta),g))$ which is a cone apexed at $s$, apex angle $2 \alpha$ and orientation \vtxt{need to discuss this -- also must include $g$}

We can now define the worst case uncertainty for a given set $\mathcal{S} = \{s_1,s_2,...,s_k\}$  of camera centers and a ground point $g$ as:
% with corresponding measurements $M = \{p_1,p_2,...,p_k\}$:

%\begin{equation}
$$\varepsilon(g, \mathcal{S}) = \max_{Cone_1 \in S(g,s_1), \ldots, Cone_k \in S(g, s_k)} ||\cap Cone_i ||$$

%\end{equation}
%where the domain of each $\theta_i$ is defined such that  $Cone((s_i,\theta_i),g) \in \mathcal{C}(s_i,g)$.
In other words, for each camera location $s_i$, a cone is chosen such that the chosen cones \emph{jointly} maximize the intersection diameter.
The advantage of this formulation is that since  the computation of $\varepsilon(g, \mathcal{S})$ implicitly generates all possible measurements for a given camera location and world point, it generates a worst case uncertainty independent of specific measurements and camera rotations.
%which finds the maximum diagonal in the intersection region as the uncertainty measurement. 
We are now ready to define the first problem.

\begin{problem}
For a given world point $g$, the set of all possible viewpoints $\mathcal{S}$, a projection error bound $\alpha$, 
and an error tolerance parameter $\rho \in \mathbb{R}$,
choose a minimum cardinality subset $\mathcal{S'} \subseteq \mathcal{S}$, such that
%$\min|\mathcal{S'}|$ for $\mathcal{S'} \subseteq \mathcal{S}$ such that 
$$\varepsilon(g,\mathcal{S'}) \leq \rho \varepsilon(g,\mathcal{S})$$
%where $\rho \in \mathbb{R}$.
\label{prob:singlep}
\end{problem}

In Problem~\ref{prob:singlep} the goal is to choose a small subset of camera locations whose worst case uncertainty when reconstructing a given point $g$ is at most with a factor $\rho$ of the worst-case uncertainty of the entire viewing set. Problem~\ref{prob:multip} generalizes it to multiple points.

%
%\begin{problem}
%For a set of points $g \in \mathcal{G}$, the problem becomes $\min|\mathcal{S'}|$ for $\mathcal{S'} \subseteq \mathcal{S}$ such that 
%$$\max_{g \in \mathcal{G}} \varepsilon(g,\mathcal{S'}) \leq \phi \max_{g \in \mathcal{G}} \varepsilon(g,\mathcal{S})$$
%where $\phi \in \mathbb{R}$.
%\label{prob:multip}
%\end{problem}

\begin{problem}
For a set of points $G \subseteq \mathcal{G}$, 
the set of all possible viewpoints $\mathcal{S}$, a projection error bound $\alpha$, 
and an error tolerance parameter $\phi \in \mathbb{R}$,
choose a minimum cardinality subset $\mathcal{S'} \subseteq \mathcal{S}$, such that
%the problem becomes $\min|\mathcal{S'}|$ for $\mathcal{S'} \subseteq \mathcal{S}$ such that 
$$\max_{g \in {G}} \varepsilon(g,\mathcal{S'}) \leq \phi \max_{g \in \mathcal{G}} \varepsilon(g,\mathcal{S})$$
%where $\phi \in \mathbb{R}$.
\label{prob:multip}
\end{problem}

In this paper, we study a specific geometric instance of these problems where  $\mathcal{G}$ and $\mathcal{S}$ are two parallel planes with distance $h$ apart. 
For a given $g \in \mathcal{G}$, we will define $\varepsilon_{\infty}(g) = \varepsilon(g,\mathcal{S})$.
%\vtxt{We will show that ...}
%
%The number of cameras in $\mathcal{S}$ are unbounded. Therefore $\varepsilon(g,\mathcal{S}) = \max_{g \in \mathcal{G}} \varepsilon(g, \mathcal{S})$. The number of cameras in $\mathcal{S}$ is also unbounded, thus we define $\varepsilon_{\infty}(g) = \varepsilon(g,\mathcal{S})$ as the minimum worst case uncertainty. 
%
%The main contribution of this paper is that for Problem~\ref{prob:singlep}, we can find two sensors $s_i,s_j$ for a single target $g$ such that 
%$$
%\varepsilon(g,\{s_i,s_j\}) \leq \sqrt{\frac{1+2\alpha}{1-4\alpha}} \cdot \varepsilon_{\infty}
%$$
%The second contribution is that for Problem~\ref{prob:multip}, we propose a camera grid for a region $\mathcal{G}$ with area $Area(\mathcal{G})$