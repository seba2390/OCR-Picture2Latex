\section{Sensor Selection for a Single Point}
In this section, we study Problem 1 where the goal is to choose cameras to reconstruct a single point.
We will start with the two dimensional (2D) case where the ground and viewing planes reduce to lines, and the uncertainty cones become wedges.

Our key result in this section is that for any point $g$, one can choose two cameras whose worst case uncertainty $\varepsilon_{2}(g)$ is almost as good as $\varepsilon_{\infty}(g)$ , which is the worst case uncertainty obtained by merging the views from {\em all} cameras. The key ideas in obtaining this result are: (1)~if we choose two cameras at locations $p$ and $q$ who view $g$ symmetrically at 90 degrees (i.e. $\angle p g q = \pi/2$), the diagonals of the worst-case uncertainty polygon (the intersection of the two wedges) are roughly of equal length. (2)~Any other camera added to the sensor set can be rotated to contain the horizontal diagonal. Therefore, it does not reduce the uncertainty drastically. 
%\vtxt{say something about the model: spherical cameras rotation and measurement uncertainty folded in a single parameter $\alpha$ and justify single plane}
%Later, in Section~\ref{xxx},  we generalize the results to the three dimensional (3D) case.



%
%will analyze the minimum worst case uncertainty $\varepsilon_{\infty}(g)$ of target $g \in \mathcal{G}$ starting from 2-D and extend the result to 3-D. 
%
%In planar case, we first choose two cameras $s_p$ and $s_q$ that creates a convex polygon $Q_{pq}$ by intersecting their cones viewing at a target $g$ with worst case uncertainty $\varepsilon_2$. A closed-form solution for all the edges and diagonals of $Q_{pq}$ is omitted. We then show that the vertical diagonal is maximized for some orientation of the cameras. 
%Later, we prove that any camera in in $\mathcal{G}$ will contain the vertical diagonal when maximized over the orientation, which results in the lower bound for $\varepsilon_\infty$. Finally, the upper bound for $\varepsilon_\infty$ is also derived. Therefore, we show that using only two cameras, the worst case uncertainty can be bounded such that $\varepsilon_2 \leq \sqrt{\frac{1+2\alpha}{1-4\alpha}} \cdot \varepsilon_\infty$. 

\subsection{The Solution of Problem 1 in 2D}
%In the 2D planar case, the ground plane G and the view plane S reduce to two parallel lines with distance $h$.
%the cameras and target will appear on the lines $g \in G$ and $s \in S$ respectively and the distance between $G$ and $S$ remains to be $h$ as the height. The number of cameras in $S$ is unbounded.  
%The minimum worst case uncertainty in 2-D is given as

%\begin{figure}[h]
%\centering
%	\includegraphics[width=0.9\columnwidth]{fig/cone2d.jpeg}
%	\caption{Uncertainty cone in 2-D}
%	\label{fig:cone2d}
%\end{figure} 

%\begin{equation}\label{eq:opt2d}
%\varepsilon_{\infty}(g) = \max_{Cone_i \in S(s_i,g), \forall s \in S} ||\cap Cone_i||
%\end{equation}
%where the $Cone(s,g)$ is a wedge in 2-D with angle $\alpha$ and apex at $s$ as shown in Fig~\ref{fig:cone2d}. 
%The set of cones that can contains target $g$ is then $C(s,g) = \{ Cone((s,\theta),g): g \in Cone((s,\theta),g), \forall \theta \in SO(3) \}
%$.
Let $A = \arg \max(\varepsilon_\infty(g))$ be the set of wedges which yield the minimum worst case uncertainty. 
For every point $c$ on the viewing plane, there is a wedge in $A$ which (i)~ is apexed at $c$, (ii) has wedge angle $\alpha$ and (iii)~contains $g$.
By definition of $\varepsilon_\infty(g)$, the wedges are rotated so as to maximize the diagonal of the intersection.
% where orientation $\theta$ has been fixed for each camera.

%\vtxt{is this right? Check consistent clockwise-counterclockwise}.
%We use the angle $\theta$ between the bisector of a wedge with respect to $S$ for the edge orientation.
%Of the two half-planes whose intersection yields the edge, the inner half plane is the one that is closer to $S$ -- i.e. the angle measured is smaller \vtxt{????}.
%
%We use $\theta$ to denote the orientation of a wedges defined as the smaller angle between $S$ and center line of the wedge such that $\theta \leq \frac{\pi}{2}$ (Figure~\ref{xxx})).
%A wedge is the intersection of two half-planes: 
%we define the inner half-plane to be the half-plane that creates a smaller angle $(\theta-\alpha)$ w.r.t. $S$. The outer half-plane is the other half-plane with a larger angle w.r.t. $S$. 
%\vtxt{need to phrase these a bit better after the figure is in -- you may just want to mark them on the figure}

%\begin{theorem}\label{thrm:opt2d}
%Given a target $g \in G$ and a set of cameras $s \in S$, where the distance between $G$ and $S$ is $h$ and the number of cameras in $S$ is unbounded, we claim the minimum worst case uncertainty denoted as $\varepsilon_\infty = \varepsilon(g,S)$ is
%\begin{equation}\label{eq:opt2cameras2d}
%\frac{2h\sin(2\alpha)}{1-\sin{2\alpha}} \leq \varepsilon_\infty \leq \sqrt{\frac{1+2\alpha}{1-4\alpha}}\frac{2h\sin(2\alpha)}{1-\sin{2\alpha}}
%%\varepsilon_\infty \leq \sqrt{\frac{1+2\alpha}{1-4\alpha}}\frac{4h\alpha}{1-2\alpha}
%\end{equation}
%\end{theorem}

\begin{theorem}\label{thrm:opt2cameras2dbound}
Consider a target $g$ on line $G$ and viewing set $S$ composed of all camera locations on $S$ parallel to $G$.
%Given a target $g \in G$ and a viewing set  $S$, such that the distance between $G$ and $S$ is $h$ and the number of cameras in $S$ is unbounded, we claim that 
There exist two cameras $s_p$ and $s_q$ which guarantee that
\begin{equation}\label{eq:opt2cameras2dbound}
\varepsilon_2 \leq \sqrt{\frac{1+2\alpha}{1-4\alpha}}  \varepsilon_\infty
\end{equation}
where $\varepsilon_\infty = \varepsilon(g,S)$  is the minimum worst case uncertainty of the entire viewing set, and 
$\varepsilon_2$ is the worst case uncertainty of  $\{s_p, s_q \}$ and $0 \leq \alpha < 1/4$ is the error threshold measured in radians.
%
%is $\varepsilon_\infty = \varepsilon(g,S)$ and worst case uncertainty from two cameras $s_p$ and $s_q$ is $\varepsilon_2$.
\end{theorem}

%In reality, the wedge angle $\alpha$ is created by pixel resolution and calibration error. Both error are normally less than $1$ pixel (sub-pixel error). Therefore, the maixmum angular error for a camera with $1080 \times 1920$ and field of view $70^\circ \times 120^\circ$ respectively. If we assume the pixel error is upper bounded by $10$ pixels, then $\alpha \leq \max(10/1920*120^\circ *\pi/180,10/1080*70^\circ *\pi/180) = 0.0113$ rad. Therefore, in the following text, we will use small angle approximation for angle $\alpha$, where $\sin(\alpha) \approx \alpha$ and $\cos(\alpha) \approx 1$.

%We will prove Theorem~\ref{thrm:opt2cameras2dbound} in the following steps. First, we find two cameras $s_p,s_q \in A$ and derive the closed-form solution for the uncertainty $\varepsilon_2$. 
%Next, we show minimum worst case uncertainty $\varepsilon_\infty$ is lower bounded by $\varepsilon_2$.

\begin{figure}[h]
\centering
	\includegraphics[width=0.5\textwidth]{fig/campq.pdf}
	\caption{(a) Notation for the two camera selection $s_p$ and $s_q$ (b) If the cone created by $s_k$ that does not contain $diag_1$, we get a contradiction (proof of Lemma~\ref{lem:diag1max})}
	\label{fig:campq}
\end{figure} 

We will prove the theorem directly by providing the two cameras, computing their worst-case uncertainty $\varepsilon_2$ and comparing it with the minimum possible worst-case uncertainty. 
First, we present the notation and the setup used in the computations.
%The following notations for camera positions, orientations and intersection polygon will be used throughout the paper. 
We set a coordinate system whose origin is at the target $g$. The $x$-axis is on $G$ and the $z$-axis points ``up" toward the viewing plane.
The locations of the two cameras are chosen as:
 $s_p = [-t/2,h]$ and $s_q = [t/2,h]$ where $t = \frac{2h}{\tan(\pi/4-\alpha)}$ and the cone orientations $\theta_p, \theta_q$ respectively (Fig~\ref{fig:campq} (a)). We use the angle $\theta$ between the bisector of a wedge with respect to $S$ for orientation. Of the two half-planes whose intersection yields the wedge, the inner half plane is the one that is closer to $S$ -- i.e. the angle measured is smaller while the other half-plane is the outer half-plane also shown in Fig~\ref{fig:campq} (a). Note that $\theta_p,\theta_q \in [\pi/4-2\alpha,\pi/4]$.
%\vtxt{Figure}
 
 %Their orientations are $\theta_p$ and $\theta_q$ respectively such that $g$ is visible from both cameras. 
Their worst case uncertainty is given by
%We will show that their  worst case uncertainty denoted as $\varepsilon_2 = \varepsilon(g,\{s_p,s_q\})$ is: 
\begin{equation}\label{eq:uncertainty2cam2d}
\varepsilon_2 = \max_{\theta_p,\theta_q}||Cone((s_p,\theta_p),g)\cap Cone((s_q,\theta_q),g)||
\end{equation}
Consider the two wedges which give the worst case uncertainty (i.e. $\arg \max$ of $\varepsilon_2$). 
Let $Q_{pq}$ be their intersection with vertices $\{v_1,v_2,v_3,v_4\}$ and  edges  $\{e_1,e_2,e_3,e_4\}$ (Fig~\ref{fig:campq} (a)).
The lengths of the edges are denoted as $r_i = ||e_i||$ and the length of the diagonals are denoted by $diag_1 = ||\overline{v_1v_3}||,diag_2 = ||\overline{v_2v_4}||$.

We now compute these quantities.

\subsubsection{Computing $\varepsilon_2$}
In order to maximize over the orientation, we first establish the closed form solution for the edges and diagonals as functions of $h$,$t$,$\theta_{p,q}$,and $\alpha$.

Using the law of cosines, $diag_1$ can be calculated as 
\begin{equation}\label{eq:diag1}
diag_1^2 = r_1^2+r_2^2
	-2  r_1  r_2  \cos(\theta_p+\theta_q)
\end{equation}
Similarly, the $diag_2$ can be calculated as 
\begin{equation}\label{eq:diag2}
diag_2^2 = r_1^2+r_4^2
	-2 r_1  r_4  \cos(\pi - \theta_p-\theta_q + 2\alpha)
\end{equation}
The detailed derivation is shown in Appendix~\ref{sec:wedgeIntersection}. 

We now consider the vertical diagonal 
whose length $diag_1$ is given in Equation~\ref{eq:diag1}. % is monotone {\color{red} we did not prove this monotone} 
%in the two wedge angles $\theta_p, \theta_q \in [\pi/4-2\alpha, \pi/4]$ . 
It is maximized when $\theta_p = \theta_q = \pi/4$.
Fig~\ref{fig:diag1_maximum} shows $diag_1$ 
as a function of the two wedge angles $\theta_p$ and $\theta_q$ and for $\alpha \leq 0.1$ rad. 
When $\theta_p = \theta_q = \pi/4$, the vertex $v_1 = g$, which means that the inner half-planes of $Cone_p$ and $Cone_q$ intersect at $g$.

\begin{figure}[h]
\centering
	\includegraphics[width=0.3\textwidth]{fig/diag1_maximum.png}
	\caption{$diag_1$ length as a function of $\theta_p$ and $\theta_q$}
	\label{fig:diag1_maximum}
\end{figure} 

We can therefore set $\theta_p = \theta_q = \pi/4$ and write the equation of $diag_1$ as a function of 
$\alpha$ and $h$: Using the law of sines on the triangle $\triangle(s_qv_1v_3)$ and $\overline{v_1s_q} = h/\sin(\pi/4-\alpha)$, we obtain:
\begin{align*}
\frac{diag_1}{\sin(2\alpha)} &= \frac{\overline{v_1s_q}}{\sin(\frac{\pi}{2}-\theta-\alpha)} \\
diag_1 &= \frac{2h\sin(2\alpha)}{1-\sin(2\alpha)}
\end{align*}

This establishes the maximum length of the diagonal $diag_1 =\frac{2h\sin(2\alpha)}{1-\sin(2\alpha)}$ in the worst case configuration of $\theta_p=\theta_q = \pi/4$. 

We now compare $\varepsilon_2(s_p, s_q) = \max||Q_{pq}||$ with $\varepsilon_\infty$.

\begin{lemma}\label{lem:diag1max}
Consider the two cameras $s_p,s_q$ in the optimal configuration described above and let  $diag_1$
be the intersection of their worst-case uncertainty polygon $Q_{pq}$.
%the intersection polygon $Q_{pq}$ and the maximum length of the diagonal
%=\frac{2h\sin(2\alpha)}{1-\sin(2\alpha)}$ when $\theta_p=\theta_q = \pi/4$. 
Any cone starting from location $s_k \in A-\{s_p,s_q\}$, can be rotated to an angle $\theta_k$ such that both $g$ and 
$diag_1$ are contained in its uncertainty wedge $Cone((s_k,\theta_k),g)$.
% after maximizing over $\theta_k$.
\end{lemma}

%\begin{figure}[h]
%\centering
%	\includegraphics[width=0.3\textwidth]{fig/maxdiag1.png}
%	\caption{If the cone created by $s_k$ that does not contain $diag_1$, we get a contradiction (proof of Lemma~\ref{lem:diag1max}).}
%	\label{fig:diag1max}
%\end{figure} 



Now that we established that two cameras suffice, we compute the uncertainty value:
\begin{lemma}\label{lem:upperbound2camerasValue2d}
Given the two cameras $s_p,s_q$, the intersection polygon $Q_{pq}$, the maximum length of the diagonal $diag_1 =\frac{2h\sin(2\alpha)}{1-\sin(2\alpha)}$ when $\theta_p=\theta_q = \pi/4$, and the worst case uncertainty $\varepsilon_2 = \max||Q_{pq}||$.
%We claim that 
\begin{equation}\label{eq:upperbound2camerasValue2d}
\varepsilon_2 \leq \sqrt{\frac{1+2\alpha}{1-4\alpha}} \cdot \frac{2h\sin(2\alpha)}{1-\sin(2\alpha)}
\end{equation}
\end{lemma}


% \begin{proof}
% We now look at the convex polygon $Q_{pq}$ constructed by cameras $s_p$ and $s_q$ and target $g$. 
% We get $\frac{diag_2}{diag_1} \leq \sqrt{\frac{1+2\alpha}{1-4\alpha}}$ (details will be given  in Appendix~\ref{append:B} of the supplementary material), 

% Therefore ,
% \begin{align*}
% \varepsilon_2 &\leq \frac{diag_2}{diag_1} \cdot diag_1 \\
% 				&\leq \sqrt{\frac{1+2\alpha}{1-4\alpha}} \cdot \frac{2h\sin(2\alpha)}{1-\sin(2\alpha)} \\
% %				&\leq \sqrt{\frac{1+2\alpha}{1-4\alpha}} \cdot \frac{4h\alpha}{1-2\alpha} \\
% \end{align*}
% \end{proof}

Now we can conclude by presenting the proof of Theorem~\ref{thrm:opt2cameras2dbound}.

\begin{proof}
Combining Lemma~\ref{lem:diag1max} and Lemma~\ref{lem:upperbound2camerasValue2d}, we can conclude that $diag_1 \leq \varepsilon_\infty \leq diag_2$. Therefore, $\varepsilon_2 \leq \sqrt{\frac{1+2\alpha}{1-4\alpha}} \cdot \varepsilon_\infty$ 
\end{proof}

In this section, we showed that there exist two cameras $s_p$ and $s_q$ with orientation $\theta_p = \theta_q = \pi/4$ such that their worst case uncertainty $\varepsilon_2 \leq \sqrt{\frac{1+2\alpha}{1-4\alpha}} \cdot \varepsilon_\infty$.
We will call the pair of cameras $s_p,s_q$ as the \textbf{optimal pair} for the rest of the paper and this configuration as the \textbf{optimal configuration} of $\{s_p, s_q\}$.

%\begin{theorem}\label{thrm:opt2cameras2dbound}
%Given a target $g \in G$ and a set of cameras $s \in S$, where the distance between $G$ and $S$ is $h$ and the number of cameras in $S$ is unbounded, we claim that there exits two cameras $s_p$ and $s_q$ that
%\begin{equation}\label{eq:opt2cameras2dbound}
%\varepsilon_2 \leq \sqrt{\frac{1+2\alpha}{1-4\alpha}}\varepsilon_\infty
%\end{equation}
%where the minimum worst case uncertainty is $\varepsilon_\infty = \varepsilon(g,S)$ and worst case uncertainty from two cameras $s_p$ and $s_q$ is $\varepsilon_2$.
%\end{theorem}
%{\color{red} Might want to use this theorem instead}

%\vtxt{reviewed up to here}

\subsection{The Solution of Problem 1 in 3D}
%\subsection{Three Dimensions}
The results of the previous section readily extend to  $\varepsilon_\infty$ in 3-D.  

\begin{theorem}\label{thrm:opt2cameras3dbound}
Given a target $g \in \mathcal{G}$ and a set of cameras $s \in \mathcal{S}$, where the distance between $\mathcal{G}$ and $\mathcal{S}$ is $h$ and the number of cameras in $\mathcal{S}$ is unbounded, we claim that the optimal pair $s_p$ and $s_q$ gives
\begin{equation}\label{eq:opt2cameras3dbound}
\varepsilon_2 \leq \sqrt{\frac{1+2\alpha}{1-4\alpha}} \cdot \varepsilon_\infty
\end{equation}
where the minimum worst case uncertainty in 3-D is $\varepsilon_\infty = \varepsilon(g,\mathcal{S})$ and worst case uncertainty from two cameras $s_p$ and $s_q$ is $\varepsilon_2$.
\end{theorem}

\begin{figure}[h]
\centering
	\includegraphics[width=0.3\textwidth]{fig/u3d2d.png}
	\caption{Uncertainty in 3D given by two intersecting cones }
	\label{fig:u3d2d}
\end{figure} 

To prove the theorem, all we have to do is to observe that the diagonal of a perpendicular cross section of the cone bounds the uncertainty in 3D as well. See Fig~\ref{fig:u3d2d}. Therefore, we can apply Theorem~\ref{thrm:opt2cameras2dbound}.

% \begin{proof}
% From Theorem~\ref{thrm:opt2cameras2dbound}, we know that the relationship holds for 2-D. Since the planer case is created by cutting a perpendicular plane w.r.t. $\mathcal{G}$ and $\mathcal{S}$, we can claim that the Theorem~\ref{thrm:opt2cameras2dbound} is valid in the cutting plane with arbitrary orientation as long as it intersects $g$. 
% In 3D case, we can show that in the truncated cone $ab'cd'$ as shown in Fig~\ref{fig:u3d2d}, the maximum uncertainty does not surpass the diagonals. Therefore, we can use the same bound from Theorem~\ref{thrm:opt2cameras2dbound}, thus conclude the proof.
% \end{proof}
