%----------------------------------------------------------------------
%%% Conclusions
%----------------------------------------------------------------------
%!TEX root = ./HBAConicMain.tex
%

We derived Hessian-barrier algorithms based on first- and second-order information on the objective $f$. We performed a detailed analysis of their worst-case iteration complexity in order to find a suitably defined approximate KKT point. Under weak regularity assumptions and in presence of general conic constraints, our Hessian-barrier algorithms share the best known complexity rates in the literature for first- and second-order  approximate KKT points. Our methods are characterized by a decomposition approach of the feasible set which leads to numerically efficient subproblems at each their iteration. Several open questions for the future remain. First, our iterations assume that the subproblems are solved exactly, and for practical reasons this should be relaxed. Second, we mentioned that $\AHBA$ can be interpreted as a discretization of the Hessian-barrier gradient system \cite{ABB04}, but the exact relationship is not explored yet. This, however, could be an important step towards understanding acceleration techniques of $\AHBA$, akin to accelerated methods for the cubic regularized Newton method. Furthermore, the cubic-regularized version has no corresponding continuous-time version yet. It will be very interesting to investigate this question further. Additionally, the question of convergence of the trajectory $(x^{k})_{k\geq 0}$ generated by either scheme is open. Another interesting direction for future research would be to allow for  higher-order Taylor expansions in the subproblems in order to boost convergence speed further, similar to \cite{CarGouToi19}. 