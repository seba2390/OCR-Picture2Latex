%----------------------------------------------------------------------
%%% Prelims
%----------------------------------------------------------------------
% !TEX root = ./HBAConicMain.tex
%%%%%%%%%%%
%The purpose of this section and Appendix \ref{app:barrier} is to introduce all the concepts and notations we need in order to define and analyze our algorithms. Readers who are more interested in the algorithmic details can skip this section for the moment and continue directly with section \ref{sec:firstorder}, while referring back to this section whenever needed.
%
\subsection{Cones and their self-concordant barriers}
Let $\bar{\setK}\subset\setE$ be a \emph{regular cone}: $\bar{\setK}$ is closed convex, solid and pointed (i.e. contains no lines). 
We assume that $\setK:=\Int(\bar{\setK}) \neq \emptyset$, where $\Int(\bar{\setK})$ is the interior of $\bar{\setK}$. Any such cone admits a self-concordant logarithmically homogeneous barrier $h(x)$ with finite parameter value $\nu$ \cite{NesNem94}.

\begin{definition}
\label{def:LHSCB}
A function $h:\bar{\setK}\to(-\infty,\infty]$ with $\dom h=\setK$ is called a $\nu$-\emph{logarithmically homogeneous self-concordant barrier} ($\nu$-LHSCB) for the cone $\bar{\setK}$ if:
\begin{itemize}
\item[(a)] $h$ is a $\nu$-\emph{self-concordant barrier} for $\bar{\setK}$, i.e., for all $x \in \setK$ and $u\in \setE$
\begin{align}
\label{eq:SC}
&\abs{D^{3}h(x)[u,u,u]}\leq 2D^{2}h(x)[u,u]^{3/2},\text{ and } \\
&\sup_{u\in\setE}\abs{2 Dh(x)[u]-D^{2}h(x)[u,u]}\leq \nu.
\end{align}
\item[(b)] $h$ is \emph{logarithmically homogeneous:} 
\[
h(tx)=h(x)-\nu\ln(t)\qquad \forall x\in \setK,t>0.
\]
\end{itemize}
We denote the set of $\nu$-logarithmically homogeneous barriers by $\scrH_{\nu}(\setK)$.
\end{definition}
%Condition \eqref{eq:SC} is saying that $h$ is a standard \ac{SC} barrier on $\bar{\setK}$. 
Given $h\in\scrH_{\nu}(\setK)$, from \cite[Thm 5.1.3]{Nes18} we know that for any $\bar{x}\in\bd(\bar{\setK})$, any sequence $(x_{k})_{k \geq 0}$ with $x_{k}\in\ \setK$ and $\lim_{k\to\infty}x_{k}=\bar{x}$ satisfies $\lim_{k\to\infty}h(x_{k})=+\infty$. For a pointed cone $\bar{\setK}$, we have $\nu\geq 1$ and the Hessian $H(x)\eqdef\nabla^{2}h(x):\setE\to\setE^{\ast}$ is a positive definite linear operator defined by $\inner{H(x)u,v}\eqdef D^{2}h(x)[u,v]$ for all $u,v\in\setE$, see  \cite[Thm. 5.1.6]{Nes18}. The Hessian gives rise to a \emph{local norm} 
\begin{equation}\label{eq:localnorm}
(\forall x\in \setK)(\forall u\in\setE):\quad \norm{u}_{x}=\inner{H(x)u,u}^{1/2}.
\end{equation}
We also define a \emph{dual norm} on $\setE^{\ast}$ as 
\begin{equation}\label{eq:dualnorm}
(\forall x\in\setK)(\forall s\in\setE^{\ast}):\quad \norm{s}_{x}^{\ast}= \inner{[H(x)]^{-1}s,s}^{1/2}.
\end{equation}
%Note that 
%\begin{equation}\label{eq:norm_represent}
%\norm{u}_{x} = \norm{[H(x)]^{1/2} u}, \text{and }\norm{s}_{x}^{\ast}= \norm{[H(x)]^{-1/2} s}\quad \forall x\in \setK, u\in\setE, s\in\setE^{\ast}. 
%\end{equation}
The \emph{Dikin ellipsoid} is defined as the open set
$\scrW(x;r)\eqdef\{u\in\setE\vert\;\norm{u-x}_{x}<r\}, r>0.$ The usage of the local norm adapts the unit ball to the local geometry of the set $\setK$. Indeed, the following classical result is key to the development of our methods. 
\begin{lemma}[Theorem 5.1.5 \cite{Nes18}]
\label{lem:Dikin}
For all $x\in\setK$ we have $\scrW(x;1)\subseteq\setK$.
\end{lemma}
%The \emph{dual cone} $\bar{\setK}^{\ast}$ is defined as $\bar{\setK}^{\ast}\eqdef\{s\in\setE^{\ast}\vert\inner{s,x}\geq 0\;\forall x\in\bar{\setK}\}$, and the \emph{dual barrier} $h_{\ast}(s)\eqdef\sup_{x\in\setK}\{\inner{-s,x}-h(x)\}$ for $s\in\bar{\setK}^{\ast}$. From \cite[Thm 3.3.1]{Ren01} we know that if $h\in\scrH_{\nu}(\setK)$, then $h_{\ast}\in\scrH_{\nu}(\setK^{\ast})$. Moreover, 
%\begin{align}
%&x \in \setK \Rightarrow -\nabla h(x) \in \setK^{\ast},\label{eq:relations_1}\\
%&s=-\nabla h(x)\iff \nabla h_{\ast}(s)=-x\Rightarrow \nabla^{2}h_{\ast}(s)=[\nabla^{2}h(x)]^{-1}. \label{eq:relations}
%\end{align}
%We will also need the following properties listed in \cite[Lemma 5.4.3]{Nes18}.
%\begin{proposition}
%\label{prop:logSCB}
%Let $h\in\scrH_{\nu}(\setK)$, $x\in\setK$, $t>0$ and recall that $H(x)=\nabla^{2}h(x)$.
%Then,
%\begin{align}
%& \nabla^2 h(t x) = t^{-2} \nabla^2 h(x), \label{eq:log_hom_scb_hess_homog_prop}\\ 
%&-\inner{\nabla h(x),x} = \nu, \label{eq:log_hom_scb_hess_prop}\\
%&\norm{x}_{x}^2 = \inner{ H(x) x ,x } = \nu, \quad \inner{ \nabla h(x), [H(x)]^{-1}\nabla h(x) } = \nu. \label{eq:log_hom_scb_norm_prop}
%\end{align}
%%\begin{align}
%&H(x)x = -\nabla h(x), \quad H'(x)x= -2H(x),  \label{eq:log_hom_scb_hess_prop} \\
%& [H(x)]^{1/2}x = \text{const}, \label{eq:log_hom_scb_center_direction} \\
%&\inner{ H(x) x ,x } = \nu, \quad \inner{ \nabla h(x), [H(x)]^{-1}\nabla h(x) } = \nu. \label{eq:log_hom_scb_norm_prop}
%\end{align}
%\end{proposition}
%We remark that $-\inner{\nabla h(x),x} = \nu$ is an immediate consequence of \eqref{eq:log_hom_scb_hess_prop} and \eqref{eq:log_hom_scb_norm_prop}. 
%\PD{We need also the property that $-\inner{\nabla h(x),x} = \nu$. I also think that we can move this proposition to Section \ref{sec:firstorder}, closer to the proofs.}
\begin{proposition}[{Theorem 5.1.9 \cite{Nes18}}]
\label{prop:SCF_upper_bound}
Let $h\in\scrH_{\nu}(\setK)$, $x\in\dom h$, and a fixed direction $d \in \setE$. For all $t \in [0,\frac{1}{\norm{d}_x})$, with the convention that $\frac{1}{\norm{d}_{x}}=+\infty$ if $\norm{d}_x=0$, we have:
\begin{equation}
\label{eq:SCF_upper_bound} 
h(x + t d) \leq h(x) + t\inner{\nabla h(x),d} + t^2 \norm{d}_{x}^2 \omega(t \norm{d}_{x}),
\end{equation}
where $\omega(t)=\frac{-t-\ln(1-t)}{t^2}$.
\end{proposition}
\noindent
We will also use the following inequality for the function $\omega(t)$ \cite[Lemma 5.1.5]{Nes18}:
\begin{equation}\label{eq:omega_upper_bound}
\omega(t) \leq \frac{1}{2(1-t)}, \; t \in [0,1).
\end{equation}
We close this section with important examples of conic domains to which our method can be applied. 
\begin{example}[The exponential cone]
Consider the exponential cone studied by \cite{Chares:2009wb} defined as 
\begin{equation}
\setK_{\exp}=\{x\in\R^{3}\vert x_{1}\geq x_{2}e^{x_{3}/x_{2}},x_{2}>0\}
\end{equation}
with closure $\bar{\setK}_{\exp}=\cl(\setK_{\exp})$. This set admits a $3$-LHSB 
\[
h(x_{1},x_{2},x_{3})\eqdef -\ln(x_{2}\ln(x_{1}/x_{2})-x_{3})-\ln(x_{1})-\ln(x_{2})\in\scrH_{3}(\setK_{\exp}).
\]
We remark that this cone is not self-dual (cf. Definition \ref{def:SymCone}), but 
\[
G\bar{\setK}_{\exp}=\bar{\setK}^{\ast}_{\exp}=\cl\left(\{y\in\R^{3}\vert y_{1}\geq -y_{3}e^{y_{2}/y_{3}-1},y_{1}>0,y_{3}<0\}\right),
\]
under the linear transformation 
$
G=\left[\begin{array}{ccc} 
1/e & 0 & 0 \\
0 & 0 & -1 \\
0 & -1 & 0 
\end{array}\right].
$
There are many convex sets that can be represented using the exponential cone; We list some example below, but refer to the PhD thesis \cite{Chares:2009wb} for further details. 
\begin{itemize}
\item Exponential: $\{(t,u)\vert t\geq e^{u}\}\iff (t,1,u)\in\bar{\setK}_{\exp}$;
\item Logarithm: $\{(t,u)\vert t\leq \ln(u)\}\iff (u,1,t)\in\bar{\setK}_{\exp}$;
\item Entropy: $t\leq -u\ln(u)\iff t\leq u\ln(1/u)\iff (1,u,t)\in\bar{\setK}_{\exp}$;
\item Relative Entropy: $t\geq u\log(u/w)\iff (w,u,t)\in\bar{\setK}_{\exp}$;
\item Softplus function: $t\geq\ln(1+e^{u})\iff a+b\leq 1,(a,1,u-t)\in\bar{\setK}_{\exp},(b,1,-t)\in\bar{\setK}_{\exp}$.  
\end{itemize}
\close
\end{example}
%The next three examples show that our method can deal with the most important conic constraints in optimization. 
\begin{example}[Non-negativity constraints]
\label{Ex:NLP}
For $\setE=\Rn$ and $\bar{\setK}=\bar{\setK}_{\text{NN}}$, we define the log-barrier $h(x)=-\sum_{i=1}^{n}\ln(x_{i})$ for all $x\in\setK_{\text{NN}}=\Rn_{++}$. It is readily seen that $h\in\scrH_{n}(\setK)$. \close
\end{example}
%%%%%%%%%%%% SOC %%%%%%%%%%%%%%%%%55
\begin{example}[SOC constraints]
Let $\setE=\R^{n+1}$ and $\bar{\setK}=\bar{\setK}_{\text{SOC}}$ defined in Example \ref{ex:SOC}. For $x=(x_{0},\underline{x})\in\bar{\setK}_{\text{SOC}}$, we define the barrier $h(x)=-\ln(x_{0}^{2}-\underline{x}^{\top}\underline{x})$. It is well known that $h\in\scrH_{2}(\setK_{\text{SOC}})$ \cite{NesNem94}. 
\close
\end{example}
%%%%%%%%%%% SDP %%%%%%%%%%%%%%
\begin{example}[SDP constraints]
Let $\setE=\symm^{n}$ and $\bar{\setK}=\bar{\setK}_{\text{SDP}}$, defined in Example \ref{ex:SDP}. Consider the barrier $h(x)=-\ln\det(x)$. It is well known that $h\in\scrH_{n}(\setK_{\text{SDP}})$.
\close
\end{example}
%%%%%%%%%%%%%%%%%%%%%%%%%

\subsection{Exploiting the structure of Symmetric Cones}
%In the classical literature on interior-point methods, 
Nesterov and Todd \cite{NesTod97} introduced \emph{self-scaled barriers}, which later have been realized as LHSCB's for \emph{symmetric cones}. Such barriers are nowadays key to define primal-dual interior point methods for convex problems with potentially larger step sizes. Our method can also exploit the additional properties of self-scaled barriers, leading to potentially larger step sizes and faster convergence in our non-convex setting as well. For a given closed convex nonempty cone $\bar{\setK}$, its \emph{dual cone} is the closed convex and nonempty conce $\bar{\setK}^{\ast}$ defined as $\bar{\setK}^{\ast}\eqdef\{s\in\setE^{\ast}\vert\inner{s,x}\geq 0\;\forall x\in\bar{\setK}\}$. If $h\in\scrH_{\nu}(\setK)$, then the \emph{dual barrier} is defined $h_{\ast}(s)\eqdef\sup_{x\in\setK}\{\inner{-s,x}-h(x)\}$ for $s\in\bar{\setK}^{\ast}$.
 %%%%%%%
\begin{definition}\label{def:SymCone}
An open convex cone is said to be \emph{self-dual} if $\setK^{\ast}=\setK$. $\setK$ is homogeneous if for all $x,y\in\setK$ there exists a linear bijection $G:\setE\to\setE$ such that $Gx=y$ and $G\setK=\setK$. An open convex cone $\setK$ is called \emph{symmetric} if it is self-dual and homogeneous. 
\end{definition}
%%%%%%%%%%%%
The class of symmetric cones can be characterized within the language of Euclidean Jordan algebras \cite{FayLu06,Fay08,FarKor94,Schmieta2003}. For optimization, the three symmetric cones of most relevance are $\bar{\setK}_{\text{NN}},\bar{\setK}_{\text{SOC}}$ and $\bar{\setK}_{\text{SDP}}$. %On symmetric cones, \cite{NesTod97} defined a long-step primal-dual method which builds upon a subset of LHSCB's, defined as follows.
%%%%%SSB%%%%%%
\begin{definition}[\cite{NesTod97}]
$h\in\scrH_{\nu}(\setK)$ is a $\nu$-\emph{self-scaled barrier} ($\nu$-SSB) if for all $x,w\in\setK$ we have $H(w)x\in\setK$ and $h_{\ast}(H(w)x)=h(x)-2h(w)-\nu$. Let $\scrB_{\nu}(\setK)$ denote the class of $\nu$-SSBs.  
\end{definition}
We emphasize that $\scrB_{\nu}(\setK)\subset\scrH_{\nu}(\setK)$. \cite{HauGul02} showed that every symmetric cone admits a $\nu$-SSB for some $\nu\geq 1$, while a characterization of the barrier parameter $\nu$ has been obtained in \cite{GulTun98}. The main advantage of working with SSB's instead of LHSCB's is that we can make potentially longer steps in the interior of the cone $\setK$ towards the direction of its boundary. Let $x \in\setK$ and $d \in \setE$. Denote
\begin{equation}
\label{eq:sigma_def}
\sigma_x(d):= (\sup \{ t: x -t d \in \setK\})^{-1}
\end{equation}
Since $\scrW(x;1) \subseteq\setK$ for all $x\in\setK$, we have that $\sigma_x(d) \leq \norm{d}_{x}$ and $\sigma_{x}(-d)\leq\norm{d}_{x}$ for all $d\in\setE$. Therefore $[0,\frac{1}{\norm{d}_{x}})\subseteq [0,\frac{1}{\sigma_{x}(d)})$. Hence, if the scalar quantity $\sigma_{x}(d)$ can be computed efficiently, it would allow us to make a larger step without violating feasibility.%The next examples illustrate how $\sigma_{x}(d)$ looks like in practically relevant situations.
%%%%%%
\begin{example}
For $\bar{\setK}=\bar{\setK}_{\text{NN}}$, to guarantee $x-td\in\setK_{\text{NN}}$, we need $x_{i}-t d_{i}>0$ for all $i\in\{1,\ldots,n\}$. Hence, if $d_{i}\leq 0$, this is satisfied for all $t\geq 0$. If $d_{i}>0$, we obtain the restriction $t\leq \frac{x_{i}}{d_{i}}$. Hence, it follows that $\sigma_{x}(-d)=\max\{\frac{d_{i}}{x_{i}}:d_{i}>0\}$. 
\close
\end{example}
%%%%%
\begin{example}
For $\bar{\setK}=\bar{\setK}_{\text{SDP}}$, we see that $x-td\succ 0$ if and only if $\Id\succ t x^{-1/2}dx^{-1/2}$, where $\Id$ is the identity matrix. Hence, if $\lambda_{\max}(x^{-1/2}dx^{-1/2})>0$, then $t<\frac{1}{\lambda_{\max}(x^{-1/2}dx^{-1/2})}$. Thus, $\sigma_{x}(d)=\max\{\lambda_{\max}(x^{-1/2}dx^{-1/2}),0\}$.
\close
\end{example}
\noindent
We will also need the analogous result to Proposition \ref{prop:SCF_upper_bound} for barriers $h\in\scrB_{\nu}(\setK)$:
\begin{proposition}[{Theorem 4.2 \cite{NesTod97}}]
\label{prop:SSB_upper_bound}
Let $h\in\scrB_{\nu}(\setK)$ and $x \in\setK$. Let $d \in \setE$ be such that $\sigma_x(-d)>0$. Then, for all $t \in [0,\frac{1}{\sigma_x(-d)})$, we have:
\begin{equation}
\label{eq:SSB_upper_bound} 
h(x + t d) \leq h(x) + t \inner{\nabla h(x),d} + t^2 \norm{d}_x^2 \omega(t \sigma_x(-d)).
\end{equation}
\end{proposition}

\subsection{Unified Notation}
Our algorithms work on any conic domain on which we can efficiently evaluate a $\nu$-LHSCB. We formalize this in the following assumption
\begin{assumption}\label{ass:barrier}
$\bar{\setK}$ is a regular cone admitting an efficient barrier setup $h\in\scrH_{\nu}(\setK)$. By this we mean that at a given query point $x\in\setK$, we can construct an oracle that returns to us information about the values $h(x),\nabla h(x)$ and $H(x)=\nabla^{2}h(x)$, with low computational efforts. 
\end{assumption}
Given the potential advantages when working on symmetric cones, it is useful to develop a unified notation handling both cases at the same time. Note that when $h \in \scrB_{\nu}(\setK)$, we have the flexibility to treat $h$ either as $h \in\scrH_{\nu}(\setK)$ or as $h \in\scrB_{\nu}(\setK)$. To unify the presentation, we define
\begin{equation}\label{eq:zeta}
(\forall (x,d)\in\setX\times \setE):\;\zeta(x,d)=\left\{\begin{array}{ll} 
\norm{d}_{x} & \text{if }h\in\scrH_{\nu}(\setK)\setminus\scrB_{\nu}(\setK),\\
\sigma_{x}(-d) & \text{if }h\in\scrB_{\nu}(\setK).
\end{array}
\right.  
\end{equation}
Note that
\begin{align}
&(\forall (x,d)\in\setX\times \setE):\; \zeta(x,d)\leq\norm{d}_{x}, \label{eq:boundzeta}\\
&(\forall (x,d)\in\setX\times \setE)(\forall t \in [0,\frac{1}{\zeta(x,d)})):\; x + td \in\setK .\label{eq:step_length_zeta}
\end{align}
Finally, for the Bregman divergence $D_{h}(u,x):= h(u)-h(x)-\inner{\nabla h(x),u-x}$ defined for $x,u\in\setK$, Proposition \ref{prop:SCF_upper_bound}, Proposition \ref{prop:SSB_upper_bound} together with eq. \eqref{eq:zeta}, give us the one-and-for-all Bregman bound
\begin{equation}\label{eq:Dbound}
D_{h}(x+t d,x)=h(x+t d)-h(x)-\inner{\nabla h(x),td}\leq  t^2 \norm{d}_{x}^2 \omega(t \zeta(x,d))
\end{equation}
valid for all $(x,d)\in\feas\times \setE$ and $t \in [0,\frac{1}{\zeta(x,d)})$.

