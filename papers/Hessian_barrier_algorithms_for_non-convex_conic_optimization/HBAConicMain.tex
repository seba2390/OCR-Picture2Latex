%*************************************************************
%*****    HBA-CONIC_v2
%*************************************************************

%!TEX TS-program =  pdflatex


%*************************************************************
%*****    DOCUMENT CLASS
%*************************************************************

\documentclass[12pt,fleqn,reqno]{article}
%----------------------------------------------------------------------
%% Basic math input
%----------------------------------------------------------------------
\usepackage{epsfig,amssymb,latexsym,amsmath,pifont,multicol,mathtools,amsmath,verbatim,amsthm,float,lipsum}
\usepackage{caption} 
\captionsetup[table]{belowskip=0pt,aboveskip=0pt}

%----------------------------------------------------------------------
%% Fonts and alphabets (beware of conflicts)
%----------------------------------------------------------------------
\usepackage[utf8]{inputenc}
\usepackage[T1]{fontenc}


%% Times
%----------------------------------------------------------------------
\usepackage[varg]{txfonts}
\let\mathbb=\varmathbb

%% Blackboard bold
%----------------------------------------------------------------------
\usepackage[sans]{dsfont}
%\let\mathbb=\mathds

\usepackage[left=2.5cm, right=2.5cm, top=2.5cm, bottom=2.5cm]{geometry}


%% Math alphabets
%----------------------------------------------------------------------
\usepackage[%
cal=euler,
bb=fourier,
scr=zapfc,
%frak=euler
]
{mathalfa}



%----------------------------------------------------------------------
%% Figures and Graphics
%----------------------------------------------------------------------
\usepackage[font=small,labelfont=bf]{caption}
\usepackage{subfigure}
%\usepackage{subcaption}
\usepackage{graphicx}
\graphicspath{{Figures/}} 
%----------------------------------------------------------------------
%% Miscellaneous
%----------------------------------------------------------------------
\usepackage{acronym}
\usepackage{latexsym}
\usepackage{paralist}
\usepackage{wasysym}
\usepackage{xspace}
\usepackage{framed}
\usepackage{palatino,pxfonts}
\usepackage{authblk}
\usepackage{enumitem}
%----------------------------------------------------------------------
%% References
%----------------------------------------------------------------------
\usepackage[numbers,sort&compress]{natbib}
\def\bibfont{\footnotesize}
\def\bibsep{\smallskipamount}
\def\bibhang{24pt}

%----------------------------------------------------------------------
%% Colors
%----------------------------------------------------------------------
\usepackage[dvipsnames,svgnames]{xcolor}
\colorlet{MyBlue}{DodgerBlue!75!Black}
\colorlet{MyGreen}{DarkGreen!95!Black}

%----------------------------------------------------------------------
%% Comments
%----------------------------------------------------------------------
\newcommand{\MS}[1]{{\color{red}[\textbf{MS}:\;#1]}}
\newcommand{\PD}[1]{{\color{blue}[\textbf{PD}:\; #1]}}
%*************************************************************
%----------------------------------------------------------------------
%% Hyperlinks
%----------------------------------------------------------------------
\usepackage{hyperref}
\hypersetup{
colorlinks=true,
linktocpage=true,
%pdfstartpage=1,
pdfstartview=FitH,
breaklinks=true,
pdfpagemode=UseNone,
pageanchor=true,
pdfpagemode=UseOutlines,
plainpages=false,
bookmarksnumbered,
bookmarksopen=false,
bookmarksopenlevel=1,
hypertexnames=true,
pdfhighlight=/O,
%hyperfootnotes=true,
%nesting=true,
%frenchlinks,
urlcolor=MyBlue!60!black,linkcolor=MyBlue!70!black,citecolor=DarkGreen!70!black, % <--- for screen
%urlcolor=black, linkcolor=black, citecolor=black, %pagecolor=black, % <--- for printing
%pagecolor=RoyalBlue,
pdftitle={},
pdfauthor={},
pdfsubject={},
pdfkeywords={},
pdfcreator={pdfLaTeX},
pdfproducer={LaTeX with hyperref}
}
%\newcommand{\EMAIL}[1]{\email{\href{mailto:#1}{#1}}}
%\newcommand{\URLADDR}[1]{\urladdr{\href{#1}{#1}}}


%----------------------------------------------------------------------
%% Cleverefs
%----------------------------------------------------------------------
\numberwithin{equation}{section}  %numberwithin goes before cleverefs when using hyperref
\usepackage[sort&compress,capitalize,nameinlink]{cleveref}
\crefname{example}{Ex.}{Exs.}
\newcommand{\crefrangeconjunction}{\textendash\hspace*{1pt}}
\crefrangeformat{equation}{\upshape(#3#1#4)\textendash(#5#2#6)}

%----------------------------------------------------------------------
%% Only referenced equations
%----------------------------------------------------------------------
%\usepackage{autonum}


%*************************************************************
%*****    MACROS
%*************************************************************

%----------------------------------------------------------------------
%% Aliases
%----------------------------------------------------------------------
\newcommand{\dd}{\:d}
\newcommand{\del}{\partial}
\newcommand{\eps}{\varepsilon}
\newcommand{\from}{\colon}
\newcommand{\injects}{\hookrightarrow}
\newcommand{\pd}{\partial}
\newcommand{\wilde}{\widetilde}
\newcommand{\dif}{\dd}
\newcommand{\bigoh}{\mathcal{O}}
\newcommand{\mg}{\succ}
\newcommand{\mgeq}{\succcurlyeq}
\newcommand{\ml}{\prec}
\newcommand{\mleq}{\preccurlyeq}
%---------------------------------------------------------
\DeclareMathOperator*{\argmin}{argmin}
\DeclareMathOperator*{\Argmin}{Argmin}
\DeclarePairedDelimiter{\ceil}{\lceil}{\rceil}
\DeclareMathOperator*{\argmax}{argmax}
\DeclareMathOperator*{\Argmax}{Argmax}
\DeclareMathOperator{\aff}{aff}
\DeclareMathOperator{\bd}{bd}
\DeclareMathOperator{\cl}{cl}
\DeclareMathOperator{\epi}{epi}
\DeclareMathOperator{\rank}{rank}
\DeclareMathOperator{\conv}{conv}
\DeclareMathOperator{\Cov}{Cov}
\DeclareMathOperator{\diag}{diag}
\DeclareMathOperator{\diam}{diam}
\DeclareMathOperator{\dist}{dist}
\DeclareMathOperator{\Div}{div \!}
\DeclareMathOperator{\dom}{dom}
\DeclareMathOperator{\lin}{Lin}
\DeclareMathOperator{\gr}{graph}
\DeclareMathOperator{\Int}{int}
\DeclareMathOperator{\nullspace}{null}
\DeclareMathOperator{\grad}{grad}
\DeclareMathOperator{\image}{im}
\DeclareMathOperator{\sgn}{sgn}
\DeclareMathOperator{\Span}{span}
\DeclareMathOperator{\supp}{supp}
\DeclareMathOperator{\tr}{tr}
\DeclareMathOperator{\Var}{\mathsf{Var}}
\DeclareMathOperator{\Id}{Id}
\renewcommand{\vec}{\mathsf{vec}}
\DeclareMathOperator{\svec}{\mathsf{svec}}
\newcommand{\const}{\mathtt{c}}
\newcommand{\ce}{\mathtt{e}}
\newcommand{\ca}{\mathtt{a}}
\newcommand{\cb}{\mathtt{b}}
\newcommand{\ct}{\mathtt{t}}
\DeclareMathOperator{\lev}{lev}
%*************************************************************
%*****    Bold
%*************************************************************
\newcommand{\bA}{{\mathbf A}}
\newcommand{\BB}{{\mathbf B}}
\newcommand{\bH}{\mathbf{H}}
\newcommand{\bI}{\mathbf{I}}
\newcommand{\bJ}{\mathbf{J}}
\newcommand{\XX}{\mathbf{X}}
\newcommand{\bb}{\mathbf{b}}
\newcommand{\B}{\mathbb{B}}
\newcommand{\D}{\mathbb{D}}
\newcommand{\T}{\mathbb{T}}
\newcommand{\G}{\mathbb{G}}
\newcommand{\gbold}{\mathbf{g}}
\newcommand{\bR}{\mathbf{R}}
\newcommand{\bP}{\mathbf{P}}
\newcommand{\UU}{\mathbf{U}}
\newcommand{\bL}{\mathbf{L}}
\newcommand{\bQ}{\mathbf{Q}}
\newcommand{\bx}{\mathbf{x}}
\newcommand{\by}{\mathbf{y}}
\newcommand{\bS}{\mathbf{S}}
\newcommand{\bT}{\mathbf{T}}
\newcommand{\bD}{\mathbf{D}}
\newcommand{\bM}{\mathbf{M}}
\newcommand{\bW}{\mathbf{W}}
\newcommand{\bK}{\mathbf{K}}
\newcommand{\bN}{\mathbf{N}}
\newcommand{\bZ}{\mathbf{Z}}
\newcommand{\bp}{\mathbf{p}}
%*************************************************************
%*****    Short Cuts
%******************************************************
\newcommand{\orth}{\bot}
\renewcommand{\iff}{\Leftrightarrow}
\renewcommand{\implies}{\Rightarrow}
\renewcommand{\emptyset}{\varnothing}
\newcommand{\eqdef}{\triangleq}
\newcommand{\wlim}{\rightharpoonup}
%*************************************************************
%*****    Sets
%*******************************************************
\newcommand{\scrA}{\mathcal{A}}
\newcommand{\scrB}{\mathcal{B}}
\newcommand{\scrC}{\mathcal{C}}
\newcommand{\setC}{\mathsf{C}}
\newcommand{\scrD}{\mathcal{D}}
\newcommand{\setD}{\mathsf{D}}
\newcommand{\scrE}{\mathcal{E}}
\newcommand{\setE}{\mathsf{E}}
\newcommand{\scrF}{\mathcal{F}}
\newcommand{\setF}{\mathsf{F}}
\newcommand{\scrG}{\mathcal{G}}
\newcommand{\setG}{\mathsf{G}}
\newcommand{\scrH}{\mathcal{H}}
\newcommand{\setH}{\mathsf{H}}
\newcommand{\scrI}{\mathcal{I}}
\newcommand{\scrJ}{\mathcal{J}}
\newcommand{\scrK}{\mathcal{K}}
\newcommand{\setK}{\mathsf{K}}
\newcommand{\scrL}{\mathcal{L}}
\newcommand{\setL}{\mathsf{L}}
\newcommand{\scrM}{\mathcal{M}}
\newcommand{\scrN}{\mathcal{N}}
\newcommand{\scrO}{\mathcal{O}}
\newcommand{\scrP}{\mathcal{P}}
\newcommand{\scrQ}{\mathcal{Q}}
\newcommand{\setQ}{\mathsf{Q}}
\newcommand{\scrR}{\mathcal{R}}
\newcommand{\scrS}{\mathcal{S}}
\newcommand{\setS}{\mathsf{S}}
\newcommand{\scrT}{\mathcal{T}}
\newcommand{\scrU}{\mathcal{U}}
\newcommand{\setU}{\mathsf{U}}
\newcommand{\scrV}{\mathcal{V}}
\newcommand{\scrW}{\mathcal{W}}
\newcommand{\scrX}{\mathcal{X}}
\newcommand{\setX}{\mathsf{X}}
\newcommand{\scrY}{\mathcal{Y}}
\newcommand{\setY}{\mathsf{Y}}
\newcommand{\scrZ}{\mathcal{Z}}
\newcommand{\setZ}{\mathsf{Z}}
%*************************************************************
%*****    Probability
%**********************************************
\renewcommand{\Pr}{\mathbb{P}}
\newcommand{\Ex}{\mathbb{E}}
\newcommand{\dlim}{\stackrel{d}{\rightarrow}}
\newcommand{\measQ}{\mathsf{Q}}
\newcommand{\M}{{\mathsf{M}}}
\newcommand{\measP}{{\mathsf{P}}}
\newcommand{\Exp}{{\mathsf{E}}}
\newcommand{\Leb}{{\mathsf{Leb}}}
\newcommand{\F}{{\mathbb{F}}}
%\newcommand{\G}{{\mathbb{G}}}
\newcommand{\Poi}{{\mathtt{Poi}}}
\newcommand{\Ber}{{\mathtt{Ber}}}
%----------------------------------------------------------------------
%% Numbers
%----------------------------------------------------------------------
\newcommand{\0}{\mathbf{0}}
\newcommand{\1}{\mathbf{1}}
\newcommand{\Rn}{\R^n}
\newcommand{\Rnp}{\R_{+}^{n}}
\newcommand{\Rp}{\R_+}
\newcommand{\C}{\mathbb{C}}
\newcommand{\R}{\mathbb{R}}
\newcommand{\Q}{\mathbb{Q}}
\newcommand{\Z}{\mathbb{Z}}
\newcommand{\N}{\mathbb{N}}
\newcommand{\K}{\mathbb{K}}
%----------------------------------------------------------------------
%% Topology
%----------------------------------------------------------------------
\newcommand{\symm}{\mathbb{S}}
\newcommand{\ball}{\mathbb{B}}
\DeclareMathOperator{\NC}{\mathsf{NC}}
\DeclareMathOperator{\TC}{\mathsf{TC}}
\DeclareMathOperator{\Ext}{ext}
\DeclareMathOperator{\Conv}{conv}
\DeclareMathOperator{\eval}{ev}
\newcommand{\bC}{{\mathbf{C}}}
\newcommand{\metric}{\mathsf{d}}
\newcommand{\Lim}{\mathsf{Lim}}
\newcommand{\Zer}{\mathsf{Zer}}
%*************************************************************
%*****    GAMES
%*************************************************************
\newcommand{\nash}{\mathbf{NE}}
\newcommand{\corr}{\mathbf{CE}}
\newcommand{\CE}{\mathsf{CE}}
\newcommand{\NE}{\mathsf{NE}}
\newcommand{\val}{\mathsf{Val}}
\newcommand{\reg}{\mathsf{Reg}}
\newcommand{\varreg}{\mathsf{VarReg}}
\newcommand{\dynreg}{\mathsf{DynReg}}
%----------------------------------------------------------------------
%% Optimization
%----------------------------------------------------------------------
\DeclareMathOperator{\VI}{VI}
\DeclareMathOperator{\SVI}{SVI}
\DeclareMathOperator{\EV}{EV}
\DeclareMathOperator{\Opt}{Opt}							% for value of problem
\DeclareMathOperator{\prox}{prox}
\newcommand{\feas}{\mathsf{X}}							% for feasible region
\newcommand{\intfeas}{\feas^{\circ}}						% for interior of region
\newcommand{\sols}{\sol[\feas]}							% for solution set
\newcommand{\Lip}{L}								% for Lipschitz constant
\DeclareMathOperator*{\essinf}{ess\,inf}
\DeclareMathOperator{\NB}{\scrN\scrB}
\DeclareMathOperator{\SB}{\scrS\scrB}
%*************************************************************
%*****    ENVIRONMENTS
%*************************************************************

\newtheorem{innercustomthm}{Proposition}
\newenvironment{customthm}[1]
  {\renewcommand\theinnercustomthm{#1}\innercustomthm}
  {\endinnercustomthm}

%----------------------------------------------------------------------
%% Algorithms
%----------------------------------------------------------------------
\usepackage[ruled,vlined]{algorithm2e}
\usepackage{algpseudocode}								% for algorithm macros
%\renewcommand{\algorithmiccomment}[1]{\hfill\texttt{\emph{\#}\,#1}}			% for algorithm comments

%----------------------------------------------------------------------
%% Theorem-like
%----------------------------------------------------------------------
\theoremstyle{plain}
\newtheorem{theorem}{Theorem}
\newtheorem{corollary}[theorem]{Corollary}
\newtheorem*{corollary*}{Corollary}
\newtheorem{lemma}[theorem]{Lemma}
\newtheorem{proposition}[theorem]{Proposition}
\newtheorem{conjecture}[theorem]{Conjecture}


%----------------------------------------------------------------------
%% Definition-like
%----------------------------------------------------------------------
\theoremstyle{definition}
\newtheorem{definition}[theorem]{Definition}
\newtheorem*{definition*}{Definition}
\newtheorem{assumption}{Assumption}
%----------------------------------------------------------------------
%% Proofs
%----------------------------------------------------------------------
\newenvironment{Proof}[1][Proof]{\begin{proof}[#1]}{\end{proof}}
%\smartqed	%This command right justifies \qed throughout the paper.
\renewcommand{\qed}{\hfill{\footnotesize$\blacksquare$}}
\newcommand{\close}{\hfill{\footnotesize$\Diamond$}}


%----------------------------------------------------------------------
%% Remark-like
%----------------------------------------------------------------------
\theoremstyle{remark}
\newtheorem{remark}{Remark}
\newtheorem*{remark*}{Remark}
\newtheorem*{notation*}{Notational remark}
\newtheorem{example}{Example}
\newtheorem{experiment}{Experiment}

%----------------------------------------------------------------------
%% Numbering
%----------------------------------------------------------------------
\numberwithin{theorem}{section}
\numberwithin{remark}{section}
\numberwithin{example}{section}


\newcommand{\as}{\textup(a.s.\textup)\xspace}
\newcommand{\textpar}[1]{\textup(#1\textup)}
\DeclarePairedDelimiter{\braces}{\{}{\}}
\DeclarePairedDelimiter{\bracks}{[}{]}
\DeclarePairedDelimiter{\parens}{(}{)}
\DeclarePairedDelimiter{\angles}{\langle}{\rangle}

\providecommand\given{} % provides an empty command for the delimiters below

\DeclarePairedDelimiter{\abs}{\lvert}{\rvert}
\DeclarePairedDelimiter{\floor}{\lfloor}{\rfloor}

\DeclarePairedDelimiter{\inner}{\langle}{\rangle}
\DeclarePairedDelimiter{\norm}{\lVert}{\rVert}
\DeclarePairedDelimiter{\dnorm}{\lVert}{\rVert_{\ast}}

%----------------------------------------------------------------------
%%% ACRONYMS
%----------------------------------------------------------------------
\newcommand{\acli}[1]{\textit{\acl{#1}}}
\newcommand{\acdef}[1]{\textit{\acl{#1}} \textup{(\acs{#1})}\acused{#1}}
\newcommand{\acdefp}[1]{\emph{\aclp{#1}} \textup(\acsp{#1}\textup)\acused{#1}}

\newacro{SC}[SC]{self-concordant}
\newacro{SCB}[SCB]{self-concordant barrier}
\newacro{SSB}[SSB]{self-scaled barrier}
\newacro{FOM}{First-order method}

\DeclareMathOperator{\HBA}{\mathbf{HBA}}
\DeclareMathOperator{\AHBA}{\mathbf{AHBA}}
\DeclareMathOperator{\SAHBA}{\mathbf{SAHBA}}

%Pavel's commands
\def\la{\langle}
\def\ra{\rangle}



%*************************************************************
%*****    MAIN DOCUMENT
%*************************************************************

\title{Hessian Barrier Algorithms for non-convex conic optimization}
\date{\today}

\author[1]{\small Pavel Dvurechensky}
\author[2]{\small Mathias Staudigl}

\affil[1]{\footnotesize Weierstrass Institute for Applied Analysis and Stochastics, Mohrenstr. 39, 10117 Berlin, Germany\\
(\href{mailto:pavel.Dvurechensky@wias-berlin.de}{pavel.dvurechensky@wias-berlin.de})}

\affil[2]{\footnotesize Maastricht University, Department of Advanced Computing Sciences, P.O. Box 616, NL\textendash 6200 MD Maastricht, The Netherlands\\
(\href{mailto:m.staudigl@maastrichtuniversity.nl}{m.staudigl@maastrichtuniversity.nl})}


\begin{document}
\maketitle
%---------------------------------------------
%%% ABSTRACT
%----------------------------------------------------------------------
\begin{abstract}
\begin{abstract}
\label{sec:abstract}

%% 1. what is the problem 
Scientific applications that run on leadership computing facilities often face the challenge 
of being unable to fit leading science cases onto accelerator devices due to memory constraints 
(memory-bound applications).
%
% 2. what is your solution 
In this work, the authors studied one such US Department of Energy mission-critical condensed matter 
physics application, Dynamical Cluster Approximation (DCA++), and this paper discusses how device memory-bound challenges were successfully reduced  by proposing an effective 
``all-to-all'' communication method---a ring communication algorithm. 
%
This implementation takes advantage of acceleration on GPUs and remote direct memory access (RDMA) for fast data exchange between GPUs. 
%
\\Additionally, the ring algorithm was optimized with sub-ring communicators
and multi-threaded support to further reduce communication overhead and 
expose more concurrency, respectively.
%
% 3. What's the cherry-picked evaluation result you want to mention
The computation and communication were also analyzed 
by using the Autonomic Performance Environment for Exascale 
(APEX) profiling tool,  and this paper further discusses the 
performance trade-off for the ring algorithm implementation. 
%
The memory analysis on the ring algorithm shows that the allocation size for the authors' most 
memory-intensive data structure per GPU is now reduced to $1/p$ of the original size, where $p$ is the number of GPUs in the ring communicator.
%
The communication analysis suggests that 
the distributed Quantum Monte Carlo execution time grows linearly as sub-ring size increases, and the cost of messages passing through the network interface connector could be a limiting factor.


%
% \todoRed{Ronnie: Next sentence needs rewrite, too much information about Green's function that no one knows in the abstract; recommend generalizing.} \emph {However, DCA++ is currently facing memory-bound challenge as 
% a larger device array $G_t$ is limited by device memory size, where
% $G_t$ is a two-particle Green's function that allows condensed matter
% scientists to explore larger and more complex (higher fidelity)
% physics cases.}

\end{abstract}

\keywords{DCA++, Quantum Monte Carlo, GPU Remote Direct Memory Access, memory-bound issue, exascale machines}

\end{abstract}



%*************************************************************
%*****    BODY TEXT
%*************************************************************
\renewcommand{\sharp}{\gamma}
\acresetall
\allowdisplaybreaks

%%%%%%%%%%%%%%%%%%%%%%%%%%%%%%%%%%%%%%%%%
%%%%%%%% INTRO%%%%%%%%%%%%%%%%%%%%%%%%%%%
%%%%%%%%%%%%%%%%%%%%%%%%%%%

\section{Introduction}
\label{sec:intro}
\section{Introduction}
\label{sec:Introduction}


The goal in top-$\size$ recommendation is to recommend to each
consumer a small set of $\size$ items from a large collection of
items~\cite{cremonesi2010performance}.  For example, Netflix may want
to recommend $\size$ appealing movies to each consumer.  Collaborative
Filtering (CF)~\cite{herlocker2002empirical,lee2012comparative} is a
common top-$\size$ recommendation method.  CF infers user interests by
analyzing partially observed user-item interaction data, such as user
ratings on movies or historical purchase
logs~\cite{kanagal2012supercharging}. The main assumption in CF is that
users with similar interaction patterns have similar interests.


Standard CF methods for top-$\size$ recommendation focus on making  suggestions  that accurately reflect the user's preference history. However, as  observed in previous work,  CF recommendations are generally biased toward  popular items, leading to a rich get richer effect~\cite{vargas2014improving,steck2011item}.  The major reasons for this are \textit{popularity bias} and \textit{sparsity} of CF interaction data (detailed in Section~\ref{sec:related-work}). In a nutshell, to maintain  accuracy, recommendations are generated from the dense regions of the data,  where the popular items lie.  

However,  accurately suggesting popular items, may not be satisfactory for the consumers. For example, in Netflix, an accuracy-focused movie recommender may recommend ``Star Wars: The Force Awakens'' to users who have seen ``Star Wars: Rogue One''.  But, those users are probably already aware of ``The Force Awakens''. Considering additional factors, such as novelty of recommendations,  can lead to more effective suggestions~\cite{cremonesi2010performance,Castells2015,zhang2008avoiding,ziegler2005improving,zhang2012auralist}. 
%Second, accuracy-focused models typically achieve a   overall item-space coverage across their recommendations,  whereas high item-space coverage helps providers of the items increase revenue
%, users satisfaction since they are  likely already aware of or can find these items on their own.  

Focusing on popular items also adversely affects the satisfaction of  the providers of the items. This is because  accuracy-focused models typically achieve a  low overall item space coverage across their recommendations, whereas   high item space coverage helps providers of the items increase their revenue~\cite{vargas2014improving,Castells2015,adomavicius2011maximizing,anderson2006thelongtail, yin2012challenging,adomavicius2012improving}.
%accuracy-focused models typically achieve a

In contrast to the relatively small number of popular items, there are copious  {\it long-tail\/} items that have fewer observations (e.g., ratings) available. More precisely,  using the Pareto  principle (i.e.,~the $80/20$ rule),  long-tail items can be defined as items that generate the lower $20\%$ of observations~\cite{yin2012challenging}. Experimentally we found that these items correspond to almost $85\%$ of the items in several datasets (Sections~\ref{sec:Notation} and \ref{sec:Experiments}). %Table~\ref{tab:DatasetStatsticsSmall})


As previously shown, one way to improve the novelty of top-$\size$ sets is to recommend interesting long-tail items~\cite{cremonesi2010performance,ge2010beyond}.  The intuition  is that since they have fewer observations available,  they are more likely to be unseen~\cite{Kaminskas:2016:DSN:3028254.2926720}.  
 %For example, in online commerce,  newly added items are long-tail items that are yet to be discovered.  
Moreover, long-tail item promotion also results in higher overall coverage of the item space%, which increases profits for providers of the items
~\cite{vargas2014improving,Castells2015,zhang2008avoiding,zhang2012auralist,adomavicius2011maximizing,anderson2006thelongtail,yin2012challenging,jambor2010optimizing}. Because long-tail promotion reduces accuracy~\cite{steck2011item}, there are trade-offs to be explored.


%original submitted to ICDE
%This work studies three aspects of top-$\size$ recommendation: accuracy, novelty, and item-space coverage, and examines their trade-offs. In most previous work, predictions of a base recommendation system are re-ranked to handle their trade-offs~\cite{adomavicius2012improving,jambor2010optimizing,zhang2013personalize,wang2009portfolio}. Due to performance considerations, however, these techniques are not customized per user. For example,  parameters that balance the trade-off between novelty and accuracy are cross-validated at a global level.  This can be detrimental since users have varying preferences for  objectives such as long-tail novelty. We explore how to  automatically infer  user  preference for long-tail novelty, and how to leverage  it to correct  the popularity bias in standard recommender models. Our work does not rely on any additional contextual data, although such data, if available, can help promote newly-added long-tail items~\cite{agarwal2009regression,Saveski:2014:ICR:2645710.2645751}.

This work studies three aspects of top-$\size$ recommendation: accuracy, novelty, and item space coverage, and examines their trade-offs. In most previous work, predictions of a base recommendation algorithm are \textit{re-ranked} to handle these trade-offs~\cite{adomavicius2012improving,jambor2010optimizing,zhang2013personalize,wang2009portfolio}. The re-ranking models are computationally efficient but suffer from two drawbacks. First, due to performance considerations,  parameters that balance the trade-off between novelty and accuracy  are not customized per user. Instead they are cross-validated at a global level.  This can be detrimental since users have varying preferences for  objectives such as long-tail novelty. Second,  the re-ranking methods are often limited to a specific base recommender  that may be sensitive to dataset density. 
As a result, the datasets are pruned and the problem is studied in dense settings~\cite{adomavicius2012improving,ho2014likes}; but real world  scenarios are often sparse~\cite{kanagal2012supercharging,liu2017experimental}.   
% Because  dataset density can impact the performance of most base recommenders (like R-SVD), which in turn affects the performance of the re-ranking model, 

\iffalse
We address these limitations by directly inferring  user  preference for long-tail novelty  from interaction data.  This  allows us to customize the re-ranking  per user, and design a \textit{generic} framework, which resolves the second problem. In particular, since the long-tail novelty preferences are estimated independently of any base  recommender model, we can  plug-in an appropriate base recommender w.r.t. the dataset sparsity.% including ones that are more suitable for sparse settings.  

Modelling  user  preference for  long-tail novelty using only item popularity statistics, e.g., the average popularity of rated items as in~\cite{jugovac2017efficient}, disregards additional information like whether the user found the item interesting and the long-tail preferences of other users  of the items. \iffalse To incorporate them, we introduce the notion of  \emph{item long-tail importance}. Both  user long-tail preferences and item long-tail importance are dependent:  a user has high preference for discovering long-tail items if she is interested in important long-tail items, and an item that is associated with many of these kinds of users is likely to be more important.  We propose a joint optimization framework to directly learn,  from interaction data, both the users' long-tail preferences and the  items' long-tail importance. \fi
We propose an optimization approach that  incorporates  this information and  directly learns,  from interaction data, the users' long-tail novelty preferences.

Next, we use these learned preferences  to design a  top-$\size$ recommendation framework thats is generic, and provides customized balance between accuracy, novelty, and coverage. We refer to it as framework as GANC.  Using GANC, we design a novel algorithm, {\it Ordered Sampling-based Locally Greedy (OSLG)\/}, that relies on the learned long-tail novelty preferences  to scalably correct for popularity bias. Our work does not rely on any additional contextual data, although such data, if available, can help promote newly-added long-tail items~\cite{agarwal2009regression,Saveski:2014:ICR:2645710.2645751}. In summary:
\fi

We address the first limitation by directly inferring  user  preference for long-tail novelty  from interaction data.   Estimating these  preferences  using only item popularity statistics, e.g., the average popularity of rated items as in~\cite{jugovac2017efficient}, disregards additional information, like whether the user found the item interesting or the long-tail preferences of other users  of the items. We propose an approach that  incorporates  this information and  learns the users' long-tail novelty preferences from interaction data.

This approach allows us to customize the re-ranking  per user, and  design a \textit{generic} re-ranking framework, which resolves the second limitation of prior work. In particular, since the long-tail novelty preferences are estimated independently of any base recommender, we can  plug-in an appropriate one w.r.t. different factors, such as the dataset sparsity.

Our top-$\size$ recommendation framework, \textbf{GANC}, is \textbf{G}eneric, and provides customized balance between \textbf{A}ccuracy, \textbf{N}ovelty, and \textbf{C}overage. % Moreover, based on the learned long-tail novelty preferences, we also design a novel algorithm, {\it Ordered Sampling-based Locally Greedy (OSLG)\/}, that relies on the learned long-tail novelty preferences  to scalably correct for popularity bias. 
Our work does not rely on any additional contextual data, although such data, if available, can help promote newly-added long-tail items~\cite{agarwal2009regression,Saveski:2014:ICR:2645710.2645751}. In summary:

%Consider  the following toy example:
\vspace{-0.2cm}
\begin{table}[htb]
\centering
\scriptsize
%\small
\begin{tabular}{ccccccc} 
%\toprule
%&\multirow{2}{*}{}&\multicolumn{7}{c}{Ratings}\\
& & \cellcolor{blue!35}$w_1$ &\cellcolor{blue!18} $w_2$ & $\dots$ &\cellcolor{blue!8} $w_{89}$  &\cellcolor{blue!8} $w_{99}$   
\\
&   &$i_1$&$i_2$&$\dots$&$i_{89}$&$i_{90}$\\ 
\cmidrule(r){3-7} 	 
%\midrule
\cellcolor{red!35}$\theta_1$  &$u_1 $   &5 &   & $\dots$ &  &   \\
\cellcolor{red!28}$\theta_2$  &$u_2$     &5 &    & $\dots$ &  &  \\
 $\theta_3=?$  &$\bf u_3$  &5 &  &   $\dots$ &  &  \\
\cellcolor{red!10}$\theta_4$ & $u_4$  &  &5   & $\dots$ & &\\ 
\cellcolor{red!10}$\theta_5$ & $u_5$  &  & 5  & $\dots$ & &\\ 
$\theta_6=?$  & $\bf u_6$ & &5  &      $\dots$& &  \\ 
 & & $\hdots$  &$\hdots$   &$\hdots$   &$\hdots$   &$\hdots$  \\
%\midrule 
\cmidrule(r){3-7} 	 
\multicolumn{2}{c}{item pop.}  & 3  & 3  & $\dots$ &50&60\\  
%\bottomrule
%$ f_i$    &3  &3  &1  &3  &1  &2  \\  \hline
\end{tabular}
%#.
\caption{Simplified user-item interaction data. The user long-tail novelty preference ($\theta_u$), item long-tail importance weight ($w_i$) are highlighted. Darker colors indicate larger values. } \label{tab:example}
\end{table} 
\vspace{-0.2cm}
\begin{example}  
In Table~\ref{tab:example}, we are interested in estimating $\theta_3$ and $\theta_6$,  the long-tail preference of users $u_3$ and $u_6$ who have each rated a single movie. Additional ratings for other users  are not included here.  Considering only rating information, we observe $i_1$ and $i_2$ are  equally popular $|\mathcal{U}_{i_1}^{\trainset}| = |\mathcal{U}_{i_2}^{\trainset}|=3$, and $r_{31}=5$ and $r_{62}=5$. Using Eq.~\ref{eq:tfidf-risk}  we have $\theta_3 = \theta_6$. However, if we were given the long-tail preferences of the each item's user set, specifically that $u_1$ and $u_2$ have high long-tail preference (darker red), while $u_4$ and $u_5$ have lower long-tail preference (lighter red), we could conclude $i_1$ is a more important long-tail item compared to $i_2$ (indicated by a darker blue shade for $w_1$), and we expect  $\theta_3 \geq \theta_6$.

% On the other hand, if we knew that $u_4$ and $u_5$ have lower long-tail preference, we could conclude $i_2$ is a  less significant long-tail item. Therefore, However, if we  consider the long-tail preferences of other users, we may reason differently.    We need another variable $w_i$ which captures this information. 
%we would conclude that $u_3$ has higher long-tail preference compared to $u_6$, since the users $i_1$ is a more prominent long-tail item. 

% Relying only  on item popularity information, we would  conclude   $u_3$ and $u_6$ have equal long-tail preference, since $i_1$ and $i_2$ are  equally popular. However, considering  the second column,  long-tail preference of users,  long-tail importance for each item,  which captures the long-tail preference of its users. Since  that  both users of $i_1$ have high long-tail preference while  the users of $i_2$ have lower preference,  we may conclude $i_1$ is a more important long-tail item compared to $i_2$. Therefore, $u_3$'s long-tail preference should be at least as large as $u_6$'s preference. Specifically, consider two  items $i_1$ and $i_2$, with the following rating data: $i_1=\{u_1:5, u_2:5, u_3:5 \}$, $i_2=\{u_4:5, u_5:5, u_6:5\}$.  

%Table~\ref{tab:example} shows  simplified rating data. We want an estimate of the long-tail preference of $u_3$ and $u_6$, who have each  rated a single movie.  Relying only  on movie popularity information, we would  conclude   $u_3$ and $u_6$ have similar long-tail preference, since $m_1$ and $m_2$ are  equally popular. However, considering the long-tail preferences of other users of those movies, we may reason differently: since $u_1$ and $u_2$ have high long-tail preference, and $u_4$ and $u_5$ have low long-tail preference, $m_1$ is a more prominent long-tail item compared to $m_2$. Therefore, it is likely that $u_3$ has higher long-tail preference compared to $u_6$.considering the long-tail preferences of other users of those movies, we may reason differently.  For example, 
\label{ex:running}
\end{example}



%------------------------------

\iffalse
\begin{example}
Table~\ref{tab:example} shows rating data for a simplified system. %Note the user-item interaction matrix is sparse.
For this example, we define popular movies as those that have received  three or more ratings; $\{m_1, m_2, m_4\}$ are popular and  $\{m_3, m_5, m_6\}$ are niche movies. We observe $u_1$ and $u_3$  have rated relatively popular movies (risk-averse) while $u_2$ and $u_4$ have rated niche movies (risk-loving). 
\label{ex:running}
\end{example}

\begin{table}[htb]
\centering
\scriptsize
\begin{tabular}{ccccccc} 
\toprule
			&$m_1$ &$m_2$   &$m_3$    &$m_4$   &$m_5$ &$m_6$  \\ \hline 
$u_1 $ &5  &4  & - &-  &-  &-   \\
$u_2$  &-  &-  &-  &-  &5  &5   \\
$u_3$  &-  &4  &-  &5  &-  &-   \\
$u_4$  &-  &-  &3  &-  &-  &4   \\ 
$u_5$  &5  &-  &-  &3  &-  &-   \\ 
$u_6$  &4  &2  &-  &4  &-  &-   \\ 
\bottomrule
%$ f_i$    &3  &3  &1  &3  &1  &2  \\  \hline
\end{tabular}
\caption{User-Movie rating data} \label{tab:example}
\end{table}

It is essential to consider consumer characteristics in designing recommender systems so that they promote long-tail items to the right group of users and spread demand evenly between hit and niche items.  

\fi





%------------------------------
\iffalse
\begin{table}[htb]
\centering
\scriptsize
\begin{tabular}{ccccccc} 
\toprule
			&$m_1$ &$m_2$   &$m_3$    &$m_4$   &$m_5$ &$m_6$  \\ \hline 
$u_1 $ &\textbf{5}  & \textbf{4}  &\textcolor{gray}{ 1.2} &-  &-  &-   \\
$u_2$  &-  &-  &-  &-  & \textbf{5}  &\textbf{5}   \\
$u_3$  &-  &\textbf{4}  &-  &\textbf{5}  &-  &-   \\
$u_4$  &-  &-  &\textbf{3}  &-  &-  &\textbf{4}   \\ 
$u_5$  &\textbf{5}  &-  &-  &\textbf{3}  &-  &-   \\ 
$u_6$  &\textbf{4}  &\textbf{2}  &-  &\textbf{4}  &-  &-   \\ 
\bottomrule
%$ f_i$    &3  &3  &1  &3  &1  &2  \\  \hline
\end{tabular}
\caption{User-Movie rating data} \label{tab:example}
\end{table}
% $\mathcal{P}^1= \{ \mathcal{P}_1^1 \{i_1,i_2,i_3\}, \mathcal{P}_2^1:\{i_2,i_3,i_5\}  \}$
 %$\mathcal{P}^2= \{ \mathcal{P}_1^2: \{i_1,i_2,i_3\}, \mathcal{P}_2^2:\{i_2,i_5,i_6\}  \}$
 %$\mathcal{P}^3= \{ \mathcal{P}_1^3: \{i_7,i_8,i_9\}, \mathcal{P}_2^3:\{i_{10},i_{11},i_{12}\}  \}$
\begin{table}[htb]
\centering
\tiny
\begin{tabular}{ccc} 
\toprule
		&$u_1$&$u_2$  \\ \hline 
$\mathcal{P}^1 $ & $\{i_1,i_2,i_3\}$ & $\{i_2,i_3,i_5\} $ \\
$\mathcal{P}^2$ & $\{i_1,i_2,i_3\}$ & $\{i_2,i_5,i_6\} $ \\
$\mathcal{P}^3$ & $\{i_7,i_8,i_9\}$ & $\{i_{10},i_{11},i_{12} \}$ \\
\bottomrule
%$ f_i$    &3  &3  &1  &3  &1  &2  \\  \hline
\end{tabular}
\caption{Top-$\size$ allocations to users.} \label{tab:paretoExamples}
\end{table}
\fi


\iffalse
When considering long-tail items, it is important to consider consumers' willingness  to explore niche or unpopular items and their propensity towards similar items. In particular, they can be characterized by their  {\it risk degree\/} and {\it focusing degree\/}, respectively.  We compute these estimates  based on historical rating information. The following example further describes these notions in the context of movie rating data. 

\begin{example}  
Table~\ref{tab:example} shows rating data for a simplified system with $6$ users, $6$ movies, and $3$ genres. $m_i^{j}$ implies that movie $m_i$ belongs to genre $j$. Note the user-item interaction matrix is sparse. 
  For this setting, we define popular movies as those that have received  three or more ratings; $\{m_1, m_2, m_4\}$ are popular and  $\{m_3, m_5, m_6\}$ are niche movies. We now profile the users according to their risk and focusing degree. E.g., $u_1$ has rated relatively popular movies belonging to the same genre (risk-averse, high focusing degree); $u_2$ has rated niches movies in the same genre (risk-loving, high focusing degree); $u_3$ has rated popular movies in two different genres (risk-averse, low focusing degree), and $u_4$ has rated niches movies in two different genres (risk-loving, low focusing degree). 
\label{ex:running}
\end{example}
\begin{table}[htb]
\centering
\tiny
\begin{tabular}{ccccccc} 
\toprule
			&$m_1^{1}$ &$m_2^{1}$   &$m_3^{2}$    &$m_4^{3}$   &$m_5^{3}$ &$m_6^{3}$  \\ \hline 
$u_1 $ &5  &4  &-  &-  &-  &-   \\
$u_2$  &-  &-  &-  &-  &5  &5   \\
$u_3$  &-  &4  &-  &5  &-  &-   \\
$u_4$  &-  &-  &3  &-  &-  &4   \\ 
$u_5$  &5  &-  &-  &3  &-  &-   \\ 
$u_6$  &4  &2  &-  &4  &-  &-   \\ 
\bottomrule
%$ f_i$    &3  &3  &1  &3  &1  &2  \\  \hline
\end{tabular}
\caption{User-Movie rating data} \label{tab:example}
\end{table}
It is essential to consider these consumer characteristics in designing recommender systems so that they promote long-tail items to the right group of users and spread demand evenly between the hit and niche items.  
\fi
\iffalse
\begin{center}
\begin{figure*}[tp]
%\scalebox{0.5}{%
\resizebox{1\textwidth}{!}{%
%\small%\addtolength{\tabcolsep}{5pt}% below sums to 8
\begin{tabularx}{1.5\textwidth}{>{\hsize=2.5\hsize}X>{\hsize=2.5\hsize}X>{\hsize=0.5\hsize}X>{\hsize=0.5\hsize}X>{\hsize=0.5\hsize}X>{\hsize=0.5\hsize}X>{\hsize=0.5\hsize}X>{\hsize=0.5\hsize}X}
    \multirow{12}{*}{\includegraphics[scale=0.3]{codeForExample/popularity-movie.png}} & \multirow{12}{*}{\includegraphics[scale=0.3]{codeForExample/scatterplot.png}} & & & & & & \\
%   & &               &       &       &       &       &       \\
    & &\multicolumn{1}{l|}{}               &$m_1^{g1}$   	&$m_2^{g1}$    	&$m_3^{g2}$    &$m_4^{g2}$      &$m_5^{g3}$    \\ \cline{3-8}%\hline
    & &\multicolumn{1}{l|}{u1}          &5  &5  &-  &-   &-  \\
    & &\multicolumn{1}{l|}{u2}    		&-  &-  &4  &4  &5  \\
    & &\multicolumn{1}{l|}{u3}   			&1  &2  &1  &-  &-   \\
    & &\multicolumn{1}{l|}{u4}     		&1  &-  &-  &-  &-  \\
    & &               &       &       &       &       &       \\
    & &               &       &       &       &       &       \\
    & &               &       &       &       &       &       \\
    & &               &       &       &       &       &	\\
    \\
\end{tabularx}}
\caption{User-Movie interaction data a) Popularity-Movie histogram b)Movie genres/clusters c) User-Movie rating data} \label{fig:example}
\end{figure*}
\end{center}
\fi



%We propose a novel approach that allows us to  promote long-tail items in a targeted manner, thereby improving the novelty of top-$\size$ sets, the overall item-space coverage across recommendations, while maintaining reasonable levels of accuracy.

%Next, we integrate these learned preferences  in a generic  top-$\size$ recommendation framework to provide customized balance between accuracy and coverage.

%sequentially make recommendations, while adjusting its parameters with regard to the set of top-$\size$ recommendations made so far. However, since  sequential parameter updates  cause  scalability issues, we propose a sampling based algorithm. This variant of our framework, called {\it Ordered Sampling-based Locally Greedy (OSLG)\/},  allows us to  correct for the popularity bias in recommendations with regard to individual user long-tail preferences. 

%ICDE submission
%Our framework differs with  prior work in the following aspects:  unlike~\cite{adomavicius2011maximizing,adomavicius2012improving,zhang2013personalize,ho2014likes},  the long-tail preference personalization in our framework is learned rather than optimized using cross-validation or parameter tuning. In other words, our personalization method is independent of the underlying base  recommendation models.  Moreover, our framework is  generic. This enables us to  plug-in several base recommenders, and evaluate their  effectiveness without requiring  extensive tuning for the accuracy and coverage trade-off. 


%\vspace{-2.8pt}
\begin{itemize}

\item  We examine various measures for estimating user long-tail novelty preference in Section~\ref{sec:lt-pref} and formulate an optimization problem  to directly learn users' preferences for long-tail  items from interaction data in Section~\ref{sec:learning-lt-pref}. %In addition, we introduce several heuristics for measuring the user preference for less common items from historical rating data.% 

\item  We integrate the user preference estimates into GANC %, a generic re-ranking framework that provides customized balance between accuracy, novelty, and coverage 
(Section~\ref{sec:RiskbasedReranking}), and  introduce {\it Ordered Sampling-based Locally Greedy (OSLG)\/}, a scalable algorithm that relies  on user long-tail preferences to correct the popularity bias (Section~\ref{sec:optimizationAlgorithm}).
%We introduce OSLG, a scalable algorithm that relies  on user long-tail preferences to  maximize item space coverage \textcolor{red}{while maintaining acceptable levels of accuracy} (Section~\ref{sec:optimizationAlgorithm}).

\item   We conduct an extensive empirical study and evaluate performance from  accuracy, novelty, and coverage perspectives (Section~\ref{sec:Experiments}).  We use five  datasets with varying density and difficulty levels. %:  Netflix, MovieTweetings, and MovieLens (100K, 1M, 10M). 
  In contrast to most related work,  our evaluation considers realistic settings that include a large number of infrequent  items and users. %This enables us to study the impact of  data density on the performance trade-offs of several  state of the art top-$\size$ recommendation algorithms. %   %,  and use the all-items ranking protocol~\cite{steck2013evaluation,vargas2014improving}, where performance is measured using all items with train data. to evaluate the performance of several  state of the art top-$\size$ recommendation algorithms 
 
\item Our empirical results confirm that the performance of re-ranking models is impacted by the underlying   base recommender and the dataset density. Our generic approach enables us to easily incorporate a suitable base recommender to devise an effective solution for both dense and sparse settings. In dense settings, we use the same base recommender as existing re-ranking approaches, and we outperform them in accuracy and coverage metrics. For sparse settings, we plug-in a more suitable base recommender, and devise an effective solution that is competitive with existing top-$\size$ recommendation methods in accuracy and novelty. 

%Directly estimating the long-tail novelty preferences allows us to customize re-ranking per user, and  devise a generic framework.   
 
\end{itemize}

Section~\ref{sec:related-work} describes related work. Section~\ref{sec:conclusion} concludes.

%%%%%%%%%%%%%%%%%%%%%%%%%%%%%%%%%%%%%%%%%
%%%%%%%% PRELIMS%%%%%%%%%%%%%%%%%%%%%%%%%%%
%%%%%%%%%%%%%%%%%%%%%%%%%%%
\section{Prelminiaries}
\label{sec:prelims}

\section{Preliminaries}\label{sec:prelims}

\subparagraph{Notations.}
For a given positive integer $k \in \mathbb{N}$, the set of integers $\{1,2,\ldots,k\}$ is denoted for short as $[k]$. Given a graph $G$, the vertex set is denoted as $V(G)$ and the edge set as $E(G)$. Given two graphs $G_1$ and $G_2$, $G_1 \cup G_2$ denotes the graph $G$ where $V(G) = V(G_1) \cup V(G_2)$ and $E(G) = E(G_1) \cup E(G_2)$. 

In this paper, a regular $n$-gon is denoted by $A_1A_2A_3...A_n$ or $B_1B_2B_3...B_n$. For convenience, we define $A_{n + 1} := A_1$, $B_{n + 1} := B_1$, $A_0 := A_n$ and $B_0 := B_n$. We use the notation $\{A_i\}$ to denote the polygon $A_1A_2A_3 \ldots A_n$ and $\{B_i\}$ to denote the polygon $B_1B_2B_3 \ldots B_n$. For any regular polygon $A_1A_2A_3...A_n$, the circumcircle of the polygon is denoted as $(A_1A_2A_3...A_n)$. Given any $n$-vertex polygon in the Euclidean plane with vertices $\mathcal P = P_1P_2P_3\ldots P_n$, and interval in $\mathcal{K}$ is a subset of consecutive vertices $P_iP_{i+1\ldots P_j}$, $i,j\in [n]$, also denoted as $[P_i,P_j]$. Here $P_i$ is considered the starting vertex of the interval and $P_j$ the ending vertex. For any $P_k$, $i \leq k\leq j$ in the interval we will also use the notation $P_i \leq P_k \leq P_j$.

Given two points $P$, $Q$ in the Euclidean plane, we denote by ${\sf dist}(P,Q)$ the Euclidean distance between $P$ and $Q$. Given a line segment $AB$ in the Euclidean plane, $\overline{AB} = {\sf dist}(A,B)$. For two distinct points $A$ and $B$, $L_{AB}$ denotes the line containing $A$ and $B$; and $\overrightarrow{AB}$ denotes the ray originating from $A$ and containing $B$. 

When we refer to a graph $\mathcal{G}$ in the Euclidean plane then $V(\mathcal{G})$ is a set of points in the Euclidean plane, and $E(\mathcal{G})$ is a subset of the family of line segments $\{P_1P_2 | P_1,P_2 \in V(\mathcal{G})\}$. For any tree $\mathcal T$ in the Euclidean plane, we denote by the notation $|\mathcal T|$ the value of $\Sigma_{e \in E(\mathcal T)} \overline{e}$. A path in a tree $\mathcal T$ is uniquely specified by the sequence of vertices on the path; therefore, $P_1$, $P_2$, $P_3$, \ldots, $P_k$ (where $P_i \in V(\mathcal T), \forall i \in [k]$ and $P_iP_{i+1} \in E(\mathcal T), \forall i \in [k-1]$) denotes the path starting from the vertex $P_1$, going through the vertices $P_2$, $P_3$, \ldots, $P_{k-1}$ and finally ending at $P_k$. Equivalently, we can specify the same path as \emph{the path from $P_1$ to $P_k$}, since $\mathcal T$ is a tree. Consider the graph $T$ such that $V(T) = \{v_P| P \in V(\mathcal{T})\}$, $E(T) = \{v_{P_1}v_{P_2}| P_1P_2 \mbox{ is a line segment in } E(\mathcal{T})\}$. Then $T$ is said to be the topology of $\mathcal{T}$ while $\mathcal{T}$ is said to realize the topology $T$. Given two trees $\mathcal{T}_1$, $\mathcal{T}_2$ in the Euclidean plane, $\mathcal{T}' = \mathcal{T}_1\cup \mathcal{T}_2$ is the graph where $V(\mathcal{T}')= V(\mathcal{T}_1) \cup V(\mathcal{T}_2)$ and $E(\mathcal{T}')= E(\mathcal{T}_1) \cup E(\mathcal{T}_2)$. 

Given any graph $G$, a Steiner minimal tree or SMT for a terminal set $\mathcal{P} \subseteq V(G)$ is the minimum length connected subgraph $G'$ of $G$ such that $\mathcal{P} \subseteq V(G')$. The {\sc Steiner Minimal Tree} problem on graphs takes as input a set $\mathcal{P}$ of terminals and aims to find a minimum length SMT for $\mathcal{P}$. For the rest of the paper, we also refer to a Euclidean Steiner minimal tree as an SMT. Given a set of points $\mathcal{P}$ in the Euclidean plane, the convex hull of $\mathcal{P}$ is denoted as $\mathrm{CH(\mathcal P)}$.

\subparagraph{Euclidean Minimum Spanning Tree (MST).}
Given a set $\mathcal P$ of $n$ points in the Euclidean plane, let $G$ be a graph where $V(G) = \{v_P| P \in \mathcal P\}$ and $E(G) = \{v_{P_i}v_{P_j} | P_i,P_j \in \mathcal P\}$. Also, a weight function $w_{G}: E(T) \rightarrow \mathbb{R}$ is defined such that for each edge $v_{P_1}v_{P_2} \in E(T)$, $w_{G}(v_{P_1}v_{P_2}) = \overline{P_1P_2}$. The Euclidean minimum spanning tree of a set $\mathcal P$ is the minimum spanning tree of the graph $G$ with edge weights $w_G$. Note that a Steiner tree may have shorter length than a minimum spanning tree of the point set $\mathcal P$. 

In the plane, the Euclidean minimum spanning tree is a subgraph of the Delaunay triangulation. Using this fact, the Euclidean minimum spanning tree for a given set of points in the Euclidean plane can be found in $\OO(n\log n)$ time as discussed in \cite{Shamos1975ClosestpointP}. 

\subparagraph{Properties of a Euclidean Steiner minimal tree.}
A Euclidean Steiner minimal tree (SMT) has certain structural properties as given in~\cite{cockayne1967steiner}. We state them in the following Proposition.

\begin{proposition}\label{smt-prop}
Consider an SMT on $n$ terminals.
 \begin{enumerate}
   \item No two edges of the SMT intersect with each other.
 
   \item Each Steiner point has degree exactly $3$ and the incident edges meet at $120^\circ$ angles. The terminals have degree at most $3$ and the incident edges form angles that are at least $120^\circ$.
  
   \item The number of Steiner points is at most $n-2$, where $n$ is the number of terminals.

\end{enumerate}
\end{proposition}

 A full Steiner tree (FST) is a Steiner tree (need not be minimal, but may include Steiner points) having exactly $n-2$ Steiner points, where $n$ is the number of terminals. In an FST, all terminals are leaves and Steiner points are interior nodes. When the length of an FST is minimized, it is called a minimum FST.

All SMTs can be decomposed into FST components such that, in each component a terminal is always a  leaf. This decomposition is unique for a given SMT~\cite{hwang1992steiner}. A topology for an FST is called a full Steiner topology and that of a Steiner tree is called a Steiner topology.


%For a tree $\mathcal T$, we would denote the set of vertices (the terminal vertices and the Steiner points) as $V(\mathcal T)$ and the set of edges as $E(\mathcal T)$. Similarly for a topology $T$, $V(T)$ and $E(T)$ denote vertex set (the terminal vertices and the Steiner points) and the edge set respectively. \todo{\color{white}Anubhav: Added this defition}

\subparagraph{Steiner Hulls.}
A Steiner hull for a given set of points is defined to be a region which is known to contain an SMT. We get the following propositions from~\cite{hwang1992steiner}.

\begin{proposition}\label{convex-steiner}
    For a given set of terminals, every SMT is always contained inside the convex hull of those points. Thus, the convex hull is also a Steiner hull.
\end{proposition}

The next two propositions are useful in restricting the structure of SMTs and the location of Steiner points.

\begin{proposition} [The Lune property]\label{lune}
    Let $\rm UV$ be any edge of an SMT. Let $L(\rm{U},\rm{V})$ be the lune-shaped intersection of circles of radius $|\rm UV|$ centered on $\rm U$ and $\rm V$. No other vertex of the SMT can lie in $L(\rm{U},\rm{V})$, except $U$ and $V$ themselves.
\end{proposition}

\begin{proposition} [The Wedge property]\label{wedge}
    Let $W$ be any open wedge-shaped region having angle $120^\circ$ or more and containing none of the points from the input terminal set $\mathcal P$. Then $W$ contains no Steiner points from an SMT of $\mathcal P$.
\end{proposition}

\subparagraph{Approximation Algorithms.}
We define all the necessary terminology required in terms of a minimization problem, as ESMT is a minimization problem.
%\begin{definition} [Approximation Factor for a Minimization Problem]
%    Let $\mathcal{P}$ be a minimization problem. An algorithm $\mathcal{A}$ for the problem $\mathcal{P}$ is called an $\alpha$ factor approximation algorithm if, for every instance $\Pi$ of $\mathcal{P}$, we have $\rm{ALG}(\Pi) \leq \alpha \rm{OPT}(\Pi)$ where $\rm{ALG}(\Pi)$ and $\rm{OPT}(\Pi)$ are the values of the output of the algorithm and optimal solution for the instance $\Pi$ respectively. $\alpha$ can be a constant or a function of the input size $n$, and is always at least $1$.
%\end{definition}

%\begin{definition} [Polynomial Time Approximation Scheme (PTAS)]
%    An algorithm is called a polynomial time approximation scheme (PTAS) for a problem if it takes an input instance and a parameter $\epsilon > 0$, and outputs a solution with approximation factor $(1+\epsilon)$ for a minimization problem in time $\OO(n^{f(1/\epsilon)})$ where $n$ is the input size and $f(1/\epsilon)$ is any computable function.
%\end{definition}

\begin{definition} [Efficient Polynomial Time Approximation Scheme (EPTAS)]
    An algorithm is called an efficient polynomial time approximation scheme (EPTAS) for a problem if it takes an input instance and a parameter $\epsilon > 0$, and outputs a solution with approximation factor $(1+\epsilon)$ for a minimization problem in time $f(1/\epsilon)n^{\OO(1)}$ where $n$ is the input size and $f(1/\epsilon)$ is any computable function.
\end{definition}

\begin{definition} [Fully Polynomial Time Approximation Scheme (FPTAS)]
    An algorithm is called a fully polynomial time approximation scheme (FPTAS) for a problem if it takes an input instance and a parameter $\epsilon > 0$, and outputs a solution with approximation factor $(1+\epsilon)$ for a minimization problem in time $(1/\epsilon)^{\OO(1)}n^{\OO(1)}$ where $n$ is the input size.
\end{definition}

% appending preliminaries.tex --- Anubhav 


\section{Approximate optimality conditions}
\label{sec:Optimality}
%----------------------------------------------------------------------
%%% Optimization
%----------------------------------------------------------------------
% !TEX root = ./HBAConicMain.tex
%
%In this section we define our notion of approximate first- and second-order KKT points. The following assumptions are in place throughout the paper. 
%\begin{assumption}\label{ass:barrier}
%$\bar{\setK}$ is a regular cone admitting an efficient barrier setup $h\in\scrH_{\nu}(\setK)$. By this we mean that at a given query point $x\in\setK$, we can construct an oracle that returns to us information about the values $h(x),\nabla h(x)$ and $H(x)=\nabla^{2}h(x)$, with low computational efforts. 
%\end{assumption}
%%%%
%\begin{assumption}\label{ass:full_rank}
%The linear operator $\bA$ has full rank: $\image(\bA)=\R^{m}$. 
%\end{assumption}
%%%%%
%Note that this assumption is not restrictive. If the linear operator maps a point $x$ to a lower-dimensional subset, then it is possible to eliminate redundant constraints, or we are working with an inconsistent system $\setL=\emptyset$. The latter is excluded from our considerations, so in fact Assumption \ref{ass:full_rank} is without loss of generality. 
%%%%%%%%%%%%%%%
%\begin{assumption}\label{ass:solution_exist}
%Problem \eqref{eq:Opt} admits a solution. We let $f_{\min}(\setX)\eqdef \argmin\{f(x)\vert x\in\bar{\setX}\}$.
%\end{assumption}
%%%%%%%%%%%%%%%%%%
%\begin{assumption}\label{ass:C1}
%The objective function $f:\setE\to\R$ is continuously differentiable on an open set containing $\setX$.
%\end{assumption}
%%%%%%%%%%%%%%%%%%%
%\begin{assumption}\label{ass:C2}
%The objective function $f:\setE\to\R$ is twice continuously differentiable on an open set containing $\setX$.
%\end{assumption}
%%%%%%%%%%%%%%%%%%%%%
The next definition specifies our notion of an approximate first-order KKT point for problem \eqref{eq:Opt}.
%%%%%%%%%%%
\begin{definition}\label{def:eps_KKT}
%Let Assumptions \ref{ass:1} and \ref{ass:full_rank} hold. 
Given $\eps \geq 0$, we call a triple $(\bar{x},\bar{y},\bar{s})\in\setE\times\R^{m}\times\setE^{\ast}$ an $\eps$-KKT point for problem \eqref{eq:Opt} if
\begin{align}
& \bA\bar{x}=b, \bar{x}\in\setK, \bar{s} \in\setK^{\ast}, \label{eq:eps_optim_equality_cones} \\
&\|\nabla f(\bar{x}) - \bA^{\ast}\bar{y}-\bar{s} \|\leq \eps,  \label{eq:eps_optim_grad} \\
& \inner{\bar{s},\bar{x}} \leq \eps. \label{eq:eps_optim_complem} 
\end{align}
\end{definition}
%%%%%%%%%%%%%%
To justify this definition, let $x^{\ast}$ be a local solution of problem \eqref{eq:Opt}. Then, for $\delta>0$ sufficiently small, the point $x^{\ast}$ is the unique global solution to the perturbed optimization problem with ball restriction $\overline{\ball(x^{\ast};\delta)}\eqdef\{x\in\setE\vert\; \norm{x-x^{\ast}}\leq\delta\}$:
\begin{equation}
\label{eq:limit_optimality_proof_0}
	\min_{x\in\setX\cap\overline{\ball(x^{\ast};\delta)}} f(x) + \frac{1}{4}\norm{x-x^{*}}^4.
\end{equation}
Next, using the barrier $h\in\scrH_{\nu}(\setK)$, we absorb the constraint $x\in\setK$ in the penalty $\mu_k h(x)$, where $\mu_k> 0$, $\mu_{k}\downarrow 0$ is a given sequence. This leads to the barrier formulation
\begin{equation}\label{eq:limit_optimality_proof_1}
\min_{x\in\setX\cap\overline{\ball(x^{\ast};\delta)}}\varphi_{k}(x)\eqdef F_{\mu_{k}}(x)+\frac{1}{4}\norm{x-x^{\ast}}^{4},\quad F_{\mu_{k}}(x)= f(x)+\mu_{k}h(x).
\end{equation}
From the classical theory of interior penalty methods \cite{FiacMcCo68}, it is known that a global solution $x^k$ exists for this problem for all $k$ and that cluster points of $x^k$ are global solutions of \eqref{eq:limit_optimality_proof_0}. Clearly, $x^{k}\in \setX\cap\overline{\ball(x^{\ast},\delta)}$ for all $k$ and $x^{k}\to x^{\ast}$. Setting $s^{k}= -\mu_{k}\nabla h(x^{k})$, which belongs to $\setK^{\ast}$ by eq. \eqref{eq:relations_1}, and exploiting the properties of the barrier function $h\in\scrH_{\nu}(\setK)$, we see that 
$
\inner{s^{k},x^{k}}=-\mu_{k}\inner{\nabla h(x^{k}),x^{k}} \stackrel{\eqref{eq:log_hom_scb_hess_prop}}{=}\mu_{k}\nu.
%=-\mu_{k}\inner{[H(x^{k})]^{-1/2}\nabla h(x^{k}),[H(x^{k})]^{1/2}x^{k}}\\
%&\overset{\eqref{eq:log_hom_scb_hess_prop}}{=}\mu_{k}\inner{[H(x^{k})]^{1/2}x^{k},[H(x^{k})]^{1/2}x^{k}}\\
%&=\mu_{k}\norm{x^{k}}^{2}_{x^{k}}=\mu_{k}\nu.
$
Consequently, $\lim_{k\to\infty}\inner{s^{k},x^{k}}=0$. Since $x^{k}\to x^{\ast}$, the restriction $x^{k}\in\overline{\ball(x^{\ast};\delta)}$ will automatically hold for $k$ sufficiently large. By the full-rank assumption, the first-order optimality conditions of problem \eqref{eq:limit_optimality_proof_1} reads as 
\[
\nabla f(x^{k})-\bA^{\ast}y^{k}-s^{k}-\norm{x^{k}-x^{\ast}}^{2}\cdot(x^{k}-x^{\ast})=0,
\]
for all $k$ large enough. Hence, setting $\delta\leq\eps^{1/3}$, $\mu_k \leq \eps/\nu$, and $\bar{x}=x^{k},\bar{s}=s^{k},\bar{y}=y^{k}$, we obtain a triple satisfying conditions \eqref{eq:eps_optim_equality_cones}-\eqref{eq:eps_optim_complem}.
%%%%%%%%%%%%%%%%%%%%%
%\begin{remark}
%For $\eps=0$ the conditions of Definition \ref{def:eps_KKT} reduce to the exact first-order optimality conditions of \cite{FayLu06}.
%\end{remark}
%%%%%%%%%%%%%%%%%%%%%%%%%
\\

Assuming twice continuous differentiability of $f$ on $\setX$, our notion of an approximate second-order KKT point for problem \eqref{eq:Opt} is defined as follows.
\begin{definition}\label{def:eps_SOKKT}
%Let Assumptions \ref{ass:1}, and \ref{ass:full_rank} hold. 
Given $\eps_1,\eps_2 \geq 0$, we call a triple $(\bar{x},\bar{y},\bar{s})\in\setE\times\R^{m}\times\setE^{\ast}$ an $(\eps_1,\eps_2)$-2KKT point for problem \eqref{eq:Opt} if
\begin{align}
& \bA\bar{x}=b, \bar{x}\in\setK, \bar{s} \in\setK^{\ast}, \label{eq:eps_SO_optim_equality_cones} \\
&\|\nabla f(\bar{x}) - \bA^{\ast}\bar{y}-\bar{s} \|\leq \eps_1,  \label{eq:eps_SO_optim_grad} \\
&\inner{\bar{s},\bar{x}} \leq \eps_1, \label{eq:eps_SO_optim_complem} \\
& \nabla^2f(\bar{x}) + \sqrt{\eps_2} H(\bar{x}) \succeq 0 \;\; \text{on} \;\; \setL_0.   \label{eq:eps_SO_optim_SO}
\end{align}
\end{definition}
The first three conditions are the same as for the $\eps$-KKT point. 
The last one can be justified as follows. Using the full-rank condition, the second-order optimality condition for problem \eqref{eq:limit_optimality_proof_1} says that $x^{k}$ satisfies 
\[
\inner{(\nabla^{2}f(x^{k})+\mu_{k}H(x^{k}))d,d} \geq -2\inner{x^{k}-x^{\ast},d}^{2} - \norm{x^{k}-x^{\ast}}^2\norm{d}^2\geq -3\delta^{2}\norm{d}_{2}^{2}\qquad\forall d\in\setL_{0}.
\]
%says that $x^{k}$ is an isolated local solution to problem \eqref{eq:limit_optimality_proof_1} for $k$ sufficiently large if $\nabla^{2}\varphi_{k}(x^{k})\succ 0$ on $\setL_{0}\eqdef\setL-\setL$. This gives directly
%\[
%\inner{(\nabla^{2}f(x^{k})+\mu_{k}H(x^{k}))d,d}> -2\inner{x^{k}-x^{\ast},d}^{2}> -\delta^{2}\norm{d}_{2}^{2}\qquad\forall d\in\setL_{0}.
%\]
Setting $\mu_k \leq \sqrt{\eps_2}$ and $\delta \leq (\eps_2/9)^{1/4}$, we see that $x^{k}$ satisfies  $\inner{(\nabla^{2}f(x^{k})+\sqrt{\eps_2}(H(x^{k})+\bI))d,d}\geq 0,\forall d\in\setL_{0}$, which is clearly implied by \eqref{eq:eps_SO_optim_SO}. 
%Letting $\delta\to 0$, then for a given $\eps_{2}>0$ we see that condition \eqref{eq:eps_SO_optim_SO} applies for all $k$ sufficiently large. 
%\begin{remark}
%\label{rem:foopt}
%We note that both proposed definitions are stronger than the ones in \cite{HaeLiuYe18}, who use the weaker infinity norm in conditions \eqref{eq:eps_optim_grad} and \eqref{eq:eps_SO_optim_grad}, respectively. Moreover, they also use the weaker infinity norm in conditions \eqref{eq:eps_optim_complem} and \eqref{eq:eps_SO_optim_complem}. 
%Since in our case $\bar{x}\in\setK, \bar{s} \in\setK^{\ast}$, in our case we have for each $i=1,...,\dim(\setE)$, $ 0 \leq \bar{s}_i\bar{x}_i \leq \inner{\bar{s},\bar{x}} \leq \eps$. We discuss this in more details in Section \ref{sec:FO_discussion}, where we compare the complexity bounds for our first-order method and the first-order method in \cite{HaeLiuYe18}, as well as in Section \ref{sec:SO_discussion}, where we compare the complexity bounds for our second-order method and the second-order method in \cite{HaeLiuYe18}. \MS{We talk about this in laters sections anyhow so I think we do not need to put it here.}
%\end{remark}
%%%%%
\begin{remark}\label{rem:soopt}
To compare our second-order condition with the ones previously formulated in the literature, we consider the particular case $\bar{\setK}=\bar{\setK}_{NN}$ as in \cite{HaeLiuYe18,NeiWr20} with the log-barrier setup giving $H(x)=\diag[x_{1}^{-2},\ldots,x_{n}^{-2}]\eqdef\XX^{-2}$. Within this setup, our second-order condition \eqref{eq:eps_SO_optim_SO} becomes, after multiplication by  $[H(x)]^{-1/2}=\XX$ from left and right,
\[
\XX\nabla^{2}f(x)\XX+\sqrt{\eps_{2}}\bI\succeq 0\qquad \text{on the set }\{d\in\setE\vert \bA\XX d=0\}.%\ker(\bA\XX).
\] 
This is equivalent to Proposition 2(c) in \cite{HaeLiuYe18}, modulo our use of $\sqrt{\eps_{2}}$ instead of $\eps$ in \cite{HaeLiuYe18}, as well as equation (1.6d) in \cite{NeiWr20}, modulo our use of $\sqrt{\eps_{2}}$ instead of $\eps_H$ in \cite{NeiWr20}.
%On the other hand, our condition \eqref{eq:eps_SO_optim_SO} does not rely on the coupling $[H(\bar{x})]^{-1/2}=\XX$ which may not hold for general cones and is formulated in terms %of the Hessian $H(x)$ without the use of $[H(\bar{x})]^{-1/2}$ which may be tricky to define.
%
%
%This representation of the second-order optimality condition makes our set of KKT conditions easier to compare with the ones previously formulated in the literature, e.g. in \cite{HaeLiuYe18}. They consider the case $\bar{\setK}=\bar{\setK}_{NN}$ with the log-barrier setup. This gives $H(x)=\diag[x_{1}^{-2},\ldots,x_{n}^{-2}]\eqdef\XX^{-2}$. Within this setup, the second-order KKT condition becomes 
%\[
%\XX\nabla^{2}f(x)\XX+\sqrt{\eps_{2}}\bI\succeq 0\qquad \text{on }\ker(\bA\XX).
%\] 
%This corresponds to Proposition 2(c) in \cite{HaeLiuYe18}, modulo our use of $\sqrt{\eps_{2}}$ instead of $\eps$ in \cite{HaeLiuYe18}, as well as equarion (1.6d) in \cite{NeiWr20}, modulo our use of $\sqrt{\eps_{2}}$ instead of $\eps_H$ in \cite{NeiWr20}.
%
%Condition \eqref{eq:eps_SO_optim_SO} is equivalent to 
%\[
%[H(\bar{x})]^{-1/2}\nabla^2f(\bar{x})[H(\bar{x})]^{-1/2} + \sqrt{\eps_2}\bI \succeq 0 \quad \text{on }\ker(\bA[H(\bar{x})]^{-1/2}).
%\]
\close
\end{remark}
\begin{remark}
If $\bar{\setK}$ is a symmetric cone, the complementarity conditions \eqref{eq:eps_optim_complem} and \eqref{eq:eps_SO_optim_complem} are equivalent to complementarity notions formulated in terms of the multiplication $\circ$ under which $\setK$ becomes an Euclidean Jordan algebra. \citep[][Prop. 2.1]{LouFukMas18} shows that $x\circ y=0$ if and only $\inner{x,y}=0$, where $\inner{\cdot,\cdot}$ is the inner product of the ambient space $\setE$. Moreover, if $\bar{\setK}$ is a primitive symmetric cone, then by \citep[][Prop. III.4.1]{FarKor94}, there exists a constant $a>0$ such that $a\tr(x\circ y)=\inner{x,y}$ for all $x,y\in\setK$. In view of this relation, our complementarity notions could be specialized to the condition $\bar{s}\circ \bar{x}\leq \eps$. Hence, our approximate KKT conditions reduce to the ones reported in \cite{AndFukHaeSanSec21}. In particular, for $\bar{\setK}=\bar{\setK}_{\text{NN}}$ we recover the standard complementary slackness condition $s^{k}_{i}x^{k}_{i}\to 0$ as $k\to\infty$ for all $i$, as in this case the Jordan product $\circ$ gives rise to the Hadamard product. See \cite{Andreani:2019uf} for more details.
\close
\end{remark}

\subsection{On the relation to scaled critical points}
In absence of differentiability at the boundary, a popular formulation of necessary optimality conditions involves the definition of scaled-critical points. Indeed, at a local minimizer $x^{\ast}$, the scaled first-order optimality condition $x_{i}^{\ast}[\nabla f(x^{\ast})]_{i}=0,1\leq i\leq n$ holds, where the product is taken to be $0$ when the derivative does not exist. Based on this characterization, one may call a point $x\in\setK_{\text{NN}}$ with $\abs{x_{i}[\nabla f(x)]_{i}}\leq\eps$ for all $i=1,\ldots,n$ and $\eps$-scaled first-order point. Algorithms designed to produce $\eps$-scaled first-order points, with some small $\eps>0$, have been introduced in \cite{BiaCheYe15} and \cite{BiaChe15}. As reported in \cite{HaeLiuYe18}, there are several problems associated with this weak definition of a critical point. First, when derivatives are available on $\bar{\setK}_{\text{NN}}$, the standard definition of a critical point would entail the inclusion $\inner{\nabla f(x),x'-x}\geq 0$ for all $x'\in \bar{\setK}_{\text{NN}}.$ Hence, $[\nabla f(x)]_{i}=0$ for $x_{i}>0$ and $[\nabla f(x)]_{i}\geq 0$ for $x_{i}=0$. It follows, $\nabla f(x)\in\bar{\setK}_{\text{NN}}$, a condition that is absent in the definition of a scaled critical point. Second, scaled critical points come with no measure of strength, as they holds trivially when $x=0$, regardless of the objective function. Third, there is a general gap between local minimizers and limits of $\eps$-scaled first-order points, when $\eps\to 0^{+}$ (see \cite{HaeLiuYe18}). Similar remarks apply to the scaled second-order condition, considered in \cite{BiaChe15}. Our definition of approximate KKT points overcome these issues. In fact, our definitions of approximate first- and second-order KKT points is continuous in $\eps$, and therefore in the limit our approximate KKT points coincide with the classical first- and second-order KKT conditions for a local minimizer. This is achieved without assuming global differentiability of the objective function or performing an additional smoothing of the problem data as in \cite{Bian:2013vd,BiaChe15}. 






%%%%%%%%%%%%%%%%%%%%%%%%%%%%%%%%%%%%%%%%%%%%%%
%%%%%%ALGORITHM%%%%%%%%%%%%%

\section{A first-order Hessian-Barrier Algorithm}
\label{sec:firstorder}
%----------------------------------------------------------------------
%%% First-Order ALGORITHM
%----------------------------------------------------------------------
% !TEX root = ./HBAConicMain.tex

In this section we introduce a first-order potential reduction method for solving \eqref{eq:Opt} that uses a barrier $h \in\scrH_{\nu}(\setK)$ and potential function \eqref{eq:potential}. %Our first-order method employs a quadratic regularization strategy for the linearization of $F_{\mu}(x)$ using the local norm at the current position $x$. 
%We are thus following classical numerical optimization ideas, related to the Levenberg–Morrison–Marquardt techniques \cite{NocWri00}, and also more recently used in \cite{CarGouToi11}. \PD{I'm afraid that this will give an impression that our results are simple and classical.}
We assume that we are able to compute an approximate analytic center at low computational cost. Specifically, our algorithm relies on the availability of a $\nu$-analytic center, i.e. a point $x^{0}\in\setX$ such that 
\begin{equation}\label{eq:analytic_center}
h(x)\geq h(x^{0})-\nu\qquad\forall x\in\setX. 
\end{equation}
To obtain such a point $x^{0}$, one can apply interior point methods to the convex programming problem $\min_{x\in \feas}h(x)$. Moreover, since $\nu \geq 1$ we do not need to solve it with high precision, making the application of computationally cheap first-order method, such as \cite{Dvurechensky:2022tu}, an appealing choice for this preprocessing step. 

\subsection{Local properties}
Given $x\in\setX$, define the set of \emph{feasible directions} as $\scrF_{x}=\{v\in\setE\vert x+v\in\feas\}.$ Lemma \ref{lem:Dikin} implies that 
\begin{equation}\label{eq:Dikinv}
\scrT_{x}=\{v\in\setE\vert \bA v=0,\norm{v}_{x}<1\}\subseteq\scrF_{x}.
\end{equation}
Upon defining $d=[H(x)]^{1/2}v$ for $v\in\scrT_{x}$, we obtain a point  $d \in \R^{{\rm dim}(\setE)}$ satisfying $\bA[H(x)]^{-1/2}d=0$ and $\norm{d}=\norm{v}_{x}$. Hence, for $x\in\setK$, we can equivalently characterize the set $\scrT_{x}$ as $\scrT_{x}=\{[H(x)]^{-1/2}d\vert \bA[H(x)]^{-1/2}d=0,\norm{d}<1\}$.\\ 
Our complexity analysis relies on the ability to control the behavior of the objective function along the set of feasible directions and with respect to the local norm. 
%%%%%%%%%%%%%%%%%%%%%%%%
\begin{assumption}[Local smoothness]
\label{ass:gradLip}
$f:\setE\to\R\cup\{+\infty\}$ is continuously differentiable on $\feas$ and there exists a constant $M>0$ such that for all $x\in\feas$ and $v\in\scrT_{x}$ we have 
\begin{equation}\label{eq:gradLip}
f(x+v) - f(x) - \inner{\nabla f(x),v} \leq \frac{M}{2}\norm{v}_x^2.
\end{equation}
\end{assumption}
\noindent
\begin{remark}
\label{rem:bounded1}
If the set $\bar{\setX}$ is bounded, we have $\lambda_{\min}(H(x)) \geq \sigma$ for some $\sigma >0$. In this case, assuming $f$ has an $M$-Lipschitz continuous gradient, the classical descent lemma \cite{Nes18} implies Assumption \ref{ass:gradLip}. Indeed,
\[
f(x+v) - f(x) - \inner{\nabla f(x),v} \leq \frac{M}{2}\norm{v}^2 \leq \frac{M}{2\sigma}\norm{v}_x^2.
\]
\close
\end{remark}
\begin{remark}
We emphasize that the local Lipschitz smoothness condition \eqref{eq:gradLip} does not require global differentiability. Consider the composite non-smooth and non-convex model \eqref{eq:composite} on $\bar{\setK}_{\text{NN}}$, with $\varphi(s)=s$ for $s \geq 0$. This means $\sum_{i=1}^{n}\varphi(x_{i}^{p})=\norm{x}_{p}^{p}$ for $p\in(0,1)$ and $x\in\bar{\setK}_{\text{NN}}$. As a concrete example for the smooth part of the problem let us consider the $L_{2}$-loss $\ell(x)=\frac{1}{2}\norm{\bN x-\bp}^{2}$. This gives rise to the $L_{2}-L_{p}$ minimization problem, an important optimization formulation arising in phase retrieval, mathematical statistics, signal processing and image recovery \cite{Fou09, GeJiaYe11,Chen:2014wx,LiLiuYaoYe17}. For $x\in\setK_{\text{NN}}$, set $M=\lambda_{\max}(\bN^{\ast}\bN)$, so that 
\[
\ell(x^{+})\leq \ell(x)+\inner{\nabla\ell(x),x^{+}-x}+\frac{M}{2}\norm{x^{+}-x}^{2},
\]
Since $t\mapsto t^{p}$ is concave for $t>0$ and $p\in(0,1)$, we have 
\[
(x_{i}^{+})^{p}\leq  x^{p}_{i}+px_{i}^{p-1}(x^{+}_{i}-x_{i})\qquad i=1,\ldots,n.
\]
Adding all these inequalities together, we immediately arrive at condition \eqref{eq:gradLip} in terms of the Euclidean norm. Over a bounded feasible set $\bar{\setX}$, Remark \ref{rem:bounded1} makes it clear that this implies Assumption \ref{ass:gradLip}. At the same time, $f$ is not differentiable at zero. \close
\end{remark}
We  emphasize that in Assumption \ref{ass:gradLip} the constant $M$ is in general either unknown or is a very conservative upper bound. Therefore, adaptive techniques should be used to estimate it and are likely to improve the practical performance of the method. 

Considering $x\in\setX,v\in\scrT_{x}$ and combining eq. \eqref{eq:gradLip} with eq. \eqref{eq:Dbound} (with $d=v$ and $t=1< \frac{1}{\norm{v}_x} \stackrel{\eqref{eq:boundzeta}}{\leq} \frac{1}{\zeta(x,v)}$) reveals a suitable quadratic model, to be used in the design of our first-order algorithm.
\begin{lemma}[Quadratic Overestimation]
For all $x\in\setX,v\in\scrT_{x}$ and $L\geq M$, we have 
\begin{equation}\label{eq:descentFOM}
F_{\mu}(x+v)\leq F_{\mu}(x)+\inner{\nabla F_{\mu}(x),v}+\frac{L}{2}\norm{v}^{2}_{x}+\mu\norm{v}^{2}_{x}\omega(\zeta(x,v)).
\end{equation}
\end{lemma}


\subsection{Algorithm description and its complexity}
\label{S:FO_descr}
Let $x \in \feas$ be given. Our first-order method employs a quadratic model $ Q^{(1)}_{\mu}(x,v)$ to compute a search direction $v_{\mu}(x)$, given by 
\begin{equation}\label{eq:search}
v_{\mu}(x) \eqdef \argmin_{v\in\setE:\bA v=0} \left\{  Q^{(1)}_{\mu}(x,v) \eqdef F_{\mu}(x) + \inner{\nabla F_{\mu}(x),v}+\frac{1}{2}\norm{v}_{x}^{2} \right\}.
\end{equation}
%\begin{equation}\label{eq:model1}
%Q^{(1)}_{\mu}(x,v) \eqdef F_{\mu}(x) + \inner{\nabla F_{\mu}(x),v}+\frac{1}{2}\norm{v}_{x}^{2}.
%\end{equation}
%We construct a search direction $v_{\mu}(x)$ as the solution of the strongly convex subproblem 
For the above problem, we have the following system of optimality conditions involving the dual variable $y_{\mu}(x)\in\R^{m}$:
\begin{align}
\nabla F_{\mu}(x) + H(x)v_{\mu}(x) - \bA^{\ast} y_{\mu}(x) &= 0, \label{eq:finder_1} \\
\bA  v_{\mu}(x) &=0. \label{eq:finder_2}
\end{align}
Since $H(x)\succ 0$ for $x\in\feas$, any standard solution method \citep{NocWri00} can be applied for the above linear system.
Moreover, this system can be solved explicitly.
Indeed, since $H(x)\succ 0$ for $x\in\feas$, and $\bA$ has full column rank, the linear operator $\bA[H(x)]^{-1}\bA^{\ast}$ is invertible. Hence, $v_{\mu}(x)$  is given explicitly as
\begin{equation*}%\label{eq:v_explicit}
v_{\mu}(x)= - ([H(x)]^{-1}\bA^{\ast}(\bA[H(x)]^{-1}\bA^{\ast})^{-1}\bA[H(x)]^{-1} - [H(x)]^{-1} ) \nabla F_{\mu}(x) \eqdef-\bS_{x}\nabla F_{\mu}(x).
\end{equation*}
To give some intuition behind this expression, observe that we can give an alternative representation of $\bS_{x}$ as $\bS_{x}v = [H(x)]^{-1/2}\Pi_{x}[H(x)]^{-1/2}v$, where
\[
\Pi_{x}v\eqdef v-[H(x)]^{-1/2}\bA^{\ast}(\bA[H(x)]^{-1}\bA^{\ast})^{-1}\bA[H(x)]^{-1/2}v.
\]
This shows that $\bS_{x}$ is just the $\norm{\cdot}_{x}$-orthogonal projection operator onto $\ker(\bA[H(x)]^{-1/2})$. Hence, we can always find a scalar $t>0$ such that $t v_{\mu}(x)\in\setL_{0}$ and $\norm{t v_{\mu}(x)}_{x}<1$. Any such scalar will be a suitable candidate for a step size. To determine an acceptable step-size, consider a point $x \in\setX$, the  search direction $v_{\mu}(x)$ gives rise to a family of parameterized arcs $x^{+}(t)\eqdef x+tv_{\mu}(x)$, where $t\geq 0$. Our aim is to choose this step-size to ensure feasibility of the iterates and decrease of the potential. By \eqref{eq:step_length_zeta} and \eqref{eq:finder_2}, we know that $x^{+}(t)\in\feas$ for all $t\in I_{x,\mu} \eqdef [0,\frac{1}{\zeta(x,v_{\mu}(x))})$. Multiplying \eqref{eq:finder_1} by $v_{\mu}(x)$ and using \eqref{eq:finder_2}, we obtain 
$\inner{\nabla F_{\mu}(x),v_{\mu}(x)}=-\norm{v_{\mu}(x)}_{x}^{2}$. Choosing $t \in  I_{x,\mu}$, we bound
\[
t^{2}\norm{v_{\mu}(x)}_{x}^{2}\omega(t\zeta(x,v_{\mu}(x))) \stackrel{\eqref{eq:omega_upper_bound}}{\leq}  \frac{t^{2}\norm{v_{\mu}(x)}_{x}^{2}}{2(1-t\zeta(x,v_{\mu}(x)))}. 
\]
Therefore, if $t\zeta(x,v_{\mu}(x))\leq 1/2$, we readily see from \eqref{eq:descentFOM} that
%\footnote{Note that since $t\zeta(x,v_{\mu}(x))=\zeta(x,tv_{\mu}(x))$ and we assumed that $t\zeta(x,v_{\mu}(x))\leq 1/2$, it is sufficient to make Assumption \ref{ass:gradLip} on a potentially smaller set $\widetilde{\scrT}_{x}\eqdef \{v\in\setE\vert \bA v=0,\zeta(x,v)\leq 1/2\}$ instead of $\scrT_x$ defined in \eqref{eq:Dikinv}.} 
\begin{align}
F_{\mu}(x^{+}(t))-F_{\mu}(x)&\leq -t\norm{v_{\mu}(x)}_{x}^{2}+\frac{t^{2}M}{2}\norm{v_{\mu}(x)}_{x}^{2}+\mu t^{2}\norm{v_{\mu}(x)}_{x}^{2} \nonumber\\
&= -t \norm{v_{\mu}(x)}_{x}^{2}\left(1-\frac{M+2\mu}{2}t\right) \eqdef -\eta_{x}(t).\label{eq:success}
\end{align}
The function $t \mapsto \eta_{x}(t)$ is strictly concave with the unique maximum at $ \frac{1}{M+2\mu}$, and two real roots at $t\in\left\{0,\frac{2}{M+2\mu}\right\}$. 
Thus, maximizing the per-iteration decrease $\eta_{x}(t)$  under the restriction $0\leq t\leq\frac{1}{2\zeta(x,v_{\mu}(x))}$, we choose the step-size
\begin{equation*}
%\label{eq:}
\ct_{\mu,M}(x)\eqdef \min \left\{\frac{1}{M+2\mu},\frac{1}{2\zeta(x,v_{\mu}(x))}\right\}.
\end{equation*}
This step-size rule, however, requires knowledge of the parameter $M$. To boost numerical performance, we employ a backtracking scheme in the spirit of \cite{NesPol06} to estimate the constant $M$ at each iteration. This procedure generates a sequence of positive numbers $(L_{k})_{k\geq 0}$ for which the local Lipschitz smoothness condition \eqref{eq:gradLip} holds. More specifically, suppose that $x^{k}$ is the current position of the algorithm with the corresponding initial local Lipschitz estimate $L_{k}$ and $v^{k}=v_{\mu}(x^{k})$ is the corresponding search direction. To determine the next iterate $x^{k+1}$, we iteratively try step-sizes $\alpha_k$ of the form $\ct_{\mu,2^{i_k}L_k}(x^{k})$ for $i_k\geq 0$ until the local smoothness condition \eqref{eq:gradLip} holds with $x=x^{k}$, $v= \alpha_k v^{k}$ and local Lipschitz estimate $M=2^{i_k}L_k$, see \eqref{eq:LS}. This process must terminate in finitely many steps, since when $2^{i_k}L_k \geq M$, inequality \eqref{eq:gradLip} with $M$ changed to $2^{i_k}L_k$, i.e., \eqref{eq:LS}, follows from Assumption \ref{ass:gradLip}. Combining the search direction finding problem \eqref{eq:search} with the just outlined backtracking strategy, yields an \underline{A}daptive first-order \underline{H}essian-\underline{B}arrier \underline{A}lgorithm ($\AHBA$, Algorithm \ref{alg:AHBA}).
%%%%%%%%%%%%%%%%%%%%%%%%%%%%%%%%
\begin{algorithm}[t]
\caption{ \underline{A}daptive first-order \underline{H}essian-\underline{B}arrier \underline{A}lgorithm  - $\AHBA(\mu,\eps,L_{0},x^{0})$}
\label{alg:AHBA}
\SetAlgoLined
\KwData{ $h \in\scrH_{\nu}(\setK)$, $\mu>0,\eps>0,L_0>0,x^{0}\in\setX$.
}
\KwResult{$(x^{k},y^{k},s^{k},L_{k})\in\setX\times\R^{m}\times\setK^{\ast}\times\R_{+}$, where $s^{k}=\nabla f(x^{k}) -\bA^{\ast}y^{k}$, and $L_{k}$ is the last estimate of the Lipschitz constant.}
%Set $L_0 > 0$ -- initial guess for $M$, 
Set $k=0$\;
\Repeat{
		$\norm{v^k}_{x^k} < \tfrac{\eps}{\nu}$ 
	}{	
		Set $i_k=0$. Find $v^k\eqdef v_{\mu}(x^k)$ and  the corresponding dual variable $y^k\eqdef y_{\mu}(x^k)$ as the solution to
		\begin{equation}\label{eq:finder}
		\min_{v\in\setE:\bA v=0}\{F_{\mu}(x^k) + \inner{\nabla F_{\mu}(x^k),v}+\frac{1}{2}\norm{v}_{x^{k}}^{2}\}. 
		\end{equation}
		\Repeat{
			\begin{equation}				
				f(z^{k}) \leq f(x^{k}) + \inner{\nabla f(x^{k}),z^{k}-x^{k}}+2^{i_{k}-1}L_{k}\norm{z^{k}-x^{k}}^{2}_{x^{k}}.
				\label{eq:LS}
			\end{equation}
		}
		{
			\begin{equation}\label{eq:alpha_k}
				\alpha_k \eqdef  \min \left\{\frac{1}{2^{i_k}L_{k} + 2 \mu},\frac{1}{2\zeta(x^k,v^k)} \right\},  	\text{where $\zeta(\cdot,\cdot)$ as in \eqref{eq:zeta}}	
			\end{equation}
			%%%%%%%%%%%%%%%
			Set $z^{k}=x^{k} + \alpha_k v^k$, $i_k=i_k+1$\;%, $i_k=i_k+1$\;
		}
		Set $L_{k+1} = 2^{i_k-1}L_{k}$, $x^{k+1}=z^{k}$, $k=k+1$\;%i_k-2 is since when we find a suitable $i_k$ we still set $i_k=i_k+1$
	}
\end{algorithm}
%%%%%%%%%%%%%%%%%%%%%%%%%%%%%%%
Our main result on the iteration complexity of Algorithm \ref{alg:AHBA} is the following Theorem, whose proof is given in Section \ref{sec:ProofFOM}. 
\begin{theorem}
\label{Th:AHBA_conv}
Let Assumptions \ref{ass:1}-\ref{ass:gradLip} hold. Fix the error tolerance $\eps>0$, the regularization parameter $\mu=\frac{\eps}{\nu}$, and some initial guess $L_0>0$ for the Lipschitz constant. Let $(x^{k})_{k\geq 0}$ be the trajectory generated by $\AHBA(\mu,\eps,L_{0},x^{0})$, where $x^{0}$ is a $\nu$-analytic center satisfying \eqref{eq:analytic_center}. Then the algorithm stops in no more than 
\begin{equation}
\label{eq:FO_main_Th_compl}
\K_{I}(\eps,x^{0})= \ceil[\bigg]{4(f(x^{0}) - f_{\min}(\setX)+ \eps) \frac{\nu^{2}(\max\{M,L_0\}+\eps/\nu)}{\eps^{2}}}
\end{equation}
outer iterations, and the number of inner iterations is no more than $2(\K_{I}(\eps,x^{0})+1)+\max\{\log_{2}(M/L_{0}),0\}$. Moreover, the last iterate obtained from $\AHBA(\mu,\eps,L_{0},x^{0})$ constitute a $2\eps$-KKT point for problem \eqref{eq:Opt} in the sense of Definition \ref{def:eps_KKT}.
%\begin{align}
%&\|\nabla f(x^{k}) - \bA^{\ast}y^{k} - s^{k} \| = 0 \leq 2\eps, \label{eq:FO_main_Th_eps_KKT_1} \\
%& |\inner{s^{k},x^{k}}| \leq 2\eps, \label{eq:FO_main_Th_eps_KKT_2}  \\
%& \bA x^{k}=b, s^{k} \in \setK^{\ast}, x^{k}\in\setK.   \label{eq:FO_main_Th_eps_KKT_3}
%\end{align}  
\end{theorem}
 %%%%%%%%
\begin{remark}
The line-search process of finding the appropriate $i_k$ is simple since only recalculating $z^k$ is needed, and repeatedly solving problem \eqref{eq:finder} is not required. Furthermore, the sequence of constants $L_k$ is allowed to decrease along subsequent iterations, which is achieved by the division by the constant factor 2 in the final updating step of each iteration. This potentially leads to longer steps and faster decrease of the potential.
\close
\end{remark}
%%%%%%%%%%%%%%%%%
\begin{remark}
\label{rem:FO_complexity_simplified}
Since $\nu \geq 1$, $f(x^{0}) - f_{\min}(\setX)$ is expected to be larger than $\eps$, and the constant $M$ is potentially large, we see that the main term in the complexity bound \eqref{eq:FO_main_Th_compl} is $O\left(\frac{M\nu^2(f(x^{0}) - f_{\min}(\setX))}{\eps^2}\right)=O(\frac{\nu^{2}}{\eps^{2}})$, i.e. has the same dependence on $\eps$ as the standard complexity bounds  \cite{CarDucHinSid19b,CarDucHinSid19,lan2020first} of first-order methods for non-convex problems under the standard Lipschitz-gradient assumption, which on bounded sets is subsumed by our Assumption \ref{ass:gradLip}. Further, if the function $f$ is quadratic, Assumption \ref{ass:gradLip} holds with $M=0$ and we can take $L_0=0$. In this case, the complexity bound \eqref{eq:FO_main_Th_compl} improves to $O\left(\frac{\nu(f(x^{0}) - f_{\min}(\setX))}{\eps}\right)$. 

Just like classical interior-point methods, the iteration complexity of $\AHBA$ depends on the barrier parameter $\nu\geq 1$. For conic domains, the characterization of this barrier parameter has thus been an active research line. \cite{GulTun98} demonstrated that for symmetric cones, the barrier parameter is equivalent to algebraic properties of the cone and identified it with the rank of the cone (see \cite{FarKor94} for a definition of the rank of a symmetric cone). This deep analysis gives an exact characterization of the optimal barrier parameter for the most important conic domains in optimization. For $\setK_{\text{NN}}$ and $\setK_{\text{SDP}}$, it is known that $\nu=n$ is optimal, whereas for $\setK_{\text{SOC}}$ the optimal barrier parameter is $\nu=2$ (and therefore independent of the ambient dimension $n$). 
\close
\end{remark}
%\begin{remark}\label{rem:KKT1st}
%Consider the special case of problem \eqref{eq:Opt} with non-negativity constraints, i.e., with $\bar{\setK}=\bar{\setK}_{\text{NN}}$. In this case, our algorithm generates $(x^{k},y^{k},s^{k})$ satisfying $x^{k}\in\setK_{\text{NN}},s^{k}=\nabla f(x^k) -\bA^{\ast}y^{k}\in\setK_{\text{NN}}$, i.e., $x^{k},s^{k}\geq 0$, and approximate complementary slackness $\sum_{i=1}^{n}\abs{x_{i}^{k}s_{i}^{k}}=\sum_{i=1}^{n}x_{i}^{k}s_{i}^{k}\leq \eps$ after $O\left(\frac{M \nu^2(f(x^{0}) - f_{\min}(\setX))}{\eps^2}\right)$ iterations. 
%A natural alternative formulation of the approximate complementary slackness condition is $\abs{\hat{x}_{i}\hat{s}_{i}}\leq \delta$ for all $i=1,\ldots,n$, or equivalently $\max_{1\leq i\leq n}\abs{\hat{x}_{i}\hat{s}_{i}}\leq \delta$, where $\hat{s}=\nabla f(\hat{x})-\bA^{\ast}\hat{y}$. This condition is imposed in \cite{HaeLiuYe18}, where, under an assumption similar to our Assumption \ref{ass:gradLip}, they guarantee that $\max_{1\leq i\leq n}\abs{\hat{x}_{i}\hat{s}_{i}}\leq \delta$ in $O\left(\frac{M (f(x^{0}) - f_{\min}(\setX))}{\delta^2}\right)$ iterations. We see that our notion of complementarity is stronger since $\max_{1\leq i\leq n}\abs{\hat{x}_{i}\hat{s}_{i}} \leq \sum_{i=1}^{n}\abs{\hat{x}_{i}\hat{s}_{i}} \leq n \max_{1\leq i\leq n}\abs{\hat{x}_{i}\hat{s}_{i}}$, where both equalities are achievable. Moreover,  to match our stronger guarantee, one has to take $\delta=\eps/n$ in the algorithm of \cite{HaeLiuYe18}, which leads to complexity $O\left(\frac{M n^2(f(x^{0}) - f_{\min}(\setX))}{\eps^2}\right)$ similar to ours since in this case $\nu=n$.	Further benefit of our algorithm is that it is designed for general cones, rather than only for $\bar{\setK}_{\text{NN}}$.
%%
%%
%%For problem \eqref{eq:Opt} on the non-negative orthant $\bar{\setK}_{\text{NN}}$, a natural formulation of the approximate complementary slackness condition is $\abs{\hat{x}_{i}[\nabla f(\hat{x})-\bA^{\ast}\hat{y}]_{i}}\leq \delta$ for all $i=1,\ldots,n$. This condition is imposed in \cite{HaeLiuYe18}, where, under the assumption similar to our Assumption \ref{ass:gradLip}, they guarantee that $\max_{1\leq i\leq n}\abs{\hat{x}_{i}[\nabla f(\hat{x})-\bA^{\ast}\hat{y}]_{i}}\leq \delta$ in $O\left(\frac{M (f(x^{0}) - f_{\min}(\setX))}{\eps^2}\right)$ iterations. In the same setting, our notion of complementarity is stronger. Indeed, since our first-order $\eps$-KKT point $(x^{k},y^{k},s^{k})$ satisfies $x^{k}\in\setK_{\text{NN}},s^{k}=\nabla f(x^k) -\bA^{\ast}y^{k}\in\setK_{\text{NN}}$, and $\sum_{i=1}^{n}x_{i}^{k}[\nabla f(x^k) -\bA^{\ast}y^{k}]_{i}\leq \eps$ after $O\left(\frac{M \nu^2(f(x^{0}) - f_{\min}(\setX))}{\eps^2}\right)$. Since $\sum_{i=1}^{n}x_{i}^{k}[\nabla f(x^k) -\bA^{\ast}y^{k}]_{i}\leq n \max_{1\leq i\leq n}\abs{\hat{x}_{i}[\nabla f(\hat{x})-\bA^{\ast}\hat{y}]_{i}}\leq \delta$ (and the equality is achievable), we have that to match our guarantee, one has to take $\delta=\eps/n$ in the algorithm of \cite{HaeLiuYe18} which leads to complexity $O\left(\frac{M n^2(f(x^{0}) - f_{\min}(\setX))}{\eps^2}\right)$ similar to ours since in this case $\nu=n$.	
%%
%%
%%in terms of the uniform norm $\max_{1\leq i\leq n}\abs{x_{i}[\nabla f(x)-\bA^{\ast}y]_{i}}\leq\eps$. In the worst case this could mean that $x_{i}[\nabla f(x)-\bA^{\ast}y]_{i}=\eps$ for all $i=1,\ldots,n$. Our complementarity measure uses the inner product between primal and dual variables. Hence, on the non-negative orthant, our first-order $\eps$-KKT point $(x^{k},y^{k},s^{k})$ satisfies $x^{k}\in\setK_{\text{NN}},s^{k}=\nabla f(x^k) -\bA^{\ast}y^{k}\in\setK_{\text{NN}}$, and $\sum_{i=1}^{n}x_{i}^{k}[\nabla f(x^k) -\bA^{\ast}y^{k}]_{i}\leq \eps$. Since $n\max_{1\leq i\leq n}\abs{x^{k}_{i}[\nabla f(x^k) -\bA^{\ast}y^{k}]_{i}}\leq \sum_{i=1}^{n}x^{k}_{i}[\nabla f(x^k) -\bA^{\ast}y^{k}]_{i}$, we see that our definition is stronger by a factor $1/n$.
%%
%%
%%For problem \eqref{eq:Opt} on the non-negative orthant $\bar{\setK}_{\text{NN}}$, a natural formulation of the approximate complementary slackness condition is $\abs{x_{i}[\nabla f(x)-\bA^{\ast}y]_{i}}\leq \eps$ for all $i=1,\ldots,n$. This condition is imposed in \cite{HaeLiuYe18} in terms of the uniform norm $\max_{1\leq i\leq n}\abs{x_{i}[\nabla f(x)-\bA^{\ast}y]_{i}}\leq\eps$. In the worst case this could mean that $x_{i}[\nabla f(x)-\bA^{\ast}y]_{i}=\eps$ for all $i=1,\ldots,n$. Our complementarity measure uses the inner product between primal and dual variables. Hence, on the non-negative orthant, our first-order $\eps$-KKT point $(x^{k},y^{k},s^{k})$ satisfies $x^{k}\in\setK_{\text{NN}},s^{k}=\nabla f(x^k) -\bA^{\ast}y^{k}\in\setK_{\text{NN}}$, and $\sum_{i=1}^{n}x_{i}^{k}[\nabla f(x^k) -\bA^{\ast}y^{k}]_{i}\leq \eps$. Since $n\max_{1\leq i\leq n}\abs{x^{k}_{i}[\nabla f(x^k) -\bA^{\ast}y^{k}]_{i}}\leq \sum_{i=1}^{n}x^{k}_{i}[\nabla f(x^k) -\bA^{\ast}y^{k}]_{i}$, we see that our definition is stronger by a factor $1/n$.
%\close
%\end{remark}

\paragraph{Connection with interior point flows on polytopes.}
Consider $\bar{\setK}=\bar{\setK}_{\text{NN}}$, and $\setX=\setK_{\text{NN}}\cap\setL$. We are given a function $f:\bar{\setX}\to\R$ which is the restriction of a smooth function $f:\Rn\to\R$. %Let $\nabla f(x)=[\partial_{1}f(x),\ldots,\partial_{n}f(x)]^{\top}$ denote the gradient in $\Rn$ at $x\in\setX$. 
The canonical barrier for this setting is $h(x)=-\sum_{i=1}^{n}\ln(x_{i})$, so that $H(x)=\diag[x_{1}^{-2},\ldots,x_{n}^{-2}]=\XX^{-2}$ for $x\in\setX$. Applying our first-order method on this domain gives the search direction 
$v_{\mu}(x)=-\bS_{x}\nabla F_{\mu}(x)=-\XX(\bI-\XX\bA^{\top}(\bA\XX^{2}\bA^{\top})^{-1}\bA\XX)\XX\nabla F_{\mu}(x)$. This explicit formula yields various interesting connections between our approach and classical methods. For $\bA=\1_{n}^{\top}$, the feasible set $\setX$ reduces to the relative interior of the $(n-1)$-dimensional unit simplex. In this case, the vector field $v_{\mu}(\cdot)$ simplifies further to 
\[
[v_{\mu}(x)]_{i}=[\XX^{2}\nabla F_{\mu}(x)]_{i}-\frac{x_{i}^{2}}{\sum_{j=1}^{n}x_{j}^{2}}\sum_{j}[\XX^{2}\nabla F_{\mu}(x)]_{j} \quad 1\leq i\leq n,
\]
Observe that $v_{\mu}(x)\in (\1_{n})^{\bot}=\ker(\1_{n}^{\top})$. For $f(x)=c^{\top}x$ and $\mu=0$, we further obtain from this formula the search direction employed in \emph{affine scaling} methods for linear programming \cite{BayLag89,BayLag89II,AdlMont91,TseBomSch11}. 
%\[
%\bS_{x}c=\XX^{2} c-\XX^{2}\bA^{\top}(\bA\XX^{2}\bA^{\top})^{-1}\bA\XX^{2}c.
%\]
%This map is the search direction in \emph{affine scaling} methods for linear programming \cite{BayLag89,BayLag89II,AdlMont91}, a fundamental class of interior-point methods. Extensions to quadratic programming have been studied in various papers, notably by \cite{Ye92,Tse04,TsuMon96,Ye98}.
%
\cite{HBA-linear} partly motivated their algorithm as a discretization of the Hessian-Riemannian gradient flows introduced in \cite{ABB04} and \cite{BolTeb03}. Heuristically, we can therefore interpret $\AHBA$ as an Euler discretization (with non-monotone adaptive step-size policies) of the gradient-like flow $\dot{x}(t)=-\bS_{x(t)}\nabla F_{\mu}(x(t))$, which resembles very much the class of dynamical systems introduced in \cite{BolTeb03}. This gives an immediate connection to a large class of interior point flows on polytopes, heavily studied in control theory \cite{HelMoo96}.
%%Proof%%%
\subsection{Proof of Theorem \ref{Th:AHBA_conv}}
\label{sec:ProofFOM}
Our proof proceeds in several steps. First, we show that procedure $\AHBA(\mu,\eps,L_{0},x^{0})$ produces points in $\setX$, and, thus, is indeed an interior-point method. Next, we show that the line-search process of finding appropriate $L_k$ in each iteration is finite, and estimate the total number of trials in this process. Then we enter the core of our analysis where we prove that if the stopping criterion does not hold at iteration $k$, i.e. $\norm{v^{k}}_{x^{k}} \geq \tfrac{\eps}{\nu}$, then the objective $f$ is decreased by a quantity $O(\eps^{2})$, and, since the objective is globally lower bounded, we conclude that the method stops in at most $O(\eps^{-2})$ iterations. Finally, we show that when the stopping criterion holds, the method has generated an $\eps$-KKT point. 

\subsubsection{Interior-point property of the iterates}
\label{S:FO_correct}
By construction $x^{0}\in\setX$. Proceeding inductively, let $x^{k}\in\setX$ be the $k$-th iterate of the algorithm, delivering the search direction $v^{k}=v_{\mu}(x^{k})$. 
By eq. \eqref{eq:alpha_k}, the step-size $\alpha_k$ satisfies $\alpha_{k}\leq \frac{1}{2\zeta(x^{k},v^{k})}$, and, hence, $\alpha_{k}\zeta(x^{k},v^{k})\leq 1/2$ for all $k\geq 0$. 
Thus, by \eqref{eq:step_length_zeta} $x^{k+1}=x^{k}+\alpha_{k}v^{k} \in\setK$. Since, by \eqref{eq:finder}, $\bA v^{k} =0$, we have that $x^{k+1} \in \setL$. Thus, $x^{k+1}\in\setK \cap \setL=\setX$. By induction, we conclude that $(x^{k})_{k\geq 0}\subset\setX$. 

\subsubsection{Bounding the number of backtracking steps}
\label{sec:backtrack1}
Let us fix iteration $k$. Since the sequence $2^{i_k} L_k $ is increasing as $i_k$ is increasing, and Assumption \ref{ass:gradLip} holds, we know that when $2^{i_k} L_k \geq \max\{M,L_k\}$, the line-search process for sure stops since inequality \eqref{eq:LS} holds. 
%Hence, if $L_0 \leq M$, we have that $2^{i_k} L_k \leq 2M$. Otherwise, if $L_0 > M$, we have 
Hence, $2^{i_k} L_k \leq 2\max\{M,L_k\}$ must be the case, and, consequently, $L_{k+1} = 2^{i_k-1} L_k \leq \max\{M,L_k\}$, which, by induction, gives $L_{k+1} \leq \bar{M}\eqdef\max\{M,L_0\}$. 
%By construction we have $L_{k+1}=\frac{1}{2}2^{i_{k}}L_{k}$. 
At the same time, $\log_{2}\left(\frac{L_{k+1}}{L_{k}}\right)= i_{k}-1$, $\forall k\geq 0$. Let $N(k)$ denote the number of inner line-search iterations up to the $k-$th iteration of $\AHBA(\mu,\eps,L_{0},x^{0})$. Then, using that $L_{k+1} \leq \bar{M}=\max\{M,L_0\}$, 
\begin{align*}
N(k)&=\sum_{j=0}^{k}(i_{j}+1)=\sum_{j=0}^{k} (\log_{2}(L_{j+1}/L_{j})+2 ) \leq 2(k+1)+\max\{\log_{2}(M/L_{0}),0\}.
\end{align*}
This shows that on average the inner loop ends after two trials. 

\subsubsection{Per-iteration analysis and a bound for the number of iterations}
%Given $\eps>0$, we choose $\mu=\eps/\nu$ and $x^{0}$ a $\nu$-analytic center. 
Let us fix iteration counter $k$. Since $L_{k+1} = 2^{i_k-1}L_k$, the step-size \eqref{eq:alpha_k} reads as $\alpha_{k}=\min \left\{\frac{1}{2L_{k+1} + 2 \mu},\frac{1}{2\zeta(x^k,v^k)} \right\}$. Hence, $\alpha_{k}\zeta(x^{k},v^{k})\leq 1/2$, and \eqref{eq:success} with the identification $t=\alpha_k = \ct_{\mu,2L_{k+1}}(x^{k})$, $M=2L_{k+1}$, $x=x^{k}$, $v_{\mu}(x^{k}) \eqdef v^k$ gives:  
\begin{equation}
\label{eq:FO_per_iter_proof_2}
F_{\mu}(x^{k+1})-F_{\mu}(x^{k})\leq -\alpha_k \norm{v^k}_{x^k}^{2}\left(1-(L_{k+1}+\mu)\alpha_k\right) \leq -\frac{\alpha_k \norm{v^k}_{x^k}^{2}}{2},
\end{equation}
where we used that $\alpha_k \leq \frac{1}{2(L_{k+1}+\mu)}$ in the last inequality. 
Substituting into \eqref{eq:FO_per_iter_proof_2} the two possible values of the step-size $\alpha_k$ in \eqref{eq:alpha_k} gives
%Next, we consider two possible cases of the value of the step-size $\alpha_k$ and substitute it into \eqref{eq:FO_per_iter_proof_2}.
\begin{equation}
\label{eq:per_iter_decr_0}
F_{\mu}(x^{k+1})-F_{\mu}(x^{k})\leq 
\left\{
\begin{array}{ll}
- \frac{\norm{v^{k}}_{x^{k}}^{2} }{4(L_{k+1}+\mu)} & \text{if }  \alpha_k=\frac{1}{2(L_{k+1}+\mu)}\\

- \frac{\norm{v^{k}}_{x^{k}}^{2} }{4\zeta(x^k,v^k)}  \stackrel{\eqref{eq:boundzeta}}{\leq} - \frac{\norm{v^{k}}_{x^{k}}}{4} & \text{if }  \alpha_k=\frac{1}{2\zeta(x^k,v^k)}.
\end{array}\right.
\end{equation}
Recalling $L_{k+1} \leq \bar{M}$ (see section \ref{sec:backtrack1}), we obtain that 
\begin{equation}
\label{eq:per_iter_decr}
F_{\mu}(x^{k+1}) - F_{\mu}(x^{k}) \leq  -\frac{\norm{v^{k}}_{x^{k}}}{4} \min\left\{1,  \frac{\norm{v^{k}}_{x^{k}}}{\bar{M}+\mu}\right\}=-\delta_{k}.
\end{equation}
Rearranging and summing these inequalities for $k$ from $0$ to $K-1$ gives
\begin{align}
&K\min_{k=0\ldots,K-1} \delta_{k} \leq  \sum_{k=0}^{K-1}\delta_{k}\leq F_{\mu}(x^{0})-F_{\mu}(x^{K}) \notag \\
& \quad\stackrel{\eqref{eq:potential}}{=} f(x^{0}) - f(x^{K}) + \mu (h(x^{0}) - h(x^{K})) \leq f(x^{0}) - f_{\min}(\setX) + \eps, \label{eq:FO_per_iter_proof_6} 
\end{align}
where we used that, by the assumptions of Theorem \ref{Th:AHBA_conv}, $x^{0}$ is a $\nu$-analytic center defined in \eqref{eq:analytic_center} and $\mu = \eps/\nu$, implying that $h(x^{0}) - h(x^{K}) \leq \nu = \eps/\mu$.
Thus, up to passing to a subsequence, $\delta_{k}\to 0$, and consequently $\norm{v^{k}}_{x^{k}} \to 0$ as $k \to \infty$. This shows that the stopping criterion in Algorithm \ref{alg:AHBA} is achievable.

Assume now that the stopping criterion $\norm{v^k}_{x^k} < \frac{\eps}{ \nu}$ does not hold for $K$ iterations of $\AHBA$. Then, for all $k=0,\ldots,K-1,$ it holds that 
$\delta_{k}\geq \min\left\{\frac{\eps}{4 \nu},\frac{\eps^{2}}{4 \nu^{2}(\bar{M}+\mu)}\right\}$. 
Together with the parameter coupling $\mu=\frac{\eps}{\nu}$, it follows from \eqref{eq:FO_per_iter_proof_6} that
\[
K\frac{\eps^{2}}{4 \nu^{2}(\bar{M}+\eps/\nu)} = K \min\left\{\frac{\eps}{4 \nu},\frac{\eps^{2}}{4 \nu^{2}(\bar{M}+\eps/\nu)}\right\}\leq f(x^{0})-f_{\min}(\setX)+\eps.
\]
Hence, recalling that $\bar{M}=\max\{M,L_0\}$,
\[
K \leq 4(f(x^{0}) - f_{\min}(\setX)+ \eps) \cdot \frac{\nu^{2}(\max\{M,L_0\}+\eps/\nu)}{\eps^{2}},%\max\left\{\frac{\nu}{\eps},\frac{\nu^{2}(M+\eps/\nu)}{\eps^{2}}\right\},
\] 
i.e., the algorithm stops for sure after no more than this number of iterations. This, combined with the bound for the number of inner steps in Section \ref{sec:backtrack1}, proves the first statement of Theorem \ref{Th:AHBA_conv}.

\subsubsection{Generating $\eps$-KKT point}
To finish the proof of Theorem \ref{Th:AHBA_conv}, we now show that when Algorithm \ref{alg:AHBA} stops for the first time, it returns a $2\eps$-KKT point of \eqref{eq:Opt} according to Definition \ref{def:eps_KKT}.

Let the stopping criterion hold at iteration $k$. By the optimality condition \eqref{eq:finder_1} and the definition of the potential \eqref{eq:potential}, we have
\begin{equation}
\label{eq:FO_KKT_proof_0}
\nabla f(x^{k})-\bA^{\ast}y^{k}+\mu \nabla h(x^{k}) =-H(x^{k})v^{k}\iff [H(x^{k})]^{-1}\left(\nabla f(x^{k})-\bA^{\ast}y^{k}+\mu \nabla h(x^{k}) \right)=-v^{k}.
\end{equation}
Denoting $g^{k}\eqdef-\mu\nabla h(x^{k})$, multiplying both equations, and using the stopping criterion $\norm{v^{k}}_{x^{k}} < \frac{\eps}{\nu}$, we conclude 
\begin{equation}
\label{eq:FO_KKT_proof_00}
\norm{\nabla f(x^{k})-\bA^{\ast}y^{k}-g^{k}}^{\ast}_{x^{k}}=\norm{v^{k}}_{x^{k}}<\frac{\eps}{\nu}.
\end{equation}
Whence, setting $s^{k}\eqdef\nabla f(x^{k})-\bA^{\ast}y^{k}\in\setE^{\ast}$, we get, by the definition of the dual norm, 
\begin{align}
\frac{\eps}{\nu} > \norm{v^{k}}_{x^{k}}&=\norm{s^{k}-g^{k}}^{\ast}_{x^{k}} \label{eq:FO_KKT_proof_1}\\
&=\norm{s^{k}-g^{k}}_{[H(x^{k})]^{-1}}\stackrel{\eqref{eq:relations}}{=}\norm{s^{k}-g^{k}}_{\nabla^{2}h_{\ast}(-\nabla h(x^{k}))} \\
&=\norm{s^{k}-g^{k}}_{\nabla^{2}h_{\ast}(\frac{1}{\mu}g^{k})} = \mu\norm{s^{k}-g^{k}}_{\nabla^{2}h_{\ast}(g^{k})}, \notag
\end{align}
where in the last equality we used that since $h_{\ast}\in\scrH_{\nu}(\setK^{\ast})$, by \eqref{eq:log_hom_scb_hess_homog_prop}, 
%$\frac{1}{t^{2}}\nabla^{2}h_{\ast}(s)=\nabla^{2}h_{\ast}(ts)$. Therefore, 
$\nabla^{2}h_{\ast}(\frac{1}{\mu}g^{k})=\mu^{2}\nabla^{2}h_{\ast}(g^{k})$.
%, and we conclude $\norm{v^{k}}_{x^{k}}=\mu\norm{s^{k}-g^{k}}_{\nabla^{2}h_{\ast}(g^{k})}.$
%Whence, since the stopping criterion $\norm{v^{k}}_{x^{k}} < \frac{\eps}{\nu}$ holds at iteration $k$,
Thus, we arrive at
\begin{align}
\norm{s^{k}-g^{k}}_{\nabla^{2}h_{\ast}(g^{k})} = \frac{\norm{v^{k}}_{x^{k}}}{\mu} < \frac{\eps}{\mu \nu} = 1, \label{eq:FO_KKT_proof_2}
\end{align}
where in the last equality we used that, by the assumptions of Theorem \ref{Th:AHBA_conv}, $\mu=\frac{\eps}{\nu}$.
Thus, since, by \eqref{eq:relations_1}, $g^{k}=-\mu\nabla h(x^{k})\in \setK^{\ast}$, we get that $s^{k}\in\setK^{\ast}$. By construction, $x^{k}\in \setK$ and $\bA x^{k} = b$. Thus, \eqref{eq:eps_optim_equality_cones} holds.
Furthermore, $\|\nabla f(x^{k})-\bA^{\ast}y^{k} - s^{k}\|=0\leq 2 \eps$, meaning that \eqref{eq:eps_optim_grad} holds. 
Finally, since $(x^{k},s^{k})\in\setK\times\setK^{\ast}$, we see 
\begin{align}
0 \leq \inner{s^{k},x^{k}}&=\inner{s^{k}-g^{k},x^{k}}+\inner{g^{k},x^{k}} \notag\\
&\leq \norm{s^{k}-g^{k}}_{x^{k}}^{\ast}\cdot\norm{x^{k}}_{x^{k}}-\mu\inner{\nabla h(x^{k}),x^{k}} \notag \\
&\stackrel{\eqref{eq:FO_KKT_proof_1},\eqref{eq:log_hom_scb_norm_prop},\eqref{eq:log_hom_scb_hess_prop}}{=}\norm{v^{k}}_{x^{k}}\sqrt{\nu}+\mu\nu \notag \\
&<\sqrt{\nu}\frac{\eps}{\nu}+\eps\leq 2\eps, \label{eq:FO_KKT_proof_3}
\end{align}
where the last inequality uses $\nu\geq 1$. Hence, the complementarity condition \eqref{eq:eps_optim_complem} holds as well. This finishes the proof of Theorem \ref{Th:AHBA_conv}.

\subsection{Discussion}
\label{sec:FO_discussion}
\paragraph{Strengthened KKT condition.}
%The main technical challenge in our analysis is to guarantee the approximate complementarity condition \eqref{eq:eps_optim_complem}. For this, we need to show that the dual variable $s^{k}$ belongs to the dual cone $\setK^{\ast}$, and second, we need to show that \eqref{eq:eps_optim_complem} holds. Another challenge is to translate the condition \eqref{eq:FO_KKT_proof_00} stated in terms of the local norm to the condition \eqref{eq:eps_optim_grad} formulated in terms of the standard Euclidean norm. In our setting of general, potentially non-symmetric, cones all this is more difficult than for the prominent (symmetric) cone case $\bar{\setK}=\bar{\setK}_{\text{NN}}$, as studied in \cite{HaeLiuYe18}. % where a first-order algorithm is also proposed. 
%Let us compare the bounds reported in that references with ours. 
For $\bar{\setK}=\bar{\setK}_{\text{NN}}$, \cite{HaeLiuYe18} consider a first-order potential reduction method employing the standard log-barrier $h(x)=-\sum_{i=1}^n \ln(x_i)$ using a trust-region subproblem for obtaining the search direction. For $x\in\setK_{\text{NN}}$, we have $\nabla h(x)=[-x_{1}^{-1},\ldots,-x_{n}^{-1}]^{\top}$, $H(x)=\diag[x_{1}^{-2},\ldots,x_{n}^{-2}]=\XX^{-2}$. Combining \eqref{eq:FO_KKT_proof_0}, the information $[H(x^{k})]^{-1/2}\nabla h(x^{k})= - \1_{n},\;\nu=n$, and the stopping criterion of Algorithm \ref{alg:AHBA} at iteration $k$, saying that $\norm{v^{k}}_{x^{k}}<\frac{\eps}{\nu}$, we see 
%\begin{equation}
%\label{eq:FO_remarks_1}
\begin{align*}
\norm{H(x^{k})^{-\frac{1}{2}} (\nabla f(x^{k})-\bA^{\ast}y^{k}) - \mu \1_{n}}_{\infty}& \leq\norm{H(x^{k})^{-\frac{1}{2}} (\nabla f(x^{k})-\bA^{\ast}y^{k}) - \mu \1_{n}  }\\
& = \norm{-H(x^{k})^{\frac{1}{2}} v^{k}} < \frac{\eps}{n}.
\end{align*}
Therefore, since $\mu=\eps/n$ and $s^{k}=\nabla f(x^{k})-\bA^{\ast}y^{k}\in\setK^{\ast}_{\text{NN}}=\Rn_{++}$, we obtain from the triangle inequality
\[
0<\norm{\XX^{k}s^{k}}_{\infty}\leq \norm{H(x^{k})^{-1/2} s^{k}-\mu\1_{n}}_{\infty}+\mu\leq\frac{2\eps}{n}.
\]
By Remark \ref{rem:FO_complexity_simplified}, these inequalities are achieved after $O\left(\frac{M n^2(f(x^{0}) - f_{\min}(\setX))}{\eps^2}\right)$ iterations of $\AHBA$, and they are seen to be by the factor $\frac{1}{n}$ sharper than the complementarity measure employed in \cite{HaeLiuYe18}. Conversely, in order to attain an approximate KKT point with the same strength as in \cite{HaeLiuYe18}, the above calculations suggest that we can weaken our tolerance from $\eps$ to $\eps\cdot n$, which results in an overall iteration complexity of   
$O\left(\frac{M  (f(x^{0}) - f_{\min}(\setX))}{\eps^2}\right)$, and a complementarity measure $\norm{\XX^{k}s^{k}}_{\infty}\leq 2\eps$. %
%
Thus, in the particular case of non-negativity constraints our general algorithm is able to obtain results similar to \cite{HaeLiuYe18}, but under weaker assumptions. At the same time, our algorithm ensures a stronger measure of complementarity. Indeed, our algorithm guarantees that $x^{k}\in\setK_{\text{NN}},s^{k}=\nabla f(x^k) -\bA^{\ast}y^{k}\in\setK_{\text{NN}}$, i.e., $x^{k},s^{k}\geq 0$, and approximate complementary $0\leq \sum_{i=1}^{n}\abs{x_{i}^{k}s_{i}^{k}}=\sum_{i=1}^{n}x_{i}^{k}s_{i}^{k}\leq 2\eps$ after $O\left(\frac{M n^2(f(x^{0}) - f_{\min}(\setX))}{\eps^2}\right)$ iterations, which is stronger than $\max_{1\leq i\leq n}\abs{x_{i}^{k}s_{i}^{k}} \leq 2\eps$ guaranteed by \cite{HaeLiuYe18}. Indeed, $\max_{1\leq i\leq n}\abs{x_{i}^{k}s_{i}^{k}} \leq \sum_{i=1}^{n}\abs{x_{i}^{k}s_{i}^{k}} \leq n \max_{1\leq i\leq n}\abs{x_{i}^{k}s_{i}^{k}}$, and both equalities are achievable. Moreover,  to match our stronger guarantee, one has to change $\eps \to\eps/n$ in the complexity bound of \cite{HaeLiuYe18}, which leads to the same $O\left(\frac{M n^2(f(x^{0}) - f_{\min}(\setX))}{\eps^2}\right)$ complexity bound. Besides this important insights, our algorithm is designed for general cones, rather than only for $\bar{\setK}_{\text{NN}}$. Therefore, we provide a unified approach for essentially all conic domains of relevance in optimization. Finally, our method does not rely on the trust-region techniques as in \cite{HaeLiuYe18} that may slow down the convergence in practice since the radius of the trust region is no grater than $O(\eps)$ leading to short steps.

\paragraph{Exploiting problem structure.}
In \eqref{eq:per_iter_decr_0} we can clearly observe the benefit of the use of $\nu$-SSB in our algorithm, whenever $\setK$ is a symmetric cone. Indeed, when $\alpha_k=\frac{1}{2\zeta(x^k,v^k)}$, the per-iteration decrease of the potential is $\frac{\norm{v^{k}}_{x^{k}}^{2} }{4\zeta(x^k,v^k)} \geq \frac{ \eps \norm{v^{k}}_{x^{k}}}{4\nu\zeta(x^k,v^k)} $ which may be large if $\zeta(x^k,v^k)=\sigma_{x^k}(-v^k) \ll \norm{v^{k}}_{x^{k}}$.


\paragraph{The role of the potential function.}
Next, we discuss more explicitly, how the algorithm and complexity bounds depend on the parameter $\mu$. The first observation is that from \eqref{eq:FO_KKT_proof_2}, to guarantee that $s^{k} \in \setK^{\ast}$, we need the stopping criterion to be $\norm{v^{k}}_{x^k} < \mu$, which by \eqref{eq:FO_KKT_proof_3} leads to the error $2 \mu \nu$ in the complementarity conditions. From the analysis following equation \eqref{eq:FO_per_iter_proof_6}, we have that  
\[
K\frac{\mu^{2}}{4  (\bar{M}+\mu)} = K \min\left\{\frac{\mu}{4},\frac{\mu^{2}}{4 \ (\bar{M}+\mu)}\right\}\leq f(x^{0})-f_{\min}(\setX)+\mu \nu.
\]
Whence, recalling that $\bar{M}=\max\{M,L_0\}$,
\[
K \leq 4(f(x^{0}) - f_{\min}(\setX)+ \mu \nu) \cdot \frac{\max\{M,L_0\}+\mu}{\mu^{2}}.%\max\left\{\frac{\nu}{\eps},\frac{\nu^{2}(M+\eps/\nu)}{\eps^{2}}\right\},
\]
Thus, we see that after $O(\mu^{-2})$ iterations the algorithm finds a $(2 \mu \nu)$-KKT point, and  if $\mu \to 0$, we have convergence to a KKT point, but the complexity bound tends to infinity and becomes non-informative. At the same time, as it is seen from \eqref{eq:finder}, when $\mu \to 0$, the algorithm itself converges to a preconditioned gradient method since $F_{\mu}(x)=f(x) + \mu h(x) \to  f(x)$. 
We also see from the above explicit expressions in terms of $\mu$ that the design of the algorithm requires careful balance between the desired accuracy of the approximate KKT point expressed mainly by the complementarity condition, stopping criterion, and complexity. Moreover, the step-size should be also taken carefully to ensure the feasibility of the iterates, and the standard for first-order methods step-size $1/M$ may not work. 





\subsection{Anytime convergence via restarting $\AHBA$}
\label{sec:path-following}
The analysis of Algorithm \ref{alg:AHBA} is based on the a-priori fixed tolerance $\eps>0$ and the parameter coupling $\mu=\eps/\nu$. This coupling allows us to embed Algorithm \ref{alg:AHBA} within a restarting scheme featuring a decreasing sequence $\{\mu_{i}\}_{i\geq 0}$, followed by restarts of $\AHBA$. This restarting strategy frees Algorithm \ref{alg:AHBA} from hard-coded parameters and connects it well to traditional barrier methods.% In this way, we obtain, to the best of our knowledge, the first path-following method with complexity guarantees for non-convex problems with linear and conic constraints. \MS{Not sure if this can not be interpreted as overselling.}

To describe this double-loop algorithm, we fix $\eps_{0}>0$ and select the starting point $x_0^{0}$ as a $\nu$-analytic centre of $\setX$ with respect to $h\in\scrH_{\nu}(\setK)$. We let $i \geq 0$ denote the counter for the restarting epochs at the start of which the value $\mu_{i}$ is decreased. In epoch $i$, we generate a sequence $\{x^{k}_{i}\}_{k=0}^{K_{i}}$ by calling $\AHBA(\mu_{i},\eps_{i},L_0^{(i)},x^{0}_{i})$ until the stopping condition is reached. This will take at most $\K_{I}(\eps_{i},x^{0}_{i})$ iterations, specified in eq. \eqref{eq:FO_main_Th_compl}. We store the last iterate $\hat{x}_{i}=x^{K_{i}}_{i}$ and the last estimate of the Lipschitz modulus $\hat{M}_{i}=L_{K_i}^{(i)}$ obtained from procedure $\AHBA(\mu_{i},\eps_{i},L_0^{(i)},x^{0}_{i})$ and then restart the algorithm using the ``warm starts'' $x^{0}_{i+1}=\hat{x}_{i}$, $L_{0}^{(i+1)}=\hat{M}_{i}/2$, $\eps_{i+1}=\eps_{i}/2$, $\mu_{i+1}=\eps_{i+1}/\nu$. If $\eps \in (0,\eps_0)$ is the target accuracy of the final solution, it suffices to perform $ \lceil \log_{2}(\eps_{0}/\eps)\rceil+1$ restarts since, by construction, $\eps_{i} = \eps_0 \cdot 2^{-i}$. 
%%%%%%%%%%%%%%%%%%%%%%%%%%%%
\begin{algorithm}[h]
\caption{Restarting $\AHBA$}
\label{alg:RestartHBA}
\SetAlgoLined
\KwData{ $h \in\scrH_{\nu}(\setK)$, $\eps_{0}>0$, $x_0^{0}\in\setX$ satisfying \eqref{eq:analytic_center}, $L_0^{(0)}>0$.}
\KwResult{Point $\hat{x}_i$, dual variables $\hat{y}_i$, $\hat{s}_i = \nabla f(\hat{x}_i) -\bA^{\ast}\hat{y}_i$.}
%Set $I=\lceil\log_{2}(\eps_{0}/\eps)\rceil+1$\;
\For{$i=0,1,\ldots$} 
{ Set $\eps_{i}=2^{-i}\eps_{0}$, $\mu_{i}=\frac{\eps_i}{\nu}$\; 
 Obtain $(\hat{x}_{i},\hat{y}_{i},\hat{s}_{i},\hat{M}_{i})$ from $\AHBA(\mu_{i},\eps_i,L_0^{(i)},x^{0}_{i})$\;
 %
% $(x_{i}^{K_{i}},L_{i}^{K_{i}})$ from $\AHBA(\mu_{i},\eps_i,x^{0}_{i})$\; 
 Set $x_{i+1}^{0}=\hat{x}_{i}$ and $L_0^{(i+1)}=\hat{M}_{i}/2$.
	}
\end{algorithm}
%%%%%%%%%%%%%%%%%	
\begin{theorem}\label{th:ComplexityPathfollowing}
Let Assumptions \ref{ass:1}-\ref{ass:gradLip} hold. 
Then, for any $\eps \in (0,\eps_0)$, Algorithm \ref{alg:RestartHBA} finds a $2\eps$-KKT point for problem \eqref{eq:Opt} in the sense of Definition \ref{def:eps_KKT} after
no more than $I(\eps):=\lceil \log_{2}(\eps_{0}/\eps)\rceil+1$ restarts and at most 
$
\left\lceil \frac{64}{3\eps^2}(f(x^{0})-f_{\min}(\setX)+\eps_0)\nu^2(\max\{M,L_0^{(0)}\}+\eps_0/\nu)\right\rceil
$
 iterations of $\AHBA$.
\end{theorem}
\begin{proof}
Let us consider a restart $i \geq 0$ and repeat the proof of Theorem \ref{Th:AHBA_conv} with the change $\eps \to \eps_i$,  $\mu \to \mu_i = \eps_i/\nu$, $L_0 \to L_0^{(i)}=\hat{M}_{i-1}/2$, $\bar{M}=\max\{M,L_0\} \to \bar{M}_i=\max\{M,L_0^{(i)}\}$, $x^0 \to x^{0}_{i}=\hat{x}_{i-1}$.
Let $K_i$ be the last iteration of $\AHBA(\mu_{i},\eps_i,L_0^{(i)},x^{0}_{i})$ meaning that $\norm{v^{K_i}}_{x^{K_i}} < \frac{\eps_i}{\nu}$ and $\norm{v^{K_i-1}}_{x^{K_i-1}} \geq \frac{\eps_i}{\nu}$. From the analysis following equation \eqref{eq:FO_per_iter_proof_6}, we have that
\begin{align}
&K_i \frac{\eps_i^{2}}{4 \nu^{2}(\bar{M}_i+\eps_i/\nu)} \leq K_i \min_{k=0\ldots,K_i-1} \delta_{k}^i \leq  \sum_{k=0}^{K_i-1}\delta_{k}^i\leq F_{\mu_i}(x^{0}_i)-F_{\mu_i}(x^{K_i}_i). \label{eq:PF_proof_1} 
\end{align}
Further, using the fact that $\mu_i$ is a decreasing sequence and \eqref{eq:analytic_center}, it is easy to deduce
\begin{align}
F_{\mu_{i+1}}(x^{0}_{i+1})&=F_{\mu_{i+1}}(x^{K_i}_{i})\stackrel{\eqref{eq:potential}}{=}f(x^{K_i}_{i}) + \mu_{i+1} h(x^{K_i}_{i}) \stackrel{\eqref{eq:potential}}{=}F_{\mu_{i}}(x^{K_i}_{i}) + (\mu_{i+1} - \mu_{i}) h(x^{K_i}_{i}) \notag \\
&\stackrel{\eqref{eq:analytic_center}}{\leq} F_{\mu_{i}}(x^{K_i}_{i}) + (\mu_{i+1} - \mu_{i}) (h(x_0^0)-\nu)\notag \\
& \stackrel{\eqref{eq:PF_proof_1}}{\leq}F_{\mu_i}(x^{0}_i)-K_i \frac{\eps_i^{2}}{4 \nu^{2}(\bar{M}_i+\eps_i/\nu)} + (\mu_{i+1} - \mu_{i}) (h(x_0^0)-\nu).
\label{eq:PF_proof_2} 
\end{align}
Letting $I\equiv I(\eps):=\left\lceil \log_2 (\frac{\eps_0}{\eps}) \right\rceil+1$, by Theorem \ref{Th:AHBA_conv} applied to the restart $I-1$, we see that $\AHBA(\mu_{I-1},\eps_{I-1},L_0^{(I-1)},x^{0}_{I-1})$ outputs a $2\eps$-KKT point for problem \eqref{eq:Opt} in the sense of Definition \ref{def:eps_KKT}.
%Clearly, to achieve any desired accuracy $\eps$, i.e., find $2\eps$-KKT point for problem \eqref{eq:Opt}, it is sufficient to make $I=I_I(\eps)=\left\lceil \log_2 \frac{\eps_0}{\eps} \right\rceil+1$ restarts $i=0,...,I-1$. 
Summing inequalities \eqref{eq:PF_proof_2} for all the performed restarts $i=0,...,I-1$ and rearranging the terms, we obtain
\begin{align}
\sum_{i=0}^{I-1} K_i \frac{\eps_i^{2}}{4 \nu^{2}(\bar{M}_i+\eps_i/\nu)} & \leq F_{\mu_0}(x^{0}_0) - F_{\mu_{I}}(x^{0}_{I}) + (\mu_{I} - \mu_{0}) (h(x_0^0)-\nu) \notag \\
& \stackrel{\eqref{eq:potential}}{=} f(x^{0}_0) + \mu_0 h(x^{0}_0)- f(x^{0}_{I}) - \mu_{I}h(x^{0}_{I}) + (\mu_{I} - \mu_{0}) (h(x_0^0)-\nu) \notag \\
& \stackrel{\eqref{eq:analytic_center}}{\leq} f(x^{0}_0) - f_{\min}(\setX) + \mu_0 h(x^{0}_0) - \mu_{I}h(x^{0}_{0}) + \mu_{I} \nu + (\mu_{I} - \mu_{0}) (h(x_0^0)-\nu) \notag \\ 
& \leq f(x^{0}_0) - f_{\min}(\setX) + \mu_0 \nu = f(x^{0}_0) - f_{\min}(\setX) + \eps_0.
\label{eq:PF_proof_3} 
\end{align}
Moreover, based on our updating choice $L_0^{(i+1)}=\hat{M}_{i}/2$, it holds that 
\begin{align*}
\bar{M}_i &= \max\{M,L_0^{(i)}\} = \max\{M,\hat{M}_{i-1}/2\}\\
& = \max\{M,L_{K_{i-1}}^{(i-1)}/2\} \leq \max\{M,\bar{M}_{i-1}\} \leq ... \leq \max\{M, \bar{M}_{0}\} \leq  \max\{M,L_0^{(0)}\}.
\end{align*}
Hence, 
%$\bar{M}_i = \max\{M,L_0^{(i)}\} = \max\{M,\hat{M}_{i-1}/2\} = \max\{M,L_{K_i-1}^{i-1}/2\} \leq \max\{M,\bar{M}_{i-1}\} \leq ... \leq \max\{M, \bar{M}_{0}\}=  \max\{M,L_0\}$
%$\bar{M}_i = \max\{M,\hat{M}_i\}\leq \max\{M,L_0\}=\bar{M}$, we obtain for each $i=0,...,I-1$ that
\begin{align}
K_i \leq  4(f(x^{0}) - f_{\min}(\setX)+  \eps_0) \cdot \frac{\nu^{2}(\bar{M}_i+\eps_i/\nu)}{\eps_i^{2}} \leq \frac{C}{\eps_i^2},
\label{eq:PF_proof_4} 
\end{align}
where $C\equiv 4(f(x^{0})-f_{\min}(\setX)+\eps_0)\nu^2(\max\{M,L_0^{(0)}\}+\eps_0/\nu)$. Finally, we obtain that the total number of iterations of procedures $\AHBA(\mu_{i},\eps_{i},L_0^{(i)},x_{i}^{0}),0\leq i \leq I-1$, to reach accuracy $\eps$ is at most
\begin{align*}
\sum_{i=0}^{I-1}K_{i}&\leq \sum_{i=0}^{I-1} \frac{C}{\eps_i^2} \leq \frac{C}{\eps_0^2} \sum_{i=0}^{I-1} (2^i)^2 
\leq \frac{C}{3\eps_0^2} \cdot (4^{2+\log_2(\frac{\eps_0}{\eps})}) = \frac{16C}{3\eps^2}\\
&=\frac{64(f(x^{0})-f_{\min}(\setX)+\eps_0)\nu^2(\max\{M,L_0^{(0)}\}+\eps_0/\nu)}{3\eps^{2}}.
\end{align*}
\end{proof}

\section{A second-order Hessian-Barrier Algorithm}
\label{sec:secondorder}
%----------------------------------------------------------------------
%%% 2nd-order
%----------------------------------------------------------------------
% !TEX root = ./HBAConicMain.tex
%

In this section we introduce a second-order potential reduction method for problem \eqref{eq:Opt} under the assumption that the second-order Taylor expansion of $f$ on the set of feasible directions $\scrT_{x}$ defined in \eqref{eq:Dikinv} is sufficiently accurate in the geometry induced by $h \in\scrH_{\nu}(\setK)$. 
%%%%%%%%%%%%
\begin{assumption}[Local second-order smoothness]
\label{ass:2ndorder}
$f:\setE\to\R\cup\{+\infty\}$ is twice continuously differentiable on $\feas$ and there exists a constant $M>0$ such that, for all $x\in\feas$ and $v\in\scrT_{x}$, we have
\begin{equation}\label{eq:SO_Lipschitz_Gradient}
\norm{\nabla f(x+v)-\nabla f(x)-\nabla^{2}f(x)v}^{\ast}_{x}\leq\frac{M}{2}\norm{v}^{2}_{x}.
\end{equation}
\end{assumption}
A sufficient condition for \eqref{eq:SO_Lipschitz_Gradient} is the following local counterpart of the global Lipschitz condition on
the Hessian of $f$:
\begin{equation}\label{eq:LipHess}
(\forall x\in\setX)(\forall u,v\in\scrF_{x}):\;\norm{\nabla^{2}f(x+u)-\nabla^{2}f(x+v)}_{\text{op},x}\leq M\norm{u-v}_{x},
\end{equation}
where 
$\norm{\BB}_{\text{op},x}\eqdef\sup_{u:\norm{u}_{x}\leq 1}\left\{\frac{\norm{\BB u}_{x}^{\ast}}{\norm{u}_{x}}\right\}$ is the induced operator norm for a linear operator $\BB:\setE\to\setE^{\ast}$. Indeed, this condition implies \eqref{eq:SO_Lipschitz_Gradient}:
\begin{align*}
&\norm{\nabla f(x+v)-\nabla f(x)-\nabla^{2}f(x)v}^{\ast}_{x}=\norm{\int_{0}^{1}(\nabla^{2}f(x+tv)-\nabla^{2}f(x))v\dif t }^{\ast}_{x}\\
&\leq \int_{0}^{1}\norm{\nabla^{2}f(x+tv)-\nabla^{2}f(x)}_{\text{op},x}\cdot \norm{v}_{x}\dif t   \leq \frac{M}{2}\norm{v}^{2}_{x}.
\end{align*}
%\begin{align*}
%&\norm{\nabla f(x+v)-\nabla f(x)-\nabla^{2}f(x)v}^{\ast}_{x}=\norm{\int_{0}^{1}(\nabla^{2}f(x+tv)-\nabla^{2}f(x))v\dif t }^{\ast}_{x}\\
%&\quad \leq\int_{0}^{1}\norm{(\nabla^{2}f(x+tv)-\nabla^{2}f(x))v}_{x}^{\ast}\dif t \leq \int_{0}^{1}\norm{\nabla^{2}f(x+tv)-\nabla^{2}f(x)}_{\text{op},x}\cdot \norm{v}_{x}\dif t  \\
%&\quad \quad \leq \frac{M}{2}\norm{v}^{2}_{x}.
%\end{align*}
Further, \eqref{eq:SO_Lipschitz_Gradient} in turn implies another important estimate 
\begin{equation}\label{eq:cubicestimate}
f(x+v)-\left[f(x)+\inner{\nabla f(x),v}+\frac{1}{2}\inner{\nabla^{2}f(x)v,v}\right]\leq\frac{M}{6}\norm{v}^{3}_{x}.
\end{equation}
Indeed, for all $x\in\setX$ and $v\in\scrT_{x}$,
\begin{align*}
&\abs{f(x+v)-f(x)-\inner{\nabla f(x),v}-\frac{1}{2}\inner{\nabla^{2}f(x)v,v}}=\abs{\int_{0}^{1}\inner{\nabla f(x+tv)-\nabla f(x)-\frac{1}{2}\nabla^{2}f(x)v,v}\dif t}\\
&\quad\leq \int_{0}^{1}\norm{\nabla f(x+tv)-\nabla f(x)-\frac{1}{2}\nabla^{2}f(x)v}_{x}^{*}\dif t\cdot\norm{v}_{x} 
\leq \frac{M}{6}\norm{v}^{3}_{x}.
\end{align*}
%%%%%%%%%%%%%%%%%%%%%%%%%%%
\begin{remark}
\label{Rm:Hess_Lip}
Assumption \ref{ass:2ndorder} subsumes, when $\bar{\setX}$ is bounded, the standard Lipschitz-Hessian setting since 
if the Hessian of $f$ is Lipschitz with modulus $M$ with respect to the Euclidean norm, we have by \cite[Eq. (2.2)]{NesPol06}.
\[
\norm{\nabla f(x+v)-\nabla f(x)-\nabla^{2}f(x)v}\leq\frac{M}{2}\norm{v}^{2}.
\]
Since $\bar{\setX}$ is bounded, one can observe that $\lambda_{\max}([H(x)]^{-1})^{-1}=\lambda_{\min}(H(x)) \geq \sigma$ for some $\sigma >0$, and \eqref{eq:SO_Lipschitz_Gradient} holds. Indeed, denoting $g=\nabla f(x+v)-\nabla f(x)-\nabla^{2}f(x)v$, we obtain
\[
(\norm{g}_x^*)^2 \leq \lambda_{\max}([H(x)]^{-1})\norm{g}^{2} \leq \frac{M^2}{4\lambda_{\min}(H(x))}\norm{v}^{4} \leq \frac{M^2}{4\sigma^{3}}\norm{v}_x^4.
\]
\close
\end{remark}

\begin{remark}
The cubic overestimation of the objective function in \eqref{eq:cubicestimate} does not rely on global second order differentiability assumptions. To illustrate this we invoke again the structured composite optimization problem \eqref{eq:composite}, assuming that the data fidelity function $\ell$ is twice continuously differentiable on an open neighborhood containing $\setX$, with Lipschitz continuous Hessian $\nabla^{2}\ell$ with modulus $\gamma$ w.r.t. the Euclidean norm. On the domain $\setK_{\text{NN}}$ we employ the canonical barrier $h(x)=-\sum_{i=1}^{n}\ln(x_{i})$, with $H(x)=\diag\{x_{1}^{-2},\ldots,x_{n}^{-2}\}=\XX^{-2}$. This means, for all $x,x^{+}\in\setX$, we have 
\[
\ell(x^{+})\leq\ell(x)+\inner{\nabla\ell(x),x^{+}-x}+\frac{1}{2}\inner{\nabla^{2}\ell(x)(x^{+}-x),x^{+}-x}+\frac{\gamma}{6}\norm{x^{+}-x}^{3}.
\]
As penalty function, we again consider the $L_{p}$ regularizer with $p\in(0,1)$. For any $t,s>0$, one has 
\[
t^{p}\leq s^{p}+ps^{p-1}(t-s)+\frac{p(p-1)}{2} s^{p-2}(t-s)^{2}+\frac{p(p-1)(p-2)}{6}s^{p-3}(t-s)^{3}.
\]
Since $v\in\scrT_{x}$ if and only if $v=[H(x)]^{-1/2}d=\XX d$ for some $d\in\R^{\dim(\setE)}$ satisfying $\bA\XX d=0$ and $\norm{d}<1$. Since $p(1-p)\leq 1/4$, it follows $p(1-p)(2-p)\leq 1/2$. Thus, using $x^{+}=x+v=x+\XX d$, we get 
\begin{align*}
f(x^{+}) &- \left((f(x)+\inner{\nabla f(x),\XX d}+\frac{1}{2}\inner{\nabla^{2}f(x)\XX d,\XX d}\right) \leq \frac{\gamma}{6}\norm{\XX d}^{3}+\frac{1}{12}\sum_{i=1}^{n}x^{p}_{i}d_{i}^{3}\\
&\leq \frac{\gamma}{6}\norm{\XX d}^{3}+\frac{1}{12}\norm{x}^{p}_{\infty}\sum_{i=1}^{n}d_{i}^{3} \leq \frac{1}{6}\left(\gamma\norm{x}_{\infty}^{3}+\frac{1}{2}\norm{x}^{p}_{\infty}\right)\norm{d}^{3}.
\end{align*}
%\begin{align*}
%f(x^{+})&\leq f(x)+\inner{\nabla f(x),\XX d}+\frac{1}{2}\inner{\nabla^{2}f(x)\XX d,\XX d}+\frac{\gamma}{6}\norm{\XX d}^{3}+\frac{1}{12}\sum_{i=1}^{n}x^{p}_{i}d_{i}^{3}\\
%&\leq f(x)+\inner{\nabla f(x),\XX d}+\frac{1}{2}\inner{\nabla^{2}f(x)\XX d,\XX d}+\frac{\gamma}{6}\norm{\XX d}^{3}+\frac{1}{12}\norm{x}^{p}_{\infty}\sum_{i=1}^{n}d_{i}^{3}\\
%&\leq f(x)+\inner{\nabla f(x),\XX d}+\frac{1}{2}\inner{\nabla^{2}f(x)\XX d,\XX d}+\frac{1}{6}\left(\gamma\norm{x}_{\infty}^{3}+\frac{1}{2}\norm{x}^{p}_{\infty}\right)\norm{d}^{3}
%\end{align*}
Assuming that $\bar{\setX}$ is bounded, there exists a universal constant $M>0$ such that $\gamma\norm{x}_{\infty}^{2}+\frac{1}{2}\norm{x}^{p}_{\infty}\leq M$. Combining this with Remark \ref{Rm:Hess_Lip}, we obtain a cubic overestimation as in eq. \eqref{eq:cubicestimate}. Importantly, $f(x)$ is not differentiable for $x \in \{x_i=0,\text{ for some } i\}$.  
\close
\end{remark}

We emphasize that in Assumption \ref{ass:2ndorder} the constant $M$ is in general unknown or may be a conservative upper bound. Therefore, adaptive techniques should be used to estimate it and are likely to improve the practical performance of the method. Assumption \ref{ass:2ndorder} also implies, by \eqref{eq:cubicestimate} and \eqref{eq:Dbound} (with $d=v$ and $t=1< \frac{1}{\norm{v}_x} \stackrel{\eqref{eq:boundzeta}}{\leq} \frac{1}{\zeta(x,v)}$), the following upper bound for the potential function $F_{\mu}$. 
\begin{lemma}[Cubic Overestimation]
\label{lem:cubic}
For all $x\in\setX,v\in\scrT_{x}$ and $L\geq M$, we have 
\begin{equation}\label{eq:cubicdecrease}
F_{\mu}(x+v)\leq F_{\mu}(x)+\inner{\nabla F_{\mu}(x),v}+\frac{1}{2}\inner{\nabla^{2}f(x)v,v}+\frac{L}{6}\norm{v}^{3}_{x}+\mu\norm{v}^{2}_{x}\omega(\zeta(x,v)).
\end{equation}
\end{lemma}

\subsection{Algorithm description and its complexity theorem}
\label{S:SO_alg_descr}
Let $x\in\setX$ be given. In order to find a search direction, we choose a parameter $L>0$, construct a cubic-regularized model of the potential $F_{\mu}$ \eqref{eq:potential}, and minimize it on the linear subspace $\setL_{0}$:
\begin{equation}\label{eq:cubicproblem}
v_{\mu,L}(x)\in\Argmin_{v\in\setE:\bA v=0}\left\{ Q^{(2)}_{\mu,L}(x,v)\eqdef F_{\mu}(x)+\inner{\nabla F_{\mu}(x),v}+\frac{1}{2}\inner{\nabla^{2}f(x)v,v}+\frac{L}{6}\norm{v}_{x}^{3} \right\},
\end{equation}
where by $\Argmin$ we denote the set of global minimizers. The model consists of three parts: linear approximation of $h$, quadratic approximation of $f$, and a cubic regularizer with penalty parameter $L>0$. Since this model and our algorithm use the second derivative of $f$, we call it a second-order method.
Our further derivations rely on the first-order optimality conditions for the problem \eqref{eq:cubicproblem}, which say that there exists $y_{\mu,L}(x)\in\R^{m}$ such that $v_{\mu,L}(x)$ satisfies
\begin{align}
\nabla F_{\mu}(x)+\nabla^{2}f(x)v_{\mu,L}(x)+\frac{L}{2}\norm{v_{\mu,L}(x)}_{x}H(x)v_{\mu,L}(x) - \bA^{\ast} y_{\mu,L}(x) &= 0,\label{eq:opt1}\\
 - \bA v_{\mu,L}(x)&=0. \label{eq:opt2}
\end{align}
We also use the following extension of \cite[Prop. 1]{NesPol06} to our setting with the local norm induced by $H(x)$.
\begin{proposition}
For all $x\in\feas$ it holds 
\begin{equation}\label{eq:PD}
\nabla^{2}f(x)+\frac{L}{2}\norm{v_{\mu,L}(x)}_{x}H(x)\succeq 0\qquad\text{ on }\;\;\setL_{0}.
\end{equation}
\end{proposition}
\begin{proof}
The proof follows the same strategy as Lemma 3.2 in \cite{CarGouToi11a}. Let $\{z_{1},\ldots,z_{p}\}$ be an orthonormal basis of $\setL_{0}$ and the linear operator $\bZ:\R^{p}\to\setL_{0}$ be defined by $\bZ w=\sum_{i=1}^{p}z_{i}w^{i}$ for all $w=[w^{1};\ldots;w^{p}]^{\top}\in\R^{p}$. With the help of this linear map, we can absorb the null-space restriction, and formulate the search-direction finding problem \eqref{eq:cubicproblem} using the projected data
\begin{equation}\label{eq:dataKernel}
\gbold\eqdef\bZ^{\ast}\nabla F_{\mu}(x),\; \bJ\eqdef\bZ^{\ast}\nabla^{2}f(x)\bZ,\;\bH\eqdef\bZ^{\ast}H(x)\bZ \succ 0.
\end{equation}
We then arrive at the cubic-regularized subproblem to find $u_{L}\in\R^{p}$ s.t.
\begin{equation}\label{eq:cubicauxiliary}
u_{L}\in  \Argmin_{u\in\R^{p}}\{\inner{\gbold,u}+\frac{1}{2}\inner{\bJ u,u}+\frac{L}{6}\norm{u}^{3}_{\bH}\},
\end{equation}
where  $\norm{\cdot}_{\bH}$ is the norm induced by the operator $\bH$. From \cite[Thm. 10]{NesPol06} we deduce  
\[
\bJ+\frac{L\norm{u_{L}}_{\bH}}{2}\bH\succeq 0.
\]
%where $u_{L}\in\R^{p}$ represents one global solution of problem \eqref{eq:cubicauxiliary}. 
Denoting $v_{\mu,L}(x) = \bZ u_{L}$, we see
\begin{align*}
\norm{u_{L}}_{\bH}=\inner{\bZ^{\ast}H(x)\bZ u_{L},u_{L}}^{1/2}=\inner{H(x)(\bZ u_{L}),\bZ u_{L}}^{1/2}&=\norm{v_{\mu,L}(x)}_{x}, \text{  and} \\
\bZ^{\ast}\left(\nabla^{2}f(x)+\frac{L}{2}\norm{v_{\mu,L}(x)}_{x}H(x)\right)\bZ&\succeq 0,
\end{align*}
which implies $\nabla^{2}f(x)+\frac{L}{2}\norm{v_{\mu,L}(x)}_{x}H(x)\succ 0$ over the null space $\setL_{0} = \{ v\in\setE:\bA v=0\}$. 
\end{proof}
\noindent
The above proposition gives some ideas on how one could numerically solve problem \eqref{eq:cubicproblem} in practice. In a preprocessing step, we once calculate matrix $\bZ$ and use it during the whole algorithm execution. At each iteration we calculate the new data using \eqref{eq:dataKernel}, leaving us with a standard \textit{unconstrained} cubic subproblem  \eqref{eq:cubicauxiliary}. \cite{NesPol06} show how such problems can be transformed to a \emph{convex} problem to which fast convex programming methods could in principle be applied. However, we can also solve it via recent efficient methods based on Lanczos' method \cite{CarGouToi11a,Jia22}. Whatever numerical tool is employed, we can recover our search direction by $v_{\mu,L}(x)$ by the matrix vector product $\bZ u_{L}$ in which $u_{L}$ denotes the solution obtained from this subroutine.
%%%%%%%%%%%%%%%%%%%%%
%\begin{remark}
%Our framework is formulated under the unrealistic assumption that the search direction can be computed exactly. An important extension of the current analysis will be to consider inexact computations. We leave this important question for future research.
%\close
%\end{remark}
%%%%%%%%%%%%%%

%step-size
Our next goal is to construct an admissible step-size policy,  given the search direction $v_{\mu,L}(x)$. 
Let $x\in\setX$ be the current position of the algorithm. Define the parameterized family of arcs $x^{+}(t)\eqdef x+t v_{\mu,L}(x)$, where $t\geq 0$ is a step-size. 
By \eqref{eq:step_length_zeta} and since $v_{\mu,L}(x)\in\setL_{0}$ by \eqref{eq:opt2}, we know that $x^{+}(t)$ is in $\setX$ provided that $t\in I_{x,\mu,L}\eqdef[0,\frac{1}{\zeta(x,v_{\mu,L}(x))})$. For all such $t$, Lemma \ref{lem:cubic} yields  
\begin{equation}\label{eq:SO_potent_upper_bound}
\begin{split}
F_{\mu}(x^{+}(t))\leq F_{\mu}(x)&+t\inner{\nabla F_{\mu}(x),v_{\mu,L}(x)} + \frac{t^2}{2}\inner{\nabla^{2}f(x)v_{\mu,L}(x),v_{\mu,L}(x)} \\
&+ \frac{Mt^3}{6}\norm{v_{\mu,L}(x)}^{3}_{x}  +\mu t^{2}\omega(t\zeta(x,v_{\mu,L}(x))).
\end{split}
\end{equation}
Since $v_{\mu,L}(x) \in \setL_0$, multiplying \eqref{eq:PD} with $v_{\mu,L}(x)$ from the left and the right, and multiplying \eqref{eq:opt1} by $v_{\mu,L}(x)$ and combining with \eqref{eq:opt2}, we obtain
\begin{align}
&\inner{\nabla^{2}f(x)v_{\mu,L}(x),v_{\mu,L}(x)}\geq-\frac{L}{2}\norm{v_{\mu,L}(x)}^{3}_{x},\label{eq:descent1}\\
&\inner{\nabla F_{\mu}(x),v_{\mu,L}(x)}+\inner{\nabla^{2}f(x)v_{\mu,L}(x),v_{\mu,L}(x)}+\frac{L}{2}\norm{v_{\mu,L}(x)}^{3}_{x}=0.\label{eq:normal}
\end{align}
Under the additional assumption that $t\leq 2$ and $L\geq M$, we obtain
\begin{align*}
&t\inner{\nabla F_{\mu}(x),v_{\mu,L}(x)} + \frac{t^2}{2}\inner{\nabla^{2}f(x)v_{\mu,L}(x),v_{\mu,L}(x)} + \frac{Mt^3}{6}\norm{v_{\mu,L}(x)}^{3}_{x}\\
&\stackrel{\eqref{eq:normal}}{=} - t \left(\inner{\nabla^{2}f(x)v_{\mu,L}(x),v_{\mu,L}(x)}+\frac{L}{2}\norm{v_{\mu,L}(x)}^{3}_{x}\right) &\\
&+ \frac{t^2}{2}\inner{\nabla^{2}f(x)v_{\mu,L}(x),v_{\mu,L}(x)} + \frac{Mt^3}{6}\norm{v_{\mu,L}(x)}^{3}_{x} \\
& = \left(\frac{t^2}{2} - t \right) \inner{\nabla^{2}f(x)v_{\mu,L}(x),v_{\mu,L}(x)} - \frac{Lt}{2}\norm{v_{\mu,L}(x)}^{3}_{x} + \frac{Mt^3}{6}\norm{v_{\mu,L}(x)}^{3}_{x} \\
& \stackrel{\eqref{eq:descent1},t\leq 2}{\leq} \left(\frac{t^2}{2} - t \right) \left(- \frac{L}{2}\norm{v_{\mu,L}(x)}^{3}_{x} \right) - \frac{Lt}{2}\norm{v_{\mu,L}(x)}^{3}_{x} + \frac{Mt^3}{6}\norm{v_{\mu,L}(x)}^{3}_{x} \\
& = - \norm{v_{\mu,L}(x)}^{3}_{x} \left(\frac{Lt^2}{4} - \frac{Mt^3}{6} \right) 
\stackrel{L \geq M}{\leq} - \norm{v_{\mu,L}(x)}^{3}_{x} \frac{Lt^2}{12} \left(3 - 2t \right).
\end{align*}
Substituting this into \eqref{eq:SO_potent_upper_bound}, we arrive at
\begin{align*}
F_{\mu}(x^{+}(t))&\leq F_{\mu}(x) - \norm{v_{\mu,L}(x)}^{3}_{x} \frac{Lt^2}{12} \left(3 - 2t \right)+\mu t^{2}\omega(t\zeta(x,v_{\mu,L}(x)))\\ 
&\stackrel{\eqref{eq:omega_upper_bound}}{\leq}  F_{\mu}(x) - \norm{v_{\mu,L}(x)}^{3}_{x} \frac{Lt^2}{12} \left(3 - 2t \right)+\mu \frac{t^{2}\norm{v_{\mu,L}(x)}_{x}^{2}}{2(1-t\zeta(x,v_{\mu,L}(x))}. 
\end{align*} 
for all  $t \in I_{x,\mu,L}$. Therefore, if $t\zeta(x,v_{\mu,L}(x))\leq 1/2$, we readily see%\footnote{
%Note that since $t\zeta(x,v_{\mu,L}(x))=\zeta(x,tv_{\mu,L}(x))$ and we assumed that $t\zeta(x,v_{\mu,L}(x))\leq 1/2$, it is sufficient to make Assumption \ref{ass:2ndorder} on a potentially smaller set $\widetilde{\scrT}_{x}\eqdef \{v\in\setE\vert \bA v=0,\zeta(x,v)\leq 1/2\}$ instead of $\scrT_x$ defined in \eqref{eq:Dikinv}.
%}  
\begin{align}
F_{\mu}(x^{+}(t))-F_{\mu}(x)&\leq - \frac{Lt^2\norm{v_{\mu,L}(x)}^{3}_{x} }{12} \left(3 - 2t \right)+\mu t^{2}\norm{v_{\mu,L}(x)}_{x}^{2} \nonumber \\
&= - \norm{v_{\mu,L}(x)}^{3}_{x} \frac{Lt^2}{12}\left(3 - 2t - \frac{12 \mu}{L\norm{v_{\mu,L}(x)}_{x}} \right) \eqdef -\eta_{x}(t).\label{eq:progressSOM}
\end{align}
Maximizing the above function $\eta_{x}(t)$ and finding a lower bound for its optimal value is technically quite challenging. Instead, we adopt the following step-size rule
\begin{equation}
\label{eq:SO_t_opt_def}
\ct_{\mu,L}(x) \eqdef \frac{1}{\max\{1,2\zeta(x,v_{\mu,L}(x))\}} = \min \left\{1, \frac{1}{2\zeta(x,v_{\mu,L}(x))}  \right\}.
\end{equation}  
Note that $\ct_{\mu,L}(x) \leq 1$ and $\ct_{\mu,L}(x)\zeta(x,v_{\mu,L}(x))\leq 1/2$. Thus, this choice of the step-size is feasible to derive \eqref{eq:progressSOM}. 
%%%%%%%%%%%%

Just like Algorithm \ref{alg:AHBA}, our second-order method employs a line-search procedure to estimate the Lipschitz constant $M$ in \eqref{eq:SO_Lipschitz_Gradient}, \eqref{eq:cubicestimate} in the spirit of \cite{NesPol06,CarGouToi12}. More specifically, suppose that $x^{k}\in\setX$ is the current position of the algorithm with the corresponding initial local Lipschitz estimate $M_{k}$. To determine the next iterate $x^{k+1}$, we solve problem \eqref{eq:cubicproblem} with $L= L_k = 2^{i_k}M_{k}$ starting with $ i_k =0$, find the corresponding search direction $v^{k}=v_{\mu,L_{k}}(x^{k})$ and the new point $x^{k+1} = x^{k} + \ct_{\mu, L_{k}}(x^{k})v^{k}$. 
Then, we check whether the inequalities \eqref{eq:SO_Lipschitz_Gradient} and \eqref{eq:cubicestimate} hold with  $M=L_{k}$, $x=x^{k}$, $v = \ct_{\mu, L_{k}}(x^{k})v^{k}$, see \eqref{eq:SO_LS_2} and \eqref{eq:SO_LS_1}. 
If they hold, we make a step to $x^{k+1}$. 
Otherwise, we increase $i_k$ by 1 and repeat the procedure. Obviously, when $L_{k} = 2^{i_k}M_k \geq M$, both inequalities \eqref{eq:SO_Lipschitz_Gradient} and \eqref{eq:cubicestimate} with $M$ changed to $L_k$, i.e., \eqref{eq:SO_LS_2} and \eqref{eq:SO_LS_1}, are satisfied and the line-search procedure ends. For the next iteration we set $M_{k+1} = \max\{2^{i_k-1}M_{k},\underline{L}\}=\max\{L_{k}/2,\underline{L}\}$, so that the estimate for the local Lipschitz constant on the one hand can decrease allowing larger step-sizes, and on the other hand is bounded from below. 
The resulting procedure gives rise to a \underline{S}econd-order \underline{A}daptive \underline{H}essian-\underline{B}arrier \underline{A}lgorithm ($\SAHBA$, Algorithm \ref{alg:SOAHBA}).
%%%%%%%%%%%%%%%%%%%%%%%%%%%%%%%%
\begin{algorithm}[t!]
\caption{\underline{S}econd-order \underline{A}daptive \underline{H}essian-\underline{B}arrier \underline{A}lgorithm - $\SAHBA(\mu,\eps,M_{0},x^{0})$}
\label{alg:SOAHBA}
\SetAlgoLined
\KwData{ $h \in\scrH_{\nu}(\setK)$, $\mu>0,\eps>0,M_0\geq 144  \eps,x^{0}\in\setX$.}
\KwResult{$(x^{k},y^{k-1},s^{k},M_{k})\in\setX\times\R^{m}\times\setK^{\ast}\times\R_{+}$, where $s^{k}=\nabla f(x^{k}) -\bA^{\ast}y^{k-1}$, and $M_{k}$ is the last estimate of the Lipschitz constant.}
Set $\underline{L} \eqdef 144  \eps$, $k=0$\;
%Set $144  \eps \eqdef \underline{L} < M_0$ -- initial guess for $M$\;% $\mu = \frac{\eps}{4\nu}$, $k=0$, $x^{0}\in\setX$ -- $4\nu$-analytic center (see \eqref{eq:analytic_center})
\Repeat{
		$\norm{v^{k-1}}_{x^{k-1}} < \Delta_{k-1}\eqdef\sqrt{\frac{\eps}{4L_{k-1}\nu}}$ \textnormal{ and }$\|v^{k}\|_{x^{k}} < \Delta_{k}\eqdef\sqrt{\frac{\eps}{4L_{k}\nu}}$
	}{	
		Set $i_k=0$.
		
		\Repeat{
		\begin{align}
			& f(z^{k}) \leq f(x^{k}) + \inner{\nabla f(x^{k}),z^{k}-x^{k}}+\frac{1}{2}\inner{\nabla^{2}f(x^{k})(z^{k}-x^{k}),z^{k}-x^{k}} \notag \\
		&\qquad\qquad\qquad\qquad\qquad+\frac{L_{k}}{6}\norm{z^{k}-x^{k}}^{3}_{x^{k}}, \quad \textnormal{and } \label{eq:SO_LS_1}\\
		&  \norm{\nabla f(z^{k})-\nabla f(x^{k})-\nabla^{2}f(x^{k})(z^{k}-x^{k})}^{\ast}_{x^{k}}\leq\frac{L_{k}}{2}\norm{z^{k}-x^{k}}^{2}_{x^{k}}. \label{eq:SO_LS_2}
		\end{align}
		}
		{
			Set $L_k = 2^{i_k}M_k$. Find $v^k \eqdef v_{\mu,L_k}(x^{k})$ and $y^k \eqdef y_{\mu,L_k}(x^{k})$ as a global solution to
			\begin{align}
				&\hspace{-2em} \min_{v:\bA v=0} \left\{F_{\mu}(x^k)+\inner{\nabla F_{\mu}(x^k),v}+\frac{1}{2}\inner{\nabla^{2}f(x^k)v,v}+\frac{L_{k}}{6}\norm{v}_{x^k}^{3} \right\}. \label{eq:SO_finder} \\
				&\hspace{-1em} \text{Set }\;\; \alpha_k\eqdef\min \left\{1, \frac{1}{2\zeta(x^k,v^k)}  \right\},  	\text{where $\zeta(\cdot,\cdot)$ as in \eqref{eq:zeta}}. \label{eq:SO_alpha_k}
			\end{align}
					
			Set $z^{k}=x^{k} + \alpha_k v^k$, $i_k=i_k+1$\;
		}
		Set $M_{k+1} =\max\{\frac{L_{k}}{2},\underline{L}\}$, %\max\{2^{i_k-1}M_{k},\underline{L}\}$,  
		$x^{k+1}=z^{k}$, $k=k+1$%i_k-2 is since when we find a suitable $i_k$ we still set $i_k=i_k+1$
	}
\end{algorithm}
Our main result on the iteration complexity of Algorithm \ref{alg:SOAHBA} is the following Theorem, whose proof is given in Section \ref{sec:ProofSOM}. 
%
\begin{theorem}
\label{Th:SOAHBA_conv}
Let Assumptions \ref{ass:1}, \ref{ass:barrier}, and \ref{ass:2ndorder} hold. Fix the error tolerance $\eps>0$, the regularization parameter $\mu= \frac{\eps}{4\nu}$, and some initial guess $M_0>144\eps$ for the Lipschitz constant. Let $(x^{k})_{k\geq 0}$ be the trajectory generated by $\SAHBA(\mu,\eps,M_{0},x^{0})$, where $x^{0}$ is a $4\nu$-analytic center satisfying \eqref{eq:analytic_center}.
Then the algorithm stops in no more than 
\begin{equation}
\label{eq:SO_main_Th_compl}
\K_{II}(\eps,x^{0})= \ceil[\bigg]{\frac{192 \nu^{3/2} \sqrt{2\max\{M,M_0\}}(f(x^{0}) - f_{\min}(\setX)+ \eps)}{\eps^{3/2} }}
\end{equation}
outer iterations, and the number of inner iterations is no more than $2(\K_{II}(\eps,x^{0})+1)+2\max\{\log_{2}(2M/M_{0}),1\}$. Moreover, the output of $\SAHBA(\mu,\eps,M_{0},x^{0})$ constitute an $(\eps,\frac{\max\{M,M_0\}\eps}{8\nu})$-2KKT point for problem \eqref{eq:Opt} in the sense of Definition \ref{def:eps_SOKKT}.
%\begin{align}
%&\|\nabla f(x^{k}) - \bA^{\ast}y^{k-1} - s^{k} \| = 0 \leq \eps, \label{eq:SO_main_Th_eps_KKT_1} \\
%& |\inner{s^{k},x^{k}}| \leq \eps,  \label{eq:SO_main_Th_eps_KKT_2}  \\
%& \bA x^{k}=b, s^{k} \in \setK^{\ast}, x^{k}\in\setK \label{eq:SO_main_Th_eps_KKT_3}  \\
%& \nabla^2f(x^k)  + H(x^k) \sqrt{\frac{M\eps}{\nu}}  \succeq 0 \;\; \text{on} \;\; \setL_0. \label{eq:SO_main_Th_eps_KKT_4} 
%\end{align}  
\end{theorem}
\begin{remark}
\label{rem:SO_complexity_simplified}
Since $f(x^{0}) - f_{\min}(\setX)$ is expected to be larger than $\eps$, and the constant $M$ is potentially large, we see that the main term in the complexity bound \eqref{eq:SO_main_Th_compl} is $O\left(\frac{\nu^{3/2}\sqrt{M}(f(x^{0}) - f_{\min}(\setX))}{\eps^{3/2}}\right)=O((\frac{\nu}{\eps})^{3/2})$.
Note that the complexity result  $O(\max\{\eps_1^{-3/2},\eps_2^{-3/2}\})$ reported in \cite{CarDucHinSid19b,CarDucHinSid19} to find an $(\eps_1,\eps_2)$-2KKT point for arbitrary $ \eps_1,\eps_2 > 0$, is known to be optimal for unconstrained smooth non-convex optimization by second-order methods under the standard Lipschitz-Hessian assumption. It can be easily obtained from our theorem by setting $\eps=\max\{\eps_1^{-3/2},\eps_2^{-3/2}\}$. 
\close
\end{remark}
%\begin{remark}
%An interesting observation is that our algorithm can be interpreted as a damped version of a cubic-regularized Newton's method. When $h \in\scrH_{\nu}(\setK)$ and we consider it as an element of $\scrH_{\nu}(\setK)$, we have that $\zeta(x^k,v^k) = \norm{v^k}_{x^k}$ and the stepsize $\alpha_k$ satisfies $\alpha_k= \min \left\{1, \frac{1}{2\norm{v^k}_{x^k}}  \right\}$. At the initial phase, when the algorithm is far from an $(\eps_1,\eps_2)$-2KKT point, we have $\norm{v^k}_{x^k} > 1/2$ and $\alpha_k = \frac{1}{2\norm{v^k}_{x^k}}<1$. When the algorithm is getting closer to $(\eps_1,\eps_2)$-2KKT point, $\norm{v^k}_{x^k}$ becomes smaller and the algorithm automatically switches to full steps $\alpha_k=1$. 
%
%At the same time our algorithm is completely different from cubic-regularized Newton's method \cite{NesPol06} applied to minimize the potential $F_{\mu}$. Indeed, we regularize by the cube of the local norm, rather than cube of the standard Euclidean norm, and we do not form a second-order Taylor expansion of $F_{\mu}$. These adjustments are needed to align the search direction subproblem with the local geometry of the feasible set. Moreover, for our algorithm, the analysis of the cubic-regularized Newton's method is not directly applicable since it relies on the stepsize 1, which may lead to infeasible iterates in our case.
%\close
%\end{remark}




\subsection{Proof of Theorem \ref{Th:SOAHBA_conv}}
\label{sec:ProofSOM}
%In this subsection we analyze the convergence of $\SAHBA$ and prove Theorem \ref{Th:SOAHBA_conv}. 
The main steps of the proof are similar to the analysis of Algorithm \ref{alg:AHBA}. We start by showing the feasibility of the iterates and correctness of the line-search process. Next, we analyze the per-iteration decrease of $F_{\mu}$ and $f$ and show that if the stopping criterion does not hold at iteration $k$, then the objective function is decreased by the value $O(\eps^{3/2})$. From this, since the objective is globally lower bounded, we conclude that the algorithm stops in  $O(\eps^{-3/2})$ iterations. Finally, we show that when the stopping criterion holds, the primal-dual pair $(x^{k}$, $y^{k-1})$ resulting from solving the cubic subproblem \eqref{eq:SO_finder} yields a dual slack variable $s^{k}$ such that this triple constitutes an second-order KKT point. 

\subsubsection{Interior point property of the iterates}
By construction $x^{0}\in\setX$. Proceeding inductively, let $x^{k}\in\setX$ be the $k$-th iterate of the algorithm, with the search direction $v^{k}\equiv v_{\mu,L}(x^{k})$.  By eq. \eqref{eq:SO_alpha_k}, the step-size $\alpha_k$ satisfies $\alpha_{k}\leq \frac{1}{2\zeta(x^{k},v^{k})}$. Consequently, $\alpha_{k}\zeta(x^{k},v^{k})\leq 1/2$ for all $k\geq 0$, and using \eqref{eq:step_length_zeta} as well as $\bA v^{k} =0$, we have that $x^{k+1}=x^{k}+\alpha_{k}v^{k}\in\setX$. By induction, it follows that $x^{k}\in\setX$ for all $k\geq 0$.

\subsubsection{Bounding the number of backtracking steps}
\label{sec:backtrack2}
%To bound the number of cycles involved in the line-search process for finding appropriate constants $L_{k}$, we proceed as in Section \ref{sec:backtrack1}. Since the derivations are analogous to the ones performed there, we skip the details, and just report here that the number of line-search iterations up to iteration $k$ of $\SAHBA(\mu,\eps,M_0,x^{0})$ is bounded as  
%\begin{align*}
%N(k)&=\sum_{j=0}^{k}(i_{j}+1)\leq i_0+1 + \sum_{j=1}^{k}\left(\log_{2}\left(\frac{L_{j}}{L_{j-1}}\right)+2\right) 
%\leq 2(k+1) + 2 \log_2\left(\frac{2\bar{M}}{M_0}\right),
%\end{align*}
%since $L_{k} \leq 2\bar{M} \eqdef 2\max\{M_0,M\}$ in the last step. Thus, on average, the inner loop ends after two trials. 
To bound the number of cycles involved in the line-search process for finding appropriate constants $L_{k}$, we proceed as in Section \ref{sec:backtrack1}. 
Let us fix an iteration $k$. The sequence $L_k = 2^{i_k} M_k$ is increasing as $i_k$ is increasing, and Assumption \ref{ass:2ndorder} holds. 
This implies \eqref{eq:cubicestimate}, and thus when $L_k = 2^{i_k} M_k \geq \max\{M,M_k\}$, the line-search process for sure stops since inequalities \eqref{eq:SO_LS_1} and \eqref{eq:SO_LS_2} hold.	
Hence, $L_k=2^{i_k} M_k \leq 2\max\{M,M_k\}$ must be the case, and, consequently, $M_{k+1}=\max\{L_k/2, \underline{L}\} \leq  \max\{\max\{M,M_k\}, \underline{L}\} = \max\{M,M_k\}$, which, by induction, gives $M_{k} \leq \bar{M}\equiv \max\{M,M_0\}$ and $L_k  \leq 2\bar{M}$. 
%
%
%$L_{k+1} = 2^{i_k-1} L_k \leq \max\{M,L_k\}$, which, by induction, gives $L_{k+1} \leq \bar{M}\eqdef\max\{M,L_0\}$. 
%Moreover, from this observation it follows that $L_k = 2^{i_k} M_k \leq 2\bar{M}=2\max\{M_0,M\}$. 
%
%
At the same time, by construction, $M_{k+1}= \max\{2^{i_k-1}M_{k},\underline{L}\} = \max\{L_k/2,\underline{L}\} \geq L_k/2 $. Hence, $L_{k+1} = 2^{i_{k+1}} M_{k+1} \geq 2^{i_{k+1}-1} L_k$ and therefore $\log_{2}\left(\frac{L_{k+1}}{L_{k}}\right)\geq i_{k+1}-1$, $\forall k\geq 0$. At the same time, at iteration $0$ we have $L_0=2^{i_0} M_0 \leq 2\bar{M}$, whence, $i_0 \leq \log_2\left(\frac{2\bar{M}}{M_0}\right)$.
Let $N(k)$ denote the number inner line-search iterations up to iteration $k$ of $\SAHBA$. Then, 
\begin{align*}
N(k)&=\sum_{j=0}^{k}(i_{j}+1)\leq i_0+1 + \sum_{j=1}^{k}\left(\log_{2}\left(\frac{L_{j}}{L_{j-1}}\right)+2\right) 
\leq 2(k+1) + 2 \log_2\left(\frac{2\bar{M}}{M_0}\right),
\end{align*}
since $L_{k} \leq 2\bar{M}= 2\max\{M,M_0\}$ in the last step. Thus, on average, the inner loop ends after two trials. 




%Next, we give a bound on the number of cycles involved in the line-search process for finding appropriate constants $L_k$. This allows us to connect the iteration complexity estimate to the number of function evaluations in the worst-case. Let us fix an iteration $k$. The sequence $L_k = 2^{i_k} M_k$ is increasing as $i_k$ is increasing and Assumption \ref{ass:2ndorder} holds. This implies \eqref{eq:cubicestimate}, and thus when $L_k = 2^{i_k} M_k \geq M$, the line-search process for sure stops since inequalities \eqref{eq:SO_LS_1} and \eqref{eq:SO_LS_2} hold.	
%Moreover, from this observation it follows that $L_k = 2^{i_k} M_k \leq 2\bar{M}=2\max\{M_0,M\}$. This also allows us to estimate the total number of backtracking steps after $k$ iterations. By construction, $M_{k+1}= \max\{2^{i_k-1}M_{k},\underline{L}\} = \max\{L_k/2,\underline{L}\} \geq L_k/2 $. Hence, $L_{k+1} = 2^{i_{k+1}} M_{k+1} \geq 2^{i_{k+1}-1} L_k$ and therefore $\log_{2}\left(\frac{L_{k+1}}{L_{k}}\right)\geq i_{k+1}-1$, $\forall k\geq 0$. At the same time, at iteration $0$ we have $L_0=2^{i_0} M_0 \leq 2\bar{M}$, whence, $i_0 \leq \log_2\left(\frac{2\bar{M}}{M_0}\right)$.
%Let $N(k)$ denote the number inner line-search iterations up to iteration $k$ of $\SAHBA$. Then, 
%\begin{align*}
%N(k)&=\sum_{j=0}^{k}(i_{j}+1)\leq i_0+1 + \sum_{j=1}^{k}\left(\log_{2}\left(\frac{L_{j}}{L_{j-1}}\right)+2\right) 
%\leq 2(k+1) + 2 \log_2\left(\frac{2\bar{M}}{M_0}\right),
%\end{align*}
%since $L_{k} \leq 2\bar{M}$ in the last step. Thus, on average, the inner loop ends after two trials. 

\subsubsection{Per-iteration analysis and a bound for the number of iterations}
Let us fix iteration counter $k$. The main assumption of this subsection is that the stopping criterion is not satisfied, i.e. either
$\|v^{k}\|_{x^{k}} \geq \Delta_{k}$ or $\|v^{k-1}\|_{x^{k-1}} \geq \Delta_{k-1}$. 
Without loss of generality, we assume that the first inequality holds, i.e., $\|v^{k}\|_{x^{k}} \geq \Delta_{k}$, and consider iteration $k$. Otherwise, if the second inequality holds, the same derivations can be made considering the iteration $k-1$ and using the second inequality $\|v^{k-1}\|_{x^{k-1}} \geq \Delta_{k-1}$. Thus, at the end of the $k$-th iteration 
\begin{equation}
\label{eq:SO_per_iter_proof_1}
\|v^{k}\|_{x^{k}} \geq \Delta_{k}=\sqrt{\frac{\eps}{4L_{k}\nu}}.
\end{equation}
Since the step-size $\alpha_k= \min\{1,\frac{1}{ 2\zeta(x^k,v^k)}\} = \ct_{\mu,L_k}(x^{k})$ in \eqref{eq:SO_alpha_k} satisfies $\alpha_k \leq 1$ and $\alpha_k \zeta(x^k,v^k) \leq 1/2$ (cf. \eqref{eq:SO_t_opt_def} and a remark after it), we can repeat the derivations of Section \ref{S:SO_alg_descr}, 
changing  \eqref{eq:cubicestimate} to  \eqref{eq:SO_LS_1}.
%changing \eqref{eq:SO_Lipschitz_Gradient} and \eqref{eq:cubicdecrease} to \eqref{eq:SO_LS_2} and \eqref{eq:SO_LS_1} respectively. 
In this way we obtain the following counterpart of \eqref{eq:progressSOM} with $t=\alpha_k$, $L=L_k$, $x=x^k$, $v_{\mu,L_k}(x^{k}) \eqdef v^k$: 
\begin{align}
F_{\mu}(x^{k+1})-F_{\mu}(x^{k})&\leq - \norm{v^{k}}^{3}_{x^{k}} \frac{L_k\alpha_k^2}{12}\left(3 - 2\alpha_k - \frac{12 \mu}{L_k\norm{v^{k}}_{x^{k}}} \right)\nonumber\\
& \leq - \norm{v^{k}}^{3}_{x^{k}} \frac{L_k\alpha_k^2}{12}\left(1 - \frac{12 \mu}{L_k\norm{v^{k}}_{x^{k}}} \right)\label{eq:SO_per_iter_proof_2},
\end{align}
where in the last inequality we used that $\alpha_k \leq 1$ by construction. 
Substituting $\mu = \frac{\eps}{4\nu}$, and using \eqref{eq:SO_per_iter_proof_1}, we obtain
\begin{align*}
1 - \frac{12 \mu}{L_k\norm{v^{k}}_{x^{k}}} &= 1 - \frac{12 \eps}{4\nu L_k\norm{v^{k}}_{x^{k}}}  %\stackrel{\eqref{eq:SO_per_iter_proof_1}}{\geq} 1 - \frac{3 \beta^2\eps}{\nu L_k\Delta_k} \\
\stackrel{\eqref{eq:SO_per_iter_proof_1}}{\geq} 1 - \frac{3 \eps}{\nu L_k\sqrt{\frac{\eps}{4L_{k}\nu}}} \\
&= 1 - \frac{6 \sqrt{\eps}}{ \sqrt{\nu L_k}} \geq 1 - \frac{6 \sqrt{\eps}}{ \sqrt{ 144 \nu \eps	}}  \geq \frac{1}{2},
\end{align*}
using that, by construction, $L_k =2^{i_k}M_k \geq \underline{L} = 144 \eps$ and that $\nu \geq 1$. 
Hence, from \eqref{eq:SO_per_iter_proof_2}, 
\begin{equation}
\label{eq:SO_per_iter_proof_3}
F_{\mu}(x^{k+1})-F_{\mu}(x^{k})\leq  - \norm{v^{k}}^{3}_{x^{k}} \frac{L_k\alpha_k^2}{24}.
\end{equation}
%Consider two possible cases of the value of the step-size $\alpha_k$ in \eqref{eq:SO_alpha_k} and substitute it into \eqref{eq:SO_per_iter_proof_3}:
Substituting into \eqref{eq:SO_per_iter_proof_3} the two possible values of the step-size $\alpha_k$ in \eqref{eq:SO_alpha_k} gives
\begin{equation}
\label{eq:SO_per_iter_proof_8}
F_{\mu}(x^{k+1})-F_{\mu}(x^{k})\leq 
\left\{
\begin{array}{ll}
- \norm{v^{k}}^{3}_{x^{k}} \frac{L_k}{24}, & \text{if }  \alpha_k=1,\\
 -  \frac{L_k\norm{v^{k}}^{3}_{x^{k}}}{96 (\zeta(x^k,v^k))^2} \stackrel{\eqref{eq:boundzeta}}{\leq} -  \frac{L_k\norm{v^{k}}_{x^{k}}}{96}& \text{if } \alpha_k=\frac{1}{2\zeta(x^k,v^k)}.
\end{array}\right.
\end{equation}
This implies
%From  \eqref{eq:SO_per_iter_proof_4} and \eqref{eq:SO_per_iter_proof_6}, we obtain that
\begin{equation}
\label{eq:SO_per_iter_proof_7}
F_{\mu}(x^{k+1})-F_{\mu}(x^{k}) \leq - \frac{L_k\norm{v^{k}}_{x^{k}}}{96} \min\left\{1, 4\norm{v^{k}}_{x^{k}}^2 \right\} \eqdef-\delta_{k}.
\end{equation}
Rearranging and summing these inequalities for $k$ from $0$ to $K-1$, and using that $L_k \geq \underline{L}$, we obtain
\begin{align}
%\label{eq:}
K\min_{k=0,...,K-1} &\frac{\underline{L}\norm{v^{k}}_{x^{k}}}{96} \min\left\{1, 4\norm{v^{k}}_{x^{k}}^2\right\} \leq  \sum_{k=0}^{K-1} \delta_k 
\leq F_{\mu}(x^{0})-F_{\mu}(x^{K}) \notag \\
&\stackrel{\eqref{eq:potential}}{=} f(x^{0}) - f(x^{K}) + \mu (h(x^{0}) - h(x^{K})) \leq f(x^{0}) - f_{\min}(\setX) + \eps, \label{eq:SO_per_iter_proof_9}
\end{align}
where we used that, by the assumptions of Theorem \ref{Th:SOAHBA_conv}, $x^{0}$ is a $4\nu$-analytic center defined in \eqref{eq:analytic_center} and $\mu = \frac{\eps}{4\nu}$, implying that $h(x^{0}) - h(x^{K}) \leq 4\nu = \eps/\mu$.
Thus, up to passing to a subsequence, we have $\norm{v^{k}}_{x^{k}} \to 0$ as $k \to \infty$, which makes the stopping criterion in Algorithm \ref{alg:SOAHBA} achievable.

Assume now that the stopping criterion does not hold for $K$ iterations of $\SAHBA$. 
Then, for all $k=0,\ldots,K-1,$ it holds that 
\begin{align}
\delta_k &= \frac{L_k}{96} \min\left\{\norm{v^{k}}_{x^{k}}, 4\norm{v^{k}}_{x^{k}}^3 \right\} 
\stackrel{\eqref{eq:SO_per_iter_proof_1}}{\geq} \frac{L_k}{96} \min\left\{  \sqrt{\frac{\eps}{4L_{k}\nu}} ,  \frac{4 \eps^{3/2}}{4^{3/2}L_{k}^{3/2}\nu^{3/2}}  \right\} \notag \\
&\stackrel{L_k \leq 2\bar{M}, \nu \geq 1}{\geq}  \frac{1}{96} \min\left\{  \frac{L_k \sqrt{\eps}}{\sqrt{8\bar{M}} \nu^{3/2}} ,  \frac{ \eps^{3/2}}{2 L_{k}^{1/2}\nu^{3/2}}  \right\}  \notag \\
&\stackrel{L_k \leq 2\bar{M},L_k \geq 144  \eps}{\geq} \frac{1}{96} \min\left\{ \frac{(144 \eps) \cdot \sqrt{\eps}}{\sqrt{8\bar{M}} \nu^{3/2}} ,  \frac{ \eps^{3/2}}{\sqrt{8\bar{M}}\nu^{3/2}}  \right\}  = \frac{\eps^{3/2}}{192 \nu^{3/2} \sqrt{2\bar{M}} },
\end{align}
Thus, from \eqref{eq:SO_per_iter_proof_9}.
\begin{align*}
K \frac{\eps^{3/2}}{192 \nu^{3/2} \sqrt{2\bar{M}} } \leq f(x^{0}) - f_{\min}(\setX) + \eps. 
\end{align*}
Hence, reacalling that $\bar{M}=\max\{M_0,M\}$, $K \leq \frac{192 \nu^{3/2} \sqrt{2\max\{M_0,M\}}(f(x^{0}) - f_{\min}(\setX)+ \eps)}{\eps^{3/2} }$,
i.e. the algorithm stops for sure after no more than this number of iterations. This, combined with the bound for the number of inner steps in Section \ref{sec:backtrack2}, proves the first statement of Theorem \ref{Th:SOAHBA_conv}.


\subsubsection{Generating a $(\eps_{1},\eps_{2})$-2KKT point}
In this section, to finish the proof of Theorem \ref{Th:SOAHBA_conv}, we show that if the stopping criterion in Algorithm \ref{alg:SOAHBA} holds, i.e. $\norm{v^{k-1}}_{x^{k-1}} < \Delta_{k-1}$ and $\norm{v^{k}}_{x^{k}} < \Delta_{k}$, then the algorithm has generated an $(\eps_{1},\eps_{2})$-2KKT point of \eqref{eq:Opt} according to Definition \ref{def:eps_SOKKT}, with $\eps_{1}=\eps$ and $\eps_{2}=\frac{\max\{M_{0},\bar{M}\}\eps}{8\nu}$.

Let the stopping criterion hold at iteration $k$. Using the first-order optimality condition \eqref{eq:opt1} for the subproblem \eqref{eq:SO_finder} solved at iteration $k-1$, there exists a dual variable $y^{k-1}\in\R^{m}$ such that \eqref{eq:opt1} holds. Now, expanding the definition of the potential \eqref{eq:potential} and adding $\nabla f(x^{k})$ to both sides, we obtain
\begin{align*}
 \nabla f(x^{k}) - & \bA^{\ast}y^{k-1} + \mu \nabla h(x^{k-1}) \\
&=\nabla f(x^{k}) - \nabla f(x^{k-1}) - \nabla^2 f(x^{k-1})v^{k-1} - \frac{L_{k-1}}{2}\norm{v^{k-1}}_{x^{k-1}}H(x^{k-1})v^{k-1}.
\end{align*}
Setting $s^{k}\eqdef \nabla f(x^{k})-\bA^{\ast}y^{k-1} \in \setE^{\ast}$ and $g^{k-1}\eqdef-\mu\nabla h(x^{k-1})$, after multiplication by $[H(x^{k-1})]^{-1}$, this is equivalent to 
\begin{align*}
[H(x^{k-1})]^{-1}\left(s^{k}-g^{k-1}\right)&=[H(x^{k-1})]^{-1}\left(\nabla f(x^{k}) - \nabla f(x^{k-1}) - \nabla^2 f(x^{k-1})v^{k-1}\right)\\
& - \frac{L_{k-1}}{2}\norm{v^{k-1}}_{x^{k-1}}v^{k-1}.
\end{align*}
Multiplying both of the above equalities, we arrive at 
\begin{align*}
\left(\norm{s^{k}-g^{k-1}}^{\ast}_{x^{k-1}}\right)^{2}
&=\left( \left\|\nabla f(x^{k}) - \nabla f(x^{k-1}) - \nabla^2 f(x^{k-1})v^{k-1}  - \frac{L_{k-1}}{2}\norm{v^{k-1}}_{x^{k-1}}H(x^{k-1})v^{k-1} \right\|_{x^{k-1}}^*\right)^{2}.
%&+\frac{L^{2}_{k-1}}{4}\norm{v^{k-1}}^{4}_{x^{k-1}}\\
%&-2\inner{\nabla f(x^{k}) - \nabla f(x^{k-1}) - \nabla^2 f(x^{k-1})v^{k-1},\frac{L_{k-1}}{2}\norm{v^{k-1}}_{x^{k-1}}v^{k-1}}.
\end{align*}
Taking the square root and applying the triangle inequality, we obtain
%Using the trivial inequality $-2\inner{a,b}\leq \norm{a}^{2}+\norm{b}^{2}$, it follows 
%\begin{align*}
%\left(\norm{s^{k}-g^{k-1}}^{\ast}_{x^{k-1}}\right)^{2}&=2\left(\norm{\nabla f(x^{k}) - \nabla f(x^{k-1}) - \nabla^2 f(x^{k-1})v^{k-1}}^{\ast}_{x^{k-1}}\right)^{2}+\frac{L^{2}_{k-1}}{2}\norm{v^{k-1}}^{4}_{x^{k-1}}.
%\end{align*}
%Whence, 
\begin{align*}
\norm{s^{k}-g^{k-1}}^{\ast}_{x^{k-1}}&\leq\norm{\nabla f(x^{k}) - \nabla f(x^{k-1}) - \nabla^2 f(x^{k-1})v^{k-1}}^{\ast}_{x^{k-1}}+\frac{L_{k-1}}{2}\norm{v^{k-1}}^{2}_{x^{k-1}}\\
&\stackrel{\eqref{eq:dualnorm},\eqref{eq:SO_LS_2},\eqref{eq:localnorm}}{\leq} \frac{L_{k-1}}{2}\norm{\alpha_{k-1}v^{k-1}}^{2}_{x_{k-1}}+\frac{L_{k-1}}{2}\norm{v^{k-1}}^{2}_{x^{k-1}}.
\end{align*}
Since the stopping criterion holds, at iteration $k-1$ we have
\begin{align}
\hspace{-1em}\zeta(x^{k-1},v^{k-1}) &\stackrel{\eqref{eq:boundzeta}}{\leq} \|v^{k-1}\|_{x^{k-1}} < \Delta_{k-1} = \sqrt{\frac{\eps}{4L_{k-1}\nu}} \leq \sqrt{\frac{\eps}{4 \cdot 144 \eps \nu}} < \frac{1}{2}\label{eq:SO_eps_KKT_proof_0},
\end{align}
where we used that, by construction, $L_{k-1} \geq \underline{L} = 144 \eps$ and that $\nu \geq 1$. Hence, by \eqref{eq:SO_alpha_k}, we have that $\alpha_{k-1}=1$ and $x^{k}=x^{k-1}+v^{k-1}$. This, in turn, implies that 
\begin{equation}
\label{eq:SO_eps_KKT_proof_1}
\norm{s^{k}-g^{k-1}}^{\ast}_{x^{k-1}}\leq  L_{k-1}\norm{v^{k-1}}^{2}_{x^{k-1}}.
\end{equation}
Now we follow the analysis of the first-order method by noting that $\norm{s^{k}-g^{k-1}}^{\ast}_{x^{k-1}}=\mu\norm{s^{k}-g^{k-1}}_{\nabla^{2}h_{\ast}(g^{k-1})}$ and $\mu=\frac{\eps}{4\nu}$, 
which implies
\begin{equation}
\label{eq:SO_eps_KKT_proof_2}
\norm{s^{k}-g^{k-1}}_{\nabla^{2}h_{\ast}(g^{k-1})}\leq\frac{L_{k-1}}{\mu}\norm{v^{k-1}}^{2}_{x^{k-1}}<\frac{L_{k-1}}{\mu}\Delta_{k-1}^{2} = \frac{L_{k-1}}{\frac{\eps}{4\nu}} \cdot \frac{\eps}{4L_{k-1}\nu} = 1.
\end{equation}
%where we used the stopping criterion and that, by the assumptions of Theorem \ref{Th:SOAHBA_conv}, $\mu=\frac{\eps}{4\nu}$.
Thus, since, by \eqref{eq:relations_1}, $g^{k-1}=-\mu\nabla h(x^{k-1})\in \setK^{\ast}$, we get that $s^{k}\in\setK^{\ast}$. By construction, $x^{k}\in \setK$ and $\bA x^{k} = b$. Thus, \eqref{eq:eps_SO_optim_equality_cones} holds. We also have that, by construction, $\|\nabla f(x^{k})-\bA^{\ast}y^{k-1} - s^{k}\|=0\leq \eps$, meaning that \eqref{eq:eps_SO_optim_grad} holds with $\eps_1=\eps$. To finish the analysis of the first-order condition, it remains to check the complementarity condition \eqref{eq:eps_SO_optim_complem}. 
We have
\begin{align*}
\inner{s^{k},x^{k}}=\inner{s^{k},x^{k-1}+v^{k-1}}=\inner{s^{k},x^{k-1}}+\inner{s^{k},v^{k-1}}.
\end{align*}
We estimate each of the two terms in the r.h.s. separately. First, 
\begin{align*}
0 \leq \inner{s^{k},x^{k-1}}&%=\inner{\nabla f(x^{k})-\bA^{\ast}y^{k-1},x^{k-1}}=
=  \inner{s^{k}-g^{k-1},x^{k-1}} + \inner{g^{k-1},x^{k-1}} \\
&\leq \norm{s^{k}-g^{k-1}}^{\ast}_{x^{k-1}}\cdot\norm{x^{k-1}}_{x^{k-1}}-\mu\inner{\nabla h(x^{k-1}),x^{k-1}} \\
&\stackrel{\eqref{eq:SO_eps_KKT_proof_1},\eqref{eq:log_hom_scb_norm_prop},\eqref{eq:log_hom_scb_hess_prop}}{\leq}  L_{k-1}\norm{v^{k-1}}^{2}_{x^{k-1}}\sqrt{\nu}+\mu\nu. %\Delta^{2}_{k-1}
\end{align*}
Second, 
\begin{align*}
 \inner{s^{k},v^{k-1}}&\leq \norm{s^{k}}^{\ast}_{x^{k-1}}\cdot\norm{v^{k-1}}_{x^{k-1}}\leq \left(\norm{s^{k}-g^{k-1}}^{\ast}_{x^{k-1}}+\norm{g^{k-1}}^{\ast}_{x^{k-1}}\right)\cdot \norm{v^{k-1}}_{x^{k-1}}\\
&\stackrel{\eqref{eq:SO_eps_KKT_proof_1},\eqref{eq:log_hom_scb_norm_prop},\eqref{eq:log_hom_scb_hess_prop}}{\leq}  \left( L_{k-1}\norm{v^{k-1}}^{2}_{x^{k-1}}+\mu\sqrt{\nu}\right)\Delta_{k-1}.
\end{align*}
Summing up, using the stopping criterion $\norm{v^{k-1}}_{x^{k-1}}<\Delta_{k-1}$ and that, by \eqref{eq:SO_eps_KKT_proof_0}, $\Delta_{k-1} \leq 1\leq\sqrt{\nu}$, we obtain
\begin{align}
 0 \leq \inner{s^k,x^k} = \inner{s^k,x^{k-1}+v^{k-1}} \leq  2L_{k-1}\Delta_{k-1}^2 \sqrt{\nu} + 2\mu \nu = 
2 L_{k-1} \frac{\eps}{4L_{k-1}\nu}  \sqrt{\nu} + 2\frac{\eps}{4\nu} \nu \leq  \eps, \label{eq:SO_eps_KKT_proof_3}
\end{align}
i.e., \eqref{eq:eps_SO_optim_complem} holds with $\eps_1=\eps$.\\
Finally, we show the second-order condition \eqref{eq:eps_SO_optim_SO}.
%It remains to show the second-order approximate stationarity \eqref{eq:SO_main_Th_eps_KKT_4}. 
By inequality \eqref{eq:PD} for subproblem \eqref{eq:SO_finder} solved at iteration $k$, we obtain on $\setL_0$
\begin{align}
\nabla^2 f(x^{k})  &\succeq - \frac{L_{k}\norm{v^{k}}_{x^{k}}}{2} H(x^{k}) \succeq -\frac{L_{k} \Delta_k}{2} H(x^{k}) \notag \\
& =  - \frac{L_{k}}{2}\sqrt{\frac{\eps}{4L_{k}\nu}} H(x^{k}) = - \frac{\sqrt{L_k \eps}}{4\nu^{1/2}}H(x^{k})  
\succeq - \frac{ \sqrt{2\bar{M}\eps}}{4\nu^{1/2}}H(x^{k}) \label{eq:SO_eps_KKT_proof_4}
\end{align}
where we used the second part of the stopping criterion, i.e. $\norm{v^{k}}_{x^{k}}< \Delta_k$ and that $L_k \leq 2\bar{M}=2\max\{M,M_0\}$ (see Section \ref{sec:backtrack2}). Thus, \eqref{eq:eps_SO_optim_SO} holds with $\eps_2=\frac{\max\{M,M_0\}\eps}{8\nu}$, which finishes the proof of Theorem \ref{Th:SOAHBA_conv}.
%Using the 	second-order optimality condition \eqref{eq:PD} for the subproblem \eqref{eq:cubicproblem} solved at iteration $k$, we obtain, on $\setL_0$
%\begin{equation}
%%\label{eq:}
%[H(x^{k})]^{-1/2} \nabla^2 f(x^{k}) [H(x^{k})]^{-1/2} \succeq - \frac{L_{k}\norm{v^{k}}_{x^{k}}}{2} I \succeq -\frac{L_{k} \Delta_k}{2} I =  - \frac{L_{k}}{2} \beta\sqrt{\frac{\eps}{4L_{k}\sqrt{\nu}}} I = - \frac{\beta \sqrt{L_k \eps}}{4\nu^{1/4}}I  \succeq - \frac{\beta \sqrt{2M\eps}}{4\nu^{1/4}}I
%\end{equation}
%where we used the second part of the stopping criterion, i.e. $\norm{v^{k}}_{x^{k}}< \Delta_k$ and that $L_k \leq 2M$. This finishes the proof of \eqref{eq:SO_main_Th_eps_KKT_4} and the proof of Theorem \ref{Th:SOAHBA_conv}.



\subsection{Discussion}
\label{sec:SO_discussion}
%The analysis of $\SAHBA$ requires to overcome similar technical challenges to that of the analysis of $\AHBA$. Namely, we need to guarantee that $s^{k} \in\setK^{\ast}$, show that \eqref{eq:eps_SO_optim_complem} holds, and obtain the condition \eqref{eq:eps_SO_optim_grad}  formulated in terms of the standard Euclidean norm based on inequality \eqref{eq:SO_eps_KKT_proof_1} formulated in terms of the local norm. Additionally, we need to guarantee that the second-order condition \eqref{eq:eps_SO_optim_SO} holds. And again, in our setting of general, potentially non-symmetric, cones all this is more difficult than for the particular case of the non-negativity constraints considered in \cite{HaeLiuYe18,NeiWr20}, where a second-order algorithm and a first-order implementation of a second-order algorithm are proposed respectively. 
\paragraph{Strengthened KKT condition.}
As in Section \ref{sec:FO_discussion}, our aim in this section is to compare our result with those available in the contemporary literature. We therefore onsider the special case $\bar{\setK}=\bar{\setK}_{\text{NN}}$, endowed with the standard log-barrier $h(x)=-\sum_{i=1}^n \ln(x_i)$. Recall that for this barrier setup we have $\nabla h(x)=[-x_{1}^{-1},\ldots,-x_{n}^{-1}]^{\top}$, $H(x)=\diag[x_{1}^{-2},\ldots,x_{n}^{-2}]=\XX^{-2}$. Assume that the stopping criterion applies at iteration $k$. Using the first-order optimality condition \eqref{eq:opt1} for the subproblem \eqref{eq:SO_finder} solved at iteration $k-1$ and expanding the definition of the potential \eqref{eq:potential}, there exists a dual variable $y^{k-1}\in\R^{m}$ such that \eqref{eq:opt1} holds, i.e., 
\begin{align*}
 \nabla f(x^{k-1}) + \mu \nabla h(x^{k-1})  + \nabla^2 f(x^{k-1})v^{k-1} - \bA^{\ast}y^{k-1} =  - \frac{L_{k-1}}{2}\norm{v^{k-1}}_{x^{k-1}}H(x^{k-1})v^{k-1}.
\end{align*}
Multiplying both sides by $H(x^{k-1})^{-1/2}$, using the stopping criterion $\norm{v^{k-1}}_{x^{k-1}}<\sqrt{\frac{\eps}{4\nu L_{k-1}}}$, since $H(x^{k-1})^{-1/2}\nabla h(x^{k-1})= - \1_{n}$ and $\nu=n$, we obtain
\begin{align}
&\norm{\XX^{k-1}(\nabla^2 f(x^{k-1})v^{k-1} + \nabla f(x^{k-1})-\bA^{\ast}y^{k-1}) - \mu \1_{n}}_{\infty} \notag \\
& \leq \norm{\XX^{k-1} (\nabla^2 f(x^{k-1})v^{k-1} + \nabla f(x^{k-1})-\bA^{\ast}y^{k-1}) - \mu \1_{n}  } = \frac{L_{k-1}}{2} \norm{-\XX^{k} v^{k-1}}^2\notag\\
& < \frac{\eps}{8n}. \label{eq:SO_remarks_1}
\end{align}
Whence, since $\mu=\frac{\eps}{4n}$, the above bound \eqref{eq:SO_remarks_1} combined with the triangle inequality yields
\begin{align}
& \norm{\XX^{k-1}(\nabla^2 f(x^{k-1})v^{k-1} +\nabla f(x^{k-1})-\bA^{\ast}y^{k-1})}_{\infty} \notag \\
&\leq \norm{\XX^{k-1} (\nabla^2 f(x^{k-1})v^{k-1} + \nabla f(x^{k-1})-\bA^{\ast}y^{k-1}) - \mu \1_{n}  } +\norm{\mu\1_{n}} \notag\\
&=\frac{L_{k-1}}{2} \norm{-\XX^{k} v^{k-1}}^2+\norm{\mu\1_{n}}<\frac{3\eps}{8n}. \label{eq:SO_remarks_2}
\end{align}
Let $\mathbf{V}^{k-1} = \diag(v^{k-1})$. Using the fact that $x^{k}=x^{k-1}+v^{k-1}$ shown after \eqref{eq:SO_eps_KKT_proof_0}, we obtain
\begin{align*}
& \norm{\XX^{k}(\nabla f(x^{k})-\bA^{\ast}y^{k-1})}_{\infty} \notag \\
& = \norm{(\XX^{k-1}+\mathbf{V}^{k-1})(\nabla^2 f(x^{k-1})v^{k-1} +\nabla f(x^{k-1})-\bA^{\ast}y^{k-1} + \nabla f(x^{k}) - \nabla f(x^{k-1}) - \nabla^2 f(x^{k-1})v^{k-1})}_{\infty} \notag  \\
& \leq \norm{\XX^{k-1}(\nabla^2 f(x^{k-1})v^{k-1} +\nabla f(x^{k-1})-\bA^{\ast}y^{k-1} )}_{\infty}\\ %\label{eq:SO_remarks_3} \\
& \hspace{1em} + \norm{\XX^{k-1} (\nabla f(x^{k}) - \nabla f(x^{k-1}) - \nabla^2 f(x^{k-1})v^{k-1})}_{\infty}\\ %\label{eq:SO_remarks_4} \\
& \hspace{1em} + \norm{\mathbf{V}^{k-1}(\nabla^2 f(x^{k-1})v^{k-1} +\nabla f(x^{k-1})-\bA^{\ast}y^{k-1} )}_{\infty}\\ %\label{eq:SO_remarks_5}\\
& \hspace{1em} + \norm{ \mathbf{V}^{k-1}(\nabla f(x^{k}) - \nabla f(x^{k-1}) - \nabla^2 f(x^{k-1})v^{k-1})}_{\infty}\\ %\label{eq:SO_remarks_6}\\
&=I+II+III+IV.
\end{align*}
Let us estimate each of the four terms $I-IV$, using two technical facts \eqref{eq:technical_1}, \eqref{eq:technical_2} proved in Appendix \ref{sec:Appendix}. We have:
\begin{align*}
 I& \stackrel{\eqref{eq:SO_remarks_2}}{<} \frac{3\eps}{8n}, \\
II& \leq \norm{\XX^{k-1} (\nabla f(x^{k}) - \nabla f(x^{k-1}) - \nabla^2 f(x^{k-1})v^{k-1})}\\
& = \norm{\nabla f(x^{k}) - \nabla f(x^{k-1}) - \nabla^2 f(x^{k-1})v^{k-1}}_{x^{k-1}}^* \stackrel{\eqref{eq:SO_LS_2}}{\leq} \frac{L_{k-1}}{2}\norm{v^{k-1}}_{x^{k-1}}^2<\frac{\eps}{8n}, \\
III&  \stackrel{\eqref{eq:technical_2}}{\leq} \norm{v^{k-1}}_{x^{k-1}} \cdot \norm{\XX^{k-1} (\nabla^2 f(x^{k-1})v^{k-1} +\nabla f(x^{k-1})-\bA^{\ast}y^{k-1} )}_{\infty}  \stackrel{\eqref{eq:SO_eps_KKT_proof_0},\eqref{eq:SO_remarks_2}}{<}\frac{3\eps}{8n},
\end{align*}
where we have used $x^{k}=z^{k-1}=x^{k-1}+v^{k-1}$ in bounding $II$, and the last bound for expression $III$ uses $\norm{v^{k-1}}_{x^{k-1}}<1$, which is implied by eq. \eqref{eq:SO_eps_KKT_proof_0}. Finally, we also obtain
\begin{align*}
IV& \stackrel{\eqref{eq:technical_1}}{\leq} \norm{v^{k-1}}_{x^{k-1}} \cdot \norm{\nabla f(x^{k}) - \nabla f(x^{k-1}) - \nabla^2 f(x^{k-1})v^{k-1}}_{x^{k-1}}^{*}\\
&\stackrel{\eqref{eq:SO_eps_KKT_proof_0},\eqref{eq:SO_LS_2}}{\leq} \frac{L_{k-1}}{2}\norm{v^{k-1}}_{x^{k-1}}^2 <\frac{\eps}{8n}.
\end{align*}
Summarizing, we arrive at
\begin{equation}
\label{eq:SO_remarks_7}
\norm{\XX^{k}(\nabla f(x^{k})-\bA^{\ast}y^{k-1})}_{\infty} \leq \frac{\eps}{n}. 
\end{equation}
Further, by Theorem \ref{Th:SOAHBA_conv}, we have that $\nabla f(x^{k})-\bA^{\ast}y^{k-1} = s^k \in \setK^{\ast}_{\text{NN}}=\Rn_{++}$, and
\[
\nabla^2f(x^k)  + H(x^k) \sqrt{\frac{M\eps}{n}}  \succeq 0 \;\; \text{on} \;\; \setL_0.
\] 
By Remark \ref{rem:SO_complexity_simplified}, these inequalities are achieved after $O\left(\frac{\sqrt{M} n^{3/2}(f(x^{0}) - f_{\min}(\setX))}{\eps^{3/2}}\right)$ iterations. Assuming that $M\geq 1$, if we change $\eps \to \tilde{\eps}=\min\{n\eps, n\eps/M\}$, we obtain from these inequalities that in 
$O\left(\frac{\sqrt{M} n^{3/2}(f(x^{0}) - f_{\min}(\setX))}{\tilde{\eps}^{3/2}}\right) = O\left(\frac{M^2 (f(x^{0}) - f_{\min}(\setX))}{\eps^{3/2}}\right)$ iterations $\SAHBA$ guarantees
\begin{align*}
& x^{k} > 0, \; \nabla f(x^{k})-\bA^{\ast}y^{k-1} >0 \\
&\norm{\XX^{k}(\nabla f(x^{k})-\bA^{\ast}y^{k-1})}_{\infty} \leq \frac{\tilde{\eps}}{n} \leq \eps, \\
& \nabla^2f(x^k)  + H(x^k) \sqrt{\eps}  \succeq \nabla^2f(x^k)  + H(x^k) \sqrt{\frac{M\tilde{\eps}}{n}}  \succeq 0 \;\; \text{on} \;\; \setL_0.
\end{align*}
In contrast, the second-order algorithm of \cite{HaeLiuYe18} requires an additional assumption that the level set of the objective $f$ is bounded in the $L_{\infty}$-norm, gives a slightly worse guarantee $\nabla f(x^{k})-\bA^{\ast}y^{k-1} > -\eps$, and requires a larger number of iterations $O\left(\frac{ \max\{M,R\}^{7/2}(f(x^{0}) - f_{\min}(\setX))}{\eps^{3/2}}\right)$ ($R$ denoting the $L_{\infty}$ upper bound of the level set corresponding to $x^{0}$).  We also can repeat the same remark as in Section \ref{sec:FO_discussion} that our measure of complementarity $0\leq \inner{s^{k},x^{k}}\leq \eps$ is stronger than $\max_{1\leq i\leq n}\abs{x_{i}^{k}s_{i}^{k}}$ used in \cite{HaeLiuYe18,NeiWr20}. Furthermore, our algorithm is applicable to general cones admitting an efficient barrier setup, rather than only for $\bar{\setK}_{\text{NN}}$. For more general cones we can not use the coupling $H(x)^{-\frac{1}{2}} = \XX$, which was seen to be very helpful in the derivations of the bound \eqref{eq:SO_remarks_7} above. Thus, to deal with general cones, we had to find and exploit suitable properties of the barrier class $\scrH_{\nu}(\setK)$ and develop a new analysis technique that works for general, potentially non-symmetric, cones. Finally, our method does not rely on the trust-region techniques as in \cite{HaeLiuYe18} that may slow down the convergence in practice since the radius of the trust region is no grater than $O(\sqrt{\eps})$ leading to short steps.

\paragraph{Exploiting problem structure.}
We note that in \eqref{eq:SO_per_iter_proof_8} we can clearly observe the benefit of the use of $\nu$-SSB in our algorithm. When $\alpha_k=\frac{1}{2\zeta(x^k,v^k)}$, the per-iteration decrease of the potential is  $\frac{L_k\norm{v^{k}}^{3}_{x^{k}}}{96 (\zeta(x^k,v^k))^2} \geq \frac{ \sqrt{\eps L_k} \norm{v^{k}}_{x^{k}}^2}{96 \sqrt{4\nu} (\zeta(x^k,v^k))^2} $ which may be large if $\zeta(x^k,v^k)=\sigma_{x^k}(-v^k) \ll \norm{v^{k}}_{x^{k}}$.

\paragraph{Dependence on parameters.}
Next, we discuss more explicitly, how the algorithm and complexity bounds depend on the parameter $\mu$. The first observation is that from \eqref{eq:SO_eps_KKT_proof_2}, to guarantee that $s^{k} \in \setK^{\ast}$, we need the stopping criterion to be $\norm{v^{k-1}}_{x^{k-1}} < \Delta_{k-1}= \sqrt{\mu/L_{k-1}}$, which by \eqref{eq:SO_eps_KKT_proof_3} leads to the error $4 \mu \nu$ in the complementarity conditions and by \eqref{eq:SO_eps_KKT_proof_4} leads to the error $\sqrt{\mu/\bar{M}}$ in the second-order condition. From the analysis following equation \eqref{eq:SO_per_iter_proof_8}, we have that  
\[
K\frac{\mu^{3/2}}{24  \sqrt{\bar{M}}} %= K \min\left\{\frac{\mu}{4},\frac{\mu^{2}}{4 \ (\bar{M}+\mu)}\right\}
\leq f(x^{0})-f_{\min}(\setX)+\mu \nu.
\]
Whence, recalling that $\bar{M}=\max\{M,M_0\}$,
\[
K \leq 24(f(x^{0}) - f_{\min}(\setX)+ \mu \nu) \cdot \frac{\sqrt{2\max\{M,M_0\}}}{\mu^{3/2}}.%\max\left\{\frac{\nu}{\eps},\frac{\nu^{2}(M+\eps/\nu)}{\eps^{2}}\right\},
\]
Thus, we see that after $O(\mu^{-3/2})$ iterations the algorithm finds a $(4 \mu \nu,\mu/\bar{M})$-KKT point, and  if $\mu \to 0$, we have convergence to a KKT point, but the complexity bound tends to infinity and becomes non-informative. At the same time, as it is seen from \eqref{eq:SO_finder}, when $\mu \to 0$, the algorithm resembles a cubic-regularized Newton method, but with the regularization with the cube of the local norm. We also see from the above explicit expressions in terms of $\mu$ that the design of the algorithm requires careful balance between the desired accuracy of the approximate KKT point expressed mainly by the complementarity conditions, stopping criterion, and complexity. Moreover, the step-size must be selected carefully to ensure the feasibility of the iterates.

\subsection{Anytime convergence via restarting $\SAHBA$}
Similarly to the restarted $\AHBA$ (Algorithm \ref{alg:RestartHBA}), we can obtain anytime convergence envoking a restarted method that uses $\SAHBA$ as an inner procedure. We fix $\eps_{0}>0$ and select the starting point $x_0^{0}$ as a $4\nu$-analytic center of $\setX$ in the sense of eq. \eqref{eq:analytic_center}. In epoch $i\geq 0$ we generate a sequence 
$\{x_{i}^{k}\}_{k=0}^{K_{i}}$  by calling $\SAHBA(\mu_{i},\eps_{i},M_0^{(i)},x^{0}_{i})$ with $\mu_{i}=\frac{\eps_{i}}{4\nu}$ until the stopping condition is reached. We know that this inner procedure terminates after at most $\K_{II}(\eps_{i},x^{0}_{i})$ iterations. Store the values $x^{K_{i}}_{i}$ and $M_{K_{i}}^{(i)}$, and set $x^{i+1}_{0}\equiv x^{K_{i}}_{i}$, as well as $M_{0}^{(i+1)}\equiv M_{K_{i}}^{(i)}/2$. Updating the parameters to $\mu_{i+1}$ and $\eps_{i+1}$, we restart by calling procedure $\SAHBA(\mu_{i+1},\eps_{i+1},M_0^{(i+1)},x^{0}_{i+1})$ anew. This is formalized in Algorithm \ref{alg:RestartSAHBA}. 
%%%%%%%%%%%%%%%%%%%%%%%%%%%%
\begin{algorithm}[h!]
\caption{Restarting $\SAHBA$}
\label{alg:RestartSAHBA}
\SetAlgoLined
\KwData{ $h \in\scrH_{\nu}(\setK)$, $\eps_{0}>0$, $x_0^{0}\in\setX$ -- $4\nu$-analytic center, $M_0^{(0)}\geq 144 \eps_0$.}
\KwResult{Point $\hat{x}_i$, dual variables $\hat{y}_i$, $\hat{s}_i = \nabla f(\hat{x}_i) -\bA^{\ast}\hat{y}_i$.}
%Set $L_{0}^{0} > 0$ -- initial guess for $M$\; 
%$\hat{x}_{0}=x^{0},\hat{M}_{0}=L_{0}$, $\mu_{0}=\frac{\eps_0}{4\nu}$\;
\For{$i=0,1,\ldots$} 
{ Set $\eps_{i}=2^{-i}\eps_{0}$, $\mu_{i}=\frac{\eps_i}{4\nu}$\; 
Obtain $(\hat{x}_{i},\hat{y}_{i},\hat{s}_{i},\hat{M}_{i})$ from $\SAHBA(\mu_{i},\eps_i,M_0^{(i)},x^{0}_{i})$\;
 %
% $(x_{i}^{K_{i}},L_{i}^{K_{i}})$ from $\AHBA(\mu_{i},\eps_i,x^{0}_{i})$\; 
 Set $x_{i+1}^{0}=\hat{x}_{i}$ and $M_0^{(i+1)}=\hat{M}_{i}/2$.
	}
\end{algorithm}
%%%%%%%%%%%%%%%%%	
\begin{theorem}\label{th:ComplexityPathfollowingSAHBA}
Let Assumptions \ref{ass:1}, \ref{ass:barrier}, \ref{ass:2ndorder} hold. 
Then, for any $\eps \in (0,\eps_0)$, Algorithm \ref{alg:RestartSAHBA} finds an $(\eps,\frac{\max\{M,M_0^{(0)}\}\eps}{8\nu})$-2KKT point for problem \eqref{eq:Opt} in the sense of Definition \ref{def:eps_SOKKT} after no more than $I(\eps)\eqdef\lceil \log_{2}(\eps_{0}/\eps)\rceil+1$ restarts and at most 
$
\left\lceil 841(f(x^{0})-f_{\min}(\setX)+\eps_0)\nu^{3/2}\eps^{-3/2}\sqrt{2\max\{M,M_0^{(0)}\}}\right\rceil
$
%841 \frac{1536}{(2\sqrt{2}-1)
 iterations of $\SAHBA$.
\end{theorem}
\begin{proof}
Let us consider a restart $i \geq 0$ and mimic the proof of Theorem \ref{Th:SOAHBA_conv} with the substitution $\eps \to \eps_i$,  $\mu \to \mu_i = \eps_i/(4\nu)$, $M_0 \to M_0^{(i)}=\hat{M}_{i-1}/2$, $\underline{L} = 144  \eps \to \underline{L}_i = 144  \eps_i$, $\bar{M}=\max\{M,M_0\} \to \bar{M}_i=\max\{M,M_0^{(i)}\}$, $x^0 \to x^{0}_{i}=\hat{x}_{i-1}$. Note that $M_0^{(i)} \geq 144 \eps_i=\underline{L}_i$ for $i\geq 0$. We verify this via induction. By construction $M_0^{(0)}\geq 144 \eps_0$. Assume the bound holds for some $i\geq 1$. Then,  $M_0^{(i+1)}=M_{K_{i}}^{(i)}/2=\max\{L_{K_{i}-1}^{(i)}/2,\underline{L}_{i}\}/2\geq 144 \eps_{i}/2=144 \eps_{i+1}$, where we have used the induction hypothesis and the definition of the sequence $\eps_{i}$.\\
Let $K_i$ be the last iteration of $\SAHBA(\mu_{i},\eps_i,M_0^{(i)},x^{0}_{i})$ meaning that the stopping criterion does not hold at the inner iterations $k=0,\ldots,K_i-1$. From the analysis following equation \eqref{eq:SO_per_iter_proof_7}, we obtain
\begin{equation} \label{eq:SO_PF_proof_1} 
K_i \frac{\eps_i^{3/2}}{192 \nu^{3/2} \sqrt{2\bar{M}_i}} \leq F_{\mu_i}(x^{0}_i)-F_{\mu_i}(x^{K_i}_i).
\end{equation}
Using that $\mu_i$ is a decreasing sequence and $x_{0}^{0}$ is a $4\nu$-analytic center, we see
\begin{align}
F_{\mu_{i+1}}(x^{0}_{i+1})&=F_{\mu_{i+1}}(x^{K_i}_{i})=f(x^{K_i}_{i}) + \mu_{i+1} h(x^{K_i}_{i})=F_{\mu_{i}}(x^{K_i}_{i}) + (\mu_{i+1} - \mu_{i}) h(x^{K_i}_{i}) \notag \\
&\stackrel{\eqref{eq:analytic_center}}{\leq} F_{\mu_{i}}(x^{K_i}_{i}) + (\mu_{i+1} - \mu_{i}) (h(x_0^0)-4\nu)\notag \\
& \stackrel{\eqref{eq:SO_PF_proof_1}}{\leq}F_{\mu_i}(x^{0}_i)-K_i \frac{\eps_i^{3/2}}{192 \nu^{3/2} \sqrt{2\bar{M}_i} }  + (\mu_{i+1} - \mu_{i}) (h(x_0^0)-4\nu).
\label{eq:SO_PF_proof_2} 
\end{align}
Let $I=I(\eps)=\left\lceil \log_2 \frac{\eps_0}{\eps} \right\rceil+1$. By Theorem \ref{Th:SOAHBA_conv} applied to the restart $I-1$, we see that $\SAHBA(\mu_{I-1},\eps_{I-1},M_0^{(I-1)},x^{0}_{I-1})$ outputs an $(\eps_{I-1},\frac{\bar{M}_{I-1}\eps_{I-1}}{8\nu})$-2KKT point for problem \eqref{eq:Opt} in the sense of Definition \ref{def:eps_SOKKT}. Since $\eps_{I-1}=\eps$ and, for all $i\geq 1$,
\begin{align}
\bar{M}_i &= \max\{M,M_0^{(i)}\} = \max\{M,\hat{M}_{i-1}/2\} = \max\{M,M_{K_{i-1}}^{(i-1)}/2\} \notag\\
&= \max\{M,\max\{L_{K_{i-1}-1}^{(i-1)}/2,\underline{L}_{i-1}\}/2\} \notag\\
&\leq  \max\{M,\max\{\bar{M}_{i-1},M_0^{(i-1)}\}/2\} \leq  \max\{M,\bar{M}_{i-1}\} \notag\\
&\leq ... \leq \max\{M,\bar{M}_{0}\} \leq  \max\{M,M_0^{(0)}\}.\label{eq:SO_PF_proof_3} 
\end{align}
it follows that actually we generate an $(\eps,\frac{\max\{M,M_{0}^{(0)}\}\eps}{8\nu})$-2KKT point. Summing inequalities \eqref{eq:SO_PF_proof_2} for all the performed restarts $i=0,...,I-1$ and rearranging the terms, we obtain
\begin{align*}
\sum_{i=0}^{I-1} K_i \frac{\eps_i^{3/2}}{192 \nu^{3/2} \sqrt{2\bar{M}_i} }  & \leq F_{\mu_0}(x^{0}_0) - F_{\mu_{I}}(x^{0}_{I}) + (\mu_{I} - \mu_{0}) (h(x_0^0)-4\nu) \notag \\
&=f(x^{0}_0) + \mu_0 h(x^{0}_0)- f(x^{0}_{I}) - \mu_{I}h(x^{0}_{I}) + (\mu_{I} - \mu_{0}) (h(x_0^0)-4\nu) \notag \\
& \stackrel{\eqref{eq:analytic_center}}{\leq} f(x^{0}_0) - f_{\min}(\setX) + \mu_0 h(x^{0}_0) - \mu_{I}h(x^{0}_{0}) + 4\mu_{I} \nu + (\mu_{I} - \mu_{0}) (h(x_0^0)-4\nu) \notag \\ 
& \leq f(x^{0}_0) - f_{\min}(\setX) + 4\mu_0 \nu = f(x^{0}_0) - f_{\min}(\setX) + \eps_0,
\end{align*}
where in the last steps we have used the coupling $\mu_{0}=\eps_{0}/\nu$. From this inequality, using \eqref{eq:SO_PF_proof_3}, we obtain
\begin{align}
K_i \leq  (f(x^{0}) - f_{\min}(\setX)+  \eps_0) \cdot \frac{192\nu^{3/2}\sqrt{2\bar{M}_i}}{\eps_i^{3/2}} \leq \frac{C}{\eps_i^{3/2}},
\label{eq:SO_PF_proof_4} 
\end{align}
where $C \equiv 192(f(x^{0})-f_{\min}(\setX)+\eps_0)\nu^{3/2}\sqrt{2\max\{M,M_0^{(0)}\}}$. Finally, we obtain that the total number of iterations of procedures $\SAHBA(\mu_{i},\eps_{i},M_0^{(i)},x_{i}^{0}),0\leq i \leq I-1$, to reach accuracy $\eps$ is at most
\begin{align*}
\sum_{i=0}^{I-1}K_{i}&\leq \sum_{i=0}^{I-1} \frac{C}{\eps_i^{3/2}} \leq \frac{C}{\eps_0^{3/2}} \sum_{i=0}^{I-1} (2^i)^{3/2} \\
&\leq \frac{C}{\eps_0^{3/2}} \cdot \frac{2^{3/2\cdot(2+\log_2(\frac{\eps_0}{\eps}))}-1}{2^{3/2}-1}\leq \frac{8C}{(\sqrt{8}-1)\eps^{3/2}}\\
&<\frac{841(f(x^{0})-f_{\min}(\setX)+\eps_0)\nu^{3/2}\sqrt{2\max\{M,M_0^{(0)}\}}}{\eps^{3/2}}.
\end{align*} 
\end{proof}


%----------------------------------------------------------------------
%%% CONCLUSIONS
%----------------------------------------------------------------------
\section{Conclusion}
\label{sec:conclusion}

\begin{comment}
\begin{figure}
\includegraphics[width=\linewidth]{figs/beyond_tss_lesion.pdf}
\caption[]{End-to-End runtime lesion study of the entire MNIST dataset and the FMA featurized music dataset. Each of DROP's contributions provides a runtime improvement.}
\label{fig:beyond_lesion}
\end{figure}
\end{comment}



\section{Conclusion}
\label{sec:conclusion}

Advanced data analytics techniques must scale to rising data volumes. 
DR techniques offer a powerful toolkit when processing these datasets, with PCA frequently outperforming popular techniques in exchange for high computational cost. 
In response, we propose DROP, a new dimensionality reduction optimizer. 
DROP combines progressive sampling, progress estimation, and online aggregation to identify high quality low dimensional bases via PCA without processing the entire dataset by balancing the runtime of downstream tasks and achieved dimensionality. 
Thus, DROP provides a first step in bridging the gap between quality and efficiency in end-to-end DR for downstream \red{analytics}. 

%We revisit canonical operators for time series dimensionality reduction and the measurement study of~\cite{keogh-study}, and show that PCA is more effective than popular alternatives in the data mining literature often by a margin of over $2\times$ on average on gold-standard time series benchmark data sets with respect to output data dimension. More surprisingly, we empirically demonstrate that a small number of samples are sufficient to accurately characterize directions of maximum variance and obtain a high-quality low-dimensional transformation.




\section*{Acknowledgments}
We would like to thank Yurii Nesterov, Anton Rodomanov, Nikita Doikov, Giovanni Grapiglia and Radu Dragomir for fruitful discussions that allowed to improve the quality of the paper. M. Staudigl acknowledges financial support from the COST Action CA16228 "European Network for Game Theory".

\appendix
\section{More results on Self-concordant barriers}
\label{app:barrier}
%----------------------------------------------------------------------
%%% Appendix - SC
%----------------------------------------------------------------------
% !TEX root = ./HBAConicMain.tex


The \emph{dual cone} $\bar{\setK}^{\ast}$ is defined as $\bar{\setK}^{\ast}\eqdef\{s\in\setE^{\ast}\vert\inner{s,x}\geq 0\;\forall x\in\bar{\setK}\}$, and the \emph{dual barrier} $h_{\ast}(s)\eqdef\sup_{x\in\setK}\{\inner{-s,x}-h(x)\}$ for $s\in\bar{\setK}^{\ast}$. From \cite[Thm 3.3.1]{Ren01} we know that if $h\in\scrH_{\nu}(\setK)$, then $h_{\ast}\in\scrH_{\nu}(\setK^{\ast})$. Moreover, 
\begin{align}
&x \in \setK \Rightarrow -\nabla h(x) \in \setK^{\ast},\label{eq:relations_1}\\
&s=-\nabla h(x)\iff \nabla h_{\ast}(s)=-x\Rightarrow \nabla^{2}h_{\ast}(s)=[\nabla^{2}h(x)]^{-1}. \label{eq:relations}
\end{align}
We will also need the following properties listed in \cite[Lemma 5.4.3]{Nes18}.
\begin{proposition}
\label{prop:logSCB}
Let $h\in\scrH_{\nu}(\setK)$, $x\in\setK$, $t>0$ and recall that $H(x)=\nabla^{2}h(x)$.
Then,
\begin{align}
& \nabla^2 h(t x) = t^{-2} \nabla^2 h(x), \label{eq:log_hom_scb_hess_homog_prop}\\ 
&-\inner{\nabla h(x),x} = \nu, \label{eq:log_hom_scb_hess_prop}\\
&\norm{x}_{x}^2 = \inner{ H(x) x ,x } = \nu, \quad \inner{ \nabla h(x), [H(x)]^{-1}\nabla h(x) } = \nu. \label{eq:log_hom_scb_norm_prop}
\end{align}
\end{proposition}

\section{Useful inequalities}
\label{sec:Appendix}
%----------------------------------------------------------------------
%%% Appendix - Inequalities
%----------------------------------------------------------------------
% !TEX root = ./HBAConicMain.tex

Consider the cone $\setK_{\text{NN}}$ with the standard log-barrier $h(x)=-\sum_{i=1}^n \ln(x_i)$ which has Hessian $H(x)=\diag[x_{1}^{-2},\ldots,x_{n}^{-2}]=\XX^{-2}$. 
Let $\mathbf{V}=\diag[v_{1},\ldots,v_{n}]=\diag(v)$, $z \in \R^n$, and $x\in\setK_{\text{NN}}$. Then, 
\begin{align}
&\norm{\mathbf{V}z}_{\infty} \leq \norm{\mathbf{V}z} \leq \norm{v}_{x} \cdot \norm{z}_x^* \label{eq:technical_1}\\
& \norm{\mathbf{V}z}_{\infty} \leq \norm{v}_{x} \cdot \norm{\XX z}_{\infty}. \label{eq:technical_2}
\end{align}
The first inequality in \eqref{eq:technical_1} is trivial. Let us prove the second inequality.
Indeed, we have
\begin{align*}
\norm{\mathbf{V}z}^2 &= \sum_{i=1}^n (v_iz_i)^2 = \sum_{i=1}^n (v_i/x_i)^2\cdot (x_iz_i)^2 \leq \left(\sum_{i=1}^n (v_i/x_i)^2 \right)\cdot \left(\sum_{i=1}^n(x_iz_i)^2 \right) \\
& = \inner{H(x)v,v} \cdot \inner{[H(x)]^{-1}z,z} = \norm{v}_{x}^2 \cdot (\norm{z}_x^*)^2,
\end{align*}
which finishes the proof of \eqref{eq:technical_1}. 
For the inequality \eqref{eq:technical_2}, we have, donoting by $v/x$ the componentwise division of $v$ by $x$,
\begin{align*}
\norm{\mathbf{V}z}_{\infty} &= \norm{\mathbf{V}\XX^{-1}\XX z}_{\infty} \leq \norm{v/x}_{\infty} \cdot \norm{\XX z}_{\infty} \leq \norm{v/x} \cdot\norm{\XX z}_{\infty}= \norm{\XX^{-1}v} \cdot\norm{\XX z}_{\infty} = \norm{H(x)^{1/2}v} \cdot\norm{\XX z}_{\infty}\\
& = \norm{v}_x \cdot \norm{\XX z}_{\infty}
%  \sum_{i=1}^n (v_iz_i)^2 = \sum_{i=1}^n (v_i/x_i)^2\cdot (x_iz_i)^2 \leq \left(\sum_{i=1}^n (v_i/x_i)^2 \right)\cdot \left(\sum_{i=1}^n(x_iz_i)^2 \right) \\
%& = \inner{H(x)v,v} \cdot \inner{[H(x)]^{-1}z,z} = \norm{v}_{x}^2 \cdot (\norm{z}_x^*)^2,
\end{align*}

\bibliographystyle{plainnat}
\bibliography{mybib}
\end{document}


