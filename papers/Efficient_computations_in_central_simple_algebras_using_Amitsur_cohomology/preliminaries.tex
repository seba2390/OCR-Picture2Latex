\subsection{Algorithmic preliminaries}\label{Sec:AlgoPrelim}
    We recall here some existing classical and quantum algorithms.
    \subsubsection{Computational presentations of central simple algebras}\label{Sec:Presentations}
        A \(k\)-algebra may be given as input to program in the form of a table of structure constants. Let \(A\) be a finite dimensional \(k\)-algebra of dimension \(n\), let \(V\) be the underlying vector space of \(A\) and let \((e_1,\hdots,e_n)\) be a basis of \(V\). Multiplication in \(A\) is a bilinear map which may be represented as a tensor \(\lambda \in V^\vee \otimes V^\vee \otimes V\). The table of structure constants of \(A\) for the basis \((e_1,\hdots,e_n)\) is then the coordinates of \(\lambda\) in the basis \(\left(e_i^\vee \otimes e_j^\vee \otimes e_k\right)\) of \(V^\vee \otimes V^\vee \otimes V\), where \(\left(e_i^\vee\right)\) is the basis of \(V^\vee\) dual to \((e_i)\).

        An \(F\)-algebra \(A \cong M_d(k)\) may be presented as a cyclic algebra \cite[Section 2.5]{gille2017central} or as a crossed-product algebra \cite[Section 2.6]{jacobson2009finite}. It follows readily from the algebraic constructions that, computationally, knowing a cyclic (resp. crossed-product) presentation of a central simple \(F\)-algebra \(A\) of degree \(d\) is equivalent to knowing an embedding \(F \to A\), where \(F/k\) is a cyclic (resp. Galois) extension of  \(k\) of degree \(d\).
        
    \subsubsection{Zero divisors and explicit isomorphisms}\label{Sec:ZeroDivGivesIsom}
        If \(k\) is a field and \(z \in M_d(k)\), let \(L_z\) be the left ideal \(M_d(k)z\). Then \(\rk z = \frac{\dim_k L_z}{d}\). Furthermore, if \(z\) has rank one, then the action of \(M_d(k)\) by left multiplication on \(L_z\) is isomorphic to the usual action of \(M_d(k)\) on a \(d\)-dimensional \(F\)-vector space.

        It follows from this fact that if one knows a rank one element in \(A \cong M_d(k)\) and one may efficiently perform computations in \(A\), then one may compute an explicit isomorphism \(A \to M_d(k)\) in polynomial time. For completeness, we record this as \cref{Algo:ExplicitIsomorphism}.

        \begin{algorithm}
        \caption{Computing an explicit isomorphism to \(M_d(F)\) from a rank one zero divisor}
        \label{Algo:ExplicitIsomorphism}
        \begin{algorithmic}
            \REQUIRE a field \(F\)
            \REQUIRE an algebra \(A \cong M_d(F)\) with basis \(e_1,\hdots,e_{d^2}\)
            \REQUIRE a rank one zero divisor \(z \in A\)
            \STATE \((f_1,\hdots,f_d) \leftarrow \Basis(Az)\)
            \FOR{\(i=1\) to \(d^2\), \(j = 1\) to \(d\)}
                \STATE \(C_{ij} \leftarrow\) column with coordinates of \(e_if_j\) in basis \((f_1,\hdots,f_d)\)
            \ENDFOR
            \RETURN Linear map whose matrix from Basis \((e_1,\hdots,e_{d^2})\) of \(A\) to the canonical basis of \(M_d(F)\) is \(\begin{pmatrix} C_{1,1} & \hdots &C_{d^2,1} \\ \vdots & \ddots & \vdots \\ C_{d^2,d} & \hdots & C_{d^2,d}\end{pmatrix}\)
        \end{algorithmic}
        \end{algorithm}
\subsubsection{Computing rings of integers and factoring ideals}
    The ability to factor integers in polynomial time allow for polynomial quantum algorithms for several tasks of computational number theory. For instance, Zassenhaus' round 2 algorithm \cite[Algorithm 6.1.8]{cohen2013course} computes the maximal order of a number field in polynomial time, provided an oracle for factoring. Thus, one may deduce a polynomial quantum algorithm for computing maximal orders in number fields.

    Likewise, one may deduce from \cite[Algorithm 2.3.22]{cohen2012advanced} a polynomial quantum algorithm for factoring ideals in a number field.
\subsubsection{Computing discrete logarithms in unit groups}\label{Sec:DiscreteLogUnitGroup}
    Here we briefly discuss a subroutine that is necessary for \Cref{algo:TrivCobound}. Let $K$ be a number field and let $h$ be a unit. Let $g_1,\dots,g_n$ be a set of generators of the unit group of $K$. The algorithmic problem is to write $h$ as a product of these generators. This is used to translate a multiplicative equation over the group of \(S\)-units into a linear equation over \(Z\) once generators for the group have been computed.

    An algorithm for computing discrete logarithms in the unit group of a number field is \cite[Algorithm 5.3.10]{cohen2012advanced}. While the source does not make any claim on the complexity, it is well known that this algorithm runs in polynomial time. The idea of the algorithm is to use the logarithmic embedding into $\mathbb{R}^n$ which reduces a multiplicative problem to an additive one. Dirichlet's Unit Theorem implies that the image of this embedding is a lattice. Now one has to represent the target unit as a linear combination of the lattice vectors. The issue is that since $\mathbb{R}$ is not a computable field, one has to work with real numbers up to some precision. What one can do is write the target unit as a linear combination by solving a system of linear equations and then round the solution up to the nearest integer vector. If this does not lead to a solution one can increase the precision as mentioned in \cite[Algorithm 5.3.10]{cohen2012advanced}. An alternative way would be to view this problem as a bounded distance decoding problem. One is essentially looking for the closest lattice vector to our target vector. Since this should potentially be extremely close one can use Babai's algorithm \cite{babai1986lovasz}. 
    
Besides the above mentioned classical algorithm, there also exists a polynomial-time quantum algorithm for this task as it can be seen as an instance of the hidden subgroup problem. If $h$ has finite order, then $h$ is a root of unity, thus writing it as a power of a primitive root of unity is an easy task. So one may assume that $h$ has infinite order and let $g_1,\dots g_k$ be generators of the torsion-free part of the unit group. Let us denote the unit group by $G$. Then one can define a homomorphism from $\mathbb{Z}^{k+1}$ to $G$ by mapping $(a_1,\dots,a_k,b)$ to $h^{-b}\prod g_i^{a_i}$. 
The kernel of this homomorphism is clearly a $\mathbb{Z}$-lattice (not full rank) in $\mathbb{Z}^{k+1}$. Generators for that can be computed in quantum polynomial time by a recent preprint of Kuperberg \cite[Theorem 4]{kuperberghidden}. Now what is left is to find an element in the kernel where the last coordinate is 1. Indeed, if $(a_1,\dots a_k,1)$ is in the kernel, then $1=h^{-1}\prod g_i^{a_i}$ which means that $h=\prod g_i^{a_i}$. If the kernel is generated by vectors with last coordinates $b_1,\dots b_j$, then finding such an element is just solving the Diophantine equation $\sum x_ib_i=1$ which is a linear Diophantine equation which can be solved with in polynomial time. 

\subsubsection{A polynomial quantum algorithm for computing class groups and groups of \(S\)-units in number fields}\label{subsec:quantSunit}
The class group of $L$, denoted by $Cl(L)$ is the group of fractional ideals modulo principal ideals. The units in $\Z_L$ form a group which is called the unit group and is denoted by $U_L$. All these are important objects in number theory and they motivate the following three algorithmic problems: 
\begin{enumerate}
    \item Compute generators of $Cl(L)$
    \item Compute generators of $U_L$ 
    \item Given a principal ideal $I$, find a generator element of $I$ 
\end{enumerate}

To the best of our knowledge, there only exist polynomial-time classical algorithms for these problems in very special cases. In \cite{biasse2014subexponential} the authors propose a subexponential algorithm for computing class groups and unit groups for arbitrary degree number fields. In \cite{bauch2017short} and \cite{biasse2017computing} the authors propose subexponential algorithms for the principal ideal problem for a certain subclass of number fields (multiquadratic extensions of $\mathbb{Q}$ and power-of-two cyclotomic fields). 

Thus these problems are good examples where quantum algorithms have a clear advantage over their classical counterparts. Hallgren \cite{hallgren2005fast} first proposed a polynomial-time quantum algorithm for computing the class group and unit group for bounded degree number fields. Later polynomial-time algorithms were proposed for these tasks for arbitrary number fields \cite{eisentrager2014quantum},\cite{biasse2016efficient}. Furthermore, \cite{biasse2016efficient} also provides an efficient algorithm for computing generators of principal ideals (dubbed the principal ideal problem) and claims an application is the computation of generators of the group of \(S\)-units.

Let $S$ be a set of prime ideals of $K$. An element $x\in K$ is called an $S$-integer if $v_{\textfrak{p}}\geq 0$ for every $\textfrak{p}\notin S$. An element $x\in K$ is called an $S$-unit if $v_{\textfrak{p}}=0$ for every $\textfrak{p}\notin S$. We denote the ring of $S$-integers by $\Z_{K,S}$ and the group of $S$-units by $U_{K,S}$. An element $x\in L$ is an $S$-integer if $v_{\textfrak{P}}\geq 0$ for all primes $\textfrak{P}$ except perhaps those above $S$. An element $x\in L$ is an $S$-unit if $v_{\textfrak{P}}= 0$ for all primes $\textfrak{P}$ except perhaps those above $S$.

Let $S=\textfrak{P}_1,\dots,\textfrak{P}_k$. We will essentially follow \cite[Algorithm 6.1]{simon2002solving}. First we compute the class group $Cl(K)$, let $g_1,\dots,g_r$ be generators whose orders are $d_1,\dots,d_r$. Note that orders of elements can be computed with Shor's algorithm in quantum polynomial time in any finite  abelian group. We also compute the unit group of $\Z_K$ using the algorithm from \cite{eisentrager2014quantum}. We introduce the following notation. Let $V$ be a vector with $k$ entries from $K$ and let $U$ be a $k\times l$ integer matrix. Then $W=V^U$ is the vector with $l$ entries defined by 
$$W_j=\prod_i V^{U_{i,j}}$$.

First determine $\beta_i\in K$ such that $g_i^{d_i}=\beta_i \Z_K$. Such $\beta_i$ can be found using the polynomial quantum algorithm for the principal ideal problem \cite{biasse2016efficient}. Let $M$ be the diagonal matrix with $d_i$ in the $i$th position and let $V=(\beta_1,\dots,\beta_r)$. Let $\textfrak{P}_j= (\prod_{i,j} g_i^{e_{i,j}})\alpha_j$. Let $M'=(-e_{i,j})$ and $V'=(\alpha_1,\dots,\alpha_k)$.

Now compute a unimodular matrix $U=\begin{pmatrix}
A&B \\
C&D
\end{pmatrix}$ such that 
$$(M|M')U=(0|H)$$ is in Hermite Normal Form. Compute $W$ such that $(W|W')=(V|V')^U$. Then the coordinates of $W$ generate the $S$-unit group (modulo units of $K$). The proof of this fact can be found in \cite[Section 6]{simon2002solving}. 

One application of $S$-unit computation (mentioned in \cite{biasse2016efficient} and \cite{simon2002solving}) is that one can compute solutions to norm equations in Galois extensions. Let $L|K$ be a Galois extension of number field and let $a\in K$. Then one would like to find a solution to  $N_{L|K}(x)=a$ (or show that it does not exist). The main result by Simon is the following: 
\begin{theorem}(\cite[Theorem 4.2]{simon2002solving})
Let $L|K$ be a Galois extension of number fields. Let $S_0$ be a set of primes generating the the relative class group $Cl_i(L|K)$. Let $S$ be a set of primes containing $S_0$ and let us denote the set of $S$-units in $L$ by $U_{L,S}$ and the set of $S$-units in $K$ by $U_{K,S}$. Then one has that: 
$$N_{L|K}(U_{L,S})=N_{L|K}(L^{*})\cap U_{K,S}$$
\end{theorem}
Informally, this means that for Galois extensions one can search for $x$ in the group of $S$-units. Then by factoring $S$ and computing the class group of $L$ one can turn finding $x$ into solving a system of linear equations. Thus computing $S$-units implies a polynomial quantum algorithm for solving norm equations in Galois extensions. We summarize this discussion into the following theorem: 
\begin{theorem}\label{thm:s-unit&norm}
Let $L|K$ be a Galois extension of number fields and let $S$ be a set of primes of $L$. Then there exists a polynomial-time quantum algorithm that computes a set of generators for the group of $S$-units of $L$. Furthermore, let $a\in K$. Then there exists a polynomial-time quantum algorithm for solving the norm equation $N_{L|K}(x)=a$. 
\end{theorem}

\subsection{Brauer factor sets and central simple algebras}\label{Sec:BrauerPrelim}
For the remainder of this section, we fix a field \(k\), a separable polynomial \(P \in k[X]\) of degree \(d\) and an etale algebra \(F = k[X]/P\) over \(k\). We let \(K\) be the smallest Galois extension of \(k\) which splits \(F\), and set \(G = \Gal(K/k)\)

We will introduce definitions that are equivalent to those of Adamson cohomology \cite{adamson1954cohomology} with values in the multiplicative group. However, while Adamson cohomology is normally defined for separable field extensions, we extend our definitions to etale algebras. One difference is then that cochains are indexed by \(k\)-algebra morphisms which are not necessarily injective. We may then use well-known results on Brauer factor sets to establish a cohomological description of the Brauer group (see \cite[Chapter 3]{jacobson2009finite} for a modern reference).

Let \(\Phi\) be the set of \(k\)-algebra morphisms from \(F\) to \(K\).
\begin{lemma}\label{lemma:BrauerAdamson}
     The map \(\varphi \mapsto \varphi(X)\) is a bijection between \(\Phi\) and the set of roots of \(P\) in \(K\). In particular, the set \(\Phi\) has \(d\) elements.
\end{lemma}

\begin{proof}
    If \(\varphi \in \Phi\), it is clear that \(\varphi(X)\) is a root of \(P\) in \(K\). Now, the morphism \(\varphi\) is entirely determined by \(\varphi(X)\) so the map \(\varphi \mapsto \varphi(X)\) is injective.

    Let \(r \in K\) be a root of \(P\). Then, let \(Q\) be the unique irreducible factor of \(P\) such that \(Q(r) = 0\). Then, \(F \simeq k[X]/Q \oplus k[X]/(P/Q)\) and we have a map \(k[X]/Q \oplus k[X]/(P/Q)\) sending \((A(X),B(X))\) to \(A(r)\). This maps precomposed with the isomorphism defined above sends \(X\) to \(r\).
\end{proof}

We may then number the maps from \(F\) to \(K\), so that \(\Phi = \{\varphi_1,\hdots,\varphi_d\}\). If \(i \in [d]\) and \(g \in G\), we write \(gi\) for the index of \(g \varphi_i\).

\begin{definition}
    Let \(n \in \N\). An \(n\)-cochain is a map \[a:\begin{array}{ccl} \Phi^{n+1} &\to &K^\times \\ (\varphi_{i_0},\hdots,\varphi_{i_n}) &\mapsto &a_{i_0,\hdots,i_n}\end{array}\] which satisfies the following homoegeneity condition: for \(i_0,\hdots,i_n \in [d]\), for \(g \in G\),
    \begin{equation} \label{eq:homogeneity}
        ga_{i_0,\hdots,i_n} = a_{gi_0,\hdots,gi_n}
    \end{equation}
    A cochain \(a\) may be denoted as the family \((a_{i_0,\hdots,i_n})_{i_0,\hdots,i_n \in [d]}\) of values that it takes over \(\Phi^{n+1}\).

    A cochain is said to be reduced if \(a_{i,\hdots,i} = 1\) for all \(i \in [n]\).
    By pointwise multiplication, the \(n\)-cochains form a group which we denote \(C^n(k,F)\). Then, the reduced cochains form a subgroup called \(C'^n(k,F)\).

    If \(a \in C^n(k,F)\) and \(i_0,\hdots,i_{n+1} \in [d]\), we set \[a_{i_0,\hdots,\widehat{i_k},\hdots,i_{n+1}} \coloneqq a_{i_0,\hdots,i_{k-1},i_{k+1},\hdots,i_{n+1}}.\]

    For \(n \in \N\), there is a group homomorphism \(\Delta^n \colon C^n(k,F) \to C^{n+1}(k,F)\) defined as follows: if \(a \in C^n(k,F)\),
    \[\Delta^n(a)_{i_0,\hdots,i_{n+1}} = \prod_{k=0}^{n+1} c_{i_0,\hdots,\widehat{i_k},\hdots,i_n}^{(-1)^k}.\]

    We set \(Z^n(k,F) = \Ker(\Delta^n)\) and call elements of this subgroup \(n\)-cocycles. We write \(Z'^n(k,F)\) for the group of reduced cocycles. For \(n\) positive, we also set \(B^n(k,F) = \im(\Delta^{n-1})\) and call its elements \(n\)-coboundaries. We note that \(B^n(k,F)\) is a subgroup of \(Z^n(k,F)\). For completeness, we let \(B^0(k,F)\) be the trivial subgroup of \(Z^0(k,F)\). Finally, we set \(H^n(k,F) = Z^n(k,F)/B^n(k,F)\).

    Two cocycles \(c_1,c_2 \in Z^n(k,F)\) are called associated, denoted by \(c_1 \sim c_2\), if they have the same class in \(H^n(k,F)\). If \(b \in B^{n+1}(k,F)\), a preimage of \(b\) by \(\Delta^n\) is called a trivialisation of \(b\).
\end{definition}

\begin{remark}
    The cochain \((1)_{i,j,k \in [d]}\) is always a cocycle and is a coboundary for \(n \geq 1\). It is called the trivial cocycle.
\end{remark}

\begin{prop}
    For \(n\) positive, every \(n\)-cocycle is associated to a reduced cocycle.
\end{prop}

\begin{proof}
    Let \(c \in Z^n(k,F)\), and consider the cochain \(a \in C^{n-1}(k,F)\) defined by \(a_{i,\hdots,i} = c_{i,\hdots,i}^{-1}\) and \(a_{i_0,\hdots,i_{n-1}} = 1\) if the \(i_k\) are not all equal to one another. Then, the cocycle \(c \Delta^{n-1}(a)\) is reduced and associated to \(c\).
\end{proof}

We may now recall the explicit description of the isomorphism \(H^2(k,F) \simeq Br(F/k)\) (that is, the group of Brauer classes of central simple \(k\)-algebras that are split by \(F\)), as exposed in \cite[Chapter 2]{jacobson2009finite}. Using Lemma \ref{lemma:BrauerAdamson}, we observe that what is called a Brauer factor set there is a \(2\)-cocycle in our notation.

\begin{definition}\label{def:BrauerCSA}
    Let \(c \in Z'^2(k,F)\). Let \(B(F,c)\) be the \(k\)-vector space of \(G\)-homogeneous maps from \(\Phi^2\) to \(K\). We turn \(B(F,c)\) into a \(k\)-algebra using the following definition. If \(\ell,\ell' \in B(F,c)\), we set \(\ell \ell' = \ell''\) such that for all \(i,j \in [d]\),
    \[\ell''_{i,j} = \sum_{k=1}^d \ell_{i,k} c_{i,k,j} \ell'_{k,j}.\]
\end{definition}

\begin{theorem}\label{thm:BrauerFSAssociated}
    If \(c\) is a reduced \(2\)-cocycle, the \(k\)-algebra \(B(F,c)\) is a central simple \(k\)-algebra. Furthermore, if \(c_1\) and \(c_2\) are associated reduced cocycles, and \(a\) is a reduced \(1\)-cochain such that \(c_2 = \Delta^1(a) c_1\), then the map
    \[\begin{array}{ccl} B(F,c_1) &\to &B(F,c_2) \\ (m_{i,j})_{i,j \in [d]} &\mapsto &(m_{i,j}a_{i,j}^{-1})_{i,j \in [d]}\end{array}\]
    is an isomorphism. Conversely, if \(A\) is a central simple \(k\)-algebra split by \(F\), then there exists a reduced cocycle \(c \in Z^2(k,F)\) such that \(A \simeq B(k,F)\) If \(B(F,c_1)\) and \(B(F,c_2)\) are isomorphic algebras, then the cocycles \(c_1\) and \(c_2\) are associated.
\end{theorem}

\begin{remark}\label{rem:ReducedAssociated}
    We note that if both \(c_1\) and \(c_2\) are reduced, then \(m\) must be reduced as well.
\end{remark}

\begin{example}\label{ex:ConstantFS}
    Since elements of \(\Phi\) are numbered from \(1\) to \(d\), a map from \(\Phi^2\) to \(K\) can naturally be seen as a matrix in \(M_d(K)\).

    If \(\1\) is the constant factor set whose values are equal to \(1\), then \(B(F,\1)\) is the subalgebra of \(G\)-homogeneous matrices in \(M_d(K)\). In particular, since \(B(F,\1)\) contains the rank \(1\) matrix with every coefficient equal to \(1\), we get \(B(F,\1) \simeq M_d(F)\) and this isomorphism may be computed explicitly using the methods from \cref{Sec:ZeroDivGivesIsom}.
\end{example}

We refer the reader to \cite[Section 2.3]{jacobson2009finite} for the detailed construction of a Brauer factor-set associated to a given central simple algebra, and we give a brief description below:
Let \(A\) be a central simple \(k\)-algebra, and let \(F\) be a separable maximal commutative subalgebra of \(A\). Then, there exists \(v \in A\) such that \(A = FvF\). Furthermore, letting \(K\) be a Galois extension of \(k\) which splits \(F\), there is a map \(A \to M_d(K)\) such that every \(u \in F\) is mapped to \(\Diag(\phi_1(x),\hdots,\phi_d(x))\). Let \(v_{ij}\) be the image of \(v\) in \(M_d(K)\). Then, every element \(uvu'\) of \(A\) becomes a matrix of the form \((\ell_{ij} v_{ij}\), where \(\ell_{ij} = \phi_i(u)\phi_j(u')\). The map \(uvu' \mapsto (\ell_{ij}\) gives an isomorphism from \(A\) to \(B(F,c)\), where \(c_{ijk} \coloneqq v_{ik}v_{kj}v_{ij}^{-1}\).