The \emph{explicit isomorphism problem} is the algorithmic problem of, given some algebra \(A\) isomorphic to \(M_d(k)\), constructing an explicit isomorphism \(\varphi\colon A \to M_d(k)\). The explicit isomorphism problem may be thought of as a natural problem in computational representation theory. Given an \(k\)-algebra \(A\), one may wish to assay it. That is, compute the Jacobson radical of \(A\), and the decomposition of the semi-simple part of \(A\) as a sum of simple \(F\)-algebras, themselves identified to some \(M_d(D)\), for \(D\) a division \(k\)-algebra. In general, the hard part of this task is to find an isomorphism \(A \to M_n(D)\) when \(A\) is simple. A general recipe for solving this problem is to identify the Brauer class of \(D\) over its center \(K/F\), find structure constants for \(M_n(D^{op})\) and then compute an explicit isomorphism \(A \otimes M_n(D^{op}) \simeq M_m(K)\) \cite{ivanyos2012splitting,gomez2022primitive,csahok2022explicit}.

Applications of the explicit isomorphism problem go beyond the mere computational theory of associative algebras. In arithmetic geometry, the problem is relevant for trivialising obstruction algebras in explicit descent over elliptic curves \cite{cremona2008explicit,cremona2009explicit,cremona2015explicit,fisher2013explicit} and computation of Cassel-Tate pairings \cite{fisher2014computing,yan2021computing}. The problem is also connected to the parametrisation of Severi-Brauer surfaces \cite{de2006lie}. Recent work in algebraic complexity theory reduced the determinant equivalence test to the explicit isomorphism problem \cite{garg2019determinant}. Finally, the explicit isomorphism problem over the function field \(\F_q(T)\) is also relevant to error correcting codes \cite{gomez2022primitive}.

It is well known that the Brauer group of equivalence classes of central simple \(k\)-algebras may be described as the second Galois cohomology group \(H^2(k,k_{sep}^\times)\). The remarkable fact that these two descriptions coincide may be seen as a consequence of the crossed-product presentation of central simple algebras containing a maximal commutative subalgebra which is a Galois extension of the base field.

When a cohomological representation is known for a certain algebra, the explicit isomorphism problem translates into a multiplicative algebraic equation. Indeed, the algebra is represented by a cocycle, and since the algebra is split, the cocycle is a coboundary. Then, an explicit isomorphism to a matrix algebra may be computed using a cochain whose differential is this coboundary. When the base field is a number field and a presentation of the algebra (either a cyclic algebra or a crossed product representation) is known, the resulting equation may be solved by computing a certain group of \(S\)-units \cite{simon2002solving,fieker2009minimizing}. 

The crossed-product presentation, introduced by Emmy Noether, is not the only existing algebraic presentation of a central simple algebra. Earlier, Dickson had introduced a presentation as a so-called cyclic algebra for algebras containing a maximal commutative subalgebra which is a cyclic Galois extension of the base field. Furthermore, Brauer had also introduced a factor-set presentation which allowed to describe an algebra without the knowledge of a maximal commutative subalgebra which is a Galois extension. We refer the reader to \cite{jacobson2009finite} for a modern description of Brauer's and Noether's presentations. It is interesting to note that, as Galois cohomology encompasses Noether's factor sets, Adamson developed a cohomology theory for non-Galois field extensions which describes Brauer factor sets as cocycles \cite{adamson1954cohomology}.

A line of work introduced Azumay algebras, which generalise the notion of central simple algebra (and therefore of the Brauer group) over arbitrary rings \cite{azumaya1951maximally,auslander1960brauer}, and then later over schemes \cite{grothendieck1968groupe,grothendieck1968groupe2,grothendieck1966groupe}. While trying to connect the usual Galois cohomological description of the Brauer group and another description introduced by Hochschild, Amitsur came upon an exact complex which yields a cohomology theory suitable to classify central simple algebras. As the complex he defined may be constructed over arbitrary rings, a line of work generalised his classification result to Azumaya algebras over increasingly general classes of ring extensions \cite{amitsur1959simple,rosenberg1960amitsur,chase1965amitsur}.

Such presentations present the potential of being used both for a computational representation of a central simple algebra and for approaching the explicit isomorphism problem. In fact, many computer algebra systems manipulate cyclic algebras using Dickson's presentation. While the constructions relying on Amitsur cohomology are abstract and do not lend themselves easily to a practical implementation, the crossed product and cyclic presentations have the inconvenient that one is required to know a Galois, or even cyclic, maximal subfield in order to convert a given algebra into these presentations. While such a subfield always exists for algebras over global fields (as per the Brauer-Hasse-Noether theorem), there is no known efficient algorithm to find one (except for quaternion and degree 3 central simple algebras). The presentation using Brauer factor sets may be constructed using an arbitrary maximal subfield, which is easy to find, but factor sets take values in the normal closure of the subfield used. Results in arithmetic statistics \cite{eberhard2022characteristic} suggest that a random element of such an algebra will generate a maximal subfield whose Galois group is the full symmetric group, and hence will have a Galois closure of degree \(n!\), preventing it to be used efficiently in a computation.

\subsection{Our contributions}
In \Cref{Sec:Amitsur}, we give a presentation for central simple algebras over a field using Amitsur cocycles. Our presentation only requires knowledge of a separable maximal commutative subalgebra, which may be found efficiently in deterministic polynomial time \cite{graaf2000finding}. Furthermore, multiplication in our representation may be computed using \(O(n^3)\) base field multiplications, making it as efficient as the naive representation using structure constants. In \Cref{Sec:AmitsurAlgebra} we construct a central simple algebra from an Amitsur \(2\)-cocycle. In \Cref{Sec:CompRepAmi}, we discuss the computational representation of Amitsur cocycles and related algebras. Finally, in \Cref{Sec:AmitsurAlgebraConstruction} we give an efficient algorithm for computing a cocycle associated to a given central simple algebra.

In \Cref{Sec:Trivialisation}, we prove that over a number field, an Amitsur \(1\)-cochain whose differential is a given \(2\)-coboundary may be computed using a certain group of \(S\)-units. This gives an algorithm for solving the explicit isomorphism problem over number fields, which is polynomial aside from some well understood number algebraic tasks, such as factorisation, class group computation and \(S\)-unit group computation. Each of theses tasks may be solved by a polynomial quantum algorithm, and our work therefore provides the first polynomial quantum algorithm for the explicit isomorphism problem over number fields.

\subsection{Related works}
Below, we review existing algorithms for solving the explicit isomorphism problem over various fields. 

In the case of a finite base field, a polynomial-time algorithm was introduced by Ronyai in \cite{ronyai1990computing}.

Instances of the problem for \(\Q\)-algebras isomorphic to \(M_n(\Q)\) were first treated separately for small values of \(n\). When \(n=2\), the problem reduces to finding a rational point on a projective conic \cite[Theorem 5.5.4]{voight2021quaternion}, which is solved for instance in \cite{cremona2003efficient}. Then, \cite{de2006lie} presented an subexponential algorithm when \(n=3\) by finding a cyclic presentation and solving a cubic norm equation. The case \(n=4\) is tackled in \cite{pilnikova2007trivializing} by reducing to the case of quaternion algebras over \(Q\) and \(\Q(\sqrt{d})\) and solving a norm equation.

In \cite{cremona2015explicit} an algorithm was given and studied mostly for the cases \(n=3\) and \(n=5\). It was then generalised in \cite{ivanyos2012splitting,ivanyos2013improved} to a \(K\)-algebra isomorphic to \(M_n(K)\), where \(n\) is a natural number and \(K\) is a number field. The complexity of this last algorithm is polynomial in the size of the structure constants of the input algebra, but depends exponentially on \(n\), the degree of \(K\) and the size of the discriminant of \(K\).

In 2018, \cite{ivanyos2018computing} exhibited a polynomial-time algorithm for algebras isomorphic to \(M_n(\F_q(T))\).

For the case of fixed \(n\) and varying base field, \cite{fisher2017higher,kutas2019splitting} independently gave an algorithm for an algebra isomorphic to \(M_2(\Q(\sqrt{d}))\) with complexity polynomial in \(\log(d)\). A similar algorithm for quadratic extensions of \(\F_q(T)\) was given in \cite{ivanyos2019explicit} for the case of odd \(q\). The case of even \(q\) was then treated in \cite{kutas2022finding}.