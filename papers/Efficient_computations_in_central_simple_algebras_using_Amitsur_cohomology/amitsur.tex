In this section, we recall the basic definitions of Amitsur cohomology and then use them to get efficient representations of cocycles as introduced in \Cref{Sec:BrauerPrelim}. We also keep notations as they were in \Cref{Sec:BrauerPrelim}.

Although the Amitsur complex and its cohomology are defined for general commutative rings, we will only need to state the definitions over fields and étale algebras. We refer to \cite[Chapter 5]{ford2017separable} for more general definitions and results.

Unless specified otherwise, all tensor products are taken over \(k\), and by \(F^{\otimes n}\) we mean the tensor product of \(n\) copies of the \(k\)-algebra \(F\).

\begin{definition}
    A \(n\)-cochain in the sense of Amitsur is an invertible element in the \(k\)-algebra \(F^{\otimes n+1}\), and we write \(C_{Am}^n(k,F)\) for the group of \(n\)-cochains. 

    For \(n \in \N\) and \(0 \leq i \leq n\), we define a \(k\)-algebra homomorphism \(\varepsilon_i^n\) from \(F^{\otimes n+1}\) to \(F^{\otimes n+2}\). The map \(\varepsilon_i^n\) is defined on the simple tensors as follows:
    \[\varepsilon_i^n(f_0 \otimes \hdots \otimes f_n) = f_0 \otimes \hdots \otimes f_{i-1} \otimes 1 \otimes f_{i} \otimes \hdots \otimes f_n.\]
    We then define the group homomorphism
    \[\Delta^n\colon\begin{array}{ccl} C_{Am}^n(k,F) &\to &C_{Am}^{n+1}(k,F) \\ a &\mapsto &\prod_{i=0}^{n+1} \varepsilon_i^n(a)^{(-1)^i}\end{array}.\]

    Define the subgroup \(Z_{Am}^n(k,F) = \Ker(\Delta_{Am}^n)\) of \(C_{Am}^n(k,F)\), and its elements are called \(n\)-cocycles in the sense of Amitsur.

    If \(n\) is positive, we also define the subgroup \(B_{Am}^n(k,F) = \im(\Delta_{Am}^{n-1})\) of \(Z_{Am}^n(k,F)\) and its elements are \(n\)-coboundaries in the sense of Amitsur. Then, the \(n\)th Amitsur cohomology group \(H_{Am}^n(k,F)\) is defined as the quotient group \(Z_{Am}^n(k,F)/B_{Am}^n(k,F)\).

    Two \(n\)-cocycles \(c_1\) and \(c_2\) are called associated if they have the same class in \(H_{Am}^n(k,F)\).
\end{definition}

\subsection{Amitsur cohomology and central simple algebras}\label{Sec:AmitsurAlgebra}

    The first result we need to leverage Amitsur cohomology in representing Brauer factor sets is the following
    \begin{lemma}\label{lemma:AdamsonAmitsur}
        Consider the map \(\Psi_n\) from \(F^{\otimes n+1}\) to the \(k\)-algebra of \(G\)-homogeneous maps from \(\Phi^{n+1}\) to \(K\), which is defined over simple tensors by
        \[f_0 \otimes \hdots \otimes f_n \mapsto \left(\begin{array}{ccl} \Phi^{n+1} &\to &K^\times \\ (\varphi_{i_0},\hdots,\varphi_{i_n}) &\mapsto &\prod_{j=0}^n \varphi_{i_j}(f_j)\end{array}\right).\]
        The map \(\Psi\) is an isomorphism of \(k\)-algebras, which also sends the unit group \(C_{Am}^n(k,F)\) onto the unit group \(C^n(k,F)\). Furthermore, \(\Psi_{n+1} \circ \Delta_{Am}^n = \Delta^n \circ \Psi_{n}\), and it follows that cocycle and coboundaries subgroup are also isomorphic to one another, and so are cohomology groups.
    \end{lemma}

    \begin{proof}
        If \(F\) is a separable field extension of \(k\), this is the content of \cite[Section 2]{rosenberg1960amitsur}. However, the proof given there readily generalises to the case that \(F\) is an etale \(k\)-algebra.
    \end{proof}

    By the universal property of the tensor products, the \(k\)-algebra maps from \(F^{\otimes n+1}\) to \(K\) are in natural bijection with \(\Phi^{n+1}\). More precisely, if \((\varphi_{i_0},\hdots,\varphi_{i_n})\) and \(f_0,\hdots,f_n \in F^{\otimes n+1}\) is a simple tensor element, then
    \[(\varphi_{i_0},\hdots,\varphi_{i_n})(f_0 \otimes \hdots \otimes f_n) = \prod_{j=0}^n \varphi_{i_j}(f_j).\]

    By the construction of \(\Psi\), it is clear that if \(a \in F^{\otimes n+1}\) and \((\varphi_{i_0},\hdots,\varphi_{i_n})\), then
    \begin{equation}\label{eq:AmitsurEmbeddings}
        \Psi_n(c)_{i_0,\hdots,i_n} = (\varphi_{i_0},\hdots,\varphi_{i_n})(c).
    \end{equation}

    \begin{definition}
        A reduced \(n\)-cocycle in the sense of Amitsur is a cocycle \(c \in Z_{Am}^n(k,F)\) such that \((\varphi_i,\hdots,\varphi_i)(c) = 1\) for all \(i \in [d]\). Equivalently, it is a cocycle \(c\) such that \(\Psi_n(c)\) is reduced.
    \end{definition}

    Adapting \Cref{def:BrauerCSA} to Amitsur cohomology, we get:
    \begin{definition}\label{def:AmitsurCSA}
        Let \(c \in Z_{Am}^2(k,F)\) be a reduced cocycle. Then the \(k\)-algebra \(A(F,c)\) is defined as follows:

        As a \(k\)-vector space, \(A(F,c)\) is \(F^{\otimes 2}\). We see \(F^{\otimes 3}\) as a \(F^{\otimes 2}\)-algebra via the embedding \(\varepsilon_1^2\colon F^{\otimes 2} \to F^{\otimes 3}\). Then, multiplication in \(A(F,c)\) is defined by
        \[a a' = \Tr_{F^{\otimes 3}/F^{\otimes 2}}\left(\varepsilon_2^2(a)c\varepsilon_0^2(a')\right).\]
        In the sequel, we use the same embedding whenever we write \(\Tr_{F^{\otimes 3}/F^{\otimes 2}}\).
    \end{definition}

    \begin{prop}
        Let \(c \in Z_{Am}^2(k,F)\). The map \(\Psi_1\) is a \(k\)-algebra isomorphism from \(A(F,c)\) to \(B(F,\Psi_2(c)\).
    \end{prop}

    \begin{proof}
        The map \(\Psi_1\) is already known to be an isomorphism of vector space between \(A(F,c)\) and \(B(F,\Psi_2(c))\), as the underlying vector spaces of these algebras are respectively \(F^{\otimes 2}\) and the space of \(G\)-homogeneous maps from \(\Phi^2\) to \(K\). We therefore only need to check that the map \(\Psi_1\) is a homomorphism of \(k\)-algebras.
        
        By definition of the trace as the sum of the conjugates of an element, if \(a \in F^{\otimes 3}\) and \(i,j \in [d]\),
        \[(\varphi_i,\varphi_j)\left(\Tr_{F^{\otimes 3}/F^{\otimes 2}}(a)\right) = \sum_{k \in [d]} (\varphi_i,\varphi_k,\varphi_j)(a).\]

        Using \Cref{eq:AmitsurEmbeddings}, we get that if \(a,a' \in A(F,c)\) and \(i,j \in [d]\),
        \begin{align*}
            \Psi_1(aa')_{i,j} &= (\varphi_i,\varphi_j)\left( \Tr_{F^{\otimes 3}/F^{\otimes 2}}\left(\varepsilon_2^1(a)c\varepsilon_0^1(a')\right)\right)\\
            &= \sum_{k \in [d]}(\varphi_i,\varphi_k,\varphi_j)\left(\varepsilon_2^1(a)c\varepsilon_0^1(a')\right) \\
            &= \sum_{k \in [d]}(\varphi_i,\varphi_k,\varphi_j)(\varepsilon_2^1(a))(\varphi_i,\varphi_k,\varphi_j)(c)(\varphi_i,\varphi_k,\varphi_j)(\varepsilon_0^1(a')) \\
            &= \sum_{k \in [d]} \Psi_1(a)_{i,k} \Psi_2(c)_{i,k,j} \Psi_1(a')_{k,j} \\
            &= \Psi_1(a) \Psi_1(a')
        \end{align*}
    \end{proof}
    
    \begin{theorem}\label{thm:IsomAmiAlgebra}
        Let \(A\) be a central simple \(k\) algebra, let \(F \subset A\) be a separable maximal commutative subalgebra, and let \(v \in A\) be such that \(A = FvF\). Then, there is a cocycle \(c \in Z^2_{Am}(k,F)\) such that the map
        \[\begin{array}{rlcl}
            \Phi_c\colon &A(F,c) &\to &A \\
            &r_1 \otimes r_2 &\mapsto &r_1 v r_2
        \end{array}\]
        is an isomorphism. Furthermore, \(c\) is the unique element of \(F^{\otimes 3}\) with this property (if we extend the definition of the multiplication in \(A(F,c)\) to non-cocycle elements).
    \end{theorem}

    \begin{proof}
        The cocycle \(c\) is constructed as a preimage through \(\Psi_2\) of the Brauer factor-set obtained from \(A = FvF\) as in the discussion at the end of \(\Cref{Sec:BrauerPrelim}\). Then, the isomorphism given in the theorem statement is simply the isomorphism \(B(K,\Psi_2(c)) \simeq A\) pulled back through \(\Psi_1\).
        
        The cocycle \(c\) is a solution to the equation
        \[\Tr_{F^{\otimes 3}/F^{\otimes 2}} \left(\varepsilon_2^1(\Phi_c^{-1}(a))c\varepsilon_0^1(\Phi_c^{-1}(a'))\right) = \Phi_c^{-1}(aa')\]
        for all \(a,a' \in A)\). Observe that \(\varepsilon_2^1(\Phi_c^{-1}(a))\varepsilon_0^1(\Phi_c^{-1}(a'))\) covers \(F^{\otimes 3}\) as \(a,a'\) range over \(A\), and then the uniqueness of \(c\) follows from the non-degeneracy of the trace for étale algebras.
    \end{proof}

    \begin{cor}\label{cor:AssoCocyIsomAmitsur}
        Let \(c,c' \in Z_{Am}^2(k,F)\) be reduced associated cocycles, and let \(a \in C_{am}^2(k,F)\) be such that \(c' = c\Delta_{Am}^1(a)\). Then, the map \(m \mapsto ma^{-1}\), where the multiplication used is the natural one in \(F^{\otimes 2}\), is an isomorphism from \(A(F,c)\) to \(A(F,c')\).
    \end{cor}

    \begin{proof}
        This is just a consequence of the analogous property for Brauer factor sets (\Cref{thm:BrauerFSAssociated}), pulled through the isomorphisms \(\Psi_i\).
    \end{proof}

    \begin{remark}
        For the sake of efficiency, we use already proven results on Brauer factor-sets and the isomorphisms \(\Psi_1\) and \(\Psi_2\) to describe the isomorphism between the groups \(H^2_{Am}(k,F)\) and \(\Br(F/k)\). However, the simplicity of the isomorphism \(\Phi_c\) described in \Cref{thm:IsomAmiAlgebra} suggests that Amitsur cohomology is a natural setting for giving an algebraic description of \(\Br(F/k)\) and that direct proofs of these facts wy be given without using references to Brauer factor-sets.
    \end{remark}

    \begin{example}\label{ex:AmiTrivialCocycle}
        Just as in \Cref{ex:ConstantFS}, the cocycle \(1 \in Z_{Am}^2(k,F)\) corresponds to a split algebra. That is, \(A(F,1) \simeq M_d(k)\). Indeed, we observe that \(\varepsilon_1^0(F)\) is a left ideal of dimension \(d\). Since \(F\) has dimension \(d\), we only need to prove that \(\varepsilon_1^0(F)\) is a left ideal. This is a consequence of the fact that \(\varepsilon_0^1(\varepsilon_0^0(F)) \subset \varepsilon_2^1(F^\otimes 2)\) (as is easily observed on simple tensor elements). Then, the multiplication formula ensures that the product \(x\varepsilon_1^0(y)\) for any \(x \in A(F,1)\) and \(y \in F\) lies in \(T_{F^{\otimes 3}/F^{\otimes 2}}(\varepsilon_2^1(F^{\otimes 2})) \subset \varepsilon_1^0(F)\).
    \end{example}

\subsection{Computational representation of Amitsur cohomology}\label{Sec:CompRepAmi}
    Once a field \(k\) and an étale \(k\)-algebra \(F = k[X]/P\) are fixed, we have isomorphisms \[R_n \coloneqq k[X_0,\hdots,X_n]/(P(X_0),\hdots,P(X_n)) \simeq F^{\otimes n+1},\] for all integers \(n \geq 1\). Therefore, an element of \(F^{\otimes n+1}\), and in particular of \(C_{Am}^n(k,F)\), may be represented uniquely as a polynomial \(R(X_0,\hdots,X_n)\) in \(k[X_0,\hdots,X_n]\) whose individual degrees respectively in \(X_0,X_1,\hdots,X_n\) are bounded by \(d-1\). 

    In this setting, the map \(\varepsilon_i^n\) simply translates as the map sending \(Q(X_0,\hdots,X_n)\) to \(Q(X_0,\hdots,X_{i-1},X_{i+1},\hdots,X_n)\). That is, 
    \[\varepsilon_i^n(X_j) = \begin{cases}
        X_j \text{ if } j < i \\
        X_{j+1} \text{ otherwise.}
    \end{cases}
    \]

    Now, the trace of \(R_2\) as an \(R_1\) algebra (via the morphism \(\varepsilon_1^1\)) may easily be computed in the \(R_1\)-basis \((X_1^i)_{0 \leq i < d}\) of \(R_2\). It follows that if \(Q_1,Q_2 \in R_1\) represent elements \(a_1,a_2\) of \(A(F,c)\), where we see \(c\) as an element of \(R_2\), then the element \(Q \in R_2\) representing the product \(a_1a_2\) may be computed practically as:
    \[Q(X_0,X_1) = \Tr_{R_2/R_1}\left(Q_1(X_0,X_1)c(X_0,X_1,X_2)Q_2(X_1,X_2)\right).\]

    
    Recall that \(\varphi_i\) is a map from \(F = k[X]/P\) to \(K\). The map \((\varphi_i,\varphi_i,\varphi_i)\colon F^{\otimes 3} \to K\) factors through \(F\) via
    \[\begin{array}{ccccl} F^{\otimes 3} &\to &F &\to &K \\
        Q(X_0,X_1,X_2) &\mapsto &Q(X,X,X) &\mapsto &Q(\varphi_i(X),\varphi_i(X),\varphi_i(X)) \end{array}\]

    We may describe the explicit computation of a reduced cocycle associated to a given cocycle:
    \begin{theorem}\label{thm:ReducingCocycleExplicit}
        Let \(c \in Z_{Am}^2(k,F)\) be represented by a polynomial \(Q\) as described above. Let \(c' \in F^{\otimes 3}\) be represented by
        \[Q' = \frac{Q(X_0,X_1,X_2)}{Q(X_1,X_1,X_1)}.\]
        Then, \(c'\) is a reduced cocycle associated to \(c\), and the map
        \[\begin{array}{rlcl}
        \rho\colon &A(F,c) &\to &A(F,c') \\
        &R(X_0,X_1) &\mapsto &Q(X_0,X_0,X_0) R(X_0,X_1)
        \end{array}\]
        is an isomorphism of \(k\)-algebras.
    \end{theorem}

    \begin{proof}
        As \(c\) is invertible in \(F^{\otimes 3}\), \(Q(X_0,X_0,X_0)\) is invertible in \(F[X_0]/(P(X_0))\), so the map given is an isomorphism of vector spaces, and we only need to check that it is compatible with multiplication.

        We let \(R(X_0,X_1),R'(X_0,X_1) \in k[X_0,X_1]/(P(X_0),P(X_1))\) and we compute:
        \begin{align*}
            \rho(R) \rho(R') &= \Tr_{F^{\otimes 3}/F^{\otimes 2}}(\varepsilon_2^1(Q(X_0,X_0,X_0)R)Q(X_1,X_1,X_1)^{-1}c\\
            &\varepsilon_0^1(Q(X_0,X_0,X_0)R')) \\
            &= \Tr_{F^{\otimes 3}/F^{\otimes 2}}\left(Q(X_0,X_0,X_0)R(X_0,X_1)Q(X_1,X_1,X_1)^{-1}c\right.\\
            &\left.Q(X_1,X_1,X_1)R'(X_1,X_2)\right)\\
            &= Q(X_0,X_0,X_0) Tr_{F^{\otimes 3}/F^{\otimes 2}}\left(R(X_0,X_1)cR'(X_1,X_2)\right)\\
            &= Q(X_0,X_0,X_0) RR' \\
            &= \rho(RR')
        \end{align*}
    \end{proof}
\subsection{Computing a cocycle representing a given algebra}\label{Sec:AmitsurAlgebraConstruction}

    We may now give an algorithm for finding a representation of a central simple algebra \(A\) over any field of sufficiently large size where linear algebra tasks may be performed efficiently.

    \begin{algorithm}
        \caption{Computing a \(2\)-cocycle representing a given central simple algebra}
        \label{algo:FindCocycle}
        \begin{algorithmic}[1]
            \REQUIRE A field \(k\)
            \REQUIRE A central simple \(k\)-algebra \(A\) such that \(|k| > \dim_k A\).
            \STATE Find \(u \in A\) such that \(F \coloneqq k[u]\) is a maximal separable commutative subalgebra of \(A\) \label{algoline:Findu}
            \STATE Compute \(P\), the minimal polynomial of \(u\)
            \STATE Find \(v \in A\) such that \(A = FvF\) \label{algoline:Findv}
            \STATE Compute the matrix of the isomorphism \(e: F^{\otimes 2} \to A\) sending \(f_1 \otimes f_2\) to \(f_1vf_2\)
            \STATE Using linear algebra, find \(c \in F^{\otimes 3}\) such that for all \(a,b \in F^{\otimes 2}\), \(e(a)a(b) = \Tr_{F^{\otimes 3}/F^{\otimes 2}}(\varepsilon_2^1(a)c\varepsilon_0^1(b))\). \label{algoline:Findc}
            \STATE Compute \(c' = c/c(X_1,X_1,X_1)\)
            \STATE Compute \(e'\) as \(e\) precomposed with \(R \mapsto c(X_0,X_0,X_0)^{-1} R\). 
            \RETURN \((u,c',e')\)
        \end{algorithmic}
    \end{algorithm}

    Before we prove the correctness efficiency of \cref{algo:FindCocycle}, we need a lemma:

    \begin{lemma}\label{lemma:Findv}
        Let \(k\) be a field and let \(A\) be a central simple \(k\)-algebra. Assume that \(|k| > \dim_k A\). Let \(u \in A\) be such that \(F \coloneqq k[u]\) is a maximal commutative subalgebra of \(A\). Then an element \(v \in A\) such that \(A = FvF\) may be found in probabilistic polynomial time.
    \end{lemma}

    \begin{proof}
        For \(v\) in \(A\), by an argument of dimensions over \(k\), we observe that \(A = FvF\) if and only if the map
        \[e\colon\begin{array}{ccl} F \otimes F &\to &A \\ f_1 \otimes f_2 &\mapsto &f_1vf_2 \end{array}\]
        is injective.

        Fixing the basis \((u^i \otimes u^j)_{0 \leq i,j \leq \deg A - 1}\) and any basis \(B = (b_1,\hdots,b_{\dim A})\) of \(A\), the determinant of \(e\) is a homogeneous polynomial on the coordinates of \(v\) with respect to \(B\), and \(A = FvF\) if and only if \(v\) is not a zero of this polynomial.

        Letting \(S\) be a finite subset of \(k\), the Schwartz-Zippel lemma ensures that a random \(v\) in \(Sb_1 \oplus \hdots \oplus Sb_{\dim A}\) satisfies this condition with probability larger than \(1 - \frac{\dim A}{|S|}\).

        Therefore, if \(|k| > \dim A\), we may pick \(S\) large enough that \(v\) has the desired property with positive probability. For instance, take \(|S| = 2\dim A\) and \(v\) has the desired property with probability larger than \(\frac{1}{2}\).
    \end{proof}
    
    \begin{theorem}\label{thm:AlgoFindCocycle}
        If \(k\) is a field over which linear algebra may be performed efficiently, and \(A\) is a central simple \(k\)-algebra such that \(|k| > \dim_k A\), then \Cref{algo:FindCocycle} returns \(u \in A\), a cocycle \(c \in Z_{Am}^2(k,k(u))\) and an isomorphism \(e\colon A(F,c) \to A\) in probabilistic polynomial time.
    \end{theorem}

    \begin{proof}
         We first prove the correctness of the algorithm. The first 3 steps compute \(u,v \in A\) as discussed above \Cref{thm:IsomAmiAlgebra}. We obtain a subalgebra \(F\) of \(A\) and \(v \in A\) such that \(A = FvF\). Then, by \Cref{thm:IsomAmiAlgebra}, there exists a cocycle \(c \in Z_{Am}^2(k,F)\) such that \(a \otimes b \mapsto avb\) is an isomorphism from \(A(k,F)\) to \(A\), and such a \(c\) is unique in \(F^{\otimes 3}\). The equation solved in \Cref{algoline:Findc} finds this \(c\).

        Finally, the last two lines perform the operation described in \Cref{thm:ReducingCocycleExplicit} for computing a reduced cocycle associated to \(c\), and the isomorphism from \(A(F,c)\) to \(A(F,c')\). Putting everything together, \cref{algo:FindCocycle} \(c\) is indeed a reduced \(2\)-Amitsur cocycle for \(F = k(u)\) and \(e\) gives an isomorphism \(A(F,c) \simeq A\).

        Now we consider the complexity of the algorithm:
        
        The element \(u \in A\) in \Cref{algoline:Findu} may be found using the polynomial algorithm given in \cite{graaf2000finding}.

        Then, \(v\) in \Cref{algoline:Findv} may be found in probabilistic polynomial time using the algorithm of \Cref{lemma:Findv} below.

        Then, the remaining lines involve arithmetic in \(A\) and bounded tensor powers of \(F\), as well as the computation of the solution of a linear system.
        
        All in all, this makes \Cref{algo:FindCocycle} a polynomial probabilistic algorithm.
    \end{proof}
