In this section, we present an algorithm for computing the trivialisation of a coboundary using \(S\)-units group computation. This result is inspired by results such as Simon's algorithm for solving norm equation in cyclic extensions \cite{simon2002solving} and Fieker's result on finding trivialisation of Galois coboundaries in groups of \(S\)-units \cite[Theorem 7]{fieker2009minimizing}.

Our strategy is to prove a vanishing lemma for the first Amitsur cohomology group with coefficients in the divisor group. Such a result is analogous to \cite[Lemma 7]{fieker2009minimizing} and allows us to adapt the proof strategy to our setting.

Let \(k\) be a number field. We let the divisor group \(\Dc(k)\) be the free Abelian group on the set \(\Pl(k)\) of finite places of \(k\). The group \(\Pc(k)\) of principal divisors is the image of the natural injection of \(k\) into \(\Dc(k)\), and it is well known that the quotient \(\Cl_k = \Dc(k)/\Pc(k)\) is finite.

We extend these definition to étale algebras over number fields:
\begin{definition}\label{def:EtaleDivisors}
    Let \(A\) be an étale \(k\)-algebra, for a number field \(k\). Then \(A\) splits as a direct sum
    \[A \simeq \bigoplus_{i \in I} K_i\]
    of finite separable extensions of \(k\). Furthermore, this decomposition is unique up to reordering and internal isomorphism of the fields. We define \(\Dc(F)\) as the free Abelian group over \(\Pl(F)\), the disjoint union of the sets \(\Pl(K_i)\).
\end{definition}

    While \(\Dc\) is not a functor, we may define \(\Dc(f)\) whenever \(f\) is a monomorphism in the category of étale \(k\)-algebras. Indeed, if \(f\colon A \to A'\), for each place of \(A\) belonging to a direct factor \(K\), the map \(f\) sends \(K\) to one of the direct factors \(K'\) of \(A'\) and we may then define \(f(P) \in \Dc(K') \subset \Dc(A')\) as in the case of field extensions.
    
    This yields a complex
    \[\hdots \to \Dc(F^{\otimes n+1}) \xrightarrow{\Dc(\Delta_{Am}^{n})} \Dc(F^{\otimes n+2}) \to \hdots.\]
of abelian groups, where we define \(\Dc(\Delta_{Am}^n) = \sum_{0 \leq i \leq n} (-1)^i \Dc(\epsilon^n_i)\), using the fact that the \(\epsilon\) maps are injective.

For the remainder of this section, we let \(F\) be an étale \(k\)-algebra.

We give a precise description of the map \(\Dc(\Delta_{Am}^n)\). Let \(Q \in \Pl(F^{\otimes n+2})\). Then, for any \(0 \leq i \leq n+1\), there is exactly one \(P \in \Pl(F^{\otimes n+1})\) such that \(Q \mid \epsilon_i^n(P)\), and we call this place \(Q_i\). Then, if \(P \in \Pl(F^{\otimes n+1})\), we get
\[\Dc(\epsilon_i^n)(P) = \sum_{\substack Q \in \Pl(F^{\otimes n+2}) \\ Q_i = P} e_{Q,i} Q,\]
where \(e_{Q,i}\) is the ramification index of \(Q\) over \(\epsilon_i^n\colon F^{\otimes n+1} \to F^{\otimes n+2}\), and it follows that 
\[\Dc(\epsilon_i^n)\left(\sum_{P \in \Pl(F^{\otimes n+1})} n_P P\right) = \sum_{Q \in \Pl(F^{\otimes n+2})} e_{Q,i} n_{Q_i} Q\]
and
\[\Dc(\Delta_{Am}^n)\left(\sum_{P \in \Pl(F^{\otimes n+1})} n_P P\right) = \sum_{Q \in \Pl(F^{\otimes n+2})} \left(\sum_{0 \leq i \leq {n+1}} (-1)^i e_{Q,i} n_{Q_i} \right) Q.\]

We first need two lemmas:

\begin{lemma}\label{lemma:CocycleTransit}
    Let \(Q,Q'\) be places of \(F^{\otimes 2}\) such that \(Q_0 = Q'_0 = P\). Then there exists a place \(R \in \Pl(F^{\otimes 3})\) such that \(R_1 = Q\) and \(R_2 = Q'\).
\end{lemma}

\begin{proof}
    We set \(P = Q_0 = Q'_0\), and consider the local field \(F_P\) obtained by first selecting the direct factor of \(F\) to which \(P\) belongs and then taking the completion at \(P\). The set of places of \(F^{\otimes 2}\) that divide \(\epsilon_1^0(P)\) is in bijection with the set of direct factors of \(F_P \otimes_F F^{\otimes 2} \simeq F_P \otimes_k F\). Likewise, the set of places of \(F^{\otimes 3}\) above \(P\) (for the embedding \(\epsilon_1^1 \circ \epsilon_1^0 = Id_F \otimes 1 \otimes 1\)) is in bijection with the set of direct factors of \(F_P \otimes_k F \otimes_k F\). Furthermore, the maps \(\epsilon_1^1\) and \(\epsilon_1^2\) may be expanded to \(F_P \otimes_k F\), and if \(R \in \Pl(F^{\otimes 3})\) is the place corresponding to some direct factor \(K\) of \(F_P \otimes F \otimes F\), then, the place \(R_1\) (resp. \(R_2\)) corresponds to the factor of \(F_P \otimes F\) which is mapped into \(K\) by \(\epsilon_1^1\) (resp. \(\epsilon_1^2\)).

    Now, let \(\alpha \in k[X]\) be a defining polynomial for \(F\). We have isomorphisms \(F_P \otimes F \simeq F_P[X]/(\alpha(X))\) and \(F_P \otimes F \otimes F \simeq F_P[X,Y]/(\alpha(X),\alpha(Y))\). A direct factor of \(F_P \otimes F\) corresponds to an irreducible factor of \(\alpha\) in \(F_P[X]\). Likewise, a factor of \(F_P \otimes F \otimes F\) is uniquely described by the choice of an irreducible factor \(\beta\) of \(\alpha(X)\) in \(F_P[X]\) and an irreducible factor \(\gamma\) of \(\alpha(Y)\) in \(\left(F_P[X]/\beta(X)\right)[Y]\). We note that each factor \(\gamma\) is obtained as a factor of some \(\beta'\), itself an irreducible factor of \(\alpha\) in \(F_P[Y]\).

    Furthermore, if \(R \in \Pl(F^{\otimes 3})\) corresponds to some factors \(\beta\) and \(\gamma\) of \(\alpha\), then \(R_2\) is the place of \(F^{\otimes 2}\) corresponding to \(\beta\) and \(R_1\) is the place of \(F^{\otimes 2}\) corresponding to the irreducible factor \(\beta'\) of \(\alpha\) in \(\F_P[X]\) corresponding to \(\gamma\) as described above.

    Now, we fix \(\beta\) and \(\beta'\) the irreducible factors of \(\alpha\) corresponding respectively to \(Q\) and \(Q'\). We let \(\gamma\) be an irreducible factor of \(\beta'(Y)\) in \(\left(F_P[X]/\beta(X)\right)[Y]\) and the place \(R\) corresponding to \(\beta\) and \(\gamma\) is as demanded.
\end{proof}

We may now prove a generalized version of Hilbert's theorem 90 in our setting. We say that a place \(P \in \Pl(F)\) is unramified if, for all \(Q \in \Pl(Q)\), if \(F = Q_0\) then \(e_{Q,0} = 1\) and if \(F = Q_1\) then \(e_{Q,1} = 1\).
\begin{lemma}\label{lemma:Hilbert90}
    Let \(D = \sum_{Q \in \Pl(F^{\otimes 2})} n_Q Q \in \Ker \Dc(\Delta_{Am}^1)\) be supported by unramified places. That is, for all \(Q \in \Pl(F^{\otimes 2})\), if \(Q_0\) or \(Q_1\) is ramified, then \(n_Q = 0\). Then, there exists \(E \in \Dc(F)\) such that \(D = \Dc(\Delta_{Am}^0)(E)\).
\end{lemma}

\begin{proof}
    We set
    \[E = \sum_{P \in \Pl(F)} \left(\min_{\substack{Q \in \Pl(F^{\otimes 2}) \\ Q_0 = P}} n_Q \right) P.\]
    Then, we get 
    \[\Dc(\epsilon_0^0)(E) = \sum_{Q \in \Pl(F^{\otimes 2})} \left(\min_{\substack{Q' \in \Pl(F^{\otimes 2}) \\ Q'_0 = Q_1}}  n_{Q'}\right) Q\]
    and
    \[\Dc(\epsilon_1^0)(E) = \sum_{Q \in \Pl(F^{\otimes 2})} \left(\min_{\substack{Q' \in \Pl(F^{\otimes 2}) \\ Q'_0 = Q_0}}  n_{Q'}\right) Q\]

    It follows that
    \[D + \Dc(\epsilon_0^0)(E) = \sum_{Q \in \Pl(F^{\otimes 2})} \left(\min_{\substack{Q' \in \Pl(F^{\otimes 2}) \\ Q'_0 = Q_1}}  n_Q + n_{Q'}\right) Q.\]
    If we fix places \(Q,Q' \in \Pl(F^{\otimes 2})\) such that \(Q'_0 = Q_1\), we apply \cref{lemma:CocycleTransit} to \(Q'\) and \(Q^\sigma\), the image of \(Q\) by the automorphism \(\sigma\colon a \otimes b \mapsto b \otimes a\) of \(F^{\otimes 2}\). We find that there exists \(R \in \Pl(F^{\otimes 3})\) such that \(R_1 = Q'\) and \(R_2 = Q^\sigma\). Then, consider \(R^\tau\), the image of \(R\) by the automorphism \(\tau \colon a \otimes b \otimes c \mapsto b \otimes a \otimes c\) of \(F^{\otimes 3}\). We get \(R^\tau_2 = Q\) and \(R^\tau_0 = Q'\). We may then set \(Q'' = R^\tau_1\) and, as \(\Dc(\Delta_{Am}^1)(D) = 0\), we get that \(n_Q + n_{Q'} = n_{Q''}\), for some \(Q''\) such that \(Q''_0 = Q_0\).

    Conversely, if we fix \(Q,Q'' \in \Pl(F^{\otimes 2})\) such that \(Q_0 = Q''_0\), then there exists \(R \in \Pl(F^{\otimes 3})\) such that \(R_2 = Q\) and \(R_1 = Q''\). We set \(Q' = R_0\) and we get that \(n_Q + n_{Q'} = n_{Q''}\), and observe that \(Q'_0 = Q_1\).

    This shows that for \(Q \in \Pl(F^{\otimes 2})\), \[\min_{\substack{Q' \in \Pl(F^{\otimes 2}) \\ Q'_0 = Q_1}}  n_Q + n_{Q'} = \min_{\substack{Q' \in \Pl(F^{\otimes 2}) \\ Q'_0 = Q_0}}  n_{Q'}.\]
    Therefore, \(D + \epsilon_0^0(E) = \epsilon_1^0(E)\). That is, \(D = \Dc(\Delta_{Am}^0)(-E)\).
\end{proof}

In what follows, if \(S\) is a set of places of \(F\), we define inductively \(S^{(1)} = S\) and
\[S^{(i+1)} = \left\{Q \in \Pl(F^{\otimes i+1}) \mid \exists 0 \leq j \leq i, P \in S^{(i)} \colon Q|\varepsilon_j^{i-1}(P)\right\}.\]
We now get our main theorem for this section:
\begin{theorem}\label{thm:SUnitTriv}
    Let \(b \in B_{Am}^2(k,F)\) be a coboundary. Let \(S\) be a finite set of places of \(F\) such that:
    \begin{itemize}
        \item \(S\) contains the infinite places of \(F\).
        \item \(S\) contains the places of \(F\) that ramify in \(F^{\otimes 2}\).
        \item The finite places of \(S\) generate the class group \(\Cl(F)\).
        \item The places in the support of \(b\) are contained in \(S^{(3)}\).
    \end{itemize}
    Then there exists a cochain \(\sigma\) in the group of \(S^{(2)}\)-units of \(F^{\otimes 2}\) such that \(b = \Delta_{Am}^1(\sigma)\)
\end{theorem}

\begin{proof}
 Let \(\alpha \in (F^{\otimes 2})^\times\) be such that \(\Delta_{Am}^1(\alpha) = b\). We consider the divisor \(D = \Dc(\alpha) = \sum_{Q \in \Pl(F^{\otimes 2})} n_Q Q\) of \(\alpha\). We set \(D_S = \sum_{Q \in S^{(2)}} n_Q Q\) and \(D_{\bar{S}} = \sum_{Q \notin S^{(2)}} n_Q Q\). Now, \(\Dc(\Delta_{Am}^1)(D)\) is the divisor of \(F^{\otimes 3}\) corresponding to \(b\) and therefore is supported by \(S^{(3)}\). Observe that if \(Q \in S^{(2)}\), then \(\Dc(\Delta_{Am}^1)(Q)\) has support in \(S^{(3)}\). It follows that \(\Dc(\Delta_{Am}^1)(D_{\bar{S}}) = 0\).

 The divisor \(D_{\bar{S}}\) has no ramified place of \(F^{\otimes 2}\) in its support. We may therefore apply \cref{lemma:Hilbert90} and get a divisor \(E \in \Dc(F)\) such that \(D_{\bar{S}} = \Delta_{Am}^0(E)\). Now, as \(S\) generates the class group of \(F\), there exists \(E' \in \Dc(F)\) with support in \(S\) and \(\gamma \in F^\times\) such that \(E = \Dc(\gamma) + E'\). Then, we get that
 \[\Dc(\Delta_{Am}^0)(\Dc(\gamma)) + \Dc(\Delta_{Am}^0)(E') = D - D_{\bar{S}}\]
 and therefore
 \[\Dc(\Delta_{Am}^0)(E') + D_{\bar{S}} = \Dc(\alpha \Delta_{Am}^0(\gamma^{-1})).\]
 Now, this shows that \(\alpha \Delta_{Am}^0(\gamma^{-1})\) is a \(S^{(2)}\)-unit. Furthermore, \[\Delta_{Am}^1(\alpha\Delta_{Am}^0(\gamma^{-1})) = \Delta_{Am}^1(\alpha) = b,\]
 and \(\alpha \Delta_{Am}^0(\gamma^{-1})\) is a cochain with the required properties.
\end{proof}

From \cref{thm:SUnitTriv} we directly get an algorithm for computing a trivialisation of a \(2\)-coboundary:

\begin{algorithm}
    \caption{Computing a trivialisation of a \(2\)-coboundary}
    \label{algo:TrivCobound}
    \begin{algorithmic}[1]
        \REQUIRE A number field \(k\), an étale \(k\)-algebra \(F\)
        \REQUIRE A coboundary \(b \in B_{Am}^2(k,F)\)
        \STATE Compute \(S_1\), the set of places of \(F\) that ramify in \(F^{(2)}\).
        \STATE Compute \(S_2\), a set of places of \(F\) which generate the class group \(\Cl(F)\).
        \STATE Compute the divisor of \(F^{\otimes 3}\) corresponding to \(b\). Let \(S_3\) be the set of places of \(F\) below the places in the support of \(b\).
        \STATE Set \(S = S_1 \cup S_2 \cup S_3\).
        \STATE Compute the sets \(S^{(2)}\) and \(S^{(3)}\).
        \STATE Compute an isomorphism \(\phi\) from the group of \(S^{(2)}\)-units of \(F^{\otimes 2}\) to \(\Z^r \oplus \Z/m\Z\).
        \STATE Compute an isomorphism \(\psi\) from the group of \(S^{(3)}\)-units of \(F^{\otimes 3}\) to \(\Z^{r'} \oplus \Z/m'\Z\).
        \STATE Solve the linear equation \((\psi \circ \Delta_{Am}^1 \circ \phi^{-1})(\alpha) = \psi(b)\)
        \RETURN \(\alpha\)
    \end{algorithmic}
\end{algorithm}

\begin{theorem}\label{thm:AlgoTrivCobound}
    Given a number field \(k\), and étale \(k\)-algebra \(F\) and a coboundary \(b \in B^2_{Am}(k,F)\), \cref{algo:TrivCobound} outputs a cochain \(\alpha \in C^1(k,F)\) such that \(\Delta_{Am}^1(\alpha) = b\). Furthermore,  \cref{algo:TrivCobound} runs in polynomial time on a quantum computer.
\end{theorem}

\begin{proof}
    Using a polynomial-time algorithm for factoring polynomials over number fields \cite{lenstra1983factoring}, one may compute splitting isomorphisms 
    \[F \simeq \bigoplus_\alpha F_\alpha,\]
    \[F^{\otimes 2} \simeq \bigoplus_\beta F^{(2)}_\beta,\]
    and
    \[F^{\otimes 3} \simeq \bigoplus_\gamma F^{(3)}_\gamma.\]

    Using these isomorphisms, the steps of the computation of \(S\) may be done over number field extensions. Then, this entails computing and factoring relative discriminants, computing class groups and factoring \(b\) in the ideal group of \(F^{\otimes 3}\). The sets \(S^{(2)}\) and \(S^{(3)}\) are computed by factoring images \(\epsilon_i^0(P)\) for \(P \in S\) and then \(\epsilon_i^1(Q)\) for \(Q \in S^{(2)}\). Then, isomorphisms \(\phi\) and \(\psi\) are computed using an algorithm for computing \(S\)-unit groups. Finally, the last step is the computation of a solution of an integral linear system.
    Each of these tasks can be accomplished in quantum polynomial time according to Theorem \ref{thm:s-unit&norm}. 
    The correctness of the algorithm relies on the fact that a cochain \(\alpha\) such that \(b = \Delta_{Am}^1(\alpha)\) exists and may be found in the group of \(S^{(2)}\)-units, which is the content of \cref{thm:SUnitTriv}.
\end{proof}

\begin{cor}\label{cor:QuantumSplit}
    There exists a polynomial quantum algorithm which, give a number field \(k\) and an algebra \(A \simeq M_n(k)\), computes an explicit algorithm from \(A\) to \(M_n(k)\).
\end{cor}

\begin{proof}
    This is simply a combination of \cref{thm:AlgoFindCocycle,thm:AlgoTrivCobound}. Indeed, using \Cref{algo:FindCocycle}, one may compute an étale \(k\) algebra \(F\) and a cocycle \(c \in \Z^2(k,F)\) representing \(A\). As \(A\) is isomorphic to \(M_n(k)\), the cocycle \(c\) is in fact a coboundary. Then, a cochain \(\alpha \in C^1(k,F)\) such that \(\Delta_{Am}^1(\alpha) = c\) may be computed using \cref{algo:TrivCobound}. Applying \Cref{cor:AssoCocyIsomAmitsur}, we obtain an explicit isomorphism \(A \simeq A(F,1)\). Finally, an isomorphism \(A(F,1) \simeq M_n(k)\) may easily be computed using the left ideal provided in \Cref{ex:AmiTrivialCocycle}.
\end{proof}
\begin{remark}
For quaternion algebras $(a,b)_K$ it is known that splitting is equivalent to solving the norm equation $N_{\mathbb{Q(\sqrt{a})}|\mathbb{Q}}(x)=b$ thus one can also directly apply Theorem \ref{thm:s-unit&norm} to find an explicit isomorphism to $M_2(K)$. A similar statement can be derived for degree three central simple algebras as there is a polynomial-time algorithm for finding a cyclic algebra presentation. 
\end{remark}