\documentclass{article}

\usepackage{amsfonts}
\usepackage{amsmath}
\usepackage{amsthm}
\usepackage{mathtools}
\usepackage{cite}
\usepackage{yfonts}
\usepackage{algorithmic}
\usepackage{algorithm}
\usepackage{tikz-cd}
\usepackage{hyperref}
\usepackage{cleveref}
\usepackage{xcolor}

\Crefname{ALC@unique}{Line}{Lines}
\newcounter{myalg}
\AtBeginEnvironment{algorithmic}{\refstepcounter{myalg}}
\makeatletter
\@addtoreset{ALC@unique}{myalg}
\makeatother

\newtheorem{theorem}{Theorem}
\newtheorem{lemma}{Lemma}
\newtheorem{prop}{Proposition}
\newtheorem{cor}{Corollary}

\theoremstyle{definition}
\newtheorem{definition}{Definition}
\newtheorem{heuristic}{Heuristic}

\theoremstyle{remark}
\newtheorem{remark}{Remark}
\newtheorem{example}{Example}


\DeclareMathOperator{\Tr}{Tr}
\DeclareMathOperator{\Diag}{Diag}
\DeclareMathOperator{\Gal}{Gal}
\DeclareMathOperator{\Aut}{Aut}
\DeclareMathOperator{\disc}{Disc}
\DeclareMathOperator{\rk}{Rk}
\DeclareMathOperator{\Ker}{Ker}
\DeclareMathOperator{\im}{Im}
\DeclareMathOperator{\Cl}{Cl}
\DeclareMathOperator{\Disc}{Disc}
\DeclareMathOperator{\Basis}{Basis}
\DeclareMathOperator{\IsIrreducible}{IsIrreducible}
\DeclareMathOperator{\Dc}{\mathcal{D}}
\DeclareMathOperator{\Pc}{\mathcal{P}}
\DeclareMathOperator{\Pl}{Pl}
\DeclareMathOperator{\Br}{Br}

\newcommand{\Q}{\mathbb{Q}}
\newcommand{\Z}{\mathbb{Z}}
\newcommand{\N}{\mathbb{N}}
\newcommand{\F}{\mathbb{F}}
\newcommand{\U}{\mathbb{U}}
\renewcommand{\S}{\mathfrak{S}}
\newcommand{\A}{\mathfrak{A}}
\newcommand{\Oc}{\mathcal{O}}
\newcommand{\B}{\mathcal{B}}
\newcommand{\Pp}{\mathbb{P}}
\newcommand{\p}{\mathfrak{p}}
\newcommand{\1}{\mathbf{1}}
\renewcommand{\epsilon}{\varepsilon}

\newcommand{\MM}[1]{\textcolor{blue}{{\sf(Mickael's comment:} {\sl{#1})}}}
\newcommand{\PK}[1]{\textcolor{red}{{\sf(Peter's comment:} {\sl{#1})}}}

% Add a serial/Oxford comma by default.
\newcommand{\creflastconjunction}{, and~}

\title{Efficient computations in central simple algebras using Amitsur cohomology}

% Authors: full names plus addresses.
%\author{Péter Kutas\thanks{Faculty of Informatics, Eötvös Loránd University and School of Computer Science, University of Birmingham
  %(\email{p.kutas@bham.ac.uk},\url{https://sites.google.com/view/peterkutas89}). The first author is supported by the Hungarian Ministry of Innovation and Technology NRDI Office within the framework of the Quantum Information National Laboratory Program, the János Bolyai Research Scholarship of the Hungarian Academy of Sciences and by the UNKP-22-5 New National Excellence Program. The first author is also partly supported by EPSRC through grant number EP/V011324/1.}
  %\and Mickaël Montessinos\thanks{Institute of Mathematics, Faculty of Mathematics and Informatics, Vilnius University
  %(\email{mickael@montessinos.fr}, \url{http://mickael.montessinos.fr}).}
%}
\author{Péter Kutas and Mickaël Montessinos}


\begin{document}

\maketitle

\begin{abstract}
    We propose a presentation for central simple algebras over a field \(k\) using Amitsur cohomology. We provide efficient algorithms for computing a cocycle corresponding to any such algebra given by structure constants. If \(k\) is a number field, we use this presentation to prove that the explicit isomorphism problem (i.e., finding an isomorphism between central simple algebras given by structure constants) reduces to \(S\)-unit group computation and other related number theoretical computational problems. This also yields the first polynomial quantum algorithm for the explicit isomorphism problem over number fields.
\end{abstract}

\section{Introduction}
% \leavevmode
% \\
% \\
% \\
% \\
% \\
\section{Introduction}
\label{introduction}

AutoML is the process by which machine learning models are built automatically for a new dataset. Given a dataset, AutoML systems perform a search over valid data transformations and learners, along with hyper-parameter optimization for each learner~\cite{VolcanoML}. Choosing the transformations and learners over which to search is our focus.
A significant number of systems mine from prior runs of pipelines over a set of datasets to choose transformers and learners that are effective with different types of datasets (e.g. \cite{NEURIPS2018_b59a51a3}, \cite{10.14778/3415478.3415542}, \cite{autosklearn}). Thus, they build a database by actually running different pipelines with a diverse set of datasets to estimate the accuracy of potential pipelines. Hence, they can be used to effectively reduce the search space. A new dataset, based on a set of features (meta-features) is then matched to this database to find the most plausible candidates for both learner selection and hyper-parameter tuning. This process of choosing starting points in the search space is called meta-learning for the cold start problem.  

Other meta-learning approaches include mining existing data science code and their associated datasets to learn from human expertise. The AL~\cite{al} system mined existing Kaggle notebooks using dynamic analysis, i.e., actually running the scripts, and showed that such a system has promise.  However, this meta-learning approach does not scale because it is onerous to execute a large number of pipeline scripts on datasets, preprocessing datasets is never trivial, and older scripts cease to run at all as software evolves. It is not surprising that AL therefore performed dynamic analysis on just nine datasets.

Our system, {\sysname}, provides a scalable meta-learning approach to leverage human expertise, using static analysis to mine pipelines from large repositories of scripts. Static analysis has the advantage of scaling to thousands or millions of scripts \cite{graph4code} easily, but lacks the performance data gathered by dynamic analysis. The {\sysname} meta-learning approach guides the learning process by a scalable dataset similarity search, based on dataset embeddings, to find the most similar datasets and the semantics of ML pipelines applied on them.  Many existing systems, such as Auto-Sklearn \cite{autosklearn} and AL \cite{al}, compute a set of meta-features for each dataset. We developed a deep neural network model to generate embeddings at the granularity of a dataset, e.g., a table or CSV file, to capture similarity at the level of an entire dataset rather than relying on a set of meta-features.
 
Because we use static analysis to capture the semantics of the meta-learning process, we have no mechanism to choose the \textbf{best} pipeline from many seen pipelines, unlike the dynamic execution case where one can rely on runtime to choose the best performing pipeline.  Observing that pipelines are basically workflow graphs, we use graph generator neural models to succinctly capture the statically-observed pipelines for a single dataset. In {\sysname}, we formulate learner selection as a graph generation problem to predict optimized pipelines based on pipelines seen in actual notebooks.

%. This formulation enables {\sysname} for effective pruning of the AutoML search space to predict optimized pipelines based on pipelines seen in actual notebooks.}
%We note that increasingly, state-of-the-art performance in AutoML systems is being generated by more complex pipelines such as Directed Acyclic Graphs (DAGs) \cite{piper} rather than the linear pipelines used in earlier systems.  
 
{\sysname} does learner and transformation selection, and hence is a component of an AutoML systems. To evaluate this component, we integrated it into two existing AutoML systems, FLAML \cite{flaml} and Auto-Sklearn \cite{autosklearn}.  
% We evaluate each system with and without {\sysname}.  
We chose FLAML because it does not yet have any meta-learning component for the cold start problem and instead allows user selection of learners and transformers. The authors of FLAML explicitly pointed to the fact that FLAML might benefit from a meta-learning component and pointed to it as a possibility for future work. For FLAML, if mining historical pipelines provides an advantage, we should improve its performance. We also picked Auto-Sklearn as it does have a learner selection component based on meta-features, as described earlier~\cite{autosklearn2}. For Auto-Sklearn, we should at least match performance if our static mining of pipelines can match their extensive database. For context, we also compared {\sysname} with the recent VolcanoML~\cite{VolcanoML}, which provides an efficient decomposition and execution strategy for the AutoML search space. In contrast, {\sysname} prunes the search space using our meta-learning model to perform hyperparameter optimization only for the most promising candidates. 

The contributions of this paper are the following:
\begin{itemize}
    \item Section ~\ref{sec:mining} defines a scalable meta-learning approach based on representation learning of mined ML pipeline semantics and datasets for over 100 datasets and ~11K Python scripts.  
    \newline
    \item Sections~\ref{sec:kgpipGen} formulates AutoML pipeline generation as a graph generation problem. {\sysname} predicts efficiently an optimized ML pipeline for an unseen dataset based on our meta-learning model.  To the best of our knowledge, {\sysname} is the first approach to formulate  AutoML pipeline generation in such a way.
    \newline
    \item Section~\ref{sec:eval} presents a comprehensive evaluation using a large collection of 121 datasets from major AutoML benchmarks and Kaggle. Our experimental results show that {\sysname} outperforms all existing AutoML systems and achieves state-of-the-art results on the majority of these datasets. {\sysname} significantly improves the performance of both FLAML and Auto-Sklearn in classification and regression tasks. We also outperformed AL in 75 out of 77 datasets and VolcanoML in 75  out of 121 datasets, including 44 datasets used only by VolcanoML~\cite{VolcanoML}.  On average, {\sysname} achieves scores that are statistically better than the means of all other systems. 
\end{itemize}


%This approach does not need to apply cleaning or transformation methods to handle different variances among datasets. Moreover, we do not need to deal with complex analysis, such as dynamic code analysis. Thus, our approach proved to be scalable, as discussed in Sections~\ref{sec:mining}.

\section{Preliminaries}\label{Sec:Prelim}
%!TEX root = hopfwright.tex
%

In this section we systematically recast the Hopf bifurcation problem in Fourier space. 
We introduce appropriate scalings, sequence spaces of Fourier coefficients and convenient operators on these spaces. 
To study Equation~\eqref{eq:FourierSequenceEquation} we consider Fourier sequences $ \{a_k\}$ and fix a Banach space in which these sequences reside. It is indispensable for our analysis that this space have an algebraic structure. 
The Wiener algebra of absolutely summable Fourier series is a natural candidate, which we use with minor modifications. 
In numerical applications, weighted sequence spaces with algebraic and geometric decay have been used to great effect to study periodic solutions which are $C^k$ and analytic, respectively~\cite{lessard2010recent,hungria2016rigorous}. 
Although it follows from Lemma~\ref{l:analytic} that the Fourier coefficients of any solution decay exponentially, we choose to work in a space of less regularity. 
The reason is that by working in a space with less regularity, we are better able to connect our results with the global estimates in \cite{neumaier2014global}, see Theorem~\ref{thm:UniqunessNbd2}.


%
%
%\begin{remark}
%	Although it follows from Lemma~\ref{l:analytic} that the Fourier coefficients of any solution decay exponentially, we choose to work in a space of less regularity, namely summable Fourier coefficients. This allows us to draw SOME MORE INTERESTING CONCLUSION LATER.
%	EXPLAIN WHY WE CHOOSE A NORM WITH ALMOST NO DECAY!
%	% of s Periodic solutions to Wright's equation are known to be real analytic and so their  Fourier coefficients must decay geometrically [Nussbaum].
%	% We do not use such a strong result;  any periodic solution must be continuously differentiable, by which it follows that $ \sum | c_k| < \infty$.
%\end{remark}


\begin{remark}\label{r:a0}
There is considerable redundancy in Equation~\eqref{eq:FourierSequenceEquation}. First, since we are considering real-valued solutions $y$, we assume $\c_{-k}$ is the complex conjugate of $\c_k$. This symmetry implies it suffices to consider Equation~\eqref{eq:FourierSequenceEquation} for $k \geq 0$.
Second, we may effectively ignore the zeroth Fourier coefficient of any periodic solution \cite{jones1962existence}, since it is necessarily equal to $0$. 
%In \cite{jones1962existence}, it is shown that if $y \not\equiv -1$ is a periodic solution of~\eqref{eq:Wright} with frequency $\omega$, then $ \int_0^{2\pi/\omega} y(t) dt =0$. 
		The self contained argument is as follows. 
		As mentioned in the introduction, any periodic solution to Wright's equation must satisfy $ y(t) > -1$ for all $t$. 
	By dividing Equation~\eqref{eq:Wright} by $(1+y(t))$, which never vanishes, we obtain
	\[
	\frac{d}{dt} \log (1 + y(t)) = - \alpha y(t-1).
	\]  
	Integrating over one period $L$ we derive the condition 
	$0=\int_0^L y(t) dt $.
	Hence $a_0=0$ for any periodic solution. 
	It will be shown in Theorem~\ref{thm:FourierEquivalence1} that a related argument implies that we do not need to consider Equation~\eqref{eq:FourierSequenceEquation} for $k=0$.
\end{remark}

%%%
%%%
%%%\begin{remark}\label{r:c0} 
%%%In \cite{jones1962existence}, it is shown that if $y \not\equiv -1$ is a periodic solution of~\eqref{eq:Wright} with frequency $\omega$, then $ \int_0^{2\pi/\omega} y(t) dt =0$. 
%%%PERHAPS TOO MUCH DETAIL HERE. The self contained argument is as follows.
%%%If $y \not\equiv -1$ then $y(t) \neq -1$ for all $t$, since if $y(t_0)=-1$ for some $t_0 \in \R$ then $y'(t_0)=0$ by~\eqref{eq:Wright} and in fact by differentiating~\eqref{eq:Wright} repeatedly one obtains that all derivatives of $y$ vanish at $t_0$. Hence $y \equiv -1$ by Lemma~\ref{l:analytic}, a contradiction. Now divide~\eqref{eq:Wright} by $(1+y(t))$, which never vanishes, to obtain
%%%\[
%%%  \frac{d}{dt} \log |1 + y(t)| = - \alpha y(t-1).
%%%\]  
%%%Integrating over one period we obtain $\int_0^L y(t) dt =0$.
%%%\end{remark}



%Furthermore, the condition that $y(t)$ is real forces $\c_{-k} = \overline{\c}_{k}$.  
%
We define the spaces of absolutely summable Fourier series
\begin{alignat*}{1}
	\ell^1 &:= \left\{ \{ \c_k \}_{k \geq 1} : 
    \sum_{k \geq 1} | \c_k| < \infty  \right\} , \\
	\ell^1_\bi &:= \left\{ \{ \c_k \}_{k \in \Z} : 
    \sum_{k \in \Z} | \c_k| < \infty  \right\} .
\end{alignat*} 
We identify any semi-infinite sequence $ \{ \c_k \}_{k \geq 1} \in \ell^1$ with the bi-infinite sequence $ \{ \c_k \}_{k \in \Z} \in \ell^1_\bi$ via the conventions (see Remark~\ref{r:a0})
\begin{equation}
  \c_0=0 \qquad\text{ and }\qquad \c_{-k} = \c_{k}^*. 
\end{equation}
In other word, we identify $\ell^1$ with the set
\begin{equation*}
   \ell^1_\sym := \left\{ \c \in \ell^1_\bi : 
	\c_0=0,~\c_{-k}=\c_k^* \right\} .
\end{equation*}
On $\ell^1$ we introduce the norm
\begin{equation}\label{e:lnorm}
  \| \c \| = \| \c \|_{\ell^1} := 2 \sum_{k = 1}^\infty |\c_k|.
\end{equation}
The factor $2$ in this norm is chosen to have a Banach algebra estimate.
Indeed, for $\c, \tilde{\c} \in \ell^1 \cong \ell^1_\sym$ we define
the discrete convolution 
\[
\left[ \c * \tilde{\c} \right]_k = \sum_{\substack{k_1,k_2\in\Z\\ k_1 + k_2 = k}} \c_{k_1} \tilde{\c}_{k_2} .
\]
Although $[\c*\tilde{\c}]_0$ does not necessarily vanish, we have $\{\c*\tilde{\c}\}_{k \geq 1} \in \ell^1 $ and 
\begin{equation*}
	\| \c*\tilde{\c} \| \leq \| \c \| \cdot  \| \tilde{\c} \| 
	\qquad\text{for all } \c , \tilde{\c} \in \ell^1, 
\end{equation*}
hence $\ell^1$ with norm~\eqref{e:lnorm} is a Banach algebra.

By Lemma~\ref{l:analytic} it is clear that any periodic solution of~\eqref{eq:Wright} has a well-defined Fourier series $\c \in \ell^1_\bi$. 
The next theorem shows that in order to study periodic orbits to Wright's equation we only need to study Equation~\eqref{eq:FourierSequenceEquation} 
for $k \geq 1$. For convenience we introduce the notation 
\[
G(\alpha,\omega,\c)_k=
( i \omega k + \alpha e^{ - i \omega k}) \c_k + \alpha \sum_{k_1 + k_2 = k} e^{- i \omega k_1} \c_{k_1} \c_{k_2} \qquad \text{for } k \in \N.
\]
We note that we may interpret the trivial solution $y(t)\equiv 0$ as a periodic solution of arbitrary period.
\begin{theorem}
\label{thm:FourierEquivalence1}
Let $\alpha>0$ and $\omega>0$.
If $\c \in \ell^1 \cong \ell^1_{\sym}$ solves
$G(\alpha,\omega,\c)_k =0$  for all $k \geq 1$,
then $y(t)$ given by~\eqref{eq:FourierEquation} is a periodic solution of~\eqref{eq:Wright} with period~$2\pi/\omega$.
Vice versa, if $y(t)$ is a periodic solution of~\eqref{eq:Wright} with period~$2\pi/\omega$ then its Fourier coefficients $\c \in \ell^1_\bi$ lie in $\ell^1_\sym \cong \ell^1$ and solve $G(\alpha,\omega,\c)_k =0$ for all $k \geq 1$.
\end{theorem}

\begin{proof}	
	If $y(t)$ is a periodic solution of~\eqref{eq:Wright} then it is real analytic by Lemma~\ref{l:analytic}, hence its Fourier series $\c$ is well-defined and $\c \in \ell^1_{\sym}$ by Remark~\ref{r:a0}.
	Plugging the Fourier series~\eqref{eq:FourierEquation} into~\eqref{eq:Wright} one easily derives that $\c$ solves~\eqref{eq:FourierSequenceEquation} for all $k \geq 1$.

To prove the reverse implication, assume that $\c \in \ell^1_\sym$ solves
Equation~\eqref{eq:FourierSequenceEquation} for all $k \geq 1$. Since $\c_{-k}
= \c_k^*$, Equation \eqref{eq:FourierSequenceEquation} is also satisfied for
all $k \leq -1$. It follows from the Banach algebra property and
\eqref{eq:FourierSequenceEquation} that $\{k \c_k\}_{k \in \Z} \in \ell^1_\bi$,
hence $y$, given by~\eqref{eq:FourierEquation}, is continuously differentiable.
% (and by bootstrapping one infers that $\{k^m c_k \} \in \ell^1_\bi$, 
% hence $y \in C^m$ for any $m \geq 1$).
	Since~\eqref{eq:FourierSequenceEquation} is satisfied for all $k \in \Z \setminus \{0\}$ (but not necessarily for $k=0$) one may perform the inverse Fourier transform on~\eqref{eq:FourierSequenceEquation} to conclude that
	$y$ satisfies the delay equation 
\begin{equation}\label{eq:delaywithK}
   	y'(t) = - \alpha y(t-1) [ 1 + y(t)] + C
\end{equation}
	for some constant $C \in \R$. 
   Finally, to prove that $C=0$ we argue by contradiction.
   Suppose $C \neq 0$. Then $y(t) \neq -1$ for all $t$.
   Namely, at any point where $y(t_0) =-1$ one would have $y'(t_0) = C$
   which has fixed sign,   hence it would follow that $y$ is not periodic
   ($y$ would not be able to cross $-1$ in the opposite direction, 
   preventing $y$  from being periodic).  
  We may thus divide~\eqref{eq:delaywithK} through by $1 + y(t)$ and obtain 
\begin{equation*}
	\frac{d}{dt} \log | 1 + y(t) | = - \alpha y(t-1) + \frac{C}{1+y(t)} .
\end{equation*}
	By integrating both sides of the equation over one period $L$ and by using that $\c_0=0$, we 
	obtain
	\[
	 C \int_0^L \frac{1}{1+y(t)} dt =0.
	\]
	Since the integrand is either strictly negative or strictly positive, this implies that $C=0$. Hence~\eqref{eq:delaywithK} reduces to~\eqref{eq:Wright},
	and $y$ satisfies Wright's equation. 
\end{proof}






To efficiently study Equation~\eqref{eq:FourierSequenceEquation}, we introduce the following linear operators on $ \ell^1$:
\begin{alignat*}{1}
   [K \c ]_k &:= k^{-1} \c_k  , \\ 
   [ U_\omega \c ]_k &:= e^{-i k \omega} \c_k  .
\end{alignat*}
The map $K$ is a compact operator, and it has a densely defined inverse $K^{-1}$. The domain of $K^{-1}$ is denoted by
\[
  \ell^K := \{ \c \in \ell^1 : K^{-1} \c \in \ell^1 \}.  
\]
The map $U_{\omega}$ is a unitary operator on $\ell^1$, but
it is discontinuous in $\omega$. 
With this notation, Theorem~\ref{thm:FourierEquivalence1} implies that our problem of finding a SOPS to~\eqref{eq:Wright} is equivalent to finding an $\c \in \ell^1$ such that
\begin{equation}
\label{e:defG}
  G(\alpha,\omega,\c) :=
  \left( i \omega K^{-1} + \alpha U_\omega \right) \c + \alpha \left[U_\omega \, \c \right] * \c  = 0.
\end{equation}


%In order for the solutions of Equation \ref{eq:FHat} to be isolated we need to impose a phase condition. 
%If there is a sequence $ \{ c_k \} $ which satisfies  Equation \ref{eq:FHat}, then $ y( t + \tau) = \sum_{k \in \Z} c_k e^{ i k \omega (t + \tau)}$ satisfies Wright's equation at parameter $\alpha$. 
%Fix $ \tau = - Arg[c_1] / \omega$ so that $ c_1  e^{ i \omega \tau} $ is a nonnegative real number. 
%By Proposition \ref{thm:FourierEquivalence1} it follows that $\{ c'_k \} =  \{c_k e^{ i \omega k \tau }   \}$ is a solution to Equation \ref{eq:FHat}, and furthermore that $ c'_1 = \epsilon$ for some $ \epsilon \geq 0$. 


Periodic solutions are invariant under time translation: if $y(t)$ solves Wright's equation, then so does $ y(t+\tau)$ for any $\tau \in \R$. 
We remove this degeneracy by adding a phase condition. 
Without loss of generality, if $\c \in \ell^1$ solves Equation~\eqref{e:defG}, we may assume that $\c_1 = \epsilon$ for some 
\emph{real non-negative}~$\epsilon$:
\[
  \ell^1_{\epsilon} := \{\c \in \ell^1 : \c_1 = \epsilon \} 
  \qquad \text{where } \epsilon \in \R,  \epsilon \geq 0.
\]
In the rest of our analysis, we will split elements $\c \in \ell^1$ into two parts: $\c_1$ and $\{\c_{k}\}_{k \geq 2}$.  
We define the basis elements $\e_j \in \ell^1$ for $j=1,2,\dots$ as
\[
  [\e_j]_k = \begin{cases}
  1 & \text{if } k=j, \\
  0 & \text{if } k \neq j.
  \end{cases}
\]
We note that $\| \e_j \|=2$. 
Then we can decompose
% We define
% \[
%   \tilde{\epsilon} := (\epsilon,0,0,0,\dots) \in \ell^1
% \]
% and
% For clarity when referring to sequences $\{c_{k}\}_{k \geq 2}$, we make the following definition:
% \[
% \ell^1_0  := \{ \tc \in \ell^1 : \tc_1 = 0 \}.
% \]
% With the
any $\c \in \ell^1_\epsilon$ uniquely as
\begin{equation}\label{e:aepsc}
  \c= \epsilon \e_1 + \tc \qquad \text{with}\quad 
  \tc \in \ell^1_0 := \{ \tc \in \ell^1 : \tc_1 = 0 \}.
\end{equation}
We follow the classical approach in studying Hopf bifurcations and consider 
$\c_1 = \epsilon$ to be a parameter, and then find periodic solutions with Fourier modes in $\ell^1_{\epsilon}$.
This approach rewrites the function $G: \R^2 \times \ell^K \to \ell^1$ as a function $\tilde{F}_\epsilon : \R^2 \times \ell^K_0 \to \ell^1$, where 
we denote 
\[
\ell^K_0 := \ell^1_0 \cap \ell^K.
\]
% I AM ACTUALLY NOT SURE IF YOU WANT TO DEFINE THIS WITH RANGE IN $\ell^1$
% OR WITH DOMAIN IN $\ell^1_0$ ?? IT SEEMS TO DEPEND ON WHICH GLOBAL STATEMENT YOU WANT/NEED TO MAKE!?
\begin{definition}
We define the $\epsilon$-parameterized family of  functions $\tilde{F}_\epsilon: \R^2 \times \ell^K_0  \to \ell^1$ 
by 
\begin{equation}
\label{eq:fourieroperators}
\tilde{F}_{\epsilon}(\alpha,\omega, \tc) := 
\epsilon [i \omega + \alpha e^{-i \omega}] \e_1 + 
( i \omega K^{-1} + \alpha U_{\omega}) \tc + 
\epsilon^2 \alpha e^{-i \omega}  \e_2  +
\alpha \epsilon L_\omega \tc + 
\alpha  [ U_{\omega} \tc] * \tc ,
\end{equation}
where
$L_\omega : \ell^1_0 \to \ell^1$ is given by
\[
   L_{\omega} := \sigma^+( e^{- i \omega} I + U_{\omega}) + \sigma^-(e^{i \omega} I + U_{\omega}),
\]
with $I$ the identity and  $\sigma^\pm$ the shift operators on $\ell^1$:
\begin{alignat*}{2}
\left[ \sigma^- a \right]_k &:=  a_{k+1}  , \\
\left[ \sigma^+ a \right]_k &:=  a_{k-1}  &\qquad&\text{with the convention } \c_0=0.
\end{alignat*}
The operator $ L_\omega$ is discontinuous in $\omega$ and $ \| L_\omega \| \leq 4$. 
\end{definition} 

%The maps $ \sigma^{+}$ and $ \sigma^-$ are shift up and shift down operators respectively. 
We reformulate Theorem~\ref{thm:FourierEquivalence1}  in terms of the map  $\tilde{F}$. 
We note that it follows from Lemma~\ref{l:analytic} and 
%\marginpar{Reformulate}
%one's choice of  
Equation~\eqref{eq:FourierSequenceEquation}  
%or Equation ~\eqref{eq:fourieroperators},
that the Fourier coefficients of any periodic solution of~\eqref{eq:Wright} lie in $\ell^K$.
These observations are summarized in the following theorem.
\begin{theorem}
\label{thm:FourierEquivalence2}
	Let $ \epsilon \geq 0$,  $\tc \in \ell^K_0$, $\alpha>0$ and $ \omega >0$. 
	Define $y: \R\to \R$ as 
\begin{equation}\label{e:ytc}
	y(t) = 
	\epsilon \left( e^{i \omega t }  + e^{- i \omega t }\right) 
	+  \sum_{k = 2}^\infty   \tc_k e^{i \omega k t }  + \tc_k^* e^{- i \omega k t } .
\end{equation}
%	and suppose that $ y(t) > -1$. 
	Then $y(t)$ solves~\eqref{eq:Wright} if and only if $\tilde{F}_{\epsilon}( \alpha , \omega , \tc) = 0$. 
	Furthermore, up to time translation, any periodic solution of~\eqref{eq:Wright} with period $2\pi/\omega$ is described by a Fourier series of the form~\eqref{e:ytc} with $\epsilon \geq 0$ and $\tc \in \ell^K_0$.
\end{theorem}


%We note that for $\epsilon>0$ such solutions are truly periodic, while for $\epsilon=0$ a zero of $\tilde{F}_\epsilon$ may either correspond to a periodic solution or to the trivial solution $y(t) \equiv 0$. 



% \begin{proof}
%  By Proposition \ref{thm:FourierEquivalence1}, it suffices to show that $\tilde{F}(\alpha,\omega,c) =0$ is equivalent to Equation \ref{eq:FourierSequenceEquation} being satisfied for $k \geq 1$.
%  Since Equation \ref{eq:FourierSequenceEquation} is equivalent to Equation \ref{eq:FHat}, we expand  Equation \ref{eq:FHat} by writing $ \hat{c} = \hat{\epsilon } + c$  where $ \hat{\epsilon} := (\epsilon,0,0,\dots) \in \ell^1$ as below:
%  \begin{equation}
%  0=  \left( i \omega K^{-1} + \alpha U_\omega \right) (\hat{\epsilon}+ c) + \alpha \left[U_\omega \, (\hat{\epsilon}+ c) \right] * (\hat{\epsilon}+ c) \label{eq:Intial}
%  \end{equation}
%  The RHS of Equation \ref{eq:Intial} is $ \tilde{F}(\alpha,\omega,c)$, so the theorem is proved.
% \end{proof}



Since we want to analyze a Hopf bifurcation, we will want to solve $\tilde{F}_\epsilon = 0$ for small values of~$\epsilon$. 
However, at the bifurcation point, $ D \tilde{F}_0(\pp  ,\pp , 0)$ is not invertible.
In order for our asymptotic analysis to be non-degenerate,
we work with a rescaled version of the problem. To this end, for any $\epsilon >0$, we rescale both $\tc$ and $\tilde{F}$ as follows. Let $\tc = \epsilon c$ and 
\begin{equation}\label{e:changeofvariables}
  \tilde{F}_\epsilon (\alpha,\omega,\epsilon c) = \epsilon F_\epsilon (\alpha,\omega,c).
\end{equation}
For $\epsilon>0$ the problem then reduces to finding zeros of 
\begin{equation}
\label{eq:FDefinition}
	F_\epsilon(\alpha,\omega, c) := 
	[i \omega + \alpha e^{-i \omega}] \e_1 + 
	( i \omega K^{-1} + \alpha U_{\omega}) c + 
	\epsilon \alpha e^{-i \omega} \e_2  +
	\alpha \epsilon L_\omega c + 
	\alpha \epsilon [ U_{\omega} c] * c.
\end{equation}
We denote the triple $(\alpha,\omega,c) \in \R^2 \times \ell^1_0$ by $x$.
To pinpoint the components of $x$ we use the projection operators
\[
   \pi_\alpha x = \alpha, \quad \pi_\omega x = \omega, \quad 
  \pi_c x = c \qquad\text{for any } x=(\alpha,\omega,c).
\]

After the change of variables~\eqref{e:changeofvariables} we now have an invertible Jacobian $D F_0(\pp  ,\pp , 0)$ at the bifurcation point.
On the other hand, for $\epsilon=0$ the zero finding problems for $\tilde{F}_\epsilon$ and $F_\epsilon$ are not equivalent. 
However, it follows from the following lemma that any nontrivial periodic solution having $ \epsilon=0$ must have a relatively large size when $ \alpha $ and $ \omega $ are close to the bifurcation point. 

\begin{lemma}\label{lem:Cone}
	Fix $ \epsilon \geq 0$ and $\alpha,\omega >0$. 
	Let
	\[
	b_* :=  \frac{\omega}{\alpha} - \frac{1}{2} - \epsilon  \left(\frac{2}{3}+ \frac{1}{2}\sqrt{2 + 2 |\omega-\pp| } \right).
	\]
Assume that $b_*> \sqrt{2} \epsilon$. 
Define
% \begin{equation*}%\label{e:zstar}
% 	z^{\pm}_* :=b_* \pm \sqrt{(b_*)^2- \epsilon^2 } .
% \end{equation*}
% \note[J]{Proposed change to match Lemma E.4}
\begin{equation}\label{e:zstar}
z^{\pm}_* :=b_* \pm \sqrt{(b_*)^2- 2 \epsilon^2 } .
\end{equation}
If there exists a $\tc \in \ell^1_0$ such that $\tilde{F}_\epsilon(\alpha, \omega,\tc) = 0$, then \\
\mbox{}\quad\textup{(a)} either $ \|\tc\| \leq  z_*^-$ or $ \|\tc\| \geq z_*^+  $.\\
\mbox{}\quad\textup{(b)} 
$ \| K^{-1} \tc \| \leq (2\epsilon^2+ \|\tc\|^2) / b_*$. 
\end{lemma}
\begin{proof}
	The proof follows from Lemmas~\ref{lem:gamma} and~\ref{lem:thecone} in Appendix~\ref{appendix:aprioribounds}, combined with the observation that
$\frac{\omega}{\alpha} - \gamma \geq b_*$,
% \[
%   \frac{\omega}{\alpha} - \gamma \geq b_*
%  \qquad\text{for all }
% | \alpha - \pp| \leq r_\alpha \text{ and } 
%   | \omega - \pp| \leq r_\omega.
% \]
with $\gamma$ as defined in Lemma~\ref{lem:gamma}.
\end{proof}

\begin{remark}\label{r:smalleps}
We note that for $\alpha < 2\omega$
\begin{alignat*}{1}
z^+_* &\geq   \frac{2 \omega - \alpha}{\alpha} 
- \epsilon \left(4/3+\sqrt{2 + 2 |\omega-\pp| } \, \right) + \cO(\epsilon^2)
\\[1mm]
z^-_* & \leq   \cO(\epsilon^2)
\end{alignat*}
for small $\epsilon$. 
Hence Lemma~\ref{lem:Cone} implies that for values of $(\alpha,\omega)$ near $(\pp,\pp)$ any solution has either $\|\tc\|$ of order 1 or $\|\tc\| =  \cO(\epsilon^2)$. 
The asymptotically small term bounding $z_*^-$ is explicitly calculated in Lemma~\ref{lem:ZminusBound}. 
A related consequence is that for $\epsilon=0$ there are no nontrivial solutions 
of $\tilde{F}_0(\alpha,\omega,\tc)=0$ with 
$\| \tc \| < \frac{2 \omega - \alpha}{\alpha} $. 
\end{remark}

\begin{remark}\label{r:rhobound}
In Section~\ref{s:contraction} we will work on subsets of $\ell^K_0$ of the form
\[
  \ell_\rho := \{ c \in \ell^K_0 : \|K^{-1} c\| \leq \rho \} .
\]
Part (b) of Lemma~\ref{lem:Cone} will be used in Section~\ref{s:global} to guarantee that we are not missing any solutions by considering $\ell_\rho$ (for some specific choice of $\rho$) rather than the full space $\ell^K_0$.
In particular, we infer from Remark~\ref{r:smalleps} that  small solutions (meaning roughly that $\|\tc\| \to 0$ as $\epsilon \to 0$)
satisfy $\| K^{-1} \tc \| = \cO(\epsilon^2)$.
\end{remark}

The following theorem guarantees that near the bifurcation point the problem of finding all periodic solutions is equivalent to considering the rescaled problem $F_\epsilon(\alpha,\omega,c)=0$.
\begin{theorem}
\label{thm:FourierEquivalence3}
\textup{(a)} Let $ \epsilon > 0$,  $c \in \ell^K_0$, $\alpha>0$ and $ \omega >0$. 
	Define $y: \R\to \R$ as 
\begin{equation}\label{e:yc}
	y(t) = 
	\epsilon \left( e^{i \omega t }  + e^{- i \omega t }\right) 
	+ \epsilon  \sum_{k = 2}^\infty   c_k e^{i \omega k t }  + c_k^* e^{- i \omega k t } .
\end{equation}
%	and suppose that $ y(t) > -1$. 
	Then $y(t)$ solves~\eqref{eq:Wright} if and only if $F_{\epsilon}( \alpha , \omega , c) = 0$.\\
\textup{(b)}
Let $y(t) \not\equiv 0$ be a periodic solution of~\eqref{eq:Wright} of period $2\pi/\omega$
 with Fourier coefficients $\c$.
Suppose $\alpha < 2\omega$ and $\| \c \| < \frac{2 \omega - \alpha}{\alpha} $.
Then, up to time translation, $y(t)$ is described by a Fourier series of the form~\eqref{e:yc} with $\epsilon > 0$ and $c \in \ell^K_0$.
\end{theorem}

\begin{proof}
Part (a) follows directly from Theorem~\ref{thm:FourierEquivalence2} and the  change of variables~\eqref{e:changeofvariables}.
To prove part (b) we need to exclude the possibility that there is a nontrivial solution with $\epsilon=0$. The asserted bound on the ratio of $\alpha$ and $\omega$ guarantees, by Lemma~\ref{lem:Cone} (see also Remark~\ref{r:smalleps}), that indeed $\epsilon>0$ for any nontrivial solution. 
\end{proof}

We note that in practice (see Section~\ref{s:global}) a bound on $\| \c \|$ is derived from a bound on $y$ or $y'$ using Parseval's identity.

\begin{remark}\label{r:cone}
It follows from Theorem~\ref{thm:FourierEquivalence3} and Remark~\ref{r:smalleps} that for values of $(\alpha,\omega)$ near $(\pp,\pp)$ any reasonably bounded solution satisfies $\| c\| =  O(\epsilon)$ as well as $\|K^{-1} c \| = O(\epsilon)$ asymptotically (as $\epsilon \to 0$).
These bounds will be made explicit (and non-asymptotic) for specific choices of the parameters in Section~\ref{s:global}.
\end{remark}

% We are able to rule out such large amplitude solutions using global estimates such as those in \cite{neumaier2014global}.
% Hence, near the bifurcation point, the problem of describing periodic solutions of~\eqref{eq:Wright} reduces to studying the family of zeros finding problems $F_\epsilon=0$.





%Specifically, if a solution having $ \epsilon = 0$ does in fact correspond to a nontrivial periodic solution and $\alpha  < 2\omega $, then $ \| \tilde{c} \| > 2 \omega \alpha^{-1} -1$. 
%%PERHAPS THIS NEEDS A FORMULATION AS A THEOREM AS WELL?
%%IN OTHER WORDS: ARE WE SURE WE HAVE FOUND ALL ZEROS OF $\tilde{F}_0$, I.E. ALL SOLUTIONS WITH $\epsilon=0$ NEAR THE BIFURCATION POINT? AFTER RESCALING THESE ARE INVISIBLE?
%%THERE SHOULD BE A STATEMENT ABOUT THIS SOMEWHERE! EITHER HERE OR SOME





We finish this section by defining a curve of approximate zeros $\bx_\epsilon$ of $F_\epsilon$ 
(see \cite{chow1977integral,hassard1981theory}). 
%(see \cite{chow1977integral,morris1976perturbative,hassard1981theory}). 


\begin{definition}\label{def:xepsilon}
Let
\begin{alignat*}{1}
	\balpha_\epsilon &:= \pp + \tfrac{\epsilon^2}{5} ( \tfrac{3\pi}{2} -1)  \\
	\bomega_\epsilon &:= \pp -  \tfrac{\epsilon^2}{5} \\
	\bc_\epsilon 	 &:= \left(\tfrac{2 - i}{5}\right) \epsilon \,  \e_2 \,.
\end{alignat*}
We define the approximate solution 
$ \bx_\epsilon := \left( \balpha_\epsilon , \bomega_\epsilon  , \bc_\epsilon \right)$
for all $\epsilon \geq 0$.
\end{definition}

We leave it to the reader to verify that both 
 $F_\epsilon(\pp,\pp,\bc_{\epsilon})=\cO(\epsilon^2)$ and $F_\epsilon(\bx_\epsilon)=\cO(\epsilon^2)$.
%%%	
%%%	
%%%	}{Better like this?}
%%%\annote[J]{ $F_\epsilon(\bx_0)=\cO(\epsilon^2)$ and $F_\epsilon(\bx_\epsilon)=\cO(\epsilon^2)$.}{I think we'd still need the $ \bar{c}_\epsilon$ term in $\bar{x}_0$ to be of order $ \epsilon$.}
%%%\remove[JB]{We show in Proposition A.1
%%%%\ref{prop:ApproximateSolutionWorks} 
%%% that any $ x \in \R^2 \times \ell^1_0$ which is $ \cO(\epsilon^2)$ close to $ \bar{x}_\epsilon $ will yield the estimate $F_\epsilon(x) = \cO(\epsilon^2)$.
%%%Hence choosing $\{ \pp , \pp, \bar{c}_\epsilon\}$ as our approximate solution would also have been a natural choice for performing an $\cO(\epsilon^2)$ analysis and would have simplified several of our calculations.
%%%However,} 
%%%
We choose to use the more accurate approximation 
for the $ \alpha$ and $ \omega $ components to improve our final quantitative results. 














%
% Values for $ (\alpha, \omega,c)$ which approximately solve $\tilde{F}(\alpha,\omega,c) = 0$  are computed in  \cite{chow1977integral,morris1976perturbative,hassard1981theory} and are as follows:
%  \begin{eqnarray}
%  \tilde{\alpha}( \epsilon) &:=& \pi /2 + \tfrac{\epsilon^2}{5} ( \tfrac{3\pi}{2} -1) \nonumber \\
%  \tilde{\omega}( \epsilon) &:=& \pi /2 -  \tfrac{\epsilon^2}{5} \label{eq:ScaleApprox} \\
%  \tc(\epsilon) 	  &:=& \{ \left(\tfrac{2 - i}{5}\right)  \epsilon^2 , 0,0, \dots \} \nonumber
%  \end{eqnarray}
% In Appendix \ref{sec:OperatorNorms} we illustrate an alternative method for deriving this approximation.
%
%
%
%
% We want to solve $ \tilde{F}(\alpha , \omega, \hat{c}) =0$ for small values of $ \epsilon$.
% However $ D \tilde{F}(\alpha , \omega , c)$ is not invertible at $ ( \pp , \pp , 0)$ when $ \epsilon = 0$.
% In order for our asymptotic analysis to be non-degenerate, we need to make the change of variables $ c \mapsto \epsilon c$.
% Under this change of variables, we define the function $ F$ below so that $ \tilde{F}(\alpha , \omega , \epsilon c) =\epsilon  F( \alpha , \omega , c)$.
%
%
%
% \begin{definition}
% Construct an $\epsilon$-parameterized family of densely defined functions  $F : \R^2 \oplus \ell^1 / \C \to \ell^1$ by:
% \begin{equation}
% \label{eq:FDefinition}
% 	F(\alpha,\omega, c) :=
% 	[i \omega + \alpha e^{-i \omega}]_1 +
% 	( i \omega K^{-1} + \alpha U_{\omega}) c +
% 	[\epsilon \alpha e^{-i \omega}]_2  +
% 	\alpha \epsilon L_\omega c +
% 	\alpha \epsilon [ U_{\omega} c] * c.
% \end{equation}
% \end{definition}

%%
%%
%%\begin{corollary}
%%	\label{thm:FourierEquivalence3}
%%	Fix $ \epsilon > 0$, and $ c \in \ell^1 / \C $, and $ \omega >0$. Define $y: \R\to \R$ as 
%%	\[
%%	y(t) = 
%%	\epsilon \left( e^{i \omega t }  + e^{- i \omega t }\right) 
%%	+  \epsilon  \left( \sum_{k = 2}^\infty   c_k e^{i \omega k t }  + \overline{c}_k e^{- i \omega k t } \right) 
%%	\]
%%	and suppose that $ y(t) > -1$. 
%%	Then $y(t)$ solves Wright's equation at parameter $ \alpha > 0 $ if and only if $ F( \alpha , \omega , c) = 0$ at parameter $ \epsilon$. 
%%	
%%	
%%	
%%\end{corollary}
%%
%%
%%\begin{proof}
%%	Since $ \tilde{F}(\alpha,\omega, \epsilon c) = \epsilon F( \alpha , \omega , c)$, the result follows from Theorem \ref{thm:FourierEquivalence2}.
%%\end{proof}

% If we can find $(\alpha , \omega, c)$ for which $ F( \alpha , \omega,c)=0$ at parameter $\epsilon$, then $ \tilde{F}(\alpha ,\omega, c)=0$.
% By Theorem \ref{thm:FourierEquivalence2} this amounts to finding a periodic solution to Wright's equation.
% Lastly, because we have performed the change of variables $ c \mapsto \epsilon c$, we need to  apply this change of variables to our approximate solution as well.
%
% \begin{definition}
% 	Define the approximate solution $ x( \epsilon) = \left\{ \alpha(\epsilon ) , \omega ( \epsilon ) , c(\epsilon) \right\}$ as below,  where $c(\epsilon) = \{ c_2( \epsilon) , 0 ,0 , \dots\} $.
% 	We may also write $ x_\epsilon = x(\epsilon) $.
% 	\begin{eqnarray}
% 	\alpha( \epsilon) &:=& \pi /2 + \tfrac{\epsilon^2}{5} ( \tfrac{3\pi}{2} -1) \nonumber \\
% 	\omega( \epsilon) &:=& \pi /2 -  \tfrac{\epsilon^2}{5} \label{eq:Approx} \\
% 	c_2(\epsilon) 	  &:=& \left(\tfrac{2 - i}{5}\right) \epsilon \nonumber
% 	\end{eqnarray}
%
% \end{definition}


%\section{Multiply transitive Galois groups and factor-sets}\label{Sec:MultiTrans}
%\input{multiple_transitivity}

\section{Explicit computations with Amitsur cohomology}\label{Sec:Amitsur}
In this section, we recall the basic definitions of Amitsur cohomology and then use them to get efficient representations of cocycles as introduced in \Cref{Sec:BrauerPrelim}. We also keep notations as they were in \Cref{Sec:BrauerPrelim}.

Although the Amitsur complex and its cohomology are defined for general commutative rings, we will only need to state the definitions over fields and étale algebras. We refer to \cite[Chapter 5]{ford2017separable} for more general definitions and results.

Unless specified otherwise, all tensor products are taken over \(k\), and by \(F^{\otimes n}\) we mean the tensor product of \(n\) copies of the \(k\)-algebra \(F\).

\begin{definition}
    A \(n\)-cochain in the sense of Amitsur is an invertible element in the \(k\)-algebra \(F^{\otimes n+1}\), and we write \(C_{Am}^n(k,F)\) for the group of \(n\)-cochains. 

    For \(n \in \N\) and \(0 \leq i \leq n\), we define a \(k\)-algebra homomorphism \(\varepsilon_i^n\) from \(F^{\otimes n+1}\) to \(F^{\otimes n+2}\). The map \(\varepsilon_i^n\) is defined on the simple tensors as follows:
    \[\varepsilon_i^n(f_0 \otimes \hdots \otimes f_n) = f_0 \otimes \hdots \otimes f_{i-1} \otimes 1 \otimes f_{i} \otimes \hdots \otimes f_n.\]
    We then define the group homomorphism
    \[\Delta^n\colon\begin{array}{ccl} C_{Am}^n(k,F) &\to &C_{Am}^{n+1}(k,F) \\ a &\mapsto &\prod_{i=0}^{n+1} \varepsilon_i^n(a)^{(-1)^i}\end{array}.\]

    Define the subgroup \(Z_{Am}^n(k,F) = \Ker(\Delta_{Am}^n)\) of \(C_{Am}^n(k,F)\), and its elements are called \(n\)-cocycles in the sense of Amitsur.

    If \(n\) is positive, we also define the subgroup \(B_{Am}^n(k,F) = \im(\Delta_{Am}^{n-1})\) of \(Z_{Am}^n(k,F)\) and its elements are \(n\)-coboundaries in the sense of Amitsur. Then, the \(n\)th Amitsur cohomology group \(H_{Am}^n(k,F)\) is defined as the quotient group \(Z_{Am}^n(k,F)/B_{Am}^n(k,F)\).

    Two \(n\)-cocycles \(c_1\) and \(c_2\) are called associated if they have the same class in \(H_{Am}^n(k,F)\).
\end{definition}

\subsection{Amitsur cohomology and central simple algebras}\label{Sec:AmitsurAlgebra}

    The first result we need to leverage Amitsur cohomology in representing Brauer factor sets is the following
    \begin{lemma}\label{lemma:AdamsonAmitsur}
        Consider the map \(\Psi_n\) from \(F^{\otimes n+1}\) to the \(k\)-algebra of \(G\)-homogeneous maps from \(\Phi^{n+1}\) to \(K\), which is defined over simple tensors by
        \[f_0 \otimes \hdots \otimes f_n \mapsto \left(\begin{array}{ccl} \Phi^{n+1} &\to &K^\times \\ (\varphi_{i_0},\hdots,\varphi_{i_n}) &\mapsto &\prod_{j=0}^n \varphi_{i_j}(f_j)\end{array}\right).\]
        The map \(\Psi\) is an isomorphism of \(k\)-algebras, which also sends the unit group \(C_{Am}^n(k,F)\) onto the unit group \(C^n(k,F)\). Furthermore, \(\Psi_{n+1} \circ \Delta_{Am}^n = \Delta^n \circ \Psi_{n}\), and it follows that cocycle and coboundaries subgroup are also isomorphic to one another, and so are cohomology groups.
    \end{lemma}

    \begin{proof}
        If \(F\) is a separable field extension of \(k\), this is the content of \cite[Section 2]{rosenberg1960amitsur}. However, the proof given there readily generalises to the case that \(F\) is an etale \(k\)-algebra.
    \end{proof}

    By the universal property of the tensor products, the \(k\)-algebra maps from \(F^{\otimes n+1}\) to \(K\) are in natural bijection with \(\Phi^{n+1}\). More precisely, if \((\varphi_{i_0},\hdots,\varphi_{i_n})\) and \(f_0,\hdots,f_n \in F^{\otimes n+1}\) is a simple tensor element, then
    \[(\varphi_{i_0},\hdots,\varphi_{i_n})(f_0 \otimes \hdots \otimes f_n) = \prod_{j=0}^n \varphi_{i_j}(f_j).\]

    By the construction of \(\Psi\), it is clear that if \(a \in F^{\otimes n+1}\) and \((\varphi_{i_0},\hdots,\varphi_{i_n})\), then
    \begin{equation}\label{eq:AmitsurEmbeddings}
        \Psi_n(c)_{i_0,\hdots,i_n} = (\varphi_{i_0},\hdots,\varphi_{i_n})(c).
    \end{equation}

    \begin{definition}
        A reduced \(n\)-cocycle in the sense of Amitsur is a cocycle \(c \in Z_{Am}^n(k,F)\) such that \((\varphi_i,\hdots,\varphi_i)(c) = 1\) for all \(i \in [d]\). Equivalently, it is a cocycle \(c\) such that \(\Psi_n(c)\) is reduced.
    \end{definition}

    Adapting \Cref{def:BrauerCSA} to Amitsur cohomology, we get:
    \begin{definition}\label{def:AmitsurCSA}
        Let \(c \in Z_{Am}^2(k,F)\) be a reduced cocycle. Then the \(k\)-algebra \(A(F,c)\) is defined as follows:

        As a \(k\)-vector space, \(A(F,c)\) is \(F^{\otimes 2}\). We see \(F^{\otimes 3}\) as a \(F^{\otimes 2}\)-algebra via the embedding \(\varepsilon_1^2\colon F^{\otimes 2} \to F^{\otimes 3}\). Then, multiplication in \(A(F,c)\) is defined by
        \[a a' = \Tr_{F^{\otimes 3}/F^{\otimes 2}}\left(\varepsilon_2^2(a)c\varepsilon_0^2(a')\right).\]
        In the sequel, we use the same embedding whenever we write \(\Tr_{F^{\otimes 3}/F^{\otimes 2}}\).
    \end{definition}

    \begin{prop}
        Let \(c \in Z_{Am}^2(k,F)\). The map \(\Psi_1\) is a \(k\)-algebra isomorphism from \(A(F,c)\) to \(B(F,\Psi_2(c)\).
    \end{prop}

    \begin{proof}
        The map \(\Psi_1\) is already known to be an isomorphism of vector space between \(A(F,c)\) and \(B(F,\Psi_2(c))\), as the underlying vector spaces of these algebras are respectively \(F^{\otimes 2}\) and the space of \(G\)-homogeneous maps from \(\Phi^2\) to \(K\). We therefore only need to check that the map \(\Psi_1\) is a homomorphism of \(k\)-algebras.
        
        By definition of the trace as the sum of the conjugates of an element, if \(a \in F^{\otimes 3}\) and \(i,j \in [d]\),
        \[(\varphi_i,\varphi_j)\left(\Tr_{F^{\otimes 3}/F^{\otimes 2}}(a)\right) = \sum_{k \in [d]} (\varphi_i,\varphi_k,\varphi_j)(a).\]

        Using \Cref{eq:AmitsurEmbeddings}, we get that if \(a,a' \in A(F,c)\) and \(i,j \in [d]\),
        \begin{align*}
            \Psi_1(aa')_{i,j} &= (\varphi_i,\varphi_j)\left( \Tr_{F^{\otimes 3}/F^{\otimes 2}}\left(\varepsilon_2^1(a)c\varepsilon_0^1(a')\right)\right)\\
            &= \sum_{k \in [d]}(\varphi_i,\varphi_k,\varphi_j)\left(\varepsilon_2^1(a)c\varepsilon_0^1(a')\right) \\
            &= \sum_{k \in [d]}(\varphi_i,\varphi_k,\varphi_j)(\varepsilon_2^1(a))(\varphi_i,\varphi_k,\varphi_j)(c)(\varphi_i,\varphi_k,\varphi_j)(\varepsilon_0^1(a')) \\
            &= \sum_{k \in [d]} \Psi_1(a)_{i,k} \Psi_2(c)_{i,k,j} \Psi_1(a')_{k,j} \\
            &= \Psi_1(a) \Psi_1(a')
        \end{align*}
    \end{proof}
    
    \begin{theorem}\label{thm:IsomAmiAlgebra}
        Let \(A\) be a central simple \(k\) algebra, let \(F \subset A\) be a separable maximal commutative subalgebra, and let \(v \in A\) be such that \(A = FvF\). Then, there is a cocycle \(c \in Z^2_{Am}(k,F)\) such that the map
        \[\begin{array}{rlcl}
            \Phi_c\colon &A(F,c) &\to &A \\
            &r_1 \otimes r_2 &\mapsto &r_1 v r_2
        \end{array}\]
        is an isomorphism. Furthermore, \(c\) is the unique element of \(F^{\otimes 3}\) with this property (if we extend the definition of the multiplication in \(A(F,c)\) to non-cocycle elements).
    \end{theorem}

    \begin{proof}
        The cocycle \(c\) is constructed as a preimage through \(\Psi_2\) of the Brauer factor-set obtained from \(A = FvF\) as in the discussion at the end of \(\Cref{Sec:BrauerPrelim}\). Then, the isomorphism given in the theorem statement is simply the isomorphism \(B(K,\Psi_2(c)) \simeq A\) pulled back through \(\Psi_1\).
        
        The cocycle \(c\) is a solution to the equation
        \[\Tr_{F^{\otimes 3}/F^{\otimes 2}} \left(\varepsilon_2^1(\Phi_c^{-1}(a))c\varepsilon_0^1(\Phi_c^{-1}(a'))\right) = \Phi_c^{-1}(aa')\]
        for all \(a,a' \in A)\). Observe that \(\varepsilon_2^1(\Phi_c^{-1}(a))\varepsilon_0^1(\Phi_c^{-1}(a'))\) covers \(F^{\otimes 3}\) as \(a,a'\) range over \(A\), and then the uniqueness of \(c\) follows from the non-degeneracy of the trace for étale algebras.
    \end{proof}

    \begin{cor}\label{cor:AssoCocyIsomAmitsur}
        Let \(c,c' \in Z_{Am}^2(k,F)\) be reduced associated cocycles, and let \(a \in C_{am}^2(k,F)\) be such that \(c' = c\Delta_{Am}^1(a)\). Then, the map \(m \mapsto ma^{-1}\), where the multiplication used is the natural one in \(F^{\otimes 2}\), is an isomorphism from \(A(F,c)\) to \(A(F,c')\).
    \end{cor}

    \begin{proof}
        This is just a consequence of the analogous property for Brauer factor sets (\Cref{thm:BrauerFSAssociated}), pulled through the isomorphisms \(\Psi_i\).
    \end{proof}

    \begin{remark}
        For the sake of efficiency, we use already proven results on Brauer factor-sets and the isomorphisms \(\Psi_1\) and \(\Psi_2\) to describe the isomorphism between the groups \(H^2_{Am}(k,F)\) and \(\Br(F/k)\). However, the simplicity of the isomorphism \(\Phi_c\) described in \Cref{thm:IsomAmiAlgebra} suggests that Amitsur cohomology is a natural setting for giving an algebraic description of \(\Br(F/k)\) and that direct proofs of these facts wy be given without using references to Brauer factor-sets.
    \end{remark}

    \begin{example}\label{ex:AmiTrivialCocycle}
        Just as in \Cref{ex:ConstantFS}, the cocycle \(1 \in Z_{Am}^2(k,F)\) corresponds to a split algebra. That is, \(A(F,1) \simeq M_d(k)\). Indeed, we observe that \(\varepsilon_1^0(F)\) is a left ideal of dimension \(d\). Since \(F\) has dimension \(d\), we only need to prove that \(\varepsilon_1^0(F)\) is a left ideal. This is a consequence of the fact that \(\varepsilon_0^1(\varepsilon_0^0(F)) \subset \varepsilon_2^1(F^\otimes 2)\) (as is easily observed on simple tensor elements). Then, the multiplication formula ensures that the product \(x\varepsilon_1^0(y)\) for any \(x \in A(F,1)\) and \(y \in F\) lies in \(T_{F^{\otimes 3}/F^{\otimes 2}}(\varepsilon_2^1(F^{\otimes 2})) \subset \varepsilon_1^0(F)\).
    \end{example}

\subsection{Computational representation of Amitsur cohomology}\label{Sec:CompRepAmi}
    Once a field \(k\) and an étale \(k\)-algebra \(F = k[X]/P\) are fixed, we have isomorphisms \[R_n \coloneqq k[X_0,\hdots,X_n]/(P(X_0),\hdots,P(X_n)) \simeq F^{\otimes n+1},\] for all integers \(n \geq 1\). Therefore, an element of \(F^{\otimes n+1}\), and in particular of \(C_{Am}^n(k,F)\), may be represented uniquely as a polynomial \(R(X_0,\hdots,X_n)\) in \(k[X_0,\hdots,X_n]\) whose individual degrees respectively in \(X_0,X_1,\hdots,X_n\) are bounded by \(d-1\). 

    In this setting, the map \(\varepsilon_i^n\) simply translates as the map sending \(Q(X_0,\hdots,X_n)\) to \(Q(X_0,\hdots,X_{i-1},X_{i+1},\hdots,X_n)\). That is, 
    \[\varepsilon_i^n(X_j) = \begin{cases}
        X_j \text{ if } j < i \\
        X_{j+1} \text{ otherwise.}
    \end{cases}
    \]

    Now, the trace of \(R_2\) as an \(R_1\) algebra (via the morphism \(\varepsilon_1^1\)) may easily be computed in the \(R_1\)-basis \((X_1^i)_{0 \leq i < d}\) of \(R_2\). It follows that if \(Q_1,Q_2 \in R_1\) represent elements \(a_1,a_2\) of \(A(F,c)\), where we see \(c\) as an element of \(R_2\), then the element \(Q \in R_2\) representing the product \(a_1a_2\) may be computed practically as:
    \[Q(X_0,X_1) = \Tr_{R_2/R_1}\left(Q_1(X_0,X_1)c(X_0,X_1,X_2)Q_2(X_1,X_2)\right).\]

    
    Recall that \(\varphi_i\) is a map from \(F = k[X]/P\) to \(K\). The map \((\varphi_i,\varphi_i,\varphi_i)\colon F^{\otimes 3} \to K\) factors through \(F\) via
    \[\begin{array}{ccccl} F^{\otimes 3} &\to &F &\to &K \\
        Q(X_0,X_1,X_2) &\mapsto &Q(X,X,X) &\mapsto &Q(\varphi_i(X),\varphi_i(X),\varphi_i(X)) \end{array}\]

    We may describe the explicit computation of a reduced cocycle associated to a given cocycle:
    \begin{theorem}\label{thm:ReducingCocycleExplicit}
        Let \(c \in Z_{Am}^2(k,F)\) be represented by a polynomial \(Q\) as described above. Let \(c' \in F^{\otimes 3}\) be represented by
        \[Q' = \frac{Q(X_0,X_1,X_2)}{Q(X_1,X_1,X_1)}.\]
        Then, \(c'\) is a reduced cocycle associated to \(c\), and the map
        \[\begin{array}{rlcl}
        \rho\colon &A(F,c) &\to &A(F,c') \\
        &R(X_0,X_1) &\mapsto &Q(X_0,X_0,X_0) R(X_0,X_1)
        \end{array}\]
        is an isomorphism of \(k\)-algebras.
    \end{theorem}

    \begin{proof}
        As \(c\) is invertible in \(F^{\otimes 3}\), \(Q(X_0,X_0,X_0)\) is invertible in \(F[X_0]/(P(X_0))\), so the map given is an isomorphism of vector spaces, and we only need to check that it is compatible with multiplication.

        We let \(R(X_0,X_1),R'(X_0,X_1) \in k[X_0,X_1]/(P(X_0),P(X_1))\) and we compute:
        \begin{align*}
            \rho(R) \rho(R') &= \Tr_{F^{\otimes 3}/F^{\otimes 2}}(\varepsilon_2^1(Q(X_0,X_0,X_0)R)Q(X_1,X_1,X_1)^{-1}c\\
            &\varepsilon_0^1(Q(X_0,X_0,X_0)R')) \\
            &= \Tr_{F^{\otimes 3}/F^{\otimes 2}}\left(Q(X_0,X_0,X_0)R(X_0,X_1)Q(X_1,X_1,X_1)^{-1}c\right.\\
            &\left.Q(X_1,X_1,X_1)R'(X_1,X_2)\right)\\
            &= Q(X_0,X_0,X_0) Tr_{F^{\otimes 3}/F^{\otimes 2}}\left(R(X_0,X_1)cR'(X_1,X_2)\right)\\
            &= Q(X_0,X_0,X_0) RR' \\
            &= \rho(RR')
        \end{align*}
    \end{proof}
\subsection{Computing a cocycle representing a given algebra}\label{Sec:AmitsurAlgebraConstruction}

    We may now give an algorithm for finding a representation of a central simple algebra \(A\) over any field of sufficiently large size where linear algebra tasks may be performed efficiently.

    \begin{algorithm}
        \caption{Computing a \(2\)-cocycle representing a given central simple algebra}
        \label{algo:FindCocycle}
        \begin{algorithmic}[1]
            \REQUIRE A field \(k\)
            \REQUIRE A central simple \(k\)-algebra \(A\) such that \(|k| > \dim_k A\).
            \STATE Find \(u \in A\) such that \(F \coloneqq k[u]\) is a maximal separable commutative subalgebra of \(A\) \label{algoline:Findu}
            \STATE Compute \(P\), the minimal polynomial of \(u\)
            \STATE Find \(v \in A\) such that \(A = FvF\) \label{algoline:Findv}
            \STATE Compute the matrix of the isomorphism \(e: F^{\otimes 2} \to A\) sending \(f_1 \otimes f_2\) to \(f_1vf_2\)
            \STATE Using linear algebra, find \(c \in F^{\otimes 3}\) such that for all \(a,b \in F^{\otimes 2}\), \(e(a)a(b) = \Tr_{F^{\otimes 3}/F^{\otimes 2}}(\varepsilon_2^1(a)c\varepsilon_0^1(b))\). \label{algoline:Findc}
            \STATE Compute \(c' = c/c(X_1,X_1,X_1)\)
            \STATE Compute \(e'\) as \(e\) precomposed with \(R \mapsto c(X_0,X_0,X_0)^{-1} R\). 
            \RETURN \((u,c',e')\)
        \end{algorithmic}
    \end{algorithm}

    Before we prove the correctness efficiency of \cref{algo:FindCocycle}, we need a lemma:

    \begin{lemma}\label{lemma:Findv}
        Let \(k\) be a field and let \(A\) be a central simple \(k\)-algebra. Assume that \(|k| > \dim_k A\). Let \(u \in A\) be such that \(F \coloneqq k[u]\) is a maximal commutative subalgebra of \(A\). Then an element \(v \in A\) such that \(A = FvF\) may be found in probabilistic polynomial time.
    \end{lemma}

    \begin{proof}
        For \(v\) in \(A\), by an argument of dimensions over \(k\), we observe that \(A = FvF\) if and only if the map
        \[e\colon\begin{array}{ccl} F \otimes F &\to &A \\ f_1 \otimes f_2 &\mapsto &f_1vf_2 \end{array}\]
        is injective.

        Fixing the basis \((u^i \otimes u^j)_{0 \leq i,j \leq \deg A - 1}\) and any basis \(B = (b_1,\hdots,b_{\dim A})\) of \(A\), the determinant of \(e\) is a homogeneous polynomial on the coordinates of \(v\) with respect to \(B\), and \(A = FvF\) if and only if \(v\) is not a zero of this polynomial.

        Letting \(S\) be a finite subset of \(k\), the Schwartz-Zippel lemma ensures that a random \(v\) in \(Sb_1 \oplus \hdots \oplus Sb_{\dim A}\) satisfies this condition with probability larger than \(1 - \frac{\dim A}{|S|}\).

        Therefore, if \(|k| > \dim A\), we may pick \(S\) large enough that \(v\) has the desired property with positive probability. For instance, take \(|S| = 2\dim A\) and \(v\) has the desired property with probability larger than \(\frac{1}{2}\).
    \end{proof}
    
    \begin{theorem}\label{thm:AlgoFindCocycle}
        If \(k\) is a field over which linear algebra may be performed efficiently, and \(A\) is a central simple \(k\)-algebra such that \(|k| > \dim_k A\), then \Cref{algo:FindCocycle} returns \(u \in A\), a cocycle \(c \in Z_{Am}^2(k,k(u))\) and an isomorphism \(e\colon A(F,c) \to A\) in probabilistic polynomial time.
    \end{theorem}

    \begin{proof}
         We first prove the correctness of the algorithm. The first 3 steps compute \(u,v \in A\) as discussed above \Cref{thm:IsomAmiAlgebra}. We obtain a subalgebra \(F\) of \(A\) and \(v \in A\) such that \(A = FvF\). Then, by \Cref{thm:IsomAmiAlgebra}, there exists a cocycle \(c \in Z_{Am}^2(k,F)\) such that \(a \otimes b \mapsto avb\) is an isomorphism from \(A(k,F)\) to \(A\), and such a \(c\) is unique in \(F^{\otimes 3}\). The equation solved in \Cref{algoline:Findc} finds this \(c\).

        Finally, the last two lines perform the operation described in \Cref{thm:ReducingCocycleExplicit} for computing a reduced cocycle associated to \(c\), and the isomorphism from \(A(F,c)\) to \(A(F,c')\). Putting everything together, \cref{algo:FindCocycle} \(c\) is indeed a reduced \(2\)-Amitsur cocycle for \(F = k(u)\) and \(e\) gives an isomorphism \(A(F,c) \simeq A\).

        Now we consider the complexity of the algorithm:
        
        The element \(u \in A\) in \Cref{algoline:Findu} may be found using the polynomial algorithm given in \cite{graaf2000finding}.

        Then, \(v\) in \Cref{algoline:Findv} may be found in probabilistic polynomial time using the algorithm of \Cref{lemma:Findv} below.

        Then, the remaining lines involve arithmetic in \(A\) and bounded tensor powers of \(F\), as well as the computation of the solution of a linear system.
        
        All in all, this makes \Cref{algo:FindCocycle} a polynomial probabilistic algorithm.
    \end{proof}


\section{Trivialisation of Amitsur cocycles}\label{Sec:Trivialisation}
In this section, we present an algorithm for computing the trivialisation of a coboundary using \(S\)-units group computation. This result is inspired by results such as Simon's algorithm for solving norm equation in cyclic extensions \cite{simon2002solving} and Fieker's result on finding trivialisation of Galois coboundaries in groups of \(S\)-units \cite[Theorem 7]{fieker2009minimizing}.

Our strategy is to prove a vanishing lemma for the first Amitsur cohomology group with coefficients in the divisor group. Such a result is analogous to \cite[Lemma 7]{fieker2009minimizing} and allows us to adapt the proof strategy to our setting.

Let \(k\) be a number field. We let the divisor group \(\Dc(k)\) be the free Abelian group on the set \(\Pl(k)\) of finite places of \(k\). The group \(\Pc(k)\) of principal divisors is the image of the natural injection of \(k\) into \(\Dc(k)\), and it is well known that the quotient \(\Cl_k = \Dc(k)/\Pc(k)\) is finite.

We extend these definition to étale algebras over number fields:
\begin{definition}\label{def:EtaleDivisors}
    Let \(A\) be an étale \(k\)-algebra, for a number field \(k\). Then \(A\) splits as a direct sum
    \[A \simeq \bigoplus_{i \in I} K_i\]
    of finite separable extensions of \(k\). Furthermore, this decomposition is unique up to reordering and internal isomorphism of the fields. We define \(\Dc(F)\) as the free Abelian group over \(\Pl(F)\), the disjoint union of the sets \(\Pl(K_i)\).
\end{definition}

    While \(\Dc\) is not a functor, we may define \(\Dc(f)\) whenever \(f\) is a monomorphism in the category of étale \(k\)-algebras. Indeed, if \(f\colon A \to A'\), for each place of \(A\) belonging to a direct factor \(K\), the map \(f\) sends \(K\) to one of the direct factors \(K'\) of \(A'\) and we may then define \(f(P) \in \Dc(K') \subset \Dc(A')\) as in the case of field extensions.
    
    This yields a complex
    \[\hdots \to \Dc(F^{\otimes n+1}) \xrightarrow{\Dc(\Delta_{Am}^{n})} \Dc(F^{\otimes n+2}) \to \hdots.\]
of abelian groups, where we define \(\Dc(\Delta_{Am}^n) = \sum_{0 \leq i \leq n} (-1)^i \Dc(\epsilon^n_i)\), using the fact that the \(\epsilon\) maps are injective.

For the remainder of this section, we let \(F\) be an étale \(k\)-algebra.

We give a precise description of the map \(\Dc(\Delta_{Am}^n)\). Let \(Q \in \Pl(F^{\otimes n+2})\). Then, for any \(0 \leq i \leq n+1\), there is exactly one \(P \in \Pl(F^{\otimes n+1})\) such that \(Q \mid \epsilon_i^n(P)\), and we call this place \(Q_i\). Then, if \(P \in \Pl(F^{\otimes n+1})\), we get
\[\Dc(\epsilon_i^n)(P) = \sum_{\substack Q \in \Pl(F^{\otimes n+2}) \\ Q_i = P} e_{Q,i} Q,\]
where \(e_{Q,i}\) is the ramification index of \(Q\) over \(\epsilon_i^n\colon F^{\otimes n+1} \to F^{\otimes n+2}\), and it follows that 
\[\Dc(\epsilon_i^n)\left(\sum_{P \in \Pl(F^{\otimes n+1})} n_P P\right) = \sum_{Q \in \Pl(F^{\otimes n+2})} e_{Q,i} n_{Q_i} Q\]
and
\[\Dc(\Delta_{Am}^n)\left(\sum_{P \in \Pl(F^{\otimes n+1})} n_P P\right) = \sum_{Q \in \Pl(F^{\otimes n+2})} \left(\sum_{0 \leq i \leq {n+1}} (-1)^i e_{Q,i} n_{Q_i} \right) Q.\]

We first need two lemmas:

\begin{lemma}\label{lemma:CocycleTransit}
    Let \(Q,Q'\) be places of \(F^{\otimes 2}\) such that \(Q_0 = Q'_0 = P\). Then there exists a place \(R \in \Pl(F^{\otimes 3})\) such that \(R_1 = Q\) and \(R_2 = Q'\).
\end{lemma}

\begin{proof}
    We set \(P = Q_0 = Q'_0\), and consider the local field \(F_P\) obtained by first selecting the direct factor of \(F\) to which \(P\) belongs and then taking the completion at \(P\). The set of places of \(F^{\otimes 2}\) that divide \(\epsilon_1^0(P)\) is in bijection with the set of direct factors of \(F_P \otimes_F F^{\otimes 2} \simeq F_P \otimes_k F\). Likewise, the set of places of \(F^{\otimes 3}\) above \(P\) (for the embedding \(\epsilon_1^1 \circ \epsilon_1^0 = Id_F \otimes 1 \otimes 1\)) is in bijection with the set of direct factors of \(F_P \otimes_k F \otimes_k F\). Furthermore, the maps \(\epsilon_1^1\) and \(\epsilon_1^2\) may be expanded to \(F_P \otimes_k F\), and if \(R \in \Pl(F^{\otimes 3})\) is the place corresponding to some direct factor \(K\) of \(F_P \otimes F \otimes F\), then, the place \(R_1\) (resp. \(R_2\)) corresponds to the factor of \(F_P \otimes F\) which is mapped into \(K\) by \(\epsilon_1^1\) (resp. \(\epsilon_1^2\)).

    Now, let \(\alpha \in k[X]\) be a defining polynomial for \(F\). We have isomorphisms \(F_P \otimes F \simeq F_P[X]/(\alpha(X))\) and \(F_P \otimes F \otimes F \simeq F_P[X,Y]/(\alpha(X),\alpha(Y))\). A direct factor of \(F_P \otimes F\) corresponds to an irreducible factor of \(\alpha\) in \(F_P[X]\). Likewise, a factor of \(F_P \otimes F \otimes F\) is uniquely described by the choice of an irreducible factor \(\beta\) of \(\alpha(X)\) in \(F_P[X]\) and an irreducible factor \(\gamma\) of \(\alpha(Y)\) in \(\left(F_P[X]/\beta(X)\right)[Y]\). We note that each factor \(\gamma\) is obtained as a factor of some \(\beta'\), itself an irreducible factor of \(\alpha\) in \(F_P[Y]\).

    Furthermore, if \(R \in \Pl(F^{\otimes 3})\) corresponds to some factors \(\beta\) and \(\gamma\) of \(\alpha\), then \(R_2\) is the place of \(F^{\otimes 2}\) corresponding to \(\beta\) and \(R_1\) is the place of \(F^{\otimes 2}\) corresponding to the irreducible factor \(\beta'\) of \(\alpha\) in \(\F_P[X]\) corresponding to \(\gamma\) as described above.

    Now, we fix \(\beta\) and \(\beta'\) the irreducible factors of \(\alpha\) corresponding respectively to \(Q\) and \(Q'\). We let \(\gamma\) be an irreducible factor of \(\beta'(Y)\) in \(\left(F_P[X]/\beta(X)\right)[Y]\) and the place \(R\) corresponding to \(\beta\) and \(\gamma\) is as demanded.
\end{proof}

We may now prove a generalized version of Hilbert's theorem 90 in our setting. We say that a place \(P \in \Pl(F)\) is unramified if, for all \(Q \in \Pl(Q)\), if \(F = Q_0\) then \(e_{Q,0} = 1\) and if \(F = Q_1\) then \(e_{Q,1} = 1\).
\begin{lemma}\label{lemma:Hilbert90}
    Let \(D = \sum_{Q \in \Pl(F^{\otimes 2})} n_Q Q \in \Ker \Dc(\Delta_{Am}^1)\) be supported by unramified places. That is, for all \(Q \in \Pl(F^{\otimes 2})\), if \(Q_0\) or \(Q_1\) is ramified, then \(n_Q = 0\). Then, there exists \(E \in \Dc(F)\) such that \(D = \Dc(\Delta_{Am}^0)(E)\).
\end{lemma}

\begin{proof}
    We set
    \[E = \sum_{P \in \Pl(F)} \left(\min_{\substack{Q \in \Pl(F^{\otimes 2}) \\ Q_0 = P}} n_Q \right) P.\]
    Then, we get 
    \[\Dc(\epsilon_0^0)(E) = \sum_{Q \in \Pl(F^{\otimes 2})} \left(\min_{\substack{Q' \in \Pl(F^{\otimes 2}) \\ Q'_0 = Q_1}}  n_{Q'}\right) Q\]
    and
    \[\Dc(\epsilon_1^0)(E) = \sum_{Q \in \Pl(F^{\otimes 2})} \left(\min_{\substack{Q' \in \Pl(F^{\otimes 2}) \\ Q'_0 = Q_0}}  n_{Q'}\right) Q\]

    It follows that
    \[D + \Dc(\epsilon_0^0)(E) = \sum_{Q \in \Pl(F^{\otimes 2})} \left(\min_{\substack{Q' \in \Pl(F^{\otimes 2}) \\ Q'_0 = Q_1}}  n_Q + n_{Q'}\right) Q.\]
    If we fix places \(Q,Q' \in \Pl(F^{\otimes 2})\) such that \(Q'_0 = Q_1\), we apply \cref{lemma:CocycleTransit} to \(Q'\) and \(Q^\sigma\), the image of \(Q\) by the automorphism \(\sigma\colon a \otimes b \mapsto b \otimes a\) of \(F^{\otimes 2}\). We find that there exists \(R \in \Pl(F^{\otimes 3})\) such that \(R_1 = Q'\) and \(R_2 = Q^\sigma\). Then, consider \(R^\tau\), the image of \(R\) by the automorphism \(\tau \colon a \otimes b \otimes c \mapsto b \otimes a \otimes c\) of \(F^{\otimes 3}\). We get \(R^\tau_2 = Q\) and \(R^\tau_0 = Q'\). We may then set \(Q'' = R^\tau_1\) and, as \(\Dc(\Delta_{Am}^1)(D) = 0\), we get that \(n_Q + n_{Q'} = n_{Q''}\), for some \(Q''\) such that \(Q''_0 = Q_0\).

    Conversely, if we fix \(Q,Q'' \in \Pl(F^{\otimes 2})\) such that \(Q_0 = Q''_0\), then there exists \(R \in \Pl(F^{\otimes 3})\) such that \(R_2 = Q\) and \(R_1 = Q''\). We set \(Q' = R_0\) and we get that \(n_Q + n_{Q'} = n_{Q''}\), and observe that \(Q'_0 = Q_1\).

    This shows that for \(Q \in \Pl(F^{\otimes 2})\), \[\min_{\substack{Q' \in \Pl(F^{\otimes 2}) \\ Q'_0 = Q_1}}  n_Q + n_{Q'} = \min_{\substack{Q' \in \Pl(F^{\otimes 2}) \\ Q'_0 = Q_0}}  n_{Q'}.\]
    Therefore, \(D + \epsilon_0^0(E) = \epsilon_1^0(E)\). That is, \(D = \Dc(\Delta_{Am}^0)(-E)\).
\end{proof}

In what follows, if \(S\) is a set of places of \(F\), we define inductively \(S^{(1)} = S\) and
\[S^{(i+1)} = \left\{Q \in \Pl(F^{\otimes i+1}) \mid \exists 0 \leq j \leq i, P \in S^{(i)} \colon Q|\varepsilon_j^{i-1}(P)\right\}.\]
We now get our main theorem for this section:
\begin{theorem}\label{thm:SUnitTriv}
    Let \(b \in B_{Am}^2(k,F)\) be a coboundary. Let \(S\) be a finite set of places of \(F\) such that:
    \begin{itemize}
        \item \(S\) contains the infinite places of \(F\).
        \item \(S\) contains the places of \(F\) that ramify in \(F^{\otimes 2}\).
        \item The finite places of \(S\) generate the class group \(\Cl(F)\).
        \item The places in the support of \(b\) are contained in \(S^{(3)}\).
    \end{itemize}
    Then there exists a cochain \(\sigma\) in the group of \(S^{(2)}\)-units of \(F^{\otimes 2}\) such that \(b = \Delta_{Am}^1(\sigma)\)
\end{theorem}

\begin{proof}
 Let \(\alpha \in (F^{\otimes 2})^\times\) be such that \(\Delta_{Am}^1(\alpha) = b\). We consider the divisor \(D = \Dc(\alpha) = \sum_{Q \in \Pl(F^{\otimes 2})} n_Q Q\) of \(\alpha\). We set \(D_S = \sum_{Q \in S^{(2)}} n_Q Q\) and \(D_{\bar{S}} = \sum_{Q \notin S^{(2)}} n_Q Q\). Now, \(\Dc(\Delta_{Am}^1)(D)\) is the divisor of \(F^{\otimes 3}\) corresponding to \(b\) and therefore is supported by \(S^{(3)}\). Observe that if \(Q \in S^{(2)}\), then \(\Dc(\Delta_{Am}^1)(Q)\) has support in \(S^{(3)}\). It follows that \(\Dc(\Delta_{Am}^1)(D_{\bar{S}}) = 0\).

 The divisor \(D_{\bar{S}}\) has no ramified place of \(F^{\otimes 2}\) in its support. We may therefore apply \cref{lemma:Hilbert90} and get a divisor \(E \in \Dc(F)\) such that \(D_{\bar{S}} = \Delta_{Am}^0(E)\). Now, as \(S\) generates the class group of \(F\), there exists \(E' \in \Dc(F)\) with support in \(S\) and \(\gamma \in F^\times\) such that \(E = \Dc(\gamma) + E'\). Then, we get that
 \[\Dc(\Delta_{Am}^0)(\Dc(\gamma)) + \Dc(\Delta_{Am}^0)(E') = D - D_{\bar{S}}\]
 and therefore
 \[\Dc(\Delta_{Am}^0)(E') + D_{\bar{S}} = \Dc(\alpha \Delta_{Am}^0(\gamma^{-1})).\]
 Now, this shows that \(\alpha \Delta_{Am}^0(\gamma^{-1})\) is a \(S^{(2)}\)-unit. Furthermore, \[\Delta_{Am}^1(\alpha\Delta_{Am}^0(\gamma^{-1})) = \Delta_{Am}^1(\alpha) = b,\]
 and \(\alpha \Delta_{Am}^0(\gamma^{-1})\) is a cochain with the required properties.
\end{proof}

From \cref{thm:SUnitTriv} we directly get an algorithm for computing a trivialisation of a \(2\)-coboundary:

\begin{algorithm}
    \caption{Computing a trivialisation of a \(2\)-coboundary}
    \label{algo:TrivCobound}
    \begin{algorithmic}[1]
        \REQUIRE A number field \(k\), an étale \(k\)-algebra \(F\)
        \REQUIRE A coboundary \(b \in B_{Am}^2(k,F)\)
        \STATE Compute \(S_1\), the set of places of \(F\) that ramify in \(F^{(2)}\).
        \STATE Compute \(S_2\), a set of places of \(F\) which generate the class group \(\Cl(F)\).
        \STATE Compute the divisor of \(F^{\otimes 3}\) corresponding to \(b\). Let \(S_3\) be the set of places of \(F\) below the places in the support of \(b\).
        \STATE Set \(S = S_1 \cup S_2 \cup S_3\).
        \STATE Compute the sets \(S^{(2)}\) and \(S^{(3)}\).
        \STATE Compute an isomorphism \(\phi\) from the group of \(S^{(2)}\)-units of \(F^{\otimes 2}\) to \(\Z^r \oplus \Z/m\Z\).
        \STATE Compute an isomorphism \(\psi\) from the group of \(S^{(3)}\)-units of \(F^{\otimes 3}\) to \(\Z^{r'} \oplus \Z/m'\Z\).
        \STATE Solve the linear equation \((\psi \circ \Delta_{Am}^1 \circ \phi^{-1})(\alpha) = \psi(b)\)
        \RETURN \(\alpha\)
    \end{algorithmic}
\end{algorithm}

\begin{theorem}\label{thm:AlgoTrivCobound}
    Given a number field \(k\), and étale \(k\)-algebra \(F\) and a coboundary \(b \in B^2_{Am}(k,F)\), \cref{algo:TrivCobound} outputs a cochain \(\alpha \in C^1(k,F)\) such that \(\Delta_{Am}^1(\alpha) = b\). Furthermore,  \cref{algo:TrivCobound} runs in polynomial time on a quantum computer.
\end{theorem}

\begin{proof}
    Using a polynomial-time algorithm for factoring polynomials over number fields \cite{lenstra1983factoring}, one may compute splitting isomorphisms 
    \[F \simeq \bigoplus_\alpha F_\alpha,\]
    \[F^{\otimes 2} \simeq \bigoplus_\beta F^{(2)}_\beta,\]
    and
    \[F^{\otimes 3} \simeq \bigoplus_\gamma F^{(3)}_\gamma.\]

    Using these isomorphisms, the steps of the computation of \(S\) may be done over number field extensions. Then, this entails computing and factoring relative discriminants, computing class groups and factoring \(b\) in the ideal group of \(F^{\otimes 3}\). The sets \(S^{(2)}\) and \(S^{(3)}\) are computed by factoring images \(\epsilon_i^0(P)\) for \(P \in S\) and then \(\epsilon_i^1(Q)\) for \(Q \in S^{(2)}\). Then, isomorphisms \(\phi\) and \(\psi\) are computed using an algorithm for computing \(S\)-unit groups. Finally, the last step is the computation of a solution of an integral linear system.
    Each of these tasks can be accomplished in quantum polynomial time according to Theorem \ref{thm:s-unit&norm}. 
    The correctness of the algorithm relies on the fact that a cochain \(\alpha\) such that \(b = \Delta_{Am}^1(\alpha)\) exists and may be found in the group of \(S^{(2)}\)-units, which is the content of \cref{thm:SUnitTriv}.
\end{proof}

\begin{cor}\label{cor:QuantumSplit}
    There exists a polynomial quantum algorithm which, give a number field \(k\) and an algebra \(A \simeq M_n(k)\), computes an explicit algorithm from \(A\) to \(M_n(k)\).
\end{cor}

\begin{proof}
    This is simply a combination of \cref{thm:AlgoFindCocycle,thm:AlgoTrivCobound}. Indeed, using \Cref{algo:FindCocycle}, one may compute an étale \(k\) algebra \(F\) and a cocycle \(c \in \Z^2(k,F)\) representing \(A\). As \(A\) is isomorphic to \(M_n(k)\), the cocycle \(c\) is in fact a coboundary. Then, a cochain \(\alpha \in C^1(k,F)\) such that \(\Delta_{Am}^1(\alpha) = c\) may be computed using \cref{algo:TrivCobound}. Applying \Cref{cor:AssoCocyIsomAmitsur}, we obtain an explicit isomorphism \(A \simeq A(F,1)\). Finally, an isomorphism \(A(F,1) \simeq M_n(k)\) may easily be computed using the left ideal provided in \Cref{ex:AmiTrivialCocycle}.
\end{proof}
\begin{remark}
For quaternion algebras $(a,b)_K$ it is known that splitting is equivalent to solving the norm equation $N_{\mathbb{Q(\sqrt{a})}|\mathbb{Q}}(x)=b$ thus one can also directly apply Theorem \ref{thm:s-unit&norm} to find an explicit isomorphism to $M_2(K)$. A similar statement can be derived for degree three central simple algebras as there is a polynomial-time algorithm for finding a cyclic algebra presentation. 
\end{remark}

%\section{Finding trivialisations of Brauer factor-sets}\label{Sec:Trivialisation}
%\input{trivialisation}

%\section{Random elements of maximal orders of matrix algebras}\label{Sec:Random}
%\input{random}

%\section{The main algorithm}\label{Sec:Algo}
%%% This declares a command \Comment
%% The argument will be surrounded by /* ... */
\SetKwComment{Comment}{/* }{ */}

\begin{algorithm}[t]
\caption{Training Scheduler}\label{alg:TS}
% \KwData{$n \geq 0$}
% \KwResult{$y = x^n$}
\LinesNumbered
\KwIn{Training data $\mathcal{D}_{train}=\{(q_i, a_i, p_i^+)\}_{i=1}^m$, \\
\qquad \quad Iteration number $L$.}
\KwOut{A set of optimal model parameters.}

\For{$l=1,\cdots, L$}{
    Sample a batch of questions $Q^{(l)}$\\
    \For{$q_i\in Q^{(l)}$}{
        $\mathcal{P}_{i}^{(l)} \gets \mathrm{arg\,max}_{p_{i,j}}(\mathrm{sim}(q_i^{en},p_{i,j}),K)$\\
        $\mathcal{P}_{Gi}^{(l)} \gets \mathcal{P}_{i}^{(l)}\cup\{p^+_i\}$\\
        Compute $\mathcal{L}^i_{retriever}$, $\mathcal{L}^i_{postranker}$, $\mathcal{L}^i_{reader}$\\ according to Eq.\ref{eq:retriever}, Eq.\ref{eq:rerank}, Eq.\ref{eq:reader}\\
    }
    % $\mathcal{L}^{(l)}_{retriever} \gets \frac{1}{|Q^{(l)}|}\sum_i\mathcal{L}^i_{retriever}$\\
    % $\mathcal{L}^{(l)}_{retriever} \gets \mathrm{Avg}(\mathcal{L}^i_{retriever})$,
    % $\mathcal{L}^{(l)}_{rerank} \gets \mathrm{Avg}(\mathcal{L}^i_{rerank})$,
    % $\mathcal{L}^{(l)}_{reader} \gets \mathrm{Avg}(\mathcal{L}^i_{reader})$\\
    % Compute $\mathcal{L}^{(l)}_{retriever}$, $\mathcal{L}^{(l)}_{rerank}$, and $\mathcal{L}^{(l)}_{reader}$ by averaging over $Q^{(l)}$\\
    $\mathcal{L}^{(l)} \gets \frac{1}{|Q^{(l)}|}\sum_i(\mathcal{L}^{i}_{retriever} + \mathcal{L}^{i}_{postranker}+ \mathcal{L}^{i}_{reader})$\\
    $\mathcal{P}^{(l)}_K\gets\{\mathcal{P}^{(l)}_i|q_i\in Q^{(l)}\}$,\quad $\mathcal{P}^{(l)}_{KG}\gets\{\mathcal{P}^{(l)}_{Gi}|q_i\in Q^{(l)}\}$\\
    Compute the coefficient $v^{(l)}$ according to Eq.~\ref{eq:v}\\
  \eIf{$ v^{(l)}=1$}{
    $\mathcal{L}^{(l)}_{final} \gets \mathcal{L}^{(l)}(\mathcal{P}_{KG}^{(l)})$\\
  }{
      $\mathcal{L}^{(l)}_{final} \gets \mathcal{L}^{(l)}(\mathcal{P}^{(l)}_{K}),$\\
    }
    Optimize $\mathcal{L}^{(l)}_{final}$
}
\end{algorithm}


%  \eIf{$ \mathcal{L}^{(l-1)}_{retriever}<\lambda$}{
%     $\mathcal{L}^{(l)}_{final} \gets \mathcal{L}^{(l)}(\mathcal{P}_K^{(l)})$\\
%   }{
%       $\mathcal{L}^{(l)}_{final} \gets \mathcal{L}^{(l)}(\mathcal{P}^{(l)}_{KG}),$\\
%     }

\bibliographystyle{plain}
\bibliography{biblio}
\end{document}
