\documentclass{article}

\usepackage{amsfonts}
\usepackage{amsmath}
\usepackage{amsthm}
\usepackage{mathtools}
\usepackage{cite}
\usepackage{yfonts}
\usepackage{algorithmic}
\usepackage{algorithm}
\usepackage{tikz-cd}
\usepackage{hyperref}
\usepackage{cleveref}
\usepackage{xcolor}

\Crefname{ALC@unique}{Line}{Lines}
\newcounter{myalg}
\AtBeginEnvironment{algorithmic}{\refstepcounter{myalg}}
\makeatletter
\@addtoreset{ALC@unique}{myalg}
\makeatother

\newtheorem{theorem}{Theorem}
\newtheorem{lemma}{Lemma}
\newtheorem{prop}{Proposition}
\newtheorem{cor}{Corollary}

\theoremstyle{definition}
\newtheorem{definition}{Definition}
\newtheorem{heuristic}{Heuristic}

\theoremstyle{remark}
\newtheorem{remark}{Remark}
\newtheorem{example}{Example}


\DeclareMathOperator{\Tr}{Tr}
\DeclareMathOperator{\Diag}{Diag}
\DeclareMathOperator{\Gal}{Gal}
\DeclareMathOperator{\Aut}{Aut}
\DeclareMathOperator{\disc}{Disc}
\DeclareMathOperator{\rk}{Rk}
\DeclareMathOperator{\Ker}{Ker}
\DeclareMathOperator{\im}{Im}
\DeclareMathOperator{\Cl}{Cl}
\DeclareMathOperator{\Disc}{Disc}
\DeclareMathOperator{\Basis}{Basis}
\DeclareMathOperator{\IsIrreducible}{IsIrreducible}
\DeclareMathOperator{\Dc}{\mathcal{D}}
\DeclareMathOperator{\Pc}{\mathcal{P}}
\DeclareMathOperator{\Pl}{Pl}
\DeclareMathOperator{\Br}{Br}

\newcommand{\Q}{\mathbb{Q}}
\newcommand{\Z}{\mathbb{Z}}
\newcommand{\N}{\mathbb{N}}
\newcommand{\F}{\mathbb{F}}
\newcommand{\U}{\mathbb{U}}
\renewcommand{\S}{\mathfrak{S}}
\newcommand{\A}{\mathfrak{A}}
\newcommand{\Oc}{\mathcal{O}}
\newcommand{\B}{\mathcal{B}}
\newcommand{\Pp}{\mathbb{P}}
\newcommand{\p}{\mathfrak{p}}
\newcommand{\1}{\mathbf{1}}
\renewcommand{\epsilon}{\varepsilon}

\newcommand{\MM}[1]{\textcolor{blue}{{\sf(Mickael's comment:} {\sl{#1})}}}
\newcommand{\PK}[1]{\textcolor{red}{{\sf(Peter's comment:} {\sl{#1})}}}

% Add a serial/Oxford comma by default.
\newcommand{\creflastconjunction}{, and~}

\title{Efficient computations in central simple algebras using Amitsur cohomology}

% Authors: full names plus addresses.
%\author{Péter Kutas\thanks{Faculty of Informatics, Eötvös Loránd University and School of Computer Science, University of Birmingham
  %(\email{p.kutas@bham.ac.uk},\url{https://sites.google.com/view/peterkutas89}). The first author is supported by the Hungarian Ministry of Innovation and Technology NRDI Office within the framework of the Quantum Information National Laboratory Program, the János Bolyai Research Scholarship of the Hungarian Academy of Sciences and by the UNKP-22-5 New National Excellence Program. The first author is also partly supported by EPSRC through grant number EP/V011324/1.}
  %\and Mickaël Montessinos\thanks{Institute of Mathematics, Faculty of Mathematics and Informatics, Vilnius University
  %(\email{mickael@montessinos.fr}, \url{http://mickael.montessinos.fr}).}
%}
\author{Péter Kutas and Mickaël Montessinos}


\begin{document}

\maketitle

\begin{abstract}
    We propose a presentation for central simple algebras over a field \(k\) using Amitsur cohomology. We provide efficient algorithms for computing a cocycle corresponding to any such algebra given by structure constants. If \(k\) is a number field, we use this presentation to prove that the explicit isomorphism problem (i.e., finding an isomorphism between central simple algebras given by structure constants) reduces to \(S\)-unit group computation and other related number theoretical computational problems. This also yields the first polynomial quantum algorithm for the explicit isomorphism problem over number fields.
\end{abstract}

\section{Introduction}
\section{Introduction}  \label{sec:introduction}

\newcommand\inexpIntro[3]{#1?(#2,#3).}
\newcommand\rinexpIntro[3]{*#1?(#2,#3).}
\newcommand\outexpIntro[3]{#1!(#2,#3).}
\newcommand\outatomIntro[3]{#1!(#2,#3)}

We propose a fully automated method for proving termination of \(\pi\)-calculus processes.
Although there have been a lot of studies on termination analysis for the \(\pi\)-calculus
and related calculi~\cite{Deng06IC,Demangeon07,SangiorgiTermination,KobayashiHybrid,Yoshida04IC,DBLP:journals/jlp/DemangeonHS10,Venet98SAS}, most of them have been rather theoretical,
and there have been surprisingly little efforts in developing  fully automated termination
verification methods and tools based on them. To our knowledge,
Kobayashi's \typical{}~\cite{TyPiCal,KobayashiHybrid} is the only exception that
can prove termination of \(\pi\)-calculus processes (extended with natural numbers)
fully automatically, but its termination analysis is quite limited (see Section~\ref{sec:relatedwork}).

Our method is based on a reduction to termination analysis for sequential programs:
we translate a \(\pi\)-calculus process \(P\) to a sequential program \(S_P\), so that
if \(S_P\) is terminating, so is \(P\). The reduction allows us to use
powerful, mature methods and tools
for termination analysis of sequential programs~\cite{heizmann2016ultimate,freqterm,DBLP:conf/lics/PodelskiR04,Kuwahara2014Termination,DBLP:journals/cacm/CookPR11}.

The idea of the translation is to convert a chain of communications on replicated input
channels to a chain of recursive function calls of the target sequential program.
Let us consider the following Fibonacci process:
\begin{align*}
    & \rinexpIntro{\fib}{n}{r}
        \ifexp{n<2}{ \soutatom{r}{1} \\ &\quad}
                   { \nuexp{s_1} \nuexp{s_2} (\outatomIntro{\fib}{n-1}{s_1} \PAR \outatomIntro{\fib}{n-2}{s_2} \PAR \sinexp{s_1}{x}\sinexp{s_2}{y}\soutatom{r}{x+y}) \\}
    & \PAR \outatomIntro{\fib}{m}{r}
\end{align*}
Here, the process
$\rinexpIntro{\fib}{n}{r} \ldots$ is a function server that computes the \(n\)-th Fibonacci number
in parallel and returns the result to \(r\),
and $\outatom{\fib}{m}{r}$ sends a request for computing the \(m\)-th Fibonacci number;
those who are not familiar with the syntax of the \(\pi\)-calculus may wish to consult
Section~\ref{sec:targetlanguage} first.
To prove that the process above is terminating for any integer \(m\),
it suffices to show that there is no infinite chain of communications on $\fib$:
\[
    \fib(m,r) \to \fib(m_1,r_1) \to \fib(m_2,r_2) \to \cdots.
\]
We convert the process above to the following program:\footnote{The actual translation
  given later is a little more complex.}
\begin{verbatim}
 let rec fib(n) = if n<2 then () else (fib(n-1) [] fib(n-2)) in
 fib(m)
\end{verbatim}
Here, \texttt{[]} represents the non-deterministic choice.
Note that, although the calculation of Fibonacci numbers is not preserved,
for each chain of communications on \texttt{fib}, there is a corresponding
sequence of recursive calls:
\[
\mathtt{fib}(m) \to \mathtt{fib}(m_1) \to \mathtt{fib}(m_2) \to \cdots.
\]
Thus, the termination of the sequential program above implies the termination of
the original process.
As shown in the example above, (i) each communication on a replicated input channel
is converted to a function call, (ii) each communication on a non-replicated input
channel is just removed (or, in the actual translation, replaced by a call of
a trivial function defined by \(f(\seq{x})=(\,)\)), and (iii) parallel composition
is replaced by a non-deterministic choice.
We formalize the translation outlined above and prove its correctness.

The basic translation sketched above sometimes loses too much information.
For example, consider the following process:
\begin{align*}
    & \rinexpIntro{\pre}{n}{r} \soutatom{r}{n-1} \\
    & \PAR \rinexpIntro{f}{n}{r} \ifexp{n<0}{ \soutatom{r}{1} }
                                       { \nuexp{s} (\outatomIntro{\pre}{n}{s} \PAR \sinexp{s}{x}\outatomIntro{f}{x}{r}) } \\
    & \PAR \outatomIntro{f}{m}{r}
\end{align*}
The translation sketched above would yield:
\begin{verbatim}
  let pred(n) = n-1 in
  let rec f(n) = if n<0 then () else (pred(n) [] f(*)) in
  f(m)
\end{verbatim}
Here, \texttt{*} represents a non-deterministic integer: since we have removed
the input $\sinatom{s}{x}$, we do not have information about the value of \( x \).
As a result, the sequential program above is non-terminating, although the original
process is terminating.
To remedy this problem, we also refine the basic translation above by using a refinement
type system for the \(\pi\)-calculus. Using the refinement type system,
we can infer that the value of \(x\) in the original process is less than \(n\),
so that we can refine the definition of \texttt{f} to:
\begin{verbatim}
 let rec f(n) = ... else (pred(n) [] let x=* in assume(x<n);f(x))
\end{verbatim}
The target program is now terminating, from which
we can deduce that the original process is also terminating.
We have implemented an automated tool based on the refined translation above.

The contributions of this paper are summarized as follows.
\begin{itemize}
\item The formalization of the basic translation from the \(\pi\)-calculus
  (extended with integers) to sequential programs, and a proof of its correctness.
\item The formalization of a refined translation based on a refinement type system.
\item An implementation of the refined translation, including automated refinement type
  inference based on CHC solving, and experiments to evaluate the effectiveness of
  our method.
\end{itemize}

The rest of this paper is structured as follows.
Section~\ref{sec:targetlanguage} introduces the source and target languages
of our translation.
Section~\ref{sec:approach} 
formalizes the basic translation, and proves its correctness.
Section~\ref{sec:refinement} refines the basic translation by using a refinement type system.
Section~\ref{sec:implementation} reports an implementation and experiments.
Section~\ref{sec:relatedwork} discusses related work,
and Section~\ref{sec:conclusion} concludes the paper.


\section{Preliminaries}\label{Sec:Prelim}
\section{Preliminaries}\label{chpt:preliminiaries}
In this chapter we will introduce some of the mathematical background and notation needed for this thesis. In particular, we will shortly introduce the differential geometric description of spacetime in Section \ref{sec:spacetime_geometry} and give an introduction to the notion of global hyperbolicity and its connection to Green- and normally-hyperbolic operators in Section \ref{sec:global_hyperbolicity}. In a bit more detail, we will introduce the notion of differential forms and give explicit definitions, also in terms of an index based notation, in Section \ref{sec:differential_forms}. For completeness, in Section \ref{sec:cat-theory}, we present basic definitions of category theory. The reader familiar with these topics can safely skip this chapter and refer to it when interested in the chosen conventions.
%
%
%
%
%%%%%%
%%SPACTIME GEOMETRY
%%%%%
%
%
%
\subsection{Spacetime geometry}\label{sec:spacetime_geometry}
In GR, the universe is mathematically described as a four dimensional \emph{spacetime}, consisting of a smooth, four dimensional manifold \gls{M} (assumed to be Hausdorff, connected, oriented, time-oriented and para-compact) and a Lorentzian metric $g$. We will assume the signature of the Lorentzian metric $g$ to be $(-,+,+,+)$. The Levi-Civita connection on $(\M,g)$ is as usual denoted by \gls{nabla}.
Throughout this thesis, we treat spacetime as fixed, implementing a gravitational background determined classically by Einstein's field equations. Hence, we neglect any back-reaction of the fields on the metric, both in the quantum and the classical case. In that sense, we treat the fields as \emph{test fields}.\par
For the basic mathematical theory regarding Lorentzian manifolds, we refer to the literature: An introduction to the topic with an emphasis on the physical application in GR is for example given in \cite{wald_GR} and \cite{carroll_spacetime-and-gr}.
Here, we will shortly recap the notion of a tangent space and tangent bundle and generalize to the notion of a vector bundle which we will use in the general description of normally hyperbolic operators and differential forms.
In the following, we generalize the setting to an arbitrary smooth manifold $\N$ of dimension $N$ with either Lorentzian or Riemannian metric $k$.\par
%
%
A \emph{tangent vector} $v_x$ at point $x \in \N$ is a linear map $v_x : C^\infty(\N , \IR) \to \IR$ that obeys the Leibniz rule, that is, for $f,g \in C^\infty (\N,\IR)$ it holds $v_x(fg) = f(x)v_x(g) + v_x(f)g(x)$.
We define the \emph{tangent space} \gls{TxN} of $\N$ at $x$ as the real $N$-dimensional vector space of all tangent vectors at point $x$.
The disjoint union of all tangent spaces is called the \emph{tangent bundle} \gls{TN} of $\N$ and is itself a manifold of dimension $2N$. A \emph{vector field} is a map $v: \N \to T\N$ such that $v(x) \in T_x\N$.
The respective dual spaces, that is the space of all linear functionals, the \emph{co-tangent space} and the \emph{co-tangent bundle}, are denoted by \gls{TsxN} and \gls{TsN} respectively.\par
%
For Lorentzian manifolds, we call a tangent vector $v$ at $x \in \N$ \emph{timelike} if $k_{\mu \nu} v^\mu v^\nu < 0$, \emph{spacelike} if $k_{\mu \nu} v^\mu v^\nu > 0$ and \emph{null} (or lightlike) if $k_{\mu \nu} v^\mu v^\nu = 0$. At every point $x \in \N$, we define the set of all \emph{causal}, that is, either timelike or null, tangent vectors in the tangent space at $x$. This set is called the \emph{light cone} at $x$ and it is split up into two distinct parts, one that we call the future light cone, and one that we call the past light cone at $x$. Since we assume the manifold to be time orientable, there exists a smooth vector field $t$ that is timelike at every $x \in \N$. Given this time orientation, we identify the future (past) light cone with the set of tangent vectors $v \in T_x\N$ such that $k_{\mu\nu} v^\mu t^\nu < 0$ (respectively $> 0$). Therefore, a tangent vector $v$ at $x$ is called \emph{future directed} (past directed) if it lies in the future (past) light cone at $x$.\\
Accordingly, a curve $\gamma : I \to \N$ is called timelike (spacelike, null, causal, future or past directed) if its tangent vector $\dot{\gamma}$ is timelike (spacelike, null, causal, future or past directed) at every $x \in \N$.  For every point $x \in \N$ we define the \emph{causal future/past} \gls{causalfuturepast} of $x$ as the set of all points $q \in \N$ that can be reached by a future directed causal curve originating in $x$. For any subset $S \in \N$ we define $J^\pm (S) = \bigcup_{x \in S} J^\pm(x)$ and $J(S) = J^+(S) \cup J^- (S)$. Finally, the future/past domain of dependence $\gls{futurepastdomainofdependence}$ of a set $S \subset \N$ is the set of all points $x \in \N$ such that every inextendible causal curve through $x$ intersects $S$. The \emph{domain of dependence} \gls{domainofdependence} of $S$ is the union of the future and past domain of dependence of the set $S$.
For more details on the causal structure of spacetime we refer to for example \cite[Chapter 8]{wald_GR}.\par
%
%
%
The notion of tangent bundles can be generalized to the notion of a vector bundle. Instead of ``attaching'' the vector spaces $T_x \N$ to every point $x$ of the manifold, we allow for the occurrence of arbitrary vector spaces, called the fibres of the vector bundle. A vector bundle then consists of the base manifold, in our case $\N$, the total space and a map $\pi$ from the total space to the base manifold, that can be locally trivialized. At each point of the base manifold, the pre-image of $\pi$ is the fibre of the vector bundle. To be precise we define, following \cite{rudolph_schmidt}:
\begin{definition}[Vector bundle]
	A smooth \emph{vector bundle} over $\N$ is a tuple $\gls{vectorbundle} = (E,\N, \pi)$, where $E$ is a smooth manifold and $\pi : E \to \N$ is a smooth surjective map satisfying:
	\begin{enumerate}
		\item For every $x \in \N$, $\pi^{-1}(x)$ is a vector space, called the fibre of the bundle at point $x$.
		\item There exists a finite dimensional vector space $F$, an open covering $\left\{ U_\alpha\right\}_\alpha$ of $\N$ and a family of diffeomorphisms $\chi_\alpha : \pi^{-1}(U_\alpha) \to U_\alpha \times F$ such that for all $\alpha$ it holds $\chi_\alpha \comp \text{pr}_1 =  \restr{\pi}{\pi^{-1}(U_\alpha)}$ and for every $x \in \N$ the map $\text{pr}_2 \comp \restr{\chi_\alpha}{\pi^{-1}(x)} : \pi^{-1}(x) \to F$ is linear.
	\end{enumerate}
\end{definition}
Here, the maps $\text{pr}_1$ and $\text{pr}_2$ denote the projection onto the first respectively second component of an element in $U_\alpha \times F$. The properties graphically mean that \emph{locally}, the vector bundle ``looks like" the product of the base manifold with the fibre. The tuples $(U_\alpha, \chi_\alpha)$ are called \emph{local trivializations} of the vector bundle. Like for vector spaces, we can define the sum and product of vector bundles, by using the according vector space definitions on the fibres of the bundle.\par
Let $\mathfrak{X}, \mathfrak{Y}$ be vector bundles over $\N$ with fibres $X_x$ and $Y_x$ at $x \in \N$. We denote by \gls{whitneysum} the \emph{Whitney sum} of the two vector bundles - the vector bundle over $\N$ whose fibres are given by the direct sum $X_x \oplus Y_x$. Similarly, one obtains the local trivializations of the Whitney sum from the trivializations of $\mathfrak{X}, \mathfrak{Y}$ and direct sums.\par
Accordingly, let $\mathfrak{X}, \mathfrak{Y}$ be vector bundles over $\N$ and $\widetilde{\N}$, with fibres $X_x$ and $Y_{\tilde{x}}$ at $x \in \N$, $\tilde{x} \in \widetilde{\N}$ respectively. We denote by \gls{outerproductbundle} the \emph{outer product} of the two vector bundles - the vector bundle over $\N \times \widetilde{\N}$ whose fibres are given by the tensor products $X_x \otimes Y_x$. Similarly, one obtains the local trivializations of the outer product from the trivializations of $\mathfrak{X}, \mathfrak{Y}$ and tensor products. \par
%
Finally, we generalize the notion of vector fields:
\begin{definition}[Sections of vector bundles]
Let $\mathfrak{X}=(E,\N,\pi)$ be a vector bundle with fibres $X_x=\pi^{-1}(x)$ at $x \in \N$. A \emph{smooth section} of the vector bundle is a smooth map $\gamma : \N \to E$ such that $\gamma(x) \in X_x$ for all $x \in \N$. The \emph{vector space of smooth sections} of $\mathfrak{X}$ is denoted by \gls{gammax}, the one with compactly supported sections is as usual denoted by \gls{gammaxzero}.
\end{definition}
In this language, a vector field $v$ is just a smooth section of the tangent bundle of a manifold, $v \in \Gamma(T\N)$. One may therefore identify the physical notion of fields with smooth sections of vector bundles. This point of view will be used to define the notion of differential forms in Section \ref{sec:differential_forms}.\par
In this thesis, we usually are interested in complex valued functions (or sections in general). Therefore, we view all occurring vector bundles as complex, in the sense that we take two distinct copies of the vector bundle, one representing the real, one the imaginary part of the bundle. A section of that complex vector bundle is just a pair of two sections of the real vector bundle under consideration. From now, if not specified explicitly, we will view all vector bundles, including the tangent bundle $T\N$, as complex vector bundles. Accordingly, smooth sections of those bundles will in general be complex valued.
%
%
%
%
%
%
%
%
%%%%%%%
%%PARTIAL DIFFERENTIAL OPERATORS AND GLOBAL HYPERBOLICITY
%%%%%%%
%
%
%
\subsection{Partial differential operators and global hyperbolicity}\label{sec:global_hyperbolicity}
When dealing with field theories, whether classical or quantum, one is, of course, interested in the dynamics of the fields. These are usually described by some partial differential equation, often of second order. In the following, we give a short introduction to the theory of certain partial differential operators acting on smooth sections of a vector bundle over the spacetime $(\M,g)$.\par
%
As we have seen, these smooth sections are generalizations of the notion of a field.  In the following, let $\mathfrak{X}$ denote a vector bundle over the manifold $\M$ and let $P: \Gamma(\mathfrak{X}) \to \Gamma(\mathfrak{X})$ be a partial differential operator acting on smooth sections of the bundle. As in the case of flat spacetime, we are interested in basic questions regarding the differential equation $Pf = j$, for example: Can we formulate a (globally) well posed initial value problem? Does the differential equation possess (unique) solutions? To answer these questions, we will now restrict to the case where $P$ is linear and of second order, as it is often the case in physical applications. One can show that for a certain class of such operators, namely normally hyperbolic partial differential operators of second order, we can rigorously treat these questions.\par
Choosing local coordinates $x=(x_\mu)$ on $\M$ and a local trivialization of $\mathfrak{X}$, a linear partial differential operator of second order is called \emph{normally hyperbolic} if it takes the form
\begin{align}
	P = - \sum_{\mu,\nu} g^{\mu \nu} \partial_\mu \partial_\nu + \sum_{\alpha} A_\alpha (x) \partial_\alpha + B(x) \formspace,
\end{align}
where $A_\alpha$ and $B$ are matrix-valued coefficients depending smoothly on the coordinate $x$ (see. \cite[Chapter 1.5]{baer_ginoux_pfaeffle}). One can also formulate a coordinate independent definition in terms of the principal symbol, which we will not present here (see for example \cite[Section 1.5]{baer_ginoux_pfaeffle} ). \par
%
Normally hyperbolic operators possess unique fundamental solutions (see for example the fundamental solutions to the wave operator as noted in Lemma \ref{lem:fundamental_solution_wave_operator}). These fundamental solutions fulfill certain physically important properties, such as a finite propagation speed smaller than the speed of light. Furthermore, specifying the initial data on some space-like hypersurface $X \in  \M$ specifies a unique solution on the domain of dependence $D(X)$ of $X$. Due to these properties, one often calls normally hyperbolic operators just \emph{wave operators}. But to state a \emph{globally} well posed initial value problem for a wave equation, we need to restrict the class of spacetimes $\M$ under consideration to those that possess space-like hypersurfaces $X$ whose domain of dependence is all of the spacetime, $D(X) = \M$. This leads to the notion of \emph{globally hyperbolic} spacetimes:
\begin{definition}[Global Hyperbolicity]
	A spacetime $\M$ is called \emph{globally hyperbolic} if there exists a Cauchy surface $\gls{sigma}$ in $\M$.
\end{definition}
\noindent Here, a Cauchy surface is a space-like hypersurface $\Sigma \subset \M$ such that every inextendible causal curve $\gamma$ intersects $\Sigma$ exactly once. One can show that Cauchy surfaces fulfill the desired property mentioned above, that is,  $D(\Sigma) = \M$. Furthermore, one can show that any globally hyperbolic spacetime $\M$ is foliated by a one-parameter family $\left\{ \Sigma_t \right\}_t$ of Cauchy surfaces (see for example \cite[Theorem 8.3.14]{wald_GR}). \par
In physical applications, one often finds the dynamics of a theory to be described by wave operators. Most prominently, the Klein-Gordon operator $(\square + m^2)$ acting on scalar fields, or its generalization, the wave operator acting on differential forms introduced in Section \ref{sec:differential_forms}, is normally hyperbolic. But there are also important physical field theories that are not described by wave operators, such as the Proca field treated in this thesis. It turns out that the Proca operator (see Definition \ref{def:proca_operator}) is a so called \emph{Green-hyperbolic} operator. These are again partial differential operators $P$ of second order acting on smooth sections of some vector bundle, such that $P$ (and its dual $P'$) posses fundamental solutions. Obviously, normally hyperbolic operators are Green-hyperbolic, but the opposite is not true. One can generalize some results obtained by studying normally hyperbolic operators to Green-hyperbolic operators. An introduction to this topic is given in \cite{baer_green-hyperbolic}, where it is also shown that the Proca operator is Green-hyperbolic but not normally hyperbolic.\par
For our application, the notion of Green-hyperbolicity is not of vast importance, but it is worth mentioning that there exists a more detailed mathematical background on the treatment of such operators.
A very detailed description of normally hyperbolic operators on Lorentzian manifolds, including proofs of the above statements regarding the initial value problem and the existence of fundamental solutions, is given in \cite{baer_ginoux_pfaeffle}, also with an overview of quantization. A shorter introduction to the topic is for example treated in \cite{baer-ginoux_classical-and-quantum-fields}, also with a description of quantization.
%
%
%
%
%
%
%%%
%
%
%
%%
%%%%%%%%%
%%%DIFFERENTIAL FORMS
%%%%%%%%
%
%
%
\subsection{Differential forms}\label{sec:differential_forms}
%
%
Differential forms provide an elegant, coordinate independent description of calculus on smooth manifolds. In particular, they generalize the notion of line- and volume-integrals that are known from analysis. Differential forms play a remarkable role in physics, as one can argue that they indeed describe fundamental physical entities. As an example, instead of viewing a classical force as a vector, one can think of it, more closely related to experiments, as a differential one-form that assigns a scalar to a tangent vector of a curve. This scalar is the (infinitesimal) work associated with the force along the curve. Also, differential forms allow for an elegant geometric description of field theories, for example the Maxwell and Proca field theories that we encounter in this thesis. In Maxwell's classical theory of electromagnetism, instead of viewing the electric and magnetic field (which are conceptually just forces) as the fundamental physical entities, one introduces the \emph{vector potential}, a one-form, consisting of the scalar electric potential and the vector potential associated with the magnet field. Experiments like the Aharonov-Bohm experiment allow for an interpretation of the vector potential as the fundamental physical object, rather than the associated electromagnetic field. \\
Even more fundamentally, the two main theories of physics, General Relativity and the Standard Model of particle physics, are field theories. They are deeply connected to a geometric interpretation and can be elegantly described using differential forms. \par
%
%
Despite of all this, differential forms are usually not part of the standard curriculum of physicists. We shall therefore introduce the basic aspects and definitions regarding differential forms that are used in this thesis. For a more detailed introduction we refer to the literature: For example \cite[Chapter 2 and 4]{rudolph_schmidt} or \cite[Appendix B]{wald_GR} provide introductions to the topic.\par
%
%
In the following, let $\N$ denote a smooth $N$-dimensional manifold, assumed to be Hausdorff, connected, oriented and para-compact, with either Lorentzian or Riemannian metric $k$ and Levi-Civita connection $\nabla$. For a Lorentzian manifold we use the sign convention $(-,+,\dots,+)$ of the metric $k$. The number of negative eigenvalues of $k$ is denoted by $s$, so $s=0$ for a Riemannian manifold and, in our convention, $s=1$ for a Lorentzian manifold.
Later, we will specify to a four dimensional (globally hyperbolic) spacetime consisting of a four dimensional manifold $\M$ with Lorentzian metric $g$ and Cauchy surface $\Sigma$ with induced Riemannian metric $h$.
%
We define:
\begin{definition}[Differential form]
	Let $p\in \{0,1,\dots,N\}$. A \emph{differential form} $\omega$ of degree $p$, or $p$-form for short, on the manifold $\N$ is an anti-symmetric tensor field of rank $(0,p)$. That is, at every point $x \in \N$, $\omega_x$ is an anti-symmetric multi-linear map
	\begin{align}
	\omega_x : \underbrace{T_x \N \times T_x \N \times \cdots \times T_x \N}_{p\text{-times}} \to \IR \formspace.
	\end{align}
	We denote the vector space\footnote{Naturally, addition and scalar multiplication are defined point-wise.} of $p$-forms on $\N$ by $\gls{omegap}$, the space with compactly supported ones by \gls{omegapz}.
\end{definition}
As an example, a zero-form $f \in \Omega^0(\N)$ is just a $C^\infty$-function from $\N$ to $\IR$, hence we can identify $\Omega^0(\N) = C^\infty (\N, \IR)$. A one-form $A \in \Omega^1(\N)$ is nothing more than a co-vector field and in a physical context usually denoted in local coordinates by $A_\mu$. Note, that alternatively one can directly define a $p$-form as a smooth section of the $p$-th exterior product of the co-tangent bundle and hence identify $\Omega^p(\N) = \Gamma \big( \largewedge^k T^*\N\big)$. As mentioned in Section \ref{sec:spacetime_geometry}, we view the tangent bundle as a complex bundle. Therefore, the sections of that bundle will be complex valued functionals. In that fashion, we will usually view the spaces $\Omega^p(\N)$ as complex valued differential forms.\par
%
Next we define the basic operations, besides addition and scalar multiplication, that one can perform on differential forms.
%
\begin{definition}[Exterior product]
	Let $A \in \Omega^p(\N)$ be a $p$-form and  $B\in \Omega^q(\N)$ a $q$-form on $\N$. \\
	The \emph{exterior product} $\gls{wedge}:\Omega^p(\N) \times \Omega^q(\N) \to \Omega^{p+q} (\N)$ is defined by
	\begin{align}
	(A \wedge B)_{\mu_1\dots\mu_p \nu_1\dots\nu_q} = \frac{(p+q)!}{p!q!}\, A_{[\mu_1 \dots \mu_p} B_{\nu_1\dots\nu_q]} \formspace,
	\end{align}
	where the anti-symmetrization of a tensor $T$ is given through
	\begin{align}
	T_{[\mu_1\dots\mu_p]} = \frac{1}{p!} \sum\limits_{\sigma\in S_N }\textrm{sgn}(\sigma) T_{\sigma(\mu_1)\dots\sigma(\mu_p)} \formspace.
	\end{align}
\end{definition}
Here, $S_N$ denotes the symmetric group\footnote{Usually the symmetric group is defined as the set of permutations of $\{1,2,\dots,N\}$ but we chose the index to run over $\{0,1,\dots,N-1\}$, identifying the time component with zero rather then one.} of degree $N$, consisting of permutations of the set $\{0,1,\dots,N-1\}$.
With this notion of multiplication, point-wise addition and scalar multiplication, the space $\gls{omega} \coloneqq \bigoplus_{p = 0}^\infty \Omega^p(\N) = \bigoplus_{p = 0}^N \Omega^p(\N)$ becomes an algebra, usually called the Grassmann- or \emph{exterior algebra} of differential forms on $\N$. We have used that obviously $\Omega^k(\N) =0$ for $k >N$ due to the anti-symmetrization.\par
Furthermore, we find a notion of how to \emph{pullback} differential forms on manifolds to another manifold, for example the pullback of a differential form on the spacetime $\M$ to differential forms on its Cauchy surface $\Sigma$. Given a $C^\infty$-map $\psi: \widetilde{\N} \to \N$, where $\N, \widetilde{\N}$ are manifolds, we can naturally define the pullback of a function $f \in \Omega^0(\N)$ to a function $(\psi^* f) \in \Omega^0(\widetilde{\N})$ by composing $f$ with $\psi$:
\begin{align}
\psi^* f \coloneqq f \comp \psi \formspace.
\end{align}
\newpage
With the pullback of functions defined, we can define how to \emph{push forward}, or carry along, vector fields on $\widetilde{\N}$ to vector fields on $\N$: Let $f\in \Omega^0(\N)$ and $\tilde{v} \in \Gamma(T\widetilde{\N})$ and $\tilde{x} \in \widetilde{\N}$. Then
\begin{align}
(\psi_* \tilde{v})_{\psi(\tilde{x})} (f) \coloneqq \tilde{v}_{\tilde{x}}(\psi^* f)
\end{align}
defines the vector field $(\psi_* v) \in \Gamma(T\N)$. With these basic operations at hand, we can generalize to define the pullback of differential forms:
\begin{definition}[Pullback]\label{def:pullback}
	Let $\N, \widetilde{\N}$ be manifolds of dimension $N,\widetilde{N}$ respectively, and let $\psi: \widetilde{\N} \to \N$ be a smooth map. Then, $\psi$ defines an algebra homomorphism $\psi^* : \Omega(\N) \to  \Omega(\widetilde{\N})$,
	called the \emph{pullback} of differential forms. For $\omega \in \Omega^p(\N)$, $\tilde{x} \in \widetilde{\N}$ and $\tilde{v}_i \in T_x \widetilde{\N}$, $i=1,2,\dots,p$, it is defined by
	\begin{align}
	\left( \psi^* \omega \right)_{\tilde{x}}  (\tilde{v}_1,\tilde{v}_2,\dots,\tilde{v}_p) \coloneqq \omega_{\psi(\tilde{x})} (\psi_* \tilde{v}_1, \dots , \psi_* \tilde{v}_p) \formspace.
	\end{align}
\end{definition}
%
%
%
%
On the exterior algebra we find a duality, provided by the Hodge operator:
\begin{definition}[Hodge dual]
	The hodge star operator $\gls{hodge}: \Omega^p(\N) \to \Omega^{N-p}(\N)$ is defined through
	\begin{align}
	B \wedge *A = \frac{1}{p!} B^{\mu_1\dots\mu_p}A_{\mu_1\dots\mu_p} \dvolk \formspace,
	\end{align}
	which yields the coordinate representation
	\begin{align}
	(*A)_{\mu_{p+1}\dots\mu_N} = \frac{\detk}{p!} \, \epsilon_{\mu_1\dots\mu_N} A^{\mu_1\dots\mu_p} \formspace.
	\end{align}
\end{definition}
Here, \gls{levicivita} denotes the fully antisymmetric tensor of rank $N$ (Levi-Civita symbol) satisfying $\epsilon_{12,\dots,N} =1$ and the \emph{volume element} \gls{dvolk} is defined by
\begin{align}
\left( \gls{dvolk} \right)_{\alpha_1\dots\alpha_N} = \detk \, \epsilon_{\alpha_1\dots\alpha_N} \formspace.
\end{align}
In a sense, the volume element describes how the curvature of the manifold deforms a unit volume.
The duality follows from the important property of the Hodge operator as stated in the following lemma:
\begin{lemma}
	Let $*$ denote the Hodge star operator on the exterior algebra $\Omega(\N) $. It holds that
	\begin{align}
	** = (-1)^{s+p(N-p)} \, \mathbbm{1} \formspace,
	\end{align}
	which is trivially equivalent to $*^{-1} = (-1)^{s+p(N-p)} \, *$.
\end{lemma}
\begin{proof}
	Let $A \in \Omega^p(\N)$ be a $p$-form on $\N$. Then:
	\begin{align}
	(*{*A})_{\mu_1 \dots \mu_p}
	&= \frac{\detk \, \detk}{p! \, (N-p)!} \; \epsilon_{\alpha_{p+1}\dots\alpha_N \mu_1 \dots \mu_p}\;\epsilon^{\alpha_{1}\dots\alpha_N}\;A_{\alpha_1\dots\alpha_p} \notag\\
	&= (-1)^{p(N-p)} \frac{\detk \, \detk}{p! \, (N-p)!} \; \epsilon_{\alpha_{p+1}\dots\alpha_N \mu_1 \dots \mu_p}\;\epsilon^{\alpha_{p+1}\dots\alpha_{N}\alpha_1\dots\alpha_p}\;A_{\alpha_1\dots\alpha_p}  \notag\\
	&= (-1)^{s+p(N-p)} \delta\indices{^{[\alpha_{1}}_{\mu_{1}}}\, \dots \, \delta\indices{^{\alpha_p ] }_{\mu_p}} \;A_{\alpha_1\dots\alpha_p} \notag\\
	&=  (-1)^{s+p(N-p)}\;A_{\mu_1\dots\mu_p} \formspace
	\end{align}
	We have used Lemma \ref{lem:epsilon_contraction} and, in the last step, that the anti-symmetrization is absorbed by contraction because $A$ is antisymmetric.
\end{proof}
%
%
%
%
%
Furthermore, we can equip the exterior algebra with a differentiable structure, introducing the notion of the exterior derivative.
\begin{definition}[Exterior derivative]
	The \emph{exterior derivative} $\gls{d}:\Omega^p(\N) \to \Omega^{p+1} (\N)$ is defined by the following properties:
	\begin{enumerate}
		\item $d$ is linear
		\item $d$ obeys a graded Leibniz rule: Let $A \in \Omega^p(\N)$ and  $B\in \Omega^q(\N)$, then
		\begin{align}
		d(A \wedge B) = dA \wedge B + (-1)^p \, A \wedge dB
		\end{align}
		\item $d$ is nilpotent, that is,  $d^2 = 0$.
	\end{enumerate}
	In local coordinates, this is equivalent to the representation
	\begin{align}
	(dA)_{\mu \alpha_1\dots\alpha_p} = (p+1)\, \nabla_{[\mu}A_{\alpha_1\dots\alpha_p]} \formspace.
	\end{align}
\end{definition}
An important property of the exterior derivative is that it commutes (or rather intertwines its action) with pullbacks (see \cite[Proposition 4.1.7]{rudolph_schmidt}).
A $p$-form $\omega \in \Omega^p(\N)$ is called \emph{exact} if there is a $(p-1)$-form $\alpha \in \Omega^{p-1}(\N)$ such that $\omega = d\alpha$. We call $\omega$ \emph{closed} if $d \omega =0$. Accordingly, the space of closed $p$-forms is denoted by \gls{omegapd}, the space of exact ones by \gls{domegap}. As usual, the ones with compact support are denoted by a subscript zero. Note, that every exact form is closed, using that $d$ is by definition nilpotent, but the reverse is in general not true. It does hold, however, on certain manifolds with trivial topology, such as Minkowski spacetime. This is expressed in the so called Poincar\'e-Lemma (see for example \cite[Chapter 4]{bott_tu}) based on the study of de Rham cohomology.\par
%
Moreover, $N$-forms can naturally be integrated. Using local coordinates and a partition of unity, we define the integral of $N$-forms via the well known integration on $\IR^N$:
\begin{definition}[Integration on manifolds]
	Let $\left\{U_\alpha, \psi_\alpha\right\}_\alpha$ be an atlas of the manifold $\N$ and $\left\{\chi_\alpha\right\}_\alpha$ a partition of unity subordinate to the locally finite open cover $\left\{U_\alpha\right\}_\alpha$. Let $x^\mu_{(\alpha)}$ be a coordinate basis of $\psi$ on $U_\alpha$. For any $N$-form $\omega \in \Omega^N_0(\M)$ we define the integral
	\begin{align}
	\int\limits_{\N} \omega &\coloneqq \sum_{\alpha} \int\limits_{\psi_\alpha (U_\alpha)} w(x_{(\alpha)}^0,\dots,x_{(\alpha)}^1)\; dx_{(\alpha)}^0 \cdots dx_{(\alpha)}^{N-1} \formspace,
	\end{align}
	where $w$ are the components of $\omega$ in the coordinates $x_{(\alpha)}^\mu$, that is $\omega = w dx_{(\alpha)}^0 \wedge \cdots \wedge dx_{(\alpha)}^{N-1}$.
	This definition is independent of the choice of the atlas and the partition of unity (see \cite[Proposition 3.3]{bott_tu}).
\end{definition}
With integration at our disposal, we present an important theorem regarding the integration of exact differential forms:
\begin{theorem}[Stoke's Theorem]\label{thm:stokes}
	Let $\N$ be an oriented manifold of dimension $N$ and let its boundary $\partial \N$ be endowed with the induced orientation. Let $\gls{inclusionmap} : \partial \N \hookrightarrow \N$ be the inclusion operator.
	Let $\omega \in \Omega^{N-1}_0(\N)$ be a compactly supported $(N-1)$-form on $\N$. Then it holds
	\begin{align}
	\int\limits_\N d\omega = \int\limits_{\partial \N} i^*\omega \formspace.
	\end{align}
\end{theorem}
\begin{proof}
	A proof is given in most of the introductory literature on differential geometry (see for example \cite[Chapter 17, Theorem 2.1]{lang}).
	Note that one can equivalently formulate Stoke's theorem on a \emph{compact} manifold but for {arbitrary} (that is, in general not compactly supported) $(N-1)$-forms on the manifold (see for example \cite[Theorem 4.2.14]{rudolph_schmidt}). This will be of importance in later calculations.
\end{proof}
%
Furthermore, we can define a bilinear map on $\Omega^p(\N)$ using the integration of $N$-forms:
\begin{definition}
	Let $A,B \in \Omega^p(\N)$ such that their supports have a compact intersection. Define the bilinear map $\gls{innerprod} : \Omega^p(\N) \times \Omega^p(\N) \to \IC$ by
	\begin{align}
	\langle A, B \rangle_\N \coloneqq  \int_{\N } A \wedge * B = \int_{\N } A_{\mu_1 \dots \mu_p}B^{\mu_1 \dots \mu_p}\,\dvolk \formspace.
	\end{align}
\end{definition}
Since by definition $A \wedge * B$ is a compactly supported $N$-form, this is well defined. We may sometimes refer to $\langle \cdot , \cdot \rangle_\N$ as an inner product for simplicity, even though it is not positive definite.
%
%
%
%
%
Using the exterior derivative, we define the interior or co-derivative:
\begin{definition}[Interior derivative]
	The \emph{interior derivative} $\gls{delta} : \Omega^p(\N) \to \Omega^{p-1}(\N)$ is defined by
	\begin{align}
	\delta \coloneqq (-1)^{s+1+N(p-1)}\, {*{d*}} \formspace.
	\end{align}
	From the defining properties of $d$ and $*$ it follows $\delta^2 =0$.
\end{definition}
Here, $s$ again denotes the number of negative eigenvalues of the metric $k$ of $\N$. In accordance with our nomenclature, we call a $p$-form $\omega$ co-exact if there exists a $\alpha \in \Omega^{p+1}(\N)$ such that $\omega = \delta \alpha$ and co-closed if $\delta \omega = 0$. Accordingly, the spaces of co-closed and co-exact $p$-forms are denoted by \gls{omegapdelta} and \gls{deltaomegap} respectively.\par
Using the exterior and interior derivative we define the partial differential operator:
\begin{definition}[D'Alembert Operator]
	The d'Alembert (or Laplace - de Rham) operator $\gls{dalembert}: \Omega^p(\N) \to \Omega^{p}(\N)$ is defined by
	\begin{align}
	\square \coloneqq \delta d +d \delta \formspace.
	\end{align}
\end{definition}
By definition of the exterior and interior derivative, it is easy to show that $\square$ commutes with both $d$ and $\delta$:
\begin{align}
\square d &= (\delta d + d \delta )d \notag \\
&= d \delta d \notag \\
&= d (\delta d + d \delta) \formspace,
\end{align}
and analogously for $\delta$.
The d'Alembert operator, and its generalization to $(\square + m^2)$ for some constant $m > 0$, are important examples for a normally hyperbolic differential operators (see Section \ref{sec:global_hyperbolicity}) and we may therefore sometimes just refer to them as \emph{wave operators}.\par
The sign convention in the definition of the exterior derivative is chosen such that on any Lorentzian or Riemannian manifold the interior derivative is formally adjoint to the exterior derivative, that is,  for $A \in \Omega^{p}(\N)$ and $B \in \Omega^{p+1}(\N)$ it holds that
\begin{align}
\langle dA , B \rangle_{\N} = \langle A , \delta B \rangle_\N \formspace,
\end{align}
which leads to a representation in local coordinates of the Manifold given by:
\begin{align}
(\delta A)_{\mu_2\dots\mu_p} = - \nabla^{\mu_1}A_{\mu_1\dots\mu_p} \formspace.
\end{align}
To see that this is consistent, let $A \in \Omega^{p-1}(\N)$ and $B \in \Omega^{p}(\N)$ such that their supports have compact intersection.
We obtain, using Stoke's Theorem \ref{thm:stokes}:
\begin{align}
0 &= \int \limits_{\partial \N} i^* (A \wedge *B) \notag\\
&= \int \limits_{\N} d(A \wedge *B)  \notag\\
&= \int \limits_{\N} dA \wedge *B + (-1)^{p-1} A \wedge d{*B} \notag\\
&= \int \limits_{\N} dA \wedge *B + (-1)^{p-1} A \wedge *{*^{-1}}\underbrace{d{*B}}_{\textrm{is a } (N-p+1) \textrm{ form.}} \notag\\
&= \int \limits_{\N} dA \wedge *B + (-1)^{p-1}(-1)^{s+(N-p+1)(N-N+p-1)} A \wedge *{*d{*B}} \notag\\
&= \int \limits_{\N} dA \wedge *B + (-1)^{p+(1-p)(p-1)} A \wedge *\delta B \formspace.
\end{align}
It can easily be proven by induction that $\big(p+(1-p)(p-1)\big)$ is odd for any $p \in \IN$, which yields the result
\begin{align}
\langle dA , B \rangle_{\N} = \langle A , \delta B \rangle_\N \formspace.
\end{align}
The definitions stated above thus fulfill the requirement of formal adjointness of the exterior and interior derivate on an arbitrary Lorentzian or Riemannian manifold $\N$.
In local coordinates we use a partial integration to obtain
\begin{align}
\langle dA , B \rangle_\N &= \int \limits_{\N} dA \wedge * B \notag\\
%&= \int \limits_{\N} \frac{1}{p!} (dA)^{\alpha_1\dots\alpha_p}\,B_{\alpha_1 \dots \alpha_p} \, \dvolk \notag\\
&= \int \limits_{\N}  \frac{p}{p!} \nabla^{[\alpha_1}A^{\alpha_2\dots\alpha_p]}\,B_{\alpha_1 \dots \alpha_p} \, \dvolk \notag\\
&= \int \limits_{\N}  \frac{1}{(p-1)!} \nabla^{\alpha_1}A^{\alpha_2\dots\alpha_p}\,B_{\alpha_1 \dots \alpha_p} \, \dvolk \notag\\
&= - \int \limits_{\N}  \frac{1}{(p-1)!} A^{\alpha_2\dots\alpha_p}\, \nabla^{\alpha_1}B_{\alpha_1 \dots \alpha_p} \, \dvolk \notag\\
&= \langle A, \delta B \rangle_\N \formspace,
\end{align}
which yields
\begin{align}
-\nabla^{\alpha_1}B_{\alpha_1 \dots \alpha p} = (\delta B)_{\alpha_2 \dots \alpha_p}\formspace.
\end{align}
On the four dimensional spacetime $(\M,g)$ the definitions of the Hodge star operator and the interior derivative simplify, such that
\begin{align}
*_{(\M)}*_{(\M)} &= (-1)^{p+1} \mathbbm{1} \\
\delta_{(\M)} &= *_{(\M)}{d_{(\M)}*_{(\M)}} \formspace ,
\end{align}
holds on the spacetime $(\M,g)$ and
\begin{align}
*_{(\Sigma)}*_{(\Sigma)} &= \mathbbm{1} \\
\delta_{(\Sigma)} &= (-1)^p *_{(\Sigma)}{d_{(\Sigma)}*_{(\Sigma)}}
\end{align}
holds on  $(\Sigma,h)$. In the following we will drop the subscript ${(\M)}$, since we will perform all the calculations on a four dimensional spacetime, except when explicitly noted (for example with a subscript $(\Sigma)$).
%
%
%
%
%
%
%
%
%%%%%%
%%CATEGORY THEORY
%%%%%%
\subsection{Category theory}\label{sec:cat-theory}
The description of Quantum Field Theory on Curved Spacetimes (QFTCS) in the framework of \name{Brunetti}, \name{Fredenhagen} and \name{Verch} \cite{Brunetti_Fredenhagen_Verch} is based on category theory. In this thesis, we will not go into detail on those categorical aspects, however we will need some basic definitions to formulate the theory rigorously, that is namely the notion of a category and that of covariant functors, since, in the used framework, the generally covariant QFTCS is a functor.\par
Here, we present definitions given in \cite[Appendix A.1]{baer_ginoux_pfaeffle} and refer to the appropriate literature for details. We define:
\begin{definition}[Category]
	A \emph{category} $\mathsf{Cat}$ consists of the following:
	\begin{enumerate}
		\item a class $\mathsf{Obj}_\mathsf{Cat}$ whose members are called \emph{objects},
		\item a set $\mathsf{Mor}_\mathsf{Cat}(A,B)$, for any two objects $A,B \in \mathsf{Obj}_\mathsf{Cat}$, whose elements are called \emph{morphisms},
		\item for any three objects $A,B,C \in \mathsf{Obj}_\mathsf{Cat}$ there is a map
		\begin{align}
\mathsf{Mor}_\mathsf{Cat}(B,C) \times \mathsf{Mor}_\mathsf{Cat}(A,B) &\to \mathsf{Mor}_\mathsf{Cat}(A,C) \notag\\
(\psi,\phi) &\mapsto \psi \comp \phi
		\end{align}
		called the composition of morphisms subject to the relations:\vspace{4mm}
		\begin{enumerate}[label=(\arabic*)]
			\item for non equal pairs $(A,B)$, $(A',B')$ of objects, the sets $\mathsf{Mor}_\mathsf{Cat}(A,B)$ and $\mathsf{Mor}_\mathsf{Cat}(A',B')$ are disjoint,
			\item for every object $A$ there exists a morphism $\text{id}_A \in \mathsf{Mor}_\mathsf{Cat}(A,A)$ such that it holds for all objects $B$, morphisms $\psi \in \mathsf{Mor}_\mathsf{Cat}(B,A)$ and $\phi \in \mathsf{Mor}_\mathsf{Cat}(A,B)$
			\begin{align}
				\text{id}_A \comp \psi &= \psi \quad \text{and}\\
				\phi \comp \text{id}_A &= \phi \quad,
			\end{align}
			\item the composition law is associative, that is for an objects $A,B,C,D$ and any morphisms $\psi \in \mathsf{Mor}_\mathsf{Cat}(A,B)$, $\phi \in \mathsf{Mor}_\mathsf{Cat}(B,C)$ and $\chi \in \mathsf{Mor}_\mathsf{Cat}(C,D)$ it holds
			\begin{align}
				(\chi \comp \phi) \comp \psi = \chi \comp (\phi \comp \psi) \formspace.
			\end{align}
		\end{enumerate}
	\end{enumerate}
\end{definition}
%
%
%
\begin{definition}[Functor]
	Let $\mathsf{Cat1}$ and $\mathsf{Cat2}$ be categories. A \emph{covariant functor} $\mathscr{A}: \mathsf{Cat1} \to \mathsf{Cat2}$ consists of the map $\mathscr{A} : \mathsf{Obj}_\mathsf{Cat1} \to \mathsf{Obj}_\mathsf{Cat2}$ and maps $\mathscr{A}: \mathsf{Mor}_\mathsf{Cat1}(A,B) \to \mathsf{Mor}_\mathsf{Cat2}\big(\mathscr{A}(A),\mathscr{A}(B)\big)$ for any two objects $A,B \in \mathsf{Obj}_\mathsf{Cat1}$ such that
	\begin{enumerate}
		\item {the composition is preserved, that is for all objects $A,B,C \in \mathsf{Obj}_\mathsf{Cat1}$ and for any morphisms $\psi \in \mathsf{Mor}_\mathsf{Cat1}(A,B)$ and $\phi \in \mathsf{Mor}_\mathsf{Cat1}(B,C)$ it holds
		\begin{align}
			\mathscr{A}(\phi \comp \psi) = \mathscr{A}(\phi) \comp \mathscr{A}(\psi) \formspace,
		\end{align}}
		\item{
			$\mathscr{A}$ maps identities to identities, that is for any object $A \in \mathsf{Obj}_\mathsf{Cat1}$ it holds
			\begin{align}
				\mathscr{A}(\text{id}_\mathsf{A}) = \text{id}_{\mathscr{A}(A)} \formspace.
			\end{align}
			}
	\end{enumerate}
\end{definition}
%
%
%
%
%
%
%
%
%
%
%
%
%%%%%%
%%SIGN CONVENTIONS
%%%%%%
%
%
\subsection{Sign conventions}\label{sec:sign_conventions}
At certain points throughout this chapter we have had a freedom of choice regarding the signs of some entities, in particular the sign of the signature of the Lorentzian metric $g$ and that of the interior derivative $\delta$. Though at this stage the choice can be made arbitrarily, we want to make it in a way that in the end allows us to make certain physical interpretations on some parameters. More precisely, we want to interpret the parameter $m$ of the Klein-Gordon equation\footnote{or its generalization on $p$-forms} $(\square + m^2) f = 0$ for a zero-form $f \in \Omega^0(\M)$ as a mass in the physical sense. With the chosen sign convention for $\delta$ we find, using ${\delta}f = 0$:
\begin{align}
	\square f
	&= (\delta d + d \delta) f \notag\\
	&= \delta d f \notag\\
	&= - \nabla^\mu \nabla_\mu f \formspace.
\end{align}
In the following heuristic (local) argument we see
\begin{align}
	\square + m^2
	&= -\nabla^\mu \nabla_\mu + m^2 \notag\\
	&\sim \partial_t^2 + \sum_i \partial_i^2 + m^2\notag\\
	&\sim -E^2 + \abs{\vector{p}}^2 + m^2
\end{align}
which yields the correct relativistic relation of energy, momentum and mass according to $E^2 = \abs{\vector{p}}^2 + m^2$.
A similar calculation holds for the Klein-Gordon operator generalized to act on one-forms. If we had found a ``wrong'' relation between energy, momentum and mass, we would have had to adapt the chosen signs. Usually one chooses the sign of the metric and the interior derivative such that they are in some sense mathematically convenient (although one might disagree with another one's choice). We have made the choice of the metric, such that the Cauchy surfaces become Riemannian rather that ``anti-Riemannian'' (with an all minus signature), which seems more natural to some. Also, a lot of the used references on spacetime geometry (in particular the book by \name{Wald} \cite{wald_GR}) use this sign convention, which makes the application of certain formulas easier. As mentioned, the sign of the interior derivative was chosen such that it is formally adjoint to the exterior derivative (with respect the specified inner product) on all Lorentzian and Riemannian manifolds. It seemed convenient for the actual calculations to fix the sign regardless of the signature of the metric of the underlying manifold. One could equivalently have fixed the opposite sign, yielding the two derivatives to be skew-adjoint, which is also done in the literature. However, in the end, one has one freedom left to make the energy-momentum-mass relation work: that is the sign in front of the mass in the Klein-Gordon equation and all other wave equations accordingly. Hence, one regularly also finds the Klein-Gordon equation to be defined with a flipped sign of the mass term. But for our case, we want the mass $m$ in any wave equation to appear with a positive sign.
%
%


%\section{Multiply transitive Galois groups and factor-sets}\label{Sec:MultiTrans}
%\input{multiple_transitivity}

\section{Explicit computations with Amitsur cohomology}\label{Sec:Amitsur}
In this section, we recall the basic definitions of Amitsur cohomology and then use them to get efficient representations of cocycles as introduced in \Cref{Sec:BrauerPrelim}. We also keep notations as they were in \Cref{Sec:BrauerPrelim}.

Although the Amitsur complex and its cohomology are defined for general commutative rings, we will only need to state the definitions over fields and étale algebras. We refer to \cite[Chapter 5]{ford2017separable} for more general definitions and results.

Unless specified otherwise, all tensor products are taken over \(k\), and by \(F^{\otimes n}\) we mean the tensor product of \(n\) copies of the \(k\)-algebra \(F\).

\begin{definition}
    A \(n\)-cochain in the sense of Amitsur is an invertible element in the \(k\)-algebra \(F^{\otimes n+1}\), and we write \(C_{Am}^n(k,F)\) for the group of \(n\)-cochains. 

    For \(n \in \N\) and \(0 \leq i \leq n\), we define a \(k\)-algebra homomorphism \(\varepsilon_i^n\) from \(F^{\otimes n+1}\) to \(F^{\otimes n+2}\). The map \(\varepsilon_i^n\) is defined on the simple tensors as follows:
    \[\varepsilon_i^n(f_0 \otimes \hdots \otimes f_n) = f_0 \otimes \hdots \otimes f_{i-1} \otimes 1 \otimes f_{i} \otimes \hdots \otimes f_n.\]
    We then define the group homomorphism
    \[\Delta^n\colon\begin{array}{ccl} C_{Am}^n(k,F) &\to &C_{Am}^{n+1}(k,F) \\ a &\mapsto &\prod_{i=0}^{n+1} \varepsilon_i^n(a)^{(-1)^i}\end{array}.\]

    Define the subgroup \(Z_{Am}^n(k,F) = \Ker(\Delta_{Am}^n)\) of \(C_{Am}^n(k,F)\), and its elements are called \(n\)-cocycles in the sense of Amitsur.

    If \(n\) is positive, we also define the subgroup \(B_{Am}^n(k,F) = \im(\Delta_{Am}^{n-1})\) of \(Z_{Am}^n(k,F)\) and its elements are \(n\)-coboundaries in the sense of Amitsur. Then, the \(n\)th Amitsur cohomology group \(H_{Am}^n(k,F)\) is defined as the quotient group \(Z_{Am}^n(k,F)/B_{Am}^n(k,F)\).

    Two \(n\)-cocycles \(c_1\) and \(c_2\) are called associated if they have the same class in \(H_{Am}^n(k,F)\).
\end{definition}

\subsection{Amitsur cohomology and central simple algebras}\label{Sec:AmitsurAlgebra}

    The first result we need to leverage Amitsur cohomology in representing Brauer factor sets is the following
    \begin{lemma}\label{lemma:AdamsonAmitsur}
        Consider the map \(\Psi_n\) from \(F^{\otimes n+1}\) to the \(k\)-algebra of \(G\)-homogeneous maps from \(\Phi^{n+1}\) to \(K\), which is defined over simple tensors by
        \[f_0 \otimes \hdots \otimes f_n \mapsto \left(\begin{array}{ccl} \Phi^{n+1} &\to &K^\times \\ (\varphi_{i_0},\hdots,\varphi_{i_n}) &\mapsto &\prod_{j=0}^n \varphi_{i_j}(f_j)\end{array}\right).\]
        The map \(\Psi\) is an isomorphism of \(k\)-algebras, which also sends the unit group \(C_{Am}^n(k,F)\) onto the unit group \(C^n(k,F)\). Furthermore, \(\Psi_{n+1} \circ \Delta_{Am}^n = \Delta^n \circ \Psi_{n}\), and it follows that cocycle and coboundaries subgroup are also isomorphic to one another, and so are cohomology groups.
    \end{lemma}

    \begin{proof}
        If \(F\) is a separable field extension of \(k\), this is the content of \cite[Section 2]{rosenberg1960amitsur}. However, the proof given there readily generalises to the case that \(F\) is an etale \(k\)-algebra.
    \end{proof}

    By the universal property of the tensor products, the \(k\)-algebra maps from \(F^{\otimes n+1}\) to \(K\) are in natural bijection with \(\Phi^{n+1}\). More precisely, if \((\varphi_{i_0},\hdots,\varphi_{i_n})\) and \(f_0,\hdots,f_n \in F^{\otimes n+1}\) is a simple tensor element, then
    \[(\varphi_{i_0},\hdots,\varphi_{i_n})(f_0 \otimes \hdots \otimes f_n) = \prod_{j=0}^n \varphi_{i_j}(f_j).\]

    By the construction of \(\Psi\), it is clear that if \(a \in F^{\otimes n+1}\) and \((\varphi_{i_0},\hdots,\varphi_{i_n})\), then
    \begin{equation}\label{eq:AmitsurEmbeddings}
        \Psi_n(c)_{i_0,\hdots,i_n} = (\varphi_{i_0},\hdots,\varphi_{i_n})(c).
    \end{equation}

    \begin{definition}
        A reduced \(n\)-cocycle in the sense of Amitsur is a cocycle \(c \in Z_{Am}^n(k,F)\) such that \((\varphi_i,\hdots,\varphi_i)(c) = 1\) for all \(i \in [d]\). Equivalently, it is a cocycle \(c\) such that \(\Psi_n(c)\) is reduced.
    \end{definition}

    Adapting \Cref{def:BrauerCSA} to Amitsur cohomology, we get:
    \begin{definition}\label{def:AmitsurCSA}
        Let \(c \in Z_{Am}^2(k,F)\) be a reduced cocycle. Then the \(k\)-algebra \(A(F,c)\) is defined as follows:

        As a \(k\)-vector space, \(A(F,c)\) is \(F^{\otimes 2}\). We see \(F^{\otimes 3}\) as a \(F^{\otimes 2}\)-algebra via the embedding \(\varepsilon_1^2\colon F^{\otimes 2} \to F^{\otimes 3}\). Then, multiplication in \(A(F,c)\) is defined by
        \[a a' = \Tr_{F^{\otimes 3}/F^{\otimes 2}}\left(\varepsilon_2^2(a)c\varepsilon_0^2(a')\right).\]
        In the sequel, we use the same embedding whenever we write \(\Tr_{F^{\otimes 3}/F^{\otimes 2}}\).
    \end{definition}

    \begin{prop}
        Let \(c \in Z_{Am}^2(k,F)\). The map \(\Psi_1\) is a \(k\)-algebra isomorphism from \(A(F,c)\) to \(B(F,\Psi_2(c)\).
    \end{prop}

    \begin{proof}
        The map \(\Psi_1\) is already known to be an isomorphism of vector space between \(A(F,c)\) and \(B(F,\Psi_2(c))\), as the underlying vector spaces of these algebras are respectively \(F^{\otimes 2}\) and the space of \(G\)-homogeneous maps from \(\Phi^2\) to \(K\). We therefore only need to check that the map \(\Psi_1\) is a homomorphism of \(k\)-algebras.
        
        By definition of the trace as the sum of the conjugates of an element, if \(a \in F^{\otimes 3}\) and \(i,j \in [d]\),
        \[(\varphi_i,\varphi_j)\left(\Tr_{F^{\otimes 3}/F^{\otimes 2}}(a)\right) = \sum_{k \in [d]} (\varphi_i,\varphi_k,\varphi_j)(a).\]

        Using \Cref{eq:AmitsurEmbeddings}, we get that if \(a,a' \in A(F,c)\) and \(i,j \in [d]\),
        \begin{align*}
            \Psi_1(aa')_{i,j} &= (\varphi_i,\varphi_j)\left( \Tr_{F^{\otimes 3}/F^{\otimes 2}}\left(\varepsilon_2^1(a)c\varepsilon_0^1(a')\right)\right)\\
            &= \sum_{k \in [d]}(\varphi_i,\varphi_k,\varphi_j)\left(\varepsilon_2^1(a)c\varepsilon_0^1(a')\right) \\
            &= \sum_{k \in [d]}(\varphi_i,\varphi_k,\varphi_j)(\varepsilon_2^1(a))(\varphi_i,\varphi_k,\varphi_j)(c)(\varphi_i,\varphi_k,\varphi_j)(\varepsilon_0^1(a')) \\
            &= \sum_{k \in [d]} \Psi_1(a)_{i,k} \Psi_2(c)_{i,k,j} \Psi_1(a')_{k,j} \\
            &= \Psi_1(a) \Psi_1(a')
        \end{align*}
    \end{proof}
    
    \begin{theorem}\label{thm:IsomAmiAlgebra}
        Let \(A\) be a central simple \(k\) algebra, let \(F \subset A\) be a separable maximal commutative subalgebra, and let \(v \in A\) be such that \(A = FvF\). Then, there is a cocycle \(c \in Z^2_{Am}(k,F)\) such that the map
        \[\begin{array}{rlcl}
            \Phi_c\colon &A(F,c) &\to &A \\
            &r_1 \otimes r_2 &\mapsto &r_1 v r_2
        \end{array}\]
        is an isomorphism. Furthermore, \(c\) is the unique element of \(F^{\otimes 3}\) with this property (if we extend the definition of the multiplication in \(A(F,c)\) to non-cocycle elements).
    \end{theorem}

    \begin{proof}
        The cocycle \(c\) is constructed as a preimage through \(\Psi_2\) of the Brauer factor-set obtained from \(A = FvF\) as in the discussion at the end of \(\Cref{Sec:BrauerPrelim}\). Then, the isomorphism given in the theorem statement is simply the isomorphism \(B(K,\Psi_2(c)) \simeq A\) pulled back through \(\Psi_1\).
        
        The cocycle \(c\) is a solution to the equation
        \[\Tr_{F^{\otimes 3}/F^{\otimes 2}} \left(\varepsilon_2^1(\Phi_c^{-1}(a))c\varepsilon_0^1(\Phi_c^{-1}(a'))\right) = \Phi_c^{-1}(aa')\]
        for all \(a,a' \in A)\). Observe that \(\varepsilon_2^1(\Phi_c^{-1}(a))\varepsilon_0^1(\Phi_c^{-1}(a'))\) covers \(F^{\otimes 3}\) as \(a,a'\) range over \(A\), and then the uniqueness of \(c\) follows from the non-degeneracy of the trace for étale algebras.
    \end{proof}

    \begin{cor}\label{cor:AssoCocyIsomAmitsur}
        Let \(c,c' \in Z_{Am}^2(k,F)\) be reduced associated cocycles, and let \(a \in C_{am}^2(k,F)\) be such that \(c' = c\Delta_{Am}^1(a)\). Then, the map \(m \mapsto ma^{-1}\), where the multiplication used is the natural one in \(F^{\otimes 2}\), is an isomorphism from \(A(F,c)\) to \(A(F,c')\).
    \end{cor}

    \begin{proof}
        This is just a consequence of the analogous property for Brauer factor sets (\Cref{thm:BrauerFSAssociated}), pulled through the isomorphisms \(\Psi_i\).
    \end{proof}

    \begin{remark}
        For the sake of efficiency, we use already proven results on Brauer factor-sets and the isomorphisms \(\Psi_1\) and \(\Psi_2\) to describe the isomorphism between the groups \(H^2_{Am}(k,F)\) and \(\Br(F/k)\). However, the simplicity of the isomorphism \(\Phi_c\) described in \Cref{thm:IsomAmiAlgebra} suggests that Amitsur cohomology is a natural setting for giving an algebraic description of \(\Br(F/k)\) and that direct proofs of these facts wy be given without using references to Brauer factor-sets.
    \end{remark}

    \begin{example}\label{ex:AmiTrivialCocycle}
        Just as in \Cref{ex:ConstantFS}, the cocycle \(1 \in Z_{Am}^2(k,F)\) corresponds to a split algebra. That is, \(A(F,1) \simeq M_d(k)\). Indeed, we observe that \(\varepsilon_1^0(F)\) is a left ideal of dimension \(d\). Since \(F\) has dimension \(d\), we only need to prove that \(\varepsilon_1^0(F)\) is a left ideal. This is a consequence of the fact that \(\varepsilon_0^1(\varepsilon_0^0(F)) \subset \varepsilon_2^1(F^\otimes 2)\) (as is easily observed on simple tensor elements). Then, the multiplication formula ensures that the product \(x\varepsilon_1^0(y)\) for any \(x \in A(F,1)\) and \(y \in F\) lies in \(T_{F^{\otimes 3}/F^{\otimes 2}}(\varepsilon_2^1(F^{\otimes 2})) \subset \varepsilon_1^0(F)\).
    \end{example}

\subsection{Computational representation of Amitsur cohomology}\label{Sec:CompRepAmi}
    Once a field \(k\) and an étale \(k\)-algebra \(F = k[X]/P\) are fixed, we have isomorphisms \[R_n \coloneqq k[X_0,\hdots,X_n]/(P(X_0),\hdots,P(X_n)) \simeq F^{\otimes n+1},\] for all integers \(n \geq 1\). Therefore, an element of \(F^{\otimes n+1}\), and in particular of \(C_{Am}^n(k,F)\), may be represented uniquely as a polynomial \(R(X_0,\hdots,X_n)\) in \(k[X_0,\hdots,X_n]\) whose individual degrees respectively in \(X_0,X_1,\hdots,X_n\) are bounded by \(d-1\). 

    In this setting, the map \(\varepsilon_i^n\) simply translates as the map sending \(Q(X_0,\hdots,X_n)\) to \(Q(X_0,\hdots,X_{i-1},X_{i+1},\hdots,X_n)\). That is, 
    \[\varepsilon_i^n(X_j) = \begin{cases}
        X_j \text{ if } j < i \\
        X_{j+1} \text{ otherwise.}
    \end{cases}
    \]

    Now, the trace of \(R_2\) as an \(R_1\) algebra (via the morphism \(\varepsilon_1^1\)) may easily be computed in the \(R_1\)-basis \((X_1^i)_{0 \leq i < d}\) of \(R_2\). It follows that if \(Q_1,Q_2 \in R_1\) represent elements \(a_1,a_2\) of \(A(F,c)\), where we see \(c\) as an element of \(R_2\), then the element \(Q \in R_2\) representing the product \(a_1a_2\) may be computed practically as:
    \[Q(X_0,X_1) = \Tr_{R_2/R_1}\left(Q_1(X_0,X_1)c(X_0,X_1,X_2)Q_2(X_1,X_2)\right).\]

    
    Recall that \(\varphi_i\) is a map from \(F = k[X]/P\) to \(K\). The map \((\varphi_i,\varphi_i,\varphi_i)\colon F^{\otimes 3} \to K\) factors through \(F\) via
    \[\begin{array}{ccccl} F^{\otimes 3} &\to &F &\to &K \\
        Q(X_0,X_1,X_2) &\mapsto &Q(X,X,X) &\mapsto &Q(\varphi_i(X),\varphi_i(X),\varphi_i(X)) \end{array}\]

    We may describe the explicit computation of a reduced cocycle associated to a given cocycle:
    \begin{theorem}\label{thm:ReducingCocycleExplicit}
        Let \(c \in Z_{Am}^2(k,F)\) be represented by a polynomial \(Q\) as described above. Let \(c' \in F^{\otimes 3}\) be represented by
        \[Q' = \frac{Q(X_0,X_1,X_2)}{Q(X_1,X_1,X_1)}.\]
        Then, \(c'\) is a reduced cocycle associated to \(c\), and the map
        \[\begin{array}{rlcl}
        \rho\colon &A(F,c) &\to &A(F,c') \\
        &R(X_0,X_1) &\mapsto &Q(X_0,X_0,X_0) R(X_0,X_1)
        \end{array}\]
        is an isomorphism of \(k\)-algebras.
    \end{theorem}

    \begin{proof}
        As \(c\) is invertible in \(F^{\otimes 3}\), \(Q(X_0,X_0,X_0)\) is invertible in \(F[X_0]/(P(X_0))\), so the map given is an isomorphism of vector spaces, and we only need to check that it is compatible with multiplication.

        We let \(R(X_0,X_1),R'(X_0,X_1) \in k[X_0,X_1]/(P(X_0),P(X_1))\) and we compute:
        \begin{align*}
            \rho(R) \rho(R') &= \Tr_{F^{\otimes 3}/F^{\otimes 2}}(\varepsilon_2^1(Q(X_0,X_0,X_0)R)Q(X_1,X_1,X_1)^{-1}c\\
            &\varepsilon_0^1(Q(X_0,X_0,X_0)R')) \\
            &= \Tr_{F^{\otimes 3}/F^{\otimes 2}}\left(Q(X_0,X_0,X_0)R(X_0,X_1)Q(X_1,X_1,X_1)^{-1}c\right.\\
            &\left.Q(X_1,X_1,X_1)R'(X_1,X_2)\right)\\
            &= Q(X_0,X_0,X_0) Tr_{F^{\otimes 3}/F^{\otimes 2}}\left(R(X_0,X_1)cR'(X_1,X_2)\right)\\
            &= Q(X_0,X_0,X_0) RR' \\
            &= \rho(RR')
        \end{align*}
    \end{proof}
\subsection{Computing a cocycle representing a given algebra}\label{Sec:AmitsurAlgebraConstruction}

    We may now give an algorithm for finding a representation of a central simple algebra \(A\) over any field of sufficiently large size where linear algebra tasks may be performed efficiently.

    \begin{algorithm}
        \caption{Computing a \(2\)-cocycle representing a given central simple algebra}
        \label{algo:FindCocycle}
        \begin{algorithmic}[1]
            \REQUIRE A field \(k\)
            \REQUIRE A central simple \(k\)-algebra \(A\) such that \(|k| > \dim_k A\).
            \STATE Find \(u \in A\) such that \(F \coloneqq k[u]\) is a maximal separable commutative subalgebra of \(A\) \label{algoline:Findu}
            \STATE Compute \(P\), the minimal polynomial of \(u\)
            \STATE Find \(v \in A\) such that \(A = FvF\) \label{algoline:Findv}
            \STATE Compute the matrix of the isomorphism \(e: F^{\otimes 2} \to A\) sending \(f_1 \otimes f_2\) to \(f_1vf_2\)
            \STATE Using linear algebra, find \(c \in F^{\otimes 3}\) such that for all \(a,b \in F^{\otimes 2}\), \(e(a)a(b) = \Tr_{F^{\otimes 3}/F^{\otimes 2}}(\varepsilon_2^1(a)c\varepsilon_0^1(b))\). \label{algoline:Findc}
            \STATE Compute \(c' = c/c(X_1,X_1,X_1)\)
            \STATE Compute \(e'\) as \(e\) precomposed with \(R \mapsto c(X_0,X_0,X_0)^{-1} R\). 
            \RETURN \((u,c',e')\)
        \end{algorithmic}
    \end{algorithm}

    Before we prove the correctness efficiency of \cref{algo:FindCocycle}, we need a lemma:

    \begin{lemma}\label{lemma:Findv}
        Let \(k\) be a field and let \(A\) be a central simple \(k\)-algebra. Assume that \(|k| > \dim_k A\). Let \(u \in A\) be such that \(F \coloneqq k[u]\) is a maximal commutative subalgebra of \(A\). Then an element \(v \in A\) such that \(A = FvF\) may be found in probabilistic polynomial time.
    \end{lemma}

    \begin{proof}
        For \(v\) in \(A\), by an argument of dimensions over \(k\), we observe that \(A = FvF\) if and only if the map
        \[e\colon\begin{array}{ccl} F \otimes F &\to &A \\ f_1 \otimes f_2 &\mapsto &f_1vf_2 \end{array}\]
        is injective.

        Fixing the basis \((u^i \otimes u^j)_{0 \leq i,j \leq \deg A - 1}\) and any basis \(B = (b_1,\hdots,b_{\dim A})\) of \(A\), the determinant of \(e\) is a homogeneous polynomial on the coordinates of \(v\) with respect to \(B\), and \(A = FvF\) if and only if \(v\) is not a zero of this polynomial.

        Letting \(S\) be a finite subset of \(k\), the Schwartz-Zippel lemma ensures that a random \(v\) in \(Sb_1 \oplus \hdots \oplus Sb_{\dim A}\) satisfies this condition with probability larger than \(1 - \frac{\dim A}{|S|}\).

        Therefore, if \(|k| > \dim A\), we may pick \(S\) large enough that \(v\) has the desired property with positive probability. For instance, take \(|S| = 2\dim A\) and \(v\) has the desired property with probability larger than \(\frac{1}{2}\).
    \end{proof}
    
    \begin{theorem}\label{thm:AlgoFindCocycle}
        If \(k\) is a field over which linear algebra may be performed efficiently, and \(A\) is a central simple \(k\)-algebra such that \(|k| > \dim_k A\), then \Cref{algo:FindCocycle} returns \(u \in A\), a cocycle \(c \in Z_{Am}^2(k,k(u))\) and an isomorphism \(e\colon A(F,c) \to A\) in probabilistic polynomial time.
    \end{theorem}

    \begin{proof}
         We first prove the correctness of the algorithm. The first 3 steps compute \(u,v \in A\) as discussed above \Cref{thm:IsomAmiAlgebra}. We obtain a subalgebra \(F\) of \(A\) and \(v \in A\) such that \(A = FvF\). Then, by \Cref{thm:IsomAmiAlgebra}, there exists a cocycle \(c \in Z_{Am}^2(k,F)\) such that \(a \otimes b \mapsto avb\) is an isomorphism from \(A(k,F)\) to \(A\), and such a \(c\) is unique in \(F^{\otimes 3}\). The equation solved in \Cref{algoline:Findc} finds this \(c\).

        Finally, the last two lines perform the operation described in \Cref{thm:ReducingCocycleExplicit} for computing a reduced cocycle associated to \(c\), and the isomorphism from \(A(F,c)\) to \(A(F,c')\). Putting everything together, \cref{algo:FindCocycle} \(c\) is indeed a reduced \(2\)-Amitsur cocycle for \(F = k(u)\) and \(e\) gives an isomorphism \(A(F,c) \simeq A\).

        Now we consider the complexity of the algorithm:
        
        The element \(u \in A\) in \Cref{algoline:Findu} may be found using the polynomial algorithm given in \cite{graaf2000finding}.

        Then, \(v\) in \Cref{algoline:Findv} may be found in probabilistic polynomial time using the algorithm of \Cref{lemma:Findv} below.

        Then, the remaining lines involve arithmetic in \(A\) and bounded tensor powers of \(F\), as well as the computation of the solution of a linear system.
        
        All in all, this makes \Cref{algo:FindCocycle} a polynomial probabilistic algorithm.
    \end{proof}


\section{Trivialisation of Amitsur cocycles}\label{Sec:Trivialisation}
In this section, we present an algorithm for computing the trivialisation of a coboundary using \(S\)-units group computation. This result is inspired by results such as Simon's algorithm for solving norm equation in cyclic extensions \cite{simon2002solving} and Fieker's result on finding trivialisation of Galois coboundaries in groups of \(S\)-units \cite[Theorem 7]{fieker2009minimizing}.

Our strategy is to prove a vanishing lemma for the first Amitsur cohomology group with coefficients in the divisor group. Such a result is analogous to \cite[Lemma 7]{fieker2009minimizing} and allows us to adapt the proof strategy to our setting.

Let \(k\) be a number field. We let the divisor group \(\Dc(k)\) be the free Abelian group on the set \(\Pl(k)\) of finite places of \(k\). The group \(\Pc(k)\) of principal divisors is the image of the natural injection of \(k\) into \(\Dc(k)\), and it is well known that the quotient \(\Cl_k = \Dc(k)/\Pc(k)\) is finite.

We extend these definition to étale algebras over number fields:
\begin{definition}\label{def:EtaleDivisors}
    Let \(A\) be an étale \(k\)-algebra, for a number field \(k\). Then \(A\) splits as a direct sum
    \[A \simeq \bigoplus_{i \in I} K_i\]
    of finite separable extensions of \(k\). Furthermore, this decomposition is unique up to reordering and internal isomorphism of the fields. We define \(\Dc(F)\) as the free Abelian group over \(\Pl(F)\), the disjoint union of the sets \(\Pl(K_i)\).
\end{definition}

    While \(\Dc\) is not a functor, we may define \(\Dc(f)\) whenever \(f\) is a monomorphism in the category of étale \(k\)-algebras. Indeed, if \(f\colon A \to A'\), for each place of \(A\) belonging to a direct factor \(K\), the map \(f\) sends \(K\) to one of the direct factors \(K'\) of \(A'\) and we may then define \(f(P) \in \Dc(K') \subset \Dc(A')\) as in the case of field extensions.
    
    This yields a complex
    \[\hdots \to \Dc(F^{\otimes n+1}) \xrightarrow{\Dc(\Delta_{Am}^{n})} \Dc(F^{\otimes n+2}) \to \hdots.\]
of abelian groups, where we define \(\Dc(\Delta_{Am}^n) = \sum_{0 \leq i \leq n} (-1)^i \Dc(\epsilon^n_i)\), using the fact that the \(\epsilon\) maps are injective.

For the remainder of this section, we let \(F\) be an étale \(k\)-algebra.

We give a precise description of the map \(\Dc(\Delta_{Am}^n)\). Let \(Q \in \Pl(F^{\otimes n+2})\). Then, for any \(0 \leq i \leq n+1\), there is exactly one \(P \in \Pl(F^{\otimes n+1})\) such that \(Q \mid \epsilon_i^n(P)\), and we call this place \(Q_i\). Then, if \(P \in \Pl(F^{\otimes n+1})\), we get
\[\Dc(\epsilon_i^n)(P) = \sum_{\substack Q \in \Pl(F^{\otimes n+2}) \\ Q_i = P} e_{Q,i} Q,\]
where \(e_{Q,i}\) is the ramification index of \(Q\) over \(\epsilon_i^n\colon F^{\otimes n+1} \to F^{\otimes n+2}\), and it follows that 
\[\Dc(\epsilon_i^n)\left(\sum_{P \in \Pl(F^{\otimes n+1})} n_P P\right) = \sum_{Q \in \Pl(F^{\otimes n+2})} e_{Q,i} n_{Q_i} Q\]
and
\[\Dc(\Delta_{Am}^n)\left(\sum_{P \in \Pl(F^{\otimes n+1})} n_P P\right) = \sum_{Q \in \Pl(F^{\otimes n+2})} \left(\sum_{0 \leq i \leq {n+1}} (-1)^i e_{Q,i} n_{Q_i} \right) Q.\]

We first need two lemmas:

\begin{lemma}\label{lemma:CocycleTransit}
    Let \(Q,Q'\) be places of \(F^{\otimes 2}\) such that \(Q_0 = Q'_0 = P\). Then there exists a place \(R \in \Pl(F^{\otimes 3})\) such that \(R_1 = Q\) and \(R_2 = Q'\).
\end{lemma}

\begin{proof}
    We set \(P = Q_0 = Q'_0\), and consider the local field \(F_P\) obtained by first selecting the direct factor of \(F\) to which \(P\) belongs and then taking the completion at \(P\). The set of places of \(F^{\otimes 2}\) that divide \(\epsilon_1^0(P)\) is in bijection with the set of direct factors of \(F_P \otimes_F F^{\otimes 2} \simeq F_P \otimes_k F\). Likewise, the set of places of \(F^{\otimes 3}\) above \(P\) (for the embedding \(\epsilon_1^1 \circ \epsilon_1^0 = Id_F \otimes 1 \otimes 1\)) is in bijection with the set of direct factors of \(F_P \otimes_k F \otimes_k F\). Furthermore, the maps \(\epsilon_1^1\) and \(\epsilon_1^2\) may be expanded to \(F_P \otimes_k F\), and if \(R \in \Pl(F^{\otimes 3})\) is the place corresponding to some direct factor \(K\) of \(F_P \otimes F \otimes F\), then, the place \(R_1\) (resp. \(R_2\)) corresponds to the factor of \(F_P \otimes F\) which is mapped into \(K\) by \(\epsilon_1^1\) (resp. \(\epsilon_1^2\)).

    Now, let \(\alpha \in k[X]\) be a defining polynomial for \(F\). We have isomorphisms \(F_P \otimes F \simeq F_P[X]/(\alpha(X))\) and \(F_P \otimes F \otimes F \simeq F_P[X,Y]/(\alpha(X),\alpha(Y))\). A direct factor of \(F_P \otimes F\) corresponds to an irreducible factor of \(\alpha\) in \(F_P[X]\). Likewise, a factor of \(F_P \otimes F \otimes F\) is uniquely described by the choice of an irreducible factor \(\beta\) of \(\alpha(X)\) in \(F_P[X]\) and an irreducible factor \(\gamma\) of \(\alpha(Y)\) in \(\left(F_P[X]/\beta(X)\right)[Y]\). We note that each factor \(\gamma\) is obtained as a factor of some \(\beta'\), itself an irreducible factor of \(\alpha\) in \(F_P[Y]\).

    Furthermore, if \(R \in \Pl(F^{\otimes 3})\) corresponds to some factors \(\beta\) and \(\gamma\) of \(\alpha\), then \(R_2\) is the place of \(F^{\otimes 2}\) corresponding to \(\beta\) and \(R_1\) is the place of \(F^{\otimes 2}\) corresponding to the irreducible factor \(\beta'\) of \(\alpha\) in \(\F_P[X]\) corresponding to \(\gamma\) as described above.

    Now, we fix \(\beta\) and \(\beta'\) the irreducible factors of \(\alpha\) corresponding respectively to \(Q\) and \(Q'\). We let \(\gamma\) be an irreducible factor of \(\beta'(Y)\) in \(\left(F_P[X]/\beta(X)\right)[Y]\) and the place \(R\) corresponding to \(\beta\) and \(\gamma\) is as demanded.
\end{proof}

We may now prove a generalized version of Hilbert's theorem 90 in our setting. We say that a place \(P \in \Pl(F)\) is unramified if, for all \(Q \in \Pl(Q)\), if \(F = Q_0\) then \(e_{Q,0} = 1\) and if \(F = Q_1\) then \(e_{Q,1} = 1\).
\begin{lemma}\label{lemma:Hilbert90}
    Let \(D = \sum_{Q \in \Pl(F^{\otimes 2})} n_Q Q \in \Ker \Dc(\Delta_{Am}^1)\) be supported by unramified places. That is, for all \(Q \in \Pl(F^{\otimes 2})\), if \(Q_0\) or \(Q_1\) is ramified, then \(n_Q = 0\). Then, there exists \(E \in \Dc(F)\) such that \(D = \Dc(\Delta_{Am}^0)(E)\).
\end{lemma}

\begin{proof}
    We set
    \[E = \sum_{P \in \Pl(F)} \left(\min_{\substack{Q \in \Pl(F^{\otimes 2}) \\ Q_0 = P}} n_Q \right) P.\]
    Then, we get 
    \[\Dc(\epsilon_0^0)(E) = \sum_{Q \in \Pl(F^{\otimes 2})} \left(\min_{\substack{Q' \in \Pl(F^{\otimes 2}) \\ Q'_0 = Q_1}}  n_{Q'}\right) Q\]
    and
    \[\Dc(\epsilon_1^0)(E) = \sum_{Q \in \Pl(F^{\otimes 2})} \left(\min_{\substack{Q' \in \Pl(F^{\otimes 2}) \\ Q'_0 = Q_0}}  n_{Q'}\right) Q\]

    It follows that
    \[D + \Dc(\epsilon_0^0)(E) = \sum_{Q \in \Pl(F^{\otimes 2})} \left(\min_{\substack{Q' \in \Pl(F^{\otimes 2}) \\ Q'_0 = Q_1}}  n_Q + n_{Q'}\right) Q.\]
    If we fix places \(Q,Q' \in \Pl(F^{\otimes 2})\) such that \(Q'_0 = Q_1\), we apply \cref{lemma:CocycleTransit} to \(Q'\) and \(Q^\sigma\), the image of \(Q\) by the automorphism \(\sigma\colon a \otimes b \mapsto b \otimes a\) of \(F^{\otimes 2}\). We find that there exists \(R \in \Pl(F^{\otimes 3})\) such that \(R_1 = Q'\) and \(R_2 = Q^\sigma\). Then, consider \(R^\tau\), the image of \(R\) by the automorphism \(\tau \colon a \otimes b \otimes c \mapsto b \otimes a \otimes c\) of \(F^{\otimes 3}\). We get \(R^\tau_2 = Q\) and \(R^\tau_0 = Q'\). We may then set \(Q'' = R^\tau_1\) and, as \(\Dc(\Delta_{Am}^1)(D) = 0\), we get that \(n_Q + n_{Q'} = n_{Q''}\), for some \(Q''\) such that \(Q''_0 = Q_0\).

    Conversely, if we fix \(Q,Q'' \in \Pl(F^{\otimes 2})\) such that \(Q_0 = Q''_0\), then there exists \(R \in \Pl(F^{\otimes 3})\) such that \(R_2 = Q\) and \(R_1 = Q''\). We set \(Q' = R_0\) and we get that \(n_Q + n_{Q'} = n_{Q''}\), and observe that \(Q'_0 = Q_1\).

    This shows that for \(Q \in \Pl(F^{\otimes 2})\), \[\min_{\substack{Q' \in \Pl(F^{\otimes 2}) \\ Q'_0 = Q_1}}  n_Q + n_{Q'} = \min_{\substack{Q' \in \Pl(F^{\otimes 2}) \\ Q'_0 = Q_0}}  n_{Q'}.\]
    Therefore, \(D + \epsilon_0^0(E) = \epsilon_1^0(E)\). That is, \(D = \Dc(\Delta_{Am}^0)(-E)\).
\end{proof}

In what follows, if \(S\) is a set of places of \(F\), we define inductively \(S^{(1)} = S\) and
\[S^{(i+1)} = \left\{Q \in \Pl(F^{\otimes i+1}) \mid \exists 0 \leq j \leq i, P \in S^{(i)} \colon Q|\varepsilon_j^{i-1}(P)\right\}.\]
We now get our main theorem for this section:
\begin{theorem}\label{thm:SUnitTriv}
    Let \(b \in B_{Am}^2(k,F)\) be a coboundary. Let \(S\) be a finite set of places of \(F\) such that:
    \begin{itemize}
        \item \(S\) contains the infinite places of \(F\).
        \item \(S\) contains the places of \(F\) that ramify in \(F^{\otimes 2}\).
        \item The finite places of \(S\) generate the class group \(\Cl(F)\).
        \item The places in the support of \(b\) are contained in \(S^{(3)}\).
    \end{itemize}
    Then there exists a cochain \(\sigma\) in the group of \(S^{(2)}\)-units of \(F^{\otimes 2}\) such that \(b = \Delta_{Am}^1(\sigma)\)
\end{theorem}

\begin{proof}
 Let \(\alpha \in (F^{\otimes 2})^\times\) be such that \(\Delta_{Am}^1(\alpha) = b\). We consider the divisor \(D = \Dc(\alpha) = \sum_{Q \in \Pl(F^{\otimes 2})} n_Q Q\) of \(\alpha\). We set \(D_S = \sum_{Q \in S^{(2)}} n_Q Q\) and \(D_{\bar{S}} = \sum_{Q \notin S^{(2)}} n_Q Q\). Now, \(\Dc(\Delta_{Am}^1)(D)\) is the divisor of \(F^{\otimes 3}\) corresponding to \(b\) and therefore is supported by \(S^{(3)}\). Observe that if \(Q \in S^{(2)}\), then \(\Dc(\Delta_{Am}^1)(Q)\) has support in \(S^{(3)}\). It follows that \(\Dc(\Delta_{Am}^1)(D_{\bar{S}}) = 0\).

 The divisor \(D_{\bar{S}}\) has no ramified place of \(F^{\otimes 2}\) in its support. We may therefore apply \cref{lemma:Hilbert90} and get a divisor \(E \in \Dc(F)\) such that \(D_{\bar{S}} = \Delta_{Am}^0(E)\). Now, as \(S\) generates the class group of \(F\), there exists \(E' \in \Dc(F)\) with support in \(S\) and \(\gamma \in F^\times\) such that \(E = \Dc(\gamma) + E'\). Then, we get that
 \[\Dc(\Delta_{Am}^0)(\Dc(\gamma)) + \Dc(\Delta_{Am}^0)(E') = D - D_{\bar{S}}\]
 and therefore
 \[\Dc(\Delta_{Am}^0)(E') + D_{\bar{S}} = \Dc(\alpha \Delta_{Am}^0(\gamma^{-1})).\]
 Now, this shows that \(\alpha \Delta_{Am}^0(\gamma^{-1})\) is a \(S^{(2)}\)-unit. Furthermore, \[\Delta_{Am}^1(\alpha\Delta_{Am}^0(\gamma^{-1})) = \Delta_{Am}^1(\alpha) = b,\]
 and \(\alpha \Delta_{Am}^0(\gamma^{-1})\) is a cochain with the required properties.
\end{proof}

From \cref{thm:SUnitTriv} we directly get an algorithm for computing a trivialisation of a \(2\)-coboundary:

\begin{algorithm}
    \caption{Computing a trivialisation of a \(2\)-coboundary}
    \label{algo:TrivCobound}
    \begin{algorithmic}[1]
        \REQUIRE A number field \(k\), an étale \(k\)-algebra \(F\)
        \REQUIRE A coboundary \(b \in B_{Am}^2(k,F)\)
        \STATE Compute \(S_1\), the set of places of \(F\) that ramify in \(F^{(2)}\).
        \STATE Compute \(S_2\), a set of places of \(F\) which generate the class group \(\Cl(F)\).
        \STATE Compute the divisor of \(F^{\otimes 3}\) corresponding to \(b\). Let \(S_3\) be the set of places of \(F\) below the places in the support of \(b\).
        \STATE Set \(S = S_1 \cup S_2 \cup S_3\).
        \STATE Compute the sets \(S^{(2)}\) and \(S^{(3)}\).
        \STATE Compute an isomorphism \(\phi\) from the group of \(S^{(2)}\)-units of \(F^{\otimes 2}\) to \(\Z^r \oplus \Z/m\Z\).
        \STATE Compute an isomorphism \(\psi\) from the group of \(S^{(3)}\)-units of \(F^{\otimes 3}\) to \(\Z^{r'} \oplus \Z/m'\Z\).
        \STATE Solve the linear equation \((\psi \circ \Delta_{Am}^1 \circ \phi^{-1})(\alpha) = \psi(b)\)
        \RETURN \(\alpha\)
    \end{algorithmic}
\end{algorithm}

\begin{theorem}\label{thm:AlgoTrivCobound}
    Given a number field \(k\), and étale \(k\)-algebra \(F\) and a coboundary \(b \in B^2_{Am}(k,F)\), \cref{algo:TrivCobound} outputs a cochain \(\alpha \in C^1(k,F)\) such that \(\Delta_{Am}^1(\alpha) = b\). Furthermore,  \cref{algo:TrivCobound} runs in polynomial time on a quantum computer.
\end{theorem}

\begin{proof}
    Using a polynomial-time algorithm for factoring polynomials over number fields \cite{lenstra1983factoring}, one may compute splitting isomorphisms 
    \[F \simeq \bigoplus_\alpha F_\alpha,\]
    \[F^{\otimes 2} \simeq \bigoplus_\beta F^{(2)}_\beta,\]
    and
    \[F^{\otimes 3} \simeq \bigoplus_\gamma F^{(3)}_\gamma.\]

    Using these isomorphisms, the steps of the computation of \(S\) may be done over number field extensions. Then, this entails computing and factoring relative discriminants, computing class groups and factoring \(b\) in the ideal group of \(F^{\otimes 3}\). The sets \(S^{(2)}\) and \(S^{(3)}\) are computed by factoring images \(\epsilon_i^0(P)\) for \(P \in S\) and then \(\epsilon_i^1(Q)\) for \(Q \in S^{(2)}\). Then, isomorphisms \(\phi\) and \(\psi\) are computed using an algorithm for computing \(S\)-unit groups. Finally, the last step is the computation of a solution of an integral linear system.
    Each of these tasks can be accomplished in quantum polynomial time according to Theorem \ref{thm:s-unit&norm}. 
    The correctness of the algorithm relies on the fact that a cochain \(\alpha\) such that \(b = \Delta_{Am}^1(\alpha)\) exists and may be found in the group of \(S^{(2)}\)-units, which is the content of \cref{thm:SUnitTriv}.
\end{proof}

\begin{cor}\label{cor:QuantumSplit}
    There exists a polynomial quantum algorithm which, give a number field \(k\) and an algebra \(A \simeq M_n(k)\), computes an explicit algorithm from \(A\) to \(M_n(k)\).
\end{cor}

\begin{proof}
    This is simply a combination of \cref{thm:AlgoFindCocycle,thm:AlgoTrivCobound}. Indeed, using \Cref{algo:FindCocycle}, one may compute an étale \(k\) algebra \(F\) and a cocycle \(c \in \Z^2(k,F)\) representing \(A\). As \(A\) is isomorphic to \(M_n(k)\), the cocycle \(c\) is in fact a coboundary. Then, a cochain \(\alpha \in C^1(k,F)\) such that \(\Delta_{Am}^1(\alpha) = c\) may be computed using \cref{algo:TrivCobound}. Applying \Cref{cor:AssoCocyIsomAmitsur}, we obtain an explicit isomorphism \(A \simeq A(F,1)\). Finally, an isomorphism \(A(F,1) \simeq M_n(k)\) may easily be computed using the left ideal provided in \Cref{ex:AmiTrivialCocycle}.
\end{proof}
\begin{remark}
For quaternion algebras $(a,b)_K$ it is known that splitting is equivalent to solving the norm equation $N_{\mathbb{Q(\sqrt{a})}|\mathbb{Q}}(x)=b$ thus one can also directly apply Theorem \ref{thm:s-unit&norm} to find an explicit isomorphism to $M_2(K)$. A similar statement can be derived for degree three central simple algebras as there is a polynomial-time algorithm for finding a cyclic algebra presentation. 
\end{remark}

%\section{Finding trivialisations of Brauer factor-sets}\label{Sec:Trivialisation}
%\input{trivialisation}

%\section{Random elements of maximal orders of matrix algebras}\label{Sec:Random}
%\input{random}

%\section{The main algorithm}\label{Sec:Algo}
%\begin{algorithm}[!ht]
\begin{algorithmic}[1]
\Require Query workload $Q$, event stream $I$, \app\ graph $G$, hash table of snapshots $S$
\Ensure Hash table of results $R$ 
\State $G \leftarrow \emptyset$, $S, R \leftarrow$ empty hash tables
\ForAll {event $e \in I$ with $e.type=E$} 
    \State $//$ \textbf{\app\ graph construction}
    \ForAll {$q \in Q$ \text{ with event types }T}
        \ForAll {$E' \in T,\ E' \neq E$}
            \State $G_{E'} \leftarrow \mathit{getGraphlet}(G,E')$,
            $G_{E'}.\mathit{active} \leftarrow \mathit{false}$
        \EndFor
    \EndFor
    \If {\textbf{not} $G_E.\mathit{active}$}
        \State $G_E \leftarrow \mathit{createGraphlet()}$, $G_{E}.\mathit{active} \leftarrow \mathit{true}$,
        $G \leftarrow G \cup G_E$
        \If {$G_E.\mathit{shared}$ by $Q_E \subseteq Q$}
            $x \leftarrow \mathit{createSnapshot()}$ 
            \ForAll {$q \in Q_E$}
                \ForAll{$E' \in \mathit{pt}(E,q), E' \neq E$}
                    \State $G_{E'} \leftarrow \mathit{getGraphlet}(G,E')$
                    \State $S(x,q) \leftarrow S(x,q) + sum(G_{E'},q)$ \hspace{0.5cm}$//$ Eq.~5
                \EndFor
            \EndFor
        \EndIf    
    \EndIf
    \State insert $e$ into $G_E$
    \State $//$ \textbf{Trend count computation}
    \If {$G_E.\mathit{shared}$ by $Q_E \subseteq Q$}
        \If {$\forall q \in Q_E\ pe(e,q)$ are identical}
            \State $count(e,Q_E) \leftarrow count(e,q)$ \hspace{2.3cm}$//$ Eq.~2
        \Else\ $y \leftarrow \mathit{createSnapshot()}$, $count(e,Q_E) = y$
            \ForAll {$q \in Q_E$}
                $S(y,q) \leftarrow count(e,q)$ \hspace{0.2cm}$//$ Eq.~2
            \EndFor
          \EndIf
    \Else\ $count(e,q)$ \hspace{5.2cm}$//$ Eq.~2
    \EndIf
    \ForAll{$q \in Q$}
  	    \If {$E \in \mathit{end}(q)$} 
  		    $R(q) \leftarrow R(q) + count(e,q)$ $//$ Eq.~3
        \EndIf
    \EndFor
\EndFor
\State \Return $R$
\end{algorithmic}
\caption{\app\ shared online trend aggregation}
\label{algo:snapshot-propagation}
\end{algorithm}


\bibliographystyle{plain}
\bibliography{biblio}
\end{document}
