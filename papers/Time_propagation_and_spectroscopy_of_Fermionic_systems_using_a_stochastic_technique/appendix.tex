\section{Appendix}
\subsection{Recap of the FCIQMC method}

The FCIQMC method \cite{BAT2009,BA2010,CBA2010} is a projector quantum Monte Carlo method based on 
the imaginary-time Schr\"odinger equation. 
It has the stationary form
\beq
  {\partial\over \partial \tau} |\Psi\rangle= -(\hat{H}-E_0)|\Psi\rangle = 0,  
\eeq 
with formal solution:
\beq
  |\Psi(\tau) \rangle = e^{-\tau (\hat{H}-E_0)} |\Psi(0) \rangle
\eeq 
which converges (up to a normalization constant) to the ground state $|\Psi_0\rangle$ of $\hat{H}$ in the large $\tau$ limit. 
We define a first-order propagator $\hat{P}$ as
\begin{equation}
  \hat{P}   =   \mathds{1}-\Delta \tau (\hat{H}-S\mathds{1}),   
\end{equation}
where $\Delta\tau$ is a time-step and $S$ an energy shift to control the
walker number.
If $\hat{H}$ has a finite spectral width $W$,
repeated application leads to the ground-state 
\beq
  |\Psi(n \Delta \tau)\rangle  & = & \hat{P}^n |\Psi(0) \rangle \\
  \lim_{n\rightarrow\infty} | \Psi(n \Delta \tau) \rangle  & \propto&  |\Psi_0 \rangle, \nonumber
\eeq 
without a time-step error, if $\Delta\tau$ is smaller than
$\frac{2}{W}$. $|\Psi(\tau)\rangle $ is expressed as a linear combination of a
complete set of basis states ${|D_i\rangle}$ 
\beq
    |\Psi(\tau) \rangle = \sum_i C_i(\tau) | D_i \rangle
\eeq
In FCIQMC, the coefficients  $C_i$  are replaced by 
an ensemble of  positive and negative {\em walkers}:
\beq
   C_i \propto N_i =  \sum_w^{N_w} s_w \delta(i-i_w) \label{deltafunc}
\eeq
where $s_w=\pm 1$ is the sign of the walker $w$, residing on Slater determinant $i_w$. $N_w$ is the number of walkers. 
The walkers evolve according  to stochastic rules
\begin{itemize}
\item A spawning step: a given walker, on $|D_i\rangle$, randomly selects another connected determinant, $|D_j\rangle$ with probability
$p_{gen}(j|i)$. It then attempts to spawn a new walker on $|D_j\rangle$ with probability $p_s=-\Delta \tau  H_{ij} /p_{gen}(j|i)$. 
\item A death/cloning step: A walker on $D_i$ attempts to die with probability $p_d = \Delta \tau(H_{ii}-S) $. 
\end{itemize}
In a following step, walkers with opposite signs cancel each other, which is essential for addressing the sign-problem. 
In the initiator version of the algorithm \cite{CBA2010,CBA2011,BTCA2012} the spawning is restricted. If the target determinant is not occupied by another walker, the spawning  is aborted if 
$|N_i|\le n_a$, where $n_a$ is the initiator parameter. This condition is
crucial for obtaining a smooth convergence without too many walkers.               

For the calculation of reduced density matrices (RDM), we use the replica method \cite{OBCA2014}, 
in which two independent simulations of walkers are propagated and elements of
the RDMs are being calculated by taking products 
involving the two replicas.

The main advantage of the FCIQMC algorithm compared to conventional exact diagonalization is that
the number of walkers needed for convergence is much smaller than the
dimension of the Hilbert space, thereby requiring drastically less
memory. Using this technique, molecular and condensed-matter systems involving
Hilbert spaces of over $10^{20}$ Slater determinants have been computed \cite{Daday2012,Shepherd2012}.

\subsection{Norm conservation}
Compared to the pure imaginary time evolution, the complex exponential in the
real-time formulation does not cause an exponential decay of contributions
from excited states, but instead gives a complex phase to the walkers, which
requires the use of both real and imaginary walkers for each determinant. Here, real and
imaginary populations are only coupled via the stochastic application of the
first-order expanded propagator
\begin{equation}
\hat{U}_1(t) = 1 - \mathrm{i}\hat{H}t\,.
\end{equation}
The annihilation step is performed separately for each of the populations.

The direct use of $\hat{U}_1$ in the time propagation leads to an exponentially increasing  
wave function, and therefore severely violates norm-conservation of unitary dynamics. 
This can be seen by considering the time evolution of a wave function $\Psi=\Psi_0$ that is already an eigenstate of the 
Hamiltonian with energy $E$.
The exact solution is 
\begin{equation}
  |\Psi(t)\rangle = e^{-\mathrm{i}Et} |\Psi_0 \rangle   
\end{equation}
According to the first-order propagator, after $n$ application of $\hat{U}_1$
we obtain:
\begin{equation}
|\Psi(t_n)\rangle = (1 - \mathrm{i} E \Delta t)^n |\Psi_0 \rangle  \,,
\end{equation}
with $t_n = n \Delta t$, we obtain:
\begin{eqnarray}\label{eq:rungekutta}
  {\rm ln}{\Psi(t_n)\over \Psi_0}  =  n {\rm ln} (1 - \mathrm{i} E \Delta t) 
   \approx & -\mathrm{i} E t_n + {E^2 t_n  \Delta t\over 2}                              
\end{eqnarray}
so that:
\begin{equation}\label{eq:rungekutta1}
|\Psi(t_n)\rangle = e^{-\mathrm{i} E t_n} e^{E^2 t_n \Delta t/2 } |\Psi_0 \rangle    
\end{equation}
which is exponentially growing in time, with an exponent ${\cal O}(\Delta t)$.                                            
This is a direct consequence of working with real time, which introduces a growing exponential factor 
in Eq.~(\ref{eq:rungekutta1}).

This problem can be greatly suppressed using a second-order short-time propagator. 
Defining:
\beq
\hat{U}_2(\Delta t) = \mathds{1} - \mathrm{i}\Delta t\hat{H} - \frac{1}{2}(\Delta t)^2 H^2
\eeq

The time evolution is implemented using a second-order Runge-Kutta
algorithm, which decomposes $\hat{U}_2$ into two steps:
\begin{equation}
\hat{U}_2(\Delta t) = \mathds{1} + \left(\hat{U}_1(\Delta t) -
  \mathds{1}\right)\hat{U}_1\left(\frac{\Delta t}{2}\right)\,.
\end{equation}
The second order propagator is applied by first applying
$\hat{U}_1\left(\frac{\Delta t}{2}\right)$ to the wavefunction, followed by applying
$\left(\hat{U}_1(\Delta t) - \mathds{1}\right)$ to the result and finally adding
the resulting wavefunction to the original one. In this way, $\hat{H}^2$ is not
explicitly applied, which is highly advantageous for the efficiency of the method.

We now have after $n$ repetitions of $\hat{U}_2$: 
\beq
|\Psi(t_n)\rangle = [1 - i E \Delta t - \frac{(E \Delta t) ^2}{2} ]^n |\Psi_0  \rangle  
\eeq
resulting in: 
\begin{eqnarray}\label{eq:second}
  \ln {\Psi(t_n)\over \Psi_0} & = & n \ln [1 - \mathrm{i} E \Delta t -  \frac{(E \Delta t)^2}{2} ] \\
  & \approx&  -\mathrm{i} E t_n  - \mathrm{i} \frac{  E^3 t_n  (\Delta t)^2}{6} + 
      \frac{E^4 t_n  (\Delta t)^3}{8}        \nonumber           
\end{eqnarray}
\beq
|  \Psi(t_n)\rangle = e^{-\mathrm{i} E t_n} e^{-\mathrm{i} E^3 t_n (\Delta t)^2/6} e^{E^4 t_n (\Delta )^3/8} |\Psi_0 \rangle  \nonumber                
\eeq
i.e. in this formulation the norm-violating factor grows only as ${\cal O}(\Delta t^3)$.       
It is possible to reduce further the scaling of norm-violation 
by employing a 4-th order propagator, but we found that improvements are typically masked           
by much larger stochastic errors. 


\begin{figure}[t]
\includegraphics[width=0.8\columnwidth]{norm_plots_panel.pdf}
\caption{Number of walkers $N_W$, norm of the wave function, and Green's function 
for the 10-site Hubbard model,  $k=(0,0)$ and  $U/t=1$, using the stochastic algorithm
 and the first and second order Runge-Kutta methods.}
\label{fig:firstOrder}
\end{figure}

Fig.~\ref{fig:firstOrder} compares the first and second order expansions in the Rung-Kutta method.
The figure illustrates how the number of walkers and the norm rapidly increase
in the first order expansion. The first order overlap $\langle \Psi_i^{\pm}(0)|\Psi^{\pm}_i(t)\rangle$
is substantially more accurate than the norm, but still not satisfactory.

Fig.~\ref{fig:normError} (right part) compares deterministic\cite{PHC2012,BSKS2015} and 
stochastic calculations of time evolutions of the norm $\langle \Psi(t)|\Psi(t)\rangle$
to second order.  The deterministic calculation only contains the errors of the second 
order Runge-Kutta, and it is very accurate over this time scale.  The stochastic calculation 
introduces substantial errors in the norm, e.g., due to excitations to high-lying states. 
In the overlap $\langle \Psi_j^{\pm}(0)|\Psi^{\pm}_i(t)\rangle$ these stochastic errors tend 
to cancel (see left part of Fig.~\ref{fig:normError}) for two reasons. Many of the stochastically 
excited states have little or no weight in the initial state and therefore give little 
or no contribution to the overlap. Furthermore, the stochastic errors due to the time evolution 
enter linearly in $\langle \Psi_j^{\pm}(0)|\Psi^{\pm}_i(t)\rangle$ and therefore tend to cancel. 
This is crucial for the accuracy of the method. 
We could alternatively have calculated $\langle\Psi_j^{\pm}(t/2)| \Psi^{\pm}_i(t/2)\rangle$, but 
in this case the stochastic errors are much larger, since the the two arguments above do not apply.   

Even though a symplectic integrator such as the Verlet method \cite{Verlet} could in
principle yield smaller discretization errors, we find that stochastic errors
play a much larger role, making it unfeasible compared to the
Runge-Kutta integrator.

\begin{figure*}[t]
\includegraphics[width=0.99\textwidth]{stochastic_norm_error_panel.pdf}

\caption{Green's function (left) and norm (right) of the wave function over time  
using the second-order algorithm for the two-dimensional 10-site Hubbard model, 
$k=(0,0)$ and $U/t=1$. Both were calculated using both stochastic and deterministic algorithms.}
\label{fig:normError}
\end{figure*}

\subsection{Computation of Green's functions and optical absorption}
Here, we provide some more details about the calculation of the Green's function.

We assume that the ground-state $|\Psi_0^N\rangle$ for $N$ electrons has been calculated.
We then want to calculate the Green's function
\begin{equation}\label{eq:g0}
G_{ij}(t)=-i\langle\Psi_0^N|T\lbrace c_i(t)c_j^{\dagger}\rbrace|\Psi_0^N\rangle,
\end{equation}
where $T$ is the time-ordering operator, $c_i$ is the annihilation operator for an electron 
with quantum numbers $i$ (including spin) and $c_i(t)={\rm exp}(i\hat Ht)c_i{\rm exp}(-i\hat Ht)$. 
For $t<0$ ($t>0$) this corresponds to (inverse) photoemission. For photoemission we make 
a variable substitution $t \to -t$. Then both photoemission and inverse photoemission 
correspond to positive time propagation, but there is now an extra minus sign in the 
Schr\"odinger equation for photoemission. We then consider the initial state
\begin{equation}\label{eq:g1}
|\Psi^{\pm}_i(0)\rangle=c_i^{\pm}|\Psi_0^N\rangle,
\end{equation}
where lower (upper) sign indicates (inverse) photoemission and $c_i^{+}=c_i^{\dagger}$ and $c_i^{-}=c_i$.
We solve the Schr\"odinger equation
\begin{equation}\label{eq:g2}
i{d\over dt}|\Psi_i^{\pm}(t)\rangle=\pm [ e^{\mp i\alpha(t)}(\hat H-E_0^N\mp \mu)]|\Psi_i^{\pm}(t)\rangle,
\end{equation}
Here $\alpha(t)$ defines the path through the complex time plane. 
$\alpha(t)\equiv 0$ ($\pi/2$)) corresponds to integration along the real (imaginary) time axis. 
The formal solution can be written as 
\begin{eqnarray}\label{eq:g3}
&&|\Psi_i^{\pm}(t)\rangle \\
&&={\rm exp}[ \mp i\int_0^t dt'e^{\mp i\alpha(t')}(\hat H-E_0^N\mp \mu)] |\Psi_i^{\pm}(0)\rangle \nonumber
\end{eqnarray}
We take the overlap to the state $\langle \Psi^{\pm}_j(0)|$ and expand this in a complete set of states
$|\Psi_n^{N \pm 1}\rangle$.
\begin{eqnarray}\label{eq:g4}
&& \langle \Psi_j^{\pm}(0)|\Psi_i^{\pm}(t)\rangle   \\
&&=\sum_n\langle \Psi_0^N|c^{\mp}_j|\Psi_n^{N\pm 1}\rangle   \langle \Psi_n^{N\pm 1}|c^{\pm}_i|\Psi_0^N\rangle  \nonumber  \\
&&\times {\rm exp}\lbrace \mp i\int_0^tdt' {\rm exp}[\mp i\alpha(t')] [E_n^{N\pm 1}-E_0^N\mp \mu]\rbrace  \nonumber \\
&&=\int d\omega A^{\pm}_{ji}(\omega){\rm exp}\lbrace -i\int_0^t dt' {\rm exp}[\mp i\alpha(t')]\omega\rbrace  \nonumber
\end{eqnarray}
Here we have introduced the spectral functions
\begin{eqnarray}\label{eq:g5}
&& A^{\pm}_{ji}=\sum_n \langle \Psi_0^N|c^{\mp}_j|\Psi_n^{N\pm 1}\rangle    \\
&&\times \langle \Psi_n^{N\pm 1}|c^{\pm}_i|\Psi_0^N\rangle \delta[\omega\mp E_n^{N\pm 1}\pm E_0^N+\mu] \nonumber
\end{eqnarray}
We finally introduce the spectral function
\begin{equation}\label{eq:g6}
A_{ij}(\omega)=A_{ij}^{+}(\omega)+A_{ij}^{-}(\omega),
\end{equation}
where we have used conventions that negative  (positive) frequencies correspond to (inverse) photoemission.
Large (small) values of $|\omega|$ correspond to excited states with large (small) excitation energy.
In a similar way we can calculate optical conductivity, by applying a current operator
to the $N$-particle state and propagating this in time.

The targeted spectral function then dictates the structure of the initial
wavefunction, and thereby also the level of correlation present in the initial
state.
As $\Psi_0^N$ is taken from a previous FCIQMC calculation, the initial state
is obtained from a stochastic sample of the true ground-state. Therefore, multiple
independent samples of $\Psi_0^N$ are taken, and the Green's function is
computed from the overlap of the initial state of one sample with the
time-evolution of another, since a Green's function from only a single sample
is quadratic in the initial state and is hence potentially biased. We find
that such a bias is problematic only for the most correlated initial states,
like the inverse photoemission for the $24$-site Hubbard model as in
Fig. ~\ref{fig:dispersion_ipe_pe}, but using a Green's function obtained from
a single sample should be avoided nevertheless.

\begin{figure*}
\includegraphics[width=\columnwidth]{dispersion_panel.pdf}
\caption{Energy levels of the non-interacting $24$-site Hubbard model with
  lattice vectors (3,3) and (-5,3). The (inverse) photoemission spectrum for $k=(0,0)$ is
  obtained by removing (adding) the electron marked in red. While removing an
  electron with $k=(0,0)$ keeps the determinant with highest weight and
  therefore creates an initial state with a unique high-weight leading
  determinant, this is not the case for the inverse photoemission. As
  $k=(0,0)$ is doubly occupied in the reference determinant of the ground
  state, the latter does not appear in the initial state and we start from a
  enormously correlated state with a high number of determinants with
  comparable weight.}
\label{fig:dispersion_ipe_pe}
\end{figure*}
 
\subsection{Complex time contour}

We use a time-dependent angle $\alpha(t)$, which is adjusted so that the number 
of walkers do not appreciably exceed a preset value. This is done in a similar way 
as the walker number control in the projective algorithm. We prescribe 
an initial value $\alpha(t=0) = \alpha_0$, typically $\alpha_0=0$. Once the
walker number exceeds a threshold value $N_\mathrm{target}$, we start to adjust $\alpha$ every $B$
steps as 
\begin{equation}
\alpha (t + B\Delta t) = \alpha (t) + \xi\, \mathrm{arctan}\left(\frac{N_W\left(t+B\Delta
      t\right)}{N_W\left(t\right)} -1 \right) \,.
\end{equation}
Here, $N_w(t)$ is the number of walkers at time $t$ and $\xi \sim 0.1\, - \, 1.0$ is a damping
parameter. Using this heuristic approach, the value of alpha is iteratively
updated to counter changes in the walker number. We use the $\mathrm{arctan}$ function
to map changes in walker number to changes in an angle, but for sufficiently
small $B \approx 10 $, we do not expect the exact choice of the function used for
this mapping to have
an impact. Using this technique, the value of $\alpha$ is increased during the
time evolution as the walker number increases, which in turn damps the walker
number growth, eventually leading to an equilibration of both the value of $\alpha$
and the number of walkers. However, depending on the chosen parameters $\xi$
and $B$, even in equilibrium, the value of $\alpha$ can be subject to rapid
fluctuations around the average value due to short-time fluctuations in the
number of walkers. This has no notable impact on the contour, however. The equilibrium value of $\alpha$ is then typically $\sim
0.05\,-\,0.25$ for the studied systems, except for the $24$-site Hubbard model
with an equilibrium value of $\alpha \sim 0.45$.
Increasing the walker threshold value $N_\mathrm{target}$ tends to decrease $\alpha$. 
 
\subsubsection{Walker number dependence}

The walker number impacts the time-evolution in two ways. The first is the
influence on the adaptation of $\alpha$, as increasing the walker number for a
fixed initial number of walkers lowers the required values of $\alpha$ for a
stable calculation with a constant walker number. The control mechanisms for
adjusting the walker number here are setting the initial value $\alpha_0$
and/or a minimum walker number which has to be reached before the value of
$\alpha$ is changed. In particular only adjusting $\alpha$ once a given
number of walkers is reached allows for targeting specific walker numbers,
similar to the variable shift mode in the projected algorithm, although the
walker number equilibration is typically slower. The values
$\alpha$ obtains in this procedure decrease as the targeted walker number is
increased, while increasing $\alpha_0$ unsurprisingly decreases the number of
walkers used. 

The second effect is a bias in the Green's function itself as shown in figure ~\ref{fig:24siteWN}.

\begin{figure*}[t]
\includegraphics[width=0.9\textwidth]{t24_wn_compare_panel.pdf}
\caption{Photoemission Green's function for $k=(0,0)$ for the Hubbard model
  with an 18-site cluster at $U/t=2$ obtained with FCIQMC and Lanczos with a)
  70000 and b) 17 million walkers, showing a bias in the Green's function due to under-sampling for the
  smaller walker number. }
\label{fig:24siteWN}
\end{figure*}

\subsection{Chemical potential shift}

Typically we are particularly interested in the spectrum relatively close to the
chemical potential (within several eV). We can emphasize these states by using the 
flexibility of the present method. Thus we study the spectra for each ${\bf k}$ at 
a time and photoemission and inverse photoemission separately. We then have the 
freedom to choose the chemical potential as $E_0^N-E_0^{N-1}({\bf k})\le \mu \le 
E_0^{N+1}({\bf k})-E_0^N$ in the spectral calculation, where $E_0^{M}({\bf k})$ 
is the lowest $M$-electron state with the wave vector ${\bf k}$. Lowering (increasing) 
$\mu$ for (inverse) photoemission leads to a slower decay of the Green's function 
for a given $\alpha(t)$. The shift increases the weight of all states. To keep the 
number of walkers fixed,  $\alpha(t)$ is then increased. This suppresses high-lying 
states (far from $\mu$) more than low-lying states, enhancing the relative weight 
of low-lying states, as the suppression scales with energy. The result is that 
low-lying states contribute to the Green's function over a longer time, and it 
then becomes easier to extract the information about these states. This should 
then also improve the signal to noise ratio for low-lying states. Fig.~\ref{fig:maxent} 
(e.g., for $\alpha_0=\pi/4$ or 0.2) illustrates how structures close to $\mu$ 
are described more accurately.

We can use 
\begin{equation}\label{eq:g7}
\mu=\left\{ \begin{array}{ll}
E_0^{N+1}({\bf k})-E_0^N &   {\rm inverse \ photoemission} \\
E_0^N-E_0^{N-1}({\bf k}) &  {\rm photoemission}
\end{array}\right.
\end{equation}
In this way the contribution to the spectrum from $|\Psi_0^{N\pm1}({\bf k})\rangle$ 
is not damped by $\alpha(t)$, and its contribution to the spectrum is therefore well 
described.

Sometimes the lowest states of the $(N\pm 1)$-system with a given ${\bf k}$ have 
little or no weight in the spectrum of interest and it may then be favorable to 
reduce (increase) $\mu$ even more for (inverse) photoemission. Eventually, however, 
these states obtain weight due to statistical noise and then grow exponentially. 
The shift of $\mu$ should therefore not be too large.

The Matsubara formalism has often been used to study the Mott metal-insulator 
transition or the formation of a pseudo gap. Then the (angular integrated)
spectrum  at $\mu$ is of particular interest, and the Matsubara formalism 
provides very useful information. However, we are often also interested in angular resolved spectra, where for a given ${\bf k}$ the leading peak may be located
well away from $\mu$. Then the separate treatment of each ${\bf k}$ in the
present formalism, and the related possibility to shift the spectrum,
becomes particularly important. Satellites are also often of interest, 
and then the use of a relatively small $\alpha(t)$ in the FCIQMC is
of great advantage.

In the Matsubara formalism the photoemission and inverse photoemission spectra
are treated simultaneously. In the ${\bf k}$-resolved case the relative weights,
and thereby the relative standard deviations, may be very different. The present
separate treatment of the two spectra then becomes an important advantage, since 
the relative standard deviations are comparable for the two spectra.

\subsection{The initiator approximation}

We make use of the initiator version of FCIQMC \cite{CBA2010, CBA2011} which
is commonly used in the projective algorithm. This limits the possibilities
for walkers to spawn to unoccupied determinants and thereby prevents sign
errors from proliferating. The adaptation made is, that spawns onto
unoccupied determinants are only accepted if they either came from a
determinant exceeding a certain threshold occupation or if another spawn
onto the same determinant occurred in the same iteration. 

In contrast to the projective algorithm, the threshold value itself is not
very significant for the purpose of Green's function calculation, as the initial
wave function already has a high number of determinants populated, and only
their population will enter the Green's function. Also, the event of two
spawns occurring onto the same determinant is common, limiting the influence of
the threshold further. 

It can then be highly beneficial to either pick a high threshold, or entirely
disable the possibility to spawn onto unoccupied determinants by single spawns 
and require two spawns to populate a new determinant. Fig.   
~\ref{fig:init_compare} shows the effect of the threshold onto the Green's
function and the spectral function for exemplary cases. The effect on the
Green's function is minor. For the $C_2$ molecule, the high-energy part of the
spectrum exhibits some sensitivity, whereas the low-energy part notices only a
constant shift which does not enter energy differences. 

\begin{figure*}[t]
\includegraphics[width=0.99\textwidth]{init_compare_panel.pdf}
\caption{(a) Green's function of the $U/t=2$ 18-site Hubbard model for fixed
  $\alpha=0.2$ for different initiator thresholds and without any initiators ($\infty$),
  allowing only double spawns to populate new determinants, and
  as obtained using Lanczos. (b) Photo absorption spectra of $C_2$ in the
  cc-pVTZ basis set for different thresholds and without initiators.
Large values of $\alpha$ were used for the smaller thresholds, leading
to broader spectra.}
\label{fig:init_compare}
\end{figure*}

\subsection{Maximum entropy}
\label{sec:maxent}
The maximum entropy method \cite{maxent,Jarrell} for calculating spectral functions is 
often applied together with the finite temperature Matsubara formalism, where the spectral data 
are then analytically continued from the imaginary to the real axis. Here we develop a 
formalism for analytic continuation from an arbitrary path in the complex
plane to the real axis, using the (inverse) photoemission spectrum as an example. 
The spectrum $A_{ij}(\omega)$ is related to the solution of the Schr\"odinger equation via
\begin{equation}\label{eq:max3}
g_k=\sum_l K_{kl}a_l, 
\end{equation}
where $g_k=\langle \Psi_i^{\pm}(0)|\Psi_j^{\pm}(t_k)\rangle$, $a_l=A_{ij}^{\pm}(\omega_l)$ and                                                           
\begin{equation}\label{eq:max1}
K_{kl}={\rm exp}\lbrace -i\int_0^{t_k}dt e^{\mp i\alpha(t)} \omega_l \rbrace f_l,
\end{equation}
where $f_l$ is a weight factor for the $\omega$ integration and the lower (upper) sign         
refers to (inverse) photoemission. The indices $i$ and $j$ have been dropped for simplicity.
We introduce the average $\bar g_k$ over many samples of $g$ and define the deviation $\chi$ of 
a spectral function $a$ giving $g$ from $\bar g$ as
\begin{equation}\label{eq:8}
\chi^2=\sum_{k=1}^L\sum_{k=1}^L({\bar g}_{k}-g_{k})^{*}[C^{-1}]_{kl}({\bar g}_{l}-g_{l}).
\end{equation}
where the sums run over the $L$ values of $g_k$ and $C$ is the covariance
matrix \cite{maxent,Jarrell} of the samples of $g$. To obtain a regular
expression for $\chi^2$, it is important to have a non-singular covariance
matrix $C$, as the inverse $C^{-1}$ is required to calculate
$\chi^2$. If few samples are used, $C$ may be ill-behaved. We have then imposed
a minimum value, $\sigma_\text{min} \approx 10^{-4}\dots 10^{-6}$, on the
diagonal entries of $C$. While this allows for regularizing $C$, it also
assumes the data to be more noisy than it actually is and hence can
affect the details of the spectra as illustrated in Fig.~\ref{fig:sigmamin}. 
Alternatively, we have split the data in batches and assumed a diagonal $C$
for each batch. This assumption can overemphasize noise, which tends to be 
compensated by averaging over batches.

\begin{figure*}[t!]
\includegraphics[width=\textwidth]{24t_sigmamin_panel.pdf}
\caption{Spectra obtained using maximum entropy obtained using different
  values of the cutoff $\sigma_\text{min}$ for the photoemission (left) and the inverse photoemission of the 22-electron 24-site (right) Hubbard model at
  $U/t=4$. While the photoemission spectrum shows sensitivity to the cutoff, the
inverse photoemission spectrum does not as long as the covariance matrix is
non-singular. For $\sigma_\text{min}= 10^{-9}$, this is no longer the case here
and the analytic continuation is ill-defined, leading to deviations in the
spectrum. For comparison, the spectrum obtained by partitioning the data in 6
batches, assuming a diagonal $C$ and averaging over the spectra obtained from
each batch is also shown.}
\label{fig:sigmamin}
\end{figure*}

We also introduce the entropy $S$ 
\begin{equation}\label{eq:11}
S=\sum_{i=1}^L[a_i-m_i-a_i{\rm ln}(a_i/m_i)]f_i,
\end{equation}
where $m_i$ is default function providing a guess for $A(\omega)$.
We minimize $\chi-\gamma S$, where $\gamma$ determines the importance of the entropy.
The most probable value of $\gamma$ is chosen \cite{maxent,Jarrell}.
This leads to a system of nonlinear equations.
This system is solved iteratively, by linearizing the equations around successive approximations $a^{(m)}_i$.
We introduce $a_i^{(m+1)}=a_i^{(m)}+\delta a_i^{(m+1)}$ and solve 
\begin{eqnarray}\label{eq:16}
&&\sum_j {\rm Re} [K^{\dagger}C^{-1}]_{kj}{\bar g}_j-\sum_{jl} {\rm Re} [K^{\dagger}C^{-1}K]_{kl}a^{(m)}_l \nonumber \\
&& -\gamma{\rm ln} { a_k^{(m)}\over m_k} =\sum_{l} \Lambda_{kl}(a_k^{(m)})\delta a_l^{(m+1)}. 
\end{eqnarray}
where 
\begin{equation}\label{eq:maxent10} 
\Lambda_{kl}(a_k^{(m)})=\lbrace {f_k\gamma\over a_k^{(m)}}\delta_{kl}+ {\rm Re} [K^{\dagger}C^{-1}K]_{kl} \rbrace
\end{equation}
Fig.~\ref{fig:maxent} show results for the Hubbard model with four different $\alpha(t)\equiv \alpha_0$.
The spectrum was obtained from exact diagonalization, transformed to complex $t$ and Gaussian noise was added.
The spectrum was then transformed back to real frequencies using maximum entropy and compared with the exact result.
For data on the imaginary axis ($\alpha_0=\pi/2$), the $\omega=0$ peak is accurately described, while the other structures 
are approximated by two peaks. For data close to the real axis ($\alpha_0=0.1$) almost all structures are reproduced.

To understand what accuracy can be obtained, we expanded the work in Ref. \onlinecite{analyticcontinuation} and introduce the eigenvectors $|\nu\rangle$ 
and eigenvalues $\varepsilon_{\nu}$ of $\Lambda$. We expand the differences $\delta a=a-a_{\rm exact}$,  $\delta m=m-a_{\rm exact}$
and the stochastic error in $\bar g$ in the eigenvectors $|\nu\rangle$ and obtain coefficient $\delta a_{\nu}$, $\delta m_{\nu}$
and $\delta g_{\nu}$, satisfying    
\begin{equation}\label{eq:maxent11}
\delta a_{\nu}={1\over \varepsilon_{\nu}}(\delta g_{\nu}+\delta m_{\nu}).
\end{equation}
Typically there are several very large $\varepsilon_{\nu}$.
The corresponding components of $a$ are then very accurately described. Other eigenvalues 
are approximately unity, and the corresponding $\delta a_{\nu}$ cannot be
trusted.                        

\begin{figure}[t!]
\includegraphics[width=\columnwidth]{maxent_alpha.pdf}
\caption{Photoemission spectrum for the Hubbard model with 18 sites, $U/t=8$ for different functions $\alpha(t)\equiv \alpha_0$ (top four figures).
The bottom two figure shows the  basis functions $|\nu\rangle$ with 
$\varepsilon_{\nu}>5$ for $\alpha_0=\pi/2$ and $\pi/4$.
The chemical potential is chosen so that one peak is at $\omega=0$.
The data have Gaussian noise with a relative standard deviation 
of about 10$^{-2}$. The spectra have been given a Lorentzian broadening 
with FWHM=0.1.
}
\label{fig:maxent}
\end{figure}


The bottom of Fig.~\ref{fig:maxent} shows the eigenvectors for $\alpha_0=\pi/2$ and $\pi/4$. 
The eigenvalues for $\alpha_0=1$ are $2\times 10^6$, $3 \times 10^4$, $5 \times 10^3$, 73, 5.   
The components of $A(\omega)$ corresponding to the first four or five eigenvectors are then described very well.
These eigenvectors do not have enough nodes to describe details away from $\omega=0$. As $\alpha_0$ is reduced 
the number of eigenvalues larger than 5 increases from 10 ($\alpha_0=\pi/4$) or 20 ($\alpha_0=0.2$) to about 40 ($\alpha_0=0.1$).
Correspondingly, more and more details of the spectrum can be described.
The $|\nu\rangle$ and $\varepsilon_{\nu}$ help  us judge which details of $A(\omega)$ can be described
and which cannot. 

\subsection{Additional Data}
In addition to the study on the half-filled 18-site Hubbard model with
$U/t=2$, calculations on the same system with $U/t=4$ have been performed, of
which the resulting spectra are displayed in Fig.~\ref{fig:HubU4}.

\begin{figure}[t!]
\includegraphics[width=\columnwidth]{hub_U4.pdf}
\caption{Spectrum for the 18-site half-filled model at $U/t=4$ obtained
with $1.4\times 10^8$ walkers for $k=(0,0)$. Both the integrated weights of
the peaks of the FCIQMC spectrum as well as the corresponding integrated
weights of the Lanczos spectrum (bracketed) are displayed, showing reasonable agreement.}
\label{fig:HubU4}
\end{figure}

For completeness, we also consider the Carbon dimer in a minimal cc-pVDZ basis
set consisting of 14 orbitals per atom in the frozen core approximation. The
Hilbert space size here is $\sim 10^8$, and photo absorption spectra can be
obtained analogously to the basis sets described in the main text, which are
shown in Fig.~\ref{fig:C2DZ}.

\begin{figure}[t!]
\includegraphics[width=\columnwidth]{c2_dz_spectrum.pdf}
\caption{Photo absorption spectra for the Carbon dimer in a cc-pVDZ basis set
  for a single excitation from the $2\sigma_u$(blue)/$1\pi_u$(red) to the
  $3\sigma_g$ orbital. Next to the real-time FCIQMC estimates of the
  excitation energies we also list the corresponding energies as obtained
  using excited-state i-FCIQMC method \cite{excited_state_neci} and the FCIQMC ground
state energy from \cite{BTCA2012}. The time step used is $\Delta t = 5 \times
10^{-3}$. Again, a rotation of time in the complex plane is performed, with an
angle of $\alpha \sim 0.33$, which is higher than for the larger basis sets
due to the larger time-step, leading to an increased broadening.}
\label{fig:C2DZ}
\end{figure}
%\clearpage

%\input{appendix_refs.bbl}
