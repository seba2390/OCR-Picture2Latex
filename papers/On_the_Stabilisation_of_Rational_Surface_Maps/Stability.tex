\documentclass[11pt, final]{amsart}
\usepackage{rmwe}
\usepackage{geometry}
\geometry{letterpaper}
\usepackage[parfill=0pt]{parskip}    % Activate to begin paragraphs with an empty line rather than an indent
\usepackage{graphicx}
\usepackage{amssymb, centernot}
\usepackage{epstopdf}
\DeclareGraphicsRule{.tif}{png}{.png}{`convert #1 `dirname #1`/`basename #1 .tif`.png}

\setlength{\textheight}{8.25in}
\setlength{\marginparwidth}{27mm}

\usepackage{enumitem}
\setlist[enumerate,1]{ref=\emph{(\roman*)}}
\setlist[enumerate,1]{label=\emph{(\roman*)}}

\usetikzlibrary{decorations.pathmorphing}

\renewcommand{\sectionautorefname}{\S\!\!}

\mynewtheorem{propdefn}{Proposition/Definition}{Prop/Defn}

\newcommand{\dashto}{\dashrightarrow}
\newcommand{\E}{\mathcal{E}}
\newcommand{\nin}{\notin}
\newcommand{\emp}{\emptyset}
\newcommand{\sgraph}{\Sigma}
\newcommand{\nice}{untangled}
\newcommand{\Nice}{Untangled}
\newcommand{\nimplies}{\centernot\Longrightarrow}
\newcommand{\lo}{o}
\newcommand{\minimalalgname}{Minimal Stabilisation Algorithm}
\newcommand{\half}{\frac 12}

\DeclareMathOperator{\Div}{Div}
\DeclareMathOperator{\Pic}{Pic}
\DeclareMathOperator{\NS}{NS}
\DeclareMathOperator{\rk}{rk}
\DeclareMathOperator{\comp}{\mathfrak e}

\title[On the stabilisation of rational surface maps]{On the stabilisation of rational surface maps}
\author{\mylongname}
\address{Department of Mathematics, University of Notre Dame, Notre Dame, IN 46656}
\email{rbirkett@nd.edu}
%\date{}                                           % Activate to display a given date or no date

\begin{document}

\begin{abstract}
%We provide two new approaches to a theorem of Diller~and~Favre. Namely, a birational self-map $f : X \dashto X$ on a smooth projective surface $X$ is birationally conjugate to a map which is algebraically stable. Among other things, the first approach confirms the validity of a longstanding practical method to stabilise birational maps. Secondly we show that a map becomes algebraically stable if one repeatedly lifts the map to its graph.

%We provide insight about when and how blowups alone can be used to birationally conjugate a rational map to an algebraically stable one on a smooth projective surface. We show that the method suggested by Diller \& Favre for birational maps is in fact the optimal such for rational maps when at all possible. Secondly we give a second `automatic' method to stabilise birational maps whilst making observations about the structure of the graph of a birational map on a surface. The paper concludes with an example of rational maps which can only be made algebraically stable with a blowdown.

%A rational self-map $f : X \dashto X$ on a smooth projective surface $X$ may or may not be birationally conjugate to a map which is algebraically stable. We explore results about the situation where this conjugation is through a birational morphism (blowups but no blowdowns). This includes a family of rational maps which can be stabilised but never by blowups.

The dynamics of a rational surface map $f : X \dashto X$ are easier to analyse when $f$ is `algebraically stable'.  Here we investigate when and how this condition can be achieved by conjugating $f$ with a birational change of coordinate.  We show that if this can be done with a birational morphism, then there is a minimal such conjugacy. For birational $f$ we also show that repeatedly lifting $f$ to its graph gives a stable conjugacy. Finally, we give an example in which $f$ can be birationally conjugated to a stable map, but the conjugacy cannot be achieved solely by blowing up.
\end{abstract}

\maketitle

\vspace{-5mm}
\section{Introduction}

%Confirm an old method works for birational maps
%Expand this to rational maps
%In parallel, provide a new method based on the graph of a birational map.
%Understand more about when maps can be stabilised
%Exhibit a rational map which shows that not all maps can be stabilised by blowups. This example belongs outside well known toric examples which one might hope were part of a small class of exceptions.

Let $f: X \dashto X$ be a rational map on a smooth projective surface over an algebraically closed field. Studying $f$ as a dynamical system is complicated by the fact that $f$ need not be continuously defined on all points of $X$. For example a rational map $f$ induces a natural pullback operator on curves $f^* : \Pic(X) \to \Pic(X)$, but this operator may not iterate well. We say $f$ is \emph{algebraically stable} iff $\forall n \in \N\ (f^*)^n = (f^n)^*$. This is equivalent to the geometric condition that $f$ has no \emph{destabilising orbits} \cite{DF}; i.e. an orbit of (closed) points $p, f(p)\dots, f^n(p)$ in $X$, for which $f^{-1}(p)$ and $f(f^n(p))$ are curves. It is natural to hope that by blowing up the points in such an orbit will improve the situation. The main theme of this paper is to discuss the extent to which this actually works.

%To be more precise, $f$ can have a finite \emph{indeterminate set} of closed points, $I(f)$, where $f$ is not a morphism.
%Given such a point $p \in I(f)$, one can meaningfully define the transform $f(p)$ as a curve in $X$. Suppose $p \in X$ is a closed point, we will say an orbit $p, f(p)\dots, f^{n-1}(p)$ is \emph{destabilising} when $f^{-1}(p)$ is a curve and $f^{n-1}(p) \cap I(f) \ne \emp$. When $f$ has no destabilising orbits, we say $f$ is \emph{algebraically stable} (AS). 
%%Although this characterisation of algebraic stability is sufficient for this paper, we note the following important equivalent formulation.

Assume for the rest of the paper that all surfaces are projective varieties over an algebraically closed field, and unless explicitly stated, also smooth. We write $\phi : X \dashto Y$ to indicate that $f$ is a rational map between surfaces, and we use a solid arrow $f : X \to Y$ when $f$ is a morphism.

\begin{defn}
 %Let $(f, X)$ denote a surface along with a rational map $f : X \dashto X$. 
 %Given a rational map $f : X \dashto X$ on a smooth projective surface, we may denote this dynamical system simply as $(f, X)$.
 We write $\phi : (g, Y) \dashto (f, X)$ to indicate that $\phi : Y \dashto X$ is a birational map conjugating $f : X \dashto X$ to $g = \phi^{-1} \circ f \circ \phi : Y \dashto Y$. When $g : Y \dashto Y$ is algebraically stable, we say that $\phi$ \emph{stabilises} $f$.
\end{defn}
%Given $(f, X)$ which is not algebraically stable, one might hope to stabilise $f$ as above with a birational map or even a morphism.
%$f$ is \emph{potentially algebraically stable} and
%Further we say $\pi$ is a \emph{stabilises}. in this case we say that $f$ is \emph{potentially algebraically stable}. The broad goals of this paper are to provide some insight to when and how blowups alone can be used to stabilise a rational map.

%\improvement{To stabilise a birational map... a successful approach has been... cite... a more elaborate }
%Now consider how we might stabilise a rational map. We call a destabilising orbit \emph{minimal} iff it does not contain any shorter destabilising orbits; in this case $p_j = f^{j-1}(p)$ are distinct closed points for $1 \le j \le n$. 
%
%To stabilise \emph{bi}rational maps, blowing up destabilising orbits has proven successful.
Diller-Favre \cite{DF} proved that for a birational map $(f, X)$ there is always a morphism $\pi : (g, Y) \to (f, X)$ which stabilises $f$. %The arguments in \cite{DF} diverge somewhat from the straight forward idea of blowing up destabilising orbits. %Their idea is that one might hope to stabilise $f$ by repeatedly blowing up its minimal destabilising orbits; this is but one reason we call this the \emph{minimal method}.
%
%Firstly, we provide a result inspired by a theorem of Diller~and~Favre \cite{DF}, namely that if $f: X \dashto X$ is a birational map then $f$ can always be stabilised and the conjugation is by a birational morphism; an unproven technique developed at the same time was what we now call the \emph{minimal method}, which blows up a selection of points on $X$ described by the dynamics of $f$.
%
Not all (non-invertible) rational maps can be stabilised, however. Favre \cite{Fav} showed for example that many monomial maps on $\P^2$ cannot be stabilised by any birational conjugacy.%, whilst also stabilising the other monomial maps through blowups and covering maps. %monomial maps on $\P^2$ with coefficients from $2\times 2$ integer matrix which is an irrational rotation of the plane

%\change{Fix...}
%Both the arguments in \cite{DF} and particular examples considered in \cite{BK06, BKTAM, BK10} suggest the following method 
%On the other hand, simply blowing up destabilising orbits has seemed to work well in practice.  been used successfully in many particular examples of birational maps \cite{BK06, BKTAM, BK10} etc. %\unsure{Maybe I should quote something other than Bedford and Kim?}%Takn BK11 BK09
%Our first results explain this success.%below, show however that when a \emph{rational map} $f : X \dashto X$ can be stabilised by a birational morphism, then repeatedly blowing up the destabilising orbits suffices. We call this special stabilisation $\pi : (\hat f, \hat X) \to (f, X)$ the \emph{minimal} one. Moreover, given any other blowup procedure $\rho : (g, Y) \to (f, X)$ which stabilises $(f, X)$ then it is just a superfluous extension of the minimal method.%, meaning $\rho$ factors as $\pi \circ \nu$.
%More informally, we not only give final confirmation that the minimal method works for birational maps, but also show that it is the best blowup procedure of all for stabilising rational maps.

In any case the arguments in Diller-Favre, and the evidence of particular examples of rational maps \cite{BK06, BKTAM, BK10, Fav} support the idea that blowing up destabilising orbits is a good approach to achieving algebraic stability. We show that when stability can be achieved via birational morphism, then this is essentially the only way.

Let us call a destabilising orbit \emph{minimal} when it does not contain any shorter ones.

\begin{prop}\label{prop:minorbit}
 Suppose that $f: X \dashto X$ is a rational map on a surface. Let\\$p, f(p), \dots, f^{n-1}(p)$ be a minimal destabilising orbit for $f$ and $\pi : X' \to X$ be the birational morphism blowing up of each $p_j$. Then any birational morphism $\rho : (g, Y) \to (f, X)$ stabilising $(f, X)$ factors as %Suppose that there exists a stabilisation $\rho : (g, Y) \to (f, X)$, then $\rho$ factors through $\pi$, meaning
 $\rho = \pi \circ \nu$ for some birational morphism $\nu : Y \to X'$. %and $\hat f = \pi^{-1} \circ f \circ \pi$. $\pi : (g, Y) \to (f, X)$
\end{prop}

%\begin{propdefn}[Minimal Method]\label{propdefn:minimal}\unsure{keep?}
%%\change{Turn the proof into minimal counter example.}
% Let $f : X \dashto X$ be a rational map on a smooth projective surface which is not algebraically stable.
% 
% Define the sequence $(f_m, X_m)$ recursively as follows.
%\begin{itemize}
%\item Let $X_1 = X$ and $f_1 = f$.
%\item If $f_m$ is not algebraically stable, pick a minimal destabilising orbit $p_1, p_2, \dots, p_n$ and blowup each of the closed points $p_j$ to produce $\pi_m : (f_{m+1}, X_{m+1}) \to (f_m, X_m)$.
%\end{itemize}
% If this sequence terminates at $f_M : X_M \dashto X_M$, we write $X_M = \hat X$, $\hat f = f_M$ and $\pi = \pi_1 \circ \cdots \circ \pi_{M-1} : \hat X \to X$. Then $(\hat f, \hat X)$ is algebraically stable and we call $\pi : (\hat f, \hat X) \to (f, X)$ the \emph{minimal stabilisation} of $(f, X)$ (by blowups) when it exists.
%\end{propdefn}

\begin{defn}[\minimalalgname]\label{defn:minimal}
Given a rational surface map\\ $f_0 : X_0 \dashto X_0$ we define a (possibly finite) sequence $\pi_m : (f_{m+1}, X_{m+1}) \to (f_m, X_m)$ for $m \ge 0$ as follows
\begin{enumerate}
 \item If $f_m$ is algebraically stable, stop.
 \item If not, then pick a minimal destabilising orbit $p_1, p_2, \dots, p_n$ and blowup each of the $p_j$ to produce $\pi_m : (f_{m+1}, X_{m+1}) \to (f_m, X_m)$.
\end{enumerate}

If this sequence terminates at $f_M : X_M \dashto X_M$, write $\pi = \pi_1 \circ \cdots \circ \pi_{M-1} : X_M \to X$. Then $(f_M, X_M)$ is algebraically stable and we call $\pi : (f_M, X_M) \to (f, X)$ a \emph{minimal stabilisation} of $(f, X)$ (by blowups) when it exists.
\end{defn}

This terminology is justified by the next theorem.

\begin{thm}\label{thm:minimal}
 Let $f : X \dashto X$ be a rational map on a surface. If there exists a birational morphism $\rho : (g, Y) \to (f, X)$ stabilising $f$ then any instance of the \minimalalgname{} terminates in a minimal stabilisation $\pi : (\hat f, \hat X) \to (f, X)$ such that $\rho = \pi \circ \nu$ for some $\nu : Y \to \hat X$. It follows that the minimal stabilisation $(\hat f, \hat X)$ is unique for $(f, X)$.% up to isomorphism (of surfaces and dynamical systems).
\end{thm}

\begin{cor}\label{cor:minbirational}
 Let $f : X \dashto X$ be a birational map on a surface. Then there exists a unique minimal stabilisation $\pi : (\hat f, \hat X) \to (f, X)$.
 \end{cor}

\begin{proof}%[Proof of \autoref{cor:minbirational}]
 By \cite{DF}, there exists a birational morphism $\rho : \tilde X \to X$ which makes $\tilde f = \rho^{-1} \circ f \circ \rho$ algebraically stable on $\tilde X$. By \autoref{thm:minimal}, the minimal stabilisation $\hat f : \hat X \dashto \hat X$ via $\pi : \hat X \to X$ exists (uniquely and factors $\rho$).
\end{proof}

\autoref{thm:minimal} and the results of \cite{Fav} may lead the reader to believe that if a rational map $f$ admits a stabilisation $\phi : (g, Y) \dashto (f, X)$, then in fact we can achieve algebraic stability through blowups alone, i.e. $\phi$ can be chosen to be a \emph{morphism} $\phi : (\hat f, \hat X) \to (f, X)$. %One might even hope that AS models are somehow ubiquitous in the world of birational conjugates of $(f, X)$.
This turns out to be false.

% might lead one to believe that if $f$ admits a stabilisation $\phi : (g, Y) \dashto (f, X)$, then $\phi$ can be chosen to be a morphism. This turns out to be false.

%We provide a negative answer to this question, by exhibiting a rational map which can be stabilised but not by any blowup procedure at all. We believe the example is new in this respect, meaning that other examples in the literature either have no birational conjugate which is algebraically stable or have a simple blowup procedure which stabilises them. %cannot be stabilised at all, or 
%See section \autoref{sec:counterex} for details.

%\begin{thm}\label{thm:counterex}\change{replace with new version}
% There exists a (family of) rational map(s) $f : X \dashto X$ on the Hirzebruch surface $X = \Sigma_1$ which are algebraically stable, and a single point blowup $\mu : (f', X) \to (f, X)$ such that for any further blowing up of $X'$ (i.e.\ birational morphism) $\rho : (g, Y) \to (f', X')$, we find that $(g, Y)$ is not algebraically stable.
%\end{thm}

\begin{thm}\label{thm:counterex}
 Let $X$ be the first Hirzebruch surface
 Let $f : \C^2 \dashto \C^2$ be given by \[(x, y) \longmapsto (x^2, x^4y^{-3} + y^3) = \left(x^2, \frac{x^4 + y^6}{y^3}\right).\]
 %Let $\phi : (\tilde f, \P^1 \times \P^1) \dashto (f, X)$ be the map blowing up $(\infty, \infty)$ and contracting the old fiber of $x=\infty$.
 The $f$ extends to an algebraically stable rational map $f : X \dashto X$ of a Hirzebruch surface $X$. 
 If however $\sigma_0 : (f_0, X_0) \to (f, X)$ is the point blowup of $(0, 0) \in X$, then there does not exist any birational morphism $\pi : (g, Y) \to (f_0, X_0)$ which stabilises $f_0$.
\end{thm}

We conclude the introduction on a somewhat different note, giving an alternative approach to the theorem of Diller-Favre for stabilising birational maps $f : X \dashto X$. In this approach one focuses on the graph of $f$, rather than on any of the destabilising orbits of $f$.

% idea is to repeatedly and \emph{completely} blow up the indeterminate set $I(f)$, with no attention to destabilising curves.

%\begin{minipage}{0.7\textwidth}
To be precise, we write $\Gamma_{\!f} \subset X \times X$ for the graph of $f$, and we call its minimal smooth desingularisation, $\sgraph_{\!f}$, the \emph{smooth graph} of $f$. Equivalently $\alpha : \sgraph_{\!f} \to X$ is the birational morphism found by blowing up $X$ (minimally) until the lift $\beta : \sgraph_{\!f} \to X$ is a morphism.
%\end{minipage}
%
%\begin{minipage}{0.3\textwidth}
  \begin{center}
 \begin{tikzcd}[column sep = 1.5em, row sep = 2.5em]
 & \sgraph_{\!f} \arrow{ld}[swap]{\alpha}\arrow{rd}{\beta}&\\
  X \arrow[dashed]{rr}{f} & & X
  \end{tikzcd}
\end{center}
%\end{minipage}
%
% \begin{defn}\label{defn:blowupev}\unsure{Name this algorithm too?}%[Graph Stabilisation Algorithm]
% Let $f_0 : X_0 \dashto X_0$ be a rational map on a surface.
%Define the sequence $\alpha_m : (f_{m+1}, X_{m+1}) \to (f_m, X_m)$ recursively as follows.
%
% Let $X_{m+1} = \sgraph_{\!f_m}$ be the smooth graph of $f_m$, and denote its first projection by $\alpha_m : \sgraph_{\!f_m} \to X_m$. Then set $f_{m+1} = \alpha_m^{-1} \circ f_m \circ \alpha_m$.
%\end{defn}
%
% \begin{thm}\label{thm:blowupev}
% Let $f : X \dashto X$ be a birational map on a surface. Let $(f_0, X_0) = (f, X)$ and consider the sequence $(f_m, X_m)$ as defined in \autoref{defn:blowupev}. Then 
%\begin{enumerate}
% \item for $m > 0$, $X_{m+1} = \sgraph_{\!f_m} = \Gamma_{\!f_m}$, i.e.\ the usual graph is smooth, and
% \item there exists an $M \in \N$ such that $\forall m \ge M$ the map $f_m$ is algebraically stable.
%\end{enumerate}
%\end{thm}

\begin{thm}\label{thm:blowupev}
  Let $f_0 : X_0 \dashto X_0$ be a birational map on a surface.
Suppose the sequence $\alpha_m : (f_{m+1}, X_{m+1}) \to (f_m, X_m)$ is defined recursively by $X_{m+1} = \sgraph_{\!f_m}$, and $\alpha_m : \sgraph_{\!f_m} \to X_m$ is the first projection from the smooth graph. Then
\begin{enumerate}
 \item $\forall m > 0$ $X_{m+1} = \sgraph_{\!f_m} = \Gamma_{\!f_m}$, i.e.\ the graph $\Gamma_{\!f_m}$ is smooth, and
 \item $\exists M \in \N$ $\forall m \ge M$ the map $f_m$ is algebraically stable.
\end{enumerate}
\end{thm}

%The particular advantage of the second algorithm is that the reader doesn't need to locate any destabilising orbits, only know whether or not a given map is algebraically stable.
A key ingredient of this proof is the observation that the lift $f_1$ on $\sgraph_{\!f}$, is \emph{\nice{}}, that is whenever a curve $C$ in $\sgraph_{\!f}$ is contracted to a point by $f_1$, then $C$ contains no indeterminate points.

The rest of this paper is organised as follows. \autoref{sec:back} provides notation for this article and recalls useful concepts for birational maps. \autoref{sec:jud} provides the proof of \autoref{thm:minimal}. In \autoref{sec:nice} we describe some interesting properties of \nice{} birational maps and also prepare for the proof of \autoref{thm:blowupev}, which constitutes \autoref{sec:blowupev}. We end the article with the example and computations for \autoref{thm:counterex} in \autoref{sec:counterex}.

In closing we mention that there is another recent proof of the theorem in \cite{DF} based on geometric group theory by Lonjou~and~Urech \cite{LU}. We also note that in the context of integrable systems, the failure of algebraic stability is related to the \emph{singularity confinement property}, see \cite{GRP} etc.

\section*{Acknowledgements}

I would like to thank my advisor Jeffrey Diller for his encouragement, helpful discussions, and expert advice on the exposition of this document. I also thank Claudiu Raicu for his excellent question asking whether the method given in \autoref{defn:minimal} (originally proved otherwise for birational maps) is indeed the `minimal' stabilisation method.

\pagebreak

\section{Background}\label{sec:back}

For the rest of this article, assume all surfaces are smooth projective over an algebraically closed field, and rational maps are dominant. An account of most of the facts below can be found in \cite[\S V]{Hart}.

Let $X, Y$ be surfaces and $f : X \dashto Y$ a rational map. Let $U$ be the largest (open) set on which $f : U \to Y$ is a morphism, then we define the \emph{indeterminate set} as $I(f) = X \sm U$. Alternatively, these are the finitely many points at which $f$ cannot be continuously defined. These are often also called \emph{fundamental points}.

An irreducible curve $C \subset X$ is \emph{exceptional} iff $f(C\sm I(f))$ is a point in $Y$. We define the \emph{exceptional set}, $\E(f)$, of $f$ to be the union of all (finitely many) irreducible exceptional curves in $X$. Denote by $\comp(f)$ the number of irreducible components in $\E(f)$.
 
Let $f : X \dashrightarrow Y$ be a rational map of surfaces. The \emph{graph} of $f$ is the subvariety \[\Gamma_{\!f} = \overline{\set{(x, f(x)) \in X\sm I(f) \times Y}} \subset X\times Y\] along with projections $\alpha : \Gamma_{\!f} \to X$ onto the first factor and $\beta : \Gamma_{\!f} \to Y$ onto the second factor which are proper. $\Gamma_{\!f}$ is irreducible because $X \sm I(f)$ is.

%\improvement{Draw triangle and then the rest. Leave room!}
% \begin{center}
% \begin{tikzcd}[column sep = 1.5em, row sep = 2.5em]
% & \Gamma_f \arrow{ld}[swap]{\pi_1}\arrow{rd}{\pi_2}&  \\
%  X \arrow[dashed]{rr}{f} & & Y
%  \end{tikzcd}
%\end{center}
%\end{defn}

$I(f)$ is the set of points where $\pi_1$ does not have a local inverse. $\alpha^{-1} : X \sm I(f) \to \Gamma_f$ is an isomorphism.  For $p \nin I(f)$ we have $\beta(\alpha^{-1}(p)) = f(p)$.
 In general for any set of points $S \subseteq X$, we may define the \emph{total transform} of $S$ by $f$ as $f(S) = \beta(\alpha^{-1}(S))$. When $\emp \ne S \subseteq I(f)$ this image has dimension $1$. 
%  Given these definitions, the graph is the set of all total transforms. \[\Gamma_f = \set[(x, y) \in X \times Y]{y \in f(x)}\]
 $\E(f) \subset X$ is the $\alpha$ projection of the set of points where $\beta$ is not finite.

$\Gamma_{\!f}$ need not be smooth and it will be more convenient to work with the minimal smooth desingularisation $\sgraph_{\!f}$ of $\Gamma_{\!f}$. Abusing notation slightly, we denote the lifted projections as $\alpha : \sgraph_{\!f} \to X$ and $\beta : \sgraph_{\!f} \to Y$. Equivalently $\alpha : \sgraph_{\!f} \to X$ is the birational morphism found by recursively blowing up the indeterminate set $I(f)$ until the lift of $f$ becomes a morphism $\beta : \sgraph_{\!f} \to Y$. When $Y = X$ and $f$ is birational, $\beta$ is also a birational morphism, and one can deduce that $\comp(\alpha) = \comp(\beta) = \rk\NS(\sgraph_{\!f}) - \rk\NS(X)$.

\begin{rmk}[Warning]
 $\sgraph$ may contain curves which appear neither in the domain $X$ or the codomain $Y$, meaning both $\alpha$ and $\beta$ map the curve to a point. This issue will be rectified in \autoref{prop:curveex} and \autoref{cor:graphnice}.
\end{rmk}

\begin{prop}\label{prop:factorblowup}
 Let $f : X \to Y$ be a birational morphism of surfaces. Then $f$ can be written as a composition of $\comp(f)$ point blowups.% (unique up to permissible reordering).
 
 Suppose $p \in I(f^{-1})$ and $\pi : Y' \to Y$ be the point blowup of $p$. Then $f$ factors as $\pi \circ f'$ where $f': X \to Y'$ is a birational morphism with $\comp(f') = \comp(f) -1$.
 
 Otherwise if $p \nin I(f^{-1})$ and $\rho : X' \to X$ is the point blowup of $f^{-1}(p)$. Then $f$ lifts to a birational morphism $f' = \pi^{-1} \circ f \circ \rho$ with $\comp(f') = \comp(f)$
\end{prop}

\begin{defn}
Let $f : X \dashrightarrow Y$ and $g : Y \dashrightarrow Z$ be rational maps on surfaces.
 We say an irreducible curve $C \subset X$ is a \emph{destabilising curve} for the composition $g \circ f$ iff $f(C\sm I(f)) = y \in I(g)$. Equivalently $C \subseteq f^{-1}(y)$ and $g(y) \supseteq C'$ for some irreducible curve $D \subset Z$, we call $D$ an \emph{inverse destabilising curve}.
 \end{defn}
 
 \begin{prop}\label{prop:graphcomp}
 Let $f : X \dashto Y$ and $g : Y \dashto Z$ be birational maps. Then \[g(f(x)) \supseteq (g\circ f)(x)\] with equality iff $x$ is not contained in a destabilising curve. Moreover $z \in g(f(x)) \iff z \in (g\circ f)(x)$ or $\exists y \in I(g)\cap I(f^{-1})$ with $z \in f(y)$ and $x \in f^{-1}(y)$ destabilising the composition $g\circ f$.
\end{prop}

\begin{defn}
Let $f : X \dashrightarrow X$ be a rational map. An irreducible curve $C$ is a \emph{destabilising curve} for $f$ iff there is an $n \in \N$ such that $f(C\sm I(f)) = p$ is a closed point, and there is an $n$ such that $f^{n-1}(p) \ni q$ and $q \in I(f)$. The \emph{destabilising orbit} here is $p, f(p)\dots, f^{n-1}(p)$ and its \emph{length} is $n$. Then we call any irreducible component of $f(q)$ an \emph{inverse destabilising curve} of $f$. 
%
Finally, we say $f : X \dashrightarrow X$ is \emph{algebraically stable} iff there are no destabilising curves for $f$.% of any length.
\end{defn}

 When $f$ is birational, the roles of destabilising and inverse destabilising curve will swap when you replace $f$ with $f^{-1}$. 
 
 %%%
 
\section{Minimal Stabilisation}\label{sec:jud}

\begin{defn}
 We say that a destabilising orbit $p, f(p), \dots, f^{n-1}(p)$ is \emph{minimal} iff the $p_j = f^{j-1}(p)$ are all distinct closed points where for $1 \le j < n$ we have $p_j \nin I(f)$ and for $1 < j \le n$ we have $p_j \nin I(f^{-1})$.
\end{defn}

Equivalently, the points of a minimal destabilising orbit do not contain any shorter destabilising orbits. One can show that any destabilising orbit of minimum length $n$ is \emph{minimal}, so these must always exist when $f$ is not algebraically stable.

%\begin{lem}\label{lem:blowuporbit}
% Suppose that $f: X \dashto X$ has a minimal destabilising orbit $p_1, p_2, \dots, p_n$. Let $\pi : \hat X \to X$ be the blowup each of the closed points $p_j$ and $\hat f = \pi^{-1} \circ f \circ \pi$. Let $\alpha : \sgraph_{\!f} \to X$ and $\hat \alpha : \sgraph_{\!\hat f} \to \hat X$ be the first projections of the smooth graphs of $f$ and $\hat f$ respectively.\\
% Then $\comp(\hat \alpha) = \comp(\alpha) -1$.
%\end{lem}
%
%\begin{proof}
% The proof is by induction on $n$. We initially define $Y_n = X$ and $g_n = f : Y_n \dashto Y_n$, then recursively define $g_j : Y_j \dashto Y_j$ for $0 \le j < n$ (in reverse order). We assert that for $j \ge 1$, $g_j$ will have a minimal destabilising orbit $(p_r)_{r=1}^j$ of length $j$, noting that $g_n$ has this orbit by hypothesis.
% 
% Let $\pi_n : Y_{n-1} \to Y_n$ be the single point blowup over $p_n$, with an exceptional curve $E \subset Y_{n-1}$. Define $g_{n-1} : Y_{n-1} \dashto Y_{n-1}$ as the lift $\pi_n^{-1} \circ g_n \circ \pi_n$. Write $\sgraph_n = \sgraph_{g_n}$ and $\alpha_n, \beta_n : \sgraph_n \to Y_n$ for the projection maps of $\sgraph_n$. 
% We aim to show that $\sgraph_{n-1}$ is a single blowup of $\sgraph_n$ when $n >1$, and that $\sgraph_0$ is isomorphic to $\sgraph_1$. In either of the cases, one can check using the minimality of $\alpha_n$ that in fact $\sgraph_{n-1}$ really is the smooth graph $\sgraph_{g_{n-1}}$.
%
% \begin{center}
% \begin{tikzcd}[column sep = 2.5em, row sep = 2.5em]
%    & \hat\sgraph \arrow[dotted]{rd}{\beta''}\arrow{d}{\rho}\arrow[dotted, swap]{ld}{\alpha''}\\
%  Y_{n-1}\arrow{d}[swap]{\pi_n} & \sgraph_n \arrow[dotted]{l}[swap]{\alpha'}\arrow[dotted]{r}{\beta'}\arrow{ld}[swap]{\alpha_n}\arrow{rd}{\beta_n}& Y_{n-1} \arrow{d}{\pi_n}\\
%  Y_n \arrow[dashed]{rr}{g_n}& & Y_n
%  \end{tikzcd}
%\end{center}
%
%First consider the left hand side of the diagram above, for any $n$. By definition $p_n \in I(g_n)$, so $p_n \in I(\alpha_n^{-1})$. This means that by \autoref{prop:factorblowup}, $\alpha_n$ factors as $\pi_n \circ \alpha'$ where $\alpha'$ is a birational morphism and $\comp(\alpha') = \comp(\alpha_n) - 1$.
%
%Second consider the case where $n > 1$. Then $p_n \nin I(g_n^{-1})$ and so $p_n \nin I(\beta_n^{-1})$. This means $\beta_n$ is a local isomorphism around $\beta_n^{-1}(p_n) \in \sgraph_n$ and $p_n \in Y_n$. Now blowup $\beta_n^{-1}(p_n)$ to get $\rho : \hat\sgraph \to \sgraph_n$. By \autoref{prop:factorblowup} we get a birational morphism $\beta'' : \hat\sgraph \to Y_{n-1}$. Now define $\alpha'' = \alpha' \circ \rho : \hat\sgraph \to Y_{n-1}$; then $g_{n-1}$ factors as $g_{n-1} = \beta'' \circ \alpha''^{-1}$ through $\hat\sgraph$.
% 
% The points $p_1, \dots, p_{n-1}$ are distinct from $p_n$; thus the $\pi_n^{-1}(p_j)$ ($1 \le j < n$) are each closed points. Because $g_n^{-1}(p_1)$ is a curve, $\pi_n^{-1}(g_n^{-1}(p_1)) \sm E$ also has dimension $1$; then one may check that $\pi_n^{-1}(p_1) \in I(g_{n-1}^{-1})$. Since $p_{n-1} \nin I(g_n)$, we have $g_{n-1}(\pi_n^{-1}(p_{n-1})) = \pi_n^{-1}(g_n(p_{n-1})) = \pi_n^{-1}(p_n) = E$ and similarly $\pi_n^{-1}(p_{j-1})$ maps to $\pi_n^{-1}(p_j)$. Therefore \[\pi_n^{-1}(p_1), \dots, \pi_n^{-1}(p_{n-1})\] is a minimal destabilising orbit for $g_{n-1}$ of length $n-1$. 
%We let $\sgraph_{n-1} = \hat\sgraph$, $\alpha_{n-1} = \alpha''$ and $\beta_{n-1} = \beta''$. Note that $\comp(\alpha_{n-1}) = \comp(\alpha') + 1 = \comp(\alpha_n) +1 - 1 = \comp(\alpha_n)$.
%
%Third suppose $n = 1$. Then $p_1 \in I(g_1^{-1})$ and so $p_1 \in I(\beta_1^{-1})$. By \autoref{prop:factorblowup}, $\beta_1$ factors as $\pi_n \circ \beta'$ where $\beta'$ is a birational morphism. % with $\comp(\beta') = \comp(\beta_n) - 1$.
%In this case set $\sgraph_0 = \sgraph_1$ with projection maps $\alpha_0 = \alpha'$ and $\beta_0 = \beta'$. Then $g_0 = \beta_0 \circ \alpha_0^{-1}$ and $\comp(\alpha_0) = \comp(\alpha_1) - 1$.
%
%Observe that $Y_0 = \hat X$, is the blowup of $X$ along the original minimal destabilising orbit $p_1, \dots, p_n$ via $\pi = \pi_n \circ \cdots \circ \pi_1$. Moreover $\hat f = g_0$, $\sgraph_{\hat f} = \sgraph_0$, $\hat \alpha = \alpha_0$, and $\hat \beta  = \beta_0$.
%%
%\[\comp(\alpha) \equiv \comp(\alpha_n) = \comp(\alpha_{n-1}) = \cdots = \comp(\alpha_1) = \comp(\alpha_0) + 1 \equiv \comp(\hat\alpha) + 1\]
%\end{proof}


%%%

\begin{proof}[Proof of \autoref{prop:minorbit}]
First note that by applying \autoref{prop:factorblowup} ($n$ times), if $\rho$ blows up all the $p_j$ at least once, then we get a new birational morphism $\nu$ which provides the factorisation $\rho = \pi \circ \nu$. Therefore we will proceed to show that $\rho$ does indeed blowup the $p_j$.

Suppose not; then there is a largest $m \le n$ such that $\hat p_m = \rho^{-1}(p_m)$ is a closed point in $Y$, so we claim this is indeterminate for $g$. Say $m = n$ and $f(p_n) = D \subset X$, then $\hat p_n \in I(g)$ if $g(\hat p_n)$ is a curve. It is enough to show that $\rho(g(\hat p_n))$ is a curve. The composition $\rho \circ g$ is stable since $I(\rho) = \emp$, therefore \[\rho(g(\hat p_n)) = \rho \circ g(\hat p_n) = f \circ \rho(\hat p_n) = f(p_n) = D\] which is a curve; note that the last step is because $\rho$ is locally an isomorphism at $\hat p_n$. If $m < n$ then $p_m \nin I(f)$, so the composition $\rho^{-1} \circ f$ is locally stable, as is $\rho^{-1} \circ f \circ \rho = g$ near $\hat p_m$. Therefore \[g(\hat p_m) = \rho^{-1} (f(\rho(\hat p_m))) = \rho^{-1}(p_{m+1})\] which is also a curve by assumption.

Suppose that $k\le m$ is minimal such that $p_k, \dots, p_m$ are not blown up by $\rho$ and either $k=1$ or $k-1$ is blown up by $\rho$. Using \autoref{prop:factorblowup} one sees that $\hat p_k \mapsto \hat p_{k+1} \mapsto \cdots \mapsto \hat p_m$. To show that this is a destabilising orbit for $g$, we only need to show that $g^{-1}(p_k)$ is a curve. This is very similar to the case of $p_m$ above.

$\rho$ is locally an isomorphism near $\hat p_k$. So it is enough to show that $\rho^{-1}\circ f^{-1}(p_k)$ is a curve. Say $k=1$ and $f^{-1}(p_1) = C \subset X$, then $\rho^{-1}\circ f^{-1}(p_1)$ contains at least $\rho^{-1}(C\sm I(f))$ which is a (most of) a curve. Otherwise $k > 1$, so $p_{k-1} \nin I(f)$ and the composition $f \circ \rho$ is algebraically stable over $\rho^{-1}(p_{k-1})$ which is a curve $\hat C$. We get that $\rho^{-1}\circ f^{-1}(p_k) = \hat C$.

 \begin{center}
\begin{tikzcd}[column sep = 2.5em, row sep = 2.5em]
Y \arrow[swap]{d}{\rho} \arrow[dashed]{r}{g} & Y \arrow{d}{\rho}\\
 X \arrow[dashed]{r}{f} & X
\end{tikzcd}
\end{center}
%
%------------\unfinished[inline]{Editing proof. Old below.}
%
% The proof is by induction on $n$. We initially define $X_n = X$, $f_n = f : X_n \dashto X_n$, and $\nu_n = \rho$, then we will recursively define $f_j : X_j \dashto X_j$ for $0 \le j < n$ (in reverse order) along with $\nu_j : Y \to X_j$. We will show that for $j \ge 1$, $f_j$ will have a minimal destabilising orbit $(p_r)_{r=1}^j$ (of length $j$), noting that $f_n$ has this orbit by hypothesis.
% 
% Let $\pi_n : X_{n-1} \to X_n$ be the single point blowup over $p_n$. Define $f_{n-1} : X_{n-1} \dashto X_{n-1}$ as the lift $\pi_n^{-1} \circ f_n \circ \pi_n$
% 
% \begin{center}
%\begin{tikzcd}[column sep = 2.5em, row sep = 2.5em]
% X_{n-1} \arrow[swap]{d}{\pi_n}& Y \arrow[swap, red]{l}{\nu_{n-1}} \arrow{ld}{\nu_n}\\
% X_n
%\end{tikzcd}
%\end{center}
%
% \begin{center}
%\begin{tikzcd}[column sep = 2.5em, row sep = 2.5em]
%Y \arrow[swap]{d}{\nu_n} \arrow[dashed]{r}{g} & Y \arrow{d}{\nu_n}\\
% X_n \arrow[dashed]{r}{f_n} & X_n
%\end{tikzcd}
%\end{center}
%
%First note that by \autoref{prop:factorblowup}, if $\nu_n$ blows up $p_n$ at least once, then we get a new birational morphism $\nu_{n-1}$ which provides the factorisation $\nu_n = \pi_n \circ \nu_{n-1}$. Therefore we want to show that $\nu_n$ does indeed blow up $p_n$.
%
%Suppose not; then $\hat p_n = \nu_n^{-1}(p_n)$ is a closed point in $Y$ and we claim this is indeterminate for $g$. Say $f_n(p_n) = D \subset X_n$, and let $\hat D \subset Y$ be the proper transform of $D$ by $\nu_n^{-1}$; therefore, avoiding destabilising curves we have at least \[\nu_n^{-1}\circ f_n(p_n) \supset \overline{\nu_n^{-1}(f_n(p_n) \sm I(\nu_n^{-1}))} = \overline{\nu_n^{-1}(D \sm I(\nu_n^{-1}))} = \hat C.\] Whence $g(\hat p_n) = \nu_n^{-1}\circ f_n \circ \nu_n(\hat p_n) \supset \hat D$ and we have shown $\hat p_n \in I(g)$.
%
%Next, say that $\nu_n$ blows up $p_k$ to a curve $\hat C\subset Y$ but merely lifts $p_{k+1}, \dots, p_n$ to the closed points $\hat p_{k+1}, \dots, \hat p_n$. Then since these points are not indeterminate for $f_n$, it is clear that $g^{n-k}(E) = \nu_n^{-1}\circ f_n^k \circ \nu_n(E) = \hat p_n$. Now we have a destabilising orbit $E, \hat p_{k+1}, \dots, \hat p_n, \hat D$, $\contra$. Otherwise we deduce that none of the $p_j$ are blown up to curves by $\nu_n$, in which case the proper transform $\hat C$ of $C = f_n^{-1}(p_1)$ provides a destabilising curve for $g$ with orbit $\hat C, \hat p_1, \dots, \hat p_n, \hat D$. This provides the final contradiction ($\contra$) to the assumption $p_n$ was not blown up by $\nu_n$.
%
%If $n=1$ the induction is complete finishing with $\nu_1 = \pi_1 \circ \nu_0$. We may write $\pi = \pi_n \circ \cdots \circ \pi_1$, $X_0 = \hat X$, $\nu_0 = \nu : Y \to \hat X$. Then \[\rho = \nu_n = \pi_n \circ \nu_{n-1} = \pi_n \circ \pi{n-1} \circ \nu_{n-2} = \cdots = \pi_n \circ \cdots \circ \pi_1 \circ \nu_0 = \pi \circ \nu\] as required.
%
%If $n > 1$ then and consider that by minimality of the orbit, $p_j\ (1 \le j \le n-1)$ lifts by $\pi_n$ to points which $f_{n-1}$ still maps continuously from one to the next, moreover the proper transform $\pi_n^{-1}(C)$ maps to $\pi_n^{-1}(p_1)$, and finally $f_{n-1}(\pi^{-1}(p_n)) = \pi^{-1}(p_n)$ is a curve. Therefore $\pi^{-1}(p_1), \dots, \pi^{-1}(p_{n-1})$ is a minimal destabilising orbit for $f_{n-1}$. That completes the inductive step.
\end{proof}

%%%

\begin{proof}[Proof of \autoref{thm:minimal}]
%In the birational case, suppose that $f_m : X_m \dashto X_m$ is not algebraically stable with minimal destabilising orbit $p_1, p_2, \dots, p_n$, and define $f_{m+1}$ as stated. Then consider the smooth graphs $\sgraph_{\!f_j}$ with projections $\alpha_j : \sgraph_{\!f_j} \to X_j$ for $j = m, m+1$.
%Now \autoref{lem:blowuporbit} says precisely that $\comp(\alpha_{m+1}) = \comp(\alpha_m) - 1$.
%
%The sequence $\comp(\alpha_m) \ge 0$ is strictly decreasing in $m$, thus it is clear that the process must terminate at some $m = M$. Whence $f_M$ is algebraically stable.
%
Given that $g : Y \dashto Y$ dominates $f$ via $\rho : Y \to X = X_1$ we may proceed inductively on $m$ with the hypothesis that $g : Y \dashto Y$ dominates $f_n : X_m \dashto X_m$ via $\nu_m : Y \to X_m$.

If $f_m$ is algebraically stable we are done, otherwise \autoref{prop:minorbit} says that because $\pi_m$ blows up a minimal destabilising orbit we have a $\nu_{m+1} : Y \to X_{m+1}$ which factors $\nu_m$ as $\nu_m = \pi_m \circ \nu_{m+1}$.
\begin{center}
\begin{tikzcd}[column sep = 2.5em, row sep = 2.5em]
 X_{m+1} \arrow[swap]{d}{\pi_m}& Y \arrow[swap, red]{l}{\nu_{m+1}} \arrow{ld}{\nu_m}\\
 X_m
\end{tikzcd}
\end{center}
Clearly there is no limit to the number of times we can do this if $f_m$ is never algebraically stable for $m \ge 1$. However overall we have shown that \[\rho = \nu_1 = \pi_1 \circ \nu_2 = \pi_1 \circ \pi_2 \circ \nu_3 = \cdots = \pi_1 \circ \cdots \circ \pi_m \circ \nu_{m+1} = \pi \circ \nu_{m+1}\] meaning that $\comp(\rho) \ge m$. Therefore $\set[m \in \N]{f_m \text{ is not AS}}$ is in fact bounded and so on the final value $m = M$, $f_M$ is algebraically stable. Moreover, $\rho = \pi \circ \nu$ where we define $\nu = \nu_M$.
%Let $X_M = \hat X$, $\hat f = f_M$ and $\pi = \pi_1 \circ \cdots \circ \pi_{M-1} : \hat X \to X$

We have shown that $\pi$ factors any birational morphism stabilising $f$, and to finish we apply this to get uniqueness of $(\hat f, \hat X)$. Suppose we proceed in the minimal stabilisation algorithm in two different ways which produce two (potentially different) models, namely $\hat f_1 : \hat X_1 \dashto \hat X_1$ via $\pi_1 : \hat X_1 \to X$ and $\hat f_2 : \hat X_2 \dashto \hat X_2$ via $\pi_2 : \hat X_2 \to X$. By the above we have that $\pi_1 = \pi_2 \circ \nu_1$ and $\pi_2 = \pi_2 \circ \nu_2$. We deduce that $\nu_1, \nu_2$ are inverse morphisms to each other, providing an isomorphism not only of surfaces but dynamical systems $\nu_1 : (\hat f_1, \hat X_1) \leftrightarrow (\hat f_2, \hat X_2)$.
%
\begin{center}
\begin{tikzcd}[column sep = 1.5em, row sep = 2.5em]
 \hat X_1 \arrow[swap]{dr}{\pi_1} \arrow[swap, bend right=10]{rr}{\nu_1} && \hat X_2 \arrow[swap, bend right=10]{ll}{\nu_2} \arrow{ld}{\pi_2}\\
 &X
\end{tikzcd}
\end{center}
\end{proof}

%\begin{rmk}
% In fact the proof shows that $M \le \comp(\alpha_1)$.
%\end{rmk}

%

%The following corollary provides an example application of \autoref{thm:minimal}. For brevity, denote the rational dynamical system $f : X \dashto X$ by $(f, X)$ and denote by $h : (f, X) \dashto (g, Y)$ a birational map $h : X \dashto Y$ along with the conjugation relation $g = h \circ f \circ h^{-1}$. 
%
%\begin{cor}[Minimal models for AS maps]
% Let $f : X \dashto X$ be an algebraically stable rational map and $M$ a minimal model with $\rho : (f, X) \to (f_M, M)$. Then the stabilisation $(\hat f_M, \hat M)$ exists and there is a $\nu : (f, X) \to (\hat f_M, \hat M)$ factoring $\rho$. Conversely, every rational map $(g, Y)$ dominating $(f_M, M)$, also dominates $(\hat f_M, \hat M)$. 
%\end{cor}

%%

\section{\Nice{} maps}\label{sec:nice}

\begin{defn}
 Let $f : X \dashto Y$ be a birational map. We say $f$ is \emph{\nice{}} iff \[\E(f) \cap I(f) = \emp.\]
\end{defn}

The following results are auxiliary to \autoref{thm:blowupev} but also of independent interest.

\begin{prop}\label{prop:curveex}
 Suppose $f : X \dashto Y$ is a birational map with smooth graph $\sgraph_{\!f}$ as in the diagram below. Then $f$ is \nice{} if and only if $\E(\alpha) \cap \E(\beta) = \emp$. In this case the following additional properties hold
%
\begin{enumerate}[ref=\emph{(\alph*)}, label=\emph{(\alph*)}]
\item $\alpha : \E(\beta) \to \E(f)$ and $\beta : \E(\alpha) \to \E(f^{-1})$ are isomorphisms;
 \item $\comp(f) = \comp(\beta)$ and $\comp(f^{-1}) = \comp(\alpha)$;
% \item $\E(\alpha) \cap \E(\beta)$ is dimension $0$;
  \item $\Gamma_{\!f} \cong \sgraph_{\!f}$ is smooth.
\end{enumerate}
  \begin{center}
 \begin{tikzcd}[column sep = 1.5em, row sep = 2.5em]
 & \sgraph_{\!f} \arrow{ld}[swap]{\alpha}\arrow{rd}{\beta}&  \\
  X \arrow[dashed]{rr}{f} & & Y
  \end{tikzcd}
\end{center}
\end{prop}
%
We leave the proof as an exercise to the reader. The next lemma gives another equivalent formulation of \nice{}ness which requires a little more argument.

\begin{lem}\label{lem:untanglingcomp}
Let $f : X \dashto Y$ be a birational map. Suppose $f$ can be written as $h^{-1} \circ g$ where $g : X \to Z,\ h : Y \to Z$ are birational morphisms, then $f$ is \nice{}.

Moreover $\E(f) \subseteq \E(g)$ and $\E(f^{-1}) \subseteq \E(h)$ with equality if and only if the composition $h^{-1} \circ g$ has no destabilising curves. Conversely an \nice{} $f$ always has such a decomposition.
\end{lem}

\begin{proof}
We prove first and second part by induction; the converse is left as a further exercise. We claim that if $h^{-1} \circ g$ has a destabilising curve then we can blowup $Z$ by $\pi : Z' \to Z$ to get a simpler decomposition $h'^{-1} \circ g'$ where $\comp(g') = \comp(g) - 1$ and $\comp(h') = \comp(h) - 1$. This produces a destabilising curve $C$ such that $C$ is mapped to a curve $D$ by $f$ (proper transform) and vice versa. When $f = h^{-1} \circ g$ has no destabilising curves, we conclude that $f$ is \nice{}. This process terminates because $\comp(g), \comp(h)$ cannot decrease below $0$.

Suppose we have such a destabilising curve, meaning $p \in Z$ such that $g^{-1}(p)$ and $h^{-1}(p)$ are curves. Now we blowup $p \in Z$ by $\pi : Z' \to Z$ with exceptional curve $E = \pi^{-1}(p)$. By \autoref{prop:factorblowup}, $g$ factors as $\pi \circ g'$ with $\comp(g') = \comp(g) - 1$ and $h$ factors as $\pi \circ h'$ with $\comp(h') = \comp(h) - 1$.

    \begin{center}
 \begin{tikzcd}[column sep = 1.5em, row sep = 2.5em]
  C \arrow[draw=none]{r}[sloped,auto=false]{\subseteq} & X \arrow{rrd}[swap]{g'} \arrow[bend right]{rrdd}[swap]{g} \arrow[dashed]{rrrr}{f}&&&& Y \arrow{lld}{h'} \arrow[bend left]{lldd}{h} &  D \arrow[draw=none]{l}[sloped,auto=false]{\supseteq} \\
&& E\arrow[draw=none]{r}[sloped,auto=false]{\subseteq} &Z' \arrow{d}{\pi} & ~\\
  && p \arrow[draw=none]{r}[sloped,auto=false]{\in}& Z
  \end{tikzcd}
  \end{center}
 
The proper transform of $E$, $C = \overline{g'^{-1}(E \sm I(g'^{-1}))} \subset X$, is an irreducible curve; or more simply we have $g'(C) = E$. Similarly there is an irreducible curve $D$ such that $h'(D) = E$. Whence $D$ is the proper transform of $C$ by $f = h'^{-1} \circ g'$. This completes the claim.
 
 It remains to show that if the decomposition $f = h^{-1} \circ g$ has no destabilising curves then $f$ is \nice{}, plus $\E(g) = \E(f)$ and $\E(h) = \E(f^{-1})$. Let $C \subset \E(g)$ be a curve, then $g(C) = p \in Z$ is a closed point. If $p \nin I(h^{-1})$ then clearly $f = h^{-1} \circ g$ is continuous on $C$ with a closed point as an image, therefore $I(f) \cap C = \emp$ and $C \subseteq \E(f)$. Otherwise if $p$ is indeterminate, then we know there is a curve $D \subseteq h^{-1}(p) \subseteq \E(h)$; whence $C$ is a destabilising curve which pairs with the inverse destabilising curve $D$, contradicting our assumption. Thus $C \subseteq \E(f)$. Conversely, if $f(C \sm I(f))$ is a closed point then certainly $g(C)$ is a closed point since $\E(h^{-1}) = \emp$; so $C \subseteq \E(g)$. We have shown that $\E(g) = \E(f)$; a similar argument shows that $\E(h) = \E(f^{-1})$, absent destabilising curves.
\end{proof}

\begin{cor}\label{cor:graphnice}
  Let $f : X \dashto X$ be a birational map and let $\hat f$ be the lift of $f$ defined by the following commutative diagram, where $\sgraph_{\!f}$ is the smooth graph of $f$. Then $\hat f$ is \nice{}.
  %
    \begin{center}
 \begin{tikzcd}[column sep = 1.5em, row sep = 2.5em]
  & \sgraph_{\!f} \arrow{ld}[swap]{\alpha} \arrow[dashed]{rr}{\hat f} \arrow{rd}{\beta}&& \sgraph_{\!f} \arrow{ld}{\alpha} \\
  X \arrow[dashed]{rr}{f} && X
  \end{tikzcd}
  \end{center}
\end{cor}

\begin{proof}
 $\hat f$ is \nice{} due to \autoref{lem:untanglingcomp} because $\hat f = \alpha^{-1} \circ \beta$.
 \end{proof}

\begin{cor}\label{cor:equivexceptionals}
  Let $f : X \dashto X$ be an \nice{} birational map and let $\hat f$ be the lift of $f$ defined as above. Then $\alpha : \E(\hat f) \hookrightarrow \E(f)$ and $\beta : \E(\hat f ^{-1}) \hookrightarrow \E(f ^{-1})$ are injections, which are surjective if and only if $f$ has no length $1$ destabilising orbits.
\end{cor}

\begin{proof}
 By \autoref{lem:untanglingcomp} $\E(\hat f) \subseteq \E(\beta)$ and $\E(\hat f^{-1}) \subseteq \E(\alpha)$ with equality iff $\alpha^{-1} \circ \beta$ has no destabilising curves; on the other hand \autoref{prop:curveex} says that $\alpha : \E(\beta) \to \E(f)$ and $\beta : \E(\alpha) \to \E(f^{-1})$ are isomorphisms. Therefore to conclude we only need to show that the composition $\alpha^{-1} \circ \beta$ has a destabilising curve $\hat C$ and inverse destabilising curve $\hat D$ if and only if the composition $f\circ f$ has a destabilising curve $C = \alpha(\hat C)$ and inverse destabilising curve $D = \beta(\hat D)$.
 %
  Indeed, by \autoref{prop:curveex} $C$ is a curve and $f(C) = p$ if and only if $\hat C$ is a curve and $\beta(\hat C) = p$; similarly $D$ is a curve and $f^{-1}(D) = p$ if and only if $\hat D$ is a curve and $\alpha(\hat D) = p$.
\end{proof}

%%%
 
\section{Stabilisation Through Graphs}\label{sec:blowupev}
 
\begin{defn}
 Let $\mathcal D(X, f)$ be the set of all triples $(C, D, n)$ such that $C$ is a destabilising curve for $f : X \dashto X$ with an orbit of length $n$ and inverse destabilising curve $D$.
 \end{defn}

The following proposition is the heart of \autoref{thm:blowupev}.
 
\begin{prop}\label{prop:blowupev}
 Let $f : X \dashto X$ be an \nice{} birational map and $\hat f : \sgraph_{\!f} \dashto \sgraph_{\!f}$ be the lift described above in \autoref{cor:graphnice}.
Then there exists a well defined injection
\begin{align*}
\iota : \mathcal D(\sgraph_{\!f}, \hat f) &\longrightarrow \mathcal D(X, f) \\
(\hat C, \hat D, n) &\longmapsto (\alpha(\hat C), \beta(\hat D), n+1).
\end{align*}
If $\iota$ is surjective (a bijection) then $\comp(\hat f) = \comp(f)$.
If $\iota$ isn't surjective then $\comp(\hat f) < \comp(f)$.
\end{prop}

\begin{proof}
To justify that $\iota$ is well defined, we claim that every destabilising orbit upstairs descends to one downstairs; to be precise, if $(\hat C, \hat D, n) \in \mathcal D(\sgraph, \hat f)$ then $(C, D, n+1) \in \mathcal D(X, f)$ where $C = \alpha(\hat C)$ and $D = \beta(\hat D)$.

Assume $\hat C$ is a destabilising curve, $\hat f(\hat C) = \hat p$, % ($\hat f$ is \nice{} so indeterminacy isn't a problem),
  $\hat f^{n-1}(\hat p) \ni \hat q \in I(\hat f)$ and $\hat f(\hat q) \supseteq \hat D$. Let $D = \beta(\hat D), C = \alpha(\hat C)$; as shown in \autoref{cor:equivexceptionals} $C \subseteq \E(f)$ with $f(C) = p = \alpha(\hat p)$, and $D \subseteq \E(f^{-1})$ with $q = \beta(\hat q) = f^{-1}(D)$. To complete the claim we need to show that $f^n(p) \ni q$.
 Indeed consider the composition $f^n = \beta \circ \hat f^{n-1} \circ \alpha^{-1}$; $\E(\alpha^{-1}) = \emp$ and $I(\beta) = \emp$ therefore by \autoref{prop:graphcomp} the total transforms are functorial. We also know that $\alpha^{-1}(p) \ni \hat p$ and $\beta(\hat q) = q$.
 %
 \[f^n(p) = \beta ( \hat f^{n-1} ( \alpha^{-1}(p))) \supseteq \beta ( \hat f^{n-1} (\{\hat p\})) \supseteq \beta(\{\hat q\}) = \{q\}\]
 Injectivity follows from the injectivity given in \autoref{cor:equivexceptionals} and the simple fact that $\iota(\hat C, \hat D, m) = (C, D, n)$ implies $m=n - 1$.
 
For the surjectivity, we claim that $\iota$ is surjective if and only if we can find a length $1$ destabilising orbit for $f$, that is $(C, D, 1) \in \mathcal D(X, f)$. Then \autoref{cor:equivexceptionals} finishes the proof since we know $\comp(\hat f) \le \comp(f)$ with equality if and only if we can find a length $1$ destabilising orbit for $f$.

Clearly, if $(C, D, 1) \in \mathcal D(X, f)$ then $(C, D, 1)$ cannot have a preimage under $\iota$ since no destabilising orbit has length $1-1=0$. Conversely we show in the remainder of this proof that when $\mathcal D(X, f)$ has no such triples we can find a preimage for $(C, D, n) \in \mathcal D(X, f)$. Write $\hat C = \alpha^{-1}(C)$, $\hat D = \beta^{-1}(D)$, $f(C) = p$, and $q = f^{-1}(D)$. By \autoref{cor:equivexceptionals}, $\hat C \subseteq \E(f)$, $\hat D \subseteq \E(f^{-1})$, $p \nin I(f) = I(\alpha^{-1})$, and $q \nin I(f^{-1}) = I(\beta^{-1})$. Write $\hat p = \hat f(\hat C) = \alpha^{-1}(p)$ and $\hat q = f^{-1}(\hat D) = \beta^{-1}(q)$, both closed points.

We wish to show that $\hat f^{n-2}(\hat p) \ni \hat q$ given that we know $f^{n-1}(p) \ni q$. Recall that $p \nin I(\alpha^{-1})$ and $q \nin I(\beta^{-1})$, so by \autoref{prop:graphcomp} the total transform of composition $\hat f^{n-2} = \beta^{-1} \circ f^{n-1} \circ \alpha$ is functorial at $\hat p$ in the following way.
 %
 \[\hat f^{n-2}(\hat p) = \beta^{-1} \circ f^{n-1} ( \alpha(\hat p)) = \beta^{-1} \circ f^{n-1} (p) \supseteq \beta^{-1}(\{q\}) = \{\hat q\}.\]
 Therefore $(\hat C, \hat D, n-1) \in \mathcal D(\sgraph_{\!f}, \hat f)$.
  \end{proof}

\begin{proof}[Proof of \autoref{thm:blowupev}]
 First, note that by \autoref{prop:curveex} $X_m$ is smooth for all $m \ge 1$, then by \autoref{cor:graphnice}, $f_m$ is \nice{} for $m \ge 1$; assume that $(X_1, f_1)$ is not algebraically stable.
 
 If $f_m$ is not algebraically stable then we may choose $(C, D, n) \in \mathcal D(X_m, f_m)$. By \autoref{prop:blowupev}, $\comp(f_{m+1}) \le \comp(f_m)$ and either we have strict inequality or all destabilising orbits lift to strictly shorter destabilising orbits. Since lengths of orbits must be positive, eventually we find an $m' \le m+n$ such that $\comp(f_{m'}) < \comp(f_m)$.
 
 The sequence $k(f_m) \ge 0$ must stabilise as $m$ increases with $\comp(f_m) = \comp(f_M)$ for all $m \ge M$. Then $\mathcal D(X_m, f_m) = \emp$ for all such $m$, otherwise we could decrease $\comp(f_m)$ further as above.
\end{proof}

\autoref{prop:blowupev} and the theorem admit a shorter proof, ignoring $\comp(f_m)$ when $\mathcal D(X, f)$ is finite. 
 With a little more work, one can show that a length $n$ destabilising orbit on $\sgraph_f$ proves the existence of a length $n+1$ orbit on $X$, even if it's not given by $\iota$ as in \autoref{prop:blowupev}. This gives the following bound.

\begin{cor}
Let $f : X \dashto X$ be a birational map and the sequence $(X_m, f_m)$ be as given in \autoref{thm:blowupev}. Suppose that $N$ is an upper bound on lengths of a destabilising orbit for $f$. Then $\forall m \ge N$, $f_m$ is algebraically stable.
\end{cor}

\section{An Example}\label{sec:counterex}

\subsection{Intro}

The goal of this section is to prove \autoref{thm:counterex}. 
This demonstrates that there are rational maps which can be stabilised by a birational conjugacy but not by blowups of the surface alone. 
Recall that $f : \C^2 \dashto \C^2$ is given by \[(x, y) \longmapsto (x^2, x^4y^{-3} + y^3) = \left(x^2, \frac{x^4 + y^6}{y^3}\right)\]

Let us initially compactify $(f, \C^2) \dashto (\tilde f, \P^1 \times \P^1)$ in the obvious way, by adding a point at infinity to each factor. Unfortunately, $\set{x=\infty}$ is a destabilising curve for $\tilde f$. % : \P^1 \times \P^1 \dashto \P^1 \times \P^1$.
One may check that $\tilde f(\set{x=\infty})$ is the indeterminate point $(\infty, \infty)$. However if we modify this fibre to create $X$, the first Hirzebruch surface, then $f : X \dashto X$ is algebraically stable.

\begin{prop}\label{prop:initialstability}
 %Let $\tilde f : \P^1 \times \P^1 \dashto \P^1 \times \P^1$ be the extension of $f$ from $\C^2$ to $\tilde f : \P^1 \times \P^1$. %Then the exceptional set $E(\tilde f)$ is $E_\infty = \set{x=\infty}$.
 Let $\psi : (\tilde f, \P^1 \times \P^1) \dashto (f, X)$ be the birational transformation obtained by blowing up $(\infty, \infty)$ and then blowing down the proper transform of $\set{x = \infty}$. Then the exceptional set, $E(f)$, is $E_\infty = \phi(\infty, \infty)$, and $f(E_\infty) \nin I(f)$ is a fixed point. In particular, $f$ is algebraically stable.
\end{prop}

This proposition can be verified by local coordinate computations which we leave to the reader. %Note that the techniques given below provide a simple proof of this after a change in coordinates.

We define $E_0 = E_{\frac 01}$ to be $\set{x=0} \subset \C \times \P^1 \subset X$. Next we define the blowup $\sigma_0 : (f_0, X_0) \to (f, X)$ centred on $(0, 0) \in \C^2 \subset X$ and let $E_1 = E_{\frac 11} = E(\sigma_0)$ be the exceptional curve. A consequence of our proof will be that $f_0$ is not algebraically stable, but we will also see this directly from \autoref{prop:mapexc}, later.

The method we provide in the next two subsections is elementary, however analysing these examples on Berkovich space is much faster and more informative. The reader who is familiar with Berkovich theory may skip to \autoref{sec:berkalt}.

\subsection{Satellite Blowups}

Before analysing $f$ further, we introduce a convenient bookkeeping system for blowups over the origin (see \cite[\S 15.1]{Jon} and \cite[appx]{DL} for precedents).

%We now introduce both a definition for blowing up $\P^1 \times \P^1$ and also a natural recursive labelling scheme for the exceptional curves produced, using Farey addition.

\begin{defn}
% Suppose we have two curves $E, E'$ on a surface $Y$, with a single transverse intersection $E \cap E' = P$. A (single) \emph{satellite blowup} $\sigma : Z \to Y$ between $E$ and $E'$ is the (single) blowup of $Y$ at $P$. We notate the exceptional curve created by $E(\sigma)$.
 %
 A birational morphism $\phi : Y \to X_0$ is \emph{satellite (relative to $E_0$ and $E_1$)} iff $\phi = \sigma_1 \circ \sigma_2 \circ \cdots \circ \sigma_n$ ($n \ge 1$) and for every $1 \le j \le n$ we have that $\sigma_j$ is the blowup of the point of intersection between two curves in the list $E_0, E_1, E(\sigma_1), E(\sigma_2), \dots , E(\sigma_{j-1})$.
\end{defn}

Note that in this definition and throughout this section we will adopt the convention that if $\phi : Y \to Z$ is a birational morphism between surfaces and $C \subset Z$ is a curve then $C$ will also denote the proper transform $\overline{\phi^{-1}(C)\sm \E(\phi)}$ of $C$ in $Y$.

 Now given any birational morphism $\phi : Y \to X_0$ which is \emph{satellite}, as above, we proceed to index the exceptional curves of $\phi$, by rational numbers $\frac ab \in (0, 1)$ in lowest terms as follows. For each $j \ge 1$, % there are, by definition two curves in the list $E_0, E_1, E(\sigma_1), E(\sigma_2), \dots , E(\sigma_{j-1})$ with single point of intersection which $\sigma_j$ blows up to produce $E(\sigma_j)$. By induction these two have already been labelled $E_{\frac ab}$ and $E_{\frac cd}$; then we label $E(\sigma_j) = E_{\frac{a+c}{b+d}}$.\change{we will index... coprime}
if $\sigma_j$ blows up the intersection $E_{\frac ab} \cap E_{\frac cd}$ of two previously indexed curves from among $E_{\frac01}, E_{\frac11}, E(\sigma_1), E(\sigma_2), \dots , E(\sigma_{j-1})$, then we declare $E(\sigma_j) = E_{\frac{a+c}{b+d}}$. Note that the Farey sum $\frac{a+c}{b+d} \in (\frac ab, \frac cd)$ is a rational number in lowest terms.
%The Farey fractions produced have the property that numerator and denominator are naturally always coprime, indeed the definition above requires this. We also have that \[\frac ab < \frac cd \implies \frac ab < \frac{a+c}{b+d} < \frac cd.\]
%One can show for any $q \in \Q\cap[0, 1]$ that there is a birational morphism $\phi$ which generates $E_q$ in this way. 

Let $0 = r_0 < r_1 < \cdots < r_n = 1$ be the full list of rational indices for the curves $E_{r_j}$ as above. The \emph{dual graph} for $\phi$, with vertices $\set[E_{r_j}]{0\le j \le n}$ and edges $\set[E_{r_j}E_{r_k}]{E_{r_j} \pitchfork E_{r_k}}$, then becomes 
\begin{center}
 \begin{tikzcd}
 E_{r_0} \arrow[-]{r} & E_{r_1} \arrow[-]{r} & \cdots \arrow[-]{r} & E_{r_{n-1}} \arrow[-]{r} & E_{r_n}
 \end{tikzcd}
\end{center}
In particular blowing up $E_{\frac ab} \cap E_{\frac cd}$ corresponds to inserting a vertex as follows.
\begin{center}
 \begin{tikzcd}
 \cdots \arrow[-]{r} & E_{\frac ab} \arrow[-]{rr} &\arrow[rightsquigarrow]{d}& E_{\frac cd} \arrow[-]{r} & \cdots\\%bend right
 \cdots \arrow[-]{r} & E_{\frac ab} \arrow[-]{r} & E_{\frac{a+c}{b+d}} \arrow[-]{r} & E_{\frac cd} \arrow[-]{r} & \cdots
 \end{tikzcd}
\end{center}
We caution however that the ordering of the dual graph does not match the ordering of the $\sigma_j$, i.e.\ $E_{r_j} \ne E(\sigma_j)$ in general.


%We caution that $E_{r_j}$ is not necessarily $E(\sigma_j)$, but this ordering of the $r_j$ reflects that of the \emph{dual graph}. This is a graph with vertices $\set[E_{r_j}]{0\le j \le n}$ and edges $\set[E_{r_j}E_{r_{j+1}}]{0\le j < n}$.
%\begin{center}
% \begin{tikzcd}
% E_{r_0} \arrow[-]{r} & E_{r_1} \arrow[-]{r} & \cdots \arrow[-]{r} & E_{r_{n-1}} \arrow[-]{r} & E_{r_n}
% \end{tikzcd}
%\end{center}
%
%%One can see directly from the recursive construction that two curves are adjacent in the graph if and only if they intersect, furthermore
%Blowing up $E_{\frac ab} \cap E_{\frac cd}$ corresponds to editing the dual graph as follows:
%\begin{center}
% \begin{tikzcd}
% \cdots \arrow[-]{r} & E_{\frac ab} \arrow[-]{rr} &\arrow[rightsquigarrow]{d}& E_{\frac cd} \arrow[-]{r} & \cdots\\%bend right
% \cdots \arrow[-]{r} & E_{\frac ab} \arrow[-]{r} & E_{\frac{a+c}{b+d}} \arrow[-]{r} & E_{\frac cd} \arrow[-]{r} & \cdots
% \end{tikzcd}
%\end{center}

The curves $E_{\frac ab}$ can be seen as `degenerations' of embeddings of the complex torus $\C^*\times \C^*$ into $Y$. For any $\frac ab$ we define the map $\gamma_{\frac ab} : \P^1 \times \P^1 \dashto Y$ given on $\C^*\times \C^*$ by
\begin{align*}
 \gamma_{\frac ab} : \C^*\times \C^* &\longrightarrow \C^*\times \C^* \subset Y\\
 (s, t) &\longmapsto (t^b, st^a).
\end{align*}

More generally, we say that a rational map $\gamma : \P^1 \times \P^1 \dashto Y$ is asymptotic to $\gamma_{\frac ab}$, or $\gamma \sim \gamma_{\frac ab}$, if and only if on $\C^* \times \C^*$ we have
%\begin{align*}
% \gamma : \C^*\times \C^* &\dashrightarrow \C^2 \subset Y\\
 \[(s, t) \longmapsto (t^d + \lo(t^d), R(s)t^c + \lo(t^c))\]
%\end{align*}
where $\frac cd = \frac ab$ and $R$ is a non-constant rational function on $\P^1$.

%\unsure{mention curves?}

%It turns out to be very convenient to view the curves $E_{\frac ab}$ through the embedded planes passing through them.
%First, given $\frac ab$ we define the map 
%\begin{align*}
% \gamma_{\frac ab} : \C^*\times \C^* &\longrightarrow \C^*\times \C^* \subset Y\\
% (s, t) &\longmapsto (t^b, st^a).
%\end{align*}
%The reader should think of this as a family of curves $t \mapsto \gamma_{\frac ab}(s, t)$ on the surface, parameterised by $s$. 
%The idea is that we then look at $\lim_{t \to 0}\gamma_{\frac ab}(s, t)$ and then vary $s$: this will sweep out the curve $E_{\frac ab}$ in our model. Second, if $\gamma : \C^* \times \C^* \to Y$ is a more general embedding we say %Second, we expand our notion of what constitutes a family like $\gamma_{\frac ab}$, by defining a relation on the following similar curves $\gamma : \C^* \times \C^* \to Y$: 
%$\gamma \sim \gamma_{\frac ab}$ if and only if 
%\begin{align*}
% \gamma : \C^*\times \C^* &\dashrightarrow \C^2 \subset Y\\
% (s, t) &\longmapsto (t^d + \lo(t^d), R(s)t^c + \lo(t^c))
%\end{align*}
%where $\frac cd = \frac ab$ and $R$ is a non-constant rational function on $\P^1$.% and $\lo$ is little-o notation with respect to $t \to 0$.

\begin{prop}\label{prop:degen}
Suppose that $\gamma \sim \gamma_{\frac ab} : \P^1\times \P^1 \dashto Y$. Let $Z$ be the proper transform of $\P^1 \times \set 0$ under $\gamma$. Then either
\begin{enumerate}
 \item $\frac ab = r_j$ and $Z$ is the curve $E_{\frac ab}$; or
 \item $r_j < \frac ab < r_{j+1}$ and $Z$ is the closed point $E_{r_j}\cap E_{r_{j+1}}$.
\end{enumerate}
%The fact this is because the image must also be irreducible, rational and dimension at most $1$.
\end{prop}


\subsection{Mapping Exceptional Curves}\label{sec:mapexc}
 %All this and more is written in the following proposition which considers the \emph{proper transform} of curves and points by $f$.

%Suppose we have a rational map $\phi : (g, Y) \to (f_0, X_0)$, a satellite blowup relative to $E_0$ and $E_1$ as described in the previous subsection.
Now we compute the images of the curves $E_{r_j}$ under $g$, using \autoref{prop:degen}. The rough idea is that the point or curve represented by $\gamma_{\frac ab}$ is mapped to the point or curve represented by $f \circ \gamma_{\frac ab} \tilde \gamma_{\frac cd}$. We now compute $\frac cd$ in terms of $\frac ab$ %To do this we apply $f$ to $\gamma_{r_j}$ to get some $\gamma \sim \gamma_r$.% and then use \autoref{prop:degen}.% we will infer for example how $E_{r_j}$ maps to $E_q$ or $E_k \cap E_{r_{k+1}}$, depending on whether $q = r_k$ or $r \in (r_k, r_{k+1})$ respectively. %Similarly we can also see how the intersections map.%; if the result is a family of curves asymptotic to $\gamma_r$ we can deduce that $E_q \mapsto E_r$.
%
 \[f\circ \gamma_{\frac ab}(s, t) = f(t^b, st^a) = \left(t^{2b}, s^{-3}t^{4b-3a} + s^3t^{3a}\right)\]
%To determine $r$ we look at the lowest order terms.
In the case where $4b-3a > 3a$ we get by looking at lowest order terms that $f\circ \gamma_{\frac ab} \sim \gamma_{\frac {3a}{2b}}$. In the case where $4b-3a < 3a$ we get that $f\circ \gamma_{\frac ab} \sim \gamma_{\frac {4b-3a}{2b}}$. Finally in the special case that $4b-3a = 3a$ we get $f\circ \gamma_{\frac ab}(s, t) = \left(t^{2b}, (s^{-3}+ s^3)t^{3a}\right) \sim \gamma_{\frac {3a}{2b}}(s, t)$. 
 %
In short $f \circ \gamma_q \sim \gamma_{T_f(q)}$, where
\[T_f : \frac ab\longmapsto 
\begin{cases}
 \frac {3a}{2b} & \frac ab \le \frac 23\\
 2 - \frac {3a}{2b} & \frac ab > \frac 23
\end{cases}
\]

%Of course it may be the case that $E_{T_f(q)}$ is not a curve in $Y$ and $r_j < T_f(q) < r_{j+1}$, but then from \autoref{prop:degen} we can infer that $E_q \mapsto E_{r_j} \cap E_{r_{j+1}}$. We can even use this technique to understand how the points $E_{r_j} \cap E_{r_{j+1}}$ are mapped by $g$.

\begin{ex}
 We can now see that $f_0 : X_0 \dashto X_0$ is not algebraically stable. $E_0$ is fixed by $f_0$ since $T_f(0) = 0$. However $T_f(1) = \half \in (0, 1) = (r_0, r_1)$, therefore $f_0(E_1) = E_0 \cap E_1 = P$. This point is indeterminate with $f_0(P) = E_1$ because $T_f((0, 1)) = (0, 1] \ni 1$.
\end{ex}

\begin{prop}\label{prop:mapexc}
 Let $f$, $g : Y \dashto Y$ and $\phi$ be as above and the irreducible curves of the fibre $\set{x=0}$ indexed by rational (Farey) parameters \[0 = r_0 < r_1 < \cdots < r_n = 1.\] Let $T_f : \Q \cap [0, 1] \to \Q \cap [0, 1]$ be such that $g \circ \gamma_q = \gamma \sim \gamma_{T_f(q)}$ for every $q \in \Q \cap [0, 1]$.
 
 Then the dynamics over $\set{x=0}$ is determined by the following:
\begin{enumerate}
 \item if $q = r_j$ and $T_f(q) = r_k$ for some $0 \le j, k \le n$, then $g_*(E_q) = E_{T_f(q)}$;
 \item if $q = r_j$ and $r_k < T_f(q) < r_{k+1}$ for some $0 \le j, k \le n$, then $g : E_q \mapsto E_{r_k} \cap E_{r_{k+1}}$;
 \item if $T_f((r_j, r_{j+1})) \subset (r_k, r_{k+1})$, then $g\left(E_{r_j} \cap E_{r_{j+1}}\right) = E_{r_k} \cap E_{r_{k+1}}$; otherwise
 \item if $[r_k, r_l] \subseteq T_f((r_j, r_{j+1}))$ with $k$ minimal and $l$ maximal, then we have \[g\!\left(E_{r_j} \cap E_{r_{j+1}}\right) = E_{r_k} \cup \cdots \cup E_{r_l}.\]
\end{enumerate}
\end{prop}

\subsection{Dynamics of Exceptional Curves}\label{sec:dynexc}

 Suppose for contradiction we have a birational morphism $\pi : (g, Y) \to (f_0, X_0)$ which stabilises $f$. Then by \autoref{thm:minimal} we may assume that $\pi : (g, Y) = (\hat f, \hat X) \to (f_0, X_0)$ is the minimal stabilisation. Consider precisely how $\pi$ blows up $X_0$ with the \minimalalgname{}.

\begin{lem}\label{lem:satellitestab}
 The \minimalalgname{} on $(f_0, X_0)$ only creates curves which are satellite relative to $E_0$ and $E_1$.
\end{lem}

Hence we can assume that $E(\pi) = E_{r_1} \cup \cdots \cup E_{r_{n-1}}$, $r_j \in (0, 1) \cap \Q$. Note that the interval $[0, 1]$ is forward invariant for $T_f$, as is $[\half, 1]$. The rest of the proof will hinge on $T_f$ being topologically mixing on this interval. In the Berkovich theory, this lemma also corresponds to the similar fact that $[\zeta(0, 1), \zeta(0, |x|)]$ is forward invariant.

\begin{proof}
%It is clear from \autoref{prop:initialstability} that $E(f_0) \subset E_1 \cup E_\infty$, and that $E_\infty$ is not a destabilising curve. So t
The only possible destabilising curve for $f_0$ is $E_1$. Any destabilising orbit beginning at $E_1$ must remain in the fibre $\set{x=0}$ which is fixed by $f_0$, and the same applies for every exceptional curve of $\pi$ created by the algorithm.

 At the first step of the algorithm we only have $E_0, E_1$. % which trivially satisfies the conclusion. %would blowup the forward orbit of $E_1$ which is just $E_{\half}$. This is not directly because $T_f(1) = \half$ but $\half \in (\frac 01, \frac 11)$ and the next Farey parameter here is $\frac 12 = \frac {0+1}{1+1}$.
 Proceeding inductively, suppose we have an intermediate surface $\pi' : (g, Y) \to (f_0, X_0)$ generated by the algorithm which is satellite relative to $E_0$ and $E_1$, with $E(\pi') = E_{r_1'} \cup \cdots \cup E_{r_m'}$. 
 On one hand, a destabilising curve must be one of the $E_{r_j'}$, but on the other hand, \autoref{prop:mapexc} says that a minimal destabilising orbit consists of finitely many points of the form $E_{r_j'}\cap E_{r_{j+1}'}$. Blowing up all of these points leads to a further map which is satellite relative to $E_0$ and $E_1$.
 %
 %Suppose in the next step there is a destabilising curve $E = E_q$ and we blowup its image. Since $E_q$ is exceptional for $g$, \autoref{prop:mapexc} says that $T_f(q) \in (r_j', r_{j+1}')$ for some $r_j', r_{j+1}'$. This means that $g(E_q) = E_{r_j'}\cap E_{r_{j+1}'}$ and blowing up this point to $E_r$ is clearly another satellite blowup relative to $E_0$ and $E_1$. The next curve to be blown up would be the image of $E_r$ if necessary, and so on, with each a satellite blowup.
\end{proof}

\begin{proof}[Proof of \autoref{thm:counterex}]

Consider the orbit of $1$ under $T_f$.%; they do not appear preperiodic.
\[1 \mapsto \frac 12 \mapsto \frac 34 \mapsto \frac 78 \mapsto \frac{11}{16} \mapsto \cdots\]
Indeed, suppose that $\frac ab = \frac a{2^n}$ with $a$ odd, then
\[\frac a{2^n} \longmapsto
\begin{cases}
 \frac {3a}{2^{n+1}} & \frac a{2^n} < \frac 23\\
 2 - \frac {3a}{2^{n+1}} = \frac{2^{n+2} - 3a}{2^{n+1}} & \frac ab > \frac 23.
\end{cases}
\]
Thus with every iterate the denominator multiplies by $2$ (since $2^{n+2} -3a$ is also odd). 
%
In particular, the sequence $(T_f^m(1))$ is infinite and there exists a smallest $m = m_1$ such that $T_f^{m-1}(1) = q \in \set{r_0, \dots, r_n}$, but $T_f(q) \in (r_j, r_{j+1})$. Thus by \autoref{prop:mapexc}(i, ii), $\hat f(E_q) = E_{r_j} \cap E_{r_{j+1}} = P$. % $\hat f^{m_1-1}(E_1) = E_q$
 We claim that for some smallest $m = m_2$, $T_f^m((r_j, r_{j+1})) \ni r_k$ for some $0 \le k \le n$. Then by \autoref{prop:mapexc}(iii, iv) we have that for $P, \hat f(P), \dots, \hat f^{m_2-1}(P) = Q$ are closed points with $\hat f(Q) \supset E_k$. Therefore $\hat f : \hat X \dashto \hat X$ still has the minimal destabilising orbit $P, \hat f(P), \dots, \hat f^{m_2-1}(P) = Q$, contradiction.

\emph{Proof of claim}:
%By \autoref{prop:mapexc}(i), $\hat f^m(E_1) = E_{T_f^m(1)}$ so long as $T_f^m(1) \in \set[r_j]{0 \le j \le n}$, but eventually this must end because the set is finite and we saw that $(T_f^m(1))$ is not preperiodic. Whence there is a least $m$ such that $\hat f^m(E_1) = E_q$ and by \autoref{prop:mapexc}(ii) we have $\hat f(E_q) = E_{r_k} \cap E_{r_{k+1}} = P$ for some $0 \le k \le n$. $E_q$ shall be our destabilising curve; note that $q \in [\half, 1]$.
%
%Next consider the orbit of $P$. If it is ever indeterminate then we have confirmed a destabilising orbit in $X$. Let $m \in \N$ and $Q = \hat f^{m}(P) = E_{r_j} \cap E_{r_{j+1}}$. By \autoref{prop:mapexc}(iii), if $T_f((r_j, r_{j+1})) \ni r_k$ for some $0 \le k \le n$ then $g(Q)$ is a curve. Conversely if $\hat f(Q)$ is a point, then (iv) in \autoref{prop:mapexc} must apply, so $\hat f(Q) = E_{r_k} \cap E_{r_{k+1}}$ where $T_f((r_j, r_{j+1})) \subset (r_k, r_{k+1})$. We can see inductively that either for some $m$, $Q = \hat f^{m}(P)$ is indeterminate (i.e.\ $\hat f$ has a destabilising orbit), or we have that $T_f^m((r_j, r_{j+1}))$ contains none of the $r_k$ for any $m \in \N$.
%
Suppose not, then for every $m, k$, $r_k \nin T_f^m((r_j, r_{j+1}))$. Since $T_f(\frac 23) = 1 = r_n$ it follows that $\forall m, \frac 23 \nin T_f^m((r_j, r_{j+1}))$. %We show how intervals mapped by $T_f$ eventually hit one of the $r_j$, namely $1 = r_n$. Let $(p, q) \subset [\half, 1]$ be an open interval.
Because $T_f$ is continuous, $T_f^m((r_j, r_{j+1}))$ is an interval $(p, q) \subset [0, 1] \sm \set {0, \frac 23, 1}$. Suppose $(p, q) \subseteq (0, \frac 23)$, then by the formula for $T_f$ we have $T_f((p, q)) = (\frac{3p}2, \frac{3q}2)$, i.e. the interval expanded by a factor of $\frac 32$. Similar goes for when $(p, q) \subseteq (\frac 23, 1)$. 
%Otherwise the interval $(p, q)$ contains $\frac 23$, thus its image contains $1$.
We find that the length of $T_f^m((r_j, r_{j+1}))$ is $(r_{j+1} - r_j)\left(\frac 32\right)^m$, but on the other hand $T_f^m((r_j, r_{j+1})) \subseteq T_f^m([0, 1]) = [0, 1]$. Clearly this is impossible.%, at least until its image includes $1$, which cannot continue forever since $T_f^m((p, q)) \subseteq [\half, 1]\ \forall m \in \N$.
%Aside: in fact it is not hard to see that there is an $N \in \N$ such that $T_f^m((p, q)) = [\half, 1]$ for all $m \ge N$. %Since we have the working assumption that $r_j \nin T_f^m((p, q))$ for any $m \in \N$ this is a contradiction.
%Therefore we have an $E_q$ such that $\hat f(E_q) = P$ and for some $m$, $Q = \hat f^{m}(P) \in I(\hat f)$, thus $\hat f$ is not algebraically stable, contradicting our key assumption.
%
\end{proof}

%%

\subsection{The Berkovich Alternative}\label{sec:berkalt}
Here we provide some details about another approach to the bookkeeping using the Berkovich projective line, $\P^1_\text{an}(K)$. We work over the field, $K$, of Puiseux series on the variable $x$ (the same $x$ as above) with $\C$ coefficients. There is a straightforward correspondence between Type II points of the form $\zeta(0, |x|^q)$ with $q \in \Q\cap[0, 1]$ and exceptional curves $E_q$ which can be obtained from satellite blowups between $E_0$ and $E_1$. The curves $E_{r_j}$ ($0 \le j \le n$) correspond to finitely many marked Type II points in $[\zeta(0, 1), \zeta(0, |x|)]$, and the intersection point $E_{r_j}\cap E_{r_{j+1}}$ corresponds to the Berkovich annulus bounded by the $\zeta(0, |x|^{r_j})$ and $\zeta(0, |x|^{r_{j+1}})$.

We see that $f$ induces a map on $\P^1_\text{an}$. This maps $\zeta(0, r)$ to $\zeta(0, R)$, where $R$ is the radius given by the magnitude of the Laurent series $x^4y^{-3} + y^3$ at $|y| = r$. The Weierstrass degree is $-3$ when $R = |x|^4r^{-3} = |x^4y^{-3}| > |y^3| = r^3$ and $3$ when $|x|^4r^{-3} = |x^4y^{-3}| < |y^3| = r^3 = R$. This means that
 \[\zeta(0, r) \longmapsto 
\begin{cases}
 \zeta(0, r^3) & r > |x|^{\frac 23}\\
 \zeta(0, |x|^4r^{-3}) & r < |x|^{\frac 23}
\end{cases}
\]
The effect of $x \mapsto x^2$ is to replace each diameter with its square root. The map $T_f$ constructed in \autoref{sec:mapexc} describes the dynamics on $(0, \infty) \subset \P^1_\text{an}$ with each $T_f(q) = r$ corresponding to $f(\zeta(0, |x|^q)) = \zeta(0, |x|^r)$.

\[T_f : q \longmapsto 
\begin{cases}
 \frac {3}{2}q & q \le \frac 23\\
 2 - \frac {3}{2}q & q > \frac 23
\end{cases}
\]

\begin{ex}
 Consider the map \[f : (x, y) \longmapsto \left((1-x)x^2, (1-x)(x^4y^{-3} + y^3)\right)\] as defined on $\P^1 \times \P^1$. Then $f$ is not algebraically stable after \emph{any} birational conjugation.
 
 Unfortunately to elaborate on this example would use the full power of new machinery developed for dynamics on Berkovich space, which will appear in the author's thesis.
\end{ex}

\bibliographystyle{alpha}
\bibliography{Stability}

\end{document} 
