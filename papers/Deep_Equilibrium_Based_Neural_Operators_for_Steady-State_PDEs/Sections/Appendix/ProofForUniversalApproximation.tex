\section{Proof of Universal Approximation}
\label{sec:proof_of_universal_approximation}

The proof of the universal approximation essentially follows from the result on the universal approximation 
capabilities of FNO layers in~\citet{kovachki2021universal}, applied to $\calG(v,f) = v - (Lv - f)$. For the sake of completeness, we reitarate the key steps.  

For simplicity, we will assume that 
$d_u = d_v = d_f = 1$. (The results straightforwardly generalize.)
We will first establish some key technical lemmas and introduce some notation and definitions useful 
for the proof for Theorem~\ref{thm:main_theorem}.

\begin{definition}
    An operator $T: L^2(\Omega; \R) \to L^2(\Omega; \R)$
    is continuous 
    at $u \in L^2(\Omega; \R)$ if for every 
    $\epsilon > 0$, there exists a $\delta > 0$,
    such that for all $v \in \ll$ with $\|u - v\|_{\ll}\leq\delta$,
    we have 
    $\|L(u) - L(v)\|_{\ll} \leq \epsilon$.
\end{definition}









First, we approximate the infinite dimensional operator $\calG: \ll \times \ll \to \ll$
by projecting the functions in $\ll$ to a finite-dimensional approximation $\lln$, and considering the action of the operator on this subspace.  
The linear projection we use is the one introduced in~\eqref{eq:projection_definition}.
More precisely we show the following result, 

\begin{lemma}
    \label{lemma:discrete_close}
    Given a continuous operator $L: \ll \to \ll$ as defined in~\eqref{eq:lu_f}, let us define an operator $\calG: \ll \times \ll \to \ll$
 as $\calG(v, f):= v - (L(v) - f)$.
    Then, for every $\epsilon > 0$ there exists an $N \in \N$
    such that for all $v, f$ in any compact set $K \subset \ll$, 
    the operator $\calG_N = \project \calG(\project v, \project f)$ 
    is an $\epsilon$-approximation of $\calG(v, f)$, i.e., we have,
    $$\sup_{v,f \in K} \|\calG(v, f) - \calG_N(v, f)\|_{\ll} \leq \epsilon.$$
\end{lemma}
\begin{proof}
    Note that for an $\epsilon > 0$
    there exists an $N = N(\epsilon, d)$ 
    such that for all $v \in K$ 
    we have 
    $$\sup_{v \in K} \|v - \project v\|_{\ll} \leq \epsilon.$$
    Therefore, using the definition of $\calG_N$ we can bound the $\ll$ norm of the difference between 
    $\calG$ and $\calG_N$ as follows,
     \begin{align*}
         &\|\calG(v, f) - \project \calG(v_n, f_n)\|_{\ll} \\
         &\leq 
            \|\calG(v, f) - \project \calG(v, f)\|_{\ll} 
            + \|\project \calG(v, f) - \project \calG(\project v, \project f)\|_{\ll} \\
        &\leq 
            \underbrace{\|\calG(v, f) - \project \calG(v, f)\|_{\ll}}_{I}
            + \underbrace{\| \calG(v, f) - \calG(\project v, \project f)\|_{\ll}}_{II}
     \end{align*}
     We first bound the term $I$ as follows:
     \begin{align*}
        &\|\calG(v, f) - \project \calG(v, f)\|_{\ll} \\
            &= \left\|v - (L(v) - f) - \project\left(v - (L(v) - f)\right)\right\|_{\ll}\\
            &= \|v - \project v \|_{\ll} + \|f - \project f\|_{\ll} + \|L(v) - \project L(v)\|_{\ll} \\
            &= \epsilon + \epsilon + \|L(v) - \project L(v)\|_{\ll} \numberthis \label{eq:lemma_1_eq1}
     \end{align*}
     Since $L$ is continuous, 
     for all compact sets $K \subset \ll$, $L(K)$ is compact as well. This is because: (1) for any $u \in K$, $\|L(u)\|_{\ll}$ is finite; (2) for any $v \in K$, $\|L(v)\|_{\ll} \leq \|L(u)\|_{\ll} + C\|u-v\|_{\ll}$.  
     Therefore, 
     for every $\epsilon > 0$, there exists an $N\in\N$ such that 
     \begin{align*}
         \sup_{v \in K} \|L(v) -  \project L(v)\|_{\ll} \leq \epsilon.
     \end{align*}
     Substituting the above result in~\eqref{eq:lemma_1_eq1}, we have
     \begin{equation}
         \label{eq:lemma1_term1_upper_bound}
         \|\calG(v, f) - \project \calG(v, f)\|_{\ll} \leq 3\epsilon.
     \end{equation}

     Similarly, for all $v \in K$ where $K$ is compact, 
     we can bound Term $II$ as following,
     \begin{align*}
         &\left \|\calG(v, f) - \calG(\project v, \project f)\right\|_{\ll} \\
         &\leq \left\|v - (L(v) - f) - \project v - (L(\project v) - \project f)\right\|_{\ll} \\
         &\leq \|v - \project v\|_{\ll} + \|f - \project f\|_{\ll} + \|L(v) - L(\project v)\|_{\ll} \\
         &\leq \epsilon + \epsilon + \|L(v) - L(\project v)\|_{\ll}. \numberthis \label{eq:lemma_2_eq2}
     \end{align*}
    Now, since $v \in K$ and $L:\ll \to \ll$ is a continuous operator, 
    there exists a modulus of continuity (an increasing real valued function)
    $\alpha \in [0, \infty)$, such that
    for all $v \in K$, we have
    \begin{align*}
        \|L(v) - L(\project v)\|_{\ll} \leq \alpha\left(\|v - \project v\|_{\ll}\right)
    \end{align*}
    Hence for every $\epsilon > 0$ 
    there exists %
    an $N \in \N$ such that, 
    $$\alpha(\|v - \project v\|_{\ll}) \leq \epsilon.$$
    Plugging these bounds in~\eqref{eq:lemma_2_eq2}, we get, 
    \begin{equation}
         \label{eq:lemma1_term2_upper_bound}
         \left \|\calG(v, f) - \calG(\project v, \project f)\right\|_{\ll} 
         \leq 3\epsilon.
    \end{equation}
    Therefore, combining~\eqref{eq:lemma1_term1_upper_bound} and~\eqref{eq:lemma1_term2_upper_bound}
    then for
    $\epsilon > 0$, there exists an 
    $N \in \N$, such that for all $v,f \in K$ 
    we have
    \begin{align}
        \sup_{v,f\in K}\left\|\calG(v, f) - \project \calG(v_n, f_n)\right\|_{\ll} \leq 6\epsilon.
    \end{align}
    Taking $\epsilon' = 6\epsilon$ proves the claim.
\end{proof}


\begin{proof}[Proof of Theorem~\ref{thm:main_theorem}]
    For Lemma~\ref{lemma:discrete_close}
    we know that there exists a finite dimensional 
    projection for the operator $\gG$, defined as $\gG_N(v, f)$ such that 
    for all $v, f \in \ll$ we have
    $$\|\gG(v, f) - \gG_N(v, f)\|_{\ll} \leq \epsilon.$$

    Now using the definition of $\gG_N(v, f)$ we have
    \begin{align*}
        \gG_N(v, f) &= \project\gG(\project v, \project f)\\
        &= \project v - \left(\project L(\project v) - \project f\right)
    \end{align*}
    From~\citet{kovachki2021universal}, Theorem 2.4 we know 
    that there exists an FNO network $G_{\theta^L}$ 
    of the form defined in 
    ~\eqref{eq:fno_layer_def} such that 
    for all $v \in K$, where $K$ is a compact set, 
    there exists an $\epsilon^L$
    we have
    \begin{equation}
    \sup_{v \in K} \|\project L(\project v) - G_{\theta^L}\|_{\ll} \leq 
        \epsilon^L
    \end{equation}
    Finally, 
    note that from Lemma~D.1 in~\cite{kovachki2021universal},
    we have
    that for any $v \in K$, 
    there exists an FNO layers
    $G_{\theta^f} \in \ll$ and $G_{\theta^v} \in \ll$ defined 
    in~\eqref{eq:fno_layer}
    such that
    \begin{equation}
    \sup_{v \in K} \|\project v - G_{\theta^v}\|_{\ll} \leq 
        \epsilon^v
    \end{equation}
    and 
    \begin{equation}
    \sup_{f \in K} \|\project f - G_{\theta^f}\|_{\ll} \leq 
        \epsilon^f
    \end{equation}
    for $\epsilon^v > 0$ and $\epsilon^f > 0$.


    Therefore there exists an $\tilde{\epsilon} > $ 
    such that there is an FNO network
    $G_\theta: \ll \times \ll \to \ll$
    where $\theta := \{\theta^L, \theta^v, \theta^f\}$
    such that
    \begin{equation}
        \label{eq:neural_network_existence}
        \sup_{v \in K, f \in \ll} 
            \|\gG_N(v, f) - G_\theta(v, f)\|_{\ll} \leq 
            \tilde{\epsilon}
    \end{equation}


    Now, since we know that $u^\star$ is the fixed point 
    of the operator $\gG$ we have from Lemma~\ref{lemma:discrete_close} and \eqref{eq:neural_network_existence},
    \begin{align*}
        \|\gG(u^\star, f) - G_\theta(u^\star, f)\|_{\ll} 
        &\leq \|u^\star - \gG_N(u^\star, f)\|_{\ll} 
            + \|\gG_N(u^\star, f) -  G_\theta(u^\star, f)\|_{\ll}\\
        &\leq \tilde{\epsilon} + \epsilon.
    \end{align*}
\end{proof}

\section{Fast Convergence for Newton Method}
\label{sec:fast_convergence}

\begin{definition}[Frechet Derivative in $\ll$]
    For a continuous operator $F: \ll \to \ll$, 
    the Frechet derivative at $u \in \ll$
    is a linear operator $F'(u): \ll \to \ll$ such that for all $v\in \ll$ we have
    \begin{align*}
        \lim_{\|v\|_{\ll} \to 0} \frac{\|F(u + v) - F(u) - F'(u)(v)\|_{\ll}}{\|v\|_{\ll}} = 0.
    \end{align*}
\end{definition}

\begin{lemma}
    \label{lemma:frechet_upperbound}
    Given the operator $L: \ll \to \ll$
    with Frechet derivative $L'$,
    such that for all $u, v \in \ll$, we have
    $\|L'(u)(v)\|_{\ll} \geq \lambda \|v\|_{\ll}$, then 
    $L'(u)^{-1}$ exists and we have,
    for all $v_1, v_2 \in \ll$:
    \begin{enumerate}
        \item $\|L'(u)^{-1}(v_1)\|_{\ll} \leq \frac{1}{\lambda}\|v_1\|_{\ll}$.
        \item $\|v_1 - v_2\|_{\ll} 
            \leq \frac{1}{\lambda}\|L(v_1) - L(v_2)\|_{\ll}$
    \end{enumerate}
\end{lemma}
\begin{proof}
    Note that for all $u, v' \in \ll$ we have,
    \begin{align*}
        \|L'(u)v'\|_{\ll} \geq \lambda \|v'\|_{\ll}
    \end{align*}
    Taking $v = L'(u)^{-1}(v')$, we have
    \begin{align*}
        \|L'(u)\left(L'(u)^{-1}(v)\right)\|_{\ll} &\geq \lambda \|L^{-1}(u)(v)\|_{\ll}\\
        \implies
        \frac{1}{\lambda}\|v\|_{\ll} &\geq  \|L^{-1}(u)(v)\|_{\ll}.
    \end{align*}

    For part $2$, note that 
    there exists a $c \in [0, 1]$ 
    such that
    \begin{align*}
        \|L(v_1) - L(v_2)\|_{\ll} \geq \inf_{c \in [0, 1]}\|L'(c v_1  + (1 - c)v_2)\|_{2}\|v_1 - v_2\|_{\ll}
        \geq \lambda \|v_1 - v_2\|_{\ll}.
    \end{align*}
\end{proof}

We now show the proof for Lemma~\ref{lemma:fast_convergence}.
The proof is standard and can be found in~\cite{farago2002numerical}, 
however we include the 
complete proof here for the sake of completeness.


We restate the Lemma here for the convenience of the reader.
\begin{lemma}[\cite{farago2002numerical}, Chapter 5]
    \label{lemma:fast_convergence}
   Consider the PDE defined Definition~\ref{def:steady_state_PDE},
   such that $d_u=d_v=d_f=1$.
   such that $L'(u)$ defines the Frechet derivative 
   of the operator $L$.
   If for all $u,v \in L^2(\Omega; \R)$ we have
   $\| L'(u) v\|_{\ll} \geq \lambda \|v\|_{\ll}$
   \footnote{We note that this condition is different from the 
   condition on the inner-product in the submitted version 
   of the paper, which had. 
   $\langle L'(u), v\rangle_{\ll} \geq \lambda \|v\|_{\ll}$.
   }
   and 
   $\|L'(u) - L'(v)\|_{\ll} \leq \Lambda \|u - v\|_{\ll}$
   for $0 < \lambda \leq \Lambda <\infty $,
   then for the Newton update,
   $
       u_{t+1} \leftarrow u_t - L'(u_t)^{-1}\left(L(u_t) - f\right),
   $
   with $u_0 \in L^2(\Omega; \R)$, there exists an $\epsilon > 0$,
   such that  $\|u_T - u^\star\|_{\ll} \leq \epsilon$
   if 
    \footnote{We note that this rate is different from the one in 
    the submitted version of the paper. }
    $
       T \geq \log 
       \left(
           \log \left(\frac{1}{\epsilon}\right) 
           /
           \log \left(\frac{2\lambda^2}{\Lambda\|L(u_0) - f\|_{\ll}}\right)
        \right).
    $
\end{lemma}


\begin{proof}[Proof of Lemma~\ref{lemma:fast_convergence}]
    Re-writing the updates in Lemma~\ref{lemma:fast_convergence} as,
    \begin{align}
        u_{t+1} &= u_t + p_t \\
        L'(u_t) p_t &= -(L(u_t) - f) \numberthis \label{eq:Lprime}
    \end{align}
    Now, upper bounding $L(u_{t+1}) - f$ for all $x \in \Omega$
    we have,
    \begin{align*}
        & L(u_{t+1}(x)) - f(x)  \\
            &= L(u_t(x)) - f(x)
            + \int_{0}^1\left(L'(u_t(x) + t(u_{t+1}(x) - u_t(x)))\right)(u_{t+1}(x) - u_t(x))dt\\
            &= L(u_t(x)) - f(x)  + L'(u_t(x))p_t(x)
            + \int_{0}^1\left(L'(u_t(x) + t(u_{t+1}(x) - u_t(x))) - L'(u_t(x))\right)p_t(x)dt\\
            &= 
            \int_{0}^1\left(L'(u_t(x) + t(u_{t+1}(x) - u_t(x))) - L'(u_t(x))\right)p_t(x)dt
    \end{align*}    
    where we use ~\eqref{eq:Lprime} in the final step.
    
    Taking $\ll$ norm on both sides and using the fact that
    $\|L'(u) - L'(v)\|_{\ll} \leq \Lambda \|u - v\|_{\ll}$, 
    we have 
    \begin{align*}
        \|L(u_{t+1}) - f\|_{\ll}
        \leq \int_0^1
        \Lambda t \|u_{t+1} - u_t\|_{\ll}\|p_t\|_{\ll} dt
    \end{align*}
    
    Noting that for all $x \in \Omega$, we have
    $u_{t+1} - u_t = p_t$, 
    and using the fact that for all $u,v$ 
    $\|L'(u)^{-1}v\|_{\ll}\leq\frac{1}{\lambda}\|v\|_{\ll}$
    we have,
    $\|L'(u_t)p_t\|_{\ll} \leq \frac{1}{\lambda} \|p_t\|_{\ll}$
    \begin{align*}
         \|L(u_{t+1}) - f\|_{\ll}
         &\leq
            \int_{0}^1 \Lambda t\|u_{t+1} - u\|_{\ll}\|p_t\|_{\ll} dt\\
        &\leq \Lambda/2 \|p_t\|_{\ll}^2\\
        &\leq \Lambda/2 \|-L'(u_t)^{-1} ( L(u_t) - f)\|_{\ll}^2\\
        &\leq \frac{\Lambda}{2\lambda^2}\|L(u_t) - f)\|_{\ll}^2
    \end{align*}    
    where we use the result from Lemma~\ref{lemma:frechet_upperbound}
    in the last step.

    Therefore we have
    \begin{align*}
         \|L(u_{t+1}) - f\|_{\ll}
         &\leq
         \left(\frac{\Lambda}{2\lambda^2}\right)^{2^t - 1}
         \left(L(u_0) - f\right)^{2^t}\\
         \implies
         \|L(u_{t+1}) - f\|_{\ll}
         &\leq
         \left(\frac{\Lambda}{2\lambda^2}\right)^{2^t - 1}
         \left(L(u_0) - L(u^\star)\right)^{2^t}\\
         \implies
         \|u_{t+1} - u^\star\|_{\ll}
         &\leq
         \frac{1}{\lambda}\left(\frac{\Lambda}{2\lambda^2}\right)^{2^t - 1}
         \left\|L(u_0) - L(u^\star)\right\|_{\ll}^{2^t}.
    \end{align*}
    Therefore, if 
    $$\frac{\Lambda}{2\lambda^2}\|L(u_0) - L(u^\star)\|_{\ll}\leq 1,$$
    then we have 
    \begin{align*}
         \|u_{t+1} - u^\star\|_{\ll}
         \leq \epsilon,
    \end{align*}
    for
   \begin{align*}
       T \geq \log 
       \left(
           \log \left(\frac{1}{\epsilon}\right) 
           /
           \log \left(\frac{2\lambda^2}{\Lambda\|L(u_0) - f\|_{\ll}}\right)
        \right).
   \end{align*}
\end{proof}

