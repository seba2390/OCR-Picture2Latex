%%%%%%%% mlsys 2022 EXAMPLE LATEX SUBMISSION FILE %%%%%%%%%%%%%%%%%
\documentclass{article}

% Recommended, but optional, packages for figures and better typesetting:
\usepackage{microtype}
\usepackage{graphicx}
\usepackage{subfigure}
\usepackage{booktabs} % for professional tables

% hyperref makes hyperlinks in the resulting PDF.
% If your build breaks (sometimes temporarily if a hyperlink spans a page)
% please comment out the following usepackage line and replace
% \usepackage{mlsys2023} with \usepackage[nohyperref]{mlsys2023} above.
\usepackage{hyperref}

% Attempt to make hyperref and algorithmic work together better:
%\newcommand{\theHalgorithm}{\arabic{algorithm}}
%%%%% NEW MATH DEFINITIONS %%%%%

\usepackage{amsmath,amsfonts,bm}

% Mark sections of captions for referring to divisions of figures
\newcommand{\figleft}{{\em (Left)}}
\newcommand{\figcenter}{{\em (Center)}}
\newcommand{\figright}{{\em (Right)}}
\newcommand{\figtop}{{\em (Top)}}
\newcommand{\figbottom}{{\em (Bottom)}}
\newcommand{\captiona}{{\em (a)}}
\newcommand{\captionb}{{\em (b)}}
\newcommand{\captionc}{{\em (c)}}
\newcommand{\captiond}{{\em (d)}}

% Highlight a newly defined term
\newcommand{\newterm}[1]{{\bf #1}}


% Figure reference, lower-case.
\def\figref#1{figure~\ref{#1}}
% Figure reference, capital. For start of sentence
\def\Figref#1{Figure~\ref{#1}}
\def\twofigref#1#2{figures \ref{#1} and \ref{#2}}
\def\quadfigref#1#2#3#4{figures \ref{#1}, \ref{#2}, \ref{#3} and \ref{#4}}
% Section reference, lower-case.
\def\secref#1{section~\ref{#1}}
% Section reference, capital.
\def\Secref#1{Section~\ref{#1}}
% Reference to two sections.
\def\twosecrefs#1#2{sections \ref{#1} and \ref{#2}}
% Reference to three sections.
\def\secrefs#1#2#3{sections \ref{#1}, \ref{#2} and \ref{#3}}
% Reference to an equation, lower-case.
% \def\eqref#1{equation~\ref{#1}}
 \def\eqref#1{(\ref{#1})}
% Reference to an equation, upper case
\def\Eqref#1{Equation~\ref{#1}}
% A raw reference to an equation---avoid using if possible
\def\plaineqref#1{\ref{#1}}
% Reference to a chapter, lower-case.
\def\chapref#1{chapter~\ref{#1}}
% Reference to an equation, upper case.
\def\Chapref#1{Chapter~\ref{#1}}
% Reference to a range of chapters
\def\rangechapref#1#2{chapters\ref{#1}--\ref{#2}}
% Reference to an algorithm, lower-case.
\def\algref#1{algorithm~\ref{#1}}
% Reference to an algorithm, upper case.
\def\Algref#1{Algorithm~\ref{#1}}
\def\twoalgref#1#2{algorithms \ref{#1} and \ref{#2}}
\def\Twoalgref#1#2{Algorithms \ref{#1} and \ref{#2}}
% Reference to a part, lower case
\def\partref#1{part~\ref{#1}}
% Reference to a part, upper case
\def\Partref#1{Part~\ref{#1}}
\def\twopartref#1#2{parts \ref{#1} and \ref{#2}}

\def\ceil#1{\lceil #1 \rceil}
\def\floor#1{\lfloor #1 \rfloor}
\def\1{\bm{1}}
\newcommand{\train}{\mathcal{D}}
\newcommand{\valid}{\mathcal{D_{\mathrm{valid}}}}
\newcommand{\test}{\mathcal{D_{\mathrm{test}}}}

\def\eps{{\epsilon}}


% Random variables
\def\reta{{\textnormal{$\eta$}}}
\def\ra{{\textnormal{a}}}
\def\rb{{\textnormal{b}}}
\def\rc{{\textnormal{c}}}
\def\rd{{\textnormal{d}}}
\def\re{{\textnormal{e}}}
\def\rf{{\textnormal{f}}}
\def\rg{{\textnormal{g}}}
\def\rh{{\textnormal{h}}}
\def\ri{{\textnormal{i}}}
\def\rj{{\textnormal{j}}}
\def\rk{{\textnormal{k}}}
\def\rl{{\textnormal{l}}}
% rm is already a command, just don't name any random variables m
\def\rn{{\textnormal{n}}}
\def\ro{{\textnormal{o}}}
\def\rp{{\textnormal{p}}}
\def\rq{{\textnormal{q}}}
\def\rr{{\textnormal{r}}}
\def\rs{{\textnormal{s}}}
\def\rt{{\textnormal{t}}}
\def\ru{{\textnormal{u}}}
\def\rv{{\textnormal{v}}}
\def\rw{{\textnormal{w}}}
\def\rx{{\textnormal{x}}}
\def\ry{{\textnormal{y}}}
\def\rz{{\textnormal{z}}}

% Random vectors
\def\rvepsilon{{\mathbf{\epsilon}}}
\def\rvtheta{{\mathbf{\theta}}}
\def\rva{{\mathbf{a}}}
\def\rvb{{\mathbf{b}}}
\def\rvc{{\mathbf{c}}}
\def\rvd{{\mathbf{d}}}
\def\rve{{\mathbf{e}}}
\def\rvf{{\mathbf{f}}}
\def\rvg{{\mathbf{g}}}
\def\rvh{{\mathbf{h}}}
\def\rvu{{\mathbf{i}}}
\def\rvj{{\mathbf{j}}}
\def\rvk{{\mathbf{k}}}
\def\rvl{{\mathbf{l}}}
\def\rvm{{\mathbf{m}}}
\def\rvn{{\mathbf{n}}}
\def\rvo{{\mathbf{o}}}
\def\rvp{{\mathbf{p}}}
\def\rvq{{\mathbf{q}}}
\def\rvr{{\mathbf{r}}}
\def\rvs{{\mathbf{s}}}
\def\rvt{{\mathbf{t}}}
\def\rvu{{\mathbf{u}}}
\def\rvv{{\mathbf{v}}}
\def\rvw{{\mathbf{w}}}
\def\rvx{{\mathbf{x}}}
\def\rvy{{\mathbf{y}}}
\def\rvz{{\mathbf{z}}}

% Elements of random vectors
\def\erva{{\textnormal{a}}}
\def\ervb{{\textnormal{b}}}
\def\ervc{{\textnormal{c}}}
\def\ervd{{\textnormal{d}}}
\def\erve{{\textnormal{e}}}
\def\ervf{{\textnormal{f}}}
\def\ervg{{\textnormal{g}}}
\def\ervh{{\textnormal{h}}}
\def\ervi{{\textnormal{i}}}
\def\ervj{{\textnormal{j}}}
\def\ervk{{\textnormal{k}}}
\def\ervl{{\textnormal{l}}}
\def\ervm{{\textnormal{m}}}
\def\ervn{{\textnormal{n}}}
\def\ervo{{\textnormal{o}}}
\def\ervp{{\textnormal{p}}}
\def\ervq{{\textnormal{q}}}
\def\ervr{{\textnormal{r}}}
\def\ervs{{\textnormal{s}}}
\def\ervt{{\textnormal{t}}}
\def\ervu{{\textnormal{u}}}
\def\ervv{{\textnormal{v}}}
\def\ervw{{\textnormal{w}}}
\def\ervx{{\textnormal{x}}}
\def\ervy{{\textnormal{y}}}
\def\ervz{{\textnormal{z}}}

% Random matrices
\def\rmA{{\mathbf{A}}}
\def\rmB{{\mathbf{B}}}
\def\rmC{{\mathbf{C}}}
\def\rmD{{\mathbf{D}}}
\def\rmE{{\mathbf{E}}}
\def\rmF{{\mathbf{F}}}
\def\rmG{{\mathbf{G}}}
\def\rmH{{\mathbf{H}}}
\def\rmI{{\mathbf{I}}}
\def\rmJ{{\mathbf{J}}}
\def\rmK{{\mathbf{K}}}
\def\rmL{{\mathbf{L}}}
\def\rmM{{\mathbf{M}}}
\def\rmN{{\mathbf{N}}}
\def\rmO{{\mathbf{O}}}
\def\rmP{{\mathbf{P}}}
\def\rmQ{{\mathbf{Q}}}
\def\rmR{{\mathbf{R}}}
\def\rmS{{\mathbf{S}}}
\def\rmT{{\mathbf{T}}}
\def\rmU{{\mathbf{U}}}
\def\rmV{{\mathbf{V}}}
\def\rmW{{\mathbf{W}}}
\def\rmX{{\mathbf{X}}}
\def\rmY{{\mathbf{Y}}}
\def\rmZ{{\mathbf{Z}}}

% Elements of random matrices
\def\ermA{{\textnormal{A}}}
\def\ermB{{\textnormal{B}}}
\def\ermC{{\textnormal{C}}}
\def\ermD{{\textnormal{D}}}
\def\ermE{{\textnormal{E}}}
\def\ermF{{\textnormal{F}}}
\def\ermG{{\textnormal{G}}}
\def\ermH{{\textnormal{H}}}
\def\ermI{{\textnormal{I}}}
\def\ermJ{{\textnormal{J}}}
\def\ermK{{\textnormal{K}}}
\def\ermL{{\textnormal{L}}}
\def\ermM{{\textnormal{M}}}
\def\ermN{{\textnormal{N}}}
\def\ermO{{\textnormal{O}}}
\def\ermP{{\textnormal{P}}}
\def\ermQ{{\textnormal{Q}}}
\def\ermR{{\textnormal{R}}}
\def\ermS{{\textnormal{S}}}
\def\ermT{{\textnormal{T}}}
\def\ermU{{\textnormal{U}}}
\def\ermV{{\textnormal{V}}}
\def\ermW{{\textnormal{W}}}
\def\ermX{{\textnormal{X}}}
\def\ermY{{\textnormal{Y}}}
\def\ermZ{{\textnormal{Z}}}

% Vectors
\def\vzero{{\bm{0}}}
\def\vone{{\bm{1}}}
\def\vmu{{\bm{\mu}}}
\def\vtheta{{\bm{\theta}}}
\def\va{{\bm{a}}}
\def\vb{{\bm{b}}}
\def\vc{{\bm{c}}}
\def\vd{{\bm{d}}}
\def\ve{{\bm{e}}}
\def\vf{{\bm{f}}}
\def\vg{{\bm{g}}}
\def\vh{{\bm{h}}}
\def\vi{{\bm{i}}}
\def\vj{{\bm{j}}}
\def\vk{{\bm{k}}}
\def\vl{{\bm{l}}}
\def\vm{{\bm{m}}}
\def\vn{{\bm{n}}}
\def\vo{{\bm{o}}}
\def\vp{{\bm{p}}}
\def\vq{{\bm{q}}}
\def\vr{{\bm{r}}}
\def\vs{{\bm{s}}}
\def\vt{{\bm{t}}}
\def\vu{{\bm{u}}}
\def\vv{{\bm{v}}}
\def\vw{{\bm{w}}}
\def\vx{{\bm{x}}}
\def\vy{{\bm{y}}}
\def\vz{{\bm{z}}}

% Elements of vectors
\def\evalpha{{\alpha}}
\def\evbeta{{\beta}}
\def\evepsilon{{\epsilon}}
\def\evlambda{{\lambda}}
\def\evomega{{\omega}}
\def\evmu{{\mu}}
\def\evpsi{{\psi}}
\def\evsigma{{\sigma}}
\def\evtheta{{\theta}}
\def\eva{{a}}
\def\evb{{b}}
\def\evc{{c}}
\def\evd{{d}}
\def\eve{{e}}
\def\evf{{f}}
\def\evg{{g}}
\def\evh{{h}}
\def\evi{{i}}
\def\evj{{j}}
\def\evk{{k}}
\def\evl{{l}}
\def\evm{{m}}
\def\evn{{n}}
\def\evo{{o}}
\def\evp{{p}}
\def\evq{{q}}
\def\evr{{r}}
\def\evs{{s}}
\def\evt{{t}}
\def\evu{{u}}
\def\evv{{v}}
\def\evw{{w}}
\def\evx{{x}}
\def\evy{{y}}
\def\evz{{z}}

% Matrix
\def\mA{{\bm{A}}}
\def\mB{{\bm{B}}}
\def\mC{{\bm{C}}}
\def\mD{{\bm{D}}}
\def\mE{{\bm{E}}}
\def\mF{{\bm{F}}}
\def\mG{{\bm{G}}}
\def\mH{{\bm{H}}}
\def\mI{{\bm{I}}}
\def\mJ{{\bm{J}}}
\def\mK{{\bm{K}}}
\def\mL{{\bm{L}}}
\def\mM{{\bm{M}}}
\def\mN{{\bm{N}}}
\def\mO{{\bm{O}}}
\def\mP{{\bm{P}}}
\def\mQ{{\bm{Q}}}
\def\mR{{\bm{R}}}
\def\mS{{\bm{S}}}
\def\mT{{\bm{T}}}
\def\mU{{\bm{U}}}
\def\mV{{\bm{V}}}
\def\mW{{\bm{W}}}
\def\mX{{\bm{X}}}
\def\mY{{\bm{Y}}}
\def\mZ{{\bm{Z}}}
\def\mBeta{{\bm{\beta}}}
\def\mPhi{{\bm{\Phi}}}
\def\mLambda{{\bm{\Lambda}}}
\def\mSigma{{\bm{\Sigma}}}

% Tensor
\DeclareMathAlphabet{\mathsfit}{\encodingdefault}{\sfdefault}{m}{sl}
\SetMathAlphabet{\mathsfit}{bold}{\encodingdefault}{\sfdefault}{bx}{n}
\newcommand{\tens}[1]{\bm{\mathsfit{#1}}}
\def\tA{{\tens{A}}}
\def\tB{{\tens{B}}}
\def\tC{{\tens{C}}}
\def\tD{{\tens{D}}}
\def\tE{{\tens{E}}}
\def\tF{{\tens{F}}}
\def\tG{{\tens{G}}}
\def\tH{{\tens{H}}}
\def\tI{{\tens{I}}}
\def\tJ{{\tens{J}}}
\def\tK{{\tens{K}}}
\def\tL{{\tens{L}}}
\def\tM{{\tens{M}}}
\def\tN{{\tens{N}}}
\def\tO{{\tens{O}}}
\def\tP{{\tens{P}}}
\def\tQ{{\tens{Q}}}
\def\tR{{\tens{R}}}
\def\tS{{\tens{S}}}
\def\tT{{\tens{T}}}
\def\tU{{\tens{U}}}
\def\tV{{\tens{V}}}
\def\tW{{\tens{W}}}
\def\tX{{\tens{X}}}
\def\tY{{\tens{Y}}}
\def\tZ{{\tens{Z}}}


% Graph
\def\gA{{\mathcal{A}}}
\def\gB{{\mathcal{B}}}
\def\gC{{\mathcal{C}}}
\def\gD{{\mathcal{D}}}
\def\gE{{\mathcal{E}}}
\def\gF{{\mathcal{F}}}
\def\gG{{\mathcal{G}}}
\def\gH{{\mathcal{H}}}
\def\gI{{\mathcal{I}}}
\def\gJ{{\mathcal{J}}}
\def\gK{{\mathcal{K}}}
\def\gL{{\mathcal{L}}}
\def\gM{{\mathcal{M}}}
\def\gN{{\mathcal{N}}}
\def\gO{{\mathcal{O}}}
\def\gP{{\mathcal{P}}}
\def\gQ{{\mathcal{Q}}}
\def\gR{{\mathcal{R}}}
\def\gS{{\mathcal{S}}}
\def\gT{{\mathcal{T}}}
\def\gU{{\mathcal{U}}}
\def\gV{{\mathcal{V}}}
\def\gW{{\mathcal{W}}}
\def\gX{{\mathcal{X}}}
\def\gY{{\mathcal{Y}}}
\def\gZ{{\mathcal{Z}}}

% Sets
\def\sA{{\mathbb{A}}}
\def\sB{{\mathbb{B}}}
\def\sC{{\mathbb{C}}}
\def\sD{{\mathbb{D}}}
% Don't use a set called E, because this would be the same as our symbol
% for expectation.
\def\sF{{\mathbb{F}}}
\def\sG{{\mathbb{G}}}
\def\sH{{\mathbb{H}}}
\def\sI{{\mathbb{I}}}
\def\sJ{{\mathbb{J}}}
\def\sK{{\mathbb{K}}}
\def\sL{{\mathbb{L}}}
\def\sM{{\mathbb{M}}}
\def\sN{{\mathbb{N}}}
\def\sO{{\mathbb{O}}}
\def\sP{{\mathbb{P}}}
\def\sQ{{\mathbb{Q}}}
\def\sR{{\mathbb{R}}}
\def\sS{{\mathbb{S}}}
\def\sT{{\mathbb{T}}}
\def\sU{{\mathbb{U}}}
\def\sV{{\mathbb{V}}}
\def\sW{{\mathbb{W}}}
\def\sX{{\mathbb{X}}}
\def\sY{{\mathbb{Y}}}
\def\sZ{{\mathbb{Z}}}

% Entries of a matrix
\def\emLambda{{\Lambda}}
\def\emA{{A}}
\def\emB{{B}}
\def\emC{{C}}
\def\emD{{D}}
\def\emE{{E}}
\def\emF{{F}}
\def\emG{{G}}
\def\emH{{H}}
\def\emI{{I}}
\def\emJ{{J}}
\def\emK{{K}}
\def\emL{{L}}
\def\emM{{M}}
\def\emN{{N}}
\def\emO{{O}}
\def\emP{{P}}
\def\emQ{{Q}}
\def\emR{{R}}
\def\emS{{S}}
\def\emT{{T}}
\def\emU{{U}}
\def\emV{{V}}
\def\emW{{W}}
\def\emX{{X}}
\def\emY{{Y}}
\def\emZ{{Z}}
\def\emSigma{{\Sigma}}

% entries of a tensor
% Same font as tensor, without \bm wrapper
\newcommand{\etens}[1]{\mathsfit{#1}}
\def\etLambda{{\etens{\Lambda}}}
\def\etA{{\etens{A}}}
\def\etB{{\etens{B}}}
\def\etC{{\etens{C}}}
\def\etD{{\etens{D}}}
\def\etE{{\etens{E}}}
\def\etF{{\etens{F}}}
\def\etG{{\etens{G}}}
\def\etH{{\etens{H}}}
\def\etI{{\etens{I}}}
\def\etJ{{\etens{J}}}
\def\etK{{\etens{K}}}
\def\etL{{\etens{L}}}
\def\etM{{\etens{M}}}
\def\etN{{\etens{N}}}
\def\etO{{\etens{O}}}
\def\etP{{\etens{P}}}
\def\etQ{{\etens{Q}}}
\def\etR{{\etens{R}}}
\def\etS{{\etens{S}}}
\def\etT{{\etens{T}}}
\def\etU{{\etens{U}}}
\def\etV{{\etens{V}}}
\def\etW{{\etens{W}}}
\def\etX{{\etens{X}}}
\def\etY{{\etens{Y}}}
\def\etZ{{\etens{Z}}}

% The true underlying data generating distribution
\newcommand{\pdata}{p_{\rm{data}}}
% The empirical distribution defined by the training set
\newcommand{\ptrain}{\hat{p}_{\rm{data}}}
\newcommand{\Ptrain}{\hat{P}_{\rm{data}}}
% The model distribution
\newcommand{\pmodel}{p_{\rm{model}}}
\newcommand{\Pmodel}{P_{\rm{model}}}
\newcommand{\ptildemodel}{\tilde{p}_{\rm{model}}}
% Stochastic autoencoder distributions
\newcommand{\pencode}{p_{\rm{encoder}}}
\newcommand{\pdecode}{p_{\rm{decoder}}}
\newcommand{\precons}{p_{\rm{reconstruct}}}

\newcommand{\laplace}{\mathrm{Laplace}} % Laplace distribution

\newcommand{\E}{\mathbb{E}}
\newcommand{\Ls}{\mathcal{L}}
\newcommand{\R}{\mathbb{R}}
\newcommand{\emp}{\tilde{p}}
\newcommand{\lr}{\alpha}
\newcommand{\reg}{\lambda}
\newcommand{\rect}{\mathrm{rectifier}}
\newcommand{\softmax}{\mathrm{softmax}}
\newcommand{\sigmoid}{\sigma}
\newcommand{\softplus}{\zeta}
\newcommand{\KL}{D_{\mathrm{KL}}}
\newcommand{\Var}{\mathrm{Var}}
\newcommand{\standarderror}{\mathrm{SE}}
\newcommand{\Cov}{\mathrm{Cov}}
% Wolfram Mathworld says $L^2$ is for function spaces and $\ell^2$ is for vectors
% But then they seem to use $L^2$ for vectors throughout the site, and so does
% wikipedia.
\newcommand{\normlzero}{L^0}
\newcommand{\normlone}{L^1}
\newcommand{\normltwo}{L^2}
\newcommand{\normlp}{L^p}
\newcommand{\normmax}{L^\infty}

\newcommand{\parents}{Pa} % See usage in notation.tex. Chosen to match Daphne's book.

\DeclareMathOperator*{\argmax}{arg\,max}
\DeclareMathOperator*{\argmin}{arg\,min}

\DeclareMathOperator{\sign}{sign}
\DeclareMathOperator{\Tr}{Tr}
\let\ab\allowbreak

\newcommand{\norm}[2]{\left\| #1 \right\|_{#2}}

\newcommand{\zz}[1]{\textcolor{blue}{ [{\em Zhihui:} #1]}}
\newcommand{\jz}[1]{\textcolor{red}{ [{\em JZ:} #1]}}
% \newcommand{\td}[1]{\textcolor{blue}{ [{\em TD:} #1]}}
\newcommand{\jj}[1]{\textcolor{pink}{ [{\em JJ:} #1]}}
% Use the following line for the initial blind version submitted for review:
\usepackage[accepted]{mlsys2023}

% %%%%% NEW MATH DEFINITIONS %%%%%

\usepackage{amsmath,amsfonts,bm}

% Mark sections of captions for referring to divisions of figures
\newcommand{\figleft}{{\em (Left)}}
\newcommand{\figcenter}{{\em (Center)}}
\newcommand{\figright}{{\em (Right)}}
\newcommand{\figtop}{{\em (Top)}}
\newcommand{\figbottom}{{\em (Bottom)}}
\newcommand{\captiona}{{\em (a)}}
\newcommand{\captionb}{{\em (b)}}
\newcommand{\captionc}{{\em (c)}}
\newcommand{\captiond}{{\em (d)}}

% Highlight a newly defined term
\newcommand{\newterm}[1]{{\bf #1}}


% Figure reference, lower-case.
\def\figref#1{figure~\ref{#1}}
% Figure reference, capital. For start of sentence
\def\Figref#1{Figure~\ref{#1}}
\def\twofigref#1#2{figures \ref{#1} and \ref{#2}}
\def\quadfigref#1#2#3#4{figures \ref{#1}, \ref{#2}, \ref{#3} and \ref{#4}}
% Section reference, lower-case.
\def\secref#1{section~\ref{#1}}
% Section reference, capital.
\def\Secref#1{Section~\ref{#1}}
% Reference to two sections.
\def\twosecrefs#1#2{sections \ref{#1} and \ref{#2}}
% Reference to three sections.
\def\secrefs#1#2#3{sections \ref{#1}, \ref{#2} and \ref{#3}}
% Reference to an equation, lower-case.
% \def\eqref#1{equation~\ref{#1}}
 \def\eqref#1{(\ref{#1})}
% Reference to an equation, upper case
\def\Eqref#1{Equation~\ref{#1}}
% A raw reference to an equation---avoid using if possible
\def\plaineqref#1{\ref{#1}}
% Reference to a chapter, lower-case.
\def\chapref#1{chapter~\ref{#1}}
% Reference to an equation, upper case.
\def\Chapref#1{Chapter~\ref{#1}}
% Reference to a range of chapters
\def\rangechapref#1#2{chapters\ref{#1}--\ref{#2}}
% Reference to an algorithm, lower-case.
\def\algref#1{algorithm~\ref{#1}}
% Reference to an algorithm, upper case.
\def\Algref#1{Algorithm~\ref{#1}}
\def\twoalgref#1#2{algorithms \ref{#1} and \ref{#2}}
\def\Twoalgref#1#2{Algorithms \ref{#1} and \ref{#2}}
% Reference to a part, lower case
\def\partref#1{part~\ref{#1}}
% Reference to a part, upper case
\def\Partref#1{Part~\ref{#1}}
\def\twopartref#1#2{parts \ref{#1} and \ref{#2}}

\def\ceil#1{\lceil #1 \rceil}
\def\floor#1{\lfloor #1 \rfloor}
\def\1{\bm{1}}
\newcommand{\train}{\mathcal{D}}
\newcommand{\valid}{\mathcal{D_{\mathrm{valid}}}}
\newcommand{\test}{\mathcal{D_{\mathrm{test}}}}

\def\eps{{\epsilon}}


% Random variables
\def\reta{{\textnormal{$\eta$}}}
\def\ra{{\textnormal{a}}}
\def\rb{{\textnormal{b}}}
\def\rc{{\textnormal{c}}}
\def\rd{{\textnormal{d}}}
\def\re{{\textnormal{e}}}
\def\rf{{\textnormal{f}}}
\def\rg{{\textnormal{g}}}
\def\rh{{\textnormal{h}}}
\def\ri{{\textnormal{i}}}
\def\rj{{\textnormal{j}}}
\def\rk{{\textnormal{k}}}
\def\rl{{\textnormal{l}}}
% rm is already a command, just don't name any random variables m
\def\rn{{\textnormal{n}}}
\def\ro{{\textnormal{o}}}
\def\rp{{\textnormal{p}}}
\def\rq{{\textnormal{q}}}
\def\rr{{\textnormal{r}}}
\def\rs{{\textnormal{s}}}
\def\rt{{\textnormal{t}}}
\def\ru{{\textnormal{u}}}
\def\rv{{\textnormal{v}}}
\def\rw{{\textnormal{w}}}
\def\rx{{\textnormal{x}}}
\def\ry{{\textnormal{y}}}
\def\rz{{\textnormal{z}}}

% Random vectors
\def\rvepsilon{{\mathbf{\epsilon}}}
\def\rvtheta{{\mathbf{\theta}}}
\def\rva{{\mathbf{a}}}
\def\rvb{{\mathbf{b}}}
\def\rvc{{\mathbf{c}}}
\def\rvd{{\mathbf{d}}}
\def\rve{{\mathbf{e}}}
\def\rvf{{\mathbf{f}}}
\def\rvg{{\mathbf{g}}}
\def\rvh{{\mathbf{h}}}
\def\rvu{{\mathbf{i}}}
\def\rvj{{\mathbf{j}}}
\def\rvk{{\mathbf{k}}}
\def\rvl{{\mathbf{l}}}
\def\rvm{{\mathbf{m}}}
\def\rvn{{\mathbf{n}}}
\def\rvo{{\mathbf{o}}}
\def\rvp{{\mathbf{p}}}
\def\rvq{{\mathbf{q}}}
\def\rvr{{\mathbf{r}}}
\def\rvs{{\mathbf{s}}}
\def\rvt{{\mathbf{t}}}
\def\rvu{{\mathbf{u}}}
\def\rvv{{\mathbf{v}}}
\def\rvw{{\mathbf{w}}}
\def\rvx{{\mathbf{x}}}
\def\rvy{{\mathbf{y}}}
\def\rvz{{\mathbf{z}}}

% Elements of random vectors
\def\erva{{\textnormal{a}}}
\def\ervb{{\textnormal{b}}}
\def\ervc{{\textnormal{c}}}
\def\ervd{{\textnormal{d}}}
\def\erve{{\textnormal{e}}}
\def\ervf{{\textnormal{f}}}
\def\ervg{{\textnormal{g}}}
\def\ervh{{\textnormal{h}}}
\def\ervi{{\textnormal{i}}}
\def\ervj{{\textnormal{j}}}
\def\ervk{{\textnormal{k}}}
\def\ervl{{\textnormal{l}}}
\def\ervm{{\textnormal{m}}}
\def\ervn{{\textnormal{n}}}
\def\ervo{{\textnormal{o}}}
\def\ervp{{\textnormal{p}}}
\def\ervq{{\textnormal{q}}}
\def\ervr{{\textnormal{r}}}
\def\ervs{{\textnormal{s}}}
\def\ervt{{\textnormal{t}}}
\def\ervu{{\textnormal{u}}}
\def\ervv{{\textnormal{v}}}
\def\ervw{{\textnormal{w}}}
\def\ervx{{\textnormal{x}}}
\def\ervy{{\textnormal{y}}}
\def\ervz{{\textnormal{z}}}

% Random matrices
\def\rmA{{\mathbf{A}}}
\def\rmB{{\mathbf{B}}}
\def\rmC{{\mathbf{C}}}
\def\rmD{{\mathbf{D}}}
\def\rmE{{\mathbf{E}}}
\def\rmF{{\mathbf{F}}}
\def\rmG{{\mathbf{G}}}
\def\rmH{{\mathbf{H}}}
\def\rmI{{\mathbf{I}}}
\def\rmJ{{\mathbf{J}}}
\def\rmK{{\mathbf{K}}}
\def\rmL{{\mathbf{L}}}
\def\rmM{{\mathbf{M}}}
\def\rmN{{\mathbf{N}}}
\def\rmO{{\mathbf{O}}}
\def\rmP{{\mathbf{P}}}
\def\rmQ{{\mathbf{Q}}}
\def\rmR{{\mathbf{R}}}
\def\rmS{{\mathbf{S}}}
\def\rmT{{\mathbf{T}}}
\def\rmU{{\mathbf{U}}}
\def\rmV{{\mathbf{V}}}
\def\rmW{{\mathbf{W}}}
\def\rmX{{\mathbf{X}}}
\def\rmY{{\mathbf{Y}}}
\def\rmZ{{\mathbf{Z}}}

% Elements of random matrices
\def\ermA{{\textnormal{A}}}
\def\ermB{{\textnormal{B}}}
\def\ermC{{\textnormal{C}}}
\def\ermD{{\textnormal{D}}}
\def\ermE{{\textnormal{E}}}
\def\ermF{{\textnormal{F}}}
\def\ermG{{\textnormal{G}}}
\def\ermH{{\textnormal{H}}}
\def\ermI{{\textnormal{I}}}
\def\ermJ{{\textnormal{J}}}
\def\ermK{{\textnormal{K}}}
\def\ermL{{\textnormal{L}}}
\def\ermM{{\textnormal{M}}}
\def\ermN{{\textnormal{N}}}
\def\ermO{{\textnormal{O}}}
\def\ermP{{\textnormal{P}}}
\def\ermQ{{\textnormal{Q}}}
\def\ermR{{\textnormal{R}}}
\def\ermS{{\textnormal{S}}}
\def\ermT{{\textnormal{T}}}
\def\ermU{{\textnormal{U}}}
\def\ermV{{\textnormal{V}}}
\def\ermW{{\textnormal{W}}}
\def\ermX{{\textnormal{X}}}
\def\ermY{{\textnormal{Y}}}
\def\ermZ{{\textnormal{Z}}}

% Vectors
\def\vzero{{\bm{0}}}
\def\vone{{\bm{1}}}
\def\vmu{{\bm{\mu}}}
\def\vtheta{{\bm{\theta}}}
\def\va{{\bm{a}}}
\def\vb{{\bm{b}}}
\def\vc{{\bm{c}}}
\def\vd{{\bm{d}}}
\def\ve{{\bm{e}}}
\def\vf{{\bm{f}}}
\def\vg{{\bm{g}}}
\def\vh{{\bm{h}}}
\def\vi{{\bm{i}}}
\def\vj{{\bm{j}}}
\def\vk{{\bm{k}}}
\def\vl{{\bm{l}}}
\def\vm{{\bm{m}}}
\def\vn{{\bm{n}}}
\def\vo{{\bm{o}}}
\def\vp{{\bm{p}}}
\def\vq{{\bm{q}}}
\def\vr{{\bm{r}}}
\def\vs{{\bm{s}}}
\def\vt{{\bm{t}}}
\def\vu{{\bm{u}}}
\def\vv{{\bm{v}}}
\def\vw{{\bm{w}}}
\def\vx{{\bm{x}}}
\def\vy{{\bm{y}}}
\def\vz{{\bm{z}}}

% Elements of vectors
\def\evalpha{{\alpha}}
\def\evbeta{{\beta}}
\def\evepsilon{{\epsilon}}
\def\evlambda{{\lambda}}
\def\evomega{{\omega}}
\def\evmu{{\mu}}
\def\evpsi{{\psi}}
\def\evsigma{{\sigma}}
\def\evtheta{{\theta}}
\def\eva{{a}}
\def\evb{{b}}
\def\evc{{c}}
\def\evd{{d}}
\def\eve{{e}}
\def\evf{{f}}
\def\evg{{g}}
\def\evh{{h}}
\def\evi{{i}}
\def\evj{{j}}
\def\evk{{k}}
\def\evl{{l}}
\def\evm{{m}}
\def\evn{{n}}
\def\evo{{o}}
\def\evp{{p}}
\def\evq{{q}}
\def\evr{{r}}
\def\evs{{s}}
\def\evt{{t}}
\def\evu{{u}}
\def\evv{{v}}
\def\evw{{w}}
\def\evx{{x}}
\def\evy{{y}}
\def\evz{{z}}

% Matrix
\def\mA{{\bm{A}}}
\def\mB{{\bm{B}}}
\def\mC{{\bm{C}}}
\def\mD{{\bm{D}}}
\def\mE{{\bm{E}}}
\def\mF{{\bm{F}}}
\def\mG{{\bm{G}}}
\def\mH{{\bm{H}}}
\def\mI{{\bm{I}}}
\def\mJ{{\bm{J}}}
\def\mK{{\bm{K}}}
\def\mL{{\bm{L}}}
\def\mM{{\bm{M}}}
\def\mN{{\bm{N}}}
\def\mO{{\bm{O}}}
\def\mP{{\bm{P}}}
\def\mQ{{\bm{Q}}}
\def\mR{{\bm{R}}}
\def\mS{{\bm{S}}}
\def\mT{{\bm{T}}}
\def\mU{{\bm{U}}}
\def\mV{{\bm{V}}}
\def\mW{{\bm{W}}}
\def\mX{{\bm{X}}}
\def\mY{{\bm{Y}}}
\def\mZ{{\bm{Z}}}
\def\mBeta{{\bm{\beta}}}
\def\mPhi{{\bm{\Phi}}}
\def\mLambda{{\bm{\Lambda}}}
\def\mSigma{{\bm{\Sigma}}}

% Tensor
\DeclareMathAlphabet{\mathsfit}{\encodingdefault}{\sfdefault}{m}{sl}
\SetMathAlphabet{\mathsfit}{bold}{\encodingdefault}{\sfdefault}{bx}{n}
\newcommand{\tens}[1]{\bm{\mathsfit{#1}}}
\def\tA{{\tens{A}}}
\def\tB{{\tens{B}}}
\def\tC{{\tens{C}}}
\def\tD{{\tens{D}}}
\def\tE{{\tens{E}}}
\def\tF{{\tens{F}}}
\def\tG{{\tens{G}}}
\def\tH{{\tens{H}}}
\def\tI{{\tens{I}}}
\def\tJ{{\tens{J}}}
\def\tK{{\tens{K}}}
\def\tL{{\tens{L}}}
\def\tM{{\tens{M}}}
\def\tN{{\tens{N}}}
\def\tO{{\tens{O}}}
\def\tP{{\tens{P}}}
\def\tQ{{\tens{Q}}}
\def\tR{{\tens{R}}}
\def\tS{{\tens{S}}}
\def\tT{{\tens{T}}}
\def\tU{{\tens{U}}}
\def\tV{{\tens{V}}}
\def\tW{{\tens{W}}}
\def\tX{{\tens{X}}}
\def\tY{{\tens{Y}}}
\def\tZ{{\tens{Z}}}


% Graph
\def\gA{{\mathcal{A}}}
\def\gB{{\mathcal{B}}}
\def\gC{{\mathcal{C}}}
\def\gD{{\mathcal{D}}}
\def\gE{{\mathcal{E}}}
\def\gF{{\mathcal{F}}}
\def\gG{{\mathcal{G}}}
\def\gH{{\mathcal{H}}}
\def\gI{{\mathcal{I}}}
\def\gJ{{\mathcal{J}}}
\def\gK{{\mathcal{K}}}
\def\gL{{\mathcal{L}}}
\def\gM{{\mathcal{M}}}
\def\gN{{\mathcal{N}}}
\def\gO{{\mathcal{O}}}
\def\gP{{\mathcal{P}}}
\def\gQ{{\mathcal{Q}}}
\def\gR{{\mathcal{R}}}
\def\gS{{\mathcal{S}}}
\def\gT{{\mathcal{T}}}
\def\gU{{\mathcal{U}}}
\def\gV{{\mathcal{V}}}
\def\gW{{\mathcal{W}}}
\def\gX{{\mathcal{X}}}
\def\gY{{\mathcal{Y}}}
\def\gZ{{\mathcal{Z}}}

% Sets
\def\sA{{\mathbb{A}}}
\def\sB{{\mathbb{B}}}
\def\sC{{\mathbb{C}}}
\def\sD{{\mathbb{D}}}
% Don't use a set called E, because this would be the same as our symbol
% for expectation.
\def\sF{{\mathbb{F}}}
\def\sG{{\mathbb{G}}}
\def\sH{{\mathbb{H}}}
\def\sI{{\mathbb{I}}}
\def\sJ{{\mathbb{J}}}
\def\sK{{\mathbb{K}}}
\def\sL{{\mathbb{L}}}
\def\sM{{\mathbb{M}}}
\def\sN{{\mathbb{N}}}
\def\sO{{\mathbb{O}}}
\def\sP{{\mathbb{P}}}
\def\sQ{{\mathbb{Q}}}
\def\sR{{\mathbb{R}}}
\def\sS{{\mathbb{S}}}
\def\sT{{\mathbb{T}}}
\def\sU{{\mathbb{U}}}
\def\sV{{\mathbb{V}}}
\def\sW{{\mathbb{W}}}
\def\sX{{\mathbb{X}}}
\def\sY{{\mathbb{Y}}}
\def\sZ{{\mathbb{Z}}}

% Entries of a matrix
\def\emLambda{{\Lambda}}
\def\emA{{A}}
\def\emB{{B}}
\def\emC{{C}}
\def\emD{{D}}
\def\emE{{E}}
\def\emF{{F}}
\def\emG{{G}}
\def\emH{{H}}
\def\emI{{I}}
\def\emJ{{J}}
\def\emK{{K}}
\def\emL{{L}}
\def\emM{{M}}
\def\emN{{N}}
\def\emO{{O}}
\def\emP{{P}}
\def\emQ{{Q}}
\def\emR{{R}}
\def\emS{{S}}
\def\emT{{T}}
\def\emU{{U}}
\def\emV{{V}}
\def\emW{{W}}
\def\emX{{X}}
\def\emY{{Y}}
\def\emZ{{Z}}
\def\emSigma{{\Sigma}}

% entries of a tensor
% Same font as tensor, without \bm wrapper
\newcommand{\etens}[1]{\mathsfit{#1}}
\def\etLambda{{\etens{\Lambda}}}
\def\etA{{\etens{A}}}
\def\etB{{\etens{B}}}
\def\etC{{\etens{C}}}
\def\etD{{\etens{D}}}
\def\etE{{\etens{E}}}
\def\etF{{\etens{F}}}
\def\etG{{\etens{G}}}
\def\etH{{\etens{H}}}
\def\etI{{\etens{I}}}
\def\etJ{{\etens{J}}}
\def\etK{{\etens{K}}}
\def\etL{{\etens{L}}}
\def\etM{{\etens{M}}}
\def\etN{{\etens{N}}}
\def\etO{{\etens{O}}}
\def\etP{{\etens{P}}}
\def\etQ{{\etens{Q}}}
\def\etR{{\etens{R}}}
\def\etS{{\etens{S}}}
\def\etT{{\etens{T}}}
\def\etU{{\etens{U}}}
\def\etV{{\etens{V}}}
\def\etW{{\etens{W}}}
\def\etX{{\etens{X}}}
\def\etY{{\etens{Y}}}
\def\etZ{{\etens{Z}}}

% The true underlying data generating distribution
\newcommand{\pdata}{p_{\rm{data}}}
% The empirical distribution defined by the training set
\newcommand{\ptrain}{\hat{p}_{\rm{data}}}
\newcommand{\Ptrain}{\hat{P}_{\rm{data}}}
% The model distribution
\newcommand{\pmodel}{p_{\rm{model}}}
\newcommand{\Pmodel}{P_{\rm{model}}}
\newcommand{\ptildemodel}{\tilde{p}_{\rm{model}}}
% Stochastic autoencoder distributions
\newcommand{\pencode}{p_{\rm{encoder}}}
\newcommand{\pdecode}{p_{\rm{decoder}}}
\newcommand{\precons}{p_{\rm{reconstruct}}}

\newcommand{\laplace}{\mathrm{Laplace}} % Laplace distribution

\newcommand{\E}{\mathbb{E}}
\newcommand{\Ls}{\mathcal{L}}
\newcommand{\R}{\mathbb{R}}
\newcommand{\emp}{\tilde{p}}
\newcommand{\lr}{\alpha}
\newcommand{\reg}{\lambda}
\newcommand{\rect}{\mathrm{rectifier}}
\newcommand{\softmax}{\mathrm{softmax}}
\newcommand{\sigmoid}{\sigma}
\newcommand{\softplus}{\zeta}
\newcommand{\KL}{D_{\mathrm{KL}}}
\newcommand{\Var}{\mathrm{Var}}
\newcommand{\standarderror}{\mathrm{SE}}
\newcommand{\Cov}{\mathrm{Cov}}
% Wolfram Mathworld says $L^2$ is for function spaces and $\ell^2$ is for vectors
% But then they seem to use $L^2$ for vectors throughout the site, and so does
% wikipedia.
\newcommand{\normlzero}{L^0}
\newcommand{\normlone}{L^1}
\newcommand{\normltwo}{L^2}
\newcommand{\normlp}{L^p}
\newcommand{\normmax}{L^\infty}

\newcommand{\parents}{Pa} % See usage in notation.tex. Chosen to match Daphne's book.

\DeclareMathOperator*{\argmax}{arg\,max}
\DeclareMathOperator*{\argmin}{arg\,min}

\DeclareMathOperator{\sign}{sign}
\DeclareMathOperator{\Tr}{Tr}
\let\ab\allowbreak

\newcommand{\norm}[2]{\left\| #1 \right\|_{#2}}

\newcommand{\zz}[1]{\textcolor{blue}{ [{\em Zhihui:} #1]}}
\newcommand{\jz}[1]{\textcolor{red}{ [{\em JZ:} #1]}}
% \newcommand{\td}[1]{\textcolor{blue}{ [{\em TD:} #1]}}
\newcommand{\jj}[1]{\textcolor{pink}{ [{\em JJ:} #1]}}
\usepackage[utf8]{inputenc} % allow utf-8 input
\usepackage[T1]{fontenc}    % use 8-bit T1 fonts
\usepackage{url}            % simple URL typesetting
\usepackage{booktabs}       % professional-quality tables
\usepackage{amsfonts}       % blackboard math symbols
\usepackage{nicefrac}       % compact symbols for 1/2, etc.
\usepackage{microtype}      % microtypography
%\usepackage{xcolor}         % colors
\usepackage{graphicx}

\usepackage[ruled,vlined]{algorithm2e}

%\usepackage{algorithm}
%\usepackage{algorithmic}
%\usepackage{algorithmicx}
%\usepackage{algpseudocode}


\usepackage{microtype}
\usepackage{graphicx}
\usepackage{subfigure}
\usepackage{enumitem}
\usepackage{pifont}
\usepackage{adjustbox}
\usepackage{caption}
\usepackage{booktabs} % for professional tables
\usepackage{color, colortbl}
\usepackage[dvipsnames]{xcolor}
\usepackage{footnote}
\usepackage{multirow}
\usepackage{listings}

\usepackage{graphicx}
\usepackage{amsmath}
\usepackage{subfigure}
\usepackage{tcolorbox}
\usepackage{multirow}
\usepackage{pifont}
%\usepackage[finalizecache,cachedir=.]{minted}

\usepackage{amssymb}
\usepackage{diagbox}

\newcommand{\eg}{\textit{e.g.}}
\newcommand{\ie}{\textit{i.e.}}
\newcommand{\etc}{\textit{etc}}

\def\sectionautorefname{Sec.}
\def\algorithmautorefname{Algo.}
\def\corollaryautorefname{Corollary}
\def\subsectionautorefname{Sec.}
\def\figureautorefname{Fig.}
\def\equationautorefname~#1\null{Eq.~(#1)\null}

\newcommand{\yh}[1]{\textcolor{blue}{[YH: #1]}}



% If accepted, instead use the following line for the camera-ready submission:
%\usepackage[accepted]{mlsys2023}

% The \mlsystitle you define below is probably too long as a header.
% Therefore, a short form for the running title is supplied here:
\mlsystitlerunning{SysNoise: Exploring and Benchmarking Training-Deployment System Inconsistency}

\begin{document}

\twocolumn[
\mlsystitle{SysNoise: Exploring and Benchmarking Training-Deployment System Inconsistency}

% It is OKAY to include author information, even for blind
% submissions: the style file will automatically remove it for you
% unless you've provided the [accepted] option to the mlsys2023
% package.

% List of affiliations: The first argument should be a (short)
% identifier you will use later to specify author affiliations
% Academic affiliations should list Department, University, City, Region, Country
% Industry affiliations should list Company, City, Region, Country

% You can specify symbols, otherwise they are numbered in order.
% Ideally, you should not use this facility. Affiliations will be numbered
% in order of appearance and this is the preferred way.
\mlsyssetsymbol{equal}{*}

\begin{mlsysauthorlist}
\mlsysauthor{Yan Wang}{equal,st}
\mlsysauthor{Yuhang Li}{equal,st,ya}
\mlsysauthor{Ruihao Gong}{equal,st,bh}
\mlsysauthor{Aishan Liu}{equal,bh}
\mlsysauthor{Yanfei Wang}{st}
\mlsysauthor{Jian Hu}{st}
\mlsysauthor{Yongqiang Yao}{st}
\mlsysauthor{Yunchen Zhang}{st}
\mlsysauthor{Tianzi Xiao}{st}
\mlsysauthor{Fengwei Yu}{st}
\mlsysauthor{Xianglong Liu}{bh}
\end{mlsysauthorlist}

\mlsysaffiliation{st}{SenseTime Research, Beijing, China}
\mlsysaffiliation{bh}{Beihang University, Beijing, China}
\mlsysaffiliation{ya}{Yale University, New Haven, USA}

\mlsyscorrespondingauthor{Yan Wang}{mrwangyan98@gmail.com}
%\mlsyscorrespondingauthor{Eee Pppp}{ep@eden.co.uk}

% You may provide any keywords that you
% find helpful for describing your paper; these are used to populate
% the "keywords" metadata in the PDF but will not be shown in the document
\mlsyskeywords{Machine Learning, MLSys}

%\vskip 0.3in

\begin{abstract}
Extensive studies have shown that deep learning models are vulnerable to adversarial and natural noises, yet little is known about model robustness on noises caused by different system implementations. In this paper, we for the first time introduce SysNoise, a frequently occurred but often overlooked noise in the deep learning training-deployment cycle. In particular, SysNoise happens when the source training system switches to a disparate target system in deployments, where various tiny system mismatch adds up to a non-negligible difference. We first identify and classify SysNoise into three categories based on the inference stage; we then build a holistic benchmark to quantitatively measure the impact of SysNoise on 20+ models, comprehending image classification, object detection, instance segmentation and natural language processing tasks. Our extensive experiments revealed that SysNoise could bring certain impacts on model robustness across different tasks and common mitigations like data augmentation and adversarial training show limited effects on it. Together, our findings open a new research topic and we hope this work will raise research attention to deep learning deployment systems accounting for model performance. We have open-sourced the benchmark and framework at \url{https://modeltc.github.io/systemnoise_web}.
\end{abstract}
]

% this must go after the closing bracket ] following \twocolumn[ ...

% This command actually creates the footnote in the first column
% listing the affiliations and the copyright notice.
% The command takes one argument, which is text to display at the start of the footnote.
% The \mlsysEqualContribution command is standard text for equal contribution.
% Remove it (just {}) if you do not need this facility.

%\printAffiliationsAndNotice{}  % leave blank if no need to mention equal contribution
\printAffiliationsAndNotice{\mlsysEqualContribution} % otherwise use the standard text.

% !TEX root = ../arxiv.tex

Unsupervised domain adaptation (UDA) is a variant of semi-supervised learning \cite{blum1998combining}, where the available unlabelled data comes from a different distribution than the annotated dataset \cite{Ben-DavidBCP06}.
A case in point is to exploit synthetic data, where annotation is more accessible compared to the costly labelling of real-world images \cite{RichterVRK16,RosSMVL16}.
Along with some success in addressing UDA for semantic segmentation \cite{TsaiHSS0C18,VuJBCP19,0001S20,ZouYKW18}, the developed methods are growing increasingly sophisticated and often combine style transfer networks, adversarial training or network ensembles \cite{KimB20a,LiYV19,TsaiSSC19,Yang_2020_ECCV}.
This increase in model complexity impedes reproducibility, potentially slowing further progress.

In this work, we propose a UDA framework reaching state-of-the-art segmentation accuracy (measured by the Intersection-over-Union, IoU) without incurring substantial training efforts.
Toward this goal, we adopt a simple semi-supervised approach, \emph{self-training} \cite{ChenWB11,lee2013pseudo,ZouYKW18}, used in recent works only in conjunction with adversarial training or network ensembles \cite{ChoiKK19,KimB20a,Mei_2020_ECCV,Wang_2020_ECCV,0001S20,Zheng_2020_IJCV,ZhengY20}.
By contrast, we use self-training \emph{standalone}.
Compared to previous self-training methods \cite{ChenLCCCZAS20,Li_2020_ECCV,subhani2020learning,ZouYKW18,ZouYLKW19}, our approach also sidesteps the inconvenience of multiple training rounds, as they often require expert intervention between consecutive rounds.
We train our model using co-evolving pseudo labels end-to-end without such need.

\begin{figure}[t]%
    \centering
    \def\svgwidth{\linewidth}
    \input{figures/preview/bars.pdf_tex}
    \caption{\textbf{Results preview.} Unlike much recent work that combines multiple training paradigms, such as adversarial training and style transfer, our approach retains the modest single-round training complexity of self-training, yet improves the state of the art for adapting semantic segmentation by a significant margin.}
    \label{fig:preview}
\end{figure}

Our method leverages the ubiquitous \emph{data augmentation} techniques from fully supervised learning \cite{deeplabv3plus2018,ZhaoSQWJ17}: photometric jitter, flipping and multi-scale cropping.
We enforce \emph{consistency} of the semantic maps produced by the model across these image perturbations.
The following assumption formalises the key premise:

\myparagraph{Assumption 1.}
Let $f: \mathcal{I} \rightarrow \mathcal{M}$ represent a pixelwise mapping from images $\mathcal{I}$ to semantic output $\mathcal{M}$.
Denote $\rho_{\bm{\epsilon}}: \mathcal{I} \rightarrow \mathcal{I}$ a photometric image transform and, similarly, $\tau_{\bm{\epsilon}'}: \mathcal{I} \rightarrow \mathcal{I}$ a spatial similarity transformation, where $\bm{\epsilon},\bm{\epsilon}'\sim p(\cdot)$ are control variables following some pre-defined density (\eg, $p \equiv \mathcal{N}(0, 1)$).
Then, for any image $I \in \mathcal{I}$, $f$ is \emph{invariant} under $\rho_{\bm{\epsilon}}$ and \emph{equivariant} under $\tau_{\bm{\epsilon}'}$, \ie~$f(\rho_{\bm{\epsilon}}(I)) = f(I)$ and $f(\tau_{\bm{\epsilon}'}(I)) = \tau_{\bm{\epsilon}'}(f(I))$.

\smallskip
\noindent Next, we introduce a training framework using a \emph{momentum network} -- a slowly advancing copy of the original model.
The momentum network provides stable, yet recent targets for model updates, as opposed to the fixed supervision in model distillation \cite{Chen0G18,Zheng_2020_IJCV,ZhengY20}.
We also re-visit the problem of long-tail recognition in the context of generating pseudo labels for self-supervision.
In particular, we maintain an \emph{exponentially moving class prior} used to discount the confidence thresholds for those classes with few samples and increase their relative contribution to the training loss.
Our framework is simple to train, adds moderate computational overhead compared to a fully supervised setup, yet sets a new state of the art on established benchmarks (\cf \cref{fig:preview}).

\section{Related Work}
\textbf{Noises Types and Benchmarks.} Extensive shreds of evidence have shown that deep learning models are unstable towards different noises, including adversarial noises and natural noises. \emph{Adversarial noises}, which are imperceptible to human vision, could easily make neural networks misclassify the input images~\cite{szegedy2013intriguing,fgsm,liu2022harnessing,liu2023x,Liu2019Perceptual,liang2021parallel,Wang_2021_CVPR,liu2020bias,liu2020spatiotemporal}. To benchmark and evaluate adversarial robustness, \cite{su2018is} first investigated the adversarial robustness of 18 models on ImageNet; \cite{Ling2019Deepsec} built the  platform DEEPSEC for adversarial robustness analysis including 16 adversarial attacks, and 13 adversarial defenses; meanwhile, RealSafe \cite{Dong2020Benchmarking} open-sourced and benchmarked adversarial robustness on image classification tasks. More recently, large-scale benchmarks on adversarial robustness regarding defense strategies (RobustBench \cite{croce2020robustbench}) and model architectures (RobustART \cite{tang2021robustart,liu2023exploring}) were developed. Besides adversarial noises, there exist another type of model-agnostic noise named \emph{natural noises} (also deemed as corruptions), which are commonly witnessed in the real-world scenario, \eg, blur, snow, and frost. Some representative datasets are constructed to simulate and benchmark the natural noises, such as ImageNet-P, ImageNet-C~\cite{hendrycks2019robustness}, and ImageNet-A, ImageNet-O~\cite{hendrycks2021nae}. \cite{hendrycks2020many} also introduced new real-world distribution shift datasets including changes in image style, geographic location \etc. However, these studies only focus on noises brought during data acquisition, while ignoring the impacts of the whole inference pipeline caused by different system implementations.

Furthermore, some studies introduce the influence of individual SysNoise. \cite{yan2021real, boltaevich2019estimationresize} show how image pre-processing progresses including image decoder, resize method, and color conversion generate noise. However, they only introduce one or two noises in image pre-processing and lack investigation on the whole training-deployment progress as well as the combination of system noise.
\cite{biterror} introduces the bit error that is caused by the Low-voltage operation of DNN accelerators, which does illustrate that training-deployment system inconsistency can bring error. And \cite{randomnessintraining} show the random noise caused by different training systems. But this work only focuses on differences in the training system and ignores the deployment system.
In addition, ~\cite{jia2021exploiting} takes the first step towards the influence of the floating-point value representation. They highlight that, to achieve practically reliable verification of neural networks, the system must accurately model the effects of any floating-point computations. However, this paper only conducts a preliminary attempt at the effect of floating-point numerical error for neural network verifiers. 

% By contrast, \emph{this paper for the first time proposes system noises, which is caused by the training-deployment system inconsistency, and accordingly builds a benchmark to comprehensively study the effect on model performance.}



\textbf{Approaches to Improving Model Robustness.} To improve model robustness against \emph{adversarial noises}, a long line of adversarial defense works have been proposed including: (1) adversarial training that adversarially train deep models using adversarial examples \cite{goodfellow2014explaining,PGD,tramer2017ensemble,shafahi2019adversarial,liu2021ANP,zhang2021interpreting}; and (2) adversarial detection that distinguishes the clean example and adversarial example \cite{grosse2017statistical,gong2017adversarial,jiang2020attack}. 

To effectively tackle the \emph{natural noises}, several studies have been devoted primarily from the perspective of data augmentation. By producing an elementwise convex combination of two images, Mixup~\cite{mixup2017} could regularize neural networks to favor simple linear behavior in-between training examples and improve model performance. Different from Mixup, AutoAugment \cite{2018AutoAugment} adopts and tunes a group of augmentations to optimize performance on a downstream task. To further improve model robustness against natural noises, AugMix \cite{hendrycks2020augmix} was proposed to mix multiple augmented images. And APR-SP~\cite{chen2021amplitude} was proposed to force the CNN to pay more attention on the structured information from phase components and keep robust to the variation of the amplitude which can help with the model's robustness of natural noise.

Test-time adaptation is another way to improve the model's performance at inference. It refers to adapting a machine learning model to a target domain at test time, without access to the source data or even any additional labeled/unlabeled samples from the target distribution to fine-tune the source model. \cite{tent} propose a method to reduce generalization error by reducing the entropy of model predictions on test data, and it reduces error for image classification on corrupted ImageNet and CIFAR-10/100 and reaches a new state-of-the-art error on ImageNet-C.



 %The adversarial examples were first introduced by~\cite{szegedy2013intriguing}. From then on, various adversarial attacks were proposed, such as FGSM~\cite{fgsm}, PGD~\cite{PGD}, C\&W~\cite{CW} and some black-box attack methods~\cite{black_box}. Driven by the emergence of adversarial noises, corresponding defense techniques also arose, including adversarial training~\cite{ganin2016domain,tramer2017ensemble,shafahi2019adversarial}, data augmentation~\cite{Maxup} and regularization~\cite{LabelSmoothing}. The adversarial examples are always dependent on the model to attack, especially for the block-box attack. Thus they suffer a low transferability and rarely occur in the practical scenario.

%\textbf{Natural Noises. } Besides the adversarial examples, the community also realizes the importance of natural noises that are widely existing in the real world. Some representative datasets are constructed to simulate the natural noise, such as ImageNet-P, ImageNet-C~\cite{hendrycks2019robustness}, and ImageNet-A, ImageNet-O~\cite{hendrycks2021nae}. These noises are model-agnostic and may cause perceptible perturbation. Natural noises such as Snow noise and Frost noise can measure the robustness of a model in the wild.



\section{Verification benchmark}
\label{sec:benchmark}

% Relative to the tokamak core, the characteristic time and spatial scales
% are compressed.  However, type-I ELMs still have the instability time
% scale associated with the fast crash is an order of magnitude faster
% than the transport-time scale associated with the processes that govern
% the build up of the pedestal structure.   This separation of time scales
% still allows the standard decomposition of studying the linear
% instabilities about an equilibrium that is used in core modes as well.
% 
% Like the core modes, these long-wavelength instabilities are dominated
% by the stiffness in the ideal MHD terms, even for the cases when they
% may be strictly ideal stable.  Multiple numerical methods have been
% developed to handle this stiffness for both linear and nonlinear codes.
% For the nonlinear codes, one numerical advantage is to 
% separate the fields into steady-state (e.g. the reconstructed fields)
% and time-dependent parts.  The pure steady-state terms are analytically
% eliminated resulting in the largest terms in the system to be removed
% from the numerical computations.
% 
% Although typically only MHD-force balance (a
% Grad-Shafranov solution) is strictly enforced for the steady state, in practice
% all fields associated are time independent. This effectively assumes the
% presence of implicit (in the sense that they are calculable but not calculated)
% sources, fluxes and sinks.  With these assumptions, if the code is run on a
% MHD-stable case, the fields do not change.  Alternatively, the initial fields
% are self-consistently modified by the presence of unstable modes. 
% {\bf SEK: OK -- I think this is a better place to put the discussion, in
%   the end this is confusing unless there is an appendix to explain
%   things in detail.  We need to discuss whether we want to add it.  I
% think not as this is really an EHO discussion.}
% 
% There is no technical reason to make this time-scale decomposition - the NIMROD
% code has the capability to compute the extended-MHD evolution of the
% reconstructed fields. However, it is well-known that physical mechanisms
% outside the scope of the extended-MHD model mediate tokamak transport such as
% neoclassical bootstrap current, toroidal viscosity, and poloidal flow damping;
% neutral beam and RF drive; turbulence; and coupling to the scrape-off layer
% (SOL), neutrals, impurities and the material boundary. Including these effects
% requires explicit calculation of the sources, fluxes and sinks. These
% transport-type calculations are possible and are becoming practical (e.g.
% \cite{held15}), but this sort of integrated modeling remains in the future.

We begin with a study of a high resolution, lower-single-null, JT-60U-like
equilibrium (`Meudas-1'), which was originally employed in a benchmark of the
MARG2D and ELITE codes \cite{Aiba07}, including a
close approach to the X-point \cite{Snyder09}.
This extends previous benchmarks~\cite{Burke10} of ELITE and NIMROD as it
includes diverted magnetic topology and a higher edge safety factor
($q_{95}=6.74$, the safety factor at 95\% of the normalized poloidal flux) that
leads to increased resolution requirements. An ideal-MHD limit is achieved in
NIMROD by using flat density and resistivity profiles inside the last closed
flux surface (LCFS) with small resistivity, $S=10^8$ where $S$ is the Lundquist
number  ($S=\tau_R/\tau_A$), $\tau_A$ is the Alfvén time ($\tau_A=R_o/v_A$),
$v_A$ is the Alfvén velocity ($B/\sqrt{m_i n_i \mu_0}$), $\tau_R$ is the
resistive diffusion time ($\tau_R=R_o^2 \mu_0/\eta$), $R_o=2.936 m$ is the
radius of the magnetic axis, $\eta$ is the electrical resistivity, $\mu_0$ is
the permeability of free space, $m_\alpha$ is a species mass (the $\alpha$
subscript denotes ions or electrons in this work), and $n_\alpha$ is a species
density. The deuteron mass ($m_i = 3.34\times 10^{-27} kg$) is used. In order to
reproduce the vacuum response model outside the LCFS that is used by ELITE, a
low density ($0.01$ of the core density) and high resistivity ($10^7$ times the
core resistivity) is prescribed beyond the LCFS (more details on these
approximations are in Ref.~\cite{Burke10}).  

\begin{figure}
  \centering
  \includegraphics[width=8cm]{ELITEComparison}
  \vspace{-4mm}
  \caption{[Color online]
  Growth rates for the `Meudas-1' benchmark. ELITE with $\Gamma=5/3$ and
  $\Gamma=0$ are compared against results from NIMROD with $\Gamma=5/3$).
  Associated NIMROD data available in Ref.~\cite{king16Z}.}
  \label{fig:ELITEComp}
\end{figure}

\begin{figure}
  \centering
  \includegraphics[width=8cm]{idealConv}
  \vspace{-4mm}
  \caption{[Color online]
  Spectral convergence of the NIMROD code for the ideal-like parameters. 
  The maximum polynomial degree (P) of the basis functions composing the 
  spectral elements in increased in each subsequent line plotted.
  Associated NIMROD data available in Ref.~\cite{king16Z}.}
  \label{idealConv}
\end{figure}

The normalized growth rates ($\gamma \tau_A$ where the linearized mode grows as
$\text{exp}[\gamma t]$) vs.~toroidal mode number ($n_\phi$) from NIMROD and ELITE are
plotted in Fig.~\ref{fig:ELITEComp}.  There is good agreement between the
codes except for $n_\phi$=4 where there is a 27\% relative difference. All
other cases have a relative difference of less than 8\% with typical
differences of 5\%. The NIMROD convergence in terms of the maximum polynomial
order of the spectral elements is shown in Fig.~\ref{idealConv}. Convergence is
most challenging at high wavenumbers where the resolution requirements are most
stringent (the poloidal mesh is composed of $72\times512$ spectral elements).
%SEK: Great point, but total troll bait for reviewers
These cases converge from the unstable side where the growth rate decreases with
enhanced resolution. Thus the excellent agreement between NIMROD and ELITE at
high $n_\phi$ in Fig.~\ref{fig:ELITEComp} may be spurious and indicate that
slightly more resolution is required for $n_\phi$>25, however, the 
shown growth rates are likely within 5\% of their converged values.
Studying nearly ideal cases with extended MHD codes such as NIMROD is challenging 
given the vanishingly small dissipation operators, and convergence is achieved
more quickly with the additional non-ideal terms in the extended-MHD equations,
as in the cases in Sec.~\ref{sec:xMHD}.

% Relative to modeling with extended MHD, ideal-MHD convergence is more challenging 
% given the vanishingly small dissipation operators and convergence is 
% achieved more quickly with all other model equations shown in this work.

\begin{figure}
  \includegraphics[width=8cm]{meudas_n11_BR}
  \caption{[Color online]
  Poloidal cross section of the radial magnetic field component of the
  $n_\phi=11$ peeling-ballooning mode from the `Meudas-1' benchmark case. }
  \vspace{-4mm}
  \label{meudas_n11_BR}
\end{figure}

Figure \ref{meudas_n11_BR} shows a poloidal cross section of the magnetic
($B_R$) eigenmode.  The mode develops an `interference-pattern' structure near
the X-point when inboard and outboard finger-like structures overlap. The
finite-element-mesh nodes are superimposed atop the smallest-scale sub-figure.
As established by Fig.~\ref{idealConv}, this simulation is spatially
and temporally converged. The high resolution required to resolve these
high-$q_{95}$, diverted cases
motivated development of memory-scaling improvements in the NIMROD code.

\section{Experiment \& Analysis}
\label{sec:exp}
In this section, we first introduce the experimental set-up. Then, we report the performances of baselines and the proposed steep slope loss on ImageNet, followed by comprehensive analyses. 
% At last, we present comprehensive analyses to help better understand the efficacy of the proposed loss.

\noindent\textbf{Experimental Set-Up}.
We use ViT B/16 \cite{Dosovitskiy_ICLR_2021} and ResNet-50 \cite{He_CVPR_2016} as the classifiers, and the respective backbones are used as the oracles' backbones. We denote the combination of oracles and classifiers as \textlangle O, C\textrangle. There are four combinations in total, \ie \textlangle ViT, ViT\textrangle, \textlangle ViT, RSN\textrangle, \textlangle RSN, ViT\textrangle, and \textlangle RSN, RSN\textrangle, where RSN stands for ResNet.
In this work, we adopt three baselines, \ie the cross entropy loss \cite{Cox_JRSS_1972}, focal loss \cite{Lin_ICCV_2017}, and TCP confidence loss \cite{Corbiere_NIPS_2019}, for comparison purposes.

The experiment is conducted on ImageNet \cite{Deng_CVPR_2009}, which consists of 1.2 million labeled training images and 50000 labeled validation images. It has 1000 visual concepts. Similar to the learning scheme in \cite{Corbiere_NIPS_2019}, the oracle is trained with training samples and evaluated on the validation set. During the training process of the oracle, the classifier works in the evaluation mode so training the oracle would not affect the parameters of the classifier. Moreover, we conduct the analyses about how well the learned oracle generalizes to the images in the unseen domains. To this end, we apply the widely-used style transfer method \cite{Geirhos_ICLR_2019} and the functional adversarial attack method \cite{Laidlaw_NeurIPS_2019} to generate two variants of the validation set, \ie stylized validation set and adversarial validation set. \REVISION{Also, we adopt ImageNet-C \cite{Hendrycks_ICLR_2018} for evaluation, which is used for evaluating robustness to common corruptions.}
% Then, the oracle trained with regular training samples would be evaluated with the samples that are in the two unseen domains.

% To understand how the learned oracle work on unseen domains, the oracle is learned with training samples and is evaluated with three types of samples, the samples on the same domain as training samples and the samples on two unseen domains. We base our experiments on ImageNet \cite{Deng_CVPR_2009}, a widely-used large-scale dataset. Except for the training set and the validation set, we use the stylized ImageNet validation set \cite{Geirhos_ICLR_2019} and an ImageNet validation set that are perturbed by the functional adversarial attack technique \cite{Laidlaw_NeurIPS_2019}.
% (Introduce models here.)
% (Introduce hyperpaprameters here.)

The oracle's backbone is initialized by the pre-trained classifier's backbone and trained by fine-tuning using training samples and the trained classifier.
% As the oracle's backbone is initialized by the pre-trained classifier's backbone, the training process of the oracles is equivalent to the process of fine-tuning the initialized oracles.
Training the oracles with all the loss functions uses the same hyperparameters, such as learning rate, weight decay, momentum, batch size, etc.
The details for the training process and the implementation are provided in \appref{sec:implementation}.

For the focal loss, we follow \cite{Lin_ICCV_2017} to use $\gamma=2$,  which leads to the best performance for object detection.
For the proposed loss, we use $\alpha^{+}=1$ and $\alpha^{-}=3$ for the oracle that is based on ViT's backbone, while we use $\alpha^{+}=2$ and $\alpha^{-}=5$ for the oracle that is based on ResNet's backbone.

Following \cite{Corbiere_NIPS_2019}, we use FPR-95\%-TPR, AUPR-Error, AUPR-Success, and AUC as the metrics.
FPR-95\%-TPR is the false positive rate (FPR) when true positive rate (TPR) is equal to 95\%. 
AUPR is the area under the precision-recall curve. 
Specifically, AUPR-Success considers the correct prediction as the positive class, whereas AUPR-Error considers the incorrect prediction as the positive class.
AUC is the area under the receiver operating characteristic curve, which is the plot of TPR versus FPR.
Moreover, we use TPR and true negative rate (TNR) as additional metrics because they assess overfitting issue, \eg TPR=100\% and TNR=0\% imply that the trustworthiness predictor is prone to view all the incorrect predictions to be trustworthy. %due to overfitting.

% \noindent\textbf{Baselines \& Metrics}.
% We adopt widely-used loss functions, \ie cross entropy and focal loss, as the baselines. To comprehensively understand and measure oracles' performance, we use KL divergence and Bhattacharya coefficient to measure the correlation between two feature distributions, use true positive rate (TPR), true negative rate (TNR), accuracy (Acc=$(TP+TN)/Total$), F1 score, precision (P), and recall (R) to measure the efficacy of predicting trustworthiness. Specifically, we add Acc\textsubscript{P} and Acc\textsubscript{N} to understand how much TP and TN contribute to Acc. This is useful when the model overfits the data, \ie classifying all the images as either positives or negatives. Moreover, to differentiate the accuracy of classification from the accuracy of predicting trustworthiness, we denote the classifier's accuracy as C-Acc, and the oracle's accuracy as O-Acc.

% \begin{table}[!t]
	\centering
	\caption{\label{tbl:avg_perf}
	    Averaged performance over the regular ImageNet validation set, the stylized ImageNet validation set, and the adversarial ImageNet validation set. The oracle is trained with the cross entropy (CE) loss, the focal loss, and the proposed steep slope (SS) loss on the ImageNet training set. The resulting oracles w.r.t. each loss are evaluated on the three validation sets. The classifier is used in the evaluation mode in the experiment. $d_{KL}$ represents KL divergence, while $c_{B}$ represents Bhattacharyya coefficient.
	}
	\adjustbox{width=1\columnwidth}{
	\begin{tabular}{L{7ex} C{8ex} C{8ex} C{8ex} C{10ex} C{8ex} C{8ex} C{8ex} C{8ex} C{8ex} C{10ex} C{8ex}}
		\toprule
		Loss & C-Acc & $d_{KL}\uparrow$ & $c_{B}\downarrow$ & TPR & TNR & O-Acc & O-Acc\textsubscript{P} & O-Acc\textsubscript{N} & F1 & P & R  \\
		\cmidrule(lr){1-1} \cmidrule(lr){2-2} \cmidrule(lr){3-3} \cmidrule(lr){4-4} \cmidrule(lr){5-5} \cmidrule(lr){6-6} \cmidrule(lr){7-7} \cmidrule(lr){8-8} \cmidrule(lr){9-9} \cmidrule(lr){10-10} \cmidrule(lr){11-11} \cmidrule(lr){12-12}
		& \multicolumn{11}{c}{Oracle: ViT, classifier: ViT} \\
		\cmidrule(lr){2-12}
		CE & 35.74 & 0.5138 & 0.9983 & 99.98 & 0.04 & 35.78 & 35.74 & 0.04 & 0.4382 & 0.3575 & 0.8444 \\
        Focal & 35.74 & 0.5224 & 0.9972 & 99.23 & 1.30 & 36.22 & 35.43 & 0.78 & 0.4374 & 0.3579 & 0.8403 \\
        SS & 35.74 & 1.0875 & 0.9302 & 73.62 & 47.23 & 63.94 & 29.84 & 34.10 & 0.4727 & 0.4430 & 0.5964 \\ \midrule
		& \multicolumn{11}{c}{Oracle: ResNet, classifier: ViT} \\
		\cmidrule(lr){2-12}
		CE & & & & & & & & & & & \\
		Focal & & & & & & & & & & & \\
		SS & & & & & & & & & & & \\
		\bottomrule	
	\end{tabular}}
\end{table}

% \begin{figure}[!t]
% 	\centering
% 	\subfloat{\includegraphics[width=0.32\textwidth]{fig/sigmoid_imagenet_trfeat}    } \hfill
% 	\subfloat{\includegraphics[width=0.32\textwidth]{fig/focal_imagenet_trfeat}    } \hfill
% 	\subfloat{\includegraphics[width=0.32\textwidth]{fig/steep_imagenet_trfeat}    } \\
% 	\subfloat{\includegraphics[width=0.32\textwidth]{fig/sigmoid_imagenet_valfeat}    } \hfill
% 	\subfloat{\includegraphics[width=0.32\textwidth]{fig/focal_imagenet_valfeat}    } \hfill
% 	\subfloat{\includegraphics[width=0.32\textwidth]{fig/steep_imagenet_valfeat}    } \\
% 	\subfloat{\includegraphics[width=0.32\textwidth]{fig/sigmoid_imagenet_valfeat_sty}    } \hfill
% 	\subfloat{\includegraphics[width=0.32\textwidth]{fig/focal_imagenet_valfeat_sty}    } \hfill
% 	\subfloat{\includegraphics[width=0.32\textwidth]{fig/steep_imagenet_valfeat_sty}    } \\
% 	\subfloat{\includegraphics[width=0.32\textwidth]{fig/sigmoid_imagenet_valfeat_adv}    } \hfill
% 	\subfloat{\includegraphics[width=0.32\textwidth]{fig/focal_imagenet_valfeat_adv}    } \hfill
% 	\subfloat{\includegraphics[width=0.32\textwidth]{fig/steep_imagenet_valfeat_adv}    }
% 	\caption{\label{fig:distribution}
%     	Feature distributions w.r.t. the cross entropy (first column), focal (second column), and the proposed steep slope (third column) losses on the ImageNet training set (second row), ImageNet validation set (first row), stylized ImageNet validation set (third row), and adversarial ImageNet validation set (fourth row). CE stands for cross entropy, while SS stands for steep slope.
%     % 	\REVISION{\textit{Baseline} indicates ResNet GEM.}
%     	}
% \end{figure}

% \begin{table}[!t]
	\centering
	\caption{\label{tbl:perf_rsn_vit}
	    Performances on the regular ImageNet validation set, the stylized ImageNet validation set, and the adversarial ImageNet validation set. In this experiment, ResNet-50 is used for the oracle backbone while ViT is used for the classifier. The classifier is used in the evaluation mode in the experiment.
	}
	\adjustbox{width=1\columnwidth}{
	\begin{tabular}{L{8ex} C{8ex} C{8ex} C{8ex} C{10ex} C{8ex} C{8ex} C{8ex}}
		\toprule
		Loss & Acc$\uparrow$ & FPR-95\%-TPR$\downarrow$ & AURP-Error$\uparrow$ & AURP-Success$\uparrow$ & AUC$\uparrow$ & TPR$\uparrow$ & TNR$\uparrow$ \\
		\cmidrule(lr){1-1} \cmidrule(lr){2-2} \cmidrule(lr){3-3} \cmidrule(lr){4-4} \cmidrule(lr){5-5} \cmidrule(lr){6-6} \cmidrule(lr){7-7} \cmidrule(lr){8-8}
		& \multicolumn{7}{c}{Regular validation set} \\
		\cmidrule(lr){1-1} \cmidrule(lr){2-8}
		CE & 83.90 & 92.58 & 14.59 & 85.57 & 53.78 & 100.00 & 0.00 \\
		Focal & 83.90 & 94.92 & 14.87 & 85.26 & 52.49 & 100.00 & 0.00 \\
		TCP & 83.90 & 91.63 & 14.17 & 86.06 & 55.37 & 100.00 & 0.00 \\
% 		SS & 83.90 & 89.86 & 11.99 & 89.49 & 62.75 & 67.74 & 48.98 \\
		SS & 83.90 & 88.63 & 11.75 & 89.87 & 64.11 & 95.41 & 10.48 \\
% 		& 83.90 & 88.63 & 11.75 & 89.87 & 64.11 & 95.41 & 10.48 \\
%         & 83.90 & 88.72 & 11.76 & 89.85 & 64.01 & 91.10 & 18.51 \\
% 		& 83.90 & 88.25 & 11.54 & 90.24 & 65.23 & 88.23 & 24.25 \\ rsn152
		\midrule
		& \multicolumn{7}{c}{Stylized validation set} \\
		\cmidrule(lr){1-1} \cmidrule(lr){2-8}
		CE & 15.94 & 86.54 & 79.32 & 22.00 & 61.74 & 100.00 & 0.00 \\
		Focal & 15.94 & 95.04 & 85.18 & 14.94 & 48.20 & 100.00 & 0.00 \\
		TCP & 15.94 & 90.82 & 80.69 & 19.96 & 58.34 & 100.00 & 0.00 \\
		SS & 15.94 & 93.80 & 82.10 & 18.19 & 54.18 & 56.94 & 48.88 \\
% 		& 15.94 & 94.27 & 83.09 & 16.97 & 52.35 & 92.13 & 9.03 \\ 52
% 		& 15.94 & 95.76 & 84.28 & 15.74 & 49.02 & 93.83 & 5.52 \\ a62
        \midrule
		& \multicolumn{7}{c}{Adversarial validation set} \\
		\cmidrule(lr){1-1} \cmidrule(lr){2-8}
        CE & & & & & & &  \\
        Focal & & & & & & &  \\
        TCP & & & & & & & \\
        SS & & & & & & &  \\
		\bottomrule	
	\end{tabular}}
\end{table}



% \begin{figure}[!t]
	\centering
	\subfloat[\textlangle ViT, ViT\textrangle]{\includegraphics[width=0.24\textwidth]{fig/risk/risk_vit_vit}} \hfill
	\subfloat[\textlangle ViT, RSN\textrangle]{\includegraphics[width=0.24\textwidth]{fig/risk/risk_vit_rsn}} \hfill
	\subfloat[\textlangle RSN, ViT\textrangle]{\includegraphics[width=0.24\textwidth]{fig/risk/risk_rsn_vit}}
	\hfill
	\subfloat[\textlangle RSN, RSN\textrangle]{\includegraphics[width=0.24\textwidth]{fig/risk/risk_rsn_rsn}}
% 	\subfloat[O: ViT, C: ViT, Loss: TCP]{\includegraphics[width=0.24\textwidth]{fig/tsne/tsne_tcp}    } \hfill
% 	\subfloat[O: ViT, C: ViT, Loss: SS]{\includegraphics[width=0.24\textwidth]{fig/tsne/tsne_steep}    } 
    \\
	\caption{\label{fig:anal_risk}
    	Curves of risk vs. coverage. Selective risk represents the percentage of errors in the remaining validation set for a given coverage.
    	The curves correspond to the oracles used in \tabref{tbl:all_perf_w_std}.
    % 	\REVISION{\textit{Baseline} indicates ResNet GEM.}
    	}
\end{figure}

% \begin{figure}[!t]
	\centering
	\subfloat[O: ViT, C: ViT, Loss: CE]{\includegraphics[width=0.24\textwidth]{fig/tsne/tsne_ce}    } \hfill
	\subfloat[O: ViT, C: ViT, Loss: Focal]{\includegraphics[width=0.24\textwidth]{fig/tsne/tsne_focal}    } \hfill
	\subfloat[O: ViT, C: ViT, Loss: TCP]{\includegraphics[width=0.24\textwidth]{fig/tsne/tsne_tcp}    } \hfill
	\subfloat[O: ViT, C: ViT, Loss: SS]{\includegraphics[width=0.24\textwidth]{fig/tsne/tsne_steep}    } \\
	\caption{\label{fig:anal_tsne}
    	Analysis of t-SNE.
    % 	\REVISION{\textit{Baseline} indicates ResNet GEM.}
    	}
\end{figure}

% \begin{table}[!t]
% 	\centering
% 	\caption{\label{tbl:noise}
% 	    Correctness of oracle on the ImageNet validation set. The oracles are trained with the ImageNet training set. The underlined architecture indicates the architecture of Bayesian network. Leave-out rate indicates the proportion of samples that are ruled out by the oracle. Ideally, it should be equivelant to 1-Acc.
% 	}
% 	\adjustbox{width=1\columnwidth}{
% 	\begin{tabular}{L{10ex} C{12ex} C{12ex} C{9ex} C{9ex} C{9ex} C{9ex} C{9ex} C{9ex} C{9ex}}
% 		\toprule
% 		Dataset & Oracle & Classifier & Acc & O-Acc & O-TP & O-FP & F1 & Precision & Recall \\
% 		\cmidrule(lr){1-1} \cmidrule(lr){2-2} \cmidrule(lr){3-3} \cmidrule(lr){4-4} \cmidrule(lr){5-5} \cmidrule(lr){6-6} \cmidrule(lr){7-7} \cmidrule(lr){8-8} \cmidrule(lr){9-9} \cmidrule(lr){10-10}
% 		Regular & ViT-sigm            & ViT & 83.90 & 83.93 & 83.41 & 15.57 & 0.9121 & 0.8426 & 0.9941    \\
% 		Regular & ViT-Gauss            & ViT & 83.90 & 83.95 & 83.26 & 15.41 & 0.9121 & 0.8438 & 0.9924    \\
% 		Regular & ViT-exp            & ViT & 83.90 & 82.11 &  &  &  &  &     \\  \midrule
% 		Stylized & ViT-sigm            & ViT & 15.93 & 20.62 & 15.36 & 78.79 & 0.2790 & 0.1631 & 0.9639    \\
% 		Stylized & ViT-Gauss            & ViT & 15.93 & 46.28 & 13.01 & 50.79 & 0.3263 & 0.2039 & 0.8163    \\
% 		Stylized & ViT-exp            & ViT & 15.93 & 72.23 &  &  &  &  &     \\ \midrule
% 		Adv & ViT-sigm            & ViT & 7.41 & 11.23 & & & 0.1307 & 0.0762 & 0.5336    \\
% 		Adv & ViT-Gauss            & ViT & 7.41 & 11.15 & 7.14 & 88.79 & 0.1270 & 0.0744 & 0.5088 \\
% 		Adv & ViT-exp            & ViT & 7.41 & 32.57 &  &  &  &  &     \\ 
% 		\bottomrule	
% 	\end{tabular}}
% \end{table}

% \begin{table}[!t]
	\centering
	\caption{\label{tbl:all_perf}
	    Performance on the ImageNet validation set. The averaged scores are computed over three runs. The oracles are trained with the ImageNet training samples. The classifier is used in the evaluation mode in the experiment. Acc is the classification accuracy (\%) and is helpful to understand the proportion of correct predictions. \textit{SS} stands for the proposed steep slope loss.
	   % For example, Acc=83.90\% implies that 83.90\% of predictions is trustworthy and 16.10\% of predictions is untrustworthy.
	}
	\adjustbox{width=1\columnwidth}{
	\begin{tabular}{C{10ex} L{9ex} C{8ex} C{10ex} C{8ex} C{8ex} C{8ex} C{8ex} C{8ex}}
		\toprule
		\textbf{\textlangle O, C\textrangle} & \textbf{Loss} & \textbf{Acc$\uparrow$} & \textbf{FPR-95\%-TPR$\downarrow$} & \textbf{AUPR-Error$\uparrow$} & \textbf{AUPR-Success$\uparrow$} & \textbf{AUC$\uparrow$} & \textbf{TPR$\uparrow$} & \textbf{TNR$\uparrow$} \\
		\cmidrule(lr){1-1} \cmidrule(lr){2-2} \cmidrule(lr){3-3} \cmidrule(lr){4-4} \cmidrule(lr){5-5} \cmidrule(lr){6-6} \cmidrule(lr){7-7} \cmidrule(lr){8-8} \cmidrule(lr){9-9} 
		\multirow{4}{*}{\textlangle ViT, ViT \textrangle} & CE & 83.90 & 93.01 & \textbf{15.80} & 84.25 & 51.62 & \textbf{99.99} & 0.02 \\
		 & Focal \cite{Lin_ICCV_2017} & 83.90 & 93.37 & 15.31 & 84.76 & 52.38 & 99.15 & 1.35 \\
		 & TCP \cite{Corbiere_NIPS_2019} & 83.90 & 88.38 & 12.96 & 87.63 & 60.14 & 99.73 & 0.00 \\
		 & SS & 83.90 & \textbf{80.48} & 10.26 & \textbf{93.01} & \textbf{73.68} & 87.52 & \textbf{38.27} \\
		\midrule
		\multirow{4}{*}{\textlangle ViT, RSN\textrangle} & CE & 68.72 & 93.43 & 30.90 & 69.13 & 51.24 & \textbf{99.90} & 0.20 \\
		 & Focal \cite{Lin_ICCV_2017} & 68.72 & 93.94 & \textbf{30.97} & 69.07 & 51.26 & 93.66 & 7.71 \\
		 & TCP \cite{Corbiere_NIPS_2019} & 68.72 & 83.55 & 23.56 & 79.04 & 66.23 & 94.25 & 0.00 \\
		 & SS & 68.72 & \textbf{77.89} & 20.91 & \textbf{85.39} & \textbf{74.31} & 68.32 & \textbf{67.53} \\
        \midrule
		\multirow{4}{*}{\textlangle RSN, ViT\textrangle} & CE & 83.90 & 93.29 & 14.74 & 85.40 & 53.43 & \textbf{100.00} & 0.00 \\
		 & Focal \cite{Lin_ICCV_2017} & 83.90 & 94.60 & \textbf{14.98} & 85.13 & 52.37 & \textbf{100.00} & 0.00 \\
		 & TCP \cite{Corbiere_NIPS_2019} & 83.90 & 91.93 & 14.12 & 86.12 & 55.55 & \textbf{100.00} & 0.00 \\
         & SS & 83.90 & \textbf{88.70} & 11.69 & \textbf{90.01} & \textbf{64.34} & 96.20 & \textbf{9.00} \\
% 		RSN & ViT & SS & 83.90 & 89.86 & 11.99 & 89.49 & 62.75 & 67.74 & 48.98 \\
        \midrule
        \multirow{4}{*}{\textlangle RSN, RSN\textrangle} & CE & 68.72 & 94.84 & 29.41 & 70.79 & 52.36 & \textbf{100.00} & 0.00 \\
		 & Focal \cite{Lin_ICCV_2017} & 68.72 & 95.16 & \textbf{29.92} & 70.23 & 51.43 & 99.86 & 0.08 \\
		 & TCP \cite{Corbiere_NIPS_2019} & 68.72 & 88.81 & 24.46 & 77.79 & 62.73 & 99.23 & 0.00 \\
         & SS & 68.72 & \textbf{86.21} & 22.53 & \textbf{81.88} & \textbf{67.92} & 79.20 & \textbf{42.09} \\
		\bottomrule	
	\end{tabular}}
\end{table}
\begin{table}[!t]
	\centering
	\caption{\label{tbl:all_perf_w_std}
	    Performance on the ImageNet validation set. The mean and the standard deviation of scores are computed over three runs. The oracles are trained with the ImageNet training samples. The classifier is used in the evaluation mode. Acc is the classification accuracy and is helpful to understand the proportion of correct predictions. \textit{SS} stands for the proposed steep slope loss.
	   % For example, Acc=83.90\% implies that 83.90\% of predictions is trustworthy and 16.10\% of predictions is untrustworthy.
	}
	\adjustbox{width=1\columnwidth}{
	\begin{tabular}{C{10ex} L{10ex} C{8ex} C{10ex} C{10ex} C{10ex} C{10ex} C{10ex} C{10ex}}
		\toprule
		\textbf{\textlangle O, C\textrangle} & \textbf{Loss} & \textbf{Acc$\uparrow$} & \textbf{FPR-95\%-TPR$\downarrow$} & \textbf{AUPR-Error$\uparrow$} & \textbf{AUPR-Success$\uparrow$} & \textbf{AUC$\uparrow$} & \textbf{TPR$\uparrow$} & \textbf{TNR$\uparrow$} \\
		\cmidrule(lr){1-1} \cmidrule(lr){2-2} \cmidrule(lr){3-3} \cmidrule(lr){4-4} \cmidrule(lr){5-5} \cmidrule(lr){6-6} \cmidrule(lr){7-7} \cmidrule(lr){8-8} \cmidrule(lr){9-9} 
		\multirow{4}{*}{\textlangle ViT, ViT \textrangle} & CE & 83.90 & 93.01$\pm$0.17 & \textbf{15.80}$\pm$0.56 & 84.25$\pm$0.57 & 51.62$\pm$0.86 & \textbf{99.99}$\pm$0.01 & 0.02$\pm$0.02 \\
		 & Focal \cite{Lin_ICCV_2017} & 83.90 & 93.37$\pm$0.52 & 15.31$\pm$0.44 & 84.76$\pm$0.50 & 52.38$\pm$0.77 & 99.15$\pm$0.14 & 1.35$\pm$0.22 \\
		 & TCP \cite{Corbiere_NIPS_2019} & 83.90 & 88.38$\pm$0.23 & 12.96$\pm$0.10 & 87.63$\pm$0.15 & 60.14$\pm$0.47 & 99.73$\pm$0.02 & 0.00$\pm$0.00 \\
		 & SS & 83.90 & \textbf{80.48}$\pm$0.66 & 10.26$\pm$0.03 & \textbf{93.01}$\pm$0.10 & \textbf{73.68}$\pm$0.27 & 87.52$\pm$0.95 & \textbf{38.27}$\pm$2.48 \\
		\midrule
		\multirow{4}{*}{\textlangle ViT, RSN\textrangle} & CE & 68.72 & 93.43$\pm$0.28 & 30.90$\pm$0.35 & 69.13$\pm$0.36 & 51.24$\pm$0.63 & \textbf{99.90}$\pm$0.04 & 0.20$\pm$0.00 \\
		 & Focal \cite{Lin_ICCV_2017} & 68.72 & 93.94$\pm$0.51 & \textbf{30.97}$\pm$0.36 & 69.07$\pm$0.35 & 51.26$\pm$0.62 & 93.66$\pm$0.29 & 7.71$\pm$0.53 \\
		 & TCP \cite{Corbiere_NIPS_2019} & 68.72 & 83.55$\pm$0.70 & 23.56$\pm$0.47 & 79.04$\pm$0.91 & 66.23$\pm$1.02 & 94.25$\pm$0.96 & 0.00$\pm$0.00 \\
		 & SS & 68.72 & \textbf{77.89}$\pm$0.39 & 20.91$\pm$0.05 & \textbf{85.39}$\pm$0.16 & \textbf{74.31}$\pm$0.21 & 68.32$\pm$0.41 & \textbf{67.53}$\pm$0.62 \\
        \midrule
		\multirow{4}{*}{\textlangle RSN, ViT\textrangle} & CE & 83.90 & 93.29$\pm$0.53 & 14.74$\pm$0.17 & 85.40$\pm$0.20 & 53.43$\pm$0.28 & \textbf{100.00}$\pm$0.00 & 0.00$\pm$0.00 \\
		 & Focal \cite{Lin_ICCV_2017} & 83.90 & 94.60$\pm$0.53 & \textbf{14.98}$\pm$0.21 & 85.13$\pm$0.24 & 52.37$\pm$0.51 & \textbf{100.00}$\pm$0.00 & 0.00$\pm$0.00 \\
		 & TCP \cite{Corbiere_NIPS_2019} & 83.90 &91.93$\pm$0.49 & 14.12$\pm$0.12 & 86.12$\pm$0.15 & 55.55$\pm$0.46 & \textbf{100.00}$\pm$0.00 & 0.00$\pm$0.00 \\
         & SS & 83.90 & \textbf{88.70}$\pm$0.08 & 11.69$\pm$0.04 & \textbf{90.01}$\pm$0.10 & \textbf{64.34}$\pm$0.16 & 96.20$\pm$0.73 & \textbf{9.00}$\pm$1.32 \\
% 		RSN & ViT & SS & 83.90 & 89.86 & 11.99 & 89.49 & 62.75 & 67.74 & 48.98 \\
        \midrule
        \multirow{4}{*}{\textlangle RSN, RSN\textrangle} & CE & 68.72 & 94.84$\pm$0.27 & 29.41$\pm$0.18 & 70.79$\pm$0.19 & 52.36$\pm$0.41 & \textbf{100.00}$\pm$0.00 & 0.00$\pm$0.00 \\
		 & Focal \cite{Lin_ICCV_2017} & 68.72 & 95.16$\pm$0.19 & \textbf{29.92}$\pm$0.38 & 70.23$\pm$0.44 & 51.43$\pm$0.50 & 99.86$\pm$0.05 & 0.08$\pm$0.03 \\
		 & TCP \cite{Corbiere_NIPS_2019} & 68.72 & 88.81$\pm$0.24 & 24.46$\pm$0.12 & 77.79$\pm$0.29 & 62.73$\pm$0.14 & 99.23$\pm$0.14 & 0.00$\pm$0.00 \\
         & SS & 68.72 & \textbf{86.21}$\pm$0.44 & 22.53$\pm$0.03 & \textbf{81.88}$\pm$0.10 & \textbf{67.92}$\pm$0.11 & 79.20$\pm$2.50 & \textbf{42.09}$\pm$3.77 \\
		\bottomrule	
	\end{tabular}}
\end{table}

\noindent\textbf{Performance on Large-Scale Dataset}. 
The result on ImageNet are reported in \tabref{tbl:all_perf_w_std}. We have two key observations. Firstly, training with the cross entropy loss, focal loss, and TCP confidence loss lead to overfitting the imbalanced training samples, \ie the dominance of trustworthy predictions. Specifically, TPR is higher than 99\% while TNR is less than 1\% in all cases. Secondly, the performance of predicting trustworthiness is contingent on both the oracle and the classifier. When a high-performance model (\ie ViT) is used as the oracle and a relatively low-performance model (\ie ResNet) is used as the classifier, cross entropy loss and focal loss achieve higher TNRs than the loss functions with the other combinations. In contrast, the two losses with \textlangle ResNet, ViT\textrangle~ lead to the lowest TNRs (\ie 0\%). %, compared to the cases with the other combinations.

Despite the combinations of oracles and classifiers, the proposed steep slope loss can achieve significantly higher TNRs than using the other loss functions, while it achieves desirable performance on FPR-95\%-TPR, AUPR-Success, and AUC. This verifies that the proposed loss is effective to improve the generalizability for predicting trustworthiness. Note that the scores of AUPR-Error and TPR yielded by the proposed loss are lower than that of the other loss functions. Recall that AUPR-Error aims to inspect how easy to detect failures and depends on the negated trustworthiness confidences w.r.t. incorrect predictions \cite{Corbiere_NIPS_2019}. The AUPR-Error correlates to TPR and TNR. When TPR is close to 100\% and TNR is close to 0\%, it indicates the oracle is prone to view all the predictions to be trustworthy. In other words, almost all the trustworthiness confidences are on the right-hand side of $p(o=1|\theta,\bm{x})=0.5$. Consequently, when taking the incorrect prediction as the positive class, the negated confidences are smaller than -0.5. On the other hand, the oracle trained with the proposed loss intends to yield the ones w.r.t. incorrect predictions that are smaller than 0.5. In general, the negated confidences w.r.t. incorrect predictions are greater than the ones yielded by the other loss functions. In summary, a high TPR score and a low TNR score leads to a high AUPR-Error.

To intuitively understand the effects of all the loss functions, we plot the histograms of trustworthiness confidences w.r.t. true positive (TP), false positive (FP), true negative (TN), and false negative (FN) in \figref{fig:histogram_part}. The result confirms that the oracles trained with the baseline loss functions are prone to predict overconfident trustworthiness for incorrect predictions, while the oracles trained with the proposed loss can properly predict trustworthiness for incorrect predictions.

% On the other hand, the proposed steep slope loss show better generalizability over the three domains, where TPR is 73.62\% and TNR is 47.23\%. Secondly, the learned oracles exhibit consistent separability over the three domains through the lens of KL divergence and Bhttacharya coefficient. This is aligned with the intuition that a model that work well on a domain is likely to work well on other domains. 

\begin{figure}[!t]
	\centering
	\subfloat[\textlangle ViT, ViT\textrangle + CE]{\includegraphics[width=0.24\textwidth]{fig/hist/ce_vit_vit_val}    } \hfill
	\subfloat[\textlangle ViT, ViT\textrangle + Focal]{\includegraphics[width=0.24\textwidth]{fig/hist/focal_vit_vit_val}    } \hfill
	\subfloat[\textlangle ViT, ViT\textrangle + TCP]{\includegraphics[width=0.24\textwidth]{fig/hist/tcp_vit_vit_val}    } \hfill
	\subfloat[\textlangle ViT, ViT\textrangle +  SS]{\includegraphics[width=0.24\textwidth]{fig/hist/ss_vit_vit_val}    } \\
	\subfloat[\textlangle ViT, RSN\textrangle + CE]{\includegraphics[width=0.24\textwidth]{fig/hist/ce_vit_rsn_val}    } \hfill
	\subfloat[\textlangle ViT, RSN\textrangle + Focal]{\includegraphics[width=0.24\textwidth]{fig/hist/focal_vit_rsn_val}    } \hfill
	\subfloat[\textlangle ViT, RSN\textrangle + TCP]{\includegraphics[width=0.24\textwidth]{fig/hist/tcp_vit_rsn_val}    } \hfill
	\subfloat[\textlangle ViT, RSN\textrangle + SS]{\includegraphics[width=0.24\textwidth]{fig/hist/ss_vit_rsn_val}    } \\
% 	\subfloat[\textlangle RSN, ViT\textrangle + CE]{\includegraphics[width=0.24\textwidth]{fig/hist/ce_rsn_vit_val}    } \hfill
% 	\subfloat[\textlangle RSN, ViT\textrangle + Focal]{\includegraphics[width=0.24\textwidth]{fig/hist/focal_rsn_vit_val}    } \hfill
% 	\subfloat[\textlangle RSN, ViT\textrangle + TCP]{\includegraphics[width=0.24\textwidth]{fig/hist/tcp_rsn_vit_val}    } \hfill
% 	\subfloat[\textlangle RSN, ViT\textrangle + SS]{\includegraphics[width=0.24\textwidth]{fig/hist/ss_rsn_vit_val}    } \\
% 	\subfloat[\textlangle RSN, RSN\textrangle + CE]{\includegraphics[width=0.24\textwidth]{fig/hist/ce_rsn_rsn_val}    } \hfill
% 	\subfloat[\textlangle RSN, RSN\textrangle + Focal]{\includegraphics[width=0.24\textwidth]{fig/hist/focal_rsn_rsn_val}    } \hfill
% 	\subfloat[\textlangle RSN, RSN\textrangle + TCP]{\includegraphics[width=0.24\textwidth]{fig/hist/tcp_rsn_rsn_val}    } \hfill
% 	\subfloat[\textlangle RSN, RSN\textrangle + SS]{\includegraphics[width=0.24\textwidth]{fig/hist/ss_rsn_rsn_val}    } \\
	\caption{\label{fig:histogram_part}
    	Histograms of trustworthiness confidences w.r.t. all the loss functions on the ImageNet validation set.
    	The oracles that are used to generate the confidences are the ones used in \tabref{tbl:all_perf_w_std}. The histograms generated with \textlangle RSN, ViT\textrangle and \textlangle RSN, RSN\textrangle are provided in \appref{sec:histogram}.
    % 	appendix \ref{sec:histogram}.
    % 	the cross entropy (first column), focal loss (second column), TCP confidence loss (third column), and the proposed steep slope loss (fourth column) on the ImageNet validation set.
    % 	\REVISION{\textit{Baseline} indicates ResNet GEM.}
    	}
    \vspace{-1ex}
\end{figure}

% \begin{wrapfigure}{r}{0.5\textwidth}
\begin{table}[!t]
	\centering
	\caption{\label{tbl:perf_mnist}
	    Performance on MNIST and CIFAR-10.
	   % We use the official TCP code, but find out that there are several bugs and we couldn't reproduce the performance reported in their paper, not even close. Below are the best results by fixing a few bugs, according to the technical details in the paper.
	}
	\adjustbox{width=1\columnwidth}{
	\begin{tabular}{C{12ex} L{15ex} C{8ex} C{10ex} C{8ex} C{8ex} C{8ex} C{8ex} C{8ex}}
		\toprule
		\textbf{Dataset} & \textbf{Loss} & \textbf{Acc$\uparrow$} & \textbf{FPR-95\%-TPR$\downarrow$} & \textbf{AUPR-Error$\uparrow$} & \textbf{AUPR-Success$\uparrow$} & \textbf{AUC$\uparrow$} & \textbf{TPR$\uparrow$} & \textbf{TNR$\uparrow$} \\
		\cmidrule(lr){1-1} \cmidrule(lr){2-2} \cmidrule(lr){3-3} \cmidrule(lr){4-4} \cmidrule(lr){5-5} \cmidrule(lr){6-6} \cmidrule(lr){7-7} \cmidrule(lr){8-8} \cmidrule(lr){9-9}
		\multirow{6}{*}{MNIST} & MCP \cite{Hendrycks_ICLR_2017} & 99.10 & 5.56 & 35.05 & \textbf{99.99} & 98.63 & 99.89 & \textbf{8.89} \\
		& MCDropout \cite{Gal_ICML_2016} & 99.10 & 5.26 & 38.50 & \textbf{99.99} & 98.65 & - & - \\
		& TrustScore \cite{Jiang_NIPS_2018} & 99.10 & 10.00 & 35.88 & 99.98 & 98.20 & - & - \\
		& TCP \cite{Corbiere_NIPS_2019} & 99.10 & 3.33 & \textbf{45.89} & \textbf{99.99} & 98.82 & 99.71 & 0.00 \\
		& TCP$\dagger$ & 99.10 & 3.33 & 45.88 & \textbf{99.99} & 98.82 & 99.72 & 0.00 \\
		& SS & 99.10 & \textbf{2.22} & 40.86 & \textbf{99.99} & \textbf{98.83} & \textbf{100.00} & 0.00 \\
		\midrule
		\multirow{6}{*}{CIFAR-10} & MCP \cite{Hendrycks_ICLR_2017} & 92.19 & 47.50 & 45.36 & 99.19 & 91.53 & 99.64 & 6.66 \\
		& MCDropout \cite{Gal_ICML_2016} & 92.19 & 49.02 & 46.40 & \textbf{99.27} & 92.08 & - & - \\
		& TrustScore \cite{Jiang_NIPS_2018} & 92.19 & 55.70 & 38.10 & 98.76 & 88.47 & - & - \\
		& TCP \cite{Corbiere_NIPS_2019} & 92.19 & 44.94 & 49.94 & 99.24 & 92.12 & \textbf{99.77} & 0.00 \\
		& TCP$\dagger$ & 92.19 & 45.07 & 49.89 & 99.24 & 92.12 & 97.88 & 0.00 \\
		& SS & 92.19 & \textbf{44.69 }& \textbf{50.28}  & 99.26 & \textbf{92.22} & 98.46 & \textbf{28.04} \\
		\bottomrule	
	\end{tabular}}
\end{table}
% \end{wrapfigure}

% \begin{figure}[!t]
% 	\centering
% 	\subfloat[Official TCP  plot]{\includegraphics[width=0.45\textwidth]{fig/hist/tcphp_mnist_tefeat}    } \hfill
% 	\subfloat[Proposed with pretrained baseline ]{\includegraphics[width=0.45\textwidth]{fig/hist/steephp_mnist_tefeat}    } \\
% 	\subfloat[TCP with trained baseline]{\includegraphics[width=0.45\textwidth]{fig/hist/tcplp_mnist_tefeat}    } \hfill
% 	\subfloat[Proposed with trained baseline ]{\includegraphics[width=0.45\textwidth]{fig/hist/steeplp_mnist_tefeat}    }
% 	\caption{
%     	Reproduction and comparison.
%     	}
% \end{figure}

\noindent\textbf{Separability between Distributions of Correct Predictions and Incorrect Predictions}.
As observed in \figref{fig:histogram_part}, the confidences w.r.t. correct and incorrect predictions follow Gaussian-like distributions.
Hence, we can compute the separability between the distributions of correct and incorrect predictions from a probabilistic perspective.
% There are two common tools to achieve the goal, \ie Kullback–Leibler (KL) divergence \cite{Kullback_AMS_1951} and Bhattacharyya distance \cite{Bhattacharyya_JSTOR_1946}.
Given the distribution of correct predictions {\small $\mathcal{N}_{1}(\mu_{1}, \sigma^{2}_{1})$} and the distribution of correct predictions {\small $\mathcal{N}_{2}(\mu_{2}, \sigma^{2}_{2})$}, we use the average Kullback–Leibler (KL) divergence {\small $\bar{d}_{KL}(\mathcal{N}_{1}, \mathcal{N}_{2})$} \cite{Kullback_AMS_1951} and Bhattacharyya distance {\small $d_{B}(\mathcal{N}_{1}, \mathcal{N}_{2})$} \cite{Bhattacharyya_JSTOR_1946} to measure the separability. 
More details and the quantitative results are reported in \appref{sec:separability}. 
In short, the proposed loss leads to larger separability than the baseline loss functions. 
This implies that the proposed loss is more effective to differentiate incorrect predictions from correct predictions.

\noindent\textbf{Performance on Small-Scale Datasets}.
We also provide comparative experimental results on small-scale datasets, \ie MNIST \cite{Lecun_IEEE_1998} and CIFAR-10 \cite{Krizhevsky_TR_2009}.
\REVISION{The results are reported in \tabref{tbl:perf_mnist}.}
% The experiment details and results are reported in \appref{sec:mnist}.
The proposed loss outperforms TCP$\dagger$ on metric FPR-95\%-TPR on both MNIST and CIFAR-10, and additionally achieved good performance on metrics AUPR-Error and TNR on CIFAR-10.
This shows the proposed loss is able to adapt to relatively simple data.
\REVISION{More details can be found in \appref{sec:mnist}.}

\noindent\textbf{Generalization to Unseen Domains}.
In practice, the oracle may run into the data in the domains that are different from the ones of training samples.
Thus, it is interesting to find out how well the learned oracles generalize to the unseen domain data.
% To this end, we apply a style transfer method \cite{Geirhos_ICLR_2019} and the functional adversarial attack method \cite{Laidlaw_NeurIPS_2019} to generate the stylized ImageNet validation set and the adversarial ImageNet validation set.
Using the oracles trained with the ImageNet training set (\ie the ones used in \tabref{tbl:all_perf_w_std}), we evaluate it on the stylized ImageNet validation set \cite{Geirhos_ICLR_2019}, adversarial ImageNet validation set \cite{Laidlaw_NeurIPS_2019}, and corrupted ImageNet validation set \cite{Hendrycks_ICLR_2018}.
% and evaluated on the two variants of the validation set.
\textlangle ViT, ViT\textrangle~ is used in the experiment.

The results on the stylized ImageNet, adversarial ImageNet, and ImageNet-C are reported in \tabref{tbl:perf_vit_vit}, \REVISION{More results on ImageNet-C are reported in \tabref{tbl:perf_imagenetc}}.
As all unseen domains are different from the one of the training set, the classification accuracies are much lower than the ones in \tabref{tbl:all_perf_w_std}. 
The adversarial validation set is also more challenging than the stylized validation set \REVISION{and the corrupted validation set}.
As a result, the difficulty affects the scores across all metrics.
The oracles trained with the baseline loss functions are still prone to recognize the incorrect prediction to be trustworthy.
The proposed loss consistently improves the performance on FPR-95\%-TPR, AUPR-Sucess, AUC, and TNR.
Note that the adversarial perturbations are computed on the fly \cite{Laidlaw_NeurIPS_2019}. Instead of truncating the sensitive pixel values and saving into the images files, we follow the experimental settings in \cite{Laidlaw_NeurIPS_2019} to evaluate the oracles on the fly.
Hence, the classification accuracies w.r.t. various loss function are slightly different but are stably around 6.14\%.

% Also, we report the performances on each domain in \tabref{tbl:perf_vit_vit} and \tabref{tbl:perf_rsn_vit}.
% They shows that the cross entropy and focal loss work well on the regular validation set, but work poorly on the stylized and adversarial validation sets. This confirms the overfitting resulted from the learning with the cross entropy and focal loss.

\begin{table}[!t]
	\centering
	\vspace{-1ex}
	\caption{\label{tbl:perf_vit_vit}
	   % Histograms of trustworthiness confidences w.r.t. all the loss functions on the stylized ImageNet validation set (stylized val) and the adversarial ImageNet validation set (adversarial val). \textlangle ViT, ViT\textrangle is used in the experiment and the domains of the two validation sets are different from the one of the training set that is used for training the oracle.
	    Performance on the stylized ImageNet validation set, the adversarial ImageNet validation set, and one (Defocus blur) of validation sets in ImageNet-C. Defocus blus is at at the highest level of severity.
	    \textlangle ViT, ViT\textrangle~ is used in the experiment and the domains of the two validation sets are different from the one of the training set that is used for training the oracle. The corresponding histograms are available in \appref{sec:histogram}. More results on ImageNet-C can be found in \tabref{tbl:perf_imagenetc}.
	   % In this experiment, ViT is used for both the oracle backbone and the classifier. The oracle is trained with the CE loss, the focal loss, and the proposed steep slope loss on the ImageNet training set. The resulting oracles w.r.t. each loss are evaluated on the three validation sets. The classifier is used in the evaluation mode in the experiment.
	}
	\adjustbox{width=1\columnwidth}{
	\begin{tabular}{C{15ex} L{10ex} C{8ex} C{10ex} C{8ex} C{8ex} C{8ex} C{8ex} C{8ex}}
		\toprule
		\textbf{Dataset} & \textbf{Loss} & \textbf{Acc$\uparrow$} & \textbf{FPR-95\%-TPR$\downarrow$} & \textbf{AUPR-Error$\uparrow$} & \textbf{AUPR-Success$\uparrow$} & \textbf{AUC$\uparrow$} & \textbf{TPR$\uparrow$} & \textbf{TNR$\uparrow$} \\
		\cmidrule(lr){1-1} \cmidrule(lr){2-2} \cmidrule(lr){3-3} \cmidrule(lr){4-4} \cmidrule(lr){5-5} \cmidrule(lr){6-6} \cmidrule(lr){7-7} \cmidrule(lr){8-8} \cmidrule(lr){9-9}
% 		& \multicolumn{7}{c}{Regular validation set} \\
% 		\cmidrule(lr){1-1} \cmidrule(lr){2-8}
% 		CE & 83.90 & 92.83 & 15.08 & 84.99 & 52.78 & 100.00 & 0.01 \\
% 		Focal & 83.90 & 92.68 & 14.69 & 85.46 & 53.47 & 99.06 & 1.61 \\
% 		TCP & 83.90 & 88.07 & 12.86 & 87.80 & 60.45 & 99.72 & 1.02 \\
% % 		TCP & 83.90 & 86.45 & 12.12 & 88.95 & 63.39 & 99.07 & 3.06 \\
% 		SS & 83.90 & 80.89 & 10.31 & 92.90 & 73.31 & 88.44 & 35.64 \\
% 		\midrule
% 		& \multicolumn{7}{c}{Stylized validation set} \\
% 		\cmidrule(lr){1-1} \cmidrule(lr){2-8}
		\multirow{4}{*}{Stylized \cite{Geirhos_ICLR_2019}} & CE & 15.94 & 95.52 & 84.18 & 15.86 & 49.07 & \textbf{99.99} & 0.02 \\
		& Focal \cite{Lin_ICCV_2017} & 15.94 & 95.96 & \textbf{85.90} & 14.30 & 46.01 & 99.71 & 0.25 \\
		& TCP \cite{Corbiere_NIPS_2019} & 15.94 & 93.42 & 80.17 & 21.25 & 57.29 & 99.27 & 0.00 \\
% 		& TCP & 15.94 & 93.19 & 78.53 & 24.52 & 60.31 & 95.41 & 6.24 \\
		& SS & 15.94 & \textbf{89.38} & 75.08 & \textbf{34.39} & \textbf{67.68} & 44.42 & \textbf{81.22} \\
        \midrule
% 		& \multicolumn{7}{c}{Adversarial validation set} \\
% 		\cmidrule(lr){1-1} \cmidrule(lr){2-8}
        \multirow{4}{*}{Adversarial \cite{Laidlaw_NeurIPS_2019}} & CE & 6.14 & 94.35 & \textbf{93.70} & 6.32 & 51.28 & \textbf{99.97} & 0.06 \\
        & Focal \cite{Lin_ICCV_2017} & 6.15 & 93.67 & 93.48 & 6.56 & 52.39 & 99.06 & 1.43 \\
        & TCP \cite{Corbiere_NIPS_2019} & 6.11 & 93.94 & 92.77 & 7.55 & 55.81 & 99.71 & 0.00 \\
        & SS  & 6.16 & \textbf{90.07} & 90.09 & \textbf{13.07} & \textbf{65.36} & 87.10 & \textbf{24.33} \\ \midrule
        \multirow{4}{*}{Defocus blur \cite{Hendrycks_ICLR_2018}} & CE & 31.83 & 94.46 & \textbf{68.56} & 31.47 & 50.13 & \textbf{99.15} & 1.07 \\
		& Focal \cite{Lin_ICCV_2017} & 31.83 & 94.98  & 66.87 & 33.24 & 51.28 & 96.70 & 3.26 \\
		& TCP \cite{Corbiere_NIPS_2019} & 31.83 & 93.50 & 64.67 & 36.05 & 54.27 & 96.71 & 4.35 \\
		& SS & 31.83 & \textbf{90.18} & 57.95 & \textbf{48.80} & \textbf{64.34} & 77.79 & \textbf{37.29} \\
		\bottomrule	
	\end{tabular}}
\end{table}

\begin{figure}[!b]
	\centering
	\subfloat[]{\includegraphics[width=0.32\textwidth]{fig/risk/risk_vit_vit} \label{fig:risk_vit}} \hfill
	\subfloat[]{\includegraphics[width=0.30\textwidth]{fig/analysis/loss} \label{fig:abl_loss}} \hfill
	\subfloat[]{\includegraphics[width=0.32\textwidth]{fig/analysis/tpr_tnr} \label{fig:abl_tpr_tnr}} 
	\caption{\label{fig:anal_abl}
    	Analyses based on \textlangle ViT, ViT\textrangle. (a) are the curves of risk vs. coverage. Selective risk represents the percentage of errors in the remaining validation set for a given coverage. (b) are the curves of loss vs. $\alpha^{-}$. (c) are TPR and TNR against various $\alpha^{-}$.
    	}
\end{figure}

\noindent\textbf{Selective Risk Analysis}.
Risk-coverage curve is an important technique for analyzing trustworthiness through the lens of the rejection mechanism in the classification task \cite{Corbiere_NIPS_2019,Geifman_NIPS_2017}. 
In the context of predicting trustworthiness, selective risk is the empirical loss that takes into account the decisions, \ie to trust or not to trust the prediction. 
Correspondingly, coverage is the probability mass of the non-rejected region. As can see in \figref{fig:risk_vit}, the proposed loss leads to significantly lower risks, compared to the other loss functions.
We present the risk-coverage curves w.r.t. all the combinations of oracles and classifiers in \appref{sec:risk}.
They consistently exhibit similar pattern.

\noindent\textbf{Ablation Study}.
In contrast to the compared loss functions, the proposed loss has more hyperparameters to be determined, \ie $\alpha^{+}$ and $\alpha^{-}$.
As the proportion of correct predictions is usually larger than that of incorrect predictions, we would prioritize $\alpha^{-}$ over $\alpha^{+}$ such that the oracle is able to recognize a certain amount of incorrect predictions.
In other words, we first search for $\alpha^{-}$ by freezing $\alpha^{+}$, and then freeze $\alpha^{-}$ and search for $\alpha^{+}$.
\figref{fig:abl_loss} and \ref{fig:abl_tpr_tnr} show how the loss, TPR, and TNR vary with various $\alpha^{-}$. In this analysis, the combination \textlangle ViT, ViT\textrangle~ is used and $\alpha^{+}=1$.
We can see that $\alpha^{-}=3$ achieves the optimal trade-off between TPR and TNR.
We follow a similar search strategy to determine $\alpha^{+}=2$ and $\alpha^{-}=5$ for training the oracle with ResNet backbone.
% With the classifier ViT and the ViT based oracle, we show how the performance vary when $\alpha^{+}$ and $\alpha^{-}$ change.  

\noindent\textbf{Effects of Using $z=\bm{w}^{\top}\bm{x}^{out}+b$}.
Using the signed distance as $z$, \ie $z = \frac{\bm{w}^{\top} \bm{x}^{out}+b}{\|\bm{w}\|}$, has a geometric interpretation as shown in \figref{fig:workflow_a}.
However, the main-stream models \cite{He_CVPR_2016,Tan_ICML_2019,Dosovitskiy_ICLR_2021} use $z=\bm{w}^{\top}\bm{x}^{out}+b$. 
Therefore, we provide the corresponding results in appendix \ref{sec:appd_z}, which are generated by the proposed loss taking the output of the linear function as input.
In comparison with the results of using $z = \frac{\bm{w}^{\top} \bm{x}^{out}+b}{\|\bm{w}\|}$, using $z=\bm{w}^{\top}\bm{x}^{out}+b$ yields comparable scores of FPR-95\%-TPR, AUPR-Error, AUPR-Success, and AUC.
Also, TPR and TNR are moderately different between $z = \frac{\bm{w}^{\top} \bm{x}^{out}+b}{\|\bm{w}\|}$ and $z=\bm{w}^{\top}\bm{x}^{out}+b$, when $\alpha^{+}$ and $\alpha^{-}$ are fixed.
This implies that TPR and TNR are sensitive to $\|\bm{w}\|$. 
% \REVISION{We discuss the reason in \appref{sec:effect_normalization}.}
% 
\REVISION{
This is because the normalization by $\|w\|$ would make $z$ more dispersed in value than the variant without normalization. 
In other words, the normalization leads to long-tailed distributions while no normalization leads to short-tailed distributions. 
Given the same threshold, TNR (TPR) is determined by the location of the distribution of negative (positive) examples and the extent of short/long tails. 
Our analysis on the histograms generated with and without $\|w\|$ normalization verifies this point.
}

% \noindent\textbf{Learning with Class Weights}. We witness the imbalancing characteristics in the learning task for predicting trustworthiness. Table xx shows that one of most common learning techniques with imbalanced data, \ie using class weights, is not effective. The reason is that applying class weights to the loss function, \eg cross entropy, it only scale up the graph along y-axis. However, the long tail regions still slow down the move of the features w.r.t. false positive or false negative towards the well-classified regions.

% \noindent\textbf{Separability between Distributions of Correct Predictions and Incorrect Predictions}.
% As observed in \figref{fig:histogram_part}, the confidences w.r.t. correct and incorrect predictions follow Gaussian-like distributions.
% Hence, we can compute the separability between the distributions of correct and incorrect predictions from a probabilistic perspective.
% % There are two common tools to achieve the goal, \ie Kullback–Leibler (KL) divergence \cite{Kullback_AMS_1951} and Bhattacharyya distance \cite{Bhattacharyya_JSTOR_1946}.
% Given the distribution of correct predictions $\mathcal{N}_{1}(\mu_{1}, \sigma^{2}_{1})$ and the distribution of correct predictions $\mathcal{N}_{2}(\mu_{2}, \sigma^{2}_{2})$, we use the average Kullback–Leibler (KL) divergence $\bar{d}_{KL}(\mathcal{N}_{1}, \mathcal{N}_{2})$ \cite{Kullback_AMS_1951} and Bhattacharyya distance $d_{B}(\mathcal{N}_{1}, \mathcal{N}_{2})$ \cite{Bhattacharyya_JSTOR_1946} to measure the separability. More details and the quantitative results are reported in \appref{sec:separability}. In short, the proposed loss leads to larger separability than the baseline loss functions. This implies that the proposed loss is more effective to differentiate incorrect predictions from correct predictions.

\noindent\textbf{Steep Slope Loss vs. Class-Balanced Loss}.
We compare the proposed loss to the class-balanced loss \cite{Cui_CVPR_2019}, which is based on a re-weighting strategy.
The results are reported in \appref{sec:cbloss}.
Overall, the proposed loss outperforms the class-balanced loss, which implies that the imbalance characteristics of predicting trustworthiness is different from that of imbalanced data classification.

% KL divergence is used to measure the difference between two distributions \cite{Cantu_Springer_2004,Luo_TNNLS_2020}, while Bhattacharyya distance is used to measure the similarity of two probability distributions. Given two Gaussian distributions $\mathcal{N}_{1}(\mu_{1}, \sigma^{2}_{1})$ and $\mathcal{N}_{2}(\mu_{2}, \sigma^{2}_{2})$, we use the averaged KL divergence, \ie $\bar{d}_{KL}(\mathcal{N}_{1}, \mathcal{N}_{2}) = (d_{KL}(\mathcal{N}_{1}, \mathcal{N}_{2}) + d_{KL}(\mathcal{N}_{2}, \mathcal{N}_{1}))/2$, where $d_{KL}(\mathcal{N}_{1}, \mathcal{N}_{2})=\log\frac{\sigma_{2}}{\sigma_{1}}+\frac{\sigma_{1}^{2}+(\mu_{1}-\mu_{2})^{2}}{2\sigma_{2}^{2}}-\frac{1}{2}$ is not symmetrical. On the other hand, Bhattacharyya distance is defined as $d_{B}(\mathcal{N}_{1}, \mathcal{N}_{2})=\frac{1}{4}\ln \left( \frac{1}{4} \left( \frac{\sigma^{2}_{1}}{\sigma^{2}_{2}}+\frac{\sigma^{2}_{2}}{\sigma^{2}_{1}}+2 \right) \right) + \frac{1}{4} \left( \frac{(\mu_{1}-\mu_{2})^{2}}{\sigma^{2}_{1}+\sigma^{2}_{2}} \right)$. A larger $\bar{d}_{KL}$ or $d_{B}$ indicates that the two distributions are further away from each other.


% We hypothesize that $x$ w.r.t. positive and negative samples both follow Gaussian distributions. The discriminativeness of features is an important characteristic that correlates to the performance, \eg accuracy. We are interested in measures of separability of feature distributions, which reflect the discriminativeness from a probabilistic perspective. There are two common tools to achieve the goal, \ie Kullback–Leibler (KL) divergence \cite{Kullback_AMS_1951} and Bhattacharyya distance \cite{Bhattacharyya_JSTOR_1946}. Usually, KL divergence is used to measure the difference between two distributions \cite{Cantu_Springer_2004,Luo_TNNLS_2020}, while Bhattacharyya distance is used to measure the similarity of two probability distributions. Given two Gaussian distributions $\mathcal{N}_{1}(\mu_{1}, \sigma^{2}_{1})$ and $\mathcal{N}_{2}(\mu_{2}, \sigma^{2}_{2})$, we use an averaged KL divergence as in this work, \ie $\bar{d}_{KL}(\mathcal{N}_{1}, \mathcal{N}_{2}) = (d_{KL}(\mathcal{N}_{1}, \mathcal{N}_{2}) + d_{KL}(\mathcal{N}_{2}, \mathcal{N}_{1}))/2$, where $d_{KL}(\mathcal{N}_{1}, \mathcal{N}_{2})$ is the KL divergence between $\mathcal{N}_{1}$ and $\mathcal{N}_{2}$ (not symmetrical). On the other hand, Bhattacharyya distance is defined as $d_{B}(\mathcal{N}_{1}, \mathcal{N}_{2})=\frac{1}{4}\ln \left( \frac{1}{4} \left( \frac{\sigma^{2}_{1}}{\sigma^{2}_{2}}+\frac{\sigma^{2}_{2}}{\sigma^{2}_{1}}+2 \right) \right) + \frac{1}{4} \left( \frac{(\mu_{1}-\mu_{2})^{2}}{\sigma^{2}_{1}+\sigma^{2}_{2}} \right)$. In this work, we use Bhattacharyya coefficient that measures the amount of overlap between two distributions, instead of Bhattacharyya distance. Bhattacharyya coefficient is defined as $c_{B}(\mathcal{N}_{1}, \mathcal{N}_{2}) = \exp(-d_{B}(\mathcal{N}_{1}, \mathcal{N}_{2}))$. $c_{B} \in [0,1]$, where 1 indicates a full overlap and 0 indicates no overlap.

% \noindent\textbf{Semantics Difference between Predicting Trustworthiness and Classification}. As we use ViT for both the oracle and classifier, it is interesting to find out what features are leaned for predicting trustworthiness, in comparison to the features learned for classification. Hence, we compute the $l_{1}$ and $l_{2}$ distances between the features generated by the learned oracle and the features generated by the pre-trained classifier. The features are the inputs to the last layer of ViT, \ie 768-dimensional vectors.

% The mean and standard deviation of distances over all the samples in the training and validation sets are provided in \tabref{tbl:anal_diff}. Note that a smaller distance indicates higher similarity between two features. Overall, the mean of distances w.r.t. the three loss functions are large, but the focal loss yields the smallest averaged distance, which implies that the oracle learned with the focal yields the most similar features as the ones generated by the pre-trained classifier. One of possible reasons is that the focal loss prohibits the oracle training.

% Comparison of classifier backbone and oracle backbone

% Per class accuracy, precision, recall, F1

% \noindent\textbf{Taking Features as Input}
% \figref{fig:anal_featinput} shows the distributions of discriminative features generated by a multi-layer perceptron (MLP),, which plays as an oracle. The MLP takes the features generated by the classifier, instead of images, as input. The MLP-based oracle is training on the training set and is evaluated on the validation set. The figure shows that the oracle barely distinguish between positives and negatives. Because all the features are on the right-hand side of the decision boundary $x=0$.

% focal loss vs proposed

% \begin{figure}[!t]
	\centering
	\subfloat{\includegraphics[width=0.32\textwidth]{fig/analysis/anal_featinput_ce}    } \hfill
	\subfloat{\includegraphics[width=0.32\textwidth]{fig/analysis/anal_featinput_focal}    } \hfill
	\subfloat{\includegraphics[width=0.32\textwidth]{fig/analysis/anal_featinput_ss}    } \\
	\caption{\label{fig:anal_featinput}
    	Analysis of taking the features of the classifier as input to the oracle on the ImageNet validation set. In this experiment, ViT is used for both the oracle backbone and the classifier. The features are 768-dimensional vectors. The classifier is used in the evaluation mode in the experiment.
    % 	\REVISION{\textit{Baseline} indicates ResNet GEM.}
    	}
\end{figure}

% \begin{table}[!t]
	\centering
	\caption{\label{tbl:anal_diff}
	    Analysis of the difference of the output features between the classifier backbone and the oracle backbone in terms of $l_{1}$ and $l_{2}$ distances. The common backbone is ViT. The oracle backbone is trained for predicting trustworthiness, while the classifier backbone is pre-trained for classification.
	}
	\adjustbox{width=1\columnwidth}{
	\begin{tabular}{L{7ex} C{14ex} C{14ex} C{14ex} C{14ex}}
		\toprule
		& \multicolumn{2}{c}{Training} & \multicolumn{2}{c}{Validation} \\
		\cmidrule(lr){2-3} \cmidrule(lr){4-5}
		Loss & $l_{1}$ & $l_{2}$ & $l_{1}$ & $l_{2}$ \\
		\cmidrule(lr){1-1} \cmidrule(lr){2-2} \cmidrule(lr){3-3} \cmidrule(lr){4-4} \cmidrule(lr){5-5}
		CE & 74.0674$\pm$23.9773 & 3.4074$\pm$1.0967 & 78.4107$\pm$24.9338 & 3.6051$\pm$1.1402 \\
        Focal & 29.0901$\pm$8.5641 & 1.3527$\pm$0.3933 & 30.6497$\pm$8.9262 & 1.4240$\pm$0.4100 \\
        SS & 70.1997$\pm$32.8220 & 3.2129$\pm$1.4973 & 77.3162$\pm$33.4536 & 3.5378$\pm$1.5271 \\
		\bottomrule	
	\end{tabular}}
\end{table}

% \noindent\textbf{Ablation Study}. With the classifier ViT and the ViT based oracle, we show how the performance vary when $\alpha^{+}$ and $\alpha^{-}$ change.  

% \noindent\textbf{Generalization to Unseen Classifier}.
% As the oracle is trained by observing what a classifier predicts the label for an image, the knowledge learned in this way highly correlates to the behaviours of the classifier. It is interesting to know how the knowledge learned by the oracle generalizes to other unseen classifiers. To this end, we use the ViT based oracle that is trained with a ViT classifier to predict the trustworthiness of a ResNet-50 on the adversarial validation set, which is the most challenging in the three sets. 
% For the proposed loss, we use $\alpha^{+}=1$ and $\alpha^{-}=3$ for the oracle that is based on ViT's backbone, while we use $\alpha^{+}=2$ and $\alpha^{-}=5$ for the oracle that is based on ResNet's backbone.


\section{Conclusion}
This paper introduces SysNoise, a harmful noise that frequently happens when the source training system switches to a disparate target system in deployments. We first identify and classify SysNoise based on the inference stage, and thereafter build a holistic benchmark and framework to quantitatively measure the impact of SysNoise on image classification, object detection, segmentation, and natural language processing tasks. Our large-scale experiments revealed that SysNoise is highly-influential and will cause model performance degeneration; additionally, common mitigations like data augmentation and adversarial training show limited effects on SysNoise.  

In the future, we will evaluate SysNoise on the real-world systems, and will continuously develop the benchmark to include more tasks. Our findings open a new research topic and we hope it will raise research attention to the performance and robustness of deep learning deployment systems. 

%%%%%%%%%%%%%%%%%%%%%%%%%%%%%%%%%%%%%%%%%%%%%%%%%%%%%%%%%%%%

\section*{Acknowledgement}
This work was supported by the National Key Research and Development Plan of China (2021ZD0110601), the National Natural Science Foundation of China (62022009 and 62206009), and the State Key Laboratory of Software Development Environment.


\clearpage
% In the unusual situation where you want a paper to appear in the
% references without citing it in the main text, use \nocite
\nocite{langley00}


\bibliography{example_paper}
\bibliographystyle{mlsys2023}


%%%%%%%%%%%%%%%%%%%%%%%%%%%%%%%%%%%%%%%%%%%%%%%%%%%%%%%%%%%%%%%%%%%%%%%%%%%%%%%
%%%%%%%%%%%%%%%%%%%%%%%%%%%%%%%%%%%%%%%%%%%%%%%%%%%%%%%%%%%%%%%%%%%%%%%%%%%%%%%
% SUPPLEMENTAL CONTENT AS APPENDIX AFTER REFERENCES
%%%%%%%%%%%%%%%%%%%%%%%%%%%%%%%%%%%%%%%%%%%%%%%%%%%%%%%%%%%%%%%%%%%%%%%%%%%%%%%
%%%%%%%%%%%%%%%%%%%%%%%%%%%%%%%%%%%%%%%%%%%%%%%%%%%%%%%%%%%%%%%%%%%%%%%%%%%%%%%

\clearpage

\appendix
\chapter{Supplementary Material}
\label{appendix}

In this appendix, we present supplementary material for the techniques and
experiments presented in the main text.

\section{Baseline Results and Analysis for Informed Sampler}
\label{appendix:chap3}

Here, we give an in-depth
performance analysis of the various samplers and the effect of their
hyperparameters. We choose hyperparameters with the lowest PSRF value
after $10k$ iterations, for each sampler individually. If the
differences between PSRF are not significantly different among
multiple values, we choose the one that has the highest acceptance
rate.

\subsection{Experiment: Estimating Camera Extrinsics}
\label{appendix:chap3:room}

\subsubsection{Parameter Selection}
\paragraph{Metropolis Hastings (\MH)}

Figure~\ref{fig:exp1_MH} shows the median acceptance rates and PSRF
values corresponding to various proposal standard deviations of plain
\MH~sampling. Mixing gets better and the acceptance rate gets worse as
the standard deviation increases. The value $0.3$ is selected standard
deviation for this sampler.

\paragraph{Metropolis Hastings Within Gibbs (\MHWG)}

As mentioned in Section~\ref{sec:room}, the \MHWG~sampler with one-dimensional
updates did not converge for any value of proposal standard deviation.
This problem has high correlation of the camera parameters and is of
multi-modal nature, which this sampler has problems with.

\paragraph{Parallel Tempering (\PT)}

For \PT~sampling, we took the best performing \MH~sampler and used
different temperature chains to improve the mixing of the
sampler. Figure~\ref{fig:exp1_PT} shows the results corresponding to
different combination of temperature levels. The sampler with
temperature levels of $[1,3,27]$ performed best in terms of both
mixing and acceptance rate.

\paragraph{Effect of Mixture Coefficient in Informed Sampling (\MIXLMH)}

Figure~\ref{fig:exp1_alpha} shows the effect of mixture
coefficient ($\alpha$) on the informed sampling
\MIXLMH. Since there is no significant different in PSRF values for
$0 \le \alpha \le 0.7$, we chose $0.7$ due to its high acceptance
rate.


% \end{multicols}

\begin{figure}[h]
\centering
  \subfigure[MH]{%
    \includegraphics[width=.48\textwidth]{figures/supplementary/camPose_MH.pdf} \label{fig:exp1_MH}
  }
  \subfigure[PT]{%
    \includegraphics[width=.48\textwidth]{figures/supplementary/camPose_PT.pdf} \label{fig:exp1_PT}
  }
\\
  \subfigure[INF-MH]{%
    \includegraphics[width=.48\textwidth]{figures/supplementary/camPose_alpha.pdf} \label{fig:exp1_alpha}
  }
  \mycaption{Results of the `Estimating Camera Extrinsics' experiment}{PRSFs and Acceptance rates corresponding to (a) various standard deviations of \MH, (b) various temperature level combinations of \PT sampling and (c) various mixture coefficients of \MIXLMH sampling.}
\end{figure}



\begin{figure}[!t]
\centering
  \subfigure[\MH]{%
    \includegraphics[width=.48\textwidth]{figures/supplementary/occlusionExp_MH.pdf} \label{fig:exp2_MH}
  }
  \subfigure[\BMHWG]{%
    \includegraphics[width=.48\textwidth]{figures/supplementary/occlusionExp_BMHWG.pdf} \label{fig:exp2_BMHWG}
  }
\\
  \subfigure[\MHWG]{%
    \includegraphics[width=.48\textwidth]{figures/supplementary/occlusionExp_MHWG.pdf} \label{fig:exp2_MHWG}
  }
  \subfigure[\PT]{%
    \includegraphics[width=.48\textwidth]{figures/supplementary/occlusionExp_PT.pdf} \label{fig:exp2_PT}
  }
\\
  \subfigure[\INFBMHWG]{%
    \includegraphics[width=.5\textwidth]{figures/supplementary/occlusionExp_alpha.pdf} \label{fig:exp2_alpha}
  }
  \mycaption{Results of the `Occluding Tiles' experiment}{PRSF and
    Acceptance rates corresponding to various standard deviations of
    (a) \MH, (b) \BMHWG, (c) \MHWG, (d) various temperature level
    combinations of \PT~sampling and; (e) various mixture coefficients
    of our informed \INFBMHWG sampling.}
\end{figure}

%\onecolumn\newpage\twocolumn
\subsection{Experiment: Occluding Tiles}
\label{appendix:chap3:tiles}

\subsubsection{Parameter Selection}

\paragraph{Metropolis Hastings (\MH)}

Figure~\ref{fig:exp2_MH} shows the results of
\MH~sampling. Results show the poor convergence for all proposal
standard deviations and rapid decrease of AR with increasing standard
deviation. This is due to the high-dimensional nature of
the problem. We selected a standard deviation of $1.1$.

\paragraph{Blocked Metropolis Hastings Within Gibbs (\BMHWG)}

The results of \BMHWG are shown in Figure~\ref{fig:exp2_BMHWG}. In
this sampler we update only one block of tile variables (of dimension
four) in each sampling step. Results show much better performance
compared to plain \MH. The optimal proposal standard deviation for
this sampler is $0.7$.

\paragraph{Metropolis Hastings Within Gibbs (\MHWG)}

Figure~\ref{fig:exp2_MHWG} shows the result of \MHWG sampling. This
sampler is better than \BMHWG and converges much more quickly. Here
a standard deviation of $0.9$ is found to be best.

\paragraph{Parallel Tempering (\PT)}

Figure~\ref{fig:exp2_PT} shows the results of \PT sampling with various
temperature combinations. Results show no improvement in AR from plain
\MH sampling and again $[1,3,27]$ temperature levels are found to be optimal.

\paragraph{Effect of Mixture Coefficient in Informed Sampling (\INFBMHWG)}

Figure~\ref{fig:exp2_alpha} shows the effect of mixture
coefficient ($\alpha$) on the blocked informed sampling
\INFBMHWG. Since there is no significant different in PSRF values for
$0 \le \alpha \le 0.8$, we chose $0.8$ due to its high acceptance
rate.



\subsection{Experiment: Estimating Body Shape}
\label{appendix:chap3:body}

\subsubsection{Parameter Selection}
\paragraph{Metropolis Hastings (\MH)}

Figure~\ref{fig:exp3_MH} shows the result of \MH~sampling with various
proposal standard deviations. The value of $0.1$ is found to be
best.

\paragraph{Metropolis Hastings Within Gibbs (\MHWG)}

For \MHWG sampling we select $0.3$ proposal standard
deviation. Results are shown in Fig.~\ref{fig:exp3_MHWG}.


\paragraph{Parallel Tempering (\PT)}

As before, results in Fig.~\ref{fig:exp3_PT}, the temperature levels
were selected to be $[1,3,27]$ due its slightly higher AR.

\paragraph{Effect of Mixture Coefficient in Informed Sampling (\MIXLMH)}

Figure~\ref{fig:exp3_alpha} shows the effect of $\alpha$ on PSRF and
AR. Since there is no significant differences in PSRF values for $0 \le
\alpha \le 0.8$, we choose $0.8$.


\begin{figure}[t]
\centering
  \subfigure[\MH]{%
    \includegraphics[width=.48\textwidth]{figures/supplementary/bodyShape_MH.pdf} \label{fig:exp3_MH}
  }
  \subfigure[\MHWG]{%
    \includegraphics[width=.48\textwidth]{figures/supplementary/bodyShape_MHWG.pdf} \label{fig:exp3_MHWG}
  }
\\
  \subfigure[\PT]{%
    \includegraphics[width=.48\textwidth]{figures/supplementary/bodyShape_PT.pdf} \label{fig:exp3_PT}
  }
  \subfigure[\MIXLMH]{%
    \includegraphics[width=.48\textwidth]{figures/supplementary/bodyShape_alpha.pdf} \label{fig:exp3_alpha}
  }
\\
  \mycaption{Results of the `Body Shape Estimation' experiment}{PRSFs and
    Acceptance rates corresponding to various standard deviations of
    (a) \MH, (b) \MHWG; (c) various temperature level combinations
    of \PT sampling and; (d) various mixture coefficients of the
    informed \MIXLMH sampling.}
\end{figure}


\subsection{Results Overview}
Figure~\ref{fig:exp_summary} shows the summary results of the all the three
experimental studies related to informed sampler.
\begin{figure*}[h!]
\centering
  \subfigure[Results for: Estimating Camera Extrinsics]{%
    \includegraphics[width=0.9\textwidth]{figures/supplementary/camPose_ALL.pdf} \label{fig:exp1_all}
  }
  \subfigure[Results for: Occluding Tiles]{%
    \includegraphics[width=0.9\textwidth]{figures/supplementary/occlusionExp_ALL.pdf} \label{fig:exp2_all}
  }
  \subfigure[Results for: Estimating Body Shape]{%
    \includegraphics[width=0.9\textwidth]{figures/supplementary/bodyShape_ALL.pdf} \label{fig:exp3_all}
  }
  \label{fig:exp_summary}
  \mycaption{Summary of the statistics for the three experiments}{Shown are
    for several baseline methods and the informed samplers the
    acceptance rates (left), PSRFs (middle), and RMSE values
    (right). All results are median results over multiple test
    examples.}
\end{figure*}

\subsection{Additional Qualitative Results}

\subsubsection{Occluding Tiles}
In Figure~\ref{fig:exp2_visual_more} more qualitative results of the
occluding tiles experiment are shown. The informed sampling approach
(\INFBMHWG) is better than the best baseline (\MHWG). This still is a
very challenging problem since the parameters for occluded tiles are
flat over a large region. Some of the posterior variance of the
occluded tiles is already captured by the informed sampler.

\begin{figure*}[h!]
\begin{center}
\centerline{\includegraphics[width=0.95\textwidth]{figures/supplementary/occlusionExp_Visual.pdf}}
\mycaption{Additional qualitative results of the occluding tiles experiment}
  {From left to right: (a)
  Given image, (b) Ground truth tiles, (c) OpenCV heuristic and most probable estimates
  from 5000 samples obtained by (d) MHWG sampler (best baseline) and
  (e) our INF-BMHWG sampler. (f) Posterior expectation of the tiles
  boundaries obtained by INF-BMHWG sampling (First 2000 samples are
  discarded as burn-in).}
\label{fig:exp2_visual_more}
\end{center}
\end{figure*}

\subsubsection{Body Shape}
Figure~\ref{fig:exp3_bodyMeshes} shows some more results of 3D mesh
reconstruction using posterior samples obtained by our informed
sampling \MIXLMH.

\begin{figure*}[t]
\begin{center}
\centerline{\includegraphics[width=0.75\textwidth]{figures/supplementary/bodyMeshResults.pdf}}
\mycaption{Qualitative results for the body shape experiment}
  {Shown is the 3D mesh reconstruction results with first 1000 samples obtained
  using the \MIXLMH informed sampling method. (blue indicates small
  values and red indicates high values)}
\label{fig:exp3_bodyMeshes}
\end{center}
\end{figure*}

\clearpage



\section{Additional Results on the Face Problem with CMP}

Figure~\ref{fig:shading-qualitative-multiple-subjects-supp} shows inference results for reflectance maps, normal maps and lights for randomly chosen test images, and Fig.~\ref{fig:shading-qualitative-same-subject-supp} shows reflectance estimation results on multiple images of the same subject produced under different illumination conditions. CMP is able to produce estimates that are closer to the groundtruth across different subjects and illumination conditions.

\begin{figure*}[h]
  \begin{center}
  \centerline{\includegraphics[width=1.0\columnwidth]{figures/face_cmp_visual_results_supp.pdf}}
  \vspace{-1.2cm}
  \end{center}
	\mycaption{A visual comparison of inference results}{(a)~Observed images. (b)~Inferred reflectance maps. \textit{GT} is the photometric stereo groundtruth, \textit{BU} is the Biswas \etal (2009) reflectance estimate and \textit{Forest} is the consensus prediction. (c)~The variance of the inferred reflectance estimate produced by \MTD (normalized across rows).(d)~Visualization of inferred light directions. (e)~Inferred normal maps.}
	\label{fig:shading-qualitative-multiple-subjects-supp}
\end{figure*}


\begin{figure*}[h]
	\centering
	\setlength\fboxsep{0.2mm}
	\setlength\fboxrule{0pt}
	\begin{tikzpicture}

		\matrix at (0, 0) [matrix of nodes, nodes={anchor=east}, column sep=-0.05cm, row sep=-0.2cm]
		{
			\fbox{\includegraphics[width=1cm]{figures/sample_3_4_X.png}} &
			\fbox{\includegraphics[width=1cm]{figures/sample_3_4_GT.png}} &
			\fbox{\includegraphics[width=1cm]{figures/sample_3_4_BISWAS.png}}  &
			\fbox{\includegraphics[width=1cm]{figures/sample_3_4_VMP.png}}  &
			\fbox{\includegraphics[width=1cm]{figures/sample_3_4_FOREST.png}}  &
			\fbox{\includegraphics[width=1cm]{figures/sample_3_4_CMP.png}}  &
			\fbox{\includegraphics[width=1cm]{figures/sample_3_4_CMPVAR.png}}
			 \\

			\fbox{\includegraphics[width=1cm]{figures/sample_3_5_X.png}} &
			\fbox{\includegraphics[width=1cm]{figures/sample_3_5_GT.png}} &
			\fbox{\includegraphics[width=1cm]{figures/sample_3_5_BISWAS.png}}  &
			\fbox{\includegraphics[width=1cm]{figures/sample_3_5_VMP.png}}  &
			\fbox{\includegraphics[width=1cm]{figures/sample_3_5_FOREST.png}}  &
			\fbox{\includegraphics[width=1cm]{figures/sample_3_5_CMP.png}}  &
			\fbox{\includegraphics[width=1cm]{figures/sample_3_5_CMPVAR.png}}
			 \\

			\fbox{\includegraphics[width=1cm]{figures/sample_3_6_X.png}} &
			\fbox{\includegraphics[width=1cm]{figures/sample_3_6_GT.png}} &
			\fbox{\includegraphics[width=1cm]{figures/sample_3_6_BISWAS.png}}  &
			\fbox{\includegraphics[width=1cm]{figures/sample_3_6_VMP.png}}  &
			\fbox{\includegraphics[width=1cm]{figures/sample_3_6_FOREST.png}}  &
			\fbox{\includegraphics[width=1cm]{figures/sample_3_6_CMP.png}}  &
			\fbox{\includegraphics[width=1cm]{figures/sample_3_6_CMPVAR.png}}
			 \\
	     };

       \node at (-3.85, -2.0) {\small Observed};
       \node at (-2.55, -2.0) {\small `GT'};
       \node at (-1.27, -2.0) {\small BU};
       \node at (0.0, -2.0) {\small MP};
       \node at (1.27, -2.0) {\small Forest};
       \node at (2.55, -2.0) {\small \textbf{CMP}};
       \node at (3.85, -2.0) {\small Variance};

	\end{tikzpicture}
	\mycaption{Robustness to varying illumination}{Reflectance estimation on a subject images with varying illumination. Left to right: observed image, photometric stereo estimate (GT)
  which is used as a proxy for groundtruth, bottom-up estimate of \cite{Biswas2009}, VMP result, consensus forest estimate, CMP mean, and CMP variance.}
	\label{fig:shading-qualitative-same-subject-supp}
\end{figure*}

\clearpage

\section{Additional Material for Learning Sparse High Dimensional Filters}
\label{sec:appendix-bnn}

This part of supplementary material contains a more detailed overview of the permutohedral
lattice convolution in Section~\ref{sec:permconv}, more experiments in
Section~\ref{sec:addexps} and additional results with protocols for
the experiments presented in Chapter~\ref{chap:bnn} in Section~\ref{sec:addresults}.

\vspace{-0.2cm}
\subsection{General Permutohedral Convolutions}
\label{sec:permconv}

A core technical contribution of this work is the generalization of the Gaussian permutohedral lattice
convolution proposed in~\cite{adams2010fast} to the full non-separable case with the
ability to perform back-propagation. Although, conceptually, there are minor
differences between Gaussian and general parameterized filters, there are non-trivial practical
differences in terms of the algorithmic implementation. The Gauss filters belong to
the separable class and can thus be decomposed into multiple
sequential one dimensional convolutions. We are interested in the general filter
convolutions, which can not be decomposed. Thus, performing a general permutohedral
convolution at a lattice point requires the computation of the inner product with the
neighboring elements in all the directions in the high-dimensional space.

Here, we give more details of the implementation differences of separable
and non-separable filters. In the following, we will explain the scalar case first.
Recall, that the forward pass of general permutohedral convolution
involves 3 steps: \textit{splatting}, \textit{convolving} and \textit{slicing}.
We follow the same splatting and slicing strategies as in~\cite{adams2010fast}
since these operations do not depend on the filter kernel. The main difference
between our work and the existing implementation of~\cite{adams2010fast} is
the way that the convolution operation is executed. This proceeds by constructing
a \emph{blur neighbor} matrix $K$ that stores for every lattice point all
values of the lattice neighbors that are needed to compute the filter output.

\begin{figure}[t!]
  \centering
    \includegraphics[width=0.6\columnwidth]{figures/supplementary/lattice_construction}
  \mycaption{Illustration of 1D permutohedral lattice construction}
  {A $4\times 4$ $(x,y)$ grid lattice is projected onto the plane defined by the normal
  vector $(1,1)^{\top}$. This grid has $s+1=4$ and $d=2$ $(s+1)^{d}=4^2=16$ elements.
  In the projection, all points of the same color are projected onto the same points in the plane.
  The number of elements of the projected lattice is $t=(s+1)^d-s^d=4^2-3^2=7$, that is
  the $(4\times 4)$ grid minus the size of lattice that is $1$ smaller at each size, in this
  case a $(3\times 3)$ lattice (the upper right $(3\times 3)$ elements).
  }
\label{fig:latticeconstruction}
\end{figure}

The blur neighbor matrix is constructed by traversing through all the populated
lattice points and their neighboring elements.
% For efficiency, we do this matrix construction recursively with shared computations
% since $n^{th}$ neighbourhood elements are $1^{st}$ neighborhood elements of $n-1^{th}$ neighbourhood elements. \pg{do not understand}
This is done recursively to share computations. For any lattice point, the neighbors that are
$n$ hops away are the direct neighbors of the points that are $n-1$ hops away.
The size of a $d$ dimensional spatial filter with width $s+1$ is $(s+1)^{d}$ (\eg, a
$3\times 3$ filter, $s=2$ in $d=2$ has $3^2=9$ elements) and this size grows
exponentially in the number of dimensions $d$. The permutohedral lattice is constructed by
projecting a regular grid onto the plane spanned by the $d$ dimensional normal vector ${(1,\ldots,1)}^{\top}$. See
Fig.~\ref{fig:latticeconstruction} for an illustration of the 1D lattice construction.
Many corners of a grid filter are projected onto the same point, in total $t = {(s+1)}^{d} -
s^{d}$ elements remain in the permutohedral filter with $s$ neighborhood in $d-1$ dimensions.
If the lattice has $m$ populated elements, the
matrix $K$ has size $t\times m$. Note that, since the input signal is typically
sparse, only a few lattice corners are being populated in the \textit{slicing} step.
We use a hash-table to keep track of these points and traverse only through
the populated lattice points for this neighborhood matrix construction.

Once the blur neighbor matrix $K$ is constructed, we can perform the convolution
by the matrix vector multiplication
\begin{equation}
\ell' = BK,
\label{eq:conv}
\end{equation}
where $B$ is the $1 \times t$ filter kernel (whose values we will learn) and $\ell'\in\mathbb{R}^{1\times m}$
is the result of the filtering at the $m$ lattice points. In practice, we found that the
matrix $K$ is sometimes too large to fit into GPU memory and we divided the matrix $K$
into smaller pieces to compute Eq.~\ref{eq:conv} sequentially.

In the general multi-dimensional case, the signal $\ell$ is of $c$ dimensions. Then
the kernel $B$ is of size $c \times t$ and $K$ stores the $c$ dimensional vectors
accordingly. When the input and output points are different, we slice only the
input points and splat only at the output points.


\subsection{Additional Experiments}
\label{sec:addexps}
In this section, we discuss more use-cases for the learned bilateral filters, one
use-case of BNNs and two single filter applications for image and 3D mesh denoising.

\subsubsection{Recognition of subsampled MNIST}\label{sec:app_mnist}

One of the strengths of the proposed filter convolution is that it does not
require the input to lie on a regular grid. The only requirement is to define a distance
between features of the input signal.
We highlight this feature with the following experiment using the
classical MNIST ten class classification problem~\cite{lecun1998mnist}. We sample a
sparse set of $N$ points $(x,y)\in [0,1]\times [0,1]$
uniformly at random in the input image, use their interpolated values
as signal and the \emph{continuous} $(x,y)$ positions as features. This mimics
sub-sampling of a high-dimensional signal. To compare against a spatial convolution,
we interpolate the sparse set of values at the grid positions.

We take a reference implementation of LeNet~\cite{lecun1998gradient} that
is part of the Caffe project~\cite{jia2014caffe} and compare it
against the same architecture but replacing the first convolutional
layer with a bilateral convolution layer (BCL). The filter size
and numbers are adjusted to get a comparable number of parameters
($5\times 5$ for LeNet, $2$-neighborhood for BCL).

The results are shown in Table~\ref{tab:all-results}. We see that training
on the original MNIST data (column Original, LeNet vs. BNN) leads to a slight
decrease in performance of the BNN (99.03\%) compared to LeNet
(99.19\%). The BNN can be trained and evaluated on sparse
signals, and we resample the image as described above for $N=$ 100\%, 60\% and
20\% of the total number of pixels. The methods are also evaluated
on test images that are subsampled in the same way. Note that we can
train and test with different subsampling rates. We introduce an additional
bilinear interpolation layer for the LeNet architecture to train on the same
data. In essence, both models perform a spatial interpolation and thus we
expect them to yield a similar classification accuracy. Once the data is of
higher dimensions, the permutohedral convolution will be faster due to hashing
the sparse input points, as well as less memory demanding in comparison to
naive application of a spatial convolution with interpolated values.

\begin{table}[t]
  \begin{center}
    \footnotesize
    \centering
    \begin{tabular}[t]{lllll}
      \toprule
              &     & \multicolumn{3}{c}{Test Subsampling} \\
       Method  & Original & 100\% & 60\% & 20\%\\
      \midrule
       LeNet &  \textbf{0.9919} & 0.9660 & 0.9348 & \textbf{0.6434} \\
       BNN &  0.9903 & \textbf{0.9844} & \textbf{0.9534} & 0.5767 \\
      \hline
       LeNet 100\% & 0.9856 & 0.9809 & 0.9678 & \textbf{0.7386} \\
       BNN 100\% & \textbf{0.9900} & \textbf{0.9863} & \textbf{0.9699} & 0.6910 \\
      \hline
       LeNet 60\% & 0.9848 & 0.9821 & 0.9740 & 0.8151 \\
       BNN 60\% & \textbf{0.9885} & \textbf{0.9864} & \textbf{0.9771} & \textbf{0.8214}\\
      \hline
       LeNet 20\% & \textbf{0.9763} & \textbf{0.9754} & 0.9695 & 0.8928 \\
       BNN 20\% & 0.9728 & 0.9735 & \textbf{0.9701} & \textbf{0.9042}\\
      \bottomrule
    \end{tabular}
  \end{center}
\vspace{-.2cm}
\caption{Classification accuracy on MNIST. We compare the
    LeNet~\cite{lecun1998gradient} implementation that is part of
    Caffe~\cite{jia2014caffe} to the network with the first layer
    replaced by a bilateral convolution layer (BCL). Both are trained
    on the original image resolution (first two rows). Three more BNN
    and CNN models are trained with randomly subsampled images (100\%,
    60\% and 20\% of the pixels). An additional bilinear interpolation
    layer samples the input signal on a spatial grid for the CNN model.
  }
  \label{tab:all-results}
\vspace{-.5cm}
\end{table}

\subsubsection{Image Denoising}

The main application that inspired the development of the bilateral
filtering operation is image denoising~\cite{aurich1995non}, there
using a single Gaussian kernel. Our development allows to learn this
kernel function from data and we explore how to improve using a \emph{single}
but more general bilateral filter.

We use the Berkeley segmentation dataset
(BSDS500)~\cite{arbelaezi2011bsds500} as a test bed. The color
images in the dataset are converted to gray-scale,
and corrupted with Gaussian noise with a standard deviation of
$\frac {25} {255}$.

We compare the performance of four different filter models on a
denoising task.
The first baseline model (`Spatial' in Table \ref{tab:denoising}, $25$
weights) uses a single spatial filter with a kernel size of
$5$ and predicts the scalar gray-scale value at the center pixel. The next model
(`Gauss Bilateral') applies a bilateral \emph{Gaussian}
filter to the noisy input, using position and intensity features $\f=(x,y,v)^\top$.
The third setup (`Learned Bilateral', $65$ weights)
takes a Gauss kernel as initialization and
fits all filter weights on the train set to minimize the
mean squared error with respect to the clean images.
We run a combination
of spatial and permutohedral convolutions on spatial and bilateral
features (`Spatial + Bilateral (Learned)') to check for a complementary
performance of the two convolutions.

\label{sec:exp:denoising}
\begin{table}[!h]
\begin{center}
  \footnotesize
  \begin{tabular}[t]{lr}
    \toprule
    Method & PSNR \\
    \midrule
    Noisy Input & $20.17$ \\
    Spatial & $26.27$ \\
    Gauss Bilateral & $26.51$ \\
    Learned Bilateral & $26.58$ \\
    Spatial + Bilateral (Learned) & \textbf{$26.65$} \\
    \bottomrule
  \end{tabular}
\end{center}
\vspace{-0.5em}
\caption{PSNR results of a denoising task using the BSDS500
  dataset~\cite{arbelaezi2011bsds500}}
\vspace{-0.5em}
\label{tab:denoising}
\end{table}
\vspace{-0.2em}

The PSNR scores evaluated on full images of the test set are
shown in Table \ref{tab:denoising}. We find that an untrained bilateral
filter already performs better than a trained spatial convolution
($26.27$ to $26.51$). A learned convolution further improve the
performance slightly. We chose this simple one-kernel setup to
validate an advantage of the generalized bilateral filter. A competitive
denoising system would employ RGB color information and also
needs to be properly adjusted in network size. Multi-layer perceptrons
have obtained state-of-the-art denoising results~\cite{burger12cvpr}
and the permutohedral lattice layer can readily be used in such an
architecture, which is intended future work.

\subsection{Additional results}
\label{sec:addresults}

This section contains more qualitative results for the experiments presented in Chapter~\ref{chap:bnn}.

\begin{figure*}[th!]
  \centering
    \includegraphics[width=\columnwidth,trim={5cm 2.5cm 5cm 4.5cm},clip]{figures/supplementary/lattice_viz.pdf}
    \vspace{-0.7cm}
  \mycaption{Visualization of the Permutohedral Lattice}
  {Sample lattice visualizations for different feature spaces. All pixels falling in the same simplex cell are shown with
  the same color. $(x,y)$ features correspond to image pixel positions, and $(r,g,b) \in [0,255]$ correspond
  to the red, green and blue color values.}
\label{fig:latticeviz}
\end{figure*}

\subsubsection{Lattice Visualization}

Figure~\ref{fig:latticeviz} shows sample lattice visualizations for different feature spaces.

\newcolumntype{L}[1]{>{\raggedright\let\newline\\\arraybackslash\hspace{0pt}}b{#1}}
\newcolumntype{C}[1]{>{\centering\let\newline\\\arraybackslash\hspace{0pt}}b{#1}}
\newcolumntype{R}[1]{>{\raggedleft\let\newline\\\arraybackslash\hspace{0pt}}b{#1}}

\subsubsection{Color Upsampling}\label{sec:color_upsampling}
\label{sec:col_upsample_extra}

Some images of the upsampling for the Pascal
VOC12 dataset are shown in Fig.~\ref{fig:Colour_upsample_visuals}. It is
especially the low level image details that are better preserved with
a learned bilateral filter compared to the Gaussian case.

\begin{figure*}[t!]
  \centering
    \subfigure{%
   \raisebox{2.0em}{
    \includegraphics[width=.06\columnwidth]{figures/supplementary/2007_004969.jpg}
   }
  }
  \subfigure{%
    \includegraphics[width=.17\columnwidth]{figures/supplementary/2007_004969_gray.pdf}
  }
  \subfigure{%
    \includegraphics[width=.17\columnwidth]{figures/supplementary/2007_004969_gt.pdf}
  }
  \subfigure{%
    \includegraphics[width=.17\columnwidth]{figures/supplementary/2007_004969_bicubic.pdf}
  }
  \subfigure{%
    \includegraphics[width=.17\columnwidth]{figures/supplementary/2007_004969_gauss.pdf}
  }
  \subfigure{%
    \includegraphics[width=.17\columnwidth]{figures/supplementary/2007_004969_learnt.pdf}
  }\\
    \subfigure{%
   \raisebox{2.0em}{
    \includegraphics[width=.06\columnwidth]{figures/supplementary/2007_003106.jpg}
   }
  }
  \subfigure{%
    \includegraphics[width=.17\columnwidth]{figures/supplementary/2007_003106_gray.pdf}
  }
  \subfigure{%
    \includegraphics[width=.17\columnwidth]{figures/supplementary/2007_003106_gt.pdf}
  }
  \subfigure{%
    \includegraphics[width=.17\columnwidth]{figures/supplementary/2007_003106_bicubic.pdf}
  }
  \subfigure{%
    \includegraphics[width=.17\columnwidth]{figures/supplementary/2007_003106_gauss.pdf}
  }
  \subfigure{%
    \includegraphics[width=.17\columnwidth]{figures/supplementary/2007_003106_learnt.pdf}
  }\\
  \setcounter{subfigure}{0}
  \small{
  \subfigure[Inp.]{%
  \raisebox{2.0em}{
    \includegraphics[width=.06\columnwidth]{figures/supplementary/2007_006837.jpg}
   }
  }
  \subfigure[Guidance]{%
    \includegraphics[width=.17\columnwidth]{figures/supplementary/2007_006837_gray.pdf}
  }
   \subfigure[GT]{%
    \includegraphics[width=.17\columnwidth]{figures/supplementary/2007_006837_gt.pdf}
  }
  \subfigure[Bicubic]{%
    \includegraphics[width=.17\columnwidth]{figures/supplementary/2007_006837_bicubic.pdf}
  }
  \subfigure[Gauss-BF]{%
    \includegraphics[width=.17\columnwidth]{figures/supplementary/2007_006837_gauss.pdf}
  }
  \subfigure[Learned-BF]{%
    \includegraphics[width=.17\columnwidth]{figures/supplementary/2007_006837_learnt.pdf}
  }
  }
  \vspace{-0.5cm}
  \mycaption{Color Upsampling}{Color $8\times$ upsampling results
  using different methods, from left to right, (a)~Low-resolution input color image (Inp.),
  (b)~Gray scale guidance image, (c)~Ground-truth color image; Upsampled color images with
  (d)~Bicubic interpolation, (e) Gauss bilateral upsampling and, (f)~Learned bilateral
  updampgling (best viewed on screen).}

\label{fig:Colour_upsample_visuals}
\end{figure*}

\subsubsection{Depth Upsampling}
\label{sec:depth_upsample_extra}

Figure~\ref{fig:depth_upsample_visuals} presents some more qualitative results comparing bicubic interpolation, Gauss
bilateral and learned bilateral upsampling on NYU depth dataset image~\cite{silberman2012indoor}.

\subsubsection{Character Recognition}\label{sec:app_character}

 Figure~\ref{fig:nnrecognition} shows the schematic of different layers
 of the network architecture for LeNet-7~\cite{lecun1998mnist}
 and DeepCNet(5, 50)~\cite{ciresan2012multi,graham2014spatially}. For the BNN variants, the first layer filters are replaced
 with learned bilateral filters and are learned end-to-end.

\subsubsection{Semantic Segmentation}\label{sec:app_semantic_segmentation}
\label{sec:semantic_bnn_extra}

Some more visual results for semantic segmentation are shown in Figure~\ref{fig:semantic_visuals}.
These include the underlying DeepLab CNN\cite{chen2014semantic} result (DeepLab),
the 2 step mean-field result with Gaussian edge potentials (+2stepMF-GaussCRF)
and also corresponding results with learned edge potentials (+2stepMF-LearnedCRF).
In general, we observe that mean-field in learned CRF leads to slightly dilated
classification regions in comparison to using Gaussian CRF thereby filling-in the
false negative pixels and also correcting some mis-classified regions.

\begin{figure*}[t!]
  \centering
    \subfigure{%
   \raisebox{2.0em}{
    \includegraphics[width=.06\columnwidth]{figures/supplementary/2bicubic}
   }
  }
  \subfigure{%
    \includegraphics[width=.17\columnwidth]{figures/supplementary/2given_image}
  }
  \subfigure{%
    \includegraphics[width=.17\columnwidth]{figures/supplementary/2ground_truth}
  }
  \subfigure{%
    \includegraphics[width=.17\columnwidth]{figures/supplementary/2bicubic}
  }
  \subfigure{%
    \includegraphics[width=.17\columnwidth]{figures/supplementary/2gauss}
  }
  \subfigure{%
    \includegraphics[width=.17\columnwidth]{figures/supplementary/2learnt}
  }\\
    \subfigure{%
   \raisebox{2.0em}{
    \includegraphics[width=.06\columnwidth]{figures/supplementary/32bicubic}
   }
  }
  \subfigure{%
    \includegraphics[width=.17\columnwidth]{figures/supplementary/32given_image}
  }
  \subfigure{%
    \includegraphics[width=.17\columnwidth]{figures/supplementary/32ground_truth}
  }
  \subfigure{%
    \includegraphics[width=.17\columnwidth]{figures/supplementary/32bicubic}
  }
  \subfigure{%
    \includegraphics[width=.17\columnwidth]{figures/supplementary/32gauss}
  }
  \subfigure{%
    \includegraphics[width=.17\columnwidth]{figures/supplementary/32learnt}
  }\\
  \setcounter{subfigure}{0}
  \small{
  \subfigure[Inp.]{%
  \raisebox{2.0em}{
    \includegraphics[width=.06\columnwidth]{figures/supplementary/41bicubic}
   }
  }
  \subfigure[Guidance]{%
    \includegraphics[width=.17\columnwidth]{figures/supplementary/41given_image}
  }
   \subfigure[GT]{%
    \includegraphics[width=.17\columnwidth]{figures/supplementary/41ground_truth}
  }
  \subfigure[Bicubic]{%
    \includegraphics[width=.17\columnwidth]{figures/supplementary/41bicubic}
  }
  \subfigure[Gauss-BF]{%
    \includegraphics[width=.17\columnwidth]{figures/supplementary/41gauss}
  }
  \subfigure[Learned-BF]{%
    \includegraphics[width=.17\columnwidth]{figures/supplementary/41learnt}
  }
  }
  \mycaption{Depth Upsampling}{Depth $8\times$ upsampling results
  using different upsampling strategies, from left to right,
  (a)~Low-resolution input depth image (Inp.),
  (b)~High-resolution guidance image, (c)~Ground-truth depth; Upsampled depth images with
  (d)~Bicubic interpolation, (e) Gauss bilateral upsampling and, (f)~Learned bilateral
  updampgling (best viewed on screen).}

\label{fig:depth_upsample_visuals}
\end{figure*}

\subsubsection{Material Segmentation}\label{sec:app_material_segmentation}
\label{sec:material_bnn_extra}

In Fig.~\ref{fig:material_visuals-app2}, we present visual results comparing 2 step
mean-field inference with Gaussian and learned pairwise CRF potentials. In
general, we observe that the pixels belonging to dominant classes in the
training data are being more accurately classified with learned CRF. This leads to
a significant improvements in overall pixel accuracy. This also results
in a slight decrease of the accuracy from less frequent class pixels thereby
slightly reducing the average class accuracy with learning. We attribute this
to the type of annotation that is available for this dataset, which is not
for the entire image but for some segments in the image. We have very few
images of the infrequent classes to combat this behaviour during training.

\subsubsection{Experiment Protocols}
\label{sec:protocols}

Table~\ref{tbl:parameters} shows experiment protocols of different experiments.

 \begin{figure*}[t!]
  \centering
  \subfigure[LeNet-7]{
    \includegraphics[width=0.7\columnwidth]{figures/supplementary/lenet_cnn_network}
    }\\
    \subfigure[DeepCNet]{
    \includegraphics[width=\columnwidth]{figures/supplementary/deepcnet_cnn_network}
    }
  \mycaption{CNNs for Character Recognition}
  {Schematic of (top) LeNet-7~\cite{lecun1998mnist} and (bottom) DeepCNet(5,50)~\cite{ciresan2012multi,graham2014spatially} architectures used in Assamese
  character recognition experiments.}
\label{fig:nnrecognition}
\end{figure*}

\definecolor{voc_1}{RGB}{0, 0, 0}
\definecolor{voc_2}{RGB}{128, 0, 0}
\definecolor{voc_3}{RGB}{0, 128, 0}
\definecolor{voc_4}{RGB}{128, 128, 0}
\definecolor{voc_5}{RGB}{0, 0, 128}
\definecolor{voc_6}{RGB}{128, 0, 128}
\definecolor{voc_7}{RGB}{0, 128, 128}
\definecolor{voc_8}{RGB}{128, 128, 128}
\definecolor{voc_9}{RGB}{64, 0, 0}
\definecolor{voc_10}{RGB}{192, 0, 0}
\definecolor{voc_11}{RGB}{64, 128, 0}
\definecolor{voc_12}{RGB}{192, 128, 0}
\definecolor{voc_13}{RGB}{64, 0, 128}
\definecolor{voc_14}{RGB}{192, 0, 128}
\definecolor{voc_15}{RGB}{64, 128, 128}
\definecolor{voc_16}{RGB}{192, 128, 128}
\definecolor{voc_17}{RGB}{0, 64, 0}
\definecolor{voc_18}{RGB}{128, 64, 0}
\definecolor{voc_19}{RGB}{0, 192, 0}
\definecolor{voc_20}{RGB}{128, 192, 0}
\definecolor{voc_21}{RGB}{0, 64, 128}
\definecolor{voc_22}{RGB}{128, 64, 128}

\begin{figure*}[t]
  \centering
  \small{
  \fcolorbox{white}{voc_1}{\rule{0pt}{6pt}\rule{6pt}{0pt}} Background~~
  \fcolorbox{white}{voc_2}{\rule{0pt}{6pt}\rule{6pt}{0pt}} Aeroplane~~
  \fcolorbox{white}{voc_3}{\rule{0pt}{6pt}\rule{6pt}{0pt}} Bicycle~~
  \fcolorbox{white}{voc_4}{\rule{0pt}{6pt}\rule{6pt}{0pt}} Bird~~
  \fcolorbox{white}{voc_5}{\rule{0pt}{6pt}\rule{6pt}{0pt}} Boat~~
  \fcolorbox{white}{voc_6}{\rule{0pt}{6pt}\rule{6pt}{0pt}} Bottle~~
  \fcolorbox{white}{voc_7}{\rule{0pt}{6pt}\rule{6pt}{0pt}} Bus~~
  \fcolorbox{white}{voc_8}{\rule{0pt}{6pt}\rule{6pt}{0pt}} Car~~ \\
  \fcolorbox{white}{voc_9}{\rule{0pt}{6pt}\rule{6pt}{0pt}} Cat~~
  \fcolorbox{white}{voc_10}{\rule{0pt}{6pt}\rule{6pt}{0pt}} Chair~~
  \fcolorbox{white}{voc_11}{\rule{0pt}{6pt}\rule{6pt}{0pt}} Cow~~
  \fcolorbox{white}{voc_12}{\rule{0pt}{6pt}\rule{6pt}{0pt}} Dining Table~~
  \fcolorbox{white}{voc_13}{\rule{0pt}{6pt}\rule{6pt}{0pt}} Dog~~
  \fcolorbox{white}{voc_14}{\rule{0pt}{6pt}\rule{6pt}{0pt}} Horse~~
  \fcolorbox{white}{voc_15}{\rule{0pt}{6pt}\rule{6pt}{0pt}} Motorbike~~
  \fcolorbox{white}{voc_16}{\rule{0pt}{6pt}\rule{6pt}{0pt}} Person~~ \\
  \fcolorbox{white}{voc_17}{\rule{0pt}{6pt}\rule{6pt}{0pt}} Potted Plant~~
  \fcolorbox{white}{voc_18}{\rule{0pt}{6pt}\rule{6pt}{0pt}} Sheep~~
  \fcolorbox{white}{voc_19}{\rule{0pt}{6pt}\rule{6pt}{0pt}} Sofa~~
  \fcolorbox{white}{voc_20}{\rule{0pt}{6pt}\rule{6pt}{0pt}} Train~~
  \fcolorbox{white}{voc_21}{\rule{0pt}{6pt}\rule{6pt}{0pt}} TV monitor~~ \\
  }
  \subfigure{%
    \includegraphics[width=.18\columnwidth]{figures/supplementary/2007_001423_given.jpg}
  }
  \subfigure{%
    \includegraphics[width=.18\columnwidth]{figures/supplementary/2007_001423_gt.png}
  }
  \subfigure{%
    \includegraphics[width=.18\columnwidth]{figures/supplementary/2007_001423_cnn.png}
  }
  \subfigure{%
    \includegraphics[width=.18\columnwidth]{figures/supplementary/2007_001423_gauss.png}
  }
  \subfigure{%
    \includegraphics[width=.18\columnwidth]{figures/supplementary/2007_001423_learnt.png}
  }\\
  \subfigure{%
    \includegraphics[width=.18\columnwidth]{figures/supplementary/2007_001430_given.jpg}
  }
  \subfigure{%
    \includegraphics[width=.18\columnwidth]{figures/supplementary/2007_001430_gt.png}
  }
  \subfigure{%
    \includegraphics[width=.18\columnwidth]{figures/supplementary/2007_001430_cnn.png}
  }
  \subfigure{%
    \includegraphics[width=.18\columnwidth]{figures/supplementary/2007_001430_gauss.png}
  }
  \subfigure{%
    \includegraphics[width=.18\columnwidth]{figures/supplementary/2007_001430_learnt.png}
  }\\
    \subfigure{%
    \includegraphics[width=.18\columnwidth]{figures/supplementary/2007_007996_given.jpg}
  }
  \subfigure{%
    \includegraphics[width=.18\columnwidth]{figures/supplementary/2007_007996_gt.png}
  }
  \subfigure{%
    \includegraphics[width=.18\columnwidth]{figures/supplementary/2007_007996_cnn.png}
  }
  \subfigure{%
    \includegraphics[width=.18\columnwidth]{figures/supplementary/2007_007996_gauss.png}
  }
  \subfigure{%
    \includegraphics[width=.18\columnwidth]{figures/supplementary/2007_007996_learnt.png}
  }\\
   \subfigure{%
    \includegraphics[width=.18\columnwidth]{figures/supplementary/2010_002682_given.jpg}
  }
  \subfigure{%
    \includegraphics[width=.18\columnwidth]{figures/supplementary/2010_002682_gt.png}
  }
  \subfigure{%
    \includegraphics[width=.18\columnwidth]{figures/supplementary/2010_002682_cnn.png}
  }
  \subfigure{%
    \includegraphics[width=.18\columnwidth]{figures/supplementary/2010_002682_gauss.png}
  }
  \subfigure{%
    \includegraphics[width=.18\columnwidth]{figures/supplementary/2010_002682_learnt.png}
  }\\
     \subfigure{%
    \includegraphics[width=.18\columnwidth]{figures/supplementary/2010_004789_given.jpg}
  }
  \subfigure{%
    \includegraphics[width=.18\columnwidth]{figures/supplementary/2010_004789_gt.png}
  }
  \subfigure{%
    \includegraphics[width=.18\columnwidth]{figures/supplementary/2010_004789_cnn.png}
  }
  \subfigure{%
    \includegraphics[width=.18\columnwidth]{figures/supplementary/2010_004789_gauss.png}
  }
  \subfigure{%
    \includegraphics[width=.18\columnwidth]{figures/supplementary/2010_004789_learnt.png}
  }\\
       \subfigure{%
    \includegraphics[width=.18\columnwidth]{figures/supplementary/2007_001311_given.jpg}
  }
  \subfigure{%
    \includegraphics[width=.18\columnwidth]{figures/supplementary/2007_001311_gt.png}
  }
  \subfigure{%
    \includegraphics[width=.18\columnwidth]{figures/supplementary/2007_001311_cnn.png}
  }
  \subfigure{%
    \includegraphics[width=.18\columnwidth]{figures/supplementary/2007_001311_gauss.png}
  }
  \subfigure{%
    \includegraphics[width=.18\columnwidth]{figures/supplementary/2007_001311_learnt.png}
  }\\
  \setcounter{subfigure}{0}
  \subfigure[Input]{%
    \includegraphics[width=.18\columnwidth]{figures/supplementary/2010_003531_given.jpg}
  }
  \subfigure[Ground Truth]{%
    \includegraphics[width=.18\columnwidth]{figures/supplementary/2010_003531_gt.png}
  }
  \subfigure[DeepLab]{%
    \includegraphics[width=.18\columnwidth]{figures/supplementary/2010_003531_cnn.png}
  }
  \subfigure[+GaussCRF]{%
    \includegraphics[width=.18\columnwidth]{figures/supplementary/2010_003531_gauss.png}
  }
  \subfigure[+LearnedCRF]{%
    \includegraphics[width=.18\columnwidth]{figures/supplementary/2010_003531_learnt.png}
  }
  \vspace{-0.3cm}
  \mycaption{Semantic Segmentation}{Example results of semantic segmentation.
  (c)~depicts the unary results before application of MF, (d)~after two steps of MF with Gaussian edge CRF potentials, (e)~after
  two steps of MF with learned edge CRF potentials.}
    \label{fig:semantic_visuals}
\end{figure*}


\definecolor{minc_1}{HTML}{771111}
\definecolor{minc_2}{HTML}{CAC690}
\definecolor{minc_3}{HTML}{EEEEEE}
\definecolor{minc_4}{HTML}{7C8FA6}
\definecolor{minc_5}{HTML}{597D31}
\definecolor{minc_6}{HTML}{104410}
\definecolor{minc_7}{HTML}{BB819C}
\definecolor{minc_8}{HTML}{D0CE48}
\definecolor{minc_9}{HTML}{622745}
\definecolor{minc_10}{HTML}{666666}
\definecolor{minc_11}{HTML}{D54A31}
\definecolor{minc_12}{HTML}{101044}
\definecolor{minc_13}{HTML}{444126}
\definecolor{minc_14}{HTML}{75D646}
\definecolor{minc_15}{HTML}{DD4348}
\definecolor{minc_16}{HTML}{5C8577}
\definecolor{minc_17}{HTML}{C78472}
\definecolor{minc_18}{HTML}{75D6D0}
\definecolor{minc_19}{HTML}{5B4586}
\definecolor{minc_20}{HTML}{C04393}
\definecolor{minc_21}{HTML}{D69948}
\definecolor{minc_22}{HTML}{7370D8}
\definecolor{minc_23}{HTML}{7A3622}
\definecolor{minc_24}{HTML}{000000}

\begin{figure*}[t]
  \centering
  \small{
  \fcolorbox{white}{minc_1}{\rule{0pt}{6pt}\rule{6pt}{0pt}} Brick~~
  \fcolorbox{white}{minc_2}{\rule{0pt}{6pt}\rule{6pt}{0pt}} Carpet~~
  \fcolorbox{white}{minc_3}{\rule{0pt}{6pt}\rule{6pt}{0pt}} Ceramic~~
  \fcolorbox{white}{minc_4}{\rule{0pt}{6pt}\rule{6pt}{0pt}} Fabric~~
  \fcolorbox{white}{minc_5}{\rule{0pt}{6pt}\rule{6pt}{0pt}} Foliage~~
  \fcolorbox{white}{minc_6}{\rule{0pt}{6pt}\rule{6pt}{0pt}} Food~~
  \fcolorbox{white}{minc_7}{\rule{0pt}{6pt}\rule{6pt}{0pt}} Glass~~
  \fcolorbox{white}{minc_8}{\rule{0pt}{6pt}\rule{6pt}{0pt}} Hair~~ \\
  \fcolorbox{white}{minc_9}{\rule{0pt}{6pt}\rule{6pt}{0pt}} Leather~~
  \fcolorbox{white}{minc_10}{\rule{0pt}{6pt}\rule{6pt}{0pt}} Metal~~
  \fcolorbox{white}{minc_11}{\rule{0pt}{6pt}\rule{6pt}{0pt}} Mirror~~
  \fcolorbox{white}{minc_12}{\rule{0pt}{6pt}\rule{6pt}{0pt}} Other~~
  \fcolorbox{white}{minc_13}{\rule{0pt}{6pt}\rule{6pt}{0pt}} Painted~~
  \fcolorbox{white}{minc_14}{\rule{0pt}{6pt}\rule{6pt}{0pt}} Paper~~
  \fcolorbox{white}{minc_15}{\rule{0pt}{6pt}\rule{6pt}{0pt}} Plastic~~\\
  \fcolorbox{white}{minc_16}{\rule{0pt}{6pt}\rule{6pt}{0pt}} Polished Stone~~
  \fcolorbox{white}{minc_17}{\rule{0pt}{6pt}\rule{6pt}{0pt}} Skin~~
  \fcolorbox{white}{minc_18}{\rule{0pt}{6pt}\rule{6pt}{0pt}} Sky~~
  \fcolorbox{white}{minc_19}{\rule{0pt}{6pt}\rule{6pt}{0pt}} Stone~~
  \fcolorbox{white}{minc_20}{\rule{0pt}{6pt}\rule{6pt}{0pt}} Tile~~
  \fcolorbox{white}{minc_21}{\rule{0pt}{6pt}\rule{6pt}{0pt}} Wallpaper~~
  \fcolorbox{white}{minc_22}{\rule{0pt}{6pt}\rule{6pt}{0pt}} Water~~
  \fcolorbox{white}{minc_23}{\rule{0pt}{6pt}\rule{6pt}{0pt}} Wood~~ \\
  }
  \subfigure{%
    \includegraphics[width=.18\columnwidth]{figures/supplementary/000010868_given.jpg}
  }
  \subfigure{%
    \includegraphics[width=.18\columnwidth]{figures/supplementary/000010868_gt.png}
  }
  \subfigure{%
    \includegraphics[width=.18\columnwidth]{figures/supplementary/000010868_cnn.png}
  }
  \subfigure{%
    \includegraphics[width=.18\columnwidth]{figures/supplementary/000010868_gauss.png}
  }
  \subfigure{%
    \includegraphics[width=.18\columnwidth]{figures/supplementary/000010868_learnt.png}
  }\\[-2ex]
  \subfigure{%
    \includegraphics[width=.18\columnwidth]{figures/supplementary/000006011_given.jpg}
  }
  \subfigure{%
    \includegraphics[width=.18\columnwidth]{figures/supplementary/000006011_gt.png}
  }
  \subfigure{%
    \includegraphics[width=.18\columnwidth]{figures/supplementary/000006011_cnn.png}
  }
  \subfigure{%
    \includegraphics[width=.18\columnwidth]{figures/supplementary/000006011_gauss.png}
  }
  \subfigure{%
    \includegraphics[width=.18\columnwidth]{figures/supplementary/000006011_learnt.png}
  }\\[-2ex]
    \subfigure{%
    \includegraphics[width=.18\columnwidth]{figures/supplementary/000008553_given.jpg}
  }
  \subfigure{%
    \includegraphics[width=.18\columnwidth]{figures/supplementary/000008553_gt.png}
  }
  \subfigure{%
    \includegraphics[width=.18\columnwidth]{figures/supplementary/000008553_cnn.png}
  }
  \subfigure{%
    \includegraphics[width=.18\columnwidth]{figures/supplementary/000008553_gauss.png}
  }
  \subfigure{%
    \includegraphics[width=.18\columnwidth]{figures/supplementary/000008553_learnt.png}
  }\\[-2ex]
   \subfigure{%
    \includegraphics[width=.18\columnwidth]{figures/supplementary/000009188_given.jpg}
  }
  \subfigure{%
    \includegraphics[width=.18\columnwidth]{figures/supplementary/000009188_gt.png}
  }
  \subfigure{%
    \includegraphics[width=.18\columnwidth]{figures/supplementary/000009188_cnn.png}
  }
  \subfigure{%
    \includegraphics[width=.18\columnwidth]{figures/supplementary/000009188_gauss.png}
  }
  \subfigure{%
    \includegraphics[width=.18\columnwidth]{figures/supplementary/000009188_learnt.png}
  }\\[-2ex]
  \setcounter{subfigure}{0}
  \subfigure[Input]{%
    \includegraphics[width=.18\columnwidth]{figures/supplementary/000023570_given.jpg}
  }
  \subfigure[Ground Truth]{%
    \includegraphics[width=.18\columnwidth]{figures/supplementary/000023570_gt.png}
  }
  \subfigure[DeepLab]{%
    \includegraphics[width=.18\columnwidth]{figures/supplementary/000023570_cnn.png}
  }
  \subfigure[+GaussCRF]{%
    \includegraphics[width=.18\columnwidth]{figures/supplementary/000023570_gauss.png}
  }
  \subfigure[+LearnedCRF]{%
    \includegraphics[width=.18\columnwidth]{figures/supplementary/000023570_learnt.png}
  }
  \mycaption{Material Segmentation}{Example results of material segmentation.
  (c)~depicts the unary results before application of MF, (d)~after two steps of MF with Gaussian edge CRF potentials, (e)~after two steps of MF with learned edge CRF potentials.}
    \label{fig:material_visuals-app2}
\end{figure*}


\begin{table*}[h]
\tiny
  \centering
    \begin{tabular}{L{2.3cm} L{2.25cm} C{1.5cm} C{0.7cm} C{0.6cm} C{0.7cm} C{0.7cm} C{0.7cm} C{1.6cm} C{0.6cm} C{0.6cm} C{0.6cm}}
      \toprule
& & & & & \multicolumn{3}{c}{\textbf{Data Statistics}} & \multicolumn{4}{c}{\textbf{Training Protocol}} \\

\textbf{Experiment} & \textbf{Feature Types} & \textbf{Feature Scales} & \textbf{Filter Size} & \textbf{Filter Nbr.} & \textbf{Train}  & \textbf{Val.} & \textbf{Test} & \textbf{Loss Type} & \textbf{LR} & \textbf{Batch} & \textbf{Epochs} \\
      \midrule
      \multicolumn{2}{c}{\textbf{Single Bilateral Filter Applications}} & & & & & & & & & \\
      \textbf{2$\times$ Color Upsampling} & Position$_{1}$, Intensity (3D) & 0.13, 0.17 & 65 & 2 & 10581 & 1449 & 1456 & MSE & 1e-06 & 200 & 94.5\\
      \textbf{4$\times$ Color Upsampling} & Position$_{1}$, Intensity (3D) & 0.06, 0.17 & 65 & 2 & 10581 & 1449 & 1456 & MSE & 1e-06 & 200 & 94.5\\
      \textbf{8$\times$ Color Upsampling} & Position$_{1}$, Intensity (3D) & 0.03, 0.17 & 65 & 2 & 10581 & 1449 & 1456 & MSE & 1e-06 & 200 & 94.5\\
      \textbf{16$\times$ Color Upsampling} & Position$_{1}$, Intensity (3D) & 0.02, 0.17 & 65 & 2 & 10581 & 1449 & 1456 & MSE & 1e-06 & 200 & 94.5\\
      \textbf{Depth Upsampling} & Position$_{1}$, Color (5D) & 0.05, 0.02 & 665 & 2 & 795 & 100 & 654 & MSE & 1e-07 & 50 & 251.6\\
      \textbf{Mesh Denoising} & Isomap (4D) & 46.00 & 63 & 2 & 1000 & 200 & 500 & MSE & 100 & 10 & 100.0 \\
      \midrule
      \multicolumn{2}{c}{\textbf{DenseCRF Applications}} & & & & & & & & &\\
      \multicolumn{2}{l}{\textbf{Semantic Segmentation}} & & & & & & & & &\\
      \textbf{- 1step MF} & Position$_{1}$, Color (5D); Position$_{1}$ (2D) & 0.01, 0.34; 0.34  & 665; 19  & 2; 2 & 10581 & 1449 & 1456 & Logistic & 0.1 & 5 & 1.4 \\
      \textbf{- 2step MF} & Position$_{1}$, Color (5D); Position$_{1}$ (2D) & 0.01, 0.34; 0.34 & 665; 19 & 2; 2 & 10581 & 1449 & 1456 & Logistic & 0.1 & 5 & 1.4 \\
      \textbf{- \textit{loose} 2step MF} & Position$_{1}$, Color (5D); Position$_{1}$ (2D) & 0.01, 0.34; 0.34 & 665; 19 & 2; 2 &10581 & 1449 & 1456 & Logistic & 0.1 & 5 & +1.9  \\ \\
      \multicolumn{2}{l}{\textbf{Material Segmentation}} & & & & & & & & &\\
      \textbf{- 1step MF} & Position$_{2}$, Lab-Color (5D) & 5.00, 0.05, 0.30  & 665 & 2 & 928 & 150 & 1798 & Weighted Logistic & 1e-04 & 24 & 2.6 \\
      \textbf{- 2step MF} & Position$_{2}$, Lab-Color (5D) & 5.00, 0.05, 0.30 & 665 & 2 & 928 & 150 & 1798 & Weighted Logistic & 1e-04 & 12 & +0.7 \\
      \textbf{- \textit{loose} 2step MF} & Position$_{2}$, Lab-Color (5D) & 5.00, 0.05, 0.30 & 665 & 2 & 928 & 150 & 1798 & Weighted Logistic & 1e-04 & 12 & +0.2\\
      \midrule
      \multicolumn{2}{c}{\textbf{Neural Network Applications}} & & & & & & & & &\\
      \textbf{Tiles: CNN-9$\times$9} & - & - & 81 & 4 & 10000 & 1000 & 1000 & Logistic & 0.01 & 100 & 500.0 \\
      \textbf{Tiles: CNN-13$\times$13} & - & - & 169 & 6 & 10000 & 1000 & 1000 & Logistic & 0.01 & 100 & 500.0 \\
      \textbf{Tiles: CNN-17$\times$17} & - & - & 289 & 8 & 10000 & 1000 & 1000 & Logistic & 0.01 & 100 & 500.0 \\
      \textbf{Tiles: CNN-21$\times$21} & - & - & 441 & 10 & 10000 & 1000 & 1000 & Logistic & 0.01 & 100 & 500.0 \\
      \textbf{Tiles: BNN} & Position$_{1}$, Color (5D) & 0.05, 0.04 & 63 & 1 & 10000 & 1000 & 1000 & Logistic & 0.01 & 100 & 30.0 \\
      \textbf{LeNet} & - & - & 25 & 2 & 5490 & 1098 & 1647 & Logistic & 0.1 & 100 & 182.2 \\
      \textbf{Crop-LeNet} & - & - & 25 & 2 & 5490 & 1098 & 1647 & Logistic & 0.1 & 100 & 182.2 \\
      \textbf{BNN-LeNet} & Position$_{2}$ (2D) & 20.00 & 7 & 1 & 5490 & 1098 & 1647 & Logistic & 0.1 & 100 & 182.2 \\
      \textbf{DeepCNet} & - & - & 9 & 1 & 5490 & 1098 & 1647 & Logistic & 0.1 & 100 & 182.2 \\
      \textbf{Crop-DeepCNet} & - & - & 9 & 1 & 5490 & 1098 & 1647 & Logistic & 0.1 & 100 & 182.2 \\
      \textbf{BNN-DeepCNet} & Position$_{2}$ (2D) & 40.00  & 7 & 1 & 5490 & 1098 & 1647 & Logistic & 0.1 & 100 & 182.2 \\
      \bottomrule
      \\
    \end{tabular}
    \mycaption{Experiment Protocols} {Experiment protocols for the different experiments presented in this work. \textbf{Feature Types}:
    Feature spaces used for the bilateral convolutions. Position$_1$ corresponds to un-normalized pixel positions whereas Position$_2$ corresponds
    to pixel positions normalized to $[0,1]$ with respect to the given image. \textbf{Feature Scales}: Cross-validated scales for the features used.
     \textbf{Filter Size}: Number of elements in the filter that is being learned. \textbf{Filter Nbr.}: Half-width of the filter. \textbf{Train},
     \textbf{Val.} and \textbf{Test} corresponds to the number of train, validation and test images used in the experiment. \textbf{Loss Type}: Type
     of loss used for back-propagation. ``MSE'' corresponds to Euclidean mean squared error loss and ``Logistic'' corresponds to multinomial logistic
     loss. ``Weighted Logistic'' is the class-weighted multinomial logistic loss. We weighted the loss with inverse class probability for material
     segmentation task due to the small availability of training data with class imbalance. \textbf{LR}: Fixed learning rate used in stochastic gradient
     descent. \textbf{Batch}: Number of images used in one parameter update step. \textbf{Epochs}: Number of training epochs. In all the experiments,
     we used fixed momentum of 0.9 and weight decay of 0.0005 for stochastic gradient descent. ```Color Upsampling'' experiments in this Table corresponds
     to those performed on Pascal VOC12 dataset images. For all experiments using Pascal VOC12 images, we use extended
     training segmentation dataset available from~\cite{hariharan2011moredata}, and used standard validation and test splits
     from the main dataset~\cite{voc2012segmentation}.}
  \label{tbl:parameters}
\end{table*}

\clearpage

\section{Parameters and Additional Results for Video Propagation Networks}

In this Section, we present experiment protocols and additional qualitative results for experiments
on video object segmentation, semantic video segmentation and video color
propagation. Table~\ref{tbl:parameters_supp} shows the feature scales and other parameters used in different experiments.
Figures~\ref{fig:video_seg_pos_supp} show some qualitative results on video object segmentation
with some failure cases in Fig.~\ref{fig:video_seg_neg_supp}.
Figure~\ref{fig:semantic_visuals_supp} shows some qualitative results on semantic video segmentation and
Fig.~\ref{fig:color_visuals_supp} shows results on video color propagation.

\newcolumntype{L}[1]{>{\raggedright\let\newline\\\arraybackslash\hspace{0pt}}b{#1}}
\newcolumntype{C}[1]{>{\centering\let\newline\\\arraybackslash\hspace{0pt}}b{#1}}
\newcolumntype{R}[1]{>{\raggedleft\let\newline\\\arraybackslash\hspace{0pt}}b{#1}}

\begin{table*}[h]
\tiny
  \centering
    \begin{tabular}{L{3.0cm} L{2.4cm} L{2.8cm} L{2.8cm} C{0.5cm} C{1.0cm} L{1.2cm}}
      \toprule
\textbf{Experiment} & \textbf{Feature Type} & \textbf{Feature Scale-1, $\Lambda_a$} & \textbf{Feature Scale-2, $\Lambda_b$} & \textbf{$\alpha$} & \textbf{Input Frames} & \textbf{Loss Type} \\
      \midrule
      \textbf{Video Object Segmentation} & ($x,y,Y,Cb,Cr,t$) & (0.02,0.02,0.07,0.4,0.4,0.01) & (0.03,0.03,0.09,0.5,0.5,0.2) & 0.5 & 9 & Logistic\\
      \midrule
      \textbf{Semantic Video Segmentation} & & & & & \\
      \textbf{with CNN1~\cite{yu2015multi}-NoFlow} & ($x,y,R,G,B,t$) & (0.08,0.08,0.2,0.2,0.2,0.04) & (0.11,0.11,0.2,0.2,0.2,0.04) & 0.5 & 3 & Logistic \\
      \textbf{with CNN1~\cite{yu2015multi}-Flow} & ($x+u_x,y+u_y,R,G,B,t$) & (0.11,0.11,0.14,0.14,0.14,0.03) & (0.08,0.08,0.12,0.12,0.12,0.01) & 0.65 & 3 & Logistic\\
      \textbf{with CNN2~\cite{richter2016playing}-Flow} & ($x+u_x,y+u_y,R,G,B,t$) & (0.08,0.08,0.2,0.2,0.2,0.04) & (0.09,0.09,0.25,0.25,0.25,0.03) & 0.5 & 4 & Logistic\\
      \midrule
      \textbf{Video Color Propagation} & ($x,y,I,t$)  & (0.04,0.04,0.2,0.04) & No second kernel & 1 & 4 & MSE\\
      \bottomrule
      \\
    \end{tabular}
    \mycaption{Experiment Protocols} {Experiment protocols for the different experiments presented in this work. \textbf{Feature Types}:
    Feature spaces used for the bilateral convolutions, with position ($x,y$) and color
    ($R,G,B$ or $Y,Cb,Cr$) features $\in [0,255]$. $u_x$, $u_y$ denotes optical flow with respect
    to the present frame and $I$ denotes grayscale intensity.
    \textbf{Feature Scales ($\Lambda_a, \Lambda_b$)}: Cross-validated scales for the features used.
    \textbf{$\alpha$}: Exponential time decay for the input frames.
    \textbf{Input Frames}: Number of input frames for VPN.
    \textbf{Loss Type}: Type
     of loss used for back-propagation. ``MSE'' corresponds to Euclidean mean squared error loss and ``Logistic'' corresponds to multinomial logistic loss.}
  \label{tbl:parameters_supp}
\end{table*}

% \begin{figure}[th!]
% \begin{center}
%   \centerline{\includegraphics[width=\textwidth]{figures/video_seg_visuals_supp_small.pdf}}
%     \mycaption{Video Object Segmentation}
%     {Shown are the different frames in example videos with the corresponding
%     ground truth (GT) masks, predictions from BVS~\cite{marki2016bilateral},
%     OFL~\cite{tsaivideo}, VPN (VPN-Stage2) and VPN-DLab (VPN-DeepLab) models.}
%     \label{fig:video_seg_small_supp}
% \end{center}
% \vspace{-1.0cm}
% \end{figure}

\begin{figure}[th!]
\begin{center}
  \centerline{\includegraphics[width=0.7\textwidth]{figures/video_seg_visuals_supp_positive.pdf}}
    \mycaption{Video Object Segmentation}
    {Shown are the different frames in example videos with the corresponding
    ground truth (GT) masks, predictions from BVS~\cite{marki2016bilateral},
    OFL~\cite{tsaivideo}, VPN (VPN-Stage2) and VPN-DLab (VPN-DeepLab) models.}
    \label{fig:video_seg_pos_supp}
\end{center}
\vspace{-1.0cm}
\end{figure}

\begin{figure}[th!]
\begin{center}
  \centerline{\includegraphics[width=0.7\textwidth]{figures/video_seg_visuals_supp_negative.pdf}}
    \mycaption{Failure Cases for Video Object Segmentation}
    {Shown are the different frames in example videos with the corresponding
    ground truth (GT) masks, predictions from BVS~\cite{marki2016bilateral},
    OFL~\cite{tsaivideo}, VPN (VPN-Stage2) and VPN-DLab (VPN-DeepLab) models.}
    \label{fig:video_seg_neg_supp}
\end{center}
\vspace{-1.0cm}
\end{figure}

\begin{figure}[th!]
\begin{center}
  \centerline{\includegraphics[width=0.9\textwidth]{figures/supp_semantic_visual.pdf}}
    \mycaption{Semantic Video Segmentation}
    {Input video frames and the corresponding ground truth (GT)
    segmentation together with the predictions of CNN~\cite{yu2015multi} and with
    VPN-Flow.}
    \label{fig:semantic_visuals_supp}
\end{center}
\vspace{-0.7cm}
\end{figure}

\begin{figure}[th!]
\begin{center}
  \centerline{\includegraphics[width=\textwidth]{figures/colorization_visuals_supp.pdf}}
  \mycaption{Video Color Propagation}
  {Input grayscale video frames and corresponding ground-truth (GT) color images
  together with color predictions of Levin et al.~\cite{levin2004colorization} and VPN-Stage1 models.}
  \label{fig:color_visuals_supp}
\end{center}
\vspace{-0.7cm}
\end{figure}

\clearpage

\section{Additional Material for Bilateral Inception Networks}
\label{sec:binception-app}

In this section of the Appendix, we first discuss the use of approximate bilateral
filtering in BI modules (Sec.~\ref{sec:lattice}).
Later, we present some qualitative results using different models for the approach presented in
Chapter~\ref{chap:binception} (Sec.~\ref{sec:qualitative-app}).

\subsection{Approximate Bilateral Filtering}
\label{sec:lattice}

The bilateral inception module presented in Chapter~\ref{chap:binception} computes a matrix-vector
product between a Gaussian filter $K$ and a vector of activations $\bz_c$.
Bilateral filtering is an important operation and many algorithmic techniques have been
proposed to speed-up this operation~\cite{paris2006fast,adams2010fast,gastal2011domain}.
In the main paper we opted to implement what can be considered the
brute-force variant of explicitly constructing $K$ and then using BLAS to compute the
matrix-vector product. This resulted in a few millisecond operation.
The explicit way to compute is possible due to the
reduction to super-pixels, e.g., it would not work for DenseCRF variants
that operate on the full image resolution.

Here, we present experiments where we use the fast approximate bilateral filtering
algorithm of~\cite{adams2010fast}, which is also used in Chapter~\ref{chap:bnn}
for learning sparse high dimensional filters. This
choice allows for larger dimensions of matrix-vector multiplication. The reason for choosing
the explicit multiplication in Chapter~\ref{chap:binception} was that it was computationally faster.
For the small sizes of the involved matrices and vectors, the explicit computation is sufficient and we had no
GPU implementation of an approximate technique that matched this runtime. Also it
is conceptually easier and the gradient to the feature transformations ($\Lambda \mathbf{f}$) is
obtained using standard matrix calculus.

\subsubsection{Experiments}

We modified the existing segmentation architectures analogous to those in Chapter~\ref{chap:binception}.
The main difference is that, here, the inception modules use the lattice
approximation~\cite{adams2010fast} to compute the bilateral filtering.
Using the lattice approximation did not allow us to back-propagate through feature transformations ($\Lambda$)
and thus we used hand-specified feature scales as will be explained later.
Specifically, we take CNN architectures from the works
of~\cite{chen2014semantic,zheng2015conditional,bell2015minc} and insert the BI modules between
the spatial FC layers.
We use superpixels from~\cite{DollarICCV13edges}
for all the experiments with the lattice approximation. Experiments are
performed using Caffe neural network framework~\cite{jia2014caffe}.

\begin{table}
  \small
  \centering
  \begin{tabular}{p{5.5cm}>{\raggedright\arraybackslash}p{1.4cm}>{\centering\arraybackslash}p{2.2cm}}
    \toprule
		\textbf{Model} & \emph{IoU} & \emph{Runtime}(ms) \\
    \midrule

    %%%%%%%%%%%% Scores computed by us)%%%%%%%%%%%%
		\deeplablargefov & 68.9 & 145ms\\
    \midrule
    \bi{7}{2}-\bi{8}{10}& \textbf{73.8} & +600 \\
    \midrule
    \deeplablargefovcrf~\cite{chen2014semantic} & 72.7 & +830\\
    \deeplabmsclargefovcrf~\cite{chen2014semantic} & \textbf{73.6} & +880\\
    DeepLab-EdgeNet~\cite{chen2015semantic} & 71.7 & +30\\
    DeepLab-EdgeNet-CRF~\cite{chen2015semantic} & \textbf{73.6} & +860\\
  \bottomrule \\
  \end{tabular}
  \mycaption{Semantic Segmentation using the DeepLab model}
  {IoU scores on the Pascal VOC12 segmentation test dataset
  with different models and our modified inception model.
  Also shown are the corresponding runtimes in milliseconds. Runtimes
  also include superpixel computations (300 ms with Dollar superpixels~\cite{DollarICCV13edges})}
  \label{tab:largefovresults}
\end{table}

\paragraph{Semantic Segmentation}
The experiments in this section use the Pascal VOC12 segmentation dataset~\cite{voc2012segmentation} with 21 object classes and the images have a maximum resolution of 0.25 megapixels.
For all experiments on VOC12, we train using the extended training set of
10581 images collected by~\cite{hariharan2011moredata}.
We modified the \deeplab~network architecture of~\cite{chen2014semantic} and
the CRFasRNN architecture from~\cite{zheng2015conditional} which uses a CNN with
deconvolution layers followed by DenseCRF trained end-to-end.

\paragraph{DeepLab Model}\label{sec:deeplabmodel}
We experimented with the \bi{7}{2}-\bi{8}{10} inception model.
Results using the~\deeplab~model are summarized in Tab.~\ref{tab:largefovresults}.
Although we get similar improvements with inception modules as with the
explicit kernel computation, using lattice approximation is slower.

\begin{table}
  \small
  \centering
  \begin{tabular}{p{6.4cm}>{\raggedright\arraybackslash}p{1.8cm}>{\raggedright\arraybackslash}p{1.8cm}}
    \toprule
    \textbf{Model} & \emph{IoU (Val)} & \emph{IoU (Test)}\\
    \midrule
    %%%%%%%%%%%% Scores computed by us)%%%%%%%%%%%%
    CNN &  67.5 & - \\
    \deconv (CNN+Deconvolutions) & 69.8 & 72.0 \\
    \midrule
    \bi{3}{6}-\bi{4}{6}-\bi{7}{2}-\bi{8}{6}& 71.9 & - \\
    \bi{3}{6}-\bi{4}{6}-\bi{7}{2}-\bi{8}{6}-\gi{6}& 73.6 &  \href{http://host.robots.ox.ac.uk:8080/anonymous/VOTV5E.html}{\textbf{75.2}}\\
    \midrule
    \deconvcrf (CRF-RNN)~\cite{zheng2015conditional} & 73.0 & 74.7\\
    Context-CRF-RNN~\cite{yu2015multi} & ~~ - ~ & \textbf{75.3} \\
    \bottomrule \\
  \end{tabular}
  \mycaption{Semantic Segmentation using the CRFasRNN model}{IoU score corresponding to different models
  on Pascal VOC12 reduced validation / test segmentation dataset. The reduced validation set consists of 346 images
  as used in~\cite{zheng2015conditional} where we adapted the model from.}
  \label{tab:deconvresults-app}
\end{table}

\paragraph{CRFasRNN Model}\label{sec:deepinception}
We add BI modules after score-pool3, score-pool4, \fc{7} and \fc{8} $1\times1$ convolution layers
resulting in the \bi{3}{6}-\bi{4}{6}-\bi{7}{2}-\bi{8}{6}
model and also experimented with another variant where $BI_8$ is followed by another inception
module, G$(6)$, with 6 Gaussian kernels.
Note that here also we discarded both deconvolution and DenseCRF parts of the original model~\cite{zheng2015conditional}
and inserted the BI modules in the base CNN and found similar improvements compared to the inception modules with explicit
kernel computaion. See Tab.~\ref{tab:deconvresults-app} for results on the CRFasRNN model.

\paragraph{Material Segmentation}
Table~\ref{tab:mincresults-app} shows the results on the MINC dataset~\cite{bell2015minc}
obtained by modifying the AlexNet architecture with our inception modules. We observe
similar improvements as with explicit kernel construction.
For this model, we do not provide any learned setup due to very limited segment training
data. The weights to combine outputs in the bilateral inception layer are
found by validation on the validation set.

\begin{table}[t]
  \small
  \centering
  \begin{tabular}{p{3.5cm}>{\centering\arraybackslash}p{4.0cm}}
    \toprule
    \textbf{Model} & Class / Total accuracy\\
    \midrule

    %%%%%%%%%%%% Scores computed by us)%%%%%%%%%%%%
    AlexNet CNN & 55.3 / 58.9 \\
    \midrule
    \bi{7}{2}-\bi{8}{6}& 68.5 / 71.8 \\
    \bi{7}{2}-\bi{8}{6}-G$(6)$& 67.6 / 73.1 \\
    \midrule
    AlexNet-CRF & 65.5 / 71.0 \\
    \bottomrule \\
  \end{tabular}
  \mycaption{Material Segmentation using AlexNet}{Pixel accuracy of different models on
  the MINC material segmentation test dataset~\cite{bell2015minc}.}
  \label{tab:mincresults-app}
\end{table}

\paragraph{Scales of Bilateral Inception Modules}
\label{sec:scales}

Unlike the explicit kernel technique presented in the main text (Chapter~\ref{chap:binception}),
we didn't back-propagate through feature transformation ($\Lambda$)
using the approximate bilateral filter technique.
So, the feature scales are hand-specified and validated, which are as follows.
The optimal scale values for the \bi{7}{2}-\bi{8}{2} model are found by validation for the best performance which are
$\sigma_{xy}$ = (0.1, 0.1) for the spatial (XY) kernel and $\sigma_{rgbxy}$ = (0.1, 0.1, 0.1, 0.01, 0.01) for color and position (RGBXY)  kernel.
Next, as more kernels are added to \bi{8}{2}, we set scales to be $\alpha$*($\sigma_{xy}$, $\sigma_{rgbxy}$).
The value of $\alpha$ is chosen as  1, 0.5, 0.1, 0.05, 0.1, at uniform interval, for the \bi{8}{10} bilateral inception module.


\subsection{Qualitative Results}
\label{sec:qualitative-app}

In this section, we present more qualitative results obtained using the BI module with explicit
kernel computation technique presented in Chapter~\ref{chap:binception}. Results on the Pascal VOC12
dataset~\cite{voc2012segmentation} using the DeepLab-LargeFOV model are shown in Fig.~\ref{fig:semantic_visuals-app},
followed by the results on MINC dataset~\cite{bell2015minc}
in Fig.~\ref{fig:material_visuals-app} and on
Cityscapes dataset~\cite{Cordts2015Cvprw} in Fig.~\ref{fig:street_visuals-app}.


\definecolor{voc_1}{RGB}{0, 0, 0}
\definecolor{voc_2}{RGB}{128, 0, 0}
\definecolor{voc_3}{RGB}{0, 128, 0}
\definecolor{voc_4}{RGB}{128, 128, 0}
\definecolor{voc_5}{RGB}{0, 0, 128}
\definecolor{voc_6}{RGB}{128, 0, 128}
\definecolor{voc_7}{RGB}{0, 128, 128}
\definecolor{voc_8}{RGB}{128, 128, 128}
\definecolor{voc_9}{RGB}{64, 0, 0}
\definecolor{voc_10}{RGB}{192, 0, 0}
\definecolor{voc_11}{RGB}{64, 128, 0}
\definecolor{voc_12}{RGB}{192, 128, 0}
\definecolor{voc_13}{RGB}{64, 0, 128}
\definecolor{voc_14}{RGB}{192, 0, 128}
\definecolor{voc_15}{RGB}{64, 128, 128}
\definecolor{voc_16}{RGB}{192, 128, 128}
\definecolor{voc_17}{RGB}{0, 64, 0}
\definecolor{voc_18}{RGB}{128, 64, 0}
\definecolor{voc_19}{RGB}{0, 192, 0}
\definecolor{voc_20}{RGB}{128, 192, 0}
\definecolor{voc_21}{RGB}{0, 64, 128}
\definecolor{voc_22}{RGB}{128, 64, 128}

\begin{figure*}[!ht]
  \small
  \centering
  \fcolorbox{white}{voc_1}{\rule{0pt}{4pt}\rule{4pt}{0pt}} Background~~
  \fcolorbox{white}{voc_2}{\rule{0pt}{4pt}\rule{4pt}{0pt}} Aeroplane~~
  \fcolorbox{white}{voc_3}{\rule{0pt}{4pt}\rule{4pt}{0pt}} Bicycle~~
  \fcolorbox{white}{voc_4}{\rule{0pt}{4pt}\rule{4pt}{0pt}} Bird~~
  \fcolorbox{white}{voc_5}{\rule{0pt}{4pt}\rule{4pt}{0pt}} Boat~~
  \fcolorbox{white}{voc_6}{\rule{0pt}{4pt}\rule{4pt}{0pt}} Bottle~~
  \fcolorbox{white}{voc_7}{\rule{0pt}{4pt}\rule{4pt}{0pt}} Bus~~
  \fcolorbox{white}{voc_8}{\rule{0pt}{4pt}\rule{4pt}{0pt}} Car~~\\
  \fcolorbox{white}{voc_9}{\rule{0pt}{4pt}\rule{4pt}{0pt}} Cat~~
  \fcolorbox{white}{voc_10}{\rule{0pt}{4pt}\rule{4pt}{0pt}} Chair~~
  \fcolorbox{white}{voc_11}{\rule{0pt}{4pt}\rule{4pt}{0pt}} Cow~~
  \fcolorbox{white}{voc_12}{\rule{0pt}{4pt}\rule{4pt}{0pt}} Dining Table~~
  \fcolorbox{white}{voc_13}{\rule{0pt}{4pt}\rule{4pt}{0pt}} Dog~~
  \fcolorbox{white}{voc_14}{\rule{0pt}{4pt}\rule{4pt}{0pt}} Horse~~
  \fcolorbox{white}{voc_15}{\rule{0pt}{4pt}\rule{4pt}{0pt}} Motorbike~~
  \fcolorbox{white}{voc_16}{\rule{0pt}{4pt}\rule{4pt}{0pt}} Person~~\\
  \fcolorbox{white}{voc_17}{\rule{0pt}{4pt}\rule{4pt}{0pt}} Potted Plant~~
  \fcolorbox{white}{voc_18}{\rule{0pt}{4pt}\rule{4pt}{0pt}} Sheep~~
  \fcolorbox{white}{voc_19}{\rule{0pt}{4pt}\rule{4pt}{0pt}} Sofa~~
  \fcolorbox{white}{voc_20}{\rule{0pt}{4pt}\rule{4pt}{0pt}} Train~~
  \fcolorbox{white}{voc_21}{\rule{0pt}{4pt}\rule{4pt}{0pt}} TV monitor~~\\


  \subfigure{%
    \includegraphics[width=.15\columnwidth]{figures/supplementary/2008_001308_given.png}
  }
  \subfigure{%
    \includegraphics[width=.15\columnwidth]{figures/supplementary/2008_001308_sp.png}
  }
  \subfigure{%
    \includegraphics[width=.15\columnwidth]{figures/supplementary/2008_001308_gt.png}
  }
  \subfigure{%
    \includegraphics[width=.15\columnwidth]{figures/supplementary/2008_001308_cnn.png}
  }
  \subfigure{%
    \includegraphics[width=.15\columnwidth]{figures/supplementary/2008_001308_crf.png}
  }
  \subfigure{%
    \includegraphics[width=.15\columnwidth]{figures/supplementary/2008_001308_ours.png}
  }\\[-2ex]


  \subfigure{%
    \includegraphics[width=.15\columnwidth]{figures/supplementary/2008_001821_given.png}
  }
  \subfigure{%
    \includegraphics[width=.15\columnwidth]{figures/supplementary/2008_001821_sp.png}
  }
  \subfigure{%
    \includegraphics[width=.15\columnwidth]{figures/supplementary/2008_001821_gt.png}
  }
  \subfigure{%
    \includegraphics[width=.15\columnwidth]{figures/supplementary/2008_001821_cnn.png}
  }
  \subfigure{%
    \includegraphics[width=.15\columnwidth]{figures/supplementary/2008_001821_crf.png}
  }
  \subfigure{%
    \includegraphics[width=.15\columnwidth]{figures/supplementary/2008_001821_ours.png}
  }\\[-2ex]



  \subfigure{%
    \includegraphics[width=.15\columnwidth]{figures/supplementary/2008_004612_given.png}
  }
  \subfigure{%
    \includegraphics[width=.15\columnwidth]{figures/supplementary/2008_004612_sp.png}
  }
  \subfigure{%
    \includegraphics[width=.15\columnwidth]{figures/supplementary/2008_004612_gt.png}
  }
  \subfigure{%
    \includegraphics[width=.15\columnwidth]{figures/supplementary/2008_004612_cnn.png}
  }
  \subfigure{%
    \includegraphics[width=.15\columnwidth]{figures/supplementary/2008_004612_crf.png}
  }
  \subfigure{%
    \includegraphics[width=.15\columnwidth]{figures/supplementary/2008_004612_ours.png}
  }\\[-2ex]


  \subfigure{%
    \includegraphics[width=.15\columnwidth]{figures/supplementary/2009_001008_given.png}
  }
  \subfigure{%
    \includegraphics[width=.15\columnwidth]{figures/supplementary/2009_001008_sp.png}
  }
  \subfigure{%
    \includegraphics[width=.15\columnwidth]{figures/supplementary/2009_001008_gt.png}
  }
  \subfigure{%
    \includegraphics[width=.15\columnwidth]{figures/supplementary/2009_001008_cnn.png}
  }
  \subfigure{%
    \includegraphics[width=.15\columnwidth]{figures/supplementary/2009_001008_crf.png}
  }
  \subfigure{%
    \includegraphics[width=.15\columnwidth]{figures/supplementary/2009_001008_ours.png}
  }\\[-2ex]




  \subfigure{%
    \includegraphics[width=.15\columnwidth]{figures/supplementary/2009_004497_given.png}
  }
  \subfigure{%
    \includegraphics[width=.15\columnwidth]{figures/supplementary/2009_004497_sp.png}
  }
  \subfigure{%
    \includegraphics[width=.15\columnwidth]{figures/supplementary/2009_004497_gt.png}
  }
  \subfigure{%
    \includegraphics[width=.15\columnwidth]{figures/supplementary/2009_004497_cnn.png}
  }
  \subfigure{%
    \includegraphics[width=.15\columnwidth]{figures/supplementary/2009_004497_crf.png}
  }
  \subfigure{%
    \includegraphics[width=.15\columnwidth]{figures/supplementary/2009_004497_ours.png}
  }\\[-2ex]



  \setcounter{subfigure}{0}
  \subfigure[\scriptsize Input]{%
    \includegraphics[width=.15\columnwidth]{figures/supplementary/2010_001327_given.png}
  }
  \subfigure[\scriptsize Superpixels]{%
    \includegraphics[width=.15\columnwidth]{figures/supplementary/2010_001327_sp.png}
  }
  \subfigure[\scriptsize GT]{%
    \includegraphics[width=.15\columnwidth]{figures/supplementary/2010_001327_gt.png}
  }
  \subfigure[\scriptsize Deeplab]{%
    \includegraphics[width=.15\columnwidth]{figures/supplementary/2010_001327_cnn.png}
  }
  \subfigure[\scriptsize +DenseCRF]{%
    \includegraphics[width=.15\columnwidth]{figures/supplementary/2010_001327_crf.png}
  }
  \subfigure[\scriptsize Using BI]{%
    \includegraphics[width=.15\columnwidth]{figures/supplementary/2010_001327_ours.png}
  }
  \mycaption{Semantic Segmentation}{Example results of semantic segmentation
  on the Pascal VOC12 dataset.
  (d)~depicts the DeepLab CNN result, (e)~CNN + 10 steps of mean-field inference,
  (f~result obtained with bilateral inception (BI) modules (\bi{6}{2}+\bi{7}{6}) between \fc~layers.}
  \label{fig:semantic_visuals-app}
\end{figure*}


\definecolor{minc_1}{HTML}{771111}
\definecolor{minc_2}{HTML}{CAC690}
\definecolor{minc_3}{HTML}{EEEEEE}
\definecolor{minc_4}{HTML}{7C8FA6}
\definecolor{minc_5}{HTML}{597D31}
\definecolor{minc_6}{HTML}{104410}
\definecolor{minc_7}{HTML}{BB819C}
\definecolor{minc_8}{HTML}{D0CE48}
\definecolor{minc_9}{HTML}{622745}
\definecolor{minc_10}{HTML}{666666}
\definecolor{minc_11}{HTML}{D54A31}
\definecolor{minc_12}{HTML}{101044}
\definecolor{minc_13}{HTML}{444126}
\definecolor{minc_14}{HTML}{75D646}
\definecolor{minc_15}{HTML}{DD4348}
\definecolor{minc_16}{HTML}{5C8577}
\definecolor{minc_17}{HTML}{C78472}
\definecolor{minc_18}{HTML}{75D6D0}
\definecolor{minc_19}{HTML}{5B4586}
\definecolor{minc_20}{HTML}{C04393}
\definecolor{minc_21}{HTML}{D69948}
\definecolor{minc_22}{HTML}{7370D8}
\definecolor{minc_23}{HTML}{7A3622}
\definecolor{minc_24}{HTML}{000000}

\begin{figure*}[!ht]
  \small % scriptsize
  \centering
  \fcolorbox{white}{minc_1}{\rule{0pt}{4pt}\rule{4pt}{0pt}} Brick~~
  \fcolorbox{white}{minc_2}{\rule{0pt}{4pt}\rule{4pt}{0pt}} Carpet~~
  \fcolorbox{white}{minc_3}{\rule{0pt}{4pt}\rule{4pt}{0pt}} Ceramic~~
  \fcolorbox{white}{minc_4}{\rule{0pt}{4pt}\rule{4pt}{0pt}} Fabric~~
  \fcolorbox{white}{minc_5}{\rule{0pt}{4pt}\rule{4pt}{0pt}} Foliage~~
  \fcolorbox{white}{minc_6}{\rule{0pt}{4pt}\rule{4pt}{0pt}} Food~~
  \fcolorbox{white}{minc_7}{\rule{0pt}{4pt}\rule{4pt}{0pt}} Glass~~
  \fcolorbox{white}{minc_8}{\rule{0pt}{4pt}\rule{4pt}{0pt}} Hair~~\\
  \fcolorbox{white}{minc_9}{\rule{0pt}{4pt}\rule{4pt}{0pt}} Leather~~
  \fcolorbox{white}{minc_10}{\rule{0pt}{4pt}\rule{4pt}{0pt}} Metal~~
  \fcolorbox{white}{minc_11}{\rule{0pt}{4pt}\rule{4pt}{0pt}} Mirror~~
  \fcolorbox{white}{minc_12}{\rule{0pt}{4pt}\rule{4pt}{0pt}} Other~~
  \fcolorbox{white}{minc_13}{\rule{0pt}{4pt}\rule{4pt}{0pt}} Painted~~
  \fcolorbox{white}{minc_14}{\rule{0pt}{4pt}\rule{4pt}{0pt}} Paper~~
  \fcolorbox{white}{minc_15}{\rule{0pt}{4pt}\rule{4pt}{0pt}} Plastic~~\\
  \fcolorbox{white}{minc_16}{\rule{0pt}{4pt}\rule{4pt}{0pt}} Polished Stone~~
  \fcolorbox{white}{minc_17}{\rule{0pt}{4pt}\rule{4pt}{0pt}} Skin~~
  \fcolorbox{white}{minc_18}{\rule{0pt}{4pt}\rule{4pt}{0pt}} Sky~~
  \fcolorbox{white}{minc_19}{\rule{0pt}{4pt}\rule{4pt}{0pt}} Stone~~
  \fcolorbox{white}{minc_20}{\rule{0pt}{4pt}\rule{4pt}{0pt}} Tile~~
  \fcolorbox{white}{minc_21}{\rule{0pt}{4pt}\rule{4pt}{0pt}} Wallpaper~~
  \fcolorbox{white}{minc_22}{\rule{0pt}{4pt}\rule{4pt}{0pt}} Water~~
  \fcolorbox{white}{minc_23}{\rule{0pt}{4pt}\rule{4pt}{0pt}} Wood~~\\
  \subfigure{%
    \includegraphics[width=.15\columnwidth]{figures/supplementary/000008468_given.png}
  }
  \subfigure{%
    \includegraphics[width=.15\columnwidth]{figures/supplementary/000008468_sp.png}
  }
  \subfigure{%
    \includegraphics[width=.15\columnwidth]{figures/supplementary/000008468_gt.png}
  }
  \subfigure{%
    \includegraphics[width=.15\columnwidth]{figures/supplementary/000008468_cnn.png}
  }
  \subfigure{%
    \includegraphics[width=.15\columnwidth]{figures/supplementary/000008468_crf.png}
  }
  \subfigure{%
    \includegraphics[width=.15\columnwidth]{figures/supplementary/000008468_ours.png}
  }\\[-2ex]

  \subfigure{%
    \includegraphics[width=.15\columnwidth]{figures/supplementary/000009053_given.png}
  }
  \subfigure{%
    \includegraphics[width=.15\columnwidth]{figures/supplementary/000009053_sp.png}
  }
  \subfigure{%
    \includegraphics[width=.15\columnwidth]{figures/supplementary/000009053_gt.png}
  }
  \subfigure{%
    \includegraphics[width=.15\columnwidth]{figures/supplementary/000009053_cnn.png}
  }
  \subfigure{%
    \includegraphics[width=.15\columnwidth]{figures/supplementary/000009053_crf.png}
  }
  \subfigure{%
    \includegraphics[width=.15\columnwidth]{figures/supplementary/000009053_ours.png}
  }\\[-2ex]




  \subfigure{%
    \includegraphics[width=.15\columnwidth]{figures/supplementary/000014977_given.png}
  }
  \subfigure{%
    \includegraphics[width=.15\columnwidth]{figures/supplementary/000014977_sp.png}
  }
  \subfigure{%
    \includegraphics[width=.15\columnwidth]{figures/supplementary/000014977_gt.png}
  }
  \subfigure{%
    \includegraphics[width=.15\columnwidth]{figures/supplementary/000014977_cnn.png}
  }
  \subfigure{%
    \includegraphics[width=.15\columnwidth]{figures/supplementary/000014977_crf.png}
  }
  \subfigure{%
    \includegraphics[width=.15\columnwidth]{figures/supplementary/000014977_ours.png}
  }\\[-2ex]


  \subfigure{%
    \includegraphics[width=.15\columnwidth]{figures/supplementary/000022922_given.png}
  }
  \subfigure{%
    \includegraphics[width=.15\columnwidth]{figures/supplementary/000022922_sp.png}
  }
  \subfigure{%
    \includegraphics[width=.15\columnwidth]{figures/supplementary/000022922_gt.png}
  }
  \subfigure{%
    \includegraphics[width=.15\columnwidth]{figures/supplementary/000022922_cnn.png}
  }
  \subfigure{%
    \includegraphics[width=.15\columnwidth]{figures/supplementary/000022922_crf.png}
  }
  \subfigure{%
    \includegraphics[width=.15\columnwidth]{figures/supplementary/000022922_ours.png}
  }\\[-2ex]


  \subfigure{%
    \includegraphics[width=.15\columnwidth]{figures/supplementary/000025711_given.png}
  }
  \subfigure{%
    \includegraphics[width=.15\columnwidth]{figures/supplementary/000025711_sp.png}
  }
  \subfigure{%
    \includegraphics[width=.15\columnwidth]{figures/supplementary/000025711_gt.png}
  }
  \subfigure{%
    \includegraphics[width=.15\columnwidth]{figures/supplementary/000025711_cnn.png}
  }
  \subfigure{%
    \includegraphics[width=.15\columnwidth]{figures/supplementary/000025711_crf.png}
  }
  \subfigure{%
    \includegraphics[width=.15\columnwidth]{figures/supplementary/000025711_ours.png}
  }\\[-2ex]


  \subfigure{%
    \includegraphics[width=.15\columnwidth]{figures/supplementary/000034473_given.png}
  }
  \subfigure{%
    \includegraphics[width=.15\columnwidth]{figures/supplementary/000034473_sp.png}
  }
  \subfigure{%
    \includegraphics[width=.15\columnwidth]{figures/supplementary/000034473_gt.png}
  }
  \subfigure{%
    \includegraphics[width=.15\columnwidth]{figures/supplementary/000034473_cnn.png}
  }
  \subfigure{%
    \includegraphics[width=.15\columnwidth]{figures/supplementary/000034473_crf.png}
  }
  \subfigure{%
    \includegraphics[width=.15\columnwidth]{figures/supplementary/000034473_ours.png}
  }\\[-2ex]


  \subfigure{%
    \includegraphics[width=.15\columnwidth]{figures/supplementary/000035463_given.png}
  }
  \subfigure{%
    \includegraphics[width=.15\columnwidth]{figures/supplementary/000035463_sp.png}
  }
  \subfigure{%
    \includegraphics[width=.15\columnwidth]{figures/supplementary/000035463_gt.png}
  }
  \subfigure{%
    \includegraphics[width=.15\columnwidth]{figures/supplementary/000035463_cnn.png}
  }
  \subfigure{%
    \includegraphics[width=.15\columnwidth]{figures/supplementary/000035463_crf.png}
  }
  \subfigure{%
    \includegraphics[width=.15\columnwidth]{figures/supplementary/000035463_ours.png}
  }\\[-2ex]


  \setcounter{subfigure}{0}
  \subfigure[\scriptsize Input]{%
    \includegraphics[width=.15\columnwidth]{figures/supplementary/000035993_given.png}
  }
  \subfigure[\scriptsize Superpixels]{%
    \includegraphics[width=.15\columnwidth]{figures/supplementary/000035993_sp.png}
  }
  \subfigure[\scriptsize GT]{%
    \includegraphics[width=.15\columnwidth]{figures/supplementary/000035993_gt.png}
  }
  \subfigure[\scriptsize AlexNet]{%
    \includegraphics[width=.15\columnwidth]{figures/supplementary/000035993_cnn.png}
  }
  \subfigure[\scriptsize +DenseCRF]{%
    \includegraphics[width=.15\columnwidth]{figures/supplementary/000035993_crf.png}
  }
  \subfigure[\scriptsize Using BI]{%
    \includegraphics[width=.15\columnwidth]{figures/supplementary/000035993_ours.png}
  }
  \mycaption{Material Segmentation}{Example results of material segmentation.
  (d)~depicts the AlexNet CNN result, (e)~CNN + 10 steps of mean-field inference,
  (f)~result obtained with bilateral inception (BI) modules (\bi{7}{2}+\bi{8}{6}) between
  \fc~layers.}
\label{fig:material_visuals-app}
\end{figure*}


\definecolor{city_1}{RGB}{128, 64, 128}
\definecolor{city_2}{RGB}{244, 35, 232}
\definecolor{city_3}{RGB}{70, 70, 70}
\definecolor{city_4}{RGB}{102, 102, 156}
\definecolor{city_5}{RGB}{190, 153, 153}
\definecolor{city_6}{RGB}{153, 153, 153}
\definecolor{city_7}{RGB}{250, 170, 30}
\definecolor{city_8}{RGB}{220, 220, 0}
\definecolor{city_9}{RGB}{107, 142, 35}
\definecolor{city_10}{RGB}{152, 251, 152}
\definecolor{city_11}{RGB}{70, 130, 180}
\definecolor{city_12}{RGB}{220, 20, 60}
\definecolor{city_13}{RGB}{255, 0, 0}
\definecolor{city_14}{RGB}{0, 0, 142}
\definecolor{city_15}{RGB}{0, 0, 70}
\definecolor{city_16}{RGB}{0, 60, 100}
\definecolor{city_17}{RGB}{0, 80, 100}
\definecolor{city_18}{RGB}{0, 0, 230}
\definecolor{city_19}{RGB}{119, 11, 32}
\begin{figure*}[!ht]
  \small % scriptsize
  \centering


  \subfigure{%
    \includegraphics[width=.18\columnwidth]{figures/supplementary/frankfurt00000_016005_given.png}
  }
  \subfigure{%
    \includegraphics[width=.18\columnwidth]{figures/supplementary/frankfurt00000_016005_sp.png}
  }
  \subfigure{%
    \includegraphics[width=.18\columnwidth]{figures/supplementary/frankfurt00000_016005_gt.png}
  }
  \subfigure{%
    \includegraphics[width=.18\columnwidth]{figures/supplementary/frankfurt00000_016005_cnn.png}
  }
  \subfigure{%
    \includegraphics[width=.18\columnwidth]{figures/supplementary/frankfurt00000_016005_ours.png}
  }\\[-2ex]

  \subfigure{%
    \includegraphics[width=.18\columnwidth]{figures/supplementary/frankfurt00000_004617_given.png}
  }
  \subfigure{%
    \includegraphics[width=.18\columnwidth]{figures/supplementary/frankfurt00000_004617_sp.png}
  }
  \subfigure{%
    \includegraphics[width=.18\columnwidth]{figures/supplementary/frankfurt00000_004617_gt.png}
  }
  \subfigure{%
    \includegraphics[width=.18\columnwidth]{figures/supplementary/frankfurt00000_004617_cnn.png}
  }
  \subfigure{%
    \includegraphics[width=.18\columnwidth]{figures/supplementary/frankfurt00000_004617_ours.png}
  }\\[-2ex]

  \subfigure{%
    \includegraphics[width=.18\columnwidth]{figures/supplementary/frankfurt00000_020880_given.png}
  }
  \subfigure{%
    \includegraphics[width=.18\columnwidth]{figures/supplementary/frankfurt00000_020880_sp.png}
  }
  \subfigure{%
    \includegraphics[width=.18\columnwidth]{figures/supplementary/frankfurt00000_020880_gt.png}
  }
  \subfigure{%
    \includegraphics[width=.18\columnwidth]{figures/supplementary/frankfurt00000_020880_cnn.png}
  }
  \subfigure{%
    \includegraphics[width=.18\columnwidth]{figures/supplementary/frankfurt00000_020880_ours.png}
  }\\[-2ex]



  \subfigure{%
    \includegraphics[width=.18\columnwidth]{figures/supplementary/frankfurt00001_007285_given.png}
  }
  \subfigure{%
    \includegraphics[width=.18\columnwidth]{figures/supplementary/frankfurt00001_007285_sp.png}
  }
  \subfigure{%
    \includegraphics[width=.18\columnwidth]{figures/supplementary/frankfurt00001_007285_gt.png}
  }
  \subfigure{%
    \includegraphics[width=.18\columnwidth]{figures/supplementary/frankfurt00001_007285_cnn.png}
  }
  \subfigure{%
    \includegraphics[width=.18\columnwidth]{figures/supplementary/frankfurt00001_007285_ours.png}
  }\\[-2ex]


  \subfigure{%
    \includegraphics[width=.18\columnwidth]{figures/supplementary/frankfurt00001_059789_given.png}
  }
  \subfigure{%
    \includegraphics[width=.18\columnwidth]{figures/supplementary/frankfurt00001_059789_sp.png}
  }
  \subfigure{%
    \includegraphics[width=.18\columnwidth]{figures/supplementary/frankfurt00001_059789_gt.png}
  }
  \subfigure{%
    \includegraphics[width=.18\columnwidth]{figures/supplementary/frankfurt00001_059789_cnn.png}
  }
  \subfigure{%
    \includegraphics[width=.18\columnwidth]{figures/supplementary/frankfurt00001_059789_ours.png}
  }\\[-2ex]


  \subfigure{%
    \includegraphics[width=.18\columnwidth]{figures/supplementary/frankfurt00001_068208_given.png}
  }
  \subfigure{%
    \includegraphics[width=.18\columnwidth]{figures/supplementary/frankfurt00001_068208_sp.png}
  }
  \subfigure{%
    \includegraphics[width=.18\columnwidth]{figures/supplementary/frankfurt00001_068208_gt.png}
  }
  \subfigure{%
    \includegraphics[width=.18\columnwidth]{figures/supplementary/frankfurt00001_068208_cnn.png}
  }
  \subfigure{%
    \includegraphics[width=.18\columnwidth]{figures/supplementary/frankfurt00001_068208_ours.png}
  }\\[-2ex]

  \subfigure{%
    \includegraphics[width=.18\columnwidth]{figures/supplementary/frankfurt00001_082466_given.png}
  }
  \subfigure{%
    \includegraphics[width=.18\columnwidth]{figures/supplementary/frankfurt00001_082466_sp.png}
  }
  \subfigure{%
    \includegraphics[width=.18\columnwidth]{figures/supplementary/frankfurt00001_082466_gt.png}
  }
  \subfigure{%
    \includegraphics[width=.18\columnwidth]{figures/supplementary/frankfurt00001_082466_cnn.png}
  }
  \subfigure{%
    \includegraphics[width=.18\columnwidth]{figures/supplementary/frankfurt00001_082466_ours.png}
  }\\[-2ex]

  \subfigure{%
    \includegraphics[width=.18\columnwidth]{figures/supplementary/lindau00033_000019_given.png}
  }
  \subfigure{%
    \includegraphics[width=.18\columnwidth]{figures/supplementary/lindau00033_000019_sp.png}
  }
  \subfigure{%
    \includegraphics[width=.18\columnwidth]{figures/supplementary/lindau00033_000019_gt.png}
  }
  \subfigure{%
    \includegraphics[width=.18\columnwidth]{figures/supplementary/lindau00033_000019_cnn.png}
  }
  \subfigure{%
    \includegraphics[width=.18\columnwidth]{figures/supplementary/lindau00033_000019_ours.png}
  }\\[-2ex]

  \subfigure{%
    \includegraphics[width=.18\columnwidth]{figures/supplementary/lindau00052_000019_given.png}
  }
  \subfigure{%
    \includegraphics[width=.18\columnwidth]{figures/supplementary/lindau00052_000019_sp.png}
  }
  \subfigure{%
    \includegraphics[width=.18\columnwidth]{figures/supplementary/lindau00052_000019_gt.png}
  }
  \subfigure{%
    \includegraphics[width=.18\columnwidth]{figures/supplementary/lindau00052_000019_cnn.png}
  }
  \subfigure{%
    \includegraphics[width=.18\columnwidth]{figures/supplementary/lindau00052_000019_ours.png}
  }\\[-2ex]




  \subfigure{%
    \includegraphics[width=.18\columnwidth]{figures/supplementary/lindau00027_000019_given.png}
  }
  \subfigure{%
    \includegraphics[width=.18\columnwidth]{figures/supplementary/lindau00027_000019_sp.png}
  }
  \subfigure{%
    \includegraphics[width=.18\columnwidth]{figures/supplementary/lindau00027_000019_gt.png}
  }
  \subfigure{%
    \includegraphics[width=.18\columnwidth]{figures/supplementary/lindau00027_000019_cnn.png}
  }
  \subfigure{%
    \includegraphics[width=.18\columnwidth]{figures/supplementary/lindau00027_000019_ours.png}
  }\\[-2ex]



  \setcounter{subfigure}{0}
  \subfigure[\scriptsize Input]{%
    \includegraphics[width=.18\columnwidth]{figures/supplementary/lindau00029_000019_given.png}
  }
  \subfigure[\scriptsize Superpixels]{%
    \includegraphics[width=.18\columnwidth]{figures/supplementary/lindau00029_000019_sp.png}
  }
  \subfigure[\scriptsize GT]{%
    \includegraphics[width=.18\columnwidth]{figures/supplementary/lindau00029_000019_gt.png}
  }
  \subfigure[\scriptsize Deeplab]{%
    \includegraphics[width=.18\columnwidth]{figures/supplementary/lindau00029_000019_cnn.png}
  }
  \subfigure[\scriptsize Using BI]{%
    \includegraphics[width=.18\columnwidth]{figures/supplementary/lindau00029_000019_ours.png}
  }%\\[-2ex]

  \mycaption{Street Scene Segmentation}{Example results of street scene segmentation.
  (d)~depicts the DeepLab results, (e)~result obtained by adding bilateral inception (BI) modules (\bi{6}{2}+\bi{7}{6}) between \fc~layers.}
\label{fig:street_visuals-app}
\end{figure*}


%%%%%%%%%%%%%%%%%%%%%%%%%%%%%%%%%%%%%%%%%%%%%%%%%%%%%%%%%%%%%%%%%%%%%%%%%%%%%%%
%%%%%%%%%%%%%%%%%%%%%%%%%%%%%%%%%%%%%%%%%%%%%%%%%%%%%%%%%%%%%%%%%%%%%%%%%%%%%%%


\end{document}


% This document was modified from the file originally made available by
% Pat Langley and Andrea Danyluk for ICML-2K. This version was created
% by Iain Murray in 2018. It was modified from a version from Dan Roy in
% 2017, which was based on a version from Lise Getoor and Tobias
% Scheffer, which was slightly modified from the 2010 version by
% Thorsten Joachims & Johannes Fuernkranz, slightly modified from the
% 2009 version by Kiri Wagstaff and Sam Roweis's 2008 version, which is
% slightly modified from Prasad Tadepalli's 2007 version which is a
% lightly changed version of the previous year's version by Andrew
% Moore, which was in turn edited from those of Kristian Kersting and
% Codrina Lauth. Alex Smola contributed to the algorithmic style files.
