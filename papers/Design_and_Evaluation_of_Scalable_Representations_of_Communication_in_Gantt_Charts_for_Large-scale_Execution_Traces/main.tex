\documentclass[journal]{vgtc}         

\ifpdf%                                % if we use pdflatex
  \pdfoutput=1\relax                   % create PDFs from pdfLaTeX
  \pdfcompresslevel=9                  % PDF Compression
  \pdfoptionpdfminorversion=7          % create PDF 1.7
  \ExecuteOptions{pdftex}
  \usepackage{graphicx}                % allow us to embed graphics files
  \DeclareGraphicsExtensions{.pdf,.png,.jpg,.jpeg} % for pdflatex we expect .pdf, .png, or .jpg files
\else%                                 % else we use pure latex
  \ExecuteOptions{dvips}
  \usepackage{graphicx}                % allow us to embed graphics files
  \DeclareGraphicsExtensions{.eps}     % for pure latex we expect eps files
\fi%

\graphicspath{{figures/}{pictures/}{images/}{./}} % where to search for the images

\usepackage{microtype}                 % use micro-typography (slightly more compact, better to read)
\PassOptionsToPackage{warn}{textcomp}  % to address font issues with \textrightarrow
\usepackage{textcomp}                  % use better special symbols
\usepackage{mathptmx}                  % use matching math font
\usepackage{times}                     % we use Times as the main font
\renewcommand*\ttdefault{txtt}         % a nicer typewriter font
\usepackage{cite}                      % needed to automatically sort the references
\usepackage{tabu}                      % only used for the table example
\usepackage{booktabs}                  % only used for the table example
\usepackage{enumerate}
\usepackage[shortlabels]{enumitem}
\usepackage{caption}
\usepackage{subcaption}

\onlineid{0}

%% declare the category of your paper, only shown in review mode
\vgtccategory{Research}

\vgtcpapertype{please specify}

%% Paper title.
\title{Design and Evaluation of Scalable Representations of Communication in Gantt Charts for Large-scale Execution Traces}

%% indicate IEEE Member or Student Member in form indicated below
\author{Connor Scully-Allison, Katherine E. Isaacs}
\authorfooter{
%% insert punctuation at end of each item
\item
 Connor Scully-Allison, University of Arizona. E-mail: cscullyallison@email.arizona.edu.
\item
 Katherine E. Isaacs, University of Arizona. E-mail: kisaacs@cs.arizona.edu.
}

\abstract{
    Gantt charts are frequently used to explore execution traces of large-scale parallel programs found in high-performance computing (HPC). In these visualizations, each parallel processor is assigned a row showing the computation state of a processor at a particular time. Lines are drawn between rows to show communication between these processors. When drawn to align equivalent calls across rows, structures can emerge reflecting communication patterns employed by the executing code. However, though these structures have the same definition at any scale, they are obscured by the density of rendered lines when displaying more than a few hundred processors. A more scalable metaphor is necessary to aid HPC experts in understanding communication in large-scale traces. To address this issue, we first conduct an exploratory study to identify what visual features are critical for determining similarity between structures shown at different scales. Based on these findings, we design a set of glyphs for displaying these structures in dense charts. We then conduct a pre-registered user study evaluating how well people interpret communication using our new representation versus their base depictions in large-scale Gantt charts. Through our evaluation, we find that our representation enables users to more accurately identify communication patterns compared to full renderings of dense charts. We discuss the results of our evaluation and findings regarding the design of metaphors for extensible structures.
} % end of abstract

\keywords{}

\CCScatlist{ % not used in journal version
 \CCScat{K.6.1}{Management of Computing and Information Systems}%
{Project and People Management}{Life Cycle};
 \CCScat{K.7.m}{The Computing Profession}{Miscellaneous}{Ethics}
}

%% A teaser figure can be included as follows
\teaser{
  \centering
  \includegraphics[width=\linewidth]{figures/header-story.pdf}
  \caption{Communication between rows in a Gantt chart are interpretable at small scales but not further, despite being highly regular. We study how people identify and distinguish similar structures at different scales and design a scale-agnostic representation.}
  \label{fig:teaser}
}

\vgtcinsertpkg

%commands
\newcommand{\note}[1]{\textcolor[rgb]{1.0,0.0,0.0}{#1}}

%%%%%%%%%%%%%%%%%%%%%%%%%%%%%%%%%%%%%%%%%%%%%%%%%%%%%%%%%%%%%%%%
%%%%%%%%%%%%%%%%%%%%%% START OF THE PAPER %%%%%%%%%%%%%%%%%%%%%%
%%%%%%%%%%%%%%%%%%%%%%%%%%%%%%%%%%%%%%%%%%%%%%%%%%%%%%%%%%%%%%%%%

\begin{document}

\firstsection{Introduction}

%% the only exception to this rule is the \firstsection command
\maketitle

% !TEX root = ../arxiv.tex

Unsupervised domain adaptation (UDA) is a variant of semi-supervised learning \cite{blum1998combining}, where the available unlabelled data comes from a different distribution than the annotated dataset \cite{Ben-DavidBCP06}.
A case in point is to exploit synthetic data, where annotation is more accessible compared to the costly labelling of real-world images \cite{RichterVRK16,RosSMVL16}.
Along with some success in addressing UDA for semantic segmentation \cite{TsaiHSS0C18,VuJBCP19,0001S20,ZouYKW18}, the developed methods are growing increasingly sophisticated and often combine style transfer networks, adversarial training or network ensembles \cite{KimB20a,LiYV19,TsaiSSC19,Yang_2020_ECCV}.
This increase in model complexity impedes reproducibility, potentially slowing further progress.

In this work, we propose a UDA framework reaching state-of-the-art segmentation accuracy (measured by the Intersection-over-Union, IoU) without incurring substantial training efforts.
Toward this goal, we adopt a simple semi-supervised approach, \emph{self-training} \cite{ChenWB11,lee2013pseudo,ZouYKW18}, used in recent works only in conjunction with adversarial training or network ensembles \cite{ChoiKK19,KimB20a,Mei_2020_ECCV,Wang_2020_ECCV,0001S20,Zheng_2020_IJCV,ZhengY20}.
By contrast, we use self-training \emph{standalone}.
Compared to previous self-training methods \cite{ChenLCCCZAS20,Li_2020_ECCV,subhani2020learning,ZouYKW18,ZouYLKW19}, our approach also sidesteps the inconvenience of multiple training rounds, as they often require expert intervention between consecutive rounds.
We train our model using co-evolving pseudo labels end-to-end without such need.

\begin{figure}[t]%
    \centering
    \def\svgwidth{\linewidth}
    \input{figures/preview/bars.pdf_tex}
    \caption{\textbf{Results preview.} Unlike much recent work that combines multiple training paradigms, such as adversarial training and style transfer, our approach retains the modest single-round training complexity of self-training, yet improves the state of the art for adapting semantic segmentation by a significant margin.}
    \label{fig:preview}
\end{figure}

Our method leverages the ubiquitous \emph{data augmentation} techniques from fully supervised learning \cite{deeplabv3plus2018,ZhaoSQWJ17}: photometric jitter, flipping and multi-scale cropping.
We enforce \emph{consistency} of the semantic maps produced by the model across these image perturbations.
The following assumption formalises the key premise:

\myparagraph{Assumption 1.}
Let $f: \mathcal{I} \rightarrow \mathcal{M}$ represent a pixelwise mapping from images $\mathcal{I}$ to semantic output $\mathcal{M}$.
Denote $\rho_{\bm{\epsilon}}: \mathcal{I} \rightarrow \mathcal{I}$ a photometric image transform and, similarly, $\tau_{\bm{\epsilon}'}: \mathcal{I} \rightarrow \mathcal{I}$ a spatial similarity transformation, where $\bm{\epsilon},\bm{\epsilon}'\sim p(\cdot)$ are control variables following some pre-defined density (\eg, $p \equiv \mathcal{N}(0, 1)$).
Then, for any image $I \in \mathcal{I}$, $f$ is \emph{invariant} under $\rho_{\bm{\epsilon}}$ and \emph{equivariant} under $\tau_{\bm{\epsilon}'}$, \ie~$f(\rho_{\bm{\epsilon}}(I)) = f(I)$ and $f(\tau_{\bm{\epsilon}'}(I)) = \tau_{\bm{\epsilon}'}(f(I))$.

\smallskip
\noindent Next, we introduce a training framework using a \emph{momentum network} -- a slowly advancing copy of the original model.
The momentum network provides stable, yet recent targets for model updates, as opposed to the fixed supervision in model distillation \cite{Chen0G18,Zheng_2020_IJCV,ZhengY20}.
We also re-visit the problem of long-tail recognition in the context of generating pseudo labels for self-supervision.
In particular, we maintain an \emph{exponentially moving class prior} used to discount the confidence thresholds for those classes with few samples and increase their relative contribution to the training loss.
Our framework is simple to train, adds moderate computational overhead compared to a fully supervised setup, yet sets a new state of the art on established benchmarks (\cf \cref{fig:preview}).


\section{Background and Motivation}

\subsection{IBM Streams}

IBM Streams is a general-purpose, distributed stream processing system. It
allows users to develop, deploy and manage long-running streaming applications
which require high-throughput and low-latency online processing.

The IBM Streams platform grew out of the research work on the Stream Processing
Core~\cite{spc-2006}.  While the platform has changed significantly since then,
that work established the general architecture that Streams still follows today:
job, resource and graph topology management in centralized services; processing
elements (PEs) which contain user code, distributed across all hosts,
communicating over typed input and output ports; brokers publish-subscribe
communication between jobs; and host controllers on each host which
launch PEs on behalf of the platform.

The modern Streams platform approaches general-purpose cluster management, as
shown in Figure~\ref{fig:streams_v4_v6}. The responsibilities of the platform
services include all job and PE life cycle management; domain name resolution
between the PEs; all metrics collection and reporting; host and resource
management; authentication and authorization; and all log collection. The
platform relies on ZooKeeper~\cite{zookeeper} for consistent, durable metadata
storage which it uses for fault tolerance.

Developers write Streams applications in SPL~\cite{spl-2017} which is a
programming language that presents streams, operators and tuples as
abstractions. Operators continuously consume and produce tuples over streams.
SPL allows programmers to write custom logic in their operators, and to invoke
operators from existing toolkits. Compiled SPL applications become archives that
contain: shared libraries for the operators; graph topology metadata which tells
both the platform and the SPL runtime how to connect those operators; and
external dependencies. At runtime, PEs contain one or more operators. Operators
inside of the same PE communicate through function calls or queues. Operators
that run in different PEs communicate over TCP connections that the PEs
establish at startup. PEs learn what operators they contain, and how to connect
to operators in other PEs, at startup from the graph topology metadata provided
by the platform.

We use ``legacy Streams'' to refer to the IBM Streams version 4 family. The
version 5 family is for Kubernetes, but is not cloud native. It uses the
lift-and-shift approach and creates a platform-within-a-platform: it deploys a
containerized version of the legacy Streams platform within Kubernetes.

\subsection{Kubernetes}

Borg~\cite{borg-2015} is a cluster management platform used internally at Google
to schedule, maintain and monitor the applications their internal infrastructure
and external applications depend on. Kubernetes~\cite{kube} is the open-source
successor to Borg that is an industry standard cloud orchestration platform.

From a user's perspective, Kubernetes abstracts running a distributed
application on a cluster of machines. Users package their applications into
containers and deploy those containers to Kubernetes, which runs those
containers in \emph{pods}. Kubernetes handles all life cycle management of pods,
including scheduling, restarting and migration in case of failures.

Internally, Kubernetes tracks all entities as \emph{objects}~\cite{kubeobjects}.
All objects have a name and a specification that describes its desired state.
Kubernetes stores objects in etcd~\cite{etcd}, making them persistent,
highly-available and reliably accessible across the cluster. Objects are exposed
to users through \emph{resources}. All resources can have
\emph{controllers}~\cite{kubecontrollers}, which react to changes in resources.
For example, when a user changes the number of replicas in a
\code{ReplicaSet}, it is the \code{ReplicaSet} controller which makes sure the
desired number of pods are running. Users can extend Kubernetes through
\emph{custom resource definitions} (CRDs)~\cite{kubecrd}. CRDs can contain
arbitrary content, and controllers for a CRD can take any kind of action.

Architecturally, a Kubernetes cluster consists of nodes. Each node runs a
\emph{kubelet} which receives pod creation requests and makes sure that the
requisite containers are running on that node. Nodes also run a
\emph{kube-proxy} which maintains the network rules for that node on behalf of
the pods. The \emph{kube-api-server} is the central point of contact: it
receives API requests, stores objects in etcd, asks the scheduler to schedule
pods, and talks to the kubelets and kube-proxies on each node. Finally,
\emph{namespaces} logically partition the cluster. Objects which should not know
about each other live in separate namespaces, which allows them to share the
same physical infrastructure without interference.

\subsection{Motivation}
\label{sec:motivation}

Systems like Kubernetes are commonly called ``container orchestration''
platforms. We find that characterization reductive to the point of being
misleading; no one would describe operating systems as ``binary executable
orchestration.'' We adopt the idea from Verma et al.~\cite{borg-2015} that
systems like Kubernetes are ``the kernel of a distributed system.'' Through CRDs
and their controllers, Kubernetes provides state-as-a-service in a distributed
system. Architectures like the one we propose are the result of taking that view 
seriously.

The Streams legacy platform has obvious parallels to the Kubernetes
architecture, and that is not a coincidence: they solve similar problems.
Both are designed to abstract running arbitrary user-code across a distributed
system.  We suspect that Streams is not unique, and that there are many
non-trivial platforms which have to provide similar levels of cluster
management.  The benefits to being cloud native and offloading the platform
to an existing cloud management system are: 
\begin{itemize}
    \item Significantly less platform code.
    \item Better scheduling and resource management, as all services on the cluster are 
        scheduled by one platform.
    \item Easier service integration.
    \item Standardized management, logging and metrics.
\end{itemize}
The rest of this paper presents the design of replacing the legacy Streams 
platform with Kubernetes itself.



\section{Preliminary Study}
\label{sec:prelim}

To aid the design of scalable representations of communication patterns and design of experiments for evaluating both new and existing representations, we executed a preliminary study to explore how people interpret communication in Gantt charts. This study focused on how communication is presently drawn and sought to understand what visual factors are considered when differentiating patterns. 

There are two ways people can view communication lines in Gantt charts. The first is to look at all PEs. We call this the {\em full} representation. The problem is when the chart has many PEs which communicate, i.e., it is {\em dense}, the lines may overlap to the point of a solid shape. The other way is to look at a subset of the PEs (rows). We call this the {\em partial} representation. The problem here is that key indicators about the pattern, such as grouping, may not appear in a given window.

Prior to the study, we identified several factors that might be used in interpreting these views: the density of a chart, the structure being shown, the grouping of a structure, the representation type ({\em  partial} or {\em full}), and the stride of a structure. A study across the full range of these factors would be unreasonably large, so we designed a qualitative study to limit our focus, following the ``factor mining'' evaluation pattern discussed by Elmqvist and Yi~\cite{elmqvist2015patterns}.

\begin{figure}
    \centering
    \includegraphics[width=\columnwidth]{figures/c1.png}
    \caption{Example image prompt from our preliminary study. These are both {\em partial} representations of a stencil pattern, emulating being zoomed-in on a Gantt chart.}
    \label{fig:interview_prompt}
\end{figure}

Our procedure was to interview participants while showing them paired images of communication patterns in idealized unit time Gantt charts. We decided on semi-structured interviews to allow probing and elaboration of ideas. The paired images varied in number of rows, representation type, and whether they were the same pattern. \autoref{fig:interview_prompt} shows an example prompt. For each prompt, we asked three questions:
\vspace{0em}
\begin{enumerate}
    \itemsep=0em
    \item Please describe the pattern of lines on the right.
    \item Please describe the pattern of lines on the left.
    \item Do you think that these two patterns are the same? Why or why not?
\end{enumerate}

We recruited seven participants, one from our university and six at the Supercomputing 2019 conference. Six reported computing experience and two reported HPC experience. Of those two, one had analyzed HPC performance. None had prior experience with trace visualization. 

The interviews were recorded, transcribed, and coded for common themes, resulting in over 170 unique codes. We describe the most frequently used. See the supplemental material for a full list of codes.

\texttt{Line angle} was used by all participants to describe patterns of lines and justify comparisons between them, with three referencing it over 10 times. \texttt{Line direction} was mentioned by all but one participant, but mostly to describe a pattern rather than compare. Although a line direction could be the result of an angle, it was coded distinct from \texttt{angle} since respondents would describe lines as going ``up," ``down," ``left," or ``right". Their intent was distinct from when they mentioned angle. A few participants mistook the multiple wrap-around lines for \texttt{one line} in ring patterns, even on repeated prompting.

Four participants made note of the \texttt{background}---the boxes and columns the lines connected. This code was also used both to describe an individual pattern and differentiate between pairs.

\texttt{Comparison} and \texttt{uncertainty} were co-occurring codes. Participants rarely felt confident comparing charts. We surmise this uncertainty comes from the fact that simple changes of height and representation alter visual factors they relied upon. Four participants discussed \texttt{transformation} during comparisons, describing how one depiction could be ``rotated'',  ``squashed/stretched'', or ``enlarged.''

Participants had the most difficulty with prompts depicting stencil patterns. They were uncertain of line extents and misidentified discrete lines as single lines or vice versa. Of the four interviewees exposed to stencil patterns, three found them viscerally off-putting, pausing with surprise when presented with them and one calling them ``a mess." 

\vspace{1ex}

\textbf{Discussion.} Participants generally relied on features such as line angle, line direction, and background markings in interpreting patterns. We expect all three factors to be harder to interpret with large numbers of PEs as many angles map to the same pixels and background markers are aggregated or removed.

The line factors are functions of the pattern stride and the height, and therefore density, of the chart, suggesting density significantly impacts recognition. This is further supported with participants' difficulty in recognizing line separation at severe angles, such as the wrap-around lines in rings.

The impetus of participants to identify patterns as similar under a stretch factor matches how the patterns are extensible to PE count. However, communication patterns are not extensible under rotation or inversion, as some discussed. We use this finding to inform tutorial material in subsequent study designs.

Participants reactions to stencils suggest that the idealized unit time depiction is too complex even at small scales. As we want to understand how people can interpret patterns as they scale up, we conclude that stencils are not appropriate for early work towards this goal as they are too difficult to interpret at small scale.

Based on the results of this preliminary study, we decided to take line angle and background into careful consideration for proposed designs. We ultimately fixed line angle at set values and obscure background features to encourage focus on pattern types over strides and to maintain discernability. We also decided to remove stencils from our pattern type factor and randomize across our other factors in further studies.

\section{Designing a Watermark}
\label{sec:taxonomy}

After the quick overview of watermarking schemes in \cref{sec:background}, we now provide more details 
about the watermarking design space. We created a unifying taxonomy under which all previous schemes 
can be expressed. We first discuss the requirements then the building blocks of a text watermark. 
%
%We provide a modular implementation of all schemes, so any of the building blocks can be combined.
%
\cref{fig:design-figure} summarizes the current design space.

\subsection{Requirements}

A useful watermarking scheme must detect watermarked texts, without falsely flagging human-generated text and without impairing the original model's performance.
%
More precisely, we want watermarks to have the following properties.
% \begin{itemize}[leftmargin=\itemlm,itemsep=2pt]
\begin{enumerate}[leftmargin=\itemlm,itemsep=2pt]
    \item \textbf{High Recall}. $\Pr[\mathcal{V}_k(T) = \texttt{True}]$ is large if $T$ is a watermarked text generated using the marking procedure $\mathcal{W}$ and secret key $k$.
    %
    \item \textbf{High Precision}. For a random key $k$, $\Pr[\mathcal{V}_k(\Tilde{T}) = \texttt{False}]$ is large if $\Tilde{T}$ is a human-generated (\emph{non-watermarked}) text.
    %
    \item \textbf{Quality}. The watermarked model should perform similarly to the original model. 
    It should be useful for the same tasks and generate similar quality text.
    %
    \item \textbf{Robustness}. A good watermark should be robust to small changes to the watermarked text (potentially caused by an adversary), 
    meaning if a sample $T$ is watermarked with key $k$, then for any text $\Tilde{T}$ that is semantically close to $T$, $\mathcal{V}_k(\Tilde{T})$ should evaluate to \text{True}.
\end{enumerate}

\noindent
A desireable (but optional) property for watermarks is diversity. 
In some settings, such as creative tasks like story-telling, users might want the model to have the ability to generate 
multiple different outputs in response to the same prompt (so they can select their favorite).
We would like watermarked outputs to preserve this capability.
% \noindent
% In addition to these properties, another desirable property for a watermark is to 
% preserve a model's diversity. Language models tend to have diverse generated text distributions: 
% they are able to generate different responses to a same prompt. This is useful in many settings, 
% such as creative tasks like story telling, so the user can  their favorite output.

% The notion of \emph{undetectability} has been defined in previous work~\citep{christ_undetectable_2023}:
Another useful property is \emph{undetectability}, also called \emph{indistinguishability}:
%
no feasible adversary should be able to distinguish watermarked text from non-watermarked text, without knowledge of the secret key~\citep{christ_undetectable_2023}. 
%
A watermark is considered undetectable if the maximum advantage at distinguishing is very small.
%
This notion is appealing; for instance, undetectability implies that watermarking does not degrade the model's quality.
%
However, we find in practice that undetectability is not necessary and may be overly restrictive:
%
minor changes to the model's output distribution are not always detrimental to its quality.

In this paper we focus on symmetric-key watermarking, where both the watermarking and verification procedures share a secret key.
%
This is most suitable for proprietary language models that served via an API.
%
We imagine that the vendor would watermark all outputs, and also provide a second API to query the verification procedure.
%
Alternatively, one could publish the key, enabling anyone to run the verification procedure.
%
\begin{figure*}
    \begin{center}
    \begin{tikzpicture}
    
    \draw[draw=black] (0,15) rectangle ++(17.5,1) node[pos=0.5, align=center] {\Large{Watermarking Taxonomy}};
    \draw[draw=black] (0,12.75) rectangle ++(8.375,2) node[pos=0.5, align=left] 
    {\\
    \\
    \textbf{Parameters:} Key $k$, Sampling $\mathcal{C}$, Randomness $\mathcal{R}$\\
    \textbf{Inputs:} Probs $\mathcal{D}_n = \{\lambda^n_1,\, \cdots, \lambda^n_d\}$, Tokens $\{T_i\}_{i < n}$\\
    \textbf{Output:} Next token 
    $T_n \leftarrow \mathcal{C}(\mathcal{R}_k( \{T_i\}_{i < n}), \mathcal{D}_n)$};
    \draw[draw=black] (9.125,12.75) rectangle ++(8.375,2) node[pos=0.5, align=left] 
    {\\
    \\
    \textbf{Parameters:} Key $k$, Score $\mathcal{S}$, Threshold $p$\\
    \textbf{Inputs:} Text $T$\\
    \textbf{Output:} Decision $\mathcal{V} \leftarrow \text{P}_{0}\left( \mathcal{S} < \mathcal{S}_k(T)\right) < p$};
    \draw (8.75,13.75) circle (0.25) node {+};
    \draw[draw=none] (0,14.25) rectangle ++(8.375,.5) node[pos=0.5, align=left] {\large{Marking $\mathcal{W}$}};
    \draw[draw=none] (9.125,14.25) rectangle ++(8.375,.5) node[pos=0.5, align=left] {\large{Verification $\mathcal{V}$}};
    
    %%%
    
    \draw[draw=black,dashed] (0,8.75) rectangle ++(17.5,3.75);
    \draw[draw=none] (0,11) rectangle ++(17.5,1.75) node[pos=0.5, align=center] {\large{Randomness Source $\mathcal{R}$}\\
    \textbf{Inputs:} Tokens $\{T_i\}_{i < n}$\,
    \textbf{Output:} Random value $r_n = \mathcal{R}_k(\{T_i\}_{i < n})$};
    \draw[draw=black] (0.25,9) rectangle ++(11.25,2.35) node[pos=0, anchor=south west] {\textbf{Text-dependent.} Hash function $h$. Context length H};
    \draw[draw=black] (0.5,9.6) rectangle ++(10.75,0.625) node[anchor=north west] at (0.5, 10.225) {\textbf{(R2) Min Hash}} node[pos=1, anchor=north east, align=left] {
    $r_n = \text{min} \left( h\left( T_{n-1} \mathbin\Vert k\right), \, \cdots, h\left( T_{n-H} \mathbin\Vert k\right) \right)$\\
    };
    \draw[draw=black] (0.5,10.475) rectangle ++(10.75,0.625) node[anchor=north west] at (0.5, 11.1) {\textbf{(R1) Sliding Window}} node[pos=1, anchor=north east, align=left] {
    $r_n = h\left( T_{n-1} \mathbin\Vert \, \cdots \mathbin\Vert T_{n-H} \mathbin\Vert k\right)$\\
    };
    \draw[draw=black] (11.75,9) rectangle ++(5.5,2.35) node[pos=0, anchor=south west] {\textbf{(R3) Fixed}} node[pos=0.5, align=left] {Key length L. Expand $k$ to\\ pseudo-random sequence $\{r^k_i\}_{i<L}$.\\ 
    $r_n = r^k_{n \text{ (mod L)}}$ \\ \\ };
    
    %%%
    
    \draw[draw=black,dashed] (0,3.25) rectangle ++(17.5,5.25);
    \draw[draw=none] (0,6.85) rectangle ++(17.5,1.75) node[pos=0.5, align=center] {\large{Sampling algorithm $\mathcal{C}$ \& Per-token statistic $s$}\\
    \textbf{Inputs:} Random value $r_n = \mathcal{R}_k( \{T_i\}_{i < n})$, Probabilities $\mathcal{D}_n = \{\lambda^n_1,\, \cdots, \lambda^n_d\}$, Logits $\mathcal{L}_n = \{l^n_1,\,\cdots,l^n_d\}$\\};
    
    %
    
    \draw[draw=black] (11,4.75) rectangle ++(6.25,2.5) node[pos=0, anchor=south west] {\textbf{(C3) Binary}} node[pos=0.5, align=left] {Binary alphabet.\\ 
    $T_n \leftarrow 0$ if $r_n < \lambda^n_0$, else $1$. \\
    $s(T_n, r) = \begin{cases} -\log(r) \text{ if } T_n = 1\\
          -\log(1-r) \text{ if } T_n = 0\\\end{cases} $};
    
    \draw[draw=black] (5,4.75) rectangle ++(5.75,2.5) node[pos=0, anchor=south west] {\textbf{(C2) Inverse Transform}} node[pos=0.5, align=left] 
    {$\pi$ keyed permutation. $\eta$ scaling func.\\
    $T_n \leftarrow \pi_k \left( \min\limits_{ j \leq d } \sum\limits_{i=1}^j \lambda^n_{\pi_k (i)} \geq r_n \right)$ \\
    $s(T_n, r) = | r - \eta \left( \pi^{-1}_k(T_n) \right) | $\\};
    
    \draw[draw=black] (0.25,4.75) rectangle ++(4.5,2.5) node[pos=0, anchor=south west] {\textbf{(C1) Exponential}} node[pos=0.5, align=left] 
    {$h$ keyed hash function. \\
    $T_n \leftarrow \argmax\limits_{i \leq d} \left\{ \frac{\log \left( h_{r_n}\left( i \right) \right)}{\lambda^n_i} \right\}$ \\
    $s(T_n, r) = -\log(1 \! - \! h_r(T_n))$\\};
    
    % 
    
    \draw[draw=black] (0.25,3.5) rectangle ++(17,1) node[pos=0, anchor=south west] {\textbf{(C4) Distribution-shift}} node[pos=0.5, align=right] {Bias $\delta$, Greenlist size $\gamma$. Keyed permutation $\pi$. $T_n$ sampled from $\widetilde{\mathcal{L}}_n = \{l^n_i + \delta \text{ if } \pi_{r_n}(i) < \gamma d \text{ else } l^n_i\, , 1 \leq i \leq d\}$\\
    $s(T_n, r) = 1 \text{ if } \pi_{r}(T_n) < \gamma d \text{ else } 0$};
    
    %%% 
    
    \draw[draw=black,dashed] (0,0) rectangle ++(17.5,3);
    \draw[draw=none] (0,1.75) rectangle ++(17.5,1.25) node[pos=0.5, align=center] {\large{Score $\mathcal{S}$}\\
    \textbf{Inputs:} Per-token statistics $s_{i,j} = s(T_i, r_j)$, where $r_j = \mathcal{R}_k( \{T_l\}_{l < j}))$. \# Tokens $N$.};
    
    % 
    
    \draw[draw=black] (8.15,0.25) rectangle ++(9.1,1.5) node[pos=0, anchor=south west] {\textbf{(S3) Edit Score}}
    
    node[pos=0.5, align=left] {
    $\mathcal{S}_{\text{edit}}^\psi = s^\psi(N,N)$,
    $
        s^\psi (i,j) = \min \begin{cases}
          s^\psi (i-1, j-1) + s_{i,j}\\
          s^\psi (i-1, j) + \psi\\
          s^\psi (i, j-1) + \psi\\
        \end{cases} 
    $};
    \draw[draw=black] (0.25,0.25) rectangle ++(2.6,1.5) node[pos=0, anchor=south west] {\textbf{(S1) Sum Score}} node[pos=0.5, align=left] {$\mathcal{S}_{\text{sum}}\! = \! \sum_{i=1}^N s_{i,i}$ \\};
    \draw[draw=black] (3.1,0.25) rectangle ++(4.8,1.5) node[pos=0, anchor=south west] {\textbf{(S2) Align Score}} node[pos=0.5, align=left] {$\mathcal{S}_{\text{align}} \!= \!\min\limits_{0 \leq j < N} \sum\limits_{i=1}^N s_{i, (i+j) \text{ mod}(N)}$ \\ \\ };
    
    \end{tikzpicture}
    \caption{Watermarking design blocks. There are three main components: randomness source, sampling algorithm (and associated per-token statistics), and score function. Each solid box within each of these three components (dashed) denotes a design choice. The choice for each component is independent and offers different trade-offs.}\label{fig:design-figure}
    \end{center}
    \end{figure*}

\subsection{Watermark Design Space}
\label{sec:watermark-design}

Designing a good watermark is a balancing act.
% 
For instance, replacing every word of the output with [WATERMARK] would achieve high recall but destroy the utility of the model.
%
%Conversely, sampling from the original distribution preserves quality but makes it impossible to watermark. 

Existing proposals have cleverly crafted marking procedures that are meant to preserve quality, provide high precision and recall, and achieve a degree of robustness.
%
Despite their apparent differences, we realized they can all be expressed within a unified framework:

\begin{itemize}[leftmargin=\itemlm,itemsep=2pt]
    \item The marking procedure $\mathcal{W}$ contains a randomness source $\mathcal{R}$ and a sampling algorithm $\mathcal{C}$.
    %
    The randomness source $\mathcal{R}$ produces a (pseudo-random) value $r_n$ for each new token, based on the secret key $k$ and the previous tokens $T_0,\cdots,T_{n-1}$.
    %
    The sampling algorithm $\mathcal{C}$ uses $r_n$ and the model's next token distribution $\mathcal{D}$ to  a token.
    \item The verification procedure $\mathcal{V}$ is a one-tailed significance test that computes a $p$-value for the null hypothesis that the text is not watermarked.
    %
    The procedure compares this $p$-value to a threshold, which enables control over the watermark's precision and recall.
    %
    % This test is done using a \emph{score function} $\mathcal{S}$ based on a per-token variable that depends on the ed sampling algorithm.
    % We call the value of this per-token test statistic $s_n$, which only depends on the random value $r_n$ and the ed token $T_n$: $s_n = s(T_n, r_n)$.
    In particular, we compute a per-token score $s_{n,m} \coloneqq s(T_n, r_m)$ for each token $T_n$ and randomness $r_m$, aggregate them to obtain an overall score $\mathcal{S}$, and compute a $p$-value from this score.
    We consider all overlaps $s_{n,m}$ instead of only $s_{n,n}$ to support scores that consider misaligned randomness and text after perturbation. 
    %the test computes \emph{score function} $\mathcal{S}$ which takes as input per-token test statistics $s_{n,m} \coloneqq s(T_n, r_m)$ for a token $T_n$ and a random value $r_m$, $\forall n,m \in [N]$.
    %
    %$s_{n,m}$ depends on the sampling algorithm (see \cref{fig:design-figure} for examples).
    %
    % \dave{I believe $s(T_n, r_m)$ is incorrect and it should be $s(T_n, r_n)$.  Also I think the score should be $s_n$ rather than $s_{n,m}$.}
    % \jp{Depending on the alignment between the key string and the text, there are times we want to refer to the score for key at position m and token at poistion n (for instance, for both the align and edit scores). I'll add some explanation for this.}
    
\end{itemize}
% \dave{I find the sheer number of fonts inelegant (blackboard bold, mathcal, mathbf, typewritter, italics, bold, etc.). In some places, algorithms are denoted by mathcal (W,V), in other places by mathbf (R,C,S).  I suggest picking one and being consistent.  I prefer mathcal.  Lots of bold feels distracting to my eyes, as does lots of font changes.}
% \jp{I changed a bunch of fonts to make it more consistent, and removed bold fonts}

Next, we show how each scheme we consider falls within this framework, each with its own choices for $\mathcal{R},\mathcal{C},\mathcal{S}$.
%Given this template, previous work introduced their own variants of the building blocks, which we will now detail. 
% \chawin{I would have liked to see a summary of which design choices belong to which paper. Maybe we can add a shorthand notation denoting each paper in \cref{fig:design-figure} or have a separate table.}
% \jp{I agree that's a good idea. A table is probably the right way to represent this.}

\subsubsection{Randomness source $\mathcal{R}$}\label{ssec:randomness}
% \textbf{Randomness source $\mathcal{R}$.}
%
% \chawin{Maybe others?} \jp{Yeah but all the other papers i've seen seem to attribute it to one of these two.}
We distinguish two main ways of generating the random values $r_n$, \emph{text-dependent} (computed as a deterministic function of the prior tokens) vs \emph{fixed} (computed as a function of the token index).
Both approaches use the standard heuristic of applying a keyed function (typically, a PRF) to some data, to produce pseudorandom values that can be treated as effectively random but can also be reproduced by the verification procedure.

\citet{aaronson_watermarking_2022} and \citet{kirchenbauer_watermark_2023}
use text-dependent randomness: $r_n = f\left(T_0,\,\cdots,T_{n-1},k\right)$.
%
This scheme has two parameters: the length of the token context window (which we call the window size H) and the aggregation function $f$.
%
\citet{aaronson_watermarking_2022} proposed using the hash of the concatenation of previous tokens, $f := h\left( T_{n-1} \mathbin\Vert \, \cdots \mathbin\Vert T_{n-H} \mathbin\Vert k\right)$; we call this (R1) sliding window.
%
\citet{kirchenbauer_watermark_2023} used this with a window size of $ H = 1$ and also introduced an alternate aggregation function $f := \text{min} \left( h\left( T_{n-1} \mathbin\Vert k\right), \, \cdots, h\left( T_{n-H} \mathbin\Vert k\right) \right)$.
%
We call this last aggregation function (R2) min hash.
%
While these two schemes propose specific choices of $H$, other values are possible. 
We use \benchmarkname{} to evaluate a range of values of $H$ with each candidate aggregation function.

% \smallskip\noindent\textbf{(R3) Fixed}
\citet{kuditipudi_robust_2023} use fixed randomness:
$r_n = f_k(n)$, where $n$ is the index (position) of the token.
We call this (R3) fixed.
%
In practice, they propose using a fixed string of length $L$ (the key length), which is repeated across the generation.
% r_n = f_k(n \bmod L)$ where $L$ is the key length.
% \dave{I don't think we need this level of detail.  I suggest deleting the preceding sentence.}
% \jp{Since we look at the impact of the key length on generations we still need to introduce the idea that the key is repeated, but I canwrite that in english for it to be more digestable}
%
We test the choice of key length in ~\cref{ssec:param_tuning}
%
In the extreme case where $L=1$ or $H=0$, both sources are identical, as $r_n$ will be the same value for every token. \citet{zhao2023provable} explored this option using the same sampling algorithm as~\citet{kirchenbauer_watermark_2023}.

\label{ssec:binary}
\citet{christ_undetectable_2023} proposed setting a target entropy for the context window instead of fixing a window size.
%
This allows to set a lower bound on the security parameter for the model's undetectability.
%
However, setting a fixed entropy makes for a less efficient detector since all context window lengths must be tried in order to detect a watermark.
%
Furthermore, in practice, provable undetectability is not needed to achieve optimal quality: we chose to keep using a fixed-size window for increased efficiency.

\subsubsection{Sampling algorithm \(\mathcal{C}\)}\label{ssec:sampling}
% \textbf{sampling algorithm $\mathcal{C}$.}
%
\noindent
We now give more details about the four sampling algorithms initially presented in~\cref{tab:marking-algorithms}.

\smallskip\noindent\textbf{(C1) Exponential}.
%
Introduced by \citet{aaronson_watermarking_2022} and also used by \citet{kuditipudi_robust_2023}. It relies on the Gumbel-max trick.
%
Let $\mathcal{D}_n = \left\{\lambda^n_i\,, 1 \leq i \leq d\right\}$ be the distribution of the language model over the next token. %(obtained after passing the logits through a softmax and applying a temperature adjustment).
%
The exponential scheme will select the next token as:
\begin{align}
    T_{n} = \argmax\limits_{i \leq d}\left\{ \frac{\log \left( h_{r_n}\left( i \right) \right)}{\lambda^n_i} \right\}
\end{align}
where $h$ is a keyed hash function using $r_n$ as its key.
%
The per-token variable used in the statistical test is either $s_n = h_{r_n}(T_n)$ or $s_n = -\log \left( 1-h_{r_n}(T_n)\right)$.
%
\citet{aaronson_watermarking_2022} and \citet{kuditipudi_robust_2023} both use the latter quantity.
%
We argue the first variable provides the same results, and unlike the log-based variable, the distribution of watermarked variables can be expressed analytically (see~\cref{app:ssec:pseudorandom-proofs} for more details).
%
We align with previous work and use the $\log$ for \benchmarkname{}.

\smallskip\noindent\textbf{(C2) Inverse transform}.
%
\citet{kuditipudi_robust_2023} introduce inverse transform sampling.
%
They derive a random permutation using the secret key $\pi_k$. The next token is selected as follows:
\begin{align}
    T_{n} = \pi_k \left( \min\limits_{ j \leq d } \sum\limits_{i=1}^j \lambda^n_{\pi_k (i)} \geq r_n \right)
\end{align}
which is the smallest index in the inverse permutation such that the CDF of the next token distribution is at least $r_n$.
%
\citet{kuditipudi_robust_2023} propose to use $s_n = | r_n - \eta \left( \pi^{-1}_k(T_n) \right) |$ as a the test variable, where $\eta$ normalizes the token index to the $[0,1]$ range.
%
% We call this scheme the \textit{inverse transform} scheme.

\smallskip\noindent\textbf{(C3) Binary}.
%
\citet{christ_undetectable_2023} propose a different sampling scheme for binary token alphabets --- however, it can be applied to any model by using a bit encoding of the tokens.
%
In our implementation, we rely on a Huffman encoding of the token set, using frequencies derived from a large corpus of natural text.
%
In this case, the distribution over the next token reduces to a single probability $p_n$ that token ``0'' is ed next, and $1-p$ that ``1'' is ed.
%
The sampling rule s 0 if $r_n < p$, and 1 otherwise. The test variable for this case is $s_n = -\log \left( T_n r_n + (1-T_n) (1-r_n) \right)$.
%
% We call this scheme the \textit{binary} scheme.
%
% At first glance, it can seem like this scheme is identical to the exponential scheme. However, because it uses a binary alphabet, the distribution of the test variable is different for both schemes.
%
% However, we show in Appendix \jp{ref} that this is not the case: the distribution of the test variable is different for both schemes.
% %
% \jp{Maybe I'll remove this if I don't have time to show it.}

\smallskip\noindent\textbf{(C4) Distribution-shift}.
%
\citet{kirchenbauer_watermark_2023} propose the distribution-shift scheme. 
%
It produces a modified distribution $D_n$ from which the next token is sampled.
%
Let $\delta > 0$ and $\gamma \in [0,1]$ be two system parameters, and $d$ be the number of tokens.
%
The scheme constructs a permutation $\pi_{r_n}$, seeded by the random value $r_n$, which is used to define a ``green list,'' containing tokens $T$ such that $\pi_{r_n} (T) < \delta d$. It then adds $\delta$ to green-list logits.
%
This modified distribution is then used by the model to sample the next token. The test variable $s_n$ is a bit equal to ``1'' if $T_n$ is in the green list defined by $\pi_{r_n}$, and ``0'' if not.
%
% We call this scheme the \textit{distribution-shift} scheme.

The advantage of this last scheme over the others is that it preserves the model's diversity: 
for a given key, the model will still generate diverse outputs.
In contrast, for a given secret key and a given prompt, the first three sampling strategies 
will always produce the same result, since the randomness value $r_n$ will be the same.
\citet{kuditipudi_robust_2023} tackles this by randomly offseting the key sequence of 
fixed randomness for each generation. We introcude a skip probability $p$ for the 
same effect on text-dependent randomness. Each token is selected without the marking 
strategy with probability $p$. In the interest of space, we leave a detailed discussion 
of generation diversity in~\cref{app:ssec:diverse}.

Another advantage of the distribution-shift scheme is that it can also be used 
at any temperature, by applying the temperature scaling \emph{after} using the 
scheme to modify the logits. Other models apply temperature before watermarking.

However the distribution-shift scheme is not indistinguishable from the original model, 
as discussed earlier in~\cref{ssec:watermark-design}.

\subsubsection{Score Function $\mathcal{S}$}\label{ssec:score}

% \paragraph{Verification procedure $\mathcal{V}.$}

% The distribution of the per-token test statistic is different for watermarked text and non-watermarked text: this is what makes detection possible. Depending on the scheme, it is either higher or lower on average in the watermarked case. Without loss of generality, we assume it is always lower for this discussion.

To determine whether an $N$-token text is watermarked, we compute a score over per-token statistics.
%
This score is then subject to a one-tailed statistical test where the null hypothesis is that the text is not watermarked.
%
In other words, if its $p$-value is under a fixed threshold, the text is watermarked.
%
Different works propose different scores.

\smallskip\noindent\textbf{(S1) Sum score}.
%
\citet{aaronson_watermarking_2022} and \citet{kirchenbauer_watermark_2023} take the sum of all individual per-token statistics:
\begin{align}
    \mathcal{S}_{\text{sum}}=\sum_{i=1}^N s_i = \sum_{i=1}^N s(T_i, r_i).
\end{align}
%
This score requires the random values $r_i$ and the tokens $T_i$ to be aligned.
%
% \chawin{Maybe this goes into limitation or discussion or appendix}
This is not a problem when using text-dependent randomness, since the random values are directly obtained from the tokens.
%
However, this score is not suited for fixed randomness: removing one token at the start of the text will offset the values of $r_i$ for the rest of the text and remove the watermark.
%
The use of the randomness shift to increase diversity will have the same effect. 

\smallskip\noindent\textbf{(S2) Alignment score}.
Proposed by \citet{kuditipudi_robust_2023}, the alignment score aims to mitigate the misalignment issue mentioned earlier.
% \citet{kuditipudi_robust_2023} proposes two alternative scores to deal with this issue.
%
% In keeping with their work, we name these scores the alignment score and the edit score.
Given the sequence of random values $r_i$ and the sequence of tokens $T_i$, the verification process now computes different versions of the per-token test statistic for each possible overlap of both sequences $s_{i,j} = s(T_i, r_j)$.
%
The alignment score is defined as:
\begin{align}
   \mathcal{S}_{\text{align}}  = \min\limits_{0 \leq j < N} \sum\limits_{i=1}^N s_{i, (i+j) \text{ mod}(N)}
\end{align}

\smallskip\noindent\textbf{(S3) Edit score}.
Similar to the alignment score, \citet{kuditipudi_robust_2023} propose the edit score as an alternative for dealing with the misalignment issue.
%
It comes with an additional parameter $\psi$ and is defined as $\mathcal{S}_{\text{edit}}^\psi = s^\psi(N,N)$, where
\begin{align}
    s^\psi (i,j) &= \min \begin{cases}
      s^\psi (i-1, j-1) + s_{i,j}\\
      s^\psi (i-1, j) + \psi\\
      s^\psi (i, j-1) + \psi\\
    \end{cases} 
\end{align}

In all three cases, the average value of the score for watermarked text will be lower than for non-watermarked text.
%
% In the case of the sum score, we can often derive the distribution of the score under the null hypothesis, allowing us to use a $z$-test to determine if the text is watermarked.
In the case of the sum score, the previous works use the $z$-test on the score to determine whether the text is watermarked, but it is also possible, or even better in certain situations, to use a different statistical test according to \citet{fernandez_three_2023}.
%
When possible, we derive the exact distribution of the scores under the null hypothesis (see \cref{app:ssec:exact_dist}) which is more precise than the $z$-test. When it is not, we rely on an empirical T-test, as proposed by \citet{kuditipudi_robust_2023}
%
% This allows one to compute 

\subsection{Limitations of the Building Blocks}\label{ssec:limit_blocks}

While we design the blocks to be as independent as possible, some combinations of the scheme and specific parameters are obviously sub-optimal.
%
Here, we list a few of these subpar block combinations as a guide for practitioners.
% Even though any of the three scores can be used with any scheme and randomness source, in practice not all combinations are useful.
\begin{itemize}[leftmargin=\itemlm,itemsep=2pt]
    \item The sum score (S1) is not robust for fixed randomness (R3).
    \item The alignment score (S2) does not make sense for the text-dependent randomness (R1, R2) since misalignment is not an issue.
    \item The edit score (S3) has a robustness benefit since it can support local misalignment caused by token insertion, deletion, or swapping. However, using it with text-dependent randomness (R1, R2) only makes sense for a window size of 1: for longer window sizes, swapping, adding, or removing tokens would actually change the random values themselves, and not just misalign them.
    \item Finally, in the corner case when a window size of 0 for the text-dependent randomness (R1, R2) or when a random sequence length of 1 for the fixed randomness (R3), both the alignment score (S2) and the edit score (S3) are unnecessary since all random values are the same and misalignment is not possible.
\end{itemize}

In our experiments (\cref{sec:experiments}), we test all reasonable configurations of the randomness source, 
the sampling protocol, and the verification score, along with their parameters. 
We list the evaluated combinations in~\cref{tab:design_space_combinations}. 
The edit score is too inefficient 
to be run on all configurations, instead we rely on the sum and align scores.
%
We hope to not only fairly compare the prior works but also investigate previously unexplored combinations in the 
design space that can produce a better result.

% \chawin{We need a table or a tree that lists all the combinations we test.}\

\begin{table}[h!]
    \centering
    \caption{Tested combinations in the design space, using notations from~\cref{fig:design-figure}.\\
    We only tested the edit score {\bf S3} on a subset of watermarks.\\
    The distribution of non-watermarked scores is known for \textcolor{orange}{orange} configurations and 
    unknown for \textcolor{blue}{blue} configuration. We rely on empirical T-tests~\cite{kuditipudi_robust_2023} for blue configurations.
    }
    \label{tab:design_space_combinations}
    \normalsize
    \begin{tabular}{|l||c|c|c|c|} 
    \hline
     & \makecell[tc]{{\bf C4}\\{\small Distribution}\\{\small Shift}} & \makecell[tc]{{\bf C1}\\{\small Exponential}} & \makecell[tc]{{\bf C2}\\{\small Binary}} & \makecell[tc]{{\bf C3}\\{\small Inverse}\\{\small Transform}} \\
    \hline
    \hline
    \makecell{{\bf S1}+{\bf R1}}  & \textcolor{orange}{X} & \textcolor{orange}{X} & \textcolor{orange}{X} & \textcolor{blue}{X} \\
    \hline
    \makecell{{\bf S1}+{\bf R2}}  & \textcolor{orange}{X} & \textcolor{orange}{X} & \textcolor{orange}{X} & \textcolor{blue}{X} \\
    \hline
    \makecell{{\bf S2}+{\bf R3}}  & \textcolor{blue}{X} & \textcolor{blue}{X} & \textcolor{blue}{X} & \textcolor{blue}{X} \\
    \hline
    \makecell{{\bf S3}+{\bf R3}}  & \textcolor{blue}{X} &  &  &  \\
    \hline
    \end{tabular}
\end{table}
    

\subsection{Analysis of the edit score.} 
\label{ssec:editscore}
We analyzed the tamper-resistance of the edit score on a subset of watermarks 
(distribution-shift with $\delta=2.5$ at a temperature of 1, for key lengths between 1 and 1024). 
We tried various $\psi$ values between 0 and 1 for the edit distance, and compared the tamper-resistance 
and watermark size of the resulting verification procedures to the align score. 
Using an edit distance does improve tamper-resistance for key lengths under 32, but at a large efficiency cost: 
for key lengths above 8, the edit score size is at least twice that of the align score. 
We do not recommend using an edit score on low entropy models such as Llama-2 chat.




\section{Experimental Evaluation}
\label{sec:experiment}
To demonstrate the viability of our modeling methodology, we show experimentally how through the deliberate combination and configuration of parallel FREEs, full control over 2DOF spacial forces can be achieved by using only the minimum combination of three FREEs.
To this end, we carefully chose the fiber angle $\Gamma$ of each of these actuators to achieve a well-balanced force zonotope (Fig.~\ref{fig:rigDiagram}).
We combined a contracting and counterclockwise twisting FREE with a fiber angle of $\Gamma = 48^\circ$, a contracting and clockwise twisting FREE with $\Gamma = -48^\circ$, and an extending FREE with $\Gamma = -85^\circ$.
All three FREEs were designed with a nominal radius of $R$ = \unit[5]{mm} and a length of $L$ = \unit[100]{mm}.
%
\begin{figure}
    \centering
    \includegraphics[width=0.75\linewidth]{figures/rigDiagram_wlabels10.pdf}
    \caption{In the experimental evaluation, we employed a parallel combination of three FREEs (top) to yield forces along and moments about the $z$-axis of an end effector.
    The FREEs were carefully selected to yield a well-balanced force zonotope (bottom) to gain full control authority over $F^{\hat{z}_e}$ and $M^{\hat{z}_e}$.
    To this end, we used one extending FREE, and two contracting FREEs which generate antagonistic moments about the end effector $z$-axis.}
    \label{fig:rigDiagram}
\end{figure}


\subsection{Experimental Setup}
To measure the forces generated by this actuator combination under a varying state $\vec{x}$ and pressure input $\vec{p}$, we developed a custom built test platform (Fig.~\ref{fig:rig}). 
%
\begin{figure}
    \centering
    \includegraphics[width=0.9\linewidth]{figures/photos/rig_labeled.pdf}
    \caption{\revcomment{1.3}{This experimental platform is used to generate a targeted displacement (extension and twist) of the end effector and to measure the forces and torques created by a parallel combination of three FREEs. A linear actuator and servomotor impose an extension and a twist, respectively, while the net force and moment generated by the FREEs is measured with a force load cell and moment load cell mounted in series.}}
    \label{fig:rig}
\end{figure}
%
In the test platform, a linear actuator (ServoCity HDA 6-50) and a rotational servomotor (Hitec HS-645mg) were used to impose a 2-dimensional displacement on the end effector. 
A force load cell (LoadStar  RAS1-25lb) and a moment load cell (LoadStar RST1-6Nm) measured the end-effector forces $F^{\hat{z_e}}$ and moments $M^{\hat{z_e}}$, respectively.
During the experiments, the pressures inside the FREEs were varied using pneumatic pressure regulators (Enfield TR-010-g10-s). 

The FREE attachment points (measured from the end effector origin) were measured to be:
\begin{align}
    \vec{d}_1 &= \bmx 0.013 & 0 & 0 \emx^T  \text{m}\\
    \vec{d}_2 &= \bmx -0.006 & 0.011 & 0 \emx^T  \text{m}\\
    \vec{d}_3 &= \bmx -0.006 & -0.011 & 0 \emx^T \text{m}
%    \vec{d}_i &= \bmx 0 & 0 & 0 \emx^T , && \text{for } i = 1,2,3
\end{align}
All three FREEs were oriented parallel to the end effector $z$-axis:
\begin{align}
    \hat{a}_i &= \bmx 0 & 0 & 1 \emx^T, \hspace{20pt} \text{for } i = 1,2,3
\end{align}
Based on this geometry, the transformation matrices $\bar{\mathcal{D}}_i$ were given by:
\begin{align}
    \bar{\mathcal{D}}_1 &= \bmx 0 & 0 & 1 & 0 & -0.013 & 0 \\ 0 & 0 & 0 & 0 & 0 & 1 \emx^T  \\
    \bar{\mathcal{D}}_2 &= \bmx 0 & 0 & 1 & 0.011 & 0.006 & 0 \\ 0 & 0 & 0 & 0 & 0 & 1 \emx^T  \\
    \bar{\mathcal{D}}_3 &= \bmx 0 & 0 & 1 & -0.011 & 0.006 & 0 \\ 0 & 0 & 0 & 0 & 0 & 1 \emx^T 
%    \bar{\mathcal{D}}_i &= \bmx 0 & 0 & 1 & 0 & 0 & 0 \\ 0 & 0 & 0 & 0 & 0 & 1 \emx^T , && \text{for } i = 1,2,3
\end{align}
These matrices were used in equation \eqref{eq:zeta} to yield the state-dependent fluid Jacobian $\bar{J}_x$ and to compute the resulting force zontopes.
%while using measured values of $\vec{\zeta}^{\,\text{meas}} (\vec{q}, \vec{P})$ and $\vec{\zeta}^{\,\text{meas}} (\vec{q}, 0)$ in \eqref{eq:fiberIso} yields the empirical measurements of the active force.



\subsection{Isolating the Active Force}
To compare our model force predictions (which focus only on the active forces induced by the fibers)
to those measured empirically on a physical system, we had to remove the elastic force components attributed to the elastomer. 
Under the assumption that the elastomer force is merely a function of the displacement $\vec{x}$ and independent of pressure $\vec{p}$ \cite{bruder2017model}, this force component can be approximated by the measured force at a pressure of $\vec{p}=0$. 
That is: 
\begin{align}
    \vec{f}_{\text{elast}} (\vec{x}) = \vec{f}_{\text{\,meas}} (\vec{x}, 0)
\end{align}
With this, the active generalized forces were measured indirectly by subtracting off the force generated at zero pressure:
\begin{align}
    \vec{f} (\vec{x}, \vec{p})  &= \vec{f}_{\text{meas}} (\vec{x}, \vec{p}) - \vec{f}_{\text{meas}} (\vec{x}, 0)     \label{eq:fiberIso}
\end{align}


%To validate our parallel force model, we compare its force predictions, $\vec{\zeta}_{\text{pred}}$, to those measured empirically on a physical system, $\vec{\zeta}_\text{meas}$. 
%From \eqref{eq:Z} and \eqref{eq:zeta}, the force at the end effector is given by:
%\begin{align}
%    \vec{\zeta}(\vec{q}, \vec{P}) &= \sum_{i=1}^n \bar{\mathcal{D}}_i \left( {\bar{J}_V}_i^T(\vec{q_i}) P_i + \vec{Z}_i^{\text{elast}} (\vec{q_i}) \right) \\
%    &= \underbrace{\sum_{i=1}^n \bar{\mathcal{D}}_i {\bar{J}_V}_i^T(\vec{q_i}) P_i}_{\vec{\zeta}^{\,\text{fiber}} (\vec{q}, \vec{P})} + \underbrace{\sum_{i=1}^n \bar{\mathcal{D}}_i \vec{Z}_i^{\text{elast}} (\vec{q_i})}_{\vec{\zeta}^{\text{elast}} (\vec{q})}   \label{eq:zetaSplit}
%     &= \vec{\zeta}^{\,\text{fiber}} (\vec{q}, \vec{P}) + \vec{\zeta}^{\text{elast}} (\vec{q})
%\end{align}
%\Dan{These will need to reflect changes made to previous section.}
%The model presented in this paper does not specify the elastomer forces, $\vec{\zeta}^{\text{elast}}$, therefore we only validate its predictions %of the fiber forces, $\vec{\zeta}^{\,\text{fiber}}$. 
%We isolate the fiber forces by noting that $\vec{\zeta}^{\text{elast}} (\vec{q}) = \vec{\zeta}(\vec{q}, 0)$ and rearranging \eqref{eq:zetaSplit}
%\begin{align}
%    \vec{\zeta}^{\,\text{fiber}} (\vec{q}, \vec{P})  &= \vec{\zeta}(\vec{q}, \vec{P}) - \vec{\zeta}(\vec{q}, 0)     \label{eq:fiberIso}
%%    \vec{\zeta}^{\,\text{fiber}}_{\text{emp}} (\vec{q}, \vec{P})  &= \vec{\zeta}_{\text{emp}}(\vec{q}, \vec{P}) - %\vec{\zeta}_{\text{emp}}(\vec{q}, 0)
%\end{align}
%Thus we measure the fiber forces indirectly by subtracting off the forces generated at zero pressure.  


\subsection{Experimental Protocol}
The force and moment generated by the parallel combination of FREEs about the end effector $z$-axis  was measured in four different geometric configurations: neutral, extended, twisted, and simultaneously extended and twisted (see Table \ref{table:RMSE} for the exact deformation amounts). 
At each of these configurations, the forces were measured at all pressure combinations in the set
\begin{align}
    \mathcal{P} &= \left\{ \bmx \alpha_1 & \alpha_2 & \alpha_3 \emx^T p^{\text{max}} \, : \, \alpha_i = \left\{ 0, \frac{1}{4}, \frac{1}{2}, \frac{3}{4}, 1 \right\} \right\}
\end{align}
with $p^{\text{max}}$ = \unit[103.4]{kPa}. 
\revcomment{3.2}{The experiment was performed twice using two different sets of FREEs to observe how fabrication variability might affect performance. The results from Trial 1 are displayed in Fig.~\ref{fig:results} and the error for both trials is given in Table \ref{table:RMSE}.}



\subsection{Results}

\begin{figure*}[ht]
\centering

\def\picScale{0.08}    % define variable for scaling all pictures evenly
\def\plotScale{0.2}    % define variable for scaling all plots evenly
\def\colWidth{0.22\linewidth}

\begin{tikzpicture} %[every node/.style={draw=black}]
% \draw[help lines] (0,0) grid (4,2);
\matrix [row sep=0cm, column sep=0cm, style={align=center}] (my matrix) at (0,0) %(2,1)
{
& \node (q1) {(a) $\Delta l = 0, \Delta \phi = 0$}; & \node (q2) {(b) $\Delta l = 5\text{mm}, \Delta \phi = 0$}; & \node (q3) {(c) $\Delta l = 0, \Delta \phi = 20^\circ$}; & \node (q4) {(d) $\Delta l = 5\text{mm}, \Delta \phi = 20^\circ$};

\\

&
\node[style={anchor=center}] {\includegraphics[width=\colWidth]{figures/photos/s0w0pic_colored.pdf}}; %\fill[blue] (0,0) circle (2pt);
&
\node[style={anchor=center}] {\includegraphics[width=\colWidth]{figures/photos/s5w0pic_colored.pdf}}; %\fill[blue] (0,0) circle (2pt);
&
\node[style={anchor=center}] {\includegraphics[width=\colWidth]{figures/photos/s0w20pic_colored.pdf}}; %\fill[blue] (0,0) circle (2pt);
&
\node[style={anchor=center}] {\includegraphics[width=\colWidth]{figures/photos/s5w20pic_colored.pdf}}; %\fill[blue] (0,0) circle (2pt);

\\

\node[rotate=90] (ylabel) {Moment, $M^{\hat{z}_e}$ (N-m)};
&
\node[style={anchor=center}] {\includegraphics[width=\colWidth]{figures/plots3/s0w0.pdf}}; %\fill[blue] (0,0) circle (2pt);
&
\node[style={anchor=center}] {\includegraphics[width=\colWidth]{figures/plots3/s5w0.pdf}}; %\fill[blue] (0,0) circle (2pt);
&
\node[style={anchor=center}] {\includegraphics[width=\colWidth]{figures/plots3/s0w20.pdf}}; %\fill[blue] (0,0) circle (2pt);
&
\node[style={anchor=center}] {\includegraphics[width=\colWidth]{figures/plots3/s5w20.pdf}}; %\fill[blue] (0,0) circle (2pt);

\\

& \node (xlabel1) {Force, $F^{\hat{z}_e}$ (N)}; & \node (xlabel2) {Force, $F^{\hat{z}_e}$ (N)}; & \node (xlabel3) {Force, $F^{\hat{z}_e}$ (N)}; & \node (xlabel4) {Force, $F^{\hat{z}_e}$ (N)};

\\
};
\end{tikzpicture}

\caption{For four different deformed configurations (top row), we compare the predicted and the measured forces for the parallel combination of three FREEs (bottom row). 
\revcomment{2.6}{Data points and predictions corresponding to the same input pressures are connected by a thin line, and the convex hull of the measured data points is outlined in black.}
The Trial 1 data is overlaid on top of the theoretical force zonotopes (grey areas) for each of the four configurations.
Identical colors indicate correspondence between a FREE and its resulting force/torque direction.}
\label{fig:results}
\end{figure*}






% & \node (a) {(a)}; & \node (b) {(b)}; & \node (c) {(c)}; & \node (d) {(d)};


For comparison, the measured forces are superimposed over the force zonotope generated by our model in Fig.~\ref{fig:results}a-~\ref{fig:results}d.
To quantify the accuracy of the model, we defined the error at the $j^{th}$ evaluation point as the difference between the modeled and measured forces
\begin{align}
%    \vec{e}_j &= \left( {\vec{\zeta}_{\,\text{mod}}} - {\vec{\zeta}_{\,\text{emp}}} \right)_j
%    e_j &= \left( F/M_{\,\text{mod}} - F/M_{\,\text{emp}} \right)_j
    e^F_j &= \left( F^{\hat{z}_e}_{\text{pred}, j} - F^{\hat{z}_e}_{\text{meas}, j} \right) \\
    e^M_j &= \left( M^{\hat{z}_e}_{\text{pred}, j} - M^{\hat{z}_e}_{\text{meas}, j} \right)
\end{align}
and evaluated the error across all $N = 125$ trials of a given end effector configuration.
% using the following metrics:
% \begin{align}
%     \text{RMSE} &= \sqrt{ \frac{\sum_{j=1}^{N} e_j^2}{N} } \\
%     \text{Max Error} &= \max \{ \left| e_j \right| : j = 1, ... , N \}
% \end{align}
As shown in Table \ref{table:RMSE}, the root-mean-square error (RMSE) is less than \unit[1.5]{N} (\unit[${8 \times 10^{-3}}$]{Nm}), and the maximum error is less than \unit[3]{N}  (\unit[${19 \times 10^{-3}}$]{Nm}) across all trials and configurations.

\begin{table}[H]
\centering
\caption{Root-mean-square error and maximum error}
\begin{tabular}{| c | c || c | c | c | c|}
    \hline
     & \rule{0pt}{2ex} \textbf{Disp.} & \multicolumn{2}{c |}{\textbf{RMSE}} & \multicolumn{2}{c |}{\textbf{Max Error}} \\ 
     \cline{2-6}
     & \rule{0pt}{2ex} (mm, $^\circ$) & F (N) & M (Nm) & F (N) & M (Nm) \\
     \hline
     \multirow{4}{*}{\rotatebox[origin=c]{90}{\textbf{Trial 1}}}
     & 0, 0 & 1.13 & $3.8 \times 10^{-3}$ & 2.96 & $7.8 \times 10^{-3}$ \\
     & 5, 0 & 0.74 & $3.2 \times 10^{-3}$ & 2.31 & $7.4 \times 10^{-3}$ \\
     & 0, 20 & 1.47 & $6.3 \times 10^{-3}$ & 2.52 & $15.6 \times 10^{-3}$\\
     & 5, 20 & 1.18 & $4.6 \times 10^{-3}$ & 2.85 & $12.4 \times 10^{-3}$ \\  
     \hline
     \multirow{4}{*}{\rotatebox[origin=c]{90}{\textbf{Trial 2}}}
     & 0, 0 & 0.93 & $6.0 \times 10^{-3}$ & 1.90 & $13.3 \times 10^{-3}$ \\
     & 5, 0 & 1.00 & $7.7 \times 10^{-3}$ & 2.97 & $19.0 \times 10^{-3}$ \\
     & 0, 20 & 0.77 & $6.9 \times 10^{-3}$ & 2.89 & $15.7 \times 10^{-3}$\\
     & 5, 20 & 0.95 & $5.3 \times 10^{-3}$ & 2.22 & $13.3 \times 10^{-3}$ \\  
     \hline
\end{tabular}
\label{table:RMSE}
\end{table}

\begin{figure}
    \centering
    \includegraphics[width=\linewidth]{figures/photos/buckling.pdf}
    \caption{At high fluid pressure the FREE with fiber angle of $-85^\circ$ started to buckle.  This effect was less pronounced when the system was extended along the $z$-axis.}
    \label{fig:buckling}
\end{figure}

%Experimental precision was limited by unmodeled material defects in the FREEs, as well as sensor inaccuracy. While the commercial force and moment sensors used have a quoted accuracy of 0.02\% for the force sensor and 0.2\% for the moment sensor (LoadStar Sensors, 2015), a drifting of up to 0.5 N away from zero was noticed on the force sensor during testing.

It should be noted, that throughout the experiments, the FREE with a fiber angle of $-85^\circ$ exhibited noticeable buckling behavior at pressures above $\approx$ \unit[50]{kPa} (Fig.~\ref{fig:buckling}). 
This behavior was more pronounced during testing in the non-extended configurations (Fig.~\ref{fig:results}a~and~\ref{fig:results}c). 
The buckling might explain the noticeable leftward offset of many of the points in Fig.~\ref{fig:results}a and Fig.~\ref{fig:results}c, since it is reasonable to assume that buckling reduces the efficacy of of the FREE to exert force in the direction normal to the force sensor. 

\begin{figure}
    \centering
    \includegraphics[width=\linewidth]{figures/zntp_vs_x4.pdf}
    \caption{A visualization of how the \emph{force zonotope} of the parallel combination of three FREEs (see Fig.~\ref{fig:rig}) changes as a function of the end effector state $x$. One can observe that the change in the zonotope ultimately limits the work-space of such a system.  In particular the zonotope will collapse for compressions of more than \unit[-10]{mm}.  For \revcomment{2.5}{scale and comparison, the convex hulls of the measured points from Fig.~\ref{fig:results}} are superimposed over their corresponding zonotope at the configurations that were evaluated experimentally.}
    % \marginnote{\#2.5}
    \label{fig:zntp_vs_x}
\end{figure}


\section{Case Studies}
\label{sec:case_studies}
In this section, we present a case study of Facebook posts from an Australian public page.
The page shifts between early 2020 (\emph{2019-2020 Australian bushfire season}) and late 2020 (\emph{COVID-19 crises}) from being a moderate-right group for discussion around climate change to a far-right extremist group for conspiracy theories.


\begin{figure*}[!tbp]
	\begin{subfigure}{0.21\textwidth}
		\includegraphics[width=\textwidth]{images/facebook1.png}
		\caption{}
		\label{subfig:first-posting}
		\includegraphics[width=0.9\textwidth]{images/facebook3.jpg}
		\caption{}
		\label{subfig:comment-post-1}
	\end{subfigure}
    \begin{subfigure}{0.28\textwidth}
		\includegraphics[width=\textwidth]{images/facebook2.jpg}
		\caption{}
		\label{subfig:second-posting}
	\end{subfigure}
    \begin{subfigure}{0.23\textwidth}
		\includegraphics[width=\textwidth]{images/facebook4.jpg}
		\caption{}
		\label{subfig:comment-post-2a}
	\end{subfigure}
    \begin{subfigure}{0.23\textwidth}
		\includegraphics[width=\textwidth]{images/facebook5.jpg}
		\caption{}
		\label{subfig:comment-post-2b}
	\end{subfigure}
	\caption{
		Examples of postings and comment threads from a public Facebook page from two periods of time early 2020 (a) and late 2020 (b)-(e), which show a shift from climate change debates to extremist and far-right messaging.
	}
	\label{fig:facebook}
\end{figure*}

We focus on a sample of 2 postings and commenting threads from one Australian Facebook page we classified as ``far-right'' based on the content on the page. 
We have anonymized the users in \Cref{fig:facebook} to avoid re-identification.
The first posting and comment thread (see \Cref{subfig:first-posting}) was collected on Jan 10, 2020, and responded to the Australian bushfire crisis that began in late 2019 and was still ongoing in January 2020. It contains an ambivalent text-based provocation that references disputes in the community regarding the validity of climate change and climate science. 

The second posting and comment thread (see \Cref{subfig:second-posting}) was collected from the same page in September 2020, months after the bushfire crisis had abated.
At that time, a new crisis was energizing and connecting the far-right groups in our dataset --- i.e., the COVID-19 pandemic and the government interventions to curb the spread of the virus. 
The post is different in style compared to the first.
It is image-based instead of text-based and highly emotive, with a photo collage bringing together images of prison inmates with iron masks on their faces (top row) juxtaposed to people wearing face masks during COVID-19 (bottom row). 
The image references the public health orders issued during Melbourne's second lockdown and suggests that being ordered to wear masks is an infringement of citizen rights and freedoms, similar to dehumanizing restraints used on prisoners.

To analyze reactions to the posts, two researchers used a deductive analytical approach to separately code and to analyze the commenting threads --- see \Cref{subfig:comment-post-1} for comments of the first posting, and \Cref{subfig:comment-post-2a,subfig:comment-post-2b} for comments on the second posting. 
Conversations were also inductively coded for emerging themes. 
During the analysis, we observed qualitative differences in the types of content users posted, interactions between commenters, tone and language of debate, linked media shared in the commenting thread, and the opinions expressed.
The rest of this section further details these differences.
To ensure this was not a random occurrence, we tested the exemplar threads against field notes collected on the group during the entire study.
We also used Facebook's search function within pages to find a sample of posts from the same period and which dealt with similar topics. 
After this analysis, we can confidently say that key changes occurred in the group between the bushfire crisis and COVID-19, that we detail next.

\subsubsection*{Exemplar 1 --- climate change skepticism.}
To explore this transformation in more depth, we analyzed comments scraped on the first posting --- \cref{sub@subfig:comment-post-1} shows a small sample of these comments.
The language used was similar to comments that we observed on numerous far-right nationalist pages at the time of the bushfires.
These comments are usually text-based, employing emojis to denote emotions, and sometimes being mocking or provocative in tone. 
Noteworthy for this commenting thread is the 50/50 split in the number of members posting in favor of action on climate change (on one side) and those who posted anti-Greens and anti-climate change science posts and memes (on the other side).
The two sides aligned strongly with political partisanship --- either with Liberal/National coalition (climate change deniers) or Labor/Green (climate change believers) parties. 
This is rather unusual for pages classified as far-right. 

We observed trolling practices between the climate change deniers and believers, which often descend into \emph{flame wars} --- i.e., online ``firefights that take place between disembodied combatants on electronic bulletin boards''~\citep{bukatman1994flame}.
The result is a boosted engagement on the post but also the frustration and confusion of community members and lurkers who came to the discussions to become informed or debate rationally on key differences between the two positions.
They often even become targeted, victimized, and baited by trolls on both sides of the partisan divide. 
The opinions expressed by deniers in commenting sections range from skepticism regarding climate change science to plain denial.
Deniers also regard a range of targets as embroiled in a climate change conspiracy to deceive the public, such as The Greens and their environmental policy, in some cases the government, the United Nations, and climate change celebrities like David Attenborough and Greta Thunberg. 
These figures are blamed for either exaggerating risks of climate change or creating a climate change hoax to increase the influence of the UN on domestic governments or to increase domestic governments' social control over citizens. 

Both coders noted that flame wars between these opposing personas contained very few links to external media. 
Where links were added, they often seemed disconnected from the rest of the conversation and were from users whose profiles suggested they believed in more radical conspiracy theories.
One such example is ``geo-engineering'' (see \cref{sub@subfig:comment-post-1}).
Its adherents believe that solar geo-engineering programs designed to combat climate change are secretly used by a global elite to depopulate the world through sterilization or to control and weaponize the weather.

Nonetheless, apart from the random comments that hijack the thread, redirecting users to external ``alternative'' news sites and Twitter, and the trolls who seem to delight in victimizing unsuspecting victims, the discussion was pretty healthy.
There are many questions, rational inquiries, and debates between users of different political persuasion and views on climate change.
This, however, changes in the span of only a couple of months.

\subsubsection*{Exemplar 2 --- posting and commenting thread.}
We observe a shift in the comment section of the post collected during the second wave of the COVID pandemic (\Cref{sub@subfig:second-posting}) --- which coincided with government laws mandating the public to wear masks and stay at home in Victoria, Australia.
There emerges much more extreme far-right content that converges with anti-vaccination opinions and content.
We also note a much higher prevalence of conspiracy theories often implicating racialized targets.
This is exemplified in the comments on the second post (\Cref{sub@subfig:comment-post-2a,sub@subfig:comment-post-2b}) where Islamophobia and antisemitism are confidently asserted alongside anti-mask rhetoric.
These comments consider face masks similar to the religious head coverings worn by some Muslim women, which users describe as ``oppressive'' and ``silencing''. 
In this way, anti-maskers cast women as a distinct, sympathetic marginalized demographic.
However, this is enacted alongside the racialization and demonization of Islam as an oppressive religion. 

Given the extreme racialization of anti-mask rhetoric, some commenters contest these positions, arguing that the page is becoming less an anti-Scott Morrison page (Australia's Prime Minister at the time) and changing into a page that harbors ``far-right dickheads''.
This questioning is actively challenged by far-right commenters and conspiracy theorists on the page, who regarded pro-mask users and the Scott Morrison government as ``puppets'' being manipulated by higher forces (see \Cref{sub@subfig:comment-post-2b}). 

This indicates a significant change on the page's membership towards the extreme-right, who employs more extreme forms of racialized imagery, with more extreme opinion being shared.
Conspiracy theorists become more active and vocal, and they consistently challenge the opinions of both center conservative and left-leaning users. 
This is evident in the final two comments in \Cref{subfig:comment-post-2b}, which reflect QAnon style conspiracy theories and language.
Public health orders to wear masks are being connected to a conspiracy that all of these decisions are directed by a secret network of global Jewish elites, who manipulate the pandemic to increase their power and control. 
This rhetoric intersects with the contemporary ``QAnon'' conspiracy theory, which evolved from the ``Pizzagate'' conspiracy theory.
They also heavily draw on well-established antisemitic blood libel conspiracy theories, which foster beliefs that a powerful global elite is controlling the decisions of organizations such as WHO and are responsible for the vaccine rollout and public health orders related to the pandemic.
The QAnon conspiracy is also influenced by Bill Gates' Microchips conspiracy theory, i.e., the theory that the WHO and the Bill Gates Foundation global vaccine programs are used to inject tracking microchips into people.

These conspiracy theories have, since COVID-19, connected formerly separate communities and discourses, uniting existing anti-vaxxer communities, older demographics who are mistrustful of technology, far-right communities suspicious of global and national left-wing agendas, communities protesting against 5G mobile networks (for fear that they will brainwash, control, or harm people), as well as generating its own followers out of those anxious during the 2020 onset of the COVID-19 pandemic.
We detect and describe some of these opinion dynamics in the next section.


\mySection{Related Works and Discussion}{}
\label{chap3:sec:discussion}

In this section we briefly discuss the similarities and differences of the model presented in this chapter, comparing it with some related work presented earlier (Chapter \ref{chap1:artifact-centric-bpm}). We will mention a few related studies and discuss directly; a more formal comparative study using qualitative and quantitative metrics should be the subject of future work.

Hull et al. \citeyearpar{hull2009facilitating} provide an interoperation framework in which, data are hosted on central infrastructures named \textit{artifact-centric hubs}. As in the work presented in this chapter, they propose mechanisms (including user views) for controlling access to these data. Compared to choreography-like approach as the one presented in this chapter, their settings has the advantage of providing a conceptual rendezvous point to exchange status information. The same purpose can be replicated in this chapter's approach by introducing a new type of agent called "\textit{monitor}", which will serve as a rendezvous point; the behaviour of the agents will therefore have to be slightly adapted to take into account the monitor and to preserve as much as possible the autonomy of agents.

Lohmann and Wolf \citeyearpar{lohmann2010artifact} abandon the concept of having a single artifact hub \cite{hull2009facilitating} and they introduce the idea of having several agents which operate on artifacts. Some of those artifacts are mobile; thus, the authors provide a systematic approach for modelling artifact location and its impact on the accessibility of actions using a Petri net. Even though we also manipulate mobile artifacts, we do not model artifact location; rather, our agents are equipped with capabilities that allow them to manipulate the artifacts appropriately (taking into account their location). Moreover, our approach considers that artifacts can not be remotely accessed, this increases the autonomy of agents.

The process design approach presented in this chapter, has some conceptual similarities with the concept of \textit{proclets} proposed by Wil M. P. van der Aalst et al. \citeyearpar{van2001proclets, van2009workflow}: they both split the process when designing it. In the model presented in this chapter, the process is split into execution scenarios and its specification consists in the diagramming of each of them. Proclets \cite{van2001proclets, van2009workflow} uses the concept of \textit{proclet-class} to model different levels of granularity and cardinality of processes. Additionally, proclets act like agents and are autonomous enough to decide how to interact with each other.

The model presented in this chapter uses an attributed grammar as its mathematical foundation. This is also the case of the AWGAG model by Badouel et al. \citeyearpar{badouel14, badouel2015active}. However, their model puts stress on modelling process data and users as first class citizens and it is designed for Adaptive Case Management.

To summarise, the proposed approach in this chapter allows the modelling and decentralized execution of administrative processes using autonomous agents. In it, process management is very simply done in two steps. The designer only needs to focus on modelling the artifacts in the form of task trees and the rest is easily deduced. Moreover, we propose a simple but powerful mechanism for securing data based on the notion of accreditation; this mechanism is perfectly composed with that of artifacts. The main strengths of our model are therefore : 
\begin{itemize}
	\item The simplicity of its syntax (process specification language), which moreover (well helped by the accreditation model), is suitable for administrative processes;
	\item The simplicity of its execution model; the latter is very close to the blockchain's execution model \cite{hull2017blockchain, mendling2018blockchains}. On condition of a formal study, the latter could possess the same qualities (fault tolerance, distributivity, security, peer autonomy, etc.) that emanate from the blockchain;
	\item Its formal character, which makes it verifiable using appropriate mathematical tools;
	\item The conformity of its execution model with the agent paradigm and service technology.
\end{itemize}
In view of all these benefits, we can say that the objectives set for this thesis have indeed been achieved. However, the proposed model is perfectible. For example, it can be modified to permit agents to respond incrementally to incoming requests as soon as any prefix of the extension of a bud is produced. This makes it possible to avoid the situation observed on figure \ref{chap3:fig:execution-figure-4} where the associated editor is informed of the evolution of the subtree resulting from $C$ only when this one is closed. All the criticisms we can make of the proposed model in particular, and of this thesis in general, have been introduced in the general conclusion (page \pageref{chap5:general-conclusion}) of this manuscript.





% \vspace{-0.5em}
\section{Conclusion}
% \vspace{-0.5em}
Recent advances in multimodal single-cell technology have enabled the simultaneous profiling of the transcriptome alongside other cellular modalities, leading to an increase in the availability of multimodal single-cell data. In this paper, we present \method{}, a multimodal transformer model for single-cell surface protein abundance from gene expression measurements. We combined the data with prior biological interaction knowledge from the STRING database into a richly connected heterogeneous graph and leveraged the transformer architectures to learn an accurate mapping between gene expression and surface protein abundance. Remarkably, \method{} achieves superior and more stable performance than other baselines on both 2021 and 2022 NeurIPS single-cell datasets.

\noindent\textbf{Future Work.}
% Our work is an extension of the model we implemented in the NeurIPS 2022 competition. 
Our framework of multimodal transformers with the cross-modality heterogeneous graph goes far beyond the specific downstream task of modality prediction, and there are lots of potentials to be further explored. Our graph contains three types of nodes. While the cell embeddings are used for predictions, the remaining protein embeddings and gene embeddings may be further interpreted for other tasks. The similarities between proteins may show data-specific protein-protein relationships, while the attention matrix of the gene transformer may help to identify marker genes of each cell type. Additionally, we may achieve gene interaction prediction using the attention mechanism.
% under adequate regulations. 
% We expect \method{} to be capable of much more than just modality prediction. Note that currently, we fuse information from different transformers with message-passing GNNs. 
To extend more on transformers, a potential next step is implementing cross-attention cross-modalities. Ideally, all three types of nodes, namely genes, proteins, and cells, would be jointly modeled using a large transformer that includes specific regulations for each modality. 

% insight of protein and gene embedding (diff task)

% all in one transformer

% \noindent\textbf{Limitations and future work}
% Despite the noticeable performance improvement by utilizing transformers with the cross-modality heterogeneous graph, there are still bottlenecks in the current settings. To begin with, we noticed that the performance variations of all methods are consistently higher in the ``CITE'' dataset compared to the ``GEX2ADT'' dataset. We hypothesized that the increased variability in ``CITE'' was due to both less number of training samples (43k vs. 66k cells) and a significantly more number of testing samples used (28k vs. 1k cells). One straightforward solution to alleviate the high variation issue is to include more training samples, which is not always possible given the training data availability. Nevertheless, publicly available single-cell datasets have been accumulated over the past decades and are still being collected on an ever-increasing scale. Taking advantage of these large-scale atlases is the key to a more stable and well-performing model, as some of the intra-cell variations could be common across different datasets. For example, reference-based methods are commonly used to identify the cell identity of a single cell, or cell-type compositions of a mixture of cells. (other examples for pretrained, e.g., scbert)


%\noindent\textbf{Future work.}
% Our work is an extension of the model we implemented in the NeurIPS 2022 competition. Now our framework of multimodal transformers with the cross-modality heterogeneous graph goes far beyond the specific downstream task of modality prediction, and there are lots of potentials to be further explored. Our graph contains three types of nodes. while the cell embeddings are used for predictions, the remaining protein embeddings and gene embeddings may be further interpreted for other tasks. The similarities between proteins may show data-specific protein-protein relationships, while the attention matrix of the gene transformer may help to identify marker genes of each cell type. Additionally, we may achieve gene interaction prediction using the attention mechanism under adequate regulations. We expect \method{} to be capable of much more than just modality prediction. Note that currently, we fuse information from different transformers with message-passing GNNs. To extend more on transformers, a potential next step is implementing cross-attention cross-modalities. Ideally, all three types of nodes, namely genes, proteins, and cells, would be jointly modeled using a large transformer that includes specific regulations for each modality. The self-attention within each modality would reconstruct the prior interaction network, while the cross-attention between modalities would be supervised by the data observations. Then, The attention matrix will provide insights into all the internal interactions and cross-relationships. With the linearized transformer, this idea would be both practical and versatile.

% \begin{acks}
% This research is supported by the National Science Foundation (NSF) and Johnson \& Johnson.
% \end{acks}

%%
%% The acknowledgments section is defined using the "acks" environment
%% (and NOT an unnumbered section). This ensures the proper
%% identification of the section in the article metadata, and the
%% consistent spelling of the heading.

\begin{acks}
This research was supported by the Natural Science Foundation of China (NSFC No.62202105), Shanghai Municipal Science and Technology (No. 21ZR1403300 and No. 21YF1402900), and Hong Kong Research Grants Council General Research Fund 16210722.
We thank the anonymous reviewers, participants in the user studies, Zhan Wang, Ziyue Lin, Dr. Xinhuan Shu, Dr. Jiaxiong Hu, and Prof. Zhenjie Zhao for valuable feedback.
\end{acks}

%\bibliographystyle{abbrv}
\bibliographystyle{abbrv-doi}
%\bibliographystyle{abbrv-doi-narrow}
%\bibliographystyle{abbrv-doi-hyperref}
%\bibliographystyle{abbrv-doi-hyperref-narrow}

\bibliography{main}
\end{document}

