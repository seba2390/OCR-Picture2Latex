\documentclass[journal]{vgtc}         

\ifpdf%                                % if we use pdflatex
  \pdfoutput=1\relax                   % create PDFs from pdfLaTeX
  \pdfcompresslevel=9                  % PDF Compression
  \pdfoptionpdfminorversion=7          % create PDF 1.7
  \ExecuteOptions{pdftex}
  \usepackage{graphicx}                % allow us to embed graphics files
  \DeclareGraphicsExtensions{.pdf,.png,.jpg,.jpeg} % for pdflatex we expect .pdf, .png, or .jpg files
\else%                                 % else we use pure latex
  \ExecuteOptions{dvips}
  \usepackage{graphicx}                % allow us to embed graphics files
  \DeclareGraphicsExtensions{.eps}     % for pure latex we expect eps files
\fi%

\graphicspath{{figures/}{pictures/}{images/}{./}} % where to search for the images

\usepackage{microtype}                 % use micro-typography (slightly more compact, better to read)
\PassOptionsToPackage{warn}{textcomp}  % to address font issues with \textrightarrow
\usepackage{textcomp}                  % use better special symbols
\usepackage{mathptmx}                  % use matching math font
\usepackage{times}                     % we use Times as the main font
\renewcommand*\ttdefault{txtt}         % a nicer typewriter font
\usepackage{cite}                      % needed to automatically sort the references
\usepackage{tabu}                      % only used for the table example
\usepackage{booktabs}                  % only used for the table example
\usepackage{enumerate}
\usepackage[shortlabels]{enumitem}
\usepackage{caption}
\usepackage{subcaption}

\onlineid{0}

%% declare the category of your paper, only shown in review mode
\vgtccategory{Research}

\vgtcpapertype{please specify}

%% Paper title.
\title{Design and Evaluation of Scalable Representations of Communication in Gantt Charts for Large-scale Execution Traces}

%% indicate IEEE Member or Student Member in form indicated below
\author{Connor Scully-Allison, Katherine E. Isaacs}
\authorfooter{
%% insert punctuation at end of each item
\item
 Connor Scully-Allison, University of Arizona. E-mail: cscullyallison@email.arizona.edu.
\item
 Katherine E. Isaacs, University of Arizona. E-mail: kisaacs@cs.arizona.edu.
}

\abstract{
    Gantt charts are frequently used to explore execution traces of large-scale parallel programs found in high-performance computing (HPC). In these visualizations, each parallel processor is assigned a row showing the computation state of a processor at a particular time. Lines are drawn between rows to show communication between these processors. When drawn to align equivalent calls across rows, structures can emerge reflecting communication patterns employed by the executing code. However, though these structures have the same definition at any scale, they are obscured by the density of rendered lines when displaying more than a few hundred processors. A more scalable metaphor is necessary to aid HPC experts in understanding communication in large-scale traces. To address this issue, we first conduct an exploratory study to identify what visual features are critical for determining similarity between structures shown at different scales. Based on these findings, we design a set of glyphs for displaying these structures in dense charts. We then conduct a pre-registered user study evaluating how well people interpret communication using our new representation versus their base depictions in large-scale Gantt charts. Through our evaluation, we find that our representation enables users to more accurately identify communication patterns compared to full renderings of dense charts. We discuss the results of our evaluation and findings regarding the design of metaphors for extensible structures.
} % end of abstract

\keywords{}

\CCScatlist{ % not used in journal version
 \CCScat{K.6.1}{Management of Computing and Information Systems}%
{Project and People Management}{Life Cycle};
 \CCScat{K.7.m}{The Computing Profession}{Miscellaneous}{Ethics}
}

%% A teaser figure can be included as follows
\teaser{
  \centering
  \includegraphics[width=\linewidth]{figures/header-story.pdf}
  \caption{Communication between rows in a Gantt chart are interpretable at small scales but not further, despite being highly regular. We study how people identify and distinguish similar structures at different scales and design a scale-agnostic representation.}
  \label{fig:teaser}
}

\vgtcinsertpkg

%commands
\newcommand{\note}[1]{\textcolor[rgb]{1.0,0.0,0.0}{#1}}

%%%%%%%%%%%%%%%%%%%%%%%%%%%%%%%%%%%%%%%%%%%%%%%%%%%%%%%%%%%%%%%%
%%%%%%%%%%%%%%%%%%%%%% START OF THE PAPER %%%%%%%%%%%%%%%%%%%%%%
%%%%%%%%%%%%%%%%%%%%%%%%%%%%%%%%%%%%%%%%%%%%%%%%%%%%%%%%%%%%%%%%%

\begin{document}

\firstsection{Introduction}

%% the only exception to this rule is the \firstsection command
\maketitle

Reinforcement learning has achieved great success in areas such as Game-playing \citep{silver2018general,vinyals2019grandmaster}, robotics \cite{kober2013reinforcement}, large language models \citep{ouyang2022training}, etc.
However, due to safety concerns or physical limitations, in some real-world reinforcement learning problems, we must consider additional constraints that may influence the optimal policy and the learning process \citep{garcia2015comprehensive}.
% For example, a robotic arm must not take actions that may cause harm to itself or the environments.
A standard framework to handle such cases is the constrained Markov Decision Process (CMDP) \citep{altman1999constrained}.
Within the CMDP framework, the agent has to maximize
the expected cumulative reward while
obeying a finite number of constraints, which are usually in the form of expected cumulative cost criteria.

However, we are sometimes concerned with the problem with a continuum of constraints.
For example,
the constraints we meet might be time-evolving or subject to uncertain parameters, which
cannot be formulated as an ordinary CMDP
(see Examples \ref{Example_Time_Evolving} and  \ref{Example_Uncertain}).
In this paper we would study a generalized CMDP  
to address the above problem.  Because the constraints are not only infinite-number but also lie
in a continuous set,
the generalization is not trivial. Fortunately, we find that we can borrow the idea behind semi-infinite programming (SIP) \citep{remez1934determination, hettich1993semi} to deal with the semi-infinite constraints.
Accordingly, we propose \emph{semi-infinitely constrained Markov decision processes} (SICMDPs)
as a novel complement to the ordinary CMDP framework.
%More specifically,  an SICMDP model %, we consider 
%contains a continuum of constraints whereas an ordinary CMDP contains a finite number of constraints. 

%This generalization is natural but not trivial. However, we can brows the idea  
%The idea is quite natural and can be backtracked
%to the practice of extending linear programming to linear semi-infinite programming (LSIP) %\cite{remez1934determination, GobernaLSIO1998}.
%In addition, 
%As a complementary approach to the ordinary CMDP framework, 
%SICMDP can be used to model these problems  which cannot be described by a finite number of constraints
%that are not covered by .
%For example,
%the restrictions we consider can be time-evolving or subject to uncertain parameters
%, thus
%cannot be described by a finite number of constraints but a continuum of constraints 
%(see Examples \ref{Example_Time_Evolving} and  \ref{Example_Uncertain}).

We also present two reinforcement learning algorithms to solve SICMDPs called SI-CRL and SI-CPO, respectively.
SI-CRL is a model-based reinforcement learning algorithm designed for tabular cases, and SI-CPO is a policy optimization algorithm for non-tabular cases.
% and analyze its performance both theoretically and empirically.
The main challenge is that we need to deal with a continuum of constraints, thus reinforcement learning algorithms for ordinary CMDPs do not work anymore.
In SI-CRL, we tackle this difficulty by first transforming the reinforcement learning problem to an equivalent LSIP problem, which can then be solved using methods in the LSIP literature like the dual exchange methods \citep{Hu1990,reemtsen1998numerical}.
In SI-CPO, we resort to the idea of cooperative stochastic approximation developed in \cite{lan2020algorithms, wei2020comirror}.
As far as we know, we are the first to introduce tools from semi-infinitely programming (SIP) into the reinforcement learning community for solving constrained reinforcement learning problems.

% To the best of our knowledge, we are the first to apply tools from semi-infinitely programming (SIP) to solve reinforcement learning problems.
Furthermore, we give theoretical analysis for both SI-CRL and SI-CPO.
We decompose the error of SI-CRL into two parts: the statistical error from approximating the true SICMDP with an offline dataset and the optimization error due to the fact that the solution of the LSIP problem obtained by the dual exchange method is inexact.
On the optimization side, we show that the iteration complexity of SI-CRL is $O\left(\left\{\mathrm{diam}(Y)L\sqrt{|\gS|^2|\gA|m}/\left[(1-\gamma)\epsilon\right]\right\}^m\right)$.
On the statistical side, we show that the sample complexity of SI-CRL is $\widetilde O\left(\frac{|S|^2|A|^2}{\epsilon^2(1-\gamma)^3}\right)$ if the offline dataset is generated by a generative model, and $\widetilde O\left(\frac{|S||A|}{\nu_{\min} \epsilon^2(1-\gamma)^3}\right)$ if the dataset is generated by a probability measure $\nu$ as considered in \cite{chen2019information}.
Here $\widetilde O$ means that all logarithm terms are discarded.
For SI-CPO, things become a little more complicated because other than the statistical error and the optimization error, we also need to consider the function approximation error, which comes from imperfect policy parametrizations.
It is shown if the function approximation error can be controlled to $O(\epsilon)$ order, the iteration complexity of SI-CPO is $\widetilde{O}\left(\frac{1}{\epsilon^2(1-\gamma)^6}\right)$ and the sample complexity of SI-CPO is $\widetilde{O}(\frac{1}{\epsilon^4(1-\gamma)^{10}})$.
Here our iteration complexity bound is equivalent to a typical $\widetilde O(1/\sqrt{T})$ global convergence rate.

We perform a set of numerical experiments to illustrate the SICMDP model and validate our proposed algorithms.
Specifically, we examine two numerical examples, namely the discharge of sewage and ship route planning.
Through the discharge of sewage example, we show the advantage of the SICMDP framework over the CMDP baseline obtained by naive discretization in modeling realistic sequential decision-making problems.
Moreover, we demonstrate the effectiveness of the SI-CRL and SI-CPO algorithms in such tabular environments. 
In the ship route planning example, we illustrate the benefits of the SICMDP framework and the ability of the SI-CPO algorithm to address complex continuous control tasks involving continuous state spaces with modern deep reinforcement learning techniques.

% In summary, our contributions are listed as follows.
% First, we present the SICMDP model, which can be viewed as a generalization of the ordinary CMDP model.
% Second, we propose an algorithm to perform reinforcement learning for SICMDPs, which is called SI-CRL, and we believe that we are the first to apply tools from SIP
% to solve reinforcement learning problems.
% Third, we give a theoretical analysis of SI-CRL and identify both its sample complexity and iteration complexity.
% In addition, we perform numerical experiments to illustrate the SICMDP model and validate the SI-CRL algorithm.
% \{This paragraph can be removed!!! \}







% Panoptic segmentation

% 3D segmentation

% Multi-object tracking

% Online 3D panoptic:

% PanopticFusion: (IROS 2019)
% https://arxiv.org/pdf/1903.01177.pdf
%
% - most similar to ours
% - PSPNet + M-RCNN + 2D fusion
% - volumetric mapping, 
% - greedy matching with IoU -> optimal only with 0.5 threshold
% - voxel & class weighting
% - CRF regularisation
%
% - good:
%
% - bad:
%  - CRF post-processing step
%  - greedy data-association
%    - can't be tuned for lower overlap ratios -> has to have high framerate, large changes in viewpoint could break this
%    - IoU: sensitive to 2D labels projecting over object borders (CRF and voxel weighting seem to alleviate this)

% Voxblox++: (Robotics & automation letters 2019)
% https://arxiv.org/pdf/1903.00268.pdf
% https://github.com/ethz-asl/voxblox-plusplus
%
% - M-RCNN + geometric segmentation + fusion 
% - data association of geometric segments with 3D overlap (no. points inside volume), fixed threshold for min number of points
% - instance label is assigned to a segment based on highest overlap
% - only one detected segment per reference label, as in PanopticFusion and Ours
% - TSDF Integration 
%
% good: 
% - because of geometric segmentation objects with no associated semantic class can also be segmented
% bad:
% - two different object segment types -> confusing, overly complicated ?
% - quite inaccurate (fixed below)

% Reconstructing Interactive 3D Scenes by Panoptic Mapping and CAD Model Alignments (ICRA 2021)
% https://arxiv.org/pdf/2103.16095.pdf
% https://github.com/hmz-15/Interactive-Scene-Reconstruction
%
% - based heavily on Voxblox++, much more accurate
% - Scene-graph ("contact graph") for mapping object relations
% - Search & replace voxels with CAD models, with geometrical and physical constraints
% - Object 6D pose
% - Format for robot interaction
%
% - Segmentation: bilateral fusion of geomatric and semantic segments -> reduce segmentation noise compared to Voxblox++
% - Fusion: triplet count improves consistency over Voxblox++ pairwise count strategy (take semantic label into account in addition to instance and geometry)
% - Fusion: instance labels are also combined if there is enough overlap with common geometric label for long enough time
%   - this means multiple detections can match the same reference unlike ours, voxblox++ and PanopticFusion ?
%

% Panoptic-MOPE: (ROBOTICS AND AUTOMATION LETTERS 2020)
% https://ieeexplore.ieee.org/stamp/stamp.jsp?tp=&arnumber=8977356
% https://github.com/hoangcuongbk80/Object-RPE/tree/panoptic-mope
%
% - novel RGB-D semantic segmentation model + M-RCNN
% - camera tracking based on "addaptively weighted optimization of geometric, appearance, and semantic cues"
% - surfel map: 
%   - how does it scale ? authors satate they tested on room-sized environments, but could be applied in larger scale as well ...
%     - could maybe be applied as VO in a SLAM algorithm ...
%   - demo only on a small pallet + surroundings, might not be applicable in large-scale SLAM

% US VS THEM:
%
% - based heavily on PanopticFusion, with modifications:
%   - instead of greedy data-association (which seems to be the case in others as well), we solve LAP (JPDA?)
%     - overlap threshold can be tuned, which renders the algorithm more flexible
%     - could be extended to dynamic tracking ?
%   - multiple options for association likelihood
%   - outlier rejection (either clustering or probabilistic)
%   - test different options for decreasing processing time
%   - no post-processing
%
% - model-agnostic:
%   - completely separated from segmentation
%   - does not care how point clouds are obtained -> applicable for LIDAR segmentation (e.g. EfficientLPS) as well
%
% - also agnostic to localisation method
%   - could, however, be utilised to find landmark locations / poses

% More compact version of this paragraph to introduction to save space?
%Panoptic segmentation -- proposed in \cite{panoptic_segmentation} -- aims to solve the unified task of semantic- and instance segmentation. Semantic classes are separated to \textit{stuff} -- amorphous, unquantifiable regions like sky, road or floor -- and \textit{things} -- quantifiable objects. The distinction between the two can vary depending on the application, but a semantic class can only belong to one or another. The article also proposes a unified panoptic evaluation metric, coined \textbf{Panoptic Quality} (PQ). Many 2D approaches to panoptic segmentation -- \textit{e.g.} \cite{panopticfpn,seamless,panoptic_deeplab,efficientps} -- have since been proposed. Deep neural networks for performing semantic- or instance segmentation directly on the 3D reconstruction -- \textit{e.g.} on \cite{scannet,s3dis,paris_lille_3d} -- have also been proposed, but since they require the reconstructed 3D scene, they are mostly offline approaches and therefore out of scope for this work. Some recent works also apply panoptic segmentation to point clouds -- \textit{e.g.} methods in the SemanticKITTI panoptic segmentation competition \cite{semantic_kitti} -- mostly aimed at segmenting LiDAR output. They are suitable for online processing, but similar to RGB-D images require a method for tracking object instances persistent in both time and space. In fact, our proposed method, as well as some others mentioned in this work, could use segmented LiDAR point clouds as an input similarly to RGB-D images.

PanopticFusion \cite{panopticfusion} is the first work to propose online integration of panoptic image segmentations to a 3D reconstruction. They integrate point clouds generated from segmented images to a TSDF voxel volume \cite{tsdf,voxblox} by greedily matching detected segments with the reconstruction and regulating each voxel's corresponding instance with a weighting function. Semantic labels are inferred in a bayesian manner based on confidence scores provided by the segmentation model. They also apply a Conditional Random Field (CRF) to regularise the reconstruction, improving results significantly. Voxblox++ \cite{voxblox++} -- introduced later the same year -- is a similar approach that also integrates segmented RGB-D images into a TSDF volume. It leverages geometric segmentation of depth images to improve instance segmentation accuracy. Both geometric and semantic segments are used to compute a pair-wise weight, which is used to greedily match them with segments in the reconstruction. Because of the geometric segmentation, the method allows segmentation of objects with no known semantic class in addition to objects recognised by the instance segmentation model. 

Recently, \cite{interactive_3d_scenes} built upon the idea of Voxblox++. They apply Voxblox++ for 3D instance integration, with two small but effective modifications: the pair-wise weight is replaced by a triplet weight that also takes semantic labels into account in the fusion, and -- in addition to geometric segments -- instance segments are fused if they overlap by a significant amount. The article introduces a method for searching and aligning CAD models to reconstructed objects based on geometry and semantic class, as well as geometrical and physical rules. With the CAD models, a contact graph and interactive virtual scene are reconstructed to allow a robot to simulate its interaction with the environment. SceneGraphFusion \cite{scenegraphfusion} is another approach that forms a scene graph online from a stream of RGB-D images, but unlike the above-mentioned approach, it generates the graph with a deep neural network, after which the panoptic labels for geometrically segmented portions of the 3D reconstruction are produced a side product.

Panoptic-MOPE \cite{panoptic_mope} is another recent approach, which integrates sequences of RGB-D images into a surfel reconstruction. Unlike other mentioned approaches -- which assume the camera pose either known or estimated elsewhere -- it also tracks camera movements based on geometric-, appearance- and semantic cues. The method also applies a novel RGB-D panoptic segmentation model. Although it is only tested on room-sized environments, the authors claim it could be scaled to larger environments as well.

\section{Preliminary Study}
\label{sec:prelim}

To aid the design of scalable representations of communication patterns and design of experiments for evaluating both new and existing representations, we executed a preliminary study to explore how people interpret communication in Gantt charts. This study focused on how communication is presently drawn and sought to understand what visual factors are considered when differentiating patterns. 

There are two ways people can view communication lines in Gantt charts. The first is to look at all PEs. We call this the {\em full} representation. The problem is when the chart has many PEs which communicate, i.e., it is {\em dense}, the lines may overlap to the point of a solid shape. The other way is to look at a subset of the PEs (rows). We call this the {\em partial} representation. The problem here is that key indicators about the pattern, such as grouping, may not appear in a given window.

Prior to the study, we identified several factors that might be used in interpreting these views: the density of a chart, the structure being shown, the grouping of a structure, the representation type ({\em  partial} or {\em full}), and the stride of a structure. A study across the full range of these factors would be unreasonably large, so we designed a qualitative study to limit our focus, following the ``factor mining'' evaluation pattern discussed by Elmqvist and Yi~\cite{elmqvist2015patterns}.

\begin{figure}
    \centering
    \includegraphics[width=\columnwidth]{figures/c1.png}
    \caption{Example image prompt from our preliminary study. These are both {\em partial} representations of a stencil pattern, emulating being zoomed-in on a Gantt chart.}
    \label{fig:interview_prompt}
\end{figure}

Our procedure was to interview participants while showing them paired images of communication patterns in idealized unit time Gantt charts. We decided on semi-structured interviews to allow probing and elaboration of ideas. The paired images varied in number of rows, representation type, and whether they were the same pattern. \autoref{fig:interview_prompt} shows an example prompt. For each prompt, we asked three questions:
\vspace{0em}
\begin{enumerate}
    \itemsep=0em
    \item Please describe the pattern of lines on the right.
    \item Please describe the pattern of lines on the left.
    \item Do you think that these two patterns are the same? Why or why not?
\end{enumerate}

We recruited seven participants, one from our university and six at the Supercomputing 2019 conference. Six reported computing experience and two reported HPC experience. Of those two, one had analyzed HPC performance. None had prior experience with trace visualization. 

The interviews were recorded, transcribed, and coded for common themes, resulting in over 170 unique codes. We describe the most frequently used. See the supplemental material for a full list of codes.

\texttt{Line angle} was used by all participants to describe patterns of lines and justify comparisons between them, with three referencing it over 10 times. \texttt{Line direction} was mentioned by all but one participant, but mostly to describe a pattern rather than compare. Although a line direction could be the result of an angle, it was coded distinct from \texttt{angle} since respondents would describe lines as going ``up," ``down," ``left," or ``right". Their intent was distinct from when they mentioned angle. A few participants mistook the multiple wrap-around lines for \texttt{one line} in ring patterns, even on repeated prompting.

Four participants made note of the \texttt{background}---the boxes and columns the lines connected. This code was also used both to describe an individual pattern and differentiate between pairs.

\texttt{Comparison} and \texttt{uncertainty} were co-occurring codes. Participants rarely felt confident comparing charts. We surmise this uncertainty comes from the fact that simple changes of height and representation alter visual factors they relied upon. Four participants discussed \texttt{transformation} during comparisons, describing how one depiction could be ``rotated'',  ``squashed/stretched'', or ``enlarged.''

Participants had the most difficulty with prompts depicting stencil patterns. They were uncertain of line extents and misidentified discrete lines as single lines or vice versa. Of the four interviewees exposed to stencil patterns, three found them viscerally off-putting, pausing with surprise when presented with them and one calling them ``a mess." 

\vspace{1ex}

\textbf{Discussion.} Participants generally relied on features such as line angle, line direction, and background markings in interpreting patterns. We expect all three factors to be harder to interpret with large numbers of PEs as many angles map to the same pixels and background markers are aggregated or removed.

The line factors are functions of the pattern stride and the height, and therefore density, of the chart, suggesting density significantly impacts recognition. This is further supported with participants' difficulty in recognizing line separation at severe angles, such as the wrap-around lines in rings.

The impetus of participants to identify patterns as similar under a stretch factor matches how the patterns are extensible to PE count. However, communication patterns are not extensible under rotation or inversion, as some discussed. We use this finding to inform tutorial material in subsequent study designs.

Participants reactions to stencils suggest that the idealized unit time depiction is too complex even at small scales. As we want to understand how people can interpret patterns as they scale up, we conclude that stencils are not appropriate for early work towards this goal as they are too difficult to interpret at small scale.

Based on the results of this preliminary study, we decided to take line angle and background into careful consideration for proposed designs. We ultimately fixed line angle at set values and obscure background features to encourage focus on pattern types over strides and to maintain discernability. We also decided to remove stencils from our pattern type factor and randomize across our other factors in further studies.

\section{Scalable Representations for Communication Patterns}
\label{sec:design}

Using the lessons learned in our preliminary studies, along with existing case studies~\cite{isaacs2014combing, Isaacs2016} using idealized unit time, we design a set of strategies for representing communication patterns when there are too many PEs to draw distinct communication lines in Gantt charts. We first describe our design goals. Then, we present our designs. Finally, we discuss initial feedback from experts familiar trace analysis in HPC.


\subsection{Design Goals}

Our goal is to design a representation of communication in execution traces that (1) aids users in recognizing and understanding what communication is occurring in that temporal and logical position in the Gantt chart and (2) is agnostic to the number of processing elements, thereby scaling to larger traces. These goals are derived from usage and scalability limitations noted in prior work~\cite{isaacs2014combing}. 

We limit our focus to scaling in PEs (y-axis) rather than time. Traces are typically explored using a time window, so we focus on that case. Adapting a design or creating a new one for compressed time settings we leave for future work.

Based on our preliminary study (\autoref{sec:prelim}), we chose to focus on offset, ring, and exchange pattern types as stencils require more design consideration even at small scales. 

\subsection{Visualization Design}

Our design process began with open brainstorming on paper, which we include in the supplemental material. We tried a variety of strategies, including linked views and added channels to the traditional Gantt chart encoding rules. However, most of these retained scaling problems, leading us to focus on designs centering on glyphs.

In designing the representation, we considered the saliency of what was to be encoded (e.g., temporal range, pattern type, grouping, stride) and efficacy of available channels, taking into account that the design needs to be incorporated in a Gantt chart. For example, temporal range is set to a horizontal position matching where a pattern would be drawn in a full chart. See supplemental materials for a table containing discussion of channel considerations.

We prioritize the type of pattern before the grouping factor or stride. The rationale is that the pattern type is fixed by the source code while the grouping and stride are often computed from the problem size and number of resources. Therefore, a user will recognize pattern type first before considering other factors. \autoref{fig:abstract_designs} shows the resulting designs.


\begin{figure*}
    \centering
    \begin{subfigure}{0.18\textwidth}
         \centering
         \includegraphics[width=\textwidth]{figures/new-basic-offset.png}
         \caption{Continuous offset pattern}
         \label{fig:noc}
    \end{subfigure}
    \begin{subfigure}{0.18\textwidth}
         \centering
         \includegraphics[width=\textwidth]{figures/new-basic-offset-grouped.png}
         \caption{Grouped offset pattern}
         \label{fig:nog}
    \end{subfigure}
    \begin{subfigure}{0.18\textwidth}
         \centering
         \includegraphics[width=\textwidth]{figures/new-basic-ring.png}
         \caption{Continuous ring pattern}
         \label{fig:nrc}
    \end{subfigure}
    \begin{subfigure}{0.18\textwidth}
         \centering
         \includegraphics[width=\textwidth]{figures/new-basic-ring-grouped.png}
         \caption{Grouped ring pattern}
         \label{fig:nrg}
    \end{subfigure}
    \begin{subfigure}{0.18\textwidth}
         \centering
         \includegraphics[width=\textwidth]{figures/new-basic-exchange.png}
         \caption{Exchange pattern}
         \label{fig:neg}
    \end{subfigure}
    \caption{Examples of our designs for five communication patterns. They are reminiscent of the underlying communication pattern encoding, but not aligned to the underlying chart and agnostic to the number of rows the underlying pattern repeats over. 
    %Note that the angle for our ring pattern is slightly shallower than the angle for offset, this reflects the difference in stride between the two patterns. 
    Grouped representations fill the vertical space to indicate that the repetition continues from the top of row to the bottom.}
    \label{fig:abstract_designs}
\end{figure*}

\vspace{1ex}

\textbf{Encoding Pattern Type.} To encode the pattern type, we started with the overall shape of of the pattern when drawn at small scale with a small stride. Offsets are drawn with angled repeating lines forming a rhombus-like shape. We use a fixed distance between lines and draw as many will fit in the relevant area.

Rings add indicators of the ``wrap-around'' communication. However, unlike fully drawn rings, we only render the protruding segments at the ends of the shape. There are two main rationales for only drawing protruding segments: (1) we want to indicate this is an abstraction and (2) participants in our interviews found the crossing lines difficult to disambiguate. The number of protruding segments is proportional to the stride of a ring. 
%For a ring with a small stride, one segment is added. This increases to a max of four for a very large stride.

Exchange patterns are drawn as a series of symmetrical ``x'' shapes and avoid direct crossings for the same reason as rings. The number of lines in each cross is proportional to stride of the exchange. Short stride exchanges will exchange between only a few PEs, a long stride exchange spans many PEs. Our glyphs approximate this by increasing the number of crossing lines as stride increases.

\vspace{1ex}

\textbf{Encoding Grouping Factor.} To represent grouping, we partition the available area vertically and repeat the pattern type drawing in those partitions. More formally, the encoding rule to show ``grouping" is repetition and alignment on a non-common scale. The number of partitions is determined by the available vertical space in a chart.


\vspace{1ex}

\textbf{Encoding Stride.} We express the notion of stride through the angle of lines used in our pattern types. As people had difficulty with steeply angled lines in our preliminary study, we limit the angles to a range of 15 degrees to 60 degrees. Therefore, these do not match the encoding of a full view. Instead, they hint at the magnitude of distance over which communication is occurring. This allows users to see that there are differences in stride between glyphs, but not necessarily calculate the exact stride visually.

\vspace{1ex}

\textbf{Temporal range.} Rather than show the exact range, we place the glyphs on the x-axis so they are centered in their range. If two structures overlap, they are placed alongside one another. 

\vspace{1ex}

\textbf{Incorporation in Gantt Charts.} These are designed to be used in Gantt charts when exact lines would be too dense to be interpreted. The underlying interval rectangles will still be drawn. The color encoding of these intervals was shown to be a secondary indicator in our preliminary study, so we preserve them. We add a slight blur effect to the background as another signifier that the glyphs are an abstraction and should not be confused for exact lines.



\subsection{Expert Feedback}
\label{sec:expertfeedback}

We sent our designs to two HPC experts for feedback regarding both the designs themselves and the overall approach. Both experts were familiar with idealized unit time representations of traces. The first expert, E1, had previously collaborated on this strategy with the authors but was not involved in any of the work presented here. The second expert, E2, had managed an integration of the strategy into an HPC center's performance tools, referencing the open-source research code~\cite{isaacs2014combing} but using an alternate calculation method and front-end technology.

We sent both experts a short email with a PDF describing the visualizations with comparisons to fully drawn traces and showing how they might be applied in practice, including a few complicated examples such as zoomed-out time and idle processes. (See supplemental materials.) We asked if and how the strategy would be useful and if there were any suggestions or concerns. E1 responded the designs ``definitely look helpful,'' noted the trade off in exactness, and then pointed out figures which led to ambiguities in his view. He also identified a error where the mock-up did not match the underlying trace. E2 noted that stride is less important and wondered how the translation from data to glyph would be calculated. He suggested the strategy might also be helpful for collective communications (e.g., broadcasts, all-to-all, reductions), a set of patterns we did not consider in this work.

We interpreted these responses to suggest the designs were worth further study, particularly E1's ability to interpret well enough to detect an error and E2's interest in further patterns. However, there are design decisions in applying these glyphs in some scenarios, particularly in zoomed-out time, that require refinement. We leave these cases for future iterations and instead focus on how the base designs could be interpreted by a wider range of users in a controlled study.

\newcommand{\twomoons}{{\tt Twomoons}}
\newcommand{\gauss}{{\tt Gauss}}
\newcommand{\sculpture}{{\tt Sculpture}}
\newcommand{\baseline}{{\tt Baseline}}
\newcommand{\MM}{{\tt MsgPassing}}
\newcommand{\blackboard}{{\tt Blackboard}}
\newcommand{\ncut}{\text{ncut}}
\newcommand{\chensays}[2][]{\textcolor{blue} {\textsc{Jiecao #1:} \emph{#2}}}

\section{Experiments}
In this section we present experimental results for  graph clustering in the message passing and blackboard models. We will compare the following three algorithms. (1) \baseline: each site sends all the data to the coordinator directly; (2) \MM: our algorithm in the message passing model (Section~\ref{sec:gcmessage}); (3) 
\blackboard: our algorithm in  the blackboard model (Section~\ref{sec:bb}).


%Since both of our algorithms are crucially based on the use of spectral scarification, our main focus in the experiments is to investigate to what extend the quality of the spectral clustering algorithms will be affected by using spectral sparsification, the saving of communication costs by using spectral sparsificaion, ...
%
%
%The goal of this experiment is not to demonstrate the effectiveness of the spectral clustering algorithm. We mainly want to investigate the following, 
%\begin{itemize}
%\item to what extend the quality of clustered results will be affected by using spectral sparsification.
%\item saving of communication costs by using spectral sparsifier.
%\item the affect of constants in algorithms of the message passing/blackboard model.
%\end{itemize}
%
%
%\subsection{The Setup}
%\paragraph{Reference Algorithms}
%We compare different algorithms in our experiment.

%Note that we can also run \MM~ in the blackboard model.

Besides giving the visualized results of these algorithms on various datasets, we also measure the qualities of the results via the {\em normalized cut}, defined as 
\[
\ncut(A_1, \ldots, A_{k}) = \frac{1}{2}\sum_{i\in[k]}\frac{w(A_i, V\backslash A_i)}{\vol(A_i)},
\]
 which is a standard objective function to be minimized for spectral clustering algorithms. 
%We will compare the communication costs of these algorithms in different settings.

%We also compare the total communication costs of different algorithms/models. As the unit does not matter in our case, we normalize all communication costs by the cost of \baseline.  Whenever possible, we will visualize the clustered results.

We implemented the algorithms using multiple languages, including Matlab, Python and C++. Our experiments were conducted on an IBM NeXtScale nx360 M4 server, which is equipped with 2 Intel Xeon E5-2652 v2 8-core processors, 32GB RAM and 250GB local storage.


\subsection{Datasets.}
We test the algorithms in the following real and synthetic datasets, which is visualized in \figref{visualization}.


\begin{figure}[h]
     \centering
     \subfigure[\twomoons]{\includegraphics[width=0.23\textwidth]{twomoons-14000-original.png}\label{fig:twomoons}}
     ~~
     \subfigure[\gauss]{\includegraphics[width=0.23\textwidth]{gauss-10000-original.png}\label{fig:gauss}}
     ~~
     \subfigure[\sculpture]{\includegraphics[width=0.13\textwidth,height=0.16\textwidth]{sculpture-11680-original.jpg}\label{fig:sculpture}}
     \caption{Visualization of the datasets for our experiments.}
     \label{fig:visualization}
\end{figure}



\vspace{-1mm}
\begin{itemize}
\item \twomoons : this dataset contains $n=14,000$ coordinates in $\mathbb{R}^2$. We consider each point to be a vertex. For any two vertices $u, v$, we add an edge with weight $w(u,v) = \exp\{-\|u-v\|_2^2/\sigma^2\}$ with $\sigma = 0.1$ when one vertex is among the $7000$-nearest points of the other.  This construction results in a graph with about $110,000,000$ edges.

\item  \gauss : this dataset contains $n = 10,000$ points in $\mathbb{R}^2$. There are $4$ clusters in this dataset, each generated using a Gaussian distribution. We construct a complete graph as the similarity graph.  For any two vertices $u, v$, we define the weight $w(u,v) = \exp\{-\|u-v\|_2^2/\sigma^2\}$ with $\sigma = 1$. The resulting graph has about $100,000,000$ edges.

\item \sculpture : a photo of \textit{The Greek Slave}~\footnote{Available in e.g., \url{http://artgallery.yale.edu/collections/objects/14794}}. We use an $80\times 150$ version of this photo where each pixel is viewed as a vertex. To construct a similarity graph, we map each pixel to a point in $\mathbb{R}^5$, i.e., $(x, y, r, g, b)$, where the latter three coordinates are the RGB values. For any two vertices $u, v$, we  put an edge between $u, v$ with weight $w(u,v) = \exp\{-\|u-v\|_2^2/\sigma^2\}$ with $\sigma = 0.5$ if one of $u, v$ is among the $5000$-nearest points of the other. This results in a graph with about $70,000,000$ edges.
\end{itemize}
\vspace{-1mm}
In the distributed model edges are randomly partitioned across $s$ sites. 

%\vspace{-1.5mm}



\subsection{Results on clustering quality}
%{\em Quality.} \
\begin{figure*}[ht]
     \centering
     \subfigure[\baseline]{\includegraphics[width=0.2\textwidth]{twomoons-14000-original-clustered.png}\label{fig:twomoons-clustered-original}}
     \subfigure[\MM]{\includegraphics[width=0.2\textwidth]{twomoons-14000-sparsify-clustered-15.png}\label{fig:twomoons-clustered-sparsify}}
     \subfigure[\blackboard]{\includegraphics[width=0.2\textwidth]{twomoons-14000-chain-clustered.png}\label{fig:twomoons-clustered-chain}}
     \caption*{\twomoons, $k = 2$;}

\subfigure[\baseline]{\includegraphics[width=0.2\textwidth]{gauss-10000-original-clustered.png}\label{fig:gauss-clustered-original}}
     \subfigure[\MM]{\includegraphics[width=0.2\textwidth]{gauss-10000-sparsify-clustered-15.png}\label{fig:gauss-clustered-sparsify}}
     \subfigure[\blackboard]{\includegraphics[width=0.2\textwidth]{gauss-10000-chain-clustered.png}\label{fig:gauss-clustered-chain}}
     \caption*{\gauss, $k = 4$}


     \subfigure[\baseline]{\includegraphics[width=0.2\textwidth,height=0.2\textwidth]{sculpture-11680-original-clustered.png}\label{fig:sculpture-clustered-original}}  
     \subfigure[\MM]{\includegraphics[width=0.2\textwidth,height=0.2\textwidth]{sculpture-11680-sparsify-clustered-15.png}\label{fig:sculpture-clustered-sparsify}}
     \subfigure[\blackboard]{\includegraphics[width=0.2\textwidth,height=0.2\textwidth]{sculpture-11680-chain-clustered.png}\label{fig:sculpture-clustered-chain}}
     \caption*{\sculpture, $k = 3$. }


     
     \caption{Visualization of the results on \twomoons, \gauss\ and \sculpture. In the message passing model each site samples $5 n$ edges; in the blackboard model all sites jointly sample $10n$ edges (in \twomoons~ and \gauss) or $20n$ edges (in \sculpture) and the chain has length $18$. $s = 15$.}
     \label{fig:quality-1}
\end{figure*}

We visualize the clustered results for 
the \twomoons, \gauss\ and \sculpture\ in Figure~\ref{fig:quality-1}.
% and visualize the clustered results for \gauss\ and \sculpture in Figure~\ref{fig:quality-2}.
It can be seen that \baseline, \MM\ and \blackboard\ give results of very similar qualities.  For simplicity, here we only present the visualization for $s=15$. Similar results were observed when we varied the values of $s$.  
%\he{To Qin: Do you plan to have two titles (Results \& Quality)?}


% \begin{figure*}[h]
%      \centering
% \subfigure[\baseline]{\includegraphics[width=0.3\textwidth]{gauss-10000-original-clustered.png}\label{fig:gauss-clustered-original}}
%      \subfigure[\MM]{\includegraphics[width=0.3\textwidth]{gauss-10000-sparsify-clustered-15.png}\label{fig:gauss-clustered-sparsify}}
%      \subfigure[\blackboard]{\includegraphics[width=0.3\textwidth]{gauss-10000-chain-clustered.png}\label{fig:gauss-clustered-chain}}
%      \caption*{\gauss, $k = 4$}


%      \subfigure[\baseline]{\includegraphics[width=0.2\textwidth]{sculpture-11680-original-clustered.png}\label{fig:sculpture-clustered-original}}  
%      \subfigure[\MM]{\includegraphics[width=0.2\textwidth]{sculpture-11680-sparsify-clustered-15.png}\label{fig:sculpture-clustered-sparsify}}
%      \subfigure[\blackboard]{\includegraphics[width=0.2\textwidth]{sculpture-11680-chain-clustered.png}\label{fig:sculpture-clustered-chain}}
%      \caption*{\sculpture, $k = 3$. }

%      \caption{Visualization of results on \gauss\ and \sculpture; in the message passing model each site samples $5 n$ edges; in the blackboard model all sites jointly sample $10n$ (in \gauss) or $20n$ (in \sculpture) edges and the chain has length $18$.}
%      \label{fig:quality-2}
% \end{figure*}


We also compare the normalized cut (ncut) values of the clustering results of different algorithms.  The results are presented in Figure \ref{fig:quality}. In all datasets, the ncut values of different algorithms are very close. The ncut value of \MM\ slightly decreases when we increase the value of $s$, while the ncut value of \blackboard\ is independent of $s$.
%We comment that in general, it is difficult to compare \MM\ and \blackboard\ directly because they are affected by different parameters.


\begin{figure*}[!ht]
  \centering
  \subfigure[\twomoons]{\includegraphics[width=0.33\textwidth]{twomoons-14000-ncut.png}\label{fig:twomoons-quality}}\hspace*{-1.1em}
  \subfigure[\gauss]{\includegraphics[width=0.31\textwidth]{gauss-10000-ncut.png}\label{fig:gauss-quality}}\hspace*{-1.1em}
  \subfigure[\sculpture]{\includegraphics[width=0.31\textwidth]{sculpture-11680-ncut.png}\label{fig:sculpture-quality}}\hspace*{-1.1em}
  \subfigure{\includegraphics[width=0.14\textwidth]{legend.png}}
     \caption{Comparisons on normalized cuts. In the message passing model, each site samples $5n$ edges; in each round of the algorithm in the blackboard model, all sites jointly sample $10n$ edges (in \twomoons~and \gauss) or $20n$ edges (in \sculpture) edges and the chain has length $18$.}
     \label{fig:quality}
\end{figure*}

%\textcolor{red}{To Jiecao: Can you put the color lines indicating baseline, message passing, and blackboard within one row in Pic 2? Withthis we can save some space.}

%\vspace{-1.5mm}

\subsection{Results on communication costs} 
\begin{figure*}[!ht]
     \centering
     \subfigure[\twomoons]{\includegraphics[width=0.3\textwidth]{twomoons-14000-communication.png}\label{fig:twomoons-communication}}
     \subfigure[\gauss]{\includegraphics[width=0.3\textwidth]{gauss-10000-communication.png}\label{fig:gauss-communication}}
     \subfigure[\sculpture]{\includegraphics[width=0.3\textwidth]{sculpture-11680-communication.png}\label{fig:sculpture-communication}}


     \subfigure[\twomoons]{\includegraphics[width=0.32\textwidth]{twomoons-14000-communication-2.png}\label{fig:twomoons-communication-2}}
     \subfigure[\gauss]{\includegraphics[width=0.32\textwidth]{gauss-10000-communication-2.png}\label{fig:gauss-communication-2}}
     \subfigure[\sculpture]{\includegraphics[width=0.32\textwidth]{sculpture-11680-communication-2.png}\label{fig:sculpture-communication-2}}
     \caption{Comparisons on communication costs. In the message passing model, each site samples $5n$ edges; in each round of the algorithm in the blackboard model, all sites jointly sample $10n$ (in \twomoons~and \gauss) or $20n$ (in \sculpture) edges and the chain has length $18$. }
     \label{fig:communication}
\end{figure*}

We compare the communication costs of different algorithms in Figure \ref{fig:communication}. We observe that while achieving similar clustering qualities as \baseline, both \MM\ and \blackboard\ are significantly more communication-efficient (by one or two orders of magnitudes in our experiments). We also notice that the value of $s$ does not affect the communication cost of \blackboard, while the communication cost of \MM\ grows almost linearly with $s$; when $s$ is large, \MM\ uses significantly more communication than \blackboard. These confirm our theory.  %In Figure~\ref{fig:mm-const} and Figure~\ref{fig:blackboard-const}   in Appendix~\ref{sec:parameters} we present how the performance of \MM\ and \blackboard\ are affected by their parameters.

%
%
%\vspace{-1.5mm}
%\paragraph{Summary.}  From our experimental results we conclude that \MM\ and \blackboard\ achieve similar clustering quality as the native algorithm \baseline, while significantly reduce the communication cost.  When the number of sites is large, \blackboard\ is more communication efficient than \MM, as predicted by our theory.



\subsection{Parameters in \MM\ and \blackboard}
\label{sec:parameters}

Figure \ref{fig:mm-const} shows in \MM how the value of ncut is affected by the number of sites and the number of edges sampled in each site. 
Here, each site samples $cn$ edges. 
When $c=3$ and $s=1$, the ncut value diverges in all datasets. This is because with such a small $c$, the algorithm does not generate a valid sparsifier. In general, increasing $c$ or $s$ will slightly decrease the ncut value. But once they are above some thresholds, the ncut values of \MM\ and \baseline\ become very close.

Figure \ref{fig:blackboard-const} shows in \blackboard  how the ncut value is affected by the number of iterations and the number of edges sampled. When the number of iterations is set to be $5$, ncut values diverge in all datasets. This is because we cannot expect to generate a valid sparsifier by using such few iterations. It can be seen from \ref{fig:bb-gauss-constant} that for a fixed $c$, performing more iterations will help to reduce ncut values. From the same figure, one can also conclude that for fixed iterations, increasing $c$ also helps to reduce the ncut values.



\begin{figure*}[h!t]
     \centering
     \subfigure[\twomoons]{\includegraphics[width=0.3\textwidth]{twomoons-c.png}\label{fig:mm-twomoons-constant}}
     \subfigure[\gauss~dataset]{\includegraphics[width=0.3\textwidth]{gauss-c.png}\label{fig:mm-gauss-constant}}
     \subfigure[\sculpture]{\includegraphics[width=0.3\textwidth]{sculpture-c.png}\label{fig:mm-sculpture-constant}}
     \caption{The pictures above show the $\ncut$ values with respect to the values of $c$ and $s$ for the \MM\ algorithm. Here  
 each site samples $c n$ edges.}
     \label{fig:mm-const}
\end{figure*}


\begin{figure*}[h!t]
     \centering
     \subfigure[\twomoons]{\includegraphics[width=0.3\textwidth]{twomoons-iter.png}\label{fig:bb-twomoons-constant}}
     \subfigure[\gauss]{\includegraphics[width=0.3\textwidth]{gauss-iter.png}\label{fig:bb-gauss-constant}}
     \subfigure[\sculpture]{\includegraphics[width=0.3\textwidth]{sculpture-iter.png}\label{fig:bb-sculpture-constant}}
     \caption{The pictures above show how the $\ncut$ values are affected by the number of iterations and the value of $c$ for the \blackboard\ algorithm. Here 
all sites jointly sample $c n$ edges. }
     \label{fig:blackboard-const}
\end{figure*}







% \vspace{-0.20in}
%\subsection{Analysis:}


\textbf{Use of Multiple Projection Heads:} The use of different projection heads for each view on OpenImages classification gives us a boost of $1.1$ mAP on Obj-Obj+Dilate crop. Pre-training on COCO and finetuning on VOC dataset for object-detection task gives a boost of $0.4$ mAP. Hence using multiple projection heads results in a consistent improvement. 

\textbf{Varying Dilation Parameter:} Table 3 (appendix) shows the effect of varying the dilation parameter. A sweet spot exists at a moderate dilation value of $\delta=0.1$ for COCO object detection. 

% \textbf{Computational Cost:} BING adds negligible time to the pre-training. Generating object proposals takes ~29 mins for the full OHMS dataset (one-time cost) and ~16 mins for COCO. Instead of pre-generating, adding the BING operator to the data loader pipeline has a trivial overhead (+$0.1\%$). %As an example, the wall-clock time taken for 1 epoch of training is 1'46'' for the Dense-CL baseline and 1'45'' for our method.
% \textbf{}



%Between two views, we measure the number of common pixels; and then measure the fraction of these common pixels that overlap with a ground truth bounding box (object). We find that this fraction for COCO is $99\%$ for object-scene crops and $92.1\%$ for the scene-scene crop. In the case of OpenImages-Subset, the numbers are, respectively, $99.1\%$ and $87.3\%$. This is another way of seeing that OpenImages-Subset can benefit more from object-scene crops, borne out by the numbers in Tables \ref{tab:ssl_comparison_classification} and \ref{tab:coco_detection}. 


% \as{Shlok: could you please make this description a little better and clear?}
% We find the overlapping pixels between two crops ($C_{int} = C_1 \cap C_2$). Next we calculate intersection of $C_{int}$  with the most overlapping ground truth object ($O$) and calculate the score $\frac{C_{int} \cap O}{C_{int}}$ for each image and average it. 
% To do this, we calculate the \% intersection of the most overlapping ground truth object with the inter
% Next we try to find the probability of an actual ground truth object co-occuring in between two crops. We find  object-overlap between both Scene-Object crops and Scene-Scene crops. To do this we firstly calculate the overlapping region between two crops. Overlapping region is the area of overlap between two crops before the resize operation. Then for all the ground truth objects present in the original image  we find the object with maximum overlap in the overlapping region. Intuitively for a object to have high overlap, the object should be present in both the crops. 

% Similarly instead of taking an crop with maximum overlap we calculate average of all the crops that are present in the image. We find this average probability to be 65.12 \% for Object-Scene crop and 73.47 \% for Scene-Scene crop. 
% This is consistent with the findings of the InfoMin \cite{tian2020contrastive} that there is a tradeoff between how much information views can share.  

% Similarly in the case of OpenImages we can see from Fig \ref{fig:radius_openimages} that as we increase the radius of the object-object crops the performance firstly increases and then decreases, suggesting there is a sweet point on mutual information on OpenImages dataset as well.
% \\

% \textbf{Performance on 5 classes per image images?}










In this paper, 2D and 3D CNN models were used to generate pelvic sCTs from T1-weighted MR images. Our sCT generation methods were fully automated, requiring no deformable registration or manual segmentation of bone tissues. As shown in Figure~\ref{fig3}, the 2D and 3D CNN models generated high quality sCTs. MAE curves shown in Figure~\ref{fig4} indicated that both models could precisely estimate soft-tissue HU values but had difficulty in reproducing air and high-density bone tissues. 

The MAEs within the body contour across all patients were 40.5 $\pm$ 5.4 HU and 37.6 $\pm$ 5.1 HU for the 2D and 3D models, respectively. The time required for generating a pelvic sCT using our CNN models was about 5.5 s. Our MAE results are comparable to previous studies. Kim $et \ al.$\cite{RN41} presented a voxel-based weighted summation method that produced an MAE of 74.3 $\pm$ 3.9 HU. However, manual contouring of bone tissues required for this method can be tedious and time-consuming. An MAE of 40.5 $\pm$ 8.2 HU was achieved by Dowling $et \ al.$\cite{RN11} using an average MRI-CT atlas from 38 patients. Andreasen $et \ al.$\cite{RN42} reported an MAE of 54 $\pm$ 8 HU using an atlas-based method with pattern recognition, and its prediction time was about 20.8 min. Another random forest model proposed by Andreasen $et \ al.$\cite{RN43} generated sCTs with an MAE of 58 $pm$ 9 HU. A hybrid method suggested by Siversson $et \ al.$ \cite{RN45} obtained an MAE of 36.5 $\pm$ 4.1 HU when ignoring errors introduced by gas cavities. This hybrid method was implemented in the cloud-based commercial software MriPlanner (Spectronic Medical AB, Helsingborg, Sweden), which required 50 to 80 min to generate a sCT.\cite{RN45} The patch-based 3D context-aware generative adversarial network presented by Nie $et \ al.$\cite{RN26} achieved an MAE of 39.0 $\pm$ 4.6 HU. 

Our CNN models reproduced low-density bone as shown in Figure ~\ref{fig4}. The bone-region DSCs were 0.81 $\pm$ 0.04 and 0.82 $\pm$ 0.04 from the 2D and 3D models, respectively. These results are comparable to reported DSC results of 0.79 $\pm$ 0.12\cite{RN10} and 0.91$\pm$0.03{\cite{RN11}}, where the authors compared bone contours manually drawn on the sCT and CT.

It was feasible to train the proposed 3D model with 16 image volumes from scratch. Results of the Wilcoxon signed-rank tests shown in Table~\ref{tab1} demonstrated a statistically significant improvement in overall MAE, bone DSC, and bone precision of the 3D model compared to the 2D model. However, as shown in Figure~\ref{fig4}, the 2D model seemed to perform better in estimating the high-density bone HU values. It should be noted that smaller overall MAEs do not guarantee improved sCT dose calculation and patient positioning performance. While the models performed well, we will continue to acquire more patient data to potentially improve model accuracy and further test model differences.

As this was a retrospective study, the MR image voxel sizes were not matched, resulting in different voxel intensities between images. This may have affected the sCT generation accuracy although we applied intensity normalization. A potential study could examine how voxel size variations affects sCT estimation. 

The proposed 3D model can be implemented on a 12 GB GPU to process volumetric images with dimensions of 256 $\times$ 256 $\times$ 30. More GPU memory would be required to process higher resolution 3D images. Considering the limited access to multi-GPU systems, a 3D architecture with fewer convolutional layers could be considered to deal with higher resolutions. However, the performance could be affected by the reduced parameters and smaller receptive fields of the less complex model. Another approach would be to extract 30-slice sub-volumes from CT and MR images for training the 3D model. The sCT could then be generated by averaging 30-slice sCT sub-volumes produced by the model. 

A number of techniques could be investigated for improving model performance.  Nie $et \ al.$\cite{RN26} showed that introducing an additional adversarial discriminator improved overall sCT quality. The same approach could be adapted in our proposed 2D and 3D CNN models.  Non-rigid deformation\cite{RN44} could also be applied to both CT and MR images in the process of the on-the-fly data augmentation to produce more training pairs. Multiple MR images acquired with different sequences could be fed into models to provide more information for distinguishing different tissues. Multi-GPU systems with more memory would enable the exploration of larger batch sizes for training CNN models, which could reduce variances in gradient estimation and accelerate the training. 




\begin{comment}
\begin{figure}
\includegraphics[width=\linewidth]{figs/beyond_tss_lesion.pdf}
\caption[]{End-to-End runtime lesion study of the entire MNIST dataset and the FMA featurized music dataset. Each of DROP's contributions provides a runtime improvement.}
\label{fig:beyond_lesion}
\end{figure}
\end{comment}



\section{Conclusion}
\label{sec:conclusion}

Advanced data analytics techniques must scale to rising data volumes. 
DR techniques offer a powerful toolkit when processing these datasets, with PCA frequently outperforming popular techniques in exchange for high computational cost. 
In response, we propose DROP, a new dimensionality reduction optimizer. 
DROP combines progressive sampling, progress estimation, and online aggregation to identify high quality low dimensional bases via PCA without processing the entire dataset by balancing the runtime of downstream tasks and achieved dimensionality. 
Thus, DROP provides a first step in bridging the gap between quality and efficiency in end-to-end DR for downstream \red{analytics}. 

%We revisit canonical operators for time series dimensionality reduction and the measurement study of~\cite{keogh-study}, and show that PCA is more effective than popular alternatives in the data mining literature often by a margin of over $2\times$ on average on gold-standard time series benchmark data sets with respect to output data dimension. More surprisingly, we empirically demonstrate that a small number of samples are sufficient to accurately characterize directions of maximum variance and obtain a high-quality low-dimensional transformation.




\section*{Acknowledgements}

All authors gratefully acknowledge financial support from
the Pilot Studies Program at the Tufts Clinical and Translational Science Institute
(Tufts CTSI NIH CTSA UL1TR002544).
Author BW is supported by K23AG055667 (NIH-NIA). 
Authors HZ and MCH thank the Office of the Vice Provost for Research at Tufts University for support for this project under a ``Tufts Springboard'' award.

%\bibliographystyle{abbrv}
\bibliographystyle{abbrv-doi}
%\bibliographystyle{abbrv-doi-narrow}
%\bibliographystyle{abbrv-doi-hyperref}
%\bibliographystyle{abbrv-doi-hyperref-narrow}

\bibliography{main}
\end{document}

