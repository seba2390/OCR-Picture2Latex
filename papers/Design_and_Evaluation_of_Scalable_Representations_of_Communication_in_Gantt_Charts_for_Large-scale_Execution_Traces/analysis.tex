\section{Results and Analysis}
\label{sec:analysis}
    We evaluate the hypotheses outlined in \autoref{sec:methodology} using an Analysis of Variance test (ANVOA) with a Tukey HSD Post-Hoc test to evaluate the impact of and pairwise significance of our independent variables. We choose a significance of $0.05$ and apply a Bonferroni correction for multiple testing, resulting in an effective significance of $0.01$. 
    
    To justify the validity of using ANOVA, we applied a Shapiro-Wilk test to evaluate the normality of each subset of data investigated in our independent variable. For each level, the Shapiro-Wilk p-value exceeded the target $0.05$, so we proceeded with the test. We further supplement the analysis with bootstrap confidence intervals (95\%, 1000 trials) which are consistent with the ANOVA results.
    
    In our analysis of pattern classification tasks, we examine three confusion matrices in an exploratory fashion for further insight into how users may have arrived at the incorrect answers. Although 8 questions were asked of participants, three were included for a potential exploratory analysis and were not sufficiently dense by our definition of $>1000$ rows and were not included in the following hypothesis testing.
            
    \begin{figure}
        \centering
        \includegraphics[width=\columnwidth]{figures/ci_one_fig.png}
        \caption{95\% Bootstrap confidence intervals. }
        \label{fig:all_ci}
    \end{figure}
    
    \vspace{1ex}
    
\textbf{H1 \& H5: {\em New} and {\em Partial} outperform {\em Full} in Overall Accuracy.} H1, that the {\em full} and {\em new} representations would have the same overall accuracy, and H5, that participants would be less accurate overall with {\em partial} representations, are not supported by the results. \autoref{fig:all_ci} (top) shows the bootstrap confidence intervals of accuracies aggregated across all tasks: pattern identification, grouping identification, and stride comparison. In the confidence intervals we see that there is no overlap between {\em new} and {\em full} representations as hypothesized. The overall effect of this factor on accuracy is statistically significant  (F$_{2,32}$ = 17.58, p $<$ .01), and the effect difference between {\em new} and {\em full} is determined to be statistically significant (p $<$ .01).

    
\begin{figure*}
    \centering
    \begin{subfigure}{0.28\textwidth}
        \centering
        \includegraphics[width=\textwidth]{figures/new_conf_mat_no_scale.png}
        \caption{Confusion matrix for {\em new} representations.}
        \label{fig:abstract_conf_mat}
    \end{subfigure}
    \begin{subfigure}{0.28\textwidth}
        \centering
        \includegraphics[width=\textwidth]{figures/full_conf_mat_no_scale.png}
        \caption{Confusion matrix for {\em full} representations. }
        \label{fig:full_conf_mat}
    \end{subfigure}
    \begin{subfigure}{0.37\textwidth}
        \centering
        \includegraphics[width=\textwidth]{figures/partial_conf_mat.png}
        \caption{Confusion matrix for {\em partial} representations.}
        \label{fig:partial_conf_mat}
    \end{subfigure}
    
    \caption{Confusion matrices showing how participants classified trial patterns in our study. To preserve space, the pattern classifications are abbreviated: O(ffset), R(ing), E(exchange), C(ontinuous), G(rouped). The matrix has an extra column because participants were able to answer that a pattern was ``Exchange Continuous" but no such patterns existed in our question set. }
    \label{fig:confusion_matrices}
\end{figure*}

The {\em new} representations performed better than {\em full} when all tasks were considered. In later analyses we will see this is driven almost entirely by the classification questions, despite poor performance with discrimination questions. Unexpectedly, the {\em partial} representations also performed better than {\em full} with no significant difference to the {\em new} representations. 


\vspace{1ex}

\textbf{H2: New Outperforms Full in Pattern Classification Accuracy.} H2, that classifications would be more accurate with the {\em new} designs for dense charts ($>1000$ PEs), is partially supported by our results. The effect of representation type on pattern classification accuracy was determined to be statistically significant (F$_{2,32}$ = 7.62, p $<$ .01). The Tukey HSD test indicated a significant difference between {\em new} and {\em full} representations (p $<$ .01), but not {\em new} and {\em partial}. \autoref{fig:all_ci} shows the bootstrap confidence intervals.


    
    Though the {\em new} representations performed better than the existing ones, absolute accuracy does not exceed an average of 59\% with a maximum of 64\% and a minimum of 53\%. To further understand this behavior, we examine a confusion matrix (\autoref{fig:abstract_conf_mat}). The majority of errors are Ring-Grouped (RG) patterns being misidentified as Exchange-Grouped (EG). These depictions are visually similar. As noted previously, an exchange is identical to a ring with a stride of half the participating PEs, so participants may have had trouble differentiating when the depicted stride was close to half, which is more likely in grouped scenarios with rings which have fewer lines in the glyph.
    
    \autoref{fig:full_conf_mat} shows the confusion matrix for the {\em full} trials. Most errors were due to user selecting Offset-Continuous (OC) when the pattern was something else. It is unclear if this was due to signifiers like spacing between groups and wrap-around lines being obscured by the density or users selecting the first option when they were unsure. A second feature to note is that Ring-Grouped (RG) and Exchanged-Grouped (EG) charts are commonly mis-classified as Exchange-Continuous (EC). As noted before, these patterns can be quite similar, both in structure and appearance.
    
    In \autoref{fig:partial_conf_mat}, mis-classifications are most densely clustered around Offset-Grouped (OG) as Offset-Continuous (OC) and Exchange-Grouped (EG) as Exchange-Continuous (EC). These errors are likely attributable to the fact that a ``break" signfiying a grouping in a pattern can occur past the top or bottom edge of a view. If the break cannot be seen, determining grouping is just a guess. Furthermore, we see that Ring-Grouped (RG) are mis-classified as EG and EC for the same reasons noted above: they are both visually and structurally very similar.

        \vspace{1ex}

\textbf{H3: {\em New} Outperforms {\em Full} and {\em Partial} in Grouping Accuracy.} H3, that participants would be more accurate with {\em new} representation than {\em full} or {\em partial} for identifying a pattern as grouped or continuous, is supported by the results. The effect of representation type on classification accuracy for ``grouping" questions is statistically significant (F$_{2,32}$ = 23.34, p $<$ .01) with the Tukey HSD tests indicating pairwise significant between {\em new}-{\em full} and {\em new}-{\em partial}. The {\em new} representation had a median accuracy of 73\%, while {\em full} and {\em partial} accuracies are within range of random guessing. \autoref{fig:all_ci} shows the corresponding bootstrap confidence intervals.

\vspace{1ex}

\textbf{H4: {\em Partial} Outperforms {\em New} and {\em Full} in Stride Estimation.} H4, that participants would be less accurate with the {\em new} representations when it came to stride estimation, is partially supported by our results. A significant difference is shown in the ANOVA (F$_{2,32}$ = 28.71, p $<$ .01), with the Tukey HSD showing a significant pairwise difference between {\em partial}-{\em new} and {\em partial}-{\em full}, but not {\em new}-{\em full}. \autoref{fig:all_ci} shows the bootstrap confidence intervals.
  
    As expected, {\em new} did not perform well on stride comparisons as the encoding is only approximate. Though the encoding is direct for {\em full} representation, the resulting visualization can be ambiguous when several strides map to the same pixel. {\em Partial} representations can contain the stride, if small enough. This may explain the low accuracy ($<65$\%) even in {\em partial} representations.
    
    \vspace{1ex}

\textbf{Post-trial Survey.} In the post-trial survey, 18 (51.4\%) of participants preferred the {\em new} representation, 14 (40.0\%) preferred the {\em partial} representation, and 3 (8.6\%) preferred the {\em full} representation. Only two commented on the designs, one stating that {\em full} was ``hard to discern'' and the other stating {\em full} looked ``the most practical''.


\subsection{Discussion}
\label{sec:analysisdiscussion}

The study results show that  {\em new} and  {\em partial} representations are strongest in overall accuracy across pattern type, grouping, and stride. Breaking this down, participants were more accurate with then  {\em new} representation for detecting grouping versus continuous patterns and the  {\em partial}  representation for detecting equivalent strides. While there was no statistically significant difference between  {\em new} and  {\em partial} in terms of pattern type, the median for  {\em new} was higher and most of the errors were attributable to patterns that are very similar to each other.

These results suggest the  {\em new} representation is helpful in what it was designed to do---provide a scalable overview of the communication pattern, emphasizing the type and grouping of a pattern first. The visualization de-prioritizes strides by design, a decision reinforced by the feedback from expert E2.

We were surprised how well the  {\em partial} representation performed overall, given the results of our pilot studies. This view occurs when zooming-in to the Gantt chart, suggesting that further emphasis be given in developing methods that take advantage of this view, such as lenses, or systems that support low latency rendering for interactive navigation of large scale traces.

One caveat to the performance of  {\em partial} representations is that our trials only included small stride values of ten or less, meaning many of them had relatively gentle angles and may have fit within the  {\em partial} view. Larger strides would have incurred steeper angles, something participants in our preliminary study struggled with in rings. 

Given the complementary strengths of the  {\em new} and  {\em partial} representations, designing an interactive Gantt chart to select between the two based on zoom-level may aid users in understanding communication.

Participant preference mostly tracked with participant performance, with the majority of participants preferring the {\em new} designs but a sizable minority preferring the {\em partial} views. This further suggests a complementary approach. Furthermore, the gap in preference to {\em full} suggests the {\em new} designs enhance satisfaction in addition to boosting accuracy in the core use case they were developed for.

Lastly, we note that the accuracy levels were not particularly high across the board, with the highest being recognizing grouped versus continuous patterns with the  {\em new} representation at 73\%. This low accuracy suggests that the present methods,  {\em full} and  {\em partial}, do not serve people well and while the  {\em new} representation improves the situation in some cases, there is still a gap in interpreting communication.
    
\subsection{Limitations and Threats to Validity}

The participants in this study were not HPC experts and did not have experience with large-scale parallel programs, the communication patterns they use, or large-scale trace visualization. The results with an expert population might differ. However, we did not think we could recruit such a population of sufficient size and even people who visualize traces regularly may not be familiar with idealized unit time as it is not yet supported by commercial trace visualization software.

The categorization task in this experiment involved three types of patterns and a binary choice between grouped and continuous. In a realistic setting, participants need to recognize patterns without constraints.

This experiment compared accuracy of static visualizations with relatively short interpretation periods to simulate browsing behavior. Results may differ in more focused analysis when a user engages with the system, panning and zooming data of interest. 

All representations were presented to participants in isolation. However, the ultimate use case is for them to be presented in the context of an interactive Gantt chart with potentially multiple patterns at the same time. This additional context could affect the strategies people take and their resulting accuracy.