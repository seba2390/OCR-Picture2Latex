\section{Preliminary Study}
\label{sec:prelim}

To aid the design of scalable representations of communication patterns and design of experiments for evaluating both new and existing representations, we executed a preliminary study to explore how people interpret communication in Gantt charts. This study focused on how communication is presently drawn and sought to understand what visual factors are considered when differentiating patterns. 

There are two ways people can view communication lines in Gantt charts. The first is to look at all PEs. We call this the {\em full} representation. The problem is when the chart has many PEs which communicate, i.e., it is {\em dense}, the lines may overlap to the point of a solid shape. The other way is to look at a subset of the PEs (rows). We call this the {\em partial} representation. The problem here is that key indicators about the pattern, such as grouping, may not appear in a given window.

Prior to the study, we identified several factors that might be used in interpreting these views: the density of a chart, the structure being shown, the grouping of a structure, the representation type ({\em  partial} or {\em full}), and the stride of a structure. A study across the full range of these factors would be unreasonably large, so we designed a qualitative study to limit our focus, following the ``factor mining'' evaluation pattern discussed by Elmqvist and Yi~\cite{elmqvist2015patterns}.

\begin{figure}
    \centering
    \includegraphics[width=\columnwidth]{figures/c1.png}
    \caption{Example image prompt from our preliminary study. These are both {\em partial} representations of a stencil pattern, emulating being zoomed-in on a Gantt chart.}
    \label{fig:interview_prompt}
\end{figure}

Our procedure was to interview participants while showing them paired images of communication patterns in idealized unit time Gantt charts. We decided on semi-structured interviews to allow probing and elaboration of ideas. The paired images varied in number of rows, representation type, and whether they were the same pattern. \autoref{fig:interview_prompt} shows an example prompt. For each prompt, we asked three questions:
\vspace{0em}
\begin{enumerate}
    \itemsep=0em
    \item Please describe the pattern of lines on the right.
    \item Please describe the pattern of lines on the left.
    \item Do you think that these two patterns are the same? Why or why not?
\end{enumerate}

We recruited seven participants, one from our university and six at the Supercomputing 2019 conference. Six reported computing experience and two reported HPC experience. Of those two, one had analyzed HPC performance. None had prior experience with trace visualization. 

The interviews were recorded, transcribed, and coded for common themes, resulting in over 170 unique codes. We describe the most frequently used. See the supplemental material for a full list of codes.

\texttt{Line angle} was used by all participants to describe patterns of lines and justify comparisons between them, with three referencing it over 10 times. \texttt{Line direction} was mentioned by all but one participant, but mostly to describe a pattern rather than compare. Although a line direction could be the result of an angle, it was coded distinct from \texttt{angle} since respondents would describe lines as going ``up," ``down," ``left," or ``right". Their intent was distinct from when they mentioned angle. A few participants mistook the multiple wrap-around lines for \texttt{one line} in ring patterns, even on repeated prompting.

Four participants made note of the \texttt{background}---the boxes and columns the lines connected. This code was also used both to describe an individual pattern and differentiate between pairs.

\texttt{Comparison} and \texttt{uncertainty} were co-occurring codes. Participants rarely felt confident comparing charts. We surmise this uncertainty comes from the fact that simple changes of height and representation alter visual factors they relied upon. Four participants discussed \texttt{transformation} during comparisons, describing how one depiction could be ``rotated'',  ``squashed/stretched'', or ``enlarged.''

Participants had the most difficulty with prompts depicting stencil patterns. They were uncertain of line extents and misidentified discrete lines as single lines or vice versa. Of the four interviewees exposed to stencil patterns, three found them viscerally off-putting, pausing with surprise when presented with them and one calling them ``a mess." 

\vspace{1ex}

\textbf{Discussion.} Participants generally relied on features such as line angle, line direction, and background markings in interpreting patterns. We expect all three factors to be harder to interpret with large numbers of PEs as many angles map to the same pixels and background markers are aggregated or removed.

The line factors are functions of the pattern stride and the height, and therefore density, of the chart, suggesting density significantly impacts recognition. This is further supported with participants' difficulty in recognizing line separation at severe angles, such as the wrap-around lines in rings.

The impetus of participants to identify patterns as similar under a stretch factor matches how the patterns are extensible to PE count. However, communication patterns are not extensible under rotation or inversion, as some discussed. We use this finding to inform tutorial material in subsequent study designs.

Participants reactions to stencils suggest that the idealized unit time depiction is too complex even at small scales. As we want to understand how people can interpret patterns as they scale up, we conclude that stencils are not appropriate for early work towards this goal as they are too difficult to interpret at small scale.

Based on the results of this preliminary study, we decided to take line angle and background into careful consideration for proposed designs. We ultimately fixed line angle at set values and obscure background features to encourage focus on pattern types over strides and to maintain discernability. We also decided to remove stencils from our pattern type factor and randomize across our other factors in further studies.