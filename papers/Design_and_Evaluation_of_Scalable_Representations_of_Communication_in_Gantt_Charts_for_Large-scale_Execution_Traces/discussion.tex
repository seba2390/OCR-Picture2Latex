\section{Reflections and Lessons Learned}
\label{sec:reflections}

We reflect on the concept of a pattern, communication recognition in Gantt charts, designing with differing participant pools, and designing in difficult-to-implement scenarios.

\vspace{1ex}

\textbf{`Pattern' is an ambiguous concept, but the domain can help ground and decompose the concept for design.} We use the term `communication pattern' because that is the term our domain expert collaborators used in previous projects. Our goal was to understand factors in interpreting these patterns and how to use them to create a scalable visual representation. However, the term {\em pattern} has many different interpretations. 

In our preliminary study, we observed that participants did not have a concrete understanding of the idea of a pattern or of the constraints the domain had on them. They discussed operations for transforming between instances, some of which could be valid in context, such as stretching and squishing, and some which could not, such as rotation. In effect, the preliminary study acted as an intervention~\cite{Bigelow2021}, correcting our assumptions on how we and others interpreted the notion of `pattern.' On reflection, we have considered alternate terms like `class,' `family,' `motif,' and `extensible structure,' but we suspect they would have the same ambiguity.

As communication patterns have an extensible, repetitive structure to them, we initially considered them similar to classes of graph structures in graph theory. However, communication patterns are not the same under rotational symmetry, something we realized was a potential factor due to the preliminary study. 

Understanding what factors participants were considering helped us in two key ways. First, it helped us to dissect the notion of `communication pattern' into three factors: type (base structure), grouping, and stride. We then assigned a priority to these factors, based on domain use cases, to drive the design. 

Second, the preliminary study revealed the visual strategies people preferred, such as line angle. In the context of the domain, this preference shifted focus on a less important factor, so we designed our solution to minimize its saliency.

\vspace{1ex}

\textbf{Present strategies for visualizing communication in Gantt charts are not sufficient.} {\em Full} representations, what is shown when Gantt charts show all rows, only narrowly outperformed random guessing (41\% vs. 33\%) in our study. Though our new designs improve accuracy versus full overviews, even in this constrained scenario, there is significant room for improvement. The performance of {\em partial} representations suggests improving access to them, for example, with lenses or smooth navigation, may also improve understanding of structures.

\vspace{1ex}


\textbf{Participants with computing experience served as a proxy for beginners in trace analysis.} Our desire to improve visual scalability in Gantt charts was motivated and informed by visualization outcomes of a long-term collaboration with domain experts. However, relying only on those experts, who are most familiar with the visualization, as participants in this next step could bias the results. Another practical issue is the difficulty of recruiting more domain experts due to demands on the time of people with this expertise. 

Following the strategy of McKenna et al.~\cite{McKenna2016}, we sought to use multiple pools of people in our design. In particular, we recruited people with some HPC familiarity in our preliminary study, people with high trace visualization familiarity for informal design feedback, and people with general computing experience for our controlled study, which required the most participants.

Our preliminary study did not suggest a noticeable effect of relative experience of people in HPC who did not have trace visualization experience. Thus, we recruited people with more general computing experience for the larger study, assuming they would be similar to an HPC worker who was new to analyzing traces visually or exposed to a new method of analyzing traces. Given that participant recruitment is a perennial problem, understanding differences in adjacent populations such as these, could aid practical aspects of study design. 

\vspace{1ex}

\textbf{There are scenarios encouraging speculative visual design.} The idealized unit time approach used in Ravel required complicated domain-aware algorithms to compute~\cite{Isaacs2015,Isaacs2016}. Suspecting identifying communication patterns might be similar, we chose to investigate the efficacy of the proposed visualization first, to determine if investing in the development domain-aware algorithms would be prudent. Our rationale was that even if a perfect algorithm could not be achieved, what we learn in the study could be applied to an imperfect one. 

The patterns we investigated could also be annotated in the source code. We learned a similar strategy was taken by E2 (\autoref{sec:expertfeedback}) for the Ravel-type visualization. E2 saw the visualization and was motivated to engineer a solution without the more intensive algorithm. This demonstrates the utility of speculative visualizations such as these. 

