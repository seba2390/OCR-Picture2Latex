Large-scale parallel computation is required to achieve meaningful results from scientific simulations that model domains like climate change, medicine, and energy. High performance computing (HPC) is concerned with making such computation possible. A key strategy is performance analysis and optimization. 

One approach in performance analysis is collecting and analyzing an execution trace---a record of events occurring during program execution. Such traces are commonly visualized with Gantt charts (\autoref{fig:simple_gantt}). Each processing element (PE), e.g., a hardware thread, is assigned a row. Function calls are drawn as rectangles over their time (x-axis) interval. 

PEs must communicate to exchange data and results. This communication is represented by overlaid lines in the Gantt chart, diagonally sweeping the PE space. Even in tiny parallel programs, i.e. those executed on 32 PEs, the depiction can become cluttered. Arranging events on a {\em logical time} axis, where events are shown based on logical relationships, rather than a physical time one, has been shown to aid in understanding the data~\cite{isaacs2014combing}. Events that ideally would have occurred at the same time are easier to compare and patterns in the communication structure are revealed.

However, logical time encodings retain the vertical scaling problem of Gantt charts. A modestly sized program executed on 1,024 PEs may induce so much occlusion that communication lines appear as a solid black shape. \autoref{fig:dense_gantt}, shows three examples with different communication structures that are difficult to identify and distinguish. These are all manifestations of code designed to cover the number of PEs allocated and thus often exhibit repetitive structure. We seek to understand how potential performance analysts interpret these structures in Gantt charts and what factors they use to differentiate them so scalable representations can be designed.

\begin{figure}[htb]
    \centering
    \includegraphics[width=\columnwidth]{figures/basic_gantt.png}
    \caption{A Gantt chart produced by Vampir from \cite{isaacs2014state}, showing a slice of time in a 16-PE execution. Black lines denote communication.}
    \label{fig:simple_gantt}
\end{figure}

We conduct a qualitative study at an HPC conference to investigate salient features in existing depictions and how people discern patterns, especially in cases where the source code is the same, but the instance differs due to PE count. Leveraging our findings, we decompose these patterns into domain-specific traits and design encodings for communication pattern representations. We then conduct a controlled user study of these designs versus the standard depictions. The results show the new designs enable users to more accurately identify key pattern traits and suggest further directions in Gantt chart research. Finally, we reflect upon the nature of the concept of patterns and designing them, recruiting participants in high-expertise domains, and speculative visualization design. 

In summary, our contributions are:

\vspace{-0.5ex}

\begin{itemize}
    \itemsep0em
    \item the design (\autoref{sec:design}) of scalable representations of communication patterns for Gantt charts, 
    
    \item evaluations (\autoref{sec:prelim}, \autoref{sec:analysis}) of efficacy of dependency depiction in Gantt charts, and
    
    \item reflections (\autoref{sec:reflections}) on visualizing patterns, participant recruitment, and speculative visualization design. 
    
\end{itemize}

We first present relevant background (\autoref{sec:background}) and related work (\autoref{sec:related}). We conclude in \autoref{sec:conclusion}.

% should we mention the reflections here?