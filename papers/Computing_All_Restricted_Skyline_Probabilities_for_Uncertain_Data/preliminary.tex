\section{Problem Definition}\label{sec:preliminary}

In this section, we first review the restricted skyline query, then formally define the all rskyline probabilities problem and investigate its conditional problem complexity.
For reference, the major notations used in this paper is summarized in Table~\ref{table:notaions}.

\begin{table}[t]
	\centering
	\caption{The summary of notations.}\vspace{-2ex}
	\label{table:notaions}
	\begin{tabularx}{\linewidth}{|c|X|}
		\hline
		{\bf Notation} & {\bf Definition} \\ \hline \hline
		$\calD$ & $d$-dimensional uncertain dataset \\ \hline
		$m$ & number of uncertain objects in $\calD$, \ie, $m = |\calD|$ \\ \hline
		$T_i$ & 1) the $i$th uncertain tuple in $\calD$ \newline 2) the set of instances $\{t_{i,1}, \cdots, t_{i, n_i}\}$ of $T_i$ \\ \hline
		$I$ & the set of all instances, \ie, $I = \bigcup_{i=1}^m T_i$ \\ \hline
		$n$ & number of instances in $I$, \ie, $n = |I|$ \\ \hline
		$t$ & instance in $I$ \\ \hline
		$\calF$ & set of monotone scoring functions \\ \hline
		$t \prec_\calF s$ & $t$ $\calF$-dominates $s$ \\ \hline
		${\rm Pr}_{\rm rsky}(t)$ & restricted skyline probability of $t$ \\ \hline
		$\omega$ & weight (preference) in the standard simplex $\simplex^{d-1}$ \\ \hline
		$S_\omega(t)$ & score of $t$ under $\omega$, \ie, $S_\omega(t) = \sum^d_{i = 1}\omega[i] \times t[i]$ \\ \hline
		$S_{_V}(t)$ & score vector of $t$ under $V = \{\omega_1, \cdots, \omega_{d'}\}$, \ie, $S_{_V}(t) = (S_{\omega_1}(t), \cdots, S_{\omega_{d'}}(t))$ \\ \hline
		$\Omega$ & preference region characterized by a set of linear constraints $A \times \omega \le b$, \ie, $\Omega = \{\omega \in \simplex^{d-1} \mid A \times \omega \le b\}$ \\ \hline
    \end{tabularx}
\end{table}

\subsection{Restricted Skyline}

Let $D$ denote a $d$-dimensional dataset $D$ consisting of $n$ tuples.
Each tuple $t \in D$ has $d$ numeric attributes, denoted as $t = (t[1], \cdots, t[d])$.
Without loss of generality, we assume that the numeric domain of each attribute is normalized into the unit interval $[0, 1]$ and the lower values are preferred than higher ones.
Given a \textit{scoring function} $f : [0, 1]^d \to \mathbb{R}^+$, the value $f(t[1], \cdots, t[d])$ is called the \textit{score} of tuple $t$ under $f$, also written as $f(t)$.
Function $f$ is called \textit{monotone} if for any tuple $t$ and $s$, it holds that $f(t) \le f(s)$ if $\forall 1 \le i \le d$, $t[i] \le s[i]$.
Let $\calF$ be a set of monotone scoring functions $\calF$, a tuple $t$ $\calF$-\textit{dominates} another tuple $s \ne t$, denoted as $t \prec_\calF s$, if $\forall f \in \calF$, $f(t) \le f(s)$.
The \textit{restricted skyline} (rskyline) of $D$ with respect to $\calF$ consists of all non-$\calF$-dominated tuples.

\subsection{Restricted Skyline Probability}

Following the uncertain data model used in previous related work~\cite{DBLP:conf/pods/AtallahQ09}, a $d$-dimensional uncertain dataset $\calD$ consists of $m$ objects $\{T_1, \cdots, T_m\}$.
Each object $T_i \in \calD$ is modeled by a set of instances $\{t_{i,1}, \cdots, t_{i, n_i}\}$ along with the probabilities $\{\Pr(t_{i, 1}), \cdots, \Pr(t_{i, n_i})\}$ for each instance to occur.
To cope with datasets of large scale, we assume $\calD$ is organized by a spatial index R-tree.
For the simplicity of notation, we also use $T_i$ to denote the set of its instances $\{t_{i, 1}, \cdots, t_{i, n_i}\}$ and write $t \in T_i$ to mean that $t$ is an instance of $T_i$.
We assume that the sum of probabilities over all instances of an object may add up to less than 1 and the existence probability may vary from one instance from another.
Moreover, each object can only take one instance at a time and objects are independent of each other.
Let $I = \bigcup_{i = 1}^m T_i$ denote the set of all instances and $n = |I| = \sum_{i = 1}^m n_i$ denote the number of instances.

Given an uncertain dataset $\calD$ and a set of monotone scoring functions $\calF$, an instance $t \in T_i$ belongs to the rskyline of $\calD$ if and only if $T_i$ occurs as $t$ and none of other objects appears as an instance that $\calF$-dominates $t$.
We refer to such probability as the \textit{rskyline probability} of $t$, denoted by $\Pr_{\rm rsky}(t)$.
With the assumption that each object can only take one instance at a time and objects are independent of each other, $\Pr_{\rm rsky}(t)$ can be computed as follows,
\begin{equation}\label{eq:rskyprob-def}
	\mathrm{Pr}_{\rm rsky}(t) = \Pr(t) \cdot \prod_{j = 1, j \ne i}^m(1 - \sum_{s \in T_j, s \prec_\calF t} \Pr(s)).
\end{equation}
The rskyline probability of an object $T_i$ is defined as the sum of rskyline probabilities of all its instances, \ie,
\begin{equation}
	\mathrm{Pr}_{\rm rsky}(T_i) = \sum_{t \in T_i} \mathrm{Pr}_{\rm rsky}(t).
\end{equation}

\begin{figure}[t]
	\includegraphics[width=\linewidth]{figures/model-illustration}\vspace{-2mm}
	\caption{An uncertain dataset and rskyline probabilities of instances.}
	\vspace{-2mm}
	\label{fig:model-illustration}
\end{figure}

\begin{example}
	Consider the uncertain dataset shown in Fig.~\ref{fig:model-illustration}.
	There are 4 objects in this uncertain dataset and their instances and existence probabilities are shown in the table.
	Given a set of scoring functions $\calF = \{\omega[1]t[1] + \omega[2]t[2] \mid \omega[1] \ge \omega[2]\}$, regions containing all instances $\calF$-dominating $b_3$ and being $\calF$-dominated by $b_3$ are shaded in gray and green, respectively.
	Thus, the rskyline probability of $b_3$ can be computed as $\Pr_{\rm rsky}(b_3) = \Pr(b_3) \times (1 - \Pr(c_1)) = 0.18$.
	Similarly, we can derive that $\Pr_{\rm rsky}(b_1) = 0.3$ and $\Pr_{\rm rsky}(b_2) = 0.018$.
	The rskyline probability of object $B$ is $\Pr_{\rm rsky}(B) = \Pr_{\rm rsky}(b_1) + \Pr_{\rm rsky}(b_2) + \Pr_{\rm rsky}(b_3) = 0.498$.
\end{example}

In this paper, we study the problem of computing rskyline probabilities of all instances, from which the rskyline probabilities of all objects can also be computed.

\noindent{\bf Problem: All RSkyline Probabilities (ARSP) Problem}\\
\noindent{\bf Input:} an uncertain dataset $\calD$ and a set of monotone scoring functions $\calF$. \\
\noindent{\bf Output:} rskyline probability for all instances in $I$, \ie,
\[{\rm ARSP} = \{(t, {\rm Pr}_{\rm rsky}(t)) \mid t \in I\}.\]


\subsection{Conditional Lower Bound}

In what follows, we prove that no algorithm can compute rskyline probabilities of all instances in truly subquadratic time without preprocessing unless the Orthogonal Vectors conjecture fails.

\noindent{$\blacktriangleright$ \bf Conjecture (OVC)~\cite{DBLP:conf/stacs/Bringmann19}.} Given two sets $A, B$, each of $n$ vectors in $\{0, 1\}^d$, for every $\delta > 0$, there is a $c \ge 1$ such that no $O(n^{2 - \delta})$-time algorithm can determine if there is a pair $(a, b) \in A\times B$ such that $a \times b = 0$ with $d = c\log{n}$.


\begin{theorem}\label{thm:lower-bound}
	Given an uncertain dataset $\calD$ and a set of monotone scoring functions $\calF$, no algorithm can compute restricted skyline probabilities for all instances within $O(n^{2-\delta})$ time for any $\delta > 0$, unless the Orthogonal Vectors conjecture fails.
\end{theorem}

\begin{proof}
	We establish a fine-grained reduction from the orthogonal vectors problem to all rskyline probabilities problem.
	Given two sets $A, B$, each of $n$ vectors in $\{0, 1\}^d$, we construct an uncertain dataset $\calD$ and a set $\calF$ of monotone scoring functions as follows.
	First, for each vector $b \in B$, we construct an uncertain tuple $T_b$ with a single instance $b$ and $\Pr(b) = 1$.
	Then, we construct an uncertain tuple $T_A$ with $n$ instances $\xi(a)$ and $\Pr(\xi(a)) = \frac{1}{n}$ for all vectors $a \in A$, where $\xi(a)[i] = \frac{3}{2}$ if $a[i] = 0$ and $\xi(a)[i] = \frac{1}{2}$ if $a[i] = 1$ for $1 \le i \le d$.
	Finally, let $\calF$ consists of $d$ linear scoring functions $f_i(t) = t[i]$ for $1 \le i \le d$, which means instance $t$ $\calF$-dominates another instance $s$ if and only if $t[i] \le s[i]$ for $1 \le i \le d$.
	We claim that for each instance $\xi(a) \in T_A$, there exists an instance $b$ from other uncertain tuple $T_b$ $\calF$-dominating $\xi(a)$ if and only if $a$ is orthogonal to $b$.
 
	Suppose there is a pair $(a, b) \in A \times B$ such that $a \times b = 0$, then $a[i] = 0$ or $b[i] = 0$ for $1 \le i \le d$.
	If $a[i] = 0$, then $b[i]$ can be either 0 or 1 and $\xi(a)[i] = \frac{3}{2} > b[i]$.
	Or if $b[i] = 0$, then $a[i]$ can be either 0 or 1 and $\xi(a)[i] \ge \frac{1}{2} > b[i]$.
	That is $b \prec_\calF \xi(a)$.
	On the other side, suppose there is a pair of instances $b$ and $\xi(a)$ such that $b \prec_\calF \xi(a)$.
	For each $1 \le i \le d$, $b[i]$ is either 0 or 1 and $\xi(a)[i]$ is either $\frac{3}{2}$ and $\frac{1}{2}$.
	If $b[i] = 0$, then $b[i]\cdot a[i] = 0$.
	Or if $b[i] = 1$, then $\xi(a)[i] = \frac{3}{2}$ since $b[i] \le \xi(a)[i]$.
	So $a[i] = 0$  according to the mapping $\xi(\cdot)$.
	Hence $a[i] \cdot b[i] = 0$.
	Thus we conclude that there is a pair $(a, b) \in A \times B$ such that $a \times b = 0$ if and only if there exists an instance $\xi(a) \in T_A$ with $\Pr_{\rm rsky}(\xi(a)) = 0$.
	Since $\calD$ can be constructed in $O(nd)$ time and whether such instance exists can be determined in $O(n)$ time, any $O(n^{2-\delta})$-time algorithm for all rskyline probabilities computation for some $\delta > 0$ would yield an algorithm for Orthogonal Vectors in $O(nd + n^{2 - \delta} + n) = O(n^{2 - \delta'})$ time for some $\delta' > 0$ when $d = \Theta(\log{n})$, which contradicts the OVC.
\end{proof}

