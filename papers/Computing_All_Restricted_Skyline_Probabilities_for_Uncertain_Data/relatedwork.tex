\section{Related Work}\label{sec:relatedwork}

In this section, we elaborate on two pieces of previous work that are most related to ours.

\noindent{\bf Queries on Uncertain Datasets.}
Pei \etal~\cite{DBLP:conf/vldb/PeiJLY07} are the first to extend skyline computation on certain datasets to probabilistic skyline computation on uncertain datasets.
They proposed two algorithms to identify those objects whose skyline probabilities are higher than a user-specified threshold $p$.
Considering the inherent limitations of threshold query, Atallah and Qi~\cite{DBLP:conf/pods/AtallahQ09} presented the first work that addresses the problem of computing exact skyline probabilities of all objects.
They proposed a $\tilde{O}(n^{2-1/(d+1)})$-time algorithm by using two basic skyline probability computation methods, weighted dominance counting method and grid method, to deal with frequent and infrequent objects, respectively. 
With a more efficient sweeping method for infrequent objects, Atallah \etal~\cite{DBLP:journals/tods/AtallahQY11} improved the time complexity to $\tilde{O}(n^{2-1/d})$.
However, the utilities of these two methods are limited to 2D datasets because of a hidden factor exponential in the dimensionality of the dataset caused by the high dimensional weighted dominance counting algorithm.
To get rid of this, Afshani \etal~\cite{DBLP:journals/mst/AfshaniAALP13} calculated the skyline probabilities of all instances by performing a pre-order traversal of a modified KD-tree.
With the well-know property of the KD-tree, it is easy to verify that the time complexity of their algorithm is $O(n^{2-1/d})$.
More practically, Kim \etal~\cite{DBLP:journals/tkde/KimIP12} introduced an in-memory Z-tree structure in all skyline probabilities computation to reduce the number of dominance tests, which has been experimentally demonstrated efficient.
However, it is non-trivial to revise algorithms for computing skyline probabilities of all objects to address the problem studied in this paper.
This is because these algorithms rely on the fact that the dominance region of an instance is a hyper-rectangle, which no longer holds under $\calF$-dominance.

Somehow related to what we study in this paper are those works on top-$k$ queries on uncertain datasets~\cite{DBLP:conf/icde/SolimanIC07, DBLP:journals/dpd/WangSY16, DBLP:conf/icde/HuaPZL08, DBLP:conf/icde/YiLKS08, DBLP:conf/sigmod/GeZM09}.
Under the possible world model, top-$k$ semantics are unclear, which give rise to different definitions, \eg, to compute the most likely top-$k$ set, the tuple with high probability to rank $i$-th, the tuples having a probability greater than a specified threshold to be included in top-$k$, \etc
Our work differs from theirs as an exactly input weight is required in these studies, whereas we focus on finding a set of non-$\calF$-dominated tuples where $\calF$ is a set of scoring functions.
In other word, our work can be regarded as extending theirs by relaxing the preference input into a region.

\noindent{\bf Operators with Restricted Preference.}
Given a set of monotone scoring functions $\calF$, Ciaccia and Martinenghi~\cite{DBLP:journals/pvldb/CiacciaM17} defined that a tuple $t$ $\calF$-dominates another tuple $s$ if $t$ scores better than $s$ for any $f \in \calF$.
Based on $\calF$-dominance, they introduced two restricted skyline operators, \textsc{ND} for retrieving the set of non-$\calF$-dominated tuples and \textsc{PO} for finding the set of tuples that are optimal according to at least one function in $\calF$.
And they designed several linear programming based methods for these two queries, respectively.
Mouratidis and Tang~\cite{DBLP:journals/pvldb/MouratidisT18} extended \textsc{PO} under top-$k$ semantic when $\calF$ is a convex preference polytope $\Omega$, \ie, they studied the problem of identifying all tuples that appear in the top-$k$ result for at least one $\omega \in \Omega$.
They first disqualified records $\calF$-dominated by $k$ or more others, and then determined the top-$k$-th in each partition of $\Omega$ among the remaining candidates.
Liu \etal~\cite{DBLP:conf/icde/Liu0ZP021} investigated a case of $\calF$-dominance where $\calF$ consists of $d-1$ constraints on the weight ratio of other dimensions to the user-specified reference dimension.
They defined {\it eclipse} query as retrieving the set of all non-eclipse-dominated tuples and proposed a series of algorithms.
These works only consider datasets without uncertainty, and we extend above dominance-based operators to uncertain datasets.
Their techniques can not be extended to our problem since the introduction of uncertainty makes the problem challenging as we need to identify all instances that $\calF$-dominate it for each instance.
