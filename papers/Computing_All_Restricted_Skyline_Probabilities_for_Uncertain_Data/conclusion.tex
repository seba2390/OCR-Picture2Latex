\section{Conclusions}\label{sec:conclusions}

In this paper, we study the problem of computing rskyline probabilities of all instances from both complexity and algorithm perspective.
By establishing a fine-grained reduction from the OVC, we prove that the problem can not be solved in truly subquadratic time unless the OVC fails.
As for algorithmic concern, when $\calF$ is a set of linear scoring functions described by linear constraints, we propose two algorithms with near-optimal time complexity $O(n^{2 - 1/d'})$ and better expected time-complexity $O(mn\log{n})$, respectively.
For a special case where the linear constraints consists of $d - 1$ ratio bounds of the form $\{l_i \le \omega[i]/\omega[d] \le h_i \mid 1 \le i < d \}$, we propose an algorithm with $O(2^{d-1}\log{n} + n)$ query time and polynomial preprocessing time for ARSP problem, and an algorithm with $O(2^{d-1}\log{n})$ query time and polynomial preprocessing time for online rskyline probability query.
Our thorough experiments over real and synthetic datasets demonstrate the effectiveness of the problem and the efficiency of the proposed algorithms.
Moreover, the extension of the sublinear-time algorithm also outperforms the state-of-the-art index-based method for the corresponding query on certain datasets.
For future directions, there are two possible ways.
On the one hand, conducting rskyline analysis on datasets with continuous uncertainty remains open, where it becomes expensive to make the integral for computing the dominance probability.
On the other hand, it is still worthwhile to investigate concrete lower bounds for ARSP problem under some specific dimensions.