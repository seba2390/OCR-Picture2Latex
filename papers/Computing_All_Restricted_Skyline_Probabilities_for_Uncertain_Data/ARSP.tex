\documentclass[conference]{IEEEtran}
\IEEEoverridecommandlockouts
% The preceding line is only needed to identify funding in the first footnote. If that is unneeded, please comment it out.
\usepackage{mathtools}
\usepackage{array}
\usepackage{xcolor}
\usepackage{framed}
\usepackage{bm}
\usepackage[mathscr]{euscript}
\usepackage{multirow}
\usepackage{xspace}
\usepackage{booktabs}
\usepackage{times,epsfig,epsf,psfrag,array,amsmath,amssymb,graphicx,graphics,verbatim}
\usepackage[ruled,vlined,linesnumbered, commentsnumbered]{algorithm2e}
\usepackage{url}
\usepackage{cite}
\usepackage{comment}
\usepackage{enumerate}
\usepackage{float}
\usepackage{mathtools}
\usepackage{amsthm}
\usepackage{tabularx}
\usepackage{subfigure}

\def\BibTeX{{\rm B\kern-.05em{\sc i\kern-.025em b}\kern-.08em
    T\kern-.1667em\lower.7ex\hbox{E}\kern-.125emX}}
\begin{document}

\newcommand{\eat}[1]{}
\newcommand{\eg}{{\textrm{e.g.}}\xspace}
\newcommand{\etc}{{\textrm{etc.}}\xspace}
\newcommand{\ie}{{\textrm{i.e.}}\xspace}
\newcommand{\etal}{{\textrm{et al.}}\xspace}
\newcommand{\resp}{{\textrm{resp.}}\xspace}
\newcommand{\calD}{\mathcal{D}}
\newcommand{\calF}{\mathcal{F}}
\newcommand{\simplex}{\mathbb{S}}
\newcommand{\real}{\mathbb{R}\xspace}
\newcommand{\SKY}{{\rm SKY}}
\newcommand{\RSKY}{{\rm RSKY}}

\newtheorem{definition}{Definition}
\newtheorem{theorem}{Theorem}
\newtheorem{lemma}{Lemma}
\newtheorem{corollary}{Corollary}
\newtheorem{proposition}{Proposition}
\newtheorem{claim}{Claim}
\newtheorem{observation}{Observation}
\newtheorem{example}{Example}
\newtheorem{problem}{Problem}

\SetKwFunction{skyprob}{\textsc{$kd$-ASP$^*$}}

\makeatletter
\newcommand{\removelatexerror}{\let\@latex@error\@gobble}
\makeatother

\title{Computing All Restricted Skyline Probabilities for Uncertain Data\\
\thanks{This work is supported by the National Natural Science Foundation of China (NSFC) Grant NOs. 61732003, 61832003, 61972110, U1811461, and U19A2059, and the National Key R\&D Program of China Grant NO.2019yfb2101900.}
}


\author{\IEEEauthorblockN{Xiangyu Gao}
\IEEEauthorblockA{
\textit{Harbin Institute of Technology}\\
Harbin, China \\
gaoxy@hit.edu.cn}
\and
\IEEEauthorblockN{Jianzhong Li} 
\IEEEauthorblockA{
\textit{Shenzhen Institute of Advanced Technology} \\
\textit{Chinese Academy of Sciences}\\
Shenzhen, China \\
lijzh@siat.ac.cn}
\and
\IEEEauthorblockN{Dongjing Miao}
\IEEEauthorblockA{
\textit{Harbin Institute of Technology}\\
Harbin, China \\
miaodongjing@hit.edu.cn}
}

\maketitle

\begin{abstract}
    Since data uncertainty is inherent in multi-criteria decision making, recent years have witnessed a dramatically increasing amount of attention devoted to conducting advanced analysis on uncertain data.
    In this paper, we revisit restricted skyline query on uncertain datasets from both complexity and algorithm perspective.
    Instead of conducting probabilistic restricted skyline analysis under threshold or top-$k$ semantics, we focus on a more general problem that aims to compute the restricted skyline probability of all objects.
    We prove that the problem can not be solved in truly subquadratic-time unless the Orthogonal Vectors conjecture fails, and propose two algorithms, one with near-optimal time complexity and the other with better expected time complexity.
	We also propose an algorithm with sublinear query time and polynomial preprocessing time for the case where the preference region is described by $d - 1$ ratio bound constraints.
	Our thorough experiments over real and synthetic datasets demonstrate the effectiveness of the problem and the efficiency of the proposed algorithms.
\end{abstract}

\begin{IEEEkeywords}
Uncertain data, probabilistic restricted skyline
\end{IEEEkeywords}

%\section{Introduction}  \label{sec:introduction}

\newcommand\inexpIntro[3]{#1?(#2,#3).}
\newcommand\rinexpIntro[3]{*#1?(#2,#3).}
\newcommand\outexpIntro[3]{#1!(#2,#3).}
\newcommand\outatomIntro[3]{#1!(#2,#3)}

We propose a fully automated method for proving termination of \(\pi\)-calculus processes.
Although there have been a lot of studies on termination analysis for the \(\pi\)-calculus
and related calculi~\cite{Deng06IC,Demangeon07,SangiorgiTermination,KobayashiHybrid,Yoshida04IC,DBLP:journals/jlp/DemangeonHS10,Venet98SAS}, most of them have been rather theoretical,
and there have been surprisingly little efforts in developing  fully automated termination
verification methods and tools based on them. To our knowledge,
Kobayashi's \typical{}~\cite{TyPiCal,KobayashiHybrid} is the only exception that
can prove termination of \(\pi\)-calculus processes (extended with natural numbers)
fully automatically, but its termination analysis is quite limited (see Section~\ref{sec:relatedwork}).

Our method is based on a reduction to termination analysis for sequential programs:
we translate a \(\pi\)-calculus process \(P\) to a sequential program \(S_P\), so that
if \(S_P\) is terminating, so is \(P\). The reduction allows us to use
powerful, mature methods and tools
for termination analysis of sequential programs~\cite{heizmann2016ultimate,freqterm,DBLP:conf/lics/PodelskiR04,Kuwahara2014Termination,DBLP:journals/cacm/CookPR11}.

The idea of the translation is to convert a chain of communications on replicated input
channels to a chain of recursive function calls of the target sequential program.
Let us consider the following Fibonacci process:
\begin{align*}
    & \rinexpIntro{\fib}{n}{r}
        \ifexp{n<2}{ \soutatom{r}{1} \\ &\quad}
                   { \nuexp{s_1} \nuexp{s_2} (\outatomIntro{\fib}{n-1}{s_1} \PAR \outatomIntro{\fib}{n-2}{s_2} \PAR \sinexp{s_1}{x}\sinexp{s_2}{y}\soutatom{r}{x+y}) \\}
    & \PAR \outatomIntro{\fib}{m}{r}
\end{align*}
Here, the process
$\rinexpIntro{\fib}{n}{r} \ldots$ is a function server that computes the \(n\)-th Fibonacci number
in parallel and returns the result to \(r\),
and $\outatom{\fib}{m}{r}$ sends a request for computing the \(m\)-th Fibonacci number;
those who are not familiar with the syntax of the \(\pi\)-calculus may wish to consult
Section~\ref{sec:targetlanguage} first.
To prove that the process above is terminating for any integer \(m\),
it suffices to show that there is no infinite chain of communications on $\fib$:
\[
    \fib(m,r) \to \fib(m_1,r_1) \to \fib(m_2,r_2) \to \cdots.
\]
We convert the process above to the following program:\footnote{The actual translation
  given later is a little more complex.}
\begin{verbatim}
 let rec fib(n) = if n<2 then () else (fib(n-1) [] fib(n-2)) in
 fib(m)
\end{verbatim}
Here, \texttt{[]} represents the non-deterministic choice.
Note that, although the calculation of Fibonacci numbers is not preserved,
for each chain of communications on \texttt{fib}, there is a corresponding
sequence of recursive calls:
\[
\mathtt{fib}(m) \to \mathtt{fib}(m_1) \to \mathtt{fib}(m_2) \to \cdots.
\]
Thus, the termination of the sequential program above implies the termination of
the original process.
As shown in the example above, (i) each communication on a replicated input channel
is converted to a function call, (ii) each communication on a non-replicated input
channel is just removed (or, in the actual translation, replaced by a call of
a trivial function defined by \(f(\seq{x})=(\,)\)), and (iii) parallel composition
is replaced by a non-deterministic choice.
We formalize the translation outlined above and prove its correctness.

The basic translation sketched above sometimes loses too much information.
For example, consider the following process:
\begin{align*}
    & \rinexpIntro{\pre}{n}{r} \soutatom{r}{n-1} \\
    & \PAR \rinexpIntro{f}{n}{r} \ifexp{n<0}{ \soutatom{r}{1} }
                                       { \nuexp{s} (\outatomIntro{\pre}{n}{s} \PAR \sinexp{s}{x}\outatomIntro{f}{x}{r}) } \\
    & \PAR \outatomIntro{f}{m}{r}
\end{align*}
The translation sketched above would yield:
\begin{verbatim}
  let pred(n) = n-1 in
  let rec f(n) = if n<0 then () else (pred(n) [] f(*)) in
  f(m)
\end{verbatim}
Here, \texttt{*} represents a non-deterministic integer: since we have removed
the input $\sinatom{s}{x}$, we do not have information about the value of \( x \).
As a result, the sequential program above is non-terminating, although the original
process is terminating.
To remedy this problem, we also refine the basic translation above by using a refinement
type system for the \(\pi\)-calculus. Using the refinement type system,
we can infer that the value of \(x\) in the original process is less than \(n\),
so that we can refine the definition of \texttt{f} to:
\begin{verbatim}
 let rec f(n) = ... else (pred(n) [] let x=* in assume(x<n);f(x))
\end{verbatim}
The target program is now terminating, from which
we can deduce that the original process is also terminating.
We have implemented an automated tool based on the refined translation above.

The contributions of this paper are summarized as follows.
\begin{itemize}
\item The formalization of the basic translation from the \(\pi\)-calculus
  (extended with integers) to sequential programs, and a proof of its correctness.
\item The formalization of a refined translation based on a refinement type system.
\item An implementation of the refined translation, including automated refinement type
  inference based on CHC solving, and experiments to evaluate the effectiveness of
  our method.
\end{itemize}

The rest of this paper is structured as follows.
Section~\ref{sec:targetlanguage} introduces the source and target languages
of our translation.
Section~\ref{sec:approach} 
formalizes the basic translation, and proves its correctness.
Section~\ref{sec:refinement} refines the basic translation by using a refinement type system.
Section~\ref{sec:implementation} reports an implementation and experiments.
Section~\ref{sec:relatedwork} discusses related work,
and Section~\ref{sec:conclusion} concludes the paper.

%\newpage
\section{Introduction}
Along with global-scale communication, cellular networks facilitate
a wide range of critical applications and services including
earthquake and tsunami warning system (ETWS), telemedicine, and smart-grid electricity distribution.
Unfortunately, cellular networks, including the most recent generation, have been often plagued with debilitating
attacks due to design weaknesses~\cite{lteinspector, TORPEDO, 5g_reasoner, 5Gformal_authentication_basin}
and deployment slip-ups \cite{privacy_ndss16, kim_ltefuzz_sp19, lte_redirection,how_not_to_break_crypto}.
Implications of these attacks range from intercepting and eavesdropping messages, tracking users' locations,
and disrupting cellular services, which in turn may severely affect the security and privacy of both
individual users and primary operations of a nation's critical infrastructures. To make matters worse,
vulnerabilities discovered in this ecosystem take a long time to generate and distribute patches as they not only
require collaboration between different stakeholders (e.g., standards body, network operator,
baseband processor manufacturer) but also
incur high operational costs. To make matters worse, different patches could potentially lead to
unforeseen errors if their integration is not accounted for.

In addition to it, although a majority of the existing work focus on discovering new attacks through analysis of the \emph{control-plane} protocol
specification or deployment \cite{lteinspector, TORPEDO, privacy_ndss16, kim_ltefuzz_sp19, 5Gformal_authentication_basin,
5g_reasoner, lte_redirection, how_not_to_break_crypto}, only a handful of efforts have focused on proposing defense mechanisms or
any apparatus to detect attack occurrences~\cite{imsi_catcher_catchers, FBSRADAR, mobile_self_defense, FBSleuth, wisec_root}.
Unfortunately, these proposed mechanisms are far from being widely adopted since they suffer from one of the following
limitations: \textbf{(i)} Requires modifications to an already deployed cellular network protocol \cite{wisec_root} which require
network operator cooperation;
\textbf{(ii)} Focuses on identifying particular attacks and hence are not easily extensible \cite{imsi_catcher_catchers, FBSRADAR,
mobile_self_defense, FBSleuth}; and
\textbf{(iii)} Fails to handle realistic scenarios (e.g., roaming) \cite{wisec_root}.

A pragmatic approach for protecting users and their devices from such a wide-variety
of vulnerabilities and dubious practices of the operators
(referred to as \textbf{\emph{undesired behavior}}\footnote{
	In our context, \emph{not} all undesired behavior are necessarily exploitable
	attacks. We also call some not-necessarily-malicious behavior (e.g., the use of null encryption by real network operators)
	undesired behavior if they can be detrimental to a user's privacy and security. In our exposition, we use
	attack, vulnerability, and undesired behavior, interchangeably.
	}
at the abstract in this paper) is to deploy a device-centric defense.
Such a defense, similar to an intrusion prevention system in principle, will monitor the network traffic at
runtime to identify undesired behavior and then take different corrective actions to possibly thwart it
(e.g., dropping a packet). In this paper, we focus on the core problem of
developing a general, lightweight, and extendable mechanism \system that can empower cellular devices
to detect various undesired behavior. %(e.g., vulnerabilities, unsafe practices employed by the stakeholders).
To limit the scope of the paper, we focus on monitoring the control-plane traffic for undesired behavior,
although \system is generalizable to data-plane traffic. Monitoring control-plane traffic is vital as
flaws in control-plane procedures, such as registration and mutual authentication, are
entry points for most attacks in both control- and data-plane procedures.

\system{}'s undesired behavior detection approach can induce different instantiations depending on the corrective
actions that are available to it. When deployed inside a baseband processor, \system can be used as a full-fledged
device-centric defense, akin to the pragmatic approach discussed above, that intercepts each message before getting
processed by the message handler and take corrective actions (e.g., drop the message, terminate the session) when it
identifies the message as part of an attack sequence. Alternatively, if \system is deployed as a mobile application that
can obtain a copy of the protocol message from the baseband processor, then one can envision building a warning system, which
notifies device owners when it detects that a protocol packet is part of an undesired behavior. Finally, \system can be
deployed and distributed as part of cellular network probes or honeypots that log protocol sessions with undesired behavior.
% that demonstrate an undesired behavior.




\paragraph{Approach.} In this paper, we follow a \emph{behavioral signature-based}
attack (or, generally undesired behavior) detection approach. It is enabled by the observation that a
substantial number of cellular network undesired behavior, which is detectable from the device's point-of-view,
often can be viewed as protocol state-machine bugs. Signatures of such undesired behavior can be constructed by
considering the relative temporal ordering of events (e.g., receiving an unprotected message
after mutual authentication).

Based on this above insight, we design a lightweight, generic, and in-device runtime
undesired behavior detection system
dubbed \system for cellular devices. In its core, \system{}'s detection has
two main components: (1) a pre-populated signature database for undesired behavior;
(2) a monitoring component that efficiently \emph{monitors} the device's cellular network
traffic for those behavioral signatures and takes corresponding corrective measures based
on its deployment (e.g., drop a message, log a message, warn the user).
%
%
Such a detection system is highly efficient and deployable as it
neither induces any extra communication overhead nor calls for any
changes in the cellular protocol. \system works with only a local view of the network,
yet is effective without provider-side support in identifying a wide array of undesired behavioral signatures.

For capturing behavioral signatures, we consider the following three different signature representations
that induce different tradeoffs in terms of space and runtime overhead, explainability, and detection accuracy:
(1) Deterministic Finite Automata (DFA);
(2) Mealy machine (MM) \cite{mealy1955method};
(3) propositional, past linear temporal (\pltl) \cite{ltl} formulas.
Cellular network security experts can add behavioral signatures in these representations
to \system{}'s  database. In case an expert is not
familiar with one of the above signature representations, they can get help/confirmation
from an \textbf{\emph{optional}} automatic signature synthesis component we propose.
We show that for all the above representations the automatic signature synthesis problem
can be viewed as an instance of the \emph{language learning from the informant} problem.
For DFA and MM representations, we rely on existing automata learning algorithms, whereas
for PLTL, we propose a new algorithm, an extension of
prior work \cite{learning_ltl}. For runtime monitoring of these signature representations
in \system, we use standard algorithms \cite{havelund2004}.
%\cite{rosu2001synthesizing}.



We consider two different instantiations for \system.
First, we implemented \system as an Android application and instantiated with the following monitors:
DFA-based, %$\system_\mathrm{DFA}$ ,
MM-based, %$\system_\mathrm{Mealy}$,
and PLTL-based. %$\system_\mathrm{PLTL}$.
In \system app, for capturing in-device cellular traffic,
%which requires root privilege in the device,
we enhanced the MobileInsight Android \cite{mobile_insight} application
to efficiently parse messages and invoke the relevant monitors.
Second, we implemented \system inside srsUE, distributed as part of
the open-source protocol stack
srsLTE \cite{gomez2016srslte}, powered by
the PLTL-based monitor---the most efficient in our evaluation,
to mimic \system{}'s deployment inside the baseband processor.

% allow \system to have baseband implementation corrective measures.

% \says{Omar}{Fix the following two paragraphs based on Mitziu's feedback.}
We evaluated \system{}'s Android app instantiation based on both testbed generated
and real-world network traffic in 3 COTS devices. In our evaluation with 15
existing cellular network attacks for 4G LTE,
we observed that in general all of the approaches were able to identify the existing
attacks with a high degree of success. Among the different monitors, however,
DFA on average produced a higher number of false positives
(21.5\%) and false negatives (17.1\%) whereas MM and PLTL
turn out to be more reliable; producing a significantly less number of false positives
($\sim\!\!$ 0.03\%) and false negatives ($\sim\!\!$ 0.01\%).
In addition, we observed that all monitors can handle a high number of
control-plane packets (i.e., 3.5K-369K packets/second).
We measured the power consumption induced by different monitors
and observed that on average, they all consume a moderate amount of energy ($\sim\!\!\!$ 2-6 mW).
Interestingly, we discover that \system, when powered by the PLTL-based monitor,
	produces no false warnings on real networks and in fact, it helped us discover
	unsafe network operator practices in three major U.S. cellular network providers.
% Note that, when evaluating \system in the wild with 3 major US network operators, we identified misconfigured
% base stations belonging to two operators whose misbehavior triggered an undesired behavior.
Finally, we evaluated \system instantiation as part of  srsUE \cite{gomez2016srslte} with testbed
generated traffic and observed that it only incurs a small memory overhead (i.e., 159.25 KB).


\paragraph*{Contributions.}
In summary, the paper makes the following  contributions:
\begin{itemize}\setlength{\itemsep}{0em}
	\item %We investigate the plausibility of having a general in-device
	%monitor for detecting cellular network control-plane attacks.
	We design an in-device, behavioral-signature based
	cellular network control-plane undesired behavior detection system called \system.
	We explore the design space of developing such a vulnerability detection system and consider
	different trade-offs.
	\item We implement \system as an Android app, which during our evaluation with
	3 COTS cellular devices in our testbed has been found to be effective
	in identifying 15 existing 4G LTE attacks while incurring a small overhead.

	\item We implement \system by extending srsUE \cite{gomez2016srslte}---mimicking  a full-fledged
	defense, and show its effectiveness at preventing
	attacks. %when capable of dropping messages.
	\item We finally show how one could automatically synthesize behavioral signatures \system expects by
	posing it as a learning from an informant problem \cite{informant_learning} and solve it with different
	techniques from automata learning and syntax-guided synthesis.
	% In particular, we present a
	% novel algorithm for synthesizing a \pltl vulnerability signature from
	% benign and vulnerable traces.
\end{itemize}

\subsection{Multitask Learning}

MTL has been succesfully used in different domains, including CV \cite{UberNet,MaskRCNN}. Some challenges appear when applying it \cite{Caruana}: \textit{learning speed} differences between tasks and  deciding \textit{what to share} according to the \textit{relatedness} between tasks in the multitask architecture \cite{Stitch, AdaptiveFeatureSharing}.

\subsection{Semantic Segmentation}

Semantic segmentation aims at partitioning parts of images belonging to the same semantic class, typically via pixel-wise classification. Fully convolutional networks (FCN) \cite{FCN} have improved both accuracy and speed for dense prediction problems by using only convolutional layers. Upsampling layers allow a segmentation output size equal to the input and skip connections add finer details. Other approaches add post-processing steps \cite{DeeplabCRF}, learnable \textit{deconvolution} layers \cite{ Deconv} or global context \cite{ParseNet}.

\subsection{Object Detection}

Object detection aims at finding in an image all instances of objects and classifying them in a number of classes. Faster R-CNN \cite{FasterRCNN} was the first to give close to real-time performance. YOLO \cite{YOLO} avoids the generation of region proposals for increased speed. SSD \cite{SSD} avoids fully-connected layers for speed and takes features at different levels for improved accuracy. 

%\cite{SpeedAccuracy} reviews the speed vs. accuracy trade-off for different object detectors.
\section{Preliminaries}

\subsection{Notation}

Let $\mX \subset \R^{I_1 \times \cdots \times I_K}$ be the space of
order-$K$ tensors, where $I_k$ denotes the dimensionality of the $k$-th
mode for $k=1,\dots,K$.  For brevity, we define
$I_{<k} := \prod_{k'<k}I_{k'}$; similarly, $I_{\leq k}, I_{k<}$ and
$I_{k \leq}$ are defined.  For a vector $Y \in \R^d$, $[Y]_i$ denotes
the $i$-th element of $Y$.  Similarly, $[X]_{i_1,\ldots,i_K}$ denotes
the $(i_1,\ldots,i_K)$ elements of a tensor $X\in\mX$. Let
$[X]_{i_1,\ldots,i_{k-1},:,i_{k+1},\ldots,i_K}$ denote an
$I_k$-dimensional vector
$(X_{i_1,\ldots,i_{k-1},j,i_{k+1},\ldots,i_K})_{j=1}^{I_k}$ called the
mode-$k$ fiber.  For a vector $Y \in \R^d$, $\|Y\| = (Y^T Y)^{1/2}$
denotes the $\ell_2$-norm and $\|Y\|_{\infty} = \max_i|[Y]_i|$ denotes
the max norm.  For tensors $X,X' \in \mX$, an inner product is defined
as
$\langle X,X' \rangle := \sum_{i_1,\ldots,i_K =1}^{I_1 \dots I_K}
X(i_1,\ldots,i_K)X'(i_1,\ldots,i_K)$
and $\|X\|_{F} = \langle X,X \rangle^{1/2}$ denotes the Frobenius
norm.  For a matrix $Z$, $\|Z\|_s := \sum_{j} \sigma_{j}(Z)$ denotes
the Schatten-1 norm, where $\sigma_j(\cdot)$ is a $j$-th singular value
of $Z$.

\subsection{Tensor Train Decomposition}

%\textit{Tensor train (TT) decomposition} is a tensor factorization
%method with a matrix product representation
%\cite{oseledets2010tt,oseledets2011tensor}.  
Let us define a tuple of positive integers $(R_1, \ldots, R_{K-1})$
and an order-$3$ tensor $G_k \in \R^{I_k \times R_{k-1} \times R_k}$
for each $k = 1,\ldots,K$.  Here, we set $R_0 = R_K = 1$.  Then, TT
decomposition represents each element of $X$ as follows:
\begin{align}
	X_{i_1,\ldots,i_K} = [G_1]_{i_1,:,:} [G_2]_{i_2,:,:} \cdots [G_K]_{i_K,:,:}. \label{eq:tt}
\end{align}
Note that $[G_k]_{i_k,:,:}$ is an $R_{k-1} \times R_k$ matrix.  We
define $\mG := \{G_k\}_{k=1}^K$ as a set of the tensors, and let $X(\mG)$
be a tensor whose elements are represented by $\mG$ as
\eqref{eq:tt}.  The tuple $(R_1, \ldots, R_{K-1})$ controls
the complexity of TT decomposition, and it is called a \textit{Tensor
  Train (TT) rank}.  Note that TT decomposition is universal, i.e.,
any tensor can be represented by TT decomposition with sufficiently
large TT rank~\cite{oseledets2010tt}.


When we evaluate the computational complexity, we assume the shape of
$\mG$ is roughly symmetric. That is, we assume there exist
$I,R\in\mathbb{N}$ such that $I_k=O(I)$ for $k=1,\dots,K$ and
$R_k=O(R)$ for $k=1,\dots,K-1$.


\subsection{Tensor Completion Problem}

Suppose there exists a true tensor $X^* \in \mX$ that is unknown, and
a part of the elements of $X^*$ is observed with some noise.  Let
$S \subset \{(j_1,j_2,
\ldots,j_K)\}_{j_1,\ldots,j_K=1}^{I_1,\ldots,I_K}$
be a set of indexes of the observed elements and
$n := |S| \leq \prod_{k=1}^K I_k$ be the number of observations.  Let
$j(i)$ be an $i$-th element of $S$ for $i=1,\ldots,n$, and $y_i$
denote $i$-th observation from $X^*$ with noise.  We consider the
following observation model:
\begin{align}
	y_i = [X^*]_{j(i)} + \epsilon_i, \label{model:obs}
\end{align}
where $\epsilon_i$ is i.i.d. noise with zero mean and variance
$\sigma^2$.  For simplicity, we introduce  observation vector
$Y := (y_1, \ldots, y_n)$, noise vector
$\mE := (\epsilon_1, \ldots , \epsilon_n)$, and rearranging operator
$\mathfrak{X} : \mX \to \mathbb{R}^n$ that randomly picks the elements of $X$.
%  $[\mathfrak{X}(X)]_i = [X]_{j(i)}$.
Then, the model \eqref{model:obs} is rewritten as follows:
\begin{align*}
	Y = \mathfrak{X}(X^*) + \mE.
\end{align*}

%%%
The goal of tensor completion is to estimate the true tensor $X^*$
from the observation vector $Y$.  Because the estimation problem is
ill-posed, we need to restrict the degree of freedom of $X^*$, such as
rank. Because the direct optimization of rank is difficult, its convex
surrogation is alternatively
used~\cite{candes2012exact,candes2010matrix, krishnamurthy2013low,
  zhang2016exact, phien2016efficient}.  For tensor
completion, the convex surrogation yields the following optimization
problem
\cite{gandy2011tensor,liu2013tensor,signoretto2011tensor,tomioka2010estimation}:
\begin{align}
	\min_{X \in \Theta} \left[ \frac{1}{2n} \|Y - \mathfrak{X}(X)\|^2 + \lambda_n \|X\|_{s^*} \right], \label{opt:general}
\end{align}
where $\Theta \subset \mX$ is a convex subset of $\mX$, 
%and
%$\Omega : \Theta \to \R_+$ is a regularization for tensors, 
$\lambda_n\geq 0$ is a regularization coefficient, and
$ \|\cdot\|_{s^*}$ is the overlapped Schatten norm defined as
$ \|X\|_{s^*} := \frac{1}{K} \sum_{k=1}^K \|\tilde{X}_{(k)}\|_s$.
Here, $\tilde{X}_{(k)}$ is the $k$-unfolding matrix defined by
concatenating the mode-$k$ fibers of $X$.  The overlapped Schatten
norm regularizes the rank of $X$ in terms of Tucker
decomposition~\cite{negahban2011estimation, tomioka2011statistical}.
Although the Tucker rank of $X^*$ is unknown in general, the convex
optimization adjusts the rank depending on $\lambda_n$.

To solve the convex problem~\eqref{opt:general}, the ADMM algorithm is often
employed~\cite{boyd2011distributed,tomioka2010estimation,
  tomioka2011statistical}.  Since the overlapped Schatten norm is not
differentiable, the ADMM algorithm avoids the differentiation of the
regularization term by alternatively minimizing the augmented
Lagrangian function iteratively.


%%% Local Variables:
%%% mode: latex
%%% TeX-master: "TTcomp_NIPS2017.tex"
%%% End:

\begin{algorithm}[!ht]
\begin{algorithmic}[1]
\Require Query workload $Q$, event stream $I$, \app\ graph $G$, hash table of snapshots $S$
\Ensure Hash table of results $R$ 
\State $G \leftarrow \emptyset$, $S, R \leftarrow$ empty hash tables
\ForAll {event $e \in I$ with $e.type=E$} 
    \State $//$ \textbf{\app\ graph construction}
    \ForAll {$q \in Q$ \text{ with event types }T}
        \ForAll {$E' \in T,\ E' \neq E$}
            \State $G_{E'} \leftarrow \mathit{getGraphlet}(G,E')$,
            $G_{E'}.\mathit{active} \leftarrow \mathit{false}$
        \EndFor
    \EndFor
    \If {\textbf{not} $G_E.\mathit{active}$}
        \State $G_E \leftarrow \mathit{createGraphlet()}$, $G_{E}.\mathit{active} \leftarrow \mathit{true}$,
        $G \leftarrow G \cup G_E$
        \If {$G_E.\mathit{shared}$ by $Q_E \subseteq Q$}
            $x \leftarrow \mathit{createSnapshot()}$ 
            \ForAll {$q \in Q_E$}
                \ForAll{$E' \in \mathit{pt}(E,q), E' \neq E$}
                    \State $G_{E'} \leftarrow \mathit{getGraphlet}(G,E')$
                    \State $S(x,q) \leftarrow S(x,q) + sum(G_{E'},q)$ \hspace{0.5cm}$//$ Eq.~5
                \EndFor
            \EndFor
        \EndIf    
    \EndIf
    \State insert $e$ into $G_E$
    \State $//$ \textbf{Trend count computation}
    \If {$G_E.\mathit{shared}$ by $Q_E \subseteq Q$}
        \If {$\forall q \in Q_E\ pe(e,q)$ are identical}
            \State $count(e,Q_E) \leftarrow count(e,q)$ \hspace{2.3cm}$//$ Eq.~2
        \Else\ $y \leftarrow \mathit{createSnapshot()}$, $count(e,Q_E) = y$
            \ForAll {$q \in Q_E$}
                $S(y,q) \leftarrow count(e,q)$ \hspace{0.2cm}$//$ Eq.~2
            \EndFor
          \EndIf
    \Else\ $count(e,q)$ \hspace{5.2cm}$//$ Eq.~2
    \EndIf
    \ForAll{$q \in Q$}
  	    \If {$E \in \mathit{end}(q)$} 
  		    $R(q) \leftarrow R(q) + count(e,q)$ $//$ Eq.~3
        \EndIf
    \EndFor
\EndFor
\State \Return $R$
\end{algorithmic}
\caption{\app\ shared online trend aggregation}
\label{algo:snapshot-propagation}
\end{algorithm}

\newcommand{\twomoons}{{\tt Twomoons}}
\newcommand{\gauss}{{\tt Gauss}}
\newcommand{\sculpture}{{\tt Sculpture}}
\newcommand{\baseline}{{\tt Baseline}}
\newcommand{\MM}{{\tt MsgPassing}}
\newcommand{\blackboard}{{\tt Blackboard}}
\newcommand{\ncut}{\text{ncut}}
\newcommand{\chensays}[2][]{\textcolor{blue} {\textsc{Jiecao #1:} \emph{#2}}}

\section{Experiments}
In this section we present experimental results for  graph clustering in the message passing and blackboard models. We will compare the following three algorithms. (1) \baseline: each site sends all the data to the coordinator directly; (2) \MM: our algorithm in the message passing model (Section~\ref{sec:gcmessage}); (3) 
\blackboard: our algorithm in  the blackboard model (Section~\ref{sec:bb}).


%Since both of our algorithms are crucially based on the use of spectral scarification, our main focus in the experiments is to investigate to what extend the quality of the spectral clustering algorithms will be affected by using spectral sparsification, the saving of communication costs by using spectral sparsificaion, ...
%
%
%The goal of this experiment is not to demonstrate the effectiveness of the spectral clustering algorithm. We mainly want to investigate the following, 
%\begin{itemize}
%\item to what extend the quality of clustered results will be affected by using spectral sparsification.
%\item saving of communication costs by using spectral sparsifier.
%\item the affect of constants in algorithms of the message passing/blackboard model.
%\end{itemize}
%
%
%\subsection{The Setup}
%\paragraph{Reference Algorithms}
%We compare different algorithms in our experiment.

%Note that we can also run \MM~ in the blackboard model.

Besides giving the visualized results of these algorithms on various datasets, we also measure the qualities of the results via the {\em normalized cut}, defined as 
\[
\ncut(A_1, \ldots, A_{k}) = \frac{1}{2}\sum_{i\in[k]}\frac{w(A_i, V\backslash A_i)}{\vol(A_i)},
\]
 which is a standard objective function to be minimized for spectral clustering algorithms. 
%We will compare the communication costs of these algorithms in different settings.

%We also compare the total communication costs of different algorithms/models. As the unit does not matter in our case, we normalize all communication costs by the cost of \baseline.  Whenever possible, we will visualize the clustered results.

We implemented the algorithms using multiple languages, including Matlab, Python and C++. Our experiments were conducted on an IBM NeXtScale nx360 M4 server, which is equipped with 2 Intel Xeon E5-2652 v2 8-core processors, 32GB RAM and 250GB local storage.


\subsection{Datasets.}
We test the algorithms in the following real and synthetic datasets, which is visualized in \figref{visualization}.


\begin{figure}[h]
     \centering
     \subfigure[\twomoons]{\includegraphics[width=0.23\textwidth]{twomoons-14000-original.png}\label{fig:twomoons}}
     ~~
     \subfigure[\gauss]{\includegraphics[width=0.23\textwidth]{gauss-10000-original.png}\label{fig:gauss}}
     ~~
     \subfigure[\sculpture]{\includegraphics[width=0.13\textwidth,height=0.16\textwidth]{sculpture-11680-original.jpg}\label{fig:sculpture}}
     \caption{Visualization of the datasets for our experiments.}
     \label{fig:visualization}
\end{figure}



\vspace{-1mm}
\begin{itemize}
\item \twomoons : this dataset contains $n=14,000$ coordinates in $\mathbb{R}^2$. We consider each point to be a vertex. For any two vertices $u, v$, we add an edge with weight $w(u,v) = \exp\{-\|u-v\|_2^2/\sigma^2\}$ with $\sigma = 0.1$ when one vertex is among the $7000$-nearest points of the other.  This construction results in a graph with about $110,000,000$ edges.

\item  \gauss : this dataset contains $n = 10,000$ points in $\mathbb{R}^2$. There are $4$ clusters in this dataset, each generated using a Gaussian distribution. We construct a complete graph as the similarity graph.  For any two vertices $u, v$, we define the weight $w(u,v) = \exp\{-\|u-v\|_2^2/\sigma^2\}$ with $\sigma = 1$. The resulting graph has about $100,000,000$ edges.

\item \sculpture : a photo of \textit{The Greek Slave}~\footnote{Available in e.g., \url{http://artgallery.yale.edu/collections/objects/14794}}. We use an $80\times 150$ version of this photo where each pixel is viewed as a vertex. To construct a similarity graph, we map each pixel to a point in $\mathbb{R}^5$, i.e., $(x, y, r, g, b)$, where the latter three coordinates are the RGB values. For any two vertices $u, v$, we  put an edge between $u, v$ with weight $w(u,v) = \exp\{-\|u-v\|_2^2/\sigma^2\}$ with $\sigma = 0.5$ if one of $u, v$ is among the $5000$-nearest points of the other. This results in a graph with about $70,000,000$ edges.
\end{itemize}
\vspace{-1mm}
In the distributed model edges are randomly partitioned across $s$ sites. 

%\vspace{-1.5mm}



\subsection{Results on clustering quality}
%{\em Quality.} \
\begin{figure*}[ht]
     \centering
     \subfigure[\baseline]{\includegraphics[width=0.2\textwidth]{twomoons-14000-original-clustered.png}\label{fig:twomoons-clustered-original}}
     \subfigure[\MM]{\includegraphics[width=0.2\textwidth]{twomoons-14000-sparsify-clustered-15.png}\label{fig:twomoons-clustered-sparsify}}
     \subfigure[\blackboard]{\includegraphics[width=0.2\textwidth]{twomoons-14000-chain-clustered.png}\label{fig:twomoons-clustered-chain}}
     \caption*{\twomoons, $k = 2$;}

\subfigure[\baseline]{\includegraphics[width=0.2\textwidth]{gauss-10000-original-clustered.png}\label{fig:gauss-clustered-original}}
     \subfigure[\MM]{\includegraphics[width=0.2\textwidth]{gauss-10000-sparsify-clustered-15.png}\label{fig:gauss-clustered-sparsify}}
     \subfigure[\blackboard]{\includegraphics[width=0.2\textwidth]{gauss-10000-chain-clustered.png}\label{fig:gauss-clustered-chain}}
     \caption*{\gauss, $k = 4$}


     \subfigure[\baseline]{\includegraphics[width=0.2\textwidth,height=0.2\textwidth]{sculpture-11680-original-clustered.png}\label{fig:sculpture-clustered-original}}  
     \subfigure[\MM]{\includegraphics[width=0.2\textwidth,height=0.2\textwidth]{sculpture-11680-sparsify-clustered-15.png}\label{fig:sculpture-clustered-sparsify}}
     \subfigure[\blackboard]{\includegraphics[width=0.2\textwidth,height=0.2\textwidth]{sculpture-11680-chain-clustered.png}\label{fig:sculpture-clustered-chain}}
     \caption*{\sculpture, $k = 3$. }


     
     \caption{Visualization of the results on \twomoons, \gauss\ and \sculpture. In the message passing model each site samples $5 n$ edges; in the blackboard model all sites jointly sample $10n$ edges (in \twomoons~ and \gauss) or $20n$ edges (in \sculpture) and the chain has length $18$. $s = 15$.}
     \label{fig:quality-1}
\end{figure*}

We visualize the clustered results for 
the \twomoons, \gauss\ and \sculpture\ in Figure~\ref{fig:quality-1}.
% and visualize the clustered results for \gauss\ and \sculpture in Figure~\ref{fig:quality-2}.
It can be seen that \baseline, \MM\ and \blackboard\ give results of very similar qualities.  For simplicity, here we only present the visualization for $s=15$. Similar results were observed when we varied the values of $s$.  
%\he{To Qin: Do you plan to have two titles (Results \& Quality)?}


% \begin{figure*}[h]
%      \centering
% \subfigure[\baseline]{\includegraphics[width=0.3\textwidth]{gauss-10000-original-clustered.png}\label{fig:gauss-clustered-original}}
%      \subfigure[\MM]{\includegraphics[width=0.3\textwidth]{gauss-10000-sparsify-clustered-15.png}\label{fig:gauss-clustered-sparsify}}
%      \subfigure[\blackboard]{\includegraphics[width=0.3\textwidth]{gauss-10000-chain-clustered.png}\label{fig:gauss-clustered-chain}}
%      \caption*{\gauss, $k = 4$}


%      \subfigure[\baseline]{\includegraphics[width=0.2\textwidth]{sculpture-11680-original-clustered.png}\label{fig:sculpture-clustered-original}}  
%      \subfigure[\MM]{\includegraphics[width=0.2\textwidth]{sculpture-11680-sparsify-clustered-15.png}\label{fig:sculpture-clustered-sparsify}}
%      \subfigure[\blackboard]{\includegraphics[width=0.2\textwidth]{sculpture-11680-chain-clustered.png}\label{fig:sculpture-clustered-chain}}
%      \caption*{\sculpture, $k = 3$. }

%      \caption{Visualization of results on \gauss\ and \sculpture; in the message passing model each site samples $5 n$ edges; in the blackboard model all sites jointly sample $10n$ (in \gauss) or $20n$ (in \sculpture) edges and the chain has length $18$.}
%      \label{fig:quality-2}
% \end{figure*}


We also compare the normalized cut (ncut) values of the clustering results of different algorithms.  The results are presented in Figure \ref{fig:quality}. In all datasets, the ncut values of different algorithms are very close. The ncut value of \MM\ slightly decreases when we increase the value of $s$, while the ncut value of \blackboard\ is independent of $s$.
%We comment that in general, it is difficult to compare \MM\ and \blackboard\ directly because they are affected by different parameters.


\begin{figure*}[!ht]
  \centering
  \subfigure[\twomoons]{\includegraphics[width=0.33\textwidth]{twomoons-14000-ncut.png}\label{fig:twomoons-quality}}\hspace*{-1.1em}
  \subfigure[\gauss]{\includegraphics[width=0.31\textwidth]{gauss-10000-ncut.png}\label{fig:gauss-quality}}\hspace*{-1.1em}
  \subfigure[\sculpture]{\includegraphics[width=0.31\textwidth]{sculpture-11680-ncut.png}\label{fig:sculpture-quality}}\hspace*{-1.1em}
  \subfigure{\includegraphics[width=0.14\textwidth]{legend.png}}
     \caption{Comparisons on normalized cuts. In the message passing model, each site samples $5n$ edges; in each round of the algorithm in the blackboard model, all sites jointly sample $10n$ edges (in \twomoons~and \gauss) or $20n$ edges (in \sculpture) edges and the chain has length $18$.}
     \label{fig:quality}
\end{figure*}

%\textcolor{red}{To Jiecao: Can you put the color lines indicating baseline, message passing, and blackboard within one row in Pic 2? Withthis we can save some space.}

%\vspace{-1.5mm}

\subsection{Results on communication costs} 
\begin{figure*}[!ht]
     \centering
     \subfigure[\twomoons]{\includegraphics[width=0.3\textwidth]{twomoons-14000-communication.png}\label{fig:twomoons-communication}}
     \subfigure[\gauss]{\includegraphics[width=0.3\textwidth]{gauss-10000-communication.png}\label{fig:gauss-communication}}
     \subfigure[\sculpture]{\includegraphics[width=0.3\textwidth]{sculpture-11680-communication.png}\label{fig:sculpture-communication}}


     \subfigure[\twomoons]{\includegraphics[width=0.32\textwidth]{twomoons-14000-communication-2.png}\label{fig:twomoons-communication-2}}
     \subfigure[\gauss]{\includegraphics[width=0.32\textwidth]{gauss-10000-communication-2.png}\label{fig:gauss-communication-2}}
     \subfigure[\sculpture]{\includegraphics[width=0.32\textwidth]{sculpture-11680-communication-2.png}\label{fig:sculpture-communication-2}}
     \caption{Comparisons on communication costs. In the message passing model, each site samples $5n$ edges; in each round of the algorithm in the blackboard model, all sites jointly sample $10n$ (in \twomoons~and \gauss) or $20n$ (in \sculpture) edges and the chain has length $18$. }
     \label{fig:communication}
\end{figure*}

We compare the communication costs of different algorithms in Figure \ref{fig:communication}. We observe that while achieving similar clustering qualities as \baseline, both \MM\ and \blackboard\ are significantly more communication-efficient (by one or two orders of magnitudes in our experiments). We also notice that the value of $s$ does not affect the communication cost of \blackboard, while the communication cost of \MM\ grows almost linearly with $s$; when $s$ is large, \MM\ uses significantly more communication than \blackboard. These confirm our theory.  %In Figure~\ref{fig:mm-const} and Figure~\ref{fig:blackboard-const}   in Appendix~\ref{sec:parameters} we present how the performance of \MM\ and \blackboard\ are affected by their parameters.

%
%
%\vspace{-1.5mm}
%\paragraph{Summary.}  From our experimental results we conclude that \MM\ and \blackboard\ achieve similar clustering quality as the native algorithm \baseline, while significantly reduce the communication cost.  When the number of sites is large, \blackboard\ is more communication efficient than \MM, as predicted by our theory.



\subsection{Parameters in \MM\ and \blackboard}
\label{sec:parameters}

Figure \ref{fig:mm-const} shows in \MM how the value of ncut is affected by the number of sites and the number of edges sampled in each site. 
Here, each site samples $cn$ edges. 
When $c=3$ and $s=1$, the ncut value diverges in all datasets. This is because with such a small $c$, the algorithm does not generate a valid sparsifier. In general, increasing $c$ or $s$ will slightly decrease the ncut value. But once they are above some thresholds, the ncut values of \MM\ and \baseline\ become very close.

Figure \ref{fig:blackboard-const} shows in \blackboard  how the ncut value is affected by the number of iterations and the number of edges sampled. When the number of iterations is set to be $5$, ncut values diverge in all datasets. This is because we cannot expect to generate a valid sparsifier by using such few iterations. It can be seen from \ref{fig:bb-gauss-constant} that for a fixed $c$, performing more iterations will help to reduce ncut values. From the same figure, one can also conclude that for fixed iterations, increasing $c$ also helps to reduce the ncut values.



\begin{figure*}[h!t]
     \centering
     \subfigure[\twomoons]{\includegraphics[width=0.3\textwidth]{twomoons-c.png}\label{fig:mm-twomoons-constant}}
     \subfigure[\gauss~dataset]{\includegraphics[width=0.3\textwidth]{gauss-c.png}\label{fig:mm-gauss-constant}}
     \subfigure[\sculpture]{\includegraphics[width=0.3\textwidth]{sculpture-c.png}\label{fig:mm-sculpture-constant}}
     \caption{The pictures above show the $\ncut$ values with respect to the values of $c$ and $s$ for the \MM\ algorithm. Here  
 each site samples $c n$ edges.}
     \label{fig:mm-const}
\end{figure*}


\begin{figure*}[h!t]
     \centering
     \subfigure[\twomoons]{\includegraphics[width=0.3\textwidth]{twomoons-iter.png}\label{fig:bb-twomoons-constant}}
     \subfigure[\gauss]{\includegraphics[width=0.3\textwidth]{gauss-iter.png}\label{fig:bb-gauss-constant}}
     \subfigure[\sculpture]{\includegraphics[width=0.3\textwidth]{sculpture-iter.png}\label{fig:bb-sculpture-constant}}
     \caption{The pictures above show how the $\ncut$ values are affected by the number of iterations and the value of $c$ for the \blackboard\ algorithm. Here 
all sites jointly sample $c n$ edges. }
     \label{fig:blackboard-const}
\end{figure*}







\begin{comment}
\begin{figure}
\includegraphics[width=\linewidth]{figs/beyond_tss_lesion.pdf}
\caption[]{End-to-End runtime lesion study of the entire MNIST dataset and the FMA featurized music dataset. Each of DROP's contributions provides a runtime improvement.}
\label{fig:beyond_lesion}
\end{figure}
\end{comment}



\section{Conclusion}
\label{sec:conclusion}

Advanced data analytics techniques must scale to rising data volumes. 
DR techniques offer a powerful toolkit when processing these datasets, with PCA frequently outperforming popular techniques in exchange for high computational cost. 
In response, we propose DROP, a new dimensionality reduction optimizer. 
DROP combines progressive sampling, progress estimation, and online aggregation to identify high quality low dimensional bases via PCA without processing the entire dataset by balancing the runtime of downstream tasks and achieved dimensionality. 
Thus, DROP provides a first step in bridging the gap between quality and efficiency in end-to-end DR for downstream \red{analytics}. 

%We revisit canonical operators for time series dimensionality reduction and the measurement study of~\cite{keogh-study}, and show that PCA is more effective than popular alternatives in the data mining literature often by a margin of over $2\times$ on average on gold-standard time series benchmark data sets with respect to output data dimension. More surprisingly, we empirically demonstrate that a small number of samples are sufficient to accurately characterize directions of maximum variance and obtain a high-quality low-dimensional transformation.




\bibliographystyle{IEEEtran}
\bibliography{ARSP_ref.bib}
	
\end{document}
