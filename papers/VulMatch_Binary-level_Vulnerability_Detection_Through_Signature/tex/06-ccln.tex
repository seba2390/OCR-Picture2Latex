\section{Conclusion}
\label{sec:ccln}

In this paper, we proposed a novel approach called \name to extract and match binary-level vulnerability-related signatures. 
\name consists of four steps: 1) data preparation, 2) locating signature instruction, 3) constructing context-aware binary-level signatures, and 4) signature matching. 
Compared to previous work, \name accurately locates vulnerability-related instructions and detects vulnerability within functions. 
Through our empirical studies, \name outperformed two state-of-the-art similarity-based vulnerability detection tools --- Asm2vec \cite{asm2vec} and Palmtree \cite{Palmtree}.
Specifically, \name achieved the most accurate results on six out of seven projects with the least ambiguities while providing reasons for vulnerable functions. 
Hence, \name effectively facilitates human understanding of its decision process during vulnerability detection. 
Our experiment on real-world firmware vulnerability detection indicates \name is practical to find vulnerabilities in real-world scenarios.
Our analysis of vulnerability distributions confirmed that \name is a versatile detector with good potential for future extension. 
