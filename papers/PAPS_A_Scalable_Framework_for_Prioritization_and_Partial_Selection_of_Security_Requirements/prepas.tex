\section{Pre-PAS Process}
\label{prepas}
The Pre-PAS process as depicted in Figure~\ref{fig_paps}, includes modeling, description and analysis of security requirements. %Firstly, the SRM of a software will be developed. Afterwards, \textit{Goal-Based Fuzzy Grammar} \cite{mougouei2013fuzzyBased} of the SRM will be constructed and fuzzy derivation rules will be identified. Risk analysis will be performed then, to identify cost and technical-ability of security requirements.

\subsection{Modeling and Description of Security Requirements}
\label{prepas_modeling_description}

%An efficient model is required to allow for prioritization and selection of security requirements \cite{roy2012scalable}. Such model should properly capture partiality of security requirements \cite{roy2012scalable,mougouei2013fuzzy}. Nonetheless, the existing security modeling techniques do not prevail due to the following reasons. Firstly, some of the existing security modeling techniques such as attack-response tree \cite{roy2012scalable} suffer from state-space explosion while others like Attack trees only capture security treats and ignore security requirements\cite{dewri2007optimal,edge2007framework,roy2010act}. Roy et. al \cite{roy2012scalable} resolved this problem by proposing Attack-Countermeasure Trees (ACT) \cite{roy2012attack}. Secondly, the existing security modeling techniques do not consider partiality of security. The attack-countermeasure trees also suffer from this problem. 

%In the following we give an overview of a goal-based modeling approach we developed in our prior works \cite{mougouei2013goal,mougouei2012goal,mougouei2012measuring,mougouei2012evaluating,mougouei2012measuring} which captures partiality of security requirements when building Security Requirement Model (SRM) of a software. Due to its inherent support for partial satisfaction~\cite{mougouei2015partial} of security requirements and capturing the influence of prioritization and selection on security goals, SRM is employed in this work to serve as the input of the PAS process in the PAPS framework. We have also made use of a fuzzy-based technique presented in our earlier work \cite{mougouei2013fuzzy,mougouei2013fuzzyBased,mougouei2014fuzzy} for description of SRMs. 

%\subsubsection{Modeling Security Requirements}
%\label{prepas_modeling}

Security Requirement Model (SRM) of a software is constructed by a goal-based modeling process presented in our earlier work\cite{mougouei2013goal}. The process starts with identification of the assets \cite{mead2006security,mougouei2013s} for a software product. Then, security goals will be developed to protect the assets against attack scenarios \cite{sindre2005eliciting}. Throughout the subsequent steps a security requirement model (SRM) of software will be constructed to mitigate security faults. Our goal-based modeling process made use of a combination of RELAX \cite{whittle2010relax} and KAOS \cite{van2004elaborating} description languages to describe security goals (requirements). The security requirement model of the OBS is illustrated in Figure \ref{fig_srm} and SRM nodes are described in Table \ref{table_description}. We have further, made use of a fuzzy-based technique presented in our earlier work \cite{mougouei2013fuzzyBased} to formally describe partiality in security requirement model of a software. In the following we have briefly described the main components of the employed fuzzy-based technique. 

\begin{table*}[t]
\caption{KAOS Description for Security Requirements (Goals) of the OBS.}
\label{table_description}
\centering
% Table generated by Excel2LaTeX from sheet 'Sheet1'
%\Huge
%\huge
%\LARGE
%\Large
%\large
%\normalsize (default)
%\small
%\footnotesize
%\scriptsize
%\tiny
\normalsize\resizebox {0.95\textwidth }{!}{
% Table generated by Excel2LaTeX from sheet 'Sheet1'
\begin{tabular}{llll}
	\toprule[1.5pt]
	\textbf{Goal} &
	\textbf{Description} &
	\textbf{Requirement} &
	\textbf{Description}
	\bigstrut\\
	\midrule[1.5pt]
	\cellcolor{blue!10}\cellcolor{blue!10}\textit{$S$} &
	\cellcolor{blue!10}\textit{maintain OBS security} &
	\multicolumn{1}{l}{\cellcolor{green!10}$R_{1}$} &
	\cellcolor{green!10}\textit{achieve request transaction code}
	\bigstrut[t]\\
	\cellcolor{blue!10}$G_{1}$ &
	\cellcolor{blue!10}\textit{avoid  transfer money out of account} &
	\multicolumn{1}{l}{\cellcolor{green!10}$R_{2}$} &
	\cellcolor{green!10}\textit{achieve latency examination }
	\\
	\cellcolor{blue!10}$G_{2}$ &
	\cellcolor{blue!10}\textit{avoid unauthorized online transfer} &
	\multicolumn{1}{l}{\cellcolor{green!10}$R_{3}$} &
	\cellcolor{green!10}\textit{achieve one-time pad}
	\\
	\cellcolor{blue!10}$G_{3}$ &
	\cellcolor{blue!10}\textit{avoid stealing id and password} &
	\multicolumn{1}{l}{\cellcolor{green!10}$R_{4}$} &
	\cellcolor{green!10}\textit{achieve SSL}
	\\
	\cellcolor{blue!10}$G_{4}$ &
	\cellcolor{blue!10}\textit{avoid man in the middle} &
	\multicolumn{1}{l}{\cellcolor{green!10}$R_{5}$} &
	\cellcolor{green!10}\textit{achieve password trial limitation}
	\\
	\cellcolor{blue!10}$G_{5}$ &
	\cellcolor{blue!10}\textit{avoid guessing id and password} &
	\multicolumn{1}{l}{\cellcolor{green!10}$R_{6}$} &
	\cellcolor{green!10}\textit{achieve password policy}
	\\
	\cellcolor{blue!10}$G_{6}$ &
	\cellcolor{blue!10}\textit{avoid dictionary attack} &
	\multicolumn{1}{l}{\cellcolor{green!10}$R_{7}$} &
	\cellcolor{green!10}\textit{achieve password encryption}
	\\
	\cellcolor{blue!10}$G_{7}$ &
	\cellcolor{blue!10}\textit{avoid guess password} &
	\multicolumn{1}{l}{\cellcolor{green!10}$R_{8}$} &
	\cellcolor{green!10}\textit{achieve random id}
	\\
	\cellcolor{blue!10}$G_{8}$ &
	\cellcolor{blue!10}\textit{avoid guess id } &
	\multicolumn{1}{l}{\cellcolor{green!10}$R_{9}$} &
	\cellcolor{green!10}\textit{achieve CAPTCHA}
	\\
	\cellcolor{blue!10}$G_{9}$ &
	\cellcolor{blue!10}\textit{avoid brute forcing} &
	\multicolumn{1}{l}{\cellcolor{green!10}$R_{10}$} &
	\cellcolor{green!10}\textit{achieve complex pin}
	\\
	\cellcolor{blue!10}$G_{10}$ &
	\cellcolor{blue!10}\textit{avoid unauthorized transfer via debit card} &
	\multicolumn{1}{l}{\cellcolor{green!10}$R_{11}$} &
	\cellcolor{green!10}\textit{achieve access control}
	\\
	\cellcolor{blue!10}$G_{11}$ &
	\cellcolor{blue!10}\textit{maintain transfer network security} &
	\multicolumn{1}{l}{\cellcolor{green!10}$R_{12}$} &
	\cellcolor{green!10}\textit{achieve redundant server}
	\\
	\cellcolor{blue!10}$G_{12}$ &
	\cellcolor{blue!10}\textit{avoid hijack server} &
	\multicolumn{1}{l}{\cellcolor{green!10}\textit{}} &
	\cellcolor{green!10}\textit{}
	\\
	\cellcolor{blue!10}$G_{13}$ &
	\cellcolor{blue!10}\textit{maintain service availability} &
	\multicolumn{1}{l}{\cellcolor{green!10}\textit{}} &
	\cellcolor{green!10}\textit{}
	\bigstrut[b]\\
	\bottomrule[1.5pt]
\end{tabular}}%

\end{table*}

\begin{figure}[!h]
	\centering
	\centerline{\includegraphics[scale=0.525]{srm.pdf}}
	\caption{SRM of the OBS}
	\label{fig_srm}
\end{figure}

%\subsubsection{Describing Security Requirements}
%\label{prepas_description}



\textit{i.Goal-Based Fuzzy Grammar (GFG)}. A GFG is defined as a quintuple of $GR = (G, R, P, S, \mu)$ in which $G$ is a set of security goals, $R$ is a set of security requirements, $P$ is a set of fuzzy derivation rules and $\mu$ denotes the membership function of derivation. $S$ represents the top-level security goal of the system. For $OBS$, $G=\{G_{1},...,G_{13}\}$, $R=\{R_{1},...,R_{12}\}$, $P=\{P_{1},...,P_{20}\}$ and \textit{S=``maintain [OBS] [security]''}. Due to its fuzziness, $GFG$ can properly capture partiality in SRM of a software. 

The elements of $P \in GR $ are expression of form given in \ref{Eq_rule} where `d' is the degree of contribution of a sub-goal `w' to satisfaction of a goal `r'. If $r_{1},..., r_{n}$ are fuzzy statements in $(G \cup R)^*$ and $r_{1} \rightarrow r_{2} \rightarrow ... \rightarrow r_{n}$, then we call this chain as goal derivation chain under the employed $GFG$. 

\begin{align}
\label{Eq_rule}
& \mu(r\rightarrow w)=d , d\in[0,1] \text{ or } \mu(r,w)=d
\end{align}

\begin{table*}[t]
	\caption{Derivation rules for SRM of the OBS.}
	\label{table_rules}
	\centering
	% Table generated by Excel2LaTeX from sheet 'Sheet1'
%\Huge
%\huge
%\LARGE
%\Large
%\large
%\normalsize (default)
%\small
%\footnotesize
%\scriptsize
%\tiny
\normalsize\rowcolors{1}{}{lightgray}
\resizebox {0.75\textwidth }{!}{
% Table generated by Excel2LaTeX from sheet 'Sheet1'
% Table generated by Excel2LaTeX from sheet 'Sheet1'
\begin{tabular}{llll}
	\toprule[1.5pt]
	\textbf{Rule}  &
	\textbf{Membership Value\quad} &
	\textbf{Rule} &
	\textbf{Membership Value}
	\bigstrut\\
	\midrule
	\textit{$P_{1}:\phantom{s} S \rightarrow G{1}G_{13}$ } &
	\phantom{ss}\textit{$ 0.95 $} &
	\textit{$P_{11}:\phantom{s} G_{4} \rightarrow R_{2}R_{3}$} &
	\phantom{ss}\textit{$ 0.75 $}
	\bigstrut[t]\\
	\textit{$P_{2}:\phantom{s} G_{1} \rightarrow G_{2}G_{10}G_{12}$} &
	\phantom{ss}\textit{$ 0.95 $} &
	\textit{$P_{12}:\phantom{s} G_{4} \rightarrow R_{4}$} &
	\phantom{ss}\textit{$ 0.9 $}
	\\
	\textit{$P_{3}:\phantom{s} G_{13} \rightarrow R_{12}$} &
	\phantom{ss}\textit{$ 0.9 $} &
	\textit{$P_{13}:\phantom{s} G_{6} \rightarrow G_{7}$} &
	\phantom{ss}\textit{$ 0.6 $}
	\\
	\textit{$P_{4}:\phantom{s} G_{2} \rightarrow R_{1}G_{3}G_{5}$} &
	\phantom{ss}\textit{$ 0.85 $} &
	\textit{$P_{14}:\phantom{s} G_{6} \rightarrow G_{8}$} &
	\phantom{ss}\textit{$ 0.6 $}
	\\
	\textit{$P_{5}:\phantom{s} G_{10} \rightarrow G_{11}$} &
	\phantom{ss}\textit{$ 0.9 $} &
	\textit{$P_{15}:\phantom{s} G{9} \rightarrow G_{7}$} &
	\phantom{ss}\textit{$ 0.65 $}
	\\
	\textit{$P_{6}:\phantom{s} G_{10} \rightarrow R_{10}$} &
	\phantom{ss}\textit{$ 0.4 $} &
	\textit{$P_{16}:\phantom{s} G_{9} \rightarrow G_{8}$} &
	\phantom{ss}\textit{$ 0.6 $}
	\\
	\textit{$P_{7}:\phantom{s} G_{12} \rightarrow R_{11}$} &
	\phantom{ss}\textit{$ 0.9 $} &
	\textit{$P_{17}:\phantom{s} G_{7} \rightarrow R_{5}$} &
	\phantom{ss}\textit{$ 0.7 $}
	\\
	\textit{$P_{8}:\phantom{s} G_{3} \rightarrow G_{4}$} &
	\phantom{ss}\textit{$ 0.85 $} &
	\textit{$P_{18}:\phantom{s} G_{7} \rightarrow R_{6}$} &
	\phantom{ss}\textit{$ 0.8 $}
	\\
	\textit{$P_{9}:\phantom{s} G_{5} \rightarrow G_{6}G_{9}$} &
	\phantom{ss}\textit{$ 0.9 $} &
	\textit{$P_{19}:\phantom{s} G_{7} \rightarrow R_{7}$} &
	\phantom{ss}\textit{$ 0.9 $}
	\\
	\textit{$P_{10}:\phantom{s} G_{11} \rightarrow R_{9}$} &
	\phantom{ss}\textit{$ 0.8 $} &
	\textit{$P_{20}:\phantom{s} G_{8} \rightarrow R_{8}$} &
	\phantom{ss}\textit{$ 0.6 $}
	\bigstrut[b]\\
	\bottomrule[1.5pt]
\end{tabular}}%
\end{table*}

\textit{ii. Extracting Derivation Rules}. The employed description technique, constructs a GFG for a given SRM and identifies the derivation rules~\cite{mougouei2013fuzzy,mougouei2013fuzzyBased,mougouei2013fuzzy1}. The degree to which the successive of a rule contributes to satisfaction of its predecessor, will specify its membership value. This value will be determined by the membership function µ of GFG \cite{mougouei2013fuzzyBased}. The extracted derivation rules for SRM of the OBS, and their corresponding membership values are listed in Table \ref{table_rules}. 

%The derivation rule $P_{11} (G_{1} \rightarrow R_{2}R_{3})$ in Table \ref{table_rules}, firstly explains that the security goal $G_{1}$ derives security requirements $R_{2}$ and $R_{3}$. Secondly specifies the AND relation between $R_{2}$ and $R_{3}$. In other words, to satisfy the security goal $G_{1}$, it is required to satisfy both $R_{2}$ and $R_{3}$.

\subsection{Risk Analysis}
\label{prepas_risk} 

During the risk analysis, the cost and technical-ability of security requirements will be identified. 

\textit{Cost of Implementation}. Owing to budget limitations we need to care for cost of implementation while selecting and prioritizing security requirements \cite{karlsson1997cost}. Table \ref{table_risk} has listed the cost and values for security requirements in the SRM of the OBS. Cost of implementation is a real number in $[1,100]$ \cite{mougouei2012measuring,mougouei2012evaluating}.  

\begin{table*}[t]
	\caption{Cost and Technical-ability of the OBS Security Requirements.}
	\label{table_risk}
	\centering
	% Table generated by Excel2LaTeX from sheet 'Sheet1'
%\Huge
%\huge
%\LARGE
%\Large
%\large
%\normalsize (default)
%\small
%\footnotesize
%\scriptsize
%\tiny
\normalsize\rowcolors{1}{}{lightgray}
\resizebox {0.75\textwidth }{!}{
% Table generated by Excel2LaTeX from sheet 'Sheet1'
% Table generated by Excel2LaTeX from sheet 'Sheet1'
% Table generated by Excel2LaTeX from sheet 'Sheet1'
\begin{tabular}{lllllllllllll}
	\toprule[1.5pt]
	\textbf{Requirement} &
	\textit{$R_1$} &
	\textit{$R_2$} &
	\textit{$R_3$} &
	\textit{$R_4$} &
	\textit{$R_5$} &
	\textit{$R_6$} &
	\textit{$R_7$} &
	\textit{$R_8$} &
	\textit{$R_9$} &
	\textit{$R_{10}$} &
	\textit{$R_{11}$} &
	\textit{$R_{12}$}
	\bigstrut\\
	\hline
	\textbf{Cost} &
	\textit{$ 0.50 $} &
	\textit{$0.70$} &
	\textit{$0.70$} &
	\textit{$ 0.30 $} &
	\textit{$ 0.05 $} &
	\textit{$ 0.50 $} &
	\textit{$ 0.20 $} &
	\textit{$ 0.01 $} &
	\textit{$ 0.60 $} &
	\textit{$ 0.10 $} &
	\textit{$0.70$} &
	\textit{$ 1.00 $}
	\bigstrut[t]\\
	\textbf{Technical-ability} &
	\textit{$ 1.00 $} &
	\textit{$ 0.20 $} &
	\textit{$ 0.10 $} &
	\textit{$ 0.90 $} &
	\textit{$ 1.00 $} &
	\textit{$ 0.30 $} &
	\textit{$ 0.20 $} &
	\textit{$ 1.00 $} &
	\textit{$ 0.10 $} &
	\textit{$ 1.00 $} &
	\textit{$ 0.20 $} &
	\textit{$ 0.20 $}
	\bigstrut[b]\\
	\bottomrule[1.5pt]
\end{tabular}}%
\end{table*}

\textit{Technical-Ability}. Technical-ability is a real number in $[0,1]$ which reflects the ease of implementation for each requirement. Technical-ability of security requirements of the OBS are computed based on (\ref{Eq_technical}) and listed in Table \ref{table_risk}.  

\begin{align}
\label{Eq_technical}
& Technical\text{-}Ability=\frac{1}{\textit{Technical Complexity of Requirement}}
\end{align}