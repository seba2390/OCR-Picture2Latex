\subsection{Data Preprocessing}
\label{pas_pre}

%Data preprocessing includes construction of SRL and calculation of impact values. 

\setlength{\fboxsep}{6pt}%
\setlength{\fboxrule}{0.5pt}%


%\subsubsection{Construction of Security Requirement List }
%\label{pas_pre_cons}
Data preprocessing includes construction of SRL and calculation of impact values. Let $GR = (G, R, P, S, \mu)$ be the GFG of a SRM. For each security goal $g \in G$, $SRL[g]$ contains security requirements which contribute to satisfaction of goal g. $SRL[g]$ is constructed for every goal \textit{g} in SRM of the system. This allows for goal-oriented prioritization of security requirements with special focus on satisfaction of individual security goals. For each security requirement $x$ in $SRL[g]$ the degree to which $x$ contributes to satisfaction of the goal $g$ is referred to as the impact of $x$ on $g$ and computed by taking maximum ($ \oplus $) over membership degree of all derivations paths that can derive $x$ from $g$. Membership of each path is computed by taking minimum ($ \otimes $) over membership degrees of all derivations rules on the path. Impacts for security requirements of the OBS are listed in Table \ref{table_impact}. 

%as follows. Firstly, the membership values of the derivation rules on derivation chain of \textit{x} will be computed based on (\ref{Eq_imp}). Then, the fuzzy membership values will be calculated for each derivation chain through taking minimum over all membership value of derivation rules on the derivation chain of \textit{x}. Finally, impact will be calculated through taking supremum over all membership values of the derivation chains which can derive \textit{x}. $ \oplus $ and $ \otimes $ denote fuzzy OR (maximum) and AND (minimum) operators respectively. Impacts for security requirements of the OBS are listed in Table \ref{table_impact}.

\begin{align}
\label{Eq_imp}
& DC_{g} (x) = \mu_{g}(x)= \oplus(\mu(g,r_1)\otimes\mu(r_1,r_2)\otimes ... \otimes\mu(r_n,x))
\end{align}

%The t-norm sign $ \oplus $ and a t-conorm sign $ \otimes $ denote fuzzy OR (maximum) and AND (minimum) operators respectively based on the Zadeh’s definition in \cite{zadeh1965fuzzy}. $\mu_{g}(x)$ specifies the strongest degree for contribution of \textit{x} to satisfaction of the goal \textit{g}. Impacts for security requirements of the OBS are listed in Table \ref{table_impact}.   %As one example, $\mu_S(R_7)$ for $R_7$ in SRM of the OBS is calculated for two derivation chains: 1) $S\rightarrow G_1\rightarrow G_2\rightarrow G_5\rightarrow G_9\rightarrow G_7\rightarrow R_7$ and 2) $S\rightarrow G_1\rightarrow G_2\rightarrow G_5\rightarrow G_6\rightarrow G_7\rightarrow R_7$, as follows: $\mu_S (X=R_7)= (0.95 \otimes 0.95 \otimes0 .85 \otimes0 .9 \otimes 0 .65 \otimes 0.9) \oplus (0.95 \otimes0 .95 \otimes0 .85 \otimes 0.9 \otimes0.6 \otimes 0.9) = 0.65$. Impact values for security requirements in SRM of OBS are calculated by function ``calculateImpact" in Figure \ref{fig_code1} and listed in Table \ref{table_impact}.  

\begin{table*}[t]
	\caption{Impact of Security Requirements in the SRM of the OBS.}
	\label{table_impact}
	\centering
	% Table generated by Excel2LaTeX from sheet 'Sheet11'
\rowcolors{1}{}{lightgray}
%| L{50mm} | C{300mm} |
\resizebox {0.75\textwidth }{!}{\begin{tabular}{lllllllllllll}
\Xhline{4\arrayrulewidth}
Goal &
  $\mu(R_1)$ &
  $\mu(R_2)$ &
  $\mu(R_3)$ &
  $\mu(R_4)$ &
  $\mu(R_5)$ &
  $\mu(R_6)$ &
  $\mu(R_7)$ &
  $\mu(R_8)$ &
  $\mu(R_9)$ &
 $ \mu(R_{10})$ &
  $\mu(R_{11})$ &
  $\mu(R_{12}$)
  \\
\Xhline{4\arrayrulewidth}
S &
  $0.85$ &
  $0.75$ &
  $0.75$ &
  $0.85$ &
  $0.65$ &
  $0.65$ &
  $0.65$ &
  $0.60$ &
  $0.80$ &
  $0.40$ &
  $0.90$ &
  $0.90$
  \bigstrut\\
$G_1$ &
  $0.85$ &
  $0.75$ &
  $0.75$ &
  $0.85$ &
  $0.65$ &
  $0.65$ &
  $0.65$ &
  $0.60$ &
  $0.80$ &
  $0.40$ &
  $0.90$ &
  $0.00$
  \\
$G_2$ &
  $0.85$ &
  $0.75$ &
  $0.75$ &
  $0.85$ &
  $0.65$ &
  $0.65$ &
  $0.65$ &
  $0.60$ &
  $0.00$ &
  $0.00$ &
  $0.00$ &
  $0.00$
  \\
$G_3$ &
  $0.00$ &
  $0.75$ &
  $0.75$ &
  $0.85$ &
  $0.00$ &
  $0.00$ &
  $0.00$ &
  $0.00$ &
  $0.00$ &
  $0.00$ &
  $0.00$ &
  $0.00$
  \\
$G_4$ &
  $0.00$ &
  $0.75$ &
  $0.75$ &
  $0.90$ &
  $0.00$ &
  $0.00$ &
  $0.00$ &
  $0.00$ &
  $0.00$ &
  $0.00$ &
  $0.00$ &
  $0.00$
  \\
$G_5$ &
  $0.00$ &
  $0.00$ &
  $0.00$ &
  $0.00$ &
  $0.65$ &
  $0.65$ &
  $0.65$ &
  $0.60$ &
  $0.00$ &
  $0.00$ &
  $0.00$ &
  $0.00$
  \\
$G_6$ &
  $0.00$ &
  $0.00$ &
  $0.00$ &
  $0.00$ &
  $0.60$ &
  $0.60$ &
  $0.60$ &
  $0.60$ &
  $0.00$ &
  $0.00$ &
  $0.00$ &
  $0.00$
  \\
$G_7$ &
  $0.00$ &
  $0.00$ &
  $0.00$ &
  $0.00$ &
  $0.70$ &
  $0.80$ &
  $0.90$ &
  $0.00$ &
  $0.00$ &
  $0.00$ &
  $0.00$ &
  $0.00$
  \\
  $G_8$ &
  $0.00$ &
  $0.00$ &
  $0.00$ &
  $0.00$ &
  $0.00$ &
  $0.00$ &
  $0.00$ &
  $0.60$ &
  $0.00$ &
  $0.00$ &
  $0.00$ &
  $0.00$
  \\
$G_9$ &
  $0.00$ &
  $0.00$ &
  $0.00$ &
  $0.00$ &
  $0.65$ &
  $0.65$ &
  $0.65$ &
  $0.65$ &
  $0.00$ &
  $0.00$ &
  $0.00$ &
  $0.00$
  \\
$G_{10}$ &
  $0.00$ &
  $0.00$ &
  $0.00$ &
  $0.00$ &
  $0.00$ &
  $0.00$ &
  $0.00$ &
  $0.00$ &
  $0.80$ &
  $0.40$ &
  $0.00$ &
  $0.00$
  \\
$G_{11}$ &
  $0.00$ &
  $0.00$ &
  $0.00$ &
  $0.00$ &
  $0.00$ &
  $0.00$ &
  $0.00$ &
  $0.00$ &
  $0.80$ &
  $0.00$ &
  $0.00$ &
  $0.00$
  \\
$G_{12}$ &
  $0.00$ &
  $0.00$ &
  $0.00$ &
  $0.00$ &
  $0.00$ &
  $0.00$ &
  $0.00$ &
  $0.00$ &
  $0.00$ &
  $0.00$ &
  $0.90$ &
  $0.00$
  \\
$G_{13}$ &
  $0.00$ &
  $0.00$ &
  $0.00$ &
  $0.00$ &
  $0.00$ &
  $0.00$ &
  $0.00$ &
  $0.00$ &
  $0.00$ &
  $0.00$ &
  $0.00$ &
  $0.90$
  \\
\Xhline{4\arrayrulewidth}
\end{tabular}}%

\end{table*}
