% This is LLNCS.DEM the demonstration file of
% the LaTeX macro package from Springer-Verlag
% for Lecture Notes in Computer Science,
% version 2.4 for LaTeX2e as of 16. April 2010
%
\documentclass{llncs}
%\usepackage{makeidx}  % allows for indexgeneration
\usepackage{llncsdoc}

\usepackage{amsmath,amsfonts,amssymb,epsfig,epstopdf,url,array}
\let\proof\relax 
\let\endproof\relax
\usepackage{amsthm}
\usepackage{color,soul}
\usepackage{algorithmic}
\usepackage{balance}
\usepackage[ruled]{algorithm2e}
\usepackage{stmaryrd}
\usepackage[table]{xcolor}
\usepackage{centernot}
\usepackage{tikz}
\usepackage{booktabs}
\usepackage{pdfpages}
\usepackage{graphicx}
\usepackage{subfigure}
\usepackage[font=small,skip=8pt]{caption}
\usepackage{tabularx}
\usepackage{comment}
\usepackage{multirow}
\usepackage{multicol}
\usepackage{bigstrut}
\usepackage{pdflscape}
\usepackage{rotating}
\setlength{\rotFPtop}{0pt plus 1fil}
\usepackage{float}
\usepackage{lipsum}
\usepackage{enumitem}
\usetikzlibrary{plotmarks,shapes,arrows,chains,hobby,backgrounds,calc,trees}
\usepackage{makecell}
\usepackage{dblfloatfix}    % To enable figures at the bottom of page
\usepackage{pdfcomment} %\pdfmarkupcomment[markup=Highlight,color=yellow]{\emph{and not accurate and efficient}}{comment}

%---------------------COMMANDS-------------------------------------------------------
\newcommand{\head}[1]{\textbf{#1}}
\definecolor{lightgray}{gray}{0.9}

\theoremstyle{definition}
\newtheorem{exmp}{Example}
\theoremstyle{definition}
\newtheorem{mydef}{Proposition}
\theoremstyle{definition}
\newtheorem{defn}{Definition}


\newcommand*{\permcomb}[4][0mu]{{{}^{#3}\mkern#1#2_{#4}}}
\newcommand*{\perm}[1][-3mu]{\permcomb[#1]{P}}
\newcommand*{\comb}[1][-1mu]{\permcomb[#1]{C}}
\newcommand\lreqn[2]{\noindent\makebox[\textwidth-2cm]{$\displaystyle#1$\hfill(#2)}\vspace{2ex}}

\renewcommand{\algorithmicrequire}{\textbf{Input:}}
\renewcommand{\algorithmicensure}{\textbf{Output:}}

\begin{document}

\mainmatter              % start of the contributions
\title{PAPS: A Scalable Framework for Prioritization and Partial Selection of Security Requirements}
\titlerunning{PAPS}  % abbreviated title (for running head)
\author{Davoud Mougouei}
\authorrunning{Davoud Mougouei et al.} % abbreviated author list (for running head)

\institute{School of Computer Science, Engineering, and Mathematics\\ Flinders University, Adelaide, Australia}

\maketitle              % typeset the title of the contribution

\begin{abstract}
\begin{abstract}
\label{sec:abstract}

%% 1. what is the problem 
Scientific applications that run on leadership computing facilities often face the challenge 
of being unable to fit leading science cases onto accelerator devices due to memory constraints 
(memory-bound applications).
%
% 2. what is your solution 
In this work, the authors studied one such US Department of Energy mission-critical condensed matter 
physics application, Dynamical Cluster Approximation (DCA++), and this paper discusses how device memory-bound challenges were successfully reduced  by proposing an effective 
``all-to-all'' communication method---a ring communication algorithm. 
%
This implementation takes advantage of acceleration on GPUs and remote direct memory access (RDMA) for fast data exchange between GPUs. 
%
\\Additionally, the ring algorithm was optimized with sub-ring communicators
and multi-threaded support to further reduce communication overhead and 
expose more concurrency, respectively.
%
% 3. What's the cherry-picked evaluation result you want to mention
The computation and communication were also analyzed 
by using the Autonomic Performance Environment for Exascale 
(APEX) profiling tool,  and this paper further discusses the 
performance trade-off for the ring algorithm implementation. 
%
The memory analysis on the ring algorithm shows that the allocation size for the authors' most 
memory-intensive data structure per GPU is now reduced to $1/p$ of the original size, where $p$ is the number of GPUs in the ring communicator.
%
The communication analysis suggests that 
the distributed Quantum Monte Carlo execution time grows linearly as sub-ring size increases, and the cost of messages passing through the network interface connector could be a limiting factor.


%
% \todoRed{Ronnie: Next sentence needs rewrite, too much information about Green's function that no one knows in the abstract; recommend generalizing.} \emph {However, DCA++ is currently facing memory-bound challenge as 
% a larger device array $G_t$ is limited by device memory size, where
% $G_t$ is a two-particle Green's function that allows condensed matter
% scientists to explore larger and more complex (higher fidelity)
% physics cases.}

\end{abstract}

\keywords{DCA++, Quantum Monte Carlo, GPU Remote Direct Memory Access, memory-bound issue, exascale machines}

\keywords{Security, Requirement, Partial, Selection, Fuzzy}
\end{abstract}

\section{Introduction}  \label{sec:introduction}

\newcommand\inexpIntro[3]{#1?(#2,#3).}
\newcommand\rinexpIntro[3]{*#1?(#2,#3).}
\newcommand\outexpIntro[3]{#1!(#2,#3).}
\newcommand\outatomIntro[3]{#1!(#2,#3)}

We propose a fully automated method for proving termination of \(\pi\)-calculus processes.
Although there have been a lot of studies on termination analysis for the \(\pi\)-calculus
and related calculi~\cite{Deng06IC,Demangeon07,SangiorgiTermination,KobayashiHybrid,Yoshida04IC,DBLP:journals/jlp/DemangeonHS10,Venet98SAS}, most of them have been rather theoretical,
and there have been surprisingly little efforts in developing  fully automated termination
verification methods and tools based on them. To our knowledge,
Kobayashi's \typical{}~\cite{TyPiCal,KobayashiHybrid} is the only exception that
can prove termination of \(\pi\)-calculus processes (extended with natural numbers)
fully automatically, but its termination analysis is quite limited (see Section~\ref{sec:relatedwork}).

Our method is based on a reduction to termination analysis for sequential programs:
we translate a \(\pi\)-calculus process \(P\) to a sequential program \(S_P\), so that
if \(S_P\) is terminating, so is \(P\). The reduction allows us to use
powerful, mature methods and tools
for termination analysis of sequential programs~\cite{heizmann2016ultimate,freqterm,DBLP:conf/lics/PodelskiR04,Kuwahara2014Termination,DBLP:journals/cacm/CookPR11}.

The idea of the translation is to convert a chain of communications on replicated input
channels to a chain of recursive function calls of the target sequential program.
Let us consider the following Fibonacci process:
\begin{align*}
    & \rinexpIntro{\fib}{n}{r}
        \ifexp{n<2}{ \soutatom{r}{1} \\ &\quad}
                   { \nuexp{s_1} \nuexp{s_2} (\outatomIntro{\fib}{n-1}{s_1} \PAR \outatomIntro{\fib}{n-2}{s_2} \PAR \sinexp{s_1}{x}\sinexp{s_2}{y}\soutatom{r}{x+y}) \\}
    & \PAR \outatomIntro{\fib}{m}{r}
\end{align*}
Here, the process
$\rinexpIntro{\fib}{n}{r} \ldots$ is a function server that computes the \(n\)-th Fibonacci number
in parallel and returns the result to \(r\),
and $\outatom{\fib}{m}{r}$ sends a request for computing the \(m\)-th Fibonacci number;
those who are not familiar with the syntax of the \(\pi\)-calculus may wish to consult
Section~\ref{sec:targetlanguage} first.
To prove that the process above is terminating for any integer \(m\),
it suffices to show that there is no infinite chain of communications on $\fib$:
\[
    \fib(m,r) \to \fib(m_1,r_1) \to \fib(m_2,r_2) \to \cdots.
\]
We convert the process above to the following program:\footnote{The actual translation
  given later is a little more complex.}
\begin{verbatim}
 let rec fib(n) = if n<2 then () else (fib(n-1) [] fib(n-2)) in
 fib(m)
\end{verbatim}
Here, \texttt{[]} represents the non-deterministic choice.
Note that, although the calculation of Fibonacci numbers is not preserved,
for each chain of communications on \texttt{fib}, there is a corresponding
sequence of recursive calls:
\[
\mathtt{fib}(m) \to \mathtt{fib}(m_1) \to \mathtt{fib}(m_2) \to \cdots.
\]
Thus, the termination of the sequential program above implies the termination of
the original process.
As shown in the example above, (i) each communication on a replicated input channel
is converted to a function call, (ii) each communication on a non-replicated input
channel is just removed (or, in the actual translation, replaced by a call of
a trivial function defined by \(f(\seq{x})=(\,)\)), and (iii) parallel composition
is replaced by a non-deterministic choice.
We formalize the translation outlined above and prove its correctness.

The basic translation sketched above sometimes loses too much information.
For example, consider the following process:
\begin{align*}
    & \rinexpIntro{\pre}{n}{r} \soutatom{r}{n-1} \\
    & \PAR \rinexpIntro{f}{n}{r} \ifexp{n<0}{ \soutatom{r}{1} }
                                       { \nuexp{s} (\outatomIntro{\pre}{n}{s} \PAR \sinexp{s}{x}\outatomIntro{f}{x}{r}) } \\
    & \PAR \outatomIntro{f}{m}{r}
\end{align*}
The translation sketched above would yield:
\begin{verbatim}
  let pred(n) = n-1 in
  let rec f(n) = if n<0 then () else (pred(n) [] f(*)) in
  f(m)
\end{verbatim}
Here, \texttt{*} represents a non-deterministic integer: since we have removed
the input $\sinatom{s}{x}$, we do not have information about the value of \( x \).
As a result, the sequential program above is non-terminating, although the original
process is terminating.
To remedy this problem, we also refine the basic translation above by using a refinement
type system for the \(\pi\)-calculus. Using the refinement type system,
we can infer that the value of \(x\) in the original process is less than \(n\),
so that we can refine the definition of \texttt{f} to:
\begin{verbatim}
 let rec f(n) = ... else (pred(n) [] let x=* in assume(x<n);f(x))
\end{verbatim}
The target program is now terminating, from which
we can deduce that the original process is also terminating.
We have implemented an automated tool based on the refined translation above.

The contributions of this paper are summarized as follows.
\begin{itemize}
\item The formalization of the basic translation from the \(\pi\)-calculus
  (extended with integers) to sequential programs, and a proof of its correctness.
\item The formalization of a refined translation based on a refinement type system.
\item An implementation of the refined translation, including automated refinement type
  inference based on CHC solving, and experiments to evaluate the effectiveness of
  our method.
\end{itemize}

The rest of this paper is structured as follows.
Section~\ref{sec:targetlanguage} introduces the source and target languages
of our translation.
Section~\ref{sec:approach} 
formalizes the basic translation, and proves its correctness.
Section~\ref{sec:refinement} refines the basic translation by using a refinement type system.
Section~\ref{sec:implementation} reports an implementation and experiments.
Section~\ref{sec:relatedwork} discusses related work,
and Section~\ref{sec:conclusion} concludes the paper.

\section{Pre-PAS Process}
\label{prepas}
The Pre-PAS process as depicted in Figure~\ref{fig_paps}, includes modeling, description and analysis of security requirements. %Firstly, the SRM of a software will be developed. Afterwards, \textit{Goal-Based Fuzzy Grammar} \cite{mougouei2013fuzzyBased} of the SRM will be constructed and fuzzy derivation rules will be identified. Risk analysis will be performed then, to identify cost and technical-ability of security requirements.

\subsection{Modeling and Description of Security Requirements}
\label{prepas_modeling_description}

%An efficient model is required to allow for prioritization and selection of security requirements \cite{roy2012scalable}. Such model should properly capture partiality of security requirements \cite{roy2012scalable,mougouei2013fuzzy}. Nonetheless, the existing security modeling techniques do not prevail due to the following reasons. Firstly, some of the existing security modeling techniques such as attack-response tree \cite{roy2012scalable} suffer from state-space explosion while others like Attack trees only capture security treats and ignore security requirements\cite{dewri2007optimal,edge2007framework,roy2010act}. Roy et. al \cite{roy2012scalable} resolved this problem by proposing Attack-Countermeasure Trees (ACT) \cite{roy2012attack}. Secondly, the existing security modeling techniques do not consider partiality of security. The attack-countermeasure trees also suffer from this problem. 

%In the following we give an overview of a goal-based modeling approach we developed in our prior works \cite{mougouei2013goal,mougouei2012goal,mougouei2012measuring,mougouei2012evaluating,mougouei2012measuring} which captures partiality of security requirements when building Security Requirement Model (SRM) of a software. Due to its inherent support for partial satisfaction~\cite{mougouei2015partial} of security requirements and capturing the influence of prioritization and selection on security goals, SRM is employed in this work to serve as the input of the PAS process in the PAPS framework. We have also made use of a fuzzy-based technique presented in our earlier work \cite{mougouei2013fuzzy,mougouei2013fuzzyBased,mougouei2014fuzzy} for description of SRMs. 

%\subsubsection{Modeling Security Requirements}
%\label{prepas_modeling}

Security Requirement Model (SRM) of a software is constructed by a goal-based modeling process presented in our earlier work\cite{mougouei2013goal}. The process starts with identification of the assets \cite{mead2006security,mougouei2013s} for a software product. Then, security goals will be developed to protect the assets against attack scenarios \cite{sindre2005eliciting}. Throughout the subsequent steps a security requirement model (SRM) of software will be constructed to mitigate security faults. Our goal-based modeling process made use of a combination of RELAX \cite{whittle2010relax} and KAOS \cite{van2004elaborating} description languages to describe security goals (requirements). The security requirement model of the OBS is illustrated in Figure \ref{fig_srm} and SRM nodes are described in Table \ref{table_description}. We have further, made use of a fuzzy-based technique presented in our earlier work \cite{mougouei2013fuzzyBased} to formally describe partiality in security requirement model of a software. In the following we have briefly described the main components of the employed fuzzy-based technique. 

\begin{table*}[t]
\caption{KAOS Description for Security Requirements (Goals) of the OBS.}
\label{table_description}
\centering
% Table generated by Excel2LaTeX from sheet 'Sheet1'
%\Huge
%\huge
%\LARGE
%\Large
%\large
%\normalsize (default)
%\small
%\footnotesize
%\scriptsize
%\tiny
\normalsize\resizebox {0.95\textwidth }{!}{
% Table generated by Excel2LaTeX from sheet 'Sheet1'
\begin{tabular}{llll}
	\toprule[1.5pt]
	\textbf{Goal} &
	\textbf{Description} &
	\textbf{Requirement} &
	\textbf{Description}
	\bigstrut\\
	\midrule[1.5pt]
	\cellcolor{blue!10}\cellcolor{blue!10}\textit{$S$} &
	\cellcolor{blue!10}\textit{maintain OBS security} &
	\multicolumn{1}{l}{\cellcolor{green!10}$R_{1}$} &
	\cellcolor{green!10}\textit{achieve request transaction code}
	\bigstrut[t]\\
	\cellcolor{blue!10}$G_{1}$ &
	\cellcolor{blue!10}\textit{avoid  transfer money out of account} &
	\multicolumn{1}{l}{\cellcolor{green!10}$R_{2}$} &
	\cellcolor{green!10}\textit{achieve latency examination }
	\\
	\cellcolor{blue!10}$G_{2}$ &
	\cellcolor{blue!10}\textit{avoid unauthorized online transfer} &
	\multicolumn{1}{l}{\cellcolor{green!10}$R_{3}$} &
	\cellcolor{green!10}\textit{achieve one-time pad}
	\\
	\cellcolor{blue!10}$G_{3}$ &
	\cellcolor{blue!10}\textit{avoid stealing id and password} &
	\multicolumn{1}{l}{\cellcolor{green!10}$R_{4}$} &
	\cellcolor{green!10}\textit{achieve SSL}
	\\
	\cellcolor{blue!10}$G_{4}$ &
	\cellcolor{blue!10}\textit{avoid man in the middle} &
	\multicolumn{1}{l}{\cellcolor{green!10}$R_{5}$} &
	\cellcolor{green!10}\textit{achieve password trial limitation}
	\\
	\cellcolor{blue!10}$G_{5}$ &
	\cellcolor{blue!10}\textit{avoid guessing id and password} &
	\multicolumn{1}{l}{\cellcolor{green!10}$R_{6}$} &
	\cellcolor{green!10}\textit{achieve password policy}
	\\
	\cellcolor{blue!10}$G_{6}$ &
	\cellcolor{blue!10}\textit{avoid dictionary attack} &
	\multicolumn{1}{l}{\cellcolor{green!10}$R_{7}$} &
	\cellcolor{green!10}\textit{achieve password encryption}
	\\
	\cellcolor{blue!10}$G_{7}$ &
	\cellcolor{blue!10}\textit{avoid guess password} &
	\multicolumn{1}{l}{\cellcolor{green!10}$R_{8}$} &
	\cellcolor{green!10}\textit{achieve random id}
	\\
	\cellcolor{blue!10}$G_{8}$ &
	\cellcolor{blue!10}\textit{avoid guess id } &
	\multicolumn{1}{l}{\cellcolor{green!10}$R_{9}$} &
	\cellcolor{green!10}\textit{achieve CAPTCHA}
	\\
	\cellcolor{blue!10}$G_{9}$ &
	\cellcolor{blue!10}\textit{avoid brute forcing} &
	\multicolumn{1}{l}{\cellcolor{green!10}$R_{10}$} &
	\cellcolor{green!10}\textit{achieve complex pin}
	\\
	\cellcolor{blue!10}$G_{10}$ &
	\cellcolor{blue!10}\textit{avoid unauthorized transfer via debit card} &
	\multicolumn{1}{l}{\cellcolor{green!10}$R_{11}$} &
	\cellcolor{green!10}\textit{achieve access control}
	\\
	\cellcolor{blue!10}$G_{11}$ &
	\cellcolor{blue!10}\textit{maintain transfer network security} &
	\multicolumn{1}{l}{\cellcolor{green!10}$R_{12}$} &
	\cellcolor{green!10}\textit{achieve redundant server}
	\\
	\cellcolor{blue!10}$G_{12}$ &
	\cellcolor{blue!10}\textit{avoid hijack server} &
	\multicolumn{1}{l}{\cellcolor{green!10}\textit{}} &
	\cellcolor{green!10}\textit{}
	\\
	\cellcolor{blue!10}$G_{13}$ &
	\cellcolor{blue!10}\textit{maintain service availability} &
	\multicolumn{1}{l}{\cellcolor{green!10}\textit{}} &
	\cellcolor{green!10}\textit{}
	\bigstrut[b]\\
	\bottomrule[1.5pt]
\end{tabular}}%

\end{table*}

\begin{figure}[!h]
	\centering
	\centerline{\includegraphics[scale=0.525]{srm.pdf}}
	\caption{SRM of the OBS}
	\label{fig_srm}
\end{figure}

%\subsubsection{Describing Security Requirements}
%\label{prepas_description}



\textit{i.Goal-Based Fuzzy Grammar (GFG)}. A GFG is defined as a quintuple of $GR = (G, R, P, S, \mu)$ in which $G$ is a set of security goals, $R$ is a set of security requirements, $P$ is a set of fuzzy derivation rules and $\mu$ denotes the membership function of derivation. $S$ represents the top-level security goal of the system. For $OBS$, $G=\{G_{1},...,G_{13}\}$, $R=\{R_{1},...,R_{12}\}$, $P=\{P_{1},...,P_{20}\}$ and \textit{S=``maintain [OBS] [security]''}. Due to its fuzziness, $GFG$ can properly capture partiality in SRM of a software. 

The elements of $P \in GR $ are expression of form given in \ref{Eq_rule} where `d' is the degree of contribution of a sub-goal `w' to satisfaction of a goal `r'. If $r_{1},..., r_{n}$ are fuzzy statements in $(G \cup R)^*$ and $r_{1} \rightarrow r_{2} \rightarrow ... \rightarrow r_{n}$, then we call this chain as goal derivation chain under the employed $GFG$. 

\begin{align}
\label{Eq_rule}
& \mu(r\rightarrow w)=d , d\in[0,1] \text{ or } \mu(r,w)=d
\end{align}

\begin{table*}[t]
	\caption{Derivation rules for SRM of the OBS.}
	\label{table_rules}
	\centering
	% Table generated by Excel2LaTeX from sheet 'Sheet1'
%\Huge
%\huge
%\LARGE
%\Large
%\large
%\normalsize (default)
%\small
%\footnotesize
%\scriptsize
%\tiny
\normalsize\rowcolors{1}{}{lightgray}
\resizebox {0.75\textwidth }{!}{
% Table generated by Excel2LaTeX from sheet 'Sheet1'
% Table generated by Excel2LaTeX from sheet 'Sheet1'
\begin{tabular}{llll}
	\toprule[1.5pt]
	\textbf{Rule}  &
	\textbf{Membership Value\quad} &
	\textbf{Rule} &
	\textbf{Membership Value}
	\bigstrut\\
	\midrule
	\textit{$P_{1}:\phantom{s} S \rightarrow G{1}G_{13}$ } &
	\phantom{ss}\textit{$ 0.95 $} &
	\textit{$P_{11}:\phantom{s} G_{4} \rightarrow R_{2}R_{3}$} &
	\phantom{ss}\textit{$ 0.75 $}
	\bigstrut[t]\\
	\textit{$P_{2}:\phantom{s} G_{1} \rightarrow G_{2}G_{10}G_{12}$} &
	\phantom{ss}\textit{$ 0.95 $} &
	\textit{$P_{12}:\phantom{s} G_{4} \rightarrow R_{4}$} &
	\phantom{ss}\textit{$ 0.9 $}
	\\
	\textit{$P_{3}:\phantom{s} G_{13} \rightarrow R_{12}$} &
	\phantom{ss}\textit{$ 0.9 $} &
	\textit{$P_{13}:\phantom{s} G_{6} \rightarrow G_{7}$} &
	\phantom{ss}\textit{$ 0.6 $}
	\\
	\textit{$P_{4}:\phantom{s} G_{2} \rightarrow R_{1}G_{3}G_{5}$} &
	\phantom{ss}\textit{$ 0.85 $} &
	\textit{$P_{14}:\phantom{s} G_{6} \rightarrow G_{8}$} &
	\phantom{ss}\textit{$ 0.6 $}
	\\
	\textit{$P_{5}:\phantom{s} G_{10} \rightarrow G_{11}$} &
	\phantom{ss}\textit{$ 0.9 $} &
	\textit{$P_{15}:\phantom{s} G{9} \rightarrow G_{7}$} &
	\phantom{ss}\textit{$ 0.65 $}
	\\
	\textit{$P_{6}:\phantom{s} G_{10} \rightarrow R_{10}$} &
	\phantom{ss}\textit{$ 0.4 $} &
	\textit{$P_{16}:\phantom{s} G_{9} \rightarrow G_{8}$} &
	\phantom{ss}\textit{$ 0.6 $}
	\\
	\textit{$P_{7}:\phantom{s} G_{12} \rightarrow R_{11}$} &
	\phantom{ss}\textit{$ 0.9 $} &
	\textit{$P_{17}:\phantom{s} G_{7} \rightarrow R_{5}$} &
	\phantom{ss}\textit{$ 0.7 $}
	\\
	\textit{$P_{8}:\phantom{s} G_{3} \rightarrow G_{4}$} &
	\phantom{ss}\textit{$ 0.85 $} &
	\textit{$P_{18}:\phantom{s} G_{7} \rightarrow R_{6}$} &
	\phantom{ss}\textit{$ 0.8 $}
	\\
	\textit{$P_{9}:\phantom{s} G_{5} \rightarrow G_{6}G_{9}$} &
	\phantom{ss}\textit{$ 0.9 $} &
	\textit{$P_{19}:\phantom{s} G_{7} \rightarrow R_{7}$} &
	\phantom{ss}\textit{$ 0.9 $}
	\\
	\textit{$P_{10}:\phantom{s} G_{11} \rightarrow R_{9}$} &
	\phantom{ss}\textit{$ 0.8 $} &
	\textit{$P_{20}:\phantom{s} G_{8} \rightarrow R_{8}$} &
	\phantom{ss}\textit{$ 0.6 $}
	\bigstrut[b]\\
	\bottomrule[1.5pt]
\end{tabular}}%
\end{table*}

\textit{ii. Extracting Derivation Rules}. The employed description technique, constructs a GFG for a given SRM and identifies the derivation rules~\cite{mougouei2013fuzzy,mougouei2013fuzzyBased,mougouei2013fuzzy1}. The degree to which the successive of a rule contributes to satisfaction of its predecessor, will specify its membership value. This value will be determined by the membership function µ of GFG \cite{mougouei2013fuzzyBased}. The extracted derivation rules for SRM of the OBS, and their corresponding membership values are listed in Table \ref{table_rules}. 

%The derivation rule $P_{11} (G_{1} \rightarrow R_{2}R_{3})$ in Table \ref{table_rules}, firstly explains that the security goal $G_{1}$ derives security requirements $R_{2}$ and $R_{3}$. Secondly specifies the AND relation between $R_{2}$ and $R_{3}$. In other words, to satisfy the security goal $G_{1}$, it is required to satisfy both $R_{2}$ and $R_{3}$.

\subsection{Risk Analysis}
\label{prepas_risk} 

During the risk analysis, the cost and technical-ability of security requirements will be identified. 

\textit{Cost of Implementation}. Owing to budget limitations we need to care for cost of implementation while selecting and prioritizing security requirements \cite{karlsson1997cost}. Table \ref{table_risk} has listed the cost and values for security requirements in the SRM of the OBS. Cost of implementation is a real number in $[1,100]$ \cite{mougouei2012measuring,mougouei2012evaluating}.  

\begin{table*}[t]
	\caption{Cost and Technical-ability of the OBS Security Requirements.}
	\label{table_risk}
	\centering
	% Table generated by Excel2LaTeX from sheet 'Sheet1'
%\Huge
%\huge
%\LARGE
%\Large
%\large
%\normalsize (default)
%\small
%\footnotesize
%\scriptsize
%\tiny
\normalsize\rowcolors{1}{}{lightgray}
\resizebox {0.75\textwidth }{!}{
% Table generated by Excel2LaTeX from sheet 'Sheet1'
% Table generated by Excel2LaTeX from sheet 'Sheet1'
% Table generated by Excel2LaTeX from sheet 'Sheet1'
\begin{tabular}{lllllllllllll}
	\toprule[1.5pt]
	\textbf{Requirement} &
	\textit{$R_1$} &
	\textit{$R_2$} &
	\textit{$R_3$} &
	\textit{$R_4$} &
	\textit{$R_5$} &
	\textit{$R_6$} &
	\textit{$R_7$} &
	\textit{$R_8$} &
	\textit{$R_9$} &
	\textit{$R_{10}$} &
	\textit{$R_{11}$} &
	\textit{$R_{12}$}
	\bigstrut\\
	\hline
	\textbf{Cost} &
	\textit{$ 0.50 $} &
	\textit{$0.70$} &
	\textit{$0.70$} &
	\textit{$ 0.30 $} &
	\textit{$ 0.05 $} &
	\textit{$ 0.50 $} &
	\textit{$ 0.20 $} &
	\textit{$ 0.01 $} &
	\textit{$ 0.60 $} &
	\textit{$ 0.10 $} &
	\textit{$0.70$} &
	\textit{$ 1.00 $}
	\bigstrut[t]\\
	\textbf{Technical-ability} &
	\textit{$ 1.00 $} &
	\textit{$ 0.20 $} &
	\textit{$ 0.10 $} &
	\textit{$ 0.90 $} &
	\textit{$ 1.00 $} &
	\textit{$ 0.30 $} &
	\textit{$ 0.20 $} &
	\textit{$ 1.00 $} &
	\textit{$ 0.10 $} &
	\textit{$ 1.00 $} &
	\textit{$ 0.20 $} &
	\textit{$ 0.20 $}
	\bigstrut[b]\\
	\bottomrule[1.5pt]
\end{tabular}}%
\end{table*}

\textit{Technical-Ability}. Technical-ability is a real number in $[0,1]$ which reflects the ease of implementation for each requirement. Technical-ability of security requirements of the OBS are computed based on (\ref{Eq_technical}) and listed in Table \ref{table_risk}.  

\begin{align}
\label{Eq_technical}
& Technical\text{-}Ability=\frac{1}{\textit{Technical Complexity of Requirement}}
\end{align}
\section{Prioritization and Selection Process}
\label{pas}

The PAS process starts with preprocessing FIS inputs. Preprocessing includes construction of SRL and calculation of impacts for security requirements in the SRL. Subsequently, impacts, costs and technical-abilities will be fuzzified \cite{bede2013fuzzy} to serve as the inputs of the FIS. A Mamdani-type \cite{mamdani1974application} fuzzy inference system then specifies the priorities of security requirements. Prioritization can be performed with special focus on satisfaction of a security goal. Finally, prioritized requirements will be partially selected (when tolerated) through RELAX-ation of their satisfaction conditions. To perform RELAX-ation, we need to obtain Required Degree of Satisfaction (RDS) for each security requirement through deffuzification of priority values. %The RELAX-ed requirements will then be listed in the optimal SRL of the system. In Section 3.1 we introduce data preprocessing activities. Prioritization activities are discussed in Section 3.2 and Section 3.3 explains partial selection of security requirements. 

%\begin{figure*}[!htbp]
%	\centering
%	\centerline{\fbox{\includegraphics[scale=0.7]{figs/code1}}}
%	\caption{Java Implementation of Data Preprocessing }
%	\label{fig_code1}
%\end{figure*}
%\vspace{1em}
%\begin{algorithm*}
%	\small
%	\caption{Computing the strengths of value dependencies}
%	\label{alg_strength}
%%	\begin{multicols}{1}
%		\begin{algorithmic}[1]
%			\STATE \textbf{class} dataPreprocessing $\{$
%			\STATE \quad dataPreprocessing $\{$ \textbackslash\textbackslash constructor
%			\STATE \quad\quad calculateImpact();
%		    \STATE \quad\quad buildRequirementList();
%		    \STATE \quad $\}$
%			\STATE $\}$
%			
%			\STATE \quad dataPreprocessing $\{$ \textbackslash\textbackslash constructor
%			\STATE \quad\quad calculateImpact();
%			\STATE \quad\quad buildRequirementList();
%			\STATE \quad $\}$
%			
%			\ENSURE $\rho^{+\infty}, \rho^{-\infty}$
%			\FOR{\textbf{each} $r_i \in R$}
%			\FOR{\textbf{each} $r_j \in R$}
%			\STATE $\rho_{i,j}^{+\infty} \leftarrow -\infty$ 
%			\STATE $\rho_{i,j}^{-\infty} \leftarrow -\infty$ 
%			\ENDFOR
%			\ENDFOR
%			\FOR{\textbf{each} $r_i \in R$}
%			\STATE $\rho_{i,i}^{+\infty} \leftarrow 0$
%			\STATE $\rho_{i,i}^{-\infty} \leftarrow 0$
%			\ENDFOR
%			\FOR{\textbf{each} $r_i \in R$}
%			\FOR{\textbf{each} $r_j \in R$}
%			\IF{$\sigma_{i,j} = +$}
%			\STATE $\rho_{i,j}^{+\infty} \leftarrow \rho_{i,j}$
%			\ELSIF{$\sigma_{i,j} = -$}
%			\STATE $\rho_{i,j}^{-\infty} \leftarrow \rho_{i,j}$
%			\ENDIF
%			\ENDFOR
%			\ENDFOR
%			\FOR{\textbf{each} $r_k \in R$}
%			\FOR{\textbf{each} $r_i \in R$}
%			\FOR{\textbf{each} $r_j \in R$}
%			\IF{$min(\rho_{i,k}^{+\infty}, \rho_{k,j}^{+\infty}) > \rho_{i,j}^{+\infty}$}
%			\STATE $\rho_{i,j}^{+\infty} \leftarrow  min(\rho_{i,k}^{+\infty}, \rho_{k,j}^{+\infty})$
%			\ENDIF
%			\IF{$min(\rho_{i,k}^{-\infty}, \rho_{k,j}^{-\infty}) > \rho_{i,j}^{+\infty}$}
%			\STATE $\rho_{i,j}^{+\infty} \leftarrow  min(\rho_{i,k}^{-\infty}, \rho_{k,j}^{-\infty})$
%			\ENDIF
%			\IF{$min(\rho_{i,k}^{+\infty}, \rho_{k,j}^{-\infty}) > \rho_{i,j}^{-\infty}$}
%			\STATE $\rho_{i,j}^{-\infty} \leftarrow  min(\rho_{i,k}^{+\infty}, \rho_{k,j}^{-\infty})$
%			\ENDIF
%			\IF{$min(\rho_{i,k}^{-\infty}, \rho_{k,j}^{+\infty}) > \rho_{i,j}^{-\infty}$}
%			\STATE $\rho_{i,j}^{-\infty} \leftarrow  min(\rho_{i,k}^{-\infty}, \rho_{k,j}^{+\infty})$
%			\ENDIF
%			\ENDFOR
%			\ENDFOR
%			\ENDFOR
%		\end{algorithmic}
%%	\end{multicols}
%\end{algorithm*}


\subsection{Data Preprocessing}
\label{pas_pre}

%Data preprocessing includes construction of SRL and calculation of impact values. 

\setlength{\fboxsep}{6pt}%
\setlength{\fboxrule}{0.5pt}%


%\subsubsection{Construction of Security Requirement List }
%\label{pas_pre_cons}
Data preprocessing includes construction of SRL and calculation of impact values. Let $GR = (G, R, P, S, \mu)$ be the GFG of a SRM. For each security goal $g \in G$, $SRL[g]$ contains security requirements which contribute to satisfaction of goal g. $SRL[g]$ is constructed for every goal \textit{g} in SRM of the system. This allows for goal-oriented prioritization of security requirements with special focus on satisfaction of individual security goals. For each security requirement $x$ in $SRL[g]$ the degree to which $x$ contributes to satisfaction of the goal $g$ is referred to as the impact of $x$ on $g$ and computed by taking maximum ($ \oplus $) over membership degree of all derivations paths that can derive $x$ from $g$. Membership of each path is computed by taking minimum ($ \otimes $) over membership degrees of all derivations rules on the path. Impacts for security requirements of the OBS are listed in Table \ref{table_impact}. 

%as follows. Firstly, the membership values of the derivation rules on derivation chain of \textit{x} will be computed based on (\ref{Eq_imp}). Then, the fuzzy membership values will be calculated for each derivation chain through taking minimum over all membership value of derivation rules on the derivation chain of \textit{x}. Finally, impact will be calculated through taking supremum over all membership values of the derivation chains which can derive \textit{x}. $ \oplus $ and $ \otimes $ denote fuzzy OR (maximum) and AND (minimum) operators respectively. Impacts for security requirements of the OBS are listed in Table \ref{table_impact}.

\begin{align}
\label{Eq_imp}
& DC_{g} (x) = \mu_{g}(x)= \oplus(\mu(g,r_1)\otimes\mu(r_1,r_2)\otimes ... \otimes\mu(r_n,x))
\end{align}

%The t-norm sign $ \oplus $ and a t-conorm sign $ \otimes $ denote fuzzy OR (maximum) and AND (minimum) operators respectively based on the Zadeh’s definition in \cite{zadeh1965fuzzy}. $\mu_{g}(x)$ specifies the strongest degree for contribution of \textit{x} to satisfaction of the goal \textit{g}. Impacts for security requirements of the OBS are listed in Table \ref{table_impact}.   %As one example, $\mu_S(R_7)$ for $R_7$ in SRM of the OBS is calculated for two derivation chains: 1) $S\rightarrow G_1\rightarrow G_2\rightarrow G_5\rightarrow G_9\rightarrow G_7\rightarrow R_7$ and 2) $S\rightarrow G_1\rightarrow G_2\rightarrow G_5\rightarrow G_6\rightarrow G_7\rightarrow R_7$, as follows: $\mu_S (X=R_7)= (0.95 \otimes 0.95 \otimes0 .85 \otimes0 .9 \otimes 0 .65 \otimes 0.9) \oplus (0.95 \otimes0 .95 \otimes0 .85 \otimes 0.9 \otimes0.6 \otimes 0.9) = 0.65$. Impact values for security requirements in SRM of OBS are calculated by function ``calculateImpact" in Figure \ref{fig_code1} and listed in Table \ref{table_impact}.  

\begin{table*}[t]
	\caption{Impact of Security Requirements in the SRM of the OBS.}
	\label{table_impact}
	\centering
	% Table generated by Excel2LaTeX from sheet 'Sheet11'
\rowcolors{1}{}{lightgray}
%| L{50mm} | C{300mm} |
\resizebox {0.75\textwidth }{!}{\begin{tabular}{lllllllllllll}
\Xhline{4\arrayrulewidth}
Goal &
  $\mu(R_1)$ &
  $\mu(R_2)$ &
  $\mu(R_3)$ &
  $\mu(R_4)$ &
  $\mu(R_5)$ &
  $\mu(R_6)$ &
  $\mu(R_7)$ &
  $\mu(R_8)$ &
  $\mu(R_9)$ &
 $ \mu(R_{10})$ &
  $\mu(R_{11})$ &
  $\mu(R_{12}$)
  \\
\Xhline{4\arrayrulewidth}
S &
  $0.85$ &
  $0.75$ &
  $0.75$ &
  $0.85$ &
  $0.65$ &
  $0.65$ &
  $0.65$ &
  $0.60$ &
  $0.80$ &
  $0.40$ &
  $0.90$ &
  $0.90$
  \bigstrut\\
$G_1$ &
  $0.85$ &
  $0.75$ &
  $0.75$ &
  $0.85$ &
  $0.65$ &
  $0.65$ &
  $0.65$ &
  $0.60$ &
  $0.80$ &
  $0.40$ &
  $0.90$ &
  $0.00$
  \\
$G_2$ &
  $0.85$ &
  $0.75$ &
  $0.75$ &
  $0.85$ &
  $0.65$ &
  $0.65$ &
  $0.65$ &
  $0.60$ &
  $0.00$ &
  $0.00$ &
  $0.00$ &
  $0.00$
  \\
$G_3$ &
  $0.00$ &
  $0.75$ &
  $0.75$ &
  $0.85$ &
  $0.00$ &
  $0.00$ &
  $0.00$ &
  $0.00$ &
  $0.00$ &
  $0.00$ &
  $0.00$ &
  $0.00$
  \\
$G_4$ &
  $0.00$ &
  $0.75$ &
  $0.75$ &
  $0.90$ &
  $0.00$ &
  $0.00$ &
  $0.00$ &
  $0.00$ &
  $0.00$ &
  $0.00$ &
  $0.00$ &
  $0.00$
  \\
$G_5$ &
  $0.00$ &
  $0.00$ &
  $0.00$ &
  $0.00$ &
  $0.65$ &
  $0.65$ &
  $0.65$ &
  $0.60$ &
  $0.00$ &
  $0.00$ &
  $0.00$ &
  $0.00$
  \\
$G_6$ &
  $0.00$ &
  $0.00$ &
  $0.00$ &
  $0.00$ &
  $0.60$ &
  $0.60$ &
  $0.60$ &
  $0.60$ &
  $0.00$ &
  $0.00$ &
  $0.00$ &
  $0.00$
  \\
$G_7$ &
  $0.00$ &
  $0.00$ &
  $0.00$ &
  $0.00$ &
  $0.70$ &
  $0.80$ &
  $0.90$ &
  $0.00$ &
  $0.00$ &
  $0.00$ &
  $0.00$ &
  $0.00$
  \\
  $G_8$ &
  $0.00$ &
  $0.00$ &
  $0.00$ &
  $0.00$ &
  $0.00$ &
  $0.00$ &
  $0.00$ &
  $0.60$ &
  $0.00$ &
  $0.00$ &
  $0.00$ &
  $0.00$
  \\
$G_9$ &
  $0.00$ &
  $0.00$ &
  $0.00$ &
  $0.00$ &
  $0.65$ &
  $0.65$ &
  $0.65$ &
  $0.65$ &
  $0.00$ &
  $0.00$ &
  $0.00$ &
  $0.00$
  \\
$G_{10}$ &
  $0.00$ &
  $0.00$ &
  $0.00$ &
  $0.00$ &
  $0.00$ &
  $0.00$ &
  $0.00$ &
  $0.00$ &
  $0.80$ &
  $0.40$ &
  $0.00$ &
  $0.00$
  \\
$G_{11}$ &
  $0.00$ &
  $0.00$ &
  $0.00$ &
  $0.00$ &
  $0.00$ &
  $0.00$ &
  $0.00$ &
  $0.00$ &
  $0.80$ &
  $0.00$ &
  $0.00$ &
  $0.00$
  \\
$G_{12}$ &
  $0.00$ &
  $0.00$ &
  $0.00$ &
  $0.00$ &
  $0.00$ &
  $0.00$ &
  $0.00$ &
  $0.00$ &
  $0.00$ &
  $0.00$ &
  $0.90$ &
  $0.00$
  \\
$G_{13}$ &
  $0.00$ &
  $0.00$ &
  $0.00$ &
  $0.00$ &
  $0.00$ &
  $0.00$ &
  $0.00$ &
  $0.00$ &
  $0.00$ &
  $0.00$ &
  $0.00$ &
  $0.90$
  \\
\Xhline{4\arrayrulewidth}
\end{tabular}}%

\end{table*}

\subsection{Prioritization}
\label{pas_prior}

Prioritization starts with fuzzification of FIS inputs (impact, cost, and technical-ability of security requirements). As depicted in Figure \ref{fig_paps}, the fuzzified values will serve as the inputs for the FIS. FIS then infers the fuzzified priority values of requirements based on fuzzy-rule-base of the PAPS framework. Priority of each requirement specifies the required satisfaction level of that requirement. 

%\begin{figure}[h!]
%	\centering
%	\centerline{\fbox{\includegraphics[scale=0.6]{figs/code2}}}
%	\caption{Java Implementation of Prioritization and Selection of Security Requirements}
%	\label{fig_code2}
%\end{figure}

%\begin{table*}[t]
%	\caption{Membership Functions for FIS Inputs/ Output.}
%	\label{table_fis}
%	\centering
%	\input{tables/table_fis}
%\end{table*}
\subsection{Fuzzification}
\label{pas_prior_fuzz}

FIS inputs (impact, cost and technical-ability) are categorized under three fuzzy categories: Low (L), Medium (M) and High (H). Consequently, three membership functions are defined for each input and its corresponding categories. We use a semi-trapezoids shape for membership functions (Figure \ref{fig_inference}). Hence, four diverse points are required to define each membership function as given by (\ref{Eq_mem1})-(\ref{Eq_mem3}). We have use a combination of Fuzzy Control Language (FCL) and jFuzzyLogic \cite{lewis1998programming} to implement the membership functions.%, each membership function is computed based on four specific points as $Y_{il}:\{(-0.35,1),(0,1),(0.1,1),(0.45,1)\}$, $Y_{im}=\{(0.2,1),(0.45,1),(0.55,1),(0.8,1)\}$, $Y_{ih}=\{(0.6,1),(0.9,1),(1,1),(1.3,1)\}$ where $ i \in \{impact,cost,technical\text{-}ability\}$, `l',`m',`h' denote \textit{low}, \textit{medium}, \textit{high}, respectively. Similarly, for the membership function of the priority we have $Y_{pl}:\{(-0.35,1),(0,1),(0.1,1),(0.45,1)\}$, $Y_{pm}=\{(0.2,1),(0.45,1),(0.55,1),(0.8,1)\}$, $Y_{ph}=\{(0.6,1),(0.9,1),(1,1),(1.3,1)\}$ where `p' stands for priority and  `o', `w', `n', `s' denote priority values \textit{optional}, \textit{weak}, \textit{normal}, \textit{strong} respectively. We have employed a combination of Fuzzy Control Language (FCL) \cite{zadeh1965fuzzy} and jFuzzyLogic \cite{lewis1998programming} to implement the membership functions.

%\begin{align}
%\label{Eq_mem}
%& \forall i \in \{impact,cost,technical-ability\} ,\\ \nonumber
%& \forall v \in \{low,medium,high\} : \exists (x_0,x_1,x_2,x_3),\\ \nonumber
%& \mu(x_0)=\mu(x_3)=0,\mu(x_1)=\mu(x_2)=1 \rightarrow \\ \nonumber
%& \mu_{iv}(x) = \max (\min(\frac{x-x_0}{x_1-x_0},1,\frac{x_3-x}{x_3-x_2}),0).
%\end{align}

\begin{align}
	\label{Eq_mem1}
	& \mu_{iv}(x) = \max (\min(\frac{x-x_0}{x_1-x_0},1,\frac{x_3-x}{x_3-x_2}),0)\\
    \label{Eq_mem2}
   	& \mu(x_0)=\mu(x_3)=0,\mu(x_1)=\mu(x_2)=1 \\ 
	\label{Eq_mem3}
	& i \in \{impact,cost,technical\text{-}ability\} ,v \in \{low,medium,high\}
\end{align}

%For each security goal `g' and requirement `r' Pseudocode 2, fuzzifies the $SRL[g][r]$. Impact, $SRL[g][r].cost$ and $SRL[g][r].technicalAbility$ using membership functions of Table~\ref{table_fis}. 

%\begin{figure}[h!]
%	\centering
%	\centerline{\includegraphics[scale=0.6]{figs/membership2}}
%	\caption{Example of a membership function〖$\mu_{iv}$}
%	\label{fig_mem}
%\end{figure}

\begin{figure}[h!]
	\centering
	\centerline{\fbox{\includegraphics[scale=0.5]{rulebase}}}
	\caption{Fuzzy Rules Implemented in FCL}
	\label{fig_rulebase}
\end{figure}
\subsection{Fuzzy Inference}
\label{pas_prior_inference}

Priorities of security requirements are inferred by a Mamdani-Type \cite{mamdani1974application} FIS using the fuzzy rule-base of Figure~\ref{fig_rulebase}. In this regard, linguistic priorities Optional (O), Weak (W), Normal (N), or Strong (S) will be assigned to security requirements. Fuzzy rules of course can be tailored to the organizational and technical concerns of stakeholders. Priorities of the OBS security requirements are listed in Table~\ref{table_rds}. Each security requirement in Table~\ref{table_rds} is assigned $ 14 $ priority values each of which computed with regard to a specific security goal to assist goal-based prioritization and selection of requirements. Goal-based PAS provides structured arguments to the stakeholders. For instance, requirement ``$R_7$" in SRL of the OBS is strongly (weakly) required for satisfaction of the goal ``$G_2$" (``$G_6$"). %Consequently other security requirements (e.g. $R_5$) might be required if satisfaction of $ G_6 $ is concerned. Figure \ref{fig_mem} demonstrates the membership functions of FIS inputs (impact, cost, technical-ability) and output (priority) for prioritization of $ R_7 $. $ R_7 $ is prioritized with respect to the top-level security goal $ S $. Membership values for FIS variables are depicted by vertical lines in their corresponding membership functions.  

\begin{figure}[!htbp]
	\centering
	\centerline{\includegraphics[scale=0.55]{inference}}
	\caption{Fuzzy Inference for $R_7$ with respect to top-level security goal S}
	\label{fig_inference}
\end{figure}
\vspace{-1em}
%\begin{table*}[t]
%	\caption{Priority Values Inferred by FIS for Security Requirements of OBS.}
%	\label{table_priority}
%	\centering
%	\input{tables/table_priority}
%\end{table*}

\subsection{Partial Selection}
\label{pas_partial}

It is happening quite often in software systems that a security requirement cannot be either fully implemented or ignored. In such cases, security requirements may be tolerated \cite{whittle2010relax,cheng2009goal} to be partially satisfied (selected). Partial satisfaction of a security requirement can be explicitly addressed through RELAX-ing \cite{cheng2009goal} its satisfaction condition (level of satisfaction). For instance, implementation of a complex password policy in a software system increases the level of security on one hand and reduces the usability of the system \cite{adams1999users} on the other hand. Alternatively, implementing a less complex password policy may be tolerated to maintain usability of the system. In this case, the requirement will be partially selected through RELAX-ing its satisfaction condition. %Consequently the RELAX-ed requirement will be partially included in the optimal SRL. Partial selection in the PAPS framework, includes defuzzification and RELAX-ation of security requirements.

\begin{table*}[t]
	\caption{RDS (Linguistic Priority) of the Security Requirements of the OBS.}
	\label{table_rds}
	\centering
	% Table generated by Excel2LaTeX from sheet 'Sheet14'
\rowcolors{1}{}{lightgray}
\resizebox {0.9\textwidth }{!}{% Table generated by Excel2LaTeX from sheet 'Sheet14'
\begin{tabular}{lllllllllllll}
\Xhline{4\arrayrulewidth}
\multirow{2}[4]{*}{Goal} &
  \multicolumn{12}{c}{RDS Values}
  \bigstrut\\
\cline{2-13} &
  $R_1$ &
  $R_2$ &
  $R_3$ &
  $R_4$ &
  $R_5$ &
  $R_6$ &
  $R_7$ &
  $R_8$ &
  $R_9$ &
  $R_{10}$ &
  $R_{11}$ &
  $R_{12}$
  \bigstrut\\
\Xhline{4\arrayrulewidth}
S &
  0.82 (S) &
  0.25 (W) &
  0.25 (W) &
  0.82 (S) &
  0.59 (N) &
  0.36 (W) &
  0.42 (N) &
  0.55 (N) &
  0.35 (W) &
  0.55 (N) &
  0.25 (W) &
  0.13 (O)
  \bigstrut\\
$G_1$ &
  0.82 (S) &
  0.25 (W) &
  0.25 (W) &
  0.82 (S) &
  0.59 (N) &
  0.36 (W) &
  0.42 (N) &
  0.55 (N) &
  0.35 (W) &
  0.55 (N) &
  0.25 (W) &
  -
  \\
$G_2$ &
  0.82 (S) &
  0.25 (W) &
  0.25 (W) &
  0.82 (S) &
  0.59 (N) &
  0.36 (W) &
  0.42 (N) &
  0.55 (N) &
  - &
  - &
  - &
  -
  \\
$G_3$ &
  - &
  0.25 (W) &
  0.25 (W) &
  0.82 (S) &
  - &
  - &
  - &
  - &
  - &
  - &
  - &
  -
  \\
$G_4$ &
  - &
  0.25 (W) &
  0.25 (W) &
  0.82 (S) &
  - &
  - &
  - &
  - &
  - &
  - &
  - &
  -
  \\
$G_5$ &
  - &
  - &
  - &
  - &
  0.59 (N) &
  0.36 (W) &
  0.42 (N) &
  0.55 (N) &
  - &
  - &
  - &
  -
  \\
$G_6$ &
  - &
  - &
  - &
  - &
  0.55 (N) &
  0.24 (O) &
  0.35 (W) &
  0.55 (N) &
  - &
  - &
  - &
  -
  \\
$G_7$ &
  - &
  - &
  - &
  - &
  0.64 (N) &
  0.6 (N)  &
  0.55 (N) &
  - &
  - &
  - &
  - &
  -
  \\
$G_8$ &
  - &
  - &
  - &
  - &
  - &
  - &
  - &
  0.55 (N) &
  - &
  - &
  - &
  -
  \\
$G_9$ &
  - &
  - &
  - &
  - &
  0.59 (N) &
  0.36 (W) &
  0.42 (N) &
  0.59 (N) &
  - &
  - &
  - &
  -
  \\
$G_{10}$ &
  - &
  - &
  - &
  - &
  - &
  - &
  - &
  - &
  0.35 (W) &
  0.55 (N) &
  - &
  -
  \\
$G_{11}$ &
  - &
  - &
  - &
  - &
  - &
  - &
  - &
  - &
  0.35 (W) &
  - &
  - &
  -
  \\
$G_{12}$ &
  - &
  - &
  - &
  - &
  - &
  - &
  - &
  - &
  - &
  - &
  0.25 (W) &
  -
  \\
$G_{13}$ &
  - &
  - &
  - &
  - &
  - &
  - &
  - &
  - &
  - &
  - &
  - &
  0.13 (O)
  \bigstrut\\
\Xhline{4\arrayrulewidth}
\end{tabular}}%
%

\end{table*}

\subsubsection{Defuzzification}
\label{partial_def}

Our employed RELAX-ation technique\cite{mougouei2013goal,mougouei2015partial} needs crisp values to specify the required level of satisfaction for RELAX-ed security requirements. Hence, defuzzification is required to map the linguistic priority values of requirements into their corresponding crisp values. To do so, we use the Center of Gravity (COG) \cite{van2006fast} formula given by (\ref{Eq_rds}) and the membership function of the priority (Figure~\ref{fig_inference}) to deffuzify the priority values. $ x $ in this equation denotes crisp priority values on the $x$ axis of the priority graph in Figure~\ref{fig_inference}. Also $\mu_{x}$ denotes the membership function of priorities. 

\begin{align}
\label{Eq_rds}
& \text{Defuzzified Priority}=\frac{\int_{0}^{1}\mu_{x}\times x\times d_{x}}{\int_{0}^{1}\mu_{x}\times d_{x}}
\end{align}
\vspace{-3em}
\begin{table}[!htbp]
	\caption{RELAX-ed Security Requirements of OBS.}
	\label{table_relaxed}
	\centering
	% Table generated by Excel2LaTeX from sheet 'Sheet15'
\rowcolors{1}{}{lightgray}
\resizebox {0.9\textwidth }{!}{\begin{tabular}{ll}
\Xhline{4\arrayrulewidth}
Requirement &
  Sample RELAX-ed Value
  \bigstrut\\
\Xhline{4\arrayrulewidth}
$R_1$: achieve request transaction code &
  [expiry rate] as close as possible to $0.82 \times OV_1$
  \bigstrut\\
$R_2$: achieve latency examination &
  [examination delay] as close as possible to  $0.25 \times OV_2$
  \\
$R_3$: achieve one-time pad &
  [randomness] as close as possible to $0.25 \times  OV_3$
  \\
$R_4$: achieve SSL &
  [entropy] as close as possible to $0.82 \times  OV_4$
  \\
$R_5$: achieve password trial limitation &
  [trial delay] as close as possible to $0.59 \times OV_5$
  \\
$R_6$: achieve password policy &
  [complexity] as close as possible to $0.36 \times  OV_6$
  \\
$R_7$: achieve password encryption &
  [length of encryption key]as many bits as $0.42 \times  OV_7$
  \\
$R_8$: achieve random id &
  [randomness] as close as possible to $0.55 \times  OV_8$
  \\
$R_9$: achieve CAPTCHA &
  [level of distortion] as close as possible to $OV_9$
  \\
$R_{10}$: achieve complex pin &
  [complexity] as close as possible to $0.55 \times  OV_{10}$
  \\
$R_{11}$: achieve access control &
  [complexity] as close as possible to $0.25 \times  OV_{11}$
  \\
$R_{12}$: achieve redundant server &
  [number of servers] as close as possible to $0.13 \times  OV_{12}$
  \bigstrut\\
\Xhline{4\arrayrulewidth}
\end{tabular}}%

\end{table}
\vspace{-2em}
\subsubsection{RELAX-ation}
\label{partial_relax}

A requirement is RELAX-ed by relaxing its satisfaction condition (level). For this purpose we use a RELAX-ation technique proposed in~\cite{whittle2010relax}. Fuzzy semantic of the RELAX statements in the technique used properly captures partiality of security requirements \cite{mougouei2013goal}. Equations (\ref{eq_relax1})-(\ref{eq_relax3}) demonstrate this logic. 

In this equation, the aim is to maximize for each requirement $R_i$ the value of the membership function $\mu(V_i - (RDS_i \times OV_i))$ which is equivalent to finding a value $V_i$ for the satisfaction condition of $R_i$ as close as possible to the relaxed value $RDS_i\times OV_i$ where $RDS_i$ denotes the required degree of satisfaction for $R_i$ and $OV_i$ is the optimal value of the satisfaction condition in the absence of all resource limitations. 

%This is explained in terms of fuzzy semantic of the $R_6$: ``$AG ((\Delta (compexity) - (R_6.RDS × R_6.OptimalValue)) \in S)$'' where  $ \Delta(R_6) $ is the complexity of password policy. Also, $\mu$ is the membership function of the variance of the password complexity $\Delta(R_6)$ from the optimal complexity $R_6.RDS\times R_6.OptimalValue$. Furthermore, \textit{AG} denotes in all states and through all execution paths the condition of (\ref{label}) must hold. 

%\begin{align}
%\label{eq_relax_ex}
%&\mu(x)\rightarrow S, \mu(0)= 1\\ \nonumber
%& S= \{(x_i,\mu(x_i))|\mu(x_i)\in [0,1] , x \in \mathbb{R} \} \Rightarrow \\ \nonumber
%& \text{if }\Delta(R_6)= R_6.RDS\times R_6.OptimalValue \rightarrow \\ \nonumber
%&\mu(\Delta(R_6) - (R_6.RDS \times R_6.optimalValue))=1 \\ \nonumber
%\end{align}

\begin{align}
\label{eq_relax1}
Maximize \phantom{s} &\mu \big(V_i - (RDS_i \times OV_i) \big),& i=1,...,n\\
\label{eq_relax2}
Subject\phantom{s}to\phantom{s} &\mu(x_j)\in [0,1], &j=1,...,n \\ 
\label{eq_relax3}
& \mu(0)= 1\\ \nonumber
\end{align}

For instance consider RELAX-ing from security requirements of the OBS ``$R_6$: achieve password policy" where based on Table~\ref{table_rds}, $ SRL[S][R6] $= `weak’. When a less complex password encryption can be tolerated in $ SRL[S] $, satisfaction condition of $R_6$ can be RELAX-ed as : ``$ R_6 $: System shall achieve password policy [complexity] as close as possible to ($ RDS_6 \times OV_6 $). The membership function $\mu$ is maximizes ($\mu(x)=1$) when the variance of the password complexity is equal to $RDS_6 \times OV_6$. In a similar way, security requirements in $ SRL[S] $ of the OBS are RELAX-ed and listed in Table~\ref{table_relaxed}. It is important to note that ‘relaxed’ attributes are constructed based on the satisfaction metrics of security requirements. As depicted in Table~\ref{table_relaxed}, satisfaction metrics are placed in the brackets. For instance, satisfaction metric of $ R_6 $ is ``complexity" of the password policy while ``length" of encryption key is measures satisfaction of $ R_7 $.   


%\input{scalability}

\begin{comment}
\begin{figure}
\includegraphics[width=\linewidth]{figs/beyond_tss_lesion.pdf}
\caption[]{End-to-End runtime lesion study of the entire MNIST dataset and the FMA featurized music dataset. Each of DROP's contributions provides a runtime improvement.}
\label{fig:beyond_lesion}
\end{figure}
\end{comment}



\section{Conclusion}
\label{sec:conclusion}

Advanced data analytics techniques must scale to rising data volumes. 
DR techniques offer a powerful toolkit when processing these datasets, with PCA frequently outperforming popular techniques in exchange for high computational cost. 
In response, we propose DROP, a new dimensionality reduction optimizer. 
DROP combines progressive sampling, progress estimation, and online aggregation to identify high quality low dimensional bases via PCA without processing the entire dataset by balancing the runtime of downstream tasks and achieved dimensionality. 
Thus, DROP provides a first step in bridging the gap between quality and efficiency in end-to-end DR for downstream \red{analytics}. 

%We revisit canonical operators for time series dimensionality reduction and the measurement study of~\cite{keogh-study}, and show that PCA is more effective than popular alternatives in the data mining literature often by a margin of over $2\times$ on average on gold-standard time series benchmark data sets with respect to output data dimension. More surprisingly, we empirically demonstrate that a small number of samples are sufficient to accurately characterize directions of maximum variance and obtain a high-quality low-dimensional transformation.



\bibliographystyle{splncs}
\bibliography{ref}
\end{document}
