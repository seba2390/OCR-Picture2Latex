\section{Conclusions and Future Work}
\label{conclusion}

We presented a scalable framework referred to as the PAPS for goal-based prioritization and selection selection of security requirements which enhances precision of prioritization and selection by considering partiality of security. The proposed framework takes the security model of a software as the input and infers linguistic priorities of security requirements using a fuzzy inference system. Satisfaction conditions of security requirements then will be RELAX-ed when tolerated to partially selected (satisfy) those requirements. The defuzzified priority of a security requirement specifies its relaxed satisfaction level. Partial selection (satisfaction) of security requirements helps reduce the number of ignored security requirements which will ultimately reduce the number of unattended security threats. Validity of the PAPS framework was verified through studying an online banking system which served as our running example.  

The work is being continued by developing a tool based on the PAPS framework to assist automated prioritization and partial selection of security requirements. Using that tool in real world software project helps evaluate the PAPAS framework and its impact on the overall security of software in a real settings.  %The work can be extended in several directions. One is to apply the PAPS framework on an industrial software. Another possible direction would be to adopt different defuzzificaton techniques and evaluate the impact of each technique on the performance of the PAPS framework. The performance of the PAPAS framework can also be evaluated against different membership functions (other than the semi-trapezoids shape adopted in this paper) and/or various fuzzy rule bases.