\section{Introduction}
\label{introduction}
Security requirements (countermeasures) are to enhance security of software products. Nonetheless, due to the resources limitations it is hardly if ever possible to implement the entire set of identified security requirements for a software system \cite{loer2006integrated}. Consequently, an efficient prioritization and selection technique is required to find an optimal subset of security requirements for software products\cite{roy2012scalable,tonella2013interactive}. However, due to the following problems existing \textit{Prioritization And Selection} (PAS) techniques \cite{laurent2007towards,buyukozkan2005group,wiegers1999first,karlsson1997managing,karlsson1996software,karlsson1997cost,mohamed2008towards} have failed to be widely adopted by software practitioners \cite{ejnioui2012software,herrmann2008requirements}. The first problem i.e. \textit{Complexity} is that most of the existing PAS techniques are impractical to large number of requirements \cite{roy2012scalable,ejnioui2012software,berander2005requirements}. Adopted in prioritization of security requirements \cite{karlsson1998evaluation,mead2006identifying}, Analytic Hierarchy Process (AHP) has been the most promising \cite{karlsson1998evaluation} prioritization and selection technique \cite{ejnioui2012software} for years. However AHP suffers from high number of required comparisons \cite{karlsson1998evaluation}. Industrial studies have demonstrated that using AHP is not practical for more than $20$ requirements \cite{lehtola2004empirical}.There have been few reported techniques \cite{ejnioui2012software,berander2005requirements} to reduce the complexity of prioritization process but they have sacrificed the precision and consistency of the process \cite{ejnioui2012software}. 

The second problem is \textit{Imprecision} of PAS techniques \cite{ejnioui2012software} resulted by neglecting partiality of security and assuming that a security requirement can either be fully satisfied or ignored. But, ignoring (postponing) security requirements even if they are of lower priorities may leave some of security threats unattended and eventually result in security breaches in software systems. Moreover, ignoring a security requirement may also negatively influence the efficiency of other security requirements which rely on that requirement. As such, ignoring a security requirement may potentially cause cascading security vulnerabilities in software systems. 

To address the complexity and imprecision problems, we have proposed caring for partiality of security in a prioritization and selection process. In doing so, we have contributed a goal-based framework referred to as the \textit{Prioritization And Partial Selection} (PAPS) which enhances precision of a prioritization and selection process by allowing for partial selection (satisfaction)~\cite{mougouei2015partial} of security requirements when tolerated rather than ignoring those requirements altogether or postponing them to the future releases. The proposed framework helps reduce the number of ignored or postponed security requirements which will ultimately reduce the number of unattended security threats and mitigate the adverse impact of ignoring (postponing) security requirements. The PAPS framework is scalable to large number of requirements and allows for prioritization and selection of security requirements with respect to security goals. 

The proposed PAPS framework is composed of two major processes as demonstrated in Figure~\ref{fig_paps}. The first process referred to as \textit{Pre Prioritization and Selection} (Pre-PAS) includes modeling, description, and analysis of security requirements. The Pre-PAS process uses our previously developed security model known as \textit{Security Requirement Model} (SRM) \cite{mougouei2013goal} to capture partiality of security requirements (goals)~\cite{loer2006integrated}. The second process referred to as \textit{Prioritization and Selection} (PAS) process on the other hand, takes the formally described SRM of a software and its corresponding analysis results as the input and constructs the Security Requirement List (SRL) of that software. Then security requirements in SRL will be prioritized using a \textit{Fuzzy Inference System} (FIS) \cite{klir1995fuzzy}. Each security requirement in SRL contributes to satisfaction of at least one security goal. The FIS, infers linguistic priority values of the security requirements with respect to their impact, cost and technical-ability. We care for partiality in the optimal SRL through RELAX-ing \cite{whittle2010relax} satisfaction conditions of security requirements when partial satisfaction of those requirements is tolerated. Security requirements then will be RELAX-ed and included in the optimal SRL of software based on their fuzzy membership values.
 
The paper continues with an overview of our employed modeling and description technique (Section \ref{pas}). Then we introduce the PAPS framework (Section \ref{pas_pre}) and verify its validity through applying it to an Online Banking System (OBS)~\cite{mougouei2013goal}. The paper will be concluded in Section \ref{conclusion} with a summary of the work and some general remarks.  

\begin{figure}[h!]
	\centering
	\centerline{\includegraphics[scale=0.7]{paps.pdf}}
	\caption{Architecture of the Proposed PAPS Framework}
	\label{fig_paps}
\end{figure}
