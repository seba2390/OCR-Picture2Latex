\subsection{Partial Selection}
\label{pas_partial}

It is happening quite often in software systems that a security requirement cannot be either fully implemented or ignored. In such cases, security requirements may be tolerated \cite{whittle2010relax,cheng2009goal} to be partially satisfied (selected). Partial satisfaction of a security requirement can be explicitly addressed through RELAX-ing \cite{cheng2009goal} its satisfaction condition (level of satisfaction). For instance, implementation of a complex password policy in a software system increases the level of security on one hand and reduces the usability of the system \cite{adams1999users} on the other hand. Alternatively, implementing a less complex password policy may be tolerated to maintain usability of the system. In this case, the requirement will be partially selected through RELAX-ing its satisfaction condition. %Consequently the RELAX-ed requirement will be partially included in the optimal SRL. Partial selection in the PAPS framework, includes defuzzification and RELAX-ation of security requirements.

\begin{table*}[t]
	\caption{RDS (Linguistic Priority) of the Security Requirements of the OBS.}
	\label{table_rds}
	\centering
	% Table generated by Excel2LaTeX from sheet 'Sheet14'
\rowcolors{1}{}{lightgray}
\resizebox {0.9\textwidth }{!}{% Table generated by Excel2LaTeX from sheet 'Sheet14'
\begin{tabular}{lllllllllllll}
\Xhline{4\arrayrulewidth}
\multirow{2}[4]{*}{Goal} &
  \multicolumn{12}{c}{RDS Values}
  \bigstrut\\
\cline{2-13} &
  $R_1$ &
  $R_2$ &
  $R_3$ &
  $R_4$ &
  $R_5$ &
  $R_6$ &
  $R_7$ &
  $R_8$ &
  $R_9$ &
  $R_{10}$ &
  $R_{11}$ &
  $R_{12}$
  \bigstrut\\
\Xhline{4\arrayrulewidth}
S &
  0.82 (S) &
  0.25 (W) &
  0.25 (W) &
  0.82 (S) &
  0.59 (N) &
  0.36 (W) &
  0.42 (N) &
  0.55 (N) &
  0.35 (W) &
  0.55 (N) &
  0.25 (W) &
  0.13 (O)
  \bigstrut\\
$G_1$ &
  0.82 (S) &
  0.25 (W) &
  0.25 (W) &
  0.82 (S) &
  0.59 (N) &
  0.36 (W) &
  0.42 (N) &
  0.55 (N) &
  0.35 (W) &
  0.55 (N) &
  0.25 (W) &
  -
  \\
$G_2$ &
  0.82 (S) &
  0.25 (W) &
  0.25 (W) &
  0.82 (S) &
  0.59 (N) &
  0.36 (W) &
  0.42 (N) &
  0.55 (N) &
  - &
  - &
  - &
  -
  \\
$G_3$ &
  - &
  0.25 (W) &
  0.25 (W) &
  0.82 (S) &
  - &
  - &
  - &
  - &
  - &
  - &
  - &
  -
  \\
$G_4$ &
  - &
  0.25 (W) &
  0.25 (W) &
  0.82 (S) &
  - &
  - &
  - &
  - &
  - &
  - &
  - &
  -
  \\
$G_5$ &
  - &
  - &
  - &
  - &
  0.59 (N) &
  0.36 (W) &
  0.42 (N) &
  0.55 (N) &
  - &
  - &
  - &
  -
  \\
$G_6$ &
  - &
  - &
  - &
  - &
  0.55 (N) &
  0.24 (O) &
  0.35 (W) &
  0.55 (N) &
  - &
  - &
  - &
  -
  \\
$G_7$ &
  - &
  - &
  - &
  - &
  0.64 (N) &
  0.6 (N)  &
  0.55 (N) &
  - &
  - &
  - &
  - &
  -
  \\
$G_8$ &
  - &
  - &
  - &
  - &
  - &
  - &
  - &
  0.55 (N) &
  - &
  - &
  - &
  -
  \\
$G_9$ &
  - &
  - &
  - &
  - &
  0.59 (N) &
  0.36 (W) &
  0.42 (N) &
  0.59 (N) &
  - &
  - &
  - &
  -
  \\
$G_{10}$ &
  - &
  - &
  - &
  - &
  - &
  - &
  - &
  - &
  0.35 (W) &
  0.55 (N) &
  - &
  -
  \\
$G_{11}$ &
  - &
  - &
  - &
  - &
  - &
  - &
  - &
  - &
  0.35 (W) &
  - &
  - &
  -
  \\
$G_{12}$ &
  - &
  - &
  - &
  - &
  - &
  - &
  - &
  - &
  - &
  - &
  0.25 (W) &
  -
  \\
$G_{13}$ &
  - &
  - &
  - &
  - &
  - &
  - &
  - &
  - &
  - &
  - &
  - &
  0.13 (O)
  \bigstrut\\
\Xhline{4\arrayrulewidth}
\end{tabular}}%
%

\end{table*}

\subsubsection{Defuzzification}
\label{partial_def}

Our employed RELAX-ation technique\cite{mougouei2013goal,mougouei2015partial} needs crisp values to specify the required level of satisfaction for RELAX-ed security requirements. Hence, defuzzification is required to map the linguistic priority values of requirements into their corresponding crisp values. To do so, we use the Center of Gravity (COG) \cite{van2006fast} formula given by (\ref{Eq_rds}) and the membership function of the priority (Figure~\ref{fig_inference}) to deffuzify the priority values. $ x $ in this equation denotes crisp priority values on the $x$ axis of the priority graph in Figure~\ref{fig_inference}. Also $\mu_{x}$ denotes the membership function of priorities. 

\begin{align}
\label{Eq_rds}
& \text{Defuzzified Priority}=\frac{\int_{0}^{1}\mu_{x}\times x\times d_{x}}{\int_{0}^{1}\mu_{x}\times d_{x}}
\end{align}
\vspace{-3em}
\begin{table}[!htbp]
	\caption{RELAX-ed Security Requirements of OBS.}
	\label{table_relaxed}
	\centering
	% Table generated by Excel2LaTeX from sheet 'Sheet15'
\rowcolors{1}{}{lightgray}
\resizebox {0.9\textwidth }{!}{\begin{tabular}{ll}
\Xhline{4\arrayrulewidth}
Requirement &
  Sample RELAX-ed Value
  \bigstrut\\
\Xhline{4\arrayrulewidth}
$R_1$: achieve request transaction code &
  [expiry rate] as close as possible to $0.82 \times OV_1$
  \bigstrut\\
$R_2$: achieve latency examination &
  [examination delay] as close as possible to  $0.25 \times OV_2$
  \\
$R_3$: achieve one-time pad &
  [randomness] as close as possible to $0.25 \times  OV_3$
  \\
$R_4$: achieve SSL &
  [entropy] as close as possible to $0.82 \times  OV_4$
  \\
$R_5$: achieve password trial limitation &
  [trial delay] as close as possible to $0.59 \times OV_5$
  \\
$R_6$: achieve password policy &
  [complexity] as close as possible to $0.36 \times  OV_6$
  \\
$R_7$: achieve password encryption &
  [length of encryption key]as many bits as $0.42 \times  OV_7$
  \\
$R_8$: achieve random id &
  [randomness] as close as possible to $0.55 \times  OV_8$
  \\
$R_9$: achieve CAPTCHA &
  [level of distortion] as close as possible to $OV_9$
  \\
$R_{10}$: achieve complex pin &
  [complexity] as close as possible to $0.55 \times  OV_{10}$
  \\
$R_{11}$: achieve access control &
  [complexity] as close as possible to $0.25 \times  OV_{11}$
  \\
$R_{12}$: achieve redundant server &
  [number of servers] as close as possible to $0.13 \times  OV_{12}$
  \bigstrut\\
\Xhline{4\arrayrulewidth}
\end{tabular}}%

\end{table}
\vspace{-2em}
\subsubsection{RELAX-ation}
\label{partial_relax}

A requirement is RELAX-ed by relaxing its satisfaction condition (level). For this purpose we use a RELAX-ation technique proposed in~\cite{whittle2010relax}. Fuzzy semantic of the RELAX statements in the technique used properly captures partiality of security requirements \cite{mougouei2013goal}. Equations (\ref{eq_relax1})-(\ref{eq_relax3}) demonstrate this logic. 

In this equation, the aim is to maximize for each requirement $R_i$ the value of the membership function $\mu(V_i - (RDS_i \times OV_i))$ which is equivalent to finding a value $V_i$ for the satisfaction condition of $R_i$ as close as possible to the relaxed value $RDS_i\times OV_i$ where $RDS_i$ denotes the required degree of satisfaction for $R_i$ and $OV_i$ is the optimal value of the satisfaction condition in the absence of all resource limitations. 

%This is explained in terms of fuzzy semantic of the $R_6$: ``$AG ((\Delta (compexity) - (R_6.RDS × R_6.OptimalValue)) \in S)$'' where  $ \Delta(R_6) $ is the complexity of password policy. Also, $\mu$ is the membership function of the variance of the password complexity $\Delta(R_6)$ from the optimal complexity $R_6.RDS\times R_6.OptimalValue$. Furthermore, \textit{AG} denotes in all states and through all execution paths the condition of (\ref{label}) must hold. 

%\begin{align}
%\label{eq_relax_ex}
%&\mu(x)\rightarrow S, \mu(0)= 1\\ \nonumber
%& S= \{(x_i,\mu(x_i))|\mu(x_i)\in [0,1] , x \in \mathbb{R} \} \Rightarrow \\ \nonumber
%& \text{if }\Delta(R_6)= R_6.RDS\times R_6.OptimalValue \rightarrow \\ \nonumber
%&\mu(\Delta(R_6) - (R_6.RDS \times R_6.optimalValue))=1 \\ \nonumber
%\end{align}

\begin{align}
\label{eq_relax1}
Maximize \phantom{s} &\mu \big(V_i - (RDS_i \times OV_i) \big),& i=1,...,n\\
\label{eq_relax2}
Subject\phantom{s}to\phantom{s} &\mu(x_j)\in [0,1], &j=1,...,n \\ 
\label{eq_relax3}
& \mu(0)= 1\\ \nonumber
\end{align}

For instance consider RELAX-ing from security requirements of the OBS ``$R_6$: achieve password policy" where based on Table~\ref{table_rds}, $ SRL[S][R6] $= `weak’. When a less complex password encryption can be tolerated in $ SRL[S] $, satisfaction condition of $R_6$ can be RELAX-ed as : ``$ R_6 $: System shall achieve password policy [complexity] as close as possible to ($ RDS_6 \times OV_6 $). The membership function $\mu$ is maximizes ($\mu(x)=1$) when the variance of the password complexity is equal to $RDS_6 \times OV_6$. In a similar way, security requirements in $ SRL[S] $ of the OBS are RELAX-ed and listed in Table~\ref{table_relaxed}. It is important to note that ‘relaxed’ attributes are constructed based on the satisfaction metrics of security requirements. As depicted in Table~\ref{table_relaxed}, satisfaction metrics are placed in the brackets. For instance, satisfaction metric of $ R_6 $ is ``complexity" of the password policy while ``length" of encryption key is measures satisfaction of $ R_7 $.   

