\documentclass{appolb}
\usepackage{graphicx}
\usepackage{amssymb,amsmath,epsfig}
\usepackage{mathrsfs}
\usepackage[usenames,dvipsnames]{color}
\usepackage[bookmarks]{hyperref}
\usepackage[svgnames]{xcolor}
\usepackage{subfigure}

\newcommand{\Natural}{\mathbb{N}}
\newcommand{\Integer}{\mathbb{Z}}
\newcommand{\Real}{\mathbb{R}}
\newcommand{\Complex}{\mathbb{C}}
\newcommand{\F}{\mathbb{F}}
\newcommand{\re}{\mbox{Re}}
\newcommand{\im}{\mbox{Im}}
\newcommand{\diag}{\mbox{diag}}
\newcommand{\divrg}{\mbox{div}\,}
\newcommand{\dvol}{\mbox{dvol}}
\newcommand{\ve}[1]{\underline{#1}}
\newcommand{\proof}{\noindent {\bf Proof. }}
\newcommand{\proofof}[1]{\noindent {\bf Proof of #1. }}
\newcommand{\qed}{\hfill \fbox{} \vspace{.3cm}}
\newcommand{\identy}{1 \! \! 1}

\newtheorem{definition}{Definition}
\newtheorem{lemma}{Lemma}
\newtheorem{proposition}{Proposition}
\newtheorem{corollary}{Corollary}
\newtheorem{theorem}{Theorem}
\newtheorem{remark}{Remark}

% graphicx package included for placing figures in the text
%------------------------------------------------------
%%%%%%%%%%%%%%%%%%%%%%%%%%%%%%%%%%%%%%%%%%%%%%%%%%
%                                                %
%    BEGINNING OF TEXT                           %
%                                                %
%%%%%%%%%%%%%%%%%%%%%%%%%%%%%%%%%%%%%%%%%%%%%%%%%%
\begin{document}
% \eqsec  % uncomment this line to get equations numbered by (sec.num)
\title{Kinetic gas disks surrounding Schwarzschild black holes
%\thanks{Presented at ...}%
% you can use '\\' to break lines
}
\author{Carlos Gabarrete and Olivier Sarbach
\address{Instituto de F\'isica y Matem\'aticas,
Universidad Michoacana de San Nicol\'as de Hidalgo,
Edificio C-3, Ciudad Universitaria, 58040 Morelia, Michoac\'an, M\'exico.}}

\maketitle

\begin{abstract}
%\textcolor{red}{200 words or less than 3\% of the total of document.}
We describe stationary and axisymmetric gas configurations surrounding black holes. They consist of a collisionless relativistic kinetic gas of identical massive particles following bound orbits in a Schwarzschild exterior spacetime and are modeled by a one-particle distribution function which is the product of a function of the energy and a function of the orbital inclination associated with the particle's trajectory. The morphology of the resulting configuration is analyzed.
\end{abstract}

\date{\today}

\PACS{04.20.-q, 04.70.-s, 05.20.Dd}
% 04.20.-q: Classical GR
% 04.20.Ex: Initial value problem, existence and uniqueness of solutions
% 04.25.-g: Approximation methods; equations of motion
% 04.25.D-: Numerical relativity
% 04.25.Nx: Post-Newtonian approximation; perturbation theory; related approximations
% 04.40.-b: Self-gravitating systems, continuous media and classical fields in curved spacetime
% 04.70.-s: Physics of black holes
% 05.20.Dd: Kinetic theory
% 97.60.Lf: Astronomy: Late stage of star evolution: black holes



\section{Introduction}
\label{Sec:Introduction}

In recent years, there has been interest in analyzing the properties of solutions to the Vlasov equation on a fixed, curved background spacetime. In particular, such an analysis has been performed for a Schwarzschild background with the aim of understanding the Bondi-Michel and Bondi-Hoyle-Littleton accretion models for a collisionless kinetic gas~\cite{pRoS17,aGetal2021,pMoA2021b}. Also, kinetic analogues of the perfect fluid ``Polish doughnuts" configurations are discussed in~\cite{cGoS2021b}. Similarly to their fluid counterparts, they describe stationary and axisymmetric disks around black holes, where the individual gas particles follow bound timelike geodesics in a Schwarzschild spacetime. In~\cite{cGoS2021b} these configurations are modeled by a one-particle distribution function (DF) depending only on the energy $E$, azimuthal $L_z$ and total angular momentum $L$ of the particles. Examples are given in which the DF is described by a generalized polytropic ansatz~\cite{eAhAaL16,eAhAaL19} depending only on $E$ and $L_z$. In this article, we provide additional examples where the DF is a function of $E$ and the inclination angle $i$ defined by $\cos i = L_z/L$. We analyze the behavior of the resulting particle density and compute the total number of particles of the gas cloud as a function of the free parameters in our ansatz.

\section{The model}
\label{Sec:Model}

We work in the Schwarzschild exterior spacetime, written in the usual coordinates $(t,r,\vartheta,\varphi)$, with metric\footnote{We use units in which the speed of light and the gravitational constant are one.}
\begin{equation}
\label{Eq:Smetric}
g := -N(r) dt^2 + \frac{dr^2}{N(r)} + r^2 \left( d\vartheta^2 + \sin^2\vartheta d\varphi^2 \right), \quad N(r) := 1-\frac{2M}{r} > 0,
\end{equation}
where $M > 0$ is the mass of the black hole. Since this spacetime is static and spherically symmetric, the particle's rest mass $m$ is conserved along with $E$, $L$ and $L_z$. In terms of the orthonormal tetrad $e_{\hat{0}} = N(r)^{-1/2}\partial_t$, $e_{\hat{1}} = N(r)^{1/2}\partial_r$, $e_{\hat{2}} = r^{-1}\partial_\vartheta$, $e_{\hat{3}} = (r\sin\vartheta)^{-1}\partial_\varphi$, the four-momentum of the particles can be parametrized as
$p = p^{\hat{\mu}} e_{\hat{\mu}}$ with (see~\cite[Eq. (58)]{aGetal2021}):
\begin{equation}
(p^{\hat{\mu}}) = \left( \frac{E}{\sqrt{N(r)}}, \epsilon_r \sqrt{\frac{E^2 - V_L(r)}{N(r)}},
\frac{\epsilon_\vartheta}{r}\sqrt{L^2 - \frac{L_z^2}{\sin^2\vartheta}}, \frac{L_z}{r\sin\vartheta} 
\right),
\label{Eq:OrthonormalBasis}
\end{equation}
where the signs $\epsilon_r = \pm 1$ and $\epsilon_\vartheta = \pm 1$ determine the direction of motion in the radial and polar directions, respectively, and $V_L(r) = N(r)(m^2 + L^2/r^2)$ is the effective potential for the radial motion.

A collisionless relativistic gas consisting of identical massive particles of mass $m$ trapped in $V_L$ is described by a DF which relaxes in time to a DF depending only on integrals of motion. This is due to phase mixing, see e.g.~\cite{pRoS18a,pRoS20} and references therein. Here we assume, in addition, that the final configuration is axisymmetric, which implies that the DF has the form
\begin{equation}
f(x, p) = F(E,L,L_z),
\label{Eq:OneParticleDistributionFunction}
\end{equation}
for some function $F$ which we shall specify shortly. The relevant spacetime observables are the particle current density vector field $J$ and the energy-momentum-stress tensor $T$ defined by
\begin{equation}
J_{\hat{\mu}}(x) := \int_{P_x^+(m)} f(x, p) p_{\hat{\mu}} \dvol_x(p), \quad
T_{\hat{\mu}\hat{\nu}} (x) := \int_{P_x^+(m)} f(x, p) p_{\hat{\mu}} p_{\hat{\nu}} \dvol_x(p),\quad
\label{Eq:JT}
\end{equation}
where $\dvol_x(p) = dp^{\hat{1}}\wedge dp^{\hat{2}}\wedge dp^{\hat{3}}/p^{\hat{0}}$ is the Lorentz-invariant volume form on the future mass hyperboloid  $P_x^+(m)$ of mass $m$ at $x$, see~\cite{rAcGoS2021} for details.

For the following, we focus on the particular ansatz
\begin{equation}
\label{Eq:OneParticleDistrFunct}
F(E,L,L_z) := F_0(E) \cos^{2s}(i),\qquad
F_0(E) = \alpha \left(1 - \frac{E}{m} \right)_+^{k-\frac{3}{2}},
\end{equation}
where $\alpha > 0$, $k > 1/2$ are constants, $i$ is the inclination angle and $s\geq 0$ is a parameter. The notation $f_+$ refers to the positive part of the quantity $f$, that is $f_+ = f$ if $f > 0$ and $f_+ = 0$ otherwise. Here, the function $F_0$ is the general relativistic generalization of the polytropic ansatz~\cite{BT} while the parameter $s$ controls the concentration of the orbits near the equatorial plane $\vartheta = \pi/2$ (see Fig.~\ref{Fig:InclinationAngle}).
\begin{figure}[htb]
\centerline{
\subfigure{\includegraphics[scale=0.2]{Inclination_angle_01.pdf}} 
\subfigure{\includegraphics[scale=0.2]{Inclination_angle_02.pdf}}} 
\caption{Illustration of the effect of the parameter $s$ (left panel for small $s$, right panel for large $s$). As $s$ increases, orbits confined to planes lying close to the equatorial one become more populated, such that the configuration becomes a thin disk in the limit $s\to \infty$.}
\label{Fig:InclinationAngle}
\end{figure}\\
For the following we introduce the dimensionless quantities $\xi := r/M$, $\lambda := L/(M m)$, $\varepsilon := E/m$, and $U_\lambda(\xi) := V_L(r)/m^2$ and parametrize the future mass hyperboloid $P_x^+(m)$ in terms of the quantities $(\varepsilon,\lambda,\chi)$, where the angle $\chi$ is defined by $(p^{\hat{2}},p^{\hat{3}}) = \frac{m\lambda}{\xi}(\cos\chi,\sin\chi)$ which implies $\cos i = \sin\vartheta \sin\chi$. For bound orbits, these quantities are restricted to the following domain (see~\cite[Appendix A]{pRoS16} and~\cite[Appendix A]{cGoS2021b}):
\begin{equation}
\label{Eq:Limits}
\varepsilon_{\text{c}}(\xi) < \varepsilon < 1, \quad \lambda_{\text{c}}(\varepsilon) \leq \lambda \leq \lambda_{\text{max}}(\varepsilon,\xi), \quad \hbox{and} \quad 0\leq \chi \leq 2\pi,
\end{equation}
where $\varepsilon_{\text{c}}(\xi)$ is the minimum energy at radius $\xi$, $\lambda_{\text{c}}(\varepsilon)$ is the critical value for the total angular momentum for which the maximum of the potential barrier in $U_\lambda(\xi)$ is exactly equal to $\varepsilon^2$, and $\lambda_{\text{max}}(\varepsilon,\xi)$ is the maximum angular momentum permitted at energy $\varepsilon$ and radius $\xi$. Note that the domain~(\ref{Eq:Limits}) is empty if $\xi < 4$, since for a Schwarzschild black hole the minimum radius for bound orbits is $r = 4M$.

For the ansatz~(\ref{Eq:OneParticleDistrFunct}) the fibre integrals in Eq.~(\ref{Eq:JT}) yield
\begin{equation}
\label{Eq:Jmu}
J_{\hat{\mu}}(x) = \frac{m^2\sin^{2s}\vartheta}{\xi^2}
\sum_{\epsilon_r = \pm 1}\int_{\varepsilon_{\text{c}}(\xi)}^1 \int_{\lambda_{\text{c}}(\varepsilon)}^{\lambda_{\text{max}}(\varepsilon,\xi)} 
\int_0^{2\pi} p_{\hat{\mu}} F_0(E) \sin^{2s}\chi \frac{d\varepsilon\; \lambda d\lambda\;d\chi}{\sqrt{\varepsilon^2 - U_\lambda(\xi)}},
\end{equation}
and similarly for $T_{\hat{\mu}\hat{\nu}}(x)$. Using the expressions~(\ref{Eq:OrthonormalBasis}) for the four-momentum, the non-vanishing orthonormal components of $J^{\hat{\mu}}$ and $T^{\hat{\mu}}{}_{\hat{\nu}}$ are
\begin{eqnarray}
\label{Eq:J0}
J^{\hat{0}} &=& 4\sqrt{\pi} \frac{\sin^{2s}\vartheta}{N^{3/2}} \frac{\Gamma(s+1/2)}{\Gamma(s+1)} m^3\int_{\varepsilon_{\text{c}}(\xi)}^1 d\varepsilon \varepsilon Y(\varepsilon,\xi)^{1/2}
F_0(m\varepsilon),\\
\label{Eq:T00}
T^{\hat{0}}{}_{\hat{0}} &=& -4\sqrt{\pi} \frac{\sin^{2s}\vartheta}{N^2} \frac{\Gamma(s+1/2)}{\Gamma(s+1)} m^4\int_{\varepsilon_{\text{c}}(\xi)}^1 d\varepsilon \varepsilon^2 
Y(\varepsilon,\xi)^{1/2}
F_0(m\varepsilon),\\
\label{Eq:T11}
T^{\hat{1}}{}_{\hat{1}} &=& \frac{4\sqrt{\pi}}{3} \frac{\sin^{2s}\vartheta}{N^2} \frac{\Gamma(s+1/2)}{\Gamma(s+1)} m^4\int_{\varepsilon_{\text{c}}(\xi)}^1 d\varepsilon 
Y(\varepsilon,\xi)^{3/2}
F_0(m\varepsilon),\\
\label{Eq:T22}
T^{\hat{2}}{}_{\hat{2}} &=& \frac{4\sqrt{\pi}}{3} \frac{\sin^{2s}\vartheta}{N^2} \frac{\Gamma(s+1/2)}{\Gamma(s+2)} m^4\int_{\varepsilon_{\text{c}}(\xi)}^1 d\varepsilon 
Y(\varepsilon,\xi)^{1/2}
Z(\varepsilon,\xi) F_0(m\varepsilon),
\qquad\\
\label{Eq:T33}
T^{\hat{3}}{}_{\hat{3}}  &=& (2s+1)T^{\hat{2}}{}_{\hat{2}},
\end{eqnarray}
where we have introduced the shorthand notation
\begin{equation}
Y(\varepsilon,\xi) := \varepsilon^2 - N(r)\left[1 + \frac{\lambda_c(\varepsilon)^2}{\xi^2} \right],\quad
Z(\varepsilon,\xi) := \varepsilon^2 - N(r)\left[1 - \frac{\lambda_c(\varepsilon)^2}{2\xi^2} \right].
\end{equation}
The quantities~(\ref{Eq:J0})--(\ref{Eq:T33}) determine the relevant macroscopic observables, namely the particle density $n = J^{\hat{0}}$, energy density $\mathcal{E} = -T^{\hat{0}}{}_{\hat{0}}$, and the principal pressures $P_1 = T^{\hat{1}}{}_{\hat{1}}$, $P_2 = T^{\hat{2}}{}_{\hat{2}}$, and $P_3 = (2s+1)P_2$. Note that all of these quantities have the dependency $\sin^{2s}\vartheta$ with respect to the polar angle $\vartheta$. In the limit $s = 0$ the configurations describe a spherical shell of gas trapped in the region $\xi > 4$, while for $s = 1/2,1,3/2,\ldots$ they are axisymmetric, the macroscopic variables being zero for $\xi \leq 4$ and along the axis $\vartheta = 0,\pi$. In the next section, we analyze the morphology of these configurations as a function of the parameters $k$ and $s$ for fixed total particle number.

\section{Total particle number and behavior of the particle density}
\label{Sec:Mass}

The (conserved) total particle number ${\cal N}$ is defined as minus the flux integral of the current density vector field with respect to a Cauchy surface. This in turn can be rewritten as an integral over the six-dimensional phase space parametrized by $(x^i,p_i)$. To compute this integral, it is convenient to transform $(x^i,p_i)$ to action-angle variables $(\mathcal{Q}^i,\mathcal{J}_i)$. The integral over the angle variables $\mathcal{Q}^i$ yields a factor $(2\pi)^3$ while the integral over the action variables can be rewritten in terms of the conserved quantities $(E,L,L_z)$, taking into account that $d^3\mathcal{J} = T(E,L) dE dL dL_z /2\pi$, where $T(E,L)$ is the period function for the radial motion. For the Schwarzschild spacetime this function can be expressed in terms of elliptic integrals and has the form $T(E,L) = 2 M \varepsilon \left[ \mathbb{H}_2 - \mathbb{H}_0 \right]$ (see~\cite[Appendix A]{pRoS18a} and~\cite{cGoS2021b} for the explicit form of $\mathbb{H}_2$ and $\mathbb{H}_0$ in the Schwarzschild case). For the ansatz~(\ref{Eq:OneParticleDistrFunct}) this yields the following expression for the total particle number:
\begin{equation}
\label{Eq:NumberParticles}
\mathcal{N} = \frac{16\pi^2}{2s+1}(M m)^3 \alpha \int_{\varepsilon_{\text{min}}}^1 d\varepsilon \; \varepsilon \left(1-\varepsilon \right)^{k-\frac{3}{2}}_+ 
\int_{\lambda_{\text{c}}(\varepsilon)}^{\lambda_{\text{ub}}(\varepsilon)} 
d\lambda\; \lambda \; (\mathbb{H}_2-\mathbb{H}_0),
\end{equation}
where $\varepsilon_{\text{min}} = \sqrt{8/9}$ and $\lambda_{\text{ub}}(\varepsilon)$ is given in~\cite[Appendix A]{pRoS16}. To compute this integral, it is convenient to re-parametrize the orbits in terms of their eccentricity $e$ and ``semi-latus rectum" $P$, related to the turning points $(\xi_1,\xi_2)$ by $\xi_1 = P/(1+e)$ and $\xi_2 = P/(1-e)$ and to the conserved quantities $(\varepsilon,\lambda)$ according to~\cite{wS02,jBmGtH15,pRoS18a,cGoS2021b} $(\varepsilon^2,\lambda^2) = ( P^{-1}[(P-2)^2 - 4e^2], P^2)/(P - e^2 -3)$. Here, $(P,e)$ are restricted to the domain $0 < e < 1$ and $P > 6 + 2e$. The resulting integral is then calculated numerically using MATHEMATICA. The total mass is simply $m{\cal N}$ and the total energy is given by the same expression as in Eq.~(\ref{Eq:NumberParticles}) with an extra factor $m\varepsilon$ inside the integral.

\begin{figure}[h!]
\centerline{
\subfigure{\includegraphics[scale=0.205]{Particle_density_01.pdf}}
\subfigure{\includegraphics[scale=0.205]{Particle_density_02.pdf}}}
\caption{Left panel: Dimensionless profile of the particle density in the equatorial plane for $k=3,4,5$ and $s=1$ in a logarithmic scale. Right panel: Same quantity multiplied with $\xi^2$ which shows that even though configurations with higher values of $k$ have a larger maximum, they have a faster decay at infinity.}
\label{Fig:ParticleDensitys1k345} 
\end{figure}
In Fig.~\ref{Fig:ParticleDensitys1k345} we show the dimensionless quantity $M^3 n/{\cal N}$ in the equatorial plane for several values of $k$ and $s=1$. In Fig.~\ref{Fig:ContourParticleDensitys1k3} we show contour plots of the same quantity in the $xz$-plane for $k=3$ and two different values of $s$.
\begin{figure}
\centerline{
\subfigure{\includegraphics[scale=0.25]{Contour_particle_density_01.jpeg}}
\subfigure{\includegraphics[scale=0.25]{Contour_particle_density_02.jpeg}}}
\caption{Contour plots for the particle density in the $xz$-plane for the configurations with $k=3$ and $s=1$ (left panel) and $k=3$ and $s=3$ (right panel). Here, $(x,z) = r(\sin\vartheta,\cos\vartheta)$, the black region represents the black hole interior and the dashed black circles the interior boundary of the disk. As is visible from these plots, the configuration with higher $s$ yields a thinner disk.}
\label{Fig:ContourParticleDensitys1k3} 
\end{figure}

\section{Conclusions}

We described a family of stationary and axisymmetric collisionless gas configurations which are trapped in the gravitational potential of a Schwarz-schild black hole. This family depends on two parameters $s$ and $k$ which control the thickness of the disk and its radial density distribution. An alternative model is discussed in detail in~\cite{cGoS2021b}. We expect these configurations to serve as a first approximation for the description of low-luminosity disks surrounding black holes.

We acknowledge support from a CIC Grant to Universidad Michoacana and CONACyT Frontier Project No. 376127.

%%%%%%%%%%%%%%%%%% REFERENCES %%%%%%%%%%%%%%%%%%

\begin{thebibliography}{99}

\expandafter\ifx\csname url\endcsname\relax
\def\url#1{\texttt{#1}}\fi
\expandafter\ifx\csname urlprefix\endcsname\relax\def\urlprefix{URL }\fi
\providecommand{\bibinfo}[2]{#2}
\providecommand{\eprint}[2][]{\url{#2}}

\bibitem{pRoS17} 
\bibinfo{author}{P.~Rioseco and O.~Sarbach,} %\emph{~et~al.}, 
%\newblock \bibinfo{title}{{Spherical steady-state accretion of a relativistic collisionless gas into a {S}chwarzschild black hole}}, 
\newblock \emph{\bibinfo{journal}{J. Phys. Conf. Ser.}}
\textbf{\bibinfo{volume}{831}}, 
\bibinfo{pages}{012009}
(\bibinfo{year}{2017}). 

\bibitem{aGetal2021} 
\bibinfo{author}{A.~Gamboa}\emph{~et~al.}, 
%\newblock \bibinfo{title}{{Accretion of a Vlasov gas onto a black hole from a sphere of finite radius and the role of angular momentum}}, 
\newblock \emph{\bibinfo{journal}{Phys. Rev. D}}
\textbf{\bibinfo{volume}{104}}, 
\bibinfo{pages}{083001}
(\bibinfo{year}{2021}). 

\bibitem{pMoA2021b} 
\bibinfo{author}{P.~Mach and A.~Odrzywo\l{}ek,} %\emph{~et~al.}, 
%\newblock \bibinfo{title}{{Accretion of Dark Matter onto a Moving Schwarzschild Black Hole: An Exact Solution}}, 
\newblock \emph{\bibinfo{journal}{Phys. Rev. Lett.}}
\textbf{\bibinfo{volume}{126}}, 
\bibinfo{pages}{101104}
(\bibinfo{year}{2021}). 

\bibitem{cGoS2021b} 
\bibinfo{author}{C.~Gabarrete and O.~Sarbach,} %\emph{~et~al.}, 
%\newblock \bibinfo{title}{{Axisymmetric, stationary collisionless gas configurations surrounding black holes}}, 
\newblock \emph{\bibinfo{journal}{In preparation.}}
%\textbf{\bibinfo{volume}{126}}, 
%\bibinfo{pages}{101104}
(\bibinfo{year}{2022}). 

\bibitem{eAhAaL16} 
\bibinfo{author}{E.~Ames}\emph{~et~al.}, 
%\newblock \bibinfo{title}{{On Axisymmetric and Stationary Solutions of the Self-Gravitating {V}lasov System}}, 
\newblock \emph{\bibinfo{journal}{Class. Quantum Grav.}}
\textbf{\bibinfo{volume}{32}}, 
\bibinfo{pages}{155008}
(\bibinfo{year}{2016}). 

\bibitem{eAhAaL19} 
\bibinfo{author}{E.~Ames}\emph{~et~al.}, 
%\newblock \bibinfo{title}{{Cosmic String and Black Hole Limits of Toroidal {V}lasov Bodies in General Relativity}}, 
\newblock \emph{\bibinfo{journal}{Phys. Rev. D}}
\textbf{\bibinfo{volume}{99}}, 
\bibinfo{pages}{024012}
(\bibinfo{year}{2019}). 

\bibitem{pRoS18a} 
\bibinfo{author}{P.~Rioseco and O.~Sarbach,} %\emph{~et~al.}, 
%\newblock \bibinfo{title}{{Phase space mixing in the equatorial plane of a {K}err black hole}}, 
\newblock \emph{\bibinfo{journal}{Phys. Rev. D}}
\textbf{\bibinfo{volume}{98}}, 
\bibinfo{pages}{124024}
(\bibinfo{year}{2018}). 

\bibitem{pRoS20} 
\bibinfo{author}{P.~Rioseco and O.~Sarbach,} %\emph{~et~al.}, 
%\newblock \bibinfo{title}{{Phase space mixing in external gravitational central potentials}}, 
\newblock \emph{\bibinfo{journal}{Class. Quantum Grav.}}
\textbf{\bibinfo{volume}{37}}, 
\bibinfo{pages}{195027}
(\bibinfo{year}{2020}). 

\bibitem{rAcGoS2021} 
\bibinfo{author}{R.~Acu{\~n}a{-}C{\'a}rdenas}\emph{~et~al.}, 
%\newblock \bibinfo{title}{{An introduction to the relativistic kinetic theory on curved spacetimes}}, 
\newblock \emph{\bibinfo{journal}{arXiv:2106.09235}}
%\textbf{\bibinfo{volume}{104}}, 
%\bibinfo{pages}{083001}
(\bibinfo{year}{2021}).

\bibitem{BT}
\bibinfo{author}{J. Binney and S. Tremaine,}
\bibinfo{title}{\emph{Galactic Dynamics},}
\bibinfo{publisher}{Princeton University Press,}
\bibinfo{address}{Princeton, New Jersey,}
\bibinfo{year}{(2008)}.

\bibitem{pRoS16} 
\bibinfo{author}{P.~Rioseco and O.~Sarbach,} %\emph{~et~al.}, 
%\newblock \bibinfo{title}{{Accretion of a relativistic, collisionless kinetic gas into a {S}chwarzschild black hole}}, 
\newblock \emph{\bibinfo{journal}{Class. Quantum Grav.}}
\textbf{\bibinfo{volume}{34}}, 
\bibinfo{pages}{095007}
(\bibinfo{year}{2017}). 

\bibitem{wS02} 
\bibinfo{author}{W.~Schmidt} %\emph{~et~al.}, 
%\newblock \bibinfo{title}{{Celestial mechanics in {K}err space-time}}, 
\newblock \emph{\bibinfo{journal}{Class. Quantum Grav.}}
\textbf{\bibinfo{volume}{19}}, 
\bibinfo{pages}{2743--2764}
(\bibinfo{year}{2002}). 

\bibitem{jBmGtH15} 
\bibinfo{author}{J.~Brink}\emph{~et~al.}, 
%\newblock \bibinfo{title}{{Astrophysics of resonant orbits in the {K}err metric}}, 
\newblock \emph{\bibinfo{journal}{Phys. Rev. D}}
\textbf{\bibinfo{volume}{91}}, 
\bibinfo{pages}{083001}
(\bibinfo{year}{2015}).

%\bibitem{Mathematica} 
%\bibinfo{author}{Wolfram~Research{,}~Inc.} %\emph{~et~al.}, 
%\newblock \bibinfo{title}{{Mathematica, {V}ersion 12.2}}, 
%\newblock \emph{\bibinfo{journal}{ }}
%\textbf{\bibinfo{volume}{114}}, 
%\bibinfo{note}{Champaign, IL.}
%(\bibinfo{year}{2021}). 

\end{thebibliography}

\end{document}