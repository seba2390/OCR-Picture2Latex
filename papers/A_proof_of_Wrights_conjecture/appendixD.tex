%!TEX root = hopfwright.tex

%%%%%%%%%%%%%%%%%%
%%% Appendix D %%%
%%%%%%%%%%%%%%%%%%

\section{Appendix: The bounding functions for $Z(\epsilon,r,\rho)$}
\label{sec:BoundingFunctions}

	In this section we calculate an upper bound on $DT$.  
	To do so we first calculate 
	\(
	DF = 
	\left[ \frac{\partial F}{\partial  \alpha}, 
	\frac{\partial F}{\partial  \omega},
	\frac{\partial F}{\partial  c}
	\right]
	\):
%\note[JB]{I don't think we need to write the partial derivatives in a proposition and no proof is needed.}
%\begin{proposition}
%	The partial derivatives of $F$ are
	\begin{alignat}{1}
	\label{eq:FpartialA}
	\frac{\partial F}{\partial  \alpha} &= e^{-i \omega} \e_1 + U_\omega c + \epsilon e^{-i \omega} \e_2 + \epsilon L_\omega c + \epsilon [ U_\omega c] * c , \\
	\label{eq:FpartialW} 
	\frac{\partial F}{\partial  \omega} &=
	i(1-\alpha e^{-i \omega}) \e_1 + 
	i K^{-1} ( I - \alpha U_{\omega} ) c  -
	i \alpha \epsilon e^{-i \omega} \e_2 + 
	\alpha \epsilon L_{\omega}' c - i \alpha \epsilon [ K^{-1} U_\omega c ] *c ,
	\\
	% \end{alignat}
	% and, writing $\frac{\partial F}{\partial  c}$ as an operator working on an element $ b \in \ell^K_0$,
	% % with $ \| b \| = 1$:
	% \begin{equation}
	\frac{\partial F}{\partial  c} \cdot b 
	& =
	( i \omega K^{-1} + \alpha U_{\omega}) b + \alpha \epsilon \left( L_\omega b  + [ U_\omega b] * c + [U_{\omega} c ]*b \right)  , \qquad \text{for all $b\in \ell^K_0$},
	\label{eq:Fcderivative}
%	\end{equation}
\end{alignat}	
%\end{proposition}
where $L_{\omega}'$ is given in~\eqref{e:Lomegaprime}, and $\frac{\partial F}{\partial  c}$ is expressed in terms of the directional derivative. 
Recall that $\II$ is used to denote the $ 3 \times 3$ identity matrix. 

% \begin{proof}
% 	Recall from Equation \ref{eq:FDefinition} that:
% \begin{equation}
% F_\epsilon(\alpha,\omega, c) =
% [i \omega + \alpha e^{-i \omega}] \e_1 +
% ( i \omega K^{-1} + \alpha U_{\omega}) c +
% \epsilon \alpha e^{-i \omega} \e_2  +
% \alpha \epsilon L_\omega c +
% \alpha \epsilon [ U_{\omega} c] * c.
% \end{equation}
% The partial derivative for $ 	\frac{\partial F}{\partial  \alpha}$ and $ 	\frac{\partial F}{\partial  c}$   clearly follow, and the partial derivative for $ 	\frac{\partial F}{\partial  \omega}$  follows from Proposition \ref{prop:OmegaDerivatives}.
%
%
%
% \end{proof}


\begin{theorem}
	\label{prop:Zdef}
	Define $\overline{A_0^{-1} A_1}$ as in Proposition \ref{prop:A0A1} and define the matrix 
%	\[
%	M := 
%	\left(
%	\begin{array}{ccc}
%	1+\frac{2}{\pi } & 0 & 0 \\
%	\frac{2}{\pi } & 0 & 0 \\
%	0 & 0 & 1 \\
%	\end{array}
%	\right)
%	\left(
%	\begin{array}{ccc}
%	f_{1,\alpha } & f_{1,\omega } & f_{1,c} \\
%	0 & 0 & 0 \\
%	f_{*,\alpha } & f_{*,\omega } & f_{*,c} \\
%	\end{array}
%	\right)
%	\]
%\note[JB]{I think we should remove the zero column and row, as below:}
%\note[J]{Updated top-right element entry to new estimate.}
	\begin{equation}\label{e:defM}
	M := 
	\left(
	\begin{array}{cc}
	\sqrt{\tfrac{4}{\pi^2}+1} & 0 \\
	\frac{2}{\pi } & 0 \\
	0 & 1 \\
	\end{array}
	\right)
	\left(
	\begin{array}{ccc}
	f_{1,\alpha } & f_{1,\omega } & f_{1,c} \\
	f_{*,\alpha } & f_{*,\omega } & f_{*,c} \\
	\end{array}
	\right) ,
	\end{equation}
	where the functions $f_{1,\cdot}(\epsilon,r,\rho)$ and $f_{*,\cdot}(\epsilon,r,\rho)$ are defined as in Propositions \ref{prop:Z1a}--\ref{prop:Zsc}. 
	If we define $Z(\epsilon,r,\rho)$ as 
	\begin{equation}
		Z(\epsilon,r,\rho) := \epsilon^2  \left(\overline{ A_0^{-1} A_1 }\right)^2  + 
		\left(\II + \epsilon \overline{ A_0^{-1} A_1 } \right) \cdot M ,
	\end{equation}
	then $Z(\epsilon,r)$ is an upper bound (in the sense of Definition~\ref{def:upperbound}) on $DT(x)$ for all $ x \in B_\epsilon(r , \rho)$. 
	Furthermore, the components of $Z(\epsilon,r,\rho)$ are increasing in  $ \epsilon$, $r$ and $\rho$. 
% \note[JB]{shouldn't we also have monotonicity in $\rho$?}
% \note[J]{Yes. Also note that in our application of radii polynomials, the value of $\rho$ is fixed for all $ 0< \epsilon \leq \epsilon_0$.}
\end{theorem}

\begin{proof}
	
	
	If we fix some $x \in B_\epsilon(r,\rho)$, then we obtain 
%	\remove[JB]{the following:}
%\note[JB]{Rearranged to separate expression for $DT$ from the upper bound on it}
\begin{alignat*}{1}
		D T( x ) &=  I - A^{\dagger}  D F( x)  \\
		&= ( I - A^{\dagger} A) - A^{\dagger} \left[ D F( x)  - A \right]\\
		&=   \epsilon^2 (A_0^{-1} A_1 )^2 -    [I - \epsilon (A_0^{-1} A_1 ) ] \cdot  A_0^{-1} \cdot   \left[ D F( x) - A \right] ,
\end{alignat*}
hence an upper bound on $DT(x)$ is given by
\begin{equation*}
\epsilon^2  \left(\overline{ A_0^{-1} A_1 }\right)^2  + 
\left(\II + \epsilon \overline{ A_0^{-1} A_1 } \right) \cdot 
\overline{A_0^{-1}  \left[D F( x ) - A \right] },
\end{equation*}
	where $\overline{A_0^{-1}  \left[D F( x ) - A \right] }$ is a yet to be determined upper bound on $A_0^{-1}  \left[D F( x ) - A \right]$.  
To calculate this upper bound, we break it up into two parts: 
	\begin{alignat}{1}
		\pi_{\alpha,\omega} A_0^{-1}  \left( D F( x) - A \right)  &= 
		A_{0,1}^{-1}  i_{\C}^{-1} \pi_1   \left( D F( x) - A \right) 
\label{eq:Zfinite}\\ 
		\pi_c 		A_0^{-1}  \left( D F( x) - A \right)  &=
		A_{0,*}^{-1}  \pi_{\geq 2} \left( DF(x) - A \right)  . \label{eq:Zstar} 
\end{alignat}
	
To calculate an upper bound on \eqref{eq:Zfinite}, we use the explicit expression for $A_{0,1}^{-1}$ to estimate  
% \note[JB]{Perhaps the factor $\frac{2}{\pi}+1$ can be improved to $\sqrt{\frac{4}{\pi^2}+1}$?; did not implement it.} 
%\note[J]{Yes; done.}
	\begin{alignat*}{1}
	\left| \pi_\alpha  A_{0,1}^{-1} \pi_1 \left( D F( x) - A \right)\right| &\leq  \sqrt{\tfrac{4}{\pi^2} + 1} \, \overline{\pi_1( DF( x ) - A) } \\
	\left| \pi_\omega  A_{0,1}^{-1} \pi_1 \left( D F( x) - A \right)\right| &\leq  \tfrac{2}{\pi} \,  \overline{ \pi_1( DF( x ) - A)  } ,
	\end{alignat*}
where $\overline{ \pi_1( DF( x ) - A)  }$ is an upper bound on $\pi_1( DF( x ) - A)$, viewed as an operator from $\R^2 \times \ell^K_0 $ to $\C$ (a straightforward generalization of Definition~\ref{def:upperbound}).
Indeed, in Propositions \ref{prop:Z1a}, \ref{prop:Z1w} and \ref{prop:Z1c} we   determine functions $f_{1,\cdot}$ such that, for all $x \in B_\epsilon(r,\rho)$, 
	\begin{alignat*}{1}
	f_{1,\alpha} (\epsilon,r,\rho) &\geq    \left|  \frac{\partial F_1 }{\partial \alpha} (x) + i  \right|  , \\
	f_{1,\omega} (\epsilon,r,\rho) &\geq   \left|  \frac{\partial F_1}{\partial \omega} (x)- (i- \pp)  \right|   , \\
	f_{1,c} (\epsilon,r,\rho) &\geq   \left|  \frac{\partial F_1}{\partial c} (x) \cdot b -  \pp \epsilon (i-1) \pi_2 b \right| ,
	\qquad\text{for all $b\in\ell^K_0$ with $\|b\| \leq 1$}.
	\end{alignat*} 
Here the projection $\pi_2$ is defined as $\pi_2 b := b_2 \in \C$ for $b=\{b_k\}_{k=1}^{\infty} \in \ell^1$.	
	Hence $ [ f_{1,\alpha} , f_{1,\omega}, f_{1,c}]$ is an upper bound on $\pi_1 (DF( x ) -A )$.  
	
	
	For calculating an upper bound on Equation~\eqref{eq:Zstar}, in Propositions  \ref{prop:Zsa}, \ref{prop:Zsw}  and \ref{prop:Zsc} we  determine functions $f_{*,\cdot}$ such that, for all $x \in B_\epsilon(r,\rho)$,   
	\begin{alignat*}{1}
	f_{*,\alpha} (\epsilon,r,\rho)&\geq  \left\| A_{0,*}^{-1}  \pi_{\geq 2} \left(
	\frac{\partial F}{\partial \alpha}(x)  +    \epsilon  \tfrac{2 +4 i}{5} \e_2  \right) \right\| , \\
	%
%	f_{*,\omega} (\epsilon,r,\rho)&\geq  \left\| A_{0,*}^{-1}  \pi_{\geq 2}\left( \frac{\partial F}{\partial \omega}(x) -  \epsilon \left[ (2 + \pi) ( \tfrac{1+ 2 i}{5})- \pp \right] \e_2  \right) \right\| ,  \\
%
	f_{*,\omega} (\epsilon,r,\rho)&\geq  \left\| A_{0,*}^{-1}  \pi_{\geq 2}\left( \frac{\partial F}{\partial \omega}(x) -  \epsilon \left[ \tfrac{4-3\pi}{10} + \tfrac{2(2 + \pi)}{5}i \right] \e_2  \right) \right\| ,  \\
	%
	f_{*,c} (\epsilon,r,\rho) &\geq    \left\|  A_{0,*}^{-1} \pi_{\geq 2} \left( \frac{\partial F}{\partial c}(x) \cdot b  - (A_{0,*} + \epsilon A_{1,*}) b \right) 
	\right\| , \qquad\text{for all $b\in\ell^K_0$ with $\|b\| \leq 1$}.
	\end{alignat*}
	Hence $ [ f_{*,\alpha} , f_{*,\omega}, f_{*,c}]$ is an upper bound on $A_{0,*}^{-1}  \pi_{\geq 2} \left( D F( x) - A \right)$, viewed as an operator from $\R^2 \times \ell^K_0$ to $\ell^1_0$. 
		We have thereby shown that $M$, as defined in~\eqref{e:defM}, is an upper bound on $\overline{A_0^{-1}  \left[D F( x ) - A \right] }$, which concludes the proof.	
\end{proof}









%%%%%%%%%%%%%%%%%%%%%%%%%%%%%%%%%%%%%%%%%%%%%%%%%%%%%%%%%%%%%%%%%%%%%%%%%%%%

%\note[JB]{I do not understand why we want to suppose $\epsilon <1$. I changed it in the D.2 but not yet in the other 5 propositions.} 
%\note[J]{You are correct that $ \epsilon <1$ is not needed. I have removed this assumption in the other propositions.}
\begin{proposition}
	\label{prop:Z1a}
	Define
	\[
	f_{1,\alpha} :=  \dw +  \epsilon \frac{\dc  (2 + \dc) }{2} .
	\]
	Then for all $x = (\alpha,\omega,c) \in B_\epsilon(r,\rho)$ 
	\[
	f_{1,\alpha} \geq   \left|  \frac{\partial F_1}{\partial \alpha} (x) + i  \right| .
	\]
\end{proposition}




\begin{proof}
	We calculate
\begin{equation*}
	\frac{\partial F_1}{\partial \alpha} (x) + i =
	e^{- i \omega} + i  
	+ \epsilon \left( e^{i \omega} + e^{-2 i  \omega} \right) \pi_2 c
	+ \epsilon \pi_1 ([ U_{\omega} c] * c) ,
\end{equation*}
hence, using Lemma~\ref{lem:deltatheta},
%\begin{equation*}
%	\left|  \frac{\partial F_1 }{\partial \alpha} F_1(x) + i  \right|   \leq 
%| e^{-i \omega } +i | + 2 \epsilon \dc + \epsilon \dc^2  
%	\leq  \dw +  \epsilon \dc  (2 + \dc) .
%\end{equation*}
%\note[JB]{Because of the factor 2 in the norm I think it should be }
\begin{equation*}
	\left|  \frac{\partial F_1}{\partial \alpha} (x) + i  \right|   \leq 
| e^{-i \omega } +i | + 2 \epsilon \frac{\dc}{2} + \epsilon \frac{1}{2} \dc^2  
	\leq  \dw +  \epsilon \frac{\dc  (2 + \dc)}{2} .
\end{equation*}
Here we have used that $|\pi_k a| \leq \frac{1}{2}\|a\|$ for $k=1,2$ and all $a \in \ell^1$.
\end{proof}

%\note[J]{I think it would be good to have an explanation for why we divide $ \dc$ by 2. This explanation wouldn't need appear every time; it could just be given once. I am not sure though where the right place for it would be. }
%\note[JB]{See my attempt above}

%%%%%%%%%%%%%%%%%%%%%%%%%%%%%%%%%%%%%%%%%%%%%%%%%%%%%%%%%%%%%%%%%%%%%%%%%%%%

\begin{proposition}
		\label{prop:Z1w}
	Define
	% \[
	% f_{1,\omega} :=
	% \da + \pp \dw + \frac{ \alpha \epsilon \dc}{2} ( 3 + \rho) .
	% \]
	% \note[J]{Proposed Change}
		\[
		f_{1,\omega} := 
		\da + \pp \dw + (\pp + \da) \frac{  \epsilon \dc}{2} ( 3 + \rho)  .
		\]
Then for all $x= (\alpha,\omega,c) \in B_\epsilon(r,\rho)$
	\[
	f_{1,\omega} \geq   \left|  \frac{\partial F_1}{\partial \omega } (x)- (i- \pp)  \right| .
	\]
\end{proposition}




\begin{proof}
We calculate 
\begin{alignat*}{1}
	\frac{\partial F_1}{\partial \omega} (x) - (i - \pp)  &=
	(i - i\alpha e^{- i \omega}) - (i - \pp) 
	+ \alpha \epsilon ( i e^{i \omega }- 2 e^{- 2 i \omega}) \pi_2 c -i \alpha \epsilon \pi_1  ([ K^{-1} U_\omega c ] *c )\\
	&= -i (\alpha - \pp) e^{-i \omega} - i \pp ( i + e^{-i\omega} )
	+ \alpha \epsilon ( i e^{i \omega }- 2 e^{- 2 i \omega}) \pi_2 c -i \alpha \epsilon \pi_1( [ K^{-1} U_\omega c ] *c) ,
\end{alignat*}
hence, using Lemma~\ref{lem:deltatheta} again,
%\begin{equation*}
%	\left|  \frac{\partial F_1}{\partial \omega} ( x)- (i- \pp)  \right|  \leq
%	\da +  \pp \dw  + 3 \alpha \epsilon \dc + \alpha \epsilon \rho \dc  .
%%	\\ &\leq &   \da + \pp \dt + \alpha \epsilon \dc ( 3 + \rho) 
%\end{equation*}
%\note[JB]{Because of the factor 2 in the norm I think it should be }
\begin{equation*}
	\left|  \frac{\partial F_1}{\partial \omega} ( x)- (i- \pp)  \right|  \leq
	\da +  \pp \dw  + \frac{3}{2} \alpha \epsilon \dc +  \frac{1}{2} \alpha \epsilon \rho \dc  .
\end{equation*}
\end{proof}

%%%%%%%%%%%%%%%%%%%%%%%%%%%%%%%%%%%%%%%%%%%%%%%%%%%%%%%%%%%%%%%%%%%%%%%%%%%%


\begin{proposition}
		\label{prop:Z1c}
	Define
	\[
	f_{1,c} := 
	\epsilon \left(  \da + \tfrac{3 \pi}{4} \dw +  (\pp + \da ) \dc   \right) .
	\]
	Then for all $x= (\alpha,\omega,c) \in B_\epsilon(r,\rho)$
	\[
	f_{1,c} \geq   \left|  \frac{\partial F_1 }{\partial c } (x) \cdot b -  \pp \epsilon (i-1) \pi_2 b \right|, 
	\qquad\text{for all $b\in\ell^K_0$ with $\|b\| \leq 1$}.
	\]
\end{proposition}



\begin{proof}
	We calculate
\begin{alignat*}{1}
		\frac{\partial F_1 }{\partial c } (x) \cdot b -  \pp \epsilon (i-1) \pi_2 b 
	& =
	\epsilon [ \alpha (e^{i \omega} + e^{-2i \omega})  - \pp(i -1)] \pi_2 b 
	+ \alpha \epsilon  \pi_1 \bigl(  [ U_{\omega} b ] * c + [ U_{\omega} c ]*b \bigr)  \\
	%
	&= \epsilon [ (\alpha - \pp) (e^{i \omega} + e^{-2i \omega})  ]  \pi_2 b  +
	\epsilon  \pp [  (e^{i \omega} + e^{-2i \omega})  - (i -1)]  \pi_2 b  \nonumber \\
	& \qquad\quad + \alpha \epsilon \pi_1 \bigl(  [ U_{\omega} b ] * c + [ U_{\omega} c ]*b \bigr)  ,
	\end{alignat*}
hence, for $\|b\| \leq 1$,
%\begin{equation*}
%	\left| 
%	\frac{\partial F_1 }{\partial c } (x) \cdot b -  \pp \epsilon (i-1) \pi_2 b 
%	\right| 
%	\leq
%  \epsilon \left( 2 \da + \pp(\dt + \dtt ) + 2 \alpha \dc   \right)  .
%	\end{equation*}
%\note[JB]{Because of the factor 2 in the norm I think it should be }
\begin{equation*}
	\left| 
	\frac{\partial F_1 }{\partial c } (x) \cdot b -  \pp \epsilon (i-1) \pi_2 b 
	\right| 
	\leq
  \epsilon \left(  \da + \tfrac{\pi}{4}(\dw + 2 \dw ) +  (\pp + \da )  \dc   \right)  .
	\end{equation*}
\end{proof}


%%%%%%%%%%%%%%%%%%%%%%%%%%%%%%%%%%%%%%%%%%%%%%%%%%%%%%%%%%%%%%%%%%%%%%%%%%%%



%%%%%%%%%%%%%%%%%%%%%%%%%%%%%%%%%%%%%%%%%%%%%%%%%%%%%%%%%%%%%%%%%%%%%%%%%%%%
\begin{proposition}
		\label{prop:Zsa}
Define  
	% \[
	% f_{*,\alpha} := \frac{2}{\pi \sqrt{5}} \left( r_c + \epsilon \dt + \epsilon \dc (4 + \dc )  \right) .
	% \]
	% \note[J]{Proposed Change}
		\[
		f_{*,\alpha} := \frac{2}{\pi \sqrt{5}}\left( r_c +  2 \dw (  \dc^0 +  \epsilon )  + \epsilon \dc (4 + \dc )  \right)  .
		\]
	Then for all $x= (\alpha,\omega,c) \in B_\epsilon(r,\rho)$
	\[
		f_{*,\alpha} \geq  \left\| A_{0,*}^{-1} \pi_{\geq 2} \left(
		\frac{\partial F}{\partial \alpha}(x)  +   \epsilon  \tfrac{2 +4 i}{5}  \e_2  \right) \right\|  .
	\]
\end{proposition}

\begin{proof}
We note that $\epsilon  \tfrac{2 +4 i}{5} \e_2= \bc_\epsilon + \epsilon i \e_2$
and calculate 
%	\[
%\pi_{\geq 2} 
%		\frac{\partial F}{\partial \alpha}(x)  +   \epsilon  \tfrac{2 +4 i}{5}  \e_2  =
%	U_\omega (c-\bc_\epsilon) + \epsilon  (e^{-i \omega} +i)\e_2 + \epsilon \pi_{\geq 2} L_\omega c + \epsilon  \pi_{\geq 2} ([ U_\omega c] * c ).
%	\]
%\note[JB]{I don't see that this is correct. It seems a term is missing:}
	\[
\pi_{\geq 2} 
		\frac{\partial F}{\partial \alpha}(x)  +   \epsilon  \tfrac{2 +4 i}{5}  \e_2  =
	U_\omega (c-\bc_\epsilon) + (1+ e^{-2i\omega}) \bc_\epsilon + \epsilon  (e^{-i \omega} +i)\e_2 + \epsilon \pi_{\geq 2} L_\omega c + \epsilon  \pi_{\geq 2} ([ U_\omega c] * c ).
	\]
By using Proposition~\ref{p:severalnorms} and Lemma~\ref{lem:deltatheta}, we obtain the estimate
%\begin{alignat*}{1}
%		\left\| A_{0,*}^{-1} \pi_{\geq 2} \left(
%				\frac{\partial F}{\partial \alpha}(x)  +   \epsilon  \tfrac{2 +4 i}{5}  \e_2  \right) \right\|
%	 &\leq \| A_{0,*}^{-1} \| \left( r_c + \epsilon | e^{-i \omega } +i| + 4 \epsilon \dc + \epsilon \dc^2 \right) \\
%	&\leq \frac{2}{\pi \sqrt{5}}\left( r_c + \epsilon \dt + \epsilon \dc (4 + \dc )  \right) .
%	\end{alignat*}
%\note[JB]{Because of the factor 2 in the norm, and the term I thought was missing, I think it should be }
%\note[J]{Yes, a term was missing.} 
% \begin{alignat*}{1}
% 		\left\| A_{0,*}^{-1} \pi_{\geq 2} \left(
% 				\frac{\partial F}{\partial \alpha}(x)  +   \epsilon  \tfrac{2 +4 i}{5}  \e_2  \right) \right\|
% 	 &\leq \| A_{0,*}^{-1} \| \left( r_c + 2 \epsilon  \frac{\sqrt{20}}{5} |1+ e^{-2i\omega} | +  2 \epsilon | e^{-i \omega } +i| + 4 \epsilon \dc + \epsilon \dc^2 \right) \\
% 	&\leq \frac{2}{\pi \sqrt{5}}\left( r_c + \epsilon \dtt \frac{4}{\sqrt{5}} + 2 \epsilon \dt + \epsilon \dc (4 + \dc )  \right) .
% 	\end{alignat*}
% 	\note[J]{Rearranged some terms}
	\begin{alignat*}{1}
	\left\| A_{0,*}^{-1} \pi_{\geq 2} \left(
	\frac{\partial F}{\partial \alpha}(x)  +   \epsilon  \tfrac{2 +4 i}{5}  \e_2  \right) \right\|
	&\leq \| A_{0,*}^{-1} \| \left( r_c +  \dc^0 |1+ e^{-2i\omega} | +  2 \epsilon | e^{-i \omega } +i| + 4 \epsilon \dc + \epsilon \dc^2 \right) \\
	&\leq \frac{2}{\pi \sqrt{5}}\left( r_c +  2 \dw (  \dc^0 +  \epsilon )  + \epsilon \dc (4 + \dc )  \right) .
	\end{alignat*}
\end{proof}
%%%%%%%%%%%%%%%%%%%%%%%%%%%%%%%%%%%%%%%%%%%%%%%%%%%%%%%%%%%%%%%%%%%%%%%%%%%%




\begin{proposition}
			\label{prop:Zsw}
Define 
%	\begin{alignat}{1}
%		f_{*,\omega} &:=
%		\frac{5}{2 \pi}  r_c 
%		+
%		\tfrac{\epsilon }{\sqrt{5}} \left(  \dt + \tfrac{2}{\pi}\da \right) \nonumber
%		+
%		\tfrac{2}{\pi} \left( \tfrac{5}{4} (\pp + \da ) r_c + \tfrac{4}{5}\da \epsilon \right) 
%		+
%		\tfrac{4 \epsilon}{5} (\dtt) 
%		\\& \qquad + 
%	   \epsilon  \tfrac{2}{ \pi}(\pp + 	 \da ) \left( \frac{1}{\sqrt{5}} ( \dc + r_c) + \frac{5}{4} \left(  \dc  + \tfrac{3}{2}r_c \right)  +  \frac{ \rho \dc   }{\sqrt{5}}\right) . \label{e:fstaromega}
%	\end{alignat}
%	\note[J]{Proposed Change}
%		\begin{alignat}{1}
%		f_{*,\omega} &:=
%		\frac{5}{2 \pi}  r_c 
%		+
%		\tfrac{2}{\sqrt{5}}  \epsilon \left(  \dw + \tfrac{2}{\pi}\da \right) \nonumber
%		+
%		\tfrac{5}{2\pi} \left( \pp r_c + \da ( r_c + \dc) \right) 
%		+
%		\tfrac{8}{5}  \epsilon \dw
%		\\& \qquad + 
%		  \tfrac{2}{ \pi} \epsilon(\pp + 	 \da ) \left( \frac{1}{\sqrt{5}} ( \dc + r_c) + \frac{5}{4} \left(  \dc  + \tfrac{3}{2}r_c \right)  +  \frac{ \rho \dc   }{\sqrt{5}}\right) . \label{e:fstaromega}
%		\end{alignat}
%	\note[JB]{Proposed Rearrangement}
		\begin{alignat}{1}
		f_{*,\omega} &:=
		\tfrac{5}{2 \pi}  (1+ \pp)  r_c 
		+
		\tfrac{2}{\sqrt{5}}  \epsilon \left( (1+\tfrac{4}{\sqrt{5}}) \dw + \tfrac{2}{\pi}\da \right) \nonumber
		+
		\tfrac{5}{2\pi} \da ( r_c + \dc) 
		\\& \qquad + 
		  \tfrac{2}{ \pi} \epsilon(\pp + 	 \da ) \left( \frac{1}{\sqrt{5}} ( \dc + r_c) + \frac{5}{4} \left(  \dc  + \tfrac{3}{2}r_c \right)  +  \frac{ \rho \dc   }{\sqrt{5}}\right) . \label{e:fstaromega}
		\end{alignat}
%
%	\[
%	f_{*,\omega} :=  \frac{5}{2 \pi} ( 1 + \alpha) r_c +  \epsilon \frac{2}{\pi \sqrt{5}} \left( \da + \pp \dt + \alpha   \rho \dc \right) + \frac{2 \alpha \epsilon}{\pi} \left(  \frac{1}{\sqrt{5}} ( \dc + r_c) + \frac{5}{4} \left(  \dc  + \frac{3}{2}r_c \right)  \right)  
%	\]
	Then for all $x= (\alpha,\omega,c) \in B_\epsilon(r,\rho)$
	\[
	f_{*,\omega} \geq  
	 \left\| A_{0,*}^{-1}  \pi_{\geq 2}\left( \frac{\partial F}{\partial \omega}(x) -  \epsilon \left[ \tfrac{4-3\pi}{10} + \tfrac{2(2 + \pi)}{5}i \right] \e_2  \right) \right\| .  
	\]
\end{proposition}

\begin{proof}
We note that 
$
\epsilon  \left[ \tfrac{4-3\pi}{10} + \tfrac{2(2 + \pi)}{5}i \right] \e_2
= i (2+\pi) \bc_\epsilon - \pp \epsilon \e_2
$
and calculate 
\begin{alignat*}{1}
\pi_{\geq 2} \frac{\partial F}{\partial \omega}(x) -  \epsilon \left[ \tfrac{4-3\pi}{10} + \tfrac{2(2 + \pi)}{5}i \right] \e_2   &=
		i K^{-1} ( I - \alpha U_{\omega} ) c  -
		 i  \alpha \epsilon e^{-i \omega} \e_2 + 
		\alpha \epsilon \pi_{\geq 2} L_{\omega}' c \\
		&\qquad - i \alpha \epsilon \pi_{\geq 2} ([ K^{-1} U_\omega c ] *c  ) - i K^{-1} ( I - \pp U_{\omega_0}) \bc_\epsilon +  \pp \epsilon \e_2\\
		&= i K^{-1} ( c - \bc_\epsilon) - \epsilon  (i  \alpha e^{-i \omega}  -  \pp ) \e_2\\
		&\qquad
		-i K^{-1} \left[  U_\omega \left(  \pp ( c - \bc_\epsilon) + ( \alpha - \pp) c  \right) + \left( U_\omega - U_{\omega_0} \right) \pp \bc_\epsilon) \right]
		\\
		&\qquad\qquad
			+ \alpha \epsilon \pi_{\geq 2} L_{\omega}' c 
			- i \alpha \epsilon \pi_{\geq 2} ([ K^{-1} U_\omega c ] *c ) .
\end{alignat*}	
Applying the operator $A_{0,*}^{-1}$ to this expression, we obtain (with $\hat{U}$ defined in~\eqref{e:defUhat})
%\note[J]{Yes, the term $(\alpha-\pp)c_2$ really should have been $(\alpha-\pp)c$.}
%	\[
%	A_{0,*}^{-1} \left( \frac{\partial F}{\partial \omega} ( x_\epsilon + a) - A   \right) =
%	\frac{2 	}{\pi} \hat{U}( I - \alpha U_{\omega}) (c - c_2(\epsilon)) + \epsilon A_{0,*}^{-1} \left( -i  [(\alpha-\pp) e^{_i \omega}]_2 -i [\pp ( e^{-i \omega} + i)]_2 + \alpha L_{\omega}' c - i \alpha [K^{-1} U_{\omega} c]* c 
%	\right)
%	\]
%	
\begin{alignat*}{1}
A_{0,*}^{-1}  \pi_{\geq 2}\left( \frac{\partial F}{\partial \omega}(x) -  \epsilon \left[ \tfrac{4-3\pi}{10} + \tfrac{2(2 + \pi)}{5}i \right] \e_2  \right) 
			&= \frac{2}{\pi} \hat{U}  ( c - \bc_\epsilon) 
			- \frac{2 \epsilon}{i \pi} \hat{U} K  (i  \alpha  e^{-i \omega}  -  \pp  ) \e_2 \\
			&\qquad
			-\frac{2}{\pi} \hat{U} \left[  U_\omega \left(  \alpha ( c - \bc_\epsilon) + ( \alpha - \pp) c  \right) \right]
		    \\	&\qquad\qquad
			-\frac{2}{\pi} \hat{U} \left( U_\omega - U_{\omega_0} \right) \pp \bc_\epsilon  
			\\
			&\qquad\qquad\qquad 
			+ \frac{2 \alpha \epsilon}{i \pi} \hat{U} K \pi_{\geq 2} \left(  L_{\omega}' c - i [ K^{-1} U_\omega c ] *c  \right)  .
	\end{alignat*}
We use the triangle inequality to estimate its norm, splitting it into the  five pieces:
% \note[JB]{There seems to be a factor 2 missing in the second and fourth estimate (due to norm); not implemented yet}
% \note[JB]{I think the second term of the third estimate is not correct: I think it should be $\frac{2}{\pi}\frac{5}{4} \da \dc$. But I did not implement it; I may be overlooking something.} \note[J]{I agree with these changes except for the factor of 2 in the fourth estimate. }
% 	\begin{alignat*}{1}
% 	\left\|	\frac{2}{\pi} \hat{U}  ( c - \bc_\epsilon) \right\|
% 				&\leq \frac{2}{\pi} \frac{5}{4} r_c
% 				= \frac{5}{2 \pi}  r_c \nonumber \\
% 				%
% 				%
% \left\|		- \frac{2 \epsilon}{i \pi} \hat{U} K  (i  \alpha  e^{-i \omega}  -  \pp  ) \e_2  \right\|
% 				&\leq   \frac{2 \epsilon }{\pi} \frac{1}{\sqrt{5}} \left( \pp \dw + \da \right)
% 				=   \tfrac{ \epsilon }{\sqrt{5}} \left(  \dt + \tfrac{2}{\pi}\da \right) \nonumber \\
% 				%
% 				%
% 	\left\| 	-\tfrac{2}{\pi} \hat{U} \left[  U_\omega \left(  \alpha ( c - \bc_\epsilon) + ( \alpha - \pp) c  \right) \right]  \right\|
% 				& \leq \tfrac{2}{\pi} \left( \tfrac{5}{4} \alpha r_c + \tfrac{2}{\sqrt{5}}\da \tfrac{2 \epsilon}{\sqrt{5}} \right)
% 				= \tfrac{2}{\pi} \left( \tfrac{5}{4} \alpha r_c + \tfrac{4}{5}\da \epsilon \right) \nonumber \\
% 				%
% 				%
% 		\left\| -\tfrac{2}{\pi} \hat{U}
% 		\left( U_\omega - U_{\omega_0} \right) \pp \bc_\epsilon   \right\|
% 				&\leq  \tfrac{2}{\pi}  \tfrac{2}{\sqrt{5}} \dtt \pp \tfrac{2 \epsilon}{\sqrt{5}}
% 				=   \tfrac{4 \epsilon}{5} \dtt \nonumber \\
% 				%
% 				%
% 		\left\|\tfrac{2 \alpha \epsilon}{i \pi} \hat{U} K \pi_{\geq 2} \left(  L_{\omega}' c - i [ K^{-1} U_\omega c ] *c  \right)  \right\|
% 		&\leq \frac{2 \alpha \epsilon}{ \pi}  \left( \| \hat{U}  K \pi_{\geq 2} L_{\omega}' c  \| +  \frac{ \rho \dc   }{\sqrt{5}}\right) ,
% 	\end{alignat*}
%
% \note[J]{Below are my changes}
	\begin{alignat*}{1}
	\left\|	\frac{2}{\pi} \hat{U}  ( c - \bc_\epsilon) \right\|
	&\leq \frac{2}{\pi} \frac{5}{4} r_c 
	= \frac{5}{2 \pi}  r_c \nonumber \\
	%
	%
	\left\|		- \frac{2 \epsilon}{i \pi} \hat{U} K  (i  \alpha  e^{-i \omega}  -  \pp  ) \e_2  \right\|
	&\leq   \frac{4 \epsilon }{\pi} \frac{1}{\sqrt{5}} \left( \pp \dw + \da \right)  
	=   \tfrac{2 \epsilon }{\sqrt{5}} \left(  \dw + \tfrac{2}{\pi}\da \right) \nonumber \\
	%
	%
	\left\| 	-\tfrac{2}{\pi} \hat{U} \left[  U_\omega \left(  \alpha ( c - \bc_\epsilon) + ( \alpha - \pp) c  \right) \right]  \right\|
	& \leq \tfrac{2}{\pi}  \tfrac{5}{4} \left(  (\pp + \da) r_c + \da \dc  \right)
	= \tfrac{5}{2\pi} \left( \pp r_c + \da ( r_c + \dc) \right) \nonumber \\
	%
	%
	\left\| -\tfrac{2}{\pi} \hat{U} 
	\left( U_\omega - U_{\omega_0} \right) \pp \bc_\epsilon   \right\| 
	&\leq  \tfrac{2}{\pi}  \tfrac{2}{\sqrt{5}} (2 \dw)  \pp \tfrac{2 \epsilon}{\sqrt{5}} 
	=   \tfrac{8 \epsilon}{5} \dw \nonumber \\
	%
	%
	\left\|\tfrac{2 \alpha \epsilon}{i \pi} \hat{U} K \pi_{\geq 2} \left(  L_{\omega}' c - i [ K^{-1} U_\omega c ] *c  \right)  \right\|
	&\leq \frac{2 \alpha \epsilon}{ \pi}  \left( \| \hat{U}  K \pi_{\geq 2} L_{\omega}' c  \| +  \frac{ \rho \dc   }{\sqrt{5}}\right) ,
	\end{alignat*} 
where we have used Proposition~\ref{p:severalnorms} and Lemma~\ref{lem:deltatheta}.
%%	Multiplying by the inverse and taking norms 
%%	\begin{eqnarray}
%%	\left\| A_{0,*}^{-1}  \frac{\partial F}{\partial \omega} ( x_\epsilon + a) - A   \right\| &\leq&
%%	\frac{5}{2 \pi} ( 1 + \alpha) r_c +  \epsilon \| A_{0,*}^{-1} \| ( \da + \pp | e^{-i \omega} +i |) + \frac{2 \alpha \epsilon}{\pi} \| \hat{U} K L_{\omega}' c \| + \alpha \epsilon \| A_{0,*}^{-1} \|  \rho \dc \nonumber \\
%%	&\leq & 
%%	\frac{5}{2 \pi} ( 1 + \alpha) r_c +  \epsilon \frac{2}{\pi \sqrt{5}} ( \da + \pp \dt + \alpha   \rho \dc ) + \frac{2 \alpha \epsilon}{\pi} \| \hat{U} K L_{\omega}' c \| \nonumber 
%%	\end{eqnarray}
Finally, we estimate 
	\begin{alignat}{1}
	\left\| \hat{U} K \pi_{\geq 2} L_{\omega}' c \right\| &= 
	\left \| \hat{U}  K \pi_{\geq 2} \left(- i \sigma^+( e^{- i \omega} I + K^{-1} U_{\omega}) + i \sigma^-(e^{i \omega} I - K^{-1} U_{\omega}) \right) c \right\| \nonumber \\
	&\leq \left \| \hat{U}  K \pi_{\geq 2}( \sigma^+ + \sigma^- ) c  \right \| + 
	\left \| \hat{U} \pi_{\geq 2} K ( \sigma^+ + \sigma^- ) K^{-1}  U_{\omega} c  \right \|  \nonumber  \\
	&\leq  \frac{1}{\sqrt{5}} ( \| \sigma^+ c\| + \|\pi_{\geq 2}\sigma^- c\|) + \frac{5}{4} \left( \| K \sigma^+ K^{-1} \| \dc  +  \| \pi_{\geq 2} K \sigma^- K^{-1} \| r_c \right) \nonumber  \\
	&\leq  \frac{1}{\sqrt{5}} ( \dc + r_c) + \frac{5}{4} \left(  \dc  + \frac{3}{2}r_c \right)  .  \label{e:longestimate}
	\end{alignat}
Hence, with $f_{*,\omega}$ as defined in~\eqref{e:fstaromega},
% 	So if we define $f_{*,\omega}$ as below
% 	\begin{eqnarray}
% f_{*,\omega} &:=&
% 	\frac{5}{2 \pi}  r_c
% +
%  \tfrac{\epsilon }{\sqrt{5}} \left(  \dt + \tfrac{2}{\pi}\da \right) \nonumber
% +
%  \tfrac{2}{\pi} \left( \tfrac{5}{4} \alpha r_c + \tfrac{4}{5}\da \epsilon \right)
%  +
%   \tfrac{4 \epsilon}{5} (\dtt)
%   \\&&+
%   \frac{2 \alpha \epsilon}{ \pi}  \left( \frac{1}{\sqrt{5}} ( \dc + r_c) + \frac{5}{4} \left(  \dc  + \tfrac{3}{2}r_c \right)  +  \frac{ \rho \dc   }{\sqrt{5}}\right)
% 	\end{eqnarray}
%
it follows that 	
	\[
	 \left\| A_{0,*}^{-1}  \pi_{\geq 2}\left( \frac{\partial F}{\partial \omega}(x) -  \epsilon \left[ \tfrac{4-3\pi}{10} + \tfrac{2(2 + \pi)}{5}i \right] \e_2  \right) \right\| \leq 	f_{*,\omega} .
		\]
	
	
	
%%%%	\begin{eqnarray}
%%%%	\left\| A_{0,*}^{-1}  \frac{\partial F}{\partial \omega} ( x_\epsilon + a) - A   \right\| &\leq & 
%%%%	\frac{5}{2 \pi} ( 1 + \alpha) r_c +  \epsilon \frac{2}{\pi \sqrt{5}} \left( \da + \pp \dt+ \alpha   \rho \dc \right) + \frac{2 \alpha \epsilon}{\pi} \left(  \frac{1}{\sqrt{5}} ( \dc + r_c) + \frac{5}{4} \left(  \dc  + \frac{3}{2}r_c \right)  \right)  \nonumber
%%%%	\end{eqnarray}
%%%%	Hence
%%%%	\[
%%%%	f_{*,\omega} \leq  \frac{5}{2 \pi} ( 1 + \alpha) r_c +  \epsilon \frac{2}{\pi \sqrt{5}} \left( \da + \pp \dt + \alpha   \rho \dc \right) + \frac{2 \alpha \epsilon}{\pi} \left(  \frac{1}{\sqrt{5}} ( \dc + r_c) + \frac{5}{4} \left(  \dc  + \frac{3}{2}r_c \right)  \right)  
%%%%	\]
\end{proof}


%%%%%%%%%%%%%%%%%%%%%%%%%%%%%%%%%%%%%%%%%%%%%%%%%%%%%%%%%%%%%%%%%%%%%%%%%%%%

\begin{proposition}
			\label{prop:Zsc}
Define 
% \note[J]{Below is the old bound.}
% 	\[
% 	f_{*,c} := \left[ \frac{5}{2} ( \tfrac{1}{2} + \tfrac{1}{\pi}) \dw +  \frac{\da }{\sqrt{5}} \right]
% 	+\epsilon \left[ \frac{8}{\pi \sqrt{5}} \da + \frac{2}{\sqrt{5}} \dt  + \frac{25}{8} \dw + \frac{2 (\pp+\da)  \dc}{\pi \sqrt{5}} \right]  .
% 	\]
% \note[J]{Below is the new bound. }
	\[
	f_{*,c} :=	\left[ \frac{5}{2} \left( \frac{1}{2} + \frac{1}{\pi} \right) \dw +  \frac{\da }{\sqrt{5}} \right] 
 +\epsilon \left[ \frac{8}{\pi \sqrt{5}} \da + \left(  \frac{2}{\sqrt{5}}   + \frac{25}{8} \right) \dw + \frac{4 (\pp+\da)  \dc}{\pi \sqrt{5}} \right] .
	\]
	Then for all $x= (\alpha,\omega,c) \in B_\epsilon(r,\rho)$
	\[
	f_{*,c} \geq    \left\|  A_{0,*}^{-1} \pi_{\geq 2} \left( \frac{\partial F}{\partial c}(x) \cdot b  - (A_{0,*} + \epsilon A_{1,*}) b \right) 
	\right\| , \qquad\text{for all $b\in\ell^K_0$ with $\|b\| \leq 1$}.
	\]
\end{proposition}


\begin{proof}
We write $A_* := A_{0,*} + \epsilon A_{1,*}$ and calculate
\begin{alignat*}{1}
\frac{\partial F}{\partial c} (x) \cdot b - A_* b 
& = 
 \bigl[( i \omega K^{-1} + \alpha U_{\omega}) - ( i \pp K^{-1} + \pp U_{\omega_0}) \bigr] b  + \alpha \epsilon  L_\omega b  - \pp \epsilon L_{\omega_0} b 
 \\& \qquad
	+ \alpha \epsilon \left[ [ U_\omega b] * c + [U_{\omega} c ]*b \right]
 \\ & =
	\bigl[ i ( \omega - \pp ) K^{-1} + ( \alpha - \pp) U_{\omega} + \pp ( U_{\omega} - U_{\omega_0}) \bigr] b 
	\\ & \qquad 
	+ \epsilon \bigl[ ( \alpha - \pp) L_{\omega} + \pp ( L_{\omega} - L_{\omega_0}) \bigr] b 
%	\\ & \qquad\qquad 
+ \alpha \epsilon \left( [U_{\omega } b] * c + [ U_{\omega } c ]*b \right) . 
\end{alignat*}
Hence, for $\|b\| \leq 1$, 
\begin{alignat}{1}
	\left\| A_{0,*}^{-1} \pi_{\geq 2} \left(   \frac{\partial F}{\partial c} (x) \cdot b -  A_* b  \right)  \right\|
	 &\leq 
	 \dw  \| A_{0,*}^{-1} K^{-1} \| + \pp \da \| A_{0,*}^{-1}  \| + \pp \| A_{0,*}^{-1}  ( U_{\omega} - U_{\omega_0}) \| \nonumber  \\
	& \qquad
	 + \epsilon  \left[ 4  \da \|  A_{0,*}^{-1}   \| + \pp \| A_{0,*}^{-1} \pi_{\geq 2} ( L_{\omega} - L_{\omega_0}) \| 
%	\\ & \qquad \qquad 
	+  2 \alpha \dc \| A_{0,*}^{-1}  \|  \right]  \label{e:intermediate},
\end{alignat}	
where all norms should be interpreted as operators on $\ell^1_0$.
	Since
$\frac{\partial U_\omega}{\partial  \omega} = - i K^{-1} U_{\omega}$
and $ A_{0,*}^{-1} = \frac{2}{i\pi} \hat{U} K$, it follows from Proposition~\ref{p:severalnorms} that  
\begin{equation}\label{e:AUoUo0}
	\| A_{0,*}^{-1}  (U_{\omega} - U_{\omega_0})  \| \leq  \frac{2}{\pi}  \dw \| \hat{U} \| 
	= \frac{5}{2 \pi } \dw .
\end{equation}
Next, we compute
	\begin{alignat*}{1}
	L_{\omega} - L_{\omega_0}  &= \sigma^+ \left[ (e^{-i \omega} + i) I + (U_{\omega} - U_{\omega_0})\right] + \sigma^- \left[ (e^{i \omega} - i) I + (U_{\omega} - U_{\omega_0}) \right] \\
	&=  (e^{-i \omega} + i)  \sigma^+ - i e^{i\omega} (i+e^{-i \omega})\sigma^- 
	+ (\sigma^+ + \sigma^-) (U_{\omega} - U_{\omega_0}) .
\end{alignat*}
Analogous to~\eqref{e:longestimate} and~\eqref{e:AUoUo0} we infer that
\begin{equation*}
	\|  A_{0,*}^{-1} \pi_{\geq 2} ( L_{\omega} - L_{\omega_0} ) \| \leq  \frac{4}{\pi \sqrt{5}} |i+ e^{-i \omega} | + \frac{5}{\pi} \| \hat{U} \| \dw \\
	\leq  \frac{4 }{\pi \sqrt{5}}\dw    + \frac{25}{4 \pi}  \dw .
\end{equation*} 
Finally, by putting all estimates together and once again using Proposition~\ref{p:severalnorms}, it follows from~\eqref{e:intermediate} that 
%\note[JB]{Probably factor 2 missing in final term.}\note[J]{Added factor of 2. }
	\begin{alignat*}{1}
\left\| A_{0,*}^{-1} \pi_{\geq 2} \left(   \frac{\partial F}{\partial c} (x) \cdot b -  A_* b  \right)  \right\|
	% &\leq
	% \left[ \dw  \| A_{0,*}^{-1} K^{-1} \| + \pp \da \| A_{0,*}^{-1}  \| + \pp \| A_{0,*}^{-1}  ( U_{\omega} - U_{\omega_0}) \| \right] \nonumber \\
	% & + \epsilon  \left[ 4  \da \|  A_{0,*}^{-1}   \| + \pp \| A_{0,*}^{-1}  ( L_{\omega} - L_{\omega_0}) \| + \alpha \dc \| A_{0,*}^{-1}  \|  \right] \\
	&\leq
	 \left[ \frac{5}{2} \left( \frac{1}{2} + \frac{1}{\pi} \right) \dw +  \frac{\da }{\sqrt{5}} \right] 
	 \\
	& \qquad  +\epsilon \left[ \frac{8}{\pi \sqrt{5}} \da + \left(  \frac{2}{\sqrt{5}}   + \frac{25}{8} \right) \dw + \frac{4 (\pp+\da)  \dc}{\pi \sqrt{5}} \right] .
	\end{alignat*}
\end{proof}

%%%%%%%%%%%%%%%%%%%%%%%%%%%%%%%%%%%%%%%%%%%%%%%%%%%%%%%%%%%%%%%%%%%%%%%%%%%%


