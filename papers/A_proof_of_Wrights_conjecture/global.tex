%!TEX root = hopfwright.tex

When deriving global results from the local results in
Section~\ref{s:local}, we need to take into account that there are some obvious
reasons why the branch of periodic solutions, described by
$F_\epsilon(\alpha,\omega,c)=0$, bifurcating from the Hopf bifurcation point at
$(\alpha,\omega)=(\pp,\pp)$ does not describe the entire set of periodic
solutions for $\alpha$ near $\pp$. First, there is the trivial solution. In
particular, one needs to quantify in what sense the trivial solution is an
isolated invariant set. This is taken care of by Remarks~\ref{r:smalleps}
and~\ref{r:cone}, which show there are no ``spurious'' small solutions in the
parameter regime of interest to us (roughly as long as we stay away from the
next Hopf bifurcation at $\alpha = \tfrac{5\pi}{2}$). Second, one can interpret any periodic
solution with frequency $\omega$ as a periodic solution with frequency
$\omega/N$ as well, for any $N \in \mathbb{N}$. Since we are working in Fourier space,
showing that there are no ``spurious'' solutions with lower frequency would
require us to perform an analysis near $(\alpha,\omega)=(\pp,\tfrac{\pi}{2N})$
for all $N \geq 2$. This obstacle can be avoided by bounding (from below)
$\omega$ away from $\pi/4$. This is done in Lemma~\ref{lem:omegalarge}.

For later use, we recall an elementary Fourier analysis bound. 
%\marginpar{There must be some theory about the best constant here?}
\begin{lemma}\label{lem:fourierbound}
	Let $y \in C^1$ be a periodic function of period $2\pi/\omega$ with Fourier coefficients $\c \in \ell^1_\sym$ (in particular this means $\c_0=0$), as described by~\eqref{eq:FourierEquation}. 
	Then 
\[
 \| \c \| \leq \sqrt{\frac{\pi }{6 \omega}}\,  \| y' \|_{L^2([0,2\pi/\omega])}
\qquad\text{and}\qquad
 \| \c \| \leq \frac{\pi}{\omega\sqrt{3}}\, \|y'\|_\infty.
 \]
\end{lemma}
\begin{proof}
From the Cauchy-Schwarz inequality and  Parseval's identity it follows that
\begin{alignat*}{1}
		\| \c \| &= 2 \sum_{k=1}^{\infty} |\c_k|
	%	 &=& 
	%	2 \sum_{k=1}^{\infty} | c_k| \\ 
	%	&=&	2 \sum_{k=1}^{\infty} k^{-1} \cdot | k \, c_k| \\
		\leq 2 \left( \sum_{k=1}^{\infty} k^{-2} \right)^{1/2}
		\left( \sum_{k=1}^{\infty} |k \, \c_k|^2 \right)^{1/2} \\
     &=  \frac{\sqrt{2}}{\omega} \left(\frac{\pi^2}{6} \right)^{1/2} 
	 	 \left(2 \sum_{k=1}^{\infty} |i \omega k \, \c_k|^2 \right)^{1/2}
		 = \frac{\pi}{\omega \sqrt{3}} 
		\left(\sum_{k \in \Z} |i \omega k \, \c_k|^2 \right)^{1/2}\\
		&= \frac{\pi}{\omega \sqrt{3}} 
		\left( \frac{\omega}{2\pi} \int_0^{2\pi/\omega} | y'(t)|^2 dt  \right)^{1/2} 
		\leq \frac{\pi}{\omega \sqrt{3}} \,  \|y'\|_\infty.
\end{alignat*}	
\end{proof}

% THIS NEEDS TO BE MENTIONED SOMEWHERE ELSE (IN EARLIER SECTION)
% Beside these more or less obvious obstacles, there is a more technical hurdle.
% The local analysis in Section~\ref{} gives us a unique solution branch $c=c(\epsilon)$ in some box $\| c - c(\epsilon) \| \leq r_c$. This translates to a unique solution curve $\tc=\tc(\epsilon)$ in some \emph{cone} $\|\tc\| \leq \epsilon r_c$. Lemma~\ref{} shows that if there are solutions outside this cone then $\| \tc \|$ must be large. In particular there are no solutions with small $\tc$ outside the cone (except for the trivial solution), provided $r_c > something$

\subsection{A proof of Wright's conjecture} 


Based on the work in \cite{neumaier2014global} and \cite{wright1955non}, in order to prove Wright's conjecture it suffices to prove that there are no slowly oscillating periodic solutions (SOPS) to Wright's equation for $ \alpha \in [1.5706,\pp]$. Moreover, in \cite{neumaier2014global} it was shown that no SOPS with $\| y \|_\infty \geq e^{0.04}-1$ exists for  $\alpha \in [1.5706,\pp]$. These results are summarized in the following proposition.

\begin{proposition}[\cite{neumaier2014global,wright1955non}]
\label{prop:neumaier}
Assume $y$ is a SOPS to Wright's equation for some $\alpha \leq \pp$. Then $\alpha \in [1.5706,\pp]$
and $\| y \|_\infty \leq e^{0.04}-1$. 
\end{proposition}
For convenience we introduce
\[
  \mu := e^{0.04}-1 \approx 0.0408.
\]
We now derive a lower bound on the frequency $\omega$ of the SOPS.
\begin{lemma}\label{lem:omegalarge}
Let $\alpha \in [1.5706,\pp]$.
Assume $y$ is a SOPS to Wright's equation with minimal period $2\pi/\omega$,
and assume that $\| y \|_\infty \leq \mu$.
Then $\omega \in [1.11,1.93]$.
\end{lemma}

\begin{proof}
%	\marginpar{JJ: I've written a new proof.}
	Without loss of generality, we assume in this proof that $ y(0) =0$, that $y(t) < 0$ for $t\in (-t_{-},0)$ and that $y(t) > 0$ for $t\in (0,t_+)$. 
	We will show that $t_-$ and $t_+$ are bounded by
	\begin{alignat*}{1}
	1+ \frac{1}{\alpha }  \frac{\log (1 + \mu)}{\mu}  < t_+ &<2 + \frac{1}{\alpha} , \\
	1+\frac{1}{\alpha} <t_- & < 3 .
	\end{alignat*}
	The lower bounds for both $t_-$ and $t_+$ follow directly from  Theorem 3.5 in \cite{jones1962nonlinear}. While Theorem 3.5 in~\cite{jones1962nonlinear} assumes $ \alpha \geq \pp$, this part of the theorem simply relies on Lemma 2.1 in \cite{jones1962nonlinear}, which only requires $ \alpha > e^{-1}$.
	

To obtain an upper bound on $t_+$, assume that $ t_+ \geq 2$. Set $t'_+ =\min\{t_+,3\}$. Then it follows from~\eqref{eq:Wright} that $y'(t) < 0$ for $t\in (1,t'_+]$, hence  $y(t-1) > y(2)$ for $ t \in [2,t'_+]$. We infer that for $t \in [2,t'+]$ we have 
$y'(t) = - \alpha y(t-1) [1+y(t)] < - \alpha y(2)$.
Solving the IVP $y'(t) < -\alpha y(2)$ with the initial condition $y(2) = y(2)$, we see that $y(t)$ hits $0$ before $t=2+\frac{1}{\alpha}$. Since $\alpha > 1$ (hence $2+\frac{1}{\alpha} < 3$), this implies that $t'_+=t_+$ and $t_+ < 2+\frac{1}{\alpha}$.
 
	

	To obtain the upper bound on $t_-$, assume for the sake of contradiction
	 that $ t_- \geq 3$. 
	 Then it follows from~\eqref{eq:Wright} that $y'(t) \geq 0$ for $t\in [-2,0]$, hence  $y(t) \leq y(-1)$ for $ t \in [-2,-1]$, and  $y'(t) \geq - \alpha y(-1) [1+y(t)]$ for $ t \in [-1,0]$.  
	Solving this IVP with the initial condition $y(-1) = -\nu$, we obtain $ y(t) \geq (1-\nu) e^{ \alpha \nu (t+1)}-1 $ for $ t \in [-1,0]$, and in particular  $y(0) \geq (1-\nu ) e^{-\alpha \nu}-1$. 
	By assumption $y(0)=0$ and $\nu=|y(-1)| \leq \mu$,
	but $  (1-\nu ) e^{-\alpha \nu}-1>0 $ 
	for $ \nu  \in (0,\mu]$
	and $\alpha \in [1.5706,\pp]$, a contradiction. Thereby $ t_- <3$. 


The bound on $\alpha$ implies that 
 the minimal period $L = t_+ + t_-$ of the SOPS must lie in $[ 3.26,5.64]$.
It then follows that $ \omega \in [1.11,1.93]$	
\end{proof}

%	Let $ t_+ $ denote the amount of time the SOPS $y(t)$ is positive and $ t_-$ denote the amount of time a SOPS is negative (measured over one minimal period). 
%
%First we show that $t_- <3$. 
%Theorem 3.4 in \cite{jones1962nonlinear} shows that if $t_- \geq 3$ then 
%$ \min y \geq  e^{-(\alpha-1)}-1$ which, for our range of $\alpha$, is within  $[0.43,0.44]$. 
%Since $\|y\|_\infty \leq \mu$, this contradiction shows that $ t_- < 3$. 


It turns out that this bound on $\omega$ can (and needs to be) sharpened.
This is the purpose of the following lemma, 
which considers solutions in  unscaled variables. 
\begin{lemma}\label{lem:ZeroNBD}
Suppose $ \tilde{F}_\epsilon(\alpha,\omega,\tc)=0$. If $\omega \in
[1.1,2]$ and $ \alpha \in [1.5,2.0]$  
%(QUITE ARBITRARY BOUNDS!)\marginpar{some illumination needed} 
then
\begin{equation}\label{e:tighterboundonomega}
   \frac{\sqrt{(\omega- \alpha)^2 + 2 \alpha \omega(1-\sin\omega)}}{2\alpha} 
   \leq 2 \epsilon + \| \tc \| .
\end{equation}
\end{lemma}
\begin{proof}
This follows from Proposition~\ref{prop:zeroneighborhood2} in
Appendix~\ref{appendix:aprioribounds}, combined with Proposition \ref{prop:G1Minimizer},  which shows that for
$\omega \in [1.1,2.0]$  and $ \alpha \in [1.5,2.0]$, the minimum in Equation~\eqref{e:minoverk} is attained for $k=1$.
%%\marginpar{Jonathan: need to say something about why this is true.}
\end{proof}

Next we derive bounds on $\epsilon$ and $\tc$, which also lead to improved bounds on $\omega$.
\begin{lemma}\label{lem:wrightbounds}
Let $\alpha \in [1.5706,\pp]$. Assume $y$ is a SOPS with $\| y \|_\infty \leq \mu$.
Then $y$ corresponds, through the Fourier representation~\eqref{e:yc}, to a zero of $F_\epsilon(\alpha,\omega,c)$ with $|\omega- \pp| \leq 0.1489$ and
\[
  0< \epsilon \leq \epseps := \mu/\sqrt{2} \leq 0.02886 ,
\] 
and 
%\change[J]{$\| c \| \leq 0.0398$}
$\| c \| \leq 0.0796$ 
and 
%\change[J]{$\| K^{-1} c \| \leq 0.08 $.}
$\| K^{-1} c \| \leq  0.16 $.
\end{lemma}
\begin{proof}
First consider the Fourier representation~\eqref{e:ytc} of $y$ in unscaled variables. 
Recall that $\c_0$ vanishes (see Remark~\ref{r:a0}).
Since $|y'(t)| \leq \alpha |y(t-1)| (1+|y(t)|) \leq \alpha \mu (1+\mu)$
we see from  Lemma~\ref{lem:fourierbound} that 
\begin{equation}\label{e:epsilontc} 
  2 \epsilon + \| \tc \|  \leq \frac{\pi}{\omega\sqrt{3}} \alpha \mu (1+\mu).
\end{equation}
Combining this with Lemma~\ref{lem:ZeroNBD} leads to the inequality
\begin{equation}
\label{eq:APomegaBound}
	\omega 	\sqrt{(\omega- \alpha)^2 + 2 \alpha \omega ( 1- \sin \omega)} 
	\leq 
	\tfrac{2 \pi}{ \sqrt{3}}  \alpha^2 \mu ( 1 + \mu).
\end{equation}
In the Mathematica file \cite{mathematicafile} we show that when $ \alpha \in [ 1.5706,\pp]$, then inequality \eqref{eq:APomegaBound} is violated for any $\omega \in [ 1.1,2.0] \, \backslash \, [1.4219, 1.6887]$.
From Lemma~\ref{lem:omegalarge} we obtain the a priori  bound
$\omega \in [1.11,1.93]$, whereby it follows that  $\omega \in [1.4219,1.6887]$, and in particular  $|\omega - \pp| \leq 0.1489$. 


Using this sharper bound on $\omega$ as well as $\alpha \in [1.5706,\pp]$
we conclude from~\eqref{e:epsilontc} that
\begin{equation}
	2 \epsilon + \| \tc \|  \leq \frac{\pi}{\omega\sqrt{3}} \alpha \mu (1+\mu)
	\leq \frac{2\omega - \alpha}{\alpha}.
	%  \ref{-- This equation is refered to in the Mathematica File --}
\end{equation} 
Since we also infer that $\alpha < 2\omega$,  Theorem~\ref{thm:FourierEquivalence3}(b) shows that the solution corresponds to a zero of $F_\epsilon(\alpha,\omega,c)$, with $\tc = \epsilon c $.
We can improve the bound on $\epsilon$ from~\eqref{e:epsilontc}
% \eqref{e:epsilontc} 
by observing that
\[
	( \epsilon^2 + \epsilon^2)^{1/2} 
	\leq \left( \sum_{k\in\Z} |\c_k|^2 \right)^{1/2}
	 =  \left( \frac{\omega}{2\pi} \int_0^{2\pi/\omega}
	                         |y|^2 dt  \right)^{1/2} \leq \mu .
\]
Hence $\epsilon \leq \epseps := \mu/\sqrt{2}$.

Finally, we derive the bounds on $c$. Namely, for $\alpha \in  [1.5706,\pp]$,
$\omega \in [1.4219,1.6887]$ and $\epsilon \leq \epseps$,
we find that $b_*$ and  $z_*^+$, as defined in~\eqref{e:zstar}, are bounded below by $ b_* \geq 0.364$ and 
$z^+_* \geq 0.72 $. 
Since it follows from~\eqref{e:epsilontc} that 
$ \| \tc \|  \leq 0.09 $  
in the same parameter range  of  $\alpha$ and $\omega$, we infer from Lemma~\ref{lem:Cone}(a) that 
$\| \tc \| \leq z^-_* $.
Via an interval arithmetic computation, the latter can be bounded above
using Lemma~\ref{lem:ZminusBound}, for $\alpha \in  [1.5706,\pp]$,
$\omega \in [1.4219,1.6887]$ and $\epsilon \leq \epseps$, by
%\change[J]{$z_*^- \leq 0.0398 \epsilon$. }
$z_*^- \leq 0.0796 \epsilon$. 
Hence 
%\change[J]{$\| c \| \leq z_*^- / \epsilon \leq 0.0398$.}
$\| c \| \leq z_*^- / \epsilon \leq 0.0796$.
Furthermore, Lemma~\ref{lem:Cone}(b) implies 
the  bound 
%\change[J]{$\| K^{-1} c \| \leq (\epsilon^2 + (z_*^-)^2 )/(\epsilon b_*) \leq 2.76 \epsilon$}
%\change[J]{$\| K^{-1} c \| \leq (\epsilon^2 + (z_*^-)^2 )/(\epsilon b_*) \leq 2.77 \epsilon$.}
$\| K^{-1} c \| \leq (2\epsilon^2 + (z_*^-)^2 )/(\epsilon b_*) \leq 5.52 \epsilon$.
Since $\epsilon \leq \epsilon_*$, it then follows that 
%\change[J]{$\| K^{-1} c \| \leq 0.08 $.}
$\| K^{-1} c \| \leq  0.16 $.
\end{proof}

With these tight bounds on the solutions, we are in a position to apply the local bifurcation result formulated in Proposition~\ref{prop:TightEstimate} to prove the ultimate step of Wright's conjecture.

\begin{theorem}
	\label{thm:WrightConjecture}
	For $ \alpha \in [0,\pp]$ there is no SOPS to Wright's equation.
\end{theorem}
\begin{proof}
By Proposition~\ref{prop:neumaier} (see also the introduction of this paper) it suffices to exclude a slowly oscillating solution $y$ for $\alpha \in [1.5706,\pp]$ with $\| y \|_\infty \leq \mu$.
By Lemma~\ref{lem:wrightbounds}, if such a solution would exist, it corresponds to a solution of 
$F_\epsilon(\alpha,\omega,c)=0$ with $|\omega- \pp| \leq 0.1489$, 
$0< \epsilon \leq \epseps = \mu / \sqrt{2}$, 
%\change[J]{$\| c \| \leq 0.0398 $}
$\| c \| \leq 0.0796 $ 
and 
%\change[J]{$\| K^{-1} c \| \leq 0.08 $.}
$\| K^{-1} c \| \leq  0.16 $.
We claim that no such solution exists.
Indeed, we define the set 
% \[
%   S :=  \{ (\alpha,\omega,c) \in X :
%   |\alpha - \pp| \leq 0.0002; \,
%    |\omega - \pp| \leq 0.1489; \,
%    \| c \| \leq 0.04; \,
%    \|K^{-1} c \| \leq 0.08  \}.
% \]
% \note[J]{Version with new $|c|$ below. Also propose simplifying $r_\omega$. }
% \[
% S :=  \{ (\alpha,\omega,c) \in X :
% |\alpha - \pp| \leq 0.0002; \,
% |\omega - \pp| \leq 0.15; \,
% \| c \| \leq 0.08; \,
% \|K^{-1} c \| \leq 0.08  \}.
% \]
% \note[J]{Proposed change}
\[
S :=  \{ (\alpha,\omega,c) \in X : 
|\alpha - \pp| \leq 0.0002; \,
|\omega - \pp| \leq 0.15; \, 
\| c \| \leq 0.08; \, 
\|K^{-1} c \| \leq 0.16  \}.
\]
To show that there is no SOPS for $\alpha \in [1.5706,\pp]$, it now suffices to show that all zeros of $F_\epsilon(\alpha,\omega,c)$ in $S$ for any $0< \epsilon \leq \epseps$ satisfy $\alpha> \pp$.

Let us consider
$B_\epsilon(r,\rho)$, which is centered at $\bx_\epsilon$ (see Definition~\ref{def:xepsilon})
with $r$ and $ \rho$ taken as in Proposition~\ref{prop:bigboxes}(a).
In the Mathematica file~\cite{mathematicafile} 
we check that the following inequalities are satisfied:
%\note[JB]{Are these checked in the Mathematica file? Perhaps refer to that?}
% \begin{alignat*}{1}
% 	r_\alpha &= 0.21 \geq  0.0002 + |\balpha_{\epseps}-\pp|,\\
%  r_\omega &= 0.16 \geq  0.1489 + |\bomega_{\epseps}-\pp| ,\\
%  r_c &= 0.09 \geq   0.04 + \| \bc_{\epseps}\|,\\
%  \rho &= 1.01 \geq 0.08 .
% \end{alignat*}
% \note[J]{Version with new numbers below.}
% \begin{alignat*}{1}
% r_\alpha &= 0.13 \geq  0.0002 + |\balpha_{\epseps}-\pp|,\\
% r_\omega &= 0.17 \geq  0.15 + |\bomega_{\epseps}-\pp| ,\\
% r_c &= 0.17 \geq   0.08 + \| \bc_{\epseps}\|,\\
% \rho &= 1.78 \geq 0.08 .
% \end{alignat*}
% \note[J]{Proposed change}
\begin{alignat*}{1}
r_\alpha &= 0.13 \geq  0.0002 + |\balpha_{\epseps}-\pp|,\\
r_\omega &= 0.17 \geq  0.15 + |\bomega_{\epseps}-\pp| ,\\
r_c &= 0.17 \geq   0.08 + \| \bc_{\epseps}\|,\\
\rho &= 1.78 \geq 0.16 . 
\end{alignat*}
By the triangle inequality we obtain that $S \subset B_\epsilon(r,\rho)$ for all $0<\epsilon\leq \epseps$.
Proposition~\ref{prop:bigboxes}(a) shows that for each $0<\epsilon\leq \epseps$
there is a unique zero $\hat{x}_\epsilon=
(\hat{\alpha}_\epsilon,\hat{\omega}_\epsilon,\hat{c}_\epsilon) \in B_\epsilon (r,\rho)$ of $F_\epsilon$.
By Proposition~\ref{prop:TightEstimate} and Remark~\ref{r:nested}
this zero satisfies $\hat{\alpha}_\epsilon > \pp$.
Hence, for any $0<\epsilon\leq\epseps$ the only zero of $F_\epsilon$ in $S$ (if there is one) satisfies $\alpha>\pp$. This completes the proof.

\end{proof}
% 
%\begin{eqnarray}
%	\|c \|_{\ell^1} &=& 2 \sum_{k=1}^{\infty} | c_k| \\ 
%	&=&	2 \sum_{k=1}^{\infty} k^{-1} \cdot | k \, c_k| \\
%	&\leq& 2 \left( \sum_{k=1}^{\infty} k^{-2} \right)^{1/2}
%	\left( \sum_{k=1}^{\infty} |k \, c_k|^2 \right)^{1/2} \\
%	&=&  \left(2 \frac{\pi^2}{6} \right)^{1/2}
%		\left(2 \omega^{-2} \sum_{k=1}^{\infty} |i \omega k \, c_k|^2 \right)^{1/2}\\
%	&=& \frac{\pi}{\omega \sqrt{3}} 
%	\left(\sum_{k=- \infty}^{\infty} |i \omega k \, c_k|^2 \right)^{1/2}\\
%	&=& \frac{\pi}{\omega \sqrt{3}} 
%		\left( \frac{1}{L} \int_0^L | y'(t)|^2 dt  \right)^{1/2}\\
%	&\leq& \frac{\pi}{\omega \sqrt{3}}  \max | y'(t)| 
%\end{eqnarray}
%Line 4.3 is obtained through Cauchy-Schwartz inequality. 
%Line 4.6 is obtained through Parseval's identity.
%

% END OF SUBSECTION
%
%
% With this bound on $\omega$ we can now use the local results from Section~\ref{s:local} to establish the ultimate step in the proof of Wright's conjecture.
%
% \begin{theorem}
% 	\label{prop:WrightConjecture}
% 	For $ \alpha \in [1.5706,\pp]$ there is no SOPS to Wright's equation.
% \end{theorem}
%
% \begin{proof}
% 	Suppose that $y$ is a slowly oscillating periodic solution to Wright's equation with frequency $ \omega >0$.
% 	By Theorem~\ref{thm:FourierEquivalence2}, there is some $\epsilon \geq 0$ and $ c \in \ell^1_0$ for which $y$ (after a time shift) can be written in the following form:
% 	\[
% 	y(t) =
% 	\epsilon \left( e^{i \omega t }  + e^{- i \omega t }\right)
% 	+  \sum_{k = 2}^\infty    \tc_k e^{i \omega k t }  + \bar{\tc}_k e^{- i \omega k t }
% 	\]
% 	By Theorem \ref{thm:FourierEquivalence2}, it suffices to show that $ \tilde{F}_\epsilon(\alpha , \omega, \tc)=0$ has no nontrivial solutions when  $ \alpha \in [1.5706,\pp]$.
% 	In Proposition \ref{prop:TightEstimate} it was proved that in a small $\epsilon$-scaled neighborhood about the bifurcation point, periodic solutions to Wright's equation only exist when $ \alpha > \pp$.
% 	To prove the theorem, we extend this neighborhood so that it  contains any solutions to $\tilde{F}(\alpha,\omega,c)=0$ that could exist for $ \alpha \in  [1.5706, \pp]$.
%
%
%
% 	We first collect global bounds on $ \epsilon$, $c$ and $ \omega $ for which a solution $ \tilde{F}_\epsilon (\alpha, \omega, c)=0$ could exist.
% 	By the results in \cite{neumaier2014global} we know that $\max |y(t)| \leq e^{0.04} -1$. Let us fix $ m := e^{0.04} -1$ and define  $ \hat{c} := \{ \epsilon , c_2 , c_3, \dots \} \in \ell^1 $.  By Parseval's  identity, we obtain the following inequality:
% 	\[
% 		( \epsilon^2 + \epsilon^2)^{1/2}
% 		\leq
% 		\| \hat{c} \|_{\ell^2}
% 		=
% 	\left( \frac{1}{L} \int_0^L |y|^2 dt  \right)^{1/2} \leq m
% 	\]
% 	Thereby we may define $ \epsilon_0 := m/\sqrt{2} \approx 0.0288$ and concern ourselves only with solutions $ \tilde{F}_\epsilon =0$ for which $ 0 \leq \epsilon \leq \epsilon_0$.
% 	Similarly, we can use Parseval's identity to bound $ \| \hat{c} \|_{\ell^1}$ as below:
% 	\begin{eqnarray}
% 	\| \hat{c} \|_{\ell^1}
% %	 &=&
% %	2 \sum_{k=1}^{\infty} | c_k| \\
% %	&=&	2 \sum_{k=1}^{\infty} k^{-1} \cdot | k \, c_k| \\
% 	&\leq& 2 \left( \sum_{k=1}^{\infty} k^{-2} \right)^{1/2}
% 	\left( \sum_{k=1}^{\infty} |k \, c_k|^2 \right)^{1/2} \\
% %	&=&  \left(2 \frac{\pi^2}{6} \right)^{1/2}
% %	\left(2 \omega^{-2} \sum_{k=1}^{\infty} |i \omega k \, c_k|^2 \right)^{1/2}\\
% 	&=& \frac{\pi}{\omega \sqrt{3}}
% 	\left(\sum_{k=- \infty}^{\infty} |i \omega k \, c_k|^2 \right)^{1/2}\\
% 	&=& \frac{\pi}{\omega \sqrt{3}}
% 	\left( \frac{1}{L} \int_0^L | y'(t)|^2 dt  \right)^{1/2} 		\label{eq:WrightConjectureEll1BoundL2}\\
% 	&\leq& \frac{\pi}{\omega \sqrt{3}}  \max | y'(t)| \\
% 	&\leq& \frac{\pi}{\omega \sqrt{3}}  \alpha m ( 1 + m)
% 	\label{eq:WrightConjectureEll1Bound}
% 	\end{eqnarray}
%
%
% 	In Proposition \ref{prop:ZeroNBD} we show if  $ \tilde{F}_\epsilon (\alpha, \omega, c)=0$  for $ \alpha \in [1.5706,\pp]$ then $\omega \in [1.11,1.93]$ and furthermore:
% 	\[
% 	\sqrt{(\omega- \alpha)^2 + 2 \alpha \omega ( 1- \sin \omega)} / (2 \alpha) \leq  \| \hat{c}\|_{\ell^1}
% 	\]
% 	Combining this with the estimate in Line \ref{eq:WrightConjectureEll1Bound}, we obtain the following inequality:
% 	\[
% 	\omega 	\sqrt{(\omega- \alpha)^2 + 2 \alpha \omega ( 1- \sin \omega)}
% 	\leq
% 	 \tfrac{2 \pi}{ \sqrt{3}}  \alpha^2 m ( 1 + m)
% 	\]
% 	Using interval arithmetic, we confirm in the supplemental \emph{Mathematica} file that the only value of $ \omega$ satisfying the above inequality are  $\omega \in [1.4219, 1.6887]$.
% 	% (also note $|\omega - \pp| < 0.1489 $).
%
%
% 	Next, we derive bounds on $c$ which scale with $ \epsilon$.
% 	From Proposition \ref{prop:Cone}, it follows that if $(\alpha,\omega,c)$ solves $ \tilde{F}_\epsilon=0$ in the range  $0 \leq \epsilon \leq \epsilon_0$ and  $\alpha \in [ 1.5706,\pp]$ and $ |\omega - \pp | < 0.1489$, then either $ \| c \| \geq 0.72$ or $ \| c \| \leq   0.0398 \epsilon$.
% 	The case $\| c\| > 0.72$ contradicts our estimate in Line \ref{eq:WrightConjectureEll1Bound}, therefore, it must be the case that $  \| c \| \leq 0.0398 \epsilon $.
% 	It then follows that if $ \epsilon =0$ then  the only solution to $ \tilde{F}_\epsilon(\alpha,\omega,c)=0$ is the trivial solution.
% 	An additional result from Proposition \ref{prop:Cone} is that any solution must satisfy $\| K^{-1} c \|_{\ell^1/\C} < 0.715 \epsilon^2$ for $ \epsilon \leq \epsilon_0$.
%
%
% 	In summary, we have shown that if there is a solution to
% 	$ \tilde{F}(\alpha , \omega, c)=0$ when  $ \alpha \in [1.5706,\pp]$, then
% 	$ 0< \epsilon \leq \epsilon_0 \approx 0.0288$ and $ |\omega - \pp| < 0.1489 $ and $ \| c \| \leq   0.0398 \epsilon$ and $\| K^{-1} c \|_{\ell^1/\C} < 0.715 \epsilon^2$.
% 		To prove that $F  \equiv 0$ has no solutions for $\alpha \in [1.5706,\pp]$, we construct balls $B_\epsilon(r,\rho)$ which both will contain this region described above, and for each $ 0<\epsilon \leq \epsilon_0$, the ball will contain a unique solution $ \tilde{F}_\epsilon(\alpha_\epsilon,\omega_\epsilon,c_\epsilon)$ for which $ \alpha_\epsilon > \pp$.
% 	To satisfy both these objectives, we fix the following constants:
% 	\begin{align}
% 	r_\alpha =&	0.21
% 	&
% 	r_\omega =&	0.16
% 	&
% 	r_c =& 	0.09
% 	&
% 	\rho =& 	1.01
% 	\end{align}
% 	If $ \tilde{F}(\alpha,\omega,c) =0$ at parameter $ 0 < \epsilon \leq \epsilon_0$, then one can apply the triangle inequality to show that the variables $ (\alpha, \omega,c)$ must satisfy the following inequalities:
% 	\begin{align}
% 	r_\alpha \geq&| \alpha -  \balpha_\epsilon | ,
% 	&
% 	r_\omega  \geq& | \omega - \omega(\epsilon) | ,
% 	&
% 	\epsilon  \, r_c \geq& \|c -  \epsilon \, c(\epsilon) \|_{\ell^1/\C} ,
% 	\end{align}
% 	where $\balpha_\epsilon$ and $\omega(\epsilon)$ and $c(\epsilon)$ are given in Equation \ref{eq:Approx}.
% 	Furthermore, if $ \tilde{F}(\alpha,\omega,c) =0 $ then $\| K^{-1} c \|_{\ell^1/\C} < 0.715 (\epsilon \times \epsilon_0  ) < 0.021 \epsilon < \rho \epsilon$.
% 	Hence the ball $B_\epsilon(r,\rho)$ contains all the solutions to $ F \equiv 0$  when $ \alpha \in [1.5706,\pp]$.
%
% 	At these values, all of the components of the radii polynomials $P(\epsilon_0,r,\rho)$ are negative.
% 	Furthermore $ \rho \geq C(\epsilon_0,r)$ as in Proposition \ref{prop:DerivativeEndo} and $ \epsilon_0 < \tfrac{5}{4}(2 + \sqrt{5})^{-1}$.
% 	By Corollary \ref{prop:RPUniformEpsilon} it follows that for all $ 0 < \epsilon \leq \epsilon_0$ there exists a unique point  $(\bar{\alpha}_\epsilon , \bar{\omega}_\epsilon , \bar{c}_\epsilon) \in B_\epsilon(r,\rho)$ for which  $F(\alpha,\omega, c) =0$.
% 	It follows from  Corollary \ref{prop:TightEstimate} that   $\bar{\alpha}_\epsilon > \pp$ for all $ 0 < \epsilon \leq 0.10$.
% 	Since
% 	$F(\alpha,\omega,\epsilon c) = \epsilon  \tilde{F}(\alpha , \omega, c)$, then $\tilde{F}(\alpha , \omega, c)=0$ has no solutions when  $ \alpha \in [1.5706,\pp]$.
%
%
%
%
% \end{proof}


%%%%%%%%%%%%%%%%%%%%%%%%%%%%%%%%%%%%%%%%%%%%%%%%%%%%%%%%%%%%%%%%%%%%%%%%%%%%


%%%%%%%%%%%%%%%%%%%%%%%%%%%%%%%%%%%%%%%%%%%%%%%%%%%%%%%%%%%%%%%%%%%%%%%%%%%%


%%%%%%%%%%%%%%%%%%%%%%%%%%%%%%%%%%%%%%%%%%%%%%%%%%%%%%%%%%%%%%%%%%%%%%%%%%%%


%
%
%As was formulated in the 2010 paper by Lessard, this can be broken up into two statements: that there are no folds in the branch of SOPS originating from the Hopf bifurcation, and that there are no isolas of periodic orbits. 
%In Proposition \ref{prop:WideEstimate} we  identified a box in $ \R^2 \oplus \ell^1 / \C$ inside which the only periodic orbits are those originating from the Hopf bifurcation.  
%In Proposition \ref{prop:TightEstimate} we  identified explicit error between our approximate solution and the true solution. 
%



\subsection{Towards Jones' conjecture}
\label{s:Jones}

Jones' conjecture states that for $ \alpha > \pp$ there exists a (globally) unique SOPS to Wright's equation.
 Theorem \ref{thm:RadPoly} shows that for a fixed small $\epsilon$ there is a (locally) unique $\alpha$ at which Wright's equation has a SOPS, represented by
 $(\hat{\alpha}_\epsilon,\hat{\omega}_\epsilon,\hat{c}_\epsilon)$. 
This is not sufficient to prove the local case of Jones conjecture.
To accomplish the latter, we show in Theorem \ref{thm:UniqunessNbd} 
 that near the bifurcation point there is, for each fixed $\alpha>\pp$, a (locally) unique SOPS to Wright's equation. 
We begin by showing that on the solution branch emanating from the Hopf bifurcation 
$\hat{\alpha}_\epsilon$ is monotonically increasing in~$\epsilon$,
i.e.\ $ \tfrac{d}{d \epsilon} \hat{\alpha}_{\epsilon} >0$.  
Since $\balpha_\epsilon = \pp + \tfrac{1}{5}(\tfrac{3 \pi}{2}-1) \epsilon^2  $,
we expect that $\tfrac{d}{d \epsilon} \hat{\alpha}(\epsilon) = \tfrac{2}{5}(\tfrac{3 \pi}{2}-1) \epsilon + \cO(\epsilon^2)$. 
For this reason it is essential that we calculate an approximation of $\tfrac{d}{d \epsilon} \hat{\alpha}_\epsilon$ which is accurate up to order $ \cO(\epsilon^2)$.   

\begin{theorem}
	\label{thm:NoFold}
For $0 < \epsilon \leq 0.1$ we have $ \tfrac{d}{d \epsilon} \hat{\alpha}_{\epsilon} >0$. 
For 
%\change[J]{$\pp  < \alpha \leq \pp + 6.2757  \times 10^{-3}$ }
$\pp  < \alpha \leq \pp + 6.830  \times 10^{-3}$ 
there are no  bifurcations in the branch of SOPS that originates from the Hopf bifurcation. 
\end{theorem}

%
%In Lessard 2010, it was shown that the branch of SOPS bifurcating from $\pp$ does not have any folds for $ \alpha \in [\pp + \epsilon_1, 2.3]$ where $\epsilon_1 = 7.3165 \times 10^{-4}$. 
%In the Floquet Multipliers paper in preparation, we show that there is a unique SOPS to Wright's equation for $ \alpha \in [1.94,6]$. 
%In Xie's 1991 thesis, he shows that there is a unique SOPS to Wright's equation for $ \alpha > 5.67$.  
%Thereby, we have effectively proved the conjecture in \cite{lessard2010recent}  that the branch of SOPS bifurcating from $\pp$ has no folds. 
%\newline 

\begin{proof}
We show that the branch of solutions  $ \hat{x}_\epsilon =  (\hat{\alpha}_\epsilon , \hat{\omega}_\epsilon , \hat{c}_\epsilon)$ obtained in Proposition \ref{prop:TightEstimate} satisfies $  \tfrac{d}{d \epsilon} \hat{\alpha}_\epsilon >0$ for $0<\epsilon \leq 0.1$.
This implies that the solution branch is (smoothly) parametrized by~$\alpha$,
i.e.,  there are no secondary nor any saddle-node bifurcations in this branch.
We then show that these $\epsilon$-values cover the range 
%\change[J]{$\pp  < \alpha \leq \pp + 6.2757  \times 10^{-3}$.}
$\pp  < \alpha \leq \pp + 6.830  \times 10^{-3}$.
	
We begin by
	differentiating the equation $ F( \hat{x}_\epsilon) =0$ with respect to $ \epsilon$:
%
% Proposition \ref{prop:TightEstimate} states that for $ \epsilon \leq 0.10$ there exists a locally unique solution $ \bar{x}_\epsilon =  (\bar{\alpha}_\epsilon , \bar{\omega}_\epsilon , \bar{c}_\epsilon)$ to $F(x)=0$. 
% We show that $ \tfrac{d}{d \epsilon} \bar{\alpha}_\epsilon >0$ by implicit differentiation, which implies that there are no subsequent bifurcations.  
\begin{equation}
 \frac{\partial F}{\partial  \epsilon}(\hat{x}_\epsilon) + D F( \hat{x}_\epsilon)  \frac{d }{d  \epsilon} \hat{x}_\epsilon  = 0 .
\end{equation}
In terms of the map $T$ we obtain the relation
\[
\left[I-DT(\hat{x}_\epsilon)  \right]  \frac{d }{d \epsilon} \hat{x}_\epsilon   
=- A^{\dagger} \frac{\partial F}{\partial  \epsilon}(\hat{x}_\epsilon)  .
\]

To isolate $\frac{d }{d \epsilon} \hat{x}_\epsilon   $, we wish to left-multiply each side of the above equation by $[I-DT(\hat{x}_\epsilon)]^{-1}$. 
To that end, we define an upper bound on $DT(\hat{x}_\epsilon)$ by  the matrix 
\begin{equation}\label{e:defZeps}
	\ZZ_\epsilon := Z(\epsilon,\epsilon^2 \rr, \rho) ,
\end{equation}
with $\rr$ and $\rho$ as in Proposition~\ref{prop:TightEstimate}.
We know from Remark~\ref{r:boundDT} 
%the proof of 
that with respect to the norm $\| \cdot \|_{\rr}$ on $\R^2 \times \ell^K_0$
\[
\| DT(\hat{x}_\epsilon)  \|_{\rr} \leq \max_{i=1,2,3} \frac{( \ZZ_\epsilon \cdot \rr)_i}{\rr_i} < 1, \qquad\text{for all } 0 \leq \epsilon \leq \epsilon_0, 
\]
with $\epsilon_0$ given in Proposition~\ref{prop:TightEstimate}. 
Hence $I-DT(\hat{x}_\epsilon) $ is invertible. In particular,
\begin{alignat*}{1}
\frac{d }{d \epsilon} \hat{x}_\epsilon   
& =- \left[I-DT(\hat{x}_\epsilon)  \right]^{-1}  A^{\dagger} \frac{\partial F}{\partial  \epsilon}(\hat{x}_\epsilon)  \\
& = - \left[I + \sum_{n=1}^\infty DT(\hat{x}_\epsilon)^n  \right]  A^{\dagger} \frac{\partial F}{\partial  \epsilon}(\hat{x}_\epsilon) .
\end{alignat*}
We  have an upper bound $\QQ_\epsilon \in \R^3_+$ on $A^{\dagger} \frac{\partial F}{\partial  \epsilon}(\hat{x}_\epsilon)$, as defined in Definition~\ref{def:upperbound}, given by Lemma~\ref{lem:Qeps}. 
We define $\II$ to be the $3 \times 3$ identity matrix.
For the $\alpha$-component we then obtain the estimate
% \begin{alignat}{1}
% \frac{d }{d \epsilon} \hat{\alpha}_\epsilon
% &\geq - \pi_\alpha  A^{\dagger} \frac{\partial F}{\partial  \epsilon}(\hat{x}_\epsilon)
% - \left( \sum_{n=1}^\infty \ZZ_\epsilon^n \QQ_\epsilon \right)_1 \nonumber \\
% & = - \pi_\alpha  A^{\dagger} \frac{\partial F}{\partial  \epsilon}(\hat{x}_\epsilon)  - \left( \ZZ_\epsilon (1-\ZZ_\epsilon)^{-1} \QQ_\epsilon \right)_1 . \label{e:alphaepsilon}
% \end{alignat}
% \note[J]{Proposed Change}
\begin{alignat}{1}
\frac{d }{d \epsilon} \hat{\alpha}_\epsilon  
&\geq - \pi_\alpha  A^{\dagger} \frac{\partial F}{\partial  \epsilon}(\hat{x}_\epsilon)  
- \left( \sum_{n=1}^\infty \ZZ_\epsilon^n \QQ_\epsilon \right)_1 \nonumber \\
& = - \pi_\alpha  A^{\dagger} \frac{\partial F}{\partial  \epsilon}(\hat{x}_\epsilon)  - \left( \ZZ_\epsilon (\II-\ZZ_\epsilon)^{-1} \QQ_\epsilon \right)_1 . \label{e:alphaepsilon}
\end{alignat}
%\annote[JB]{We note that the elements of the matrix $\ZZ_\epsilon$ are $\cO(\epsilon^2)$.}{Shall we just remove this sentence? It is cryptic (no argument is given) and anyway this is repeated and made precise in Eqn (4.10).}
We approximate $\frac{\partial F}{\partial  \epsilon}(\hat{x}_\epsilon)$ by 
\[
	\Gamma_\epsilon := \pp \tfrac{3i -1}{5} \epsilon \, \e_1 - i \pp \,  \e_2 - \pp \tfrac{3+i}{5} \epsilon \, \e_3 ,
\]
which is accurate up to quadratic terms in $\epsilon$.
In Lemma \ref{lem:ImplicitApprox} it is shown that
\begin{equation}\label{e:linearepsilon}
- \pi_\alpha A^{\dagger} \Gamma _\epsilon = \tfrac{2}{5} ( \tfrac{3 \pi}{2} -1) \epsilon.
\end{equation}
It remains to incorporate two explicit bounds for the remaining terms in~\eqref{e:alphaepsilon}. 
In Lemma~\ref{lem:Meps} we define $M_\epsilon$ and $M'_\epsilon$ that satisfy the following inequalities:
% \begin{alignat}{1}
% \left| \pi_\alpha A^{\dagger} \left( \tfrac{\partial F}{\partial  \epsilon}(\hat{x}_\epsilon) - \Gamma_\epsilon \right)  \right| &\leq
% \epsilon^2 M_\epsilon  , \label{e:boundM} \\
% %
% \left( \ZZ_\epsilon (1-\ZZ_\epsilon)^{-1} \QQ_\epsilon \right)_1 &\leq
%  \epsilon^2 M'_\epsilon . \label{e:boundMp}
% \end{alignat}
% \note[J]{Proposed Change}
\begin{alignat}{1}
\left| \pi_\alpha A^{\dagger} \left( \tfrac{\partial F}{\partial  \epsilon}(\hat{x}_\epsilon) - \Gamma_\epsilon \right)  \right| &\leq 
\epsilon^2 M_\epsilon  , \label{e:boundM} \\
%
\left( \ZZ_\epsilon (\II-\ZZ_\epsilon)^{-1} \QQ_\epsilon \right)_1 &\leq 
\epsilon^2 M'_\epsilon . \label{e:boundMp} 
\end{alignat}
Moreover, we infer from Lemma~\ref{lem:Meps} that $M_\epsilon$ and $M'_\epsilon$ are positive, increasing in $\epsilon$, and can be 
obtained explicitly by performing an interval arithmetic computation, using the explicit expressions for the matrix $\ZZ_\epsilon$ and the vector $\QQ_\epsilon$ given by Equation~\eqref{e:defZeps} and Lemma~\ref{lem:Qeps}, respectively (the expression for $Z(\epsilon,r,\rho)$ is provided in Appendix~\ref{sec:BoundingFunctions}).
 
Finally, we combine~\eqref{e:alphaepsilon},~\eqref{e:linearepsilon},~\eqref{e:boundM} and~\eqref{e:boundMp} to obtain
\[
 \frac{d }{d \epsilon} \hat{\alpha}_\epsilon  \geq 
 \tfrac{2}{5} ( \tfrac{3 \pi}{2} -1) \epsilon - \epsilon^2 ( M_{\epsilon} + M'_{\epsilon}).
\]
From the monotonicity of the bounds $M_\epsilon$ and $M'_\epsilon$ in terms of $\epsilon$, we infer that in order to conclude that  $\frac{d }{d \epsilon} \hat{\alpha}_\epsilon >0 $ for $0<\epsilon\leq\epsilon_0$  it suffices to check, using interval arithmetic, that
\begin{equation}
 \tfrac{2}{5} ( \tfrac{3 \pi}{2} -1) \epsilon_0 - \epsilon_0^2 (M_{\epsilon_0} + M'_{\epsilon_0})  > 0 . \label{e:Mepsilon0}
\end{equation} 
In the Mathematica file~\cite{mathematicafile} we check that~\eqref{e:Mepsilon0} is satisfied 
for $\epsilon_0 = 0.1$.
Since $\balpha_{\epsilon_0} \geq \pp + 7.4247\times 10^{-3}$,
and taking into account the control provided by Proposition~\ref{prop:TightEstimate} on the distance between $\hat{\alpha}_\epsilon$ and $\balpha_\epsilon$ in terms of $\rr_\alpha$, we
find that  
%\change[J]{$\hat{\alpha}_{\epsilon_0} \geq \balpha_{\epsilon_0} - \epsilon_0^2 \rr_\alpha \geq \pp + 6.2757  \times 10^{-3}$.}
 $\hat{\alpha}_{\epsilon_0} \geq \balpha_{\epsilon_0} - \epsilon_0^2 \rr_\alpha \geq \pp + 6.830  \times 10^{-3}$.
Hence  there can be no bifurcation on the solution branch for 
%\change[J]{$ \pp < \alpha \leq \pp + 6.2757   \times 10^{-3}$.}
 $ \pp < \alpha \leq \pp + 6.830   \times 10^{-3}$.
\end{proof}	



% Since $\ZZ_\epsilon$ is $\cO(\epsilon^2)$,
% NEXT DETERMINE $Q$ AND (SOME APPROXIMATION OF) $\pi_\alpha A^{\dagger} \frac{\partial F}{\partial  \epsilon}(\hat{x}_\epsilon)$.
% NOW I HAVE TO UNDERSTAND APPENDIX E, WHICH DOESN'T SEEM THAT EASY.
%
% We can calculate $\frac{d }{d  \epsilon} \bar{x}_\epsilon $ using the approximate inverse $A^{\dagger}$.
% \begin{eqnarray}
%  D F( x_\epsilon) \cdot \frac{\partial }{\partial  \epsilon} \bar{x}_\epsilon
% &=&
% - \frac{\partial F}{\partial  \epsilon}(\bar{x}_\epsilon) \\
%  %
%  \left[I -(I- A^{\dagger} D F( \bar{x}_\epsilon) )  \right] \cdot \frac{\partial }{\partial  \epsilon} \bar{x}_\epsilon
%  &=&
% - A^{\dagger} \frac{\partial F}{\partial  \epsilon}(\bar{x}_\epsilon) \\
% %
%  \left[I-DT(\bar{x}_\epsilon)  \right] \cdot \frac{\partial }{\partial  \epsilon} \bar{x}_\epsilon
% &=&
% - A^{\dagger} \frac{\partial F}{\partial  \epsilon}(\bar{x}_\epsilon)
% \end{eqnarray}
% As a result of the radii polynomial method being successful in Proposition \ref{prop:TightEstimate}, it follows that $\| DT(\bar{x}_\epsilon) \|_{r} <1$ for $ \epsilon < 0.10$, whereby $I - DT(\bar{x}_\epsilon)$ is invertible.
% To shorten our notation, we introduce the variable $ B:=DT(\bar{x}_\epsilon) $.
% We calculate further:
% \begin{eqnarray}
% \frac{\partial }{\partial  \epsilon} \bar{x}_\epsilon &=& - (I - B)^{-1} A^{\dagger} \frac{\partial F}{\partial  \epsilon}(\bar{x}_\epsilon)  \\
% &=& - \left( I + \sum_{k=1}^\infty B^k \right)   A^{\dagger} \frac{\partial F}{\partial  \epsilon}(\bar{x}_\epsilon)
% \end{eqnarray}
% Choosing $r$ as in Proposition \ref{prop:TightEstimate} gives us the bound $ \bar{x}_\epsilon \in B_{\epsilon}(r \epsilon^2 , \rho)$.
% Hence $\ZZ_\epsilon := Z(\epsilon ,\epsilon^2 r , \rho)$ is an upper bound for $ DT(\bar{x}_\epsilon)$.
% Suppose that we have  an upper bound	$\overline{A^{\dagger} \frac{\partial F}{\partial  \epsilon}(\bar{x}_\epsilon) } $ on the operator $A^{\dagger} \frac{\partial F}{\partial  \epsilon}(\bar{x}_\epsilon) $.
% Then we may proceed to obtain a lower bound on $ [\frac{\partial }{\partial  \epsilon} \bar{x}_\epsilon ]_{\alpha}$ as follows:
% \begin{eqnarray}
% \left[ \frac{\partial }{\partial  \epsilon} \bar{x}_\epsilon\right]_{\alpha}
% &=&
%  -  \left[ A^{\dagger} \frac{\partial F}{\partial  \epsilon}(\bar{x}_\epsilon)  \right]_{\alpha} -
% \left[ \left( \sum_{k=1}^\infty B^k \right)   A^{\dagger} \frac{\partial F}{\partial  \epsilon}(\bar{x}_\epsilon)   \right]_{\alpha}\\
% %
% \left[ \frac{\partial }{\partial  \epsilon} \bar{x}_\epsilon\right]_{\alpha}
% &\geq &
%    \left[ -A^{\dagger} \frac{\partial F}{\partial  \epsilon}(\bar{x}_\epsilon)  \right]_{\alpha} - \left[
% \left( \sum_{k=1}^\infty \ZZ_\epsilon ^k \right)
% \overline{  A^{\dagger} \frac{\partial F}{\partial  \epsilon}(\bar{x}_\epsilon) } \right]_{\alpha}    \\
% %
% &=&
%  \left[- A^{\dagger} \frac{\partial F}{\partial  \epsilon}(\bar{x}_\epsilon)  \right]_{\alpha} -
%  \left[
% 	  \ZZ_\epsilon( I -\ZZ_\epsilon)^{-1} \overline{A^{\dagger} \frac{\partial F}{\partial  \epsilon}(\bar{x}_\epsilon) }   \,
% \right]_{\alpha} \label{eq:NoFoldIneq}
% \end{eqnarray}
%
%
%
%
%
%
% 	In order to show that $ \tfrac{\partial}{\partial \epsilon} \bar{\alpha}_\epsilon >0$, it suffices to show that the RHS of Line \ref{eq:NoFoldIneq} is positive.
% 	Since $ \ZZ_\epsilon$ is of order $ \cO(\epsilon^2)$, then to obtain a $ \cO(\epsilon^2)$  approximation of $ \tfrac{\partial}{\partial \epsilon} \bar{\alpha}_\epsilon$, we define a $ \cO(\epsilon^2)$ approximation of $\tfrac{\partial}{\partial \epsilon}F(\bar{x}_\epsilon)$ as below:
% 	\begin{equation}
% 	\label{eq:GammaDef}
% 	\Gamma := \pp[\tfrac{3i -1}{5} \epsilon] e_1 - \pp [i] e_2 - \pp[\tfrac{3+i}{5} \epsilon] e_3
% 	\end{equation}
% 	In Proposition \ref{prop:ImplicitApprox} we show that $[-A^{\dagger} \Gamma ]_\alpha =\tfrac{2}{5} ( \tfrac{3 \pi}{2} -1) \epsilon$, which is what we expected the first order approximation of $ \alpha'(\epsilon)$ to be.
% 	Hence, showing that Line \ref{eq:NoFoldIneq} is positive is equivalent to proving the inequality below:
%
% 	\begin{equation}
% 	\tfrac{2}{5} ( \tfrac{3 \pi}{2} -1) \epsilon
% 	%
% 	>
% 	%
% 	 \left[- A^{\dagger} \left( \tfrac{\partial F}{\partial  \epsilon}(\bar{x}_\epsilon) - \Gamma \right) \right]_{\alpha} +
% 	 \left[
% 	 \ZZ_\epsilon( I -\ZZ_\epsilon)^{-1} \overline{A^{\dagger} \tfrac{\partial F}{\partial  \epsilon}(\bar{x}_\epsilon) }   \,
% 	 \right]_{\alpha}
% 	 \label{eq:ImplicitDiffEq2}
% 	\end{equation}
% %To facilitate computing  the RHS of Line \ref{eq:ImplicitDiffEq2},
% In Proposition \ref{prop:ImplicitLast}  we explicitly define an upper bound on the vector $
% 	A^{\dagger} \left( \tfrac{\partial F}{\partial  \epsilon}(\bar{x}_\epsilon) - \Gamma \right) $ which by definition is of order $\cO(\epsilon^2)$ and  we denote here as $C$.
% 	Using this bound, we can expand Line \ref{eq:ImplicitDiffEq2} as follows:
% 	\begin{equation}
% 	\tfrac{2}{5} ( \tfrac{3 \pi}{2} -1) \epsilon
% 	%
% 	\geq
% 	%
% 	\left[
% 	C
% 		 \right]_{\alpha}
% 	%
% 	+
% 	%
% 	\left[
% 	\ZZ_\epsilon( I -\ZZ_\epsilon)^{-1}  \left(A^{\dagger} \Gamma +
% 	C  \right) \,
% 	\right]_{\alpha}
% 	\label{eq:ImplicitInequality}
% 	\end{equation}
% 	Since $\ZZ_\epsilon$ and $C$ are both of order $\cO(\epsilon^2)$, then it follows that the RHS of Line \ref{eq:ImplicitInequality} is of order $ \cO(\epsilon^2)$.
% 	Hence, there exists some $\epsilon>0$ for which Inequality \ref{eq:ImplicitInequality} is satisfied, and furthermore, if some $\epsilon_0 >0$ satisfies Inequality \ref{eq:ImplicitInequality}, then the inequality is satisfied for all $0 < \epsilon \leq \epsilon_0$.
%
% 	Using interval arithmetic, we check that Inequality \ref{eq:ImplicitInequality} is satisfied for $\epsilon_0 = 0.10$. Whereby $ \alpha'(\epsilon) > 0$ for $ 0 < \epsilon \leq \epsilon_0$.
% 	Since $ \alpha(\epsilon_0)= \pp + 7.4247\times 10^{-3}$, then by accounting for our error in approximating $\alpha$ with our estimate from Proposition \ref{prop:TightEstimate}, there can be no subsequent bifurcations for $ \pp < \alpha < \pp + 6.3779 \times 10^{-3}$.
%
%


%%	Since we have computed an $ \cO(\epsilon^2)$ approximation, then for a fixed $ \rho$,  the components of  $ \ZZ_\epsilon$ and 
%%	$\overline{ 
%%		A^{\dagger} \left( \tfrac{\partial F}{\partial  \epsilon}(x_\epsilon) - \Gamma \right) }  $  are polynomials with non-negative coefficients, of order $\epsilon^2$ and higher. 
%%	It then follows that $ \epsilon^{-2}$ times the RHS of Equation \ref{eq:ImplicitInequality} is well defined at $ \epsilon = 0$ and non-decreasing in $\epsilon$. 




The above theorem provides the missing piece in the proof of Theorem~\ref{thm:IntroNoFold}.

\begin{corollary}\label{cor:collectreformulatedJones}
The branch of SOPS originating from the Hopf bifurcation at $\alpha = \pp$ has no folds or secondary bifurcations for any $\alpha > \pp$. 
\end{corollary}
\begin{proof}
	
	We prove the corollary by combining results on four overlapping subintervals of $ ( \pp, \infty)$. 
	In Theorem~\ref{thm:NoFold} we show that the (continuous) branch of SOPS originating from the Hopf bifurcation does not have any folds or secondary bifurcations for 
		$ \alpha \in ( \pp  , \pp + \delta_3] $ where 
%\change[J]{ $ \delta_3 = 6.2757  \times 10^{-3}$. 
 $ \delta_3 = 6.830  \times 10^{-3}$. 
	In~\cite{lessard2010recent} the same result is proved for $ \alpha \in [ \pp + \delta_1, 2.3]$, where $\delta_1 = 7.3165 \times 10^{-4}$. 
	In~\cite{jlm2016Floquet} 
	 it is shown that there is a unique SOPS  for $ \alpha $ in the interval $[1.94,6.00]$. 
	 Since $1.94 \leq 2.3$, then the SOPS in this interval belong to the branch originating from the Hopf bifurcation, and since they are unique for each $\alpha$, the branch is continuous and cannot have any folds or secondary bifurcations. 
	  In~\cite{xie1991thesis} it is shown that there is a unique SOPS for $ \alpha $ in the interval $ [5.67, +\infty)$, and by a similar argument the branch of SOPS cannot have any folds or secondary bifurcations in this interval either. 
	 Since 
	\[
	(\pp, \infty) = (\pp, \pp + \delta_3] \cup [ \pp + \delta_1,2.3] \cup [1.94,6.00] \cup [5.67, \infty) ,
	\]
	it follows that branch of SOPS originating from the Hopf bifurcation at $\alpha = \pp$ has no folds or secondary bifurcations for any $\alpha > \pp$. 
\end{proof}

To prove Jones' conjecture, it is insufficient to prove only locally that Wright's equation has a unique SOPS.  
We must be able to connect our local results with global estimates. 
When we make the change of variables $\tc = \epsilon c$ in defining the function $F_\epsilon$, we restrict ourselves to proving local results.  
Theorems~\ref{thm:UniqunessNbd} and~\ref{thm:UniqunessNbd2} connect these local results with a global argument, and construct neighborhoods, independent of any $ \epsilon$-scaling, within which the only SOPS to Wright's equation are those originating from the Hopf bifurcation.  

The next theorem uses the large radius calculation from Proposition \ref{prop:bigboxes}(b) to show that for  
%\change[J]{$\alpha \in ( \pp , \pp+ 4.75 \times 10^{-3} ]$}
$\alpha \in ( \pp , \pp+ 5.53 \times 10^{-3} ]$
all periodic solutions in a neighborhood of $0$ lie on the Hopf bifurcation curve, which has neither folds nor secondary bifurcations.  

% \note[J]{Almost every number in the Theorem and proof below has changed, so I did not always use track changes. Please check this. }

\begin{theorem}
	\label{thm:UniqunessNbd}
	For each $\alpha  \in  (\pp , \pp + 5.53 \times 10^{-3} ] $ there is a unique triple $ ( \epsilon, \omega, c)$ in the range 
%\change[J]{$ 0 < \epsilon \leq 0.087$}
$ 0 < \epsilon \leq 0.09$
	and 
%\change[J]{$ | \omega - \pp| < 0.061 $}
$ | \omega - \pp| < 0.0924 $
	and  
%\change[J]{$ \| c \| \leq 0.13135 $}
$ \| c \| \leq 0.30232 $ 
such that $ F_\epsilon(\alpha, \omega, c)=0$. 	
\end{theorem}
 
\begin{proof}
Fix $ \alpha \in ( \pp , \pp + 5.53 \times 10^{-3}]$ and let $F_\epsilon(\alpha, \omega, c)=0$ for some $\epsilon, \omega, c$ satisfying the assumed bounds. 
%Let $0 \leq \epsilon \leq \epsilon_0:=0.087$ and let $F_\epsilon(\alpha, \omega, c)=0$ for some $(\alpha, \omega, c)$ satisfying the assumed bounds.
From Lemma~\ref{lem:Cone}(b) it follows that 
%\change[J]{$\|K^{-1}c\| \leq \epsilon^2 (1+ \|c\|^2) /(\epsilon b_*) \leq 0.32$}
$\|K^{-1}c\| \leq \epsilon^2 (2+ \|c\|^2) /(\epsilon b_*) \leq 0.61$
for $\epsilon \leq \epsilon_0$, 
since $b_* \geq 0.31$. 
Hence the zeros under consideration all lie in the set 
% \[
% \tS :=  \{ (\alpha,\omega,c) \in X : | \alpha - \pp | \leq 0.00553 , |\omega - \pp| \leq 0.0924, \| c \| \leq 0.30232, \|K^{-1} c \| \leq 0.32  \}.
% \]
% \note[J]{Proposed change}
\[
\tS :=  \{ (\alpha,\omega,c) \in X : | \alpha - \pp | \leq 0.00553 , |\omega - \pp| \leq 0.0924, \| c \| \leq 0.30232, \|K^{-1} c \| \leq 0.61  \}.
\]
Proposition~\ref{prop:bigboxes}(b) shows that for each $0\leq\epsilon\leq 0.09$
there is a unique zero $\hat{x}_\epsilon=
(\hat{\alpha}_\epsilon,\hat{\omega}_\epsilon,\hat{c}_\epsilon) \in B_\epsilon(r,\rho)$ of $F_\epsilon$,
with $r=(r_\alpha,r_\omega,r_c) = (0.1753,0.0941,0.3829)$ and $\rho= 1.5940$.
For each $0 \leq \epsilon \leq 0.09$ it follows from the triangle inequality  that $\tS \subset B_\epsilon(r,\rho)$.  
This shows that $F_\epsilon$ has at most one zero in $\tS$ for each $ 0 \leq \epsilon \leq \epsilon_0$. 
By Remark~\ref{r:nested} this solution lies on the branch $\hat{x}_\epsilon$ originating from the Hopf bifurcation, in particular $\hat{x}_0=(\pp,\pp,0) \in \tS$.
Proposition~\ref{prop:TightEstimate} gives us tight bounds 
\[
|\hat{\omega}_\epsilon - \pp| \leq |\bomega_\epsilon - \pp|  + \rr_\omega \epsilon^2 \leq 0.0924
\qquad\text{and}\qquad \| \hat{c}_\epsilon \| \leq \| \bc_\epsilon\|  + \rr_c \epsilon^2 \leq 0.30232
\]
for all $0 \leq \epsilon \leq \epsilon_0$.  
Moreover, from similar considerations it follows that $\hat{\alpha}_{\epsilon_0} \geq  \balpha_{\epsilon_0} - r_\alpha \epsilon_0^2 > 0.00553$. Hence $\hat{x}_{\epsilon_0} \notin \tS$ and the solution curve leaves $\tS$ through $|\alpha- \pp| = 0.00553$ for some $0<\epsilon <\epsilon_0$.
Since $0.00553  < 6.830 \times 10^{-3}$ the assertion now follows directly from Theorem~\ref{thm:NoFold}.
\end{proof}
%

% OLD PROOF BELOW
%Fix $ \alpha \in ( \pp , \pp + 4.750 \times 10^{-3}]$ and let $F_\epsilon(\alpha, \omega, c)=0$ for some $\epsilon, \omega, c$ satisfying the assumed bounds. 
%%Let $0 \leq \epsilon \leq \epsilon_0:=0.087$ and let $F_\epsilon(\alpha, \omega, c)=0$ for some $(\alpha, \omega, c)$ satisfying the assumed bounds.
%From Lemma~\ref{lem:Cone}(b) it follows that $\|K^{-1}c\| \leq \epsilon^2 (1+ \|c\|^2) /(\epsilon b_*) \leq 0.27$ for $\epsilon \leq \epsilon_0$, 
%since $b_* \geq 0.33$. 
%Hence the zeros under consideration all lie in the set 
%\[
%  \tS :=  \{ (\alpha,\omega,c) \in X : | \alpha - \pp | \leq 0.00475 , |\omega - \pp| \leq 0.061, \| c \| \leq 0.13135, \|K^{-1} c \| \leq 0.27  \}.
%\]
%Proposition~\ref{prop:bigboxes}(b) shows that for each $0\leq\epsilon\leq 0.087$
%there is a unique zero $\hat{x}_\epsilon=
%(\hat{\alpha}_\epsilon,\hat{\omega}_\epsilon,\hat{c}_\epsilon) \in B_\epsilon(r,\rho)$ of $F_\epsilon$,
%with $r=(r_\alpha,r_\omega,r_c) = (0.1501,0.0626,0.2092)$ and $\rho= 0.5672$.
%For each $0 \leq \epsilon \leq 0.087$ it follows from the triangle inequality  that $\tS \subset B_\epsilon(r,\rho)$.  
%This shows that $F_\epsilon$ has at most one zero in $\tS$ for each $ 0 \leq \epsilon \leq \epsilon_0$. 
%By Remark~\ref{r:nested} this solution lies on the branch $\hat{x}_\epsilon$ originating from the Hopf bifurcation, in particular $\hat{x}_0=(\pp,\pp,0) \in \tS$.
%Proposition~\ref{prop:TightEstimate} gives us tight bounds 
%\[
%  |\hat{\omega}_\epsilon - \pp| \leq |\bomega_\epsilon - \pp|  + \rr_\omega \epsilon^2 \leq 0.061
%  \qquad\text{and}\qquad \| \hat{c}_\epsilon \| \leq \| \bc_\epsilon\|  + \rr_c \epsilon^2 \leq 0.657
%\]
%for all $0 \leq \epsilon \leq \epsilon_0$.  
%Moreover, from similar considerations it follows that $\hat{\alpha}_{\epsilon_0} \geq  \balpha_{\epsilon_0} - r_\alpha \epsilon_0^2 > 0.00475$. Hence $\hat{x}_{\epsilon_0} \notin \tS$ and the solution curve leaves $\tS$ through $|\alpha- \pp| = 0.00475$ for some $0<\epsilon <\epsilon_0$.
%Since $0.00475  < 6.2757 \times 10^{-3}$ the assertion now follows directly from Theorem~\ref{thm:NoFold}.
%\end{proof}

% OLDER PROOF BELOW
%
% \begin{proof}
%
%
% By Proposition \ref{prop:Cone}, if $\tilde{F}(\alpha,\omega,c)=0$ and $ |\alpha - \pp | \leq 0.0044$ and $| \omega - \pp| \leq 0.0811$, then either $ \| c\|_{\ell^1 / \C} < 0.121691 \times \epsilon$ or $ \| c\|_{\ell^1 / \C} > 0.657$.
% Hence, by our initial assumption we may take $ \| c\|_{\ell^1 / \C} < 0.121691 \times \epsilon$.
% Furthermore, Proposition \ref{prop:Cone} tells us that $ \| K^{-1} c \|_{\ell^1 / \C} <  (\epsilon^2+ \|c\|^2)/(2 b_*)$.
% Plugging in the appropriate values produces the estimate $ \| K^{-1} c \|_{\ell^1 / \C} <  \epsilon^2 \times 0.840672$.
% To apply the method of radii polynomials, we must  make the change of variables $ c \mapsto \epsilon c$.
% That is, if $ F( \alpha , \omega, c) =0$ then we may conclude that  $ \| c\|_{\ell^1 / \C} < 0.121691$ and  $ \| K^{-1} c \|_{\ell^1 / \C} <  \epsilon \times 0.840672$.
%
% To perform the method of radii polynomials on a specific ball, fix the following constants:
% \begin{align}
%  \epsilon_0 =& 0.080
% &
% r_\alpha =&	0.190814
% &
% r_\omega =&	0.0823824
% &
% r_c =& 	0.193644
% &
% \rho =& 	0.73103633
% \end{align}
% These constants are chosen so that the ball $ B_{\epsilon}( r , \rho)$ will contain any possible solutions to $ F( \alpha, \omega, c) =0$.
% That  is,  fix $ \alpha \in ( \pp, \pp + 0.0044]$, $ | \omega - \pp| < 0.0811$ and  $ \|c \|_{\ell^1 / \C} <   0.121691$.
% The constants $ r_\alpha$ and $ r_\omega$ and $ r_c$ were chosen that for all values of $ 0 < \epsilon \leq \epsilon_0$, then  an application of the triangle inequality relative to $\{ \balpha_\epsilon , \omega(\epsilon), c(\epsilon) \}$ implies the following inequalities:
% \begin{align}
% r_\alpha \geq& | \alpha -  \balpha_\epsilon |
% &
% r_\omega  \geq& | \omega - \omega(\epsilon) |
% &
% r_c \geq& \|c - c(\epsilon)\|
% \end{align}
% The value of $\rho$ was chosen to be the constant $ C(\epsilon,r)$ from Proposition \ref{prop:DerivativeEndo}.
% Since we earlier determined that $ \| K^{-1} c \|_{\ell^1 / \C} < \epsilon_0 \times 0.840672 < 0.068$ then it follows that  $ \| K^{-1} c \|_{\ell^1 / \C} \leq \rho$.
% Hence, if the variables $\epsilon$, $ \alpha$, $\omega,$ and $ c$ are chosen as per the assumptions of this theorem, the solutions $ F(\alpha,\omega,c)=0$ satisfy  $(\alpha,\omega,c) \in B_{\epsilon}(r,\rho)$.
%
%
% For these values of $ \epsilon_0, r, \rho$ all components of the radii polynomials $ P(\epsilon_0,r,\rho)$ are negative, as verified in the supplemental \emph{Mathematica} file.
% By Theorem \ref{prop:RPUniformEpsilon}, it follows that for all $ 0 < \epsilon \leq 0.08$, there exists a unique point  $ \bar{x}_\epsilon \in B_\epsilon(r,\rho)$ for which $ F( \bar{x}_\epsilon)=0$.
% Using our approximation defined in Equation \ref{eq:Approx} we calculate that $\alpha(0.08) = \pp + 0.00475$, so by applying our estimates from Proposition \ref{prop:TightEstimate} we may deduce that if $ \bar{\alpha}_\epsilon \in ( \pp , \pp+ 0.0044]$ then $\epsilon \in (0,0.08]$.
% By Theorem \ref{thm:NoFold}, we know that each  $\alpha \in ( \pp  , \pp +  6.3779  \times 10^{-3}] $  corresponds to a unique solution $ F(\bar{x}_\epsilon) =0$.
% In conclusion, if $ \alpha \in (\pp,\pp + 0.0044]$, then there exists a unique $ \epsilon \in ( 0,0.08]$ and $ | \omega - \pp | < 0.0811$ and $ \| c \|_{\ell^1 / \C} < 0.6766$ such that $ F( \alpha ,\omega, c)=0$ at parameter $\epsilon$.
%
%
%
%
%
%
% %then there are no folds in the branch of periodic orbits.
% %By Proposition \ref{prop:TightEstimate} we know that $| \bar{\alpha}_\epsilon - \alpha(\epsilon )| < 0.0411766 \epsilon^2$, so then for each $ \alpha \in (\pp , \pp + 0.0054]$ there exists a unique $0<\epsilon \leq 0.085$ for which $ \bar{\alpha}_\epsilon = \alpha$ and $ F(\bar{x}_\epsilon)=0$.
% %
% %
% %
%
%
% \end{proof}



Finally, we translate this result to function space.

\begin{theorem}
For each 
%	\change[J]{$\alpha  \in  (\pp , \pp + 0.00475] $ }
$\alpha  \in  (\pp , \pp + 5.53 \times 10^{-3} ] $ 
there is at most one (up to time translation) periodic solution to Wright's equation satisfying 
%\change[J]{$ \| y' \|_{L^2([0,2\pi/\omega])} \leq  0.295$}
$ \| y' \|_{L^2([0,2\pi/\omega])} \leq  0.302$
	and having frequency 
%\change[J]{$ | \omega - \pp | \leq 0.061$}
$ | \omega - \pp | \leq 0.0924$. 
	\label{thm:UniqunessNbd2}
\end{theorem}

\begin{proof}
We show that any periodic solution~$y$ to Wright's equation of period $2\pi/\omega$ that satisfies 
%\change[J]{$ \| y' \|_{L^2} \leq 0.295$ }
$ \| y' \|_{L^2} \leq 0.302$ 
has Fourier coefficients satisfying the bounds in Theorem~\ref{thm:UniqunessNbd}.
For the Fourier coefficients $a$ of $y$ we infer from Lemma~\ref{lem:fourierbound} that 
%\change[J]{$\| a \| \leq \sqrt{\frac{\pi}{6\omega}} \cdot 0.295 \leq 0.174 $.}
$\| a \| \leq \sqrt{\frac{\pi}{6\omega}} \cdot 0.302 \leq 0.18 $. 
Furthermore, for the parameter range of $\alpha$ and $\omega$ under consideration we conclude that $\alpha < 2\omega $ and 
$\|a\| < \frac{2\omega-\alpha}{\alpha}$. Hence we see from Theorem~\ref{thm:FourierEquivalence3}
that $y$ corresponds to a zero of $F_\epsilon$. 
The a priori bound on $\|a\|$ translates via~\eqref{e:aepsc} into the bounds
% \[
%   \epsilon \leq 0.087
%   \qquad\text{and}\qquad
%   \| \tc \| \leq 0.174 .
% \]
% \note[J]{New Version}
\[
\epsilon \leq 0.09
\qquad\text{and}\qquad
\| \tc \| \leq 0.18 .
\]
We now derive further bounds on $c=\tc/\epsilon$, 
 as in the proof of Lemma~\ref{lem:wrightbounds}.
Namely, for 
% \change[J]{$|\alpha-\pp| \leq 0.00475$,
% 	$|\omega-\pp| \leq 0.061$ and  $\epsilon \leq 0.087$,}
	$|\alpha-\pp| \leq 0.00553$,
	$|\omega-\pp| \leq 0.0924$ and  $\epsilon \leq 0.09$,
we find that  $z_*^+$, as defined in~\eqref{e:zstar}, is bounded below by 
%\change[J]{$z^+_* \geq 0.662$}
 $z^+_* \geq 0.595$. 
It follows that 
%\change[J]{$ \| \tc \|  \leq 0.174 \leq z^+_*$,}
$ \| \tc \|  \leq 0.18 \leq z^+_*$, 
so we infer from Lemma~\ref{lem:Cone}(a) that 
$\| \tc \| \leq z^-_* $.
Via Lemma~\ref{lem:ZminusBound} and an interval arithmetic computation, the latter can be bounded above, for 
% \change[J]{$|\alpha-\pp| \leq 0.00475$,
% 	$|\omega-\pp| \leq 0.061$ and  $\epsilon \leq 0.087$, by
% 	$z_*^- \leq 0.13135 \epsilon$. }
	 $|\alpha-\pp| \leq 0.00553$,
	$|\omega-\pp| \leq 0.0924$ and  $\epsilon \leq 0.09$, by
	$z_*^- \leq 0.30226 \epsilon$. 
Hence 
%\change[J]{$\| c \| \leq z_*^- / \epsilon \leq 0.13135$.}
 $\| c \| \leq z_*^- / \epsilon \leq 0.30232$.
%
We conclude that $y$ corresponds to a zero of $F_\epsilon(\alpha,\omega,c)$ in the parameter set described by Theorem~\ref{thm:UniqunessNbd}, which implies uniqueness.
\end{proof}

%%%
%%%
%%%Existence: I THINK EXISTENCE IS NOT GOING TO WORK FOR THE WHOLE RANGE OF $\alpha$. NAMELY $y \approx 2 \epsilon \cos (\pp t)$, hence  $y' \approx - 2 \pp \epsilon \sin (\pp t)$, and with $\omega \approx \pp$ this gives AT THE VERY BEST
%%%\[  
%%% \| y' \|_{L^2([0,2\pi/\omega])} \approx \sqrt{\int_0^4 \pi^2 \epsilon^2 \sin^2 (\pp t) dt } = \sqrt{2} \pi \epsilon .
%%%\]
%%%Since $\alpha-\pp \approx \frac{1}{5}(\frac{3\pi}{2}-1) \epsilon^2 = 0.74 \epsilon^2 $, we get
%%%$\| y' \|_{L^2} \approx 5.1 \sqrt{\alpha-\pp}$ AT THE VERY BEST (ON IN FACT WE WILL LOSE SOME IN THE ESTIMATE).
%%%
%%%WE NEED TO REFORMULATE THE THEOREM, PROBABLY WITH at most one INSTEAD OF a unique, BUT THAT DEPENDS ON WHAT JONATHAN NEEDS IN HIS OTHER PAPERS. 
%%%we still need to show that the unique solution described by Theorem~\ref{thm:UniqunessNbd} satisfies the bound $ \| y' \|_{L^2} \leq  0.295$.
%%%WE NEED TO ESTABLISH THAT THE solution given by Theorem~\ref{thm:UniqunessNbd} satisfies a tighter bound on $c$.
%%%We then do some estimate estimate.
%%%\[
%%%  \| y' \|_{L^2([0,2\pi/\omega])}^2 = \frac{4\pi}{\omega} \sum_{k=1}^\infty k^2 |a_k|^2 \leq  ???
%%%\] 




% OLD PROOF
%
% \begin{proof}
%
% 	If $ y$ is a periodic solution to Wright's equation, then by time translation, we may assume that $y$ can be written in the form:
% 	\[
% 	y(t) =
% 	\epsilon \left( e^{i \omega t }  + e^{- i \omega t }\right)
% 	+  \sum_{k = 2}^\infty    c_k e^{i \omega k t }  + \bar{c}_k e^{- i \omega k t }
% 	\]
% 	for some $ \epsilon  \geq 0$ and $ c \in \ell^1 / \C$.
% 	By Theorem \ref{thm:FourierEquivalence2} we know that $ y$ is a solution to Wright's equation at parameter $ \alpha$ if and only if $ \tilde{F}(\alpha ,\omega, c) =0$ at parameter  $\epsilon$.
%
% 	If we can show that $ \epsilon \in (0,0.08]$ and $ \| c \|_{ \ell^1 / \C} \leq 0.657$ then the hypothesis of Theorem~\ref{thm:UniqunessNbd} will be satisfied and we are done.
% 	By the calculation done in Line \ref{eq:WrightConjectureEll1BoundL2} it follows that  $( 2 \epsilon +  \|c\|_{\ell^1 / \C} ) < \tfrac{\pi}{\omega \sqrt{3}} \| y' \|_{L^2} $.
% 	For $|  \omega - \pp | < 0.0811$ and $ \| y'\|_{L^2} < 0.131$, it follows that $ ( 2 \epsilon +  \|c\|_{\ell^1 / \C} ) < 0.16$ whereby  $ \| c \|_{\ell^1 / \C} < 0.16 $ and  $\epsilon$ is in the range $ 0 \leq \epsilon \leq 0.08$.
% 	To show that $\epsilon$ is positive, we apply  Proposition \ref{prop:Cone}.
% 	This result tells us that if $0 \leq  \epsilon \leq 0.08$ and  $ |\alpha - \pp| < 0.0044$ and $ | \omega - \pp| < 0.0811$, then either $ \| c\|_{\ell^1 / \C} < 0.121691 \times \epsilon $ or $ \| c\|_{\ell^1 / \C} > 0.657$.
% 	Hence if $ \epsilon = 0$ and $\| c\|_{\ell^1 / \C} < 0.16$  then the only solution to $ \tilde{F}(\alpha,\omega,c)=0$ is the trivial solution, thus proving that $ \epsilon \in ( 0 , 0.08]$.
% 	Since the hypothesis of Theorem~\ref{thm:UniqunessNbd} is satisfied, then for each $ \alpha \in ( \pp, \pp + 0.0044]$ there exists a unique periodic orbit $y$ with frequency $ | \omega - \pp| < 0.0811$ and $\| y'\|_{L^2} < 0.131$.
%
%
%
%
%
% \end{proof}
%

