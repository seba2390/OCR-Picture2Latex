%!TEX root = hopfwright.tex

%%%%%%%%%%%%%%%%%%
%%% Appendix E %%%
%%%%%%%%%%%%%%%%%%

\section{Appendix: A priori bounds on periodic orbits}
\label{appendix:aprioribounds}

In order to isolate periodic orbits, we need to separate them from the trivial solution. In this appendix we prove some lower bounds on the size of periodic orbits. First we work in the original Fourier coordinates. Then we derive refined bounds in rescaled coordinates.

Recall that periodic orbits of Wright's equation corresponds to  zeros of 
$G(\alpha,\omega,\c)=0$, as defined in~\eqref{e:defG}. Clearly $G(\alpha,\omega,0)=0$ for all frequencies $\omega>0$ and parameter values $\alpha>0$. There are bifurcations from this trivial solution for $\alpha=\alpha_n:=\pp(4n+1)$ for all $n\geq 0$. The corresponding natural frequency is $\omega=\alpha_n$, but there are bifurcations for any $\omega = \alpha_n/ \tilde{n}$ with $\tilde{n} \in \N$ as well, which are essentially copies of the primary bifurcation. The following proposition quantifies that away from these bifurcation points the trivial solution is isolated.
% \begin{proposition}
% 	\label{prop:zeroneighborhood}
% 	Suppose $G(\alpha,\omega,\c)=0$ for some $\alpha,\omega >0$.
% Then either $\c \equiv 0$ or
% 	\begin{equation}
% 	\| \c \|^2 \geq \frac{1}{4\alpha^2} \min_{k \in \N} (\alpha-k\omega)^2 + 2 \alpha k \omega ( 1- \sin k \omega )
% 	\end{equation}
% \end{proposition}
%
% \begin{proof}
% We fix $\alpha,\omega>0$ and define
% \[
%   \beta_1 := \min_{k \in \N} (\alpha-k\omega)^2 + 2 \alpha k \omega ( 1- \sin k \omega ).
% \]
% If $\beta_1=0$ there is nothing to prove. From now on we assume that $\beta_1>0$.
% We define a Newton-like map
% $N : \ell^1 \to \ell^1$ by
% \begin{equation}
% N(\c) := \c - (i \omega K^{-1} + \alpha U_{\omega})^{-1} G(\alpha,\omega,\c).
% \end{equation}
% We note that $i \omega K^{-1} + \alpha U_{\omega}$, which is the derivative of $G$ at $\c=0$, is invertible, since
% for any $k \in \N$
% \begin{alignat*}{1}
%   |ik\omega + \alpha e^{-i k \omega}|^2
%   & =  (\alpha \cos k \omega )^2 + ( \omega - \alpha \sin  k \omega )^2  \\
%   &  = (k \omega)^2 + \alpha^2 - 2 \alpha  k \omega \sin k \omega   \\
% &= (\alpha - k\omega)^2 + 2 \alpha k \omega ( 1 - \sin k \omega)  \\
% &\geq \beta_1 >0.
% \end{alignat*}
% Fixed points of $N$ thus correspond to zeros of $G$.
% Naturally, $ N(0) =0$.
% Next we show that $N$ is a contraction map on any ball $B_R := \{ \c \in \ell^1 : \|\c\| \leq R \} $ with $R < \beta_1^{1/2} / (2\alpha)$. We then apply the Banach fixed point theorem to conclude that there are no nontrivial fixed points with $\|\c\| < \beta_1^{1/2} / (2\alpha)$.
% The derivative of $N$ is
% \begin{equation}
% DN(\c)\tilde{\c} =  - \alpha (i \omega K^{-1} + \alpha U_{\omega})^{-1}  (
% [U_\omega \c ]  * \tilde{\c} ) .
% \end{equation}
% We estimate
% \[
% \| DN(\c)\tilde{\c} \| \leq 2 \alpha \|\c \| \cdot \|\tilde{\c}\| \cdot
% \| \omega K^{-1} +  \alpha U_{\omega})^{-1} \| .
% \]
% Since $\| ( \omega K^{-1} + \alpha U_{\omega})^{-1} \| = \beta_1^{-1/2}$, we find that
% $\| DN(\c)\| \leq 2 \alpha R \beta_1^{-1/2} < 1$ for all $\c \in B_R$.
% Hence $N$ is a contraction for $R < \beta_1^{1/2}/ (2\alpha)$.
% \end{proof}

\begin{proposition}
	\label{prop:zeroneighborhood2}
	Suppose $G(\alpha,\omega,\c)=0$ for some $\alpha,\omega >0$. 
Then either $\c \equiv 0$ or 
	\begin{equation}\label{e:minoverk}
	\| \c \| \geq \min_{k \in \N}  \sqrt{ \left(1-k \,\frac{\omega}{\alpha} \right)^2 + 2  k \, \frac{\omega}{\alpha} \bigl( 1- \sin k \omega \bigr)} .
	\end{equation}	
\end{proposition}

\begin{proof}
We fix $\alpha,\omega>0$ and define 
\[
  \beta_1 := \min_{k \in \N} \, (\alpha-k\omega)^2 + 2 \alpha k \omega ( 1- \sin k \omega ).
\]
If $\beta_1=0$ then there is nothing to prove. From now on we assume that $\beta_1>0$. 
We recall that 
\[ G(\alpha,\omega,\c) = (i \omega K^{-1} + \alpha U_{\omega}) \c + \alpha \left[U_\omega \, \c \right] * \c .
\]
We note that $i \omega K^{-1} + \alpha U_{\omega}$ is invertible, since
for any $k \in \N$
\begin{alignat*}{1}
  |ik\omega + \alpha e^{-i k \omega}|^2
  & =  (\alpha \cos k \omega )^2 + ( \omega - \alpha \sin  k \omega )^2  \\
  &  = (k \omega)^2 + \alpha^2 - 2 \alpha  k \omega \sin k \omega   \\
&= (\alpha - k\omega)^2 + 2 \alpha k \omega ( 1 - \sin k \omega)  \\
&\geq \beta_1 >0.
\end{alignat*}
We may thus rewrite $G(\alpha,\omega,\c) = 0$ as 
\begin{equation}\label{e:quadratic}
 \c = - \alpha  (i \omega K^{-1} + \alpha U_{\omega})^{-1} ( \left[U_\omega \, \c \right] * \c ) .
\end{equation}
Since $\| ( \omega K^{-1} + \alpha U_{\omega})^{-1} \| = \beta_1^{-1/2}$
and $\| \left[U_\omega \, \c \right] * \c \| \leq \| \c \|^2 $,
we infer from~\eqref{e:quadratic} that
\[
  \| \c \| \leq  \alpha \beta_1^{-1/2} \|\c\|^2.
\]
We conclude that either $\c \equiv 0$ or $\| \c \| \geq \beta_1^{1/2} /\alpha $.
\end{proof}

\begin{proposition}
	\label{prop:G1Minimizer}
	Suppose that $ \omega \geq 1.1$ and $ \alpha \in (0,2]$. Define 
	\begin{equation}
g_k(\omega,\alpha) =		\left(1-k \,\tfrac{\omega}{\alpha} \right)^2 + 2  k \, \tfrac{\omega}{\alpha} \bigl( 1- \sin k \omega \bigr) .
	\end{equation}
Then $ g_1 <g_k$ for all $ k \geq 2$. 
\end{proposition}

\begin{proof}
%%%	We first prove that $ g_k > g_1$ for all $ k\geq 3$.   
%%%	For all $k$ we may estimate $ g_k > (1 - k \tfrac{\omega}{\alpha})^2$. 	
%%%	If $ \omega \in [1.1,2]$  then $1- \sin ( \omega ) \in [0,0.11]$, and so 
%%%	\[
%%%	g_1 < ( 1- \tfrac{\omega}{\alpha})^2 + 0.22 \tfrac{\omega}{\alpha}
%%%	\]
%%%	If we write $ x = \tfrac{\omega}{\alpha}$ then these estimates produce:
%%%	\begin{eqnarray}
%%%		g_k &>& (kx)^2 - 2 k x +1  \\
%%%		g_1 &<& x^2 - 1.78x +1 
%%%	\end{eqnarray}
%%%	Hence we can show $ g_k > g_1$ by showing 
%%%	\[
%%%	x - 1.78 < k^2 x - 2 k
%%%	\]
%%%	or equivalently 
%%%\begin{equation}
%%%		(2k-1.78) < (k^2-1)x  \label{eq:G3G1}
%%%\end{equation}
%%%	For our range of $ \alpha$ and $\omega$ it follows that $  x \in [0.55, 4/3]$.  
%%%	Since Equation \ref{eq:G3G1} is true for $ x = 0.55$ and $ k \geq 3$, then it follows that $ g_1 < g_k$ for all $ k \geq 3$. 
%%%	
%%%	We now show that $ g_1 < g_2$. 
%%%	
%%	
%%	
	
This is equivalent to showing that 
\[
(1- \tfrac{\omega}{\alpha})^2 + 2 \tfrac{\omega}{\alpha} ( 1 - \sin \omega ) 
<
(1- k \tfrac{\omega}{\alpha})^2 + 2k  \tfrac{\omega}{\alpha} ( 1 - \sin k\omega )  
\qquad\text{for } k\geq 2.
\]
Making the substitution $ x = \tfrac{\omega}{\alpha}$, we can simplify this to the equivalent inequality 
% \begin{eqnarray}
% (1 - 2 x + x^2 ) + 2 x ( 1 - \sin \omega )
% &<&
% (1 - 2k x + k^2 x^2 ) + 2k x ( 1 - \sin k\omega )
% \\
% 2 x ( 1 - \sin \omega )
% &<&
% - 2(k-1) x + (k^2-1)x^2  + 2k x ( 1 - \sin 2\omega )
% \\
% -2  \sin \omega
% &<&
\[
 (k^2-1) x  + 2  \sin \omega - 2k  \sin k\omega >0 .
\]
% \end{eqnarray}
Since $\alpha \leq 2$, we have $x\geq \omega/2$. Hence it suffices to prove that
%
%
% $ x = \tfrac{\omega}{\alpha} $ is minimized for large $\alpha$, we may simplify this as below using our initial range of $ \alpha \in [1.5,2]$.
\begin{equation}
  h_k(\omega) := \frac{k^2-1}{2} \omega + 2 \sin \omega  - 2 k \sin k\omega  >0 
  \qquad\text{for all } k\geq 2.
\label{eq:GminRedux}
\end{equation}
We first consider $k=2$. It is clear that $h_2(\omega) > 0$ for $\omega > 4$.
We note that $h_2$ has a simple zero at $ \omega \approx 1.07146$ and it is easy to check using interval arithmetic that $h_2(\omega)$ is positive for $\omega \in [1.1,4]$. Hence $h_2(\omega) > 0$ for all $\omega \geq 1.1$.


For $k=3$ and $k=4$ we can repeat a similar argument. For $k \geq 5$ it is immediate that $h_k(\omega) > \frac{k^2-1}{2}-2-2k \geq 0$ for $\omega > 1$.
%
% One can further check that the the RHS of Equation \ref{eq:GminRedux} is positive for all $ k \geq 2 $ and $ \omega \in [1.1,2]$.
% Thereby, it follows that
% \[
% g_1(\omega,\alpha) < g_{k} (\omega,\alpha)
% \]
% for $ \alpha \in [1.5,2]$ and $ \omega \in [1.1,2]$.
%
%
%
\end{proof}


As discussed in Section~\ref{s:preliminaries},
the function $G(\alpha,\omega,\c)$ gets replaced by $\tF_\epsilon(\alpha,\omega,\tc)$ in rescaled coordinates. 
In these coordinates we derive a result analogous to Proposition~\ref{prop:zeroneighborhood2} below, see Lemma~\ref{lem:thecone}.
First we bound the inverse of the operator $\B \in B(\ell^1_0)$ defined by
%\marginpar{NOTE THAT $\B = \alpha^{-1} B K $}
\[
  \B:= i \frac{\omega}{\alpha} I +  U_{\omega} K +  \epsilon L_{\omega} K,
\]
where $K$, $U_{\omega}$ and $L_\omega$ have been introduced in Section~\ref{s:preliminaries}.
\begin{lemma}\label{lem:gamma}
	Let $\epsilon \geq 0$ and $\alpha,\omega>0$. Let
\[
  \gamma := 	  \frac{1}{2} +
  \epsilon \left( \frac{2}{3} + \max\left\{  \frac{\sqrt{2 - 2 \sin (\omega-\pp) }}{2} ,\frac{2}{3} \right\} \right).
\] 
If $\gamma < \omega / \alpha$ then the operator $\B$ is invertible and the inverse is bounded by
\[
	  \| \B^{-1} \| \leq \frac{1}{\frac{\omega}{\alpha}- \gamma}.
\]
\end{lemma} 

\begin{proof}
Writing 
\[
  \B= i \frac{\omega}{\alpha} \left( I + \frac{\alpha}{i\omega} \left( U_{\omega} +  \epsilon L_{\omega} \right) K \right)
\]
and using a (formal) Neumann series argument, we obtain
% \marginpar{Side remark: did you consider the splitting $\B = i
% \frac{\omega}{\alpha} [I+ \frac{\alpha}{i\omega} \epsilon L_\omega K Q^{-1}] Q
% $ with $Q=I+\frac{\alpha}{i\omega} U_\omega K $ ? Using the estimate from
% Prop~\ref{prop:zeroneighborhood2} this may or may not lead to a better bound.}
\[
  \| \B^{-1} \| \leq \frac{\alpha}{\omega}
   \sum_{n=0}^\infty \left( \frac{ \alpha}{ \omega} \right)^n \|(U_{\omega} + \epsilon L_{\omega}) K\|^n
   \leq \frac{\frac{\alpha}{\omega} }{1- \frac{ \alpha}{ \omega} \|(U_{\omega} + \epsilon L_{\omega}) K\|} 
   = \frac{1}{\frac{\omega}{\alpha}- \|(U_{\omega} + \epsilon L_{\omega}) K\|} .
\]
It remains to prove the estimate $\|(U_{\omega} + \epsilon L_{\omega}) K\| \leq \gamma$.
Then, in particular, for $\gamma < \omega/\alpha$ the formal argument is rigorous.

Recalling that $L_\omega= \sigma^+ (e^{-i\omega} I  + U_\omega) + \sigma^- (e^{i\omega} I  + U_\omega)$, we use the triangle inequality
\[
\|(U_{\omega} + \epsilon L_{\omega}) K\| 
\leq \| U_\omega K \|  + \epsilon \| \sigma^+ (e^{-i\omega} I  + U_\omega) K \| + \epsilon \| \sigma^- (e^{i\omega} I  + U_\omega) K \|,
\] 
and estimate each term separately as operator on $\ell^1_0$. 
We recall the formula~\eqref{e:operatornorm} for the operator norm.
Using that $\| K \tc \| \leq \frac{1}{2} \|\tc\|$ for all $\tc \in \ell^1_0$,
the first term is bounded by $\| U_\omega K \| \leq \frac{1}{2}$.
Since $\sigma^-$ shifts the sequence to the left and we consider the operators acting on $\ell^1_0$, we obtain $\| \sigma^- (e^{i\omega} I  + U_\omega) K \| \leq \frac{2}{3}$.
For the final term, $\| \sigma^+ (e^{-i\omega} I  + U_\omega) K \|$, 
to obtain a slightly more refined estimate,
we first consider the action of $\sigma^+ (e^{-i\omega} I  + U_\omega) K$ 
on $\e_2$. We observe that 
\[ 
  | e^{- i \omega} + e^{-2 i \omega}| = \sqrt{2 - 2 \sin (\omega-\pp) },
\]
hence $\| \sigma^+ (e^{-i\omega} I  + U_\omega) K \e_2 \| \leq 
\sqrt{2 - 2 \sin (\omega-\pp) } $, 
leading to
\[
 \| \sigma^+ (e^{-i\omega} I  + U_\omega) K \| 
 \leq \max\left\{  \frac{\sqrt{2 - 2 \sin (\omega-\pp) }}{2} ,\frac{2}{3} \right\}.
\]
We conclude that 
\[
\|(U_{\omega} + \epsilon L_{\omega}) K\| 
\leq 
  \frac{1}{2} +
  \epsilon \left( \frac{2}{3} + \max\left\{  \frac{\sqrt{2 - 2 \sin (\omega-\pp) }}{2} ,\frac{2}{3} \right\} \right).
\]
\end{proof}

\begin{lemma}\label{lem:thecone}
	Fix $ \epsilon \geq 0$, $\alpha,\omega>0$.
	Assume that $\B$ is invertible.
	Let $b_0$ be a bound on $\| \B^{-1} \|$.
	Define 
	\[
	z^{\pm} = b_0^{-1} \pm \sqrt{b_0^{-2}-  2\epsilon^2 } .
	\]
	Let $ \tc \in \ell^1_0$ be such that $\tF_\epsilon(\alpha, \omega,\tc) = 0$, then either $ \|\tc\| \leq  z^-$ or $  \|\tc\| \geq z^+ $. 
	\noindent
	Additionally, $ \| K^{-1} \tc \| \leq b_0 (2\epsilon^2+ \|\tc\|^2)$.
\end{lemma}

\begin{proof}
	If  $ \tF_\epsilon( \alpha, \omega, \tc) =0$ then it follows that the equations $\pi_c \tF_\epsilon=0$ can be rearranged as 
\begin{equation}\label{e:eBc2}
  \tc = - K \B^{-1} (  \epsilon^2  e^{- i \omega} \e_2 + \ [ U_{\omega} \tc ] * \tc ) .
\end{equation}
	Taking norms, and using that $\| K \tc \| \leq \frac{1}{2} \| \tc\|$ for all $\tc\in \ell^1_0$, we obtain 
\begin{equation}\label{e:quadineq}
\|\tc \|  \leq  \frac{1}{2} \| B^{-1}\| \left( \epsilon^2 \|\e_2\|  + \| [ U_{\omega} \tc ] * \tc \| \right)
\leq \frac{1}{2} b_0 \left( 2 \epsilon^2  + \| \tc \|^2 \right).
\end{equation}
The quadratic $x^2 - 2 b_0^{-1} x +   2\epsilon^2 $
has two zeros $z^+$ and $ z^-$ given by
	\[
	z^{\pm} = b_0^{-1} \pm \sqrt{b_0^{-2}-  2\epsilon^2 } .
	\]
The inequality~\eqref{e:quadineq} thus implies that either $ \|\tc\| \leq z^-$ or $ \|\tc\| \geq z^+$.

Furthermore, it follows from~\eqref{e:eBc2} that $\| K^{-1} \tc \| \leq  \| \B^{-1} \| \, (2 \epsilon^2 + \|\tc \|^2) \leq b_0 (2 \epsilon^2+ \|\tc\|^2)$.
\end{proof}

In practice we use the bound $\| \B^{-1} \| \leq b_*^{-1}$,
where
\[
  b_*(\epsilon) := \frac{\omega}{\alpha} - \frac{1}{2} - \epsilon  \left(\frac{2}{3}+ \frac{1}{2}\sqrt{2 + 2 |\omega-\pp| } \right).
\]
When doing so, we will refer to $z^\pm$ as $ z^\pm_*$. 
Additionally, we will need the following monotonicity property.
\begin{lemma}
	\label{lem:ZminusBound}
	Fix $\alpha, \omega, \epsilon_0 >0$ and assume that $ \epsilon_0 \leq b_*(\epsilon_0) /\sqrt{2}$.
	Define 
\[
  z_*^- (\epsilon):= b_*(\epsilon)-\sqrt{(b_*(\epsilon))^2 -2 \epsilon^2}.
 \]
Let $C_0 := \frac{z_*^-(\epsilon_0)}{\epsilon_0}$.
Then
	\begin{equation}
	 z_*^-(\epsilon) \leq C_0 \epsilon
	 \qquad \text{for all } 0 \leq \epsilon \leq \epsilon_0.
	 \label{eq:ConeLemma}
	\end{equation}
\end{lemma}


\begin{proof}
	Let $x:=\sqrt{2} \epsilon/b_*(\epsilon) \geq 0$. Clearly $\frac{d}{d\epsilon} x >0$.
It thus suffices to observe that
\[
  \frac{z_*^- (\epsilon)}{\epsilon} = \sqrt{2}\,\frac{1 - \sqrt{1-x^2}}{x}
\]	
is increasing for $x \in [0,1]$. 
%\marginpar{or $y-\sqrt{y^2-1}$ is decreasing}
%
% Throughout, we write $ b_* := b_*(\epsilon)$ and
% 	$ z_*^- =z_*^-(\epsilon)$.
% 	Writing $x=\epsilon/b_*(\epsilon)$
% 	First we may rewrite $ z_*^-$ as follows:
% 	\begin{eqnarray}
% 	 z_*^- &=& b_*-\sqrt{(b_*)^2 -\epsilon^2} \\
% 	 &=& b_*\left(1 - \sqrt{1-(\epsilon/b_*)^2} \right)
% 	\end{eqnarray}
% 	By assumption $|\epsilon_0/ b_*| <1 $ so for all $0 \leq  \epsilon \leq \epsilon_0$ the following Taylor expansion is valid:
% 	\begin{equation}
% 		z_*^-(\epsilon) = \frac{\epsilon^2}{2 b_*} + \frac{\epsilon^4}{8 (b_*)^3} + \frac{\epsilon^6}{16 (b_*)^5} + \dots
% 	\end{equation}
% In particular, we note that $z_*^-(\epsilon)$ can be expressed as $b_*$ times a power series in $(\epsilon/b_*)^2$ with strictly positive coefficients.
% 	Since $ \tfrac{d}{d \epsilon} b_*(\epsilon) = - \left( \tfrac{2}{3} + \tfrac{1}{2} \sqrt{2+2|\omega - \pp|}\right) < 0  $, then it follows that all of the functions:
% 	\begin{align}
% 		z_*^-(\epsilon), && \frac{z_*^-(\epsilon)}{\epsilon}, &&\frac{z_*^-(\epsilon)}{\epsilon^2}
% 	\end{align}
% 	are well defined at $ \epsilon=0$ and monotonically increasing in $\epsilon$.
% 	Hence, for all $ 0 \leq \epsilon < \epsilon_0$,  we have the inequality
% 	\[
% 	\frac{z_*^-(\epsilon)}{\epsilon} < 	\frac{z_*^-(\epsilon_0)}{\epsilon_0}
% 	\]
% 	whereby Equation \ref{eq:ConeLemma} follows.
%
\end{proof}

%%%%%%%%%%%%%%%%%%%%%%%%%%%%%%%%%%%%%%%%%%%%%%%%%%%%%%%%%%%%%%%%%%%%%

