%!TEX root = hopfwright.tex

%%%%%%%%%%%%%%%%%%
%%% Appendix F %%%
%%%%%%%%%%%%%%%%%%

\section{Appendix: Implicit Differentiation}
\label{sec:Appendix_Implicit_Diff}

%\note[JB]{I have shortened this initial paragraph considerably}.
% We calculate  $\frac{\partial F}{\partial  \epsilon} (x) $ to be:
% \[
% \frac{\partial F}{\partial  \epsilon}( x ) = \alpha e^{-i \omega} \e_2 + \alpha L_\omega c + \alpha [ U_\omega c] * c
% \]
% In keeping with our $ \cO(\epsilon^2)$ approximations, we will use
% \[
% \tilde{x}_\epsilon = ( \pp, \pp, c_2(\epsilon), 0, 0, \dots)
% \]
% as the center of our approximation.
% Furthermore, we can drop the $ \alpha [U_\omega c] * c$ term in our initial expansion.
% We thereby obtain an approximation which we will call $\Gamma$:
%
% \begin{eqnarray}
% \frac{\partial F}{\partial  \epsilon} (\bar{x}_\epsilon) + \cO(\epsilon^2) &=& [-\pp i]_2 + \pp L_{\omega_0} [ \tfrac{2-i}{5} \epsilon]_2 \\
% &=& [-\pp i]_2 + \pp \left[ \sigma^+ ( -i-1) + \sigma^-(i-1) \right] [\tfrac{2-i}{5} \epsilon]_2 \\
% \Gamma &:=& \pp[\tfrac{3i -1}{5} \epsilon]_1 - \pp [i]_2 - \pp[\tfrac{3+i}{5} \epsilon]_3
% \end{eqnarray}
%

We will approximate
\[
\frac{\partial F}{\partial  \epsilon}( x ) = \alpha e^{-i \omega} \e_2 + \alpha L_\omega c + \alpha [ U_\omega c] * c 
\]
by
\begin{alignat}{1}
\Gamma & := \pp \tfrac{3i -1}{5} \epsilon \e_1 - \pp i \e_2 - \pp \tfrac{3+i}{5} \epsilon \e_3 \label{e:defGamma}  \\
&=   -\pp i \e_2 + \pp L_{\omega_0} \bc_\epsilon , \label{e:defGamma2}
\end{alignat}
which has been chosen so that $\frac{\partial F}{\partial  \epsilon} (\pp,\pp,\bc_\epsilon) - \Gamma = \cO(\epsilon^2)$.
%\remove[JB]{Given our approximation $\Gamma$, we can then calculate $ A^\dagger \Gamma$, as we do in the following lemma:}
\begin{lemma}
	\label{lem:ImplicitApprox}
When we write 
$A^\dagger \Gamma = (\alpha', \omega', c') \in \R^2 \times \ell^K_0$, then 
% \note[JB]{shouldn't $\tfrac{1-2i}{5}$ be $\tfrac{1+2i}{5}$? I think the second expression for $c'$ is better.} \note[J]{Agreed}
\begin{alignat*}{1}
		\alpha ' &= - \tfrac{2}{5} ( \tfrac{3 \pi}{2}-1) \epsilon ,\\
		\omega ' &= \tfrac{2}{5} \epsilon , \\
%		c '	 &= \left[ (\tfrac{1-2i}{5}) - 
%		\tfrac{\epsilon^2}{25} ( \tfrac{29 - 22i}{5} + \tfrac{1 + 7 i }{2})
%		 \right] \e_2 + ( \tfrac{3i-1}{10} )\epsilon \, \e_3 . \\
		c '	 &= \left[ (\tfrac{1+2i}{5}) - 
		\epsilon^2 \tfrac{9 }{250} (7-i)
		 \right] \e_2 + \epsilon \tfrac{3i-1}{10} \, \e_3 .
	\end{alignat*}
\end{lemma}

\begin{proof}
	First we calculate the $ \alpha$ and $ \omega$ components of the image of $ A^\dagger $:	
%	\begin{eqnarray}
%	[ A^\dagger]_{\alpha,\omega}  &=& \left[ A_0^{-1} [ I - \epsilon A_1 A_0^{-1}] \right]_{\alpha,\omega} \\
%	&=&   A_{0,1}^{-1} [ I - \epsilon \pp L_{\omega_0}  A_{0,*}^{-1}] \\
%	&=&   A_{0,1}^{-1} [ I - \epsilon \sigma^{-} ( iI + U_{\omega_0} ) ( i K^{-1} + U_{\omega_0})^{-1}]\\ 
%	&=&  A_{0,1}^{-1} [ e_1^* - \epsilon ( \tfrac{3 +i}{5}) e_2^* ]
%	\end{eqnarray}
%\note[JB]{Notational cleanup:}
	\begin{alignat}{1}
	\pi_{\alpha,\omega} A^\dagger  &= A_{0,1}^{-1} i_\C^{-1} \pi_1 [ I - \epsilon A_1 A_0^{-1}] ] \nonumber \\
	&=   A_{0,1}^{-1} i_\C^{-1} \pi_1 [ I - \epsilon \pp L_{\omega_0}  A_{0,*}^{-1}] \nonumber \\
	&=   A_{0,1}^{-1} i_\C^{-1} \pi_1 [ I - \epsilon \sigma^{-} ( iI + U_{\omega_0} ) ( i K^{-1} + U_{\omega_0})^{-1}] \nonumber \\ 
	&=  A_{0,1}^{-1} i_\C^{-1} [ \pi_1- \epsilon ( \tfrac{3 +i}{5}) \pi_2 ].
	\label{e:piaoAdag}
	\end{alignat}
Here we have used projections $\pi_k a = a_k$ for $a=\{a_k\}_{k \geq 1} \in \ell^1$.
	We now calculate the $\alpha $ and $\omega$ components of $ A^\dagger \Gamma$. 
	It follows from~\eqref{e:defGamma} and~\eqref{e:piaoAdag} that
	\begin{alignat*}{1}
	\pi_{\alpha,\omega} A^\dagger \Gamma  
%	&= A_{0,1}^{-1} [ e_1^* - \epsilon ( \tfrac{3 +i}{5}) e_2^* ] \left(\pp[\tfrac{3i -1}{5} \epsilon]_1 - \pp [i]_2 - \pp[\tfrac{3+i}{5} \epsilon]_3 \right) \\
	&= A_{0,1}^{-1} i_\C^{-1} \left[ \pp \tfrac{3i -1}{5} \epsilon +  \pp  \tfrac{3 +i}{5}  i \epsilon \right]
	\\
	&= \tfrac{\pi \epsilon}{5} A_{0,1}^{-1} i_\C^{-1} ( 3 i -1) 
	 \\
	&= - \frac{2 \epsilon}{5}
	\left(
	\begin{array}{c}
	\tfrac{3\pi}{2}-1 \\
	-1
	\end{array}
	\right) .
	\end{alignat*}
	
	We now calculate 
%\note[JB]{Notational cleanup, but does not look pretty.} 
%	\[
%	\pi_c A^{\dagger} \Gamma  = A_{0,*}^{-1}  \pi_{\geq 2} [ I - \epsilon ( \e_2 [i_\C A_{1,2}A_{0,1}^{-1} i_\C^{-1} \pi_1] +A_{1,*} A_{0,*}^{-1} \pi_{\geq 2} ) ]\Gamma .
%	\]
%\note[JB]{It seems to me it is better to write this as}
\begin{equation}\label{e:picAdagGamma}
	\pi_c A^{\dagger} \Gamma  = A_{0,*}^{-1}  \pi_{\geq 2} [ I - \epsilon A_1 A_0^{-1} ]\Gamma ,
\end{equation}
where
$A_1 A_0^{-1}$ decomposes as
\begin{equation}\label{e:A1A0decomposition}
  A_1 A_0^{-1} = 
  \e_2 [i_\C A_{1,2}A_{0,1}^{-1} i_\C^{-1} \pi_1] 
  +A_{1,*} A_{0,*}^{-1}  \pi_{\geq 2} . 
\end{equation}
We first calculate  
% \note[JB]{Here I think the $\tfrac{1-2i}{5}$ should be $\tfrac{1+2i}{5}$, but I did not implement it yet.} \note[J]{I agree. I've changed it now.}
\begin{alignat}{1}
	A_{0,*}^{-1} \pi_{\geq 2} \Gamma &= \tfrac{2}{ \pi } ( i K^{-1} + U_{\omega_0} )^{-1}  [ - \pp i \e_2 - \pp \tfrac{3+i}{5}\epsilon \e_3 ] \nonumber\\
	&= -(2i-1)^{-1} \e_2 - (3i+i)^{-1} \tfrac{3+i}{5} \epsilon \e_3 \nonumber\\
	&= \tfrac{1+2i}{5} \e_2 + \epsilon \tfrac{ 3 i-1}{20} \e_3  \label{e:A0piGamma}. 
\end{alignat}

%\note[JB]{Rearranged the rest of the proof; I found it hard to follow the flow of the argument.}
Since $\Gamma$ has three nonzero components only, we next compute the action of $A_{0,*}^{-1}  \pi_{\geq 2} A_1 A_0^{-1} $ on each of these.
Taking into account the decomposition~\eqref{e:A1A0decomposition},
we first compute its action on $\lambda \e_1$ for $\lambda \in \C$.
After a straightforward but tedious calculation we obtain
%\note[JB]{Hope this is correct.} \note[J]{This checks out.}
\begin{alignat*}{1}
A_{0,*}^{-1}  \pi_{\geq 2} A_1 A_0^{-1}  \lambda \e_1
&= 
 [i_\C A_{1,2}A_{0,1}^{-1} i_\C^{-1} \lambda ] A_{0,*}^{-1} \e_2
\\
&= -\tfrac{2}{25\pi} \bigl[ (11+2i) \text{Re} \lambda  + (-6+8i) \text{Im} \lambda \bigr]  \e_2.
\end{alignat*}
Next, we compute the action of $A_{0,*}^{-1}  \pi_{\geq 2} A_1 A_0^{-1} $
on  $\e_k$ for $k=2,3$:
%\note[JB]{Hope these are correct} \note[J]{This also checks out. }
\begin{alignat*}{1}
A_{0,*}^{-1}  \pi_{\geq 2} A_1 A_0^{-1} \e_2
&=
 A_{0,*}^{-1} [ \pp \sigma^+ (  e^{-i\omega_0} I  + U_{\omega_0}) ]A_{0,*}^{-1} \e_2 
\\& 
%= - \tfrac{2}{\pi} ( 3i +i)^{-1} (-i-1)(2i-1)^{-1} \e_3 
=  \tfrac{2}{\pi} \tfrac{3+i}{20} \e_3  ,
  \\
A_{0,*}^{-1}  \pi_{\geq 2} A_1 A_0^{-1} \e_3
&=
 A_{0,*}^{-1} [ \pp \sigma^- ( e^{i\omega_0} I + U_{\omega_0}) ]A_{0,*}^{-1} \e_3  \\
& = 
-\tfrac{2}{\pi} \tfrac{1+2i}{10}  \e_2 ,
\end{alignat*}
where we have used that $(e^{-i\omega_0}I + U_{\omega_0}) \e_3$ vanishes.
Hence, by using the explicit expression~\eqref{e:defGamma} for~$\Gamma$ we obtain
\begin{equation}\label{e:actionek}
- \epsilon A_{0,*}^{-1}  \pi_{\geq 2}  A_1 A_0^{-1} \Gamma 
= 
 -\epsilon^2 \frac{29-22i}{125} \e_2 + \epsilon \frac{3i-1}{20} \e_3 -\epsilon^2 \frac{1+7i}{50} \e_2.
\end{equation}
Finally, combining~\eqref{e:picAdagGamma}, \eqref{e:A0piGamma} and~\eqref{e:actionek} completes the proof.
%
% 	\begin{alignat*}{1}
% 	\left[ - \epsilon A_0^{-1} A_1 A_{0}^{-1} [ - \pp i ]_2 \right]_c &=  - \epsilon A_{0,*}^{-1} [ \pp \sigma^+ ( -iI + U_{\omega_0}) ]A_{0,*}^{-1} [ - \pp i ]_2 \\
% 	&= \epsilon ( 3i +i)^{-1} (-i-1)(2i-1)^{-1} [i]_3 \\
% 	&= \left[ \frac{3 i-1}{20} \epsilon \right]_3 .
% 	\end{alignat*}
% \change[JB]{We now calculate the last part of  $A^\dagger \Gamma$. }{Third,}
% 	\begin{eqnarray}
% 	- \epsilon A_0^{-1} A_1 A_0^{-1} \left( \pp [ \tfrac{3i-1}{5} \epsilon]_1 - \pp \left[ \tfrac{3+i}{5} \epsilon \right]_3 \right) &=&   - \pp \epsilon^2  A_{0,*}^{-1} A_{1,2} A_{0,1}^{-1} [ \tfrac{3 i-1}{5} ]_1
% 	+\pp \epsilon^2  A_{0,*}^{-1} A_{1,*} A_{0,*}^{-1} [ \tfrac{3 +i}{5} ]_3  \nonumber
% 	\end{eqnarray}
% 	Then we compute the two summands on the RHS.
% 	\begin{eqnarray}
% 	- \pp \epsilon^2  A_{0,*}^{-1} A_{1,2} A_{0,1}^{-1} [ \tfrac{3 i-1}{5} ]_1  &=& - \frac{\epsilon^2}{5} A_{0,*}^{-1}  [ \pi \frac{3+16i}{10}]_2 \\
% 	&=& \left[ \frac{- \epsilon^2}{125} (29-22i)\right]_2
% 	\end{eqnarray}
%
%
%
% 	\begin{eqnarray}
% 	\pp \epsilon^2  A_{0,*}^{-1} A_{1,*} A_{0,*}^{-1} [ \tfrac{3 +i}{5} ]_3 &=&
% 	\frac{\epsilon^2}{5} (i K^{-1} + U_{\omega})^{-1} L_{\omega_0}  ( 3 i + i)^{-1} [ 3 + i ]_{3} \\
% 	&=& \frac{\epsilon^2}{5} (iK^{-1} + U_{\omega_0})^{-1} [\tfrac{3+i}{2}  ]_2 \\
% 	&=&\frac{-\epsilon^2}{50} [1+7i ]_2
% 	\end{eqnarray}
% 	Combining all of these results together, we obtain our theorem.
\end{proof}
 
\begin{lemma}
	\label{lem:ImplicitLast}
Let 
%	\begin{alignat}{1}
%	\hat{f}_{\epsilon,1} &:= \frac{\epsilon}{\sqrt{5}} \left[  \da \sqrt{2} + \alpha ( \dt + \dtt ) \right] +
%	\alpha r_c \left[ 2  +  \frac{2\epsilon }{\sqrt{5}} + r_c \right] , \label{e:feps1} \\ 
%	\hat{f}_{\epsilon,c} &:= \frac{2}{\pi \sqrt{5}} \left[ 
%	\da + \pp \dt + 
%	\frac{\epsilon}{\sqrt{5}} \left[ \sqrt{2} \da + \alpha ( \dt + \dtt) \right]
%	+\alpha ( 4 r_c + \dc^2)
%	\right] .\label{e:fepsc}
%	\end{alignat}
%	\note[J]{Proposed Changes}
%	\begin{alignat}{1}
%	\hat{f}_{\epsilon,1} &:= \tfrac{1}{2} \dc^0 \left(   \sqrt{2} \da  +3  \dw ( \pp + \da ) \right)  + \tfrac{1}{2} r_c  ( \pp + \da) 
%	\left(2 + 2\dc^0  + r_c \right)  , \label{e:feps1} \\
%	%
%	% 
%	\hat{f}_{\epsilon,c} &:= 
%	\frac{2}{\pi \sqrt{5}} \left[ 
%	 2 \left( \da + \pp \dw \right) + \dc^0  [ \sqrt{2} \da + 3 \dw (\pp+\da) ]
%	+(\pp+\da) ( 4 r_c + \dc^2)
%	\right] .\label{e:fepsc}
%	\end{alignat}
%	\note[JB]{Proposed Rearrangement}
	\begin{alignat}{1}
	\hat{f}_{\epsilon,1} &:= \tfrac{1}{2} \dc^0 \left(   \sqrt{2} \da  +3  \dw ( \pp + \da ) \right)  +  r_c  ( \pp + \da) 
	\left(1 + \dc^0  + \tfrac{1}{2} r_c \right)  , \label{e:feps1} \\
	%
	% 
	\hat{f}_{\epsilon,c} &:= 
	\tfrac{2}{\pi \sqrt{5}} \left[ 
	 2 \left( \da + \pp \dw \right) + \dc^0  [ \sqrt{2} \da + 3 \dw (\pp+\da) ]
	+(\pp+\da) ( 4 r_c + \dc^2)
	\right] .\label{e:fepsc}
	\end{alignat}
	Then  the vector 
% 	\change[J]{
% 	$
% 	[(1+\tfrac{2}{\pi})\hat{f}_{\epsilon,1},\tfrac{2}{\pi} \hat{f}_{\epsilon,1}, \hat{f}_{\epsilon,c}]^{T}
% 	$
	$
	[	(1+\tfrac{4}{\pi^2})^{1/2}\hat{f}_{\epsilon,1},\tfrac{2}{\pi} \hat{f}_{\epsilon,1}, \hat{f}_{\epsilon,c}]^{T}
	$
	is an upper bound on $A_0^{-1}  ( \tfrac{\partial F}{\partial  \epsilon} (x) -\Gamma )$ for any $ x \in B_\epsilon ( r,\rho)$.
\end{lemma}


\begin{proof}
	The $\alpha$- and $\omega$-component of  $A_0^{-1}  ( \tfrac{\partial F}{\partial  \epsilon} (x) -\Gamma )$ are given by $ A_{0,1}^{-1} i_\C^{-1} \pi_1 [  \tfrac{\partial F}{\partial  \epsilon} (x) -\Gamma ]$.
	If we can show that  $| \pi_1 [  \tfrac{\partial F}{\partial  \epsilon} (x) -\Gamma ]   |  \leq \hat{f}_{\epsilon,1}$, then it follows from the explicit expression for $A_{0,1}^{-1}$ that
	 $[ (1+\tfrac{4}{\pi^2})^{1/2}\hat{f}_{\epsilon,1} ,\tfrac{2}{\pi} \hat{f}_{\epsilon,1} ]^T$ 
	is an upper bound on 
	 $ \pi_{\alpha,\omega} A_0^{-1}  ( \tfrac{\partial F}{\partial  \epsilon} (x) -\Gamma )$.
	Let us write $ c = \bce +h_c$ for some $ h_c \in \ell^1_0$ with $ \|h_c\| \leq r_c$.  Recalling  \eqref{e:defGamma2}, we obtain
	\begin{alignat*}{1}
	\pi_1[  \tfrac{\partial F}{\partial  \epsilon} (x) -\Gamma ] &=   \pi_1 \bigl[ \alpha L_\omega c + \alpha [ U_\omega c ] * c - \pp L_{\omega_0} \bce  \bigr]  \\
	&= \pi_1 \bigl[ \alpha \sigma^{-}( e^{i \omega} + e^{-2 i \omega} ) \bce- \pp \sigma^{-}(i-1) \bce \bigr] 
%	\\ & \qquad 
	+ \pi_1  \bigl[ \alpha \sigma^{-}( e^{i \omega} + e^{-2 i \omega} )h_c + \alpha [ U_\omega c] * c  \bigr] \\
	& = \pi_1 \bigl[ (\alpha - \pp) (i-1) \bce  + \alpha ( e^{i \omega} -i + e^{-2 i \omega} +1)\bce \bigr] 
	\\ & \hspace*{6.55cm}
	+ \pi_1  \bigl[ \alpha \sigma^{-}( e^{i \omega} + e^{-2 i \omega} )h_c + \alpha [ U_\omega c] * c  \bigr] .
	\end{alignat*}
We note that
\[
  \pi_1 ( [ U_\omega c] * c ) =  \pi_1 ([ U_\omega (\bce+h_c)] * (\bce+h_c) )
  =  \pi_1 ( [ U_\omega \bce] * h_c + [ U_\omega h_c] * \bce +
   [ U_\omega h_c] * h_c ).
 \]
Hence, using Lemma~\ref{lem:deltatheta} we obtain the estimate 
% \note[JB]{Better to replace $\alpha$ by $\pp+\da$? Due to factor 2 in norm, term $2\alpha r_c$ should be $\alpha r_c$? And term $\alpha r_c ( \tfrac{2}{\sqrt{5}}  \epsilon + r_c)$ should be $\alpha r_c ( \tfrac{2}{\sqrt{5}}  \epsilon + \tfrac{1}{2} r_c)$?}
% 	\begin{equation*}
% 	\bigl| \pi_1 [  \tfrac{\partial F}{\partial  \epsilon} (x) -\Gamma ] \bigr|
% 	% &\leq&  | (\alpha - \pp) (i-1) \bce | + \alpha |( e^{i \omega} -i + e^{-2 i \omega} +1)\bce| +  2 \alpha r_c + \alpha r_c ( 2 |\bce| + r_c) \nonumber  \\
% 	 \leq \da \tfrac{\sqrt{2}}{\sqrt{5}} \epsilon + \tfrac{\alpha }{\sqrt{5}}\epsilon ( \dt + \dtt ) + 2 \alpha r_c + \alpha r_c ( \tfrac{2}{\sqrt{5}}  \epsilon + r_c)  .
% 	\end{equation*}
%
% 	\note[J]{Proposed Change}
		\begin{equation*}
		\bigl| \pi_1 [  \tfrac{\partial F}{\partial  \epsilon} (x) -\Gamma ] \bigr|
		 \leq \tfrac{1}{2} \dc^0 \left(   \sqrt{2} \da  +3  \dw ( \pp + \da ) \right)  +  r_c  ( \pp + \da) 
		 \left(1 + \dc^0  + \tfrac{1}{2} r_c \right) .
		\end{equation*}
We thus find that
	% Thereby by defining
	% \[
	% \hat{f}_{\epsilon,1} := \frac{\epsilon}{\sqrt{5}} [  \da \sqrt{2} + \alpha ( \dt + \dtt ) ] +
	% \alpha r_c [2  +  \tfrac{2}{\sqrt{5}} \epsilon + r_c] .
	% \]
%	it follows that 
	$ | \pi_1 [  \tfrac{\partial F}{\partial  \epsilon} (x) -\Gamma ]|   \leq \hat{f}_{\epsilon,1} $, with $\hat{f}_{\epsilon,1}$ defined in~\eqref{e:feps1}.
	
	The $c$-component of  $A_0^{-1}  ( \tfrac{\partial F}{\partial  \epsilon} (x) -\Gamma )$ is given by $ A_{0,*}^{-1}  \pi_{\geq 2} [  \tfrac{\partial F}{\partial  \epsilon} (x) -\Gamma ]$.
	We will use the estimate $ \| A_{0,*}^{-1}\| \leq \frac{2}{\pi \sqrt{5}}$, so that it remains to determine a bound on $ \| \pi_{\geq 2} [  \tfrac{\partial F}{\partial  \epsilon} (x) -\Gamma ]\|$.  
Using~\eqref{e:defGamma2} we compute
	\begin{equation*}
	\pi_{\geq 2} [  \tfrac{\partial F}{\partial  \epsilon} (x) -\Gamma ] =  \alpha e^{-i \omega} \e_2  +  \pp i \e_2 + 
	\pi_{\geq 2} \bigl( \alpha L_{\omega} \bce - \pp L_{\omega_0} \bce + \alpha L_{\omega} h_c + \alpha [U_\omega c] * c  \bigr).
	\end{equation*}
We split the right hand side into three parts, which we estimate separately. First
% 	\[
% \left\| \pi_2 \bigl[ \alpha L_{\omega} h_c + \alpha [U_\omega c] * c  \bigr] \right\| \leq  \alpha ( 4 r_c + \dc^2).
% 	\]
% 	\note[J]{Proposed Change}
		\[
		\left\| \pi_2 \bigl[ \alpha L_{\omega} h_c + \alpha [U_\omega c] * c  \bigr] \right\| \leq  (\pp+ \da)  ( 4 r_c + \dc^2).
		\]	
Next, we calculate  
	\begin{alignat*}{1}
\pi_{\geq 2}	\left[\alpha L_{\omega} \bce - \pp L_{\omega_0} \bce \right] &= \alpha  \sigma^+ (e^{-i \omega} +e^{-2 i \omega}) \bce - \pp \sigma^+ (-i-1) \bce \\
	&= \left[ (\alpha - \pp) ( -i-1)\tfrac{2-i}{5}\epsilon  + \alpha ( e^{-i\omega} +e^{-2 i \omega} -(i+1)) \tfrac{2-i}{5}\epsilon \right] \e_3 ,
\end{alignat*}
hence 
%\note[JB]{Probably factor 2 missing.}
% \begin{equation*}
% \left\|	\pi_{\geq 2}	\left[\alpha L_{\omega} \bce - \pp L_{\omega_0} \bce \right] \right\| \leq  \frac{\epsilon}{\sqrt{5}} [ \sqrt{2} \da + \alpha ( \dt + \dtt)] .
% 	\end{equation*}
% 	\note[J]{Proposed Change}
	\begin{equation*}
	\left\|	\pi_{\geq 2}	\left[\alpha L_{\omega} \bce - \pp L_{\omega_0} \bce \right] \right\| \leq 
	\dc^0  [ \sqrt{2} \da + 3 \dw (\pp+\da) ] .
	\end{equation*}
Finally, we estimate 
% \note[JB]{Since $\|\e_2\|=2$ I think there is a factor 2 missing.}
% 	\begin{equation*}
% 	\left\| ( \alpha e^{-i\omega}  + \pp i) \e_2 \right\| =
% 	 \left| ( \alpha - \pp) e^{-i \omega} + \pp ( e^{-i \omega} +i) \right|
% 	\leq \da + \pp \dt.
% 	\end{equation*}
% 	\note[J]{Proposed Change}
	\begin{equation*}
	\left\| ( \alpha e^{-i\omega}  + \pp i) \e_2 \right\| = 
	2 \left| ( \alpha - \pp) e^{-i \omega} + \pp ( e^{-i \omega} +i) \right|
	\leq 2 \left( \da + \pp \dw \right).
	\end{equation*}
Collecting all estimates, we thus find that
	$ \| \pi_c A_{0}^{-1}  [  \tfrac{\partial F}{\partial  \epsilon} (x) -\Gamma ] \|   \leq \hat{f}_{\epsilon,c} $, with $\hat{f}_{\epsilon,c}$ defined in~\eqref{e:fepsc}.
	%
	%
	%
	% Thereby, by defining
	% \[
	% \hat{f}_{\epsilon,*} = \frac{2}{\pi \sqrt{5}} \left[
	% \da + \pp \dt +
	% \frac{\epsilon}{\sqrt{5}} [ \sqrt{2} \da + \alpha ( \dt + \dtt)]
	% +\alpha ( 4 r_c + \dc^2)
	% \right]
	% \]
	% it follows that
	% $ \left| A_{0,*}^{-1} \left[  \tfrac{\partial F}{\partial  \epsilon} (x) -\Gamma \right]_{k \geq 2} \right| <  \hat{f}_{\epsilon,*}$.
	%
\end{proof}

Recall that $\II$ is used to denote the $ 3 \times 3 $ identity matrix.
% \note[JB]{We may want to introduce this notation in section 4.2 as well.} \note[J]{I've added this notation inside the proof of Theorem 4.7.}
\begin{corollary}
	\label{cor:QUpperBound}
Let $\overline{A_0^{-1} A_1} $ be defined in Proposition~\ref{prop:A0A1}.	
The vector 
	% \[
	% (\II+\epsilon \overline{A_0^{-1} A_1} ) \cdot  [(1+\tfrac{2}{\pi})\hat{f}_{\epsilon,1},\tfrac{2}{\pi} \hat{f}_{\epsilon,1}, \hat{f}_{\epsilon,c}]^{T}
	% \]
	% \note[J]{Proposed Change}
	\[
	(\II+\epsilon \overline{A_0^{-1} A_1} ) \cdot  [(1+\tfrac{4}{\pi^2})^{1/2}\hat{f}_{\epsilon,1},\tfrac{2}{\pi} \hat{f}_{\epsilon,1}, \hat{f}_{\epsilon,c}]^{T}
	\]
	is an upper bound on $ A^\dagger ( \tfrac{\partial F}{\partial  \epsilon} (x) -\Gamma ) $ for any $x \in B_\epsilon(r,\rho)$.
\end{corollary}
\begin{sloppypar}
\begin{proof}
	From Lemma \ref{lem:ImplicitLast} it follows that 
% \change[J]{	$
% 	[(1+\tfrac{2}{\pi})\hat{f}_{\epsilon,1},\tfrac{2}{\pi} \hat{f}_{\epsilon,1}, \hat{f}_{\epsilon,c}]^T
% $
% }{
$
[(1+\tfrac{4}{\pi^2})^{1/2}\hat{f}_{\epsilon,1},\tfrac{2}{\pi} \hat{f}_{\epsilon,1}, \hat{f}_{\epsilon,c}]^T
$
	is an upper bound on $A_0^{-1}  ( \tfrac{\partial F}{\partial  \epsilon} (x) -\Gamma )$. 
Since 
	$A^\dagger = (I-\epsilon A_0^{-1}A_1) A_0^{-1}$ and 
	$\II+\epsilon \overline{A_0^{-1} A_1}$ is an upper bound on $I-\epsilon A_0^{-1}A_1$, the result follows from  Lemma~\ref{lem:ImplicitLast}. 
\end{proof}
\end{sloppypar}


We combine Lemmas~\ref{lem:ImplicitApprox} and~\ref{lem:ImplicitLast}
into an upper bound on $A^{\dagger} \frac{\partial F}{\partial  \epsilon}(\hat{x}_\epsilon)$.

\begin{lemma}\label{lem:Qeps}
% \note[JB]{Introduced $\QQ_\epsilon^0$; need to check that third component is correct!}
% \note[J]{I think the $\frac{9}{5\sqrt{50}}  \epsilon^2 $ term should be $ \frac{18}{5\sqrt{50}}  \epsilon^2$ . Also, inserted the $(1+\tfrac{4}{\pi^2})^{1/2}$ term.}
%
	Define $\QQ_\epsilon^0 , \QQ_\epsilon \in \R_+^3$ as follows:
% 	\begin{alignat}{1}
% 		\QQ_\epsilon^0 &:=
% \left[ \frac{2}{5}\left(\frac{3\pi}{2}-1 \right)  \epsilon,
%  \frac{2}{5} \epsilon ,
%  \frac{2}{\sqrt{5}} + \frac{2}{\sqrt{10}}\epsilon  +
%  \frac{9}{5\sqrt{50}}  \epsilon^2
%   \right]^T , \nonumber \\
% 	\QQ_\epsilon &:= \QQ_\epsilon^0
% 	+
% 	(\II+\epsilon \overline{A_0^{-1} A_1} ) \cdot  \bigl[(1+\tfrac{2}{\pi})\hat{f}_{\epsilon,1},\tfrac{2}{\pi} \hat{f}_{\epsilon,1}, \hat{f}_{\epsilon,c}\bigr]^{T} . \label{e:defQeps}
% \end{alignat}
% \note[J]{Proposed Change}
\begin{alignat}{1}
\QQ_\epsilon^0 &:= 
\left[ \frac{2}{5}\left(\frac{3\pi}{2}-1 \right)  \epsilon,
\frac{2}{5} \epsilon , 
\frac{2}{\sqrt{5}} + \frac{2}{\sqrt{10}}\epsilon  +
\frac{18}{5\sqrt{50}}  \epsilon^2
\right]^T , \nonumber \\
\QQ_\epsilon &:= \QQ_\epsilon^0  + 
(\II+\epsilon \overline{A_0^{-1} A_1} ) \cdot  \bigl[(1+\tfrac{4}{\pi^2})^{1/2}\hat{f}_{\epsilon,1},\tfrac{2}{\pi} \hat{f}_{\epsilon,1}, \hat{f}_{\epsilon,c}\bigr]^{T} . \label{e:defQeps}
\end{alignat}
Then the vector $\QQ_\epsilon \in \R^3_+$ is an upper bound  on $A^{\dagger} \frac{\partial F}{\partial  \epsilon}(x)$ 
for any  $x \in B_\epsilon(r,\rho)$.
\end{lemma}
\begin{proof}
It follows from Lemma \ref{lem:ImplicitApprox} that the vector 
$\QQ_\epsilon^0$
is an upper bound on $A^{\dagger} \Gamma$
(for example, the third component of $\QQ_\epsilon^0$ is a bound on $\|c'\|$).
It follows from Corollary~\ref{cor:QUpperBound} that 
% \change[J]{$	(I+\epsilon \overline{A_0^{-1} A_1} ) \cdot  [(1+\tfrac{2}{\pi})\hat{f}_{\epsilon,1},\tfrac{2}{\pi} \hat{f}_{\epsilon,1}, \hat{f}_{\epsilon,c}]^{T}$
% 	}{
%\note[JB]{displayed; formula was just too big}
\[
	(\II+\epsilon \overline{A_0^{-1} A_1} ) \cdot  [(1+\tfrac{4}{\pi^2})^{1/2}\hat{f}_{\epsilon,1},\tfrac{2}{\pi} \hat{f}_{\epsilon,1}, \hat{f}_{\epsilon,c}]^{T}
\]
 is an upper bound on $ A^\dagger ( \tfrac{\partial F}{\partial  \epsilon} (x) -\Gamma ) $.
We conclude from the triangle inequality that $\QQ_\epsilon $ is an upper bound on $A^\dagger  \tfrac{\partial F}{\partial  \epsilon} (x)$. 
\end{proof}
 
Finally, we prove the bounds needed to control the derivative $\frac{d}{d\epsilon} \hat{\alpha}_\epsilon$ in Section~\ref{s:Jones}
(in particular the implicit differentiation argument in Theorem~\ref{thm:NoFold}).

% \marginpar{remove newpage when done}
% \newpage

%%%%%%%%%%%%%%% Final Lemma %%%%%%%%%%%%%%

\begin{lemma}
		\label{lem:Meps}
		% \remove[J]{		Fix $\epsilon>0$, $r \in \R^3_+$, $\rho >0$ and
		% 	assume $x \in B_\epsilon(r,\rho)$. }
		% \add[J]{
		Fix $ \epsilon_0 > 0 , \rr \in \R^3_+ $ and $\rho >0$ as in the hypothesis of Proposition~\ref{prop:TightEstimate}. 
Let $ 0 < \epsilon \leq \epsilon_0$ and let $ \hat{x}_\epsilon \in B_\epsilon(\epsilon^2  \rr,\rho)$ denote the unique solution to $F(x) = 0$. 
Recall the definitions of  
$\ZZ_\epsilon \in  \emph{Mat}(\R_+^3 , \R_+^3)$
and $\QQ_\epsilon \in  \R_+^3$ in Equations~\eqref{e:defZeps} and~\eqref{e:defQeps}.  
Define 
		\begin{alignat*}{1}
		M_\epsilon &:= \frac{1}{\epsilon^2} \left(	(\II+\epsilon \overline{A_0^{-1} A_1} ) \cdot  \bigl[(1+\tfrac{4}{\pi^2})^{1/2}\hat{f}_{\epsilon,1},\tfrac{2}{\pi} \hat{f}_{\epsilon,1}, \hat{f}_{\epsilon,c} \bigr]^{T} \right)_1  ,\\
		M'_\epsilon  &:=  \frac{1}{ \epsilon^2 } \bigl( \ZZ_\epsilon (\II-\ZZ_\epsilon)^{-1} \QQ_\epsilon \bigr)_1  ,
		\end{alignat*}
where the subscript denotes the first component of the vector.
Then $M_\epsilon$ and $M'_\epsilon$ are positive, increasing in $\epsilon$, and satisfy the inequalities
 \begin{alignat}{1}
 \left| \pi_\alpha A^{\dagger} \left( \tfrac{\partial F}{\partial  \epsilon}(\hat{x}_\epsilon) - \Gamma_\epsilon \right)  \right|  &\leq\epsilon^2 M_\epsilon , \label{eq:Mepsilon}\\
 %
 \left( \ZZ_\epsilon (\II-\ZZ_\epsilon)^{-1} \QQ_\epsilon \right)_1 &\leq \epsilon^2 M'_\epsilon . \label{eq:MMepsilon}
 \end{alignat}
\end{lemma}

\begin{proof}
To first show that $(\II - \ZZ_{\epsilon })^{-1}$ is well defined,  we note that by Proposition~\ref{prop:TightEstimate} 
the radii polynomials $ P(\epsilon,\epsilon^2 \rr,\rho)$ are all negative. As was shown in the proof of Theorem~\ref{thm:RadPoly},
the operator norm of $ \ZZ_\epsilon$ on $ \R^3$ equipped with the norm $ \| \cdot \|_{\epsilon^2 \rr}$ is given by some $\kappa <1$, whereby the Neumann series of $ (\II - \ZZ_\epsilon)^{-1}$ converges. 
	
From the definition of $M_\epsilon$ and Corollary~\ref{cor:QUpperBound}, inequality \eqref{eq:Mepsilon} follows. Inequality \eqref{eq:MMepsilon} is a direct consequence of the definition of $M'_\epsilon$.
	Since the functions $\hat{f}_{\epsilon,1}$ and $\hat{f}_{\epsilon,c}$ are positive, then $M_\epsilon$ and $\QQ_\epsilon$ are positive. 
	Since the matrix $ \ZZ_\epsilon$ has positive entries only,  the Neumann series for $(\II-\ZZ_\epsilon)^{-1}$  has summands with exclusively positive entries, whereby $M'_\epsilon$ is positive. 

	Next we show that the components of $\ZZ_\epsilon$ and $\QQ_\epsilon - \QQ_\epsilon^0$ are polynomials in $\epsilon$ with positive coefficients and their lowest degree terms are at least quadratic. 
	To do so, it suffices to prove as much for the functions $ \hat{f}_{\epsilon,1},\hat{f}_{\epsilon,c},f_{1,\alpha},f_{1,\omega}, f_{1,c} , f_{*,\alpha}, f_{*,\omega},f_{*,c}$. 
	We note that all of these functions are given as polynomials with positive coefficients in the variables $ \epsilon, \da,\dw,\dc,r_c,\dc^0$ 
	(recall that $\rho $ is fixed and does not vary with $\epsilon$).
	Since $(r_\alpha , r_\omega,r_c) = \epsilon^2 (\rr_\alpha , \rr_\omega,\rr_c)$, then by Definition~\ref{def:DeltaDef} the terms $\da,\dw,r_c$ are all $ \cO(\epsilon^2)$. 
	Furthermore, whenever any of the terms  $ \epsilon, \dc, \dc^0$ 
appears, it is multiplied by another term of order at least $\cO(\epsilon)$.
	It  follows that every component of 
	 $ \ZZ_\epsilon$ and $ \QQ_{\epsilon} - \QQ_{\epsilon}^0$ is a polynomial in $ \epsilon$ with positive coefficients for which the lowest degree term is at least quadratic. 

% 	\remove[J]{
% 	We now show that $M_\epsilon$ and $ M'_\epsilon$ are increasing in $\epsilon$.
% First note that Proposition~3.16 shows that the solution  satisfies $\hat{x}_\epsilon \in B_\epsilon(\epsilon^2 \rr, \rho)$.
% Thereby all of the variables introduced in Definition B.1 are polynomials with nonnegative coefficients in $\epsilon$.
% Moreover, all of the variables which begin with  ``$\Delta$'' have a lowest order term of $\epsilon^2$, and only $\dc$ has its lowest order term being $\epsilon^1$.
% It is thus straightforward to see that the functions $\hat{f}_{\epsilon,1}$ and $\hat{f}_{\epsilon,c}$, as well as all of the component functions of $\ZZ_{\epsilon}$, are all polynomials in $\epsilon$ with positive coefficients and their lowest order terms are at least quadratic. }

From these considerations it follows that the components of both 
$M_\epsilon = \epsilon^{-2}(\QQ_{\epsilon} - \QQ_{\epsilon}^0)_1$ and $ \epsilon^{-2}\ZZ_\epsilon$ are polynomials in $\epsilon$ with positive coefficients. 
It also follows that both $\QQ_\epsilon$ and $(\II-\ZZ_\epsilon)^{-1}$ are increasing in $\epsilon$, whereby $M'_\epsilon$ is increasing in $\epsilon$.
\end{proof}





