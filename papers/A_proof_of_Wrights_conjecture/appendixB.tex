%!TEX root = hopfwright.tex

%%%%%%%%%%%%%%%%%%
%%% Appendix B %%%
%%%%%%%%%%%%%%%%%%

\section{Appendix: Endomorphism on a Compact Domain}
\label{sec:CompactDomain}



In order to construct the Newton-like map $T$ we defined operators $ A =  DF(\bar{x}_\epsilon) + \cO(\epsilon^2)$ and $A^{\dagger} = A^{-1} + \cO(\epsilon^2)$. 
However, as $(\bar{\alpha}_\epsilon,\bar{\omega}_\epsilon,\bar{c}_\epsilon) = (\pp,\pp,\bar{c}_\epsilon) + \cO(\epsilon^2)$,  the map $A$ can be better thought of as an $\cO(\epsilon^2)$ approximation of $DF(\pp,\pp,\bar{c}_\epsilon)$. 
Thus, when working with the map $T$ and considering points $ x \in  B_\epsilon(r,\rho)$ in its domain, we will often have to measure the distances of $ \alpha$ and $ \omega $ from $ \pp$. 
To that end, we define the following variables which will be used throughout the rest of the appendices. 
\begin{definition}
	\label{def:DeltaDef}
For $ \epsilon \geq 0$, and $r_\alpha,r_\omega,r_c >0$ we define 
\begin{alignat*}{2}
	\da^0 	&:= \tfrac{\epsilon^2}{5} ( 3 \pp -1) & \qquad\qquad
	\da 	&:= \da^0 + r_\alpha \\
	\dw^0 &:=  \tfrac{\epsilon^2}{5} &
	\dw &:=  \dw^0 + r_{\omega} \\ 
	\dc^0 &:=  \tfrac{2 \epsilon}{\sqrt{5}} &
	\dc &:=  \dc^0 + r_c . 
	% \\
	% \dt^0  &:= \dw^0 + \tfrac{1}{2} (\dw^0)^2 &
	% \dt  &:= \dw + \tfrac{1}{2} \dw^2 \\
	% \dtt^0  &:= 2 \dw^0 + \tfrac{1}{2} (2\dw^0)^2 &
	% \dtt  &:= 2 \dw + \tfrac{1}{2} (2\dw)^2  .
\end{alignat*}
\end{definition}


% \note[J]{
% 	I believe that we can replace the bounds $\dt$ by $\dw$  and $\dtt$ by $2 \dw$.In short, this follows from the following estimate.
% 	\[
% 	| e^{-i \omega }+i| \leq \int_{\pp}^\omega |\tfrac{\partial}{\partial \omega}  e^{-i \omega} | d\omega \leq  \int_{\pp}^\omega |1| d\omega = |\omega - \pp| .
% 	\]
% 	I have not gone through and done this yet. }
% \note[JB]{I think you are right. I think it also follows from $|e^{-i(\pp+\dw)}+i|^2=|e^{-i\dw}-1|^2 = (\cos \dw -1)^2+(\sin \dw)^2=2(1-\cos\dw) \leq 2 \cdot \frac{1}{2} \dw^2$.}
%
When considering an element $ ( \alpha , \omega, c)$ for our $\cO(\epsilon^2)$ analysis, we are often concerned with the 
 distances $|\alpha - \pp|$, $|\omega - \pp|$ and $ \| c - \bar{c}_\epsilon\|$, each of which is of order $\epsilon^2$.  
To create some  notational consistency in these definitions, $\da^0$ and $\dw^0$ are of order $\epsilon^2$, whereas $\dc^0$ is not capitalized as it is of order $\epsilon$. 
Using these definitions, it follows that for any $\rho>0$ and all  $(\alpha, \omega, c ) \in B_\epsilon(r,\rho)$ we have: 
\begin{alignat*}{1}
| \alpha - \pp | & \leq  \da       \\ 
	 | \omega - \pp| & \leq  \dw   \\
	\|c \| &\leq  \dc  .
	%  \\
	% | e^{- i \omega} + i| &\leq  \dt \\
	% | e^{-2 i \omega } +1| &\leq \dtt  .
\end{alignat*}
In this interpretation the superscript $0$ simply refers to $r=0$, i.e., the center of the ball $(\alpha,\omega,c) = \bx_\epsilon$.

The following elementary lemma will be used frequently in the estimates. 
\begin{lemma}\label{lem:deltatheta}
For all $x\in \R$ we have $|e^{ix}-1| \leq |x|$.
Furthermore, for all $|\omega - \bomega_\epsilon  | \leq r_\omega$  
%\note[JB]{I think this should be $|\omega - \bomega_\epsilon| \leq r_\omega$, no?} \note[J]{Yes, that is correct } 
we have 
$ |e^{- i \omega} + i| \leq  \dw$ and
$ | e^{-2 i \omega } +1| \leq 2 \dw $ .
\end{lemma}
\begin{proof}
We start with
\[
  |e^{ix}-1|^2 = (\cos x -1)^2+(\sin x)^2=2(1-\cos x) \leq 2 \cdot \tfrac{1}{2} x^2 = x^2.
\]
% Let $w = \omega - \pp$. Then $|w| \leq \dw$ and, using the previous inequality,
% \[
% | e^{- i \omega} + i|^2=
% |e^{-i(\pp+w)}+i|^2=|e^{-i w}-1|^2 \leq  w^2 =  \dw^2.
% \]
% \note[J]{To avoid using $w$ and $\omega$ in the same line, I propose we switch $ w \mapsto \theta$, as below. Also the last equality should be an inequality.}

Let $\theta = \omega - \pp$. Then $|\theta| \leq \dw$ and, using the previous inequality,
\[
| e^{- i \omega} + i|^2=
|e^{-i(\pp+\theta)}+i|^2=|e^{-i\theta}-1|^2 \leq  \theta^2 \leq  \dw^2.
\]
The final asserted inequality follows from an analogous argument.
\end{proof}


While the operators $U_\omega$ and $L_\omega$ are not continuous in $ \omega$ on all of $ \ell^1_0$, they are within the compact set $ B_\epsilon(r,\rho)$. 
To denote the derivative of these operators, we  define
\begin{alignat}{1}
	U_{\omega}' &:=  - i K^{-1} U_{\omega} \nonumber \\
	L_{\omega}' &:= - i \sigma^+( e^{- i \omega} I + K^{-1} U_{\omega}) + i \sigma^-(e^{i \omega} I - K^{-1} U_{\omega})  , \label{e:Lomegaprime}
\end{alignat}
and we derive Lipschitz bounds on $U_\omega$ and $L_\omega$ in the following proposition.
 
\begin{proposition}
	\label{prop:OmegaDerivatives}
	For the definitions above, $ \frac{\partial }{\partial  \omega} U_\omega = U_{\omega}' $ and $ \frac{\partial }{\partial  \omega}  L_\omega= L_{\omega}' $. 
	Furthermore,  for any $ (\alpha, \omega,c) \in B_\epsilon(r,\rho)$, we have the norm estimates
	\begin{alignat}{1}
	\| (U_{\omega} - U_{\omega_0} )c \| &\leq   \dw  \rho \nonumber  \\
	\|( L_{\omega} - L_{\omega_0} )c \| &\leq  2  \dw (  \dc +  \rho) .
	\label{e:LomegaLip}
	\end{alignat}
\end{proposition}
% \note[J]{ There was a mistake in the statement of this proposition. I changed the estimate $\| U_{\omega} - U_{\omega_0}  \| $ to $ \| (U_{\omega} - U_{\omega_0} )c \| $. Likewise for $ L_\omega$. }

\begin{proof}
One easily calculates that $ \frac{\partial U_\omega}{\partial  \omega} =  U_{\omega}'$,  whereby
$
	\| (U_{\omega} - U_{\omega_0} )c \| \leq \int_{\omega_0}^\omega \| \tfrac{\partial}{\partial \omega} U_\omega c \|  \leq    \dw  \rho  
$. 
Calculating $ \frac{\partial }{\partial  \omega}  L_{\omega} $, we obtain the following:
\begin{alignat*}{1}
 \frac{\partial }{\partial  \omega}  L_{\omega} 
&=  \frac{\partial }{\partial  \omega} \left[  \sigma^+( e^{- i \omega} I + U_{\omega}) + \sigma^-(e^{i \omega} I + U_{\omega}) \right] \\
&= - i \sigma^+( e^{- i \omega} I + K^{-1} U_{\omega}) + i \sigma^-(e^{i \omega} I - K^{-1} U_{\omega}) ,
\end{alignat*}
thus proving $ \frac{\partial L_\omega}{\partial  \omega} =  L_{\omega}'$,
and 
$\|( L_{\omega} - L_{\omega_0} )c \| \leq  \int_{\omega_0}^\omega \| \tfrac{\partial}{\partial \omega} L_\omega c \|  \leq   \dw ( 2  \dc + 2 \rho)$.
\end{proof}

\begin{proposition}
	Let $\epsilon\geq 0$ and  $r=(r_\alpha,r_\omega,r_c) \in \R^3_+$. 
	For any $ \rho > 0$ the map 
	 $T:B_{\epsilon}(r,\rho) \to \R^2 \times \ell^K_0 $ is well defined. 	
	We define functions 
% \note[J]{New definitions for $C_0$ and $C_1$ as the old ones did not quite match the estimates proven below. }
	\begin{alignat*}{1}
%	C_0 &:=  \frac{2 \epsilon^2}{\pi} 
%	\left[
%		\frac{8}{5},\frac{8}{5\sqrt{5}} \sqrt{\left(1-3 \pi /4 \right)^2+(2+\pi )^2},\frac{5 \pi }{2} 
%	\right]
%	\cdot \overline{A_0^{-1} A_1} \cdot [ \da , \dw , \dc ]^T ,
%%	\\
	% C_0 &:=  \frac{2 \epsilon^2}{\pi}
	% \left[
	% 	\frac{8}{5},\frac{2}{5} \sqrt{16+ 8\pi + 5 \pi^2},\frac{5 \pi }{2}
	% \right]
	% \cdot \overline{A_0^{-1} A_1} \cdot [ \da , \dw , \dc ]^T ,
	% \\
	C_0 &:=  \frac{2 \epsilon^2}{\pi} 
	\left[
	\frac{8}{5},\frac{2}{5} \sqrt{16+ 8\pi + 5 \pi^2},\frac{5 \pi }{2} 
	\right]
	\cdot \overline{A_0^{-1} A_1} \cdot [0,0 , \dc ]^T ,
	\\
%	C_1 &:= \frac{5 }{2 \pi} \left(1 +   \frac{4 \epsilon  }{5} \left(2+\sqrt{5}\right) \right) , \\
	% C_1 &:= \frac{5 }{2 \pi} + 2  \epsilon   \left(2+\sqrt{5}\right)  , \\
	C_1 &:= \frac{5 }{2 \pi} + \frac{\epsilon \sqrt{10}}{\pi}, \\
	C_2 &:= \dw  \left[  (1 + \pp) + \epsilon \pi  \right] , \\
	C_3 &:=  
	\da (2+ \dc) +	2 \dw (1+\pp) 
		+ \epsilon \left[ \pi + 2\da  + 4 \dc \da + \pi \dw \dc  + (\pp + \da ) \dc^2 \right] ,
	\end{alignat*}
where the expression for $C_0$ should be read as a product of a row vector, a $(3 \times 3)$ matrix and a column vector.
Furthermore we define, for any $\epsilon,r_\omega$ such that $C_1 C_2 <1$,
	\begin{equation}
		C(\epsilon,r_\alpha,r_\omega,r_c) := \frac{C_0+ C_1 C_3}{1 - C_1 C_2}
		 \, .
		\label{eq:RhoConstant}
	\end{equation}
	All of the functions $C_0,C_1,C_2,C_3$ and $C$ are nonnegative and monotonically increasing in their arguments $\epsilon$ and~$r$. 
	Furthermore, if  $C_1 C_2 < 1$ and $	C(\epsilon,r_\alpha,r_\omega,r_c) \leq \rho $
	then $\| K^{-1} \pi_c  T( x) \| \leq \rho $
	for $x \in B_{\epsilon}(r,\rho)$. 
	\label{prop:DerivativeEndo}
\end{proposition}

% \marginpar{This proposition is vague about the actual spaces being used, ie. $\ell_1,\ell^K_0$, etc.}

\begin{proof}
	Given their definitions, it is straightforward to check that the functions $C_i$ and $C$ are monotonically increasing in their arguments.  
	To prove the second half of the proposition, we split 
	$K^{-1} \pi_c  T(x)$ into several pieces. 
%\note[JB]{$\pi_c$ and $\pi_{\ge 2}$ added. Jonathan, could you please go through this and check?}
%\note[JB]{This did not work, since we do not have that $x$ is bounded by $[\da,\dw,\dc]^T$. Jonathan: what you probably meant was what I introduce as $\pi_c^0 x$, but could you please check?} 
	We define the projection $\pi_c^0 x = (0,0,\pi_c x)$.
We then obtain
	\begin{alignat*}{1}
	K^{-1} \pi_c  T(x)  &= K^{-1} \pi_c   [ x - A^{\dagger} F(x) ]   \\
	&= K^{-1} \pi_c  [ I \pi_c^0 x -    A^{\dagger} ( A \pi_c^0 x + F(x) - A \pi_c^0 x)]  \\
	&= \epsilon^2 K^{-1} \pi_c (A_0^{-1}A_{1})^2 \pi_c^0 x + K^{-1} \pi_c A^{\dagger} (F(x) - A \pi_c^0 x) \nonumber \\
	&=  \frac{2 \epsilon^2}{i\pi} \hat{U} \pi_{\ge 2} A_1 A_0^{-1}A_{1} \pi_c^0 x +\frac{2 }{i\pi} \hat{U}  \pi_{\ge 2} (I-\epsilon A_1 A_0^{-1}) (F(x) - A\pi_c^0 x)  ,
\end{alignat*}
where we have used that $K^{-1} \pi_c A_0^{-1} = \frac{2}{i\pi} \hat{U} \pi_{\ge 2}$, with the projection $\pi_{\ge 2}$ defined in~\eqref{e:pige2}.
By using $\| \hat{U} \| \leq \frac{5}{4}$, see Proposition~\ref{p:severalnorms}, we obtain the estimate
%\note[J]{Changed $[\da,\dw,\dc ]^T$ to $[0,0,\dc ]^T$}
\begin{equation}
	\| K^{-1} \pi_c T(x) \| \leq   \frac{2 \epsilon^2}{\pi} \overline{\hat{U}\pi_{\ge 2} A_1} \cdot  \overline{ A_0^{-1}A_{1}}  \cdot
	[0,0,\dc ]^T +\frac{5 }{2 \pi} \left(1 + \epsilon \| A_1 A_0^{-1} \| \right) \|F(x) - A\pi_c^0 x \| .
	\label{eq:DerivativeEndo}
\end{equation}
Here the $(1 \times 3)$ row vector $\overline{\hat{U}\pi_{\ge 2} A_1}$ is an upper bound on $\hat{U}\pi_{\ge 2} A_1$ interpreted as a linear operator from $\R^2 \times \ell^1_0$ to $\ell^1_0$, thus extending in a straightforward manner the definition of upper bounds given in  Definition~\ref{def:upperbound}.
	
	
	We have already calculated  an expression for
	 $ \overline{ A_0^{-1}A_{1}}$ in Proposition~\ref{prop:A0A1},  and  $  \| A_1 A_0^{-1}\| =\frac{2\sqrt{10}}{5}$ by Proposition~\ref{prop:A1A0}.  In order to finish the calculation of the right hand side of Equation \eqref{eq:DerivativeEndo}, we need to  estimate  $\| F(x) - A\pi_c^0 x \|$ and $\overline{\hat{U} \pi_{\ge 2} A_1} $. 
	We first calculate a bound on $\hat{U} \pi_{\ge 2} A_1 $. 
	We note that $ \hat{U} \pi_{\ge 2} A_1  =  \hat{U} \e_2 ( i_\C A_{1,2} \pi_{\alpha,\omega})+ \hat{U} \pi_{\ge 2}A_{1,*} \pi_c$.	
As $\|\hat{U} e_2\| = \| \tfrac{4-2i}{5} \e_2\|$,
it follows from the definition of $A_{1,2}$ 
that 
\[
	 \left| i_\C  A_{1,2}
	 \left( \!\!\begin{array}{c}\alpha \\ \omega \end{array} \!\!\right) \right|  
	 \cdot \| \hat{U} \e_2 \| 
	 \leq 
	 \left(\frac{\sqrt{20}}{5} |\alpha| +  \frac{\sqrt{(2-3 \pi/2)^2 +4(2+\pi)^2}}{5} |\omega| \right)  \cdot \frac{4}{\sqrt{5}}.
\]
	To calculate $ \| \hat{U} \pi_{\ge 2} A_{1,*} \|$ we note that $ \| \hat{U}\| \leq \frac{5}{4}$ and $ \|A_{1,*}\| = \pp \| L_{\omega_0} \| \leq 2 \pi$. 
	Hence $ \| \hat{U} \pi_{\ge 2} A_{1,*} \| \leq \frac{5 \pi}{2}$. 
	Combining these results, we obtain  that
%\note[JB]{I think the second, rearranged version, of the root looks ``nicer''. }
%	\[
%	\overline{\hat{U} \pi_{\ge 2}  A_1 } = \left[\frac{8}{5},\frac{8}{5\sqrt{5}} \sqrt{\left(1-3 \pi /4 \right)^2+(2+\pi )^2},\frac{5 \pi }{2} \right].
%	\] 
	\[
	\overline{\hat{U} \pi_{\ge 2}  A_1 } = \left[\frac{8}{5},\frac{2}{5} \sqrt{16 + 8 \pi + 5 \pi^2},\frac{5 \pi }{2} \right].
	\] 
Thereby, it follows from~\eqref{eq:DerivativeEndo} that 
\begin{equation}\label{e:C0C1}
	\| K^{-1} \pi_c T(x) \| \leq C_0 + C_1 \| F(x) - A \pi_c^0 x\|. 
\end{equation}
We now calculate
	\begin{alignat*}{1}
	F(x) - A \pi_c^0 x &= 
	(i \omega + \alpha e^{-i \omega} ) \e_1 + 
	( i \omega K^{-1} + \alpha U_{\omega}) c + 
	\epsilon \alpha e^{-i \omega} \e_2  +
	\alpha \epsilon L_\omega c + 
	\alpha \epsilon [ U_{\omega} c] * c  
	\\ &\qquad 
	- \pp (i K^{-1} + U_{\omega_0} + \epsilon L_{\omega_0} ) c \\
	&= i ( \omega - \pp) K^{-1} c + ( \alpha - \pp) U_{\omega} c +  \pp ( U_{\omega} - U_{\omega_0})c  \nonumber \\
	&\qquad  + \left[i ( \omega - \pp ) + ( \alpha - \pp) e^{-i \omega} + \pp( e^{- i \omega }+ i)\right] \e_1  \nonumber
	\\ 
	&\qquad  +\epsilon  \alpha   e^{-i \omega}  \e_2  
+  ( \alpha- \pp)  \epsilon L_{\omega} c + \pp \epsilon ( L_{\omega} - L_{\omega_0}) c + \alpha \epsilon [ U_{\omega} c ] * c .
	\end{alignat*}
Taking norms and using~\eqref{e:LomegaLip} and Lemma~\ref{lem:deltatheta}, we obtain 
	\begin{alignat*}{1}
	\| F(x) - A \pi_c^0 x\|& \leq  
	 \dw \rho + \da \dc + \pp \dw \rho
    +	2 (\dw + \da + \pp \dw)  
	   \\
	&\qquad + \epsilon \left[ 2(\pp + \da ) + 4 \dc \da + \pi  \dw (  \dc + \rho) + (\pp + \da ) \dc^2 \right]  \\
		&= \dw [ (1+\pp) +   \epsilon \pi ] \rho \nonumber \\ 
	&\qquad +  \da (2 + \dc)
	+	2 \dw (1+\pp) 
	+ \epsilon \left[ \pi + 2\da  + 4 \dc \da + \pi \dw \dc  + (\pp + \da ) \dc^2 \right].  
	\end{alignat*}


	We have now computed all of the necessary constants. Thus $ \| F(x) - A \pi_c^0 x \| \leq C_2 \rho + C_3$, and from~\eqref{e:C0C1}   we obtain 
	\begin{eqnarray*}
	\| K^{-1} \pi_c T(c) \|
	&\leq & C_0 +  C_1 ( C_2  \rho + C_3),
	\end{eqnarray*}
with the constants defined in the statement of the proposition.
We would like to select values of $\rho$ for which 
	\[
	\| K^{-1} \pi_c T(c) \| \leq \rho
	\]
	This is true if  
	$	C_0 +  C_1 ( C_2  \rho + C_3) \leq \rho$, 
	or equivalently 
	\[
	\frac{C_0 + C_1 C_3 }{1 - C_1 C_2} \leq \rho.
	\]
	This proves the theorem.
\end{proof}

%%%%%%%%%%%%%%%%%%
%%% Appendix C %%%
%%%%%%%%%%%%%%%%%%

