\documentclass{article}
\usepackage{amsmath, geometry,amssymb,amscd, float, verbatim, latexsym,subfig,wrapfig}
\usepackage{algorithmic,algorithm,cite,setspace}
\usepackage[pdftex]{graphicx}
\usepackage[all]{xy}
\usepackage{lscape}
\usepackage{url}
\usepackage{amsthm}

\newtheorem{theorem}{Theorem}[section]
\newtheorem{lemma}[theorem]{Lemma}
\newtheorem{proposition}[theorem]{Proposition}
\newtheorem{corollary}[theorem]{Corollary}
\newtheorem{claim}[theorem]{Claim}
\newtheorem{conjecture}[theorem]{Conjecture}
\newtheorem{remark}[theorem]{Remark}
\newtheorem{definition}[theorem]{Definition}

\numberwithin{equation}{section}

% \newenvironment{definition}[1][Definition]{\begin{trivlist}
% \item[\hskip \labelsep {\bfseries #1}]}{\end{trivlist}}

\newcommand{\bi}{\text{\textup{bi}}}
\newcommand{\sym}{\text{\textup{sym}}}
\renewcommand{\c}{a}
\newcommand{\e}{\mathrm{e}}
\newcommand{\tc}{\tilde{c}}

\newcommand{\R}{\mathbb{R}}
\newcommand{\C}{\mathbb{C}}
\newcommand{\Q}{\mathbb{Q}}
\newcommand{\N}{\mathbb{N}}
\newcommand{\Z}{\mathbb{Z}}
\newcommand{\pp}{\tfrac{\pi}{2}}
\newcommand{\II}{I_3}

\newcommand{\B}{\widehat{B}}
\newcommand{\tF}{\widetilde{F}}
\newcommand{\epseps}{\epsilon_*}
\newcommand{\tS}{\widetilde{S}}

\newcommand{\da}{\Delta_\alpha}
\newcommand{\dw}{\Delta_\omega}
\newcommand{\dt}{\Delta_{\theta}}
\newcommand{\dtt}{\Delta_{2\theta}}
\newcommand{\dc}{\delta_c}

\newcommand{\bx}{\bar{x}}
\newcommand{\balpha}{\bar{\alpha}}
\newcommand{\bomega}{\bar{\omega}}
\newcommand{\bc}{\bar{c}}
\newcommand{\bce}{\bc_{\epsilon}}
\newcommand{\rr}{\check{r}}

\newcommand{\ZZ}{\mathcal{Z}}
\newcommand{\QQ}{\mathcal{Q}}

\newcommand{\LL}{\zeta}
\newcommand{\upperbound}[1]{\overline{#1}}
%\newcommand{\upperbound}[1]{\overline{\overline{#1}}}

\newcommand{\cA}{\mathcal{A}}
\newcommand{\cB}{\mathcal{B}}
\newcommand{\cD}{\mathcal{D}}
\newcommand{\cE}{\mathcal{E}}
\newcommand{\cG}{\mathcal{G}}
\newcommand{\cH}{\mathcal{H}}
\newcommand{\cL}{\mathcal{L}}
\newcommand{\cM}{\mathcal{M}}
\newcommand{\cN}{\mathcal{N}}
\newcommand{\cO}{\mathcal{O}}

\newcommand{\X}[1]{\mathbf{X_{#1}}}

%\usepackage[inline]{trackchanges}
%\addeditor{J}
%\addeditor{JB}

\title{A proof of Wright's conjecture}
\author{Jan Bouwe van den Berg\thanks{Partially supported by NWO VICI-grant 639.033.109} \thanks{Department of Mathematics, VU Amsterdam, de Boelelaan 1081, 1081 HV Amsterdam, The Netherlands, janbouwe@few.vu.nl}
\and
Jonathan Jaquette \thanks{Partially supported by NSF DMS 0915019,	NSF DMS 1248071} \thanks{Department of Mathematics, Rutgers, The State University Of New Jersey, 110 Frelinghuysen Rd., Piscataway, NJ 08854-8019, USA, jaquette@math.rutgers.edu}
}

\begin{document}

\maketitle



\begin{abstract}
Wright's conjecture states that the origin is the global attractor for the delay differential equation $y'(t) = - \alpha y(t-1) [ 1 + y(t) ] $  for all $\alpha \in (0,\pp]$. This has been proven to be true for a subset of parameter values $\alpha$. We extend the result to the full parameter range $\alpha \in (0,\pp]$, and thus prove Wright's conjecture to be true. Our approach relies on a careful investigation of the neighborhood of the Hopf bifurcation occurring at $\alpha = \pp$. This analysis fills the gap left by complementary work on Wright's conjecture, which covers parameter values further away from the bifurcation point. Furthermore, we show that the branch of (slowly oscillating) periodic orbits originating from this Hopf bifurcation does not have any subsequent bifurcations (and in particular no folds) for
$\alpha\in(\pp , \pp + 6.830 \times10^{-3}]$. 
When combined with other results, this proves that the branch of slowly oscillating solutions that originates from the Hopf bifurcation at $\alpha=\pp$ is globally parametrized by $\alpha > \pp$.
\end{abstract}



\begin{center}
	{\bf \small Keywords.} 
	{ \small Delay Differential Equation,
		Hopf Bifurcation, 
		Wright's Conjecture, \\
		Supercritical Bifurcation Branch,
		Newton-Kantorovich Theorem
		}
\end{center}

\section{Introduction}
\label{s:introduction}
% \leavevmode
% \\
% \\
% \\
% \\
% \\
\section{Introduction}
\label{introduction}

AutoML is the process by which machine learning models are built automatically for a new dataset. Given a dataset, AutoML systems perform a search over valid data transformations and learners, along with hyper-parameter optimization for each learner~\cite{VolcanoML}. Choosing the transformations and learners over which to search is our focus.
A significant number of systems mine from prior runs of pipelines over a set of datasets to choose transformers and learners that are effective with different types of datasets (e.g. \cite{NEURIPS2018_b59a51a3}, \cite{10.14778/3415478.3415542}, \cite{autosklearn}). Thus, they build a database by actually running different pipelines with a diverse set of datasets to estimate the accuracy of potential pipelines. Hence, they can be used to effectively reduce the search space. A new dataset, based on a set of features (meta-features) is then matched to this database to find the most plausible candidates for both learner selection and hyper-parameter tuning. This process of choosing starting points in the search space is called meta-learning for the cold start problem.  

Other meta-learning approaches include mining existing data science code and their associated datasets to learn from human expertise. The AL~\cite{al} system mined existing Kaggle notebooks using dynamic analysis, i.e., actually running the scripts, and showed that such a system has promise.  However, this meta-learning approach does not scale because it is onerous to execute a large number of pipeline scripts on datasets, preprocessing datasets is never trivial, and older scripts cease to run at all as software evolves. It is not surprising that AL therefore performed dynamic analysis on just nine datasets.

Our system, {\sysname}, provides a scalable meta-learning approach to leverage human expertise, using static analysis to mine pipelines from large repositories of scripts. Static analysis has the advantage of scaling to thousands or millions of scripts \cite{graph4code} easily, but lacks the performance data gathered by dynamic analysis. The {\sysname} meta-learning approach guides the learning process by a scalable dataset similarity search, based on dataset embeddings, to find the most similar datasets and the semantics of ML pipelines applied on them.  Many existing systems, such as Auto-Sklearn \cite{autosklearn} and AL \cite{al}, compute a set of meta-features for each dataset. We developed a deep neural network model to generate embeddings at the granularity of a dataset, e.g., a table or CSV file, to capture similarity at the level of an entire dataset rather than relying on a set of meta-features.
 
Because we use static analysis to capture the semantics of the meta-learning process, we have no mechanism to choose the \textbf{best} pipeline from many seen pipelines, unlike the dynamic execution case where one can rely on runtime to choose the best performing pipeline.  Observing that pipelines are basically workflow graphs, we use graph generator neural models to succinctly capture the statically-observed pipelines for a single dataset. In {\sysname}, we formulate learner selection as a graph generation problem to predict optimized pipelines based on pipelines seen in actual notebooks.

%. This formulation enables {\sysname} for effective pruning of the AutoML search space to predict optimized pipelines based on pipelines seen in actual notebooks.}
%We note that increasingly, state-of-the-art performance in AutoML systems is being generated by more complex pipelines such as Directed Acyclic Graphs (DAGs) \cite{piper} rather than the linear pipelines used in earlier systems.  
 
{\sysname} does learner and transformation selection, and hence is a component of an AutoML systems. To evaluate this component, we integrated it into two existing AutoML systems, FLAML \cite{flaml} and Auto-Sklearn \cite{autosklearn}.  
% We evaluate each system with and without {\sysname}.  
We chose FLAML because it does not yet have any meta-learning component for the cold start problem and instead allows user selection of learners and transformers. The authors of FLAML explicitly pointed to the fact that FLAML might benefit from a meta-learning component and pointed to it as a possibility for future work. For FLAML, if mining historical pipelines provides an advantage, we should improve its performance. We also picked Auto-Sklearn as it does have a learner selection component based on meta-features, as described earlier~\cite{autosklearn2}. For Auto-Sklearn, we should at least match performance if our static mining of pipelines can match their extensive database. For context, we also compared {\sysname} with the recent VolcanoML~\cite{VolcanoML}, which provides an efficient decomposition and execution strategy for the AutoML search space. In contrast, {\sysname} prunes the search space using our meta-learning model to perform hyperparameter optimization only for the most promising candidates. 

The contributions of this paper are the following:
\begin{itemize}
    \item Section ~\ref{sec:mining} defines a scalable meta-learning approach based on representation learning of mined ML pipeline semantics and datasets for over 100 datasets and ~11K Python scripts.  
    \newline
    \item Sections~\ref{sec:kgpipGen} formulates AutoML pipeline generation as a graph generation problem. {\sysname} predicts efficiently an optimized ML pipeline for an unseen dataset based on our meta-learning model.  To the best of our knowledge, {\sysname} is the first approach to formulate  AutoML pipeline generation in such a way.
    \newline
    \item Section~\ref{sec:eval} presents a comprehensive evaluation using a large collection of 121 datasets from major AutoML benchmarks and Kaggle. Our experimental results show that {\sysname} outperforms all existing AutoML systems and achieves state-of-the-art results on the majority of these datasets. {\sysname} significantly improves the performance of both FLAML and Auto-Sklearn in classification and regression tasks. We also outperformed AL in 75 out of 77 datasets and VolcanoML in 75  out of 121 datasets, including 44 datasets used only by VolcanoML~\cite{VolcanoML}.  On average, {\sysname} achieves scores that are statistically better than the means of all other systems. 
\end{itemize}


%This approach does not need to apply cleaning or transformation methods to handle different variances among datasets. Moreover, we do not need to deal with complex analysis, such as dynamic code analysis. Thus, our approach proved to be scalable, as discussed in Sections~\ref{sec:mining}.

\section{Preliminaries}
\label{s:preliminaries}
%!TEX root = hopfwright.tex
%

In this section we systematically recast the Hopf bifurcation problem in Fourier space. 
We introduce appropriate scalings, sequence spaces of Fourier coefficients and convenient operators on these spaces. 
To study Equation~\eqref{eq:FourierSequenceEquation} we consider Fourier sequences $ \{a_k\}$ and fix a Banach space in which these sequences reside. It is indispensable for our analysis that this space have an algebraic structure. 
The Wiener algebra of absolutely summable Fourier series is a natural candidate, which we use with minor modifications. 
In numerical applications, weighted sequence spaces with algebraic and geometric decay have been used to great effect to study periodic solutions which are $C^k$ and analytic, respectively~\cite{lessard2010recent,hungria2016rigorous}. 
Although it follows from Lemma~\ref{l:analytic} that the Fourier coefficients of any solution decay exponentially, we choose to work in a space of less regularity. 
The reason is that by working in a space with less regularity, we are better able to connect our results with the global estimates in \cite{neumaier2014global}, see Theorem~\ref{thm:UniqunessNbd2}.


%
%
%\begin{remark}
%	Although it follows from Lemma~\ref{l:analytic} that the Fourier coefficients of any solution decay exponentially, we choose to work in a space of less regularity, namely summable Fourier coefficients. This allows us to draw SOME MORE INTERESTING CONCLUSION LATER.
%	EXPLAIN WHY WE CHOOSE A NORM WITH ALMOST NO DECAY!
%	% of s Periodic solutions to Wright's equation are known to be real analytic and so their  Fourier coefficients must decay geometrically [Nussbaum].
%	% We do not use such a strong result;  any periodic solution must be continuously differentiable, by which it follows that $ \sum | c_k| < \infty$.
%\end{remark}


\begin{remark}\label{r:a0}
There is considerable redundancy in Equation~\eqref{eq:FourierSequenceEquation}. First, since we are considering real-valued solutions $y$, we assume $\c_{-k}$ is the complex conjugate of $\c_k$. This symmetry implies it suffices to consider Equation~\eqref{eq:FourierSequenceEquation} for $k \geq 0$.
Second, we may effectively ignore the zeroth Fourier coefficient of any periodic solution \cite{jones1962existence}, since it is necessarily equal to $0$. 
%In \cite{jones1962existence}, it is shown that if $y \not\equiv -1$ is a periodic solution of~\eqref{eq:Wright} with frequency $\omega$, then $ \int_0^{2\pi/\omega} y(t) dt =0$. 
		The self contained argument is as follows. 
		As mentioned in the introduction, any periodic solution to Wright's equation must satisfy $ y(t) > -1$ for all $t$. 
	By dividing Equation~\eqref{eq:Wright} by $(1+y(t))$, which never vanishes, we obtain
	\[
	\frac{d}{dt} \log (1 + y(t)) = - \alpha y(t-1).
	\]  
	Integrating over one period $L$ we derive the condition 
	$0=\int_0^L y(t) dt $.
	Hence $a_0=0$ for any periodic solution. 
	It will be shown in Theorem~\ref{thm:FourierEquivalence1} that a related argument implies that we do not need to consider Equation~\eqref{eq:FourierSequenceEquation} for $k=0$.
\end{remark}

%%%
%%%
%%%\begin{remark}\label{r:c0} 
%%%In \cite{jones1962existence}, it is shown that if $y \not\equiv -1$ is a periodic solution of~\eqref{eq:Wright} with frequency $\omega$, then $ \int_0^{2\pi/\omega} y(t) dt =0$. 
%%%PERHAPS TOO MUCH DETAIL HERE. The self contained argument is as follows.
%%%If $y \not\equiv -1$ then $y(t) \neq -1$ for all $t$, since if $y(t_0)=-1$ for some $t_0 \in \R$ then $y'(t_0)=0$ by~\eqref{eq:Wright} and in fact by differentiating~\eqref{eq:Wright} repeatedly one obtains that all derivatives of $y$ vanish at $t_0$. Hence $y \equiv -1$ by Lemma~\ref{l:analytic}, a contradiction. Now divide~\eqref{eq:Wright} by $(1+y(t))$, which never vanishes, to obtain
%%%\[
%%%  \frac{d}{dt} \log |1 + y(t)| = - \alpha y(t-1).
%%%\]  
%%%Integrating over one period we obtain $\int_0^L y(t) dt =0$.
%%%\end{remark}



%Furthermore, the condition that $y(t)$ is real forces $\c_{-k} = \overline{\c}_{k}$.  
%
We define the spaces of absolutely summable Fourier series
\begin{alignat*}{1}
	\ell^1 &:= \left\{ \{ \c_k \}_{k \geq 1} : 
    \sum_{k \geq 1} | \c_k| < \infty  \right\} , \\
	\ell^1_\bi &:= \left\{ \{ \c_k \}_{k \in \Z} : 
    \sum_{k \in \Z} | \c_k| < \infty  \right\} .
\end{alignat*} 
We identify any semi-infinite sequence $ \{ \c_k \}_{k \geq 1} \in \ell^1$ with the bi-infinite sequence $ \{ \c_k \}_{k \in \Z} \in \ell^1_\bi$ via the conventions (see Remark~\ref{r:a0})
\begin{equation}
  \c_0=0 \qquad\text{ and }\qquad \c_{-k} = \c_{k}^*. 
\end{equation}
In other word, we identify $\ell^1$ with the set
\begin{equation*}
   \ell^1_\sym := \left\{ \c \in \ell^1_\bi : 
	\c_0=0,~\c_{-k}=\c_k^* \right\} .
\end{equation*}
On $\ell^1$ we introduce the norm
\begin{equation}\label{e:lnorm}
  \| \c \| = \| \c \|_{\ell^1} := 2 \sum_{k = 1}^\infty |\c_k|.
\end{equation}
The factor $2$ in this norm is chosen to have a Banach algebra estimate.
Indeed, for $\c, \tilde{\c} \in \ell^1 \cong \ell^1_\sym$ we define
the discrete convolution 
\[
\left[ \c * \tilde{\c} \right]_k = \sum_{\substack{k_1,k_2\in\Z\\ k_1 + k_2 = k}} \c_{k_1} \tilde{\c}_{k_2} .
\]
Although $[\c*\tilde{\c}]_0$ does not necessarily vanish, we have $\{\c*\tilde{\c}\}_{k \geq 1} \in \ell^1 $ and 
\begin{equation*}
	\| \c*\tilde{\c} \| \leq \| \c \| \cdot  \| \tilde{\c} \| 
	\qquad\text{for all } \c , \tilde{\c} \in \ell^1, 
\end{equation*}
hence $\ell^1$ with norm~\eqref{e:lnorm} is a Banach algebra.

By Lemma~\ref{l:analytic} it is clear that any periodic solution of~\eqref{eq:Wright} has a well-defined Fourier series $\c \in \ell^1_\bi$. 
The next theorem shows that in order to study periodic orbits to Wright's equation we only need to study Equation~\eqref{eq:FourierSequenceEquation} 
for $k \geq 1$. For convenience we introduce the notation 
\[
G(\alpha,\omega,\c)_k=
( i \omega k + \alpha e^{ - i \omega k}) \c_k + \alpha \sum_{k_1 + k_2 = k} e^{- i \omega k_1} \c_{k_1} \c_{k_2} \qquad \text{for } k \in \N.
\]
We note that we may interpret the trivial solution $y(t)\equiv 0$ as a periodic solution of arbitrary period.
\begin{theorem}
\label{thm:FourierEquivalence1}
Let $\alpha>0$ and $\omega>0$.
If $\c \in \ell^1 \cong \ell^1_{\sym}$ solves
$G(\alpha,\omega,\c)_k =0$  for all $k \geq 1$,
then $y(t)$ given by~\eqref{eq:FourierEquation} is a periodic solution of~\eqref{eq:Wright} with period~$2\pi/\omega$.
Vice versa, if $y(t)$ is a periodic solution of~\eqref{eq:Wright} with period~$2\pi/\omega$ then its Fourier coefficients $\c \in \ell^1_\bi$ lie in $\ell^1_\sym \cong \ell^1$ and solve $G(\alpha,\omega,\c)_k =0$ for all $k \geq 1$.
\end{theorem}

\begin{proof}	
	If $y(t)$ is a periodic solution of~\eqref{eq:Wright} then it is real analytic by Lemma~\ref{l:analytic}, hence its Fourier series $\c$ is well-defined and $\c \in \ell^1_{\sym}$ by Remark~\ref{r:a0}.
	Plugging the Fourier series~\eqref{eq:FourierEquation} into~\eqref{eq:Wright} one easily derives that $\c$ solves~\eqref{eq:FourierSequenceEquation} for all $k \geq 1$.

To prove the reverse implication, assume that $\c \in \ell^1_\sym$ solves
Equation~\eqref{eq:FourierSequenceEquation} for all $k \geq 1$. Since $\c_{-k}
= \c_k^*$, Equation \eqref{eq:FourierSequenceEquation} is also satisfied for
all $k \leq -1$. It follows from the Banach algebra property and
\eqref{eq:FourierSequenceEquation} that $\{k \c_k\}_{k \in \Z} \in \ell^1_\bi$,
hence $y$, given by~\eqref{eq:FourierEquation}, is continuously differentiable.
% (and by bootstrapping one infers that $\{k^m c_k \} \in \ell^1_\bi$, 
% hence $y \in C^m$ for any $m \geq 1$).
	Since~\eqref{eq:FourierSequenceEquation} is satisfied for all $k \in \Z \setminus \{0\}$ (but not necessarily for $k=0$) one may perform the inverse Fourier transform on~\eqref{eq:FourierSequenceEquation} to conclude that
	$y$ satisfies the delay equation 
\begin{equation}\label{eq:delaywithK}
   	y'(t) = - \alpha y(t-1) [ 1 + y(t)] + C
\end{equation}
	for some constant $C \in \R$. 
   Finally, to prove that $C=0$ we argue by contradiction.
   Suppose $C \neq 0$. Then $y(t) \neq -1$ for all $t$.
   Namely, at any point where $y(t_0) =-1$ one would have $y'(t_0) = C$
   which has fixed sign,   hence it would follow that $y$ is not periodic
   ($y$ would not be able to cross $-1$ in the opposite direction, 
   preventing $y$  from being periodic).  
  We may thus divide~\eqref{eq:delaywithK} through by $1 + y(t)$ and obtain 
\begin{equation*}
	\frac{d}{dt} \log | 1 + y(t) | = - \alpha y(t-1) + \frac{C}{1+y(t)} .
\end{equation*}
	By integrating both sides of the equation over one period $L$ and by using that $\c_0=0$, we 
	obtain
	\[
	 C \int_0^L \frac{1}{1+y(t)} dt =0.
	\]
	Since the integrand is either strictly negative or strictly positive, this implies that $C=0$. Hence~\eqref{eq:delaywithK} reduces to~\eqref{eq:Wright},
	and $y$ satisfies Wright's equation. 
\end{proof}






To efficiently study Equation~\eqref{eq:FourierSequenceEquation}, we introduce the following linear operators on $ \ell^1$:
\begin{alignat*}{1}
   [K \c ]_k &:= k^{-1} \c_k  , \\ 
   [ U_\omega \c ]_k &:= e^{-i k \omega} \c_k  .
\end{alignat*}
The map $K$ is a compact operator, and it has a densely defined inverse $K^{-1}$. The domain of $K^{-1}$ is denoted by
\[
  \ell^K := \{ \c \in \ell^1 : K^{-1} \c \in \ell^1 \}.  
\]
The map $U_{\omega}$ is a unitary operator on $\ell^1$, but
it is discontinuous in $\omega$. 
With this notation, Theorem~\ref{thm:FourierEquivalence1} implies that our problem of finding a SOPS to~\eqref{eq:Wright} is equivalent to finding an $\c \in \ell^1$ such that
\begin{equation}
\label{e:defG}
  G(\alpha,\omega,\c) :=
  \left( i \omega K^{-1} + \alpha U_\omega \right) \c + \alpha \left[U_\omega \, \c \right] * \c  = 0.
\end{equation}


%In order for the solutions of Equation \ref{eq:FHat} to be isolated we need to impose a phase condition. 
%If there is a sequence $ \{ c_k \} $ which satisfies  Equation \ref{eq:FHat}, then $ y( t + \tau) = \sum_{k \in \Z} c_k e^{ i k \omega (t + \tau)}$ satisfies Wright's equation at parameter $\alpha$. 
%Fix $ \tau = - Arg[c_1] / \omega$ so that $ c_1  e^{ i \omega \tau} $ is a nonnegative real number. 
%By Proposition \ref{thm:FourierEquivalence1} it follows that $\{ c'_k \} =  \{c_k e^{ i \omega k \tau }   \}$ is a solution to Equation \ref{eq:FHat}, and furthermore that $ c'_1 = \epsilon$ for some $ \epsilon \geq 0$. 


Periodic solutions are invariant under time translation: if $y(t)$ solves Wright's equation, then so does $ y(t+\tau)$ for any $\tau \in \R$. 
We remove this degeneracy by adding a phase condition. 
Without loss of generality, if $\c \in \ell^1$ solves Equation~\eqref{e:defG}, we may assume that $\c_1 = \epsilon$ for some 
\emph{real non-negative}~$\epsilon$:
\[
  \ell^1_{\epsilon} := \{\c \in \ell^1 : \c_1 = \epsilon \} 
  \qquad \text{where } \epsilon \in \R,  \epsilon \geq 0.
\]
In the rest of our analysis, we will split elements $\c \in \ell^1$ into two parts: $\c_1$ and $\{\c_{k}\}_{k \geq 2}$.  
We define the basis elements $\e_j \in \ell^1$ for $j=1,2,\dots$ as
\[
  [\e_j]_k = \begin{cases}
  1 & \text{if } k=j, \\
  0 & \text{if } k \neq j.
  \end{cases}
\]
We note that $\| \e_j \|=2$. 
Then we can decompose
% We define
% \[
%   \tilde{\epsilon} := (\epsilon,0,0,0,\dots) \in \ell^1
% \]
% and
% For clarity when referring to sequences $\{c_{k}\}_{k \geq 2}$, we make the following definition:
% \[
% \ell^1_0  := \{ \tc \in \ell^1 : \tc_1 = 0 \}.
% \]
% With the
any $\c \in \ell^1_\epsilon$ uniquely as
\begin{equation}\label{e:aepsc}
  \c= \epsilon \e_1 + \tc \qquad \text{with}\quad 
  \tc \in \ell^1_0 := \{ \tc \in \ell^1 : \tc_1 = 0 \}.
\end{equation}
We follow the classical approach in studying Hopf bifurcations and consider 
$\c_1 = \epsilon$ to be a parameter, and then find periodic solutions with Fourier modes in $\ell^1_{\epsilon}$.
This approach rewrites the function $G: \R^2 \times \ell^K \to \ell^1$ as a function $\tilde{F}_\epsilon : \R^2 \times \ell^K_0 \to \ell^1$, where 
we denote 
\[
\ell^K_0 := \ell^1_0 \cap \ell^K.
\]
% I AM ACTUALLY NOT SURE IF YOU WANT TO DEFINE THIS WITH RANGE IN $\ell^1$
% OR WITH DOMAIN IN $\ell^1_0$ ?? IT SEEMS TO DEPEND ON WHICH GLOBAL STATEMENT YOU WANT/NEED TO MAKE!?
\begin{definition}
We define the $\epsilon$-parameterized family of  functions $\tilde{F}_\epsilon: \R^2 \times \ell^K_0  \to \ell^1$ 
by 
\begin{equation}
\label{eq:fourieroperators}
\tilde{F}_{\epsilon}(\alpha,\omega, \tc) := 
\epsilon [i \omega + \alpha e^{-i \omega}] \e_1 + 
( i \omega K^{-1} + \alpha U_{\omega}) \tc + 
\epsilon^2 \alpha e^{-i \omega}  \e_2  +
\alpha \epsilon L_\omega \tc + 
\alpha  [ U_{\omega} \tc] * \tc ,
\end{equation}
where
$L_\omega : \ell^1_0 \to \ell^1$ is given by
\[
   L_{\omega} := \sigma^+( e^{- i \omega} I + U_{\omega}) + \sigma^-(e^{i \omega} I + U_{\omega}),
\]
with $I$ the identity and  $\sigma^\pm$ the shift operators on $\ell^1$:
\begin{alignat*}{2}
\left[ \sigma^- a \right]_k &:=  a_{k+1}  , \\
\left[ \sigma^+ a \right]_k &:=  a_{k-1}  &\qquad&\text{with the convention } \c_0=0.
\end{alignat*}
The operator $ L_\omega$ is discontinuous in $\omega$ and $ \| L_\omega \| \leq 4$. 
\end{definition} 

%The maps $ \sigma^{+}$ and $ \sigma^-$ are shift up and shift down operators respectively. 
We reformulate Theorem~\ref{thm:FourierEquivalence1}  in terms of the map  $\tilde{F}$. 
We note that it follows from Lemma~\ref{l:analytic} and 
%\marginpar{Reformulate}
%one's choice of  
Equation~\eqref{eq:FourierSequenceEquation}  
%or Equation ~\eqref{eq:fourieroperators},
that the Fourier coefficients of any periodic solution of~\eqref{eq:Wright} lie in $\ell^K$.
These observations are summarized in the following theorem.
\begin{theorem}
\label{thm:FourierEquivalence2}
	Let $ \epsilon \geq 0$,  $\tc \in \ell^K_0$, $\alpha>0$ and $ \omega >0$. 
	Define $y: \R\to \R$ as 
\begin{equation}\label{e:ytc}
	y(t) = 
	\epsilon \left( e^{i \omega t }  + e^{- i \omega t }\right) 
	+  \sum_{k = 2}^\infty   \tc_k e^{i \omega k t }  + \tc_k^* e^{- i \omega k t } .
\end{equation}
%	and suppose that $ y(t) > -1$. 
	Then $y(t)$ solves~\eqref{eq:Wright} if and only if $\tilde{F}_{\epsilon}( \alpha , \omega , \tc) = 0$. 
	Furthermore, up to time translation, any periodic solution of~\eqref{eq:Wright} with period $2\pi/\omega$ is described by a Fourier series of the form~\eqref{e:ytc} with $\epsilon \geq 0$ and $\tc \in \ell^K_0$.
\end{theorem}


%We note that for $\epsilon>0$ such solutions are truly periodic, while for $\epsilon=0$ a zero of $\tilde{F}_\epsilon$ may either correspond to a periodic solution or to the trivial solution $y(t) \equiv 0$. 



% \begin{proof}
%  By Proposition \ref{thm:FourierEquivalence1}, it suffices to show that $\tilde{F}(\alpha,\omega,c) =0$ is equivalent to Equation \ref{eq:FourierSequenceEquation} being satisfied for $k \geq 1$.
%  Since Equation \ref{eq:FourierSequenceEquation} is equivalent to Equation \ref{eq:FHat}, we expand  Equation \ref{eq:FHat} by writing $ \hat{c} = \hat{\epsilon } + c$  where $ \hat{\epsilon} := (\epsilon,0,0,\dots) \in \ell^1$ as below:
%  \begin{equation}
%  0=  \left( i \omega K^{-1} + \alpha U_\omega \right) (\hat{\epsilon}+ c) + \alpha \left[U_\omega \, (\hat{\epsilon}+ c) \right] * (\hat{\epsilon}+ c) \label{eq:Intial}
%  \end{equation}
%  The RHS of Equation \ref{eq:Intial} is $ \tilde{F}(\alpha,\omega,c)$, so the theorem is proved.
% \end{proof}



Since we want to analyze a Hopf bifurcation, we will want to solve $\tilde{F}_\epsilon = 0$ for small values of~$\epsilon$. 
However, at the bifurcation point, $ D \tilde{F}_0(\pp  ,\pp , 0)$ is not invertible.
In order for our asymptotic analysis to be non-degenerate,
we work with a rescaled version of the problem. To this end, for any $\epsilon >0$, we rescale both $\tc$ and $\tilde{F}$ as follows. Let $\tc = \epsilon c$ and 
\begin{equation}\label{e:changeofvariables}
  \tilde{F}_\epsilon (\alpha,\omega,\epsilon c) = \epsilon F_\epsilon (\alpha,\omega,c).
\end{equation}
For $\epsilon>0$ the problem then reduces to finding zeros of 
\begin{equation}
\label{eq:FDefinition}
	F_\epsilon(\alpha,\omega, c) := 
	[i \omega + \alpha e^{-i \omega}] \e_1 + 
	( i \omega K^{-1} + \alpha U_{\omega}) c + 
	\epsilon \alpha e^{-i \omega} \e_2  +
	\alpha \epsilon L_\omega c + 
	\alpha \epsilon [ U_{\omega} c] * c.
\end{equation}
We denote the triple $(\alpha,\omega,c) \in \R^2 \times \ell^1_0$ by $x$.
To pinpoint the components of $x$ we use the projection operators
\[
   \pi_\alpha x = \alpha, \quad \pi_\omega x = \omega, \quad 
  \pi_c x = c \qquad\text{for any } x=(\alpha,\omega,c).
\]

After the change of variables~\eqref{e:changeofvariables} we now have an invertible Jacobian $D F_0(\pp  ,\pp , 0)$ at the bifurcation point.
On the other hand, for $\epsilon=0$ the zero finding problems for $\tilde{F}_\epsilon$ and $F_\epsilon$ are not equivalent. 
However, it follows from the following lemma that any nontrivial periodic solution having $ \epsilon=0$ must have a relatively large size when $ \alpha $ and $ \omega $ are close to the bifurcation point. 

\begin{lemma}\label{lem:Cone}
	Fix $ \epsilon \geq 0$ and $\alpha,\omega >0$. 
	Let
	\[
	b_* :=  \frac{\omega}{\alpha} - \frac{1}{2} - \epsilon  \left(\frac{2}{3}+ \frac{1}{2}\sqrt{2 + 2 |\omega-\pp| } \right).
	\]
Assume that $b_*> \sqrt{2} \epsilon$. 
Define
% \begin{equation*}%\label{e:zstar}
% 	z^{\pm}_* :=b_* \pm \sqrt{(b_*)^2- \epsilon^2 } .
% \end{equation*}
% \note[J]{Proposed change to match Lemma E.4}
\begin{equation}\label{e:zstar}
z^{\pm}_* :=b_* \pm \sqrt{(b_*)^2- 2 \epsilon^2 } .
\end{equation}
If there exists a $\tc \in \ell^1_0$ such that $\tilde{F}_\epsilon(\alpha, \omega,\tc) = 0$, then \\
\mbox{}\quad\textup{(a)} either $ \|\tc\| \leq  z_*^-$ or $ \|\tc\| \geq z_*^+  $.\\
\mbox{}\quad\textup{(b)} 
$ \| K^{-1} \tc \| \leq (2\epsilon^2+ \|\tc\|^2) / b_*$. 
\end{lemma}
\begin{proof}
	The proof follows from Lemmas~\ref{lem:gamma} and~\ref{lem:thecone} in Appendix~\ref{appendix:aprioribounds}, combined with the observation that
$\frac{\omega}{\alpha} - \gamma \geq b_*$,
% \[
%   \frac{\omega}{\alpha} - \gamma \geq b_*
%  \qquad\text{for all }
% | \alpha - \pp| \leq r_\alpha \text{ and } 
%   | \omega - \pp| \leq r_\omega.
% \]
with $\gamma$ as defined in Lemma~\ref{lem:gamma}.
\end{proof}

\begin{remark}\label{r:smalleps}
We note that for $\alpha < 2\omega$
\begin{alignat*}{1}
z^+_* &\geq   \frac{2 \omega - \alpha}{\alpha} 
- \epsilon \left(4/3+\sqrt{2 + 2 |\omega-\pp| } \, \right) + \cO(\epsilon^2)
\\[1mm]
z^-_* & \leq   \cO(\epsilon^2)
\end{alignat*}
for small $\epsilon$. 
Hence Lemma~\ref{lem:Cone} implies that for values of $(\alpha,\omega)$ near $(\pp,\pp)$ any solution has either $\|\tc\|$ of order 1 or $\|\tc\| =  \cO(\epsilon^2)$. 
The asymptotically small term bounding $z_*^-$ is explicitly calculated in Lemma~\ref{lem:ZminusBound}. 
A related consequence is that for $\epsilon=0$ there are no nontrivial solutions 
of $\tilde{F}_0(\alpha,\omega,\tc)=0$ with 
$\| \tc \| < \frac{2 \omega - \alpha}{\alpha} $. 
\end{remark}

\begin{remark}\label{r:rhobound}
In Section~\ref{s:contraction} we will work on subsets of $\ell^K_0$ of the form
\[
  \ell_\rho := \{ c \in \ell^K_0 : \|K^{-1} c\| \leq \rho \} .
\]
Part (b) of Lemma~\ref{lem:Cone} will be used in Section~\ref{s:global} to guarantee that we are not missing any solutions by considering $\ell_\rho$ (for some specific choice of $\rho$) rather than the full space $\ell^K_0$.
In particular, we infer from Remark~\ref{r:smalleps} that  small solutions (meaning roughly that $\|\tc\| \to 0$ as $\epsilon \to 0$)
satisfy $\| K^{-1} \tc \| = \cO(\epsilon^2)$.
\end{remark}

The following theorem guarantees that near the bifurcation point the problem of finding all periodic solutions is equivalent to considering the rescaled problem $F_\epsilon(\alpha,\omega,c)=0$.
\begin{theorem}
\label{thm:FourierEquivalence3}
\textup{(a)} Let $ \epsilon > 0$,  $c \in \ell^K_0$, $\alpha>0$ and $ \omega >0$. 
	Define $y: \R\to \R$ as 
\begin{equation}\label{e:yc}
	y(t) = 
	\epsilon \left( e^{i \omega t }  + e^{- i \omega t }\right) 
	+ \epsilon  \sum_{k = 2}^\infty   c_k e^{i \omega k t }  + c_k^* e^{- i \omega k t } .
\end{equation}
%	and suppose that $ y(t) > -1$. 
	Then $y(t)$ solves~\eqref{eq:Wright} if and only if $F_{\epsilon}( \alpha , \omega , c) = 0$.\\
\textup{(b)}
Let $y(t) \not\equiv 0$ be a periodic solution of~\eqref{eq:Wright} of period $2\pi/\omega$
 with Fourier coefficients $\c$.
Suppose $\alpha < 2\omega$ and $\| \c \| < \frac{2 \omega - \alpha}{\alpha} $.
Then, up to time translation, $y(t)$ is described by a Fourier series of the form~\eqref{e:yc} with $\epsilon > 0$ and $c \in \ell^K_0$.
\end{theorem}

\begin{proof}
Part (a) follows directly from Theorem~\ref{thm:FourierEquivalence2} and the  change of variables~\eqref{e:changeofvariables}.
To prove part (b) we need to exclude the possibility that there is a nontrivial solution with $\epsilon=0$. The asserted bound on the ratio of $\alpha$ and $\omega$ guarantees, by Lemma~\ref{lem:Cone} (see also Remark~\ref{r:smalleps}), that indeed $\epsilon>0$ for any nontrivial solution. 
\end{proof}

We note that in practice (see Section~\ref{s:global}) a bound on $\| \c \|$ is derived from a bound on $y$ or $y'$ using Parseval's identity.

\begin{remark}\label{r:cone}
It follows from Theorem~\ref{thm:FourierEquivalence3} and Remark~\ref{r:smalleps} that for values of $(\alpha,\omega)$ near $(\pp,\pp)$ any reasonably bounded solution satisfies $\| c\| =  O(\epsilon)$ as well as $\|K^{-1} c \| = O(\epsilon)$ asymptotically (as $\epsilon \to 0$).
These bounds will be made explicit (and non-asymptotic) for specific choices of the parameters in Section~\ref{s:global}.
\end{remark}

% We are able to rule out such large amplitude solutions using global estimates such as those in \cite{neumaier2014global}.
% Hence, near the bifurcation point, the problem of describing periodic solutions of~\eqref{eq:Wright} reduces to studying the family of zeros finding problems $F_\epsilon=0$.





%Specifically, if a solution having $ \epsilon = 0$ does in fact correspond to a nontrivial periodic solution and $\alpha  < 2\omega $, then $ \| \tilde{c} \| > 2 \omega \alpha^{-1} -1$. 
%%PERHAPS THIS NEEDS A FORMULATION AS A THEOREM AS WELL?
%%IN OTHER WORDS: ARE WE SURE WE HAVE FOUND ALL ZEROS OF $\tilde{F}_0$, I.E. ALL SOLUTIONS WITH $\epsilon=0$ NEAR THE BIFURCATION POINT? AFTER RESCALING THESE ARE INVISIBLE?
%%THERE SHOULD BE A STATEMENT ABOUT THIS SOMEWHERE! EITHER HERE OR SOME





We finish this section by defining a curve of approximate zeros $\bx_\epsilon$ of $F_\epsilon$ 
(see \cite{chow1977integral,hassard1981theory}). 
%(see \cite{chow1977integral,morris1976perturbative,hassard1981theory}). 


\begin{definition}\label{def:xepsilon}
Let
\begin{alignat*}{1}
	\balpha_\epsilon &:= \pp + \tfrac{\epsilon^2}{5} ( \tfrac{3\pi}{2} -1)  \\
	\bomega_\epsilon &:= \pp -  \tfrac{\epsilon^2}{5} \\
	\bc_\epsilon 	 &:= \left(\tfrac{2 - i}{5}\right) \epsilon \,  \e_2 \,.
\end{alignat*}
We define the approximate solution 
$ \bx_\epsilon := \left( \balpha_\epsilon , \bomega_\epsilon  , \bc_\epsilon \right)$
for all $\epsilon \geq 0$.
\end{definition}

We leave it to the reader to verify that both 
 $F_\epsilon(\pp,\pp,\bc_{\epsilon})=\cO(\epsilon^2)$ and $F_\epsilon(\bx_\epsilon)=\cO(\epsilon^2)$.
%%%	
%%%	
%%%	}{Better like this?}
%%%\annote[J]{ $F_\epsilon(\bx_0)=\cO(\epsilon^2)$ and $F_\epsilon(\bx_\epsilon)=\cO(\epsilon^2)$.}{I think we'd still need the $ \bar{c}_\epsilon$ term in $\bar{x}_0$ to be of order $ \epsilon$.}
%%%\remove[JB]{We show in Proposition A.1
%%%%\ref{prop:ApproximateSolutionWorks} 
%%% that any $ x \in \R^2 \times \ell^1_0$ which is $ \cO(\epsilon^2)$ close to $ \bar{x}_\epsilon $ will yield the estimate $F_\epsilon(x) = \cO(\epsilon^2)$.
%%%Hence choosing $\{ \pp , \pp, \bar{c}_\epsilon\}$ as our approximate solution would also have been a natural choice for performing an $\cO(\epsilon^2)$ analysis and would have simplified several of our calculations.
%%%However,} 
%%%
We choose to use the more accurate approximation 
for the $ \alpha$ and $ \omega $ components to improve our final quantitative results. 














%
% Values for $ (\alpha, \omega,c)$ which approximately solve $\tilde{F}(\alpha,\omega,c) = 0$  are computed in  \cite{chow1977integral,morris1976perturbative,hassard1981theory} and are as follows:
%  \begin{eqnarray}
%  \tilde{\alpha}( \epsilon) &:=& \pi /2 + \tfrac{\epsilon^2}{5} ( \tfrac{3\pi}{2} -1) \nonumber \\
%  \tilde{\omega}( \epsilon) &:=& \pi /2 -  \tfrac{\epsilon^2}{5} \label{eq:ScaleApprox} \\
%  \tc(\epsilon) 	  &:=& \{ \left(\tfrac{2 - i}{5}\right)  \epsilon^2 , 0,0, \dots \} \nonumber
%  \end{eqnarray}
% In Appendix \ref{sec:OperatorNorms} we illustrate an alternative method for deriving this approximation.
%
%
%
%
% We want to solve $ \tilde{F}(\alpha , \omega, \hat{c}) =0$ for small values of $ \epsilon$.
% However $ D \tilde{F}(\alpha , \omega , c)$ is not invertible at $ ( \pp , \pp , 0)$ when $ \epsilon = 0$.
% In order for our asymptotic analysis to be non-degenerate, we need to make the change of variables $ c \mapsto \epsilon c$.
% Under this change of variables, we define the function $ F$ below so that $ \tilde{F}(\alpha , \omega , \epsilon c) =\epsilon  F( \alpha , \omega , c)$.
%
%
%
% \begin{definition}
% Construct an $\epsilon$-parameterized family of densely defined functions  $F : \R^2 \oplus \ell^1 / \C \to \ell^1$ by:
% \begin{equation}
% \label{eq:FDefinition}
% 	F(\alpha,\omega, c) :=
% 	[i \omega + \alpha e^{-i \omega}]_1 +
% 	( i \omega K^{-1} + \alpha U_{\omega}) c +
% 	[\epsilon \alpha e^{-i \omega}]_2  +
% 	\alpha \epsilon L_\omega c +
% 	\alpha \epsilon [ U_{\omega} c] * c.
% \end{equation}
% \end{definition}

%%
%%
%%\begin{corollary}
%%	\label{thm:FourierEquivalence3}
%%	Fix $ \epsilon > 0$, and $ c \in \ell^1 / \C $, and $ \omega >0$. Define $y: \R\to \R$ as 
%%	\[
%%	y(t) = 
%%	\epsilon \left( e^{i \omega t }  + e^{- i \omega t }\right) 
%%	+  \epsilon  \left( \sum_{k = 2}^\infty   c_k e^{i \omega k t }  + \overline{c}_k e^{- i \omega k t } \right) 
%%	\]
%%	and suppose that $ y(t) > -1$. 
%%	Then $y(t)$ solves Wright's equation at parameter $ \alpha > 0 $ if and only if $ F( \alpha , \omega , c) = 0$ at parameter $ \epsilon$. 
%%	
%%	
%%	
%%\end{corollary}
%%
%%
%%\begin{proof}
%%	Since $ \tilde{F}(\alpha,\omega, \epsilon c) = \epsilon F( \alpha , \omega , c)$, the result follows from Theorem \ref{thm:FourierEquivalence2}.
%%\end{proof}

% If we can find $(\alpha , \omega, c)$ for which $ F( \alpha , \omega,c)=0$ at parameter $\epsilon$, then $ \tilde{F}(\alpha ,\omega, c)=0$.
% By Theorem \ref{thm:FourierEquivalence2} this amounts to finding a periodic solution to Wright's equation.
% Lastly, because we have performed the change of variables $ c \mapsto \epsilon c$, we need to  apply this change of variables to our approximate solution as well.
%
% \begin{definition}
% 	Define the approximate solution $ x( \epsilon) = \left\{ \alpha(\epsilon ) , \omega ( \epsilon ) , c(\epsilon) \right\}$ as below,  where $c(\epsilon) = \{ c_2( \epsilon) , 0 ,0 , \dots\} $.
% 	We may also write $ x_\epsilon = x(\epsilon) $.
% 	\begin{eqnarray}
% 	\alpha( \epsilon) &:=& \pi /2 + \tfrac{\epsilon^2}{5} ( \tfrac{3\pi}{2} -1) \nonumber \\
% 	\omega( \epsilon) &:=& \pi /2 -  \tfrac{\epsilon^2}{5} \label{eq:Approx} \\
% 	c_2(\epsilon) 	  &:=& \left(\tfrac{2 - i}{5}\right) \epsilon \nonumber
% 	\end{eqnarray}
%
% \end{definition}


\section{Local results}
\label{s:local}
%!TEX root = hopfwright.tex

\subsection{Constructing a Newton-like operator}
\label{s:newtonlike}

In this section and in the appendices we often suppress the subscript in $F=F_\epsilon$.
We will find solutions to the equation $F(\alpha ,\omega , c)=0$ by the
constructing a Newton-like operator $T$ such that fixed points of $T$
corresponds precisely to zeros of $F$. In order to construct the map $T$ we
need an operator $A^{\dagger}$ which is an approximate inverse of 
$DF(\bx_\epsilon)$. 
% Since
% $\bx_\epsilon$ is an approximate zero of $F_\epsilon$ up to order
% $\cO(\epsilon^2)$ correction terms,
We will use an approximation $A$ of 
$DF( \bx_\epsilon )$ that is linear in~$\epsilon$ and correct up to $\cO(\epsilon^2)$.
% (recall that $F(\bx_\epsilon)=\cO(\eps^2)$). 
Likewise, we define $A^{\dagger}$ to be linear in $\epsilon$ (and again correct up to $\cO(\epsilon^2)$). 

It will be convenient to use the usual identification $i_\C : \R^2 \to \C$ given by $i_\C (x,y) = x+iy $. We also use $\omega_0 := \pi/2$.
 % order
% Since $x(\epsilon)$ is only correct up to order $\cO(\epsilon^2)$, then it only makes sense to compute our approximate derivative up to order $\cO( \epsilon^2)$.

% \marginpar{Jonathan: I tried to be careful about the spaces here, but it all seems a bit of a distraction since everything is explicit in coordinates}
\begin{definition}\label{def:A}
We introduce the linear maps $A:  \R^2 \times \ell^K_0 \to \ell^1$ and 
$ A^{\dagger}:  \ell^1 \to  \R^2 \times \ell^K_0 $ by
\begin{alignat*}{1}
A &:= A_0 + \epsilon A_1 \, , \\
A^{\dagger} &:= A_0^{-1} - \epsilon A_0^{-1} A_1 A_0^{-1} \,  ,
%\label{eq:ADagger}
\end{alignat*}
where the linear maps $ A_0 , A_1 : \R^2 \times \ell^K_0 \to \ell^1$  are defined below. Writing $x=(\alpha,\omega,c)$, we set
\begin{alignat*}{1}
A_0	x = A_0 (\alpha,\omega,c) & := i_\C A_{0,1} 
\!\left[\!\! \begin{array}{c} \alpha \\ \omega \end{array} \!\!\right]  \e_1
 + A_{0,*}  c , \\
A_1 x =	A_1 (\alpha,\omega,c) & := i_\C  A_{1,2}
\!\left[\!\! \begin{array}{c} \alpha \\ \omega \end{array} \!\!\right]  \e_2
 + A_{1,*}  c .
%\label{eq:ApproxDFdef}
\end{alignat*}
Here the matrices $A_{0,1}$ and $A_{1,2}$ are given by
\begin{equation}
A_{0,1} := 
\left[
\begin{matrix}
0 & - \pp \\
-1  & 1 
\end{matrix} 
\right]
\qquad\text{and}\qquad
A_{1,2} := \frac{1}{5}
\left[
\begin{matrix}
-2 & 2-\tfrac{3 \pi}{2} \\
-4  & 2(2+\pi) 
\end{matrix}  
\right]  ,
\label{eq:defA12}
\end{equation}
and the linear maps $A_{0,*} : \ell^K_0 \to \ell^1_0$ and
$A_{1,*} : \ell^K_0 \to \ell^1$
are given by
\begin{equation*}
% A_{0,*} :& \ell^1_0 \to \ell^1_0
% &
% A_{1,*} :& \ell^1_0 \to \ell^1  \\
% %%%%%%
% A_{0,1} :& \{ \alpha, \omega\} \to \{ Re \, F_1 , Im\, F_1 \}
% &
% A_{1,2} :& \{ \alpha, \omega\} \to \{ Re \,F_2 , Im \, F_2 \}
% \end{align*}
% and given by the equations below, taking $ \omega_0 = \pp$.
% \begin{align}
A_{0,*} 	 := \tfrac{\pi}{2} ( i K^{-1} + U_{\omega_0}) 
\qquad\text{and}\qquad
A_{1,*} 	:= \tfrac{\pi}{2} L_{\omega_0} .
\end{equation*}
%%%%%%%%%%%%%%%%%%%%
% A_{0,1} := &
% \left[
% \begin{matrix}
% 0 & - \pp \\
% -1  & 1
% \end{matrix}
% \right]
% &
% A_{1,2} :=& \frac{1}{5}
% \left[
% \begin{matrix}
% -2 & 2-\tfrac{3 \pi}{2} \\
% -4  & 2(2+\pi)
% \end{matrix}
% \right]
% \label{eq:defA12}
% \end{align}
\end{definition}

Since $K$ and $U_{\omega_0}$ both act as diagonal operators, the inverse 
$A_{0,*}^{-1} : \ell^1_0 \to \ell^K_0$ of $A_{0,*}$ is given by
\begin{equation*}
	  (A_{0,*}^{-1} a)_k = \frac{2}{\pi} \frac{a_k}{ik+e^{-ik\omega_0}} 
	  \qquad\text{for all } k \geq 2.
\end{equation*} 
An explicit computation, which we leave to the reader, shows that these approximations are indeed correct up to $\cO(\epsilon^2)$. 
In particular, $A^{\dagger} = \left[ DF( \bx_\epsilon ) \right]^{-1} + \cO(\epsilon^2)$.
In Appendix~\ref{sec:OperatorNorms} several additional properties of these operators are derived. The most important one is the following.
% \note[J]{I've tried to make this change for the new injectivity bound throughout.} \note[JB]{Seems fine, but wouldn't it be nicer to write $\tfrac{\sqrt{10}}{4}$ instead of $\tfrac{5}{2 \sqrt{10}}$?} \note[J]{Yes it would, made changes below. }
\begin{proposition}
	\label{prop:Injective}
	For 
%\change[J]{$0 \leq \epsilon < \tfrac{5}{2} ( 4 + \sqrt{10})^{-1} \approx 0.349$}
	$0 \leq \epsilon < \tfrac{\sqrt{10}}{4} \approx 0.790$
	 the operator $ A^{\dagger}$ is injective. 
\end{proposition}
\begin{proof}
	In order to show that $ A^{\dagger}$ is injective we show that 
	it has a left inverse. 
	Note that $ A A^{\dagger} = I - \epsilon^2 ( A_1 A_0^{-1})^2$. 
	By Proposition \ref{prop:A1A0} it follows that 
%	\change[J]{$ \| A_1 A_0^{-1} \| \leq \tfrac{2}{5} ( 4 + \sqrt{10})$}
	 $ \| A_1 A_0^{-1} \| \leq \tfrac{2 \sqrt{10}}{5} $.  
	By choosing 
%\change[J]{$ \epsilon < \tfrac{5}{2} ( 4 + \sqrt{10})^{-1}$}
$ \epsilon < \tfrac{\sqrt{10}}{4}$ 
we obtain 
	$\|  \epsilon^2 ( A_1 A_0^{-1})^2 \| < 1$, whereby $ A A^{\dagger}$ is 
	invertible, and so $ A^{\dagger}$ is injective. 
\end{proof}


\begin{definition}
We define the operator $ T: \R^2 \times \ell^K_0 \to \R^2 \times \ell^K_0 $ by
\begin{equation*}
	T(x) :=  x - A^{\dagger} F(x) ,
\end{equation*}
	where  $F$ is defined in Equation~\eqref{eq:FDefinition}  and $A^{\dagger}$ in Definition~\ref{def:A}.
	We note that $F$, $A^{\dagger}$ and $T$ depend on the parameter $\epsilon \geq 0$, although we suppress this in the notation.
\end{definition}

% \note[J]{When we got the better bound on $\|A_1 A_{0}^{-1}\|$, then $A^{\dagger}$ being injective ceased to be a bottleneck for doing a Hopf bifurcation. I don't think we'd lose much if we just delete this remark.  }
% \begin{remark}
% 	\label{r:Injective}
% \remove[J]{
% 	If $A^{\dagger}$ is injective, which is true for
% 	$0 \le \epsilon <  \tfrac{5}{2} ( 4 + \sqrt{10})^{-1}$ by Proposition 3.2, then the fixed points of $T$ correspond bijectively with the zeros of $F$.
% 	Since the periodic solution having $ \epsilon_0 = \tfrac{5}{2} ( 4 + \sqrt{10})^{-1}$ corresponds approximately to $\bar{\alpha}_{\epsilon_0} = \pp + 0.090$, above this value we cannot use the Newton-like operator $T$ to reliably study the SOPS to Wright's equation.
% 	Hence $ \alpha = \pp + 0.09$ represents an upper bound for doing an $\cO(\epsilon^2)$ Hopf bifurcation analysis.}
% \end{remark}
%


\subsection{Explicit contraction bounds}
\label{s:contraction}


The map $T$ is not continuous on all of $\R^2 \times \ell^K_0$,
since $ U_{\omega} c $ is not continuous in $\omega$.
While continuity is ``recovered'' for terms of the form $A^{\dagger} U_{\omega} c$,  this is not the case for the nonlinear part $ - \alpha \epsilon A^{\dagger} [ U_{\omega} c ] * c$.  
% The problem is that while in the $ U_{\omega} c$ term and that  $ \tfrac{\partial}{\partial \omega} U_{\omega} = - i K^{-1} U_{\omega}$.
% Since  the map $ A^{\dagger}$ is approximately $\tfrac{2 }{\pi i} K$, then the  $  A^{\dagger}  U_{\omega} c$ component of $ T$ is continuous in $\omega$.
%
%For any $ \omega_1, \omega_2\in \R$  then $ \| U_{\omega_1} - U_{\omega_2} \|  = 2$ when $ 2 \pi $ does not divide $ \omega_1 - \omega_2$.  
%This is not a problem for the $ A^{\dagger} ( i \omega K^{-1} + \alpha U_{\omega} ) c$ component of $T$; 
%the and then $ A^{\dagger } U_{\omega}$ is continuous in $\omega$.  
% However, since $ U_{\omega}$ is inside a convolution in the nonlinear part, this type of simplification cannot happen.
% 
We overcome this difficulty by fixing some $ \rho > 0$ and restricting the domain of $T$ to sets of the form 
%$\R^2 \times X_\rho$, where
\[
  \R^2 \times  \{ c \in \ell^K_0 : \|K^{-1} c\| \leq \rho \} = \R^2 \times \ell_\rho.
\]
Since we wish to center the domain of $T$ about the approximate solution~$\bx_\epsilon$, we introduce the following definition, which uses a triple of radii $r \in \R^3_+$, for which it will be convenient to use two different notations:
\[
  r = ( r_{\alpha } , r_{\omega} , r_c) = (r_1,r_2,r_3).
\]
\begin{definition}
	Fix   $ r \in \R^3_+$ and $ \rho > 0$ and let  $ \bx_\epsilon = ( \balpha_\epsilon , \bomega_\epsilon , \bc_\epsilon )$ be as defined in Definition~\ref{def:xepsilon}. 
    We define the $\rho$-ball $B_\epsilon(r,\rho) \subset \R^2 \times \ell^1_0$
    of radius $r$ centered at $\bx_\epsilon$ to be the set of points satisfying 
\begin{alignat*}{1}
	|  \alpha -\balpha_\epsilon | & \leq  r_\alpha  \\
	| \omega - \bomega_\epsilon  | & \leq  r_{\omega} \\
	\| c - \bc_\epsilon  \| & \leq r_c \\
	\|K^{-1} c\| & \leq  \rho .
\end{alignat*}
\end{definition}

We want to show that $T$ is a contraction map on some $\rho$-ball 
$B_\epsilon(r,\rho) \subset \R^2 \times \ell^1_0$ using a Newton-Kantorovich argument. 
This will require us to develop a bound on $DT$ using some norm on  $ X$.  
Unfortunately there is no natural choice of norm on the product space $ X$. 
Furthermore, it will not become apparent if one norm is better than another until after  significant calculation.  
For this reason, we use a notion of an ``upper bound'' which allows us to delay our choice of norm. 
We first introduce the operator $\zeta:  X  \to \R^3_{+}$
which consists of the norms of the three components:
\[
  \LL(x) :=   ( |\pi_\alpha x|, |\pi_\omega x|, \|\pi_c x\| )^T \in \R^3_{+}
  \qquad\text{for any } x \in X.
\]
% \note[JB]{Propose to revert to using single overlines for the upper bound; the double overlines I had introduced just look ridiculous to me now. There is a potential for confusion with $\bx_\epsilon$, but I can live with it.} \note[J]{I would also prefer using only one overline. }
\begin{definition}[upper bound]\label{def:upperbound}
We call $\upperbound{x} \in \R^3_+$ an upper bound on $x$ if $\LL(x) \leq \upperbound{x}$, where the inequality is interpreted componentwise in $\R^3$. 
Let $X'$ be a subspace of $X$ and let $X''$ be a subset of $X'$.   
An upper bound on a linear operator $A' : X' \to X $ over $X''$  is 
a $3 \times 3$ matrix $\upperbound{A'} \in \text{\textup{Mat}}(\R^3 , \R^3)$ such that
\[
   \LL(A' x ) \leq \upperbound{A'} \cdot \LL(x)  
     \qquad\text{for any }  x \in X'',
\]
where the inequality is again interpreted componentwise in $\R^3$. 
The notion of upper bound conveniently encapsulates bounds on the different components of the operator $A'$ on the product space $X$. Clearly the components of the matrix $\upperbound{A'}$ are nonnegative.


		% 	Let $ (\alpha , \omega , c) \in \R^2 \times \ell^1_0$ and $  \upperbound{x} = ( x_{ \alpha } , x_{\omega} , x_{c}) \in \R^3_+$.
		% 	Then $ \upperbound{x}$ is an \emph{upper bound} on $(\alpha , \omega , c)$ if $ | \alpha | \leq x_\alpha$ and $ | \omega | \leq x_{\omega}$ and $ \| c\|\leq x_{c}$.
		% Similarly, suppose that $ A' : X \to \R^2 \times \ell^1_0$ is a linear operator, defined on some domain $ X \subset \R^2 \times \ell^1_0$.
		% Then $ \upperbound{A'} \in Mat(\R^3 , \R^3)$ is an \emph{upper bound } on $ A'$  if $ \upperbound{A'} \cdot \upperbound{x} $ is an upper bound on $ A' x$ whenever $ \upperbound{x} \in \R^3_+$ is an upper bound on all $ x \in X$.
\end{definition}
		% The notion of upper bounds commutes with vector addition and matrix multiplication.
		% That is, if $\upperbound{x} , \upperbound{y} \in \R^3_+ $ are upper bounds on $ x,y \in \R^2 \times \ell^1_0$, then $ \upperbound{x} + \upperbound{y}$ is an upper bound on $x +y$.
		% Similarly, if we have two linear operators $A'$ and $A''$ with upper bounds
		% $\upperbound{A'}$ and $\upperbound{A''}$, respectively, then $ \upperbound{A'} \cdot \upperbound{A''}$ is an upper bound for $ A' \circ A''$.
		% Furthermore, if $\upperbound{A'} \in Mat(\R^3,\R^3)$ is an upper bound, then the entries of this matrix are necessarily non-negative.
For example, in Proposition \ref{prop:A0A1} we calculate an upper  bound on the map $A_0^{-1} A_1$.  
As for the domain of definition of $T$, in practice we use $X' = \R^2 \times  \ell^K_0  $ and  $X'' = \R^2 \times  \ell_\rho  $.
The subset $X''$ does not always affect the upper bound calculation (such as in Proposition \ref{prop:A0A1}). 
However, operators such as $U_{\omega} - U_{\omega_0}$ have upper bounds which contain $\rho$-terms (see for example Proposition \ref{prop:OmegaDerivatives}).

Using this terminology, we state a ``radii polynomial'' theorem, which allows us to check whether $T$ is a contraction map. This technique has been used frequently in a computer-assisted setting in the past decade. Early application include~\cite{daylessardmischaikow,lessardvandenberg}, while a previous implementation in the context of Wright's delay equation can be found in~\cite{lessard2010recent}. 
Although we use radii polynomials as well, our approach differs significantly from the computer-assisted setting mentioned above. 
While we do engage a computer (namely the Mathematica file~\cite{mathematicafile}) to optimize our quantitative results, the analysis is performed essentially in terms of pencil-and-paper mathematics (in particular, our operators do not involve any floating point numbers).
In our current setup we employ \emph{three} radii as a priori unknown variables,
which builds on an idea introduced in~\cite{vandenberg}.
We note that in most of the papers mentioned above the notation of $A$ and $A^\dagger$ is reversed compared to the current paper.

As preparation, the following lemma (of which the proof can be found in Appendix~\ref{sec:CompactDomain})  provides an explicit choice for $\rho$, as a function of $\epsilon$ and $r$, for which we have proper control on the image of $B_\epsilon(r,\rho)$ under $T$.
\begin{lemma}\label{lem:Crho}
For any $\epsilon \geq 0$ and $r \in\R^3_+$, let $C=C(\epsilon,r)$ be given  by Equation~\eqref{eq:RhoConstant}. 
If $C(\epsilon,r) >0$  then 
% Proposition~\ref{prop:DerivativeEndo} states that
\begin{equation}\label{e:Cepsr} 
  \| K^{-1} \pi_c  T(x) \| \leq \rho 
  \quad\text{whenever } x \in B_\epsilon(r,\rho) \text{ and } \rho \geq C(\epsilon,r).
\end{equation}
%%%%\marginpar{Jonathan: please fix appendix to reflect this (and define $C$ there)}
Moreover, $C(\epsilon,r)$ is nondecreasing in $\epsilon$ and $r$. 
\end{lemma}

\begin{proof}
See Proposition~\ref{prop:DerivativeEndo}.
\end{proof}


\begin{theorem}
	\label{thm:RadPoly}
	Let  
%\change[J]{$0 \leq \epsilon < \tfrac{5}{2} ( 4 + \sqrt{10})^{-1} $}
 $0 \leq \epsilon < \tfrac{\sqrt{10}}{4} $  
 and fix $r = (r_\alpha, r_\omega, r_c) \in \R^3_+$. Fix $\rho > 0$ such that $ \rho \geq C(\epsilon,r)$, as given by Lemma~\ref{lem:Crho}.
 % as in Proposition \ref{prop:DerivativeEndo} (REFORMULATE TO POINT TO THE PREPARATION ABOVE).
%
Suppose that $Y(\epsilon) $ is an upper bound on $ T(\bx_\epsilon) - \bx_\epsilon$ and $Z(\epsilon , r ,\rho) $ a (uniform) upper bound on $ DT(x) $ for all $ x \in B_\epsilon(r,\rho)$. 
Define the \emph{radii polynomials}
$P :\R^5_+ \to \R^3 $  by 
 \begin{equation}
 \label{eq:RadPolyDef}
  P(\epsilon,r,\rho) := Y(\epsilon) - \left[ I - Z( \epsilon,r,\rho) \right] \cdot r  \,  .
 \end{equation}
If each component of $P(\epsilon,r,\rho)$ is negative, then there is  a unique $\hat{x}_\epsilon \in B_\epsilon( r , \rho)$ such that $F(\hat{x}_\epsilon) =0$. 
\end{theorem}

\begin{proof}    
Let $r \in \R^3_+$ be a triple such that $P(\epsilon,r,\rho)<0$.
By Proposition \ref{prop:Injective}, if 
%\change[J]{$\epsilon < \tfrac{5}{2} ( 4 + \sqrt{10})^{-1} $}
$\epsilon <\tfrac{\sqrt{10}}{4} $
then $ A^{\dagger}$ is injective. 
Hence $ \hat{x}_{\epsilon} $ is a fixed point of $T$ if and only if $ F( \hat{x}_{\epsilon}) = 0$.  
In order to show  there is a unique fixed point $ \hat{x}_{\epsilon}$, we show that $T$ maps  $ B_{\epsilon}(r,\rho) $ into itself and that $ T $ is a contraction mapping. 

We first show that $T: B_\epsilon(r,\rho) \to B_\epsilon(r,\rho)$. 
Since $ \rho \geq C(\epsilon,r)$ then by Equation~\eqref{e:Cepsr} it follows that $ \| K^{-1} \pi_c T( x) \| \leq \rho$ for all $ x \in B_\epsilon(r,\rho)$.
In order to show that $T(x) \in B_\epsilon(r,\rho)$, it suffices to show that $ r=(r_\alpha , r_\omega, r_c)$ is an upper bound on $ T(x) - \bx_\epsilon$
for all $ x \in B_\epsilon(r,\rho)$.  
We decompose 
%by breaking $T(x) - \bx_\epsilon$ into two parts: 
\begin{equation}\label{e:Tsplit}
	T(x) - \bx_\epsilon = [T(\bx_\epsilon) -\bx_\epsilon] +
	[T(x) - T(\bx_\epsilon)],
\end{equation}
and estimate each part separately. Concerning the first term,
by assumption, $Y(\epsilon)$ is an upper bound on $T(\bx_\epsilon) - \bx_\epsilon$. 
%
Concerning the second term, we claim that $ Z(\epsilon,r,\rho) \cdot r$ is an upper bound on $T(x) - T(\bx_\epsilon)$.
Indeed, we have the following somewhat stronger bound: 
% if $ x,y\in B_\epsilon(r,\rho)$ and $\upperbound{\xi}$ is an upper bound on $y-x$, then
% $Z(\epsilon,r,\rho) \cdot \upperbound{\xi}$ is an upper bound on $T(y) - T(x)$,
% i.e.,
\begin{equation}\label{e:DTisboundedbyZ}
	\LL(T(y) - T(x)) \leq Z(\epsilon,r,\rho) \cdot \LL(y-x)
	\qquad\text{for all } x,y \in B_\epsilon(r,\rho) .
\end{equation}
The latter follows from the mean value theorem, since 
$T$ is continuously Fr\'echet differentiable on $B_\epsilon(r,\rho)$.
%
% MORE DETAILED ARGUMENT PROBABLY NOT NEEDED?
% \begin{equation}
% \label{eq:ZIntegrationBound}
% 	T( y) - T(x) \leq Z(\epsilon,r,\rho) \cdot \upperbound{\xi}
% \end{equation}
% Since $T$ is continuously Frechet differentiable on $B_\epsilon(r,\rho)$,
% then it follows that $T(y) - T(x) = \int_x^y DT( z) dz  $.
% %Since $ B_\epsilon(r,\rho)$ is convex, then $z= x+ s(y-x) \in B_\epsilon(r,\rho)$ for all $s \in [0,1]$.
% By assumption $Z(\epsilon,r,\rho)$ is an upper bound on $DT(z)$ for all $z \in B_\epsilon(r,\rho)$.
% If $ \upperbound{\xi} $ is an upper bound on $ y-x$, then we obtain the inequality $ T(y) - T(x) \leq \int_0^1 Z(\epsilon,r,\rho) \cdot \upperbound{\xi} \, ds$, from which Equation \ref{eq:ZIntegrationBound} follows.
%
Since $r$ is an upper bound on $x - \bx_\epsilon$ for all $ x \in B_\epsilon(r,\rho)$, we find, by using~\eqref{e:Tsplit}, that  
% have obtained that
% \begin{eqnarray}
% 	T(\bx_\epsilon) - \bx_\epsilon &=& \left[ T(\bx_\epsilon) - \bx_\epsilon \right] +
% 	\left[ T(x) - T(\bx_\epsilon) \right] \\
% 	&\leq& Y(\epsilon) + Z(\epsilon,r,\rho) \cdot r
% \end{eqnarray}
% By assumption each component of Equation \ref{eq:RadPolyDef} is negative, so
$Y(\epsilon) + Z(\epsilon,r,\rho) \cdot r \leq r$ (with the inequality, interpreted componentwise, following from $P(\epsilon,r,\rho)<0$) is an upper bound on $T(x) - \bx_\epsilon$
for all $ x \in B_\epsilon(r,\rho)$.  
%It follows that $r$ is an upper bound on $T(x) - \bx_\epsilon $. 
That is to  say, if all of the  radii polynomials are negative, 
then  $T$ maps $B_\epsilon(r,\rho) $ into itself.

To finish the proof we show that $T$ is a contraction mapping. 
We abbreviate $Z=Z(\epsilon,r,\rho)$ and  recall that $r=(r_\alpha,r_\omega,r_c)=(r_1,r_2,r_3) \in \R^3_+$
is such that $Z \cdot r < r$, hence for some $\kappa <1$ we have
\begin{equation}\label{e:defkappa}
  \frac{(Z \cdot r)_i}{r_i} \leq \kappa  \qquad\text{for } i=1,2,3.
\end{equation}

We now need to choose a norm on $X$. 
We define a norm $ \| \cdot \|_r$ on elements $x = (\alpha,\omega,c) \in X$
by
\[  
\| (\alpha, \omega, c) \|_r := \max 
\left\{  		  
	 \frac{|\alpha|}{r_\alpha},
	 \frac{|\omega|}{r_\omega},
	 \frac{\|c\|}{r_{c}} \right\} , 
\]
or
\[
  \|x\|_r = \max_{i=1,2,3} \frac{ \LL(x)_i}{r_i}
  \qquad \text{for all } x \in X.
\]
% We also introduce the compatible norm $ \| \cdot \|_{\tilde{r}}$ on $\R^3$ by
% $ \| (y_1, y_2, y_3) \|_{\tilde{r}} = \max_{i=1,2,3 }
% \{\frac{|y_i|}/{r_i} \}$, so that $\|x\|_r = \| \LL(x) \|_{\tilde{r}}$ for all $x \in X$.
%
By using the upper bound $Z$, we bound the Lipschitz constant of $T$ on $B_\epsilon(r, \rho)$ as follows:
\begin{alignat*}{1}
 \| T(y) - T(x) \|_r 
 % &= \|\LL(T(y) - T(x)) \|_{\tilde{r} } \\
    &= \max_{i=1,2,3} \frac{\LL(T(y) - T(x))_i} {r_i} \\
    &\leq  \max_{i=1,2,3}  \frac{(Z \cdot \LL(y-x))_i}{r_i} \\
    &\leq  \max_{i=1,2,3} \max_{j=1,2,3}\frac{\LL(y-x)_j}{r_j}  
				   \frac{(Z \cdot r )_i}{r_i} \\
    & = \| y-x \|_r \max_{i=1,2,3} \frac{(Z \cdot r )_i}{r_i} \\
    & \leq \kappa \| y-x \|_r,
\end{alignat*}
where we have used~\eqref{e:DTisboundedbyZ} and~\eqref{e:defkappa} with $\kappa<1$.
% \sup_{x,y \in B_\epsilon(r,\rho)} \frac{\|T(y) - T(x) \|_r}{\| y- x\|_r}
% \leq
% \sup_{ x,y \in B_\epsilon(r,\rho)}
% \frac{\left \|
% Z(\epsilon,r,\rho) \cdot \upperbound{\xi}
% \right\|_{\tilde{r}} }{  \| y -x\|_{r}} ,
% \]
% where $\upperbound{\xi}$ is any upper bound on $y-x$, as before.
% \marginpar{THIS IS STILL NOT ENTIRELY CLEAR!}
% If $u \in \R^3$ and $ \|u\|_{\tilde{r}} =1$, then $ \| Z \cdot u\|_{\tilde{r}}$ is maximized when $u=r$.
% Hence $ Lip(T) \leq \| Z(\epsilon,r,\rho) \cdot r \|_r$.
% Since all of the radii polynomials are negative, then $ Z \cdot r < r$component wise, thus proving that $ \|Z \cdot r\|_r <1$ and
Hence $T:B_{\epsilon}(r,\rho) \to B_{\epsilon}(r,\rho)$ is a contraction with respect to the $\| \cdot \|_r$ norm.

% We have thereby proved that  $T:B_{\epsilon}(r,\rho) \to B_{\epsilon}(r,\rho)$ is a contraction mapping.
Since $B_\epsilon(r,\rho)$ with this norm is a complete metric space, by the Banach fixed point theorem $T$~has a unique fixed point $ \hat{x}_\epsilon \in B_\epsilon(r,\rho)$. 
Since $A^\dagger$ is injective,  it follows that $ \hat{x}_\epsilon$   is the unique point in $B_\epsilon(r,\rho)$ for which $ F(\hat{x}_\epsilon) =0$. 
\end{proof}

\begin{remark}\label{r:boundDT}
Under the assumptions in Theorem~\ref{thm:RadPoly},
essentially the same calculation as in the proof above
leads to the estimate
\[
  \| DT(x) y \|_r \leq \kappa \|y\|_r 
  \qquad \text{for all } y \in \R^2 \times \ell^K_0 , 
  \, x \in B_\epsilon(r,\rho),
\]
where $\kappa := \max_{i=1,2,3} (Z\cdot r)_i / r_i$.
\end{remark}


In Appendix \ref{sec:YBoundingFunctions} and Appendix \ref{sec:BoundingFunctions} we construct explicit upper bounds 
$Y(\epsilon)$ and $ Z(\epsilon,r,\rho)$, respectively.  
These functions are constructed such that their components are (multivariate) polynomials in $\epsilon$, $r$ and $ \rho$ with nonnegative coefficients, hence they are increasing in these variables. 
This construction enables us to make use of the uniform contraction principle. 

\begin{corollary}\label{cor:eps0}
Let 
%\change[J]{$0 <\epsilon_0 < \tfrac{5}{2} ( 4 + \sqrt{10})^{-1} $}
 $0 < \epsilon_0 < \tfrac{\sqrt{10}}{4} $ 
and fix some $r = (r_\alpha, r_\omega, r_c) \in \R^3_+$.  
Fix $\rho > 0$ such that $ \rho \geq C(\epsilon_0,  r)$, as given by Lemma~\ref{lem:Crho}.
% as in Proposition \ref{prop:DerivativeEndo}. 
%
Let $Y(\epsilon)$ and $Z(\epsilon,r,\rho)$ be the upper bounds as given in  Propositions~\ref{prop:Ydef} and~\ref{prop:Zdef}. 
Let the radii polynomials $P$ be defined by Equation~\eqref{eq:RadPolyDef}.


If each component of  $P(\epsilon_0, r,\rho)$ is negative, 
then for all $ 0 \leq \epsilon \leq \epsilon_0$ there exists a unique $ \hat{x}_\epsilon \in B_\epsilon(  r , \rho)$ such that $ F(\hat{x}_\epsilon) =0$.  
The solution $\hat{x}_\epsilon$ depends smoothly on $\epsilon$.
\end{corollary}
\begin{proof} 
	Let $0 \leq  \epsilon \leq \epsilon_0$ be arbitrary.
	Because $\rho \geq C(\epsilon_0, r) \geq C(\epsilon, r)$ by Lemma~\ref{lem:Crho},
	Theorem~\ref{thm:RadPoly} implies that it suffices to show that $ P(\epsilon, r ,\rho) <0$. 	
Since  the bounds 
$Y(\epsilon)$ and $ Z(\epsilon,r,\rho)$ are monotonically increasing in their arguments, it follows that $ P(\epsilon,r,\rho) \leq P(\epsilon_0,r,\rho) <0$.  
Continuous and smooth dependence on $\epsilon$ of the fixed point follows from the uniform contraction principle (see for example~\cite{ChowHale}). 
\end{proof}


Given the upper bounds $ Y(\epsilon)$ and $ Z( \epsilon ,r , \rho)$, 
trying to apply Corollary~\ref{cor:eps0} amounts to finding values of $ \epsilon, r_\alpha, r_\omega, r_c,\rho$ for which the radii polynomials are negative.
Selecting a value for $ \rho$ is straightforward: all estimates improve with smaller values of $\rho$, and Proposition \ref{prop:DerivativeEndo} (see also Lemma~\ref{lem:Crho}) explicitly describes the smallest allowable choice of $\rho$ in terms of $ \epsilon,r_\alpha,r_\omega,r_c$. 

Beyond selecting a value for $ \rho$, it is difficult to pinpoint what constitutes an ``optimal'' choice of these variables. 
In general it is interesting to find such  viable radii (i.e.\ radii such that $P(r)<0$) which are both large and small.  
The smaller radius tells us how close the true solution is to our approximate solution. 
The larger radius tells us in how large a neighborhood our solution is unique.  With regard to $\epsilon$, larger values allow us to describe functions whose first Fourier mode is large. However this will ``grow'' the smallest viable radius and ``shrink'' the largest viable radius. 

Proposition \ref{prop:bigboxes} presents two selections of variables which satisfy the hypothesis of Corollary~\ref{cor:eps0}.  
We check the hypothesis is indeed satisfied by using interval arithmetic.
All details are provided in the Mathematica file~\cite{mathematicafile}. 
While the specific numbers used may appear to be somewhat arbitrary (see also the discussion in Remark~\ref{r:largeradii})  they have been chosen to be used later in Theorem 
\ref{thm:WrightConjecture} and Theorem \ref{thm:UniqunessNbd}.  


%%%
%%%BY DOING SOME CHOICES THAT HAVE NO MOTIVATION AT THIS POINT, BUT THAT WILL TURN OUT TO BE USEFUL IN SECTION~\ref{s:global} WE PROVE THE FOLLOWING USING MATHEMATICA  FILES.\marginpar{todo}

\begin{proposition}
		\label{prop:bigboxes}
Fix the constants $ \epsilon_0$, $(r_\alpha, r_\omega,r_c)$  and $\rho$ according to one of the following choices:
% \begin{enumerate}
% \item[\textup{(a)}]  $ \epsilon_0 = 0.029 $ and $ (r_\alpha , r_ \omega , r_c) = (  0.21, \, 0.16 , \, 0.09 ) $ and $\rho = 1.01$;
% \item[\textup{(b)}]  $ \epsilon_0 = 0.087 $ and $ (r_\alpha , r_ \omega , r_c) = (  0.1501, \, 0.0626 , \, 0.2092 ) $ and $\rho = 0.5672$.
% \end{enumerate}
% \note[J]{Version with new numbers below}
\begin{enumerate}
	\item[\textup{(a)}]  $ \epsilon_0 = 0.029 $ and $ (r_\alpha , r_ \omega , r_c) = (  0.13, \, 0.17 , \, 0.17 ) $ and $\rho = 1.78$; 
	\item[\textup{(b)}]  $ \epsilon_0 = 0.09 $ and $ (r_\alpha , r_ \omega , r_c) = (  0.1753, \, 0.0941 , \, 0.3829 ) $ and $\rho = 1.5940$. 
\end{enumerate}
For either of the choices (a) and (b) we have the following: 
for all $0 \leq \epsilon \leq \epsilon_0$ there exists a unique point 
$(\hat{\alpha}_\epsilon,\hat{\omega}_\epsilon,\hat{c}_\epsilon) \in B_{\epsilon}(r,\rho)$ 
satisfying $F_\epsilon(\hat{\alpha}_\epsilon,\hat{\omega}_\epsilon,\hat{c}_\epsilon) = 0$ and 
\[ 	
 | \hat{\alpha}_\epsilon - \balpha_\epsilon| \leq r_\alpha , 
 \quad
 |\hat{\omega}_\epsilon - \bomega_\epsilon| \leq  r_\omega  ,
 \quad
 \| \hat{c}_\epsilon - \bc_\epsilon\| \leq r_c     ,
 \quad
 \| K^{-1} \hat{c}_\epsilon \| \leq \rho  .
\]
\end{proposition}
\begin{proof}
In the Mathematica file~\cite{mathematicafile}  we check, using interval arithmetic, that  $\rho \geq C(\epsilon_0, r)$ and  the radii polynomials $P(\epsilon_0,r,\rho)$ are negative for the choices (a) and (b). The result then follows from Corollary~\ref{cor:eps0}.	
\end{proof}


\begin{remark}\label{r:largeradii}	
In Proposition~\ref{prop:bigboxes} we aimed for large balls on which the solution is unique.
Even for a fixed value of $ \epsilon$, it is not immediately obvious how to find a ``largest'' viable radius $r$, 
since $r$ has three components. In particular, there is a trade-off between the different components of $r$. On the other hand, as explained in Remark~\ref{r:smallradii}, no such difficulty arises when looking for a ``smallest'' viable radius.
\end{remark}




We will also need a rescaled version of the radii polynomials, which takes into account the asymptotic behavior of the bound $Y$ on the residue $T(\bar{x}_\epsilon) -\bar{x}_\epsilon = - A^\dagger F(\bx_\epsilon)$  as $\epsilon \to 0$, namely it is of the form
$Y(\epsilon)= \epsilon^2 \tilde{Y}(\epsilon)$,
see Proposition~\ref{prop:Ydef}.
The proofs of the following monotonicity properties can be found in 
Appendices~\ref{sec:YBoundingFunctions} and~\ref{sec:BoundingFunctions}. 
\begin{lemma}\label{lem:YZ}
Let $\epsilon \geq 0$, $\rho >0$ 
and $r \in\R^3_+$. 
Then there are upper bounds
$Y(\epsilon) =\epsilon^2 \tilde{Y}(\epsilon)$ on $ T(\bx_\epsilon) - \bx_\epsilon$ and a (uniform) upper bound 
$Z(\epsilon , r ,\rho) $  on $ DT(x) $ for all $ x \in B_\epsilon(r,\rho)$.
These bounds are given explicitly by Propositions~\ref{prop:Ydef} and~\ref{prop:Zdef}, respectively. Moreover, $\tilde{Y}(\epsilon)$ is nondecreasing in $\epsilon$,
while $Z(\epsilon , r ,\rho) $ is nondecreasing in $\epsilon$, $r$ and $\rho$.
\end{lemma}

This implies, roughly speaking, that if we are able to show that $T$ is a contraction map on 
$B_{\epsilon_0}( \epsilon_0^2 \rr,\rho)$ for a particular choice of $ \epsilon_0$, then it will be a contraction map on $B_\epsilon( \epsilon^2 \rr,\rho)$ for all $ 0 \leq \epsilon \leq \epsilon_0$. Here, and in what follows, we use the notation $r = \epsilon^2 \rr$ for the $\epsilon$-scaled version of the radii. 



\begin{corollary}
	\label{cor:RPUniformEpsilon}
	Let  
	 $0 < \epsilon_0 < \tfrac{\sqrt{10}}{4} $ 
	and fix some $\rr = (\rr_\alpha, \rr_\omega, \rr_c) \in \R^3_+$. 
	Fix $\rho > 0$ such that $ \rho \geq C(\epsilon_0, \epsilon_0^2 \rr)$, as given by Lemma~\ref{lem:Crho}. 
	Let $Y(\epsilon)$ and $Z(\epsilon,r,\rho)$ be the upper bounds as given by Lemma~\ref{lem:YZ}.  
Let the radii polynomials $P$ be defined by~\eqref{eq:RadPolyDef}. 

	If each component of  $P(\epsilon_0,\epsilon_0^2 \rr,\rho)$ is negative, 
	then for all $ 0 \leq \epsilon \leq \epsilon_0$ 
	there exists a unique $ \hat{x}_\epsilon \in B_\epsilon(\epsilon^2  \rr , \rho)$ 
	such that $ F(\hat{x}_\epsilon) =0$. 
	Furthermore, $\hat{x}_\epsilon$ depends smoothly on $\epsilon$.
\end{corollary}

\begin{proof}
	 Let $0 \leq  \epsilon < \epsilon_0$ be arbitrary.
	 Because $\rho \geq C(\epsilon_0,\epsilon_0^2 \rr) \geq C(\epsilon,\epsilon^2 \rr)$ by Lemma~\ref{lem:Crho},
	Theorem~\ref{thm:RadPoly} implies that it suffices to show that $ P(\epsilon,\epsilon^2 \rr ,\rho) <0$. 
	By using the monotonicity provided by Lemma~\ref{lem:YZ}, we obtain
	\begin{alignat*}{1}
		P(\epsilon,\epsilon^2 \rr ,\rho) &= Y(\epsilon) 
- \left[ I - Z(\epsilon,\epsilon^2 \rr,\rho)\right] \cdot \epsilon^2 \rr \\
		&=  (\epsilon / \epsilon_0)^{2} \left[ \epsilon_0^2   
		  \tilde{Y}(\epsilon) - \epsilon_0^2 \rr 
		+  Z(\epsilon,\epsilon^2 \rr,\rho) \cdot \epsilon_0^2 \rr  \right] \\
		&\leq  (\epsilon / \epsilon_0)^{2} \left[ \epsilon_0^2  
		  \tilde{Y}(\epsilon_0)  - \epsilon_0^2 \rr 
   +  Z(\epsilon_0,\epsilon_0^2 \rr,\rho) \cdot \epsilon_0^2 \rr  \right] \\
		&= (\epsilon / \epsilon_0)^{ 2} P(\epsilon_0 , \epsilon_0^2 \rr,\rho) \\
		& < 0,
	\end{alignat*}
where inequalities are interpreted componentwise in $\R^3$, as usual.
\end{proof}




%%%%%
%%%%%		THIS IS THE OLD VERSION OF THE UNIFORM \EPSILON^2 THEOREM
%%%%%
%%%%%\begin{corollary}
%%%%%	\label{prop:RPUniformEpsilon}
%%%%%	Let $ 0 < \epsilon_0 < \tfrac{5}{2} ( 4 + \sqrt{10})^{-1}$ and fix some $r = (r_\alpha, r_\omega, r_c) \in \R^3_+$ and 
%%%%%	fix $ k \in \{ 0,1,2\}$.  
%%%%%	Fix $\rho > 0$ such that $ \rho \geq C(\epsilon_0, (\epsilon_0)^2 r)$, as given by Lemma~\ref{lem:Crho}. 
%%%%%	Let $Y(\epsilon)$ and $Z(\epsilon,r,\rho)$ be the upper bounds as given by~\ref{lem:YZ}.  
%%%%%	Let the radii polynomials $P$ be defined by~\eqref{eq:RadPolyDef}.
%%%%%	If each component of  $P(\epsilon_0,{\epsilon_0}^k r,\rho)$ is negative, 
%%%%%	then for all $ 0 \leq \epsilon \leq \epsilon_0$ there exists a unique $ \hat{x}_\epsilon \in B_\epsilon(\epsilon^k  r , \rho)$ such that $ F(\hat{x}_\epsilon) =0$. Furthermore, $\hat{x}_\epsilon$ depends smoothly on $\epsilon$.
%%%%%\end{corollary}
%%%%%
%%%%%\begin{proof}
%%%%%	Let $0 \leq  \epsilon < \epsilon_0$ be arbitrary.
%%%%%	Because $\rho \geq C(\epsilon_0,\epsilon_0^k r) \geq C(\epsilon_0,\epsilon_0^k r)$ by Lemma~\ref{lem:Crho},
%%%%%	Theorem~\ref{thm:RadPoly} implies that it suffices to show that $ P(\epsilon,\epsilon^k r ,\rho) <0$. 
%%%%%	By using the monotonicity provided by Lemma~\ref{lem:YZ}, we obtain
%%%%%	\begin{alignat*}{1}
%%%%%	P(\epsilon,\epsilon^k r ,\rho) &= Y(\epsilon) - \left[ I - Z(\epsilon,\epsilon^k r,\rho)\right] \cdot \epsilon^k r \\
%%%%%	&=  (\epsilon / \epsilon_0)^{k} \left[ \epsilon_0^k  \epsilon^{2-k}  \tilde{Y}(\epsilon) - \epsilon_0^k r +  Z(\epsilon,\epsilon^k r,\rho) \cdot \epsilon_0^k r  \right] \\
%%%%%	&\leq  (\epsilon / \epsilon_0)^{k} \left[ \epsilon_0^k  \epsilon_0^{2-k}  \tilde{Y}(\epsilon_0)  - \epsilon_0^k r +  Z(\epsilon_0,\epsilon_0^k r,\rho) \cdot \epsilon_0^k r  \right] \\
%%%%%	&= (\epsilon / \epsilon_0)^{ k} P(\epsilon_0 , \epsilon_0^k r,\rho) \\
%%%%%	& < 0,
%%%%%	\end{alignat*}
%%%%%	where inequalities are interpreted componentwise in $\R^3$, as usual.
%%%%%\end{proof}
%%%%%




%%%%%%%%%%%%%%%%%%%%%%%%%%%%%%%%%%%%%%%%%%%%%%%%%%%%%%%%%%%%%%%%%%%%%%%%%%%%
%\subsection{Application of Radii Polynomials}

%\begin{remark}


These $\epsilon$-rescaled variables are used in
Proposition~\ref{prop:TightEstimate} below to derive \emph{tight} bounds on the
solution (in particular, tight enough to conclude that the bifurcation is
supercritical). The following remark explains that the monotonicity properties of
the bounds $Y$ and $Z$ imply that looking for small(est) radii which satisfy $P(r)<0$, is
a well-defined problem.


\begin{remark}\label{r:smallradii}
The set $R$ of radii for which the radii polynomials are negative is given by 
\[
  R := \{ r \in \R^3_+ : r_j > 0,  P_i(r) < 0 \text{ for } i,j=1,2,3 \} .
\] 
This set has the property that if
	$r,r' \in R$, then $r''\in R$, where $r''_j=\min\{ r_j,r'_j\}$.
Namely, the main observation is that we can write 
	$P_i(r)= \tilde{P}_i(r)-r_i$, where $\partial_{r_j} \tilde{P}_i \geq 0$ for all $i,j=1,2,3$.
Now fix any $i$; we want to show that $P_i(r'') < 0$.	
We have either $r''_i=r_i$ or $r''_i=r'_i$, hence assume $r''_i=r_i$ (otherwise
just exchange the roles of $r$ and $r'$). We infer that $P_i(r'') \leq P_i(r) <
0$, since $\partial_{r_j} P_i \geq 0$ for $j \neq i$.
We conclude that there are no trade-offs in looking for minimal/tight radii, as
opposed to looking for large radii, see Remark~\ref{r:largeradii}.
\end{remark}

%%%
%%%The optimization problem is simplified to a degree because the region $ P(\epsilon_0,r,\rho_0) <0$ is convex for fixed $\epsilon_0$ and $ \rho_0 $.  
%%%This is because the function $Z(\epsilon,r,\rho)$ is constructed out of polynomials with non-negative coefficients, whereby $\tfrac{\partial}{ \partial r_i} \tfrac{\partial }{\partial r_j} P_{r_k}(\epsilon_0,r,\rho_0) >0$ for all $ i,j,k \in \{ \alpha, \omega,c\}$. 
%%%\marginpar{I believe this is true, right? -JJ}


\begin{proposition}
		\label{prop:TightEstimate}
	Fix $ \epsilon_0 = 0.10$ and 
%\change[J]{$ (\rr_\alpha , \rr_ \omega , \rr_c) = (  0.1149, \, 0.0470 , \, 0.4711 ) $}
$ (\rr_\alpha , \rr_ \omega , \rr_c) = (  0.0594, \, 0.0260 , \, 0.4929 ) $ 
and 
%\change[J]{$\rho = 0.0279$}
$\rho = 0.3191$. 
	For all $0< \epsilon \leq \epsilon_0$ there exists a unique point $\hat{x}_\epsilon = (\hat{\alpha}_\epsilon,\hat{\omega}_\epsilon,\hat{c}_\epsilon)$ 
	satisfying $F(\hat{x}_\epsilon) = 0$ and 
	\begin{align}
	\label{eq:TightBound}
 | \hat{\alpha}_\epsilon - \balpha_\epsilon| <& \rr_\alpha \epsilon^2 , 
 %
 &|\hat{\omega}_\epsilon - \bomega_\epsilon| <&  \rr_\omega \epsilon^2 ,
 %
 &
 \| \hat{c}_\epsilon - \bc_\epsilon\| <& \rr_c  \epsilon^2   ,
  %
  &
  \| K^{-1} \hat{c}_\epsilon \| <& \rho  .
	\end{align}
Furthermore, $\hat{\alpha}_\epsilon > \pp$ for $ 0 < \epsilon < \epsilon_0$.
\end{proposition}

\begin{proof}
	In the Mathematica file~\cite{mathematicafile}  we check, using interval arithmetic, that  $\rho \geq C(\epsilon_0, \epsilon_0^2 \rr)$ and  the radii polynomials $P(\epsilon_0,\epsilon_0^2 \rr,\rho)$ are negative.  
	%I DO NOT UNDERSTAND THE NEXT SENTENCE
 The inequalities in Equation~\eqref{eq:TightBound} follow from Corollary~\ref{cor:RPUniformEpsilon}. 
 Since $\hat{\alpha}_\epsilon \geq \balpha_\epsilon - \rr_\alpha \epsilon^2
 = \pp +\frac{1}{5}(\frac{3\pi}{2}-1)\epsilon^2 - \rr_\alpha \epsilon^2$ and $ \rr_\alpha < \tfrac{1}{5} ( \tfrac{3 \pi}{2} -1) $, it follows that $ \hat{\alpha}_\epsilon > \pp $ for all $ 0 < \epsilon \leq \epsilon_0$. 
%%
%%STILL NEEDS AN EXPLANATION
%%\marginpar{Jonathan: I am not sure what the argument is \dots}
%% WHY IT IS UNIQUE IN THE BALL GIVEN BY~\eqref{eq:TightBound}. CLEARLY IT IS UNIQUE IN $B_\epsilon(r,\rho)$
%%WITH $\rho= C( \epsilon_0,\epsilon_0^2 r)$. WHY CAN THERE BE NO SOLUTIONS WITH
%%$\| K^{-1} c \| > \rho$ SATISFYING~\eqref{eq:TightBound} ? 
\end{proof}

\begin{remark}\label{r:nested}
% The pivotal result in Proposition~\ref{prop:TightEstimate} is that $\hat{\alpha}_\epsilon > \pp$, which implies that the bifurcation is subcritical.
Since $\epsilon_0^2\rr < r$ for the choices (a) and (b) in Proposition~\ref{prop:bigboxes},
and the choices of $\rho$ and $\epsilon_0$ are compatible as well, the solutions found in Proposition~\ref{prop:bigboxes} are the same as those described by Proposition~\ref{prop:TightEstimate}. While the former proposition provides large isolation/uniqueness neighborhoods for the solutions,
the latter provides tight bounds and confirms the  supercriticality of the bifurcation suggested in Definition \ref{def:xepsilon}.

% The bifurcation is supercritical (see eg.  \cite{faria2006normal} p 252
	
	
	
%		We note that for each (appropriate) $\epsilon$, the ball 
%	$ B_{\epsilon}(r,\rho)$ from Proposition \ref{prop:TightEstimate} is contained within the balls   
%	$ B_{\epsilon}(r_a,\rho_a)$  and 
%	$ B_{\epsilon}(r_b,\rho_b)$ from Proposition \ref{prop:bigboxes}. 
%	This means that the fixed points $  \hat{x}_{\epsilon} \in B_{\epsilon}(r,\rho)$ is the same fixed point $\hat{x}_{\epsilon} \in B_{\epsilon}(r_a,\rho_a)$ .

\end{remark}


%
%The method of radii polynomials is versatile. 
%With the goal of later proving Corollary \ref{prop:UniqunessNbd}, we added additional constraints to \emph{Mathematica}'s function \emph{NMaximize} to find the parameters for Proposition \ref{prop:WideEstimate}.
%
%When searching for the largest viable radius we add an additional constraint. 
%In Proposition \ref{prop:Cone}, we showed that for a given selection of $ \epsilon$, $r_\alpha$ and $ r_\omega$, then the unscaled variable $ \|c\|$ is either $\cO(1)$ or $\cO(\epsilon^2)$. 
%When we scale $c \to \epsilon c$, then we are only able to prove uniqueness of our solution in an $ \epsilon-$cone about the approximate solution. 
%We use this Proposition to select $r_c = ????$ in terms of $\epsilon$, $r_\alpha$ and $r_\omega$ so that any unscaled  solution $c$ is either $\cO(1)$ or $ c \in B_{\epsilon}(r,\rho)$.
%
%Even still, the larger we choose $ \epsilon$, the smaller we will need to take $ r_\omega$ in order to have a proof. 
%For the following theorem, we fixed $ \epsilon_0 =0.085$ and used \emph{NMaximize} to find a choice of variables $(\epsilon,r_\alpha,r_\omega,r_c)$  which maximized the objective function $r_w$ and for which all the radii polynomials were negative. 
%By slightly shrinking the estimate for the optimal radii, we obtain the following theorem.




\section{Global results}
\label{s:global}
%!TEX root = hopfwright.tex

When deriving global results from the local results in
Section~\ref{s:local}, we need to take into account that there are some obvious
reasons why the branch of periodic solutions, described by
$F_\epsilon(\alpha,\omega,c)=0$, bifurcating from the Hopf bifurcation point at
$(\alpha,\omega)=(\pp,\pp)$ does not describe the entire set of periodic
solutions for $\alpha$ near $\pp$. First, there is the trivial solution. In
particular, one needs to quantify in what sense the trivial solution is an
isolated invariant set. This is taken care of by Remarks~\ref{r:smalleps}
and~\ref{r:cone}, which show there are no ``spurious'' small solutions in the
parameter regime of interest to us (roughly as long as we stay away from the
next Hopf bifurcation at $\alpha = \tfrac{5\pi}{2}$). Second, one can interpret any periodic
solution with frequency $\omega$ as a periodic solution with frequency
$\omega/N$ as well, for any $N \in \mathbb{N}$. Since we are working in Fourier space,
showing that there are no ``spurious'' solutions with lower frequency would
require us to perform an analysis near $(\alpha,\omega)=(\pp,\tfrac{\pi}{2N})$
for all $N \geq 2$. This obstacle can be avoided by bounding (from below)
$\omega$ away from $\pi/4$. This is done in Lemma~\ref{lem:omegalarge}.

For later use, we recall an elementary Fourier analysis bound. 
%\marginpar{There must be some theory about the best constant here?}
\begin{lemma}\label{lem:fourierbound}
	Let $y \in C^1$ be a periodic function of period $2\pi/\omega$ with Fourier coefficients $\c \in \ell^1_\sym$ (in particular this means $\c_0=0$), as described by~\eqref{eq:FourierEquation}. 
	Then 
\[
 \| \c \| \leq \sqrt{\frac{\pi }{6 \omega}}\,  \| y' \|_{L^2([0,2\pi/\omega])}
\qquad\text{and}\qquad
 \| \c \| \leq \frac{\pi}{\omega\sqrt{3}}\, \|y'\|_\infty.
 \]
\end{lemma}
\begin{proof}
From the Cauchy-Schwarz inequality and  Parseval's identity it follows that
\begin{alignat*}{1}
		\| \c \| &= 2 \sum_{k=1}^{\infty} |\c_k|
	%	 &=& 
	%	2 \sum_{k=1}^{\infty} | c_k| \\ 
	%	&=&	2 \sum_{k=1}^{\infty} k^{-1} \cdot | k \, c_k| \\
		\leq 2 \left( \sum_{k=1}^{\infty} k^{-2} \right)^{1/2}
		\left( \sum_{k=1}^{\infty} |k \, \c_k|^2 \right)^{1/2} \\
     &=  \frac{\sqrt{2}}{\omega} \left(\frac{\pi^2}{6} \right)^{1/2} 
	 	 \left(2 \sum_{k=1}^{\infty} |i \omega k \, \c_k|^2 \right)^{1/2}
		 = \frac{\pi}{\omega \sqrt{3}} 
		\left(\sum_{k \in \Z} |i \omega k \, \c_k|^2 \right)^{1/2}\\
		&= \frac{\pi}{\omega \sqrt{3}} 
		\left( \frac{\omega}{2\pi} \int_0^{2\pi/\omega} | y'(t)|^2 dt  \right)^{1/2} 
		\leq \frac{\pi}{\omega \sqrt{3}} \,  \|y'\|_\infty.
\end{alignat*}	
\end{proof}

% THIS NEEDS TO BE MENTIONED SOMEWHERE ELSE (IN EARLIER SECTION)
% Beside these more or less obvious obstacles, there is a more technical hurdle.
% The local analysis in Section~\ref{} gives us a unique solution branch $c=c(\epsilon)$ in some box $\| c - c(\epsilon) \| \leq r_c$. This translates to a unique solution curve $\tc=\tc(\epsilon)$ in some \emph{cone} $\|\tc\| \leq \epsilon r_c$. Lemma~\ref{} shows that if there are solutions outside this cone then $\| \tc \|$ must be large. In particular there are no solutions with small $\tc$ outside the cone (except for the trivial solution), provided $r_c > something$

\subsection{A proof of Wright's conjecture} 


Based on the work in \cite{neumaier2014global} and \cite{wright1955non}, in order to prove Wright's conjecture it suffices to prove that there are no slowly oscillating periodic solutions (SOPS) to Wright's equation for $ \alpha \in [1.5706,\pp]$. Moreover, in \cite{neumaier2014global} it was shown that no SOPS with $\| y \|_\infty \geq e^{0.04}-1$ exists for  $\alpha \in [1.5706,\pp]$. These results are summarized in the following proposition.

\begin{proposition}[\cite{neumaier2014global,wright1955non}]
\label{prop:neumaier}
Assume $y$ is a SOPS to Wright's equation for some $\alpha \leq \pp$. Then $\alpha \in [1.5706,\pp]$
and $\| y \|_\infty \leq e^{0.04}-1$. 
\end{proposition}
For convenience we introduce
\[
  \mu := e^{0.04}-1 \approx 0.0408.
\]
We now derive a lower bound on the frequency $\omega$ of the SOPS.
\begin{lemma}\label{lem:omegalarge}
Let $\alpha \in [1.5706,\pp]$.
Assume $y$ is a SOPS to Wright's equation with minimal period $2\pi/\omega$,
and assume that $\| y \|_\infty \leq \mu$.
Then $\omega \in [1.11,1.93]$.
\end{lemma}

\begin{proof}
%	\marginpar{JJ: I've written a new proof.}
	Without loss of generality, we assume in this proof that $ y(0) =0$, that $y(t) < 0$ for $t\in (-t_{-},0)$ and that $y(t) > 0$ for $t\in (0,t_+)$. 
	We will show that $t_-$ and $t_+$ are bounded by
	\begin{alignat*}{1}
	1+ \frac{1}{\alpha }  \frac{\log (1 + \mu)}{\mu}  < t_+ &<2 + \frac{1}{\alpha} , \\
	1+\frac{1}{\alpha} <t_- & < 3 .
	\end{alignat*}
	The lower bounds for both $t_-$ and $t_+$ follow directly from  Theorem 3.5 in \cite{jones1962nonlinear}. While Theorem 3.5 in~\cite{jones1962nonlinear} assumes $ \alpha \geq \pp$, this part of the theorem simply relies on Lemma 2.1 in \cite{jones1962nonlinear}, which only requires $ \alpha > e^{-1}$.
	

To obtain an upper bound on $t_+$, assume that $ t_+ \geq 2$. Set $t'_+ =\min\{t_+,3\}$. Then it follows from~\eqref{eq:Wright} that $y'(t) < 0$ for $t\in (1,t'_+]$, hence  $y(t-1) > y(2)$ for $ t \in [2,t'_+]$. We infer that for $t \in [2,t'+]$ we have 
$y'(t) = - \alpha y(t-1) [1+y(t)] < - \alpha y(2)$.
Solving the IVP $y'(t) < -\alpha y(2)$ with the initial condition $y(2) = y(2)$, we see that $y(t)$ hits $0$ before $t=2+\frac{1}{\alpha}$. Since $\alpha > 1$ (hence $2+\frac{1}{\alpha} < 3$), this implies that $t'_+=t_+$ and $t_+ < 2+\frac{1}{\alpha}$.
 
	

	To obtain the upper bound on $t_-$, assume for the sake of contradiction
	 that $ t_- \geq 3$. 
	 Then it follows from~\eqref{eq:Wright} that $y'(t) \geq 0$ for $t\in [-2,0]$, hence  $y(t) \leq y(-1)$ for $ t \in [-2,-1]$, and  $y'(t) \geq - \alpha y(-1) [1+y(t)]$ for $ t \in [-1,0]$.  
	Solving this IVP with the initial condition $y(-1) = -\nu$, we obtain $ y(t) \geq (1-\nu) e^{ \alpha \nu (t+1)}-1 $ for $ t \in [-1,0]$, and in particular  $y(0) \geq (1-\nu ) e^{-\alpha \nu}-1$. 
	By assumption $y(0)=0$ and $\nu=|y(-1)| \leq \mu$,
	but $  (1-\nu ) e^{-\alpha \nu}-1>0 $ 
	for $ \nu  \in (0,\mu]$
	and $\alpha \in [1.5706,\pp]$, a contradiction. Thereby $ t_- <3$. 


The bound on $\alpha$ implies that 
 the minimal period $L = t_+ + t_-$ of the SOPS must lie in $[ 3.26,5.64]$.
It then follows that $ \omega \in [1.11,1.93]$	
\end{proof}

%	Let $ t_+ $ denote the amount of time the SOPS $y(t)$ is positive and $ t_-$ denote the amount of time a SOPS is negative (measured over one minimal period). 
%
%First we show that $t_- <3$. 
%Theorem 3.4 in \cite{jones1962nonlinear} shows that if $t_- \geq 3$ then 
%$ \min y \geq  e^{-(\alpha-1)}-1$ which, for our range of $\alpha$, is within  $[0.43,0.44]$. 
%Since $\|y\|_\infty \leq \mu$, this contradiction shows that $ t_- < 3$. 


It turns out that this bound on $\omega$ can (and needs to be) sharpened.
This is the purpose of the following lemma, 
which considers solutions in  unscaled variables. 
\begin{lemma}\label{lem:ZeroNBD}
Suppose $ \tilde{F}_\epsilon(\alpha,\omega,\tc)=0$. If $\omega \in
[1.1,2]$ and $ \alpha \in [1.5,2.0]$  
%(QUITE ARBITRARY BOUNDS!)\marginpar{some illumination needed} 
then
\begin{equation}\label{e:tighterboundonomega}
   \frac{\sqrt{(\omega- \alpha)^2 + 2 \alpha \omega(1-\sin\omega)}}{2\alpha} 
   \leq 2 \epsilon + \| \tc \| .
\end{equation}
\end{lemma}
\begin{proof}
This follows from Proposition~\ref{prop:zeroneighborhood2} in
Appendix~\ref{appendix:aprioribounds}, combined with Proposition \ref{prop:G1Minimizer},  which shows that for
$\omega \in [1.1,2.0]$  and $ \alpha \in [1.5,2.0]$, the minimum in Equation~\eqref{e:minoverk} is attained for $k=1$.
%%\marginpar{Jonathan: need to say something about why this is true.}
\end{proof}

Next we derive bounds on $\epsilon$ and $\tc$, which also lead to improved bounds on $\omega$.
\begin{lemma}\label{lem:wrightbounds}
Let $\alpha \in [1.5706,\pp]$. Assume $y$ is a SOPS with $\| y \|_\infty \leq \mu$.
Then $y$ corresponds, through the Fourier representation~\eqref{e:yc}, to a zero of $F_\epsilon(\alpha,\omega,c)$ with $|\omega- \pp| \leq 0.1489$ and
\[
  0< \epsilon \leq \epseps := \mu/\sqrt{2} \leq 0.02886 ,
\] 
and 
%\change[J]{$\| c \| \leq 0.0398$}
$\| c \| \leq 0.0796$ 
and 
%\change[J]{$\| K^{-1} c \| \leq 0.08 $.}
$\| K^{-1} c \| \leq  0.16 $.
\end{lemma}
\begin{proof}
First consider the Fourier representation~\eqref{e:ytc} of $y$ in unscaled variables. 
Recall that $\c_0$ vanishes (see Remark~\ref{r:a0}).
Since $|y'(t)| \leq \alpha |y(t-1)| (1+|y(t)|) \leq \alpha \mu (1+\mu)$
we see from  Lemma~\ref{lem:fourierbound} that 
\begin{equation}\label{e:epsilontc} 
  2 \epsilon + \| \tc \|  \leq \frac{\pi}{\omega\sqrt{3}} \alpha \mu (1+\mu).
\end{equation}
Combining this with Lemma~\ref{lem:ZeroNBD} leads to the inequality
\begin{equation}
\label{eq:APomegaBound}
	\omega 	\sqrt{(\omega- \alpha)^2 + 2 \alpha \omega ( 1- \sin \omega)} 
	\leq 
	\tfrac{2 \pi}{ \sqrt{3}}  \alpha^2 \mu ( 1 + \mu).
\end{equation}
In the Mathematica file \cite{mathematicafile} we show that when $ \alpha \in [ 1.5706,\pp]$, then inequality \eqref{eq:APomegaBound} is violated for any $\omega \in [ 1.1,2.0] \, \backslash \, [1.4219, 1.6887]$.
From Lemma~\ref{lem:omegalarge} we obtain the a priori  bound
$\omega \in [1.11,1.93]$, whereby it follows that  $\omega \in [1.4219,1.6887]$, and in particular  $|\omega - \pp| \leq 0.1489$. 


Using this sharper bound on $\omega$ as well as $\alpha \in [1.5706,\pp]$
we conclude from~\eqref{e:epsilontc} that
\begin{equation}
	2 \epsilon + \| \tc \|  \leq \frac{\pi}{\omega\sqrt{3}} \alpha \mu (1+\mu)
	\leq \frac{2\omega - \alpha}{\alpha}.
	%  \ref{-- This equation is refered to in the Mathematica File --}
\end{equation} 
Since we also infer that $\alpha < 2\omega$,  Theorem~\ref{thm:FourierEquivalence3}(b) shows that the solution corresponds to a zero of $F_\epsilon(\alpha,\omega,c)$, with $\tc = \epsilon c $.
We can improve the bound on $\epsilon$ from~\eqref{e:epsilontc}
% \eqref{e:epsilontc} 
by observing that
\[
	( \epsilon^2 + \epsilon^2)^{1/2} 
	\leq \left( \sum_{k\in\Z} |\c_k|^2 \right)^{1/2}
	 =  \left( \frac{\omega}{2\pi} \int_0^{2\pi/\omega}
	                         |y|^2 dt  \right)^{1/2} \leq \mu .
\]
Hence $\epsilon \leq \epseps := \mu/\sqrt{2}$.

Finally, we derive the bounds on $c$. Namely, for $\alpha \in  [1.5706,\pp]$,
$\omega \in [1.4219,1.6887]$ and $\epsilon \leq \epseps$,
we find that $b_*$ and  $z_*^+$, as defined in~\eqref{e:zstar}, are bounded below by $ b_* \geq 0.364$ and 
$z^+_* \geq 0.72 $. 
Since it follows from~\eqref{e:epsilontc} that 
$ \| \tc \|  \leq 0.09 $  
in the same parameter range  of  $\alpha$ and $\omega$, we infer from Lemma~\ref{lem:Cone}(a) that 
$\| \tc \| \leq z^-_* $.
Via an interval arithmetic computation, the latter can be bounded above
using Lemma~\ref{lem:ZminusBound}, for $\alpha \in  [1.5706,\pp]$,
$\omega \in [1.4219,1.6887]$ and $\epsilon \leq \epseps$, by
%\change[J]{$z_*^- \leq 0.0398 \epsilon$. }
$z_*^- \leq 0.0796 \epsilon$. 
Hence 
%\change[J]{$\| c \| \leq z_*^- / \epsilon \leq 0.0398$.}
$\| c \| \leq z_*^- / \epsilon \leq 0.0796$.
Furthermore, Lemma~\ref{lem:Cone}(b) implies 
the  bound 
%\change[J]{$\| K^{-1} c \| \leq (\epsilon^2 + (z_*^-)^2 )/(\epsilon b_*) \leq 2.76 \epsilon$}
%\change[J]{$\| K^{-1} c \| \leq (\epsilon^2 + (z_*^-)^2 )/(\epsilon b_*) \leq 2.77 \epsilon$.}
$\| K^{-1} c \| \leq (2\epsilon^2 + (z_*^-)^2 )/(\epsilon b_*) \leq 5.52 \epsilon$.
Since $\epsilon \leq \epsilon_*$, it then follows that 
%\change[J]{$\| K^{-1} c \| \leq 0.08 $.}
$\| K^{-1} c \| \leq  0.16 $.
\end{proof}

With these tight bounds on the solutions, we are in a position to apply the local bifurcation result formulated in Proposition~\ref{prop:TightEstimate} to prove the ultimate step of Wright's conjecture.

\begin{theorem}
	\label{thm:WrightConjecture}
	For $ \alpha \in [0,\pp]$ there is no SOPS to Wright's equation.
\end{theorem}
\begin{proof}
By Proposition~\ref{prop:neumaier} (see also the introduction of this paper) it suffices to exclude a slowly oscillating solution $y$ for $\alpha \in [1.5706,\pp]$ with $\| y \|_\infty \leq \mu$.
By Lemma~\ref{lem:wrightbounds}, if such a solution would exist, it corresponds to a solution of 
$F_\epsilon(\alpha,\omega,c)=0$ with $|\omega- \pp| \leq 0.1489$, 
$0< \epsilon \leq \epseps = \mu / \sqrt{2}$, 
%\change[J]{$\| c \| \leq 0.0398 $}
$\| c \| \leq 0.0796 $ 
and 
%\change[J]{$\| K^{-1} c \| \leq 0.08 $.}
$\| K^{-1} c \| \leq  0.16 $.
We claim that no such solution exists.
Indeed, we define the set 
% \[
%   S :=  \{ (\alpha,\omega,c) \in X :
%   |\alpha - \pp| \leq 0.0002; \,
%    |\omega - \pp| \leq 0.1489; \,
%    \| c \| \leq 0.04; \,
%    \|K^{-1} c \| \leq 0.08  \}.
% \]
% \note[J]{Version with new $|c|$ below. Also propose simplifying $r_\omega$. }
% \[
% S :=  \{ (\alpha,\omega,c) \in X :
% |\alpha - \pp| \leq 0.0002; \,
% |\omega - \pp| \leq 0.15; \,
% \| c \| \leq 0.08; \,
% \|K^{-1} c \| \leq 0.08  \}.
% \]
% \note[J]{Proposed change}
\[
S :=  \{ (\alpha,\omega,c) \in X : 
|\alpha - \pp| \leq 0.0002; \,
|\omega - \pp| \leq 0.15; \, 
\| c \| \leq 0.08; \, 
\|K^{-1} c \| \leq 0.16  \}.
\]
To show that there is no SOPS for $\alpha \in [1.5706,\pp]$, it now suffices to show that all zeros of $F_\epsilon(\alpha,\omega,c)$ in $S$ for any $0< \epsilon \leq \epseps$ satisfy $\alpha> \pp$.

Let us consider
$B_\epsilon(r,\rho)$, which is centered at $\bx_\epsilon$ (see Definition~\ref{def:xepsilon})
with $r$ and $ \rho$ taken as in Proposition~\ref{prop:bigboxes}(a).
In the Mathematica file~\cite{mathematicafile} 
we check that the following inequalities are satisfied:
%\note[JB]{Are these checked in the Mathematica file? Perhaps refer to that?}
% \begin{alignat*}{1}
% 	r_\alpha &= 0.21 \geq  0.0002 + |\balpha_{\epseps}-\pp|,\\
%  r_\omega &= 0.16 \geq  0.1489 + |\bomega_{\epseps}-\pp| ,\\
%  r_c &= 0.09 \geq   0.04 + \| \bc_{\epseps}\|,\\
%  \rho &= 1.01 \geq 0.08 .
% \end{alignat*}
% \note[J]{Version with new numbers below.}
% \begin{alignat*}{1}
% r_\alpha &= 0.13 \geq  0.0002 + |\balpha_{\epseps}-\pp|,\\
% r_\omega &= 0.17 \geq  0.15 + |\bomega_{\epseps}-\pp| ,\\
% r_c &= 0.17 \geq   0.08 + \| \bc_{\epseps}\|,\\
% \rho &= 1.78 \geq 0.08 .
% \end{alignat*}
% \note[J]{Proposed change}
\begin{alignat*}{1}
r_\alpha &= 0.13 \geq  0.0002 + |\balpha_{\epseps}-\pp|,\\
r_\omega &= 0.17 \geq  0.15 + |\bomega_{\epseps}-\pp| ,\\
r_c &= 0.17 \geq   0.08 + \| \bc_{\epseps}\|,\\
\rho &= 1.78 \geq 0.16 . 
\end{alignat*}
By the triangle inequality we obtain that $S \subset B_\epsilon(r,\rho)$ for all $0<\epsilon\leq \epseps$.
Proposition~\ref{prop:bigboxes}(a) shows that for each $0<\epsilon\leq \epseps$
there is a unique zero $\hat{x}_\epsilon=
(\hat{\alpha}_\epsilon,\hat{\omega}_\epsilon,\hat{c}_\epsilon) \in B_\epsilon (r,\rho)$ of $F_\epsilon$.
By Proposition~\ref{prop:TightEstimate} and Remark~\ref{r:nested}
this zero satisfies $\hat{\alpha}_\epsilon > \pp$.
Hence, for any $0<\epsilon\leq\epseps$ the only zero of $F_\epsilon$ in $S$ (if there is one) satisfies $\alpha>\pp$. This completes the proof.

\end{proof}
% 
%\begin{eqnarray}
%	\|c \|_{\ell^1} &=& 2 \sum_{k=1}^{\infty} | c_k| \\ 
%	&=&	2 \sum_{k=1}^{\infty} k^{-1} \cdot | k \, c_k| \\
%	&\leq& 2 \left( \sum_{k=1}^{\infty} k^{-2} \right)^{1/2}
%	\left( \sum_{k=1}^{\infty} |k \, c_k|^2 \right)^{1/2} \\
%	&=&  \left(2 \frac{\pi^2}{6} \right)^{1/2}
%		\left(2 \omega^{-2} \sum_{k=1}^{\infty} |i \omega k \, c_k|^2 \right)^{1/2}\\
%	&=& \frac{\pi}{\omega \sqrt{3}} 
%	\left(\sum_{k=- \infty}^{\infty} |i \omega k \, c_k|^2 \right)^{1/2}\\
%	&=& \frac{\pi}{\omega \sqrt{3}} 
%		\left( \frac{1}{L} \int_0^L | y'(t)|^2 dt  \right)^{1/2}\\
%	&\leq& \frac{\pi}{\omega \sqrt{3}}  \max | y'(t)| 
%\end{eqnarray}
%Line 4.3 is obtained through Cauchy-Schwartz inequality. 
%Line 4.6 is obtained through Parseval's identity.
%

% END OF SUBSECTION
%
%
% With this bound on $\omega$ we can now use the local results from Section~\ref{s:local} to establish the ultimate step in the proof of Wright's conjecture.
%
% \begin{theorem}
% 	\label{prop:WrightConjecture}
% 	For $ \alpha \in [1.5706,\pp]$ there is no SOPS to Wright's equation.
% \end{theorem}
%
% \begin{proof}
% 	Suppose that $y$ is a slowly oscillating periodic solution to Wright's equation with frequency $ \omega >0$.
% 	By Theorem~\ref{thm:FourierEquivalence2}, there is some $\epsilon \geq 0$ and $ c \in \ell^1_0$ for which $y$ (after a time shift) can be written in the following form:
% 	\[
% 	y(t) =
% 	\epsilon \left( e^{i \omega t }  + e^{- i \omega t }\right)
% 	+  \sum_{k = 2}^\infty    \tc_k e^{i \omega k t }  + \bar{\tc}_k e^{- i \omega k t }
% 	\]
% 	By Theorem \ref{thm:FourierEquivalence2}, it suffices to show that $ \tilde{F}_\epsilon(\alpha , \omega, \tc)=0$ has no nontrivial solutions when  $ \alpha \in [1.5706,\pp]$.
% 	In Proposition \ref{prop:TightEstimate} it was proved that in a small $\epsilon$-scaled neighborhood about the bifurcation point, periodic solutions to Wright's equation only exist when $ \alpha > \pp$.
% 	To prove the theorem, we extend this neighborhood so that it  contains any solutions to $\tilde{F}(\alpha,\omega,c)=0$ that could exist for $ \alpha \in  [1.5706, \pp]$.
%
%
%
% 	We first collect global bounds on $ \epsilon$, $c$ and $ \omega $ for which a solution $ \tilde{F}_\epsilon (\alpha, \omega, c)=0$ could exist.
% 	By the results in \cite{neumaier2014global} we know that $\max |y(t)| \leq e^{0.04} -1$. Let us fix $ m := e^{0.04} -1$ and define  $ \hat{c} := \{ \epsilon , c_2 , c_3, \dots \} \in \ell^1 $.  By Parseval's  identity, we obtain the following inequality:
% 	\[
% 		( \epsilon^2 + \epsilon^2)^{1/2}
% 		\leq
% 		\| \hat{c} \|_{\ell^2}
% 		=
% 	\left( \frac{1}{L} \int_0^L |y|^2 dt  \right)^{1/2} \leq m
% 	\]
% 	Thereby we may define $ \epsilon_0 := m/\sqrt{2} \approx 0.0288$ and concern ourselves only with solutions $ \tilde{F}_\epsilon =0$ for which $ 0 \leq \epsilon \leq \epsilon_0$.
% 	Similarly, we can use Parseval's identity to bound $ \| \hat{c} \|_{\ell^1}$ as below:
% 	\begin{eqnarray}
% 	\| \hat{c} \|_{\ell^1}
% %	 &=&
% %	2 \sum_{k=1}^{\infty} | c_k| \\
% %	&=&	2 \sum_{k=1}^{\infty} k^{-1} \cdot | k \, c_k| \\
% 	&\leq& 2 \left( \sum_{k=1}^{\infty} k^{-2} \right)^{1/2}
% 	\left( \sum_{k=1}^{\infty} |k \, c_k|^2 \right)^{1/2} \\
% %	&=&  \left(2 \frac{\pi^2}{6} \right)^{1/2}
% %	\left(2 \omega^{-2} \sum_{k=1}^{\infty} |i \omega k \, c_k|^2 \right)^{1/2}\\
% 	&=& \frac{\pi}{\omega \sqrt{3}}
% 	\left(\sum_{k=- \infty}^{\infty} |i \omega k \, c_k|^2 \right)^{1/2}\\
% 	&=& \frac{\pi}{\omega \sqrt{3}}
% 	\left( \frac{1}{L} \int_0^L | y'(t)|^2 dt  \right)^{1/2} 		\label{eq:WrightConjectureEll1BoundL2}\\
% 	&\leq& \frac{\pi}{\omega \sqrt{3}}  \max | y'(t)| \\
% 	&\leq& \frac{\pi}{\omega \sqrt{3}}  \alpha m ( 1 + m)
% 	\label{eq:WrightConjectureEll1Bound}
% 	\end{eqnarray}
%
%
% 	In Proposition \ref{prop:ZeroNBD} we show if  $ \tilde{F}_\epsilon (\alpha, \omega, c)=0$  for $ \alpha \in [1.5706,\pp]$ then $\omega \in [1.11,1.93]$ and furthermore:
% 	\[
% 	\sqrt{(\omega- \alpha)^2 + 2 \alpha \omega ( 1- \sin \omega)} / (2 \alpha) \leq  \| \hat{c}\|_{\ell^1}
% 	\]
% 	Combining this with the estimate in Line \ref{eq:WrightConjectureEll1Bound}, we obtain the following inequality:
% 	\[
% 	\omega 	\sqrt{(\omega- \alpha)^2 + 2 \alpha \omega ( 1- \sin \omega)}
% 	\leq
% 	 \tfrac{2 \pi}{ \sqrt{3}}  \alpha^2 m ( 1 + m)
% 	\]
% 	Using interval arithmetic, we confirm in the supplemental \emph{Mathematica} file that the only value of $ \omega$ satisfying the above inequality are  $\omega \in [1.4219, 1.6887]$.
% 	% (also note $|\omega - \pp| < 0.1489 $).
%
%
% 	Next, we derive bounds on $c$ which scale with $ \epsilon$.
% 	From Proposition \ref{prop:Cone}, it follows that if $(\alpha,\omega,c)$ solves $ \tilde{F}_\epsilon=0$ in the range  $0 \leq \epsilon \leq \epsilon_0$ and  $\alpha \in [ 1.5706,\pp]$ and $ |\omega - \pp | < 0.1489$, then either $ \| c \| \geq 0.72$ or $ \| c \| \leq   0.0398 \epsilon$.
% 	The case $\| c\| > 0.72$ contradicts our estimate in Line \ref{eq:WrightConjectureEll1Bound}, therefore, it must be the case that $  \| c \| \leq 0.0398 \epsilon $.
% 	It then follows that if $ \epsilon =0$ then  the only solution to $ \tilde{F}_\epsilon(\alpha,\omega,c)=0$ is the trivial solution.
% 	An additional result from Proposition \ref{prop:Cone} is that any solution must satisfy $\| K^{-1} c \|_{\ell^1/\C} < 0.715 \epsilon^2$ for $ \epsilon \leq \epsilon_0$.
%
%
% 	In summary, we have shown that if there is a solution to
% 	$ \tilde{F}(\alpha , \omega, c)=0$ when  $ \alpha \in [1.5706,\pp]$, then
% 	$ 0< \epsilon \leq \epsilon_0 \approx 0.0288$ and $ |\omega - \pp| < 0.1489 $ and $ \| c \| \leq   0.0398 \epsilon$ and $\| K^{-1} c \|_{\ell^1/\C} < 0.715 \epsilon^2$.
% 		To prove that $F  \equiv 0$ has no solutions for $\alpha \in [1.5706,\pp]$, we construct balls $B_\epsilon(r,\rho)$ which both will contain this region described above, and for each $ 0<\epsilon \leq \epsilon_0$, the ball will contain a unique solution $ \tilde{F}_\epsilon(\alpha_\epsilon,\omega_\epsilon,c_\epsilon)$ for which $ \alpha_\epsilon > \pp$.
% 	To satisfy both these objectives, we fix the following constants:
% 	\begin{align}
% 	r_\alpha =&	0.21
% 	&
% 	r_\omega =&	0.16
% 	&
% 	r_c =& 	0.09
% 	&
% 	\rho =& 	1.01
% 	\end{align}
% 	If $ \tilde{F}(\alpha,\omega,c) =0$ at parameter $ 0 < \epsilon \leq \epsilon_0$, then one can apply the triangle inequality to show that the variables $ (\alpha, \omega,c)$ must satisfy the following inequalities:
% 	\begin{align}
% 	r_\alpha \geq&| \alpha -  \balpha_\epsilon | ,
% 	&
% 	r_\omega  \geq& | \omega - \omega(\epsilon) | ,
% 	&
% 	\epsilon  \, r_c \geq& \|c -  \epsilon \, c(\epsilon) \|_{\ell^1/\C} ,
% 	\end{align}
% 	where $\balpha_\epsilon$ and $\omega(\epsilon)$ and $c(\epsilon)$ are given in Equation \ref{eq:Approx}.
% 	Furthermore, if $ \tilde{F}(\alpha,\omega,c) =0 $ then $\| K^{-1} c \|_{\ell^1/\C} < 0.715 (\epsilon \times \epsilon_0  ) < 0.021 \epsilon < \rho \epsilon$.
% 	Hence the ball $B_\epsilon(r,\rho)$ contains all the solutions to $ F \equiv 0$  when $ \alpha \in [1.5706,\pp]$.
%
% 	At these values, all of the components of the radii polynomials $P(\epsilon_0,r,\rho)$ are negative.
% 	Furthermore $ \rho \geq C(\epsilon_0,r)$ as in Proposition \ref{prop:DerivativeEndo} and $ \epsilon_0 < \tfrac{5}{4}(2 + \sqrt{5})^{-1}$.
% 	By Corollary \ref{prop:RPUniformEpsilon} it follows that for all $ 0 < \epsilon \leq \epsilon_0$ there exists a unique point  $(\bar{\alpha}_\epsilon , \bar{\omega}_\epsilon , \bar{c}_\epsilon) \in B_\epsilon(r,\rho)$ for which  $F(\alpha,\omega, c) =0$.
% 	It follows from  Corollary \ref{prop:TightEstimate} that   $\bar{\alpha}_\epsilon > \pp$ for all $ 0 < \epsilon \leq 0.10$.
% 	Since
% 	$F(\alpha,\omega,\epsilon c) = \epsilon  \tilde{F}(\alpha , \omega, c)$, then $\tilde{F}(\alpha , \omega, c)=0$ has no solutions when  $ \alpha \in [1.5706,\pp]$.
%
%
%
%
% \end{proof}


%%%%%%%%%%%%%%%%%%%%%%%%%%%%%%%%%%%%%%%%%%%%%%%%%%%%%%%%%%%%%%%%%%%%%%%%%%%%


%%%%%%%%%%%%%%%%%%%%%%%%%%%%%%%%%%%%%%%%%%%%%%%%%%%%%%%%%%%%%%%%%%%%%%%%%%%%


%%%%%%%%%%%%%%%%%%%%%%%%%%%%%%%%%%%%%%%%%%%%%%%%%%%%%%%%%%%%%%%%%%%%%%%%%%%%


%
%
%As was formulated in the 2010 paper by Lessard, this can be broken up into two statements: that there are no folds in the branch of SOPS originating from the Hopf bifurcation, and that there are no isolas of periodic orbits. 
%In Proposition \ref{prop:WideEstimate} we  identified a box in $ \R^2 \oplus \ell^1 / \C$ inside which the only periodic orbits are those originating from the Hopf bifurcation.  
%In Proposition \ref{prop:TightEstimate} we  identified explicit error between our approximate solution and the true solution. 
%



\subsection{Towards Jones' conjecture}
\label{s:Jones}

Jones' conjecture states that for $ \alpha > \pp$ there exists a (globally) unique SOPS to Wright's equation.
 Theorem \ref{thm:RadPoly} shows that for a fixed small $\epsilon$ there is a (locally) unique $\alpha$ at which Wright's equation has a SOPS, represented by
 $(\hat{\alpha}_\epsilon,\hat{\omega}_\epsilon,\hat{c}_\epsilon)$. 
This is not sufficient to prove the local case of Jones conjecture.
To accomplish the latter, we show in Theorem \ref{thm:UniqunessNbd} 
 that near the bifurcation point there is, for each fixed $\alpha>\pp$, a (locally) unique SOPS to Wright's equation. 
We begin by showing that on the solution branch emanating from the Hopf bifurcation 
$\hat{\alpha}_\epsilon$ is monotonically increasing in~$\epsilon$,
i.e.\ $ \tfrac{d}{d \epsilon} \hat{\alpha}_{\epsilon} >0$.  
Since $\balpha_\epsilon = \pp + \tfrac{1}{5}(\tfrac{3 \pi}{2}-1) \epsilon^2  $,
we expect that $\tfrac{d}{d \epsilon} \hat{\alpha}(\epsilon) = \tfrac{2}{5}(\tfrac{3 \pi}{2}-1) \epsilon + \cO(\epsilon^2)$. 
For this reason it is essential that we calculate an approximation of $\tfrac{d}{d \epsilon} \hat{\alpha}_\epsilon$ which is accurate up to order $ \cO(\epsilon^2)$.   

\begin{theorem}
	\label{thm:NoFold}
For $0 < \epsilon \leq 0.1$ we have $ \tfrac{d}{d \epsilon} \hat{\alpha}_{\epsilon} >0$. 
For 
%\change[J]{$\pp  < \alpha \leq \pp + 6.2757  \times 10^{-3}$ }
$\pp  < \alpha \leq \pp + 6.830  \times 10^{-3}$ 
there are no  bifurcations in the branch of SOPS that originates from the Hopf bifurcation. 
\end{theorem}

%
%In Lessard 2010, it was shown that the branch of SOPS bifurcating from $\pp$ does not have any folds for $ \alpha \in [\pp + \epsilon_1, 2.3]$ where $\epsilon_1 = 7.3165 \times 10^{-4}$. 
%In the Floquet Multipliers paper in preparation, we show that there is a unique SOPS to Wright's equation for $ \alpha \in [1.94,6]$. 
%In Xie's 1991 thesis, he shows that there is a unique SOPS to Wright's equation for $ \alpha > 5.67$.  
%Thereby, we have effectively proved the conjecture in \cite{lessard2010recent}  that the branch of SOPS bifurcating from $\pp$ has no folds. 
%\newline 

\begin{proof}
We show that the branch of solutions  $ \hat{x}_\epsilon =  (\hat{\alpha}_\epsilon , \hat{\omega}_\epsilon , \hat{c}_\epsilon)$ obtained in Proposition \ref{prop:TightEstimate} satisfies $  \tfrac{d}{d \epsilon} \hat{\alpha}_\epsilon >0$ for $0<\epsilon \leq 0.1$.
This implies that the solution branch is (smoothly) parametrized by~$\alpha$,
i.e.,  there are no secondary nor any saddle-node bifurcations in this branch.
We then show that these $\epsilon$-values cover the range 
%\change[J]{$\pp  < \alpha \leq \pp + 6.2757  \times 10^{-3}$.}
$\pp  < \alpha \leq \pp + 6.830  \times 10^{-3}$.
	
We begin by
	differentiating the equation $ F( \hat{x}_\epsilon) =0$ with respect to $ \epsilon$:
%
% Proposition \ref{prop:TightEstimate} states that for $ \epsilon \leq 0.10$ there exists a locally unique solution $ \bar{x}_\epsilon =  (\bar{\alpha}_\epsilon , \bar{\omega}_\epsilon , \bar{c}_\epsilon)$ to $F(x)=0$. 
% We show that $ \tfrac{d}{d \epsilon} \bar{\alpha}_\epsilon >0$ by implicit differentiation, which implies that there are no subsequent bifurcations.  
\begin{equation}
 \frac{\partial F}{\partial  \epsilon}(\hat{x}_\epsilon) + D F( \hat{x}_\epsilon)  \frac{d }{d  \epsilon} \hat{x}_\epsilon  = 0 .
\end{equation}
In terms of the map $T$ we obtain the relation
\[
\left[I-DT(\hat{x}_\epsilon)  \right]  \frac{d }{d \epsilon} \hat{x}_\epsilon   
=- A^{\dagger} \frac{\partial F}{\partial  \epsilon}(\hat{x}_\epsilon)  .
\]

To isolate $\frac{d }{d \epsilon} \hat{x}_\epsilon   $, we wish to left-multiply each side of the above equation by $[I-DT(\hat{x}_\epsilon)]^{-1}$. 
To that end, we define an upper bound on $DT(\hat{x}_\epsilon)$ by  the matrix 
\begin{equation}\label{e:defZeps}
	\ZZ_\epsilon := Z(\epsilon,\epsilon^2 \rr, \rho) ,
\end{equation}
with $\rr$ and $\rho$ as in Proposition~\ref{prop:TightEstimate}.
We know from Remark~\ref{r:boundDT} 
%the proof of 
that with respect to the norm $\| \cdot \|_{\rr}$ on $\R^2 \times \ell^K_0$
\[
\| DT(\hat{x}_\epsilon)  \|_{\rr} \leq \max_{i=1,2,3} \frac{( \ZZ_\epsilon \cdot \rr)_i}{\rr_i} < 1, \qquad\text{for all } 0 \leq \epsilon \leq \epsilon_0, 
\]
with $\epsilon_0$ given in Proposition~\ref{prop:TightEstimate}. 
Hence $I-DT(\hat{x}_\epsilon) $ is invertible. In particular,
\begin{alignat*}{1}
\frac{d }{d \epsilon} \hat{x}_\epsilon   
& =- \left[I-DT(\hat{x}_\epsilon)  \right]^{-1}  A^{\dagger} \frac{\partial F}{\partial  \epsilon}(\hat{x}_\epsilon)  \\
& = - \left[I + \sum_{n=1}^\infty DT(\hat{x}_\epsilon)^n  \right]  A^{\dagger} \frac{\partial F}{\partial  \epsilon}(\hat{x}_\epsilon) .
\end{alignat*}
We  have an upper bound $\QQ_\epsilon \in \R^3_+$ on $A^{\dagger} \frac{\partial F}{\partial  \epsilon}(\hat{x}_\epsilon)$, as defined in Definition~\ref{def:upperbound}, given by Lemma~\ref{lem:Qeps}. 
We define $\II$ to be the $3 \times 3$ identity matrix.
For the $\alpha$-component we then obtain the estimate
% \begin{alignat}{1}
% \frac{d }{d \epsilon} \hat{\alpha}_\epsilon
% &\geq - \pi_\alpha  A^{\dagger} \frac{\partial F}{\partial  \epsilon}(\hat{x}_\epsilon)
% - \left( \sum_{n=1}^\infty \ZZ_\epsilon^n \QQ_\epsilon \right)_1 \nonumber \\
% & = - \pi_\alpha  A^{\dagger} \frac{\partial F}{\partial  \epsilon}(\hat{x}_\epsilon)  - \left( \ZZ_\epsilon (1-\ZZ_\epsilon)^{-1} \QQ_\epsilon \right)_1 . \label{e:alphaepsilon}
% \end{alignat}
% \note[J]{Proposed Change}
\begin{alignat}{1}
\frac{d }{d \epsilon} \hat{\alpha}_\epsilon  
&\geq - \pi_\alpha  A^{\dagger} \frac{\partial F}{\partial  \epsilon}(\hat{x}_\epsilon)  
- \left( \sum_{n=1}^\infty \ZZ_\epsilon^n \QQ_\epsilon \right)_1 \nonumber \\
& = - \pi_\alpha  A^{\dagger} \frac{\partial F}{\partial  \epsilon}(\hat{x}_\epsilon)  - \left( \ZZ_\epsilon (\II-\ZZ_\epsilon)^{-1} \QQ_\epsilon \right)_1 . \label{e:alphaepsilon}
\end{alignat}
%\annote[JB]{We note that the elements of the matrix $\ZZ_\epsilon$ are $\cO(\epsilon^2)$.}{Shall we just remove this sentence? It is cryptic (no argument is given) and anyway this is repeated and made precise in Eqn (4.10).}
We approximate $\frac{\partial F}{\partial  \epsilon}(\hat{x}_\epsilon)$ by 
\[
	\Gamma_\epsilon := \pp \tfrac{3i -1}{5} \epsilon \, \e_1 - i \pp \,  \e_2 - \pp \tfrac{3+i}{5} \epsilon \, \e_3 ,
\]
which is accurate up to quadratic terms in $\epsilon$.
In Lemma \ref{lem:ImplicitApprox} it is shown that
\begin{equation}\label{e:linearepsilon}
- \pi_\alpha A^{\dagger} \Gamma _\epsilon = \tfrac{2}{5} ( \tfrac{3 \pi}{2} -1) \epsilon.
\end{equation}
It remains to incorporate two explicit bounds for the remaining terms in~\eqref{e:alphaepsilon}. 
In Lemma~\ref{lem:Meps} we define $M_\epsilon$ and $M'_\epsilon$ that satisfy the following inequalities:
% \begin{alignat}{1}
% \left| \pi_\alpha A^{\dagger} \left( \tfrac{\partial F}{\partial  \epsilon}(\hat{x}_\epsilon) - \Gamma_\epsilon \right)  \right| &\leq
% \epsilon^2 M_\epsilon  , \label{e:boundM} \\
% %
% \left( \ZZ_\epsilon (1-\ZZ_\epsilon)^{-1} \QQ_\epsilon \right)_1 &\leq
%  \epsilon^2 M'_\epsilon . \label{e:boundMp}
% \end{alignat}
% \note[J]{Proposed Change}
\begin{alignat}{1}
\left| \pi_\alpha A^{\dagger} \left( \tfrac{\partial F}{\partial  \epsilon}(\hat{x}_\epsilon) - \Gamma_\epsilon \right)  \right| &\leq 
\epsilon^2 M_\epsilon  , \label{e:boundM} \\
%
\left( \ZZ_\epsilon (\II-\ZZ_\epsilon)^{-1} \QQ_\epsilon \right)_1 &\leq 
\epsilon^2 M'_\epsilon . \label{e:boundMp} 
\end{alignat}
Moreover, we infer from Lemma~\ref{lem:Meps} that $M_\epsilon$ and $M'_\epsilon$ are positive, increasing in $\epsilon$, and can be 
obtained explicitly by performing an interval arithmetic computation, using the explicit expressions for the matrix $\ZZ_\epsilon$ and the vector $\QQ_\epsilon$ given by Equation~\eqref{e:defZeps} and Lemma~\ref{lem:Qeps}, respectively (the expression for $Z(\epsilon,r,\rho)$ is provided in Appendix~\ref{sec:BoundingFunctions}).
 
Finally, we combine~\eqref{e:alphaepsilon},~\eqref{e:linearepsilon},~\eqref{e:boundM} and~\eqref{e:boundMp} to obtain
\[
 \frac{d }{d \epsilon} \hat{\alpha}_\epsilon  \geq 
 \tfrac{2}{5} ( \tfrac{3 \pi}{2} -1) \epsilon - \epsilon^2 ( M_{\epsilon} + M'_{\epsilon}).
\]
From the monotonicity of the bounds $M_\epsilon$ and $M'_\epsilon$ in terms of $\epsilon$, we infer that in order to conclude that  $\frac{d }{d \epsilon} \hat{\alpha}_\epsilon >0 $ for $0<\epsilon\leq\epsilon_0$  it suffices to check, using interval arithmetic, that
\begin{equation}
 \tfrac{2}{5} ( \tfrac{3 \pi}{2} -1) \epsilon_0 - \epsilon_0^2 (M_{\epsilon_0} + M'_{\epsilon_0})  > 0 . \label{e:Mepsilon0}
\end{equation} 
In the Mathematica file~\cite{mathematicafile} we check that~\eqref{e:Mepsilon0} is satisfied 
for $\epsilon_0 = 0.1$.
Since $\balpha_{\epsilon_0} \geq \pp + 7.4247\times 10^{-3}$,
and taking into account the control provided by Proposition~\ref{prop:TightEstimate} on the distance between $\hat{\alpha}_\epsilon$ and $\balpha_\epsilon$ in terms of $\rr_\alpha$, we
find that  
%\change[J]{$\hat{\alpha}_{\epsilon_0} \geq \balpha_{\epsilon_0} - \epsilon_0^2 \rr_\alpha \geq \pp + 6.2757  \times 10^{-3}$.}
 $\hat{\alpha}_{\epsilon_0} \geq \balpha_{\epsilon_0} - \epsilon_0^2 \rr_\alpha \geq \pp + 6.830  \times 10^{-3}$.
Hence  there can be no bifurcation on the solution branch for 
%\change[J]{$ \pp < \alpha \leq \pp + 6.2757   \times 10^{-3}$.}
 $ \pp < \alpha \leq \pp + 6.830   \times 10^{-3}$.
\end{proof}	



% Since $\ZZ_\epsilon$ is $\cO(\epsilon^2)$,
% NEXT DETERMINE $Q$ AND (SOME APPROXIMATION OF) $\pi_\alpha A^{\dagger} \frac{\partial F}{\partial  \epsilon}(\hat{x}_\epsilon)$.
% NOW I HAVE TO UNDERSTAND APPENDIX E, WHICH DOESN'T SEEM THAT EASY.
%
% We can calculate $\frac{d }{d  \epsilon} \bar{x}_\epsilon $ using the approximate inverse $A^{\dagger}$.
% \begin{eqnarray}
%  D F( x_\epsilon) \cdot \frac{\partial }{\partial  \epsilon} \bar{x}_\epsilon
% &=&
% - \frac{\partial F}{\partial  \epsilon}(\bar{x}_\epsilon) \\
%  %
%  \left[I -(I- A^{\dagger} D F( \bar{x}_\epsilon) )  \right] \cdot \frac{\partial }{\partial  \epsilon} \bar{x}_\epsilon
%  &=&
% - A^{\dagger} \frac{\partial F}{\partial  \epsilon}(\bar{x}_\epsilon) \\
% %
%  \left[I-DT(\bar{x}_\epsilon)  \right] \cdot \frac{\partial }{\partial  \epsilon} \bar{x}_\epsilon
% &=&
% - A^{\dagger} \frac{\partial F}{\partial  \epsilon}(\bar{x}_\epsilon)
% \end{eqnarray}
% As a result of the radii polynomial method being successful in Proposition \ref{prop:TightEstimate}, it follows that $\| DT(\bar{x}_\epsilon) \|_{r} <1$ for $ \epsilon < 0.10$, whereby $I - DT(\bar{x}_\epsilon)$ is invertible.
% To shorten our notation, we introduce the variable $ B:=DT(\bar{x}_\epsilon) $.
% We calculate further:
% \begin{eqnarray}
% \frac{\partial }{\partial  \epsilon} \bar{x}_\epsilon &=& - (I - B)^{-1} A^{\dagger} \frac{\partial F}{\partial  \epsilon}(\bar{x}_\epsilon)  \\
% &=& - \left( I + \sum_{k=1}^\infty B^k \right)   A^{\dagger} \frac{\partial F}{\partial  \epsilon}(\bar{x}_\epsilon)
% \end{eqnarray}
% Choosing $r$ as in Proposition \ref{prop:TightEstimate} gives us the bound $ \bar{x}_\epsilon \in B_{\epsilon}(r \epsilon^2 , \rho)$.
% Hence $\ZZ_\epsilon := Z(\epsilon ,\epsilon^2 r , \rho)$ is an upper bound for $ DT(\bar{x}_\epsilon)$.
% Suppose that we have  an upper bound	$\overline{A^{\dagger} \frac{\partial F}{\partial  \epsilon}(\bar{x}_\epsilon) } $ on the operator $A^{\dagger} \frac{\partial F}{\partial  \epsilon}(\bar{x}_\epsilon) $.
% Then we may proceed to obtain a lower bound on $ [\frac{\partial }{\partial  \epsilon} \bar{x}_\epsilon ]_{\alpha}$ as follows:
% \begin{eqnarray}
% \left[ \frac{\partial }{\partial  \epsilon} \bar{x}_\epsilon\right]_{\alpha}
% &=&
%  -  \left[ A^{\dagger} \frac{\partial F}{\partial  \epsilon}(\bar{x}_\epsilon)  \right]_{\alpha} -
% \left[ \left( \sum_{k=1}^\infty B^k \right)   A^{\dagger} \frac{\partial F}{\partial  \epsilon}(\bar{x}_\epsilon)   \right]_{\alpha}\\
% %
% \left[ \frac{\partial }{\partial  \epsilon} \bar{x}_\epsilon\right]_{\alpha}
% &\geq &
%    \left[ -A^{\dagger} \frac{\partial F}{\partial  \epsilon}(\bar{x}_\epsilon)  \right]_{\alpha} - \left[
% \left( \sum_{k=1}^\infty \ZZ_\epsilon ^k \right)
% \overline{  A^{\dagger} \frac{\partial F}{\partial  \epsilon}(\bar{x}_\epsilon) } \right]_{\alpha}    \\
% %
% &=&
%  \left[- A^{\dagger} \frac{\partial F}{\partial  \epsilon}(\bar{x}_\epsilon)  \right]_{\alpha} -
%  \left[
% 	  \ZZ_\epsilon( I -\ZZ_\epsilon)^{-1} \overline{A^{\dagger} \frac{\partial F}{\partial  \epsilon}(\bar{x}_\epsilon) }   \,
% \right]_{\alpha} \label{eq:NoFoldIneq}
% \end{eqnarray}
%
%
%
%
%
%
% 	In order to show that $ \tfrac{\partial}{\partial \epsilon} \bar{\alpha}_\epsilon >0$, it suffices to show that the RHS of Line \ref{eq:NoFoldIneq} is positive.
% 	Since $ \ZZ_\epsilon$ is of order $ \cO(\epsilon^2)$, then to obtain a $ \cO(\epsilon^2)$  approximation of $ \tfrac{\partial}{\partial \epsilon} \bar{\alpha}_\epsilon$, we define a $ \cO(\epsilon^2)$ approximation of $\tfrac{\partial}{\partial \epsilon}F(\bar{x}_\epsilon)$ as below:
% 	\begin{equation}
% 	\label{eq:GammaDef}
% 	\Gamma := \pp[\tfrac{3i -1}{5} \epsilon] e_1 - \pp [i] e_2 - \pp[\tfrac{3+i}{5} \epsilon] e_3
% 	\end{equation}
% 	In Proposition \ref{prop:ImplicitApprox} we show that $[-A^{\dagger} \Gamma ]_\alpha =\tfrac{2}{5} ( \tfrac{3 \pi}{2} -1) \epsilon$, which is what we expected the first order approximation of $ \alpha'(\epsilon)$ to be.
% 	Hence, showing that Line \ref{eq:NoFoldIneq} is positive is equivalent to proving the inequality below:
%
% 	\begin{equation}
% 	\tfrac{2}{5} ( \tfrac{3 \pi}{2} -1) \epsilon
% 	%
% 	>
% 	%
% 	 \left[- A^{\dagger} \left( \tfrac{\partial F}{\partial  \epsilon}(\bar{x}_\epsilon) - \Gamma \right) \right]_{\alpha} +
% 	 \left[
% 	 \ZZ_\epsilon( I -\ZZ_\epsilon)^{-1} \overline{A^{\dagger} \tfrac{\partial F}{\partial  \epsilon}(\bar{x}_\epsilon) }   \,
% 	 \right]_{\alpha}
% 	 \label{eq:ImplicitDiffEq2}
% 	\end{equation}
% %To facilitate computing  the RHS of Line \ref{eq:ImplicitDiffEq2},
% In Proposition \ref{prop:ImplicitLast}  we explicitly define an upper bound on the vector $
% 	A^{\dagger} \left( \tfrac{\partial F}{\partial  \epsilon}(\bar{x}_\epsilon) - \Gamma \right) $ which by definition is of order $\cO(\epsilon^2)$ and  we denote here as $C$.
% 	Using this bound, we can expand Line \ref{eq:ImplicitDiffEq2} as follows:
% 	\begin{equation}
% 	\tfrac{2}{5} ( \tfrac{3 \pi}{2} -1) \epsilon
% 	%
% 	\geq
% 	%
% 	\left[
% 	C
% 		 \right]_{\alpha}
% 	%
% 	+
% 	%
% 	\left[
% 	\ZZ_\epsilon( I -\ZZ_\epsilon)^{-1}  \left(A^{\dagger} \Gamma +
% 	C  \right) \,
% 	\right]_{\alpha}
% 	\label{eq:ImplicitInequality}
% 	\end{equation}
% 	Since $\ZZ_\epsilon$ and $C$ are both of order $\cO(\epsilon^2)$, then it follows that the RHS of Line \ref{eq:ImplicitInequality} is of order $ \cO(\epsilon^2)$.
% 	Hence, there exists some $\epsilon>0$ for which Inequality \ref{eq:ImplicitInequality} is satisfied, and furthermore, if some $\epsilon_0 >0$ satisfies Inequality \ref{eq:ImplicitInequality}, then the inequality is satisfied for all $0 < \epsilon \leq \epsilon_0$.
%
% 	Using interval arithmetic, we check that Inequality \ref{eq:ImplicitInequality} is satisfied for $\epsilon_0 = 0.10$. Whereby $ \alpha'(\epsilon) > 0$ for $ 0 < \epsilon \leq \epsilon_0$.
% 	Since $ \alpha(\epsilon_0)= \pp + 7.4247\times 10^{-3}$, then by accounting for our error in approximating $\alpha$ with our estimate from Proposition \ref{prop:TightEstimate}, there can be no subsequent bifurcations for $ \pp < \alpha < \pp + 6.3779 \times 10^{-3}$.
%
%


%%	Since we have computed an $ \cO(\epsilon^2)$ approximation, then for a fixed $ \rho$,  the components of  $ \ZZ_\epsilon$ and 
%%	$\overline{ 
%%		A^{\dagger} \left( \tfrac{\partial F}{\partial  \epsilon}(x_\epsilon) - \Gamma \right) }  $  are polynomials with non-negative coefficients, of order $\epsilon^2$ and higher. 
%%	It then follows that $ \epsilon^{-2}$ times the RHS of Equation \ref{eq:ImplicitInequality} is well defined at $ \epsilon = 0$ and non-decreasing in $\epsilon$. 




The above theorem provides the missing piece in the proof of Theorem~\ref{thm:IntroNoFold}.

\begin{corollary}\label{cor:collectreformulatedJones}
The branch of SOPS originating from the Hopf bifurcation at $\alpha = \pp$ has no folds or secondary bifurcations for any $\alpha > \pp$. 
\end{corollary}
\begin{proof}
	
	We prove the corollary by combining results on four overlapping subintervals of $ ( \pp, \infty)$. 
	In Theorem~\ref{thm:NoFold} we show that the (continuous) branch of SOPS originating from the Hopf bifurcation does not have any folds or secondary bifurcations for 
		$ \alpha \in ( \pp  , \pp + \delta_3] $ where 
%\change[J]{ $ \delta_3 = 6.2757  \times 10^{-3}$. 
 $ \delta_3 = 6.830  \times 10^{-3}$. 
	In~\cite{lessard2010recent} the same result is proved for $ \alpha \in [ \pp + \delta_1, 2.3]$, where $\delta_1 = 7.3165 \times 10^{-4}$. 
	In~\cite{jlm2016Floquet} 
	 it is shown that there is a unique SOPS  for $ \alpha $ in the interval $[1.94,6.00]$. 
	 Since $1.94 \leq 2.3$, then the SOPS in this interval belong to the branch originating from the Hopf bifurcation, and since they are unique for each $\alpha$, the branch is continuous and cannot have any folds or secondary bifurcations. 
	  In~\cite{xie1991thesis} it is shown that there is a unique SOPS for $ \alpha $ in the interval $ [5.67, +\infty)$, and by a similar argument the branch of SOPS cannot have any folds or secondary bifurcations in this interval either. 
	 Since 
	\[
	(\pp, \infty) = (\pp, \pp + \delta_3] \cup [ \pp + \delta_1,2.3] \cup [1.94,6.00] \cup [5.67, \infty) ,
	\]
	it follows that branch of SOPS originating from the Hopf bifurcation at $\alpha = \pp$ has no folds or secondary bifurcations for any $\alpha > \pp$. 
\end{proof}

To prove Jones' conjecture, it is insufficient to prove only locally that Wright's equation has a unique SOPS.  
We must be able to connect our local results with global estimates. 
When we make the change of variables $\tc = \epsilon c$ in defining the function $F_\epsilon$, we restrict ourselves to proving local results.  
Theorems~\ref{thm:UniqunessNbd} and~\ref{thm:UniqunessNbd2} connect these local results with a global argument, and construct neighborhoods, independent of any $ \epsilon$-scaling, within which the only SOPS to Wright's equation are those originating from the Hopf bifurcation.  

The next theorem uses the large radius calculation from Proposition \ref{prop:bigboxes}(b) to show that for  
%\change[J]{$\alpha \in ( \pp , \pp+ 4.75 \times 10^{-3} ]$}
$\alpha \in ( \pp , \pp+ 5.53 \times 10^{-3} ]$
all periodic solutions in a neighborhood of $0$ lie on the Hopf bifurcation curve, which has neither folds nor secondary bifurcations.  

% \note[J]{Almost every number in the Theorem and proof below has changed, so I did not always use track changes. Please check this. }

\begin{theorem}
	\label{thm:UniqunessNbd}
	For each $\alpha  \in  (\pp , \pp + 5.53 \times 10^{-3} ] $ there is a unique triple $ ( \epsilon, \omega, c)$ in the range 
%\change[J]{$ 0 < \epsilon \leq 0.087$}
$ 0 < \epsilon \leq 0.09$
	and 
%\change[J]{$ | \omega - \pp| < 0.061 $}
$ | \omega - \pp| < 0.0924 $
	and  
%\change[J]{$ \| c \| \leq 0.13135 $}
$ \| c \| \leq 0.30232 $ 
such that $ F_\epsilon(\alpha, \omega, c)=0$. 	
\end{theorem}
 
\begin{proof}
Fix $ \alpha \in ( \pp , \pp + 5.53 \times 10^{-3}]$ and let $F_\epsilon(\alpha, \omega, c)=0$ for some $\epsilon, \omega, c$ satisfying the assumed bounds. 
%Let $0 \leq \epsilon \leq \epsilon_0:=0.087$ and let $F_\epsilon(\alpha, \omega, c)=0$ for some $(\alpha, \omega, c)$ satisfying the assumed bounds.
From Lemma~\ref{lem:Cone}(b) it follows that 
%\change[J]{$\|K^{-1}c\| \leq \epsilon^2 (1+ \|c\|^2) /(\epsilon b_*) \leq 0.32$}
$\|K^{-1}c\| \leq \epsilon^2 (2+ \|c\|^2) /(\epsilon b_*) \leq 0.61$
for $\epsilon \leq \epsilon_0$, 
since $b_* \geq 0.31$. 
Hence the zeros under consideration all lie in the set 
% \[
% \tS :=  \{ (\alpha,\omega,c) \in X : | \alpha - \pp | \leq 0.00553 , |\omega - \pp| \leq 0.0924, \| c \| \leq 0.30232, \|K^{-1} c \| \leq 0.32  \}.
% \]
% \note[J]{Proposed change}
\[
\tS :=  \{ (\alpha,\omega,c) \in X : | \alpha - \pp | \leq 0.00553 , |\omega - \pp| \leq 0.0924, \| c \| \leq 0.30232, \|K^{-1} c \| \leq 0.61  \}.
\]
Proposition~\ref{prop:bigboxes}(b) shows that for each $0\leq\epsilon\leq 0.09$
there is a unique zero $\hat{x}_\epsilon=
(\hat{\alpha}_\epsilon,\hat{\omega}_\epsilon,\hat{c}_\epsilon) \in B_\epsilon(r,\rho)$ of $F_\epsilon$,
with $r=(r_\alpha,r_\omega,r_c) = (0.1753,0.0941,0.3829)$ and $\rho= 1.5940$.
For each $0 \leq \epsilon \leq 0.09$ it follows from the triangle inequality  that $\tS \subset B_\epsilon(r,\rho)$.  
This shows that $F_\epsilon$ has at most one zero in $\tS$ for each $ 0 \leq \epsilon \leq \epsilon_0$. 
By Remark~\ref{r:nested} this solution lies on the branch $\hat{x}_\epsilon$ originating from the Hopf bifurcation, in particular $\hat{x}_0=(\pp,\pp,0) \in \tS$.
Proposition~\ref{prop:TightEstimate} gives us tight bounds 
\[
|\hat{\omega}_\epsilon - \pp| \leq |\bomega_\epsilon - \pp|  + \rr_\omega \epsilon^2 \leq 0.0924
\qquad\text{and}\qquad \| \hat{c}_\epsilon \| \leq \| \bc_\epsilon\|  + \rr_c \epsilon^2 \leq 0.30232
\]
for all $0 \leq \epsilon \leq \epsilon_0$.  
Moreover, from similar considerations it follows that $\hat{\alpha}_{\epsilon_0} \geq  \balpha_{\epsilon_0} - r_\alpha \epsilon_0^2 > 0.00553$. Hence $\hat{x}_{\epsilon_0} \notin \tS$ and the solution curve leaves $\tS$ through $|\alpha- \pp| = 0.00553$ for some $0<\epsilon <\epsilon_0$.
Since $0.00553  < 6.830 \times 10^{-3}$ the assertion now follows directly from Theorem~\ref{thm:NoFold}.
\end{proof}
%

% OLD PROOF BELOW
%Fix $ \alpha \in ( \pp , \pp + 4.750 \times 10^{-3}]$ and let $F_\epsilon(\alpha, \omega, c)=0$ for some $\epsilon, \omega, c$ satisfying the assumed bounds. 
%%Let $0 \leq \epsilon \leq \epsilon_0:=0.087$ and let $F_\epsilon(\alpha, \omega, c)=0$ for some $(\alpha, \omega, c)$ satisfying the assumed bounds.
%From Lemma~\ref{lem:Cone}(b) it follows that $\|K^{-1}c\| \leq \epsilon^2 (1+ \|c\|^2) /(\epsilon b_*) \leq 0.27$ for $\epsilon \leq \epsilon_0$, 
%since $b_* \geq 0.33$. 
%Hence the zeros under consideration all lie in the set 
%\[
%  \tS :=  \{ (\alpha,\omega,c) \in X : | \alpha - \pp | \leq 0.00475 , |\omega - \pp| \leq 0.061, \| c \| \leq 0.13135, \|K^{-1} c \| \leq 0.27  \}.
%\]
%Proposition~\ref{prop:bigboxes}(b) shows that for each $0\leq\epsilon\leq 0.087$
%there is a unique zero $\hat{x}_\epsilon=
%(\hat{\alpha}_\epsilon,\hat{\omega}_\epsilon,\hat{c}_\epsilon) \in B_\epsilon(r,\rho)$ of $F_\epsilon$,
%with $r=(r_\alpha,r_\omega,r_c) = (0.1501,0.0626,0.2092)$ and $\rho= 0.5672$.
%For each $0 \leq \epsilon \leq 0.087$ it follows from the triangle inequality  that $\tS \subset B_\epsilon(r,\rho)$.  
%This shows that $F_\epsilon$ has at most one zero in $\tS$ for each $ 0 \leq \epsilon \leq \epsilon_0$. 
%By Remark~\ref{r:nested} this solution lies on the branch $\hat{x}_\epsilon$ originating from the Hopf bifurcation, in particular $\hat{x}_0=(\pp,\pp,0) \in \tS$.
%Proposition~\ref{prop:TightEstimate} gives us tight bounds 
%\[
%  |\hat{\omega}_\epsilon - \pp| \leq |\bomega_\epsilon - \pp|  + \rr_\omega \epsilon^2 \leq 0.061
%  \qquad\text{and}\qquad \| \hat{c}_\epsilon \| \leq \| \bc_\epsilon\|  + \rr_c \epsilon^2 \leq 0.657
%\]
%for all $0 \leq \epsilon \leq \epsilon_0$.  
%Moreover, from similar considerations it follows that $\hat{\alpha}_{\epsilon_0} \geq  \balpha_{\epsilon_0} - r_\alpha \epsilon_0^2 > 0.00475$. Hence $\hat{x}_{\epsilon_0} \notin \tS$ and the solution curve leaves $\tS$ through $|\alpha- \pp| = 0.00475$ for some $0<\epsilon <\epsilon_0$.
%Since $0.00475  < 6.2757 \times 10^{-3}$ the assertion now follows directly from Theorem~\ref{thm:NoFold}.
%\end{proof}

% OLDER PROOF BELOW
%
% \begin{proof}
%
%
% By Proposition \ref{prop:Cone}, if $\tilde{F}(\alpha,\omega,c)=0$ and $ |\alpha - \pp | \leq 0.0044$ and $| \omega - \pp| \leq 0.0811$, then either $ \| c\|_{\ell^1 / \C} < 0.121691 \times \epsilon$ or $ \| c\|_{\ell^1 / \C} > 0.657$.
% Hence, by our initial assumption we may take $ \| c\|_{\ell^1 / \C} < 0.121691 \times \epsilon$.
% Furthermore, Proposition \ref{prop:Cone} tells us that $ \| K^{-1} c \|_{\ell^1 / \C} <  (\epsilon^2+ \|c\|^2)/(2 b_*)$.
% Plugging in the appropriate values produces the estimate $ \| K^{-1} c \|_{\ell^1 / \C} <  \epsilon^2 \times 0.840672$.
% To apply the method of radii polynomials, we must  make the change of variables $ c \mapsto \epsilon c$.
% That is, if $ F( \alpha , \omega, c) =0$ then we may conclude that  $ \| c\|_{\ell^1 / \C} < 0.121691$ and  $ \| K^{-1} c \|_{\ell^1 / \C} <  \epsilon \times 0.840672$.
%
% To perform the method of radii polynomials on a specific ball, fix the following constants:
% \begin{align}
%  \epsilon_0 =& 0.080
% &
% r_\alpha =&	0.190814
% &
% r_\omega =&	0.0823824
% &
% r_c =& 	0.193644
% &
% \rho =& 	0.73103633
% \end{align}
% These constants are chosen so that the ball $ B_{\epsilon}( r , \rho)$ will contain any possible solutions to $ F( \alpha, \omega, c) =0$.
% That  is,  fix $ \alpha \in ( \pp, \pp + 0.0044]$, $ | \omega - \pp| < 0.0811$ and  $ \|c \|_{\ell^1 / \C} <   0.121691$.
% The constants $ r_\alpha$ and $ r_\omega$ and $ r_c$ were chosen that for all values of $ 0 < \epsilon \leq \epsilon_0$, then  an application of the triangle inequality relative to $\{ \balpha_\epsilon , \omega(\epsilon), c(\epsilon) \}$ implies the following inequalities:
% \begin{align}
% r_\alpha \geq& | \alpha -  \balpha_\epsilon |
% &
% r_\omega  \geq& | \omega - \omega(\epsilon) |
% &
% r_c \geq& \|c - c(\epsilon)\|
% \end{align}
% The value of $\rho$ was chosen to be the constant $ C(\epsilon,r)$ from Proposition \ref{prop:DerivativeEndo}.
% Since we earlier determined that $ \| K^{-1} c \|_{\ell^1 / \C} < \epsilon_0 \times 0.840672 < 0.068$ then it follows that  $ \| K^{-1} c \|_{\ell^1 / \C} \leq \rho$.
% Hence, if the variables $\epsilon$, $ \alpha$, $\omega,$ and $ c$ are chosen as per the assumptions of this theorem, the solutions $ F(\alpha,\omega,c)=0$ satisfy  $(\alpha,\omega,c) \in B_{\epsilon}(r,\rho)$.
%
%
% For these values of $ \epsilon_0, r, \rho$ all components of the radii polynomials $ P(\epsilon_0,r,\rho)$ are negative, as verified in the supplemental \emph{Mathematica} file.
% By Theorem \ref{prop:RPUniformEpsilon}, it follows that for all $ 0 < \epsilon \leq 0.08$, there exists a unique point  $ \bar{x}_\epsilon \in B_\epsilon(r,\rho)$ for which $ F( \bar{x}_\epsilon)=0$.
% Using our approximation defined in Equation \ref{eq:Approx} we calculate that $\alpha(0.08) = \pp + 0.00475$, so by applying our estimates from Proposition \ref{prop:TightEstimate} we may deduce that if $ \bar{\alpha}_\epsilon \in ( \pp , \pp+ 0.0044]$ then $\epsilon \in (0,0.08]$.
% By Theorem \ref{thm:NoFold}, we know that each  $\alpha \in ( \pp  , \pp +  6.3779  \times 10^{-3}] $  corresponds to a unique solution $ F(\bar{x}_\epsilon) =0$.
% In conclusion, if $ \alpha \in (\pp,\pp + 0.0044]$, then there exists a unique $ \epsilon \in ( 0,0.08]$ and $ | \omega - \pp | < 0.0811$ and $ \| c \|_{\ell^1 / \C} < 0.6766$ such that $ F( \alpha ,\omega, c)=0$ at parameter $\epsilon$.
%
%
%
%
%
%
% %then there are no folds in the branch of periodic orbits.
% %By Proposition \ref{prop:TightEstimate} we know that $| \bar{\alpha}_\epsilon - \alpha(\epsilon )| < 0.0411766 \epsilon^2$, so then for each $ \alpha \in (\pp , \pp + 0.0054]$ there exists a unique $0<\epsilon \leq 0.085$ for which $ \bar{\alpha}_\epsilon = \alpha$ and $ F(\bar{x}_\epsilon)=0$.
% %
% %
% %
%
%
% \end{proof}



Finally, we translate this result to function space.

\begin{theorem}
For each 
%	\change[J]{$\alpha  \in  (\pp , \pp + 0.00475] $ }
$\alpha  \in  (\pp , \pp + 5.53 \times 10^{-3} ] $ 
there is at most one (up to time translation) periodic solution to Wright's equation satisfying 
%\change[J]{$ \| y' \|_{L^2([0,2\pi/\omega])} \leq  0.295$}
$ \| y' \|_{L^2([0,2\pi/\omega])} \leq  0.302$
	and having frequency 
%\change[J]{$ | \omega - \pp | \leq 0.061$}
$ | \omega - \pp | \leq 0.0924$. 
	\label{thm:UniqunessNbd2}
\end{theorem}

\begin{proof}
We show that any periodic solution~$y$ to Wright's equation of period $2\pi/\omega$ that satisfies 
%\change[J]{$ \| y' \|_{L^2} \leq 0.295$ }
$ \| y' \|_{L^2} \leq 0.302$ 
has Fourier coefficients satisfying the bounds in Theorem~\ref{thm:UniqunessNbd}.
For the Fourier coefficients $a$ of $y$ we infer from Lemma~\ref{lem:fourierbound} that 
%\change[J]{$\| a \| \leq \sqrt{\frac{\pi}{6\omega}} \cdot 0.295 \leq 0.174 $.}
$\| a \| \leq \sqrt{\frac{\pi}{6\omega}} \cdot 0.302 \leq 0.18 $. 
Furthermore, for the parameter range of $\alpha$ and $\omega$ under consideration we conclude that $\alpha < 2\omega $ and 
$\|a\| < \frac{2\omega-\alpha}{\alpha}$. Hence we see from Theorem~\ref{thm:FourierEquivalence3}
that $y$ corresponds to a zero of $F_\epsilon$. 
The a priori bound on $\|a\|$ translates via~\eqref{e:aepsc} into the bounds
% \[
%   \epsilon \leq 0.087
%   \qquad\text{and}\qquad
%   \| \tc \| \leq 0.174 .
% \]
% \note[J]{New Version}
\[
\epsilon \leq 0.09
\qquad\text{and}\qquad
\| \tc \| \leq 0.18 .
\]
We now derive further bounds on $c=\tc/\epsilon$, 
 as in the proof of Lemma~\ref{lem:wrightbounds}.
Namely, for 
% \change[J]{$|\alpha-\pp| \leq 0.00475$,
% 	$|\omega-\pp| \leq 0.061$ and  $\epsilon \leq 0.087$,}
	$|\alpha-\pp| \leq 0.00553$,
	$|\omega-\pp| \leq 0.0924$ and  $\epsilon \leq 0.09$,
we find that  $z_*^+$, as defined in~\eqref{e:zstar}, is bounded below by 
%\change[J]{$z^+_* \geq 0.662$}
 $z^+_* \geq 0.595$. 
It follows that 
%\change[J]{$ \| \tc \|  \leq 0.174 \leq z^+_*$,}
$ \| \tc \|  \leq 0.18 \leq z^+_*$, 
so we infer from Lemma~\ref{lem:Cone}(a) that 
$\| \tc \| \leq z^-_* $.
Via Lemma~\ref{lem:ZminusBound} and an interval arithmetic computation, the latter can be bounded above, for 
% \change[J]{$|\alpha-\pp| \leq 0.00475$,
% 	$|\omega-\pp| \leq 0.061$ and  $\epsilon \leq 0.087$, by
% 	$z_*^- \leq 0.13135 \epsilon$. }
	 $|\alpha-\pp| \leq 0.00553$,
	$|\omega-\pp| \leq 0.0924$ and  $\epsilon \leq 0.09$, by
	$z_*^- \leq 0.30226 \epsilon$. 
Hence 
%\change[J]{$\| c \| \leq z_*^- / \epsilon \leq 0.13135$.}
 $\| c \| \leq z_*^- / \epsilon \leq 0.30232$.
%
We conclude that $y$ corresponds to a zero of $F_\epsilon(\alpha,\omega,c)$ in the parameter set described by Theorem~\ref{thm:UniqunessNbd}, which implies uniqueness.
\end{proof}

%%%
%%%
%%%Existence: I THINK EXISTENCE IS NOT GOING TO WORK FOR THE WHOLE RANGE OF $\alpha$. NAMELY $y \approx 2 \epsilon \cos (\pp t)$, hence  $y' \approx - 2 \pp \epsilon \sin (\pp t)$, and with $\omega \approx \pp$ this gives AT THE VERY BEST
%%%\[  
%%% \| y' \|_{L^2([0,2\pi/\omega])} \approx \sqrt{\int_0^4 \pi^2 \epsilon^2 \sin^2 (\pp t) dt } = \sqrt{2} \pi \epsilon .
%%%\]
%%%Since $\alpha-\pp \approx \frac{1}{5}(\frac{3\pi}{2}-1) \epsilon^2 = 0.74 \epsilon^2 $, we get
%%%$\| y' \|_{L^2} \approx 5.1 \sqrt{\alpha-\pp}$ AT THE VERY BEST (ON IN FACT WE WILL LOSE SOME IN THE ESTIMATE).
%%%
%%%WE NEED TO REFORMULATE THE THEOREM, PROBABLY WITH at most one INSTEAD OF a unique, BUT THAT DEPENDS ON WHAT JONATHAN NEEDS IN HIS OTHER PAPERS. 
%%%we still need to show that the unique solution described by Theorem~\ref{thm:UniqunessNbd} satisfies the bound $ \| y' \|_{L^2} \leq  0.295$.
%%%WE NEED TO ESTABLISH THAT THE solution given by Theorem~\ref{thm:UniqunessNbd} satisfies a tighter bound on $c$.
%%%We then do some estimate estimate.
%%%\[
%%%  \| y' \|_{L^2([0,2\pi/\omega])}^2 = \frac{4\pi}{\omega} \sum_{k=1}^\infty k^2 |a_k|^2 \leq  ???
%%%\] 




% OLD PROOF
%
% \begin{proof}
%
% 	If $ y$ is a periodic solution to Wright's equation, then by time translation, we may assume that $y$ can be written in the form:
% 	\[
% 	y(t) =
% 	\epsilon \left( e^{i \omega t }  + e^{- i \omega t }\right)
% 	+  \sum_{k = 2}^\infty    c_k e^{i \omega k t }  + \bar{c}_k e^{- i \omega k t }
% 	\]
% 	for some $ \epsilon  \geq 0$ and $ c \in \ell^1 / \C$.
% 	By Theorem \ref{thm:FourierEquivalence2} we know that $ y$ is a solution to Wright's equation at parameter $ \alpha$ if and only if $ \tilde{F}(\alpha ,\omega, c) =0$ at parameter  $\epsilon$.
%
% 	If we can show that $ \epsilon \in (0,0.08]$ and $ \| c \|_{ \ell^1 / \C} \leq 0.657$ then the hypothesis of Theorem~\ref{thm:UniqunessNbd} will be satisfied and we are done.
% 	By the calculation done in Line \ref{eq:WrightConjectureEll1BoundL2} it follows that  $( 2 \epsilon +  \|c\|_{\ell^1 / \C} ) < \tfrac{\pi}{\omega \sqrt{3}} \| y' \|_{L^2} $.
% 	For $|  \omega - \pp | < 0.0811$ and $ \| y'\|_{L^2} < 0.131$, it follows that $ ( 2 \epsilon +  \|c\|_{\ell^1 / \C} ) < 0.16$ whereby  $ \| c \|_{\ell^1 / \C} < 0.16 $ and  $\epsilon$ is in the range $ 0 \leq \epsilon \leq 0.08$.
% 	To show that $\epsilon$ is positive, we apply  Proposition \ref{prop:Cone}.
% 	This result tells us that if $0 \leq  \epsilon \leq 0.08$ and  $ |\alpha - \pp| < 0.0044$ and $ | \omega - \pp| < 0.0811$, then either $ \| c\|_{\ell^1 / \C} < 0.121691 \times \epsilon $ or $ \| c\|_{\ell^1 / \C} > 0.657$.
% 	Hence if $ \epsilon = 0$ and $\| c\|_{\ell^1 / \C} < 0.16$  then the only solution to $ \tilde{F}(\alpha,\omega,c)=0$ is the trivial solution, thus proving that $ \epsilon \in ( 0 , 0.08]$.
% 	Since the hypothesis of Theorem~\ref{thm:UniqunessNbd} is satisfied, then for each $ \alpha \in ( \pp, \pp + 0.0044]$ there exists a unique periodic orbit $y$ with frequency $ | \omega - \pp| < 0.0811$ and $\| y'\|_{L^2} < 0.131$.
%
%
%
%
%
% \end{proof}
%



\bibliographystyle{abbrv}
\bibliography{BibWright}

%%%%%%%%%%%%%%%%%%%%%%%%%%%%%%%%%%%%%%%%%%%%%%%%%%%%%%%%%%%%%%%%%%
%%%%%%%%%%%%%%%%%%%%%%%%  Appendices  %%%%%%%%%%%%%%%%%%%%%%%%%%%%
%%%%%%%%%%%%%%%%%%%%%%%%%%%%%%%%%%%%%%%%%%%%%%%%%%%%%%%%%%%%%%%%%%

\appendix

%\begin{appendices}

\section{Simple versions of the algorithms}
\label{appendix:simple_algs}

\begin{algorithm}
\caption{Online algorithm}\label{online_simple}
\begin{algorithmic}
\State Initialize \textsc{Student} learning algorithm
\State Initialize expected return $Q(a)=0$ for all $N$ tasks
\For{t=1,\ldots,T}
\State Choose task $a_t$ based on $|Q|$ using $\epsilon$-greedy or Boltzmann policy
\State Train \textsc{Student} using task $a_t$ and observe reward $r_t = x_t^{(a_t)} - x_{t'}^{(a_t)}$
\State Update expected return $Q(a_t) = \alpha r_t + (1 - \alpha) Q(a_t)$
\EndFor
\end{algorithmic}
\end{algorithm}

\begin{algorithm}
\caption{Naive algorithm}\label{naive_simple}
\begin{algorithmic}
\State Initialize \textsc{Student} learning algorithm
\State Initialize expected return $Q(a)=0$ for all $N$ tasks
\For{t=1,...,T}
\State Choose task $a_t$ based on $|Q|$ using $\epsilon$-greedy or Boltzmann policy
\State Reset $D=\emptyset$
\For{k=1,...,K}
\State Train \textsc{Student} using task $a_t$ and observe score $o_t = x_t^{(a_t)}$
\State Store score $o_t$ in list $D$
\EndFor
\State Apply linear regression to $D$ and extract the coefficient as $r_t$
\State Update expected return $Q(a_t) = \alpha r_t + (1 - \alpha) Q(a_t)$
\EndFor
\end{algorithmic}
\end{algorithm}

\begin{algorithm}
\caption{Window algorithm}\label{window_simple}
\begin{algorithmic}
\State Initialize \textsc{Student} learning algorithm
\State Initialize FIFO buffers $D(a)$ and $E(a)$ with length $K$ for all $N$ tasks
\State Initialize expected return $Q(a)=0$ for all $N$ tasks
\For{t=1,\ldots,T}
\State Choose task $a_t$ based on $|Q|$ using $\epsilon$-greedy or Boltzmann policy
\State Train \textsc{Student} using task $a_t$ and observe score $o_t = x_t^{(a_t)}$
\State Store score $o_t$ in $D(a_t)$ and timestep $t$ in $E(a_t)$
\State Use linear regression to predict $D(a_t)$ from $E(a_t)$ and use the coef. as $r_t$
%\State Update expected return $Q(a_t) := r_t$
\State Update expected return $Q(a_t) = \alpha r_t + (1 - \alpha) Q(a_t)$
\EndFor
\end{algorithmic}
\end{algorithm}

\begin{algorithm}
\caption{Sampling algorithm}\label{sampling_simple}
\begin{algorithmic}
\State Initialize \textsc{Student} learning algorithm
\State Initialize FIFO buffers $D(a)$ with length $K$ for all $N$ tasks
\For{t=1,\ldots,T}
\State Sample reward $\tilde{r}_a$ from $D(a)$ for each task (if $|D(a)|=0$ then $\tilde{r}_a=1$)
\State Choose task $a_t = \argmax_a |\tilde{r}_a|$
\State Train \textsc{Student} using task $a_t$ and observe reward $r_t = x_t^{(a_t)} - x_{t'}^{(a_t)}$
\State Store reward $r_t$ in $D(a_t)$
\EndFor
\end{algorithmic}
\end{algorithm}

\newpage
\section{Batch versions of the algorithms}
\label{appendix:batch_algs}

\begin{algorithm}
\caption{Online algorithm}\label{online_batch}
\begin{algorithmic}
\State Initialize \textsc{Student} learning algorithm
\State Initialize expected return $Q(a)=0$ for all $N$ tasks
\For{t=1,\ldots,T}
\State Create prob. dist. $\vec{a_t}=(p_t^{(1)}, ..., p_t^{(N)})$ based on $|Q|$ using $\epsilon$-greedy or Boltzmann policy
\State Train \textsc{Student} using prob. dist. $\vec{a_t}$ and observe scores $\vec{o_t} = (x_t^{(1)}, ..., x_t^{(N)})$
\State Calculate score changes $\vec{r_t} = \vec{o_t} - \vec{o_{t-1}}$
%\State Calculate score change $\hat{r}_t = o_t - o_{t-1}$
%\State Calculate corrected reward $r_t = \hat{r}_t / a_t$ ($a_t$ is prob. dist.)
\State Update expected return $\vec{Q} = \alpha \vec{r_t} + (1 - \alpha) \vec{Q}$
\EndFor
\end{algorithmic}
\end{algorithm}

\begin{algorithm}
\caption{Naive algorithm}\label{online_naive}
\begin{algorithmic}
\State Initialize \textsc{Student} learning algorithm
\State Initialize expected return $Q(a)=0$ for all $N$ tasks
\For{t=1,\ldots,T}
\State Create prob. dist. $\vec{a_t}=(p_t^{(1)}, ..., p_t^{(N)})$ based on $|Q|$ using $\epsilon$-greedy or Boltzmann policy
\State Reset $D(a)=\emptyset$ for all tasks
\For{k=1,\ldots,K}
\State Train \textsc{Student} using prob. dist. $\vec{a_t}$ and observe scores $\vec{o_t} = (x_t^{(1)}, ..., x_t^{(N)})$
\State Store score $o_t^{(a)}$ in list $D(a)$ for each task $a$
\EndFor
\State Apply linear regression to each $D(a)$ and extract the coefficients as vector $\vec{r_t}$
%\State Apply linear regression to each $D(a)$ and extract the coefficients as $\hat{r}_t$
%\State Calculate corrected rewards $r_t = \hat{r}_t / a_t$ ($a_t$ is prob. dist.)
\State Update expected return $\vec{Q} = \alpha \vec{r_t} + (1 - \alpha) \vec{Q}$
\EndFor
\end{algorithmic}
\end{algorithm}

\begin{algorithm}
\caption{Window algorithm}\label{online_window}
\begin{algorithmic}
\State Initialize \textsc{Student} learning algorithm
\State Initialize FIFO buffers $D(a)$ with length $K$ for all $N$ tasks
\State Initialize expected return $Q(a)=0$ for all $N$ tasks
\For{t=1,\ldots,T}
\State Create prob. dist. $\vec{a_t}=(p_t^{(1)}, ..., p_t^{(N)})$ based on $|Q|$ using $\epsilon$-greedy or Boltzmann policy
\State Train \textsc{Student} using prob. dist. $\vec{a_t}$ and observe scores $\vec{o_t} = (x_t^{(1)}, ..., x_t^{(N)})$
\State Store score $o_t^{(a)}$ in $D(a)$ for all tasks $a$
\State Apply linear regression to each $D(a)$ and extract the coefficients as vector $\vec{r_t}$
%\State Apply linear regression to each $D(a)$ and extract the coefficients as $\hat{r}_t$
%\State Calculate corrected rewards $r_t = \hat{r}_t / a_t$ ($a_t$ is prob. dist.)
\State Update expected return $\vec{Q} = \alpha \vec{r_t} + (1 - \alpha) \vec{Q}$
%\State Update expected return $Q = r_t$
\EndFor
\end{algorithmic}
\end{algorithm}

\begin{algorithm}
\caption{Sampling algorithm}\label{online_sampling}
\begin{algorithmic}
\State Initialize \textsc{Student} learning algorithm
\State Initialize FIFO buffers $D(a)$ with length $K$ for all $N$ tasks
\For{t=1,\ldots,T}
\State Sample reward $\tilde{r}_a$ from $D(a)$ for each task (if $|D(a)|=0$ then $\tilde{r}_a=1$)
\State Create one-hot prob. dist. $\vec{\tilde{a}_t}=(p_t^{(1)}, ..., p_t^{(N)})$ based on $\argmax\nolimits_a |\tilde{r}_a|$
\State Mix in uniform dist. : $\vec{a_t} = (1 - \epsilon) \vec{\tilde{a}_t} + \epsilon/N$
\State Train \textsc{Student} using prob. dist. $\vec{a_t}$ and observe scores $\vec{o_t} = (x_t^{(1)}, ..., x_t^{(N)})$
\State Calculate score changes $\vec{r_t} = \vec{o_t} - \vec{o_{t-1}}$
%\State Calculate score change $\hat{r}_t = o_t - o_{t-1}$
%\State Calculate corrected rewards $r_t = \hat{r}_t / a_t$ ($a_t$ is prob. dist.)
\State Store reward $r_t^{(a)}$ in $D(a)$ for each task $a$
\EndFor
\end{algorithmic}
\end{algorithm}

\clearpage
\section{Decimal Number Addition Training Details}
\label{appendix:addition}

Our reimplementation of decimal addition is based on Keras \citep{chollet2015keras}. The encoder and decoder are both LSTMs with 128 units. In contrast to the original implementation, the hidden state is not passed from encoder to decoder, instead the last output of the encoder is provided to all inputs of the decoder. One curriculum training step consists of training on 40,960 samples. Validation set consists of 4,096 samples and 4,096 is also the batch size. Adam optimizer \citep{kingma2014adam} is used for training with default learning rate of 0.001. Both input and output are padded to a fixed size.

In the experiments we used the number of steps until 99\% validation set accuracy is reached as a comparison metric. The exploration coefficient $\epsilon$ was fixed to 0.1, the temperature $\tau$ was fixed to 0.0004, the learning rate $\alpha$ was 0.1, and the window size $K$ was 10 in all experiments.
 
\section{Minecraft Training Details}
\label{appendix:minecraft}

The Minecraft task consisted of navigating through randomly generated mazes. The maze ends with a target block and the agent gets 1,000 points by touching it. Each move costs -0.1 and dying in lava or getting a timeout yields -1,000 points. Timeout is 30 seconds (1,500 steps) in the first task and 45 seconds (2,250 steps) in the subsequent tasks.

For learning we used the \textit{proximal policy optimization} (PPO) algorithm \citep{schulman2017proximal} implemented using Keras \citep{chollet2015keras} and optimized for real-time environments. The policy network used four convolutional layers and one LSTM layer. Input to the network was $40\times 30$ color image and outputs were two Gaussian actions: move forward/backward and turn left/right. In addition the policy network had state value output, which was used as the baseline. Figure \ref{f14} shows the network architecture.

\begin{figure}[h]
  \includegraphics[scale=0.4]{figures/minecraft_network}
\caption{Network architecture used for Minecraft.}
\label{f14}
\end{figure}

For training we used a setup with 10 parallel Minecraft instances. The agent code was separated into runners, that interact with the environment, and a trainer, that performs batch training on GPU, similar to \cite{babaeizadeh2016reinforcement}. Runners regularly update their snapshot of the current policy weights, but they only perform prediction (forward pass), never training. After a fixed number of steps they use FIFO buffers to send collected states, actions and rewards to the trainer. Trainer collects those experiences from all runners, assembles them into batches and performs training. FIFO buffers shield the runners and the trainer from occasional hiccups. This also means that the trainer is not completely on-policy, but this problem is handled by the importance sampling in PPO.

\begin{figure}[h]
  \includegraphics[scale=0.4]{figures/minecraft_training}
\caption{Training scheme used for Minecraft.}
\label{f14}
\end{figure}

During training we also used frame skipping, i.e. processed only every 5th frame. This sped up the learning considerably and the resulting policy also worked without frame skip. Also, we used auxiliary loss for predicting the depth as suggested in \citep{mirowski2016learning}. Surprisingly this resulted only in minor improvements.

For automatic curriculum learning we only implemented the Window algorithm for the Minecraft task, because other algorithms rely on score change, which is not straightforward to calculate for parallel training scheme. Window size was defined in timesteps and fixed to 10,000 in the experiments, exploration rate was set to 0.1.

The idea of the first task in the curriculum was to make the agent associate the target with a reward. In practice this task proved to be too simple - the agent could achieve almost the same reward by doing backwards circles in the room. For this reason we added penalty for moving backwards to the policy loss function. This fixed the problem in most cases, but we occasionally still had to discard some unsuccessful runs. Results only reflect the successful runs.

We also had some preliminary success combining continuous (Gaussian) actions with binary (Bernoulli) actions for "jump" and "use" controls, as shown on figure \ref{f14}. This allowed the agent to learn to cope also with rooms that involve doors, switches or jumping obstacles, see \url{https://youtu.be/e1oKiPlAv74}.

\end{appendices}

\section[]{The effects of other parameters}

We discussed the effect of the bulge and disk masses on the
development of bars and spiral arms in the main text. Here we briefly
summarize the effects of the other parameters we investigated.


\subsection{The halo spin}

The spin of the halo is known to be an important parameter that 
affects the bar's secular evolution. 
\citet{2014ApJ...783L..18L} showed that a co-rotating disk and halo 
speed up the bar formation, but decrease its final length. This 
is due to the angular momentum transfer between the disk and halo.
If the halo does not spin it absorbs the bar's angular momentum, 
which slows down the bar and increases its length. 
A co-rotating halo, however, returns angular momentum to the disk instead of 
just absorbing it. 
This stabilizes the angular momentum transfer, and the bar evolution ceases.

We setup a few models, based on model md1mb1, but now with a rotating halo. 
In order to give spin to the halo we change the sign of the angular momentum $z$ component, $L_{\rm z}$.
For models md1mb1s0.65 and md1mb1s0.8, 65 and 85\,\% of the halo particles are rotating in the same 
direction as the disk. For models without rotation, this value is 50\,\%. 

To compare our results with previous studies, we measure the spin 
parameter~\citet{1969ApJ...155..393P,1971A&A....11..377P}:
\begin{eqnarray}
\lambda = \frac{J|E|^{1/2}}{GM_{\rm h}^{5/2}},
\end{eqnarray}
where $J$ is the magnitude of the angular momentum vector, and $E$ is the total 
energy.
In our models, $\alpha_{\rm h}=0.65$ (0.8) correspond 
to $\lambda\sim0.03$ (0.06).


In Fig.~\ref{fig:snapshots_spin_b} we present the effect that the halo spin
has on models md1mb1s0.65 and md1mb1s0.8. 
The results indicate that  
the bar is shorter for the models with a stronger halo spin.

In Fig.~\ref{fig:A2_max_spin} we show the length and maximum amplitude of
the resulting bars.
These results are consistent with previous results which show that
the length of the bar and its amplitude decay when the halo spin increases.
However, in contrast to~\citet{2013MNRAS.434.1287S} and \citet{2014ApJ...783L..18L} ,
we find that the epoch of bar formation in our models is similar, 
whereas a faster formation was expected based on the larger halo spin. 
In order to rule out the effect of run-to-run variations~\citep{2009MNRAS.398.1279S},
we performed four additional simulations for each of models md1mb1, md1mb1s0.65 and md1mb1s0.8.
For the bar formation epochs we calculated the average and standard deviation. 

The average bar formation-epoch is $0.674 \pm 0.053$,
$0.691\pm 0.083$, and $0.610\pm 0.069$\,Gyr for models md1mb1, md1mb1s0.65 and md1mb1s0.8, respectively.
This may be caused by the relatively early bar formation (within $\sim0.8$\,Gyr)
compared to the previous
studies; 1--2\,Gyr for \citet{2014ApJ...783L..18L} and 3--4\,Gyr for
\citet{2013MNRAS.434.1287S}.
Indeed, in~\citet{2014ApJ...783L..18L} the bar formation epoch starts slightly earlier when 
a moderate spin parameter ($\lambda=0.045$ and 0.06) is introduced. The
dependence of the bar formation-epoch on the halo spin is even clearer in
\citet{2013MNRAS.434.1287S}, where the formation time is longer
than in~\citet{2014ApJ...783L..18L}.
We therefore argue that the rapid bar formation in our models may hide the sequential delay
of the bar formation as caused by the halo spin.

%
\begin{figure}
\begin{center}
  \includegraphics[width=40mm]{figures/md1_mb1_a0.65_100M_1024c.pdf}
  \includegraphics[width=40mm]{figures/md1_mb1_a0.8_100M_1024c.pdf}\\
    \caption{Snapshots for models md1mb1s0.65 (left) and md1mb1s0.8 (right), which are the same as model md1mb1 (Fig.~\ref{fig:snapshots_10Gyr},far most right panel) but now including halo spin.}\label{fig:snapshots_spin_b}
\end{center}
\end{figure}



\begin{figure*}
\includegraphics[width=\columnwidth]{figures/mode_A2max_spin.pdf}\includegraphics[width=\columnwidth]{figures/mode_Dbar_mb1s.pdf}
\caption{Time evolution of the maximum amplitude for $m=2$ (left) and the bar length (averaged for every $\sim 0.1$\,Gyr) for models md1mb1, md1mb1s0.65, and md1mb1s0.8. For each model, we performed four simulations changing the random seed (varying positions and velocities of the particles) when generating the initial realizations.
\label{fig:A2_max_spin}}
\end{figure*}

In addition to the bar forming models above, we also added halo spin to a model that 
shows no bar formation within 10\, Gyr. This model, md0.5Rd1.5s, is based on md0.5Rd1.5
but now with a halo spin of 0.8. 
In Fig.~\ref{fig:snapshots_spin_sp}, we present the snapshots of the above models at $t=10$\,Gyr. 
In contrast to the barred galaxies, their spiral structures look quite similar. 
To quantitatively compare the spiral amplitudes we use the total amplitude of the spiral arms 
given by $\sum ^{10}_{m=1} |A_m|^2$, where $A_m$ is the Fourier amplitude (Eq.~\ref{eq:Fourier}).
Instead of the bar amplitude, we measured the spirals total amplitude at 
$2.2R_{\rm d}$ and at $4.5R_{\rm d}$ (for this model 9.5 and 19.5\,kpc, respectively), 
the results are shown in Fig.~\ref{fig:mode_spin_sp}. 
The evolution of the spiral amplitudes are quite similar for both models, 
just like the pitch angle  $24^{\circ}$--$29^{\circ}$ (with)
$24^{\circ}$--$26^{\circ}$ (without halo spin) and the number of 
spiral arms $m=7$--8 for $R=10$--14\,kpc (see Table~\ref{tb:pitch_angle}).


In addition, in Fig.~\ref{fig:AM} we investigate the angular-momentum flow for 
the disk and halo as a function of time and cylindrical radius.
Following \citet{2014ApJ...783L..18L} and \citet{2009ApJ...707..218V},
we measure the change in angular momentum of the $z$-component at
every $\sim 10$\,Myr.
For the halos (top panels) there is no continuous angular
momentum transfer from the disk to the halo, but we only discern random variations
in the angular momentum. These fluctuations look stronger at outer
radii, but this is because the angular momentum changes are normalized by
the disk' angular momentum, which is smaller in the outer regions.

The angular momentum of the disks vary with time (see the red and blue
stripes in the bottom panels), but overall the disk loses only 1.9\,\%
of its initial angular momentum for models with spin and 1.7\,\% for
models without.
The amplitude of the stripes for the disks roughly corresponds to the
amplitude of the spiral pattern. In Fig.~\ref{fig:amplitude_ev}, we show the
total power as a function of cylindrical radius and time for models
md0.5Rd1.5 (left) and md0.5Rd1.5s (right).
From this we conclude that for spiral arms the angular momentum transfer between the disk and 
the halo is not efficient.
On the other hand, for barred galaxies the angular momentum flow 
from the disk to the halo is considerably smaller for models with 
a larger halo spin \citep[see Fig. 3 in][]{2014ApJ...783L..18L}.


\begin{figure}
\includegraphics[width=40mm]{figures/md0.5_Rd1.5_110M_1024c.pdf}\includegraphics[width=40mm]{figures/md0.5_Rd1.5_a0.8_110M_1024c.pdf}\\
\caption{Snapshots for models md0.5Rd1.5 (left) and md0.5Rd1.5s (right). \label{fig:snapshots_spin_sp}}
\end{figure}

\begin{figure*}
  \includegraphics[width=\columnwidth]{figures/mode_rot_spiral_R10.pdf}
  \includegraphics[width=\columnwidth]{figures/mode_rot_spiral_R19.pdf}\\
\caption{Total power for models md0.5Rd1.5 and md0.5Rd1.5s at $R=9.5$ kpc (left) and 19.5 kpc (right). \label{fig:mode_spin_sp}}
\end{figure*}

\begin{figure*}
  \includegraphics[width=\columnwidth]{figures/AM_evolution_halo_md0.5mb1Rd1.5.pdf}\includegraphics[width=\columnwidth]{figures/AM_evolution_halo_md0.5mb1Rd1.5s.pdf}\\
  \includegraphics[width=\columnwidth]{figures/AM_evolution_disk_md0.5mb1Rd1.5.pdf}\includegraphics[width=\columnwidth]{figures/AM_evolution_disk_md0.5mb1Rd1.5s.pdf}\\
\caption{Angular momentum flow of the halo (top) and the disk (bottom) as a function of cylindrical radius and time for models md0.5mb1Rd1.5 (left) and md0.5mb1Rd1.5s (right). The angular momentum flow is calculated from the angular momentum's change in the $z$-component for every $\sim 10$\,Myr. The value (color) is scaled to the initial angular momentum of the disk at each radius for both the disks and halos.}
\label{fig:AM}
\end{figure*}


\begin{figure*}
  \includegraphics[width=\columnwidth]{figures/amplitude_evolution_md0.5mb1Rd1.5.pdf}
  \includegraphics[width=\columnwidth]{figures/amplitude_evolution_md0.5mb1Rd1.5s.pdf}\\
\caption{Total power as a function of cylindrical radius and time for models md0.5Rd1.5 (left) and md0.5Rd1.5s (right).}\label{fig:amplitude_ev}
\end{figure*}



\subsection{Initial Q value}

To verify the expectation that the initial value of Toomre's $Q$ parameter 
($Q_0$) influences the bar and spiral structure, we created a set of models in 
which we varied this parameter. 

The models are based on md0.5mb0, with one having an initially unstable disk
(md0.5mb0Q0.5) and 
the other having a large $Q_0$, in which no spiral arms develop (md0.5mb0Q2.0).
The time evolution of the bar's amplitude and length is presented in
Fig.~\ref{fig:A2_max_Q} 
and the surface densities are shown in Fig.~\ref{fig:snapshots_Q}.
For md0.5mb0Q2.0 there is no sign of spiral or bar structure until $\sim 5$\,Gyr, but 
a bar develops shortly after that (left panel of Fig.~\ref{fig:A2_max_Q}).
This matches with the 
expectation that $Q_0$ influences the bar formation epoch,
the smaller the $Q_0$ value the faster the bar forms.
The peak amplitude just after the bar formation is higher for
the larger $Q_0$, but the final amplitude is similar
(see the left panel of Fig.~\ref{fig:A2_max_Q}).
We also confirmed that the final bar length does not depend on $Q_0$
(see the right panel of Fig.~\ref{fig:A2_max_Q}). 
However, the radius that gives the maximum amplitude is different for 
the models with a large or a small value of $Q$.
The radius for $A_{\rm 2, max}$ is 2.6 and 4.9\,kpc for models with $Q_0=0.5$
and $2.0$, respectively. This result is qualitatively consistent with 
\citet{2012PASJ...64....5H} where an initially colder disk forms
a weaker and more compact bar due to the smaller velocity dispersion of the disk
(although they stopped their simulation just after the first amplitude peak).

This further proves (as discussed in Section~3.3) that the growth 
rate of swing amplification governs the bar formation timescale.
The growth rate decreases 
as $Q$ increases \citep{1981seng.proc..111T} which is confirmed by our simulations. 
With $Q_0=2.0$, the disk is initially stable and hence the spiral structure has 
to be induced by the bar. These ring-like spiral arms are sometimes seen in SB0--SBa 
galaxies such as NGC\,5101 \citep{2011ApJS..197...21H}.


\begin{figure*}
\includegraphics[width=\columnwidth]{figures/mode_A2max_Q.pdf}\includegraphics[width=\columnwidth]{figures/mode_Dbar_Q.pdf}
\caption{Same as Fig.~\ref{fig:A2_max_mdisk}, but for models md0.5mb0Q0.5 and md0.5mb0Q2.0.
\label{fig:A2_max_Q}}
\end{figure*}


\begin{figure}
\begin{center}
\includegraphics[width=40mm]{figures/md0.5_mb0_Q0.5_1536c.pdf}\includegraphics[width=40mm]{figures/md0.5_mb0_Q2.0_1536c.pdf}\\
    \caption{Snapshots for models md0.5mb0Q0.5 (left) and md0.5mb0Q2.0 (right).\label{fig:snapshots_Q}}
\end{center}
\end{figure}

\subsection{Disk scale length}

We further examine models md1mb1Rd1.5 and md0.5mb1Rd1.5, which
have a larger disk length scale. For these models the total disk mass is the same 
as that of models md1mb1 and md0.5mb1, but the disk scale length 
is larger. The changed disk scale length results in different rotation 
curves (see Fig.~\ref{fig:snapshots_Rdisk}). Given  Eq.~\ref{eq:mX} we expect 
that this leads to fewer spiral arms. The top views of these models are presented in
Fig.~\ref{fig:snapshots_Rdisk} (right panels) and the evolution of 
the bar's amplitude and length in Fig.~\ref{fig:A2_max_Rd}. 
The bar formation epoch of model md1mb1Rd1.5 (2\,Gyr) is
later than that of model mdmb1 (1\,Gyr). Model md0.5mb1Rd1.5 did not form 
a bar within 10\,Gyr, although model md0.5mb1 formed a bar at $\sim6$\,Gyr.
The difference
between these models is that the disk mass fraction ($f_{\rm d}$) for model
md1mb1R1.5 and md0.5mb1R1.5 is smaller than those for model md1mb1 
and md0.5mb1 (see Table~\ref{tb:bar_crit}).
Although the bar formation starts later for model md1mb1Rd1.5, the bar grows
faster, and 
the final bar length at 10\,Gyr is comparable for these models.
The bar's secular evolution, however, may continue further. 
In order to understand what decides the final bar length further simulations
are required. 


\begin{figure}
\includegraphics[width=50mm]{figures/rotation_curve_md1_Rd1.5.pdf}\includegraphics[width=38mm]{figures/md1_Rd1.5_80M_1024c.pdf}\\
\includegraphics[width=50mm]{figures/rotation_curve_md0.5_Rd1.5.pdf}\includegraphics[width=38mm]{figures/md0.5_Rd1.5_110M_1024c.pdf}\\
    \caption{Rotation curves (left) and snapshots at 10 Gyr (right) for models md1mb1Rd1.5 (top) and md0.5mb1Rd1.5 (bottom). 
    The gray dashed curve is the same as the one in Fig.~\ref{fig:snapshots_mb10}. \label{fig:snapshots_Rdisk}}
\end{figure}


\begin{figure*}
\includegraphics[width=\columnwidth]{figures/mode_A2max_Rd.pdf}\includegraphics[width=\columnwidth]{figures/mode_Dbar_Rd.pdf}
\caption{Same as Fig.~\ref{fig:A2_max_mdisk}, but now for models md1mb1Rd1.5 and md0.5mb1Rd1.5 with md1mb1 shown as reference.
\label{fig:A2_max_Rd}}
\end{figure*}


%!TEX root = hopfwright.tex

%%%%%%%%%%%%%%%%%%
%%% Appendix B %%%
%%%%%%%%%%%%%%%%%%

\section{Appendix: Endomorphism on a Compact Domain}
\label{sec:CompactDomain}



In order to construct the Newton-like map $T$ we defined operators $ A =  DF(\bar{x}_\epsilon) + \cO(\epsilon^2)$ and $A^{\dagger} = A^{-1} + \cO(\epsilon^2)$. 
However, as $(\bar{\alpha}_\epsilon,\bar{\omega}_\epsilon,\bar{c}_\epsilon) = (\pp,\pp,\bar{c}_\epsilon) + \cO(\epsilon^2)$,  the map $A$ can be better thought of as an $\cO(\epsilon^2)$ approximation of $DF(\pp,\pp,\bar{c}_\epsilon)$. 
Thus, when working with the map $T$ and considering points $ x \in  B_\epsilon(r,\rho)$ in its domain, we will often have to measure the distances of $ \alpha$ and $ \omega $ from $ \pp$. 
To that end, we define the following variables which will be used throughout the rest of the appendices. 
\begin{definition}
	\label{def:DeltaDef}
For $ \epsilon \geq 0$, and $r_\alpha,r_\omega,r_c >0$ we define 
\begin{alignat*}{2}
	\da^0 	&:= \tfrac{\epsilon^2}{5} ( 3 \pp -1) & \qquad\qquad
	\da 	&:= \da^0 + r_\alpha \\
	\dw^0 &:=  \tfrac{\epsilon^2}{5} &
	\dw &:=  \dw^0 + r_{\omega} \\ 
	\dc^0 &:=  \tfrac{2 \epsilon}{\sqrt{5}} &
	\dc &:=  \dc^0 + r_c . 
	% \\
	% \dt^0  &:= \dw^0 + \tfrac{1}{2} (\dw^0)^2 &
	% \dt  &:= \dw + \tfrac{1}{2} \dw^2 \\
	% \dtt^0  &:= 2 \dw^0 + \tfrac{1}{2} (2\dw^0)^2 &
	% \dtt  &:= 2 \dw + \tfrac{1}{2} (2\dw)^2  .
\end{alignat*}
\end{definition}


% \note[J]{
% 	I believe that we can replace the bounds $\dt$ by $\dw$  and $\dtt$ by $2 \dw$.In short, this follows from the following estimate.
% 	\[
% 	| e^{-i \omega }+i| \leq \int_{\pp}^\omega |\tfrac{\partial}{\partial \omega}  e^{-i \omega} | d\omega \leq  \int_{\pp}^\omega |1| d\omega = |\omega - \pp| .
% 	\]
% 	I have not gone through and done this yet. }
% \note[JB]{I think you are right. I think it also follows from $|e^{-i(\pp+\dw)}+i|^2=|e^{-i\dw}-1|^2 = (\cos \dw -1)^2+(\sin \dw)^2=2(1-\cos\dw) \leq 2 \cdot \frac{1}{2} \dw^2$.}
%
When considering an element $ ( \alpha , \omega, c)$ for our $\cO(\epsilon^2)$ analysis, we are often concerned with the 
 distances $|\alpha - \pp|$, $|\omega - \pp|$ and $ \| c - \bar{c}_\epsilon\|$, each of which is of order $\epsilon^2$.  
To create some  notational consistency in these definitions, $\da^0$ and $\dw^0$ are of order $\epsilon^2$, whereas $\dc^0$ is not capitalized as it is of order $\epsilon$. 
Using these definitions, it follows that for any $\rho>0$ and all  $(\alpha, \omega, c ) \in B_\epsilon(r,\rho)$ we have: 
\begin{alignat*}{1}
| \alpha - \pp | & \leq  \da       \\ 
	 | \omega - \pp| & \leq  \dw   \\
	\|c \| &\leq  \dc  .
	%  \\
	% | e^{- i \omega} + i| &\leq  \dt \\
	% | e^{-2 i \omega } +1| &\leq \dtt  .
\end{alignat*}
In this interpretation the superscript $0$ simply refers to $r=0$, i.e., the center of the ball $(\alpha,\omega,c) = \bx_\epsilon$.

The following elementary lemma will be used frequently in the estimates. 
\begin{lemma}\label{lem:deltatheta}
For all $x\in \R$ we have $|e^{ix}-1| \leq |x|$.
Furthermore, for all $|\omega - \bomega_\epsilon  | \leq r_\omega$  
%\note[JB]{I think this should be $|\omega - \bomega_\epsilon| \leq r_\omega$, no?} \note[J]{Yes, that is correct } 
we have 
$ |e^{- i \omega} + i| \leq  \dw$ and
$ | e^{-2 i \omega } +1| \leq 2 \dw $ .
\end{lemma}
\begin{proof}
We start with
\[
  |e^{ix}-1|^2 = (\cos x -1)^2+(\sin x)^2=2(1-\cos x) \leq 2 \cdot \tfrac{1}{2} x^2 = x^2.
\]
% Let $w = \omega - \pp$. Then $|w| \leq \dw$ and, using the previous inequality,
% \[
% | e^{- i \omega} + i|^2=
% |e^{-i(\pp+w)}+i|^2=|e^{-i w}-1|^2 \leq  w^2 =  \dw^2.
% \]
% \note[J]{To avoid using $w$ and $\omega$ in the same line, I propose we switch $ w \mapsto \theta$, as below. Also the last equality should be an inequality.}

Let $\theta = \omega - \pp$. Then $|\theta| \leq \dw$ and, using the previous inequality,
\[
| e^{- i \omega} + i|^2=
|e^{-i(\pp+\theta)}+i|^2=|e^{-i\theta}-1|^2 \leq  \theta^2 \leq  \dw^2.
\]
The final asserted inequality follows from an analogous argument.
\end{proof}


While the operators $U_\omega$ and $L_\omega$ are not continuous in $ \omega$ on all of $ \ell^1_0$, they are within the compact set $ B_\epsilon(r,\rho)$. 
To denote the derivative of these operators, we  define
\begin{alignat}{1}
	U_{\omega}' &:=  - i K^{-1} U_{\omega} \nonumber \\
	L_{\omega}' &:= - i \sigma^+( e^{- i \omega} I + K^{-1} U_{\omega}) + i \sigma^-(e^{i \omega} I - K^{-1} U_{\omega})  , \label{e:Lomegaprime}
\end{alignat}
and we derive Lipschitz bounds on $U_\omega$ and $L_\omega$ in the following proposition.
 
\begin{proposition}
	\label{prop:OmegaDerivatives}
	For the definitions above, $ \frac{\partial }{\partial  \omega} U_\omega = U_{\omega}' $ and $ \frac{\partial }{\partial  \omega}  L_\omega= L_{\omega}' $. 
	Furthermore,  for any $ (\alpha, \omega,c) \in B_\epsilon(r,\rho)$, we have the norm estimates
	\begin{alignat}{1}
	\| (U_{\omega} - U_{\omega_0} )c \| &\leq   \dw  \rho \nonumber  \\
	\|( L_{\omega} - L_{\omega_0} )c \| &\leq  2  \dw (  \dc +  \rho) .
	\label{e:LomegaLip}
	\end{alignat}
\end{proposition}
% \note[J]{ There was a mistake in the statement of this proposition. I changed the estimate $\| U_{\omega} - U_{\omega_0}  \| $ to $ \| (U_{\omega} - U_{\omega_0} )c \| $. Likewise for $ L_\omega$. }

\begin{proof}
One easily calculates that $ \frac{\partial U_\omega}{\partial  \omega} =  U_{\omega}'$,  whereby
$
	\| (U_{\omega} - U_{\omega_0} )c \| \leq \int_{\omega_0}^\omega \| \tfrac{\partial}{\partial \omega} U_\omega c \|  \leq    \dw  \rho  
$. 
Calculating $ \frac{\partial }{\partial  \omega}  L_{\omega} $, we obtain the following:
\begin{alignat*}{1}
 \frac{\partial }{\partial  \omega}  L_{\omega} 
&=  \frac{\partial }{\partial  \omega} \left[  \sigma^+( e^{- i \omega} I + U_{\omega}) + \sigma^-(e^{i \omega} I + U_{\omega}) \right] \\
&= - i \sigma^+( e^{- i \omega} I + K^{-1} U_{\omega}) + i \sigma^-(e^{i \omega} I - K^{-1} U_{\omega}) ,
\end{alignat*}
thus proving $ \frac{\partial L_\omega}{\partial  \omega} =  L_{\omega}'$,
and 
$\|( L_{\omega} - L_{\omega_0} )c \| \leq  \int_{\omega_0}^\omega \| \tfrac{\partial}{\partial \omega} L_\omega c \|  \leq   \dw ( 2  \dc + 2 \rho)$.
\end{proof}

\begin{proposition}
	Let $\epsilon\geq 0$ and  $r=(r_\alpha,r_\omega,r_c) \in \R^3_+$. 
	For any $ \rho > 0$ the map 
	 $T:B_{\epsilon}(r,\rho) \to \R^2 \times \ell^K_0 $ is well defined. 	
	We define functions 
% \note[J]{New definitions for $C_0$ and $C_1$ as the old ones did not quite match the estimates proven below. }
	\begin{alignat*}{1}
%	C_0 &:=  \frac{2 \epsilon^2}{\pi} 
%	\left[
%		\frac{8}{5},\frac{8}{5\sqrt{5}} \sqrt{\left(1-3 \pi /4 \right)^2+(2+\pi )^2},\frac{5 \pi }{2} 
%	\right]
%	\cdot \overline{A_0^{-1} A_1} \cdot [ \da , \dw , \dc ]^T ,
%%	\\
	% C_0 &:=  \frac{2 \epsilon^2}{\pi}
	% \left[
	% 	\frac{8}{5},\frac{2}{5} \sqrt{16+ 8\pi + 5 \pi^2},\frac{5 \pi }{2}
	% \right]
	% \cdot \overline{A_0^{-1} A_1} \cdot [ \da , \dw , \dc ]^T ,
	% \\
	C_0 &:=  \frac{2 \epsilon^2}{\pi} 
	\left[
	\frac{8}{5},\frac{2}{5} \sqrt{16+ 8\pi + 5 \pi^2},\frac{5 \pi }{2} 
	\right]
	\cdot \overline{A_0^{-1} A_1} \cdot [0,0 , \dc ]^T ,
	\\
%	C_1 &:= \frac{5 }{2 \pi} \left(1 +   \frac{4 \epsilon  }{5} \left(2+\sqrt{5}\right) \right) , \\
	% C_1 &:= \frac{5 }{2 \pi} + 2  \epsilon   \left(2+\sqrt{5}\right)  , \\
	C_1 &:= \frac{5 }{2 \pi} + \frac{\epsilon \sqrt{10}}{\pi}, \\
	C_2 &:= \dw  \left[  (1 + \pp) + \epsilon \pi  \right] , \\
	C_3 &:=  
	\da (2+ \dc) +	2 \dw (1+\pp) 
		+ \epsilon \left[ \pi + 2\da  + 4 \dc \da + \pi \dw \dc  + (\pp + \da ) \dc^2 \right] ,
	\end{alignat*}
where the expression for $C_0$ should be read as a product of a row vector, a $(3 \times 3)$ matrix and a column vector.
Furthermore we define, for any $\epsilon,r_\omega$ such that $C_1 C_2 <1$,
	\begin{equation}
		C(\epsilon,r_\alpha,r_\omega,r_c) := \frac{C_0+ C_1 C_3}{1 - C_1 C_2}
		 \, .
		\label{eq:RhoConstant}
	\end{equation}
	All of the functions $C_0,C_1,C_2,C_3$ and $C$ are nonnegative and monotonically increasing in their arguments $\epsilon$ and~$r$. 
	Furthermore, if  $C_1 C_2 < 1$ and $	C(\epsilon,r_\alpha,r_\omega,r_c) \leq \rho $
	then $\| K^{-1} \pi_c  T( x) \| \leq \rho $
	for $x \in B_{\epsilon}(r,\rho)$. 
	\label{prop:DerivativeEndo}
\end{proposition}

% \marginpar{This proposition is vague about the actual spaces being used, ie. $\ell_1,\ell^K_0$, etc.}

\begin{proof}
	Given their definitions, it is straightforward to check that the functions $C_i$ and $C$ are monotonically increasing in their arguments.  
	To prove the second half of the proposition, we split 
	$K^{-1} \pi_c  T(x)$ into several pieces. 
%\note[JB]{$\pi_c$ and $\pi_{\ge 2}$ added. Jonathan, could you please go through this and check?}
%\note[JB]{This did not work, since we do not have that $x$ is bounded by $[\da,\dw,\dc]^T$. Jonathan: what you probably meant was what I introduce as $\pi_c^0 x$, but could you please check?} 
	We define the projection $\pi_c^0 x = (0,0,\pi_c x)$.
We then obtain
	\begin{alignat*}{1}
	K^{-1} \pi_c  T(x)  &= K^{-1} \pi_c   [ x - A^{\dagger} F(x) ]   \\
	&= K^{-1} \pi_c  [ I \pi_c^0 x -    A^{\dagger} ( A \pi_c^0 x + F(x) - A \pi_c^0 x)]  \\
	&= \epsilon^2 K^{-1} \pi_c (A_0^{-1}A_{1})^2 \pi_c^0 x + K^{-1} \pi_c A^{\dagger} (F(x) - A \pi_c^0 x) \nonumber \\
	&=  \frac{2 \epsilon^2}{i\pi} \hat{U} \pi_{\ge 2} A_1 A_0^{-1}A_{1} \pi_c^0 x +\frac{2 }{i\pi} \hat{U}  \pi_{\ge 2} (I-\epsilon A_1 A_0^{-1}) (F(x) - A\pi_c^0 x)  ,
\end{alignat*}
where we have used that $K^{-1} \pi_c A_0^{-1} = \frac{2}{i\pi} \hat{U} \pi_{\ge 2}$, with the projection $\pi_{\ge 2}$ defined in~\eqref{e:pige2}.
By using $\| \hat{U} \| \leq \frac{5}{4}$, see Proposition~\ref{p:severalnorms}, we obtain the estimate
%\note[J]{Changed $[\da,\dw,\dc ]^T$ to $[0,0,\dc ]^T$}
\begin{equation}
	\| K^{-1} \pi_c T(x) \| \leq   \frac{2 \epsilon^2}{\pi} \overline{\hat{U}\pi_{\ge 2} A_1} \cdot  \overline{ A_0^{-1}A_{1}}  \cdot
	[0,0,\dc ]^T +\frac{5 }{2 \pi} \left(1 + \epsilon \| A_1 A_0^{-1} \| \right) \|F(x) - A\pi_c^0 x \| .
	\label{eq:DerivativeEndo}
\end{equation}
Here the $(1 \times 3)$ row vector $\overline{\hat{U}\pi_{\ge 2} A_1}$ is an upper bound on $\hat{U}\pi_{\ge 2} A_1$ interpreted as a linear operator from $\R^2 \times \ell^1_0$ to $\ell^1_0$, thus extending in a straightforward manner the definition of upper bounds given in  Definition~\ref{def:upperbound}.
	
	
	We have already calculated  an expression for
	 $ \overline{ A_0^{-1}A_{1}}$ in Proposition~\ref{prop:A0A1},  and  $  \| A_1 A_0^{-1}\| =\frac{2\sqrt{10}}{5}$ by Proposition~\ref{prop:A1A0}.  In order to finish the calculation of the right hand side of Equation \eqref{eq:DerivativeEndo}, we need to  estimate  $\| F(x) - A\pi_c^0 x \|$ and $\overline{\hat{U} \pi_{\ge 2} A_1} $. 
	We first calculate a bound on $\hat{U} \pi_{\ge 2} A_1 $. 
	We note that $ \hat{U} \pi_{\ge 2} A_1  =  \hat{U} \e_2 ( i_\C A_{1,2} \pi_{\alpha,\omega})+ \hat{U} \pi_{\ge 2}A_{1,*} \pi_c$.	
As $\|\hat{U} e_2\| = \| \tfrac{4-2i}{5} \e_2\|$,
it follows from the definition of $A_{1,2}$ 
that 
\[
	 \left| i_\C  A_{1,2}
	 \left( \!\!\begin{array}{c}\alpha \\ \omega \end{array} \!\!\right) \right|  
	 \cdot \| \hat{U} \e_2 \| 
	 \leq 
	 \left(\frac{\sqrt{20}}{5} |\alpha| +  \frac{\sqrt{(2-3 \pi/2)^2 +4(2+\pi)^2}}{5} |\omega| \right)  \cdot \frac{4}{\sqrt{5}}.
\]
	To calculate $ \| \hat{U} \pi_{\ge 2} A_{1,*} \|$ we note that $ \| \hat{U}\| \leq \frac{5}{4}$ and $ \|A_{1,*}\| = \pp \| L_{\omega_0} \| \leq 2 \pi$. 
	Hence $ \| \hat{U} \pi_{\ge 2} A_{1,*} \| \leq \frac{5 \pi}{2}$. 
	Combining these results, we obtain  that
%\note[JB]{I think the second, rearranged version, of the root looks ``nicer''. }
%	\[
%	\overline{\hat{U} \pi_{\ge 2}  A_1 } = \left[\frac{8}{5},\frac{8}{5\sqrt{5}} \sqrt{\left(1-3 \pi /4 \right)^2+(2+\pi )^2},\frac{5 \pi }{2} \right].
%	\] 
	\[
	\overline{\hat{U} \pi_{\ge 2}  A_1 } = \left[\frac{8}{5},\frac{2}{5} \sqrt{16 + 8 \pi + 5 \pi^2},\frac{5 \pi }{2} \right].
	\] 
Thereby, it follows from~\eqref{eq:DerivativeEndo} that 
\begin{equation}\label{e:C0C1}
	\| K^{-1} \pi_c T(x) \| \leq C_0 + C_1 \| F(x) - A \pi_c^0 x\|. 
\end{equation}
We now calculate
	\begin{alignat*}{1}
	F(x) - A \pi_c^0 x &= 
	(i \omega + \alpha e^{-i \omega} ) \e_1 + 
	( i \omega K^{-1} + \alpha U_{\omega}) c + 
	\epsilon \alpha e^{-i \omega} \e_2  +
	\alpha \epsilon L_\omega c + 
	\alpha \epsilon [ U_{\omega} c] * c  
	\\ &\qquad 
	- \pp (i K^{-1} + U_{\omega_0} + \epsilon L_{\omega_0} ) c \\
	&= i ( \omega - \pp) K^{-1} c + ( \alpha - \pp) U_{\omega} c +  \pp ( U_{\omega} - U_{\omega_0})c  \nonumber \\
	&\qquad  + \left[i ( \omega - \pp ) + ( \alpha - \pp) e^{-i \omega} + \pp( e^{- i \omega }+ i)\right] \e_1  \nonumber
	\\ 
	&\qquad  +\epsilon  \alpha   e^{-i \omega}  \e_2  
+  ( \alpha- \pp)  \epsilon L_{\omega} c + \pp \epsilon ( L_{\omega} - L_{\omega_0}) c + \alpha \epsilon [ U_{\omega} c ] * c .
	\end{alignat*}
Taking norms and using~\eqref{e:LomegaLip} and Lemma~\ref{lem:deltatheta}, we obtain 
	\begin{alignat*}{1}
	\| F(x) - A \pi_c^0 x\|& \leq  
	 \dw \rho + \da \dc + \pp \dw \rho
    +	2 (\dw + \da + \pp \dw)  
	   \\
	&\qquad + \epsilon \left[ 2(\pp + \da ) + 4 \dc \da + \pi  \dw (  \dc + \rho) + (\pp + \da ) \dc^2 \right]  \\
		&= \dw [ (1+\pp) +   \epsilon \pi ] \rho \nonumber \\ 
	&\qquad +  \da (2 + \dc)
	+	2 \dw (1+\pp) 
	+ \epsilon \left[ \pi + 2\da  + 4 \dc \da + \pi \dw \dc  + (\pp + \da ) \dc^2 \right].  
	\end{alignat*}


	We have now computed all of the necessary constants. Thus $ \| F(x) - A \pi_c^0 x \| \leq C_2 \rho + C_3$, and from~\eqref{e:C0C1}   we obtain 
	\begin{eqnarray*}
	\| K^{-1} \pi_c T(c) \|
	&\leq & C_0 +  C_1 ( C_2  \rho + C_3),
	\end{eqnarray*}
with the constants defined in the statement of the proposition.
We would like to select values of $\rho$ for which 
	\[
	\| K^{-1} \pi_c T(c) \| \leq \rho
	\]
	This is true if  
	$	C_0 +  C_1 ( C_2  \rho + C_3) \leq \rho$, 
	or equivalently 
	\[
	\frac{C_0 + C_1 C_3 }{1 - C_1 C_2} \leq \rho.
	\]
	This proves the theorem.
\end{proof}

%%%%%%%%%%%%%%%%%%
%%% Appendix C %%%
%%%%%%%%%%%%%%%%%%


\section[]{Relation between bulge-to-disk mass ratio and shear rate}

In Section 3.4 we showed that the bulge-to-disk mass ratio ($B/D$) is not 
always a good indicator for the shear rate ($\Gamma$), because $\Gamma$ 
also depends on other parameters such as the disk-mass fraction ($f_{\rm d}$). 
Here, we construct additional initial conditions by sequentially changing some 
parameters in order to investigate their importance. We do not simulate 
these models, but measure $B/D$ and $\Gamma$ in the generated models at $t=0$.
All parameters of these models are summarized in Tables \ref{tb:models_add} and 
 \ref{tb:models_add2}.

In Fig.~\ref{fig:Gamma_BD}, we present the relation between $B/D$ and $\Gamma$
calculated from the additional initial conditions.
If we keep both $f_{\rm d}$ and the bulge scale length ($r_{\rm b}$) constant,
$\Gamma$ monotonically increases as $B/D$ increases (square symbols).
But if we increase $r_{\rm b}$ while keeping $f_{\rm d}$ constant, then $\Gamma$ 
also increases (triangle symbols). 
If we increase $f_{\rm d}$ and keep $B/D$ and $r_{\rm b}$ constant,
$\Gamma$ increases (diamond symbols). 
The halo scale length ($r_{\rm h}$) and scale velocity 
($\sigma_{\rm h}$), on the other hand, barely affect the relation between $B/D$ and $\Gamma$
(circle symbols).


\begin{figure}
\includegraphics[width=\columnwidth]{figures/shear_rate_BD_fd.pdf}
\caption{Relation between bulge-to-disk mass ratio ($B/D$) and shear rate ($\Gamma$). \label{fig:Gamma_BD}}
\end{figure}


\begin{table*}
\begin{center}
%\rotate
\caption{Parameters for additional initial conditions\label{tb:models_add}}
\begin{tabular}{lccccccccccc}
%\tabletypesize{\scriptsize}
%\tablewidth{0pt}
%\startdata 
\hline
           &  \multicolumn{3}{l}{Halo} &  \multicolumn{4}{l}{Disk} &  \multicolumn{3}{l}{Bulge} \\
Parameters &  $a_{\rm h}$ & $\sigma_{\rm h}$ & $1-\epsilon_{\rm h}$ & $M_{\rm d}$ & $R_{\rm d}$ & $z_{\rm d}$ & $\sigma_{R0}$  & $a_{\rm b}$ & $\sigma_{\rm b}$ & $1-\epsilon_{\rm b}$ \\ 
Model   &  (kpc) & ($\kms$) &  &  $(10^{10}M_{\odot})$ & (kpc) & (kpc) & ($\kms$)  & (kpc) & $(\kms)$\\
\hline \hline
Add1 & 8.2 & 350 & 0.9  & 2.45 & 2.8 & 0.36 & 105 & 0.64 & 300 & 1.0  \\ %t13
Add2 & 11.5 & 443 & 0.9  & 2.45 & 2.8 & 0.36 & 105 & 0.65 & 400 & 1.0  \\ %t5
Add3 & 8.2 & 370 & 0.9  & 2.45 & 2.8 & 0.36 & 105 & 0.64 & 500 & 1.0  \\ %t14
Add4 & 10 & 340 & 0.9  & 2.45 & 2.8 & 0.36 & 105 & 0.64 & 550 & 1.0  \\ %t8
Add5 & 8.2 & 295 & 0.9  & 2.45 & 2.8 & 0.36 & 105 & 0.65 & 600 & 1.0  \\ %t7
Add6 & 8.2 & 284 & 0.9  & 2.45 & 2.8 & 0.36 & 105 & 1.3 & 370 & 1.0  \\ %t11
Add7 & 8.2 & 330 & 0.9  & 2.45 & 2.8 & 0.36 & 105 & 0.8 & 380 & 1.0   \\  %t12
Add8 & 8.2 & 330 & 1.0  & 2.45 & 2.8 & 0.36 & 105 & 0.64 & 550 & 1.0  \\ %t10
Add9 & 12 & 400 & 1.0  & 2.45 & 2.8 & 0.36 & 105 & 0.64 & 540 & 1.0  \\ %t9
Add10 & 8.2 & 370 & 0.9  & 1.47 & 2.8 & 0.36 & 105 & 0.64 & 390 & 1.0  \\ %t15
Add11 & 12 & 330 & 0.9  & 2.45 & 2.8 & 0.36 & 105 & 0.64 & 486 & 1.0  \\ %t17
\hline
\end{tabular}
\end{center}
\end{table*}


\begin{table*}
\begin{center}
\caption{Obtained values for additional initial conditions\label{tb:models_add2}}
\begin{tabular}{lccccccccc}
%\tabletypesize{\scriptsize}
%\rotate
%\tablewidth{0pt}
%\startdata 
\hline
Model    & $M_{\rm d}$ & $M_{\rm b}$ & $M_{\rm h}$ & $M_{\rm b}/M_{\rm d}$& $R_{\rm d, t}$ & $r_{\rm b, t}$ & $r_{\rm h, t}$ & $f_{\rm d}$  &  $\Gamma$\\ 
   & ($10^{10}M_{\odot}$) & ($10^{10}M_{\odot}$) & ($10^{10}M_{\odot}$) & (kpc) & (kpc) & (kpc) &  &   &  \\ 
\hline  \hline
Add1 & 2.57 & 0.514 & 56.0 & 0.20 & 31.6 & 3.57 & 284 &  0.346 & 0.682 \\ %test13
Add2 & 2.58 & 1.21 & 137 & 0.47 & 31.6 & 5.32 & 330 & 0.343 & 0.706 \\ %test5
Add3 & 2.69 & 2.03 & 94.6 & 0.75 & 31.6 & 6.65 & 234 &  0.321 & 0.895 \\ %test14
Add4 & 2.59 & 2.74 & 124 & 1.05 & 31.6 & 8.67 & 288 &   0.340 & 1.04 \\ %test8
Add5 & 2.61 & 3.29 & 93.2 & 1.26 & 31.6 & 9.48 & 270 &  0.307 & 1.10 \\ %test7
Add6 & 2.59 & 1.19 & 45.9 & 0.46 & 31.6 & 6.28& 265 &  0.332 & 0.869 \\ %test11
Add7 & 2.58 & 1.11 & 61.3 & 0.43 & 31.6 & 5.31 & 251 & 0.348  & 0.789 \\ %test12
Add8 & 2.61 & 2.62 & 108 & 1.0 & 31.6 & 7.98 & 494 &  0.322 &  0.996\\ %test10
Add9 & 2.59 & 2.64 & 195 & 1.02 & 31.6 & 8.46 & 687 &  0.341 & 1.00 \\  % test9
Add10 & 1.58 & 1.18 & 74.8 & 0.74 & 31.6 & 5.65 & 234 &  0.251  & 0.675 \\ %test15
Add11 & 2.75 & 2.07 & 130 & 0.75 & 31.6 & 8.00 & 324 &  0.401 & 0.992 \\ %test17
\hline
\end{tabular}
\end{center}
\medskip
\end{table*}



%!TEX root = hopfwright.tex

%%%%%%%%%%%%%%%%%%
%%% Appendix D %%%
%%%%%%%%%%%%%%%%%%

\section{Appendix: The bounding functions for $Z(\epsilon,r,\rho)$}
\label{sec:BoundingFunctions}

	In this section we calculate an upper bound on $DT$.  
	To do so we first calculate 
	\(
	DF = 
	\left[ \frac{\partial F}{\partial  \alpha}, 
	\frac{\partial F}{\partial  \omega},
	\frac{\partial F}{\partial  c}
	\right]
	\):
%\note[JB]{I don't think we need to write the partial derivatives in a proposition and no proof is needed.}
%\begin{proposition}
%	The partial derivatives of $F$ are
	\begin{alignat}{1}
	\label{eq:FpartialA}
	\frac{\partial F}{\partial  \alpha} &= e^{-i \omega} \e_1 + U_\omega c + \epsilon e^{-i \omega} \e_2 + \epsilon L_\omega c + \epsilon [ U_\omega c] * c , \\
	\label{eq:FpartialW} 
	\frac{\partial F}{\partial  \omega} &=
	i(1-\alpha e^{-i \omega}) \e_1 + 
	i K^{-1} ( I - \alpha U_{\omega} ) c  -
	i \alpha \epsilon e^{-i \omega} \e_2 + 
	\alpha \epsilon L_{\omega}' c - i \alpha \epsilon [ K^{-1} U_\omega c ] *c ,
	\\
	% \end{alignat}
	% and, writing $\frac{\partial F}{\partial  c}$ as an operator working on an element $ b \in \ell^K_0$,
	% % with $ \| b \| = 1$:
	% \begin{equation}
	\frac{\partial F}{\partial  c} \cdot b 
	& =
	( i \omega K^{-1} + \alpha U_{\omega}) b + \alpha \epsilon \left( L_\omega b  + [ U_\omega b] * c + [U_{\omega} c ]*b \right)  , \qquad \text{for all $b\in \ell^K_0$},
	\label{eq:Fcderivative}
%	\end{equation}
\end{alignat}	
%\end{proposition}
where $L_{\omega}'$ is given in~\eqref{e:Lomegaprime}, and $\frac{\partial F}{\partial  c}$ is expressed in terms of the directional derivative. 
Recall that $\II$ is used to denote the $ 3 \times 3$ identity matrix. 

% \begin{proof}
% 	Recall from Equation \ref{eq:FDefinition} that:
% \begin{equation}
% F_\epsilon(\alpha,\omega, c) =
% [i \omega + \alpha e^{-i \omega}] \e_1 +
% ( i \omega K^{-1} + \alpha U_{\omega}) c +
% \epsilon \alpha e^{-i \omega} \e_2  +
% \alpha \epsilon L_\omega c +
% \alpha \epsilon [ U_{\omega} c] * c.
% \end{equation}
% The partial derivative for $ 	\frac{\partial F}{\partial  \alpha}$ and $ 	\frac{\partial F}{\partial  c}$   clearly follow, and the partial derivative for $ 	\frac{\partial F}{\partial  \omega}$  follows from Proposition \ref{prop:OmegaDerivatives}.
%
%
%
% \end{proof}


\begin{theorem}
	\label{prop:Zdef}
	Define $\overline{A_0^{-1} A_1}$ as in Proposition \ref{prop:A0A1} and define the matrix 
%	\[
%	M := 
%	\left(
%	\begin{array}{ccc}
%	1+\frac{2}{\pi } & 0 & 0 \\
%	\frac{2}{\pi } & 0 & 0 \\
%	0 & 0 & 1 \\
%	\end{array}
%	\right)
%	\left(
%	\begin{array}{ccc}
%	f_{1,\alpha } & f_{1,\omega } & f_{1,c} \\
%	0 & 0 & 0 \\
%	f_{*,\alpha } & f_{*,\omega } & f_{*,c} \\
%	\end{array}
%	\right)
%	\]
%\note[JB]{I think we should remove the zero column and row, as below:}
%\note[J]{Updated top-right element entry to new estimate.}
	\begin{equation}\label{e:defM}
	M := 
	\left(
	\begin{array}{cc}
	\sqrt{\tfrac{4}{\pi^2}+1} & 0 \\
	\frac{2}{\pi } & 0 \\
	0 & 1 \\
	\end{array}
	\right)
	\left(
	\begin{array}{ccc}
	f_{1,\alpha } & f_{1,\omega } & f_{1,c} \\
	f_{*,\alpha } & f_{*,\omega } & f_{*,c} \\
	\end{array}
	\right) ,
	\end{equation}
	where the functions $f_{1,\cdot}(\epsilon,r,\rho)$ and $f_{*,\cdot}(\epsilon,r,\rho)$ are defined as in Propositions \ref{prop:Z1a}--\ref{prop:Zsc}. 
	If we define $Z(\epsilon,r,\rho)$ as 
	\begin{equation}
		Z(\epsilon,r,\rho) := \epsilon^2  \left(\overline{ A_0^{-1} A_1 }\right)^2  + 
		\left(\II + \epsilon \overline{ A_0^{-1} A_1 } \right) \cdot M ,
	\end{equation}
	then $Z(\epsilon,r)$ is an upper bound (in the sense of Definition~\ref{def:upperbound}) on $DT(x)$ for all $ x \in B_\epsilon(r , \rho)$. 
	Furthermore, the components of $Z(\epsilon,r,\rho)$ are increasing in  $ \epsilon$, $r$ and $\rho$. 
% \note[JB]{shouldn't we also have monotonicity in $\rho$?}
% \note[J]{Yes. Also note that in our application of radii polynomials, the value of $\rho$ is fixed for all $ 0< \epsilon \leq \epsilon_0$.}
\end{theorem}

\begin{proof}
	
	
	If we fix some $x \in B_\epsilon(r,\rho)$, then we obtain 
%	\remove[JB]{the following:}
%\note[JB]{Rearranged to separate expression for $DT$ from the upper bound on it}
\begin{alignat*}{1}
		D T( x ) &=  I - A^{\dagger}  D F( x)  \\
		&= ( I - A^{\dagger} A) - A^{\dagger} \left[ D F( x)  - A \right]\\
		&=   \epsilon^2 (A_0^{-1} A_1 )^2 -    [I - \epsilon (A_0^{-1} A_1 ) ] \cdot  A_0^{-1} \cdot   \left[ D F( x) - A \right] ,
\end{alignat*}
hence an upper bound on $DT(x)$ is given by
\begin{equation*}
\epsilon^2  \left(\overline{ A_0^{-1} A_1 }\right)^2  + 
\left(\II + \epsilon \overline{ A_0^{-1} A_1 } \right) \cdot 
\overline{A_0^{-1}  \left[D F( x ) - A \right] },
\end{equation*}
	where $\overline{A_0^{-1}  \left[D F( x ) - A \right] }$ is a yet to be determined upper bound on $A_0^{-1}  \left[D F( x ) - A \right]$.  
To calculate this upper bound, we break it up into two parts: 
	\begin{alignat}{1}
		\pi_{\alpha,\omega} A_0^{-1}  \left( D F( x) - A \right)  &= 
		A_{0,1}^{-1}  i_{\C}^{-1} \pi_1   \left( D F( x) - A \right) 
\label{eq:Zfinite}\\ 
		\pi_c 		A_0^{-1}  \left( D F( x) - A \right)  &=
		A_{0,*}^{-1}  \pi_{\geq 2} \left( DF(x) - A \right)  . \label{eq:Zstar} 
\end{alignat}
	
To calculate an upper bound on \eqref{eq:Zfinite}, we use the explicit expression for $A_{0,1}^{-1}$ to estimate  
% \note[JB]{Perhaps the factor $\frac{2}{\pi}+1$ can be improved to $\sqrt{\frac{4}{\pi^2}+1}$?; did not implement it.} 
%\note[J]{Yes; done.}
	\begin{alignat*}{1}
	\left| \pi_\alpha  A_{0,1}^{-1} \pi_1 \left( D F( x) - A \right)\right| &\leq  \sqrt{\tfrac{4}{\pi^2} + 1} \, \overline{\pi_1( DF( x ) - A) } \\
	\left| \pi_\omega  A_{0,1}^{-1} \pi_1 \left( D F( x) - A \right)\right| &\leq  \tfrac{2}{\pi} \,  \overline{ \pi_1( DF( x ) - A)  } ,
	\end{alignat*}
where $\overline{ \pi_1( DF( x ) - A)  }$ is an upper bound on $\pi_1( DF( x ) - A)$, viewed as an operator from $\R^2 \times \ell^K_0 $ to $\C$ (a straightforward generalization of Definition~\ref{def:upperbound}).
Indeed, in Propositions \ref{prop:Z1a}, \ref{prop:Z1w} and \ref{prop:Z1c} we   determine functions $f_{1,\cdot}$ such that, for all $x \in B_\epsilon(r,\rho)$, 
	\begin{alignat*}{1}
	f_{1,\alpha} (\epsilon,r,\rho) &\geq    \left|  \frac{\partial F_1 }{\partial \alpha} (x) + i  \right|  , \\
	f_{1,\omega} (\epsilon,r,\rho) &\geq   \left|  \frac{\partial F_1}{\partial \omega} (x)- (i- \pp)  \right|   , \\
	f_{1,c} (\epsilon,r,\rho) &\geq   \left|  \frac{\partial F_1}{\partial c} (x) \cdot b -  \pp \epsilon (i-1) \pi_2 b \right| ,
	\qquad\text{for all $b\in\ell^K_0$ with $\|b\| \leq 1$}.
	\end{alignat*} 
Here the projection $\pi_2$ is defined as $\pi_2 b := b_2 \in \C$ for $b=\{b_k\}_{k=1}^{\infty} \in \ell^1$.	
	Hence $ [ f_{1,\alpha} , f_{1,\omega}, f_{1,c}]$ is an upper bound on $\pi_1 (DF( x ) -A )$.  
	
	
	For calculating an upper bound on Equation~\eqref{eq:Zstar}, in Propositions  \ref{prop:Zsa}, \ref{prop:Zsw}  and \ref{prop:Zsc} we  determine functions $f_{*,\cdot}$ such that, for all $x \in B_\epsilon(r,\rho)$,   
	\begin{alignat*}{1}
	f_{*,\alpha} (\epsilon,r,\rho)&\geq  \left\| A_{0,*}^{-1}  \pi_{\geq 2} \left(
	\frac{\partial F}{\partial \alpha}(x)  +    \epsilon  \tfrac{2 +4 i}{5} \e_2  \right) \right\| , \\
	%
%	f_{*,\omega} (\epsilon,r,\rho)&\geq  \left\| A_{0,*}^{-1}  \pi_{\geq 2}\left( \frac{\partial F}{\partial \omega}(x) -  \epsilon \left[ (2 + \pi) ( \tfrac{1+ 2 i}{5})- \pp \right] \e_2  \right) \right\| ,  \\
%
	f_{*,\omega} (\epsilon,r,\rho)&\geq  \left\| A_{0,*}^{-1}  \pi_{\geq 2}\left( \frac{\partial F}{\partial \omega}(x) -  \epsilon \left[ \tfrac{4-3\pi}{10} + \tfrac{2(2 + \pi)}{5}i \right] \e_2  \right) \right\| ,  \\
	%
	f_{*,c} (\epsilon,r,\rho) &\geq    \left\|  A_{0,*}^{-1} \pi_{\geq 2} \left( \frac{\partial F}{\partial c}(x) \cdot b  - (A_{0,*} + \epsilon A_{1,*}) b \right) 
	\right\| , \qquad\text{for all $b\in\ell^K_0$ with $\|b\| \leq 1$}.
	\end{alignat*}
	Hence $ [ f_{*,\alpha} , f_{*,\omega}, f_{*,c}]$ is an upper bound on $A_{0,*}^{-1}  \pi_{\geq 2} \left( D F( x) - A \right)$, viewed as an operator from $\R^2 \times \ell^K_0$ to $\ell^1_0$. 
		We have thereby shown that $M$, as defined in~\eqref{e:defM}, is an upper bound on $\overline{A_0^{-1}  \left[D F( x ) - A \right] }$, which concludes the proof.	
\end{proof}









%%%%%%%%%%%%%%%%%%%%%%%%%%%%%%%%%%%%%%%%%%%%%%%%%%%%%%%%%%%%%%%%%%%%%%%%%%%%

%\note[JB]{I do not understand why we want to suppose $\epsilon <1$. I changed it in the D.2 but not yet in the other 5 propositions.} 
%\note[J]{You are correct that $ \epsilon <1$ is not needed. I have removed this assumption in the other propositions.}
\begin{proposition}
	\label{prop:Z1a}
	Define
	\[
	f_{1,\alpha} :=  \dw +  \epsilon \frac{\dc  (2 + \dc) }{2} .
	\]
	Then for all $x = (\alpha,\omega,c) \in B_\epsilon(r,\rho)$ 
	\[
	f_{1,\alpha} \geq   \left|  \frac{\partial F_1}{\partial \alpha} (x) + i  \right| .
	\]
\end{proposition}




\begin{proof}
	We calculate
\begin{equation*}
	\frac{\partial F_1}{\partial \alpha} (x) + i =
	e^{- i \omega} + i  
	+ \epsilon \left( e^{i \omega} + e^{-2 i  \omega} \right) \pi_2 c
	+ \epsilon \pi_1 ([ U_{\omega} c] * c) ,
\end{equation*}
hence, using Lemma~\ref{lem:deltatheta},
%\begin{equation*}
%	\left|  \frac{\partial F_1 }{\partial \alpha} F_1(x) + i  \right|   \leq 
%| e^{-i \omega } +i | + 2 \epsilon \dc + \epsilon \dc^2  
%	\leq  \dw +  \epsilon \dc  (2 + \dc) .
%\end{equation*}
%\note[JB]{Because of the factor 2 in the norm I think it should be }
\begin{equation*}
	\left|  \frac{\partial F_1}{\partial \alpha} (x) + i  \right|   \leq 
| e^{-i \omega } +i | + 2 \epsilon \frac{\dc}{2} + \epsilon \frac{1}{2} \dc^2  
	\leq  \dw +  \epsilon \frac{\dc  (2 + \dc)}{2} .
\end{equation*}
Here we have used that $|\pi_k a| \leq \frac{1}{2}\|a\|$ for $k=1,2$ and all $a \in \ell^1$.
\end{proof}

%\note[J]{I think it would be good to have an explanation for why we divide $ \dc$ by 2. This explanation wouldn't need appear every time; it could just be given once. I am not sure though where the right place for it would be. }
%\note[JB]{See my attempt above}

%%%%%%%%%%%%%%%%%%%%%%%%%%%%%%%%%%%%%%%%%%%%%%%%%%%%%%%%%%%%%%%%%%%%%%%%%%%%

\begin{proposition}
		\label{prop:Z1w}
	Define
	% \[
	% f_{1,\omega} :=
	% \da + \pp \dw + \frac{ \alpha \epsilon \dc}{2} ( 3 + \rho) .
	% \]
	% \note[J]{Proposed Change}
		\[
		f_{1,\omega} := 
		\da + \pp \dw + (\pp + \da) \frac{  \epsilon \dc}{2} ( 3 + \rho)  .
		\]
Then for all $x= (\alpha,\omega,c) \in B_\epsilon(r,\rho)$
	\[
	f_{1,\omega} \geq   \left|  \frac{\partial F_1}{\partial \omega } (x)- (i- \pp)  \right| .
	\]
\end{proposition}




\begin{proof}
We calculate 
\begin{alignat*}{1}
	\frac{\partial F_1}{\partial \omega} (x) - (i - \pp)  &=
	(i - i\alpha e^{- i \omega}) - (i - \pp) 
	+ \alpha \epsilon ( i e^{i \omega }- 2 e^{- 2 i \omega}) \pi_2 c -i \alpha \epsilon \pi_1  ([ K^{-1} U_\omega c ] *c )\\
	&= -i (\alpha - \pp) e^{-i \omega} - i \pp ( i + e^{-i\omega} )
	+ \alpha \epsilon ( i e^{i \omega }- 2 e^{- 2 i \omega}) \pi_2 c -i \alpha \epsilon \pi_1( [ K^{-1} U_\omega c ] *c) ,
\end{alignat*}
hence, using Lemma~\ref{lem:deltatheta} again,
%\begin{equation*}
%	\left|  \frac{\partial F_1}{\partial \omega} ( x)- (i- \pp)  \right|  \leq
%	\da +  \pp \dw  + 3 \alpha \epsilon \dc + \alpha \epsilon \rho \dc  .
%%	\\ &\leq &   \da + \pp \dt + \alpha \epsilon \dc ( 3 + \rho) 
%\end{equation*}
%\note[JB]{Because of the factor 2 in the norm I think it should be }
\begin{equation*}
	\left|  \frac{\partial F_1}{\partial \omega} ( x)- (i- \pp)  \right|  \leq
	\da +  \pp \dw  + \frac{3}{2} \alpha \epsilon \dc +  \frac{1}{2} \alpha \epsilon \rho \dc  .
\end{equation*}
\end{proof}

%%%%%%%%%%%%%%%%%%%%%%%%%%%%%%%%%%%%%%%%%%%%%%%%%%%%%%%%%%%%%%%%%%%%%%%%%%%%


\begin{proposition}
		\label{prop:Z1c}
	Define
	\[
	f_{1,c} := 
	\epsilon \left(  \da + \tfrac{3 \pi}{4} \dw +  (\pp + \da ) \dc   \right) .
	\]
	Then for all $x= (\alpha,\omega,c) \in B_\epsilon(r,\rho)$
	\[
	f_{1,c} \geq   \left|  \frac{\partial F_1 }{\partial c } (x) \cdot b -  \pp \epsilon (i-1) \pi_2 b \right|, 
	\qquad\text{for all $b\in\ell^K_0$ with $\|b\| \leq 1$}.
	\]
\end{proposition}



\begin{proof}
	We calculate
\begin{alignat*}{1}
		\frac{\partial F_1 }{\partial c } (x) \cdot b -  \pp \epsilon (i-1) \pi_2 b 
	& =
	\epsilon [ \alpha (e^{i \omega} + e^{-2i \omega})  - \pp(i -1)] \pi_2 b 
	+ \alpha \epsilon  \pi_1 \bigl(  [ U_{\omega} b ] * c + [ U_{\omega} c ]*b \bigr)  \\
	%
	&= \epsilon [ (\alpha - \pp) (e^{i \omega} + e^{-2i \omega})  ]  \pi_2 b  +
	\epsilon  \pp [  (e^{i \omega} + e^{-2i \omega})  - (i -1)]  \pi_2 b  \nonumber \\
	& \qquad\quad + \alpha \epsilon \pi_1 \bigl(  [ U_{\omega} b ] * c + [ U_{\omega} c ]*b \bigr)  ,
	\end{alignat*}
hence, for $\|b\| \leq 1$,
%\begin{equation*}
%	\left| 
%	\frac{\partial F_1 }{\partial c } (x) \cdot b -  \pp \epsilon (i-1) \pi_2 b 
%	\right| 
%	\leq
%  \epsilon \left( 2 \da + \pp(\dt + \dtt ) + 2 \alpha \dc   \right)  .
%	\end{equation*}
%\note[JB]{Because of the factor 2 in the norm I think it should be }
\begin{equation*}
	\left| 
	\frac{\partial F_1 }{\partial c } (x) \cdot b -  \pp \epsilon (i-1) \pi_2 b 
	\right| 
	\leq
  \epsilon \left(  \da + \tfrac{\pi}{4}(\dw + 2 \dw ) +  (\pp + \da )  \dc   \right)  .
	\end{equation*}
\end{proof}


%%%%%%%%%%%%%%%%%%%%%%%%%%%%%%%%%%%%%%%%%%%%%%%%%%%%%%%%%%%%%%%%%%%%%%%%%%%%



%%%%%%%%%%%%%%%%%%%%%%%%%%%%%%%%%%%%%%%%%%%%%%%%%%%%%%%%%%%%%%%%%%%%%%%%%%%%
\begin{proposition}
		\label{prop:Zsa}
Define  
	% \[
	% f_{*,\alpha} := \frac{2}{\pi \sqrt{5}} \left( r_c + \epsilon \dt + \epsilon \dc (4 + \dc )  \right) .
	% \]
	% \note[J]{Proposed Change}
		\[
		f_{*,\alpha} := \frac{2}{\pi \sqrt{5}}\left( r_c +  2 \dw (  \dc^0 +  \epsilon )  + \epsilon \dc (4 + \dc )  \right)  .
		\]
	Then for all $x= (\alpha,\omega,c) \in B_\epsilon(r,\rho)$
	\[
		f_{*,\alpha} \geq  \left\| A_{0,*}^{-1} \pi_{\geq 2} \left(
		\frac{\partial F}{\partial \alpha}(x)  +   \epsilon  \tfrac{2 +4 i}{5}  \e_2  \right) \right\|  .
	\]
\end{proposition}

\begin{proof}
We note that $\epsilon  \tfrac{2 +4 i}{5} \e_2= \bc_\epsilon + \epsilon i \e_2$
and calculate 
%	\[
%\pi_{\geq 2} 
%		\frac{\partial F}{\partial \alpha}(x)  +   \epsilon  \tfrac{2 +4 i}{5}  \e_2  =
%	U_\omega (c-\bc_\epsilon) + \epsilon  (e^{-i \omega} +i)\e_2 + \epsilon \pi_{\geq 2} L_\omega c + \epsilon  \pi_{\geq 2} ([ U_\omega c] * c ).
%	\]
%\note[JB]{I don't see that this is correct. It seems a term is missing:}
	\[
\pi_{\geq 2} 
		\frac{\partial F}{\partial \alpha}(x)  +   \epsilon  \tfrac{2 +4 i}{5}  \e_2  =
	U_\omega (c-\bc_\epsilon) + (1+ e^{-2i\omega}) \bc_\epsilon + \epsilon  (e^{-i \omega} +i)\e_2 + \epsilon \pi_{\geq 2} L_\omega c + \epsilon  \pi_{\geq 2} ([ U_\omega c] * c ).
	\]
By using Proposition~\ref{p:severalnorms} and Lemma~\ref{lem:deltatheta}, we obtain the estimate
%\begin{alignat*}{1}
%		\left\| A_{0,*}^{-1} \pi_{\geq 2} \left(
%				\frac{\partial F}{\partial \alpha}(x)  +   \epsilon  \tfrac{2 +4 i}{5}  \e_2  \right) \right\|
%	 &\leq \| A_{0,*}^{-1} \| \left( r_c + \epsilon | e^{-i \omega } +i| + 4 \epsilon \dc + \epsilon \dc^2 \right) \\
%	&\leq \frac{2}{\pi \sqrt{5}}\left( r_c + \epsilon \dt + \epsilon \dc (4 + \dc )  \right) .
%	\end{alignat*}
%\note[JB]{Because of the factor 2 in the norm, and the term I thought was missing, I think it should be }
%\note[J]{Yes, a term was missing.} 
% \begin{alignat*}{1}
% 		\left\| A_{0,*}^{-1} \pi_{\geq 2} \left(
% 				\frac{\partial F}{\partial \alpha}(x)  +   \epsilon  \tfrac{2 +4 i}{5}  \e_2  \right) \right\|
% 	 &\leq \| A_{0,*}^{-1} \| \left( r_c + 2 \epsilon  \frac{\sqrt{20}}{5} |1+ e^{-2i\omega} | +  2 \epsilon | e^{-i \omega } +i| + 4 \epsilon \dc + \epsilon \dc^2 \right) \\
% 	&\leq \frac{2}{\pi \sqrt{5}}\left( r_c + \epsilon \dtt \frac{4}{\sqrt{5}} + 2 \epsilon \dt + \epsilon \dc (4 + \dc )  \right) .
% 	\end{alignat*}
% 	\note[J]{Rearranged some terms}
	\begin{alignat*}{1}
	\left\| A_{0,*}^{-1} \pi_{\geq 2} \left(
	\frac{\partial F}{\partial \alpha}(x)  +   \epsilon  \tfrac{2 +4 i}{5}  \e_2  \right) \right\|
	&\leq \| A_{0,*}^{-1} \| \left( r_c +  \dc^0 |1+ e^{-2i\omega} | +  2 \epsilon | e^{-i \omega } +i| + 4 \epsilon \dc + \epsilon \dc^2 \right) \\
	&\leq \frac{2}{\pi \sqrt{5}}\left( r_c +  2 \dw (  \dc^0 +  \epsilon )  + \epsilon \dc (4 + \dc )  \right) .
	\end{alignat*}
\end{proof}
%%%%%%%%%%%%%%%%%%%%%%%%%%%%%%%%%%%%%%%%%%%%%%%%%%%%%%%%%%%%%%%%%%%%%%%%%%%%




\begin{proposition}
			\label{prop:Zsw}
Define 
%	\begin{alignat}{1}
%		f_{*,\omega} &:=
%		\frac{5}{2 \pi}  r_c 
%		+
%		\tfrac{\epsilon }{\sqrt{5}} \left(  \dt + \tfrac{2}{\pi}\da \right) \nonumber
%		+
%		\tfrac{2}{\pi} \left( \tfrac{5}{4} (\pp + \da ) r_c + \tfrac{4}{5}\da \epsilon \right) 
%		+
%		\tfrac{4 \epsilon}{5} (\dtt) 
%		\\& \qquad + 
%	   \epsilon  \tfrac{2}{ \pi}(\pp + 	 \da ) \left( \frac{1}{\sqrt{5}} ( \dc + r_c) + \frac{5}{4} \left(  \dc  + \tfrac{3}{2}r_c \right)  +  \frac{ \rho \dc   }{\sqrt{5}}\right) . \label{e:fstaromega}
%	\end{alignat}
%	\note[J]{Proposed Change}
%		\begin{alignat}{1}
%		f_{*,\omega} &:=
%		\frac{5}{2 \pi}  r_c 
%		+
%		\tfrac{2}{\sqrt{5}}  \epsilon \left(  \dw + \tfrac{2}{\pi}\da \right) \nonumber
%		+
%		\tfrac{5}{2\pi} \left( \pp r_c + \da ( r_c + \dc) \right) 
%		+
%		\tfrac{8}{5}  \epsilon \dw
%		\\& \qquad + 
%		  \tfrac{2}{ \pi} \epsilon(\pp + 	 \da ) \left( \frac{1}{\sqrt{5}} ( \dc + r_c) + \frac{5}{4} \left(  \dc  + \tfrac{3}{2}r_c \right)  +  \frac{ \rho \dc   }{\sqrt{5}}\right) . \label{e:fstaromega}
%		\end{alignat}
%	\note[JB]{Proposed Rearrangement}
		\begin{alignat}{1}
		f_{*,\omega} &:=
		\tfrac{5}{2 \pi}  (1+ \pp)  r_c 
		+
		\tfrac{2}{\sqrt{5}}  \epsilon \left( (1+\tfrac{4}{\sqrt{5}}) \dw + \tfrac{2}{\pi}\da \right) \nonumber
		+
		\tfrac{5}{2\pi} \da ( r_c + \dc) 
		\\& \qquad + 
		  \tfrac{2}{ \pi} \epsilon(\pp + 	 \da ) \left( \frac{1}{\sqrt{5}} ( \dc + r_c) + \frac{5}{4} \left(  \dc  + \tfrac{3}{2}r_c \right)  +  \frac{ \rho \dc   }{\sqrt{5}}\right) . \label{e:fstaromega}
		\end{alignat}
%
%	\[
%	f_{*,\omega} :=  \frac{5}{2 \pi} ( 1 + \alpha) r_c +  \epsilon \frac{2}{\pi \sqrt{5}} \left( \da + \pp \dt + \alpha   \rho \dc \right) + \frac{2 \alpha \epsilon}{\pi} \left(  \frac{1}{\sqrt{5}} ( \dc + r_c) + \frac{5}{4} \left(  \dc  + \frac{3}{2}r_c \right)  \right)  
%	\]
	Then for all $x= (\alpha,\omega,c) \in B_\epsilon(r,\rho)$
	\[
	f_{*,\omega} \geq  
	 \left\| A_{0,*}^{-1}  \pi_{\geq 2}\left( \frac{\partial F}{\partial \omega}(x) -  \epsilon \left[ \tfrac{4-3\pi}{10} + \tfrac{2(2 + \pi)}{5}i \right] \e_2  \right) \right\| .  
	\]
\end{proposition}

\begin{proof}
We note that 
$
\epsilon  \left[ \tfrac{4-3\pi}{10} + \tfrac{2(2 + \pi)}{5}i \right] \e_2
= i (2+\pi) \bc_\epsilon - \pp \epsilon \e_2
$
and calculate 
\begin{alignat*}{1}
\pi_{\geq 2} \frac{\partial F}{\partial \omega}(x) -  \epsilon \left[ \tfrac{4-3\pi}{10} + \tfrac{2(2 + \pi)}{5}i \right] \e_2   &=
		i K^{-1} ( I - \alpha U_{\omega} ) c  -
		 i  \alpha \epsilon e^{-i \omega} \e_2 + 
		\alpha \epsilon \pi_{\geq 2} L_{\omega}' c \\
		&\qquad - i \alpha \epsilon \pi_{\geq 2} ([ K^{-1} U_\omega c ] *c  ) - i K^{-1} ( I - \pp U_{\omega_0}) \bc_\epsilon +  \pp \epsilon \e_2\\
		&= i K^{-1} ( c - \bc_\epsilon) - \epsilon  (i  \alpha e^{-i \omega}  -  \pp ) \e_2\\
		&\qquad
		-i K^{-1} \left[  U_\omega \left(  \pp ( c - \bc_\epsilon) + ( \alpha - \pp) c  \right) + \left( U_\omega - U_{\omega_0} \right) \pp \bc_\epsilon) \right]
		\\
		&\qquad\qquad
			+ \alpha \epsilon \pi_{\geq 2} L_{\omega}' c 
			- i \alpha \epsilon \pi_{\geq 2} ([ K^{-1} U_\omega c ] *c ) .
\end{alignat*}	
Applying the operator $A_{0,*}^{-1}$ to this expression, we obtain (with $\hat{U}$ defined in~\eqref{e:defUhat})
%\note[J]{Yes, the term $(\alpha-\pp)c_2$ really should have been $(\alpha-\pp)c$.}
%	\[
%	A_{0,*}^{-1} \left( \frac{\partial F}{\partial \omega} ( x_\epsilon + a) - A   \right) =
%	\frac{2 	}{\pi} \hat{U}( I - \alpha U_{\omega}) (c - c_2(\epsilon)) + \epsilon A_{0,*}^{-1} \left( -i  [(\alpha-\pp) e^{_i \omega}]_2 -i [\pp ( e^{-i \omega} + i)]_2 + \alpha L_{\omega}' c - i \alpha [K^{-1} U_{\omega} c]* c 
%	\right)
%	\]
%	
\begin{alignat*}{1}
A_{0,*}^{-1}  \pi_{\geq 2}\left( \frac{\partial F}{\partial \omega}(x) -  \epsilon \left[ \tfrac{4-3\pi}{10} + \tfrac{2(2 + \pi)}{5}i \right] \e_2  \right) 
			&= \frac{2}{\pi} \hat{U}  ( c - \bc_\epsilon) 
			- \frac{2 \epsilon}{i \pi} \hat{U} K  (i  \alpha  e^{-i \omega}  -  \pp  ) \e_2 \\
			&\qquad
			-\frac{2}{\pi} \hat{U} \left[  U_\omega \left(  \alpha ( c - \bc_\epsilon) + ( \alpha - \pp) c  \right) \right]
		    \\	&\qquad\qquad
			-\frac{2}{\pi} \hat{U} \left( U_\omega - U_{\omega_0} \right) \pp \bc_\epsilon  
			\\
			&\qquad\qquad\qquad 
			+ \frac{2 \alpha \epsilon}{i \pi} \hat{U} K \pi_{\geq 2} \left(  L_{\omega}' c - i [ K^{-1} U_\omega c ] *c  \right)  .
	\end{alignat*}
We use the triangle inequality to estimate its norm, splitting it into the  five pieces:
% \note[JB]{There seems to be a factor 2 missing in the second and fourth estimate (due to norm); not implemented yet}
% \note[JB]{I think the second term of the third estimate is not correct: I think it should be $\frac{2}{\pi}\frac{5}{4} \da \dc$. But I did not implement it; I may be overlooking something.} \note[J]{I agree with these changes except for the factor of 2 in the fourth estimate. }
% 	\begin{alignat*}{1}
% 	\left\|	\frac{2}{\pi} \hat{U}  ( c - \bc_\epsilon) \right\|
% 				&\leq \frac{2}{\pi} \frac{5}{4} r_c
% 				= \frac{5}{2 \pi}  r_c \nonumber \\
% 				%
% 				%
% \left\|		- \frac{2 \epsilon}{i \pi} \hat{U} K  (i  \alpha  e^{-i \omega}  -  \pp  ) \e_2  \right\|
% 				&\leq   \frac{2 \epsilon }{\pi} \frac{1}{\sqrt{5}} \left( \pp \dw + \da \right)
% 				=   \tfrac{ \epsilon }{\sqrt{5}} \left(  \dt + \tfrac{2}{\pi}\da \right) \nonumber \\
% 				%
% 				%
% 	\left\| 	-\tfrac{2}{\pi} \hat{U} \left[  U_\omega \left(  \alpha ( c - \bc_\epsilon) + ( \alpha - \pp) c  \right) \right]  \right\|
% 				& \leq \tfrac{2}{\pi} \left( \tfrac{5}{4} \alpha r_c + \tfrac{2}{\sqrt{5}}\da \tfrac{2 \epsilon}{\sqrt{5}} \right)
% 				= \tfrac{2}{\pi} \left( \tfrac{5}{4} \alpha r_c + \tfrac{4}{5}\da \epsilon \right) \nonumber \\
% 				%
% 				%
% 		\left\| -\tfrac{2}{\pi} \hat{U}
% 		\left( U_\omega - U_{\omega_0} \right) \pp \bc_\epsilon   \right\|
% 				&\leq  \tfrac{2}{\pi}  \tfrac{2}{\sqrt{5}} \dtt \pp \tfrac{2 \epsilon}{\sqrt{5}}
% 				=   \tfrac{4 \epsilon}{5} \dtt \nonumber \\
% 				%
% 				%
% 		\left\|\tfrac{2 \alpha \epsilon}{i \pi} \hat{U} K \pi_{\geq 2} \left(  L_{\omega}' c - i [ K^{-1} U_\omega c ] *c  \right)  \right\|
% 		&\leq \frac{2 \alpha \epsilon}{ \pi}  \left( \| \hat{U}  K \pi_{\geq 2} L_{\omega}' c  \| +  \frac{ \rho \dc   }{\sqrt{5}}\right) ,
% 	\end{alignat*}
%
% \note[J]{Below are my changes}
	\begin{alignat*}{1}
	\left\|	\frac{2}{\pi} \hat{U}  ( c - \bc_\epsilon) \right\|
	&\leq \frac{2}{\pi} \frac{5}{4} r_c 
	= \frac{5}{2 \pi}  r_c \nonumber \\
	%
	%
	\left\|		- \frac{2 \epsilon}{i \pi} \hat{U} K  (i  \alpha  e^{-i \omega}  -  \pp  ) \e_2  \right\|
	&\leq   \frac{4 \epsilon }{\pi} \frac{1}{\sqrt{5}} \left( \pp \dw + \da \right)  
	=   \tfrac{2 \epsilon }{\sqrt{5}} \left(  \dw + \tfrac{2}{\pi}\da \right) \nonumber \\
	%
	%
	\left\| 	-\tfrac{2}{\pi} \hat{U} \left[  U_\omega \left(  \alpha ( c - \bc_\epsilon) + ( \alpha - \pp) c  \right) \right]  \right\|
	& \leq \tfrac{2}{\pi}  \tfrac{5}{4} \left(  (\pp + \da) r_c + \da \dc  \right)
	= \tfrac{5}{2\pi} \left( \pp r_c + \da ( r_c + \dc) \right) \nonumber \\
	%
	%
	\left\| -\tfrac{2}{\pi} \hat{U} 
	\left( U_\omega - U_{\omega_0} \right) \pp \bc_\epsilon   \right\| 
	&\leq  \tfrac{2}{\pi}  \tfrac{2}{\sqrt{5}} (2 \dw)  \pp \tfrac{2 \epsilon}{\sqrt{5}} 
	=   \tfrac{8 \epsilon}{5} \dw \nonumber \\
	%
	%
	\left\|\tfrac{2 \alpha \epsilon}{i \pi} \hat{U} K \pi_{\geq 2} \left(  L_{\omega}' c - i [ K^{-1} U_\omega c ] *c  \right)  \right\|
	&\leq \frac{2 \alpha \epsilon}{ \pi}  \left( \| \hat{U}  K \pi_{\geq 2} L_{\omega}' c  \| +  \frac{ \rho \dc   }{\sqrt{5}}\right) ,
	\end{alignat*} 
where we have used Proposition~\ref{p:severalnorms} and Lemma~\ref{lem:deltatheta}.
%%	Multiplying by the inverse and taking norms 
%%	\begin{eqnarray}
%%	\left\| A_{0,*}^{-1}  \frac{\partial F}{\partial \omega} ( x_\epsilon + a) - A   \right\| &\leq&
%%	\frac{5}{2 \pi} ( 1 + \alpha) r_c +  \epsilon \| A_{0,*}^{-1} \| ( \da + \pp | e^{-i \omega} +i |) + \frac{2 \alpha \epsilon}{\pi} \| \hat{U} K L_{\omega}' c \| + \alpha \epsilon \| A_{0,*}^{-1} \|  \rho \dc \nonumber \\
%%	&\leq & 
%%	\frac{5}{2 \pi} ( 1 + \alpha) r_c +  \epsilon \frac{2}{\pi \sqrt{5}} ( \da + \pp \dt + \alpha   \rho \dc ) + \frac{2 \alpha \epsilon}{\pi} \| \hat{U} K L_{\omega}' c \| \nonumber 
%%	\end{eqnarray}
Finally, we estimate 
	\begin{alignat}{1}
	\left\| \hat{U} K \pi_{\geq 2} L_{\omega}' c \right\| &= 
	\left \| \hat{U}  K \pi_{\geq 2} \left(- i \sigma^+( e^{- i \omega} I + K^{-1} U_{\omega}) + i \sigma^-(e^{i \omega} I - K^{-1} U_{\omega}) \right) c \right\| \nonumber \\
	&\leq \left \| \hat{U}  K \pi_{\geq 2}( \sigma^+ + \sigma^- ) c  \right \| + 
	\left \| \hat{U} \pi_{\geq 2} K ( \sigma^+ + \sigma^- ) K^{-1}  U_{\omega} c  \right \|  \nonumber  \\
	&\leq  \frac{1}{\sqrt{5}} ( \| \sigma^+ c\| + \|\pi_{\geq 2}\sigma^- c\|) + \frac{5}{4} \left( \| K \sigma^+ K^{-1} \| \dc  +  \| \pi_{\geq 2} K \sigma^- K^{-1} \| r_c \right) \nonumber  \\
	&\leq  \frac{1}{\sqrt{5}} ( \dc + r_c) + \frac{5}{4} \left(  \dc  + \frac{3}{2}r_c \right)  .  \label{e:longestimate}
	\end{alignat}
Hence, with $f_{*,\omega}$ as defined in~\eqref{e:fstaromega},
% 	So if we define $f_{*,\omega}$ as below
% 	\begin{eqnarray}
% f_{*,\omega} &:=&
% 	\frac{5}{2 \pi}  r_c
% +
%  \tfrac{\epsilon }{\sqrt{5}} \left(  \dt + \tfrac{2}{\pi}\da \right) \nonumber
% +
%  \tfrac{2}{\pi} \left( \tfrac{5}{4} \alpha r_c + \tfrac{4}{5}\da \epsilon \right)
%  +
%   \tfrac{4 \epsilon}{5} (\dtt)
%   \\&&+
%   \frac{2 \alpha \epsilon}{ \pi}  \left( \frac{1}{\sqrt{5}} ( \dc + r_c) + \frac{5}{4} \left(  \dc  + \tfrac{3}{2}r_c \right)  +  \frac{ \rho \dc   }{\sqrt{5}}\right)
% 	\end{eqnarray}
%
it follows that 	
	\[
	 \left\| A_{0,*}^{-1}  \pi_{\geq 2}\left( \frac{\partial F}{\partial \omega}(x) -  \epsilon \left[ \tfrac{4-3\pi}{10} + \tfrac{2(2 + \pi)}{5}i \right] \e_2  \right) \right\| \leq 	f_{*,\omega} .
		\]
	
	
	
%%%%	\begin{eqnarray}
%%%%	\left\| A_{0,*}^{-1}  \frac{\partial F}{\partial \omega} ( x_\epsilon + a) - A   \right\| &\leq & 
%%%%	\frac{5}{2 \pi} ( 1 + \alpha) r_c +  \epsilon \frac{2}{\pi \sqrt{5}} \left( \da + \pp \dt+ \alpha   \rho \dc \right) + \frac{2 \alpha \epsilon}{\pi} \left(  \frac{1}{\sqrt{5}} ( \dc + r_c) + \frac{5}{4} \left(  \dc  + \frac{3}{2}r_c \right)  \right)  \nonumber
%%%%	\end{eqnarray}
%%%%	Hence
%%%%	\[
%%%%	f_{*,\omega} \leq  \frac{5}{2 \pi} ( 1 + \alpha) r_c +  \epsilon \frac{2}{\pi \sqrt{5}} \left( \da + \pp \dt + \alpha   \rho \dc \right) + \frac{2 \alpha \epsilon}{\pi} \left(  \frac{1}{\sqrt{5}} ( \dc + r_c) + \frac{5}{4} \left(  \dc  + \frac{3}{2}r_c \right)  \right)  
%%%%	\]
\end{proof}


%%%%%%%%%%%%%%%%%%%%%%%%%%%%%%%%%%%%%%%%%%%%%%%%%%%%%%%%%%%%%%%%%%%%%%%%%%%%

\begin{proposition}
			\label{prop:Zsc}
Define 
% \note[J]{Below is the old bound.}
% 	\[
% 	f_{*,c} := \left[ \frac{5}{2} ( \tfrac{1}{2} + \tfrac{1}{\pi}) \dw +  \frac{\da }{\sqrt{5}} \right]
% 	+\epsilon \left[ \frac{8}{\pi \sqrt{5}} \da + \frac{2}{\sqrt{5}} \dt  + \frac{25}{8} \dw + \frac{2 (\pp+\da)  \dc}{\pi \sqrt{5}} \right]  .
% 	\]
% \note[J]{Below is the new bound. }
	\[
	f_{*,c} :=	\left[ \frac{5}{2} \left( \frac{1}{2} + \frac{1}{\pi} \right) \dw +  \frac{\da }{\sqrt{5}} \right] 
 +\epsilon \left[ \frac{8}{\pi \sqrt{5}} \da + \left(  \frac{2}{\sqrt{5}}   + \frac{25}{8} \right) \dw + \frac{4 (\pp+\da)  \dc}{\pi \sqrt{5}} \right] .
	\]
	Then for all $x= (\alpha,\omega,c) \in B_\epsilon(r,\rho)$
	\[
	f_{*,c} \geq    \left\|  A_{0,*}^{-1} \pi_{\geq 2} \left( \frac{\partial F}{\partial c}(x) \cdot b  - (A_{0,*} + \epsilon A_{1,*}) b \right) 
	\right\| , \qquad\text{for all $b\in\ell^K_0$ with $\|b\| \leq 1$}.
	\]
\end{proposition}


\begin{proof}
We write $A_* := A_{0,*} + \epsilon A_{1,*}$ and calculate
\begin{alignat*}{1}
\frac{\partial F}{\partial c} (x) \cdot b - A_* b 
& = 
 \bigl[( i \omega K^{-1} + \alpha U_{\omega}) - ( i \pp K^{-1} + \pp U_{\omega_0}) \bigr] b  + \alpha \epsilon  L_\omega b  - \pp \epsilon L_{\omega_0} b 
 \\& \qquad
	+ \alpha \epsilon \left[ [ U_\omega b] * c + [U_{\omega} c ]*b \right]
 \\ & =
	\bigl[ i ( \omega - \pp ) K^{-1} + ( \alpha - \pp) U_{\omega} + \pp ( U_{\omega} - U_{\omega_0}) \bigr] b 
	\\ & \qquad 
	+ \epsilon \bigl[ ( \alpha - \pp) L_{\omega} + \pp ( L_{\omega} - L_{\omega_0}) \bigr] b 
%	\\ & \qquad\qquad 
+ \alpha \epsilon \left( [U_{\omega } b] * c + [ U_{\omega } c ]*b \right) . 
\end{alignat*}
Hence, for $\|b\| \leq 1$, 
\begin{alignat}{1}
	\left\| A_{0,*}^{-1} \pi_{\geq 2} \left(   \frac{\partial F}{\partial c} (x) \cdot b -  A_* b  \right)  \right\|
	 &\leq 
	 \dw  \| A_{0,*}^{-1} K^{-1} \| + \pp \da \| A_{0,*}^{-1}  \| + \pp \| A_{0,*}^{-1}  ( U_{\omega} - U_{\omega_0}) \| \nonumber  \\
	& \qquad
	 + \epsilon  \left[ 4  \da \|  A_{0,*}^{-1}   \| + \pp \| A_{0,*}^{-1} \pi_{\geq 2} ( L_{\omega} - L_{\omega_0}) \| 
%	\\ & \qquad \qquad 
	+  2 \alpha \dc \| A_{0,*}^{-1}  \|  \right]  \label{e:intermediate},
\end{alignat}	
where all norms should be interpreted as operators on $\ell^1_0$.
	Since
$\frac{\partial U_\omega}{\partial  \omega} = - i K^{-1} U_{\omega}$
and $ A_{0,*}^{-1} = \frac{2}{i\pi} \hat{U} K$, it follows from Proposition~\ref{p:severalnorms} that  
\begin{equation}\label{e:AUoUo0}
	\| A_{0,*}^{-1}  (U_{\omega} - U_{\omega_0})  \| \leq  \frac{2}{\pi}  \dw \| \hat{U} \| 
	= \frac{5}{2 \pi } \dw .
\end{equation}
Next, we compute
	\begin{alignat*}{1}
	L_{\omega} - L_{\omega_0}  &= \sigma^+ \left[ (e^{-i \omega} + i) I + (U_{\omega} - U_{\omega_0})\right] + \sigma^- \left[ (e^{i \omega} - i) I + (U_{\omega} - U_{\omega_0}) \right] \\
	&=  (e^{-i \omega} + i)  \sigma^+ - i e^{i\omega} (i+e^{-i \omega})\sigma^- 
	+ (\sigma^+ + \sigma^-) (U_{\omega} - U_{\omega_0}) .
\end{alignat*}
Analogous to~\eqref{e:longestimate} and~\eqref{e:AUoUo0} we infer that
\begin{equation*}
	\|  A_{0,*}^{-1} \pi_{\geq 2} ( L_{\omega} - L_{\omega_0} ) \| \leq  \frac{4}{\pi \sqrt{5}} |i+ e^{-i \omega} | + \frac{5}{\pi} \| \hat{U} \| \dw \\
	\leq  \frac{4 }{\pi \sqrt{5}}\dw    + \frac{25}{4 \pi}  \dw .
\end{equation*} 
Finally, by putting all estimates together and once again using Proposition~\ref{p:severalnorms}, it follows from~\eqref{e:intermediate} that 
%\note[JB]{Probably factor 2 missing in final term.}\note[J]{Added factor of 2. }
	\begin{alignat*}{1}
\left\| A_{0,*}^{-1} \pi_{\geq 2} \left(   \frac{\partial F}{\partial c} (x) \cdot b -  A_* b  \right)  \right\|
	% &\leq
	% \left[ \dw  \| A_{0,*}^{-1} K^{-1} \| + \pp \da \| A_{0,*}^{-1}  \| + \pp \| A_{0,*}^{-1}  ( U_{\omega} - U_{\omega_0}) \| \right] \nonumber \\
	% & + \epsilon  \left[ 4  \da \|  A_{0,*}^{-1}   \| + \pp \| A_{0,*}^{-1}  ( L_{\omega} - L_{\omega_0}) \| + \alpha \dc \| A_{0,*}^{-1}  \|  \right] \\
	&\leq
	 \left[ \frac{5}{2} \left( \frac{1}{2} + \frac{1}{\pi} \right) \dw +  \frac{\da }{\sqrt{5}} \right] 
	 \\
	& \qquad  +\epsilon \left[ \frac{8}{\pi \sqrt{5}} \da + \left(  \frac{2}{\sqrt{5}}   + \frac{25}{8} \right) \dw + \frac{4 (\pp+\da)  \dc}{\pi \sqrt{5}} \right] .
	\end{alignat*}
\end{proof}

%%%%%%%%%%%%%%%%%%%%%%%%%%%%%%%%%%%%%%%%%%%%%%%%%%%%%%%%%%%%%%%%%%%%%%%%%%%%



%!TEX root = hopfwright.tex

%%%%%%%%%%%%%%%%%%
%%% Appendix E %%%
%%%%%%%%%%%%%%%%%%

\section{Appendix: A priori bounds on periodic orbits}
\label{appendix:aprioribounds}

In order to isolate periodic orbits, we need to separate them from the trivial solution. In this appendix we prove some lower bounds on the size of periodic orbits. First we work in the original Fourier coordinates. Then we derive refined bounds in rescaled coordinates.

Recall that periodic orbits of Wright's equation corresponds to  zeros of 
$G(\alpha,\omega,\c)=0$, as defined in~\eqref{e:defG}. Clearly $G(\alpha,\omega,0)=0$ for all frequencies $\omega>0$ and parameter values $\alpha>0$. There are bifurcations from this trivial solution for $\alpha=\alpha_n:=\pp(4n+1)$ for all $n\geq 0$. The corresponding natural frequency is $\omega=\alpha_n$, but there are bifurcations for any $\omega = \alpha_n/ \tilde{n}$ with $\tilde{n} \in \N$ as well, which are essentially copies of the primary bifurcation. The following proposition quantifies that away from these bifurcation points the trivial solution is isolated.
% \begin{proposition}
% 	\label{prop:zeroneighborhood}
% 	Suppose $G(\alpha,\omega,\c)=0$ for some $\alpha,\omega >0$.
% Then either $\c \equiv 0$ or
% 	\begin{equation}
% 	\| \c \|^2 \geq \frac{1}{4\alpha^2} \min_{k \in \N} (\alpha-k\omega)^2 + 2 \alpha k \omega ( 1- \sin k \omega )
% 	\end{equation}
% \end{proposition}
%
% \begin{proof}
% We fix $\alpha,\omega>0$ and define
% \[
%   \beta_1 := \min_{k \in \N} (\alpha-k\omega)^2 + 2 \alpha k \omega ( 1- \sin k \omega ).
% \]
% If $\beta_1=0$ there is nothing to prove. From now on we assume that $\beta_1>0$.
% We define a Newton-like map
% $N : \ell^1 \to \ell^1$ by
% \begin{equation}
% N(\c) := \c - (i \omega K^{-1} + \alpha U_{\omega})^{-1} G(\alpha,\omega,\c).
% \end{equation}
% We note that $i \omega K^{-1} + \alpha U_{\omega}$, which is the derivative of $G$ at $\c=0$, is invertible, since
% for any $k \in \N$
% \begin{alignat*}{1}
%   |ik\omega + \alpha e^{-i k \omega}|^2
%   & =  (\alpha \cos k \omega )^2 + ( \omega - \alpha \sin  k \omega )^2  \\
%   &  = (k \omega)^2 + \alpha^2 - 2 \alpha  k \omega \sin k \omega   \\
% &= (\alpha - k\omega)^2 + 2 \alpha k \omega ( 1 - \sin k \omega)  \\
% &\geq \beta_1 >0.
% \end{alignat*}
% Fixed points of $N$ thus correspond to zeros of $G$.
% Naturally, $ N(0) =0$.
% Next we show that $N$ is a contraction map on any ball $B_R := \{ \c \in \ell^1 : \|\c\| \leq R \} $ with $R < \beta_1^{1/2} / (2\alpha)$. We then apply the Banach fixed point theorem to conclude that there are no nontrivial fixed points with $\|\c\| < \beta_1^{1/2} / (2\alpha)$.
% The derivative of $N$ is
% \begin{equation}
% DN(\c)\tilde{\c} =  - \alpha (i \omega K^{-1} + \alpha U_{\omega})^{-1}  (
% [U_\omega \c ]  * \tilde{\c} ) .
% \end{equation}
% We estimate
% \[
% \| DN(\c)\tilde{\c} \| \leq 2 \alpha \|\c \| \cdot \|\tilde{\c}\| \cdot
% \| \omega K^{-1} +  \alpha U_{\omega})^{-1} \| .
% \]
% Since $\| ( \omega K^{-1} + \alpha U_{\omega})^{-1} \| = \beta_1^{-1/2}$, we find that
% $\| DN(\c)\| \leq 2 \alpha R \beta_1^{-1/2} < 1$ for all $\c \in B_R$.
% Hence $N$ is a contraction for $R < \beta_1^{1/2}/ (2\alpha)$.
% \end{proof}

\begin{proposition}
	\label{prop:zeroneighborhood2}
	Suppose $G(\alpha,\omega,\c)=0$ for some $\alpha,\omega >0$. 
Then either $\c \equiv 0$ or 
	\begin{equation}\label{e:minoverk}
	\| \c \| \geq \min_{k \in \N}  \sqrt{ \left(1-k \,\frac{\omega}{\alpha} \right)^2 + 2  k \, \frac{\omega}{\alpha} \bigl( 1- \sin k \omega \bigr)} .
	\end{equation}	
\end{proposition}

\begin{proof}
We fix $\alpha,\omega>0$ and define 
\[
  \beta_1 := \min_{k \in \N} \, (\alpha-k\omega)^2 + 2 \alpha k \omega ( 1- \sin k \omega ).
\]
If $\beta_1=0$ then there is nothing to prove. From now on we assume that $\beta_1>0$. 
We recall that 
\[ G(\alpha,\omega,\c) = (i \omega K^{-1} + \alpha U_{\omega}) \c + \alpha \left[U_\omega \, \c \right] * \c .
\]
We note that $i \omega K^{-1} + \alpha U_{\omega}$ is invertible, since
for any $k \in \N$
\begin{alignat*}{1}
  |ik\omega + \alpha e^{-i k \omega}|^2
  & =  (\alpha \cos k \omega )^2 + ( \omega - \alpha \sin  k \omega )^2  \\
  &  = (k \omega)^2 + \alpha^2 - 2 \alpha  k \omega \sin k \omega   \\
&= (\alpha - k\omega)^2 + 2 \alpha k \omega ( 1 - \sin k \omega)  \\
&\geq \beta_1 >0.
\end{alignat*}
We may thus rewrite $G(\alpha,\omega,\c) = 0$ as 
\begin{equation}\label{e:quadratic}
 \c = - \alpha  (i \omega K^{-1} + \alpha U_{\omega})^{-1} ( \left[U_\omega \, \c \right] * \c ) .
\end{equation}
Since $\| ( \omega K^{-1} + \alpha U_{\omega})^{-1} \| = \beta_1^{-1/2}$
and $\| \left[U_\omega \, \c \right] * \c \| \leq \| \c \|^2 $,
we infer from~\eqref{e:quadratic} that
\[
  \| \c \| \leq  \alpha \beta_1^{-1/2} \|\c\|^2.
\]
We conclude that either $\c \equiv 0$ or $\| \c \| \geq \beta_1^{1/2} /\alpha $.
\end{proof}

\begin{proposition}
	\label{prop:G1Minimizer}
	Suppose that $ \omega \geq 1.1$ and $ \alpha \in (0,2]$. Define 
	\begin{equation}
g_k(\omega,\alpha) =		\left(1-k \,\tfrac{\omega}{\alpha} \right)^2 + 2  k \, \tfrac{\omega}{\alpha} \bigl( 1- \sin k \omega \bigr) .
	\end{equation}
Then $ g_1 <g_k$ for all $ k \geq 2$. 
\end{proposition}

\begin{proof}
%%%	We first prove that $ g_k > g_1$ for all $ k\geq 3$.   
%%%	For all $k$ we may estimate $ g_k > (1 - k \tfrac{\omega}{\alpha})^2$. 	
%%%	If $ \omega \in [1.1,2]$  then $1- \sin ( \omega ) \in [0,0.11]$, and so 
%%%	\[
%%%	g_1 < ( 1- \tfrac{\omega}{\alpha})^2 + 0.22 \tfrac{\omega}{\alpha}
%%%	\]
%%%	If we write $ x = \tfrac{\omega}{\alpha}$ then these estimates produce:
%%%	\begin{eqnarray}
%%%		g_k &>& (kx)^2 - 2 k x +1  \\
%%%		g_1 &<& x^2 - 1.78x +1 
%%%	\end{eqnarray}
%%%	Hence we can show $ g_k > g_1$ by showing 
%%%	\[
%%%	x - 1.78 < k^2 x - 2 k
%%%	\]
%%%	or equivalently 
%%%\begin{equation}
%%%		(2k-1.78) < (k^2-1)x  \label{eq:G3G1}
%%%\end{equation}
%%%	For our range of $ \alpha$ and $\omega$ it follows that $  x \in [0.55, 4/3]$.  
%%%	Since Equation \ref{eq:G3G1} is true for $ x = 0.55$ and $ k \geq 3$, then it follows that $ g_1 < g_k$ for all $ k \geq 3$. 
%%%	
%%%	We now show that $ g_1 < g_2$. 
%%%	
%%	
%%	
	
This is equivalent to showing that 
\[
(1- \tfrac{\omega}{\alpha})^2 + 2 \tfrac{\omega}{\alpha} ( 1 - \sin \omega ) 
<
(1- k \tfrac{\omega}{\alpha})^2 + 2k  \tfrac{\omega}{\alpha} ( 1 - \sin k\omega )  
\qquad\text{for } k\geq 2.
\]
Making the substitution $ x = \tfrac{\omega}{\alpha}$, we can simplify this to the equivalent inequality 
% \begin{eqnarray}
% (1 - 2 x + x^2 ) + 2 x ( 1 - \sin \omega )
% &<&
% (1 - 2k x + k^2 x^2 ) + 2k x ( 1 - \sin k\omega )
% \\
% 2 x ( 1 - \sin \omega )
% &<&
% - 2(k-1) x + (k^2-1)x^2  + 2k x ( 1 - \sin 2\omega )
% \\
% -2  \sin \omega
% &<&
\[
 (k^2-1) x  + 2  \sin \omega - 2k  \sin k\omega >0 .
\]
% \end{eqnarray}
Since $\alpha \leq 2$, we have $x\geq \omega/2$. Hence it suffices to prove that
%
%
% $ x = \tfrac{\omega}{\alpha} $ is minimized for large $\alpha$, we may simplify this as below using our initial range of $ \alpha \in [1.5,2]$.
\begin{equation}
  h_k(\omega) := \frac{k^2-1}{2} \omega + 2 \sin \omega  - 2 k \sin k\omega  >0 
  \qquad\text{for all } k\geq 2.
\label{eq:GminRedux}
\end{equation}
We first consider $k=2$. It is clear that $h_2(\omega) > 0$ for $\omega > 4$.
We note that $h_2$ has a simple zero at $ \omega \approx 1.07146$ and it is easy to check using interval arithmetic that $h_2(\omega)$ is positive for $\omega \in [1.1,4]$. Hence $h_2(\omega) > 0$ for all $\omega \geq 1.1$.


For $k=3$ and $k=4$ we can repeat a similar argument. For $k \geq 5$ it is immediate that $h_k(\omega) > \frac{k^2-1}{2}-2-2k \geq 0$ for $\omega > 1$.
%
% One can further check that the the RHS of Equation \ref{eq:GminRedux} is positive for all $ k \geq 2 $ and $ \omega \in [1.1,2]$.
% Thereby, it follows that
% \[
% g_1(\omega,\alpha) < g_{k} (\omega,\alpha)
% \]
% for $ \alpha \in [1.5,2]$ and $ \omega \in [1.1,2]$.
%
%
%
\end{proof}


As discussed in Section~\ref{s:preliminaries},
the function $G(\alpha,\omega,\c)$ gets replaced by $\tF_\epsilon(\alpha,\omega,\tc)$ in rescaled coordinates. 
In these coordinates we derive a result analogous to Proposition~\ref{prop:zeroneighborhood2} below, see Lemma~\ref{lem:thecone}.
First we bound the inverse of the operator $\B \in B(\ell^1_0)$ defined by
%\marginpar{NOTE THAT $\B = \alpha^{-1} B K $}
\[
  \B:= i \frac{\omega}{\alpha} I +  U_{\omega} K +  \epsilon L_{\omega} K,
\]
where $K$, $U_{\omega}$ and $L_\omega$ have been introduced in Section~\ref{s:preliminaries}.
\begin{lemma}\label{lem:gamma}
	Let $\epsilon \geq 0$ and $\alpha,\omega>0$. Let
\[
  \gamma := 	  \frac{1}{2} +
  \epsilon \left( \frac{2}{3} + \max\left\{  \frac{\sqrt{2 - 2 \sin (\omega-\pp) }}{2} ,\frac{2}{3} \right\} \right).
\] 
If $\gamma < \omega / \alpha$ then the operator $\B$ is invertible and the inverse is bounded by
\[
	  \| \B^{-1} \| \leq \frac{1}{\frac{\omega}{\alpha}- \gamma}.
\]
\end{lemma} 

\begin{proof}
Writing 
\[
  \B= i \frac{\omega}{\alpha} \left( I + \frac{\alpha}{i\omega} \left( U_{\omega} +  \epsilon L_{\omega} \right) K \right)
\]
and using a (formal) Neumann series argument, we obtain
% \marginpar{Side remark: did you consider the splitting $\B = i
% \frac{\omega}{\alpha} [I+ \frac{\alpha}{i\omega} \epsilon L_\omega K Q^{-1}] Q
% $ with $Q=I+\frac{\alpha}{i\omega} U_\omega K $ ? Using the estimate from
% Prop~\ref{prop:zeroneighborhood2} this may or may not lead to a better bound.}
\[
  \| \B^{-1} \| \leq \frac{\alpha}{\omega}
   \sum_{n=0}^\infty \left( \frac{ \alpha}{ \omega} \right)^n \|(U_{\omega} + \epsilon L_{\omega}) K\|^n
   \leq \frac{\frac{\alpha}{\omega} }{1- \frac{ \alpha}{ \omega} \|(U_{\omega} + \epsilon L_{\omega}) K\|} 
   = \frac{1}{\frac{\omega}{\alpha}- \|(U_{\omega} + \epsilon L_{\omega}) K\|} .
\]
It remains to prove the estimate $\|(U_{\omega} + \epsilon L_{\omega}) K\| \leq \gamma$.
Then, in particular, for $\gamma < \omega/\alpha$ the formal argument is rigorous.

Recalling that $L_\omega= \sigma^+ (e^{-i\omega} I  + U_\omega) + \sigma^- (e^{i\omega} I  + U_\omega)$, we use the triangle inequality
\[
\|(U_{\omega} + \epsilon L_{\omega}) K\| 
\leq \| U_\omega K \|  + \epsilon \| \sigma^+ (e^{-i\omega} I  + U_\omega) K \| + \epsilon \| \sigma^- (e^{i\omega} I  + U_\omega) K \|,
\] 
and estimate each term separately as operator on $\ell^1_0$. 
We recall the formula~\eqref{e:operatornorm} for the operator norm.
Using that $\| K \tc \| \leq \frac{1}{2} \|\tc\|$ for all $\tc \in \ell^1_0$,
the first term is bounded by $\| U_\omega K \| \leq \frac{1}{2}$.
Since $\sigma^-$ shifts the sequence to the left and we consider the operators acting on $\ell^1_0$, we obtain $\| \sigma^- (e^{i\omega} I  + U_\omega) K \| \leq \frac{2}{3}$.
For the final term, $\| \sigma^+ (e^{-i\omega} I  + U_\omega) K \|$, 
to obtain a slightly more refined estimate,
we first consider the action of $\sigma^+ (e^{-i\omega} I  + U_\omega) K$ 
on $\e_2$. We observe that 
\[ 
  | e^{- i \omega} + e^{-2 i \omega}| = \sqrt{2 - 2 \sin (\omega-\pp) },
\]
hence $\| \sigma^+ (e^{-i\omega} I  + U_\omega) K \e_2 \| \leq 
\sqrt{2 - 2 \sin (\omega-\pp) } $, 
leading to
\[
 \| \sigma^+ (e^{-i\omega} I  + U_\omega) K \| 
 \leq \max\left\{  \frac{\sqrt{2 - 2 \sin (\omega-\pp) }}{2} ,\frac{2}{3} \right\}.
\]
We conclude that 
\[
\|(U_{\omega} + \epsilon L_{\omega}) K\| 
\leq 
  \frac{1}{2} +
  \epsilon \left( \frac{2}{3} + \max\left\{  \frac{\sqrt{2 - 2 \sin (\omega-\pp) }}{2} ,\frac{2}{3} \right\} \right).
\]
\end{proof}

\begin{lemma}\label{lem:thecone}
	Fix $ \epsilon \geq 0$, $\alpha,\omega>0$.
	Assume that $\B$ is invertible.
	Let $b_0$ be a bound on $\| \B^{-1} \|$.
	Define 
	\[
	z^{\pm} = b_0^{-1} \pm \sqrt{b_0^{-2}-  2\epsilon^2 } .
	\]
	Let $ \tc \in \ell^1_0$ be such that $\tF_\epsilon(\alpha, \omega,\tc) = 0$, then either $ \|\tc\| \leq  z^-$ or $  \|\tc\| \geq z^+ $. 
	\noindent
	Additionally, $ \| K^{-1} \tc \| \leq b_0 (2\epsilon^2+ \|\tc\|^2)$.
\end{lemma}

\begin{proof}
	If  $ \tF_\epsilon( \alpha, \omega, \tc) =0$ then it follows that the equations $\pi_c \tF_\epsilon=0$ can be rearranged as 
\begin{equation}\label{e:eBc2}
  \tc = - K \B^{-1} (  \epsilon^2  e^{- i \omega} \e_2 + \ [ U_{\omega} \tc ] * \tc ) .
\end{equation}
	Taking norms, and using that $\| K \tc \| \leq \frac{1}{2} \| \tc\|$ for all $\tc\in \ell^1_0$, we obtain 
\begin{equation}\label{e:quadineq}
\|\tc \|  \leq  \frac{1}{2} \| B^{-1}\| \left( \epsilon^2 \|\e_2\|  + \| [ U_{\omega} \tc ] * \tc \| \right)
\leq \frac{1}{2} b_0 \left( 2 \epsilon^2  + \| \tc \|^2 \right).
\end{equation}
The quadratic $x^2 - 2 b_0^{-1} x +   2\epsilon^2 $
has two zeros $z^+$ and $ z^-$ given by
	\[
	z^{\pm} = b_0^{-1} \pm \sqrt{b_0^{-2}-  2\epsilon^2 } .
	\]
The inequality~\eqref{e:quadineq} thus implies that either $ \|\tc\| \leq z^-$ or $ \|\tc\| \geq z^+$.

Furthermore, it follows from~\eqref{e:eBc2} that $\| K^{-1} \tc \| \leq  \| \B^{-1} \| \, (2 \epsilon^2 + \|\tc \|^2) \leq b_0 (2 \epsilon^2+ \|\tc\|^2)$.
\end{proof}

In practice we use the bound $\| \B^{-1} \| \leq b_*^{-1}$,
where
\[
  b_*(\epsilon) := \frac{\omega}{\alpha} - \frac{1}{2} - \epsilon  \left(\frac{2}{3}+ \frac{1}{2}\sqrt{2 + 2 |\omega-\pp| } \right).
\]
When doing so, we will refer to $z^\pm$ as $ z^\pm_*$. 
Additionally, we will need the following monotonicity property.
\begin{lemma}
	\label{lem:ZminusBound}
	Fix $\alpha, \omega, \epsilon_0 >0$ and assume that $ \epsilon_0 \leq b_*(\epsilon_0) /\sqrt{2}$.
	Define 
\[
  z_*^- (\epsilon):= b_*(\epsilon)-\sqrt{(b_*(\epsilon))^2 -2 \epsilon^2}.
 \]
Let $C_0 := \frac{z_*^-(\epsilon_0)}{\epsilon_0}$.
Then
	\begin{equation}
	 z_*^-(\epsilon) \leq C_0 \epsilon
	 \qquad \text{for all } 0 \leq \epsilon \leq \epsilon_0.
	 \label{eq:ConeLemma}
	\end{equation}
\end{lemma}


\begin{proof}
	Let $x:=\sqrt{2} \epsilon/b_*(\epsilon) \geq 0$. Clearly $\frac{d}{d\epsilon} x >0$.
It thus suffices to observe that
\[
  \frac{z_*^- (\epsilon)}{\epsilon} = \sqrt{2}\,\frac{1 - \sqrt{1-x^2}}{x}
\]	
is increasing for $x \in [0,1]$. 
%\marginpar{or $y-\sqrt{y^2-1}$ is decreasing}
%
% Throughout, we write $ b_* := b_*(\epsilon)$ and
% 	$ z_*^- =z_*^-(\epsilon)$.
% 	Writing $x=\epsilon/b_*(\epsilon)$
% 	First we may rewrite $ z_*^-$ as follows:
% 	\begin{eqnarray}
% 	 z_*^- &=& b_*-\sqrt{(b_*)^2 -\epsilon^2} \\
% 	 &=& b_*\left(1 - \sqrt{1-(\epsilon/b_*)^2} \right)
% 	\end{eqnarray}
% 	By assumption $|\epsilon_0/ b_*| <1 $ so for all $0 \leq  \epsilon \leq \epsilon_0$ the following Taylor expansion is valid:
% 	\begin{equation}
% 		z_*^-(\epsilon) = \frac{\epsilon^2}{2 b_*} + \frac{\epsilon^4}{8 (b_*)^3} + \frac{\epsilon^6}{16 (b_*)^5} + \dots
% 	\end{equation}
% In particular, we note that $z_*^-(\epsilon)$ can be expressed as $b_*$ times a power series in $(\epsilon/b_*)^2$ with strictly positive coefficients.
% 	Since $ \tfrac{d}{d \epsilon} b_*(\epsilon) = - \left( \tfrac{2}{3} + \tfrac{1}{2} \sqrt{2+2|\omega - \pp|}\right) < 0  $, then it follows that all of the functions:
% 	\begin{align}
% 		z_*^-(\epsilon), && \frac{z_*^-(\epsilon)}{\epsilon}, &&\frac{z_*^-(\epsilon)}{\epsilon^2}
% 	\end{align}
% 	are well defined at $ \epsilon=0$ and monotonically increasing in $\epsilon$.
% 	Hence, for all $ 0 \leq \epsilon < \epsilon_0$,  we have the inequality
% 	\[
% 	\frac{z_*^-(\epsilon)}{\epsilon} < 	\frac{z_*^-(\epsilon_0)}{\epsilon_0}
% 	\]
% 	whereby Equation \ref{eq:ConeLemma} follows.
%
\end{proof}

%%%%%%%%%%%%%%%%%%%%%%%%%%%%%%%%%%%%%%%%%%%%%%%%%%%%%%%%%%%%%%%%%%%%%


%!TEX root = hopfwright.tex

%%%%%%%%%%%%%%%%%%
%%% Appendix F %%%
%%%%%%%%%%%%%%%%%%

\section{Appendix: Implicit Differentiation}
\label{sec:Appendix_Implicit_Diff}

%\note[JB]{I have shortened this initial paragraph considerably}.
% We calculate  $\frac{\partial F}{\partial  \epsilon} (x) $ to be:
% \[
% \frac{\partial F}{\partial  \epsilon}( x ) = \alpha e^{-i \omega} \e_2 + \alpha L_\omega c + \alpha [ U_\omega c] * c
% \]
% In keeping with our $ \cO(\epsilon^2)$ approximations, we will use
% \[
% \tilde{x}_\epsilon = ( \pp, \pp, c_2(\epsilon), 0, 0, \dots)
% \]
% as the center of our approximation.
% Furthermore, we can drop the $ \alpha [U_\omega c] * c$ term in our initial expansion.
% We thereby obtain an approximation which we will call $\Gamma$:
%
% \begin{eqnarray}
% \frac{\partial F}{\partial  \epsilon} (\bar{x}_\epsilon) + \cO(\epsilon^2) &=& [-\pp i]_2 + \pp L_{\omega_0} [ \tfrac{2-i}{5} \epsilon]_2 \\
% &=& [-\pp i]_2 + \pp \left[ \sigma^+ ( -i-1) + \sigma^-(i-1) \right] [\tfrac{2-i}{5} \epsilon]_2 \\
% \Gamma &:=& \pp[\tfrac{3i -1}{5} \epsilon]_1 - \pp [i]_2 - \pp[\tfrac{3+i}{5} \epsilon]_3
% \end{eqnarray}
%

We will approximate
\[
\frac{\partial F}{\partial  \epsilon}( x ) = \alpha e^{-i \omega} \e_2 + \alpha L_\omega c + \alpha [ U_\omega c] * c 
\]
by
\begin{alignat}{1}
\Gamma & := \pp \tfrac{3i -1}{5} \epsilon \e_1 - \pp i \e_2 - \pp \tfrac{3+i}{5} \epsilon \e_3 \label{e:defGamma}  \\
&=   -\pp i \e_2 + \pp L_{\omega_0} \bc_\epsilon , \label{e:defGamma2}
\end{alignat}
which has been chosen so that $\frac{\partial F}{\partial  \epsilon} (\pp,\pp,\bc_\epsilon) - \Gamma = \cO(\epsilon^2)$.
%\remove[JB]{Given our approximation $\Gamma$, we can then calculate $ A^\dagger \Gamma$, as we do in the following lemma:}
\begin{lemma}
	\label{lem:ImplicitApprox}
When we write 
$A^\dagger \Gamma = (\alpha', \omega', c') \in \R^2 \times \ell^K_0$, then 
% \note[JB]{shouldn't $\tfrac{1-2i}{5}$ be $\tfrac{1+2i}{5}$? I think the second expression for $c'$ is better.} \note[J]{Agreed}
\begin{alignat*}{1}
		\alpha ' &= - \tfrac{2}{5} ( \tfrac{3 \pi}{2}-1) \epsilon ,\\
		\omega ' &= \tfrac{2}{5} \epsilon , \\
%		c '	 &= \left[ (\tfrac{1-2i}{5}) - 
%		\tfrac{\epsilon^2}{25} ( \tfrac{29 - 22i}{5} + \tfrac{1 + 7 i }{2})
%		 \right] \e_2 + ( \tfrac{3i-1}{10} )\epsilon \, \e_3 . \\
		c '	 &= \left[ (\tfrac{1+2i}{5}) - 
		\epsilon^2 \tfrac{9 }{250} (7-i)
		 \right] \e_2 + \epsilon \tfrac{3i-1}{10} \, \e_3 .
	\end{alignat*}
\end{lemma}

\begin{proof}
	First we calculate the $ \alpha$ and $ \omega$ components of the image of $ A^\dagger $:	
%	\begin{eqnarray}
%	[ A^\dagger]_{\alpha,\omega}  &=& \left[ A_0^{-1} [ I - \epsilon A_1 A_0^{-1}] \right]_{\alpha,\omega} \\
%	&=&   A_{0,1}^{-1} [ I - \epsilon \pp L_{\omega_0}  A_{0,*}^{-1}] \\
%	&=&   A_{0,1}^{-1} [ I - \epsilon \sigma^{-} ( iI + U_{\omega_0} ) ( i K^{-1} + U_{\omega_0})^{-1}]\\ 
%	&=&  A_{0,1}^{-1} [ e_1^* - \epsilon ( \tfrac{3 +i}{5}) e_2^* ]
%	\end{eqnarray}
%\note[JB]{Notational cleanup:}
	\begin{alignat}{1}
	\pi_{\alpha,\omega} A^\dagger  &= A_{0,1}^{-1} i_\C^{-1} \pi_1 [ I - \epsilon A_1 A_0^{-1}] ] \nonumber \\
	&=   A_{0,1}^{-1} i_\C^{-1} \pi_1 [ I - \epsilon \pp L_{\omega_0}  A_{0,*}^{-1}] \nonumber \\
	&=   A_{0,1}^{-1} i_\C^{-1} \pi_1 [ I - \epsilon \sigma^{-} ( iI + U_{\omega_0} ) ( i K^{-1} + U_{\omega_0})^{-1}] \nonumber \\ 
	&=  A_{0,1}^{-1} i_\C^{-1} [ \pi_1- \epsilon ( \tfrac{3 +i}{5}) \pi_2 ].
	\label{e:piaoAdag}
	\end{alignat}
Here we have used projections $\pi_k a = a_k$ for $a=\{a_k\}_{k \geq 1} \in \ell^1$.
	We now calculate the $\alpha $ and $\omega$ components of $ A^\dagger \Gamma$. 
	It follows from~\eqref{e:defGamma} and~\eqref{e:piaoAdag} that
	\begin{alignat*}{1}
	\pi_{\alpha,\omega} A^\dagger \Gamma  
%	&= A_{0,1}^{-1} [ e_1^* - \epsilon ( \tfrac{3 +i}{5}) e_2^* ] \left(\pp[\tfrac{3i -1}{5} \epsilon]_1 - \pp [i]_2 - \pp[\tfrac{3+i}{5} \epsilon]_3 \right) \\
	&= A_{0,1}^{-1} i_\C^{-1} \left[ \pp \tfrac{3i -1}{5} \epsilon +  \pp  \tfrac{3 +i}{5}  i \epsilon \right]
	\\
	&= \tfrac{\pi \epsilon}{5} A_{0,1}^{-1} i_\C^{-1} ( 3 i -1) 
	 \\
	&= - \frac{2 \epsilon}{5}
	\left(
	\begin{array}{c}
	\tfrac{3\pi}{2}-1 \\
	-1
	\end{array}
	\right) .
	\end{alignat*}
	
	We now calculate 
%\note[JB]{Notational cleanup, but does not look pretty.} 
%	\[
%	\pi_c A^{\dagger} \Gamma  = A_{0,*}^{-1}  \pi_{\geq 2} [ I - \epsilon ( \e_2 [i_\C A_{1,2}A_{0,1}^{-1} i_\C^{-1} \pi_1] +A_{1,*} A_{0,*}^{-1} \pi_{\geq 2} ) ]\Gamma .
%	\]
%\note[JB]{It seems to me it is better to write this as}
\begin{equation}\label{e:picAdagGamma}
	\pi_c A^{\dagger} \Gamma  = A_{0,*}^{-1}  \pi_{\geq 2} [ I - \epsilon A_1 A_0^{-1} ]\Gamma ,
\end{equation}
where
$A_1 A_0^{-1}$ decomposes as
\begin{equation}\label{e:A1A0decomposition}
  A_1 A_0^{-1} = 
  \e_2 [i_\C A_{1,2}A_{0,1}^{-1} i_\C^{-1} \pi_1] 
  +A_{1,*} A_{0,*}^{-1}  \pi_{\geq 2} . 
\end{equation}
We first calculate  
% \note[JB]{Here I think the $\tfrac{1-2i}{5}$ should be $\tfrac{1+2i}{5}$, but I did not implement it yet.} \note[J]{I agree. I've changed it now.}
\begin{alignat}{1}
	A_{0,*}^{-1} \pi_{\geq 2} \Gamma &= \tfrac{2}{ \pi } ( i K^{-1} + U_{\omega_0} )^{-1}  [ - \pp i \e_2 - \pp \tfrac{3+i}{5}\epsilon \e_3 ] \nonumber\\
	&= -(2i-1)^{-1} \e_2 - (3i+i)^{-1} \tfrac{3+i}{5} \epsilon \e_3 \nonumber\\
	&= \tfrac{1+2i}{5} \e_2 + \epsilon \tfrac{ 3 i-1}{20} \e_3  \label{e:A0piGamma}. 
\end{alignat}

%\note[JB]{Rearranged the rest of the proof; I found it hard to follow the flow of the argument.}
Since $\Gamma$ has three nonzero components only, we next compute the action of $A_{0,*}^{-1}  \pi_{\geq 2} A_1 A_0^{-1} $ on each of these.
Taking into account the decomposition~\eqref{e:A1A0decomposition},
we first compute its action on $\lambda \e_1$ for $\lambda \in \C$.
After a straightforward but tedious calculation we obtain
%\note[JB]{Hope this is correct.} \note[J]{This checks out.}
\begin{alignat*}{1}
A_{0,*}^{-1}  \pi_{\geq 2} A_1 A_0^{-1}  \lambda \e_1
&= 
 [i_\C A_{1,2}A_{0,1}^{-1} i_\C^{-1} \lambda ] A_{0,*}^{-1} \e_2
\\
&= -\tfrac{2}{25\pi} \bigl[ (11+2i) \text{Re} \lambda  + (-6+8i) \text{Im} \lambda \bigr]  \e_2.
\end{alignat*}
Next, we compute the action of $A_{0,*}^{-1}  \pi_{\geq 2} A_1 A_0^{-1} $
on  $\e_k$ for $k=2,3$:
%\note[JB]{Hope these are correct} \note[J]{This also checks out. }
\begin{alignat*}{1}
A_{0,*}^{-1}  \pi_{\geq 2} A_1 A_0^{-1} \e_2
&=
 A_{0,*}^{-1} [ \pp \sigma^+ (  e^{-i\omega_0} I  + U_{\omega_0}) ]A_{0,*}^{-1} \e_2 
\\& 
%= - \tfrac{2}{\pi} ( 3i +i)^{-1} (-i-1)(2i-1)^{-1} \e_3 
=  \tfrac{2}{\pi} \tfrac{3+i}{20} \e_3  ,
  \\
A_{0,*}^{-1}  \pi_{\geq 2} A_1 A_0^{-1} \e_3
&=
 A_{0,*}^{-1} [ \pp \sigma^- ( e^{i\omega_0} I + U_{\omega_0}) ]A_{0,*}^{-1} \e_3  \\
& = 
-\tfrac{2}{\pi} \tfrac{1+2i}{10}  \e_2 ,
\end{alignat*}
where we have used that $(e^{-i\omega_0}I + U_{\omega_0}) \e_3$ vanishes.
Hence, by using the explicit expression~\eqref{e:defGamma} for~$\Gamma$ we obtain
\begin{equation}\label{e:actionek}
- \epsilon A_{0,*}^{-1}  \pi_{\geq 2}  A_1 A_0^{-1} \Gamma 
= 
 -\epsilon^2 \frac{29-22i}{125} \e_2 + \epsilon \frac{3i-1}{20} \e_3 -\epsilon^2 \frac{1+7i}{50} \e_2.
\end{equation}
Finally, combining~\eqref{e:picAdagGamma}, \eqref{e:A0piGamma} and~\eqref{e:actionek} completes the proof.
%
% 	\begin{alignat*}{1}
% 	\left[ - \epsilon A_0^{-1} A_1 A_{0}^{-1} [ - \pp i ]_2 \right]_c &=  - \epsilon A_{0,*}^{-1} [ \pp \sigma^+ ( -iI + U_{\omega_0}) ]A_{0,*}^{-1} [ - \pp i ]_2 \\
% 	&= \epsilon ( 3i +i)^{-1} (-i-1)(2i-1)^{-1} [i]_3 \\
% 	&= \left[ \frac{3 i-1}{20} \epsilon \right]_3 .
% 	\end{alignat*}
% \change[JB]{We now calculate the last part of  $A^\dagger \Gamma$. }{Third,}
% 	\begin{eqnarray}
% 	- \epsilon A_0^{-1} A_1 A_0^{-1} \left( \pp [ \tfrac{3i-1}{5} \epsilon]_1 - \pp \left[ \tfrac{3+i}{5} \epsilon \right]_3 \right) &=&   - \pp \epsilon^2  A_{0,*}^{-1} A_{1,2} A_{0,1}^{-1} [ \tfrac{3 i-1}{5} ]_1
% 	+\pp \epsilon^2  A_{0,*}^{-1} A_{1,*} A_{0,*}^{-1} [ \tfrac{3 +i}{5} ]_3  \nonumber
% 	\end{eqnarray}
% 	Then we compute the two summands on the RHS.
% 	\begin{eqnarray}
% 	- \pp \epsilon^2  A_{0,*}^{-1} A_{1,2} A_{0,1}^{-1} [ \tfrac{3 i-1}{5} ]_1  &=& - \frac{\epsilon^2}{5} A_{0,*}^{-1}  [ \pi \frac{3+16i}{10}]_2 \\
% 	&=& \left[ \frac{- \epsilon^2}{125} (29-22i)\right]_2
% 	\end{eqnarray}
%
%
%
% 	\begin{eqnarray}
% 	\pp \epsilon^2  A_{0,*}^{-1} A_{1,*} A_{0,*}^{-1} [ \tfrac{3 +i}{5} ]_3 &=&
% 	\frac{\epsilon^2}{5} (i K^{-1} + U_{\omega})^{-1} L_{\omega_0}  ( 3 i + i)^{-1} [ 3 + i ]_{3} \\
% 	&=& \frac{\epsilon^2}{5} (iK^{-1} + U_{\omega_0})^{-1} [\tfrac{3+i}{2}  ]_2 \\
% 	&=&\frac{-\epsilon^2}{50} [1+7i ]_2
% 	\end{eqnarray}
% 	Combining all of these results together, we obtain our theorem.
\end{proof}
 
\begin{lemma}
	\label{lem:ImplicitLast}
Let 
%	\begin{alignat}{1}
%	\hat{f}_{\epsilon,1} &:= \frac{\epsilon}{\sqrt{5}} \left[  \da \sqrt{2} + \alpha ( \dt + \dtt ) \right] +
%	\alpha r_c \left[ 2  +  \frac{2\epsilon }{\sqrt{5}} + r_c \right] , \label{e:feps1} \\ 
%	\hat{f}_{\epsilon,c} &:= \frac{2}{\pi \sqrt{5}} \left[ 
%	\da + \pp \dt + 
%	\frac{\epsilon}{\sqrt{5}} \left[ \sqrt{2} \da + \alpha ( \dt + \dtt) \right]
%	+\alpha ( 4 r_c + \dc^2)
%	\right] .\label{e:fepsc}
%	\end{alignat}
%	\note[J]{Proposed Changes}
%	\begin{alignat}{1}
%	\hat{f}_{\epsilon,1} &:= \tfrac{1}{2} \dc^0 \left(   \sqrt{2} \da  +3  \dw ( \pp + \da ) \right)  + \tfrac{1}{2} r_c  ( \pp + \da) 
%	\left(2 + 2\dc^0  + r_c \right)  , \label{e:feps1} \\
%	%
%	% 
%	\hat{f}_{\epsilon,c} &:= 
%	\frac{2}{\pi \sqrt{5}} \left[ 
%	 2 \left( \da + \pp \dw \right) + \dc^0  [ \sqrt{2} \da + 3 \dw (\pp+\da) ]
%	+(\pp+\da) ( 4 r_c + \dc^2)
%	\right] .\label{e:fepsc}
%	\end{alignat}
%	\note[JB]{Proposed Rearrangement}
	\begin{alignat}{1}
	\hat{f}_{\epsilon,1} &:= \tfrac{1}{2} \dc^0 \left(   \sqrt{2} \da  +3  \dw ( \pp + \da ) \right)  +  r_c  ( \pp + \da) 
	\left(1 + \dc^0  + \tfrac{1}{2} r_c \right)  , \label{e:feps1} \\
	%
	% 
	\hat{f}_{\epsilon,c} &:= 
	\tfrac{2}{\pi \sqrt{5}} \left[ 
	 2 \left( \da + \pp \dw \right) + \dc^0  [ \sqrt{2} \da + 3 \dw (\pp+\da) ]
	+(\pp+\da) ( 4 r_c + \dc^2)
	\right] .\label{e:fepsc}
	\end{alignat}
	Then  the vector 
% 	\change[J]{
% 	$
% 	[(1+\tfrac{2}{\pi})\hat{f}_{\epsilon,1},\tfrac{2}{\pi} \hat{f}_{\epsilon,1}, \hat{f}_{\epsilon,c}]^{T}
% 	$
	$
	[	(1+\tfrac{4}{\pi^2})^{1/2}\hat{f}_{\epsilon,1},\tfrac{2}{\pi} \hat{f}_{\epsilon,1}, \hat{f}_{\epsilon,c}]^{T}
	$
	is an upper bound on $A_0^{-1}  ( \tfrac{\partial F}{\partial  \epsilon} (x) -\Gamma )$ for any $ x \in B_\epsilon ( r,\rho)$.
\end{lemma}


\begin{proof}
	The $\alpha$- and $\omega$-component of  $A_0^{-1}  ( \tfrac{\partial F}{\partial  \epsilon} (x) -\Gamma )$ are given by $ A_{0,1}^{-1} i_\C^{-1} \pi_1 [  \tfrac{\partial F}{\partial  \epsilon} (x) -\Gamma ]$.
	If we can show that  $| \pi_1 [  \tfrac{\partial F}{\partial  \epsilon} (x) -\Gamma ]   |  \leq \hat{f}_{\epsilon,1}$, then it follows from the explicit expression for $A_{0,1}^{-1}$ that
	 $[ (1+\tfrac{4}{\pi^2})^{1/2}\hat{f}_{\epsilon,1} ,\tfrac{2}{\pi} \hat{f}_{\epsilon,1} ]^T$ 
	is an upper bound on 
	 $ \pi_{\alpha,\omega} A_0^{-1}  ( \tfrac{\partial F}{\partial  \epsilon} (x) -\Gamma )$.
	Let us write $ c = \bce +h_c$ for some $ h_c \in \ell^1_0$ with $ \|h_c\| \leq r_c$.  Recalling  \eqref{e:defGamma2}, we obtain
	\begin{alignat*}{1}
	\pi_1[  \tfrac{\partial F}{\partial  \epsilon} (x) -\Gamma ] &=   \pi_1 \bigl[ \alpha L_\omega c + \alpha [ U_\omega c ] * c - \pp L_{\omega_0} \bce  \bigr]  \\
	&= \pi_1 \bigl[ \alpha \sigma^{-}( e^{i \omega} + e^{-2 i \omega} ) \bce- \pp \sigma^{-}(i-1) \bce \bigr] 
%	\\ & \qquad 
	+ \pi_1  \bigl[ \alpha \sigma^{-}( e^{i \omega} + e^{-2 i \omega} )h_c + \alpha [ U_\omega c] * c  \bigr] \\
	& = \pi_1 \bigl[ (\alpha - \pp) (i-1) \bce  + \alpha ( e^{i \omega} -i + e^{-2 i \omega} +1)\bce \bigr] 
	\\ & \hspace*{6.55cm}
	+ \pi_1  \bigl[ \alpha \sigma^{-}( e^{i \omega} + e^{-2 i \omega} )h_c + \alpha [ U_\omega c] * c  \bigr] .
	\end{alignat*}
We note that
\[
  \pi_1 ( [ U_\omega c] * c ) =  \pi_1 ([ U_\omega (\bce+h_c)] * (\bce+h_c) )
  =  \pi_1 ( [ U_\omega \bce] * h_c + [ U_\omega h_c] * \bce +
   [ U_\omega h_c] * h_c ).
 \]
Hence, using Lemma~\ref{lem:deltatheta} we obtain the estimate 
% \note[JB]{Better to replace $\alpha$ by $\pp+\da$? Due to factor 2 in norm, term $2\alpha r_c$ should be $\alpha r_c$? And term $\alpha r_c ( \tfrac{2}{\sqrt{5}}  \epsilon + r_c)$ should be $\alpha r_c ( \tfrac{2}{\sqrt{5}}  \epsilon + \tfrac{1}{2} r_c)$?}
% 	\begin{equation*}
% 	\bigl| \pi_1 [  \tfrac{\partial F}{\partial  \epsilon} (x) -\Gamma ] \bigr|
% 	% &\leq&  | (\alpha - \pp) (i-1) \bce | + \alpha |( e^{i \omega} -i + e^{-2 i \omega} +1)\bce| +  2 \alpha r_c + \alpha r_c ( 2 |\bce| + r_c) \nonumber  \\
% 	 \leq \da \tfrac{\sqrt{2}}{\sqrt{5}} \epsilon + \tfrac{\alpha }{\sqrt{5}}\epsilon ( \dt + \dtt ) + 2 \alpha r_c + \alpha r_c ( \tfrac{2}{\sqrt{5}}  \epsilon + r_c)  .
% 	\end{equation*}
%
% 	\note[J]{Proposed Change}
		\begin{equation*}
		\bigl| \pi_1 [  \tfrac{\partial F}{\partial  \epsilon} (x) -\Gamma ] \bigr|
		 \leq \tfrac{1}{2} \dc^0 \left(   \sqrt{2} \da  +3  \dw ( \pp + \da ) \right)  +  r_c  ( \pp + \da) 
		 \left(1 + \dc^0  + \tfrac{1}{2} r_c \right) .
		\end{equation*}
We thus find that
	% Thereby by defining
	% \[
	% \hat{f}_{\epsilon,1} := \frac{\epsilon}{\sqrt{5}} [  \da \sqrt{2} + \alpha ( \dt + \dtt ) ] +
	% \alpha r_c [2  +  \tfrac{2}{\sqrt{5}} \epsilon + r_c] .
	% \]
%	it follows that 
	$ | \pi_1 [  \tfrac{\partial F}{\partial  \epsilon} (x) -\Gamma ]|   \leq \hat{f}_{\epsilon,1} $, with $\hat{f}_{\epsilon,1}$ defined in~\eqref{e:feps1}.
	
	The $c$-component of  $A_0^{-1}  ( \tfrac{\partial F}{\partial  \epsilon} (x) -\Gamma )$ is given by $ A_{0,*}^{-1}  \pi_{\geq 2} [  \tfrac{\partial F}{\partial  \epsilon} (x) -\Gamma ]$.
	We will use the estimate $ \| A_{0,*}^{-1}\| \leq \frac{2}{\pi \sqrt{5}}$, so that it remains to determine a bound on $ \| \pi_{\geq 2} [  \tfrac{\partial F}{\partial  \epsilon} (x) -\Gamma ]\|$.  
Using~\eqref{e:defGamma2} we compute
	\begin{equation*}
	\pi_{\geq 2} [  \tfrac{\partial F}{\partial  \epsilon} (x) -\Gamma ] =  \alpha e^{-i \omega} \e_2  +  \pp i \e_2 + 
	\pi_{\geq 2} \bigl( \alpha L_{\omega} \bce - \pp L_{\omega_0} \bce + \alpha L_{\omega} h_c + \alpha [U_\omega c] * c  \bigr).
	\end{equation*}
We split the right hand side into three parts, which we estimate separately. First
% 	\[
% \left\| \pi_2 \bigl[ \alpha L_{\omega} h_c + \alpha [U_\omega c] * c  \bigr] \right\| \leq  \alpha ( 4 r_c + \dc^2).
% 	\]
% 	\note[J]{Proposed Change}
		\[
		\left\| \pi_2 \bigl[ \alpha L_{\omega} h_c + \alpha [U_\omega c] * c  \bigr] \right\| \leq  (\pp+ \da)  ( 4 r_c + \dc^2).
		\]	
Next, we calculate  
	\begin{alignat*}{1}
\pi_{\geq 2}	\left[\alpha L_{\omega} \bce - \pp L_{\omega_0} \bce \right] &= \alpha  \sigma^+ (e^{-i \omega} +e^{-2 i \omega}) \bce - \pp \sigma^+ (-i-1) \bce \\
	&= \left[ (\alpha - \pp) ( -i-1)\tfrac{2-i}{5}\epsilon  + \alpha ( e^{-i\omega} +e^{-2 i \omega} -(i+1)) \tfrac{2-i}{5}\epsilon \right] \e_3 ,
\end{alignat*}
hence 
%\note[JB]{Probably factor 2 missing.}
% \begin{equation*}
% \left\|	\pi_{\geq 2}	\left[\alpha L_{\omega} \bce - \pp L_{\omega_0} \bce \right] \right\| \leq  \frac{\epsilon}{\sqrt{5}} [ \sqrt{2} \da + \alpha ( \dt + \dtt)] .
% 	\end{equation*}
% 	\note[J]{Proposed Change}
	\begin{equation*}
	\left\|	\pi_{\geq 2}	\left[\alpha L_{\omega} \bce - \pp L_{\omega_0} \bce \right] \right\| \leq 
	\dc^0  [ \sqrt{2} \da + 3 \dw (\pp+\da) ] .
	\end{equation*}
Finally, we estimate 
% \note[JB]{Since $\|\e_2\|=2$ I think there is a factor 2 missing.}
% 	\begin{equation*}
% 	\left\| ( \alpha e^{-i\omega}  + \pp i) \e_2 \right\| =
% 	 \left| ( \alpha - \pp) e^{-i \omega} + \pp ( e^{-i \omega} +i) \right|
% 	\leq \da + \pp \dt.
% 	\end{equation*}
% 	\note[J]{Proposed Change}
	\begin{equation*}
	\left\| ( \alpha e^{-i\omega}  + \pp i) \e_2 \right\| = 
	2 \left| ( \alpha - \pp) e^{-i \omega} + \pp ( e^{-i \omega} +i) \right|
	\leq 2 \left( \da + \pp \dw \right).
	\end{equation*}
Collecting all estimates, we thus find that
	$ \| \pi_c A_{0}^{-1}  [  \tfrac{\partial F}{\partial  \epsilon} (x) -\Gamma ] \|   \leq \hat{f}_{\epsilon,c} $, with $\hat{f}_{\epsilon,c}$ defined in~\eqref{e:fepsc}.
	%
	%
	%
	% Thereby, by defining
	% \[
	% \hat{f}_{\epsilon,*} = \frac{2}{\pi \sqrt{5}} \left[
	% \da + \pp \dt +
	% \frac{\epsilon}{\sqrt{5}} [ \sqrt{2} \da + \alpha ( \dt + \dtt)]
	% +\alpha ( 4 r_c + \dc^2)
	% \right]
	% \]
	% it follows that
	% $ \left| A_{0,*}^{-1} \left[  \tfrac{\partial F}{\partial  \epsilon} (x) -\Gamma \right]_{k \geq 2} \right| <  \hat{f}_{\epsilon,*}$.
	%
\end{proof}

Recall that $\II$ is used to denote the $ 3 \times 3 $ identity matrix.
% \note[JB]{We may want to introduce this notation in section 4.2 as well.} \note[J]{I've added this notation inside the proof of Theorem 4.7.}
\begin{corollary}
	\label{cor:QUpperBound}
Let $\overline{A_0^{-1} A_1} $ be defined in Proposition~\ref{prop:A0A1}.	
The vector 
	% \[
	% (\II+\epsilon \overline{A_0^{-1} A_1} ) \cdot  [(1+\tfrac{2}{\pi})\hat{f}_{\epsilon,1},\tfrac{2}{\pi} \hat{f}_{\epsilon,1}, \hat{f}_{\epsilon,c}]^{T}
	% \]
	% \note[J]{Proposed Change}
	\[
	(\II+\epsilon \overline{A_0^{-1} A_1} ) \cdot  [(1+\tfrac{4}{\pi^2})^{1/2}\hat{f}_{\epsilon,1},\tfrac{2}{\pi} \hat{f}_{\epsilon,1}, \hat{f}_{\epsilon,c}]^{T}
	\]
	is an upper bound on $ A^\dagger ( \tfrac{\partial F}{\partial  \epsilon} (x) -\Gamma ) $ for any $x \in B_\epsilon(r,\rho)$.
\end{corollary}
\begin{sloppypar}
\begin{proof}
	From Lemma \ref{lem:ImplicitLast} it follows that 
% \change[J]{	$
% 	[(1+\tfrac{2}{\pi})\hat{f}_{\epsilon,1},\tfrac{2}{\pi} \hat{f}_{\epsilon,1}, \hat{f}_{\epsilon,c}]^T
% $
% }{
$
[(1+\tfrac{4}{\pi^2})^{1/2}\hat{f}_{\epsilon,1},\tfrac{2}{\pi} \hat{f}_{\epsilon,1}, \hat{f}_{\epsilon,c}]^T
$
	is an upper bound on $A_0^{-1}  ( \tfrac{\partial F}{\partial  \epsilon} (x) -\Gamma )$. 
Since 
	$A^\dagger = (I-\epsilon A_0^{-1}A_1) A_0^{-1}$ and 
	$\II+\epsilon \overline{A_0^{-1} A_1}$ is an upper bound on $I-\epsilon A_0^{-1}A_1$, the result follows from  Lemma~\ref{lem:ImplicitLast}. 
\end{proof}
\end{sloppypar}


We combine Lemmas~\ref{lem:ImplicitApprox} and~\ref{lem:ImplicitLast}
into an upper bound on $A^{\dagger} \frac{\partial F}{\partial  \epsilon}(\hat{x}_\epsilon)$.

\begin{lemma}\label{lem:Qeps}
% \note[JB]{Introduced $\QQ_\epsilon^0$; need to check that third component is correct!}
% \note[J]{I think the $\frac{9}{5\sqrt{50}}  \epsilon^2 $ term should be $ \frac{18}{5\sqrt{50}}  \epsilon^2$ . Also, inserted the $(1+\tfrac{4}{\pi^2})^{1/2}$ term.}
%
	Define $\QQ_\epsilon^0 , \QQ_\epsilon \in \R_+^3$ as follows:
% 	\begin{alignat}{1}
% 		\QQ_\epsilon^0 &:=
% \left[ \frac{2}{5}\left(\frac{3\pi}{2}-1 \right)  \epsilon,
%  \frac{2}{5} \epsilon ,
%  \frac{2}{\sqrt{5}} + \frac{2}{\sqrt{10}}\epsilon  +
%  \frac{9}{5\sqrt{50}}  \epsilon^2
%   \right]^T , \nonumber \\
% 	\QQ_\epsilon &:= \QQ_\epsilon^0
% 	+
% 	(\II+\epsilon \overline{A_0^{-1} A_1} ) \cdot  \bigl[(1+\tfrac{2}{\pi})\hat{f}_{\epsilon,1},\tfrac{2}{\pi} \hat{f}_{\epsilon,1}, \hat{f}_{\epsilon,c}\bigr]^{T} . \label{e:defQeps}
% \end{alignat}
% \note[J]{Proposed Change}
\begin{alignat}{1}
\QQ_\epsilon^0 &:= 
\left[ \frac{2}{5}\left(\frac{3\pi}{2}-1 \right)  \epsilon,
\frac{2}{5} \epsilon , 
\frac{2}{\sqrt{5}} + \frac{2}{\sqrt{10}}\epsilon  +
\frac{18}{5\sqrt{50}}  \epsilon^2
\right]^T , \nonumber \\
\QQ_\epsilon &:= \QQ_\epsilon^0  + 
(\II+\epsilon \overline{A_0^{-1} A_1} ) \cdot  \bigl[(1+\tfrac{4}{\pi^2})^{1/2}\hat{f}_{\epsilon,1},\tfrac{2}{\pi} \hat{f}_{\epsilon,1}, \hat{f}_{\epsilon,c}\bigr]^{T} . \label{e:defQeps}
\end{alignat}
Then the vector $\QQ_\epsilon \in \R^3_+$ is an upper bound  on $A^{\dagger} \frac{\partial F}{\partial  \epsilon}(x)$ 
for any  $x \in B_\epsilon(r,\rho)$.
\end{lemma}
\begin{proof}
It follows from Lemma \ref{lem:ImplicitApprox} that the vector 
$\QQ_\epsilon^0$
is an upper bound on $A^{\dagger} \Gamma$
(for example, the third component of $\QQ_\epsilon^0$ is a bound on $\|c'\|$).
It follows from Corollary~\ref{cor:QUpperBound} that 
% \change[J]{$	(I+\epsilon \overline{A_0^{-1} A_1} ) \cdot  [(1+\tfrac{2}{\pi})\hat{f}_{\epsilon,1},\tfrac{2}{\pi} \hat{f}_{\epsilon,1}, \hat{f}_{\epsilon,c}]^{T}$
% 	}{
%\note[JB]{displayed; formula was just too big}
\[
	(\II+\epsilon \overline{A_0^{-1} A_1} ) \cdot  [(1+\tfrac{4}{\pi^2})^{1/2}\hat{f}_{\epsilon,1},\tfrac{2}{\pi} \hat{f}_{\epsilon,1}, \hat{f}_{\epsilon,c}]^{T}
\]
 is an upper bound on $ A^\dagger ( \tfrac{\partial F}{\partial  \epsilon} (x) -\Gamma ) $.
We conclude from the triangle inequality that $\QQ_\epsilon $ is an upper bound on $A^\dagger  \tfrac{\partial F}{\partial  \epsilon} (x)$. 
\end{proof}
 
Finally, we prove the bounds needed to control the derivative $\frac{d}{d\epsilon} \hat{\alpha}_\epsilon$ in Section~\ref{s:Jones}
(in particular the implicit differentiation argument in Theorem~\ref{thm:NoFold}).

% \marginpar{remove newpage when done}
% \newpage

%%%%%%%%%%%%%%% Final Lemma %%%%%%%%%%%%%%

\begin{lemma}
		\label{lem:Meps}
		% \remove[J]{		Fix $\epsilon>0$, $r \in \R^3_+$, $\rho >0$ and
		% 	assume $x \in B_\epsilon(r,\rho)$. }
		% \add[J]{
		Fix $ \epsilon_0 > 0 , \rr \in \R^3_+ $ and $\rho >0$ as in the hypothesis of Proposition~\ref{prop:TightEstimate}. 
Let $ 0 < \epsilon \leq \epsilon_0$ and let $ \hat{x}_\epsilon \in B_\epsilon(\epsilon^2  \rr,\rho)$ denote the unique solution to $F(x) = 0$. 
Recall the definitions of  
$\ZZ_\epsilon \in  \emph{Mat}(\R_+^3 , \R_+^3)$
and $\QQ_\epsilon \in  \R_+^3$ in Equations~\eqref{e:defZeps} and~\eqref{e:defQeps}.  
Define 
		\begin{alignat*}{1}
		M_\epsilon &:= \frac{1}{\epsilon^2} \left(	(\II+\epsilon \overline{A_0^{-1} A_1} ) \cdot  \bigl[(1+\tfrac{4}{\pi^2})^{1/2}\hat{f}_{\epsilon,1},\tfrac{2}{\pi} \hat{f}_{\epsilon,1}, \hat{f}_{\epsilon,c} \bigr]^{T} \right)_1  ,\\
		M'_\epsilon  &:=  \frac{1}{ \epsilon^2 } \bigl( \ZZ_\epsilon (\II-\ZZ_\epsilon)^{-1} \QQ_\epsilon \bigr)_1  ,
		\end{alignat*}
where the subscript denotes the first component of the vector.
Then $M_\epsilon$ and $M'_\epsilon$ are positive, increasing in $\epsilon$, and satisfy the inequalities
 \begin{alignat}{1}
 \left| \pi_\alpha A^{\dagger} \left( \tfrac{\partial F}{\partial  \epsilon}(\hat{x}_\epsilon) - \Gamma_\epsilon \right)  \right|  &\leq\epsilon^2 M_\epsilon , \label{eq:Mepsilon}\\
 %
 \left( \ZZ_\epsilon (\II-\ZZ_\epsilon)^{-1} \QQ_\epsilon \right)_1 &\leq \epsilon^2 M'_\epsilon . \label{eq:MMepsilon}
 \end{alignat}
\end{lemma}

\begin{proof}
To first show that $(\II - \ZZ_{\epsilon })^{-1}$ is well defined,  we note that by Proposition~\ref{prop:TightEstimate} 
the radii polynomials $ P(\epsilon,\epsilon^2 \rr,\rho)$ are all negative. As was shown in the proof of Theorem~\ref{thm:RadPoly},
the operator norm of $ \ZZ_\epsilon$ on $ \R^3$ equipped with the norm $ \| \cdot \|_{\epsilon^2 \rr}$ is given by some $\kappa <1$, whereby the Neumann series of $ (\II - \ZZ_\epsilon)^{-1}$ converges. 
	
From the definition of $M_\epsilon$ and Corollary~\ref{cor:QUpperBound}, inequality \eqref{eq:Mepsilon} follows. Inequality \eqref{eq:MMepsilon} is a direct consequence of the definition of $M'_\epsilon$.
	Since the functions $\hat{f}_{\epsilon,1}$ and $\hat{f}_{\epsilon,c}$ are positive, then $M_\epsilon$ and $\QQ_\epsilon$ are positive. 
	Since the matrix $ \ZZ_\epsilon$ has positive entries only,  the Neumann series for $(\II-\ZZ_\epsilon)^{-1}$  has summands with exclusively positive entries, whereby $M'_\epsilon$ is positive. 

	Next we show that the components of $\ZZ_\epsilon$ and $\QQ_\epsilon - \QQ_\epsilon^0$ are polynomials in $\epsilon$ with positive coefficients and their lowest degree terms are at least quadratic. 
	To do so, it suffices to prove as much for the functions $ \hat{f}_{\epsilon,1},\hat{f}_{\epsilon,c},f_{1,\alpha},f_{1,\omega}, f_{1,c} , f_{*,\alpha}, f_{*,\omega},f_{*,c}$. 
	We note that all of these functions are given as polynomials with positive coefficients in the variables $ \epsilon, \da,\dw,\dc,r_c,\dc^0$ 
	(recall that $\rho $ is fixed and does not vary with $\epsilon$).
	Since $(r_\alpha , r_\omega,r_c) = \epsilon^2 (\rr_\alpha , \rr_\omega,\rr_c)$, then by Definition~\ref{def:DeltaDef} the terms $\da,\dw,r_c$ are all $ \cO(\epsilon^2)$. 
	Furthermore, whenever any of the terms  $ \epsilon, \dc, \dc^0$ 
appears, it is multiplied by another term of order at least $\cO(\epsilon)$.
	It  follows that every component of 
	 $ \ZZ_\epsilon$ and $ \QQ_{\epsilon} - \QQ_{\epsilon}^0$ is a polynomial in $ \epsilon$ with positive coefficients for which the lowest degree term is at least quadratic. 

% 	\remove[J]{
% 	We now show that $M_\epsilon$ and $ M'_\epsilon$ are increasing in $\epsilon$.
% First note that Proposition~3.16 shows that the solution  satisfies $\hat{x}_\epsilon \in B_\epsilon(\epsilon^2 \rr, \rho)$.
% Thereby all of the variables introduced in Definition B.1 are polynomials with nonnegative coefficients in $\epsilon$.
% Moreover, all of the variables which begin with  ``$\Delta$'' have a lowest order term of $\epsilon^2$, and only $\dc$ has its lowest order term being $\epsilon^1$.
% It is thus straightforward to see that the functions $\hat{f}_{\epsilon,1}$ and $\hat{f}_{\epsilon,c}$, as well as all of the component functions of $\ZZ_{\epsilon}$, are all polynomials in $\epsilon$ with positive coefficients and their lowest order terms are at least quadratic. }

From these considerations it follows that the components of both 
$M_\epsilon = \epsilon^{-2}(\QQ_{\epsilon} - \QQ_{\epsilon}^0)_1$ and $ \epsilon^{-2}\ZZ_\epsilon$ are polynomials in $\epsilon$ with positive coefficients. 
It also follows that both $\QQ_\epsilon$ and $(\II-\ZZ_\epsilon)^{-1}$ are increasing in $\epsilon$, whereby $M'_\epsilon$ is increasing in $\epsilon$.
\end{proof}








\end{document}

