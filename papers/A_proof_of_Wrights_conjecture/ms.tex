\documentclass{article}
\usepackage{amsmath, geometry,amssymb,amscd, float, verbatim, latexsym,subfig,wrapfig}
\usepackage{algorithmic,algorithm,cite,setspace}
\usepackage[pdftex]{graphicx}
\usepackage[all]{xy}
\usepackage{lscape}
\usepackage{url}
\usepackage{amsthm}

\newtheorem{theorem}{Theorem}[section]
\newtheorem{lemma}[theorem]{Lemma}
\newtheorem{proposition}[theorem]{Proposition}
\newtheorem{corollary}[theorem]{Corollary}
\newtheorem{claim}[theorem]{Claim}
\newtheorem{conjecture}[theorem]{Conjecture}
\newtheorem{remark}[theorem]{Remark}
\newtheorem{definition}[theorem]{Definition}

\numberwithin{equation}{section}

% \newenvironment{definition}[1][Definition]{\begin{trivlist}
% \item[\hskip \labelsep {\bfseries #1}]}{\end{trivlist}}

\newcommand{\bi}{\text{\textup{bi}}}
\newcommand{\sym}{\text{\textup{sym}}}
\renewcommand{\c}{a}
\newcommand{\e}{\mathrm{e}}
\newcommand{\tc}{\tilde{c}}

\newcommand{\R}{\mathbb{R}}
\newcommand{\C}{\mathbb{C}}
\newcommand{\Q}{\mathbb{Q}}
\newcommand{\N}{\mathbb{N}}
\newcommand{\Z}{\mathbb{Z}}
\newcommand{\pp}{\tfrac{\pi}{2}}
\newcommand{\II}{I_3}

\newcommand{\B}{\widehat{B}}
\newcommand{\tF}{\widetilde{F}}
\newcommand{\epseps}{\epsilon_*}
\newcommand{\tS}{\widetilde{S}}

\newcommand{\da}{\Delta_\alpha}
\newcommand{\dw}{\Delta_\omega}
\newcommand{\dt}{\Delta_{\theta}}
\newcommand{\dtt}{\Delta_{2\theta}}
\newcommand{\dc}{\delta_c}

\newcommand{\bx}{\bar{x}}
\newcommand{\balpha}{\bar{\alpha}}
\newcommand{\bomega}{\bar{\omega}}
\newcommand{\bc}{\bar{c}}
\newcommand{\bce}{\bc_{\epsilon}}
\newcommand{\rr}{\check{r}}

\newcommand{\ZZ}{\mathcal{Z}}
\newcommand{\QQ}{\mathcal{Q}}

\newcommand{\LL}{\zeta}
\newcommand{\upperbound}[1]{\overline{#1}}
%\newcommand{\upperbound}[1]{\overline{\overline{#1}}}

\newcommand{\cA}{\mathcal{A}}
\newcommand{\cB}{\mathcal{B}}
\newcommand{\cD}{\mathcal{D}}
\newcommand{\cE}{\mathcal{E}}
\newcommand{\cG}{\mathcal{G}}
\newcommand{\cH}{\mathcal{H}}
\newcommand{\cL}{\mathcal{L}}
\newcommand{\cM}{\mathcal{M}}
\newcommand{\cN}{\mathcal{N}}
\newcommand{\cO}{\mathcal{O}}

\newcommand{\X}[1]{\mathbf{X_{#1}}}

%\usepackage[inline]{trackchanges}
%\addeditor{J}
%\addeditor{JB}

\title{A proof of Wright's conjecture}
\author{Jan Bouwe van den Berg\thanks{Partially supported by NWO VICI-grant 639.033.109} \thanks{Department of Mathematics, VU Amsterdam, de Boelelaan 1081, 1081 HV Amsterdam, The Netherlands, janbouwe@few.vu.nl}
\and
Jonathan Jaquette \thanks{Partially supported by NSF DMS 0915019,	NSF DMS 1248071} \thanks{Department of Mathematics, Rutgers, The State University Of New Jersey, 110 Frelinghuysen Rd., Piscataway, NJ 08854-8019, USA, jaquette@math.rutgers.edu}
}

\begin{document}

\maketitle



\begin{abstract}
Wright's conjecture states that the origin is the global attractor for the delay differential equation $y'(t) = - \alpha y(t-1) [ 1 + y(t) ] $  for all $\alpha \in (0,\pp]$. This has been proven to be true for a subset of parameter values $\alpha$. We extend the result to the full parameter range $\alpha \in (0,\pp]$, and thus prove Wright's conjecture to be true. Our approach relies on a careful investigation of the neighborhood of the Hopf bifurcation occurring at $\alpha = \pp$. This analysis fills the gap left by complementary work on Wright's conjecture, which covers parameter values further away from the bifurcation point. Furthermore, we show that the branch of (slowly oscillating) periodic orbits originating from this Hopf bifurcation does not have any subsequent bifurcations (and in particular no folds) for
$\alpha\in(\pp , \pp + 6.830 \times10^{-3}]$. 
When combined with other results, this proves that the branch of slowly oscillating solutions that originates from the Hopf bifurcation at $\alpha=\pp$ is globally parametrized by $\alpha > \pp$.
\end{abstract}



\begin{center}
	{\bf \small Keywords.} 
	{ \small Delay Differential Equation,
		Hopf Bifurcation, 
		Wright's Conjecture, \\
		Supercritical Bifurcation Branch,
		Newton-Kantorovich Theorem
		}
\end{center}

\section{Introduction}
\label{s:introduction}
\section{Introduction}  \label{sec:introduction}

\newcommand\inexpIntro[3]{#1?(#2,#3).}
\newcommand\rinexpIntro[3]{*#1?(#2,#3).}
\newcommand\outexpIntro[3]{#1!(#2,#3).}
\newcommand\outatomIntro[3]{#1!(#2,#3)}

We propose a fully automated method for proving termination of \(\pi\)-calculus processes.
Although there have been a lot of studies on termination analysis for the \(\pi\)-calculus
and related calculi~\cite{Deng06IC,Demangeon07,SangiorgiTermination,KobayashiHybrid,Yoshida04IC,DBLP:journals/jlp/DemangeonHS10,Venet98SAS}, most of them have been rather theoretical,
and there have been surprisingly little efforts in developing  fully automated termination
verification methods and tools based on them. To our knowledge,
Kobayashi's \typical{}~\cite{TyPiCal,KobayashiHybrid} is the only exception that
can prove termination of \(\pi\)-calculus processes (extended with natural numbers)
fully automatically, but its termination analysis is quite limited (see Section~\ref{sec:relatedwork}).

Our method is based on a reduction to termination analysis for sequential programs:
we translate a \(\pi\)-calculus process \(P\) to a sequential program \(S_P\), so that
if \(S_P\) is terminating, so is \(P\). The reduction allows us to use
powerful, mature methods and tools
for termination analysis of sequential programs~\cite{heizmann2016ultimate,freqterm,DBLP:conf/lics/PodelskiR04,Kuwahara2014Termination,DBLP:journals/cacm/CookPR11}.

The idea of the translation is to convert a chain of communications on replicated input
channels to a chain of recursive function calls of the target sequential program.
Let us consider the following Fibonacci process:
\begin{align*}
    & \rinexpIntro{\fib}{n}{r}
        \ifexp{n<2}{ \soutatom{r}{1} \\ &\quad}
                   { \nuexp{s_1} \nuexp{s_2} (\outatomIntro{\fib}{n-1}{s_1} \PAR \outatomIntro{\fib}{n-2}{s_2} \PAR \sinexp{s_1}{x}\sinexp{s_2}{y}\soutatom{r}{x+y}) \\}
    & \PAR \outatomIntro{\fib}{m}{r}
\end{align*}
Here, the process
$\rinexpIntro{\fib}{n}{r} \ldots$ is a function server that computes the \(n\)-th Fibonacci number
in parallel and returns the result to \(r\),
and $\outatom{\fib}{m}{r}$ sends a request for computing the \(m\)-th Fibonacci number;
those who are not familiar with the syntax of the \(\pi\)-calculus may wish to consult
Section~\ref{sec:targetlanguage} first.
To prove that the process above is terminating for any integer \(m\),
it suffices to show that there is no infinite chain of communications on $\fib$:
\[
    \fib(m,r) \to \fib(m_1,r_1) \to \fib(m_2,r_2) \to \cdots.
\]
We convert the process above to the following program:\footnote{The actual translation
  given later is a little more complex.}
\begin{verbatim}
 let rec fib(n) = if n<2 then () else (fib(n-1) [] fib(n-2)) in
 fib(m)
\end{verbatim}
Here, \texttt{[]} represents the non-deterministic choice.
Note that, although the calculation of Fibonacci numbers is not preserved,
for each chain of communications on \texttt{fib}, there is a corresponding
sequence of recursive calls:
\[
\mathtt{fib}(m) \to \mathtt{fib}(m_1) \to \mathtt{fib}(m_2) \to \cdots.
\]
Thus, the termination of the sequential program above implies the termination of
the original process.
As shown in the example above, (i) each communication on a replicated input channel
is converted to a function call, (ii) each communication on a non-replicated input
channel is just removed (or, in the actual translation, replaced by a call of
a trivial function defined by \(f(\seq{x})=(\,)\)), and (iii) parallel composition
is replaced by a non-deterministic choice.
We formalize the translation outlined above and prove its correctness.

The basic translation sketched above sometimes loses too much information.
For example, consider the following process:
\begin{align*}
    & \rinexpIntro{\pre}{n}{r} \soutatom{r}{n-1} \\
    & \PAR \rinexpIntro{f}{n}{r} \ifexp{n<0}{ \soutatom{r}{1} }
                                       { \nuexp{s} (\outatomIntro{\pre}{n}{s} \PAR \sinexp{s}{x}\outatomIntro{f}{x}{r}) } \\
    & \PAR \outatomIntro{f}{m}{r}
\end{align*}
The translation sketched above would yield:
\begin{verbatim}
  let pred(n) = n-1 in
  let rec f(n) = if n<0 then () else (pred(n) [] f(*)) in
  f(m)
\end{verbatim}
Here, \texttt{*} represents a non-deterministic integer: since we have removed
the input $\sinatom{s}{x}$, we do not have information about the value of \( x \).
As a result, the sequential program above is non-terminating, although the original
process is terminating.
To remedy this problem, we also refine the basic translation above by using a refinement
type system for the \(\pi\)-calculus. Using the refinement type system,
we can infer that the value of \(x\) in the original process is less than \(n\),
so that we can refine the definition of \texttt{f} to:
\begin{verbatim}
 let rec f(n) = ... else (pred(n) [] let x=* in assume(x<n);f(x))
\end{verbatim}
The target program is now terminating, from which
we can deduce that the original process is also terminating.
We have implemented an automated tool based on the refined translation above.

The contributions of this paper are summarized as follows.
\begin{itemize}
\item The formalization of the basic translation from the \(\pi\)-calculus
  (extended with integers) to sequential programs, and a proof of its correctness.
\item The formalization of a refined translation based on a refinement type system.
\item An implementation of the refined translation, including automated refinement type
  inference based on CHC solving, and experiments to evaluate the effectiveness of
  our method.
\end{itemize}

The rest of this paper is structured as follows.
Section~\ref{sec:targetlanguage} introduces the source and target languages
of our translation.
Section~\ref{sec:approach} 
formalizes the basic translation, and proves its correctness.
Section~\ref{sec:refinement} refines the basic translation by using a refinement type system.
Section~\ref{sec:implementation} reports an implementation and experiments.
Section~\ref{sec:relatedwork} discusses related work,
and Section~\ref{sec:conclusion} concludes the paper.


\section{Preliminaries}
\label{s:preliminaries}
\section{Preliminaries}\label{chpt:preliminiaries}
In this chapter we will introduce some of the mathematical background and notation needed for this thesis. In particular, we will shortly introduce the differential geometric description of spacetime in Section \ref{sec:spacetime_geometry} and give an introduction to the notion of global hyperbolicity and its connection to Green- and normally-hyperbolic operators in Section \ref{sec:global_hyperbolicity}. In a bit more detail, we will introduce the notion of differential forms and give explicit definitions, also in terms of an index based notation, in Section \ref{sec:differential_forms}. For completeness, in Section \ref{sec:cat-theory}, we present basic definitions of category theory. The reader familiar with these topics can safely skip this chapter and refer to it when interested in the chosen conventions.
%
%
%
%
%%%%%%
%%SPACTIME GEOMETRY
%%%%%
%
%
%
\subsection{Spacetime geometry}\label{sec:spacetime_geometry}
In GR, the universe is mathematically described as a four dimensional \emph{spacetime}, consisting of a smooth, four dimensional manifold \gls{M} (assumed to be Hausdorff, connected, oriented, time-oriented and para-compact) and a Lorentzian metric $g$. We will assume the signature of the Lorentzian metric $g$ to be $(-,+,+,+)$. The Levi-Civita connection on $(\M,g)$ is as usual denoted by \gls{nabla}.
Throughout this thesis, we treat spacetime as fixed, implementing a gravitational background determined classically by Einstein's field equations. Hence, we neglect any back-reaction of the fields on the metric, both in the quantum and the classical case. In that sense, we treat the fields as \emph{test fields}.\par
For the basic mathematical theory regarding Lorentzian manifolds, we refer to the literature: An introduction to the topic with an emphasis on the physical application in GR is for example given in \cite{wald_GR} and \cite{carroll_spacetime-and-gr}.
Here, we will shortly recap the notion of a tangent space and tangent bundle and generalize to the notion of a vector bundle which we will use in the general description of normally hyperbolic operators and differential forms.
In the following, we generalize the setting to an arbitrary smooth manifold $\N$ of dimension $N$ with either Lorentzian or Riemannian metric $k$.\par
%
%
A \emph{tangent vector} $v_x$ at point $x \in \N$ is a linear map $v_x : C^\infty(\N , \IR) \to \IR$ that obeys the Leibniz rule, that is, for $f,g \in C^\infty (\N,\IR)$ it holds $v_x(fg) = f(x)v_x(g) + v_x(f)g(x)$.
We define the \emph{tangent space} \gls{TxN} of $\N$ at $x$ as the real $N$-dimensional vector space of all tangent vectors at point $x$.
The disjoint union of all tangent spaces is called the \emph{tangent bundle} \gls{TN} of $\N$ and is itself a manifold of dimension $2N$. A \emph{vector field} is a map $v: \N \to T\N$ such that $v(x) \in T_x\N$.
The respective dual spaces, that is the space of all linear functionals, the \emph{co-tangent space} and the \emph{co-tangent bundle}, are denoted by \gls{TsxN} and \gls{TsN} respectively.\par
%
For Lorentzian manifolds, we call a tangent vector $v$ at $x \in \N$ \emph{timelike} if $k_{\mu \nu} v^\mu v^\nu < 0$, \emph{spacelike} if $k_{\mu \nu} v^\mu v^\nu > 0$ and \emph{null} (or lightlike) if $k_{\mu \nu} v^\mu v^\nu = 0$. At every point $x \in \N$, we define the set of all \emph{causal}, that is, either timelike or null, tangent vectors in the tangent space at $x$. This set is called the \emph{light cone} at $x$ and it is split up into two distinct parts, one that we call the future light cone, and one that we call the past light cone at $x$. Since we assume the manifold to be time orientable, there exists a smooth vector field $t$ that is timelike at every $x \in \N$. Given this time orientation, we identify the future (past) light cone with the set of tangent vectors $v \in T_x\N$ such that $k_{\mu\nu} v^\mu t^\nu < 0$ (respectively $> 0$). Therefore, a tangent vector $v$ at $x$ is called \emph{future directed} (past directed) if it lies in the future (past) light cone at $x$.\\
Accordingly, a curve $\gamma : I \to \N$ is called timelike (spacelike, null, causal, future or past directed) if its tangent vector $\dot{\gamma}$ is timelike (spacelike, null, causal, future or past directed) at every $x \in \N$.  For every point $x \in \N$ we define the \emph{causal future/past} \gls{causalfuturepast} of $x$ as the set of all points $q \in \N$ that can be reached by a future directed causal curve originating in $x$. For any subset $S \in \N$ we define $J^\pm (S) = \bigcup_{x \in S} J^\pm(x)$ and $J(S) = J^+(S) \cup J^- (S)$. Finally, the future/past domain of dependence $\gls{futurepastdomainofdependence}$ of a set $S \subset \N$ is the set of all points $x \in \N$ such that every inextendible causal curve through $x$ intersects $S$. The \emph{domain of dependence} \gls{domainofdependence} of $S$ is the union of the future and past domain of dependence of the set $S$.
For more details on the causal structure of spacetime we refer to for example \cite[Chapter 8]{wald_GR}.\par
%
%
%
The notion of tangent bundles can be generalized to the notion of a vector bundle. Instead of ``attaching'' the vector spaces $T_x \N$ to every point $x$ of the manifold, we allow for the occurrence of arbitrary vector spaces, called the fibres of the vector bundle. A vector bundle then consists of the base manifold, in our case $\N$, the total space and a map $\pi$ from the total space to the base manifold, that can be locally trivialized. At each point of the base manifold, the pre-image of $\pi$ is the fibre of the vector bundle. To be precise we define, following \cite{rudolph_schmidt}:
\begin{definition}[Vector bundle]
	A smooth \emph{vector bundle} over $\N$ is a tuple $\gls{vectorbundle} = (E,\N, \pi)$, where $E$ is a smooth manifold and $\pi : E \to \N$ is a smooth surjective map satisfying:
	\begin{enumerate}
		\item For every $x \in \N$, $\pi^{-1}(x)$ is a vector space, called the fibre of the bundle at point $x$.
		\item There exists a finite dimensional vector space $F$, an open covering $\left\{ U_\alpha\right\}_\alpha$ of $\N$ and a family of diffeomorphisms $\chi_\alpha : \pi^{-1}(U_\alpha) \to U_\alpha \times F$ such that for all $\alpha$ it holds $\chi_\alpha \comp \text{pr}_1 =  \restr{\pi}{\pi^{-1}(U_\alpha)}$ and for every $x \in \N$ the map $\text{pr}_2 \comp \restr{\chi_\alpha}{\pi^{-1}(x)} : \pi^{-1}(x) \to F$ is linear.
	\end{enumerate}
\end{definition}
Here, the maps $\text{pr}_1$ and $\text{pr}_2$ denote the projection onto the first respectively second component of an element in $U_\alpha \times F$. The properties graphically mean that \emph{locally}, the vector bundle ``looks like" the product of the base manifold with the fibre. The tuples $(U_\alpha, \chi_\alpha)$ are called \emph{local trivializations} of the vector bundle. Like for vector spaces, we can define the sum and product of vector bundles, by using the according vector space definitions on the fibres of the bundle.\par
Let $\mathfrak{X}, \mathfrak{Y}$ be vector bundles over $\N$ with fibres $X_x$ and $Y_x$ at $x \in \N$. We denote by \gls{whitneysum} the \emph{Whitney sum} of the two vector bundles - the vector bundle over $\N$ whose fibres are given by the direct sum $X_x \oplus Y_x$. Similarly, one obtains the local trivializations of the Whitney sum from the trivializations of $\mathfrak{X}, \mathfrak{Y}$ and direct sums.\par
Accordingly, let $\mathfrak{X}, \mathfrak{Y}$ be vector bundles over $\N$ and $\widetilde{\N}$, with fibres $X_x$ and $Y_{\tilde{x}}$ at $x \in \N$, $\tilde{x} \in \widetilde{\N}$ respectively. We denote by \gls{outerproductbundle} the \emph{outer product} of the two vector bundles - the vector bundle over $\N \times \widetilde{\N}$ whose fibres are given by the tensor products $X_x \otimes Y_x$. Similarly, one obtains the local trivializations of the outer product from the trivializations of $\mathfrak{X}, \mathfrak{Y}$ and tensor products. \par
%
Finally, we generalize the notion of vector fields:
\begin{definition}[Sections of vector bundles]
Let $\mathfrak{X}=(E,\N,\pi)$ be a vector bundle with fibres $X_x=\pi^{-1}(x)$ at $x \in \N$. A \emph{smooth section} of the vector bundle is a smooth map $\gamma : \N \to E$ such that $\gamma(x) \in X_x$ for all $x \in \N$. The \emph{vector space of smooth sections} of $\mathfrak{X}$ is denoted by \gls{gammax}, the one with compactly supported sections is as usual denoted by \gls{gammaxzero}.
\end{definition}
In this language, a vector field $v$ is just a smooth section of the tangent bundle of a manifold, $v \in \Gamma(T\N)$. One may therefore identify the physical notion of fields with smooth sections of vector bundles. This point of view will be used to define the notion of differential forms in Section \ref{sec:differential_forms}.\par
In this thesis, we usually are interested in complex valued functions (or sections in general). Therefore, we view all occurring vector bundles as complex, in the sense that we take two distinct copies of the vector bundle, one representing the real, one the imaginary part of the bundle. A section of that complex vector bundle is just a pair of two sections of the real vector bundle under consideration. From now, if not specified explicitly, we will view all vector bundles, including the tangent bundle $T\N$, as complex vector bundles. Accordingly, smooth sections of those bundles will in general be complex valued.
%
%
%
%
%
%
%
%
%%%%%%%
%%PARTIAL DIFFERENTIAL OPERATORS AND GLOBAL HYPERBOLICITY
%%%%%%%
%
%
%
\subsection{Partial differential operators and global hyperbolicity}\label{sec:global_hyperbolicity}
When dealing with field theories, whether classical or quantum, one is, of course, interested in the dynamics of the fields. These are usually described by some partial differential equation, often of second order. In the following, we give a short introduction to the theory of certain partial differential operators acting on smooth sections of a vector bundle over the spacetime $(\M,g)$.\par
%
As we have seen, these smooth sections are generalizations of the notion of a field.  In the following, let $\mathfrak{X}$ denote a vector bundle over the manifold $\M$ and let $P: \Gamma(\mathfrak{X}) \to \Gamma(\mathfrak{X})$ be a partial differential operator acting on smooth sections of the bundle. As in the case of flat spacetime, we are interested in basic questions regarding the differential equation $Pf = j$, for example: Can we formulate a (globally) well posed initial value problem? Does the differential equation possess (unique) solutions? To answer these questions, we will now restrict to the case where $P$ is linear and of second order, as it is often the case in physical applications. One can show that for a certain class of such operators, namely normally hyperbolic partial differential operators of second order, we can rigorously treat these questions.\par
Choosing local coordinates $x=(x_\mu)$ on $\M$ and a local trivialization of $\mathfrak{X}$, a linear partial differential operator of second order is called \emph{normally hyperbolic} if it takes the form
\begin{align}
	P = - \sum_{\mu,\nu} g^{\mu \nu} \partial_\mu \partial_\nu + \sum_{\alpha} A_\alpha (x) \partial_\alpha + B(x) \formspace,
\end{align}
where $A_\alpha$ and $B$ are matrix-valued coefficients depending smoothly on the coordinate $x$ (see. \cite[Chapter 1.5]{baer_ginoux_pfaeffle}). One can also formulate a coordinate independent definition in terms of the principal symbol, which we will not present here (see for example \cite[Section 1.5]{baer_ginoux_pfaeffle} ). \par
%
Normally hyperbolic operators possess unique fundamental solutions (see for example the fundamental solutions to the wave operator as noted in Lemma \ref{lem:fundamental_solution_wave_operator}). These fundamental solutions fulfill certain physically important properties, such as a finite propagation speed smaller than the speed of light. Furthermore, specifying the initial data on some space-like hypersurface $X \in  \M$ specifies a unique solution on the domain of dependence $D(X)$ of $X$. Due to these properties, one often calls normally hyperbolic operators just \emph{wave operators}. But to state a \emph{globally} well posed initial value problem for a wave equation, we need to restrict the class of spacetimes $\M$ under consideration to those that possess space-like hypersurfaces $X$ whose domain of dependence is all of the spacetime, $D(X) = \M$. This leads to the notion of \emph{globally hyperbolic} spacetimes:
\begin{definition}[Global Hyperbolicity]
	A spacetime $\M$ is called \emph{globally hyperbolic} if there exists a Cauchy surface $\gls{sigma}$ in $\M$.
\end{definition}
\noindent Here, a Cauchy surface is a space-like hypersurface $\Sigma \subset \M$ such that every inextendible causal curve $\gamma$ intersects $\Sigma$ exactly once. One can show that Cauchy surfaces fulfill the desired property mentioned above, that is,  $D(\Sigma) = \M$. Furthermore, one can show that any globally hyperbolic spacetime $\M$ is foliated by a one-parameter family $\left\{ \Sigma_t \right\}_t$ of Cauchy surfaces (see for example \cite[Theorem 8.3.14]{wald_GR}). \par
In physical applications, one often finds the dynamics of a theory to be described by wave operators. Most prominently, the Klein-Gordon operator $(\square + m^2)$ acting on scalar fields, or its generalization, the wave operator acting on differential forms introduced in Section \ref{sec:differential_forms}, is normally hyperbolic. But there are also important physical field theories that are not described by wave operators, such as the Proca field treated in this thesis. It turns out that the Proca operator (see Definition \ref{def:proca_operator}) is a so called \emph{Green-hyperbolic} operator. These are again partial differential operators $P$ of second order acting on smooth sections of some vector bundle, such that $P$ (and its dual $P'$) posses fundamental solutions. Obviously, normally hyperbolic operators are Green-hyperbolic, but the opposite is not true. One can generalize some results obtained by studying normally hyperbolic operators to Green-hyperbolic operators. An introduction to this topic is given in \cite{baer_green-hyperbolic}, where it is also shown that the Proca operator is Green-hyperbolic but not normally hyperbolic.\par
For our application, the notion of Green-hyperbolicity is not of vast importance, but it is worth mentioning that there exists a more detailed mathematical background on the treatment of such operators.
A very detailed description of normally hyperbolic operators on Lorentzian manifolds, including proofs of the above statements regarding the initial value problem and the existence of fundamental solutions, is given in \cite{baer_ginoux_pfaeffle}, also with an overview of quantization. A shorter introduction to the topic is for example treated in \cite{baer-ginoux_classical-and-quantum-fields}, also with a description of quantization.
%
%
%
%
%
%
%%%
%
%
%
%%
%%%%%%%%%
%%%DIFFERENTIAL FORMS
%%%%%%%%
%
%
%
\subsection{Differential forms}\label{sec:differential_forms}
%
%
Differential forms provide an elegant, coordinate independent description of calculus on smooth manifolds. In particular, they generalize the notion of line- and volume-integrals that are known from analysis. Differential forms play a remarkable role in physics, as one can argue that they indeed describe fundamental physical entities. As an example, instead of viewing a classical force as a vector, one can think of it, more closely related to experiments, as a differential one-form that assigns a scalar to a tangent vector of a curve. This scalar is the (infinitesimal) work associated with the force along the curve. Also, differential forms allow for an elegant geometric description of field theories, for example the Maxwell and Proca field theories that we encounter in this thesis. In Maxwell's classical theory of electromagnetism, instead of viewing the electric and magnetic field (which are conceptually just forces) as the fundamental physical entities, one introduces the \emph{vector potential}, a one-form, consisting of the scalar electric potential and the vector potential associated with the magnet field. Experiments like the Aharonov-Bohm experiment allow for an interpretation of the vector potential as the fundamental physical object, rather than the associated electromagnetic field. \\
Even more fundamentally, the two main theories of physics, General Relativity and the Standard Model of particle physics, are field theories. They are deeply connected to a geometric interpretation and can be elegantly described using differential forms. \par
%
%
Despite of all this, differential forms are usually not part of the standard curriculum of physicists. We shall therefore introduce the basic aspects and definitions regarding differential forms that are used in this thesis. For a more detailed introduction we refer to the literature: For example \cite[Chapter 2 and 4]{rudolph_schmidt} or \cite[Appendix B]{wald_GR} provide introductions to the topic.\par
%
%
In the following, let $\N$ denote a smooth $N$-dimensional manifold, assumed to be Hausdorff, connected, oriented and para-compact, with either Lorentzian or Riemannian metric $k$ and Levi-Civita connection $\nabla$. For a Lorentzian manifold we use the sign convention $(-,+,\dots,+)$ of the metric $k$. The number of negative eigenvalues of $k$ is denoted by $s$, so $s=0$ for a Riemannian manifold and, in our convention, $s=1$ for a Lorentzian manifold.
Later, we will specify to a four dimensional (globally hyperbolic) spacetime consisting of a four dimensional manifold $\M$ with Lorentzian metric $g$ and Cauchy surface $\Sigma$ with induced Riemannian metric $h$.
%
We define:
\begin{definition}[Differential form]
	Let $p\in \{0,1,\dots,N\}$. A \emph{differential form} $\omega$ of degree $p$, or $p$-form for short, on the manifold $\N$ is an anti-symmetric tensor field of rank $(0,p)$. That is, at every point $x \in \N$, $\omega_x$ is an anti-symmetric multi-linear map
	\begin{align}
	\omega_x : \underbrace{T_x \N \times T_x \N \times \cdots \times T_x \N}_{p\text{-times}} \to \IR \formspace.
	\end{align}
	We denote the vector space\footnote{Naturally, addition and scalar multiplication are defined point-wise.} of $p$-forms on $\N$ by $\gls{omegap}$, the space with compactly supported ones by \gls{omegapz}.
\end{definition}
As an example, a zero-form $f \in \Omega^0(\N)$ is just a $C^\infty$-function from $\N$ to $\IR$, hence we can identify $\Omega^0(\N) = C^\infty (\N, \IR)$. A one-form $A \in \Omega^1(\N)$ is nothing more than a co-vector field and in a physical context usually denoted in local coordinates by $A_\mu$. Note, that alternatively one can directly define a $p$-form as a smooth section of the $p$-th exterior product of the co-tangent bundle and hence identify $\Omega^p(\N) = \Gamma \big( \largewedge^k T^*\N\big)$. As mentioned in Section \ref{sec:spacetime_geometry}, we view the tangent bundle as a complex bundle. Therefore, the sections of that bundle will be complex valued functionals. In that fashion, we will usually view the spaces $\Omega^p(\N)$ as complex valued differential forms.\par
%
Next we define the basic operations, besides addition and scalar multiplication, that one can perform on differential forms.
%
\begin{definition}[Exterior product]
	Let $A \in \Omega^p(\N)$ be a $p$-form and  $B\in \Omega^q(\N)$ a $q$-form on $\N$. \\
	The \emph{exterior product} $\gls{wedge}:\Omega^p(\N) \times \Omega^q(\N) \to \Omega^{p+q} (\N)$ is defined by
	\begin{align}
	(A \wedge B)_{\mu_1\dots\mu_p \nu_1\dots\nu_q} = \frac{(p+q)!}{p!q!}\, A_{[\mu_1 \dots \mu_p} B_{\nu_1\dots\nu_q]} \formspace,
	\end{align}
	where the anti-symmetrization of a tensor $T$ is given through
	\begin{align}
	T_{[\mu_1\dots\mu_p]} = \frac{1}{p!} \sum\limits_{\sigma\in S_N }\textrm{sgn}(\sigma) T_{\sigma(\mu_1)\dots\sigma(\mu_p)} \formspace.
	\end{align}
\end{definition}
Here, $S_N$ denotes the symmetric group\footnote{Usually the symmetric group is defined as the set of permutations of $\{1,2,\dots,N\}$ but we chose the index to run over $\{0,1,\dots,N-1\}$, identifying the time component with zero rather then one.} of degree $N$, consisting of permutations of the set $\{0,1,\dots,N-1\}$.
With this notion of multiplication, point-wise addition and scalar multiplication, the space $\gls{omega} \coloneqq \bigoplus_{p = 0}^\infty \Omega^p(\N) = \bigoplus_{p = 0}^N \Omega^p(\N)$ becomes an algebra, usually called the Grassmann- or \emph{exterior algebra} of differential forms on $\N$. We have used that obviously $\Omega^k(\N) =0$ for $k >N$ due to the anti-symmetrization.\par
Furthermore, we find a notion of how to \emph{pullback} differential forms on manifolds to another manifold, for example the pullback of a differential form on the spacetime $\M$ to differential forms on its Cauchy surface $\Sigma$. Given a $C^\infty$-map $\psi: \widetilde{\N} \to \N$, where $\N, \widetilde{\N}$ are manifolds, we can naturally define the pullback of a function $f \in \Omega^0(\N)$ to a function $(\psi^* f) \in \Omega^0(\widetilde{\N})$ by composing $f$ with $\psi$:
\begin{align}
\psi^* f \coloneqq f \comp \psi \formspace.
\end{align}
\newpage
With the pullback of functions defined, we can define how to \emph{push forward}, or carry along, vector fields on $\widetilde{\N}$ to vector fields on $\N$: Let $f\in \Omega^0(\N)$ and $\tilde{v} \in \Gamma(T\widetilde{\N})$ and $\tilde{x} \in \widetilde{\N}$. Then
\begin{align}
(\psi_* \tilde{v})_{\psi(\tilde{x})} (f) \coloneqq \tilde{v}_{\tilde{x}}(\psi^* f)
\end{align}
defines the vector field $(\psi_* v) \in \Gamma(T\N)$. With these basic operations at hand, we can generalize to define the pullback of differential forms:
\begin{definition}[Pullback]\label{def:pullback}
	Let $\N, \widetilde{\N}$ be manifolds of dimension $N,\widetilde{N}$ respectively, and let $\psi: \widetilde{\N} \to \N$ be a smooth map. Then, $\psi$ defines an algebra homomorphism $\psi^* : \Omega(\N) \to  \Omega(\widetilde{\N})$,
	called the \emph{pullback} of differential forms. For $\omega \in \Omega^p(\N)$, $\tilde{x} \in \widetilde{\N}$ and $\tilde{v}_i \in T_x \widetilde{\N}$, $i=1,2,\dots,p$, it is defined by
	\begin{align}
	\left( \psi^* \omega \right)_{\tilde{x}}  (\tilde{v}_1,\tilde{v}_2,\dots,\tilde{v}_p) \coloneqq \omega_{\psi(\tilde{x})} (\psi_* \tilde{v}_1, \dots , \psi_* \tilde{v}_p) \formspace.
	\end{align}
\end{definition}
%
%
%
%
On the exterior algebra we find a duality, provided by the Hodge operator:
\begin{definition}[Hodge dual]
	The hodge star operator $\gls{hodge}: \Omega^p(\N) \to \Omega^{N-p}(\N)$ is defined through
	\begin{align}
	B \wedge *A = \frac{1}{p!} B^{\mu_1\dots\mu_p}A_{\mu_1\dots\mu_p} \dvolk \formspace,
	\end{align}
	which yields the coordinate representation
	\begin{align}
	(*A)_{\mu_{p+1}\dots\mu_N} = \frac{\detk}{p!} \, \epsilon_{\mu_1\dots\mu_N} A^{\mu_1\dots\mu_p} \formspace.
	\end{align}
\end{definition}
Here, \gls{levicivita} denotes the fully antisymmetric tensor of rank $N$ (Levi-Civita symbol) satisfying $\epsilon_{12,\dots,N} =1$ and the \emph{volume element} \gls{dvolk} is defined by
\begin{align}
\left( \gls{dvolk} \right)_{\alpha_1\dots\alpha_N} = \detk \, \epsilon_{\alpha_1\dots\alpha_N} \formspace.
\end{align}
In a sense, the volume element describes how the curvature of the manifold deforms a unit volume.
The duality follows from the important property of the Hodge operator as stated in the following lemma:
\begin{lemma}
	Let $*$ denote the Hodge star operator on the exterior algebra $\Omega(\N) $. It holds that
	\begin{align}
	** = (-1)^{s+p(N-p)} \, \mathbbm{1} \formspace,
	\end{align}
	which is trivially equivalent to $*^{-1} = (-1)^{s+p(N-p)} \, *$.
\end{lemma}
\begin{proof}
	Let $A \in \Omega^p(\N)$ be a $p$-form on $\N$. Then:
	\begin{align}
	(*{*A})_{\mu_1 \dots \mu_p}
	&= \frac{\detk \, \detk}{p! \, (N-p)!} \; \epsilon_{\alpha_{p+1}\dots\alpha_N \mu_1 \dots \mu_p}\;\epsilon^{\alpha_{1}\dots\alpha_N}\;A_{\alpha_1\dots\alpha_p} \notag\\
	&= (-1)^{p(N-p)} \frac{\detk \, \detk}{p! \, (N-p)!} \; \epsilon_{\alpha_{p+1}\dots\alpha_N \mu_1 \dots \mu_p}\;\epsilon^{\alpha_{p+1}\dots\alpha_{N}\alpha_1\dots\alpha_p}\;A_{\alpha_1\dots\alpha_p}  \notag\\
	&= (-1)^{s+p(N-p)} \delta\indices{^{[\alpha_{1}}_{\mu_{1}}}\, \dots \, \delta\indices{^{\alpha_p ] }_{\mu_p}} \;A_{\alpha_1\dots\alpha_p} \notag\\
	&=  (-1)^{s+p(N-p)}\;A_{\mu_1\dots\mu_p} \formspace
	\end{align}
	We have used Lemma \ref{lem:epsilon_contraction} and, in the last step, that the anti-symmetrization is absorbed by contraction because $A$ is antisymmetric.
\end{proof}
%
%
%
%
%
Furthermore, we can equip the exterior algebra with a differentiable structure, introducing the notion of the exterior derivative.
\begin{definition}[Exterior derivative]
	The \emph{exterior derivative} $\gls{d}:\Omega^p(\N) \to \Omega^{p+1} (\N)$ is defined by the following properties:
	\begin{enumerate}
		\item $d$ is linear
		\item $d$ obeys a graded Leibniz rule: Let $A \in \Omega^p(\N)$ and  $B\in \Omega^q(\N)$, then
		\begin{align}
		d(A \wedge B) = dA \wedge B + (-1)^p \, A \wedge dB
		\end{align}
		\item $d$ is nilpotent, that is,  $d^2 = 0$.
	\end{enumerate}
	In local coordinates, this is equivalent to the representation
	\begin{align}
	(dA)_{\mu \alpha_1\dots\alpha_p} = (p+1)\, \nabla_{[\mu}A_{\alpha_1\dots\alpha_p]} \formspace.
	\end{align}
\end{definition}
An important property of the exterior derivative is that it commutes (or rather intertwines its action) with pullbacks (see \cite[Proposition 4.1.7]{rudolph_schmidt}).
A $p$-form $\omega \in \Omega^p(\N)$ is called \emph{exact} if there is a $(p-1)$-form $\alpha \in \Omega^{p-1}(\N)$ such that $\omega = d\alpha$. We call $\omega$ \emph{closed} if $d \omega =0$. Accordingly, the space of closed $p$-forms is denoted by \gls{omegapd}, the space of exact ones by \gls{domegap}. As usual, the ones with compact support are denoted by a subscript zero. Note, that every exact form is closed, using that $d$ is by definition nilpotent, but the reverse is in general not true. It does hold, however, on certain manifolds with trivial topology, such as Minkowski spacetime. This is expressed in the so called Poincar\'e-Lemma (see for example \cite[Chapter 4]{bott_tu}) based on the study of de Rham cohomology.\par
%
Moreover, $N$-forms can naturally be integrated. Using local coordinates and a partition of unity, we define the integral of $N$-forms via the well known integration on $\IR^N$:
\begin{definition}[Integration on manifolds]
	Let $\left\{U_\alpha, \psi_\alpha\right\}_\alpha$ be an atlas of the manifold $\N$ and $\left\{\chi_\alpha\right\}_\alpha$ a partition of unity subordinate to the locally finite open cover $\left\{U_\alpha\right\}_\alpha$. Let $x^\mu_{(\alpha)}$ be a coordinate basis of $\psi$ on $U_\alpha$. For any $N$-form $\omega \in \Omega^N_0(\M)$ we define the integral
	\begin{align}
	\int\limits_{\N} \omega &\coloneqq \sum_{\alpha} \int\limits_{\psi_\alpha (U_\alpha)} w(x_{(\alpha)}^0,\dots,x_{(\alpha)}^1)\; dx_{(\alpha)}^0 \cdots dx_{(\alpha)}^{N-1} \formspace,
	\end{align}
	where $w$ are the components of $\omega$ in the coordinates $x_{(\alpha)}^\mu$, that is $\omega = w dx_{(\alpha)}^0 \wedge \cdots \wedge dx_{(\alpha)}^{N-1}$.
	This definition is independent of the choice of the atlas and the partition of unity (see \cite[Proposition 3.3]{bott_tu}).
\end{definition}
With integration at our disposal, we present an important theorem regarding the integration of exact differential forms:
\begin{theorem}[Stoke's Theorem]\label{thm:stokes}
	Let $\N$ be an oriented manifold of dimension $N$ and let its boundary $\partial \N$ be endowed with the induced orientation. Let $\gls{inclusionmap} : \partial \N \hookrightarrow \N$ be the inclusion operator.
	Let $\omega \in \Omega^{N-1}_0(\N)$ be a compactly supported $(N-1)$-form on $\N$. Then it holds
	\begin{align}
	\int\limits_\N d\omega = \int\limits_{\partial \N} i^*\omega \formspace.
	\end{align}
\end{theorem}
\begin{proof}
	A proof is given in most of the introductory literature on differential geometry (see for example \cite[Chapter 17, Theorem 2.1]{lang}).
	Note that one can equivalently formulate Stoke's theorem on a \emph{compact} manifold but for {arbitrary} (that is, in general not compactly supported) $(N-1)$-forms on the manifold (see for example \cite[Theorem 4.2.14]{rudolph_schmidt}). This will be of importance in later calculations.
\end{proof}
%
Furthermore, we can define a bilinear map on $\Omega^p(\N)$ using the integration of $N$-forms:
\begin{definition}
	Let $A,B \in \Omega^p(\N)$ such that their supports have a compact intersection. Define the bilinear map $\gls{innerprod} : \Omega^p(\N) \times \Omega^p(\N) \to \IC$ by
	\begin{align}
	\langle A, B \rangle_\N \coloneqq  \int_{\N } A \wedge * B = \int_{\N } A_{\mu_1 \dots \mu_p}B^{\mu_1 \dots \mu_p}\,\dvolk \formspace.
	\end{align}
\end{definition}
Since by definition $A \wedge * B$ is a compactly supported $N$-form, this is well defined. We may sometimes refer to $\langle \cdot , \cdot \rangle_\N$ as an inner product for simplicity, even though it is not positive definite.
%
%
%
%
%
Using the exterior derivative, we define the interior or co-derivative:
\begin{definition}[Interior derivative]
	The \emph{interior derivative} $\gls{delta} : \Omega^p(\N) \to \Omega^{p-1}(\N)$ is defined by
	\begin{align}
	\delta \coloneqq (-1)^{s+1+N(p-1)}\, {*{d*}} \formspace.
	\end{align}
	From the defining properties of $d$ and $*$ it follows $\delta^2 =0$.
\end{definition}
Here, $s$ again denotes the number of negative eigenvalues of the metric $k$ of $\N$. In accordance with our nomenclature, we call a $p$-form $\omega$ co-exact if there exists a $\alpha \in \Omega^{p+1}(\N)$ such that $\omega = \delta \alpha$ and co-closed if $\delta \omega = 0$. Accordingly, the spaces of co-closed and co-exact $p$-forms are denoted by \gls{omegapdelta} and \gls{deltaomegap} respectively.\par
Using the exterior and interior derivative we define the partial differential operator:
\begin{definition}[D'Alembert Operator]
	The d'Alembert (or Laplace - de Rham) operator $\gls{dalembert}: \Omega^p(\N) \to \Omega^{p}(\N)$ is defined by
	\begin{align}
	\square \coloneqq \delta d +d \delta \formspace.
	\end{align}
\end{definition}
By definition of the exterior and interior derivative, it is easy to show that $\square$ commutes with both $d$ and $\delta$:
\begin{align}
\square d &= (\delta d + d \delta )d \notag \\
&= d \delta d \notag \\
&= d (\delta d + d \delta) \formspace,
\end{align}
and analogously for $\delta$.
The d'Alembert operator, and its generalization to $(\square + m^2)$ for some constant $m > 0$, are important examples for a normally hyperbolic differential operators (see Section \ref{sec:global_hyperbolicity}) and we may therefore sometimes just refer to them as \emph{wave operators}.\par
The sign convention in the definition of the exterior derivative is chosen such that on any Lorentzian or Riemannian manifold the interior derivative is formally adjoint to the exterior derivative, that is,  for $A \in \Omega^{p}(\N)$ and $B \in \Omega^{p+1}(\N)$ it holds that
\begin{align}
\langle dA , B \rangle_{\N} = \langle A , \delta B \rangle_\N \formspace,
\end{align}
which leads to a representation in local coordinates of the Manifold given by:
\begin{align}
(\delta A)_{\mu_2\dots\mu_p} = - \nabla^{\mu_1}A_{\mu_1\dots\mu_p} \formspace.
\end{align}
To see that this is consistent, let $A \in \Omega^{p-1}(\N)$ and $B \in \Omega^{p}(\N)$ such that their supports have compact intersection.
We obtain, using Stoke's Theorem \ref{thm:stokes}:
\begin{align}
0 &= \int \limits_{\partial \N} i^* (A \wedge *B) \notag\\
&= \int \limits_{\N} d(A \wedge *B)  \notag\\
&= \int \limits_{\N} dA \wedge *B + (-1)^{p-1} A \wedge d{*B} \notag\\
&= \int \limits_{\N} dA \wedge *B + (-1)^{p-1} A \wedge *{*^{-1}}\underbrace{d{*B}}_{\textrm{is a } (N-p+1) \textrm{ form.}} \notag\\
&= \int \limits_{\N} dA \wedge *B + (-1)^{p-1}(-1)^{s+(N-p+1)(N-N+p-1)} A \wedge *{*d{*B}} \notag\\
&= \int \limits_{\N} dA \wedge *B + (-1)^{p+(1-p)(p-1)} A \wedge *\delta B \formspace.
\end{align}
It can easily be proven by induction that $\big(p+(1-p)(p-1)\big)$ is odd for any $p \in \IN$, which yields the result
\begin{align}
\langle dA , B \rangle_{\N} = \langle A , \delta B \rangle_\N \formspace.
\end{align}
The definitions stated above thus fulfill the requirement of formal adjointness of the exterior and interior derivate on an arbitrary Lorentzian or Riemannian manifold $\N$.
In local coordinates we use a partial integration to obtain
\begin{align}
\langle dA , B \rangle_\N &= \int \limits_{\N} dA \wedge * B \notag\\
%&= \int \limits_{\N} \frac{1}{p!} (dA)^{\alpha_1\dots\alpha_p}\,B_{\alpha_1 \dots \alpha_p} \, \dvolk \notag\\
&= \int \limits_{\N}  \frac{p}{p!} \nabla^{[\alpha_1}A^{\alpha_2\dots\alpha_p]}\,B_{\alpha_1 \dots \alpha_p} \, \dvolk \notag\\
&= \int \limits_{\N}  \frac{1}{(p-1)!} \nabla^{\alpha_1}A^{\alpha_2\dots\alpha_p}\,B_{\alpha_1 \dots \alpha_p} \, \dvolk \notag\\
&= - \int \limits_{\N}  \frac{1}{(p-1)!} A^{\alpha_2\dots\alpha_p}\, \nabla^{\alpha_1}B_{\alpha_1 \dots \alpha_p} \, \dvolk \notag\\
&= \langle A, \delta B \rangle_\N \formspace,
\end{align}
which yields
\begin{align}
-\nabla^{\alpha_1}B_{\alpha_1 \dots \alpha p} = (\delta B)_{\alpha_2 \dots \alpha_p}\formspace.
\end{align}
On the four dimensional spacetime $(\M,g)$ the definitions of the Hodge star operator and the interior derivative simplify, such that
\begin{align}
*_{(\M)}*_{(\M)} &= (-1)^{p+1} \mathbbm{1} \\
\delta_{(\M)} &= *_{(\M)}{d_{(\M)}*_{(\M)}} \formspace ,
\end{align}
holds on the spacetime $(\M,g)$ and
\begin{align}
*_{(\Sigma)}*_{(\Sigma)} &= \mathbbm{1} \\
\delta_{(\Sigma)} &= (-1)^p *_{(\Sigma)}{d_{(\Sigma)}*_{(\Sigma)}}
\end{align}
holds on  $(\Sigma,h)$. In the following we will drop the subscript ${(\M)}$, since we will perform all the calculations on a four dimensional spacetime, except when explicitly noted (for example with a subscript $(\Sigma)$).
%
%
%
%
%
%
%
%
%%%%%%
%%CATEGORY THEORY
%%%%%%
\subsection{Category theory}\label{sec:cat-theory}
The description of Quantum Field Theory on Curved Spacetimes (QFTCS) in the framework of \name{Brunetti}, \name{Fredenhagen} and \name{Verch} \cite{Brunetti_Fredenhagen_Verch} is based on category theory. In this thesis, we will not go into detail on those categorical aspects, however we will need some basic definitions to formulate the theory rigorously, that is namely the notion of a category and that of covariant functors, since, in the used framework, the generally covariant QFTCS is a functor.\par
Here, we present definitions given in \cite[Appendix A.1]{baer_ginoux_pfaeffle} and refer to the appropriate literature for details. We define:
\begin{definition}[Category]
	A \emph{category} $\mathsf{Cat}$ consists of the following:
	\begin{enumerate}
		\item a class $\mathsf{Obj}_\mathsf{Cat}$ whose members are called \emph{objects},
		\item a set $\mathsf{Mor}_\mathsf{Cat}(A,B)$, for any two objects $A,B \in \mathsf{Obj}_\mathsf{Cat}$, whose elements are called \emph{morphisms},
		\item for any three objects $A,B,C \in \mathsf{Obj}_\mathsf{Cat}$ there is a map
		\begin{align}
\mathsf{Mor}_\mathsf{Cat}(B,C) \times \mathsf{Mor}_\mathsf{Cat}(A,B) &\to \mathsf{Mor}_\mathsf{Cat}(A,C) \notag\\
(\psi,\phi) &\mapsto \psi \comp \phi
		\end{align}
		called the composition of morphisms subject to the relations:\vspace{4mm}
		\begin{enumerate}[label=(\arabic*)]
			\item for non equal pairs $(A,B)$, $(A',B')$ of objects, the sets $\mathsf{Mor}_\mathsf{Cat}(A,B)$ and $\mathsf{Mor}_\mathsf{Cat}(A',B')$ are disjoint,
			\item for every object $A$ there exists a morphism $\text{id}_A \in \mathsf{Mor}_\mathsf{Cat}(A,A)$ such that it holds for all objects $B$, morphisms $\psi \in \mathsf{Mor}_\mathsf{Cat}(B,A)$ and $\phi \in \mathsf{Mor}_\mathsf{Cat}(A,B)$
			\begin{align}
				\text{id}_A \comp \psi &= \psi \quad \text{and}\\
				\phi \comp \text{id}_A &= \phi \quad,
			\end{align}
			\item the composition law is associative, that is for an objects $A,B,C,D$ and any morphisms $\psi \in \mathsf{Mor}_\mathsf{Cat}(A,B)$, $\phi \in \mathsf{Mor}_\mathsf{Cat}(B,C)$ and $\chi \in \mathsf{Mor}_\mathsf{Cat}(C,D)$ it holds
			\begin{align}
				(\chi \comp \phi) \comp \psi = \chi \comp (\phi \comp \psi) \formspace.
			\end{align}
		\end{enumerate}
	\end{enumerate}
\end{definition}
%
%
%
\begin{definition}[Functor]
	Let $\mathsf{Cat1}$ and $\mathsf{Cat2}$ be categories. A \emph{covariant functor} $\mathscr{A}: \mathsf{Cat1} \to \mathsf{Cat2}$ consists of the map $\mathscr{A} : \mathsf{Obj}_\mathsf{Cat1} \to \mathsf{Obj}_\mathsf{Cat2}$ and maps $\mathscr{A}: \mathsf{Mor}_\mathsf{Cat1}(A,B) \to \mathsf{Mor}_\mathsf{Cat2}\big(\mathscr{A}(A),\mathscr{A}(B)\big)$ for any two objects $A,B \in \mathsf{Obj}_\mathsf{Cat1}$ such that
	\begin{enumerate}
		\item {the composition is preserved, that is for all objects $A,B,C \in \mathsf{Obj}_\mathsf{Cat1}$ and for any morphisms $\psi \in \mathsf{Mor}_\mathsf{Cat1}(A,B)$ and $\phi \in \mathsf{Mor}_\mathsf{Cat1}(B,C)$ it holds
		\begin{align}
			\mathscr{A}(\phi \comp \psi) = \mathscr{A}(\phi) \comp \mathscr{A}(\psi) \formspace,
		\end{align}}
		\item{
			$\mathscr{A}$ maps identities to identities, that is for any object $A \in \mathsf{Obj}_\mathsf{Cat1}$ it holds
			\begin{align}
				\mathscr{A}(\text{id}_\mathsf{A}) = \text{id}_{\mathscr{A}(A)} \formspace.
			\end{align}
			}
	\end{enumerate}
\end{definition}
%
%
%
%
%
%
%
%
%
%
%
%
%%%%%%
%%SIGN CONVENTIONS
%%%%%%
%
%
\subsection{Sign conventions}\label{sec:sign_conventions}
At certain points throughout this chapter we have had a freedom of choice regarding the signs of some entities, in particular the sign of the signature of the Lorentzian metric $g$ and that of the interior derivative $\delta$. Though at this stage the choice can be made arbitrarily, we want to make it in a way that in the end allows us to make certain physical interpretations on some parameters. More precisely, we want to interpret the parameter $m$ of the Klein-Gordon equation\footnote{or its generalization on $p$-forms} $(\square + m^2) f = 0$ for a zero-form $f \in \Omega^0(\M)$ as a mass in the physical sense. With the chosen sign convention for $\delta$ we find, using ${\delta}f = 0$:
\begin{align}
	\square f
	&= (\delta d + d \delta) f \notag\\
	&= \delta d f \notag\\
	&= - \nabla^\mu \nabla_\mu f \formspace.
\end{align}
In the following heuristic (local) argument we see
\begin{align}
	\square + m^2
	&= -\nabla^\mu \nabla_\mu + m^2 \notag\\
	&\sim \partial_t^2 + \sum_i \partial_i^2 + m^2\notag\\
	&\sim -E^2 + \abs{\vector{p}}^2 + m^2
\end{align}
which yields the correct relativistic relation of energy, momentum and mass according to $E^2 = \abs{\vector{p}}^2 + m^2$.
A similar calculation holds for the Klein-Gordon operator generalized to act on one-forms. If we had found a ``wrong'' relation between energy, momentum and mass, we would have had to adapt the chosen signs. Usually one chooses the sign of the metric and the interior derivative such that they are in some sense mathematically convenient (although one might disagree with another one's choice). We have made the choice of the metric, such that the Cauchy surfaces become Riemannian rather that ``anti-Riemannian'' (with an all minus signature), which seems more natural to some. Also, a lot of the used references on spacetime geometry (in particular the book by \name{Wald} \cite{wald_GR}) use this sign convention, which makes the application of certain formulas easier. As mentioned, the sign of the interior derivative was chosen such that it is formally adjoint to the exterior derivative (with respect the specified inner product) on all Lorentzian and Riemannian manifolds. It seemed convenient for the actual calculations to fix the sign regardless of the signature of the metric of the underlying manifold. One could equivalently have fixed the opposite sign, yielding the two derivatives to be skew-adjoint, which is also done in the literature. However, in the end, one has one freedom left to make the energy-momentum-mass relation work: that is the sign in front of the mass in the Klein-Gordon equation and all other wave equations accordingly. Hence, one regularly also finds the Klein-Gordon equation to be defined with a flipped sign of the mass term. But for our case, we want the mass $m$ in any wave equation to appear with a positive sign.
%
%


\section{Local results}
\label{s:local}
%!TEX root = hopfwright.tex

\subsection{Constructing a Newton-like operator}
\label{s:newtonlike}

In this section and in the appendices we often suppress the subscript in $F=F_\epsilon$.
We will find solutions to the equation $F(\alpha ,\omega , c)=0$ by the
constructing a Newton-like operator $T$ such that fixed points of $T$
corresponds precisely to zeros of $F$. In order to construct the map $T$ we
need an operator $A^{\dagger}$ which is an approximate inverse of 
$DF(\bx_\epsilon)$. 
% Since
% $\bx_\epsilon$ is an approximate zero of $F_\epsilon$ up to order
% $\cO(\epsilon^2)$ correction terms,
We will use an approximation $A$ of 
$DF( \bx_\epsilon )$ that is linear in~$\epsilon$ and correct up to $\cO(\epsilon^2)$.
% (recall that $F(\bx_\epsilon)=\cO(\eps^2)$). 
Likewise, we define $A^{\dagger}$ to be linear in $\epsilon$ (and again correct up to $\cO(\epsilon^2)$). 

It will be convenient to use the usual identification $i_\C : \R^2 \to \C$ given by $i_\C (x,y) = x+iy $. We also use $\omega_0 := \pi/2$.
 % order
% Since $x(\epsilon)$ is only correct up to order $\cO(\epsilon^2)$, then it only makes sense to compute our approximate derivative up to order $\cO( \epsilon^2)$.

% \marginpar{Jonathan: I tried to be careful about the spaces here, but it all seems a bit of a distraction since everything is explicit in coordinates}
\begin{definition}\label{def:A}
We introduce the linear maps $A:  \R^2 \times \ell^K_0 \to \ell^1$ and 
$ A^{\dagger}:  \ell^1 \to  \R^2 \times \ell^K_0 $ by
\begin{alignat*}{1}
A &:= A_0 + \epsilon A_1 \, , \\
A^{\dagger} &:= A_0^{-1} - \epsilon A_0^{-1} A_1 A_0^{-1} \,  ,
%\label{eq:ADagger}
\end{alignat*}
where the linear maps $ A_0 , A_1 : \R^2 \times \ell^K_0 \to \ell^1$  are defined below. Writing $x=(\alpha,\omega,c)$, we set
\begin{alignat*}{1}
A_0	x = A_0 (\alpha,\omega,c) & := i_\C A_{0,1} 
\!\left[\!\! \begin{array}{c} \alpha \\ \omega \end{array} \!\!\right]  \e_1
 + A_{0,*}  c , \\
A_1 x =	A_1 (\alpha,\omega,c) & := i_\C  A_{1,2}
\!\left[\!\! \begin{array}{c} \alpha \\ \omega \end{array} \!\!\right]  \e_2
 + A_{1,*}  c .
%\label{eq:ApproxDFdef}
\end{alignat*}
Here the matrices $A_{0,1}$ and $A_{1,2}$ are given by
\begin{equation}
A_{0,1} := 
\left[
\begin{matrix}
0 & - \pp \\
-1  & 1 
\end{matrix} 
\right]
\qquad\text{and}\qquad
A_{1,2} := \frac{1}{5}
\left[
\begin{matrix}
-2 & 2-\tfrac{3 \pi}{2} \\
-4  & 2(2+\pi) 
\end{matrix}  
\right]  ,
\label{eq:defA12}
\end{equation}
and the linear maps $A_{0,*} : \ell^K_0 \to \ell^1_0$ and
$A_{1,*} : \ell^K_0 \to \ell^1$
are given by
\begin{equation*}
% A_{0,*} :& \ell^1_0 \to \ell^1_0
% &
% A_{1,*} :& \ell^1_0 \to \ell^1  \\
% %%%%%%
% A_{0,1} :& \{ \alpha, \omega\} \to \{ Re \, F_1 , Im\, F_1 \}
% &
% A_{1,2} :& \{ \alpha, \omega\} \to \{ Re \,F_2 , Im \, F_2 \}
% \end{align*}
% and given by the equations below, taking $ \omega_0 = \pp$.
% \begin{align}
A_{0,*} 	 := \tfrac{\pi}{2} ( i K^{-1} + U_{\omega_0}) 
\qquad\text{and}\qquad
A_{1,*} 	:= \tfrac{\pi}{2} L_{\omega_0} .
\end{equation*}
%%%%%%%%%%%%%%%%%%%%
% A_{0,1} := &
% \left[
% \begin{matrix}
% 0 & - \pp \\
% -1  & 1
% \end{matrix}
% \right]
% &
% A_{1,2} :=& \frac{1}{5}
% \left[
% \begin{matrix}
% -2 & 2-\tfrac{3 \pi}{2} \\
% -4  & 2(2+\pi)
% \end{matrix}
% \right]
% \label{eq:defA12}
% \end{align}
\end{definition}

Since $K$ and $U_{\omega_0}$ both act as diagonal operators, the inverse 
$A_{0,*}^{-1} : \ell^1_0 \to \ell^K_0$ of $A_{0,*}$ is given by
\begin{equation*}
	  (A_{0,*}^{-1} a)_k = \frac{2}{\pi} \frac{a_k}{ik+e^{-ik\omega_0}} 
	  \qquad\text{for all } k \geq 2.
\end{equation*} 
An explicit computation, which we leave to the reader, shows that these approximations are indeed correct up to $\cO(\epsilon^2)$. 
In particular, $A^{\dagger} = \left[ DF( \bx_\epsilon ) \right]^{-1} + \cO(\epsilon^2)$.
In Appendix~\ref{sec:OperatorNorms} several additional properties of these operators are derived. The most important one is the following.
% \note[J]{I've tried to make this change for the new injectivity bound throughout.} \note[JB]{Seems fine, but wouldn't it be nicer to write $\tfrac{\sqrt{10}}{4}$ instead of $\tfrac{5}{2 \sqrt{10}}$?} \note[J]{Yes it would, made changes below. }
\begin{proposition}
	\label{prop:Injective}
	For 
%\change[J]{$0 \leq \epsilon < \tfrac{5}{2} ( 4 + \sqrt{10})^{-1} \approx 0.349$}
	$0 \leq \epsilon < \tfrac{\sqrt{10}}{4} \approx 0.790$
	 the operator $ A^{\dagger}$ is injective. 
\end{proposition}
\begin{proof}
	In order to show that $ A^{\dagger}$ is injective we show that 
	it has a left inverse. 
	Note that $ A A^{\dagger} = I - \epsilon^2 ( A_1 A_0^{-1})^2$. 
	By Proposition \ref{prop:A1A0} it follows that 
%	\change[J]{$ \| A_1 A_0^{-1} \| \leq \tfrac{2}{5} ( 4 + \sqrt{10})$}
	 $ \| A_1 A_0^{-1} \| \leq \tfrac{2 \sqrt{10}}{5} $.  
	By choosing 
%\change[J]{$ \epsilon < \tfrac{5}{2} ( 4 + \sqrt{10})^{-1}$}
$ \epsilon < \tfrac{\sqrt{10}}{4}$ 
we obtain 
	$\|  \epsilon^2 ( A_1 A_0^{-1})^2 \| < 1$, whereby $ A A^{\dagger}$ is 
	invertible, and so $ A^{\dagger}$ is injective. 
\end{proof}


\begin{definition}
We define the operator $ T: \R^2 \times \ell^K_0 \to \R^2 \times \ell^K_0 $ by
\begin{equation*}
	T(x) :=  x - A^{\dagger} F(x) ,
\end{equation*}
	where  $F$ is defined in Equation~\eqref{eq:FDefinition}  and $A^{\dagger}$ in Definition~\ref{def:A}.
	We note that $F$, $A^{\dagger}$ and $T$ depend on the parameter $\epsilon \geq 0$, although we suppress this in the notation.
\end{definition}

% \note[J]{When we got the better bound on $\|A_1 A_{0}^{-1}\|$, then $A^{\dagger}$ being injective ceased to be a bottleneck for doing a Hopf bifurcation. I don't think we'd lose much if we just delete this remark.  }
% \begin{remark}
% 	\label{r:Injective}
% \remove[J]{
% 	If $A^{\dagger}$ is injective, which is true for
% 	$0 \le \epsilon <  \tfrac{5}{2} ( 4 + \sqrt{10})^{-1}$ by Proposition 3.2, then the fixed points of $T$ correspond bijectively with the zeros of $F$.
% 	Since the periodic solution having $ \epsilon_0 = \tfrac{5}{2} ( 4 + \sqrt{10})^{-1}$ corresponds approximately to $\bar{\alpha}_{\epsilon_0} = \pp + 0.090$, above this value we cannot use the Newton-like operator $T$ to reliably study the SOPS to Wright's equation.
% 	Hence $ \alpha = \pp + 0.09$ represents an upper bound for doing an $\cO(\epsilon^2)$ Hopf bifurcation analysis.}
% \end{remark}
%


\subsection{Explicit contraction bounds}
\label{s:contraction}


The map $T$ is not continuous on all of $\R^2 \times \ell^K_0$,
since $ U_{\omega} c $ is not continuous in $\omega$.
While continuity is ``recovered'' for terms of the form $A^{\dagger} U_{\omega} c$,  this is not the case for the nonlinear part $ - \alpha \epsilon A^{\dagger} [ U_{\omega} c ] * c$.  
% The problem is that while in the $ U_{\omega} c$ term and that  $ \tfrac{\partial}{\partial \omega} U_{\omega} = - i K^{-1} U_{\omega}$.
% Since  the map $ A^{\dagger}$ is approximately $\tfrac{2 }{\pi i} K$, then the  $  A^{\dagger}  U_{\omega} c$ component of $ T$ is continuous in $\omega$.
%
%For any $ \omega_1, \omega_2\in \R$  then $ \| U_{\omega_1} - U_{\omega_2} \|  = 2$ when $ 2 \pi $ does not divide $ \omega_1 - \omega_2$.  
%This is not a problem for the $ A^{\dagger} ( i \omega K^{-1} + \alpha U_{\omega} ) c$ component of $T$; 
%the and then $ A^{\dagger } U_{\omega}$ is continuous in $\omega$.  
% However, since $ U_{\omega}$ is inside a convolution in the nonlinear part, this type of simplification cannot happen.
% 
We overcome this difficulty by fixing some $ \rho > 0$ and restricting the domain of $T$ to sets of the form 
%$\R^2 \times X_\rho$, where
\[
  \R^2 \times  \{ c \in \ell^K_0 : \|K^{-1} c\| \leq \rho \} = \R^2 \times \ell_\rho.
\]
Since we wish to center the domain of $T$ about the approximate solution~$\bx_\epsilon$, we introduce the following definition, which uses a triple of radii $r \in \R^3_+$, for which it will be convenient to use two different notations:
\[
  r = ( r_{\alpha } , r_{\omega} , r_c) = (r_1,r_2,r_3).
\]
\begin{definition}
	Fix   $ r \in \R^3_+$ and $ \rho > 0$ and let  $ \bx_\epsilon = ( \balpha_\epsilon , \bomega_\epsilon , \bc_\epsilon )$ be as defined in Definition~\ref{def:xepsilon}. 
    We define the $\rho$-ball $B_\epsilon(r,\rho) \subset \R^2 \times \ell^1_0$
    of radius $r$ centered at $\bx_\epsilon$ to be the set of points satisfying 
\begin{alignat*}{1}
	|  \alpha -\balpha_\epsilon | & \leq  r_\alpha  \\
	| \omega - \bomega_\epsilon  | & \leq  r_{\omega} \\
	\| c - \bc_\epsilon  \| & \leq r_c \\
	\|K^{-1} c\| & \leq  \rho .
\end{alignat*}
\end{definition}

We want to show that $T$ is a contraction map on some $\rho$-ball 
$B_\epsilon(r,\rho) \subset \R^2 \times \ell^1_0$ using a Newton-Kantorovich argument. 
This will require us to develop a bound on $DT$ using some norm on  $ X$.  
Unfortunately there is no natural choice of norm on the product space $ X$. 
Furthermore, it will not become apparent if one norm is better than another until after  significant calculation.  
For this reason, we use a notion of an ``upper bound'' which allows us to delay our choice of norm. 
We first introduce the operator $\zeta:  X  \to \R^3_{+}$
which consists of the norms of the three components:
\[
  \LL(x) :=   ( |\pi_\alpha x|, |\pi_\omega x|, \|\pi_c x\| )^T \in \R^3_{+}
  \qquad\text{for any } x \in X.
\]
% \note[JB]{Propose to revert to using single overlines for the upper bound; the double overlines I had introduced just look ridiculous to me now. There is a potential for confusion with $\bx_\epsilon$, but I can live with it.} \note[J]{I would also prefer using only one overline. }
\begin{definition}[upper bound]\label{def:upperbound}
We call $\upperbound{x} \in \R^3_+$ an upper bound on $x$ if $\LL(x) \leq \upperbound{x}$, where the inequality is interpreted componentwise in $\R^3$. 
Let $X'$ be a subspace of $X$ and let $X''$ be a subset of $X'$.   
An upper bound on a linear operator $A' : X' \to X $ over $X''$  is 
a $3 \times 3$ matrix $\upperbound{A'} \in \text{\textup{Mat}}(\R^3 , \R^3)$ such that
\[
   \LL(A' x ) \leq \upperbound{A'} \cdot \LL(x)  
     \qquad\text{for any }  x \in X'',
\]
where the inequality is again interpreted componentwise in $\R^3$. 
The notion of upper bound conveniently encapsulates bounds on the different components of the operator $A'$ on the product space $X$. Clearly the components of the matrix $\upperbound{A'}$ are nonnegative.


		% 	Let $ (\alpha , \omega , c) \in \R^2 \times \ell^1_0$ and $  \upperbound{x} = ( x_{ \alpha } , x_{\omega} , x_{c}) \in \R^3_+$.
		% 	Then $ \upperbound{x}$ is an \emph{upper bound} on $(\alpha , \omega , c)$ if $ | \alpha | \leq x_\alpha$ and $ | \omega | \leq x_{\omega}$ and $ \| c\|\leq x_{c}$.
		% Similarly, suppose that $ A' : X \to \R^2 \times \ell^1_0$ is a linear operator, defined on some domain $ X \subset \R^2 \times \ell^1_0$.
		% Then $ \upperbound{A'} \in Mat(\R^3 , \R^3)$ is an \emph{upper bound } on $ A'$  if $ \upperbound{A'} \cdot \upperbound{x} $ is an upper bound on $ A' x$ whenever $ \upperbound{x} \in \R^3_+$ is an upper bound on all $ x \in X$.
\end{definition}
		% The notion of upper bounds commutes with vector addition and matrix multiplication.
		% That is, if $\upperbound{x} , \upperbound{y} \in \R^3_+ $ are upper bounds on $ x,y \in \R^2 \times \ell^1_0$, then $ \upperbound{x} + \upperbound{y}$ is an upper bound on $x +y$.
		% Similarly, if we have two linear operators $A'$ and $A''$ with upper bounds
		% $\upperbound{A'}$ and $\upperbound{A''}$, respectively, then $ \upperbound{A'} \cdot \upperbound{A''}$ is an upper bound for $ A' \circ A''$.
		% Furthermore, if $\upperbound{A'} \in Mat(\R^3,\R^3)$ is an upper bound, then the entries of this matrix are necessarily non-negative.
For example, in Proposition \ref{prop:A0A1} we calculate an upper  bound on the map $A_0^{-1} A_1$.  
As for the domain of definition of $T$, in practice we use $X' = \R^2 \times  \ell^K_0  $ and  $X'' = \R^2 \times  \ell_\rho  $.
The subset $X''$ does not always affect the upper bound calculation (such as in Proposition \ref{prop:A0A1}). 
However, operators such as $U_{\omega} - U_{\omega_0}$ have upper bounds which contain $\rho$-terms (see for example Proposition \ref{prop:OmegaDerivatives}).

Using this terminology, we state a ``radii polynomial'' theorem, which allows us to check whether $T$ is a contraction map. This technique has been used frequently in a computer-assisted setting in the past decade. Early application include~\cite{daylessardmischaikow,lessardvandenberg}, while a previous implementation in the context of Wright's delay equation can be found in~\cite{lessard2010recent}. 
Although we use radii polynomials as well, our approach differs significantly from the computer-assisted setting mentioned above. 
While we do engage a computer (namely the Mathematica file~\cite{mathematicafile}) to optimize our quantitative results, the analysis is performed essentially in terms of pencil-and-paper mathematics (in particular, our operators do not involve any floating point numbers).
In our current setup we employ \emph{three} radii as a priori unknown variables,
which builds on an idea introduced in~\cite{vandenberg}.
We note that in most of the papers mentioned above the notation of $A$ and $A^\dagger$ is reversed compared to the current paper.

As preparation, the following lemma (of which the proof can be found in Appendix~\ref{sec:CompactDomain})  provides an explicit choice for $\rho$, as a function of $\epsilon$ and $r$, for which we have proper control on the image of $B_\epsilon(r,\rho)$ under $T$.
\begin{lemma}\label{lem:Crho}
For any $\epsilon \geq 0$ and $r \in\R^3_+$, let $C=C(\epsilon,r)$ be given  by Equation~\eqref{eq:RhoConstant}. 
If $C(\epsilon,r) >0$  then 
% Proposition~\ref{prop:DerivativeEndo} states that
\begin{equation}\label{e:Cepsr} 
  \| K^{-1} \pi_c  T(x) \| \leq \rho 
  \quad\text{whenever } x \in B_\epsilon(r,\rho) \text{ and } \rho \geq C(\epsilon,r).
\end{equation}
%%%%\marginpar{Jonathan: please fix appendix to reflect this (and define $C$ there)}
Moreover, $C(\epsilon,r)$ is nondecreasing in $\epsilon$ and $r$. 
\end{lemma}

\begin{proof}
See Proposition~\ref{prop:DerivativeEndo}.
\end{proof}


\begin{theorem}
	\label{thm:RadPoly}
	Let  
%\change[J]{$0 \leq \epsilon < \tfrac{5}{2} ( 4 + \sqrt{10})^{-1} $}
 $0 \leq \epsilon < \tfrac{\sqrt{10}}{4} $  
 and fix $r = (r_\alpha, r_\omega, r_c) \in \R^3_+$. Fix $\rho > 0$ such that $ \rho \geq C(\epsilon,r)$, as given by Lemma~\ref{lem:Crho}.
 % as in Proposition \ref{prop:DerivativeEndo} (REFORMULATE TO POINT TO THE PREPARATION ABOVE).
%
Suppose that $Y(\epsilon) $ is an upper bound on $ T(\bx_\epsilon) - \bx_\epsilon$ and $Z(\epsilon , r ,\rho) $ a (uniform) upper bound on $ DT(x) $ for all $ x \in B_\epsilon(r,\rho)$. 
Define the \emph{radii polynomials}
$P :\R^5_+ \to \R^3 $  by 
 \begin{equation}
 \label{eq:RadPolyDef}
  P(\epsilon,r,\rho) := Y(\epsilon) - \left[ I - Z( \epsilon,r,\rho) \right] \cdot r  \,  .
 \end{equation}
If each component of $P(\epsilon,r,\rho)$ is negative, then there is  a unique $\hat{x}_\epsilon \in B_\epsilon( r , \rho)$ such that $F(\hat{x}_\epsilon) =0$. 
\end{theorem}

\begin{proof}    
Let $r \in \R^3_+$ be a triple such that $P(\epsilon,r,\rho)<0$.
By Proposition \ref{prop:Injective}, if 
%\change[J]{$\epsilon < \tfrac{5}{2} ( 4 + \sqrt{10})^{-1} $}
$\epsilon <\tfrac{\sqrt{10}}{4} $
then $ A^{\dagger}$ is injective. 
Hence $ \hat{x}_{\epsilon} $ is a fixed point of $T$ if and only if $ F( \hat{x}_{\epsilon}) = 0$.  
In order to show  there is a unique fixed point $ \hat{x}_{\epsilon}$, we show that $T$ maps  $ B_{\epsilon}(r,\rho) $ into itself and that $ T $ is a contraction mapping. 

We first show that $T: B_\epsilon(r,\rho) \to B_\epsilon(r,\rho)$. 
Since $ \rho \geq C(\epsilon,r)$ then by Equation~\eqref{e:Cepsr} it follows that $ \| K^{-1} \pi_c T( x) \| \leq \rho$ for all $ x \in B_\epsilon(r,\rho)$.
In order to show that $T(x) \in B_\epsilon(r,\rho)$, it suffices to show that $ r=(r_\alpha , r_\omega, r_c)$ is an upper bound on $ T(x) - \bx_\epsilon$
for all $ x \in B_\epsilon(r,\rho)$.  
We decompose 
%by breaking $T(x) - \bx_\epsilon$ into two parts: 
\begin{equation}\label{e:Tsplit}
	T(x) - \bx_\epsilon = [T(\bx_\epsilon) -\bx_\epsilon] +
	[T(x) - T(\bx_\epsilon)],
\end{equation}
and estimate each part separately. Concerning the first term,
by assumption, $Y(\epsilon)$ is an upper bound on $T(\bx_\epsilon) - \bx_\epsilon$. 
%
Concerning the second term, we claim that $ Z(\epsilon,r,\rho) \cdot r$ is an upper bound on $T(x) - T(\bx_\epsilon)$.
Indeed, we have the following somewhat stronger bound: 
% if $ x,y\in B_\epsilon(r,\rho)$ and $\upperbound{\xi}$ is an upper bound on $y-x$, then
% $Z(\epsilon,r,\rho) \cdot \upperbound{\xi}$ is an upper bound on $T(y) - T(x)$,
% i.e.,
\begin{equation}\label{e:DTisboundedbyZ}
	\LL(T(y) - T(x)) \leq Z(\epsilon,r,\rho) \cdot \LL(y-x)
	\qquad\text{for all } x,y \in B_\epsilon(r,\rho) .
\end{equation}
The latter follows from the mean value theorem, since 
$T$ is continuously Fr\'echet differentiable on $B_\epsilon(r,\rho)$.
%
% MORE DETAILED ARGUMENT PROBABLY NOT NEEDED?
% \begin{equation}
% \label{eq:ZIntegrationBound}
% 	T( y) - T(x) \leq Z(\epsilon,r,\rho) \cdot \upperbound{\xi}
% \end{equation}
% Since $T$ is continuously Frechet differentiable on $B_\epsilon(r,\rho)$,
% then it follows that $T(y) - T(x) = \int_x^y DT( z) dz  $.
% %Since $ B_\epsilon(r,\rho)$ is convex, then $z= x+ s(y-x) \in B_\epsilon(r,\rho)$ for all $s \in [0,1]$.
% By assumption $Z(\epsilon,r,\rho)$ is an upper bound on $DT(z)$ for all $z \in B_\epsilon(r,\rho)$.
% If $ \upperbound{\xi} $ is an upper bound on $ y-x$, then we obtain the inequality $ T(y) - T(x) \leq \int_0^1 Z(\epsilon,r,\rho) \cdot \upperbound{\xi} \, ds$, from which Equation \ref{eq:ZIntegrationBound} follows.
%
Since $r$ is an upper bound on $x - \bx_\epsilon$ for all $ x \in B_\epsilon(r,\rho)$, we find, by using~\eqref{e:Tsplit}, that  
% have obtained that
% \begin{eqnarray}
% 	T(\bx_\epsilon) - \bx_\epsilon &=& \left[ T(\bx_\epsilon) - \bx_\epsilon \right] +
% 	\left[ T(x) - T(\bx_\epsilon) \right] \\
% 	&\leq& Y(\epsilon) + Z(\epsilon,r,\rho) \cdot r
% \end{eqnarray}
% By assumption each component of Equation \ref{eq:RadPolyDef} is negative, so
$Y(\epsilon) + Z(\epsilon,r,\rho) \cdot r \leq r$ (with the inequality, interpreted componentwise, following from $P(\epsilon,r,\rho)<0$) is an upper bound on $T(x) - \bx_\epsilon$
for all $ x \in B_\epsilon(r,\rho)$.  
%It follows that $r$ is an upper bound on $T(x) - \bx_\epsilon $. 
That is to  say, if all of the  radii polynomials are negative, 
then  $T$ maps $B_\epsilon(r,\rho) $ into itself.

To finish the proof we show that $T$ is a contraction mapping. 
We abbreviate $Z=Z(\epsilon,r,\rho)$ and  recall that $r=(r_\alpha,r_\omega,r_c)=(r_1,r_2,r_3) \in \R^3_+$
is such that $Z \cdot r < r$, hence for some $\kappa <1$ we have
\begin{equation}\label{e:defkappa}
  \frac{(Z \cdot r)_i}{r_i} \leq \kappa  \qquad\text{for } i=1,2,3.
\end{equation}

We now need to choose a norm on $X$. 
We define a norm $ \| \cdot \|_r$ on elements $x = (\alpha,\omega,c) \in X$
by
\[  
\| (\alpha, \omega, c) \|_r := \max 
\left\{  		  
	 \frac{|\alpha|}{r_\alpha},
	 \frac{|\omega|}{r_\omega},
	 \frac{\|c\|}{r_{c}} \right\} , 
\]
or
\[
  \|x\|_r = \max_{i=1,2,3} \frac{ \LL(x)_i}{r_i}
  \qquad \text{for all } x \in X.
\]
% We also introduce the compatible norm $ \| \cdot \|_{\tilde{r}}$ on $\R^3$ by
% $ \| (y_1, y_2, y_3) \|_{\tilde{r}} = \max_{i=1,2,3 }
% \{\frac{|y_i|}/{r_i} \}$, so that $\|x\|_r = \| \LL(x) \|_{\tilde{r}}$ for all $x \in X$.
%
By using the upper bound $Z$, we bound the Lipschitz constant of $T$ on $B_\epsilon(r, \rho)$ as follows:
\begin{alignat*}{1}
 \| T(y) - T(x) \|_r 
 % &= \|\LL(T(y) - T(x)) \|_{\tilde{r} } \\
    &= \max_{i=1,2,3} \frac{\LL(T(y) - T(x))_i} {r_i} \\
    &\leq  \max_{i=1,2,3}  \frac{(Z \cdot \LL(y-x))_i}{r_i} \\
    &\leq  \max_{i=1,2,3} \max_{j=1,2,3}\frac{\LL(y-x)_j}{r_j}  
				   \frac{(Z \cdot r )_i}{r_i} \\
    & = \| y-x \|_r \max_{i=1,2,3} \frac{(Z \cdot r )_i}{r_i} \\
    & \leq \kappa \| y-x \|_r,
\end{alignat*}
where we have used~\eqref{e:DTisboundedbyZ} and~\eqref{e:defkappa} with $\kappa<1$.
% \sup_{x,y \in B_\epsilon(r,\rho)} \frac{\|T(y) - T(x) \|_r}{\| y- x\|_r}
% \leq
% \sup_{ x,y \in B_\epsilon(r,\rho)}
% \frac{\left \|
% Z(\epsilon,r,\rho) \cdot \upperbound{\xi}
% \right\|_{\tilde{r}} }{  \| y -x\|_{r}} ,
% \]
% where $\upperbound{\xi}$ is any upper bound on $y-x$, as before.
% \marginpar{THIS IS STILL NOT ENTIRELY CLEAR!}
% If $u \in \R^3$ and $ \|u\|_{\tilde{r}} =1$, then $ \| Z \cdot u\|_{\tilde{r}}$ is maximized when $u=r$.
% Hence $ Lip(T) \leq \| Z(\epsilon,r,\rho) \cdot r \|_r$.
% Since all of the radii polynomials are negative, then $ Z \cdot r < r$component wise, thus proving that $ \|Z \cdot r\|_r <1$ and
Hence $T:B_{\epsilon}(r,\rho) \to B_{\epsilon}(r,\rho)$ is a contraction with respect to the $\| \cdot \|_r$ norm.

% We have thereby proved that  $T:B_{\epsilon}(r,\rho) \to B_{\epsilon}(r,\rho)$ is a contraction mapping.
Since $B_\epsilon(r,\rho)$ with this norm is a complete metric space, by the Banach fixed point theorem $T$~has a unique fixed point $ \hat{x}_\epsilon \in B_\epsilon(r,\rho)$. 
Since $A^\dagger$ is injective,  it follows that $ \hat{x}_\epsilon$   is the unique point in $B_\epsilon(r,\rho)$ for which $ F(\hat{x}_\epsilon) =0$. 
\end{proof}

\begin{remark}\label{r:boundDT}
Under the assumptions in Theorem~\ref{thm:RadPoly},
essentially the same calculation as in the proof above
leads to the estimate
\[
  \| DT(x) y \|_r \leq \kappa \|y\|_r 
  \qquad \text{for all } y \in \R^2 \times \ell^K_0 , 
  \, x \in B_\epsilon(r,\rho),
\]
where $\kappa := \max_{i=1,2,3} (Z\cdot r)_i / r_i$.
\end{remark}


In Appendix \ref{sec:YBoundingFunctions} and Appendix \ref{sec:BoundingFunctions} we construct explicit upper bounds 
$Y(\epsilon)$ and $ Z(\epsilon,r,\rho)$, respectively.  
These functions are constructed such that their components are (multivariate) polynomials in $\epsilon$, $r$ and $ \rho$ with nonnegative coefficients, hence they are increasing in these variables. 
This construction enables us to make use of the uniform contraction principle. 

\begin{corollary}\label{cor:eps0}
Let 
%\change[J]{$0 <\epsilon_0 < \tfrac{5}{2} ( 4 + \sqrt{10})^{-1} $}
 $0 < \epsilon_0 < \tfrac{\sqrt{10}}{4} $ 
and fix some $r = (r_\alpha, r_\omega, r_c) \in \R^3_+$.  
Fix $\rho > 0$ such that $ \rho \geq C(\epsilon_0,  r)$, as given by Lemma~\ref{lem:Crho}.
% as in Proposition \ref{prop:DerivativeEndo}. 
%
Let $Y(\epsilon)$ and $Z(\epsilon,r,\rho)$ be the upper bounds as given in  Propositions~\ref{prop:Ydef} and~\ref{prop:Zdef}. 
Let the radii polynomials $P$ be defined by Equation~\eqref{eq:RadPolyDef}.


If each component of  $P(\epsilon_0, r,\rho)$ is negative, 
then for all $ 0 \leq \epsilon \leq \epsilon_0$ there exists a unique $ \hat{x}_\epsilon \in B_\epsilon(  r , \rho)$ such that $ F(\hat{x}_\epsilon) =0$.  
The solution $\hat{x}_\epsilon$ depends smoothly on $\epsilon$.
\end{corollary}
\begin{proof} 
	Let $0 \leq  \epsilon \leq \epsilon_0$ be arbitrary.
	Because $\rho \geq C(\epsilon_0, r) \geq C(\epsilon, r)$ by Lemma~\ref{lem:Crho},
	Theorem~\ref{thm:RadPoly} implies that it suffices to show that $ P(\epsilon, r ,\rho) <0$. 	
Since  the bounds 
$Y(\epsilon)$ and $ Z(\epsilon,r,\rho)$ are monotonically increasing in their arguments, it follows that $ P(\epsilon,r,\rho) \leq P(\epsilon_0,r,\rho) <0$.  
Continuous and smooth dependence on $\epsilon$ of the fixed point follows from the uniform contraction principle (see for example~\cite{ChowHale}). 
\end{proof}


Given the upper bounds $ Y(\epsilon)$ and $ Z( \epsilon ,r , \rho)$, 
trying to apply Corollary~\ref{cor:eps0} amounts to finding values of $ \epsilon, r_\alpha, r_\omega, r_c,\rho$ for which the radii polynomials are negative.
Selecting a value for $ \rho$ is straightforward: all estimates improve with smaller values of $\rho$, and Proposition \ref{prop:DerivativeEndo} (see also Lemma~\ref{lem:Crho}) explicitly describes the smallest allowable choice of $\rho$ in terms of $ \epsilon,r_\alpha,r_\omega,r_c$. 

Beyond selecting a value for $ \rho$, it is difficult to pinpoint what constitutes an ``optimal'' choice of these variables. 
In general it is interesting to find such  viable radii (i.e.\ radii such that $P(r)<0$) which are both large and small.  
The smaller radius tells us how close the true solution is to our approximate solution. 
The larger radius tells us in how large a neighborhood our solution is unique.  With regard to $\epsilon$, larger values allow us to describe functions whose first Fourier mode is large. However this will ``grow'' the smallest viable radius and ``shrink'' the largest viable radius. 

Proposition \ref{prop:bigboxes} presents two selections of variables which satisfy the hypothesis of Corollary~\ref{cor:eps0}.  
We check the hypothesis is indeed satisfied by using interval arithmetic.
All details are provided in the Mathematica file~\cite{mathematicafile}. 
While the specific numbers used may appear to be somewhat arbitrary (see also the discussion in Remark~\ref{r:largeradii})  they have been chosen to be used later in Theorem 
\ref{thm:WrightConjecture} and Theorem \ref{thm:UniqunessNbd}.  


%%%
%%%BY DOING SOME CHOICES THAT HAVE NO MOTIVATION AT THIS POINT, BUT THAT WILL TURN OUT TO BE USEFUL IN SECTION~\ref{s:global} WE PROVE THE FOLLOWING USING MATHEMATICA  FILES.\marginpar{todo}

\begin{proposition}
		\label{prop:bigboxes}
Fix the constants $ \epsilon_0$, $(r_\alpha, r_\omega,r_c)$  and $\rho$ according to one of the following choices:
% \begin{enumerate}
% \item[\textup{(a)}]  $ \epsilon_0 = 0.029 $ and $ (r_\alpha , r_ \omega , r_c) = (  0.21, \, 0.16 , \, 0.09 ) $ and $\rho = 1.01$;
% \item[\textup{(b)}]  $ \epsilon_0 = 0.087 $ and $ (r_\alpha , r_ \omega , r_c) = (  0.1501, \, 0.0626 , \, 0.2092 ) $ and $\rho = 0.5672$.
% \end{enumerate}
% \note[J]{Version with new numbers below}
\begin{enumerate}
	\item[\textup{(a)}]  $ \epsilon_0 = 0.029 $ and $ (r_\alpha , r_ \omega , r_c) = (  0.13, \, 0.17 , \, 0.17 ) $ and $\rho = 1.78$; 
	\item[\textup{(b)}]  $ \epsilon_0 = 0.09 $ and $ (r_\alpha , r_ \omega , r_c) = (  0.1753, \, 0.0941 , \, 0.3829 ) $ and $\rho = 1.5940$. 
\end{enumerate}
For either of the choices (a) and (b) we have the following: 
for all $0 \leq \epsilon \leq \epsilon_0$ there exists a unique point 
$(\hat{\alpha}_\epsilon,\hat{\omega}_\epsilon,\hat{c}_\epsilon) \in B_{\epsilon}(r,\rho)$ 
satisfying $F_\epsilon(\hat{\alpha}_\epsilon,\hat{\omega}_\epsilon,\hat{c}_\epsilon) = 0$ and 
\[ 	
 | \hat{\alpha}_\epsilon - \balpha_\epsilon| \leq r_\alpha , 
 \quad
 |\hat{\omega}_\epsilon - \bomega_\epsilon| \leq  r_\omega  ,
 \quad
 \| \hat{c}_\epsilon - \bc_\epsilon\| \leq r_c     ,
 \quad
 \| K^{-1} \hat{c}_\epsilon \| \leq \rho  .
\]
\end{proposition}
\begin{proof}
In the Mathematica file~\cite{mathematicafile}  we check, using interval arithmetic, that  $\rho \geq C(\epsilon_0, r)$ and  the radii polynomials $P(\epsilon_0,r,\rho)$ are negative for the choices (a) and (b). The result then follows from Corollary~\ref{cor:eps0}.	
\end{proof}


\begin{remark}\label{r:largeradii}	
In Proposition~\ref{prop:bigboxes} we aimed for large balls on which the solution is unique.
Even for a fixed value of $ \epsilon$, it is not immediately obvious how to find a ``largest'' viable radius $r$, 
since $r$ has three components. In particular, there is a trade-off between the different components of $r$. On the other hand, as explained in Remark~\ref{r:smallradii}, no such difficulty arises when looking for a ``smallest'' viable radius.
\end{remark}




We will also need a rescaled version of the radii polynomials, which takes into account the asymptotic behavior of the bound $Y$ on the residue $T(\bar{x}_\epsilon) -\bar{x}_\epsilon = - A^\dagger F(\bx_\epsilon)$  as $\epsilon \to 0$, namely it is of the form
$Y(\epsilon)= \epsilon^2 \tilde{Y}(\epsilon)$,
see Proposition~\ref{prop:Ydef}.
The proofs of the following monotonicity properties can be found in 
Appendices~\ref{sec:YBoundingFunctions} and~\ref{sec:BoundingFunctions}. 
\begin{lemma}\label{lem:YZ}
Let $\epsilon \geq 0$, $\rho >0$ 
and $r \in\R^3_+$. 
Then there are upper bounds
$Y(\epsilon) =\epsilon^2 \tilde{Y}(\epsilon)$ on $ T(\bx_\epsilon) - \bx_\epsilon$ and a (uniform) upper bound 
$Z(\epsilon , r ,\rho) $  on $ DT(x) $ for all $ x \in B_\epsilon(r,\rho)$.
These bounds are given explicitly by Propositions~\ref{prop:Ydef} and~\ref{prop:Zdef}, respectively. Moreover, $\tilde{Y}(\epsilon)$ is nondecreasing in $\epsilon$,
while $Z(\epsilon , r ,\rho) $ is nondecreasing in $\epsilon$, $r$ and $\rho$.
\end{lemma}

This implies, roughly speaking, that if we are able to show that $T$ is a contraction map on 
$B_{\epsilon_0}( \epsilon_0^2 \rr,\rho)$ for a particular choice of $ \epsilon_0$, then it will be a contraction map on $B_\epsilon( \epsilon^2 \rr,\rho)$ for all $ 0 \leq \epsilon \leq \epsilon_0$. Here, and in what follows, we use the notation $r = \epsilon^2 \rr$ for the $\epsilon$-scaled version of the radii. 



\begin{corollary}
	\label{cor:RPUniformEpsilon}
	Let  
	 $0 < \epsilon_0 < \tfrac{\sqrt{10}}{4} $ 
	and fix some $\rr = (\rr_\alpha, \rr_\omega, \rr_c) \in \R^3_+$. 
	Fix $\rho > 0$ such that $ \rho \geq C(\epsilon_0, \epsilon_0^2 \rr)$, as given by Lemma~\ref{lem:Crho}. 
	Let $Y(\epsilon)$ and $Z(\epsilon,r,\rho)$ be the upper bounds as given by Lemma~\ref{lem:YZ}.  
Let the radii polynomials $P$ be defined by~\eqref{eq:RadPolyDef}. 

	If each component of  $P(\epsilon_0,\epsilon_0^2 \rr,\rho)$ is negative, 
	then for all $ 0 \leq \epsilon \leq \epsilon_0$ 
	there exists a unique $ \hat{x}_\epsilon \in B_\epsilon(\epsilon^2  \rr , \rho)$ 
	such that $ F(\hat{x}_\epsilon) =0$. 
	Furthermore, $\hat{x}_\epsilon$ depends smoothly on $\epsilon$.
\end{corollary}

\begin{proof}
	 Let $0 \leq  \epsilon < \epsilon_0$ be arbitrary.
	 Because $\rho \geq C(\epsilon_0,\epsilon_0^2 \rr) \geq C(\epsilon,\epsilon^2 \rr)$ by Lemma~\ref{lem:Crho},
	Theorem~\ref{thm:RadPoly} implies that it suffices to show that $ P(\epsilon,\epsilon^2 \rr ,\rho) <0$. 
	By using the monotonicity provided by Lemma~\ref{lem:YZ}, we obtain
	\begin{alignat*}{1}
		P(\epsilon,\epsilon^2 \rr ,\rho) &= Y(\epsilon) 
- \left[ I - Z(\epsilon,\epsilon^2 \rr,\rho)\right] \cdot \epsilon^2 \rr \\
		&=  (\epsilon / \epsilon_0)^{2} \left[ \epsilon_0^2   
		  \tilde{Y}(\epsilon) - \epsilon_0^2 \rr 
		+  Z(\epsilon,\epsilon^2 \rr,\rho) \cdot \epsilon_0^2 \rr  \right] \\
		&\leq  (\epsilon / \epsilon_0)^{2} \left[ \epsilon_0^2  
		  \tilde{Y}(\epsilon_0)  - \epsilon_0^2 \rr 
   +  Z(\epsilon_0,\epsilon_0^2 \rr,\rho) \cdot \epsilon_0^2 \rr  \right] \\
		&= (\epsilon / \epsilon_0)^{ 2} P(\epsilon_0 , \epsilon_0^2 \rr,\rho) \\
		& < 0,
	\end{alignat*}
where inequalities are interpreted componentwise in $\R^3$, as usual.
\end{proof}




%%%%%
%%%%%		THIS IS THE OLD VERSION OF THE UNIFORM \EPSILON^2 THEOREM
%%%%%
%%%%%\begin{corollary}
%%%%%	\label{prop:RPUniformEpsilon}
%%%%%	Let $ 0 < \epsilon_0 < \tfrac{5}{2} ( 4 + \sqrt{10})^{-1}$ and fix some $r = (r_\alpha, r_\omega, r_c) \in \R^3_+$ and 
%%%%%	fix $ k \in \{ 0,1,2\}$.  
%%%%%	Fix $\rho > 0$ such that $ \rho \geq C(\epsilon_0, (\epsilon_0)^2 r)$, as given by Lemma~\ref{lem:Crho}. 
%%%%%	Let $Y(\epsilon)$ and $Z(\epsilon,r,\rho)$ be the upper bounds as given by~\ref{lem:YZ}.  
%%%%%	Let the radii polynomials $P$ be defined by~\eqref{eq:RadPolyDef}.
%%%%%	If each component of  $P(\epsilon_0,{\epsilon_0}^k r,\rho)$ is negative, 
%%%%%	then for all $ 0 \leq \epsilon \leq \epsilon_0$ there exists a unique $ \hat{x}_\epsilon \in B_\epsilon(\epsilon^k  r , \rho)$ such that $ F(\hat{x}_\epsilon) =0$. Furthermore, $\hat{x}_\epsilon$ depends smoothly on $\epsilon$.
%%%%%\end{corollary}
%%%%%
%%%%%\begin{proof}
%%%%%	Let $0 \leq  \epsilon < \epsilon_0$ be arbitrary.
%%%%%	Because $\rho \geq C(\epsilon_0,\epsilon_0^k r) \geq C(\epsilon_0,\epsilon_0^k r)$ by Lemma~\ref{lem:Crho},
%%%%%	Theorem~\ref{thm:RadPoly} implies that it suffices to show that $ P(\epsilon,\epsilon^k r ,\rho) <0$. 
%%%%%	By using the monotonicity provided by Lemma~\ref{lem:YZ}, we obtain
%%%%%	\begin{alignat*}{1}
%%%%%	P(\epsilon,\epsilon^k r ,\rho) &= Y(\epsilon) - \left[ I - Z(\epsilon,\epsilon^k r,\rho)\right] \cdot \epsilon^k r \\
%%%%%	&=  (\epsilon / \epsilon_0)^{k} \left[ \epsilon_0^k  \epsilon^{2-k}  \tilde{Y}(\epsilon) - \epsilon_0^k r +  Z(\epsilon,\epsilon^k r,\rho) \cdot \epsilon_0^k r  \right] \\
%%%%%	&\leq  (\epsilon / \epsilon_0)^{k} \left[ \epsilon_0^k  \epsilon_0^{2-k}  \tilde{Y}(\epsilon_0)  - \epsilon_0^k r +  Z(\epsilon_0,\epsilon_0^k r,\rho) \cdot \epsilon_0^k r  \right] \\
%%%%%	&= (\epsilon / \epsilon_0)^{ k} P(\epsilon_0 , \epsilon_0^k r,\rho) \\
%%%%%	& < 0,
%%%%%	\end{alignat*}
%%%%%	where inequalities are interpreted componentwise in $\R^3$, as usual.
%%%%%\end{proof}
%%%%%




%%%%%%%%%%%%%%%%%%%%%%%%%%%%%%%%%%%%%%%%%%%%%%%%%%%%%%%%%%%%%%%%%%%%%%%%%%%%
%\subsection{Application of Radii Polynomials}

%\begin{remark}


These $\epsilon$-rescaled variables are used in
Proposition~\ref{prop:TightEstimate} below to derive \emph{tight} bounds on the
solution (in particular, tight enough to conclude that the bifurcation is
supercritical). The following remark explains that the monotonicity properties of
the bounds $Y$ and $Z$ imply that looking for small(est) radii which satisfy $P(r)<0$, is
a well-defined problem.


\begin{remark}\label{r:smallradii}
The set $R$ of radii for which the radii polynomials are negative is given by 
\[
  R := \{ r \in \R^3_+ : r_j > 0,  P_i(r) < 0 \text{ for } i,j=1,2,3 \} .
\] 
This set has the property that if
	$r,r' \in R$, then $r''\in R$, where $r''_j=\min\{ r_j,r'_j\}$.
Namely, the main observation is that we can write 
	$P_i(r)= \tilde{P}_i(r)-r_i$, where $\partial_{r_j} \tilde{P}_i \geq 0$ for all $i,j=1,2,3$.
Now fix any $i$; we want to show that $P_i(r'') < 0$.	
We have either $r''_i=r_i$ or $r''_i=r'_i$, hence assume $r''_i=r_i$ (otherwise
just exchange the roles of $r$ and $r'$). We infer that $P_i(r'') \leq P_i(r) <
0$, since $\partial_{r_j} P_i \geq 0$ for $j \neq i$.
We conclude that there are no trade-offs in looking for minimal/tight radii, as
opposed to looking for large radii, see Remark~\ref{r:largeradii}.
\end{remark}

%%%
%%%The optimization problem is simplified to a degree because the region $ P(\epsilon_0,r,\rho_0) <0$ is convex for fixed $\epsilon_0$ and $ \rho_0 $.  
%%%This is because the function $Z(\epsilon,r,\rho)$ is constructed out of polynomials with non-negative coefficients, whereby $\tfrac{\partial}{ \partial r_i} \tfrac{\partial }{\partial r_j} P_{r_k}(\epsilon_0,r,\rho_0) >0$ for all $ i,j,k \in \{ \alpha, \omega,c\}$. 
%%%\marginpar{I believe this is true, right? -JJ}


\begin{proposition}
		\label{prop:TightEstimate}
	Fix $ \epsilon_0 = 0.10$ and 
%\change[J]{$ (\rr_\alpha , \rr_ \omega , \rr_c) = (  0.1149, \, 0.0470 , \, 0.4711 ) $}
$ (\rr_\alpha , \rr_ \omega , \rr_c) = (  0.0594, \, 0.0260 , \, 0.4929 ) $ 
and 
%\change[J]{$\rho = 0.0279$}
$\rho = 0.3191$. 
	For all $0< \epsilon \leq \epsilon_0$ there exists a unique point $\hat{x}_\epsilon = (\hat{\alpha}_\epsilon,\hat{\omega}_\epsilon,\hat{c}_\epsilon)$ 
	satisfying $F(\hat{x}_\epsilon) = 0$ and 
	\begin{align}
	\label{eq:TightBound}
 | \hat{\alpha}_\epsilon - \balpha_\epsilon| <& \rr_\alpha \epsilon^2 , 
 %
 &|\hat{\omega}_\epsilon - \bomega_\epsilon| <&  \rr_\omega \epsilon^2 ,
 %
 &
 \| \hat{c}_\epsilon - \bc_\epsilon\| <& \rr_c  \epsilon^2   ,
  %
  &
  \| K^{-1} \hat{c}_\epsilon \| <& \rho  .
	\end{align}
Furthermore, $\hat{\alpha}_\epsilon > \pp$ for $ 0 < \epsilon < \epsilon_0$.
\end{proposition}

\begin{proof}
	In the Mathematica file~\cite{mathematicafile}  we check, using interval arithmetic, that  $\rho \geq C(\epsilon_0, \epsilon_0^2 \rr)$ and  the radii polynomials $P(\epsilon_0,\epsilon_0^2 \rr,\rho)$ are negative.  
	%I DO NOT UNDERSTAND THE NEXT SENTENCE
 The inequalities in Equation~\eqref{eq:TightBound} follow from Corollary~\ref{cor:RPUniformEpsilon}. 
 Since $\hat{\alpha}_\epsilon \geq \balpha_\epsilon - \rr_\alpha \epsilon^2
 = \pp +\frac{1}{5}(\frac{3\pi}{2}-1)\epsilon^2 - \rr_\alpha \epsilon^2$ and $ \rr_\alpha < \tfrac{1}{5} ( \tfrac{3 \pi}{2} -1) $, it follows that $ \hat{\alpha}_\epsilon > \pp $ for all $ 0 < \epsilon \leq \epsilon_0$. 
%%
%%STILL NEEDS AN EXPLANATION
%%\marginpar{Jonathan: I am not sure what the argument is \dots}
%% WHY IT IS UNIQUE IN THE BALL GIVEN BY~\eqref{eq:TightBound}. CLEARLY IT IS UNIQUE IN $B_\epsilon(r,\rho)$
%%WITH $\rho= C( \epsilon_0,\epsilon_0^2 r)$. WHY CAN THERE BE NO SOLUTIONS WITH
%%$\| K^{-1} c \| > \rho$ SATISFYING~\eqref{eq:TightBound} ? 
\end{proof}

\begin{remark}\label{r:nested}
% The pivotal result in Proposition~\ref{prop:TightEstimate} is that $\hat{\alpha}_\epsilon > \pp$, which implies that the bifurcation is subcritical.
Since $\epsilon_0^2\rr < r$ for the choices (a) and (b) in Proposition~\ref{prop:bigboxes},
and the choices of $\rho$ and $\epsilon_0$ are compatible as well, the solutions found in Proposition~\ref{prop:bigboxes} are the same as those described by Proposition~\ref{prop:TightEstimate}. While the former proposition provides large isolation/uniqueness neighborhoods for the solutions,
the latter provides tight bounds and confirms the  supercriticality of the bifurcation suggested in Definition \ref{def:xepsilon}.

% The bifurcation is supercritical (see eg.  \cite{faria2006normal} p 252
	
	
	
%		We note that for each (appropriate) $\epsilon$, the ball 
%	$ B_{\epsilon}(r,\rho)$ from Proposition \ref{prop:TightEstimate} is contained within the balls   
%	$ B_{\epsilon}(r_a,\rho_a)$  and 
%	$ B_{\epsilon}(r_b,\rho_b)$ from Proposition \ref{prop:bigboxes}. 
%	This means that the fixed points $  \hat{x}_{\epsilon} \in B_{\epsilon}(r,\rho)$ is the same fixed point $\hat{x}_{\epsilon} \in B_{\epsilon}(r_a,\rho_a)$ .

\end{remark}


%
%The method of radii polynomials is versatile. 
%With the goal of later proving Corollary \ref{prop:UniqunessNbd}, we added additional constraints to \emph{Mathematica}'s function \emph{NMaximize} to find the parameters for Proposition \ref{prop:WideEstimate}.
%
%When searching for the largest viable radius we add an additional constraint. 
%In Proposition \ref{prop:Cone}, we showed that for a given selection of $ \epsilon$, $r_\alpha$ and $ r_\omega$, then the unscaled variable $ \|c\|$ is either $\cO(1)$ or $\cO(\epsilon^2)$. 
%When we scale $c \to \epsilon c$, then we are only able to prove uniqueness of our solution in an $ \epsilon-$cone about the approximate solution. 
%We use this Proposition to select $r_c = ????$ in terms of $\epsilon$, $r_\alpha$ and $r_\omega$ so that any unscaled  solution $c$ is either $\cO(1)$ or $ c \in B_{\epsilon}(r,\rho)$.
%
%Even still, the larger we choose $ \epsilon$, the smaller we will need to take $ r_\omega$ in order to have a proof. 
%For the following theorem, we fixed $ \epsilon_0 =0.085$ and used \emph{NMaximize} to find a choice of variables $(\epsilon,r_\alpha,r_\omega,r_c)$  which maximized the objective function $r_w$ and for which all the radii polynomials were negative. 
%By slightly shrinking the estimate for the optimal radii, we obtain the following theorem.




\section{Global results}
\label{s:global}
%!TEX root = hopfwright.tex

When deriving global results from the local results in
Section~\ref{s:local}, we need to take into account that there are some obvious
reasons why the branch of periodic solutions, described by
$F_\epsilon(\alpha,\omega,c)=0$, bifurcating from the Hopf bifurcation point at
$(\alpha,\omega)=(\pp,\pp)$ does not describe the entire set of periodic
solutions for $\alpha$ near $\pp$. First, there is the trivial solution. In
particular, one needs to quantify in what sense the trivial solution is an
isolated invariant set. This is taken care of by Remarks~\ref{r:smalleps}
and~\ref{r:cone}, which show there are no ``spurious'' small solutions in the
parameter regime of interest to us (roughly as long as we stay away from the
next Hopf bifurcation at $\alpha = \tfrac{5\pi}{2}$). Second, one can interpret any periodic
solution with frequency $\omega$ as a periodic solution with frequency
$\omega/N$ as well, for any $N \in \mathbb{N}$. Since we are working in Fourier space,
showing that there are no ``spurious'' solutions with lower frequency would
require us to perform an analysis near $(\alpha,\omega)=(\pp,\tfrac{\pi}{2N})$
for all $N \geq 2$. This obstacle can be avoided by bounding (from below)
$\omega$ away from $\pi/4$. This is done in Lemma~\ref{lem:omegalarge}.

For later use, we recall an elementary Fourier analysis bound. 
%\marginpar{There must be some theory about the best constant here?}
\begin{lemma}\label{lem:fourierbound}
	Let $y \in C^1$ be a periodic function of period $2\pi/\omega$ with Fourier coefficients $\c \in \ell^1_\sym$ (in particular this means $\c_0=0$), as described by~\eqref{eq:FourierEquation}. 
	Then 
\[
 \| \c \| \leq \sqrt{\frac{\pi }{6 \omega}}\,  \| y' \|_{L^2([0,2\pi/\omega])}
\qquad\text{and}\qquad
 \| \c \| \leq \frac{\pi}{\omega\sqrt{3}}\, \|y'\|_\infty.
 \]
\end{lemma}
\begin{proof}
From the Cauchy-Schwarz inequality and  Parseval's identity it follows that
\begin{alignat*}{1}
		\| \c \| &= 2 \sum_{k=1}^{\infty} |\c_k|
	%	 &=& 
	%	2 \sum_{k=1}^{\infty} | c_k| \\ 
	%	&=&	2 \sum_{k=1}^{\infty} k^{-1} \cdot | k \, c_k| \\
		\leq 2 \left( \sum_{k=1}^{\infty} k^{-2} \right)^{1/2}
		\left( \sum_{k=1}^{\infty} |k \, \c_k|^2 \right)^{1/2} \\
     &=  \frac{\sqrt{2}}{\omega} \left(\frac{\pi^2}{6} \right)^{1/2} 
	 	 \left(2 \sum_{k=1}^{\infty} |i \omega k \, \c_k|^2 \right)^{1/2}
		 = \frac{\pi}{\omega \sqrt{3}} 
		\left(\sum_{k \in \Z} |i \omega k \, \c_k|^2 \right)^{1/2}\\
		&= \frac{\pi}{\omega \sqrt{3}} 
		\left( \frac{\omega}{2\pi} \int_0^{2\pi/\omega} | y'(t)|^2 dt  \right)^{1/2} 
		\leq \frac{\pi}{\omega \sqrt{3}} \,  \|y'\|_\infty.
\end{alignat*}	
\end{proof}

% THIS NEEDS TO BE MENTIONED SOMEWHERE ELSE (IN EARLIER SECTION)
% Beside these more or less obvious obstacles, there is a more technical hurdle.
% The local analysis in Section~\ref{} gives us a unique solution branch $c=c(\epsilon)$ in some box $\| c - c(\epsilon) \| \leq r_c$. This translates to a unique solution curve $\tc=\tc(\epsilon)$ in some \emph{cone} $\|\tc\| \leq \epsilon r_c$. Lemma~\ref{} shows that if there are solutions outside this cone then $\| \tc \|$ must be large. In particular there are no solutions with small $\tc$ outside the cone (except for the trivial solution), provided $r_c > something$

\subsection{A proof of Wright's conjecture} 


Based on the work in \cite{neumaier2014global} and \cite{wright1955non}, in order to prove Wright's conjecture it suffices to prove that there are no slowly oscillating periodic solutions (SOPS) to Wright's equation for $ \alpha \in [1.5706,\pp]$. Moreover, in \cite{neumaier2014global} it was shown that no SOPS with $\| y \|_\infty \geq e^{0.04}-1$ exists for  $\alpha \in [1.5706,\pp]$. These results are summarized in the following proposition.

\begin{proposition}[\cite{neumaier2014global,wright1955non}]
\label{prop:neumaier}
Assume $y$ is a SOPS to Wright's equation for some $\alpha \leq \pp$. Then $\alpha \in [1.5706,\pp]$
and $\| y \|_\infty \leq e^{0.04}-1$. 
\end{proposition}
For convenience we introduce
\[
  \mu := e^{0.04}-1 \approx 0.0408.
\]
We now derive a lower bound on the frequency $\omega$ of the SOPS.
\begin{lemma}\label{lem:omegalarge}
Let $\alpha \in [1.5706,\pp]$.
Assume $y$ is a SOPS to Wright's equation with minimal period $2\pi/\omega$,
and assume that $\| y \|_\infty \leq \mu$.
Then $\omega \in [1.11,1.93]$.
\end{lemma}

\begin{proof}
%	\marginpar{JJ: I've written a new proof.}
	Without loss of generality, we assume in this proof that $ y(0) =0$, that $y(t) < 0$ for $t\in (-t_{-},0)$ and that $y(t) > 0$ for $t\in (0,t_+)$. 
	We will show that $t_-$ and $t_+$ are bounded by
	\begin{alignat*}{1}
	1+ \frac{1}{\alpha }  \frac{\log (1 + \mu)}{\mu}  < t_+ &<2 + \frac{1}{\alpha} , \\
	1+\frac{1}{\alpha} <t_- & < 3 .
	\end{alignat*}
	The lower bounds for both $t_-$ and $t_+$ follow directly from  Theorem 3.5 in \cite{jones1962nonlinear}. While Theorem 3.5 in~\cite{jones1962nonlinear} assumes $ \alpha \geq \pp$, this part of the theorem simply relies on Lemma 2.1 in \cite{jones1962nonlinear}, which only requires $ \alpha > e^{-1}$.
	

To obtain an upper bound on $t_+$, assume that $ t_+ \geq 2$. Set $t'_+ =\min\{t_+,3\}$. Then it follows from~\eqref{eq:Wright} that $y'(t) < 0$ for $t\in (1,t'_+]$, hence  $y(t-1) > y(2)$ for $ t \in [2,t'_+]$. We infer that for $t \in [2,t'+]$ we have 
$y'(t) = - \alpha y(t-1) [1+y(t)] < - \alpha y(2)$.
Solving the IVP $y'(t) < -\alpha y(2)$ with the initial condition $y(2) = y(2)$, we see that $y(t)$ hits $0$ before $t=2+\frac{1}{\alpha}$. Since $\alpha > 1$ (hence $2+\frac{1}{\alpha} < 3$), this implies that $t'_+=t_+$ and $t_+ < 2+\frac{1}{\alpha}$.
 
	

	To obtain the upper bound on $t_-$, assume for the sake of contradiction
	 that $ t_- \geq 3$. 
	 Then it follows from~\eqref{eq:Wright} that $y'(t) \geq 0$ for $t\in [-2,0]$, hence  $y(t) \leq y(-1)$ for $ t \in [-2,-1]$, and  $y'(t) \geq - \alpha y(-1) [1+y(t)]$ for $ t \in [-1,0]$.  
	Solving this IVP with the initial condition $y(-1) = -\nu$, we obtain $ y(t) \geq (1-\nu) e^{ \alpha \nu (t+1)}-1 $ for $ t \in [-1,0]$, and in particular  $y(0) \geq (1-\nu ) e^{-\alpha \nu}-1$. 
	By assumption $y(0)=0$ and $\nu=|y(-1)| \leq \mu$,
	but $  (1-\nu ) e^{-\alpha \nu}-1>0 $ 
	for $ \nu  \in (0,\mu]$
	and $\alpha \in [1.5706,\pp]$, a contradiction. Thereby $ t_- <3$. 


The bound on $\alpha$ implies that 
 the minimal period $L = t_+ + t_-$ of the SOPS must lie in $[ 3.26,5.64]$.
It then follows that $ \omega \in [1.11,1.93]$	
\end{proof}

%	Let $ t_+ $ denote the amount of time the SOPS $y(t)$ is positive and $ t_-$ denote the amount of time a SOPS is negative (measured over one minimal period). 
%
%First we show that $t_- <3$. 
%Theorem 3.4 in \cite{jones1962nonlinear} shows that if $t_- \geq 3$ then 
%$ \min y \geq  e^{-(\alpha-1)}-1$ which, for our range of $\alpha$, is within  $[0.43,0.44]$. 
%Since $\|y\|_\infty \leq \mu$, this contradiction shows that $ t_- < 3$. 


It turns out that this bound on $\omega$ can (and needs to be) sharpened.
This is the purpose of the following lemma, 
which considers solutions in  unscaled variables. 
\begin{lemma}\label{lem:ZeroNBD}
Suppose $ \tilde{F}_\epsilon(\alpha,\omega,\tc)=0$. If $\omega \in
[1.1,2]$ and $ \alpha \in [1.5,2.0]$  
%(QUITE ARBITRARY BOUNDS!)\marginpar{some illumination needed} 
then
\begin{equation}\label{e:tighterboundonomega}
   \frac{\sqrt{(\omega- \alpha)^2 + 2 \alpha \omega(1-\sin\omega)}}{2\alpha} 
   \leq 2 \epsilon + \| \tc \| .
\end{equation}
\end{lemma}
\begin{proof}
This follows from Proposition~\ref{prop:zeroneighborhood2} in
Appendix~\ref{appendix:aprioribounds}, combined with Proposition \ref{prop:G1Minimizer},  which shows that for
$\omega \in [1.1,2.0]$  and $ \alpha \in [1.5,2.0]$, the minimum in Equation~\eqref{e:minoverk} is attained for $k=1$.
%%\marginpar{Jonathan: need to say something about why this is true.}
\end{proof}

Next we derive bounds on $\epsilon$ and $\tc$, which also lead to improved bounds on $\omega$.
\begin{lemma}\label{lem:wrightbounds}
Let $\alpha \in [1.5706,\pp]$. Assume $y$ is a SOPS with $\| y \|_\infty \leq \mu$.
Then $y$ corresponds, through the Fourier representation~\eqref{e:yc}, to a zero of $F_\epsilon(\alpha,\omega,c)$ with $|\omega- \pp| \leq 0.1489$ and
\[
  0< \epsilon \leq \epseps := \mu/\sqrt{2} \leq 0.02886 ,
\] 
and 
%\change[J]{$\| c \| \leq 0.0398$}
$\| c \| \leq 0.0796$ 
and 
%\change[J]{$\| K^{-1} c \| \leq 0.08 $.}
$\| K^{-1} c \| \leq  0.16 $.
\end{lemma}
\begin{proof}
First consider the Fourier representation~\eqref{e:ytc} of $y$ in unscaled variables. 
Recall that $\c_0$ vanishes (see Remark~\ref{r:a0}).
Since $|y'(t)| \leq \alpha |y(t-1)| (1+|y(t)|) \leq \alpha \mu (1+\mu)$
we see from  Lemma~\ref{lem:fourierbound} that 
\begin{equation}\label{e:epsilontc} 
  2 \epsilon + \| \tc \|  \leq \frac{\pi}{\omega\sqrt{3}} \alpha \mu (1+\mu).
\end{equation}
Combining this with Lemma~\ref{lem:ZeroNBD} leads to the inequality
\begin{equation}
\label{eq:APomegaBound}
	\omega 	\sqrt{(\omega- \alpha)^2 + 2 \alpha \omega ( 1- \sin \omega)} 
	\leq 
	\tfrac{2 \pi}{ \sqrt{3}}  \alpha^2 \mu ( 1 + \mu).
\end{equation}
In the Mathematica file \cite{mathematicafile} we show that when $ \alpha \in [ 1.5706,\pp]$, then inequality \eqref{eq:APomegaBound} is violated for any $\omega \in [ 1.1,2.0] \, \backslash \, [1.4219, 1.6887]$.
From Lemma~\ref{lem:omegalarge} we obtain the a priori  bound
$\omega \in [1.11,1.93]$, whereby it follows that  $\omega \in [1.4219,1.6887]$, and in particular  $|\omega - \pp| \leq 0.1489$. 


Using this sharper bound on $\omega$ as well as $\alpha \in [1.5706,\pp]$
we conclude from~\eqref{e:epsilontc} that
\begin{equation}
	2 \epsilon + \| \tc \|  \leq \frac{\pi}{\omega\sqrt{3}} \alpha \mu (1+\mu)
	\leq \frac{2\omega - \alpha}{\alpha}.
	%  \ref{-- This equation is refered to in the Mathematica File --}
\end{equation} 
Since we also infer that $\alpha < 2\omega$,  Theorem~\ref{thm:FourierEquivalence3}(b) shows that the solution corresponds to a zero of $F_\epsilon(\alpha,\omega,c)$, with $\tc = \epsilon c $.
We can improve the bound on $\epsilon$ from~\eqref{e:epsilontc}
% \eqref{e:epsilontc} 
by observing that
\[
	( \epsilon^2 + \epsilon^2)^{1/2} 
	\leq \left( \sum_{k\in\Z} |\c_k|^2 \right)^{1/2}
	 =  \left( \frac{\omega}{2\pi} \int_0^{2\pi/\omega}
	                         |y|^2 dt  \right)^{1/2} \leq \mu .
\]
Hence $\epsilon \leq \epseps := \mu/\sqrt{2}$.

Finally, we derive the bounds on $c$. Namely, for $\alpha \in  [1.5706,\pp]$,
$\omega \in [1.4219,1.6887]$ and $\epsilon \leq \epseps$,
we find that $b_*$ and  $z_*^+$, as defined in~\eqref{e:zstar}, are bounded below by $ b_* \geq 0.364$ and 
$z^+_* \geq 0.72 $. 
Since it follows from~\eqref{e:epsilontc} that 
$ \| \tc \|  \leq 0.09 $  
in the same parameter range  of  $\alpha$ and $\omega$, we infer from Lemma~\ref{lem:Cone}(a) that 
$\| \tc \| \leq z^-_* $.
Via an interval arithmetic computation, the latter can be bounded above
using Lemma~\ref{lem:ZminusBound}, for $\alpha \in  [1.5706,\pp]$,
$\omega \in [1.4219,1.6887]$ and $\epsilon \leq \epseps$, by
%\change[J]{$z_*^- \leq 0.0398 \epsilon$. }
$z_*^- \leq 0.0796 \epsilon$. 
Hence 
%\change[J]{$\| c \| \leq z_*^- / \epsilon \leq 0.0398$.}
$\| c \| \leq z_*^- / \epsilon \leq 0.0796$.
Furthermore, Lemma~\ref{lem:Cone}(b) implies 
the  bound 
%\change[J]{$\| K^{-1} c \| \leq (\epsilon^2 + (z_*^-)^2 )/(\epsilon b_*) \leq 2.76 \epsilon$}
%\change[J]{$\| K^{-1} c \| \leq (\epsilon^2 + (z_*^-)^2 )/(\epsilon b_*) \leq 2.77 \epsilon$.}
$\| K^{-1} c \| \leq (2\epsilon^2 + (z_*^-)^2 )/(\epsilon b_*) \leq 5.52 \epsilon$.
Since $\epsilon \leq \epsilon_*$, it then follows that 
%\change[J]{$\| K^{-1} c \| \leq 0.08 $.}
$\| K^{-1} c \| \leq  0.16 $.
\end{proof}

With these tight bounds on the solutions, we are in a position to apply the local bifurcation result formulated in Proposition~\ref{prop:TightEstimate} to prove the ultimate step of Wright's conjecture.

\begin{theorem}
	\label{thm:WrightConjecture}
	For $ \alpha \in [0,\pp]$ there is no SOPS to Wright's equation.
\end{theorem}
\begin{proof}
By Proposition~\ref{prop:neumaier} (see also the introduction of this paper) it suffices to exclude a slowly oscillating solution $y$ for $\alpha \in [1.5706,\pp]$ with $\| y \|_\infty \leq \mu$.
By Lemma~\ref{lem:wrightbounds}, if such a solution would exist, it corresponds to a solution of 
$F_\epsilon(\alpha,\omega,c)=0$ with $|\omega- \pp| \leq 0.1489$, 
$0< \epsilon \leq \epseps = \mu / \sqrt{2}$, 
%\change[J]{$\| c \| \leq 0.0398 $}
$\| c \| \leq 0.0796 $ 
and 
%\change[J]{$\| K^{-1} c \| \leq 0.08 $.}
$\| K^{-1} c \| \leq  0.16 $.
We claim that no such solution exists.
Indeed, we define the set 
% \[
%   S :=  \{ (\alpha,\omega,c) \in X :
%   |\alpha - \pp| \leq 0.0002; \,
%    |\omega - \pp| \leq 0.1489; \,
%    \| c \| \leq 0.04; \,
%    \|K^{-1} c \| \leq 0.08  \}.
% \]
% \note[J]{Version with new $|c|$ below. Also propose simplifying $r_\omega$. }
% \[
% S :=  \{ (\alpha,\omega,c) \in X :
% |\alpha - \pp| \leq 0.0002; \,
% |\omega - \pp| \leq 0.15; \,
% \| c \| \leq 0.08; \,
% \|K^{-1} c \| \leq 0.08  \}.
% \]
% \note[J]{Proposed change}
\[
S :=  \{ (\alpha,\omega,c) \in X : 
|\alpha - \pp| \leq 0.0002; \,
|\omega - \pp| \leq 0.15; \, 
\| c \| \leq 0.08; \, 
\|K^{-1} c \| \leq 0.16  \}.
\]
To show that there is no SOPS for $\alpha \in [1.5706,\pp]$, it now suffices to show that all zeros of $F_\epsilon(\alpha,\omega,c)$ in $S$ for any $0< \epsilon \leq \epseps$ satisfy $\alpha> \pp$.

Let us consider
$B_\epsilon(r,\rho)$, which is centered at $\bx_\epsilon$ (see Definition~\ref{def:xepsilon})
with $r$ and $ \rho$ taken as in Proposition~\ref{prop:bigboxes}(a).
In the Mathematica file~\cite{mathematicafile} 
we check that the following inequalities are satisfied:
%\note[JB]{Are these checked in the Mathematica file? Perhaps refer to that?}
% \begin{alignat*}{1}
% 	r_\alpha &= 0.21 \geq  0.0002 + |\balpha_{\epseps}-\pp|,\\
%  r_\omega &= 0.16 \geq  0.1489 + |\bomega_{\epseps}-\pp| ,\\
%  r_c &= 0.09 \geq   0.04 + \| \bc_{\epseps}\|,\\
%  \rho &= 1.01 \geq 0.08 .
% \end{alignat*}
% \note[J]{Version with new numbers below.}
% \begin{alignat*}{1}
% r_\alpha &= 0.13 \geq  0.0002 + |\balpha_{\epseps}-\pp|,\\
% r_\omega &= 0.17 \geq  0.15 + |\bomega_{\epseps}-\pp| ,\\
% r_c &= 0.17 \geq   0.08 + \| \bc_{\epseps}\|,\\
% \rho &= 1.78 \geq 0.08 .
% \end{alignat*}
% \note[J]{Proposed change}
\begin{alignat*}{1}
r_\alpha &= 0.13 \geq  0.0002 + |\balpha_{\epseps}-\pp|,\\
r_\omega &= 0.17 \geq  0.15 + |\bomega_{\epseps}-\pp| ,\\
r_c &= 0.17 \geq   0.08 + \| \bc_{\epseps}\|,\\
\rho &= 1.78 \geq 0.16 . 
\end{alignat*}
By the triangle inequality we obtain that $S \subset B_\epsilon(r,\rho)$ for all $0<\epsilon\leq \epseps$.
Proposition~\ref{prop:bigboxes}(a) shows that for each $0<\epsilon\leq \epseps$
there is a unique zero $\hat{x}_\epsilon=
(\hat{\alpha}_\epsilon,\hat{\omega}_\epsilon,\hat{c}_\epsilon) \in B_\epsilon (r,\rho)$ of $F_\epsilon$.
By Proposition~\ref{prop:TightEstimate} and Remark~\ref{r:nested}
this zero satisfies $\hat{\alpha}_\epsilon > \pp$.
Hence, for any $0<\epsilon\leq\epseps$ the only zero of $F_\epsilon$ in $S$ (if there is one) satisfies $\alpha>\pp$. This completes the proof.

\end{proof}
% 
%\begin{eqnarray}
%	\|c \|_{\ell^1} &=& 2 \sum_{k=1}^{\infty} | c_k| \\ 
%	&=&	2 \sum_{k=1}^{\infty} k^{-1} \cdot | k \, c_k| \\
%	&\leq& 2 \left( \sum_{k=1}^{\infty} k^{-2} \right)^{1/2}
%	\left( \sum_{k=1}^{\infty} |k \, c_k|^2 \right)^{1/2} \\
%	&=&  \left(2 \frac{\pi^2}{6} \right)^{1/2}
%		\left(2 \omega^{-2} \sum_{k=1}^{\infty} |i \omega k \, c_k|^2 \right)^{1/2}\\
%	&=& \frac{\pi}{\omega \sqrt{3}} 
%	\left(\sum_{k=- \infty}^{\infty} |i \omega k \, c_k|^2 \right)^{1/2}\\
%	&=& \frac{\pi}{\omega \sqrt{3}} 
%		\left( \frac{1}{L} \int_0^L | y'(t)|^2 dt  \right)^{1/2}\\
%	&\leq& \frac{\pi}{\omega \sqrt{3}}  \max | y'(t)| 
%\end{eqnarray}
%Line 4.3 is obtained through Cauchy-Schwartz inequality. 
%Line 4.6 is obtained through Parseval's identity.
%

% END OF SUBSECTION
%
%
% With this bound on $\omega$ we can now use the local results from Section~\ref{s:local} to establish the ultimate step in the proof of Wright's conjecture.
%
% \begin{theorem}
% 	\label{prop:WrightConjecture}
% 	For $ \alpha \in [1.5706,\pp]$ there is no SOPS to Wright's equation.
% \end{theorem}
%
% \begin{proof}
% 	Suppose that $y$ is a slowly oscillating periodic solution to Wright's equation with frequency $ \omega >0$.
% 	By Theorem~\ref{thm:FourierEquivalence2}, there is some $\epsilon \geq 0$ and $ c \in \ell^1_0$ for which $y$ (after a time shift) can be written in the following form:
% 	\[
% 	y(t) =
% 	\epsilon \left( e^{i \omega t }  + e^{- i \omega t }\right)
% 	+  \sum_{k = 2}^\infty    \tc_k e^{i \omega k t }  + \bar{\tc}_k e^{- i \omega k t }
% 	\]
% 	By Theorem \ref{thm:FourierEquivalence2}, it suffices to show that $ \tilde{F}_\epsilon(\alpha , \omega, \tc)=0$ has no nontrivial solutions when  $ \alpha \in [1.5706,\pp]$.
% 	In Proposition \ref{prop:TightEstimate} it was proved that in a small $\epsilon$-scaled neighborhood about the bifurcation point, periodic solutions to Wright's equation only exist when $ \alpha > \pp$.
% 	To prove the theorem, we extend this neighborhood so that it  contains any solutions to $\tilde{F}(\alpha,\omega,c)=0$ that could exist for $ \alpha \in  [1.5706, \pp]$.
%
%
%
% 	We first collect global bounds on $ \epsilon$, $c$ and $ \omega $ for which a solution $ \tilde{F}_\epsilon (\alpha, \omega, c)=0$ could exist.
% 	By the results in \cite{neumaier2014global} we know that $\max |y(t)| \leq e^{0.04} -1$. Let us fix $ m := e^{0.04} -1$ and define  $ \hat{c} := \{ \epsilon , c_2 , c_3, \dots \} \in \ell^1 $.  By Parseval's  identity, we obtain the following inequality:
% 	\[
% 		( \epsilon^2 + \epsilon^2)^{1/2}
% 		\leq
% 		\| \hat{c} \|_{\ell^2}
% 		=
% 	\left( \frac{1}{L} \int_0^L |y|^2 dt  \right)^{1/2} \leq m
% 	\]
% 	Thereby we may define $ \epsilon_0 := m/\sqrt{2} \approx 0.0288$ and concern ourselves only with solutions $ \tilde{F}_\epsilon =0$ for which $ 0 \leq \epsilon \leq \epsilon_0$.
% 	Similarly, we can use Parseval's identity to bound $ \| \hat{c} \|_{\ell^1}$ as below:
% 	\begin{eqnarray}
% 	\| \hat{c} \|_{\ell^1}
% %	 &=&
% %	2 \sum_{k=1}^{\infty} | c_k| \\
% %	&=&	2 \sum_{k=1}^{\infty} k^{-1} \cdot | k \, c_k| \\
% 	&\leq& 2 \left( \sum_{k=1}^{\infty} k^{-2} \right)^{1/2}
% 	\left( \sum_{k=1}^{\infty} |k \, c_k|^2 \right)^{1/2} \\
% %	&=&  \left(2 \frac{\pi^2}{6} \right)^{1/2}
% %	\left(2 \omega^{-2} \sum_{k=1}^{\infty} |i \omega k \, c_k|^2 \right)^{1/2}\\
% 	&=& \frac{\pi}{\omega \sqrt{3}}
% 	\left(\sum_{k=- \infty}^{\infty} |i \omega k \, c_k|^2 \right)^{1/2}\\
% 	&=& \frac{\pi}{\omega \sqrt{3}}
% 	\left( \frac{1}{L} \int_0^L | y'(t)|^2 dt  \right)^{1/2} 		\label{eq:WrightConjectureEll1BoundL2}\\
% 	&\leq& \frac{\pi}{\omega \sqrt{3}}  \max | y'(t)| \\
% 	&\leq& \frac{\pi}{\omega \sqrt{3}}  \alpha m ( 1 + m)
% 	\label{eq:WrightConjectureEll1Bound}
% 	\end{eqnarray}
%
%
% 	In Proposition \ref{prop:ZeroNBD} we show if  $ \tilde{F}_\epsilon (\alpha, \omega, c)=0$  for $ \alpha \in [1.5706,\pp]$ then $\omega \in [1.11,1.93]$ and furthermore:
% 	\[
% 	\sqrt{(\omega- \alpha)^2 + 2 \alpha \omega ( 1- \sin \omega)} / (2 \alpha) \leq  \| \hat{c}\|_{\ell^1}
% 	\]
% 	Combining this with the estimate in Line \ref{eq:WrightConjectureEll1Bound}, we obtain the following inequality:
% 	\[
% 	\omega 	\sqrt{(\omega- \alpha)^2 + 2 \alpha \omega ( 1- \sin \omega)}
% 	\leq
% 	 \tfrac{2 \pi}{ \sqrt{3}}  \alpha^2 m ( 1 + m)
% 	\]
% 	Using interval arithmetic, we confirm in the supplemental \emph{Mathematica} file that the only value of $ \omega$ satisfying the above inequality are  $\omega \in [1.4219, 1.6887]$.
% 	% (also note $|\omega - \pp| < 0.1489 $).
%
%
% 	Next, we derive bounds on $c$ which scale with $ \epsilon$.
% 	From Proposition \ref{prop:Cone}, it follows that if $(\alpha,\omega,c)$ solves $ \tilde{F}_\epsilon=0$ in the range  $0 \leq \epsilon \leq \epsilon_0$ and  $\alpha \in [ 1.5706,\pp]$ and $ |\omega - \pp | < 0.1489$, then either $ \| c \| \geq 0.72$ or $ \| c \| \leq   0.0398 \epsilon$.
% 	The case $\| c\| > 0.72$ contradicts our estimate in Line \ref{eq:WrightConjectureEll1Bound}, therefore, it must be the case that $  \| c \| \leq 0.0398 \epsilon $.
% 	It then follows that if $ \epsilon =0$ then  the only solution to $ \tilde{F}_\epsilon(\alpha,\omega,c)=0$ is the trivial solution.
% 	An additional result from Proposition \ref{prop:Cone} is that any solution must satisfy $\| K^{-1} c \|_{\ell^1/\C} < 0.715 \epsilon^2$ for $ \epsilon \leq \epsilon_0$.
%
%
% 	In summary, we have shown that if there is a solution to
% 	$ \tilde{F}(\alpha , \omega, c)=0$ when  $ \alpha \in [1.5706,\pp]$, then
% 	$ 0< \epsilon \leq \epsilon_0 \approx 0.0288$ and $ |\omega - \pp| < 0.1489 $ and $ \| c \| \leq   0.0398 \epsilon$ and $\| K^{-1} c \|_{\ell^1/\C} < 0.715 \epsilon^2$.
% 		To prove that $F  \equiv 0$ has no solutions for $\alpha \in [1.5706,\pp]$, we construct balls $B_\epsilon(r,\rho)$ which both will contain this region described above, and for each $ 0<\epsilon \leq \epsilon_0$, the ball will contain a unique solution $ \tilde{F}_\epsilon(\alpha_\epsilon,\omega_\epsilon,c_\epsilon)$ for which $ \alpha_\epsilon > \pp$.
% 	To satisfy both these objectives, we fix the following constants:
% 	\begin{align}
% 	r_\alpha =&	0.21
% 	&
% 	r_\omega =&	0.16
% 	&
% 	r_c =& 	0.09
% 	&
% 	\rho =& 	1.01
% 	\end{align}
% 	If $ \tilde{F}(\alpha,\omega,c) =0$ at parameter $ 0 < \epsilon \leq \epsilon_0$, then one can apply the triangle inequality to show that the variables $ (\alpha, \omega,c)$ must satisfy the following inequalities:
% 	\begin{align}
% 	r_\alpha \geq&| \alpha -  \balpha_\epsilon | ,
% 	&
% 	r_\omega  \geq& | \omega - \omega(\epsilon) | ,
% 	&
% 	\epsilon  \, r_c \geq& \|c -  \epsilon \, c(\epsilon) \|_{\ell^1/\C} ,
% 	\end{align}
% 	where $\balpha_\epsilon$ and $\omega(\epsilon)$ and $c(\epsilon)$ are given in Equation \ref{eq:Approx}.
% 	Furthermore, if $ \tilde{F}(\alpha,\omega,c) =0 $ then $\| K^{-1} c \|_{\ell^1/\C} < 0.715 (\epsilon \times \epsilon_0  ) < 0.021 \epsilon < \rho \epsilon$.
% 	Hence the ball $B_\epsilon(r,\rho)$ contains all the solutions to $ F \equiv 0$  when $ \alpha \in [1.5706,\pp]$.
%
% 	At these values, all of the components of the radii polynomials $P(\epsilon_0,r,\rho)$ are negative.
% 	Furthermore $ \rho \geq C(\epsilon_0,r)$ as in Proposition \ref{prop:DerivativeEndo} and $ \epsilon_0 < \tfrac{5}{4}(2 + \sqrt{5})^{-1}$.
% 	By Corollary \ref{prop:RPUniformEpsilon} it follows that for all $ 0 < \epsilon \leq \epsilon_0$ there exists a unique point  $(\bar{\alpha}_\epsilon , \bar{\omega}_\epsilon , \bar{c}_\epsilon) \in B_\epsilon(r,\rho)$ for which  $F(\alpha,\omega, c) =0$.
% 	It follows from  Corollary \ref{prop:TightEstimate} that   $\bar{\alpha}_\epsilon > \pp$ for all $ 0 < \epsilon \leq 0.10$.
% 	Since
% 	$F(\alpha,\omega,\epsilon c) = \epsilon  \tilde{F}(\alpha , \omega, c)$, then $\tilde{F}(\alpha , \omega, c)=0$ has no solutions when  $ \alpha \in [1.5706,\pp]$.
%
%
%
%
% \end{proof}


%%%%%%%%%%%%%%%%%%%%%%%%%%%%%%%%%%%%%%%%%%%%%%%%%%%%%%%%%%%%%%%%%%%%%%%%%%%%


%%%%%%%%%%%%%%%%%%%%%%%%%%%%%%%%%%%%%%%%%%%%%%%%%%%%%%%%%%%%%%%%%%%%%%%%%%%%


%%%%%%%%%%%%%%%%%%%%%%%%%%%%%%%%%%%%%%%%%%%%%%%%%%%%%%%%%%%%%%%%%%%%%%%%%%%%


%
%
%As was formulated in the 2010 paper by Lessard, this can be broken up into two statements: that there are no folds in the branch of SOPS originating from the Hopf bifurcation, and that there are no isolas of periodic orbits. 
%In Proposition \ref{prop:WideEstimate} we  identified a box in $ \R^2 \oplus \ell^1 / \C$ inside which the only periodic orbits are those originating from the Hopf bifurcation.  
%In Proposition \ref{prop:TightEstimate} we  identified explicit error between our approximate solution and the true solution. 
%



\subsection{Towards Jones' conjecture}
\label{s:Jones}

Jones' conjecture states that for $ \alpha > \pp$ there exists a (globally) unique SOPS to Wright's equation.
 Theorem \ref{thm:RadPoly} shows that for a fixed small $\epsilon$ there is a (locally) unique $\alpha$ at which Wright's equation has a SOPS, represented by
 $(\hat{\alpha}_\epsilon,\hat{\omega}_\epsilon,\hat{c}_\epsilon)$. 
This is not sufficient to prove the local case of Jones conjecture.
To accomplish the latter, we show in Theorem \ref{thm:UniqunessNbd} 
 that near the bifurcation point there is, for each fixed $\alpha>\pp$, a (locally) unique SOPS to Wright's equation. 
We begin by showing that on the solution branch emanating from the Hopf bifurcation 
$\hat{\alpha}_\epsilon$ is monotonically increasing in~$\epsilon$,
i.e.\ $ \tfrac{d}{d \epsilon} \hat{\alpha}_{\epsilon} >0$.  
Since $\balpha_\epsilon = \pp + \tfrac{1}{5}(\tfrac{3 \pi}{2}-1) \epsilon^2  $,
we expect that $\tfrac{d}{d \epsilon} \hat{\alpha}(\epsilon) = \tfrac{2}{5}(\tfrac{3 \pi}{2}-1) \epsilon + \cO(\epsilon^2)$. 
For this reason it is essential that we calculate an approximation of $\tfrac{d}{d \epsilon} \hat{\alpha}_\epsilon$ which is accurate up to order $ \cO(\epsilon^2)$.   

\begin{theorem}
	\label{thm:NoFold}
For $0 < \epsilon \leq 0.1$ we have $ \tfrac{d}{d \epsilon} \hat{\alpha}_{\epsilon} >0$. 
For 
%\change[J]{$\pp  < \alpha \leq \pp + 6.2757  \times 10^{-3}$ }
$\pp  < \alpha \leq \pp + 6.830  \times 10^{-3}$ 
there are no  bifurcations in the branch of SOPS that originates from the Hopf bifurcation. 
\end{theorem}

%
%In Lessard 2010, it was shown that the branch of SOPS bifurcating from $\pp$ does not have any folds for $ \alpha \in [\pp + \epsilon_1, 2.3]$ where $\epsilon_1 = 7.3165 \times 10^{-4}$. 
%In the Floquet Multipliers paper in preparation, we show that there is a unique SOPS to Wright's equation for $ \alpha \in [1.94,6]$. 
%In Xie's 1991 thesis, he shows that there is a unique SOPS to Wright's equation for $ \alpha > 5.67$.  
%Thereby, we have effectively proved the conjecture in \cite{lessard2010recent}  that the branch of SOPS bifurcating from $\pp$ has no folds. 
%\newline 

\begin{proof}
We show that the branch of solutions  $ \hat{x}_\epsilon =  (\hat{\alpha}_\epsilon , \hat{\omega}_\epsilon , \hat{c}_\epsilon)$ obtained in Proposition \ref{prop:TightEstimate} satisfies $  \tfrac{d}{d \epsilon} \hat{\alpha}_\epsilon >0$ for $0<\epsilon \leq 0.1$.
This implies that the solution branch is (smoothly) parametrized by~$\alpha$,
i.e.,  there are no secondary nor any saddle-node bifurcations in this branch.
We then show that these $\epsilon$-values cover the range 
%\change[J]{$\pp  < \alpha \leq \pp + 6.2757  \times 10^{-3}$.}
$\pp  < \alpha \leq \pp + 6.830  \times 10^{-3}$.
	
We begin by
	differentiating the equation $ F( \hat{x}_\epsilon) =0$ with respect to $ \epsilon$:
%
% Proposition \ref{prop:TightEstimate} states that for $ \epsilon \leq 0.10$ there exists a locally unique solution $ \bar{x}_\epsilon =  (\bar{\alpha}_\epsilon , \bar{\omega}_\epsilon , \bar{c}_\epsilon)$ to $F(x)=0$. 
% We show that $ \tfrac{d}{d \epsilon} \bar{\alpha}_\epsilon >0$ by implicit differentiation, which implies that there are no subsequent bifurcations.  
\begin{equation}
 \frac{\partial F}{\partial  \epsilon}(\hat{x}_\epsilon) + D F( \hat{x}_\epsilon)  \frac{d }{d  \epsilon} \hat{x}_\epsilon  = 0 .
\end{equation}
In terms of the map $T$ we obtain the relation
\[
\left[I-DT(\hat{x}_\epsilon)  \right]  \frac{d }{d \epsilon} \hat{x}_\epsilon   
=- A^{\dagger} \frac{\partial F}{\partial  \epsilon}(\hat{x}_\epsilon)  .
\]

To isolate $\frac{d }{d \epsilon} \hat{x}_\epsilon   $, we wish to left-multiply each side of the above equation by $[I-DT(\hat{x}_\epsilon)]^{-1}$. 
To that end, we define an upper bound on $DT(\hat{x}_\epsilon)$ by  the matrix 
\begin{equation}\label{e:defZeps}
	\ZZ_\epsilon := Z(\epsilon,\epsilon^2 \rr, \rho) ,
\end{equation}
with $\rr$ and $\rho$ as in Proposition~\ref{prop:TightEstimate}.
We know from Remark~\ref{r:boundDT} 
%the proof of 
that with respect to the norm $\| \cdot \|_{\rr}$ on $\R^2 \times \ell^K_0$
\[
\| DT(\hat{x}_\epsilon)  \|_{\rr} \leq \max_{i=1,2,3} \frac{( \ZZ_\epsilon \cdot \rr)_i}{\rr_i} < 1, \qquad\text{for all } 0 \leq \epsilon \leq \epsilon_0, 
\]
with $\epsilon_0$ given in Proposition~\ref{prop:TightEstimate}. 
Hence $I-DT(\hat{x}_\epsilon) $ is invertible. In particular,
\begin{alignat*}{1}
\frac{d }{d \epsilon} \hat{x}_\epsilon   
& =- \left[I-DT(\hat{x}_\epsilon)  \right]^{-1}  A^{\dagger} \frac{\partial F}{\partial  \epsilon}(\hat{x}_\epsilon)  \\
& = - \left[I + \sum_{n=1}^\infty DT(\hat{x}_\epsilon)^n  \right]  A^{\dagger} \frac{\partial F}{\partial  \epsilon}(\hat{x}_\epsilon) .
\end{alignat*}
We  have an upper bound $\QQ_\epsilon \in \R^3_+$ on $A^{\dagger} \frac{\partial F}{\partial  \epsilon}(\hat{x}_\epsilon)$, as defined in Definition~\ref{def:upperbound}, given by Lemma~\ref{lem:Qeps}. 
We define $\II$ to be the $3 \times 3$ identity matrix.
For the $\alpha$-component we then obtain the estimate
% \begin{alignat}{1}
% \frac{d }{d \epsilon} \hat{\alpha}_\epsilon
% &\geq - \pi_\alpha  A^{\dagger} \frac{\partial F}{\partial  \epsilon}(\hat{x}_\epsilon)
% - \left( \sum_{n=1}^\infty \ZZ_\epsilon^n \QQ_\epsilon \right)_1 \nonumber \\
% & = - \pi_\alpha  A^{\dagger} \frac{\partial F}{\partial  \epsilon}(\hat{x}_\epsilon)  - \left( \ZZ_\epsilon (1-\ZZ_\epsilon)^{-1} \QQ_\epsilon \right)_1 . \label{e:alphaepsilon}
% \end{alignat}
% \note[J]{Proposed Change}
\begin{alignat}{1}
\frac{d }{d \epsilon} \hat{\alpha}_\epsilon  
&\geq - \pi_\alpha  A^{\dagger} \frac{\partial F}{\partial  \epsilon}(\hat{x}_\epsilon)  
- \left( \sum_{n=1}^\infty \ZZ_\epsilon^n \QQ_\epsilon \right)_1 \nonumber \\
& = - \pi_\alpha  A^{\dagger} \frac{\partial F}{\partial  \epsilon}(\hat{x}_\epsilon)  - \left( \ZZ_\epsilon (\II-\ZZ_\epsilon)^{-1} \QQ_\epsilon \right)_1 . \label{e:alphaepsilon}
\end{alignat}
%\annote[JB]{We note that the elements of the matrix $\ZZ_\epsilon$ are $\cO(\epsilon^2)$.}{Shall we just remove this sentence? It is cryptic (no argument is given) and anyway this is repeated and made precise in Eqn (4.10).}
We approximate $\frac{\partial F}{\partial  \epsilon}(\hat{x}_\epsilon)$ by 
\[
	\Gamma_\epsilon := \pp \tfrac{3i -1}{5} \epsilon \, \e_1 - i \pp \,  \e_2 - \pp \tfrac{3+i}{5} \epsilon \, \e_3 ,
\]
which is accurate up to quadratic terms in $\epsilon$.
In Lemma \ref{lem:ImplicitApprox} it is shown that
\begin{equation}\label{e:linearepsilon}
- \pi_\alpha A^{\dagger} \Gamma _\epsilon = \tfrac{2}{5} ( \tfrac{3 \pi}{2} -1) \epsilon.
\end{equation}
It remains to incorporate two explicit bounds for the remaining terms in~\eqref{e:alphaepsilon}. 
In Lemma~\ref{lem:Meps} we define $M_\epsilon$ and $M'_\epsilon$ that satisfy the following inequalities:
% \begin{alignat}{1}
% \left| \pi_\alpha A^{\dagger} \left( \tfrac{\partial F}{\partial  \epsilon}(\hat{x}_\epsilon) - \Gamma_\epsilon \right)  \right| &\leq
% \epsilon^2 M_\epsilon  , \label{e:boundM} \\
% %
% \left( \ZZ_\epsilon (1-\ZZ_\epsilon)^{-1} \QQ_\epsilon \right)_1 &\leq
%  \epsilon^2 M'_\epsilon . \label{e:boundMp}
% \end{alignat}
% \note[J]{Proposed Change}
\begin{alignat}{1}
\left| \pi_\alpha A^{\dagger} \left( \tfrac{\partial F}{\partial  \epsilon}(\hat{x}_\epsilon) - \Gamma_\epsilon \right)  \right| &\leq 
\epsilon^2 M_\epsilon  , \label{e:boundM} \\
%
\left( \ZZ_\epsilon (\II-\ZZ_\epsilon)^{-1} \QQ_\epsilon \right)_1 &\leq 
\epsilon^2 M'_\epsilon . \label{e:boundMp} 
\end{alignat}
Moreover, we infer from Lemma~\ref{lem:Meps} that $M_\epsilon$ and $M'_\epsilon$ are positive, increasing in $\epsilon$, and can be 
obtained explicitly by performing an interval arithmetic computation, using the explicit expressions for the matrix $\ZZ_\epsilon$ and the vector $\QQ_\epsilon$ given by Equation~\eqref{e:defZeps} and Lemma~\ref{lem:Qeps}, respectively (the expression for $Z(\epsilon,r,\rho)$ is provided in Appendix~\ref{sec:BoundingFunctions}).
 
Finally, we combine~\eqref{e:alphaepsilon},~\eqref{e:linearepsilon},~\eqref{e:boundM} and~\eqref{e:boundMp} to obtain
\[
 \frac{d }{d \epsilon} \hat{\alpha}_\epsilon  \geq 
 \tfrac{2}{5} ( \tfrac{3 \pi}{2} -1) \epsilon - \epsilon^2 ( M_{\epsilon} + M'_{\epsilon}).
\]
From the monotonicity of the bounds $M_\epsilon$ and $M'_\epsilon$ in terms of $\epsilon$, we infer that in order to conclude that  $\frac{d }{d \epsilon} \hat{\alpha}_\epsilon >0 $ for $0<\epsilon\leq\epsilon_0$  it suffices to check, using interval arithmetic, that
\begin{equation}
 \tfrac{2}{5} ( \tfrac{3 \pi}{2} -1) \epsilon_0 - \epsilon_0^2 (M_{\epsilon_0} + M'_{\epsilon_0})  > 0 . \label{e:Mepsilon0}
\end{equation} 
In the Mathematica file~\cite{mathematicafile} we check that~\eqref{e:Mepsilon0} is satisfied 
for $\epsilon_0 = 0.1$.
Since $\balpha_{\epsilon_0} \geq \pp + 7.4247\times 10^{-3}$,
and taking into account the control provided by Proposition~\ref{prop:TightEstimate} on the distance between $\hat{\alpha}_\epsilon$ and $\balpha_\epsilon$ in terms of $\rr_\alpha$, we
find that  
%\change[J]{$\hat{\alpha}_{\epsilon_0} \geq \balpha_{\epsilon_0} - \epsilon_0^2 \rr_\alpha \geq \pp + 6.2757  \times 10^{-3}$.}
 $\hat{\alpha}_{\epsilon_0} \geq \balpha_{\epsilon_0} - \epsilon_0^2 \rr_\alpha \geq \pp + 6.830  \times 10^{-3}$.
Hence  there can be no bifurcation on the solution branch for 
%\change[J]{$ \pp < \alpha \leq \pp + 6.2757   \times 10^{-3}$.}
 $ \pp < \alpha \leq \pp + 6.830   \times 10^{-3}$.
\end{proof}	



% Since $\ZZ_\epsilon$ is $\cO(\epsilon^2)$,
% NEXT DETERMINE $Q$ AND (SOME APPROXIMATION OF) $\pi_\alpha A^{\dagger} \frac{\partial F}{\partial  \epsilon}(\hat{x}_\epsilon)$.
% NOW I HAVE TO UNDERSTAND APPENDIX E, WHICH DOESN'T SEEM THAT EASY.
%
% We can calculate $\frac{d }{d  \epsilon} \bar{x}_\epsilon $ using the approximate inverse $A^{\dagger}$.
% \begin{eqnarray}
%  D F( x_\epsilon) \cdot \frac{\partial }{\partial  \epsilon} \bar{x}_\epsilon
% &=&
% - \frac{\partial F}{\partial  \epsilon}(\bar{x}_\epsilon) \\
%  %
%  \left[I -(I- A^{\dagger} D F( \bar{x}_\epsilon) )  \right] \cdot \frac{\partial }{\partial  \epsilon} \bar{x}_\epsilon
%  &=&
% - A^{\dagger} \frac{\partial F}{\partial  \epsilon}(\bar{x}_\epsilon) \\
% %
%  \left[I-DT(\bar{x}_\epsilon)  \right] \cdot \frac{\partial }{\partial  \epsilon} \bar{x}_\epsilon
% &=&
% - A^{\dagger} \frac{\partial F}{\partial  \epsilon}(\bar{x}_\epsilon)
% \end{eqnarray}
% As a result of the radii polynomial method being successful in Proposition \ref{prop:TightEstimate}, it follows that $\| DT(\bar{x}_\epsilon) \|_{r} <1$ for $ \epsilon < 0.10$, whereby $I - DT(\bar{x}_\epsilon)$ is invertible.
% To shorten our notation, we introduce the variable $ B:=DT(\bar{x}_\epsilon) $.
% We calculate further:
% \begin{eqnarray}
% \frac{\partial }{\partial  \epsilon} \bar{x}_\epsilon &=& - (I - B)^{-1} A^{\dagger} \frac{\partial F}{\partial  \epsilon}(\bar{x}_\epsilon)  \\
% &=& - \left( I + \sum_{k=1}^\infty B^k \right)   A^{\dagger} \frac{\partial F}{\partial  \epsilon}(\bar{x}_\epsilon)
% \end{eqnarray}
% Choosing $r$ as in Proposition \ref{prop:TightEstimate} gives us the bound $ \bar{x}_\epsilon \in B_{\epsilon}(r \epsilon^2 , \rho)$.
% Hence $\ZZ_\epsilon := Z(\epsilon ,\epsilon^2 r , \rho)$ is an upper bound for $ DT(\bar{x}_\epsilon)$.
% Suppose that we have  an upper bound	$\overline{A^{\dagger} \frac{\partial F}{\partial  \epsilon}(\bar{x}_\epsilon) } $ on the operator $A^{\dagger} \frac{\partial F}{\partial  \epsilon}(\bar{x}_\epsilon) $.
% Then we may proceed to obtain a lower bound on $ [\frac{\partial }{\partial  \epsilon} \bar{x}_\epsilon ]_{\alpha}$ as follows:
% \begin{eqnarray}
% \left[ \frac{\partial }{\partial  \epsilon} \bar{x}_\epsilon\right]_{\alpha}
% &=&
%  -  \left[ A^{\dagger} \frac{\partial F}{\partial  \epsilon}(\bar{x}_\epsilon)  \right]_{\alpha} -
% \left[ \left( \sum_{k=1}^\infty B^k \right)   A^{\dagger} \frac{\partial F}{\partial  \epsilon}(\bar{x}_\epsilon)   \right]_{\alpha}\\
% %
% \left[ \frac{\partial }{\partial  \epsilon} \bar{x}_\epsilon\right]_{\alpha}
% &\geq &
%    \left[ -A^{\dagger} \frac{\partial F}{\partial  \epsilon}(\bar{x}_\epsilon)  \right]_{\alpha} - \left[
% \left( \sum_{k=1}^\infty \ZZ_\epsilon ^k \right)
% \overline{  A^{\dagger} \frac{\partial F}{\partial  \epsilon}(\bar{x}_\epsilon) } \right]_{\alpha}    \\
% %
% &=&
%  \left[- A^{\dagger} \frac{\partial F}{\partial  \epsilon}(\bar{x}_\epsilon)  \right]_{\alpha} -
%  \left[
% 	  \ZZ_\epsilon( I -\ZZ_\epsilon)^{-1} \overline{A^{\dagger} \frac{\partial F}{\partial  \epsilon}(\bar{x}_\epsilon) }   \,
% \right]_{\alpha} \label{eq:NoFoldIneq}
% \end{eqnarray}
%
%
%
%
%
%
% 	In order to show that $ \tfrac{\partial}{\partial \epsilon} \bar{\alpha}_\epsilon >0$, it suffices to show that the RHS of Line \ref{eq:NoFoldIneq} is positive.
% 	Since $ \ZZ_\epsilon$ is of order $ \cO(\epsilon^2)$, then to obtain a $ \cO(\epsilon^2)$  approximation of $ \tfrac{\partial}{\partial \epsilon} \bar{\alpha}_\epsilon$, we define a $ \cO(\epsilon^2)$ approximation of $\tfrac{\partial}{\partial \epsilon}F(\bar{x}_\epsilon)$ as below:
% 	\begin{equation}
% 	\label{eq:GammaDef}
% 	\Gamma := \pp[\tfrac{3i -1}{5} \epsilon] e_1 - \pp [i] e_2 - \pp[\tfrac{3+i}{5} \epsilon] e_3
% 	\end{equation}
% 	In Proposition \ref{prop:ImplicitApprox} we show that $[-A^{\dagger} \Gamma ]_\alpha =\tfrac{2}{5} ( \tfrac{3 \pi}{2} -1) \epsilon$, which is what we expected the first order approximation of $ \alpha'(\epsilon)$ to be.
% 	Hence, showing that Line \ref{eq:NoFoldIneq} is positive is equivalent to proving the inequality below:
%
% 	\begin{equation}
% 	\tfrac{2}{5} ( \tfrac{3 \pi}{2} -1) \epsilon
% 	%
% 	>
% 	%
% 	 \left[- A^{\dagger} \left( \tfrac{\partial F}{\partial  \epsilon}(\bar{x}_\epsilon) - \Gamma \right) \right]_{\alpha} +
% 	 \left[
% 	 \ZZ_\epsilon( I -\ZZ_\epsilon)^{-1} \overline{A^{\dagger} \tfrac{\partial F}{\partial  \epsilon}(\bar{x}_\epsilon) }   \,
% 	 \right]_{\alpha}
% 	 \label{eq:ImplicitDiffEq2}
% 	\end{equation}
% %To facilitate computing  the RHS of Line \ref{eq:ImplicitDiffEq2},
% In Proposition \ref{prop:ImplicitLast}  we explicitly define an upper bound on the vector $
% 	A^{\dagger} \left( \tfrac{\partial F}{\partial  \epsilon}(\bar{x}_\epsilon) - \Gamma \right) $ which by definition is of order $\cO(\epsilon^2)$ and  we denote here as $C$.
% 	Using this bound, we can expand Line \ref{eq:ImplicitDiffEq2} as follows:
% 	\begin{equation}
% 	\tfrac{2}{5} ( \tfrac{3 \pi}{2} -1) \epsilon
% 	%
% 	\geq
% 	%
% 	\left[
% 	C
% 		 \right]_{\alpha}
% 	%
% 	+
% 	%
% 	\left[
% 	\ZZ_\epsilon( I -\ZZ_\epsilon)^{-1}  \left(A^{\dagger} \Gamma +
% 	C  \right) \,
% 	\right]_{\alpha}
% 	\label{eq:ImplicitInequality}
% 	\end{equation}
% 	Since $\ZZ_\epsilon$ and $C$ are both of order $\cO(\epsilon^2)$, then it follows that the RHS of Line \ref{eq:ImplicitInequality} is of order $ \cO(\epsilon^2)$.
% 	Hence, there exists some $\epsilon>0$ for which Inequality \ref{eq:ImplicitInequality} is satisfied, and furthermore, if some $\epsilon_0 >0$ satisfies Inequality \ref{eq:ImplicitInequality}, then the inequality is satisfied for all $0 < \epsilon \leq \epsilon_0$.
%
% 	Using interval arithmetic, we check that Inequality \ref{eq:ImplicitInequality} is satisfied for $\epsilon_0 = 0.10$. Whereby $ \alpha'(\epsilon) > 0$ for $ 0 < \epsilon \leq \epsilon_0$.
% 	Since $ \alpha(\epsilon_0)= \pp + 7.4247\times 10^{-3}$, then by accounting for our error in approximating $\alpha$ with our estimate from Proposition \ref{prop:TightEstimate}, there can be no subsequent bifurcations for $ \pp < \alpha < \pp + 6.3779 \times 10^{-3}$.
%
%


%%	Since we have computed an $ \cO(\epsilon^2)$ approximation, then for a fixed $ \rho$,  the components of  $ \ZZ_\epsilon$ and 
%%	$\overline{ 
%%		A^{\dagger} \left( \tfrac{\partial F}{\partial  \epsilon}(x_\epsilon) - \Gamma \right) }  $  are polynomials with non-negative coefficients, of order $\epsilon^2$ and higher. 
%%	It then follows that $ \epsilon^{-2}$ times the RHS of Equation \ref{eq:ImplicitInequality} is well defined at $ \epsilon = 0$ and non-decreasing in $\epsilon$. 




The above theorem provides the missing piece in the proof of Theorem~\ref{thm:IntroNoFold}.

\begin{corollary}\label{cor:collectreformulatedJones}
The branch of SOPS originating from the Hopf bifurcation at $\alpha = \pp$ has no folds or secondary bifurcations for any $\alpha > \pp$. 
\end{corollary}
\begin{proof}
	
	We prove the corollary by combining results on four overlapping subintervals of $ ( \pp, \infty)$. 
	In Theorem~\ref{thm:NoFold} we show that the (continuous) branch of SOPS originating from the Hopf bifurcation does not have any folds or secondary bifurcations for 
		$ \alpha \in ( \pp  , \pp + \delta_3] $ where 
%\change[J]{ $ \delta_3 = 6.2757  \times 10^{-3}$. 
 $ \delta_3 = 6.830  \times 10^{-3}$. 
	In~\cite{lessard2010recent} the same result is proved for $ \alpha \in [ \pp + \delta_1, 2.3]$, where $\delta_1 = 7.3165 \times 10^{-4}$. 
	In~\cite{jlm2016Floquet} 
	 it is shown that there is a unique SOPS  for $ \alpha $ in the interval $[1.94,6.00]$. 
	 Since $1.94 \leq 2.3$, then the SOPS in this interval belong to the branch originating from the Hopf bifurcation, and since they are unique for each $\alpha$, the branch is continuous and cannot have any folds or secondary bifurcations. 
	  In~\cite{xie1991thesis} it is shown that there is a unique SOPS for $ \alpha $ in the interval $ [5.67, +\infty)$, and by a similar argument the branch of SOPS cannot have any folds or secondary bifurcations in this interval either. 
	 Since 
	\[
	(\pp, \infty) = (\pp, \pp + \delta_3] \cup [ \pp + \delta_1,2.3] \cup [1.94,6.00] \cup [5.67, \infty) ,
	\]
	it follows that branch of SOPS originating from the Hopf bifurcation at $\alpha = \pp$ has no folds or secondary bifurcations for any $\alpha > \pp$. 
\end{proof}

To prove Jones' conjecture, it is insufficient to prove only locally that Wright's equation has a unique SOPS.  
We must be able to connect our local results with global estimates. 
When we make the change of variables $\tc = \epsilon c$ in defining the function $F_\epsilon$, we restrict ourselves to proving local results.  
Theorems~\ref{thm:UniqunessNbd} and~\ref{thm:UniqunessNbd2} connect these local results with a global argument, and construct neighborhoods, independent of any $ \epsilon$-scaling, within which the only SOPS to Wright's equation are those originating from the Hopf bifurcation.  

The next theorem uses the large radius calculation from Proposition \ref{prop:bigboxes}(b) to show that for  
%\change[J]{$\alpha \in ( \pp , \pp+ 4.75 \times 10^{-3} ]$}
$\alpha \in ( \pp , \pp+ 5.53 \times 10^{-3} ]$
all periodic solutions in a neighborhood of $0$ lie on the Hopf bifurcation curve, which has neither folds nor secondary bifurcations.  

% \note[J]{Almost every number in the Theorem and proof below has changed, so I did not always use track changes. Please check this. }

\begin{theorem}
	\label{thm:UniqunessNbd}
	For each $\alpha  \in  (\pp , \pp + 5.53 \times 10^{-3} ] $ there is a unique triple $ ( \epsilon, \omega, c)$ in the range 
%\change[J]{$ 0 < \epsilon \leq 0.087$}
$ 0 < \epsilon \leq 0.09$
	and 
%\change[J]{$ | \omega - \pp| < 0.061 $}
$ | \omega - \pp| < 0.0924 $
	and  
%\change[J]{$ \| c \| \leq 0.13135 $}
$ \| c \| \leq 0.30232 $ 
such that $ F_\epsilon(\alpha, \omega, c)=0$. 	
\end{theorem}
 
\begin{proof}
Fix $ \alpha \in ( \pp , \pp + 5.53 \times 10^{-3}]$ and let $F_\epsilon(\alpha, \omega, c)=0$ for some $\epsilon, \omega, c$ satisfying the assumed bounds. 
%Let $0 \leq \epsilon \leq \epsilon_0:=0.087$ and let $F_\epsilon(\alpha, \omega, c)=0$ for some $(\alpha, \omega, c)$ satisfying the assumed bounds.
From Lemma~\ref{lem:Cone}(b) it follows that 
%\change[J]{$\|K^{-1}c\| \leq \epsilon^2 (1+ \|c\|^2) /(\epsilon b_*) \leq 0.32$}
$\|K^{-1}c\| \leq \epsilon^2 (2+ \|c\|^2) /(\epsilon b_*) \leq 0.61$
for $\epsilon \leq \epsilon_0$, 
since $b_* \geq 0.31$. 
Hence the zeros under consideration all lie in the set 
% \[
% \tS :=  \{ (\alpha,\omega,c) \in X : | \alpha - \pp | \leq 0.00553 , |\omega - \pp| \leq 0.0924, \| c \| \leq 0.30232, \|K^{-1} c \| \leq 0.32  \}.
% \]
% \note[J]{Proposed change}
\[
\tS :=  \{ (\alpha,\omega,c) \in X : | \alpha - \pp | \leq 0.00553 , |\omega - \pp| \leq 0.0924, \| c \| \leq 0.30232, \|K^{-1} c \| \leq 0.61  \}.
\]
Proposition~\ref{prop:bigboxes}(b) shows that for each $0\leq\epsilon\leq 0.09$
there is a unique zero $\hat{x}_\epsilon=
(\hat{\alpha}_\epsilon,\hat{\omega}_\epsilon,\hat{c}_\epsilon) \in B_\epsilon(r,\rho)$ of $F_\epsilon$,
with $r=(r_\alpha,r_\omega,r_c) = (0.1753,0.0941,0.3829)$ and $\rho= 1.5940$.
For each $0 \leq \epsilon \leq 0.09$ it follows from the triangle inequality  that $\tS \subset B_\epsilon(r,\rho)$.  
This shows that $F_\epsilon$ has at most one zero in $\tS$ for each $ 0 \leq \epsilon \leq \epsilon_0$. 
By Remark~\ref{r:nested} this solution lies on the branch $\hat{x}_\epsilon$ originating from the Hopf bifurcation, in particular $\hat{x}_0=(\pp,\pp,0) \in \tS$.
Proposition~\ref{prop:TightEstimate} gives us tight bounds 
\[
|\hat{\omega}_\epsilon - \pp| \leq |\bomega_\epsilon - \pp|  + \rr_\omega \epsilon^2 \leq 0.0924
\qquad\text{and}\qquad \| \hat{c}_\epsilon \| \leq \| \bc_\epsilon\|  + \rr_c \epsilon^2 \leq 0.30232
\]
for all $0 \leq \epsilon \leq \epsilon_0$.  
Moreover, from similar considerations it follows that $\hat{\alpha}_{\epsilon_0} \geq  \balpha_{\epsilon_0} - r_\alpha \epsilon_0^2 > 0.00553$. Hence $\hat{x}_{\epsilon_0} \notin \tS$ and the solution curve leaves $\tS$ through $|\alpha- \pp| = 0.00553$ for some $0<\epsilon <\epsilon_0$.
Since $0.00553  < 6.830 \times 10^{-3}$ the assertion now follows directly from Theorem~\ref{thm:NoFold}.
\end{proof}
%

% OLD PROOF BELOW
%Fix $ \alpha \in ( \pp , \pp + 4.750 \times 10^{-3}]$ and let $F_\epsilon(\alpha, \omega, c)=0$ for some $\epsilon, \omega, c$ satisfying the assumed bounds. 
%%Let $0 \leq \epsilon \leq \epsilon_0:=0.087$ and let $F_\epsilon(\alpha, \omega, c)=0$ for some $(\alpha, \omega, c)$ satisfying the assumed bounds.
%From Lemma~\ref{lem:Cone}(b) it follows that $\|K^{-1}c\| \leq \epsilon^2 (1+ \|c\|^2) /(\epsilon b_*) \leq 0.27$ for $\epsilon \leq \epsilon_0$, 
%since $b_* \geq 0.33$. 
%Hence the zeros under consideration all lie in the set 
%\[
%  \tS :=  \{ (\alpha,\omega,c) \in X : | \alpha - \pp | \leq 0.00475 , |\omega - \pp| \leq 0.061, \| c \| \leq 0.13135, \|K^{-1} c \| \leq 0.27  \}.
%\]
%Proposition~\ref{prop:bigboxes}(b) shows that for each $0\leq\epsilon\leq 0.087$
%there is a unique zero $\hat{x}_\epsilon=
%(\hat{\alpha}_\epsilon,\hat{\omega}_\epsilon,\hat{c}_\epsilon) \in B_\epsilon(r,\rho)$ of $F_\epsilon$,
%with $r=(r_\alpha,r_\omega,r_c) = (0.1501,0.0626,0.2092)$ and $\rho= 0.5672$.
%For each $0 \leq \epsilon \leq 0.087$ it follows from the triangle inequality  that $\tS \subset B_\epsilon(r,\rho)$.  
%This shows that $F_\epsilon$ has at most one zero in $\tS$ for each $ 0 \leq \epsilon \leq \epsilon_0$. 
%By Remark~\ref{r:nested} this solution lies on the branch $\hat{x}_\epsilon$ originating from the Hopf bifurcation, in particular $\hat{x}_0=(\pp,\pp,0) \in \tS$.
%Proposition~\ref{prop:TightEstimate} gives us tight bounds 
%\[
%  |\hat{\omega}_\epsilon - \pp| \leq |\bomega_\epsilon - \pp|  + \rr_\omega \epsilon^2 \leq 0.061
%  \qquad\text{and}\qquad \| \hat{c}_\epsilon \| \leq \| \bc_\epsilon\|  + \rr_c \epsilon^2 \leq 0.657
%\]
%for all $0 \leq \epsilon \leq \epsilon_0$.  
%Moreover, from similar considerations it follows that $\hat{\alpha}_{\epsilon_0} \geq  \balpha_{\epsilon_0} - r_\alpha \epsilon_0^2 > 0.00475$. Hence $\hat{x}_{\epsilon_0} \notin \tS$ and the solution curve leaves $\tS$ through $|\alpha- \pp| = 0.00475$ for some $0<\epsilon <\epsilon_0$.
%Since $0.00475  < 6.2757 \times 10^{-3}$ the assertion now follows directly from Theorem~\ref{thm:NoFold}.
%\end{proof}

% OLDER PROOF BELOW
%
% \begin{proof}
%
%
% By Proposition \ref{prop:Cone}, if $\tilde{F}(\alpha,\omega,c)=0$ and $ |\alpha - \pp | \leq 0.0044$ and $| \omega - \pp| \leq 0.0811$, then either $ \| c\|_{\ell^1 / \C} < 0.121691 \times \epsilon$ or $ \| c\|_{\ell^1 / \C} > 0.657$.
% Hence, by our initial assumption we may take $ \| c\|_{\ell^1 / \C} < 0.121691 \times \epsilon$.
% Furthermore, Proposition \ref{prop:Cone} tells us that $ \| K^{-1} c \|_{\ell^1 / \C} <  (\epsilon^2+ \|c\|^2)/(2 b_*)$.
% Plugging in the appropriate values produces the estimate $ \| K^{-1} c \|_{\ell^1 / \C} <  \epsilon^2 \times 0.840672$.
% To apply the method of radii polynomials, we must  make the change of variables $ c \mapsto \epsilon c$.
% That is, if $ F( \alpha , \omega, c) =0$ then we may conclude that  $ \| c\|_{\ell^1 / \C} < 0.121691$ and  $ \| K^{-1} c \|_{\ell^1 / \C} <  \epsilon \times 0.840672$.
%
% To perform the method of radii polynomials on a specific ball, fix the following constants:
% \begin{align}
%  \epsilon_0 =& 0.080
% &
% r_\alpha =&	0.190814
% &
% r_\omega =&	0.0823824
% &
% r_c =& 	0.193644
% &
% \rho =& 	0.73103633
% \end{align}
% These constants are chosen so that the ball $ B_{\epsilon}( r , \rho)$ will contain any possible solutions to $ F( \alpha, \omega, c) =0$.
% That  is,  fix $ \alpha \in ( \pp, \pp + 0.0044]$, $ | \omega - \pp| < 0.0811$ and  $ \|c \|_{\ell^1 / \C} <   0.121691$.
% The constants $ r_\alpha$ and $ r_\omega$ and $ r_c$ were chosen that for all values of $ 0 < \epsilon \leq \epsilon_0$, then  an application of the triangle inequality relative to $\{ \balpha_\epsilon , \omega(\epsilon), c(\epsilon) \}$ implies the following inequalities:
% \begin{align}
% r_\alpha \geq& | \alpha -  \balpha_\epsilon |
% &
% r_\omega  \geq& | \omega - \omega(\epsilon) |
% &
% r_c \geq& \|c - c(\epsilon)\|
% \end{align}
% The value of $\rho$ was chosen to be the constant $ C(\epsilon,r)$ from Proposition \ref{prop:DerivativeEndo}.
% Since we earlier determined that $ \| K^{-1} c \|_{\ell^1 / \C} < \epsilon_0 \times 0.840672 < 0.068$ then it follows that  $ \| K^{-1} c \|_{\ell^1 / \C} \leq \rho$.
% Hence, if the variables $\epsilon$, $ \alpha$, $\omega,$ and $ c$ are chosen as per the assumptions of this theorem, the solutions $ F(\alpha,\omega,c)=0$ satisfy  $(\alpha,\omega,c) \in B_{\epsilon}(r,\rho)$.
%
%
% For these values of $ \epsilon_0, r, \rho$ all components of the radii polynomials $ P(\epsilon_0,r,\rho)$ are negative, as verified in the supplemental \emph{Mathematica} file.
% By Theorem \ref{prop:RPUniformEpsilon}, it follows that for all $ 0 < \epsilon \leq 0.08$, there exists a unique point  $ \bar{x}_\epsilon \in B_\epsilon(r,\rho)$ for which $ F( \bar{x}_\epsilon)=0$.
% Using our approximation defined in Equation \ref{eq:Approx} we calculate that $\alpha(0.08) = \pp + 0.00475$, so by applying our estimates from Proposition \ref{prop:TightEstimate} we may deduce that if $ \bar{\alpha}_\epsilon \in ( \pp , \pp+ 0.0044]$ then $\epsilon \in (0,0.08]$.
% By Theorem \ref{thm:NoFold}, we know that each  $\alpha \in ( \pp  , \pp +  6.3779  \times 10^{-3}] $  corresponds to a unique solution $ F(\bar{x}_\epsilon) =0$.
% In conclusion, if $ \alpha \in (\pp,\pp + 0.0044]$, then there exists a unique $ \epsilon \in ( 0,0.08]$ and $ | \omega - \pp | < 0.0811$ and $ \| c \|_{\ell^1 / \C} < 0.6766$ such that $ F( \alpha ,\omega, c)=0$ at parameter $\epsilon$.
%
%
%
%
%
%
% %then there are no folds in the branch of periodic orbits.
% %By Proposition \ref{prop:TightEstimate} we know that $| \bar{\alpha}_\epsilon - \alpha(\epsilon )| < 0.0411766 \epsilon^2$, so then for each $ \alpha \in (\pp , \pp + 0.0054]$ there exists a unique $0<\epsilon \leq 0.085$ for which $ \bar{\alpha}_\epsilon = \alpha$ and $ F(\bar{x}_\epsilon)=0$.
% %
% %
% %
%
%
% \end{proof}



Finally, we translate this result to function space.

\begin{theorem}
For each 
%	\change[J]{$\alpha  \in  (\pp , \pp + 0.00475] $ }
$\alpha  \in  (\pp , \pp + 5.53 \times 10^{-3} ] $ 
there is at most one (up to time translation) periodic solution to Wright's equation satisfying 
%\change[J]{$ \| y' \|_{L^2([0,2\pi/\omega])} \leq  0.295$}
$ \| y' \|_{L^2([0,2\pi/\omega])} \leq  0.302$
	and having frequency 
%\change[J]{$ | \omega - \pp | \leq 0.061$}
$ | \omega - \pp | \leq 0.0924$. 
	\label{thm:UniqunessNbd2}
\end{theorem}

\begin{proof}
We show that any periodic solution~$y$ to Wright's equation of period $2\pi/\omega$ that satisfies 
%\change[J]{$ \| y' \|_{L^2} \leq 0.295$ }
$ \| y' \|_{L^2} \leq 0.302$ 
has Fourier coefficients satisfying the bounds in Theorem~\ref{thm:UniqunessNbd}.
For the Fourier coefficients $a$ of $y$ we infer from Lemma~\ref{lem:fourierbound} that 
%\change[J]{$\| a \| \leq \sqrt{\frac{\pi}{6\omega}} \cdot 0.295 \leq 0.174 $.}
$\| a \| \leq \sqrt{\frac{\pi}{6\omega}} \cdot 0.302 \leq 0.18 $. 
Furthermore, for the parameter range of $\alpha$ and $\omega$ under consideration we conclude that $\alpha < 2\omega $ and 
$\|a\| < \frac{2\omega-\alpha}{\alpha}$. Hence we see from Theorem~\ref{thm:FourierEquivalence3}
that $y$ corresponds to a zero of $F_\epsilon$. 
The a priori bound on $\|a\|$ translates via~\eqref{e:aepsc} into the bounds
% \[
%   \epsilon \leq 0.087
%   \qquad\text{and}\qquad
%   \| \tc \| \leq 0.174 .
% \]
% \note[J]{New Version}
\[
\epsilon \leq 0.09
\qquad\text{and}\qquad
\| \tc \| \leq 0.18 .
\]
We now derive further bounds on $c=\tc/\epsilon$, 
 as in the proof of Lemma~\ref{lem:wrightbounds}.
Namely, for 
% \change[J]{$|\alpha-\pp| \leq 0.00475$,
% 	$|\omega-\pp| \leq 0.061$ and  $\epsilon \leq 0.087$,}
	$|\alpha-\pp| \leq 0.00553$,
	$|\omega-\pp| \leq 0.0924$ and  $\epsilon \leq 0.09$,
we find that  $z_*^+$, as defined in~\eqref{e:zstar}, is bounded below by 
%\change[J]{$z^+_* \geq 0.662$}
 $z^+_* \geq 0.595$. 
It follows that 
%\change[J]{$ \| \tc \|  \leq 0.174 \leq z^+_*$,}
$ \| \tc \|  \leq 0.18 \leq z^+_*$, 
so we infer from Lemma~\ref{lem:Cone}(a) that 
$\| \tc \| \leq z^-_* $.
Via Lemma~\ref{lem:ZminusBound} and an interval arithmetic computation, the latter can be bounded above, for 
% \change[J]{$|\alpha-\pp| \leq 0.00475$,
% 	$|\omega-\pp| \leq 0.061$ and  $\epsilon \leq 0.087$, by
% 	$z_*^- \leq 0.13135 \epsilon$. }
	 $|\alpha-\pp| \leq 0.00553$,
	$|\omega-\pp| \leq 0.0924$ and  $\epsilon \leq 0.09$, by
	$z_*^- \leq 0.30226 \epsilon$. 
Hence 
%\change[J]{$\| c \| \leq z_*^- / \epsilon \leq 0.13135$.}
 $\| c \| \leq z_*^- / \epsilon \leq 0.30232$.
%
We conclude that $y$ corresponds to a zero of $F_\epsilon(\alpha,\omega,c)$ in the parameter set described by Theorem~\ref{thm:UniqunessNbd}, which implies uniqueness.
\end{proof}

%%%
%%%
%%%Existence: I THINK EXISTENCE IS NOT GOING TO WORK FOR THE WHOLE RANGE OF $\alpha$. NAMELY $y \approx 2 \epsilon \cos (\pp t)$, hence  $y' \approx - 2 \pp \epsilon \sin (\pp t)$, and with $\omega \approx \pp$ this gives AT THE VERY BEST
%%%\[  
%%% \| y' \|_{L^2([0,2\pi/\omega])} \approx \sqrt{\int_0^4 \pi^2 \epsilon^2 \sin^2 (\pp t) dt } = \sqrt{2} \pi \epsilon .
%%%\]
%%%Since $\alpha-\pp \approx \frac{1}{5}(\frac{3\pi}{2}-1) \epsilon^2 = 0.74 \epsilon^2 $, we get
%%%$\| y' \|_{L^2} \approx 5.1 \sqrt{\alpha-\pp}$ AT THE VERY BEST (ON IN FACT WE WILL LOSE SOME IN THE ESTIMATE).
%%%
%%%WE NEED TO REFORMULATE THE THEOREM, PROBABLY WITH at most one INSTEAD OF a unique, BUT THAT DEPENDS ON WHAT JONATHAN NEEDS IN HIS OTHER PAPERS. 
%%%we still need to show that the unique solution described by Theorem~\ref{thm:UniqunessNbd} satisfies the bound $ \| y' \|_{L^2} \leq  0.295$.
%%%WE NEED TO ESTABLISH THAT THE solution given by Theorem~\ref{thm:UniqunessNbd} satisfies a tighter bound on $c$.
%%%We then do some estimate estimate.
%%%\[
%%%  \| y' \|_{L^2([0,2\pi/\omega])}^2 = \frac{4\pi}{\omega} \sum_{k=1}^\infty k^2 |a_k|^2 \leq  ???
%%%\] 




% OLD PROOF
%
% \begin{proof}
%
% 	If $ y$ is a periodic solution to Wright's equation, then by time translation, we may assume that $y$ can be written in the form:
% 	\[
% 	y(t) =
% 	\epsilon \left( e^{i \omega t }  + e^{- i \omega t }\right)
% 	+  \sum_{k = 2}^\infty    c_k e^{i \omega k t }  + \bar{c}_k e^{- i \omega k t }
% 	\]
% 	for some $ \epsilon  \geq 0$ and $ c \in \ell^1 / \C$.
% 	By Theorem \ref{thm:FourierEquivalence2} we know that $ y$ is a solution to Wright's equation at parameter $ \alpha$ if and only if $ \tilde{F}(\alpha ,\omega, c) =0$ at parameter  $\epsilon$.
%
% 	If we can show that $ \epsilon \in (0,0.08]$ and $ \| c \|_{ \ell^1 / \C} \leq 0.657$ then the hypothesis of Theorem~\ref{thm:UniqunessNbd} will be satisfied and we are done.
% 	By the calculation done in Line \ref{eq:WrightConjectureEll1BoundL2} it follows that  $( 2 \epsilon +  \|c\|_{\ell^1 / \C} ) < \tfrac{\pi}{\omega \sqrt{3}} \| y' \|_{L^2} $.
% 	For $|  \omega - \pp | < 0.0811$ and $ \| y'\|_{L^2} < 0.131$, it follows that $ ( 2 \epsilon +  \|c\|_{\ell^1 / \C} ) < 0.16$ whereby  $ \| c \|_{\ell^1 / \C} < 0.16 $ and  $\epsilon$ is in the range $ 0 \leq \epsilon \leq 0.08$.
% 	To show that $\epsilon$ is positive, we apply  Proposition \ref{prop:Cone}.
% 	This result tells us that if $0 \leq  \epsilon \leq 0.08$ and  $ |\alpha - \pp| < 0.0044$ and $ | \omega - \pp| < 0.0811$, then either $ \| c\|_{\ell^1 / \C} < 0.121691 \times \epsilon $ or $ \| c\|_{\ell^1 / \C} > 0.657$.
% 	Hence if $ \epsilon = 0$ and $\| c\|_{\ell^1 / \C} < 0.16$  then the only solution to $ \tilde{F}(\alpha,\omega,c)=0$ is the trivial solution, thus proving that $ \epsilon \in ( 0 , 0.08]$.
% 	Since the hypothesis of Theorem~\ref{thm:UniqunessNbd} is satisfied, then for each $ \alpha \in ( \pp, \pp + 0.0044]$ there exists a unique periodic orbit $y$ with frequency $ | \omega - \pp| < 0.0811$ and $\| y'\|_{L^2} < 0.131$.
%
%
%
%
%
% \end{proof}
%



\bibliographystyle{abbrv}
\bibliography{BibWright}

%%%%%%%%%%%%%%%%%%%%%%%%%%%%%%%%%%%%%%%%%%%%%%%%%%%%%%%%%%%%%%%%%%
%%%%%%%%%%%%%%%%%%%%%%%%  Appendices  %%%%%%%%%%%%%%%%%%%%%%%%%%%%
%%%%%%%%%%%%%%%%%%%%%%%%%%%%%%%%%%%%%%%%%%%%%%%%%%%%%%%%%%%%%%%%%%

\appendix

%
	\section{Proof of Proposition~\ref{prop-quasi-order}}
\begin{proof}
	We will prove the result for relation $\sqsubseteq$, the proof for $\preceq$ being similar. We need to prove that $\sqsubseteq$ is reflexive and transitive. For reflexivity, it is obvious that since $G\subseteq G$ for any goal $G$, we have $G\sqsubseteq_\theta G$ for the empty substitution $\theta$. For transitivity, suppose that for goals $G_1$, $G_2$ and $G_3$, it holds that $G_1 \sqsubseteq_{\theta_1} G_2$ and $G_2 \sqsubseteq_{\theta_2} G_3$. Then by Definition~\ref{def-generalization}, there exist sets of atoms $\Delta_1$ and $\Delta_2$ such that $G_1\theta_1 \cup \Delta_1 = G_2$ and $G_2\theta_2\cup\Delta_2 = G_3$. In other words it holds that $(G_1\theta_1\cup\Delta_1)\theta_2\cup\Delta_2 = G_3$ or equivalently, $(G_1\theta_1)\theta_2 \cup (\Delta_1\theta_2 \cup\Delta_2) = G_3$. As the composition of two substitutions is a substitution, by defining $\theta_3 = \theta_2\circ\theta_1$ and $\Delta_3 = \Delta_1\theta_2 \cup\Delta_2$, we have $G_1\theta_3 \cup \Delta_3 = G_3$, so $G_1\sqsubseteq_{\theta_3} G_3$, which concludes the proof.
\end{proof}
	

	\section{Proof of Proposition~\ref{prop-msg-lcg}}
	
		First, observe the following property that holds for both relations, essentially stating that a common generalization that is not a lcg has a direct extension obtained by the addition of one atom. 
		
		\begin{proposition}\label{prop-lcg-extensible}
			Let $G_1, \dots, G_n$ and $G$ be goals such that $G$ is a $\leqslant$-common generalization, but not a $\leqslant$-lcg, of $\{G_1, \dots, G_n\}$. Then there exists an atom $A\notin G$ such that $G\cup\{A\}$ is a $\leqslant$-common generalization of $\{G_1, \dots, G_n\}$.
		\end{proposition} 
	
	
		\begin{proof}
		
		Let us suppose the existence of some goal $G$, a $\leqslant$-common generalization that is not a $\leqslant$-lcg of $\{G_1, \dots, G_n\}$, and let us try and extend $G$ into a $\leqslant$-common generalization $G\cup\{A\}$ with $A\notin G$ an atom. As $G$ is not a lcg, there must exist another goal $G'$ being a $\leqslant$-lcg of $G_1$ and $G_2$ and obviously we have $|G'|>|G|$. As a consequence at this point there are three groups of atoms that can be identified: let us denote by $\hat{A}_1, \dots, \hat{A}_p$ the $p (\ge 0)$ atom(s) that are both in $G$ and in $G'$; by $A_1, \dots, A_m$ the $m (\ge 0)$ atom(s) that are part of $G$ but not of $G'$; and by $B_1, \dots, B_l$ the $l (\ge 1)$ atom(s) that are part of $G'$ but not of $G$. For an element $A$ of any of these sets, we denote by $A^1, \dots, A_n$ the atom in respectively $G_1, \dots, G_n$ whose anti-unification led to having $A$ as part of the generalizations.
		
		From the fact that $|G'|>|G|$ it follows that $l>m$. Now each $A_i (i \in 1..m)$ is such that $\exists  h\in 1..n : A_i^h\in \{B_i^h|i\in 1..l\}$: if not, it would be possible to add an atom generalizing $\{A_i^1, \dots, A_i^n\}$ (such as $A_i$) in $G'$ and get a larger generalization, which is impossible given that $G'$ is a lcg. Also note that for two atoms $B_i$ and $B_j (1\le i < j \le l)$, for $g, h \in 1..n : g\neq h$, if $B_i^g$ is anti-unifiable with an atom $B_j^h$ then $B_j^g$ is also anti-unifiable with $B_i^h$ (as it means that all four base atoms are a call to one and the same predicate (with relation $\sqsubseteq$) or have the exact same inner structure save for variables (with relation $\preceq$)), so that it is possible to switch the atoms $B_i^g$ and $B_j^h$, compute the anti-unification of $\{B_i^g, B_j^h)$ and $(B_j^g, B_i^h)$, and get an equally valid anti-unification. Thanks to this we can, where necessary, perform switches so as to rearrange the atoms $B_i (1\le i \le l)$ into $\{\hat{B}_i|i \in 1..l\}$ in such a way that $\{A_i|i\in 1..m\} \subset \{\hat{B}_i|i \in 1..l\}$ and for each atom $\hat{B}_i$, either $\hat{B}_i \in \{A_k|k\in 1..m\}$ or $\exists g, h \in 1..n : g\neq h \wedge \hat{B}_i^g\notin \{A_k^g|k\in 1..m\}\wedge \hat{B}_i^h\notin \{A_k^h|k\in 1..m\}$. We can now define a new generalization $\hat{G}$ defined as the union of these rearranged atoms and those that are common to $G$ and $G'$, i.e. $\hat{G} = \{\hat{A_i}|i\in 1..p\}\cup\{\hat{B}_i|i\in 1..l\}$. Since $|\hat{G}| = |G'|$ and $G\subset \hat{G}$, it suffices to add one of the atoms $A \in \hat{G}\setminus G$ to $G$ in order to obtain $G\cup\{A\}$, a $\leqslant$-common generalization of $\{G_1, \dots, G_n\}$ by construction.
	\end{proof}


		Next, we prove Proposition~\ref{prop-msg-lcg}.
		
	\begin{proof}
	We prove that any $\leqslant$-msg is a $\leqslant$-lcg by contradiction. Let us suppose that some goal $G$ is both a $\leqslant$-msg and not a $\leqslant$-lcg of the set of $\{G_1, \dots, G_n\}$. According to Proposition~\ref{prop-lcg-extensible} it must then be possible to select an atom $A \notin G$ such that $G\cup\{A\}$ is a $\leqslant$-common generalization of $\{G_1, \dots, G_n\}$. Since $A\notin G$ and any atom has a $\tau$-value of at least 1, it follows that $|\tau(G\cup\{A\})|>|\tau(G)|$. Consequently $G$ cannot be a $\leqslant$-most specific generalization of $G_1$ and $G_2$: a contradiction.
	
	As for the fact that any $\preceq$-lcg is a $\preceq$-msg, we prove this also by contradiction. Let $G$ represent a $\preceq$-lcg of the set of goals $\{G_1, \dots, G_n\}$ and let us suppose that $G$ is not a $\preceq$-msg. Then there must exist another goal that is a $\preceq$-msg of $\{G_1, \dots, G_n\}$, say $G'$, such that $|\tau(G')|>|\tau(G)|$ and, according to the first part of the proposition, $|G'|=|G|$. 
	Now, observe that for a set of atoms $\{A_1, \dots, A_n\}$ to be anti-unified with $\preceq$ into an atom $A$, necessarily all $A_i (1\le i\le n)$ must have the same $\terms$-value. Indeed relation $\preceq$ is defined upon renamings so that only variables (having a $\terms$-value of zero) are impacted by the generalization process. Therefore, the only possibility for the inequality $|\tau(G')|>|\tau(G)|$ to be true is that some atoms $B_1, \dots, B_n$ of respective goals $G_1,\dots,G_n$ appear in a generalized form (say $B$) in $G'$, while these atoms have not been generalized in $G$. This means that it is possible to add a (possibly renamed) version of $B$ in $G$ and obtain $G\cup\{B\}$, also a $\preceq$-common generalization and larger than $G$: a contradiction.
	\end{proof}

	
	\section{Detailed proof of Lemma~\ref{lemma-au-op}}
	\begin{proof}
The lemma will be shown correct by the definition of three anti-unification operators. A first anti-unification operator, based on $\sqsubseteq$ is the following. 

\begin{definition}%[Simple anti-unification operator]
	\label{def-atoms-au}
	Given a variabilization function $\Phi$, let $\au^\Phi_\sqsubseteq$ (or simply $\au_\sqsubseteq$ if $\Phi$ is clear from the context) denote the anti-unification operator such that for any two atoms $A = a(t^A_1, \dots, t^A_n)$ and $B = b(t^B_1, \dots, t^B_m)$, it holds that \[\au^{\Phi}_\sqsubseteq(A,B)=\left\{\begin{array}{l}
		a\big(\Phi(t^A_1, t^B_1), \dots, \Phi(t^A_n, t^B_n)\big) \\ \qquad  \mbox{if } a = b \mbox{ and } n = m\\
		\bot \\
		\qquad \mbox{otherwise}\\
	\end{array}\right. \]
\end{definition} 

\begin{example}\label{ex-au-sq}
	In Table~\ref{table:sqsubseteq}, we show three atomic anti-unification results obtained by the application of $\au_\sqsubseteq^\Phi$ with $\Phi$ a given variabilization function. Note how in the first example, the predicates used in $A_1$ and $A_2$ differ (resp. $p/2$ and $p/3$), leading to an impossible anti-unification.
\end{example}

\begin{table*}
	\caption{Example results for $au_\sqsubseteq^\Phi$}
	\label{table:sqsubseteq}
	\centering
	\begin{tabular}{l|l|l}
		%\hline 
		$\bm{A_1}$ & $\bm{A_2}$ & $\bm{\au_\sqsubseteq^\Phi(A_1, A_2)}$\\\hline 
		$p(X, 5, q(Y,4))$ & $p(W,t(Z))$ & $\bot$\\\hline 
		$p(r(X,3), t(5))$ & $p(W, t(Z))$ & $p(\Phi(r(X,3),W), \Phi(t(5), t(Z)))$\\\hline 
		$p(r(X,3), t(Y))$ & $p(r(W,3),t(Z))$ & $p(\Phi(r(X,3),r(W,3)), \Phi(t(Y),t(Z)))$ %\\\hline 
	\end{tabular} 
\end{table*}

Note that the anti-unification operator defined in Definition~\ref{def-atoms-au} differs from the traditional subsumption operator in the ordered case (i.e. when goals are ordered sequences of atoms). The difference comes from the fact that our goals being sets, all the possible couples of atoms have to be considered, whereas traditional subsumption must handle one atom at the time, making the anti-unification operator more straigtforward.     

Let us now introduce a second anti-unification operator that will allow to compute a $\preceq$-lcg. 
%and to prove Theorem~\ref{thm-preceq-lcg}. 
Since the result of this operator should be a $\preceq$-common generalization, the operator need only to anti-unify the \textit{variables} occurring at the corresponding positions in the atoms under investigation. 
The operator must thus go deeper into the term structure of the atoms than $\au_\sqsubseteq$ does, as it needs to only anti-unify those atoms that harbor the exact same structure at the level of their non-variable terms.
\begin{definition}%[Variable anti-unification operator]
	\label{def-term-au-through-variables}
	Given some variabilization function $\Phi$, let $\au^\Phi_\preceq$ (or simply $\au_\preceq$ if $\Phi$ is clear from the context) denote the function such that for any two terms $T = t(t_1, \dots, t_n)$ and $U = u(u_1, \dots, u_m)$ it holds that
	\[\au^\Phi_\preceq(T,U)=\left\{\begin{array}{l}
		\Phi(T,U) 
		\\ \qquad \mbox{if } T\in\mathcal{V}\mbox{ and } U\in\mathcal{V}
		\\t\big(\au^\Phi_\preceq(t_1,u_1), \dots, \au^\Phi_\preceq(t_n, u_n)\big) 
		\\ \qquad \mbox{if } t = u \mbox{ and } n = m 
		\\ \qquad \mbox{and } \forall i \in 1..n: \au^\Phi_\preceq(t_i,u_i)\neq\bot
		\\ \bot
		\\ \qquad  \mbox{otherwise}
	\end{array}\right.\]
	and for any two atoms $A = a(t^A_1, \dots, t^A_n)$ and $B = b(u^B_1, \dots, u^B_m)$, it holds that
	\[\au^\Phi_\preceq(A,B)=\left\{\begin{array}{l}
		a\big(\au^\Phi_\preceq(t^A_1, u^B_1),\dots, \au^\Phi_\preceq(t^A_n, u^B_n)\big) 
		\\ \qquad \mbox{if } a = b \mbox{ and } n = m 
		\\ \qquad \mbox{and } \forall i \in 1..n: \au^\Phi_\preceq(t^A_i, u^B_i) \neq\bot
		\\ \bot  
		\\ \qquad \mbox{otherwise}
	\end{array}\right.\]
\end{definition}

\begin{example}
	In Table~\ref{table:preceq}, we treat the anti-unification of the same atoms as above, this time with the use of $\au_\preceq^\Phi$ with $\Phi$ a given variabilization function. Note how $\au_\preceq$ behaves differently than $\au_\sqsubseteq$ on the second and third couple of atoms as it requires its arguments to exhibit a similar structure in order to be anti-unifiable.
\end{example}

\begin{table*}
	\caption{Example results for $au_\preceq^\Phi$}
	\label{table:preceq}
	\centering
	\begin{tabular}{l|l|l}
		%\hline
		$\bm{A_1}$ & $\bm{A_2}$ & $\bm{\au_\preceq^\Phi(A_1, A_2)}$\\\hline 
		$p(X, 5, q(Y,4))$ & $p(W,t(Z))$ & $\bot$\\\hline 
		$p(r(X,3), t(5))$ & $p(W, t(Z))$ & $\bot$\\\hline 
		$p(r(X,3), t(Y))$ & $p(r(W,3),t(Z))$ & $p(r(\Phi(X,W),3), t(\Phi(Y,Z)))$ %\\\hline
	\end{tabular}
\end{table*}

Now, in order to compute $\sqsubseteq$-msgs, we need a more precise anti-unification operator: one that goes deeper into detail when comparing atoms so as not to miss their maximal common structure. 
\begin{definition} %[Deep anti-unification operator]\label{def-deep-operator}
	Given some variabilization function $\Phi$, let $\dau^\Phi_\sqsubseteq$ (or simply $\dau_\sqsubseteq$ if $\Phi$ is clear from the context) denote the function such that for any two terms $T = t(t_1, \dots, t_n)$ and $U = u(u_1, \dots, u_m)$ it holds that 
	\[\dau^\Phi_\sqsubseteq(T,U)=\left\{\begin{array}{l}
		
		t\big(\dau^\Phi_\sqsubseteq(t_1,u_1), \dots, \dau^\Phi_\sqsubseteq(t_n, u_n)\big) 
		\\ \qquad \mbox{if } t = u \mbox{ and } n = m 
		\\ \qquad \mbox{and } T \notin \mathcal{V} \mbox{ and } U \notin \mathcal{V}
		\\ \Phi(T,U) 
		\\ \qquad \mbox{otherwise}
	\end{array}\right.\]
	and for any two atoms $A = a(t^A_1, \dots, t^A_n)$ and $B = b(u^B_1, \dots, u^B_m)$, it holds that
	\[\dau^\Phi_\sqsubseteq(A,B)=\left\{\begin{array}{l}
		a\big(\dau^\Phi_\sqsubseteq(t^A_1, u^B_1),\dots, \dau^\Phi_\sqsubseteq(t^A_n, u^B_n)\big) 
		\\ \qquad \mbox{if } a = b \mbox{ and } n = m 
		\\ \bot
		\\ \qquad \mbox{otherwise}
	\end{array}\right.\]
\end{definition}

When applied on atoms, it is easy to see that $\dau_\sqsubseteq$ is an anti-unification operator based on relation $\sqsubseteq$.

\begin{example}
	Let us once more consider the anti-unification of the atoms introduced in Example~\ref{ex-au-sq}. This time we make use of $\dau_\sqsubseteq^\Phi$ with $\Phi$ a given variabilization function, to anti-unify the three pairs of atoms. The result is shown in Table~\ref{table:dau}. Notice how the operator preserves as much non-variable atomic structure as possible in the process.
\end{example}
\begin{table*}
	\caption{Example results for $dau_\sqsubseteq^\Phi$}
	\label{table:dau}
	\centering
	\begin{tabular}{l|l|l}
		%\hline 
		$\bm{A_1}$ & $\bm{A_2}$ & $\bm{\dau_\sqsubseteq^\Phi(A_1, A_2)}$\\\hline 
		$p(X, 5, q(Y,4))$ & $p(W,t(Z))$ & $\bot$\\\hline 
		$p(r(X,3), t(5))$ & $p(W, t(Z))$ & $p(\Phi(r(X,3),W),t(\Phi(5,Z)))$\\\hline 
		$p(r(X,3), t(Y))$ & $p(r(W,3),t(Z))$ & $p(r(\Phi(X,W),3), t(\Phi(Y,Z)))$ %\\\hline 
	\end{tabular} 
\end{table*}
The existence of these operators proves Lemma~\ref{lemma-au-op}.
	\end{proof}
	
	\section{Proof of Theorem~\ref{thm-ausqsubseteq}}
	\begin{proof}
	Obviously the $\au_\sqsubseteq(A_1,A_2)$ operation can be achieved in a time linear with respect to the arity $n$ of $A_1$. In the worst case, the operation needs to be performed for each atom in $G_1$ with respect to each atom in $G_2$. Hence the first result.
	
	
	It is also easy to see that the $\au_\preceq(A_1,A_2)$ operation can be achieved in linear time with respect to the maximum number of function applications in the argument terms of the atom $A_1$ under scrutiny. In the worst case, the operation needs to be performed for each atom in $G_1$ with respect to each atom in $G_2$. Hence the second result.
	\end{proof}
		
%	
%	\section{Maximum Weight Matching of Example~\ref{example-mwm}}
%	See Fig.~\ref{fig:mwm}.
	
	
	\section{Proof of Theorem~\ref{thm-sqsubseteq-msg}}
\begin{proof}
	First note how the atomic anti-unifications and the weights of the associated bipartite graph's edges can be computed simultaneously, by working out $\dau_\sqsubseteq(A_1,A_2)$ for each possible couple $(A_1,A_2)$ in $G_1\times G_2$ and keeping account of the number of non-variable terms encountered during the operation (or $-1$). Given that $\dau_\sqsubseteq(A_1,A_2)$ can obviously operate linearly in the number of terms appearing in $A_1$ (denoted $N$), the computation of all weights is carried out in a time not exceeding $\mathcal{O}(|G_1|.|G_2|.N)$.
	
	Now the obtained assignment problem can be solved by existing algorithms (such as the Hungarian method~\cite{assignment}) that compute a MWM in $\mathcal{O}(n^3)$, where $n$ is the number of vertexes appearing on the side of the bipartite graph that has the most vertexes. In our case, there are $|G_1|$ left vertexes and $|G_2|$ right vertexes so that a MWM algorithm can be ran in $\mathcal{O}(max(|G_1|,|G_2|)^3)$.
\end{proof}
	
	\section{Proof of Theorem~\ref{thm-dataflow-np-complete}}
\begin{proof}
	First, let us consider MSG-MIN. It clearly belongs to NP. Indeed, given an arbitrary generalization $G$, we can verify in polynomial time whether it is a most specific generalization. The procedure is as follows. We can compute at least one $\leqslant$-msg, say $G'$, in polynomial time (see Theorem~\ref{thm-sqsubseteq-msg}). It suffices then to compare the $\tau$-value of $G'$ with that of $G$ in order to decide whether $G$ is a msg. Next, verifying whether the number of variables in $G$ is bounded by a constant is obviously achieved in polynomial time as well.
	
	In order to prove NP-hardness, we will construct a reduction from the well-known set cover problem (known to be NP-complete~\cite{karp}) to MSG-MIN. The set cover problem in its decision-problem version (denoted SCP), can be formulated as follows. Given a constant $p \in \mathbb{N}_0$, a universe $U$ of values and a collection $S$ composed of $n$ sets $\{S_1, \dots, S_n\}$ that cover $U$, i.e. $U = \underset{i=1}{\overset{n}{\cup}}S_i$, the problem is to decide whether there exists $p$ subsets from $S$ that still cover $U$.
	
	We can transform an arbitrary instance of SCP into MSG-MIN as follows. Let us consider without loss of generality a universe $U$ where the elements are lowercase strings and $p \in \mathbb{N}_0$ a constant. Given a collection of sets $S=\{S_1, \dots, S_n\}$ we construct an instance of MSG-MIN as follows. In our construction we use $n+1$ different variables, namely $V$ and $(W_i)_{i\in1..n}$. We use $x_j$ to denote some element of $U$; these elements being strings, we can easily use them as predicate names. The construction of goals $G_1$ and $G_2$ proceeds then as follows:
	
	\begin{algorithmic}
		\State $G_1 = \{\}$ 
		\State $G_2 = \{\}$ 
		\For {each ($S_i \in S$)}
		\For {each ($x_j \in S_i$)}
		\State $G_1 \gets G_1\cup \{x_j(V)\}$				
		\State $G_2 \gets G_2\cup \{x_j(W_i)\}$
		\EndFor
		\EndFor
	\end{algorithmic}
	Note that all the atoms in $G_1$ have the same argument (namely the variable $V$) and there are as many atoms in $G_1$ as there are distinct elements in $S$. In $G_2$, however, there is an atom of the form $x_j(W_i)$ for each element $x_j$ occurring in $S_i$.
	
	The construction is such that any $\leqslant$-msg of $G_1$ and $G_2$ will be a version of $G_1$ where each occurrence of a variable $V$ is replaced by $\Phi(V, W_k)$ for some $W_k\in\vars(G_2)$ (where $\Phi$ is a variabilization function). Now, introducing such a variable $\Phi(V, W_k)$ in the generalization will allow to reuse the same variable for all the atoms $x_j(V)$ in $G_1$ that have a corresponding $x_j(W_k)$ in $G_2$. In other words, choosing to have variable $\Phi(V,W_k)$ in the $\leqslant$-msg is the same as selecting the subset $S_k$ to be part of the solution of the set cover problem. Consequently, using this transformation MSG-MIN can be used to decide SCP. Since the transformation can clearly be done in polynomial time, and since SCP is known to be NP-complete, we conclude that MSG-MIN is NP-complete as well.
	
	Now let us prove the result for LCG-MIN. We know that a $\leqslant$-lcg can be computed in polynomial time, so that a positive instance of LCG-MIN can be verified just like it can be for MSG-MIN. Moreover, the absence of non-variable terms in the transformation from SCP to MSG-MIN above allows us to reuse said transformation as-is to prove that LCG-MIN is NP-hard. Indeed, since the obtained anti-unification problem doesn't harbor terms other than variables, it is both an instance of MSG-MIN and LCG-MIN. LCG-MIN is therefore also NP-complete.
\end{proof}
	
	\section{Proof of Theorem~\ref{thm-inj-np-complete}}
\begin{proof}
	INJ is in NP: given a relation $\leqslant^\iota$, goals $G_1$ and $G_2$ and a substitution (or renaming) $\theta$, it is possible to verify in polynomial time whether the application of $\theta$ on $G_1$ results on a subset of $G_2$ or not.
	As for the proof of NP-hardness, we refer to~\cite{gen} in which the problem ``is $G_1$ a $\preceq^\iota$-lcg of $G_1$ and $G_2$?'' has been proved to be NP-complete using a polynomial reduction from the Induced Subgraph Isomorphism Problem~\cite{SYSLO198291}. The same reduction can be used for the other cases, leading to the conclusion that INJ is NP-complete.
\end{proof}
	
%!TEX root = hopfwright.tex

%%%%%%%%%%%%%%%%%%
%%% Appendix A %%%
%%%%%%%%%%%%%%%%%%

\section{Appendix: Operator Norms}
\label{sec:OperatorNorms}
%%
%%\note[JB]{I think we should get rid of this proposition. It is not used anywhere (although there is a vague reference to it in the main text).}
%%
%%\begin{proposition} 
%%	\label{prop:ApproximateSolutionWorks}
%%Define $\bar{x}_{\epsilon}$ as in Definition \ref{def:xepsilon} and let $ x \in \ell^1_0$. 
%%If $ \| x - \bar{x}_\epsilon \| = \cO(\epsilon^2)$ then $F_\epsilon(x) = \cO(\epsilon^2)$. 
%%%%%
%%%%%	 \begin{eqnarray}
%%%%%	 \tilde{\alpha}( \epsilon) &:=& \pi /2 + \tfrac{\epsilon^2}{5} ( \tfrac{3\pi}{2} -1) \nonumber \\
%%%%%	 \tilde{\omega}( \epsilon) &:=& \pi /2 -  \tfrac{\epsilon^2}{5}  \nonumber \\
%%%%%	 \tilde{c}(\epsilon) 	  &:=& \{ \left(\tfrac{2 - i}{5}\right)  \epsilon^2 , 0,0, \dots \} \nonumber
%%%%%	 \end{eqnarray}
%%%%%Then $ \tilde{F}( \tilde{\alpha} (\epsilon) , \tilde{\omega}(\epsilon) , \tilde{c}(\epsilon) ) = \cO(\epsilon^3)$. 
%%\end{proposition}
%%
%%
%%\begin{proof}
%%	It suffices to prove the theorem centering our calculation around $ \{ \pp, \pp, \bar{c}_\epsilon \}$.  
%%	Since $ [ \bar{c}_\epsilon]_{k\geq 3 } =0$ and $ \|\bar{c}_\epsilon \| = \cO(\epsilon)$, then we may expand the function $F$ out  to order $ \cO(\epsilon^2)$ as follows: 
%%	\begin{eqnarray}
%%	\, [F(\alpha,\omega, c)]_1 &=& 
%%			i \omega + \alpha e^{-i \omega} + 
%%	\cO(\epsilon^2)  \\ 
%%	\, [F(\alpha,\omega, c)]_2 &=& 
%%			(  2 i \omega  + \alpha e^{ - 2 i \omega} ) c_2 + 
%%	\epsilon \alpha e^{-i \omega}  +
%%	\cO(\epsilon^2) \\ 
%%	\, [ F(\alpha,\omega, c) ]_{k \geq 3} &=&  
%%			\cO(\epsilon^2)
%%	\end{eqnarray}
%%	When  $ \{\alpha , \omega, c \} = \{ \pp, \pp , \bar{c}_{\epsilon} \}$ then both $0 = 	i \omega + \alpha e^{-i \omega} $ and $ 0 = (  2 i \omega  + \alpha e^{ - 2 i \omega} ) c_2 + 
%%	\epsilon \alpha e^{-i \omega}  $. 
%%	Hence, for any $ \| x - \{\pp,\pp, \bar{c}_{\epsilon}\} \| = \cO(\epsilon^2)$ it follows that $ F(x) = \cO(\epsilon^2)$. 
%%	
%%	
%%\end{proof}

%%%%	THIS IS THE OLD PROOF OF PROPOSITION A.1
%%%%
%%%%\begin{proof}
%%%%	Since $ c_{k\geq 3 } =0$ and $ \|c\| = \cO(\epsilon)$, then we may expand the function $F$ out  to order $ \cO(\epsilon^2)$ as follows: 
%%%%	\begin{eqnarray}
%%%%	\, [F(\alpha,\omega, c)]_1 &=& 
%%%%	i \omega + \alpha e^{-i \omega} + 
%%%%	\cO(\epsilon^2)  \\ 
%%%%	\, [F(\alpha,\omega, c)]_2 &=& 
%%%%	(  2 i \omega  + \alpha e^{ - 2 i \omega} ) c_2 + 
%%%%	\epsilon \alpha e^{-i \omega}  +
%%%%	\cO(\epsilon^2) \\ 
%%%%	\, [ F(\alpha,\omega, c) ]_{k \geq 3} &=&  
%%%%	\cO(\epsilon^2)
%%%%	\end{eqnarray}
%%%%	
%%%%	
%%%%	
%%%%	Hence, if we want to solve $ \tilde{F}(\alpha,\omega, c) = 0 + \cO( \epsilon^3)$, then $c_{k \geq 3} =0$. 
%%%%	To solve for $ \alpha, \omega,$ and $ c_2$, we rescale the first equation by $\epsilon$ and then attempt to solve the following system of equations.
%%%%	\begin{eqnarray}
%%%%	0 &=& 
%%%%	i \omega + \alpha e^{-i \omega} + 
%%%%	\alpha \left(e^{i \omega } + e^{-2 i \omega} \right)  c_2 \\
%%%%	0 &=& 
%%%%	( 2 i \omega  + \alpha e^{ - 2 i \omega} ) c_2 + 
%%%%	\epsilon^2 \alpha e^{-i \omega}  
%%%%	\end{eqnarray}
%%%%	Since we are solving for two real variables ($\alpha$ and $ \omega$) and one complex variable ($c_2$), if the equations are non-degenerate then there should be a unique solution. 
%%%%	We solve this equation by making the change of variables $ \alpha = \pp ( 1 + \alpha_1)$ and $ \omega = \pp ( 1 + \omega_1)$.
%%%%	Dividing through by $\pp$ in both equations and then linearizing in terms of $ \alpha_1, \omega_1 $ and $ c_2$ results in the following system of equations:
%%%%	\begin{eqnarray}
%%%%	0 &=&  -i \alpha_1 +(i - \pp) \omega_1  + (-1 + i) c_2 \\
%%%%	i \epsilon^2  &=&   - i \epsilon^2 \alpha_1 - \pp \epsilon ^2 \omega_1 + ( -1 + 2 i) c_2  
%%%%	\end{eqnarray}
%%%%	Separating this into real and imaginary parts, we obtain a system of four real equations and four real variables. 
%%%%	Solving this matrix equation then results in the desired  approximations. 
%%%%	
%%%%	
%%%%\end{proof}



% THIS PARAGRAPH CAN BE REMOVED TO REDUCE THE LENGTH
%  We evaluate the derivative in $ \tilde{x}_\epsilon = (\pi/2,\pi/2,c_2(\epsilon),0,0,\dots) $. The corrections of order $\epsilon^2$ in the $\alpha$- and $\omega$-component that are incorporated in $x_\epsilon$, see Definition~\ref{def:xepsilon}, are not needed here because we only need ????   then $ DF(x_\epsilon) + \cO(\epsilon^2)$ can be calculated to be:
%  	\begin{eqnarray}
%  	\frac{\partial F}{\partial  \alpha} \left(\tilde{x}_\epsilon \right)
%  	&=&
%  	[- i]_1 + [ -\left(\tfrac{2 - i}{5}\right) \epsilon]_2 + [-i \epsilon]_2 +  cO(\epsilon^2) \\
%  	%
%  	\frac{\partial F}{\partial  \omega}  (\tilde{x}_\epsilon)
%  	&=&
%  	[i - \tfrac{\pi}{2}]_1 +
%  	\epsilon [ (2 + \pi) ( \tfrac{1+ 2 i}{5})- \pp ]_2 + \cO(\epsilon^2) \\
%  	%
%  	\frac{\partial F}{\partial  c}  (\tilde{x}_\epsilon)
%  	&=&
%  	\tfrac{\pi}{2} ( i K^{-1} + U_{\omega_0} + \epsilon L_{\omega_0}) + cO(\epsilon^2)
%  	\end{eqnarray}
%  	Here, $\omega_0 = \omega(0) = \pi/2$.
%  	These equations are used to define $ A = A_0 + \epsilon A_1$, whence $A = F( x_\epsilon) + \cO(\epsilon^2)$.
%  	 By our choice of $\epsilon$ we know that  $ \epsilon\|  A_1 A_0^{-1}\| < $, so we can write the power series expansion of $A^{-1} = A_0^{-1} ( I + epsilon A_1 A_0^{-1})$ as follows
%  	 \[
%  	  A^{-1} = A_0^{-1} \sum_{k=0}^{\infty} \left( - \epsilon A_1 A_0^{-1} \right)^k
%  	 \]
%  	 If we truncate this power series and define $ A^{\dagger } := A_0^{-1} - \epsilon A_{0}^{-1} A_1 A_0^{-1}$, then it follows that $ A^{\dagger} = A^{-1} + \cO(\epsilon^2)$.
%  	 Thus, we have proven that $ A^{\dagger} = [DF(x_\epsilon)]^{-1} + \cO(\epsilon^2)$.
%
% \end{proof}

%
% \begin{proposition}
% 	Fix $ \epsilon \geq 0$ and suppose that  $ | \alpha - \pp| < r_\alpha$ and $| \omega - \pp| < r_\omega$ and  $\frac{1+4 \epsilon}{2} <\frac{\omega }{\alpha }$ and  define
% 	\[
% 	b_* = 2  \frac{\pp - r_{\omega}}{\pp + r_\alpha} -1 - \epsilon  (4/3+\sqrt{2 + 2 r_\omega } )
% 	\]
% 	and
% 	\[
% 	z^{\pm}_* =\frac{b_* \pm \sqrt{(b_*)^2- 4 \epsilon^2 }}{2 } .
% 	\]
% 	If there exists for some $ c \in \ell^1 / \C$ such that $\tilde{F}(\alpha, \omega,c) = 0$, then either $ |c| \leq  z_*^-$ or $ z_*^+ \leq |c| $.
%
% 	\noindent
% 	Additionally, $ \| K^{-1} c \| < 2 (\epsilon^2+ \|c\|^2)/ b_*$.
% 	\label{prop:Cone}
% \end{proposition}
%
%
% \begin{proof}
%
% 	Let us define the linear operator $B: \ell^1 / \C \to \ell^1 / \C$ by
% 	\[
% 	B = i \omega K^{-1} + \alpha U_{\omega} + \alpha \epsilon L_{\omega}.
% 	\]
% 	If  $ \tilde{F}( \alpha, \omega, c) =0$ then it follows that for the equations $\tilde{F}_{k\geq 2}$ we have
% 	\begin{eqnarray}
% 	0 &=& \tilde{F}(\alpha , \omega, c) \\
% 	0 &=& [ \epsilon^2 \alpha e^{- i \omega}]_2 + Bc  + \alpha [ U_{\omega} c ] * c \\
% 	- B c &=& [ \epsilon^2 \alpha e^{- i \omega}]_2 + \alpha [ U_{\omega} c ] * c \\
% 	c &=& - \alpha B^{-1} ( [ \epsilon^2  e^{- i \omega}]_2 + \ [ U_{\omega} c ] * c )
% 	\end{eqnarray}
% 	Taking norms, we obtain the following:
% 	\begin{eqnarray}
% 	|c | & \leq & \alpha \| B^{-1}\| \left( \epsilon^2  + \| [U_\omega c] * c \| \right)  \\
% 	(\alpha \| B^{-1}\|)^{-1} |c | & \leq &  ( \epsilon^2 + |c |^2) \\
% 	0 & \leq & |c|^2 - (\alpha \| B^{-1}\|)^{-1} |c| +  \epsilon^2
% 	\end{eqnarray}
% 	Let us define $ b = ( \alpha \| B^{-1} \|)^{-1}$
% 	The above quadratic has two zeros $z^+$ and $ z^-$ given by
% 	\[
% 	z^{\pm} =\frac{b \pm \sqrt{b^2- 4 \epsilon^2 }}{2 }
% 	\]
% 	These zeros have the property that either $ |c| \leq z^-$ or $ |c| \geq z^+$
% 	We calculate $ \| B^{-1}\|$.
% 	Since $ \frac{1+4 \epsilon}{2} <\frac{\omega }{\alpha }$ then $\| \frac{\alpha}{i \omega} (U_{\omega} + \epsilon L_{\omega})K\| <1$ and we can expand $B^{-1}$ using a geometric series.
% 	First we evaluate $B^{-1}$.
% 	\begin{eqnarray}
% 	B 	&=& i \omega K^{-1} + \alpha U_{\omega} + \alpha \epsilon L_{\omega} \\
% 	&=& i \omega \left[I + \frac{\alpha}{i \omega} (U_{\omega} + \epsilon L_{\omega}) K \right] K^{-1} \\
% 	B^{-1} &=& \frac{1}{i \omega } K \left[I + \frac{\alpha}{i \omega} (U_{\omega} + \epsilon L_{\omega})K \right]^{-1} \\
% 	&=& \frac{1}{i \omega } K  \sum_{n=0}^\infty \left( \frac{ - \alpha}{i \omega} \right)^n [(U_{\omega} + \epsilon L_{\omega})K]^n
% 		\end{eqnarray}
% 		We now calculate $\| B^{-1} \|$.
% 		\begin{eqnarray}
% 	\| B^{-1} \| &\leq &  \frac{\| K\|}{ \omega }   \sum_{n=0}^\infty \left( \frac{ \alpha}{ \omega} \right)^n \|(U_{\omega} + \epsilon L_{\omega}) K\|^n \\
% 	&=& \frac{\| K \|}{\omega  - \alpha \|(U_{\omega} + \epsilon L_{\omega}) K\|} \\
% 	&\leq & \frac{\| K \|}{\omega  - \alpha ( \| K \| + 2 \epsilon \| \sigma^+ K \| + 2 \epsilon \|\sigma^- K\|)} \\
% 	&\leq & \frac{1/2}{\omega  - \alpha (1/2+ 2\epsilon( 1/2 + 1/3) )} \\
% 	&=& \frac{1}{2 \omega  - \alpha (1 + \tfrac{10}{3}\epsilon)}.
% 	\end{eqnarray}
% 	Thereby, we have obtained the inequality $b \geq 2 \tfrac{\omega}{\alpha} - 1 - \tfrac{10}{3} \epsilon $.
%
% 	We can further improve this constant with the following observation.
% 	The norm of $ ( e^{-i \omega } I + U_{\omega}) K$ is concentrated in the first equation.
% 	One calculates that $ | [e^{-i \omega } I + U_{\omega} ]_2 |=  | e^{- i \omega} + e^{-2 i \omega}| = \sqrt{2 - 2 \sin (\omega-\pp) } $. So then $ \|\sigma^+ ( e^{-i \omega } I + U_{\omega}) K\| \leq  \sqrt{2 (1+ r_\omega) } /2$.
% 	Consequently $	\| B^{-1} \| \leq  [ 2 \omega - \alpha ( 1 + \epsilon (4/3+\sqrt{2 + 2 r_\omega } ))]^{-1}$.
% 	We can define a lower approximation to $b$ as follows:
% 	\[
% 	b_* = 2  \frac{\pp - r_{\omega}}{\pp + r_\alpha} -1 - \epsilon  (4/3+\sqrt{2 + 2 r_\omega } )
% 	\]
% 	If we calculate the derivative of $ z^{\pm}$ with respect to $b$, we find that $\frac{\partial }{\partial b} z^+>0 $ and $ \frac{\partial }{\partial b} z^-<0 $.
% 	To minimize $ z^+$ and maximize $z^-$ let us then define approximations to $ z^{\pm}$ as
% 	\[
% 	z^{\pm}_* =\frac{b_* \pm \sqrt{(b_*)^2- 4 \epsilon^2 }}{2 } .
% 	\]
% 	It then follows that either $ |c| \leq  z_*^-$ or $ z_*^+ \leq |c| $.
%
% 	Furthermore, since $\| K^{-1} c \| \leq  \alpha \| K^{-1} B^{-1} \| \, ( \epsilon^2 + \|c \|^2)$ and $ \| K^{-1} B^{-1} \| \leq ( \omega - \alpha \| ( U_\omega + \epsilon L_\omega ) K \|)^{-1}$, then it follows from our calculation of $ b_*$ that $ \| K^{-1} c \| < 2 (\epsilon^2+ \|c\|^2)/b_*$.
% \end{proof}



% Below, we make some definitions and prove some small propositions to assist with future calculations.
We set $\omega_0 = \pp$ and recall that 
\begin{alignat*}{1}
	[U_\omega a]_k & =  e^{-i k\omega} a_k \\
	[U_{\omega_0} a]_k & = (-i)^k a_k \\
	L_{\omega}  & =  \sigma^+( e^{-i\omega}  I + U_{\omega}) + \sigma^-( e^{i\omega} I + U_{\omega})  \\
	L_{\omega_0} & = \sigma^+( -i  I + U_{\omega_0}) + \sigma^-( i I + U_{\omega_0})  .
\end{alignat*}
% For future reference we compute $L_{\omega_0}$ where, as in Definition \ref{def:A}, we take $ \omega_0 = \pp$.
% \begin{eqnarray}
% L_{\omega_0} &=& \sigma^+( e^{- i \pp} I + U_{\omega_0}) + \sigma^-(e^{i \pp} I + U_{\omega_0}) \\
% &=& \sigma^+( -i  I + U_{\omega_0}) + \sigma^-( i I + U_{\omega_0})
% \end{eqnarray}
To more efficiently express the inverse of $ A_{0,*}$ we define an operator $\hat{U}: \ell^1_0 \to \ell^1_0 $ by
%
% \begin{definition}
% Define the map  $ \hat{U} : \ell^1_0 \to \ell^1_0 $ by:
\begin{equation}\label{e:defUhat}
[\hat{U} c]_{k\geq 2} := (1 - i k^{-1}e^{-i k \pi /2} )^{-1} c_k,
\end{equation}
so that  
$ A_{0,*}^{-1}=  \frac{2}{ i \pi } \hat{U} K $.
% \end{definition}
%
%
% \begin{proposition}
% The inverse of $ A_{0,*}$ is given by $A_{0,*}^{-1}=  \frac{2}{ i \pi } \hat{U} K $.
% \end{proposition}

% \begin{proof}
% 	The map $A_{0,*}$ is a diagonal operator and its inverse can be calculated as follows:
% 	\begin{eqnarray*}
% 	A_{0,*}^{-1} &=& \frac{2}{ \pi } (i K^{-1} + U_{\omega_0})^{-1} \\
% 	&=& \frac{2}{  \pi } [  (iK^{-1}) (I -i K U_{\omega_0}) ]^{-1} \\
% 	&=& \frac{2}{ i \pi } (I - i U_{\omega_0}K )^{-1} K   \\
% 	&=& \frac{2}{ i \pi } \hat{U} K
% 	\end{eqnarray*}
% \end{proof}
%

The operator norm of  $Q \in B(\ell^1_0,\ell^1)$ can be expressed using the basis elements $\e_k$ (which have norm $\|\e_k\|=2$):
\begin{equation}\label{e:operatornorm}
  \| Q \| = \frac{1}{2} \sup_{k \geq 2} \|Q \e_k\| .
\end{equation}
Some of the operators in $B(\ell^1_0,\ell^1)$ considered in these appendices restrict naturally to $B(\ell^1_0)$, with the same expression for the norm. For operators in $B(\ell^1)$ a similar expression for the norm holds (the supremum being over $k\geq 1$). We will abuse the notation $\|Q\|$ by not indicating explicitly which of these operator norms is considered; this will always be clear from the context.
%\note[JB]{This is not very clean of course, but I am not inclined to go over the paper and put subscripts everywhere to make all these distinctions.}
\begin{proposition}\label{p:severalnorms}
	The operators $\hat{U}, \hat{U} K, L_{\omega}, A_{0,*}^{-1}   $ and $A_{1,*}$ in $B(\ell^1_0,\ell^1)$  satisfy the bounds
\begin{align*}
\| \hat{U} \| 		=& \tfrac{5}{4} 						&\| A_{0,*}^{-1} \| =& \tfrac{2}{ \pi \sqrt{5}}	\\ 
\| \hat{U} K \| 	=& \tfrac{1}{ \sqrt{5}}	&\| A_{1,*} \| \leq& 2 \pi	 \\
\| L_{\omega} \| \leq& 4
\end{align*}
\end{proposition}
\begin{proof}
		The value $\| \hat{U} \e_k \|$ is maximized when $k=5$, whence  $\| \hat{U} \| = 5/4$. 
		The value $\| \hat{U} K \e_k \|$ is maximized when $k=2$, whence $\| \hat{U} K \| 	= \frac{1}{ \sqrt{5}}$ and $\| A_{0,*}^{-1}\| = \frac{2}{\pi \sqrt{5}} $. 
It follows from the definition of $L_\omega$ and the fact that $U_\omega$ is unitary that $ \| L_{\omega} \| \leq 4$, whereby it follows that  
$ \| A_{1,*} \| = \| \pp L_{\omega_0} \|  \leq 2 \pi$.
\end{proof}

We recall, for any $a\in \ell^1$, the splitting $a=a_1 \e_1 + \tilde{a}$ with $a_1 \in \C$ and $\tilde{a} \in \ell^1_0 $,  and as a tool in the estimates below we  introduce the projections 
\begin{alignat}{1}
	\pi_1 a &= a_1  \in \C \\
	\pi_{\geq 2} a & = \tilde{a}. \label{e:pige2}
\end{alignat}

\begin{proposition}
	\label{prop:A1A0}
	We have for the map $ A_1 A_0^{-1} : \ell^1 \to \ell^1$ that 
%\note[JB]{I don't think we should have a strict inequality here.}
	\begin{equation}
	\label{eq:A1A0}	
	\| A_1 A_0^{-1}\| = \frac{2 \sqrt{10}}{5} \, .
	\end{equation}
%\note[JB]{Perhaps the estimate is now
%$ \| A_1 A_0^{-1}\| = \max\{ \frac{1}{5}\sqrt{\frac{45+5\sqrt{17}}{2}}, \frac{2 \sqrt{10}}{5} \}  = \frac{2 \sqrt{10}}{5}$ ?}
\end{proposition}

\begin{proof}
%%	Expanding $A_1A_{0}^{-1}$  we obtain 
%%\note[JB]{This expansion uses imaginative notation. I propose to remove it.}
%%\begin{eqnarray}
%%A_1 A_0^{-1} &=&  (  A_{1,2} + A_{1,*})(  A_{0,1}^{-1} + A_{0,*}^{-1} ) \\
%%&=&   A_{1,2} A_{0,1}^{-1}  +  A_{1,*}     A_{0,*}^{-1} 
%%\end{eqnarray}
%%\note[JB]{I propose to replace it by the following}
Expanding $A_1A_{0}^{-1}$ we see that it splits into two parts: $A_{1,2} A_{0,1}^{-1}$ and $A_{1,*}     A_{0,*}^{-1}$, which we estimate separately. To be precise
\[
  A_1A_{0}^{-1} a = (i_\C A_{1,2} A_{0,1} i_\C^{-1} \pi_1 a) \e_2 
                    +  A_{1,*} A_{0,*}^{-1} \pi_{\ge 2} a.
\]
First, we calculate the matrix
\[
  A_{1,2} A_{0,1}^{-1}  = 
   \frac{1}{5}
  \left[
  \begin{matrix}
  3 & 2 \\
  -4  & 4 
  \end{matrix} 
  \right] .
\]
Using the identification of $\R^2$ and $\C$, which is an isometry if one uses the $2$-norm on $\R^2$,
this matrix contributes to $A_1 A_0^{-1}$
as an operator mapping the (complex) one-dimensional subspace spanned by~$\e_1$ to the (complex) one-dimensional subspace spanned by~$\e_2$. 
To determine its contribution to the estimate of the norm of $A_1 A_0^{-1}$,
we thus need to determine the $2$-norm of the matrix (as a linear map from $\R^2 \to \R^2$):
\[
  \| A_{1,2} A_{0,1}^{-1} \|  = \frac{1}{5} \sqrt{\frac{45+5\sqrt{17}}{2}}.
\]
%\note[JB]{Explained and improved the bound. Still needs to be reflected in the statement of the Proposition though.}
% We calculate the map $A_{1,2} A_{0,1}^{-1} :\{e_1\}   \to \{ \alpha , \omega \} \to \{e_2\} $
% \begin{eqnarray}
% A_{1,2} A_{0,1}^{-1} 		&=&
% \frac{1}{5}
% \left[
% \begin{matrix}
% -2 & 2-\tfrac{3 \pi}{2} \\
% -4  & 2(2+\pi)
% \end{matrix}
% \right]
% \cdot
% \left[
% \begin{matrix}
% 0 & - \pp \\
% -1  & 1
% \end{matrix}
% \right]^{-1}
% \\
% &=&  \frac{1}{5}
% \left[
% \begin{matrix}
% 3 & 2 \\
% -4  & 4
% \end{matrix}
% \right] \\
% \| A_{1,2} A_{0,1}^{-1} \| &\leq& \frac{8}{5} \label{eq:AppendixCheck}.
% \end{eqnarray}
% \marginpar{JJ: todo - Is the estimate in \ref{eq:AppendixCheck} right/ could it be sharpened?}
Next, we calculate a bound on the map $ A_{1,*}     A_{0,*}^{-1}: \ell^1_0 \to \ell^1$:
\begin{equation}\label{eq:LUK}
  \| A_{1,*}     A_{0,*}^{-1} \| =  \| L_{\omega_0} \hat{U} K \| .
\end{equation}
% \begin{eqnarray}
% A_{1,*}     A_{0,*}^{-1} 	&=&  \frac{\pi}{2} L_{\omega_0} \frac{2}{ i \pi } \hat{U} K  \\
% \| A_{1,*}     A_{0,*}^{-1} \| &= & \| L_{\omega_0} \hat{U} K \|  \label{eq:LUK}
% \end{eqnarray}
To bound \eqref{eq:LUK} we first compute how $L_{\omega_0} K \hat{U}    $ operates on basis elements $\e_k$ for $k\geq 2$: 
\[
L_{\omega_0} K \hat{U}     \e_{k} = \frac{ -i+(-i)^k }{k-i (-i)^{k}}   \e_{k+1}
+
\frac{ i+(-i)^k  }{k-i (-i)^{k}}   \e_{k-1} .
\]
Since the norm of this expression is maximized when $k=2$ and $  \| L_{\omega_0} K \hat{U}    \e_2 \| = \tfrac{4\sqrt{10}}{5}$,
%\note[JB]{Isn't that $\tfrac{4\sqrt{10}}{5}$? (norm below still correct)}
 we have calculated the $B(\ell^1_0,\ell^1)$ operator norm $ \|L_{\omega_0} K \hat{U}      \| = \tfrac{2\sqrt{10}}{5}$. 
%\change[J]{By combining the estimates above and using the triangle inequality, we arrive at the asserted bound.}{ 
As $\|A_1A_{0}^{-1}\|$ is equal to the maximum of  $ \| A_{1,2} A_{0,1}^{-1}\|$ and $\|A_{1,*}     A_{0,*}^{-1}\|$, it follows that 
	$ \| A_1 A_0^{-1}\| = \max\{ \frac{1}{5}\sqrt{\frac{45+5\sqrt{17}}{2}}, \frac{2 \sqrt{10}}{5} \}  = \frac{2 \sqrt{10}}{5}$.
%
% \begin{equation*}
% \| A_1 A_0^{-1}\|  \leq  \|  A_{1,2} A_{0,1}^{-1} 	 \| + \| A_{1,*}     A_{0,*}^{-1} \|  \frac{2}{5} \left(4+\sqrt{10}\right)
% \end{equation}
\end{proof}

\begin{proposition}
		\label{prop:A0A1}
		Define $\overline{A_0^{-1} A_1 } \in \text{\textup{Mat}}((\R^3,\R^3)$ by
	\[
\overline{A_0^{-1} A_1 } :=
\left(
\begin{array}{ccc}
0 & 0 & \tfrac{1}{2}\sqrt{2+\frac{\pi ^2}{2}} \\
0 & 0 & \frac{1}{\sqrt{2}}  \\
\frac{8}{5 \pi } & \frac{2\sqrt{16+8 \pi +5 \pi ^2}}{5 \pi } & \frac{2}{\sqrt{5}} \\
\end{array}
\right)
	\] 
%	\note[J]{Reflected changes in the statement of the proposition.}
	Then $\overline{A_0^{-1} A_1 } $ is an upper bound (as defined in Definition~\ref{def:upperbound}) for $A_0^{-1} A_1 $.
\end{proposition}
\begin{proof}
%%We expand $  A_0^{-1} A_1 $ as follows: 
%%\begin{eqnarray}
%%A_0^{-1} A_1 &=& ( A_{0,1}^{-1} + A_{0,*}^{-1}) ( A_{1,2} + A_{1,*}) \\
%%&=&  A_{0,*}^{-1}  A_{1,2} +  A_{0,1}^{-1} A_{1,*} + A_{0,*}^{-1} A_{1,*}
%%\end{eqnarray}
%%\note[JB]{Although the splitting is morally correct, the notation makes no sense to me. I propose to replace it by what is below.}
We write $x=(\alpha,\omega,c)$.
Let $\pi_{\alpha,\omega}$ be the projection onto $\R^2$, whereas $\pi_c$ is the projection onto $\ell^1_0$. 
Then we can expand $  A_0^{-1} A_1 $ as follows:
\begin{alignat}{1}
  \pi_{\alpha,\omega} A_0^{-1} A_1 x &=   A_{0,1}^{-1}  i_\C^{-1} \pi_{1} A_{1,*} \pi_c x  \label{e:complicated1} \\
  \pi_{c} A_0^{-1} A_1 x  &= 
  A_{0,*}^{-1} (( i_\C A_{1,2} \pi_{\alpha,\omega} x)   \e_2 ) + A_{0,*}^{-1} \pi_{\geq 2} A_{1,*} \pi_c x . \label{e:complicated2}
\end{alignat}
We estimate the three operators that appear separately.

First, we note that the term $A_{0,*}^{-1} (( i_\C A_{1,2} \pi_{\alpha,\omega} x)   \e_2 ) $ in~\eqref{e:complicated2} essentially represents an operator from $\R^2$ to the (complex) one-dimensional subspace spanned by $\e_2$. Using the identification of $\C$ with $\R^2$, this map is represented by the matrix 
\[
  \frac{-2 }{25 \pi }
  \left[
  \begin{matrix}
  1 & -2 \\
  2 & 1
  \end{matrix} 
  \right] 
  \cdot
  \left[
  \begin{matrix}
  -2 & 2-\tfrac{3 \pi}{2} \\
  -4  & 2(2+\pi) 
  \end{matrix} 
  \right] \\
  = \frac{2 }{25 \pi }
  \left[
  \begin{matrix}
  -6 & 6+ 11 \pp \\
  8  & \pi -8
  \end{matrix} 
  \right] .
\]
It then follows that 
%\note[J]{Added factor $2$ and changed $r_\alpha \mapsto |\alpha|$ and $r_\omega \mapsto |\omega|$}
\begin{alignat*}{1}
  \| A_{0,*}^{-1} (( i_\C A_{1,2} \pi_{\alpha,\omega} x)   \e_2 ) \|
&\leq 	\frac{4}{25 \pi} 
	\left(|\alpha| \sqrt{(-6)^2+8^2}   + 
	 |\omega|  \sqrt{( 6+ 11 \pp )^2 + (\pi-8)^2} 
	 \right) \\
	&= \frac{4}{5 \pi} \left( 2 |\alpha| + 
	\frac{\sqrt{16+8 \pi +5 \pi ^2}}{2} |\omega| \right) .
\end{alignat*}

% \begin{eqnarray*}
% A_{0,*}^{-1}  A_{1,2} &=&
% \frac{-2 }{25 \pi }
% \left[
% \begin{matrix}
% 1 & -2 \\
% 2 & 1
% \end{matrix}
% \right]
% \cdot
% \left[
% \begin{matrix}
% -2 & 2-\tfrac{3 \pi}{2} \\
% -4  & 2(2+\pi)
% \end{matrix}
% \right] \\
% &=& \frac{2 }{25 \pi }
% \left[
% \begin{matrix}
% -6 & 6+ 11 \pp \\
% 8  & \pi -8
% \end{matrix}
% \right]
% \end{eqnarray*}
% It then follows that
% \begin{eqnarray*}
% 	A_{0,*}^{-1}  A_{1,2} &\leq&
% 	\frac{2}{25 \pi}
% 	\left(r_\alpha \sqrt{(-6)^2+8^2}   +
% 	 r_\omega  \sqrt{( 6+ 11 \pp )^2 + (\pi-8)^2}
% 	 \right) \\
% 	&=& \frac{2}{5 \pi} \left( 2 r_\alpha +
% 	\frac{\sqrt{16+8 \pi +5 \pi ^2}}{2} r_\omega \right)
% \end{eqnarray*}
%
%
%\[
%A_{0,*}^{-1}  A_{1,2} \leq \frac{2}{5 \pi} \left( 2 r_\alpha + 
%\frac{\sqrt{16+8 \pi +5 \pi ^2}}{2} r_\omega \right) 
%\]
%
%

Next, we note that the term $A_{0,1}^{-1}  i_\C^{-1} \pi_{1} A_{1,*} \pi_c x $
in~\eqref{e:complicated1} essentially represents an operator from the (complex) one-dimensional subspace spanned by $\e_2$ to $\R^2$. Using the identification of $\C$ with $\R^2$, this map is represented by the matrix 
\[
 \pp 
 \left[
 \begin{matrix}
 0 & - \pp \\
 -1  & 1 
 \end{matrix} 
 \right]^{-1}
 \cdot
 \left[
 \begin{matrix}
 -1 & -1 \\
 1 &-1 
 \end{matrix} 
 \right] \\
 =
 \left[
 \begin{matrix}
 1 - \pp & 1 + \pp \\
 1  & 1
 \end{matrix} 
 \right] ,
\]
because $\pi_1 A_{1,*} \e_2 = \pp (i-1)$.
% Noting that $A_{0,1}^{-1}  A_{1,*} :\{ e_2 \} \to \{ \alpha , \omega\} $, we calculate:
% \begin{eqnarray}
% A_{0,1}^{-1}  A_{1,*} &=& A_{0,1}^{-1} [ \pp ( e^{ i \omega_0} +e^{-2 i \omega_0} )] \\
% &=& A_{0,1}^{-1} [ \pp ( i-1 )] \\
% &=& \pp
% \left[
% \begin{matrix}
% 0 & - \pp \\
% -1  & 1
% \end{matrix}
% \right]^{-1}
% \cdot
% \left[
% \begin{matrix}
% -1 & -1 \\
% 1 &-1
% \end{matrix}
% \right] \\
% &=&
% \left[
% \begin{matrix}
% 1 - \pp & 1 + \pp \\
% 1  & 1
% \end{matrix}
% \right]
% \end{eqnarray}
Hence
%\note[J]{Added factor of $\frac{1}{2}$.}
\begin{alignat*}{1}
  | \pi_\alpha  A_0^{-1} A_1 x |  &\leq  \tfrac{1}{2} \sqrt{ 2 + \tfrac{\pi^2}{2}}  \|c\| \\ 
  | \pi_\omega  A_0^{-1} A_1 x |  &\leq  \tfrac{1}{2}  \sqrt{2} \|c\|  . 
\end{alignat*}

% If we convert this into  $ \{ e_2 , \overline{ e_2} \}$ coordinates, then it follows that
% \[
% | \{ \alpha , \omega \} \cdot  A_{0,1}^{-1}  A_{1,*} \cdot c_2 | \leq
% \left\{ \sqrt{ 2 + \tfrac{\pi^2}{2}} , \sqrt{2} \right\}
% \]
%
%
%
 
Finally, note that the term $A_{0,*}^{-1}  \pi_{\geq 2} A_{1,*}$
appearing in~\eqref{e:complicated2} maps $\ell^1_0$ to itself. It can be expressed as
\[
A_{0,*}^{-1}  \pi_{\geq 2} A_{1,*}  = - i K \hat{U} \pi_{\geq 2} L_{\omega_0} .
\]
The operator $K \hat{U} \pi_{\geq 2} L_{\omega_0}$
 acts on basis elements $\{ \e_k\}_{k \geq 2}$ as follows:
\begin{alignat*}{1}
	K \hat{U} \pi_{\geq 2}  L_{\omega_0}  \e_2 &= -\frac{1+i}{4} \e_{3}  \\
	K \hat{U} \pi_{\geq 2}  L_{\omega_0}  \e_{k} &= \frac{-i+(-i)^k}{(k+1)-i (-i)^{k+1}} \e_{k+1} + \frac{i+(-i)^k}{(k-1) - i (-i)^{k-1}} \e_{k-1}
	\qquad\text{for } k \geq 3.
\end{alignat*}
Since $\max_{k\geq 2 } \| K \hat{U} \pi_{\geq 2} L_{\omega_0} \e_k \| = \| K
\hat{U} L_{\omega_0} \e_3 \| = \tfrac{4}{\sqrt{5}}$, 
the operator norm of $A_{0,*}^{-1}  \pi_{\geq 2} A_{1,*}$ is $\tfrac{2}{\sqrt{5}}$.

These three bounds on the three operators appearing in~\eqref{e:complicated1} and~\eqref{e:complicated2} lead to the asserted upper bound.
%
% \begin{eqnarray}
% A_{0,*}^{-1}  A_{1,*}  &=& \left( \frac{2}{i \pi } K \hat{U} \right) \left( \frac{\pi}{2} L_{\omega_0}\right) \\
% &=& - i K \hat{U} L_{\omega_0}
% %&=& - i K \hat{U} \left(  \sigma^+ \left( - i I + U_{\omega_0} \right)  + \sigma^- \left(  i I + U_{\omega_0} \right) \right)
% %
% %\| A_{0,*}^{-1}  A_{1,*} \|  &\leq& \| K \hat{U}   L_{\omega_0} \| \\
% %&\leq& \| K \hat{U} \sigma^+ \left( - i I + U_{\omega_0} \right) \| + \| K \hat{U} \sigma^- \left(  i I + U_{\omega_0} \right) \|\\
% %&\leq & \frac{1}{2 \sqrt{2}}+ \frac{2}{\sqrt{17}} .
% \end{eqnarray}
% Noting that $e_1$ is not in the domain of $\hat{U}$, we compute how $K \hat{U}   L_{\omega_0} $ operates on basis elements $\e_k$.
% \begin{eqnarray}
% 	K \hat{U}   L_{\omega_0}  e_2 &=& -\frac{1+i}{4} e_{3} \\
% 	K \hat{U}   L_{\omega_0}  e_{k\geq 3} &=& \frac{-i+(-i)^k}{(k+1)-i (-i)^{k+1}} e_{k+1} + \frac{i+(-i)^k}{(k-1) - i (-i)^{k-1}} e_{k-1}
% \end{eqnarray}
% Since $\max_{k\geq 2 } \| K \hat{U}   L_{\omega_0}  e_k \|_{\ell^1_0} =  \| K \hat{U}   L_{\omega_0}  e_3 \|_{\ell^1_0} = \tfrac{2}{\sqrt{5}}$, then we have calculated the $\ell^1_0$ operator norm  $ \| K \hat{U}   L_{\omega_0} \| = \tfrac{2}{\sqrt{5}}$.
% Hence, we obtain
% \[
%  A_0^{-1} A_1 \cdot r \leq
% \left(
% \begin{array}{ccc}
% 	0 & 0 & \sqrt{2+\frac{\pi ^2}{2}} \\
% 	0 & 0 & \sqrt{2} \\
% 	\frac{4}{5 \pi } & \frac{\sqrt{16+8 \pi +5 \pi ^2}}{5 \pi } & \frac{2}{
% 		\sqrt{5}}
% \end{array}
% \right) \cdot
% \left(
% \begin{array}{c}
% r_{\alpha } \\
% r_{\omega } \\
% r_c \\
% \end{array}
% \right)
% \]
\end{proof}

%%%%%%%%%%%%%%%%%%%%%%%%%%%%%%%%%%%%%%%%%%%%%%%%%%%%%%%%%%%%%%%%%%%%%%%%


%!TEX root = hopfwright.tex

%%%%%%%%%%%%%%%%%%
%%% Appendix B %%%
%%%%%%%%%%%%%%%%%%

\section{Appendix: Endomorphism on a Compact Domain}
\label{sec:CompactDomain}



In order to construct the Newton-like map $T$ we defined operators $ A =  DF(\bar{x}_\epsilon) + \cO(\epsilon^2)$ and $A^{\dagger} = A^{-1} + \cO(\epsilon^2)$. 
However, as $(\bar{\alpha}_\epsilon,\bar{\omega}_\epsilon,\bar{c}_\epsilon) = (\pp,\pp,\bar{c}_\epsilon) + \cO(\epsilon^2)$,  the map $A$ can be better thought of as an $\cO(\epsilon^2)$ approximation of $DF(\pp,\pp,\bar{c}_\epsilon)$. 
Thus, when working with the map $T$ and considering points $ x \in  B_\epsilon(r,\rho)$ in its domain, we will often have to measure the distances of $ \alpha$ and $ \omega $ from $ \pp$. 
To that end, we define the following variables which will be used throughout the rest of the appendices. 
\begin{definition}
	\label{def:DeltaDef}
For $ \epsilon \geq 0$, and $r_\alpha,r_\omega,r_c >0$ we define 
\begin{alignat*}{2}
	\da^0 	&:= \tfrac{\epsilon^2}{5} ( 3 \pp -1) & \qquad\qquad
	\da 	&:= \da^0 + r_\alpha \\
	\dw^0 &:=  \tfrac{\epsilon^2}{5} &
	\dw &:=  \dw^0 + r_{\omega} \\ 
	\dc^0 &:=  \tfrac{2 \epsilon}{\sqrt{5}} &
	\dc &:=  \dc^0 + r_c . 
	% \\
	% \dt^0  &:= \dw^0 + \tfrac{1}{2} (\dw^0)^2 &
	% \dt  &:= \dw + \tfrac{1}{2} \dw^2 \\
	% \dtt^0  &:= 2 \dw^0 + \tfrac{1}{2} (2\dw^0)^2 &
	% \dtt  &:= 2 \dw + \tfrac{1}{2} (2\dw)^2  .
\end{alignat*}
\end{definition}


% \note[J]{
% 	I believe that we can replace the bounds $\dt$ by $\dw$  and $\dtt$ by $2 \dw$.In short, this follows from the following estimate.
% 	\[
% 	| e^{-i \omega }+i| \leq \int_{\pp}^\omega |\tfrac{\partial}{\partial \omega}  e^{-i \omega} | d\omega \leq  \int_{\pp}^\omega |1| d\omega = |\omega - \pp| .
% 	\]
% 	I have not gone through and done this yet. }
% \note[JB]{I think you are right. I think it also follows from $|e^{-i(\pp+\dw)}+i|^2=|e^{-i\dw}-1|^2 = (\cos \dw -1)^2+(\sin \dw)^2=2(1-\cos\dw) \leq 2 \cdot \frac{1}{2} \dw^2$.}
%
When considering an element $ ( \alpha , \omega, c)$ for our $\cO(\epsilon^2)$ analysis, we are often concerned with the 
 distances $|\alpha - \pp|$, $|\omega - \pp|$ and $ \| c - \bar{c}_\epsilon\|$, each of which is of order $\epsilon^2$.  
To create some  notational consistency in these definitions, $\da^0$ and $\dw^0$ are of order $\epsilon^2$, whereas $\dc^0$ is not capitalized as it is of order $\epsilon$. 
Using these definitions, it follows that for any $\rho>0$ and all  $(\alpha, \omega, c ) \in B_\epsilon(r,\rho)$ we have: 
\begin{alignat*}{1}
| \alpha - \pp | & \leq  \da       \\ 
	 | \omega - \pp| & \leq  \dw   \\
	\|c \| &\leq  \dc  .
	%  \\
	% | e^{- i \omega} + i| &\leq  \dt \\
	% | e^{-2 i \omega } +1| &\leq \dtt  .
\end{alignat*}
In this interpretation the superscript $0$ simply refers to $r=0$, i.e., the center of the ball $(\alpha,\omega,c) = \bx_\epsilon$.

The following elementary lemma will be used frequently in the estimates. 
\begin{lemma}\label{lem:deltatheta}
For all $x\in \R$ we have $|e^{ix}-1| \leq |x|$.
Furthermore, for all $|\omega - \bomega_\epsilon  | \leq r_\omega$  
%\note[JB]{I think this should be $|\omega - \bomega_\epsilon| \leq r_\omega$, no?} \note[J]{Yes, that is correct } 
we have 
$ |e^{- i \omega} + i| \leq  \dw$ and
$ | e^{-2 i \omega } +1| \leq 2 \dw $ .
\end{lemma}
\begin{proof}
We start with
\[
  |e^{ix}-1|^2 = (\cos x -1)^2+(\sin x)^2=2(1-\cos x) \leq 2 \cdot \tfrac{1}{2} x^2 = x^2.
\]
% Let $w = \omega - \pp$. Then $|w| \leq \dw$ and, using the previous inequality,
% \[
% | e^{- i \omega} + i|^2=
% |e^{-i(\pp+w)}+i|^2=|e^{-i w}-1|^2 \leq  w^2 =  \dw^2.
% \]
% \note[J]{To avoid using $w$ and $\omega$ in the same line, I propose we switch $ w \mapsto \theta$, as below. Also the last equality should be an inequality.}

Let $\theta = \omega - \pp$. Then $|\theta| \leq \dw$ and, using the previous inequality,
\[
| e^{- i \omega} + i|^2=
|e^{-i(\pp+\theta)}+i|^2=|e^{-i\theta}-1|^2 \leq  \theta^2 \leq  \dw^2.
\]
The final asserted inequality follows from an analogous argument.
\end{proof}


While the operators $U_\omega$ and $L_\omega$ are not continuous in $ \omega$ on all of $ \ell^1_0$, they are within the compact set $ B_\epsilon(r,\rho)$. 
To denote the derivative of these operators, we  define
\begin{alignat}{1}
	U_{\omega}' &:=  - i K^{-1} U_{\omega} \nonumber \\
	L_{\omega}' &:= - i \sigma^+( e^{- i \omega} I + K^{-1} U_{\omega}) + i \sigma^-(e^{i \omega} I - K^{-1} U_{\omega})  , \label{e:Lomegaprime}
\end{alignat}
and we derive Lipschitz bounds on $U_\omega$ and $L_\omega$ in the following proposition.
 
\begin{proposition}
	\label{prop:OmegaDerivatives}
	For the definitions above, $ \frac{\partial }{\partial  \omega} U_\omega = U_{\omega}' $ and $ \frac{\partial }{\partial  \omega}  L_\omega= L_{\omega}' $. 
	Furthermore,  for any $ (\alpha, \omega,c) \in B_\epsilon(r,\rho)$, we have the norm estimates
	\begin{alignat}{1}
	\| (U_{\omega} - U_{\omega_0} )c \| &\leq   \dw  \rho \nonumber  \\
	\|( L_{\omega} - L_{\omega_0} )c \| &\leq  2  \dw (  \dc +  \rho) .
	\label{e:LomegaLip}
	\end{alignat}
\end{proposition}
% \note[J]{ There was a mistake in the statement of this proposition. I changed the estimate $\| U_{\omega} - U_{\omega_0}  \| $ to $ \| (U_{\omega} - U_{\omega_0} )c \| $. Likewise for $ L_\omega$. }

\begin{proof}
One easily calculates that $ \frac{\partial U_\omega}{\partial  \omega} =  U_{\omega}'$,  whereby
$
	\| (U_{\omega} - U_{\omega_0} )c \| \leq \int_{\omega_0}^\omega \| \tfrac{\partial}{\partial \omega} U_\omega c \|  \leq    \dw  \rho  
$. 
Calculating $ \frac{\partial }{\partial  \omega}  L_{\omega} $, we obtain the following:
\begin{alignat*}{1}
 \frac{\partial }{\partial  \omega}  L_{\omega} 
&=  \frac{\partial }{\partial  \omega} \left[  \sigma^+( e^{- i \omega} I + U_{\omega}) + \sigma^-(e^{i \omega} I + U_{\omega}) \right] \\
&= - i \sigma^+( e^{- i \omega} I + K^{-1} U_{\omega}) + i \sigma^-(e^{i \omega} I - K^{-1} U_{\omega}) ,
\end{alignat*}
thus proving $ \frac{\partial L_\omega}{\partial  \omega} =  L_{\omega}'$,
and 
$\|( L_{\omega} - L_{\omega_0} )c \| \leq  \int_{\omega_0}^\omega \| \tfrac{\partial}{\partial \omega} L_\omega c \|  \leq   \dw ( 2  \dc + 2 \rho)$.
\end{proof}

\begin{proposition}
	Let $\epsilon\geq 0$ and  $r=(r_\alpha,r_\omega,r_c) \in \R^3_+$. 
	For any $ \rho > 0$ the map 
	 $T:B_{\epsilon}(r,\rho) \to \R^2 \times \ell^K_0 $ is well defined. 	
	We define functions 
% \note[J]{New definitions for $C_0$ and $C_1$ as the old ones did not quite match the estimates proven below. }
	\begin{alignat*}{1}
%	C_0 &:=  \frac{2 \epsilon^2}{\pi} 
%	\left[
%		\frac{8}{5},\frac{8}{5\sqrt{5}} \sqrt{\left(1-3 \pi /4 \right)^2+(2+\pi )^2},\frac{5 \pi }{2} 
%	\right]
%	\cdot \overline{A_0^{-1} A_1} \cdot [ \da , \dw , \dc ]^T ,
%%	\\
	% C_0 &:=  \frac{2 \epsilon^2}{\pi}
	% \left[
	% 	\frac{8}{5},\frac{2}{5} \sqrt{16+ 8\pi + 5 \pi^2},\frac{5 \pi }{2}
	% \right]
	% \cdot \overline{A_0^{-1} A_1} \cdot [ \da , \dw , \dc ]^T ,
	% \\
	C_0 &:=  \frac{2 \epsilon^2}{\pi} 
	\left[
	\frac{8}{5},\frac{2}{5} \sqrt{16+ 8\pi + 5 \pi^2},\frac{5 \pi }{2} 
	\right]
	\cdot \overline{A_0^{-1} A_1} \cdot [0,0 , \dc ]^T ,
	\\
%	C_1 &:= \frac{5 }{2 \pi} \left(1 +   \frac{4 \epsilon  }{5} \left(2+\sqrt{5}\right) \right) , \\
	% C_1 &:= \frac{5 }{2 \pi} + 2  \epsilon   \left(2+\sqrt{5}\right)  , \\
	C_1 &:= \frac{5 }{2 \pi} + \frac{\epsilon \sqrt{10}}{\pi}, \\
	C_2 &:= \dw  \left[  (1 + \pp) + \epsilon \pi  \right] , \\
	C_3 &:=  
	\da (2+ \dc) +	2 \dw (1+\pp) 
		+ \epsilon \left[ \pi + 2\da  + 4 \dc \da + \pi \dw \dc  + (\pp + \da ) \dc^2 \right] ,
	\end{alignat*}
where the expression for $C_0$ should be read as a product of a row vector, a $(3 \times 3)$ matrix and a column vector.
Furthermore we define, for any $\epsilon,r_\omega$ such that $C_1 C_2 <1$,
	\begin{equation}
		C(\epsilon,r_\alpha,r_\omega,r_c) := \frac{C_0+ C_1 C_3}{1 - C_1 C_2}
		 \, .
		\label{eq:RhoConstant}
	\end{equation}
	All of the functions $C_0,C_1,C_2,C_3$ and $C$ are nonnegative and monotonically increasing in their arguments $\epsilon$ and~$r$. 
	Furthermore, if  $C_1 C_2 < 1$ and $	C(\epsilon,r_\alpha,r_\omega,r_c) \leq \rho $
	then $\| K^{-1} \pi_c  T( x) \| \leq \rho $
	for $x \in B_{\epsilon}(r,\rho)$. 
	\label{prop:DerivativeEndo}
\end{proposition}

% \marginpar{This proposition is vague about the actual spaces being used, ie. $\ell_1,\ell^K_0$, etc.}

\begin{proof}
	Given their definitions, it is straightforward to check that the functions $C_i$ and $C$ are monotonically increasing in their arguments.  
	To prove the second half of the proposition, we split 
	$K^{-1} \pi_c  T(x)$ into several pieces. 
%\note[JB]{$\pi_c$ and $\pi_{\ge 2}$ added. Jonathan, could you please go through this and check?}
%\note[JB]{This did not work, since we do not have that $x$ is bounded by $[\da,\dw,\dc]^T$. Jonathan: what you probably meant was what I introduce as $\pi_c^0 x$, but could you please check?} 
	We define the projection $\pi_c^0 x = (0,0,\pi_c x)$.
We then obtain
	\begin{alignat*}{1}
	K^{-1} \pi_c  T(x)  &= K^{-1} \pi_c   [ x - A^{\dagger} F(x) ]   \\
	&= K^{-1} \pi_c  [ I \pi_c^0 x -    A^{\dagger} ( A \pi_c^0 x + F(x) - A \pi_c^0 x)]  \\
	&= \epsilon^2 K^{-1} \pi_c (A_0^{-1}A_{1})^2 \pi_c^0 x + K^{-1} \pi_c A^{\dagger} (F(x) - A \pi_c^0 x) \nonumber \\
	&=  \frac{2 \epsilon^2}{i\pi} \hat{U} \pi_{\ge 2} A_1 A_0^{-1}A_{1} \pi_c^0 x +\frac{2 }{i\pi} \hat{U}  \pi_{\ge 2} (I-\epsilon A_1 A_0^{-1}) (F(x) - A\pi_c^0 x)  ,
\end{alignat*}
where we have used that $K^{-1} \pi_c A_0^{-1} = \frac{2}{i\pi} \hat{U} \pi_{\ge 2}$, with the projection $\pi_{\ge 2}$ defined in~\eqref{e:pige2}.
By using $\| \hat{U} \| \leq \frac{5}{4}$, see Proposition~\ref{p:severalnorms}, we obtain the estimate
%\note[J]{Changed $[\da,\dw,\dc ]^T$ to $[0,0,\dc ]^T$}
\begin{equation}
	\| K^{-1} \pi_c T(x) \| \leq   \frac{2 \epsilon^2}{\pi} \overline{\hat{U}\pi_{\ge 2} A_1} \cdot  \overline{ A_0^{-1}A_{1}}  \cdot
	[0,0,\dc ]^T +\frac{5 }{2 \pi} \left(1 + \epsilon \| A_1 A_0^{-1} \| \right) \|F(x) - A\pi_c^0 x \| .
	\label{eq:DerivativeEndo}
\end{equation}
Here the $(1 \times 3)$ row vector $\overline{\hat{U}\pi_{\ge 2} A_1}$ is an upper bound on $\hat{U}\pi_{\ge 2} A_1$ interpreted as a linear operator from $\R^2 \times \ell^1_0$ to $\ell^1_0$, thus extending in a straightforward manner the definition of upper bounds given in  Definition~\ref{def:upperbound}.
	
	
	We have already calculated  an expression for
	 $ \overline{ A_0^{-1}A_{1}}$ in Proposition~\ref{prop:A0A1},  and  $  \| A_1 A_0^{-1}\| =\frac{2\sqrt{10}}{5}$ by Proposition~\ref{prop:A1A0}.  In order to finish the calculation of the right hand side of Equation \eqref{eq:DerivativeEndo}, we need to  estimate  $\| F(x) - A\pi_c^0 x \|$ and $\overline{\hat{U} \pi_{\ge 2} A_1} $. 
	We first calculate a bound on $\hat{U} \pi_{\ge 2} A_1 $. 
	We note that $ \hat{U} \pi_{\ge 2} A_1  =  \hat{U} \e_2 ( i_\C A_{1,2} \pi_{\alpha,\omega})+ \hat{U} \pi_{\ge 2}A_{1,*} \pi_c$.	
As $\|\hat{U} e_2\| = \| \tfrac{4-2i}{5} \e_2\|$,
it follows from the definition of $A_{1,2}$ 
that 
\[
	 \left| i_\C  A_{1,2}
	 \left( \!\!\begin{array}{c}\alpha \\ \omega \end{array} \!\!\right) \right|  
	 \cdot \| \hat{U} \e_2 \| 
	 \leq 
	 \left(\frac{\sqrt{20}}{5} |\alpha| +  \frac{\sqrt{(2-3 \pi/2)^2 +4(2+\pi)^2}}{5} |\omega| \right)  \cdot \frac{4}{\sqrt{5}}.
\]
	To calculate $ \| \hat{U} \pi_{\ge 2} A_{1,*} \|$ we note that $ \| \hat{U}\| \leq \frac{5}{4}$ and $ \|A_{1,*}\| = \pp \| L_{\omega_0} \| \leq 2 \pi$. 
	Hence $ \| \hat{U} \pi_{\ge 2} A_{1,*} \| \leq \frac{5 \pi}{2}$. 
	Combining these results, we obtain  that
%\note[JB]{I think the second, rearranged version, of the root looks ``nicer''. }
%	\[
%	\overline{\hat{U} \pi_{\ge 2}  A_1 } = \left[\frac{8}{5},\frac{8}{5\sqrt{5}} \sqrt{\left(1-3 \pi /4 \right)^2+(2+\pi )^2},\frac{5 \pi }{2} \right].
%	\] 
	\[
	\overline{\hat{U} \pi_{\ge 2}  A_1 } = \left[\frac{8}{5},\frac{2}{5} \sqrt{16 + 8 \pi + 5 \pi^2},\frac{5 \pi }{2} \right].
	\] 
Thereby, it follows from~\eqref{eq:DerivativeEndo} that 
\begin{equation}\label{e:C0C1}
	\| K^{-1} \pi_c T(x) \| \leq C_0 + C_1 \| F(x) - A \pi_c^0 x\|. 
\end{equation}
We now calculate
	\begin{alignat*}{1}
	F(x) - A \pi_c^0 x &= 
	(i \omega + \alpha e^{-i \omega} ) \e_1 + 
	( i \omega K^{-1} + \alpha U_{\omega}) c + 
	\epsilon \alpha e^{-i \omega} \e_2  +
	\alpha \epsilon L_\omega c + 
	\alpha \epsilon [ U_{\omega} c] * c  
	\\ &\qquad 
	- \pp (i K^{-1} + U_{\omega_0} + \epsilon L_{\omega_0} ) c \\
	&= i ( \omega - \pp) K^{-1} c + ( \alpha - \pp) U_{\omega} c +  \pp ( U_{\omega} - U_{\omega_0})c  \nonumber \\
	&\qquad  + \left[i ( \omega - \pp ) + ( \alpha - \pp) e^{-i \omega} + \pp( e^{- i \omega }+ i)\right] \e_1  \nonumber
	\\ 
	&\qquad  +\epsilon  \alpha   e^{-i \omega}  \e_2  
+  ( \alpha- \pp)  \epsilon L_{\omega} c + \pp \epsilon ( L_{\omega} - L_{\omega_0}) c + \alpha \epsilon [ U_{\omega} c ] * c .
	\end{alignat*}
Taking norms and using~\eqref{e:LomegaLip} and Lemma~\ref{lem:deltatheta}, we obtain 
	\begin{alignat*}{1}
	\| F(x) - A \pi_c^0 x\|& \leq  
	 \dw \rho + \da \dc + \pp \dw \rho
    +	2 (\dw + \da + \pp \dw)  
	   \\
	&\qquad + \epsilon \left[ 2(\pp + \da ) + 4 \dc \da + \pi  \dw (  \dc + \rho) + (\pp + \da ) \dc^2 \right]  \\
		&= \dw [ (1+\pp) +   \epsilon \pi ] \rho \nonumber \\ 
	&\qquad +  \da (2 + \dc)
	+	2 \dw (1+\pp) 
	+ \epsilon \left[ \pi + 2\da  + 4 \dc \da + \pi \dw \dc  + (\pp + \da ) \dc^2 \right].  
	\end{alignat*}


	We have now computed all of the necessary constants. Thus $ \| F(x) - A \pi_c^0 x \| \leq C_2 \rho + C_3$, and from~\eqref{e:C0C1}   we obtain 
	\begin{eqnarray*}
	\| K^{-1} \pi_c T(c) \|
	&\leq & C_0 +  C_1 ( C_2  \rho + C_3),
	\end{eqnarray*}
with the constants defined in the statement of the proposition.
We would like to select values of $\rho$ for which 
	\[
	\| K^{-1} \pi_c T(c) \| \leq \rho
	\]
	This is true if  
	$	C_0 +  C_1 ( C_2  \rho + C_3) \leq \rho$, 
	or equivalently 
	\[
	\frac{C_0 + C_1 C_3 }{1 - C_1 C_2} \leq \rho.
	\]
	This proves the theorem.
\end{proof}

%%%%%%%%%%%%%%%%%%
%%% Appendix C %%%
%%%%%%%%%%%%%%%%%%


\section[]{Relation between bulge-to-disk mass ratio and shear rate}

In Section 3.4 we showed that the bulge-to-disk mass ratio ($B/D$) is not 
always a good indicator for the shear rate ($\Gamma$), because $\Gamma$ 
also depends on other parameters such as the disk-mass fraction ($f_{\rm d}$). 
Here, we construct additional initial conditions by sequentially changing some 
parameters in order to investigate their importance. We do not simulate 
these models, but measure $B/D$ and $\Gamma$ in the generated models at $t=0$.
All parameters of these models are summarized in Tables \ref{tb:models_add} and 
 \ref{tb:models_add2}.

In Fig.~\ref{fig:Gamma_BD}, we present the relation between $B/D$ and $\Gamma$
calculated from the additional initial conditions.
If we keep both $f_{\rm d}$ and the bulge scale length ($r_{\rm b}$) constant,
$\Gamma$ monotonically increases as $B/D$ increases (square symbols).
But if we increase $r_{\rm b}$ while keeping $f_{\rm d}$ constant, then $\Gamma$ 
also increases (triangle symbols). 
If we increase $f_{\rm d}$ and keep $B/D$ and $r_{\rm b}$ constant,
$\Gamma$ increases (diamond symbols). 
The halo scale length ($r_{\rm h}$) and scale velocity 
($\sigma_{\rm h}$), on the other hand, barely affect the relation between $B/D$ and $\Gamma$
(circle symbols).


\begin{figure}
\includegraphics[width=\columnwidth]{figures/shear_rate_BD_fd.pdf}
\caption{Relation between bulge-to-disk mass ratio ($B/D$) and shear rate ($\Gamma$). \label{fig:Gamma_BD}}
\end{figure}


\begin{table*}
\begin{center}
%\rotate
\caption{Parameters for additional initial conditions\label{tb:models_add}}
\begin{tabular}{lccccccccccc}
%\tabletypesize{\scriptsize}
%\tablewidth{0pt}
%\startdata 
\hline
           &  \multicolumn{3}{l}{Halo} &  \multicolumn{4}{l}{Disk} &  \multicolumn{3}{l}{Bulge} \\
Parameters &  $a_{\rm h}$ & $\sigma_{\rm h}$ & $1-\epsilon_{\rm h}$ & $M_{\rm d}$ & $R_{\rm d}$ & $z_{\rm d}$ & $\sigma_{R0}$  & $a_{\rm b}$ & $\sigma_{\rm b}$ & $1-\epsilon_{\rm b}$ \\ 
Model   &  (kpc) & ($\kms$) &  &  $(10^{10}M_{\odot})$ & (kpc) & (kpc) & ($\kms$)  & (kpc) & $(\kms)$\\
\hline \hline
Add1 & 8.2 & 350 & 0.9  & 2.45 & 2.8 & 0.36 & 105 & 0.64 & 300 & 1.0  \\ %t13
Add2 & 11.5 & 443 & 0.9  & 2.45 & 2.8 & 0.36 & 105 & 0.65 & 400 & 1.0  \\ %t5
Add3 & 8.2 & 370 & 0.9  & 2.45 & 2.8 & 0.36 & 105 & 0.64 & 500 & 1.0  \\ %t14
Add4 & 10 & 340 & 0.9  & 2.45 & 2.8 & 0.36 & 105 & 0.64 & 550 & 1.0  \\ %t8
Add5 & 8.2 & 295 & 0.9  & 2.45 & 2.8 & 0.36 & 105 & 0.65 & 600 & 1.0  \\ %t7
Add6 & 8.2 & 284 & 0.9  & 2.45 & 2.8 & 0.36 & 105 & 1.3 & 370 & 1.0  \\ %t11
Add7 & 8.2 & 330 & 0.9  & 2.45 & 2.8 & 0.36 & 105 & 0.8 & 380 & 1.0   \\  %t12
Add8 & 8.2 & 330 & 1.0  & 2.45 & 2.8 & 0.36 & 105 & 0.64 & 550 & 1.0  \\ %t10
Add9 & 12 & 400 & 1.0  & 2.45 & 2.8 & 0.36 & 105 & 0.64 & 540 & 1.0  \\ %t9
Add10 & 8.2 & 370 & 0.9  & 1.47 & 2.8 & 0.36 & 105 & 0.64 & 390 & 1.0  \\ %t15
Add11 & 12 & 330 & 0.9  & 2.45 & 2.8 & 0.36 & 105 & 0.64 & 486 & 1.0  \\ %t17
\hline
\end{tabular}
\end{center}
\end{table*}


\begin{table*}
\begin{center}
\caption{Obtained values for additional initial conditions\label{tb:models_add2}}
\begin{tabular}{lccccccccc}
%\tabletypesize{\scriptsize}
%\rotate
%\tablewidth{0pt}
%\startdata 
\hline
Model    & $M_{\rm d}$ & $M_{\rm b}$ & $M_{\rm h}$ & $M_{\rm b}/M_{\rm d}$& $R_{\rm d, t}$ & $r_{\rm b, t}$ & $r_{\rm h, t}$ & $f_{\rm d}$  &  $\Gamma$\\ 
   & ($10^{10}M_{\odot}$) & ($10^{10}M_{\odot}$) & ($10^{10}M_{\odot}$) & (kpc) & (kpc) & (kpc) &  &   &  \\ 
\hline  \hline
Add1 & 2.57 & 0.514 & 56.0 & 0.20 & 31.6 & 3.57 & 284 &  0.346 & 0.682 \\ %test13
Add2 & 2.58 & 1.21 & 137 & 0.47 & 31.6 & 5.32 & 330 & 0.343 & 0.706 \\ %test5
Add3 & 2.69 & 2.03 & 94.6 & 0.75 & 31.6 & 6.65 & 234 &  0.321 & 0.895 \\ %test14
Add4 & 2.59 & 2.74 & 124 & 1.05 & 31.6 & 8.67 & 288 &   0.340 & 1.04 \\ %test8
Add5 & 2.61 & 3.29 & 93.2 & 1.26 & 31.6 & 9.48 & 270 &  0.307 & 1.10 \\ %test7
Add6 & 2.59 & 1.19 & 45.9 & 0.46 & 31.6 & 6.28& 265 &  0.332 & 0.869 \\ %test11
Add7 & 2.58 & 1.11 & 61.3 & 0.43 & 31.6 & 5.31 & 251 & 0.348  & 0.789 \\ %test12
Add8 & 2.61 & 2.62 & 108 & 1.0 & 31.6 & 7.98 & 494 &  0.322 &  0.996\\ %test10
Add9 & 2.59 & 2.64 & 195 & 1.02 & 31.6 & 8.46 & 687 &  0.341 & 1.00 \\  % test9
Add10 & 1.58 & 1.18 & 74.8 & 0.74 & 31.6 & 5.65 & 234 &  0.251  & 0.675 \\ %test15
Add11 & 2.75 & 2.07 & 130 & 0.75 & 31.6 & 8.00 & 324 &  0.401 & 0.992 \\ %test17
\hline
\end{tabular}
\end{center}
\medskip
\end{table*}



%!TEX root = hopfwright.tex

%%%%%%%%%%%%%%%%%%
%%% Appendix D %%%
%%%%%%%%%%%%%%%%%%

\section{Appendix: The bounding functions for $Z(\epsilon,r,\rho)$}
\label{sec:BoundingFunctions}

	In this section we calculate an upper bound on $DT$.  
	To do so we first calculate 
	\(
	DF = 
	\left[ \frac{\partial F}{\partial  \alpha}, 
	\frac{\partial F}{\partial  \omega},
	\frac{\partial F}{\partial  c}
	\right]
	\):
%\note[JB]{I don't think we need to write the partial derivatives in a proposition and no proof is needed.}
%\begin{proposition}
%	The partial derivatives of $F$ are
	\begin{alignat}{1}
	\label{eq:FpartialA}
	\frac{\partial F}{\partial  \alpha} &= e^{-i \omega} \e_1 + U_\omega c + \epsilon e^{-i \omega} \e_2 + \epsilon L_\omega c + \epsilon [ U_\omega c] * c , \\
	\label{eq:FpartialW} 
	\frac{\partial F}{\partial  \omega} &=
	i(1-\alpha e^{-i \omega}) \e_1 + 
	i K^{-1} ( I - \alpha U_{\omega} ) c  -
	i \alpha \epsilon e^{-i \omega} \e_2 + 
	\alpha \epsilon L_{\omega}' c - i \alpha \epsilon [ K^{-1} U_\omega c ] *c ,
	\\
	% \end{alignat}
	% and, writing $\frac{\partial F}{\partial  c}$ as an operator working on an element $ b \in \ell^K_0$,
	% % with $ \| b \| = 1$:
	% \begin{equation}
	\frac{\partial F}{\partial  c} \cdot b 
	& =
	( i \omega K^{-1} + \alpha U_{\omega}) b + \alpha \epsilon \left( L_\omega b  + [ U_\omega b] * c + [U_{\omega} c ]*b \right)  , \qquad \text{for all $b\in \ell^K_0$},
	\label{eq:Fcderivative}
%	\end{equation}
\end{alignat}	
%\end{proposition}
where $L_{\omega}'$ is given in~\eqref{e:Lomegaprime}, and $\frac{\partial F}{\partial  c}$ is expressed in terms of the directional derivative. 
Recall that $\II$ is used to denote the $ 3 \times 3$ identity matrix. 

% \begin{proof}
% 	Recall from Equation \ref{eq:FDefinition} that:
% \begin{equation}
% F_\epsilon(\alpha,\omega, c) =
% [i \omega + \alpha e^{-i \omega}] \e_1 +
% ( i \omega K^{-1} + \alpha U_{\omega}) c +
% \epsilon \alpha e^{-i \omega} \e_2  +
% \alpha \epsilon L_\omega c +
% \alpha \epsilon [ U_{\omega} c] * c.
% \end{equation}
% The partial derivative for $ 	\frac{\partial F}{\partial  \alpha}$ and $ 	\frac{\partial F}{\partial  c}$   clearly follow, and the partial derivative for $ 	\frac{\partial F}{\partial  \omega}$  follows from Proposition \ref{prop:OmegaDerivatives}.
%
%
%
% \end{proof}


\begin{theorem}
	\label{prop:Zdef}
	Define $\overline{A_0^{-1} A_1}$ as in Proposition \ref{prop:A0A1} and define the matrix 
%	\[
%	M := 
%	\left(
%	\begin{array}{ccc}
%	1+\frac{2}{\pi } & 0 & 0 \\
%	\frac{2}{\pi } & 0 & 0 \\
%	0 & 0 & 1 \\
%	\end{array}
%	\right)
%	\left(
%	\begin{array}{ccc}
%	f_{1,\alpha } & f_{1,\omega } & f_{1,c} \\
%	0 & 0 & 0 \\
%	f_{*,\alpha } & f_{*,\omega } & f_{*,c} \\
%	\end{array}
%	\right)
%	\]
%\note[JB]{I think we should remove the zero column and row, as below:}
%\note[J]{Updated top-right element entry to new estimate.}
	\begin{equation}\label{e:defM}
	M := 
	\left(
	\begin{array}{cc}
	\sqrt{\tfrac{4}{\pi^2}+1} & 0 \\
	\frac{2}{\pi } & 0 \\
	0 & 1 \\
	\end{array}
	\right)
	\left(
	\begin{array}{ccc}
	f_{1,\alpha } & f_{1,\omega } & f_{1,c} \\
	f_{*,\alpha } & f_{*,\omega } & f_{*,c} \\
	\end{array}
	\right) ,
	\end{equation}
	where the functions $f_{1,\cdot}(\epsilon,r,\rho)$ and $f_{*,\cdot}(\epsilon,r,\rho)$ are defined as in Propositions \ref{prop:Z1a}--\ref{prop:Zsc}. 
	If we define $Z(\epsilon,r,\rho)$ as 
	\begin{equation}
		Z(\epsilon,r,\rho) := \epsilon^2  \left(\overline{ A_0^{-1} A_1 }\right)^2  + 
		\left(\II + \epsilon \overline{ A_0^{-1} A_1 } \right) \cdot M ,
	\end{equation}
	then $Z(\epsilon,r)$ is an upper bound (in the sense of Definition~\ref{def:upperbound}) on $DT(x)$ for all $ x \in B_\epsilon(r , \rho)$. 
	Furthermore, the components of $Z(\epsilon,r,\rho)$ are increasing in  $ \epsilon$, $r$ and $\rho$. 
% \note[JB]{shouldn't we also have monotonicity in $\rho$?}
% \note[J]{Yes. Also note that in our application of radii polynomials, the value of $\rho$ is fixed for all $ 0< \epsilon \leq \epsilon_0$.}
\end{theorem}

\begin{proof}
	
	
	If we fix some $x \in B_\epsilon(r,\rho)$, then we obtain 
%	\remove[JB]{the following:}
%\note[JB]{Rearranged to separate expression for $DT$ from the upper bound on it}
\begin{alignat*}{1}
		D T( x ) &=  I - A^{\dagger}  D F( x)  \\
		&= ( I - A^{\dagger} A) - A^{\dagger} \left[ D F( x)  - A \right]\\
		&=   \epsilon^2 (A_0^{-1} A_1 )^2 -    [I - \epsilon (A_0^{-1} A_1 ) ] \cdot  A_0^{-1} \cdot   \left[ D F( x) - A \right] ,
\end{alignat*}
hence an upper bound on $DT(x)$ is given by
\begin{equation*}
\epsilon^2  \left(\overline{ A_0^{-1} A_1 }\right)^2  + 
\left(\II + \epsilon \overline{ A_0^{-1} A_1 } \right) \cdot 
\overline{A_0^{-1}  \left[D F( x ) - A \right] },
\end{equation*}
	where $\overline{A_0^{-1}  \left[D F( x ) - A \right] }$ is a yet to be determined upper bound on $A_0^{-1}  \left[D F( x ) - A \right]$.  
To calculate this upper bound, we break it up into two parts: 
	\begin{alignat}{1}
		\pi_{\alpha,\omega} A_0^{-1}  \left( D F( x) - A \right)  &= 
		A_{0,1}^{-1}  i_{\C}^{-1} \pi_1   \left( D F( x) - A \right) 
\label{eq:Zfinite}\\ 
		\pi_c 		A_0^{-1}  \left( D F( x) - A \right)  &=
		A_{0,*}^{-1}  \pi_{\geq 2} \left( DF(x) - A \right)  . \label{eq:Zstar} 
\end{alignat}
	
To calculate an upper bound on \eqref{eq:Zfinite}, we use the explicit expression for $A_{0,1}^{-1}$ to estimate  
% \note[JB]{Perhaps the factor $\frac{2}{\pi}+1$ can be improved to $\sqrt{\frac{4}{\pi^2}+1}$?; did not implement it.} 
%\note[J]{Yes; done.}
	\begin{alignat*}{1}
	\left| \pi_\alpha  A_{0,1}^{-1} \pi_1 \left( D F( x) - A \right)\right| &\leq  \sqrt{\tfrac{4}{\pi^2} + 1} \, \overline{\pi_1( DF( x ) - A) } \\
	\left| \pi_\omega  A_{0,1}^{-1} \pi_1 \left( D F( x) - A \right)\right| &\leq  \tfrac{2}{\pi} \,  \overline{ \pi_1( DF( x ) - A)  } ,
	\end{alignat*}
where $\overline{ \pi_1( DF( x ) - A)  }$ is an upper bound on $\pi_1( DF( x ) - A)$, viewed as an operator from $\R^2 \times \ell^K_0 $ to $\C$ (a straightforward generalization of Definition~\ref{def:upperbound}).
Indeed, in Propositions \ref{prop:Z1a}, \ref{prop:Z1w} and \ref{prop:Z1c} we   determine functions $f_{1,\cdot}$ such that, for all $x \in B_\epsilon(r,\rho)$, 
	\begin{alignat*}{1}
	f_{1,\alpha} (\epsilon,r,\rho) &\geq    \left|  \frac{\partial F_1 }{\partial \alpha} (x) + i  \right|  , \\
	f_{1,\omega} (\epsilon,r,\rho) &\geq   \left|  \frac{\partial F_1}{\partial \omega} (x)- (i- \pp)  \right|   , \\
	f_{1,c} (\epsilon,r,\rho) &\geq   \left|  \frac{\partial F_1}{\partial c} (x) \cdot b -  \pp \epsilon (i-1) \pi_2 b \right| ,
	\qquad\text{for all $b\in\ell^K_0$ with $\|b\| \leq 1$}.
	\end{alignat*} 
Here the projection $\pi_2$ is defined as $\pi_2 b := b_2 \in \C$ for $b=\{b_k\}_{k=1}^{\infty} \in \ell^1$.	
	Hence $ [ f_{1,\alpha} , f_{1,\omega}, f_{1,c}]$ is an upper bound on $\pi_1 (DF( x ) -A )$.  
	
	
	For calculating an upper bound on Equation~\eqref{eq:Zstar}, in Propositions  \ref{prop:Zsa}, \ref{prop:Zsw}  and \ref{prop:Zsc} we  determine functions $f_{*,\cdot}$ such that, for all $x \in B_\epsilon(r,\rho)$,   
	\begin{alignat*}{1}
	f_{*,\alpha} (\epsilon,r,\rho)&\geq  \left\| A_{0,*}^{-1}  \pi_{\geq 2} \left(
	\frac{\partial F}{\partial \alpha}(x)  +    \epsilon  \tfrac{2 +4 i}{5} \e_2  \right) \right\| , \\
	%
%	f_{*,\omega} (\epsilon,r,\rho)&\geq  \left\| A_{0,*}^{-1}  \pi_{\geq 2}\left( \frac{\partial F}{\partial \omega}(x) -  \epsilon \left[ (2 + \pi) ( \tfrac{1+ 2 i}{5})- \pp \right] \e_2  \right) \right\| ,  \\
%
	f_{*,\omega} (\epsilon,r,\rho)&\geq  \left\| A_{0,*}^{-1}  \pi_{\geq 2}\left( \frac{\partial F}{\partial \omega}(x) -  \epsilon \left[ \tfrac{4-3\pi}{10} + \tfrac{2(2 + \pi)}{5}i \right] \e_2  \right) \right\| ,  \\
	%
	f_{*,c} (\epsilon,r,\rho) &\geq    \left\|  A_{0,*}^{-1} \pi_{\geq 2} \left( \frac{\partial F}{\partial c}(x) \cdot b  - (A_{0,*} + \epsilon A_{1,*}) b \right) 
	\right\| , \qquad\text{for all $b\in\ell^K_0$ with $\|b\| \leq 1$}.
	\end{alignat*}
	Hence $ [ f_{*,\alpha} , f_{*,\omega}, f_{*,c}]$ is an upper bound on $A_{0,*}^{-1}  \pi_{\geq 2} \left( D F( x) - A \right)$, viewed as an operator from $\R^2 \times \ell^K_0$ to $\ell^1_0$. 
		We have thereby shown that $M$, as defined in~\eqref{e:defM}, is an upper bound on $\overline{A_0^{-1}  \left[D F( x ) - A \right] }$, which concludes the proof.	
\end{proof}









%%%%%%%%%%%%%%%%%%%%%%%%%%%%%%%%%%%%%%%%%%%%%%%%%%%%%%%%%%%%%%%%%%%%%%%%%%%%

%\note[JB]{I do not understand why we want to suppose $\epsilon <1$. I changed it in the D.2 but not yet in the other 5 propositions.} 
%\note[J]{You are correct that $ \epsilon <1$ is not needed. I have removed this assumption in the other propositions.}
\begin{proposition}
	\label{prop:Z1a}
	Define
	\[
	f_{1,\alpha} :=  \dw +  \epsilon \frac{\dc  (2 + \dc) }{2} .
	\]
	Then for all $x = (\alpha,\omega,c) \in B_\epsilon(r,\rho)$ 
	\[
	f_{1,\alpha} \geq   \left|  \frac{\partial F_1}{\partial \alpha} (x) + i  \right| .
	\]
\end{proposition}




\begin{proof}
	We calculate
\begin{equation*}
	\frac{\partial F_1}{\partial \alpha} (x) + i =
	e^{- i \omega} + i  
	+ \epsilon \left( e^{i \omega} + e^{-2 i  \omega} \right) \pi_2 c
	+ \epsilon \pi_1 ([ U_{\omega} c] * c) ,
\end{equation*}
hence, using Lemma~\ref{lem:deltatheta},
%\begin{equation*}
%	\left|  \frac{\partial F_1 }{\partial \alpha} F_1(x) + i  \right|   \leq 
%| e^{-i \omega } +i | + 2 \epsilon \dc + \epsilon \dc^2  
%	\leq  \dw +  \epsilon \dc  (2 + \dc) .
%\end{equation*}
%\note[JB]{Because of the factor 2 in the norm I think it should be }
\begin{equation*}
	\left|  \frac{\partial F_1}{\partial \alpha} (x) + i  \right|   \leq 
| e^{-i \omega } +i | + 2 \epsilon \frac{\dc}{2} + \epsilon \frac{1}{2} \dc^2  
	\leq  \dw +  \epsilon \frac{\dc  (2 + \dc)}{2} .
\end{equation*}
Here we have used that $|\pi_k a| \leq \frac{1}{2}\|a\|$ for $k=1,2$ and all $a \in \ell^1$.
\end{proof}

%\note[J]{I think it would be good to have an explanation for why we divide $ \dc$ by 2. This explanation wouldn't need appear every time; it could just be given once. I am not sure though where the right place for it would be. }
%\note[JB]{See my attempt above}

%%%%%%%%%%%%%%%%%%%%%%%%%%%%%%%%%%%%%%%%%%%%%%%%%%%%%%%%%%%%%%%%%%%%%%%%%%%%

\begin{proposition}
		\label{prop:Z1w}
	Define
	% \[
	% f_{1,\omega} :=
	% \da + \pp \dw + \frac{ \alpha \epsilon \dc}{2} ( 3 + \rho) .
	% \]
	% \note[J]{Proposed Change}
		\[
		f_{1,\omega} := 
		\da + \pp \dw + (\pp + \da) \frac{  \epsilon \dc}{2} ( 3 + \rho)  .
		\]
Then for all $x= (\alpha,\omega,c) \in B_\epsilon(r,\rho)$
	\[
	f_{1,\omega} \geq   \left|  \frac{\partial F_1}{\partial \omega } (x)- (i- \pp)  \right| .
	\]
\end{proposition}




\begin{proof}
We calculate 
\begin{alignat*}{1}
	\frac{\partial F_1}{\partial \omega} (x) - (i - \pp)  &=
	(i - i\alpha e^{- i \omega}) - (i - \pp) 
	+ \alpha \epsilon ( i e^{i \omega }- 2 e^{- 2 i \omega}) \pi_2 c -i \alpha \epsilon \pi_1  ([ K^{-1} U_\omega c ] *c )\\
	&= -i (\alpha - \pp) e^{-i \omega} - i \pp ( i + e^{-i\omega} )
	+ \alpha \epsilon ( i e^{i \omega }- 2 e^{- 2 i \omega}) \pi_2 c -i \alpha \epsilon \pi_1( [ K^{-1} U_\omega c ] *c) ,
\end{alignat*}
hence, using Lemma~\ref{lem:deltatheta} again,
%\begin{equation*}
%	\left|  \frac{\partial F_1}{\partial \omega} ( x)- (i- \pp)  \right|  \leq
%	\da +  \pp \dw  + 3 \alpha \epsilon \dc + \alpha \epsilon \rho \dc  .
%%	\\ &\leq &   \da + \pp \dt + \alpha \epsilon \dc ( 3 + \rho) 
%\end{equation*}
%\note[JB]{Because of the factor 2 in the norm I think it should be }
\begin{equation*}
	\left|  \frac{\partial F_1}{\partial \omega} ( x)- (i- \pp)  \right|  \leq
	\da +  \pp \dw  + \frac{3}{2} \alpha \epsilon \dc +  \frac{1}{2} \alpha \epsilon \rho \dc  .
\end{equation*}
\end{proof}

%%%%%%%%%%%%%%%%%%%%%%%%%%%%%%%%%%%%%%%%%%%%%%%%%%%%%%%%%%%%%%%%%%%%%%%%%%%%


\begin{proposition}
		\label{prop:Z1c}
	Define
	\[
	f_{1,c} := 
	\epsilon \left(  \da + \tfrac{3 \pi}{4} \dw +  (\pp + \da ) \dc   \right) .
	\]
	Then for all $x= (\alpha,\omega,c) \in B_\epsilon(r,\rho)$
	\[
	f_{1,c} \geq   \left|  \frac{\partial F_1 }{\partial c } (x) \cdot b -  \pp \epsilon (i-1) \pi_2 b \right|, 
	\qquad\text{for all $b\in\ell^K_0$ with $\|b\| \leq 1$}.
	\]
\end{proposition}



\begin{proof}
	We calculate
\begin{alignat*}{1}
		\frac{\partial F_1 }{\partial c } (x) \cdot b -  \pp \epsilon (i-1) \pi_2 b 
	& =
	\epsilon [ \alpha (e^{i \omega} + e^{-2i \omega})  - \pp(i -1)] \pi_2 b 
	+ \alpha \epsilon  \pi_1 \bigl(  [ U_{\omega} b ] * c + [ U_{\omega} c ]*b \bigr)  \\
	%
	&= \epsilon [ (\alpha - \pp) (e^{i \omega} + e^{-2i \omega})  ]  \pi_2 b  +
	\epsilon  \pp [  (e^{i \omega} + e^{-2i \omega})  - (i -1)]  \pi_2 b  \nonumber \\
	& \qquad\quad + \alpha \epsilon \pi_1 \bigl(  [ U_{\omega} b ] * c + [ U_{\omega} c ]*b \bigr)  ,
	\end{alignat*}
hence, for $\|b\| \leq 1$,
%\begin{equation*}
%	\left| 
%	\frac{\partial F_1 }{\partial c } (x) \cdot b -  \pp \epsilon (i-1) \pi_2 b 
%	\right| 
%	\leq
%  \epsilon \left( 2 \da + \pp(\dt + \dtt ) + 2 \alpha \dc   \right)  .
%	\end{equation*}
%\note[JB]{Because of the factor 2 in the norm I think it should be }
\begin{equation*}
	\left| 
	\frac{\partial F_1 }{\partial c } (x) \cdot b -  \pp \epsilon (i-1) \pi_2 b 
	\right| 
	\leq
  \epsilon \left(  \da + \tfrac{\pi}{4}(\dw + 2 \dw ) +  (\pp + \da )  \dc   \right)  .
	\end{equation*}
\end{proof}


%%%%%%%%%%%%%%%%%%%%%%%%%%%%%%%%%%%%%%%%%%%%%%%%%%%%%%%%%%%%%%%%%%%%%%%%%%%%



%%%%%%%%%%%%%%%%%%%%%%%%%%%%%%%%%%%%%%%%%%%%%%%%%%%%%%%%%%%%%%%%%%%%%%%%%%%%
\begin{proposition}
		\label{prop:Zsa}
Define  
	% \[
	% f_{*,\alpha} := \frac{2}{\pi \sqrt{5}} \left( r_c + \epsilon \dt + \epsilon \dc (4 + \dc )  \right) .
	% \]
	% \note[J]{Proposed Change}
		\[
		f_{*,\alpha} := \frac{2}{\pi \sqrt{5}}\left( r_c +  2 \dw (  \dc^0 +  \epsilon )  + \epsilon \dc (4 + \dc )  \right)  .
		\]
	Then for all $x= (\alpha,\omega,c) \in B_\epsilon(r,\rho)$
	\[
		f_{*,\alpha} \geq  \left\| A_{0,*}^{-1} \pi_{\geq 2} \left(
		\frac{\partial F}{\partial \alpha}(x)  +   \epsilon  \tfrac{2 +4 i}{5}  \e_2  \right) \right\|  .
	\]
\end{proposition}

\begin{proof}
We note that $\epsilon  \tfrac{2 +4 i}{5} \e_2= \bc_\epsilon + \epsilon i \e_2$
and calculate 
%	\[
%\pi_{\geq 2} 
%		\frac{\partial F}{\partial \alpha}(x)  +   \epsilon  \tfrac{2 +4 i}{5}  \e_2  =
%	U_\omega (c-\bc_\epsilon) + \epsilon  (e^{-i \omega} +i)\e_2 + \epsilon \pi_{\geq 2} L_\omega c + \epsilon  \pi_{\geq 2} ([ U_\omega c] * c ).
%	\]
%\note[JB]{I don't see that this is correct. It seems a term is missing:}
	\[
\pi_{\geq 2} 
		\frac{\partial F}{\partial \alpha}(x)  +   \epsilon  \tfrac{2 +4 i}{5}  \e_2  =
	U_\omega (c-\bc_\epsilon) + (1+ e^{-2i\omega}) \bc_\epsilon + \epsilon  (e^{-i \omega} +i)\e_2 + \epsilon \pi_{\geq 2} L_\omega c + \epsilon  \pi_{\geq 2} ([ U_\omega c] * c ).
	\]
By using Proposition~\ref{p:severalnorms} and Lemma~\ref{lem:deltatheta}, we obtain the estimate
%\begin{alignat*}{1}
%		\left\| A_{0,*}^{-1} \pi_{\geq 2} \left(
%				\frac{\partial F}{\partial \alpha}(x)  +   \epsilon  \tfrac{2 +4 i}{5}  \e_2  \right) \right\|
%	 &\leq \| A_{0,*}^{-1} \| \left( r_c + \epsilon | e^{-i \omega } +i| + 4 \epsilon \dc + \epsilon \dc^2 \right) \\
%	&\leq \frac{2}{\pi \sqrt{5}}\left( r_c + \epsilon \dt + \epsilon \dc (4 + \dc )  \right) .
%	\end{alignat*}
%\note[JB]{Because of the factor 2 in the norm, and the term I thought was missing, I think it should be }
%\note[J]{Yes, a term was missing.} 
% \begin{alignat*}{1}
% 		\left\| A_{0,*}^{-1} \pi_{\geq 2} \left(
% 				\frac{\partial F}{\partial \alpha}(x)  +   \epsilon  \tfrac{2 +4 i}{5}  \e_2  \right) \right\|
% 	 &\leq \| A_{0,*}^{-1} \| \left( r_c + 2 \epsilon  \frac{\sqrt{20}}{5} |1+ e^{-2i\omega} | +  2 \epsilon | e^{-i \omega } +i| + 4 \epsilon \dc + \epsilon \dc^2 \right) \\
% 	&\leq \frac{2}{\pi \sqrt{5}}\left( r_c + \epsilon \dtt \frac{4}{\sqrt{5}} + 2 \epsilon \dt + \epsilon \dc (4 + \dc )  \right) .
% 	\end{alignat*}
% 	\note[J]{Rearranged some terms}
	\begin{alignat*}{1}
	\left\| A_{0,*}^{-1} \pi_{\geq 2} \left(
	\frac{\partial F}{\partial \alpha}(x)  +   \epsilon  \tfrac{2 +4 i}{5}  \e_2  \right) \right\|
	&\leq \| A_{0,*}^{-1} \| \left( r_c +  \dc^0 |1+ e^{-2i\omega} | +  2 \epsilon | e^{-i \omega } +i| + 4 \epsilon \dc + \epsilon \dc^2 \right) \\
	&\leq \frac{2}{\pi \sqrt{5}}\left( r_c +  2 \dw (  \dc^0 +  \epsilon )  + \epsilon \dc (4 + \dc )  \right) .
	\end{alignat*}
\end{proof}
%%%%%%%%%%%%%%%%%%%%%%%%%%%%%%%%%%%%%%%%%%%%%%%%%%%%%%%%%%%%%%%%%%%%%%%%%%%%




\begin{proposition}
			\label{prop:Zsw}
Define 
%	\begin{alignat}{1}
%		f_{*,\omega} &:=
%		\frac{5}{2 \pi}  r_c 
%		+
%		\tfrac{\epsilon }{\sqrt{5}} \left(  \dt + \tfrac{2}{\pi}\da \right) \nonumber
%		+
%		\tfrac{2}{\pi} \left( \tfrac{5}{4} (\pp + \da ) r_c + \tfrac{4}{5}\da \epsilon \right) 
%		+
%		\tfrac{4 \epsilon}{5} (\dtt) 
%		\\& \qquad + 
%	   \epsilon  \tfrac{2}{ \pi}(\pp + 	 \da ) \left( \frac{1}{\sqrt{5}} ( \dc + r_c) + \frac{5}{4} \left(  \dc  + \tfrac{3}{2}r_c \right)  +  \frac{ \rho \dc   }{\sqrt{5}}\right) . \label{e:fstaromega}
%	\end{alignat}
%	\note[J]{Proposed Change}
%		\begin{alignat}{1}
%		f_{*,\omega} &:=
%		\frac{5}{2 \pi}  r_c 
%		+
%		\tfrac{2}{\sqrt{5}}  \epsilon \left(  \dw + \tfrac{2}{\pi}\da \right) \nonumber
%		+
%		\tfrac{5}{2\pi} \left( \pp r_c + \da ( r_c + \dc) \right) 
%		+
%		\tfrac{8}{5}  \epsilon \dw
%		\\& \qquad + 
%		  \tfrac{2}{ \pi} \epsilon(\pp + 	 \da ) \left( \frac{1}{\sqrt{5}} ( \dc + r_c) + \frac{5}{4} \left(  \dc  + \tfrac{3}{2}r_c \right)  +  \frac{ \rho \dc   }{\sqrt{5}}\right) . \label{e:fstaromega}
%		\end{alignat}
%	\note[JB]{Proposed Rearrangement}
		\begin{alignat}{1}
		f_{*,\omega} &:=
		\tfrac{5}{2 \pi}  (1+ \pp)  r_c 
		+
		\tfrac{2}{\sqrt{5}}  \epsilon \left( (1+\tfrac{4}{\sqrt{5}}) \dw + \tfrac{2}{\pi}\da \right) \nonumber
		+
		\tfrac{5}{2\pi} \da ( r_c + \dc) 
		\\& \qquad + 
		  \tfrac{2}{ \pi} \epsilon(\pp + 	 \da ) \left( \frac{1}{\sqrt{5}} ( \dc + r_c) + \frac{5}{4} \left(  \dc  + \tfrac{3}{2}r_c \right)  +  \frac{ \rho \dc   }{\sqrt{5}}\right) . \label{e:fstaromega}
		\end{alignat}
%
%	\[
%	f_{*,\omega} :=  \frac{5}{2 \pi} ( 1 + \alpha) r_c +  \epsilon \frac{2}{\pi \sqrt{5}} \left( \da + \pp \dt + \alpha   \rho \dc \right) + \frac{2 \alpha \epsilon}{\pi} \left(  \frac{1}{\sqrt{5}} ( \dc + r_c) + \frac{5}{4} \left(  \dc  + \frac{3}{2}r_c \right)  \right)  
%	\]
	Then for all $x= (\alpha,\omega,c) \in B_\epsilon(r,\rho)$
	\[
	f_{*,\omega} \geq  
	 \left\| A_{0,*}^{-1}  \pi_{\geq 2}\left( \frac{\partial F}{\partial \omega}(x) -  \epsilon \left[ \tfrac{4-3\pi}{10} + \tfrac{2(2 + \pi)}{5}i \right] \e_2  \right) \right\| .  
	\]
\end{proposition}

\begin{proof}
We note that 
$
\epsilon  \left[ \tfrac{4-3\pi}{10} + \tfrac{2(2 + \pi)}{5}i \right] \e_2
= i (2+\pi) \bc_\epsilon - \pp \epsilon \e_2
$
and calculate 
\begin{alignat*}{1}
\pi_{\geq 2} \frac{\partial F}{\partial \omega}(x) -  \epsilon \left[ \tfrac{4-3\pi}{10} + \tfrac{2(2 + \pi)}{5}i \right] \e_2   &=
		i K^{-1} ( I - \alpha U_{\omega} ) c  -
		 i  \alpha \epsilon e^{-i \omega} \e_2 + 
		\alpha \epsilon \pi_{\geq 2} L_{\omega}' c \\
		&\qquad - i \alpha \epsilon \pi_{\geq 2} ([ K^{-1} U_\omega c ] *c  ) - i K^{-1} ( I - \pp U_{\omega_0}) \bc_\epsilon +  \pp \epsilon \e_2\\
		&= i K^{-1} ( c - \bc_\epsilon) - \epsilon  (i  \alpha e^{-i \omega}  -  \pp ) \e_2\\
		&\qquad
		-i K^{-1} \left[  U_\omega \left(  \pp ( c - \bc_\epsilon) + ( \alpha - \pp) c  \right) + \left( U_\omega - U_{\omega_0} \right) \pp \bc_\epsilon) \right]
		\\
		&\qquad\qquad
			+ \alpha \epsilon \pi_{\geq 2} L_{\omega}' c 
			- i \alpha \epsilon \pi_{\geq 2} ([ K^{-1} U_\omega c ] *c ) .
\end{alignat*}	
Applying the operator $A_{0,*}^{-1}$ to this expression, we obtain (with $\hat{U}$ defined in~\eqref{e:defUhat})
%\note[J]{Yes, the term $(\alpha-\pp)c_2$ really should have been $(\alpha-\pp)c$.}
%	\[
%	A_{0,*}^{-1} \left( \frac{\partial F}{\partial \omega} ( x_\epsilon + a) - A   \right) =
%	\frac{2 	}{\pi} \hat{U}( I - \alpha U_{\omega}) (c - c_2(\epsilon)) + \epsilon A_{0,*}^{-1} \left( -i  [(\alpha-\pp) e^{_i \omega}]_2 -i [\pp ( e^{-i \omega} + i)]_2 + \alpha L_{\omega}' c - i \alpha [K^{-1} U_{\omega} c]* c 
%	\right)
%	\]
%	
\begin{alignat*}{1}
A_{0,*}^{-1}  \pi_{\geq 2}\left( \frac{\partial F}{\partial \omega}(x) -  \epsilon \left[ \tfrac{4-3\pi}{10} + \tfrac{2(2 + \pi)}{5}i \right] \e_2  \right) 
			&= \frac{2}{\pi} \hat{U}  ( c - \bc_\epsilon) 
			- \frac{2 \epsilon}{i \pi} \hat{U} K  (i  \alpha  e^{-i \omega}  -  \pp  ) \e_2 \\
			&\qquad
			-\frac{2}{\pi} \hat{U} \left[  U_\omega \left(  \alpha ( c - \bc_\epsilon) + ( \alpha - \pp) c  \right) \right]
		    \\	&\qquad\qquad
			-\frac{2}{\pi} \hat{U} \left( U_\omega - U_{\omega_0} \right) \pp \bc_\epsilon  
			\\
			&\qquad\qquad\qquad 
			+ \frac{2 \alpha \epsilon}{i \pi} \hat{U} K \pi_{\geq 2} \left(  L_{\omega}' c - i [ K^{-1} U_\omega c ] *c  \right)  .
	\end{alignat*}
We use the triangle inequality to estimate its norm, splitting it into the  five pieces:
% \note[JB]{There seems to be a factor 2 missing in the second and fourth estimate (due to norm); not implemented yet}
% \note[JB]{I think the second term of the third estimate is not correct: I think it should be $\frac{2}{\pi}\frac{5}{4} \da \dc$. But I did not implement it; I may be overlooking something.} \note[J]{I agree with these changes except for the factor of 2 in the fourth estimate. }
% 	\begin{alignat*}{1}
% 	\left\|	\frac{2}{\pi} \hat{U}  ( c - \bc_\epsilon) \right\|
% 				&\leq \frac{2}{\pi} \frac{5}{4} r_c
% 				= \frac{5}{2 \pi}  r_c \nonumber \\
% 				%
% 				%
% \left\|		- \frac{2 \epsilon}{i \pi} \hat{U} K  (i  \alpha  e^{-i \omega}  -  \pp  ) \e_2  \right\|
% 				&\leq   \frac{2 \epsilon }{\pi} \frac{1}{\sqrt{5}} \left( \pp \dw + \da \right)
% 				=   \tfrac{ \epsilon }{\sqrt{5}} \left(  \dt + \tfrac{2}{\pi}\da \right) \nonumber \\
% 				%
% 				%
% 	\left\| 	-\tfrac{2}{\pi} \hat{U} \left[  U_\omega \left(  \alpha ( c - \bc_\epsilon) + ( \alpha - \pp) c  \right) \right]  \right\|
% 				& \leq \tfrac{2}{\pi} \left( \tfrac{5}{4} \alpha r_c + \tfrac{2}{\sqrt{5}}\da \tfrac{2 \epsilon}{\sqrt{5}} \right)
% 				= \tfrac{2}{\pi} \left( \tfrac{5}{4} \alpha r_c + \tfrac{4}{5}\da \epsilon \right) \nonumber \\
% 				%
% 				%
% 		\left\| -\tfrac{2}{\pi} \hat{U}
% 		\left( U_\omega - U_{\omega_0} \right) \pp \bc_\epsilon   \right\|
% 				&\leq  \tfrac{2}{\pi}  \tfrac{2}{\sqrt{5}} \dtt \pp \tfrac{2 \epsilon}{\sqrt{5}}
% 				=   \tfrac{4 \epsilon}{5} \dtt \nonumber \\
% 				%
% 				%
% 		\left\|\tfrac{2 \alpha \epsilon}{i \pi} \hat{U} K \pi_{\geq 2} \left(  L_{\omega}' c - i [ K^{-1} U_\omega c ] *c  \right)  \right\|
% 		&\leq \frac{2 \alpha \epsilon}{ \pi}  \left( \| \hat{U}  K \pi_{\geq 2} L_{\omega}' c  \| +  \frac{ \rho \dc   }{\sqrt{5}}\right) ,
% 	\end{alignat*}
%
% \note[J]{Below are my changes}
	\begin{alignat*}{1}
	\left\|	\frac{2}{\pi} \hat{U}  ( c - \bc_\epsilon) \right\|
	&\leq \frac{2}{\pi} \frac{5}{4} r_c 
	= \frac{5}{2 \pi}  r_c \nonumber \\
	%
	%
	\left\|		- \frac{2 \epsilon}{i \pi} \hat{U} K  (i  \alpha  e^{-i \omega}  -  \pp  ) \e_2  \right\|
	&\leq   \frac{4 \epsilon }{\pi} \frac{1}{\sqrt{5}} \left( \pp \dw + \da \right)  
	=   \tfrac{2 \epsilon }{\sqrt{5}} \left(  \dw + \tfrac{2}{\pi}\da \right) \nonumber \\
	%
	%
	\left\| 	-\tfrac{2}{\pi} \hat{U} \left[  U_\omega \left(  \alpha ( c - \bc_\epsilon) + ( \alpha - \pp) c  \right) \right]  \right\|
	& \leq \tfrac{2}{\pi}  \tfrac{5}{4} \left(  (\pp + \da) r_c + \da \dc  \right)
	= \tfrac{5}{2\pi} \left( \pp r_c + \da ( r_c + \dc) \right) \nonumber \\
	%
	%
	\left\| -\tfrac{2}{\pi} \hat{U} 
	\left( U_\omega - U_{\omega_0} \right) \pp \bc_\epsilon   \right\| 
	&\leq  \tfrac{2}{\pi}  \tfrac{2}{\sqrt{5}} (2 \dw)  \pp \tfrac{2 \epsilon}{\sqrt{5}} 
	=   \tfrac{8 \epsilon}{5} \dw \nonumber \\
	%
	%
	\left\|\tfrac{2 \alpha \epsilon}{i \pi} \hat{U} K \pi_{\geq 2} \left(  L_{\omega}' c - i [ K^{-1} U_\omega c ] *c  \right)  \right\|
	&\leq \frac{2 \alpha \epsilon}{ \pi}  \left( \| \hat{U}  K \pi_{\geq 2} L_{\omega}' c  \| +  \frac{ \rho \dc   }{\sqrt{5}}\right) ,
	\end{alignat*} 
where we have used Proposition~\ref{p:severalnorms} and Lemma~\ref{lem:deltatheta}.
%%	Multiplying by the inverse and taking norms 
%%	\begin{eqnarray}
%%	\left\| A_{0,*}^{-1}  \frac{\partial F}{\partial \omega} ( x_\epsilon + a) - A   \right\| &\leq&
%%	\frac{5}{2 \pi} ( 1 + \alpha) r_c +  \epsilon \| A_{0,*}^{-1} \| ( \da + \pp | e^{-i \omega} +i |) + \frac{2 \alpha \epsilon}{\pi} \| \hat{U} K L_{\omega}' c \| + \alpha \epsilon \| A_{0,*}^{-1} \|  \rho \dc \nonumber \\
%%	&\leq & 
%%	\frac{5}{2 \pi} ( 1 + \alpha) r_c +  \epsilon \frac{2}{\pi \sqrt{5}} ( \da + \pp \dt + \alpha   \rho \dc ) + \frac{2 \alpha \epsilon}{\pi} \| \hat{U} K L_{\omega}' c \| \nonumber 
%%	\end{eqnarray}
Finally, we estimate 
	\begin{alignat}{1}
	\left\| \hat{U} K \pi_{\geq 2} L_{\omega}' c \right\| &= 
	\left \| \hat{U}  K \pi_{\geq 2} \left(- i \sigma^+( e^{- i \omega} I + K^{-1} U_{\omega}) + i \sigma^-(e^{i \omega} I - K^{-1} U_{\omega}) \right) c \right\| \nonumber \\
	&\leq \left \| \hat{U}  K \pi_{\geq 2}( \sigma^+ + \sigma^- ) c  \right \| + 
	\left \| \hat{U} \pi_{\geq 2} K ( \sigma^+ + \sigma^- ) K^{-1}  U_{\omega} c  \right \|  \nonumber  \\
	&\leq  \frac{1}{\sqrt{5}} ( \| \sigma^+ c\| + \|\pi_{\geq 2}\sigma^- c\|) + \frac{5}{4} \left( \| K \sigma^+ K^{-1} \| \dc  +  \| \pi_{\geq 2} K \sigma^- K^{-1} \| r_c \right) \nonumber  \\
	&\leq  \frac{1}{\sqrt{5}} ( \dc + r_c) + \frac{5}{4} \left(  \dc  + \frac{3}{2}r_c \right)  .  \label{e:longestimate}
	\end{alignat}
Hence, with $f_{*,\omega}$ as defined in~\eqref{e:fstaromega},
% 	So if we define $f_{*,\omega}$ as below
% 	\begin{eqnarray}
% f_{*,\omega} &:=&
% 	\frac{5}{2 \pi}  r_c
% +
%  \tfrac{\epsilon }{\sqrt{5}} \left(  \dt + \tfrac{2}{\pi}\da \right) \nonumber
% +
%  \tfrac{2}{\pi} \left( \tfrac{5}{4} \alpha r_c + \tfrac{4}{5}\da \epsilon \right)
%  +
%   \tfrac{4 \epsilon}{5} (\dtt)
%   \\&&+
%   \frac{2 \alpha \epsilon}{ \pi}  \left( \frac{1}{\sqrt{5}} ( \dc + r_c) + \frac{5}{4} \left(  \dc  + \tfrac{3}{2}r_c \right)  +  \frac{ \rho \dc   }{\sqrt{5}}\right)
% 	\end{eqnarray}
%
it follows that 	
	\[
	 \left\| A_{0,*}^{-1}  \pi_{\geq 2}\left( \frac{\partial F}{\partial \omega}(x) -  \epsilon \left[ \tfrac{4-3\pi}{10} + \tfrac{2(2 + \pi)}{5}i \right] \e_2  \right) \right\| \leq 	f_{*,\omega} .
		\]
	
	
	
%%%%	\begin{eqnarray}
%%%%	\left\| A_{0,*}^{-1}  \frac{\partial F}{\partial \omega} ( x_\epsilon + a) - A   \right\| &\leq & 
%%%%	\frac{5}{2 \pi} ( 1 + \alpha) r_c +  \epsilon \frac{2}{\pi \sqrt{5}} \left( \da + \pp \dt+ \alpha   \rho \dc \right) + \frac{2 \alpha \epsilon}{\pi} \left(  \frac{1}{\sqrt{5}} ( \dc + r_c) + \frac{5}{4} \left(  \dc  + \frac{3}{2}r_c \right)  \right)  \nonumber
%%%%	\end{eqnarray}
%%%%	Hence
%%%%	\[
%%%%	f_{*,\omega} \leq  \frac{5}{2 \pi} ( 1 + \alpha) r_c +  \epsilon \frac{2}{\pi \sqrt{5}} \left( \da + \pp \dt + \alpha   \rho \dc \right) + \frac{2 \alpha \epsilon}{\pi} \left(  \frac{1}{\sqrt{5}} ( \dc + r_c) + \frac{5}{4} \left(  \dc  + \frac{3}{2}r_c \right)  \right)  
%%%%	\]
\end{proof}


%%%%%%%%%%%%%%%%%%%%%%%%%%%%%%%%%%%%%%%%%%%%%%%%%%%%%%%%%%%%%%%%%%%%%%%%%%%%

\begin{proposition}
			\label{prop:Zsc}
Define 
% \note[J]{Below is the old bound.}
% 	\[
% 	f_{*,c} := \left[ \frac{5}{2} ( \tfrac{1}{2} + \tfrac{1}{\pi}) \dw +  \frac{\da }{\sqrt{5}} \right]
% 	+\epsilon \left[ \frac{8}{\pi \sqrt{5}} \da + \frac{2}{\sqrt{5}} \dt  + \frac{25}{8} \dw + \frac{2 (\pp+\da)  \dc}{\pi \sqrt{5}} \right]  .
% 	\]
% \note[J]{Below is the new bound. }
	\[
	f_{*,c} :=	\left[ \frac{5}{2} \left( \frac{1}{2} + \frac{1}{\pi} \right) \dw +  \frac{\da }{\sqrt{5}} \right] 
 +\epsilon \left[ \frac{8}{\pi \sqrt{5}} \da + \left(  \frac{2}{\sqrt{5}}   + \frac{25}{8} \right) \dw + \frac{4 (\pp+\da)  \dc}{\pi \sqrt{5}} \right] .
	\]
	Then for all $x= (\alpha,\omega,c) \in B_\epsilon(r,\rho)$
	\[
	f_{*,c} \geq    \left\|  A_{0,*}^{-1} \pi_{\geq 2} \left( \frac{\partial F}{\partial c}(x) \cdot b  - (A_{0,*} + \epsilon A_{1,*}) b \right) 
	\right\| , \qquad\text{for all $b\in\ell^K_0$ with $\|b\| \leq 1$}.
	\]
\end{proposition}


\begin{proof}
We write $A_* := A_{0,*} + \epsilon A_{1,*}$ and calculate
\begin{alignat*}{1}
\frac{\partial F}{\partial c} (x) \cdot b - A_* b 
& = 
 \bigl[( i \omega K^{-1} + \alpha U_{\omega}) - ( i \pp K^{-1} + \pp U_{\omega_0}) \bigr] b  + \alpha \epsilon  L_\omega b  - \pp \epsilon L_{\omega_0} b 
 \\& \qquad
	+ \alpha \epsilon \left[ [ U_\omega b] * c + [U_{\omega} c ]*b \right]
 \\ & =
	\bigl[ i ( \omega - \pp ) K^{-1} + ( \alpha - \pp) U_{\omega} + \pp ( U_{\omega} - U_{\omega_0}) \bigr] b 
	\\ & \qquad 
	+ \epsilon \bigl[ ( \alpha - \pp) L_{\omega} + \pp ( L_{\omega} - L_{\omega_0}) \bigr] b 
%	\\ & \qquad\qquad 
+ \alpha \epsilon \left( [U_{\omega } b] * c + [ U_{\omega } c ]*b \right) . 
\end{alignat*}
Hence, for $\|b\| \leq 1$, 
\begin{alignat}{1}
	\left\| A_{0,*}^{-1} \pi_{\geq 2} \left(   \frac{\partial F}{\partial c} (x) \cdot b -  A_* b  \right)  \right\|
	 &\leq 
	 \dw  \| A_{0,*}^{-1} K^{-1} \| + \pp \da \| A_{0,*}^{-1}  \| + \pp \| A_{0,*}^{-1}  ( U_{\omega} - U_{\omega_0}) \| \nonumber  \\
	& \qquad
	 + \epsilon  \left[ 4  \da \|  A_{0,*}^{-1}   \| + \pp \| A_{0,*}^{-1} \pi_{\geq 2} ( L_{\omega} - L_{\omega_0}) \| 
%	\\ & \qquad \qquad 
	+  2 \alpha \dc \| A_{0,*}^{-1}  \|  \right]  \label{e:intermediate},
\end{alignat}	
where all norms should be interpreted as operators on $\ell^1_0$.
	Since
$\frac{\partial U_\omega}{\partial  \omega} = - i K^{-1} U_{\omega}$
and $ A_{0,*}^{-1} = \frac{2}{i\pi} \hat{U} K$, it follows from Proposition~\ref{p:severalnorms} that  
\begin{equation}\label{e:AUoUo0}
	\| A_{0,*}^{-1}  (U_{\omega} - U_{\omega_0})  \| \leq  \frac{2}{\pi}  \dw \| \hat{U} \| 
	= \frac{5}{2 \pi } \dw .
\end{equation}
Next, we compute
	\begin{alignat*}{1}
	L_{\omega} - L_{\omega_0}  &= \sigma^+ \left[ (e^{-i \omega} + i) I + (U_{\omega} - U_{\omega_0})\right] + \sigma^- \left[ (e^{i \omega} - i) I + (U_{\omega} - U_{\omega_0}) \right] \\
	&=  (e^{-i \omega} + i)  \sigma^+ - i e^{i\omega} (i+e^{-i \omega})\sigma^- 
	+ (\sigma^+ + \sigma^-) (U_{\omega} - U_{\omega_0}) .
\end{alignat*}
Analogous to~\eqref{e:longestimate} and~\eqref{e:AUoUo0} we infer that
\begin{equation*}
	\|  A_{0,*}^{-1} \pi_{\geq 2} ( L_{\omega} - L_{\omega_0} ) \| \leq  \frac{4}{\pi \sqrt{5}} |i+ e^{-i \omega} | + \frac{5}{\pi} \| \hat{U} \| \dw \\
	\leq  \frac{4 }{\pi \sqrt{5}}\dw    + \frac{25}{4 \pi}  \dw .
\end{equation*} 
Finally, by putting all estimates together and once again using Proposition~\ref{p:severalnorms}, it follows from~\eqref{e:intermediate} that 
%\note[JB]{Probably factor 2 missing in final term.}\note[J]{Added factor of 2. }
	\begin{alignat*}{1}
\left\| A_{0,*}^{-1} \pi_{\geq 2} \left(   \frac{\partial F}{\partial c} (x) \cdot b -  A_* b  \right)  \right\|
	% &\leq
	% \left[ \dw  \| A_{0,*}^{-1} K^{-1} \| + \pp \da \| A_{0,*}^{-1}  \| + \pp \| A_{0,*}^{-1}  ( U_{\omega} - U_{\omega_0}) \| \right] \nonumber \\
	% & + \epsilon  \left[ 4  \da \|  A_{0,*}^{-1}   \| + \pp \| A_{0,*}^{-1}  ( L_{\omega} - L_{\omega_0}) \| + \alpha \dc \| A_{0,*}^{-1}  \|  \right] \\
	&\leq
	 \left[ \frac{5}{2} \left( \frac{1}{2} + \frac{1}{\pi} \right) \dw +  \frac{\da }{\sqrt{5}} \right] 
	 \\
	& \qquad  +\epsilon \left[ \frac{8}{\pi \sqrt{5}} \da + \left(  \frac{2}{\sqrt{5}}   + \frac{25}{8} \right) \dw + \frac{4 (\pp+\da)  \dc}{\pi \sqrt{5}} \right] .
	\end{alignat*}
\end{proof}

%%%%%%%%%%%%%%%%%%%%%%%%%%%%%%%%%%%%%%%%%%%%%%%%%%%%%%%%%%%%%%%%%%%%%%%%%%%%



%!TEX root = hopfwright.tex

%%%%%%%%%%%%%%%%%%
%%% Appendix E %%%
%%%%%%%%%%%%%%%%%%

\section{Appendix: A priori bounds on periodic orbits}
\label{appendix:aprioribounds}

In order to isolate periodic orbits, we need to separate them from the trivial solution. In this appendix we prove some lower bounds on the size of periodic orbits. First we work in the original Fourier coordinates. Then we derive refined bounds in rescaled coordinates.

Recall that periodic orbits of Wright's equation corresponds to  zeros of 
$G(\alpha,\omega,\c)=0$, as defined in~\eqref{e:defG}. Clearly $G(\alpha,\omega,0)=0$ for all frequencies $\omega>0$ and parameter values $\alpha>0$. There are bifurcations from this trivial solution for $\alpha=\alpha_n:=\pp(4n+1)$ for all $n\geq 0$. The corresponding natural frequency is $\omega=\alpha_n$, but there are bifurcations for any $\omega = \alpha_n/ \tilde{n}$ with $\tilde{n} \in \N$ as well, which are essentially copies of the primary bifurcation. The following proposition quantifies that away from these bifurcation points the trivial solution is isolated.
% \begin{proposition}
% 	\label{prop:zeroneighborhood}
% 	Suppose $G(\alpha,\omega,\c)=0$ for some $\alpha,\omega >0$.
% Then either $\c \equiv 0$ or
% 	\begin{equation}
% 	\| \c \|^2 \geq \frac{1}{4\alpha^2} \min_{k \in \N} (\alpha-k\omega)^2 + 2 \alpha k \omega ( 1- \sin k \omega )
% 	\end{equation}
% \end{proposition}
%
% \begin{proof}
% We fix $\alpha,\omega>0$ and define
% \[
%   \beta_1 := \min_{k \in \N} (\alpha-k\omega)^2 + 2 \alpha k \omega ( 1- \sin k \omega ).
% \]
% If $\beta_1=0$ there is nothing to prove. From now on we assume that $\beta_1>0$.
% We define a Newton-like map
% $N : \ell^1 \to \ell^1$ by
% \begin{equation}
% N(\c) := \c - (i \omega K^{-1} + \alpha U_{\omega})^{-1} G(\alpha,\omega,\c).
% \end{equation}
% We note that $i \omega K^{-1} + \alpha U_{\omega}$, which is the derivative of $G$ at $\c=0$, is invertible, since
% for any $k \in \N$
% \begin{alignat*}{1}
%   |ik\omega + \alpha e^{-i k \omega}|^2
%   & =  (\alpha \cos k \omega )^2 + ( \omega - \alpha \sin  k \omega )^2  \\
%   &  = (k \omega)^2 + \alpha^2 - 2 \alpha  k \omega \sin k \omega   \\
% &= (\alpha - k\omega)^2 + 2 \alpha k \omega ( 1 - \sin k \omega)  \\
% &\geq \beta_1 >0.
% \end{alignat*}
% Fixed points of $N$ thus correspond to zeros of $G$.
% Naturally, $ N(0) =0$.
% Next we show that $N$ is a contraction map on any ball $B_R := \{ \c \in \ell^1 : \|\c\| \leq R \} $ with $R < \beta_1^{1/2} / (2\alpha)$. We then apply the Banach fixed point theorem to conclude that there are no nontrivial fixed points with $\|\c\| < \beta_1^{1/2} / (2\alpha)$.
% The derivative of $N$ is
% \begin{equation}
% DN(\c)\tilde{\c} =  - \alpha (i \omega K^{-1} + \alpha U_{\omega})^{-1}  (
% [U_\omega \c ]  * \tilde{\c} ) .
% \end{equation}
% We estimate
% \[
% \| DN(\c)\tilde{\c} \| \leq 2 \alpha \|\c \| \cdot \|\tilde{\c}\| \cdot
% \| \omega K^{-1} +  \alpha U_{\omega})^{-1} \| .
% \]
% Since $\| ( \omega K^{-1} + \alpha U_{\omega})^{-1} \| = \beta_1^{-1/2}$, we find that
% $\| DN(\c)\| \leq 2 \alpha R \beta_1^{-1/2} < 1$ for all $\c \in B_R$.
% Hence $N$ is a contraction for $R < \beta_1^{1/2}/ (2\alpha)$.
% \end{proof}

\begin{proposition}
	\label{prop:zeroneighborhood2}
	Suppose $G(\alpha,\omega,\c)=0$ for some $\alpha,\omega >0$. 
Then either $\c \equiv 0$ or 
	\begin{equation}\label{e:minoverk}
	\| \c \| \geq \min_{k \in \N}  \sqrt{ \left(1-k \,\frac{\omega}{\alpha} \right)^2 + 2  k \, \frac{\omega}{\alpha} \bigl( 1- \sin k \omega \bigr)} .
	\end{equation}	
\end{proposition}

\begin{proof}
We fix $\alpha,\omega>0$ and define 
\[
  \beta_1 := \min_{k \in \N} \, (\alpha-k\omega)^2 + 2 \alpha k \omega ( 1- \sin k \omega ).
\]
If $\beta_1=0$ then there is nothing to prove. From now on we assume that $\beta_1>0$. 
We recall that 
\[ G(\alpha,\omega,\c) = (i \omega K^{-1} + \alpha U_{\omega}) \c + \alpha \left[U_\omega \, \c \right] * \c .
\]
We note that $i \omega K^{-1} + \alpha U_{\omega}$ is invertible, since
for any $k \in \N$
\begin{alignat*}{1}
  |ik\omega + \alpha e^{-i k \omega}|^2
  & =  (\alpha \cos k \omega )^2 + ( \omega - \alpha \sin  k \omega )^2  \\
  &  = (k \omega)^2 + \alpha^2 - 2 \alpha  k \omega \sin k \omega   \\
&= (\alpha - k\omega)^2 + 2 \alpha k \omega ( 1 - \sin k \omega)  \\
&\geq \beta_1 >0.
\end{alignat*}
We may thus rewrite $G(\alpha,\omega,\c) = 0$ as 
\begin{equation}\label{e:quadratic}
 \c = - \alpha  (i \omega K^{-1} + \alpha U_{\omega})^{-1} ( \left[U_\omega \, \c \right] * \c ) .
\end{equation}
Since $\| ( \omega K^{-1} + \alpha U_{\omega})^{-1} \| = \beta_1^{-1/2}$
and $\| \left[U_\omega \, \c \right] * \c \| \leq \| \c \|^2 $,
we infer from~\eqref{e:quadratic} that
\[
  \| \c \| \leq  \alpha \beta_1^{-1/2} \|\c\|^2.
\]
We conclude that either $\c \equiv 0$ or $\| \c \| \geq \beta_1^{1/2} /\alpha $.
\end{proof}

\begin{proposition}
	\label{prop:G1Minimizer}
	Suppose that $ \omega \geq 1.1$ and $ \alpha \in (0,2]$. Define 
	\begin{equation}
g_k(\omega,\alpha) =		\left(1-k \,\tfrac{\omega}{\alpha} \right)^2 + 2  k \, \tfrac{\omega}{\alpha} \bigl( 1- \sin k \omega \bigr) .
	\end{equation}
Then $ g_1 <g_k$ for all $ k \geq 2$. 
\end{proposition}

\begin{proof}
%%%	We first prove that $ g_k > g_1$ for all $ k\geq 3$.   
%%%	For all $k$ we may estimate $ g_k > (1 - k \tfrac{\omega}{\alpha})^2$. 	
%%%	If $ \omega \in [1.1,2]$  then $1- \sin ( \omega ) \in [0,0.11]$, and so 
%%%	\[
%%%	g_1 < ( 1- \tfrac{\omega}{\alpha})^2 + 0.22 \tfrac{\omega}{\alpha}
%%%	\]
%%%	If we write $ x = \tfrac{\omega}{\alpha}$ then these estimates produce:
%%%	\begin{eqnarray}
%%%		g_k &>& (kx)^2 - 2 k x +1  \\
%%%		g_1 &<& x^2 - 1.78x +1 
%%%	\end{eqnarray}
%%%	Hence we can show $ g_k > g_1$ by showing 
%%%	\[
%%%	x - 1.78 < k^2 x - 2 k
%%%	\]
%%%	or equivalently 
%%%\begin{equation}
%%%		(2k-1.78) < (k^2-1)x  \label{eq:G3G1}
%%%\end{equation}
%%%	For our range of $ \alpha$ and $\omega$ it follows that $  x \in [0.55, 4/3]$.  
%%%	Since Equation \ref{eq:G3G1} is true for $ x = 0.55$ and $ k \geq 3$, then it follows that $ g_1 < g_k$ for all $ k \geq 3$. 
%%%	
%%%	We now show that $ g_1 < g_2$. 
%%%	
%%	
%%	
	
This is equivalent to showing that 
\[
(1- \tfrac{\omega}{\alpha})^2 + 2 \tfrac{\omega}{\alpha} ( 1 - \sin \omega ) 
<
(1- k \tfrac{\omega}{\alpha})^2 + 2k  \tfrac{\omega}{\alpha} ( 1 - \sin k\omega )  
\qquad\text{for } k\geq 2.
\]
Making the substitution $ x = \tfrac{\omega}{\alpha}$, we can simplify this to the equivalent inequality 
% \begin{eqnarray}
% (1 - 2 x + x^2 ) + 2 x ( 1 - \sin \omega )
% &<&
% (1 - 2k x + k^2 x^2 ) + 2k x ( 1 - \sin k\omega )
% \\
% 2 x ( 1 - \sin \omega )
% &<&
% - 2(k-1) x + (k^2-1)x^2  + 2k x ( 1 - \sin 2\omega )
% \\
% -2  \sin \omega
% &<&
\[
 (k^2-1) x  + 2  \sin \omega - 2k  \sin k\omega >0 .
\]
% \end{eqnarray}
Since $\alpha \leq 2$, we have $x\geq \omega/2$. Hence it suffices to prove that
%
%
% $ x = \tfrac{\omega}{\alpha} $ is minimized for large $\alpha$, we may simplify this as below using our initial range of $ \alpha \in [1.5,2]$.
\begin{equation}
  h_k(\omega) := \frac{k^2-1}{2} \omega + 2 \sin \omega  - 2 k \sin k\omega  >0 
  \qquad\text{for all } k\geq 2.
\label{eq:GminRedux}
\end{equation}
We first consider $k=2$. It is clear that $h_2(\omega) > 0$ for $\omega > 4$.
We note that $h_2$ has a simple zero at $ \omega \approx 1.07146$ and it is easy to check using interval arithmetic that $h_2(\omega)$ is positive for $\omega \in [1.1,4]$. Hence $h_2(\omega) > 0$ for all $\omega \geq 1.1$.


For $k=3$ and $k=4$ we can repeat a similar argument. For $k \geq 5$ it is immediate that $h_k(\omega) > \frac{k^2-1}{2}-2-2k \geq 0$ for $\omega > 1$.
%
% One can further check that the the RHS of Equation \ref{eq:GminRedux} is positive for all $ k \geq 2 $ and $ \omega \in [1.1,2]$.
% Thereby, it follows that
% \[
% g_1(\omega,\alpha) < g_{k} (\omega,\alpha)
% \]
% for $ \alpha \in [1.5,2]$ and $ \omega \in [1.1,2]$.
%
%
%
\end{proof}


As discussed in Section~\ref{s:preliminaries},
the function $G(\alpha,\omega,\c)$ gets replaced by $\tF_\epsilon(\alpha,\omega,\tc)$ in rescaled coordinates. 
In these coordinates we derive a result analogous to Proposition~\ref{prop:zeroneighborhood2} below, see Lemma~\ref{lem:thecone}.
First we bound the inverse of the operator $\B \in B(\ell^1_0)$ defined by
%\marginpar{NOTE THAT $\B = \alpha^{-1} B K $}
\[
  \B:= i \frac{\omega}{\alpha} I +  U_{\omega} K +  \epsilon L_{\omega} K,
\]
where $K$, $U_{\omega}$ and $L_\omega$ have been introduced in Section~\ref{s:preliminaries}.
\begin{lemma}\label{lem:gamma}
	Let $\epsilon \geq 0$ and $\alpha,\omega>0$. Let
\[
  \gamma := 	  \frac{1}{2} +
  \epsilon \left( \frac{2}{3} + \max\left\{  \frac{\sqrt{2 - 2 \sin (\omega-\pp) }}{2} ,\frac{2}{3} \right\} \right).
\] 
If $\gamma < \omega / \alpha$ then the operator $\B$ is invertible and the inverse is bounded by
\[
	  \| \B^{-1} \| \leq \frac{1}{\frac{\omega}{\alpha}- \gamma}.
\]
\end{lemma} 

\begin{proof}
Writing 
\[
  \B= i \frac{\omega}{\alpha} \left( I + \frac{\alpha}{i\omega} \left( U_{\omega} +  \epsilon L_{\omega} \right) K \right)
\]
and using a (formal) Neumann series argument, we obtain
% \marginpar{Side remark: did you consider the splitting $\B = i
% \frac{\omega}{\alpha} [I+ \frac{\alpha}{i\omega} \epsilon L_\omega K Q^{-1}] Q
% $ with $Q=I+\frac{\alpha}{i\omega} U_\omega K $ ? Using the estimate from
% Prop~\ref{prop:zeroneighborhood2} this may or may not lead to a better bound.}
\[
  \| \B^{-1} \| \leq \frac{\alpha}{\omega}
   \sum_{n=0}^\infty \left( \frac{ \alpha}{ \omega} \right)^n \|(U_{\omega} + \epsilon L_{\omega}) K\|^n
   \leq \frac{\frac{\alpha}{\omega} }{1- \frac{ \alpha}{ \omega} \|(U_{\omega} + \epsilon L_{\omega}) K\|} 
   = \frac{1}{\frac{\omega}{\alpha}- \|(U_{\omega} + \epsilon L_{\omega}) K\|} .
\]
It remains to prove the estimate $\|(U_{\omega} + \epsilon L_{\omega}) K\| \leq \gamma$.
Then, in particular, for $\gamma < \omega/\alpha$ the formal argument is rigorous.

Recalling that $L_\omega= \sigma^+ (e^{-i\omega} I  + U_\omega) + \sigma^- (e^{i\omega} I  + U_\omega)$, we use the triangle inequality
\[
\|(U_{\omega} + \epsilon L_{\omega}) K\| 
\leq \| U_\omega K \|  + \epsilon \| \sigma^+ (e^{-i\omega} I  + U_\omega) K \| + \epsilon \| \sigma^- (e^{i\omega} I  + U_\omega) K \|,
\] 
and estimate each term separately as operator on $\ell^1_0$. 
We recall the formula~\eqref{e:operatornorm} for the operator norm.
Using that $\| K \tc \| \leq \frac{1}{2} \|\tc\|$ for all $\tc \in \ell^1_0$,
the first term is bounded by $\| U_\omega K \| \leq \frac{1}{2}$.
Since $\sigma^-$ shifts the sequence to the left and we consider the operators acting on $\ell^1_0$, we obtain $\| \sigma^- (e^{i\omega} I  + U_\omega) K \| \leq \frac{2}{3}$.
For the final term, $\| \sigma^+ (e^{-i\omega} I  + U_\omega) K \|$, 
to obtain a slightly more refined estimate,
we first consider the action of $\sigma^+ (e^{-i\omega} I  + U_\omega) K$ 
on $\e_2$. We observe that 
\[ 
  | e^{- i \omega} + e^{-2 i \omega}| = \sqrt{2 - 2 \sin (\omega-\pp) },
\]
hence $\| \sigma^+ (e^{-i\omega} I  + U_\omega) K \e_2 \| \leq 
\sqrt{2 - 2 \sin (\omega-\pp) } $, 
leading to
\[
 \| \sigma^+ (e^{-i\omega} I  + U_\omega) K \| 
 \leq \max\left\{  \frac{\sqrt{2 - 2 \sin (\omega-\pp) }}{2} ,\frac{2}{3} \right\}.
\]
We conclude that 
\[
\|(U_{\omega} + \epsilon L_{\omega}) K\| 
\leq 
  \frac{1}{2} +
  \epsilon \left( \frac{2}{3} + \max\left\{  \frac{\sqrt{2 - 2 \sin (\omega-\pp) }}{2} ,\frac{2}{3} \right\} \right).
\]
\end{proof}

\begin{lemma}\label{lem:thecone}
	Fix $ \epsilon \geq 0$, $\alpha,\omega>0$.
	Assume that $\B$ is invertible.
	Let $b_0$ be a bound on $\| \B^{-1} \|$.
	Define 
	\[
	z^{\pm} = b_0^{-1} \pm \sqrt{b_0^{-2}-  2\epsilon^2 } .
	\]
	Let $ \tc \in \ell^1_0$ be such that $\tF_\epsilon(\alpha, \omega,\tc) = 0$, then either $ \|\tc\| \leq  z^-$ or $  \|\tc\| \geq z^+ $. 
	\noindent
	Additionally, $ \| K^{-1} \tc \| \leq b_0 (2\epsilon^2+ \|\tc\|^2)$.
\end{lemma}

\begin{proof}
	If  $ \tF_\epsilon( \alpha, \omega, \tc) =0$ then it follows that the equations $\pi_c \tF_\epsilon=0$ can be rearranged as 
\begin{equation}\label{e:eBc2}
  \tc = - K \B^{-1} (  \epsilon^2  e^{- i \omega} \e_2 + \ [ U_{\omega} \tc ] * \tc ) .
\end{equation}
	Taking norms, and using that $\| K \tc \| \leq \frac{1}{2} \| \tc\|$ for all $\tc\in \ell^1_0$, we obtain 
\begin{equation}\label{e:quadineq}
\|\tc \|  \leq  \frac{1}{2} \| B^{-1}\| \left( \epsilon^2 \|\e_2\|  + \| [ U_{\omega} \tc ] * \tc \| \right)
\leq \frac{1}{2} b_0 \left( 2 \epsilon^2  + \| \tc \|^2 \right).
\end{equation}
The quadratic $x^2 - 2 b_0^{-1} x +   2\epsilon^2 $
has two zeros $z^+$ and $ z^-$ given by
	\[
	z^{\pm} = b_0^{-1} \pm \sqrt{b_0^{-2}-  2\epsilon^2 } .
	\]
The inequality~\eqref{e:quadineq} thus implies that either $ \|\tc\| \leq z^-$ or $ \|\tc\| \geq z^+$.

Furthermore, it follows from~\eqref{e:eBc2} that $\| K^{-1} \tc \| \leq  \| \B^{-1} \| \, (2 \epsilon^2 + \|\tc \|^2) \leq b_0 (2 \epsilon^2+ \|\tc\|^2)$.
\end{proof}

In practice we use the bound $\| \B^{-1} \| \leq b_*^{-1}$,
where
\[
  b_*(\epsilon) := \frac{\omega}{\alpha} - \frac{1}{2} - \epsilon  \left(\frac{2}{3}+ \frac{1}{2}\sqrt{2 + 2 |\omega-\pp| } \right).
\]
When doing so, we will refer to $z^\pm$ as $ z^\pm_*$. 
Additionally, we will need the following monotonicity property.
\begin{lemma}
	\label{lem:ZminusBound}
	Fix $\alpha, \omega, \epsilon_0 >0$ and assume that $ \epsilon_0 \leq b_*(\epsilon_0) /\sqrt{2}$.
	Define 
\[
  z_*^- (\epsilon):= b_*(\epsilon)-\sqrt{(b_*(\epsilon))^2 -2 \epsilon^2}.
 \]
Let $C_0 := \frac{z_*^-(\epsilon_0)}{\epsilon_0}$.
Then
	\begin{equation}
	 z_*^-(\epsilon) \leq C_0 \epsilon
	 \qquad \text{for all } 0 \leq \epsilon \leq \epsilon_0.
	 \label{eq:ConeLemma}
	\end{equation}
\end{lemma}


\begin{proof}
	Let $x:=\sqrt{2} \epsilon/b_*(\epsilon) \geq 0$. Clearly $\frac{d}{d\epsilon} x >0$.
It thus suffices to observe that
\[
  \frac{z_*^- (\epsilon)}{\epsilon} = \sqrt{2}\,\frac{1 - \sqrt{1-x^2}}{x}
\]	
is increasing for $x \in [0,1]$. 
%\marginpar{or $y-\sqrt{y^2-1}$ is decreasing}
%
% Throughout, we write $ b_* := b_*(\epsilon)$ and
% 	$ z_*^- =z_*^-(\epsilon)$.
% 	Writing $x=\epsilon/b_*(\epsilon)$
% 	First we may rewrite $ z_*^-$ as follows:
% 	\begin{eqnarray}
% 	 z_*^- &=& b_*-\sqrt{(b_*)^2 -\epsilon^2} \\
% 	 &=& b_*\left(1 - \sqrt{1-(\epsilon/b_*)^2} \right)
% 	\end{eqnarray}
% 	By assumption $|\epsilon_0/ b_*| <1 $ so for all $0 \leq  \epsilon \leq \epsilon_0$ the following Taylor expansion is valid:
% 	\begin{equation}
% 		z_*^-(\epsilon) = \frac{\epsilon^2}{2 b_*} + \frac{\epsilon^4}{8 (b_*)^3} + \frac{\epsilon^6}{16 (b_*)^5} + \dots
% 	\end{equation}
% In particular, we note that $z_*^-(\epsilon)$ can be expressed as $b_*$ times a power series in $(\epsilon/b_*)^2$ with strictly positive coefficients.
% 	Since $ \tfrac{d}{d \epsilon} b_*(\epsilon) = - \left( \tfrac{2}{3} + \tfrac{1}{2} \sqrt{2+2|\omega - \pp|}\right) < 0  $, then it follows that all of the functions:
% 	\begin{align}
% 		z_*^-(\epsilon), && \frac{z_*^-(\epsilon)}{\epsilon}, &&\frac{z_*^-(\epsilon)}{\epsilon^2}
% 	\end{align}
% 	are well defined at $ \epsilon=0$ and monotonically increasing in $\epsilon$.
% 	Hence, for all $ 0 \leq \epsilon < \epsilon_0$,  we have the inequality
% 	\[
% 	\frac{z_*^-(\epsilon)}{\epsilon} < 	\frac{z_*^-(\epsilon_0)}{\epsilon_0}
% 	\]
% 	whereby Equation \ref{eq:ConeLemma} follows.
%
\end{proof}

%%%%%%%%%%%%%%%%%%%%%%%%%%%%%%%%%%%%%%%%%%%%%%%%%%%%%%%%%%%%%%%%%%%%%


%!TEX root = hopfwright.tex

%%%%%%%%%%%%%%%%%%
%%% Appendix F %%%
%%%%%%%%%%%%%%%%%%

\section{Appendix: Implicit Differentiation}
\label{sec:Appendix_Implicit_Diff}

%\note[JB]{I have shortened this initial paragraph considerably}.
% We calculate  $\frac{\partial F}{\partial  \epsilon} (x) $ to be:
% \[
% \frac{\partial F}{\partial  \epsilon}( x ) = \alpha e^{-i \omega} \e_2 + \alpha L_\omega c + \alpha [ U_\omega c] * c
% \]
% In keeping with our $ \cO(\epsilon^2)$ approximations, we will use
% \[
% \tilde{x}_\epsilon = ( \pp, \pp, c_2(\epsilon), 0, 0, \dots)
% \]
% as the center of our approximation.
% Furthermore, we can drop the $ \alpha [U_\omega c] * c$ term in our initial expansion.
% We thereby obtain an approximation which we will call $\Gamma$:
%
% \begin{eqnarray}
% \frac{\partial F}{\partial  \epsilon} (\bar{x}_\epsilon) + \cO(\epsilon^2) &=& [-\pp i]_2 + \pp L_{\omega_0} [ \tfrac{2-i}{5} \epsilon]_2 \\
% &=& [-\pp i]_2 + \pp \left[ \sigma^+ ( -i-1) + \sigma^-(i-1) \right] [\tfrac{2-i}{5} \epsilon]_2 \\
% \Gamma &:=& \pp[\tfrac{3i -1}{5} \epsilon]_1 - \pp [i]_2 - \pp[\tfrac{3+i}{5} \epsilon]_3
% \end{eqnarray}
%

We will approximate
\[
\frac{\partial F}{\partial  \epsilon}( x ) = \alpha e^{-i \omega} \e_2 + \alpha L_\omega c + \alpha [ U_\omega c] * c 
\]
by
\begin{alignat}{1}
\Gamma & := \pp \tfrac{3i -1}{5} \epsilon \e_1 - \pp i \e_2 - \pp \tfrac{3+i}{5} \epsilon \e_3 \label{e:defGamma}  \\
&=   -\pp i \e_2 + \pp L_{\omega_0} \bc_\epsilon , \label{e:defGamma2}
\end{alignat}
which has been chosen so that $\frac{\partial F}{\partial  \epsilon} (\pp,\pp,\bc_\epsilon) - \Gamma = \cO(\epsilon^2)$.
%\remove[JB]{Given our approximation $\Gamma$, we can then calculate $ A^\dagger \Gamma$, as we do in the following lemma:}
\begin{lemma}
	\label{lem:ImplicitApprox}
When we write 
$A^\dagger \Gamma = (\alpha', \omega', c') \in \R^2 \times \ell^K_0$, then 
% \note[JB]{shouldn't $\tfrac{1-2i}{5}$ be $\tfrac{1+2i}{5}$? I think the second expression for $c'$ is better.} \note[J]{Agreed}
\begin{alignat*}{1}
		\alpha ' &= - \tfrac{2}{5} ( \tfrac{3 \pi}{2}-1) \epsilon ,\\
		\omega ' &= \tfrac{2}{5} \epsilon , \\
%		c '	 &= \left[ (\tfrac{1-2i}{5}) - 
%		\tfrac{\epsilon^2}{25} ( \tfrac{29 - 22i}{5} + \tfrac{1 + 7 i }{2})
%		 \right] \e_2 + ( \tfrac{3i-1}{10} )\epsilon \, \e_3 . \\
		c '	 &= \left[ (\tfrac{1+2i}{5}) - 
		\epsilon^2 \tfrac{9 }{250} (7-i)
		 \right] \e_2 + \epsilon \tfrac{3i-1}{10} \, \e_3 .
	\end{alignat*}
\end{lemma}

\begin{proof}
	First we calculate the $ \alpha$ and $ \omega$ components of the image of $ A^\dagger $:	
%	\begin{eqnarray}
%	[ A^\dagger]_{\alpha,\omega}  &=& \left[ A_0^{-1} [ I - \epsilon A_1 A_0^{-1}] \right]_{\alpha,\omega} \\
%	&=&   A_{0,1}^{-1} [ I - \epsilon \pp L_{\omega_0}  A_{0,*}^{-1}] \\
%	&=&   A_{0,1}^{-1} [ I - \epsilon \sigma^{-} ( iI + U_{\omega_0} ) ( i K^{-1} + U_{\omega_0})^{-1}]\\ 
%	&=&  A_{0,1}^{-1} [ e_1^* - \epsilon ( \tfrac{3 +i}{5}) e_2^* ]
%	\end{eqnarray}
%\note[JB]{Notational cleanup:}
	\begin{alignat}{1}
	\pi_{\alpha,\omega} A^\dagger  &= A_{0,1}^{-1} i_\C^{-1} \pi_1 [ I - \epsilon A_1 A_0^{-1}] ] \nonumber \\
	&=   A_{0,1}^{-1} i_\C^{-1} \pi_1 [ I - \epsilon \pp L_{\omega_0}  A_{0,*}^{-1}] \nonumber \\
	&=   A_{0,1}^{-1} i_\C^{-1} \pi_1 [ I - \epsilon \sigma^{-} ( iI + U_{\omega_0} ) ( i K^{-1} + U_{\omega_0})^{-1}] \nonumber \\ 
	&=  A_{0,1}^{-1} i_\C^{-1} [ \pi_1- \epsilon ( \tfrac{3 +i}{5}) \pi_2 ].
	\label{e:piaoAdag}
	\end{alignat}
Here we have used projections $\pi_k a = a_k$ for $a=\{a_k\}_{k \geq 1} \in \ell^1$.
	We now calculate the $\alpha $ and $\omega$ components of $ A^\dagger \Gamma$. 
	It follows from~\eqref{e:defGamma} and~\eqref{e:piaoAdag} that
	\begin{alignat*}{1}
	\pi_{\alpha,\omega} A^\dagger \Gamma  
%	&= A_{0,1}^{-1} [ e_1^* - \epsilon ( \tfrac{3 +i}{5}) e_2^* ] \left(\pp[\tfrac{3i -1}{5} \epsilon]_1 - \pp [i]_2 - \pp[\tfrac{3+i}{5} \epsilon]_3 \right) \\
	&= A_{0,1}^{-1} i_\C^{-1} \left[ \pp \tfrac{3i -1}{5} \epsilon +  \pp  \tfrac{3 +i}{5}  i \epsilon \right]
	\\
	&= \tfrac{\pi \epsilon}{5} A_{0,1}^{-1} i_\C^{-1} ( 3 i -1) 
	 \\
	&= - \frac{2 \epsilon}{5}
	\left(
	\begin{array}{c}
	\tfrac{3\pi}{2}-1 \\
	-1
	\end{array}
	\right) .
	\end{alignat*}
	
	We now calculate 
%\note[JB]{Notational cleanup, but does not look pretty.} 
%	\[
%	\pi_c A^{\dagger} \Gamma  = A_{0,*}^{-1}  \pi_{\geq 2} [ I - \epsilon ( \e_2 [i_\C A_{1,2}A_{0,1}^{-1} i_\C^{-1} \pi_1] +A_{1,*} A_{0,*}^{-1} \pi_{\geq 2} ) ]\Gamma .
%	\]
%\note[JB]{It seems to me it is better to write this as}
\begin{equation}\label{e:picAdagGamma}
	\pi_c A^{\dagger} \Gamma  = A_{0,*}^{-1}  \pi_{\geq 2} [ I - \epsilon A_1 A_0^{-1} ]\Gamma ,
\end{equation}
where
$A_1 A_0^{-1}$ decomposes as
\begin{equation}\label{e:A1A0decomposition}
  A_1 A_0^{-1} = 
  \e_2 [i_\C A_{1,2}A_{0,1}^{-1} i_\C^{-1} \pi_1] 
  +A_{1,*} A_{0,*}^{-1}  \pi_{\geq 2} . 
\end{equation}
We first calculate  
% \note[JB]{Here I think the $\tfrac{1-2i}{5}$ should be $\tfrac{1+2i}{5}$, but I did not implement it yet.} \note[J]{I agree. I've changed it now.}
\begin{alignat}{1}
	A_{0,*}^{-1} \pi_{\geq 2} \Gamma &= \tfrac{2}{ \pi } ( i K^{-1} + U_{\omega_0} )^{-1}  [ - \pp i \e_2 - \pp \tfrac{3+i}{5}\epsilon \e_3 ] \nonumber\\
	&= -(2i-1)^{-1} \e_2 - (3i+i)^{-1} \tfrac{3+i}{5} \epsilon \e_3 \nonumber\\
	&= \tfrac{1+2i}{5} \e_2 + \epsilon \tfrac{ 3 i-1}{20} \e_3  \label{e:A0piGamma}. 
\end{alignat}

%\note[JB]{Rearranged the rest of the proof; I found it hard to follow the flow of the argument.}
Since $\Gamma$ has three nonzero components only, we next compute the action of $A_{0,*}^{-1}  \pi_{\geq 2} A_1 A_0^{-1} $ on each of these.
Taking into account the decomposition~\eqref{e:A1A0decomposition},
we first compute its action on $\lambda \e_1$ for $\lambda \in \C$.
After a straightforward but tedious calculation we obtain
%\note[JB]{Hope this is correct.} \note[J]{This checks out.}
\begin{alignat*}{1}
A_{0,*}^{-1}  \pi_{\geq 2} A_1 A_0^{-1}  \lambda \e_1
&= 
 [i_\C A_{1,2}A_{0,1}^{-1} i_\C^{-1} \lambda ] A_{0,*}^{-1} \e_2
\\
&= -\tfrac{2}{25\pi} \bigl[ (11+2i) \text{Re} \lambda  + (-6+8i) \text{Im} \lambda \bigr]  \e_2.
\end{alignat*}
Next, we compute the action of $A_{0,*}^{-1}  \pi_{\geq 2} A_1 A_0^{-1} $
on  $\e_k$ for $k=2,3$:
%\note[JB]{Hope these are correct} \note[J]{This also checks out. }
\begin{alignat*}{1}
A_{0,*}^{-1}  \pi_{\geq 2} A_1 A_0^{-1} \e_2
&=
 A_{0,*}^{-1} [ \pp \sigma^+ (  e^{-i\omega_0} I  + U_{\omega_0}) ]A_{0,*}^{-1} \e_2 
\\& 
%= - \tfrac{2}{\pi} ( 3i +i)^{-1} (-i-1)(2i-1)^{-1} \e_3 
=  \tfrac{2}{\pi} \tfrac{3+i}{20} \e_3  ,
  \\
A_{0,*}^{-1}  \pi_{\geq 2} A_1 A_0^{-1} \e_3
&=
 A_{0,*}^{-1} [ \pp \sigma^- ( e^{i\omega_0} I + U_{\omega_0}) ]A_{0,*}^{-1} \e_3  \\
& = 
-\tfrac{2}{\pi} \tfrac{1+2i}{10}  \e_2 ,
\end{alignat*}
where we have used that $(e^{-i\omega_0}I + U_{\omega_0}) \e_3$ vanishes.
Hence, by using the explicit expression~\eqref{e:defGamma} for~$\Gamma$ we obtain
\begin{equation}\label{e:actionek}
- \epsilon A_{0,*}^{-1}  \pi_{\geq 2}  A_1 A_0^{-1} \Gamma 
= 
 -\epsilon^2 \frac{29-22i}{125} \e_2 + \epsilon \frac{3i-1}{20} \e_3 -\epsilon^2 \frac{1+7i}{50} \e_2.
\end{equation}
Finally, combining~\eqref{e:picAdagGamma}, \eqref{e:A0piGamma} and~\eqref{e:actionek} completes the proof.
%
% 	\begin{alignat*}{1}
% 	\left[ - \epsilon A_0^{-1} A_1 A_{0}^{-1} [ - \pp i ]_2 \right]_c &=  - \epsilon A_{0,*}^{-1} [ \pp \sigma^+ ( -iI + U_{\omega_0}) ]A_{0,*}^{-1} [ - \pp i ]_2 \\
% 	&= \epsilon ( 3i +i)^{-1} (-i-1)(2i-1)^{-1} [i]_3 \\
% 	&= \left[ \frac{3 i-1}{20} \epsilon \right]_3 .
% 	\end{alignat*}
% \change[JB]{We now calculate the last part of  $A^\dagger \Gamma$. }{Third,}
% 	\begin{eqnarray}
% 	- \epsilon A_0^{-1} A_1 A_0^{-1} \left( \pp [ \tfrac{3i-1}{5} \epsilon]_1 - \pp \left[ \tfrac{3+i}{5} \epsilon \right]_3 \right) &=&   - \pp \epsilon^2  A_{0,*}^{-1} A_{1,2} A_{0,1}^{-1} [ \tfrac{3 i-1}{5} ]_1
% 	+\pp \epsilon^2  A_{0,*}^{-1} A_{1,*} A_{0,*}^{-1} [ \tfrac{3 +i}{5} ]_3  \nonumber
% 	\end{eqnarray}
% 	Then we compute the two summands on the RHS.
% 	\begin{eqnarray}
% 	- \pp \epsilon^2  A_{0,*}^{-1} A_{1,2} A_{0,1}^{-1} [ \tfrac{3 i-1}{5} ]_1  &=& - \frac{\epsilon^2}{5} A_{0,*}^{-1}  [ \pi \frac{3+16i}{10}]_2 \\
% 	&=& \left[ \frac{- \epsilon^2}{125} (29-22i)\right]_2
% 	\end{eqnarray}
%
%
%
% 	\begin{eqnarray}
% 	\pp \epsilon^2  A_{0,*}^{-1} A_{1,*} A_{0,*}^{-1} [ \tfrac{3 +i}{5} ]_3 &=&
% 	\frac{\epsilon^2}{5} (i K^{-1} + U_{\omega})^{-1} L_{\omega_0}  ( 3 i + i)^{-1} [ 3 + i ]_{3} \\
% 	&=& \frac{\epsilon^2}{5} (iK^{-1} + U_{\omega_0})^{-1} [\tfrac{3+i}{2}  ]_2 \\
% 	&=&\frac{-\epsilon^2}{50} [1+7i ]_2
% 	\end{eqnarray}
% 	Combining all of these results together, we obtain our theorem.
\end{proof}
 
\begin{lemma}
	\label{lem:ImplicitLast}
Let 
%	\begin{alignat}{1}
%	\hat{f}_{\epsilon,1} &:= \frac{\epsilon}{\sqrt{5}} \left[  \da \sqrt{2} + \alpha ( \dt + \dtt ) \right] +
%	\alpha r_c \left[ 2  +  \frac{2\epsilon }{\sqrt{5}} + r_c \right] , \label{e:feps1} \\ 
%	\hat{f}_{\epsilon,c} &:= \frac{2}{\pi \sqrt{5}} \left[ 
%	\da + \pp \dt + 
%	\frac{\epsilon}{\sqrt{5}} \left[ \sqrt{2} \da + \alpha ( \dt + \dtt) \right]
%	+\alpha ( 4 r_c + \dc^2)
%	\right] .\label{e:fepsc}
%	\end{alignat}
%	\note[J]{Proposed Changes}
%	\begin{alignat}{1}
%	\hat{f}_{\epsilon,1} &:= \tfrac{1}{2} \dc^0 \left(   \sqrt{2} \da  +3  \dw ( \pp + \da ) \right)  + \tfrac{1}{2} r_c  ( \pp + \da) 
%	\left(2 + 2\dc^0  + r_c \right)  , \label{e:feps1} \\
%	%
%	% 
%	\hat{f}_{\epsilon,c} &:= 
%	\frac{2}{\pi \sqrt{5}} \left[ 
%	 2 \left( \da + \pp \dw \right) + \dc^0  [ \sqrt{2} \da + 3 \dw (\pp+\da) ]
%	+(\pp+\da) ( 4 r_c + \dc^2)
%	\right] .\label{e:fepsc}
%	\end{alignat}
%	\note[JB]{Proposed Rearrangement}
	\begin{alignat}{1}
	\hat{f}_{\epsilon,1} &:= \tfrac{1}{2} \dc^0 \left(   \sqrt{2} \da  +3  \dw ( \pp + \da ) \right)  +  r_c  ( \pp + \da) 
	\left(1 + \dc^0  + \tfrac{1}{2} r_c \right)  , \label{e:feps1} \\
	%
	% 
	\hat{f}_{\epsilon,c} &:= 
	\tfrac{2}{\pi \sqrt{5}} \left[ 
	 2 \left( \da + \pp \dw \right) + \dc^0  [ \sqrt{2} \da + 3 \dw (\pp+\da) ]
	+(\pp+\da) ( 4 r_c + \dc^2)
	\right] .\label{e:fepsc}
	\end{alignat}
	Then  the vector 
% 	\change[J]{
% 	$
% 	[(1+\tfrac{2}{\pi})\hat{f}_{\epsilon,1},\tfrac{2}{\pi} \hat{f}_{\epsilon,1}, \hat{f}_{\epsilon,c}]^{T}
% 	$
	$
	[	(1+\tfrac{4}{\pi^2})^{1/2}\hat{f}_{\epsilon,1},\tfrac{2}{\pi} \hat{f}_{\epsilon,1}, \hat{f}_{\epsilon,c}]^{T}
	$
	is an upper bound on $A_0^{-1}  ( \tfrac{\partial F}{\partial  \epsilon} (x) -\Gamma )$ for any $ x \in B_\epsilon ( r,\rho)$.
\end{lemma}


\begin{proof}
	The $\alpha$- and $\omega$-component of  $A_0^{-1}  ( \tfrac{\partial F}{\partial  \epsilon} (x) -\Gamma )$ are given by $ A_{0,1}^{-1} i_\C^{-1} \pi_1 [  \tfrac{\partial F}{\partial  \epsilon} (x) -\Gamma ]$.
	If we can show that  $| \pi_1 [  \tfrac{\partial F}{\partial  \epsilon} (x) -\Gamma ]   |  \leq \hat{f}_{\epsilon,1}$, then it follows from the explicit expression for $A_{0,1}^{-1}$ that
	 $[ (1+\tfrac{4}{\pi^2})^{1/2}\hat{f}_{\epsilon,1} ,\tfrac{2}{\pi} \hat{f}_{\epsilon,1} ]^T$ 
	is an upper bound on 
	 $ \pi_{\alpha,\omega} A_0^{-1}  ( \tfrac{\partial F}{\partial  \epsilon} (x) -\Gamma )$.
	Let us write $ c = \bce +h_c$ for some $ h_c \in \ell^1_0$ with $ \|h_c\| \leq r_c$.  Recalling  \eqref{e:defGamma2}, we obtain
	\begin{alignat*}{1}
	\pi_1[  \tfrac{\partial F}{\partial  \epsilon} (x) -\Gamma ] &=   \pi_1 \bigl[ \alpha L_\omega c + \alpha [ U_\omega c ] * c - \pp L_{\omega_0} \bce  \bigr]  \\
	&= \pi_1 \bigl[ \alpha \sigma^{-}( e^{i \omega} + e^{-2 i \omega} ) \bce- \pp \sigma^{-}(i-1) \bce \bigr] 
%	\\ & \qquad 
	+ \pi_1  \bigl[ \alpha \sigma^{-}( e^{i \omega} + e^{-2 i \omega} )h_c + \alpha [ U_\omega c] * c  \bigr] \\
	& = \pi_1 \bigl[ (\alpha - \pp) (i-1) \bce  + \alpha ( e^{i \omega} -i + e^{-2 i \omega} +1)\bce \bigr] 
	\\ & \hspace*{6.55cm}
	+ \pi_1  \bigl[ \alpha \sigma^{-}( e^{i \omega} + e^{-2 i \omega} )h_c + \alpha [ U_\omega c] * c  \bigr] .
	\end{alignat*}
We note that
\[
  \pi_1 ( [ U_\omega c] * c ) =  \pi_1 ([ U_\omega (\bce+h_c)] * (\bce+h_c) )
  =  \pi_1 ( [ U_\omega \bce] * h_c + [ U_\omega h_c] * \bce +
   [ U_\omega h_c] * h_c ).
 \]
Hence, using Lemma~\ref{lem:deltatheta} we obtain the estimate 
% \note[JB]{Better to replace $\alpha$ by $\pp+\da$? Due to factor 2 in norm, term $2\alpha r_c$ should be $\alpha r_c$? And term $\alpha r_c ( \tfrac{2}{\sqrt{5}}  \epsilon + r_c)$ should be $\alpha r_c ( \tfrac{2}{\sqrt{5}}  \epsilon + \tfrac{1}{2} r_c)$?}
% 	\begin{equation*}
% 	\bigl| \pi_1 [  \tfrac{\partial F}{\partial  \epsilon} (x) -\Gamma ] \bigr|
% 	% &\leq&  | (\alpha - \pp) (i-1) \bce | + \alpha |( e^{i \omega} -i + e^{-2 i \omega} +1)\bce| +  2 \alpha r_c + \alpha r_c ( 2 |\bce| + r_c) \nonumber  \\
% 	 \leq \da \tfrac{\sqrt{2}}{\sqrt{5}} \epsilon + \tfrac{\alpha }{\sqrt{5}}\epsilon ( \dt + \dtt ) + 2 \alpha r_c + \alpha r_c ( \tfrac{2}{\sqrt{5}}  \epsilon + r_c)  .
% 	\end{equation*}
%
% 	\note[J]{Proposed Change}
		\begin{equation*}
		\bigl| \pi_1 [  \tfrac{\partial F}{\partial  \epsilon} (x) -\Gamma ] \bigr|
		 \leq \tfrac{1}{2} \dc^0 \left(   \sqrt{2} \da  +3  \dw ( \pp + \da ) \right)  +  r_c  ( \pp + \da) 
		 \left(1 + \dc^0  + \tfrac{1}{2} r_c \right) .
		\end{equation*}
We thus find that
	% Thereby by defining
	% \[
	% \hat{f}_{\epsilon,1} := \frac{\epsilon}{\sqrt{5}} [  \da \sqrt{2} + \alpha ( \dt + \dtt ) ] +
	% \alpha r_c [2  +  \tfrac{2}{\sqrt{5}} \epsilon + r_c] .
	% \]
%	it follows that 
	$ | \pi_1 [  \tfrac{\partial F}{\partial  \epsilon} (x) -\Gamma ]|   \leq \hat{f}_{\epsilon,1} $, with $\hat{f}_{\epsilon,1}$ defined in~\eqref{e:feps1}.
	
	The $c$-component of  $A_0^{-1}  ( \tfrac{\partial F}{\partial  \epsilon} (x) -\Gamma )$ is given by $ A_{0,*}^{-1}  \pi_{\geq 2} [  \tfrac{\partial F}{\partial  \epsilon} (x) -\Gamma ]$.
	We will use the estimate $ \| A_{0,*}^{-1}\| \leq \frac{2}{\pi \sqrt{5}}$, so that it remains to determine a bound on $ \| \pi_{\geq 2} [  \tfrac{\partial F}{\partial  \epsilon} (x) -\Gamma ]\|$.  
Using~\eqref{e:defGamma2} we compute
	\begin{equation*}
	\pi_{\geq 2} [  \tfrac{\partial F}{\partial  \epsilon} (x) -\Gamma ] =  \alpha e^{-i \omega} \e_2  +  \pp i \e_2 + 
	\pi_{\geq 2} \bigl( \alpha L_{\omega} \bce - \pp L_{\omega_0} \bce + \alpha L_{\omega} h_c + \alpha [U_\omega c] * c  \bigr).
	\end{equation*}
We split the right hand side into three parts, which we estimate separately. First
% 	\[
% \left\| \pi_2 \bigl[ \alpha L_{\omega} h_c + \alpha [U_\omega c] * c  \bigr] \right\| \leq  \alpha ( 4 r_c + \dc^2).
% 	\]
% 	\note[J]{Proposed Change}
		\[
		\left\| \pi_2 \bigl[ \alpha L_{\omega} h_c + \alpha [U_\omega c] * c  \bigr] \right\| \leq  (\pp+ \da)  ( 4 r_c + \dc^2).
		\]	
Next, we calculate  
	\begin{alignat*}{1}
\pi_{\geq 2}	\left[\alpha L_{\omega} \bce - \pp L_{\omega_0} \bce \right] &= \alpha  \sigma^+ (e^{-i \omega} +e^{-2 i \omega}) \bce - \pp \sigma^+ (-i-1) \bce \\
	&= \left[ (\alpha - \pp) ( -i-1)\tfrac{2-i}{5}\epsilon  + \alpha ( e^{-i\omega} +e^{-2 i \omega} -(i+1)) \tfrac{2-i}{5}\epsilon \right] \e_3 ,
\end{alignat*}
hence 
%\note[JB]{Probably factor 2 missing.}
% \begin{equation*}
% \left\|	\pi_{\geq 2}	\left[\alpha L_{\omega} \bce - \pp L_{\omega_0} \bce \right] \right\| \leq  \frac{\epsilon}{\sqrt{5}} [ \sqrt{2} \da + \alpha ( \dt + \dtt)] .
% 	\end{equation*}
% 	\note[J]{Proposed Change}
	\begin{equation*}
	\left\|	\pi_{\geq 2}	\left[\alpha L_{\omega} \bce - \pp L_{\omega_0} \bce \right] \right\| \leq 
	\dc^0  [ \sqrt{2} \da + 3 \dw (\pp+\da) ] .
	\end{equation*}
Finally, we estimate 
% \note[JB]{Since $\|\e_2\|=2$ I think there is a factor 2 missing.}
% 	\begin{equation*}
% 	\left\| ( \alpha e^{-i\omega}  + \pp i) \e_2 \right\| =
% 	 \left| ( \alpha - \pp) e^{-i \omega} + \pp ( e^{-i \omega} +i) \right|
% 	\leq \da + \pp \dt.
% 	\end{equation*}
% 	\note[J]{Proposed Change}
	\begin{equation*}
	\left\| ( \alpha e^{-i\omega}  + \pp i) \e_2 \right\| = 
	2 \left| ( \alpha - \pp) e^{-i \omega} + \pp ( e^{-i \omega} +i) \right|
	\leq 2 \left( \da + \pp \dw \right).
	\end{equation*}
Collecting all estimates, we thus find that
	$ \| \pi_c A_{0}^{-1}  [  \tfrac{\partial F}{\partial  \epsilon} (x) -\Gamma ] \|   \leq \hat{f}_{\epsilon,c} $, with $\hat{f}_{\epsilon,c}$ defined in~\eqref{e:fepsc}.
	%
	%
	%
	% Thereby, by defining
	% \[
	% \hat{f}_{\epsilon,*} = \frac{2}{\pi \sqrt{5}} \left[
	% \da + \pp \dt +
	% \frac{\epsilon}{\sqrt{5}} [ \sqrt{2} \da + \alpha ( \dt + \dtt)]
	% +\alpha ( 4 r_c + \dc^2)
	% \right]
	% \]
	% it follows that
	% $ \left| A_{0,*}^{-1} \left[  \tfrac{\partial F}{\partial  \epsilon} (x) -\Gamma \right]_{k \geq 2} \right| <  \hat{f}_{\epsilon,*}$.
	%
\end{proof}

Recall that $\II$ is used to denote the $ 3 \times 3 $ identity matrix.
% \note[JB]{We may want to introduce this notation in section 4.2 as well.} \note[J]{I've added this notation inside the proof of Theorem 4.7.}
\begin{corollary}
	\label{cor:QUpperBound}
Let $\overline{A_0^{-1} A_1} $ be defined in Proposition~\ref{prop:A0A1}.	
The vector 
	% \[
	% (\II+\epsilon \overline{A_0^{-1} A_1} ) \cdot  [(1+\tfrac{2}{\pi})\hat{f}_{\epsilon,1},\tfrac{2}{\pi} \hat{f}_{\epsilon,1}, \hat{f}_{\epsilon,c}]^{T}
	% \]
	% \note[J]{Proposed Change}
	\[
	(\II+\epsilon \overline{A_0^{-1} A_1} ) \cdot  [(1+\tfrac{4}{\pi^2})^{1/2}\hat{f}_{\epsilon,1},\tfrac{2}{\pi} \hat{f}_{\epsilon,1}, \hat{f}_{\epsilon,c}]^{T}
	\]
	is an upper bound on $ A^\dagger ( \tfrac{\partial F}{\partial  \epsilon} (x) -\Gamma ) $ for any $x \in B_\epsilon(r,\rho)$.
\end{corollary}
\begin{sloppypar}
\begin{proof}
	From Lemma \ref{lem:ImplicitLast} it follows that 
% \change[J]{	$
% 	[(1+\tfrac{2}{\pi})\hat{f}_{\epsilon,1},\tfrac{2}{\pi} \hat{f}_{\epsilon,1}, \hat{f}_{\epsilon,c}]^T
% $
% }{
$
[(1+\tfrac{4}{\pi^2})^{1/2}\hat{f}_{\epsilon,1},\tfrac{2}{\pi} \hat{f}_{\epsilon,1}, \hat{f}_{\epsilon,c}]^T
$
	is an upper bound on $A_0^{-1}  ( \tfrac{\partial F}{\partial  \epsilon} (x) -\Gamma )$. 
Since 
	$A^\dagger = (I-\epsilon A_0^{-1}A_1) A_0^{-1}$ and 
	$\II+\epsilon \overline{A_0^{-1} A_1}$ is an upper bound on $I-\epsilon A_0^{-1}A_1$, the result follows from  Lemma~\ref{lem:ImplicitLast}. 
\end{proof}
\end{sloppypar}


We combine Lemmas~\ref{lem:ImplicitApprox} and~\ref{lem:ImplicitLast}
into an upper bound on $A^{\dagger} \frac{\partial F}{\partial  \epsilon}(\hat{x}_\epsilon)$.

\begin{lemma}\label{lem:Qeps}
% \note[JB]{Introduced $\QQ_\epsilon^0$; need to check that third component is correct!}
% \note[J]{I think the $\frac{9}{5\sqrt{50}}  \epsilon^2 $ term should be $ \frac{18}{5\sqrt{50}}  \epsilon^2$ . Also, inserted the $(1+\tfrac{4}{\pi^2})^{1/2}$ term.}
%
	Define $\QQ_\epsilon^0 , \QQ_\epsilon \in \R_+^3$ as follows:
% 	\begin{alignat}{1}
% 		\QQ_\epsilon^0 &:=
% \left[ \frac{2}{5}\left(\frac{3\pi}{2}-1 \right)  \epsilon,
%  \frac{2}{5} \epsilon ,
%  \frac{2}{\sqrt{5}} + \frac{2}{\sqrt{10}}\epsilon  +
%  \frac{9}{5\sqrt{50}}  \epsilon^2
%   \right]^T , \nonumber \\
% 	\QQ_\epsilon &:= \QQ_\epsilon^0
% 	+
% 	(\II+\epsilon \overline{A_0^{-1} A_1} ) \cdot  \bigl[(1+\tfrac{2}{\pi})\hat{f}_{\epsilon,1},\tfrac{2}{\pi} \hat{f}_{\epsilon,1}, \hat{f}_{\epsilon,c}\bigr]^{T} . \label{e:defQeps}
% \end{alignat}
% \note[J]{Proposed Change}
\begin{alignat}{1}
\QQ_\epsilon^0 &:= 
\left[ \frac{2}{5}\left(\frac{3\pi}{2}-1 \right)  \epsilon,
\frac{2}{5} \epsilon , 
\frac{2}{\sqrt{5}} + \frac{2}{\sqrt{10}}\epsilon  +
\frac{18}{5\sqrt{50}}  \epsilon^2
\right]^T , \nonumber \\
\QQ_\epsilon &:= \QQ_\epsilon^0  + 
(\II+\epsilon \overline{A_0^{-1} A_1} ) \cdot  \bigl[(1+\tfrac{4}{\pi^2})^{1/2}\hat{f}_{\epsilon,1},\tfrac{2}{\pi} \hat{f}_{\epsilon,1}, \hat{f}_{\epsilon,c}\bigr]^{T} . \label{e:defQeps}
\end{alignat}
Then the vector $\QQ_\epsilon \in \R^3_+$ is an upper bound  on $A^{\dagger} \frac{\partial F}{\partial  \epsilon}(x)$ 
for any  $x \in B_\epsilon(r,\rho)$.
\end{lemma}
\begin{proof}
It follows from Lemma \ref{lem:ImplicitApprox} that the vector 
$\QQ_\epsilon^0$
is an upper bound on $A^{\dagger} \Gamma$
(for example, the third component of $\QQ_\epsilon^0$ is a bound on $\|c'\|$).
It follows from Corollary~\ref{cor:QUpperBound} that 
% \change[J]{$	(I+\epsilon \overline{A_0^{-1} A_1} ) \cdot  [(1+\tfrac{2}{\pi})\hat{f}_{\epsilon,1},\tfrac{2}{\pi} \hat{f}_{\epsilon,1}, \hat{f}_{\epsilon,c}]^{T}$
% 	}{
%\note[JB]{displayed; formula was just too big}
\[
	(\II+\epsilon \overline{A_0^{-1} A_1} ) \cdot  [(1+\tfrac{4}{\pi^2})^{1/2}\hat{f}_{\epsilon,1},\tfrac{2}{\pi} \hat{f}_{\epsilon,1}, \hat{f}_{\epsilon,c}]^{T}
\]
 is an upper bound on $ A^\dagger ( \tfrac{\partial F}{\partial  \epsilon} (x) -\Gamma ) $.
We conclude from the triangle inequality that $\QQ_\epsilon $ is an upper bound on $A^\dagger  \tfrac{\partial F}{\partial  \epsilon} (x)$. 
\end{proof}
 
Finally, we prove the bounds needed to control the derivative $\frac{d}{d\epsilon} \hat{\alpha}_\epsilon$ in Section~\ref{s:Jones}
(in particular the implicit differentiation argument in Theorem~\ref{thm:NoFold}).

% \marginpar{remove newpage when done}
% \newpage

%%%%%%%%%%%%%%% Final Lemma %%%%%%%%%%%%%%

\begin{lemma}
		\label{lem:Meps}
		% \remove[J]{		Fix $\epsilon>0$, $r \in \R^3_+$, $\rho >0$ and
		% 	assume $x \in B_\epsilon(r,\rho)$. }
		% \add[J]{
		Fix $ \epsilon_0 > 0 , \rr \in \R^3_+ $ and $\rho >0$ as in the hypothesis of Proposition~\ref{prop:TightEstimate}. 
Let $ 0 < \epsilon \leq \epsilon_0$ and let $ \hat{x}_\epsilon \in B_\epsilon(\epsilon^2  \rr,\rho)$ denote the unique solution to $F(x) = 0$. 
Recall the definitions of  
$\ZZ_\epsilon \in  \emph{Mat}(\R_+^3 , \R_+^3)$
and $\QQ_\epsilon \in  \R_+^3$ in Equations~\eqref{e:defZeps} and~\eqref{e:defQeps}.  
Define 
		\begin{alignat*}{1}
		M_\epsilon &:= \frac{1}{\epsilon^2} \left(	(\II+\epsilon \overline{A_0^{-1} A_1} ) \cdot  \bigl[(1+\tfrac{4}{\pi^2})^{1/2}\hat{f}_{\epsilon,1},\tfrac{2}{\pi} \hat{f}_{\epsilon,1}, \hat{f}_{\epsilon,c} \bigr]^{T} \right)_1  ,\\
		M'_\epsilon  &:=  \frac{1}{ \epsilon^2 } \bigl( \ZZ_\epsilon (\II-\ZZ_\epsilon)^{-1} \QQ_\epsilon \bigr)_1  ,
		\end{alignat*}
where the subscript denotes the first component of the vector.
Then $M_\epsilon$ and $M'_\epsilon$ are positive, increasing in $\epsilon$, and satisfy the inequalities
 \begin{alignat}{1}
 \left| \pi_\alpha A^{\dagger} \left( \tfrac{\partial F}{\partial  \epsilon}(\hat{x}_\epsilon) - \Gamma_\epsilon \right)  \right|  &\leq\epsilon^2 M_\epsilon , \label{eq:Mepsilon}\\
 %
 \left( \ZZ_\epsilon (\II-\ZZ_\epsilon)^{-1} \QQ_\epsilon \right)_1 &\leq \epsilon^2 M'_\epsilon . \label{eq:MMepsilon}
 \end{alignat}
\end{lemma}

\begin{proof}
To first show that $(\II - \ZZ_{\epsilon })^{-1}$ is well defined,  we note that by Proposition~\ref{prop:TightEstimate} 
the radii polynomials $ P(\epsilon,\epsilon^2 \rr,\rho)$ are all negative. As was shown in the proof of Theorem~\ref{thm:RadPoly},
the operator norm of $ \ZZ_\epsilon$ on $ \R^3$ equipped with the norm $ \| \cdot \|_{\epsilon^2 \rr}$ is given by some $\kappa <1$, whereby the Neumann series of $ (\II - \ZZ_\epsilon)^{-1}$ converges. 
	
From the definition of $M_\epsilon$ and Corollary~\ref{cor:QUpperBound}, inequality \eqref{eq:Mepsilon} follows. Inequality \eqref{eq:MMepsilon} is a direct consequence of the definition of $M'_\epsilon$.
	Since the functions $\hat{f}_{\epsilon,1}$ and $\hat{f}_{\epsilon,c}$ are positive, then $M_\epsilon$ and $\QQ_\epsilon$ are positive. 
	Since the matrix $ \ZZ_\epsilon$ has positive entries only,  the Neumann series for $(\II-\ZZ_\epsilon)^{-1}$  has summands with exclusively positive entries, whereby $M'_\epsilon$ is positive. 

	Next we show that the components of $\ZZ_\epsilon$ and $\QQ_\epsilon - \QQ_\epsilon^0$ are polynomials in $\epsilon$ with positive coefficients and their lowest degree terms are at least quadratic. 
	To do so, it suffices to prove as much for the functions $ \hat{f}_{\epsilon,1},\hat{f}_{\epsilon,c},f_{1,\alpha},f_{1,\omega}, f_{1,c} , f_{*,\alpha}, f_{*,\omega},f_{*,c}$. 
	We note that all of these functions are given as polynomials with positive coefficients in the variables $ \epsilon, \da,\dw,\dc,r_c,\dc^0$ 
	(recall that $\rho $ is fixed and does not vary with $\epsilon$).
	Since $(r_\alpha , r_\omega,r_c) = \epsilon^2 (\rr_\alpha , \rr_\omega,\rr_c)$, then by Definition~\ref{def:DeltaDef} the terms $\da,\dw,r_c$ are all $ \cO(\epsilon^2)$. 
	Furthermore, whenever any of the terms  $ \epsilon, \dc, \dc^0$ 
appears, it is multiplied by another term of order at least $\cO(\epsilon)$.
	It  follows that every component of 
	 $ \ZZ_\epsilon$ and $ \QQ_{\epsilon} - \QQ_{\epsilon}^0$ is a polynomial in $ \epsilon$ with positive coefficients for which the lowest degree term is at least quadratic. 

% 	\remove[J]{
% 	We now show that $M_\epsilon$ and $ M'_\epsilon$ are increasing in $\epsilon$.
% First note that Proposition~3.16 shows that the solution  satisfies $\hat{x}_\epsilon \in B_\epsilon(\epsilon^2 \rr, \rho)$.
% Thereby all of the variables introduced in Definition B.1 are polynomials with nonnegative coefficients in $\epsilon$.
% Moreover, all of the variables which begin with  ``$\Delta$'' have a lowest order term of $\epsilon^2$, and only $\dc$ has its lowest order term being $\epsilon^1$.
% It is thus straightforward to see that the functions $\hat{f}_{\epsilon,1}$ and $\hat{f}_{\epsilon,c}$, as well as all of the component functions of $\ZZ_{\epsilon}$, are all polynomials in $\epsilon$ with positive coefficients and their lowest order terms are at least quadratic. }

From these considerations it follows that the components of both 
$M_\epsilon = \epsilon^{-2}(\QQ_{\epsilon} - \QQ_{\epsilon}^0)_1$ and $ \epsilon^{-2}\ZZ_\epsilon$ are polynomials in $\epsilon$ with positive coefficients. 
It also follows that both $\QQ_\epsilon$ and $(\II-\ZZ_\epsilon)^{-1}$ are increasing in $\epsilon$, whereby $M'_\epsilon$ is increasing in $\epsilon$.
\end{proof}








\end{document}

