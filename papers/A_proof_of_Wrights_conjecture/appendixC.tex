%!TEX root = hopfwright.tex

%%%%%%%%%%%%%%%%%%
%%% Appendix C %%%
%%%%%%%%%%%%%%%%%%

\section{Appendix: The bounding functions for $Y(\epsilon)$}
\label{sec:YBoundingFunctions}

We need to define $Y(\epsilon)$ so that it bounds $ T(\bar{x}_\epsilon) -\bar{x}_\epsilon = A^{\dagger} F(\bar{x}_\epsilon)$. 
We introduce $c_2(\epsilon) := \frac{2-i}{5} \epsilon$. 
We can explicitly calculate $F(\bar{x}_\epsilon)$ as follows:
\begin{alignat*}{1}
	F_1( \bar{x}_\epsilon) &=
	( i \bar{\omega}_\epsilon + \bar{\alpha}_\epsilon e^{-i \bar{\omega}_\epsilon}) + 
	\bar{\alpha}_\epsilon \epsilon ( e^{i \bar{\omega}_\epsilon} + e^{-2 i \bar{\omega}_\epsilon}) c_2( \epsilon) \\
	F_2( \bar{x}_\epsilon) &= 
	( 2 i \bar{\omega}_\epsilon + \bar{\alpha}_\epsilon e^{- 2 i \bar{\omega}_\epsilon}) c_2( \epsilon)+ 
	\bar{\alpha}_\epsilon \epsilon  e^{ -i \bar{\omega}_\epsilon} \\
	F_3( \bar{x}_\epsilon) &= \bar{\alpha}_\epsilon \epsilon ( e^{- i \bar{\omega}_\epsilon} + e^{-2 i \bar{\omega}_\epsilon}) c_2( \epsilon) \\
	F_4(\bar{x}_\epsilon) &=  \bar{\alpha}_\epsilon \epsilon  e^{-2 i \bar{\omega}_\epsilon} c_2( \epsilon)^2 \\
	F_{k}(\bar{x}_\epsilon) &=  0 \qquad\text{for all } k\geq 5.
\end{alignat*}
By using the definition of $ A^{\dagger} = A_{0}^{-1} - \epsilon A_{0}^{-1} A_1 A_{0}^{-1} $ we can calculate $A^{\dagger} F(\bar{x}_\epsilon)$ explicitly using a finite number of operations.  
However, proving $ \epsilon^{-2} Y (\epsilon) $ is well defined and increasing requires more work.  
To estimate $A^{\dagger} F(\bar{x}_\epsilon)$ in Theorem~\ref{prop:Ydef} below, we will take entry-wise absolute values in the constituents of $A^{\dagger}$, as clarified in the next remark.
\begin{remark}
 Since $F(\bar{x}_\epsilon)$ is a finite linear combination of the basis elements $\e_k$, and the operators $A_0$ and $A_1$ are diagonal and tridiagonal, respectively,  we can represent $A_0^{-1} \cdot F(\bar{x}_\epsilon)$ and $A_{0}^{-1} A_1 A_0^{-1} \cdot F(\bar{x}_\epsilon)$ by finite dimensional matrix-vector products. 
 By $|A_0^{-1}|$ and $|A_{0}^{-1} A_1 A_0^{-1} |$ we denote the entry-wise absolute values of these matrices. 
\end{remark}

\begin{theorem} \label{prop:Ydef}
	Let  $ f_i: \R \to \R$ for $i=1,2,3,4$ be defined as in Propositions \ref{prop:YfBound1}, \ref{prop:YfBound2},  \ref{prop:YfBound3}, and  \ref{prop:YfBound4} below. 
	%\note[J]{Note: The function $f$ bounds the coefficients of $ F(\bar{x}_\epsilon) \in \ell^1$. } 
Define $f(\epsilon)= \sum_{i=1}^4 f_i \e_i \in \ell^1$ and 
define the function $\hat{Y}: \R \to \R^2 \times \ell^1_0$ 
to be 
\begin{equation}
		\hat{Y}(\epsilon) : =  \left| A_0^{-1}  \right| \cdot  f(\epsilon)    + \epsilon \left| A_0^{-1} A_1  A_0^{-1}   \right|    \cdot  f(\epsilon)   .
\end{equation}	
Then the only nonzero components of   $\hat{Y}=(\hat{Y}_\alpha,\hat{Y}_\omega,\hat{Y}_c)$ are $\hat{Y}_\alpha$, $\hat{Y}_\omega$ and $(\hat{Y}_c)_k$ for $k=2,3,4,5$.
Furthermore, define 
\begin{align}
	Y_\alpha(\epsilon) :=& \hat{Y}_\alpha(\epsilon) &
	Y_\omega(\epsilon) :=& \hat{Y}_\omega(\epsilon) &	
	Y_c(\epsilon) :=& 2 \sum_{k=2}^5 (\hat{Y}_c)_k(\epsilon) 
\end{align}
	Then $[Y_\alpha(\epsilon),Y_\omega(\epsilon),Y_c(\epsilon) ]^T$ is an upper bound on $ T(\bar{x}_\epsilon ) - \bar{x}_\epsilon$, and $ \epsilon^{-2} [Y_\alpha(\epsilon),Y_\omega(\epsilon),Y_c(\epsilon) ]$ is non-decreasing in $\epsilon$.  
\end{theorem}
\begin{proof}
	By Propositions \ref{prop:YfBound1}, \ref{prop:YfBound2},  \ref{prop:YfBound3} and  \ref{prop:YfBound4} it follows that $|F_i(\bar{x}_\epsilon)| \leq f_i(\epsilon)$ for $i=1,2,3,4$. 
	By taking the entry-wise absolute values $ \left| A_0^{-1}  \right|   $  and $\left| A_0^{-1} A_1  A_0^{-1}   \right| $, it follows that $ |T(\bar{x}_\epsilon ) - \bar{x}_\epsilon| \leq \hat{Y}$, where the absolute values and inequalities are taken element-wise. 
We note that  in defining $ Y_c$ the factor~$2$ arises from our choice of norm in \eqref{e:lnorm}. 
	To see that  $(\hat{Y}_c)_k$ is non-zero for $k = 2,3,4,5$ only, we note that while $A_0^{-1}$ is a block diagonal operator, $A_1$  has off-diagonal terms. In particular, $ A_{1,*} \e_k = \pp (-i+(-i)^k) \e_{k+1} + \pp (i+(-i)^k) \e_{k-1}$ for $k \geq 2$, whereby  $( \hat{Y})_k =0$ for $ k \geq 6$. 


Next we show that  $\epsilon^{-2} [ Y_\alpha(\epsilon), Y_\omega(\epsilon), Y_c(\epsilon)]^T$ is nondecreasing in $\epsilon$. We note that it follows from Definition \ref{def:DeltaDef} that each function $f_i(\epsilon)$ is a polynomial in $\epsilon$ with nonnegative coefficients, and the lowest degree term is at least $ \epsilon^2$. 
Additionally, $\left| A_0^{-1}  \right| \cdot  f(\epsilon)   $  is a positive linear combination of the functions $\{f_i(\epsilon)\}_{i=1}^4$, 
whereas $\left| A_0^{-1} A_1  A_0^{-1}   \right|    \cdot  f(\epsilon)  $ is $ \epsilon$ times a positive linear combination of  $\{f_i(\epsilon)\}_{i=1}^4$. 
It follows that each component of  $\hat{Y}$  is a polynomial in $\epsilon$ with nonnegative coefficients, and the lowest degree term is at least $ \epsilon^2$. 
	Thereby $\epsilon^{-2} [ Y_\alpha(\epsilon),Y_\omega(\epsilon),Y_c(\epsilon)]^T$ is nondecreasing in $\epsilon$. 
	

	

\end{proof}
Before presenting Propositions \ref{prop:YfBound1}, \ref{prop:YfBound2},  \ref{prop:YfBound3} and  \ref{prop:YfBound4}, 
we recall that the definitions of $\da^0$, $\dw^0$ and $\dc^0$ are given  in  Definition~\ref{def:DeltaDef}. 
\begin{proposition}
	\label{prop:YfBound1}
Define 
% \begin{equation}\label{e:f1}
% f_1 (\epsilon) := \pp ( \tfrac{1}{2} (\dw^0)^2 + \tfrac{1}{6} (\dw^0)^3) + \da^0 \dt^0 +  \da^0 \epsilon \dc^0 + \tfrac{\pi   }{4} \epsilon \dc^0 ( \dt^0 + \dtt^0) .
% \end{equation}
% \note[J]{Proposed Change}
% \begin{equation}\label{e:f1}
% f_1 (\epsilon) := \pp ( \tfrac{1}{2} (\dw^0)^2 + \tfrac{1}{6} (\dw^0)^3) + \da^0 \dw^0 +  \da^0 \epsilon \dc^0 + \tfrac{3\pi   }{4} \epsilon \dc^0 \dw^0.
% \end{equation}
% \note[JB]{Proposed Change}
\begin{equation}\label{e:f1}
f_1 (\epsilon) := \pp ( \tfrac{1}{2} (\dw^0)^2 + \tfrac{1}{6} (\dw^0)^3) + \da^0 \dw^0 +  \da^0 \epsilon \dc^0 + \tfrac{3\pi   }{4} \dw^0 \epsilon \dc^0 .
\end{equation}
	Then $| F_1(\bx_\epsilon) | \leq  f_1(\epsilon)$. 
\end{proposition}
%
\begin{proof}
Note that 
\begin{equation}\label{e:F1}
F_1(\bar{x}_\epsilon ) =
 i \bar{\omega}_\epsilon + \bar{\alpha}_\epsilon e^{-i \bar{\omega}_\epsilon} + 
\bar{\alpha}_\epsilon \epsilon  c_2( \epsilon)  ( e^{i \bar{\omega}_\epsilon} + e^{-2 i \bar{\omega}_\epsilon}) .
\end{equation}
We will show that all of the $ \cO(\epsilon^3)$ in $F_1(\bx_\epsilon)$ cancel. 
We first expand the first summand~\eqref{e:F1}:
\begin{equation*}
	i \bar{\omega}_\epsilon  = i \pp -  i \dw^0   \label{e:Y1a}.
\end{equation*}
Next, we expand the second summand in~\eqref{e:F1}:
\begin{alignat}{1}
\bar{\alpha}_\epsilon e^{- i \bar{\omega}_\epsilon }	&= - i \bar{\alpha}_\epsilon e^{i \dw^0  }	
=  -i \left( \pp  e^{i \dw^0 } +\da^0  e^{i \dw^0 } \right) \nonumber\\
&= -i \left(\pp  (1 + i \dw^0 )   +\da^0  \right)  
 -i \left( \pp  (e^{i \dw^0 }-1- i \dw^0) + \da^0 (e^{i \dw^0 }-1)\right) . \label{e:Y1b}
\end{alignat}
Finally, we expand the third summand~\eqref{e:F1} as
\begin{alignat}{1}
\bar{\alpha}_\epsilon \epsilon^2 \tfrac{2-i}{5} ( e^{i \bar{\omega}_\epsilon} + e^{-2 i \bar{\omega}_\epsilon})   &= 
\pp \epsilon^2 \tfrac{2-i}{5} (i  - 1) + \pp \epsilon^2 \tfrac{2-i}{5} \left(i (e^{- i \dw^0} -1)-(e^{2 i \dw^0}-1) \right) \nonumber \\
&\hspace*{3cm} +\da^0 \epsilon^2 \tfrac{2-i}{5} \left(i e^{- i \dw^0} - e^{2 i \dw^0} \right)  \label{e:Y1c}.
\end{alignat}
% \begin{alignat}{1}
% \bar{\alpha}_\epsilon \epsilon^2 \tfrac{2-i}{5} ( e^{i \bar{\omega}_\epsilon} + e^{-2 i \bar{\omega}_\epsilon})   &= \pp \epsilon^2 \tfrac{2-i}{5} (i e^{- i \epsilon^2/5} - e^{2 i \epsilon^2/5}) \nonumber \\
% &\qquad + \da \epsilon^2 \tfrac{2-i}{5} (i e^{- i \epsilon^2/5} - e^{2 i \epsilon^2/5}) \\
% %
% \pp \epsilon^2 \tfrac{2-i}{5} (i e^{- i \epsilon^2/5} - e^{2 i \epsilon^2/5})  \label{eq:Y1b} \\
% &=
% \pp \epsilon^2 \tfrac{2-i}{5} (i  - 1) \nonumber \\
% &\qquad + \pp \epsilon^2 \tfrac{2-i}{5} (i (e^{- i \epsilon^2/5} -1)-(e^{2 i \epsilon^2/5}-1)) . \label{eq:Y1c}
% \end{alignat}
If we now collect the final term from~\eqref{e:Y1b} and the final two terms from \eqref{e:Y1c}
in 
\begin{alignat*}{1}
g(\epsilon) &:=
-i \left( \pp  (e^{i \dw^0 }-1- i \dw^0) + \da^0  (e^{i \dw^0 }-1)\right) \\
& \quad\qquad +
\da^0 \epsilon^2 \tfrac{2-i}{5} \left(i e^{- i \dw^0} - e^{2 i \dw^0}\right) \\
& \quad\qquad\qquad+
\pp \epsilon^2 \tfrac{2-i}{5} \left(i (e^{- i \dw^0} -1)-(e^{2 i \dw^0}-1)\right) ,
\end{alignat*}
then we can write $F_1(\bar{x}_\epsilon)$ as 
\begin{alignat*}{1}
F_1(\bar{x}_\epsilon) &= g(\epsilon) 
+ i \pp -  i \dw^0  
-i \left(\pp  (1 + i \dw^0 )   + \da^0  \right)
+ \pp \epsilon^2 \tfrac{2-i}{5} (i  - 1) \\
&= g(\epsilon).
\end{alignat*}
Using Lemma~\ref{lem:deltatheta} it  is not difficult to see that $|g(\epsilon)|$ can be bounded by $f_1(\epsilon)$, as defined in~\eqref{e:f1}.
% This means that all of the $\cO(\epsilon^3)$ terms in $F_1(\bx_\epsilon)$ cancel, and so it follows that $F_1(\epsilon) = g(\epsilon)$.
% Then for our definition of $ f_1$, it follows that $ | F_1(\epsilon) | \leq f_1( \epsilon)$.
%
\end{proof}


%%%%%%
\begin{proposition}
		\label{prop:YfBound2}
	Define 
		% \begin{equation}\label{e:f2}
		% f_2(\epsilon) :=		(\pp + \da^0) ( \tfrac{1}{2} \dtt^0 \dc^0 + \epsilon \dt^0 ) +   \tfrac{1}{2}\dc^0 ( 2 \dw^0 + \da^0 ) + \epsilon \da^0.
		% \end{equation}
		% \note[J]{Proposed Change}
		\begin{equation}\label{e:f2}
		f_2(\epsilon) :=		(\pp + \da^0) \dw^0 ( \dc^0 + \epsilon  ) +   \tfrac{1}{2}\dc^0 ( 2 \dw^0 + \da^0 ) + \epsilon \da^0.
		\end{equation}
	Then $| F_2(\bx_\epsilon) | \leq f_2(\epsilon)$. 
\end{proposition}
%
\begin{proof}
	First note that 
\begin{alignat}{1}
	F_2(\bar{x}_\epsilon) &= 
	( 2 i \bar{\omega}_\epsilon + \bar{\alpha}_\epsilon e^{- 2 i \bar{\omega}_\epsilon}) c_2( \epsilon)+ 
	\bar{\alpha}_\epsilon \epsilon  e^{ -i \bar{\omega}_\epsilon} 
	%
= \left(  2 i \bar{\omega}_\epsilon - \bar{\alpha}_\epsilon e^{ 2 i \dw^0 } \right) \tfrac{2-i}{5} \epsilon - i	\bar{\alpha}_\epsilon \epsilon  e^{ i \dw^0 }   \nonumber \\
%
		&=  \left(  2 i \bar{\omega}_\epsilon - \bar{\alpha}_\epsilon  \right) \tfrac{2-i}{5} \epsilon - i	\bar{\alpha}_\epsilon \epsilon   
   - \bar{\alpha}_\epsilon (e^{ 2 i \dw^0 } -1)  \tfrac{2-i}{5} \epsilon -i	\bar{\alpha}_\epsilon \epsilon  (e^{ i \dw^0 }-1).
\label{e:F2part}
\end{alignat}
We expand the first part of the right hand side in~\eqref{e:F2part} as
	\begin{alignat*}{1}
 \left(  2 i \bar{\omega}_\epsilon - \bar{\alpha}_\epsilon  \right) \tfrac{2-i}{5} \epsilon - i	\bar{\alpha}_\epsilon \epsilon  &= \left( 2 i \pp - \pp \right) \tfrac{2-i}{5} \epsilon - i \pp \epsilon 
 +  \left(  - 2 i \dw^0 - \da^0 \right) \tfrac{2-i}{5} \epsilon -	i \da^0  \epsilon
 \\
& = - \left(   2 i \dw^0 + \da^0 \right) \tfrac{2-i}{5} \epsilon -	i \da^0  \epsilon .
\end{alignat*}
	% Since $0=  \left( 2 i \pp - \pp \right) \tfrac{2-i}{5} \epsilon - i \pp \epsilon $,
Hence, we can rewrite $F_2(\epsilon)$ as 
	\begin{equation*}
F_2(\bx_\epsilon) =   - \bar{\alpha}_\epsilon (e^{ 2 i \dw^0 } -1)  \tfrac{2-i}{5} \epsilon -i	\bar{\alpha}_\epsilon \epsilon  (e^{ i \dw^0 }-1)
 -\left(   2 i \dw^0 + \da^0 \right) \tfrac{2-i}{5} \epsilon -	i \da^0  \epsilon .
	\end{equation*}
Using Lemma~\ref{lem:deltatheta} it  is then not difficult to see that $|F_2(\bx_\epsilon)|$ can be bounded by $f_2(\epsilon)$, as defined in~\eqref{e:f2}.
\end{proof}


\begin{proposition}
		\label{prop:YfBound3}
	Define 
		\begin{equation}\label{e:f3} 
		f_3(\epsilon):= \tfrac{1}{2}(\pp + \da^0 )  ( \sqrt{2} + 3 \dw^0 )
			 \epsilon \dc^0   .
		\end{equation} 
	Then $| F_3(\bx_\epsilon) | \leq  f_3(\epsilon)$. 
\end{proposition}
%
\begin{proof}
Note that 
\[
F_3( \bar{x}_\epsilon) = \bar{\alpha}_\epsilon \epsilon ( e^{- i \bar{\omega}_\epsilon} + e^{-2 i \bar{\omega}_\epsilon}) c_2( \epsilon)  . \\
\]
We expand this as
\begin{alignat*}{1}
F_3( \bar{x}_\epsilon) &= -\bar{\alpha}_\epsilon \epsilon^2 \tfrac{2-i}{5} ( i e^{ i \dw^0 } + e^{2 i \dw^0 }) \\
&= -\bar{\alpha}_\epsilon \epsilon^2 \tfrac{2-i}{5} ( i+1) -\bar{\alpha}_\epsilon \epsilon^2 \tfrac{2-i}{5} \left( i ( e^{ i \dw^0 }-1) +( e^{2 i \dw^0 } -1)\right)  .
\end{alignat*}
Using Lemma~\ref{lem:deltatheta} it is then not difficult to see that $|F_3(\bx_\epsilon)|$ can be bounded by $f_3(\epsilon)$, as defined in~\eqref{e:f3}.
\end{proof}



\begin{proposition}
		\label{prop:YfBound4}
Define
\begin{equation}\label{e:f4}
	f_4(\epsilon) := \tfrac{1}{5} (\pp+\da^0) \epsilon^3   %	\dc^2
\end{equation}
	Then $| F_4(\bx_\epsilon) | \leq  f_4(\epsilon)$. 
\end{proposition}
%
\begin{proof}
Note that 
\[
F_4( \bar{x}_\epsilon ) =  \bar{\alpha}_\epsilon \epsilon  e^{-2 i \bar{\omega}_\epsilon} \left(\tfrac{2-i}{5}\epsilon \right)^2 ,
\]
from which it follows that $|F_4(\bx_\epsilon)|$ can be bounded by $f_4(\epsilon)$, as defined in~\eqref{e:f4}.
\end{proof}



%%
%%\begin{proposition}
%% The functions $\epsilon^{-4} Y_{\alpha}(\epsilon)$ and  $\epsilon^{-4} Y_{\omega }(\epsilon)$ and  $\epsilon^{-2} Y_{c}(\epsilon)$ are well defined and non-decreasing with respect to $\epsilon$.
%%\end{proposition}
%%
%%
%%\begin{proof}
%%	
%%Evaluating $A^{\dagger} f(\epsilon)$ produces the bounds 
%%\begin{eqnarray}
%%y_\alpha(\epsilon) &=& \frac{88-147 \pi  }{500} \epsilon ^4 \\
%%y_\omega(\epsilon) &=& \frac{7 \epsilon ^4}{125} \\
%%y_2(\epsilon) &=& \frac{\epsilon ^3}{\pi } \left(\frac{3}{1250}+\frac{21 i}{1250}\right)  \left( (1+i) (3 \pi -4) \epsilon ^2+15 i \pi \right)\\
%%y_3(\epsilon) &=& \left(-\frac{3}{8500}+\frac{29 i}{8500}\right) \epsilon ^2 \left((30+6 i)
%%\epsilon ^2+(45+10 i)\right)\\
%%y_4(\epsilon) &=& \left(\frac{13}{425}+\frac{16 i}{425}\right) \epsilon ^3 \\
%%y_5(\epsilon) &=& \left(-\frac{3}{1700}+\frac{29 i}{1700}\right) \epsilon ^4
%%\end{eqnarray}
%%The norm of each of these values are all increasing in $\epsilon$. We then define $y_c(\epsilon) := |y_2(\epsilon)| + | y_3(\epsilon)| + |y_4(\epsilon)| + |y_5(\epsilon)|$. Note that the quantities $\epsilon^{-4} y_\alpha(\epsilon)$,  $\epsilon^{-4} y_\omega(\epsilon)$ and  $\epsilon^{-2} y_c(\epsilon)$ are all well defined at $\epsilon=0$ and non-decreasing in $\epsilon$. This shows that $ \left| A^{\dagger} f(\epsilon)   \right|$ satisfies the proposition. 
%%Since all of the functions $h$ are $\cO(\epsilon^4)$, then the result holds for $ Y_\alpha$, $Y_\omega$ and $Y_c$. 
%%
%%
%%
%%\end{proof}


%%%%%%%%%%%%%%%%%%%%%%%%%%%%%%%%%%%%%%%%%%%%%%%%%%%%%%%%%%%%%%%%%%%%%%%%%%%%

