%!TEX root = hopfwright.tex

In many biological and physical systems the dependency of future states relies not only on the present situation, but on a broader history of the system. 
For simplicity, mathematical models often ignore the causal influence of all but the present state.  
However, in a wide variety of applications delayed feedback loops play an inextricable role in the qualitative dynamics of a system \cite{kolmanovskii2013introduction}. 
These phenomena can be modeled using delay and integro-differential equations, the theory of which has developed significantly over the past 60 years \cite{Hale2006}. 
A canonical and well-studied example of a nonlinear delay differential equation is Wright's equation:
\begin{equation}
y'(t) = - \alpha \,y(t-1) \left[  1+ y(t)  \right] .
\label{eq:Wright}
\end{equation}
Here $\alpha$ is considered to be both real and positive. 
This equation has been a central example considered in the development of much of the theory of functional differential equations. 
For a short overview of this equation, we refer the reader to \cite{hale1971functional}. We cite some basic properties of its global dynamics \cite{wright1955non}:  
\begin{itemize}
	\item Corresponding to every 
	 $y \in C^0([-1,0])$, there is a unique solution of 
 \eqref{eq:Wright}  for all $t>0$.
	\item Wright's equation has two equilibria $ y \equiv -1$ and $ y \equiv 0$. Moreover, 
	solutions cannot cross $-1$. Any solution with $y(t_0)=-1$ (for some $t_0 \in \mathbb{R}$) is identically equal to $-1$.
	\item When $y<-1$ then the solution decreases monotonically without bound.
	 \item When $ y > -1$ then $ y(t) $ is globally bounded as $ t \to + \infty$. 
\end{itemize}



Henceforth we restrict our attention to $y>-1$.
In Wright’s seminal 1955 paper \cite{wright1955non}, he showed that if $ \alpha \leq \tfrac{3}{2}$ then any solution having $ y > -1$ is attracted to $ 0$ as $ t \to + \infty$.  
At $ \alpha = \pp$, the equilibrium $ y \equiv 0$ changes from asymptotically stable to unstable, and Wright formulated the following  conjecture: 

\begin{conjecture}[Wright's Conjecture]
	For every $0< \alpha \leq \pp  $, the zero solution to~\eqref{eq:Wright} is globally attractive. 
	\label{conj:ConjWright}
\end{conjecture}
For $ \alpha > \pp$, Wright proved the existence of oscillatory solutions to \eqref{eq:Wright} which do not tend towards $0$, and whose zeros are spaced at distances greater than the delay. 
Such a periodic solution is said to be \emph{slowly oscillating}, and formally defined as follows: 
\begin{definition}
	A \emph{slowly oscillating periodic solution (SOPS)}  is a periodic solution $y(t)$ which up to a time translation satisfies the following property: there exists some $t_{-}, t_{+} >1$ and $ L = t_{-} +t_{+} $ such that 
	$ y( t ) >0$ for $ t\in (0,t_{+})$,
	$y(t) < 0$ for $ t \in (-t_{-},0)$, 
	and $y(t+L) = y(t)$ for all $ t$,
	so that $ L $ is the minimal period of $y(t)$.  
\end{definition}



In Jones' 1962 paper \cite{jones1962existence} he proved that for $ \alpha > \pp $  there exists a  slowly oscillating periodic solution to~\eqref{eq:Wright}.  
Based on numerical calculations \cite{jones1962nonlinear}  Jones made the following conjecture: 



\begin{conjecture}[Jones' Conjecture]
	For every $ \alpha > \pp  $ there exists a unique slowly oscillating periodic solution to~\eqref{eq:Wright}. 
	\label{conj:ConjJones}
\end{conjecture}

Slowly oscillating periodic solutions play a critical role in the global dynamics of~\eqref{eq:Wright}. 
The global attractor of~\eqref{eq:Wright} admits a Morse decomposition \cite{mccord1996global,mallet1988morse} and if there is a unique SOPS and $ \alpha > \pp$ then it must be asymptotically stable \cite{xie1991thesis,xie1993uniqueness}.  
In \cite{neumaier2014global} it is shown that there are no homoclinic solutions from $0$ to itself for $0 \leq   \alpha \leq \pp$. A corollary is the following theorem.
\begin{theorem}[Theorem 3.1 in \cite{neumaier2014global}]% This is Theorem 3.1 in their paper. 
	\label{thm:AttractiveNonexistenceEquivalence}
	The zero solution of \eqref{eq:Wright} is globally attracting if and only if \eqref{eq:Wright}
	has no slowly oscillating periodic solution. 
\end{theorem}



Despite a considerable amount of work studying Wright's equation, complete resolution of these conjectures has remained elusive (see the survey paper \cite{walther2014topics} and the references contained therein). 
To describe a few  results,
the global bifurcation analysis in \cite{nussbaum1975global} proved that for $ \alpha > \pp$ there is a continuum of pairs $(\phi,\alpha)$ where $ \phi$ is a periodic solution to~\eqref{eq:Wright}, and these pairs form a $2-$dimensional manifold~\cite{regala1989periodic}.
In \cite{chow1977integral} it was shown that Wright's equation has a supercritical Hopf bifurcation at $ \alpha = \pp$. 
By studying the Floquet multipliers of periodic solutions for large $ \alpha$, Xie showed in  \cite{xie1991thesis} that Conjecture 1.2 holds for $ \alpha \geq 5.67$.  




Recent results using computer assisted proofs have narrowed the gap to resolving both conjectures. 
In the work in preparation \cite{jlm2016Floquet} it is shown that Conjecture \ref{conj:ConjJones} holds for $ [1.94,6.00]$. 
In \cite{lessard2010recent}, it is shown that the branch of periodic orbits emanating from the Hopf bifurcation does not have any subsequent bifurcations in the interval $ \alpha \in  [ \pp + \delta_1 , 2.3]$ where $ \delta_1 = 7.3165 \times 10^{-4}$.  
In \cite{neumaier2014global} it was shown that Conjecture \ref{conj:ConjWright} holds for $ \alpha \in [1.5, \pp - \delta_2 ]$ where $ \delta_2 = 1.9633 \times 10^{-4}$, and the authors remark that ``\emph{substantial improvement of the theoretical
	part of the present proof is needed to prove Wright's conjecture fully.}''





Many normal form techniques for functional differential equations have been developed  to transform a given equation into a simpler expression having the same qualitative behavior as the original equation (see \cite{faria2006normal} and references contained therein). 
While this transformation is valid in some neighborhood about the bifurcation point, such results usually do not describe the size of this neighborhood explicitly. 
In this paper we develop an explicit description of a neighborhood wherein the only periodic solutions are those originating from the Hopf bifurcation. 
The main result of this analysis is the resolution of Wright's conjecture. 

\begin{theorem}
	For every $0 < \alpha \leq \pp  $, the zero solution to~\eqref{eq:Wright} is globally attractive.
	\label{thm:IntroWrightConjecture} 
\end{theorem}

This result follows from Theorem~\ref{thm:WrightConjecture} combined with Theorem~\ref{thm:AttractiveNonexistenceEquivalence}. 
Roughly, by the work in \cite{neumaier2014global}, 
to prove Wright's conjecture
it is sufficient  to show that there do not exist any slowly oscillating periodic solutions for $ \alpha \in [ \pp - \delta_2, \pp]$, where $\delta_2 = 1.9633 \times 10^{-4}$.
Indeed, we construct an explicit neighborhood about $ \alpha = \pp$ for which the bifurcation branch of periodic orbits are the only periodic orbits. 
Then we show that throughout this entire neighborhood the solution branch behaves as expected from a supercritical bifurcation branch, i.e., it does not bend back into the parameter region $\alpha \leq \pp$.


Rather than trying to resolve all small bounded solutions near the bifurcation point through a center manifold analysis, we focus on periodic orbits only. In particular, we ignore orbits that connect the trivial state to the periodic states, since those are not relevant for our analysis.
The advantage is that, by restricting our attention to periodic solution, we can perform our analysis in Fourier space. We first note that all periodic solutions are smooth, as was established in~\cite{wright1955non} and more generally in~\cite{nussbaum-analytic}.
\begin{lemma}[\cite{nussbaum-analytic}]\label{l:analytic}
	All periodic solutions of~\eqref{eq:Wright} are real analytic.
\end{lemma}

%We follow the approach in \cite{lessard2010recent} and use a Fourier analysis. 
For  a periodic function $y:\R \to \R$ with frequency $\omega >0$ we write 
\begin{equation}
y(t)  = \sum_{k \in \Z} \c_{k} e^{ i \omega k t } ,
\label{eq:FourierEquation}
\end{equation}
where $\c_k \in \C$. 
This transforms the delay equation~\eqref{eq:Wright} into 
\begin{equation}
( i \omega k + \alpha e^{ - i \omega k}) \c_k + \alpha \sum_{k_1 + k_2 = k} e^{- i \omega k_1} \c_{k_1} \c_{k_2} = 0 \qquad\text{for all } k \in \Z.
\label{eq:FourierSequenceEquation}
\end{equation}
In effect, the problem of finding periodic solutions to Wright's equation can be reformulated as finding a parameter $ \alpha$, a frequency $\omega$, and a sequence $ \{a_k\}$  for which~\eqref{eq:FourierSequenceEquation} is satisfied. 
In Section \ref{s:preliminaries} we define an appropriate sequence space to work in, and define a zero finding problem $ F_\epsilon( \alpha,\omega,c)=0$  equivalent to~\eqref{eq:FourierSequenceEquation}.  
The auxiliary variable $\epsilon$, which represents the dominant Fourier mode, corresponds to the rescaling $y \mapsto \epsilon y$ canonical to the study of Hopf bifurcations. 




In Section \ref{s:local} we construct a Newton-like operator $T_\epsilon$ whose fixed points correspond to the zeros of $F_\epsilon(\alpha, \omega, c)$.  
By applying a Newton-Kantorovich like theorem, we identify explicit neighborhoods $B_\epsilon$ wherein $T_\epsilon: B_\epsilon \to B_\epsilon$ is a uniform contraction mapping. 
By the nature of our argument, we have the freedom to construct both large and small balls $ B_\epsilon$ on which we may apply the Banach fixed point theorem. Using smaller balls will produce tighter  approximations of the periodic solutions, while using larger balls will produce a larger region within which the periodic solution is unique. 


These results are leveraged in  Section \ref{s:global} to derive global results such  as Theorem \ref{thm:IntroWrightConjecture}, as well as Theorem \ref{thm:IntroNoFold} which helps to resolve one part of the reformulated Jones conjecture presented in~\cite{lessard2010recent}. 
This result shows that the branch of solutions that bifurcates from the Hopf bifurcation at $\alpha = \pp$ provides a unique SOPS for every $\alpha > \pp$. 
\begin{theorem}
\label{thm:IntroNoFold}
There are no bifurcations in the branch of SOPS originating from the Hopf bifurcation for $\alpha > \pp  $. 
\end{theorem}

This result follows from Theorem~\ref{thm:NoFold} combined with the results in~\cite{lessard2010recent,jlm2016Floquet,xie1991thesis}, see Corollary~\ref{cor:collectreformulatedJones}. 
Roughly, by the work in~\cite{lessard2010recent,jlm2016Floquet,xie1991thesis},
to prove Theorem~\ref{thm:IntroNoFold} 
it suffices to show that there are no subsequent bifurcations for $ \alpha \in ( \pp , \pp + \delta_1)$, where $\delta_1 = 7.3165 \times 10^{-4}$. 
We prove in Proposition~\ref{prop:TightEstimate} that for  $ 0 < \epsilon \leq 0.1$  
 there is a locally unique $( \hat{\alpha}_\epsilon , \hat{\omega}_\epsilon, \hat{c}_\epsilon)$ which solves $F_\epsilon ( \hat{\alpha}_\epsilon , \hat{\omega}_\epsilon, \hat{c}_\epsilon)=0$. 
However, this is not sufficient. 
To show that the branch of periodic solutions does not have any subsequent bifurcations, we prove that $\hat{\alpha}_\epsilon $ is monotonically increasing in $ \epsilon$. 
Since  
$\frac{d}{d\epsilon} \hat{\alpha}_\epsilon \approx \tfrac{2 \epsilon}{5} ( \tfrac{3 \pi}{2} -1 ) $, 
in order to have any hope of proving $\frac{d}{d\epsilon} \hat{\alpha}_\epsilon >0$, it is imperative that we derive an $\cO(\epsilon^2)$ approximation of $ \frac{d}{d\epsilon} \hat{\alpha}_\epsilon$, an approach we take from the beginning of our analysis. 


Theorem~\ref{thm:IntroNoFold} does not fully resolve Conjecture~\ref{conj:ConjJones}, as it makes no claims about the (non)existence of isolas of solutions (disjoint from the Hopf bifurcation branch).
Hence, as a corollary to Theorem \ref{thm:IntroNoFold}, we are able to reduce the Jones' conjecture~\ref{conj:ConjJones} to the following statement:
\begin{conjecture}
	The only slowly oscillating periodic solution to Wright's equation are those originating from the Hopf bifurcation. In particular, there are no isolas of SOPS.	
	\label{conj:ConjRemainder}
\end{conjecture}

Resolving Conjecture \ref{conj:ConjRemainder} is still a nontrivial task.
For $\alpha \geq 5.67$ a (purely analytic) proof is given in~\cite{xie1991thesis}, whereas for $\alpha \in [1.94,6]$ a (computer assisted) proof is provided in~\cite{jlm2016Floquet}. This leaves a gap of parameter values $\alpha \in (\pp, 1.94)$. 
In Theorem~\ref{thm:UniqunessNbd2} we prove a partial result: we construct a neighborhood about the bifurcation point independent of any $ \epsilon$-scaling such that the only periodic orbits for 
$\alpha \in ( \pp , \pp + 0.00553]$
are those originating from the Hopf bifurcation.
This implies that there are no ``spurious'' solutions (for example on isolas) 
in this explicit neighborhood of the bifurcation point. 
By applying the techniques used in \cite{neumaier2014global,jlm2016Floquet}  to rule out solutions which have either a large amplitude or a frequency dissimilar from $\pp$, we expect the Conjecture  \ref{conj:ConjRemainder}  could be proved for 
$ \alpha \in ( \pp , \pp + 0.00553]$. 


