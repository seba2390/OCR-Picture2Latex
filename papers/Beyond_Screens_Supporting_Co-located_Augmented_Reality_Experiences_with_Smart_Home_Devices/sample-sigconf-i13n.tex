%%
%% This is file `sample-sigconf-i13n.tex',
%% generated with the docstrip utility.
%%
%% The original source files were:
%%
%% samples.dtx  (with options: `sigconf-i13n')
%% 
%% IMPORTANT NOTICE:
%% 
%% For the copyright see the source file.
%% 
%% Any modified versions of this file must be renamed
%% with new filenames distinct from sample-sigconf-i13n.tex.
%% 
%% For distribution of the original source see the terms
%% for copying and modification in the file samples.dtx.
%% 
%% This generated file may be distributed as long as the
%% original source files, as listed above, are part of the
%% same distribution. (The sources need not necessarily be
%% in the same archive or directory.)
%%
%%
%% Commands for TeXCount
%TC:macro \cite [option:text,text]
%TC:macro \citep [option:text,text]
%TC:macro \citet [option:text,text]
%TC:envir table 0 1
%TC:envir table* 0 1
%TC:envir tabular [ignore] word
%TC:envir displaymath 0 word
%TC:envir math 0 word
%TC:envir comment 0 0
%%
%%
%% The first command in your LaTeX source must be the \documentclass command.
\documentclass[sigconf, language=french,
language=german, language=spanish, language=english]{acmart}

%%
%% \BibTeX command to typeset BibTeX logo in the docs
\AtBeginDocument{%
  \providecommand\BibTeX{{%
    \normalfont B\kern-0.5em{\scshape i\kern-0.25em b}\kern-0.8em\TeX}}}

%% Rights management information.  This information is sent to you
%% when you complete the rights form.  These commands have SAMPLE
%% values in them; it is your responsibility as an author to replace
%% the commands and values with those provided to you when you
%% complete the rights form.
\setcopyright{acmcopyright}
\copyrightyear{2018}
\acmYear{2018}
\acmDOI{XXXXXXX.XXXXXXX}

%% These commands are for a PROCEEDINGS abstract or paper.
\acmConference[Conference’17]{}{July 2017}{Washington, DC, USA}
\acmPrice{15.00}
\acmISBN{978-1-4503-XXXX-X/18/06}


%%
%% Submission ID.
%% Use this when submitting an article to a sponsored event. You'll
%% receive a unique submission ID from the organizers
%% of the event, and this ID should be used as the parameter to this command.
%%\acmSubmissionID{123-A56-BU3}

%%
%% For managing citations, it is recommended to use bibliography
%% files in BibTeX format.
%%
%% You can then either use BibTeX with the ACM-Reference-Format style,
%% or BibLaTeX with the acmnumeric or acmauthoryear sytles, that include
%% support for advanced citation of software artefact from the
%% biblatex-software package, also separately available on CTAN.
%%
%% Look at the sample-*-biblatex.tex files for templates showcasing
%% the biblatex styles.
%%

%%
%% The majority of ACM publications use numbered citations and
%% references.  The command \citestyle{authoryear} switches to the
%% "author year" style.
%%
%% If you are preparing content for an event
%% sponsored by ACM SIGGRAPH, you must use the "author year" style of
%% citations and references.
%% Uncommenting
%% the next command will enable that style.
%%\citestyle{acmauthoryear}

%%
%% end of the preamble, start of the body of the document source.
\begin{document}

%%
%% The "title" command has an optional parameter,
%% allowing the author to define a "short title" to be used in page headers.
\title{Beyond Screens: Supporting Co-located Augmented Reality
Experiences with Smart Home Devices}

%%
%% The "author" command and its associated commands are used to define
%% the authors and their affiliations.
%% Of note is the shared affiliation of the first two authors, and the
%% "authornote" and "authornotemark" commands
%% used to denote shared contribution to the research.
%\author{Ben Trovato}
%\authornote{Both authors contributed equally to this research.}
%\email{trovato@corporation.com}
%\orcid{1234-5678-9012}
%\authornotemark[1]
%\email{webmaster@marysville-ohio.com}
%\affiliation{%
  %\institution{Institute for Clarity in Documentation}
 %\streetaddress{P.O. Box 1212}
  %\city{Dublin}
  %\state{Ohio}
  %\country{USA}
  %\postcode{43017-6221}}

\author{Ava Robinson, Yu Jiang Tham, Rajan Vaish, Andrés Monroy-Hernández}
\affiliation{%
  \institution{Snap Inc.}
  \country{USA}}
\email{arobinson,yujiang,rvaish,amh@snap.com}

%%
%% By default, the full list of authors will be used in the page
%% headers. Often, this list is too long, and will overlap
%% other information printed in the page headers. This command allows
%% the author to define a more concise list
%% of authors' names for this purpose.
\renewcommand{\shortauthors}{Trovato and Tobin, et al.}

%%
%% The abstract is a short summary of the work to be presented in the
%% article.
\begin{abstract}
We introduce Spooky Spirits, an AR game that makes novel use of
everyday smart home devices to support co-located play. Recent
exploration of co-located AR experiences consists mainly of digital
visual augmentations on mobile or head-mounted screens. In this
work, we leverage widely adopted smart lightbulbs to expand AR
capabilities beyond the digital and into the physical world, further
leveraging the physicality of users’ shared environment.

\end{abstract}


%%
%% The code below is generated by the tool at http://dl.acm.org/ccs.cfm.
%% Please copy and paste the code instead of the example below.
%%
\begin{CCSXML}
<ccs2012>
 <concept>
  <concept_id>10010520.10010553.10010562</concept_id>
  <concept_desc>Computer systems organization~Embedded systems</concept_desc>
  <concept_significance>500</concept_significance>
 </concept>
 <concept>
  <concept_id>10010520.10010575.10010755</concept_id>
  <concept_desc>Computer systems organization~Redundancy</concept_desc>
  <concept_significance>300</concept_significance>
 </concept>
 <concept>
  <concept_id>10010520.10010553.10010554</concept_id>
  <concept_desc>Computer systems organization~Robotics</concept_desc>
  <concept_significance>100</concept_significance>
 </concept>
 <concept>
  <concept_id>10003033.10003083.10003095</concept_id>
  <concept_desc>Networks~Network reliability</concept_desc>
  <concept_significance>100</concept_significance>
 </concept>
</ccs2012>
\end{CCSXML}

\ccsdesc[500]{Human-centered computing Ubiquitous and mobile computing systems and tools; Ubiquitous and mobile computing}


%%
%% Keywords. The author(s) should pick words that accurately describe
%% the work being presented. Separate the keywords with commas.
\keywords{Co-Located, Augmented Reality, IoT, Embodied, Social, Mobile AR,
Smart Home Devices
}

%% A "teaser" image appears between the author and affiliation
%% information and the body of the document, and typically spans the
%% page.



%\received{20 February 2007}
%\received[revised]{12 March 2009}
%\received[accepted]{5 June 2009}

%%
%% This command processes the author and affiliation and title
%% information and builds the first part of the formatted document.
\maketitle

\section{Introduction}
Recent research has shown how co-located AR experiences can
enable people to have fun together, but most consist of only digital
visual augmentations using phone-based AR \cite{dagan2022project}. Additionally, prior
work has shown that AR is well-suited for co-located experiences
because it can be grounded in the environment \cite{wetzel2008guidelines}, for example
by using physical objects as enablers, or inputs, of the experience
\cite{dagan2022project}. However, these augmentations are constrained to pixels on a
screen rather than augmenting the physical space itself or using
physical objects as outputs in the experience. The recent increase
of IoT devices in people’s homes \cite{koskela2004evolution} presents an opportunity to use
smart home devices to extend the augmentation of the physical
world for co-located experiences.


\begin{figure}[h!] \includegraphics[width=0.48\textwidth]{figure1.png}
  \caption{Left: Users playing Spooky Spirits with the smart
light in the background and a user doing the T Pose body
gesture. Middle: Game prompting users to ask a question
while the light illuminates the room white indicating “spirits”
listening. Right: Room getting illuminated red by the light,
representing a “no” answer.
}
  \label{fig1}
\end{figure}

Leveraging this opportunity, we explore using IoT devices, in
particular a smart light, as a core aspect of an experience and novel form of interaction. We present Spooky Spirits, a mobile AR game\footnote{Implemented as a Snapchat Lens using Lens Studio \url{https://ar.snap.com/}} built for two users to play together in person along with a smart
light \footnote{Used Kasa Smart Lightbulbs \url{https://www.kasasmart.com/us/products/smart-lighting}}. Our system allows users to easily connect their smart lights
to the mobile experience via a web interface.

We evaluated our system with 10 pairs of participants by observing how they use the experience and interviewing them to
learn more about the immersive value and impact of augmenting a
shared physical space using an IoT device. We hope to motivate and
inform future designs of mobile AR and IoT co-located experiences.

\section{DESIGN AND EXPERIENCE OVERVIEW}
The Spooky Spirits experience piggybacks off of existing spirit
summoning and fortune-telling games such as Ouija boards\footnote{Ouija board game:\url{https://en.wikipedia.org/wiki/Ouija}}and
Magic 8-Balls\footnote{Magic 8-Ball game: \url{https://en.wikipedia.org/wiki/Magic_8_Ball}}
. This narrative embraces the magic and spookiness
of a smart light changing colors without manual input. To set up
the experience users must 1) have a smart light in their space 2)
connect their light to the mobile game using a website, and 3) find
a partner to play with in person.


Device arrangement is a key design consideration for co-located
experiences \cite{isbister2018social}\cite{lundgren2015designing}. Our experience involves two users, a phone
running the game, and a smart light. When playing Spooky Spirits, one user, player 1, holds the phone with the game running and
guides their partner, player 2, through the playful “summoning
ritual” by communicating the instructions they receive via the
mobile UI. When the experience first begins, the game creates the
spooky ambiance by turning the smart light blue. Player 1 then
directs their partner to stand in front of the camera and position
their body in a specific gesture, for example, a T Pose, which is
detected using Full Body Triggers \footnote{https://docs.snap.com/lens-studio/references/templates/object/full-body-triggers}
(see Fig. 1 - left). After the body
gesture is detected, the light turns white providing visual feedback
for both users indicating that the “spirits” are listening and users are
prompted to ask a yes-no question about the future aloud (see Fig.
1 - Middle). For example, users might ask "Will I win the lottery?".


To make the light a salient part of the experience we chose to
use it not only as an enhancement to ambiance but also as a source
of information in the narrative of the experience. After users ask a
yes-no question aloud, the light will turn green or red representing
the answer as yes or no respectively (see Fig. 1 - Right), making the
experience dependent on the information from the state of the light.
In this design, the smart light acts as the sole source of immersion
and augmentation for player 2. This allows us to explore if the
experience is immersive even without mobile visual augmentation
for all players.


This design encourages users to work together and leverages
the physicality of users’ space by augmenting their environment
beyond pixels on a screen. This provides an opportunity to observe
the impact and immersive value that the smart home device brings
to the experience since only one user can see the UI and AR on the
mobile device, while the other user can only see the effects of the
light in their environment. This experience extends augmentation
beyond mobile for both users.

\section{ SMART DEVICE INTEGRATION SYSTEM}


\begin{figure}[h!] \includegraphics[width=0.3\textwidth]{figure2.png}
  \caption{Website for users to launch the game and generate
a 5-digit code to connect to their smart light.}
  \label{fig2}
\end{figure}

\begin{figure}[h!] \includegraphics[width=0.3\textwidth]{figure3.png}
  \caption{Overview of the backend system for integrating
Kasa Devices into the game.}
  \label{fig3}
\end{figure}

Our system involves a web app\footnote{https://spookyspirits.letsplayirl.com/} we built for users to connect to
their smart lights (see Fig. 2). We chose to use Kasa Smart Light
Bulbs as they have a public API, are easy for non-technical users to
set up, and are relatively inexpensive. As set up for our experience,
users use our web interface to log in to their existing Kasa account,
select the bulb they would like to use, and obtain a 5-digit randomly
generated code that is stored in DynamoDB. Next, users launch
the game\footnote{They launched by scanning a code in Snapchat}
and enter the 5-digit code to connect the game to their
smart light. By using an external interface for login we hoped
to simplify the connection process and reduce friction within the
mobile experience. The details of the system consist of AWS Amplify
running a Next.js app with 1) a server-side component that handles
API calls to get and set the settings of the smart light and 2) a client side component that displays the website to the user (see Fig. 3).
The Next.js app also polls from a DynamoDB database periodically
to check for changes for each device, represented by the generated
5-digit code, and if there are changes, it sends the new data to the
Kasa Cloud using the Kasa public API, which modifies the state of
the physical devices. The Snapchat Lens will make get and set calls
to the AWS Remote API using the entered 5-digit code to change
the state of the device.

\section{USER STUDY OBSERVATIONS}

We performed a pilot study with 10 pairs of participants playing
with the Spooky Spirits experience together with a smart light and
conducted semi-structured interviews.


We aimed to explore the impact of a smart home device as a core
aspect of a co-located AR experience to help inform future designs
of similar experiences. Here are six main observations


\begin{enumerate}
\item  Augmentation of the physical world beyond mobile superseded the setup effort. Users noted that the setup friction
was high due to the smart light, however, they felt that the
light added to the overall immersive value.
\item  Augmentation via IoT devices enabled immersion beyond mobile and overcame asymmetric device access. Although one
user did not experience mobile visual augmentation, generally they still felt immersed and involved in the experience
because of the augmentations from the smart light.
\item  Different setup decisions led to new interactions. We observed
users who placed the bulb in a ceiling socket try to reach up
towards the light and pay direct attention to the bulb itself,
whereas users who placed the bulb in a desk or standing
lamp did not try to touch the physical bulb and instead paid
attention to the effect the bulb had on overall lighting.
\item  The state of the physical environment affected its augmentation. Users who played in dark lit rooms felt the experience
was more immersive than those who played in rooms with
more secondary light because the changes in the color and
brightness of the light were more salient.
\item  IoT devices could have played a more important role in giving feedback. Users sometimes struggled to know what to
expect from the smart light or when to wait for the light to
change.

\item  IoT devices were able to serve multiple roles, from creating
ambiance to providing information. During the game, blue
light was used to create spooky ambiance, and green or
red light encoded information (yes or no answers to users’
questions) that users easily decoded.
\end{enumerate}
In the future, we hope to build and study more co-located AR
experiences that use a variety of IoT devices as novel forms of
augmentation and interactions. We hope this inspires future designers to continue to design more mobile AR and IoT co-located
experiences.




%%
%% The acknowledgments section is defined using the "acks" environment
%% (and NOT an unnumbered section). This ensures the proper
%% identification of the section in the article metadata, and the
%% consistent spelling of the heading.
\begin{acks}
Many thanks to Melissa Powers, Erica Principe Cruz, Samantha
Reig, Tim Chong, Jennifer He, and Eunice Kim for their help and support in designing, building, and studying this experience. Also,
many thanks to all our study participants for their time.

\end{acks}

%%
%% The next two lines define the bibliography style to be used, and
%% the bibliography file.
\bibliographystyle{ACM-Reference-Format}
\bibliography{final}

\end{document}
\endinput
%%
%% End of file `sample-sigconf-i13n.tex'.
