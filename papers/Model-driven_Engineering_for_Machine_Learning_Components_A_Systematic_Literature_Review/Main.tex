%% 
%% Copyright 2019-2020 Elsevier Ltd
%% 
%% This file is part of the 'CAS Bundle'.
%% --------------------------------------
%% 
%% It may be distributed under the conditions of the LaTeX Project Public
%% License, either version 1.2 of this license or (at your option) any
%% later version.  The latest version of this license is in
%%    http://www.latex-project.org/lppl.txt
%% and version 1.2 or later is part of all distributions of LaTeX
%% version 1999/12/01 or later.
%% 
%% The list of all files belonging to the 'CAS Bundle' is
%% given in the file `manifest.txt'.
%% 
%% Template article for cas-sc documentclass for 
%% double column output.

% \documentclass[a4paper,fleqn,longmktitle]{cas-sc}
\documentclass[a4paper]{cas-sc}
\usepackage[numbers]{natbib}
%\usepackage[authoryear]{natbib}
%\usepackage[authoryear,longnamesfirst]{natbib}
\usepackage{caption}
\usepackage{subcaption}
% \usepackage[resetlabels,labeled]{multibib}
% \newcites{S}{Primary Studies}
\usepackage{caption}
\usepackage{float}
\usepackage{changepage} % adjusting paragraph width
\usepackage{hyperref} % 
\usepackage[most]{tcolorbox}
\newcommand{\todo}[1]{}
\renewcommand{\todo}[1]{{\color{red} TODO: {#1}}}

\newcommand{\sectopic}[1]{\vspace{0.2em}\par\noindent{\textit{\bfseries #1}}}


%%%Author definitions
\def\tsc#1{\csdef{#1}{\textsc{\lowercase{#1}}\xspace}}
\tsc{WGM}
\tsc{QE}
\tsc{EP}
\tsc{PMS}
\tsc{BEC}
\tsc{DE}
%%%
\newcommand\Tstrut{\rule{0pt}{2.6ex}}       % "top" strut
\newcommand\Bstrut{\rule[-0.9ex]{0pt}{0pt}} % "bottom" strut
\newcommand{\TBstrut}{\Tstrut\Bstrut} % top&bottom struts

\newtcolorbox{myframe}[1][]{
  enhanced,
  arc=0pt,
  outer arc=0pt,
  colback=white,
  boxrule=0.8pt,
  #1
}
%%%% set another level of sub-headings
% \usepackage{titlesec} 

% \setcounter{secnumdepth}{4}

% \titleformat{\paragraph}
% {\normalfont\normalsize\bfseries}{\theparagraph}{1em}{}
% \titlespacing*{\paragraph}
% {0pt}{3.25ex plus 1ex minus .2ex}{1.5ex plus .2ex}

%%%%%%%%%%%%%%%%%%%%

% Uncomment and use as if needed
%\newtheorem{theorem}{Theorem}
%\newtheorem{lemma}[theorem]{Lemma}
%\newdefinition{rmk}{Remark}
%\newproof{pf}{Proof}
%\newproof{pot}{Proof of Theorem \ref{thm}}

\begin{document}
\let\WriteBookmarks\relax
\def\floatpagepagefraction{1}
\def\textpagefraction{.001}

% Short title
\shorttitle{Model Driven Engineering for Machine Learning Components}

% Short author
\shortauthors{Naveed et~al.}

% Main title of the paper
\title [mode = title]{Model Driven Engineering for Machine Learning Components: A Systematic Literature Review}     
% Title footnote mark
% eg: \tnotemark[1]
%\tnotemark[1,2]

% Title footnote 1.
% eg: \tnotetext[1]{Title footnote text}
% \tnotetext[<tnote number>]{<tnote text>} 
%\tnotetext[1]{}

%\tnotetext[2]{}


% First author
\author[1]{Hira Naveed}

                        % [type=editor,
                        % auid=000,bioid=1,
                        % prefix=Sir,
                        % role=Researcher,
                        % orcid=]

\ead{hira.naveed@monash.edu}
\affiliation[1]{organization={Monash University}, 
    %addressline={}, 
    city={Clayton},
    % citysep={}, % Uncomment if no comma needed between city and postcode
    %postcode={}, 
    state={VIC},
    country={Australia}}

\author[1]{Chetan Arora}
\ead{chetan.arora@monash.edu}



% Second autor
\author[2]{Hourieh Khalajzadeh}
\ead{hourieh.khalajzadeh@deakin.edu.au}
\affiliation[2]{organization={Deakin University}, 
    %addressline={ }, 
    city={Burwood},
    % citysep={}, % Uncomment if no comma needed between city and postcode
    %postcode={}, 
    state={VIC},
    country={Australia}}

% Fifth author
\author[1]{John Grundy}
\ead{john.grundy@monash.edu}

\author[1]{Omar Haggag}
\ead{omar.haggag@monash.edu}
% Here goes the abstract
\begin{abstract}
\textbf{\textit{\textbf{Context}: }} Machine Learning (ML) has become widely adopted as a component in many modern software applications. Due to the large volumes of data available, organizations want to increasingly leverage their data to extract meaningful insights and enhance business profitability. ML components enable predictive capabilities, anomaly detection, recommendation, accurate image and text processing, and informed decision-making. However, developing systems with ML components is not trivial; it requires time, effort, knowledge, and expertise in ML, data processing, and software engineering. There have been several studies on the use of model-driven engineering (MDE) techniques to address these challenges when developing traditional software and cyber-physical systems. Recently, there has been a growing interest in applying MDE for systems with ML components. \\
\textbf{\textit{Objective: }} The goal of this study is to further explore the promising intersection of MDE with ML (MDE4ML) through a systematic literature review (SLR). Through this SLR, we wanted to analyze existing studies, including their motivations, MDE solutions, evaluation techniques, key benefits and limitations.\\
\textbf{\textit{Method: }} Our SLR is conducted following the well-established guidelines by Kitchenham. We started by devising a protocol and systematically searching seven databases, which resulted in 3,934 papers. After iterative filtering, we selected 46 highly relevant primary studies for data extraction, synthesis, and reporting. \\
\textbf{\textit{Results: }} We analyzed selected studies with respect to several areas of interest and identified the following: 1) the key motivations behind using MDE4ML; 2) a variety of MDE solutions applied, such as modeling languages, model transformations, tool support, targeted ML aspects, contributions and more; 3) the evaluation techniques and metrics used; and 4) the limitations and directions for future work. We also discuss the gaps in existing literature and provide recommendations for future research.  \\
\textbf{\textit{Conclusion: }} This SLR highlights current trends, gaps and future research directions in the field of MDE4ML, benefiting both researchers and practitioners.
\end{abstract}

% Use if graphical abstract is present
% \begin{graphicalabstract}
% \includegraphics{figs/grabs.pdf}
% \end{graphicalabstract}

% Research highlights
% \begin{highlights}
% % \item Research highlights item 1
% % \item Research highlights item 2
% % \item Research highlights item 3
% \end{highlights}

% Keywords
% Each keyword is separated by \sep
\begin{keywords}
model driven engineering \sep 
software engineering \sep artificial intelligence \sep machine learning \sep systematic literature review
\end{keywords}


\maketitle

\section{Introduction}
\label{sec:Introduction}


The goal in top-$\size$ recommendation is to recommend to each
consumer a small set of $\size$ items from a large collection of
items~\cite{cremonesi2010performance}.  For example, Netflix may want
to recommend $\size$ appealing movies to each consumer.  Collaborative
Filtering (CF)~\cite{herlocker2002empirical,lee2012comparative} is a
common top-$\size$ recommendation method.  CF infers user interests by
analyzing partially observed user-item interaction data, such as user
ratings on movies or historical purchase
logs~\cite{kanagal2012supercharging}. The main assumption in CF is that
users with similar interaction patterns have similar interests.


Standard CF methods for top-$\size$ recommendation focus on making  suggestions  that accurately reflect the user's preference history. However, as  observed in previous work,  CF recommendations are generally biased toward  popular items, leading to a rich get richer effect~\cite{vargas2014improving,steck2011item}.  The major reasons for this are \textit{popularity bias} and \textit{sparsity} of CF interaction data (detailed in Section~\ref{sec:related-work}). In a nutshell, to maintain  accuracy, recommendations are generated from the dense regions of the data,  where the popular items lie.  

However,  accurately suggesting popular items, may not be satisfactory for the consumers. For example, in Netflix, an accuracy-focused movie recommender may recommend ``Star Wars: The Force Awakens'' to users who have seen ``Star Wars: Rogue One''.  But, those users are probably already aware of ``The Force Awakens''. Considering additional factors, such as novelty of recommendations,  can lead to more effective suggestions~\cite{cremonesi2010performance,Castells2015,zhang2008avoiding,ziegler2005improving,zhang2012auralist}. 
%Second, accuracy-focused models typically achieve a   overall item-space coverage across their recommendations,  whereas high item-space coverage helps providers of the items increase revenue
%, users satisfaction since they are  likely already aware of or can find these items on their own.  

Focusing on popular items also adversely affects the satisfaction of  the providers of the items. This is because  accuracy-focused models typically achieve a  low overall item space coverage across their recommendations, whereas   high item space coverage helps providers of the items increase their revenue~\cite{vargas2014improving,Castells2015,adomavicius2011maximizing,anderson2006thelongtail, yin2012challenging,adomavicius2012improving}.
%accuracy-focused models typically achieve a

In contrast to the relatively small number of popular items, there are copious  {\it long-tail\/} items that have fewer observations (e.g., ratings) available. More precisely,  using the Pareto  principle (i.e.,~the $80/20$ rule),  long-tail items can be defined as items that generate the lower $20\%$ of observations~\cite{yin2012challenging}. Experimentally we found that these items correspond to almost $85\%$ of the items in several datasets (Sections~\ref{sec:Notation} and \ref{sec:Experiments}). %Table~\ref{tab:DatasetStatsticsSmall})


As previously shown, one way to improve the novelty of top-$\size$ sets is to recommend interesting long-tail items~\cite{cremonesi2010performance,ge2010beyond}.  The intuition  is that since they have fewer observations available,  they are more likely to be unseen~\cite{Kaminskas:2016:DSN:3028254.2926720}.  
 %For example, in online commerce,  newly added items are long-tail items that are yet to be discovered.  
Moreover, long-tail item promotion also results in higher overall coverage of the item space%, which increases profits for providers of the items
~\cite{vargas2014improving,Castells2015,zhang2008avoiding,zhang2012auralist,adomavicius2011maximizing,anderson2006thelongtail,yin2012challenging,jambor2010optimizing}. Because long-tail promotion reduces accuracy~\cite{steck2011item}, there are trade-offs to be explored.


%original submitted to ICDE
%This work studies three aspects of top-$\size$ recommendation: accuracy, novelty, and item-space coverage, and examines their trade-offs. In most previous work, predictions of a base recommendation system are re-ranked to handle their trade-offs~\cite{adomavicius2012improving,jambor2010optimizing,zhang2013personalize,wang2009portfolio}. Due to performance considerations, however, these techniques are not customized per user. For example,  parameters that balance the trade-off between novelty and accuracy are cross-validated at a global level.  This can be detrimental since users have varying preferences for  objectives such as long-tail novelty. We explore how to  automatically infer  user  preference for long-tail novelty, and how to leverage  it to correct  the popularity bias in standard recommender models. Our work does not rely on any additional contextual data, although such data, if available, can help promote newly-added long-tail items~\cite{agarwal2009regression,Saveski:2014:ICR:2645710.2645751}.

This work studies three aspects of top-$\size$ recommendation: accuracy, novelty, and item space coverage, and examines their trade-offs. In most previous work, predictions of a base recommendation algorithm are \textit{re-ranked} to handle these trade-offs~\cite{adomavicius2012improving,jambor2010optimizing,zhang2013personalize,wang2009portfolio}. The re-ranking models are computationally efficient but suffer from two drawbacks. First, due to performance considerations,  parameters that balance the trade-off between novelty and accuracy  are not customized per user. Instead they are cross-validated at a global level.  This can be detrimental since users have varying preferences for  objectives such as long-tail novelty. Second,  the re-ranking methods are often limited to a specific base recommender  that may be sensitive to dataset density. 
As a result, the datasets are pruned and the problem is studied in dense settings~\cite{adomavicius2012improving,ho2014likes}; but real world  scenarios are often sparse~\cite{kanagal2012supercharging,liu2017experimental}.   
% Because  dataset density can impact the performance of most base recommenders (like R-SVD), which in turn affects the performance of the re-ranking model, 

\iffalse
We address these limitations by directly inferring  user  preference for long-tail novelty  from interaction data.  This  allows us to customize the re-ranking  per user, and design a \textit{generic} framework, which resolves the second problem. In particular, since the long-tail novelty preferences are estimated independently of any base  recommender model, we can  plug-in an appropriate base recommender w.r.t. the dataset sparsity.% including ones that are more suitable for sparse settings.  

Modelling  user  preference for  long-tail novelty using only item popularity statistics, e.g., the average popularity of rated items as in~\cite{jugovac2017efficient}, disregards additional information like whether the user found the item interesting and the long-tail preferences of other users  of the items. \iffalse To incorporate them, we introduce the notion of  \emph{item long-tail importance}. Both  user long-tail preferences and item long-tail importance are dependent:  a user has high preference for discovering long-tail items if she is interested in important long-tail items, and an item that is associated with many of these kinds of users is likely to be more important.  We propose a joint optimization framework to directly learn,  from interaction data, both the users' long-tail preferences and the  items' long-tail importance. \fi
We propose an optimization approach that  incorporates  this information and  directly learns,  from interaction data, the users' long-tail novelty preferences.

Next, we use these learned preferences  to design a  top-$\size$ recommendation framework thats is generic, and provides customized balance between accuracy, novelty, and coverage. We refer to it as framework as GANC.  Using GANC, we design a novel algorithm, {\it Ordered Sampling-based Locally Greedy (OSLG)\/}, that relies on the learned long-tail novelty preferences  to scalably correct for popularity bias. Our work does not rely on any additional contextual data, although such data, if available, can help promote newly-added long-tail items~\cite{agarwal2009regression,Saveski:2014:ICR:2645710.2645751}. In summary:
\fi

We address the first limitation by directly inferring  user  preference for long-tail novelty  from interaction data.   Estimating these  preferences  using only item popularity statistics, e.g., the average popularity of rated items as in~\cite{jugovac2017efficient}, disregards additional information, like whether the user found the item interesting or the long-tail preferences of other users  of the items. We propose an approach that  incorporates  this information and  learns the users' long-tail novelty preferences from interaction data.

This approach allows us to customize the re-ranking  per user, and  design a \textit{generic} re-ranking framework, which resolves the second limitation of prior work. In particular, since the long-tail novelty preferences are estimated independently of any base recommender, we can  plug-in an appropriate one w.r.t. different factors, such as the dataset sparsity.

Our top-$\size$ recommendation framework, \textbf{GANC}, is \textbf{G}eneric, and provides customized balance between \textbf{A}ccuracy, \textbf{N}ovelty, and \textbf{C}overage. % Moreover, based on the learned long-tail novelty preferences, we also design a novel algorithm, {\it Ordered Sampling-based Locally Greedy (OSLG)\/}, that relies on the learned long-tail novelty preferences  to scalably correct for popularity bias. 
Our work does not rely on any additional contextual data, although such data, if available, can help promote newly-added long-tail items~\cite{agarwal2009regression,Saveski:2014:ICR:2645710.2645751}. In summary:

%Consider  the following toy example:
\vspace{-0.2cm}
\begin{table}[htb]
\centering
\scriptsize
%\small
\begin{tabular}{ccccccc} 
%\toprule
%&\multirow{2}{*}{}&\multicolumn{7}{c}{Ratings}\\
& & \cellcolor{blue!35}$w_1$ &\cellcolor{blue!18} $w_2$ & $\dots$ &\cellcolor{blue!8} $w_{89}$  &\cellcolor{blue!8} $w_{99}$   
\\
&   &$i_1$&$i_2$&$\dots$&$i_{89}$&$i_{90}$\\ 
\cmidrule(r){3-7} 	 
%\midrule
\cellcolor{red!35}$\theta_1$  &$u_1 $   &5 &   & $\dots$ &  &   \\
\cellcolor{red!28}$\theta_2$  &$u_2$     &5 &    & $\dots$ &  &  \\
 $\theta_3=?$  &$\bf u_3$  &5 &  &   $\dots$ &  &  \\
\cellcolor{red!10}$\theta_4$ & $u_4$  &  &5   & $\dots$ & &\\ 
\cellcolor{red!10}$\theta_5$ & $u_5$  &  & 5  & $\dots$ & &\\ 
$\theta_6=?$  & $\bf u_6$ & &5  &      $\dots$& &  \\ 
 & & $\hdots$  &$\hdots$   &$\hdots$   &$\hdots$   &$\hdots$  \\
%\midrule 
\cmidrule(r){3-7} 	 
\multicolumn{2}{c}{item pop.}  & 3  & 3  & $\dots$ &50&60\\  
%\bottomrule
%$ f_i$    &3  &3  &1  &3  &1  &2  \\  \hline
\end{tabular}
%#.
\caption{Simplified user-item interaction data. The user long-tail novelty preference ($\theta_u$), item long-tail importance weight ($w_i$) are highlighted. Darker colors indicate larger values. } \label{tab:example}
\end{table} 
\vspace{-0.2cm}
\begin{example}  
In Table~\ref{tab:example}, we are interested in estimating $\theta_3$ and $\theta_6$,  the long-tail preference of users $u_3$ and $u_6$ who have each rated a single movie. Additional ratings for other users  are not included here.  Considering only rating information, we observe $i_1$ and $i_2$ are  equally popular $|\mathcal{U}_{i_1}^{\trainset}| = |\mathcal{U}_{i_2}^{\trainset}|=3$, and $r_{31}=5$ and $r_{62}=5$. Using Eq.~\ref{eq:tfidf-risk}  we have $\theta_3 = \theta_6$. However, if we were given the long-tail preferences of the each item's user set, specifically that $u_1$ and $u_2$ have high long-tail preference (darker red), while $u_4$ and $u_5$ have lower long-tail preference (lighter red), we could conclude $i_1$ is a more important long-tail item compared to $i_2$ (indicated by a darker blue shade for $w_1$), and we expect  $\theta_3 \geq \theta_6$.

% On the other hand, if we knew that $u_4$ and $u_5$ have lower long-tail preference, we could conclude $i_2$ is a  less significant long-tail item. Therefore, However, if we  consider the long-tail preferences of other users, we may reason differently.    We need another variable $w_i$ which captures this information. 
%we would conclude that $u_3$ has higher long-tail preference compared to $u_6$, since the users $i_1$ is a more prominent long-tail item. 

% Relying only  on item popularity information, we would  conclude   $u_3$ and $u_6$ have equal long-tail preference, since $i_1$ and $i_2$ are  equally popular. However, considering  the second column,  long-tail preference of users,  long-tail importance for each item,  which captures the long-tail preference of its users. Since  that  both users of $i_1$ have high long-tail preference while  the users of $i_2$ have lower preference,  we may conclude $i_1$ is a more important long-tail item compared to $i_2$. Therefore, $u_3$'s long-tail preference should be at least as large as $u_6$'s preference. Specifically, consider two  items $i_1$ and $i_2$, with the following rating data: $i_1=\{u_1:5, u_2:5, u_3:5 \}$, $i_2=\{u_4:5, u_5:5, u_6:5\}$.  

%Table~\ref{tab:example} shows  simplified rating data. We want an estimate of the long-tail preference of $u_3$ and $u_6$, who have each  rated a single movie.  Relying only  on movie popularity information, we would  conclude   $u_3$ and $u_6$ have similar long-tail preference, since $m_1$ and $m_2$ are  equally popular. However, considering the long-tail preferences of other users of those movies, we may reason differently: since $u_1$ and $u_2$ have high long-tail preference, and $u_4$ and $u_5$ have low long-tail preference, $m_1$ is a more prominent long-tail item compared to $m_2$. Therefore, it is likely that $u_3$ has higher long-tail preference compared to $u_6$.considering the long-tail preferences of other users of those movies, we may reason differently.  For example, 
\label{ex:running}
\end{example}



%------------------------------

\iffalse
\begin{example}
Table~\ref{tab:example} shows rating data for a simplified system. %Note the user-item interaction matrix is sparse.
For this example, we define popular movies as those that have received  three or more ratings; $\{m_1, m_2, m_4\}$ are popular and  $\{m_3, m_5, m_6\}$ are niche movies. We observe $u_1$ and $u_3$  have rated relatively popular movies (risk-averse) while $u_2$ and $u_4$ have rated niche movies (risk-loving). 
\label{ex:running}
\end{example}

\begin{table}[htb]
\centering
\scriptsize
\begin{tabular}{ccccccc} 
\toprule
			&$m_1$ &$m_2$   &$m_3$    &$m_4$   &$m_5$ &$m_6$  \\ \hline 
$u_1 $ &5  &4  & - &-  &-  &-   \\
$u_2$  &-  &-  &-  &-  &5  &5   \\
$u_3$  &-  &4  &-  &5  &-  &-   \\
$u_4$  &-  &-  &3  &-  &-  &4   \\ 
$u_5$  &5  &-  &-  &3  &-  &-   \\ 
$u_6$  &4  &2  &-  &4  &-  &-   \\ 
\bottomrule
%$ f_i$    &3  &3  &1  &3  &1  &2  \\  \hline
\end{tabular}
\caption{User-Movie rating data} \label{tab:example}
\end{table}

It is essential to consider consumer characteristics in designing recommender systems so that they promote long-tail items to the right group of users and spread demand evenly between hit and niche items.  

\fi





%------------------------------
\iffalse
\begin{table}[htb]
\centering
\scriptsize
\begin{tabular}{ccccccc} 
\toprule
			&$m_1$ &$m_2$   &$m_3$    &$m_4$   &$m_5$ &$m_6$  \\ \hline 
$u_1 $ &\textbf{5}  & \textbf{4}  &\textcolor{gray}{ 1.2} &-  &-  &-   \\
$u_2$  &-  &-  &-  &-  & \textbf{5}  &\textbf{5}   \\
$u_3$  &-  &\textbf{4}  &-  &\textbf{5}  &-  &-   \\
$u_4$  &-  &-  &\textbf{3}  &-  &-  &\textbf{4}   \\ 
$u_5$  &\textbf{5}  &-  &-  &\textbf{3}  &-  &-   \\ 
$u_6$  &\textbf{4}  &\textbf{2}  &-  &\textbf{4}  &-  &-   \\ 
\bottomrule
%$ f_i$    &3  &3  &1  &3  &1  &2  \\  \hline
\end{tabular}
\caption{User-Movie rating data} \label{tab:example}
\end{table}
% $\mathcal{P}^1= \{ \mathcal{P}_1^1 \{i_1,i_2,i_3\}, \mathcal{P}_2^1:\{i_2,i_3,i_5\}  \}$
 %$\mathcal{P}^2= \{ \mathcal{P}_1^2: \{i_1,i_2,i_3\}, \mathcal{P}_2^2:\{i_2,i_5,i_6\}  \}$
 %$\mathcal{P}^3= \{ \mathcal{P}_1^3: \{i_7,i_8,i_9\}, \mathcal{P}_2^3:\{i_{10},i_{11},i_{12}\}  \}$
\begin{table}[htb]
\centering
\tiny
\begin{tabular}{ccc} 
\toprule
		&$u_1$&$u_2$  \\ \hline 
$\mathcal{P}^1 $ & $\{i_1,i_2,i_3\}$ & $\{i_2,i_3,i_5\} $ \\
$\mathcal{P}^2$ & $\{i_1,i_2,i_3\}$ & $\{i_2,i_5,i_6\} $ \\
$\mathcal{P}^3$ & $\{i_7,i_8,i_9\}$ & $\{i_{10},i_{11},i_{12} \}$ \\
\bottomrule
%$ f_i$    &3  &3  &1  &3  &1  &2  \\  \hline
\end{tabular}
\caption{Top-$\size$ allocations to users.} \label{tab:paretoExamples}
\end{table}
\fi


\iffalse
When considering long-tail items, it is important to consider consumers' willingness  to explore niche or unpopular items and their propensity towards similar items. In particular, they can be characterized by their  {\it risk degree\/} and {\it focusing degree\/}, respectively.  We compute these estimates  based on historical rating information. The following example further describes these notions in the context of movie rating data. 

\begin{example}  
Table~\ref{tab:example} shows rating data for a simplified system with $6$ users, $6$ movies, and $3$ genres. $m_i^{j}$ implies that movie $m_i$ belongs to genre $j$. Note the user-item interaction matrix is sparse. 
  For this setting, we define popular movies as those that have received  three or more ratings; $\{m_1, m_2, m_4\}$ are popular and  $\{m_3, m_5, m_6\}$ are niche movies. We now profile the users according to their risk and focusing degree. E.g., $u_1$ has rated relatively popular movies belonging to the same genre (risk-averse, high focusing degree); $u_2$ has rated niches movies in the same genre (risk-loving, high focusing degree); $u_3$ has rated popular movies in two different genres (risk-averse, low focusing degree), and $u_4$ has rated niches movies in two different genres (risk-loving, low focusing degree). 
\label{ex:running}
\end{example}
\begin{table}[htb]
\centering
\tiny
\begin{tabular}{ccccccc} 
\toprule
			&$m_1^{1}$ &$m_2^{1}$   &$m_3^{2}$    &$m_4^{3}$   &$m_5^{3}$ &$m_6^{3}$  \\ \hline 
$u_1 $ &5  &4  &-  &-  &-  &-   \\
$u_2$  &-  &-  &-  &-  &5  &5   \\
$u_3$  &-  &4  &-  &5  &-  &-   \\
$u_4$  &-  &-  &3  &-  &-  &4   \\ 
$u_5$  &5  &-  &-  &3  &-  &-   \\ 
$u_6$  &4  &2  &-  &4  &-  &-   \\ 
\bottomrule
%$ f_i$    &3  &3  &1  &3  &1  &2  \\  \hline
\end{tabular}
\caption{User-Movie rating data} \label{tab:example}
\end{table}
It is essential to consider these consumer characteristics in designing recommender systems so that they promote long-tail items to the right group of users and spread demand evenly between the hit and niche items.  
\fi
\iffalse
\begin{center}
\begin{figure*}[tp]
%\scalebox{0.5}{%
\resizebox{1\textwidth}{!}{%
%\small%\addtolength{\tabcolsep}{5pt}% below sums to 8
\begin{tabularx}{1.5\textwidth}{>{\hsize=2.5\hsize}X>{\hsize=2.5\hsize}X>{\hsize=0.5\hsize}X>{\hsize=0.5\hsize}X>{\hsize=0.5\hsize}X>{\hsize=0.5\hsize}X>{\hsize=0.5\hsize}X>{\hsize=0.5\hsize}X}
    \multirow{12}{*}{\includegraphics[scale=0.3]{codeForExample/popularity-movie.png}} & \multirow{12}{*}{\includegraphics[scale=0.3]{codeForExample/scatterplot.png}} & & & & & & \\
%   & &               &       &       &       &       &       \\
    & &\multicolumn{1}{l|}{}               &$m_1^{g1}$   	&$m_2^{g1}$    	&$m_3^{g2}$    &$m_4^{g2}$      &$m_5^{g3}$    \\ \cline{3-8}%\hline
    & &\multicolumn{1}{l|}{u1}          &5  &5  &-  &-   &-  \\
    & &\multicolumn{1}{l|}{u2}    		&-  &-  &4  &4  &5  \\
    & &\multicolumn{1}{l|}{u3}   			&1  &2  &1  &-  &-   \\
    & &\multicolumn{1}{l|}{u4}     		&1  &-  &-  &-  &-  \\
    & &               &       &       &       &       &       \\
    & &               &       &       &       &       &       \\
    & &               &       &       &       &       &       \\
    & &               &       &       &       &       &	\\
    \\
\end{tabularx}}
\caption{User-Movie interaction data a) Popularity-Movie histogram b)Movie genres/clusters c) User-Movie rating data} \label{fig:example}
\end{figure*}
\end{center}
\fi



%We propose a novel approach that allows us to  promote long-tail items in a targeted manner, thereby improving the novelty of top-$\size$ sets, the overall item-space coverage across recommendations, while maintaining reasonable levels of accuracy.

%Next, we integrate these learned preferences  in a generic  top-$\size$ recommendation framework to provide customized balance between accuracy and coverage.

%sequentially make recommendations, while adjusting its parameters with regard to the set of top-$\size$ recommendations made so far. However, since  sequential parameter updates  cause  scalability issues, we propose a sampling based algorithm. This variant of our framework, called {\it Ordered Sampling-based Locally Greedy (OSLG)\/},  allows us to  correct for the popularity bias in recommendations with regard to individual user long-tail preferences. 

%ICDE submission
%Our framework differs with  prior work in the following aspects:  unlike~\cite{adomavicius2011maximizing,adomavicius2012improving,zhang2013personalize,ho2014likes},  the long-tail preference personalization in our framework is learned rather than optimized using cross-validation or parameter tuning. In other words, our personalization method is independent of the underlying base  recommendation models.  Moreover, our framework is  generic. This enables us to  plug-in several base recommenders, and evaluate their  effectiveness without requiring  extensive tuning for the accuracy and coverage trade-off. 


%\vspace{-2.8pt}
\begin{itemize}

\item  We examine various measures for estimating user long-tail novelty preference in Section~\ref{sec:lt-pref} and formulate an optimization problem  to directly learn users' preferences for long-tail  items from interaction data in Section~\ref{sec:learning-lt-pref}. %In addition, we introduce several heuristics for measuring the user preference for less common items from historical rating data.% 

\item  We integrate the user preference estimates into GANC %, a generic re-ranking framework that provides customized balance between accuracy, novelty, and coverage 
(Section~\ref{sec:RiskbasedReranking}), and  introduce {\it Ordered Sampling-based Locally Greedy (OSLG)\/}, a scalable algorithm that relies  on user long-tail preferences to correct the popularity bias (Section~\ref{sec:optimizationAlgorithm}).
%We introduce OSLG, a scalable algorithm that relies  on user long-tail preferences to  maximize item space coverage \textcolor{red}{while maintaining acceptable levels of accuracy} (Section~\ref{sec:optimizationAlgorithm}).

\item   We conduct an extensive empirical study and evaluate performance from  accuracy, novelty, and coverage perspectives (Section~\ref{sec:Experiments}).  We use five  datasets with varying density and difficulty levels. %:  Netflix, MovieTweetings, and MovieLens (100K, 1M, 10M). 
  In contrast to most related work,  our evaluation considers realistic settings that include a large number of infrequent  items and users. %This enables us to study the impact of  data density on the performance trade-offs of several  state of the art top-$\size$ recommendation algorithms. %   %,  and use the all-items ranking protocol~\cite{steck2013evaluation,vargas2014improving}, where performance is measured using all items with train data. to evaluate the performance of several  state of the art top-$\size$ recommendation algorithms 
 
\item Our empirical results confirm that the performance of re-ranking models is impacted by the underlying   base recommender and the dataset density. Our generic approach enables us to easily incorporate a suitable base recommender to devise an effective solution for both dense and sparse settings. In dense settings, we use the same base recommender as existing re-ranking approaches, and we outperform them in accuracy and coverage metrics. For sparse settings, we plug-in a more suitable base recommender, and devise an effective solution that is competitive with existing top-$\size$ recommendation methods in accuracy and novelty. 

%Directly estimating the long-tail novelty preferences allows us to customize re-ranking per user, and  devise a generic framework.   
 
\end{itemize}

Section~\ref{sec:related-work} describes related work. Section~\ref{sec:conclusion} concludes.


\section{Background}
\label{sec:background}

%\SHAN{I think the background section is too long, maybe we can remove the SO3, SE3 parts, only keep SIM3, ellipsoid, SDF}

Rigid body orientation, pose, and similarity are represented using the $\text{SO}(3)$, $\text{SE}(3)$, and $\text{SIM}(3)$ Lie groups, respectively, defined as:
%
\begin{gather}
\label{eq:LieGroups}
\scaleMathLine{\begin{aligned}
\text{SO}(3) &\triangleq \crl{ \bfR \in \bbR^{3 \times 3} \mid \bfR^\top\bfR = \bfI, \det(\bfR) = 1},\\
\text{SE}(3) &\triangleq \crl{ \begin{bmatrix} \bfR & \bft\\\mathbf{0}^\top & 1 \end{bmatrix} \in \bbR^{4 \times 4} \,\bigg\vert\, \bfR \in SO(3), \bft \in \bbR^3}, \\
\text{SIM}(3) &\triangleq \crl{ \begin{bmatrix} s\bfR & \bft\\\mathbf{0}^\top & 1 \end{bmatrix} \in \bbR^{4 \times 4} \,\bigg\vert\, \bfR \in SO(3), \bft \in \bbR^3, s \in \bbR}.
\end{aligned}}
\raisetag{10ex}
\end{gather}
%
We overload $\bfxi_\times$ to denote a mapping from a vector in $\bbR^3$ or $\bbR^6$ or $\bbR^7$ to the Lie algebra $\mathfrak{so}(3)$, $\mathfrak{se}(3)$, or $\mathfrak{sim}(3)$, associated with the Lie groups in \eqref{eq:LieGroups}, defined as:
%
\begin{gather}
\label{eq:LieAlgebras}
\begin{aligned}
\mathfrak{so}(3) &\triangleq \crl{\bfxi_\times = \begin{bmatrix}0 & -\xi_3 & \xi_2\\\xi_3 & 0 & -\xi_1\\-\xi_2 & \xi_1 & 0 \end{bmatrix} \,\bigg\vert\, \bfxi \in \bbR^3},\\
\mathfrak{se}(3) &\triangleq \crl{\bfxi_\times = \begin{bmatrix} \bftheta_\times & \bfrho \\ \mathbf{0}^\top & 0 \end{bmatrix}\,\bigg\vert\, \bfxi = \begin{bmatrix} \bfrho \\ \bftheta\end{bmatrix} \in \bbR^6},\\
\mathfrak{sim}(3) &\triangleq \crl{\bfxi_\times = \begin{bmatrix} \sigma \bfI + \bftheta_\times & \bfrho \\ \mathbf{0}^\top & 0 \end{bmatrix} \,\bigg\vert\, \bfxi = \begin{bmatrix} \bfrho \\ \bftheta \\ \sigma \end{bmatrix} \in \bbR^7}.
\end{aligned}
\raisetag{12ex}
\end{gather}
%
We define an infinitesimal change of a Lie group element $\bfT$ via a left perturbation $\exp\prl{\bfxi_\times}\bfT$, using the exponential map $\exp\prl{\bfxi_\times}$ to retract a Lie algebra element $\bfxi_\times$ to the Lie group. Please refer to \cite[Ch.7]{BarfootBook} or \cite{Gao2017SLAM} for details. 

The coarse shape of a rigid body is represented using a \emph{quadric shape} \cite[Ch.3]{MVGBook}, $\crl{ \bfx \in \bbR^3 \mid \underline{\bfx}^\top \bfQ \underline{\bfx} \leq 0}$, where $\underline{\bfx} \triangleq [\bfx^\top, 1]^\top$ denotes the homogeneous coordinates of $\bfx$ and $\bfQ \in \bbR^{4 \times 4}$ is a symmetric matrix. An axis-aligned ellipsoid centered at the origin:
%
\begin{equation}
\label{eq:ellipsoid}
\mathcal{E}_{\bfu} \triangleq \crl{\bfx \in \bbR^3 \mid \bfx^\top \bfU^{-\top}\bfU^{-1}\bfx \leq 1},
\end{equation}
%
where $\bfU \triangleq \diag(\bfu)$ and the elements of the vector $\bfu \in \bbR^3$ specify the lengths of the semi-axes of $\mathcal{E}_{\bfu}$. An ellipsoid $\mathcal{E}_{\bfu}$ is a special case of a quadric shape with $\bfQ = \diag(\bfU^{-2},-1)$. 
% Instead of as the collection of points $\bfx$ contained in it, 
A quadric shape can also be defined in dual form, as the set of planes $\underline{\boldsymbol{\pi}} = \bfQ\underline{\bfx}$ that are tangent to the shape surface at each $\bfx$. This dual quadric surface definition is $\crl{ \bfpi \in \bbR^3 \mid \underline{\bfpi}^\top \bfQ^* \underline{\bfpi} = 0}$, where $\bfQ^* = \adj(\bfQ)$ is the adjugate of $\bfQ$.
%\footnote{If $\bfQ$ is invertible, $\bfQ^* = \adj(\bfQ) \triangleq \det(\bfQ)\bfQ^{-1}$ can be simplified to $\bfQ^* = \bfQ^{-1} = \diag(\bfU^2, -1)$ due to the scale-invariance of the dual quadric definition.}.
A dual quadric defined by $\bfQ^*$ can be scaled, rotated, or translated by a similarity transform $\bfT \in \text{SIM}(3)$ as $\bfT \bfQ^* \bfT^\top$. Similarity, a dual quadric can be projected to a lower-dimensional space by a projection matrix $\bfP = \begin{bmatrix} \bfI & \mathbf{0} \end{bmatrix}$ as $\bfP \bfQ^* \bfP^\top$.

The fine shape of a rigid body is represented as $\crl{\bfx \in \bbR^3 \mid f(\bfx) \leq 0}$ using the \emph{signed distance field} of a set $\calS \subset \bbR^3$:
%
\begin{equation}
f(\bfx) = \begin{cases}
  -d(\bfx,\partial\calS), & \bfx \in \calS,\\
  \phantom{-}d(\bfx,\partial\calS), & \bfx \notin \calS,
\end{cases}
\end{equation}
%
where $d(\bfx,\partial\calS)$ denotes the Euclidean distance from a point $\bfx \in \bbR^3$ to the boundary $\partial\calS$ of $\calS$.























%% \subsection{SE3}

%This section introduces the necessary mathmatical background. 
%The transformation in $\text{SE}(3)$ can be expressed as:
%\begin{equation}
%\bfT \triangleq \begin{bmatrix} \bfR & \bft\\\mathbf{0}^\top & 1 \end{bmatrix} \in \text{SE}(3)
%\end{equation}
%We overload $\bftheta_\times$ to denote the mapping from an axis-angle vector $\bftheta \in \mathbb{R}^3$ to a $3 \times 3$ skew-symmetric matrix $\bftheta_\times \in \mathfrak{so}(3)$ and the mapping from a position-rotation vector $\bfxi \in \mathbb{R}^6$ to a $4 \times 4$ twist matrix $\bfxi_\times \in \mathfrak{se}(3)$. We define an infinitesimal change of pose $\bfT \in SE(3)$ using a left perturbation $\exp\prl{\bfxi_\times}\bfT \in \text{SE}(3)$ (see~\cite[Ch.7]{BarfootBook}).

%% \subsection{SIM(3)}

%We use the space $\text{SIM}(3)$ of similarity transformations to capture scale $s$ in addition to pose: 
%\begin{equation}
%\bfT \triangleq \begin{bmatrix} s\bfR & \bft\\\mathbf{0}^\top & 1 \end{bmatrix} \in \text{SIM}(3).
%\end{equation}
%We also use $\bfxi$ to denote the corresponding Lie algebra $\mathfrak{sim}(3)$, as in \cite{Gao2017SLAM}:
%\begin{equation}
%\label{eq:sim3_to_SIM3}
%\scaleMathLine[0.91]{
%\begin{aligned}
%\mathfrak{sim}(3) \triangleq
%\left\{\bfxi_\times=
%\left[\begin{array}{cc}
%{\sigma \bfI+\bfphi_\times} & {\bfrho} \\
%\mathbf{0}^\top & {0}
%\end{array}\right] \in \mathbb{R}^{4 \times 4} \;\bigg\vert\; \bfxi = \left[\begin{array}{c}
%{\bfrho} \\
%{\bfphi} \\
%{\sigma}
%\end{array}\right] \in \mathbb{R}^{7}\right\}
%\end{aligned}
%}
%\end{equation}
%We define the operator:
%%\NA{what is the operator $\underline{\bfx}^\circledcirc$?}
%%\SHAN{I don't think we will use $\underline{\bfx}^\circledcirc$ anywhere}
%\begin{equation}
%\underline{\bfx}^\odot \triangleq \begin{bmatrix} \bfI_3 & -\bfx_\times & \bfx\\ \mathbf{0}^\top & \mathbf{0}^\top & 0 \end{bmatrix} \in \mathbb{R}^{4 \times 7}
%\end{equation}




%\begin{definition*}
%The \textit{signed distance field} of a set $\calS \subset \mathbb{R}^3$ is
%\begin{equation}
%f(\bfx) = \begin{cases}
%  -d(\bfx,\partial\calS) & \bfx \in \calS\\
%  d(\bfx,\partial\calS) & \bfx \notin \calS
%\end{cases}
%\end{equation}
%where $d(\bfx,\partial\calS)$ is the distance from a point $\bfx \in \mathbb{R}^3$ to the set boundary $\partial\calS$, and we use $d$ as a shorthand notation to denote this distance.
%\end{definition*}


%\begin{definition*}
%\textit{Huber error function} \cite{Huber1964Robust} with parameter $\delta > 0$ is:
%\begin{equation}
%\rho(r) \triangleq 
%\begin{cases}
%\frac{1}{2}r^2 & |r|\leq \delta,\\
%\delta(|r|-\frac{1}{2}\delta) & \text{else}.
%\end{cases}
%\end{equation}
%\end{definition*}
%whose gradient can be computed as 
%\[
%  \frac{\partial \rho(r)}{\partial r}
%  =\left\{\begin{array}{ll}
%    r & |r| \leq \delta \\
%    \text{sign}(r)\delta  & \text{ else}. 
%    \end{array}\right.
%\] 

%An axis-aligned ellipsoid centered at the origin can be described as:
%\begin{equation}
%\mathcal{E}_{\bfu} \triangleq \crl{\bfx \mid \bfx^\top \bfU^{-\top}\bfU^{-1}\bfx \leq 1},
%\end{equation}
%where $\bfU \triangleq \diag(\bfu)$ and the elements of the vector $\bfu = [a,b,c]^{\top}$ are the lengths of the semi-axes of $\mathcal{E}_{\bfu}$. In homogeneous coordinates, $\mathcal{E}_{\bfu}$ can be represented as a quadric surface~\cite[Ch.3]{MVGBook}, $\crl{ \bfx \mid \underline{\bfx}^\top \bfQ_{\bfu} \underline{\bfx} \leq 0}$, where $\bfQ_{\bfu} = \mathbf{diag}(\bfU^{-2},-1)$. This describes the ellipsoid as a collection of points lying on its surface. 

%Alternatively, a quadric can be defined by the set of planes $\underline{\boldsymbol{\pi}} = \bfQ_{\bfu}\underline{\bfx}$ tangent to its surface at $\underline{\bfx}$. This dual quadric surface is defined as $\crl{ \bfpi \mid \underline{\bfpi}^\top \bfQ_{\bfu}^* \underline{\bfpi} = 0}$, where $\bfQ_{\bfu}^* = \mathbf{adj}(\bfQ_{\bfu})$~\footnote{If $\bfQ_{\bfu}$ is invertible, $\bfQ_{\bfu}^* = \mathbf{adj}(\bfQ_{\bfu}) = \det(\bfQ_{\bfu})\bfQ_{\bfu}^{-1}$ can be simplified to $\bfQ_{\bfu}^* = \bfQ_{\bfu}^{-1} = \diag(\bfU^2, -1)$ due to the scale-invariance of the dual quadric definition.}. A dual quadric defined by $\bfQ_{\bfu}^*$ can be transformed by $\bfT \in SE(3)$ to another reference frame as $\bfQ^* = \bfT \bfQ_{\bfu}^* \bfT^\top$, which can be projected to a lower-dimensional space by $\bfP \in \mathbb{R}^{3\times 4}$ as $\bfP \bfQ^* \bfP^\top$. 
%$\bfQ^*$ can be parameterized as:
%\begin{equation}
%\label{eq:ellipsoid}
%\begin{aligned}
%\bfQ^*
%&=
%\mathbf{T} \bfQ_{\bfu}^* \mathbf{T}^{\top}=
%\left[\begin{array}{cc}
%\mathbf{R} & \bft \\
%\mathbf{0}^{\top} & 1
%\end{array}\right]\left[\begin{array}{cc}
%\mathbf{U} \mathbf{U}^{\top} & \mathbf{0} \\
%\mathbf{0} & -1
%\end{array}\right]\left[\begin{array}{ll}
%\mathbf{R}^{\top} & \mathbf{0} \\
%\bft^{\top} & 1
%\end{array}\right] \\ 
%&=
%\begin{pmatrix} 
%\mathbf{R} \mathbf{U} \mathbf{U}^\top \mathbf{R}^\top -  \bft \bft^\top & - \bft \\ -\bft^\top & -1
%\end{pmatrix}
%\end{aligned}
%\end{equation}





\begin{figure*}[!h]
 \centering
 \centerline{\includegraphics[width=17cm]{framework}}
 \caption{The proposed multi-source remote sensing image registration framework of MS-HLMO, including Harris feature point detection, Histogram of Local Main Orientation feature extraction, and multi-scale registration strategy.}
 \label{fig:framework}
\end{figure*}

The framework of the proposed MS-HLMO registration algorithm is shown in Fig.\ref{fig:framework}. The input multi-source image pair to be registered is preprocessed, which includes data normalization and basic denoising. Then, the preprocessed single-band images are used for feature points detection and feature extraction. Harris corner point detection, which contains detail treatments for multi-source images, is adopted to generate feature points between the image pair for matching. The key process of the proposed HLMO feature extraction is carried out in a multi-scale strategy, in which Gaussian pyramids are built to create a scale-space of the images. The HLMO feature descriptors of each Harris corner point are extracted on the PMOM of the images. The feature points between the image pair are then matched according to the descriptors, and Fast Sample Consensus (FSC) is carried out to remove the outliers. The matching results in the scale-space are combined through a multi-scale matching strategy. Finally, the spatial transformation between the original image pair is determined by the coordinate relationship between matched feature points according to a selected transformation model.



\subsection{Harris Feature Point Detection}
\label{ssec:subhead}
Harris corner \cite{harris1988combined} is one of the most stable feature points, which is slightly affected by intensity and scale difference and has high computational efficiency \cite{gao2021multi,2021Multi}. It has the advantage in multi-source remote sensing images with multi-modal properties and large data size. Here, the similar strategy \cite{2021Multi} is used for feature points detection. The Harris corner response of each pixel is calculated by:
\begin{equation}
cornerness = \frac{{{\rm{det}}(\textbf{\emph{M}})}}{{{\rm{tr}}(\textbf{\emph{M}})}}
\end{equation}
\begin{equation}
\textbf{\emph{M}} = \left[ {\begin{array}{*{20}{l}}
{\sum\limits_{{\bf{W}_\sigma}} {{{\bf{G}}_x}^2} }&{\sum\limits_{{\bf{W}_\sigma}} {{{\bf{G}}_x} {{\bf{G}}_y}}}\\
{\sum\limits_{{\bf{W}_\sigma}} {{{\bf{G}}_x} {{\bf{G}}_y}} }&{\sum\limits_{{\bf{W}_\sigma}} {{{\bf{G}}_y}^2}}
\end{array}} \right]
\end{equation}
where $\rm{det}(\textbf{\emph{M}})$ and $\rm{tr}(\textbf{\emph{M}})$ are the determinant and trace of $\textbf{\emph{M}}$, respectively, ${{\bf{G}}_x}$ and ${{\bf{G}}_y}$ are the image's gradient along $x$ and $y$ directions, respectively, and $\bf{W}_\sigma$ is a Gaussian window with variance $\sigma$. Pixels with strong response are considered to be feature points with distinct structure and stability between multi-source images.

An important issue in practical multi-source remote sensing image registration is that the data size and scale relations between images to be registered are diverse. For example, an image with high resolution covers a smaller spatial area. Many of the existing algorithms only deal with the ideal case that the image pair has the same scale and size. This paper focuses on solving several key problems at the same time, that is, two uncertain factors of image scale and size should be considered simultaneously. The proposed MS-HLMO adopts local non-maximum suppression (LNMS) to solve this problem. Since the size and scale difference of the image pairs are uncertain, in Harris corner detection, it is expected that the feature points in the image pair are distributed as uniformly as possible with the use of LNSM. Then, the ratio of the window size in LNMS is set depending on the ratio of the data size of the image pair:
\begin{equation}
ratio = \sqrt {\frac{{M \times N}}{{m \times {\rm{n}}}}}
\end{equation}
where $M,N$ and $m,n$ are the length and width of the two images, respectively.

\begin{figure}[!h]
    \centering
        \subfloat[]{\includegraphics[width=2.5in]{keypoints_1.pdf}
        \label{fig:harris:a}}
    \hfil
        \subfloat[]{\includegraphics[width=2.5in]{keypoints_2.pdf}
        \label{fig:harris:b}}
    \caption{Examples of feature points detection results with LNMS. (a) Detection results of the image pair with different scale. (b) Detection results of the image pair with different scale and size.}
    \label{fig:harris}
\end{figure}

Consequently, the distribution of feature points is consistent with the images' scale proportion when there are scale differences, as shown in Fig.\ref{fig:harris:a}. When there is a size difference, the feature points are far more uniformly distributed, and the repeatability is higher, as shown in Fig.\ref{fig:harris:b}.



\subsection{Histogram of Local Main Orientation}
\label{ssec:subhead}

%\subsubsection{Robust Feature of Gradient Orientation}
%
%Most feature-point-based registration algorithms use a local descriptor to extract the neighborhood information of the keypoints, and generate their feature vectors for similarity matching.  For example, SIFT, SURF, HOG and PIIFD use local gradient information for statistics. However, the performance of these algorithms is greatly reduced when processing multi-source , especially multi-sensor images. Through practice and analysis on , it is believed that multi-modal images will have unpredictable and serious differences in the magnitude of gradient, but the orientation of gradient is a more stable feature.
%
%\begin{figure}[h!]
% \begin{center}
%  \includegraphics[width=3.5in]{Robust_Gradient_Orientation.pdf}
%  \caption{Robust gradient orientation.}
%  \label{fig:res}
% \end{center}
%\end{figure}
%
%To intuitively show the characteristics of this orientation, we briefly summarize the common intensity distortion in multi-modal images, as shown in Fig.0. The four images can be taken as the edge of an object or the interface of two substances in the image. Assume that the center point is the detected feature point, and the image block represents the intensity information of the neighborhood of it. In Fig.0 (a), the intensity amplitude of the left part is lower than that of the right part. Obviously, the gradient orientation is horizontal to the right, which is 0°; Take Fig.0 (a) as the original or reference image, then Fig.0 (b), Fig.0 (c), and Fig.0 (d) is considered as intensity distortions ${{\cal F}_{{\rm{Ra}}}}( \bullet )$ of (a). In (b), the magnitude relationship of the two parts remains unchanged, but the difference between them changes. So the gradient amplitude changes, but the orientation remains 0°; In (c), due to large intensity distortion, the magnitude relationship changes. At this time, the gradient orientation is reversed compared with (a), but notice that it is still on the same line with the orientation in (a). If the gradient orientation is limited to [0°, 180°), the orientation in (c) is still 0°. (d) represents a general situation, which is considered as non-linear distortion of intensity, or some degradation of (a), such as down-sampling or blurring. At this time, the amplitude is hard to determine, but the orientation is still 0° to the right horizontally. A typical multi-source data is shown in Fig.0, which is an optical-infrared image pair. It basically contains the above intensity distortion problems, which is discussed in detail in the next subsection.
%
%In multi-source images, due to the differences in sensors, tempors, environments, etc., various intensity distortions may be caused, resulting in the multi-modal attributes of the image. The morphology of the detailed part of the image is basically in the four cases in Fig.0. In summary, the magnitude of the local gradient varies, but the orientation is basically stable, which defines the feature information that should be focused.


\subsubsection{Partial main orientation map}
Feature-point-based registration algorithms often use a local descriptor to extract the neighborhood information of the keypoints, and generate their feature vectors for similarity matching. For example, SIFT \cite{lowe2004distinctive}, HOG \cite{dalal2005histograms}, SURF \cite{bay2006surf} and PIIFD \cite{chen2010partial} employ gradient information as the basic feature. However, the performance of these algorithms is greatly degraded when processing multi-source, especially multi-sensor images. So, it is critical to extract invariant feature that is robust to ${{\cal F}_{\rm{Int}}}( \bullet )$, ${{\cal F}_{\rm{Rot}}}( \bullet )$, and ${{\cal F}_{\rm{Chg}}}( \bullet )$ for feature points description. A partial main orientation map (PMOM) is designed as the feature map in MS-HLMO, in which the Average Squared Gradient (ASG)\cite{kass1987analyzing} is adopted.

The ASG is a gradient weighting method. The elementary gradient of the image along $x$ and $y$ directions, i.e., ${\bf{G}}_x$ and ${\bf{G}}_y$ are calculated as
\begin{equation} \label{eqPMOM1}
\left[ \begin{array}{l}
{{\bf{G}}_x}(x,y)\\
{{\bf{G}}_y}(x,y)
\end{array} \right] = \left[ \begin{array}{l}
\frac{\partial }{{\partial x}}{\bf{I}}(x,y)\\
\frac{\partial }{{\partial y}}{\bf{I}}(x,y)
\end{array} \right]
\end{equation}
where ${\bf{I}}(x,y)$ represents the single-layer gray-scale image. The magnitude and orientation of its gradient, i.e., ${\bf{G}}_\rho$ and ${\bf{G}}_\varphi$ are
\begin{equation} \label{eqPMOM2}
\left[ \begin{array}{l}
{{\bf{G}}_\rho}\\
{{\bf{G}}_\varphi}
\end{array} \right] = \left[ \begin{array}{l}
\sqrt {{{\bf{G}}_x}^2 + {{\bf{G}}_y}^2} \\
\arctan \frac{{\bf{G}}_y}{{\bf{G}}_x}
\end{array} \right]\end
{equation}

In the ASG, a locally weighted squared gradient \cite{kass1987analyzing} along $x$ and $y$ directions, ${\bf{G}}_{{{\bf{W}}_\sigma},s,x}$ and ${\bf{G}}_{{\bf{W}_\sigma},s,y}$ are obtained as
\begin{equation} \label{eqPMOM3}
\left[ \begin{array}{l}
{{\bf{G}}_{{{\bf{W}}_\sigma},s,x}}\\
{{\bf{G}}_{{{\bf{W}}_\sigma},s,y}}
\end{array} \right] = \left[ \begin{array}{l}
\sum\limits_{{\bf{W}}_\sigma} {{{\bf{G}}_x}^2 - {{\bf{G}}_y}^2} \\
\sum\limits_{{\bf{W}}_\sigma} {2{{\bf{G}}_x}{{\bf{G}}_y}}
\end{array} \right]
\end{equation}
where ${\bf{W}}_\sigma$ is a Gaussian window with variance $\sigma$. Accordingly, the orientation of this gradient is
\begin{equation} \label{eqPMOM4}
{{\bf{G}}_{{{\bf{W}}_\sigma },s,\varphi }} = \angle ({{\bf{G}}_{{{\bf{W}}_\sigma },s,x}},{{\bf{G}}_{{{\bf{W}}_\sigma },s,y}})
\end{equation}
where $\angle (X,Y)$ is defined as
\begin{equation} \label{eqPMOM5}
\angle (X,Y) = \left\{ \begin{array}{l}
\arctan (\frac{Y}{X}), X \ge 0\\
\arctan (\frac{Y}{X}) + \pi , X < 0,Y \ge 0\\
\arctan (\frac{Y}{X}) - \pi , X < 0,Y < 0
\end{array} \right.
\end{equation}
making ${\bf{G}}_{{{\bf{W}}_\sigma },s,\varphi }$ within $(-\pi,\pi)$. According to \cite{kass1987analyzing}, this gradient is obtained by doubling the angle of the original gradient, so the orientation of the ASG is
\begin{equation} \label{eqPMOM6}
{{\bf{G}}_{{{\bf{W}}_\sigma},\varphi }} = \frac{1}{2}{{\bf{G}}_{{{\bf{W}}_\sigma},s,\varphi }}
\end{equation}

Compared with the classical gradient operator, the ASG orientation ${{\bf{G}}_{{{\bf{W}}_\sigma},\varphi}}\in(-\frac{\pi}{2},\frac{\pi}{2})$ reflects the weighted gradient orientation of a local region ${\bf{W}}_\sigma$, which is more robust and stable. In addition, the $x$ direction gradient is constant, and this characteristic meets the requirement that not affected by the reversal of gradient in intensity difference. Note that when $\sigma$ increases, the scale of ASG increases, which makes the local orientation more invariant to intensity difference and noise, but the uniqueness of local features decreases. From this, the following function is defined:
\begin{equation} \label{eqPMOM7}
{{\bf{G}}_{PMOM}} = \frac{1}{2}\angle (\sum\limits_\sigma  {\sum\limits_{{\bf{W}}_\sigma } {{{\bf{G}}_x}^2 - {{\bf{G}}_y}^2}} ,\sum\limits_\sigma  {\sum\limits_{{\bf{W}}_\sigma } {2{{\bf{G}}_x}{{\bf{G}}_y}} } )
\end{equation}
where a series of scale $\sigma$ are preset, the weighted responses ${\bf{G}}_{{{\bf{W}}_\sigma},s,x}$ and ${\bf{G}}_{{{\bf{W}}_\sigma},s,y}$ at each scale are added, and the ASG orientation is obtained. By filtering the image with Eq.(\ref{eqPMOM7}), the PMOM is obtained, where its value reflects the overall orientation of multiple scales in each partial area of the image.

\begin{figure}[h!]
 \begin{center}
  \includegraphics[width=3.5in]{Comparion_of_Feature_Maps.pdf}
  \caption{Comparison of feature maps of a selected scene, including a visible and an infrared image.}
  \label{fig:maps}
 \end{center}
\end{figure}

A visualized comparison of PMOM with other feature maps of typical multi-source data is shown in Fig.\ref{fig:maps}. The original data is a visible-infrared image pair, which contains obvious intensity difference. For comparison purposes, the images have been registered manually, basically eliminating the spatial differences of scale, rotation, and size. The magnitude and orientation of images' gradient are shown in Fig.\ref{fig:maps}, which are obtained using Eqs.(\ref{eqPMOM1})-(\ref{eqPMOM2}). These are the basic feature information in most algorithms \cite{lowe2004distinctive,dalal2005histograms,bay2006surf,chen2010partial,2021Multi}. It is observed that, due to the multi-modal properties of the original image pair, these two feature maps have large differences and instability, which is the main reason for the failure of most algorithms. The MIM \cite{li2019rift} of RIFT shown in Fig.\ref{fig:maps} also focuses on the local orientation of the image, where the maximum index is the main orientation among several ones. Compared with the directional gradient, Gabor transformation has a more stable response, which leads to RIFT robust to intensity difference. However, its value will also mutate due to local changes in images, and the rotation invariance is slightly poor. The image pair's PMOMs are shown at the bottom of Fig.\ref{fig:maps}. Compared with MIM, the proposed PMOM is not only more robust and stable between multi-modal images, but also continuous in value, which is conducive to achieving effective rotation invariance. In HLMO, PMOM is used as the unique feature information to extract local features of keypoints that are invariant to multi-modal properties.


\subsubsection{Descriptor extraction}

After determining the feature points and the specific feature for discrimination, the next step is to make use of the local feature information around each point and generate descriptors. Gradient Location and Orientation Histogram (GLOH) has shown excellent ability through experiments \cite{mikolajczyk2005performance}, and has been applied in multi-source remote sensing image registration \cite{dellinger2014sar,ma2016remote}. The original GLOH descriptor is a circular region divided by three circles, similar to that shown in Fig.\ref{fig:des180}, in which the two outer circular regions are divided into 4 parts, and the radius of the circular region divided are 5, 9, and 11. The partition size and the number are then improved \cite{mikolajczyk2005performance,dellinger2014sar,ma2016remote}. However, different parameters have various effects when treating multifarious types of images. In addition, if the number of outer ring regions is too small, the character of feature points is not significant, which makes it difficult to match accurately. If it is too large, the features are unstable, and the dimension of the descriptor is too high, which increases the burden of redundant calculation. To deal with this, a generalized GLOH-like (GGLOH) descriptor is proposed, and its structure of which is shown in Fig.\ref{fig:des:a}.

\begin{figure}[!h]
    \centering
        \subfloat[]{\includegraphics[width=2.2in]{Descriptor_Structure_a.png}%
        \label{fig:des:a}}
    \hfil
        \subfloat[]{\includegraphics[width=1.1in]{Descriptor_Structure_b.png}%
        \label{fig:des:b}}
    \caption{Descriptor structure of the proposed GGLOH. (a) Subregion partition of the local neighborhood of a feature point. (b) Angle quantification within each subregion.}
    \label{fig:des}
\end{figure}

Let ${{\rm{A}}^0}$ denote the central circular region, and ${\rm{A}}_j^i,(i=1,2, j=1,...,{N_{\rm{A}}})$ represent the sector subregion $j$ in the outer ring region $i$. Let ${N_{\rm{A}}}$ be the number of the subregions in each out ring region, which is even, ${\theta _{\rm{0}}}$ be the main orientation of the feature point, and $R_0$, $R_1$, $R_2$ be the radii of the central and outer regions, respectively. Note that the orientations of pixels’ gradient in each region are counted as feature information, therefore, fair use of information in each region is expected. The number of pixels in each region should be roughly the same, and the weight of the outer regions should not change due to the change of ${N_{\rm{A}}}$. So the area of each region should be the same, that is
\begin{equation} \label{eqggloh}
{N_{\rm{A}}} \cdot \pi {\rm{R}}_{\rm{0}}^2 = \pi ({\rm{R}}_{\rm{1}}^2 - {\rm{R}}_{\rm{0}}^2) = \pi ({\rm{R}}_{\rm{2}}^2 - {\rm{R}}_{\rm{1}}^2)
\end{equation}
which also fixes the relationship between $R_0$, $R_1$, $R_2$ and ${N_{\rm{A}}}$. When ${N_{\rm{A}}}$ is given different values, the stability and importance of each region’s feature remains the same. In HLMO, the GGLOH is used to extract local features on the PMOM, where the orientation values within $(-\frac{\pi}{2},\frac{\pi}{2})$ are uniformly quantified to ${N_{\rm{O}}}$ values, as shown in Fig.\ref{fig:des:b}, where ${\phi_k}(k=1,2,...,{N_{\rm{O}}})$ is the angle after quantization. A histogram with ${N_{\rm{O}}}$ bins is obtained in each region.

It is simple to achieve rotation invariance of HLMO. For each keypoint, the PMOM value at its location is the main orientation, that is, the reference orientation ${\theta _0}$ of the GGLOH. Then, all of the PMOM values within the local area of GGLOH also take ${\theta _0}$ as the reference (0°), that is, all angle values minus ${\theta _0}$, and those beyond $(-\frac{\pi}{2},\frac{\pi}{2})$ are flipped to their opposite angles.

\begin{figure}[h!]
 \begin{center}
  \includegraphics[width=3.3in]{Problem_of_180.pdf}
  \caption{The problem caused by the jump of main orientation near $-\frac{\pi}{2}$ or $\frac{\pi}{2}$.}
  \label{fig:des180}
 \end{center}
\end{figure}

Another key problem is that the rotation and nonlinear intensity difference may cause the jump of the main orientations of some feature points near $-\frac{\pi}{2}$ and $\frac{\pi}{2}$. An example is shown in Fig.\ref{fig:des180}. In PIIFD \cite{chen2010partial}, a similar problem has been discovered and improvement has been made for SIFT. Then a similar strategy is adopted to process GLOH-like descriptors,

\begin{equation}
{{\bf{D}}_1} = \left[ {\begin{array}{*{20}{c}}
{\begin{array}{*{20}{c}}
{{\bf{H}}_1^1}&{{\bf{H}}_2^1}& \cdots &{{\bf{H}}_{{{{N_{\rm{A}}}} \mathord{\left/
 {\vphantom {{{N_{\rm{A}}}} 2}} \right.
 \kern-\nulldelimiterspace} 2}}^1}
\end{array}}\\
{\begin{array}{*{20}{c}}
{{\bf{H}}_1^2}&{{\bf{H}}_2^2}& \cdots &{{\bf{H}}_{{{{N_{\rm{A}}}} \mathord{\left/
 {\vphantom {{{N_{\rm{A}}}} 2}} \right.
 \kern-\nulldelimiterspace} 2}}^2}
\end{array}}
\end{array}} \right]
\end{equation}

\begin{equation}
{{\bf{D}}_2} = \left[ {\begin{array}{*{20}{c}}
{\begin{array}{*{20}{c}}
{{\bf{H}}_{{{{N_{\rm{A}}}} \mathord{\left/
 {\vphantom {{{N_{\rm{A}}}} 2}} \right.
 \kern-\nulldelimiterspace} 2} + 1}^1}&{{\bf{H}}_{{{{N_{\rm{A}}}} \mathord{\left/
 {\vphantom {{{N_{\rm{A}}}} 2}} \right.
 \kern-\nulldelimiterspace} 2} + 2}^1}& \cdots &{{\bf{H}}_{{N_{\rm{A}}}}^1}
\end{array}}\\
{\begin{array}{*{20}{c}}
{{\bf{H}}_{{{{N_{\rm{A}}}} \mathord{\left/
 {\vphantom {{{N_{\rm{A}}}} 2}} \right.
 \kern-\nulldelimiterspace} 2} + 1}^2}&{{\bf{H}}_{{{{N_{\rm{A}}}} \mathord{\left/
 {\vphantom {{{N_{\rm{A}}}} 2}} \right.
 \kern-\nulldelimiterspace} 2} + 2}^2}& \cdots &{{\bf{H}}_{{N_{\rm{A}}}}^2}
\end{array}}
\end{array}} \right]
\end{equation}

\begin{equation} \label{eqdes}
{\bf{D}} = \left[ {\begin{array}{*{20}{c}}
{{{\bf{D}}_1} + {{\bf{D}}_2}}\\
{c\left| {{{\bf{D}}_1} - {{\bf{D}}_2}} \right|}
\end{array}} \right]
\end{equation}
where ${\bf{H}}_j^i$ is the histogram vector of gradient orientation of region ${\rm{A}}_j^i$. In this way, no matter whether the main orientation of feature points is reversed 180° or not, descriptor ${\bf{D}}$ is composed of the addition and subtraction of the upper and lower parts of GGLOH according to the main orientation axis, without changing the regions' order. Finally, a descriptor vector ${\bf{D}}_P$ is generated for the feature point $P$, whose dimension is $(2 \cdot {N_{\rm{A}}}+1) \cdot {N_{\rm{O}}}$.

%\subsection{Advanced Outlier Removal}
%\label{ssec:subhead}


\subsection{Multi-scale Registration Strategy}
\label{ssec:subhead}
Scale difference ${{\cal F}_{{\rm{Sc}}}}( \bullet )$ of multi-source images has a great influence on local features. Some algorithms have quantitative scale judging methods, such as SIFT \cite{lowe2004distinctive} and LHOPC. However, it is found that these methods are invalid in images with large modal differences. The reason is that when images do not belong to the same degradation model, it is not credible to judge the scale quantitatively through local image feature information. Multi-source images often have scale differences, and sometimes the scale proportion is unknown. In order to deal with this key problem and realize scale robustness, a multi-scale feature extraction and matching strategy is designed in MS-HLMO.
%由于传感器成像能力或成像条件的差异,图像显示出了尺度的差异。这种尺度差异往往体现在两个方面,一个是分辨率,一个是模糊程度。其中分辨率是指图像中一个像素对应实际空间的尺寸,另外一个是指

\begin{figure}[h!]
 \begin{center}
  \includegraphics[width=2.5in]{Pyramid.pdf}
  \caption{Structure of the Gaussian pyramid used in MS-HLMO.}
  \label{fig:pyramid}
 \end{center}
\end{figure}

Local information of feature points is extracted in the scale-space of the images. Based on the scale-space theory \cite{lindeberg1994scale}, the method of building image's Gaussian pyramids is adopted. The schematic diagram of establishing Gaussian pyramid of the image in the proposed algorithm is shown in Fig.\ref{fig:pyramid}. The original image is first sampled down step by step to obtain a series of images with different resolutions, that is, the first layer in each octave. Then in each octave, a series of Gaussian blurs are performed:
\begin{equation}\label{eqGauss1}
{\bf{L}} = {\bf{G}} * {\bf{I}}
\end{equation}
\begin{equation}\label{eqGauss2}
{\bf{G}} = \frac{1}{{\sqrt {2\pi {\sigma ^2}} }}{e^{\frac{{-({x^2} + {y^2})}}{{2{\sigma ^2}}}}}
\end{equation}
where ${\bf{I}}$ is the original image, ${\bf{G}}$ is a Gaussian kernel with a standard deviation of $\sigma$, and ${\bf{L}}$ is the Gaussian blur image.
%高斯尺度空间模拟出了图像尺度变化的效果,为获取图像多个尺度层面的特征提供了重要帮助。

\begin{algorithm}[htp]
\caption{\label{multi1} \footnotesize Proposed MS-HLMO Feature Extraction}
\begin{algorithmic}
\footnotesize
\STATE {\bf Input:} single-band image $\mathbf{I}$, feature point set $P_{\mathbf{I}}$, total number of octaves ${N_{\rm{GO}}}$ and layers ${N_{\rm{GL}}}$ in Gaussian pyramid, subregion and angle partition parameters ${N_{\rm{A}}}$, ${N_{\rm{O}}}$ in GGLOH , patch size $S$ of HLMO.
\STATE Through down-sampling and Eq.(\ref{eqGauss1})(\ref{eqGauss2}), the Gaussian pyramid ${\bf{G}}_{\bf{I}}(O,L)$ of image $\mathbf{I}$ is built with ${N_{\rm{GO}}}$ octaves and ${N_{\rm{GL}}}$ layers in each octave.
\STATE In each layer of ${\bf{G}}_{\bf{I}}(O,L)$:
\STATE \hspace*{0.1in}Calculate the PMOM of this layer to get ${\bf{F}}_{\bf{I}}(O,L)$ according to Eq.(\ref{eqPMOM1})(\ref{eqPMOM7})
\STATE \hspace*{0.1in}For each feature point $P$ in $P_{\bf{I}}$:
\STATE \hspace*{0.2in}Calculate the corresponding position
\STATE \hspace*{0.2in}Take the PMOM value at the position as the main orientation ${\theta_0}$
\STATE \hspace*{0.2in}Taking the main orientation as the reference direction ($0^{\circ}$), establish a GGLOH window with size (diameter) of S
\STATE \hspace*{0.2in}Statistics the PMOM value within each region of GGLOH to obtain the basic feature descriptor $D_{1}(P,O,L)$ and $D_{2}(P,O,L)$
\STATE \hspace*{0.2in}Obtain the descriptor $D(P,O,L)$ of $P$ with Eq.\ref{eqdes}.
\STATE {\bf Output:} feature descriptor set $D_{P_{\bf{I}}}(O,L)$
\end{algorithmic}
\end{algorithm}

After the scale-space of the images is established, for each Harris corner point, the HLMO descriptor is calculated by obtaining the local information at the corresponding location of each feature point in the scale-space. The proposed multi-scale HLMO feature extraction method is provided in Algorithm 1, where $O$ is the octave number in the Gaussian pyramid, and $L$ is the layer number. The algorithm outputs the feature point descriptor set $D_{P_{\bf{I}}}(O,L)$, which contains $(2 \cdot {N_{\rm{A}}}+1) \cdot {N_{\rm{O}}}$-dimensional vectors for each feature point at each scale.

The next step is to match the feature point sets of the image pair according to the descriptor sets. The process of the multi-scale feature matching is provided in Algorithm 2. In the process, each scale is matched in turn. Then the matching results are merged step by step while the outlier removal is carried out to realize the optimization of matching points. The final matching results are used to determine the spatial transformation between images. The most critical is to combine all the matching results of feature points and remove outliers, so as to maximize the correct matches of all scales. Fig.\ref{fig:pyramids} shows this process visually. Obviously, this is a general approach to handle all kinds of unknown scale differences. When the scale proportion of images is known or can be estimated, then the above process can be greatly simplified. In this case, the proposed multi-scale strategy still has the advantages of enhancing feature matching and maximizing the number of matching points.

\begin{algorithm}[htp]
\caption{\label{multi2} \footnotesize Proposed MS-HLMO Feature Matching}
\begin{algorithmic}
\footnotesize
\STATE {\bf Input:} feature point set of the image pair $P_{{\bf{I}}1}$, $P_{{\bf{I}}2}$, feature descriptor set of the image pair $D_{P_{{\bf{I}}1}}(O_{1},L_{1})$, $D_{P_{{\bf{I}}2}}(O_{2},L_{2})$
\STATE Take each layer of $D_{P_{{\bf{I}}1}}(O_{1},L_{1})$:
\STATE \hspace*{0.1in}Take each layer of $D_{P_{{\bf{I}}2}}(O_{2},L_{2})$:
\STATE \hspace*{0.2in}Match $P_{{\bf{I}}1}$ and $P_{{\bf{I}}2}$ using Euclidean distance of the descriptorss
\STATE \hspace*{0.2in}Remove outliers, producing the matching result of a single scale $M(O_{1},O_{2},L_{1},L_{2})$
\STATE The matching results of all layers in each octave of the scale-space are union and then optimized with outlier removal, producing the matching result $M_{L}(O_{1},O_{2})$
\STATE The matching results of all octaves in $M_{L}(O_{1},O_{2})$ are union and then optimized with outlier removal, producing the final matching result $M_{OL}(P_{{\bf{I}}1},P_{{\bf{I}}2})$
\STATE {\bf Output:} feature points matching set $M_{OL}(P_{{\bf{I}}1},P_{{\bf{I}}2})$
\end{algorithmic}
\end{algorithm}

\begin{figure}[h!]
 \begin{center}
  \includegraphics[width=3.5in]{MS_Matching.pdf}
  \caption{Multi-scale keypoints matching strategy in MS-HLMO.}
  \label{fig:pyramids}
 \end{center}
\end{figure}

\section{Results}
The accuracy of proposed method is assessed using $100$+ pairs of image frames with diverse resolutions spanning from $50 \times 50$ to $1000 \times 1000$ pixels, and including different categories (e.g., animals, cars, airplane, people, and abstract images). Additionally, the impact of noisy image frames to the accuracy of the proposed method, is assessed using image frames with up to $70$\% of random noise

The evaluations are designed as it follows
\begin{inlinelist}
	\item \textit{Shape A} is a BMP image, or an abstract shape composed of lines, circles, and random noise drawn using features integrated in the implemented tool
	\item \textit{Shape B} is obtained by $\theta$ degrees rotation of \textit{Shape A} plus a random percentage of noise
	\item \textit{Shape A} and \textit{Shape B} are used as inputs for the proposed method
	\item the central tendency of determined rotations is measured as weighted arithmetic mean among top-3 (WM3) rotations, and it is compared with $\theta$. 
\end{inlinelist}
The assessments are performed with default parameters which are: $\omega = 3$, $\lambda=10$, $\epsilon=10$, and the similarity between two segments is calculated excluding neighbor segments. The results of the experiments are discussed as it follows.

\subsection{Top transformations converge rapidly}
The fundamental argument of iterations is to progressively increase the level of details on the image frame abstraction, and accordingly, iteratively improve the accuracy of the calculated approximated transformations, until a user-defined precision criterion is met. Weighted sample variance among top-3 (WV3) approximated transformations provides a measure of dispersion on top approximations. The WV3 reflects the variability in the top-3 approximated transformations, such that: a small WV3 suggests a very reliable WM3, while a large WV3 reflects an uncertainty about the ``best'' linear mapping transformation. According to the experiments, WV3 gets closer to $1$ in a few iterations which yields (a) rapid convergence among top approximated transformations (this confirms the validation of iteration procedure discussed in Section~\ref{section: Validation}), (b) $\textit{WV3} \approx 1$ in few iterations ($>6$) confirms the accuracy of rapidly converged approximated transformations.


\begin{figure*}[!ht]
	\centering
	\includegraphics[width=0.9\textwidth]{Figures/RotationExample.pdf}
	\caption
	{
		\textit{Shape A} is loaded from a BMP image, and \textit{Shape B} is obtained by $270^\circ$ rotation of \textit{Shape A}. The $\Gamma$ matrices of both shapes at different iterations are presented by circular heatmaps. T: determined transformation, S: standard deviation among top-3 determined transformations, D: difference between actual and determined transformations. The normalized similarity index $J(\Gamma_A, \delta \Gamma_B), \forall\delta \in \Delta$ is plotted using a circular heapmap for all the iterations, see panels A2 and B2.  
	}
	\label{Figure: 270}
\end{figure*}


\subsection{Tuning out the cognitive noise}
Selective and visual attention filter irrelevant stimuli to the subject's task by mechanisms such as habituation and cognitive inhibition. There have been promising efforts to model the ability (e.g.,~\cite{tsotsos1995modeling}) since the \textit{spotlight}~\cite{eriksen1972temporal} and \textit{zoom lens}~\cite{eriksen1986visual} models. Additionally, perceived visual information are function of an observer's distance to an object. This aspect has variety of applications namely is Olivia et al.~\cite{oliva2006hybrid} that incorporates this aspect with hybrid images. A hybrid image is composed of two image frames with low and high spatial frequencies, such that either is perceived as noise as a function of observer's distance to the hybrid image frame. In other words, the image of high spatial frequency is dominant at closer distance, while the image with low frequency is perceived at far distance. Whether the noise is a masked image or it is an irrelevant stimuli, it does not impact the perceived information from an image frame. Therefore, the performance of proposed method in approximating linear mapping transformation using noisy image frames, is assessed by experiments where a percentage of \textit{Shape B} is covered with random noise. 

To this extend, an experiment of four tests, $T1$, $T2$, $T3$, and $T4$ is conducted (see Fig.~\ref{Figure: NoiseImpact}). The tests have \textit{Shape A} in common which is a BMP image of a bee. The \textit{Shape B} is created by $234^\circ$ rotation of \textit{Shape A}, and differs among test in the amount of incorporated random noise. The subject in the \textit{Shape A} (i.e., the bee) is represented by $\approx230$K pixels (of $584$K pixels of the image frame). A portion of $120$K pixels (out of the $\approx230K$ pixels) is subject to random noise. This portion is intentionally chosen to cover the body of the bee which presents the majority of perceptible features of the subject. Given that the pixels are binary and the figure is represented by pixels of value 1 (see Section~\ref{section: Shapre Representation}), the random noise is created by setting the value of a random pixel to $1$ in the subject-to-noise portion of \textit{Shape B}. The random noise is added through an iteration of $0$, $5$K, $50$K, and $500$K random pixel selections (a pixel can be selected multiple times) respectively for $T1$, $T2$, $T3$, and $T4$ (see Fig.~\ref{Figure: NoiseImpact}); such that, the majority of perceptible features on \textit{Shape B} are covered with random noise at $T4$. 

The initial segmentation parameter ($\omega = 3$) provides a limited number of variant initial approximations (see Section~\ref{section: Iteration}). Therefore, the WV3 at first iteration (i.e., $l=3$) of the $T1$, $T2$, $T3$, and $T4$ show relatively high dispersion, which indicate the inconsistency of WM3 (see Fig.~\ref{Figure: NoiseImpact}). The initial approximations are tuned at second iteration (i.e., $l=4$) which improve WV3 tenfold (from $118$ to $18$) for the $T1$, $T2$, and $T3$. Despite of a minor discrepancy, WM3 of the tests $T1$, $T2$, and $T3$ are relatively close to actual transformation (i.e., $234^\circ$). However, the considerable noise of $T4$ prevents its WV3 convergence at the same rate as of $T1$, $T2$, and $T3$ (see Fig.~\ref{Figure: NoiseImpact}). The third iteration (i.e., $l=5$) improves approximations, and it brings WV3 of all the test to a same scale, and accordingly provides reliable WM3. Further iterations squeeze the approximations and reach to $\text{WV3}=1.1$ for all tests at sixth iteration (i.e., $l=8$) which indicates a considerable consistency of WM3. Therefore, the method determines WV3 and WM3 for all test at the same scale, given the considerable amount of noise (specially at $T4$). This confirms that even a low amount of perceptible features of the figures is adequate to tune the initial approximations to reliable approximations. For details of the noise impact on other approximations, refer to Supp. Fig.2.17-20.

\begin{figure}[!t]
	\centering
	\includegraphics[width=\columnwidth]{Figures/NoiseImpact.pdf}
	\caption
	{
		Evaluation of random noise impact on transformation determination.s
	}
	\label{Figure: NoiseImpact}
\end{figure}


\subsection{Image resolution defines maximum number of iterations}
When abstracting an image frame, up until a certain iteration, a segment consists of multiple pixels. However beyond that iteration, a segment might be smaller than a pixel (i.e. one pixel belongs to multiple segments). To determine a segment to which a pixel belongs to, the method rounds the position of the pixel. Therefore, beyond a certain iteration, the rounding procedure could potentially increase the distance between the abstractions of two image frames. In such condition, the WV3 converges up-until a certain iteration, and it is saturated beyond that iteration, and accordingly is the WM3 (see Supp. Fig.2.22-23). Therefore, maximum number of iterations, and accordingly the number of \textit{segments} and \textit{sectors} are the function of shape resolution. 


\subsection{Pin-pointed transformation vs. condensed approximations}
The linear mapping transformation between two shapes is determined either as a single transformation with considerable discrepancy with the rest of the approximations (e.g., panel A on Fig.~\ref{Figure: 270}), or a condensed distribution of approximated transformations around actual transformation (e.g., panel B on Fig.~\ref{Figure: 270}). This behavior originates from the discreet representation of image frames (raster graphics); such that, when drawing a \textit{Shape B} from \textit{Shape A}, a pixel of \textit{Shape A} is mapped to a rounded position on \textit{Shape B}. Therefore, pixels of \textit{Shape A} could overlap as mapped on \textit{Shape B}. For instance, the two pixels at $\langle x_1=4, y_1=4 \rangle$ , $\langle x_2=4, y_2=5 \rangle$ belonging to the segment/sector $V_{nm}$ of \textit{Shape A}, with $70^\circ$ rotation, respectively map to positions $\langle x_1^\prime = -0.562, y_1^\prime = 5.628 \rangle$ and $\langle x_2^\prime = -1.33, y_2^\prime = 6.262 \rangle$. As the coordinates are rounded, the two pixels map to position $\langle -1, 6 \rangle$ belonging to the segment/sector $V_{n'm'}$ of \textit{Shape B}. Therefore, two pixels of \textit{Shape A} map to one pixel on \textit{Shape B} (surjective linear transformation). Accordingly, as abstracting the shapes using aggregation function \textit{count} (see Section~\ref{section: Shape Segmentation}), the abstraction parameters are calculated as it follows: $\gamma_{nm} = 2$ and $\gamma_{n'm'} = 1$ (e.g., see comparison of $\gamma$ value distribution plots on Supp. Fig.2.1-22). Hence, comparing $\gamma_{nm}$ and $\gamma_{n'm'}$ results to $j(\gamma_{nm}, \gamma_{n'm'}) = 0.33$ as opposed to expected $j(\gamma_{nm}, \gamma_{n'm'}) = 1$. Such scenarios prevents ``pin-pointing'' the actual transformation (in this case $70^\circ$) and rather provides a condensed distribution of transformations around actual transformation (e.g., see panel B on Fig.~\ref{Figure: 270}).


\subsection{A small similarity is sufficient to determine a reliable linear mapping approximation}
Ideal scenario for comparing two shapes is when there exist a one-to-one correspondence (injective/surjective) between pixels of two the shapes. However, for variety of reasons discussed as it follows, the rotation function on raster graphics is surjective. For instance, rotation function may map multiple pixels of \textit{Shape A} to one pixel of \textit{Shape B}, causing a percentage of deformation on \textit{Shape B} (e.g., see supp. Fig.2.21), and preventing ``pin-pointing'' actual transformation (as above-discussed). Additionally, shapes are possibly subject to noise, which would prevent one-to-one correspondence between the two shapes (non-surjective). Moreover, \textit{Shape A} may consist of congruent figures (e.g., two side-by-side circles of the same radius), and if \textit{Shape B} is determined by $\theta^\circ$ rotation of \textit{Shape A}, then in addition to $\theta^\circ$, multiple rotation angles may also map the congruent shapes on each other. In such cases, actual transformation is determined using incongruent elements (e.g., saddle area, or pedal of the bicycle on Fig.~\ref{Figure: 270}). Such prominent details not only improve approximations for congruent shapes, but are also advantageous when the majority of the shape is covered by noise (e.g., Fig.\ref{Figure: NoiseImpact}) or is deformed (e.g., Supp. Fig.2.21). 

The method discussed in present study, minimizes the impact of such discrepancies on linear mapping transformation determination, by calculating the similarity of two corresponding segments independently from the rest of the segments (an adaptive neighborhood operation of custom range is optionally enabled). Therefore, a higher similarity between few segments is adequate to determine mapping transformation with considerable accuracy. The experiments on deformed, congruent, and noisy image frames illustrate the accuracy of the proposed method on such scenarios.





\section{Threats to Validity}~\label{sec:Threats}
%This section describes the threats to the validity of our study and our attempts to mitigate these threats.

\subsection{Internal Validity}
To mitigate threats to internal validity, an SLR protocol was developed by the first author and reviewed by the other authors before conducting the study. The search string was modified and executed several times on multiple scientific databases to optimize the results. Since the Science Direct database does not allow searching with long strings, we created multiple smaller combinations of our search string and executed those. The studies were filtered in various rounds by the first author and validated by the other authors. The first round of filtering was based on the title and abstract. The second round was based on a brief reading of the paper, and the third round on a detailed reading. These measures ensure minimal selection bias in our study. After selecting the final pool of studies, a data extraction form was created, and all the authors participated in pilot tests for extracting data from these papers. 

\subsection{Construct Validity}
 We attempted to reduce the threat to construct validity by searching seven relevant scientific databases and employing two search strategies (automated and manual). The selected primary studies were highly relevant to MDE for ML and our RQs. After several rounds of discussions, we refined our inclusion and exclusion criteria to ensure that our criteria support selecting the most suitable studies for this SLR. Some of the chosen studies use inconsistent terminology for ML, which is a potential threat to our study. However, all ambiguities were discussed with the second and the third authors to reach a consensus.
 
\subsection{Conclusion Validity}
We aimed to minimize threats to conclusion validity through a well-planned and validated search and data extraction process. A data extraction form was created with questions based on our RQs, ensuring the selected data was relevant to the study. The first author extracted data using a data extraction form for a small subset of studies. All other authors followed the same method and extracted data for the same subset of studies. We compared the data extracted by the first author and other authors and found a close match between them, after which the first author proceeded with data extraction of the remaining studies. To reduce bias during data analysis and synthesis all authors had several rounds of discussion on how to best categorize and represent data.

\subsection{External Validity}
To mitigate threats to external validity, we employed a systematic search process combining automatic search and manual search (snowballing) from the widely accepted guidelines in \cite{kitchenham2009systematic} and \cite{wohlin2014guidelines}. For both searches, we had clearly defined inclusion and exclusion criteria. To ensure the quality of studies considered in our SLR, we only included peer-reviewed academic studies, excluding grey literature, book chapters, opinion-, vision-, and comparison papers. %because this SLR is focused on high-quality research studies on MDE for ML. 
%Authors of research studies usually publish their work in peer-reviewed research venues hence this bias should not have a significant impact on our study. 
We only included studies in the English language since it is the most widely used language for reporting research studies. While we acknowledge that this may have led to the exclusion of some potentially relevant studies, we deem the impact of this bias on our research is minimal. We did not exclude any study based on publication quality to eliminate publication bias in our study. Additionally, our search was not restricted to any time frame to capture all developments in the area of MDE for ML.




% !TEX root = Guillon2017_arxiv.tex

%% Short Recap

Graph analysis of brain networks have been largely exploited in the study of AD with the aim to extract new predictive diagnostics of disease progression.
Typical approaches in functional neuroimaging, characterized by oscillatory dynamics, analyze brain networks separately at different frequencies thus neglecting the available multivariate spectral information.
Here, we adopted a method to formally take into account the topological information of multi-frequency connectomes obtained from source-reconstructed MEG signals in a group of AD and healthy subjects during EC resting states.

%% Multiplex Results

Main results showed that, while flattening networks of different frequency bands attenuates differences between AD and HC populations, keeping the multiplex nature of MEG connectomes allow to capture higher-order discriminant information.
AD subjects exhibited an aberrant multiplex brain network structure that significantly reduced the global propensity to facilitate information propagation across frequency bands as compared to HC subjects (\autoref{fig:participation}b, inset). This could be in part explained by the higher variability of the individual node degrees across bands (\autoref{fig:coefficient_of_variation}).

% NOTE: High MPCi does not necessarily mean high oi (autoref fig:mpca) but it also seems that having a high number of connections (high oi) and a low MPC is not possible in the case of the brain.

% NOTE: In general, a ROI with a high MPC but with low oi, will have an even higher MPC if its oi increase (for another subject for instance). In clear: for a given i (i.e. a given ROI), the corrcoeff between oi and MPCi is always positive.

% NOTE: I tested different thresholds and with the ImCoh to check if it was not because of the week noisy connections, but the distribution of MPCi values seems to be always the same. Even with an average degree of 1 meaning that the brain always tends to keep connections in multiples frequency bands in the same time. Could it be explained by the fact that coherence is influenced by harmonics?

Such loss of inter-frequency centrality was mostly localized in association areas as well as in the cingulate cortex (\autoref{fig:participation}b; \autoref{tab:local_participation}), which resulted the most important hub promoting interaction across bands in the HC group (\autoref{fig:mpc}a).
Because all these areas are typically affected by AD atrophy \citep{wenk_neuropathologic_2003} we hypothesize that the anatomical withering might have impacted the neural oscillatory mechanisms supporting large-scale brain functional integration. Notably, the significant alteration of the connectivity across bands observed in the cingulate cortex could be ascribed to typical M/EEG connectivity changes observed in AD, such as reduced $alpha$ coherence \citep{stam_magnetoencephalographic_2006,jeong_eeg_2004,dauwels_diagnosis_2010,wang_power_2015} (\autoref{fig:mpc}b).
We also found a significant decrease in the primary motor cortex (right precentral gyrus). While previous studies have identified this specific region as a connector hub in human brain networks \citep{tijms_alzheimers_2013}, its role in AD still needs to be clarified in terms of node centrality's changes with respect to healthy conditions.
%For these affected ROIs the decreased centrality was reflected by fewer interactions with higher sensory rhythms ($>20$ Hz) \citep{basar_review_2013} and more connections to lower attentional ones ($<13$ Hz) \citep{klimesch_EEG_1999} (\autoref{fig:participation}c).

% Single-Layer Results
While flattening network layers represents in general an oversimplification, analyzing single layers can still be a valid approach that is worth of investigation.
Because the $MPC$ is a pure multiplex quantity, we considered the conceptually akin version for single-layer networks, the standard participation coefficient $PC$, which evaluates the tendency of nodes to integrate information from different modules, rather than from different layers \citep{guimera_cartography_2005, battiston_structural_2014}.
AD patients exhibited lower inter-modular connectivity in the \textit{gamma} band with respect to HC subjects (\autoref{fig:participation}a; \autoref{tab:local_participation}) that was localized in association areas including frontal, temporal, and parietal cortices (\autoref{fig:participation}a; \autoref{tab:local_participation}).
%
Damages to these regions can lead to deficits in attention, recognition and planning \citep{purves_neuroscience_2001}. Our results support the hypothesis that AD could include a disconnection syndrome  \citep{pearson_anatomical_1985,arnold_topographical_1991,catani_rises_2005}.
Furthermore, they are in line with previous findings showing $PC$ decrements in AD, although those declines were more evident in lower frequency bands and therefore ascribed to possible long-range low-frequency connectivity alteration \citep{de_haan_disrupted_2012,tijms_alzheimers_2013}.

%% Conclusion
Put together, our findings indicated that AD alters the global brain network organization through connection disruption in several association regions, which play important roles in sensory processing by integrating information from other cortical regions through high-frequency channels \citep{miltner_coherence_1999-1,buschman_top-down_2007, siegel_neuronal_2008, gregoriou_high-frequency_2009, hipp_oscillatory_2011}.
%
Notably, we showed that the global loss of inter-modular interactions in the \textit{gamma} band is paralleled by a diffused decrease of inter-frequency centrality.
Future studies, involving recordings of limbic structures and/or stimulation-based techniques, should elucidate whether these two distinct reorganizational processes are truly independent or linked through possible cross-frequency mechanisms which are known to be essential for normal memory formation \citep{canolty_high_2006,axmacher_cross-frequency_2010, goutagny_alterations_2013}.


%% Classification Results

As a confirmation of the complementary information carried out by the multi-layer approach, we reported an increased classification accuracy when combining the local $PC$ and $MPC$ features.
The observed diagnostic power is in line with previous accuracy values obtained with standard graph theoretic approaches (around $80\%$) but exhibits slightly higher sensitivity ($>90\%$), which is often desired to avoid false negatives \citep{li_discriminant_2012, wang_disrupted_2013, wee_enriched_2011, wee_identification_2012, horwitz_functional_2011}.
Other approaches should determine if and to what extent the use of more sophisticated machine learning algorithms, or the inclusion of basic connectivity features \citep{hutchison_network-based_2011, shao_prediction_2012, zhou_hierarchical_2011} and different imaging modalities \citep{dai_discriminative_2012}, can lead to higher classification performance and better diagnosis \citep{tijms_alzheimers_2013}.

%% Correlation With MMSE

Previous works have documented relationships between brain network properties and neuropsychological measurements in AD, suggesting a potential impact for monitoring disease progression and for the development of new therapies
\citep{de_haan_functional_2009,lo_diffusion_2010,sanz-arigita_loss_2010-1,shu_disrupted_2012,stam_small-world_2007,wang_disrupted_2013}.
This is especially true for the standard $PC$ which has exhibited stronger correlations and larger between-group differences \citep{tijms_alzheimers_2013}.
In line with this prediction, we also reported significant correlations between the MMSE cognitive scores and the $PC$ values of the AD patients in the \textit{gamma} band (\autoref{fig:correlations}a).
%
An even stronger correlation was found, however, for the global $MPC$ values and the TR scores (\autoref{fig:correlations}b, \autoref{tab:local_correlation}).
Recent studies suggest that TR scores could be more specific for AD \citep{grober_free_2010, velayudhan_review_2014} as compared to MMSE scores which could be biased by differences in years of education, lack of sensitivity to progressive changes occurring with AD, as well as fail in detecting impairment caused by focal lesions \citep{tombaugh_mini-mental_1992}.
Locally, the regions whose $MPC$ correlated with TR were part of the default-mode network (DMN) (\autoref{tab:local_correlation}), which is heavily involved in memory formation and retrieval \citep{buckner_brains_2008,sperling_functional_2010}. According to recent hypothesis, these areas are directly affected by atrophy and metabolism disruption, as well as amyloid-$\beta$ deposition \citep{buckner_molecular_2005, greicius_default-mode_2004}.
Put together, our results suggest that AD symptoms related to episodic memory losses could be determined by the lower capacity of strategic DMN association areas to let information flow across different frequency channels.

\subsection*{Methodological considerations}

We estimated brain networks by means of spectral coherence, a connectivity measure widely used in the electrophysiological literature because of its simplicity and relatively intuitive interpretation \citep{srinivasan_eeg_2007}.
While this measure is known to suffer from possible volume conduction effects, recent evidence showed that source reconstruction techniques, like the one we adopted here, could at least mitigate this bias \citep{schoffelen_source_2009} and generate connectivity patterns consistent within and between subjects \citep{colclough_how_2016}.
In a separate analysis, we used the imaginary coherence as a candidate alternative to eliminate volume conduction effects \citep{nolte_identifying_2004}. We demonstrated that while no significant between-group differences could be obtained in terms of $MPC$ (data not shown here), the spatial distribution of the $MPC$ values was very similar to that observed in the brain networks obtained with the spectral coherence, especially for the internal regions along the longitudinal fissure (\autoref{fig:mpc_imcoh}).

Differently from other multiplex network quantities, such as those based on paths and walks \citep{boccaletti_structure_2014}, the $MPC$ has the advantage to not depend on the weights of the inter-layer links which, in general, are difficult to estimate or to assign from empirically obtained biological data. This is especially true in network neuroscience where, so far, the strength of the inter-layer connections is parametric and subject to arbitrariness \citep{de_domenico_mapping_2016} or estimated through measures of cross-frequency coupling \citep{brookes_multi-layer_2016-1} whose biological interpretation remains still to be completely elucidated \citep{jirsa_cross-frequency_2013}.


\section{Conclusion}
\label{sec:conclusion}
This paper presents a generic top-$\size$ recommendation framework for  trading-off accuracy, novelty, and coverage. To achieve this, we profile the users according to their preference for long-tail novelty. We examine various measures, and formulate an optimization problem to learn these user preferences from interaction data.  We then integrate the user preference estimates in our generic framework, GANC.  Extensive experiments on several datasets confirm that there are trade-offs between accuracy, coverage, and novelty. Almost all re-ranking models increase coverage and novelty at the cost of accuracy. However, existing re-ranking models typically rely on rating prediction models, and are hence more effective in dense settings. Using a generic approach, we can easily incorporate a suitable base accuracy recommender to devise an effective solution for both sparse and dense settings.  %Our results  also indicate there is no single method that outperforms other methods in all metrics. However our techniques obtain a significant improvement in coverage, while  . 
Although we integrated the  long-tail novelty preference estimates into a re-ranking framework, their use-case is not limited to these frameworks. In  the future, we intend to explore the temporal and topical dynamics of long-tail novelty preference, particularly in settings where contextual information is  available.  
%We achieve these objectives without using any additional contextual information.


\iffalse
While we focused on promoting long-tail items across users, we did not consider diversity of individual top-$\size$ recommendations, a factor that has been shown to positively affect consumer satisfaction. This is one direction for future work. Moreover, the sequential setting  in our work, creates a dependency between different batches, where,  the items recommended to a batch of users, depends on those recommended to previous batches. This dependency is created through the parameter $\mathbf{f}$, that is updated every time a top-$\size$ set  is allocated to a batch of users. A future direction for our work is to estimate a distribution over $\mathbf{f}$ that allows us to independently solve the problem for each user, leading to improvements across all performance metrics, including recommendation time. 

We design algorithms that take advantage of the structure in the value functions to obtain both efficient and scalable solutions. 
We design an algorithm that takes advantage of the structure in the value functions to obtain both efficient and scalable solutions. 

\textcolor{red}{Our  sequential  algorithms can be applied for batch recommendation contexts,~e.g., personalized email marketing, where based on prior interaction data between users and items,  a new round of recommendations must be sent to all users in the system.  However, the independent coverage algorithms lift the sequential setting restrictions and allow it be applied for re-ranking the output of base recommender in any setting. }A future direction for our work is to incorporate explicit diversity metrics in the framework. 
\fi


%We have a presented a submodular maximization framework to systematically trade-off relevance and diversity in recommendations to individual users and coverage across the item-space. This ensures both consumer and producer satisfaction. We model users according to their risk and focusing degrees and promote long-tail items to the right group of consumers. Consequently, we obtain a significant improvement in coverage while maintaining reasonable levels of user satisfaction. Furthermore, our methods are able to achieve a more balanced distribution across the set of recommended items. In the future, we plan to investigate the effect of using alternative base recommender systems. 

%Future Work
%However most of these methods assume that the ratings are missing at random (MAR). Since our method of generating recommendations is based on the completed matrix, assuming MAR might introduce additional bias, we will use methods which assume that the ratings at missing not at random (MNAR),explored in~\cite{steck2010training, icml2014c2_hernandez-lobatob14}. 	 
%Long Tail %Recently, authors in~\cite{cremonesi2010performance} conducted extensive experiments to evaluate the performances of various matrix factorization-based algorithms and neighborhood models on the task of recommending long tail items. Their experimental results show that long tail recommendation leads to a decrease in accuracy for all algorithms. They also showed that for this task, SVD outperforms other state-of-the-art algorithms. 


\newpage
\section{Dataset Visualizations}
\label{sec:app_dataset_visuals}

%%%%%%
%%
%%
\subsection{Examples of each view class}
\newcommand{\BC}{0.33}
\setlength{\tabcolsep}{0.1cm}
\begin{figure}[!h]
\begin{tabular}{c c c c}
    PLAX  & PSAX & OTHER 
    \\
    \includegraphics[width=\BC\textwidth]{figures/small_appendix/Appendix_PLAX1.jpg}
    &
    \includegraphics[width=\BC\textwidth]{figures/small_appendix/Appendix_PSAX1.jpg}
    &
    \includegraphics[width=\BC\textwidth]{figures/small_appendix/Appendix_Other1.jpg}
    &
   
    \\
    
    \includegraphics[width=\BC\textwidth]{figures/small_appendix/Appendix_PLAX2.jpg}
    &
    \includegraphics[width=\BC\textwidth]{figures/small_appendix/Appendix_PSAX2.jpg}
    &
    \includegraphics[width=\BC\textwidth]{figures/small_appendix/Appendix_Other2.jpg}
    &
   
     \\
     
     \includegraphics[width=\BC\textwidth]{figures/small_appendix/Appendix_PLAX3.jpg}
    &
    \includegraphics[width=\BC\textwidth]{figures/small_appendix/Appendix_PSAX3.jpg}
    &
    \includegraphics[width=\BC\textwidth]{figures/small_appendix/Appendix_Other3.jpg}
    &
   
     \\
     
     \includegraphics[width=\BC\textwidth]{figures/small_appendix/Appendix_PLAX4.jpg}
    &
    \includegraphics[width=\BC\textwidth]{figures/small_appendix/Appendix_PSAX4.jpg}
    &
    \includegraphics[width=\BC\textwidth]{figures/small_appendix/Appendix_Other4.jpg}
    &
   
    \end{tabular}	
    \caption{Examples of images for each possible view label in our dataset. \emph{From left to right:} Four examples of peristernal long axis (PLAX) view, four examples of peristernal short axis (PSAX) view, and four examples of other kinds of view in our ``Other'' class. }
    \label{fig:VIEW_SAMPLES_APPENDIX}
\end{figure}

%%%%%%
%%
%%
\newpage
\subsection{Examples of each view for a Severe AS patient}
\newcommand{\BA}{0.33}
\setlength{\tabcolsep}{0.1cm}
\begin{figure}[!h]
\begin{tabular}{c c c c}
    PLAX  & PSAX & OTHER 
    \\
    \includegraphics[width=\BA\textwidth]{figures/small_appendix/SevereAS_11112007_PLAX1.jpg}
    &
    \includegraphics[width=\BA\textwidth]{figures/small_appendix/SevereAS_11112007_PSAX1.jpg}
    &
    \includegraphics[width=\BA\textwidth]{figures/small_appendix/SevereAS_11112007_Other1.jpg}
    &
    
    \\
    
    \includegraphics[width=\BA\textwidth]{figures/small_appendix/SevereAS_11112007_PLAX2.jpg}
    &
    \includegraphics[width=\BA\textwidth]{figures/small_appendix/SevereAS_11112007_PSAX2.jpg}
    &
    \includegraphics[width=\BA\textwidth]{figures/small_appendix/SevereAS_11112007_Other2.jpg}
    &
   
     \\
     
     \includegraphics[width=\BA\textwidth]{figures/small_appendix/SevereAS_11112007_PLAX3.jpg}
    &
    \includegraphics[width=\BA\textwidth]{figures/small_appendix/SevereAS_11112007_PSAX3.jpg}
    &
    \includegraphics[width=\BA\textwidth]{figures/small_appendix/SevereAS_11112007_Other3.jpg}
    &
  
    \end{tabular}	
    \caption{Examples of images from a patient with Severe AS in our dataset. \emph{From left to right:} Three examples of parasternal long axis (PLAX) view, three examples of parasternal short axis (PSAX) view, and three examples of other kinds of view in our ``Other'' class. }
    \label{fig:PatientSevereAS}
\end{figure}


%%%%%%
%%
%%
\newpage
\subsection{Examples of each view for a No AS patient}
\newcommand{\BB}{0.33}
\setlength{\tabcolsep}{0.1cm}
\begin{figure}[!h]
\begin{tabular}{c c c c}
    PLAX  & PSAX & OTHER 
    \\
    \includegraphics[width=\BB\textwidth]{figures/small_appendix/NoAS_1996889_PLAX1.jpg}
    &
    \includegraphics[width=\BB\textwidth]{figures/small_appendix/NoAS_1996889_PSAX1.jpg}
    &
    \includegraphics[width=\BB\textwidth]{figures/small_appendix/NoAS_1996889_Other1.jpg}
    &
    
    \\
    
    \includegraphics[width=\BB\textwidth]{figures/small_appendix/NoAS_1996889_PLAX2.jpg}
    &
    \includegraphics[width=\BB\textwidth]{figures/small_appendix/NoAS_1996889_PSAX2.jpg}
    &
    \includegraphics[width=\BB\textwidth]{figures/small_appendix/NoAS_1996889_Other2.jpg}
    &
   
     \\
     
     \includegraphics[width=\BB\textwidth]{figures/small_appendix/NoAS_1996889_PLAX3.jpg}
    &
    \includegraphics[width=\BB\textwidth]{figures/small_appendix/NoAS_1996889_PSAX3.jpg}
    &
    \includegraphics[width=\BB\textwidth]{figures/small_appendix/NoAS_1996889_Other3.jpg}
    &
  
    \end{tabular}	
    \caption{Examples of images from a patient with No AS in our dataset. \emph{From left to right:} Three examples of parasternal long axis (PLAX) view, three examples of parasternal short axis (PSAX) view, and three examples of other kinds of view in our ``Other'' class. }
    \label{fig:PatientNoAS}
\end{figure}



\newpage 
\section{Further Results}

\subsection{Assessment of ensembling}

Table~\ref{tab:best_single_checkpoint_VS_ensemble_FS_echo260} compares using a single checkpoint (one point estimate of neural network weight vector $\theta$) to using an ensemble of parameters aggregated from the last 25 checkpoints (one per epoch).

\begin{table}[!h]
    \centering
    \begin{tabular}{c|cccc|c}
    \textit{Diagnosis classification} & Split 1  & Split 2 & Split 3 & Split 4 & Average\\
    \hline
    Best single checkpoint  & 61.81 & 59.79 & 56.05 & 64.21 & 60.46\\
    Ensemble  & 62.95 & 61.03 & 56.58 & 63.84 & \textbf{61.13}
	\\ \hline
    \textit{View classification}  &   &  &  &  & 
    \\ \hline
    Best single checkpoint  & 93.03 & 93.24 & 92.39 & 93.79 & 93.11\\
    Ensemble  & 92.37 & 93.24 & 93.72 & 93.87 & \textbf{93.30}\\
    \end{tabular}
    \caption{Comparing best single checkpoint performance with ensemble performance on \textbf{Full-size \datasetName-156-52}}
    \label{tab:best_single_checkpoint_VS_ensemble_FS_echo260}
\end{table}


%%%%%%
%%
%%
\subsection{Patient-level diagnosis performance on bonus heldout set}

Table~\ref{tab:diagnosis classification patient unlabeled_heldout_174} examines the performance of the best labeled-set-only methods and MixMatch methods on the 174 patient studies that have diagnosis but no view labels.
 While the images used here were originally included in the unlabeled training set (which was used to train SSL methods like MixMatch), the diagnosis labels were not provided at all during training time. 
 We thus still believe this is an authentic test of generalization given the scarcity of labeled data available for our task.
 Of course, additional independent evaluation (especially from another institution) is needed.

\begin{table}[!h]
    \centering
    \begin{tabular}{l l l|rrrr|c}
    Pretrain & Method & Voting
    & Split 1  & Split 2 & Split 3 & Split 4 & average\\
    \hline
    & Basic WRN & Simple average & 76.73 & 75.25 & 76.87 & 81.88 & 77.68\\
    & Basic WRN & View-prioritized & 73.63 & 83.21 & 79.70 & 80.08 & 79.18\\
    %SSL & FS & MixMatch & Priority view + confidence & 94.58 & 84.17 & 77.50 & 92.5 & 87.19\\
    \hline
    & MixMatch & Simple average & 85.32 & 76.29 & 74.14 & 79.95 & 78.93\\
    view & MixMatch & Simple average & 83.36 & 77.96 & 75.61 & 81.37 & 79.58\\
    & MixMatch & View-prioritized & 83.27 & 83.76 & 82.34 & 82.83 & \textbf{83.05}\\
    view & MixMatch & View-prioritized & 82.53 & 86.15 & 79.62 & 83.27 & 82.89\\
    %view & MixMatch & LR with view-priority & 80.42 & 84.24 & 76.58 & 80.67 & 80.48\\
    %(MixMatch transfered) + MysteryMethod & NA & NA & NA\\ 
    \end{tabular}
    \caption{Patient-level AS Severity Diagnosis Classification on the \textbf{bonus heldout set} of 174 patients for whom we have diagnosis labels only (no view labels). We show balanced accuracy on models trained on each of the four folds on four \textbf{full-size \datasetName-156-52} dataset.
    }%endcaption
    \label{tab:diagnosis classification patient unlabeled_heldout_174}
\end{table}


%%%%%%
%%
%%
\subsection{Assessment of MixMatch hyperparameter sensitivity}

In Table~\ref{tab:MixMatch hyperparameters ablation study}, we consider four possible strategies for setting the hyperparameters of MixMatch, varying two  key settings for the weight on unlabeled loss $\lambda$. First, we vary whether the final value of $\lambda$ is set to its \emph{best} value among a grid of candidates (based on validation set performance), or \emph{fixed} to a constant.
Second, we vary whether $\lambda$ remains fixed over iterations throughout a training run, or is updated over iterations on a linear ramp schedule from 0 to its final target value. 

From this comparison, we see we consistent gains across splits (average gain across splits of over 1.6\% balanced accuracy) for using a delayed ramp up schedule with target value selected via grid search.

\begin{table}[!h]
    \centering
    \begin{tabular}{l l| rrrr | r}
    Final $\lambda$ value & $\lambda$ update schedule & Split 1  & Split 2 & Split 3 & Split 4 & Average\\
    \hline
    best on val & Delayed ramp-up  & 65.57 & 62.69 & 60.87 & 66.29 & 63.86\\
    best on val & Immediate ramp-up & 65.07 & 61.87 & 60.82 & 65.37 & 63.28\\
    best on val & Constant  & 65.03 & 61.52 & 58.87 & 65.22 & 62.66\\
    100 (fixed) & Constant & 63.94 & 61.79 & 58.87 & 64.35 & 62.24\\
    \end{tabular}
    \caption{Ablation study of different settings of the unlabeled loss weight $\lambda$ for MixMatch. AS severity diagnosis classification for individual images on the \textbf{full-size \datasetName-156-52} dataset. showing balanced accuracy averaged over the test sets from multiple folds (each fold’s test set contains all images from 52 patients). }%endcaption
    \label{tab:MixMatch hyperparameters ablation study}
\end{table}



%%%%%%
%%
%%
\subsection{Assessment of alternative view prioritization strategy using thresholding}


An anonymous reviewer suggested an alternative strategy for prioritizing images of relevant view.
The alternative strategy works as follows: for each image, we compute the predicted probability that the image is a ``relevant view'' (either PLAX and PSAX) by summing the probabilities of each view type.
However, instead of using this raw probability as a weight (as our chosen method does), we use a \emph{cutoff threshold} and simply average the diagnosis predictions of images whose relevant view probability is above the cutoff.
For each patient, we use the majority vote prediction of the diagnosis from the images of relevant views.
The value of the cutoff threshold is selected using the validation set to maximize balanced accuracy.

Table~\ref{tab:Suggested_Aggregation_Ablation} shows the performance of this strategy (``threshold-then-average'') on the full-size dataset.
Using this alternative prioritization strategy together with our suggested methodology for patient-level diagnosis (using MixMatch, pretraining on view), we find the average test set balanced accuracy is around 85.8\%, while the weighted average strategy in the main paper achieves over 90\% balanced accuracy. We take this as reasonably decisive evidence that a weighted average (rather than a simple cutoff) should be preferred.

\begin{table}[!h]
    \centering
    \begin{tabular}{l l l|rrrr|c}
    Pretrain & Method & Aggregation across images
    & Split 1  & Split 2 & Split 3 & Split 4 & average\\
    \hline
    & Basic WRN & Threshold-then-Average & 85.42 & 86.25 & 79.17 & 92.50 & 85.84 \\
    %SSL & FS & MixMatch & Priority view + confidence & 94.58 & 84.17 & 77.50 & 92.5 & 87.19\\
    & MixMatch & Threshold-then-Average & 83.33 & 84.17 & 77.50 & 94.58 & 84.90 \\
    view & MixMatch & Threshold-then-Averagen & 86.67 & 80.00 & 82.50 & 94.17 & 85.84\\
    %view & MixMatch & LR with view-priority & 87.08 & 82.08 & 85.00 & 88.75 & 85.73\\
    %(MixMatch transfered) + MysteryMethod & NA & NA & NA\\ 
    \end{tabular}
    \caption{Alternative view-prioritizing strategy for patient-level AS severity diagnosis classification on the \textbf{full-size \datasetName-156-52} dataset, showing balanced accuracy on the test set across multiple folds (each fold’s test set contains 52 patients).}
    %endcaption
    \label{tab:Suggested_Aggregation_Ablation}
\end{table}



%%%%%%
%%
%%
\subsection{ROC Curve of patient-level diagnosis: no AS vs. mild/moderate/severe AS}

Fig.~\ref{fig: No AS vs Some AS} shows receiver operating curves for several methods for the task of distinguishing no AS vs Some AS (which aggregates both the mild/moderate and severe levels in the 3-level diagnosis task of the main paper).

\begin{figure}[!h]
\begin{tabular}{c c}
	\includegraphics[width=0.43\textwidth]{figures/fold0_multitask_PatientLevel_NoVSSome_NormalizedPriorityStrategyClassProbabilityScore.pdf}
	&
    \includegraphics[width=0.43\textwidth]{figures/fold1_multitask_PatientLevel_NoVSSome_NormalizedPriorityStrategyClassProbabilityScore.pdf}
	\\
	(a) Split 1 & (b) Split 2
	\\
	\includegraphics[width=0.43\textwidth]{figures/fold2_multitask_PatientLevel_NoVSSome_NormalizedPriorityStrategyClassProbabilityScore.pdf}
	&
    \includegraphics[width=0.43\textwidth]{figures/fold3_multitask_PatientLevel_NoVSSome_NormalizedPriorityStrategyClassProbabilityScore.pdf}
	\\
	(c) Split 3 & (d) Split 4
\end{tabular}
    
\caption{ROC curves for binary diagnosis task (no AS vs ``mild/moderate/severe AS'') on \textbf{full-size \datasetName-156-52}.
    }%endcaption
    \label{fig: No AS vs Some AS}
\end{figure}

\section{Methodological Details}

\subsection{Image processing details}
\label{sec:removing_doppler}

\paragraph{Removing doppler images.}
In the raw data of all imagery available for an echocardiogram study, 
we obtained TIFF files that represent both cineloops and Doppler images.

We verified in our labeled set that all Doppler images have one of the following landscape aspect ratio $(831, 323)$, $(901, 384)$, $(901, 390)$, $(704, 305)$, $(831, 421)$, $(901, 469)$ or $(563, 294)$. Only the Dopplers have these aspect ratios. We thus filtered out Doppler completely via these aspect ratios. 

\paragraph{Downsizing}
The original images are provided as high-resolution TIFF format images (hundreds of pixels per side) of varying aspect ratios. Generally, we can expect that both view and diagnosis classifiers would perform better given higher-resolution input (and holding other factors the same). The main trade-off of processing higher-resolution images is increased runtime and memory requirements. In our preliminary experiments, we compared downsizing all images to a standard square aspect ratio at 3 possible sizes: 32x32, 64x64 and 128x128. We found that 64x64 achieves a good balance between model performance and computation cost. 
A prior study by \citet{madaniDeepEchocardiographyDataefficient2018} provides a more extensive study of optimal resolution size. The interested reader can refer to their work for more details. 


\subsection{Architecture Settings and Hyperparameters}
\label{sec:arch_and_hyperparameters}

\paragraph{Weighted cross-entropy for labeled loss}
To counteract the effect of class imbalance in the dataset, we use weighted cross-entropy for the labeled loss. For an input image $x$ whose true label $y$ indicates it belongs to class $c$, the weighted cross-entropy assumes the following form:
\begin{align}
\mathcal{L}^L(\theta, x) = - w_{c} \log \hat{p}_{c}(\theta, x),
\end{align}
where $\hat{p}_{c}$ is the predicted probability of class $c$. The weight $w_{c}$ is calculated using the training set statistics as follow:
\begin{align}
w_{c} = \frac{\prod_{k\neq c}{N_{k}}}{\sum_{j}\prod_{k \neq j}{N_{k}}}
\end{align}
where $N_{k}$ is the number of images of class $k$ in the training set.

\paragraph{Common architecture.}
Following~\citet{oliverRealisticEvaluationDeep2018}, for all considered methods, we use the \emph{same} backbone neural network architecture: a wide residual network~\citep{zagoruykoWideResidualNetworks2017} with 28 layers (WRN-28), which has total of 5,931,683 parameters.
This same network architecture is used in the original MixMatch evaluation~\citep{berthelotMixmatchHolisticApproach2019} with promising results.

\paragraph{Common training protocol.}
All SSL methods we consider follow the loss minimization framework with two primary losses (one for ``labeled'' data and one for ``unlabeled'' data) in Eq.~\eqref{eq:standard-SSL-loss-template}.
We allow every method to train for 32 epochs (where each epoch processes $2^{16}$ images, as in \citet{berthelotMixmatchHolisticApproach2019}).
Our preliminary experiments suggest that after 30 epochs all methods effectively converge in terms of validation balanced accuracy. 

\paragraph{Common regularization.}
For all methods, we expect performance will be vulnerable to overfitting, so we impose an L2-norm penalty on the weights $\theta$, also known as weight decay. Each method selects its preferred value of this penalty strength hyperparameter. We searched values in [0.0002, 0.002, 0.02].

\paragraph{Common optimization.}
We use ADAM \citep{kingma2014adam} to optimize each model.
Each method selects the value of the step size (learning rate) as a hyperparameter. We experimented with 0.002 and 0.0007
%HZ: 'performance being sensitive to learning rate' is very reasonable. But we don't have an ablation to back it. 
%We find performance is sensitive to the step size (learning rate) hyperparameter, so we perform a grid search and select the value that maximizes balanced accuracy on the validation set.

\paragraph{Hyperparameters for Pseudo-Label.}
Beyond the usual hyperparameters for our loss-minimization SSL framework, another important hyperparameter for pseudo-label is the threshold $\tau$. We find that performance is not very sensitive to the chosen $\tau$ value as long as it is within a certain range. We set $\tau$ to 0.95, as done in past literature that evaluates Pseudo-Label as an SSL method ~\citep{oliverRealisticEvaluationDeep2018,berthelotMixmatchHolisticApproach2019, berthelotRemixmatchSemisupervisedLearning2019, sohnFixmatchSimplifyingSemisupervised2020}.


\paragraph{Hyperparameters for VAT.}
Beyond the usual hyperparameters for our SSL framework, for VAT we need to select a value for $\epsilon$.
In \citet{miyatoVirtualAdversarialTraining2019}, the authors claimed that they can achieve superior performance by tuning only $\epsilon$ and fixing $\lambda$ to 1. In our experiment, we used the default $\lambda$ as in \cite{berthelotMixmatchHolisticApproach2019} and searched the value of $\epsilon$ in [2, 6, 18], together with learning rate and weight decay. We select the best hyperparameters using validation set performance. 


\paragraph{Hyperparameters for MixMatch.}
Beyond the usual hyperparameters for our SSL framework, the key hyperparameters for MixMatch include the number of augmentations $K$, the temperature $T>0$ used for sharpening, interpolation hyperparameter $\alpha$ and unlabeled loss coefficient $\lambda$. We set $K=2$, $T=0.5$, and $\alpha=0.75$ as done in \citet{berthelotMixmatchHolisticApproach2019}, and search for $\lambda$ in the range [10, 30, 75, 100, 130] using validation set. 

\paragraph{Hyperparameters for Multitask training.}
We searched $\gamma$, the hyperparameter that control the strength of the auxilliary view loss in Eq.~\eqref{eq:multitask}, in the range [10, 3, 1, 0.3, 0.1]. The best $\alpha$ is selected together with other hyperparameters on validation set. 


% Appendix sections are coded under \verb+\appendix+.

% \verb+\printcredits+ command is used after appendix sections to list 
% author credit taxonomy contribution roles tagged using \verb+\credit+ 

\printcredits

\bibliographystyle{elsarticle-num}

\bibliography{refs}



\end{document}

