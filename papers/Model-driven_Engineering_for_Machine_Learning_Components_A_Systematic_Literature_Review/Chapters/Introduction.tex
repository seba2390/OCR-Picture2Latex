\section{Introduction}\label{sec:introduction}

The ability of Machine Learning (ML) to autonomously learn data patterns and predict outcomes has tremendous potential to solve complex problems~\cite{zhang2003machine}. The proliferation of data and advances in hardware processing capabilities have contributed to the rapid growth and adoption of ML in recent years. ML components have now becoming integral to software systems with application domains including healthcare~\cite{ghassemi2020review,beam2018big}, finance~\cite{goodell2021artificial,dixon2020machine}, transport~\cite{zantalis2019review}, entertainment~\cite{galway2008machine, bennett2007netflix}, and many more. However, the development, integration, and maintenance processes of traditional software components and ML components  differ significantly~\cite{ahmad2023requirements}. Traditional software components are deterministic and developed through a {deductive} development process by explicitly coding the system's required behaviour. In contrast, ML component development follows an {inductive} process by exploring data and recognizing patterns~\cite{khomh2018software}. ML components are dynamic and it is extremely difficult to specify their `well-defined' behavior~\cite{ahmad2023requirements,arora2023advancing}. Unlike software engineers, ML engineers need to perform exploratory steps to identify and curate datasets, select relevant features, select the most suitable ML model, tune hyperparameters, monitor the ML component, and re-train in case of performance degradation~\cite{lwakatare2019taxonomy}. Hence, it is challenging to build and integrate ML components into software systems ~\cite{khomh2018software}. 

The Model-driven Engineering (MDE) paradigm offers a potential solution to reduce the aforementioned complexities through abstraction~\cite{hutchinson2011model,moin2022model}. MDE advocates using software models at different abstraction levels to (semi-)automatically build software systems~\cite{brambilla2017model, moin2022model}. It has been effectively applied over several years in aerospace, automotive, telecommunication, business information systems, and mobile apps~\cite{mussbacher2014relevance, hutchinson2011model, hutchinson2011empirical, shamsujjoha2021developing}. Similar MDE techniques apply to software systems with ML components provided that existing modeling techniques are customized for ML-specific information, e.g., software architecture models describe classical software components and ML components, and the generated artifacts are tailored to ML, e.g., ML models or training code~\cite{moin2022model}. MDE has the potential to significantly enhance the development of ML-based systems~\cite{bucchiarone2020grand} by hiding complexities, increasing productivity, and improving system quality~\cite{moin2022model, mohagheghi2008proof}. MDE approaches, with their capability to model ML-based systems at a high level of abstraction, facilitate ML novices and experts~\cite{bucchiarone2020grand} in the development, integration, and maintenance of the systems. The effort and time required to develop and maintain an ML-based system may also be reduced through automated artifact generation~\cite{volter2013model}. 
%
Figure \ref{fig:MDE4ML} models an example ML component by specifying the ML algorithm (Random Forest), training parameters, and the training and test datasets. By applying MDE, this model is automatically transformed into code and documentation for the ML component with lower technical barriers and higher efficiency. Additional benefits of MDE include easier system management~\cite{bhattacharjee2019stratum}, early detection of bugs~\cite{mohagheghi2008proof}, lower costs~\cite{fleurey2007model, mohagheghi2013empirical}, and improved understanding and collaboration among diverse stakeholders~\cite{khalajzadeh2020end}.

\begin{figure*}[htbp]
    \centering
    \includegraphics[width=1\textwidth]{Images/MDE4ML.png}
    \caption{Model-driven Engineering for Machine Learning}
    \label{fig:MDE4ML}
\end{figure*}

To explore this promising and relatively new research area, we conducted a systematic literature review (SLR) focusing on MDE approaches used to develop systems with ML components (that we term `MDE4ML'). Through this study, we aim to collate, summarize, and report interesting findings in the literature on MDE4ML. We identify the goals of existing studies, key MDE approaches used, and the modeling languages, frameworks, and model transformation tools applied to develop ML-based systems. We also analyze the ML aspects addressed in the studies, evaluation methods, existing limitations, and future opportunities. Our analysis reveals that most MDE solutions for ML lack maturity and good tooling, often ignore data pre-processing steps, responsible ML development practices and scalability considerations, and have limited emphasis on ML aspects other than design, development, and training. Our findings can help future researchers efficiently identify current research trends and limitations in studies on MDE for ML components and guide future research. We followed the well-known and widely accepted SLR guidelines by Kitchenham et al.~\cite{kitchenham2009systematic, kitchenham2007guidelines}. The main contributions of this SLR are as follows:
\begin{itemize}
    \item Identification, analysis, data extraction, and synthesis of 46 primary studies highly relevant to MDE4ML;
    \item Insights into current trends in MDE for ML components, e.g., most MDE approaches focus on supervised learning;.
    \item Key limitations in existing studies on MDE for ML components, e.g., scalability -- one of the most important concerns in ML development -- is seldom considered in MDE4ML; and
    \item Key future research directions and recommendations for further studies on MDE for ML.
\end{itemize}

The rest of the paper is organised as follows: Section~\ref{sec:background} provides an overview of the background and related work on MDE4ML. Section~\ref{sec:Method} presents details of our research methodology. Section~\ref{sec:Results}
reports our findings from the selected primary studies. Section~\ref{sec:Threats} addresses the threats to validity. Section~\ref{sec:Discussion}  discusses the interesting results of our SLR and recommendations for future research, and Section~\ref{sec:Conclusion}  concludes the paper.


