


\section{Conclusion}~\label{sec:Conclusion}
Software engineering for ML-based systems is in many ways more complicated and challenging for developers compared to traditional software systems~\cite{atouani2021artifact}. In this article, we report on an SLR of MDE solutions for software systems with ML components. The goal was to explore the potential of MDE for ML-based systems and identify trends, benefits, and limitations of existing work. For the SLR, we followed the systematic process described by Kitchenham et al.~\cite{kitchenham2009systematic} and selected 46 highly relevant primary studies from an initial pool of 3,496 papers. We have explored many interesting aspects of the MDE solutions and reported our findings along with gaps and potential directions for future research. Our key findings suggest that over the last five years, there has been a significant increase in the studies on MDE4ML; with the rapidly growing popularity of ML and AI, we expect this trend to continue in the future. 

Our examination of selected studies suggests that: 1) there are few studies on the data required for ML; 2) proposed solutions are limited to the design, development, and training of ML components (less studies on requirements engineering, integration, ML pipelines, automated deployment, monitoring and documentation); 3) there are limited studies for unsupervised learning and reinforcement learning; 4) MDE steps are not comprehensively explained; 5) MDE solutions for ML-based systems require more maturity and better tool support; 6) there is a lack of MDE4ML solutions for domain experts; 7) studies use inconsistent terminology for describing ML; 8) there is a need for more focus on solution scalability; 9) there is a lack of responsible ML and human-centric development practices in MDE approaches for ML-based systems; 10) most evaluations lack rigor and are conducted in an academic setting.

\section*{Acknowledgements}

Naveed is supported by a Faculty of IT PhD scholarship. Grundy is supported by ARC Laureate Fellowship FL190100035. This work has been partially supported by ARC Discovery Project DP200100020.