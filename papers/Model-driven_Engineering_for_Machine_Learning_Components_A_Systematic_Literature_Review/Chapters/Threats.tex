\section{Threats to Validity}~\label{sec:Threats}
%This section describes the threats to the validity of our study and our attempts to mitigate these threats.

\subsection{Internal Validity}
To mitigate threats to internal validity, an SLR protocol was developed by the first author and reviewed by the other authors before conducting the study. The search string was modified and executed several times on multiple scientific databases to optimize the results. Since the Science Direct database does not allow searching with long strings, we created multiple smaller combinations of our search string and executed those. The studies were filtered in various rounds by the first author and validated by the other authors. The first round of filtering was based on the title and abstract. The second round was based on a brief reading of the paper, and the third round on a detailed reading. These measures ensure minimal selection bias in our study. After selecting the final pool of studies, a data extraction form was created, and all the authors participated in pilot tests for extracting data from these papers. 

\subsection{Construct Validity}
 We attempted to reduce the threat to construct validity by searching seven relevant scientific databases and employing two search strategies (automated and manual). The selected primary studies were highly relevant to MDE for ML and our RQs. After several rounds of discussions, we refined our inclusion and exclusion criteria to ensure that our criteria support selecting the most suitable studies for this SLR. Some of the chosen studies use inconsistent terminology for ML, which is a potential threat to our study. However, all ambiguities were discussed with the second and the third authors to reach a consensus.
 
\subsection{Conclusion Validity}
We aimed to minimize threats to conclusion validity through a well-planned and validated search and data extraction process. A data extraction form was created with questions based on our RQs, ensuring the selected data was relevant to the study. The first author extracted data using a data extraction form for a small subset of studies. All other authors followed the same method and extracted data for the same subset of studies. We compared the data extracted by the first author and other authors and found a close match between them, after which the first author proceeded with data extraction of the remaining studies. To reduce bias during data analysis and synthesis all authors had several rounds of discussion on how to best categorize and represent data.

\subsection{External Validity}
To mitigate threats to external validity, we employed a systematic search process combining automatic search and manual search (snowballing) from the widely accepted guidelines in \cite{kitchenham2009systematic} and \cite{wohlin2014guidelines}. For both searches, we had clearly defined inclusion and exclusion criteria. To ensure the quality of studies considered in our SLR, we only included peer-reviewed academic studies, excluding grey literature, book chapters, opinion-, vision-, and comparison papers. %because this SLR is focused on high-quality research studies on MDE for ML. 
%Authors of research studies usually publish their work in peer-reviewed research venues hence this bias should not have a significant impact on our study. 
We only included studies in the English language since it is the most widely used language for reporting research studies. While we acknowledge that this may have led to the exclusion of some potentially relevant studies, we deem the impact of this bias on our research is minimal. We did not exclude any study based on publication quality to eliminate publication bias in our study. Additionally, our search was not restricted to any time frame to capture all developments in the area of MDE for ML.


