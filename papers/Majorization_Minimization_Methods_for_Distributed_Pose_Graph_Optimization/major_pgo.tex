In this section, following a similar procedure to our previous works \cite{fan2019proximal,fan2020mm}, we  present surrogate functions $G(X|\Xk)$ and $\lG(X|\Xk)$ that majorize the objective function $F(X)$. The surrogate functions $G(X|\Xk)$ and $\lG(X|\Xk)$ decouple unknown poses of different nodes, and thus, are critical to our MM methods for distributed PGO.

%Instead of using \cref{eq::proxF}, we need to seek for other surrogate functions that not only majorize $F(X)$ but are convenient for distributed PGO as well. 

\subsection{The Majorization of $F_{ij}^{\ab}(X)$}
For any matrices $B,\,C$ and $P\in \R^{m\times n}$, it can be shown that
\begin{equation}\label{eq::inequality}
\frac{1}{2}\|B-C\|_{\nM_{ij}^{\ab}}^2 \leq \|B-P\|_{\nM_{ij}^{\ab}}^2 + \|C-P\|_{\nM_{ij}^{\ab}}^2
\end{equation}
as long as $\nM_{ij}^{\ab}\in\R^{n\times n}$ is positive semidefinite, where ``$=$'' holds if
$$P=\frac{1}{2}B+\frac{1}{2}C.$$
If we let $P=\0$, \cref{eq::inequality} becomes
\begin{equation}\label{eq::inequality2}
\frac{1}{2}\|B-C\|_{\nM_{ij}^{\ab}}^2 \leq \|B\|_{\nM_{ij}^{\ab}}^2 + \|C\|_{\nM_{ij}^{\ab}}^2,
\end{equation}
{\highlight which holds for any $B$ and $C\in \R^{m\times n}$.} Applying \cref{eq::inequality2} on the right-hand side of \cref{eq::Mab}, we obtain
\begin{equation}\label{eq::inequality3}
\begin{aligned}
\frac{1}{2}\|X\|_{\nM_{ij}^{\ab}}^2\leq&\;\kappa_{ij}^{\alpha\beta}\|R_i^\alpha\nR_{ij}^{\alpha\beta}\|^2+\kappa_{ij}^{\alpha\beta}\|R_j^\beta\|^2+\\
&\;\tau_{ij}^{\alpha\beta}\|R_i^\alpha \nt_{ij}^{\alpha\beta}+t_i^\alpha\|^2+\tau_{ij}^{\alpha\beta}\|t_j^\beta\|^2\\
=&\kappa_{ij}^{\alpha\beta}\|R_i^\alpha\|^2+\kappa_{ij}^{\alpha\beta}\|R_j^\beta\|^2+\\
&\tau_{ij}^{\alpha\beta}\|R_i^\alpha \nt_{ij}^{\alpha\beta}+t_i^\alpha\|^2+\tau_{ij}^{\alpha\beta}\|t_j^\beta\|^2,
\end{aligned}
\end{equation}
where the last equality is due to $\big(\nR_{ij}^{\ab}\big)^\transpose\nR_{ij}^{\ab}=\nR_{ij}^{\ab}\big(\nR_{ij}^{\ab}\big)^\transpose=\I$. Furthermore, there exists a positive semidefinite matrix $\nH_{ij}^{\alpha\beta}\in \R^{(d+1)n\times (d+1)n}$ such that the right-hand side of \cref{eq::inequality3} can be rewritten as
\begin{equation}\label{eq::Hab}
\begin{aligned}
\frac{1}{2}\|X\|_{\nH_{ij}^{\ab}}^2
=&\kappa_{ij}^{\alpha\beta}\|R_i^\alpha\|^2+\kappa_{ij}^{\alpha\beta}\|R_j^\beta\|^2+\\
 &\tau_{ij}^{\alpha\beta}\|R_i^\alpha \nt_{ij}^{\alpha\beta}+t_i^\alpha\|^2+\tau_{ij}^{\alpha\beta}\|t_j^\beta\|^2,
\end{aligned}
\end{equation}
where $\nH_{ij}^{\ab}$ is a block diagonal matrix decoupling unknown poses of different nodes. Replacing the right-hand side of \cref{eq::inequality3} with \cref{eq::Hab} results in
\begin{equation}
\nonumber
 \frac{1}{2}\|X\|_{\nM_{ij}^{\ab}}^2 \leq \frac{1}{2}\|X\|_{\nH_{ij}^{\ab}}^2
\end{equation}
for any $X\in\R^{d\times(d+1)n}$, which suggests 
\begin{equation}\label{eq::MH}
\nH_{ij}^{\ab}\succeq\nM_{ij}^{\ab}.
\end{equation}
With $\nH_{ij}^{\ab}\in\R^{(d+1)n\times(d+1)n}$ in \cref{eq::MH,eq::Hab}, we define $E_{ij}^{\ab}(\cdot|\Xk):\R^{d\times(d+1)n}\rightarrow \R$:
\begin{multline}\label{eq::Eab}
E_{ij}^{\ab}(X|\Xk)\triangleq \frac{1}{2}\wabijk\|X-\Xk\|_{\nH_{ij}^{\ab}}^2+\\
\innprod{\nabla F_{ij}^{\ab}(\Xk)}{X-\Xk}+ F_{ij}^{\ab}(\Xk),
\end{multline}
where $\wabijk$ is given in \cref{eq::wabij}. From the equation above, it can be concluded that $E_{ij}^{\ab}(X|\Xk)$ majorizes $F_{ij}^{\ab}(X)$  as the following proposition states, which is important for the construction of surrogate functions for distributed PGO.

\begin{prop}\label{prop::upper}
	Given any nodes $\alpha,\,\beta\in \AA$ with either $\alpha = \beta$ or $\alpha\neq\beta$, if $\rho(\cdot):\R^+\rightarrow \R$ is a loss kernel that satisfies \cref{assumption::loss}, then we obtain
	\begin{equation}\label{eq::EF}
	E_{ij}^{\ab}(X|\Xk)\geq F_{ij}^{\ab}(X).
	\end{equation} 
	for any $X\in \R^{d\times(d+1)n}$. In the equation above, the equality ``$=$'' holds if $X=\Xk$.
\end{prop}
\begin{proof}
	See \hyperapp{appendix::C}{C}.
\end{proof}

\subsection{The Majorization of $F(X)$}
From \cref{prop::upper}, it is straightforward to construct surrogate functions majorizing $F(X)$ in \cref{eq::obj} as the following proposition states.

\begin{prop}\label{prop::G}

	Let $X^{(\sk)}=\begin{bmatrix}
	X^{1(\sk)} & \cdots & X^{|\AA|(\sk)}
	\end{bmatrix}\in \XX$ with $X^{\alpha(\sk)}\in \XX^\alpha$ be an iterate of $X\in \XX$ in \cref{eq::pgo}. Suppose $G(\cdot|\Xk): \R^{d\times(d+1)n}\rightarrow\R$ is a function:
	\begin{multline}\label{eq::G}
		G(X|\Xk)\triangleq \sum_{\alpha\in\AA}\sum_{(i,j)\in \aEE^{\alpha\alpha}}F_{ij}^{\aa}(X)+\\
		\sum_{\substack{\alpha,\beta\in\AA,\\
				\alpha\neq \beta}}\sum_{(i,j)\in \aEE^{\alpha\beta}} E_{ij}^{\ab}(X|\Xk)+\frac{\xi}{2}\big\|X-\Xk\big\|^2
	\end{multline}
	where $\xi\in\R$ and $\xi\geq 0$. Then, we have the following results:
{\highlight
	\begin{enumerate}[(a)]
	\item \label{prop::G1}  For any $X\in\R^{d\times(d+1)}$ and $X^{(\sk)}\in\XX$, 	
	\begin{equation}\label{eq::GF}
		G(X|X^{(\sk)})\geq F(X)
	\end{equation}
	where the equality ``$=$'' holds if  $X=X^{(\sk)}$.
	\item \label{prop::G2} $G(X|\Xk)$ is equivalent to
	\begin{equation}\label{eq::G2}
		G(X|\Xk)=\sum_{\alpha\in\AA} G^{\alpha}(\Xa|\Xk)  + F(\Xk),
	\end{equation}
	where $G^\alpha(\Xa|X^{(\sk)})$ is a function of $\Xa\in\XXa$ within a single node $\alpha$. 
	\item\label{prop::G3}  For any node $\alpha\in\AA$, there exists positive-semidefinite matrices $\nGamma^{\ak}\in \R^{(d+1)n_\alpha\times (d+1)n_\alpha}$ such that 
	\begin{multline}\label{eq::GMa}
	G^\alpha(\Xa|X^{(\sk)}) \triangleq \frac{1}{2}\|\Xa-\Xak\|_{\nGamma^{\ak}}^2+\\
	\big\langle\nabla_{\Xa} F(X^{(\sk)}),\,{\Xa-\Xak}\big\rangle
	\end{multline}
	where $\nabla_{\Xa} F(X^{(\sk)})$ is the Euclidean gradient of $F(X)$ with respect to $\Xa\in \XXa$ at $\Xk\in\XX$.
	\end{enumerate}
}
\end{prop}
\begin{proof}
	See \hyperapp{appendix::D}{D}.
\end{proof}




Following \cref{prop::G}, we might further majorize $F(X)$ as well as $G(X|\Xk)$ by applying \cref{eq::EF} to \cref{eq::G} and replacing $F_{ij}^{\aa}(X|\Xk)$ with $E_{ij}^{\aa}(X|\Xk)$, which results in the following proposition.

\begin{prop}\label{prop::lG}
Let $X^{(\sk)}=\begin{bmatrix}
	X^{1(\sk)} & \cdots & X^{|\AA|(\sk)}
\end{bmatrix}\in \XX$ with $X^{\alpha(\sk)}\in \XX^\alpha$ be an iterate of $X\in \XX$ in \cref{eq::pgo}, and $X_i^{\ak}=\begin{bmatrix}
t^{\alpha(\sk)} & R^{\alpha(\sk)}
\end{bmatrix}\in SE(d)$ the corresponding iterate of $X_i^\alpha \in  SE(d)$. Suppose $\lG(\cdot|\Xk): \R^{d\times(d+1)n}\rightarrow\R$ is a function:
\vspace{-0.25em}
\begin{multline}\label{eq::lG}
\lG(X|\Xk)= \sum_{\alpha\in\AA}\sum_{(i,j)\in \aEE^{\alpha\alpha}}E_{ij}^{\aa}(X|\Xk)+\\
\sum_{\substack{\alpha,\beta\in\AA,\\
		\alpha\neq \beta}}\sum_{(i,j)\in \aEE^{\alpha\beta}} E_{ij}^{\ab}(X|\Xk) + \frac{\zeta}{2} \big\|X-\Xk\big\|^2
\end{multline}
where $\zeta\in\R$ and $\zeta\geq\xi\geq 0$. Note that $\xi\in \R$ is given in \cref{eq::G}. Then,  we have the following results:
{\highlight
\begin{enumerate}[(a)]
	\item\label{prop::lG1}  For any $X\in\R^{d\times(d+1)}$ and $X^{(\sk)}\in\XX$, 	
	\begin{equation}\label{eq::lGF}
		\lG (X|X^{(\sk)})\geq G(X|X^{(\sk)})\geq F(X)
	\end{equation}
	where the equality ``$=$'' holds if  $X=X^{(\sk)}$.
	\item\label{prop::lG2} $\lG(X|\Xk)$ is equivalent to
	\begin{equation}\label{eq::lG2}
		\begin{aligned}
			\lG(X|\Xk)&=\sum_{\alpha\in\AA} H^{\alpha}(\Xa|\Xk)  + F(\Xk)\\
							&=\sum_{\alpha\in\AA}\sum_{i=1}^{n_\alpha} \lG_i^{\alpha}(\Xa_i|\Xk) + F(\Xk)
		\end{aligned}
	\end{equation}
	where $\lG^\alpha(\Xa|X^{(\sk)})$ is a function of $\Xa\in\XXa$ within a single node $\alpha$, and  $\lG_i^\alpha(\Xa_i|X^{(\sk)})$ is a  function of a single pose $\Xa_i\in SE(d)\subset \R^{d\times (d+1)}$. 
	
	\item\label{prop::lG3} For any node $\alpha\in\AA$ and $i\!\in\!\{1,\,\cdots,\, n_\alpha\}$, there exists positive-semidefinite matrices $\lnGamma^{\ak}\in \R^{(d+1)n_\alpha\times (d+1)n_\alpha}$ and $\lnGamma_i^{\ak}\in \R^{(d+1)\times (d+1)}$ such that
	\vspace{-0.25em}
	\begin{multline}\label{eq::lGMa}
		\lG^\alpha(\Xa|X^{(\sk)}) =\frac{1}{2}\|\Xa-\Xak\|_{\lnGamma^{\ak}}^2+\\
		\big\langle\nabla_{\Xa} F(X^{(\sk)}),\,{\Xa-\Xak}\big\rangle,
	\end{multline}
	\vspace{-1.25em}
	\begin{multline}\label{eq::lGMai}
		\lG_i^\alpha(X_i^\alpha|X^{(\sk)}) =\frac{1}{2}\|X_i^\alpha-X_i^{\ak}\|_{\lnGamma_i^{\ak}}^2+\\
		\big\langle\nabla_{X_i^{\alpha}} F(X^{(\sk)}),\,{X_i^{\alpha}-X_i^{\ak}}\big\rangle
	\end{multline}
	 where $\nabla_{\Xa} F(X^{(\sk)})$ and $\nabla_{\Xa_i} F(X^{(\sk)})$ are the Euclidean gradients of $F(X)$ with respect to $\Xa\in\XXa$ and $\Xa_i\in SE(d)$ at $\Xk\in\XX$, respectively. 
\end{enumerate}
}
\end{prop}
\begin{proof}
The proof is similar to that of \cref{prop::G}.
\end{proof}

\begin{remark}
As a result of \cref{eq::lG2}, $\lG^\alpha(\Xa|\Xk)$ can be rewritten as the sum of $\lG_i^\alpha(\Xa_i|\Xk)$, i.e.,
\begin{equation}\label{eq::lGMsum}
	\lG^\alpha(\Xa|X^{(\sk)})=\sum_{i=1}^{n_\alpha} \lG_i^{\alpha}(\Xa_i|\Xk).
\end{equation}
Note that $\lG_i^\alpha(\Xa_i|\Xk)$ in \cref{eq::lG2,eq::lGMsum} relies on a single pose $X_i^\alpha\in SE(d)\subset \R^{d\times (d+1)}$. This will be later exploited  in \cref{section::mm,section::amm,section::ammc} to improve the computational efficiency of distributed PGO.
\end{remark}

\begin{remark}
Propositions \cref{prop::G,prop::lG}  indicate that $G(X|\Xk)$ and $H(X|\Xk)$ not only majorize $F(X)$ but also decouple  poses from different nodes through  $G^\alpha(\Xa|\Xk)$ and $H^\alpha(\Xa|\Xk)$, making it possible to implement majorization minimization methods for distributed PGO.
\end{remark}

\begin{remark}
	\highlight
	$\zeta$ and $\xi$  in \cref{eq::G,eq::lG} are important for the convergence analysis of MM methods for distributed PGO in \cref{section::amm,section::ammc,section::mm} . It is  recommended to be set $\zeta>\xi>0$ but close to zero such that $G(X|\Xk)$ and $\lG(X|\Xk)$ are tighter upper bounds of $F(X)$ and yield faster convergence.
\end{remark}

Recall that $G^\alpha(\Xa|\Xk)$ and $\lG^\alpha(\Xa|\Xk)$ in \cref{eq::GMa,eq::lGMa} rely on $\nGammaak$, $\lnGammaak$,
$\nabla_{\Xa} F(\Xk)$, which---according to \cref{eq::Faaabij,eq::F}---are only related to $F_{ij}^{\aa}(X)$, $F_{ij}^{\ab}(X)$, $F_{ji}^{\beta\alpha}(X)$.
Also, \cref{eq::Faaabij} indicates that $F_{ij}^{\aa}(X)$ depends on $X_i^\alpha$ and $X_j^\alpha$, while $F_{ij}^{\ab}(X)$ and $F_{ji}^{\beta\alpha}(X)$ on $X_i^\alpha$ and $X_j^\beta$. Therefore, $\nGammaak$, $\lnGammaak$,
$\nabla_{\Xa} F(\Xk)$ can be evaluated  as long as node $\alpha$ have access to $\Xb$ from its neighbor $\beta$. Furthermore, $G^\alpha(\Xa|\Xk)$ and  $\lG^\alpha(\Xa|X^{(\sk)})$  can be constructed in a distributed setting with one inter-node communication round between neighboring nodes $\alpha$ and $\beta$.

In the next sections, we will present MM methods for distributed PGO using $G(X|\Xk)$ and $\lG(X|\Xk)$ that are guaranteed to converge to first-order critical points.