\noindent\textbf{Proof of \ref{prop::G1}.\;} 
According to \cref{prop::upper}, $E_{ij}^{\ab}(X|\Xk)$ majorizes $F_{ij}^{\ab}(X)$ and $E_{ij}^{\ab}(X|\Xk)=F_{ij}^{\ab}(\Xk)$ if $X=\Xk$. Then, as a result \cref{eq::F,eq::G}, it can be concluded that $G(X|\Xk)$ majorizes $F(X)$ and $G(X|\Xk)=F(X)$ if $X=\Xk$. The proof is completed.
\vspace{1em}

\noindent\textbf{Proof of \ref{prop::G2}.\;}
From \cref{eq::Hab}, we obtain
\vspace{-0.3em}
\begin{multline}\label{eq::HXab}
\frac{1}{2}\|X-\Xk\|_{\nH_{ij}^{\ab}}^2=\kappa_{ij}^{\alpha\beta}\|R_i^\alpha-R_i^{\ak}\|^2+\\
\tau_{ij}^{\alpha\beta}\|(R_i^\alpha - R_i^{\ak})\nt_{ij}^{\alpha\beta}+t_i^\alpha-t_i^{\ak}\|^2+\\
\kappa_{ij}^{\alpha\beta}\|R_j^\beta-R_j^{\beta(\sk)}\|^2+\tau_{ij}^{\alpha\beta}\|t_j^\beta-t_j^{\beta(\sk)}\|^2.
\end{multline}
From \cref{eq::Fij2}, it is by definition that $F_{ij}^{\ab}(X)$  is a function related with $X^\alpha\in \XX^\alpha$ and $X^\beta\in\XX^\beta$ only, and thus, $\nabla F_{ij}^{\ab}(X)$ is sparse, which suggests
\vspace{-0.3em}
\begin{multline}\label{eq::DFabXX}
\innprod{\nabla F_{ij}^{\ab}(\Xk)}{X-\Xk}=\\
\innprod{\nabla_{\Xa} F_{ij}^{\ab}(\Xk)}{\Xa-\Xak}+\\
\innprod{\nabla_{X^{\beta}} F_{ij}^{\ab}(\Xk)}{X^{\beta}-X^{\beta(\sk)}}.
\end{multline}
In \cref{eq::DFabXX}, $\nabla_{\Xa} F_{ij}^{\ab}(\Xk)$ is the Euclidean gradient of $F_{ij}^{\ab}(X)$ with respect to $\Xa\in\XXa$ at $\Xk\in\XX$.  Substituting \cref{eq::HXab,eq::DFabXX} into \cref{eq::Eab}, we obtain
\vspace{-0.3em}
\begin{multline}\label{eq::Eij3}
E_{ij}^{\ab}(X|\Xk)=\wabijk\cdot\left(\kappa_{ij}^{\alpha\beta}\|R_i^\alpha-R_i^{\ak}\|^2+\right.\\
\tau_{ij}^{\alpha\beta}\|(R_i^\alpha - R_i^{\ak})\nt_{ij}^{\alpha\beta}+t_i^\alpha-t_i^{\ak}\|^2+\\
\left.\kappa_{ij}^{\alpha\beta}\|R_j^\beta-R_j^{\beta(\sk)}\|^2+\tau_{ij}^{\alpha\beta}\|t_j^\beta-t_j^{\beta(\sk)}\|^2\right)+\\
\innprod{\nabla_{\Xa} F_{ij}^{\ab}(\Xk)}{\Xa-\Xak}+\\
\innprod{\nabla_{X^{\beta}} F_{ij}^{\ab}(\Xk)}{X^{\beta}-X^{\beta(\sk)}}+F_{ij}^{\ab}(\Xk).
\end{multline}
In a similar way, $F_{ij}^{\aa}(X)$ in \cref{eq::Fij2} can be rewritten as
\vspace{-0.3em}
\begin{multline}\label{eq::Fij3}
F_{ij}^{\aa}(X)=\frac{1}{2}\kappa_{ij}^{\aa}\|(R_i^\alpha-R_i^{\ak})\nR_{ij}^{\aa} -(R_j^\alpha-R_j^{\alpha(\sk)})\|^2 +\\ 
\frac{1}{2}\tau_{ij}^{\aa}\|(R_i^\alpha-R_i^{\ak}) \nt_{ij}^{\aa}+t_i^\alpha - t_i^{\ak} - (t_j^\alpha-t_j^{\alpha(\sk)})\|^2+\\
\innprod{\nabla_{\Xa} F_{ij}^{\alpha\alpha}(\Xk)}{\Xa-\Xak} + F_{ij}^{\alpha\alpha}(\Xk).
\end{multline}
Substituting \cref{eq::Eij3,eq::Fij3} into \cref{eq::G} and simplifying the resulting equation, we obtain
\begin{equation}\label{eq::Gsum}
G(X|X^{(\sk)})=\sum_{\alpha\in\AA} G^{\alpha}(X^\alpha|\Xk) + F(\Xk)
\end{equation}
where $G^\alpha(\Xa|\Xk)$ is a function that is related with $\Xa\in \XX^\alpha$ only. The proof is completed.

\vspace{1em}

\noindent\textbf{Proof of \ref{prop::G3}.\;}
A tedious but straightforward mathematical manipulation from \cref{eq::Fij3,eq::Eij3,eq::DFabXX,eq::Gsum} indicates that there exists positive-semidefinite matrices $\nGamma^{\ak}\in \R^{(d+1)n_\alpha\times (d+1)n_\alpha}$ such that $G^\alpha(X^\alpha|\Xk)$ in the equation above can be written as
\begin{multline}
	\nonumber
	G^\alpha(X^\alpha|\Xk)=
	\frac{1}{2}\|\Xa-\Xak\|_{\nGamma^{\ak}}^2+\\
	\big\langle\nabla_{\Xa} F(X^{(\sk)}),\,{\Xa-\Xak}\big\rangle,
\end{multline}
in which the formulation of $\nGamma^{\ak}$ is given in Appendix \hyperref[appendix::I]{I}. The proof is completed.

%\noindent\textbf{Proof of \ref{prop::G2}.\;} Substituting \cref{eq::GMa} into \cref{eq::G2} results in
%\begin{multline}\label{eq::EXk}
%G(X|\Xk)=\sum_{\alpha\in\AA}\Big[\frac{1}{2}\|\Xa-\Xak\|_{\nGamma^{\ak}}^2+\\
%\big\langle\nabla_{\Xa} F(X^{(\sk)}),\,{\Xa-\Xak}\big\rangle\Big] + F(\Xk).
%\end{multline}
%Furthermore, it can be shown that
%\begin{equation}
%\nonumber
%\frac{1}{2}\|X-\Xk\|_{\nGammak}^2=\sum_{\alpha\in\AA}\frac{1}{2}\|\Xa-\Xak\|_{\nGamma^{\ak}}^2,
%\end{equation}
%in which $\nGammak\in\R^{(d+1)n\times(d+1)n}$ is defined as \cref{eq::nG}, and  
%\begin{multline}
%\nonumber
%\innprod{\nabla F(\Xk)}{X-\Xk}=\\
%\sum_{\alpha\in\AA}\innprod{\nabla_{\Xa} F(\Xk)}{X^\alpha-\Xak}.
%\end{multline}
%Thus, \cref{eq::EXk} is equivalent to \cref{eq::GM}, i.e.,
%\begin{multline}\label{eq::E3}
%G(X|\Xk) = \frac{1}{2}\|X-\Xk\|_{\nGammak}^2+\\
%\innprod{\nabla F(X^{(\sk)})}{{X-X^{(\sk)}}}+ F(X^{(\sk)}).
%\end{multline}
%\vspace{1.2em}
%
%\noindent\textbf{Proof of \ref{prop::G3}.\;}From \cref{eq::G,eq::Faa,eq::Eab,eq::E3}, we rewrite $\nGammak\in\R^{(d+1)n\times(d+1)n}$ as
%\begin{multline}\label{eq::H}
%\nGammak=\sum_{\alpha\in\AA}\sum_{(i,j)\in \aEE^{\alpha\alpha}}\nM_{ij}^{\aa}+\\
%\sum_{\substack{\alpha,\beta\in\AA,\\\alpha\neq \beta}}\sum_{(i,j)\in \aEE^{\alpha\beta}}\wabijk\cdot \nH_{ij}^{\ab}+\xi\cdot\I,
%\end{multline}
%in which $\nH_{ij}^{\ab}\succeq\nM_{ij}^{\ab}$ by \cref{eq::MH} and $\xi\geq 0$. Then, as a result of \cref{eq::M,eq::H,eq::MH}, it is straightforward to conclude that
%\begin{equation}\label{eq::GMk}
%\nGammak\succeq\nMk+\xi \cdot\I\succeq \nMk.
%\end{equation}
%The proof is completed.
%\vspace{1.2em}
%
%\noindent\textbf{Proof of \ref{prop::G4}.\;}  Let $\nGamma\in\R^{(d+1)n\times(d+1)n}$ be defined as
%\begin{equation}\label{eq::HH}
%	\nGamma\triangleq\sum_{\alpha\in\AA}\sum_{(i,j)\in \aEE^{\alpha\alpha}}\nM_{ij}^{\aa}+
%	\sum_{\substack{\alpha,\beta\in\AA,\\\alpha\neq \beta}}\sum_{(i,j)\in \aEE^{\alpha\beta}} \nH_{ij}^{\ab}+\xi\cdot\I.
%\end{equation}
%From \cref{assumption::loss}\ref{assumption::loss_drho} and \cref{eq::wabij}, it can be concluded that
%\begin{equation}\label{eq::wabijkbnd}
%	0\leq \wabijk \leq 1
%\end{equation} 
%for any $\Xk\in\R^{d\times(d+1)n}$. Furthermore, it is known that $\nH_{ij}^{\ab} \succeq 0$, then \cref{eq::HH,eq::H,eq::wabijkbnd} result in $\nGamma\succeq\nGammak$ for any $\Xk\in\R^{d\times(d+1)n}$. The proof is completed.