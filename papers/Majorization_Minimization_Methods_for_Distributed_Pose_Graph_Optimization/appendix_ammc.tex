\noindent\textbf{Proof of \ref{prop::ammc1}.\;} In this proof, we will prove $F(\Xk)\leq \sF^{(\sk)}$ and $\sF^{(\skp)}\leq \sF^{(\sk)}$ by induction.
\begin{enumerate}[leftmargin=0.45cm]%[wide, labelwidth=!, labelindent=0pt]
\item From lines~\ref{line::alg3::s0}, \ref{line::alg3::lF0}, \ref{line::alg3::lFk} of \cref{algorithm::ammc}, it can be shown that
\begin{equation}\label{eq::sF0011}
	F(X^{(-1)})=F(X^{(0)})=\lF^{(-1)}=\lF^{(0)}.
\end{equation}
\item Suppose $\sk\geq 0$ and $F(\Xk)\leq \lF^{(\sk)}$ holds at $\sk$-th iteration. In terms of $\Xkh$, if the adaptive restart scheme for $\Xkh$  is not triggered, it is immediate to show from line~\ref{line::alg4::restart_s1} of \cref{algorithm::amm_c_x} that
\begin{equation}\label{eq::Fah1}
F(\Xkh) \leq \lF^{(\sk)}.
\end{equation}
On the other hand, if the adaptive restart scheme for $\Xkh$ is triggered, line~\ref{line::alg4::xlG2} of \cref{algorithm::amm_c_x} results in
\begin{equation}\label{eq::Hah}
H^\alpha(\Xakh|\Xk) \leq H^\alpha(\Xak|\Xk)= 0,
\end{equation}
where $H^{\alpha}(\Xak|\Xk) = 0$ is from \cref{eq::lGMa}. Then, \cref{eq::lG2,eq::Hah,eq::lGF} indicate
\begin{equation}\label{eq::Fah2}
F(\Xkh)\leq H(\Xkh|\Xk) \leq F(\Xk) \leq \sF^{(\sk)}.
\end{equation}
Therefore, no matter whether the adaptive restart scheme is triggered or not, we conclude from \cref{eq::Fah1,eq::Fah2} that
\begin{equation}\label{eq::Fah}
F(\Xkh) \leq \lF^{(\sk)}
\end{equation}
always holds.

Furthermore, as a result of lines~\ref{line::alg4::restart_s3} to \ref{line::alg4::restart_e3} of \cref{algorithm::amm_c_x}, we obtain
\begin{equation}\label{eq::FkpF}
F(\Xkp)-\sF^{(\sk)} \leq \phi\cdot\Big(F(\Xkh)-\lF^{(\sk)}\Big) \leq 0.
\end{equation}
From line~\ref{line::alg3::lFk} of \cref{algorithm::ammc} and $F(\Xkp)-\sF^{(\sk)} \leq 0$ in \cref{eq::FkpF}, we obtain
\begin{equation}
\nonumber
F(\Xkp)- \sF^{(\skp)} = (1-\eta)\cdot \big(F(\Xkp) - \sF^{(\sk)} \big)\leq 0
\end{equation}
and
\begin{equation}
\nonumber
\sF^{(\skp)}-\sF^{(\sk)} = \eta\cdot \big(F(\Xkp) - \sF^{(\sk)} \big)\leq 0,
\end{equation}
which suggest $F(\Xkp)\leq \sF^{(\skp)}\leq \sF^{(\sk)}$. 
\item Therefore, it can be concluded that $F(\Xk)\leq \sF^{(\sk)}$ and $\sF^{(\skp)}\leq \sF^{(\sk)}$, which suggests that $\sF^{(\sk)}$ is nonincreasing. The proof is completed.
\end{enumerate}

\vspace{0.8em}
\noindent\textbf{Proof of \ref{prop::ammc2}.\;} From line~\ref{line::alg3::lFk} of \cref{algorithm::ammc}, we obtain
\begin{equation}\label{eq::lFkp}
	\lF^{(\skp)} = (1-\eta)\cdot \lF^{(\sk)} + \eta\cdot F(\Xkp),
\end{equation}
which and \cref{eq::sF0011} suggest that $\lF^{(\sk)}$ is a convex combination of $F(X^{(0)}),\,F(X^{(1)}),\,\cdots F(\Xk)$ as long as $\eta\in(0,\,1]$. Since $F(\Xk)\geq 0$, we obtain $\lF^{(\sk)} \geq 0$ as well, i.e., $\lF^{(\sk)}$ is bounded below. \cref{prop::ammc}\ref{prop::ammc1} indicates that $\lF^{(\sk)}$ is nonincreasing, and thus, there exists $F^{\infty}$ such that $\lF^{(\sk)}\rightarrow F^{\infty}$. Then, it can be still concluded from \cref{eq::lFkp} that $F(\Xk)\rightarrow F^{\infty}$.

\vspace{0.8em}
\noindent\textbf{Proof of \ref{prop::ammc3}.\;}
If $\sk=-1$, line~\ref{line::alg3::s0} of \cref{algorithm::ammc} and \cref{eq::lFk} suggest $X^{(-1)}=X^{(0)}$  and $\lF^{(-1)}=\lF^{(0)}=F(X^{(0)})$, respectively, from which we conclude
\begin{equation}\label{eq::lF01}
	\lF^{(-1)}-\lF^{(0)}=\|X^{(-1)}-X^{(0)}\|^2.
\end{equation}

If $\sk\geq 0$, there exists three possible cases for $\Xkp$:
\begin{enumerate}[leftmargin=0.45cm]
\item\label{cases::prop6::1} If $\Xkp$ is from line~\ref{line::alg4::xG1} of \cref{algorithm::amm_c_x}, then the adaptive restart scheme is not triggered and line~\ref{line::alg4::restart_s1} of \cref{algorithm::amm_c_x} results in
\begin{equation}\label{eq::FlF1}
	 \sF^{(\sk)}-F(\Xkp) \geq \psi\cdot\big\|\Xkp-\Xk\big\|^2.
\end{equation}
\item\label{cases::prop6::2} If $\Xkp$ is from line~\ref{line::alg4::xG2} of \cref{algorithm::amm_c_x}, then the adaptive restart scheme is triggered and \cref{eq::FkFkp} holds as well, from which and \cref{eq::GMk} we obtain
\begin{equation}
\nonumber
F(\Xk)-F(\Xkp)\geq \frac{\xi}{2}\big\|\Xkp-\Xk\big\|^2.
\end{equation}
In the proof of \cref{prop::ammc}\ref{prop::ammc1}, it is known that $\sF^{(\sk)}\geq F(\Xk)$, then the equation above results in
\begin{equation}\label{eq::FlF2}
\sF^{(\sk)}-F(\Xkp)\geq \frac{\xi}{2}\big\|\Xkp-\Xk\big\|^2.
\end{equation}
\item\label{cases::prop6::3} If $\Xkp$ is from line~\ref{line::alg4::xG3} of \cref{algorithm::amm_c_x}, then we obtain $\Xkp=\Xkh$ and $F(\Xkp)=F(\Xkh)$. Then, similar to the derivations of \cref{eq::FlF1,eq::FlF2}, lines~\ref{line::alg4::xlG1} and \ref{line::alg4::xlG2} of \cref{algorithm::amm_c_x} result in
\begin{equation}\label{eq::FlF3}
\sF^{(\sk)}-F(\Xkh) \geq \psi\cdot\big\|\Xkh-\Xk\big\|^2
\end{equation}
and
\begin{equation}\label{eq::FlF4}
\sF^{(\sk)}-F(\Xkh)\geq \frac{\zeta}{2}\big\|\Xkh-\Xk\big\|^2,
\end{equation}
respectively, from which and $\Xkp=\Xkh$ and $F(\Xkp)=F(\Xkh)$ we obtain either
\begin{equation}
\nonumber
\sF^{(\sk)}-F(\Xkp) \geq \psi\cdot\big\|\Xkp-\Xk\big\|^2
\end{equation}
or
\begin{equation}
\nonumber
\sF^{(\sk)}-F(\Xkp)\geq \frac{\zeta}{2}\big\|\Xkp-\Xk\big\|^2.
\end{equation}
\end{enumerate}
Therefore, if $0<\eta\leq 1$, $0<\psi\leq 1$, $\xi>0$, $\zeta>0$, it can be shown from cases \ref{cases::prop6::1}) to \ref{cases::prop6::3}) above that there exists a constant scalar $\sigma>0$ such that
\begin{equation}\label{eq::sFF0}
	\sF^{(\sk)}-F(\Xkp) \geq  \frac{\sigma}{2}\|\Xkp-\Xk\|^2
\end{equation}
holds for any $\sk\geq 0$. In addition, note that \cref{eq::lFkp} is equivalent to
\begin{equation}\label{eq::lFklFkp}
	\lF^{(\sk)}-\lF^{(\skp)}= \eta\cdot\Big(\sF^{(\sk)}-F(\Xkp)\Big).
\end{equation} 
From \cref{eq::sFF0,eq::lFklFkp}, we conclude that there exists $\delta > 0$ such that
\begin{multline}\label{eq::sFF01}
\sF^{(\sk)}-\sF^{(\skp)}=\eta\cdot\Big(\sF^{(\sk)}-F(\Xkp)\Big) \geq \\
\frac{\eta\sigma}{2}\|\Xkp-\Xk\|^2 \geq \frac{\delta}{2}\|\Xkp-\Xk\|^2
\end{multline}
for any $\sk\geq 0$. 

As a result of \cref{eq::lF01,eq::sFF01}, we further obtain
\begin{equation}\label{eq::sFF1}
\sF^{(\sk)}-\sF^{(\skp)}\geq \frac{\delta}{2}\|\Xkp-\Xk\|^2
\end{equation}
for any $\sk\geq -1$. Recall $\lF^{(\sk)}\rightarrow F^\infty$ from \cref{prop::ammc}\ref{prop::ammc2}, which suggests
\begin{equation}\label{eq::sFF2}
	\sF^{(\sk)} - \sF^{(\skp)}\rightarrow 0.
\end{equation}
Therefore, \cref{eq::sFF1,eq::sFF2} indicate that
\begin{equation}
	\nonumber
	\|X^{(\sk+1)}-X^{(\sk)}\| \rightarrow 0.
\end{equation}
The proof is completed.

\vspace{0.8em}
\noindent\textbf{Proof of \ref{prop::ammc4}.\;} Note that \cref{eq::FlF3,eq::FlF4} suggest that there exists $\sigma'>0$ such that
\begin{equation}\label{eq::FlF6}
	\sF^{(\sk)}-F(\Xkh)\geq \frac{\sigma'}{2}\big\|\Xkh-\Xk\big\|^2
\end{equation}
always holds. From lines~\ref{line::alg4::restart_s3} to \ref{line::alg4::restart_e3} of \cref{algorithm::amm_c_x}, we obtain 
\begin{multline}\label{eq::lFkFkp0}
\lF^{(\sk)}-F(\Xkp) \geq \phi\cdot\Big(\lF^{(\sk)} - F(\Xkh)\Big)\geq\\
\frac{\phi\sigma'}{2}\big\|\Xkh-\Xk\big\|^2,
\end{multline}
where the last inequality is due to \cref{eq::FlF6}.
From \cref{eq::lFklFkp,eq::lFkFkp0}, it can be shown that
\begin{multline}\label{eq::FlF5}
\lF^{(\sk)}-\lF^{(\skp)}= \eta\cdot\Big(\sF^{(\sk)}-F(\Xkp)\Big)\geq\\
\frac{\eta\phi\sigma'}{2}\big\|\Xkh-\Xk\big\|^2.
\end{multline} 
The equation above suggests that there exists a constant scalar $\delta'>0$ such that
\begin{equation}\label{eq::sFF3}
\lF^{(\sk)}-\lF^{(\skp)}\geq \frac{\delta'}{2}\|\Xkh-\Xk\|^2.
\end{equation}
Note that \cref{prop::ammc}\ref{prop::ammc2} results in $\sF^{(\sk)} - \sF^{(\skp)}\rightarrow 0$. Thus, similar to the proof of \cref{prop::ammc}\ref{prop::ammc3}, it can be concluded from \cref{eq::sFF3} that
\begin{equation}
	\nonumber
	\|\Xkh-\Xk\| \rightarrow 0
\end{equation}
if $\zeta>\xi>0$. The proof is completed.

\vspace{0.8em}
\noindent\textbf{Proof of \ref{prop::ammc5}.\;} For any $\sk\geq 0$,  there are two possible cases about $\Xakh\in\XXa$:
\begin{enumerate}[leftmargin=0.45cm]%[wide, labelwidth=!, labelindent=0pt]
\item If $\Xakh\in\XXa$ is from line~\ref{line::alg4::xlG1} of \cref{algorithm::amm_c_x}, we obtain
\begin{equation}\label{eq::minH1}
	\Xakh\gets\arg\min\limits_{\Xa\in\XX^\alpha }\lG^\alpha(\Xa|\Yk).
\end{equation}
From \cref{eq::lGMa}, it can be shown that
\begin{equation}
	\nonumber
	\begin{aligned}
		&\nabla \lG^\alpha(\Xakh|\Yk)\\
		=&\nabla_{\Xa} F(\Yk)+(\Xakh-\Yak)\lnGamma^{\ak}\\
		=&\nabla_{\Xa} F(\Xkh)+(\Xakh-\Yak)\lnGamma^{\ak}\, +\\
		&\big(\nabla_{\Xa} F(\Yk) -\nabla_{\Xa} F(\Xkh)\big).
	\end{aligned}
\end{equation}
The equation above suggests
\begin{equation}\label{eq::gradHa}
\begin{aligned}
&\grad\, \lG^\alpha(\Xakh|\Yk)\\
=&\grad_{\alpha} F(\Xkh)+\\
&\QQ_{\Xakh}^\alpha\big((\Xakh-\Yak)\lnGamma^{\ak}\big) +\\
&\QQ_{\Xakh}^\alpha\big(\nabla_{\Xa} F(\Yk) -\nabla_{\Xa} F(\Xkh)\big),
\end{aligned}
\end{equation}
where $\grad_\alpha F(X)$ is the Riemannian gradient of $F(X)$ with respect to $\Xa \in\XX^\alpha$, and $\QQ_{\Xa}^\alpha: \R^{d\times dn_\alpha}\rightarrow \R^{d\times dn_\alpha}$ is a linear operator that extracts the $\alpha$-th block of $\QQ_{X}(\cdot)$ in \cref{eq::QQx}. Since $\Xakh$ is an optimal solution to \cref{eq::minH1}, we obtain
\begin{equation}\label{eq::gradHa0}
\grad\, H^\alpha(\Xakh|\Yk)=\0.
\end{equation}
From \cref{eq::gradHa,eq::gradHa0}, a straightforward mathematical manipulation indicates
\begin{equation}\label{eq::gradnFh1}
	\begin{aligned}
		&\grad_{\alpha} F(\Xkh) \\
		=&\QQ_{\Xakh}^\alpha\big((\Yak-\Xakh)\lnGamma^{\ak}\big)+\\
		&\QQ_{\Xakh}^\alpha\big(\nabla_{\Xa} F(\Xkh)-\nabla_{\Xa} F(\Yk)\big).
	\end{aligned}
\end{equation}
From \cref{eq::gradnFh1}, we further obtain
\begin{equation}
\nonumber
\begin{aligned}
&\|\grad_{\alpha} F(\Xkh)\|\\
\leq& \|\QQ_{\Xakh}^\alpha\|_2\cdot\|(\Yak-\Xakh)\lnGamma^{\ak}\|+\\
& \|\QQ_{\Xakh}^\alpha\|_2\cdot\|\nabla_{\Xa} F(\Xkh)-\nabla_{\Xa} F(\Yk)\|\\
\leq&\|\QQ_{\Xakh}^\alpha\|_2\cdot\|(\Yk-\Xkh)\lnGammak\|+\\
&\|\QQ_{\Xakh}^\alpha\|_2\cdot\|\nabla F(\Xkh)-\nabla F(\Yk)\|\\
\leq&\|\QQ_{\Xakh}^\alpha\|_2\cdot\|\lnGammak\|_2\cdot\|\Xkh-\Yk\|+\\
&\|\QQ_{\Xakh}^\alpha\|_2\cdot\|\nabla F(\Xkh)-\nabla F(\Yk)\|,
\end{aligned}
\end{equation}
where $\|\cdot\|_2$ denotes the induced 2-norm of linear operators. Recall from Lemmas \ref{lemma::lG}\ref{lemma::lGc} and \ref{prop::DF} that $\lnGamma\succeq\lnGammak$ and $\|\nabla F(\Xkh)-\nabla F(\Yk)\|\leq \mu\cdot\|\Xkh-\Yk\|$. Therefore, the right-hand side of the equation above can be further upper-bounded as
\begin{equation}\label{eq::gradak}
\begin{aligned}
	&\|\grad_\alpha F(\Xkh)\|\\
\leq&\|\QQ_{\Xakh}^\alpha\|_2\cdot \|\lnGamma\|_2\cdot\|\Xkh-\Yk\|+\\
	&\|\QQ_{\Xakh}^\alpha\|_2\cdot\mu\cdot\|\Xkh-\Yk\|.
\end{aligned}
\end{equation}
Similar to \cref{eq::gradFk,eq::gradFh}, $\|\QQ_{\Xakh}^\alpha\|_2$ in \cref{eq::gradak} is bounded as well. Therefore, there exists a constant scalar $\nu^\alpha >0$ such that the right-hand side of \cref{eq::gradak} can be upper-bounded:
\begin{equation}\label{eq::gradFa1}
	\|\grad_{\alpha} F(\Xkh)\|\leq \nu^\alpha \|\Xkh-\Yk\|.
\end{equation}
Recall that $\Yk\in\R^{d\times(d+1)n}$ results from line~\ref{line::alg3::Yk} of \cref{algorithm::ammc}:
\begin{equation}\label{eq::Y}
	\Yk = X^{(\sk)}+\big(X^{(\sk)}-X^{(\sk-1)}\big) \lambda^{(\sk)}.
\end{equation}
In \cref{eq::Y}, $\lambda^{(\sk)}\in\R^{(d+1)n\times(d+1)n}$ is a diagonal matrix
\begin{equation} 
	\nonumber
	\lambda^{(\sk)}\triangleq\diag\{\lambda^{1(\sk)}\cdot\I^1,\,\cdots,\,\lambda^{|\AA|(\sk)}\cdot\I^{|\AA|}\}\in \R^{(d+1)n\times(d+1)n},
\end{equation}
where $\lambda^{\ak}\in\R$ is given by line~\ref{line::alg3::sk} of \cref{algorithm::ammc} and $\I^\alpha\in \R^{(d+1)n_\alpha\times (d+1)n_\alpha}$ is the identity matrix. From \cref{eq::Y,eq::gradFa1}, it can be shown that
\begin{equation}\label{eq::gradnFbnd}
\begin{aligned}
	&\|\grad_\alpha F(\Xkh)\|\\
\leq&\nu^\alpha \|\Xkh-\Xk-\big(\Xk-\Xkm\big) \lambda^{(\sk)}\|\\
\leq&\nu^\alpha\|\big(\Xk\!-\Xkm\big) \lambda^{(\sk)}\|+\nu^\alpha \|\Xkh\!-\!\Xk\|\\
\leq&\nu^\alpha\|\lambda^{(\sk)}\|_2\cdot\|\Xk-\Xkm\|+\\
	&\nu^\alpha \|\Xkh-\Xk\|.
\end{aligned}
\end{equation} 
From line~\ref{line::alg3::sk} of \cref{algorithm::ammc}, we obtain $s^{\alpha(\sk)}\geq1$, and thus,
\begin{equation}
	\nonumber
	\lambda^{\ak}=\frac{\sqrt{4{s^{\alpha(\sk)}}^2+1}-1}{2s^{\alpha(\sk)}}=\frac{2s^{\alpha(\sk)}}{\sqrt{4{s^{\alpha(\sk)}}^2+1}+1}\in(0,\,1),
\end{equation}
which suggests $\|\lambda^{(\sk)}\|_2\in(0,\,1)$. Then, we upper-bound the right-hand side of \cref{eq::gradnFbnd} using $\|\lambda^{(\sk)}\|_2\in(0,\,1)$:
\begin{multline}\label{eq::gradanFF1}
	\|\grad_{\alpha} F(\Xkh)\|\leq \nu^\alpha\|\Xk-\Xkm\|+\\\nu^\alpha\|\Xkh-\Xk\|.
\end{multline}
\item If $\Xakh\in\XXa$ is from line~\ref{line::alg4::xlG2} of \cref{algorithm::amm_c_x}, we obtain
\begin{equation}\label{eq::minH2}
	\Xakh\gets\arg\min\limits_{\Xa\in\XX^\alpha }\lG^\alpha(\Xa|\Xk).
\end{equation}
A similar procedure to the derivation of \cref{eq::gradFa1} yields
\begin{equation}
	\nonumber
	\|\grad_{\alpha} F(\Xkh)\|\leq \nu^\alpha \|\Xkh-\Xk\|,
\end{equation}
where $\nu^\alpha>0$ is the same as that in \cref{eq::gradFa1}. Thus, we obtain
\begin{multline}\label{eq::gradanFF2}
	\|\grad_{\alpha} F(\Xkh)\leq \nu^\alpha\|\Xkh-\Xk\|\leq
	\\  \nu^\alpha\|\Xk-\Xkm\|+\nu^\alpha\|\Xkh-\Xk\|.
\end{multline}
\end{enumerate}
Therefore, as long as $\sk\geq 0$, it can be concluded from \cref{eq::gradanFF1,eq::gradanFF2} that
\begin{multline}\label{eq::gradanF2}
\|\grad_{\alpha} F(\Xkh)\|\leq \nu^\alpha\|\Xk-\Xkm\|+\\ \nu^\alpha\|\Xkh-\Xk\|
\end{multline}
holds for any node $\alpha\in\AA$. 

If $\sk\geq 0$, as a result of \cref{eq::gradanF2}, there exists a constant scalar $\nu\triangleq\sum_{\alpha\in\AA}\nu^\alpha>0$ such that
\begin{equation}
\nonumber
\begin{aligned}
	&\|\grad\, F(\Xkh)\|\\
\leq&\,\sum_{\alpha\in\AA}\|\grad_\alpha F(\Xkh)\|\\
\leq&\,\sum_{\alpha\in\AA}\nu^\alpha\cdot\big(\|\Xk-\Xkm\| + \|\Xkh-\Xk\|\big)\\
=&\, \nu\|\Xk-\Xkm\| + \nu\|\Xkh-\Xk\|\\
\leq&\sqrt{2}\nu\sqrt{ \|\Xk-\Xkm\|^2 + \|\Xkh-\Xk\|^2},
\end{aligned}
\end{equation}
which is equivalent to
\begin{multline}\label{eq::gradF2}
\|\grad\,F(\Xkh)\|^2\leq 2\nu^2\cdot\|\Xk-\Xkm\|^2 + \\
2\nu^2\cdot\|\Xkh-\Xk\|^2.
\end{multline}
Note that \cref{eq::sFF1,eq::sFF3} hold as long as $\zeta>\xi>0$, from which we might upper-bound $\|\Xk-\Xkm\|^2$ and $\|\Xkh-\Xk\|^2$ in \cref{eq::gradF2} and obtain
\begin{multline}\label{eq::gradFF2}
	\|\grad\,F(\Xkh)\|^2\leq\frac{4\nu^2}{\delta}\big(\lF^{(\skm)}-\lF^{(\sk)}\big) +\\
	 \frac{4\nu^2}{\delta'}\big(\lF^{(\sk)}-\lF^{(\skp)}\big).
\end{multline}
Recall from \cref{prop::ammc}\ref{prop::amm1} that 
\begin{equation}\label{eq::sFkmk}
\sF^{(\skm)}-\sF^{(\sk)}\geq 0
\end{equation}
and 
\begin{equation}\label{eq::sFkkp}
\sF^{(\sk)}-\sF^{(\skp)}\geq 0.
\end{equation}
Then, if we let $\epsilon \triangleq \min\{\frac{\delta}{2\nu^2},\,\frac{\delta'}{2\nu^2}\}>0$, \cref{eq::gradFF2,eq::sFkmk,eq::sFkkp} lead to
\begin{equation}\label{eq::gradF3}
	\lF^{(\skm)}-\lF^{(\skp)} \geq \frac{\epsilon}{2}\|\grad\,F(\Xkh)\|^2.
\end{equation}
A telescoping sum of \cref{eq::gradF3} over $\sk$ from $0$ to $\sK$ yields
\begin{equation}
	\nonumber
	\lF^{(-1)} + \lF^{(0)} - \lF^{(\sk)}-\lF^{(\skp)}\geq \\
	\frac{\epsilon}{2} \sum_{\sk=0}^{\sK}\|\grad\,F(\Xkh)\|^2,
\end{equation}
and thus,
\begin{multline}\label{eq::lFbnd2}
	\lF^{(-1)} + \lF^{(0)} - \lF^{(\sk)}-\lF^{(\skp)}\geq \\
	\frac{\epsilon(\sK+1)}{2}\min_{0\leq\sk\leq\sK}\|\grad\, F(\Xkh)\|^2.
\end{multline}
From lines~\ref{line::alg3::lF0}, \ref{line::alg3::lFk} of \cref{algorithm::ammc}, we obtain
\begin{equation}\label{eq::lF0}
	\lF^{(-1)}=\lF^{(0)}=F(X^{(0)}),
\end{equation} 
and Propositions \ref{prop::ammc}\ref{prop::ammc1} and \ref{prop::ammc}\ref{prop::ammc2} indicate
\begin{equation}\label{eq::Finf}
	\lF^{(\sk)}\geq \lF^{(\skp)}\geq F^{\infty}.
\end{equation}
As a result of \cref{eq::lF0,eq::Finf,eq::lFbnd2}, it can be concluded that
\begin{multline}
	\nonumber
	F(X^{(0)})-F^\infty \geq \frac{\epsilon(\sK+1)}{4}\min_{0\leq\sk\leq\sK}\|\grad\, F(\Xkh)\|^2,
\end{multline}
which is equivalent to
\begin{equation}
	\min\limits_{0\leq\sk< \mathsf{K}}\|\grad\, F(\Xkh)\|\leq 2\sqrt{\frac{1}{\epsilon}\cdot\dfrac{F(X^{(0)})-F^\infty}{{\sK+1}}}.
\end{equation}
The proof is completed.


\vspace{0.5em}
\noindent\textbf{Proof of \ref{prop::ammc6}.\;} From Propositions \ref{prop::ammc}\ref{prop::ammc3} and \ref{prop::ammc}\ref{prop::ammc4}, we obtain
\begin{equation}
	\nonumber
	\|\Xkp-\Xk\|\rightarrow 0
\end{equation}
and
\begin{equation}
	\nonumber
	\|\Xkh-\Xk\|\rightarrow 0
\end{equation}
as long as $\zeta>\xi>0$, from which and \cref{eq::gradF2}, it is trivial to show that
\begin{equation}\label{eq::gradFkh}
	\grad\, F(\Xkh)\rightarrow \0.
\end{equation}
In addition, note that $\grad\,F(X)$ is continuous by \cref{assumption::loss}\ref{assumption::loss_cont}, which suggests
\begin{equation}\label{eq::gradFkkh}
	\grad\,F(\Xk)\rightarrow\grad\,F(\Xkh).
\end{equation}
From \cref{eq::gradFkh,eq::gradFkkh}, it can be concluded that
\begin{equation}
	\nonumber
	\grad\, F(\Xk) \rightarrow \0.
\end{equation}
The proof is completed.

%If the adaptive restart scheme is not triggered for node $\alpha\in\AA$, then line~\ref{line::alg4::xlG1} of \cref{algorithm::amm_c_x}
%
%for all $\alpha\in\AA$, which is equivalent to
%\begin{equation}\label{eq::minH}
%	\Xkh\gets\arg\min\limits_{X\in\XX}\lG(X|Y^{(\sk)}).
%\end{equation}
%From \cref{eq::lGM}, we obtain
%\begin{equation}
%	\nonumber
%	\begin{aligned}
%		&\nabla \lG(\Xkh|\Yk)\\
%		=&\nabla_{X} F(\Yk)+(\Xkh-\Yk)\lnGammak\\
%		=&\nabla_{X} F(\Xkh)+(\Xkh-\Yk)\lnGammak\, +\\
%		&\big(\nabla_{\Xa} F(\Yk) -\nabla_{\Xa} F(\Xkh)\big),
%	\end{aligned}
%\end{equation}
%which suggests
%\begin{equation}\label{eq::gradHY1}
%\begin{aligned}
%&\grad\,H(\Xkh|\Yk)\\
%=&\grad\, F(\Xkh) + \QQ_{\Xkh}\big((\Xkh-\Yk)\lnGammak\big)+\\
%&\QQ_{\Xkh}\big(\nabla F(\Yk)-\nabla F(\Xkh)\big).
%\end{aligned}
%\end{equation}
%In addition, \cref{eq::minHY} results in
%\begin{equation}\label{eq::gradHY2}
%\grad\,H(\Xkh|\Yk)=\0.
%\end{equation} 
%As a result of \cref{eq::gradHY1,eq::gradHY2}, it can be concluded that
%\begin{multline}
%	\nonumber
%	\grad\, F(\Xkh)=\QQ_{\Xkh}\big((\Yk-\Xkh)\lnGammak\big)+\\
%	\QQ_{\Xkh}\big(\nabla F(\Xkh)-\nabla F(\Yk)\big).
%\end{multline}
%From the equation above, there exists a constant scalar $\nu > 0$ such that
%\begin{equation}\label{eq::gradnFkh}
%	\|\grad\, F(\Xkh)\|\leq \nu \|\Xkh-\Yk\|,
%\end{equation} 
%whose derivation is almost the same as that of \cref{eq::gradFh}. 
