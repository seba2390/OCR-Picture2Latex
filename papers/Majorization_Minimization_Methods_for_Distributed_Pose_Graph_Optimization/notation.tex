

\textbf{Miscellaneous Sets.}\; $\R$ denotes the sets of real numbers; $\R^+$ denotes the sets of nonnegative real numbers; $\R^{m\times n}$ and $\R^n$ denote the sets of $m\times n$ matrices and $n\times 1$ vectors, respectively. $SO(d)$ denotes the set of special orthogonal groups and  $SE(d)$ denotes the set of special Euclidean groups.  $|\cdot|$ denotes the cardinality of a set.

\textbf{Matrices.}\; For a matrix $X\in \R^{m\times n}$, $[X]_{ij}$ denotes the $(i,\,j)$-th entry or $(i,\,j)$-th block of $X$, and $[X]_i$ denotes the $i$-th entry or $i$-th block of $X$. For symmetric matrices $X,\, Y\in \R^{n\times n}$, $X\succeq Y$ (or $Y\preceq X$) and $X\succ Y$ (or $Y\prec X$) mean that $X-Y$ is positive (or negative) semidefinite and definite, respectively.

\textbf{Inner Products and Norms.}\; For a matrix $M\in\R^{n\times n}$, $\innprod{\cdot}{\cdot}_{M}:\R^{m\times n}\times \R^{m\times n}\rightarrow \R$ denotes the function
\vspace{-0.25em}
\begin{equation}\label{eq::innerM}
	\innprod{X}{Y}_M\triangleq \trace(XMY^\transpose)
\end{equation}
where $X,\,Y\in \R^{m\times n}$. If $M$ is the identity matrix, $\innprod{\cdot}{\cdot}_M$ is also represented as $\innprod{\cdot}{\cdot}:\R^{m\times n}\times\R^{m\times n}\rightarrow \R$ such that
\vspace{-0.15em}
\begin{equation}\label{eq::inner}
\innprod{X}{Y}\triangleq\trace(XY^\transpose).
\vspace{-0.1em}
\end{equation}
For a positive semidefinite matrix $M\in\R^{n\times n}$, $\|\cdot\|_M:\R^{m\times n}\rightarrow\R^+$ denotes the function
\vspace{-0.15em}
\begin{equation}\label{eq::normM}
\|X\|_M\triangleq\sqrt{\trace(X M X^\transpose)}
\vspace{-0.1em}
\end{equation}
where $X\in \R^{m\times n}$.  Also, $\|\cdot\|$ denotes the Frobenius norm of matrices and vectors, and $\|\cdot\|_2$ denotes the induced $2$-norms of matrices and linear operators.  

\textbf{Riemannian Geometry.}\; If $F(\cdot):\R^{m\times n}\rightarrow\R $ is a function, $\mathcal{M}\subset \R^{m\times n}$ is a Riemannian manifold and $X\in \mathcal{M}$, then {\highlight $\nabla F(X)$ and $\mathrm{grad}\, F(X)$ denote the Euclidean and Riemannian gradients, respectively. }
\vspace{0.15em}

\text{\textbf{Graph Theory.}}\quad  PGO is represented as a directed graph $\aGG=(\VV,\,\aEE)$ where $\VV$ and $\EE$ are the sets of vertices and edges, respectively \cite{rosen2016se}. In distributed PGO, each vertex is described as  an ordered pair $(\alpha,\,i)\in\VV$ where $\alpha$ is the node index and $i$ the local index of the vertex within node $\alpha$. For any  nodes $\alpha$ and $\beta$ in distributed PGO, $\aEE^{\ab}$ denotes the set of edges between nodes $\alpha$ and $\beta$: 
\vspace{-0.15em}
\begin{equation}\label{eq::aEE}
	\aEE^{\ab}\triangleq\{(i,\,j)|((\alpha,\,i),\,(\beta,\,j))\in\aEE\};
\vspace{-0.15em}
\end{equation} 
and $\NN_-^\alpha$  denotes the set of nodes with edges from node $\alpha$:
\vspace{-0.15em}
\begin{equation}\label{eq::NN-}
	\NN_-^\alpha\triangleq\{\beta|\aEE^{\ab}\neq \emptyset\text{ and }\alpha\neq\beta\};
\vspace{-0.15em}
\end{equation} 
and $\NN_+^\alpha$ denotes  the set of nodes with edges to node $\alpha$:
\vspace{-0.15em}
\begin{equation}\label{eq::NN+}
	\NN_+^\alpha\triangleq\{\beta|\aEE^{\ba}\neq \emptyset\text{ and }\alpha\neq\beta\};
\vspace{-0.15em}
\end{equation} 
and $\NN^\alpha$ denotes the set of nodes with edges  from or to node $\alpha$:
\vspace{-0.15em}
\begin{equation}\label{eq::NN}
\!	\NN^\alpha\!\triangleq \NN_-^\alpha \cup \NN_+^\alpha \triangleq \{\beta|\aEE^{\ab}\!\neq\! \emptyset\text{ or }\! \aEE^{\ba}\!\neq\! \emptyset\text{ and }\alpha \!\neq\! \beta\}. 
\vspace{-0.1em}
\end{equation}

\textbf{Optimization.}\quad For optimization variables $X$, $X^\alpha$, $R^\alpha$, $t^\alpha$, etc., the notation $\Xk$, $X^{\ak}$, $R^{\ak}$, $t^{\ak}$, etc. denotes the $\sk$-th iterate of corresponding optimization variables.