Substituting \cref{eq::lnGammaki} into \cref{eq::xlGi} and rewriting the resulting equation in terms of $\Xa_i=\begin{bmatrix}
t_i^\alpha & R_i^\alpha
\end{bmatrix}\in\R^{d\times(d+1)}$ leads to
\begin{multline}
\nonumber
\big(t_i^{\akh},\,R_i^{\akh}\big)=\\
\arg\min_{\substack{t_i^\alpha\in\R^d,\, R_i^\alpha\in SO(d)}}\; \frac{1}{2} \|\tai-\taik\|_{\lnGammataik}^2 +\\
\lnGammanaik(\Rai-\Raik)^\transpose(\tai-\taik) +\\
\frac{1}{2}\|\Rai-\Raik\|_{\lnGammaRaik}^2+\\
\innprod{\nabla_{\tai}F(\Xk)}{\tai-\taik} +\\ \innprod{\nabla_{\Rai}F(\Xk)}{\Rai-\Raik}.
\end{multline}
For notational simplicity, the equation above is simplified to
\begin{multline}\label{eq::opti}
\min_{\substack{t_i^\alpha\in\R^d,\, R_i^\alpha\in SO(d)}}\; \frac{1}{2} \|\tai\|_{\lnGammataik}^2 +
\lnGammanaik{\Rai}^\transpose\tai +\\
\frac{1}{2}\|\Rai\|_{\lnGammaRaik}^2+  \innprod{\gamma_{i}^{\tau,\ak}}{\tai} + \innprod{\gamma_{i}^{\kappa,\ak}}{\Rai},
\end{multline}
in which
\begin{equation}
\gamma_{i}^{\tau,\ak}=\nabla_{\tai}F(\Xk)-\taik\lnGammataik\in\R^d
\end{equation}
and
\begin{equation}
\gamma_{i}^{\kappa,\ak}=\nabla_{\Rai}F(\Xk)-\Raik\lnGammaRaik\in\R^{\d\times d}.
\end{equation}
Recalling from \cref{eq::lnGammaki} that $\lnGammataik\in\R$ and $\lnGammataik>0$,  we obtain that
\begin{equation}\label{eq::tsol}
\tai = -\Rai{\lnGammanaik}^\transpose{\lnGammataik}^{-1}-\gamma_{i}^{\tau,\ak}{\lnGammataik}^{-1}
\end{equation}
minimizes \cref{eq::opti} if $\Rai\in SO(d)$ is given. Then, substituting \cref{eq::tsol} into \cref{eq::opti} yields
\begin{equation}\label{eq::opti2}
R_i^{\akh}=\arg\min_{\Rai\in SO(d)} \frac{1}{2}\|\Rai\|_{\Xi_i^{\ak}}^2 - \innprod{\upsilon_i^{\ak}}{\Rai},
\end{equation}
in which
\begin{equation}
\Xi_i^{\ak}=\lnGammaRaik-{\lnGammanaik}^\transpose{\lnGammataik}^{-1}\lnGammanaik\in\R^{d\times d}
\end{equation}
and
\begin{equation}
\upsilon_i^{\ak} = \gamma_{i}^{\tau,\ak}{\lnGammataik}^{-1}\lnGammanaik - \gamma_{i}^{\kappa,\ak} \in\R^{d\times d}.
\end{equation}
If we apply ${\Rai}^\transpose\Rai=\I$ on \cref{eq::opti2}, then \cref{eq::xlGi} is equivalent to
\begin{equation}\label{eq::opti4}
R_i^{\akh} = \arg\max_{\Rai\in SO(d)} \innprod{\upsilon_i^{\ak}}{\Rai}.
\end{equation} 
Thus, \cref{eq::xlGi} is reduced to an optimization problem on $SO(d)$, which has a closed-form solution as follows.

Following \cite{umeyama1991least}, if $\upsilon_i^{\ak}\in \R^{d\times d}$ admits a singular value decomposition $\upsilon_i^{\ak}=U_i^{\ak}\mathrm{\Sigma}_i^{\ak} {V_i^{\ak}}^\transpose$ in which $U_i^{\ak}$ and $V_i^{\ak}\in O(d)$ are orthogonal (not necessarily special orthogonal) matrices, and $\mathrm{\Sigma}_i^{\ak}=\diag\{\sigma_1^{\ak},\,\sigma_2^{\ak},\,\cdots,\,\sigma_d^{\ak}\}\in \R^{d\times d}$ is a diagonal matrix, and $\sigma_1^{\ak}\geq\sigma_2^{\ak}\geq\cdots\geq \sigma_d^{\ak}\geq 0$ are singular values of $\upsilon_i^{\ak}$, then the optimal solution to \cref{eq::opti4} is
\begin{equation}\label{eq::Rsol}
	R_i^{\akh}=\begin{cases}
		U_i^{\ak}\mathrm{\Lambda}^+ {V_i^{\ak}}^\transpose, & \det\big(U_i^{\ak}{V_i^{\ak}}^\transpose\big) > 0,\\[0.3em]
		U_i^{\ak}\mathrm{\Lambda}^- {V_i^{\ak}}^\transpose, & \text{otherwise},
	\end{cases}
	\vspace{-0.25em}
\end{equation}
in which $\mathrm{\Lambda}^+=\diag\{1,\,1,\,\cdots,\,1\}\in\R^{d\times d}$ and $\mathrm{\Lambda}^-=\diag\{1,\,1,\,\cdots,\,-1\}\in\R^{d\times d}$. If $d = 2$, the equation above is equivalent to the polar decomposition of $2\times 2$ matrices, and if $d=3$, there are fast algorithms for singular value decomposition of $3\times 3$ matrices \cite{mcadams2011computing}. As a result, \cref{eq::opti4} can be efficiently solved in the case of $SO(2)$ and $SO(3)$, both of which are commonly used in SLAM.

As long as $R_i^{\akh}\in SO(d)$ is known, $t_i^{\akh}\in\R^d$ can be exactly recovered using \cref{eq::tsol}:
\begin{multline}
t_i^{\akh} = -R_i^{\akh}{\lnGammanaik}^\transpose{\lnGammataik}^{-1}-\\ \gamma_{i}^{\tau,\ak}{\lnGammataik}^{-1}.
\end{multline}
Therefore, $X_i^{\akh}=\begin{bmatrix}
t_i^{\akh} & R_i^{\akh}
\end{bmatrix}\in \R^{d\times(d+1)}$ is exactly solved, whose computation only involves matrix multiplication and singular value decomposition.
