\thispagestyle{empty}
\selectlanguage{ngerman}

\chapter*{Zusammenfassung}
Bei der klassischen Planung besteht das Ziel darin, einen Handlungsablauf zu finden, der es einem intelligenten Agenten erm{\"o}glicht, aus jeder Situation, in der er sich befindet, in eine Situation zu gelangen, die seine Ziele erf{\"u}llt.
Die klassische Planung gilt als dom{\"a}nenunabh{\"a}ngig, d.h. sie ist nicht auf eine bestimmte Anwendung beschr{\"a}nkt und kann zur L{\"o}sung verschiedener Arten von Logikproblemen verwendet werden.
In der Praxis erfordern jedoch einige Eigenschaften eines vorliegenden Planungsproblems eine ausdrucksstarke Erweiterung des klassischen Standardplanungsformalismus, um sie zu erfassen und zu modellieren. 
Obwohl die Bedeutung vieler dieser Erweiterungen bekannt ist, unterst{\"u}tzen die meisten Planer, insbesondere optimale Planer, diese erweiterten Planungsformalismen nicht. 
Die fehlende Unterst{\"u}tzung schr{\"a}nkt nicht nur die Verwendung dieser Planer f{\"u}r bestimmte Probleme ein, sondern selbst wenn es m{\"o}glich ist, die Probleme ohne diese Erweiterungen zu modellieren, f{\"u}hrt dies oft zu einem erh{\"o}hten Aufwand bei der Modellierung oder macht die Modellierung praktisch unm{\"o}glich, da die erforderliche Problemkodierungsgr{\"o}{\ss}e exponentiell ansteigt.

In dieser Arbeit schlagen wir vor, die symbolische Suche f{\"u}r kostenoptimale Planung f{\"u}r verschiedene ausdrucksstarke Erweiterungen der klassischen Planung zu verwenden, die alle unterschiedliche Aspekte des Problems erfassen.
Insbesondere untersuchen wir die Planung mit Axiomen, die Planung mit zustandsabh{\"a}ngigen Aktionskosten, die \say{Oversubscription Planung} und die \say{Top-$k$ Planung}. 
F{\"u}r alle Formalismen pr{\"a}sentieren wir Ergebnisse zur Komplexit{\"a}t und Kompilierbarkeit, wobei wir hervorheben, dass es w{\"u}nschenswert und sogar notwendig ist, die entsprechenden Aspekte nativ zu unterst{\"u}tzen.
Wir analysieren die symbolische heuristische Suche und zeigen, dass die Suchleistung nicht immer von der Verwendung einer Heuristik profitiert und dass die Suchleistung selbst unter den bestm{\"o}glichen Umst{\"a}nden, n{\"a}mlich der perfekten Heuristik, exponentiell abnehmen kann.
Dies unterstreicht, dass die symbolische blinde Suche heutzutage die dominierende symbolische Suchstrategie ist, {\"a}nlich gut wie anderen modernen kostenoptimalen Planungsstrategien.
Basierend auf dieser Beobachtung und dem Mangel an guten Heuristiken f{\"u}r Planungsformalismen mit ausdrucksstarken Erweiterungen, erweist sich die symbolische Suche als ein starker Ansatz.
Wir erweitern die symbolische Suche, um jeden der Formalismen einzeln und in Kombination zu unterst{\"u}tzen, was zu optimalen, korrekten und vollst{\"a}ndigen Planungsalgorithmen f{\"u}hrt, die sich im Vergleich zu anderen Ans{\"a}tzen empirisch besser verhalten.
\selectlanguage{english}

