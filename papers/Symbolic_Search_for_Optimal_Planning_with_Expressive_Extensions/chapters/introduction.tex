\chapter{Introduction}
\chapterquote{If you fail to plan, you are planning to fail.}{Benjamin Franklin}

Automated planning is the science of designing and engineering machines, in particular computer programs, that can automatically derive behaviors to achieve goals.
The generation of such strategies, i.e., thinking before acting, is understood as an intelligent behavior and is one of the oldest areas in the field of Artificial Intelligence.
Nowadays, a classical planning problem can be informally described as the problem of finding a course of actions that allows an intelligent agent to move from any situation it finds itself in to one that satisfies its goals.
Since planning is not limited to a specific application domain, it was originally labeled as general problem solving \autocite{helmert-2008,newell-simon-cat1963} and can be used for different types of reasoning problems, such as elevator control \autocite{koehler-schuster-aips2000}, greenhouse logistics \autocite{helmert-lasinger-icaps2010}, natural-language generation \autocite{koller-hoffmann-icaps2010}, robot control \autocite{nilsson-tr1984,speck-et-al-ral2017,karpas-magazzeni-2020}, risk management \autocite{sohrabi-et-al-aaai2018,katz-et-al-icaps2021wsfinplan}, and many more.

\setboolean{kiviatBlue}{true}
\setboolean{kiviatRed}{false}
\renewcommand{\kiviatTopk}{1}
\renewcommand{\kiviatGoal}{1}
\renewcommand{\kiviatCost}{1}
\renewcommand{\kiviatPredicates}{1}
\begin{figure}[t]
    \begin{center}
        \begin{tikzpicture}
    \tkzKiviatDiagram[
    radial style/.style ={-{Latex[length=3mm, width=2mm]}},
    scale=0.85,
    label space=1.5,
    radial = 1,
    gap = 1.5,
    step = 1,
    lattice = 2]{
    \hspace{2.5cm}\mbox{\parbox{4cm}{{\begin{center}\textbf{Goal}\\\textbf{Description} \end{center}}}},
    {\textbf{Number of Plans}},
    \hspace{-1.5cm}\mbox{\parbox{3cm}{{\begin{center}\textbf{State}\\\textbf{Description}\end{center}}}},
    \textbf{Cost Function}}

    \ifbool{kiviatRed}{
        \tkzKiviatLine[very thick,color=red!50,
            fill=red!30,
            opacity=.35,
            mark=ball,
            mark size=2.5pt,
            ball color=red](\kiviatGoal{},\kiviatTopk{},\kiviatPredicates{},\kiviatCost{})
    }

    \ifbool{kiviatBlue}{
        \tkzKiviatLine[very thick,color=blue!50,
            fill=blue!30,
            opacity=.35,
            mark=ball,
            mark size=2.5pt,
            ball color=blue](1,1,1,1)
    }

    \ifbool{kiviatFull}{
    \tkzKiviatLine[very thick,color=darkgreen!50,
        fill=green!30,
        opacity=.35,
        mark=ball,
        mark size=2.5pt,
        ball color=darkgreen](1,2,2,2)
    \tkzKiviatLine[very thick,color=blue!50,
        fill=blue!30,
        opacity=.35,
        mark=ball,
        mark size=2.5pt,
        ball color=blue](1,1,1,1)
    \tkzKiviatLine[very thick,color=red!50,
        %fill=red!30,
        %opacity=.35,
        mark=ball,
        mark size=2.5pt,
        ball color=red](\kiviatGoal{},\kiviatTopk{},\kiviatPredicates{},\kiviatCost{})
    \LegendBox[shift={(-0cm,-2.25cm)}]{current bounding box.south west}%
    {
    red!100/{Forward symbolic search (contribution of this thesis)},
    asparagus!100/{Bidirectional symbolic search (contribution of this thesis)},
    blue!100/{Bidirectional symbolic search (previous state of the art)}}
    }
    {}

    % goal specification
    \draw[] node[align=center,fill=none] at (1.5,-0.5) {\small Hard};
    \draw[] node[align=center,fill=none] at (3.25,-0.75) {\small Oversub-\\ \small scription};
    %\draw[] node[] at (1.75,-0.6) {\rotatebox{-45}{hard}};
    %\draw[] node[align=center] at (3.85,-1.25) {\rotatebox{-45}{oversubscription}};

    % number of plans
    \draw[] node[align=center,fill=none] at (0.7,1.5) {\small Top-$1$};
    \draw[] node[align=center,fill=none] at (0.7,3.0) {\small Top-$k$};

    % derived predicates
    \draw[] node[align=center,fill=none] at (-1.25,0.75) {\small State\\ \small Variables};
    \draw[] node[align=center,fill=none] at (-3.25,0.75) {\small $+$ \small Derived\\ \small Variables};

    % cost function
    \draw[] node[align=center,fill=none] at (-1,-1.5) {\small Constant};
    \draw[] node[align=center,fill=none] at (-1.6,-3.0) {\small State-Dependent\phantom{0}};
\end{tikzpicture}
    \end{center}
    \caption[Overview of extension for classical planning.]{
        Overview of extensions for classical planning, where the blue color denotes the planning formalism supported by symbolic search approaches from the literature.
        %Overview of extensions for classical planning, where the red color indicates which planning formalism is supported by symbolic search approaches from the literature.
        %Overview of extensions for classical planning, where the red color denotes  the subset of supported extensions supported by symbolic search from the literature.
    }\label{fig:intro:kiviat}
\end{figure}
\renewcommand{\kiviatTopk}{1}
\renewcommand{\kiviatGoal}{1}
\renewcommand{\kiviatCost}{1}
\renewcommand{\kiviatPredicates}{1}
\setboolean{kiviatBlue}{false}
\setboolean{kiviatRed}{true}

In classical planning, some properties of a planning problem at hand require an expressive extension of the standard classical planning formalism to capture them in a concise way.
In this thesis, we consider four different extensions of classical planning (\Cref{fig:intro:kiviat}), all of which capture and extend different aspects of classical planning while retaining the core of the formalism: Single-agent planning problems in a fully observable, static, discrete, deterministic, and fully known environment \autocite{russell-norvig-2003}.
Unfortunately, most classical planners do not support any of these expressive extensions because they are usually based on heuristic search and it appears very challenging to design informative and fast to compute heuristics (goal-distance estimators) that consider additional problem properties.
This is especially true for cost-optimal planners, which additionally require that a heuristic is admissible, i.e., that the heuristic never overestimates the cost of reaching a goal state.
However, it is well known that such extensions are critical for an elegant and compact modeling of many real-world problems.
For example, extending the state description to include derived variables allows aspects of a planning problem that are not directly affected by the actions but are derived from the values of other variables to be modeled concisely using a set of logical axioms \autocite{thiebaux-et-al-aij2005}.
While in probabilistic planning a concise encoding of action costs or rewards in form of Markov decision processes has long been standard, in classical planning it is common to consider a potentially exponentially larger representation with constant action costs \autocite{geisser-phd2018}.
In many real-world applications, the description of goals is oversubscribed, i.e., there are a large number of desirable, often competing goals of varying value, and a system may not be able to achieve all of them with the available resources \autocite{smith-icaps2004}.
Unfortunately, conventional classical planning cannot handle such a scenario, because the goal is treated as a hard constraint that can either be accomplished or not.
Finally, in some cases, a single plan may be sufficient, but in practice it is often better to have many good alternatives to choose from in order to be more flexible \autocite{nguyen-et-al-aij2012}.

In this thesis, we investigate and discuss the computational complexity and compilability of classical planning incorporating the four expressive extensions mentioned earlier.
This analysis shows and highlights that for many real-world problems, it is desirable and even necessary to support these features natively.
We propose for the first time to use symbolic search for cost-optimal planning for those four expressive extensions (see \Cref{fig:intro:kiviat}).
Symbolic search provides a good basis for native support of these planning formalisms for several reasons.
A theoretical and empirical analysis of symbolic heuristic search in the form of \bddastar{} shows that the use of a heuristic in symbolic search does not always improve search performance.
This observation strengthens the fact that nowadays symbolic blind search, i.e., without heuristics, is the dominant search strategy of symbolic search, on par with explicit heuristic search in cost-optimal planning.
Thus, since symbolic search does not necessarily need heuristics to be efficient, the search efficiency does not suffer from the lack of efficient and informative heuristics for the proposed planning scenarios.
Based on these observations, we enhance symbolic search to obtain optimal, sound, and complete algorithms for planning with the expressive extensions under consideration.
Our empirical evaluations show that the presented symbolic search approaches perform favorably in all these planning settings compared with other state-of-the-art approaches.
Finally, we show that the proposed symbolic search approach is able to support planning tasks that use and require all expressive extensions at once.

\section{Outline}
This thesis is structured as follows.
%
In \Cref{ch:background}, we introduce the relevant background for this thesis concerning classical planning. In particular, we introduce the concepts of complexity, compilability and expressive power and present symbolic search for classical planning
together with different types of decision diagrams.
%
In \Cref{ch:symbolic_heuristic_search}, we investigate the question, why heuristics do not seem to pay off in symbolic search.
Specifically, we theoretically and empirically analyze the search behavior of symbolic heuristic search in form of \bddastar{} \autocite{speck-et-al-icaps2020}.
%
In \Cref{ch:axioms}, we summarize computational complexity and compilability results for planning with axioms from the literature.
We introduce three ways to extend symbolic search algorithms to support axioms natively and present an empirical evaluation \autocite{speck-et-al-icaps2019}.
%
In \Cref{ch:sdac}, we present computational complexity and compilability results for planning with state-dependent action costs \autocite{speck-et-al-icaps2021}.
Then, symbolic search algorithms for planning with state-dependent action costs are presented and an empirical evaluation is conducted \autocite{speck-et-al-icaps2018}.
%
In \Cref{ch:osp}, we discuss computational complexity and compilability results for oversubscription planning.
A symbolic search approach is presented and explained for oversubscription planning and an empirical evaluation is conducted \autocite{speck-katz-aaai2021}.
%
In \Cref{ch:topk}, we present computational complexity and compilability results for top-$k$ planning and introduce symbolic search for it, which we analyze theoretically and empirically \autocite{speck-et-al-aaai2020}.
%
In \Cref{ch:discussion}, we discuss the combination of the previously analyzed extensions for classical planning and the use of symbolic search for such a setting.
Finally, possible future work related to this thesis is
discussed.
%
\Cref{ch:conclusion} concludes and summarizes this thesis.

\section{Published Work}
The core results presented in this work have been published at leading AI and automated planning conferences.
In particular, the following publications form the backbone of this work.
At the beginning of each chapter, we indicate which publications form the basis for the content of the chapter.
\ifbool{attachPDFs}{
    Each of these core papers is attached in its publication form at the end of the thesis.
}

\renewcommand{\citebf}[1]{\textbf{#1}}
\begin{itemize}
    \item \renewcommand{\citeaddendum}{\\\textbf{(Best Student Paper Runner-Up Award)}}\fullcite{speck-et-al-icaps2021}\renewcommand{\citeaddendum}{}
    \item \fullcite{speck-katz-aaai2021}
    \item \fullcite{speck-et-al-icaps2020}
    \item \fullcite{speck-et-al-aaai2020}
    \item \fullcite{speck-et-al-icaps2019}
    \item \renewcommand{\citeaddendum}{\\\textbf{(Partly based on ideas from my master thesis)}}\fullcite{speck-et-al-icaps2018}\renewcommand{\citeaddendum}{}
\end{itemize}
\renewcommand{\citebf}[1]{#1}

The following publications also resulted from my doctoral research, but are not a central part of this thesis.
However, some of the ideas and concepts presented in this thesis were used in these research papers.

\renewcommand{\citebf}[1]{\textbf{#1}}
\begin{itemize}
    \item \fullcite{speck-et-al-icaps2021b}
          \begin{itemize}
              \item \renewcommand{\citeaddendum}{\textbf{ (superseded)}}\fullcite{speck-et-al-icaps2020wsprl}\renewcommand{\citeaddendum}{}
                    %\item \renewcommand{\citeaddendum}{\textbf{ (superseded)}}\fullcite{speck-et-al-arxiv2020}\renewcommand{\citeaddendum}{}
          \end{itemize}
    \item \fullcite{drexler-et-al-icaps2021}
          \begin{itemize}
              \item \renewcommand{\citeaddendum}{\textbf{ (superseded)}}\fullcite{drexler-et-al-icaps2020wshsdip}\renewcommand{\citeaddendum}{}
          \end{itemize}
    \item \fullcite{behnke-speck-aaai2021}
    \item \fullcite{geisser-et-al-socs2020}
    \item \fullcite{corraya-et-al-ki2019}
    \item \fullcite{geisser-et-al-icaps2019wsipc}
    \item \fullcite{speck-et-al-ipc2018}
    \item \renewcommand{\citeaddendum}{\textbf{ (Winner of the Discrete Markov Decision Process Track)}}\fullcite{geisser-speck-ippc2018}\renewcommand{\citeaddendum}{}
\end{itemize}
\renewcommand{\citebf}[1]{#1}

Finally, the following patents were filed based on work conducted during my doctoral process.

\renewcommand{\citebf}[1]{\textbf{#1}}
\begin{itemize}
    \item \fullcite{speck-et-al-patentus2021}
    \item \fullcite{speck-et-al-patenteu2020}
\end{itemize}
\renewcommand{\citebf}[1]{#1}

\section{Awards}
Together with my colleagues, I received the following awards for my work during my doctoral process.

\renewcommand{\citebf}[1]{\textbf{#1}}
\begin{itemize}
    \item \textbf{Best Student Paper Runner-Up Award} for the paper \say{On the Compilability and Expressive Power of State-Dependent Action Costs} \autocite{speck-et-al-icaps2021} at ICAPS 2021
    \item \textbf{Winner of the Discrete Markov Decision Process Track} for the planning system \say{Prost-DD} \autocite{geisser-speck-ippc2018} at the Sixth International Probabilistic Planning Competition at ICAPS 2018
\end{itemize}
\renewcommand{\citebf}[1]{#1}
