\chapter{Top-k Planning}\label{ch:topk}
\chapterquote{If plan A fails, remember there are 25 more letters.}{Chris Guillebeau}
%\chapterquote{If plan A doesn't work, the alphabet has 25 more letters -- 204 if you're in Japan.}{Claire Cook, Seven Year Switch}

\renewcommand{\kiviatTopk}{2}
\begin{figure}[t]
    \begin{center}
        \begin{tikzpicture}
    \tkzKiviatDiagram[
    radial style/.style ={-{Latex[length=3mm, width=2mm]}},
    scale=0.85,
    label space=1.5,
    radial = 1,
    gap = 1.5,
    step = 1,
    lattice = 2]{
    \hspace{2.5cm}\mbox{\parbox{4cm}{{\begin{center}\textbf{Goal}\\\textbf{Description} \end{center}}}},
    {\textbf{Number of Plans}},
    \hspace{-1.5cm}\mbox{\parbox{3cm}{{\begin{center}\textbf{State}\\\textbf{Description}\end{center}}}},
    \textbf{Cost Function}}

    \ifbool{kiviatRed}{
        \tkzKiviatLine[very thick,color=red!50,
            fill=red!30,
            opacity=.35,
            mark=ball,
            mark size=2.5pt,
            ball color=red](\kiviatGoal{},\kiviatTopk{},\kiviatPredicates{},\kiviatCost{})
    }

    \ifbool{kiviatBlue}{
        \tkzKiviatLine[very thick,color=blue!50,
            fill=blue!30,
            opacity=.35,
            mark=ball,
            mark size=2.5pt,
            ball color=blue](1,1,1,1)
    }

    \ifbool{kiviatFull}{
    \tkzKiviatLine[very thick,color=darkgreen!50,
        fill=green!30,
        opacity=.35,
        mark=ball,
        mark size=2.5pt,
        ball color=darkgreen](1,2,2,2)
    \tkzKiviatLine[very thick,color=blue!50,
        fill=blue!30,
        opacity=.35,
        mark=ball,
        mark size=2.5pt,
        ball color=blue](1,1,1,1)
    \tkzKiviatLine[very thick,color=red!50,
        %fill=red!30,
        %opacity=.35,
        mark=ball,
        mark size=2.5pt,
        ball color=red](\kiviatGoal{},\kiviatTopk{},\kiviatPredicates{},\kiviatCost{})
    \LegendBox[shift={(-0cm,-2.25cm)}]{current bounding box.south west}%
    {
    red!100/{Forward symbolic search (contribution of this thesis)},
    asparagus!100/{Bidirectional symbolic search (contribution of this thesis)},
    blue!100/{Bidirectional symbolic search (previous state of the art)}}
    }
    {}

    % goal specification
    \draw[] node[align=center,fill=none] at (1.5,-0.5) {\small Hard};
    \draw[] node[align=center,fill=none] at (3.25,-0.75) {\small Oversub-\\ \small scription};
    %\draw[] node[] at (1.75,-0.6) {\rotatebox{-45}{hard}};
    %\draw[] node[align=center] at (3.85,-1.25) {\rotatebox{-45}{oversubscription}};

    % number of plans
    \draw[] node[align=center,fill=none] at (0.7,1.5) {\small Top-$1$};
    \draw[] node[align=center,fill=none] at (0.7,3.0) {\small Top-$k$};

    % derived predicates
    \draw[] node[align=center,fill=none] at (-1.25,0.75) {\small State\\ \small Variables};
    \draw[] node[align=center,fill=none] at (-3.25,0.75) {\small $+$ \small Derived\\ \small Variables};

    % cost function
    \draw[] node[align=center,fill=none] at (-1,-1.5) {\small Constant};
    \draw[] node[align=center,fill=none] at (-1.6,-3.0) {\small State-Dependent\phantom{0}};
\end{tikzpicture}
    \end{center}
    \caption[Overview of extension for classical planning (top-$k$).]{Overview of extensions for classical planning, where the red color denotes the planning formalism supported by the proposed symbolic search approach.}\label{fig:top_k:kiviat}
\end{figure}
\renewcommand{\kiviatTopk}{1}

\section*{Core Publication of this Chapter}
\renewcommand{\citebf}[1]{\textbf{#1}}
\begin{itemize}
    \item \fullcite{speck-et-al-aaai2020}
\end{itemize}
\renewcommand{\citebf}[1]{#1}

In planning, it is common to assume that the model is fully specified and that the objective is to find a single plan.
While in some cases a single plan may be sufficient, in practice it is often better to have many good alternative plans \autocite{nguyen-et-al-aij2012}.
The possibility of obtaining multiple high-quality plans can be of great importance in various fields and applications, such as task and motion planning \autocite{lavalle-2006,ren-et-al-arxiv2021,lozano-et-al-iros2014}, diverse planning \autocite{katz-et-al-aaai2020}, scenario planning \autocite{sohrabi-et-al-aaai2018}, goal recognition \autocite{sohrabi-et-al-ijcai2016} or planning with user preferences \autocite{nguyen-et-al-aij2012,seimetz-et-al-ijcai2021}.

The problem of determining $k$ shortest paths for a given graph is a well-studied topic, going back to 1957 \autocite{bock-et-al-1957}.
However, the problem of determining $k$ cheapest plans, known as top-$k$ planning, has only been studied more recently \autocite{riabov-et-al-icaps2014wsspark}.
In addition to generating a set of possible solutions for all the above mentioned applications, the systematic enumeration of plans in an anytime behavior enables the realization of a \say{generate-and-test framework} by searching for plans that must satisfy certain complex requirements.
Thus, top-$k$ planning can also be used to search in a simplified version of a given problem, e.g., in an overapproximated, abstracted, or decomposed model representation \autocite{hoeller-icaps2021}.

In this chapter, we define and motivate top-$k$ planning (\Cref{fig:top_k:kiviat}) before presenting the work of \textcite{speck-et-al-aaai2020} on this topic.
On the theoretical side, we discuss the question of whether top-$k$ planning is computationally more difficult than ordinary planning.
On the practical side, we describe a sound and complete symbolic search approach to top-$k$ planning.
The empirical evaluation shows that symbolic search performs better than other state-of-the-art search methods for both a small and a large number of desired plans $k$.

\section{Formalism}

The objective of top-$k$ planning is to determine a set of $k$ different plans with lowest cost for a given classical planning task (\Cref{def:planning-task}).
We formally define top-$k$ planning as follows \autocite{riabov-et-al-icaps2014wsspark,katz-et-al-icaps2018,speck-et-al-aaai2020}.

\begin{definition}[Top-k Planning]\label{def:top-k-planning}
    Given a planning task $\task$ and a natural number $k \in \mathbb{N} \cup \{ \infty \}$, top-$k$ planning is the problem of determining a set of plans $P \subseteq P_\task$ such that:
    \begin{enumerate}
        \item there exists no plan $\plan' \in P_\task$ with $\plan' \not\in P$ that is cheaper than some plan $\plan \in P$, and
        \item $|P| = k$ if $|P_\task| \geq k$, and $|P| = |P_\task{}|$, otherwise.
    \end{enumerate}
\end{definition}

We emphasize that top-$k$ planning is a generalization of classical planning, where $k=1$ corresponds to cost-optimal classical planning.


There can be infinitely many plans, since according to \Cref{def:top-k-planning} plans with cycles, i.e., plans which visit the same state multiple times, are allowed.
Moreover, there can be infinitely many plans with optimal costs if cycles with $0$ costs exist.
In general, top-$k$ planners not only provide a way to generate the $k$ best plans, but also to enumerate all possible plans if no $0$-actions or, more precisely, no $0$-cost cycles exist.
Depending on the application, plans with cycles can be important to provide all possibilities, e.g., when user preferences are not fully known or the model specification is not complete.
In addition, the sequential generation of all plans allows the realization of complete search strategies that search in a simplified version of a problem at hand \autocite{hoeller-icaps2021}.
The symbolic approach presented in this chapter can be modified to generate only simple plans, i.e., plans without cycles \autocite{vontschammer-et-al-icaps2022}.
However, we will focus on \say{ordinary} top-$k$ planning, which includes all possible plans.
\Cref{ex:top_k} illustrates the idea of top-$k$ planning when user preferences are not fully known and the model is incomplete.

\begin{example}\label{ex:top_k}
    Consider \Cref{ex:rover}, in which we identified an optimal plan where the rover first navigates to (7,1) before the drone flies and takes an image at (6,1) and then at (10,1).
    Finally, the rover equipped with the drone navigates to (0,5).
    A closer look reveals that the order in which the drone takes the images at the two desired locations is irrelevant to the overall cost of the plan.
    With top-$k$ planning we can determine both plans, one where the drone flies first to (6,1) and then to (10,1) and vice versa.
    Based on these alternatives, it can be decided at which of the two locations the images should be taken first, depending on what the user prefers.
    In addition, there are various plans with different navigation routes of the rover, which allow to choose a different path during the execution of the plan without replanning and thus to react to unforeseen events that were not considered in the model.
\end{example}


\section{Complexity and Compilability}
Clearly, top-$k$ planning is as hard as classical planning, since it is a straightforward generalization.
However, the question arises to what extend it is harder to determine multiple plans instead of a single one.
Unlike the previously discussed extension to classical planning, top-$k$ planning is not a direct model extension.
The concept of compilability cannot (readily) be applied here, since the output changes from a single plan to a set of plans, which raises a conceptually different decision problem.
Thus, \textcite{speck-et-al-aaai2020} examined the computational complexity of top-$k$ planning and defined the \emph{bounded top-$k$ plans existence} problem (\Cref{def:bounded-top-k-plan-existence}) to answer the question of whether top-$k$ planning is as difficult as classical planning.

\begin{definition}[Bounded Top-k Plans Existence]\label{def:bounded-top-k-plan-existence}
    \emph{Bounded top-$k$ plans existence} is the decision problem: Given a planning task $\Pi$ and two natural numbers $\ell$ and $k$, are there at least $k$ different plans of length at most $\ell$?
\end{definition}

First of all, \textcite{speck-et-al-aaai2020} showed that the bounded top-$k$ plans existence problem in general is \PSPACE-complete (\Cref{thm:bounded-top-k-plan-existence}), which is surprising because the ordinary bounded plan existence problem, i.e., answering the question whether only one such plan exists, is also \PSPACE-complete (\Cref{thm:bounded-plan-existence}).

\begin{theorem}[\cite{speck-et-al-aaai2020}]\label{thm:bounded-top-k-plan-existence}
    Bounded top-$k$ plans existence is \PSPACE-complete. \qed
\end{theorem}

Considering polynomially bounded plans, it turns out that the bounded \mbox{top-$k$} plan existence decision problem is \PP-hard \autocite{gill-siam-1977,speck-et-al-aaai2020}, while the ordinary bounded plan existence problem is \NP-complete \autocite{bylander-aij1994}.

\begin{theorem}[\cite{speck-et-al-aaai2020}]\label{thm:bounded-top-k-plan-existence-short}
    Bounded top-$k$ plans existence is \PP-hard if the plan length $\ell$ is polynomially bounded by the task size, i.e., $\ell \leq p(\size{\task})$ for some polynomial $p$. \qed
\end{theorem}

\Cref{thm:bounded-top-k-plan-existence-short} indicates that top-$k$ planning is much harder in practice than ordinary classical planning.
The assumption that both problems have the same complexity when the plan length is polynomially bounded would imply a collapse of the polynomial hierarchy at $\mathsf{P}^{\text{\NP}}$ \autocite{toda-siam1991}, which is considered very unlikely.

\section{Symbolic Search}
\textcite{speck-et-al-aaai2020} propose a symbolic search approach to top-$k$ planning that is sound and complete.
Recall that symbolic search for classical planning is divided into two phases: 1) a reachability phase, in which states with increasing costs are generated until a goal state is found, and 2) a plan reconstruction phase, in which the search is regressed to reconstruct a corresponding goal path and plan.
Since symbolic search expands entire sets of states at once, it is not unusual to find multiple goal states at once.
If multiple goal states are found, then of course multiple plans are found as well.
But even if only one goal state is found, it is possible that multiple plans are found that lead to that particular goal state.
Usually, symbolic search reconstructs and reports only one such plan, ignoring all others.
Based on this observation, \textcite{speck-et-al-aaai2020} show that three modifications to ordinary symbolic search lead to a sound and complete symbolic search algorithm for top-$k$ planning.
First, they propose that after a goal state is expanded, all plans leading to that goal state are reconstructed.
Second, during the search, states are not closed, i.e., all newly generated states with their corresponding reachability costs are insert in the open list without filtering them by already expanded states, otherwise suboptimal plans may be lost.
Third and finally, the termination criterion is adjusted so that the algorithm terminates if either $k$ plans are found or the open list contains only states that have already been expanded at least once and are not part of a goal path induced by a plan that has already been found.

\Cref{ex:gripper} illustrates the functioning of this modified symbolic search for top-$k$ planning.
For simplicity, this example describes symbolic forward search for a task with unit cost.

\begin{figure}
    \centering
    \subfloat[Visualization of the dynamics of the planning task.\label{fig:gripper_example_a}]{
        \centering
        \makebox[1\textwidth][c]{
            \newcommand{\gripperFloor}[3]
{
    \draw [ultra thick] (#1,#2) -- (#1 + 0.6, #2);
    \draw [ultra thick] (#1 + 0.8, #2) -- (#1 + 1.4,#2);
    \node [draw,minimum size=0.65cm] at (#1 + 0.7, #2 - 0.75) {$s_#3$};

}

\newcommand{\gripperHand}[2]
{
    \draw [ultra thick] (#1 + 0.3,#2) -- (#1 + 0.3, #2 + 0.4);
    \draw [ultra thick] (#1, #2 - 0.4) --(#1,#2) -- (#1 + 0.6,#2) -- (#1 + 0.6, #2 - 0.4);
}

\newcommand{\gripperBall}[2]
{
    \draw [ultra thick] (#1,#2) circle [radius=0.225];
}

\begin{tikzpicture}[]
    \gripperFloor{0}{0}{0 = \init}
    \gripperBall{0.3}{0.325}
    \gripperHand{0.0}{1.2}

    \gripperFloor{3}{0}{1}
    \gripperBall{3.3}{0.85}
    \gripperHand{3}{1.2}

    \gripperFloor{6}{0}{2}
    \gripperBall{7.1}{0.85}
    \gripperHand{6.8}{1.2}

    \gripperFloor{9}{0}{3 \models \goal}
    \gripperBall{10.1}{0.325}
    \gripperHand{9.8}{1.2}

    \draw[>=latex, <->] (1.1,1.85) to[ultra thick, bend left=45] node [above, align=center] {pick-up (p) \\ drop (d)} (3.3,1.85);
    \draw[>=latex, ->] (4.1,1.85) to[ultra thick, bend left=45] node [above, align=center] {move (m)} (6.3,1.85);
    \draw[>=latex, <->] (7.1,1.85) to[ultra thick, bend left=45] node [above, align=center] {drop (d) \\ pick-up (p)} (9.3,1.85);

    \node at (0,-1.1) {};

    %\draw [ultra thick, dashed] (0,-1) -- (7, -1);
\end{tikzpicture}

        }
    }
    \\ \medskip
    \subfloat[Reachable states generated and expanded by the proposed symbolic search approach.\label{fig:gripper_example_b}]{
        \centering
        \makebox[1\textwidth][c]{
            %\resizebox{1\textwidth}{!}{
            \begin{tikzpicture}[]
    \newcommand*{\xd}{1.9}
    %%% Fwd reachable
    \node[draw,ellipse] at (0*\xd, -2) {$s_0$};
    \node[draw,ellipse] at (1*\xd, -2) {$s_1$};
    \node[draw,ellipse, align=center] at (2*\xd, -2) {$s_0$ \\ $s_2$};
    \node[draw,ellipse, align=center] at (3*\xd, -2) {$s_1$ \\ $\mathbf{s_3}$};
    \node[draw,ellipse, align=center] at (4*\xd, -2) {$s_0$ \\ $s_2$};
    \node[draw,ellipse, align=center] at (5*\xd, -2) {$s_1$ \\ $\mathbf{s_3}$};
    \node[align=center] at (5.75*\xd, -2) {\Large \textbf{\dots}};

    \node[] at (0*\xd, -3.25) {$S_0$};
    \node[] at (1*\xd, -3.25) {$S_1$};
    \node[] at (2*\xd, -3.25) {$S_2$};
    \node[] at (3*\xd, -3.25) {$S_3$};
    \node[] at (4*\xd, -3.25) {$S_4$};
    \node[] at (5*\xd, -3.25) {$S_5$};

    \path [draw, ->, thick] (0.0, -4) to (-0.5+6*\xd, -4);
    \node at (-0.5, -4.5) {$g$};
    \foreach \x in {0,...,5}
    \node  at (\xd*\x,-4.5) {\x};


    % %%% Reconstruction
    \draw[>=latex, ->,ultra thick, red] (0.25+0*\xd,-2) to[ultra thick] node [above] {p} (-0.25+1*\xd,-2);
    \draw[>=latex, ->,ultra thick, dashed] (0.25+1*\xd,-2) to[ultra thick] node [above] {d} (-0.25+2*\xd,-1.8);
    \draw[>=latex, ->,ultra thick, red] (0.25+1*\xd,-2) to[ultra thick] node [below] {m} (-0.25+2*\xd,-2.2);
    \draw[>=latex, ->,ultra thick, red] (0.25+2*\xd,-2.2) to[ultra thick] node [below] {d} (-0.25+3*\xd,-2.2);
    \draw[>=latex, ->,ultra thick, dashed] (0.25+2*\xd,-1.8) to[ultra thick] node [above] {p} (-0.25+3*\xd,-1.8);
    \draw[>=latex, ->,ultra thick, dashed] (0.25+3*\xd,-1.8) to[ultra thick] node [above] {m} (-0.25+4*\xd,-2.2);
    \draw[>=latex, ->,ultra thick, dashed] (0.25+3*\xd,-2.2) to[ultra thick] node [below] {p} (-0.25+4*\xd,-2.2);
    \draw[>=latex, ->,ultra thick, dashed] (0.25+4*\xd,-2.2) to[ultra thick] node [below] {d} (-0.25+5*\xd,-2.2);
\end{tikzpicture}

            %}
        }
    }
    \caption[Visualization of symbolic search for top-$k$ planning.]{Visualization of the \name{gripper} planning task and the functioning of the proposed symbolic search approach for top-$k$ planning \autocite{speck-et-al-aaai2020}.}
    \label{fig:gripper_example}
\end{figure}

\begin{example}[\cite{speck-et-al-aaai2020}]\label{ex:gripper}
    Consider a unit cost planning task with two rooms and a robot with a gripper, as shown in \Cref{fig:gripper_example_a}.
    The robot can move from room A to B if it carries the ball, but it cannot return.
    There is a possibility to pick-up and drop the ball in each room.
    The goal is to get the ball from room A to room B.

    Assuming that the desired number of plans is $k = 3$, \Cref{fig:gripper_example_b} illustrates the functioning of the proposed symbolic search approach.
    First, all states $S_0$ reachable with cost $0$, i.e., only the initial state, are expanded resulting in the set of states $S_1 = \{s_1 \}$ reachable with cost $1$.
    The subsequent expansion of $S_1$ yields the state set $S_2 = \{ s_0, s_2 \}$, whose cost is $2$.
    Note that $s_0$ is a part of $S_2$, although it has been previously expanded and therefore would no longer be considered in ordinary symbolic search.
    The next expansion yields $S_3 = \{ s_1, s_3 \}$ with cost $3$, leading to the expansion of $S_3$, which contains the goal state $s_3$.
    Therefore, the plan reconstruction procedure is executed that yields exactly one plan $\plan_1 = \langle \textit{pick-up}, \textit{move}, \textit{drop} \rangle$ visualized in red in \Cref{fig:gripper_example_b}.
    Since a total of $3$ plans is desired, the search continues from here until the next goal states are expanded, which occurs with the expansion of $S_5$.
    Plan reconstruction yields two plans with a cost of $5$ each, namely $\plan_2 = \langle \textit{pick-up}, \textit{drop}, \textit{pick-up}, \textit{move}, \textit{drop} \rangle$ and $\plan_3 = \langle \textit{pick-up}, \textit{move}, \textit{drop}, \textit{pick-up},  \textit{drop} \rangle$.
    As the desired number of plans is achieved, the algorithm terminates with $\{\plan_1, \plan_2, \plan_3\}$.
    However, if more plans were desired, the search would continue.
\end{example}

Looking more closely at the reconstruction phase, we can see that we perform an exhaustive greedy backward search using the provided perfect heuristic of the reachability phase.
Although this exhaustive search may seem expensive, it is goal-driven due to the perfect heuristic, and each time the initial state is reached, a new plan is created.
Finally, \textcite{speck-et-al-aaai2020} also define and describe the support of general operator costs, including zero costs, and defined symbolic backward search and symbolic bidirectional search, which can significantly improve planning performance.

\begin{figure}
    \begin{center}
        \includegraphics[width=\textwidth]{pictures/topk_plot.png}
    \end{center}
    \caption[The $k$-coverage for top-$k$ planning.]{The $k$-coverage (number of instances for which $k$ best plans were found) for explicit and symbolic search algorithms on classical planning domains \autocite{speck-et-al-aaai2020}.}
    \label{fig:coverage_time_topk}
\end{figure}


\paragraph{Empirical Evaluation.}
To compare different approaches to top-$k$ planning, we consider $k$-coverage, i.e., the number of instances for which a planner reports a set of $k$ best plans or reports only $k' < k$ plans but proves that only $k'$ plans exist (see \Cref{def:top-k-planning}).
\Cref{fig:coverage_time_topk} compares the $k$-coverage of the discussed symbolic search approach based on BDDs from \textcite{speck-et-al-aaai2020} with other state-of-the-art approaches from the literature on domains of the optimal track of the International Planning Competitions 1998-2018.
The $x$-axis indicates the number of desired plans $k$, while the $y$-axis represents the number of instances in which the corresponding $k$-coverage has been reached.
\kstar{} \autocite{aljazzar-leue-aij2011,katz-et-al-icaps2018} generates and processes parts of the implicit search tree as needed and can be enhanced by a heuristic.
However, the evaluation only includes \kstar{} with the blind heuristic because \kstar{} is very memory intensive, so the performance differences reported by \textcite{speck-et-al-aaai2020} between \kstar{} with and without sophisticated heuristics are negligible.
\forbidk{} \autocite{katz-et-al-icaps2018} is an iterative approach that searches for additional plans through a replanning loop that forbids already found plans and preserves all other plans.
The underlying search is an orbit space search with structural symmetries \autocite{alkhazraji-et-al-ipc2014,domshlak-et-al-tr2015} and the \name{LM}-cut heuristic \autocite{helmert-domshlak-icaps2009}.
The three versions of \forbidk{} considered differ in a plan reordering step, where different reordering strategies (none, naive, and neighbor) can be used to generate multiple plans from a single plan.

Overall, we can see that symbolic bidirectional search performs as well as \forbidk{} when only one plan ($k=1$) is requested.
But already for $k=2$ symbolic bidirectional search performs best and for $k=4$ symbolic forward search surpasses Forbid-k, while symbolic bidirectional search already shows a large performance advantage.
Most importantly, symbolic search scales much better to larger $k$ than all other approaches.
\kstar{} solves only six more instances for $k = 1$ than for $k = 10\,000$, due to the high memory consumption that is the bottleneck of the approach.
The replanning of \forbidk{} turns out to be too expensive when a larger number of plans is desired, resulting in a large performance drop.
Finally, \textcite{speck-et-al-aaai2020} expect the dominant performance of symbolic search to hold even for very large $k$ until the high number of plan reports is the limiting factor, i.e., the limiting factor is writing the plans to disk.
