\chapter{Discussion}\label{ch:discussion}
\chapterquote{Be happy, but never satisfied.}{Bruce Lee}

\setboolean{kiviatFull}{true}
\renewcommand{\kiviatTopk}{2}
\renewcommand{\kiviatGoal}{2}
\renewcommand{\kiviatCost}{2}
\renewcommand{\kiviatPredicates}{2}
\begin{figure}[t]
    \begin{center}
        \begin{tikzpicture}
    \tkzKiviatDiagram[
    radial style/.style ={-{Latex[length=3mm, width=2mm]}},
    scale=0.85,
    label space=1.5,
    radial = 1,
    gap = 1.5,
    step = 1,
    lattice = 2]{
    \hspace{2.5cm}\mbox{\parbox{4cm}{{\begin{center}\textbf{Goal}\\\textbf{Description} \end{center}}}},
    {\textbf{Number of Plans}},
    \hspace{-1.5cm}\mbox{\parbox{3cm}{{\begin{center}\textbf{State}\\\textbf{Description}\end{center}}}},
    \textbf{Cost Function}}

    \ifbool{kiviatRed}{
        \tkzKiviatLine[very thick,color=red!50,
            fill=red!30,
            opacity=.35,
            mark=ball,
            mark size=2.5pt,
            ball color=red](\kiviatGoal{},\kiviatTopk{},\kiviatPredicates{},\kiviatCost{})
    }

    \ifbool{kiviatBlue}{
        \tkzKiviatLine[very thick,color=blue!50,
            fill=blue!30,
            opacity=.35,
            mark=ball,
            mark size=2.5pt,
            ball color=blue](1,1,1,1)
    }

    \ifbool{kiviatFull}{
    \tkzKiviatLine[very thick,color=darkgreen!50,
        fill=green!30,
        opacity=.35,
        mark=ball,
        mark size=2.5pt,
        ball color=darkgreen](1,2,2,2)
    \tkzKiviatLine[very thick,color=blue!50,
        fill=blue!30,
        opacity=.35,
        mark=ball,
        mark size=2.5pt,
        ball color=blue](1,1,1,1)
    \tkzKiviatLine[very thick,color=red!50,
        %fill=red!30,
        %opacity=.35,
        mark=ball,
        mark size=2.5pt,
        ball color=red](\kiviatGoal{},\kiviatTopk{},\kiviatPredicates{},\kiviatCost{})
    \LegendBox[shift={(-0cm,-2.25cm)}]{current bounding box.south west}%
    {
    red!100/{Forward symbolic search (contribution of this thesis)},
    asparagus!100/{Bidirectional symbolic search (contribution of this thesis)},
    blue!100/{Bidirectional symbolic search (previous state of the art)}}
    }
    {}

    % goal specification
    \draw[] node[align=center,fill=none] at (1.5,-0.5) {\small Hard};
    \draw[] node[align=center,fill=none] at (3.25,-0.75) {\small Oversub-\\ \small scription};
    %\draw[] node[] at (1.75,-0.6) {\rotatebox{-45}{hard}};
    %\draw[] node[align=center] at (3.85,-1.25) {\rotatebox{-45}{oversubscription}};

    % number of plans
    \draw[] node[align=center,fill=none] at (0.7,1.5) {\small Top-$1$};
    \draw[] node[align=center,fill=none] at (0.7,3.0) {\small Top-$k$};

    % derived predicates
    \draw[] node[align=center,fill=none] at (-1.25,0.75) {\small State\\ \small Variables};
    \draw[] node[align=center,fill=none] at (-3.25,0.75) {\small $+$ \small Derived\\ \small Variables};

    % cost function
    \draw[] node[align=center,fill=none] at (-1,-1.5) {\small Constant};
    \draw[] node[align=center,fill=none] at (-1.6,-3.0) {\small State-Dependent\phantom{0}};
\end{tikzpicture}
    \end{center}
    \caption[Overview of the contribution of this thesis.]{
        Overview of the contribution of this thesis in terms of extending symbolic search to support different expressive extensions of classical planning.
    }\label{fig:dicussion:kiviat}
\end{figure}

\setboolean{kiviatFull}{false}

As we discussed in \Cref{sec:symbolic_search_discussion}, symbolic blind search, i.e., without any heuristics, is the dominant search strategy for symbolic search, on par with explicit heuristic search in optimal planning.
For this reason, symbolic search provides a good basis for supporting several extensions of classical planning.
In this chapter, we discuss the combination of the previously considered extensions and how symbolic search with the introduced modifications can support this setting.
Finally, possible future work related to this thesis is discussed.

\section{Combination of Extensions}\label{sec:combination}
As we have seen in the previous chapters, it is possible to extend symbolic search to support various extensions of classical planning.
A natural question that arises is, of course, about combining these extensions, which makes intuitive sense since they all capture and extend different aspects of classical planning.
The key question is whether the different enhancements to symbolic search can be combined to support several of these planning extensions simultaneously.
The short answer to this question is: Yes, with some minor restrictions on the search directions.
This is remarkable because the support of only one of these features is quite rare, and the support of more than one feature is almost inexistent in the literature on cost-optimal planning.
The reason is the same why very few planning techniques and planners support a single feature, namely that almost all optimal planners are based on heuristic search and it is very difficult to design admissible, informative, and fast to evaluate heuristics that take into account the different extensions to classical planning.
We will now briefly discuss the combination of the discussed extensions for classical planning and how and why the proposed modifications to symbolic search can support all of these extensions simultaneously.

\paragraph{Model Extensions.}
Let us first focus on the three discussed model extensions that result in \emph{oversubscription planning with axioms and state-dependent action costs}.
This planning formalism is simply defined by combining all the features of the extension and allowing the state-dependent cost function and the utility function to take into account derived variables \derivedvars such that the cost function of an operator $o$ is given by $\costfun_o : \extendedstates \to \mathbb{N}_0$ and the utility function $\utility : \extendedstates \to \mathbb{N}_0$.
This makes it possible to concisely encode complex state-dependent action cost functions and state utility functions.
In the empirical evaluation of \Cref{ch:sdac}, we have already mentioned that derived variables can be used in a state-dependent action cost function in the Power Supply Restoration (PSR) domain \autocite{thiebaux-cordier-ecp2001,thiebaux-et-al-ijcai2013,hoffmann-et-al-jair2006}, to encode whether certain parts of the power system are energized.
A closer look, however, reveals an interesting connection between these model extensions.
While derived variables in the cost or utility function can lead to a simple and concise representation, it might be possible to simulate the evaluation of the values of the derived variables by the cost or utility function, which could make the consideration of derived variables in the cost or utility function obsolete.
Moreover, it might be possible to simplify a complex cost or utility function by introducing new derived variables and thus shifting the complexity into the computation of the derived variables.
This shows that these extensions are closely related and a more detailed analysis of the connection is an exciting question for future work (\Cref{sec:future_work}).

All proposed symbolic encodings of axioms and derived variables in the context of symbolic search (\Cref{ch:axioms}) allow derived variables to exist in the domain of operator cost functions and utility functions.
Thus, the proposed enhancements to symbolic search can be directly used for oversubscription planning with axioms and state-dependent action costs.
This is especially true for the dominant axiom encoding strategy, symbolic translation, in which each derived variable is replaced by its primary representation (\Cref{def:primary_representation}).
However, as \Cref{fig:dicussion:kiviat} illustrates, in oversubscription planning tasks, i.e., in the presence of utilities, only forward symbolic search can be performed, since it is not clear how to perform backward or bidirectional symbolic search efficiently.

\paragraph{Top-k Planning with Model Extensions.}
The search for the best $k$ plans is independent of the support of axioms and state-dependent action costs and can be easily combined.
However, if one combines oversubscription planning and top-$k$ planning, the question arises which are the best $k$ plans.
Although this question has never been answered in the literature, it is natural to rank plans first by utility and then by cost, i.e., a plan $\pi$ is better than a plan $\pi'$ iff 1) $\utility(\pi) > \utility(\pi')$ or 2) $\utility(\pi) = \utility(\pi')$ and $\cost(\pi) < \cost(\pi')$.
And indeed, this can be supported with minor modifications to the proposed symbolic search approach by maintaining not only the states with the best utility found so far, but all relevant ones and executing exhaustive plan extraction as done in symbolic search for top-$k$ planning (\Cref{fig:dicussion:kiviat}).
\medskip

Overall, the proposed extension of symbolic search yields an optimal, sound, and complete approach to \emph{top-$k$ oversubscription planning with axioms and state-dependent action costs}.
As \Cref{fig:dicussion:kiviat} illustrates, this not only enhances the applicability and state of the art of symbolic search for classical planning with expressive extensions, but also provides the first planner that can support several of these extensions at once.\footnote{Available online: \url{https://github.com/speckdavid/symk}}
Finally, the proposed planning algorithm that supports the interaction of all these features does not suffer from its generality, because if certain features are not present, the approach becomes standard symbolic search.

\section{Future Work}\label{sec:future_work}
%There are various directions for research in the future related to the content of this thesis. 
While we have highlighted some future challenges and research in the individual chapters, here we focus on future research directions related to symbolic (heuristic) search and planning with expressive extensions in general.

We have seen in this thesis different types of decision diagrams such as EVMDDs, ADDs and BDDs used in the context of symbolic search for classical planning.
A more detailed comparison in theory and practice could provide more insight into when to use which decision diagrams.
Other types of decision diagrams such as Functional Decision Diagrams \autocite{kebschull-et-al-dac1992} or Kronecker Functional Decision Diagrams \autocite{drechsler-becker-tcad1998} could also lead to an even more concise and efficient representation.

Throughout this thesis, we have assumed a fixed static variable order.
Since the order of the variables plays an important role in the efficiency of symbolic search, as the size of the decision diagrams strongly depends on the chosen order of the variables, it is important to find good orders.
While it is known that the computation of an optimal order for decision diagrams is \coNP{}-complete \autocite{bryant-ieeecomp1986}, it is still a challenging question whether and how to find good orders in practice \autocite{kissmann-hoffmann-icaps2013,kissmann-hoffmann-jair2014}.
Moreover, dynamic reordering techniques \autocite{rudell-iccad1993} are used in other fields, such as model checking \autocite{yang-et-al-fmcad1998} or logic synthesis \autocite{scholl-et-al-tcad1999}, which have been explored up to now only sparsely in planning \autocite{kissmann-hoffmann-jair2014,kissmann-et-al-ipc2014}.

The efficient integration of heuristics into symbolic search is still an open research task.
We believe that the main goal should be to keep the decision diagrams involved as small as possible.
\textcite{fiser-et-al-icaps2021wshsdip} introduced operator potentials based on potential heuristics \autocite{pommerening-et-al-aaai2015}, which seem to provide concise representation in symbolic forward search, while efficiently pruning states based on their heuristic values.
More work certainly needs to be done here to better understand the search behavior of symbolic heuristic search.
In particular, it is interesting to see if it is possible to find heuristics that give any size guarantees for the decision diagrams involved, and if and how the results can be generalized to symbolic bidirectional heuristic search.

Considering the discussed extensions of classical planning, an in-depth theoretical and empirical evaluation of the combination of several such extensions and the symbolic search approach presented for this purpose is certainly a promising research direction.
In particular, it is interesting to examine the relationship between different extensions.
It might be possible that one extension can be compiled away (with small overhead) in the presence of another extension.
Moreover, several of these extensions are not yet covered by more expressive planning formalisms, such as
fully observable nondeterministic planning \autocite{cimatti-et-al-aij2003}, partially observable nondeterministic planning \autocite{bertoli-et-al-aij2006,speck-et-al-ki2015}, or hierarchical task network planning \autocite{erol-et-al-amai1996,geier-bercher-ijcai2011}.
An analysis of how and to what extent all these features can enhance these planning formalisms is an important future line of research, and whether the proposed ideas can also be used for symbolic search approaches for these settings \autocite{kissmann-edelkamp-ki2009,behnke-speck-aaai2021,bertoli-et-al-aij2006}.
%\section*{Core publications of this chapter}
%This chapter describes and discusses the combination of all extensions and %clearly relies on all core publications, but goes beyond their content.

% \todo[caption={}]{
%     \textbf{Complexity and Compilability:}
%     \begin{itemize}
%         \item OSP + Derived Predicates + SDAC are all essientiel features and can lead to a multiple exponential blowup by compiling them away.
%         \item OSP + Derived Predicates + SDAC is PSPACE-complete (given functions in PSPACE).
%         \item Hardness: Reduce one version (e.g SDAC) to it.
%         \item Membership: Guess actions until the cost bound is exhausted or the current state has a high enough utility. For evaluation of derived variables we require only PSPACE and for evaluation of cost function again PSPACE. No problem.
%     \end{itemize}

% }

% \section{Formalism}
% \section{Complexity and Compilability}
% \section{Extending Symbolic Search}
% \section{Summary}
