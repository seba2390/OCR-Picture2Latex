
%% bare_conf.tex
%% V1.4b
%% 2015/08/26
%% by Michael Shell
%% See:
%% http://www.michaelshell.org/
%% for current contact information.
%%
%% This is a skeleton file demonstrating the use of IEEEtran.cls
%% (requires IEEEtran.cls version 1.8b or later) with an IEEE
%% conference paper.
%%
%% Support sites:
%% http://www.michaelshell.org/tex/ieeetran/
%% http://www.ctan.org/pkg/ieeetran
%% and
%% http://www.ieee.org/

%%*************************************************************************
%% Legal Notice:
%% This code is offered as-is without any warranty either expressed or
%% implied; without even the implied warranty of MERCHANTABILITY or
%% FITNESS FOR A PARTICULAR PURPOSE! 
%% User assumes all risk.
%% In no event shall the IEEE or any contributor to this code be liable for
%% any damages or losses, including, but not limited to, incidental,
%% consequential, or any other damages, resulting from the use or misuse
%% of any information contained here.
%%
%% All comments are the opinions of their respective authors and are not
%% necessarily endorsed by the IEEE.
%%
%% This work is distributed under the LaTeX Project Public License (LPPL)
%% ( http://www.latex-project.org/ ) version 1.3, and may be freely used,
%% distributed and modified. A copy of the LPPL, version 1.3, is included
%% in the base LaTeX documentation of all distributions of LaTeX released
%% 2003/12/01 or later.
%% Retain all contribution notices and credits.
%% ** Modified files should be clearly indicated as such, including  **
%% ** renaming them and changing author support contact information. **
%%*************************************************************************


% *** Authors should verify (and, if needed, correct) their LaTeX system  ***
% *** with the testflow diagnostic prior to trusting their LaTeX platform ***
% *** with production work. The IEEE's font choices and paper sizes can   ***
% *** trigger bugs that do not appear when using other class files.       ***                          ***
% The testflow support page is at:
% http://www.michaelshell.org/tex/testflow/



\documentclass[conference]{IEEEtran}
% Some Computer Society conferences also require the compsoc mode option,
% but others use the standard conference format.
%
% If IEEEtran.cls has not been installed into the LaTeX system files,
% manually specify the path to it like:
% \documentclass[conference]{../sty/IEEEtran}


\usepackage{amsmath}
\usepackage{amssymb}
\usepackage{graphicx}


% Some very useful LaTeX packages include:
% (uncomment the ones you want to load)


% *** MISC UTILITY PACKAGES ***
%
%\usepackage{ifpdf}
% Heiko Oberdiek's ifpdf.sty is very useful if you need conditional
% compilation based on whether the output is pdf or dvi.
% usage:
% \ifpdf
%   % pdf code
% \else
%   % dvi code
% \fi
% The latest version of ifpdf.sty can be obtained from:
% http://www.ctan.org/pkg/ifpdf
% Also, note that IEEEtran.cls V1.7 and later provides a builtin
% \ifCLASSINFOpdf conditional that works the same way.
% When switching from latex to pdflatex and vice-versa, the compiler may
% have to be run twice to clear warning/error messages.






% *** CITATION PACKAGES ***
%
%\usepackage{cite}
% cite.sty was written by Donald Arseneau
% V1.6 and later of IEEEtran pre-defines the format of the cite.sty package
% \cite{} output to follow that of the IEEE. Loading the cite package will
% result in citation numbers being automatically sorted and properly
% "compressed/ranged". e.g., [1], [9], [2], [7], [5], [6] without using
% cite.sty will become [1], [2], [5]--[7], [9] using cite.sty. cite.sty's
% \cite will automatically add leading space, if needed. Use cite.sty's
% noadjust option (cite.sty V3.8 and later) if you want to turn this off
% such as if a citation ever needs to be enclosed in parenthesis.
% cite.sty is already installed on most LaTeX systems. Be sure and use
% version 5.0 (2009-03-20) and later if using hyperref.sty.
% The latest version can be obtained at:
% http://www.ctan.org/pkg/cite
% The documentation is contained in the cite.sty file itself.






% *** GRAPHICS RELATED PACKAGES ***
%
\ifCLASSINFOpdf
  % \usepackage[pdftex]{graphicx}
  % declare the path(s) where your graphic files are
  % \graphicspath{{../pdf/}{../jpeg/}}
  % and their extensions so you won't have to specify these with
  % every instance of \includegraphics
  % \DeclareGraphicsExtensions{.pdf,.jpeg,.png}
\else
  % or other class option (dvipsone, dvipdf, if not using dvips). graphicx
  % will default to the driver specified in the system graphics.cfg if no
  % driver is specified.
  % \usepackage[dvips]{graphicx}
  % declare the path(s) where your graphic files are
  % \graphicspath{{../eps/}}
  % and their extensions so you won't have to specify these with
  % every instance of \includegraphics
  % \DeclareGraphicsExtensions{.eps}
\fi
% graphicx was written by David Carlisle and Sebastian Rahtz. It is
% required if you want graphics, photos, etc. graphicx.sty is already
% installed on most LaTeX systems. The latest version and documentation
% can be obtained at: 
% http://www.ctan.org/pkg/graphicx
% Another good source of documentation is "Using Imported Graphics in
% LaTeX2e" by Keith Reckdahl which can be found at:
% http://www.ctan.org/pkg/epslatex
%
% latex, and pdflatex in dvi mode, support graphics in encapsulated
% postscript (.eps) format. pdflatex in pdf mode supports graphics
% in .pdf, .jpeg, .png and .mps (metapost) formats. Users should ensure
% that all non-photo figures use a vector format (.eps, .pdf, .mps) and
% not a bitmapped formats (.jpeg, .png). The IEEE frowns on bitmapped formats
% which can result in "jaggedy"/blurry rendering of lines and letters as
% well as large increases in file sizes.
%
% You can find documentation about the pdfTeX application at:
% http://www.tug.org/applications/pdftex





% *** MATH PACKAGES ***
%
%\usepackage{amsmath}
% A popular package from the American Mathematical Society that provides
% many useful and powerful commands for dealing with mathematics.
%
% Note that the amsmath package sets \interdisplaylinepenalty to 10000
% thus preventing page breaks from occurring within multiline equations. Use:
%\interdisplaylinepenalty=2500
% after loading amsmath to restore such page breaks as IEEEtran.cls normally
% does. amsmath.sty is already installed on most LaTeX systems. The latest
% version and documentation can be obtained at:
% http://www.ctan.org/pkg/amsmath





% *** SPECIALIZED LIST PACKAGES ***
%
%\usepackage{algorithmic}
% algorithmic.sty was written by Peter Williams and Rogerio Brito.
% This package provides an algorithmic environment fo describing algorithms.
% You can use the algorithmic environment in-text or within a figure
% environment to provide for a floating algorithm. Do NOT use the algorithm
% floating environment provided by algorithm.sty (by the same authors) or
% algorithm2e.sty (by Christophe Fiorio) as the IEEE does not use dedicated
% algorithm float types and packages that provide these will not provide
% correct IEEE style captions. The latest version and documentation of
% algorithmic.sty can be obtained at:
% http://www.ctan.org/pkg/algorithms
% Also of interest may be the (relatively newer and more customizable)
% algorithmicx.sty package by Szasz Janos:
% http://www.ctan.org/pkg/algorithmicx




% *** ALIGNMENT PACKAGES ***
%
%\usepackage{array}
% Frank Mittelbach's and David Carlisle's array.sty patches and improves
% the standard LaTeX2e array and tabular environments to provide better
% appearance and additional user controls. As the default LaTeX2e table
% generation code is lacking to the point of almost being broken with
% respect to the quality of the end results, all users are strongly
% advised to use an enhanced (at the very least that provided by array.sty)
% set of table tools. array.sty is already installed on most systems. The
% latest version and documentation can be obtained at:
% http://www.ctan.org/pkg/array


% IEEEtran contains the IEEEeqnarray family of commands that can be used to
% generate multiline equations as well as matrices, tables, etc., of high
% quality.




% *** SUBFIGURE PACKAGES ***
%\ifCLASSOPTIONcompsoc
%  \usepackage[caption=false,font=normalsize,labelfont=sf,textfont=sf]{subfig}
%\else
%  \usepackage[caption=false,font=footnotesize]{subfig}
%\fi
% subfig.sty, written by Steven Douglas Cochran, is the modern replacement
% for subfigure.sty, the latter of which is no longer maintained and is
% incompatible with some LaTeX packages including fixltx2e. However,
% subfig.sty requires and automatically loads Axel Sommerfeldt's caption.sty
% which will override IEEEtran.cls' handling of captions and this will result
% in non-IEEE style figure/table captions. To prevent this problem, be sure
% and invoke subfig.sty's "caption=false" package option (available since
% subfig.sty version 1.3, 2005/06/28) as this is will preserve IEEEtran.cls
% handling of captions.
% Note that the Computer Society format requires a larger sans serif font
% than the serif footnote size font used in traditional IEEE formatting
% and thus the need to invoke different subfig.sty package options depending
% on whether compsoc mode has been enabled.
%
% The latest version and documentation of subfig.sty can be obtained at:
% http://www.ctan.org/pkg/subfig




% *** FLOAT PACKAGES ***
%
%\usepackage{fixltx2e}
% fixltx2e, the successor to the earlier fix2col.sty, was written by
% Frank Mittelbach and David Carlisle. This package corrects a few problems
% in the LaTeX2e kernel, the most notable of which is that in current
% LaTeX2e releases, the ordering of single and double column floats is not
% guaranteed to be preserved. Thus, an unpatched LaTeX2e can allow a
% single column figure to be placed prior to an earlier double column
% figure.
% Be aware that LaTeX2e kernels dated 2015 and later have fixltx2e.sty's
% corrections already built into the system in which case a warning will
% be issued if an attempt is made to load fixltx2e.sty as it is no longer
% needed.
% The latest version and documentation can be found at:
% http://www.ctan.org/pkg/fixltx2e


%\usepackage{stfloats}
% stfloats.sty was written by Sigitas Tolusis. This package gives LaTeX2e
% the ability to do double column floats at the bottom of the page as well
% as the top. (e.g., "\begin{figure*}[!b]" is not normally possible in
% LaTeX2e). It also provides a command:
%\fnbelowfloat
% to enable the placement of footnotes below bottom floats (the standard
% LaTeX2e kernel puts them above bottom floats). This is an invasive package
% which rewrites many portions of the LaTeX2e float routines. It may not work
% with other packages that modify the LaTeX2e float routines. The latest
% version and documentation can be obtained at:
% http://www.ctan.org/pkg/stfloats
% Do not use the stfloats baselinefloat ability as the IEEE does not allow
% \baselineskip to stretch. Authors submitting work to the IEEE should note
% that the IEEE rarely uses double column equations and that authors should try
% to avoid such use. Do not be tempted to use the cuted.sty or midfloat.sty
% packages (also by Sigitas Tolusis) as the IEEE does not format its papers in
% such ways.
% Do not attempt to use stfloats with fixltx2e as they are incompatible.
% Instead, use Morten Hogholm'a dblfloatfix which combines the features
% of both fixltx2e and stfloats:
%
% \usepackage{dblfloatfix}
% The latest version can be found at:
% http://www.ctan.org/pkg/dblfloatfix




% *** PDF, URL AND HYPERLINK PACKAGES ***
%
%\usepackage{url}
% url.sty was written by Donald Arseneau. It provides better support for
% handling and breaking URLs. url.sty is already installed on most LaTeX
% systems. The latest version and documentation can be obtained at:
% http://www.ctan.org/pkg/url
% Basically, \url{my_url_here}.




% *** Do not adjust lengths that control margins, column widths, etc. ***
% *** Do not use packages that alter fonts (such as pslatex).         ***
% There should be no need to do such things with IEEEtran.cls V1.6 and later.
% (Unless specifically asked to do so by the journal or conference you plan
% to submit to, of course. )


% correct bad hyphenation here
\hyphenation{op-tical net-works semi-conduc-tor}


\begin{document}
%
% paper title
% Titles are generally capitalized except for words such as a, an, and, as,
% at, but, by, for, in, nor, of, on, or, the, to and up, which are usually
% not capitalized unless they are the first or last word of the title.
% Linebreaks \\ can be used within to get better formatting as desired.
% Do not put math or special symbols in the title.
\title{Non-Negative Matrix Factorization Test Cases}

% author names and affiliations
% use a multiple column layout for up to three different
% affiliations
\author{\IEEEauthorblockN{
  Connor Sell and Jeremy Kepner \\
  Massachusetts Institute of Technology, Cambridge, MA 02139 \\
  email: csell@mit.edu, kepner@ll.mit.edu
}
}

%\author{\IEEEauthorblockN{Connor Sell}
%\IEEEauthorblockA{Lincoln Laboratory\\
%Massachusetts Institute of Technology\\
%Cambridge, Massachusetts 02139\\
%Email: csell@mit.edu}
%\and
%\IEEEauthorblockN{Dr. Jeremy Kepner}
%\IEEEauthorblockA{Lincoln Laboratory\\
%Cambridge, Massachusetts 02139\\
%Email: kepner@ll.mit.edu}}

% conference papers do not typically use \thanks and this command
% is locked out in conference mode. If really needed, such as for
% the acknowledgment of grants, issue a \IEEEoverridecommandlockouts
% after \documentclass

% for over three affiliations, or if they all won't fit within the width
% of the page, use this alternative format:
% 
%\author{\IEEEauthorblockN{Michael Shell\IEEEauthorrefmark{1},
%Homer Simpson\IEEEauthorrefmark{2},
%James Kirk\IEEEauthorrefmark{3}, 
%Montgomery Scott\IEEEauthorrefmark{3} and
%Eldon Tyrell\IEEEauthorrefmark{4}}
%\IEEEauthorblockA{\IEEEauthorrefmark{1}School of Electrical and Computer Engineering\\
%Georgia Institute of Technology,
%Atlanta, Georgia 30332--0250\\ Email: see http://www.michaelshell.org/contact.html}
%\IEEEauthorblockA{\IEEEauthorrefmark{2}Twentieth Century Fox, Springfield, USA\\
%Email: homer@thesimpsons.com}
%\IEEEauthorblockA{\IEEEauthorrefmark{3}Starfleet Academy, San Francisco, California 96678-2391\\
%Telephone: (800) 555--1212, Fax: (888) 555--1212}
%\IEEEauthorblockA{\IEEEauthorrefmark{4}Tyrell Inc., 123 Replicant Street, Los Angeles, California 90210--4321}}




% use for special paper notices
%\IEEEspecialpapernotice{(Invited Paper)}




% make the title area
\maketitle

% As a general rule, do not put math, special symbols or citations
% in the abstract
\begin{abstract}
Non-negative matrix factorization (NMF) is a problem with many applications, ranging from facial recognition to document clustering.  However, due to the variety of algorithms that solve NMF, the randomness involved in these algorithms, and the somewhat subjective nature of the problem, there is no clear ``correct answer'' to any particular NMF problem, and as a result, it can be hard to test new algorithms.  This paper suggests some test cases for NMF algorithms derived from matrices with enumerable exact non-negative factorizations and perturbations of these matrices.  Three algorithms using widely divergent approaches to NMF all give similar solutions over these test cases, suggesting that these test cases could be used as test cases for implementations of these existing NMF algorithms as well as potentially new NMF algorithms.  This paper also describes how the proposed test cases could be used in practice.
\end{abstract}

% no keywords




% For peer review papers, you can put extra information on the cover
% page as needed:
% \ifCLASSOPTIONpeerreview
% \begin{center} \bfseries EDICS Category: 3-BBND \end{center}
% \fi
%
% For peerreview papers, this IEEEtran command inserts a page break and
% creates the second title. It will be ignored for other modes.
\IEEEpeerreviewmaketitle



\section{Introduction}
% no \IEEEPARstart
\let\thefootnote\relax\footnotetext{This material is based in part upon work supported by the NSF under grant number DMS-1312831.  Any opinions, findings, and conclusions or recommendations expressed in this material are those of the authors and do not necessarily reflect the views of the National Science Foundation.} What do document clustering, recommender systems, and audio signal processing have in common?  All of them are problems that involve finding patterns buried in noisy data.  As a result, these three problems are common applications of algorithms that solve non-negative matrix factorization, or NMF \cite{Gemulla:2011:LMF:2020408.2020426,NIPS2005_2757,wang2010instantaneous}.

Non-negative matrix factorization involves factoring some matrix $ \mathbf{A} $, usually large and sparse, into two factors $ \mathbf{W} $ and $ \mathbf{H} $, usually of low rank
\begin{equation}
\mathbf{A} = \mathbf{WH}
\end{equation}
Because all of the entries in $ \mathbf{A} $, $ \mathbf{W} $, and $ \mathbf{H} $ must be non-negative, and because of the imposition of low rank on $ \mathbf{W} $ and $ \mathbf{H} $, an exact factorization rarely exists.  Thus NMF algorithms often seek an approximate factorization, where $ \mathbf{WH} $ is close to $ \mathbf{A} $.  Despite the imprecision, however, the low rank of $ \mathbf{W} $ and $ \mathbf{H} $ forces the solution to describe $ \mathbf{A} $ using fewer parameters, which tends to find underlying patterns in $ \mathbf{A} $.   These underlying patterns are what make NMF of interest to a wide range of applications.

In the decades since NMF was introduced by Seung and Lee \cite{NIPS2000_1861}, a variety of algorithms have been published that compute NMF \cite{ALS}.  However, the non-deterministic nature of these NMF algorithms make them difficult to test.  First, NMF asks for approximations rather than exact solutions, so whether or not an output is correct is somewhat subjective.  Although cost functions can quantitatively indicate how close a given solution is to being optimal, most algorithms do not claim to find the globally optimal solution, so whether or not an algorithm gives useful solutions can be ambiguous.  Secondly, all of the algorithms produced so far are stochastic algorithms, so running the algorithm on the same input multiple times can give different outputs if they use different random number sequences.  Thirdly, the algorithms themselves, though often simple to implement, can have very complex behavior that is difficult to understand.  As a result, it can be hard to determine whether a proposed algorithm really ``solves'' NMF.

This paper proposes some test cases that NMF algorithms should solve verifiably.  The approach uses very simple input, such as matrices that have exact non-negative factorizations, that reduce the space of possible solutions and ensure that the algorithm finds correct patterns with little noise.  In addition, small perturbations of these simple matrices are also used, to ensure that small variations in the matrix $ \mathbf{A} $ do not drastically change the generated solution.

% You must have at least 2 lines in the paragraph with the drop letter
% (should never be an issue)

\section{Perturbations of order $ \epsilon $}
Suppose NMF is applied to a non-negative matrix $ \mathbf{A} $ to get non-negative matrices $ \mathbf{W} $ and $ \mathbf{H} $ such that $ \mathbf{A} \approx \mathbf{W} \mathbf{H} $.  If $ \mathbf{A} $ is chosen to have an exact non-negative factorization, then the optimal solution satisfies $ \mathbf{A} = \mathbf{WH} $. Furthermore, if $ \mathbf{A} $ is simple enough, most ``good'' NMF algorithms will find the exact solution.

For example, suppose $ \mathbf{A}_0 $ is a non-negative square diagonal matrix, and the output $ \mathbf{W}_0 $ and $ \mathbf{H}_0 $ is also specified to be square.   Let the diagonal $ n \times n $ matrix $ \mathbf{A}_0 $ be denoted $ \mathbf{A}_0 = \text{diag}(\mathbf{a}_0) $, where $ \mathbf{a}_0 $ is an $ n $-dimensional vector, so that the diagonal entries $ \mathbf{A}_0(i,i) $ are $ \mathbf{a}_0(i) $.  It is easy to show that $ \mathbf{W}_0 $ and $ \mathbf{H}_0 $ must be monomial matrices (diagonal matrices under a permutation) \cite{proof}.  Ignoring the permutation and similarly denoting $ \mathbf{W}_0 = \text{diag}(\mathbf{w}_0) $ and $ \mathbf{H}_0 = \text{diag}(\mathbf{h}_0) $, then $ \mathbf{a}_0(i) = \mathbf{w}_0(i) \mathbf{h}_0(i) $ for applicable $ i $.  Such diagonal matrices $ \mathbf{A}_0 $ were given as input to the known NMF algorithms described in the next section, and all of the algorithms successfully found exact solutions in the form of monomial matrices for $ \mathbf{W}_0 $ and $ \mathbf{H}_0 $.

One way to analyze the properties of an algorithm is to perturb the input by a small amount $ \mathbf{\epsilon} > 0 $ and see how the output changes.  Formally, if the input $ \mathbf{A}_0 $ gives output $ \mathbf{W}_0\mathbf{H}_0 $, then the output generated from $ \mathbf{A}_0 + \epsilon \mathbf{A}_1 $ can be approximated as $ (\mathbf{W}_0 + \epsilon \mathbf{W}_1)(\mathbf{H}_0 + \epsilon \mathbf{H}_1) $.  It is assumed that $ \epsilon $ is sufficiently small that $ \epsilon^2 $ terms are negligible.

For the test case, the nonzero entries of $ \mathbf{A}_1 $ were chosen to be the on the superdiagonal (the first diagonal directly above the main diagonal).  This matrix is denoted  as $ \mathbf{A}_1 = \text{diag}(\mathbf{a}_1, 1) $, where $ \mathbf{a}_1 $ is an $ n-1 $-dimensional vector such that $ \mathbf{A}_1(i,i+1) = \mathbf{a}_1(i) $.  The resulting matrix $ \mathbf{A}_0 + \epsilon \mathbf{A}_1 $ has $O(1)$ entries on its main diagonal, $O(\epsilon) $ entries on the superdiagonal, and zeroes elsewhere.  It is assumed that all the vector entries $ \mathbf{a}_0(i) $ and $ \mathbf{a}_1(i) $ are of comparable magnitude.

\section{Results from Various Algorithms}
Three published NMF algorithms were implemented and run with input of the form $ \mathbf{A} = \mathbf{A}_0 + \epsilon \mathbf{A}_1 $ as described above.  Algorithm 1 was the multiplicative update algorithm described by Seung and Lee in their groundbreaking paper \cite{NIPS2000_1861}, which was run for $ 10^6 $ iterations in each test.  Algorithm 2 was the ALS algorithm described in \cite{ALS}, and which was run for $ 10^6 $ iterations as well.  Algorithm 3 was a gradient descent method as described by Guan and Tao \cite{GradDes}, and was run for $ 10^4 $ iterations.  These three algorithms were chosen because they were representative and easy-to-implement algorithms of three distinct types.  Many published NMF algorithms are variations of these three algorithms.

The experiments began with the simplest nontrivial case, in which $ \mathbf{A} $ is a $ 2 \times 2 $ matrix with only three nonzero entries, with fixed $ \mathbf{a}_0 = [1~1] $ and $ \mathbf{a}_1 = [1] $, while  $ \epsilon $ was varied over several different values.  Each of the algorithms used randomness in the form of initial seed values for $ \mathbf{W} $ and $ \mathbf{H} $. The random seeds were held constant as $ \epsilon $ varied.  As a result, the outputs from the algorithms with different values of $ \epsilon $ were comparable within each test case.

For the $ 2 \times 2 $ case, it is possible to enumerate all of the non-negative exact factorizations of $ \mathbf{A} $.  Given that the factors $ \mathbf{W} $ and $ \mathbf{H} $ are also $ 2 \times 2 $ matrices, they can be written as shown below.
\begin{equation}
\left[ \begin{array}{cc} m & n \\ p & q \end{array} \right] \left[ \begin{array}{cc} r & s \\ t & u \end{array} \right] = \left[ \begin{array}{cc} 1 & \epsilon \\ & 1 \end{array} \right]
\end{equation}
Multiplying the matrices directly produces the the following four equations:
\begin{align}
mr + nt & = 1 \\
ms + nu & = \epsilon \\
pr + qt & = 0 \\
ps + qu & = 1
\end{align}
Recall that all entries must be non-negative, so from equation (5), either $ p $ or $ r $ must be 0, and either $ q $ or $ t $ must be 0.  Furthermore, it cannot be that $ p = q = 0 $ because that would contradict equation (6), and it cannot be that $ r = t = 0 $ because that would contradict equation (3).  Thus two cases remain: $ p = t = 0 $ and $ q = r = 0 $.

Substituting $ p = t = 0 $ into equations (3), (4), and (6) and solving for $ r $, $ s $, and $ u $ gives
\begin{equation}
r = \frac{1}{m}, ~~~
s = \frac{1}{m}\left( \epsilon - \frac{n}{q} \right), ~~~
u = \frac{1}{q}
\end{equation}
Likewise, substituting $ q = r = 0 $ into (3), (4), and (6) and solving for $ s $, $ t $, and $ u $ to gives
\begin{align}
s = \frac{1}{p}, ~~~
t = \frac{1}{n}, ~~~
u = \frac{1}{n} \left( \epsilon - \frac{m}{p} \right)
\end{align}
Observe that these two solutions look similar.  In fact, they differ merely by a permutation.  In the first case,  $ \mathbf{W} $ and $ \mathbf{H} $ have the same main diagonal and superdiagonal format as $ \mathbf{A} $, and can be written in matrix notation as
\begin{equation}
\mathbf{WH} = \left[ \begin{array}{cc} \mathbf{w}_0(1) & \mathbf{w}_1(1) \\ & \mathbf{w}_0(2) \end{array} \right] \left[ \begin{array}{cc} \frac{1}{\mathbf{w}_0(1)} & \frac{1}{\mathbf{w}_0(1)}(\epsilon - \frac{\mathbf{w}_1(1)}{\mathbf{w}_0(2)}) \\ & \frac{1}{\mathbf{w}_0(2)} \end{array} \right]
\end{equation}
The second case can be written as $ (\mathbf{WP})(\mathbf{P}^{-1}\mathbf{H}) $, where $ \mathbf{P} $ is the permutation matrix $ \left[ \begin{array}{cc} & 1 \\ 1 & \end{array} \right] $.

All three of the algorithms tested gave solutions of this form 1000 times out of 1000, for each of several values of $ \epsilon $.  The consistency of the solutions enabled further analysis.  The change in the solution can be measured by the change in the three parameters $ \mathbf{w}_0(1) $, $ \mathbf{w}_0(2) $, and $ \mathbf{w}_1(1) $ (ignoring the permutation if present).  Figure 1 shows the change in each of the three parameters from the base case $ \mathbf{A}_0 $ for several different values of $ \epsilon $ when input into Algorithm 1.  Each of the values is the arithmetic mean of the corresponding values generated from 1000 different random seeds.
\begin{figure}
\begin{center}
\includegraphics[scale=0.22]{NMF_fig1v2}
\caption{The figure shows the slope associated with the change in each of the three parameters for each of several values of $ \epsilon $.  As $ \epsilon $ approaches zero on the right of the graph, the values of the slopes converge, showing that for sufficiently small $ \epsilon $, each of the parameters is linear in $ \epsilon $.}
\end{center}
\end{figure}
Of course, the precise values depend on the distribution of randomness used.  But notice that as $ \epsilon $ approaches 0, the values of the three parameters become very nearly linear in $ \epsilon $.  The results for Algorithms 2 and 3 were very similar - they also showed linearity of the parameters in $ \epsilon $, with comparable slopes.

However, $ \mathbf{w}_1(1) $ was not always linear in $ \epsilon $, even for small $ \epsilon $.  In some cases, the difference approached 0 much more quickly.  To see why this occurred, consider that the entries in $ \mathbf{H} $ could have been chosen to be the parameters rather than the entries in $ \mathbf{W} $.  Also, recall that in the base case $ \mathbf{A}_0 $, in which $ \epsilon = 0 $, $ \mathbf{w}_1(1) = \mathbf{h}_1(1) = 0 $ since both entries are off the diagonal.  Thus, when either is linear in $ \epsilon $, they are of the form $ \epsilon x $ for some slope $ x $.  Since the solution is exact, it can be deduced that
\begin{equation}
\mathbf{w}_0(1) \mathbf{h}_1(1) + \mathbf{w}_1(1) \mathbf{h}_0(2) = \epsilon
\end{equation}
Therefore, in the cases that $ \mathbf{w}_1(1) $ approaches 0 very quickly, since $ \mathbf{w}_0(1) $ approaches a large, stable value as $ \epsilon $ approaches 0, $ \mathbf{h}_1(1) $ must be nearly linear in $ \epsilon $.  So in the cases that $ \mathbf{w}_1(1) $ is not linear in $ \epsilon $, its symmetrical counterpart, $ \mathbf{h}_1(1) $, is.  To simplify this complication out of the data, the parameters in $ \mathbf{W} $ were chosen when $ \mathbf{w}_1(1) $ was closer to linearity in $ \epsilon $, and the parameters in $ \mathbf{H} $ were chosen when $ \mathbf{h}_1(1) $ was closer to linearity in $ \epsilon $.

Curiously, although it was possible for $ \mathbf{w}_1(1) $ and $ \mathbf{h}_1(1) $ to ``split'' the nonlinearity so that both were somewhat linear, this rarely occurred.  All three algorithms preferred to make one of them very close to linear at the expense of the other.  When $ \mathbf{w}_1(1) $ approached zero very rapidly, by equations (3) and (4), $ \mathbf{h}_1(1) = \epsilon \mathbf{h}_0(1) $, and similarly, when $ \mathbf{h}_1(1) $ is negligible, $ \mathbf{w}_1(1) = \epsilon \mathbf{h}_0(2) $.

Next, different values for the entries of $ \mathbf{a}_0 $ and $ \mathbf{a}_1 $ were tried, so they had a range of entries rather than all 1's. The algorithms all behaved similarly; up to permutation, they satisfied the following formula
\begin{equation}
\mathbf{WH} = \left[ \begin{array}{cc} \mathbf{w}_0(1) & \mathbf{w}_1(1) \\ & \mathbf{w}_0(2) \end{array} \right] \left[ \begin{array}{cc} \frac{\mathbf{a}_0(1)}{\mathbf{w}_0(1)} & \frac{\mathbf{a}_1(1)}{\mathbf{w}_0(1)}(\epsilon - \frac{\mathbf{w}_1(1)\mathbf{a}_0(2)}{\mathbf{a}_1(1)\mathbf{w}_0(2)}) \\ & \frac{\mathbf{a}_0(2)}{\mathbf{w}_0(2)} \end{array} \right]
\end{equation}
Note that equation (9) is just a special case of this equation in which $ \mathbf{a}_0(1) = \mathbf{a}_0(2) = \mathbf{a}_1(1) = 1 $.   The same phenomena was also observed in which the algorithm usually made one of $ \mathbf{w}_1(1) $ and $ \mathbf{h}_1(1) $ be nearly linear in $ \epsilon $ and the other approach zero rapidly, rather than having both entries be non-negligible.  As long as the entries of $ \mathbf{a}_0 $ and $ \mathbf{a}_1 $ are roughly on the order of 1, the algorithms operated similarly.

The next case examined set $ \mathbf{A} $ to be a $ 3 \times 3 $ matrix.  Using similar logic to the $ 2 \times 2 $ case, it can be deduced that any exact factorization of $ \mathbf{A} $ is likely to be of the form
\begin{equation}
\left[ \begin{array}{ccc} \mathbf{w}_0(1) & \mathbf{w}_1(1) & \\ & \mathbf{w}_0(2) & \mathbf{w}_1(2) \\ & & \mathbf{w}_0(3) \end{array} \right] \left[ \begin{array}{ccc} \mathbf{h}_0(1) & \mathbf{h}_1(1) & \\ & \mathbf{h}_0(2) & \mathbf{h}_1(2) \\ & & \mathbf{h}_0(3) \end{array} \right]
\end{equation}
Indeed, all three algorithms always gave solutions of this form.  In fact, most of the time there were two more zero entries than necessary - either $ \mathbf{w}_1(1) $ or $ \mathbf{h}_1(1) $, and either $ \mathbf{w}_1(2) $ or $ \mathbf{h}_1(2) $.  This is similar to the way that $ \mathbf{w}_1(1) $ or $ \mathbf{h}_1(1) $ often approached 0 rapidly in the $ 2 \times 2 $ case.  To note another similarity to the $ 2 \times 2 $ case, whenever $ \mathbf{w}_1(i) $ was significant and $ \mathbf{h}_1(i) $ was not, $ \mathbf{w}_1(i) $ was very close to $ \epsilon \mathbf{w}_0(i+1) $ - in similar situations $ \mathbf{h}_1(i) $ was approximately $ \epsilon \mathbf{h}_0(i) $.

As a result, there were 4 distinct configurations of the nonzero elements in the solutions, as given by Figure 2.
\begin{figure}
\begin{center}
\begin{tabular}{|c|c|} \hline
Type & equal to 0 \\ \hline
Type I & $ \mathbf{w}_1(1), \mathbf{w}_1(2) $ \\ \hline
Type II & $ \mathbf{h}_1(1), \mathbf{h}_1(2) $ \\ \hline
Type III & $ \mathbf{w}_1(1), \mathbf{h}_1(2) $ \\ \hline
Type IV & $ \mathbf{h}_1(1), \mathbf{w}_1(2) $ \\ \hline
\end{tabular}
\caption{We categorized the solutions when $ \mathbf{A} $ was a $ 3 \times 3 $ matrix by where the non-negligible entries in the solution were.  For each type, this table shows which entries that are usually positive are negligible.}
\end{center}
\end{figure}
Note that Type IV appears to be an inexact solution; since it has positive $ \mathbf{w}_1(1) $ and $ \mathbf{h}_1(2) $, the entry at position $ \mathbf{A}(1,3) = \mathbf{w}_1(1)\mathbf{h}_1(2) $ in the product $ WH $ would have to be nonzero.  However, both $ \mathbf{w}_1(1) $ and $ \mathbf{h}_1(2) $, like all entries on the superdiagonal, are $ O(\epsilon) $, so $ \mathbf{w}_1(1) \mathbf{h}_1(2) $ is $ O(\epsilon^2) $, and is considered  negligible.  In fact, most of the solutions generated by the algorithms had nonzero values for entries that were supposed to be zero, but for this analysis anything below $ O(\epsilon^2) $ was considered negligible.
%We suspect that these negligible values arise because we cut off the algorithms before they were fully able to converge, as that would take a very long time.

Each algorithm was run 100 times on the $ 3 \times 3 $ input with $ \mathbf{w}_0 = [1~1~1] $, $ \mathbf{w}_1 = [1~1] $, and $ \epsilon = 10^{-3} $.  The solutions were categorized by the solution type in Figure 2.  The distributions of the solutions by algorithm type are given in Figure 2.  Note that some solutions did not have two negligible entries among $ \mathbf{w}_1(1) $, $ \mathbf{w}_1(2) $, $ \mathbf{h}_1(1) $, and $ \mathbf{h}_1(2) $, in which case the smaller entry was ignored for the sake of sorting - this accounted for about 20\% of the three algorithms, the majority occurring in Algorithm 1.
\begin{figure}
\begin{center}
\includegraphics[scale=0.45]{NMF_fig2}
\caption{Categorized the solutions for $ \mathbf{A} $  being a $ 3 \times 3 $ matrix by where the non-negligible entries in the solution were.  This chart shows how often each algorithm generated a solution of each type out of 100 cases.  Type II (in which $ \mathbf{H} $ is diagonal) was the most common among all the algorithms, but by differing amounts.}
\end{center}
\end{figure}
It is significant to note that even the solutions that didn't fall cleanly into a ``type'' still satisfied the pattern given in (12).  It seems that an NMF algorithm should satisfy this pattern, but little more is required.

Next, entries in $ \mathbf{a}_0 $ and $ \mathbf{a}_1 $, were changed as in the $ 2 \times 2 $ case.  As long as the entries were $O(1)$  (as opposed to $ O(\epsilon) $ or $ O(\frac{1}{\epsilon}) $), the behavior of the algorithms was similar.

Finally, $ \mathbf{A} $ larger than $ 3 \times 3 $ were examined.  Several different sizes of matrices were tested, ranging from $ 4 \times 4 $ to $ 20 \times 20 $, always keeping $ \mathbf{A} $, $ \mathbf{W} $, and $ \mathbf{H} $ square, with positive entries only on the main diagonal and the superdiagonal.  The experiments followed the same general pattern; nonzero entries in $ \mathbf{W} $ and $ \mathbf{H} $ appeared only on the main diagonal and superdiagonal.  Using similar logic to the $ 2 \times 2 $ and $ 3 \times 3 $ cases, it can be shown that these are the only exact solutions.  However, in practice, as the matrices get larger, exceptions to this pattern become more common, particularly in Algorithm 3.  The general rule seems to mostly hold (over half the time) until $ \mathbf{A} $ becomes around $ 20 \times 20 $.  Note, however, that because the run-time of the algorithms are cubic in the size of the matrix, at best, the sample size for large matrices is small.

\section{Proposed Tests for NMF Algorithms}
Since all three algorithms, which cover a variety of approaches to NMF, had a lot in common in their solutions, it is propose that these inputs $ \mathbf{A} $ could be used as a test case of an NMF algorithm implementation.  In this section, it is proposed how such test cases could be executed.

The test begins with input of the form
\begin{equation}
\mathbf{A} = \mathbf{A}_0 + \epsilon \mathbf{A}_1 = \text{diag}(\mathbf{a}_0) + \epsilon \text{diag}(\mathbf{a}_1,1)
\end{equation}
$ \mathbf{A} $ is square, and preferably somewhere between $ 3 \times 3 $ and $ 8 \times 8 $ in size, although bigger inputs may be useful as well.  The entries should vary between tests.  Each test should start by using $ \epsilon = 0 $ so that $ \mathbf{A} $ is diagonal.  The results of this test should have $ \mathbf{W} $ and $ \mathbf{H} $ monomial - only one nonzero element in each row and column.  Ignore entries that are below $ O(10^{-10}) $, for the entirety of testing, as any such entries are negligible.

If $ \mathbf{W} $ or $ \mathbf{H} $ is not monomial, or if the product $ \mathbf{WH} $ is not equal to $ \mathbf{A} $ to within a negligible margin of error, the algorithm fails the test.  Otherwise, the generated solution can be used to find the permutation matrix $ \mathbf{P} $ that makes $ \mathbf{WP} $ and $ \mathbf{P}^{-1}\mathbf{H} $ diagonal by replacing the nonzero entries of $ \mathbf{H} $ with 1's.  Since $ \mathbf{A} = \mathbf{WH} $ is diagonal, $ \mathbf{WP} $ is also diagonal, and since $ \mathbf{I} = \mathbf{P}^{-1}\mathbf{P} $ is diagonal, so is $ \mathbf{P}^{-1}\mathbf{H} $.  Knowing $ \mathbf{P} $ will make the rest of the testing much simpler since it is easier to identify whether a solution is of the form given above when it is not permuted.

Next, run the test again using a positive value for $ \epsilon $; $ \epsilon = 10^{-3} $ seems to work well, although using a variety of $ \epsilon $ is also recommended.  Make sure to use the same random seeds that were used in the $ \epsilon = 0 $ test to produce corresponding output.  Then check that the $ \mathbf{W} $ and $ \mathbf{H} $ given by the algorithm are such that $ \mathbf{WP} $ and $ \mathbf{P}^{-1}\mathbf{H} $ have nonzero entries only on the two diagonals that they are supposed to.  If this doesn't hold, changing $ \epsilon $ might have changed which permutation returns $ \mathbf{W} $ and $ \mathbf{H} $ to the proper form, so check again; this happens more commonly among larger matrices than smaller ones.  However, if $ \mathbf{W} $ and $ \mathbf{H} $ really do break the form, or $ \mathbf{A} \neq \mathbf{WH} $, the algorithm fails the test on this input.  Otherwise, it passes.

Note that even widely accepted algorithms do fail these tests occasionally, especially with matrices larger than $ 8 \times 8 $, so it's advisable to perform the test many times to get a more accurate idea of an algorithm's performance.

%Note that we didn't check for some patterns we found in the previous section, such as the preference for Type II in the $ 3 \times 3 $ case, because they can vary from algorithm to algorithm and we have no mathematical reason to think they should hold in general.

% An example of a floating figure using the graphicx package.
% Note that \label must occur AFTER (or within) \caption.
% For figures, \caption should occur after the \includegraphics.
% Note that IEEEtran v1.7 and later has special internal code that
% is designed to preserve the operation of \label within \caption
% even when the captionsoff option is in effect. However, because
% of issues like this, it may be the safest practice to put all your
% \label just after \caption rather than within \caption{}.
%
% Reminder: the "draftcls" or "draftclsnofoot", not "draft", class
% option should be used if it is desired that the figures are to be
% displayed while in draft mode.
%
%\begin{figure}[!t]
%\centering
%\includegraphics[width=2.5in]{myfigure}
% where an .eps filename suffix will be assumed under latex, 
% and a .pdf suffix will be assumed for pdflatex; or what has been declared
% via \DeclareGraphicsExtensions.
%\caption{Simulation results for the network.}
%\label{fig_sim}
%\end{figure}

% Note that the IEEE typically puts floats only at the top, even when this
% results in a large percentage of a column being occupied by floats.


% An example of a double column floating figure using two subfigures.
% (The subfig.sty package must be loaded for this to work.)
% The subfigure \label commands are set within each subfloat command,
% and the \label for the overall figure must come after \caption.
% \hfil is used as a separator to get equal spacing.
% Watch out that the combined width of all the subfigures on a 
% line do not exceed the text width or a line break will occur.
%
%\begin{figure*}[!t]
%\centering
%\subfloat[Case I]{\includegraphics[width=2.5in]{box}%
%\label{fig_first_case}}
%\hfil
%\subfloat[Case II]{\includegraphics[width=2.5in]{box}%
%\label{fig_second_case}}
%\caption{Simulation results for the network.}
%\label{fig_sim}
%\end{figure*}
%
% Note that often IEEE papers with subfigures do not employ subfigure
% captions (using the optional argument to \subfloat[]), but instead will
% reference/describe all of them (a), (b), etc., within the main caption.
% Be aware that for subfig.sty to generate the (a), (b), etc., subfigure
% labels, the optional argument to \subfloat must be present. If a
% subcaption is not desired, just leave its contents blank,
% e.g., \subfloat[].


% An example of a floating table. Note that, for IEEE style tables, the
% \caption command should come BEFORE the table and, given that table
% captions serve much like titles, are usually capitalized except for words
% such as a, an, and, as, at, but, by, for, in, nor, of, on, or, the, to
% and up, which are usually not capitalized unless they are the first or
% last word of the caption. Table text will default to \footnotesize as
% the IEEE normally uses this smaller font for tables.
% The \label must come after \caption as always.
%
%\begin{table}[!t]
%% increase table row spacing, adjust to taste
%\renewcommand{\arraystretch}{1.3}
% if using array.sty, it might be a good idea to tweak the value of
% \extrarowheight as needed to properly center the text within the cells
%\caption{An Example of a Table}
%\label{table_example}
%\centering
%% Some packages, such as MDW tools, offer better commands for making tables
%% than the plain LaTeX2e tabular which is used here.
%\begin{tabular}{|c||c|}
%\hline
%One & Two\\
%\hline
%Three & Four\\
%\hline
%\end{tabular}
%\end{table}


% Note that the IEEE does not put floats in the very first column
% - or typically anywhere on the first page for that matter. Also,
% in-text middle ("here") positioning is typically not used, but it
% is allowed and encouraged for Computer Society conferences (but
% not Computer Society journals). Most IEEE journals/conferences use
% top floats exclusively. 
% Note that, LaTeX2e, unlike IEEE journals/conferences, places
% footnotes above bottom floats. This can be corrected via the
% \fnbelowfloat command of the stfloats package.




\section{Conclusion}
This paper proposes an approach to the problem of testing NMF algorithms by running the algorithms on simple input that can produce an exact non-negative factorization, and perturbations of such input.  In particular,  square matrices with $O(1)$ entries on the main diagonal and $ O(\epsilon)$ entries on the superdiagonal are proposed, because they have exact solutions that can enumerated mathematically, or because they are perturbations of matrices with exact solutions.
%, and because they're simple enough for useful algorithms to find exact solutions.

The test cases have been used as input on three known NMF algorithms that represent a variety of algorithms, and all of them behaved similarly, which suggests testable, quantifiable behaviors that many NMF algorithms share.  These test cases offer one approach for testing candidate NMF implentations to help determine whether it behaves as it should.



% conference papers do not normally have an appendix


% use section* for acknowledgment
\section*{Acknowledgment}
The authors would like to thank Dr. Alan Edelman for providing and overseeing this research opportunity, and Dr. Vijay Gadepally for his advice and expertise.





% trigger a \newpage just before the given reference
% number - used to balance the columns on the last page
% adjust value as needed - may need to be readjusted if
% the document is modified later
%\IEEEtriggeratref{8}
% The "triggered" command can be changed if desired:
%\IEEEtriggercmd{\enlargethispage{-5in}}

% references section

% can use a bibliography generated by BibTeX as a .bbl file
% BibTeX documentation can be easily obtained at:
% http://mirror.ctan.org/biblio/bibtex/contrib/doc/
% The IEEEtran BibTeX style support page is at:
% http://www.michaelshell.org/tex/ieeetran/bibtex/
%\bibliographystyle{IEEEtran}
% argument is your BibTeX string definitions and bibliography database(s)
%\bibliography{IEEEabrv,../bib/paper}
%
% <OR> manually copy in the resultant .bbl file
% set second argument of \begin to the number of references
% (used to reserve space for the reference number labels box)
\begin{thebibliography}{6}

\bibitem{ALS}
Berry Browne~Langville Pauca and Plemmons, \emph{Algorithms and applications
  for approximate nonnegative matrix factorization}, Computational Statistics
  and Data Analysis \textbf{52} (2007), 155--173.

\bibitem{Gemulla:2011:LMF:2020408.2020426}
Rainer Gemulla, Erik Nijkamp, Peter~J. Haas, and Yannis Sismanis,
  \emph{Large-scale matrix factorization with distributed stochastic gradient
  descent}, Proceedings of the 17th ACM SIGKDD International Conference on
  Knowledge Discovery and Data Mining (New York, NY, USA), KDD '11, ACM, 2011,
  pp.~69--77.
  
\bibitem{proof}
John Gilbert, personal communication, Sep 2015.

\bibitem{GradDes}
N.~Guan, D.~Tao, Z.~Luo, and B.~Yuan, \emph{Nenmf: An optimal gradient method
  for nonnegative matrix factorization}, IEEE Transactions on Signal Processing
  \textbf{60} (2012), no.~6, 2882--2898.

\bibitem{NIPS2000_1861}
Daniel~D. Lee and H.~Sebastian Seung, \emph{Algorithms for non-negative matrix
  factorization}, Advances in Neural Information Processing Systems 13 (T.~K.
  Leen, T.~G. Dietterich, and V.~Tresp, eds.), MIT Press, 2001, pp.~556--562.

\bibitem{NIPS2005_2757}
Suvrit Sra and Inderjit~S. Dhillon, \emph{Generalized nonnegative matrix
  approximations with bregman divergences}, Advances in Neural Information
  Processing Systems 18 (Y.~Weiss, B.~Sch\"{o}lkopf, and J.~C. Platt, eds.),
  MIT Press, 2006, pp.~283--290.

\bibitem{wang2010instantaneous}
Wenwu Wang, \emph{Instantaneous vs. convolutive non-negative matrix
  factorization: Models, algorithms and applications}, Machine Audition:
  Principles, Algorithms and Systems: Principles, Algorithms and Systems
  (2010), 353.

\end{thebibliography}


% that's all folks
\end{document}