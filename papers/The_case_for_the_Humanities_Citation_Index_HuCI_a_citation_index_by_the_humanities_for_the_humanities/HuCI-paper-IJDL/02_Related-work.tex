\section{Related Work}\label{sec:related-work}

\subsection{On scholarly information retrieval in the A\&H}\label{sec:scholarly-IR}

The needs and behaviours of humanities scholars in terms of information seeking has been an active area of study especially in the field of Library and Information Science (LIS), where research on this topic started in the 1980s and early 1990s \cite{stone_humanities_1982, ellis_behavioural_1989, watson-boone_information_1994}. For a thorough review of the early literature on this topic see \cite{wiberley_jr_humanities_2009}[p. 2198] and \cite{benardou_understanding_2010}[pp. 19-21].
Determining the information needs and behaviours of humanities scholars was essential for librarians in order to support scholars in their research by devising new library systems or by improving the guidelines for abstracting publications to cater for the specific needs of humanities
scholars \cite{tibbo_abstracting_1993}. What emerges from this literature are also the key strategies for finding bibliographic information that characterise humanities scholarship. Firstly, scholars use proper names extensively when searching as compared with scholars in other disciplines \cite{wiberley_patterns_1989, bates_getty_1996, palmer_scholarly_2009}. Secondly, a prominent behaviour among humanities scholars is to search for bibliographic information by browsing \cite{bates_getty_1996, ellis_behavioural_1989, meho_modeling_2003}. A typical example is browsing books in the stacks or shelves of a library. What characterises browsing as opposed to a targeted search is that it favours the serendipitous discovery of relevant information: the physical proximity of books on library shelves, which is related to their subject classification, may in some cases transcend the boundaries of subjects. Finally, a third prominent search strategy is the already mentioned citation chaining with its two variants of backward and forward chaining \cite{ellis_behavioural_1989,buchanan_information_2005}. The former consists of starting from one publication – the seed document -- and then following up the references it contains in order to expand the initial search and to discover other related publications. The latter consists of starting from a seed document and then finding which other publications cite it. Moreover, an empirical study of the information seeking strategies of humanities scholars reports that searching and browsing proved to be rather ineffective strategies for locating information and that citation chaining was the most common behavioural pattern \cite{buchanan_information_2005}[pp. 227–228].

\subsection{Citation indexing and the Humanities}\label{sec:citation-indexing}

Citation indexing is commonplace for Science, Technology, Engineering and Mathematics (STEM) literature. Mainstream indexes such as Google Scholar, the Web of Science, Scopus, Dimensions or Semantic Scholar are largely capable of indexing most citations accurately. To be sure, their coverage is still uneven and far from uniform \cite{martin-martin_google_2021, visser_large-scale_2021}. One of the critical problems which are left open is the uneven coverage of different disciplines, with humanities disciplines usually faring worse than most \cite{harzing_google_2016}. Several reasons for this state of affairs have been individuated, which can be grouped into two categories. Intrinsic factors, which depend on the literature itself, and extrinsic factors, which depend on the information environment where citation mining is performed \cite{colavizza_citation_2019-1}. 

\textit{Intrinsic factors} which act as obstacles to citation indexing in the humanities include the more limited availability of born digital or digitized publications, a higher variety of languages and publication venues in use, the practice to publish monographs, complex referencing practices and motivations which limit their automatic processing. These topics have been amply discussed in the literature \cite{kulczycki_publication_2018, hicks_difficulty_1999, nederhof_bibliometric_2006, huang_characteristics_2008, santos_citing_2021}. \textit{Extrinsic factors} have been less the focus of previous work and include, instead, the variety and fragmentation of catalogs, information systems and other sources of unique identifiers and authoritative metadata. These issues are well-known more generally in the Galleries, Libraries, Archives, and Museums (GLAM) sector. A recent study on metadata aggregation highlights several characteristics of this landscape, among which these fall within what we here refer as extrinsic factors \cite{freire_cultural_2020}:
\begin{itemize}
\item Each GLAM sub-domain (libraries, archives and museums) applies its specific resource description practices and data models.
\item All sub-domains embrace the adoption and definition of standards-based solutions addressing description of resources, but to different extents.
\item Interoperability of systems and data is scarce across sub-domains, but it is somewhat more common within each sub-domain, at the national and the international levels.
\end{itemize}

As a consequence of the limitations enacted by both intrinsic and extrinsic factors, it is more difficult to comprehensively index the humanities via citations, a condition that limited the use of quantitative bibliometric methods in this area \cite{ardanuy_sixty_2013}, despite clear progress over recent time \cite{petr_journal_2021, hammarfelt_beyond_2016}. The lack of a comprehensive and reliable citation index remains a known and open problem in the humanities \cite{heinzkill_characteristics_1980, linmans_why_2009, sula_citations_2014}. Our contribution proposes a way forward which mainly addresses the obstacles posed by extrinsic factors, and is true to the way the humanities communicate research and retrieve scholarly information.