\section{The need for a citation index for the humanities}\label{sec:citation-index-AH-needs}

Scholarship in the humanities rests on solid traditions, most crucially developed in the archives, libraries and information studies communities. It is thus worth asking the question: \textbf{why do we need a citation index for the humanities?} We advance four motivations: to dramatically improve current scholars’ information retrieval capabilities; to interlink presently siloed GLAM information systems; to foster best practices in terms of referencing and metadata; to provide research data for bibliometrics and science studies.

\subsection{Improve scholarly information retrieval}

From an information retrieval point of view, citation indexes seem to be the natural evolution of disciplinary and thematic bibliographies (e.g., the \textit{Annual Bibliography of English Language and Literature}\footnote{\url{https://www.mla.org/Publications/MLA-International-Bibliography}.} or \textit{L'Année Philologique}\footnote{\url{https://about.brepolis.net/lannee-philologique-aph}.}), which are widely used by scholars across the humanities to conduct literature search. A citation index, in fact, can be seen a bibliography whose entries are linked with one another depending on the citations that are found in the full-text of the catalogued publications. Moreover, thematic bibliographies such as the \textit{World Shakespeare bibliography}\footnote{\url{https://www.worldshakesbib.org}.} or the \textit{International Dante Bibliography}\footnote{\url{[https://bibliografia.dantesca.it](https://bibliografia.dantesca.it)}.} often provide users with the ability to search for publications related to specific literary works -- a functionality that could also be provided by a citation index which captures references to primary sources.

Despite the existence of bibliographies and bibliographic databases, humanities scholars cannot yet fully rely on citation indexes when searching for secondary literature, nor to keep up to date with recent developments (e.g., via citation alerts). As we highlighted above, it is the limited coverage of existing citation indexes more than any intrinsic limitation that has been the decisive factor in discouraging their more systematic adoption in retrieval practices. This need not be a sealed fate. Assuming sufficient coverage, in both quality and quantity, a citation index for the humanities can first and foremost serve the same information retrieval needs these tools provide for in the sciences since decades. It is likely that a non-negligible fraction of humanities scholars already uses services such as Google Scholar and Google Books \cite{chen_exploring_2019}, even in the absence of comprehensive evidence on their coverage and reliability.
 
Furthermore, a variable yet non-negligible amount of references in the humanities are given to primary sources, such as archival documents or literary works \cite{knievel_citation_2005}. There has never been a way to count and retrieve all references to a given primary source without painstaking manual work. Knowledge about primary sources, in terms of their existence, location and means of access, takes up a substantial amount of time and training in the humanities, sometimes becoming all too treasured. In principle, both primary and secondary sources should be indexed in the humanities citation index. This will allow anyone to immediately gauge which sources have been used together, where and by whom. In practice, several open challenges will need to be overcome first, including programmatic access to uniform GLAM metadata.

\subsection{Interlink GLAM collections via citations}\label{sec:interlinking-collections-via-citations}

GLAM information ecosystems often exist in isolated silos: metadata and data are largely made accessible by the specific institution that creates and curates them. Notable exceptions exist, for example national library catalogs and projects such as Europeana. Nevertheless, to the best of our knowledge, no encompassing information retrieval infrastructure exists spanning across GLAM institutional categories, for example interlinking libraries (L) with archives (A). Citation links extracted from scholarly literature can do just that.
 
The literature in the humanities in fact contains a wealth of references to primary sources, accumulated over centuries of scholarly work. Within the scope of one project alone, some of the authors were able to extract nearly 700,000 references to primary sources from approximately 1900 books and 5500 journal articles (Venice Scholar Index\footnote{\url{https://venicescholar.dhlab.epfl.ch}} \cite{colavizza_linked_2018}). Citation links connect secondary literature, hence library catalogs, with  information systems of archives, galleries and museums. They also connect archives, galleries and museums directly by virtue of co-citation relationships (i.e., two resources are connected if they are cited together by a third one). These links effectively constitute a dormant virtual information system which awaits to be digitally materialized. By so doing, a significant acceleration and democratization to the access of primary sources can be realized, contributing to a broader scholarly and public engagement with these collections as it is currently the case.
 
Digitally materializing citation links would create incentives to make GLAM information retrieval and research infrastructure increasingly more interoperable and interdependent, to the great benefit of the research community. Citation indexing requires publication data and metadata, which must be made available by publishers and GLAM institutions. We argue that, once the benefits of citation indexing will have been made tangible to a sufficient degree, this will create a positive feedback loop for all stakeholders to gradually improve on their practices in order to make citation indexing increasingly easier and to a large extent automatic.

\subsection{Improve current practices}
The automatic extraction and indexing of structured information, such as bibliographic citations, typically requires a high degree of openness and standardization in the ecosystem it happens in. Citation indexing requires open, standardized and programmatically accessible metadata about primary and secondary sources alike, as well as access to the full text of scholarly publications. It also benefits from a high degree of uniformity in the referencing practices of authors, which makes reference parsing all the more feasible. Yet, all this is costly, hard and time consuming. For all stakeholders to strive to higher openness, standardization and accessibility, we require a positive incentive. We argue that citation indexing, once it reaches a certain threshold, actually provides for one: if a community starts using citation indexes for information retrieval, being indexed increasingly becomes a necessity, hence related investments will be made.
 
Citation indexing starts with authors. Referencing practices, sometimes less than uniform and coherent, pose a significant challenge to the automatic extraction of citations (e.g., \cite{nederhof_bibliometric_2006}). Yet, once references become data, and their value as links is immediately made tangible via citation indexing, authors might have more incentives to make their referencing practices syntactically and stylistically more uniform in view of improving their harvesting and correct indexing.%\footnote{On this matter, also see the results and recommendations of the Citation Capture report (Research Libraries UK et al. 2018).} 
 
A similar point in case can be made for publishers. On the one hand, proof-checking work can make sure to provide for uniform references with sufficient information for their indexing, similarly to what is provided by several scientific publishers. On the other hand, and more importantly, publishers could sign up (and effectively contribute) to the Crossref and OpenCitations initiatives, making their metadata and citation data available. The existence of a citation index for the humanities should foster participation in such initiatives. Failing that, or considering the backlog of already published publications (especially if printed), GLAM institutions themselves can take a leading role, as we discuss below.
 
The positive incentive to expose open, standardized and programmatically accessible metadata provided by the citation index will also apply to GLAM institutions, once the benefits of interlinked collections and increased searchability will become apparent. A crucial challenge for us will be to reach a critical mass of citation data to provide for an indispensable service to a sizable share of the research community and, at the same time, initiate the positive incentive for all stakeholders.

\subsection{Research data for bibliometrics and science studies}
 
It is well known that the humanities are significantly understudied by the bibliometrics and quantitative science studies community, largely because of the lack of citation data \cite{colavizza_citation_2019}. This has several consequences, among which the separation of qualitative studies on the humanities from analyses grounded in (bigger) data \cite{franssen_science_2019}. Furthermore, it also causes a widespread science-as-the-norm/humanities-as-an-exception mindset in bibliometrics and research evaluation as a whole, as if it were the case that citations cannot be used to study the humanities. To be sure, indexing citations in the humanities is challenging, yet it would allow the bibliometrics and quantitative science studies communities to finally approach the humanities on equal ground with respect to the sciences.
The proposed citation index for the humanities can radically alter this state of affairs. First of all, a bibliometrics for the humanities grounded in data as well as theory could finally be developed, in full recognition of the specificities of the humanities \cite{hammarfelt_beyond_2016}. Secondly, citation data in the humanities is very rich, if we consider the varied publication typologies, languages, primary and secondary sources that come into play. As a consequence, citation data from the humanities will require novel methods and approaches that might not only provide insights into these data, but as well inform further developments when applied to citation data from the sciences. The HuCI can essentially put an end to the age of the so-called ``non-bibliometric'' humanities.

Citation data, once available, have been used for research evaluation. Indicators such as citation counts or the H-index, are widespread and have been amply discussed by the bibliometrics community \cite{waltman_review_2016}. Recently, public efforts have been made to call for a redress and improvement in the use of citation-based indicators \cite{hicks_bibliometrics_2015}.\footnote{Also see the San Francisco Declaration on Research Assessment (DORA): \url{https://sfdora.org/read}.} It will be likely unavoidable to face similar discussions if and when the HuCI materializes. We believe these worries should not prevent it from happening, for the very reasons we just detailed. Furthermore, HuCI could provide for an opportunity to rethink the way we use citation-based indicators in research evaluation. The humanities have a long-lasting tradition of peer review assessment which, when mixed with situated and contextualized metrics (which in turn need not be just citation-based), has the potential to inform research evaluation in the  sciences too.