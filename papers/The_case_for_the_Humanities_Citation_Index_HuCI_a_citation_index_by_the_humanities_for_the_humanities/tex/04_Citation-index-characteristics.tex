\section{The characteristics of a citation index for the A\&H}\label{sec:citation-index-characteristics}

Having clarified why we believe a citation index for the humanities is motivated, we detail here four requirements we propose it should have. These are: comprehensive source coverage and chronological depth, rich information provided to contextualize citations, and a growth strategy driven by institutional collections. We note that we intend these requirements as something to aspire to: they represent end goals more than necessary conditions to begin with.

\subsection{Source coverage}
Scholars in the humanities use a complementary variety of publication typologies, such as monographs, journal articles and contributions in edited volumes. Journal articles, the main focus of existing commercial citation indexes, in general account for a small fraction of the output in all the humanities \cite{knievel_citation_2005}. We thus argue that the first requirement of a citation index for the humanities is complete coverage in terms of publication typologies. A related requirement, or pain point, is multilingualism. Scholarly literature in national languages abounds in most of the humanities, yet this variety is not often captured by digital resources. A case in point is the situation of Classics: 75\% of Classics publications contained in JSTOR are written in English, while the language of publications reviewed in L’Année Philologique (APh, the most important bibliography in this field) is much more evenly distributed between English, German, Italian and French. In fact, Scheidel \cite{scheidel_continuity_1997} reports that, of the publications reviewed by APh in 1992, 30\% were written in English, roughly 25\% in Italian, 20\% in French and 20\% in German. Ideally, language should not be a source of bias in the citation index.\footnote{Promoting measures against language bias in the context of research assessment is one of the three key recommendations made by the Helsinki Initiative \cite{federation_of_finnish_learned_societies_helsinki_2019}.}
 
As we anticipated above, the second requirement we put forth is the full indexation of citations to primary sources. Interestingly, this requirement compels a discussion of citation granularity: what is the object of a reference which should be considered in a citation index? Typically, for secondary literature we use the level of the work in FRBR terms \cite{ifla_functional_1997}. Hence, citations are accumulated for, say, a journal article aggregating over all its expressions (e.g., in pre-print and printed versions) or for a book over all its editions, excluding those with major revisions that justify calling it a new work. For primary sources, we typically consider unique items (the lowest FRBR level), for example archival documents or unique artworks. We consider instead works when, say, dealing with critical editions of a classic author, where the editing activity is considered scholarly and the source is printed into editions. All this to say that the choice of the citation aggregation object is far from straightforward for primary sources, and a good rule of thumb is that further aggregation is always possible, while disaggregation can be more difficult to undo. Hence, we recommend lower FRBR citation aggregation levels when in doubt.

\subsection{Chronological depth}
The humanities are known to publish at a relatively slower pace than other sciences and to keep citing older relevant literature (e.g., \cite{nederhof_bibliometric_2006, hellqvist_referencing_2010}). This has two consequences for the citation index: first, and foremost, it is crucial to index older literature as well, spanning back ideally to when systematic scholarly referencing became commonplace \cite{grafton_footnote_1999}. Secondly, and this is not a requirement but an opportunity we highlight, digitizing and making openly available old and out-of-copyright literature, in conjunction with its indexation via citations, would constitute a great service to scholars. It would not only improve the use of such literature, but open up opportunities to study the history of scholarship in the humanities at unprecedented scale and comprehensiveness.

\subsection{Rich in context}
Previous work has elucidated how the citation semantics in the humanities tend to be rich and varied \cite{hellqvist_referencing_2010}. This is of crucial importance when using citations for information retrieval: is a citation supportive or dismissing? Is it contextual, perfunctory or does it substantially underpin an argument? The citation index we propose will need to make every effort possible to offer its users all the means necessary to appreciate and understand every citation link. This is mainly done by providing relevant context, within the bounds of existing copyrights.
 
Citation contexts are the excerpts of text preceding and following a citation. The most common context, in this sense, is the sentence where a citation is made. Nevertheless, a context can cover any relevant span, e.g., a few sentences or a whole paragraph. Another source of contextual information is given by proximal co-citations: which other sources are cited with the one under consideration, within the same citing publication? Lastly, providing the exact details of the citation, such as the page number it refers to, also helps to specify its scope. It is possible to see citations and their contexts in aggregate, from the point of view of either the citing or cited sources. This is the case when we consider, for example, all the other sources a given source is co-cited with. It is also possible to consider every citation as situated in a quite specific location of a publication. For example, by considering co-cited sources within the same paragraph of a well-defined citing publication. Both views, the aggregate and the detail, provide for relevant contextual information for a scholar to interpret citation links, and use them for information retrieval.

\subsection{Collection driven}
We conclude this section not by discussing a requirement, but by suggesting a growth strategy for HuCI. Mainstream citation indexes convey the impression, and sometimes the illusion, of comprehensive coverage. Only when we are able to trust a citation index in this sense, we, as scholars, can rely on it for our work. If a citation index is manifestly incomplete, and especially if what is missing is unknown or hard to qualify, it will be difficult for it to succeed. Given the daunting task we have set ourselves to with the humanities citation index and the stated requirements, we also need a reasonable growth strategy. Our proposal is to be topic/collection driven. That is to say, we recommend to index topically coherent batches of scholarly literature, by leveraging the specialized collections of research libraries.
 
In our previous work on the historiography of Venice, we faced the task of defining the limits of what pertains to this topic and what can be left outside. By relying on a set of finding aids – library catalogs, bibliographies, shelving strategies and specialized collections – we were able to create a coherent citation corpus \cite{colavizza_references_2018}. We suggest here that this approach can make HuCI scale, one topic/collection at the time. In so doing, the citation index can gradually serve more and more and larger and larger humanities communities.

Creating the humanities citation index requires not only a growth strategy, but first and foremost a research infrastructure which provides for the right affordances to build the index as a collaborative, distributed and open effort. We propose its design in what follows.

\subsection{Metadata ecosystem and requirements}\label{sec:metadata-ecosystem-requirements}

In the research infrastructure needed to build an A\&H citation index, libraries play a key role not only as holders of digitized collections but also as potential providers of data that can greatly support the citation extraction process. In fact, library catalogues constitute highly valuable knowledge bases of bibliographic information that can be exploited when doing citation mining, and especially citation matching. 

We identify a set of key technical requirements that need to be met if library catalogue metadata are to be seamlessly integrated into the HuCI infrastructure. These requirements are:
\begin{enumerate}
    \item ability to handle the heterogeneity of metadata formats;
    \item provision of unique persistent identifiers;
    \item machine-aided creation, delivery and exchange of metadata;
    \item fine-grained/granular metadata descriptions;  
    \item open licensing of metadata.
\end{enumerate}

\textbf{Metadata formats}. From the point of view of citation mining pipelines and processes there is a need to have metadata expressed in concise and ``easy-to-process'' formats. Such concerns become even more relevant when the metadata processing happens at a large scale, as the needs arise for optimising processing time and for efficient data storage. For example, in the context of previous work carried out by some of the authors \cite{colavizza_linked_2018,colavizza_citation_2019}, the Central Institute for the Union Catalogue of Italian Libraries and for Bibliographic Information (ICCU) has created a dump of 15 million records by transforming its data from MARC to a JSON-based representation, so as to facilitate their use in the project’s citation mining pipeline. Along similar lines, \cite{bergamin_new_2018} have successfully tested a workflow for mapping ICCU’s UNIMARC data onto Wikibase Data Model, which would allow for using Wikibase as an environment to manage and edit bibliographic data, as well as exposing such data in an easier to process format.  

MARC, however, is only one of the many formats that characterise the landscape of library metadata, where a plethora of old and new formats co-exist \cite{tennant_bibliographic_2004}. This situation makes it seem rather unlikely that libraries will converge to a common and widely-adopted metadata format in the near future. As a result, a key requirement of the HuCI infrastructure is the ability to handle this heterogeneity of bibliographic metadata formats, achievable by developing code modules that read these formats and map them onto a common one. 

\textbf{Provision of unique identifiers}. In addition to the granularity of descriptions, the provision of unique, persistent identifiers to identify bibliographic resources is another key requirement for metadata that are meant to support citation mining processes. Ideally, any primary or secondary source of which we are interested in tracking the citations ought to be identifiable by means of a unique, persistent identifier (e.g., a resolvable URI). Naturally, what is considered as a primary source varies from domain to domain: archival documents in History, various types of texts in Classics (e.g., canonical, papyri, inscriptions), manuscripts in Medieval Literature Studies, inscriptions and papyri in Egyptology, and so forth. Once these identifiers are in place, it is possible to use them to link `disambiguated' citations. However, it cannot be the task of a single project to mint and provide these identifiers. This process should be happening in each discipline -- and it has already been happening over the past years e.g., in Classics \cite{romanello_using_2017} -- but it can be fostered and accelerated by large-scale initiatives involving libraries and cultural heritage institutions, such as the European Open Science Cloud \cite{hellstrom_second_2020}.

\textbf{Machine-aided creation, delivery and exchange of metadata.} There is an urgent need to take humans \textit{out} of the loop insofar as access to and exchange of library metadata are concerned. Libraries -- and especially aggregators of library metadata (e.g., national aggregators, library consortia, etc.) -- ought to provide, at the very least, regular data dumps of their bibliographic metadata so as to facilitate their consumption and further reuse. Data dumps, however, being frozen snapshots of a dataset, raise the issue of synchronisation between the data at the source and the copy of the data used by other systems and processes. A partial solution to this problem is to provide streams of data (e.g., via APIs) in addition to regular dumps. 

\textbf{Granularity of bibliographic descriptions.} The granularity of bibliographic descriptions is an apt example of gaps currently existing between the needs and requirements of citation mining projects, on the one hand, and the cataloguing practices currently adopted by the majority of libraries, on the other hand. Types of publications where granularity matters the most are journal articles, book chapters and individual essays within collective volumes. In fact, while the citation unit of such publications is often the most granular (e.g., a given journal article, as opposed to the entire journal), cataloguing practices often do not reach that level of granularity in bibliographic descriptions. 

\textbf{Open licensing.} Despite a declared willingness to share, often libraries and other cultural heritage institutions make available online data dumps that do not come with explicitly defined (open) licenses. They ought to be encouraged to always provide explicit license statements, as their absence hinders the reuse of shared data by others.