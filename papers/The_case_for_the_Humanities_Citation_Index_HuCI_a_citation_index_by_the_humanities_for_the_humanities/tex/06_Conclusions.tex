\section{Conclusions} \label{sec:conclusions}

In this article we have listed the main aspects that are necessary to devise the creation of a Humanities Citation Index (HuCI). We propose HuCI to be a decentralised and federated research infrastructure for gathering, sharing, elaborating, exposing bibliographic metadata and citation data of Humanities publications, that offers hooks for the development of further applications to keep track of the evolution of the Humanities research. 

The technical guidelines we have provided for the creation of such an infrastructure follows current trends shared by the Open Science community around the globe. Several of the principles regarding data sharing we proposed are grounded in existing guidelines such as the FAIR (findability, accessibility, interoperability, and reusability) data principles \cite{wilkinson_fair_2016}, which are considered a common and shared good practice in the field -- where the word \emph{data} in this context is an umbrella term including research data spreadsheets, software, workflows, slides, and other research objects that accompany a \emph{traditional} publication (e.g., a book, a journal article, a conference paper).

Several guidelines for enabling the creation of new open infrastructures -- including their technological compliance, plans for their long-term sustainability and governance -- have been proposed in the past five years, and have directly guided our work on HuCI. The Principles for Open Scholarly Infrastructures \cite{bilder_principles_2020}, the work done by the Confederation Of Open Access Repositories (COAR) on best practices for implementing digital repositories \cite{coar_wg_next_generation_repositories_behaviours_2017, confederation_of_open_access_repositories_coar_2020}, and other principles proposed by independent scholars such as the TRUST (transparency, responsibility, user focus, sustainability and technology) principles \cite{lin_trust_2020} have been extensively reused and adapted to devise the various component of the technical research infrastructure in HuCI. The very same principles characterise several national and international initiatives, such as the community workshop held in 2021 with the aim of shaping the main technical and organisational aspects for the creation of a open knowledge base of scholarly information for the Netherlands \cite{neylon_open_2021}, and organisations created to help open infrastructures flourishing, such as the Global Sustainability Coalition for Open Science Services (SCOSS)\footnote{\url{https://scoss.org}.} and Invest in Open Infrastructures (IOI)\footnote{\url{https://investinopen.org}.}.

As part of our future work towards the creation of HuCI, we plan to conduct a survey among Humanities scholars in order to elicit their views and desiderata with respect to the prospects and usefulness of such a citation index. This survey could be conducted in coordination with ongoing international activities on the topic of bibliographic data in the Humanities, notably the DARIAH-EU Bibliographic Data Working Group\footnote{\url{https://www.dariah.eu/activities/working-groups/bibliographical-data-bibliodata}.}. Nevertheless, given the striking similarities that citation indexes bear with thematic bibliographies (both printed and digital) -- which are widely used by scholars across the Humanities -- it does seem plausible to postulate that such a citation index will meet the interests of many scholars.

Our hope is that the guidelines, principles, and technological approaches described in this work can be an appropriate starting point for the implementation of HuCI, a fundamental tool for Humanities research. The goals depicted by HuCI, and their technical implementation, are possible only if the Humanities scholars and institutions act together in a decentralised and coordinated fashion, by sharing efforts, resources, and services towards a common objective, of which the suggestions in this article represent only the starting point. 