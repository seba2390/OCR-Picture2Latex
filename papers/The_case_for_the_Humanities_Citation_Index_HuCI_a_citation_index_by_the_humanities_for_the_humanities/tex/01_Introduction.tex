\section{Introduction}\label{sec:introduction}

%\textit{Expand on the purpose of the paper and contextualise. Most crucially, provide and discuss reference to recent work in terms of motivating the need for the HuCI. Clarify we include arts. Introduce GLAM as a whole sector.}

%\vspace{0.5cm}
%Introduce terminology:
%\begin{itemize}
    %\item HuCI is the Arts\&Humanities citation corpus. HuCI will exist as a virtual corpus, materialized by several co-existing repositories (e.g. OpenCitations, Wikidata) providing access to their citation data via SPARQL and other endpoints. HuCI is, in this sense, fully distributed and existing by means of an infrastructure proposed in what follows.
    %\item Scholar Index is the application layer which will allow to distribute the creation and curation of citation data via a digital library application embedding the necessary machine learning components, and to centrally expose them via a citation index.
%\end{itemize}

Citation  indexes  are  by  now  part  of  the  research  infrastructure  in  use  by  most  scientists:  a  necessary  tool  in  order  to  cope  with  the  increasing  amounts  of  scientific  literature  being  published. However, existing commercial  citation  indexes  are  designed  for  the  sciences  and  have  uneven  coverage  and  unsatisfactory  characteristics  for humanities\footnote{Throughout this paper we use the term \textit{humanities} as a shorthand for Arts \& Humanities (A\&H). To a degree, the Social Sciences are also concerned.} scholars. This situation has both discouraged the usage of citation indexes and hindered bibliometric studies of humanities disciplines.  

The creation of a citation index for the humanities may well appear as a daunting task due to several characteristics of this field, such as its fragmentation into several sub-disciplines, the common practice of publishing research in languages other than English, as well as the amount of scholarship from past centuries that is still waiting to be digitised.

Notwithstanding these challenges, we argue that the creation of such an index can be highly beneficial to humanities scholars for, at least, the following reasons. Firstly, humanities scholars have long been relying on information seeking behaviours that leverage citations and references lists for the discovery of relevant publications -- a strategy that citation indexes are designed to support and facilitate. Secondly, a comprehensive citation index for the humanities will be a valuable source of data for researchers willing to conduct bibliometric studies of the humanities. Lastly, capturing the wealth of references to primary and secondary sources contained in humanities literature will allow to create links between archives, galleries, libraries and museums where digitized copies of these sources can increasingly be found.

Before continuing with this paper, we introduce key terminology related to citation indexing that will be used throughout this paper, adopting the definitions from \cite{peroni_opencitations_2018}. These are: bibliographic entity, bibliographic resource and bibliographic citation.
%We provide herein some definitions so as to avoid ambiguities when these terms will be mentioned in the rest of the paper. 
A \textbf{bibliographic entity} is any entity which can be part of the bibliographic metadata of a bibliographic artifact: it can be a person, an article, an identifier for a particular entity (e.g., a DOI), a particular role held by a person (e.g., being an author) in the context of defining another entity (e.g., a journal article), and so forth. A \textbf{bibliographic resource} is a kind of bibliographic entity that can cite or be cited by other bibliographic resources (e.g., a journal article), or that contains other  resources (e.g., a journal). A \textbf{bibliographic citation} is another kind of bibliographic entity: a conceptual directional link from a citing bibliographic resource to a cited bibliographic resource.
The citation data defining a particular citation must include the representation of the conceptual directional link of the citation and the basic metadata of the involved bibliographic resources, that is to say sufficient information to create or retrieve textual bibliographic references for each of the bibliographic resources. Following \cite{peroni_open_2018}, we say that a bibliographic citation is an open citation when the citation data needed to define it are compliant with the following principles: structured, separate, open, identifiable, available.

The remaining of this paper is organised as follows. In Section \ref{sec:related-work} we discuss previous work on analysing the behaviour of humanities scholars in relation to information retrieval. We also present the main limitations of existing citation indexes, seen from the perspective of the humanities, and outline the main obstacle that citation indexing has faced in this area. In Section \ref{sec:citation-index-AH-needs} we argue for the need of a Humanities Citation Index (HuCI from now onward) and in Section \ref{sec:citation-index-characteristics} we present what we believe are the essential characteristics that such an index should have. We then propose a possible implementation of HuCI, based on a federated and distributed research infrastructure (Section \ref{sec:research-infrastructure}). We conclude with some considerations on how HuCI relates to recent efforts to create open infrastructures for research.