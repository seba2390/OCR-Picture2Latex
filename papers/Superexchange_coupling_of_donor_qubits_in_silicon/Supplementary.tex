\section{Convergence of multi-electron states}
The convergence of multi-electron states calculated using FCI depends on the number of single-electron states taken into the basis. As in this work, we investigate the effect of superexchange, i.e. the energy difference $\Delta E$ between outer-dot singlet and triplet states, $S^l$ and $T^l_0$ respectively, we use $\Delta E$ also to determine the convergence of our calculated four-electron states. We gradually increase the number of single-electron basis states in FCI calculations and observe when the changes in $\Delta E$ are negligible within some numerical tolerance. In Figure \ref{fig:convergence} we show examples of such analysis for donors separated in both [100] and [110] crystal directions. We find that in most of the cases considered in the paper, it is sufficient to use 56 basis states, i.e. 28 valley-orbital states, with two-fold spin degeneracy. In a few cases, we have used up to 72 basis states to reach convergence in the energy difference. 

\begin{figure}[htb!]
    \centering
    \includegraphics[width=0.8\textwidth]{All_convergence.pdf}
    \caption[Convergence of superexchange as a function of a number of single-electron basis states]{\textbf{Convergence of superexchange as a function of a number of single-electron basis states.} Plots (a,c) show results for equidistant dots with $r_1=r_2=r_M=18a_0$ along the [100] direction where $a_0 = 0.5431$ nm and $r_1=r_2=r_M=24a$ along the [110] direction where $a = a_0/\sqrt{2}$, respectively. Plots (b,d) show results for non-equidistant cases $r_1=r_2=20a_0$, $r_M=14a_0$ in [100] and $r_1=r_2=26a$, $r_M=20a$ in [110], respectively.}
    \label{fig:convergence}
\end{figure}



\section{Analysis of four-electron eigenstates}

% As discussed in the main text, the Schrieffer-Wolff approximation of the spin Hamiltonian holds only when $j_1,j_2\ll j_M$. In this regime, the lowest energy manifold is characterized by a singlet state formed within the two middle dots and is well separated in energy from higher energy states. In that case, the superexchange is well-defined and the coherent coupling of the outer-dots spins is possible. 


% In this section, we analyze the 4-electron states in terms of contributions from different Slater determinants, i.e. consisting of different 1-electron basis states. This will help us to determine the regimes of $j_1,j_2,j_M$ where the eigenstates can be confidently described in terms of singlet-triplet symmetries.


% \begin{figure}[htbp]
%     \centering
%     \includegraphics[scale=0.6]{spin_model_eval_contrib.pdf}
%     \caption{(a) Energies of 4-electron eigenstates calculated with effective Hamiltonian $H_{eff}$ as a function of middle donors exchange $j_M$, for constant $j_1=j_2=0.1$. (b) Contribution of inner-singlet and inner-triplet basis states in the ground state of $H_{eff}$ for the same $j$ values as in (a).}
%     \label{fig:effective}
% \end{figure}

% First, we look at the problem from the effective spin Hamiltonian perspective. As mentioned in the main text we restrict ourselves to the basis consisting only of $S_z=0$ states, i.e. $\ket{\uparrow S \downarrow}, \ket{\downarrow S \uparrow}$, $\ket{\uparrow T_0 \downarrow}, \ket{\downarrow T_0 \uparrow}$, $\ket{\uparrow T_- \uparrow}, \ket{\downarrow T_+ \downarrow}$. For $j_1,j_2\ll j_M$ the two lowest-energy states can be well described by $S^l$ and $T^l_0$, as defined in the main text. However, when we move closer to the $j_1,j_2 \approx j_M$ region $S^l$ and $T^l_0$ states begin to acquire some inner-dots triplet admixtures. This can be seen in Fig. \ref{fig:effective}. In Fig. \ref{fig:effective}(a) we plot energies of all the eigenstates of the effective Hamiltonian $H_{eff}$ for $j_1=j_1=0.1$ and $j_M$ varying from 0 to 1. We can see that for $j_M=1$ the $S^l$ and $T^l_0$ states are well separated in energy from all excited states. In \ref{fig:effective}(b) in blue we can see $S^l$ contributions from inner-dots singlet states (i.e. $\ket{\uparrow S \downarrow}$ and $\ket{\downarrow S \uparrow}$) and in red contributions from inner-dots triplet states (all the remaining basis states). We can see the singlet contribution reaches 99.1\% for $j_M=1$. When decreasing $j_M$ the energies of $S^l$ and $T^l_0$ are closing to the excited states and around $j_1=j_2=j_M$ we can see clear anticrossing between lower and upper energy manifolds. In this region, the inner-dots singlet and triplet contributions in $S_l$ are both approximately 50\%. In this regime, we do no longer have well defined two-level system of $S^l$ and $T^l_0$, separated by superexchange which can be used to manipulate the states coherently. The specific value of $j_M/j_{1,2}$ ratio required for the coherent manipulation in real device applications depends strongly on the desired fidelity and the details of the experiment.


\begin{figure}[htbp]
    \centering
    \includegraphics[width=\columnwidth]{wf16_v2.pdf}
    \caption{\textbf{Single electron basis states calculated using an atomistic tight-binding approach.} The wavefunctions of the first 16 valley-orbit molecular states are plotted here on a logarithmic scale. The titles indicate the binding energy of each of the states.}
    \label{fig:basis_states}
\end{figure}

In this section, we analyze the 4-electron states in terms of their contributions from different Slater determinants, i.e. consisting of different 1-electron atomistic tight-binding basis states. This will help us to determine the regimes of $j_1,j_2,j_M$ where the eigenstates can be confidently described in terms of singlet-triplet symmetries. First, we focus on [100] crystal direction and the equidistant case $r_1=r_2=r_M=18a_0$. In Figure \ref{fig:basis_states} we plot several of the lowest-energy single electron states. 

\begin{itemize}
    \item 
We can see that the ground and first-excited orbitals (states numbered 1 and 3 in the figure) are localized mainly in the two middle dots. This is because superposing the four donors confining potentials in TB calculations results in the total potential being slightly deeper within the middle dots than the outer ones. 
    \item 
The next two excited orbitals (states 5 and 7) are localized mainly within two outer dots. 
    \item
We consider the even-numbered states (2,4,6,8...) to be of the same orbital as odd (1,3,5,7...) but of opposite spin. 
    \item
The 4-electron ground state, labelled by us as $S_l$, has main contributions from following configurations: 14.7\% $\ket{1,3,6,8}$, 14.7\% $\ket{2,4,5,7}$, 10.7\% $\ket{1,2,5,6}$, 9.5\% $\ket{1,2,7,8}$, 8.9\% $\ket{3,4,7,8}$, 7.8\% $\ket{3,4,5,6}$, and smaller contributions from other configurations.
    \item
The first two contributions can be interpreted as inner-triplet outer-triplet states ($\ket{1,3,6,8}$ is $T_-$ in inner dots and $T_+$ in outer dots,  $\ket{2,4,5,7}$ is opposite to that). 
    \item
Other configurations are inner-singlet outer-singlet states. 
    \item
We can then see that a significant part of the ground state is described as an inner-dot triplet, which is specific for $j_{1,2}\approx j_M$ regime and has been expected by us following the above discussion on effective Hamiltonian states. 
    \item
To obtain a rough estimate of the percentage of singlet and triplet contributions within the middle dots we sum up contributions from states 
$\{\ket{1,2,i,j},\ket{3,4,i,j}\}$ 
for singlet and $\{\ket{1,3,i,j},\ket{1,4,i,j}$,
$\ket{2,3,i,j}$,
$\ket{2,4,i,j}\}$ for triplet. 
\end{itemize}
We show the results in Table \ref{tab:contrib}. As expected from the effective Hamiltonian model here both contributions are close to 50\%. The presented numbers are, however, just an estimate, not a precise evaluation of singlet and triplet admixtures in the middle dots as i) we do not sum up all the higher-order Slater determinant probabilities and ii) the orbitals 1 and 3 have small but non-zero probabilities also in the outer dots.


Next, we look at the non-equidistant case, with $r_1=r_2>r_M$ which guarantees $j_1=j_2<j_M$. In Table \ref{tab:contrib} we compare inner-singlet and inner-triplet contributions in $S_l$ for 3 values of $r_M$, i.e. $18a_0$, $16a_0$ and $14a_0$, while keeping the total four-donor separation $R$ constant at $54a_0$. We can see the inner-singlet (inner-triplet) contribution is significantly increasing (decreasing) when the middle donors are pulled closer together.  


\begin{table}[htb!]
    \centering
    \begin{tabular}{|c|c|c|}
    \hline
     Separation ($a_0$) & Inner singlet contribution & Inner triplet contribution  \\ \hline
     18 & 0.4927 & 0.483  \\ \hline
     16 & 0.9209 & 0.06706 \\ \hline
     14 & 0.97904 & 0.00238 \\ \hline
\end{tabular} 
    \caption{\textbf{Approximate contributions of the inner-singlet and inner-triplet configurations in the FCI ground state of a 4-donor chain separated along the [100] direction} Here we show 3 different middle donor separations $18a_0$, $16a_0$ and $14a_0$ with a constant outer donor separation of $54a_0$. We see that the inner singlet contribution to the ground state dramatically increases from $\sim49\%$ to $\sim98\%$ as the separation decreases by $4a_0$.}
    \label{tab:contrib}
\end{table}


