% ****** Start of file apssamp.tex ******
%
%   This file is part of the APS files in the REVTeX 4.2 distribution.
%   Version 4.2a of REVTeX, December 2014
%
%   Copyright (c) 2014 The American Physical Society.
%
%   See the REVTeX 4 README file for restrictions and more information.
%
% TeX'ing this file requires that you have AMS-LaTeX 2.0 installed
% as well as the rest of the prerequisites for REVTeX 4.2
%
% See the REVTeX 4 README file
% It also requires running BibTeX. The commands are as follows:
%
%  1)  latex apssamp.tex
%  2)  bibtex apssamp
%  3)  latex apssamp.tex
%  4)  latex apssamp.tex
%
\documentclass[%
 %reprint,
%superscriptaddress,
%groupedaddress,
%unsortedaddress,
%runinaddress,
%frontmatterverbose, 
%preprint,
%preprintnumbers,
%nofootinbib,
%nobibnotes,
%bibnotes,
 amsmath,amssymb,
 aps,
%pra,
%prb,
%rmp,
%prstab,
%prstper,
%floatfix,
10.5pt]{revtex4-2}
\usepackage{mathrsfs}
%\usepackage{bm}% bold math
\usepackage{siunitx}
\usepackage[dvips]{epsfig}
%\usepackage{bbm}
\usepackage{amsfonts}
\usepackage{color}
%\usepackage{pict2e}
\def\bm#1{\mbox{\boldmath{$#1$}}}
\def\rr#1{(\ref{#1})}
\def\k#1{\kappa{#1}}
\def\cprime{$'$}
\newcommand{\be}{\begin{equation}}
\newcommand{\ee}{\end{equation}}
\usepackage{graphicx}% Include figure files
\usepackage{dcolumn}% Align table columns on decimal point
\usepackage{bm}% bold math


\renewcommand{\thesection}{S-\Roman{section}}
\renewcommand{\thetable}{S-\Roman{table}}
\renewcommand{\theequation}{S\arabic{equation}}
\renewcommand{\thefigure}{S-\Roman{figure}}
\renewcommand{\bibnumfmt}[1]{[S#1]}
\renewcommand{\citenumfont}[1]{S#1}





\DeclareMathOperator{\sech}{sech}

%\usepackage{hyperref}% add hypertext capabilities
%\usepackage[mathlines]{lineno}% Enable numbering of text and display math
%\linenumbers\relax % Commence numbering lines

%\usepackage[showframe,%Uncomment any one of the following lines to test 
%%scale=0.7, marginratio={1:1, 2:3}, ignoreall,% default settings
%%text={7in,10in},centering,
%%margin=1.5in,
%%total={6.5in,8.75in}, top=1.2in, left=0.9in, includefoot,
%%height=10in,a5paper,hmargin={3cm,0.8in},
%]{geometry}

\begin{document}

\preprint{APS/123-QED}

\title{Thermocapillary migrating odd viscous droplets; Supplementary Information}% Force line breaks with \\
%\thanks{A footnote to the article title}%
\author{A.Aggarwal}
\affiliation{Department of Materials Science \& Engineering, Robert R. McCormick School of Engineering and Applied Science, Northwestern University, Evanston IL 60208 USA and \\Center for Computation and Theory of Soft Materials, Northwestern University, Evanston IL 60208 USA
}
\author{E.Kirkinis}
\affiliation{Department of Materials Science \& Engineering, Robert R. McCormick School of Engineering and Applied Science, Northwestern University, Evanston IL 60208 USA and \\Center for Computation and Theory of Soft Materials, Northwestern University, Evanston IL 60208 USA
}
%\author{A. V. Andreev}
% %\homepage{http://www.Second.institution.edu/~Charlie.Author}
%\affiliation{Department of Physics, University of Washington, Seattle WA 98195 USA}
%\collaboration{MUSO Collaboration}%\noaffiliation
%
%\affiliation{
% Third institution, the second for Charlie Author
%}%
%\author{Delta Author}
%\affiliation{%
% Authors' institution and/or address\\
% This line break forced with \textbackslash\textbackslash
%}%
%
%\collaboration{CLEO Collaboration}%\noaffiliation
\author{M. Olvera de la Cruz}
\email{m-olvera@northwestern.edu}
\affiliation{Department of Materials Science \& Engineering, Robert R. McCormick School of Engineering and Applied Science, Northwestern University, Evanston IL 60208 USA and \\Center for Computation and Theory of Soft Materials, Northwestern University, Evanston IL 60208 USA
}

\date{\today}% It is always \today, today,
             %  but any date may be explicitly specified

%\keywords{Suggested keywords}%Use showkeys class option if keyword
                              %display desired
\maketitle
\section{Results for the migration of a three-dimensional odd thermocapillary droplet}
{In the main body of this article we have followed the two dimensional geometry that
is exclusively employed in the literature to computationally and theoretically study the migration of droplets. 
In this section we build on the theory developed in the main article 
and show that a \emph{three-dimensional} thermocapillary droplet composed of an odd viscous liquid
shows the same migration trend as in the two-dimensional case. To this end we employ the 
constitutive law \citep[\S13]{Landau1981} for the odd viscous stress
\be \label{sigma1}
\bm{\sigma}^o = \eta_o 
\left(\begin{array}{ccc}
-\left(\partial_x v + \partial_y u \right) &  \partial_x u  - \partial_y v & 0\\
 \partial_x u  - \partial_y v & \partial_x v + \partial_y u   & 0\\
0&0&0
\end{array}
\right).
\ee
which is similar to Eq (6) of the main article but here the liquid velocity is three dimensional
$(u,v,w)$, satisfies the incompressibility condition $\partial_x u + \partial_y v + \partial_zw=0$ and we take the $z$ axis to point out of the page as in panel (b) of Fig. \ref{odd_viscosity_droplet}. 

We numerically simulate the three-dimensional droplet by employing the same material parameters
as in the two-dimensional case with the heat transfer coefficient 
$\alpha_{th} = 500 \textrm{W/(m}^2\cdot K)$ and employing the same software package, as 
this is described in detail at the end of this SI. In the main article the qualitative 
comparison of panel (b) of Fig. \ref{odd_viscosity_droplet} with panel (a) of Fig. 3 of the main article
showed that the three-dimensional droplet follows the same trend as its two-dimensional counterpart.  
Panels (a) and (c) of Fig. \ref{odd_viscosity_droplet} also display the trend we would expect from
studying the droplet configurations of the main article. In the absence of odd viscosity (left column) 
a toroidally shaped cell encompases the center axis, so that the contact angles, cell and velocity field
are all axisymmetric. In the presence of odd viscosity (right column) the axial symmetry is lost and 
the droplet starts migrating to the right. 
}
\begin{figure*}
\vspace{-5pt}
\begin{center}
\includegraphics[height=7in,width=7in]{3dmigration}%{fig2_v2}
\end{center}
\vspace{-5pt}
\caption{{Numerically determined thermocapillary droplet configurations in the absence (left column) and the presence (right column) 
of odd viscosity. (a) Left: in the absence of odd viscosity 
the thermocapillary effect leads to axisymmetric circulations. The arrows depict
the velocity field at the liquid-gas interface poining towards the summit. Right: In the presence of odd
viscosity, where the anisotropy axis is parallel to the solid substrate the axial symmetry of the contact
angles break leading to a recircularing cell deformation and the velocity field to point towards 
a point located off the geometric pole. This leads to droplet migrating to the right. (b) Cross section of the droplets displayed in panel (a). 
\textbf{Left:} Counterrotating cells in a
thermocapillary droplet which remains immobile in the absence of odd viscosity.  
\textbf{Right:} In the presence of odd viscosity the symmetry breaks, leading to a left-right asymmetry in the cells and the onset of migration.
(c) Top view of the three-dimensional droplet of panel (a). It is clear that the velocity field is axisymmetric
in the left (zero odd viscosity) configuration and asymmetric in the right (nonzero odd viscosity) configuration
leading to droplet migration.}
\label{odd_viscosity_droplet} }
\vspace{-15pt}
\end{figure*}


\section{Governing equations and boundary conditions}
\subsection{Constitutive laws}
In an incompressible liquid endowed with odd viscosity the Cauchy stress tensor $\bm{\sigma}$
becomes \citep{Avron1998}
\be\label{totalstress}
\bm{\sigma} =-p\bm{I}+ \bm{\sigma}^e + \bm{\sigma}^o. 
\ee 
In two dimensions 
\be \label{evenstress}
\sigma^e_{ik} = \eta_e \left( \frac{\partial u_i}{\partial x_k} + \frac{\partial u_k}{\partial x_i}
\right) , \quad i,k=1,2,
\ee
is the standard deviatoric part of the Cauchy stress tensor with shear viscosity
$\eta_e$. 
$\bm{\sigma}^o$ is the odd part of the Cauchy stress tensor
\cite{Avron1998,Lapa2014}
\be  \label{oddstress}
\sigma^o_{ik} = - \eta_o (\delta_{i1}\delta_{k1} - \delta_{i3}\delta_{k3} ) 
\left( \frac{\partial u_1}{\partial x_3} +\frac{\partial u_3}{\partial x_1}  \right)
+ \eta_o \left( \delta_{i1}\delta_{k3} + \delta_{i3} \delta_{k1}    \right) 
\left( \frac{\partial u_1}{\partial x_1} - \frac{\partial u_3}{\partial x_3}\right),
\ee
%for $i,k=1,2$, or explicitly, letting $u_1=u$ and $u_3 = w$, 
%\be \label{sigmaodd3311}
%\sigma^o_{13} = \sigma^o_{31} = \eta_o \left(\partial_x u - \partial_y v\right) \quad
%\textrm{and} \quad \sigma^o_{33} = - \sigma^o_{11} = \eta_o\left( \partial_x v
%+\partial_y u\right),
%\ee
where $\eta_o$ is the odd viscosity coefficient. 
%The components of the even and odd parts of the stress
%are collected in Table \ref{table:evenodd}. 
%
%\begin{table}
%  \begin{center}
%\def~{\hphantom{0}}
%  \begin{tabular}{ccc}
%Stress component &even stress $\sigma^e$& odd stress $\sigma^o$\\
%&& \\
%$\sigma_{xx}$ &$2\eta_e \partial_x u $&$ -\eta_o (\partial_xv + \partial_y u)$\\
%$\sigma_{xy} $ &$\eta_e (\partial_xv + \partial_y u) $ & $\eta_o (\partial_x u  - \partial_yv)$\\ 
%$\sigma_{yy} $ & $2\eta_e\partial_y v$    & $ \eta_o (\partial_xv + \partial_y u)$\\
%&& \\
%$\sigma_{rr}$ & $      2\eta_e\partial_r v_r $          &  $-\eta_o \left(  \partial_r v_\phi - \frac{1}{r}v_\phi + \frac{1}{r}\partial_\phi v_r\right)$\\
%$\sigma_{r\phi}$ & $      \eta_e\left(  \partial_r v_\phi - \frac{1}{r}v_\phi + \frac{1}{r}\partial_\phi v_r\right) $          &  2$\eta_o \partial_r v_r$\\
%$\sigma_{\phi\phi}$ & $      2\eta_e\left(\frac{1}{r}\partial_\phi v_\phi + \frac{1}{r}v_r \right)$         &  $\eta_o \left(  \partial_r v_\phi - \frac{1}{r}v_\phi + \frac{1}{r}\partial_\phi v_r\right)$
%  \end{tabular}
%\caption{Explicit formulas for even and odd stress defined in Eq. \rr{evenstress} and \rr{oddstress}, respectively,
%in Cartesian and plane-polar coordinates. }
%  \label{table:evenodd}
%  \end{center}
%\end{table}


Incorporating in \rr{totalstress} the contributions \rr{evenstress} and \rr{oddstress} we obtain a more
transparent form for the Cauchy stress tensor, cf. \cite{Kirkinis2023a}, 
\be \label{seso}
\bm{\sigma} =-p\bm{I}+
%=  \left( \begin{array}{cc}
% 1 & - \frac{\eta_o}{\eta_e} \\
%  \frac{\eta_o}{\eta_e} & 1
% \end{array} \right) \sigma^e
2 \left( \begin{array}{cc}
 \eta_e & -\eta_o \\
 \eta_o & \eta_e
 \end{array} \right)
\mathcal{D},
%\equiv
% \left( \begin{array}{cc}
% \eta_e & -\eta_o \\
% \eta_o & \eta_e
% \end{array} \right)
%  \left( \begin{array}{cc}
% 2u_x & u_z+w_x \\
% u_z+w_x & 2w_z
% \end{array} \right)
\ee
where $\mathcal{D} =  \frac{1}{2} \left(\frac{\partial u_i}{\partial x_j} + \frac{\partial u_j}{\partial x_i}\right)$ is the rate-of-strain tensor. 



\subsection{Equations of motion and boundary conditions}


{The Navier-Stokes equations (for temperature-independent viscosity and density) become}
\begin{eqnarray}  \label{nsx}
\rho\left[ u_t + uu_x +wu_z\right] & = & -p_x  + \eta_e(u_{xx} + u_{zz}) - \eta_o(w_{xx} +  w_{zz}) , \\
\rho\left[ w_t + uw_x +ww_z\right] & = & -p_z + \eta_e (w_{xx} + w_{zz}) + \eta_o(u_{xx} + 
u_{zz}) - \rho g, \label{nsy}
\end{eqnarray}
where $(u,w)$ is the liquid velocity, $g$ the acceleration of gravity and $\rho$ the liquid density. 
At the liquid-gas interface $z = h(x,t)$, the shear and normal stresses are respectively
\be  \label{freebc1}
\mathbf{t} \bm{\sigma}\mathbf{n} = \frac{\partial \gamma}{\partial s} + \tau_0, \quad \textrm{and}
\quad \mathbf{n} \bm{\sigma}\mathbf{n} = 2\kappa \gamma + \Pi,  \quad \textrm{at}\quad
z = h(x,t),
\ee
where $\gamma$ is the (variable) surface tension, $s$ is arc length along the interface, $\kappa$
the mean curvature of the surface, $\tau_0$ is an external shear stress and $\Pi$ an external normal 
stress on the boundary \cite{Oron1997}. 

Boundary conditions at the liquid-solid interface are 
\be \label{solidbc1}
u = 0, \quad \textrm{and} \quad w=0, \quad \textrm{at} \quad z=0.
\ee

The liquid-gas interface is described by the function $z=h(x,t)$. The tangent and normal vectors to the interface are 
\be
\mathbf{t} = \frac{(1,h_x)}{\sqrt{1 + h^2_x}}, \quad \textrm{and}\quad \mathbf{n} = \frac{(-h_x,1)}{\sqrt{1 + h^2_x}}.
\ee
The components of stress tensor on the liquid-gas interface become
\be \label{tsn}
\mathbf{t} \bm{\sigma}\mathbf{n} = \frac{1}{1+h_x^2} (1,h_x)
\left( \begin{array}{cc}
\sigma_{11} & \sigma_{13} \\
\sigma_{31} & \sigma_{33}
\end{array}
\right)\left( \begin{array}{c}
-h_x \\ 1
\end{array} \right) = \frac{1}{1+h_x^2} \left[ \sigma_{13} - h_x^2 \sigma_{31} + h_x ( \sigma_{33} - \sigma_{11} ) \right],
\ee
and 
\be \label{nsn}
\mathbf{n} \bm{\sigma} \mathbf{n} = \frac{1}{1+h_x^2} (-h_x,1)
\left( \begin{array}{cc}
\sigma_{11} & \sigma_{13} \\
\sigma_{31} & \sigma_{33}
\end{array}
\right) 
\left( \begin{array}{c}
-h_x \\ 1
\end{array} \right) = \frac{1}{1+h_x^2} \left[ \sigma_{33} + h_x^2 \sigma_{11} - h_x ( \sigma_{13} + \sigma_{31} ) \right], 
\ee
so that
\be\label{freebc2a}
\frac{1}{1+h_x^2} \left[ \sigma_{13} - h_x^2 \sigma_{31} + h_x ( \sigma_{33} - \sigma_{11} ) \right]= \frac{\partial \gamma}{\partial s} + \tau_0,
\ee
and
\be
 \frac{1}{1+h_x^2} \left[ \sigma_{33} + h_x^2 \sigma_{11} - h_x ( \sigma_{13} + \sigma_{31} ) \right]=
2\kappa \gamma + \Pi, \label{freebc2b} 
\ee
at $z=h(x,t)$.  
The kinematic condition at $z=h(x,t)$ reads
\be \label{kinematic1}
w = h_t + u h_x.
\ee




%Considering first the even viscosity contributions we have \cite{Oron1997,Davis2002}
%\begin{eqnarray} \label{even1a}
%\mathbf{t} \bm{\sigma}^e\mathbf{n} &=& \frac{1}{1+ h_x^2}
%\eta_e \left[ 2(-u_x  + w_z) h_x + (1-h_x^2) (w_x + u_z)                                          \right],
% \end{eqnarray}
%and
%\begin{eqnarray}\label{even2a}
%\mathbf{n} \bm{\sigma}^e\mathbf{n} &=& \frac{1}{1+ h_x^2} \left[
%-p +2\eta_e w_z - 2\eta_e h_x (w_x + u_z)  + h_x^2(-p + 2\eta_e u_x) \right]   .                       
% \end{eqnarray}
%
%On the other hand, the contribution of the odd viscosity to the shear and normal stress is
%\begin{eqnarray} \label{odd1a}
%\mathbf{t} \bm{\sigma}^o\mathbf{n} &=& \frac{1}{1+ h_x^2}
%\eta_o \left[ (u_x  - w_z) (1-h_x^2) + 2h_x (w_x + u_z)           \right]                                ,  \end{eqnarray}
%and
%\begin{eqnarray}
%\mathbf{n} \bm{\sigma}^o\mathbf{n} &=& \frac{1}{1+ h_x^2}
%\eta_o \left[ (w_x + u_z) (1-h_x^2) - 2h_x (u_x - w_z) \right],
%\label{odd2a}             
% \end{eqnarray}
%respectively. 



\subsection{Normal and shear components of rate-of-strain tensor with respect to $\eta_o$}
Invoking the form of the Cauchy stress tensor \rr{seso},  
the
boundary conditions \rr{freebc1} at a free surface can be rewritten as 
\be \label{tsn2}
%\mathbf{t} \sigma \mathbf{n} \equiv 
2\left(\eta_e\mathbf{t}-{\eta_o} \mathbf{n} \right)
\mathcal{D} \mathbf{n} = \frac{\partial \gamma}{\partial s} + \tau_0, \quad \textrm{at}\quad
z = h(x,t),
\ee
and 
\be \label{nsn2}
%\mathbf{n} \sigma \mathbf{n} 
 -p + 2\left(\eta_e\mathbf{n}+{\eta_o}\mathbf{t} \right)
\mathcal{D} \mathbf{n}=2\kappa \gamma + \Pi,  \quad \textrm{at}\quad
z = h(x,t). 
\ee




Equations \rr{tsn2} and \rr{nsn2} can be considered as a system of two simultaneous equations for the
two unknowns $\mathbf{t} \mathcal{D} \mathbf{n} $ and $\mathbf{n} \mathcal{D} \mathbf{n} $.  
We can therefore rewrite each of the familiar components of the twice rate-of-strain tensor 
at an interface  in the form
\be \label{tDn}
\mathbf{t} \mathcal{D} \mathbf{n} = \frac{1}{2(\eta_e^2  + \eta_o^2)} \left[ \eta_e\left( \frac{\partial \gamma}{\partial s} + \tau_0 \right) + \eta_o (p + 2\kappa \gamma +\Pi )        \right]
\quad \textrm{at}\quad
z = h(x,t),
\ee
and 
\be \label{nDn}
\mathbf{n} \mathcal{D} \mathbf{n} = \frac{1}{2(\eta_e^2  + \eta_o^2)} \left[ -\eta_o\left( \frac{\partial \gamma}{\partial s} + \tau_0 \right) + \eta_e (p + 2\kappa \gamma     +\Pi )  \right]
\quad \textrm{at}\quad
z = h(x,t),
\ee

This is the main result of the first part of the SI. It expresses the fact that in the presence of odd viscosity 
the liquid will experience a 
normal stress-induced extra shear stress (the term multiplying $\eta_o$ in \rr{tDn})
and a shear stress-induced extra normal stress (the term multiplying $-\eta_o$ in \rr{nDn}) at the free surface. 








\subsection{Thermocapillary flow\label{sec: thermo}}
In the system considered, there are temperature variations due to 
the thermocapillary effect and the applied temperature gradient. 
Conservation of energy 
{(for temperature-independent thermal conductivity and density)} reads
\be \label{energy1}
\rho c_p (T_t + \mathbf{u}\cdot \textrm{grad} T)  =  k_{th} \nabla^2 T
\ee
where $k_{th}$ is the liquid thermal conductivity, and $c_p$ the thermal expansion coefficient. 
In the geometry of our problem (cf. Fig.~\ref{fig1}),  Eq.~\rr{energy1} reduces to 
\be\label{energy1.1}
\rho c_p (T_t + uT_x + wT_z) = k_{th}(T_{xx} + T_{zz}), 
\ee
with boundary conditions 
\be \label{bc1}
T = T_0 \quad \textrm{at} \quad z=0 \quad \textrm{and} \quad 
k_{th} \mathbf{n}\cdot \nabla T + \alpha_{th} (T - T_\infty)=0, \quad \textrm{at} \quad
z = h(x,t),
\ee
where $T_0$ is the temperature of the rigid plane, $T_\infty$ 
the temperature of the gas phase and $\alpha_{th}$ the heat transfer coefficient. 
%describing the rate of heat transfer from the liquid to the ambient gas phase. 


The surface tension is considered to be temperature dependent of the form
\be 
\gamma = \gamma_0  +  \frac{d\gamma}{dT} (T - T_0),
\ee
where $-\frac{d\gamma}{dT}$ is positive for common liquids \citep{Oron1997,Davis2002}.
We define 
$\Delta \gamma$ to be 
the variation of surface tension over the temperature domain 
$\Delta T = T_0 - T_\infty$
so that
\be \label{Deltasigma}
\Delta \gamma =\frac{d\gamma}{dT} \Delta T.
\ee




\subsection{Droplet migration}
\begin{figure}[t]
\begin{center}
\vspace{30pt}
\setlength{\unitlength}{1mm}
\begin{picture}(50,40)
\thicklines
\put(40,20){\vector(1,0){4}}
\put(40,22){$U$}

% interface
\qbezier
(0,5)(20,68)(40,5)
\put(-5,5){\line(1,0){50}}
\put(32,4){\vector(-1,0){10}} % a(t)
\put(30,4){\vector(1,0){10}}
\put(30,1){$a(t)$}
\put(16,6){$\bar c(t)$}
\put(21,4){\line(0,1){2}}
\thicklines
% streamlines
\qbezier
(35,15)(19,52)(21,15)
\qbezier
(5,15)(21,52)(19,15)
\put(19,15){\vector(0,-1){1}}
\put(21,15){\vector(0,-1){1}}

\put(50,49){$\displaystyle \theta $}
\put(62,36){$\displaystyle U $}
\put(52,36){\vector(0,1){15}}
\put(42,36){\vector(1,0){20}}  % hysteresis diagram
\thicklines

%\put(56,48){\line(1,0){5}}
%\put(43,40){\line(1,0){5}}
%\qbezier(52,46)(53,48)(56,48)
\qbezier(52,46)(53,48)(60,50)
%\qbezier(48,40)(51,40)(52,42)
\qbezier(44,38)(51,40)(52,42)
\put(47,45){$\displaystyle {\Theta}_A $}
\put(52,40){$\displaystyle {\Theta}_R $}





\put(9,10){viscous liquid}
\put(-6,46){cold gas}
\put(10,0){hot solid}
%interface 
\thicklines
%\put(34,39.5){\vector(-2,-1){8}}
%\put(26,40){interface $\displaystyle z=h(x,t)$}
\put(25,36){interface}
\put(28,33){$\displaystyle z=h(x,t)$}
\thicklines
% angles
\qbezier
(2,12)(7,10)(8,5)
\put(0,5){\line(1,4){4}}
\put(3,6){$\displaystyle \theta_r $}
\put(-1,2){$\displaystyle c_r(t) $}

\qbezier
(38,12)(34,10)(32,5)
\put(40,5){\line(-1,4){4}}
\put(35,6){$\displaystyle \theta_a $}
\put(39,2){$\displaystyle c_a(t) $}


% odd pressure
\put(0,30){\vector(2,-1){8}}
\put(-21,31){odd compressive stress}

% odd pressure
\put(31,28){\vector(2,1){8}}
\put(32,26){odd tensile stress}



%angle separate cartoon
%\put(-20,35){\line(1,0){14}}
%\put(-19,35){\line(1,4){3}}
%\put(-19,35){\line(1,2){3}}
%\qbezier(-17,39)(-15,38)(-14,35)
%\qbezier(-17,45)(-10,44)(-8,35)
%\put(-18,35.5){$\displaystyle \theta_r $}
%\put(-15,38.5){$\displaystyle \Theta_R $}


\put(-19,8){\line(1,0){12}}
\put(-19,8){\line(1,4){3}}
\put(-19,8){\line(1,2){3}}
\qbezier(-17,12)(-15,11)(-14,8)
\qbezier(-17,18)(-10,17)(-8,8)
\put(-18,8.5){$\displaystyle \theta_r $}
\put(-15,11.5){$\displaystyle \Theta_R $}
%box
\put(-20,7){\line(1,0){14}}
\put(-20,22){\line(1,0){14}}
\put(-20,7){\line(0,1){15}}
\put(-6,7){\line(0,1){15}}
\put(-5.2,7){\vector(3,-1){5}}
%
%\put(-4,8){\line(1,0){14}}
%\put(9,8){\line(-1,4){3}}
%\put(9,8){\line(-1,2){3}}
%\qbezier(7,13)(3,12)(2,8)
%\qbezier(7,18)(0,17)(-2,8)
%\put(3.8,8.8){$\displaystyle \Theta_A $}
%\put(0.5,11.5){$\displaystyle \theta_a $}

\put(48,8){\line(1,0){11}}
\put(59,8){\line(-1,4){3}}
\put(59,8){\line(-1,2){3}}
\qbezier(56.7,13)(53,12)(52,8)
\qbezier(57,18)(50,17)(48,8)
\put(53.7,8.8){$\displaystyle \Theta_A $}
\put(50.5,11.5){$\displaystyle \theta_a $}
%
%box
\put(46,7){\line(1,0){14}}
\put(46,22){\line(1,0){14}}
\put(46,7){\line(0,1){15}}
\put(60,7){\line(0,1){15}}
\put(46,7){\vector(-3,-1){5}}


\put(-15,44){$\displaystyle z $}
\put(-2,36){$\displaystyle x $}
\put(-12,36){\vector(0,1){10}}
\put(-12,36){\vector(1,0){10}}  % coordinate system


\put(8,40){$\displaystyle \mathbf{n} $}
\put(20,40){$\displaystyle \mathbf{t} $}
\put(14,34){\vector(1,1){6}}
\put(14,34){\vector(-1,1){6}} 
\thicklines
\end{picture}
\caption{
\textbf{Top right}: Experiment-inspired hysteretic diagram of a single contact line moving with velocity $U_{\textrm{cl}}$ vs.  dynamic contact angle $\theta(t)$ (cf. Eq.\rr{Ucl} and \citep{Dussan1979}). Contact lines move only when the dynamic contact angle $\theta(t)$ 
lies outside the interval $[ {\Theta}_R, {\Theta}_A]$, 
determined by the \emph{static} advancing and receding contact angles ${\Theta}_A$ and ${\Theta}_R$, respectively.
\textbf{Main figure:}
Droplet configuration at the onset of migration:
the viscous liquid droplet on a hot solid substrate surrounded by a colder ambient gas phase. 
The thermocapillary effect
creates in the droplet two identical cells with opposite sense of circulation \cite{Ehrhard1991}. 
Odd viscosity induces a compressive stress on the left cell and tensile on the right. 
The droplet tilts to the right and forces the right dynamic contact angle $\theta_a(t)$ to exceed its static
advancing counterpart $\Theta_A$ (right inset) and the left dynamic contact angle $\theta_r$ to lag behind the static receding contact angle $\Theta_R$ (left inset). 
This leads the droplet to migrate to the right with velocity $U$ according to the experiments of Dussan V
\cite{Dussan1979} (cf. the upper right hysteretic diagram). 
%Notation follows Smith \cite{Smith1995}. 
\label{fig1} }
\end{center}
\vspace{-10pt}
\end{figure}

The problem of determining the motion of a droplet reduces to 
solving an evolution equation for the liquid-gas interface $z=h(x,t)$ with boundary conditions
\be \label{bcmob0}
h(c_a(t),t) = h(c_r(t),t) =0, \quad \int_{c_r(t)}^{c_a(t)} h(x,t)dx = V_0,
\ee
where 
\be \label{cathetaa}
c_a(t) \quad \textrm{and}\quad \theta_a(t),\quad \quad c_r(t) \quad \textrm{and}\quad \theta_r(t)
\ee
is the $x$-location of the contact line and dynamic contact angle at the right and the left contact lines
of the droplet, respectively (cf. Fig. \ref{fig1}) {and $V_0$ denotes the constant area of the two-dimensional
droplet}. 
Employing the nondimensional units introduced in \cite{Smith1995, Aggarwal2023} (which are not 
repeated here to avoid possible duplication), we adopt the mobility law 
\be \label{mob0}
\frac{dc_a}{dt} = \left\{ 
\begin{array}{cc}
(\theta_a-\Theta_A)^m, &  \theta_a>\Theta_A\\
0, & \Theta_R\leq \theta_a\leq \Theta_A\\
-(\Theta_R - \theta_a)^m& \theta_a<\Theta_R
\end{array} \right.
\quad
\textrm{and}
\quad
\frac{dc_r}{dt} = \left\{ 
\begin{array}{cc}
-(\theta_r-\Theta_A)^m, &  \theta_r>\Theta_A\\
0, & \Theta_R\leq \theta_r\leq \Theta_A\\
(\Theta_R - \theta_r)^m& \theta_r<\Theta_R
\end{array} \right.
\ee
where $\theta_a(t) = -h_x(c_a(t))$ and $\theta_r(t) = h_x(c_r(t))$ (cf. Fig. \ref{fig1})
and 
\be  \label{barthetaa}
\Theta_A\quad \textrm{and}\quad \Theta_R
\ee
are the advancing and receding \emph{static} contact angles, respectively (cf. upper right corner
of Fig. \ref{fig1}) { and we tacitly assumed that angles are small}. 













\section{\label{sec: symmetry}Droplet migration is induced by symmetry-breaking}


\begin{figure}
\vspace{20pt}
\begin{center}
\setlength{\unitlength}{1.2mm}
\begin{picture}(90,30)
\thicklines




% droplet pressure gradient
\put(10,7){\line(1,0){25}}
\qbezier(12,7)(23,50)(34,7)
\put(12,4){$u(-x,z) = -u(x,z)$}
%\put(12,1){$\omega(-x,z) = -\omega(x,z)$}
%\put(27,12){\circle{1}}
%\qbezier
%(25,12)(27,15)(29,12)
%\put(29,12){\vector(1,-1){1}}

%%\put(19,12){\circle{1}}
%\qbezier
%(17,12)(19,15)(21,12)
%%\put(17,12){\vector(-1,-1){1}}

\put(18,30){$\eta_o \equiv 0$}
\put(15,8){viscous liquid}
\put(30,25){gas}


% streamlines
\qbezier
(31,12)(24,40)(24,12)
\qbezier
(15,12)(22,40)(22,12)
\put(24,12){\vector(0,-1){1}}
\put(22,12){\vector(0,-1){1}}




% droplet torque
\put(50,7){\line(1,0){25}}
\qbezier(52,7)(73,50)(74,7)
\put(12,4){$u(-x,z) = -u(x,z)$}

% streamlines
\qbezier
(72,12)(70,38)(68,12)
\qbezier
(56,12)(72,40)(66,12)
\put(68,12){\vector(0,-1){1}}
\put(66,12){\vector(0,-1){1}}

\put(52,4){symmetry-breaking}
\put(62,30){$\eta_o \neq 0$}
\put(57,8){viscous liquid}
\put(58,25){gas}

\end{picture}
\caption{Symmetries of the equations of motion for a thermocapillary droplet (left) in the absence of odd viscosity and symmetry breaking in a thermocapillary droplet in the presence of odd viscosity (right).
In the former case the Navier-Stokes equations, boundary conditions and contact angles are invariant
with respect to the transformation $x\rightarrow -x, u\rightarrow - u$. This gives rise to two recirculating
cells inside the droplet \cite{Ehrhard1991}. In the latter case odd viscosity 
breaks all former symmetries. This leads the droplet to tilt and migrate to the right. 
\label{fig3} }
\end{center}
\vspace{0pt}
\end{figure}


In this article we consider the case displayed to the right of Fig.\ref{fig3}. 
A droplet in the presence of odd viscosity breaks the reflection symmetry $x\rightarrow -x, u\rightarrow - u$, 
in the equations of motion \rr{nsx} and \rr{nsy}
and in the stress boundary conditions.
The loss of symmetry in the above system becomes evident in the interfacial boundary conditions
written with respect to the rate-of-strain tensor as was done in the main body of the Letter.  
Employing the general expressions \rr{tDn} and \rr{nDn}, the normal component of the rate-of-strain tensor $\mathcal{D}$ at $
z = h(x,t)$ 
\be \label{nDn1}
\mathbf{n} \mathcal{D} \mathbf{n} = \frac{  -\eta_o\frac{\partial \gamma}{\partial s}+ \eta_e (p + 2\kappa \gamma )  }{2(\eta_e^2  + \eta_o^2)}. 
\ee 
Thus, odd viscosity gives rise to an extra normal shear stress proportional to $\eta_o$. It is 
compressive on the left liquid-gas interface (cf. Fig. \rr{fig1}) since $\frac{\partial \gamma}{\partial s} >0$ there and tensile on the right, since $\frac{\partial \gamma}{\partial s} <0$ there, ($s$ is arc length along
the interface). This is the driving mechanism behind the droplet migration and 
is depicted in Fig. \ref{fig1}. Notice that this term breaks the symmetry
$x\rightarrow -x, u\rightarrow - u$ that existed in its absence. 

Likewise, in the shear component of the rate-of-strain tensor on the liquid-gas interface $
z = h(x,t)$ 
\be \label{tDn1}
\mathbf{t} \mathcal{D} \mathbf{n} = \frac{ \eta_e \frac{\partial \gamma}{\partial s}  + \eta_o (p + 2\kappa \gamma  )}{2(\eta_e^2  + \eta_o^2)}, 
\ee
the odd viscosity term breaks the $x\rightarrow -x, u\rightarrow - u$ symmetry that existed in its 
absence. In the context of the present problem the symmetry-breaking terms in \rr{tDn} lead
to a renormalization of the thermocapillary effect. In other cases however (cf. \cite{Soni2019})
the combination of \rr{nDn1} with \rr{tDn1} may lead to additional effects such as unidirectional
wave propagation. 

{The symmetry breaking mechanism of the contact lines was explained in detail for the case of 
a droplet migrating under the action 
of a magnetic torque (in the absence of odd viscosity) \cite{Aggarwal2023}. The analysis carried-out
in Eq. (7.3)-(7.6) and Fig. 14 of \cite{Aggarwal2023} can be employed here as well by making
the substitution
\be
N \rightarrow \eta M
\ee
where $N$ is the magnetic torque defined in \cite{Aggarwal2023}.

The magnetic torque-actuated \emph{isothermal} droplet (with no odd viscosity and no thermocapillarity) \cite{Aggarwal2023}
is driven by the motion of the liquid gas interface due to the collective rotation of the ferroparticles
suspended in the liquid and gives rise to a \emph{single} circulating cell. This differs
significantly from the \emph{non-isothermal} effect considered here where the thermocapillary effect
produces two cells with opposite sense of circulation and the droplet is set into motion purely by
the presence of odd viscosity. 

There are similarities however. In both the magnetic torque paper \cite{Aggarwal2023}
and in the present contribution we employ the same constitutive law for the droplet 
motion (Eq. \rr{mob0}), giving rise to the same symmetry-breaking mechanism based 
on the asymmetry of the contact angles. 
}








%
%
%The effects of symmetry breaking are also inherited by the mobility law \rr{mob0}. 
%In the absence of odd viscosity but in the presence of a temperature gradient whose strength is
%denoted by the letter $M$ (a Marangoni number) the midpoint of the droplet is immobile \cite{Ehrhard1991}:
%\be
%\frac{d (c_a + c_r)}{dt} =0.
%\ee
%This is because in the cases considered by \cite{Ehrhard1991}, the left and right dynamic contact angles are equal to each other
%$\theta_a = -h_x(c_a) = h_x(c_r) = \theta_r$ and the droplet can 
%only undergo spreading or retraction.
%Analytically, 
%\be  \label{dthetadM}
%\frac{d (c_a + c_r)}{dt} = (\theta_a(M) -\Theta_A)^m - (\theta_r(M) - \Theta_A)^m
%\sim m(\theta_a(0) - \Theta_A)^{m-1} M
%\left(\frac{d\theta_a}{dM} - \frac{d\theta_r}{dM}\right) =0
%\ee
%because $\frac{d\theta_a}{dM} = \frac{d\theta_r}{dM}$ (see the left two 
%wedges of Fig.\ref{fig4} and the mobility law \rr{mob0} - in this calculation
%we considered the approximation $\theta_a(M) \sim \theta_a(0) +M\frac{d\theta_a}{dM}(0)$, 
%$\theta_r(M) \sim \theta_r(0) +M\frac{d\theta_r}{dM}(0)$ and the fact 
%that $\theta_a(0)=\theta_r(0)$). 
%
%




%
%\begin{figure}
%\vspace{20pt}
%\begin{center}
%\setlength{\unitlength}{1.2mm}
%\begin{picture}(50,20)
%\thicklines
%
%
%% M case 
%% leftmost wedge
%\put(-25,7){\line(1,0){18}}
%\put(-25,7){\line(3,1){18}}
%\put(-25,7){\line(2,1){15}}
%\put(-20,10){\vector(-1,0){5}}
%\put(-18,7.5){$\displaystyle \theta_r(0)$}
%\qbezier(-12,11)(-10,9)(-10, 7)
%\qbezier(-10,15)(-8,14)(-8, 13)
%\put(-12,17){$\displaystyle M\frac{d\theta_r}{dM}$}
%
%%middle right wedge
%\put(20,7){\line(-1,0){18}}
%\put(20,7){\line(-3,1){18}}
%\put(20,7){\line(-2,1){15}}
%\put(10,12){\vector(1,0){5}}
%\put(5,7.5){$\displaystyle \theta_a(0)$}
%\qbezier(4,7)(4,9)(6,11)
%\qbezier(3,13)(3,14)(5, 15)
%\put(2,17){$\displaystyle M\frac{d\theta_a}{dM}$}
%
%
%\put(-10,3){spreading, $\eta \equiv \frac{\eta_o}{\eta_e} = 0$}
%
%
%
%%middle right wedge
%\put(30,7){\line(1,0){18}}
%\put(30,7){\line(3,1){15}}
%\put(30,7){\line(5,1){18}}
%\put(47,10){\vector(1,0){5}}
%\put(38,7.5){$\displaystyle \theta_r(0)$}
%\qbezier(43,11)(45,9)(45, 7)
%%\qbezier(41,13)(45,9)(45, 7)
%\qbezier(45.5,12)(48,12)(48.5, 11)
%\put(44,14){$\displaystyle \eta M\frac{d\theta_r}{d(\eta M)}$}
%
%%middle right wedge
%\put(75,7){\line(-1,0){18}}
%\put(75,7){\line(-3,1){18}}
%\put(75,7){\line(-2,1){15}}
%\put(65,12){\vector(1,0){5}}
%\put(60,7.5){$\displaystyle \theta_a(0)$}
%\qbezier(59,7)(59,9)(61,11)
%\qbezier(58,13)(58,14)(60, 15)
%\put(57,17){$\displaystyle \eta M\frac{d\theta_a}{d(\eta M)}$}
%
%\put(45,3){migration, $\eta \equiv \frac{\eta_o}{\eta_e} \neq 0$}
%
%
%\end{picture}
%\caption{\textbf{Left two wedges}: Explanation of the calculation of the mid-point velocity
%of the droplet in Eq. \rr{dthetadM}, under a vertical temperature gradient 
%\cite{Ehrhard1991}. 
%The droplet whose contact angles, in the absence of the 
%temperature gradient, are $\theta_r(M=0) =\theta_a(M=0)$, spreads under the influence of the gradient of strength $M$. The contact angle increments generated
%by it are equal: $M\frac{d\theta_r}{dM}(M=0) = M\frac{d\theta_a}{dM}(M=0)$. 
%\textbf{Right two wedges}: Explanation of the calculation of the mid-point velocity
%of the droplet in Eq. \rr{dthetadN}, under a horizontal temperature gradient 
%\cite{Smith1995} and odd viscosity.  
%The droplet whose contact angles, in the absence of the 
%temperature gradient, are $\theta_r(\eta M=0) =\theta_a(\eta M=0)$, \emph{migrates} due to the 
%presence of odd viscosity (whose strength is denoted by $\eta \equiv \eta_o/\eta_e$). The contact angle increments generated
%by it have equal magnitudes but opposite signs : $\eta M\frac{d\theta_r}{d(\eta M)}(\eta M=0) = -\eta M\frac{d\theta_a}{d(\eta M)}(\eta M=0)$. 
%\label{fig4} }
%\end{center}
%\vspace{0pt}
%\end{figure}
%
%
%
%
%In the presence of odd viscosity however, the droplet midpoint
%is instead mobile:
%\be
%\frac{d (c_a + c_r)}{dt} \neq 0.
%\ee
%Thus, in contrast to the case of spreading described by Eq.\rr{dthetadM} 
%\cite{Ehrhard1991} we
%now have $\frac{d\theta_a}{d\eta M} = -\frac{d\theta_r}{d\eta M}$ (see Fig.\ref{fig4} right two wedges; this
%can also be understood by considering Eq. \rr{F} below which is a consequence of the lubrication
%approximation)
%with $\eta \equiv \frac{\eta_o}{\eta_e}$. Thus, 
%\begin{align}
%\frac{d (c_a + c_r)}{dt} &= (\theta_a(M,\eta M) -\Theta_A)^m + (\Theta_R - \theta_r(M,\eta M))^m
%\nonumber \\
%&\sim m\eta M(\theta_a(M,0) - \Theta_A)^{m-1}
%\left(\frac{d\theta_a}{d\eta M} - \frac{d\theta_r}{d\eta M}\right) =
%2\eta M
%\frac{d\theta_a}{d\eta M} \neq 0 \label{dthetadN}, 
%\end{align}
%where in the last equality we set $m=1$ and $\theta_a(M,0) =\Theta_A$ and $\Theta_R = \theta_r(M,0)$
%(since there is no migration in the absence of odd viscosity, see \cite{Ehrhard1991}). 
%Thus, symmetry-breaking, evident in the Navier-Stokes equations, its various reductions and 
%constitutive laws, accompanies
%the migration of liquid droplets. 



\begin{table*}[t]%The best place to locate the table environment is directly after its first reference in text
\caption{\label{tab:table1}%
Experimental data taken from \cite{Blake2019} and \cite{Haynes2016} for di-n-butyl phthalate (DBP), valid
between $\SI{15}{\celsius}$ and $\SI{55}{\celsius}$ and curve-fitted.}
\begin{ruledtabular}
\begin{tabular}{lcl}
\textrm{Quantity}&
\textrm{Value}&
%\multicolumn{1}{c}{\textrm{Decimal}}&
\textrm{Definition}\\
\colrule
$\rho$ ( $\textrm{kg}\:\textrm{m}^{-3} $)    & $\frac{1}{80000} T^{4}-\frac{11}{6000} T^{3}+\frac{151}{1600} T^{2}-\frac{661}{240} T +\frac{137553}{128}$        & density \\
$\eta_e$ ( mPa$\cdot$ sec)    & $4.166666667\times 10^{-6} T^{4}- 0.0008T^{3}+ 0.06495833333 T^{2}- 2.73T + 55.2234375$        & shear (or even) viscosity \\
$a$ (m)  & $1\times10^{-3}$& characteristic droplet radius\\
$\gamma$ (mN $\textrm{m}^{-1}$) &$4.16666\times 10^{-7} T^{4}- 0.000066666 T^{3}+ 0.003895833 T^{2}- 0.198333 T + 37.3023$& surface tension \\
$k_{th}$ (W $\textrm{m}^{-1}K^{-1}$) &$1.066666\times 10^{-8} T^{3}- 1.6\times 10^{-6} T^{2}- 0.00008666T + 0.139$& thermal conductivity  \\
$T_0$  \SI{}{\celsius} &50 & solid substrate temperature\\
$T_\infty$  \SI{}{\celsius} &20 & gas phase temperature\\
$\Theta_A, \Theta_R$ degrees& 86.45&  static contact angles
%$\Theta_R$ degrees & 86.45 &  static receding contact angle
\end{tabular}
\end{ruledtabular}
\end{table*}



\section{\label{generalF}General expression for migration velocity in the perturbative regime}
Broken symmetry and a general expression for the migration velocity 
depend on the constitutive law \rr{mob0} and is thus identical to the migration of droplets
induced by different mechanisms \cite[Appendix B]{Aggarwal2023}. Thus, to avoid 
possible duplication here, as in the previous section, \emph{we have} to state only the final results. 
%We recall that in the case of thermocapillary spreading of a viscous droplet 
%\citep{Ehrhard1991}, the left and right contact lines are
%simultaneously advancing or receding. Without loss of generality, we consider only the 
%advancing case (as was developed in the thermocapillarity problem \citep{Ehrhard1991}): the left and right contact angles are always equal
%\be \label{Fthermo}
%\theta_a= F_0 + M F_1, \quad \textrm{and} 
%\quad
%\theta_r= F_0 + MF_1,
%\ee
%where $F_0 = -\frac{3}{2a^2}$, $F_1 = \frac{3}{2}a$, $M\ll1$
%%$F_0$ and $F_1$ are two functions depending on the natural parameters of the 
%%liquid (surface tension, viscosity) and external fields such as gravity and $M$
% is the Marangoni number
%(proportional to the temperature gradients responsible for the spreading behavior)
%and $a=a(t)$ is the droplet spreading radius. 


In the presence of odd viscosity, the left-right symmetry of contact angles breaks
%or purely magnetic 
%torque migration with the nanoparticles are immobile at this interface 
%($\Lambda=0, \mathcal{N}\neq 0$): Let 
\be \label{F}
\theta_a= F_0 +\eta M F_1, \quad \textrm{and} 
\quad
\theta_r= F_0 - \eta MF_1. 
\ee
The explicit form of $F_0$ and $F_1$ is not needed 
here {(the expression $F_0$ in \rr{F} depends on $M$, see Eq. \rr{thetara} below).}
%When the droplet has reached  its steady-state migration shape and velocity (i.e. moving as a whole without deforming and at constant speed), the contact
%angles are related by 
%\be
%\theta_a - \Theta_A = \Theta_R - \theta_r, \quad \textrm{and} \quad \theta_a-\Theta_A>0,
%\quad 
%\theta_r\geq 0 .
%\ee
%Employing Eq. \rr{F}, the above expressions lead to 
%\be \label{taptr}
%\theta_a+\theta_r = 2F_0 = \Theta_A + \Theta_R, \quad
%\ee
%and the inequality 
%\be
%0< \frac{1}{2}(\Theta_A - \Theta_R) < \eta M F_1,
%\ee
%{where the latter inequality is a consequence of $\theta_r-\Theta_r<0$, $\theta_a-\Theta_A>0$ leading
%$\theta_a-\Theta_A>\theta_r-\Theta_R$ and using Eq. \rr{F}. In the Letter and the SI we have tacitly 
%assumed that $\eta_o>0$. }
%Employing Eq. \rr{taptr}
Employing the arguments on contact angles stated in the Appendix of \cite{Aggarwal2023}, 
the migration velocity is then given by 
\be \label{UAR}
U = (\theta_a - \Theta_A)^m =\left[ F_0 + \eta M F_1 - \Theta_A\right]^m = \left[\eta M  F_1 - \frac{\Theta_A-\Theta_R}{2}\right]^m.
\ee
In the absence of hysteresis effects ($\Theta_A = \Theta_R \neq 0$) the migration velocity reduces to 
\be \label{U}
U = \left(\eta M F_1\right)^m. 
\ee



\section{Quantitative description of migration velocity}
To gain a quantitative insight into the odd viscosity droplet migration effect, we consider the problem of migration by employing the existing formulation
of droplet spreading and migration developed by Ehrhard \& Davis \cite{Ehrhard1991} and Smith \cite{Smith1995}, based on the lubrication approximation, 
and proceed on an \emph{ad-hoc} basis. 
The acquired closed form droplet migration velocity $U$ agrees with our numerical simulations, as this was established in
the main section of this Letter (for instance, by comparing Eq. (2) of the Letter with the Letter's Fig. 2). To this end, and starting from equation \rr{nDn} we consider the 
following line of thought
\be
\frac{\partial \gamma}{\partial s}\sim \frac{d \gamma}{dT}\frac{dT}{dh}h_x = 
\frac{\alpha_{th} \Delta T }{k_{th}} \frac{d\gamma}{dT} h_x
\ee
where we employed the temperature profile $T(h) = \frac{\Delta T}{1 + Bi h/h_0} + T_\infty$
and $Bi = \frac{\alpha_{th} h_0}{k_{th}}$
\cite{Ehrhard1991}, $h_0$ is a characteristic size for the droplet and the arc length has been replaced
by the horizontal coordinate (strictly, valid only in the long wavelength approximation). $T_\infty$ is the 
ambient gas phase temperature {and we employed the small Biot number limit which is characteristic for
most liquids \cite{Oron1997}.  }

Consider a viscous droplet, incorporating the effects of odd viscosity, sitting on a heated solid substrate and 
surrounded by a colder ambient gas phase and denote its liquid-gas interface by $z=h(x,t)$ (cf.~Fig.~\ref{fig1}).    
Conservation of momentum and energy
leads to velocity profiles derived by Ehrhard and Davis \cite[Eq.~(4.11p)]{Ehrhard1991}
with an amended pressure gradient due to the term $\eta_o\frac{\partial \gamma}{\partial s}$
appearing in the normal stress boundary condition \rr{nDn1}. 
%This analysis  shows that the flow is dominated by a pressure gradient
%\be
%\bar p_x = \frac{1}{C}\left( h_{xx} - Gh - \eta\frac{M h_x}{(1+Bh)^2} \right), 
%\ee
%and the expression for thermocapillary shear stresses  
%\be
%\sigma_x = \frac{1}{C}\frac{M h_x}{(1+Bh)^2} 
%\ee
In the quasistatic limit of small Biot and capillary numbers 
the dynamics is determined by the balance of pressure and thermocapillary boundary
motion
\be\label{bal2}
-\frac{1}{3}h^3 \bar p_x + \frac{1}{2}h^2\sigma_x =0,
\ee
where 
$-\bar p_x=\left[ h_{xx} - \eta {M h_x}
\right]_x$ and $\sigma_x = Mh_x$ by employing the dimensionless units developed in \cite{Ehrhard1991}. 
%$h_{xx}$ is the interface curvature. When positive, this pressure pushes troughs up
%and when negative pulls peaks down. 
$M$ is the product
of Marangoni number (measures the ratio of the temperature gradient to the mobility of the contact line),
Biot number (measures the ratio of 
the rate of heat transfer from liquid to gas, to liquid thermal conductivity) and capillary number (measures 
the ratio of contact line mobility to surface tension) and is given by 
%$M$ above is the product of Capillary, Biot and Marangoni numbers
%\be
%Ca = \frac{U_0 \eta_e}{\theta_0^3 \gamma_0}, \quad 
%Bi = \frac{\alpha_{th} a_0 \theta_0}{k_{th}}, \quad
%Ma = \frac{\theta_0k\Delta T}{\eta_e U_0}.
%\ee
\be
M =  \frac{ \alpha_{th}a\Delta T}{\theta_0\gamma k_{th}}\left|\frac{d \gamma}{dT}\right|
\ee
where $\theta_0$ is a small angle which, however, drops out when dimensional units are used, 
$|\cdot|$ denotes the absolute value of a real number and $a$ is the droplet radius.  

The effect discussed in this Letter arises due to the term 
$\eta M h_x$ in the pressure gradient, where $\eta$ is the ratio of odd viscosity to its (even) standard
counterpart.
%the effect is caused by surface shear
%due to surface-tension gradients (induced by the applied temperature-gradients).  
When the plate is heated ($M>0$) there is a compressive pressure on the left cell
(clockwise circulating) since $h_x>0$ there, and a tensile pressure on the 
right cell (anticlockwise circulating) where $h_x<0$. 
When the plate is cooled, ($M<0$) the circulation patterns and pressure directions just described are 
reversed. 


Reverting to a frame of reference whose origin $\bar{c}(t)$ lies at the mid-point of the droplet base and denoting its spreading
radius by $a(t)$ \cite{Smith1995}, (cf.~Fig.~\ref{fig1}) we can solve Eq.~\rr{bal2} considering that the droplet height vanishes 
at the two contact angles
$
h(\pm a,t) = 0, $
and its volume is a constant
$ \int_{-a}^{a} h(x,t) dx = 1 
$
{(we employ dimensionless units in this calculation that have been defined in \cite{Smith1995} and \cite{Aggarwal2023}. The radius $a$ was nondimensionalized by its counterpart $a_0$
attained during migration).}
We expand all fields in a perturbation series
$
h = h_0  + M h_1 +..., 
$
and in the absence of gravity
we obtain the following liquid-gas interface (dimensionless) profile
\begin{widetext}
\be \label{h01}
h(x,t) \sim \frac{3}{4} \left\{ \frac{a^2 - x^2}{a^3} +M \left[\frac{\eta x (a^2 - x^2)}{3a^3} + 
2(a^2 + x^2)\ln(2a) - \frac{4}{3}(a^2  - x^2) - (a-x)^2 \ln(a-x) - (a+x)^2 \ln(a+x) \right] \right\}.
 \ee
\end{widetext}
%
%\begin{figure} [tbp]
%\begin{center}
%\includegraphics[height=1.6in,width=2.2in]{droplet_streamlines_eta0}
%\end{center}
%\vspace{-5pt}
%\caption{Nonisothermal spreading in the absence of odd viscosity ($\eta=0$ and $M=0.2$, cf. \cite[Fig.~5]{Ehrhard1991}). A vertical temperature gradient
%creates two identical cells with opposite sense of circulation that oppose spreading. 
%\label{droplet_streamlines_eta0} }
%\end{figure}


Notice the asymmetry induced on the liquid-gas interface profile \rr{h01} by odd viscosity. Since
$|x|\leq a$, the profile is elevated to the right of the droplet and is suppressed to the left. When $\eta\equiv 0$ one recovers the spreading results of Ehrhard and Davis \cite[Eq.~(7.2p)]{Ehrhard1991} where
two identical cells with opposite sense of circulation are formed in the interior of the droplet (cf.~Fig. 1 and 2
of the main part of this Letter). {Also note that Eq. \rr{h01} was also derived by Smith \cite{Smith1995}
with the exception of the term multiplying the coefficient $\eta$. }

Employing \rr{h01}, the left and right dynamic contact angles $\theta_r = h_x(-a)$ and $\theta_a = -h_x(a)$ are, respectively, 
\be \label{thetara}
\theta_{a,r} = \frac{3}{2a^2} (1 - \frac{1}{3}Ma^3 \pm \frac{1}{3} \eta M a).
\ee
 


Surface roughness, chemical contaminations and solutes may lead to contact line pinning whenever
the contact angle $\theta$ lies within a finite interval $\Theta_R<\theta<\Theta_A$ \cite{deGennes1985}. 
This is the phenomenon of contact-angle hysteresis and has been experimentally documented by Dussan V \cite[Fig.~2]{Dussan1979}
relating contact angle with contact line velocity $U_{cl}$. In modeling this behavior, one finds that the contact
line moves with a velocity \cite{Oron1997}
\be \label{Ucl}
U_{cl} \sim \pm
\left\{  \begin{array}{cc}
(\theta - \Theta_A)^m, & \theta>\Theta_A,\\  
-(\Theta_R - \theta)^m, & \theta < \Theta_R
\end{array}
\right.
\ee
where $\theta$ is the dynamic contact angle (here denoted either as $\theta_a$ or $\theta_r$, cf. Fig.~\ref{fig1})
and the upper/lower sign is identified with the motion of the right/left contact line (Eq. \rr{Ucl} is a 
simplified version of the exact formula \rr{mob0}).  
Theoretical analyses \cite{Ehrhard1991,deGennes1985} and multiple experiments from different groups %of Tanner \cite{Tanner1979}, Chen \cite{Chen1988}, Ehrhard \cite{Ehrhard1993}
\cite{Marsh1993,*Ehrhard1993,*Tanner1979, *Chen1988} determine the value of the exponent $m$. 
In this Letter we determine that $m\sim 1$ employing numerical simulations. 

The motion of the droplet is composed of a transient and a steady-state configuration. Contact angle hysteresis endows the 
transient with diverse behavior:
spreading, pinning and migration where left and right contact line velocities are unequal.
These cases, exhaustively classified by Smith \cite{Smith1995} for a different problem, 
are applicable here as well. For instance, Eq.~\rr{thetara} shows that the droplet may spread
but the right-hand contact angle spreading will always be faster. 



We are here interested on the steady-state configurations
attained by the migrating droplet. When the droplet has reached  its steady-state migration shape and velocity, 
Eq.~\rr{Ucl} implies that the dynamic contact
angles are related by (cf.~Fig.~\ref{fig1} for the definition of the angles)
\be
\theta_a - \Theta_A = \Theta_R - \theta_r, \quad \textrm{and} \quad \theta_a-\Theta_A>0. 
\ee
Employing Eq. \rr{thetara}, the first of the above expressions leads to 
$
\theta_a+\theta_r = \frac{3}{a^2}(1 - \frac{1}{3}Ma^3) = \Theta_A + \Theta_R, \quad
$
and the second to the inequality 
$
0< \frac{1}{2}(\Theta_A - \Theta_R) < \frac{\eta M}{2R}. 
$
Eq.~\rr{Ucl} leads to the migration (dimensionless) velocity
\be
U = (\theta_a - \Theta_A)^m = \left[ \frac{\eta M}{2a} - \frac{\Theta_A-\Theta_R}{2}\right]^m,
\ee
(applicable for a general $m$, see the discussion is section \ref{generalF}. 
In this article we consider migration of droplets in the absence of hysteresis ($\Theta_A = \Theta_R \neq 0$) and set $m=1$, thus, the migration velocity reduces to 
\be \label{UM2a}
U = \frac{\eta M}{2a}
\ee
which is the dimensionless counterpart of Eq. (2) of the Letter. 
Thus, droplet migration on a vertical temperature gradient is a purely odd viscosity-induced effect. 

The dimensional counterpart of Eq. \rr{UM2a} is given by (this is Eq.(2) of the Letter)
\be \label{Ucl0}
U \sim K \frac{\eta_o}{\eta_e}\frac{ \alpha_{th}a\Delta T}{2 k_{th}\gamma}\frac{d \gamma}{dT}.
\ee
Here, $\alpha_{th}$ is the 
heat transfer coefficient, $k_{th}$ is the liquid thermal conductivity satisfying $-k_{th}\mathbf{n} \cdot\nabla T + \alpha_{th}(T-T_\infty)=0$, at the liquid-gas interface, $T_\infty$ is the ambient gas phase temperature, $a$ is the droplet radius, $\Delta T$ the temperature difference between the hot wall
and the cold air, 
%$\theta_0$ a reference small contact angle 
and $\gamma_0$ is the surface tension at wall 
temperature. $K$ is the mobility coefficient (units cm/sec). 
We have performed extensive numerical checks by varying \emph{all} the parameters appearing in 
Eq. \rr{Ucl0} and verified that they agree well in the vicinity of the currently available experimental 
values for the viscosity ratio $\eta_o/\eta_e\sim 1/3$ and that the exponent $m$ remains equal
to $1$. 


This Letter paves the way to utilize odd viscosity in manipulating structures that incorporate
recirculating cell patterns. Interfacial driving mechanisms are particularly important in microfluidic devices
due to their large surface-to-volume ratio. Current interest lies on discrete or 
continuous microscopic control of small-scale systems and applications vary 
from clinical and forensic analysis to semiconductor devices and environmental 
monitoring \cite{Darhuber2005}. 

%\begin{figure}[t] 
%\vspace{-5pt}
%\begin{center}
%\includegraphics[height=3in,width=4in]{SI_fig3low}
%\end{center}
%\vspace{-15pt}
%\caption{Droplet migration velocity for large values of the the viscosity ratio shows a dip. 
%The source of this behavior is the mismatch between the time-scales of rotation of the thermocapillary cells and the degrees of freedom associated with odd viscosity, as this is discussed in section \ref{sec: dip}.
%Thus, for the ratio of viscosities $<1$, the slopes is equal to $1$, as is the slope of the timescale ratio
%in Eq. \rr{tauMtau0}. For ratio of viscosities $>1$ the trend is (approximately) 
%like the inverse of viscosities ratio, as is the slope of the timescale ratio
%in Eq. \rr{tauMtau0} (the latter is exactly equal to the inverse of viscosities ratio). 
%The two
%curves correspond to different values of the slip length at the liquid-solid interface.  
%\label{SI_fig3} }
%\vspace{-5pt}
%\end{figure}
%
%\section{\label{sec: dip}Justification for the high odd viscosity migration velocity dip}
%Fig.\ref{SI_fig3} shows that for values of odd viscosity close to its even (shear) viscosity counterpart, 
%the migration velocity reaches a maximum. It starts decaying when odd viscosity exceeds the 
%shear viscosity value. Below we provide an explanation of this effect. 
%%
%%\begin{figure}[t] 
%%\vspace{-5pt}
%%\begin{center}
%%\includegraphics[height=3in,width=4in]{omega0omegaM2}
%%\end{center}
%%\vspace{-15pt}
%%\caption{Plot of the ratio of frequencies in \rr{tauMtau0} versus odd viscosity with a shear viscosity
%%$\eta_e$ for paraffin oil. 
%%\label{omega0omegaM} }
%%\vspace{-5pt}
%%\end{figure}
%
%The frequency over which thermocapillary liquid circulation takes place is \cite{Tam2009}
%\be \label{f1}
%\frac{1}{\tau_M} = \frac{\frac{d\gamma}{dT} \Delta T}{\eta_e a_0}. 
%\ee
%On the other hand the condition \rr{nDn1} at the liquid-gas interface, that is responsible for 
%the effect discussed in this Letter, gives rise to a second time-scale
%\be\label{f2}
%\frac{1}{\tau_{\eta_o}} = \frac{\eta_o\frac{d\gamma}{dT} \Delta T}{(\eta_e^2 + \eta_o^2) a_0}.
%\ee
%For example, for the physical values 
%low volatility
%ester, di-n-butyl phthalate \cite{Blake2019}, cf. Table I of the main article
%$\Delta T=\SI{25}{\celsius}$, $ T=\SI{25}{\celsius}$ and $\eta_o/\eta_e = 1/3$ 
%we obtain
%%---------------------------------------------
%% Estimates in Hall viscosity/di-n-butyl.mw
%%---------------------------------------------
%\be
%\frac{1}{\tau_M} =153 \textrm{ Hz}, \quad \frac{1}{\tau_{\eta_o}} = 49 \textrm{ Hz}.
%\ee
%\begin{figure}[t] 
%\vspace{-5pt}
%\begin{center}
%\includegraphics[height=3in,width=3.5in]{omega0omegaM3}
%\end{center}
%\vspace{-15pt}
%\caption{Plot of the ratio of timescales in \rr{tauMtau0} versus the viscosities ratio for comparison with Fig. \ref{SI_fig3}. Thus, for the ratio of viscosities $<1$, the slope is equal to $1$, as is the slope of the curve
%in Fig. \ref{SI_fig3}. For ratio of viscosities $>1$ the trend goes like
%the inverse of viscosities ratio, as is the slope of the curve
%in Fig. \ref{SI_fig3} (approximately).  
%\label{omega0omegaM} }
%\vspace{-5pt}
%\end{figure}
%%Odd viscosity is associated with rotating the existing shear viscosity-dominated flow by ninety degrees. 
%To provide a physical explanation of the velocity dip in Fig. \ref{SI_fig3} we  
%identify the value of odd viscosity with the intrinsic angular momentum of constituent matter
%(eg. magnetic nanoparticles) that rotate under the application of an external torque \cite{Soni2019}
%(but we make no such reference in this Letter). 
%We can make this association because it is well known in the literature of the fractional quantum Hall effect 
%\cite{Avron1995} that odd viscosity is proportional to the average orbital spin per particle, and 
%in addition, it is proportional to the intrinsic angular momentum of a two-dimensional fluid \cite{Monteiro2016}. 
%Thus, the dip experienced by the migration velocity can be associated with the asynchronous motion
%of these particles with respect to the liquid circulation. When particles start rotating faster than
%the liquid circulation due to thermocapillarity, a decrease in migration velocity is observed. 
%This is a common characteristic of 
%spinning particles in active matter (see for instance \cite[Fig. S1]{Driscoll2017}). 
%This type of behavior was also observed when odd viscosity was present in a liquid with a free surface
%in a corner, populated by Moffatt vortices \cite{Kirkinis2022a}. In this case, increase of odd viscosity
%led to decrease in the size and intensity of the vortices. 
%
%
%
%The ratio of the two frequencies \rr{f1} and \rr{f2} gives
%\be\label{tauMtau0}
%\frac{\tau_M}{\tau_{\eta_o}} = \frac{\eta_o \eta_e}{\eta_e^2 + \eta_o^2}. 
%\ee
%The curve reaches a maximum when $\eta_o\sim \eta_e$ and then starts decaying. 
%In figure \ref{omega0omegaM} we display the ratio \rr{tauMtau0} for a number of viscosity ratio
%$\eta_o/\eta_e$ values. Note that the slopes of the curves in both Figures \ref{SI_fig3} 
%and \ref{omega0omegaM} follow the general trend
%\be
%\frac{\tau_M}{\tau_{\eta_o}} \sim
%\left\{ \begin{array}{cc}
%\frac{\eta_o}{\eta_e}, & \frac{\eta_o}{\eta_e} \ll1, \\
%\frac{1}{\frac{\eta_o}{\eta_e}}, & \frac{\eta_o}{\eta_e} \gg1. 
%\end{array}
%\right.
%\ee


%In figure \ref{omega0omegaM} we display the ratio \rr{tauMtau0} for a number of odd viscosity values. 
%The curve reaches a maximum when $\eta_o\sim \eta_e$ and then starts decaying. 



%\begin{table}[t]%The best place to locate the table environment is directly after its first reference in text
%\caption{\label{tab:table1}%
%Experimental data taken from \cite{Blake2019} and \cite{Haynes2016} for di-n-butyl phthalate (DBP), valid
%between $\SI{15}{\celsius}$ and $\SI{55}{\celsius}$}
%\begin{ruledtabular}
%\begin{tabular}{llllll}
%\textrm{Temp.} $\SI{}{\celsius}$&
%$\eta$ ( mPa sec) &
%$\gamma$ (mN $\textrm{m}^{-1}$) &
%$\rho$ (kg $\textrm{m}^{-3}$) &
%\textrm{Temp.} $\SI{}{\celsius}$&
%$k_{th}$ (W $\textrm{m}^{-1}K^{-1}$) 
% \\
%\colrule
%15 &26.4 &35.0 &1049 &0& 0.139 \\
%25 &16.7 &33.9 &1041&25&0.136\\
%35 &11.2 &32.9 &1034&50&0.132\\
%45 &8.1 &31.9 &1026&75&0.128\\
%55 &6.6 &30.9 &1018&100& 0.125
%\end{tabular}
%\end{ruledtabular}
%\end{table}







\section{Numerical method}
We use a commercial finite element software, COMSOL\cite{comsol2022comsol} to model the odd viscous droplets studied in this letter. COMSOL uses the \textit{moving mesh} method defined within the framework of Arbitrary Lagrangian-Eulerian (ALE) formulation \cite{donea2004arbitrary} to define a deformable computational domain for the fluid (droplet). The geometric surface of the fluid domain acts as the free surface that separates the fluid from the surroundings. The interfacial effects due to the surrounding gas phase are modeled via surface tension (or any other boundary forces) applied directly on the fluid surface as boundary conditions. The advantage of using the moving mesh method is that only the fluid domain is explicitly defined. This leads to a substantial improvement in performance compared to other commonly used interface capturing methods such as the level set \cite{sethian2003level} or volume of fluid \cite{hirt1981volume} methods because we only solve for the fluid domain and ignore the gas domain. Additionally, the moving mesh method provides a much sharper and accurate interface because it is modeled as a geometric surface. However, it has the limitation of not being able to handle topological changes, such as a droplet splitting into two. Fortunately, this limitation is not relevant for the droplet system we study here.










\subsection{Modeling the equation of fluid motion}
We use COMSOL's in-built \textit{laminar flow} and \textit{heat transfer in fluids} modules to model the system. However, COMSOL assumes the viscosity to be a scalar field. In order to account for the odd stress $\boldsymbol{\sigma^o}$ in the model, we need to look at the weak formulation of governing equations and input the extra terms manually.

The governing equations for incompressible fluid motion are 

\begin{equation} \label{eq:NS-eom}
    \rho (\partial_t \boldsymbol{u} + (\boldsymbol{u}\cdot \nabla)\boldsymbol{u}) = -\nabla p +\nabla \cdot \boldsymbol{\sigma} + \boldsymbol{f}
\end{equation}
\begin{equation}\label{eq:NS-incomp}
    \nabla \cdot \boldsymbol{u} = 0
\end{equation}
where $\boldsymbol{\sigma}$ is the Cauchy stress tensor and includes both the even and odd stress tensors i.e. $\boldsymbol{\sigma}=\boldsymbol{\sigma^e}+\boldsymbol{\sigma^o}$. The boundary condition on the liquid-gas interface (free surface) is

\begin{equation} \label{eq:bc_free}
    \boldsymbol{\sigma} \cdot \boldsymbol{n_f} = \gamma (\nabla_s \cdot \boldsymbol{n_f}) \boldsymbol{n_f} + \frac{\partial \gamma}{\partial s} \boldsymbol{t_f}
\end{equation}
where $\boldsymbol{n_f}$ and $\boldsymbol{t_f}$ are normal and tangent vectors to the free surface respectively. The boundary conditions on the solid-liquid interface are:
{
\begin{align}
    \boldsymbol{u}\cdot \boldsymbol{n_s} &= 0 \nonumber\\ 
    \boldsymbol{\sigma \cdot n_s} &= -\frac{1}{\beta} \begin{pmatrix} \eta_e & -\eta_o \\ \eta_o & \eta_e \end{pmatrix} (\boldsymbol{u} \cdot \boldsymbol{t_s}) \boldsymbol{t_s}  \label{eq:bc_solid} 
\end{align}
where $\beta$ is the slip length} and $\boldsymbol{n_s}$ and $\boldsymbol{t_s}$ are the normal and tangent vectors to the solid surface. The above equations are no penetration and the Navier-slip conditions respectively with the full viscosity tensor. However, we can show that the odd viscous part of the Navier-slip condition gets evaluated to zero due to the no-penetration condition (see section \ref{sec:weak_formulation}). 

\subsection{The weak formulation} \label{sec:weak_formulation}
To derive the weak form of the Navier-Stokes equations, we choose suitable test functions $\boldsymbol{\nu}$ and $q$, multiply them with equations \ref{eq:NS-eom} and \ref{eq:NS-incomp} respectively and integrate over the computational domain $\Omega$: 

\begin{equation} \label{eq:wf_1}
    \int_\Omega \rho \left(\partial_t \boldsymbol{u} + (\boldsymbol{u}\cdot\nabla)\boldsymbol{u} \right)\cdot \boldsymbol{\nu} d\Omega = \int_\Omega (\nabla \cdot \boldsymbol{\sigma}) \cdot \boldsymbol{\nu} d\Omega + \int_\Omega \boldsymbol{f} \cdot \boldsymbol{\nu} d\Omega
\end{equation}
\begin{equation}\label{eq:wf_2}
    \int_\Omega (\nabla \cdot \boldsymbol{u}) q d\Omega = 0
\end{equation}




\begin{figure}[t] 
\vspace{-5pt}
\begin{center}
\includegraphics[height=3in,width=4in]{SI_fig1low}
\end{center}
\vspace{-15pt}
\caption{The effect of slip length on migration velocity
\label{SI_fig1} }
\vspace{-5pt}
\end{figure}




We focus here on the term containing the stress tensor as all other terms have already been implement in COMSOL. For a full derivation of the weak formulation of Navier-Stokes equations, please refer to \cite{ganesan2015,wind2014finite}. From eq. \ref{eq:wf_1}, expanding the term with the stress tensor,

\begin{align}
        \int_\Omega (\nabla \cdot \boldsymbol{\sigma})\cdot \boldsymbol{\nu} &= \int_\Omega \partial_k (\sigma_{ik})\nu_i d\Omega \nonumber\\
        &=  - \int_\Omega \sigma_{ik} \partial_k \nu_i d\Omega +\int_\Omega \partial_k (\sigma_{ik} \nu_i) d\Omega \nonumber\\
        &=  - \int_\Omega (\boldsymbol{\sigma^e}+\boldsymbol{\sigma^o}): (\nabla \boldsymbol{\nu}) d\Omega+ \int_{\partial \Omega}\boldsymbol{\nu} \cdot (\boldsymbol{\sigma^e}+\boldsymbol{\sigma^o} )\cdot \boldsymbol{n} ds \label{eq:weak_form}
\end{align}
where `:' is the is the Frobenious inner product i.e. $\boldsymbol{A}:\boldsymbol{B} = tr(\boldsymbol{A}^T\boldsymbol{B})$ and $\boldsymbol{n}$ is the normal vector to the domain boundary. Analysing the volume integral from eq. \ref{eq:weak_form}, we find that the even stress terms are already implemented by COMSOL. We can therefore isolate the `extra' volumetric term from the weak expression, that is

\begin{align}
    \int_{\Omega} -  \boldsymbol{\sigma^o}: (\nabla \boldsymbol{\nu}) d\Omega &= \int_{\Omega} -  \sigma^o_{ik} \partial_k \nu_i d\Omega\\
    &= \int_{\Omega} - \left(\sigma^o_{xx} \partial_x \nu_1 + \sigma^o_{xy} \partial_y \nu_1 + \sigma^o_{yx} \partial_x \nu_2 + \sigma^o_{yy} \partial_y \nu_2 \right) d\Omega
\end{align}

The test function provided by COMSOL is given by $\nu_i = ( test(u), test(v) )$ and its derivatives $\partial_k \nu_i$ are given by $test(u_x)$, $test(u_y)$, $test(v_x)$ and $test(v_y)$. Therefore, the weak expression is

\begin{equation} \label{eq:extra_weak_contri}
    \int_{\Omega} - \left(\sigma^o_{xx} test(u_x) + \sigma^o_{xy} test(u_y) + \sigma^o_{yx} test(v_x) + \sigma^o_{yy} test(v_y) \right)
\end{equation}
We use COMSOL's \textit{weak contribution} functionality to add this `extra' term to the built-in Navier-Stokes equation. 

Similarly, we analyse the boundary integral from eq. \ref{eq:weak_form} to isolate any `extra' boundary terms that need to be implemented. The boundary integral is split into the free surface boundary $\partial \Omega_f$ and the solid-liquid interface boundary $\partial \Omega_s$, 
\begin{align} \label{eq:boundary_integral}
    \int_{\partial \Omega}\boldsymbol{\nu} \cdot (\boldsymbol{\sigma^e}+\boldsymbol{\sigma^o}) \cdot \boldsymbol{n} ds = \int_{\partial \Omega_f}\boldsymbol{\nu} \cdot (\boldsymbol{\sigma^e}+\boldsymbol{\sigma^o} ) \cdot \boldsymbol{n_f} ds + \int_{\partial \Omega_s}\boldsymbol{\nu} \cdot (\boldsymbol{\sigma^e}+\boldsymbol{\sigma^o} ) \cdot \boldsymbol{n_s} ds
\end{align}
For the free surface integral, substituting boundary conditions from eq. \ref{eq:bc_free}, we get

\begin{equation}
    \int_{\partial \Omega_f} \boldsymbol{\nu} \cdot (\boldsymbol{\sigma^e}+\boldsymbol{\sigma^o} ) \cdot \boldsymbol{n_f} ds = \int_{\partial \Omega_f} \boldsymbol{\nu} \cdot \left(\gamma (\nabla_s . n_f) \boldsymbol{n_f} + \frac{\partial \gamma}{\partial s} \boldsymbol{t_f} \right) ds
\end{equation}
which is the standard boundary condition implemented by COMSOL's laminar flow module on the free surface. On the solid-liquid interface boundary, substituting eq. \ref{eq:bc_solid} in eq. \ref{eq:boundary_integral},  

\begin{align}
    \int_{\partial \Omega_s} \boldsymbol{\nu} \cdot (\boldsymbol{\sigma^e}+\boldsymbol{\sigma^o}) \cdot \boldsymbol{n_s} ds &= \int_{\partial \Omega_s}\boldsymbol{\nu} \cdot \left( -\frac{\eta_e}{\beta} (\boldsymbol{u} \cdot \boldsymbol{t_s}) \boldsymbol{t_s} - \frac{\eta_o}{\beta} \begin{pmatrix} 0 & -1 \\ 1 & 0 \end{pmatrix} (\boldsymbol{u} \cdot \boldsymbol{t_s}) \boldsymbol{t_s} \right)ds \nonumber\\
    &= \int_{\partial \Omega_s}\left[-\frac{\eta_e}{\beta} (\boldsymbol{u} \cdot \boldsymbol{t_s}) (\boldsymbol{\nu} \cdot \boldsymbol{t_s}) - \frac{\eta_o}{\beta}(\boldsymbol{u} \cdot \boldsymbol{t_s}) \left(\boldsymbol{\nu} \cdot \begin{pmatrix} 0 & -1 \\ 1 & 0 \end{pmatrix} \boldsymbol{t_s}\right) \right] ds \label{eq:slip_wf}
\end{align}
Notice that the second term on the right side of the above equation (eq. \ref{eq:slip_wf}) gets evaluated to zero because of the no penetration condition $\boldsymbol{\nu} \cdot \boldsymbol{n_s} = 0$. Therefore, we are left with the following equation

\begin{align}
    \int_{\partial \Omega_s} \boldsymbol{\nu} \cdot (\boldsymbol{\sigma^e}+\boldsymbol{\sigma^o}) \cdot \boldsymbol{n_s} ds = \int_{\partial \Omega_s}\left(-\frac{\eta_e}{\beta} (\boldsymbol{u} \cdot \boldsymbol{t_s}) (\boldsymbol{\nu} \cdot \boldsymbol{t_s}) \right) ds  
\end{align}
which is again the standard Navier-slip condition used by COMSOL's laminar flow module on the solid-liquid interface. Therefore, the weak expression given by eq. \ref{eq:extra_weak_contri} is the only term that needs to be added to the Navier-Stokes equations in COMSOL to model the odd viscous liquid.
In Fig. \ref{SI_fig1} we display the effect of slip length on migration velocity. 



\begin{figure}
    \centering
    \includegraphics[width=\linewidth]{benchmarklow}
    \caption{(a) The geometry of a two-dimensional pipe with odd viscous fluid flowing under a constant pressure gradient. (b) Plot of transverse pressure, $P_\perp$ as a function of $\eta_o/\eta_e$ for the Poiseuille flow with odd viscosity. The plot benchmarks the FEM results against the analytical expression for different sets of system parameters. }
    \label{fig:benchmark}
\end{figure}


{The equations of heat transfer are given by:

\begin{equation}
    \rho c_p (\partial_t T + \boldsymbol{u}\cdot \nabla T) = k_{th} \nabla^2 T, 
\end{equation}
where $T$ is the temperature, $c_p$ is the heat capacity, $k_{th}$ is the thermal conductivity of the fluid. The boundary condition used on the fluid-surface interface is $T_0 = T_{gas} + \delta T$, where $\delta T$ was set as $30^\circ C$. On the fluid-gas interface, we use:

\begin{equation}
    k_{th} \boldsymbol{n}\cdot\nabla T = \alpha_{th} (T_\infty - T)
\end{equation}
where $\alpha_{th}$ is the heat transfer coefficient.

The initial conditions used are:
\begin{align}
    \boldsymbol{u} &= 0,  \nonumber\\ 
    p_{init} &= p_{hydrostatic} = \rho g (h-h_{ref}),  \nonumber\\ 
    T_{\infty} &= 293.15K,  \nonumber\\
    T_0 &= T_\infty +\delta T = (293.15 + 30) K = 323.15 K. \nonumber
\end{align}
}




\subsection{Benchmark}
In order to validate the implementation of the model, we benchmark our numerical results against a test problem that can be analytically solved. \cite{vitelli2022} describe the case of two-dimensional Poiseuille flow with odd viscosity through a pipe under a constant pressure gradient. We simulate the system using our model and fig. \ref{fig:benchmark}a shows the geometry of the problem with horizontal pressure gradient, $G = 1 Pa/mm$ and and pipe width, $h = 1 mm$. For a normal fluid ($\eta_o = 0$), the transverse pressure difference across the pipe (labeled as $\Delta P_\perp$ in fig. \ref{fig:benchmark}a) is zero, however, in the presence of odd viscosity ($\eta_o \neq 0$), $\Delta P_\perp$ attains a non-zero value given by \cite{vitelli2022}: 

\begin{equation} \label{eq:benchmark}
    \Delta P_{\perp} = Gh \frac{\eta_o}{\eta_e}
\end{equation} 

Fig. \ref{fig:benchmark}b plots $\Delta P_{\perp}$ as a function of $\eta_o/\eta_e$ for different values of $G$ and $h$. The numerical results from finite element method (FEM) are marked as cross symbols in fig. \ref{fig:benchmark}b  and they match consistently with the analytical solution (eq. \ref{eq:benchmark}) represented as solid lines. This validates the finite element implementation of the odd viscous fluid model.









%\bibliographystyle{unsrt}
%\bibliography{/Users/eks400/Documents/Bibliography/disorder,/Users/eks400/Documents/Bibliography/materialscience,/Users/eks400/Documents/Bibliography/perturbations,/Users/eks400/Documents/Bibliography/fluids,/Users/eks400/Documents/Bibliography/physiology,/Users/eks400/Documents/Bibliography/Compressible, biblio}

\def\cprime{$'$}
\begin{thebibliography}{10}
\makeatletter

\let\auto@bib@innerbib\@empty


\bibitem{Landau1981}
E.~M. Lifshitz and L.~P. Pitaevskii.
\newblock {\em {Course of theoretical physics. {V}ol. 10: Physical Kinetics}}.
\newblock Pergamon Press, 1981.

\bibitem{Avron1998}
J.E. Avron.
\newblock Odd viscosity.
\newblock {\em {Journal of Statistical Physics}}, 92(3-4):543--557, 1998.

\bibitem{Lapa2014}
M.F. Lapa and T.L. Hughes.
\newblock {Swimming at low Reynolds number in fluids with odd, or Hall,
  viscosity}.
\newblock {\em {Physical Review E}}, 89(4):043019, 2014.

\bibitem{Kirkinis2023a}
E.~Kirkinis and M.~Olvera de~la Cruz.
\newblock {Activity-induced separation of passive chiral particles in liquids}.
\newblock {\em Physical Review Fluids}, 8(2):023302, 2023.

\bibitem{Oron1997}
A.~Oron, S.H. Davis, and S.G. Bankoff.
\newblock Long-scale evolution of thin liquid films.
\newblock {\em Reviews of Modern Physics}, 69(3):931--980, 1997.

\bibitem{Davis2002}
S.H. Davis.
\newblock Interfacial fluid dynamics.
\newblock In G.~K. {Batchelor}, H.~K. {Moffatt}, and M.~G. {Worster}, editors,
  {\em Perspectives in Fluid Dynamics: A Collective Introduction to Current
  Research}, pages 1--51. Cambridge University Press, Cambridge, 2002.

\bibitem{Dussan1979}
E.B. Dussan~V.
\newblock On the spreading of liquids on solid surfaces: static and dynamic
  contact lines.
\newblock {\em {Annual Review of Fluid Mechanics}}, 11(1):371--400, 1979.

\bibitem{Ehrhard1991}
P.~Ehrhard and S.H. Davis.
\newblock Non-isothermal spreading of liquid drops on horizontal plates.
\newblock {\em {Journal of Fluid Mechanics}}, 229:365--388, 1991.

\bibitem{Smith1995}
M.K. Smith.
\newblock Thermocapillary migration of a two-dimensional liquid droplet on a
  solid surface.
\newblock {\em {Journal of Fluid Mechanics}}, 294:209--230, 1995.

\bibitem{Aggarwal2023}
A.~Aggarwal, E.~Kirkinis, and M.~Olvera de~la Cruz.
\newblock Activity-induced migration of viscous droplets on a solid substrate.
\newblock {\em Journal of Fluid Mechanics}, 955:A10, 2023.

\bibitem{Soni2019}
V.~Soni, E.S. Bililign, S.~Magkiriadou, S.~Sacanna, D.~Bartolo, M.J. Shelley,
  and W.T.M. Irvine.
\newblock The odd free surface flows of a colloidal chiral fluid.
\newblock {\em Nature Physics}, 15(11):1188--1194, 2019.

\bibitem{Blake2019}
T.D. Blake and G.N. Batts.
\newblock The temperature-dependence of the dynamic contact angle.
\newblock {\em {Journal of Colloid and Interface Science}}, 553:108--116, 2019.

\bibitem{Haynes2016}
W.M. Haynes.
\newblock {\em CRC Handbook of Chemistry and Physics}.
\newblock CRC press, Boca Raton, 2016.

\bibitem{deGennes1985}
P.G. de~Gennes.
\newblock Wetting: statics and dynamics.
\newblock {\em {Reviews of Modern Physics}}, 57(3):827, 1985.

\bibitem{Marsh1993}
J.A. Marsh, S.~Garoff, and E.B. Dussan~V.
\newblock Dynamic contact angles and hydrodynamics near a moving contact line.
\newblock {\em {Physical Review Letters}}, 70(18):2778, 1993.

\bibitem{Ehrhard1993}
P.~Ehrhard.
\newblock Experiments on isothermal and non-isothermal spreading.
\newblock {\em {Journal of Fluid Mechanics}}, 257:463--483, 1993.

\bibitem{Tanner1979}
L.H. Tanner.
\newblock The spreading of silicone oil drops on horizontal surfaces.
\newblock {\em Journal of Physics D: Applied Physics}, 12(9):1473, 1979.

\bibitem{Chen1988}
J.D. Chen.
\newblock Experiments on a spreading drop and its contact angle on a solid.
\newblock {\em Journal of colloid and interface science}, 122(1):60--72, 1988.

\bibitem{Darhuber2005}
A.A. Darhuber and S.M. Troian.
\newblock Principles of microfluidic actuation by modulation of surface
  stresses.
\newblock {\em {Annual Review of Fluid Mechanics}}, 37:425--455, 2005.

\bibitem{comsol2022comsol}
{COMSOL Multiphysics\textsuperscript{\tiny\textregistered} v. 6.1 }.
\newblock {www.comsol.com.}

\bibitem{donea2004arbitrary}
Jean Donea, Antonio Huerta, J-Ph Ponthot, and Antonio Rodr{\'i}guez-Ferran.
\newblock Arbitrary l agrangian--e ulerian methods.
\newblock {\em Encyclopedia of computational mechanics}, 2004.

\bibitem{sethian2003level}
James~A Sethian and Peter Smereka.
\newblock Level set methods for fluid interfaces.
\newblock {\em Annual review of fluid mechanics}, 35(1):341--372, 2003.

\bibitem{hirt1981volume}
Cyril~W Hirt and Billy~D Nichols.
\newblock Volume of fluid (vof) method for the dynamics of free boundaries.
\newblock {\em Journal of computational physics}, 39(1):201--225, 1981.

\bibitem{ganesan2015}
Sashikumaar Ganesan.
\newblock Simulations of impinging droplets with surfactant-dependent dynamic
  contact angle.
\newblock {\em Journal of Computational Physics}, 301:178--200, 2015.

\bibitem{wind2014finite}
{\O}istein Wind-Willassen and Mads~Peter S{\o}rensen.
\newblock A finite-element method model for droplets moving down a hydrophobic
  surface.
\newblock {\em The European Physical Journal E}, 37:1--9, 2014.

\bibitem{vitelli2022}
Michel Fruchart, Colin Scheibner, and Vincenzo Vitelli.
\newblock Odd viscosity and odd elasticity.
\newblock {\em arXiv preprint arXiv:2207.00071}, 2022.

\end{thebibliography}



\end{document}






