% ****** Start of file apssamp.tex ******
%
%   This file is part of the APS files in the REVTeX 4.2 distribution.
%   Version 4.2a of REVTeX, December 2014
%
%   Copyright (c) 2014 The American Physical Society.
%
%   See the REVTeX 4 README file for restrictions and more information.
%
% TeX'ing this file requires that you have AMS-LaTeX 2.0 installed
% as well as the rest of the prerequisites for REVTeX 4.2
%
% See the REVTeX 4 README file
% It also requires running BibTeX. The commands are as follows:
%
%  1)  latex apssamp.tex
%  2)  bibtex apssamp
%  3)  latex apssamp.tex
%  4)  latex apssamp.tex
%
\documentclass[%
 reprint,
%superscriptaddress,
%groupedaddress,
%unsortedaddress,
%runinaddress,
%frontmatterverbose, 
%preprint,
%preprintnumbers,
%nofootinbib,
%nobibnotes,
%bibnotes,
 amsmath,amssymb,
 aps,
%pra,
%prb,
%rmp,
%prstab,
%prstper,
%floatfix,
10.5pt]{revtex4-2}
\usepackage{mathrsfs}
%\usepackage{bm}% bold math
%\usepackage{caption}
%\usepackage{subcaption}
%\usepackage{bbm}
\usepackage{siunitx}
\usepackage[dvips]{epsfig}
%\usepackage{bbm}
\usepackage{amsfonts}
\usepackage{color}
%\usepackage{pict2e}
\def\bm#1{\mbox{\boldmath{$#1$}}}
\def\rr#1{(\ref{#1})}
\def\k#1{\kappa{#1}}
\def\cprime{$'$}
\newcommand{\be}{\begin{equation}}
\newcommand{\ee}{\end{equation}}
\usepackage{graphicx}% Include figure files
\usepackage{dcolumn}% Align table columns on decimal point
\usepackage{bm}% bold math

\DeclareMathOperator{\sech}{sech}

%\usepackage{hyperref}% add hypertext capabilities
%\usepackage[mathlines]{lineno}% Enable numbering of text and display math
%\linenumbers\relax % Commence numbering lines

%\usepackage[showframe,%Uncomment any one of the following lines to test 
%%scale=0.7, marginratio={1:1, 2:3}, ignoreall,% default settings
%%text={7in,10in},centering,
%%margin=1.5in,
%%total={6.5in,8.75in}, top=1.2in, left=0.9in, includefoot,
%%height=10in,a5paper,hmargin={3cm,0.8in},
%]{geometry}

\begin{document}

\preprint{APS/123-QED}

\title{Thermocapillary migrating odd viscous droplets}% Force line breaks with \\
%\thanks{A footnote to the article title}%
\author{A.Aggarwal}
\affiliation{Department of Materials Science \& Engineering, Robert R. McCormick School of Engineering and Applied Science, Northwestern University, Evanston IL 60208 USA and \\Center for Computation and Theory of Soft Materials, Northwestern University, Evanston IL 60208 USA
}
\author{E.Kirkinis}
%\email{kirkinis@northwestern.edu}
\affiliation{Department of Materials Science \& Engineering, Robert R. McCormick School of Engineering and Applied Science, Northwestern University, Evanston IL 60208 USA and \\Center for Computation and Theory of Soft Materials, Northwestern University, Evanston IL 60208 USA
}
%\author{A. V. Andreev}
% %\homepage{http://www.Second.institution.edu/~Charlie.Author}
%\affiliation{Department of Physics, University of Washington, Seattle WA 98195 USA}
%\collaboration{MUSO Collaboration}%\noaffiliation
%
%\affiliation{
% Third institution, the second for Charlie Author
%}%
%\author{Delta Author}
%\affiliation{%
% Authors' institution and/or address\\
% This line break forced with \textbackslash\textbackslash
%}%
%
%\collaboration{CLEO Collaboration}%\noaffiliation
\author{M.Olvera de la Cruz}
\email{m-olvera@northwestern.edu}
\affiliation{Department of Materials Science \& Engineering, Robert R. McCormick School of Engineering and Applied Science, Northwestern University, Evanston IL 60208 USA and \\Center for Computation and Theory of Soft Materials, Northwestern University, Evanston IL 60208 USA
}

\date{\today}% It is always \today, today,
             %  but any date may be explicitly specified

\begin{abstract}
A droplet of a classical liquid surrounded by a cold gas placed on a hot substrate is accompanied by unremitting internal circulations, while the droplet remains immobile. Two identical cells with opposite sense of circulation form in the interior due to the thermocapillary effect induced by the gas and substrate temperature difference. Under the same conditions, a droplet composed of an odd viscous liquid exerts a compressive stress on the cell rotating in one sense and tensile on the cell rotating in the opposite sense resulting in a tilted droplet configuration. A sufficiently strong thermal gradient leads the contact angles to overcome hysteresis effects and induces droplet migration.
%Heating a substrate below a 
%droplet of a classical liquid surrounded by a cold gas phase, is accompanied by unremitting 
%internal circulations, while the droplet remains immobile. Two identical
%cells with opposite sense of circulation are formed in the interior of the droplet as a consequence of the thermocapillary effect induced
%by the temperature difference $\Delta T$ between the cold gas and the hot substrate.   
%However, under the same conditions, a droplet composed of an \emph{odd viscous} liquid will
%become mobile and will migrate along the hot solid substrate. 
%The effect arises because an odd viscous liquid exerts an extra stress on a moving free surface,
%whose parity depends on flow direction. Thus, the stress will be compressive on the cell rotating in one sense and tensile on the cell 
%rotating in the opposite sense. As a consequence, 
%the droplet tilts away from a symmetric configuration, breaking the reflection symmetry with respect to the center axis, and leading to a left-right asymmetry of the
%contact angles. A sufficiently strong thermal gradient leads the contact angles to overcome hysteresis effects and induces droplet migration.
%This Letter paves the way to utilize odd viscosity in manipulating structures that incorporate
%recirculating cell patterns.
\end{abstract}

%\keywords{Suggested keywords}%Use showkeys class option if keyword
                              %display desired
\maketitle
%----------------------------------------------------------------------------------------------------------------------
%Wetting Behavior of Silicone Oils on Solid Substrates Immersed in Aqueous Electrolyte Solutions
%
%    T. Svitova, O. Theodoly, S. Christiano, R. M. Hill, and C. J. Radke
%Interfacial-Wetting Behavior of Silicone Oils. Table 2 reports that the studied silicone oils all exhibit low surface tensions of approximately 20 mN/m, which is lower than the critical-wetting surface tension of most low-energy substrates, excluding fluorocarbons. 23 Correspondingly, these oils spread completely on all the studied substrates in air. 
%-----------------------------------------------------------------------------------------------------------------------
%  Dynamic contact angles of G.E. Si oil SF-65 up to 66.5 degrees were determined by Hoffmann 1975
% Table III.  His table II shows that the Si oil has 0 static contact angle. See also our PRL2013 with Davis
%
%--------------------------------------------------------------------------------------------------------------------------
%  Liquid with nonzero static contact angle and very high dynamic angles di-n-butyl phthalate (DBP) on poly(ethylene terephthalate) (PET) tapeThe temperature-dependence of the dynamic contact angle T.D. Blake a,⇑,1, G.N. Batts b,1  2019
%---------------------------------------------------------------------------------------------------------------------------
% SLIP LENGTH DEPENDENCE Lubrication models with small to large slip lengths
%A. MU¨ NCH1, B. WAGNER1 and T.P. WITELSKI2
%---------------------------------------------------------------------------------------------------------------------------
%Curve fitting for temperature dependence of of a low volatility ester, di-n-butyl phthalate The temperature-dependence of the dynamic contact angle
%T.D. Blake a,⇑,1, G.N. Batts b,1. I take values from the CRC handbook of chemistry and physics; 
% MAPLE:  Andreev\Hall viscosity\di-n-butyl_curve_fitting.mw
%-----------------------------------------------------------------------------------------------------------------------------

\begin{figure*}
\vspace{-15pt}
\begin{center}
\includegraphics[height=2in,width=7in]{fig1}
\end{center}
\vspace{-5pt}
\caption{\textbf{Left:} Two recirculating cells formed by thermocapillarity in a viscous droplet sitting on a 
hot substrate and surrounded by a cold ambient gas phase, are symmetric with respect 
to the center axis of the droplet.% \cite{Ehrhard1991}. 
%The Navier-Stokes equations, boundary conditions and contact angles are invariant
%with respect to the transformation $x\rightarrow -x, u\rightarrow - u$, where $u$ is the horizontal 
%velocity. 
\textbf{Right:} 
Odd viscosity breaks these symmetries and exerts a compressive stress on one cell and a tensile stress on the other
leading to a deformation of the equilibrium shape of the droplet and the onset of migration with velocity $U$. 
By $\mathbf{n}$ and $\mathbf{t}$ we denote the unit normal and tangent 
vectors at the liquid-gas interface. {This same two-dimensional geometry (slice of a 3D droplet) has 
been adopted by many experiments %\cite{Tanner1978,*Tanner1979} 
and theoretical works on droplet motion.
%\cite{Greenspan1978,Ehrhard1991} 
See Fig. \ref{odd_viscosity_droplet3} for corresponding
results obtained with a \emph{three-dimensional} thermocapillary droplet}. 
\label{fig1} }
\vspace{-5pt}
\end{figure*}

%Thermal gradients on substrates are known to lead to motion of liquid droplets \cite{Darhuber2005}. 
%With temperature gradients lying horizontal to the substrate, a liquid droplet migrates towards the high surface tension region \cite{Brochard1989, *Brzoska1993}. 



Fluidic devices suffer from a lack of direct sample accessibility and increased friction. On the other hand
free-surface flows, enjoying a large surface-to-volume ratio, can circumvent these 
challenges and lead to effective liquid 
driving through interfacial processes, for example via thermal gradients \cite{Darhuber2005, *Ehrhard1991, *Ehrhard1993, Brochard1989, *Brzoska1993}.

A \emph{nonisothermal}
droplet sitting on solid heated substrated surrounded by a cold ambient gas phase
is accompanied by internal liquid circulations in the form of two identical cells rotating in an opposite
sense (cf. left of Fig.\ref{fig1}).
% and Fig. \ref{odd_viscosity_droplet}(a)). 
This is because the thermocapillary effect transports interfacial liquid from
the hot contact lines, where surface tension is lowest, to the cool droplet peak where surface tension is highest. 
Because the liquid
is viscous, it drags the bulk liquid as well. 

Avron \cite{Avron1998} showed that a classical liquid is endowed with an extra viscosity coefficient, 
the odd or Hall viscosity, in the presence of time-reversal symmetry breaking and that the accompanying stress is non-dissipative. The existence of this odd viscous stress was recently established experimentally \cite{Soni2019}. 

In this Letter we show that, odd viscosity induces migration of the aforementioned internally-circulating \emph{nonisothermal droplets}
on a heated solid substrate surrounded by a colder ambient gas phase. 
Because the two cells formed by thermocapillarity have an opposite
sense of circulation, 
odd viscosity applies a compressive stress on one cell and a tensile stress on the other (cf.~right of Fig.~\ref{fig1})
% and Fig. \ref{odd_viscosity_droplet}(a)) 
leading to a deformation of the equilibrium shape of the droplet and the onset of migration. 
We
%predominantly be concerned with the effect odd viscosity has on thermocapillary droplet migration. We will 
tacitly assume that odd viscosity has already been established in the liquid without referring
explicitly to its underlying mechanism (cf. \cite{Souslov2019}) and proceed to establish
its effects on thermocapillary droplet migration. 









\begin{figure}[b]
\vspace{-5pt}
\begin{center}
\includegraphics[height=2.2in,width=3in]{fig3oddthermal}%{fig3_v2}
\end{center}
\vspace{-15pt}
\caption{Droplet migration velocity versus the ratio of odd to even (shear) viscosity.
Finite-element numerical simulations of the full Navier-Stokes equations show that the velocity increases linearly with respect to odd viscosity in 
the proximity of experimentally determined odd viscosity values \cite{Soni2019}, where $\eta_o/\eta_e\sim 1/3$. Here we vary $\eta_o$ scaled by an $\eta_e$ that is determined from Table \ref{tab:table1} at \SI{38}{\celsius}.
%One thus obtains
%expression \rr{Ucl1},  $U_{cl} \sim \frac{\eta_o}{\eta_e}\frac{M}{2}$ for the contact line velocity
%during migration, where $M$ is the Marangoni number.  
\label{vel_v_eta} }
\vspace{-5pt}
\end{figure}


\begin{figure*}
\vspace{-5pt}
\begin{center}
\includegraphics[height=4.5in,width=7in]{new_fig2_v1}%{fig2_v2}
\end{center}
\vspace{-5pt}
\caption{(a) Numerically determined thermocapillary droplet configurations in the absence and the presence
of odd viscosity. \textbf{Left:} Internal circulations of an opposite sense in a
thermocapillary droplet remaining immobile in the absence of odd viscosity \cite{Ehrhard1991,*Ehrhard1993}.  
\textbf{Right:} In the presence of odd viscosity the symmetry \rr{symmetry} breaks, leading to a left-right asymmetry in the cells and contact angles (cf. panel (c)), 
a commensurate droplet profile deformation and the onset of migration (cf. Fig. \ref{vel_v_eta}).
{ Compare panel (a) with the corresponding configurations of a \emph{three-dimensional} thermocapillary droplet displayed in Fig. \ref{odd_viscosity_droplet3}}. 
(b) Explanation of the odd viscosity-induced droplet migration effect in terms of contact angle asymmetry.
(b) \textbf{Top right}: Experiment-inspired hysteretic diagram of a single contact line moving with velocity $U$ vs.  dynamic contact angle $\theta(t)$ (cf. Eq.\rr{Ucl}). Contact lines move only when the dynamic contact angle $\theta(t)$ 
lies outside the interval $[ {\Theta}_R, {\Theta}_A]$, 
determined by the \emph{static} advancing and receding contact angles ${\Theta}_A$ and ${\Theta}_R$, respectively.
(b) \textbf{Main figure:}
%Droplet configuration at the onset of migration:
%the viscous liquid droplet on a hot solid substrate surrounded by a colder ambient gas phase. 
%The thermocapillary effect
%creates in the droplet two identical cells with opposite sense of circulation \cite{Ehrhard1991}. 
Odd viscosity induces a compressive stress on the left cell and tensile on the right. 
The droplet tilts to the right and forces the right dynamic contact angle $\theta_a(t)$ to exceed its static
advancing counterpart $\Theta_A$ (right inset) and the left dynamic contact angle $\theta_r$ to lag behind the static receding contact angle $\Theta_R$ (left inset). 
This leads the droplet to migrate to the right with velocity $U$. 
$a(t)$ is the radius and $\bar{c}(t)$ is the middle point of the droplet. 
%according to the experiments of Dussan V
%\cite{Dussan1979} (cf. the upper right hysteretic diagram). 
(c) right and left dynamic contact angles during droplet migration as a function of the viscosity ratio. 
Here we vary $\eta_o$ scaled by an $\eta_e$ that is determined from Table \ref{tab:table1} at \SI{38}{\celsius}.
\label{odd_viscosity_droplet} }
\vspace{-15pt}
\end{figure*}

{It is important to note here that the effect we develop describes how a stationary droplet with an 
effective temperature gradient \emph{normal} to the substrate will migrate because of odd viscosity. 
This effect differs from ones that employ temperature gradients lying \emph{horizontally} 
to the substrate and can give rise to liquid droplet migration even in the absence of odd viscosity \cite{Brochard1989, *Brzoska1993}.}

The numerically-determined dependence of migration velocity $U$ on odd viscosity, is displayed in Fig. \ref{vel_v_eta}
%close to the experimentally-determined
%value $\eta_o/\eta_e\sim 0.3$ \cite{Soni2019}, 
where $\eta_e$ is the even (shear) viscosity of the liquid. 
For instance, when $\eta_o/\eta_e\sim 1/3$ we obtain
\be \label{nest}
U \sim 5\; \mu \textrm{m}/\textrm{sec}. 
\ee

{To derive this result we considered \emph{temperature-dependent} material parameters} employed in 
representative advancing contact line experiments of a low volatility
ester, di-n-butyl phthalate \cite{Blake2019}, cf. Table \ref{tab:table1}, and a hydrophobic
substrate giving rise to a large static contact angle. We derived analogous results for other materials, 
such as silicon oils employed in spreading experiments \cite{Ehrhard1993}. 

A theoretical scaling law for the migration velocity can be established by following standard 
lubrication approximation arguments  (cf. \cite{Oron1997, *Davis2002,*Ehrhard1991,*Smith1995,*Aggarwal2023} and the SI) 
\be \label{Ucl0}
U \sim K \frac{\eta_o}{\eta_e}\frac{ \alpha_{th}a\Delta T}{2 k_{th}\gamma}\frac{d \gamma}{dT}
\ee
and was also numerically confirmed. 
Here, $\alpha_{th}$ is the 
heat transfer coefficient, $k_{th}$ is the liquid thermal conductivity satisfying $k_{th}\mathbf{n}\cdot\nabla T + \alpha_{th}(T-T_\infty)=0$, at the liquid-gas interface, $T_\infty$ is the ambient gas phase temperature, $a$ is the droplet radius, $\Delta T$ the temperature difference between the hot wall
and the cold air, 
%$\theta_0$ a reference small contact angle 
$\gamma$ is surface tension and $\mathbf{n}$ is the normal, outward pointing, unit vector on the 
liquid-gas interface (cf. Fig. \ref{fig1}). 
$K$ is the mobility coefficient (units cm/sec, cf. Eq. \rr{Ucl}) which has to be determined by experiment. 
{Eq. \rr{Ucl0} was derived in the small $\Delta T$ limit where material parameters, other than
surface tension, do vary with temperature. To maintain accessibility, we tacitly assume in this article that the odd viscosity coefficient $\eta_o$ is a positive real number. Results with an odd viscosity coefficient of the opposite sign can then be obtained by making appropriate changes to our results.  }


%---------------------------------------------
% Estimates in Hall viscosity/di-n-butyl.mw
%---------------------------------------------


Considering material values from Tables \ref{tab:table1} \& \ref{tab:table2} with 
$\Delta T=\SI{30}{\celsius}$, $T=\SI{38}{\celsius}$, $\alpha_{th} = 100 \textrm{W/(m}^2\cdot K)$ and 
$\eta_o/\eta_e = 1/3$,
we obtain from the scaling law \rr{Ucl0} that $U/K \sim 0.011$. A rough estimate of the migration velocity can be obtained by employing an order of magnitude assessment of $K\sim 10^{-3}$ m/sec from representative spreading experiments \cite{Ehrhard1993}. 
This gives
\be
U \sim 11\; \mu \textrm{m}/\textrm{sec}, 
\ee
which is in good order-of-magnitude agreement with the numerically-determined value \rr{nest}. 
%Being classical, material properties of odd viscous liquids are expected to resemble their non-odd viscous counterparts. 
{This result should only be followed as a guide
since currently there is only one experimental verfication for the presence of odd viscosity in a classical liquid
\cite{Soni2019}. }



%
%Symmetry-breaking, the appearance of extra stresses on the droplet's liquid-gas interface and 
%the commencement of contact line mobility, are three fundamental and interrelated concepts
%by which the aforementioned odd viscosity-induced effect can be described. 









Droplet migration is a consequence of symmetry-breaking.  In the absence of odd viscosity ($\eta_o \equiv 0$) the Navier-Stokes 
equations 
and boundary conditions are invariant with respect to the reflection symmetry
\be \label{symmetry}
 x \rightarrow -x, \quad u \rightarrow - u, 
\ee
where $u$ is the horizontal component of liquid velocity. This invariance is associated with the two symmetric internally-circulating cells of a thermocapillary droplet \cite{Ehrhard1991}, also depicted in the left droplet configuration of Figures \ref{fig1} and \ref{odd_viscosity_droplet}(a) (the latter is the numerically-determined 
droplet configuration). This invariance is also inherited by the contact angles: in the absence of odd viscosity
both left and right contact angles are equal to each other during spreading, retraction and equilibrium
as shown in the left droplet configuration of both Figures \ref{fig1} and \ref{odd_viscosity_droplet}(a).
On the other hand, in the presence of odd viscosity, this symmetry breaks in both the Navier-Stokes equations, 
and boundary conditions. This broken symmetry leads to a left-right asymmetry in the cells (cf. right configurations of Figures \ref{fig1} and \ref{odd_viscosity_droplet}(a), where the latter is numerically determined), unequal left and right (dynamic) contact angles, 
droplet tilting and migration. Migration is a general effect that arises every time 
this reflection symmetry breaks \footnote{A discussion of symmetry-breaking in the context of 
thermocapillary droplets is delegated to the Supplementary Information}.
%------------------------------------------------------------------------------------------------------------------------------
% You need to run bibtex for the footnote to compile *********************************************
%------------------------------------------------------------------------------------------------------------------------------

A physical explanation of the odd viscosity-induced droplet migration effect developed in this Letter, is based on analyzing the stresses imparted on the liquid-gas interface %$z = h(x,t)$ 
by odd viscosity. %The shear and normal stresses at the liquid-gas interface $z = h(x,t)$ are  
The traction at the liquid-gas interface % $z= h(x,t)$ 
is \citep{Oron1997, *Davis2002}
\be  \label{freebc1}
\bm{\sigma}\mathbf{n} = \frac{\partial \gamma}{\partial s} \mathbf{t} + 2\kappa \gamma\mathbf{n}, 
\ee
%\be  \label{freebc1}
%\mathbf{t} \bm{\sigma}\mathbf{n} = \frac{\partial \gamma}{\partial s}, \quad \textrm{and}
%\quad \mathbf{n} \bm{\sigma}\mathbf{n} = 2\kappa \gamma, 
%\ee
%respectively,
where $\gamma$ is the (temperature-dependent) surface tension, $s$ is arc length along the interface, $\kappa$
its mean curvature and $\mathbf{t}$ and $\mathbf{n}$ are the unit tangent and (outward-pointing) normal vectors at the interface, see Fig. \ref{fig1}. The total stress incorporating
the effects of both even (shear) viscosity $\eta_e$ and odd viscosity $\eta_o$ is \cite{Avron1998,Khain2022} 
\be \label{seso}
\bm{\sigma} =-p\bm{I}+
 2\left( \begin{array}{cc}
 \eta_e & -\eta_o \\
 \eta_o & \eta_e
 \end{array} \right)
\mathcal{D}
\ee
where $\mathcal{D}_{ij} = \frac{1}{2}\left(\frac{\partial u_i}{\partial x_j} + \frac{\partial u_j}{\partial x_i}\right)$, $i=1,2$, $j=1,2$ is the rate-of-strain tensor. 






%\begin{table*}[t]%The best place to locate the table environment is directly after its first reference in text
%\caption{\label{tab:table1}%
%Experimental data taken from \cite{Blake2019} and \cite{Haynes2016} for di-n-butyl phthalate (DBP), valid
%between $\SI{15}{\celsius}$ and $\SI{55}{\celsius}$ and curve-fitted.}
%\begin{ruledtabular}
%\begin{tabular}{lcl}
%\textrm{Quantity}&
%\textrm{Value}&
%%\multicolumn{1}{c}{\textrm{Decimal}}&
%\textrm{Definition}\\
%\colrule
%$\rho$ ( $\textrm{kg}\:\textrm{m}^{-3} $)    & $\frac{1}{80000} T^{4}-\frac{11}{6000} T^{3}+\frac{151}{1600} T^{2}-\frac{661}{240} T +\frac{137553}{128}$        & density \\
%$\eta_e$ ( mPa$\cdot$ sec)    & $4.166666667\times 10^{-6} T^{4}- 0.0008T^{3}+ 0.06495833333 T^{2}- 2.73T + 55.2234375$        & shear (or even) viscosity \\
%$a$ (m)  & $1\times10^{-3}$& characteristic droplet radius\\
%$\gamma$ (mN $\textrm{m}^{-1}$) &$4.16666\times 10^{-7} T^{4}- 0.000066666 T^{3}+ 0.003895833 T^{2}- 0.198333 T + 37.3023$& surface tension \\
%$k_{th}$ (W $\textrm{m}^{-1}K^{-1}$) &$1.066666\times 10^{-8} T^{3}- 1.6\times 10^{-6} T^{2}- 0.00008666T + 0.139$& thermal conductivity  \\
%$T_0$  \SI{}{\celsius} &50 & solid substrate temperature\\
%$T_\infty$  \SI{}{\celsius} &20 & gas phase temperature\\
%$\Theta_A, \Theta_R$ degrees& 86.45&  static contact angles
%%$\Theta_R$ degrees & 86.45 &  static receding contact angle
%\end{tabular}
%\end{ruledtabular}
%\end{table*}
%



\begin{table*}[t]%The best place to locate the table environment is directly after its first reference in text
\caption{\label{tab:table1}%
Experimental data taken from \cite{Blake2019} and \cite{Haynes2016} for di-n-butyl phthalate (DBP). Curve
fitting of these material values is provided in the Supplementary Information. {Note that the temperature range for thermal conductivity differs from the range associated with other material parameters as they emanate
from two different references.}}
\begin{ruledtabular}
\begin{tabular}{llllll}
\textrm{Temp.} $\SI{}{\celsius}$&
$\eta_e$ ( mPa sec) &
$\gamma$ (mN $\textrm{m}^{-1}$) &
$\rho$ (kg $\textrm{m}^{-3}$) &
\textrm{Temp.} $\SI{}{\celsius}$&
$k_{th}$ (W $\textrm{m}^{-1}K^{-1}$) 
 \\
\colrule
15 &26.4 &35.0 &1049 &0& 0.139 \\
25 &16.7 &33.9 &1041&25&0.136\\
35 &11.2 &32.9 &1034&50&0.132\\
45 &8.1 &31.9 &1026&75&0.128\\
55 &6.6 &30.9 &1018&100& 0.125
\end{tabular}
\end{ruledtabular}
\end{table*}



The mechanical
behavior of interfaces between liquids is customarily described by the rate-of-strain tensor. For instance, $\mathbf{n} \mathcal{D} {\mathbf{n}}$
is the rate of extension, per unit length, of a material line element, which, in the current configuration, 
lies in the 
direction ${\mathbf{n}}$ \cite{Chadwick1976, *Truesdell1992}. It is also the deviatoric (zero-trace) part of
the normal stress at an interface. 
%Likewise, $2\hat{\mathbf{t}} \mathcal{D} \hat{\mathbf{n}}$
%is the rate of decrease of the angle between two material elements.  
With respect to this description we provide below a qualitative interpretation
for the effect odd viscosity has on the droplet liquid-gas interface of Fig. \ref{fig1}. 
Employing \rr{freebc1} and \rr{seso}, the normal component of the rate-of-strain tensor $\mathcal{D}$ at 
the liquid-gas interface reads
\be \label{nDn}
{\mathbf{n}} \mathcal{D} {\mathbf{n}} = \frac{  -\eta_o\frac{\partial \gamma}{\partial s}+ \eta_e (p + 2\kappa \gamma )  }{2(\eta_e^2  + \eta_o^2)}. 
\ee 
Thus, odd viscosity gives rise to an extra normal stress proportional to $\eta_o$. It is 
compressive on the left liquid-gas interface (cf. Fig. \rr{fig1}) since $\frac{\partial \gamma}{\partial s} >0$ there, and tensile on the right, since $\frac{\partial \gamma}{\partial s} <0$ ($s$ is arc length along
the interface). This is the driving mechanism behind the droplet migration and 
is depicted in Fig. \ref{fig1}. 
The direction of the streamlines in Fig. \ref{odd_viscosity_droplet} is compatible with this interpretation.
This leads to a tank-tread interface motion, 
also compatible with the experiments of Dussan V and Davis \cite{Dussan1974a}. Note also that it is 
the term $\eta_o\frac{\partial \gamma}{\partial s}$ that breaks the reflection symmetry \rr{symmetry} in the 
boundary condition \rr{nDn}. 









Symmetry between the contact angles also breaks in the presence of odd viscosity. 
While in the absence of odd viscosity the left
and right contact angles are equal, in the presence of odd viscosity they become mismatched and the 
droplet tilts with respect to its center axis (see the exaggerated configuration to the right of Fig. \ref{fig1}
and panel (b) in Fig. \ref{odd_viscosity_droplet}). 
This takes place when the aforementioned compressive and tensile stresses at the liquid-gas interface 
emanating from the term $\eta_o\frac{\partial \gamma}{\partial s}$ in Eq. \rr{nDn}, enable the contact lines to 
overcome hysteresis effects. In this case, the right dynamic contact angle $\theta_a$ exceeds the 
static advancing contact angle $\Theta_A$, (cf. right inset of Fig.~\ref{odd_viscosity_droplet} (b)) \emph{and} the left dynamic contact angle $\theta_r$ falls short of its receding static counterpart $\Theta_R$, (cf. left inset of Fig.~\ref{odd_viscosity_droplet} (b)). This leads to migration with a velocity $U$.
%We thus proceed, by investigating the dependence of contact line velocity on odd viscosity. 





Surface roughness, chemical contamination and solutes may lead to contact line pinning whenever
the dynamic contact angle $\theta(t)$ lies within a finite interval $\Theta_R<\theta(t)<\Theta_A$ \cite{deGennes1985}. 
This is the phenomenon of contact-angle hysteresis (cf. upper cartoon in panel (b) of Fig. \ref{odd_viscosity_droplet}) experimentally documented by Dussan V \cite[Fig.~2]{Dussan1979}
relating contact angle with contact line velocity $U_{cl}$. In modeling this behavior, one finds that the contact
line moves with a velocity \cite{Oron1997}
\be \label{Ucl}
U_{cl} =  \pm K
\left\{  \begin{array}{cc}
(\theta - \Theta_A)^m, & \theta>\Theta_A,\\  
-(\Theta_R - \theta)^m, & \theta < \Theta_R
\end{array}
\right.
\ee
where $\theta$ is the dynamic contact angle (here denoted either as $\theta_a$ or $\theta_r$, cf. Fig.~\ref{odd_viscosity_droplet})
and the upper/lower sign is identified with the motion of the right/left contact line. $K$ is the mobility
coefficient appearing in the scaling law \rr{Ucl0}. 
Theoretical analyses \cite{Ehrhard1991,deGennes1985} and multiple spreading experiments from different groups %of Tanner \cite{Tanner1979}, Chen \cite{Chen1988}, Ehrhard \cite{Ehrhard1993}
\cite{Marsh1993,*Ehrhard1993,*Tanner1979, *Chen1988} established the consistency of the 
constitutive law \rr{Ucl} with experiment and determined values of the exponent $m$ according to the 
physical systems under consideration. 





Odd viscosity-assisted migration of thermocapillary droplets is also expected to follow the law \rr{Ucl}. Neglecting hysteresis effects for simplicity, we determine numerically (cf. Fig. \ref{vel_v_eta}) that $m\sim1$, 
for odd viscosity values close to experiment ($\eta_o/\eta_e\sim 1/3$, \cite{Soni2019}). 
The value $m=1$ was employed in deriving the theoretical scaling law Eq. \rr{Ucl0}. 
Its derivation
follows general principles of the lubrication approximation 
and is delegated to the Supplementary Information.  












\begin{table}[b]%The best place to locate the table environment is directly after its first reference in text
\vspace{-10pt}
\caption{\label{tab:table2}%
Parameters employed in this article}
\begin{ruledtabular}
\begin{tabular}{lcl}
\textrm{Quantity}&
\textrm{Value}&
%\multicolumn{1}{c}{\textrm{Decimal}}&
\textrm{Definition}\\
\colrule
$a$ (m)  & $1\times10^{-3}$& characteristic droplet radius\\
$T_0$  \SI{}{\celsius} &50 & solid substrate temperature\\
$T_\infty$  \SI{}{\celsius} &20 & gas phase temperature\\
$\Theta_A, \Theta_R$ degrees& 86.45&  static contact angles
\end{tabular}
\end{ruledtabular}
\end{table}




\begin{figure*}
\vspace{-5pt}
\begin{center}
\includegraphics[height=2.5in,width=7in]{3dmigrationsection}
\end{center}
\vspace{-5pt}
\caption{{Numerically-determined typical cross-sections of a \emph{three-dimensional} thermocapillary droplet  in the absence and in the presence
of odd viscosity. 
Compare with panel (a) of Fig. \ref{odd_viscosity_droplet} depicting
the corresponding states of a two-dimensional thermocapillary droplet. 
See Fig. S-I of the supplementary information for 
three-dimensional views taken from different viewing angles. 
\textbf{Left:} Counterrotating cells in a
thermocapillary droplet which remains immobile in the absence of odd viscosity. Observables in this
three-dimensional configuration such as contact angles, cell geometry and velocity field are symmetric with respect to the center axis. 
\textbf{Right:} In the presence of odd viscosity the axial symmetry is lost, leading to the onset of migration.
\label{odd_viscosity_droplet3}} }
\vspace{-5pt}
\end{figure*}



{In this Letter we employed a two-dimensional
geometry which permitted some analytical investigation of the odd viscosity-induced droplet migration effect. }Experimentally, two-dimensional droplet migration has 
been realized by placing oil films on vertical surfaces, for instance a wire,
and let it drop off in the form of a thin cylindrical stream \cite{Tanner1978, *Tanner1979}. 
%Hence, from these experiments one obtains a spreading radius $a \sim t^{0.148}$. 
This same two-dimensional
geometry was also employed theoretically to study spreading of droplets \cite{Ehrhard1991} and migration
under a horizontal temperature gradient \cite{Smith1995}
%It was 
%determined that for $m=1$ ($m$ is the exponent appearing
%in the constitutive law \rr{Ucl}) one obtains 
%that a plane droplet has a spreading radius $a \sim t^{\frac{1}{3}}$. For $m=3$, 
%$a \sim t^{\frac{1}{7}}$, see \cite{Ehrhard1991}. 
in the gravity or surface tension
dominated regime. 

{It is instructive to qualitatively compare our numerical results on the migration of 
a \emph{two-dimensional} thermocapillary droplet depicted in Fig. \ref{odd_viscosity_droplet},
with its
\emph{three-dimensional} counterpart. In Fig. \ref{odd_viscosity_droplet3} we display
typical cross sections of such three-dimensional droplets in the presence and in the absence of 
odd viscosity, to be compared with panel (a) of Fig. \ref{odd_viscosity_droplet} depicting
the corresponding states of a two-dimensional thermocapillary droplet. We witness the 
same symmetric recirculating counter-rotating cells in the absence of odd viscosity, where
the droplet necessarily remains immobile (left configuration of Fig. \ref{odd_viscosity_droplet3}
and compare with the left configuration of panel (a) in Fig. \ref{odd_viscosity_droplet}). 
In the presence of odd viscosity however, the symmetry of the three-dimensional droplet 
with respect to the center axis breaks leading to migration (right configuration of Fig. \ref{odd_viscosity_droplet3}
and compare with the right configuration of panel (a) in Fig. \ref{odd_viscosity_droplet}). 
In Fig. S-I of the supplementary information of this Letter we display different three dimensional
views (top view, 90 degree-elevated view) of the three-dimensional thermocapillary droplet 
whose cross-section is displayed in Fig. \ref{odd_viscosity_droplet3}. }









%A three-dimensional realization of the migration of thermocapillary 
%odd viscous droplet is expected to bear similarities to the present work. In particular, an
%anisotropy axis, directed for instance, parallel to the substrate, will be associated with odd 
%viscosity and, depending on parameter values, may effectively introduce a situation where the flow along the axis will become
%independent to the flow in the lateral plane (perpendicular to the axis). Thus, one could say 
%that the flow will become effectively two-dimensional, in a manner
%that resembles the Taylor-Proudman decomposition \cite{Batchelor1967} of a three-dimensional flow. 
%These effects are 
%beyond the scope of the present Letter and will be addressed elsewhere. 
%For a collection of these axisymmetric data see Table 1 of \cite{Ehrhard1991}. 
%Evaporation/condensation effects can also be taken into account by following for instance
%the development of Ref. \cite{Oron1997}. 








%The effect described in this Letter has some similarities with existing literature. In particular, Avron \cite{Avron1998} showed that a rotating disk in an odd viscous fluid experiences an extra odd viscosity-induced pressure whose orientation depends on the disk's sense of rotation. In the present Letter 
%we showed that extra stresses are present on a liquid-gas interface, whose direction depend on the 
%the interfacial velocity's sense of orientation. 



%
%
%In addition, there is a secondary odd viscosity effect arising in the shear component of the rate-of-strain tensor on the liquid-gas interface $
%z = h(x,t)$ 
%\be \label{tDn}
%\hat{\mathbf{t}} \mathcal{D} \hat{\mathbf{n}} = \frac{ \eta_e \frac{\partial \gamma}{\partial s}  + \eta_o (p + 2\kappa \gamma  )}{2(\eta_e^2  + \eta_o^2)}. 
%\ee
%Assuming,  that pressure is due to capillarity and surface tension gradients only, \rr{nDn} gives
%$p \sim \eta_o\frac{\partial \gamma}{\partial s} - 2\kappa \gamma$. Substituting into \rr{tDn}
%leads to $\hat{\mathbf{t}} \mathcal{D} \hat{\mathbf{n}}\propto  \eta_e \left[1 + 
%\left(\frac{\eta_o}{\eta_e}\right)^2\right] \frac{\partial \gamma}{\partial s} $ which is an odd viscosity-induced enhancement of
%thermocapillarity.


%Since curvature is always negative, there is an odd viscosity-induced shear stress along the interface
%directed from the right towards the left contact line. 
%This is the same stress identified by 
%\citep[Fig. S22]{Soni2019}, being out-of-phase with respect to interface undulations, acting at its inflection
%points and leading to unidirectional wave propagation. 
%Here the effect of the extra stress is limited to 
%enhancing (to the droplet right) or reducing (to the left) the thermocapillary flow. 






%\begin{acknowledgments}
We thank the Department of Energy, Office of
Basic Energy Sciences for support under contract DE-FG02-08ER46539 and 
three anonymous referees for comments that improved the manuscript. 
%\end{acknowledgments}


%
%
%$M$ above is the product of Capillary, Biot and Marangoni numbers
%\be
%Ca = \frac{U_0 \eta_e}{\theta_0^3 \gamma_0}, \quad 
%Bi = \frac{\alpha_{th} a_0 \theta_0}{k_{th}}, \quad
%Ma = \frac{\theta_0k\Delta T}{\eta_e U_0}.
%\ee
%How other thermal coefficients \cite{Landau1981} affect the flow










\bibliographystyle{apsrev4-2}
%\bibliography{bib_files/disorder,bib_files/perturbations,bib_files/fluids,bib_files/physiology,bib_files/Compressible}


%apsrev4-2.bst 2019-01-14 (MD) hand-edited version of apsrev4-1.bst
%Control: key (0)
%Control: author (72) initials jnrlst
%Control: editor formatted (1) identically to author
%Control: production of article title (-1) disabled
%Control: page (0) single
%Control: year (1) truncated
%Control: production of eprint (0) enabled
\def\cprime{$'$}
\begin{thebibliography}{25}%
\makeatletter
\providecommand \@ifxundefined [1]{%
 \@ifx{#1\undefined}
}%
\providecommand \@ifnum [1]{%
 \ifnum #1\expandafter \@firstoftwo
 \else \expandafter \@secondoftwo
 \fi
}%
\providecommand \@ifx [1]{%
 \ifx #1\expandafter \@firstoftwo
 \else \expandafter \@secondoftwo
 \fi
}%
\providecommand \natexlab [1]{#1}%
\providecommand \enquote  [1]{``#1''}%
\providecommand \bibnamefont  [1]{#1}%
\providecommand \bibfnamefont [1]{#1}%
\providecommand \citenamefont [1]{#1}%
\providecommand \href@noop [0]{\@secondoftwo}%
\providecommand \href [0]{\begingroup \@sanitize@url \@href}%
\providecommand \@href[1]{\@@startlink{#1}\@@href}%
\providecommand \@@href[1]{\endgroup#1\@@endlink}%
\providecommand \@sanitize@url [0]{\catcode `\\12\catcode `\$12\catcode
  `\&12\catcode `\#12\catcode `\^12\catcode `\_12\catcode `\%12\relax}%
\providecommand \@@startlink[1]{}%
\providecommand \@@endlink[0]{}%
\providecommand \url  [0]{\begingroup\@sanitize@url \@url }%
\providecommand \@url [1]{\endgroup\@href {#1}{\urlprefix }}%
\providecommand \urlprefix  [0]{URL }%
\providecommand \Eprint [0]{\href }%
\providecommand \doibase [0]{https://doi.org/}%
\providecommand \selectlanguage [0]{\@gobble}%
\providecommand \bibinfo  [0]{\@secondoftwo}%
\providecommand \bibfield  [0]{\@secondoftwo}%
\providecommand \translation [1]{[#1]}%
\providecommand \BibitemOpen [0]{}%
\providecommand \bibitemStop [0]{}%
\providecommand \bibitemNoStop [0]{.\EOS\space}%
\providecommand \EOS [0]{\spacefactor3000\relax}%
\providecommand \BibitemShut  [1]{\csname bibitem#1\endcsname}%
\let\auto@bib@innerbib\@empty
%</preamble>
\bibitem [{\citenamefont {Darhuber}\ and\ \citenamefont
  {Troian}(2005)}]{Darhuber2005}%
  \BibitemOpen
  \bibfield  {author} {\bibinfo {author} {\bibfnamefont {A.}~\bibnamefont
  {Darhuber}}\ and\ \bibinfo {author} {\bibfnamefont {S.}~\bibnamefont
  {Troian}},\ }\href@noop {} {\bibfield  {journal} {\bibinfo  {journal}
  {{Annual Review of Fluid Mechanics}}\ }\textbf {\bibinfo {volume} {37}},\
  \bibinfo {pages} {425} (\bibinfo {year} {2005})}\BibitemShut {NoStop}%
\bibitem [{\citenamefont {Ehrhard}\ and\ \citenamefont
  {Davis}(1991)}]{Ehrhard1991}%
  \BibitemOpen
  \bibfield  {author} {\bibinfo {author} {\bibfnamefont {P.}~\bibnamefont
  {Ehrhard}}\ and\ \bibinfo {author} {\bibfnamefont {S.}~\bibnamefont
  {Davis}},\ }\href@noop {} {\bibfield  {journal} {\bibinfo  {journal}
  {{Journal of Fluid Mechanics}}\ }\textbf {\bibinfo {volume} {229}},\ \bibinfo
  {pages} {365} (\bibinfo {year} {1991})}\BibitemShut {NoStop}%
\bibitem [{\citenamefont {Ehrhard}(1993)}]{Ehrhard1993}%
  \BibitemOpen
  \bibfield  {author} {\bibinfo {author} {\bibfnamefont {P.}~\bibnamefont
  {Ehrhard}},\ }\href@noop {} {\bibfield  {journal} {\bibinfo  {journal}
  {{Journal of Fluid Mechanics}}\ }\textbf {\bibinfo {volume} {257}},\ \bibinfo
  {pages} {463} (\bibinfo {year} {1993})}\BibitemShut {NoStop}%
\bibitem [{\citenamefont {Brochard}(1989)}]{Brochard1989}%
  \BibitemOpen
  \bibfield  {author} {\bibinfo {author} {\bibfnamefont {F.}~\bibnamefont
  {Brochard}},\ }\href@noop {} {\bibfield  {journal} {\bibinfo  {journal}
  {Langmuir}\ }\textbf {\bibinfo {volume} {5}},\ \bibinfo {pages} {432}
  (\bibinfo {year} {1989})}\BibitemShut {NoStop}%
\bibitem [{\citenamefont {Brzoska}\ \emph {et~al.}(1993)\citenamefont
  {Brzoska}, \citenamefont {Brochard-Wyart},\ and\ \citenamefont
  {Rondelez}}]{Brzoska1993}%
  \BibitemOpen
  \bibfield  {author} {\bibinfo {author} {\bibfnamefont {J.}~\bibnamefont
  {Brzoska}}, \bibinfo {author} {\bibfnamefont {F.}~\bibnamefont
  {Brochard-Wyart}},\ and\ \bibinfo {author} {\bibfnamefont {F.}~\bibnamefont
  {Rondelez}},\ }\href@noop {} {\bibfield  {journal} {\bibinfo  {journal}
  {Langmuir}\ }\textbf {\bibinfo {volume} {9}},\ \bibinfo {pages} {2220}
  (\bibinfo {year} {1993})}\BibitemShut {NoStop}%
\bibitem [{\citenamefont {Avron}(1998)}]{Avron1998}%
  \BibitemOpen
  \bibfield  {author} {\bibinfo {author} {\bibfnamefont {J.}~\bibnamefont
  {Avron}},\ }\href@noop {} {\bibfield  {journal} {\bibinfo  {journal}
  {{Journal of Statistical Physics}}\ }\textbf {\bibinfo {volume} {92}},\
  \bibinfo {pages} {543} (\bibinfo {year} {1998})}\BibitemShut {NoStop}%
\bibitem [{\citenamefont {Soni}\ \emph {et~al.}(2019)\citenamefont {Soni},
  \citenamefont {Bililign}, \citenamefont {Magkiriadou}, \citenamefont
  {Sacanna}, \citenamefont {Bartolo}, \citenamefont {Shelley},\ and\
  \citenamefont {Irvine}}]{Soni2019}%
  \BibitemOpen
  \bibfield  {author} {\bibinfo {author} {\bibfnamefont {V.}~\bibnamefont
  {Soni}}, \bibinfo {author} {\bibfnamefont {E.}~\bibnamefont {Bililign}},
  \bibinfo {author} {\bibfnamefont {S.}~\bibnamefont {Magkiriadou}}, \bibinfo
  {author} {\bibfnamefont {S.}~\bibnamefont {Sacanna}}, \bibinfo {author}
  {\bibfnamefont {D.}~\bibnamefont {Bartolo}}, \bibinfo {author} {\bibfnamefont
  {M.}~\bibnamefont {Shelley}},\ and\ \bibinfo {author} {\bibfnamefont
  {W.}~\bibnamefont {Irvine}},\ }\href@noop {} {\bibfield  {journal} {\bibinfo
  {journal} {Nature Physics}\ }\textbf {\bibinfo {volume} {15}},\ \bibinfo
  {pages} {1188} (\bibinfo {year} {2019})}\BibitemShut {NoStop}%
\bibitem [{\citenamefont {Souslov}\ \emph {et~al.}(2019)\citenamefont
  {Souslov}, \citenamefont {Dasbiswas}, \citenamefont {Fruchart}, \citenamefont
  {Vaikuntanathan},\ and\ \citenamefont {Vitelli}}]{Souslov2019}%
  \BibitemOpen
  \bibfield  {author} {\bibinfo {author} {\bibfnamefont {A.}~\bibnamefont
  {Souslov}}, \bibinfo {author} {\bibfnamefont {K.}~\bibnamefont {Dasbiswas}},
  \bibinfo {author} {\bibfnamefont {M.}~\bibnamefont {Fruchart}}, \bibinfo
  {author} {\bibfnamefont {S.}~\bibnamefont {Vaikuntanathan}},\ and\ \bibinfo
  {author} {\bibfnamefont {V.}~\bibnamefont {Vitelli}},\ }\href@noop {}
  {\bibfield  {journal} {\bibinfo  {journal} {{Physical Review Letters}}\
  }\textbf {\bibinfo {volume} {122}},\ \bibinfo {pages} {128001} (\bibinfo
  {year} {2019})}\BibitemShut {NoStop}%
\bibitem [{\citenamefont {Blake}\ and\ \citenamefont
  {Batts}(2019)}]{Blake2019}%
  \BibitemOpen
  \bibfield  {author} {\bibinfo {author} {\bibfnamefont {T.}~\bibnamefont
  {Blake}}\ and\ \bibinfo {author} {\bibfnamefont {G.}~\bibnamefont {Batts}},\
  }\href@noop {} {\bibfield  {journal} {\bibinfo  {journal} {{Journal of
  Colloid and Interface Science}}\ }\textbf {\bibinfo {volume} {553}},\
  \bibinfo {pages} {108} (\bibinfo {year} {2019})}\BibitemShut {NoStop}%
\bibitem [{\citenamefont {Oron}\ \emph {et~al.}(1997)\citenamefont {Oron},
  \citenamefont {Davis},\ and\ \citenamefont {Bankoff}}]{Oron1997}%
  \BibitemOpen
  \bibfield  {author} {\bibinfo {author} {\bibfnamefont {A.}~\bibnamefont
  {Oron}}, \bibinfo {author} {\bibfnamefont {S.}~\bibnamefont {Davis}},\ and\
  \bibinfo {author} {\bibfnamefont {S.}~\bibnamefont {Bankoff}},\ }\href@noop
  {} {\bibfield  {journal} {\bibinfo  {journal} {Reviews of Modern Physics}\
  }\textbf {\bibinfo {volume} {69}},\ \bibinfo {pages} {931} (\bibinfo {year}
  {1997})}\BibitemShut {NoStop}%
\bibitem [{\citenamefont {Davis}(2002)}]{Davis2002}%
  \BibitemOpen
  \bibfield  {author} {\bibinfo {author} {\bibfnamefont {S.}~\bibnamefont
  {Davis}},\ }in\ \href@noop {} {\emph {\bibinfo {booktitle} {Perspectives in
  Fluid Dynamics: A Collective Introduction to Current Research}}},\ \bibinfo
  {editor} {edited by\ \bibinfo {editor} {\bibfnamefont {G.~K.}\ \bibnamefont
  {{Batchelor}}}, \bibinfo {editor} {\bibfnamefont {H.~K.}\ \bibnamefont
  {{Moffatt}}},\ and\ \bibinfo {editor} {\bibfnamefont {M.~G.}\ \bibnamefont
  {{Worster}}}}\ (\bibinfo  {publisher} {Cambridge University Press},\ \bibinfo
  {address} {Cambridge},\ \bibinfo {year} {2002})\ pp.\ \bibinfo {pages}
  {1--51}\BibitemShut {NoStop}%
\bibitem [{\citenamefont {Smith}(1995)}]{Smith1995}%
  \BibitemOpen
  \bibfield  {author} {\bibinfo {author} {\bibfnamefont {M.}~\bibnamefont
  {Smith}},\ }\href@noop {} {\bibfield  {journal} {\bibinfo  {journal}
  {{Journal of Fluid Mechanics}}\ }\textbf {\bibinfo {volume} {294}},\ \bibinfo
  {pages} {209} (\bibinfo {year} {1995})}\BibitemShut {NoStop}%
\bibitem [{\citenamefont {Aggarwal}\ \emph {et~al.}(2023)\citenamefont
  {Aggarwal}, \citenamefont {Kirkinis},\ and\ \citenamefont {Olvera de~la
  Cruz}}]{Aggarwal2023}%
  \BibitemOpen
  \bibfield  {author} {\bibinfo {author} {\bibfnamefont {A.}~\bibnamefont
  {Aggarwal}}, \bibinfo {author} {\bibfnamefont {E.}~\bibnamefont {Kirkinis}},\
  and\ \bibinfo {author} {\bibfnamefont {M.}~\bibnamefont {Olvera de~la
  Cruz}},\ }\href {https://doi.org/10.1017/jfm.2022.1051} {\bibfield  {journal}
  {\bibinfo  {journal} {Journal of Fluid Mechanics}\ }\textbf {\bibinfo
  {volume} {955}},\ \bibinfo {pages} {A10} (\bibinfo {year}
  {2023})}\BibitemShut {NoStop}%
\bibitem [{Note1()}]{Note1}%
  \BibitemOpen
  \bibinfo {note} {A discussion of symmetry-breaking in the context of
  thermocapillary droplets is delegated to the Supplementary
  Information}\BibitemShut {NoStop}%
\bibitem [{\citenamefont {Khain}\ \emph {et~al.}(2022)\citenamefont {Khain},
  \citenamefont {Scheibner}, \citenamefont {Fruchart},\ and\ \citenamefont
  {Vitelli}}]{Khain2022}%
  \BibitemOpen
  \bibfield  {author} {\bibinfo {author} {\bibfnamefont {T.}~\bibnamefont
  {Khain}}, \bibinfo {author} {\bibfnamefont {C.}~\bibnamefont {Scheibner}},
  \bibinfo {author} {\bibfnamefont {M.}~\bibnamefont {Fruchart}},\ and\
  \bibinfo {author} {\bibfnamefont {V.}~\bibnamefont {Vitelli}},\ }\href@noop
  {} {\bibfield  {journal} {\bibinfo  {journal} {{Journal of Fluid Mechanics}}\
  }\textbf {\bibinfo {volume} {934}},\ \bibinfo {pages} {A23} (\bibinfo {year}
  {2022})}\BibitemShut {NoStop}%
\bibitem [{\citenamefont {Haynes}(2016)}]{Haynes2016}%
  \BibitemOpen
  \bibfield  {author} {\bibinfo {author} {\bibfnamefont {W.}~\bibnamefont
  {Haynes}},\ }\href@noop {} {\emph {\bibinfo {title} {CRC Handbook of
  Chemistry and Physics}}}\ (\bibinfo  {publisher} {CRC press, Boca Raton},\
  \bibinfo {year} {2016})\BibitemShut {NoStop}%
\bibitem [{\citenamefont {Chadwick}(1976)}]{Chadwick1976}%
  \BibitemOpen
  \bibfield  {author} {\bibinfo {author} {\bibfnamefont {P.}~\bibnamefont
  {Chadwick}},\ }\href@noop {} {\emph {\bibinfo {title} {Continuum
  mechanics}}}\ (\bibinfo  {publisher} {Halsted Press [John Wiley \& Sons], New
  York (Now in Dover)},\ \bibinfo {year} {1976})\ p.\ \bibinfo {pages} {174},\
  \bibinfo {note} {concise theory and problems}\BibitemShut {NoStop}%
\bibitem [{\citenamefont {Truesdell}\ and\ \citenamefont
  {Noll}(1992)}]{Truesdell1992}%
  \BibitemOpen
  \bibfield  {author} {\bibinfo {author} {\bibfnamefont {C.}~\bibnamefont
  {Truesdell}}\ and\ \bibinfo {author} {\bibfnamefont {W.}~\bibnamefont
  {Noll}},\ }\href@noop {} {\emph {\bibinfo {title} {The nonlinear field
  theories of mechanics}}},\ \bibinfo {edition} {2nd}\ ed.\ (\bibinfo
  {publisher} {Springer-Verlag},\ \bibinfo {address} {Berlin},\ \bibinfo {year}
  {1992})\ pp.\ \bibinfo {pages} {x+591}\BibitemShut {NoStop}%
\bibitem [{\citenamefont {Dussan~V}\ and\ \citenamefont
  {Davis}(1974)}]{Dussan1974a}%
  \BibitemOpen
  \bibfield  {author} {\bibinfo {author} {\bibfnamefont {E.}~\bibnamefont
  {Dussan~V}}\ and\ \bibinfo {author} {\bibfnamefont {S.}~\bibnamefont
  {Davis}},\ }\href@noop {} {\bibfield  {journal} {\bibinfo  {journal} {Journal
  of Fluid Mechanics}\ }\textbf {\bibinfo {volume} {65}},\ \bibinfo {pages}
  {71} (\bibinfo {year} {1974})}\BibitemShut {NoStop}%
\bibitem [{\citenamefont {de~Gennes}(1985)}]{deGennes1985}%
  \BibitemOpen
  \bibfield  {author} {\bibinfo {author} {\bibfnamefont {P.}~\bibnamefont
  {de~Gennes}},\ }\href@noop {} {\bibfield  {journal} {\bibinfo  {journal}
  {{Reviews of Modern Physics}}\ }\textbf {\bibinfo {volume} {57}},\ \bibinfo
  {pages} {827} (\bibinfo {year} {1985})}\BibitemShut {NoStop}%
\bibitem [{\citenamefont {Dussan~V}(1979)}]{Dussan1979}%
  \BibitemOpen
  \bibfield  {author} {\bibinfo {author} {\bibfnamefont {E.}~\bibnamefont
  {Dussan~V}},\ }\href@noop {} {\bibfield  {journal} {\bibinfo  {journal}
  {{Annual Review of Fluid Mechanics}}\ }\textbf {\bibinfo {volume} {11}},\
  \bibinfo {pages} {371} (\bibinfo {year} {1979})}\BibitemShut {NoStop}%
\bibitem [{\citenamefont {Marsh}\ \emph {et~al.}(1993)\citenamefont {Marsh},
  \citenamefont {Garoff},\ and\ \citenamefont {Dussan~V}}]{Marsh1993}%
  \BibitemOpen
  \bibfield  {author} {\bibinfo {author} {\bibfnamefont {J.}~\bibnamefont
  {Marsh}}, \bibinfo {author} {\bibfnamefont {S.}~\bibnamefont {Garoff}},\ and\
  \bibinfo {author} {\bibfnamefont {E.}~\bibnamefont {Dussan~V}},\ }\href@noop
  {} {\bibfield  {journal} {\bibinfo  {journal} {{Physical Review Letters}}\
  }\textbf {\bibinfo {volume} {70}},\ \bibinfo {pages} {2778} (\bibinfo {year}
  {1993})}\BibitemShut {NoStop}%
\bibitem [{\citenamefont {Tanner}(1979)}]{Tanner1979}%
  \BibitemOpen
  \bibfield  {author} {\bibinfo {author} {\bibfnamefont {L.}~\bibnamefont
  {Tanner}},\ }\href@noop {} {\bibfield  {journal} {\bibinfo  {journal}
  {Journal of Physics D: Applied Physics}\ }\textbf {\bibinfo {volume} {12}},\
  \bibinfo {pages} {1473} (\bibinfo {year} {1979})}\BibitemShut {NoStop}%
\bibitem [{\citenamefont {Chen}(1988)}]{Chen1988}%
  \BibitemOpen
  \bibfield  {author} {\bibinfo {author} {\bibfnamefont {J.}~\bibnamefont
  {Chen}},\ }\href@noop {} {\bibfield  {journal} {\bibinfo  {journal} {Journal
  of colloid and interface science}\ }\textbf {\bibinfo {volume} {122}},\
  \bibinfo {pages} {60} (\bibinfo {year} {1988})}\BibitemShut {NoStop}%
\bibitem [{\citenamefont {Tanner}(1978)}]{Tanner1978}%
  \BibitemOpen
  \bibfield  {author} {\bibinfo {author} {\bibfnamefont {L.}~\bibnamefont
  {Tanner}},\ }\href@noop {} {\bibfield  {journal} {\bibinfo  {journal}
  {{Optics \& Laser Technology}}\ }\textbf {\bibinfo {volume} {10}},\ \bibinfo
  {pages} {125} (\bibinfo {year} {1978})}\BibitemShut {NoStop}%
\end{thebibliography}%



\end{document}






