
\documentclass[conference]{ieeetran}






  \let\proof\relax
  \let\endproof\relax


  \usepackage{amsmath,amssymb,amsfonts,amsthm,mathrsfs}
    

  \newcommand{\T}{^\mathsf{T}}
  \newcommand{\tr}{\mathrm{tr}}
  \newcommand{\R}{\mathbb{R}}
  \newcommand{\E}{\mathrm{E}}
  \newcommand{\var}{\mathrm{var}}
  \newcommand{\cov}{\mathrm{cov}}
  \providecommand{\norm}[1]{\|#1\|}
  \providecommand{\innerp}[1]{\langle#1\rangle}
  \newcommand{\diag}[1]{\mathrm{diag}(#1)}
  \renewcommand{\vec}[1]{\mathrm{vec}(#1)}
  \providecommand{\op}[1]{\mathcal{#1}}
  \providecommand{\Hspace}[1]{\mathscr{#1}}
  \providecommand{\Fourier}{\mathscr{F}}
  \newcommand{\N}{\mathrm{N}}
  \newcommand{\GP}{\mathcal{GP}}
  \newcommand{\dd}{\,\mathrm{d}}
  \newcommand{\diff}{\,\mathrm{d}}
  \newcommand{\kron}{\raisebox{1pt}{\ensuremath{\:\otimes\:}}}
  \newcommand{\hadamard}{\circ}


  \usepackage{bm}
  \newcommand{\mathbold}[1]{\bm{#1}}
  \newcommand{\mbf}[1]{\mathbf{#1}}
  \newcommand{\vect}[1]{\mbf{#1}}
  \newcommand{\vectb}[1]{\bm{#1}}
  \newcommand{\mat}[1]{\mbf{#1}}
  \newcommand{\matb}[1]{\bm{#1}}


  \newcommand{\eg}{\textit{e.g.}}
  \newcommand{\ie}{\textit{i.e.}}
  \newcommand{\cf}{\textit{cf.}}
  \newcommand{\etc}{\textit{etc.}}
  \newcommand{\etal}{\textit{et~al.}}


  \usepackage{xcolor}

 

  \usepackage{graphicx}
  \graphicspath{ {./}{./fig/} }






  \usepackage{tikz,pgfplots}
  \usetikzlibrary{plotmarks,shapes,arrows}
  \pgfplotsset{compat=newest} 


  \newlength\figureheight
  \newlength\figurewidth


  \usepackage{hyperref}
  \hypersetup{
    bookmarks=true,
    pdfstartview={FitH},
    colorlinks=true,
    linkcolor=black,
    citecolor=black,
    filecolor=black,
    urlcolor=black
  }


  \urlstyle{same}



  \usepackage{color}
  \newcommand{\arno}[1]{\textcolor{blue}{\textbf{[Arno: #1]}}}
  \newcommand{\santiago}[1]{\textcolor{orange}{\textbf{[Santiago: #1]}}}
  \newcommand{\juho}[1]{\textcolor{purple}{\textbf{[Juho: #1]}}}
  \newcommand{\esa}[1]{\textcolor{green}{\textbf{[Esa: #1]}}}



  \usetikzlibrary{external}





  \makeatletter
  \let\NAT@parse\undefined
  \makeatother


  \makeatletter
  \let\NAT@parse\undefined
  \makeatother
  \usepackage[square,numbers,sort&compress]{natbib}



\title{Inertial Odometry on Handheld Smartphones}


\author{
  \IEEEauthorblockN{Arno Solin}
  \IEEEauthorblockA{Aalto University\\
  Espoo, Finland\\
  arno.solin@aalto.fi}
  \and 
  \IEEEauthorblockN{Santiago Cortes}
  \IEEEauthorblockA{Aalto University\\
  Espoo, Finland\\
  santiago.cortesreina@aalto.fi}
  \and
  \IEEEauthorblockN{Esa Rahtu}
  \IEEEauthorblockA{Tampere Univ.\ of Tech.\\
  Tampere, Finland\\
  esa.rahtu@tut.fi}
  \and
  \IEEEauthorblockN{Juho Kannala}
  \IEEEauthorblockA{Aalto University\\
  Espoo, Finland\\
  juho.kannala@aalto.fi}
}




\begin{document}

\maketitle
\thispagestyle{empty}
\pagestyle{empty}

\begin{abstract}
  Building a complete inertial navigation system using the limited quality data provided by current smartphones has been regarded challenging, if not impossible. This paper shows that by careful crafting and accounting for the weak information in the sensor samples, smartphones are capable of pure inertial navigation. We present a probabilistic approach for orientation and use-case free inertial odometry, which is based on double-integrating rotated accelerations. The strength of the model is in learning additive and multiplicative IMU biases online. We are able to track the phone position, velocity, and pose in real-time and in a computationally lightweight fashion by solving the inference with an extended Kalman filter. The information fusion is completed with zero-velocity updates (if the phone remains stationary), altitude correction from barometric pressure readings (if available), and pseudo-updates constraining the momentary speed. We demonstrate our approach using an iPad and iPhone in several indoor dead-reckoning applications and in a measurement tool setup.
\end{abstract}





\section{Introduction}
\label{sec:intro}
\noindent
The deployment of global navigation satellite systems (GNSSs) has solved many large-scale positioning problems. However, these systems are not suited for precise tracking or for indoor use, which is where people spend most of their time. Accurate and fast indoor localization and tracking has many potential uses, including safety and emergency assistance, security, resource efficiency, navigation and augmented reality.

The idea of an inertial navigation system (INS, see \cite{Jekeli:2001,Britting:2010}) is to use the fusion of inertial sensors (accelerometers and gyroscopes) to continuously estimate the position, orientation, and velocity of a moving object. This type of tracking, known as dead-reckoning, is typically associated with aircraft, submarines, and missile technology. Recent advances in MEMS sensors have brought motion and rotation sensors to standard consumer smartphones and devices, and introduced the potential for new INS applications.

Smartphones and tablet devices are equipped with MEMS sensors in order to enhance human-computer interaction and enable new applications. For example, thanks to the accelerometer, devices can automatically rotate the screen based on the device orientation with respect to gravity. Furthermore, gyroscopes have enabled new ways to interact with digital content, such as watching of panoramic video or controlling games by rotating the device. In fact, besides gravitation sensing and tracking \cite{Sarkka+Tolvanen+Kannala+Rahtu:2015}, information fusion from accelerometers, gyroscopes and magnetometers can be utilised for robust real-time tracking of the full device orientation \cite{Madgwick+Harrison+Vaidyanathan:2011, Renaudin+Combettes:2014}. Such approaches are sometimes referred to as \emph{attitude and heading reference systems} (AHRS). 


\begin{figure*}[!t]
  %
  \tikzexternaldisable
  %

  \tikzsetnextfilename{tikz-intro}


  \setlength{\figurewidth}{.65\textwidth}
  \setlength{\figureheight}{0.4780\figurewidth}
  %

  \pgfplotsset{
    trim axis right,
    yticklabel style={rotate=90},
  }
  %
  \footnotesize\centering%
  \hspace*{\fill}
  % !TEX root = ../arxiv.tex

Unsupervised domain adaptation (UDA) is a variant of semi-supervised learning \cite{blum1998combining}, where the available unlabelled data comes from a different distribution than the annotated dataset \cite{Ben-DavidBCP06}.
A case in point is to exploit synthetic data, where annotation is more accessible compared to the costly labelling of real-world images \cite{RichterVRK16,RosSMVL16}.
Along with some success in addressing UDA for semantic segmentation \cite{TsaiHSS0C18,VuJBCP19,0001S20,ZouYKW18}, the developed methods are growing increasingly sophisticated and often combine style transfer networks, adversarial training or network ensembles \cite{KimB20a,LiYV19,TsaiSSC19,Yang_2020_ECCV}.
This increase in model complexity impedes reproducibility, potentially slowing further progress.

In this work, we propose a UDA framework reaching state-of-the-art segmentation accuracy (measured by the Intersection-over-Union, IoU) without incurring substantial training efforts.
Toward this goal, we adopt a simple semi-supervised approach, \emph{self-training} \cite{ChenWB11,lee2013pseudo,ZouYKW18}, used in recent works only in conjunction with adversarial training or network ensembles \cite{ChoiKK19,KimB20a,Mei_2020_ECCV,Wang_2020_ECCV,0001S20,Zheng_2020_IJCV,ZhengY20}.
By contrast, we use self-training \emph{standalone}.
Compared to previous self-training methods \cite{ChenLCCCZAS20,Li_2020_ECCV,subhani2020learning,ZouYKW18,ZouYLKW19}, our approach also sidesteps the inconvenience of multiple training rounds, as they often require expert intervention between consecutive rounds.
We train our model using co-evolving pseudo labels end-to-end without such need.

\begin{figure}[t]%
    \centering
    \def\svgwidth{\linewidth}
    \input{figures/preview/bars.pdf_tex}
    \caption{\textbf{Results preview.} Unlike much recent work that combines multiple training paradigms, such as adversarial training and style transfer, our approach retains the modest single-round training complexity of self-training, yet improves the state of the art for adapting semantic segmentation by a significant margin.}
    \label{fig:preview}
\end{figure}

Our method leverages the ubiquitous \emph{data augmentation} techniques from fully supervised learning \cite{deeplabv3plus2018,ZhaoSQWJ17}: photometric jitter, flipping and multi-scale cropping.
We enforce \emph{consistency} of the semantic maps produced by the model across these image perturbations.
The following assumption formalises the key premise:

\myparagraph{Assumption 1.}
Let $f: \mathcal{I} \rightarrow \mathcal{M}$ represent a pixelwise mapping from images $\mathcal{I}$ to semantic output $\mathcal{M}$.
Denote $\rho_{\bm{\epsilon}}: \mathcal{I} \rightarrow \mathcal{I}$ a photometric image transform and, similarly, $\tau_{\bm{\epsilon}'}: \mathcal{I} \rightarrow \mathcal{I}$ a spatial similarity transformation, where $\bm{\epsilon},\bm{\epsilon}'\sim p(\cdot)$ are control variables following some pre-defined density (\eg, $p \equiv \mathcal{N}(0, 1)$).
Then, for any image $I \in \mathcal{I}$, $f$ is \emph{invariant} under $\rho_{\bm{\epsilon}}$ and \emph{equivariant} under $\tau_{\bm{\epsilon}'}$, \ie~$f(\rho_{\bm{\epsilon}}(I)) = f(I)$ and $f(\tau_{\bm{\epsilon}'}(I)) = \tau_{\bm{\epsilon}'}(f(I))$.

\smallskip
\noindent Next, we introduce a training framework using a \emph{momentum network} -- a slowly advancing copy of the original model.
The momentum network provides stable, yet recent targets for model updates, as opposed to the fixed supervision in model distillation \cite{Chen0G18,Zheng_2020_IJCV,ZhengY20}.
We also re-visit the problem of long-tail recognition in the context of generating pseudo labels for self-supervision.
In particular, we maintain an \emph{exponentially moving class prior} used to discount the confidence thresholds for those classes with few samples and increase their relative contribution to the training loss.
Our framework is simple to train, adds moderate computational overhead compared to a fully supervised setup, yet sets a new state of the art on established benchmarks (\cf \cref{fig:preview}).

  %
  \hspace*{\fill}
  %
  \tikzsetnextfilename{tikz-person}%
  \begin{tikzpicture}
    \node[anchor=south west,inner sep=0] (image) at (0,0) %
      {\includegraphics[width=.4\columnwidth,keepaspectratio]{./fig/phone-person-light}};
    \begin{scope}[x={(image.south east)},y={(image.north west)}]


      \tikzstyle{label} = [text width=1.5cm, align=center]
      \tikzstyle{line}  = [draw, very thick, blue!50]
      \tikzstyle{point} = [circle,draw,very thick,blue!50,fill=none,
                           minimum size=5mm,inner sep=0]


      \node [label] (lab_hand)   at (0.05,0.90) {Phone in hand};
      \node [label] (lab_pocket) at (0.05,0.10) {Phone in pocket};
      \node [label] (lab_bag)    at (0.95,0.45) {Phone in bag};


      \node [point] (hand)       at (0.25,0.70) {};
      \node [point] (pocket)     at (0.45,0.45) {};
      \node [point] (bag)        at (0.80,0.15) {};


      \path [line] (lab_hand)   |- (hand);
      \path [line] (lab_pocket) |- (pocket);
      \path [line] (lab_bag)    |- (bag);

    \end{scope}
  \end{tikzpicture}
  %
  \hspace*{\fill}
  \vspace{-1em}
  %
  \tikzexternalenable
  %
  \caption{Features of the INS system summarized into one figure: The path was started on the ground floor with zero-velocity updates. After walking up the stairs to the first floor a position fix was given, after which the phone was put in a \textbf{closed bag}. Then the phone was put in the \textbf{pocket}. Before descending to the ground floor, the phone was taken out of the pocket and a second position fix was given (aligning the path to the map). On the ground floor a manual loop-closure was given. The data was collected by an iPhone~6 and calibrations were performed on the fly.}
  \label{fig:intro} 
\end{figure*}


Tracking the translational motion of devices based on inertial sensors is considerably harder than orientation tracking. However, certain applications, like pedestrian tracking and indoor positioning, would greatly benefit from accurate inertial navigation on smartphones. The difficulty of inertial navigation is due to the need to double-integrate the observed accelerations, which rapidly accumulates errors from the high noise-level of MEMS accelerometers. Small errors in the attitude estimation will make this even more challenging as the gravitation may `leak' to the integrated accelerations~\cite{Sachs:2010}. 

In order to solve the aforementioned challenges, many current systems resort to additional hardware, such as foot-mounted sensors \cite{Foxlin:2005,Nilsson+Zachariah+Skog+Handel:2013} or video cameras.
While providing accurate results, these are quite impractical for wide use in consumer applications. For example, camera-based approaches do not work when the device is in a closed bag or pocket, and capturing and processing video consumes a lot of energy compromising battery longevity. Further, while foot-mounted sensors can provide accurate tracking thanks to frequent zero-velocity updates and high-quality sensors, they are inconvenient for large-scale consumer use and the current solutions do not work well when the movement happens without steps, for example in a trolley, elevator, or escalator. 



In this paper we show how an inertial navigation system can be built to work on the limited-quality data provided by a standard smartphone. We propose a general inertial navigation approach which is not based on detecting steps and therefore works in various use cases, covering both legged motion and motion with wheels, as well as motion in elevators and escalators. Moreover, the approach does not require constraining the device orientation, and thus the device can be held freely. In addition, the approach is computationally light-weight and capable for real-time processing on a smartphone. To the best of our knowledge, this is the first paper demonstrating such a system with a standard smartphone.

Figure~\ref{fig:intro} summarizes the features of the proposed INS system in a test performed with a standard iPhone~6. In this example, the path was started on the ground floor with zero-velocity updates for calibrating the sensors (no pre-calibrations done). After walking up the stairs to the first floor holding the phone in the hand, a position fix was given, after which the phone was put in a bag. Next, the phone was taken out of the bag and put in the trouser pocket. Before descending to the ground floor, the phone was taken out of the pocket and a second position fix was given, which aligned the path to the map. On the ground floor a manual loop-closure indicated that we were where we started.

The contributions of this paper are two-fold: 
\begin{itemize}
  \item We show that inertial navigation on a standard smartphone is feasible by careful crafting of the model, taking advantage of weak signals, and accounting for uncertainties in data.
  \item We present a streamlined estimation approach for the INS problem which builds upon learning the dynamical sensor bias parameters as a part of the state variables. The probabilistic inference is solved by a sequential filtering scheme, where the only approximations come from the linearizations inside the extended Kalman filter. The approach is complemented with zero-velocity updates and pseudo-measurements limiting the momentary speed. 
\end{itemize} 

This paper is structured as follows. In the next section we provide a brief literature review of previous work. In Section~\ref{sec:methods} we present the INS model. The exact model is presented in detail, and measurement updates for fusing measurements with dynamics are described. Section~\ref{sec:experiments} presents empirical studies where the inertial navigation algorithm is employed in pedestrian dead-reckoning examples, a generalized dead-reckoning example, and as a measurement tool.  Finally, the results are discussed in Section~\ref{sec:discussion}.



\section{Related Work}
\label{sec:literature}
\noindent
Inertial navigation systems have been studied for decades. The classical literature cover primarily navigation applications for aircraft and large vehicles \cite{Jekeli:2001,Bar-Shalom+Li+Kirubarajan:2001,Titterton+Weston:2004,Britting:2010}. The development of handheld consumer-grade devices has awakened an interest in pedestrian navigation applications, where the challenges are slightly different from those in the classical approaches. That is, the limited quality of smartphone MEMS sensors and abrupt motions of hand-held devices pose additional challenges which have so far prevented generic inertial navigation solutions for smartphone applications.


In order to focus on the relevant previous literature, we restrict our scope to tracking algorithms that use the sensors available in a smartphone, primarily accelerometers, gyroscopes, and magnetometers.

The extensive survey by Harle \cite{Harle:2013} covers many approaches with different constraints for the use of inertial sensors for pedestrian dead-reckoning (PDR). Typically INS systems either constrain the motion model or rely on external sensors. In fact, we are not aware of any previous system which would have all the capabilities that we demonstrate in this paper.

One prominent INS solution relying on external hardware is the OpenShoe project \cite{Nilsson+Zachariah+Skog+Handel:2013, Nilsson+Gupta+Handel:2014}. It uses foot-mounted inertial sensors with several pairs of accelerometers and gyroscopes to estimate the step-by-step PDR (in an INS-SHS framework, see below). The model is constrained by zero-velocity updates (ZUPTs) on each step once the foot touches the ground.


Step and heading systems (SHS, see \eg\ \cite{Woodman:2010,Renaudin+Combettes:2014, Kang+Han:2015, Yuan+Yu+Zhang+Wang+Liu:2015, Chen+Meng+Wang+Zhang+Tian+Yang:2015}) use the inertial sensor to estimate the heading and the step length of the user. These are introduced into a constrained model that estimates the walking path by accumulating the step vectors in order to do PDR. These systems have been proven to work well for PDR in short and medium range but they typically impose constraints for the device orientation. For example, the device orientation is often known or fixed with respect to the walking direction. Further, they are very sensitive to changing gaits and are prone to false positives (see discussion in \cite{Harle:2013}). A recent approach \cite{Xiao+Wen+Markham+Trigoni:2014} uses bipedal locomotion models to model the periodical behavior of the INS in a smartphone and, thus, estimate steps. Although they are able to relax the constraint of known and fixed device orientation to some extent, their approach is still step-based, and heading estimation is error-prone, especially if there are frequent and abrupt changes in orientation.


Besides inertial PDR systems, there exist many camera-aided inertial tracking solutions (visual-inertial odometry), which can provide accurate tracking in visually distinguishable environments (\eg\ \cite{Li+Kim+Mourikis:2013, Hesch+Kottas+Bowman+Roumeliotis:2014, Bloesch+Omari+Hutter+Siegwart:2015,Solin+Cortes+Rahtu+Kannala:2018-WACV}). However, as these approaches require constant use of a video camera, causing increased battery usage, and unobstructed visibility of surroundings, they are not directly comparable to our approach.




Finally, it should be noted that often odometry estimation techniques, either inertial or visual, are part of larger localization systems, which combine odometry with various kinds of maps or fingerprinting methods that provide reference positions. Examples of mapped signals, which have been utilized for indoor localization, include signal strengths of Wi-Fi and Bluetooth radio beacons \cite{Mirowski+Ho+Yi+MacDonald:2013}, cellular communications radio \cite{Martin+Vinyals+Friedland+Bajcsy:2010}, RFID tags \cite{Ruiz+Granja+Honorato+Rosas:2012}, and variations of the ambient magnetic field \cite{Solin+Kok+Wahlstrom+Schon+Sarkka:2015,Solin+Sarkka+Kannala+Rahtu:2016}.
  


\section{Methods}
\label{sec:methods}

\noindent






Even though, the physical interpretation of how an inertial navigation system works is straight-forward, this setup has many pitfalls. All inertial navigation systems suffer from integration drift. Small errors in the measurements of acceleration and angular velocity cause progressively larger errors in velocity---and even greater errors in position. The dominating component in the accelerometer data is gravity, which means that even slight errors in orientation make the gravity `leak' into the estimates. The sequential nature of the problem makes the errors accumulate. Once the estimates start to drift, they quickly diverge.

These problems underline the importance of accurately modelling and handling the inherent noises, sampling times, uncertainties, and numerical instabilities in the system. We use the data provided by the inertial measurement unit (IMU) in the smartphone to continuously infer the relative change in position, velocity and orientation of the device with respect to a starting point (see \cite{Jekeli:2001,Britting:2010}). The three-axis IMU measures data of the specific force (accelerometer data) and angular rate (gyroscope data).

\subsection{Non-Linear Estimation}
\label{sec:nonlin-estimation}
\noindent
An inertial navigation system is non-linear both in the dynamics and observations. Non-linear filtering methods (see \cite{Sarkka:2013} for an overview) are concerned with this kind of estimation problems. Consider a non-linear state-space equation model of form
\begin{align}
  \vect{x}_k &= \vect{f}_k(\vect{x}_{k-1}, \vectb{\varepsilon}_k), \label{eq:dynamic} \\
  \vect{y}_k &= \vect{h}_k(\vect{x}_k, \vectb{\gamma}_k), \label{eq:measurement} 
\end{align}
where $\vect{x}_k \in \R^n$ is the state at time step $t_k$, $k=1,2,\ldots$, $\vect{y}_k \in \R^m$ is a measurement, $\vectb{\varepsilon}_k \sim \N(\vectb{0}, \vect{Q}_k)$ is the Gaussian process noise, and  $\vectb{\gamma}_k \sim \N(\vectb{0},\vect{R}_k)$ is the Gaussian measurement noise. The dynamics and measurements are specified in terms of the dynamical model function $\vect{f}_k(\cdot)$ and the measurement model function $\vect{h}_k(\cdot)$, both of which can depend on the time step $k$.

We employ the extended Kalman filter (EKF, \cite{Bar-Shalom+Li+Kirubarajan:2001}) which provides a means of approximating the state distributions 
\begin{equation}
  p(\vect{x}_k \mid \vect{y}_{1:k}) \simeq \N(\vect{x}_k \mid \vect{m}_{k}, \vect{P}_{k})
\end{equation}
with Gaussians through first-order linearizations. In the experiments, we also employ the fixed-interval extended Rauch--Tung--Striebel smoother (see \cite{Sarkka:2013} for detailed presentation) for obtaining the state distributions $p(\vect{x}_k \mid \vect{y}_{1:N})$ conditioned on the entire track of observations.


\subsection{Dynamical Model}
\noindent
The state variables hold the knowledge of the system state at any given time step. The state variables are:
\begin{equation}
  \vect{x}_k = (\vect{p}_k, \vect{v}_k, \vect{q}_k, \vect{b}_k^\mathrm{a}, \vect{b}_k^{\omega}, \vect{T}_k^\mathrm{a}),
\end{equation}
where $\vect{p}_k \in \R^3$ is the position, $\vect{v}_k \in \R^3$ the velocity, and $\vect{q}_k$ the orientation unit quaternion at time step $t_k$. The remaining components are the additive accelerometer and gyroscope bias components, and $\vect{T}_k^\mathrm{a}$ denotes the diagonal multiplicative scale error of the accelerometer.

The dynamical model (Eq.~\ref{eq:dynamic}) is based on the assumption that position is velocity once integrated, and velocity is acceleration (with the influence of gravity removed) once integrated. The orientation of the acceleration is tracked with gyroscope measurements. The accelerometer and gyroscope readings are regarded as control signals, and their measurement noises are seen as the process noise of the system.

The dynamical model given by the mechanization equations (see, \eg, \cite{Titterton+Weston:2004,Nilsson+Zachariah+Skog+Handel:2013} for similar model formulations) is
\begin{equation}\label{eq:ins-model}
  \begin{pmatrix}
    \vect{p}_k \\ \vect{v}_k \\ \vect{q}_k
  \end{pmatrix}
  =
  \begin{pmatrix}
    \vect{p}_{k-1} + \vect{v}_{k-1}\Delta t_k \\
    \vect{v}_{k-1} + [\vect{q}_k (\tilde{\vect{a}}_k + \vectb{\varepsilon}^\mathrm{a}_k) \vect{q}_k^\star - \vect{g}] \Delta t_k \\
    \vectb{\Omega}[(\tilde{\vectb{\omega}}_k + \vectb{\varepsilon}^\omega_k) \Delta t_k] \vect{q}_{k-1}
  \end{pmatrix},
\end{equation}
where the time step length is given by $\Delta t_k = t_{k} - t_{k-1}$ (note that we {\em do not} assume equidistant sampling times), the accelerometer input is denoted by $\tilde{\vect{a}}_k$ and the gyroscope input by $\tilde{\vectb{\omega}}_k$. Gravity $\vect{g}$ is a constant vector. The quaternion rotation is denoted by the $\vect{q}_k [\cdot] \vect{q}_k^\star$ notation, and the quaternion rotation update is given by the function $\vectb{\Omega}: \R^3 \to \R^{4 \times 4}$ (see \cite{Titterton+Weston:2004} for details). 


The system is deterministic up to the uncertainties (measurement noises and biases) associated with the accelerometer and gyroscope data. The process noises associated with the inputs are modelled as i.i.d.\ Gaussian noise $\vectb{\varepsilon}^\mathrm{a}_k \sim \N(\vectb{0},\vectb{\Sigma}^\mathrm{a} \Delta t_k)$ and $\vectb{\varepsilon}^\omega_k \sim \N(\vectb{0},\vectb{\Sigma}^\omega \Delta t_k)$. The Jacobians of \eqref{eq:ins-model}, required for the linearizations in filtering, can be constructed in closed-form.

The accelerometer and gyroscope readings provided by the low-cost sensors in the mobile device may suffer from misalignment errors and scale errors in addition to white measurement noise. These are taken into account inside the dynamic model as follows:
\begin{equation}
\begin{split}
  \tilde{\vect{a}}_k &= \vect{T}_k^\mathrm{a} \, \vect{a}_k - \vect{b}^\mathrm{a}_k, \\
  \tilde{\vectb{\omega}}_k &= \hphantom{\vect{T}_k^\mathrm{a} \, }\vectb{\omega}_k - \vect{b}^\omega_k,
\end{split}
\end{equation}
where the accelerometer and gyroscope sensor readings at $t_k$ are $\vect{a}_k$ and $\vectb{\omega}_k$. The additive biases are denoted by $\vect{b}^\mathrm{a}_k$ and $\vect{b}^\omega_k$, respectively. The diagonal scale error matrix $\vect{T}_k^\mathrm{a}$ accounts for miscalibrations in the accelerometer scale.

The biases and diagonal scale error terms are estimated online as a part of the state estimation problem. They are considered fixed over the entire time horizon, thus the dynamic model for their part is fixed and without any process noise: 
\begin{equation}
  \vect{b}^\mathrm{a}_k = \vect{b}^\mathrm{a}_{k-1}, \quad
  \vect{b}^{\omega}_k = \vect{b}^{\omega}_{k-1}, \quad \text{and} \quad
  \vect{T}_k^\mathrm{a} = \vect{T}_{k-1}^\mathrm{a}.
\end{equation}
This means that their values are controlled by the prior state and information provided by the measurement updates.

The complete dynamical model must be differentiated both in terms of the state variables and process noise terms in order to fit the EKF estimation scheme (see  \cite{Sarkka:2013}). These derivatives can be derived in closed-form in order to preserve the stability of the system.
The initial (prior) state is given by $\vect{p}_0 \sim \N(\vect{0}, \vectb{\Sigma}_0^\mathrm{p})$,  $\vect{v}_0 \sim \N(\vect{0}, \vectb{\Sigma}_0^\mathrm{v})$, and  $\vect{q}_0$ chosen such that it defines the initial orientation (deduced from gravity direction). The additive biases are initialized to zero and the scale bias to an identity matrix.



\begin{figure*}[!t]
  %

  %

  %
  \begin{minipage}{.66\textwidth}


  \setlength{\figurewidth}{.90\textwidth}
  \setlength{\figureheight}{0.15\figurewidth}
  %

  \pgfplotsset{
    trim axis right,
    yticklabel style={rotate=90},
    legend columns=3,
    legend style={draw=none},
    ylabel absolute,
    ylabel style={yshift=-0.6cm}
  }
  %

  \footnotesize
  \tikzsetnextfilename{tikz-altitude}%
  % This file was created by matlab2tikz.
%
%The latest updates can be retrieved from
%  http://www.mathworks.com/matlabcentral/fileexchange/22022-matlab2tikz-matlab2tikz
%where you can also make suggestions and rate matlab2tikz.
%
\begin{tikzpicture}

\begin{axis}[%
width=0.951\figurewidth,
height=\figureheight,
at={(0\figurewidth,0\figureheight)},
scale only axis,
xmin=0,
xmax=120,
xtick={0,10,20,30,40,50,60,70,80,90,100,110,120},
xticklabels={{\phantom{0}},{},{},{},{},{},{},{},{},{},{},{},{\phantom{120}}},
ymin=-1,
ymax=7,
ylabel={$z$-displacement (m)},
axis background/.style={fill=white},
axis on top
]

\addplot[area legend,solid,draw=white!90!black,fill=white!90!black,forget plot]
table[row sep=crcr] {%
x	y\\
0.11	-1\\
4.6799	-1\\
4.6799	7\\
0.11	7\\
}--cycle;

\addplot[area legend,solid,draw=white!90!black,fill=white!90!black,forget plot]
table[row sep=crcr] {%
x	y\\
8.6097	-1\\
10.3196	-1\\
10.3196	7\\
8.6097	7\\
}--cycle;

\addplot[area legend,solid,draw=white!90!black,fill=white!90!black,forget plot]
table[row sep=crcr] {%
x	y\\
11.3996	-1\\
12.7295	-1\\
12.7295	7\\
11.3996	7\\
}--cycle;

\addplot[area legend,solid,draw=white!90!black,fill=white!90!black,forget plot]
table[row sep=crcr] {%
x	y\\
14.5695	-1\\
18.5394	-1\\
18.5394	7\\
14.5695	7\\
}--cycle;

\addplot[area legend,solid,draw=white!90!black,fill=white!90!black,forget plot]
table[row sep=crcr] {%
x	y\\
100.7695	-1\\
104.2096	-1\\
104.2096	7\\
100.7695	7\\
}--cycle;

\addplot[area legend,solid,draw=white!90!black,fill=white!90!black,forget plot]
table[row sep=crcr] {%
x	y\\
106.1596	-1\\
110.4496	-1\\
110.4496	7\\
106.1596	7\\
}--cycle;

\addplot[area legend,solid,draw=white!90!black,fill=white!90!black,forget plot]
table[row sep=crcr] {%
x	y\\
113.3997	-1\\
116.0098	-1\\
116.0098	7\\
113.3997	7\\
}--cycle;
\addplot [color=white!50!blue,solid,line width=1.2pt,forget plot]
  table[row sep=crcr]{%
0	-1.87975225658343e-06\\
0.1	-3.95233790056774e-06\\
0.2	9.5748990316825e-07\\
0.3	-6.90647953755608e-06\\
0.4	-7.21304864894235e-06\\
0.5	-3.45682753533453e-05\\
0.6	-6.51720871052018e-05\\
0.7	-5.88181254080637e-05\\
0.8	-3.83036671394802e-05\\
0.9	-5.09511449326168e-05\\
1	-6.55642357500169e-05\\
1.0999	-8.22475656907677e-05\\
1.1999	-0.000117099104381337\\
1.2999	-0.00015479003061075\\
1.3999	-0.000172072715830446\\
1.4999	-0.000204099466677942\\
1.5999	-0.000205151844659795\\
1.6999	-0.00021350659252649\\
1.7999	-0.000212661611210045\\
1.8999	-0.000207261158370475\\
1.9999	-0.000180823064912421\\
2.0999	-0.000193068151832662\\
2.1999	-0.000209781749895989\\
2.2999	-0.000194828726654977\\
2.3999	-0.000144321114233324\\
2.4999	-7.1851544883757e-05\\
2.5999	-9.50720120665883e-06\\
2.6999	-5.73246873714794e-06\\
2.7999	-3.43250317817387e-06\\
2.8999	7.87175162023676e-06\\
2.9999	-5.6493089472734e-07\\
3.0999	-2.02990026449737e-05\\
3.1999	-3.78395685395913e-05\\
3.2999	-7.8021965137486e-05\\
3.3999	-0.000138170375082389\\
3.4999	-0.00021578201955849\\
3.5999	-0.000328689082951419\\
3.6999	-0.000455014147425631\\
3.7999	-0.000569488851796345\\
3.8999	-0.000659591596221743\\
3.9999	-0.000704345080559703\\
4.0999	-0.000720897287555995\\
4.1999	-0.000736769584172472\\
4.2999	-0.000746287717177177\\
4.3999	-0.00074095419822411\\
4.4999	-0.00074117024347005\\
4.5999	-0.000812971982702793\\
4.6999	-0.00113683786094241\\
4.7999	-0.00175321891360218\\
4.8998	-0.00262445543757096\\
4.9998	-0.00371652329061227\\
5.0998	-0.00508834467184851\\
5.1998	-0.00666802414318113\\
5.2998	-0.00847014012987658\\
5.3998	-0.010538395100638\\
5.4997	-0.0128470306189879\\
5.5997	-0.0153627473894661\\
5.6998	-0.0180830484928482\\
5.7998	-0.0210525099254649\\
5.8998	-0.0241705526802297\\
5.9998	-0.0275326112694747\\
6.0997	-0.031099250181402\\
6.1997	-0.0347656089792937\\
6.2997	-0.0379581152918918\\
6.3997	-0.0393196803855779\\
6.4997	-0.0285301247471528\\
6.5997	0.0182694140371341\\
6.6997	0.109744558862144\\
6.7997	0.246930744682299\\
6.8997	0.412732031604365\\
6.9997	0.574536762752511\\
7.0997	0.723106040634789\\
7.1997	0.839395116182898\\
7.2997	0.927559800186816\\
7.3997	0.974373063759743\\
7.4997	0.985223755421538\\
7.5997	0.998318290443318\\
7.6997	1.01017938626672\\
7.7997	1.01978454840261\\
7.8997	1.02416896580155\\
7.9997	1.024783708764\\
8.0997	1.02237433997482\\
8.1997	1.01910918701042\\
8.2997	1.01743509782342\\
8.3997	1.01742481043725\\
8.4997	1.0163350670534\\
8.5997	1.01588066761827\\
8.6997	1.01585768640149\\
8.7997	1.01582281056801\\
8.8997	1.01580404204294\\
8.9997	1.0157771113334\\
9.0997	1.01573325683623\\
9.1997	1.01566475308417\\
9.2997	1.01564283220598\\
9.3997	1.01560166327051\\
9.4997	1.01558456866448\\
9.5997	1.01558751963652\\
9.6997	1.01557440148523\\
9.7997	1.01556173099236\\
9.8997	1.01554590168986\\
9.9996	1.01553137266157\\
10.0996	1.01552355457574\\
10.1996	1.01552111589466\\
10.2996	1.01551849518833\\
10.3996	1.01552535766073\\
10.4996	1.01552233671574\\
10.5996	1.01547036511497\\
10.6996	1.01542213643605\\
10.7996	1.01531693530728\\
10.8996	1.01519274912691\\
10.9996	1.0150937380637\\
11.0996	1.01500282447666\\
11.1996	1.01490238375018\\
11.2996	1.01483866393004\\
11.3996	1.01481992823591\\
11.4995	1.01479344660282\\
11.5995	1.01479726094326\\
11.6996	1.01478887011548\\
11.7996	1.01475069143543\\
11.8995	1.01469850021115\\
11.9996	1.01468159882894\\
12.0995	1.01465189070953\\
12.1995	1.01463538444617\\
12.2996	1.01462982609084\\
12.3995	1.01460927609317\\
12.4995	1.01460169151861\\
12.5995	1.01462247910838\\
12.6995	1.0146375799397\\
12.7995	1.01474390595975\\
12.8995	1.01509350791338\\
12.9995	1.01570312564199\\
13.0995	1.01656595797055\\
13.1995	1.01764523298226\\
13.2995	1.018948039789\\
13.3995	1.02053189933885\\
13.4995	1.02258228086969\\
13.5995	1.02401376005534\\
13.6995	1.01609344432303\\
13.7995	1.00315316957334\\
13.8995	0.995255914720928\\
13.9995	0.989061703243052\\
14.0995	0.986236037954068\\
14.1995	0.985005042216415\\
14.2995	0.986176426160534\\
14.3995	0.987639554369184\\
14.4995	0.988458042310904\\
14.5995	0.98855878963824\\
14.6995	0.988555965105578\\
14.7995	0.988536575148237\\
14.8995	0.988532965197864\\
14.9994	0.98852046917976\\
15.0994	0.988520202060814\\
15.1994	0.988513691921095\\
15.2994	0.988508922854116\\
15.3994	0.988506049305301\\
15.4994	0.98849206470653\\
15.5994	0.988502117451073\\
15.6994	0.988489746631592\\
15.7994	0.988486753178019\\
15.8994	0.988489418071113\\
15.9994	0.988488631371618\\
16.0994	0.988472130380593\\
16.1994	0.988471835696868\\
16.2994	0.988475679770146\\
16.3994	0.988475688950029\\
16.4994	0.988459484000047\\
16.5994	0.988450030323153\\
16.6994	0.988448374734263\\
16.7994	0.988443733641026\\
16.8994	0.988445335243886\\
16.9994	0.988449233869707\\
17.0994	0.98847951297243\\
17.1994	0.988505110512333\\
17.2994	0.988515161561656\\
17.3994	0.988507986054376\\
17.4994	0.988482920939916\\
17.5994	0.988459059707262\\
17.6994	0.988418925125005\\
17.7994	0.988422751668284\\
17.8994	0.988449446284792\\
17.9994	0.988480611729569\\
18.0994	0.988491666326431\\
18.1994	0.988495594777787\\
18.2994	0.988504053605904\\
18.3994	0.988521022472122\\
18.4994	0.98853699858554\\
18.5994	0.988604974751197\\
18.6994	0.98859268126929\\
18.7994	0.988067172672462\\
18.8994	0.986683015920979\\
18.9994	0.989424479085829\\
19.0994	0.996233350164033\\
19.1994	1.00510249098586\\
19.2994	1.0174351981846\\
19.3994	1.03389825077305\\
19.4994	1.05072208919096\\
19.5994	1.06148788351807\\
19.6994	1.06582087783733\\
19.7994	1.06548217466511\\
19.8994	1.0645147454993\\
19.9993	1.06279094341134\\
20.0993	1.06508669083458\\
20.1993	1.06650820775136\\
20.2993	1.06792536758075\\
20.3993	1.06284230203798\\
20.4993	1.05312642359881\\
20.5993	1.05246915027852\\
20.6993	1.06413693748897\\
20.7993	1.07220860264207\\
20.8993	1.07270651491997\\
20.9993	1.0650964251478\\
21.0993	1.0543643994484\\
21.1993	1.05323670485637\\
21.2993	1.06806610961822\\
21.3993	1.08500278370902\\
21.4993	1.09443743441376\\
21.5993	1.08765530016493\\
21.6993	1.07713309449889\\
21.7993	1.08367052203337\\
21.8993	1.10609250775922\\
21.9993	1.11556108035798\\
22.0993	1.10432015984173\\
22.1993	1.08310676923316\\
22.2993	1.08005826902109\\
22.3993	1.10084235801592\\
22.4993	1.12237456821006\\
22.5993	1.1215597039139\\
22.6993	1.1021009748202\\
22.7993	1.0908910048719\\
22.8993	1.1046164305209\\
22.9993	1.13006836130489\\
23.0993	1.1273337783992\\
23.1993	1.10774276977415\\
23.2993	1.10354831293744\\
23.3993	1.12087903726385\\
23.4993	1.13547297986603\\
23.5993	1.12958934488348\\
23.6993	1.11082602616383\\
23.7993	1.1049956328475\\
23.8993	1.12235533204173\\
23.9993	1.13960949875559\\
24.0993	1.13257186518345\\
24.1993	1.1203226715375\\
24.2993	1.1300108745208\\
24.3993	1.15479059514184\\
24.4993	1.16799563721001\\
24.5993	1.1601653629698\\
24.6993	1.14355720508349\\
24.7993	1.14351180009837\\
24.8993	1.16212763001138\\
24.9992	1.17113238133369\\
25.0992	1.15707620361935\\
25.1992	1.14500647688248\\
25.2992	1.15378172901527\\
25.3992	1.16660326553334\\
25.4992	1.16986358386197\\
25.5992	1.15972676619182\\
25.6992	1.14392949610498\\
25.7992	1.14867495241087\\
25.8992	1.17105053394885\\
25.9992	1.17588062257325\\
26.0992	1.15759471028811\\
26.1992	1.15095227637972\\
26.2992	1.16763220333553\\
26.3992	1.18802426089168\\
26.4992	1.19085860826031\\
26.5992	1.17505063752381\\
26.6992	1.16495203729463\\
26.7992	1.1819156569447\\
26.8992	1.20444557961298\\
26.9992	1.20161562138112\\
27.0992	1.18043154383729\\
27.1992	1.17036755028715\\
27.2992	1.18712942804701\\
27.3992	1.20560059937511\\
27.4992	1.20567764243563\\
27.5992	1.18755006312254\\
27.6992	1.17907459444784\\
27.7992	1.19981999981316\\
27.8992	1.23060893036645\\
27.9992	1.23604712210188\\
28.0992	1.21320236340291\\
28.1992	1.19589917185957\\
28.2992	1.20822935196673\\
28.3992	1.23316346090777\\
28.4992	1.24238537771997\\
28.5992	1.22631609806485\\
28.6992	1.20792776670565\\
28.7992	1.2127080674048\\
28.8992	1.24405915193153\\
28.9992	1.26495454487593\\
29.0992	1.26602066713471\\
29.1992	1.26306983806566\\
29.2992	1.28045182975407\\
29.3992	1.31933916678032\\
29.4992	1.35710073050163\\
29.5992	1.3847083664759\\
29.6992	1.4086651306106\\
29.7992	1.44497023735738\\
29.8992	1.50338931243886\\
29.9991	1.54454650067393\\
30.0991	1.5675664652082\\
30.1991	1.59100912944806\\
30.2991	1.63686274860906\\
30.3991	1.69142263589293\\
30.4991	1.72746364451086\\
30.5991	1.74573787969188\\
30.6991	1.77588656458745\\
30.7991	1.82841034673777\\
30.8991	1.87988520483344\\
30.9991	1.91318789697416\\
31.0991	1.94016976762215\\
31.1991	1.98496048533192\\
31.2991	2.03654943735012\\
31.3991	2.07749137213539\\
31.4991	2.09872345971944\\
31.5991	2.12670440754931\\
31.6991	2.17401916506129\\
31.7991	2.22421857930937\\
31.8991	2.25692444837568\\
31.9991	2.27549327556317\\
32.0991	2.30575909220172\\
32.1991	2.35868612534274\\
32.2991	2.40553510171573\\
32.3991	2.43585402619744\\
32.4991	2.45251172735864\\
32.5991	2.48411232610227\\
32.6991	2.53991355725278\\
32.7991	2.59600066690024\\
32.8991	2.63033901934116\\
32.9991	2.63790098337009\\
33.0991	2.6402605109386\\
33.1991	2.66261303568704\\
33.2991	2.7096368867681\\
33.3991	2.75372855616966\\
33.4991	2.79009150751617\\
33.5991	2.81000087749872\\
33.6991	2.81552478654929\\
33.7991	2.84436540806099\\
33.8991	2.91374573499573\\
33.9991	2.9726503059199\\
34.0991	3.00803557126545\\
34.1991	3.02483300406141\\
34.2991	3.05114238959872\\
34.3991	3.09993458641396\\
34.4991	3.14509279806364\\
34.5991	3.17317075929253\\
34.6991	3.19387555405599\\
34.7991	3.22333468916196\\
34.8991	3.28130020686098\\
34.9991	3.33672236627109\\
35.0991	3.3687814153004\\
35.1991	3.38416869629049\\
35.2991	3.41705095551872\\
35.3991	3.47948327773926\\
35.4991	3.53316983838142\\
35.5991	3.56609907644365\\
35.6991	3.5984483553851\\
35.7991	3.65500699299043\\
35.8991	3.72072876669568\\
35.9991	3.75654931517578\\
36.0991	3.77046118754559\\
36.1991	3.79141263877228\\
36.2991	3.84303614431632\\
36.3991	3.89478601325352\\
36.4991	3.9292044970054\\
36.5991	3.94814607783793\\
36.6991	3.97913982144182\\
36.7991	4.03813855518044\\
36.8991	4.10121907396275\\
36.9991	4.13574657720063\\
37.0991	4.15370900659873\\
37.1991	4.1866855122732\\
37.2991	4.23704169872066\\
37.3991	4.2780174159012\\
37.4991	4.30364442954554\\
37.5991	4.32635879044086\\
37.6991	4.37794799746636\\
37.7991	4.44740767860282\\
37.8991	4.49399228635786\\
37.9991	4.51608160408674\\
38.0991	4.53792937215744\\
38.1991	4.58445218215251\\
38.2991	4.63503188883567\\
38.3991	4.66691630350443\\
38.4991	4.68656200275492\\
38.5991	4.71681418933862\\
38.6991	4.77454905470059\\
38.7991	4.84691167142906\\
38.8991	4.89210881663631\\
38.9991	4.91303455627479\\
39.0991	4.93156155045823\\
39.1991	4.97981550408897\\
39.2991	5.03307240830369\\
39.3991	5.0728217136647\\
39.4991	5.09196464999923\\
39.5991	5.11895259200473\\
39.6991	5.17412991075279\\
39.7991	5.23589161271382\\
39.8991	5.27266384064064\\
39.9991	5.29570298731872\\
40.0991	5.33059466154758\\
40.1991	5.38823015282702\\
40.2991	5.4384245135714\\
40.3991	5.47364458338295\\
40.499	5.49878011677992\\
40.5991	5.54291176238629\\
40.6991	5.61439805316901\\
40.7991	5.67472220735379\\
40.8991	5.71452671846159\\
40.9991	5.73789712341395\\
41.0991	5.77589485074953\\
41.1991	5.83788177343553\\
41.2991	5.894596284582\\
41.399	5.93534027702684\\
41.4991	5.95488084945047\\
41.5991	5.96256446576681\\
41.6991	5.98700437896173\\
41.7991	6.04770439443058\\
41.8991	6.10515500940704\\
41.999	6.1388789253592\\
42.0991	6.14380079318515\\
42.1991	6.13318332987053\\
42.2991	6.13230682649418\\
42.3991	6.15855786695548\\
42.4991	6.18622192954967\\
42.5991	6.20527249821477\\
42.6991	6.20870782456456\\
42.7991	6.20528835349744\\
42.8991	6.21015541826612\\
42.9991	6.23170622621011\\
43.0991	6.25590538057425\\
43.1991	6.25914634244849\\
43.2991	6.25155764670323\\
43.3991	6.24768281098106\\
43.4991	6.26047507541726\\
43.5991	6.28112686710642\\
43.6991	6.29983313438619\\
43.7991	6.30958825729462\\
43.8991	6.31202784246618\\
43.9991	6.32059145013364\\
44.0991	6.33937091264305\\
44.1991	6.35487935560896\\
44.2991	6.35438735850128\\
44.3991	6.34234605555139\\
44.4991	6.32613311467161\\
44.5991	6.31207524068159\\
44.6991	6.29500905450829\\
44.7991	6.280029941485\\
44.8991	6.26168738097011\\
44.999	6.2371395444342\\
45.099	6.20881125991807\\
45.199	6.19331960258703\\
45.299	6.18158843861412\\
45.399	6.1628159579644\\
45.499	6.13292616728073\\
45.599	6.10252774814818\\
45.699	6.08330511803042\\
45.799	6.0734610309098\\
45.899	6.06958275004735\\
45.999	6.05588511767612\\
46.099	6.02438470714001\\
46.199	5.99829373467508\\
46.299	5.98109235913359\\
46.399	5.98415713305247\\
46.499	5.97727723709491\\
46.599	5.94740078638223\\
46.699	5.91281251977713\\
46.799	5.89059286183862\\
46.899	5.86377992423236\\
46.999	5.83311585162598\\
47.099	5.79656026702734\\
47.199	5.74958448308241\\
47.299	5.73060397641962\\
47.399	5.7194881346176\\
47.499	5.72283535322697\\
47.599	5.72831027665843\\
47.699	5.70649727067935\\
47.799	5.69548475402381\\
47.899	5.68738940006194\\
47.999	5.6800393858493\\
48.099	5.66158355883033\\
48.199	5.62941924277735\\
48.299	5.60849324535944\\
48.399	5.60926864070637\\
48.4991	5.63088417052095\\
48.599	5.64390577819877\\
48.6991	5.63734198802203\\
48.7991	5.61403092581097\\
48.899	5.60462174112081\\
48.9991	5.62356179858617\\
49.0991	5.6439733602104\\
49.1991	5.64067082900929\\
49.2991	5.62504990737452\\
49.3991	5.6252719281696\\
49.499	5.6540461378781\\
49.5991	5.67998205698406\\
49.6991	5.67567854374324\\
49.7991	5.64252941821029\\
49.8991	5.61332453639043\\
49.999	5.61797701710537\\
50.099	5.63407820172447\\
50.199	5.63705657420417\\
50.299	5.60895031768225\\
50.399	5.58390793231065\\
50.499	5.59759421989245\\
50.599	5.62701348026682\\
50.699	5.64090995706021\\
50.799	5.62570882061578\\
50.899	5.60405610002698\\
50.999	5.6123968580355\\
51.099	5.64640249308634\\
51.199	5.67311410591465\\
51.299	5.67130225075249\\
51.399	5.65260004915592\\
51.499	5.6470087195636\\
51.599	5.67205794397709\\
51.699	5.69594271285828\\
51.799	5.69564674677136\\
51.899	5.66902306781478\\
51.999	5.64937103313872\\
52.099	5.66830146540463\\
52.199	5.70161415635398\\
52.299	5.7150356657989\\
52.399	5.704425031929\\
52.499	5.69232266269771\\
52.599	5.70589120354809\\
52.699	5.72676774301318\\
52.799	5.7312372806327\\
52.899	5.70898770424335\\
52.999	5.68751573482855\\
53.099	5.69119039411791\\
53.199	5.71535497843585\\
53.2991	5.73430955658394\\
53.3991	5.72764167483611\\
53.4991	5.71732525504175\\
53.5991	5.72334007498784\\
53.6991	5.74860139824117\\
53.7991	5.76560731074258\\
53.8991	5.76010456522758\\
53.999	5.74426004451496\\
54.0991	5.74294257852199\\
54.1991	5.77245018251229\\
54.299	5.80877487493622\\
54.3991	5.81736276701522\\
54.4991	5.80269456730148\\
54.599	5.78512229828477\\
54.6991	5.79890413036834\\
54.7991	5.82166481597054\\
54.8991	5.83722083365636\\
54.999	5.83778772762765\\
55.099	5.84622590235572\\
55.199	5.87871956112129\\
55.299	5.93055639758517\\
55.399	5.973759134858\\
55.499	5.99820497900944\\
55.599	6.02677673814792\\
55.699	6.05570548435146\\
55.799	6.09312959980215\\
55.899	6.13340778427858\\
55.999	6.16386641008113\\
56.099	6.18045896100801\\
56.199	6.19501356886504\\
56.299	6.20991877830906\\
56.399	6.22948817847612\\
56.499	6.2377737773344\\
56.599	6.22808757722736\\
56.699	6.22884050463363\\
56.799	6.25664138349061\\
56.899	6.29560380034235\\
56.999	6.31217429844381\\
57.099	6.3057097537028\\
57.199	6.29274324098653\\
57.299	6.29968200413777\\
57.399	6.3206706163166\\
57.4991	6.33440768947044\\
57.5991	6.33050463440568\\
57.6991	6.33290724478882\\
57.7991	6.35546911229082\\
57.8991	6.41118732620308\\
57.9991	6.47141623569708\\
58.0991	6.50161895048845\\
58.1991	6.50213625109927\\
58.2991	6.50285169403688\\
58.399	6.53045712616126\\
58.4991	6.57739958806292\\
58.5991	6.60556731342955\\
58.6991	6.59882425709332\\
58.7991	6.5851751087076\\
58.8991	6.59528923242464\\
58.9991	6.60864760539422\\
59.0991	6.61592434420667\\
59.1991	6.60532120610696\\
59.2991	6.58613755917429\\
59.3991	6.59382621905267\\
59.4991	6.60882456060149\\
59.5991	6.61075564957963\\
59.6991	6.59695065253572\\
59.7991	6.57749512442724\\
59.8991	6.57738146598899\\
59.999	6.58416228527372\\
60.099	6.59366201970303\\
60.199	6.58152086129073\\
60.299	6.54788182529704\\
60.399	6.53927918465832\\
60.499	6.55304010574823\\
60.599	6.56170251818758\\
60.699	6.55363200219301\\
60.799	6.52834972386207\\
60.899	6.51352487137652\\
60.999	6.52127571776708\\
61.099	6.55167703552514\\
61.199	6.56031613447699\\
61.299	6.52869728716875\\
61.399	6.50711082603525\\
61.499	6.52371151039488\\
61.599	6.53456273730568\\
61.699	6.53200288465383\\
61.799	6.50841710647714\\
61.899	6.48285990789117\\
61.999	6.4802831761874\\
62.099	6.50669876454306\\
62.199	6.52811687584546\\
62.299	6.50156686788208\\
62.399	6.47346790709107\\
62.499	6.48082530688247\\
62.599	6.49335642590519\\
62.699	6.49604129537234\\
62.799	6.47538009705581\\
62.899	6.44404467265589\\
62.999	6.43388636608436\\
63.099	6.44894210566466\\
63.199	6.47821249589278\\
63.299	6.47189661117326\\
63.399	6.4340806652726\\
63.499	6.43009504907627\\
63.599	6.44589075642159\\
63.699	6.45160413864606\\
63.799	6.43659514546618\\
63.8991	6.40483738252572\\
63.999	6.38333256508367\\
64.099	6.38845033618503\\
64.1991	6.41693884059673\\
64.2991	6.42257108865889\\
64.3991	6.38702469561529\\
64.4991	6.36674300408181\\
64.599	6.38054207366718\\
64.6991	6.38757650715508\\
64.7991	6.38079008247882\\
64.8991	6.35195535798283\\
64.999	6.32077346888356\\
65.099	6.31325157925614\\
65.199	6.33506264453298\\
65.299	6.35430373698031\\
65.399	6.32835118300011\\
65.499	6.29672765928344\\
65.599	6.30384176975217\\
65.699	6.31278930655489\\
65.799	6.31122253025044\\
65.899	6.28807305223758\\
65.999	6.25393296440783\\
66.099	6.24117748969337\\
66.199	6.25452941250189\\
66.299	6.279609921061\\
66.399	6.26703416330723\\
66.499	6.22826646327333\\
66.599	6.22520904427097\\
66.699	6.23610887320306\\
66.799	6.23878475267362\\
66.899	6.2312591308252\\
66.999	6.20721422785565\\
67.099	6.19037671887932\\
67.1991	6.20752007352789\\
67.299	6.24652856334967\\
67.3991	6.26969677322022\\
67.4991	6.25978990634225\\
67.5991	6.23567998919315\\
67.6991	6.22681677856657\\
67.7991	6.25246406428974\\
67.8991	6.30235841561492\\
67.9991	6.34595771652266\\
68.0991	6.36248616498254\\
68.1991	6.37094082522274\\
68.2991	6.38268008794714\\
68.3991	6.3850224532071\\
68.4991	6.37124208909096\\
68.5991	6.34779569893624\\
68.6991	6.33297417657656\\
68.7991	6.33005091367788\\
68.8991	6.33076050568373\\
68.9991	6.32387198246379\\
69.0991	6.30065397172439\\
69.1991	6.27424004212748\\
69.2991	6.26547490745406\\
69.3991	6.26688291349126\\
69.4991	6.25588672723067\\
69.5991	6.22734933834159\\
69.6991	6.19859092016921\\
69.7991	6.18734580729592\\
69.8991	6.18924420444853\\
69.999	6.18536320685114\\
70.099	6.16890442772727\\
70.199	6.14785377103694\\
70.2991	6.13993945326816\\
70.3991	6.14160490338042\\
70.4991	6.13436979203892\\
70.599	6.11251310972244\\
70.699	6.08822464499977\\
70.7991	6.07454309728165\\
70.8991	6.07780301223333\\
70.9991	6.08612591430268\\
71.0991	6.08208586332644\\
71.1991	6.0612299316412\\
71.2991	6.04315866385975\\
71.3991	6.04272793067952\\
71.4991	6.0497697479957\\
71.5991	6.04708346837336\\
71.6991	6.02919937066548\\
71.7991	6.01323685167784\\
71.8991	6.01218052721842\\
71.9991	6.02517941002898\\
72.0991	6.03541972913516\\
72.1991	6.03055461655173\\
72.2991	6.01416006870189\\
72.3991	6.00489149251115\\
72.4991	6.02044491950657\\
72.5991	6.03468330950279\\
72.6991	6.03304078696805\\
72.7991	6.01390107858152\\
72.8991	6.00212900372068\\
72.9991	6.00944920651704\\
73.0991	6.02223658614661\\
73.1991	6.03321475237159\\
73.2991	6.03216333390021\\
73.3991	6.02551342826131\\
73.4991	6.01757002047852\\
73.5991	6.02005056357744\\
73.6991	6.01736204596707\\
73.7991	6.00554966558011\\
73.8991	5.96963408660062\\
73.9991	5.90385660578694\\
74.0991	5.84108466583577\\
74.1991	5.80366112091963\\
74.2991	5.78717506587476\\
74.3991	5.76030064780778\\
74.4991	5.70109128221431\\
74.5991	5.63791780885373\\
74.6991	5.59224458135761\\
74.7991	5.5606596680233\\
74.8991	5.50879864863271\\
74.9991	5.44317986537535\\
75.099	5.39672348859399\\
75.1991	5.37333528086231\\
75.2991	5.34636855542766\\
75.3991	5.29707693145028\\
75.4991	5.23886949154155\\
75.5991	5.19819460781429\\
75.6991	5.16623232780569\\
75.7991	5.1179513778446\\
75.8991	5.05824227514389\\
75.9991	5.01384279573999\\
76.0991	4.99062345427741\\
76.1991	4.96493577261551\\
76.2991	4.91488445274296\\
76.3991	4.86350022585841\\
76.4991	4.82663464112831\\
76.5991	4.796390851505\\
76.6991	4.74731991927018\\
76.7991	4.69347266712468\\
76.8991	4.64690652002797\\
76.9991	4.61495500069803\\
77.0991	4.58014993133977\\
77.1991	4.52853007345135\\
77.2991	4.48111100374678\\
77.3991	4.44770298767615\\
77.4991	4.41839462526251\\
77.5991	4.36793151435139\\
77.6991	4.31798728466593\\
77.7991	4.29064226092557\\
77.8991	4.26372792140273\\
77.9991	4.22062092204316\\
78.0991	4.1692196303714\\
78.1991	4.1255131916208\\
78.2991	4.09338384789571\\
78.3991	4.05800293277525\\
78.4991	3.99735443725046\\
78.5991	3.94701255260017\\
78.6991	3.91285212205618\\
78.7991	3.87568345675142\\
78.8991	3.8295354071878\\
78.9991	3.7846310528354\\
79.0991	3.7522357443725\\
79.1992	3.72182471251812\\
79.2992	3.67352276123046\\
79.3992	3.61119613141033\\
79.4992	3.57256763871261\\
79.5991	3.55034693489514\\
79.6992	3.52342217701273\\
79.7992	3.47179505591514\\
79.8992	3.42993684536733\\
79.9991	3.38825061850389\\
80.0991	3.34808995525574\\
80.1991	3.28872081360728\\
80.2991	3.24279923521175\\
80.3991	3.20523861146103\\
80.4991	3.17879418676442\\
80.5992	3.13360483404801\\
80.6992	3.08627649727287\\
80.7991	3.04797045006443\\
80.8991	3.01240746923984\\
80.9992	2.96205445428549\\
81.0992	2.91927628891871\\
81.1992	2.88912645849085\\
81.2992	2.87341684605418\\
81.3991	2.85539569576377\\
81.4992	2.83422657350177\\
81.5992	2.83304729007637\\
81.6992	2.85229459169481\\
81.7992	2.86590474684357\\
81.8992	2.86601368526278\\
81.9992	2.86067958564585\\
82.0992	2.85563522939899\\
82.1992	2.85859955111449\\
82.2992	2.85660947899623\\
82.3992	2.84604773010826\\
82.4992	2.80707035876503\\
82.5992	2.74487854584152\\
82.6992	2.69234383440228\\
82.7992	2.66696216509651\\
82.8992	2.63883350869331\\
82.9992	2.57998262607165\\
83.0992	2.50868830548029\\
83.1992	2.45446009208683\\
83.2992	2.4224231996974\\
83.3992	2.37846941543114\\
83.4992	2.30371509347809\\
83.5992	2.23510146619762\\
83.6992	2.20694284299775\\
83.7992	2.19453597077968\\
83.8992	2.15365486998901\\
83.9992	2.09312139735044\\
84.0992	2.04644444425058\\
84.1992	2.04280368650679\\
84.2992	2.03249969170637\\
84.3992	1.98607956180968\\
84.4992	1.92867722975531\\
84.5993	1.88072482213687\\
84.6992	1.84432658148978\\
84.7993	1.78379228295364\\
84.8993	1.72648043691729\\
84.9991	1.69213071642096\\
85.0991	1.67240271895535\\
85.1992	1.63183233012803\\
85.2992	1.58071338254755\\
85.3992	1.52973847758994\\
85.4992	1.49826323015588\\
85.5992	1.45696652765988\\
85.6992	1.40803115240256\\
85.7992	1.36536113777363\\
85.8992	1.3327915893903\\
85.9992	1.29955750810168\\
86.0992	1.24831910898889\\
86.1992	1.21291518947382\\
86.2992	1.2160041846716\\
86.3992	1.22397321564499\\
86.4992	1.20820892363946\\
86.5992	1.18752603052223\\
86.6992	1.18947891060615\\
86.7992	1.21362444371667\\
86.8992	1.23070753777337\\
86.9992	1.23016534271755\\
87.0992	1.21514818994786\\
87.1992	1.20719100197666\\
87.2992	1.22947281231026\\
87.3992	1.25588377668982\\
87.4992	1.2523480753131\\
87.5992	1.22785473884503\\
87.6992	1.21490499640408\\
87.7992	1.22874661795673\\
87.8992	1.24631812192033\\
87.9992	1.24597063458863\\
88.0992	1.23183332471009\\
88.1992	1.21938174780441\\
88.2992	1.22309253303561\\
88.3993	1.23628615271362\\
88.4993	1.22905015389054\\
88.5992	1.20322071496083\\
88.6993	1.18351497895381\\
88.7993	1.18776612603865\\
88.8993	1.20533586317549\\
88.9993	1.20896856618053\\
89.0993	1.19334556310163\\
89.1993	1.17015302103818\\
89.2993	1.16133547241127\\
89.3993	1.17342131101055\\
89.4993	1.17420237676001\\
89.5993	1.16022057204152\\
89.6993	1.14147853306593\\
89.7993	1.13641375601575\\
89.8993	1.14817762171498\\
89.9993	1.16175489433148\\
90.0993	1.15966082267573\\
90.1993	1.14790439342741\\
90.2993	1.13610079169894\\
90.3993	1.13946170605563\\
90.4993	1.15129218840218\\
90.5993	1.14701579787819\\
90.6993	1.12907931916985\\
90.7993	1.11415405995576\\
90.8993	1.11317683169437\\
90.9993	1.12752556933436\\
91.0993	1.1403946963163\\
91.1993	1.1416959575915\\
91.2993	1.13111412897473\\
91.3993	1.12672500080029\\
91.4993	1.14328690630128\\
91.5993	1.15847387599806\\
91.6993	1.15116932463245\\
91.7993	1.1303325468827\\
91.8993	1.1155181303447\\
91.9993	1.11774097371043\\
92.0993	1.12604094705761\\
92.1993	1.12998485701329\\
92.2993	1.11894477391606\\
92.3993	1.10308936684464\\
92.4993	1.10215375179839\\
92.5993	1.12429038742098\\
92.6993	1.13498105968864\\
92.7993	1.12452600954576\\
92.8993	1.10566065323806\\
92.9994	1.0994238137279\\
93.0993	1.11473553557376\\
93.1994	1.13053458916793\\
93.2994	1.12938969067675\\
93.3994	1.1175868102603\\
93.4994	1.11019125135986\\
93.5994	1.1175808393095\\
93.6994	1.12657718668332\\
93.7994	1.12301927167329\\
93.8994	1.10584082103538\\
93.9994	1.09543206448299\\
94.0994	1.10656423995846\\
94.1994	1.12835218499736\\
94.2994	1.14170083000552\\
94.3994	1.13257421903456\\
94.4994	1.11423575238418\\
94.5994	1.10761108806818\\
94.6994	1.12144688789936\\
94.7994	1.12621810356654\\
94.8994	1.11633371445261\\
94.9993	1.09474427488225\\
95.0993	1.07185385611318\\
95.1993	1.05305655596526\\
95.2994	1.02685313064688\\
95.3994	0.992403535937593\\
95.4994	0.946990592043059\\
95.5994	0.905193115092062\\
95.6994	0.888101245846062\\
95.7994	0.885342470697922\\
95.8994	0.87821515405646\\
95.9994	0.861466484817101\\
96.0994	0.850008667397665\\
96.1994	0.85819582041099\\
96.2994	0.888209879840735\\
96.3994	0.913423187232523\\
96.4994	0.92307805911983\\
96.5994	0.937041187557684\\
96.6994	0.970006361041794\\
96.7994	1.00539692906697\\
96.8994	1.0172845839621\\
96.9994	1.01198349282275\\
97.0994	1.0158162628329\\
97.1994	1.03051778130726\\
97.2994	1.05453459664914\\
97.3994	1.05704607325461\\
97.4994	1.04457232048421\\
97.5994	1.03728035302095\\
97.6994	1.04936090822296\\
97.7994	1.06264740807494\\
97.8994	1.0622013397189\\
97.9994	1.05008246931524\\
98.0994	1.04962012844897\\
98.1995	1.05989586293789\\
98.2995	1.07476644048911\\
98.3995	1.07636384378572\\
98.4995	1.06622203841892\\
98.5995	1.04907828217361\\
98.6995	1.04543288463163\\
98.7995	1.05936910017988\\
98.8995	1.07854090544593\\
98.9995	1.09208083092521\\
99.0995	1.10200480591715\\
99.1995	1.11880801234574\\
99.2995	1.14057340563755\\
99.3995	1.15750191288773\\
99.4995	1.16220085891863\\
99.5995	1.16028942076529\\
99.6995	1.15737932798067\\
99.7995	1.15618757388502\\
99.8995	1.15544738467589\\
99.9994	1.15208932261284\\
100.0994	1.1511700811166\\
100.1995	1.15006951459578\\
100.2995	1.1501397907843\\
100.3995	1.14972373636243\\
100.4995	1.14973259076023\\
100.5995	1.14927816366122\\
100.6995	1.1493842225252\\
100.7995	1.14950807808685\\
100.8995	1.14948554283531\\
100.9995	1.14947818596541\\
101.0995	1.14948544129942\\
101.1995	1.14950208597568\\
101.2995	1.14950582252439\\
101.3995	1.14948767174245\\
101.4995	1.14946123728883\\
101.5995	1.14945898093058\\
101.6995	1.14945529202646\\
101.7995	1.14949269948976\\
101.8995	1.14949974708042\\
101.9995	1.14947357972985\\
102.0995	1.14944859711973\\
102.1995	1.14943053828264\\
102.2995	1.14939385964618\\
102.3995	1.14939099825663\\
102.4995	1.14936345514095\\
102.5995	1.14934156434535\\
102.6995	1.14936631125493\\
102.7995	1.14942117928113\\
102.8995	1.14944780869392\\
102.9995	1.149450004182\\
103.0995	1.14943634610984\\
103.1996	1.14942971512472\\
103.2996	1.14943132276337\\
103.3996	1.14946672714376\\
103.4996	1.14948455951632\\
103.5996	1.14947335914761\\
103.6996	1.14944814740188\\
103.7996	1.14943790605388\\
103.8996	1.14945475628895\\
103.9996	1.14945209950052\\
104.0996	1.14952070903089\\
104.1996	1.1499289029925\\
104.2996	1.15062886818305\\
104.3996	1.1508287077097\\
104.4996	1.15257845399078\\
104.5996	1.15612191864239\\
104.6996	1.14742937055719\\
104.7996	1.1369705141513\\
104.8996	1.13326389083592\\
104.9996	1.13478754282704\\
105.0996	1.13677728807296\\
105.1996	1.1398763529937\\
105.2996	1.14006609310183\\
105.3996	1.14063051594144\\
105.4996	1.14093332346937\\
105.5996	1.14099367574982\\
105.6996	1.14089717761639\\
105.7996	1.14077941627267\\
105.8996	1.14066008350389\\
105.9996	1.14059159233387\\
106.0996	1.14053164136457\\
106.1996	1.14051923441015\\
106.2996	1.14052757294746\\
106.3996	1.14051570252148\\
106.4996	1.14051861295625\\
106.5996	1.14054923596298\\
106.6996	1.14055654593638\\
106.7996	1.14055385275403\\
106.8996	1.14056592371083\\
106.9996	1.14056484976848\\
107.0996	1.14056144618709\\
107.1996	1.14056580203786\\
107.2996	1.14058112826812\\
107.3996	1.14058015218956\\
107.4996	1.14055703672438\\
107.5996	1.14054715718996\\
107.6996	1.14055824150546\\
107.7996	1.14055592958752\\
107.8996	1.14056355576479\\
107.9996	1.14058589510019\\
108.0996	1.14057162050838\\
108.1997	1.14057427343549\\
108.2996	1.14058614764871\\
108.3997	1.14058292088614\\
108.4997	1.14057944735934\\
108.5997	1.14058229203155\\
108.6997	1.14056787781624\\
108.7997	1.14056661657999\\
108.8997	1.14056942914001\\
108.9997	1.14059809173604\\
109.0997	1.14062198661955\\
109.1997	1.14063462447133\\
109.2997	1.14065390797498\\
109.3997	1.14064335708696\\
109.4997	1.14066550831912\\
109.5997	1.14071350513399\\
109.6997	1.1407287104664\\
109.7997	1.14074913862543\\
109.8997	1.14074157455658\\
109.9996	1.14074390544773\\
110.0996	1.14077311163171\\
110.1996	1.14079701726136\\
110.2996	1.14084097136305\\
110.3996	1.14091748317375\\
110.4996	1.14105087545793\\
110.5996	1.14102402212052\\
110.6996	1.14106405412355\\
110.7997	1.14183060820583\\
110.8997	1.14296689217332\\
110.9997	1.14377157521213\\
111.0997	1.13642812423969\\
111.1997	1.10407365970695\\
111.2997	1.05110443545902\\
111.3997	0.982281725404427\\
111.4997	0.901906872162306\\
111.5997	0.803170142310618\\
111.6997	0.691052912986905\\
111.7997	0.570350094664094\\
111.8997	0.45225527377077\\
111.9997	0.342710340064204\\
112.0997	0.244882837781081\\
112.1997	0.161433829122911\\
112.2997	0.0986329371470531\\
112.3997	0.0532388640682253\\
112.4997	0.0202206239072178\\
112.5997	-0.00261835603743976\\
112.6997	-0.0178890090615875\\
112.7997	-0.0258031406327432\\
112.8997	-0.0277105716495132\\
112.9997	-0.029982557670251\\
113.0997	-0.0314376592666128\\
113.1997	-0.032814536810398\\
113.2997	-0.0331839718605429\\
113.3997	-0.0332553934294288\\
113.4997	-0.0332428812381931\\
113.5997	-0.0332143798702602\\
113.6997	-0.0331817085186179\\
113.7998	-0.0331621757468452\\
113.8998	-0.0331734351712869\\
113.9998	-0.0331678087787608\\
114.0998	-0.0331866533797014\\
114.1998	-0.0332065732474729\\
114.2998	-0.0332045775291003\\
114.3998	-0.0331835369372362\\
114.4998	-0.0331776040511627\\
114.5998	-0.0331658102369512\\
114.6998	-0.0331485319462563\\
114.7998	-0.0331553781901532\\
114.8998	-0.033163266799886\\
114.9997	-0.0331570210278016\\
115.0997	-0.0331286249001792\\
115.1997	-0.0331034685483589\\
115.2997	-0.0330713633221813\\
115.3997	-0.0330611655869666\\
115.4997	-0.0330591029108316\\
115.5997	-0.0330516321633916\\
115.6997	-0.0330517389396585\\
115.7998	-0.0330443129849174\\
115.8997	-0.0330531737962854\\
115.9997	-0.0330512500393656\\
116.0998	-0.0330449174913825\\
116.1998	-0.0330949258866255\\
116.2998	-0.0331045086706836\\
116.3998	-0.0331195238560802\\
116.4998	-0.0331342084551289\\
116.5998	-0.0331006948002559\\
116.6998	-0.0330723560650059\\
116.7998	-0.0330606566349108\\
116.8998	-0.033067346047583\\
116.9998	-0.0330833930569687\\
117.0998	-0.0330779693529084\\
117.1998	-0.0330579819557251\\
117.2998	-0.03303419665588\\
117.3998	-0.0330173119596398\\
117.4998	-0.033035069507906\\
117.5998	-0.0330677983302337\\
117.6998	-0.0330700389735499\\
117.7998	-0.0330670214416415\\
117.8998	-0.033041310071696\\
117.9998	-0.0330424105277094\\
118.0998	-0.0330274339666008\\
118.1998	-0.0330202381349547\\
118.2998	-0.0330200273111385\\
118.3998	-0.0329890168981595\\
118.4998	-0.0329827387299748\\
118.5998	-0.0329702844769457\\
118.6998	-0.0329644699522091\\
118.7998	-0.0329453933616888\\
118.8998	-0.0329454703826626\\
118.9998	-0.0329565670269611\\
119.0999	-0.0329750957414585\\
119.1999	-0.0329845552744248\\
119.2999	-0.0330183639603376\\
119.3999	-0.0330362687156174\\
119.4998	-0.0330414491358848\\
119.5999	-0.0330655312287929\\
119.6999	-0.0330759522099182\\
119.7999	-0.0330838988504951\\
};
\addplot [color=black,dashed,forget plot]
  table[row sep=crcr]{%
41.6291	-1\\
41.6291	7\\
};
\addplot [color=black,dashed,forget plot]
  table[row sep=crcr]{%
45	-1\\
45	7\\
};
\addplot [color=black,dashed,forget plot]
  table[row sep=crcr]{%
60	-1\\
60	7\\
};
\addplot [color=black,dashed,forget plot]
  table[row sep=crcr]{%
73.3591	-1\\
73.3591	7\\
};
\node[right, align=left, text=black]
at (axis cs:41.829,0.75) {\footnotesize Position fix};
\node[right, align=left, text=black]
at (axis cs:45.2,2.5) {\footnotesize Bag};
\node[right, align=left, text=black]
at (axis cs:60.2,2.5) {\footnotesize Pocket};
\node[right, align=left, text=black]
at (axis cs:73.559,0.75) {\footnotesize Position fix};
\end{axis}
\end{tikzpicture}% \\ 
  \tikzsetnextfilename{tikz-velocity}%
  % This file was created by matlab2tikz.
%
%The latest updates can be retrieved from
%  http://www.mathworks.com/matlabcentral/fileexchange/22022-matlab2tikz-matlab2tikz
%where you can also make suggestions and rate matlab2tikz.
%
\definecolor{mycolor1}{rgb}{0.00000,0.44700,0.74100}%
\definecolor{mycolor2}{rgb}{0.85000,0.32500,0.09800}%
\definecolor{mycolor3}{rgb}{0.92900,0.69400,0.12500}%
\definecolor{mycolor4}{rgb}{0.49400,0.18400,0.55600}%
\definecolor{mycolor5}{rgb}{0.46600,0.67400,0.18800}%
\definecolor{mycolor6}{rgb}{0.30100,0.74500,0.93300}%
%
\begin{tikzpicture}

\begin{axis}[%
width=0.951\figurewidth,
height=\figureheight,
at={(0\figurewidth,0\figureheight)},
scale only axis,
xmin=0,
xmax=120,
xtick={0,10,20,30,40,50,60,70,80,90,100,110,120},
xticklabels={{\phantom{0}},{},{},{},{},{},{},{},{},{},{},{},{\phantom{120}}},
ymin=-2.2,
ymax=2.2,
ytick={-2,  0,  2},
ylabel={Velocity (m/s)},
axis background/.style={fill=white},
legend style={at={(0.97,0.03)},anchor=south east,legend cell align=left,align=left,draw=none,fill=none},
axis on top
]
\addplot [color=mycolor1,solid]
  table[row sep=crcr]{%
0	0.000108870003349578\\
};
\addlegendentry{$x$};

\addplot [color=mycolor2,solid]
  table[row sep=crcr]{%
0	-0.00012245986462566\\
};
\addlegendentry{$y$};

\addplot [color=mycolor3,solid]
  table[row sep=crcr]{%
0	3.30982285945311e-07\\
};
\addlegendentry{$z$};


\addplot[area legend,solid,draw=white!90!black,fill=white!90!black,forget plot]
table[row sep=crcr] {%
x	y\\
0.11	-2.2\\
4.6799	-2.2\\
4.6799	2.2\\
0.11	2.2\\
}--cycle;

\addplot[area legend,solid,draw=white!90!black,fill=white!90!black,forget plot]
table[row sep=crcr] {%
x	y\\
8.6097	-2.2\\
10.3196	-2.2\\
10.3196	2.2\\
8.6097	2.2\\
}--cycle;

\addplot[area legend,solid,draw=white!90!black,fill=white!90!black,forget plot]
table[row sep=crcr] {%
x	y\\
11.3996	-2.2\\
12.7295	-2.2\\
12.7295	2.2\\
11.3996	2.2\\
}--cycle;

\addplot[area legend,solid,draw=white!90!black,fill=white!90!black,forget plot]
table[row sep=crcr] {%
x	y\\
14.5695	-2.2\\
18.5394	-2.2\\
18.5394	2.2\\
14.5695	2.2\\
}--cycle;

\addplot[area legend,solid,draw=white!90!black,fill=white!90!black,forget plot]
table[row sep=crcr] {%
x	y\\
100.7695	-2.2\\
104.2096	-2.2\\
104.2096	2.2\\
100.7695	2.2\\
}--cycle;

\addplot[area legend,solid,draw=white!90!black,fill=white!90!black,forget plot]
table[row sep=crcr] {%
x	y\\
106.1596	-2.2\\
110.4496	-2.2\\
110.4496	2.2\\
106.1596	2.2\\
}--cycle;

\addplot[area legend,solid,draw=white!90!black,fill=white!90!black,forget plot]
table[row sep=crcr] {%
x	y\\
113.3997	-2.2\\
116.0098	-2.2\\
116.0098	2.2\\
113.3997	2.2\\
}--cycle;
\addplot [color=mycolor4,solid,forget plot]
  table[row sep=crcr]{%
0	0.000108870003349578\\
0.1	0.00787491027524915\\
0.2	-0.0110521848656225\\
0.3	-0.00925081461938934\\
0.4	-0.000328890267193925\\
0.5	-0.000366657049258367\\
0.6	-0.00014972643671795\\
0.7	0.000141405488305227\\
0.8	0.000288059799692694\\
0.9	-8.86807437575512e-05\\
1	-0.000244110864179785\\
1.0999	-0.000445608765685011\\
1.1999	-0.00049586428886041\\
1.2999	-0.000835915930658387\\
1.3999	-0.000212345236921559\\
1.4999	-0.000121780828578515\\
1.5999	-0.000420550689642524\\
1.6999	-0.000124043379717562\\
1.7999	-0.000577441469440169\\
1.8999	-0.000839135009904089\\
1.9999	-0.00132737569607175\\
2.0999	-0.00126747520351729\\
2.1999	-0.00123112791074031\\
2.2999	-0.000838184644655779\\
2.3999	-0.000749250668738654\\
2.4999	-0.000127710868359977\\
2.5999	-0.000119163700981923\\
2.6999	0.000311588938131075\\
2.7999	0.000247980731546773\\
2.8999	-1.11422066570649e-07\\
2.9999	-0.00035297777675503\\
3.0999	3.80313724549342e-06\\
3.1999	-0.000323482888460586\\
3.2999	-0.000861789184599291\\
3.3999	-0.000730125147555537\\
3.4999	-0.00054165923728221\\
3.5999	-0.000408189471442608\\
3.6999	-0.000463135085139969\\
3.7999	-0.000281641130368171\\
3.8999	-0.000310058317650622\\
3.9999	-1.2978129650078e-06\\
4.0999	0.000203147515164454\\
4.1999	0.000319744952539585\\
4.2999	0.000111645271112689\\
4.3999	-7.20550099841583e-05\\
4.4999	-6.11417533692862e-05\\
4.5999	0.000738988381675457\\
4.6999	0.00136398600037911\\
4.7999	0.00205607017012065\\
4.8998	0.00254240909308611\\
4.9998	0.00292957868198033\\
5.0998	0.00361578026504271\\
5.1998	0.00413703798822537\\
5.2998	0.0049753633990653\\
5.3998	0.00530115732787967\\
5.4997	0.00598326857462112\\
5.5997	0.0070349656375598\\
5.6998	0.00876406037363373\\
5.7998	0.00993727072534817\\
5.8998	0.010819644099288\\
5.9998	0.0123853749423685\\
6.0997	0.0145504510357094\\
6.1997	0.0107448401143718\\
6.2997	0.0143322019086093\\
6.3997	0.0363031497999808\\
6.4997	0.094688204268124\\
6.5997	0.161116465010697\\
6.6997	0.0877756946389005\\
6.7997	-0.0649152351598027\\
6.8997	-0.131302801410049\\
6.9997	-0.107966071388831\\
7.0997	-0.0640017088402598\\
7.1997	-0.0655269462631566\\
7.2997	-0.0441760910333624\\
7.3997	-0.0223037374965367\\
7.4997	0.0536803568826999\\
7.5997	0.0399211011695009\\
7.6997	0.023237102591116\\
7.7997	0.00787901146202023\\
7.8997	-0.00542319156995173\\
7.9997	-0.00587804652777572\\
8.0997	-0.00625642746984492\\
8.1997	0.0201878161994428\\
8.2997	0.0018210974504786\\
8.3997	0.00407770571045107\\
8.4997	-0.00123407492092692\\
8.5997	-0.000260917169874052\\
8.6997	-0.000243018839530356\\
8.7997	7.01172200312209e-05\\
8.8997	-0.000306251216863282\\
8.9997	-0.000585216333416315\\
9.0997	-0.000123608861890736\\
9.1997	9.19623583880633e-05\\
9.2997	0.000142245937682732\\
9.3997	0.000403012764651748\\
9.4997	0.000112438282648647\\
9.5997	-0.000183045240369772\\
9.6997	-5.9964803735455e-06\\
9.7997	0.000146835535625914\\
9.8997	0.000522516271843338\\
9.9996	0.000125477408172775\\
10.0996	0.000549463478546241\\
10.1996	0.000655560399406569\\
10.2996	0.000124980711086883\\
10.3996	0.000826364136665232\\
10.4996	0.0019084013424872\\
10.5996	0.00202525309858716\\
10.6996	0.00275724513796055\\
10.7996	0.00286587795126343\\
10.8996	0.00328110130365816\\
10.9996	0.00269527516485992\\
11.0996	0.00249039292032935\\
11.1996	0.00161676007827622\\
11.2996	0.000950234888272827\\
11.3996	2.02050755230759e-06\\
11.4995	0.000307941743780484\\
11.5995	0.000444088820317046\\
11.6996	0.000148307763663291\\
11.7996	4.64893645119779e-05\\
11.8995	-0.000151830282831339\\
11.9996	0.000303241542875875\\
12.0995	-0.000117266246647738\\
12.1995	0.00021769289370787\\
12.2996	0.000204068637821372\\
12.3995	1.66451729538125e-05\\
12.4995	0.000466445305026078\\
12.5995	0.000432171320620229\\
12.6995	1.28106629758871e-05\\
12.7995	0.00655865577095382\\
12.8995	0.0133100019125512\\
12.9995	0.0200792378324572\\
13.0995	0.0264289630231325\\
13.1995	0.0324219813488934\\
13.2995	0.0381165311490187\\
13.3995	0.0441295424737553\\
13.4995	0.0206699189040602\\
13.5995	-0.0534792896574159\\
13.6995	-0.0368386905244625\\
13.7995	0.0384345399619126\\
13.8995	0.0942441394779109\\
13.9995	0.0893859115265655\\
14.0995	0.0329463328420389\\
14.1995	-0.0189013208590906\\
14.2995	0.00331087617009913\\
14.3995	-0.00107561157955159\\
14.4995	-0.00103277584943862\\
14.5995	9.04690857225408e-07\\
14.6995	0.000458406507436451\\
14.7995	0.000218000742505086\\
14.8995	8.54270414837406e-05\\
14.9994	0.000225735930140055\\
15.0994	-0.000132301251207455\\
15.1994	-2.10411524338137e-05\\
15.2994	2.38673145730298e-05\\
15.3994	-0.000636725118357561\\
15.4994	-0.000270265389657475\\
15.5994	-0.00031868773512787\\
15.6994	-0.000391783434598291\\
15.7994	-1.49806901378548e-05\\
15.8994	-9.09376486433913e-05\\
15.9994	0.000139614593604141\\
16.0994	0.000154815255200269\\
16.1994	4.65199677294925e-05\\
16.2994	-0.0001865263164354\\
16.3994	-0.00036577949745826\\
16.4994	-0.000548145844730711\\
16.5994	-0.000233424816180268\\
16.6994	9.70244635742431e-05\\
16.7994	-0.000175076167061742\\
16.8994	-4.18335952470121e-05\\
16.9994	-6.47033528583696e-05\\
17.0994	0.000191113381682964\\
17.1994	-0.000555831952028803\\
17.2994	-0.000123728416200494\\
17.3994	0.000403788195769649\\
17.4994	0.000617587671689583\\
17.5994	0.000578243278359282\\
17.6994	0.000444411054273721\\
17.7994	0.000709497235108811\\
17.8994	0.000620008840205209\\
17.9994	0.000425241496746734\\
18.0994	0.000646050799104486\\
18.1994	0.000147826591415747\\
18.2994	-0.000172060084208349\\
18.3994	9.24572610939018e-06\\
18.4994	0.00446918928251536\\
18.5994	0.00940417454075202\\
18.6994	0.0124786214195847\\
18.7994	0.0133137330941053\\
18.8994	0.0726979752576192\\
18.9994	0.132879667377755\\
19.0994	0.159898567911437\\
19.1994	0.193248413394958\\
19.2994	0.261056415516391\\
19.3994	0.355970695277995\\
19.4994	0.422075706027339\\
19.5994	0.450224904184506\\
19.6994	0.512687454202973\\
19.7994	0.568788665396044\\
19.8994	0.617003494855707\\
19.9993	0.600952017685966\\
20.0993	0.585722469940912\\
20.1993	0.525058520068861\\
20.2993	0.531112041034827\\
20.3993	0.507231383899057\\
20.4993	0.519317043373362\\
20.5993	0.528584417540206\\
20.6993	0.520957688224814\\
20.7993	0.533907019412691\\
20.8993	0.526160620599174\\
20.9993	0.540998537596347\\
21.0993	0.61201302398155\\
21.1993	0.627949015038438\\
21.2993	0.662466480924187\\
21.3993	0.667956823546773\\
21.4993	0.682983581979687\\
21.5993	0.690694704669854\\
21.6993	0.672303210884239\\
21.7993	0.645186684424695\\
21.8993	0.642421230955651\\
21.9993	0.685950268804482\\
22.0993	0.806662678358968\\
22.1993	0.859971875810002\\
22.2993	0.914004028410475\\
22.3993	0.82232635804377\\
22.4993	0.753923879551412\\
22.5993	0.765595429593602\\
22.6993	0.79986257603152\\
22.7993	0.881236804727976\\
22.8993	0.813604911730848\\
22.9993	0.82048694957966\\
23.0993	0.885269184665623\\
23.1993	0.955335417792281\\
23.2993	0.973869246745713\\
23.3993	0.977997546140011\\
23.4993	0.925370220349027\\
23.5993	0.90194756715527\\
23.6993	0.894373279128154\\
23.7993	0.83009069494641\\
23.8993	0.782622559935844\\
23.9993	0.835265199157034\\
24.0993	0.911319885201937\\
24.1993	0.927582755073319\\
24.2993	0.935391495527235\\
24.3993	0.912774553924744\\
24.4993	0.872366653834143\\
24.5993	0.861471928869732\\
24.6993	0.898076918889154\\
24.7993	0.834291370650666\\
24.8993	0.761200523099311\\
24.9992	0.849001528224504\\
25.0992	0.923915064976792\\
25.1992	0.927826498862369\\
25.2992	0.913649567999128\\
25.3992	0.846123839098251\\
25.4992	0.874564771130733\\
25.5992	0.910772185175965\\
25.6992	0.891850703557576\\
25.7992	0.854158688319072\\
25.8992	0.866493236933638\\
25.9992	0.939551696685686\\
26.0992	1.03018728948674\\
26.1992	1.06246586913374\\
26.2992	0.966208246373879\\
26.3992	0.890366416299317\\
26.4992	0.917346095232614\\
26.5992	0.919872108139657\\
26.6992	0.922268428035426\\
26.7992	0.87714398821666\\
26.8992	0.920039874925814\\
26.9992	0.98771879366128\\
27.0992	1.02447696683367\\
27.1992	1.04879077760684\\
27.2992	0.955895360409013\\
27.3992	0.888355983124665\\
27.4992	0.886087164687204\\
27.5992	0.894024550884608\\
27.6992	0.888551917008125\\
27.7992	0.866270350527267\\
27.8992	0.860781035345901\\
27.9992	0.907497441929768\\
28.0992	0.945653227279997\\
28.1992	0.992720497824139\\
28.2992	0.94005621947908\\
28.3992	0.888496881988369\\
28.4992	0.923668639015267\\
28.5992	0.967275000488856\\
28.6992	0.998857430928508\\
28.7992	0.897426034651315\\
28.8992	0.822729059252451\\
28.9992	0.806838449708171\\
29.0992	0.778297354272148\\
29.1992	0.730731997967628\\
29.2992	0.769107636274426\\
29.3992	0.714656778772931\\
29.4992	0.702774791587824\\
29.5992	0.668842878104547\\
29.6992	0.62125573721305\\
29.7992	0.469860161413907\\
29.8992	0.397858236875401\\
29.9991	0.447334237451298\\
30.0991	0.519646191786624\\
30.1991	0.549913966157845\\
30.2991	0.529014131491604\\
30.3991	0.546942568973033\\
30.4991	0.566562109865421\\
30.5991	0.527556557566954\\
30.6991	0.426212814003461\\
30.7991	0.366270305585004\\
30.8991	0.460773395324858\\
30.9991	0.562438229749407\\
31.0991	0.600177335387031\\
31.1991	0.585411976565551\\
31.2991	0.600995538337069\\
31.3991	0.566034697704216\\
31.4991	0.558265234589642\\
31.5991	0.447466227266439\\
31.6991	0.349141149171258\\
31.7991	0.345431478755657\\
31.8991	0.442248823322421\\
31.9991	0.534675255234254\\
32.0991	0.560723264133561\\
32.1991	0.58077163714282\\
32.2991	0.605191013405948\\
32.3991	0.609507470492158\\
32.4991	0.610007025985244\\
32.5991	0.565968333221757\\
32.6991	0.509386079592712\\
32.7991	0.537167504740757\\
32.8991	0.627302143507856\\
32.9991	0.744455749813606\\
33.0991	0.81355610683228\\
33.1991	0.849836983987822\\
33.2991	0.810610802739766\\
33.3991	0.827392691420068\\
33.4991	0.859599992962598\\
33.5991	0.928198359725732\\
33.6991	0.935373130341363\\
33.7991	0.814846431206035\\
33.8991	0.680006398015305\\
33.9991	0.651002425612627\\
34.0991	0.62225583441385\\
34.1991	0.664091583644171\\
34.2991	0.638220731613326\\
34.3991	0.619191216081251\\
34.4991	0.637508667603807\\
34.5991	0.650948464765845\\
34.6991	0.624682745286201\\
34.7991	0.526787763138073\\
34.8991	0.409777388675506\\
34.9991	0.443340511999495\\
35.0991	0.534975388949411\\
35.1991	0.602450735824755\\
35.2991	0.562886981058765\\
35.3991	0.605336270904558\\
35.4991	0.656155114347698\\
35.5991	0.634927647348787\\
35.6991	0.487520833688179\\
35.7991	0.386991800912376\\
35.8991	0.368630477645939\\
35.9991	0.464760918271407\\
36.0991	0.553372503222255\\
36.1991	0.549953746800337\\
36.2991	0.564326914772487\\
36.3991	0.62480523625336\\
36.4991	0.639426935742775\\
36.5991	0.614138598532581\\
36.6991	0.502475753343156\\
36.7991	0.422772210362694\\
36.8991	0.46817857482941\\
36.9991	0.557525720201903\\
37.0991	0.60938299148298\\
37.1991	0.595828429237714\\
37.2991	0.592712380739031\\
37.3991	0.61076457923414\\
37.4991	0.62542533549748\\
37.5991	0.528662650975256\\
37.6991	0.443610175677054\\
37.7991	0.395103285633679\\
37.8991	0.470426161566871\\
37.9991	0.543387295836629\\
38.0991	0.544101454475598\\
38.1991	0.505828523090724\\
38.2991	0.510540182047423\\
38.3991	0.553420747246154\\
38.4991	0.592380743247703\\
38.5991	0.542766085698737\\
38.6991	0.46745769408878\\
38.7991	0.477470603814329\\
38.8991	0.513766951367486\\
38.9991	0.584018197839604\\
39.0991	0.595778776804995\\
39.1991	0.578079245950803\\
39.2991	0.590554494579479\\
39.3991	0.621256158949281\\
39.4991	0.643462408913316\\
39.5991	0.54785252513374\\
39.6991	0.448021884681621\\
39.7991	0.490581259563144\\
39.8991	0.530813386799849\\
39.9991	0.550334735272318\\
40.0991	0.536311706372489\\
40.1991	0.527483212052227\\
40.2991	0.581700449404668\\
40.3991	0.596500520872313\\
40.499	0.561875306666139\\
40.5991	0.476057018825433\\
40.6991	0.42358468777511\\
40.7991	0.459829017982186\\
40.8991	0.513677268471487\\
40.9991	0.563831907084187\\
41.0991	0.547851172498502\\
41.1991	0.485892183513057\\
41.2991	0.468951553410682\\
41.399	0.455645169143844\\
41.4991	0.502264766422759\\
41.5991	0.557855726510321\\
41.6991	0.556747303455461\\
41.7991	0.522253009116751\\
41.8991	0.607701547485618\\
41.999	0.721115250837509\\
42.0991	0.854475822394522\\
42.1991	0.995807416219022\\
42.2991	1.09274573015355\\
42.3991	1.07842489716885\\
42.4991	0.992815114546979\\
42.5991	0.92965914232417\\
42.6991	0.941931439044987\\
42.7991	0.920814826618956\\
42.8991	0.837676615648414\\
42.9991	0.628688174123853\\
43.0991	0.492333501569849\\
43.1991	0.466976208486484\\
43.2991	0.420259060878648\\
43.3991	0.349681815234552\\
43.4991	0.245303966618544\\
43.5991	0.148902746888113\\
43.6991	0.0629164113186981\\
43.7991	0.0346358707533536\\
43.8991	-0.00610482502247185\\
43.9991	-0.0745150956337093\\
44.0991	-0.195559825414752\\
44.1991	-0.27325810361658\\
44.2991	-0.329426919911946\\
44.3991	-0.442235605218055\\
44.4991	-0.60199722234737\\
44.5991	-0.706077980650977\\
44.6991	-0.816268863554885\\
44.7991	-0.900750231482366\\
44.8991	-1.0178628315708\\
44.999	-1.16670494283724\\
45.099	-1.24322596893635\\
45.199	-1.30923401842482\\
45.299	-1.27903406212234\\
45.399	-1.18181384915151\\
45.499	-1.1304327888696\\
45.599	-1.13059040487417\\
45.699	-1.04139807186463\\
45.799	-0.995381461978402\\
45.899	-1.03647847675406\\
45.999	-1.09628637689049\\
46.099	-1.12133762280408\\
46.199	-1.06790235088155\\
46.299	-1.01224101158923\\
46.399	-1.06636162339455\\
46.499	-1.19219195942518\\
46.599	-1.28026770585792\\
46.699	-1.41922652595562\\
46.799	-1.49512204857809\\
46.899	-1.52009255093918\\
46.999	-1.51635380944806\\
47.099	-1.48154890831389\\
47.199	-1.40829475612832\\
47.299	-1.3402145114745\\
47.399	-1.29176761562268\\
47.499	-1.21978656115445\\
47.599	-1.16210845059316\\
47.699	-1.10641287625956\\
47.799	-1.13624730492078\\
47.899	-1.21803943245672\\
47.999	-1.25314406768532\\
48.099	-1.25761771870384\\
48.199	-1.24362257573034\\
48.299	-1.29605235256078\\
48.399	-1.39917227451292\\
48.4991	-1.45764939086051\\
48.599	-1.43716750713343\\
48.6991	-1.35782763707859\\
48.7991	-1.3363462866776\\
48.899	-1.42445680661782\\
48.9991	-1.50892435793496\\
49.0991	-1.51454454413074\\
49.1991	-1.48315399111523\\
49.2991	-1.51827671395205\\
49.3991	-1.61711028312768\\
49.499	-1.70834016131135\\
49.5991	-1.74293607776956\\
49.6991	-1.67279670289319\\
49.7991	-1.58009515325389\\
49.8991	-1.53301837653015\\
49.999	-1.54133691014913\\
50.099	-1.5615757390639\\
50.199	-1.51761572217824\\
50.299	-1.43625953517874\\
50.399	-1.38794585563325\\
50.499	-1.37414138850171\\
50.599	-1.36411448416255\\
50.699	-1.28135049974281\\
50.799	-1.20237137778176\\
50.899	-1.26363533574605\\
50.999	-1.45095243348588\\
51.099	-1.62630964452588\\
51.199	-1.69957818610216\\
51.299	-1.72469085409174\\
51.399	-1.74179954688414\\
51.499	-1.79235331278109\\
51.599	-1.79145601678161\\
51.699	-1.69913842841047\\
51.799	-1.53584701710973\\
51.899	-1.37022609704671\\
51.999	-1.25996746736486\\
52.099	-1.25688191213773\\
52.199	-1.29090917287568\\
52.299	-1.28873788894649\\
52.399	-1.3004963708044\\
52.499	-1.37483760204628\\
52.599	-1.49446593669886\\
52.699	-1.53652607853889\\
52.799	-1.50060615009487\\
52.899	-1.43603992181888\\
52.999	-1.42238391106655\\
53.099	-1.47614770596442\\
53.199	-1.48558560318294\\
53.2991	-1.45151791477604\\
53.3991	-1.3909861532807\\
53.4991	-1.38810666888169\\
53.5991	-1.42283554074064\\
53.6991	-1.47959695417801\\
53.7991	-1.4817681596434\\
53.8991	-1.41908595586113\\
53.999	-1.40114287609374\\
54.0991	-1.45858253733288\\
54.1991	-1.52361777576328\\
54.299	-1.53946807351567\\
54.3991	-1.51620252216532\\
54.4991	-1.47009064367713\\
54.599	-1.47396928487667\\
54.6991	-1.50389306883379\\
54.7991	-1.47541766387888\\
54.8991	-1.4191041069886\\
54.999	-1.39173344178277\\
55.099	-1.41923375342473\\
55.199	-1.48593675888954\\
55.299	-1.50806776343609\\
55.399	-1.48927340588864\\
55.499	-1.44919979703568\\
55.599	-1.39988174401292\\
55.699	-1.34025733477428\\
55.799	-1.2609616559541\\
55.899	-1.2031330117133\\
55.999	-1.16781487452003\\
56.099	-1.13539800967042\\
56.199	-1.06765584461675\\
56.299	-0.985642708307431\\
56.399	-0.91180444676518\\
56.499	-0.936165816843602\\
56.599	-0.898530634276868\\
56.699	-0.938837207096189\\
56.799	-0.973584525162681\\
56.899	-0.959660287230952\\
56.999	-1.04480959002629\\
57.099	-1.07905913732277\\
57.199	-1.12379635640692\\
57.299	-1.14327872413987\\
57.399	-1.11408604088286\\
57.4991	-1.21477598703392\\
57.5991	-1.29831998184384\\
57.6991	-1.26417531433493\\
57.7991	-1.34226733664506\\
57.8991	-1.26386656832842\\
57.9991	-1.17951179230682\\
58.0991	-1.08152660147035\\
58.1991	-1.11275629111335\\
58.2991	-1.0252340504194\\
58.399	-0.849408359854228\\
58.4991	-0.719267971359956\\
58.5991	-0.51252528337387\\
58.6991	-0.476131821124194\\
58.7991	-0.594301954881031\\
58.8991	-0.743832028572136\\
58.9991	-0.828731207017876\\
59.0991	-0.874636859195335\\
59.1991	-0.976039660765552\\
59.2991	-1.15443259425466\\
59.3991	-1.14274171735991\\
59.4991	-0.993931261653384\\
59.5991	-1.02386849726982\\
59.6991	-1.1186504660225\\
59.7991	-1.1656752078474\\
59.8991	-1.28111229651263\\
59.999	-1.09698587405346\\
60.099	-1.14733347031991\\
60.199	-1.04777560430836\\
60.299	-1.10939523470458\\
60.399	-1.18928360166023\\
60.499	-1.0192069937753\\
60.599	-1.03157284681434\\
60.699	-1.05099612962859\\
60.799	-1.08301764545934\\
60.899	-1.34019308943307\\
60.999	-1.37300898075399\\
61.099	-1.27563115251885\\
61.199	-1.19435312859281\\
61.299	-1.24636868874518\\
61.399	-1.41422890820339\\
61.499	-1.16850291810604\\
61.599	-1.09645986711425\\
61.699	-1.09192500114905\\
61.799	-1.13394427096917\\
61.899	-1.41292680145955\\
61.999	-1.45924181077676\\
62.099	-1.34768509490271\\
62.199	-1.27215435295541\\
62.299	-1.33882430395775\\
62.399	-1.49189267157479\\
62.499	-1.25608876185892\\
62.599	-1.09135435470571\\
62.699	-1.03089612731444\\
62.799	-1.11448343201565\\
62.899	-1.30621724055898\\
62.999	-1.47768661497996\\
63.099	-1.34337683834527\\
63.199	-1.27326330695029\\
63.299	-1.25404127471887\\
63.399	-1.35217889054351\\
63.499	-1.31451224647227\\
63.599	-1.07654416840459\\
63.699	-1.05984191132563\\
63.799	-1.07416578828368\\
63.8991	-1.23421055783542\\
63.999	-1.51371520118509\\
64.099	-1.43650097415965\\
64.1991	-1.31321925060357\\
64.2991	-1.24708821636519\\
64.3991	-1.36499469322081\\
64.4991	-1.45007923776854\\
64.599	-1.16849841875068\\
64.6991	-1.09203760152541\\
64.7991	-1.09093724809158\\
64.8991	-1.19574257683435\\
64.999	-1.48022484445095\\
65.099	-1.47891226978534\\
65.199	-1.33672256082584\\
65.299	-1.26066439477884\\
65.399	-1.34211271257016\\
65.499	-1.47495498535229\\
65.599	-1.24565344739651\\
65.699	-1.09432155857341\\
65.799	-1.08231304382993\\
65.899	-1.16535046143442\\
65.999	-1.3784848276404\\
66.099	-1.50880849285894\\
66.199	-1.35871780671121\\
66.299	-1.28827101763881\\
66.399	-1.2620673540179\\
66.499	-1.37400075479881\\
66.599	-1.30474846681098\\
66.699	-1.08710186756176\\
66.799	-1.02490166381788\\
66.899	-1.09346618639361\\
66.999	-1.27462794702309\\
67.099	-1.51613369829213\\
67.1991	-1.36348824791032\\
67.299	-1.2999547231683\\
67.3991	-1.15388507051997\\
67.4991	-1.22952525042575\\
67.5991	-1.78139568100099\\
67.6991	-2.22696068852789\\
67.7991	-2.18259383837287\\
67.8991	-1.98711094706425\\
67.9991	-1.68422724412759\\
68.0991	-1.48723614487401\\
68.1991	-1.23867956526315\\
68.2991	-1.07317678375325\\
68.3991	-1.04610492045751\\
68.4991	-1.087761624683\\
68.5991	-1.12129803532731\\
68.6991	-1.11287677581527\\
68.7991	-0.985783381034132\\
68.8991	-0.928216755295122\\
68.9991	-0.962916009038167\\
69.0991	-0.926071429069969\\
69.1991	-0.771443060233726\\
69.2991	-0.557355443699194\\
69.3991	-0.416013972816038\\
69.4991	-0.373986863193733\\
69.5991	-0.236118371664102\\
69.6991	-0.117127206799884\\
69.7991	0.0944459809756322\\
69.8991	0.171910722113127\\
69.999	0.293472652555028\\
70.099	0.447576435542505\\
70.199	0.597462192092489\\
70.2991	0.718744371533613\\
70.3991	0.761004218166966\\
70.4991	0.86184560449531\\
70.599	0.963006931538462\\
70.699	1.04699253287463\\
70.7991	1.07635811942936\\
70.8991	1.13279906018068\\
70.9991	1.15345259473162\\
71.0991	1.23617575207368\\
71.1991	1.27811234591665\\
71.2991	1.32344639729195\\
71.3991	1.20416020190418\\
71.4991	1.17270392506363\\
71.5991	1.26721799244624\\
71.6991	1.38433243886575\\
71.7991	1.45846100570202\\
71.8991	1.44348011129996\\
71.9991	1.40136345555404\\
72.0991	1.40840550226523\\
72.1991	1.43092020371099\\
72.2991	1.46462156101053\\
72.3991	1.40842445123123\\
72.4991	1.27795047769045\\
72.5991	1.22510849937148\\
72.6991	1.27280238917908\\
72.7991	1.29822117203404\\
72.8991	1.24154689747855\\
72.9991	1.11107608313974\\
73.0991	1.01088910805332\\
73.1991	0.923679882714716\\
73.2991	0.91058410546617\\
73.3991	0.863657122362503\\
73.4991	0.841889039224116\\
73.5991	0.780411746392811\\
73.6991	0.761383306981389\\
73.7991	0.740811193263751\\
73.8991	0.747671372484859\\
73.9991	0.785214866323325\\
74.0991	0.803527907527367\\
74.1991	0.743645024522522\\
74.2991	0.705796045531061\\
74.3991	0.717715884021737\\
74.4991	0.765696076203828\\
74.5991	0.782746182174473\\
74.6991	0.749652598457516\\
74.7991	0.71883254716024\\
74.8991	0.753999158472933\\
74.9991	0.7481099428505\\
75.099	0.705132337821686\\
75.1991	0.696903147038222\\
75.2991	0.729456530826614\\
75.3991	0.777303891466835\\
75.4991	0.751674318883489\\
75.5991	0.721369344044876\\
75.6991	0.703647547114001\\
75.7991	0.73576524040132\\
75.8991	0.773717375163059\\
75.9991	0.718353652494329\\
76.0991	0.687029311660018\\
76.1991	0.717657854959943\\
76.2991	0.742464852727379\\
76.3991	0.700790877161022\\
76.4991	0.662320299628981\\
76.5991	0.713124177532482\\
76.6991	0.750911303752641\\
76.7991	0.704069071715715\\
76.8991	0.7210766572999\\
76.9991	0.66390479211313\\
77.0991	0.740534734534393\\
77.1991	0.744017407712974\\
77.2991	0.726315249875278\\
77.3991	0.644780966105938\\
77.4991	0.665636846015476\\
77.5991	0.71000254390639\\
77.6991	0.668758736042878\\
77.7991	0.623157637463396\\
77.8991	0.618412852668828\\
77.9991	0.678642765694063\\
78.0991	0.665452600914437\\
78.1991	0.626808019552062\\
78.2991	0.601987059541631\\
78.3991	0.719200610493693\\
78.4991	0.719080338112742\\
78.5991	0.648942655365346\\
78.6991	0.646009104873073\\
78.7991	0.676359021793742\\
78.8991	0.748999509793982\\
78.9991	0.721855981421636\\
79.0991	0.693125647936055\\
79.1992	0.666869674776237\\
79.2992	0.724000995609315\\
79.3992	0.738889397740727\\
79.4992	0.712094756367528\\
79.5991	0.699562960107694\\
79.6992	0.762701251132405\\
79.7992	0.743926573110876\\
79.8992	0.688630887411197\\
79.9991	0.688803100789954\\
80.0991	0.742024883355623\\
80.1991	0.776765001476212\\
80.2991	0.712731229581161\\
80.3991	0.650988844517859\\
80.4991	0.679618584261492\\
80.5992	0.731377626375575\\
80.6992	0.675888880463048\\
80.7991	0.654259838098755\\
80.8991	0.734214115737162\\
80.9992	0.796364349064956\\
81.0992	0.821633002056918\\
81.1992	0.855133237845688\\
81.2992	0.927306153085364\\
81.3991	1.06275205211028\\
81.4992	1.12523278641559\\
81.5992	1.09090827211465\\
81.6992	1.00684492754556\\
81.7992	0.973422497015308\\
81.8992	0.961445248776323\\
81.9992	0.9606126285997\\
82.0992	0.881875876765984\\
82.1992	0.795667260962278\\
82.2992	0.717988927298185\\
82.3992	0.684668197855934\\
82.4992	0.673900965660097\\
82.5992	0.630323742734046\\
82.6992	0.549938053816634\\
82.7992	0.512831571255377\\
82.8992	0.636790470655928\\
82.9992	0.724251948484611\\
83.0992	0.687532342331566\\
83.1992	0.508056587661125\\
83.2992	0.396368890009715\\
83.3992	0.447009758954797\\
83.4992	0.537366427869068\\
83.5992	0.596360148007891\\
83.6992	0.546644156138381\\
83.7992	0.657550352156958\\
83.8992	0.753891590014189\\
83.9992	0.78205619132006\\
84.0992	0.728933237236095\\
84.1992	0.603095367487415\\
84.2992	0.63030278400237\\
84.3992	0.644760611764871\\
84.4992	0.57429933996752\\
84.5993	0.548222424516989\\
84.6992	0.631081245099471\\
84.7993	0.745893644333294\\
84.8993	0.68542506374484\\
84.9991	0.580640805783992\\
85.0991	0.600529581179463\\
85.1992	0.6576521442926\\
85.2992	0.652420197322546\\
85.3992	0.621818899244723\\
85.4992	0.636091705275287\\
85.5992	0.715558109493281\\
85.6992	0.699486907508112\\
85.7992	0.714789818930886\\
85.8992	0.730616969263007\\
85.9992	0.851510348058524\\
86.0992	0.986264879296691\\
86.1992	0.971988600111188\\
86.2992	0.832828093125778\\
86.3992	0.822660210960601\\
86.4992	0.884385260332864\\
86.5992	0.900074201930412\\
86.6992	0.816818216459752\\
86.7992	0.756439009199572\\
86.8992	0.737954098342472\\
86.9992	0.771306143790307\\
87.0992	0.779182268565608\\
87.1992	0.724053118730255\\
87.2992	0.579288611439964\\
87.3992	0.453613113922538\\
87.4992	0.435116606595106\\
87.5992	0.432205839925428\\
87.6992	0.37281082222657\\
87.7992	0.302330262941644\\
87.8992	0.23183983445257\\
87.9992	0.190163614794213\\
88.0992	0.13258381045468\\
88.1992	0.0237019108803893\\
88.2992	-0.11936299493568\\
88.3993	-0.230502940997985\\
88.4993	-0.341511174999797\\
88.5992	-0.492140069904044\\
88.6993	-0.656629663032107\\
88.7993	-0.735575293254767\\
88.8993	-0.769600692676906\\
88.9993	-0.869724469820794\\
89.0993	-0.971868463738837\\
89.1993	-1.05524089307874\\
89.2993	-1.04848898821498\\
89.3993	-1.03888388212736\\
89.4993	-1.07119699006523\\
89.5993	-1.06195066280575\\
89.6993	-1.03823984417281\\
89.7993	-0.941033530008604\\
89.8993	-0.79772303267703\\
89.9993	-0.688335482273942\\
90.0993	-0.632725366261503\\
90.1993	-0.598304650283832\\
90.2993	-0.431214136058001\\
90.3993	-0.200098788967205\\
90.4993	-0.0271826392492369\\
90.5993	0.0698966918655484\\
90.6993	0.17447587312597\\
90.7993	0.33176047878127\\
90.8993	0.492593449093828\\
90.9993	0.595587976155771\\
91.0993	0.719732007686506\\
91.1993	0.836336553241662\\
91.2993	0.953439032550036\\
91.3993	1.00777350290915\\
91.4993	0.968149232271677\\
91.5993	1.01350371994424\\
91.6993	1.04644457875875\\
91.7993	1.06710555569502\\
91.8993	1.05644503298829\\
91.9993	0.98613874612956\\
92.0993	0.916192381025045\\
92.1993	0.916351630473185\\
92.2993	0.926269511478222\\
92.3993	0.923296067478958\\
92.4993	0.825048015329325\\
92.5993	0.700343977374498\\
92.6993	0.670445931495955\\
92.7993	0.695588188260185\\
92.8993	0.727119323669747\\
92.9994	0.690693395019924\\
93.0993	0.66929688700224\\
93.1994	0.660897469712861\\
93.2994	0.691829475994593\\
93.3994	0.672190839037604\\
93.4994	0.612615873720438\\
93.5994	0.56551430576181\\
93.6994	0.486916539606969\\
93.7994	0.434404655083884\\
93.8994	0.466668932704428\\
93.9994	0.466044944757927\\
94.0994	0.488609784990754\\
94.1994	0.496331184881161\\
94.2994	0.510209825226866\\
94.3994	0.527711735152818\\
94.4994	0.537017592709822\\
94.5994	0.478645592803722\\
94.6994	0.390168721084753\\
94.7994	0.327049246886142\\
94.8994	0.258843641633506\\
94.9993	0.260595239322312\\
95.0993	0.214389882992244\\
95.1993	0.148252228156288\\
95.2994	0.0929780855196625\\
95.3994	0.0945957943831099\\
95.4994	0.0965060708054835\\
95.5994	0.0790958232669277\\
95.6994	0.0471878314135985\\
95.7994	-5.59944456886008e-05\\
95.8994	-0.0447717173993998\\
95.9994	-0.0999143186311344\\
96.0994	-0.10508117334233\\
96.1994	-0.0836798143348463\\
96.2994	-0.0675600048781675\\
96.3994	-0.0238517088284471\\
96.4994	-0.0533173451973159\\
96.5994	-0.0673107915861455\\
96.6994	-0.0402329169969877\\
96.7994	-0.111577977428944\\
96.8994	-0.120415247973749\\
96.9994	-0.155560662582548\\
97.0994	-0.187236107212161\\
97.1994	-0.15543751767005\\
97.2994	-0.133929365146158\\
97.3994	-0.103310189796741\\
97.4994	-0.0801703152535262\\
97.5994	-0.147125315157968\\
97.6994	-0.0295281941952967\\
97.7994	-0.0697485885696252\\
97.8994	-0.0861662157115175\\
97.9994	-0.102656514697123\\
98.0994	-0.116981880203651\\
98.1995	-0.103970401936962\\
98.2995	-0.0567100360713244\\
98.3995	-0.0299206643553314\\
98.4995	0.00168692399010473\\
98.5995	-0.0186458763929549\\
98.6995	-0.00696437400285488\\
98.7995	-0.0441447880945365\\
98.8995	-0.120167532675594\\
98.9995	-0.165589488750126\\
99.0995	-0.1633512973213\\
99.1995	-0.130511177514509\\
99.2995	-0.0743856378447698\\
99.3995	-0.0576852980308282\\
99.4995	-0.0362034065047072\\
99.5995	0.0115168010265032\\
99.6995	0.0246882365010261\\
99.7995	0.00469005382444942\\
99.8995	-0.0116593954623134\\
99.9994	0.016847225576115\\
100.0994	0.0207349754533106\\
100.1995	0.0144919779479897\\
100.2995	-0.00135966684613686\\
100.3995	-0.00551453893364195\\
100.4995	0.00285985114384602\\
100.5995	0.00516668511156171\\
100.6995	0.0128284155935257\\
100.7995	-4.85856557647345e-05\\
100.8995	-0.000338350847249758\\
100.9995	-4.43439777210106e-05\\
101.0995	-0.000195095553275592\\
101.1995	-0.000324928774404683\\
101.2995	-0.000227252816059939\\
101.3995	6.93259228507015e-05\\
101.4995	6.87210501882996e-05\\
101.5995	-7.45142224478103e-05\\
101.6995	0.000491182938613784\\
101.7995	0.000402823325705249\\
101.8995	0.000169612189749256\\
101.9995	-7.82138997858194e-05\\
102.0995	8.93689331093429e-08\\
102.1995	-0.000126491529956857\\
102.2995	-0.000110292928393516\\
102.3995	2.14518685202072e-05\\
102.4995	0.000231709850595219\\
102.5995	0.000217287638428499\\
102.6995	-6.16935435716025e-07\\
102.7995	0.000163513964943489\\
102.8995	-3.9645823780278e-05\\
102.9995	0.000151554592182268\\
103.0995	0.000220926270800066\\
103.1996	5.29563381027554e-05\\
103.2996	-0.000218405481893721\\
103.3996	-3.48837767480143e-05\\
103.4996	8.4809325354469e-05\\
103.5996	6.39116119115013e-05\\
103.6996	0.000135632351320912\\
103.7996	0.000120925183953393\\
103.8996	0.000170902408298182\\
103.9996	4.3888157729031e-05\\
104.0996	-0.000219444507684436\\
104.1996	-0.00111693702375488\\
104.2996	-0.00640430971976998\\
104.3996	0.00373978782088415\\
104.4996	0.0100187386442392\\
104.5996	-0.139712598640242\\
104.6996	-0.0669720725163047\\
104.7996	0.0172004905621125\\
104.8996	0.0291633044106776\\
104.9996	-0.00840434764753189\\
105.0996	-0.0643200872611749\\
105.1996	-0.0372239051432827\\
105.2996	-0.0156565853027432\\
105.3996	-0.0199173425203881\\
105.4996	-0.0174304304595651\\
105.5996	-0.0139574038743608\\
105.6996	-0.0106563696490982\\
105.7996	-0.00779985232316614\\
105.8996	-0.00505842301395139\\
105.9996	-0.00277059941612609\\
106.0996	-0.00105101573138522\\
106.1996	-0.00029868744837686\\
106.2996	-0.000326863889826441\\
106.3996	6.90643015808682e-05\\
106.4996	0.000207677619448931\\
106.5996	0.000219791306713094\\
106.6996	1.27464302489383e-05\\
106.7996	0.00060866388961998\\
106.8996	0.000313349779301997\\
106.9996	0.000334476914115466\\
107.0996	0.000114954207350789\\
107.1996	0.000154423527384105\\
107.2996	-3.34333230533924e-05\\
107.3996	0.000473277321903015\\
107.4996	0.000505975021535084\\
107.5996	0.000196983113335759\\
107.6996	-5.27462148307295e-05\\
107.7996	0.000213345880604065\\
107.8996	-9.78545348373136e-05\\
107.9996	-0.000117383624446486\\
108.0996	1.84461900900377e-06\\
108.1997	-0.000331397465274787\\
108.2996	-0.00023770401461164\\
108.3997	1.61294983185044e-05\\
108.4997	-0.000282963107709452\\
108.5997	-0.000223385969624491\\
108.6997	-0.000178431461733335\\
108.7997	-7.43943637968435e-05\\
108.8997	-0.000121390336935488\\
108.9997	-0.000101239702830842\\
109.0997	-0.000290287511286911\\
109.1997	-0.000225777159628269\\
109.2997	-0.000234058412211639\\
109.3997	0.000220492132881266\\
109.4997	0.000456604709333095\\
109.5997	0.000228297475011215\\
109.6997	7.9908670619933e-05\\
109.7997	0.000527827828669374\\
109.8997	0.00052512951654965\\
109.9996	-4.1535214714385e-05\\
110.0996	0.000261589992498069\\
110.1996	0.000299343898739121\\
110.2996	0.000183181152308528\\
110.3996	-0.00105353593387927\\
110.4996	-0.00155133897125389\\
110.5996	-0.0143213952415844\\
110.6996	-0.010816826557705\\
110.7997	-0.0161067813992174\\
110.8997	-0.0180235386398327\\
110.9997	-0.0195675299271177\\
111.0997	0.0347499205846222\\
111.1997	0.0403486291990428\\
111.2997	0.0392952167614624\\
111.3997	0.0817741870031223\\
111.4997	0.120519729885519\\
111.5997	0.132246085753861\\
111.6997	0.0994952031254168\\
111.7997	0.0555132421564458\\
111.8997	0.0281441011852405\\
111.9997	-0.0561948975917611\\
112.0997	-0.0323208226230794\\
112.1997	0.00472805752745826\\
112.2997	-0.00930924859887429\\
112.3997	-0.0353296818363878\\
112.4997	-0.029524456615643\\
112.5997	-0.00883089177161865\\
112.6997	-0.0213190258270368\\
112.7997	-0.018867019653861\\
112.8997	-0.0131872543729318\\
112.9997	-0.00261248826544736\\
113.0997	-0.000308916504540867\\
113.1997	0.00439690483377425\\
113.2997	0.00407911192297288\\
113.3997	8.35125972044859e-07\\
113.4997	-0.000192017569261287\\
113.5997	-0.000305539763356643\\
113.6997	0.000178170728975314\\
113.7998	6.46719115256159e-05\\
113.8998	-0.000138239328088987\\
113.9998	6.0129663390487e-05\\
114.0998	0.000127311896370965\\
114.1998	-0.000203804597984714\\
114.2998	-0.000360788623738954\\
114.3998	-0.000153971213796604\\
114.4998	0.000182439149320865\\
114.5998	-0.000407573436906731\\
114.6998	-0.00038151444703931\\
114.7998	-0.000148937796726006\\
114.8998	-0.000273117106601441\\
114.9997	-0.000120085914150272\\
115.0997	-0.00062763108194039\\
115.1997	-0.000806648456212222\\
115.2997	-0.00133318131010294\\
115.3997	-0.00125967158998725\\
115.4997	-0.000759787947226668\\
115.5997	-0.000571180316981886\\
115.6997	-0.000240961540665495\\
115.7998	-0.000137569794890854\\
115.8997	-0.000171617522944477\\
115.9997	-0.000160299143528096\\
116.0998	-0.000297934748336578\\
116.1998	-0.000517059730151094\\
116.2998	-0.00117257457518323\\
116.3998	-0.000590954191171709\\
116.4998	-0.000825080391353728\\
116.5998	-0.00058191320752933\\
116.6998	-0.000490922637958453\\
116.7998	-0.000707500506878903\\
116.8998	-0.000630546274011008\\
116.9998	0.00014478859393801\\
117.0998	-0.000139911369796891\\
117.1998	-0.000469317120745347\\
117.2998	8.70130456596119e-05\\
117.3998	-1.30496420236315e-05\\
117.4998	-0.000319151055239009\\
117.5998	-0.000193431327671576\\
117.6998	1.25695027452897e-05\\
117.7998	0.000259460437151285\\
117.8998	-0.000194755562863428\\
117.9998	3.38098205007744e-05\\
118.0998	6.62525992818706e-05\\
118.1998	-0.000143794756024191\\
118.2998	-0.000285869879063088\\
118.3998	-0.000164565995865839\\
118.4998	2.54973572525817e-06\\
118.5998	0.000233476533777839\\
118.6998	9.39832837838638e-05\\
118.7998	8.9468341134906e-05\\
118.8998	-9.67326980623884e-05\\
118.9998	-8.6843317588073e-05\\
119.0999	-7.19570204274356e-05\\
119.1999	0.000158347402191528\\
119.2999	-4.1144445796332e-05\\
119.3999	9.6210740177584e-08\\
119.4998	1.81459306257594e-05\\
119.5999	-0.000245790858353881\\
119.6999	8.09484071283163e-05\\
119.7999	-0.00037154951360327\\
};
\addplot [color=mycolor5,solid,forget plot]
  table[row sep=crcr]{%
0	-0.00012245986462566\\
0.1	-0.00733871326655438\\
0.2	0.0111741823101223\\
0.3	0.0101682244456489\\
0.4	8.00316905606811e-05\\
0.5	0.000217985382884468\\
0.6	-0.000208068066901399\\
0.7	0.000297067801510178\\
0.8	-0.000187662401783613\\
0.9	2.91583412332554e-05\\
1	-0.000450182024869041\\
1.0999	-0.000485681585222412\\
1.1999	-0.000244768381089882\\
1.2999	0.000253502203808573\\
1.3999	0.000182617675150396\\
1.4999	-5.2697329798144e-06\\
1.5999	-8.4116474174078e-05\\
1.6999	-0.000400003713529935\\
1.7999	0.00019367291427908\\
1.8999	-0.000361271102280187\\
1.9999	-0.0011145303677371\\
2.0999	-0.000554481429692633\\
2.1999	-0.000735935253966247\\
2.2999	-0.0010552319413747\\
2.3999	-0.00129551290982691\\
2.4999	-0.000757967912672652\\
2.5999	1.75530385895292e-05\\
2.6999	-0.000256543965518702\\
2.7999	-0.000151207652715692\\
2.8999	-2.20763115273192e-07\\
2.9999	-8.03650751660232e-05\\
3.0999	-0.00029630914725341\\
3.1999	0.000143287337028481\\
3.2999	-0.000264081279126634\\
3.3999	-0.000196357575756349\\
3.4999	-0.000945769042658393\\
3.5999	-0.00132164937813117\\
3.6999	-0.00132932155820365\\
3.7999	-0.00116205318677883\\
3.8999	-0.000796144514774869\\
3.9999	-1.4213177066376e-06\\
4.0999	-0.000450126405191224\\
4.1999	-9.27712029650433e-05\\
4.2999	-0.000185382899212233\\
4.3999	-9.22225524789277e-05\\
4.4999	-0.00016605239396082\\
4.5999	-0.0038747219102929\\
4.6999	-0.00772867430956902\\
4.7999	-0.0120953280724679\\
4.8998	-0.0168369390980655\\
4.9998	-0.0209729502129498\\
5.0998	-0.0233004353269271\\
5.1998	-0.0264972413164198\\
5.2998	-0.0304475153253326\\
5.3998	-0.0345254981635643\\
5.4997	-0.038100470719069\\
5.5997	-0.0417326198503954\\
5.6998	-0.0460300145793778\\
5.7998	-0.050291313618791\\
5.8998	-0.0537021846627537\\
5.9998	-0.0568576614401115\\
6.0997	-0.0608905787457922\\
6.1997	-0.0602749033589711\\
6.2997	-0.0678149081011377\\
6.3997	-0.157585685155238\\
6.4997	-0.290900464479703\\
6.5997	-0.415009303026131\\
6.6997	-0.412442385870612\\
6.7997	-0.17511427481706\\
6.8997	0.0884265886453184\\
6.9997	0.153822886152315\\
7.0997	0.145374572954698\\
7.1997	0.137369718973636\\
7.2997	0.143763406529402\\
7.3997	0.217465275947574\\
7.4997	0.106174013602993\\
7.5997	0.00121851354333952\\
7.6997	0.00332706357898496\\
7.7997	0.0681244485429064\\
7.8997	0.0463969014986787\\
7.9997	0.00697128353859837\\
8.0997	0.00996240449634346\\
8.1997	0.0407335162406039\\
8.2997	0.0169631405243931\\
8.3997	-0.00592259710962006\\
8.4997	-0.00393359972470429\\
8.5997	-8.75908009945747e-05\\
8.6997	9.59254934445198e-06\\
8.7997	-0.000193081593583897\\
8.8997	1.61741459574328e-05\\
8.9997	0.000175658543123687\\
9.0997	8.0020556038592e-06\\
9.1997	0.000327526508579486\\
9.2997	0.000375206448611131\\
9.3997	9.57957901915581e-05\\
9.4997	8.53375381896929e-05\\
9.5997	0.000272260106743899\\
9.6997	0.000357177486794082\\
9.7997	-0.00029800747207845\\
9.8997	-4.44856193325903e-05\\
9.9996	-3.47084925492128e-05\\
10.0996	-7.84274104928648e-05\\
10.1996	-9.62892333304067e-05\\
10.2996	0.000190734051623756\\
10.3996	-0.000450148237703925\\
10.4996	-0.000713918519938188\\
10.5996	-0.00013560254277683\\
10.6996	-0.000119827449851719\\
10.7996	0.000202746602530088\\
10.8996	1.68981896711434e-05\\
10.9996	0.000508805200281203\\
11.0996	0.000386293603877703\\
11.1996	0.000910574616434341\\
11.2996	0.000728603884587513\\
11.3996	-8.62033471922389e-07\\
11.4995	7.38595735243549e-05\\
11.5995	0.000132979806528181\\
11.6996	-2.18881739750614e-05\\
11.7996	1.53592537046083e-05\\
11.8995	9.50492523214807e-05\\
11.9996	-0.000328315536804855\\
12.0995	-7.13688131960759e-05\\
12.1995	-5.83044688000589e-05\\
12.2996	-0.000121157704465141\\
12.3995	-4.2415159527485e-05\\
12.4995	5.08196283714831e-05\\
12.5995	5.11956728288894e-05\\
12.6995	1.71683782786245e-05\\
12.7995	0.00148279280862984\\
12.8995	0.00347263947635272\\
12.9995	0.00500542890305275\\
13.0995	0.00677021137377666\\
13.1995	0.00912048397718394\\
13.2995	0.0117989067742868\\
13.3995	0.0134797039196958\\
13.4995	-0.00662458752438212\\
13.5995	-0.0775344818216897\\
13.6995	-0.0646081348001932\\
13.7995	-0.00614210161251948\\
13.8995	0.0713947108852099\\
13.9995	0.085702219239074\\
14.0995	0.0682169113896556\\
14.1995	0.012085564178458\\
14.2995	0.0167889587089386\\
14.3995	0.00333046477848891\\
14.4995	-0.000412994909047004\\
14.5995	-0.000606909578094327\\
14.6995	1.72512953827567e-05\\
14.7995	-7.91824544624283e-05\\
14.8995	-0.000113208394582286\\
14.9994	1.46875725686303e-05\\
15.0994	9.66041235236017e-05\\
15.1994	0.00037302054167943\\
15.2994	0.000218067208161622\\
15.3994	-0.00026292273102836\\
15.4994	-0.000700068872300848\\
15.5994	-0.000468984965868706\\
15.6994	3.40351625931498e-05\\
15.7994	0.000202945266243775\\
15.8994	9.44246246950606e-05\\
15.9994	-0.00013727904642924\\
16.0994	0.000295699253856431\\
16.1994	0.000409413211799895\\
16.2994	3.05095995074182e-05\\
16.3994	-0.000245512003114645\\
16.4994	-0.000410168627128409\\
16.5994	-0.000955969727078768\\
16.6994	5.12627814181913e-05\\
16.7994	1.12334478868743e-05\\
16.8994	0.000182665391047258\\
16.9994	0.000169588621448419\\
17.0994	0.000309037699088872\\
17.1994	0.00031857103086958\\
17.2994	-7.63639775621255e-05\\
17.3994	-0.00050666916891149\\
17.4994	-0.00107138363267849\\
17.5994	-0.00117683925149342\\
17.6994	-0.00120586224269829\\
17.7994	-0.000947968216247985\\
17.8994	-0.000789096297872397\\
17.9994	-0.00028904579362478\\
18.0994	2.15369404155466e-06\\
18.1994	-8.3848865826335e-05\\
18.2994	0.000321441911705667\\
18.3994	5.2196058181857e-05\\
18.4994	0.00674018898874792\\
18.5994	0.00963107428990942\\
18.6994	0.0156147536051053\\
18.7994	0.00496849276673958\\
18.8994	-0.0379079930175205\\
18.9994	-0.0971481472572604\\
19.0994	-0.146355573629167\\
19.1994	-0.240730627090941\\
19.2994	-0.317254818265384\\
19.3994	-0.364068589272648\\
19.4994	-0.456008962532457\\
19.5994	-0.571412337379109\\
19.6994	-0.706645444418111\\
19.7994	-0.753290534514912\\
19.8994	-0.780692406142053\\
19.9993	-0.690722670552819\\
20.0993	-0.702983516780463\\
20.1993	-0.677514398346081\\
20.2993	-0.68396130218559\\
20.3993	-0.715333627025786\\
20.4993	-0.826033927999257\\
20.5993	-0.856133730673682\\
20.6993	-0.858595105993079\\
20.7993	-0.838894245513024\\
20.8993	-0.844882090857481\\
20.9993	-0.907900548452689\\
21.0993	-0.977314312693779\\
21.1993	-0.999135932274145\\
21.2993	-1.01982508567572\\
21.3993	-1.04444503989258\\
21.4993	-1.10530180613475\\
21.5993	-1.19632394148026\\
21.6993	-1.20759414176396\\
21.7993	-1.16705463585101\\
21.8993	-1.0832772192893\\
21.9993	-1.06251518080734\\
22.0993	-1.13620034079011\\
22.1993	-1.21701298868134\\
22.2993	-1.24950690911237\\
22.3993	-1.23034176989462\\
22.4993	-1.27867557490891\\
22.5993	-1.31325693314349\\
22.6993	-1.32494117524076\\
22.7993	-1.27869436337168\\
22.8993	-1.12948936337539\\
22.9993	-1.10261131351718\\
23.0993	-1.1895624385048\\
23.1993	-1.23612441357203\\
23.2993	-1.24488448721497\\
23.3993	-1.21937514452292\\
23.4993	-1.24041106650463\\
23.5993	-1.29795173128918\\
23.6993	-1.33433132231021\\
23.7993	-1.27088691057419\\
23.8993	-1.21219977658576\\
23.9993	-1.20580330801581\\
24.0993	-1.28898216288646\\
24.1993	-1.330841173597\\
24.2993	-1.3244392797477\\
24.3993	-1.33639739806413\\
24.4993	-1.35756898563117\\
24.5993	-1.39741867644023\\
24.6993	-1.37207651211441\\
24.7993	-1.27044161490424\\
24.8993	-1.15764189127376\\
24.9992	-1.19561305791988\\
25.0992	-1.27235307094052\\
25.1992	-1.31615977365479\\
25.2992	-1.31947405481064\\
25.3992	-1.27534611661075\\
25.4992	-1.28935139784511\\
25.5992	-1.30519501359136\\
25.6992	-1.27818954416024\\
25.7992	-1.18993303088554\\
25.8992	-1.15604894811712\\
25.9992	-1.17868056772371\\
26.0992	-1.19863154345851\\
26.1992	-1.17717777986204\\
26.2992	-1.15800510890591\\
26.3992	-1.15925875195143\\
26.4992	-1.19650990889923\\
26.5992	-1.21524941318343\\
26.6992	-1.17751481322875\\
26.7992	-1.09415677054311\\
26.8992	-1.0435351233975\\
26.9992	-1.08856572694491\\
27.0992	-1.1293153689467\\
27.1992	-1.13421802345432\\
27.2992	-1.09968006320709\\
27.3992	-1.09642911283497\\
27.4992	-1.14150732920169\\
27.5992	-1.1760108546294\\
27.6992	-1.15824912370738\\
27.7992	-1.04477181727837\\
27.8992	-0.966097530981795\\
27.9992	-0.994131437153012\\
28.0992	-1.02308026343074\\
28.1992	-0.987882800327922\\
28.2992	-0.886776134754038\\
28.3992	-0.83006324946605\\
28.4992	-0.861024704768018\\
28.5992	-0.895949647481183\\
28.6992	-0.864218976174625\\
28.7992	-0.630592912882654\\
28.8992	-0.378214037044017\\
28.9992	-0.228436731045987\\
29.0992	-0.0921079625663234\\
29.1992	-0.0149439112266451\\
29.2992	-0.00587150039871809\\
29.3992	-0.0490928063093358\\
29.4992	-0.0311519781558621\\
29.5992	-0.0281023737118778\\
29.6992	0.0388934422358047\\
29.7992	0.0594229708639126\\
29.8992	0.0884470286687552\\
29.9991	0.125599179602909\\
30.0991	0.137950685061971\\
30.1991	0.0766716467776432\\
30.2991	-0.00130313430658791\\
30.3991	0.00700204009987204\\
30.4991	0.00693700711948853\\
30.5991	0.0334422993674353\\
30.6991	0.0462833737469078\\
30.7991	0.0743208011335038\\
30.8991	0.131705179030715\\
30.9991	0.142779470074498\\
31.0991	0.108625256035096\\
31.1991	0.0452895118513655\\
31.2991	0.00160007870841838\\
31.3991	-0.000921342212975684\\
31.4991	0.0102442311314417\\
31.5991	0.0186261664643378\\
31.6991	0.0198218362111043\\
31.7991	0.0694211794200497\\
31.8991	0.108194381856449\\
31.9991	0.138426793987333\\
32.0991	0.0885448314629119\\
32.1991	0.011481295607864\\
32.2991	-0.00284060550271514\\
32.3991	-0.039533411555123\\
32.4991	-0.038087511013951\\
32.5991	-0.0148436190419003\\
32.6991	-0.00718666837483661\\
32.7991	0.0635610595394379\\
32.8991	0.111255321315111\\
32.9991	0.145865950728491\\
33.0991	0.12627837180523\\
33.1991	0.128806919027058\\
33.2991	0.0771349819266252\\
33.3991	0.0135749916543332\\
33.4991	-0.0445930931859093\\
33.5991	-0.00720578996741494\\
33.6991	0.00454491436364446\\
33.7991	9.19030592525516e-05\\
33.8991	0.047358459404073\\
33.9991	0.120320415186101\\
34.0991	0.10244644345562\\
34.1991	0.116096346446424\\
34.2991	0.0721614270495036\\
34.3991	-0.0276570453172683\\
34.4991	-0.0592718986083778\\
34.5991	-0.0892878040155612\\
34.6991	-0.0506764551126555\\
34.7991	0.005122917734524\\
34.8991	0.0196409466053933\\
34.9991	0.0818300625404045\\
35.0991	0.0981295074228619\\
35.1991	0.0729862172772553\\
35.2991	-0.00821028603189222\\
35.3991	-0.085687318968948\\
35.4991	-0.087709707559253\\
35.5991	-0.0663079901697009\\
35.6991	-0.0342187056906389\\
35.7991	-0.0201007050815494\\
35.8991	0.0171033348781309\\
35.9991	0.0587057879137789\\
36.0991	0.0691394549874265\\
36.1991	0.0125243762688105\\
36.2991	-0.0411006402962091\\
36.3991	-0.0404407258711266\\
36.4991	-0.0638337468859858\\
36.5991	-0.0671053809194191\\
36.6991	-0.0550685865262532\\
36.7991	-0.0457967025979065\\
36.8991	0.0167227866174788\\
36.9991	0.0576133530006759\\
37.0991	0.0771680729606257\\
37.1991	0.0638863144368191\\
37.2991	-0.039915950309082\\
37.3991	-0.0921532437904125\\
37.4991	-0.118701931089721\\
37.5991	-0.127919545525683\\
37.6991	-0.0887283318563254\\
37.7991	-0.0166092653636603\\
37.8991	0.0304974653702303\\
37.9991	0.0427369894157175\\
38.0991	0.018294181581207\\
38.1991	-0.0385790558936945\\
38.2991	-0.105390647737558\\
38.3991	-0.116482925582626\\
38.4991	-0.113001408648005\\
38.5991	-0.102617136314503\\
38.6991	-0.0962227270394989\\
38.7991	-0.0303145060512204\\
38.8991	-0.00379509375694131\\
38.9991	0.0414866932843965\\
39.0991	0.0200900564410715\\
39.1991	-0.0504071238337325\\
39.2991	-0.105352851943905\\
39.3991	-0.132573086990023\\
39.4991	-0.117350646301123\\
39.5991	-0.103578509852181\\
39.6991	-0.0738366958227721\\
39.7991	-0.00568936080281035\\
39.8991	-0.0151729259931201\\
39.9991	-0.0185950646725976\\
40.0991	-0.0559286920206972\\
40.1991	-0.128240758414006\\
40.2991	-0.124341950719663\\
40.3991	-0.148736704776792\\
40.499	-0.135109988798704\\
40.5991	-0.119124683646435\\
40.6991	-0.0780074213491193\\
40.7991	-0.0430049876456162\\
40.8991	-0.045372290048135\\
40.9991	-0.0505120265755916\\
41.0991	-0.0886268492182218\\
41.1991	-0.186387909275088\\
41.2991	-0.189957813406936\\
41.399	-0.190383671208048\\
41.4991	-0.18389467453237\\
41.5991	-0.153905218751635\\
41.6991	-0.162828522897462\\
41.7991	-0.0864952711298067\\
41.8991	-0.0346712391410925\\
41.999	0.000349446465500947\\
42.0991	0.0189166019282896\\
42.1991	0.101994567529793\\
42.2991	0.254205191066877\\
42.3991	0.343858838162924\\
42.4991	0.395931780902161\\
42.5991	0.464978296073429\\
42.6991	0.52439641791122\\
42.7991	0.609432693404145\\
42.8991	0.715239391895904\\
42.9991	0.779849264209782\\
43.0991	0.858300939178817\\
43.1991	0.920883066331358\\
43.2991	0.952104426949405\\
43.3991	0.976823059414162\\
43.4991	0.929826904349745\\
43.5991	0.919902362291046\\
43.6991	0.906093334594918\\
43.7991	0.985788873166441\\
43.8991	1.01033969623428\\
43.9991	1.02168027962944\\
44.0991	0.905632773561258\\
44.1991	0.859621361805272\\
44.2991	0.920263560496007\\
44.3991	0.975829931508068\\
44.4991	0.934808022842734\\
44.5991	0.833471463495832\\
44.6991	0.795466383188493\\
44.7991	0.782441904084999\\
44.8991	0.814777447856992\\
44.999	0.716829855663352\\
45.099	0.441571600809887\\
45.199	0.240245629626815\\
45.299	0.0378703532064599\\
45.399	-0.0962332872889552\\
45.499	-0.122600293580532\\
45.599	-0.143214799550952\\
45.699	-0.0979921372621694\\
45.799	-0.0444700493097607\\
45.899	0.0432920657220701\\
45.999	0.0818704889203854\\
46.099	0.105129894455487\\
46.199	0.0286750667257802\\
46.299	-0.121898391574538\\
46.399	-0.108700551035198\\
46.499	-0.152889191336232\\
46.599	-0.197639622127193\\
46.699	-0.253352923250182\\
46.799	-0.225148240357786\\
46.899	-0.162341298956094\\
46.999	-0.11497534937962\\
47.099	-0.0921563482798371\\
47.199	-0.126680951178235\\
47.299	-0.242169956177381\\
47.399	-0.298764967038107\\
47.499	-0.351661726122668\\
47.599	-0.389258893259521\\
47.699	-0.406897247256321\\
47.799	-0.455802643918862\\
47.899	-0.444467080997297\\
47.999	-0.399406766160455\\
48.099	-0.364718750591621\\
48.199	-0.321631633362094\\
48.299	-0.3003787278853\\
48.399	-0.268781704744547\\
48.4991	-0.244337136339301\\
48.599	-0.232758519155712\\
48.6991	-0.230492776313534\\
48.7991	-0.2416601248826\\
48.899	-0.262646964135742\\
48.9991	-0.280060792653076\\
49.0991	-0.257536078261942\\
49.1991	-0.211516994838155\\
49.2991	-0.159716861839745\\
49.3991	-0.121322734926741\\
49.499	-0.132224229991772\\
49.5991	-0.125724995822978\\
49.6991	-0.142716875612212\\
49.7991	-0.141994992914342\\
49.8991	-0.162846063594126\\
49.999	-0.178879744052989\\
50.099	-0.185684005973439\\
50.199	-0.15157045742799\\
50.299	-0.0776782610703612\\
50.399	-0.0149851446001596\\
50.499	0.0360454265623549\\
50.599	0.0662609207250433\\
50.699	0.0982772139839629\\
50.799	0.114436499060798\\
50.899	0.0576731232912853\\
50.999	-0.0238847032055585\\
51.099	-0.141737163638647\\
51.199	-0.191776245656668\\
51.299	-0.225634399631543\\
51.399	-0.246021900946513\\
51.499	-0.200058082645842\\
51.599	-0.15644640404092\\
51.699	-0.0495541362778038\\
51.799	0.0175216854286915\\
51.899	0.0625822478429126\\
51.999	0.0702523344143882\\
52.099	0.0353575823306032\\
52.199	-0.00561838249100655\\
52.299	-0.0173943213753809\\
52.399	-0.0191573540207326\\
52.499	-0.0257507642920718\\
52.599	-0.056205688370393\\
52.699	-0.0855647727979396\\
52.799	-0.0827654101378266\\
52.899	-0.0987896554217176\\
52.999	-0.130113648887577\\
53.099	-0.150198804108473\\
53.199	-0.129136518863444\\
53.2991	-0.0786490959675694\\
53.3991	-0.0262718815393268\\
53.4991	0.0351792732775573\\
53.5991	0.0960517996666006\\
53.6991	0.105891733325767\\
53.7991	0.105470046018822\\
53.8991	0.0933377375555238\\
53.999	0.039139637569634\\
54.0991	-0.020548583174075\\
54.1991	-0.0752457261577484\\
54.299	-0.0795898822695058\\
54.3991	-0.0761526754504973\\
54.4991	-0.0351705730917642\\
54.599	0.0380958420577255\\
54.6991	0.0789691408180277\\
54.7991	0.127543331578154\\
54.8991	0.127040256909776\\
54.999	0.0750108135501568\\
55.099	0.00743096006310484\\
55.199	-0.0521295944411833\\
55.299	-0.0509340670465512\\
55.399	-0.029085990082534\\
55.499	-0.0173974433476367\\
55.599	0.0567961416037863\\
55.699	0.125254515989653\\
55.799	0.236267359309729\\
55.899	0.281856610237371\\
55.999	0.280736736093695\\
56.099	0.219717581544631\\
56.199	0.102691327990548\\
56.299	-0.0374046148309124\\
56.399	-0.111176544638434\\
56.499	-0.170234168808827\\
56.599	-0.186491504230073\\
56.699	-0.214433523364737\\
56.799	-0.206526179709485\\
56.899	-0.232669272274755\\
56.999	-0.187132150256482\\
57.099	0.0728546658867151\\
57.199	0.0150765631612138\\
57.299	0.0282989697205207\\
57.399	0.12924729296631\\
57.4991	0.119231143732074\\
57.5991	0.178979265351856\\
57.6991	0.140499489592236\\
57.7991	0.0913769855218565\\
57.8991	0.0314550822338258\\
57.9991	-0.0175500756920552\\
58.0991	-0.0716945339597577\\
58.1991	-0.148936937922991\\
58.2991	-0.138461518598132\\
58.399	-0.102909005443353\\
58.4991	-0.0627278524212969\\
58.5991	0.0279609188964876\\
58.6991	0.215801376837216\\
58.7991	0.347419307661212\\
58.8991	0.358476588216114\\
58.9991	0.279709006636843\\
59.0991	0.247422222934171\\
59.1991	0.167148360600452\\
59.2991	0.0990204561010768\\
59.3991	0.104647562348659\\
59.4991	0.0284572481076366\\
59.5991	0.108408270212025\\
59.6991	0.153600454553817\\
59.7991	0.292524747393492\\
59.8991	0.277857313398746\\
59.999	0.101977807555816\\
60.099	0.107295296712714\\
60.199	0.0498574742653459\\
60.299	0.0682857685648552\\
60.399	0.0692086945956687\\
60.499	0.0467899696868543\\
60.599	0.0536113350612286\\
60.699	0.0475205584075553\\
60.799	0.123206603952976\\
60.899	0.28884980880805\\
60.999	0.206882765423276\\
61.099	0.159203381080975\\
61.199	0.107130562935181\\
61.299	0.0553479463717674\\
61.399	0.0956140478444731\\
61.499	-0.013055944610497\\
61.599	0.0446053408192393\\
61.699	0.0495179771552645\\
61.799	0.0650822977403362\\
61.899	0.219295430614491\\
61.999	0.193724739071667\\
62.099	0.109567439892003\\
62.199	0.059626635032433\\
62.299	0.0280023825804161\\
62.399	0.00728985902490598\\
62.499	-0.0302396326794598\\
62.599	-0.0680578866822379\\
62.699	-0.0741418888195393\\
62.799	-0.073127649736425\\
62.899	-0.0142707279155583\\
62.999	0.0158699004088851\\
63.099	-0.00760174783363254\\
63.199	-0.0643134274333952\\
63.299	-0.0657712083524231\\
63.399	-0.0924366509431653\\
63.499	-0.0771665137551265\\
63.599	-0.0886385061604971\\
63.699	-0.0694669631543521\\
63.799	-0.0576711011867284\\
63.8991	0.0222975523882845\\
63.999	0.104327187940847\\
64.099	0.0829537607764439\\
64.1991	0.0216131892158273\\
64.2991	-0.0105861024619465\\
64.3991	-0.0407918812568382\\
64.4991	-0.011077787310569\\
64.599	-0.0396242384664895\\
64.6991	-0.0264677158287752\\
64.7991	-0.0333982443877274\\
64.8991	-0.0118060522411092\\
64.999	0.0671328693129798\\
65.099	0.0297443090395717\\
65.199	-0.00501407722485547\\
65.299	-0.0439270034322705\\
65.399	-0.0845934994262345\\
65.499	-0.0633451528382842\\
65.599	-0.0620627858818832\\
65.699	-0.0547136784526199\\
65.799	-0.0690657988675336\\
65.899	-0.0465719931800872\\
65.999	0.0407412754548928\\
66.099	0.0665215807621675\\
66.199	0.0368070798225206\\
66.299	-0.0134662053234575\\
66.399	-0.0511378267163001\\
66.499	-0.0530762285038575\\
66.599	-0.0458841990372227\\
66.699	-0.0788737113160334\\
66.799	-0.0702525924218054\\
66.899	-0.0360576381614339\\
66.999	0.0264720592773529\\
67.099	0.104201507373237\\
67.1991	-0.0031513704769468\\
67.299	0.0398344788663847\\
67.3991	-0.0880505539212133\\
67.4991	-0.195862655054573\\
67.5991	-0.113058294567363\\
67.6991	-0.216750353214354\\
67.7991	-0.0986277927912242\\
67.8991	-0.0484488534097482\\
67.9991	0.0460347824627312\\
68.0991	0.00489898223124685\\
68.1991	-0.103866260267367\\
68.2991	-0.243807334862166\\
68.3991	-0.343250204546271\\
68.4991	-0.423642554097184\\
68.5991	-0.476386021641532\\
68.6991	-0.48352002204447\\
68.7991	-0.489935947664856\\
68.8991	-0.491556003954079\\
68.9991	-0.519665380384183\\
69.0991	-0.587164928134464\\
69.1991	-0.664627169257991\\
69.2991	-0.690447990580343\\
69.3991	-0.755056768480917\\
69.4991	-0.900664381441326\\
69.5991	-0.984030990216278\\
69.6991	-1.05123595509808\\
69.7991	-0.969914405048118\\
69.8991	-0.893442846117567\\
69.999	-0.893979921870851\\
70.099	-0.919648907747741\\
70.199	-0.908558333012208\\
70.2991	-0.807134032977381\\
70.3991	-0.646566743939319\\
70.4991	-0.600505993848387\\
70.599	-0.597213995796771\\
70.699	-0.601971518424704\\
70.7991	-0.599734936406053\\
70.8991	-0.498618726542285\\
70.9991	-0.408690066382902\\
71.0991	-0.410798931594454\\
71.1991	-0.411005268336302\\
71.2991	-0.334312891123145\\
71.3991	-0.235950720632285\\
71.4991	-0.162017372365939\\
71.5991	-0.111087261319295\\
71.6991	-0.0957838900254866\\
71.7991	-0.0654007556669649\\
71.8991	-0.0571526804947911\\
71.9991	-0.0773514494483827\\
72.0991	-0.0676009939483686\\
72.1991	-0.0797558718918689\\
72.2991	-0.0729336800904976\\
72.3991	-0.0338851382884227\\
72.4991	0.019365347443923\\
72.5991	0.0814675172277473\\
72.6991	0.118125336273458\\
72.7991	0.127169405923567\\
72.8991	0.124647505227404\\
72.9991	0.0585845864350629\\
73.0991	0.0247265085714456\\
73.1991	0.0322021277833016\\
73.2991	0.0281079249561969\\
73.3991	-0.00640288350190155\\
73.4991	0.00101925227611704\\
73.5991	0.0415268188218691\\
73.6991	0.0830694994904225\\
73.7991	0.0853678768885706\\
73.8991	0.085004748424018\\
73.9991	0.0953961424180181\\
74.0991	0.0987897145607796\\
74.1991	0.0962127550260463\\
74.2991	0.0658001766248406\\
74.3991	0.0137226787059916\\
74.4991	0.000430836306179128\\
74.5991	0.0196700857191104\\
74.6991	0.0871475257381786\\
74.7991	0.0735979440240994\\
74.8991	0.0851488355574008\\
74.9991	0.0805859736361927\\
75.099	0.086939645018633\\
75.1991	0.0358642906185531\\
75.2991	0.024054412804381\\
75.3991	0.0124780441675882\\
75.4991	-0.0123595388408214\\
75.5991	0.0318807724499568\\
75.6991	0.0495437253744826\\
75.7991	0.08853072374487\\
75.8991	0.121607356877851\\
75.9991	0.108787574774277\\
76.0991	0.0322295934657591\\
76.1991	0.00456117498970388\\
76.2991	-0.0141615175900535\\
76.3991	-0.022519499446868\\
76.4991	0.000366442152398117\\
76.5991	0.0188912592947952\\
76.6991	0.0538500390697916\\
76.7991	0.010852567068676\\
76.8991	0.0323893083579306\\
76.9991	-0.0332555729281071\\
77.0991	-0.0583883162428814\\
77.1991	-0.102317278418782\\
77.2991	-0.0614775358037879\\
77.3991	-0.0308690806385548\\
77.4991	0.0033008075075367\\
77.5991	0.0124488644538193\\
77.6991	0.0195767052992374\\
77.7991	-0.0639374054306636\\
77.8991	-0.0819781245207749\\
77.9991	-0.0795533645570136\\
78.0991	-0.104945115100805\\
78.1991	-0.0882368573327309\\
78.2991	-0.0810399286811703\\
78.3991	0.00188236810491921\\
78.4991	-0.040890869097202\\
78.5991	-0.0477609997104743\\
78.6991	-0.063091608762685\\
78.7991	-0.0685675830816335\\
78.8991	-0.110483502769021\\
78.9991	-0.151113081593719\\
79.0991	-0.126442666779083\\
79.1992	-0.10930243546366\\
79.2992	-0.0454937617946434\\
79.3992	-0.0547264105392164\\
79.4992	-0.0703961480949531\\
79.5991	-0.122051181347808\\
79.6992	-0.135454586772343\\
79.7992	-0.165727566114127\\
79.8992	-0.184035230748567\\
79.9991	-0.108297789031133\\
80.0991	-0.0496729547356162\\
80.1991	-0.0208785966423869\\
80.2991	-0.0812827346320056\\
80.3991	-0.0784838801201496\\
80.4991	-0.120209039546577\\
80.5992	-0.138128822298337\\
80.6992	-0.129429494494473\\
80.7991	-0.136616839561144\\
80.8991	-0.072767163271937\\
80.9992	-0.0738174902911709\\
81.0992	-0.09171652436196\\
81.1992	-0.0825890928206977\\
81.2992	-0.0898509563828275\\
81.3991	-0.10102980934434\\
81.4992	-0.141449593402495\\
81.5992	-0.13220720269685\\
81.6992	-0.0925691495966602\\
81.7992	-0.0814323780343755\\
81.8992	-0.0666136707040759\\
81.9992	-0.0476141016436105\\
82.0992	-0.0673891079009775\\
82.1992	-0.0891137230896527\\
82.2992	-0.0880450784532239\\
82.3992	-0.115097675169336\\
82.4992	-0.123398060065912\\
82.5992	-0.13454618400702\\
82.6992	-0.113996660316786\\
82.7992	-0.10196720663892\\
82.8992	-0.0675706836808287\\
82.9992	-0.0689178002164077\\
83.0992	-0.0342376568951412\\
83.1992	-0.0267847367165474\\
83.2992	-0.0303556843948318\\
83.3992	-0.130381485310576\\
83.4992	-0.177836220856302\\
83.5992	-0.21606002156302\\
83.6992	-0.247525571890787\\
83.7992	-0.205649287386702\\
83.8992	-0.121170900452209\\
83.9992	-0.11782981715925\\
84.0992	-0.12395501649601\\
84.1992	-0.151760176254679\\
84.2992	-0.204942871990065\\
84.3992	-0.22631760732805\\
84.4992	-0.21246961328191\\
84.5993	-0.178358943551488\\
84.6992	-0.102782784818747\\
84.7993	-0.118075660240562\\
84.8993	-0.092772789904664\\
84.9991	-0.0555597468628719\\
85.0991	-0.0866350957344662\\
85.1992	-0.111517929333987\\
85.2992	-0.14099133939811\\
85.3992	-0.116379803681328\\
85.4992	-0.0335401838321206\\
85.5992	0.075568702632404\\
85.6992	0.176095360113957\\
85.7992	0.256621753347797\\
85.8992	0.323180219136657\\
85.9992	0.363090683494871\\
86.0992	0.446145201704964\\
86.1992	0.509833578836116\\
86.2992	0.626271576636988\\
86.3992	0.703345549715092\\
86.4992	0.79036558881198\\
86.5992	0.909737494078454\\
86.6992	1.0065393676418\\
86.7992	1.01556286005966\\
86.8992	1.03856386673963\\
86.9992	1.09101419406847\\
87.0992	1.1590014368218\\
87.1992	1.1737141698832\\
87.2992	1.15779427980405\\
87.3992	1.19231128322551\\
87.4992	1.25985836689118\\
87.5992	1.32720445868375\\
87.6992	1.36117109426115\\
87.7992	1.30892281127212\\
87.8992	1.28701301566586\\
87.9992	1.30989364172061\\
88.0992	1.35252227422162\\
88.1992	1.32847695510886\\
88.2992	1.10025772060498\\
88.3993	1.01545378425933\\
88.4993	1.1016799880317\\
88.5992	1.12232093155388\\
88.6993	1.04625361942334\\
88.7993	0.839973178214231\\
88.8993	0.665850389579191\\
88.9993	0.634782671443643\\
89.0993	0.56643329895065\\
89.1993	0.428062441629755\\
89.2993	0.175727311105663\\
89.3993	-0.0728653288468322\\
89.4993	-0.212841296300155\\
89.5993	-0.330010917151804\\
89.6993	-0.472873938085678\\
89.7993	-0.615464599945827\\
89.8993	-0.703263712368112\\
89.9993	-0.786403114154306\\
90.0993	-0.844378898604151\\
90.1993	-0.896554356400991\\
90.2993	-0.936568712924171\\
90.3993	-0.901823085600535\\
90.4993	-0.902069384060147\\
90.5993	-0.96086999142399\\
90.6993	-1.00010253149252\\
90.7993	-1.03791123878424\\
90.8993	-0.945070587768595\\
90.9993	-0.873988821709344\\
91.0993	-0.804294139436884\\
91.1993	-0.752565074802952\\
91.2993	-0.642888620422476\\
91.3993	-0.385658157110303\\
91.4993	-0.154015680191464\\
91.5993	-0.00636398199426663\\
91.6993	0.0686336587925118\\
91.7993	0.150939941601564\\
91.8993	0.274705899204302\\
91.9993	0.360780738762365\\
92.0993	0.428578549532822\\
92.1993	0.49887892799291\\
92.2993	0.574025054802392\\
92.3993	0.646229204984851\\
92.4993	0.672433912814793\\
92.5993	0.763132136311497\\
92.6993	0.879505084410944\\
92.7993	0.964108449427106\\
92.8993	1.06495366021141\\
92.9994	1.13209025801843\\
93.0993	1.09524273844422\\
93.1994	1.14756380639213\\
93.2994	1.23547707696451\\
93.3994	1.30954571357333\\
93.4994	1.3069750514517\\
93.5994	1.25454731287398\\
93.6994	1.24567364027803\\
93.7994	1.34140066677268\\
93.8994	1.44132777408186\\
93.9994	1.48757292690169\\
94.0994	1.48388432748011\\
94.1994	1.39600591810195\\
94.2994	1.41634243037505\\
94.3994	1.46414675528574\\
94.4994	1.49878198186416\\
94.5994	1.4370673149884\\
94.6994	1.32127115178929\\
94.7994	1.34699349311618\\
94.8994	1.45910311344739\\
94.9993	1.49736024160525\\
95.0993	1.41884232421216\\
95.1993	1.31020831456671\\
95.2994	1.25607749526932\\
95.3994	1.36312691607354\\
95.4994	1.43548410187878\\
95.5994	1.40401926923231\\
95.6994	1.31284663656404\\
95.7994	1.30530755477274\\
95.8994	1.34176490096272\\
95.9994	1.46237097033181\\
96.0994	1.55295833818356\\
96.1994	1.54238799665802\\
96.2994	1.55278712693812\\
96.3994	1.60190449973432\\
96.4994	1.65384541211892\\
96.5994	1.66735786691185\\
96.6994	1.59836081968737\\
96.7994	1.59381383335326\\
96.8994	1.65326922893015\\
96.9994	1.68428916054009\\
97.0994	1.68056390506748\\
97.1994	1.59855955324135\\
97.2994	1.55051058473314\\
97.3994	1.58792711309936\\
97.4994	1.59173849068569\\
97.5994	1.58582951457135\\
97.6994	1.49440414980415\\
97.7994	1.49508857373078\\
97.8994	1.54628874571205\\
97.9994	1.58409318807076\\
98.0994	1.53443285239863\\
98.1995	1.4804550299062\\
98.2995	1.39854688377196\\
98.3995	1.39121441630678\\
98.4995	1.38664261558233\\
98.5995	1.30534749125703\\
98.6995	1.18922431168534\\
98.7995	1.04650482544642\\
98.8995	1.04649449835055\\
98.9995	1.0357407561603\\
99.0995	0.942652088385023\\
99.1995	0.777356350258255\\
99.2995	0.607704507790135\\
99.3995	0.438981249215451\\
99.4995	0.345960592772358\\
99.5995	0.269116505107058\\
99.6995	0.167316359275897\\
99.7995	0.117175570857752\\
99.8995	0.142914584473838\\
99.9994	0.127100908877616\\
100.0994	0.0264262530330023\\
100.1995	0.0308265535422332\\
100.2995	0.0385558149351204\\
100.3995	0.0374631728523275\\
100.4995	0.0318069909560814\\
100.5995	0.0410476462343121\\
100.6995	0.0500045957520499\\
100.7995	0.000394491981907208\\
100.8995	0.000893310433234929\\
100.9995	0.000633108994694698\\
101.0995	0.00010742460872454\\
101.1995	-0.000111410857620313\\
101.2995	-0.000116141785569796\\
101.3995	0.000192717312941197\\
101.4995	0.000170431708350732\\
101.5995	-0.000256761248074773\\
101.6995	-0.000184718129507142\\
101.7995	0.000196561444828612\\
101.8995	0.000120936122596997\\
101.9995	-4.29771787346725e-05\\
102.0995	-1.32934163046892e-06\\
102.1995	-0.000257086181818417\\
102.2995	-0.00023029624241257\\
102.3995	1.55462732084586e-05\\
102.4995	0.000183926358399986\\
102.5995	-1.96809697229541e-06\\
102.6995	0.000149379703604346\\
102.7995	0.000371435813027107\\
102.8995	-0.000227520292409803\\
102.9995	-5.04798990326695e-05\\
103.0995	2.88883124735257e-06\\
103.1996	7.92712121497241e-05\\
103.2996	0.000107860348018917\\
103.3996	-0.00037200962392981\\
103.4996	-0.000136694940945883\\
103.5996	-4.97768223254955e-05\\
103.6996	-0.000140630758122386\\
103.7996	-0.000151196710625977\\
103.8996	-0.000343099470376726\\
103.9996	-2.36519456810148e-05\\
104.0996	0.000898342526554121\\
104.1996	0.00222853716184867\\
104.2996	0.0037716444104736\\
104.3996	0.00176722800958536\\
104.4996	-0.0334548165227669\\
104.5996	-0.123854790082784\\
104.6996	-0.0499014783348286\\
104.7996	0.0381611186069009\\
104.8996	0.0800015148134731\\
104.9996	0.0784805958568004\\
105.0996	0.0222956010977975\\
105.1996	0.0241087502294251\\
105.2996	0.0101156554522659\\
105.3996	0.00274277806307823\\
105.4996	0.00195213571192276\\
105.5996	0.00480260429871689\\
105.6996	0.0049687686916712\\
105.7996	0.00384044732069097\\
105.8996	0.00241863398414402\\
105.9996	0.00132076657012388\\
106.0996	7.20153397408552e-05\\
106.1996	-0.000119591290711142\\
106.2996	-0.00040417384963782\\
106.3996	-2.66411191153226e-05\\
106.4996	9.81677462425466e-06\\
106.5996	-0.000113903963542271\\
106.6996	-0.000151046573873999\\
106.7996	-0.000403044692312866\\
106.8996	0.000216710567376346\\
106.9996	-0.000289889151527891\\
107.0996	-0.000589715958264758\\
107.1996	-0.000735669923717752\\
107.2996	2.15034268668357e-05\\
107.3996	6.77179752998966e-05\\
107.4996	9.81922929261709e-05\\
107.5996	-7.73341140466923e-05\\
107.6996	-0.000126160806123485\\
107.7996	5.41597657999356e-06\\
107.8996	-0.000276605838446279\\
107.9996	-0.000295699499940357\\
108.0996	-7.49419182433894e-06\\
108.1997	-0.000247163261636309\\
108.2996	-4.7358338858443e-05\\
108.3997	-0.00010320205267439\\
108.4997	-0.000233153526771823\\
108.5997	-0.000126612462170499\\
108.6997	-0.000487174127790294\\
108.7997	-0.000121330928707627\\
108.8997	-0.000150796354324602\\
108.9997	-0.000460575189026429\\
109.0997	-6.96370807772937e-05\\
109.1997	0.000192758756411565\\
109.2997	0.000135864961650313\\
109.3997	-0.000128087842298863\\
109.4997	-0.000125463191684728\\
109.5997	-0.000182955756899524\\
109.6997	-0.000323514498316572\\
109.7997	-0.000386283399810877\\
109.8997	-6.65713933946776e-05\\
109.9996	-0.00031101944097334\\
110.0996	-0.000353635676811747\\
110.1996	0.000308624421800398\\
110.2996	0.00456164024042593\\
110.3996	0.00867883292398134\\
110.4996	0.0133958290173003\\
110.5996	0.0165306489451317\\
110.6996	0.0222693108426534\\
110.7997	0.0263658461542603\\
110.8997	0.021808930166429\\
110.9997	-0.000149638457441377\\
111.0997	-0.0437365514212504\\
111.1997	-0.13667224308051\\
111.2997	-0.183629725718053\\
111.3997	-0.127105041479202\\
111.4997	-0.0405262973072839\\
111.5997	0.0196035090533948\\
111.6997	0.0495489851376779\\
111.7997	0.0575432925285963\\
111.8997	0.0961512973304171\\
111.9997	0.154278731211698\\
112.0997	0.191370648957893\\
112.1997	0.190035562842436\\
112.2997	0.1933205592185\\
112.3997	0.178307084054494\\
112.4997	0.15330238248322\\
112.5997	0.1623285886954\\
112.6997	0.125657664344682\\
112.7997	0.124440141090025\\
112.8997	0.081868048816756\\
112.9997	0.0517875214745306\\
113.0997	0.0313710541587302\\
113.1997	0.0208399336271201\\
113.2997	0.0102791967453409\\
113.3997	2.11893848724591e-05\\
113.4997	7.01404898737841e-05\\
113.5997	0.000561807866838519\\
113.6997	0.000907854780519547\\
113.7998	0.000809116047863032\\
113.8998	0.000192120244569913\\
113.9998	-2.74382299885448e-05\\
114.0998	0.000316854033383955\\
114.1998	7.23897001051491e-05\\
114.2998	0.00036330236013379\\
114.3998	0.000107667659342971\\
114.4998	-0.000361298676020237\\
114.5998	0.000131568467184631\\
114.6998	1.73255091444231e-05\\
114.7998	-0.000238105009575424\\
114.8998	-0.000224346422083483\\
114.9997	-0.000315138268421484\\
115.0997	-0.000319108652676361\\
115.1997	-0.0010046673185792\\
115.2997	-0.000251310875431625\\
115.3997	-0.000237492296128523\\
115.4997	0.000224815792388552\\
115.5997	0.000364787670590725\\
115.6997	-0.000245112733801914\\
115.7998	5.84094610855856e-05\\
115.8997	0.000146870027215009\\
115.9997	-8.76666592613623e-05\\
116.0998	-5.01896228181375e-05\\
116.1998	0.000325936512706914\\
116.2998	9.3266541667668e-05\\
116.3998	-9.92947268739826e-05\\
116.4998	0.000354207482819694\\
116.5998	-0.000381617558631718\\
116.6998	-0.000408637011554937\\
116.7998	-0.000128411257108366\\
116.8998	0.000214732783268241\\
116.9998	-4.57293027889098e-05\\
117.0998	0.000290247012389622\\
117.1998	0.000590784756274854\\
117.2998	0.00107059138067515\\
117.3998	0.000956142664812179\\
117.4998	0.000575148339070577\\
117.5998	0.000127306053605615\\
117.6998	0.00018054715710902\\
117.7998	0.000637364196784321\\
117.8998	6.94434514794525e-05\\
117.9998	0.000410758305220008\\
118.0998	0.000241925195435237\\
118.1998	0.000164392934988332\\
118.2998	-0.000155578639247141\\
118.3998	-0.000167060457463156\\
118.4998	2.29764891757819e-05\\
118.5998	3.55052772494635e-05\\
118.6998	0.000232573209579483\\
118.7998	0.000359402800540815\\
118.8998	-0.000158429686718037\\
118.9998	4.58231779868847e-05\\
119.0999	0.000299995708237196\\
119.1999	-1.94036114012947e-05\\
119.2999	0.000158634674052326\\
119.3999	-7.64551689130701e-07\\
119.4998	-0.000103579248829597\\
119.5999	1.15540352562497e-05\\
119.6999	-0.000287217200311343\\
119.7999	-0.000204403373757175\\
};
\addplot [color=mycolor6,solid,forget plot]
  table[row sep=crcr]{%
0	3.30982285945311e-07\\
0.1	-0.000157523255294273\\
0.2	-3.64917117496932e-06\\
0.3	2.48081510571441e-05\\
0.4	0.000181690069710611\\
0.5	0.000208042703531355\\
0.6	0.000258004250364518\\
0.7	-0.00016575631497241\\
0.8	-0.000104136182213584\\
0.9	-2.34391417698856e-05\\
1	0.00020231608976888\\
1.0999	0.000183065393861081\\
1.1999	0.000670685928142049\\
1.2999	0.000239832859537703\\
1.3999	0.000482478603862768\\
1.4999	-0.000140303550597627\\
1.5999	0.000102055628328844\\
1.6999	9.67011765997234e-05\\
1.7999	-0.000206379542656741\\
1.8999	-0.000329541091770412\\
1.9999	-4.37716496856229e-05\\
2.0999	0.000202142948810888\\
2.1999	-0.000191782161026304\\
2.2999	-0.000339443138651902\\
2.3999	-0.000562984068786248\\
2.4999	-0.000931829046868715\\
2.5999	-9.94237061921647e-05\\
2.6999	3.47996456175297e-05\\
2.7999	3.87187025013602e-05\\
2.8999	2.90248347036944e-06\\
2.9999	-8.77480011415052e-06\\
3.0999	0.000242743593998539\\
3.1999	0.000433359307296982\\
3.2999	0.000392891491288285\\
3.3999	0.000842389153715591\\
3.4999	0.000746474363916299\\
3.5999	0.00127855601191024\\
3.6999	0.0011739037931536\\
3.7999	0.00116843753931919\\
3.8999	0.000534277546668752\\
3.9999	4.98694134774015e-06\\
4.0999	7.03542308742628e-07\\
4.1999	0.000267413243427725\\
4.2999	-0.000216337259728946\\
4.3999	-0.000158118004813569\\
4.4999	0.000116197763001163\\
4.5999	0.00209918513221148\\
4.6999	0.00471880436917523\\
4.7999	0.00776276522569452\\
4.8998	0.00987625382153775\\
4.9998	0.0124250327475834\\
5.0998	0.0150471148646078\\
5.1998	0.0168636907865695\\
5.2998	0.019653762239308\\
5.3998	0.0220499553405428\\
5.4997	0.0242068416635126\\
5.5997	0.0262175400728511\\
5.6998	0.028765928080752\\
5.7998	0.0302881895865664\\
5.8998	0.0324136829729172\\
5.9998	0.034991394865438\\
6.0997	0.036746693246535\\
6.1997	0.0360804562261979\\
6.2997	0.0266485988196685\\
6.3997	-0.021289833321427\\
6.4997	-0.27143378156819\\
6.5997	-0.7080178397081\\
6.6997	-1.16644530216375\\
6.7997	-1.60567041819674\\
6.8997	-1.64010064219348\\
6.9997	-1.57808331840175\\
7.0997	-1.2993352985516\\
7.1997	-1.01768678371771\\
7.2997	-0.685342140386936\\
7.3997	-0.213992028357258\\
7.4997	-0.0824670993451134\\
7.5997	-0.121905178032648\\
7.6997	-0.118073427841483\\
7.7997	-0.0571422580775394\\
7.8997	-0.025689120430244\\
7.9997	0.0146568873814413\\
8.0997	0.0218880997811726\\
8.1997	0.0325537746953887\\
8.2997	-0.00528061617363303\\
8.3997	0.0116334887707116\\
8.4997	0.00739592985562764\\
8.5997	0.000635944669492149\\
8.6997	0.000366851616263623\\
8.7997	0.000374157548608945\\
8.8997	0.000226076821390644\\
8.9997	0.000410449680475447\\
9.0997	0.000483937061151864\\
9.1997	0.000458414702242114\\
9.2997	0.00028128721804395\\
9.3997	0.000334160877053629\\
9.4997	1.15049796241256e-05\\
9.5997	-9.19316783659664e-05\\
9.6997	0.00015536319428272\\
9.7997	-0.000148185546387352\\
9.8997	0.000373743615877784\\
9.9996	3.97239337263586e-05\\
10.0996	-1.1932387112223e-05\\
10.1996	0.000127346399464853\\
10.2996	-0.000103969396194415\\
10.3996	-9.50507242540892e-05\\
10.4996	0.000275123115236268\\
10.5996	0.000544255330619645\\
10.6996	0.000696116135778016\\
10.7996	0.00115905373954459\\
10.8996	0.00110280747999449\\
10.9996	0.000899501061776884\\
11.0996	0.000969874420031724\\
11.1996	0.000773567168012659\\
11.2996	0.000644650827569565\\
11.3996	2.70375197655047e-07\\
11.4995	0.000180914032655714\\
11.5995	0.000177675159507954\\
11.6996	-1.38186211052835e-05\\
11.7996	0.000763726435357799\\
11.8995	0.000205639182205967\\
11.9996	0.000157575340188772\\
12.0995	0.000339284050956351\\
12.1995	-2.19663284156432e-05\\
12.2996	0.000198136789904213\\
12.3995	0.00027365420498799\\
12.4995	-8.41593526194864e-05\\
12.5995	-0.00044421349068892\\
12.6995	-6.17932676832141e-06\\
12.7995	-0.00238202504458754\\
12.8995	-0.00491329206279766\\
12.9995	-0.00760072397173737\\
13.0995	-0.00989430061491585\\
13.1995	-0.0120418225767311\\
13.2995	-0.014477192448852\\
13.3995	-0.0172700190510622\\
13.4995	-0.0155196587596292\\
13.5995	0.0276654466525233\\
13.6995	0.127946495463214\\
13.7995	0.0980758858492679\\
13.8995	0.0754166415261134\\
13.9995	0.0608796042114548\\
14.0995	0.0358201696342184\\
14.1995	-0.0202347338253113\\
14.2995	-0.0208539627581287\\
14.3995	-0.0103831871196266\\
14.4995	-0.00261028553500789\\
14.5995	-8.7315479943079e-05\\
14.6995	0.000145711857884726\\
14.7995	0.000105181942628238\\
14.8995	0.00021317618424521\\
14.9994	2.60168395085688e-05\\
15.0994	2.05023081100044e-05\\
15.1994	7.50380425436037e-05\\
15.2994	-5.86364372172315e-05\\
15.3994	0.00030935776494648\\
15.4994	-0.000171686349895877\\
15.5994	6.38619196063894e-06\\
15.6994	0.000244524036340245\\
15.7994	-2.30986265708438e-05\\
15.8994	-0.000102376457695792\\
15.9994	5.11850975250211e-05\\
16.0994	0.000164965598464041\\
16.1994	-0.000164259456119678\\
16.2994	-0.000263212763548178\\
16.3994	0.000107444803205542\\
16.4994	0.000302115168075615\\
16.5994	-8.67035716465782e-05\\
16.6994	6.89765260518735e-06\\
16.7994	-0.00013344463035991\\
16.8994	1.27985129118086e-05\\
16.9994	-0.000119958945075599\\
17.0994	-0.000489132099966329\\
17.1994	-0.000117923061450437\\
17.2994	1.55310262369313e-05\\
17.3994	0.000130413313856332\\
17.4994	0.000274299706852774\\
17.5994	0.000275267587444463\\
17.6994	7.91341598286386e-05\\
17.7994	-0.000341780321539025\\
17.8994	-0.000140323920600378\\
17.9994	-0.000313848518377222\\
18.0994	-0.000120488229445553\\
18.1994	-0.00010881418196168\\
18.2994	-0.000226528727856229\\
18.3994	5.89552653144486e-07\\
18.4994	-0.000262463198598539\\
18.5994	-0.000122447215806842\\
18.6994	0.00189797708424949\\
18.7994	0.0107626548935054\\
18.8994	-0.00457101812942092\\
18.9994	-0.052221704424605\\
19.0994	-0.0788830016003644\\
19.1994	-0.0883683147794089\\
19.2994	-0.15340450948432\\
19.3994	-0.170149578312634\\
19.4994	-0.141786675540449\\
19.5994	-0.0775740458306209\\
19.6994	-0.00933607059735837\\
19.7994	-0.0105981507599021\\
19.8994	0.0428871080331761\\
19.9993	-0.023555012122246\\
20.0993	-0.00842263856117113\\
20.1993	-0.0347301277459367\\
20.2993	0.0251043661831088\\
20.3993	0.0762037900604759\\
20.4993	0.0842698346514461\\
20.5993	-0.0784367219380089\\
20.6993	-0.124668583454478\\
20.7993	-0.0162334886961171\\
20.8993	0.0417509561310158\\
20.9993	0.114202314427028\\
21.0993	0.0872732701253423\\
21.1993	-0.0835332067854649\\
21.2993	-0.184765658917671\\
21.3993	-0.15562680667982\\
21.4993	-0.00744870628759669\\
21.5993	0.121709263553912\\
21.6993	0.0457049752100142\\
21.7993	-0.206773345870429\\
21.8993	-0.173576758335066\\
21.9993	0.0158256599431928\\
22.0993	0.202917823057729\\
22.1993	0.163707882441628\\
22.2993	-0.10463594550193\\
22.3993	-0.26444324291102\\
22.4993	-0.103712701597673\\
22.5993	0.142061446035885\\
22.6993	0.187343497982408\\
22.7993	0.000148017042645929\\
22.8993	-0.260517095040657\\
22.9993	-0.133344543408801\\
23.0993	0.186820836017201\\
23.1993	0.151556426428771\\
23.2993	-0.0874140051174994\\
23.3993	-0.209057040738045\\
23.4993	-0.0237739935505675\\
23.5993	0.157198526760592\\
23.6993	0.165777212464291\\
23.7993	-0.0684145689058887\\
23.8993	-0.208880814097882\\
23.9993	-0.0586300738135528\\
24.0993	0.165630812305691\\
24.1993	0.0176724170702448\\
24.2993	-0.204364280645565\\
24.3993	-0.202978414649183\\
24.4993	-0.0217460957997516\\
24.5993	0.182700552779589\\
24.6993	0.0917707681557613\\
24.7993	-0.152485196622169\\
24.8993	-0.179339494492974\\
24.9992	0.0479548339488881\\
25.0992	0.200142020220171\\
25.1992	-0.00218588918414331\\
25.2992	-0.169764306300185\\
25.3992	-0.0751505390739485\\
25.4992	0.0352680934964004\\
25.5992	0.184563237661557\\
25.6992	0.0786389487750146\\
25.7992	-0.228018119406148\\
25.8992	-0.140772850322377\\
25.9992	0.0866086905510284\\
26.0992	0.16601543341196\\
26.1992	-0.0595973001105282\\
26.2992	-0.25390846043638\\
26.3992	-0.135739057093829\\
26.4992	0.0997124823540702\\
26.5992	0.176353873517807\\
26.6992	-0.0381323399293497\\
26.7992	-0.251178155776502\\
26.8992	-0.136956482941277\\
26.9992	0.185140772867918\\
27.0992	0.194838755704541\\
27.1992	-0.0469848208250165\\
27.2992	-0.255580902165707\\
27.3992	-0.083233534905554\\
27.4992	0.131006623643142\\
27.5992	0.170211582846135\\
27.6992	-0.0507057276500219\\
27.7992	-0.327589970868189\\
27.8992	-0.228675310310041\\
27.9992	0.134799333274483\\
28.0992	0.256142251651699\\
28.1992	0.0269623896623334\\
28.2992	-0.271399024686438\\
28.3992	-0.193263792483144\\
28.4992	0.0558762408687418\\
28.5992	0.231676480922403\\
28.6992	0.0811809069300406\\
28.7992	-0.271786053427536\\
28.8992	-0.302815276390602\\
28.9992	-0.087516520654983\\
29.0992	0.0618547041531735\\
29.1992	-0.0762469226440397\\
29.2992	-0.280518928533492\\
29.3992	-0.42240301091606\\
29.4992	-0.320250078673754\\
29.5992	-0.23052975741847\\
29.6992	-0.312551191513179\\
29.7992	-0.530842828254335\\
29.8992	-0.514454595330067\\
29.9991	-0.284474627008509\\
30.0991	-0.185746285934052\\
30.1991	-0.326219632320805\\
30.2991	-0.594868919347682\\
30.3991	-0.479590300108902\\
30.4991	-0.216092410922004\\
30.5991	-0.229599415997919\\
30.6991	-0.375795531861474\\
30.7991	-0.604669778788064\\
30.8991	-0.406800191584401\\
30.9991	-0.272118781074999\\
31.0991	-0.34562824407212\\
31.1991	-0.558845676173032\\
31.2991	-0.468951210716584\\
31.3991	-0.30008970034287\\
31.4991	-0.188140366868958\\
31.5991	-0.375940102999816\\
31.6991	-0.558257890360082\\
31.7991	-0.425791072697613\\
31.8991	-0.2452206465011\\
31.9991	-0.187975046058858\\
32.0991	-0.427632716102397\\
32.1991	-0.540420363333168\\
32.2991	-0.381152977157565\\
32.3991	-0.195511611482857\\
32.4991	-0.216301964796334\\
32.5991	-0.432074898423728\\
32.6991	-0.644717175795621\\
32.7991	-0.4503159659506\\
32.8991	-0.203274287903352\\
32.9991	0.0259412524243764\\
33.0991	-0.142546764538623\\
33.1991	-0.355556312847733\\
33.2991	-0.483703304182199\\
33.3991	-0.381749627326764\\
33.4991	-0.30960652171548\\
33.5991	-0.0710759620365012\\
33.6991	-0.132962944003842\\
33.7991	-0.535660385872995\\
33.8991	-0.693700220909366\\
33.9991	-0.447994180453958\\
34.0991	-0.235728489839631\\
34.1991	-0.164161853798573\\
34.2991	-0.382953837543591\\
34.3991	-0.486748647212731\\
34.4991	-0.369252572724063\\
34.5991	-0.208736471028974\\
34.6991	-0.239842164829112\\
34.7991	-0.399253303778028\\
34.8991	-0.657951975971723\\
34.9991	-0.414628352866335\\
35.0991	-0.201340610404073\\
35.1991	-0.206905616271726\\
35.2991	-0.516623360052238\\
35.3991	-0.608520029037657\\
35.4991	-0.423356614967949\\
35.5991	-0.275132193693677\\
35.6991	-0.440683106903786\\
35.7991	-0.714979062881234\\
35.8991	-0.522824556743983\\
35.9991	-0.200404450139705\\
36.0991	-0.119103446272967\\
36.1991	-0.36692355347197\\
36.2991	-0.599082855395337\\
36.3991	-0.440791925307652\\
36.4991	-0.236561300994664\\
36.5991	-0.206288508910547\\
36.6991	-0.431577582676058\\
36.7991	-0.703884352907279\\
36.8991	-0.481220425846262\\
36.9991	-0.221955281793767\\
37.0991	-0.231840877385145\\
37.1991	-0.456771181806509\\
37.2991	-0.456695286917427\\
37.3991	-0.325837073097564\\
37.4991	-0.173429710122054\\
37.5991	-0.388888842783336\\
37.6991	-0.692229198298798\\
37.7991	-0.622150492410988\\
37.8991	-0.300657441421502\\
37.9991	-0.132107033268656\\
38.0991	-0.340778490802814\\
38.1991	-0.560268592505958\\
38.2991	-0.426195487522126\\
38.3991	-0.228928299773779\\
38.4991	-0.210097887873273\\
38.5991	-0.428369210444498\\
38.6991	-0.778485925295606\\
38.7991	-0.59048468485449\\
38.8991	-0.315780896413584\\
38.9991	-0.123350706871083\\
39.0991	-0.324296828888078\\
39.1991	-0.602150010551698\\
39.2991	-0.473966344437976\\
39.3991	-0.28898278952079\\
39.4991	-0.156708836799931\\
39.5991	-0.375566563509217\\
39.6991	-0.705680628598415\\
39.7991	-0.481874476988314\\
39.8991	-0.260296186410167\\
39.9991	-0.260843688775348\\
40.0991	-0.482363145427026\\
40.1991	-0.55570399866254\\
40.2991	-0.420623337149258\\
40.3991	-0.259120868179206\\
40.499	-0.32950864590433\\
40.5991	-0.598777652178645\\
40.6991	-0.708254333219215\\
40.7991	-0.474449801296203\\
40.8991	-0.291688249435177\\
40.9991	-0.26582464071371\\
41.0991	-0.531502715988196\\
41.1991	-0.620785182352574\\
41.2991	-0.474640604810956\\
41.399	-0.293429370602442\\
41.4991	-0.0994575443859588\\
41.5991	-0.117316471945036\\
41.6991	-0.425265460711133\\
41.7991	-0.665591724970584\\
41.8991	-0.457695568001019\\
41.999	-0.188218113227178\\
42.0991	0.0751568754148704\\
42.1991	0.0994979422294964\\
42.2991	-0.14343607204018\\
42.3991	-0.307312083327084\\
42.4991	-0.247154914487575\\
42.5991	-0.103541540675253\\
42.6991	0.0247349472315738\\
42.7991	0.0168107961620313\\
42.8991	-0.121288139242492\\
42.9991	-0.253014213098178\\
43.0991	-0.146623940820145\\
43.1991	0.0556649167657542\\
43.2991	0.0788869155363709\\
43.3991	-0.0563140736929125\\
43.4991	-0.221233141470529\\
43.5991	-0.187236519010201\\
43.6991	-0.157226766093991\\
43.7991	-0.035440310734246\\
43.8991	-0.0552083864776387\\
43.9991	-0.128911995723037\\
44.0991	-0.199274289607362\\
44.1991	-0.0574824727148845\\
44.2991	0.074013435375755\\
44.3991	0.161735585183559\\
44.4991	0.135186141842518\\
44.5991	0.151571635017719\\
44.6991	0.169240768367505\\
44.7991	0.171392192473096\\
44.8991	0.186493754852784\\
44.999	0.276248855721019\\
45.099	0.266811814473414\\
45.199	0.119126673059641\\
45.299	0.149294768389753\\
45.399	0.270984114897811\\
45.499	0.321408878364833\\
45.599	0.245274416554083\\
45.699	0.137677552863764\\
45.799	0.060978958272604\\
45.899	0.0312121364485984\\
45.999	0.261344818234845\\
46.099	0.326857937768025\\
46.199	0.208880024119659\\
46.299	0.00445470223491847\\
46.399	0.010484079132662\\
46.499	0.202704037790009\\
46.599	0.381883928243584\\
46.699	0.287974061941027\\
46.799	0.218573351879463\\
46.899	0.26970114104131\\
46.999	0.35430582383158\\
47.099	0.42501938230686\\
47.199	0.413773824011761\\
47.299	0.086243558954565\\
47.399	0.0474191592071671\\
47.499	-0.0824942100132054\\
47.599	0.0656104821718175\\
47.699	0.15767980952374\\
47.799	0.113353140972224\\
47.899	0.0196846472168686\\
47.999	0.137142209140978\\
48.099	0.28703507978806\\
48.199	0.293365321375594\\
48.299	0.0899940996842425\\
48.399	-0.177303259777644\\
48.4991	-0.173315243921897\\
48.599	-0.0440375618750482\\
48.6991	0.189115999589055\\
48.7991	0.197700235731703\\
48.899	-0.060909556330298\\
48.9991	-0.254286907596351\\
49.0991	-0.0871224634623631\\
49.1991	0.155604865231647\\
49.2991	0.0841980038194389\\
49.3991	-0.175764817399319\\
49.499	-0.326524418064009\\
49.5991	-0.126314135139251\\
49.6991	0.242570584656295\\
49.7991	0.371239155246541\\
49.8991	0.138549866105835\\
49.999	-0.188958139444204\\
50.099	-0.110579302578128\\
50.199	0.140001513067034\\
50.299	0.355991990630616\\
50.399	0.0603170316750643\\
50.499	-0.287122270837024\\
50.599	-0.25415608313526\\
50.699	0.0219620013932397\\
50.799	0.262647455855893\\
50.899	0.081015076937062\\
50.999	-0.284370029213186\\
51.099	-0.34905178530388\\
51.199	-0.122994686767395\\
51.299	0.138104428971735\\
51.399	0.198394485417251\\
51.499	-0.17787470489246\\
51.599	-0.286147447625587\\
51.699	-0.14994019565658\\
51.799	0.179049391627727\\
51.899	0.291266132461945\\
51.999	-0.000342308758320907\\
52.099	-0.328892344712589\\
52.199	-0.280721594019409\\
52.299	0.0430567166614802\\
52.399	0.147543643379243\\
52.499	0.013870672817185\\
52.599	-0.230617917169056\\
52.699	-0.162315899576219\\
52.799	0.114543747800586\\
52.899	0.286121121244698\\
52.999	0.0860835886040716\\
53.099	-0.184522156850233\\
53.199	-0.270077224540948\\
53.2991	-0.0465129223087015\\
53.3991	0.116811159184689\\
53.4991	0.0776280463635659\\
53.5991	-0.250672407471649\\
53.6991	-0.210613780674947\\
53.7991	-0.0709834268878667\\
53.8991	0.161228436394843\\
53.999	0.114357951266799\\
54.0991	-0.133782912689182\\
54.1991	-0.398360202546029\\
54.299	-0.243301044577547\\
54.3991	0.0881439768741479\\
54.4991	0.191457988816341\\
54.599	0.0609196173316826\\
54.6991	-0.233234662045996\\
54.7991	-0.222353035272412\\
54.8991	-0.0473889858164126\\
54.999	-0.0174236831260257\\
55.099	-0.189449543174622\\
55.199	-0.493394015551648\\
55.299	-0.511320029243786\\
55.399	-0.306415669827179\\
55.499	-0.266095966194402\\
55.599	-0.293668575868908\\
55.699	-0.261248183133321\\
55.799	-0.381763418944836\\
55.899	-0.370141935445408\\
55.999	-0.207509146865249\\
56.099	-0.150032988736191\\
56.199	-0.136495237306204\\
56.299	-0.193276214332476\\
56.399	-0.164354785076916\\
56.499	0.0525344418474933\\
56.599	0.0874387399945405\\
56.699	-0.154576598363254\\
56.799	-0.405500578439462\\
56.899	-0.280237039866876\\
56.999	-0.0580687314630345\\
57.099	0.162470291164307\\
57.199	0.031746572891941\\
57.299	-0.178235011257481\\
57.399	-0.211078415620628\\
57.4991	0.0019448205569041\\
57.5991	0.0329796015170605\\
57.6991	-0.060578612886796\\
57.7991	-0.431391470345397\\
57.8991	-0.666244438376624\\
57.9991	-0.467328642956706\\
58.0991	-0.111217064774819\\
58.1991	0.0732285241311892\\
58.2991	-0.144207146931422\\
58.399	-0.426082778334095\\
58.4991	-0.421924888027117\\
58.5991	-0.0856313644386255\\
58.6991	0.171287370858018\\
58.7991	0.0125546774238451\\
58.8991	-0.141205728630309\\
58.9991	-0.120561774459124\\
59.0991	0.0335786907943453\\
59.1991	0.18656068785047\\
59.2991	0.107310926094576\\
59.3991	-0.192338185179556\\
59.4991	-0.104508037712576\\
59.5991	0.120241030194757\\
59.6991	0.1704155180616\\
59.7991	0.164613166451286\\
59.8991	-0.128562307146299\\
59.999	-0.0024964470572991\\
60.099	-0.0537188637816235\\
60.199	0.410908370418569\\
60.299	0.22912199264461\\
60.399	-0.177307746562799\\
60.499	-0.100738668308545\\
60.599	-0.00373819408902018\\
60.699	0.201517714353742\\
60.799	0.222115878909476\\
60.899	0.000320864594229073\\
60.999	-0.193466038499157\\
61.099	-0.261875371322967\\
61.199	0.168912903052146\\
61.299	0.376239353079328\\
61.399	-0.0579369559894313\\
61.499	-0.0942244601987114\\
61.599	-0.0778605595684121\\
61.699	0.150230025841298\\
61.799	0.318145907877162\\
61.899	0.153015264777848\\
61.999	-0.0763971428595876\\
62.099	-0.367666673743209\\
62.199	0.0592674939390366\\
62.299	0.421197522351342\\
62.399	0.0869724131947987\\
62.499	-0.175844244911805\\
62.599	-0.0932564003313595\\
62.699	0.111885106242066\\
62.799	0.323370210115488\\
62.899	0.230008408634865\\
62.999	-0.0368905940716927\\
63.099	-0.337184550819657\\
63.199	-0.162216937208154\\
63.299	0.344840070140296\\
63.399	0.207387949648445\\
63.499	-0.205628560226467\\
63.599	-0.106039324959353\\
63.699	0.0357078115942661\\
63.799	0.263667455008518\\
63.8991	0.298173019871865\\
63.999	0.0364703437281849\\
64.099	-0.189452928594553\\
64.1991	-0.226282771986893\\
64.2991	0.201269106240635\\
64.3991	0.394626946425632\\
64.4991	-0.0418568013012134\\
64.599	-0.101355417701347\\
64.6991	-0.0370829444140175\\
64.7991	0.206687824538902\\
64.8991	0.376631353761938\\
64.999	0.22400189820808\\
65.099	-0.0573100279327648\\
65.199	-0.324522749309138\\
65.299	0.0450380760392866\\
65.399	0.438414342605245\\
65.499	0.122895064936346\\
65.599	-0.130569326112264\\
65.699	-0.0563035200504156\\
65.799	0.143566343617037\\
65.899	0.33677177388475\\
65.999	0.271507636971374\\
66.099	-0.0024860311088778\\
66.199	-0.296655170718382\\
66.299	-0.102799335944575\\
66.399	0.382649961855241\\
66.499	0.213342439201225\\
66.599	-0.144443481962538\\
66.699	-0.0530531455493677\\
66.799	-0.015250605342778\\
66.899	0.165681626489502\\
66.999	0.272837016536153\\
67.099	-0.029023987942515\\
67.1991	-0.216621580050483\\
67.299	-0.448283786578229\\
67.3991	-0.0653246269245481\\
67.4991	0.226533067104677\\
67.5991	0.208503793950274\\
67.6991	-0.137454737689824\\
67.7991	-0.377341994361453\\
67.8991	-0.520961421357182\\
67.9991	-0.284533070642924\\
68.0991	-0.0738424512059005\\
68.1991	-0.111965197927652\\
68.2991	-0.0930643942864323\\
68.3991	0.050492884017877\\
68.4991	0.230560554703331\\
68.5991	0.199479805276766\\
68.6991	0.0971474939404919\\
68.7991	0.0301002136197823\\
68.8991	-0.00983042483401451\\
68.9991	0.170291019374198\\
69.0991	0.295371127018739\\
69.1991	0.196462810866371\\
69.2991	0.00355899249798133\\
69.3991	-0.000627126592424532\\
69.4991	0.232859667588341\\
69.5991	0.308501206651645\\
69.6991	0.225219709497889\\
69.7991	-0.0283101696356159\\
69.8991	-0.00686360683519169\\
69.999	0.11556470210961\\
70.099	0.205225478426908\\
70.199	0.159794041524256\\
70.2991	0.00198821127505452\\
70.3991	0.000676509223121818\\
70.4991	0.164601850129953\\
70.599	0.248854991463412\\
70.699	0.211941317390155\\
70.7991	0.0323892433472841\\
70.8991	-0.0368799721151623\\
70.9991	-0.0812221118858364\\
71.0991	0.164366968655759\\
71.1991	0.223785909105352\\
71.2991	0.0851887291228887\\
71.3991	-0.0865359514442484\\
71.4991	-0.044584257793077\\
71.5991	0.119300228769629\\
71.6991	0.207371618097043\\
71.7991	0.0888184785438464\\
71.8991	-0.105882235264155\\
71.9991	-0.117309043228891\\
72.0991	-0.0434702960734816\\
72.1991	0.145396048527127\\
72.2991	0.151625064195509\\
72.3991	-0.0292049181660471\\
72.4991	-0.180707668461673\\
72.5991	-0.0768992288974991\\
72.6991	0.133261579300052\\
72.7991	0.193485225013174\\
72.8991	0.022970500415445\\
72.9991	-0.134188406752419\\
73.0991	-0.13214778518265\\
73.1991	-0.0545717432926605\\
73.2991	0.0715611505486436\\
73.3991	0.0722062694243979\\
73.4991	0.0264202080156023\\
73.5991	-0.0186529703263953\\
73.6991	0.0755254357591431\\
73.7991	0.228370547844435\\
73.8991	0.558829230305104\\
73.9991	0.704968157457994\\
74.0991	0.517362366666947\\
74.1991	0.242779783739972\\
74.2991	0.139838467157679\\
74.3991	0.453421467248413\\
74.4991	0.673949798349493\\
74.5991	0.590765592808636\\
74.6991	0.335661246114675\\
74.7991	0.408783210707931\\
74.8991	0.642506032190615\\
74.9991	0.587187394827147\\
75.099	0.291265527037544\\
75.1991	0.1668743737199\\
75.2991	0.381448128057192\\
75.3991	0.587439679398396\\
75.4991	0.502241922698975\\
75.5991	0.35035453545923\\
75.6991	0.389865886676415\\
75.7991	0.590073666603833\\
75.8991	0.546774964806708\\
75.9991	0.303013521799687\\
76.0991	0.180399451771032\\
76.1991	0.38594057850745\\
76.2991	0.528678504975701\\
76.3991	0.448116830107734\\
76.4991	0.301617472753182\\
76.5991	0.414605023432313\\
76.6991	0.546133764978822\\
76.7991	0.498998741301498\\
76.8991	0.403776427770237\\
76.9991	0.267675793182628\\
77.0991	0.464048268187398\\
77.1991	0.512617564384663\\
77.2991	0.382898001114131\\
77.3991	0.23908956751472\\
77.4991	0.372528016599336\\
77.5991	0.563750654077229\\
77.6991	0.355955476078561\\
77.7991	0.266907207376401\\
77.8991	0.355160310785756\\
77.9991	0.530127804724971\\
78.0991	0.498819223947749\\
78.1991	0.346350113546508\\
78.2991	0.249877104016786\\
78.3991	0.515095294056303\\
78.4991	0.618288466703451\\
78.5991	0.35978563523172\\
78.6991	0.31903789269889\\
78.7991	0.411845164106709\\
78.8991	0.500338022279664\\
78.9991	0.39310411710236\\
79.0991	0.323486168549295\\
79.1992	0.361158702311743\\
79.2992	0.616323286175116\\
79.3992	0.510233339592545\\
79.4992	0.237082799973903\\
79.5991	0.166231888038616\\
79.6992	0.450424853432082\\
79.7992	0.46042097036657\\
79.8992	0.392010866168709\\
79.9991	0.370590439282357\\
80.0991	0.558811720936387\\
80.1991	0.564593488547164\\
80.2991	0.407162730613911\\
80.3991	0.251911319404966\\
80.4991	0.343055095361952\\
80.5992	0.515934476338039\\
80.6992	0.385513497508863\\
80.7991	0.341088008728845\\
80.8991	0.430033463448194\\
80.9992	0.49255599787863\\
81.0992	0.374506883838977\\
81.1992	0.207385639935031\\
81.2992	0.130855655543675\\
81.3991	0.23999794486133\\
81.4992	0.1084937556261\\
81.5992	-0.132224094338621\\
81.6992	-0.17520208614068\\
81.7992	-0.0365571608281109\\
81.8992	0.0440472184752163\\
81.9992	0.0508764018653961\\
82.0992	0.00762777362136577\\
82.1992	0.0325495969690072\\
82.2992	0.0212155468534142\\
82.3992	0.237768514381389\\
82.4992	0.549531185530376\\
82.5992	0.636245329561827\\
82.6992	0.384834954278336\\
82.7992	0.193804902587358\\
82.8992	0.424175857357377\\
82.9992	0.701217535282912\\
83.0992	0.689734188893217\\
83.1992	0.388396150274224\\
83.2992	0.30904059984684\\
83.3992	0.653665006351009\\
83.4992	0.766062559707776\\
83.5992	0.486450791931305\\
83.6992	0.104035023844526\\
83.7992	0.264913364446577\\
83.8992	0.571829043097161\\
83.9992	0.572335473233355\\
84.0992	0.240468868038986\\
84.1992	-0.0364906923431725\\
84.2992	0.312772092376389\\
84.3992	0.574492860161764\\
84.4992	0.513025429476274\\
84.5993	0.381002164412245\\
84.6992	0.474450987785274\\
84.7993	0.660838376658668\\
84.8993	0.4923645475501\\
84.9991	0.155038899925223\\
85.0991	0.302209131370923\\
85.1992	0.501869334159365\\
85.2992	0.519514484069709\\
85.3992	0.427356709984438\\
85.4992	0.297380923757203\\
85.5992	0.511704430265878\\
85.6992	0.442100745630232\\
85.7992	0.420335794327682\\
85.8992	0.218987128549352\\
85.9992	0.47757513718925\\
86.0992	0.47187298648456\\
86.1992	0.138848719753156\\
86.2992	-0.127993515248642\\
86.3992	0.0619315693273602\\
86.4992	0.253012273538566\\
86.5992	0.0997643007287435\\
86.6992	-0.176788647952257\\
86.7992	-0.214750105977079\\
86.8992	-0.0978683892127911\\
86.9992	0.131754054465219\\
87.0992	0.146163071220337\\
87.1992	-0.0531521121578742\\
87.2992	-0.353827719792259\\
87.3992	-0.103183633084971\\
87.4992	0.193045518522756\\
87.5992	0.233361272728951\\
87.6992	-0.0192122483413616\\
87.7992	-0.211540596418332\\
87.8992	-0.110373199535329\\
87.9992	0.10185781891973\\
88.0992	0.142958069264836\\
88.1992	0.0619678362117946\\
88.2992	-0.136847084813721\\
88.3993	-0.0542841771983371\\
88.4993	0.200433664339338\\
88.5992	0.272211115121353\\
88.6993	0.075843626913209\\
88.7993	-0.196357121215256\\
88.8993	-0.13080843450474\\
88.9993	0.0814743384833123\\
89.0993	0.227363819340893\\
89.1993	0.164892742984355\\
89.2993	-0.0635324725227796\\
89.3993	-0.107645555536719\\
89.4993	0.108229916482743\\
89.5993	0.179283234342954\\
89.6993	0.126999245228245\\
89.7993	-0.0650416581346481\\
89.8993	-0.128942418301038\\
89.9993	-0.0574033724074592\\
90.0993	0.0717593501944149\\
90.1993	0.164192265652279\\
90.2993	0.0631855444005761\\
90.3993	-0.145445054722309\\
90.4993	-0.0537153123704723\\
90.5993	0.142282970540864\\
90.6993	0.184276565124039\\
90.7993	0.0905831665860973\\
90.8993	-0.15336352461688\\
90.9993	-0.122543144716222\\
91.0993	-0.0788897848537602\\
91.1993	0.0752403044472574\\
91.2993	0.111209085164441\\
91.3993	-0.059588577890598\\
91.4993	-0.202337052958359\\
91.5993	-0.0307063455430146\\
91.6993	0.165290038079242\\
91.7993	0.212351099040269\\
91.8993	0.0514600534429472\\
91.9993	-0.0933625373243559\\
92.0993	-0.0924123383755172\\
92.1993	0.0694083302308101\\
92.2993	0.146180357937979\\
92.3993	0.103961225999356\\
92.4993	-0.153193878964071\\
92.5993	-0.204600700884031\\
92.6993	0.034902208121588\\
92.7993	0.163651768318537\\
92.8993	0.1577788301851\\
92.9994	-0.0650427699191831\\
93.0993	-0.19276077566976\\
93.1994	-0.0813514129954112\\
93.2994	0.101436346906374\\
93.3994	0.120866027998259\\
93.4994	-0.00511010639137782\\
93.5994	-0.115910297193249\\
93.6994	-0.0133612578552572\\
93.7994	0.136734807085984\\
93.8994	0.16603295335466\\
93.9994	-0.0165699302473294\\
94.0994	-0.202943461897783\\
94.1994	-0.219385598634449\\
94.2994	-0.00224646217383542\\
94.3994	0.178130281140818\\
94.4994	0.151691426025879\\
94.5994	-0.100755250295278\\
94.6994	-0.0907261303493365\\
94.7994	0.00720226362876095\\
94.8994	0.197346990248007\\
94.9993	0.238657521246865\\
95.0993	0.218911293636909\\
95.1993	0.254732807804956\\
95.2994	0.263527937187953\\
95.3994	0.412017992339771\\
95.4994	0.470357568440219\\
95.5994	0.285390514324587\\
95.6994	0.0850609304215089\\
95.7994	0.027488792351355\\
95.8994	0.144344764364546\\
95.9994	0.156817864905149\\
96.0994	0.0249847378726763\\
96.1994	-0.217667197281438\\
96.2994	-0.317539803456355\\
96.3994	-0.153864857677224\\
96.4994	-0.0719675454197697\\
96.5994	-0.249213030009124\\
96.6994	-0.394831525000279\\
96.7994	-0.239470923710374\\
96.8994	0.0491801356576751\\
96.9994	0.0109236622693426\\
97.0994	-0.0770617961460334\\
97.1994	-0.234479197009833\\
97.2994	-0.141865927773319\\
97.3994	0.094650100428822\\
97.4994	0.0890731231072982\\
97.5994	0.00525506650985362\\
97.6994	-0.183102378701491\\
97.7994	-0.0603407062933801\\
97.8994	0.0834537752630558\\
97.9994	0.104970304490087\\
98.0994	-0.0536927455336473\\
98.1995	-0.175503089917003\\
98.2995	-0.0894716966510501\\
98.3995	0.0593900789288099\\
98.4995	0.153715783332641\\
98.5995	0.127871588154486\\
98.6995	-0.097719655007754\\
98.7995	-0.194820014699324\\
98.8995	-0.165657382296034\\
98.9995	-0.109546203214869\\
99.0995	-0.135589113797754\\
99.1995	-0.206633321661956\\
99.2995	-0.200961240224952\\
99.3995	-0.111460313314027\\
99.4995	0.00403357290329992\\
99.5995	0.0306263550406445\\
99.6995	0.0242496646853131\\
99.7995	-0.00709238896177813\\
99.8995	0.0385599603353861\\
99.9994	0.0272385013916861\\
100.0994	-0.00150797226423774\\
100.1995	3.51202603517953e-05\\
100.2995	0.00732954680948918\\
100.3995	-0.00516000080715119\\
100.4995	-0.0019759791247556\\
100.5995	-0.00372969364011058\\
100.6995	0.000874115110587437\\
100.7995	-0.000110081059260703\\
100.8995	9.28580186951005e-05\\
100.9995	-0.000118230270494377\\
101.0995	-0.000260868340702402\\
101.1995	-3.28826932259252e-05\\
101.2995	-3.86563446223426e-05\\
101.3995	0.00040372591411686\\
101.4995	0.000195977735435342\\
101.5995	0.000139510921324946\\
101.6995	-0.000224975335907831\\
101.7995	-0.000294439817556088\\
101.8995	0.000139010140756306\\
101.9995	0.000336423784877718\\
102.0995	-1.30395245852976e-07\\
102.1995	0.000371698714182108\\
102.2995	0.000189014461558734\\
102.3995	0.000145861226901905\\
102.4995	0.00053070525381006\\
102.5995	-5.77315031198653e-05\\
102.6995	-0.000316190245510457\\
102.7995	-0.000449198890126372\\
102.8995	-0.000390547230257537\\
102.9995	0.000107764965811747\\
103.0995	7.36484881790796e-05\\
103.1996	1.58131580828417e-05\\
103.2996	-0.000222633718505863\\
103.3996	-0.000230834849598582\\
103.4996	-0.000169489149461018\\
103.5996	0.000139660572111034\\
103.6996	0.000353138679093413\\
103.7996	0.000121483832062091\\
103.8996	-0.000194896933883767\\
103.9996	0.000225479043560361\\
104.0996	-0.00240664833865936\\
104.1996	-0.00600206632226686\\
104.2996	-0.00261542414594715\\
104.3996	-0.00613831318718862\\
104.4996	-0.0475367984864148\\
104.5996	0.0200759581576459\\
104.6996	0.113751469830555\\
104.7996	0.0731774093351257\\
104.8996	0.0130527938340618\\
104.9996	-0.00549397280588691\\
105.0996	-0.024071765701623\\
105.1996	-0.0190764580650862\\
105.2996	-0.000628942661163642\\
105.3996	-0.00245920672677017\\
105.4996	-0.00259660160946841\\
105.5996	0.0015342888047902\\
105.6996	0.00128432671075347\\
105.7996	0.00130242415527275\\
105.8996	0.000895141373037894\\
105.9996	0.000668191191929545\\
106.0996	0.000414409631795301\\
106.1996	-0.000120294115852509\\
106.2996	-3.09642368639916e-05\\
106.3996	1.07145239595097e-05\\
106.4996	-0.000184035959308206\\
106.5996	-0.000272165754116326\\
106.6996	-6.70872686348898e-05\\
106.7996	-0.000196949053486819\\
106.8996	-3.9533586845308e-05\\
106.9996	7.56939260498691e-05\\
107.0996	-9.9926452869769e-05\\
107.1996	-0.00017830133571564\\
107.2996	-0.000129092243388677\\
107.3996	0.000258856677355959\\
107.4996	0.00038433173565167\\
107.5996	-7.15687309868628e-05\\
107.6996	-0.000225223565863115\\
107.7996	-6.20222180132694e-06\\
107.8996	-0.000210500320179509\\
107.9996	7.48923774979141e-05\\
108.0996	-6.84908318568575e-07\\
108.1997	-7.71742910090323e-05\\
108.2996	-8.88314643953841e-05\\
108.3997	0.000154094062562746\\
108.4997	2.85909685406364e-05\\
108.5997	-4.94307928737793e-05\\
108.6997	0.000140213944895005\\
108.7997	-0.000122383966240451\\
108.8997	-0.000175234508581917\\
108.9997	-0.000242927022500463\\
109.0997	-0.000242634087582257\\
109.1997	-0.000168096171333254\\
109.2997	-6.42098609780531e-05\\
109.3997	-0.000119633339761924\\
109.4997	-0.000412038399578443\\
109.5997	-0.000350301200609762\\
109.6997	-0.00019440076688354\\
109.7997	-2.44914655092615e-05\\
109.8997	-3.19731473841134e-05\\
109.9996	-6.13537732680756e-05\\
110.0996	-0.00041134102904564\\
110.1996	-7.69835422355482e-05\\
110.2996	-0.000616498634441188\\
110.3996	-0.00105503816045751\\
110.4996	-0.00142070052647006\\
110.5996	0.000991902451783416\\
110.6996	-0.000967425925579463\\
110.7997	-0.0173700828341233\\
110.8997	-0.00860845754006241\\
110.9997	0.00518669072112493\\
111.0997	0.200238339432385\\
111.1997	0.46620063771264\\
111.2997	0.624731928848136\\
111.3997	0.744685433704152\\
111.4997	0.901416582491533\\
111.5997	1.07713283688952\\
111.6997	1.17637013015356\\
111.7997	1.20527360527906\\
111.8997	1.13519669089885\\
111.9997	1.02858335646546\\
112.0997	0.90629157009523\\
112.1997	0.730010620844812\\
112.2997	0.524179759541811\\
112.3997	0.380521449294775\\
112.4997	0.279951573760677\\
112.5997	0.177520446693704\\
112.6997	0.120618827327851\\
112.7997	0.0224343947117589\\
112.8997	0.0206041757039254\\
112.9997	0.0170697843131362\\
113.0997	0.0109734627563853\\
113.1997	0.00435807003451443\\
113.2997	0.000984191453200149\\
113.3997	3.83148091852803e-06\\
113.4997	-0.000558074641515323\\
113.5997	-0.00013569225511535\\
113.6997	-0.000263857236157731\\
113.7998	3.24918916255056e-05\\
113.8998	-0.000130344649714148\\
113.9998	5.58552806618303e-05\\
114.0998	0.00044873817005078\\
114.1998	7.41203671139825e-05\\
114.2998	-0.000304788467084866\\
114.3998	-0.000289433835064034\\
114.4998	-8.86066951594844e-06\\
114.5998	-0.000157410071443457\\
114.6998	-9.76333587674415e-06\\
114.7998	0.000397960021109366\\
114.8998	-5.12123578391054e-05\\
114.9997	-0.000184486601824793\\
115.0997	-0.000173459899027241\\
115.1997	-0.000472019059785012\\
115.2997	-0.000138533359456576\\
115.3997	-0.000115099485444345\\
115.4997	-7.4293636130988e-05\\
115.5997	-0.000204176364320782\\
115.6997	3.72352888259927e-05\\
115.7998	4.81453247446127e-05\\
115.8997	2.53290094159331e-05\\
115.9997	-9.55835022422024e-05\\
116.0998	0.000412380926416942\\
116.1998	0.000437620179533249\\
116.2998	2.39439599394847e-05\\
116.3998	0.000204388505016087\\
116.4998	-8.51815828746606e-05\\
116.5998	-0.000586146722609032\\
116.6998	-0.000236635365259768\\
116.7998	4.77421770093637e-05\\
116.8998	5.45023533293125e-05\\
116.9998	2.73250120733058e-05\\
117.0998	-4.2455851292854e-05\\
117.1998	-9.95935332344782e-05\\
117.2998	-0.000329236528398652\\
117.3998	3.58384064686728e-05\\
117.4998	0.000276466440254932\\
117.5998	0.000110860951165397\\
117.6998	-4.08690723250944e-05\\
117.7998	-0.000148052567176966\\
117.8998	8.48187863251922e-05\\
117.9998	7.73904211361784e-05\\
118.0998	-0.000270846800506525\\
118.1998	0.000135736677125677\\
118.2998	-0.000230205800449054\\
118.3998	-0.000287226726796195\\
118.4998	-0.000101944116077882\\
118.5998	-5.77753773675211e-05\\
118.6998	-0.000161680423735372\\
118.7998	-0.000179218576797035\\
118.8998	0.000198377417074454\\
118.9998	0.000259796471101271\\
119.0999	0.000157187410380749\\
119.1999	0.000155350050384963\\
119.2999	0.000132330162921244\\
119.3999	-2.23479756811311e-06\\
119.4998	9.19127920709106e-05\\
119.5999	0.000178777387720399\\
119.6999	0.000126726137381934\\
119.7999	-1.52813244170245e-05\\
};
\addplot [color=black,dashed,forget plot]
  table[row sep=crcr]{%
41.6291	-2.2\\
41.6291	2.2\\
};
\addplot [color=black,dashed,forget plot]
  table[row sep=crcr]{%
45	-2.2\\
45	2.2\\
};
\addplot [color=black,dashed,forget plot]
  table[row sep=crcr]{%
60	-2.2\\
60	2.2\\
};
\addplot [color=black,dashed,forget plot]
  table[row sep=crcr]{%
73.3591	-2.2\\
73.3591	2.2\\
};
\end{axis}
\end{tikzpicture}% \\
  \tikzsetnextfilename{tikz-orientation}%
  % This file was created by matlab2tikz.
%
%The latest updates can be retrieved from
%  http://www.mathworks.com/matlabcentral/fileexchange/22022-matlab2tikz-matlab2tikz
%where you can also make suggestions and rate matlab2tikz.
%
\definecolor{mycolor1}{rgb}{0.00000,0.44700,0.74100}%
\definecolor{mycolor2}{rgb}{0.85000,0.32500,0.09800}%
\definecolor{mycolor3}{rgb}{0.92900,0.69400,0.12500}%
%
\begin{tikzpicture}

\begin{axis}[%
width=0.951\figurewidth,
height=\figureheight,
at={(0\figurewidth,0\figureheight)},
scale only axis,
xmin=0,
xmax=120,
xtick={  0,  10,  20,  30,  40,  50,  60,  70,  80,  90, 100, 110, 120},
xlabel={Time (seconds)},
ymin=-5,
ymax=25,
ylabel={Orientation (rad)},
axis background/.style={fill=white},
axis on top
]

\addplot[area legend,solid,draw=white!90!black,fill=white!90!black,forget plot]
table[row sep=crcr] {%
x	y\\
0.11	-5\\
4.6799	-5\\
4.6799	25\\
0.11	25\\
}--cycle;

\addplot[area legend,solid,draw=white!90!black,fill=white!90!black,forget plot]
table[row sep=crcr] {%
x	y\\
8.6097	-5\\
10.3196	-5\\
10.3196	25\\
8.6097	25\\
}--cycle;

\addplot[area legend,solid,draw=white!90!black,fill=white!90!black,forget plot]
table[row sep=crcr] {%
x	y\\
11.3996	-5\\
12.7295	-5\\
12.7295	25\\
11.3996	25\\
}--cycle;

\addplot[area legend,solid,draw=white!90!black,fill=white!90!black,forget plot]
table[row sep=crcr] {%
x	y\\
14.5695	-5\\
18.5394	-5\\
18.5394	25\\
14.5695	25\\
}--cycle;

\addplot[area legend,solid,draw=white!90!black,fill=white!90!black,forget plot]
table[row sep=crcr] {%
x	y\\
100.7695	-5\\
104.2096	-5\\
104.2096	25\\
100.7695	25\\
}--cycle;

\addplot[area legend,solid,draw=white!90!black,fill=white!90!black,forget plot]
table[row sep=crcr] {%
x	y\\
106.1596	-5\\
110.4496	-5\\
110.4496	25\\
106.1596	25\\
}--cycle;

\addplot[area legend,solid,draw=white!90!black,fill=white!90!black,forget plot]
table[row sep=crcr] {%
x	y\\
113.3997	-5\\
116.0098	-5\\
116.0098	25\\
113.3997	25\\
}--cycle;
\addplot [color=mycolor1,solid,forget plot]
  table[row sep=crcr]{%
0	2.70005772716004\\
0.1	2.71571637059392\\
0.2	2.85045386066523\\
0.3	3.0105552154614\\
0.4	3.13515856759789\\
0.5	3.13554844565737\\
0.6	3.13598210531579\\
0.7	3.13595311281562\\
0.8	3.13579185921114\\
0.9	3.13563324018923\\
1	3.13557306724333\\
1.0999	3.13551211678849\\
1.1999	3.13545210873815\\
1.2999	3.13537216222291\\
1.3999	3.13530142891235\\
1.4999	3.13518567711294\\
1.5999	3.13519900364079\\
1.6999	3.1351793611091\\
1.7999	3.13510246864054\\
1.8999	3.13505647451671\\
1.9999	3.13503238293507\\
2.0999	3.13496538187081\\
2.1999	3.13490830626572\\
2.2999	3.13487435237421\\
2.3999	3.13480731801769\\
2.4999	3.13478139287267\\
2.5999	3.13475761731696\\
2.6999	3.1347241205603\\
2.7999	3.1347025321126\\
2.8999	3.13464802549216\\
2.9999	3.13462212994071\\
3.0999	3.13452933059977\\
3.1999	3.13449349087048\\
3.2999	3.13448524725989\\
3.3999	3.13445490076681\\
3.4999	3.13438169158953\\
3.5999	3.13434129784671\\
3.6999	3.13425888566853\\
3.7999	3.13419754386678\\
3.8999	3.134178193269\\
3.9999	3.1341478705771\\
4.0999	3.13416286521118\\
4.1999	3.13416709239096\\
4.2999	3.13415143530082\\
4.3999	3.13416002328627\\
4.4999	3.1341455661174\\
4.5999	3.13411140700133\\
4.6999	3.13407814271095\\
4.7999	3.13411895691013\\
4.8998	3.1341076238677\\
4.9998	3.13409608800466\\
5.0998	3.13407265054615\\
5.1998	3.13404928621588\\
5.2998	3.13404706162959\\
5.3998	3.13400164436625\\
5.4997	3.13396806434354\\
5.5997	3.13393233658392\\
5.6998	3.13393929160473\\
5.7998	3.13395856255884\\
5.8998	3.1339358776505\\
5.9998	3.13392303119261\\
6.0997	3.13355934538218\\
6.1997	3.13017246970196\\
6.2997	3.12214002398113\\
6.3997	3.11713548006994\\
6.4997	3.18883600738549\\
6.5997	3.33686103117504\\
6.6997	3.4694551569733\\
6.7997	3.63737515511264\\
6.8997	3.897499825992\\
6.9997	4.15001158688452\\
7.0997	4.39361258107752\\
7.1997	4.52396925797112\\
7.2997	4.58214419463837\\
7.3997	4.61294765065286\\
7.4997	4.61562754484543\\
7.5997	4.62111384585165\\
7.6997	4.61892407716334\\
7.7997	4.62394530144139\\
7.8997	4.62883689230595\\
7.9997	4.63762743221954\\
8.0997	4.64576850775787\\
8.1997	4.65238114499091\\
8.2997	4.67619505947193\\
8.3997	4.7145214647023\\
8.4997	4.72236540820834\\
8.5997	4.74040805691552\\
8.6997	4.7409271501598\\
8.7997	4.74127124009236\\
8.8997	4.74152696953421\\
8.9997	4.74178034999466\\
9.0997	4.74174567103842\\
9.1997	4.7415700385048\\
9.2997	4.74155006887903\\
9.3997	4.74156795487617\\
9.4997	4.74145070728494\\
9.5997	4.74143966653447\\
9.6997	4.7415555283968\\
9.7997	4.74157487370317\\
9.8997	4.74166794895788\\
9.9996	4.74162150731594\\
10.0996	4.74161141162782\\
10.1996	4.74158416412176\\
10.2996	4.74161204032974\\
10.3996	4.74165490842235\\
10.4996	4.74161680630755\\
10.5996	4.74160870943192\\
10.6996	4.74160801736302\\
10.7996	4.74161124584154\\
10.8996	4.74168698262872\\
10.9996	4.74163667373736\\
11.0996	4.74169718914808\\
11.1996	4.7417404602689\\
11.2996	4.74170903607948\\
11.3996	4.74172108310177\\
11.4995	4.74167245479144\\
11.5995	4.74172678107576\\
11.6996	4.74177511915245\\
11.7996	4.7418007223844\\
11.8995	4.74183112584136\\
11.9996	4.74180059580652\\
12.0995	4.74181768439612\\
12.1995	4.74179407304841\\
12.2996	4.7417803889979\\
12.3995	4.74179336563917\\
12.4995	4.74181553076495\\
12.5995	4.74185965799879\\
12.6995	4.74186470599301\\
12.7995	4.74189867202212\\
12.8995	4.74195572006988\\
12.9995	4.7419643816968\\
13.0995	4.74194832345092\\
13.1995	4.7418681856958\\
13.2995	4.74158583013823\\
13.3995	4.74230767957896\\
13.4995	4.73392273728265\\
13.5995	4.71539052453252\\
13.6995	4.66273303182531\\
13.7995	4.61730158903572\\
13.8995	4.67821704833054\\
13.9995	4.90312635934529\\
14.0995	5.67656989496802\\
14.1995	6.50221314288185\\
14.2995	9.1527657561378\\
14.3995	9.0372106579698\\
14.4995	8.83206611301059\\
14.5995	8.63247541319349\\
14.6995	8.62549357435795\\
14.7995	8.62223252536651\\
14.8995	8.62221957050297\\
14.9994	8.62121970301694\\
15.0994	8.6281717746252\\
15.1994	8.63343475277344\\
15.2994	8.64006288088708\\
15.3994	8.6439347666055\\
15.4994	8.64480016232062\\
15.5994	8.64934611292149\\
15.6994	8.65561124735169\\
15.7994	8.66253336195958\\
15.8994	8.66861598174129\\
15.9994	8.67208165430442\\
16.0994	8.67520577708697\\
16.1994	8.67811864557131\\
16.2994	8.68104907094485\\
16.3994	8.68734017486972\\
16.4994	8.69451163351946\\
16.5994	8.70068595937289\\
16.6994	8.70503760386794\\
16.7994	8.70876276221062\\
16.8994	8.71384839961389\\
16.9994	8.71762961796428\\
17.0994	8.7219262292332\\
17.1994	8.72615435832429\\
17.2994	8.73080697022861\\
17.3994	8.73588238358504\\
17.4994	8.73897103894333\\
17.5994	8.74547790300406\\
17.6994	8.75045878322779\\
17.7994	8.75392549994841\\
17.8994	8.75717735295085\\
17.9994	8.76160597062976\\
18.0994	8.76457435380059\\
18.1994	8.76774004375197\\
18.2994	8.77086492377077\\
18.3994	8.77458872983001\\
18.4994	8.77597946495034\\
18.5994	8.665441588253\\
18.6994	7.31022872229701\\
18.7994	6.78426084574149\\
18.8994	6.21206407687741\\
18.9994	6.30433655755713\\
19.0994	6.40953042455696\\
19.1994	6.32703899082346\\
19.2994	6.32507453515736\\
19.3994	6.2791337357719\\
19.4994	6.27384839721492\\
19.5994	6.25107208937868\\
19.6994	6.19665675418485\\
19.7994	6.12095071386322\\
19.8994	5.85330621499063\\
19.9993	5.35535198139084\\
20.0993	4.74734560107204\\
20.1993	3.81521422534435\\
20.2993	3.72394763534002\\
20.3993	3.65220466804806\\
20.4993	3.61366452257353\\
20.5993	3.58370689335048\\
20.6993	3.53867039305607\\
20.7993	3.4969196835629\\
20.8993	3.41319974025079\\
20.9993	3.36597159044908\\
21.0993	3.38847356646168\\
21.1993	3.3709501294658\\
21.2993	3.35359652565375\\
21.3993	3.32793768976135\\
21.4993	3.30966162340679\\
21.5993	3.32020223893551\\
21.6993	3.3372188394699\\
21.7993	3.2925125043244\\
21.8993	3.29730130826482\\
21.9993	3.24432539923178\\
22.0993	3.22934170327089\\
22.1993	3.23608689941793\\
22.2993	3.26193434749263\\
22.3993	3.25277877259649\\
22.4993	3.23068014330324\\
22.5993	3.24762746265704\\
22.6993	3.26736996597822\\
22.7993	3.26485966611472\\
22.8993	3.26016015623982\\
22.9993	3.23648867000896\\
23.0993	3.21905848134682\\
23.1993	3.23898024793092\\
23.2993	3.24223427570722\\
23.3993	3.24545847152102\\
23.4993	3.223378312921\\
23.5993	3.23834710770599\\
23.6993	3.2780029884248\\
23.7993	3.28182352999459\\
23.8993	3.26135394254116\\
23.9993	3.23100413255264\\
24.0993	3.22203032421715\\
24.1993	3.23719694123304\\
24.2993	3.25006229671513\\
24.3993	3.28148986948022\\
24.4993	3.25323983883546\\
24.5993	3.26986981668725\\
24.6993	3.28227252324018\\
24.7993	3.23947171197229\\
24.8993	3.21671450311431\\
24.9992	3.21233216840278\\
25.0992	3.2178779229035\\
25.1992	3.25563614133763\\
25.2992	3.23344586423855\\
25.3992	3.21099745300516\\
25.4992	3.20418272136279\\
25.5992	3.24459880354159\\
25.6992	3.24749557126642\\
25.7992	3.21041868247107\\
25.8992	3.20199944946944\\
25.9992	3.21432110397561\\
26.0992	3.23271536303671\\
26.1992	3.26173196289879\\
26.2992	3.26699843067747\\
26.3992	3.24802090498257\\
26.4992	3.2559976292177\\
26.5992	3.28268970369657\\
26.6992	3.2789103616037\\
26.7992	3.27272053747724\\
26.8992	3.24743914150857\\
26.9992	3.23363188578491\\
27.0992	3.25766187762544\\
27.1992	3.25518578948439\\
27.2992	3.24693521761228\\
27.3992	3.22555536557677\\
27.4992	3.24065115773212\\
27.5992	3.27069516491235\\
27.6992	3.27622534642063\\
27.7992	3.27119350813497\\
27.8992	3.25943994715573\\
27.9992	3.26021623384294\\
28.0992	3.25847324310202\\
28.1992	3.25501306988423\\
28.2992	3.25056176617114\\
28.3992	3.24586479249253\\
28.4992	3.25718815520655\\
28.5992	3.271454899971\\
28.6992	3.29298945017806\\
28.7992	3.26587091876252\\
28.8992	3.28032643971615\\
28.9992	3.26671783155486\\
29.0992	3.23204501663901\\
29.1992	3.20436788811575\\
29.2992	3.18934494054941\\
29.3992	3.20443832172732\\
29.4992	3.21404271572288\\
29.5992	3.20431749684162\\
29.6992	3.2230986550152\\
29.7992	3.15321841964793\\
29.8992	3.16861816725044\\
29.9991	3.14685068781118\\
30.0991	3.16339883465016\\
30.1991	3.18045939019943\\
30.2991	3.15861758403314\\
30.3991	3.18291309915063\\
30.4991	3.20055730209457\\
30.5991	3.22032688719281\\
30.6991	3.20813836979997\\
30.7991	3.17407089648922\\
30.8991	3.14845727862015\\
30.9991	3.1672219945422\\
31.0991	3.19977639305695\\
31.1991	3.19705251954335\\
31.2991	3.21565972826488\\
31.3991	3.23640830892528\\
31.4991	3.26121948594574\\
31.5991	3.2744792084616\\
31.6991	3.23038390496933\\
31.7991	3.19894664379656\\
31.8991	3.17010477798343\\
31.9991	3.1909370768771\\
32.0991	3.19979668959104\\
32.1991	3.20618529210443\\
32.2991	3.20054760822222\\
32.3991	3.22420090403214\\
32.4991	3.24804056309683\\
32.5991	3.20483222045512\\
32.6991	3.18754661758508\\
32.7991	3.17379236235125\\
32.8991	3.17481116008317\\
32.9991	3.18653312435328\\
33.0991	3.19493104288519\\
33.1991	3.15670713872426\\
33.2991	3.18267492813587\\
33.3991	3.17529299289262\\
33.4991	3.18230739592299\\
33.5991	3.21003923623154\\
33.6991	3.22848673606103\\
33.7991	3.18103948433507\\
33.8991	3.23170852550682\\
33.9991	3.22785895748379\\
34.0991	3.22767355891908\\
34.1991	3.23828711817358\\
34.2991	3.21496650025247\\
34.3991	3.20725629674297\\
34.4991	3.20242328887964\\
34.5991	3.20400107548179\\
34.6991	3.23863837999547\\
34.7991	3.22582789119459\\
34.8991	3.21090007523836\\
34.9991	3.19009377690562\\
35.0991	3.1874360238543\\
35.1991	3.21281479521531\\
35.2991	3.17375826901402\\
35.3991	3.19481819802651\\
35.4991	3.2152386054754\\
35.5991	3.24618703905288\\
35.6991	3.26636181982389\\
35.7991	3.22842710847451\\
35.8991	3.2282856659094\\
35.9991	3.20460527037165\\
36.0991	3.23192721833617\\
36.1991	3.23151207577766\\
36.2991	3.22729428157631\\
36.3991	3.23442101726946\\
36.4991	3.24081779917426\\
36.5991	3.27303482116158\\
36.6991	3.25812299456027\\
36.7991	3.22996209432988\\
36.8991	3.220228995501\\
36.9991	3.2145066278854\\
37.0991	3.2372653903794\\
37.1991	3.21754518917913\\
37.2991	3.22320104040917\\
37.3991	3.21583413289329\\
37.4991	3.23426628927983\\
37.5991	3.25046512066962\\
37.6991	3.21955738601592\\
37.7991	3.22379865643634\\
37.8991	3.19089948247647\\
37.9991	3.19952259699859\\
38.0991	3.21935994637793\\
38.1991	3.21074259753583\\
38.2991	3.19952005357029\\
38.3991	3.19346079129497\\
38.4991	3.22335071230423\\
38.5991	3.24072797854638\\
38.6991	3.22065298535441\\
38.7991	3.26159728860877\\
38.8991	3.25098632532767\\
38.9991	3.25144824074281\\
39.0991	3.25416364449181\\
39.1991	3.24021201078261\\
39.2991	3.23878352708258\\
39.3991	3.23692358328433\\
39.4991	3.24965817429472\\
39.5991	3.23091692387884\\
39.6991	3.20487482483413\\
39.7991	3.19246488390342\\
39.8991	3.19428948219214\\
39.9991	3.21927502624444\\
40.0991	3.19622860534115\\
40.1991	3.19312030952557\\
40.2991	3.18068127070763\\
40.3991	3.19388128134594\\
40.499	3.21647207097472\\
40.5991	3.19445727061084\\
40.6991	3.1876653922258\\
40.7991	3.15001388150909\\
40.8991	3.13931917398905\\
40.9991	3.15436795883924\\
41.0991	3.16717148120891\\
41.1991	3.19882714870507\\
41.2991	3.20916213652188\\
41.399	3.21335611701583\\
41.4991	3.22600902931551\\
41.5991	3.23097908400016\\
41.6991	3.17910087175155\\
41.7991	3.1870770709507\\
41.8991	3.16462536868641\\
41.999	3.15716761697426\\
42.0991	3.16180190785327\\
42.1991	3.16554797411239\\
42.2991	3.17867305119856\\
42.3991	3.20920914363115\\
42.4991	3.23213194634513\\
42.5991	3.23847650123996\\
42.6991	3.25053592886507\\
42.7991	3.26731812283444\\
42.8991	3.26183290854032\\
42.9991	3.24488222931547\\
43.0991	3.23686876097706\\
43.1991	3.22810637777909\\
43.2991	3.22384363777153\\
43.3991	3.24151626959079\\
43.4991	3.25222222935559\\
43.5991	3.27053268215327\\
43.6991	3.27178340157676\\
43.7991	3.28167884390856\\
43.8991	3.30948243120333\\
43.9991	3.32864122923863\\
44.0991	3.31644487212672\\
44.1991	3.28830286652686\\
44.2991	3.27168735321451\\
44.3991	3.26298008160564\\
44.4991	3.25298716552364\\
44.5991	3.21421993986586\\
44.6991	3.16921758600663\\
44.7991	3.11736516060861\\
44.8991	3.09101247337321\\
44.999	3.06805000260359\\
45.099	3.0434862972151\\
45.199	3.03662688125583\\
45.299	3.01864312855137\\
45.399	3.00370278352832\\
45.499	3.03458526102294\\
45.599	3.06823820369994\\
45.699	3.06434591964876\\
45.799	3.06854112342051\\
45.899	3.11500143141775\\
45.999	3.18176500159467\\
46.099	3.28719540887304\\
46.199	3.22654013841165\\
46.299	3.21061012907878\\
46.399	3.16248013563938\\
46.499	3.15930128916137\\
46.599	3.16364876289819\\
46.699	3.18092356815455\\
46.799	3.18800758804948\\
46.899	3.20556920801114\\
46.999	3.23476346966132\\
47.099	3.24390515993635\\
47.199	3.24998043631064\\
47.299	3.24857811604586\\
47.399	3.25605971573072\\
47.499	3.26308362688905\\
47.599	3.26684917166167\\
47.699	3.26905181783712\\
47.799	3.28227461838424\\
47.899	3.29761205727968\\
47.999	3.31588226137388\\
48.099	3.32931810813351\\
48.199	3.31635821984098\\
48.299	3.28722602084704\\
48.399	3.26562150194961\\
48.4991	3.26486590601465\\
48.599	3.27706580731621\\
48.6991	3.27594862010485\\
48.7991	3.24288831514404\\
48.899	3.20810530347117\\
48.9991	3.19684040945487\\
49.0991	3.20728487823545\\
49.1991	3.21505791637408\\
49.2991	3.2081078582067\\
49.3991	3.19917214618486\\
49.499	3.21412496406483\\
49.5991	3.25679853997278\\
49.6991	3.30827773363297\\
49.7991	3.3266795527246\\
49.8991	3.29925371573573\\
49.999	3.25531314391309\\
50.099	3.25144864639162\\
50.199	3.26661635994807\\
50.299	3.27039972433147\\
50.399	3.24042359757878\\
50.499	3.20982535126561\\
50.599	3.21730458297772\\
50.699	3.23167266101527\\
50.799	3.2309769251928\\
50.899	3.16894717672305\\
50.999	3.11000843165026\\
51.099	3.11065609777523\\
51.199	3.14493541422649\\
51.299	3.1895493806593\\
51.399	3.21963756824064\\
51.499	3.22813422371022\\
51.599	3.26952366738968\\
51.699	3.31744152316571\\
51.799	3.36569237031168\\
51.899	3.35645789340454\\
51.999	3.29978441633218\\
52.099	3.23211914947022\\
52.199	3.21082948529775\\
52.299	3.20554483603879\\
52.399	3.17407761758146\\
52.499	3.13648790918951\\
52.599	3.11881406398506\\
52.699	3.13889490654195\\
52.799	3.18084707063408\\
52.899	3.19853679304529\\
52.999	3.17303159672511\\
53.099	3.1510687307513\\
53.199	3.15169194258652\\
53.2991	3.16735683417269\\
53.3991	3.17746589273365\\
53.4991	3.16958560334281\\
53.5991	3.14775157251452\\
53.6991	3.16375068489355\\
53.7991	3.19359270708681\\
53.8991	3.22273869308149\\
53.999	3.2077810782265\\
54.0991	3.17254666241079\\
54.1991	3.15420147643814\\
54.299	3.16122437319386\\
54.3991	3.16894835691924\\
54.4991	3.15421219015724\\
54.599	3.12006698421956\\
54.6991	3.11008368709108\\
54.7991	3.13300083667391\\
54.8991	3.16174028148465\\
54.999	3.17619034741852\\
55.099	3.19743696091615\\
55.199	3.21990101701101\\
55.299	3.2533969037455\\
55.399	3.28093398909989\\
55.499	3.30730297562572\\
55.599	3.31937155187449\\
55.699	3.3170211159205\\
55.799	3.2903232536755\\
55.899	3.26117847729297\\
55.999	3.24675990649328\\
56.099	3.27193056441524\\
56.199	3.30376474276535\\
56.299	3.31570075408322\\
56.399	3.3152729047713\\
56.499	3.31421216490283\\
56.599	3.31767652070822\\
56.699	3.29902598997585\\
56.799	3.28347294281681\\
56.899	3.2815369264772\\
56.999	3.28580964449986\\
57.099	3.32233629715131\\
57.199	3.33226304135177\\
57.299	3.31595678861308\\
57.399	3.33376175921167\\
57.4991	3.35489935210403\\
57.5991	3.13647442253486\\
57.6991	2.73238263970044\\
57.7991	2.58031153981766\\
57.8991	2.54854027588226\\
57.9991	2.55894086000952\\
58.0991	2.55891906894927\\
58.1991	2.47112625349836\\
58.2991	2.39004655402653\\
58.399	2.38546145436287\\
58.4991	2.45526324920931\\
58.5991	2.53465132602877\\
58.6991	2.46552019603059\\
58.7991	2.30053890801122\\
58.8991	2.16841725489972\\
58.9991	2.14682107428542\\
59.0991	2.01462535192772\\
59.1991	1.80557566091433\\
59.2991	1.7156442307818\\
59.3991	1.76482408699092\\
59.4991	1.74084764363325\\
59.5991	1.67219105815112\\
59.6991	1.54040767811879\\
59.7991	1.48880600292187\\
59.8991	1.57891599304857\\
59.999	1.62152199420087\\
60.099	1.6084529070247\\
60.199	1.69655415389293\\
60.299	1.72946174183075\\
60.399	1.74332555514968\\
60.499	1.71630114141419\\
60.599	1.63873343402042\\
60.699	1.58361418922151\\
60.799	1.54128738375567\\
60.899	1.54704502154043\\
60.999	1.62596678221508\\
61.099	1.68646259644655\\
61.199	1.73378481267135\\
61.299	1.7227553905316\\
61.399	1.75070294095048\\
61.499	1.73508554967038\\
61.599	1.66600451667815\\
61.699	1.58394963257872\\
61.799	1.50621059911892\\
61.899	1.4876498694124\\
61.999	1.56052261243797\\
62.099	1.64135754132168\\
62.199	1.6980612016242\\
62.299	1.71322196275589\\
62.399	1.74677757439024\\
62.499	1.70907013622456\\
62.599	1.65926266829728\\
62.699	1.57922907702344\\
62.799	1.50622308375871\\
62.899	1.479332204368\\
62.999	1.49986151810946\\
63.099	1.59614684989096\\
63.199	1.66741277634165\\
63.299	1.73031506609986\\
63.399	1.75558120778827\\
63.499	1.74356643098997\\
63.599	1.7055679294794\\
63.699	1.62122865015367\\
63.799	1.54042537910045\\
63.8991	1.49675177672987\\
63.999	1.47799791113455\\
64.099	1.56580431409578\\
64.1991	1.64322447506647\\
64.2991	1.71213227155606\\
64.3991	1.72330392109064\\
64.4991	1.75569384097091\\
64.599	1.71965519806182\\
64.6991	1.63485464406178\\
64.7991	1.55628250588769\\
64.8991	1.49208779619258\\
64.999	1.46999915061968\\
65.099	1.53157889610741\\
65.199	1.61451197288691\\
65.299	1.68316932328649\\
65.399	1.71066906611906\\
65.499	1.7537656812221\\
65.599	1.74461834271095\\
65.699	1.68832170001041\\
65.799	1.59870481575826\\
65.899	1.49364226627832\\
65.999	1.46190502097447\\
66.099	1.51036796164416\\
66.199	1.59036910924927\\
66.299	1.65989098678932\\
66.399	1.70735781172178\\
66.499	1.74377557714713\\
66.599	1.75092321894473\\
66.699	1.71772436362169\\
66.799	1.62042804279288\\
66.899	1.5188827379005\\
66.999	1.47354801151745\\
67.099	1.44529809098684\\
67.1991	1.47037499286523\\
67.299	1.40746303405687\\
67.3991	1.35071817957677\\
67.4991	1.31074721671148\\
67.5991	1.30212433046953\\
67.6991	1.6352193041813\\
67.7991	3.18563908726236\\
67.8991	3.81878406642255\\
67.9991	3.89955244645275\\
68.0991	3.96189163975288\\
68.1991	3.9964637977245\\
68.2991	4.00877060741462\\
68.3991	3.95692116241442\\
68.4991	3.86943578686717\\
68.5991	3.75845635703565\\
68.6991	3.55421992360036\\
68.7991	3.42676647658415\\
68.8991	3.46916346099176\\
68.9991	3.50806738683988\\
69.0991	3.52837389991875\\
69.1991	3.52341479290869\\
69.2991	3.48507253970362\\
69.3991	3.43622230109696\\
69.4991	3.42084949526239\\
69.5991	3.41627785431926\\
69.6991	3.42513204917613\\
69.7991	3.38882619561832\\
69.8991	3.37472597206703\\
69.999	3.3694646060476\\
70.099	3.3805727846809\\
70.199	3.39300950777023\\
70.2991	3.38410811490068\\
70.3991	3.37832858051886\\
70.4991	3.34237128948779\\
70.599	3.31401785250267\\
70.699	3.30187423061589\\
70.7991	3.28153715291585\\
70.8991	3.30507506338704\\
70.9991	3.26906304174081\\
71.0991	3.29533375846851\\
71.1991	3.30853660393745\\
71.2991	3.31402110847818\\
71.3991	3.28464588245349\\
71.4991	3.2744776516747\\
71.5991	3.26685020792246\\
71.6991	3.27524909038652\\
71.7991	3.31722408719125\\
71.8991	3.29663696568032\\
71.9991	3.30456703876679\\
72.0991	3.28772744692455\\
72.1991	3.29320708688376\\
72.2991	3.29719152044146\\
72.3991	3.27828369645343\\
72.4991	3.26357240699657\\
72.5991	3.2623548685861\\
72.6991	3.25085518402809\\
72.7991	3.24272676266868\\
72.8991	3.25404168712299\\
72.9991	3.25829885510775\\
73.0991	3.27034575256159\\
73.1991	3.25986143245396\\
73.2991	3.26676608968032\\
73.3991	3.26701225693229\\
73.4991	3.24183192915811\\
73.5991	3.2363445609461\\
73.6991	3.22462308661597\\
73.7991	3.21319418200455\\
73.8991	3.21845038678334\\
73.9991	3.22265953836254\\
74.0991	3.22407986816457\\
74.1991	3.21185646864793\\
74.2991	3.18622878553858\\
74.3991	3.18196108732049\\
74.4991	3.20048282994914\\
74.5991	3.22996297371797\\
74.6991	3.2217104091671\\
74.7991	3.19296774856109\\
74.8991	3.17977150174819\\
74.9991	3.19167257236764\\
75.099	3.18862121058643\\
75.1991	3.16768537874201\\
75.2991	3.14011292860429\\
75.3991	3.17384000963181\\
75.4991	3.17596362902873\\
75.5991	3.16692511294674\\
75.6991	3.15411793695941\\
75.7991	3.15761197436219\\
75.8991	3.16609559359558\\
75.9991	3.15159510554481\\
76.0991	3.13050086355944\\
76.1991	3.13575370700791\\
76.2991	3.15788421230293\\
76.3991	3.1828812205997\\
76.4991	3.1586672483952\\
76.5991	3.14985608145634\\
76.6991	3.16760532955505\\
76.7991	3.18608532339027\\
76.8991	3.18648199914545\\
76.9991	3.14983240288148\\
77.0991	3.15425766357925\\
77.1991	3.18183914835006\\
77.2991	3.19737747436258\\
77.3991	3.16283398715205\\
77.4991	3.15099373479761\\
77.5991	3.16564501317966\\
77.6991	3.1720660678135\\
77.7991	3.19922036348926\\
77.8991	3.19951015661276\\
77.9991	3.20765356648629\\
78.0991	3.22343070081344\\
78.1991	3.23885275262912\\
78.2991	3.19246697492443\\
78.3991	3.22163279479902\\
78.4991	3.2621565230509\\
78.5991	3.23514197896382\\
78.6991	3.20990259373655\\
78.7991	3.16853861838044\\
78.8991	3.18380592788973\\
78.9991	3.24367123020887\\
79.0991	3.26744772909021\\
79.1992	3.24647997594781\\
79.2992	3.25183973081412\\
79.3992	3.24262107746108\\
79.4992	3.23420921252937\\
79.5991	3.2142267945026\\
79.6992	3.22383485355654\\
79.7992	3.23396195414373\\
79.8992	3.26289449319917\\
79.9991	3.22518759668703\\
80.0991	3.24298035201538\\
80.1991	3.23639468252404\\
80.2991	3.24779461431784\\
80.3991	3.22429988650167\\
80.4991	3.20983790466214\\
80.5992	3.22148053650631\\
80.6992	3.24419656758755\\
80.7991	3.22598643777553\\
80.8991	3.21346000957364\\
80.9992	3.20121061932712\\
81.0992	3.1997535314455\\
81.1992	3.18025143580277\\
81.2992	3.16799325826119\\
81.3991	3.17783978269047\\
81.4992	3.19804260175948\\
81.5992	3.19858972556961\\
81.6992	3.23146068899668\\
81.7992	3.24669453443246\\
81.8992	3.23076932826346\\
81.9992	3.2645154225968\\
82.0992	3.28468638176461\\
82.1992	3.29700813320918\\
82.2992	3.28966898307106\\
82.3992	3.30626948511768\\
82.4992	3.34304391489886\\
82.5992	3.3870730174647\\
82.6992	3.34628825062129\\
82.7992	3.31051350794045\\
82.8992	3.3383596829508\\
82.9992	3.28769857357352\\
83.0992	3.33914393626005\\
83.1992	3.29824727494193\\
83.2992	3.26313142593921\\
83.3992	3.33115699834119\\
83.4992	3.3592946702773\\
83.5992	3.38164399499265\\
83.6992	3.31510736863704\\
83.7992	3.29737086027779\\
83.8992	3.31588712518675\\
83.9992	3.34331332878635\\
84.0992	3.34644816545163\\
84.1992	3.34624561873859\\
84.2992	3.37505784639301\\
84.3992	3.39063764632624\\
84.4992	3.38459266778485\\
84.5993	3.34124913445687\\
84.6992	3.35560664710386\\
84.7993	3.35806691169143\\
84.8993	3.35923317176884\\
84.9991	3.33028427431922\\
85.0991	3.32721924627694\\
85.1992	3.32944912407613\\
85.2992	3.32302883664032\\
85.3992	3.29276772737166\\
85.4992	3.26323668743663\\
85.5992	3.25661339987778\\
85.6992	3.24558722793436\\
85.7992	3.27099657088333\\
85.8992	3.24118637425952\\
85.9992	3.24802019420965\\
86.0992	3.24545292826215\\
86.1992	3.2275231409569\\
86.2992	3.20688634753183\\
86.3992	3.17679725205624\\
86.4992	3.15971791126602\\
86.5992	3.19938691016122\\
86.6992	3.19631479888488\\
86.7992	3.2277906356568\\
86.8992	3.22570552420998\\
86.9992	3.25455427444441\\
87.0992	3.28832126170263\\
87.1992	3.27434423244\\
87.2992	3.24992601761037\\
87.3992	3.26224364942722\\
87.4992	3.25612234962252\\
87.5992	3.27009282036295\\
87.6992	3.27659073419634\\
87.7992	3.29246268002833\\
87.8992	3.29120730505621\\
87.9992	3.28655506388417\\
88.0992	3.32716527345772\\
88.1992	3.35306109527121\\
88.2992	3.31907819577495\\
88.3993	3.28527060144928\\
88.4993	3.25360141069145\\
88.5992	3.27591358169035\\
88.6993	3.28973106433656\\
88.7993	3.28552528461384\\
88.8993	3.29105088168994\\
88.9993	3.29844372129064\\
89.0993	3.30205504617757\\
89.1993	3.30805581837214\\
89.2993	3.27138788120834\\
89.3993	3.26637250910647\\
89.4993	3.26059254703853\\
89.5993	3.25878344912987\\
89.6993	3.2504470124016\\
89.7993	3.22803767111148\\
89.8993	3.26184537203696\\
89.9993	3.27201534251441\\
90.0993	3.26644288070856\\
90.1993	3.28501239286081\\
90.2993	3.31146460815728\\
90.3993	3.2717376484899\\
90.4993	3.26367779566387\\
90.5993	3.24443876050293\\
90.6993	3.23420973731595\\
90.7993	3.24304959787128\\
90.8993	3.21321257762422\\
90.9993	3.24645138094832\\
91.0993	3.26768539164388\\
91.1993	3.28256299923242\\
91.2993	3.30246993280749\\
91.3993	3.30494939022369\\
91.4993	3.28125320111172\\
91.5993	3.27593565831792\\
91.6993	3.24757748487891\\
91.7993	3.24662460991326\\
91.8993	3.23727867900624\\
91.9993	3.24097842235818\\
92.0993	3.26176675679428\\
92.1993	3.26830119813229\\
92.2993	3.27970096939566\\
92.3993	3.29275697621632\\
92.4993	3.25658023076413\\
92.5993	3.25946839606462\\
92.6993	3.23880192438317\\
92.7993	3.23534710457401\\
92.8993	3.24558551644503\\
92.9994	3.2335194390543\\
93.0993	3.24212850703364\\
93.1994	3.22269922464958\\
93.2994	3.20801987117823\\
93.3994	3.24616862097851\\
93.4994	3.27872161169215\\
93.5994	3.28533577660578\\
93.6994	3.26483830628661\\
93.7994	3.23719081872805\\
93.8994	3.23160572049757\\
93.9994	3.24955155587929\\
94.0994	3.25996121230456\\
94.1994	3.26724900067134\\
94.2994	3.28244178933115\\
94.3994	3.31878740593489\\
94.4994	3.34888249277067\\
94.5994	3.32328087923519\\
94.6994	3.31329930788108\\
94.7994	3.29309530803696\\
94.8994	3.31708365138339\\
94.9993	3.34168898226412\\
95.0993	3.31755264562128\\
95.1993	3.30229309521358\\
95.2994	3.26706285863331\\
95.3994	3.22961601179812\\
95.4994	3.2209839789927\\
95.5994	3.18485469321741\\
95.6994	3.17562462652906\\
95.7994	3.15507975661494\\
95.8994	3.14153368139422\\
95.9994	3.14006480971711\\
96.0994	3.12936172787407\\
96.1994	3.13949461866403\\
96.2994	3.15926893857125\\
96.3994	3.1703112875541\\
96.4994	3.19196822862566\\
96.5994	3.19879011731891\\
96.6994	3.20476248602512\\
96.7994	3.21563790345118\\
96.8994	3.17007976279294\\
96.9994	3.17180519032517\\
97.0994	3.19104947928503\\
97.1994	3.19071503357902\\
97.2994	3.18163256073711\\
97.3994	3.18169959802496\\
97.4994	3.18989071002821\\
97.5994	3.20536883225637\\
97.6994	3.18047428363503\\
97.7994	3.15596625990955\\
97.8994	3.15532217986554\\
97.9994	3.1726624099255\\
98.0994	3.21281444564603\\
98.1995	3.22936411536054\\
98.2995	3.23799584540098\\
98.3995	3.26231435081409\\
98.4995	3.28993273357319\\
98.5995	3.26592575692718\\
98.6995	3.2410951231316\\
98.7995	3.26691427196469\\
98.8995	3.32192624240629\\
98.9995	3.39685448612721\\
99.0995	3.53051787420444\\
99.1995	3.72650360411009\\
99.2995	3.88203572218032\\
99.3995	4.00851130787778\\
99.4995	4.2321446060357\\
99.5995	4.36876092273989\\
99.6995	4.44560405051105\\
99.7995	4.54239677788583\\
99.8995	4.59722686837268\\
99.9994	4.65828491630474\\
100.0994	4.6987612442454\\
100.1995	4.70779303133549\\
100.2995	4.69694505232741\\
100.3995	4.71246432916539\\
100.4995	4.72041212395126\\
100.5995	4.70040722881805\\
100.6995	4.69790429713444\\
100.7995	4.71795460286937\\
100.8995	4.71774713955546\\
100.9995	4.71749861863495\\
101.0995	4.7169635836407\\
101.1995	4.71664823114745\\
101.2995	4.71669534063226\\
101.3995	4.71678542389864\\
101.4995	4.71683138470381\\
101.5995	4.71684375366029\\
101.6995	4.71679157436093\\
101.7995	4.71685001795506\\
101.8995	4.71695235679636\\
101.9995	4.7169998606506\\
102.0995	4.71703575017061\\
102.1995	4.71702754254492\\
102.2995	4.71703161596937\\
102.3995	4.71706731052321\\
102.4995	4.71717845525666\\
102.5995	4.71722758339954\\
102.6995	4.71731596364712\\
102.7995	4.71730842940739\\
102.8995	4.71734485035813\\
102.9995	4.71740246291315\\
103.0995	4.71740503835903\\
103.1996	4.71748518580519\\
103.2996	4.71747836418719\\
103.3996	4.71752457673083\\
103.4996	4.71753811019101\\
103.5996	4.71758329121602\\
103.6996	4.71757618142509\\
103.7996	4.71762264556585\\
103.8996	4.71767788586456\\
103.9996	4.71771152059113\\
104.0996	4.71778002584006\\
104.1996	4.71821377605562\\
104.2996	4.71882103228586\\
104.3996	4.71901838001523\\
104.4996	4.70176570619695\\
104.5996	4.7293943372123\\
104.6996	4.77817879604937\\
104.7996	4.9464004620116\\
104.8996	5.40898905745287\\
104.9996	5.92969535306602\\
105.0996	6.14740274428188\\
105.1996	6.27636838981355\\
105.2996	6.31141639952219\\
105.3996	6.32580563683451\\
105.4996	6.56719012304191\\
105.5996	6.79460484784984\\
105.6996	8.26400301190302\\
105.7996	9.34764042426668\\
105.8996	9.52928892372942\\
105.9996	9.5284865740946\\
106.0996	9.5368772126361\\
106.1996	9.53732083703543\\
106.2996	9.53818204250748\\
106.3996	9.5504419774901\\
106.4996	9.63336486534844\\
106.5996	9.6773433058956\\
106.6996	9.66469982677454\\
106.7996	9.65604516305913\\
106.8996	9.64620279037466\\
106.9996	9.65056710068992\\
107.0996	9.64381170495409\\
107.1996	9.63490733559577\\
107.2996	9.63039112044265\\
107.3996	9.6265229885713\\
107.4996	9.66872918074578\\
107.5996	9.65545990130842\\
107.6996	9.6495839784406\\
107.7996	9.63811895320758\\
107.8996	9.62248100846992\\
107.9996	9.60990893976202\\
108.0996	9.60122613456417\\
108.1997	9.59240803120201\\
108.2996	9.58698034126516\\
108.3997	9.58568973505556\\
108.4997	9.58134323314792\\
108.5997	9.57552801939333\\
108.6997	9.57196489568238\\
108.7997	9.56660383948001\\
108.8997	9.56188467562916\\
108.9997	9.55714682476344\\
109.0997	9.5529820498808\\
109.1997	9.55247603197164\\
109.2997	9.54998175309129\\
109.3997	9.54852963293873\\
109.4997	9.54400504035436\\
109.5997	9.54023937381011\\
109.6997	9.54066195516924\\
109.7997	9.53902128351993\\
109.8997	9.53423276623986\\
109.9996	9.53151661798909\\
110.0996	9.53018354998456\\
110.1996	9.52708867177795\\
110.2996	9.52484880746699\\
110.3996	9.52196586820351\\
110.4996	9.51284234319423\\
110.5996	9.85828389750743\\
110.6996	9.89244302138368\\
110.7997	10.1263643225692\\
110.8997	11.3413931603683\\
110.9997	11.4603681404966\\
111.0997	10.6450843066065\\
111.1997	10.1519705660895\\
111.2997	9.95756573810436\\
111.3997	9.80542560696052\\
111.4997	9.70537875881815\\
111.5997	9.59979596824171\\
111.6997	9.52827685918804\\
111.7997	9.46785684933833\\
111.8997	9.42493901994554\\
111.9997	9.40480288272202\\
112.0997	9.39842190373862\\
112.1997	9.41352011271422\\
112.2997	9.43592949400422\\
112.3997	9.4446610780468\\
112.4997	9.44876277310649\\
112.5997	9.45214330322742\\
112.6997	9.45419486339003\\
112.7997	9.4232498858842\\
112.8997	9.41860358236719\\
112.9997	9.39422104749006\\
113.0997	9.40258584222123\\
113.1997	9.41745833880244\\
113.2997	9.41753695214653\\
113.3997	9.41767446911616\\
113.4997	9.41769803214016\\
113.5997	9.41769075268626\\
113.6997	9.41772564959541\\
113.7998	9.41769665711839\\
113.8998	9.41773174770659\\
113.9998	9.41778951301774\\
114.0998	9.41777344179154\\
114.1998	9.41776586306768\\
114.2998	9.41782271255239\\
114.3998	9.4178238364067\\
114.4998	9.41788086317934\\
114.5998	9.41793769654874\\
114.6998	9.41796252831129\\
114.7998	9.41796673098197\\
114.8998	9.41799074958433\\
114.9997	9.41801482919047\\
115.0997	9.4179855281659\\
115.1997	9.41800913232956\\
115.2997	9.41802221802456\\
115.3997	9.41802582065237\\
115.4997	9.41804170867891\\
115.5997	9.41804440417346\\
115.6997	9.41805809969405\\
115.7998	9.41808188781461\\
115.8997	9.41806393541275\\
115.9997	9.41811936911782\\
116.0998	9.41816510391312\\
116.1998	9.41817901303986\\
116.2998	9.41826664192411\\
116.3998	9.41826957494676\\
116.4998	9.418314205826\\
116.5998	9.4183381378502\\
116.6998	9.41837396621974\\
116.7998	9.41836579674277\\
116.8998	9.4183384489841\\
116.9998	9.41837207090651\\
117.0998	9.41843875609598\\
117.1998	9.41846274915386\\
117.2998	9.41846452006805\\
117.3998	9.41850951802363\\
117.4998	9.41854330812182\\
117.5998	9.41860129693869\\
117.6998	9.41864716915088\\
117.7998	9.41866092165964\\
117.8998	9.418695782277\\
117.9998	9.41879474388884\\
118.0998	9.41877570584919\\
118.1998	9.4188108788287\\
118.2998	9.41885893502169\\
118.3998	9.41884089026302\\
118.4998	9.41888674534646\\
118.5998	9.41893158244834\\
118.6998	9.41895544112555\\
118.7998	9.418970293862\\
118.8998	9.41900580285733\\
118.9998	9.41901896215087\\
119.0999	9.41902060608348\\
119.1999	9.41905532151407\\
119.2999	9.41905926319833\\
119.3999	9.41909477922228\\
119.4998	9.41910681105521\\
119.5999	9.41908964822592\\
119.6999	9.41912470162388\\
119.7999	9.41911662560162\\
};
\addplot [color=mycolor2,solid,forget plot]
  table[row sep=crcr]{%
0	-0.517971719131706\\
0.1	-0.506108992360757\\
0.2	-0.324186168435578\\
0.3	-0.127465440340966\\
0.4	0.0028666908376974\\
0.5	0.00361088630201005\\
0.6	0.00434896300602196\\
0.7	0.00442048930500632\\
0.8	0.00421811664180379\\
0.9	0.00402199866852956\\
1	0.00397268533657161\\
1.0999	0.00390313395526299\\
1.1999	0.00392654771730053\\
1.2999	0.003832751166954\\
1.3999	0.0037611451512129\\
1.4999	0.00367697714733788\\
1.5999	0.00363366398238283\\
1.6999	0.0036522691573743\\
1.7999	0.00362612226386778\\
1.8999	0.00359860898700291\\
1.9999	0.00359181354153541\\
2.0999	0.00356209417853013\\
2.1999	0.00355419695170077\\
2.2999	0.00353543432864151\\
2.3999	0.00347444341481841\\
2.4999	0.00353958813681403\\
2.5999	0.00349986307751014\\
2.6999	0.00349593035107341\\
2.7999	0.00345972118913837\\
2.8999	0.0034546129201537\\
2.9999	0.00341588954502513\\
3.0999	0.0033983548578096\\
3.1999	0.00341455365171011\\
3.2999	0.00341513661056141\\
3.3999	0.0034024581184656\\
3.4999	0.00342204341093311\\
3.5999	0.00342120073748095\\
3.6999	0.00346027478137927\\
3.7999	0.00345946103363585\\
3.8999	0.00344758400378261\\
3.9999	0.00345772316626995\\
4.0999	0.00344882473812781\\
4.1999	0.00349326978679389\\
4.2999	0.00353793912093877\\
4.3999	0.00356561609248568\\
4.4999	0.00361307053144184\\
4.5999	0.0036428119837176\\
4.6999	0.00367229670866914\\
4.7999	0.00367097349832828\\
4.8998	0.00368053077970728\\
4.9998	0.00367867559015681\\
5.0998	0.00367696806727427\\
5.1998	0.00368545853320075\\
5.2998	0.00369351490396027\\
5.3998	0.0036709918016256\\
5.4997	0.00367844370163841\\
5.5997	0.00370974736818628\\
5.6998	0.00371839323422848\\
5.7998	0.00380136433144836\\
5.8998	0.00382043404806382\\
5.9998	0.00383020873594809\\
6.0997	0.00472449596045412\\
6.1997	-0.0066580749331069\\
6.2997	-0.00936886933229806\\
6.3997	0.00416653042522745\\
6.4997	-0.123539033460958\\
6.5997	-0.257920089111072\\
6.6997	-0.3470310040709\\
6.7997	-0.38637013444267\\
6.8997	-0.384932536275471\\
6.9997	-0.330461402438704\\
7.0997	-0.25941650722904\\
7.1997	-0.200892536212137\\
7.2997	-0.166945935770985\\
7.3997	-0.165740620869101\\
7.4997	-0.134301082596554\\
7.5997	-0.110988818954546\\
7.6997	-0.109873726084408\\
7.7997	-0.113551003292448\\
7.8997	-0.0913298243721541\\
7.9997	-0.0784582254133837\\
8.0997	-0.0731395255540517\\
8.1997	-0.0699051218854949\\
8.2997	-0.0742677788659025\\
8.3997	-0.0696586582206157\\
8.4997	-0.0553558938351634\\
8.5997	-0.054832951701484\\
8.6997	-0.0549159665915722\\
8.7997	-0.0549968803843653\\
8.8997	-0.0550801847137508\\
8.9997	-0.0551560148697616\\
9.0997	-0.055223419820706\\
9.1997	-0.0553200464865553\\
9.2997	-0.0553615200655987\\
9.3997	-0.0554358524881296\\
9.4997	-0.0555232972798386\\
9.5997	-0.0556014543154784\\
9.6997	-0.0556918986151912\\
9.7997	-0.05581555501434\\
9.8997	-0.0559084089053825\\
9.9996	-0.0560558196016519\\
10.0996	-0.0561638163522507\\
10.1996	-0.0562489729538463\\
10.2996	-0.0563645350930243\\
10.3996	-0.056458053201185\\
10.4996	-0.0565760633323647\\
10.5996	-0.0566419827219131\\
10.6996	-0.056740164181798\\
10.7996	-0.0568357520003877\\
10.8996	-0.0569164012887429\\
10.9996	-0.0570451155684312\\
11.0996	-0.0571402861991652\\
11.1996	-0.0572219334098846\\
11.2996	-0.0573629701088506\\
11.3996	-0.0574708268948815\\
11.4995	-0.0575121938990287\\
11.5995	-0.0576520859821387\\
11.6996	-0.0577350772619442\\
11.7996	-0.0578297903147169\\
11.8995	-0.0580020383917342\\
11.9996	-0.0581090734986197\\
12.0995	-0.05819200549677\\
12.1995	-0.058299252429973\\
12.2996	-0.0583630338962481\\
12.3995	-0.0584464763219004\\
12.4995	-0.0585514070704626\\
12.5995	-0.0586678731022827\\
12.6995	-0.0587978054764353\\
12.7995	-0.0589754874857401\\
12.8995	-0.0590575315328766\\
12.9995	-0.0591632285641412\\
13.0995	-0.0592702194889315\\
13.1995	-0.0593621453879767\\
13.2995	-0.0594397806948063\\
13.3995	-0.059315529053165\\
13.4995	-0.0724016606732691\\
13.5995	-0.182262814241587\\
13.6995	-0.52125525073843\\
13.7995	-0.908996691399662\\
13.8995	-1.23837318115223\\
13.9995	-1.45588613245643\\
14.0995	-1.54457075255646\\
14.1995	-1.56478082259044\\
14.2995	-1.5562248095642\\
14.3995	-1.53933891326118\\
14.4995	-1.54350839975821\\
14.5995	-1.54941096213956\\
14.6995	-1.5495850409576\\
14.7995	-1.54961456340087\\
14.8995	-1.54962464084474\\
14.9994	-1.54966807849195\\
15.0994	-1.54962134258112\\
15.1994	-1.54955282009961\\
15.2994	-1.54945248058314\\
15.3994	-1.54948648361615\\
15.4994	-1.54948513208047\\
15.5994	-1.54944410141939\\
15.6994	-1.5493779001356\\
15.7994	-1.54934084917284\\
15.8994	-1.54927271708623\\
15.9994	-1.54917803340242\\
16.0994	-1.54911992846761\\
16.1994	-1.54911477104841\\
16.2994	-1.54912342652439\\
16.3994	-1.54906045079962\\
16.4994	-1.54901026305694\\
16.5994	-1.54898279476735\\
16.6994	-1.54893575637931\\
16.7994	-1.54890417027699\\
16.8994	-1.5488308963546\\
16.9994	-1.54874435647269\\
17.0994	-1.54865574634431\\
17.1994	-1.54860511433162\\
17.2994	-1.54857558588214\\
17.3994	-1.54849993310431\\
17.4994	-1.54844424381109\\
17.5994	-1.5484080139783\\
17.6994	-1.54839711976588\\
17.7994	-1.54838316780612\\
17.8994	-1.54830228392306\\
17.9994	-1.54823913773858\\
18.0994	-1.54816386388653\\
18.1994	-1.54813382132103\\
18.2994	-1.5480888625213\\
18.3994	-1.54799896211533\\
18.4994	-1.54796062120104\\
18.5994	-1.55070728713165\\
18.6994	-1.55479168986456\\
18.7994	-1.53146559562085\\
18.8994	-1.52753929125219\\
18.9994	-1.48411411074909\\
19.0994	-1.4030580498255\\
19.1994	-1.33926383132372\\
19.2994	-1.2730964296325\\
19.3994	-1.17933298692168\\
19.4994	-1.08641218853107\\
19.5994	-1.03981421930885\\
19.6994	-1.04145326783273\\
19.7994	-1.11359080278306\\
19.8994	-1.23021315930282\\
19.9993	-1.33099561023891\\
20.0993	-1.3987482776111\\
20.1993	-1.30479227437331\\
20.2993	-1.16883464043325\\
20.3993	-1.05337700876513\\
20.4993	-0.956902377238823\\
20.5993	-0.84475683824902\\
20.6993	-0.786725771941926\\
20.7993	-0.730180552426217\\
20.8993	-0.705846481025437\\
20.9993	-0.616459529356398\\
21.0993	-0.530172663723741\\
21.1993	-0.50380301993218\\
21.2993	-0.481444177419944\\
21.3993	-0.436220452592309\\
21.4993	-0.373596450843164\\
21.5993	-0.354231566722846\\
21.6993	-0.326835179093975\\
21.7993	-0.332434058787243\\
21.8993	-0.311315438720196\\
21.9993	-0.299559899170993\\
22.0993	-0.2855680825462\\
22.1993	-0.270322482072485\\
22.2993	-0.259518144265412\\
22.3993	-0.199926770314247\\
22.4993	-0.099487999683823\\
22.5993	-0.0927506411636756\\
22.6993	-0.120403108210076\\
22.7993	-0.153979039201436\\
22.8993	-0.10929711334978\\
22.9993	-0.104002226203965\\
23.0993	-0.112021691100167\\
23.1993	-0.128522310387614\\
23.2993	-0.178446113446371\\
23.3993	-0.144901410146701\\
23.4993	-0.124580442826882\\
23.5993	-0.107806227589605\\
23.6993	-0.104196860253504\\
23.7993	-0.1022657209443\\
23.8993	-0.0933157477012704\\
23.9993	-0.0822983858800112\\
24.0993	-0.057882425762515\\
24.1993	-0.0592258880727689\\
24.2993	-0.0579471583197779\\
24.3993	-0.0347952303658639\\
24.4993	-0.0306305197783794\\
24.5993	-0.0457803338982916\\
24.6993	-0.0534677967115041\\
24.7993	-0.0378171683871057\\
24.8993	-0.0157480636066879\\
24.9992	-0.00317968419469676\\
25.0992	-0.0337021914429911\\
25.1992	-0.0515994225106338\\
25.2992	-0.0720426704992994\\
25.3992	-0.0461872587508386\\
25.4992	-0.029546227846397\\
25.5992	-0.0276485044558009\\
25.6992	-0.0229185818009867\\
25.7992	-0.0224242836163304\\
25.8992	-0.0458107514761252\\
25.9992	-0.0494136492568655\\
26.0992	-0.0463324178833759\\
26.1992	-0.0654425787429638\\
26.2992	-0.0505710421138977\\
26.3992	-0.0325966283039208\\
26.4992	-0.0275003682940291\\
26.5992	-0.0329844703753182\\
26.6992	-0.0372231549701619\\
26.7992	-0.0392149749721055\\
26.8992	-0.032419245988124\\
26.9992	-0.027538841806258\\
27.0992	-0.0463929465545287\\
27.1992	-0.0494134898378681\\
27.2992	-0.0100877964469354\\
27.3992	0.0165728229509246\\
27.4992	0.0120518998611302\\
27.5992	0.00979018182087152\\
27.6992	0.00815905710360579\\
27.7992	0.00889863684286425\\
27.8992	0.00337681349936446\\
27.9992	-0.00310765165576941\\
28.0992	-0.00514782145482505\\
28.1992	-0.0231998441566328\\
28.2992	-0.0127261871538831\\
28.3992	0.00397861093295574\\
28.4992	0.0195547985111108\\
28.5992	0.0232207137245692\\
28.6992	0.022857963713324\\
28.7992	0.0144414123703887\\
28.8992	0.00267898779684773\\
28.9992	-0.00448001231269751\\
29.0992	-0.0112957403400395\\
29.1992	-0.00487143504125461\\
29.2992	-0.0214610777582906\\
29.3992	-0.00383865873494316\\
29.4992	0.0256087312632109\\
29.5992	0.0119168651948072\\
29.6992	0.0113765489665696\\
29.7992	0.0391530541862278\\
29.8992	0.0224714981569392\\
29.9991	0.012899564648416\\
30.0991	0.0224455076348832\\
30.1991	0.0214840254340873\\
30.2991	0.0206624721639533\\
30.3991	0.0306268666195155\\
30.4991	0.0235111044291664\\
30.5991	0.0320329693940864\\
30.6991	0.0337190543120168\\
30.7991	0.0497727336738228\\
30.8991	0.0461028886691173\\
30.9991	0.0540803800474511\\
31.0991	0.0597865671814774\\
31.1991	0.0426868465068279\\
31.2991	0.0473737557054176\\
31.3991	0.0469750084167016\\
31.4991	0.0379545417678864\\
31.5991	0.0299811090325284\\
31.6991	0.0385183600338181\\
31.7991	0.0287134750747565\\
31.8991	0.0395131366995006\\
31.9991	0.0610842833651645\\
32.0991	0.046205645653642\\
32.1991	0.0214520994321244\\
32.2991	0.0415550876384285\\
32.3991	0.0418095927739678\\
32.4991	0.041578387306887\\
32.5991	0.0425837561063358\\
32.6991	0.0575379112353018\\
32.7991	0.0584180079425723\\
32.8991	0.0589589727950854\\
32.9991	0.0672483472708318\\
33.0991	0.0687559662683002\\
33.1991	0.0456322457109491\\
33.2991	0.056259015471439\\
33.3991	0.0693928222042623\\
33.4991	0.0651414290957165\\
33.5991	0.0520009918720146\\
33.6991	0.0744307958049703\\
33.7991	0.0830291253369599\\
33.8991	0.0623786791359491\\
33.9991	0.0598699901272741\\
34.0991	0.055770695886979\\
34.1991	0.0713690214529938\\
34.2991	0.0655850013197658\\
34.3991	0.0587284667474516\\
34.4991	0.0693338525887291\\
34.5991	0.0681637448877184\\
34.6991	0.0591059157443986\\
34.7991	0.0563099229644776\\
34.8991	0.0502369373604382\\
34.9991	0.0510971026658671\\
35.0991	0.0580086017917989\\
35.1991	0.0727397358344596\\
35.2991	0.0589892417893213\\
35.3991	0.0591069969674343\\
35.4991	0.0622258158697348\\
35.5991	0.0494394971973217\\
35.6991	0.0484611412257928\\
35.7991	0.054517134251043\\
35.8991	0.0400691398363354\\
35.9991	0.0546524033810277\\
36.0991	0.0570241049437528\\
36.1991	0.0456458709904709\\
36.2991	0.0306435170999157\\
36.3991	0.0502368318293639\\
36.4991	0.0460933811492218\\
36.5991	0.0327095641833061\\
36.6991	0.0279582159252076\\
36.7991	0.0450260500096062\\
36.8991	0.0169969860727347\\
36.9991	0.02235197825315\\
37.0991	0.0422130368649511\\
37.1991	0.0277212425549142\\
37.2991	0.0350503435635845\\
37.3991	0.0586318191231818\\
37.4991	0.0418253209283658\\
37.5991	0.0450546102121109\\
37.6991	0.0476418652477464\\
37.7991	0.0306283039219144\\
37.8991	0.0277764495570971\\
37.9991	0.0301081227257238\\
38.0991	0.0285652099031547\\
38.1991	0.0325954399253048\\
38.2991	0.0473580995629099\\
38.3991	0.0586243730094417\\
38.4991	0.0401582378081765\\
38.5991	0.0258988317172659\\
38.6991	0.0351660588658493\\
38.7991	0.0208082904822243\\
38.8991	0.0267883270452878\\
38.9991	0.0264764359219024\\
39.0991	0.0370891610677469\\
39.1991	0.0122966782571019\\
39.2991	0.0183889538981648\\
39.3991	0.011564367094019\\
39.4991	0.0109092120746706\\
39.5991	0.00601278533327617\\
39.6991	0.0186750041034126\\
39.7991	0.0170595540236127\\
39.8991	0.0459581431534496\\
39.9991	0.0532139057070945\\
40.0991	0.0318753977688715\\
40.1991	0.0142009345747535\\
40.2991	0.0271419753022664\\
40.3991	0.0214560791859268\\
40.499	0.030591203039296\\
40.5991	0.0347175337591183\\
40.6991	0.0033104244226019\\
40.7991	0.027349037717407\\
40.8991	0.0375644457048377\\
40.9991	0.036154218411022\\
41.0991	0.0123846531309047\\
41.1991	0.00527209704824519\\
41.2991	0.00756526972159418\\
41.399	-0.0105857046410411\\
41.4991	-0.016095365593406\\
41.5991	-0.0193255600208254\\
41.6991	0.0193854856658254\\
41.7991	0.0113023503839043\\
41.8991	-0.00636658124991305\\
41.999	-0.0194817616216778\\
42.0991	-0.0261814161015768\\
42.1991	-0.0399608825097745\\
42.2991	-0.0431649167999723\\
42.3991	-0.0595309272177957\\
42.4991	-0.0408094456123647\\
42.5991	-0.0461459391789894\\
42.6991	-0.056215202497716\\
42.7991	-0.0650683574040291\\
42.8991	-0.0616527252129435\\
42.9991	-0.0591687262575384\\
43.0991	-0.0717224907670907\\
43.1991	-0.0726655981872526\\
43.2991	-0.097982375294473\\
43.3991	-0.117888937787529\\
43.4991	-0.119317171223666\\
43.5991	-0.10384216849757\\
43.6991	-0.0966721215147413\\
43.7991	-0.105517495538704\\
43.8991	-0.108270766529312\\
43.9991	-0.100096707116052\\
44.0991	-0.0998763660784146\\
44.1991	-0.118348625708666\\
44.2991	-0.116464059355282\\
44.3991	-0.117378581304592\\
44.4991	-0.103517766988341\\
44.5991	-0.0630984400750377\\
44.6991	0.0115497173822597\\
44.7991	0.0731246398167339\\
44.8991	0.100110085235537\\
44.999	0.10298096753307\\
45.099	0.13984120057955\\
45.199	0.171381571880834\\
45.299	0.214771744976258\\
45.399	0.182483505469726\\
45.499	0.1795819526826\\
45.599	0.131129441279822\\
45.699	0.131006423846392\\
45.799	0.0859792441234457\\
45.899	0.184724230841458\\
45.999	0.244004167736126\\
46.099	0.420146399201618\\
46.199	0.313592161222176\\
46.299	-0.0188585445870424\\
46.399	-0.0713946633266264\\
46.499	-0.075799058058959\\
46.599	-0.0733384263035594\\
46.699	-0.0580919775503161\\
46.799	-0.0665511995306026\\
46.899	-0.0888905190796695\\
46.999	-0.145768808033756\\
47.099	-0.217758286728694\\
47.199	-0.281511904649551\\
47.299	-0.308392994857763\\
47.399	-0.322410321194329\\
47.499	-0.309430856257894\\
47.599	-0.282716749241426\\
47.699	-0.256063710837523\\
47.799	-0.203356806696889\\
47.899	-0.179542830661747\\
47.999	-0.168372577141451\\
48.099	-0.175528272262827\\
48.199	-0.175535166276411\\
48.299	-0.157198390067449\\
48.399	-0.15982082901269\\
48.4991	-0.179943854884001\\
48.599	-0.211388838988926\\
48.6991	-0.233009772139351\\
48.7991	-0.236300284344093\\
48.899	-0.226240595796987\\
48.9991	-0.217391124789693\\
49.0991	-0.215844593382462\\
49.1991	-0.21776528405132\\
49.2991	-0.214925223866255\\
49.3991	-0.206433302037907\\
49.499	-0.184174220862777\\
49.5991	-0.203319808679053\\
49.6991	-0.236541308554526\\
49.7991	-0.255691972195264\\
49.8991	-0.247519714415311\\
49.999	-0.225586916548432\\
50.099	-0.220362852337656\\
50.199	-0.220298087440793\\
50.299	-0.23484209616737\\
50.399	-0.234581243078552\\
50.499	-0.225332037872032\\
50.599	-0.231923194801022\\
50.699	-0.257056377610856\\
50.799	-0.277631868359727\\
50.899	-0.26145910537728\\
50.999	-0.216381266027694\\
51.099	-0.187495061225364\\
51.199	-0.19127838981514\\
51.299	-0.207872060901808\\
51.399	-0.208378443629223\\
51.499	-0.19539205591557\\
51.599	-0.184991548358389\\
51.699	-0.209081553135503\\
51.799	-0.242072545549901\\
51.899	-0.270749633962603\\
51.999	-0.277463721762047\\
52.099	-0.261379690023243\\
52.199	-0.249854208234528\\
52.299	-0.2564743886904\\
52.399	-0.252804871081667\\
52.499	-0.22784006925251\\
52.599	-0.204426916995622\\
52.699	-0.213972822970237\\
52.799	-0.23893201158012\\
52.899	-0.246531516153466\\
52.999	-0.227583676407393\\
53.099	-0.210668698932279\\
53.199	-0.202588564313617\\
53.2991	-0.204359058499486\\
53.3991	-0.216450381419928\\
53.4991	-0.216761163276619\\
53.5991	-0.199349536459137\\
53.6991	-0.196633166800117\\
53.7991	-0.217786298054771\\
53.8991	-0.241359600259968\\
53.999	-0.233887108833466\\
54.0991	-0.214711066578362\\
54.1991	-0.196796210026218\\
54.299	-0.192755758172114\\
54.3991	-0.208703199041279\\
54.4991	-0.200174761314281\\
54.599	-0.196170272928821\\
54.6991	-0.198080018631183\\
54.7991	-0.220906052674014\\
54.8991	-0.240813383353016\\
54.999	-0.233267376157856\\
55.099	-0.215026395542206\\
55.199	-0.195295793306517\\
55.299	-0.176076635947291\\
55.399	-0.1841574745357\\
55.499	-0.186217709746378\\
55.599	-0.184823420823401\\
55.699	-0.179787802698948\\
55.799	-0.19265335246006\\
55.899	-0.226330347958142\\
55.999	-0.277783019156689\\
56.099	-0.333285804655094\\
56.199	-0.357394749757074\\
56.299	-0.338341800604348\\
56.399	-0.304986144142461\\
56.499	-0.270752985932626\\
56.599	-0.241213559678426\\
56.699	-0.205622975638458\\
56.799	-0.169040537419118\\
56.899	-0.136419173826265\\
56.999	-0.123384832692909\\
57.099	-0.0847999403929419\\
57.199	-0.127295836728646\\
57.299	-0.129371735979704\\
57.399	-0.111716052753684\\
57.4991	-0.13496525603974\\
57.5991	-0.155152583374243\\
57.6991	-0.111932623445115\\
57.7991	-0.202641195768389\\
57.8991	-0.280372505947142\\
57.9991	-0.333124371680956\\
58.0991	-0.35032689014861\\
58.1991	-0.365498923291631\\
58.2991	-0.414841323092044\\
58.399	-0.441031545083371\\
58.4991	-0.484568838611874\\
58.5991	-0.541695401917583\\
58.6991	-0.68312875783946\\
58.7991	-0.785102270110739\\
58.8991	-0.888835598686855\\
58.9991	-0.951095234292321\\
59.0991	-0.972862923727402\\
59.1991	-0.99910390541383\\
59.2991	-1.0042792472382\\
59.3991	-1.07318926111566\\
59.4991	-1.06458020810495\\
59.5991	-1.02774836434425\\
59.6991	-0.956262060152292\\
59.7991	-0.712406247525704\\
59.8991	-0.526669142615633\\
59.999	-0.349839863261904\\
60.099	-0.18431632083281\\
60.199	-0.196841956230207\\
60.299	-0.182772862228316\\
60.399	-0.222111868121316\\
60.499	-0.179933537534756\\
60.599	-0.104107089152048\\
60.699	-0.0523174222575404\\
60.799	0.0529512194563634\\
60.899	0.0615363627721769\\
60.999	-0.0297837530635385\\
61.099	-0.149894573444468\\
61.199	-0.205455519592874\\
61.299	-0.161380745874099\\
61.399	-0.215151996177521\\
61.499	-0.19864455585148\\
61.599	-0.109746344885567\\
61.699	-0.0139529954927944\\
61.799	0.104907636432438\\
61.899	0.14419951027625\\
61.999	0.0226529077781026\\
62.099	-0.121362902441561\\
62.199	-0.236454759149425\\
62.299	-0.206984299459032\\
62.399	-0.21047720932813\\
62.499	-0.218487564196573\\
62.599	-0.140031916869348\\
62.699	-0.058265306908183\\
62.799	0.0443620996272989\\
62.899	0.11083047406921\\
62.999	0.0487127844253541\\
63.099	-0.0877546409307253\\
63.199	-0.23136290529496\\
63.299	-0.244451155182089\\
63.399	-0.191730612446961\\
63.499	-0.236242549717115\\
63.599	-0.175400491711313\\
63.699	-0.0864951525822058\\
63.799	0.0143483688487845\\
63.8991	0.109178978673099\\
63.999	0.085263258457416\\
64.099	-0.0554531176666961\\
64.1991	-0.220174122575047\\
64.2991	-0.280601110227017\\
64.3991	-0.219438645595555\\
64.4991	-0.253524661087422\\
64.599	-0.230349477233936\\
64.6991	-0.136717551427896\\
64.7991	-0.0451013808260092\\
64.8991	0.0545044274025737\\
64.999	0.0923302184122713\\
65.099	-0.0339590349362444\\
65.199	-0.205303747876981\\
65.299	-0.311974116046908\\
65.399	-0.263258099090321\\
65.499	-0.2584790410718\\
65.599	-0.266371315192946\\
65.699	-0.167484574669774\\
65.799	-0.0740632335432492\\
65.899	0.0153010264813282\\
65.999	0.0740304018485236\\
66.099	-0.00738654493859107\\
66.199	-0.164819296417849\\
66.299	-0.32319251507406\\
66.399	-0.335042511107551\\
66.499	-0.287199450160102\\
66.599	-0.300388874738065\\
66.699	-0.194562736164521\\
66.799	-0.0937250192702672\\
66.899	0.0631692440141289\\
66.999	0.155485212855159\\
67.099	0.158271634987535\\
67.1991	0.0227963426393785\\
67.299	-0.112611107794075\\
67.3991	-0.263262658012665\\
67.4991	-0.516600324904527\\
67.5991	-0.838216920909677\\
67.6991	-1.1928366986369\\
67.7991	-1.16642690596196\\
67.8991	-0.900093324066457\\
67.9991	-0.583924004337258\\
68.0991	-0.5157051094413\\
68.1991	-0.620788682931694\\
68.2991	-0.766820297608481\\
68.3991	-0.885981028963714\\
68.4991	-0.908251140478945\\
68.5991	-0.841282293347287\\
68.6991	-0.744029048951517\\
68.7991	-0.571074835450357\\
68.8991	-0.495034293112513\\
68.9991	-0.446304137301952\\
69.0991	-0.432258006147329\\
69.1991	-0.398377211847479\\
69.2991	-0.367539062413784\\
69.3991	-0.357360840407565\\
69.4991	-0.354891857228928\\
69.5991	-0.359230672588056\\
69.6991	-0.355344674429598\\
69.7991	-0.331732222723474\\
69.8991	-0.299528797999474\\
69.999	-0.265251532241116\\
70.099	-0.236943124520694\\
70.199	-0.202276253899663\\
70.2991	-0.17534227698277\\
70.3991	-0.162631391272178\\
70.4991	-0.14017606857898\\
70.599	-0.121725850317149\\
70.699	-0.130929258544551\\
70.7991	-0.144810667349747\\
70.8991	-0.162437247477771\\
70.9991	-0.144713784291688\\
71.0991	-0.120720215377832\\
71.1991	-0.100388553074112\\
71.2991	-0.0832249759428651\\
71.3991	-0.0746105205948268\\
71.4991	-0.0357453204794384\\
71.5991	0.00273549761833549\\
71.6991	0.0358598862023784\\
71.7991	0.0171844273680264\\
71.8991	-0.00844090918906204\\
71.9991	-0.000630026731937661\\
72.0991	0.00625170387054804\\
72.1991	0.0211025183727089\\
72.2991	0.0202561053638867\\
72.3991	0.0317745005142614\\
72.4991	0.00442522358861165\\
72.5991	-0.00782897406504431\\
72.6991	-0.0256249633344468\\
72.7991	-0.0309478628028515\\
72.8991	-0.0337513926945426\\
72.9991	-0.0329044709916757\\
73.0991	-0.0121033735571228\\
73.1991	-0.0159362996560553\\
73.2991	-0.0482482543387435\\
73.3991	-0.042711981967798\\
73.4991	-0.0209960311791982\\
73.5991	-0.0267753127376355\\
73.6991	-0.0141432606362577\\
73.7991	-0.00584929188791358\\
73.8991	-0.0136776725955048\\
73.9991	-0.00200276302660749\\
74.0991	-0.00072277478395771\\
74.1991	-0.00357927707103395\\
74.2991	0.0205635968217247\\
74.3991	0.0149525054081349\\
74.4991	0.0350454153932205\\
74.5991	0.0202582617556469\\
74.6991	0.0326786192292606\\
74.7991	0.0226261054232544\\
74.8991	0.0427982797401374\\
74.9991	0.0673371405133111\\
75.099	0.0674423612882201\\
75.1991	0.08565269765059\\
75.2991	0.0670050322737669\\
75.3991	0.0693532008797731\\
75.4991	0.0745754614760731\\
75.5991	0.0618474514077191\\
75.6991	0.0567060928928096\\
75.7991	0.049336761624241\\
75.8991	0.0474286749649269\\
75.9991	0.0590416959294056\\
76.0991	0.0780531883276956\\
76.1991	0.0671724459209799\\
76.2991	0.0875967434968596\\
76.3991	0.0851147664001462\\
76.4991	0.0742735862495184\\
76.5991	0.0514077705873558\\
76.6991	0.0486879989852608\\
76.7991	0.0261031153684686\\
76.8991	0.0487892268817974\\
76.9991	0.0799080571723276\\
77.0991	0.0698890433265544\\
77.1991	0.0811181007896254\\
77.2991	0.0740889454370279\\
77.3991	0.0736469113440094\\
77.4991	0.0675394203717905\\
77.5991	0.0727295753353819\\
77.6991	0.0892608917347794\\
77.7991	0.0797846657758746\\
77.8991	0.0758433528601153\\
77.9991	0.0706981754088262\\
78.0991	0.065600679140242\\
78.1991	0.0656105011159413\\
78.2991	0.0524403629303648\\
78.3991	0.0450292762494816\\
78.4991	0.0715595795709664\\
78.5991	0.0655954322779286\\
78.6991	0.0881858064038508\\
78.7991	0.0944059727209124\\
78.8991	0.0927527607919555\\
78.9991	0.0992087774488185\\
79.0991	0.0685678187593243\\
79.1992	0.0756876771963977\\
79.2992	0.0763641878444143\\
79.3992	0.110102184584274\\
79.4992	0.10666720330843\\
79.5991	0.111462578754072\\
79.6992	0.0731563156940724\\
79.7992	0.11987904729543\\
79.8992	0.0618810179210731\\
79.9991	0.0687672393632398\\
80.0991	0.0579156557511502\\
80.1991	0.0707553759731601\\
80.2991	0.0410624982088003\\
80.3991	0.0937694609280736\\
80.4991	0.0815732644372219\\
80.5992	0.0811473235214707\\
80.6992	0.0777683621278228\\
80.7991	0.0656902980896819\\
80.8991	0.0709423493484612\\
80.9992	0.0838021405516864\\
81.0992	0.0574379420662428\\
81.1992	0.0458882306743127\\
81.2992	0.0550359910533453\\
81.3991	0.0607662913293118\\
81.4992	0.0951492287564324\\
81.5992	0.108436208434895\\
81.6992	0.0939135053124154\\
81.7992	0.0908913234480691\\
81.8992	0.0906072373538339\\
81.9992	0.0840451891641969\\
82.0992	0.0708953658823135\\
82.1992	0.0493025646541926\\
82.2992	0.0728381196990985\\
82.3992	0.0621721428901687\\
82.4992	0.0468663612449316\\
82.5992	0.0428429167077894\\
82.6992	-0.00161877941894545\\
82.7992	-0.035925986207813\\
82.8992	-0.0498657589702664\\
82.9992	-0.0963154186719557\\
83.0992	-0.0779785005195483\\
83.1992	-0.0577839591483951\\
83.2992	-0.00419411113760573\\
83.3992	-0.0366594039581476\\
83.4992	-0.0184340304976296\\
83.5992	0.0833656350457689\\
83.6992	0.119686367137228\\
83.7992	0.0657330507742579\\
83.8992	0.0363996720335447\\
83.9992	0.014470853060048\\
84.0992	0.0477767783479469\\
84.1992	0.0669926366717132\\
84.2992	0.0199730538384236\\
84.3992	0.00212959772288897\\
84.4992	0.00679019340892919\\
84.5993	0.021359055723051\\
84.6992	-0.0274866034857374\\
84.7993	-0.0474711189357118\\
84.8993	-0.0505226764129667\\
84.9991	0.0357518484172637\\
85.0991	0.00789575041583597\\
85.1992	0.0169498401980142\\
85.2992	0.0147344395579053\\
85.3992	0.017824458576728\\
85.4992	0.00885374702266923\\
85.5992	-0.00509786543779773\\
85.6992	0.0159281771661928\\
85.7992	0.00870630544882511\\
85.8992	0.0743105013676766\\
85.9992	0.0420288908262529\\
86.0992	0.0425098073482102\\
86.1992	0.015240695600556\\
86.2992	0.027440886498028\\
86.3992	0.0102226069628743\\
86.4992	0.00510704256218905\\
86.5992	-0.000333168630420165\\
86.6992	-0.0183193407216087\\
86.7992	-0.0130715731203785\\
86.8992	0.014412342685283\\
86.9992	-0.00854923293185231\\
87.0992	-0.0206869746427724\\
87.1992	-0.00272575331548305\\
87.2992	0.000855608634686876\\
87.3992	-0.0158611364048157\\
87.4992	-0.0162420834027689\\
87.5992	-0.0238208776794423\\
87.6992	-0.0421734720003814\\
87.7992	-0.0309651996818346\\
87.8992	-0.02748795246382\\
87.9992	-0.0254134972326442\\
88.0992	-0.0372420381007374\\
88.1992	-0.0494127535864623\\
88.2992	-0.0585080384287473\\
88.3993	-0.0698047816432212\\
88.4993	-0.0746981146346007\\
88.5992	-0.0795900957675358\\
88.6993	-0.10944933874691\\
88.7993	-0.0969701221348783\\
88.8993	-0.11429565519638\\
88.9993	-0.121363105374491\\
89.0993	-0.133841068619012\\
89.1993	-0.121936426773503\\
89.2993	-0.105330296870699\\
89.3993	-0.0954479668444851\\
89.4993	-0.077614513231921\\
89.5993	-0.0827552363713354\\
89.6993	-0.103615553675284\\
89.7993	-0.118264708856206\\
89.8993	-0.110826067320171\\
89.9993	-0.121167127555661\\
90.0993	-0.123979367920646\\
90.1993	-0.127586447827998\\
90.2993	-0.111073784426418\\
90.3993	-0.102146381214309\\
90.4993	-0.105112421836733\\
90.5993	-0.102397761313683\\
90.6993	-0.106536894380476\\
90.7993	-0.125368590851079\\
90.8993	-0.119186342360765\\
90.9993	-0.128549819903186\\
91.0993	-0.134749513341265\\
91.1993	-0.160149884320536\\
91.2993	-0.168626028210547\\
91.3993	-0.138319705907103\\
91.4993	-0.135556365991277\\
91.5993	-0.128038615742632\\
91.6993	-0.0974386239888378\\
91.7993	-0.086519386454415\\
91.8993	-0.0919078412344703\\
91.9993	-0.076991805174185\\
92.0993	-0.0235149901150068\\
92.1993	-0.0148637449546612\\
92.2993	-0.00958779873248761\\
92.3993	0.00698705921613584\\
92.4993	0.0450965301932834\\
92.5993	0.03935741333682\\
92.6993	0.0353595746028481\\
92.7993	0.0292914860615654\\
92.8993	0.0206987955047203\\
92.9994	0.0255958079081883\\
93.0993	0.0447567563016985\\
93.1994	0.0468630365695997\\
93.2994	0.0407891691192593\\
93.3994	0.0308165074726981\\
93.4994	0.0429319462262716\\
93.5994	0.0567344473733278\\
93.6994	0.0780538117692588\\
93.7994	0.0600961077887329\\
93.8994	0.0323701562523072\\
93.9994	0.0100620397831717\\
94.0994	0.00930468364671247\\
94.1994	0.0308122064261455\\
94.2994	0.0356174497943863\\
94.3994	0.0240532795857434\\
94.4994	0.0206218090169964\\
94.5994	0.0290570844095691\\
94.6994	0.0309429260476186\\
94.7994	0.0343635916950165\\
94.8994	0.0147581460557676\\
94.9993	-0.00594817376935588\\
95.0993	-0.0110252983587691\\
95.1993	0.0141821392847075\\
95.2994	0.0379925791067636\\
95.3994	0.0498728879051365\\
95.4994	0.0669558868897166\\
95.5994	0.107115551887461\\
95.6994	0.0849247844613562\\
95.7994	0.0673085258263969\\
95.8994	0.0541858738367079\\
95.9994	0.0545093655478872\\
96.0994	0.0327553549887437\\
96.1994	0.0538954890248572\\
96.2994	0.0629853572244752\\
96.3994	0.0547281708968295\\
96.4994	0.08009164686988\\
96.5994	0.117461165682173\\
96.6994	0.114420092339731\\
96.7994	0.0956129827252032\\
96.8994	-0.00363514320682391\\
96.9994	-0.0307220272935099\\
97.0994	-0.0731355625278061\\
97.1994	-0.0451377469888859\\
97.2994	-0.0332589236979546\\
97.3994	-0.0223693595823483\\
97.4994	-0.0906016072855447\\
97.5994	-0.0640729393788722\\
97.6994	-0.0584688438857222\\
97.7994	-0.0481996187238705\\
97.8994	-0.0444647027527499\\
97.9994	-0.0437883259378982\\
98.0994	-0.0950064693281243\\
98.1995	-0.0748594488377384\\
98.2995	-0.0799203699153603\\
98.3995	-0.0740430146153197\\
98.4995	-0.11108438565874\\
98.5995	-0.160773460422442\\
98.6995	-0.194500990613403\\
98.7995	-0.195828179226061\\
98.8995	-0.208238048065299\\
98.9995	-0.212358226990779\\
99.0995	-0.200004034951833\\
99.1995	-0.13085161064714\\
99.2995	-0.13105755823047\\
99.3995	-0.0976445287871821\\
99.4995	-0.0876913583169812\\
99.5995	-0.0390581849413663\\
99.6995	-0.0234659779451134\\
99.7995	-0.0140484564216069\\
99.8995	0.00992671054346062\\
99.9994	0.00754867464795773\\
100.0994	0.00428412233544191\\
100.1995	-0.00273385711137243\\
100.2995	0.000171540630025209\\
100.3995	0.00416739292204047\\
100.4995	0.0132217724704133\\
100.5995	0.0328973353347906\\
100.6995	0.0293515096111901\\
100.7995	0.0293043046903586\\
100.8995	0.0293167948457966\\
100.9995	0.0292872175515074\\
101.0995	0.0292386236419351\\
101.1995	0.0292091027866643\\
101.2995	0.0292037102963789\\
101.3995	0.0291783023845399\\
101.4995	0.0291762072486558\\
101.5995	0.0291753620527891\\
101.6995	0.0291427825943001\\
101.7995	0.0291503503551933\\
101.8995	0.0291347610827333\\
101.9995	0.0291112887854984\\
102.0995	0.0290759109152071\\
102.1995	0.029065063761976\\
102.2995	0.0290650179590653\\
102.3995	0.0290418721961754\\
102.4995	0.0290473266131298\\
102.5995	0.0290636074665223\\
102.6995	0.0290404649306882\\
102.7995	0.0290384666593911\\
102.8995	0.0290583073013176\\
102.9995	0.0291088429530144\\
103.0995	0.0290648862074793\\
103.1996	0.0291045234047494\\
103.2996	0.0290823373121962\\
103.3996	0.0290811485651471\\
103.4996	0.0290689054988532\\
103.5996	0.029046672347637\\
103.6996	0.0290674865769359\\
103.7996	0.0290450023530521\\
103.8996	0.0290546982212409\\
103.9996	0.0290335084697871\\
104.0996	0.0290309503782683\\
104.1996	0.0290194451474189\\
104.2996	0.0265090254610963\\
104.3996	0.00505384801731487\\
104.4996	0.00338032325399826\\
104.5996	-0.188100943778884\\
104.6996	-0.678258614738186\\
104.7996	-1.1386741242017\\
104.8996	-1.37339424001984\\
104.9996	-1.44521374106801\\
105.0996	-1.47196584825912\\
105.1996	-1.4805731173632\\
105.2996	-1.51247248058064\\
105.3996	-1.51580508889862\\
105.4996	-1.52786327514283\\
105.5996	-1.54973080845392\\
105.6996	-1.56640958068064\\
105.7996	-1.56304298310818\\
105.8996	-1.56150777303875\\
105.9996	-1.56145707446667\\
106.0996	-1.5615803279126\\
106.1996	-1.56135693819246\\
106.2996	-1.56141356334255\\
106.3996	-1.56178890828857\\
106.4996	-1.56255657775091\\
106.5996	-1.56193889461692\\
106.6996	-1.56170886955025\\
106.7996	-1.56138297479674\\
106.8996	-1.56102864531369\\
106.9996	-1.56124322052099\\
107.0996	-1.5610557470571\\
107.1996	-1.56061231097553\\
107.2996	-1.56038795640024\\
107.3996	-1.56033601546815\\
107.4996	-1.56063303592186\\
107.5996	-1.5602470200195\\
107.6996	-1.5600605877495\\
107.7996	-1.55980063503463\\
107.8996	-1.55928896530055\\
107.9996	-1.55881573276635\\
108.0996	-1.55831090708701\\
108.1997	-1.55780836412714\\
108.2996	-1.55720114559838\\
108.3997	-1.556961364761\\
108.4997	-1.55659028204582\\
108.5997	-1.55626511134503\\
108.6997	-1.55597744838121\\
108.7997	-1.55565071296562\\
108.8997	-1.55522628476765\\
108.9997	-1.5549213760596\\
109.0997	-1.55458268125685\\
109.1997	-1.55426758519582\\
109.2997	-1.55386089625568\\
109.3997	-1.55358996447798\\
109.4997	-1.55313320378549\\
109.5997	-1.55283594602053\\
109.6997	-1.5526033320152\\
109.7997	-1.55225821269904\\
109.8997	-1.55184412811243\\
109.9996	-1.55164082075668\\
110.0996	-1.55139036601421\\
110.1996	-1.55120749053301\\
110.2996	-1.55095849561688\\
110.3996	-1.55049874868885\\
110.4996	-1.54911146531975\\
110.5996	-1.54442680262554\\
110.6996	-1.5428954995302\\
110.7997	-1.55107111178053\\
110.8997	-1.55634609100567\\
110.9997	-1.53864477661108\\
111.0997	-1.49241237226128\\
111.1997	-1.27810064906901\\
111.2997	-0.967007264945952\\
111.3997	-0.677361861979185\\
111.4997	-0.443170216882345\\
111.5997	-0.306068318544363\\
111.6997	-0.192517927326138\\
111.7997	-0.133868319403911\\
111.8997	-0.0861201361692361\\
111.9997	-0.0646741974079417\\
112.0997	-0.0403584683175968\\
112.1997	-0.0313884003869807\\
112.2997	-0.022302413429121\\
112.3997	-0.0208709458099586\\
112.4997	-0.0214639044957077\\
112.5997	-0.0247476854697095\\
112.6997	-0.0231738616586829\\
112.7997	-0.0102063603786938\\
112.8997	-0.00559829503649104\\
112.9997	-0.00185576182244817\\
113.0997	0.000913621903441175\\
113.1997	0.0112788374190603\\
113.2997	0.00864405445138271\\
113.3997	0.00862874185089583\\
113.4997	0.00860011587775563\\
113.5997	0.00863678868227454\\
113.6997	0.0087136755440295\\
113.7998	0.00872669463921997\\
113.8998	0.00869783646455595\\
113.9998	0.00871126511138844\\
114.0998	0.00873583920560268\\
114.1998	0.00880070741557348\\
114.2998	0.00881305607739387\\
114.3998	0.00884960999059319\\
114.4998	0.00885204431089876\\
114.5998	0.0088762726223397\\
114.6998	0.00891148910190681\\
114.7998	0.00891490100910229\\
114.8998	0.00893914312822552\\
114.9997	0.00896235883468013\\
115.0997	0.00905023947481978\\
115.1997	0.00907203984198907\\
115.2997	0.00908619049130904\\
115.3997	0.00913097564566633\\
115.4997	0.00914496247765476\\
115.5997	0.00916879969676379\\
115.6997	0.00919262828027654\\
115.7998	0.00919537230769174\\
115.8997	0.00920787938767625\\
115.9997	0.00921048172155194\\
116.0998	0.00923401550508738\\
116.1998	0.00923579110881277\\
116.2998	0.00926017355869404\\
116.3998	0.00930587137746239\\
116.4998	0.0093493153564783\\
116.5998	0.00938306878104567\\
116.6998	0.00941655637375138\\
116.7998	0.00944103650201191\\
116.8998	0.00945338749886379\\
116.9998	0.00950871044995192\\
117.0998	0.00955385398429497\\
117.1998	0.00962005014636091\\
117.2998	0.00958809163865587\\
117.3998	0.00964227753124276\\
117.4998	0.00969631805694516\\
117.5998	0.00972047664809322\\
117.6998	0.0097644979599815\\
117.7998	0.00977730535331305\\
117.8998	0.00981117393986514\\
117.9998	0.00985505006164365\\
118.0998	0.00988889692884437\\
118.1998	0.00993286506902482\\
118.2998	0.00993469087876197\\
118.3998	0.009989422719763\\
118.4998	0.01002307946511\\
118.5998	0.0100356463213676\\
118.6998	0.0100792197337732\\
118.7998	0.010111781838882\\
118.8998	0.0101461299018632\\
118.9998	0.0101367389431684\\
119.0999	0.0101381509762934\\
119.1999	0.0101605026845366\\
119.2999	0.010149011153126\\
119.3999	0.0101811653249945\\
119.4998	0.0102451079811729\\
119.5999	0.0102781707525856\\
119.6999	0.0103105080048486\\
119.7999	0.0103207392874583\\
};
\addplot [color=mycolor3,solid,forget plot]
  table[row sep=crcr]{%
0	2.95389958399442\\
0.1	2.9496533760129\\
0.2	3.00868782783369\\
0.3	3.1006166802347\\
0.4	3.1850694773704\\
0.5	3.18755051234766\\
0.6	3.19002118791447\\
0.7	3.19005877146735\\
0.8	3.18903220049079\\
0.9	3.18809313264711\\
1	3.18776461354869\\
1.0999	3.1873808697125\\
1.1999	3.18703166956896\\
1.2999	3.18665912264225\\
1.3999	3.18632928224346\\
1.4999	3.18599555280169\\
1.5999	3.18580013501166\\
1.6999	3.18560532187905\\
1.7999	3.18537446223886\\
1.8999	3.18514238112112\\
1.9999	3.18487913574454\\
2.0999	3.18463701613476\\
2.1999	3.18437372312696\\
2.2999	3.18414246519761\\
2.3999	3.18388944621443\\
2.4999	3.18366852142627\\
2.5999	3.18343731889874\\
2.6999	3.18319439967659\\
2.7999	3.18291970285728\\
2.8999	3.18266778833695\\
2.9999	3.1824659851735\\
3.0999	3.18229703674237\\
3.1999	3.18211057877888\\
3.2999	3.18194282617984\\
3.3999	3.18174316531517\\
3.4999	3.18158626279656\\
3.5999	3.18137548874872\\
3.6999	3.18115740384406\\
3.7999	3.18095958711746\\
3.8999	3.18075982524446\\
3.9999	3.18057086718611\\
4.0999	3.18047262384294\\
4.1999	3.1803839917534\\
4.2999	3.18025840338065\\
4.3999	3.18015542081382\\
4.4999	3.18004139320381\\
4.5999	3.1800193454301\\
4.6999	3.17991175548616\\
4.7999	3.17983639583514\\
4.8998	3.17975014435885\\
4.9998	3.1796318846265\\
5.0998	3.17952451314223\\
5.1998	3.17944905808059\\
5.2998	3.17936480120867\\
5.3998	3.17930044239086\\
5.4997	3.1791931479953\\
5.5997	3.17912857359013\\
5.6998	3.17899973890344\\
5.7998	3.17889263518136\\
5.8998	3.1788060022897\\
5.9998	3.17868742085852\\
6.0997	3.18496883313049\\
6.1997	3.18196382538807\\
6.2997	3.1755645407822\\
6.3997	3.15340877745655\\
6.4997	3.21959741548301\\
6.5997	3.26084971473957\\
6.6997	3.24584580015496\\
6.7997	3.22185324290654\\
6.8997	3.16057774587422\\
6.9997	3.08466759919451\\
7.0997	3.01873974480879\\
7.1997	3.00098704092068\\
7.2997	3.01389450937224\\
7.3997	3.01108618825271\\
7.4997	3.0251845168771\\
7.5997	3.04916735236001\\
7.6997	3.06899935178842\\
7.7997	3.07286481798454\\
7.8997	3.08050094978233\\
7.9997	3.10017308036552\\
8.0997	3.11552171130042\\
8.1997	3.12124255222061\\
8.2997	3.12247856005279\\
8.3997	3.1519495717264\\
8.4997	3.16439448728395\\
8.5997	3.16278154270103\\
8.6997	3.16192419491125\\
8.7997	3.16106856539458\\
8.8997	3.16064078811253\\
8.9997	3.16053811518643\\
9.0997	3.16079537848426\\
9.1997	3.161265203151\\
9.2997	3.16145325945747\\
9.3997	3.16169963614959\\
9.4997	3.16205286127067\\
9.5997	3.16218226500188\\
9.6997	3.16232359248013\\
9.7997	3.16250812156916\\
9.8997	3.16273916155425\\
9.9996	3.16295911458952\\
10.0996	3.16329671956339\\
10.1996	3.16357121091569\\
10.2996	3.16382386499531\\
10.3996	3.16403685282937\\
10.4996	3.16433430745384\\
10.5996	3.16466336375155\\
10.6996	3.16503457466445\\
10.7996	3.1653507640521\\
10.8996	3.1655311166053\\
10.9996	3.16584935802495\\
11.0996	3.16611448184676\\
11.1996	3.16629283514406\\
11.2996	3.16661977281614\\
11.3996	3.16693668896594\\
11.4995	3.16720632798305\\
11.5995	3.16748299974351\\
11.6996	3.16769646250124\\
11.7996	3.16793102761416\\
11.8995	3.16822497591719\\
11.9996	3.16850163712163\\
12.0995	3.16871306023393\\
12.1995	3.16899823330438\\
12.2996	3.16928141817692\\
12.3995	3.16949900183829\\
12.4995	3.16971633343098\\
12.5995	3.16994150535183\\
12.6995	3.17024946298383\\
12.7995	3.17044029348864\\
12.8995	3.17060990569719\\
12.9995	3.17083394163279\\
13.0995	3.17110105020107\\
13.1995	3.17156906237015\\
13.2995	3.17267793578159\\
13.3995	3.17199380274258\\
13.4995	3.17464598667729\\
13.5995	3.22420660426957\\
13.6995	3.27260704168168\\
13.7995	3.30505823934111\\
13.8995	3.2317995211292\\
13.9995	2.95938451377091\\
14.0995	2.15832695715212\\
14.1995	1.35685407153344\\
14.2995	-1.28717411081674\\
14.3995	-1.1555247166926\\
14.4995	-0.950299649120266\\
14.5995	-0.74736997789687\\
14.6995	-0.740464579954713\\
14.7995	-0.737430863109922\\
14.8995	-0.737599107524624\\
14.9994	-0.736748526811782\\
15.0994	-0.743628883284276\\
15.1994	-0.748804185808926\\
15.2994	-0.755408958806337\\
15.3994	-0.759253791594116\\
15.4994	-0.759988358969917\\
15.5994	-0.764458324229176\\
15.6994	-0.770615849583018\\
15.7994	-0.777396908800048\\
15.8994	-0.783426398842925\\
15.9994	-0.786729256346265\\
16.0994	-0.789745294649165\\
16.1994	-0.792486010344012\\
16.2994	-0.795315833086967\\
16.3994	-0.801519812264247\\
16.4994	-0.808604349261105\\
16.5994	-0.814689507075611\\
16.6994	-0.818952477703427\\
16.7994	-0.822609234796007\\
16.8994	-0.827639795455869\\
16.9994	-0.831332587083659\\
17.0994	-0.835530865142919\\
17.1994	-0.839594624028896\\
17.2994	-0.844179988366222\\
17.3994	-0.849145394970844\\
17.4994	-0.852167344154769\\
17.5994	-0.858541335583723\\
17.6994	-0.863443965479822\\
17.7994	-0.866778031998074\\
17.8994	-0.869952191161907\\
17.9994	-0.87427940431306\\
18.0994	-0.877136005054024\\
18.1994	-0.880181703782364\\
18.2994	-0.883196744600005\\
18.3994	-0.886533789192421\\
18.4994	-0.887505940998793\\
18.5994	-0.776615799272299\\
18.6994	0.57745582933453\\
18.7994	1.10850696977855\\
18.8994	1.75639179056995\\
18.9994	1.75323926049205\\
19.0994	1.67352634268372\\
19.1994	1.83748481814736\\
19.2994	1.91190015774384\\
19.3994	1.96309868669428\\
19.4994	2.02335851654346\\
19.5994	2.13179404005065\\
19.6994	2.29439219189217\\
19.7994	2.43640376335606\\
19.8994	2.6986432279675\\
19.9993	3.19907621348574\\
20.0993	3.98613613441284\\
20.1993	5.13038288854631\\
20.2993	5.40027825754317\\
20.3993	5.70904342780798\\
20.4993	5.89867087755019\\
20.5993	6.05701350338874\\
20.6993	6.21064202698557\\
20.7993	6.34733417557902\\
20.8993	6.47313753994272\\
20.9993	6.59955280231861\\
21.0993	6.63764573444712\\
21.1993	6.68290862968628\\
21.2993	6.71403298906636\\
21.3993	6.76094251855555\\
21.4993	6.7911489887769\\
21.5993	6.80434639470441\\
21.6993	6.82543136504071\\
21.7993	6.8714784675438\\
21.8993	6.87034257070026\\
21.9993	6.89885407173123\\
22.0993	6.93908823002936\\
22.1993	6.95761630439229\\
22.2993	6.95714435707807\\
22.3993	6.97401463203033\\
22.4993	6.97735764537192\\
22.5993	6.95055614059919\\
22.6993	6.93017529868907\\
22.7993	6.93483446676508\\
22.8993	6.97581986376921\\
22.9993	6.99478890498964\\
23.0993	7.00384180798073\\
23.1993	7.00399233760712\\
23.2993	6.99731727497241\\
23.3993	7.0024127628521\\
23.4993	7.01605638244613\\
23.5993	7.01297659166393\\
23.6993	7.00975298072921\\
23.7993	7.02428956563473\\
23.8993	7.03132268119199\\
23.9993	7.0314674681448\\
24.0993	7.04526975042126\\
24.1993	7.05497101504149\\
24.2993	7.04670383834471\\
24.3993	7.035362418408\\
24.4993	7.03702243108239\\
24.5993	7.02749452585923\\
24.6993	7.03979334352546\\
24.7993	7.08525096616367\\
24.8993	7.10114706222201\\
24.9992	7.10593526486428\\
25.0992	7.10982308275506\\
25.1992	7.10214117825746\\
25.2992	7.08767650767534\\
25.3992	7.07447191070481\\
25.4992	7.04888363633947\\
25.5992	7.02876054519608\\
25.6992	7.02747807766299\\
25.7992	7.05943665232257\\
25.8992	7.05802542685851\\
25.9992	7.06136952292248\\
26.0992	7.07410924812666\\
26.1992	7.06821466262812\\
26.2992	7.06710157920442\\
26.3992	7.06905781551971\\
26.4992	7.0574612569107\\
26.5992	7.05060324896779\\
26.6992	7.05020303418008\\
26.7992	7.07449369333689\\
26.8992	7.09367217025034\\
26.9992	7.11976893434645\\
27.0992	7.13161261774056\\
27.1992	7.12172344335022\\
27.2992	7.12183994503448\\
27.3992	7.1118487725088\\
27.4992	7.09590287807616\\
27.5992	7.10133683951979\\
27.6992	7.10953138549261\\
27.7992	7.14500624575631\\
27.8992	7.16864701354229\\
27.9992	7.18729267035436\\
28.0992	7.20422119373442\\
28.1992	7.20532197597935\\
28.2992	7.22858478243133\\
28.3992	7.23589082334846\\
28.4992	7.24197431019214\\
28.5992	7.26508778314107\\
28.6992	7.29181770092232\\
28.7992	7.36331983694273\\
28.8992	7.42182120401831\\
28.9992	7.5021432903575\\
29.0992	7.60029125047172\\
29.1992	7.68152089443833\\
29.2992	7.73795449215687\\
29.3992	7.76005302150738\\
29.4992	7.77984196549561\\
29.5992	7.80964973973038\\
29.6992	7.83192573334784\\
29.7992	7.90675958413309\\
29.8992	7.9214091125994\\
29.9991	7.94194406711916\\
30.0991	7.94602812673542\\
30.1991	7.94775449929367\\
30.2991	7.94145373913137\\
30.3991	7.91530984658467\\
30.4991	7.9032521953503\\
30.5991	7.90824985660779\\
30.6991	7.91752605066967\\
30.7991	7.93704980332045\\
30.8991	7.94498024460383\\
30.9991	7.9585370572747\\
31.0991	7.96839595519818\\
31.1991	7.96330147815117\\
31.2991	7.93980680318109\\
31.3991	7.92619812887057\\
31.4991	7.92170801232666\\
31.5991	7.92633197667008\\
31.6991	7.95203246013386\\
31.7991	7.94797171895985\\
31.8991	7.96266801625038\\
31.9991	7.97367720581942\\
32.0991	7.97307620071185\\
32.1991	7.95604694995398\\
32.2991	7.93212126282811\\
32.3991	7.91360534314215\\
32.4991	7.91761145148323\\
32.5991	7.93802327101927\\
32.6991	7.94691811024663\\
32.7991	7.9383746869141\\
32.8991	7.94870579200471\\
32.9991	7.96676309863271\\
33.0991	7.97785425617977\\
33.1991	7.9785566754976\\
33.2991	7.94783290997979\\
33.3991	7.92797274327805\\
33.4991	7.91299118689963\\
33.5991	7.8975381929206\\
33.6991	7.90014992168169\\
33.7991	7.92820053133021\\
33.8991	7.92757425878781\\
33.9991	7.95650048456431\\
34.0991	7.97818814837542\\
34.1991	7.99320019791701\\
34.2991	7.99378701003993\\
34.3991	7.96346096159096\\
34.4991	7.93334688310452\\
34.5991	7.92025916481754\\
34.6991	7.91829161103342\\
34.7991	7.94315160461594\\
34.8991	7.95568029571827\\
34.9991	7.959918814052\\
35.0991	7.96722073862034\\
35.1991	7.97452118519608\\
35.2991	7.97605252310581\\
35.3991	7.92959088320035\\
35.4991	7.90405649459264\\
35.5991	7.89436413873945\\
35.6991	7.89745293229199\\
35.7991	7.92185704407764\\
35.8991	7.92629553719163\\
35.9991	7.93630399785155\\
36.0991	7.9497337446986\\
36.1991	7.95102883207206\\
36.2991	7.91904834707378\\
36.3991	7.89421690265721\\
36.4991	7.90197086530015\\
36.5991	7.9128822242355\\
36.6991	7.91968699576621\\
36.7991	7.93554088494625\\
36.8991	7.93680904043242\\
36.9991	7.95332760878894\\
37.0991	7.97643772324087\\
37.1991	7.98182224669208\\
37.2991	7.95912817299947\\
37.3991	7.93961282886614\\
37.4991	7.93405353320798\\
37.5991	7.93802469619944\\
37.6991	7.94919884880572\\
37.7991	7.94783219731644\\
37.8991	7.95774064807087\\
37.9991	7.95673070107649\\
38.0991	7.95199170828724\\
38.1991	7.94463676122053\\
38.2991	7.91170487995122\\
38.3991	7.89572448473793\\
38.4991	7.89426082441339\\
38.5991	7.88936815740002\\
38.6991	7.89900358004482\\
38.7991	7.89239757383481\\
38.8991	7.9085717916092\\
38.9991	7.91732220651275\\
39.0991	7.93593340452153\\
39.1991	7.9283389410965\\
39.2991	7.8997795862141\\
39.3991	7.88669491433659\\
39.4991	7.88445707296937\\
39.5991	7.88887047324573\\
39.6991	7.89298807501562\\
39.7991	7.89610518816989\\
39.8991	7.90355397081819\\
39.9991	7.91005799987118\\
40.0991	7.91192325050995\\
40.1991	7.88218955809025\\
40.2991	7.87232834421987\\
40.3991	7.86632966371714\\
40.499	7.87071163708176\\
40.5991	7.89452999789336\\
40.6991	7.9073721943395\\
40.7991	7.91734947980277\\
40.8991	7.92267579978019\\
40.9991	7.93507656517056\\
41.0991	7.94073926007554\\
41.1991	7.9134776490984\\
41.2991	7.89263379002261\\
41.399	7.88292390170129\\
41.4991	7.87686100399176\\
41.5991	7.86997934937882\\
41.6991	7.89876844493871\\
41.7991	7.89513886666646\\
41.8991	7.90213240704953\\
41.999	7.92590902369325\\
42.0991	7.95463495246653\\
42.1991	7.98249655140558\\
42.2991	8.00643474798126\\
42.3991	8.0468366801986\\
42.4991	8.09590931777179\\
42.5991	8.15977567942548\\
42.6991	8.22827085928153\\
42.7991	8.30047198302198\\
42.8991	8.38661503281634\\
42.9991	8.49587143938372\\
43.0991	8.60961595799028\\
43.1991	8.7223159517858\\
43.2991	8.80995707673496\\
43.3991	8.85828936606258\\
43.4991	8.89966268561341\\
43.5991	8.95027713570602\\
43.6991	9.02701208011384\\
43.7991	9.10196834408532\\
43.8991	9.17596542217479\\
43.9991	9.24868303246012\\
44.0991	9.34341506069685\\
44.1991	9.42219840587821\\
44.2991	9.49713676683842\\
44.3991	9.58277173851662\\
44.4991	9.67133619249794\\
44.5991	9.76868473519879\\
44.6991	9.8806914053753\\
44.7991	10.0142319528396\\
44.8991	10.162807756591\\
44.999	10.3225047975178\\
45.099	10.5115107860352\\
45.199	10.6929376304564\\
45.299	10.8521360730937\\
45.399	11.0335324845645\\
45.499	11.1857371247704\\
45.599	11.3017861062261\\
45.699	11.3636908828117\\
45.799	11.3552843410202\\
45.899	11.3939749810388\\
45.999	11.4718420339058\\
46.099	11.477249063395\\
46.199	11.4141057764186\\
46.299	11.3517277622458\\
46.399	11.3523851727536\\
46.499	11.3778390419021\\
46.599	11.4706321831653\\
46.699	11.5759381527385\\
46.799	11.5685294063475\\
46.899	11.5636964903507\\
46.999	11.5929499612511\\
47.099	11.6258974276715\\
47.199	11.6661305083965\\
47.299	11.6908569650763\\
47.399	11.6598902461174\\
47.499	11.6546421347832\\
47.599	11.6652145298156\\
47.699	11.6777131164626\\
47.799	11.704682394063\\
47.899	11.6786161147124\\
47.999	11.6893116946433\\
48.099	11.7451989720014\\
48.199	11.8202654484163\\
48.299	11.8383213448368\\
48.399	11.7854729268655\\
48.4991	11.6755273725635\\
48.599	11.5779843437784\\
48.6991	11.5352732319971\\
48.7991	11.5342781781319\\
48.899	11.532726909836\\
48.9991	11.462650552097\\
49.0991	11.3626435100295\\
49.1991	11.3301340238882\\
49.2991	11.3854512152325\\
49.3991	11.4822193703967\\
49.499	11.5640490006041\\
49.5991	11.5918623877463\\
49.6991	11.615007838088\\
49.7991	11.6631130720843\\
49.8991	11.7343974503339\\
49.999	11.771194123526\\
50.099	11.6456290273966\\
50.199	11.4368905211495\\
50.299	11.2660090729971\\
50.399	11.2219257152339\\
50.499	11.2456372271519\\
50.599	11.2090171145344\\
50.699	11.2057842452409\\
50.799	11.257920849927\\
50.899	11.4297976631663\\
50.999	11.6305799604915\\
51.099	11.6899729964324\\
51.199	11.6538665005306\\
51.299	11.5931610387022\\
51.399	11.5429859556402\\
51.499	11.5491812437473\\
51.599	11.5109814870781\\
51.699	11.4368427731488\\
51.799	11.3896110679196\\
51.899	11.4004613558951\\
51.999	11.4617345688203\\
52.099	11.5480928310845\\
52.199	11.4990775306784\\
52.299	11.4143328095821\\
52.399	11.4206147905284\\
52.499	11.5095770188961\\
52.599	11.6248388066066\\
52.699	11.6613834877075\\
52.799	11.6191503023767\\
52.899	11.5897938195941\\
52.999	11.6227130831496\\
53.099	11.6072393018883\\
53.199	11.4667136252091\\
53.2991	11.3086726005075\\
53.3991	11.2153470440556\\
53.4991	11.2322445372957\\
53.5991	11.3459032170912\\
53.6991	11.4744185675411\\
53.7991	11.5572508660068\\
53.8991	11.6219004550791\\
53.999	11.7207682309153\\
54.0991	11.7943935777093\\
54.1991	11.7536703163766\\
54.299	11.6098762938435\\
54.3991	11.4554233378619\\
54.4991	11.4064139649525\\
54.599	11.4475304135472\\
54.6991	11.4608177891725\\
54.7991	11.4410569393201\\
54.8991	11.5171779902445\\
54.999	11.6290080816328\\
55.099	11.6997183688885\\
55.199	11.6894133636314\\
55.299	11.6434274111958\\
55.399	11.6206199442094\\
55.499	11.5675412461887\\
55.599	11.503853126009\\
55.699	11.4389047243158\\
55.799	11.4117323328037\\
55.899	11.4482852101377\\
55.999	11.4702420198324\\
56.099	11.4122816928993\\
56.199	11.322928254346\\
56.299	11.2454446533438\\
56.399	11.2252274144207\\
56.499	11.2202132988214\\
56.599	11.19955718578\\
56.699	11.2250903262802\\
56.799	11.1829561961433\\
56.899	11.1533537089213\\
56.999	11.1069522429133\\
57.099	11.1406620399099\\
57.199	11.1966775396597\\
57.299	11.193883920792\\
57.399	11.1477706775752\\
57.4991	11.0839633852882\\
57.5991	11.0593510642277\\
57.6991	10.9615859552389\\
57.7991	10.9818095773985\\
57.8991	10.90721499675\\
57.9991	10.9213471012585\\
58.0991	10.9190950598026\\
58.1991	10.8983277847286\\
58.2991	10.9097622848696\\
58.399	10.8893189797021\\
58.4991	10.8789091063722\\
58.5991	10.8851173971633\\
58.6991	10.9765178860218\\
58.7991	11.2039466131611\\
58.8991	11.4498329968022\\
58.9991	11.599419977667\\
59.0991	11.8153091298355\\
59.1991	11.9631030765786\\
59.2991	11.9926407312694\\
59.3991	11.9243042406167\\
59.4991	11.8943138484746\\
59.5991	11.9236655657595\\
59.6991	12.0335376885948\\
59.7991	12.1483692355158\\
59.8991	12.1960177302803\\
59.999	12.3063295597102\\
60.099	12.3879209335799\\
60.199	12.3781251672336\\
60.299	12.3501559855946\\
60.399	12.3189334396975\\
60.499	12.3171592852404\\
60.599	12.3046870510071\\
60.699	12.2403673386008\\
60.799	12.227645485099\\
60.899	12.2518470833306\\
60.999	12.2656743991391\\
61.099	12.2617925727802\\
61.199	12.2070676764962\\
61.299	12.1694400013225\\
61.399	12.1669658285853\\
61.499	12.165806290883\\
61.599	12.2141044856606\\
61.699	12.1971432694835\\
61.799	12.1625065692246\\
61.899	12.1972369849371\\
61.999	12.2195637736803\\
62.099	12.2653854080581\\
62.199	12.234072391135\\
62.299	12.1840922364456\\
62.399	12.1519893291196\\
62.499	12.1816977894059\\
62.599	12.2543814642931\\
62.699	12.3035914546421\\
62.799	12.2952435119718\\
62.899	12.3287211740153\\
62.999	12.3343543125218\\
63.099	12.3587486029275\\
63.199	12.3623121787907\\
63.299	12.266835599156\\
63.399	12.1844014438331\\
63.499	12.2183586794465\\
63.599	12.2337962808293\\
63.699	12.264438256907\\
63.799	12.2629072068838\\
63.8991	12.2699623419631\\
63.999	12.2775226898048\\
64.099	12.2863383014574\\
64.1991	12.2950623326301\\
64.2991	12.228758896173\\
64.3991	12.1669854713612\\
64.4991	12.1547833213684\\
64.599	12.186603792642\\
64.6991	12.2580067053917\\
64.7991	12.2778957828794\\
64.8991	12.2590281627806\\
64.999	12.2826228723852\\
65.099	12.276263124207\\
65.199	12.2953629636321\\
65.299	12.2549115804104\\
65.399	12.1911055394673\\
65.499	12.1391455298406\\
65.599	12.1788115286992\\
65.699	12.2385354374514\\
65.799	12.2830654687282\\
65.899	12.2548621737297\\
65.999	12.2749830706799\\
66.099	12.2675096004093\\
66.199	12.3040251821852\\
66.299	12.2902398659174\\
66.399	12.1961969727438\\
66.499	12.1276544181628\\
66.599	12.1631696889789\\
66.699	12.2163998955064\\
66.799	12.2753405410441\\
66.899	12.2931695107723\\
66.999	12.3264607481801\\
67.099	12.3329806530253\\
67.1991	12.2556431233888\\
67.299	12.2761373440363\\
67.3991	12.2444167381794\\
67.4991	12.2319529978664\\
67.5991	12.2972955805789\\
67.6991	12.1412876104621\\
67.7991	10.5904891531096\\
67.8991	10.3417298481621\\
67.9991	10.4966777748504\\
68.0991	10.6705536700766\\
68.1991	10.7970260039346\\
68.2991	10.8461338607982\\
68.3991	10.9322077215577\\
68.4991	11.0568938837637\\
68.5991	11.2380634009207\\
68.6991	11.410795886974\\
68.7991	11.5348715696013\\
68.8991	11.5816450639717\\
68.9991	11.6309053694453\\
69.0991	11.6931789110476\\
69.1991	11.7906944182096\\
69.2991	11.9145514968293\\
69.3991	12.0355440460964\\
69.4991	12.149914803427\\
69.5991	12.2854265911652\\
69.6991	12.4112065012598\\
69.7991	12.549072546842\\
69.8991	12.6785899222592\\
69.999	12.8088860681922\\
70.099	12.9392661730888\\
70.199	13.0764773594481\\
70.2991	13.2270055302957\\
70.3991	13.3543395022425\\
70.4991	13.4485249911621\\
70.599	13.5489535066508\\
70.699	13.6357042201813\\
70.7991	13.6975556457778\\
70.8991	13.7522650587659\\
70.9991	13.8217881216684\\
71.0991	13.8775500529147\\
71.1991	13.93200549509\\
71.2991	13.9929068421761\\
71.3991	14.0647852891246\\
71.4991	14.0977853538805\\
71.5991	14.1261398696769\\
71.6991	14.152224137035\\
71.7991	14.1636988609923\\
71.8991	14.1724778575478\\
71.9991	14.1491385549253\\
72.0991	14.138635240032\\
72.1991	14.1399933395635\\
72.2991	14.1340678360378\\
72.3991	14.1523601834222\\
72.4991	14.1730750868271\\
72.5991	14.1863365554955\\
72.6991	14.1919136946125\\
72.7991	14.2038474848656\\
72.8991	14.2102906753294\\
72.9991	14.2059644866432\\
73.0991	14.1885296526618\\
73.1991	14.1841680528277\\
73.2991	14.1786010579753\\
73.3991	14.1904505398137\\
73.4991	14.1986181399318\\
73.5991	14.2045999572416\\
73.6991	14.221987302431\\
73.7991	14.2304238109173\\
73.8991	14.2226728889209\\
73.9991	14.2107982930946\\
74.0991	14.1905832785644\\
74.1991	14.1816218238433\\
74.2991	14.1790092698391\\
74.3991	14.1773114320765\\
74.4991	14.1802309242846\\
74.5991	14.1963175113818\\
74.6991	14.2391434077206\\
74.7991	14.2493911386833\\
74.8991	14.2371916255523\\
74.9991	14.223579276536\\
75.099	14.2055701613133\\
75.1991	14.1979787202801\\
75.2991	14.1895031288712\\
75.3991	14.1854883313786\\
75.4991	14.2006165692027\\
75.5991	14.2125778783728\\
75.6991	14.2099222163786\\
75.7991	14.2014211250451\\
75.8991	14.1927941117726\\
75.9991	14.2000295343307\\
76.0991	14.2016002649211\\
76.1991	14.1985679082112\\
76.2991	14.2061265984014\\
76.3991	14.2234773991201\\
76.4991	14.2616339061985\\
76.5991	14.260164191655\\
76.6991	14.2532204739014\\
76.7991	14.2295404962291\\
76.8991	14.2060984899196\\
76.9991	14.2197057350067\\
77.0991	14.1913229550428\\
77.1991	14.191137608444\\
77.2991	14.197629054412\\
77.3991	14.2243130414127\\
77.4991	14.2222606076652\\
77.5991	14.2101925036489\\
77.6991	14.197010990093\\
77.7991	14.1693221671579\\
77.8991	14.1440745372286\\
77.9991	14.1408338750875\\
78.0991	14.163897847697\\
78.1991	14.1882573095619\\
78.2991	14.2053377260004\\
78.3991	14.1983710886787\\
78.4991	14.2126732862668\\
78.5991	14.1920835442121\\
78.6991	14.1765178358024\\
78.7991	14.1817566706773\\
78.8991	14.187564482543\\
78.9991	14.1994341966952\\
79.0991	14.2108593578625\\
79.1992	14.2146540449035\\
79.2992	14.2117200243941\\
79.3992	14.2188902584234\\
79.4992	14.2150215946213\\
79.5991	14.2097274262823\\
79.6992	14.1974783831801\\
79.7992	14.211319653434\\
79.8992	14.2263479339017\\
79.9991	14.2470778188534\\
80.0991	14.2167891345335\\
80.1991	14.2204460825806\\
80.2991	14.1948201236846\\
80.3991	14.1792822022632\\
80.4991	14.1828054920921\\
80.5992	14.1996797976307\\
80.6992	14.2285855599909\\
80.7991	14.2343643767286\\
80.8991	14.2282731596256\\
80.9992	14.2279163432742\\
81.0992	14.2089782979788\\
81.1992	14.1877868394995\\
81.2992	14.1629840164352\\
81.3991	14.1486714065032\\
81.4992	14.1596298111986\\
81.5992	14.1827980198234\\
81.6992	14.2017174896715\\
81.7992	14.223279602588\\
81.8992	14.2478299740015\\
81.9992	14.2609904234744\\
82.0992	14.2843904041941\\
82.1992	14.2663514110565\\
82.2992	14.2705317502884\\
82.3992	14.2776082114584\\
82.4992	14.2992995426214\\
82.5992	14.3483845684037\\
82.6992	14.3979694908427\\
82.7992	14.4659203848992\\
82.8992	14.5085806471285\\
82.9992	14.5064307566197\\
83.0992	14.4442703915951\\
83.1992	14.4315225727141\\
83.2992	14.4365512091338\\
83.3992	14.4274707472573\\
83.4992	14.4016597136819\\
83.5992	14.5373710187888\\
83.6992	14.5584172930144\\
83.7992	14.5038900560994\\
83.8992	14.4797150376592\\
83.9992	14.4466858315593\\
84.0992	14.4280956547391\\
84.1992	14.4323125808793\\
84.2992	14.4206738044864\\
84.3992	14.4080089554308\\
84.4992	14.4284474802446\\
84.5993	14.4615916867119\\
84.6992	14.4416811313058\\
84.7993	14.4703655173926\\
84.8993	14.4298326018342\\
84.9991	14.4222227806906\\
85.0991	14.4217740236756\\
85.1992	14.4363167228597\\
85.2992	14.4580598616084\\
85.3992	14.484917244594\\
85.4992	14.48465911688\\
85.5992	14.4771985964127\\
85.6992	14.4851057578593\\
85.7992	14.4942812314417\\
85.8992	14.5687183171838\\
85.9992	14.6253665871442\\
86.0992	14.6793119612213\\
86.1992	14.7418961198112\\
86.2992	14.8096088855074\\
86.3992	14.8955078321198\\
86.4992	14.9514048465008\\
86.5992	14.9746735655628\\
86.6992	14.9918946404131\\
86.7992	14.9918092283751\\
86.8992	15.0328722875433\\
86.9992	15.0710634780984\\
87.0992	15.119517373113\\
87.1992	15.1889198466917\\
87.2992	15.2702781818305\\
87.3992	15.3303609048585\\
87.4992	15.3978622897488\\
87.5992	15.4458878712343\\
87.6992	15.4754549830544\\
87.7992	15.5135402143063\\
87.8992	15.5787282242886\\
87.9992	15.6578095198257\\
88.0992	15.7184141667849\\
88.1992	15.797131068191\\
88.2992	15.9121206600449\\
88.3993	16.0037939175583\\
88.4993	16.0966096521531\\
88.5992	16.2107461440975\\
88.6993	16.3338510096388\\
88.7993	16.4812315238804\\
88.8993	16.6164502948407\\
88.9993	16.7194703655959\\
89.0993	16.840013765722\\
89.1993	16.9757777091093\\
89.2993	17.1483811593386\\
89.3993	17.301485013513\\
89.4993	17.4646799693182\\
89.5993	17.6223611327695\\
89.6993	17.7630105992379\\
89.7993	17.8878458843668\\
89.8993	18.007049196464\\
89.9993	18.1495991369404\\
90.0993	18.3019985667377\\
90.1993	18.4416332444876\\
90.2993	18.588322852175\\
90.3993	18.7661850188141\\
90.4993	18.920541954158\\
90.5993	19.073770692915\\
90.6993	19.2117054587143\\
90.7993	19.3244966146589\\
90.8993	19.455970647489\\
90.9993	19.5578909280003\\
91.0993	19.6827894560278\\
91.1993	19.8088695374916\\
91.2993	19.9386468445185\\
91.3993	20.1042968953267\\
91.4993	20.2807342048579\\
91.5993	20.4381078477925\\
91.6993	20.5810902361629\\
91.7993	20.7098312215329\\
91.8993	20.8107260985712\\
91.9993	20.9077965799387\\
92.0993	21.0203292757805\\
92.1993	21.1282894748792\\
92.2993	21.2264136048604\\
92.3993	21.3338023692893\\
92.4993	21.4437340727469\\
92.5993	21.5329629785885\\
92.6993	21.6238656634282\\
92.7993	21.6953473467057\\
92.8993	21.7381485121181\\
92.9994	21.7616292411888\\
93.0993	21.7812347916035\\
93.1994	21.7991069218585\\
93.2994	21.8045143858437\\
93.3994	21.8260317131971\\
93.4994	21.8607325761881\\
93.5994	21.9020387445783\\
93.6994	21.9031285951246\\
93.7994	21.8941577283582\\
93.8994	21.9004920773929\\
93.9994	21.9161481508155\\
94.0994	21.9284190061385\\
94.1994	21.9331764079043\\
94.2994	21.9186700360517\\
94.3994	21.9132726338799\\
94.4994	21.9219987142783\\
94.5994	21.9740076913452\\
94.6994	22.0093170473418\\
94.7994	22.0338347905602\\
94.8994	22.0701739947251\\
94.9993	22.0973633121415\\
95.0993	22.1127844627711\\
95.1993	22.1180648609109\\
95.2994	22.1659030587333\\
95.3994	22.2064853942793\\
95.4994	22.2453623292833\\
95.5994	22.2851101290777\\
95.6994	22.3012685029465\\
95.7994	22.2854643655275\\
95.8994	22.2685412018816\\
95.9994	22.2592674885984\\
96.0994	22.2448614916407\\
96.1994	22.2237901393059\\
96.2994	22.1988220346347\\
96.3994	22.1686930386739\\
96.4994	22.1477391213059\\
96.5994	22.1632074925108\\
96.6994	22.1850428895273\\
96.7994	22.1937880158116\\
96.8994	22.2044404170761\\
96.9994	22.2113789733055\\
97.0994	22.2133510680556\\
97.1994	22.2193779254954\\
97.2994	22.2146240511467\\
97.3994	22.2000504524495\\
97.4994	22.1501258492248\\
97.5994	22.1833947266395\\
97.6994	22.2179978601312\\
97.7994	22.2102620453259\\
97.8994	22.2186896360936\\
97.9994	22.2302358811347\\
98.0994	22.2212060559399\\
98.1995	22.2196490823757\\
98.2995	22.1923914917398\\
98.3995	22.1769710723383\\
98.4995	22.1598372487577\\
98.5995	22.137344485037\\
98.6995	22.1664327113965\\
98.7995	22.1714805420668\\
98.8995	22.1637964946197\\
98.9995	22.1509681672009\\
99.0995	22.141467036038\\
99.1995	22.0919337530234\\
99.2995	22.049738603316\\
99.3995	22.0304013864529\\
99.4995	21.9496416478659\\
99.5995	21.9022124079266\\
99.6995	21.8787669349101\\
99.7995	21.9381213056482\\
99.8995	22.0239277671382\\
99.9994	22.0831708256772\\
100.0994	22.0676330914829\\
100.1995	22.0733279075969\\
100.2995	22.0929422522225\\
100.3995	22.0954127665185\\
100.4995	22.095541749071\\
100.5995	22.1049484415121\\
100.6995	22.1085718874683\\
100.7995	22.0945451324079\\
100.8995	22.0942829045946\\
100.9995	22.0940223426241\\
101.0995	22.0935356966078\\
101.1995	22.0932093699093\\
101.2995	22.0933093377484\\
101.3995	22.0934093776241\\
101.4995	22.0934365292087\\
101.5995	22.0934084297742\\
101.6995	22.0933262342438\\
101.7995	22.09343629973\\
101.8995	22.0935774245227\\
101.9995	22.0936895991411\\
102.0995	22.093779709347\\
102.1995	22.0937618536597\\
102.2995	22.093786934392\\
102.3995	22.0938328984953\\
102.4995	22.0939630034006\\
102.5995	22.0941261770569\\
102.6995	22.0941620982137\\
102.7995	22.094166356819\\
102.8995	22.0942027377416\\
102.9995	22.0942511936992\\
103.0995	22.0942569530062\\
103.1996	22.0943492772822\\
103.2996	22.0943780946383\\
103.3996	22.0944070203105\\
103.4996	22.0944458591522\\
103.5996	22.0944748781489\\
103.6996	22.0944932693347\\
103.7996	22.0945658697865\\
103.8996	22.094597577615\\
103.9996	22.0946071136933\\
104.0996	22.0946814087326\\
104.1996	22.0952791525478\\
104.2996	22.0959660080031\\
104.3996	22.10151633538\\
104.4996	22.1252268561083\\
104.5996	22.113276923546\\
104.6996	22.0845046382211\\
104.7996	21.9004818584975\\
104.8996	21.4159268652411\\
104.9996	20.8958450266081\\
105.0996	20.7028848747349\\
105.1996	20.5753231716514\\
105.2996	20.4937436485724\\
105.3996	20.4672536500285\\
105.4996	20.2212894633664\\
105.5996	19.9948491172307\\
105.6996	18.5265708853749\\
105.7996	17.4461897009212\\
105.8996	17.2654885709094\\
105.9996	17.2664207325661\\
106.0996	17.2581935224888\\
106.1996	17.2579613705238\\
106.2996	17.2572774276569\\
106.3996	17.2450342680427\\
106.4996	17.1613154018743\\
106.5996	17.1174415259511\\
106.6996	17.1304138796603\\
106.7996	17.139394466888\\
106.8996	17.1495848365891\\
106.9996	17.1455588764183\\
107.0996	17.1526824158566\\
107.1996	17.1619284721067\\
107.2996	17.1667781745358\\
107.3996	17.1709887633806\\
107.4996	17.1290102510854\\
107.5996	17.1427498830893\\
107.6996	17.1489776662968\\
107.7996	17.1607618945732\\
107.8996	17.176729219624\\
107.9996	17.1896954564799\\
108.0996	17.1987722099104\\
108.1997	17.2079100087016\\
108.2996	17.2137213637243\\
108.3997	17.2152879726541\\
108.4997	17.219920936746\\
108.5997	17.2260019057404\\
108.6997	17.2298651073005\\
108.7997	17.235513169275\\
108.8997	17.2404766823762\\
108.9997	17.2455133985208\\
109.0997	17.2499449771238\\
109.1997	17.2507301507826\\
109.2997	17.2534912071817\\
109.3997	17.2551551350163\\
109.4997	17.2599485514363\\
109.5997	17.2639839583641\\
109.6997	17.2638433600325\\
109.7997	17.2657772683803\\
109.8997	17.2708165764159\\
109.9996	17.273807208202\\
110.0996	17.2754352331783\\
110.1996	17.2787185864311\\
110.2996	17.2812126858724\\
110.3996	17.2843700112249\\
110.4996	17.2938963802369\\
110.5996	16.9490006965486\\
110.6996	16.9147582971586\\
110.7997	16.6780356572963\\
110.8997	15.4615726303194\\
110.9997	15.3386366073887\\
111.0997	16.1151390639016\\
111.1997	16.5208216724923\\
111.2997	16.5551424893241\\
111.3997	16.4964151141859\\
111.4997	16.368090528418\\
111.5997	16.2568097688681\\
111.6997	16.1365156697749\\
111.7997	16.059197093485\\
111.8997	16.0069415120775\\
111.9997	15.98668577486\\
112.0997	15.9569545412333\\
112.1997	15.8988326173261\\
112.2997	15.8571121388759\\
112.3997	15.8494210865729\\
112.4997	15.8490896419489\\
112.5997	15.8565871074363\\
112.6997	15.8499662393899\\
112.7997	15.8533462865162\\
112.8997	15.8301697162408\\
112.9997	15.8183051300998\\
113.0997	15.805417258159\\
113.1997	15.7978094091521\\
113.2997	15.7978168713438\\
113.3997	15.8016225979013\\
113.4997	15.8016861566091\\
113.5997	15.8016522462549\\
113.6997	15.8016641644263\\
113.7998	15.8016734655532\\
113.8998	15.8016952310298\\
113.9998	15.8016732924995\\
114.0998	15.8016832530126\\
114.1998	15.8017381911611\\
114.2998	15.8017286429381\\
114.3998	15.8017499279465\\
114.4998	15.8017605421662\\
114.5998	15.8017501442394\\
114.6998	15.8017396725557\\
114.7998	15.8017726841304\\
114.8998	15.8017856188434\\
114.9997	15.8018086982191\\
115.0997	15.8017885561194\\
115.1997	15.8017578611186\\
115.2997	15.8017802208812\\
115.3997	15.8017493663287\\
115.4997	15.80174972184\\
115.5997	15.8017290093235\\
115.6997	15.8017082969047\\
115.7998	15.801699314625\\
115.8997	15.801702770376\\
115.9997	15.8016842015034\\
116.0998	15.8016747852338\\
116.1998	15.8016986869323\\
116.2998	15.8016792178457\\
116.3998	15.8016925035815\\
116.4998	15.8016849131547\\
116.5998	15.8017093877685\\
116.6998	15.8017108593436\\
116.7998	15.8017123890478\\
116.8998	15.8016949950094\\
116.9998	15.8016743241862\\
117.0998	15.8016655122803\\
117.1998	15.8016921017908\\
117.2998	15.8016953089528\\
117.3998	15.8017219754251\\
117.4998	15.8017266439005\\
117.5998	15.8017515013941\\
117.6998	15.8017654488015\\
117.7998	15.8017700846004\\
117.8998	15.8017855931951\\
117.9998	15.8018122212171\\
118.0998	15.8018389888273\\
118.1998	15.8018866966856\\
118.2998	15.8019239005826\\
118.3998	15.8019508956066\\
118.4998	15.8019670458036\\
118.5998	15.8019826029072\\
118.6998	15.8019999789754\\
118.7998	15.8020164208786\\
118.8998	15.8019988961314\\
118.9998	15.8020479311487\\
119.0999	15.8020530658649\\
119.1999	15.8020473046971\\
119.2999	15.8020748473934\\
119.3999	15.802060028706\\
119.4998	15.8020892961438\\
119.5999	15.8020848604122\\
119.6999	15.8021244654085\\
119.7999	15.8021523310936\\
};
\addplot [color=black,dashed,forget plot]
  table[row sep=crcr]{%
41.6291	-5\\
41.6291	25\\
};
\addplot [color=black,dashed,forget plot]
  table[row sep=crcr]{%
45	-5\\
45	25\\
};
\addplot [color=black,dashed,forget plot]
  table[row sep=crcr]{%
60	-5\\
60	25\\
};
\addplot [color=black,dashed,forget plot]
  table[row sep=crcr]{%
73.3591	-5\\
73.3591	25\\
};
\end{axis}
\end{tikzpicture}%
  %
  \vspace*{1em}
  \hspace{3cm}{\footnotesize {(a)~Timeline view of the example in Figure~1}} \\
  \vspace*{0.1em}
  %
  \end{minipage}%
  %
  \hspace*{0em}
  %
  \begin{minipage}{.33\textwidth}
  %

  \tikzsetnextfilename{tikz-vertical}


  \setlength{\figurewidth}{.85\columnwidth}
  \setlength{\figureheight}{1.19\figurewidth}
  %

  \pgfplotsset{
    trim axis right,
    yticklabel style={rotate=90},
  }
  %

  \footnotesize\centering%
  % This file was created by matlab2tikz.
%
%The latest updates can be retrieved from
%  http://www.mathworks.com/matlabcentral/fileexchange/22022-matlab2tikz-matlab2tikz
%where you can also make suggestions and rate matlab2tikz.
%
\begin{tikzpicture}

\begin{axis}[%
width=\figurewidth,
height=\figureheight,
at={(0cm,0cm)},
scale only axis,
xmin=-10,
xmax=4,
xtick={-8, -4,  0,  4},
xlabel={Horizontal ($y$) displacement (meters)},
ymin=-8,
ymax=2,
ytick={-8, -6, -4, -2,  0,  2},
ylabel={Vertical ($z$) displacement (meters)},
axis background/.style={fill=white}
]
\addplot [color=blue!50,solid,line width=1.2pt,forget plot]
  table[row sep=crcr]{%
0.000260210758031161	-3.67442395606126e-06\\
0.0102712320685737	-0.0002458943779126\\
0.0335731570319752	-0.000867555610648548\\
0.057201296717475	-0.00164908290059347\\
0.0707168740397457	-0.00225945716170036\\
0.0677773541687605	-0.00240873099198311\\
0.0479789601523208	-0.00201275368024285\\
0.0189094599728637	-0.0011438902287891\\
-0.00119098878427244	-0.000437717739962379\\
-0.00143846451077478	-0.000418108720434245\\
-0.00107098743284276	-0.000412039076726102\\
-0.000935272248149048	-0.000408609267189677\\
-0.000855858293766857	-0.000415156262695271\\
-0.000746438652402328	-0.000430545152882792\\
-0.000664283260123917	-0.000417111077371552\\
-0.000614316590182789	-0.000449028030026484\\
-0.000596454914280173	-0.000459766590344015\\
-0.000484296343346836	-0.000525029000214083\\
-0.000276266016936894	-0.000631663800065451\\
-2.15519503298495e-05	-0.00077914003224697\\
0.000247717582902355	-0.000983613676550685\\
0.000511843608659939	-0.0012049364649808\\
0.000806759217886567	-0.0014483455524148\\
0.00111775019534061	-0.00166217972830578\\
0.00141699265568452	-0.00184578808679025\\
0.00166315253081889	-0.00198378513438774\\
0.00184904284307979	-0.0020641186873023\\
0.0019273364517855	-0.00213357798698287\\
0.00194607103180675	-0.00215945754057502\\
0.00196400690666429	-0.00217825277888437\\
0.00195757939965117	-0.00217901134419224\\
0.00196056010360805	-0.00216844162212923\\
0.00198577637254989	-0.00218145544748646\\
0.00199039322197221	-0.00221363238028733\\
0.00198639480470511	-0.00221306663372019\\
0.00199516827832248	-0.00223224566316764\\
0.002008590558507	-0.00224410172965115\\
0.00204125343888043	-0.00222130159140063\\
0.0020778063057832	-0.0022214124741742\\
0.00215153910470869	-0.00221099318958885\\
0.00223976203992746	-0.00217822561832802\\
0.00234874017884928	-0.00214944699031699\\
0.00244470262810223	-0.0021259222080924\\
0.00250485138627578	-0.00209434292770406\\
0.00252644490618649	-0.0020969566475503\\
0.00256231145979135	-0.00211859939922338\\
0.00260255720751133	-0.00222572910784557\\
0.00265384340850993	-0.00237968895048926\\
0.00268980837485693	-0.00260325650851498\\
0.00378851480903764	-0.00284761677269728\\
0.00450687583182346	-0.00323599065531976\\
0.00442304246635416	-0.00328069196633959\\
0.00288472921691849	-0.00205640740273174\\
-0.00326129853448354	0.00430994461680299\\
-0.0144479319638725	0.0142294511886678\\
-0.0312280256636075	0.0373695631916677\\
-0.0572017353719534	0.0860090931344231\\
-0.0883426167244363	0.159456710437599\\
-0.115534795698489	0.258268647835183\\
-0.14024207322084	0.375944824595903\\
-0.162760880350252	0.508758908751875\\
-0.182710512191425	0.652760852263167\\
-0.198105942769541	0.799520201097134\\
-0.206089182668137	0.928059632663726\\
-0.205068673453374	1.04385452208345\\
-0.193134706990629	1.14758040415155\\
-0.168550555873199	1.24242167109999\\
-0.136756980195444	1.32763114480986\\
-0.102768618781284	1.40317879667564\\
-0.0729226098924146	1.46526058832243\\
-0.0550441394320347	1.50559523279564\\
-0.0443696094384813	1.5267967242517\\
-0.0389269388335714	1.5361486355262\\
-0.0383726943778688	1.54089354470306\\
-0.0343263767268808	1.54352322319183\\
-0.0335532147045744	1.54352892010597\\
-0.035794151034056	1.54149641804949\\
-0.0364123959075847	1.5415439717393\\
-0.0371929617254714	1.54169934706887\\
-0.0370941320143932	1.54176092713499\\
-0.0373784730970154	1.54207732918316\\
-0.0373601498847929	1.54206230759595\\
-0.0373707269013475	1.54202351749899\\
-0.0373647191325441	1.54201244788845\\
-0.0373628473773109	1.54199090867014\\
-0.0373624173349566	1.54200354342865\\
-0.0373578124143597	1.54200995806663\\
-0.0373763213515004	1.54200541031523\\
-0.0373924034375014	1.5420221082981\\
-0.0373796982575754	1.54201717625822\\
-0.0373552092985376	1.54200722952425\\
-0.0373446197394121	1.54201719106489\\
-0.0373134386288015	1.54200011327157\\
-0.0373037534349698	1.54198194184807\\
-0.0373200080304009	1.54199629796064\\
-0.0373313022053028	1.5420145780891\\
-0.0373319447510409	1.54200931557319\\
-0.0373403719341228	1.54200494096651\\
-0.0373522523920152	1.54199560467658\\
-0.0373376182279052	1.54200264107049\\
-0.0372656287603154	1.5419759547974\\
-0.0371565623852565	1.54197593113892\\
-0.0370309346225977	1.54201663338309\\
-0.0369037339200702	1.54207113175278\\
-0.0367414617232448	1.54214388639219\\
-0.0365503616743182	1.54218468689913\\
-0.0363469664144804	1.54217702295392\\
-0.0360904804831857	1.54214994235534\\
-0.035805586553279	1.54212788197515\\
-0.0354876227071533	1.54212592554528\\
-0.0351931437209326	1.54212356460807\\
-0.0349212751096168	1.54206807167067\\
-0.0347374049883228	1.5420459520212\\
-0.0346611930038597	1.54203517816745\\
-0.0346604684739811	1.54203649083949\\
-0.0346660460853389	1.54203209496972\\
-0.0346557427967068	1.54203093008281\\
-0.0346431518619887	1.54203796549754\\
-0.0346313174037261	1.54203097313912\\
-0.034625098066215	1.54202348433699\\
-0.034602530563119	1.54202738494563\\
-0.0345863589245198	1.54200625138069\\
-0.0345286410032903	1.54198302590937\\
-0.0344620330057424	1.54199974496426\\
-0.0343974854602994	1.54196451274011\\
-0.0343441202038303	1.54194048896636\\
-0.034321202910414	1.54191129500659\\
-0.034312718621088	1.54190106748265\\
-0.0343181981052	1.54189047369346\\
-0.0343219674579147	1.54188468744621\\
-0.0343389922729177	1.54187151909123\\
-0.0343687157638485	1.54183799309676\\
-0.0343846923162504	1.54184267242264\\
-0.0343945053388946	1.54190396741403\\
-0.0343812057987325	1.54193910617789\\
-0.0343666814923506	1.54194988847452\\
-0.0343391925084134	1.54195243810643\\
-0.0343225303447408	1.54193366707756\\
-0.0340053713880646	1.54211416804219\\
-0.0314139242965693	1.54298439710552\\
-0.0357979630469518	1.54992579013792\\
-0.0424617790347373	1.56085551542512\\
-0.0462373953981838	1.57277341083311\\
-0.0510006238372471	1.58382910656175\\
-0.050098484586266	1.58958289474183\\
-0.0434690011456014	1.58993526487872\\
-0.0370091558997099	1.5871542920739\\
-0.0257096184442991	1.58311470229995\\
-0.0114424900233001	1.58084576751837\\
0.0014243087269484	1.58006102179705\\
0.0138687568903572	1.58274075561791\\
0.0254278342017447	1.58489018903617\\
0.0329161131192114	1.58392780531166\\
0.0379898188479323	1.58249394632439\\
0.0370933926695493	1.583508656071\\
0.0372863192000549	1.58409685482106\\
0.0369117523927194	1.58427920778441\\
0.0371697340582317	1.58439617485877\\
0.0368224910245633	1.58451740358516\\
0.0367822166434585	1.58454952795947\\
0.0368241357225899	1.58456730594601\\
0.0368411038407805	1.58458283905804\\
0.0369878405508126	1.58459418460415\\
0.0373554330219861	1.58459191559299\\
0.0378474957785927	1.58459172902664\\
0.038372750224283	1.58454651717498\\
0.038950372321461	1.58449584783598\\
0.0396378679052803	1.58450134407046\\
0.04040852317565	1.58453677734311\\
0.0412090875577498	1.5845874190054\\
0.0420243470288048	1.58466717846514\\
0.0428508176447309	1.58473602553012\\
0.0436700868618073	1.58477282848398\\
0.0443856897465322	1.58481028254348\\
0.0450459193910378	1.58485782609664\\
0.0456535518644152	1.58489981268373\\
0.046220517401486	1.58497561030283\\
0.0467210955169549	1.58506752923102\\
0.0471164509946404	1.58514973834143\\
0.0474184682238714	1.58521129906739\\
0.0476034085955188	1.5852216880521\\
0.0476398899172165	1.58521423412462\\
0.0476445111101652	1.58521478499722\\
0.0476528462523068	1.58521337188343\\
0.0476605990770734	1.58520306345473\\
0.0476622856087775	1.58518123598875\\
0.0476659745215829	1.58515115420213\\
0.0476882834556997	1.58514778155304\\
0.0477336620347344	1.58511927760971\\
0.0478141552040557	1.58505608514997\\
0.047861718705458	1.58504596789422\\
0.0478750870265119	1.58505140662749\\
0.04789928042078	1.58506283986058\\
0.0479160976170036	1.58506979340498\\
0.0479144690779222	1.58508353572647\\
0.0479115242593349	1.58507836687417\\
0.0479048685926403	1.58508582694734\\
0.0479162504617962	1.58507674386635\\
0.0479233160329214	1.58506227327483\\
0.0479501851542623	1.5850656868254\\
0.0479806103547493	1.5850647512088\\
0.0479843358542073	1.58505969285688\\
0.0479795167853334	1.58536847526926\\
0.0479662771557492	1.58603567989362\\
0.0478243468337763	1.58707446641453\\
0.0472292478039565	1.58850468740454\\
0.0426705017998727	1.59183494753734\\
0.0257514712994476	1.59742296496244\\
-0.0018001477411432	1.596310014927\\
-0.0366513040243938	1.57663095330981\\
-0.0657885379921491	1.53052202775271\\
-0.0907502927816075	1.4546289159815\\
-0.103747963888108	1.34097549974381\\
-0.111773781644225	1.19369825789036\\
-0.116644609219793	1.01326686139079\\
-0.124265281141211	0.814393089441181\\
-0.124323053287607	0.618905789082734\\
-0.106351967753767	0.454722551290688\\
-0.0759939889049813	0.322607148548287\\
-0.0472878961217309	0.228007032018246\\
-0.0266739234541846	0.168084399168128\\
-0.0088173775240047	0.133727346506577\\
0.00307304368913752	0.119065271127531\\
0.0128247969960355	0.105649364398891\\
0.0218057147669135	0.0994596225269757\\
0.0274817265497567	0.0989833443893964\\
0.0336357315208068	0.0995184724354074\\
0.0383190900680808	0.100766855051317\\
0.0424931215034037	0.100520532757774\\
0.0476894360146261	0.0999037333211881\\
0.0525619582569585	0.0999389418282206\\
0.0557847316028047	0.0992448341614511\\
0.0584437616561918	0.09776660623978\\
0.0605617056389201	0.0957263022252292\\
0.0615987972372039	0.0944182096893502\\
0.0622992320057191	0.0946340791962413\\
0.0623116386783462	0.0947808090518867\\
0.0623095527347765	0.0947971900589101\\
0.0623205104742475	0.0948306999614386\\
0.0622085014687969	0.0947865350520374\\
0.0619792272864259	0.0946789163434743\\
0.0617224314887911	0.0945562599883032\\
0.0617160860123846	0.094538564425019\\
0.061706558516662	0.0944875643564179\\
0.0616939221277319	0.09446696278419\\
0.0616661817543836	0.0944119044352194\\
0.0615582475257142	0.0943291055460289\\
0.0617994123381894	0.0944155421573085\\
0.0617985470142607	0.0944052221285041\\
0.0616451594338062	0.0942945443033824\\
0.0615743473401201	0.0942471607985819\\
0.0615780365087422	0.0942496598677039\\
0.0616092852437399	0.0942099651291843\\
0.0616016384412232	0.0942046600872866\\
0.0615959760712819	0.0941728210609827\\
0.0615975658582655	0.0941582346089806\\
0.0615939254050893	0.094158826371979\\
0.0616270737105236	0.0941347417523614\\
0.0616543805955959	0.094073076037539\\
0.0616741219640709	0.0940098569217049\\
0.0616808739585262	0.0939746097947669\\
0.061692915286594	0.093936138224714\\
0.061704267596278	0.0938468576218998\\
0.0617189053201414	0.0937302854776541\\
0.0617119319273672	0.0936191741478376\\
0.0617369503947725	0.0935200747477101\\
0.0617864294766841	0.0934730390842687\\
0.0618229019107525	0.0933985432530415\\
0.0618244021755074	0.0933449972390419\\
0.0618228891768131	0.0933289775901578\\
0.0618200359306954	0.0933202261174249\\
0.0617959109965546	0.0933062169207751\\
0.0617880014460598	0.0933087291049048\\
0.0617859726642963	0.0933333621272713\\
0.0617795515447052	0.0933301222048857\\
0.0617670864227183	0.0933210335593025\\
0.061752633559469	0.0933586748129884\\
0.0617356777381016	0.0933621514790684\\
0.0617128223576592	0.0933046409196351\\
0.0616915439095744	0.0932509541313582\\
0.061693239218602	0.0932054514039776\\
0.0617125836738403	0.0932418243008388\\
0.0617524060417403	0.0933202353675049\\
0.0618483049800352	0.0934323749172582\\
0.0620261525122427	0.0935837766918592\\
0.0623278845813391	0.0953581476585911\\
0.0585694267715656	0.105669925560195\\
0.0529555232206822	0.114770449233945\\
0.0447421733713035	0.118738953522038\\
0.0364113522223449	0.1175567972416\\
0.030536750978422	0.11547001688805\\
0.0292977111080078	0.111208789497664\\
0.0304484312345552	0.105229713343622\\
0.0276753525322587	0.0940636441714275\\
0.0293683336710871	0.084187329663414\\
0.0342574639959024	0.0780684582545626\\
0.0378266355180149	0.0711858212898571\\
0.042958430988015	0.0672907208604267\\
0.0478969842511562	0.0666351370124955\\
0.0537645678017288	0.0678274077682514\\
0.0565874069115487	0.0681618803454361\\
0.0570432150004282	0.0617622999941732\\
0.0549374558504477	0.0594432684399588\\
0.0540292401117629	0.058307521216732\\
0.0539069811221243	0.0578226651995898\\
0.053848992734149	0.0579419205874981\\
0.0532210938252047	0.0580564549815078\\
0.0533005940043918	0.0582918678714977\\
0.0534856195938093	0.0584990486404089\\
0.0533830299388177	0.0585983267377117\\
0.0537294424378564	0.0586041309134498\\
0.0538484323378916	0.0586152893500532\\
0.0537940404874519	0.0586412509034021\\
0.0537780894192282	0.058637763411107\\
0.0537904869567816	0.0586528170762109\\
0.0537894152941592	0.0586719568938895\\
0.053791543593013	0.058669244510848\\
0.0537992548146937	0.0586847186941352\\
0.0537767664940754	0.0586736404093379\\
0.0537546521246324	0.0586544119629624\\
0.0537647772239949	0.0586447814650516\\
0.053778593211857	0.0586540268885862\\
0.0538032386740313	0.0586613542392155\\
0.0538003228795521	0.0586291601039174\\
0.053802932722771	0.0586018515599341\\
0.053820962730551	0.058539619736012\\
0.053837911637774	0.0584464594514165\\
0.0538287228472683	0.0583856151550809\\
0.0538308855063693	0.0583574752780995\\
0.0538313396512644	0.0583336396212132\\
0.0538411788746932	0.0583234095956914\\
0.0538388410371919	0.0582948544647327\\
0.0538259765539113	0.0582580364523407\\
0.0538096819608507	0.0582555459156722\\
0.0538086064471303	0.0582669484405501\\
0.0538187243160513	0.0582738666670582\\
0.0538246468232161	0.0582696093726966\\
0.05382423731507	0.0582531040019116\\
0.0538060293577191	0.0582479858377616\\
0.0538387753873312	0.0582410153618806\\
0.05390953144573	0.0582461264408677\\
0.0540020006488893	0.0582894836721061\\
0.0541100634838074	0.0583505079497559\\
0.052777202305488	0.0583420924426811\\
0.0404000755909405	0.0651578147248536\\
0.0213813945474143	0.087880957764466\\
0.00324118827488898	0.131094472763366\\
-0.0135885527713667	0.191161436502938\\
-0.0333014406126662	0.267041478962153\\
-0.0563992324393112	0.361442563813011\\
-0.0824956433641379	0.472605300265867\\
-0.111890834865152	0.592225282477487\\
-0.149186918492778	0.71464527732138\\
-0.197091907832262	0.836875343515684\\
-0.263509021254066	0.95593661731606\\
-0.339634108347918	1.05653456836152\\
-0.419549642415297	1.13908535386339\\
-0.499676840160571	1.20676132628462\\
-0.582412344759808	1.25298846154696\\
-0.669261770693532	1.28110248678159\\
-0.759475306747379	1.29126223082669\\
-0.857652104306589	1.28863593751157\\
-0.968918291219603	1.27804350946329\\
-1.08594046103357	1.26628372362519\\
-1.20070838920707	1.26629193485901\\
-1.30441949745218	1.27152801770248\\
-1.39232380627356	1.27548074046181\\
-1.47399888037022	1.26875945636178\\
-1.55639276385977	1.25784630131065\\
-1.65106721551436	1.2468523712326\\
-1.75057753210542	1.24559547424528\\
-1.8422420560831	1.26240172691057\\
-1.93125193397433	1.27124900319487\\
-2.02543481792669	1.26366329262546\\
-2.13523620878966	1.24179553202797\\
-2.26087807707304	1.21712004160551\\
-2.3956885565139	1.20817568541068\\
-2.53030080178051	1.22406065768358\\
-2.66385287037412	1.2412055220809\\
-2.79822454540735	1.24285678470313\\
-2.93560468947533	1.22671841508839\\
-3.07060312009828	1.21923877414383\\
-3.20061462508619	1.23421181488179\\
-3.32922533224102	1.25453812120164\\
-3.45864273746668	1.25834812245271\\
-3.59170634205034	1.24021184961462\\
-3.72771952284923	1.22573871047321\\
-3.8647997889403	1.23266757102953\\
-3.99551959645747	1.25358905597079\\
-4.1210848742747	1.27209190698592\\
-4.24784344258378	1.26860478168629\\
-4.37351407160272	1.2516632037741\\
-4.49512249846681	1.24821733128852\\
-4.6110743069099	1.26820033668468\\
-4.71985834389588	1.28769323495407\\
-4.82474811257895	1.28722989141933\\
-4.92579935397892	1.27355127141319\\
-5.01627369986288	1.26013314047796\\
-5.08703060083318	1.26084161654139\\
-5.13348621751251	1.26649173515053\\
-5.15773934957293	1.26997922296582\\
-5.16484315389958	1.25782556171309\\
-5.15975891273735	1.24157260839483\\
-5.15499787542522	1.237784589997\\
-5.15869144408788	1.25943358682782\\
-5.16617032336516	1.28248267670459\\
-5.16857590969198	1.29222263031762\\
-5.1640326067594	1.28743940574469\\
-5.15328777015126	1.27875565739057\\
-5.13755685125978	1.27972260656352\\
-5.11334530042771	1.29709517150194\\
-5.08282865775955	1.31582790833879\\
-5.04865477085403	1.32664329557043\\
-5.00515511381199	1.3233948540434\\
-4.95056821760295	1.32049842199254\\
-4.88885856647489	1.3308269820375\\
-4.8261164014371	1.34370850949774\\
-4.76741437609044	1.34792023227432\\
-4.71587080204179	1.34909606897364\\
-4.67164842132482	1.34133459319798\\
-4.6359416659881	1.33666884070767\\
-4.60962597628387	1.33811147023108\\
-4.59333243153338	1.34763219999723\\
-4.58613867195975	1.35442669131827\\
-4.58452689228826	1.35453160790221\\
-4.58486342703224	1.35612559104801\\
-4.58529245281606	1.3559221601637\\
-4.58867605687431	1.35925755869988\\
-4.5932364259463	1.35907580176939\\
-4.59459930127602	1.3581616423033\\
-4.59241338191913	1.3581436337441\\
-4.58865944718313	1.36088931085651\\
-4.58739354928017	1.36336855703705\\
-4.58971616443557	1.36385580620955\\
-4.59115940564777	1.36455991628969\\
-4.59041620166713	1.36712598162226\\
-4.58742367375875	1.36588979458654\\
-4.58197773072229	1.3664071947916\\
-4.57343583846731	1.36834767210121\\
-4.56470954280626	1.36870178897976\\
-4.55552422762491	1.3718055485821\\
-4.54950924727798	1.374338263561\\
-4.5456069464573	1.37261174384928\\
-4.54508001703309	1.37001738249561\\
-4.54331790865044	1.36186781733402\\
-4.54124261830499	1.36115821266521\\
-4.53765201276837	1.36239435941117\\
-4.52823119488056	1.36439714455355\\
-4.51015942636286	1.36503643146705\\
-4.4791952963071	1.36076454842731\\
-4.43466739947716	1.35502130570851\\
-4.38007509761583	1.35143457616591\\
-4.32266242659851	1.35000775997887\\
-4.2642919519607	1.36196685487273\\
-4.20658480095356	1.38069410123625\\
-4.15623099663849	1.40032664308035\\
-4.11216883914232	1.41085452465605\\
-4.07243746787086	1.41356459236759\\
-4.03711618209776	1.41354945485674\\
-3.99914222080801	1.41028337909857\\
-3.95505296443544	1.41201894620182\\
-3.91081746633428	1.4209544346509\\
-3.87368060349783	1.43170599596489\\
-3.8456396884958	1.43748108161417\\
-3.82622048224307	1.44463133665965\\
-3.81054758521731	1.44418066293562\\
-3.79764087818537	1.44393083992823\\
-3.78098825568993	1.45000241383798\\
-3.76432788946191	1.46713178895981\\
-3.74867051698107	1.48331228297831\\
-3.73822012822038	1.49437898518899\\
-3.73878241188493	1.49734326646877\\
-3.75852582503515	1.4972651719508\\
-3.80141550582664	1.4937035309323\\
-3.85124012575909	1.49218838794157\\
-3.88258638426903	1.49675341548551\\
-3.89586312163288	1.51359353251913\\
-3.89010468063993	1.52387843167478\\
-3.87121534099541	1.52920663534274\\
-3.84423925541128	1.53413072898185\\
-3.80951411180423	1.53129204602995\\
-3.77025078882086	1.52613592613013\\
-3.7268097849738	1.52515108968948\\
-3.67945020897575	1.5340635488808\\
-3.6258530694606	1.54513191288531\\
-3.56159202052323	1.54819901661557\\
-3.4842784839521	1.53988555367286\\
-3.39499504141447	1.5282587910066\\
-3.29490455864094	1.5193987017558\\
-3.19267290595737	1.52413822039792\\
-3.09149742205411	1.52872981634795\\
-2.99150881903088	1.52883982529423\\
-2.89249978592094	1.51544819069292\\
-2.79755398826234	1.50016444364581\\
-2.71134719714819	1.50298955602329\\
-2.63012338730897	1.51847012831446\\
-2.5527038902033	1.51810967778039\\
-2.4760190411787	1.48291862247725\\
-2.39763255516995	1.42594725073656\\
-2.3197695334249	1.36723793222877\\
-2.25072597365392	1.33145441478502\\
-2.18524244121238	1.30357229153359\\
-2.1162822130081	1.26489172011586\\
-2.03861530978603	1.21328641401305\\
-1.96416206141358	1.17493022055721\\
-1.89115169590374	1.14022498920594\\
-1.812728830119	1.09450011548112\\
-1.73290579326216	1.05005233449297\\
-1.65276736955595	1.00601770138635\\
-1.5776370775511	0.973413859027655\\
-1.50429324553697	0.930994707139119\\
-1.43237710610207	0.884735440096251\\
-1.36448314973103	0.84106477388931\\
-1.29289237337376	0.797084632248271\\
-1.21774326398674	0.756034873741557\\
-1.14492100469741	0.720666169436757\\
-1.07084060102614	0.683903595177994\\
-0.996113187568144	0.645927009015489\\
-0.921112721146573	0.601425338450436\\
-0.846978950479588	0.557248588644902\\
-0.771641362423195	0.5140334802281\\
-0.689411465508009	0.463134964687411\\
-0.609058499286284	0.419511973136482\\
-0.529465875300972	0.371527458821112\\
-0.451921976099297	0.329868196068929\\
-0.376636433185694	0.283222523864081\\
-0.303384510548066	0.237805427349749\\
-0.231999645119537	0.19159803069378\\
-0.157425153214362	0.144515331282674\\
-0.078649918577459	0.0996670640600483\\
-0.00147741470699359	0.0579989844029496\\
0.0727856923844641	0.0163480663060191\\
0.14553051159714	-0.0191959462894982\\
0.22069046286951	-0.0635289221841635\\
0.28961420557301	-0.100766093427717\\
0.352833211159266	-0.135431874287136\\
0.420510406319071	-0.181856768918078\\
0.485785569898042	-0.225820344798269\\
0.540809182032143	-0.242805515560682\\
0.580059231194075	-0.23762527641886\\
0.609797228299106	-0.245919623647909\\
0.633378490383717	-0.261192611057989\\
0.648765319927649	-0.274112576338841\\
0.655034852608373	-0.277998262496979\\
0.643352905276975	-0.258051909270129\\
0.615173630723057	-0.245548165828985\\
0.573623018022706	-0.232331650283898\\
0.519446554874655	-0.227921640162729\\
0.453120574890037	-0.228984686551787\\
0.375481170792749	-0.23464330838994\\
0.284746764636948	-0.244852293641644\\
0.186757031862903	-0.241428105182344\\
0.0866836458354601	-0.234472842303992\\
-0.00558893865934085	-0.223460463721758\\
-0.0904132013050916	-0.229836681851829\\
-0.169834931151876	-0.254835014427867\\
-0.24792128075255	-0.292530566781041\\
-0.325449770361531	-0.331649071819332\\
-0.393550552500472	-0.358751671166969\\
-0.448777365702133	-0.36961971963297\\
-0.508343401732812	-0.386744325024546\\
-0.575590925538731	-0.425953579341921\\
-0.649055108177858	-0.479759088410694\\
-0.720953910648053	-0.526788105438257\\
-0.780255913763727	-0.548987984748706\\
-0.835242817378206	-0.559103021806795\\
-0.897792976600505	-0.593647381085016\\
-0.963787540991154	-0.640196560657724\\
-1.02880695065284	-0.67881313352363\\
-1.08171316425587	-0.691027677631627\\
-1.13767276659765	-0.708585849131625\\
-1.20439305438062	-0.754049470183998\\
-1.27592851934026	-0.810321494841347\\
-1.34715032426511	-0.855405158986444\\
-1.40928174320865	-0.874081046125578\\
-1.46697814054953	-0.889607549427502\\
-1.53171690596058	-0.935274529627635\\
-1.59975963908358	-0.997742690550556\\
-1.66450900529874	-1.04110137237678\\
-1.72251700965034	-1.05184940357751\\
-1.79085811512681	-1.07680566124156\\
-1.86862061630481	-1.13756823528895\\
-1.94628731733924	-1.19837246726147\\
-2.011492938625	-1.22422861012239\\
-2.06970191516998	-1.23388989156828\\
-2.13247856461638	-1.2716396413741\\
-2.20114944037212	-1.32915798001745\\
-2.26946773315193	-1.3724966292026\\
-2.33197779515241	-1.38715267421594\\
-2.39797134218303	-1.40983247902265\\
-2.47296900632911	-1.46388478006277\\
-2.5498650536644	-1.52111950217281\\
-2.62004147326145	-1.55164621200466\\
-2.68240778462435	-1.56289810283361\\
-2.74938760765089	-1.58784058803879\\
-2.82241156839364	-1.63853369978678\\
-2.89454718972632	-1.68972778706626\\
-2.96091174059	-1.7160430785028\\
-3.02043900156642	-1.72536863022004\\
-3.08834975917085	-1.76288906129924\\
-3.16044131589871	-1.82629186999028\\
-3.2300304859735	-1.88103367920802\\
-3.29141103158632	-1.89809780204046\\
-3.35044395529521	-1.90351811863858\\
-3.41588785608553	-1.93513704936982\\
-3.48974778085937	-1.98757199050681\\
-3.56202841606686	-2.0310973999343\\
-3.61550500660012	-2.03522706615647\\
-3.66065219613691	-2.03188281822788\\
-3.69421013613962	-2.03909501106414\\
-3.70973183940306	-2.04925876394982\\
-3.71737640574691	-2.05733978783525\\
-3.71433444533782	-2.05760186740324\\
-3.68970743873164	-2.05510798332094\\
-3.64844538446438	-2.04246551171543\\
-3.59601057015779	-2.04513288652202\\
-3.53200796806207	-2.06368392406685\\
-3.45192087669159	-2.0798408088046\\
-3.36135862133301	-2.06896884689442\\
-3.2655960125905	-2.04949263270647\\
-3.16271023948485	-2.04320290020993\\
-3.0579279780269	-2.03779187510092\\
-2.95317599629343	-2.04830361612881\\
-2.84363187116111	-2.0732039286498\\
-2.72359507062128	-2.09061928324504\\
-2.59592966849858	-2.09028259756668\\
-2.48202305948993	-2.07831975168842\\
-2.39433739678451	-2.05877456023723\\
-2.31543275444519	-2.05090305921181\\
-2.23492271288693	-2.06459802115728\\
-2.1518739920525	-2.10994737632139\\
-2.07325340833126	-2.16252202584298\\
-2.00749467526826	-2.18879571096012\\
-1.95590385058701	-2.20453206708278\\
-1.90149857702095	-2.2447233607492\\
-1.83362480643297	-2.30985640308196\\
-1.76023663628749	-2.36224821772745\\
-1.69493723194767	-2.39283736757672\\
-1.63899491768483	-2.41363049254489\\
-1.58071681381357	-2.45683661240944\\
-1.51850085691793	-2.51307828482709\\
-1.45846807702944	-2.55525215775448\\
-1.40449980931875	-2.57936969248644\\
-1.34089036509982	-2.62396849493793\\
-1.26868889255658	-2.68217575168049\\
-1.20334436004727	-2.72229976780946\\
-1.14498943893897	-2.75384881596882\\
-1.08449789840248	-2.78787769690075\\
-1.01758508721417	-2.83698469995429\\
-0.948260843872341	-2.88658247327159\\
-0.882749775158341	-2.92140847394414\\
-0.819807220988239	-2.94813651788414\\
-0.748957855548964	-2.99672730508347\\
-0.676870356432998	-3.054084367349\\
-0.611916968687297	-3.09739952902176\\
-0.551987305423895	-3.12184268828273\\
-0.488734872073677	-3.16219471742586\\
-0.421487184164642	-3.2136754917061\\
-0.356505930069138	-3.258190267434\\
-0.299497240608616	-3.28546593552994\\
-0.235458400332654	-3.33180653320148\\
-0.168405797757329	-3.39081119477694\\
-0.109343173532138	-3.43652160324937\\
-0.0582361263883722	-3.46633567016957\\
-0.00740885941265934	-3.49785916667679\\
0.047728807017295	-3.54207263449716\\
0.0994627969482871	-3.5727631367052\\
0.133857073464305	-3.57514493113326\\
0.161973591893401	-3.57846970812268\\
0.18484568189792	-3.58825292044234\\
0.201848750885363	-3.59545043420476\\
0.209840166871622	-3.59665855133728\\
0.203136412266385	-3.60254131524747\\
0.190627738127988	-3.6025690546701\\
0.180063431147471	-3.6266951971888\\
0.172843732529781	-3.67888473019644\\
0.162433201773158	-3.72758318840533\\
0.144347374117182	-3.74450792739264\\
0.121528860734418	-3.75599860482498\\
0.0932788320215467	-3.80177963608218\\
0.0535913516015316	-3.85739520237488\\
0.00428266209128347	-3.89801172321312\\
-0.0522767271329607	-3.91756209438543\\
-0.113361397978194	-3.91392999162531\\
-0.188010200679055	-3.92473225378624\\
-0.273364279613547	-3.94115285301646\\
-0.351226815658781	-3.93042054110177\\
-0.425820837727519	-3.91800429620164\\
-0.50005908139348	-3.90742333481465\\
-0.582026785292914	-3.91980997274883\\
-0.673295709214212	-3.97140118546093\\
-0.765727694962262	-4.03797968022181\\
-0.845870199914217	-4.09073811364462\\
-0.917244150143739	-4.13213392928272\\
-0.984936377269202	-4.16957969092242\\
-1.0625238056382	-4.22686872786252\\
-1.13896463071816	-4.27460942389842\\
-1.20485447524356	-4.28952979867895\\
-1.27590894479029	-4.32069553387334\\
-1.35749053895783	-4.37735348574078\\
-1.43935850648161	-4.43481129970657\\
-1.51428317257894	-4.4736331031717\\
-1.58768562764618	-4.49704238560704\\
-1.66785435048097	-4.54016145848911\\
-1.75176347988103	-4.60081401440788\\
-1.83159116602335	-4.64583062624153\\
-1.9045866559044	-4.6750242020562\\
-1.98256552863752	-4.72201115806595\\
-2.06312254430151	-4.77383131436426\\
-2.14035530006157	-4.81399466674278\\
-2.21743399271053	-4.84992138376291\\
-2.29927666193468	-4.89800964733423\\
-2.38287253064459	-4.95763796143456\\
-2.46189769064411	-5.00048057166149\\
-2.53594847369226	-5.0311396774881\\
-2.61341120921778	-5.07963498361569\\
-2.6951451928397	-5.1428093119007\\
-2.77094685548073	-5.18679807434332\\
-2.84330783717883	-5.21620271986114\\
-2.91693127946234	-5.25418108670358\\
-2.99269066172149	-5.30206607272485\\
-3.06657274286209	-5.34538634141543\\
-3.14075215770739	-5.38451662537044\\
-3.22098261937266	-5.42848590336976\\
-3.3077478555927	-5.47750895310197\\
-3.39043975011253	-5.51957627586634\\
-3.46354617860691	-5.55668276621793\\
-3.53430133590371	-5.59689986475411\\
-3.60985456390952	-5.64817258703916\\
-3.68633044538533	-5.69639507893407\\
-3.75626067729911	-5.72628951419716\\
-3.82851195044375	-5.76498896833781\\
-3.90598058842814	-5.81473178415334\\
-3.98069307386386	-5.85589091250711\\
-4.04918233017264	-5.88747189856604\\
-4.11857950024115	-5.92472368095922\\
-4.1961171352375	-5.98568929211921\\
-4.28027698710904	-6.04405367945736\\
-4.36085564891461	-6.08607112751329\\
-4.43368287816152	-6.11314598220131\\
-4.5087558169902	-6.14606054552743\\
-4.59060166414667	-6.18027112467571\\
-4.67094985043684	-6.18852139857466\\
-4.74024729584146	-6.18099418133885\\
-4.80999274746513	-6.18562651405564\\
-4.88484636699315	-6.2063431131424\\
-4.95610639489826	-6.21504727920933\\
-5.02439726419853	-6.19438945298082\\
-5.09035291232885	-6.15527178597576\\
-5.15660531886301	-6.12824084490823\\
-5.22677678677918	-6.11034421128705\\
-5.30080995694233	-6.1091833867341\\
-5.37247834230368	-6.12269005860841\\
-5.43567950280778	-6.13032993731887\\
-5.48357860349868	-6.12305633216192\\
-5.5220595767202	-6.11772181867395\\
-5.5489263432516	-6.11794841909996\\
-5.56642124937382	-6.1255103457041\\
-5.57630641977237	-6.13317277099888\\
-5.58084200134722	-6.13449439885867\\
-5.58701434787717	-6.13644150952297\\
-5.59562681796157	-6.13926973397834\\
-5.60450566398778	-6.1362593539584\\
-5.616564130872	-6.13543239168018\\
-5.63412727542072	-6.13772530509869\\
-5.66013199687857	-6.14295805806712\\
-5.6927750224768	-6.14845732513021\\
-5.73034092356496	-6.15528113127911\\
-5.77493130298419	-6.1645122160083\\
-5.82828587174613	-6.17368705159488\\
-5.89321174615585	-6.17597318110428\\
-5.96288705941894	-6.17168993093564\\
-6.03457786316301	-6.16329677126677\\
-6.10680490531035	-6.15956446819145\\
-6.17933937382391	-6.16019721151386\\
-6.25916667519366	-6.1715167022402\\
-6.35022353757521	-6.18814596473487\\
-6.45140913328946	-6.20162049295047\\
-6.55782014139306	-6.20329418569077\\
-6.66297446574848	-6.18999322528673\\
-6.76724574817247	-6.1798116195008\\
-6.87243472012363	-6.17812206897279\\
-6.98404418050705	-6.19627989171502\\
-7.1029141634458	-6.21990871444626\\
-7.22926972268671	-6.22488675319933\\
-7.35563532544154	-6.20956252627226\\
-7.47542434697158	-6.19991409838758\\
-7.59803432442617	-6.20628161480438\\
-7.7322654698376	-6.22716414835139\\
-7.87420807181372	-6.23604529214956\\
-8.01106562071879	-6.21572218589601\\
-8.1332109882098	-6.19744047009585\\
-8.24883126083344	-6.19318809409883\\
-8.36622716290041	-6.20788696018718\\
-8.4821245466104	-6.22302142177601\\
-8.58525122465495	-6.22048441507903\\
-8.6600273144359	-6.2032852525575\\
-8.71102830147822	-6.19974483152896\\
-8.75237787232564	-6.2133584256945\\
-8.78615060817968	-6.23074913121558\\
-8.81418322639172	-6.23779621492809\\
-8.83276172289612	-6.22335711924535\\
-8.84451391418895	-6.20722112740216\\
-8.8503104507863	-6.19828178352845\\
-8.8539734132862	-6.20085783556972\\
-8.85845566952205	-6.21301717092228\\
-8.86618419071486	-6.21826363103194\\
-8.86865547005246	-6.20550285582424\\
-8.85957523294128	-6.19691446042173\\
-8.84855998987124	-6.19876371863018\\
-8.83945624190526	-6.20211914808245\\
-8.83462782543384	-6.2046877603879\\
-8.83246663657844	-6.20400987107218\\
-8.83389614537642	-6.20383629181131\\
-8.83435968728035	-6.20511708919514\\
-8.83058178416904	-6.20790096675342\\
-8.82481504488625	-6.20669160887512\\
-8.81912255482099	-6.20931114995857\\
-8.81682661229087	-6.21356364236022\\
-8.81616257390354	-6.21799156228816\\
-8.81396938244828	-6.22195635398084\\
-8.8105355772205	-6.22667281567035\\
-8.80670560110489	-6.23220081099395\\
-8.80251987068285	-6.23926774736184\\
-8.79897008168173	-6.24275111188338\\
-8.79886321374722	-6.24180679994421\\
-8.80067235266874	-6.24020494842107\\
-8.7994694262665	-6.24111493909071\\
-8.80064183554717	-6.24240401968596\\
-8.81242545482969	-6.24217127185535\\
-8.83114204835248	-6.24280022200331\\
-8.84974368347318	-6.23780621989278\\
-8.85710513716979	-6.23123694175817\\
-8.85619018895283	-6.22419364389701\\
-8.84578384263171	-6.22317005781768\\
-8.82561393943962	-6.2297614205001\\
-8.79629033957263	-6.24371520672682\\
-8.75947306930422	-6.26679261041857\\
-8.72291203744642	-6.29041234669473\\
-8.68884338751824	-6.30164065093875\\
-8.65491719255915	-6.30286757819888\\
-8.622234320593	-6.28349611143969\\
-8.58109354830966	-6.27453827430164\\
-8.53073488733536	-6.28485782824301\\
-8.47835358313554	-6.3042606809153\\
-8.41917801725684	-6.31241840409696\\
-8.3407376665626	-6.29218798275586\\
-8.25332213981116	-6.27754310337051\\
-8.16094344283882	-6.28318173494298\\
-8.05979384831022	-6.30639285175951\\
-7.94681712715174	-6.32384044369852\\
-7.82372716434642	-6.31864904772837\\
-7.69393283364848	-6.29824558550708\\
-7.56302892081618	-6.29784333883087\\
-7.42747854806796	-6.32624934022725\\
-7.28521509406147	-6.35239528806642\\
-7.13980697808998	-6.35063582323336\\
-6.99784362798526	-6.331398709525\\
-6.85324449760732	-6.33296736467821\\
-6.69973020025387	-6.3586559603598\\
-6.5420596446925	-6.37798918196216\\
-6.39316335221671	-6.37024702693804\\
-6.2537097226999	-6.34568062495194\\
-6.10671052441565	-6.34085596773022\\
-5.94983681142981	-6.36564791924776\\
-5.78562333671847	-6.38402977935512\\
-5.61915343120253	-6.3766399828273\\
-5.45764964190406	-6.35720482154131\\
-5.29640952538578	-6.3568202474672\\
-5.1309689703634	-6.38143658800581\\
-4.96695389617408	-6.40052216784481\\
-4.80952771113291	-6.39688860364478\\
-4.65948883559038	-6.37439679010239\\
-4.50720415149856	-6.37294206167048\\
-4.34771660505222	-6.39695638199859\\
-4.18314565785927	-6.41927778347686\\
-4.0195784094275	-6.41640895518876\\
-3.86012235002184	-6.4025300456913\\
-3.70211427840442	-6.40245163903047\\
-3.54148667885187	-6.42444972990954\\
-3.38291830668256	-6.44414572046901\\
-3.22988393674033	-6.44261358934135\\
-3.08331305904799	-6.41966091075209\\
-2.94025188954913	-6.40550333106932\\
-2.79396048309953	-6.4230934553602\\
-2.64437627943594	-6.44323377907122\\
-2.49464138645584	-6.44185991533519\\
-2.34913807846441	-6.42737021152439\\
-2.20848915480708	-6.42239854882578\\
-2.06868192687087	-6.44119183731677\\
-1.92716249563784	-6.46516881042952\\
-1.79277198655746	-6.4705187236114\\
-1.67098370883594	-6.45033240930901\\
-1.5609424895943	-6.43045088704829\\
-1.45308380176287	-6.43569323460621\\
-1.34708943534804	-6.45334363058344\\
-1.24568600040163	-6.46461863974718\\
-1.15486989960526	-6.46324185102698\\
-1.08153159794358	-6.4536378291666\\
-1.02496277387889	-6.44891764133847\\
-0.982288635620616	-6.45042213367976\\
-0.95054573753657	-6.45633151426309\\
-0.927361413781518	-6.45829784441181\\
-0.909926757218453	-6.45895549016307\\
-0.897715646496514	-6.46330058744767\\
-0.891159811804907	-6.46776919040646\\
-0.887565071455638	-6.47242801350501\\
-0.886259449343038	-6.47615839101775\\
-0.887255898416502	-6.47902191315053\\
-0.893103876673034	-6.481630338301\\
-0.904850486409207	-6.4840757298564\\
-0.918831260502541	-6.4896434868759\\
-0.932586117707136	-6.49277177393198\\
-0.946448003186815	-6.49649285444053\\
-0.959824881623136	-6.50401338984084\\
-0.976738363827578	-6.51345030134743\\
-0.998581671362277	-6.52373458722696\\
-1.02447937410938	-6.5333800544747\\
-1.05325180196843	-6.54362573321259\\
-1.08296810599332	-6.55310527536738\\
-1.10879217048331	-6.56129712195069\\
-1.12755068441977	-6.56524194996597\\
-1.13462144352114	-6.57308416921414\\
-1.12464310183424	-6.58024833551175\\
-1.10322263071909	-6.58279498147471\\
-1.07777863061625	-6.58395313599564\\
-1.0432570552999	-6.58829714849671\\
-0.996477683067567	-6.59468535419673\\
-0.936718533178443	-6.60345525622669\\
-0.865605220205032	-6.60867765434558\\
-0.783281440277546	-6.6061723166738\\
-0.691989491040596	-6.5972225715438\\
-0.591503556797229	-6.59280960665063\\
-0.481810639187117	-6.59656047698822\\
-0.365049192973837	-6.61117280501616\\
-0.240241045365863	-6.62596272291428\\
-0.111694031828473	-6.62046781018783\\
0.0119787367811162	-6.58532177963033\\
0.132513233256398	-6.55683321911567\\
0.258627949833683	-6.551591191004\\
0.393647438329005	-6.56962753227447\\
0.534951562206749	-6.58503502004409\\
0.679152745402083	-6.57623932305963\\
0.815859465679525	-6.55880458924112\\
0.944781110741804	-6.54853410559025\\
1.07179085940142	-6.56424299329704\\
1.19998034620558	-6.58294357571075\\
1.32869292146437	-6.58881450171176\\
1.45089350953225	-6.5744296944715\\
1.56828798887826	-6.56121612925516\\
1.684611557474	-6.56544058952681\\
1.80034436276753	-6.585591155425\\
1.91615718718557	-6.6023075130426\\
2.02298507218237	-6.59114931658111\\
2.11382738826751	-6.58238944708801\\
2.18732729287296	-6.58031105067393\\
2.24863375411507	-6.59796784026763\\
2.30205148771084	-6.6210481063283\\
2.3523546425386	-6.62466215482843\\
2.39766505931082	-6.60107014483611\\
2.43645603598365	-6.58558115822658\\
2.47306319574464	-6.58295543791545\\
2.50629641209093	-6.59016746912673\\
2.53438290875263	-6.59417416312215\\
2.55193518342614	-6.57897694985869\\
2.5542284569164	-6.55384103574793\\
2.5438346725487	-6.52978525819772\\
2.51999074782251	-6.51410591250375\\
2.48691330913411	-6.49792479093298\\
2.45621018564966	-6.47251135137709\\
2.43268877004295	-6.43336183857711\\
2.41553007305285	-6.39537350823128\\
2.4036380324347	-6.37067324539313\\
2.39484552958749	-6.36237912502452\\
2.38897639516698	-6.36373140020945\\
2.38484235337584	-6.36445991733082\\
2.37388007936213	-6.36507462787621\\
2.35904656297239	-6.36765779676513\\
2.35052024627628	-6.37296490204892\\
2.34491927173944	-6.37751071215204\\
2.33758370848062	-6.3803863702877\\
2.32973939408246	-6.38104790311697\\
2.32405340356294	-6.38220266882632\\
2.32201786729724	-6.38465469221065\\
2.31957298113838	-6.38725060802036\\
2.31803375254584	-6.390063193576\\
2.3158323926942	-6.39223083920801\\
2.31380247087885	-6.39507553183032\\
2.31160560773658	-6.39726161062224\\
2.30966925302061	-6.39929450587153\\
2.30843868506597	-6.4012333750655\\
2.30716251808946	-6.40291265798821\\
2.30609358261096	-6.40419413888023\\
2.30532349092036	-6.40489970267161\\
2.30477541621139	-6.40536419770359\\
2.30475996977738	-6.40553802754815\\
2.30481667308277	-6.4054955803878\\
2.30490062914633	-6.4055119870899\\
2.30504540511044	-6.40555724848839\\
2.30518295676807	-6.40567383123208\\
2.30531888798074	-6.40571511962538\\
2.30544539500152	-6.40582806272163\\
2.30556925562137	-6.40594092789907\\
2.30569069897368	-6.40606315102917\\
2.30580563772168	-6.40619954014787\\
2.3059017429113	-6.40630927062118\\
2.30598265315409	-6.40641033588074\\
2.30608090687425	-6.40648145317026\\
2.30620485102191	-6.40650571764724\\
2.30627178186576	-6.40648614679132\\
2.30630430330527	-6.40644154632305\\
2.30631244214342	-6.40639059198114\\
2.30628805633682	-6.40638873324039\\
2.30624313714592	-6.40637254867339\\
2.30623440568789	-6.40638393009777\\
2.306249635228	-6.40639745676128\\
2.30625829986466	-6.40641288714372\\
2.30625052581852	-6.40640927511429\\
2.30625656323652	-6.40643335257723\\
2.30625706124307	-6.40643881132573\\
2.30623798258196	-6.40642945978137\\
2.30619970595558	-6.40642634110856\\
2.306205073737	-6.40642699943907\\
2.30623920004758	-6.40645275629429\\
2.30624315543478	-6.40646856085927\\
2.30627446296121	-6.40644266135972\\
2.30632547871254	-6.40643084620528\\
2.30634390535767	-6.4064207119317\\
2.3063351545642	-6.40639486344315\\
2.30635205594127	-6.40637597871934\\
2.30635305846247	-6.40641614944637\\
2.30637032045127	-6.40645852063875\\
2.30637402134284	-6.40643616092129\\
2.30639208385908	-6.40633421720182\\
2.30655046919523	-6.40617994104397\\
2.30685949276776	-6.40597844215921\\
2.30635497078314	-6.40648880527636\\
2.30574811752826	-6.40693471645209\\
2.30623630421456	-6.40515537342274\\
2.31048115616632	-6.40239612647725\\
2.31595573735614	-6.40714534734529\\
2.32292930958773	-6.42946749648908\\
2.33550834236584	-6.48132136787569\\
2.37008833838863	-6.5627648217406\\
2.42254909979971	-6.68845630429379\\
2.48011912859347	-6.85282625597706\\
2.5343015071163	-7.03370256196762\\
2.57793551067738	-7.20412535578974\\
2.60526722710078	-7.35513261344248\\
2.61603717123514	-7.48695898656828\\
2.61900971676108	-7.59855015277731\\
2.62467593206569	-7.67647421779583\\
2.62896952218781	-7.71733509730389\\
2.633359415088	-7.73723127594091\\
2.63337680111935	-7.75128752949312\\
2.63460675375679	-7.76124550126326\\
2.63436219199417	-7.76187404420901\\
2.63428014003802	-7.76228126561164\\
2.63415483218182	-7.76251626525009\\
2.63337483200226	-7.76248754333455\\
2.63313311159795	-7.76208866509205\\
2.63321037380499	-7.76224256826098\\
2.63311621247011	-7.76229256968205\\
2.63335079811172	-7.76281769416749\\
2.63355055554903	-7.76271897247055\\
2.63392206417184	-7.76266277864469\\
2.63383096229967	-7.7623809633804\\
2.63382285894476	-7.76221507143524\\
2.63381640862526	-7.76217172861131\\
2.63379437309254	-7.76218078810476\\
2.6337798364499	-7.76217127966463\\
2.63377536136563	-7.76215692034237\\
2.63375437866445	-7.76214491749734\\
2.63370718066454	-7.7621429794665\\
2.63362622537923	-7.76214605678009\\
2.6335520946765	-7.76212476231236\\
2.63349977041962	-7.76207404598721\\
2.63348016992186	-7.76207114142381\\
2.63347677465023	-7.76205244690096\\
2.63348001597023	-7.7620566224116\\
2.63348432760925	-7.76207003772378\\
2.63351310446533	-7.76208790208937\\
2.63351553793714	-7.76211198170497\\
2.63351585143909	-7.76213897535759\\
2.63353955350975	-7.76218114011034\\
2.63356672489932	-7.76225368585457\\
2.63358075790345	-7.76228576403453\\
2.63359153247161	-7.76229689278784\\
2.63364288136547	-7.76228701687617\\
2.63369999300878	-7.76225963518379\\
2.63375335671481	-7.76224323408933\\
2.63380535492291	-7.76224234753706\\
2.6337806178096	-7.76223698060259\\
2.63374567271753	-7.7622361013019\\
2.63373579497779	-7.76223035696564\\
2.63373944070439	-7.76227409820248\\
2.63372488796685	-7.76229988852016\\
2.63372070634195	-7.76230330971996\\
2.63371667647007	-7.76231973154939\\
2.63371172878963	-7.76235100200139\\
2.63357018582982	-7.7623907467583\\
2.63309956867089	-7.76253176351756\\
2.63228945542664	-7.76272447509832\\
2.63114391285563	-7.7629573149205\\
2.62933856406969	-7.76323534997005\\
2.62652600002372	-7.76350914615815\\
2.62322117404812	-7.76310297348091\\
2.61944959841897	-7.762057600073\\
2.61494566560516	-7.75450556022524\\
2.60854503031838	-7.71930789626372\\
2.60287743393066	-7.64350038296948\\
2.59153046367599	-7.53470818388671\\
2.57887313910084	-7.40115737566735\\
2.55766990834678	-7.25070458006259\\
2.53233840115393	-7.10131951305579\\
2.51128400855459	-6.96090888600581\\
2.49759593315492	-6.83899587761062\\
2.4877023324719	-6.74803131898516\\
2.48703711448999	-6.68599763713117\\
2.49705055380226	-6.65040984179727\\
2.50726194052094	-6.6324413600243\\
2.52039907966403	-6.62508754506468\\
2.54127090307706	-6.62613215095979\\
2.57518261284961	-6.63059378008276\\
2.62230126418319	-6.63757103699262\\
2.66915330746555	-6.63649371594078\\
2.70893036172093	-6.62221560215538\\
2.73549884846103	-6.61634923697399\\
2.75373433448793	-6.61766980794673\\
2.76966371610571	-6.62678196481496\\
2.78634788569459	-6.6382585873109\\
2.80394751506211	-6.63498789563239\\
2.81502404377914	-6.61912642285245\\
2.82258036594844	-6.60611315798207\\
2.82896326677288	-6.59954051357114\\
2.83360634574989	-6.60454777580166\\
2.83686121256338	-6.61949714318202\\
2.84231184324848	-6.62981817077759\\
2.84963799775048	-6.62714331416639\\
2.8611359190314	-6.61481715884874\\
2.88221963413063	-6.60818421948257\\
2.912936373495	-6.60760752425029\\
2.95296582546488	-6.61722751227416\\
2.99797877976896	-6.63804319933576\\
3.03836910650153	-6.66140912489294\\
3.0663570806493	-6.68304951772375\\
3.07846654421367	-6.6951664541961\\
3.07839852673866	-6.72000124107082\\
3.07243285803896	-6.75920583306912\\
3.06707575993898	-6.81299847639241\\
3.06776653426796	-6.88423869386584\\
3.06660604193937	-6.97342451141171\\
3.06285404193669	-7.07731358062589\\
3.06218780096931	-7.18613843589451\\
3.06963329948369	-7.30445108888167\\
3.08611070053451	-7.42653177567177\\
3.11028606788847	-7.55134653356531\\
3.13715008649753	-7.65832567756978\\
3.1527374791321	-7.72941982600405\\
3.15117776154412	-7.75980613034096\\
3.148249327395	-7.77612557075872\\
3.14542944155098	-7.78512955211119\\
3.14235996905832	-7.78859384313347\\
3.14044548550368	-7.79137666457704\\
3.13695937185811	-7.79460659794507\\
3.13424186073934	-7.79698548544055\\
3.13164341911327	-7.79944952205942\\
3.12953318196558	-7.80166383129176\\
3.12788457151025	-7.80324655965547\\
3.12643072665412	-7.80317255608752\\
3.12464918139433	-7.80214298168259\\
3.12294241916801	-7.80111929862471\\
3.12132152109321	-7.80012922636644\\
3.11979310587587	-7.79915050025219\\
3.11835696668205	-7.79825911866253\\
3.11695967679522	-7.79745535314282\\
3.11561946014786	-7.79673760311375\\
3.11440764866216	-7.79604524953889\\
3.11323421618003	-7.79542773017918\\
3.11219159775934	-7.79486765044379\\
3.1112332012928	-7.79438025894183\\
3.11037358056068	-7.79397601701895\\
3.10963015661668	-7.79362508649782\\
3.10901019499665	-7.79334268306086\\
3.10846640117448	-7.79311867623619\\
3.10805174005294	-7.79291853349232\\
3.10778533569326	-7.79274230915478\\
3.10760692792253	-7.79257375153889\\
3.10759265187161	-7.79250612507598\\
3.10761433362297	-7.79249950682406\\
3.10762809853553	-7.79252309030105\\
3.10759028081413	-7.79255520288268\\
3.10750559452458	-7.79259393397032\\
3.10737970036186	-7.79267274226542\\
3.10723764352586	-7.79278552635792\\
3.10707047113312	-7.79291503400112\\
3.10695482436433	-7.79299676588838\\
3.10680181949373	-7.79307910272846\\
3.10664916024908	-7.79318382744872\\
3.10653855752583	-7.79329833602222\\
3.10645695463076	-7.7934077519865\\
3.10633478256048	-7.79359079988659\\
3.1061915950788	-7.79380041869701\\
3.10608488584012	-7.79400172479582\\
3.10600116806658	-7.79416756017785\\
3.10591756523409	-7.7930021406765\\
3.10585632611704	-7.79105851083187\\
3.10584104056919	-7.7892058068231\\
3.10579152098446	-7.78674853810479\\
3.10577504793276	-7.78235756567752\\
3.10577893975794	-7.77450331681811\\
3.10582794151721	-7.76181378762313\\
3.10596326105584	-7.74341842918725\\
3.106123163062	-7.71936853756262\\
3.10631627336875	-7.69006463956798\\
3.10655715251311	-7.655624013267\\
3.10682915935746	-7.61586675843585\\
3.10712731656347	-7.57070955172128\\
3.10745073569035	-7.52028353181105\\
3.10786013094075	-7.46480469210941\\
3.10826257994289	-7.40430278190748\\
3.1086346218125	-7.33862246327755\\
3.1089851633751	-7.26788162544592\\
3.10941907315318	-7.19191004069415\\
3.10988733124955	-7.11105202598514\\
3.11046210401718	-7.02527235018399\\
3.11095241409954	-6.93502139351145\\
3.11156780584795	-6.84149654464726\\
3.11213635627858	-6.74588946741088\\
3.11282171377689	-6.64801426271645\\
3.11348430911613	-6.54857970818734\\
3.11424864677405	-6.44814088293154\\
3.11490278321632	-6.34770315755768\\
3.11575888906243	-6.24781716089319\\
3.11656965281446	-6.14820268161292\\
3.11746941688148	-6.04812204316035\\
3.11835602211227	-5.94792253107609\\
3.11926091109529	-5.84779344515398\\
3.12022965801782	-5.74781320653868\\
3.12118370857529	-5.64786728026146\\
3.12222775735728	-5.54787044002352\\
3.12329809203118	-5.44789213878621\\
3.12434652269951	-5.34774520791411\\
3.12550550805933	-5.24759853365676\\
3.12664010233448	-5.14751224287033\\
3.12799206302647	-5.04719700607722\\
3.12926561584421	-4.94712314906211\\
3.13080443182172	-4.84680429580261\\
3.13228400810676	-4.74656543023295\\
3.13377063720124	-4.64644706903184\\
3.13544861957923	-4.54655494868999\\
3.13703559235687	-4.44632834161522\\
3.13875403591166	-4.34611101009902\\
3.14047587638373	-4.24619432698521\\
3.14223906180094	-4.14638591609562\\
3.14424139741825	-4.04644643778891\\
3.14621629166724	-3.94654398898182\\
3.14849959687238	-3.84651208327151\\
3.1507094242118	-3.74651613650488\\
3.15320839641731	-3.64638866910226\\
3.15560634143339	-3.54634563372591\\
3.15826393343493	-3.44634925414657\\
3.16131533228407	-3.34592951088475\\
3.16438333558051	-3.24570566737631\\
3.1675054055382	-3.14587243643874\\
3.17075513779133	-3.04589076149217\\
3.17413854144242	-2.94561795176429\\
3.17764812167889	-2.84547266231148\\
3.18133586461608	-2.74543342241189\\
3.18522341794227	-2.64535837954884\\
3.18919897061318	-2.54521711493391\\
3.19337275611884	-2.44503218648348\\
3.19764288488947	-2.34494883909119\\
3.20205051918809	-2.24492768442467\\
3.20655187700012	-2.14501176815774\\
3.21120705496578	-2.04503987229434\\
3.21595124528611	-1.94502767901208\\
3.2209501942972	-1.8449635725565\\
3.22594573659565	-1.74486369836046\\
3.23105995189191	-1.6451276339648\\
3.23635858237672	-1.54572806083464\\
3.24170519438184	-1.44558676255589\\
3.24724529933414	-1.34530450430942\\
3.25282741582826	-1.24610978363405\\
3.25871490685884	-1.14861042557701\\
3.26454672001561	-1.05304653533366\\
3.27071307247813	-0.960264234510373\\
3.27700525576525	-0.870601996883784\\
3.28351049477379	-0.785240888561452\\
3.29002330219908	-0.705040835037402\\
3.29679719909514	-0.629692830358838\\
3.30358683918061	-0.558877198975291\\
3.31069539927484	-0.492120402226079\\
3.31785466815332	-0.429835703585238\\
3.32527038912862	-0.372096810015573\\
3.33282263932739	-0.318822115546452\\
3.34056090083997	-0.269801923169674\\
3.34843887237236	-0.225073377049118\\
3.35654639211137	-0.184768430587287\\
3.36478114325832	-0.149001786089451\\
3.37325879146749	-0.117457875777957\\
3.3819058086518	-0.090159688554851\\
3.3907164877735	-0.0670665272884576\\
3.39975395449412	-0.0475487199234561\\
3.40899056674281	-0.0309221998934746\\
3.41841323645288	-0.016721356977552\\
3.42806632377595	-0.00465488022073107\\
3.43786664135402	0.00551208401045622\\
3.44792543133281	0.0140096302229377\\
3.45817089503965	0.0212694468179854\\
3.46860270901156	0.0273882723789331\\
3.47920551917715	0.0327647534230427\\
3.4900780315619	0.0373486526566308\\
3.50143120383946	0.0412254707271874\\
3.51282759489348	0.044765367137264\\
3.5244917188681	0.0478880376180619\\
3.53643785209056	0.0504175257881874\\
3.54734609399722	0.052474774611761\\
3.55990760816168	0.0546820363545192\\
3.57241867035562	0.0578513582967936\\
3.58497299591904	0.0745149166685339\\
3.59982876153436	0.130449698158484\\
3.61111467883517	0.229308943870724\\
3.61404248804616	0.365300686243002\\
3.60499441932765	0.52740475228315\\
3.57804277820087	0.693435354609058\\
3.5350165848012	0.838075718260123\\
3.48882279650977	0.948925411597708\\
3.44424878515476	1.03537624515083\\
3.39908061117308	1.094853507064\\
3.34930648549423	1.11899287474077\\
3.30157706043882	1.1213528400136\\
3.25708925194522	1.11191270944835\\
3.21444390040668	1.11202344881102\\
3.17791816811813	1.13859489437628\\
3.1528701441441	1.15558770080051\\
3.13685272941741	1.15722619325175\\
3.12188260979853	1.14140390321804\\
3.10887803403411	1.12193331470275\\
3.09591424969775	1.12232235815729\\
3.08544880841823	1.14160861413308\\
3.07673593103038	1.16704605629633\\
3.07355320212911	1.17641872439494\\
3.07177498391821	1.16426897400863\\
3.06500474337901	1.15103643633583\\
3.04973234434185	1.16560506849768\\
3.03148569525511	1.18177877274056\\
3.0034575915182	1.18690502532035\\
2.97387629628176	1.16988261887779\\
2.94645893680193	1.14201049416112\\
2.90588459826138	1.13248387080741\\
2.84050869146378	1.15305417129544\\
2.75880962855688	1.17051447785032\\
2.67273926900081	1.16204334440794\\
2.57752785882914	1.13191600014048\\
2.46937329418704	1.10359854499873\\
2.35156810270701	1.10131997824874\\
2.23268315824058	1.11209393661553\\
2.11438612232633	1.11814186406826\\
1.9946556029172	1.10479451903305\\
1.8740944540107	1.08331757094385\\
1.75600098845019	1.07355754691453\\
1.63981903081364	1.08487340987801\\
1.52676570210999	1.08862338894962\\
1.40809441408235	1.07626280370794\\
1.28063416438191	1.05164248754871\\
1.14484022202424	1.02654284578014\\
1.00407150789797	1.01623736421078\\
0.867182591884283	1.01309691037131\\
0.737070110200605	0.995462448082423\\
0.610997887060226	0.963754318164762\\
0.487203964664022	0.921446134629917\\
0.374534076632173	0.865983801331611\\
0.285618421744722	0.79887222773418\\
0.212800049418938	0.727296609068578\\
0.161407901471648	0.647211287733589\\
0.128042175337434	0.553357713577955\\
0.106511442166131	0.445274060879358\\
0.0847363168650261	0.337087612945474\\
0.0627934082500232	0.233526561837186\\
0.0444871082516443	0.150375449253399\\
0.0293066696216542	0.0877782254913504\\
0.0147192336607418	0.0461920059073693\\
0.00189791618724655	0.0278590931761167\\
0.00609582213058957	0.0178993436413346\\
0.00221746161611946	0.0134145959056208\\
0.00256342399280385	0.0126647246005205\\
0.00286824566359201	0.0106150394746354\\
0.0034664048741746	0.0104123935309728\\
0.00358813605207331	0.0101469735214937\\
0.00447069456399163	0.0095648145863525\\
0.00408817052888111	0.00902193924743167\\
0.00386550830876331	0.00910589165504938\\
0.00376249174515664	0.00919991601646197\\
0.00378834449048693	0.00926799253458971\\
0.00378466266286814	0.00928032312672133\\
0.00377121333541552	0.00928749442750293\\
0.00376989187949039	0.00927483605231721\\
0.00377258611348452	0.00926694984128684\\
0.00377226326350133	0.00927159557446572\\
0.0037657755164745	0.00928800222608196\\
0.00377105425235143	0.00930477757349212\\
0.00376722732943953	0.00931136308390568\\
0.00379663799006157	0.00932609368193222\\
0.00380922024191745	0.00934199767318129\\
0.00382829242889658	0.00933754028375312\\
0.00384376874955631	0.00928051328188293\\
0.00386332848780638	0.00918034979021219\\
0.00387347047241043	0.00909999795519435\\
0.0038462100298774	0.00906885437044007\\
0.00384308986882154	0.00907519956297182\\
0.00384176262401537	0.0090867792228747\\
0.00377171525436146	0.0091052314956273\\
0.00367889878213171	0.00907831483694186\\
0.00360742165035133	0.00900866856026147\\
0.00360286630332128	0.00894192041530623\\
0.00360106386900126	0.00888380523874156\\
0.00358516115374505	0.0088653130830109\\
0.00358514787177094	0.00881619380147063\\
0.00358250496100393	0.00876975533528987\\
0.00353911505284144	0.00877458481823624\\
0.00356107348930866	0.00879198697615288\\
0.00360541698465978	0.00879134751091057\\
0.00362362915512757	0.00879581184361516\\
0.00362613522308535	0.00878781183535956\\
0.00364195648865525	0.00879948376819013\\
0.00364558641497026	0.00879161609240233\\
0.0036296317872544	0.00878008859944517\\
0.00363584801783387	0.00876428578217427\\
0.00363456466116585	0.00875055799443935\\
0.00363409003891716	0.00874126391215706\\
0.00362918055161883	0.00873835769307243\\
0.00492859237114	0.00885145751531385\\
0.00633410552007736	0.00954754236638348\\
0.00866414620468599	0.0118789303947033\\
0.00976440901200113	0.0180447869042716\\
0.0170855143140172	0.0299288886838246\\
0.0318374229662901	0.0618054160021131\\
0.0512538241733429	0.113199663977901\\
0.0672521083486014	0.184577612627635\\
0.0737699227050572	0.272384296691588\\
0.0622909864289257	0.368095245863352\\
0.0466260763975261	0.461560238932614\\
0.0356144250269571	0.53768738561193\\
0.0281691899866261	0.588267020673505\\
0.0200187384376402	0.622210374325296\\
0.0118676898060204	0.644817520646712\\
0.00441360500632178	0.656235220401568\\
-0.00364972116560786	0.658194348221601\\
-0.00763602850819002	0.658584240543558\\
-0.00895091112041009	0.659499264744593\\
-0.0121845151786494	0.659453527073871\\
-0.0143286175410882	0.65893688002024\\
-0.0136822085657117	0.65830036426713\\
-0.0124051480142159	0.657925118300045\\
-0.0114590758497998	0.657591996640104\\
-0.0102374018695203	0.65723648709931\\
-0.00926512713997064	0.656911710019075\\
-0.00848910399415366	0.65665338695514\\
-0.00801369865797538	0.656490072692028\\
-0.00777835251294299	0.656401827912159\\
-0.00776387150697708	0.656385192428316\\
-0.00781208542266233	0.656377483579327\\
-0.00784509257939955	0.656413850729339\\
-0.00781803958838477	0.656413787928382\\
-0.0077873615206044	0.656408558621253\\
-0.00778355771382285	0.656450496977521\\
-0.00777236358518042	0.656443596370603\\
-0.0077901926852176	0.656413940618155\\
-0.0078294148905643	0.656376976687954\\
-0.00788500034642315	0.656343762021843\\
-0.0079019218932957	0.656320562974688\\
-0.0078770530066069	0.656276855617608\\
-0.00786607971234572	0.656245652948178\\
-0.00786900560986845	0.656230465199772\\
-0.00784360010301694	0.656185872175078\\
-0.0078387536048268	0.656088701908765\\
-0.00784446995797044	0.655990028861715\\
-0.00787522957858295	0.656004221583918\\
-0.00789394067615815	0.656038486476204\\
-0.0079550997786948	0.656049922844985\\
-0.00800502003454829	0.656042984073239\\
-0.00804549118078796	0.656081844243044\\
-0.00812961560453583	0.656106409512113\\
-0.00823776109400001	0.656130228569778\\
-0.00834271507625902	0.656149337110941\\
-0.00842756635368357	0.656151918955882\\
-0.00851275757106166	0.656159297204801\\
-0.00854967651413449	0.656153518163938\\
-0.00856020280406373	0.656145819981481\\
-0.00858551461266561	0.656150447142177\\
-0.00860031400662995	0.65616998078014\\
-0.00860336769712694	0.656154800287106\\
-0.00863021204733192	0.656136557964554\\
-0.00863763553653979	0.656156235837375\\
-0.00873904598170129	0.656374545058324\\
-0.00883124157529506	0.656825444518136\\
-0.00870343924661577	0.657453534993826\\
-0.00673243565667594	0.659121452354921\\
-0.000936670301659852	0.663021634084022\\
0.0184618840749325	0.671938043954053\\
0.0388735907475223	0.690187606263907\\
0.0567026059373696	0.720751944628469\\
0.0694814307096801	0.758422939929258\\
0.0828901420630656	0.80193599781841\\
0.0984281464900858	0.856537501430802\\
0.112992462833124	0.921400533768963\\
0.129076916072373	0.992161003142966\\
0.151778125433671	1.06564173855017\\
0.174006319951788	1.13308235931699\\
0.194158433667067	1.18817521910754\\
0.210294517032505	1.22666335669827\\
0.223962751938115	1.24831145165234\\
0.23327374812258	1.25797990122014\\
0.236170769354072	1.25865911632176\\
0.236153228886025	1.25593373025396\\
0.233868530430949	1.2559107501073\\
};
\addplot [color=black,dashed,forget plot]
  table[row sep=crcr]{%
-10	0\\
4	0\\
};
\node[right, align=left, text=black]
at (axis cs:-8,0.3) {Floor level 3};
\addplot [color=black,dashed,forget plot]
  table[row sep=crcr]{%
-10	-3.90190800343127\\
4	-3.90190800343127\\
};
\node[right, align=left, text=black]
at (axis cs:-8,-3.602) {Floor level 2};
\addplot [color=black,dashed,forget plot]
  table[row sep=crcr]{%
-10	-7.80381600686254\\
4	-7.80381600686254\\
};
\node[right, align=left, text=black]
at (axis cs:-8,-7.504) {Floor level 1};
\node[right, align=left, rotate=90, text=black]
at (axis cs:2.2,-3.6) {Elevator ride};
\node[right, align=left, rotate=40, text=black]
at (axis cs:-3,-1.3) {Stairs};
\end{axis}
\end{tikzpicture}
  %
  \vspace*{1em}
  {\footnotesize {(b)~Multi-floor example}}\\

  %
  \end{minipage}%
  %
  \tikzexternalenable
  %
  \vspace*{-1.4em}
  \caption{(a)~The altitude (vertical) profile, the velocities, and the orientations of the phone along the path in Figure~\ref{fig:intro}. The shading shows the stationarity detection outcome, where zero-velocity updates were triggered. The subtle periodicity in the path is due to walking, and the drop in altitude on the first floor is because of the phone being in a bag for parts of the path. (b)~A PDR example with first descending two levels down and then taking the elevator back up. The path was started at origin and the path ends with a loop-closure in the same place. The points where the phone touches the floor level (the sharp drops in vertical position) are zero-velocity updates. No absolute position info was given to the model.}
  \label{fig:states} 
  \vspace*{-.5em}
\end{figure*}


\subsection{Position Fixes and Loop-Closures}
\noindent
In terms of the sequential inference scheme all auxiliary observation data is combined with the model through the measurement model in Equation~\ref{eq:measurement}. Position fixes are noisy measurements of the position vectors $\vect{p}_k$ in the state (\ie\ $\vect{h}_\mathrm{pos.}(\vect{x}) = \vect{p}$). The additive Gaussian measurement noise represents the uncertainty associated with the given position.

Position fixes provide uncertain information of the position and thus also the distance travelled between the position fixes. This first-hand information helps the model pin down the bias estimates very accurately. Loop-closure points do not provide any exact position information, but indicate that at two different points in time, the positions are the same. Also this information is valuable in inferring sensor biases.

Manual loop-closures can be combined with the estimation scheme by augmenting the current position estimate in the state by a Kalman update at loop-opening. The state dimension grows by three at opening the loop, and the state becomes
\begin{equation}
  \vect{x} = (\vect{x}^\mathrm{old}, \vect{p}^\mathrm{LC}), 
\end{equation}
where the prior $\vect{p}^\mathrm{LC} \sim \N(\vect{0}, \vectb{\Sigma}_0^\mathrm{LC})$ with the $ \vectb{\Sigma}_0^\mathrm{LC}$ sufficiently large indicating the non-informativity of the initial location of the loop-closure point. In practice, both at $t_\mathrm{open}$ and $t_\mathrm{close}$ (the loop can be closed many times) the measurement model
\begin{equation}
  \vect{h}_\mathrm{LC}(\vect{x}) = \vect{p} - \vect{p}^\mathrm{LC}
\end{equation}
defines an observation $\vect{y}=\vect{0}$ with some measurement noise $\vectb{\gamma} \sim \N(\vect{0}, \vectb{\Sigma}^\mathrm{LC})$. The measurement noise covariance $\vectb{\Sigma}^\mathrm{LC}$ should reflect the mismatch of the user not exactly being at the loop-closure spot. Both the position fix and loop-closures are linear observations of the state, and can thus be implemented using a standard (linear) Kalman update.



\subsection{Zero-Velocity Updates}
\noindent
In this paper, the most important source of auxiliary information is so called zero-velocity updates (ZUPTs, see \cite{Nilsson+Zachariah+Skog+Handel:2013} for discussion). Once the phone is detected to be stationary for any period of time, it is known to the model that the system velocity must be zero ($\vect{v}=\vect{0}$). In terms of the measurement model, this means
\begin{equation}
  \vect{h}_\mathrm{ZUPT}(\vect{x}) = \vect{v}
\end{equation}
and the additive measurement noise $\vectb{\gamma}$ is small (by only specifying a pseudo-noise scale). This update can be performed as a standard (linear) Kalman update.

For triggering, we use an iterative Dickey--Fuller stationarity test \cite{Dickey+Fuller:1979} on a rolling window of accelerometer data (window size 250~ms) with an additional requirement of the sample standard deviation being small. This means that trends in the data are used as a proxy for movement.

\subsection{Pseudo-Measurement Updates}
\noindent
Without position fixes, loop-closures, or ZUPTs the inertial navigation system quickly becomes unstable. Once the estimates start diverging, they easily loose their numerical precision. The main source of these problems is gravity `leaking' into the acceleration input and corrupting the velocity vector. Once the velocity starts to drift, the position diverges almost instantly.
However, even without other auxiliary information, it is possible to keep the system informed about a reasonable scale of velocity. In our model, we present a simple yet powerful pseudo-update formulation that keeps the speed in the range of some meters per second and discourages the system from accelerating into higher velocities.

The pseudo-update model is defined in terms of the speed of the object, when it is not stationary. The speed (the Euclidean norm of the velocity) is
\begin{equation}
  h_\mathrm{pseudo}(\vect{x}) = \norm{\vect{v}}.
\end{equation}
In our experiments the pseudo-updates are parametrized as follows. The speed observation $y = 0.75$~m/s with a measurement noise $\gamma = \N(0,2^2)$. The large measurement noise variance keeps the update non-informative in comparison to other information sources.


\subsection{Barometer Readings}
\noindent
Barometric air pressure data (typically also available in high-end smartphones) can be mapped to heights through linearization of the barometric formula around sea level. Over short time periods the air pressure at a given altitude tends to stay constant. In this case the barometer readings relative to the starting point can provide absolute height updates (corresponding to position fixes as presented above).

The barometric pressure drifts over longer time horizons (in the order of tens of minutes), leading to accumulation of measurement errors. Another approach is to only use the relative pressure differences between two consecutive barometer observations mitigating drift issues. This alternative corresponds to opening an altitude loop-closure point on each barometer observation and closing them on the next.


\section{Experiments}
\label{sec:experiments}
\noindent
In the examples, the interest was put on Apple phones and tablets---mostly because of their uniform hardware and good software compatibility between devices. The device models used in the examples are the iPhone~6 and the iPad~Pro (12.9-inch model). Both these models are equipped with built-in IMUs (InvenSense MP67B) and a barometric sensor (Bosch Sensortec BMP280). In all experiments, the IMU sensor data and the associated timestamps were collected at 100~Hz, and the barometer data (when used) at approximately 0.75~Hz. The data was collected using an in-house developed data collection application, and the paths were reconstructed on the iPhone hardware off-line after the data acquisition.


\subsection{Pedestrian Dead-Reckoning}
\noindent
The most apparent use case for the presented model is to apply it to pedestrian dead-reckoning, where the mobile phone (iPhone~6) is carried by the user indoors. There exist a multitude of methods for dead-reckoning using data provided by mobile phones. Therefore the aim of this experiment is to show how this method differs from others by its generality.

Figures~\ref{fig:intro} and \ref{fig:states}(b) summarize features of the proposed INS system; the example includes the use of zero-velocity updates, position fixes, pseudo-measurements constraining the speed, and barometer observations. This experiment covers traditional navigation-like PDR use cases (walking with the phone in a fixed orientation), where SHS systems are often used, cross-floor tracking, where visual tracking methods are usually the method of choice, and bag/pocket use cases, which currently often require resorting to radio based positioning. The generality of our INS system can cover them all with only one method and no external hardware.

In the first example, the path was started on the ground floor with zero-velocity updates  (no pre-calibrations done). First the user walked up a flight of stairs to the first floor holding the phone in the hand. On the first floor a position fix was given, after which the phone was put in a bag. Next, the phone was taken out of the bag and put in the trouser pocket. Before descending to the ground floor, the phone was taken out of the pocket and a second uncertain position observation was given, which aligned the path to the map and was able to provide absolute information of the scale. On the ground floor a manual loop-closure was given to indicate that the phone had returned to the starting point, and the phone was placed on the floor for some final zero-velocity updates. The tracking path is accurate and follows the true path up to decimetres. 

Figure~\ref{fig:states}(a) shows the altitude profile of the path. The ZUPTs where the phone is placed on the floor are clearly showing, as well as the stair climbing. The drop in altitude on the first floor is due to the phone being in the bag for a part of the path. The figure also shows the estimated velocity. The periodicity is due to walking. This effect is less evident when the phone is in the bag, and at clearest when the phone is in the trouser pocket firmly attached to the body.

We briefly present a second PDR example which is shown in Figure~\ref{fig:states}(b). In this example the path was started at origin with zero-velocity updates in different phone orientations. After this the user walked two floors down. When waiting for the elevator on floor level~1, further ZUPTs were done. The path is completed with taking the elevator back to floor level~3 and closing the loop at the starting point. In this example no absolute position information was given. The scale comes entirely from the accelerometer data. In both examples, an backward smoother pass (see Sec.~\ref{sec:nonlin-estimation}) is run after every update, thus also correcting the past estimates.

These expeiments demostrate the unconventional nature of the proposed method; this odometry method delivers a combination of use cases, which cannot be delivered with other methods running on the same device. Visual methods fail in the bag/pocket, and SHS methods fail when the device orientation is not fixed or steps/motion cannot be observed.




\begin{figure}[!t]   
  \centering\footnotesize

  \tikzsetnextfilename{tikz-stroller-setup}%
  \begin{tikzpicture}
    \node[anchor=south west,inner sep=0] (image) at (0,0) %
      {\includegraphics[width=.6\columnwidth,keepaspectratio]{./fig/stroller-setup-lo}};
    \begin{scope}[x={(image.south east)},y={(image.north west)}]


      \tikzstyle{label} = [text width=3cm, align=center]
      \tikzstyle{line} = [draw, very thick, blue!50]
      \tikzstyle{point} = [circle,draw,very thick,blue!50,fill=none,minimum size=1.5cm,inner sep=0]


      \node [label] (lab_phone) at (0.05,0.90) {iPhone 6};
      \node [label] (lab_baby)  at (0.75,0.80) {Baby};
      \node [label] (lab_wheel) at (0.20,0.10) {Freely turning front wheels};


      \node [point] (phone)     at (0.38,0.68) {};
      \node [point] (baby)      at (0.75,0.55) {};
      \node [point] (wheel)     at (0.80,0.28) {};


      \path [line] (lab_phone) |- (phone);
      \path [line] (lab_baby) |- (baby);
      \path [line] (lab_wheel) -| (wheel);

    \end{scope}
  \end{tikzpicture}
  \tikzexternalenable

  \caption{The experiment setup with a wheeled baby pushchair/stroller and an iPhone placed on the top for tracking.}
  \label{fig:stroller-setup}
\end{figure} 

\begin{figure}[!t]
  %
  \tikzexternaldisable
  %




  \setlength{\figurewidth}{.84\columnwidth}
  \setlength{\figureheight}{0.7507\figurewidth}
  %

  \pgfplotsset{
    trim axis right,
    yticklabel style={rotate=90},
    grid style={very thin,gray!25},
    ylabel={Meters}
  }
  %

  \footnotesize\centering%


  \tikzsetnextfilename{tikz-stroller-2}%
  % This file was created by matlab2tikz.
%
%The latest updates can be retrieved from
%  http://www.mathworks.com/matlabcentral/fileexchange/22022-matlab2tikz-matlab2tikz
%where you can also make suggestions and rate matlab2tikz.
%
\begin{tikzpicture}

\begin{axis}[%
width=\figurewidth,
height=0.444\figureheight,
at={(0\figurewidth,0\figureheight)},
scale only axis,
xmin=-25,
xmax=20,
xtick={-25,-20,-15,-10,-5,0,5,10,15,20,25},
xticklabels={\empty},
xlabel={$t = 37.0$~s},
xmajorgrids,
ymin=-5,
ymax=10,
ytick={-5,0,5,10},
yticklabels={{0},{5},{10},{15}},
ymajorgrids,
axis background/.style={fill=white}
]
\addplot [color=gray,dashed,line width=0.8pt,forget plot]
  table[row sep=crcr]{%
0.0103269389045126	-0.0336058328639577\\
0.278501789313486	-0.489906141589696\\
0.87749219437659	-1.21624250065453\\
1.11002403027368	-1.12285157681853\\
0.236314881988431	-0.1614294835538\\
-0.00397140892669651	0.00457826309944895\\
-0.00385065029903127	0.00462123317922545\\
-0.00378682509306162	0.00463463813091785\\
-0.0038074453593265	0.00462647572328424\\
-0.00383703220824572	0.00464188919059538\\
-0.00384672332672233	0.00466634399517023\\
-0.00385491589714484	0.00466420406674007\\
-0.00385581603152646	0.00468880068059153\\
-0.00386602158423475	0.0046999517858656\\
-0.00389226345218809	0.00470178959877942\\
-0.00390561150997635	0.00468993728131046\\
-0.00396259582380767	0.00463665065591376\\
-0.00399555681061864	0.0045828079544999\\
-0.00399690419136082	0.00457609661836634\\
-0.00402294171754303	0.00460315490248991\\
-0.00406255722176316	0.00460156446717354\\
-0.00404601777895617	0.00461232105548733\\
-0.00410903777507603	0.00465619072567736\\
-0.00411891627082658	0.00465566564470737\\
-0.00411466731667848	0.00463970759044145\\
-0.0041101756396055	0.00464126795239423\\
-0.0040881330043217	0.00464008907172103\\
-0.0040263407232255	0.00462807605408912\\
-0.00401890938999203	0.00461943082647761\\
-0.0040016449646977	0.0046107413778973\\
-0.00397817267439755	0.00457726894503156\\
-0.00399019939068386	0.00458205620001794\\
-0.00399617666118063	0.00458868824217421\\
-0.00401903425636391	0.00458731642471325\\
-0.00405500677721846	0.00462496375715975\\
-0.00412045863549711	0.00462373856900511\\
-0.00417734900331046	0.00466708119920671\\
-0.00442110836114524	0.00494887558079991\\
-0.00483164894518048	0.00543494957027028\\
-0.00545618745747107	0.00605545558018452\\
-0.00634452938695394	0.00694504523098606\\
-0.00737688917409191	0.00804368875799442\\
-0.00860297217670184	0.00925691801854914\\
-0.00991584254575038	0.0106109272799758\\
-0.0114619641539311	0.0122747472614617\\
-0.012876754947015	0.0142345292795463\\
-0.0144123590696398	0.0163660619435325\\
-0.016016773937082	0.0183798694929277\\
-0.0174920030375568	0.0205746985111592\\
-0.01713053090629	0.0225791052846711\\
-0.0133756136000719	0.0236659905310159\\
-0.00430034222882562	0.0249018479511279\\
0.0116939104720748	0.0268415903980102\\
0.035216378482335	0.027217648259604\\
0.0663588597466251	0.0269971875194111\\
0.105248359618161	0.0268590607571381\\
0.150953446947431	0.0254118910735926\\
0.203690816827192	0.0233676680131207\\
0.262752420524217	0.0181427652109263\\
0.326183145096281	0.0113128131382109\\
0.394281186545766	0.00603436750471615\\
0.465208355637247	-0.0049524111812778\\
0.542475855831044	-0.0162371445294915\\
0.626418161910732	-0.0164332889858583\\
0.715999174540469	-0.0142951909911941\\
0.810423787520278	-0.0133483084101371\\
0.910048572775709	-0.0143056595945424\\
1.01445484164042	-0.0131329718326544\\
1.11869572779703	-0.0128303357868091\\
1.22638179816935	-0.0127447592178104\\
1.33245697238448	-0.00859811107491129\\
1.44139658382791	-0.00861634454747132\\
1.55170928359023	-0.00821617455881603\\
1.66555497277893	-0.00551180674490376\\
1.78340033580841	-0.00577699096982132\\
1.901138076309	-0.00294694200322526\\
2.01668419261958	4.74608163979207e-05\\
2.13424336773358	-0.00289951207851205\\
2.25039699582256	-0.00600697309026296\\
2.37027803990069	-0.00739180633388667\\
2.494219637974	-0.0105013807417679\\
2.61768245151388	-0.0131071271166303\\
2.74151639243623	-0.0173824739257015\\
2.86627502596162	-0.0250415154272412\\
2.99207777505397	-0.0380910010501901\\
3.11806537028834	-0.0514775007847373\\
3.24786623300026	-0.0581713494736401\\
3.37860174093443	-0.0706888677274701\\
3.50663343025307	-0.0858279157182961\\
3.6371536325667	-0.0972472701366352\\
3.76605890425648	-0.106563108701319\\
3.89537771065028	-0.11516815041483\\
4.02490455420131	-0.118617560728042\\
4.15306191359771	-0.121380507913208\\
4.28495039441583	-0.12280845931653\\
4.4131272329558	-0.122493781285735\\
4.54380628893909	-0.115886035583055\\
4.67134031301862	-0.0997260376421804\\
4.79492932462867	-0.0770566470880049\\
4.92004815767136	-0.0525229574925304\\
5.0426749734133	-0.0229559031667613\\
5.16408758218846	0.0135039730871678\\
5.27958952261709	0.0605647647234617\\
5.39712948270113	0.109928253414812\\
5.51440532422004	0.156636659937874\\
5.63024525834151	0.20697215965992\\
5.74309423429395	0.263082195171674\\
5.85725319290907	0.323262431934729\\
5.9712721377629	0.383811827257659\\
6.08201915250973	0.44211433638803\\
6.19706309240174	0.498910822812343\\
6.30929517032533	0.555214697521271\\
6.42140585239913	0.613911339831118\\
6.53203430691706	0.678487119498162\\
6.63956848309798	0.739226866066907\\
6.74650232563251	0.794323534881919\\
6.85309434734913	0.849836381500601\\
6.96075995977918	0.907103436471046\\
7.06953792455724	0.964564173989664\\
7.1778458103267	1.02359028848562\\
7.28907478486214	1.07892909987716\\
7.40190486770525	1.13037432258513\\
7.51186625035077	1.18201896066502\\
7.62614565327529	1.23213010400705\\
7.73803908079967	1.28214759784659\\
7.84754893889926	1.33110411220787\\
7.95718093846889	1.37666208488657\\
8.06741897367414	1.41702028096296\\
8.17759939794053	1.45414195385857\\
8.28716769594472	1.48873161254696\\
8.39539077216543	1.51964783003731\\
8.50140213968702	1.54167879076829\\
8.59979744254983	1.55867433725474\\
8.69481663401933	1.5738051403289\\
8.784964228022	1.58956612322886\\
8.8685487162298	1.60107709520195\\
8.94441052805033	1.60657467304933\\
9.01365080766615	1.61135958368202\\
9.07533631364789	1.61560807904911\\
9.12839056481216	1.61801949950707\\
9.17357210792354	1.61922697320233\\
9.20778241488074	1.61989634941204\\
9.23201760994683	1.62151180789206\\
9.24793094279027	1.62269187487398\\
9.25644762052972	1.62444037425635\\
9.26178864171396	1.62707750843622\\
9.2637814118162	1.62967363844485\\
9.26397466196395	1.6313098046899\\
9.2635695515283	1.63318376510264\\
9.262959924607	1.635189055653\\
9.26050055343164	1.63689720348422\\
9.25851009450988	1.63839982771162\\
9.25603876758261	1.63947419197701\\
9.25410853801571	1.64048720703006\\
9.25251175575484	1.64126541790145\\
9.25144091434786	1.64168957205983\\
9.25075900814933	1.64182765209988\\
9.25060876187831	1.64181238637562\\
9.25054607107374	1.64175249837659\\
9.25054901418395	1.64173185986774\\
9.25049082296123	1.64182573584587\\
9.25047715367717	1.64193999776716\\
9.25049273453673	1.64197231114744\\
9.25047853723937	1.64206090360196\\
9.25044625545543	1.64212533190375\\
9.2504619668665	1.64211910848319\\
9.25048867569463	1.64211080663775\\
9.2505041509171	1.64213091287724\\
9.25047362040631	1.64209391588056\\
9.2504305096644	1.6420532571652\\
9.25041069212886	1.64203097837218\\
9.25038582956779	1.64200796861395\\
9.25034292616562	1.64201940788822\\
9.25031957660305	1.64202003913031\\
9.25028294046754	1.64202791542441\\
9.25027687059844	1.64202572486801\\
9.25024209283712	1.64202386365655\\
9.25026826823144	1.64201231824038\\
9.25026225278583	1.64200957899892\\
9.25027250356363	1.64200994606289\\
9.250266681071	1.64199846419118\\
9.25028391742417	1.64199377361963\\
9.250262538289	1.64198316119434\\
9.25025835934023	1.6419847822127\\
9.25025746579413	1.64196672649488\\
9.25030857611288	1.64195765163767\\
9.2503407913386	1.64196075671332\\
9.25040702901152	1.64196541165326\\
9.2504425779603	1.6419829120317\\
9.25084140896116	1.64196888923235\\
9.25142224312645	1.6420982248462\\
9.25295049998587	1.64197541077519\\
9.25794062267128	1.64168654480054\\
9.26875705801955	1.64082791471881\\
9.28748421412418	1.63903325451586\\
9.31570186729328	1.63633575742149\\
9.35396600314039	1.63472801458212\\
9.40490276244721	1.63186227654239\\
9.46528024470586	1.62687575684683\\
9.53583635510757	1.61988559362491\\
9.61575326106376	1.60943020059734\\
9.70142261796248	1.59995110510971\\
9.79165363077104	1.58866393397526\\
9.88288032407644	1.57214643138389\\
9.97667942553024	1.55376669675814\\
10.0753550668132	1.53748938715095\\
10.1746166866511	1.52379758706141\\
10.2811441050324	1.50571652205691\\
10.3914736754222	1.48363672257644\\
10.5015929947603	1.46444522668809\\
10.6170797217091	1.44591301600615\\
10.7328829802264	1.42751369200134\\
10.8506109657954	1.40838773126901\\
10.9724595220891	1.39059427770646\\
11.0938570582468	1.37390642319482\\
11.2173970130595	1.35998018921816\\
11.3427713448625	1.34524118135857\\
11.4639933616538	1.32779595678834\\
11.5910507422516	1.31387382080072\\
11.7213136582956	1.29945380774123\\
11.8502670655447	1.28352742493704\\
11.9817855179115	1.27122542482536\\
12.1101041389456	1.25846511801565\\
12.2406977629523	1.24241144267876\\
12.3760231398966	1.23062705517877\\
12.5094470523416	1.22470657715516\\
12.6443884126553	1.21891990739537\\
12.7799924653766	1.21021658150317\\
12.9135539816107	1.20544703617319\\
13.0503484284216	1.2040649752512\\
13.1880208296483	1.20094376389081\\
13.3233546182681	1.20154670766975\\
13.4591584175667	1.20803829281434\\
13.5949225333203	1.21337690731593\\
13.7290017070322	1.21971593515107\\
13.8637052736797	1.23499698461786\\
13.9961167336946	1.25641481543602\\
14.1289896528746	1.28231577306095\\
14.2613825687776	1.31298227887303\\
14.3888589382315	1.34711683513207\\
14.5218297200331	1.38730428674016\\
14.6506637318569	1.42947188474825\\
14.7749227378466	1.47896088429781\\
14.8979205086719	1.53632400878001\\
15.0175958212671	1.59778123568317\\
15.1388621562201	1.66010091317191\\
15.2541057479562	1.73338289611146\\
15.3661098721842	1.81591701898454\\
15.4726053144358	1.90246607430537\\
15.5761435613946	1.99231153941791\\
15.6779563101508	2.08082868158116\\
15.7765085663889	2.1778457782653\\
15.8663674860024	2.2823986322891\\
15.9502337857409	2.39223538213668\\
16.0293935866165	2.50777916807673\\
16.1016575751607	2.61970827060563\\
16.1736060737978	2.73298432208145\\
16.2409683559766	2.85103786833708\\
16.3007253274328	2.97392123108219\\
16.3548070052648	3.09763326325894\\
16.4075128789106	3.21926314213346\\
16.4582030023846	3.34093451822254\\
16.5038693056004	3.46837811293086\\
16.5418880941823	3.59922776718389\\
16.5722135466436	3.73435679091016\\
16.6005950308892	3.86844348129196\\
16.6243232743927	3.99993935970328\\
16.6438840301607	4.13352987432718\\
16.6578062827302	4.2672406813032\\
16.6671445418413	4.40453695274579\\
16.6707293688651	4.54021929959789\\
16.6714237693954	4.67317501664795\\
16.6680142047858	4.80707109565718\\
16.6594395074225	4.94161952399884\\
16.6490210993222	5.07681985493497\\
16.6341501128229	5.21222208217043\\
16.6171241700261	5.3447591800358\\
16.599046471706	5.47392171946008\\
16.5757071775535	5.60441024190898\\
16.5474522102481	5.73539330494972\\
16.5168159363755	5.86440988993181\\
16.4817942863559	5.99367715132427\\
16.4424370585272	6.12078599878202\\
16.4003936556137	6.24554565293671\\
16.3552283870017	6.37126594018821\\
16.3049360108457	6.49351014483119\\
16.2502437901382	6.61216001143138\\
16.1964417707759	6.73048170508476\\
16.1395713312637	6.84764935884045\\
16.0798645973678	6.9651023948386\\
16.0172291087945	7.07849610858098\\
15.9505127646269	7.18850495793284\\
15.8812228108286	7.29633318991668\\
15.8121036747558	7.40310464573685\\
15.7390468831284	7.50746755072875\\
15.6612373069975	7.60942533381729\\
15.5819264635917	7.71269795820961\\
15.4994093431756	7.80948903731251\\
15.4163499757416	7.90426488631477\\
15.3354081887654	7.99931549429373\\
15.2522039207761	8.09192353506612\\
15.1658398053246	8.17540708716605\\
15.0783006558137	8.25262887379566\\
14.9943006490147	8.32636567717311\\
14.9158959910506	8.40212137030252\\
14.8355797153284	8.46994959045886\\
14.7485005420541	8.53226633046858\\
14.6623213745983	8.59387746633567\\
14.5770863861545	8.65637085241178\\
14.4921665033328	8.7118226441906\\
14.4043721444894	8.76195170161644\\
14.3180314666358	8.81205086601585\\
14.2332685342467	8.85837235624441\\
14.1480297032657	8.89754032165848\\
14.0674615318587	8.92865627301121\\
13.990664695405	8.95537233922407\\
13.9173182865316	8.97941381372292\\
13.8485934900534	9.00000784606318\\
13.7824981946652	9.01737996819088\\
13.7216823487106	9.03086723942614\\
13.6644283661557	9.0392976381526\\
13.6155394905903	9.045543459099\\
13.5750175051401	9.05190790379228\\
13.5422303171644	9.05625822861468\\
13.5168203652657	9.05740145429537\\
13.4985305433862	9.05897005644958\\
13.486100148344	9.06091618496389\\
13.4778586150701	9.06197859041248\\
13.472971041419	9.06178902371255\\
13.4701643542022	9.06214907169022\\
13.4692672601274	9.06294233525307\\
13.4692647281559	9.06270572382741\\
13.4697425884381	9.06223065048548\\
13.4698598556491	9.06211659247461\\
13.4695992222452	9.06177361082215\\
13.4690065150716	9.06113498955688\\
13.4679828073941	9.06074996503337\\
13.4671463719493	9.06053084160427\\
13.466381654897	9.06013864191982\\
13.4657147340284	9.059902042399\\
13.4648436009525	9.05965000212206\\
13.4641549187424	9.05931312021426\\
13.4636186597998	9.05910832051968\\
13.463435956788	9.059087183208\\
13.4633782956286	9.05879495670787\\
13.463309534724	9.05875224784219\\
13.4632825622388	9.05876049318771\\
13.4632542318601	9.05847631100894\\
13.4632574240701	9.05850125610002\\
13.4632798842914	9.05851826951383\\
13.4633326879677	9.05841193930063\\
13.4633507023809	9.05852965459937\\
13.4633533856229	9.05868708451989\\
13.463338011982	9.05868738186939\\
13.4632763117668	9.0588234716843\\
13.4632370495206	9.05890952860264\\
13.4631911559879	9.05891045140912\\
13.4632119968526	9.05895915435102\\
13.4632202044357	9.05895753678758\\
13.4631951054619	9.0589104926545\\
13.4631827425723	9.05889840908532\\
13.4631840744636	9.05890189894758\\
13.463223597501	9.05884286207664\\
13.4632233103903	9.05881956845674\\
13.4632249020451	9.0587849155505\\
13.4632563350171	9.05875407281572\\
13.4632740588249	9.05874876042301\\
13.463290278308	9.05876647895255\\
13.4632934494669	9.0588009144232\\
13.463244753047	9.058805766478\\
13.4631985206491	9.05882744805644\\
13.4633187485655	9.05875086159461\\
13.4629224621753	9.05910446440041\\
13.4602393949048	9.05979969159865\\
13.453119412728	9.0599315919168\\
13.4394481529651	9.0600539939569\\
13.4176572548676	9.06140228116975\\
13.3855356678185	9.06152347168285\\
13.3427863081361	9.05882369351287\\
13.2900122036728	9.05528958042339\\
13.2270463509498	9.05067041372384\\
13.1557740775883	9.04355539889044\\
13.079793433591	9.02982126899012\\
12.9985593687419	9.0096271048649\\
12.9136038296545	8.98896282588228\\
12.8269211538492	8.96272112904956\\
12.7392505481338	8.92870659726051\\
12.6483443244265	8.88913483368786\\
12.5551277967114	8.8453839727284\\
12.4602598444806	8.79384463813354\\
12.368261454597	8.73488059989262\\
12.2769914099883	8.67289609875325\\
12.1854488622084	8.61078106503657\\
12.0950382642891	8.53977533686385\\
12.0055266719867	8.4618942129087\\
11.9171486901297	8.37953326175558\\
11.8323147944693	8.29156204626061\\
11.7493754347791	8.20230756382731\\
11.6666391298933	8.11259071088053\\
11.5841763463833	8.01789382123911\\
11.5041419524301	7.92244229738386\\
11.4204859222238	7.82608896581356\\
11.3348734797012	7.73250866237616\\
11.2508445214749	7.64337198521719\\
11.1643936685241	7.55731104731318\\
11.0702886265564	7.47116382135284\\
10.974376735735	7.3886533327033\\
10.8799067874887	7.30504465131841\\
10.7814575749445	7.22171309585684\\
10.6779886551929	7.1468890634001\\
10.5722784462175	7.07047287062497\\
10.4630711259327	6.99326334892379\\
10.3492209550645	6.92237204082993\\
10.232294651634	6.8565369227511\\
10.1155537476475	6.79075188756224\\
9.99794506287057	6.72692567508149\\
9.87675086212485	6.66680797579755\\
9.75318685163088	6.61132365226119\\
9.62710413553661	6.55924465823375\\
9.50002843380128	6.51346701904514\\
9.37306766025639	6.46798400768186\\
9.2429874543177	6.42327699225196\\
9.10910463922886	6.38644740660042\\
8.97411312852411	6.35246092163036\\
8.83572766775398	6.31803465473672\\
8.70070913531952	6.28463135099351\\
8.56511876103578	6.25751710129093\\
8.42517920363083	6.23455786676026\\
8.28450625312795	6.21184679637219\\
8.14841722489797	6.19273341884587\\
8.00783753631045	6.17500188583927\\
7.8700977327396	6.15628579188258\\
7.73462237631482	6.14200703177832\\
7.59351196492898	6.13131609534772\\
7.45443487026879	6.12179950239596\\
7.3165978014135	6.1077792905897\\
7.17788348162871	6.09499295653531\\
7.03799861313196	6.08765367586544\\
6.89814751742361	6.08066907311093\\
6.75799565673137	6.07188642492231\\
6.61588169929492	6.06479581277499\\
6.47643913324609	6.05801415554759\\
6.33715302940159	6.04866167845097\\
6.19348066443017	6.04619991369662\\
6.05211156379989	6.04396746032669\\
5.90967716316949	6.03637408171243\\
5.76364751818759	6.02600508616127\\
5.62051026521014	6.0203496582249\\
5.47586615361119	6.01954642140475\\
5.32892109662997	6.0184431180986\\
5.18120115938864	6.01437710539688\\
5.03394643064646	6.00777154017654\\
4.8846774925337	6.00250673343391\\
4.73731409082565	6.00169567975507\\
4.59125046057075	6.00345759339506\\
4.44208012516938	6.00025000478072\\
4.29427658547132	5.99289992721384\\
4.14739662157714	5.98547883344516\\
4.00104235541819	5.98475687641608\\
3.85466139537104	5.98468971827389\\
3.70736263562793	5.97969799703192\\
3.55797294375675	5.97469035574344\\
3.40939957048286	5.97369909083288\\
3.26442632841279	5.97131933068691\\
3.11781874859768	5.97145036874881\\
2.97083938837511	5.96788518397652\\
2.82611507291501	5.96023044230304\\
2.67756854634337	5.95583235656968\\
2.53323131626212	5.95475309078086\\
2.39250319765962	5.95177296756577\\
2.24907031077274	5.94288561248268\\
2.10869794290739	5.93520992508363\\
1.97173137770089	5.93073242360556\\
1.83538434261827	5.92710041673153\\
1.70172976378748	5.92142653239001\\
1.56923227070891	5.91543765807853\\
1.4394624182375	5.90656192779744\\
1.31287553384125	5.90046380607169\\
1.19106254580116	5.89501382142563\\
1.0764856646797	5.88882617154235\\
0.962545612598274	5.8852006299273\\
0.855220834226946	5.87988391307629\\
0.756708871438657	5.87129866708636\\
0.664275347399113	5.86412998994399\\
0.578846594783561	5.86038430256742\\
0.504191956376299	5.85625558987713\\
0.436687600722196	5.85165076641582\\
0.380113002376999	5.84822787364657\\
0.332739434383852	5.84549030580905\\
0.292993356782038	5.84303365367697\\
0.263604738789397	5.84072293432662\\
0.240737165210177	5.83899378364481\\
0.2253677489733	5.83803222403294\\
0.214270074977396	5.83720034287128\\
0.206170322405864	5.83653609641065\\
0.200405729393371	5.83661590580498\\
0.196473622057645	5.83673711580015\\
0.194240422674213	5.83628573337145\\
0.193208701278066	5.83634211391775\\
0.192878024572207	5.83668569998948\\
0.192778691529331	5.83656872118827\\
0.192657261505016	5.83655054464418\\
0.192414638098174	5.83671562126093\\
0.192197641576818	5.83657164122002\\
0.192027646031834	5.83637716781703\\
0.192089410119591	5.83641880118548\\
0.192193632031215	5.83632839784038\\
0.192265271841128	5.83622691220537\\
0.192220794151321	5.83628379152211\\
0.192218028862463	5.83626702611131\\
0.192223693482835	5.83626109631625\\
0.192206797203417	5.83632699495828\\
0.192162558853579	5.83635327173616\\
0.192138277180388	5.83634079055766\\
0.192089069417325	5.83639207149698\\
0.192056367511352	5.83639269507781\\
0.192027256601988	5.83634212925286\\
0.192021666224881	5.83632969874173\\
0.192027814705353	5.83630908418187\\
0.192039891841005	5.83626165089078\\
0.19199409826531	5.83625955581342\\
0.192017118600983	5.83626081777702\\
0.19205382235145	5.83626385289706\\
0.192065393916505	5.83626106949741\\
0.192094363211912	5.8362383461099\\
0.192091925462566	5.83624180024431\\
0.192091327218238	5.8362433075822\\
0.192065659427502	5.83621889398323\\
0.19205628157254	5.83620775694055\\
0.192021351830379	5.83622880567102\\
0.191994890485469	5.83625864898512\\
0.192033748418168	5.83628123872977\\
0.19203080493265	5.83627702984269\\
0.192028040969289	5.83629153744254\\
0.191977744613689	5.83628773607425\\
0.191988544903619	5.83627653310999\\
0.19197579785653	5.83628721900115\\
0.191959675810059	5.8363037829727\\
0.191969658825357	5.83631029491055\\
0.191983089094947	5.83633946692264\\
0.19201818666682	5.83634548065676\\
0.192011862734825	5.83634327834629\\
0.191982988146857	5.83642662911672\\
0.191955138864175	5.83646416071095\\
0.191970744467447	5.83647541624309\\
0.192019873689103	5.83654872061129\\
0.192031763210394	5.83656331717799\\
0.192034611132813	5.83647132811937\\
0.192072949418045	5.83645145042756\\
0.192081903725611	5.83641442006252\\
0.192077982749621	5.83636976581086\\
0.192067103188284	5.83636698412639\\
0.192073817942831	5.8363523401119\\
0.192053987030156	5.83632781247386\\
0.192067730000451	5.83633205085277\\
0.192087667223252	5.83634812697968\\
0.192146931393284	5.836317025711\\
0.192133408993539	5.83634257771984\\
0.192180920583757	5.83649727926139\\
0.191528800174748	5.83650130580017\\
0.190267261436765	5.8362086433566\\
0.184637368496135	5.83683128782321\\
0.17250208292248	5.83608579522434\\
0.150973757157279	5.83344780477957\\
0.117461507083797	5.83023788496557\\
0.0721609589167941	5.82693513152253\\
0.0168377300820675	5.820953289644\\
-0.0513025638468028	5.81397376929278\\
-0.128458052723002	5.80700320285275\\
-0.214047046161554	5.7995435793628\\
-0.307222273019431	5.79027646031215\\
-0.408082305316133	5.78019743038333\\
-0.513130132125113	5.77032489679014\\
-0.62624813643627	5.7595454928174\\
-0.744592188718582	5.74551523174158\\
-0.869426618997852	5.73413044497448\\
-0.997422927103059	5.72528965045686\\
-1.12455751606655	5.71458499294802\\
-1.25091535988869	5.70493695264556\\
-1.38007133869957	5.69477822438606\\
-1.50938597127305	5.68297042760668\\
-1.63877619102581	5.67213449969741\\
-1.77464319243149	5.66169321364343\\
-1.9128440399898	5.65545559801821\\
-2.05116389407095	5.64730885946172\\
-2.19148727017903	5.63651656410109\\
-2.33421964904621	5.62611336130674\\
-2.47302745144498	5.62065137434931\\
-2.61230192404688	5.61697352648635\\
-2.75267693996776	5.60709281430635\\
-2.88956785144703	5.59673947129324\\
-3.0291517694222	5.59027970090562\\
-3.16973235409861	5.58527627299929\\
-3.31125306700997	5.57730772478058\\
-3.45358317387205	5.56873712655363\\
-3.59622509970132	5.56102894766276\\
-3.73954316637866	5.55317146377811\\
-3.88352290170339	5.54445484549892\\
-4.02615743464784	5.54175633884196\\
-4.16976407797467	5.5358000083324\\
-4.31126037333373	5.52482495120206\\
-4.45891165491369	5.51876873704534\\
-4.60420434774172	5.51612632683563\\
-4.7484472952641	5.5132058764837\\
-4.89548342867606	5.50193233286841\\
-5.04348659046376	5.49207674232509\\
-5.18811845931405	5.48713873449133\\
-5.33679053036774	5.48548079593594\\
-5.48485969228556	5.48189137424147\\
-5.63204579328475	5.47343881051361\\
-5.77969828136424	5.46291974589439\\
-5.92620755101303	5.45827307612996\\
-6.07164935625616	5.45453637491025\\
-6.22216757636865	5.45205008455213\\
-6.37200116630795	5.44575908646512\\
-6.51969615217807	5.4353132705985\\
-6.66950617255401	5.42766597850916\\
-6.81735286420405	5.42490765843028\\
-6.96578395920413	5.42291037650871\\
-7.11748348836652	5.41440348605005\\
-7.26970403303207	5.40098260200454\\
-7.42089698610081	5.39567723311886\\
-7.57236446615009	5.39511448272837\\
-7.72500227544779	5.38898205393461\\
-7.87744282370555	5.37918988851715\\
-8.03306151445366	5.37265793336039\\
-8.18523609368713	5.3707832118316\\
-8.33493468519019	5.36643076470396\\
-8.48816429268027	5.35906957622685\\
-8.63958134138863	5.35103266803356\\
-8.79237188600173	5.3463056105526\\
-8.94448495367411	5.34006091328896\\
-9.09289772626449	5.33472682445092\\
-9.24493678963999	5.32491240559329\\
-9.39975994652957	5.31393306662686\\
-9.54986708871186	5.30731784759526\\
-9.70060437925911	5.30177661081584\\
-9.85271657336519	5.29191092904254\\
-10.0007957114571	5.28118576503335\\
-10.1525685096271	5.2710908288637\\
-10.3008884678805	5.26166382031407\\
-10.4456696699597	5.25008883659091\\
-10.5923503846985	5.2341162273825\\
-10.7378716566918	5.2173363980478\\
-10.8824282657979	5.20450395657684\\
-11.0272756689425	5.18818434437143\\
-11.1693334882543	5.16700812008829\\
-11.3094887196801	5.14473382907937\\
-11.4487183358903	5.11999713041633\\
-11.5917949385453	5.0961900756865\\
-11.7325106113252	5.07157742248859\\
-11.8736934490823	5.0417857832085\\
-12.0138917930326	5.00374397838384\\
-12.1499519656331	4.96455073496946\\
-12.2858838840365	4.92161942429734\\
-12.4206352522558	4.87503858304951\\
-12.5528302932694	4.8275956344368\\
-12.6800743159332	4.76884219284885\\
-12.8040806552427	4.70743218967958\\
-12.928706828419	4.64522942057427\\
-13.0504853872525	4.57835722791146\\
-13.1691507148802	4.50864205386596\\
-13.2865857196511	4.43132275178574\\
-13.3997784996304	4.35192135961469\\
-13.5090610362673	4.27029395248132\\
-13.6164539580017	4.1828486701697\\
-13.7224359104157	4.0943487687301\\
-13.8247141158087	4.00480284311215\\
-13.9273702145182	3.91458526933169\\
-14.0253198230477	3.82488516976469\\
-14.1251649602036	3.73678743433979\\
-14.2228014060626	3.6473091620237\\
-14.3211684933456	3.56081159769459\\
-14.4191429533553	3.47436563448848\\
-14.518138722825	3.39506606114366\\
-14.6227284137722	3.32159219671099\\
-14.7287344659322	3.2502198537676\\
-14.8319773033408	3.18182123030033\\
-14.9394154185512	3.11451851939459\\
-15.0531960209555	3.0532189166384\\
-15.1628592525349	2.99419076554395\\
-15.2776844382139	2.93838249338209\\
-15.3931296796979	2.88734731506338\\
-15.5073712304988	2.83954337827642\\
-15.620894795901	2.79316573975845\\
-15.7321162491215	2.75511248432896\\
-15.8396714345892	2.72539150636919\\
-15.9446292741432	2.69522901740752\\
-16.0492503237197	2.66834475834248\\
-16.1502624126818	2.64547359040358\\
-16.2468427151435	2.6235972675644\\
-16.3377466511541	2.60661381423737\\
-16.4253224098355	2.59531762795035\\
-16.5048692209586	2.5844331269779\\
-16.5793164795662	2.57431297617609\\
-16.647125246511	2.56882631616587\\
-16.7062349640419	2.56597558618037\\
-16.7552467221284	2.56112665655648\\
-16.7942569923228	2.5578968713733\\
-16.8230189687621	2.55769051742551\\
-16.8432141136712	2.55646306868059\\
-16.857105351206	2.55538628559035\\
-16.8667224981701	2.55676500952284\\
-16.8739337510723	2.55808592815465\\
-16.8794841771865	2.55873660855177\\
-16.8833345275572	2.55994573257958\\
-16.8860381783037	2.56159095576183\\
-16.8874470831396	2.56329929908368\\
-16.887742193971	2.56495738393393\\
-16.8879815107101	2.56663989064887\\
-16.8878845192262	2.56829405770246\\
-16.8880842839073	2.56975949997548\\
-16.8883507113728	2.57109798816129\\
-16.8887934888046	2.57216656605684\\
-16.8891737910379	2.57336281444146\\
-16.8896233114219	2.57454255101965\\
-16.8897521150168	2.57541908171845\\
-16.8898322839442	2.57623821335329\\
-16.889745160578	2.57709216514254\\
-16.8897442289093	2.57762692538957\\
-16.8897034596198	2.57805129613497\\
-16.8897591520427	2.57840344964781\\
-16.8897233185884	2.57851518831848\\
-16.8897614941948	2.57859236698521\\
-16.8897185722177	2.57870193980281\\
-16.889689939499	2.57871789819091\\
-16.8896449994401	2.57874705466066\\
-16.8896754536475	2.57879635440201\\
-16.8896766393635	2.57877576857469\\
-16.8896583187586	2.57877298661095\\
-16.8896377602304	2.57878999222955\\
-16.8896265823937	2.57879970719254\\
-16.8896293364579	2.57877682650643\\
-16.8896218298523	2.57877743566601\\
-16.8895749449707	2.57876963696298\\
-16.889568898678	2.57876408544163\\
-16.8895902885903	2.57875709820867\\
-16.8896312926592	2.57873361775927\\
-16.8896484467459	2.57874356071093\\
-16.889655922689	2.57873572490657\\
-16.8896314319052	2.57868633506163\\
-16.8896681133853	2.57866040258414\\
-16.8898006130585	2.57858669202766\\
-16.8905756391323	2.57860393229322\\
-16.8923151681515	2.57869193191294\\
-16.8976956821392	2.57792808312864\\
-16.9089862693849	2.57707605518932\\
-16.9279349436601	2.57446329655217\\
-16.9560660014578	2.56949065634764\\
-16.9938234652911	2.56411948552449\\
-17.0413227678388	2.55790224956308\\
-17.0979729758774	2.54937983738815\\
-17.163316582185	2.53771258249646\\
-17.236820215967	2.52516431832639\\
-17.3177174460584	2.51212577728499\\
-17.4045342255223	2.49939873788995\\
-17.4960047621209	2.48611764614644\\
-17.5922447753565	2.46904708338596\\
-17.6945617193809	2.44936428175797\\
-17.8016481820209	2.4320320483926\\
-17.9110894890666	2.41520852246721\\
-18.0207249786857	2.39501127319995\\
-18.1294537474674	2.37010772708401\\
-18.2421361564129	2.34055834309476\\
-18.3559655053926	2.30990135300324\\
-18.4703220892696	2.27353249844683\\
-18.5848908546111	2.22942867115659\\
-18.6974867811721	2.18267405791444\\
-18.8064264881713	2.13398232752445\\
-18.912036185135	2.0784557241892\\
-19.0102327925961	2.01269594121921\\
-19.1076564344779	1.94137177644199\\
-19.2046480984412	1.86602811880957\\
-19.294819318223	1.78341283313786\\
-19.3774851559884	1.69399272739932\\
-19.4556186300384	1.60167559891559\\
-19.531481552923	1.5024486672399\\
-19.5995506585858	1.39988011747814\\
-19.6611794737827	1.2911806631594\\
-19.7123338080165	1.1761398314181\\
-19.7559106581905	1.06126854877971\\
-19.7993214917541	0.945648376129782\\
-19.8380073543923	0.828664589022003\\
-19.8647525503513	0.708852225043078\\
-19.8840766184833	0.58580674281244\\
-19.8985301535263	0.4630603253744\\
-19.9095625966528	0.341442286980434\\
-19.916202556449	0.219538583101421\\
-19.9152658473629	0.094345081587789\\
-19.9071180124539	-0.0307132724639111\\
-19.8933263399812	-0.153353247569208\\
-19.8740846394168	-0.277544868229196\\
-19.8502184821226	-0.398750500483338\\
-19.8246890817996	-0.518765606516356\\
-19.7913519001527	-0.638980069221455\\
-19.747278820008	-0.75376273748963\\
-19.6967323269565	-0.866942735599506\\
-19.6463027356185	-0.980302871433052\\
-19.5916581392254	-1.09001016901715\\
-19.5298989865015	-1.19716369971428\\
-19.4604823085608	-1.30033691019711\\
-19.3839304724689	-1.39791132193262\\
-19.3044885071713	-1.492624749267\\
-19.2212673352365	-1.58110704241762\\
-19.1323906540324	-1.66583385134984\\
-19.0408598089465	-1.74863058597174\\
-18.9452094517538	-1.82322922646955\\
-18.8430146969619	-1.89352053342814\\
-18.740187181012	-1.95811644580365\\
-18.6337407576425	-2.02030779177414\\
-18.5220644489562	-2.07655707410579\\
-18.4075035849489	-2.12574070247495\\
-18.28852684374	-2.16929811640854\\
-18.1662022858575	-2.20642072981896\\
-18.0455901591807	-2.23526895178916\\
-17.9194376441778	-2.26067671286219\\
-17.7925301616184	-2.28022445648651\\
-17.6634004577104	-2.294191933974\\
-17.5350112610869	-2.29947189390461\\
-17.4070907452283	-2.29934676625181\\
-17.2790007589517	-2.2965556182463\\
-17.1519960874774	-2.29280075365218\\
-17.0195330490404	-2.28814496882084\\
-16.8905516601539	-2.27591249639827\\
-16.7604968952462	-2.25963393271291\\
-16.631291432649	-2.24108795695927\\
-16.500622093709	-2.22290521066557\\
-16.3690365105924	-2.2072739156375\\
-16.2419236931027	-2.18500602526075\\
-16.1077813735065	-2.15613843469059\\
-15.9792008446619	-2.13235033183341\\
-15.8470246911112	-2.10923299186298\\
-15.7156089499173	-2.08547359096646\\
-15.5853446232215	-2.06061116979571\\
-15.4528641440805	-2.03282228508533\\
-15.3216703523965	-2.00298563679159\\
-15.1910006223904	-1.97490245806323\\
-15.0602201620703	-1.9502050354226\\
-14.9293216382364	-1.92022495564558\\
-14.7985979840155	-1.88727803412142\\
-14.6677633251412	-1.8562671775705\\
-14.5380207035069	-1.82804773113092\\
-14.4083437420802	-1.80184830031135\\
-14.2787733561742	-1.77504996571519\\
-14.1507364328908	-1.74383854595525\\
-14.0207137377463	-1.71073515175093\\
-13.8925569554473	-1.68169082104958\\
-13.761726994518	-1.65619034510569\\
-13.6293884320535	-1.6296400977796\\
-13.4974177501548	-1.59948277048558\\
-13.3677639543776	-1.56631229139042\\
-13.2390933387295	-1.53380886206797\\
-13.110019614334	-1.50839163486185\\
-12.982924192846	-1.4826480886422\\
-12.8532301598594	-1.45089828469383\\
-12.7246347719329	-1.4197579384602\\
-12.5971791340563	-1.39037008586078\\
-12.4673215643662	-1.36083541255139\\
-12.3382155978623	-1.33694393770552\\
-12.2070182418999	-1.30890980335578\\
-12.0794642997285	-1.27404948695515\\
-11.9489657349855	-1.23993453967464\\
-11.8195787654616	-1.21219559528404\\
-11.6923946600391	-1.18502684261564\\
-11.5670411504254	-1.15236806024461\\
-11.4429988691747	-1.12242464242604\\
-11.3240232647005	-1.09430060882595\\
-11.2046215058369	-1.06516797450544\\
-11.0924563104531	-1.04108031897572\\
-10.9782041864308	-1.01838137067365\\
-10.8663119171102	-0.990561149729751\\
-10.7589394446463	-0.963349871735852\\
-10.6569957962752	-0.940703771774692\\
-10.563937897853	-0.919206027922718\\
-10.4759820901314	-0.898761204839082\\
-10.3972908337602	-0.882957914183179\\
-10.330375639635	-0.869936680898707\\
-10.2724034066117	-0.856533550869932\\
-10.2260623729326	-0.846036539699312\\
-10.1897783975209	-0.837392253899091\\
-10.1612857122825	-0.831452506262806\\
-10.1417296690856	-0.827035138396354\\
-10.1280661097699	-0.824345456515281\\
-10.1188064623019	-0.822803784661156\\
-10.1130953199586	-0.821714900533247\\
-10.1091912662481	-0.821361811651817\\
-10.1072134830779	-0.821352497160289\\
-10.1063281189331	-0.821417064845345\\
-10.1062628296248	-0.821621253257378\\
-10.1064906979024	-0.821885842973069\\
-10.1069694581503	-0.82224023169958\\
-10.1073322408246	-0.822425121399114\\
-10.1074668069244	-0.822537537043669\\
-10.1074542969894	-0.822632812485737\\
-10.1074368337898	-0.822663344466892\\
-10.1073949770432	-0.822633311892085\\
-10.1072868856298	-0.822625273197502\\
-10.107277334749	-0.822623566988714\\
-10.1073268204879	-0.822595115867721\\
-10.1073522071708	-0.822589751519583\\
-10.1073861987275	-0.822591780634284\\
-10.1074771527301	-0.822557641943454\\
-10.1075896863093	-0.822561655768572\\
-10.1075769145613	-0.822563993001148\\
-10.1075406978211	-0.822559494601024\\
-10.1075239092873	-0.82255748185518\\
-10.1074922664978	-0.822590723171904\\
-10.1074981457356	-0.822588780744828\\
-10.1075256077492	-0.82260630011713\\
-10.1075221681079	-0.822622361096856\\
-10.1075040370553	-0.82261382909641\\
-10.10748727266	-0.822648416349649\\
-10.1074890335853	-0.822700708838657\\
-10.1075070011922	-0.822720181418125\\
-10.1075182783712	-0.822733799511384\\
-10.1075312082149	-0.8227510781509\\
-10.1074725342754	-0.82280299134352\\
-10.1059600237561	-0.82299303005442\\
-10.1014091050906	-0.821526922779003\\
-10.0903541444258	-0.819162681193516\\
-10.0702416467413	-0.81436649613715\\
-10.0398904269262	-0.805668475801171\\
-9.99724062479655	-0.794211617053428\\
-9.94293348932661	-0.78050343547816\\
-9.87542112098584	-0.765366863250363\\
-9.79814755837289	-0.748987646364983\\
-9.71189134743076	-0.729496272147711\\
-9.61783891972089	-0.70523442394262\\
-9.51857316705396	-0.679666670281398\\
-9.41538921923652	-0.656401819102942\\
-9.30933650656943	-0.631907081640096\\
-9.19968131870841	-0.603149421143961\\
-9.08204220678033	-0.5766202985443\\
-8.96556051031865	-0.554039647562945\\
-8.84427894723264	-0.524545632434934\\
-8.72452268553704	-0.492228342207594\\
-8.6036828041501	-0.464887831854657\\
-8.47897423876673	-0.438172660935063\\
-8.35541570036433	-0.410831293517628\\
-8.23148937956161	-0.382570068230915\\
-8.1060723174697	-0.355792027585466\\
-7.98503488745689	-0.326612810495622\\
-7.85862656266228	-0.294717625178855\\
-7.72990527034049	-0.269933314307017\\
-7.60061076093677	-0.246224951308284\\
-7.47022168501084	-0.213624004430629\\
-7.33970457175974	-0.179554385128228\\
-7.21008465001023	-0.152104942862888\\
-7.07744321200762	-0.125095787125837\\
-6.94472399395039	-0.0982988884584991\\
-6.80970248079524	-0.070102025589887\\
-6.6750762686454	-0.0384495053591518\\
-6.54194783455209	-0.00683222139056115\\
-6.40721547553335	0.0204337553849392\\
-6.27050490270175	0.0456691275951886\\
-6.13292163576028	0.0766029810009343\\
-5.99837699657556	0.111424754215601\\
-5.86748677013673	0.144453291130636\\
-5.73426589745025	0.176733407655623\\
-5.59880503415058	0.204926382270445\\
-5.46727796740665	0.233173391775473\\
-5.3345024050363	0.267177248436844\\
-5.19955261010616	0.300264439773778\\
-5.06723688059976	0.331145227824994\\
-4.93241103189932	0.357819979943867\\
-4.80206080557779	0.386437048647827\\
-4.66926192813861	0.419865890694154\\
-4.54037551038165	0.455959483957998\\
-4.40934349962551	0.486089807062471\\
-4.27951825198112	0.508104441510145\\
-4.15216850696719	0.53317338802407\\
-4.02390058217803	0.561749677692947\\
-3.89893465669151	0.586741210757513\\
-3.77232817312292	0.609911819559437\\
-3.64730686486922	0.63057577883909\\
-3.51931557954242	0.646581167805286\\
-3.38981845400159	0.664996467991612\\
-3.26127578589177	0.686657615709844\\
-3.13034495402016	0.707483996709437\\
-2.99962948851108	0.721599368641348\\
-2.86992634002333	0.728582414095892\\
-2.73721809665444	0.739332727621004\\
-2.60766425495171	0.750104828203993\\
-2.47834314750955	0.756539728066056\\
-2.34583016820923	0.75947347144413\\
-2.21732650575343	0.759675880864654\\
-2.08639429705994	0.7587403791709\\
-1.95519572601096	0.756798060750315\\
-1.82451649474835	0.754552090441267\\
-1.69464513507949	0.751119407765913\\
-1.56577809525865	0.742005415026537\\
-1.43676589061439	0.728348511043674\\
-1.30980023338614	0.717073405398076\\
-1.18393895354155	0.708697618407748\\
-1.05900737828149	0.693105873983302\\
-0.937096452559004	0.673954289009285\\
-0.820586634006304	0.656878916044376\\
-0.700953844395633	0.63994263202884\\
-0.585830832893412	0.623731815038984\\
-0.473778843305896	0.605435461652165\\
-0.367130262318813	0.58829320967983\\
-0.263429412646291	0.573460611744371\\
-0.168995701973718	0.561781275314086\\
-0.0802119712231269	0.55181782316043\\
0.00196259870201621	0.547107990632065\\
0.0793110758995115	0.545766563859418\\
0.150213475555143	0.544935410835604\\
0.213611322361117	0.547431533011075\\
0.266067827077746	0.553694631391817\\
0.308304183466038	0.558935869594833\\
0.338821048874056	0.563958085025466\\
0.360628663659536	0.568526394657454\\
0.375832969221941	0.571927512533977\\
0.386710164767274	0.5750536229208\\
0.395281977649544	0.57605526931106\\
0.400927223902961	0.577405749055991\\
0.405869362565782	0.578950461697472\\
0.409953604400958	0.578856184487236\\
0.412060476870128	0.579103870667439\\
0.412305721565966	0.579097950790085\\
0.411868954103004	0.578330509467443\\
0.411374188410234	0.577880896819786\\
0.410898846550918	0.577784099618098\\
0.410946829710643	0.577443283360592\\
0.410986751740597	0.577459783979815\\
0.411126698463146	0.577647903860697\\
0.411317826187041	0.577650165281403\\
0.411358523330154	0.577716657182636\\
0.4113869036374	0.577800906413886\\
0.411473618802465	0.57774551755352\\
0.411457393289283	0.577823465893689\\
0.41151117225884	0.577916806472097\\
0.411512455871821	0.57791705146631\\
0.411532956402455	0.577952798698104\\
0.411512049602722	0.577977223771869\\
0.411528500090786	0.577949794490501\\
0.411534938979846	0.577922610012269\\
0.411558477046665	0.577891324249795\\
0.411611464326033	0.577782915972485\\
0.411615025373077	0.577731451153643\\
0.411593328652256	0.577706999809956\\
0.411613880476581	0.577626135808754\\
0.411566748321058	0.577586560503794\\
0.411544878253334	0.577620742974867\\
0.411516624031199	0.577575032941944\\
0.411517930006304	0.577570110565303\\
0.411578399644241	0.577512287494666\\
0.411595095529217	0.577466441587206\\
0.411605680888979	0.577460304734264\\
0.411606062567954	0.577440409357378\\
0.411618295352803	0.577417624902671\\
0.411648195900969	0.577426674933932\\
0.41166685688162	0.577440946318967\\
0.411690227174573	0.577453011040394\\
0.411693321777629	0.577466942828084\\
0.411681862836751	0.577474810579435\\
0.411694768874393	0.57751574479305\\
0.411732981989668	0.577547947074371\\
0.411737680103852	0.577552268641736\\
0.411764334356131	0.57757082210455\\
0.41173056410406	0.577581112337578\\
0.411723681590242	0.577640362569583\\
0.411723409556966	0.577657376161157\\
0.411716040496342	0.577660829635377\\
0.411700747935951	0.577662833821857\\
0.411695445224028	0.577659069927428\\
0.411695898433635	0.577665173966188\\
0.411727169420266	0.57765400660157\\
0.41174603864449	0.577665567973183\\
0.411756976114258	0.57765924432787\\
0.411765538143698	0.577629629238749\\
0.411765816284174	0.577622079084756\\
0.411764107267364	0.577597206413899\\
0.4117724473484	0.577570754027016\\
0.411764865604992	0.577544520860951\\
0.411768310362079	0.577537646398383\\
0.411766536160633	0.577539535448562\\
0.411771008574618	0.577557599242415\\
0.411769404465588	0.577568390267755\\
0.411785618588373	0.577662796127751\\
0.411793892575265	0.577981746273396\\
0.411825175267394	0.578488279513507\\
0.411800122460608	0.579153413751195\\
0.412274317990363	0.580139255490988\\
};
\addplot [color=white!50!blue,solid,line width=1.2pt,forget plot]
  table[row sep=crcr]{%
0.0103269823534534	-0.0336058675918701\\
0.278498658487298	-0.489906659376576\\
0.877482397347522	-1.2162433417812\\
1.11000730294917	-1.12285185417652\\
0.236293999918745	-0.161429606872482\\
-0.00398903528474201	0.00458044359918179\\
-0.00385740055316794	0.00463326372282488\\
-0.00378745842670844	0.00465212229216835\\
-0.00380466723889792	0.00464337836162281\\
-0.00382966764209012	0.00465821760533158\\
-0.00383787697664022	0.00468236251032577\\
-0.00384248043794552	0.00467834225186829\\
-0.00384077522492551	0.00470169508887383\\
-0.00384946236348108	0.00471168825757222\\
-0.00387254211323035	0.00471093986927166\\
-0.00388459686168647	0.00469791032517133\\
-0.00393952768821206	0.00464174119004131\\
-0.00396999110396235	0.00458494269735934\\
-0.00397025529589443	0.00457708274733207\\
-0.0039937671349356	0.00460132252485147\\
-0.00403206695367829	0.00459793666087328\\
-0.00401368340922758	0.00460698551345954\\
-0.00407493572605793	0.00464842179204133\\
-0.00408405345082543	0.00464686152749979\\
-0.0040773425401991	0.00462801975499271\\
-0.00407112957845736	0.00462770866216906\\
-0.00404751360748155	0.00462495613655026\\
-0.00398271071625972	0.00461010102454707\\
-0.00397432817081999	0.00460033601526624\\
-0.00395486162021378	0.00458908591364685\\
-0.00392918227657114	0.0045530241770519\\
-0.00394033027893459	0.00455665926296297\\
-0.00394372817523522	0.00456044342262818\\
-0.00396562796727158	0.00455770978966289\\
-0.00400022824584234	0.00459357375765199\\
-0.00406390923859019	0.00458979857454027\\
-0.00408008979065157	0.00459798689507499\\
-0.00410494123442473	0.00469193265894346\\
-0.00410953438371765	0.00483049594016084\\
-0.00414441072416352	0.00494698532786592\\
-0.00426291514219672	0.005180863004369\\
-0.00434744277636828	0.00547507640433561\\
-0.00445182551047816	0.0057382764837321\\
-0.00447079997162544	0.00600000229320099\\
-0.00455528133017514	0.00643473320212453\\
-0.00434011377761509	0.00703081391044524\\
-0.00408349969143037	0.0076679441969962\\
-0.00373476819243557	0.00805618512152984\\
-0.00310378147241704	0.00850450976658792\\
-0.000465853501796735	0.00862742590026459\\
0.00573379114468453	0.00769310863618782\\
0.017440029776179	0.00676644025657672\\
0.036262727788961	0.00640137136031333\\
0.0628001087686835	0.00431494598865308\\
0.0971483125031696	0.00148482347104711\\
0.139436974381931	-0.00139977825965724\\
0.188717652239999	-0.00572593330068801\\
0.245211192463939	-0.0107666995872215\\
0.308191271098824	-0.0191115037356444\\
0.375690166993579	-0.0291507803300305\\
0.448039816446336	-0.0377020508860003\\
0.523321799784842	-0.0520622885923235\\
0.605112476786832	-0.0668278793036696\\
0.693831978255435	-0.0705586642663583\\
0.788357388141584	-0.0720730327397634\\
0.887853148368245	-0.0749211728781727\\
0.992669882742021	-0.079827886123981\\
1.10242481312888	-0.0827248583476053\\
1.21209387587477	-0.0865827816120392\\
1.32532676073339	-0.0907782716285578\\
1.43705814952128	-0.0909467407996144\\
1.55172002535251	-0.0954167799208428\\
1.6678421214769	-0.0995526061935763\\
1.78762247265378	-0.101476224601015\\
1.9114691376579	-0.106503247074947\\
2.03531366276849	-0.108471867047148\\
2.15702133646745	-0.110298077510912\\
2.28075967822679	-0.118172663664603\\
2.40314824400536	-0.126228671561624\\
2.52937687140485	-0.132620805443049\\
2.65973528618528	-0.14081810476055\\
2.78967278141663	-0.148505300733343\\
2.92001349802251	-0.15786661457934\\
3.05129880279836	-0.170629796848474\\
3.18361158148213	-0.188807725083351\\
3.31614473521121	-0.207296028652265\\
3.45266460809366	-0.219063393686493\\
3.59007662482478	-0.236662623586754\\
3.72476423480829	-0.256816941976646\\
3.86204658092791	-0.273218747900958\\
3.99775602245579	-0.287437906974197\\
4.13391667250189	-0.300916796371126\\
4.27038442283204	-0.309180740041025\\
4.40549106871838	-0.316711045457768\\
4.54439308925235	-0.322949218243574\\
4.67958215475683	-0.327373214934417\\
4.81740890649598	-0.32552881350475\\
4.95223560654166	-0.314057610776166\\
5.08319602698427	-0.296029309054474\\
5.21572093890105	-0.276239718232205\\
5.34581906602449	-0.251446013670603\\
5.4748152045607	-0.219823040933863\\
5.59805324488271	-0.177560555486797\\
5.72338393588166	-0.133175523440251\\
5.84837784940554	-0.0916173802561425\\
5.9719921888423	-0.0465523423940205\\
6.092699580403	0.00420299053820386\\
6.21480355759982	0.0588424072512339\\
6.33675104697064	0.113672137701288\\
6.45532891891286	0.16612788726184\\
6.57817959511835	0.216796117176014\\
6.6981516848721	0.266847140207214\\
6.81801958707665	0.319122968301685\\
6.93649078830183	0.37716663529876\\
7.05171812526015	0.43124505884281\\
7.16616912687205	0.479491367459102\\
7.28023960656821	0.5280045621951\\
7.39538214051076	0.578096689934452\\
7.51158687649598	0.628211360074012\\
7.62728591386424	0.679770649189998\\
7.74576347957613	0.72741861175184\\
7.86567986841481	0.770989349666054\\
7.98262068746981	0.814719093980423\\
8.10378278652501	0.856693446913163\\
8.2224433761014	0.898540169644973\\
8.33857031492353	0.939301116667228\\
8.45462224115241	0.976578035937038\\
8.57103284089245	1.00856462640523\\
8.68717211344389	1.03726951058104\\
8.80250667316199	1.06343783913703\\
8.91626415221395	1.08596587934982\\
9.02742919866997	1.09965630022079\\
9.13066332800137	1.10854974979953\\
9.230318138812	1.11573913662995\\
9.32495396522047	1.12379832593345\\
9.41272893382189	1.12788830208583\\
9.49242297402105	1.12628845966501\\
9.56529754375127	1.1243184920664\\
9.630419722529	1.12219774756709\\
9.68666619948205	1.11866413215326\\
9.7348208485529	1.11434707016548\\
9.77178400416063	1.1100150909274\\
9.79859495190742	1.10715478273062\\
9.81688333011419	1.10432000130533\\
9.82760004290614	1.10250953937898\\
9.83500486844784	1.10191906748238\\
9.83889153842181	1.10162403617489\\
9.84078768018709	1.10065694916668\\
9.84193794411857	1.10018917530425\\
9.8427332572883	1.10010508557515\\
9.84150063883675	1.10003029278021\\
9.84057519612963	1.0999872137961\\
9.83898923264658	1.09978651241706\\
9.83778077668368	1.09976796045566\\
9.83673172887119	1.09976786098175\\
9.8360287536323	1.09966396407713\\
9.83553334423682	1.09953339352975\\
9.83540899396933	1.09948024131377\\
9.83534582936293	1.09941905566496\\
9.83534941690406	1.09939607665136\\
9.8352955544401	1.09949081717442\\
9.83528817360898	1.09960301775165\\
9.83530594250004	1.09963378003098\\
9.83529693405911	1.09972066784535\\
9.83526870654892	1.09978409108941\\
9.83528526261648	1.09977595850404\\
9.83531404659161	1.09976344011887\\
9.835331403255	1.0997815556225\\
9.83530090297265	1.09974313407586\\
9.83525802543458	1.09970091508737\\
9.83523808071944	1.09967815016567\\
9.83521432632001	1.09965285755293\\
9.83517308199814	1.09966388373528\\
9.83515091713515	1.0996636245264\\
9.83511665487955	1.09966974425556\\
9.83511142713189	1.09966663514184\\
9.83507835376228	1.0996632465514\\
9.83510598333084	1.09964787986049\\
9.83510074894227	1.09964400671439\\
9.83511336273457	1.09964077563201\\
9.83510812546368	1.09962795367504\\
9.83512699328406	1.09962014634289\\
9.83510726364645	1.09960741381359\\
9.8351039496987	1.09960807794342\\
9.83510436081712	1.09958691851043\\
9.83515657144164	1.09957369910514\\
9.83515122014085	1.09961528726197\\
9.83509511842394	1.09974763082898\\
9.8349207456532	1.09998884904999\\
9.83501944206647	1.10028172888007\\
9.83521340267281	1.10081341685534\\
9.83625171710078	1.10115758222125\\
9.84066106341733	1.10129380536348\\
9.8507955029917	1.1007145787946\\
9.86872919838488	1.0989618486406\\
9.89604709958479	1.09600105576003\\
9.93339402327092	1.09381227631593\\
9.98329935786886	1.08992853901225\\
10.0424727817209	1.08363554322107\\
10.1116595816815	1.07502903798631\\
10.1899677929764	1.06268479855399\\
10.2739782995285	1.05123579892745\\
10.3623685169979	1.03795749895408\\
10.4513950974548	1.01959470235429\\
10.5427937886994	0.999448265873787\\
10.6390664516275	0.981394357803998\\
10.7359253766291	0.966123833118673\\
10.8397391471348	0.946328254416207\\
10.9470473784754	0.922567007632807\\
11.0541502728959	0.901941560964264\\
11.1665414089884	0.881954185122448\\
11.2791167244043	0.862311876284597\\
11.3934483184707	0.842076559829641\\
11.5118370272879	0.823206313786621\\
11.6296778208019	0.805698548926077\\
11.7496529486853	0.791093651059492\\
11.8712704496454	0.775824153776242\\
11.9884151184119	0.75828267722198\\
12.111421496738	0.744231298814341\\
12.237450874948	0.72974747628594\\
12.3619054049601	0.714075122393386\\
12.4889504398645	0.702141550104943\\
12.6125793383334	0.690149334432131\\
12.7381312270157	0.674973502332672\\
12.8684634598081	0.664065643830918\\
12.9970015109908	0.659367489943185\\
13.1268683452091	0.654960159754524\\
13.2570379184317	0.647824830530179\\
13.3851603359896	0.644968809514955\\
13.5164948525152	0.645561660085715\\
13.6483994615371	0.644586176844604\\
13.7779398948034	0.64768410922558\\
13.9080482886767	0.656874571832453\\
14.0378188072063	0.665113547443712\\
14.1657165449826	0.674628632944563\\
14.2945005613612	0.6932398044409\\
14.4210877886647	0.718277640010711\\
14.5481433632997	0.747920612748465\\
14.6747353721215	0.782487793444174\\
14.7963364706693	0.820910671411512\\
14.9235343488248	0.865192239368115\\
15.0464332789081	0.911758163728052\\
15.1649110344794	0.965984017465067\\
15.282321964061	1.02822141211435\\
15.3963811929504	1.09477839424237\\
15.5118325177814	1.16210978972249\\
15.621637614475	1.24076630217745\\
15.7285150179645	1.32883458681047\\
15.8298783988853	1.42119841189733\\
15.9282501642877	1.51697674536153\\
16.0245956190469	1.61143891518263\\
16.1179823237732	1.7145106662219\\
16.2029234724331	1.82553312342158\\
16.2819986890285	1.94207774642046\\
16.356533415342	2.06448612158479\\
16.4237720636097	2.18354065007523\\
16.4906211565996	2.30380032395631\\
16.5530284082725	2.42895609210531\\
16.6079870841051	2.55924272165481\\
16.6571887668819	2.69052414672927\\
16.7047665347886	2.81961245330689\\
16.7502242650944	2.94866055224558\\
16.7909300044618	3.08359435646733\\
16.8241079548833	3.22220896677998\\
16.8498068416133	3.36535755535457\\
16.8734418504796	3.507362075713\\
16.89221047749	3.6468345940874\\
16.9069125886679	3.78843393594877\\
16.9159583427947	3.93027389202389\\
16.9206398168882	4.07575757174995\\
16.9194550074382	4.21975359928867\\
16.915202053192	4.36095941289461\\
16.9069156677259	4.50312872853796\\
16.8935201443475	4.64604143718203\\
16.8783449584154	4.78947443829164\\
16.8587588001382	4.93315689507532\\
16.8368633974449	5.07386762436323\\
16.8137220016962	5.21102291606717\\
16.7854939290548	5.34959446675586\\
16.7524236044362	5.48875615783239\\
16.7169275442085	5.62585033289602\\
16.677151474563	5.76325093211606\\
16.6330004371296	5.89853673407556\\
16.586091452377	6.03141051234725\\
16.5362405420395	6.16521765776777\\
16.4811417516993	6.29565806512361\\
16.4215199710172	6.42256839835516\\
16.3628966632989	6.54884599734826\\
16.3012650814587	6.67394473307532\\
16.2369669670887	6.79929522173323\\
16.1695978033297	6.92056475115756\\
16.0980889032444	7.03851495658659\\
16.02402588382	7.15424765670514\\
15.9502400300235	7.26868781769445\\
15.872538361117	7.38078823849718\\
15.7901124605214	7.49061181232918\\
15.7064879904417	7.60165036703746\\
15.6194009209912	7.70624332361762\\
15.5318436874865	7.80866168816932\\
15.4466490263732	7.91100524696691\\
15.3592534499151	8.01089348444518\\
15.2682753848053	8.10171830526298\\
15.175913782064	8.1862070723686\\
15.087084504828	8.26678369164034\\
15.0042574297279	8.34879934720824\\
14.9192008112283	8.42288929382113\\
14.8272482280463	8.49184483626002\\
14.7364438589788	8.5598870842636\\
14.6469208771388	8.62860719477935\\
14.5574885088182	8.69015319727765\\
14.4650930687525	8.74649476275634\\
14.3744653622632	8.80259469908377\\
14.2854524077618	8.85471418535367\\
14.1957540652596	8.89966291480488\\
14.1104420180049	8.93615449831571\\
14.0289069719258	8.96791718895927\\
13.9509687903719	8.99670885664654\\
13.8777424291163	9.02167971036904\\
13.8072652098654	9.0432330365362\\
13.742135768854	9.06050700580988\\
13.6805535204942	9.07248690595578\\
13.6275383355024	9.08166739268546\\
13.5832714317787	9.09035330949624\\
13.5469628394229	9.09647947920059\\
13.5181596748408	9.09889115382639\\
13.4968852506436	9.10123218815277\\
13.481887140631	9.1035609104583\\
13.4713920511233	9.10475598988517\\
13.4645350495304	9.10451804823581\\
13.4601906491722	9.10474677886114\\
13.4581764517234	9.10533960111623\\
13.4573594579284	9.10492186205719\\
13.4573899350515	9.10431576877125\\
13.4574395185954	9.10417879483428\\
13.4571608616839	9.10385658537504\\
13.4565365329637	9.10326558702708\\
13.4555120768243	9.10296189333004\\
13.4546942723589	9.10280659866237\\
13.4539382020393	9.10247233842911\\
13.4532956017976	9.10228437412088\\
13.4524496010005	9.1020984068665\\
13.4517779293904	9.10181288728481\\
13.4512674936203	9.10164673065462\\
13.4511213352283	9.1016343766202\\
13.4510716923433	9.10134263547163\\
13.4510259612253	9.10130159656234\\
13.4510182540195	9.10130911366167\\
13.4509746321678	9.10102712315736\\
13.4509813512602	9.1010515271589\\
13.451008142975	9.10106617284986\\
13.4510554925141	9.10095518372317\\
13.4510874987485	9.10107036384587\\
13.4511060631797	9.10122663104027\\
13.4510912960347	9.10122821416034\\
13.4510429139902	9.10136893684506\\
13.4510119124884	9.10145798032054\\
13.4509670778811	9.10146287409555\\
13.4509931126791	9.10150935339909\\
13.4510015180262	9.10150693170839\\
13.4509733263684	9.10146220088453\\
13.4509606358041	9.10145116129289\\
13.450962863495	9.10145444443514\\
13.4509978934062	9.10139192144776\\
13.4509959124765	9.10136867979488\\
13.4509952983848	9.10133389098308\\
13.4510246997355	9.10130018482031\\
13.4510423015678	9.10129315815503\\
13.4510610583965	9.10130911467787\\
13.4510678391116	9.10134303728861\\
13.4510207130617	9.10135210262043\\
13.4509778007093	9.10137754201083\\
};
\addplot [color=blue,solid,forget plot]
  table[row sep=crcr]{%
13.1124836545986	9.10137754201083\\
13.1194126496551	9.02513696305272\\
13.1399159609531	8.95201768235504\\
13.1731541805189	8.88501321186071\\
13.2177665317382	8.82686672246445\\
13.2719265797104	8.77995873827408\\
13.3334170057078	8.74620967767379\\
13.3997203844655	8.72700123116991\\
13.4681222478666	8.72311979481921\\
13.5358222155868	8.73472427508193\\
13.600048642993	8.76133958317265\\
13.6581720926011	8.80187608525097\\
13.7078129835549	8.85467421215897\\
13.7469390119461	8.91757240237751\\
13.7739483535753	8.98799559660882\\
13.7877352428227	9.06306066100671\\
13.7877352428227	9.13969442301495\\
13.7739483535753	9.21475948741283\\
13.7469390119461	9.28518268164415\\
13.7078129835549	9.34808087186269\\
13.6581720926011	9.40087899877068\\
13.600048642993	9.441415500849\\
13.5358222155868	9.46803080893972\\
13.4681222478666	9.47963528920245\\
13.3997203844655	9.47575385285175\\
13.3334170057078	9.45654540634787\\
13.2719265797104	9.42279634574758\\
13.2177665317382	9.37588836155721\\
13.1731541805189	9.31774187216095\\
13.1399159609531	9.25073740166662\\
13.1194126496551	9.17761812096894\\
13.1124836545986	9.10137754201083\\
};
\end{axis}
\end{tikzpicture}%


  \tikzsetnextfilename{tikz-stroller-4}%
  % This file was created by matlab2tikz.
%
%The latest updates can be retrieved from
%  http://www.mathworks.com/matlabcentral/fileexchange/22022-matlab2tikz-matlab2tikz
%where you can also make suggestions and rate matlab2tikz.
%
\begin{tikzpicture}

\begin{axis}[%
width=\figurewidth,
height=0.444\figureheight,
at={(0\figurewidth,0\figureheight)},
scale only axis,
xmin=-25,
xmax=20,
xtick={-25,-20,-15,-10,-5,0,5,10,15,20,25},
xticklabels={\empty},
xlabel={$t = 74.2$~s},
xmajorgrids,
ymin=-5,
ymax=10,
ytick={-5,0,5,10},
yticklabels={{0},{5},{10},{15}},
ymajorgrids,
axis background/.style={fill=white}
]
\addplot [color=gray,dashed,line width=0.8pt,forget plot]
  table[row sep=crcr]{%
0.0103269389045126	-0.0336058328639577\\
0.278501789313486	-0.489906141589696\\
0.87749219437659	-1.21624250065453\\
1.11002403027368	-1.12285157681853\\
0.236314881988431	-0.1614294835538\\
-0.00397140892669651	0.00457826309944895\\
-0.00385065029903127	0.00462123317922545\\
-0.00378682509306162	0.00463463813091785\\
-0.0038074453593265	0.00462647572328424\\
-0.00383703220824572	0.00464188919059538\\
-0.00384672332672233	0.00466634399517023\\
-0.00385491589714484	0.00466420406674007\\
-0.00385581603152646	0.00468880068059153\\
-0.00386602158423475	0.0046999517858656\\
-0.00389226345218809	0.00470178959877942\\
-0.00390561150997635	0.00468993728131046\\
-0.00396259582380767	0.00463665065591376\\
-0.00399555681061864	0.0045828079544999\\
-0.00399690419136082	0.00457609661836634\\
-0.00402294171754303	0.00460315490248991\\
-0.00406255722176316	0.00460156446717354\\
-0.00404601777895617	0.00461232105548733\\
-0.00410903777507603	0.00465619072567736\\
-0.00411891627082658	0.00465566564470737\\
-0.00411466731667848	0.00463970759044145\\
-0.0041101756396055	0.00464126795239423\\
-0.0040881330043217	0.00464008907172103\\
-0.0040263407232255	0.00462807605408912\\
-0.00401890938999203	0.00461943082647761\\
-0.0040016449646977	0.0046107413778973\\
-0.00397817267439755	0.00457726894503156\\
-0.00399019939068386	0.00458205620001794\\
-0.00399617666118063	0.00458868824217421\\
-0.00401903425636391	0.00458731642471325\\
-0.00405500677721846	0.00462496375715975\\
-0.00412045863549711	0.00462373856900511\\
-0.00417734900331046	0.00466708119920671\\
-0.00442110836114524	0.00494887558079991\\
-0.00483164894518048	0.00543494957027028\\
-0.00545618745747107	0.00605545558018452\\
-0.00634452938695394	0.00694504523098606\\
-0.00737688917409191	0.00804368875799442\\
-0.00860297217670184	0.00925691801854914\\
-0.00991584254575038	0.0106109272799758\\
-0.0114619641539311	0.0122747472614617\\
-0.012876754947015	0.0142345292795463\\
-0.0144123590696398	0.0163660619435325\\
-0.016016773937082	0.0183798694929277\\
-0.0174920030375568	0.0205746985111592\\
-0.01713053090629	0.0225791052846711\\
-0.0133756136000719	0.0236659905310159\\
-0.00430034222882562	0.0249018479511279\\
0.0116939104720748	0.0268415903980102\\
0.035216378482335	0.027217648259604\\
0.0663588597466251	0.0269971875194111\\
0.105248359618161	0.0268590607571381\\
0.150953446947431	0.0254118910735926\\
0.203690816827192	0.0233676680131207\\
0.262752420524217	0.0181427652109263\\
0.326183145096281	0.0113128131382109\\
0.394281186545766	0.00603436750471615\\
0.465208355637247	-0.0049524111812778\\
0.542475855831044	-0.0162371445294915\\
0.626418161910732	-0.0164332889858583\\
0.715999174540469	-0.0142951909911941\\
0.810423787520278	-0.0133483084101371\\
0.910048572775709	-0.0143056595945424\\
1.01445484164042	-0.0131329718326544\\
1.11869572779703	-0.0128303357868091\\
1.22638179816935	-0.0127447592178104\\
1.33245697238448	-0.00859811107491129\\
1.44139658382791	-0.00861634454747132\\
1.55170928359023	-0.00821617455881603\\
1.66555497277893	-0.00551180674490376\\
1.78340033580841	-0.00577699096982132\\
1.901138076309	-0.00294694200322526\\
2.01668419261958	4.74608163979207e-05\\
2.13424336773358	-0.00289951207851205\\
2.25039699582256	-0.00600697309026296\\
2.37027803990069	-0.00739180633388667\\
2.494219637974	-0.0105013807417679\\
2.61768245151388	-0.0131071271166303\\
2.74151639243623	-0.0173824739257015\\
2.86627502596162	-0.0250415154272412\\
2.99207777505397	-0.0380910010501901\\
3.11806537028834	-0.0514775007847373\\
3.24786623300026	-0.0581713494736401\\
3.37860174093443	-0.0706888677274701\\
3.50663343025307	-0.0858279157182961\\
3.6371536325667	-0.0972472701366352\\
3.76605890425648	-0.106563108701319\\
3.89537771065028	-0.11516815041483\\
4.02490455420131	-0.118617560728042\\
4.15306191359771	-0.121380507913208\\
4.28495039441583	-0.12280845931653\\
4.4131272329558	-0.122493781285735\\
4.54380628893909	-0.115886035583055\\
4.67134031301862	-0.0997260376421804\\
4.79492932462867	-0.0770566470880049\\
4.92004815767136	-0.0525229574925304\\
5.0426749734133	-0.0229559031667613\\
5.16408758218846	0.0135039730871678\\
5.27958952261709	0.0605647647234617\\
5.39712948270113	0.109928253414812\\
5.51440532422004	0.156636659937874\\
5.63024525834151	0.20697215965992\\
5.74309423429395	0.263082195171674\\
5.85725319290907	0.323262431934729\\
5.9712721377629	0.383811827257659\\
6.08201915250973	0.44211433638803\\
6.19706309240174	0.498910822812343\\
6.30929517032533	0.555214697521271\\
6.42140585239913	0.613911339831118\\
6.53203430691706	0.678487119498162\\
6.63956848309798	0.739226866066907\\
6.74650232563251	0.794323534881919\\
6.85309434734913	0.849836381500601\\
6.96075995977918	0.907103436471046\\
7.06953792455724	0.964564173989664\\
7.1778458103267	1.02359028848562\\
7.28907478486214	1.07892909987716\\
7.40190486770525	1.13037432258513\\
7.51186625035077	1.18201896066502\\
7.62614565327529	1.23213010400705\\
7.73803908079967	1.28214759784659\\
7.84754893889926	1.33110411220787\\
7.95718093846889	1.37666208488657\\
8.06741897367414	1.41702028096296\\
8.17759939794053	1.45414195385857\\
8.28716769594472	1.48873161254696\\
8.39539077216543	1.51964783003731\\
8.50140213968702	1.54167879076829\\
8.59979744254983	1.55867433725474\\
8.69481663401933	1.5738051403289\\
8.784964228022	1.58956612322886\\
8.8685487162298	1.60107709520195\\
8.94441052805033	1.60657467304933\\
9.01365080766615	1.61135958368202\\
9.07533631364789	1.61560807904911\\
9.12839056481216	1.61801949950707\\
9.17357210792354	1.61922697320233\\
9.20778241488074	1.61989634941204\\
9.23201760994683	1.62151180789206\\
9.24793094279027	1.62269187487398\\
9.25644762052972	1.62444037425635\\
9.26178864171396	1.62707750843622\\
9.2637814118162	1.62967363844485\\
9.26397466196395	1.6313098046899\\
9.2635695515283	1.63318376510264\\
9.262959924607	1.635189055653\\
9.26050055343164	1.63689720348422\\
9.25851009450988	1.63839982771162\\
9.25603876758261	1.63947419197701\\
9.25410853801571	1.64048720703006\\
9.25251175575484	1.64126541790145\\
9.25144091434786	1.64168957205983\\
9.25075900814933	1.64182765209988\\
9.25060876187831	1.64181238637562\\
9.25054607107374	1.64175249837659\\
9.25054901418395	1.64173185986774\\
9.25049082296123	1.64182573584587\\
9.25047715367717	1.64193999776716\\
9.25049273453673	1.64197231114744\\
9.25047853723937	1.64206090360196\\
9.25044625545543	1.64212533190375\\
9.2504619668665	1.64211910848319\\
9.25048867569463	1.64211080663775\\
9.2505041509171	1.64213091287724\\
9.25047362040631	1.64209391588056\\
9.2504305096644	1.6420532571652\\
9.25041069212886	1.64203097837218\\
9.25038582956779	1.64200796861395\\
9.25034292616562	1.64201940788822\\
9.25031957660305	1.64202003913031\\
9.25028294046754	1.64202791542441\\
9.25027687059844	1.64202572486801\\
9.25024209283712	1.64202386365655\\
9.25026826823144	1.64201231824038\\
9.25026225278583	1.64200957899892\\
9.25027250356363	1.64200994606289\\
9.250266681071	1.64199846419118\\
9.25028391742417	1.64199377361963\\
9.250262538289	1.64198316119434\\
9.25025835934023	1.6419847822127\\
9.25025746579413	1.64196672649488\\
9.25030857611288	1.64195765163767\\
9.2503407913386	1.64196075671332\\
9.25040702901152	1.64196541165326\\
9.2504425779603	1.6419829120317\\
9.25084140896116	1.64196888923235\\
9.25142224312645	1.6420982248462\\
9.25295049998587	1.64197541077519\\
9.25794062267128	1.64168654480054\\
9.26875705801955	1.64082791471881\\
9.28748421412418	1.63903325451586\\
9.31570186729328	1.63633575742149\\
9.35396600314039	1.63472801458212\\
9.40490276244721	1.63186227654239\\
9.46528024470586	1.62687575684683\\
9.53583635510757	1.61988559362491\\
9.61575326106376	1.60943020059734\\
9.70142261796248	1.59995110510971\\
9.79165363077104	1.58866393397526\\
9.88288032407644	1.57214643138389\\
9.97667942553024	1.55376669675814\\
10.0753550668132	1.53748938715095\\
10.1746166866511	1.52379758706141\\
10.2811441050324	1.50571652205691\\
10.3914736754222	1.48363672257644\\
10.5015929947603	1.46444522668809\\
10.6170797217091	1.44591301600615\\
10.7328829802264	1.42751369200134\\
10.8506109657954	1.40838773126901\\
10.9724595220891	1.39059427770646\\
11.0938570582468	1.37390642319482\\
11.2173970130595	1.35998018921816\\
11.3427713448625	1.34524118135857\\
11.4639933616538	1.32779595678834\\
11.5910507422516	1.31387382080072\\
11.7213136582956	1.29945380774123\\
11.8502670655447	1.28352742493704\\
11.9817855179115	1.27122542482536\\
12.1101041389456	1.25846511801565\\
12.2406977629523	1.24241144267876\\
12.3760231398966	1.23062705517877\\
12.5094470523416	1.22470657715516\\
12.6443884126553	1.21891990739537\\
12.7799924653766	1.21021658150317\\
12.9135539816107	1.20544703617319\\
13.0503484284216	1.2040649752512\\
13.1880208296483	1.20094376389081\\
13.3233546182681	1.20154670766975\\
13.4591584175667	1.20803829281434\\
13.5949225333203	1.21337690731593\\
13.7290017070322	1.21971593515107\\
13.8637052736797	1.23499698461786\\
13.9961167336946	1.25641481543602\\
14.1289896528746	1.28231577306095\\
14.2613825687776	1.31298227887303\\
14.3888589382315	1.34711683513207\\
14.5218297200331	1.38730428674016\\
14.6506637318569	1.42947188474825\\
14.7749227378466	1.47896088429781\\
14.8979205086719	1.53632400878001\\
15.0175958212671	1.59778123568317\\
15.1388621562201	1.66010091317191\\
15.2541057479562	1.73338289611146\\
15.3661098721842	1.81591701898454\\
15.4726053144358	1.90246607430537\\
15.5761435613946	1.99231153941791\\
15.6779563101508	2.08082868158116\\
15.7765085663889	2.1778457782653\\
15.8663674860024	2.2823986322891\\
15.9502337857409	2.39223538213668\\
16.0293935866165	2.50777916807673\\
16.1016575751607	2.61970827060563\\
16.1736060737978	2.73298432208145\\
16.2409683559766	2.85103786833708\\
16.3007253274328	2.97392123108219\\
16.3548070052648	3.09763326325894\\
16.4075128789106	3.21926314213346\\
16.4582030023846	3.34093451822254\\
16.5038693056004	3.46837811293086\\
16.5418880941823	3.59922776718389\\
16.5722135466436	3.73435679091016\\
16.6005950308892	3.86844348129196\\
16.6243232743927	3.99993935970328\\
16.6438840301607	4.13352987432718\\
16.6578062827302	4.2672406813032\\
16.6671445418413	4.40453695274579\\
16.6707293688651	4.54021929959789\\
16.6714237693954	4.67317501664795\\
16.6680142047858	4.80707109565718\\
16.6594395074225	4.94161952399884\\
16.6490210993222	5.07681985493497\\
16.6341501128229	5.21222208217043\\
16.6171241700261	5.3447591800358\\
16.599046471706	5.47392171946008\\
16.5757071775535	5.60441024190898\\
16.5474522102481	5.73539330494972\\
16.5168159363755	5.86440988993181\\
16.4817942863559	5.99367715132427\\
16.4424370585272	6.12078599878202\\
16.4003936556137	6.24554565293671\\
16.3552283870017	6.37126594018821\\
16.3049360108457	6.49351014483119\\
16.2502437901382	6.61216001143138\\
16.1964417707759	6.73048170508476\\
16.1395713312637	6.84764935884045\\
16.0798645973678	6.9651023948386\\
16.0172291087945	7.07849610858098\\
15.9505127646269	7.18850495793284\\
15.8812228108286	7.29633318991668\\
15.8121036747558	7.40310464573685\\
15.7390468831284	7.50746755072875\\
15.6612373069975	7.60942533381729\\
15.5819264635917	7.71269795820961\\
15.4994093431756	7.80948903731251\\
15.4163499757416	7.90426488631477\\
15.3354081887654	7.99931549429373\\
15.2522039207761	8.09192353506612\\
15.1658398053246	8.17540708716605\\
15.0783006558137	8.25262887379566\\
14.9943006490147	8.32636567717311\\
14.9158959910506	8.40212137030252\\
14.8355797153284	8.46994959045886\\
14.7485005420541	8.53226633046858\\
14.6623213745983	8.59387746633567\\
14.5770863861545	8.65637085241178\\
14.4921665033328	8.7118226441906\\
14.4043721444894	8.76195170161644\\
14.3180314666358	8.81205086601585\\
14.2332685342467	8.85837235624441\\
14.1480297032657	8.89754032165848\\
14.0674615318587	8.92865627301121\\
13.990664695405	8.95537233922407\\
13.9173182865316	8.97941381372292\\
13.8485934900534	9.00000784606318\\
13.7824981946652	9.01737996819088\\
13.7216823487106	9.03086723942614\\
13.6644283661557	9.0392976381526\\
13.6155394905903	9.045543459099\\
13.5750175051401	9.05190790379228\\
13.5422303171644	9.05625822861468\\
13.5168203652657	9.05740145429537\\
13.4985305433862	9.05897005644958\\
13.486100148344	9.06091618496389\\
13.4778586150701	9.06197859041248\\
13.472971041419	9.06178902371255\\
13.4701643542022	9.06214907169022\\
13.4692672601274	9.06294233525307\\
13.4692647281559	9.06270572382741\\
13.4697425884381	9.06223065048548\\
13.4698598556491	9.06211659247461\\
13.4695992222452	9.06177361082215\\
13.4690065150716	9.06113498955688\\
13.4679828073941	9.06074996503337\\
13.4671463719493	9.06053084160427\\
13.466381654897	9.06013864191982\\
13.4657147340284	9.059902042399\\
13.4648436009525	9.05965000212206\\
13.4641549187424	9.05931312021426\\
13.4636186597998	9.05910832051968\\
13.463435956788	9.059087183208\\
13.4633782956286	9.05879495670787\\
13.463309534724	9.05875224784219\\
13.4632825622388	9.05876049318771\\
13.4632542318601	9.05847631100894\\
13.4632574240701	9.05850125610002\\
13.4632798842914	9.05851826951383\\
13.4633326879677	9.05841193930063\\
13.4633507023809	9.05852965459937\\
13.4633533856229	9.05868708451989\\
13.463338011982	9.05868738186939\\
13.4632763117668	9.0588234716843\\
13.4632370495206	9.05890952860264\\
13.4631911559879	9.05891045140912\\
13.4632119968526	9.05895915435102\\
13.4632202044357	9.05895753678758\\
13.4631951054619	9.0589104926545\\
13.4631827425723	9.05889840908532\\
13.4631840744636	9.05890189894758\\
13.463223597501	9.05884286207664\\
13.4632233103903	9.05881956845674\\
13.4632249020451	9.0587849155505\\
13.4632563350171	9.05875407281572\\
13.4632740588249	9.05874876042301\\
13.463290278308	9.05876647895255\\
13.4632934494669	9.0588009144232\\
13.463244753047	9.058805766478\\
13.4631985206491	9.05882744805644\\
13.4633187485655	9.05875086159461\\
13.4629224621753	9.05910446440041\\
13.4602393949048	9.05979969159865\\
13.453119412728	9.0599315919168\\
13.4394481529651	9.0600539939569\\
13.4176572548676	9.06140228116975\\
13.3855356678185	9.06152347168285\\
13.3427863081361	9.05882369351287\\
13.2900122036728	9.05528958042339\\
13.2270463509498	9.05067041372384\\
13.1557740775883	9.04355539889044\\
13.079793433591	9.02982126899012\\
12.9985593687419	9.0096271048649\\
12.9136038296545	8.98896282588228\\
12.8269211538492	8.96272112904956\\
12.7392505481338	8.92870659726051\\
12.6483443244265	8.88913483368786\\
12.5551277967114	8.8453839727284\\
12.4602598444806	8.79384463813354\\
12.368261454597	8.73488059989262\\
12.2769914099883	8.67289609875325\\
12.1854488622084	8.61078106503657\\
12.0950382642891	8.53977533686385\\
12.0055266719867	8.4618942129087\\
11.9171486901297	8.37953326175558\\
11.8323147944693	8.29156204626061\\
11.7493754347791	8.20230756382731\\
11.6666391298933	8.11259071088053\\
11.5841763463833	8.01789382123911\\
11.5041419524301	7.92244229738386\\
11.4204859222238	7.82608896581356\\
11.3348734797012	7.73250866237616\\
11.2508445214749	7.64337198521719\\
11.1643936685241	7.55731104731318\\
11.0702886265564	7.47116382135284\\
10.974376735735	7.3886533327033\\
10.8799067874887	7.30504465131841\\
10.7814575749445	7.22171309585684\\
10.6779886551929	7.1468890634001\\
10.5722784462175	7.07047287062497\\
10.4630711259327	6.99326334892379\\
10.3492209550645	6.92237204082993\\
10.232294651634	6.8565369227511\\
10.1155537476475	6.79075188756224\\
9.99794506287057	6.72692567508149\\
9.87675086212485	6.66680797579755\\
9.75318685163088	6.61132365226119\\
9.62710413553661	6.55924465823375\\
9.50002843380128	6.51346701904514\\
9.37306766025639	6.46798400768186\\
9.2429874543177	6.42327699225196\\
9.10910463922886	6.38644740660042\\
8.97411312852411	6.35246092163036\\
8.83572766775398	6.31803465473672\\
8.70070913531952	6.28463135099351\\
8.56511876103578	6.25751710129093\\
8.42517920363083	6.23455786676026\\
8.28450625312795	6.21184679637219\\
8.14841722489797	6.19273341884587\\
8.00783753631045	6.17500188583927\\
7.8700977327396	6.15628579188258\\
7.73462237631482	6.14200703177832\\
7.59351196492898	6.13131609534772\\
7.45443487026879	6.12179950239596\\
7.3165978014135	6.1077792905897\\
7.17788348162871	6.09499295653531\\
7.03799861313196	6.08765367586544\\
6.89814751742361	6.08066907311093\\
6.75799565673137	6.07188642492231\\
6.61588169929492	6.06479581277499\\
6.47643913324609	6.05801415554759\\
6.33715302940159	6.04866167845097\\
6.19348066443017	6.04619991369662\\
6.05211156379989	6.04396746032669\\
5.90967716316949	6.03637408171243\\
5.76364751818759	6.02600508616127\\
5.62051026521014	6.0203496582249\\
5.47586615361119	6.01954642140475\\
5.32892109662997	6.0184431180986\\
5.18120115938864	6.01437710539688\\
5.03394643064646	6.00777154017654\\
4.8846774925337	6.00250673343391\\
4.73731409082565	6.00169567975507\\
4.59125046057075	6.00345759339506\\
4.44208012516938	6.00025000478072\\
4.29427658547132	5.99289992721384\\
4.14739662157714	5.98547883344516\\
4.00104235541819	5.98475687641608\\
3.85466139537104	5.98468971827389\\
3.70736263562793	5.97969799703192\\
3.55797294375675	5.97469035574344\\
3.40939957048286	5.97369909083288\\
3.26442632841279	5.97131933068691\\
3.11781874859768	5.97145036874881\\
2.97083938837511	5.96788518397652\\
2.82611507291501	5.96023044230304\\
2.67756854634337	5.95583235656968\\
2.53323131626212	5.95475309078086\\
2.39250319765962	5.95177296756577\\
2.24907031077274	5.94288561248268\\
2.10869794290739	5.93520992508363\\
1.97173137770089	5.93073242360556\\
1.83538434261827	5.92710041673153\\
1.70172976378748	5.92142653239001\\
1.56923227070891	5.91543765807853\\
1.4394624182375	5.90656192779744\\
1.31287553384125	5.90046380607169\\
1.19106254580116	5.89501382142563\\
1.0764856646797	5.88882617154235\\
0.962545612598274	5.8852006299273\\
0.855220834226946	5.87988391307629\\
0.756708871438657	5.87129866708636\\
0.664275347399113	5.86412998994399\\
0.578846594783561	5.86038430256742\\
0.504191956376299	5.85625558987713\\
0.436687600722196	5.85165076641582\\
0.380113002376999	5.84822787364657\\
0.332739434383852	5.84549030580905\\
0.292993356782038	5.84303365367697\\
0.263604738789397	5.84072293432662\\
0.240737165210177	5.83899378364481\\
0.2253677489733	5.83803222403294\\
0.214270074977396	5.83720034287128\\
0.206170322405864	5.83653609641065\\
0.200405729393371	5.83661590580498\\
0.196473622057645	5.83673711580015\\
0.194240422674213	5.83628573337145\\
0.193208701278066	5.83634211391775\\
0.192878024572207	5.83668569998948\\
0.192778691529331	5.83656872118827\\
0.192657261505016	5.83655054464418\\
0.192414638098174	5.83671562126093\\
0.192197641576818	5.83657164122002\\
0.192027646031834	5.83637716781703\\
0.192089410119591	5.83641880118548\\
0.192193632031215	5.83632839784038\\
0.192265271841128	5.83622691220537\\
0.192220794151321	5.83628379152211\\
0.192218028862463	5.83626702611131\\
0.192223693482835	5.83626109631625\\
0.192206797203417	5.83632699495828\\
0.192162558853579	5.83635327173616\\
0.192138277180388	5.83634079055766\\
0.192089069417325	5.83639207149698\\
0.192056367511352	5.83639269507781\\
0.192027256601988	5.83634212925286\\
0.192021666224881	5.83632969874173\\
0.192027814705353	5.83630908418187\\
0.192039891841005	5.83626165089078\\
0.19199409826531	5.83625955581342\\
0.192017118600983	5.83626081777702\\
0.19205382235145	5.83626385289706\\
0.192065393916505	5.83626106949741\\
0.192094363211912	5.8362383461099\\
0.192091925462566	5.83624180024431\\
0.192091327218238	5.8362433075822\\
0.192065659427502	5.83621889398323\\
0.19205628157254	5.83620775694055\\
0.192021351830379	5.83622880567102\\
0.191994890485469	5.83625864898512\\
0.192033748418168	5.83628123872977\\
0.19203080493265	5.83627702984269\\
0.192028040969289	5.83629153744254\\
0.191977744613689	5.83628773607425\\
0.191988544903619	5.83627653310999\\
0.19197579785653	5.83628721900115\\
0.191959675810059	5.8363037829727\\
0.191969658825357	5.83631029491055\\
0.191983089094947	5.83633946692264\\
0.19201818666682	5.83634548065676\\
0.192011862734825	5.83634327834629\\
0.191982988146857	5.83642662911672\\
0.191955138864175	5.83646416071095\\
0.191970744467447	5.83647541624309\\
0.192019873689103	5.83654872061129\\
0.192031763210394	5.83656331717799\\
0.192034611132813	5.83647132811937\\
0.192072949418045	5.83645145042756\\
0.192081903725611	5.83641442006252\\
0.192077982749621	5.83636976581086\\
0.192067103188284	5.83636698412639\\
0.192073817942831	5.8363523401119\\
0.192053987030156	5.83632781247386\\
0.192067730000451	5.83633205085277\\
0.192087667223252	5.83634812697968\\
0.192146931393284	5.836317025711\\
0.192133408993539	5.83634257771984\\
0.192180920583757	5.83649727926139\\
0.191528800174748	5.83650130580017\\
0.190267261436765	5.8362086433566\\
0.184637368496135	5.83683128782321\\
0.17250208292248	5.83608579522434\\
0.150973757157279	5.83344780477957\\
0.117461507083797	5.83023788496557\\
0.0721609589167941	5.82693513152253\\
0.0168377300820675	5.820953289644\\
-0.0513025638468028	5.81397376929278\\
-0.128458052723002	5.80700320285275\\
-0.214047046161554	5.7995435793628\\
-0.307222273019431	5.79027646031215\\
-0.408082305316133	5.78019743038333\\
-0.513130132125113	5.77032489679014\\
-0.62624813643627	5.7595454928174\\
-0.744592188718582	5.74551523174158\\
-0.869426618997852	5.73413044497448\\
-0.997422927103059	5.72528965045686\\
-1.12455751606655	5.71458499294802\\
-1.25091535988869	5.70493695264556\\
-1.38007133869957	5.69477822438606\\
-1.50938597127305	5.68297042760668\\
-1.63877619102581	5.67213449969741\\
-1.77464319243149	5.66169321364343\\
-1.9128440399898	5.65545559801821\\
-2.05116389407095	5.64730885946172\\
-2.19148727017903	5.63651656410109\\
-2.33421964904621	5.62611336130674\\
-2.47302745144498	5.62065137434931\\
-2.61230192404688	5.61697352648635\\
-2.75267693996776	5.60709281430635\\
-2.88956785144703	5.59673947129324\\
-3.0291517694222	5.59027970090562\\
-3.16973235409861	5.58527627299929\\
-3.31125306700997	5.57730772478058\\
-3.45358317387205	5.56873712655363\\
-3.59622509970132	5.56102894766276\\
-3.73954316637866	5.55317146377811\\
-3.88352290170339	5.54445484549892\\
-4.02615743464784	5.54175633884196\\
-4.16976407797467	5.5358000083324\\
-4.31126037333373	5.52482495120206\\
-4.45891165491369	5.51876873704534\\
-4.60420434774172	5.51612632683563\\
-4.7484472952641	5.5132058764837\\
-4.89548342867606	5.50193233286841\\
-5.04348659046376	5.49207674232509\\
-5.18811845931405	5.48713873449133\\
-5.33679053036774	5.48548079593594\\
-5.48485969228556	5.48189137424147\\
-5.63204579328475	5.47343881051361\\
-5.77969828136424	5.46291974589439\\
-5.92620755101303	5.45827307612996\\
-6.07164935625616	5.45453637491025\\
-6.22216757636865	5.45205008455213\\
-6.37200116630795	5.44575908646512\\
-6.51969615217807	5.4353132705985\\
-6.66950617255401	5.42766597850916\\
-6.81735286420405	5.42490765843028\\
-6.96578395920413	5.42291037650871\\
-7.11748348836652	5.41440348605005\\
-7.26970403303207	5.40098260200454\\
-7.42089698610081	5.39567723311886\\
-7.57236446615009	5.39511448272837\\
-7.72500227544779	5.38898205393461\\
-7.87744282370555	5.37918988851715\\
-8.03306151445366	5.37265793336039\\
-8.18523609368713	5.3707832118316\\
-8.33493468519019	5.36643076470396\\
-8.48816429268027	5.35906957622685\\
-8.63958134138863	5.35103266803356\\
-8.79237188600173	5.3463056105526\\
-8.94448495367411	5.34006091328896\\
-9.09289772626449	5.33472682445092\\
-9.24493678963999	5.32491240559329\\
-9.39975994652957	5.31393306662686\\
-9.54986708871186	5.30731784759526\\
-9.70060437925911	5.30177661081584\\
-9.85271657336519	5.29191092904254\\
-10.0007957114571	5.28118576503335\\
-10.1525685096271	5.2710908288637\\
-10.3008884678805	5.26166382031407\\
-10.4456696699597	5.25008883659091\\
-10.5923503846985	5.2341162273825\\
-10.7378716566918	5.2173363980478\\
-10.8824282657979	5.20450395657684\\
-11.0272756689425	5.18818434437143\\
-11.1693334882543	5.16700812008829\\
-11.3094887196801	5.14473382907937\\
-11.4487183358903	5.11999713041633\\
-11.5917949385453	5.0961900756865\\
-11.7325106113252	5.07157742248859\\
-11.8736934490823	5.0417857832085\\
-12.0138917930326	5.00374397838384\\
-12.1499519656331	4.96455073496946\\
-12.2858838840365	4.92161942429734\\
-12.4206352522558	4.87503858304951\\
-12.5528302932694	4.8275956344368\\
-12.6800743159332	4.76884219284885\\
-12.8040806552427	4.70743218967958\\
-12.928706828419	4.64522942057427\\
-13.0504853872525	4.57835722791146\\
-13.1691507148802	4.50864205386596\\
-13.2865857196511	4.43132275178574\\
-13.3997784996304	4.35192135961469\\
-13.5090610362673	4.27029395248132\\
-13.6164539580017	4.1828486701697\\
-13.7224359104157	4.0943487687301\\
-13.8247141158087	4.00480284311215\\
-13.9273702145182	3.91458526933169\\
-14.0253198230477	3.82488516976469\\
-14.1251649602036	3.73678743433979\\
-14.2228014060626	3.6473091620237\\
-14.3211684933456	3.56081159769459\\
-14.4191429533553	3.47436563448848\\
-14.518138722825	3.39506606114366\\
-14.6227284137722	3.32159219671099\\
-14.7287344659322	3.2502198537676\\
-14.8319773033408	3.18182123030033\\
-14.9394154185512	3.11451851939459\\
-15.0531960209555	3.0532189166384\\
-15.1628592525349	2.99419076554395\\
-15.2776844382139	2.93838249338209\\
-15.3931296796979	2.88734731506338\\
-15.5073712304988	2.83954337827642\\
-15.620894795901	2.79316573975845\\
-15.7321162491215	2.75511248432896\\
-15.8396714345892	2.72539150636919\\
-15.9446292741432	2.69522901740752\\
-16.0492503237197	2.66834475834248\\
-16.1502624126818	2.64547359040358\\
-16.2468427151435	2.6235972675644\\
-16.3377466511541	2.60661381423737\\
-16.4253224098355	2.59531762795035\\
-16.5048692209586	2.5844331269779\\
-16.5793164795662	2.57431297617609\\
-16.647125246511	2.56882631616587\\
-16.7062349640419	2.56597558618037\\
-16.7552467221284	2.56112665655648\\
-16.7942569923228	2.5578968713733\\
-16.8230189687621	2.55769051742551\\
-16.8432141136712	2.55646306868059\\
-16.857105351206	2.55538628559035\\
-16.8667224981701	2.55676500952284\\
-16.8739337510723	2.55808592815465\\
-16.8794841771865	2.55873660855177\\
-16.8833345275572	2.55994573257958\\
-16.8860381783037	2.56159095576183\\
-16.8874470831396	2.56329929908368\\
-16.887742193971	2.56495738393393\\
-16.8879815107101	2.56663989064887\\
-16.8878845192262	2.56829405770246\\
-16.8880842839073	2.56975949997548\\
-16.8883507113728	2.57109798816129\\
-16.8887934888046	2.57216656605684\\
-16.8891737910379	2.57336281444146\\
-16.8896233114219	2.57454255101965\\
-16.8897521150168	2.57541908171845\\
-16.8898322839442	2.57623821335329\\
-16.889745160578	2.57709216514254\\
-16.8897442289093	2.57762692538957\\
-16.8897034596198	2.57805129613497\\
-16.8897591520427	2.57840344964781\\
-16.8897233185884	2.57851518831848\\
-16.8897614941948	2.57859236698521\\
-16.8897185722177	2.57870193980281\\
-16.889689939499	2.57871789819091\\
-16.8896449994401	2.57874705466066\\
-16.8896754536475	2.57879635440201\\
-16.8896766393635	2.57877576857469\\
-16.8896583187586	2.57877298661095\\
-16.8896377602304	2.57878999222955\\
-16.8896265823937	2.57879970719254\\
-16.8896293364579	2.57877682650643\\
-16.8896218298523	2.57877743566601\\
-16.8895749449707	2.57876963696298\\
-16.889568898678	2.57876408544163\\
-16.8895902885903	2.57875709820867\\
-16.8896312926592	2.57873361775927\\
-16.8896484467459	2.57874356071093\\
-16.889655922689	2.57873572490657\\
-16.8896314319052	2.57868633506163\\
-16.8896681133853	2.57866040258414\\
-16.8898006130585	2.57858669202766\\
-16.8905756391323	2.57860393229322\\
-16.8923151681515	2.57869193191294\\
-16.8976956821392	2.57792808312864\\
-16.9089862693849	2.57707605518932\\
-16.9279349436601	2.57446329655217\\
-16.9560660014578	2.56949065634764\\
-16.9938234652911	2.56411948552449\\
-17.0413227678388	2.55790224956308\\
-17.0979729758774	2.54937983738815\\
-17.163316582185	2.53771258249646\\
-17.236820215967	2.52516431832639\\
-17.3177174460584	2.51212577728499\\
-17.4045342255223	2.49939873788995\\
-17.4960047621209	2.48611764614644\\
-17.5922447753565	2.46904708338596\\
-17.6945617193809	2.44936428175797\\
-17.8016481820209	2.4320320483926\\
-17.9110894890666	2.41520852246721\\
-18.0207249786857	2.39501127319995\\
-18.1294537474674	2.37010772708401\\
-18.2421361564129	2.34055834309476\\
-18.3559655053926	2.30990135300324\\
-18.4703220892696	2.27353249844683\\
-18.5848908546111	2.22942867115659\\
-18.6974867811721	2.18267405791444\\
-18.8064264881713	2.13398232752445\\
-18.912036185135	2.0784557241892\\
-19.0102327925961	2.01269594121921\\
-19.1076564344779	1.94137177644199\\
-19.2046480984412	1.86602811880957\\
-19.294819318223	1.78341283313786\\
-19.3774851559884	1.69399272739932\\
-19.4556186300384	1.60167559891559\\
-19.531481552923	1.5024486672399\\
-19.5995506585858	1.39988011747814\\
-19.6611794737827	1.2911806631594\\
-19.7123338080165	1.1761398314181\\
-19.7559106581905	1.06126854877971\\
-19.7993214917541	0.945648376129782\\
-19.8380073543923	0.828664589022003\\
-19.8647525503513	0.708852225043078\\
-19.8840766184833	0.58580674281244\\
-19.8985301535263	0.4630603253744\\
-19.9095625966528	0.341442286980434\\
-19.916202556449	0.219538583101421\\
-19.9152658473629	0.094345081587789\\
-19.9071180124539	-0.0307132724639111\\
-19.8933263399812	-0.153353247569208\\
-19.8740846394168	-0.277544868229196\\
-19.8502184821226	-0.398750500483338\\
-19.8246890817996	-0.518765606516356\\
-19.7913519001527	-0.638980069221455\\
-19.747278820008	-0.75376273748963\\
-19.6967323269565	-0.866942735599506\\
-19.6463027356185	-0.980302871433052\\
-19.5916581392254	-1.09001016901715\\
-19.5298989865015	-1.19716369971428\\
-19.4604823085608	-1.30033691019711\\
-19.3839304724689	-1.39791132193262\\
-19.3044885071713	-1.492624749267\\
-19.2212673352365	-1.58110704241762\\
-19.1323906540324	-1.66583385134984\\
-19.0408598089465	-1.74863058597174\\
-18.9452094517538	-1.82322922646955\\
-18.8430146969619	-1.89352053342814\\
-18.740187181012	-1.95811644580365\\
-18.6337407576425	-2.02030779177414\\
-18.5220644489562	-2.07655707410579\\
-18.4075035849489	-2.12574070247495\\
-18.28852684374	-2.16929811640854\\
-18.1662022858575	-2.20642072981896\\
-18.0455901591807	-2.23526895178916\\
-17.9194376441778	-2.26067671286219\\
-17.7925301616184	-2.28022445648651\\
-17.6634004577104	-2.294191933974\\
-17.5350112610869	-2.29947189390461\\
-17.4070907452283	-2.29934676625181\\
-17.2790007589517	-2.2965556182463\\
-17.1519960874774	-2.29280075365218\\
-17.0195330490404	-2.28814496882084\\
-16.8905516601539	-2.27591249639827\\
-16.7604968952462	-2.25963393271291\\
-16.631291432649	-2.24108795695927\\
-16.500622093709	-2.22290521066557\\
-16.3690365105924	-2.2072739156375\\
-16.2419236931027	-2.18500602526075\\
-16.1077813735065	-2.15613843469059\\
-15.9792008446619	-2.13235033183341\\
-15.8470246911112	-2.10923299186298\\
-15.7156089499173	-2.08547359096646\\
-15.5853446232215	-2.06061116979571\\
-15.4528641440805	-2.03282228508533\\
-15.3216703523965	-2.00298563679159\\
-15.1910006223904	-1.97490245806323\\
-15.0602201620703	-1.9502050354226\\
-14.9293216382364	-1.92022495564558\\
-14.7985979840155	-1.88727803412142\\
-14.6677633251412	-1.8562671775705\\
-14.5380207035069	-1.82804773113092\\
-14.4083437420802	-1.80184830031135\\
-14.2787733561742	-1.77504996571519\\
-14.1507364328908	-1.74383854595525\\
-14.0207137377463	-1.71073515175093\\
-13.8925569554473	-1.68169082104958\\
-13.761726994518	-1.65619034510569\\
-13.6293884320535	-1.6296400977796\\
-13.4974177501548	-1.59948277048558\\
-13.3677639543776	-1.56631229139042\\
-13.2390933387295	-1.53380886206797\\
-13.110019614334	-1.50839163486185\\
-12.982924192846	-1.4826480886422\\
-12.8532301598594	-1.45089828469383\\
-12.7246347719329	-1.4197579384602\\
-12.5971791340563	-1.39037008586078\\
-12.4673215643662	-1.36083541255139\\
-12.3382155978623	-1.33694393770552\\
-12.2070182418999	-1.30890980335578\\
-12.0794642997285	-1.27404948695515\\
-11.9489657349855	-1.23993453967464\\
-11.8195787654616	-1.21219559528404\\
-11.6923946600391	-1.18502684261564\\
-11.5670411504254	-1.15236806024461\\
-11.4429988691747	-1.12242464242604\\
-11.3240232647005	-1.09430060882595\\
-11.2046215058369	-1.06516797450544\\
-11.0924563104531	-1.04108031897572\\
-10.9782041864308	-1.01838137067365\\
-10.8663119171102	-0.990561149729751\\
-10.7589394446463	-0.963349871735852\\
-10.6569957962752	-0.940703771774692\\
-10.563937897853	-0.919206027922718\\
-10.4759820901314	-0.898761204839082\\
-10.3972908337602	-0.882957914183179\\
-10.330375639635	-0.869936680898707\\
-10.2724034066117	-0.856533550869932\\
-10.2260623729326	-0.846036539699312\\
-10.1897783975209	-0.837392253899091\\
-10.1612857122825	-0.831452506262806\\
-10.1417296690856	-0.827035138396354\\
-10.1280661097699	-0.824345456515281\\
-10.1188064623019	-0.822803784661156\\
-10.1130953199586	-0.821714900533247\\
-10.1091912662481	-0.821361811651817\\
-10.1072134830779	-0.821352497160289\\
-10.1063281189331	-0.821417064845345\\
-10.1062628296248	-0.821621253257378\\
-10.1064906979024	-0.821885842973069\\
-10.1069694581503	-0.82224023169958\\
-10.1073322408246	-0.822425121399114\\
-10.1074668069244	-0.822537537043669\\
-10.1074542969894	-0.822632812485737\\
-10.1074368337898	-0.822663344466892\\
-10.1073949770432	-0.822633311892085\\
-10.1072868856298	-0.822625273197502\\
-10.107277334749	-0.822623566988714\\
-10.1073268204879	-0.822595115867721\\
-10.1073522071708	-0.822589751519583\\
-10.1073861987275	-0.822591780634284\\
-10.1074771527301	-0.822557641943454\\
-10.1075896863093	-0.822561655768572\\
-10.1075769145613	-0.822563993001148\\
-10.1075406978211	-0.822559494601024\\
-10.1075239092873	-0.82255748185518\\
-10.1074922664978	-0.822590723171904\\
-10.1074981457356	-0.822588780744828\\
-10.1075256077492	-0.82260630011713\\
-10.1075221681079	-0.822622361096856\\
-10.1075040370553	-0.82261382909641\\
-10.10748727266	-0.822648416349649\\
-10.1074890335853	-0.822700708838657\\
-10.1075070011922	-0.822720181418125\\
-10.1075182783712	-0.822733799511384\\
-10.1075312082149	-0.8227510781509\\
-10.1074725342754	-0.82280299134352\\
-10.1059600237561	-0.82299303005442\\
-10.1014091050906	-0.821526922779003\\
-10.0903541444258	-0.819162681193516\\
-10.0702416467413	-0.81436649613715\\
-10.0398904269262	-0.805668475801171\\
-9.99724062479655	-0.794211617053428\\
-9.94293348932661	-0.78050343547816\\
-9.87542112098584	-0.765366863250363\\
-9.79814755837289	-0.748987646364983\\
-9.71189134743076	-0.729496272147711\\
-9.61783891972089	-0.70523442394262\\
-9.51857316705396	-0.679666670281398\\
-9.41538921923652	-0.656401819102942\\
-9.30933650656943	-0.631907081640096\\
-9.19968131870841	-0.603149421143961\\
-9.08204220678033	-0.5766202985443\\
-8.96556051031865	-0.554039647562945\\
-8.84427894723264	-0.524545632434934\\
-8.72452268553704	-0.492228342207594\\
-8.6036828041501	-0.464887831854657\\
-8.47897423876673	-0.438172660935063\\
-8.35541570036433	-0.410831293517628\\
-8.23148937956161	-0.382570068230915\\
-8.1060723174697	-0.355792027585466\\
-7.98503488745689	-0.326612810495622\\
-7.85862656266228	-0.294717625178855\\
-7.72990527034049	-0.269933314307017\\
-7.60061076093677	-0.246224951308284\\
-7.47022168501084	-0.213624004430629\\
-7.33970457175974	-0.179554385128228\\
-7.21008465001023	-0.152104942862888\\
-7.07744321200762	-0.125095787125837\\
-6.94472399395039	-0.0982988884584991\\
-6.80970248079524	-0.070102025589887\\
-6.6750762686454	-0.0384495053591518\\
-6.54194783455209	-0.00683222139056115\\
-6.40721547553335	0.0204337553849392\\
-6.27050490270175	0.0456691275951886\\
-6.13292163576028	0.0766029810009343\\
-5.99837699657556	0.111424754215601\\
-5.86748677013673	0.144453291130636\\
-5.73426589745025	0.176733407655623\\
-5.59880503415058	0.204926382270445\\
-5.46727796740665	0.233173391775473\\
-5.3345024050363	0.267177248436844\\
-5.19955261010616	0.300264439773778\\
-5.06723688059976	0.331145227824994\\
-4.93241103189932	0.357819979943867\\
-4.80206080557779	0.386437048647827\\
-4.66926192813861	0.419865890694154\\
-4.54037551038165	0.455959483957998\\
-4.40934349962551	0.486089807062471\\
-4.27951825198112	0.508104441510145\\
-4.15216850696719	0.53317338802407\\
-4.02390058217803	0.561749677692947\\
-3.89893465669151	0.586741210757513\\
-3.77232817312292	0.609911819559437\\
-3.64730686486922	0.63057577883909\\
-3.51931557954242	0.646581167805286\\
-3.38981845400159	0.664996467991612\\
-3.26127578589177	0.686657615709844\\
-3.13034495402016	0.707483996709437\\
-2.99962948851108	0.721599368641348\\
-2.86992634002333	0.728582414095892\\
-2.73721809665444	0.739332727621004\\
-2.60766425495171	0.750104828203993\\
-2.47834314750955	0.756539728066056\\
-2.34583016820923	0.75947347144413\\
-2.21732650575343	0.759675880864654\\
-2.08639429705994	0.7587403791709\\
-1.95519572601096	0.756798060750315\\
-1.82451649474835	0.754552090441267\\
-1.69464513507949	0.751119407765913\\
-1.56577809525865	0.742005415026537\\
-1.43676589061439	0.728348511043674\\
-1.30980023338614	0.717073405398076\\
-1.18393895354155	0.708697618407748\\
-1.05900737828149	0.693105873983302\\
-0.937096452559004	0.673954289009285\\
-0.820586634006304	0.656878916044376\\
-0.700953844395633	0.63994263202884\\
-0.585830832893412	0.623731815038984\\
-0.473778843305896	0.605435461652165\\
-0.367130262318813	0.58829320967983\\
-0.263429412646291	0.573460611744371\\
-0.168995701973718	0.561781275314086\\
-0.0802119712231269	0.55181782316043\\
0.00196259870201621	0.547107990632065\\
0.0793110758995115	0.545766563859418\\
0.150213475555143	0.544935410835604\\
0.213611322361117	0.547431533011075\\
0.266067827077746	0.553694631391817\\
0.308304183466038	0.558935869594833\\
0.338821048874056	0.563958085025466\\
0.360628663659536	0.568526394657454\\
0.375832969221941	0.571927512533977\\
0.386710164767274	0.5750536229208\\
0.395281977649544	0.57605526931106\\
0.400927223902961	0.577405749055991\\
0.405869362565782	0.578950461697472\\
0.409953604400958	0.578856184487236\\
0.412060476870128	0.579103870667439\\
0.412305721565966	0.579097950790085\\
0.411868954103004	0.578330509467443\\
0.411374188410234	0.577880896819786\\
0.410898846550918	0.577784099618098\\
0.410946829710643	0.577443283360592\\
0.410986751740597	0.577459783979815\\
0.411126698463146	0.577647903860697\\
0.411317826187041	0.577650165281403\\
0.411358523330154	0.577716657182636\\
0.4113869036374	0.577800906413886\\
0.411473618802465	0.57774551755352\\
0.411457393289283	0.577823465893689\\
0.41151117225884	0.577916806472097\\
0.411512455871821	0.57791705146631\\
0.411532956402455	0.577952798698104\\
0.411512049602722	0.577977223771869\\
0.411528500090786	0.577949794490501\\
0.411534938979846	0.577922610012269\\
0.411558477046665	0.577891324249795\\
0.411611464326033	0.577782915972485\\
0.411615025373077	0.577731451153643\\
0.411593328652256	0.577706999809956\\
0.411613880476581	0.577626135808754\\
0.411566748321058	0.577586560503794\\
0.411544878253334	0.577620742974867\\
0.411516624031199	0.577575032941944\\
0.411517930006304	0.577570110565303\\
0.411578399644241	0.577512287494666\\
0.411595095529217	0.577466441587206\\
0.411605680888979	0.577460304734264\\
0.411606062567954	0.577440409357378\\
0.411618295352803	0.577417624902671\\
0.411648195900969	0.577426674933932\\
0.41166685688162	0.577440946318967\\
0.411690227174573	0.577453011040394\\
0.411693321777629	0.577466942828084\\
0.411681862836751	0.577474810579435\\
0.411694768874393	0.57751574479305\\
0.411732981989668	0.577547947074371\\
0.411737680103852	0.577552268641736\\
0.411764334356131	0.57757082210455\\
0.41173056410406	0.577581112337578\\
0.411723681590242	0.577640362569583\\
0.411723409556966	0.577657376161157\\
0.411716040496342	0.577660829635377\\
0.411700747935951	0.577662833821857\\
0.411695445224028	0.577659069927428\\
0.411695898433635	0.577665173966188\\
0.411727169420266	0.57765400660157\\
0.41174603864449	0.577665567973183\\
0.411756976114258	0.57765924432787\\
0.411765538143698	0.577629629238749\\
0.411765816284174	0.577622079084756\\
0.411764107267364	0.577597206413899\\
0.4117724473484	0.577570754027016\\
0.411764865604992	0.577544520860951\\
0.411768310362079	0.577537646398383\\
0.411766536160633	0.577539535448562\\
0.411771008574618	0.577557599242415\\
0.411769404465588	0.577568390267755\\
0.411785618588373	0.577662796127751\\
0.411793892575265	0.577981746273396\\
0.411825175267394	0.578488279513507\\
0.411800122460608	0.579153413751195\\
0.412274317990363	0.580139255490988\\
};
\addplot [color=white!50!blue,solid,line width=1.2pt,forget plot]
  table[row sep=crcr]{%
0.0103269384988584	-0.0336058400010778\\
0.27849987018299	-0.489906354445425\\
0.877485930817473	-1.21624312263884\\
1.11001313230075	-1.12285239284775\\
0.236301096947788	-0.161430627353994\\
-0.00398456019324209	0.00457779802289303\\
-0.00386107307075567	0.00462454366501821\\
-0.00379566908132129	0.00464008512247189\\
-0.00381546113768328	0.00463215580710129\\
-0.00384405379624822	0.00464797271771014\\
-0.00385345590070224	0.00467256954026628\\
-0.00386089870516954	0.00467040517838031\\
-0.00386129104654566	0.00469505676435914\\
-0.00387121958294955	0.00470613344431623\\
-0.00389686865637895	0.00470773156158011\\
-0.00390996458522319	0.00469572886514641\\
-0.00396658117898397	0.0046418931177305\\
-0.00399904942031218	0.0045875494463465\\
-0.00400018925792513	0.0045806860658221\\
-0.0040258373091311	0.00460741804977214\\
-0.00406526443087539	0.00460554673035783\\
-0.00404835615859119	0.00461614689363586\\
-0.0041111848151734	0.00465969984933518\\
-0.00412095490100471	0.00465901900630043\\
-0.00411625762893342	0.00464266011147295\\
-0.00411146272670343	0.00464399295648777\\
-0.00408911527283053	0.00464264863925009\\
-0.00402669256093813	0.00463035100288737\\
-0.00401908102328578	0.00462155137028347\\
-0.00400141669149138	0.00461252893847458\\
-0.00397749265835054	0.00457869156001465\\
-0.00398941240714763	0.00458332105754802\\
-0.00399493202455781	0.00458961733167181\\
-0.00401768152575699	0.00458803842668037\\
-0.004053621855588	0.00462546507751181\\
-0.00411892261042213	0.00462382613915146\\
-0.00416970568758391	0.00466278245483152\\
-0.00438063285385231	0.00492131880000003\\
-0.0047302196254191	0.00536435761806431\\
-0.00526620991821146	0.00592229636379819\\
-0.00603926726187389	0.00673057157951232\\
-0.0069296899926979	0.007729585632473\\
-0.0079878697227575	0.00882497763960239\\
-0.00910696418062701	0.0100436014874733\\
-0.0104346667259973	0.0115552540855369\\
-0.0116056170792295	0.0133468348992796\\
-0.0128732861658076	0.0152940767899244\\
-0.0141854841712418	0.0171070077685879\\
-0.0153456227578023	0.0190862613495766\\
-0.0146401104242591	0.0208595718937794\\
-0.0105091812713777	0.0216990241647751\\
-0.001020991104848	0.022674285282622\\
0.015426921509867	0.0243422608388283\\
0.0394451992486073	0.0244293207738211\\
0.0711255192710819	0.0239056379222028\\
0.110594758933927	0.0234523818907045\\
0.156918399702248	0.0216751523121295\\
0.210313291044651	0.0192895149048727\\
0.270069921899359	0.0137062357386454\\
0.334225919661469	0.00650583602181062\\
0.403083291197777	0.000855356097574325\\
0.474793729414659	-0.0105259665346114\\
0.552880561756063	-0.0222162784719574\\
0.637682806308049	-0.0228014206193017\\
0.728156789021406	-0.0210602190621922\\
0.823502451631803	-0.0205289950247475\\
0.924076761476456	-0.0219234643781255\\
1.02946331530064	-0.0211974319132003\\
1.13469618764652	-0.0213564526178604\\
1.24339826042507	-0.0217469841938297\\
1.35049967569383	-0.018075150180337\\
1.46048285621864	-0.0185924786003512\\
1.57185334315509	-0.0187023172422179\\
1.68677962653082	-0.0165148041556105\\
1.80572311329778	-0.0173192100707473\\
1.92457227352535	-0.0150284635216453\\
2.0412336027952	-0.012577512168455\\
2.15991721924664	-0.0160927497119446\\
2.2772007364521	-0.0197735086548126\\
2.39823395200945	-0.021735559259741\\
2.5233450109636	-0.0254357417458697\\
2.64798359391243	-0.0286311129620043\\
2.77299834785591	-0.0334992500291089\\
2.89894224310681	-0.0417586261561298\\
3.02592973898085	-0.0554206575071878\\
3.15310526790712	-0.0694174483664272\\
3.28412245992228	-0.0767041927399408\\
3.41606857069634	-0.089826175392085\\
3.54530137881872	-0.105566486783014\\
3.67703987560009	-0.117575452650712\\
3.80716349534274	-0.127467376489531\\
3.93770426203955	-0.136643455642791\\
4.06846465809207	-0.140647146737363\\
4.19785178415601	-0.143957713481857\\
4.3309844594354	-0.145936710848155\\
4.46039676980182	-0.14616186865624\\
4.59233432101576	-0.140086012828405\\
4.72114144402774	-0.124430740938845\\
4.8460072642559	-0.102248188179511\\
4.97241015169718	-0.0782124829589917\\
5.0963260374073	-0.0491380297157656\\
5.21904231072616	-0.0131671204868377\\
5.33586067025611	0.0334307859183082\\
5.45472893767561	0.0823116794796457\\
5.57332206104917	0.128510066584496\\
5.69048653818137	0.178327234125046\\
5.80466868201796	0.233919225207303\\
5.92017716874493	0.293564558912266\\
6.03554310085084	0.353555879317993\\
6.14761659794005	0.411280689213789\\
6.26399102542703	0.467458047167551\\
6.37753878678779	0.523125767162114\\
6.49096830724594	0.581167141817255\\
6.60292754546741	0.645083685026528\\
6.71176458655658	0.705140765441732\\
6.81997385571702	0.759518567054632\\
6.92783679792651	0.814292467654065\\
7.03677701385523	0.870797859380816\\
7.14682566843322	0.927471548868242\\
7.25639974546965	0.985695206852337\\
7.36887850056449	1.04019065969554\\
7.48293682043383	1.09075735832494\\
7.59410678618958	1.1415168625865\\
7.70958840790826	1.19070391359742\\
7.822664057325	1.23979042027846\\
7.93333051729804	1.28780816337547\\
8.04408946199801	1.33240578592925\\
8.15541710374844	1.37177784444474\\
8.26665462114234	1.40789649820064\\
8.37725044432226	1.44147418028452\\
8.48646392672766	1.47137308343598\\
8.59340238350899	1.49237432857004\\
8.69266444339017	1.50836483157555\\
8.78851526990923	1.522506272253\\
8.87946604714001	1.53731028635566\\
8.96379649864577	1.54789557599452\\
9.04033470628765	1.55250275562266\\
9.11021133478471	1.55644351392337\\
9.1724920095512	1.55990103603497\\
9.22609041216591	1.56157885502205\\
9.27177043487076	1.56211039331208\\
9.30642843852748	1.56217987614039\\
9.33106968783767	1.5632731653477\\
9.34734564718737	1.56399596852204\\
9.35618793775326	1.56535150816801\\
9.36183077238708	1.56763885238857\\
9.36409561343162	1.56992746194909\\
9.36452992476215	1.57128827792404\\
9.36434450602919	1.57291605567649\\
9.36393345231976	1.57470230452257\\
9.36164433317503	1.57622718743391\\
9.35980214126028	1.57757028549297\\
9.35745260421693	1.57851420526652\\
9.35562210101099	1.57942152743272\\
9.3541008820933	1.58012009709701\\
9.35308071415328	1.58048983661799\\
9.35242418463035	1.58060033466233\\
9.35227731529746	1.58058121431943\\
9.35221436167068	1.58052110960183\\
9.35221737181311	1.5805000886928\\
9.3521598897825	1.58059432136965\\
9.35214728668924	1.58070849005723\\
9.35216325823117	1.58074061809257\\
9.35214991424294	1.58082914477724\\
9.35211825950249	1.58089358750294\\
9.35213411424318	1.58088705900621\\
9.35216116853708	1.58087809253165\\
9.35217697902405	1.58089791989235\\
9.35214635691283	1.58086069319534\\
9.35210315988571	1.58081979806841\\
9.35208326133421	1.58079744209951\\
9.35205849840975	1.58077409178596\\
9.35201579326258	1.58078554931965\\
9.35199259398725	1.58078608332666\\
9.35195629049303	1.58079376350682\\
9.35195038036486	1.5807914431684\\
9.35191580100508	1.58078940136972\\
9.351942224745	1.58077722388119\\
9.35193631505864	1.5807743213471\\
9.35194693703545	1.58077415042017\\
9.35194117790764	1.58076246364886\\
9.35195865995452	1.58075728518682\\
9.35193749597692	1.58074638178512\\
9.35193344603243	1.58074787174294\\
9.35193271025915	1.5807293312796\\
9.35198403742301	1.58071954872014\\
9.35200742788799	1.58072827585247\\
9.35204502964456	1.58075165445834\\
9.35203163205236	1.58080211814515\\
9.35236131886092	1.58083266597337\\
9.35285340059599	1.58102054414217\\
9.35427134012155	1.58096369659986\\
9.35913565121947	1.58072797887524\\
9.36981247457943	1.57988914245819\\
9.38838744115023	1.57806266222611\\
9.41644361567652	1.5752684572848\\
9.45455476860759	1.57349906847172\\
9.50533196801378	1.57038054872318\\
9.56552799285103	1.56507685958691\\
9.63588235843731	1.55770100284131\\
9.71556219664704	1.54679577130913\\
9.80099047045009	1.53684498555062\\
9.8909488547112	1.52507156501189\\
9.98183213025632	1.5080799782776\\
10.0752505784647	1.4892297360929\\
10.1735488889735	1.47247349775936\\
10.2724317677499	1.45833445351362\\
10.3785276126848	1.4397641478117\\
10.4883691567682	1.41718591598439\\
10.5979993859076	1.39753733640383\\
10.7129858699305	1.37853460838935\\
10.8282627029558	1.35969512526235\\
10.9454329731638	1.34014291178241\\
11.066715828761	1.32192060199364\\
11.1875272937086	1.30484425828192\\
11.3104803398983	1.29055022034527\\
11.4352311970191	1.27546066962235\\
11.5557619804436	1.2577345869182\\
11.6821387746138	1.24351946411317\\
11.8116889202229	1.22880853609129\\
11.939875418223	1.21263959769215\\
12.0706346652952	1.20011166590973\\
12.1981487738207	1.18719121370621\\
12.3278724250277	1.17098285261223\\
12.4623435001203	1.15903804223749\\
12.5949320580115	1.15302111668315\\
12.7290027902902	1.1471576061303\\
12.8636674480258	1.13839918634443\\
12.9962871777499	1.13363636258118\\
13.1321412739363	1.13226682408075\\
13.2688160810445	1.12917843783764\\
13.4031468952295	1.12987739701024\\
13.5379683164959	1.13650322597173\\
13.6726932046637	1.14200456206076\\
13.8056970684512	1.14855249547157\\
13.9393803293122	1.16407610504075\\
14.0707910857571	1.18579359754581\\
14.2026685076981	1.21201689172609\\
14.334072294293	1.24303678781686\\
14.4605425096764	1.27759969554251\\
14.5925352868852	1.31817989768296\\
14.7203596193743	1.36079823527883\\
14.8436366243104	1.41081062592003\\
14.9656914454988	1.46873200977276\\
15.0844174362588	1.53079620824706\\
15.2047010374468	1.59370613694145\\
15.3190323953566	1.66766481864382\\
15.4301846753818	1.75092146103522\\
15.5358232651838	1.83825642877503\\
15.6384967610199	1.92891931866391\\
15.7393873098377	2.01825904827417\\
15.8370756476089	2.11613678438781\\
15.9261114769501	2.2216482236097\\
16.0091741228984	2.33250422658545\\
16.08755793138	2.44911333899519\\
16.1589606514369	2.56216375895034\\
16.230035155145	2.67654462314759\\
16.296547131666	2.79574272695611\\
16.3554747820418	2.91984656904098\\
16.4087041372912	3.04482373604522\\
16.460507781053	3.16770517938543\\
16.5102741981139	3.2906234717347\\
16.5550648751371	3.41935540036139\\
16.5922232392258	3.55156379748306\\
16.6217214646953	3.68812015316\\
16.6492494358634	3.82362470909668\\
16.6720770570319	3.95655972681151\\
16.6907523036214	4.09161143589415\\
16.7037785061581	4.22682034559239\\
16.7122575179277	4.36564346210943\\
16.7149530845429	4.50288881946712\\
16.7147191585698	4.63740473337407\\
16.7103888515759	4.77288004902339\\
16.7008968817712	4.9090403347584\\
16.6895705626196	5.04584073558012\\
16.6737919085396	5.1828658987361\\
16.6558226263668	5.31701395325418\\
16.6367607172673	5.44776040085607\\
16.6124635412125	5.5798662988114\\
16.583257567029	5.71249962551248\\
16.5516579694788	5.84315778586757\\
16.515686193339	5.97409082563452\\
16.4753638988974	6.10288463095873\\
16.4323369696407	6.22932672281226\\
16.3862181629599	6.35673774259116\\
16.334940203465	6.48070293904847\\
16.2792301685809	6.60109429260529\\
16.2244313603823	6.72110817200826\\
16.1665703056138	6.83997348227636\\
16.1059011834423	6.95912978042937\\
16.0422698500988	7.0742297162307\\
15.9745381750722	7.18596502017721\\
15.9042311976203	7.29552060146978\\
15.8341152725068	7.40398209060164\\
15.7600590935892	7.51005640429243\\
15.6812482108916	7.61375896485819\\
15.6009930860053	7.71876822193608\\
15.5174741433954	7.81730459679839\\
15.4334251336312	7.91380078162549\\
15.3515443104188	8.01051124872358\\
15.2674088538374	8.10478262197502\\
15.1800238297675	8.18993962099996\\
15.0914184110072	8.26881954507803\\
15.00635388629	8.34413157636568\\
14.926972706492	8.42135624000272\\
14.845612527727	8.49065166892352\\
14.7574520626484	8.55451142092788\\
14.6702416145126	8.6176291752396\\
14.5840431190131	8.68159528088574\\
14.4981133267754	8.7384887929747\\
14.4092854327736	8.79008049206758\\
14.3219747166704	8.84160316294384\\
14.2362495540861	8.88930516662911\\
14.1500029271629	8.92984189378596\\
14.0683733894099	8.96223475828122\\
13.9905184401603	8.99015440993313\\
13.9161450714578	9.01533369985549\\
13.8464147263183	9.0369820360297\\
13.7793380852549	9.05535933208939\\
13.7175591881156	9.06976126276585\\
13.6593411462992	9.07904435924047\\
13.6095379551814	9.08600748066905\\
13.56818814169	9.09295654166435\\
13.5346271288779	9.09776869685363\\
13.5084780632234	9.0992566886691\\
13.4895408952812	9.10106010583922\\
13.4765555136312	9.10315220632564\\
13.4678287941565	9.10429862690395\\
13.4625192680773	9.10414483071212\\
13.459382723088	9.104514842855\\
13.4582460989505	9.10529524208146\\
13.4580697158034	9.10503917852947\\
13.4584526256475	9.10454574193248\\
13.4585550395185	9.10442790652187\\
13.4582856114923	9.10408666759934\\
13.4576765384625	9.10345301993921\\
13.4566384835372	9.1030785976076\\
13.4557890078141	9.1028655297498\\
13.4550071243854	9.10247705046054\\
13.4543249163497	9.10224191605264\\
13.4534376664127	9.10199459249579\\
13.4527314859011	9.10165937127529\\
13.4521804811061	9.10145414006446\\
13.4519871859478	9.10142737110027\\
13.4519155214766	9.1011283880615\\
13.4518387849681	9.10108073515244\\
13.4518068541715	9.10108516645416\\
13.4517709909144	9.10079923470393\\
13.451774212569	9.10082381374136\\
13.4517959774187	9.10083970331985\\
13.451845575224	9.10073136629673\\
13.4518644802574	9.1008478642801\\
13.4518692912137	9.10100478842024\\
13.4518537445869	9.10100523249811\\
13.4517940821561	9.10114220348259\\
13.4517561684867	9.10122886872594\\
13.4517099859642	9.10123042034749\\
13.4517313610017	9.10127850605297\\
13.4517394062457	9.10127663411997\\
13.451713066268	9.10122975439814\\
13.4517002299135	9.10121770164353\\
13.4517014161453	9.10122101222465\\
13.4517394854379	9.10116093418361\\
13.451738627898	9.10113752633141\\
13.4517392539779	9.10110258040232\\
13.451769834052	9.10107090691049\\
13.4517873767167	9.10106513239145\\
13.4518036232633	9.10108228019762\\
13.4518073118155	9.10111655479298\\
13.4517583359278	9.10112207392068\\
13.4517120955044	9.10114433863066\\
13.4518237551277	9.10105718754432\\
13.4513955600839	9.10137423482898\\
13.44865293504	9.10203887975614\\
13.441427667257	9.10218760819542\\
13.4276153499214	9.1024161718525\\
13.4056712343945	9.10399225645121\\
13.3733369387741	9.10450652065635\\
13.3302852280443	9.10237239493804\\
13.2771586399877	9.09956826784904\\
13.2137860676277	9.09584966586301\\
13.1420265926953	9.08977204963922\\
13.0653998374284	9.07714327109579\\
12.9833656876972	9.05813675852652\\
12.8975760284766	9.03871865078753\\
12.8099258891732	9.01374401231506\\
12.7211116511869	8.98100585141111\\
12.6289289655763	8.94276901424798\\
12.534330910814	8.90039757226331\\
12.4379044070807	8.85027198251505\\
12.3441780634949	8.79266863556906\\
12.2510977256492	8.7320397013212\\
12.1577184469272	8.67129941765146\\
12.0652674090408	8.60166114072883\\
11.9735509996172	8.52514912365652\\
11.8828494258929	8.4441563185757\\
11.7955510623932	8.35750592429469\\
11.7100928082005	8.26956097071765\\
11.6247996431063	8.18117971103367\\
11.5396491624818	8.08784362076888\\
11.4568830875239	7.99373472555821\\
11.3704492722315	7.89883275881753\\
11.2820916616731	7.80677902747583\\
11.1953853046731	7.71916811155254\\
11.106297636529	7.6347150618069\\
11.0095324124145	7.55036700594631\\
10.9110148210674	7.46972282088103\\
10.8138960216373	7.38797754254082\\
10.712786510451	7.30662093871199\\
10.6068202559538	7.23390472372781\\
10.4985631753612	7.15966849272508\\
10.3867809403601	7.08473612653861\\
10.2704764413225	7.01624212276907\\
10.1511934178181	6.95288987710757\\
10.0320866078134	6.88960200871632\\
9.91214519749772	6.8283084378044\\
9.78869039976643	6.77081547448421\\
9.66295896662599	6.71801984601661\\
9.53477660873191	6.6686952371298\\
9.40573095901864	6.62570298060895\\
9.27679841348774	6.58301043352166\\
9.14475616219019	6.54116779061807\\
9.00907708488474	6.50728957736621\\
8.87234693897896	6.47628217617223\\
8.73221109882591	6.44491368158807\\
8.5954584829903	6.41449385584913\\
8.45827236569222	6.39037515977377\\
8.31682865517146	6.37050700760494\\
8.17465398305781	6.35090186844976\\
8.03713792496089	6.33478838347477\\
7.89516157001782	6.3201557979635\\
7.75599749401182	6.30447118930669\\
7.61919258989988	6.29316778638568\\
7.47682998470903	6.28557580535479\\
7.33652415289401	6.27910872380821\\
7.1973536362278	6.26810797933235\\
7.05733369088693	6.25835795754097\\
6.91626596153759	6.25407701394982\\
6.77523730059193	6.2501470178603\\
6.633864518081	6.24442423327761\\
6.49056785393978	6.24043674601745\\
6.34994624698102	6.23669439556844\\
6.20942032452498	6.23037575845792\\
6.06467041706972	6.23104579515887\\
5.92222800173169	6.23188970915979\\
5.77859861275426	6.22739901686919\\
5.63131478377754	6.22021695023939\\
5.48703268990354	6.21767575857674\\
5.34136203643639	6.22001846483283\\
5.1933876804776	6.22211353873177\\
5.0445750921147	6.22126564392174\\
4.89617211826696	6.21786804136949\\
4.74579181840672	6.21585746065542\\
4.59742586391851	6.21825294297304\\
4.45042512540266	6.22318995264528\\
4.30020811900419	6.22323630737379\\
4.15126498357451	6.2191113476517\\
4.00325069798721	6.21489419981715\\
3.85593106020439	6.21736115909977\\
3.70860839240925	6.22048466465186\\
3.56025926875534	6.21871034610075\\
3.40983094132251	6.21697282477346\\
3.26032584162448	6.21923187256461\\
3.11439460045954	6.22001812271561\\
2.96690303643667	6.22335817288659\\
2.81896025984146	6.22301827048285\\
2.67318528657036	6.21854187662843\\
2.52368506726978	6.21741955017712\\
2.37848617507937	6.21951665555876\\
2.2368587857998	6.21963263156328\\
2.09239588185018	6.2139202432093\\
1.95103661913632	6.20935080228687\\
1.8131764773885	6.20790029261827\\
1.67597512702912	6.20728795732282\\
1.54143098803082	6.20457686611315\\
1.40805329357709	6.20153156165906\\
1.27734869412006	6.19554286563596\\
1.14991497536646	6.19225950236466\\
1.02728798055942	6.1895139412654\\
0.91189151434787	6.18585899068216\\
0.797218000601986	6.18476056268181\\
0.689132132789129	6.18181744777021\\
0.589788606599921	6.17539750748009\\
0.496576751689031	6.17025128440449\\
0.410474740173852	6.16835941642585\\
0.335150742376788	6.16582204716519\\
0.266986801868858	6.16263995440723\\
0.209798898363312	6.16037132079791\\
0.161849816905854	6.15856477561694\\
0.121559475469498	6.1568570073784\\
0.091650997360442	6.15504181473548\\
0.068304452154025	6.15365418742598\\
0.052499215283619	6.15285404464129\\
0.0409958966891297	6.15208837787942\\
0.0325223796518919	6.15142839130431\\
0.0264324763140895	6.15146796984149\\
0.022206247494539	6.15151869519437\\
0.0196937553135679	6.15097055724242\\
0.0184280344003092	6.15091677462001\\
0.0179036041209182	6.15115052915518\\
0.0176320576119489	6.15093769271207\\
0.0173756164602814	6.15084415237089\\
0.0170380911113273	6.15095728774732\\
0.0167535912892431	6.15078182109617\\
0.0165510076688806	6.15057588557663\\
0.0166126019112261	6.15061500057262\\
0.0167139366413325	6.15052143342367\\
0.0167825588308231	6.15041785206513\\
0.0167393332669676	6.15047567498331\\
0.0167356430359571	6.15045863708092\\
0.0167409760572987	6.1504524294699\\
0.0167254838502643	6.15051844660412\\
0.0166817013610824	6.1505456990692\\
0.016656824176329	6.15053370028805\\
0.0166085983667047	6.15058597650527\\
0.0165757454739402	6.15058737414563\\
0.0165448176709073	6.15053732629064\\
0.0165385125619489	6.15052477261559\\
0.0165439089957832	6.15050385354252\\
0.0165542256611622	6.15045576447137\\
0.0165082012361403	6.15045479017683\\
0.0165309225956819	6.15045516593347\\
0.016567284295559	6.15045686839511\\
0.0165786108305256	6.15045363967909\\
0.0166064816922458	6.15042977778347\\
0.0166038768997756	6.15043308208146\\
0.0166030643924226	6.15043442140863\\
0.0165762627010196	6.15041041531043\\
0.0165663972781568	6.15039942045867\\
0.0165316964483257	6.15042118493971\\
0.0165057695683325	6.15045154287891\\
0.016545043778561	6.15047286342025\\
0.0165415426723863	6.15046841955621\\
0.0165390223542341	6.15048286601036\\
0.0164882872925161	6.15048024232412\\
0.0164983659783248	6.15046844259419\\
0.0164857587418511	6.15047936971807\\
0.0164696718642632	6.15049606803795\\
0.0164796128113336	6.15050210821307\\
0.0164936086516796	6.15053067362552\\
0.0165284327825365	6.15053535666501\\
0.0165218934529722	6.15053322722015\\
0.0164950851612382	6.15061711556554\\
0.0164680622015582	6.15065525426547\\
0.0164837907207732	6.15066588580263\\
0.0165346116193476	6.15073740433662\\
0.0165467366597734	6.15075151930357\\
0.016546484374534	6.15065918897328\\
0.0165838802344176	6.1506379192373\\
0.0165915448607179	6.15060049985274\\
0.016585805419906	6.1505556458768\\
0.0165746085278724	6.1505530303058\\
0.0165805600014988	6.15053795741779\\
0.016559584068017	6.1505137372439\\
0.0165732765447359	6.15051744651894\\
0.0165931768725365	6.15053258105674\\
0.0166510576270838	6.15049945189052\\
0.0166159358889245	6.15051310603219\\
0.0165965183941677	6.15062704580754\\
0.0158252873841213	6.1505850550181\\
0.0143870787351467	6.15023854477491\\
0.00856326548873759	6.15091391811248\\
-0.00385217893337214	6.15039406209403\\
-0.0257615971264813	6.14824348091676\\
-0.0597137713797334	6.14586211128575\\
-0.105497758602209	6.14372407864116\\
-0.161427896769905	6.13919111009434\\
-0.230244008350986	6.13403070363194\\
-0.308116506228286	6.12913296427585\\
-0.394480143868541	6.12398299825306\\
-0.488527405588608	6.11723776634281\\
-0.590324039888515	6.10989639220296\\
-0.696344792030904	6.10286900740427\\
-0.81050075246981	6.09516420951112\\
-0.930025206733984	6.08435040384289\\
-1.05599546769492	6.07636141752693\\
-1.18508427246344	6.07099326180228\\
-1.31341116749857	6.063714366466\\
-1.44096803256235	6.05744721478179\\
-1.5713746113873	6.05073757387603\\
-1.70203073207704	6.04236567480356\\
-1.83277315828076	6.03494784394274\\
-1.97000670501945	6.02811124328606\\
-2.10947774242723	6.02553079026069\\
-2.24916525600227	6.02102792648699\\
-2.39097478636177	6.01392639226974\\
-2.53521112028803	6.00727385484816\\
-2.67540386197061	6.00541759681479\\
-2.81603821526099	6.0053437495269\\
-2.95799982801236	5.99909074564799\\
-3.09653136467736	5.99223808102298\\
-3.23766102140674	5.98934830669917\\
-3.37977261000342	5.98793145609234\\
-3.52294850448422	5.98356777801751\\
-3.66697994362211	5.97861532068852\\
-3.81132267677119	5.97452161065533\\
-3.95637661517502	5.97028714621333\\
-4.10214639556529	5.96520274436271\\
-4.24640384541651	5.96607670964161\\
-4.39176563902338	5.96371497966159\\
-4.53521083390555	5.95625700060105\\
-4.68466550299457	5.95390506654983\\
-4.83167970475413	5.95487614207293\\
-4.97768384057137	5.95552355537997\\
-5.12677239111417	5.94790720432349\\
-5.27680461136272	5.94172976036011\\
-5.42333294218758	5.94034494372789\\
-5.57380895840575	5.94236235229253\\
-5.72377102529339	5.94242080251364\\
-5.87303303740171	5.93758196863664\\
-6.02284880580382	5.93068529895212\\
-6.17134604073639	5.92961040728909\\
-6.31876722335292	5.92940081939097\\
-6.47123369613901	5.93060341967493\\
-6.62316065187298	5.92797452075653\\
-6.7731097108018	5.92111535067289\\
-6.92509103303407	5.91711749275701\\
-7.07496639209194	5.91793227103148\\
-7.22541820064477	5.91952183529325\\
-7.37936842961552	5.91471129443092\\
-7.5340189318529	5.90500435258549\\
-7.68738463543175	5.9033664727491\\
-7.84087727676872	5.90647218914481\\
-7.99574254845225	5.90404814134234\\
-8.15054810665703	5.89795628248686\\
-8.30842508857855	5.89522642442928\\
-8.46271343662686	5.89702821594089\\
-8.61462671997964	5.89626611063478\\
-8.77017757930082	5.89261634935749\\
-8.92395257277472	5.88822629428319\\
-9.07898894447952	5.88718879216421\\
-9.23340724348691	5.88460977699315\\
-9.38410430273212	5.88281297605861\\
-9.53858619500025	5.87666139119346\\
-9.6958931551049	5.86944308097612\\
-9.84834267192988	5.8664207581659\\
-10.0013899605355	5.86449341866235\\
-10.1559628052233	5.8582944376217\\
-10.3065437086457	5.851096917145\\
-10.4607930665382	5.84465996151243\\
-10.6115725672525	5.83877163934596\\
-10.7588949618008	5.83061616594505\\
-10.9082641462357	5.81813772836077\\
-11.0565033281464	5.80481551724084\\
-11.2036406363077	5.79540700255056\\
-11.3511871317421	5.78252996701445\\
-11.496117451396	5.76470771054264\\
-11.6391853908431	5.74572693549006\\
-11.7814119076463	5.72425963373823\\
-11.9274392658251	5.70386490762877\\
-12.0711400147388	5.68258670391227\\
-12.2154776336773	5.65616227997184\\
-12.35911528237	5.6214704426173\\
-12.4986571423104	5.58549000144543\\
-12.6381924504089	5.54578325642096\\
-12.7766768781576	5.50239616682566\\
-12.9126287030502	5.458070555856\\
-13.0440287127106	5.40228192043997\\
-13.1722813883569	5.34373730928023\\
-13.3011666913551	5.28443904120548\\
-13.4273678255378	5.22038901627854\\
-13.5505530448762	5.15340332666407\\
-13.6727684056664	5.07879558698412\\
-13.7908111591031	5.00197239017551\\
-13.9050183976686	4.92280230102412\\
-14.0175372135272	4.83777162785752\\
-14.1286720587912	4.75165656439665\\
-14.2361345070307	4.66438199783405\\
-14.3439817535222	4.57647123259808\\
-14.4470946062731	4.48892912162639\\
-14.5520340328141	4.40307646295856\\
-14.654800206169	4.31578820267063\\
-14.7581702981285	4.23152837841691\\
-14.8611262130002	4.14733078500681\\
-14.9648221905639	4.07033410555177\\
-15.0738670517154	3.99939007540719\\
-15.1842256230911	3.93062426652983\\
-15.2917000697957	3.86475180329914\\
-15.4032971479582	3.80015822496021\\
-15.520978974912	3.74182526370344\\
-15.6344455027012	3.68563412592599\\
-15.7529175817097	3.63288089156706\\
-15.8718049227806	3.58494663378681\\
-15.9893415218061	3.54022350296962\\
-16.1060761782955	3.49692730219513\\
-16.2201675777116	3.46188745824641\\
-16.3302559201958	3.43506003831484\\
-16.437736048383	3.40772302264589\\
-16.5447225590756	3.38367762055722\\
-16.6479220811223	3.36353338516599\\
-16.746623623748	3.34424474750012\\
-16.8394388040262	3.32965674515108\\
-16.9286762859082	3.32065272290223\\
-17.0098500341128	3.31178354096954\\
-17.0858654439695	3.30351393153473\\
-17.1550395538036	3.29964788231555\\
-17.2153843621968	3.29811373884201\\
-17.2656822742094	3.29422842264046\\
-17.3058909908283	3.29160795276169\\
-17.3357099398982	3.29164495391973\\
-17.3569726726678	3.29036500723568\\
-17.3718940535422	3.28902550209676\\
-17.3824123122454	3.29000661042688\\
-17.3904894881742	3.29087098229221\\
-17.3968929210771	3.29103618252969\\
-17.4015354716482	3.29172785821193\\
-17.4049729954741	3.29284566102939\\
-17.4070724304152	3.29401152552624\\
-17.4080178706837	3.29511961471144\\
-17.4088615010269	3.29628628984917\\
-17.4093253775443	3.29744817067523\\
-17.4100467989001	3.29847027969978\\
-17.4107929781592	3.29940561929489\\
-17.4116786998786	3.30011575219705\\
-17.4124485081065	3.30098934218189\\
-17.4132387240806	3.30188762478828\\
-17.4136710910919	3.30250922184968\\
-17.4140064524952	3.30311071683978\\
-17.4141221921682	3.30377981926818\\
-17.4142846533798	3.30417344905383\\
-17.4143591672722	3.30449542803232\\
-17.4144793881954	3.30478968792359\\
-17.4144638471313	3.3048811960608\\
-17.4144992989808	3.30495956725817\\
-17.4144526596759	3.30506697955611\\
-17.4144236404178	3.30508165773086\\
-17.4143781569999	3.30510869903131\\
-17.4144070012612	3.30515896082063\\
-17.414409275065	3.30513830816696\\
-17.4143916525354	3.30513449209651\\
-17.414370668684	3.3051505224499\\
-17.414359596877	3.30515952590607\\
-17.4143637810608	3.30513651481595\\
-17.4143564245401	3.30513671061598\\
-17.4143104100992	3.30512672771287\\
-17.4143048633179	3.30512077953856\\
-17.4143268275483	3.30511448922575\\
-17.4143692686459	3.30509242869858\\
-17.4143861971845	3.3051029489999\\
-17.4143943633582	3.30509509768301\\
-17.4143720014027	3.30504442693502\\
-17.4143112733477	3.30499217251118\\
};
\addplot [color=blue,solid,forget plot]
  table[row sep=crcr]{%
-17.8267402205454	3.30499217251118\\
-17.8182977757671	3.21110297692872\\
-17.7933160761134	3.12105761604491\\
-17.7528178753032	3.03854255751423\\
-17.6984611744362	2.96693597752802\\
-17.6324713432324	2.90916945788434\\
-17.5575500131348	2.86760796668467\\
-17.4767644722004	2.84395303626201\\
-17.3934220899671	2.83917310224718\\
-17.3109349133558	2.85346385570038\\
-17.2326799770688	2.88624023149614\\
-17.1618610473922	2.93616036095832\\
-17.1013774596278	3.00118050812137\\
-17.0537054189653	3.07863874052089\\
-17.0207966243506	3.1653639090215\\
-17.0039983657019	3.25780547503499\\
-17.0039983657019	3.35217886998737\\
-17.0207966243506	3.44462043600087\\
-17.0537054189653	3.53134560450147\\
-17.1013774596278	3.60880383690099\\
-17.1618610473922	3.67382398406404\\
-17.2326799770688	3.72374411352622\\
-17.3109349133558	3.75652048932198\\
-17.3934220899671	3.77081124277518\\
-17.4767644722004	3.76603130876035\\
-17.5575500131348	3.74237637833769\\
-17.6324713432324	3.70081488713803\\
-17.6984611744362	3.64304836749434\\
-17.7528178753032	3.57144178750814\\
-17.7933160761134	3.48892672897745\\
-17.8182977757671	3.39888136809365\\
-17.8267402205454	3.30499217251118\\
};
\end{axis}
\end{tikzpicture}%


  \tikzsetnextfilename{tikz-stroller-6}%
  % This file was created by matlab2tikz.
%
%The latest updates can be retrieved from
%  http://www.mathworks.com/matlabcentral/fileexchange/22022-matlab2tikz-matlab2tikz
%where you can also make suggestions and rate matlab2tikz.
%
\begin{tikzpicture}

\begin{axis}[%
width=\figurewidth,
height=0.444\figureheight,
at={(0\figurewidth,0\figureheight)},
scale only axis,
xmin=-25,
xmax=20,
xtick={-25,-20,-15,-10,-5,0,5,10,15,20,25},
xticklabels={\empty},
xlabel={$t = 108.3$~s},
xmajorgrids,
ymin=-5,
ymax=10,
ytick={-5,0,5,10},
yticklabels={{0},{5},{10},{15}},
ymajorgrids,
axis background/.style={fill=white}
]
\addplot [color=gray,dashed,line width=0.8pt,forget plot]
  table[row sep=crcr]{%
0.0103269389045126	-0.0336058328639577\\
0.278501789313486	-0.489906141589696\\
0.87749219437659	-1.21624250065453\\
1.11002403027368	-1.12285157681853\\
0.236314881988431	-0.1614294835538\\
-0.00397140892669651	0.00457826309944895\\
-0.00385065029903127	0.00462123317922545\\
-0.00378682509306162	0.00463463813091785\\
-0.0038074453593265	0.00462647572328424\\
-0.00383703220824572	0.00464188919059538\\
-0.00384672332672233	0.00466634399517023\\
-0.00385491589714484	0.00466420406674007\\
-0.00385581603152646	0.00468880068059153\\
-0.00386602158423475	0.0046999517858656\\
-0.00389226345218809	0.00470178959877942\\
-0.00390561150997635	0.00468993728131046\\
-0.00396259582380767	0.00463665065591376\\
-0.00399555681061864	0.0045828079544999\\
-0.00399690419136082	0.00457609661836634\\
-0.00402294171754303	0.00460315490248991\\
-0.00406255722176316	0.00460156446717354\\
-0.00404601777895617	0.00461232105548733\\
-0.00410903777507603	0.00465619072567736\\
-0.00411891627082658	0.00465566564470737\\
-0.00411466731667848	0.00463970759044145\\
-0.0041101756396055	0.00464126795239423\\
-0.0040881330043217	0.00464008907172103\\
-0.0040263407232255	0.00462807605408912\\
-0.00401890938999203	0.00461943082647761\\
-0.0040016449646977	0.0046107413778973\\
-0.00397817267439755	0.00457726894503156\\
-0.00399019939068386	0.00458205620001794\\
-0.00399617666118063	0.00458868824217421\\
-0.00401903425636391	0.00458731642471325\\
-0.00405500677721846	0.00462496375715975\\
-0.00412045863549711	0.00462373856900511\\
-0.00417734900331046	0.00466708119920671\\
-0.00442110836114524	0.00494887558079991\\
-0.00483164894518048	0.00543494957027028\\
-0.00545618745747107	0.00605545558018452\\
-0.00634452938695394	0.00694504523098606\\
-0.00737688917409191	0.00804368875799442\\
-0.00860297217670184	0.00925691801854914\\
-0.00991584254575038	0.0106109272799758\\
-0.0114619641539311	0.0122747472614617\\
-0.012876754947015	0.0142345292795463\\
-0.0144123590696398	0.0163660619435325\\
-0.016016773937082	0.0183798694929277\\
-0.0174920030375568	0.0205746985111592\\
-0.01713053090629	0.0225791052846711\\
-0.0133756136000719	0.0236659905310159\\
-0.00430034222882562	0.0249018479511279\\
0.0116939104720748	0.0268415903980102\\
0.035216378482335	0.027217648259604\\
0.0663588597466251	0.0269971875194111\\
0.105248359618161	0.0268590607571381\\
0.150953446947431	0.0254118910735926\\
0.203690816827192	0.0233676680131207\\
0.262752420524217	0.0181427652109263\\
0.326183145096281	0.0113128131382109\\
0.394281186545766	0.00603436750471615\\
0.465208355637247	-0.0049524111812778\\
0.542475855831044	-0.0162371445294915\\
0.626418161910732	-0.0164332889858583\\
0.715999174540469	-0.0142951909911941\\
0.810423787520278	-0.0133483084101371\\
0.910048572775709	-0.0143056595945424\\
1.01445484164042	-0.0131329718326544\\
1.11869572779703	-0.0128303357868091\\
1.22638179816935	-0.0127447592178104\\
1.33245697238448	-0.00859811107491129\\
1.44139658382791	-0.00861634454747132\\
1.55170928359023	-0.00821617455881603\\
1.66555497277893	-0.00551180674490376\\
1.78340033580841	-0.00577699096982132\\
1.901138076309	-0.00294694200322526\\
2.01668419261958	4.74608163979207e-05\\
2.13424336773358	-0.00289951207851205\\
2.25039699582256	-0.00600697309026296\\
2.37027803990069	-0.00739180633388667\\
2.494219637974	-0.0105013807417679\\
2.61768245151388	-0.0131071271166303\\
2.74151639243623	-0.0173824739257015\\
2.86627502596162	-0.0250415154272412\\
2.99207777505397	-0.0380910010501901\\
3.11806537028834	-0.0514775007847373\\
3.24786623300026	-0.0581713494736401\\
3.37860174093443	-0.0706888677274701\\
3.50663343025307	-0.0858279157182961\\
3.6371536325667	-0.0972472701366352\\
3.76605890425648	-0.106563108701319\\
3.89537771065028	-0.11516815041483\\
4.02490455420131	-0.118617560728042\\
4.15306191359771	-0.121380507913208\\
4.28495039441583	-0.12280845931653\\
4.4131272329558	-0.122493781285735\\
4.54380628893909	-0.115886035583055\\
4.67134031301862	-0.0997260376421804\\
4.79492932462867	-0.0770566470880049\\
4.92004815767136	-0.0525229574925304\\
5.0426749734133	-0.0229559031667613\\
5.16408758218846	0.0135039730871678\\
5.27958952261709	0.0605647647234617\\
5.39712948270113	0.109928253414812\\
5.51440532422004	0.156636659937874\\
5.63024525834151	0.20697215965992\\
5.74309423429395	0.263082195171674\\
5.85725319290907	0.323262431934729\\
5.9712721377629	0.383811827257659\\
6.08201915250973	0.44211433638803\\
6.19706309240174	0.498910822812343\\
6.30929517032533	0.555214697521271\\
6.42140585239913	0.613911339831118\\
6.53203430691706	0.678487119498162\\
6.63956848309798	0.739226866066907\\
6.74650232563251	0.794323534881919\\
6.85309434734913	0.849836381500601\\
6.96075995977918	0.907103436471046\\
7.06953792455724	0.964564173989664\\
7.1778458103267	1.02359028848562\\
7.28907478486214	1.07892909987716\\
7.40190486770525	1.13037432258513\\
7.51186625035077	1.18201896066502\\
7.62614565327529	1.23213010400705\\
7.73803908079967	1.28214759784659\\
7.84754893889926	1.33110411220787\\
7.95718093846889	1.37666208488657\\
8.06741897367414	1.41702028096296\\
8.17759939794053	1.45414195385857\\
8.28716769594472	1.48873161254696\\
8.39539077216543	1.51964783003731\\
8.50140213968702	1.54167879076829\\
8.59979744254983	1.55867433725474\\
8.69481663401933	1.5738051403289\\
8.784964228022	1.58956612322886\\
8.8685487162298	1.60107709520195\\
8.94441052805033	1.60657467304933\\
9.01365080766615	1.61135958368202\\
9.07533631364789	1.61560807904911\\
9.12839056481216	1.61801949950707\\
9.17357210792354	1.61922697320233\\
9.20778241488074	1.61989634941204\\
9.23201760994683	1.62151180789206\\
9.24793094279027	1.62269187487398\\
9.25644762052972	1.62444037425635\\
9.26178864171396	1.62707750843622\\
9.2637814118162	1.62967363844485\\
9.26397466196395	1.6313098046899\\
9.2635695515283	1.63318376510264\\
9.262959924607	1.635189055653\\
9.26050055343164	1.63689720348422\\
9.25851009450988	1.63839982771162\\
9.25603876758261	1.63947419197701\\
9.25410853801571	1.64048720703006\\
9.25251175575484	1.64126541790145\\
9.25144091434786	1.64168957205983\\
9.25075900814933	1.64182765209988\\
9.25060876187831	1.64181238637562\\
9.25054607107374	1.64175249837659\\
9.25054901418395	1.64173185986774\\
9.25049082296123	1.64182573584587\\
9.25047715367717	1.64193999776716\\
9.25049273453673	1.64197231114744\\
9.25047853723937	1.64206090360196\\
9.25044625545543	1.64212533190375\\
9.2504619668665	1.64211910848319\\
9.25048867569463	1.64211080663775\\
9.2505041509171	1.64213091287724\\
9.25047362040631	1.64209391588056\\
9.2504305096644	1.6420532571652\\
9.25041069212886	1.64203097837218\\
9.25038582956779	1.64200796861395\\
9.25034292616562	1.64201940788822\\
9.25031957660305	1.64202003913031\\
9.25028294046754	1.64202791542441\\
9.25027687059844	1.64202572486801\\
9.25024209283712	1.64202386365655\\
9.25026826823144	1.64201231824038\\
9.25026225278583	1.64200957899892\\
9.25027250356363	1.64200994606289\\
9.250266681071	1.64199846419118\\
9.25028391742417	1.64199377361963\\
9.250262538289	1.64198316119434\\
9.25025835934023	1.6419847822127\\
9.25025746579413	1.64196672649488\\
9.25030857611288	1.64195765163767\\
9.2503407913386	1.64196075671332\\
9.25040702901152	1.64196541165326\\
9.2504425779603	1.6419829120317\\
9.25084140896116	1.64196888923235\\
9.25142224312645	1.6420982248462\\
9.25295049998587	1.64197541077519\\
9.25794062267128	1.64168654480054\\
9.26875705801955	1.64082791471881\\
9.28748421412418	1.63903325451586\\
9.31570186729328	1.63633575742149\\
9.35396600314039	1.63472801458212\\
9.40490276244721	1.63186227654239\\
9.46528024470586	1.62687575684683\\
9.53583635510757	1.61988559362491\\
9.61575326106376	1.60943020059734\\
9.70142261796248	1.59995110510971\\
9.79165363077104	1.58866393397526\\
9.88288032407644	1.57214643138389\\
9.97667942553024	1.55376669675814\\
10.0753550668132	1.53748938715095\\
10.1746166866511	1.52379758706141\\
10.2811441050324	1.50571652205691\\
10.3914736754222	1.48363672257644\\
10.5015929947603	1.46444522668809\\
10.6170797217091	1.44591301600615\\
10.7328829802264	1.42751369200134\\
10.8506109657954	1.40838773126901\\
10.9724595220891	1.39059427770646\\
11.0938570582468	1.37390642319482\\
11.2173970130595	1.35998018921816\\
11.3427713448625	1.34524118135857\\
11.4639933616538	1.32779595678834\\
11.5910507422516	1.31387382080072\\
11.7213136582956	1.29945380774123\\
11.8502670655447	1.28352742493704\\
11.9817855179115	1.27122542482536\\
12.1101041389456	1.25846511801565\\
12.2406977629523	1.24241144267876\\
12.3760231398966	1.23062705517877\\
12.5094470523416	1.22470657715516\\
12.6443884126553	1.21891990739537\\
12.7799924653766	1.21021658150317\\
12.9135539816107	1.20544703617319\\
13.0503484284216	1.2040649752512\\
13.1880208296483	1.20094376389081\\
13.3233546182681	1.20154670766975\\
13.4591584175667	1.20803829281434\\
13.5949225333203	1.21337690731593\\
13.7290017070322	1.21971593515107\\
13.8637052736797	1.23499698461786\\
13.9961167336946	1.25641481543602\\
14.1289896528746	1.28231577306095\\
14.2613825687776	1.31298227887303\\
14.3888589382315	1.34711683513207\\
14.5218297200331	1.38730428674016\\
14.6506637318569	1.42947188474825\\
14.7749227378466	1.47896088429781\\
14.8979205086719	1.53632400878001\\
15.0175958212671	1.59778123568317\\
15.1388621562201	1.66010091317191\\
15.2541057479562	1.73338289611146\\
15.3661098721842	1.81591701898454\\
15.4726053144358	1.90246607430537\\
15.5761435613946	1.99231153941791\\
15.6779563101508	2.08082868158116\\
15.7765085663889	2.1778457782653\\
15.8663674860024	2.2823986322891\\
15.9502337857409	2.39223538213668\\
16.0293935866165	2.50777916807673\\
16.1016575751607	2.61970827060563\\
16.1736060737978	2.73298432208145\\
16.2409683559766	2.85103786833708\\
16.3007253274328	2.97392123108219\\
16.3548070052648	3.09763326325894\\
16.4075128789106	3.21926314213346\\
16.4582030023846	3.34093451822254\\
16.5038693056004	3.46837811293086\\
16.5418880941823	3.59922776718389\\
16.5722135466436	3.73435679091016\\
16.6005950308892	3.86844348129196\\
16.6243232743927	3.99993935970328\\
16.6438840301607	4.13352987432718\\
16.6578062827302	4.2672406813032\\
16.6671445418413	4.40453695274579\\
16.6707293688651	4.54021929959789\\
16.6714237693954	4.67317501664795\\
16.6680142047858	4.80707109565718\\
16.6594395074225	4.94161952399884\\
16.6490210993222	5.07681985493497\\
16.6341501128229	5.21222208217043\\
16.6171241700261	5.3447591800358\\
16.599046471706	5.47392171946008\\
16.5757071775535	5.60441024190898\\
16.5474522102481	5.73539330494972\\
16.5168159363755	5.86440988993181\\
16.4817942863559	5.99367715132427\\
16.4424370585272	6.12078599878202\\
16.4003936556137	6.24554565293671\\
16.3552283870017	6.37126594018821\\
16.3049360108457	6.49351014483119\\
16.2502437901382	6.61216001143138\\
16.1964417707759	6.73048170508476\\
16.1395713312637	6.84764935884045\\
16.0798645973678	6.9651023948386\\
16.0172291087945	7.07849610858098\\
15.9505127646269	7.18850495793284\\
15.8812228108286	7.29633318991668\\
15.8121036747558	7.40310464573685\\
15.7390468831284	7.50746755072875\\
15.6612373069975	7.60942533381729\\
15.5819264635917	7.71269795820961\\
15.4994093431756	7.80948903731251\\
15.4163499757416	7.90426488631477\\
15.3354081887654	7.99931549429373\\
15.2522039207761	8.09192353506612\\
15.1658398053246	8.17540708716605\\
15.0783006558137	8.25262887379566\\
14.9943006490147	8.32636567717311\\
14.9158959910506	8.40212137030252\\
14.8355797153284	8.46994959045886\\
14.7485005420541	8.53226633046858\\
14.6623213745983	8.59387746633567\\
14.5770863861545	8.65637085241178\\
14.4921665033328	8.7118226441906\\
14.4043721444894	8.76195170161644\\
14.3180314666358	8.81205086601585\\
14.2332685342467	8.85837235624441\\
14.1480297032657	8.89754032165848\\
14.0674615318587	8.92865627301121\\
13.990664695405	8.95537233922407\\
13.9173182865316	8.97941381372292\\
13.8485934900534	9.00000784606318\\
13.7824981946652	9.01737996819088\\
13.7216823487106	9.03086723942614\\
13.6644283661557	9.0392976381526\\
13.6155394905903	9.045543459099\\
13.5750175051401	9.05190790379228\\
13.5422303171644	9.05625822861468\\
13.5168203652657	9.05740145429537\\
13.4985305433862	9.05897005644958\\
13.486100148344	9.06091618496389\\
13.4778586150701	9.06197859041248\\
13.472971041419	9.06178902371255\\
13.4701643542022	9.06214907169022\\
13.4692672601274	9.06294233525307\\
13.4692647281559	9.06270572382741\\
13.4697425884381	9.06223065048548\\
13.4698598556491	9.06211659247461\\
13.4695992222452	9.06177361082215\\
13.4690065150716	9.06113498955688\\
13.4679828073941	9.06074996503337\\
13.4671463719493	9.06053084160427\\
13.466381654897	9.06013864191982\\
13.4657147340284	9.059902042399\\
13.4648436009525	9.05965000212206\\
13.4641549187424	9.05931312021426\\
13.4636186597998	9.05910832051968\\
13.463435956788	9.059087183208\\
13.4633782956286	9.05879495670787\\
13.463309534724	9.05875224784219\\
13.4632825622388	9.05876049318771\\
13.4632542318601	9.05847631100894\\
13.4632574240701	9.05850125610002\\
13.4632798842914	9.05851826951383\\
13.4633326879677	9.05841193930063\\
13.4633507023809	9.05852965459937\\
13.4633533856229	9.05868708451989\\
13.463338011982	9.05868738186939\\
13.4632763117668	9.0588234716843\\
13.4632370495206	9.05890952860264\\
13.4631911559879	9.05891045140912\\
13.4632119968526	9.05895915435102\\
13.4632202044357	9.05895753678758\\
13.4631951054619	9.0589104926545\\
13.4631827425723	9.05889840908532\\
13.4631840744636	9.05890189894758\\
13.463223597501	9.05884286207664\\
13.4632233103903	9.05881956845674\\
13.4632249020451	9.0587849155505\\
13.4632563350171	9.05875407281572\\
13.4632740588249	9.05874876042301\\
13.463290278308	9.05876647895255\\
13.4632934494669	9.0588009144232\\
13.463244753047	9.058805766478\\
13.4631985206491	9.05882744805644\\
13.4633187485655	9.05875086159461\\
13.4629224621753	9.05910446440041\\
13.4602393949048	9.05979969159865\\
13.453119412728	9.0599315919168\\
13.4394481529651	9.0600539939569\\
13.4176572548676	9.06140228116975\\
13.3855356678185	9.06152347168285\\
13.3427863081361	9.05882369351287\\
13.2900122036728	9.05528958042339\\
13.2270463509498	9.05067041372384\\
13.1557740775883	9.04355539889044\\
13.079793433591	9.02982126899012\\
12.9985593687419	9.0096271048649\\
12.9136038296545	8.98896282588228\\
12.8269211538492	8.96272112904956\\
12.7392505481338	8.92870659726051\\
12.6483443244265	8.88913483368786\\
12.5551277967114	8.8453839727284\\
12.4602598444806	8.79384463813354\\
12.368261454597	8.73488059989262\\
12.2769914099883	8.67289609875325\\
12.1854488622084	8.61078106503657\\
12.0950382642891	8.53977533686385\\
12.0055266719867	8.4618942129087\\
11.9171486901297	8.37953326175558\\
11.8323147944693	8.29156204626061\\
11.7493754347791	8.20230756382731\\
11.6666391298933	8.11259071088053\\
11.5841763463833	8.01789382123911\\
11.5041419524301	7.92244229738386\\
11.4204859222238	7.82608896581356\\
11.3348734797012	7.73250866237616\\
11.2508445214749	7.64337198521719\\
11.1643936685241	7.55731104731318\\
11.0702886265564	7.47116382135284\\
10.974376735735	7.3886533327033\\
10.8799067874887	7.30504465131841\\
10.7814575749445	7.22171309585684\\
10.6779886551929	7.1468890634001\\
10.5722784462175	7.07047287062497\\
10.4630711259327	6.99326334892379\\
10.3492209550645	6.92237204082993\\
10.232294651634	6.8565369227511\\
10.1155537476475	6.79075188756224\\
9.99794506287057	6.72692567508149\\
9.87675086212485	6.66680797579755\\
9.75318685163088	6.61132365226119\\
9.62710413553661	6.55924465823375\\
9.50002843380128	6.51346701904514\\
9.37306766025639	6.46798400768186\\
9.2429874543177	6.42327699225196\\
9.10910463922886	6.38644740660042\\
8.97411312852411	6.35246092163036\\
8.83572766775398	6.31803465473672\\
8.70070913531952	6.28463135099351\\
8.56511876103578	6.25751710129093\\
8.42517920363083	6.23455786676026\\
8.28450625312795	6.21184679637219\\
8.14841722489797	6.19273341884587\\
8.00783753631045	6.17500188583927\\
7.8700977327396	6.15628579188258\\
7.73462237631482	6.14200703177832\\
7.59351196492898	6.13131609534772\\
7.45443487026879	6.12179950239596\\
7.3165978014135	6.1077792905897\\
7.17788348162871	6.09499295653531\\
7.03799861313196	6.08765367586544\\
6.89814751742361	6.08066907311093\\
6.75799565673137	6.07188642492231\\
6.61588169929492	6.06479581277499\\
6.47643913324609	6.05801415554759\\
6.33715302940159	6.04866167845097\\
6.19348066443017	6.04619991369662\\
6.05211156379989	6.04396746032669\\
5.90967716316949	6.03637408171243\\
5.76364751818759	6.02600508616127\\
5.62051026521014	6.0203496582249\\
5.47586615361119	6.01954642140475\\
5.32892109662997	6.0184431180986\\
5.18120115938864	6.01437710539688\\
5.03394643064646	6.00777154017654\\
4.8846774925337	6.00250673343391\\
4.73731409082565	6.00169567975507\\
4.59125046057075	6.00345759339506\\
4.44208012516938	6.00025000478072\\
4.29427658547132	5.99289992721384\\
4.14739662157714	5.98547883344516\\
4.00104235541819	5.98475687641608\\
3.85466139537104	5.98468971827389\\
3.70736263562793	5.97969799703192\\
3.55797294375675	5.97469035574344\\
3.40939957048286	5.97369909083288\\
3.26442632841279	5.97131933068691\\
3.11781874859768	5.97145036874881\\
2.97083938837511	5.96788518397652\\
2.82611507291501	5.96023044230304\\
2.67756854634337	5.95583235656968\\
2.53323131626212	5.95475309078086\\
2.39250319765962	5.95177296756577\\
2.24907031077274	5.94288561248268\\
2.10869794290739	5.93520992508363\\
1.97173137770089	5.93073242360556\\
1.83538434261827	5.92710041673153\\
1.70172976378748	5.92142653239001\\
1.56923227070891	5.91543765807853\\
1.4394624182375	5.90656192779744\\
1.31287553384125	5.90046380607169\\
1.19106254580116	5.89501382142563\\
1.0764856646797	5.88882617154235\\
0.962545612598274	5.8852006299273\\
0.855220834226946	5.87988391307629\\
0.756708871438657	5.87129866708636\\
0.664275347399113	5.86412998994399\\
0.578846594783561	5.86038430256742\\
0.504191956376299	5.85625558987713\\
0.436687600722196	5.85165076641582\\
0.380113002376999	5.84822787364657\\
0.332739434383852	5.84549030580905\\
0.292993356782038	5.84303365367697\\
0.263604738789397	5.84072293432662\\
0.240737165210177	5.83899378364481\\
0.2253677489733	5.83803222403294\\
0.214270074977396	5.83720034287128\\
0.206170322405864	5.83653609641065\\
0.200405729393371	5.83661590580498\\
0.196473622057645	5.83673711580015\\
0.194240422674213	5.83628573337145\\
0.193208701278066	5.83634211391775\\
0.192878024572207	5.83668569998948\\
0.192778691529331	5.83656872118827\\
0.192657261505016	5.83655054464418\\
0.192414638098174	5.83671562126093\\
0.192197641576818	5.83657164122002\\
0.192027646031834	5.83637716781703\\
0.192089410119591	5.83641880118548\\
0.192193632031215	5.83632839784038\\
0.192265271841128	5.83622691220537\\
0.192220794151321	5.83628379152211\\
0.192218028862463	5.83626702611131\\
0.192223693482835	5.83626109631625\\
0.192206797203417	5.83632699495828\\
0.192162558853579	5.83635327173616\\
0.192138277180388	5.83634079055766\\
0.192089069417325	5.83639207149698\\
0.192056367511352	5.83639269507781\\
0.192027256601988	5.83634212925286\\
0.192021666224881	5.83632969874173\\
0.192027814705353	5.83630908418187\\
0.192039891841005	5.83626165089078\\
0.19199409826531	5.83625955581342\\
0.192017118600983	5.83626081777702\\
0.19205382235145	5.83626385289706\\
0.192065393916505	5.83626106949741\\
0.192094363211912	5.8362383461099\\
0.192091925462566	5.83624180024431\\
0.192091327218238	5.8362433075822\\
0.192065659427502	5.83621889398323\\
0.19205628157254	5.83620775694055\\
0.192021351830379	5.83622880567102\\
0.191994890485469	5.83625864898512\\
0.192033748418168	5.83628123872977\\
0.19203080493265	5.83627702984269\\
0.192028040969289	5.83629153744254\\
0.191977744613689	5.83628773607425\\
0.191988544903619	5.83627653310999\\
0.19197579785653	5.83628721900115\\
0.191959675810059	5.8363037829727\\
0.191969658825357	5.83631029491055\\
0.191983089094947	5.83633946692264\\
0.19201818666682	5.83634548065676\\
0.192011862734825	5.83634327834629\\
0.191982988146857	5.83642662911672\\
0.191955138864175	5.83646416071095\\
0.191970744467447	5.83647541624309\\
0.192019873689103	5.83654872061129\\
0.192031763210394	5.83656331717799\\
0.192034611132813	5.83647132811937\\
0.192072949418045	5.83645145042756\\
0.192081903725611	5.83641442006252\\
0.192077982749621	5.83636976581086\\
0.192067103188284	5.83636698412639\\
0.192073817942831	5.8363523401119\\
0.192053987030156	5.83632781247386\\
0.192067730000451	5.83633205085277\\
0.192087667223252	5.83634812697968\\
0.192146931393284	5.836317025711\\
0.192133408993539	5.83634257771984\\
0.192180920583757	5.83649727926139\\
0.191528800174748	5.83650130580017\\
0.190267261436765	5.8362086433566\\
0.184637368496135	5.83683128782321\\
0.17250208292248	5.83608579522434\\
0.150973757157279	5.83344780477957\\
0.117461507083797	5.83023788496557\\
0.0721609589167941	5.82693513152253\\
0.0168377300820675	5.820953289644\\
-0.0513025638468028	5.81397376929278\\
-0.128458052723002	5.80700320285275\\
-0.214047046161554	5.7995435793628\\
-0.307222273019431	5.79027646031215\\
-0.408082305316133	5.78019743038333\\
-0.513130132125113	5.77032489679014\\
-0.62624813643627	5.7595454928174\\
-0.744592188718582	5.74551523174158\\
-0.869426618997852	5.73413044497448\\
-0.997422927103059	5.72528965045686\\
-1.12455751606655	5.71458499294802\\
-1.25091535988869	5.70493695264556\\
-1.38007133869957	5.69477822438606\\
-1.50938597127305	5.68297042760668\\
-1.63877619102581	5.67213449969741\\
-1.77464319243149	5.66169321364343\\
-1.9128440399898	5.65545559801821\\
-2.05116389407095	5.64730885946172\\
-2.19148727017903	5.63651656410109\\
-2.33421964904621	5.62611336130674\\
-2.47302745144498	5.62065137434931\\
-2.61230192404688	5.61697352648635\\
-2.75267693996776	5.60709281430635\\
-2.88956785144703	5.59673947129324\\
-3.0291517694222	5.59027970090562\\
-3.16973235409861	5.58527627299929\\
-3.31125306700997	5.57730772478058\\
-3.45358317387205	5.56873712655363\\
-3.59622509970132	5.56102894766276\\
-3.73954316637866	5.55317146377811\\
-3.88352290170339	5.54445484549892\\
-4.02615743464784	5.54175633884196\\
-4.16976407797467	5.5358000083324\\
-4.31126037333373	5.52482495120206\\
-4.45891165491369	5.51876873704534\\
-4.60420434774172	5.51612632683563\\
-4.7484472952641	5.5132058764837\\
-4.89548342867606	5.50193233286841\\
-5.04348659046376	5.49207674232509\\
-5.18811845931405	5.48713873449133\\
-5.33679053036774	5.48548079593594\\
-5.48485969228556	5.48189137424147\\
-5.63204579328475	5.47343881051361\\
-5.77969828136424	5.46291974589439\\
-5.92620755101303	5.45827307612996\\
-6.07164935625616	5.45453637491025\\
-6.22216757636865	5.45205008455213\\
-6.37200116630795	5.44575908646512\\
-6.51969615217807	5.4353132705985\\
-6.66950617255401	5.42766597850916\\
-6.81735286420405	5.42490765843028\\
-6.96578395920413	5.42291037650871\\
-7.11748348836652	5.41440348605005\\
-7.26970403303207	5.40098260200454\\
-7.42089698610081	5.39567723311886\\
-7.57236446615009	5.39511448272837\\
-7.72500227544779	5.38898205393461\\
-7.87744282370555	5.37918988851715\\
-8.03306151445366	5.37265793336039\\
-8.18523609368713	5.3707832118316\\
-8.33493468519019	5.36643076470396\\
-8.48816429268027	5.35906957622685\\
-8.63958134138863	5.35103266803356\\
-8.79237188600173	5.3463056105526\\
-8.94448495367411	5.34006091328896\\
-9.09289772626449	5.33472682445092\\
-9.24493678963999	5.32491240559329\\
-9.39975994652957	5.31393306662686\\
-9.54986708871186	5.30731784759526\\
-9.70060437925911	5.30177661081584\\
-9.85271657336519	5.29191092904254\\
-10.0007957114571	5.28118576503335\\
-10.1525685096271	5.2710908288637\\
-10.3008884678805	5.26166382031407\\
-10.4456696699597	5.25008883659091\\
-10.5923503846985	5.2341162273825\\
-10.7378716566918	5.2173363980478\\
-10.8824282657979	5.20450395657684\\
-11.0272756689425	5.18818434437143\\
-11.1693334882543	5.16700812008829\\
-11.3094887196801	5.14473382907937\\
-11.4487183358903	5.11999713041633\\
-11.5917949385453	5.0961900756865\\
-11.7325106113252	5.07157742248859\\
-11.8736934490823	5.0417857832085\\
-12.0138917930326	5.00374397838384\\
-12.1499519656331	4.96455073496946\\
-12.2858838840365	4.92161942429734\\
-12.4206352522558	4.87503858304951\\
-12.5528302932694	4.8275956344368\\
-12.6800743159332	4.76884219284885\\
-12.8040806552427	4.70743218967958\\
-12.928706828419	4.64522942057427\\
-13.0504853872525	4.57835722791146\\
-13.1691507148802	4.50864205386596\\
-13.2865857196511	4.43132275178574\\
-13.3997784996304	4.35192135961469\\
-13.5090610362673	4.27029395248132\\
-13.6164539580017	4.1828486701697\\
-13.7224359104157	4.0943487687301\\
-13.8247141158087	4.00480284311215\\
-13.9273702145182	3.91458526933169\\
-14.0253198230477	3.82488516976469\\
-14.1251649602036	3.73678743433979\\
-14.2228014060626	3.6473091620237\\
-14.3211684933456	3.56081159769459\\
-14.4191429533553	3.47436563448848\\
-14.518138722825	3.39506606114366\\
-14.6227284137722	3.32159219671099\\
-14.7287344659322	3.2502198537676\\
-14.8319773033408	3.18182123030033\\
-14.9394154185512	3.11451851939459\\
-15.0531960209555	3.0532189166384\\
-15.1628592525349	2.99419076554395\\
-15.2776844382139	2.93838249338209\\
-15.3931296796979	2.88734731506338\\
-15.5073712304988	2.83954337827642\\
-15.620894795901	2.79316573975845\\
-15.7321162491215	2.75511248432896\\
-15.8396714345892	2.72539150636919\\
-15.9446292741432	2.69522901740752\\
-16.0492503237197	2.66834475834248\\
-16.1502624126818	2.64547359040358\\
-16.2468427151435	2.6235972675644\\
-16.3377466511541	2.60661381423737\\
-16.4253224098355	2.59531762795035\\
-16.5048692209586	2.5844331269779\\
-16.5793164795662	2.57431297617609\\
-16.647125246511	2.56882631616587\\
-16.7062349640419	2.56597558618037\\
-16.7552467221284	2.56112665655648\\
-16.7942569923228	2.5578968713733\\
-16.8230189687621	2.55769051742551\\
-16.8432141136712	2.55646306868059\\
-16.857105351206	2.55538628559035\\
-16.8667224981701	2.55676500952284\\
-16.8739337510723	2.55808592815465\\
-16.8794841771865	2.55873660855177\\
-16.8833345275572	2.55994573257958\\
-16.8860381783037	2.56159095576183\\
-16.8874470831396	2.56329929908368\\
-16.887742193971	2.56495738393393\\
-16.8879815107101	2.56663989064887\\
-16.8878845192262	2.56829405770246\\
-16.8880842839073	2.56975949997548\\
-16.8883507113728	2.57109798816129\\
-16.8887934888046	2.57216656605684\\
-16.8891737910379	2.57336281444146\\
-16.8896233114219	2.57454255101965\\
-16.8897521150168	2.57541908171845\\
-16.8898322839442	2.57623821335329\\
-16.889745160578	2.57709216514254\\
-16.8897442289093	2.57762692538957\\
-16.8897034596198	2.57805129613497\\
-16.8897591520427	2.57840344964781\\
-16.8897233185884	2.57851518831848\\
-16.8897614941948	2.57859236698521\\
-16.8897185722177	2.57870193980281\\
-16.889689939499	2.57871789819091\\
-16.8896449994401	2.57874705466066\\
-16.8896754536475	2.57879635440201\\
-16.8896766393635	2.57877576857469\\
-16.8896583187586	2.57877298661095\\
-16.8896377602304	2.57878999222955\\
-16.8896265823937	2.57879970719254\\
-16.8896293364579	2.57877682650643\\
-16.8896218298523	2.57877743566601\\
-16.8895749449707	2.57876963696298\\
-16.889568898678	2.57876408544163\\
-16.8895902885903	2.57875709820867\\
-16.8896312926592	2.57873361775927\\
-16.8896484467459	2.57874356071093\\
-16.889655922689	2.57873572490657\\
-16.8896314319052	2.57868633506163\\
-16.8896681133853	2.57866040258414\\
-16.8898006130585	2.57858669202766\\
-16.8905756391323	2.57860393229322\\
-16.8923151681515	2.57869193191294\\
-16.8976956821392	2.57792808312864\\
-16.9089862693849	2.57707605518932\\
-16.9279349436601	2.57446329655217\\
-16.9560660014578	2.56949065634764\\
-16.9938234652911	2.56411948552449\\
-17.0413227678388	2.55790224956308\\
-17.0979729758774	2.54937983738815\\
-17.163316582185	2.53771258249646\\
-17.236820215967	2.52516431832639\\
-17.3177174460584	2.51212577728499\\
-17.4045342255223	2.49939873788995\\
-17.4960047621209	2.48611764614644\\
-17.5922447753565	2.46904708338596\\
-17.6945617193809	2.44936428175797\\
-17.8016481820209	2.4320320483926\\
-17.9110894890666	2.41520852246721\\
-18.0207249786857	2.39501127319995\\
-18.1294537474674	2.37010772708401\\
-18.2421361564129	2.34055834309476\\
-18.3559655053926	2.30990135300324\\
-18.4703220892696	2.27353249844683\\
-18.5848908546111	2.22942867115659\\
-18.6974867811721	2.18267405791444\\
-18.8064264881713	2.13398232752445\\
-18.912036185135	2.0784557241892\\
-19.0102327925961	2.01269594121921\\
-19.1076564344779	1.94137177644199\\
-19.2046480984412	1.86602811880957\\
-19.294819318223	1.78341283313786\\
-19.3774851559884	1.69399272739932\\
-19.4556186300384	1.60167559891559\\
-19.531481552923	1.5024486672399\\
-19.5995506585858	1.39988011747814\\
-19.6611794737827	1.2911806631594\\
-19.7123338080165	1.1761398314181\\
-19.7559106581905	1.06126854877971\\
-19.7993214917541	0.945648376129782\\
-19.8380073543923	0.828664589022003\\
-19.8647525503513	0.708852225043078\\
-19.8840766184833	0.58580674281244\\
-19.8985301535263	0.4630603253744\\
-19.9095625966528	0.341442286980434\\
-19.916202556449	0.219538583101421\\
-19.9152658473629	0.094345081587789\\
-19.9071180124539	-0.0307132724639111\\
-19.8933263399812	-0.153353247569208\\
-19.8740846394168	-0.277544868229196\\
-19.8502184821226	-0.398750500483338\\
-19.8246890817996	-0.518765606516356\\
-19.7913519001527	-0.638980069221455\\
-19.747278820008	-0.75376273748963\\
-19.6967323269565	-0.866942735599506\\
-19.6463027356185	-0.980302871433052\\
-19.5916581392254	-1.09001016901715\\
-19.5298989865015	-1.19716369971428\\
-19.4604823085608	-1.30033691019711\\
-19.3839304724689	-1.39791132193262\\
-19.3044885071713	-1.492624749267\\
-19.2212673352365	-1.58110704241762\\
-19.1323906540324	-1.66583385134984\\
-19.0408598089465	-1.74863058597174\\
-18.9452094517538	-1.82322922646955\\
-18.8430146969619	-1.89352053342814\\
-18.740187181012	-1.95811644580365\\
-18.6337407576425	-2.02030779177414\\
-18.5220644489562	-2.07655707410579\\
-18.4075035849489	-2.12574070247495\\
-18.28852684374	-2.16929811640854\\
-18.1662022858575	-2.20642072981896\\
-18.0455901591807	-2.23526895178916\\
-17.9194376441778	-2.26067671286219\\
-17.7925301616184	-2.28022445648651\\
-17.6634004577104	-2.294191933974\\
-17.5350112610869	-2.29947189390461\\
-17.4070907452283	-2.29934676625181\\
-17.2790007589517	-2.2965556182463\\
-17.1519960874774	-2.29280075365218\\
-17.0195330490404	-2.28814496882084\\
-16.8905516601539	-2.27591249639827\\
-16.7604968952462	-2.25963393271291\\
-16.631291432649	-2.24108795695927\\
-16.500622093709	-2.22290521066557\\
-16.3690365105924	-2.2072739156375\\
-16.2419236931027	-2.18500602526075\\
-16.1077813735065	-2.15613843469059\\
-15.9792008446619	-2.13235033183341\\
-15.8470246911112	-2.10923299186298\\
-15.7156089499173	-2.08547359096646\\
-15.5853446232215	-2.06061116979571\\
-15.4528641440805	-2.03282228508533\\
-15.3216703523965	-2.00298563679159\\
-15.1910006223904	-1.97490245806323\\
-15.0602201620703	-1.9502050354226\\
-14.9293216382364	-1.92022495564558\\
-14.7985979840155	-1.88727803412142\\
-14.6677633251412	-1.8562671775705\\
-14.5380207035069	-1.82804773113092\\
-14.4083437420802	-1.80184830031135\\
-14.2787733561742	-1.77504996571519\\
-14.1507364328908	-1.74383854595525\\
-14.0207137377463	-1.71073515175093\\
-13.8925569554473	-1.68169082104958\\
-13.761726994518	-1.65619034510569\\
-13.6293884320535	-1.6296400977796\\
-13.4974177501548	-1.59948277048558\\
-13.3677639543776	-1.56631229139042\\
-13.2390933387295	-1.53380886206797\\
-13.110019614334	-1.50839163486185\\
-12.982924192846	-1.4826480886422\\
-12.8532301598594	-1.45089828469383\\
-12.7246347719329	-1.4197579384602\\
-12.5971791340563	-1.39037008586078\\
-12.4673215643662	-1.36083541255139\\
-12.3382155978623	-1.33694393770552\\
-12.2070182418999	-1.30890980335578\\
-12.0794642997285	-1.27404948695515\\
-11.9489657349855	-1.23993453967464\\
-11.8195787654616	-1.21219559528404\\
-11.6923946600391	-1.18502684261564\\
-11.5670411504254	-1.15236806024461\\
-11.4429988691747	-1.12242464242604\\
-11.3240232647005	-1.09430060882595\\
-11.2046215058369	-1.06516797450544\\
-11.0924563104531	-1.04108031897572\\
-10.9782041864308	-1.01838137067365\\
-10.8663119171102	-0.990561149729751\\
-10.7589394446463	-0.963349871735852\\
-10.6569957962752	-0.940703771774692\\
-10.563937897853	-0.919206027922718\\
-10.4759820901314	-0.898761204839082\\
-10.3972908337602	-0.882957914183179\\
-10.330375639635	-0.869936680898707\\
-10.2724034066117	-0.856533550869932\\
-10.2260623729326	-0.846036539699312\\
-10.1897783975209	-0.837392253899091\\
-10.1612857122825	-0.831452506262806\\
-10.1417296690856	-0.827035138396354\\
-10.1280661097699	-0.824345456515281\\
-10.1188064623019	-0.822803784661156\\
-10.1130953199586	-0.821714900533247\\
-10.1091912662481	-0.821361811651817\\
-10.1072134830779	-0.821352497160289\\
-10.1063281189331	-0.821417064845345\\
-10.1062628296248	-0.821621253257378\\
-10.1064906979024	-0.821885842973069\\
-10.1069694581503	-0.82224023169958\\
-10.1073322408246	-0.822425121399114\\
-10.1074668069244	-0.822537537043669\\
-10.1074542969894	-0.822632812485737\\
-10.1074368337898	-0.822663344466892\\
-10.1073949770432	-0.822633311892085\\
-10.1072868856298	-0.822625273197502\\
-10.107277334749	-0.822623566988714\\
-10.1073268204879	-0.822595115867721\\
-10.1073522071708	-0.822589751519583\\
-10.1073861987275	-0.822591780634284\\
-10.1074771527301	-0.822557641943454\\
-10.1075896863093	-0.822561655768572\\
-10.1075769145613	-0.822563993001148\\
-10.1075406978211	-0.822559494601024\\
-10.1075239092873	-0.82255748185518\\
-10.1074922664978	-0.822590723171904\\
-10.1074981457356	-0.822588780744828\\
-10.1075256077492	-0.82260630011713\\
-10.1075221681079	-0.822622361096856\\
-10.1075040370553	-0.82261382909641\\
-10.10748727266	-0.822648416349649\\
-10.1074890335853	-0.822700708838657\\
-10.1075070011922	-0.822720181418125\\
-10.1075182783712	-0.822733799511384\\
-10.1075312082149	-0.8227510781509\\
-10.1074725342754	-0.82280299134352\\
-10.1059600237561	-0.82299303005442\\
-10.1014091050906	-0.821526922779003\\
-10.0903541444258	-0.819162681193516\\
-10.0702416467413	-0.81436649613715\\
-10.0398904269262	-0.805668475801171\\
-9.99724062479655	-0.794211617053428\\
-9.94293348932661	-0.78050343547816\\
-9.87542112098584	-0.765366863250363\\
-9.79814755837289	-0.748987646364983\\
-9.71189134743076	-0.729496272147711\\
-9.61783891972089	-0.70523442394262\\
-9.51857316705396	-0.679666670281398\\
-9.41538921923652	-0.656401819102942\\
-9.30933650656943	-0.631907081640096\\
-9.19968131870841	-0.603149421143961\\
-9.08204220678033	-0.5766202985443\\
-8.96556051031865	-0.554039647562945\\
-8.84427894723264	-0.524545632434934\\
-8.72452268553704	-0.492228342207594\\
-8.6036828041501	-0.464887831854657\\
-8.47897423876673	-0.438172660935063\\
-8.35541570036433	-0.410831293517628\\
-8.23148937956161	-0.382570068230915\\
-8.1060723174697	-0.355792027585466\\
-7.98503488745689	-0.326612810495622\\
-7.85862656266228	-0.294717625178855\\
-7.72990527034049	-0.269933314307017\\
-7.60061076093677	-0.246224951308284\\
-7.47022168501084	-0.213624004430629\\
-7.33970457175974	-0.179554385128228\\
-7.21008465001023	-0.152104942862888\\
-7.07744321200762	-0.125095787125837\\
-6.94472399395039	-0.0982988884584991\\
-6.80970248079524	-0.070102025589887\\
-6.6750762686454	-0.0384495053591518\\
-6.54194783455209	-0.00683222139056115\\
-6.40721547553335	0.0204337553849392\\
-6.27050490270175	0.0456691275951886\\
-6.13292163576028	0.0766029810009343\\
-5.99837699657556	0.111424754215601\\
-5.86748677013673	0.144453291130636\\
-5.73426589745025	0.176733407655623\\
-5.59880503415058	0.204926382270445\\
-5.46727796740665	0.233173391775473\\
-5.3345024050363	0.267177248436844\\
-5.19955261010616	0.300264439773778\\
-5.06723688059976	0.331145227824994\\
-4.93241103189932	0.357819979943867\\
-4.80206080557779	0.386437048647827\\
-4.66926192813861	0.419865890694154\\
-4.54037551038165	0.455959483957998\\
-4.40934349962551	0.486089807062471\\
-4.27951825198112	0.508104441510145\\
-4.15216850696719	0.53317338802407\\
-4.02390058217803	0.561749677692947\\
-3.89893465669151	0.586741210757513\\
-3.77232817312292	0.609911819559437\\
-3.64730686486922	0.63057577883909\\
-3.51931557954242	0.646581167805286\\
-3.38981845400159	0.664996467991612\\
-3.26127578589177	0.686657615709844\\
-3.13034495402016	0.707483996709437\\
-2.99962948851108	0.721599368641348\\
-2.86992634002333	0.728582414095892\\
-2.73721809665444	0.739332727621004\\
-2.60766425495171	0.750104828203993\\
-2.47834314750955	0.756539728066056\\
-2.34583016820923	0.75947347144413\\
-2.21732650575343	0.759675880864654\\
-2.08639429705994	0.7587403791709\\
-1.95519572601096	0.756798060750315\\
-1.82451649474835	0.754552090441267\\
-1.69464513507949	0.751119407765913\\
-1.56577809525865	0.742005415026537\\
-1.43676589061439	0.728348511043674\\
-1.30980023338614	0.717073405398076\\
-1.18393895354155	0.708697618407748\\
-1.05900737828149	0.693105873983302\\
-0.937096452559004	0.673954289009285\\
-0.820586634006304	0.656878916044376\\
-0.700953844395633	0.63994263202884\\
-0.585830832893412	0.623731815038984\\
-0.473778843305896	0.605435461652165\\
-0.367130262318813	0.58829320967983\\
-0.263429412646291	0.573460611744371\\
-0.168995701973718	0.561781275314086\\
-0.0802119712231269	0.55181782316043\\
0.00196259870201621	0.547107990632065\\
0.0793110758995115	0.545766563859418\\
0.150213475555143	0.544935410835604\\
0.213611322361117	0.547431533011075\\
0.266067827077746	0.553694631391817\\
0.308304183466038	0.558935869594833\\
0.338821048874056	0.563958085025466\\
0.360628663659536	0.568526394657454\\
0.375832969221941	0.571927512533977\\
0.386710164767274	0.5750536229208\\
0.395281977649544	0.57605526931106\\
0.400927223902961	0.577405749055991\\
0.405869362565782	0.578950461697472\\
0.409953604400958	0.578856184487236\\
0.412060476870128	0.579103870667439\\
0.412305721565966	0.579097950790085\\
0.411868954103004	0.578330509467443\\
0.411374188410234	0.577880896819786\\
0.410898846550918	0.577784099618098\\
0.410946829710643	0.577443283360592\\
0.410986751740597	0.577459783979815\\
0.411126698463146	0.577647903860697\\
0.411317826187041	0.577650165281403\\
0.411358523330154	0.577716657182636\\
0.4113869036374	0.577800906413886\\
0.411473618802465	0.57774551755352\\
0.411457393289283	0.577823465893689\\
0.41151117225884	0.577916806472097\\
0.411512455871821	0.57791705146631\\
0.411532956402455	0.577952798698104\\
0.411512049602722	0.577977223771869\\
0.411528500090786	0.577949794490501\\
0.411534938979846	0.577922610012269\\
0.411558477046665	0.577891324249795\\
0.411611464326033	0.577782915972485\\
0.411615025373077	0.577731451153643\\
0.411593328652256	0.577706999809956\\
0.411613880476581	0.577626135808754\\
0.411566748321058	0.577586560503794\\
0.411544878253334	0.577620742974867\\
0.411516624031199	0.577575032941944\\
0.411517930006304	0.577570110565303\\
0.411578399644241	0.577512287494666\\
0.411595095529217	0.577466441587206\\
0.411605680888979	0.577460304734264\\
0.411606062567954	0.577440409357378\\
0.411618295352803	0.577417624902671\\
0.411648195900969	0.577426674933932\\
0.41166685688162	0.577440946318967\\
0.411690227174573	0.577453011040394\\
0.411693321777629	0.577466942828084\\
0.411681862836751	0.577474810579435\\
0.411694768874393	0.57751574479305\\
0.411732981989668	0.577547947074371\\
0.411737680103852	0.577552268641736\\
0.411764334356131	0.57757082210455\\
0.41173056410406	0.577581112337578\\
0.411723681590242	0.577640362569583\\
0.411723409556966	0.577657376161157\\
0.411716040496342	0.577660829635377\\
0.411700747935951	0.577662833821857\\
0.411695445224028	0.577659069927428\\
0.411695898433635	0.577665173966188\\
0.411727169420266	0.57765400660157\\
0.41174603864449	0.577665567973183\\
0.411756976114258	0.57765924432787\\
0.411765538143698	0.577629629238749\\
0.411765816284174	0.577622079084756\\
0.411764107267364	0.577597206413899\\
0.4117724473484	0.577570754027016\\
0.411764865604992	0.577544520860951\\
0.411768310362079	0.577537646398383\\
0.411766536160633	0.577539535448562\\
0.411771008574618	0.577557599242415\\
0.411769404465588	0.577568390267755\\
0.411785618588373	0.577662796127751\\
0.411793892575265	0.577981746273396\\
0.411825175267394	0.578488279513507\\
0.411800122460608	0.579153413751195\\
0.412274317990363	0.580139255490988\\
};
\addplot [color=white!50!blue,solid,line width=1.2pt,forget plot]
  table[row sep=crcr]{%
0.0103269389045126	-0.0336058328639577\\
0.278501789313486	-0.489906141589696\\
0.87749219437659	-1.21624250065453\\
1.11002403027368	-1.12285157681853\\
0.236314881988431	-0.1614294835538\\
-0.00397140892669651	0.00457826309944895\\
-0.00385065029903127	0.00462123317922545\\
-0.00378682509306162	0.00463463813091785\\
-0.0038074453593265	0.00462647572328424\\
-0.00383703220824572	0.00464188919059538\\
-0.00384672332672233	0.00466634399517023\\
-0.00385491589714484	0.00466420406674007\\
-0.00385581603152646	0.00468880068059153\\
-0.00386602158423475	0.0046999517858656\\
-0.00389226345218809	0.00470178959877942\\
-0.00390561150997635	0.00468993728131046\\
-0.00396259582380767	0.00463665065591376\\
-0.00399555681061864	0.0045828079544999\\
-0.00399690419136082	0.00457609661836634\\
-0.00402294171754303	0.00460315490248991\\
-0.00406255722176316	0.00460156446717354\\
-0.00404601777895617	0.00461232105548733\\
-0.00410903777507603	0.00465619072567736\\
-0.00411891627082658	0.00465566564470737\\
-0.00411466731667848	0.00463970759044145\\
-0.0041101756396055	0.00464126795239423\\
-0.0040881330043217	0.00464008907172103\\
-0.0040263407232255	0.00462807605408912\\
-0.00401890938999203	0.00461943082647761\\
-0.0040016449646977	0.0046107413778973\\
-0.00397817267439755	0.00457726894503156\\
-0.00399019939068386	0.00458205620001794\\
-0.00399617666118063	0.00458868824217421\\
-0.00401903425636391	0.00458731642471325\\
-0.00405500677721846	0.00462496375715975\\
-0.00412045863549711	0.00462373856900511\\
-0.00417734900331046	0.00466708119920671\\
-0.00442110836114524	0.00494887558079991\\
-0.00483164894518048	0.00543494957027028\\
-0.00545618745747107	0.00605545558018452\\
-0.00634452938695394	0.00694504523098606\\
-0.00737688917409191	0.00804368875799442\\
-0.00860297217670184	0.00925691801854914\\
-0.00991584254575038	0.0106109272799758\\
-0.0114619641539311	0.0122747472614617\\
-0.012876754947015	0.0142345292795463\\
-0.0144123590696398	0.0163660619435325\\
-0.016016773937082	0.0183798694929277\\
-0.0174920030375568	0.0205746985111592\\
-0.01713053090629	0.0225791052846711\\
-0.0133756136000719	0.0236659905310159\\
-0.00430034222882562	0.0249018479511279\\
0.0116939104720748	0.0268415903980102\\
0.035216378482335	0.027217648259604\\
0.0663588597466251	0.0269971875194111\\
0.105248359618161	0.0268590607571381\\
0.150953446947431	0.0254118910735926\\
0.203690816827192	0.0233676680131207\\
0.262752420524217	0.0181427652109263\\
0.326183145096281	0.0113128131382109\\
0.394281186545766	0.00603436750471615\\
0.465208355637247	-0.0049524111812778\\
0.542475855831044	-0.0162371445294915\\
0.626418161910732	-0.0164332889858583\\
0.715999174540469	-0.0142951909911941\\
0.810423787520278	-0.0133483084101371\\
0.910048572775709	-0.0143056595945424\\
1.01445484164042	-0.0131329718326544\\
1.11869572779703	-0.0128303357868091\\
1.22638179816935	-0.0127447592178104\\
1.33245697238448	-0.00859811107491129\\
1.44139658382791	-0.00861634454747132\\
1.55170928359023	-0.00821617455881603\\
1.66555497277893	-0.00551180674490376\\
1.78340033580841	-0.00577699096982132\\
1.901138076309	-0.00294694200322526\\
2.01668419261958	4.74608163979207e-05\\
2.13424336773358	-0.00289951207851205\\
2.25039699582256	-0.00600697309026296\\
2.37027803990069	-0.00739180633388667\\
2.494219637974	-0.0105013807417679\\
2.61768245151388	-0.0131071271166303\\
2.74151639243623	-0.0173824739257015\\
2.86627502596162	-0.0250415154272412\\
2.99207777505397	-0.0380910010501901\\
3.11806537028834	-0.0514775007847373\\
3.24786623300026	-0.0581713494736401\\
3.37860174093443	-0.0706888677274701\\
3.50663343025307	-0.0858279157182961\\
3.6371536325667	-0.0972472701366352\\
3.76605890425648	-0.106563108701319\\
3.89537771065028	-0.11516815041483\\
4.02490455420131	-0.118617560728042\\
4.15306191359771	-0.121380507913208\\
4.28495039441583	-0.12280845931653\\
4.4131272329558	-0.122493781285735\\
4.54380628893909	-0.115886035583055\\
4.67134031301862	-0.0997260376421804\\
4.79492932462867	-0.0770566470880049\\
4.92004815767136	-0.0525229574925304\\
5.0426749734133	-0.0229559031667613\\
5.16408758218846	0.0135039730871678\\
5.27958952261709	0.0605647647234617\\
5.39712948270113	0.109928253414812\\
5.51440532422004	0.156636659937874\\
5.63024525834151	0.20697215965992\\
5.74309423429395	0.263082195171674\\
5.85725319290907	0.323262431934729\\
5.9712721377629	0.383811827257659\\
6.08201915250973	0.44211433638803\\
6.19706309240174	0.498910822812343\\
6.30929517032533	0.555214697521271\\
6.42140585239913	0.613911339831118\\
6.53203430691706	0.678487119498162\\
6.63956848309798	0.739226866066907\\
6.74650232563251	0.794323534881919\\
6.85309434734913	0.849836381500601\\
6.96075995977918	0.907103436471046\\
7.06953792455724	0.964564173989664\\
7.1778458103267	1.02359028848562\\
7.28907478486214	1.07892909987716\\
7.40190486770525	1.13037432258513\\
7.51186625035077	1.18201896066502\\
7.62614565327529	1.23213010400705\\
7.73803908079967	1.28214759784659\\
7.84754893889926	1.33110411220787\\
7.95718093846889	1.37666208488657\\
8.06741897367414	1.41702028096296\\
8.17759939794053	1.45414195385857\\
8.28716769594472	1.48873161254696\\
8.39539077216543	1.51964783003731\\
8.50140213968702	1.54167879076829\\
8.59979744254983	1.55867433725474\\
8.69481663401933	1.5738051403289\\
8.784964228022	1.58956612322886\\
8.8685487162298	1.60107709520195\\
8.94441052805033	1.60657467304933\\
9.01365080766615	1.61135958368202\\
9.07533631364789	1.61560807904911\\
9.12839056481216	1.61801949950707\\
9.17357210792354	1.61922697320233\\
9.20778241488074	1.61989634941204\\
9.23201760994683	1.62151180789206\\
9.24793094279027	1.62269187487398\\
9.25644762052972	1.62444037425635\\
9.26178864171396	1.62707750843622\\
9.2637814118162	1.62967363844485\\
9.26397466196395	1.6313098046899\\
9.2635695515283	1.63318376510264\\
9.262959924607	1.635189055653\\
9.26050055343164	1.63689720348422\\
9.25851009450988	1.63839982771162\\
9.25603876758261	1.63947419197701\\
9.25410853801571	1.64048720703006\\
9.25251175575484	1.64126541790145\\
9.25144091434786	1.64168957205983\\
9.25075900814933	1.64182765209988\\
9.25060876187831	1.64181238637562\\
9.25054607107374	1.64175249837659\\
9.25054901418395	1.64173185986774\\
9.25049082296123	1.64182573584587\\
9.25047715367717	1.64193999776716\\
9.25049273453673	1.64197231114744\\
9.25047853723937	1.64206090360196\\
9.25044625545543	1.64212533190375\\
9.2504619668665	1.64211910848319\\
9.25048867569463	1.64211080663775\\
9.2505041509171	1.64213091287724\\
9.25047362040631	1.64209391588056\\
9.2504305096644	1.6420532571652\\
9.25041069212886	1.64203097837218\\
9.25038582956779	1.64200796861395\\
9.25034292616562	1.64201940788822\\
9.25031957660305	1.64202003913031\\
9.25028294046754	1.64202791542441\\
9.25027687059844	1.64202572486801\\
9.25024209283712	1.64202386365655\\
9.25026826823144	1.64201231824038\\
9.25026225278583	1.64200957899892\\
9.25027250356363	1.64200994606289\\
9.250266681071	1.64199846419118\\
9.25028391742417	1.64199377361963\\
9.250262538289	1.64198316119434\\
9.25025835934023	1.6419847822127\\
9.25025746579413	1.64196672649488\\
9.25030857611288	1.64195765163767\\
9.2503407913386	1.64196075671332\\
9.25040702901152	1.64196541165326\\
9.2504425779603	1.6419829120317\\
9.25084140896116	1.64196888923235\\
9.25142224312645	1.6420982248462\\
9.25295049998587	1.64197541077519\\
9.25794062267128	1.64168654480054\\
9.26875705801955	1.64082791471881\\
9.28748421412418	1.63903325451586\\
9.31570186729328	1.63633575742149\\
9.35396600314039	1.63472801458212\\
9.40490276244721	1.63186227654239\\
9.46528024470586	1.62687575684683\\
9.53583635510757	1.61988559362491\\
9.61575326106376	1.60943020059734\\
9.70142261796248	1.59995110510971\\
9.79165363077104	1.58866393397526\\
9.88288032407644	1.57214643138389\\
9.97667942553024	1.55376669675814\\
10.0753550668132	1.53748938715095\\
10.1746166866511	1.52379758706141\\
10.2811441050324	1.50571652205691\\
10.3914736754222	1.48363672257644\\
10.5015929947603	1.46444522668809\\
10.6170797217091	1.44591301600615\\
10.7328829802264	1.42751369200134\\
10.8506109657954	1.40838773126901\\
10.9724595220891	1.39059427770646\\
11.0938570582468	1.37390642319482\\
11.2173970130595	1.35998018921816\\
11.3427713448625	1.34524118135857\\
11.4639933616538	1.32779595678834\\
11.5910507422516	1.31387382080072\\
11.7213136582956	1.29945380774123\\
11.8502670655447	1.28352742493704\\
11.9817855179115	1.27122542482536\\
12.1101041389456	1.25846511801565\\
12.2406977629523	1.24241144267876\\
12.3760231398966	1.23062705517877\\
12.5094470523416	1.22470657715516\\
12.6443884126553	1.21891990739537\\
12.7799924653766	1.21021658150317\\
12.9135539816107	1.20544703617319\\
13.0503484284216	1.2040649752512\\
13.1880208296483	1.20094376389081\\
13.3233546182681	1.20154670766975\\
13.4591584175667	1.20803829281434\\
13.5949225333203	1.21337690731593\\
13.7290017070322	1.21971593515107\\
13.8637052736797	1.23499698461786\\
13.9961167336946	1.25641481543602\\
14.1289896528746	1.28231577306095\\
14.2613825687776	1.31298227887303\\
14.3888589382315	1.34711683513207\\
14.5218297200331	1.38730428674016\\
14.6506637318569	1.42947188474825\\
14.7749227378466	1.47896088429781\\
14.8979205086719	1.53632400878001\\
15.0175958212671	1.59778123568317\\
15.1388621562201	1.66010091317191\\
15.2541057479562	1.73338289611146\\
15.3661098721842	1.81591701898454\\
15.4726053144358	1.90246607430537\\
15.5761435613946	1.99231153941791\\
15.6779563101508	2.08082868158116\\
15.7765085663889	2.1778457782653\\
15.8663674860024	2.2823986322891\\
15.9502337857409	2.39223538213668\\
16.0293935866165	2.50777916807673\\
16.1016575751607	2.61970827060563\\
16.1736060737978	2.73298432208145\\
16.2409683559766	2.85103786833708\\
16.3007253274328	2.97392123108219\\
16.3548070052648	3.09763326325894\\
16.4075128789106	3.21926314213346\\
16.4582030023846	3.34093451822254\\
16.5038693056004	3.46837811293086\\
16.5418880941823	3.59922776718389\\
16.5722135466436	3.73435679091016\\
16.6005950308892	3.86844348129196\\
16.6243232743927	3.99993935970328\\
16.6438840301607	4.13352987432718\\
16.6578062827302	4.2672406813032\\
16.6671445418413	4.40453695274579\\
16.6707293688651	4.54021929959789\\
16.6714237693954	4.67317501664795\\
16.6680142047858	4.80707109565718\\
16.6594395074225	4.94161952399884\\
16.6490210993222	5.07681985493497\\
16.6341501128229	5.21222208217043\\
16.6171241700261	5.3447591800358\\
16.599046471706	5.47392171946008\\
16.5757071775535	5.60441024190898\\
16.5474522102481	5.73539330494972\\
16.5168159363755	5.86440988993181\\
16.4817942863559	5.99367715132427\\
16.4424370585272	6.12078599878202\\
16.4003936556137	6.24554565293671\\
16.3552283870017	6.37126594018821\\
16.3049360108457	6.49351014483119\\
16.2502437901382	6.61216001143138\\
16.1964417707759	6.73048170508476\\
16.1395713312637	6.84764935884045\\
16.0798645973678	6.9651023948386\\
16.0172291087945	7.07849610858098\\
15.9505127646269	7.18850495793284\\
15.8812228108286	7.29633318991668\\
15.8121036747558	7.40310464573685\\
15.7390468831284	7.50746755072875\\
15.6612373069975	7.60942533381729\\
15.5819264635917	7.71269795820961\\
15.4994093431756	7.80948903731251\\
15.4163499757416	7.90426488631477\\
15.3354081887654	7.99931549429373\\
15.2522039207761	8.09192353506612\\
15.1658398053246	8.17540708716605\\
15.0783006558137	8.25262887379566\\
14.9943006490147	8.32636567717311\\
14.9158959910506	8.40212137030252\\
14.8355797153284	8.46994959045886\\
14.7485005420541	8.53226633046858\\
14.6623213745983	8.59387746633567\\
14.5770863861545	8.65637085241178\\
14.4921665033328	8.7118226441906\\
14.4043721444894	8.76195170161644\\
14.3180314666358	8.81205086601585\\
14.2332685342467	8.85837235624441\\
14.1480297032657	8.89754032165848\\
14.0674615318587	8.92865627301121\\
13.990664695405	8.95537233922407\\
13.9173182865316	8.97941381372292\\
13.8485934900534	9.00000784606318\\
13.7824981946652	9.01737996819088\\
13.7216823487106	9.03086723942614\\
13.6644283661557	9.0392976381526\\
13.6155394905903	9.045543459099\\
13.5750175051401	9.05190790379228\\
13.5422303171644	9.05625822861468\\
13.5168203652657	9.05740145429537\\
13.4985305433862	9.05897005644958\\
13.486100148344	9.06091618496389\\
13.4778586150701	9.06197859041248\\
13.472971041419	9.06178902371255\\
13.4701643542022	9.06214907169022\\
13.4692672601274	9.06294233525307\\
13.4692647281559	9.06270572382741\\
13.4697425884381	9.06223065048548\\
13.4698598556491	9.06211659247461\\
13.4695992222452	9.06177361082215\\
13.4690065150716	9.06113498955688\\
13.4679828073941	9.06074996503337\\
13.4671463719493	9.06053084160427\\
13.466381654897	9.06013864191982\\
13.4657147340284	9.059902042399\\
13.4648436009525	9.05965000212206\\
13.4641549187424	9.05931312021426\\
13.4636186597998	9.05910832051968\\
13.463435956788	9.059087183208\\
13.4633782956286	9.05879495670787\\
13.463309534724	9.05875224784219\\
13.4632825622388	9.05876049318771\\
13.4632542318601	9.05847631100894\\
13.4632574240701	9.05850125610002\\
13.4632798842914	9.05851826951383\\
13.4633326879677	9.05841193930063\\
13.4633507023809	9.05852965459937\\
13.4633533856229	9.05868708451989\\
13.463338011982	9.05868738186939\\
13.4632763117668	9.0588234716843\\
13.4632370495206	9.05890952860264\\
13.4631911559879	9.05891045140912\\
13.4632119968526	9.05895915435102\\
13.4632202044357	9.05895753678758\\
13.4631951054619	9.0589104926545\\
13.4631827425723	9.05889840908532\\
13.4631840744636	9.05890189894758\\
13.463223597501	9.05884286207664\\
13.4632233103903	9.05881956845674\\
13.4632249020451	9.0587849155505\\
13.4632563350171	9.05875407281572\\
13.4632740588249	9.05874876042301\\
13.463290278308	9.05876647895255\\
13.4632934494669	9.0588009144232\\
13.463244753047	9.058805766478\\
13.4631985206491	9.05882744805644\\
13.4633187485655	9.05875086159461\\
13.4629224621753	9.05910446440041\\
13.4602393949048	9.05979969159865\\
13.453119412728	9.0599315919168\\
13.4394481529651	9.0600539939569\\
13.4176572548676	9.06140228116975\\
13.3855356678185	9.06152347168285\\
13.3427863081361	9.05882369351287\\
13.2900122036728	9.05528958042339\\
13.2270463509498	9.05067041372384\\
13.1557740775883	9.04355539889044\\
13.079793433591	9.02982126899012\\
12.9985593687419	9.0096271048649\\
12.9136038296545	8.98896282588228\\
12.8269211538492	8.96272112904956\\
12.7392505481338	8.92870659726051\\
12.6483443244265	8.88913483368786\\
12.5551277967114	8.8453839727284\\
12.4602598444806	8.79384463813354\\
12.368261454597	8.73488059989262\\
12.2769914099883	8.67289609875325\\
12.1854488622084	8.61078106503657\\
12.0950382642891	8.53977533686385\\
12.0055266719867	8.4618942129087\\
11.9171486901297	8.37953326175558\\
11.8323147944693	8.29156204626061\\
11.7493754347791	8.20230756382731\\
11.6666391298933	8.11259071088053\\
11.5841763463833	8.01789382123911\\
11.5041419524301	7.92244229738386\\
11.4204859222238	7.82608896581356\\
11.3348734797012	7.73250866237616\\
11.2508445214749	7.64337198521719\\
11.1643936685241	7.55731104731318\\
11.0702886265564	7.47116382135284\\
10.974376735735	7.3886533327033\\
10.8799067874887	7.30504465131841\\
10.7814575749445	7.22171309585684\\
10.6779886551929	7.1468890634001\\
10.5722784462175	7.07047287062497\\
10.4630711259327	6.99326334892379\\
10.3492209550645	6.92237204082993\\
10.232294651634	6.8565369227511\\
10.1155537476475	6.79075188756224\\
9.99794506287057	6.72692567508149\\
9.87675086212485	6.66680797579755\\
9.75318685163088	6.61132365226119\\
9.62710413553661	6.55924465823375\\
9.50002843380128	6.51346701904514\\
9.37306766025639	6.46798400768186\\
9.2429874543177	6.42327699225196\\
9.10910463922886	6.38644740660042\\
8.97411312852411	6.35246092163036\\
8.83572766775398	6.31803465473672\\
8.70070913531952	6.28463135099351\\
8.56511876103578	6.25751710129093\\
8.42517920363083	6.23455786676026\\
8.28450625312795	6.21184679637219\\
8.14841722489797	6.19273341884587\\
8.00783753631045	6.17500188583927\\
7.8700977327396	6.15628579188258\\
7.73462237631482	6.14200703177832\\
7.59351196492898	6.13131609534772\\
7.45443487026879	6.12179950239596\\
7.3165978014135	6.1077792905897\\
7.17788348162871	6.09499295653531\\
7.03799861313196	6.08765367586544\\
6.89814751742361	6.08066907311093\\
6.75799565673137	6.07188642492231\\
6.61588169929492	6.06479581277499\\
6.47643913324609	6.05801415554759\\
6.33715302940159	6.04866167845097\\
6.19348066443017	6.04619991369662\\
6.05211156379989	6.04396746032669\\
5.90967716316949	6.03637408171243\\
5.76364751818759	6.02600508616127\\
5.62051026521014	6.0203496582249\\
5.47586615361119	6.01954642140475\\
5.32892109662997	6.0184431180986\\
5.18120115938864	6.01437710539688\\
5.03394643064646	6.00777154017654\\
4.8846774925337	6.00250673343391\\
4.73731409082565	6.00169567975507\\
4.59125046057075	6.00345759339506\\
4.44208012516938	6.00025000478072\\
4.29427658547132	5.99289992721384\\
4.14739662157714	5.98547883344516\\
4.00104235541819	5.98475687641608\\
3.85466139537104	5.98468971827389\\
3.70736263562793	5.97969799703192\\
3.55797294375675	5.97469035574344\\
3.40939957048286	5.97369909083288\\
3.26442632841279	5.97131933068691\\
3.11781874859768	5.97145036874881\\
2.97083938837511	5.96788518397652\\
2.82611507291501	5.96023044230304\\
2.67756854634337	5.95583235656968\\
2.53323131626212	5.95475309078086\\
2.39250319765962	5.95177296756577\\
2.24907031077274	5.94288561248268\\
2.10869794290739	5.93520992508363\\
1.97173137770089	5.93073242360556\\
1.83538434261827	5.92710041673153\\
1.70172976378748	5.92142653239001\\
1.56923227070891	5.91543765807853\\
1.4394624182375	5.90656192779744\\
1.31287553384125	5.90046380607169\\
1.19106254580116	5.89501382142563\\
1.0764856646797	5.88882617154235\\
0.962545612598274	5.8852006299273\\
0.855220834226946	5.87988391307629\\
0.756708871438657	5.87129866708636\\
0.664275347399113	5.86412998994399\\
0.578846594783561	5.86038430256742\\
0.504191956376299	5.85625558987713\\
0.436687600722196	5.85165076641582\\
0.380113002376999	5.84822787364657\\
0.332739434383852	5.84549030580905\\
0.292993356782038	5.84303365367697\\
0.263604738789397	5.84072293432662\\
0.240737165210177	5.83899378364481\\
0.2253677489733	5.83803222403294\\
0.214270074977396	5.83720034287128\\
0.206170322405864	5.83653609641065\\
0.200405729393371	5.83661590580498\\
0.196473622057645	5.83673711580015\\
0.194240422674213	5.83628573337145\\
0.193208701278066	5.83634211391775\\
0.192878024572207	5.83668569998948\\
0.192778691529331	5.83656872118827\\
0.192657261505016	5.83655054464418\\
0.192414638098174	5.83671562126093\\
0.192197641576818	5.83657164122002\\
0.192027646031834	5.83637716781703\\
0.192089410119591	5.83641880118548\\
0.192193632031215	5.83632839784038\\
0.192265271841128	5.83622691220537\\
0.192220794151321	5.83628379152211\\
0.192218028862463	5.83626702611131\\
0.192223693482835	5.83626109631625\\
0.192206797203417	5.83632699495828\\
0.192162558853579	5.83635327173616\\
0.192138277180388	5.83634079055766\\
0.192089069417325	5.83639207149698\\
0.192056367511352	5.83639269507781\\
0.192027256601988	5.83634212925286\\
0.192021666224881	5.83632969874173\\
0.192027814705353	5.83630908418187\\
0.192039891841005	5.83626165089078\\
0.19199409826531	5.83625955581342\\
0.192017118600983	5.83626081777702\\
0.19205382235145	5.83626385289706\\
0.192065393916505	5.83626106949741\\
0.192094363211912	5.8362383461099\\
0.192091925462566	5.83624180024431\\
0.192091327218238	5.8362433075822\\
0.192065659427502	5.83621889398323\\
0.19205628157254	5.83620775694055\\
0.192021351830379	5.83622880567102\\
0.191994890485469	5.83625864898512\\
0.192033748418168	5.83628123872977\\
0.19203080493265	5.83627702984269\\
0.192028040969289	5.83629153744254\\
0.191977744613689	5.83628773607425\\
0.191988544903619	5.83627653310999\\
0.19197579785653	5.83628721900115\\
0.191959675810059	5.8363037829727\\
0.191969658825357	5.83631029491055\\
0.191983089094947	5.83633946692264\\
0.19201818666682	5.83634548065676\\
0.192011862734825	5.83634327834629\\
0.191982988146857	5.83642662911672\\
0.191955138864175	5.83646416071095\\
0.191970744467447	5.83647541624309\\
0.192019873689103	5.83654872061129\\
0.192031763210394	5.83656331717799\\
0.192034611132813	5.83647132811937\\
0.192072949418045	5.83645145042756\\
0.192081903725611	5.83641442006252\\
0.192077982749621	5.83636976581086\\
0.192067103188284	5.83636698412639\\
0.192073817942831	5.8363523401119\\
0.192053987030156	5.83632781247386\\
0.192067730000451	5.83633205085277\\
0.192087667223252	5.83634812697968\\
0.192146931393284	5.836317025711\\
0.192133408993539	5.83634257771984\\
0.192180920583757	5.83649727926139\\
0.191528800174748	5.83650130580017\\
0.190267261436765	5.8362086433566\\
0.184637368496135	5.83683128782321\\
0.17250208292248	5.83608579522434\\
0.150973757157279	5.83344780477957\\
0.117461507083797	5.83023788496557\\
0.0721609589167941	5.82693513152253\\
0.0168377300820675	5.820953289644\\
-0.0513025638468028	5.81397376929278\\
-0.128458052723002	5.80700320285275\\
-0.214047046161554	5.7995435793628\\
-0.307222273019431	5.79027646031215\\
-0.408082305316133	5.78019743038333\\
-0.513130132125113	5.77032489679014\\
-0.62624813643627	5.7595454928174\\
-0.744592188718582	5.74551523174158\\
-0.869426618997852	5.73413044497448\\
-0.997422927103059	5.72528965045686\\
-1.12455751606655	5.71458499294802\\
-1.25091535988869	5.70493695264556\\
-1.38007133869957	5.69477822438606\\
-1.50938597127305	5.68297042760668\\
-1.63877619102581	5.67213449969741\\
-1.77464319243149	5.66169321364343\\
-1.9128440399898	5.65545559801821\\
-2.05116389407095	5.64730885946172\\
-2.19148727017903	5.63651656410109\\
-2.33421964904621	5.62611336130674\\
-2.47302745144498	5.62065137434931\\
-2.61230192404688	5.61697352648635\\
-2.75267693996776	5.60709281430635\\
-2.88956785144703	5.59673947129324\\
-3.0291517694222	5.59027970090562\\
-3.16973235409861	5.58527627299929\\
-3.31125306700997	5.57730772478058\\
-3.45358317387205	5.56873712655363\\
-3.59622509970132	5.56102894766276\\
-3.73954316637866	5.55317146377811\\
-3.88352290170339	5.54445484549892\\
-4.02615743464784	5.54175633884196\\
-4.16976407797467	5.5358000083324\\
-4.31126037333373	5.52482495120206\\
-4.45891165491369	5.51876873704534\\
-4.60420434774172	5.51612632683563\\
-4.7484472952641	5.5132058764837\\
-4.89548342867606	5.50193233286841\\
-5.04348659046376	5.49207674232509\\
-5.18811845931405	5.48713873449133\\
-5.33679053036774	5.48548079593594\\
-5.48485969228556	5.48189137424147\\
-5.63204579328475	5.47343881051361\\
-5.77969828136424	5.46291974589439\\
-5.92620755101303	5.45827307612996\\
-6.07164935625616	5.45453637491025\\
-6.22216757636865	5.45205008455213\\
-6.37200116630795	5.44575908646512\\
-6.51969615217807	5.4353132705985\\
-6.66950617255401	5.42766597850916\\
-6.81735286420405	5.42490765843028\\
-6.96578395920413	5.42291037650871\\
-7.11748348836652	5.41440348605005\\
-7.26970403303207	5.40098260200454\\
-7.42089698610081	5.39567723311886\\
-7.57236446615009	5.39511448272837\\
-7.72500227544779	5.38898205393461\\
-7.87744282370555	5.37918988851715\\
-8.03306151445366	5.37265793336039\\
-8.18523609368713	5.3707832118316\\
-8.33493468519019	5.36643076470396\\
-8.48816429268027	5.35906957622685\\
-8.63958134138863	5.35103266803356\\
-8.79237188600173	5.3463056105526\\
-8.94448495367411	5.34006091328896\\
-9.09289772626449	5.33472682445092\\
-9.24493678963999	5.32491240559329\\
-9.39975994652957	5.31393306662686\\
-9.54986708871186	5.30731784759526\\
-9.70060437925911	5.30177661081584\\
-9.85271657336519	5.29191092904254\\
-10.0007957114571	5.28118576503335\\
-10.1525685096271	5.2710908288637\\
-10.3008884678805	5.26166382031407\\
-10.4456696699597	5.25008883659091\\
-10.5923503846985	5.2341162273825\\
-10.7378716566918	5.2173363980478\\
-10.8824282657979	5.20450395657684\\
-11.0272756689425	5.18818434437143\\
-11.1693334882543	5.16700812008829\\
-11.3094887196801	5.14473382907937\\
-11.4487183358903	5.11999713041633\\
-11.5917949385453	5.0961900756865\\
-11.7325106113252	5.07157742248859\\
-11.8736934490823	5.0417857832085\\
-12.0138917930326	5.00374397838384\\
-12.1499519656331	4.96455073496946\\
-12.2858838840365	4.92161942429734\\
-12.4206352522558	4.87503858304951\\
-12.5528302932694	4.8275956344368\\
-12.6800743159332	4.76884219284885\\
-12.8040806552427	4.70743218967958\\
-12.928706828419	4.64522942057427\\
-13.0504853872525	4.57835722791146\\
-13.1691507148802	4.50864205386596\\
-13.2865857196511	4.43132275178574\\
-13.3997784996304	4.35192135961469\\
-13.5090610362673	4.27029395248132\\
-13.6164539580017	4.1828486701697\\
-13.7224359104157	4.0943487687301\\
-13.8247141158087	4.00480284311215\\
-13.9273702145182	3.91458526933169\\
-14.0253198230477	3.82488516976469\\
-14.1251649602036	3.73678743433979\\
-14.2228014060626	3.6473091620237\\
-14.3211684933456	3.56081159769459\\
-14.4191429533553	3.47436563448848\\
-14.518138722825	3.39506606114366\\
-14.6227284137722	3.32159219671099\\
-14.7287344659322	3.2502198537676\\
-14.8319773033408	3.18182123030033\\
-14.9394154185512	3.11451851939459\\
-15.0531960209555	3.0532189166384\\
-15.1628592525349	2.99419076554395\\
-15.2776844382139	2.93838249338209\\
-15.3931296796979	2.88734731506338\\
-15.5073712304988	2.83954337827642\\
-15.620894795901	2.79316573975845\\
-15.7321162491215	2.75511248432896\\
-15.8396714345892	2.72539150636919\\
-15.9446292741432	2.69522901740752\\
-16.0492503237197	2.66834475834248\\
-16.1502624126818	2.64547359040358\\
-16.2468427151435	2.6235972675644\\
-16.3377466511541	2.60661381423737\\
-16.4253224098355	2.59531762795035\\
-16.5048692209586	2.5844331269779\\
-16.5793164795662	2.57431297617609\\
-16.647125246511	2.56882631616587\\
-16.7062349640419	2.56597558618037\\
-16.7552467221284	2.56112665655648\\
-16.7942569923228	2.5578968713733\\
-16.8230189687621	2.55769051742551\\
-16.8432141136712	2.55646306868059\\
-16.857105351206	2.55538628559035\\
-16.8667224981701	2.55676500952284\\
-16.8739337510723	2.55808592815465\\
-16.8794841771865	2.55873660855177\\
-16.8833345275572	2.55994573257958\\
-16.8860381783037	2.56159095576183\\
-16.8874470831396	2.56329929908368\\
-16.887742193971	2.56495738393393\\
-16.8879815107101	2.56663989064887\\
-16.8878845192262	2.56829405770246\\
-16.8880842839073	2.56975949997548\\
-16.8883507113728	2.57109798816129\\
-16.8887934888046	2.57216656605684\\
-16.8891737910379	2.57336281444146\\
-16.8896233114219	2.57454255101965\\
-16.8897521150168	2.57541908171845\\
-16.8898322839442	2.57623821335329\\
-16.889745160578	2.57709216514254\\
-16.8897442289093	2.57762692538957\\
-16.8897034596198	2.57805129613497\\
-16.8897591520427	2.57840344964781\\
-16.8897233185884	2.57851518831848\\
-16.8897614941948	2.57859236698521\\
-16.8897185722177	2.57870193980281\\
-16.889689939499	2.57871789819091\\
-16.8896449994401	2.57874705466066\\
-16.8896754536475	2.57879635440201\\
-16.8896766393635	2.57877576857469\\
-16.8896583187586	2.57877298661095\\
-16.8896377602304	2.57878999222955\\
-16.8896265823937	2.57879970719254\\
-16.8896293364579	2.57877682650643\\
-16.8896218298523	2.57877743566601\\
-16.8895749449707	2.57876963696298\\
-16.889568898678	2.57876408544163\\
-16.8895902885903	2.57875709820867\\
-16.8896312926592	2.57873361775927\\
-16.8896484467459	2.57874356071093\\
-16.889655922689	2.57873572490657\\
-16.8896314319052	2.57868633506163\\
-16.8896681133853	2.57866040258414\\
-16.8898006130585	2.57858669202766\\
-16.8905756391323	2.57860393229322\\
-16.8923151681515	2.57869193191294\\
-16.8976956821392	2.57792808312864\\
-16.9089862693849	2.57707605518932\\
-16.9279349436601	2.57446329655217\\
-16.9560660014578	2.56949065634764\\
-16.9938234652911	2.56411948552449\\
-17.0413227678388	2.55790224956308\\
-17.0979729758774	2.54937983738815\\
-17.163316582185	2.53771258249646\\
-17.236820215967	2.52516431832639\\
-17.3177174460584	2.51212577728499\\
-17.4045342255223	2.49939873788995\\
-17.4960047621209	2.48611764614644\\
-17.5922447753565	2.46904708338596\\
-17.6945617193809	2.44936428175797\\
-17.8016481820209	2.4320320483926\\
-17.9110894890666	2.41520852246721\\
-18.0207249786857	2.39501127319995\\
-18.1294537474674	2.37010772708401\\
-18.2421361564129	2.34055834309476\\
-18.3559655053926	2.30990135300324\\
-18.4703220892696	2.27353249844683\\
-18.5848908546111	2.22942867115659\\
-18.6974867811721	2.18267405791444\\
-18.8064264881713	2.13398232752445\\
-18.912036185135	2.0784557241892\\
-19.0102327925961	2.01269594121921\\
-19.1076564344779	1.94137177644199\\
-19.2046480984412	1.86602811880957\\
-19.294819318223	1.78341283313786\\
-19.3774851559884	1.69399272739932\\
-19.4556186300384	1.60167559891559\\
-19.531481552923	1.5024486672399\\
-19.5995506585858	1.39988011747814\\
-19.6611794737827	1.2911806631594\\
-19.7123338080165	1.1761398314181\\
-19.7559106581905	1.06126854877971\\
-19.7993214917541	0.945648376129782\\
-19.8380073543923	0.828664589022003\\
-19.8647525503513	0.708852225043078\\
-19.8840766184833	0.58580674281244\\
-19.8985301535263	0.4630603253744\\
-19.9095625966528	0.341442286980434\\
-19.916202556449	0.219538583101421\\
-19.9152658473629	0.094345081587789\\
-19.9071180124539	-0.0307132724639111\\
-19.8933263399812	-0.153353247569208\\
-19.8740846394168	-0.277544868229196\\
-19.8502184821226	-0.398750500483338\\
-19.8246890817996	-0.518765606516356\\
-19.7913519001527	-0.638980069221455\\
-19.747278820008	-0.75376273748963\\
-19.6967323269565	-0.866942735599506\\
-19.6463027356185	-0.980302871433052\\
-19.5916581392254	-1.09001016901715\\
-19.5298989865015	-1.19716369971428\\
-19.4604823085608	-1.30033691019711\\
-19.3839304724689	-1.39791132193262\\
-19.3044885071713	-1.492624749267\\
-19.2212673352365	-1.58110704241762\\
-19.1323906540324	-1.66583385134984\\
-19.0408598089465	-1.74863058597174\\
-18.9452094517538	-1.82322922646955\\
-18.8430146969619	-1.89352053342814\\
-18.740187181012	-1.95811644580365\\
-18.6337407576425	-2.02030779177414\\
-18.5220644489562	-2.07655707410579\\
-18.4075035849489	-2.12574070247495\\
-18.28852684374	-2.16929811640854\\
-18.1662022858575	-2.20642072981896\\
-18.0455901591807	-2.23526895178916\\
-17.9194376441778	-2.26067671286219\\
-17.7925301616184	-2.28022445648651\\
-17.6634004577104	-2.294191933974\\
-17.5350112610869	-2.29947189390461\\
-17.4070907452283	-2.29934676625181\\
-17.2790007589517	-2.2965556182463\\
-17.1519960874774	-2.29280075365218\\
-17.0195330490404	-2.28814496882084\\
-16.8905516601539	-2.27591249639827\\
-16.7604968952462	-2.25963393271291\\
-16.631291432649	-2.24108795695927\\
-16.500622093709	-2.22290521066557\\
-16.3690365105924	-2.2072739156375\\
-16.2419236931027	-2.18500602526075\\
-16.1077813735065	-2.15613843469059\\
-15.9792008446619	-2.13235033183341\\
-15.8470246911112	-2.10923299186298\\
-15.7156089499173	-2.08547359096646\\
-15.5853446232215	-2.06061116979571\\
-15.4528641440805	-2.03282228508533\\
-15.3216703523965	-2.00298563679159\\
-15.1910006223904	-1.97490245806323\\
-15.0602201620703	-1.9502050354226\\
-14.9293216382364	-1.92022495564558\\
-14.7985979840155	-1.88727803412142\\
-14.6677633251412	-1.8562671775705\\
-14.5380207035069	-1.82804773113092\\
-14.4083437420802	-1.80184830031135\\
-14.2787733561742	-1.77504996571519\\
-14.1507364328908	-1.74383854595525\\
-14.0207137377463	-1.71073515175093\\
-13.8925569554473	-1.68169082104958\\
-13.761726994518	-1.65619034510569\\
-13.6293884320535	-1.6296400977796\\
-13.4974177501548	-1.59948277048558\\
-13.3677639543776	-1.56631229139042\\
-13.2390933387295	-1.53380886206797\\
-13.110019614334	-1.50839163486185\\
-12.982924192846	-1.4826480886422\\
-12.8532301598594	-1.45089828469383\\
-12.7246347719329	-1.4197579384602\\
-12.5971791340563	-1.39037008586078\\
-12.4673215643662	-1.36083541255139\\
-12.3382155978623	-1.33694393770552\\
-12.2070182418999	-1.30890980335578\\
-12.0794642997285	-1.27404948695515\\
-11.9489657349855	-1.23993453967464\\
-11.8195787654616	-1.21219559528404\\
-11.6923946600391	-1.18502684261564\\
-11.5670411504254	-1.15236806024461\\
-11.4429988691747	-1.12242464242604\\
-11.3240232647005	-1.09430060882595\\
-11.2046215058369	-1.06516797450544\\
-11.0924563104531	-1.04108031897572\\
-10.9782041864308	-1.01838137067365\\
-10.8663119171102	-0.990561149729751\\
-10.7589394446463	-0.963349871735852\\
-10.6569957962752	-0.940703771774692\\
-10.563937897853	-0.919206027922718\\
-10.4759820901314	-0.898761204839082\\
-10.3972908337602	-0.882957914183179\\
-10.330375639635	-0.869936680898707\\
-10.2724034066117	-0.856533550869932\\
-10.2260623729326	-0.846036539699312\\
-10.1897783975209	-0.837392253899091\\
-10.1612857122825	-0.831452506262806\\
-10.1417296690856	-0.827035138396354\\
-10.1280661097699	-0.824345456515281\\
-10.1188064623019	-0.822803784661156\\
-10.1130953199586	-0.821714900533247\\
-10.1091912662481	-0.821361811651817\\
-10.1072134830779	-0.821352497160289\\
-10.1063281189331	-0.821417064845345\\
-10.1062628296248	-0.821621253257378\\
-10.1064906979024	-0.821885842973069\\
-10.1069694581503	-0.82224023169958\\
-10.1073322408246	-0.822425121399114\\
-10.1074668069244	-0.822537537043669\\
-10.1074542969894	-0.822632812485737\\
-10.1074368337898	-0.822663344466892\\
-10.1073949770432	-0.822633311892085\\
-10.1072868856298	-0.822625273197502\\
-10.107277334749	-0.822623566988714\\
-10.1073268204879	-0.822595115867721\\
-10.1073522071708	-0.822589751519583\\
-10.1073861987275	-0.822591780634284\\
-10.1074771527301	-0.822557641943454\\
-10.1075896863093	-0.822561655768572\\
-10.1075769145613	-0.822563993001148\\
-10.1075406978211	-0.822559494601024\\
-10.1075239092873	-0.82255748185518\\
-10.1074922664978	-0.822590723171904\\
-10.1074981457356	-0.822588780744828\\
-10.1075256077492	-0.82260630011713\\
-10.1075221681079	-0.822622361096856\\
-10.1075040370553	-0.82261382909641\\
-10.10748727266	-0.822648416349649\\
-10.1074890335853	-0.822700708838657\\
-10.1075070011922	-0.822720181418125\\
-10.1075182783712	-0.822733799511384\\
-10.1075312082149	-0.8227510781509\\
-10.1074725342754	-0.82280299134352\\
-10.1059600237561	-0.82299303005442\\
-10.1014091050906	-0.821526922779003\\
-10.0903541444258	-0.819162681193516\\
-10.0702416467413	-0.81436649613715\\
-10.0398904269262	-0.805668475801171\\
-9.99724062479655	-0.794211617053428\\
-9.94293348932661	-0.78050343547816\\
-9.87542112098584	-0.765366863250363\\
-9.79814755837289	-0.748987646364983\\
-9.71189134743076	-0.729496272147711\\
-9.61783891972089	-0.70523442394262\\
-9.51857316705396	-0.679666670281398\\
-9.41538921923652	-0.656401819102942\\
-9.30933650656943	-0.631907081640096\\
-9.19968131870841	-0.603149421143961\\
-9.08204220678033	-0.5766202985443\\
-8.96556051031865	-0.554039647562945\\
-8.84427894723264	-0.524545632434934\\
-8.72452268553704	-0.492228342207594\\
-8.6036828041501	-0.464887831854657\\
-8.47897423876673	-0.438172660935063\\
-8.35541570036433	-0.410831293517628\\
-8.23148937956161	-0.382570068230915\\
-8.1060723174697	-0.355792027585466\\
-7.98503488745689	-0.326612810495622\\
-7.85862656266228	-0.294717625178855\\
-7.72990527034049	-0.269933314307017\\
-7.60061076093677	-0.246224951308284\\
-7.47022168501084	-0.213624004430629\\
-7.33970457175974	-0.179554385128228\\
-7.21008465001023	-0.152104942862888\\
-7.07744321200762	-0.125095787125837\\
-6.94472399395039	-0.0982988884584991\\
-6.80970248079524	-0.070102025589887\\
-6.6750762686454	-0.0384495053591518\\
-6.54194783455209	-0.00683222139056115\\
-6.40721547553335	0.0204337553849392\\
-6.27050490270175	0.0456691275951886\\
-6.13292163576028	0.0766029810009343\\
-5.99837699657556	0.111424754215601\\
-5.86748677013673	0.144453291130636\\
-5.73426589745025	0.176733407655623\\
-5.59880503415058	0.204926382270445\\
-5.46727796740665	0.233173391775473\\
-5.3345024050363	0.267177248436844\\
-5.19955261010616	0.300264439773778\\
-5.06723688059976	0.331145227824994\\
-4.93241103189932	0.357819979943867\\
-4.80206080557779	0.386437048647827\\
-4.66926192813861	0.419865890694154\\
-4.54037551038165	0.455959483957998\\
-4.40934349962551	0.486089807062471\\
-4.27951825198112	0.508104441510145\\
-4.15216850696719	0.53317338802407\\
-4.02390058217803	0.561749677692947\\
-3.89893465669151	0.586741210757513\\
-3.77232817312292	0.609911819559437\\
-3.64730686486922	0.63057577883909\\
-3.51931557954242	0.646581167805286\\
-3.38981845400159	0.664996467991612\\
-3.26127578589177	0.686657615709844\\
-3.13034495402016	0.707483996709437\\
-2.99962948851108	0.721599368641348\\
-2.86992634002333	0.728582414095892\\
-2.73721809665444	0.739332727621004\\
-2.60766425495171	0.750104828203993\\
-2.47834314750955	0.756539728066056\\
-2.34583016820923	0.75947347144413\\
-2.21732650575343	0.759675880864654\\
-2.08639429705994	0.7587403791709\\
-1.95519572601096	0.756798060750315\\
-1.82451649474835	0.754552090441267\\
-1.69464513507949	0.751119407765913\\
-1.56577809525865	0.742005415026537\\
-1.43676589061439	0.728348511043674\\
-1.30980023338614	0.717073405398076\\
-1.18393895354155	0.708697618407748\\
-1.05900737828149	0.693105873983302\\
-0.937096452559004	0.673954289009285\\
-0.820586634006304	0.656878916044376\\
-0.700953844395633	0.63994263202884\\
-0.585830832893412	0.623731815038984\\
-0.473778843305896	0.605435461652165\\
-0.367130262318813	0.58829320967983\\
-0.263429412646291	0.573460611744371\\
-0.168995701973718	0.561781275314086\\
-0.0802119712231269	0.55181782316043\\
0.00196259870201621	0.547107990632065\\
0.0793110758995115	0.545766563859418\\
0.150213475555143	0.544935410835604\\
0.213611322361117	0.547431533011075\\
0.266067827077746	0.553694631391817\\
0.308304183466038	0.558935869594833\\
0.338821048874056	0.563958085025466\\
0.360628663659536	0.568526394657454\\
0.375832969221941	0.571927512533977\\
0.386710164767274	0.5750536229208\\
0.395281977649544	0.57605526931106\\
0.400927223902961	0.577405749055991\\
0.405869362565782	0.578950461697472\\
0.409953604400958	0.578856184487236\\
0.412060476870128	0.579103870667439\\
0.412305721565966	0.579097950790085\\
0.411868954103004	0.578330509467443\\
0.411374188410234	0.577880896819786\\
0.410898846550918	0.577784099618098\\
0.410946829710643	0.577443283360592\\
0.410986751740597	0.577459783979815\\
0.411126698463146	0.577647903860697\\
0.411317826187041	0.577650165281403\\
0.411358523330154	0.577716657182636\\
0.4113869036374	0.577800906413886\\
0.411473618802465	0.57774551755352\\
0.411457393289283	0.577823465893689\\
0.41151117225884	0.577916806472097\\
0.411512455871821	0.57791705146631\\
0.411532956402455	0.577952798698104\\
0.411512049602722	0.577977223771869\\
0.411528500090786	0.577949794490501\\
0.411534938979846	0.577922610012269\\
0.411558477046665	0.577891324249795\\
0.411611464326033	0.577782915972485\\
0.411615025373077	0.577731451153643\\
0.411593328652256	0.577706999809956\\
0.411613880476581	0.577626135808754\\
0.411566748321058	0.577586560503794\\
0.411544878253334	0.577620742974867\\
0.411516624031199	0.577575032941944\\
0.411517930006304	0.577570110565303\\
0.411578399644241	0.577512287494666\\
0.411595095529217	0.577466441587206\\
0.411605680888979	0.577460304734264\\
0.411606062567954	0.577440409357378\\
0.411618295352803	0.577417624902671\\
0.411648195900969	0.577426674933932\\
0.41166685688162	0.577440946318967\\
0.411690227174573	0.577453011040394\\
0.411693321777629	0.577466942828084\\
0.411681862836751	0.577474810579435\\
0.411694768874393	0.57751574479305\\
0.411732981989668	0.577547947074371\\
0.411737680103852	0.577552268641736\\
0.411764334356131	0.57757082210455\\
0.41173056410406	0.577581112337578\\
0.411723681590242	0.577640362569583\\
0.411723409556966	0.577657376161157\\
0.411716040496342	0.577660829635377\\
0.411700747935951	0.577662833821857\\
0.411695445224028	0.577659069927428\\
0.411695898433635	0.577665173966188\\
0.411727169420266	0.57765400660157\\
0.41174603864449	0.577665567973183\\
0.411756976114258	0.57765924432787\\
0.411765538143698	0.577629629238749\\
0.411765816284174	0.577622079084756\\
0.411764107267364	0.577597206413899\\
0.4117724473484	0.577570754027016\\
0.411764865604992	0.577544520860951\\
0.411768310362079	0.577537646398383\\
0.411766536160633	0.577539535448562\\
0.411771008574618	0.577557599242415\\
0.411769404465588	0.577568390267755\\
0.411785618588373	0.577662796127751\\
0.411793892575265	0.577981746273396\\
0.411825175267394	0.578488279513507\\
0.411800122460608	0.579153413751195\\
0.412274317990363	0.580139255490988\\
};
\addplot [color=blue,solid,forget plot]
  table[row sep=crcr]{%
-0.063605745267399	0.580139255490988\\
-0.053864452415746	0.474779042078895\\
-0.0250393835346869	0.373732288183194\\
0.0216893596689859	0.281135859780862\\
0.0844086969581198	0.20078066553031\\
0.160550891294889	0.13595645652549\\
0.246998672296817	0.0893171434995797\\
0.340212857652503	0.0627721454077477\\
0.436377247661283	0.0574082175907777\\
0.531554860897361	0.0734449598837313\\
0.621849114691958	0.110225826172898\\
0.703563351685599	0.166245003470831\\
0.773352181414749	0.239209060076979\\
0.828358440990457	0.326130838940973\\
0.866330167682732	0.423451752223246\\
0.885712794539305	0.527187470299301\\
0.885712794539305	0.633091040682674\\
0.866330167682732	0.736826758758729\\
0.828358440990457	0.834147672041003\\
0.773352181414749	0.921069450904997\\
0.703563351685599	0.994033507511145\\
0.621849114691958	1.05005268480908\\
0.531554860897361	1.08683355109824\\
0.436377247661283	1.1028702933912\\
0.340212857652503	1.09750636557423\\
0.246998672296817	1.0709613674824\\
0.160550891294889	1.02432205445649\\
0.0844086969581198	0.959497845451666\\
0.0216893596689859	0.879142651201114\\
-0.0250393835346869	0.786546222798782\\
-0.0538644524157459	0.685499468903081\\
-0.063605745267399	0.580139255490988\\
};
\end{axis}
\end{tikzpicture}%
  %
  \tikzexternalenable
  %
  \caption{The evolution of the position estimate and path are visualized along the way at those points where the pushchair was momentarily stopped. The dashed line shows the final path for reference. The phone remained leaning on the top for the entire experiment. No position fixes nor loop-closures were used.}
  \vspace*{-2em}
  \label{fig:stroller} 
\end{figure}





\subsection{Generalized Dead-Reckoning}
\noindent
Wheel based motion and general non-legged motion are use cases which are not covered by conventional step counting PDR methods. We now set the method in a more general scope for general dead-reckoning that can be applied to any wheeled, sliding, or flying objects indoors or outdoors. Applications include push-carts, trolleys, robots, hover boards, quadcopters, \etc\
We include an example with a human manoeuvred wheeled object with an intrinsic noise source---that is a baby pushchair/stroller with a baby on board. Figure~\ref{fig:stroller-setup} shows the test setup, where the phone is placed leaning on the top. The phone (iPhone~6) remains fixed to the pushchair body for the entire experiment.

Walking was started from a stationary state, where the phone automatically performed ZUPTs. Along the route the pushchair was stopped irregularly and ZUPTs triggered if it became stationary enough. INo position fixes nor loop-closures were used. The only measurement data are the automatic zero-velocity updates, the barometer observations, and pseudo-updates constraining the momentary speed. The total path length was {$\sim$}93~m.
Figure~\ref{fig:stroller} shows the path estimate at the times when the pushchair was stopped. The dashed line is the path estimate at $t=108.3$~s which is shown for reference. The zero-velocity updates in the various heading angles of the pushchair are clearly enough to capture the bias estimates and stabilize the system. The final estimate is off from the starting point by {$\sim$}0.73~m (0.78\%). 
As the phone orientation is fixed to the pushchair and only moves in a plane, this use case is well suited for SHS sytems \cite{Harle:2013}. For comparison we implemented a 2D odometry method combining movement detection with turn rates projected to the horizontal plane \cite{Sarkka+Tolvanen+Kannala+Rahtu:2015}. The final estimate is only 1.80~m (1.94\%) off from the starting point at the end, which is good for an SHS method.


\subsection{Comments on Computational Complexity}
\noindent
The odometry method was implemented in C++ with wrappers in Objective-C for running on the device. The implementation uses the Eigen matrix library. The computational complexity scales linearly with the number of data points, meaning a constant computational burden per sample.

For development purposes, the method was run on the device hardware, but not online. For example, running the odometry for the track in Figure~\ref{fig:stroller} (108.3~s of data) took 0.30~s on the iPhone~6 hardware (single-threaded). Thus the method is capable of running in a real-time application.


\subsection{Sensing of Surroundings}
\noindent
The model proposed in this paper has many potential applications beyond simple odometry and tracking. For example, the orientation and ZUPT information can be used as a measurement tool per se. By consecutively placing the phone flat on the walls of a room, a model of the geometry of the room can be built. From each ZUPT on the trajectory, a wall can be projected parallel to the phone screen, thus capturing the geometry of the room. Associating several ZUPTs to the same wall with the additional knowledge that the points span a plane through the space, can also make it possible to better estimate the wall placement and orientation.

The model is flexible enough that new constraints---in form of estimated quantities and prior information about them---can be introduced. For this particular application there are several useful constraints, such as coplanarity between some ZUPT positions. A similar smartphone application that delivers these functionalities is publicly available. Therefore we seek to deliver comparable results to the RoomScan (Locometric~Ltd, \url{http://locometric.com}) application.

Conventional loop-closures are not suited for this particular purpose. In this case the loop-closure points are touching the same plane. This plane is a line in the $xy$-plane, and the coefficients of the equation of the line for each wall can be augmented in the state vector. Each ZUPT is thus an observation of a point and orientation on a line representing a wall. In our setup, we do not enforce any prior information about walls being orthogonal to each other, whereas we speculate that the RoomScan application enforces some shape constraint for the room.

Markers were placed upon the walls of a room of known geometry (7.30~m $\times$ 8.45~m). The phone was moved along the walls stopping for 3~s at each marker until arriving at the starting point, such that the first two markers were visited twice.
Figure~\ref{fig:roomscan} shows the results obtained by using our model. Measurement~\#1 was done by stopping at all the available markers, while measurement~\#2 was done using only every second marker. The resulting rooms are not exactly rectangular, but remarkably close considering no orthogonality constrains were implemented. The estimated size of the room was approximately 7.4~m $\times$ 8.3~m for measurement setup~\#1 and 7.4~m $\times$  8.4~m for setup \#2. In both cases the figure shows that in the beginning of the capture the ZUPTs have not matched the wall that well, but the next observations are well matching the wall planes.

For comparison, RoomScan measurements were performed on the same markers and following a similar trajectory between them. The RoomScan application gave a rooms size of 7.3~m $\times$ 8.3~m for measurement setup~\#1 and 7.0~m $\times$ 8.6~m for setup \#2. This means that the proposed method can deliver comparable results to the black-box method implemented in RoomScan. More testing would be necessary to make an objective comparison, but the purpose of the experiment was to showcase the flexibility and potential uses of the general model presented in this paper.






\begin{figure}[!t]
  %

  \pgfplotsset{
    trim axis right,
    yticklabel style={rotate=90},
    grid style={very thin,gray!25}
  }
  %

  \footnotesize\centering%
  %
  \begin{minipage}{.49\columnwidth}

    \tikzsetnextfilename{tikz-roomscan-1}
    %

    \setlength{\figurewidth}{1.15\columnwidth}
    \setlength{\figureheight}{0.7498\figurewidth}
    %
    \centering
    % This file was created by matlab2tikz.
%
%The latest updates can be retrieved from
%  http://www.mathworks.com/matlabcentral/fileexchange/22022-matlab2tikz-matlab2tikz
%where you can also make suggestions and rate matlab2tikz.
%
\begin{tikzpicture}

\begin{axis}[%
width=0.702\figurewidth,
height=\figureheight,
scale only axis,
xmin=-6.45410416957491,
xmax=3.03011617853089,
ymin=-9.95069585050696,
ymax=0.179977242785308,
hide axis,
axis x line*=bottom,
axis y line*=left
]
\addplot [color=white!50!blue,solid,forget plot]
  table[row sep=crcr]{%
-2.55559413998945e-06	-6.47523944921723e-07\\
-0.00180174437817575	-0.000314516006627426\\
-0.00332821462148213	-0.000676006897329831\\
-0.00474822811663937	-0.00100647822222753\\
-0.00637894705387544	-0.00137152294125098\\
-0.00838596440929132	-0.00185934423740939\\
-0.0101818814450529	-0.00225618934690969\\
-0.0117433904789081	-0.00272337600617772\\
-0.013233318776036	-0.00318393930335923\\
-0.0148774307196698	-0.0036815816899184\\
-0.016843152177464	-0.00425595085770948\\
-0.0186580299133136	-0.00485368831429146\\
-0.0202109182029744	-0.005359682983686\\
-0.0217010392488438	-0.00591544687399162\\
-0.0233440355972778	-0.00652614139012451\\
-0.0252620196399865	-0.00716511631267354\\
-0.0269950827859409	-0.00782203529559533\\
-0.0284878499430715	-0.00843763397958198\\
-0.0298930726559216	-0.00904931986853032\\
-0.0314401776610342	-0.0096904235106453\\
-0.0333112709117463	-0.0104710289676168\\
-0.0349552296423452	-0.0111508596785373\\
-0.0363587808475793	-0.0117810701627718\\
-0.0377234695538827	-0.0124430642149137\\
-0.0392356638556254	-0.0131842367566714\\
-0.0410290255847215	-0.0140672115436381\\
-0.0486008577260334	-0.0209212601422886\\
-0.0691312636048174	-0.0411458886481855\\
-0.0894351516253535	-0.0682295838579893\\
-0.0953675337155311	-0.104292789996428\\
-0.0725861980182308	-0.149607095984581\\
0.00783938000119833	-0.215315790303681\\
0.169777771089884	-0.29222422129447\\
0.399330296057479	-0.383873031289167\\
0.687184165423676	-0.516453251389339\\
0.989651553277339	-0.705296689414774\\
1.24978811234961	-0.907050177221814\\
1.46650225813912	-1.08216770255506\\
1.6371040002799	-1.20969145353697\\
1.77101364485133	-1.29288191756437\\
1.8676949220502	-1.33618058146159\\
1.91988474306431	-1.33513226975409\\
1.91832713852887	-1.3129051414105\\
1.87670225524746	-1.28830514403519\\
1.81877859258007	-1.26797379699342\\
1.748690599488	-1.24803908038823\\
1.67352399664002	-1.22346484551605\\
1.60442763448252	-1.19608281053197\\
1.54659609220633	-1.17122136755605\\
1.50792283784049	-1.15435575557855\\
1.49279023849524	-1.1484540998421\\
1.4919090747395	-1.14838655092129\\
1.49044237395314	-1.14844625215423\\
1.48680965670009	-1.14822072702827\\
1.48552459641314	-1.14793459546997\\
1.48474436288665	-1.14796386175408\\
1.48227524009183	-1.14892759776814\\
1.48079421483945	-1.14934988010357\\
1.47973801480659	-1.14955558672045\\
1.47888277852564	-1.14967013160533\\
1.47811505328915	-1.14974462835724\\
1.47734593173692	-1.14972108472646\\
1.47677618362103	-1.14964796028598\\
1.47640163199704	-1.14956888376124\\
1.47603959582703	-1.14949119044193\\
1.47568597206979	-1.14935990771759\\
1.47533111836829	-1.14920659890202\\
1.47503161139298	-1.14902744832542\\
1.47483921863848	-1.1488677443343\\
1.47469300805998	-1.14871236728842\\
1.4745365652031	-1.14850634766155\\
1.47437683026602	-1.14823266321972\\
1.47095475413079	-1.14476109273913\\
1.45027728596357	-1.13451444844226\\
1.3975892879104	-1.12724976993218\\
1.29991155418173	-1.12864348253497\\
1.18295729971698	-1.14852256108306\\
1.07447006080786	-1.20316905371932\\
0.978835182274458	-1.30838577051684\\
0.91523281617299	-1.42003165763891\\
0.898690092761159	-1.50851592748767\\
0.932194568002691	-1.57174797424539\\
1.01304822453612	-1.60301188291175\\
1.10661375337276	-1.58739394709746\\
1.1905735123182	-1.52461234880648\\
1.25139269290109	-1.44298535025696\\
1.28562443573248	-1.34535432094991\\
1.28587554350346	-1.25766124961136\\
1.26812657188182	-1.18249729907674\\
1.248617079941	-1.130121410359\\
1.23237968868647	-1.09228735197943\\
1.21992367131594	-1.0618662216784\\
1.21098371785818	-1.04202143251271\\
1.2072719575299	-1.03050947844838\\
1.20680023645442	-1.02971055825401\\
1.20679923377267	-1.02966589947727\\
1.20684517722532	-1.02959102312781\\
1.20690843301735	-1.02946937355406\\
1.20693101092311	-1.0293713373245\\
1.20697115215981	-1.02926374686312\\
1.20505584674366	-1.03080137287796\\
1.2005323948337	-1.03938882450679\\
1.19719014125841	-1.0499573214447\\
1.19507040434294	-1.05714846465238\\
1.19353232090299	-1.0621796620153\\
1.19226124827954	-1.06636518940366\\
1.19115629047012	-1.07018402829039\\
1.19019178986632	-1.07370120299682\\
1.1895719493209	-1.07601862354501\\
1.18917376465075	-1.07752400926077\\
1.18885709869558	-1.07865556512475\\
1.18857304955551	-1.07951509410693\\
1.1884180318683	-1.07997785906403\\
1.18730960012319	-1.07978876791835\\
1.17997428972806	-1.08046808407358\\
1.14167705194304	-1.09460028678701\\
1.05011934614488	-1.14468316239002\\
0.92955236977296	-1.23438166322769\\
0.837946230165023	-1.37514170938799\\
0.777706266626213	-1.55606107852483\\
0.746552492438414	-1.76011285630954\\
0.769540179950792	-1.93365779684475\\
0.853759646805683	-2.05039018040908\\
0.965297485043449	-2.13536944550113\\
1.08921286179768	-2.18753555595489\\
1.18603950081604	-2.2182142978164\\
1.25667698560399	-2.23311376209646\\
1.3014542003442	-2.24502884460256\\
1.3228659199299	-2.25315183802965\\
1.3261756076776	-2.26785822993549\\
1.32593464668102	-2.27916658454351\\
1.32334033989856	-2.2936070867106\\
1.32038205149506	-2.30606350099775\\
1.3153866891216	-2.31576436519687\\
1.31249065939768	-2.32237985422006\\
1.30956833952544	-2.3249508600896\\
1.30937042253682	-2.32503063982398\\
1.30936753860663	-2.32504172318615\\
1.309360314311	-2.32505087526871\\
1.30930957346881	-2.32502971722687\\
1.30909030560998	-2.32539218849077\\
1.30883022590015	-2.3254807659908\\
1.30862810057348	-2.32554153456758\\
1.30840040463765	-2.32559120563839\\
1.3081618620657	-2.32562163755115\\
1.30799440649431	-2.32566674583888\\
1.30793063628484	-2.32571070264268\\
1.30785604647284	-2.32575824159398\\
1.30779125889646	-2.32578539656766\\
1.30771998037533	-2.32577176714008\\
1.30770541164941	-2.32573329046483\\
1.30762439676932	-2.32576658806133\\
1.30795276286724	-2.32576418886596\\
1.31070714132395	-2.32460131646498\\
1.31153663038739	-2.32347496962899\\
1.30679219396237	-2.3263146514056\\
1.26467614548698	-2.35842592010192\\
1.18697375064151	-2.43788032566181\\
1.09719294880194	-2.55964696676821\\
1.0369705858539	-2.7368209549582\\
1.01334830026177	-2.92365075542853\\
1.03445475308208	-3.08671170675351\\
1.09231758659579	-3.24272156409418\\
1.19455232075077	-3.35816377956547\\
1.31936792792319	-3.37449699981994\\
1.42329387523757	-3.34697446598444\\
1.50473941286261	-3.31641749646319\\
1.5660404494235	-3.29412938123585\\
1.6094758784596	-3.28632423260198\\
1.63385721809718	-3.28369233865159\\
1.64977353662134	-3.28495541140718\\
1.65786003467423	-3.29010985299774\\
1.667202964102	-3.29717742650545\\
1.67305305866573	-3.30167376752141\\
1.67492053918092	-3.30259884642372\\
1.67485389047768	-3.30272677300176\\
1.67481541940309	-3.30271829897754\\
1.67488803498913	-3.30264468266385\\
1.67491152928824	-3.30262677266549\\
1.67498036244566	-3.30258368367262\\
1.6750961964159	-3.30259068365452\\
1.67517115920939	-3.30264636782677\\
1.6752018510786	-3.30265446274733\\
1.67522500245405	-3.30265158833065\\
1.67519959493225	-3.30264473889117\\
1.67521716716349	-3.30262548218991\\
1.67520151752825	-3.30258325058507\\
1.67515861116268	-3.30261260858899\\
1.67514101440334	-3.30264522925683\\
1.67512482981854	-3.30265565991994\\
1.67513741129594	-3.30264928879647\\
1.67511431423983	-3.30261874356462\\
1.67506166737098	-3.3026678327415\\
1.6750790596854	-3.30271104530123\\
1.67476940077994	-3.30294882118026\\
1.67305212945294	-3.30393481263977\\
1.66892218026567	-3.31012997347965\\
1.63196003441883	-3.34861576141754\\
1.5742755150682	-3.4413542933856\\
1.51802582543079	-3.58941592288085\\
1.46477521458953	-3.7861325713302\\
1.44071879438125	-3.96777949359356\\
1.47755601187108	-4.09533995374523\\
1.56476205089907	-4.17913068677133\\
1.66708494178339	-4.22770147866403\\
1.76104261338757	-4.25114366731011\\
1.84220260367057	-4.26511439770208\\
1.89867783436813	-4.27031138777243\\
1.92956646476263	-4.2702080792933\\
1.94036917769607	-4.27245995452349\\
1.94258031370364	-4.27982401386186\\
1.94134433586816	-4.28445367053623\\
1.94185736154348	-4.28680581137156\\
1.94139872967882	-4.28704524903401\\
1.941225048679	-4.28704029273315\\
1.94113657587402	-4.28709931053555\\
1.94106376066494	-4.28712844794022\\
1.94100046232382	-4.28706540602978\\
1.94094785447696	-4.28704026322678\\
1.94093912391633	-4.2870486985604\\
1.94095227450017	-4.28701542613328\\
1.940979784863	-4.28700951381923\\
1.94101023111072	-4.28701162672728\\
1.94100342883237	-4.28705500441518\\
1.94103683134265	-4.2870646744507\\
1.94101881712715	-4.28706888650117\\
1.94099726681328	-4.28709004610299\\
1.94097677639702	-4.28710097701698\\
1.94100722854499	-4.28707838540605\\
1.94102845922458	-4.28708426797287\\
1.94105678230817	-4.28708686052096\\
1.9410880408173	-4.28705293099211\\
1.94110368278844	-4.28708885287443\\
1.94099858022132	-4.28743745869174\\
1.93676622877941	-4.29007755392227\\
1.89174017529592	-4.3039203206156\\
1.79101359004797	-4.33209371390514\\
1.681632468712	-4.38755819863519\\
1.59177738859588	-4.48430331757547\\
1.53698292898923	-4.64528069641548\\
1.49797770919804	-4.84043629431061\\
1.50863830400116	-5.01134832703461\\
1.59595350137685	-5.13952699054002\\
1.7255302223792	-5.23474848446789\\
1.86282613884277	-5.2879220260394\\
1.98801382131882	-5.30178800045964\\
2.09058576687315	-5.28989818982088\\
2.17053241718456	-5.26953708259921\\
2.21333711755085	-5.25590650060362\\
2.23652861742935	-5.2530387964444\\
2.25123822184659	-5.25330355245661\\
2.26036764170208	-5.25538868630126\\
2.26614144452038	-5.26104678276303\\
2.26810974909371	-5.26341081469942\\
2.27044666634221	-5.26451483176574\\
2.27055100556702	-5.26524679337105\\
2.27024784016675	-5.26482404993575\\
2.27024304086418	-5.26483896056349\\
2.27016375406794	-5.26476151246584\\
2.27010730125301	-5.26475677264064\\
2.27007003068398	-5.26478718433621\\
2.27008533362769	-5.26477933473058\\
2.2701164666742	-5.26472640011545\\
2.27006269189466	-5.2646944106748\\
2.27009975367116	-5.26469122200979\\
2.27009344448676	-5.2646820457414\\
2.27010782861921	-5.26467917773658\\
2.27004319621023	-5.26470779764773\\
2.27000794759619	-5.26475962246557\\
2.27000430608154	-5.26477483829381\\
2.26997858288752	-5.2647891646646\\
2.26999098949719	-5.26480437084029\\
2.26996293896587	-5.26485437119342\\
2.26998011502396	-5.26483016822054\\
2.27013842217975	-5.26519218772467\\
2.26621636954315	-5.26361681365946\\
2.2333783521268	-5.27921270436897\\
2.15470508480583	-5.33545967673732\\
2.04432933637736	-5.40862548143618\\
1.94510320305504	-5.51373830612304\\
1.88899227677734	-5.66077521652555\\
1.85677626615147	-5.83940308606307\\
1.87092375819145	-5.99585476052089\\
1.92670048452317	-6.10626181699459\\
2.03481528947512	-6.19463980824855\\
2.16863330286767	-6.25065198397038\\
2.2899641129184	-6.27200585202963\\
2.3849521749942	-6.26377610080181\\
2.45853427469431	-6.26170782121125\\
2.50368404171097	-6.25807671767233\\
2.52698367919788	-6.25357567927561\\
2.54207055812326	-6.25077152664724\\
2.55145009769151	-6.24794612989504\\
2.55436393735059	-6.24933009421144\\
2.55567747011667	-6.25044909522822\\
2.5555165531243	-6.25123500109444\\
2.55552147488517	-6.25126751928292\\
2.5556377460627	-6.25158642047116\\
2.55535712788958	-6.25076690144483\\
2.55548301595366	-6.25092128953365\\
2.55548484536403	-6.2509245667777\\
2.5555276902577	-6.25092325753502\\
2.5555348634761	-6.25090642971566\\
2.55557001151284	-6.2508955864047\\
2.55560354473159	-6.25093283228744\\
2.55568886530035	-6.25091044506833\\
2.55575226194157	-6.25089611385827\\
2.55584252373876	-6.25087319904322\\
2.55594152158557	-6.25084622092903\\
2.55597567255411	-6.25080568362294\\
2.55602922373857	-6.25076456598448\\
2.55615971939839	-6.25072423223952\\
2.55623795610676	-6.2506999456633\\
2.5563759983565	-6.25064232212507\\
2.55645935969851	-6.25060068812601\\
2.55650433676701	-6.25056217698402\\
2.55662363719362	-6.25052011967979\\
2.55666082463942	-6.25048377392859\\
2.55698385841349	-6.25044137818929\\
2.56142847975506	-6.24977956020332\\
2.56685060905435	-6.25083321861139\\
2.55987330261504	-6.25883311641201\\
2.51489756193407	-6.30550019472602\\
2.4359818923568	-6.39852643300253\\
2.35249048524387	-6.53490309570503\\
2.30134522185674	-6.69602778725013\\
2.2891014656077	-6.85242467536337\\
2.33453817506565	-6.98146028897714\\
2.42636427432276	-7.08525679846494\\
2.53641248243317	-7.15489803296146\\
2.64350605577676	-7.17959523873768\\
2.7327487832476	-7.18131354975987\\
2.79527431860447	-7.1843228855115\\
2.83153448073273	-7.17991943071328\\
2.85152601696091	-7.1739022917497\\
2.86462990006791	-7.16711849040921\\
2.86950151981013	-7.16430145640329\\
2.86959997586211	-7.16431365334712\\
2.86964427849157	-7.16429885506816\\
2.86971577200107	-7.16427235998471\\
2.86979496775979	-7.1642476070898\\
2.86983378390927	-7.16419460743467\\
2.86985174157825	-7.16422518560749\\
2.86986901879007	-7.16424165164262\\
2.86986359223844	-7.16424234974945\\
2.86982958645061	-7.16426742032331\\
2.86982550872743	-7.16425520319119\\
2.86988143593758	-7.16426525465046\\
2.86988574695147	-7.16424748259974\\
2.86988339318919	-7.1642553464566\\
2.86989796013988	-7.16427776757361\\
2.86986174620411	-7.16429840315493\\
2.86983656470286	-7.16431409415508\\
2.86986525101142	-7.1642933839989\\
2.87002132289456	-7.16451890344222\\
2.87237264922978	-7.16819041469328\\
2.86982183653876	-7.17540593705414\\
2.83828172215561	-7.20538439911647\\
2.78307069423373	-7.2720815695996\\
2.71918097986257	-7.36670445684345\\
2.66566934877832	-7.48685891616854\\
2.63443224069241	-7.60305082394477\\
2.62994182042775	-7.69118966781953\\
2.64951848898393	-7.76720970734051\\
2.68360468086913	-7.82727191891841\\
2.71122445672862	-7.87545239817672\\
2.72730365631033	-7.90176960197697\\
2.73285226304991	-7.9153503296111\\
2.73454636505036	-7.92279221888186\\
2.73457102928391	-7.92296861932432\\
2.73450237804792	-7.92284772194804\\
2.73446265759515	-7.9228094937937\\
2.73447532015031	-7.922876443077\\
2.73442137909244	-7.92286270830895\\
2.73432430777966	-7.92289363321161\\
2.734203854817	-7.9229022911489\\
2.73416381926002	-7.92290700889922\\
2.73426630614382	-7.92299817544612\\
2.73422974002072	-7.92297504749346\\
2.73422731530023	-7.92297241434554\\
2.73420865790725	-7.92295130874326\\
2.73422753567078	-7.92296423763955\\
2.73423292775699	-7.92293773323718\\
2.7341934867077	-7.92291735327503\\
2.73419290401684	-7.92287455638994\\
2.73417215360444	-7.92285717353116\\
2.73415624854242	-7.92287569474619\\
2.73411849994563	-7.92287267409633\\
2.73408765949125	-7.92279387765946\\
2.73293964143982	-7.92207511400816\\
2.71351268280217	-7.90007524502409\\
2.63766283720242	-7.83400346657291\\
2.52655519409259	-7.78451110134507\\
2.393513444633	-7.75307633554775\\
2.24267272772226	-7.72364065204667\\
2.07921940733531	-7.71531309591621\\
1.94970616846214	-7.74830609023793\\
1.85708906933395	-7.83967547393765\\
1.79709100913544	-7.95241224378862\\
1.76841070225422	-8.04834427056134\\
1.77392487898798	-8.12106827095282\\
1.78784909996355	-8.1721723075377\\
1.80197449881626	-8.19593481362372\\
1.80946872739831	-8.20183887146755\\
1.81131768297205	-8.20168948901789\\
1.81121076284982	-8.20167341323646\\
1.81120058874674	-8.20163950274665\\
1.81119861087844	-8.20158219561102\\
1.8111815329064	-8.20158495281171\\
1.8111477466562	-8.20154083495878\\
1.81112413886412	-8.2015532744328\\
1.81113155290372	-8.20157629912046\\
1.81112037968229	-8.20158339655083\\
1.81110368700075	-8.20158024717184\\
1.81110593938576	-8.20158430464411\\
1.81111051801921	-8.20158925064351\\
1.81109078784612	-8.20158456449279\\
1.81109382453382	-8.20160230595339\\
1.81108979931182	-8.20163077697908\\
1.81107653684674	-8.20162914306028\\
1.81108551067082	-8.20166197242457\\
1.81110153781702	-8.20170994878789\\
1.81111164191058	-8.20174720935346\\
1.811124383348	-8.20173413784877\\
1.81107122820812	-8.2017414442488\\
1.80864166337772	-8.20240054033257\\
1.80326432733981	-8.19880638388614\\
1.7749832015191	-8.16286634048201\\
1.70810173981485	-8.10634503190651\\
1.57668007073745	-8.06166877661325\\
1.37888319344962	-8.02374910460484\\
1.14938668407738	-8.00584649878778\\
0.986889519103124	-8.04462291925189\\
0.891944934853891	-8.13159833017524\\
0.85103035744609	-8.2315806813034\\
0.843606994604099	-8.31587984409503\\
0.851226441703104	-8.38465312131135\\
0.864943099188587	-8.43236633195534\\
0.876613370980793	-8.45312488789909\\
0.885064205668065	-8.46150923241481\\
0.891140552845754	-8.46779817005118\\
0.890955688989363	-8.46839299395193\\
0.890958474145519	-8.46773281665852\\
0.89098262919477	-8.46762843434119\\
0.8910024685444	-8.46765915613994\\
0.890983664626444	-8.46767605058693\\
0.890978297671485	-8.46766588657688\\
0.891008623398046	-8.46768710643876\\
0.891036530306346	-8.46776785049014\\
0.891039186560472	-8.46779507741653\\
0.891037060738128	-8.46777585580492\\
0.891058159444152	-8.46776658958863\\
0.891050113646337	-8.4677399736408\\
0.891044791385918	-8.46774523917908\\
0.89104595371959	-8.46778614003831\\
0.891037478150687	-8.46779839951072\\
0.891029979802228	-8.46777395754564\\
0.891021290120768	-8.46780508730859\\
0.891019980220853	-8.46779110798307\\
0.891055299623426	-8.46779149471835\\
0.891039288762773	-8.4678004998804\\
0.890997847271253	-8.46781216858019\\
0.889888187790342	-8.46800524004242\\
0.887796264134016	-8.46488881982604\\
0.875055721320652	-8.43534306758133\\
0.823283298963927	-8.39160383465802\\
0.718927405855125	-8.35311645192236\\
0.550315579135835	-8.32805536094657\\
0.351234389988531	-8.31484232928111\\
0.180668587720833	-8.32266460228708\\
0.0586717372299277	-8.36867261831326\\
-0.0194883945219833	-8.45332831169661\\
-0.0684748160321265	-8.55363107981856\\
-0.0975693944164399	-8.62994162062829\\
-0.103708098208781	-8.69304976230396\\
-0.0922042357076032	-8.72964402418765\\
-0.0863491672100673	-8.74551220130129\\
-0.0754373907453621	-8.7506653048954\\
-0.0668410803891836	-8.75309298557174\\
-0.0633250115504922	-8.75476197559431\\
-0.0626760381238074	-8.754571808093\\
-0.0628183836726457	-8.75458395486672\\
-0.0629654312030569	-8.75463635425889\\
-0.0630010425826595	-8.75467994318734\\
-0.0629645407629057	-8.7546826906505\\
-0.062963654847649	-8.75469552510444\\
-0.0630011789508333	-8.7546900967825\\
-0.063027665989785	-8.75464559100157\\
-0.063070446368616	-8.7546494703955\\
-0.0630983381852723	-8.75466846226599\\
-0.0631606311139928	-8.75468181286888\\
-0.0632053237336132	-8.75464684230442\\
-0.0632119512247064	-8.75463885570183\\
-0.0632069401217911	-8.75462252046495\\
-0.0632334003024925	-8.75462318960325\\
-0.0632379948470614	-8.75463152181889\\
-0.063254474610047	-8.75465400416045\\
-0.0632276776794863	-8.75466677941481\\
-0.0631943294206444	-8.75469742525838\\
-0.0632232168118493	-8.75471365664265\\
-0.0661567906091274	-8.7537106411356\\
-0.0841663412332678	-8.73209347616692\\
-0.144655695913502	-8.68334921522094\\
-0.263009257603937	-8.64163123856051\\
-0.423693014068594	-8.61644905166917\\
-0.603841570457029	-8.61230292590939\\
-0.764003282671045	-8.6379515436349\\
-0.884332462010141	-8.69407429043006\\
-0.986696295191715	-8.78327218712173\\
-1.04942111663158	-8.87196903362229\\
-1.06697726890876	-8.94052672294394\\
-1.06286523072727	-8.9971493661314\\
-1.06240170671051	-9.03101572538209\\
-1.05733082346881	-9.04922674646407\\
-1.04644749565696	-9.05729086671865\\
-1.04526307427344	-9.05802750062329\\
-1.04525615987279	-9.05804164385455\\
-1.04527028089667	-9.05804096315372\\
-1.04529680961054	-9.05802709369265\\
-1.04530533524148	-9.05802564263729\\
-1.045273495762	-9.05801565079515\\
-1.04527852378847	-9.05804353943278\\
-1.04530903234433	-9.05803292924578\\
-1.04531199693635	-9.05798573763448\\
-1.04530877473111	-9.05799043739652\\
-1.04529871199711	-9.05802817212648\\
-1.04531626277415	-9.05809140376642\\
-1.04531361694592	-9.05810450115618\\
-1.04532011070546	-9.05811113184917\\
-1.04533237468482	-9.05813263307016\\
-1.04530759439333	-9.05811149555037\\
-1.04529210996694	-9.0581038171235\\
-1.04529532524827	-9.05810955313704\\
-1.04530679968007	-9.05809661499683\\
-1.04545785909077	-9.05808021711641\\
-1.05035082960945	-9.0542934359835\\
-1.08978046866984	-9.00590162293957\\
-1.1923139335464	-8.93752179774184\\
-1.35550484274811	-8.86259630489843\\
-1.54447523769326	-8.80834703696216\\
-1.71415445489546	-8.80441817027932\\
-1.8489218608357	-8.8483021325671\\
-1.93705627578011	-8.93729282736583\\
-1.97782919148624	-9.04782426467189\\
-1.98131787158318	-9.13992692932526\\
-1.96249856457703	-9.21321879509052\\
-1.94201759627113	-9.26639039996462\\
-1.92730320069944	-9.29299866139621\\
-1.91990565930722	-9.30468116861022\\
-1.92073027548532	-9.30807113915369\\
-1.92059026377224	-9.30967801516749\\
-1.9205902135438	-9.30966009273989\\
-1.92059738412602	-9.30964795177947\\
-1.92059795388335	-9.30962397245801\\
-1.92060501643403	-9.30962370673023\\
-1.92062340083344	-9.30962749830302\\
-1.92063560561756	-9.30963510912957\\
-1.92062671090153	-9.30955919018241\\
-1.92063941223785	-9.30953959536593\\
-1.92066068730186	-9.30949872003272\\
-1.92068556182092	-9.3094810986139\\
-1.92070077695282	-9.30945059071419\\
-1.9207310337556	-9.30944612014607\\
-1.92075335913325	-9.30940445010727\\
-1.92076057935769	-9.30934358436609\\
-1.92077915564329	-9.30929464900596\\
-1.92084434300023	-9.30924092813993\\
-1.92389030375773	-9.30590287733763\\
-1.93865138564783	-9.27579190056022\\
-1.97459185587489	-9.19166603130624\\
-2.03116659381504	-9.07263612060942\\
-2.10912220434032	-8.91240443566839\\
-2.20951378726117	-8.69065465564138\\
-2.29637659699212	-8.43148764441527\\
-2.37556126592682	-8.14307880550279\\
-2.45239378387032	-7.85995493835585\\
-2.53957995155007	-7.64617688179315\\
-2.66450998172687	-7.47933676095346\\
-2.84094894203706	-7.35103965952796\\
-3.01969426633998	-7.27503674391039\\
-3.18773528937912	-7.23127569131784\\
-3.39429226979332	-7.21242264628404\\
-3.61327240578952	-7.25974584560924\\
-3.81228911406177	-7.36815654101036\\
-3.98637561999927	-7.51228676816115\\
-4.0999290020181	-7.67145302094918\\
-4.14829552501987	-7.83940525484213\\
-4.15033628411612	-8.00626909481995\\
-4.10564657625229	-8.16834055281893\\
-4.04015264786898	-8.36023140265848\\
-3.95742890016548	-8.56866847139622\\
-3.87534401925774	-8.7722563506107\\
-3.77937143266804	-8.98614463708649\\
-3.65403437025069	-9.17654362404088\\
-3.51826748658563	-9.31963707482874\\
-3.39316336638635	-9.44540578117146\\
-3.2931148771604	-9.55190427280085\\
-3.23811008698107	-9.62301811122249\\
-3.22278493498124	-9.65630359896042\\
-3.22158301036093	-9.68060570990711\\
-3.2234737846916	-9.69542902005288\\
-3.22775861177055	-9.69828451349093\\
-3.23100554016502	-9.69852366626635\\
-3.23174222626373	-9.69860979637126\\
-3.23169108518917	-9.69860919746136\\
-3.23167200338763	-9.69858785418742\\
-3.23167381944011	-9.69867213611926\\
-3.23164518810568	-9.69865642743461\\
-3.23160167083074	-9.6986546453316\\
-3.23161947379922	-9.69872089113668\\
-3.23158707848986	-9.69877199926396\\
-3.2315514046735	-9.69880503665463\\
-3.23154982297057	-9.69883389433684\\
-3.23154039480411	-9.69886825571272\\
-3.23153384707737	-9.69889953029843\\
-3.2315452614137	-9.69894858662823\\
-3.23155443677052	-9.69898037486868\\
-3.23154689893549	-9.69897985541634\\
-3.23155946574586	-9.69898417030679\\
-3.23155248025911	-9.69901350304949\\
-3.23158773758273	-9.6989945162645\\
-3.23118079763989	-9.69649463822278\\
-3.23119498722119	-9.6854666301572\\
-3.26004804344716	-9.64523129305003\\
-3.36048277640166	-9.59822618842427\\
-3.52184402652058	-9.55504751099722\\
-3.71659053152815	-9.50140096924709\\
-3.91770453309399	-9.42853419278041\\
-4.07302042027222	-9.35348178496127\\
-4.19565789598864	-9.30001045697089\\
-4.28929380337966	-9.26617651869243\\
-4.34524293222172	-9.24409305997379\\
-4.36690573108167	-9.23943517643234\\
-4.38097877829673	-9.23413121375806\\
-4.38839324021141	-9.229544548572\\
-4.38854720752388	-9.22833517465928\\
-4.38784333063029	-9.22735352039696\\
-4.38744461591054	-9.22724027966684\\
-4.38736230409212	-9.22766969209545\\
-4.3873700758503	-9.22773598448761\\
-4.38740190493684	-9.22773380379614\\
-4.38740755091312	-9.22775793838868\\
-4.38745295482237	-9.22774529554299\\
-4.38742017141488	-9.22776677592192\\
-4.3873880181307	-9.22781460131106\\
-4.38736436745113	-9.22779197986368\\
-4.38735887892422	-9.22779603002747\\
-4.38738140152474	-9.22782840355084\\
-4.38732694522161	-9.22782708155357\\
-4.38729412780289	-9.2278407339976\\
-4.38728116866375	-9.22782699835128\\
-4.38724849733091	-9.2278038209016\\
-4.38716412625923	-9.22779489018276\\
-4.38593639823484	-9.22831049495881\\
-4.3817447616481	-9.23167262987717\\
-4.34824054937238	-9.22332366347889\\
-4.29547182742179	-9.17708783700046\\
-4.24200163512261	-9.08455831410594\\
-4.1925847823773	-8.94193809072413\\
-4.15036969145817	-8.77554950940702\\
-4.15121334931292	-8.6194468678924\\
-4.20350347268727	-8.48119486929506\\
-4.2975994019893	-8.36488529150764\\
-4.40214995961014	-8.2804285471826\\
-4.489041427608	-8.25095969013439\\
-4.55892681709454	-8.2412728173413\\
-4.60026279187439	-8.24221366019092\\
-4.61662359488384	-8.24514033200852\\
-4.62701941965598	-8.24500289768381\\
-4.62972955631481	-8.24576102793131\\
-4.63282124880262	-8.24832105496777\\
-4.63597279319632	-8.2485073689562\\
-4.63606774465141	-8.24830647660792\\
-4.6359996465961	-8.24833062776093\\
-4.63591258002139	-8.24832262234839\\
-4.6358398402614	-8.24833377915561\\
-4.63586065120829	-8.24831289365229\\
-4.63587061858638	-8.2483452324952\\
-4.6358924589397	-8.24841449192293\\
-4.63599654329119	-8.248456418756\\
-4.63603909027258	-8.24846902257832\\
-4.63601229253395	-8.24846269696434\\
-4.63602689109766	-8.24844962515119\\
-4.63603929532102	-8.24844070901377\\
-4.63602478864633	-8.2484705086712\\
-4.63610127025772	-8.24846458314262\\
-4.6360594658843	-8.24841636674787\\
-4.63587683069682	-8.2483475625515\\
-4.63586302264982	-8.24840134130213\\
-4.63624747119615	-8.24775825613064\\
-4.63631830299328	-8.24798570835979\\
-4.60386813132345	-8.24212403509138\\
-4.53273965353245	-8.19451873463271\\
-4.46815899056194	-8.08734815703581\\
-4.43870121452029	-7.92684913994552\\
-4.44048820711274	-7.76344763611998\\
-4.48208996446733	-7.64188341262376\\
-4.56870034718914	-7.54958173011322\\
-4.679484967999	-7.47550178656583\\
-4.76743313350451	-7.42241136595623\\
-4.83908212640835	-7.39511911293455\\
-4.89316596620998	-7.37827128302509\\
-4.92448871071878	-7.36340744071714\\
-4.94397226063828	-7.35196663241905\\
-4.95303616606578	-7.3454566830201\\
-4.9575346707424	-7.35121483679481\\
-4.95796694595582	-7.35523556513903\\
-4.95827492984313	-7.35626846223645\\
-4.95836907013378	-7.35605058549627\\
-4.95836842247591	-7.35609806357834\\
-4.95835684282592	-7.35617319074879\\
-4.95842579186467	-7.35617832069646\\
-4.95844775492298	-7.35625400935695\\
-4.9584971925316	-7.3562911308506\\
-4.95853909885765	-7.35635607147045\\
-4.95854603761411	-7.35638020094375\\
-4.95852042046056	-7.35634744440028\\
-4.95859373457252	-7.35639557784945\\
-4.95861061432999	-7.35642107892425\\
-4.95861233618162	-7.35645971869496\\
-4.95863370945013	-7.35650363671435\\
-4.95865756534122	-7.35656764999954\\
-4.95907825595104	-7.35665987347193\\
-4.96040108400561	-7.35717280314157\\
-4.96246795172169	-7.35806290809885\\
-4.96486615165143	-7.35910294203166\\
-4.94363520809863	-7.3712358206658\\
-4.87616489227735	-7.4056752863191\\
-4.78474864793915	-7.43889829779342\\
-4.63095727938036	-7.43841701129208\\
-4.4586829779895	-7.39861305396411\\
-4.28357434778156	-7.32813746971568\\
-4.14886628244768	-7.20733641291331\\
-4.04970134344595	-7.03838741168151\\
-4.00234581760279	-6.84508013130109\\
-4.04137822141505	-6.70776229551107\\
-4.14169703618308	-6.59776731478223\\
-4.26427686061017	-6.47993270931074\\
-4.41204348861259	-6.38036558029433\\
-4.57721319409101	-6.3399748063755\\
-4.7550075483602	-6.33162407960508\\
-4.9201981300496	-6.33282514728402\\
-5.05740743989864	-6.34662140404752\\
-5.15049074509351	-6.36919886626936\\
-5.1998749611919	-6.39194241185148\\
-5.2182119115491	-6.4022177828236\\
-5.21754873922927	-6.40238713529456\\
-5.21869684478818	-6.40212214800408\\
-5.2185120329085	-6.40130302989767\\
-5.21897968543064	-6.39942138835233\\
-5.2190036294288	-6.39943268411754\\
-5.21900995243635	-6.3993915432183\\
-5.21901039794364	-6.3993980250419\\
-5.2189905568183	-6.39939208287941\\
-5.21899658731367	-6.39940565047941\\
-5.21903973690527	-6.39944157919751\\
-5.2190664470915	-6.39950243822003\\
-5.21906523746285	-6.39951859057772\\
-5.21905749143904	-6.39954965105132\\
-5.21903685623672	-6.39956906621574\\
-5.21906252276902	-6.3996394187509\\
-5.2191233079474	-6.39972801426501\\
-5.21909429652508	-6.39974918669521\\
-5.21909177734686	-6.39976259279166\\
-5.21908676384851	-6.39977185608234\\
-5.21908051850496	-6.39979956955199\\
-5.21908442482394	-6.39979500200161\\
-5.21906617134542	-6.39977431532264\\
-5.21899254350688	-6.39955041584755\\
-5.21848880069028	-6.39801480653107\\
-5.21642947551692	-6.39630848734607\\
-5.19643081492214	-6.39187151225026\\
-5.14393051378383	-6.3511525244162\\
-5.06855610516371	-6.26354323406176\\
-5.00743185527826	-6.12221051826061\\
-4.97244903682821	-5.95164855325066\\
-4.96577273747464	-5.77751915984252\\
-5.01010544388714	-5.63758646809116\\
-5.1048247845225	-5.55046217632643\\
-5.19867983571528	-5.49966942880953\\
-5.28454419567391	-5.4770439376131\\
-5.35931274980302	-5.47802217773198\\
-5.40967412231173	-5.46859514524215\\
-5.44134070858276	-5.47065379855763\\
-5.45093605340194	-5.47130267321708\\
-5.45923630587328	-5.47249611754448\\
-5.4656360366325	-5.47149722813043\\
-5.47007060155139	-5.47077305921361\\
-5.46878429733678	-5.47150750981032\\
-5.46786507108843	-5.47285003334189\\
-5.46789775559551	-5.47289320656594\\
-5.46789297983389	-5.47296244970457\\
-5.46787706725358	-5.47303368212085\\
-5.46782087356956	-5.4730565315363\\
-5.46779752819909	-5.47305148479932\\
-5.46783846549557	-5.47306031204565\\
-5.46782988916273	-5.47308342321384\\
-5.46782838403806	-5.47311221231562\\
-5.46782334834527	-5.47310254660927\\
-5.46772357804192	-5.47307107393274\\
-5.46768728836009	-5.47308846096688\\
-5.46769678484197	-5.47310189921066\\
-5.46766176037498	-5.47310328787597\\
-5.46764224995486	-5.4731270895966\\
-5.46765421101314	-5.4731340646927\\
-5.46764056486843	-5.473163551356\\
-5.46759772386109	-5.47316393198276\\
-5.46761113467873	-5.47310649830015\\
-5.46753823389911	-5.47314657529419\\
-5.46671047954043	-5.47326636805199\\
-5.46230039878303	-5.47376781267152\\
-5.44910431491993	-5.46994087629143\\
-5.40770626244183	-5.45574529004686\\
-5.36801079689822	-5.4088208271125\\
-5.31363133118515	-5.34386274521381\\
-5.24492092050589	-5.26457707515831\\
-5.19010632500728	-5.15063424596614\\
-5.15968361553718	-4.99500373050464\\
-5.14896599333586	-4.84562234597696\\
-5.18964471177557	-4.71485722217826\\
-5.3086260659979	-4.61455893994098\\
-5.46633199891805	-4.51119943120664\\
-5.57981910105237	-4.44974134847772\\
-5.66180525680189	-4.42685132329835\\
-5.71789219093452	-4.41660759249502\\
-5.75420210426331	-4.41631389572889\\
-5.76834599256148	-4.42538262261046\\
-5.76787695578385	-4.43585078090015\\
-5.76844943951078	-4.44600090115881\\
-5.76998719108394	-4.45264005135869\\
-5.77107329274282	-4.45536682921868\\
-5.77187861200848	-4.45703544100279\\
-5.77149165228721	-4.4577020134668\\
-5.77129696633033	-4.45632797796747\\
-5.77192355906023	-4.45303162729118\\
-5.77177095440821	-4.45306909832409\\
-5.77145201159696	-4.4533680754708\\
-5.77127621217513	-4.453879652486\\
-5.77130084426376	-4.45387108559403\\
-5.77127909825411	-4.45379639590214\\
-5.77127768403885	-4.45360979575811\\
-5.77129631702887	-4.45344395321447\\
-5.77126579833694	-4.45338296347969\\
-5.77122180836686	-4.45338467458119\\
-5.7711516908522	-4.45335859124737\\
-5.77111108531157	-4.45332064006668\\
-5.77109980359106	-4.45330486021937\\
-5.77113273754083	-4.45333537484473\\
-5.7711251723977	-4.45329284496894\\
-5.7711295104881	-4.45326275045935\\
-5.77116042433025	-4.45329054619666\\
-5.77116775405355	-4.45328318517251\\
-5.77117115187834	-4.45327665719092\\
-5.77121717985786	-4.45327543165548\\
-5.77121798360839	-4.4533152664252\\
-5.77120286248191	-4.45329977171608\\
-5.77123535256416	-4.45330607902658\\
-5.77120798345621	-4.4532810870649\\
-5.7711481948492	-4.45324952938431\\
-5.77117656147252	-4.453236496954\\
-5.77116416879358	-4.45326507967699\\
-5.77114260781149	-4.4533346604175\\
-5.77110899134546	-4.45336758848379\\
-5.7709432086022	-4.45396258248303\\
-5.76987033225675	-4.45779770680961\\
-5.76815886551431	-4.46355325354978\\
-5.75803855907256	-4.46929874605703\\
-5.74640284347195	-4.47432963656443\\
-5.71287843729217	-4.45989023631199\\
-5.67651846266053	-4.39598401961421\\
-5.6742842954554	-4.29956703063259\\
-5.70508549653335	-4.15608319401128\\
-5.7729087645596	-3.97046664455027\\
-5.86281582602009	-3.78810152301123\\
-5.93580331039181	-3.65597579218781\\
-5.97665374514461	-3.61684421423461\\
-6.00610900730609	-3.61263204180441\\
-6.01676651239971	-3.60348098509461\\
-6.01567395794824	-3.61330462447111\\
-6.01399618502467	-3.61597040420093\\
-6.01346087089769	-3.62443108958484\\
-6.01308692182083	-3.63129697587462\\
-6.01073163990714	-3.63590406396519\\
-6.00926101446999	-3.63819965231042\\
-6.00855542706557	-3.63849747875665\\
-6.00843679534523	-3.63849494861749\\
-6.0084847782852	-3.63848892320226\\
-6.00815888145236	-3.63741992027697\\
-6.00916886304609	-3.63615387862372\\
-6.00958698448306	-3.63472721135983\\
-6.0097698938173	-3.63373834666641\\
-6.0097134361053	-3.63363523386164\\
-6.0098181102481	-3.63352730491581\\
-6.00982605364702	-3.63359038807411\\
-6.00985165349238	-3.63360448982296\\
-6.00983541662433	-3.63358933760941\\
-6.00980736775285	-3.63357780543645\\
-6.00977142772204	-3.63359970593833\\
-6.0097949997168	-3.63367601550252\\
-6.00979861313478	-3.63370605110735\\
-6.00979747305087	-3.63371885135694\\
-6.00981372830132	-3.63375641643161\\
-6.00979227035193	-3.63376886021844\\
-6.00977384095155	-3.63380611674061\\
-6.00978831699	-3.63381909480151\\
-6.00980883347043	-3.63388439726985\\
-6.00979670151731	-3.63389754147991\\
-6.00979103663681	-3.63392668383371\\
-6.00981534981391	-3.63395464945832\\
-6.00976258475488	-3.63393869900224\\
-6.01004806912072	-3.63194454488343\\
-6.00882224582293	-3.63098620656351\\
-6.00342015268459	-3.63692940441333\\
-5.99047815079792	-3.64235148181906\\
-5.95915178819255	-3.65061638751019\\
-5.92351656121802	-3.66557767234791\\
-5.85272523409188	-3.70437785144883\\
-5.7470080565869	-3.76472072698195\\
-5.62644148519383	-3.84088833382491\\
-5.49016767133487	-3.88918931799437\\
-5.32645960588736	-3.90733547135132\\
-5.1239274061296	-3.90933509181476\\
-4.93048681787089	-3.89066415452085\\
-4.77960782772483	-3.82102317569107\\
-4.66328458248206	-3.70465158338975\\
-4.57964017102503	-3.5703594025467\\
-4.54370106795801	-3.45488376563673\\
-4.53252919903863	-3.33447435326603\\
-4.53222761290786	-3.20227650794912\\
-4.54885733372021	-3.07746085182371\\
-4.61675938782686	-2.94279539574402\\
-4.72584293726194	-2.80759839671749\\
-4.82330652809118	-2.7100890807961\\
-4.94099659026218	-2.61989403602209\\
-5.10285368312341	-2.52439677123013\\
-5.30704451072179	-2.44851035395134\\
-5.48447307735558	-2.42400475072372\\
-5.64358758175562	-2.43053695286666\\
-5.80253736580783	-2.45933168572723\\
-5.94001851230187	-2.51668257463953\\
-6.0392435638102	-2.56840768568407\\
-6.12264025213344	-2.63048657938241\\
-6.16961154887903	-2.67811598533449\\
-6.19794162358328	-2.72631812616079\\
-6.21510763460544	-2.76055879165372\\
-6.22420913443486	-2.78695171064146\\
-6.22475600447389	-2.80001490033211\\
-6.221903992065	-2.80790295029312\\
-6.21927846285704	-2.80866570918322\\
-6.21830505641654	-2.81147148037513\\
-6.21993593111166	-2.81200399730129\\
-6.21858643435064	-2.81454823121264\\
-6.21976148358005	-2.81703283119697\\
-6.22065398259371	-2.81973658724977\\
-6.21915927497031	-2.81994520132113\\
-6.2188931049896	-2.81849377669176\\
-6.21800779414109	-2.81777366022546\\
-6.21736458432649	-2.8175109022982\\
-6.21704442855131	-2.81772646310222\\
-6.2169610490101	-2.81812877176575\\
-6.21699468526492	-2.81806490246504\\
-6.21695355545139	-2.8181544941597\\
-6.21702192930884	-2.81819344856539\\
-6.21704946473579	-2.81821950418011\\
-6.21709368724584	-2.81819117269049\\
-6.21713437774812	-2.81816154291338\\
-6.21717052680154	-2.81817140265936\\
-6.21715229382632	-2.81822250220566\\
-6.21717364717853	-2.81820388246607\\
-6.2171789333473	-2.81821590338863\\
-6.21714161417301	-2.81821403184388\\
-6.21713078257435	-2.81818989011946\\
-6.21716766424434	-2.81832518498721\\
-6.2171630650593	-2.81854348085722\\
-6.21673667910822	-2.81950290664332\\
-6.21671791205278	-2.81984412434468\\
-6.21202543687586	-2.8171018067107\\
-6.1830958740209	-2.78736322363465\\
-6.13259993218386	-2.70307815110688\\
-6.08126465812687	-2.5756446897894\\
-6.03617871378021	-2.41544874300626\\
-5.9779580878006	-2.24917152759074\\
-5.92933869428845	-2.10196297595417\\
-5.92380335124514	-1.98609881004469\\
-5.93706519066058	-1.90929834328058\\
-5.9577330688193	-1.86050204033385\\
-5.96375243540578	-1.84211063876913\\
-5.96854598232772	-1.83225350278376\\
-5.97139113019453	-1.82613145560958\\
-5.97155727628285	-1.82475646641042\\
-5.96988054727987	-1.82328086858414\\
-5.97112211775922	-1.82303665160165\\
-5.97255866859673	-1.82310068215187\\
-5.97255017007536	-1.82295391996773\\
-5.97231001428093	-1.82274713944925\\
-5.97180269530775	-1.82256620988211\\
-5.97161958155225	-1.82254035345315\\
-5.97161766885057	-1.82257882102647\\
-5.97158628317476	-1.82258196574846\\
-5.9716183920154	-1.82256729306377\\
-5.97159865992197	-1.82259218901546\\
-5.97157556244447	-1.82261965052363\\
-5.97159722670332	-1.8226602480668\\
-5.97157108161143	-1.82267181315701\\
-5.97160086686209	-1.82266327556079\\
-5.97163955319902	-1.82267740865836\\
-5.97166689749878	-1.82265759335839\\
-5.97167775318	-1.82264540268505\\
-5.97169697602937	-1.82266846499389\\
-5.97169570960027	-1.8226575675728\\
-5.97170606038097	-1.82265779746215\\
-5.97144767924959	-1.82263798936707\\
-5.96973810860333	-1.82319587626217\\
-5.96864035385705	-1.83201684759067\\
-5.96585248737132	-1.83851472030341\\
-5.96919076374661	-1.86384919234376\\
-5.95811763086939	-1.95035587557123\\
-5.9030664003882	-2.0630474398872\\
-5.80121709951033	-2.14553804474854\\
-5.66357937875886	-2.21959401667754\\
-5.48656767745789	-2.29590221378579\\
-5.29378408148959	-2.3718034796466\\
-5.11647011946987	-2.45981471327845\\
-4.9619173204187	-2.56647750664267\\
-4.81010664403037	-2.65367940725578\\
-4.64957850109953	-2.70497636251568\\
-4.50108955270142	-2.71076729520494\\
-4.35126413381684	-2.67652183168921\\
-4.24634669996583	-2.59393558485593\\
-4.18697170978263	-2.46447492464769\\
-4.12611698401056	-2.31204248128708\\
-4.07917499132757	-2.13855158148124\\
-4.06210697366248	-1.97651979982104\\
-4.06425073872858	-1.8124705250783\\
-4.05025312200576	-1.64609042332506\\
-4.05602173816493	-1.5101099943403\\
-4.0875672072386	-1.41310413929987\\
-4.12023281446636	-1.35232170078708\\
-4.14254658501064	-1.325555116572\\
-4.16541908916251	-1.31022576916626\\
-4.18823245295401	-1.30183818345223\\
-4.20258178185564	-1.29503777433062\\
-4.21689045496659	-1.29544097869057\\
-4.22529052601491	-1.29155111806097\\
-4.22960554234588	-1.29011385344295\\
-4.23460924482084	-1.29053082949443\\
-4.23675103902286	-1.29102480523237\\
-4.23749217331874	-1.2913610133983\\
-4.23745221431713	-1.29131295776719\\
-4.23747074084604	-1.29133300516159\\
-4.23749440030917	-1.29134835465143\\
-4.23754134749636	-1.29136883697464\\
-4.23757713766401	-1.29137003902018\\
-4.23758719630051	-1.29141093719747\\
-4.23759860210875	-1.29143347898923\\
-4.23763855004351	-1.29137688442197\\
-4.23767083383348	-1.29130868337134\\
-4.23768870981103	-1.29128518433896\\
-4.23770643246249	-1.29126843440242\\
-4.23773144946509	-1.29127357166419\\
-4.23772585107889	-1.29125716207514\\
-4.23775290243614	-1.2912407879289\\
-4.23776096187649	-1.29122831637596\\
-4.23776787728714	-1.29122792292115\\
-4.2377911449889	-1.29120738436149\\
-4.23783633622267	-1.29116558756264\\
-4.23730554691283	-1.29136017746541\\
-4.23732096802127	-1.29233991534059\\
-4.23714444289051	-1.29416032581307\\
-4.23497824715549	-1.29686221561868\\
-4.23556885065326	-1.30642979310013\\
-4.2322871618865	-1.33988800686836\\
-4.21378175145022	-1.40209602250214\\
-4.17557595719274	-1.50564859149553\\
-4.1156303423858	-1.64596648213391\\
-4.0372952489067	-1.8033973614328\\
-3.95932148536353	-1.97168066839447\\
-3.87081249948878	-2.13942807231163\\
-3.76500269849665	-2.28841094575783\\
-3.63332235815298	-2.40200076798257\\
-3.5150044642429	-2.4561144302918\\
-3.38372482966237	-2.47871103445687\\
-3.23032546029431	-2.48507824381979\\
-3.07071146849338	-2.48177559631994\\
-2.90027539912545	-2.48009385361231\\
-2.71512053417608	-2.45913988111035\\
-2.55785024417246	-2.4079432932105\\
-2.43018476914772	-2.31092042326203\\
-2.31911454063085	-2.17202797140751\\
-2.2511959241578	-2.01383105232989\\
-2.21957555451067	-1.84672546684515\\
-2.23293218675722	-1.68365161967582\\
-2.28841211154752	-1.53707262686684\\
-2.35925384248588	-1.39482209612574\\
-2.43876783039225	-1.23072882169923\\
-2.51542710412797	-1.08646637934597\\
-2.57729810731115	-0.981305099161201\\
-2.61881036565883	-0.912813095914652\\
-2.65066551499392	-0.872429941913855\\
-2.66879077419384	-0.851013640797896\\
-2.67937073071691	-0.839807472582718\\
-2.68577562037362	-0.838792345059261\\
-2.69000243792646	-0.840803850832825\\
-2.69281545988352	-0.84077402585156\\
-2.69644338086508	-0.83454821884108\\
-2.69648278239049	-0.834571338945329\\
-2.69645246870079	-0.83464192626108\\
-2.69644950758666	-0.834671215651036\\
-2.69650135390952	-0.834657028782115\\
-2.69654477990436	-0.834667559380593\\
-2.69647373853117	-0.834671036163195\\
-2.69645972737874	-0.834650319436218\\
-2.69647937709882	-0.83462362620566\\
-2.69648911188012	-0.834640375449039\\
-2.69649423772512	-0.834624179686641\\
-2.69648158120681	-0.834639791130087\\
-2.6964711766951	-0.834654472688735\\
-2.6964508726144	-0.834692449353265\\
-2.69643124467674	-0.834685390529246\\
-2.69644909019924	-0.83466652782733\\
-2.6964554257336	-0.834646964399706\\
-2.69645353197516	-0.834690829304898\\
-2.69645676638785	-0.834699143741971\\
-2.69646180275396	-0.834690493895248\\
-2.69645020389916	-0.834717640511952\\
-2.69748847128004	-0.835304510550009\\
-2.69487093872717	-0.84129204586859\\
-2.68718216770633	-0.866902727237474\\
-2.65264658139344	-0.919035996867654\\
-2.58060982966379	-1.00558809348595\\
-2.45977225926068	-1.10386001224438\\
-2.3081539691754	-1.20544039749215\\
-2.14716822758353	-1.29291264132804\\
-2.03312639520059	-1.33529011754733\\
-1.95320567085319	-1.31540754981386\\
-1.92589781111113	-1.20859921136802\\
-1.91855555880625	-1.06456099659243\\
-1.91620468393526	-0.932695293347906\\
-1.93281663877021	-0.817923355821289\\
-1.95441725851569	-0.723302091729654\\
-1.9738613826807	-0.665313480394463\\
-1.98356798402501	-0.63414156151617\\
-1.99479712761808	-0.625542984145897\\
-1.99606211828207	-0.62429730683744\\
-1.99875092708224	-0.624645172212548\\
-2.00108852407122	-0.625035005536721\\
-2.00284048228691	-0.625942149337818\\
-2.00347114941423	-0.626153652204214\\
-2.00343683679132	-0.626115566622965\\
-2.00342112640721	-0.626134910015501\\
-2.00342348984384	-0.626141049469158\\
-2.00342052529078	-0.626105484559646\\
-2.00341876751633	-0.626100983226375\\
-2.00342658260528	-0.626123392480462\\
-2.00340303765068	-0.626152449935006\\
-2.00340447149805	-0.626160477727713\\
-2.00343203607739	-0.626159962665572\\
-2.0034400612998	-0.62616785616651\\
-2.00344637834524	-0.626164544794016\\
-2.00345728397402	-0.626108388146441\\
-2.00348952376852	-0.626057379779534\\
-2.00348582831559	-0.62606217135272\\
-2.00347830783339	-0.62609014738404\\
-2.00350626545094	-0.62600927751049\\
-2.00349673199743	-0.626006848833512\\
-2.00298517618717	-0.625867323630631\\
-1.9994114313859	-0.624869443558243\\
-1.99637363841625	-0.624811938688192\\
-1.99143414303516	-0.631658085609689\\
-1.97642325135814	-0.667086624888073\\
-1.92744766919235	-0.743812428372444\\
-1.84761962315224	-0.831654126303778\\
-1.75541349776444	-0.932955055570445\\
-1.65131837441743	-1.02145576220355\\
-1.54018290968773	-1.07180187665854\\
-1.40592506466883	-1.07776431589722\\
-1.25092416070457	-1.02307161751144\\
-1.12232475495911	-0.934870266083981\\
-1.03472589727795	-0.826600326421753\\
-0.999051890447899	-0.692696040143526\\
-0.990798981003678	-0.552162335050933\\
-0.994900758067297	-0.442912204845109\\
-1.00861731179083	-0.378815465579536\\
-1.02355245269495	-0.341450783005488\\
-1.02876041180073	-0.328068333247302\\
-1.03401347399712	-0.325409480982385\\
-1.03744522536351	-0.326879443152657\\
-1.03993678158122	-0.32725881486967\\
-1.04578501125973	-0.328354022106259\\
-1.04836129288388	-0.328451000744307\\
-1.0497265149469	-0.328280698376246\\
-1.04982973148817	-0.328352457846127\\
-1.04985188904814	-0.32837469650473\\
-1.04976052155794	-0.328256325760273\\
-1.04977602547726	-0.328215246403807\\
-1.04977342943775	-0.328263490464145\\
-1.04978227478789	-0.32828966908424\\
-1.0498128915123	-0.328286901156235\\
-1.04983857522712	-0.328289179409331\\
-1.0498360098994	-0.328304783950284\\
-1.04987634445943	-0.328227142148843\\
-1.04989156953853	-0.3282218892314\\
-1.04991106616595	-0.328202637943217\\
-1.04994230671065	-0.328203485465103\\
-1.04995631608588	-0.328241699107832\\
-1.04998494243086	-0.328206610196936\\
-1.0499957494482	-0.328193195841254\\
-1.0500036841795	-0.328220360690616\\
-1.04977188463613	-0.328108347337534\\
-1.04515866965663	-0.326137764901436\\
-1.03609814400194	-0.322162081333196\\
-1.02228573975311	-0.318899340644958\\
-1.00269566539857	-0.320188359106488\\
-0.970458686134337	-0.324880791651904\\
-0.919120646163011	-0.347439614866275\\
-0.82046347860032	-0.385671604442249\\
-0.669245998363143	-0.444074770560005\\
-0.503885757358252	-0.538085057359458\\
-0.322998526049949	-0.663876651467739\\
-0.144022171921411	-0.799043212907574\\
0.0174613525526044	-0.904614470068198\\
0.160181637441939	-0.963797514781306\\
0.299469694482059	-0.98606172507326\\
0.443905079188047	-0.97726732040369\\
0.564358182949746	-0.96652460164295\\
0.65906134414695	-0.958306466512831\\
0.73013268085777	-0.945441279367418\\
0.777517552803266	-0.9352098055194\\
0.817789280158328	-0.931649167956456\\
0.847038366767474	-0.924138631585975\\
0.865892611110267	-0.920051080375841\\
0.888899106940058	-0.918471532331943\\
0.903647522954601	-0.915608383113604\\
0.913422831486732	-0.913838686946298\\
0.91985309506155	-0.914650454912826\\
0.923937453946629	-0.916057165036185\\
0.924651762496287	-0.91389527443188\\
0.92464555554388	-0.913777952427624\\
0.924647317130586	-0.913787366544221\\
0.924661963977536	-0.913774012451257\\
0.924620151918117	-0.913798616356277\\
0.924576436437011	-0.913753668515241\\
0.924542323873186	-0.913766474897774\\
0.924526523739895	-0.913769659663117\\
0.9245641053479	-0.913758308315242\\
0.924559105288437	-0.913784830547101\\
0.924568395681075	-0.913801510955865\\
0.924575048220684	-0.913804175697096\\
0.92456133209917	-0.913827836672834\\
0.924535052380107	-0.913830579485226\\
0.9245476609144	-0.913861020684108\\
0.924524628260771	-0.913855593980693\\
0.924534884865473	-0.913835777524694\\
0.924518332604285	-0.91384749805786\\
0.924512385864578	-0.913872218231494\\
0.924580176070513	-0.913868544053147\\
0.924582252514485	-0.913901030500588\\
0.922063376084449	-0.91372285553299\\
0.912720924792729	-0.911600478996444\\
0.896572327854075	-0.907728706389161\\
0.873661034406918	-0.903060635180314\\
0.833848591578056	-0.896282103594966\\
0.792746825933251	-0.890479256673222\\
0.715345844837932	-0.877227068700087\\
0.601180401070303	-0.863976697522689\\
0.491684033077892	-0.854693230301536\\
0.390156671570942	-0.833924504192976\\
0.300885578799509	-0.805348412454256\\
0.205674553691179	-0.789099098911981\\
0.0954768829327027	-0.784442235454819\\
};
\addplot [color=black,solid,line width=1.2pt,forget plot]
  table[row sep=crcr]{%
-6.45410416957491	-1.9684783210128\\
-5.67142140838488	-1.72976103614634\\
-4.88873864719485	-1.49104375127989\\
-4.10605588600482	-1.25232646641343\\
-3.32337312481479	-1.01360918154697\\
-2.54069036362477	-0.774891896680518\\
-1.75800760243474	-0.536174611814062\\
-0.975324841244711	-0.297457326947605\\
-0.192642080054682	-0.0587400420811486\\
0.590040681135345	0.179977242785308\\
};
\node[align=center, rotate=-163.038358227543, text=black]
at (axis cs:-3.078,-0.416) {7.36 m};
\addplot [color=black,solid,line width=1.2pt,forget plot]
  table[row sep=crcr]{%
0.590040681135345	0.179977242785308\\
0.861160180845961	-0.707696886989836\\
1.13227968055658	-1.59537101676498\\
1.40339918026719	-2.48304514654012\\
1.67451867997781	-3.37071927631527\\
1.94563817968843	-4.25839340609041\\
2.21675767939904	-5.14606753586555\\
2.48787717910966	-6.0337416656407\\
2.75899667882027	-6.92141579541584\\
3.03011617853089	-7.80908992519099\\
};
\node[align=center, rotate=106.984077665488, text=black]
at (axis cs:2.288,-3.669) {8.35 m};
\addplot [color=black,solid,line width=1.2pt,forget plot]
  table[row sep=crcr]{%
-4.07415152196298	-9.95069585050696\\
-3.28478844413032	-9.71273963658296\\
-2.49542536629767	-9.47478342265896\\
-1.70606228846502	-9.23682720873497\\
-0.916699210632369	-8.99887099481097\\
-0.127336132799717	-8.76091478088697\\
0.662026945032935	-8.52295856696298\\
1.45139002286559	-8.28500235303898\\
2.24075310069824	-8.04704613911498\\
3.03011617853089	-7.80908992519099\\
};
\node[align=center, rotate=16.775613532021, text=black]
at (axis cs:-0.378,-9.359) {7.42 m};
\addplot [color=black,solid,line width=1.2pt,forget plot]
  table[row sep=crcr]{%
-6.45410416957491	-1.9684783210128\\
-6.18966498650691	-2.85539137984549\\
-5.92522580343892	-3.74230443867817\\
-5.66078662037093	-4.62921749751085\\
-5.39634743730294	-5.51613055634354\\
-5.13190825423494	-6.40304361517622\\
-4.86746907116695	-7.2899566740089\\
-4.60302988809896	-8.17686973284159\\
-4.33859070503097	-9.06378279167427\\
-4.07415152196298	-9.95069585050696\\
};
\node[align=center, rotate=-73.3976911404205, text=black]
at (axis cs:-5.743,-6.102) {8.33 m};
\addplot [color=red,solid,line width=1.2pt,forget plot]
  table[row sep=crcr]{%
0.0952170677457612	-0.0245498227730007\\
-0.0988205565021127	0.0239207907597458\\
};
\addplot [color=red,solid,line width=1.2pt,forget plot]
  table[row sep=crcr]{%
0.0951837462795098	-0.017863024074299\\
-0.101840175522474	0.0165110102796393\\
};
\addplot [color=red,solid,line width=1.2pt,forget plot]
  table[row sep=crcr]{%
0.0947495124931381	-0.0110164539239683\\
-0.104245968726417	0.00900349747951324\\
};
\addplot [color=red,solid,line width=1.2pt,forget plot]
  table[row sep=crcr]{%
0.0935825825544667	-0.00414507632284333\\
-0.106340476662218	0.00140203044034137\\
};
\addplot [color=red,solid,line width=1.2pt,forget plot]
  table[row sep=crcr]{%
0.0915067822006203	0.00277090140704582\\
-0.108278711019203	-0.00648958988186459\\
};
\addplot [color=red,solid,line width=1.2pt,forget plot]
  table[row sep=crcr]{%
0.0890676082584025	0.00997241619925601\\
-0.109431371148508	-0.0144847948930754\\
};
\addplot [color=red,solid,line width=1.2pt,forget plot]
  table[row sep=crcr]{%
0.0870309543191489	0.0128852373119412\\
-0.110517735276965	-0.0183319893242967\\
};
\addplot [color=red,solid,line width=1.2pt,forget plot]
  table[row sep=crcr]{%
0.0855284502907514	0.0125040488021486\\
-0.111995087842823	-0.018871927408867\\
};
\addplot [color=red,solid,line width=1.2pt,forget plot]
  table[row sep=crcr]{%
0.0838581948948792	0.0121701133142007\\
-0.113613056334219	-0.0195332766940375\\
};
\addplot [color=red,solid,line width=1.2pt,forget plot]
  table[row sep=crcr]{%
0.0818411996110765	0.0119118786835611\\
-0.115527503966005	-0.0204237803989801\\
};
\addplot [color=red,solid,line width=1.2pt,forget plot]
  table[row sep=crcr]{%
0.0799529049484381	0.011756053576881\\
-0.117268964775065	-0.0214634302054639\\
};
\addplot [color=red,solid,line width=1.2pt,forget plot]
  table[row sep=crcr]{%
0.078311773835443	0.0117657120045515\\
-0.118733610241392	-0.0224850779719235\\
};
\addplot [color=red,solid,line width=1.2pt,forget plot]
  table[row sep=crcr]{%
0.0749452269620017	0.0118918081578848\\
-0.121633298156557	-0.0249440909381339\\
};
\addplot [color=red,solid,line width=1.2pt,forget plot]
  table[row sep=crcr]{%
0.0728848198494895	0.0119972921535669\\
-0.123408859129462	-0.026327524778914\\
};
\addplot [color=red,solid,line width=1.2pt,forget plot]
  table[row sep=crcr]{%
0.0709855356990419	0.0121729241017189\\
-0.124975701270924	-0.0278169946929095\\
};
\addplot [color=red,solid,line width=1.2pt,forget plot]
  table[row sep=crcr]{%
0.0693075565886936	0.0124443788852447\\
-0.126283256474837	-0.0293196468444087\\
};
\addplot [color=red,solid,line width=1.2pt,forget plot]
  table[row sep=crcr]{%
0.0677039770472827	0.0127409506490712\\
-0.127490122359126	-0.0308395903861318\\
};
\addplot [color=red,solid,line width=1.2pt,forget plot]
  table[row sep=crcr]{%
0.0659334083648754	0.0130776406243398\\
-0.128813763686944	-0.0324584876456304\\
};
\addplot [color=red,solid,line width=1.2pt,forget plot]
  table[row sep=crcr]{%
0.0638086360674902	0.0133559235933897\\
-0.130431177890983	-0.0342979815286232\\
};
\addplot [color=red,solid,line width=1.2pt,forget plot]
  table[row sep=crcr]{%
0.0618828763499674	0.0137967098989108\\
-0.131793335634658	-0.0360984292559854\\
};
\addplot [color=red,solid,line width=1.2pt,forget plot]
  table[row sep=crcr]{%
0.0601702465529565	0.0143368713546336\\
-0.132887808248115	-0.0378990116801772\\
};
\addplot [color=red,solid,line width=1.2pt,forget plot]
  table[row sep=crcr]{%
0.0584669002920592	0.0148957863260455\\
-0.133913839399825	-0.0397819147558729\\
};
\addplot [color=red,solid,line width=1.2pt,forget plot]
  table[row sep=crcr]{%
0.0565892934132554	0.015409076524884\\
-0.135060621124506	-0.0417775500382267\\
};
\addplot [color=red,solid,line width=1.2pt,forget plot]
  table[row sep=crcr]{%
0.0543947332527942	0.0158377420480269\\
-0.136452784422237	-0.0439721651353032\\
};
\addplot [color=red,solid,line width=1.2pt,forget plot]
  table[row sep=crcr]{%
1.55539791375839	-1.2264302436179\\
1.43018256323209	-1.07047795606631\\
};
\addplot [color=red,solid,line width=1.2pt,forget plot]
  table[row sep=crcr]{%
1.55029593880677	-1.22957136541508\\
1.43352221067223	-1.0672017364275\\
};
\addplot [color=red,solid,line width=1.2pt,forget plot]
  table[row sep=crcr]{%
1.54421798204127	-1.23275629885539\\
1.43666676586501	-1.06413620545307\\
};
\addplot [color=red,solid,line width=1.2pt,forget plot]
  table[row sep=crcr]{%
1.52945339392148	-1.2377692254824\\
1.44159579890479	-1.05809996545753\\
};
\addplot [color=red,solid,line width=1.2pt,forget plot]
  table[row sep=crcr]{%
1.52448586825519	-1.2397277585275\\
1.44500285751811	-1.05619996498067\\
};
\addplot [color=red,solid,line width=1.2pt,forget plot]
  table[row sep=crcr]{%
1.52243385636497	-1.2405097223287\\
1.44211662381869	-1.05734547320757\\
};
\addplot [color=red,solid,line width=1.2pt,forget plot]
  table[row sep=crcr]{%
1.52126165751326	-1.24079596303468\\
1.44032677216565	-1.05790379717247\\
};
\addplot [color=red,solid,line width=1.2pt,forget plot]
  table[row sep=crcr]{%
1.520467063843	-1.24088545349245\\
1.43900896577018	-1.05822571994846\\
};
\addplot [color=red,solid,line width=1.2pt,forget plot]
  table[row sep=crcr]{%
1.5198288805741	-1.24090289286485\\
1.43793667647718	-1.0584373703458\\
};
\addplot [color=red,solid,line width=1.2pt,forget plot]
  table[row sep=crcr]{%
1.51923490791652	-1.24089920890821\\
1.43699519866177	-1.05859004780627\\
};
\addplot [color=red,solid,line width=1.2pt,forget plot]
  table[row sep=crcr]{%
1.5185887517988	-1.24082009566151\\
1.43610311167503	-1.05862207379142\\
};
\addplot [color=red,solid,line width=1.2pt,forget plot]
  table[row sep=crcr]{%
1.51809104261453	-1.24071432304604\\
1.43546132462754	-1.05858159752592\\
};
\addplot [color=red,solid,line width=1.2pt,forget plot]
  table[row sep=crcr]{%
1.51775234820026	-1.24061897035912\\
1.43505091579383	-1.05851879716336\\
};
\addplot [color=red,solid,line width=1.2pt,forget plot]
  table[row sep=crcr]{%
1.5173969023172	-1.24053828375016\\
1.43468228933685	-1.0584440971337\\
};
\addplot [color=red,solid,line width=1.2pt,forget plot]
  table[row sep=crcr]{%
1.5170293272186	-1.24041333700523\\
1.43434261692098	-1.05830647842995\\
};
\addplot [color=red,solid,line width=1.2pt,forget plot]
  table[row sep=crcr]{%
1.51663153438753	-1.24027951274931\\
1.43403070234904	-1.05813368505472\\
};
\addplot [color=red,solid,line width=1.2pt,forget plot]
  table[row sep=crcr]{%
1.5162617459799	-1.24013220123077\\
1.43380147680606	-1.05792269542008\\
};
\addplot [color=red,solid,line width=1.2pt,forget plot]
  table[row sep=crcr]{%
1.51599689315747	-1.24000525487847\\
1.43368154411948	-1.05773023379013\\
};
\addplot [color=red,solid,line width=1.2pt,forget plot]
  table[row sep=crcr]{%
1.51575384779346	-1.23989354648885\\
1.43363216832649	-1.05753118808799\\
};
\addplot [color=red,solid,line width=1.2pt,forget plot]
  table[row sep=crcr]{%
1.51547244146775	-1.23974369765838\\
1.43360068893845	-1.05726899766471\\
};
\addplot [color=red,solid,line width=1.2pt,forget plot]
  table[row sep=crcr]{%
1.5151559138687	-1.23954020037042\\
1.43359774666335	-1.05692512606902\\
};
\addplot [color=red,solid,line width=1.2pt,forget plot]
  table[row sep=crcr]{%
1.24370953704617	-1.12264980469935\\
1.16989093586267	-0.936771311808664\\
};
\addplot [color=red,solid,line width=1.2pt,forget plot]
  table[row sep=crcr]{%
1.24331052756146	-1.12276222285187\\
1.17028793998389	-0.936569576102661\\
};
\addplot [color=red,solid,line width=1.2pt,forget plot]
  table[row sep=crcr]{%
1.24297429009166	-1.12283632973083\\
1.17071606435898	-0.936345716524785\\
};
\addplot [color=red,solid,line width=1.2pt,forget plot]
  table[row sep=crcr]{%
1.24260240687572	-1.12288211481669\\
1.17121445915898	-0.936056632291437\\
};
\addplot [color=red,solid,line width=1.2pt,forget plot]
  table[row sep=crcr]{%
1.24215873964723	-1.12296090473957\\
1.171703282199	-0.935781769909425\\
};
\addplot [color=red,solid,line width=1.2pt,forget plot]
  table[row sep=crcr]{%
1.24169627292307	-1.1230409615156\\
1.17224603139655	-0.93548653221064\\
};
\addplot [color=red,solid,line width=1.2pt,forget plot]
  table[row sep=crcr]{%
1.23939050158528	-1.12472225169043\\
1.17072119190205	-0.936880494065489\\
};
\addplot [color=red,solid,line width=1.2pt,forget plot]
  table[row sep=crcr]{%
1.23494353296617	-1.13328170800936\\
1.16612125670124	-0.94549594100421\\
};
\addplot [color=red,solid,line width=1.2pt,forget plot]
  table[row sep=crcr]{%
1.23167636808053	-1.14382265135715\\
1.16270391443629	-0.956091991532249\\
};
\addplot [color=red,solid,line width=1.2pt,forget plot]
  table[row sep=crcr]{%
1.22960183576841	-1.15099717398665\\
1.16053897291746	-0.963299755318115\\
};
\addplot [color=red,solid,line width=1.2pt,forget plot]
  table[row sep=crcr]{%
1.22808756724815	-1.1560196052699\\
1.15897707455783	-0.968339718760708\\
};
\addplot [color=red,solid,line width=1.2pt,forget plot]
  table[row sep=crcr]{%
1.22681888355469	-1.16020425293367\\
1.15770361300438	-0.972526125873647\\
};
\addplot [color=red,solid,line width=1.2pt,forget plot]
  table[row sep=crcr]{%
1.22569252515182	-1.16403097013727\\
1.15662005578843	-0.976337086443517\\
};
\addplot [color=red,solid,line width=1.2pt,forget plot]
  table[row sep=crcr]{%
1.22468130687955	-1.16756532402394\\
1.15570227285308	-0.979837081969697\\
};
\addplot [color=red,solid,line width=1.2pt,forget plot]
  table[row sep=crcr]{%
1.22399547651199	-1.16990696561007\\
1.15514842212982	-0.982130281479957\\
};
\addplot [color=red,solid,line width=1.2pt,forget plot]
  table[row sep=crcr]{%
1.22350803561157	-1.17144502840765\\
1.15483949368994	-0.983602990113888\\
};
\addplot [color=red,solid,line width=1.2pt,forget plot]
  table[row sep=crcr]{%
1.22308095832911	-1.17261687328819\\
1.15463323906205	-0.984694256961315\\
};
\addplot [color=red,solid,line width=1.2pt,forget plot]
  table[row sep=crcr]{%
1.2226872641149	-1.17351626625385\\
1.15445883499612	-0.98551392196\\
};
\addplot [color=red,solid,line width=1.2pt,forget plot]
  table[row sep=crcr]{%
1.22243184661574	-1.17401540691768\\
1.15440421712086	-0.985940311210384\\
};
\addplot [color=red,solid,line width=1.2pt,forget plot]
  table[row sep=crcr]{%
1.34000421579222	-2.42022292265607\\
1.27873662928141	-2.22983835699189\\
};
\addplot [color=red,solid,line width=1.2pt,forget plot]
  table[row sep=crcr]{%
1.339709825537	-2.4203273234825\\
1.27902525167625	-2.22975612288981\\
};
\addplot [color=red,solid,line width=1.2pt,forget plot]
  table[row sep=crcr]{%
1.33942495280446	-2.42042444347425\\
1.27929567581753	-2.22967730706317\\
};
\addplot [color=red,solid,line width=1.2pt,forget plot]
  table[row sep=crcr]{%
1.33911501843824	-2.42048460421635\\
1.27950412849938	-2.22957483023739\\
};
\addplot [color=red,solid,line width=1.2pt,forget plot]
  table[row sep=crcr]{%
1.33881587530895	-2.42087197957805\\
1.27936473591101	-2.22991239740348\\
};
\addplot [color=red,solid,line width=1.2pt,forget plot]
  table[row sep=crcr]{%
1.33861755622198	-2.42094130732751\\
1.27904289557831	-2.23002022465409\\
};
\addplot [color=red,solid,line width=1.2pt,forget plot]
  table[row sep=crcr]{%
1.33847467418711	-2.42098356956017\\
1.27878152695984	-2.230099499575\\
};
\addplot [color=red,solid,line width=1.2pt,forget plot]
  table[row sep=crcr]{%
1.33829466088518	-2.42101831626952\\
1.27850614839011	-2.23016409500725\\
};
\addplot [color=red,solid,line width=1.2pt,forget plot]
  table[row sep=crcr]{%
1.33810243267059	-2.42103422703416\\
1.27822129146082	-2.23020904806814\\
};
\addplot [color=red,solid,line width=1.2pt,forget plot]
  table[row sep=crcr]{%
1.33796209443256	-2.42107082164089\\
1.27802671855607	-2.23026267003686\\
};
\addplot [color=red,solid,line width=1.2pt,forget plot]
  table[row sep=crcr]{%
1.33791870041807	-2.42110837562028\\
1.27794257215161	-2.23031302966508\\
};
\addplot [color=red,solid,line width=1.2pt,forget plot]
  table[row sep=crcr]{%
1.33786815613442	-2.42114835257861\\
1.27784393681125	-2.23036813060934\\
};
\addplot [color=red,solid,line width=1.2pt,forget plot]
  table[row sep=crcr]{%
1.33782351190219	-2.42116916761599\\
1.27775900589074	-2.23040162551933\\
};
\addplot [color=red,solid,line width=1.2pt,forget plot]
  table[row sep=crcr]{%
1.33777134872675	-2.42114951748159\\
1.2776686120239	-2.23039401679857\\
};
\addplot [color=red,solid,line width=1.2pt,forget plot]
  table[row sep=crcr]{%
1.33776484507095	-2.42110849931061\\
1.27764597822786	-2.23035808161905\\
};
\addplot [color=red,solid,line width=1.2pt,forget plot]
  table[row sep=crcr]{%
1.33768557264162	-2.42114124772087\\
1.27756322089701	-2.23039192840178\\
};
\addplot [color=red,solid,line width=1.2pt,forget plot]
  table[row sep=crcr]{%
1.33801888874061	-2.42113728819154\\
1.27788663699387	-2.23039108954038\\
};
\addplot [color=red,solid,line width=1.2pt,forget plot]
  table[row sep=crcr]{%
1.70464934250394	-3.39818477967774\\
1.64505843845141	-3.20726876632579\\
};
\addplot [color=red,solid,line width=1.2pt,forget plot]
  table[row sep=crcr]{%
1.70440260827504	-3.39824106199587\\
1.64522823053114	-3.20719553595922\\
};
\addplot [color=red,solid,line width=1.2pt,forget plot]
  table[row sep=crcr]{%
1.70436452294007	-3.39820166387443\\
1.6454115470382	-3.20708770145327\\
};
\addplot [color=red,solid,line width=1.2pt,forget plot]
  table[row sep=crcr]{%
1.70425196789512	-3.39822561503482\\
1.64557109068136	-3.20702793029616\\
};
\addplot [color=red,solid,line width=1.2pt,forget plot]
  table[row sep=crcr]{%
1.70417154897039	-3.39822820588523\\
1.64578917592094	-3.20693916146001\\
};
\addplot [color=red,solid,line width=1.2pt,forget plot]
  table[row sep=crcr]{%
1.70426447914483	-3.39824219322257\\
1.64592791368697	-3.20693917408648\\
};
\addplot [color=red,solid,line width=1.2pt,forget plot]
  table[row sep=crcr]{%
1.70444817532621	-3.39826465230884\\
1.64589414309258	-3.20702808334469\\
};
\addplot [color=red,solid,line width=1.2pt,forget plot]
  table[row sep=crcr]{%
1.70452802111826	-3.39825768314709\\
1.64587568103894	-3.20705124234758\\
};
\addplot [color=red,solid,line width=1.2pt,forget plot]
  table[row sep=crcr]{%
1.70455399052479	-3.3982539442574\\
1.64589601438331	-3.20704923240389\\
};
\addplot [color=red,solid,line width=1.2pt,forget plot]
  table[row sep=crcr]{%
1.7046182483248	-3.3982195411801\\
1.6457809415397	-3.20706993660223\\
};
\addplot [color=red,solid,line width=1.2pt,forget plot]
  table[row sep=crcr]{%
1.7046058355793	-3.39820950893468\\
1.64582849874768	-3.20704145544513\\
};
\addplot [color=red,solid,line width=1.2pt,forget plot]
  table[row sep=crcr]{%
1.70460446868262	-3.3981628847307\\
1.64579856637388	-3.20700361643943\\
};
\addplot [color=red,solid,line width=1.2pt,forget plot]
  table[row sep=crcr]{%
1.70458342962851	-3.39818551300364\\
1.64573379269684	-3.20703970417433\\
};
\addplot [color=red,solid,line width=1.2pt,forget plot]
  table[row sep=crcr]{%
1.7046125563177	-3.39820373603247\\
1.64566947248897	-3.20708672248119\\
};
\addplot [color=red,solid,line width=1.2pt,forget plot]
  table[row sep=crcr]{%
1.70461122813115	-3.39820958351573\\
1.64563843150593	-3.20710173632415\\
};
\addplot [color=red,solid,line width=1.2pt,forget plot]
  table[row sep=crcr]{%
1.7045929420888	-3.39821273215067\\
1.64568188050307	-3.20708584544228\\
};
\addplot [color=red,solid,line width=1.2pt,forget plot]
  table[row sep=crcr]{%
1.70461946172614	-3.39816687945335\\
1.64560916675353	-3.20707060767589\\
};
\addplot [color=red,solid,line width=1.2pt,forget plot]
  table[row sep=crcr]{%
1.70451728238747	-3.3982312501354\\
1.64560605235449	-3.2071044153476\\
};
\addplot [color=red,solid,line width=1.2pt,forget plot]
  table[row sep=crcr]{%
1.70469983893731	-3.39822339757312\\
1.64545828043349	-3.20719869302934\\
};
\addplot [color=red,solid,line width=1.2pt,forget plot]
  table[row sep=crcr]{%
1.70431285089032	-3.39848512085518\\
1.64522595066957	-3.20741252150535\\
};
\addplot [color=red,solid,line width=1.2pt,forget plot]
  table[row sep=crcr]{%
1.97071874907439	-4.3826503559198\\
1.91207871028325	-4.19144014214822\\
};
\addplot [color=red,solid,line width=1.2pt,forget plot]
  table[row sep=crcr]{%
1.97069014257531	-4.38260078792693\\
1.91175995478268	-4.19147979753937\\
};
\addplot [color=red,solid,line width=1.2pt,forget plot]
  table[row sep=crcr]{%
1.97054270534606	-4.38267796688451\\
1.91173044640197	-4.19152065418659\\
};
\addplot [color=red,solid,line width=1.2pt,forget plot]
  table[row sep=crcr]{%
1.9704206109066	-4.38272225183927\\
1.91170691042327	-4.19153464404118\\
};
\addplot [color=red,solid,line width=1.2pt,forget plot]
  table[row sep=crcr]{%
1.97031374582282	-4.38267257840983\\
1.91168717882482	-4.19145823364974\\
};
\addplot [color=red,solid,line width=1.2pt,forget plot]
  table[row sep=crcr]{%
1.97020716393345	-4.38266396745398\\
1.91168854502047	-4.19141655899959\\
};
\addplot [color=red,solid,line width=1.2pt,forget plot]
  table[row sep=crcr]{%
1.97020380850822	-4.38267075791773\\
1.91167443932444	-4.19142663920306\\
};
\addplot [color=red,solid,line width=1.2pt,forget plot]
  table[row sep=crcr]{%
1.97023692035325	-4.38263137416111\\
1.91166762864709	-4.19139947810546\\
};
\addplot [color=red,solid,line width=1.2pt,forget plot]
  table[row sep=crcr]{%
1.97027233378508	-4.38262304098797\\
1.91168723594092	-4.19139598665049\\
};
\addplot [color=red,solid,line width=1.2pt,forget plot]
  table[row sep=crcr]{%
1.9703063379294	-4.38262406381194\\
1.91171412429204	-4.19139918964261\\
};
\addplot [color=red,solid,line width=1.2pt,forget plot]
  table[row sep=crcr]{%
1.97030098713926	-4.38266699674487\\
1.91170587052548	-4.19144301208548\\
};
\addplot [color=red,solid,line width=1.2pt,forget plot]
  table[row sep=crcr]{%
1.97033625886675	-4.3826760939923\\
1.91173740381854	-4.1914532549091\\
};
\addplot [color=red,solid,line width=1.2pt,forget plot]
  table[row sep=crcr]{%
1.9703223225546	-4.38267905630367\\
1.91171531169969	-4.19145871669867\\
};
\addplot [color=red,solid,line width=1.2pt,forget plot]
  table[row sep=crcr]{%
1.97029566544918	-4.38270178093388\\
1.91169886817738	-4.1914783112721\\
};
\addplot [color=red,solid,line width=1.2pt,forget plot]
  table[row sep=crcr]{%
1.97027481889933	-4.38271282097751\\
1.9116787338947	-4.19148913305645\\
};
\addplot [color=red,solid,line width=1.2pt,forget plot]
  table[row sep=crcr]{%
1.97030663158822	-4.38268981244964\\
1.91170782550177	-4.19146695836247\\
};
\addplot [color=red,solid,line width=1.2pt,forget plot]
  table[row sep=crcr]{%
1.97032989171358	-4.38269507308449\\
1.91172702673558	-4.19147346286125\\
};
\addplot [color=red,solid,line width=1.2pt,forget plot]
  table[row sep=crcr]{%
1.97034452311732	-4.38270186059049\\
1.91176904149902	-4.19147186045143\\
};
\addplot [color=red,solid,line width=1.2pt,forget plot]
  table[row sep=crcr]{%
1.97037508478092	-4.38266814450896\\
1.91180099685369	-4.19143771747525\\
};
\addplot [color=red,solid,line width=1.2pt,forget plot]
  table[row sep=crcr]{%
1.97040771141714	-4.38269886231972\\
1.91179965415975	-4.19147884342914\\
};
\addplot [color=red,solid,line width=1.2pt,forget plot]
  table[row sep=crcr]{%
2.29983035933858	-5.36034825922035\\
2.24066532099491	-5.16929984065115\\
};
\addplot [color=red,solid,line width=1.2pt,forget plot]
  table[row sep=crcr]{%
2.29980727269032	-5.36036883126648\\
2.24067880903805	-5.16930908986049\\
};
\addplot [color=red,solid,line width=1.2pt,forget plot]
  table[row sep=crcr]{%
2.29966974493255	-5.3603093879163\\
2.24065776320333	-5.16921363701537\\
};
\addplot [color=red,solid,line width=1.2pt,forget plot]
  table[row sep=crcr]{%
2.2995712157586	-5.36031763147943\\
2.24064338674741	-5.16919591380185\\
};
\addplot [color=red,solid,line width=1.2pt,forget plot]
  table[row sep=crcr]{%
2.29950505088845	-5.36035694726133\\
2.24063501047952	-5.1692174214111\\
};
\addplot [color=red,solid,line width=1.2pt,forget plot]
  table[row sep=crcr]{%
2.29948475233231	-5.36036005548446\\
2.24068591492306	-5.16919861397671\\
};
\addplot [color=red,solid,line width=1.2pt,forget plot]
  table[row sep=crcr]{%
2.29948637608383	-5.36031619256878\\
2.24074655726458	-5.16913660766212\\
};
\addplot [color=red,solid,line width=1.2pt,forget plot]
  table[row sep=crcr]{%
2.29942560659477	-5.36028635196858\\
2.24069977719454	-5.16910246938102\\
};
\addplot [color=red,solid,line width=1.2pt,forget plot]
  table[row sep=crcr]{%
2.29947864988066	-5.36027825281163\\
2.24072085746166	-5.16910419120795\\
};
\addplot [color=red,solid,line width=1.2pt,forget plot]
  table[row sep=crcr]{%
2.29949043953987	-5.36026351194614\\
2.24069644943365	-5.16910057953667\\
};
\addplot [color=red,solid,line width=1.2pt,forget plot]
  table[row sep=crcr]{%
2.29951318533769	-5.36025807183072\\
2.24070247190073	-5.16910028364244\\
};
\addplot [color=red,solid,line width=1.2pt,forget plot]
  table[row sep=crcr]{%
2.29946000221188	-5.36028316855813\\
2.24062639020858	-5.16913242673733\\
};
\addplot [color=red,solid,line width=1.2pt,forget plot]
  table[row sep=crcr]{%
2.29942840125071	-5.36033387060157\\
2.24058749394166	-5.16918537432957\\
};
\addplot [color=red,solid,line width=1.2pt,forget plot]
  table[row sep=crcr]{%
2.2994394827078	-5.36034455304164\\
2.24056912945528	-5.16920512354598\\
};
\addplot [color=red,solid,line width=1.2pt,forget plot]
  table[row sep=crcr]{%
2.29941662406706	-5.36035799709187\\
2.24054054170799	-5.16922033223733\\
};
\addplot [color=red,solid,line width=1.2pt,forget plot]
  table[row sep=crcr]{%
2.29943513520488	-5.36037132267797\\
2.2405468437895	-5.16923741900262\\
};
\addplot [color=red,solid,line width=1.2pt,forget plot]
  table[row sep=crcr]{%
2.29941516343133	-5.36041883359465\\
2.24051071450041	-5.16928990879219\\
};
\addplot [color=red,solid,line width=1.2pt,forget plot]
  table[row sep=crcr]{%
2.29944869517642	-5.36038958839353\\
2.24051153487151	-5.16927074804755\\
};
\addplot [color=red,solid,line width=1.2pt,forget plot]
  table[row sep=crcr]{%
2.58475720599756	-6.34689843492386\\
2.52628574377279	-6.15563660364198\\
};
\addplot [color=red,solid,line width=1.2pt,forget plot]
  table[row sep=crcr]{%
2.58474620092278	-6.34635080378329\\
2.52596805485638	-6.15518299910636\\
};
\addplot [color=red,solid,line width=1.2pt,forget plot]
  table[row sep=crcr]{%
2.58483468877709	-6.34651668326403\\
2.52613134313022	-6.15532589580328\\
};
\addplot [color=red,solid,line width=1.2pt,forget plot]
  table[row sep=crcr]{%
2.58481869999523	-6.34652542960056\\
2.52615099073282	-6.15532370395484\\
};
\addplot [color=red,solid,line width=1.2pt,forget plot]
  table[row sep=crcr]{%
2.58483437680994	-6.34653245229854\\
2.52622100370546	-6.1553140627715\\
};
\addplot [color=red,solid,line width=1.2pt,forget plot]
  table[row sep=crcr]{%
2.58481408335406	-6.34652403940832\\
2.52625564359813	-6.155288820023\\
};
\addplot [color=red,solid,line width=1.2pt,forget plot]
  table[row sep=crcr]{%
2.5848255059218	-6.3465204578916\\
2.52631451710387	-6.15527071491779\\
};
\addplot [color=red,solid,line width=1.2pt,forget plot]
  table[row sep=crcr]{%
2.5848355168174	-6.34656489703549\\
2.52637157264579	-6.1553007675394\\
};
\addplot [color=red,solid,line width=1.2pt,forget plot]
  table[row sep=crcr]{%
2.58492720978512	-6.34654056172511\\
2.52645052081558	-6.15528032841155\\
};
\addplot [color=red,solid,line width=1.2pt,forget plot]
  table[row sep=crcr]{%
2.58501293900952	-6.34651939960774\\
2.52649158487363	-6.15527282810879\\
};
\addplot [color=red,solid,line width=1.2pt,forget plot]
  table[row sep=crcr]{%
2.58511367998273	-6.34649327754215\\
2.52657136749479	-6.15525312054429\\
};
\addplot [color=red,solid,line width=1.2pt,forget plot]
  table[row sep=crcr]{%
2.58522651608572	-6.34646206217475\\
2.52665652708543	-6.15523037968331\\
};
\addplot [color=red,solid,line width=1.2pt,forget plot]
  table[row sep=crcr]{%
2.58526365846465	-6.34642060861535\\
2.52668768664357	-6.15519075863053\\
};
\addplot [color=red,solid,line width=1.2pt,forget plot]
  table[row sep=crcr]{%
2.58533697864098	-6.34637343326457\\
2.52672146883616	-6.1551556987044\\
};
\addplot [color=red,solid,line width=1.2pt,forget plot]
  table[row sep=crcr]{%
2.58547314638673	-6.34633136062542\\
2.52684629241004	-6.15511710385361\\
};
\addplot [color=red,solid,line width=1.2pt,forget plot]
  table[row sep=crcr]{%
2.58555747585438	-6.34630520577888\\
2.52691843635915	-6.15509468554771\\
};
\addplot [color=red,solid,line width=1.2pt,forget plot]
  table[row sep=crcr]{%
2.58570494981537	-6.34624468928364\\
2.52704704689762	-6.15503995496649\\
};
\addplot [color=red,solid,line width=1.2pt,forget plot]
  table[row sep=crcr]{%
2.58578633924544	-6.34620366020666\\
2.52713238015158	-6.15499771604536\\
};
\addplot [color=red,solid,line width=1.2pt,forget plot]
  table[row sep=crcr]{%
2.58583036559901	-6.34616544069894\\
2.52717830793502	-6.1549589132691\\
};
\addplot [color=red,solid,line width=1.2pt,forget plot]
  table[row sep=crcr]{%
2.58594015980065	-6.34612629888507\\
2.5273071145866	-6.15491394047452\\
};
\addplot [color=red,solid,line width=1.2pt,forget plot]
  table[row sep=crcr]{%
2.58600910951088	-6.34608020783229\\
2.52731253976795	-6.1548873400249\\
};
\addplot [color=red,solid,line width=1.2pt,forget plot]
  table[row sep=crcr]{%
2.58632242647177	-6.34604079463254\\
2.52764529035522	-6.15484196174604\\
};
\addplot [color=red,solid,line width=1.2pt,forget plot]
  table[row sep=crcr]{%
2.89882776935944	-7.25994699526968\\
2.84037218236478	-7.06868031142456\\
};
\addplot [color=red,solid,line width=1.2pt,forget plot]
  table[row sep=crcr]{%
2.89886758233657	-7.2599335690186\\
2.84042097464656	-7.06866414111771\\
};
\addplot [color=red,solid,line width=1.2pt,forget plot]
  table[row sep=crcr]{%
2.89893387046597	-7.25990866439938\\
2.84049767353617	-7.06863605557004\\
};
\addplot [color=red,solid,line width=1.2pt,forget plot]
  table[row sep=crcr]{%
2.89901248161187	-7.25988409010912\\
2.84057745390771	-7.06861112407049\\
};
\addplot [color=red,solid,line width=1.2pt,forget plot]
  table[row sep=crcr]{%
2.89903822699886	-7.25983508267306\\
2.84062934081968	-7.06855413219627\\
};
\addplot [color=red,solid,line width=1.2pt,forget plot]
  table[row sep=crcr]{%
2.89906281839411	-7.25986363494278\\
2.84064066476239	-7.06858673627221\\
};
\addplot [color=red,solid,line width=1.2pt,forget plot]
  table[row sep=crcr]{%
2.89908857547234	-7.25987751054132\\
2.8406494621078	-7.06860579274392\\
};
\addplot [color=red,solid,line width=1.2pt,forget plot]
  table[row sep=crcr]{%
2.89908066609278	-7.25987896718958\\
2.8406465183841	-7.06860573230932\\
};
\addplot [color=red,solid,line width=1.2pt,forget plot]
  table[row sep=crcr]{%
2.89906248413085	-7.25989920213886\\
2.84059668877037	-7.06863563850776\\
};
\addplot [color=red,solid,line width=1.2pt,forget plot]
  table[row sep=crcr]{%
2.89907354193555	-7.25988235704122\\
2.84057747551931	-7.06862804934116\\
};
\addplot [color=red,solid,line width=1.2pt,forget plot]
  table[row sep=crcr]{%
2.89913897702444	-7.25988949995232\\
2.84062389485071	-7.0686410093486\\
};
\addplot [color=red,solid,line width=1.2pt,forget plot]
  table[row sep=crcr]{%
2.89915431048135	-7.25986835474017\\
2.8406171834216	-7.0686266104593\\
};
\addplot [color=red,solid,line width=1.2pt,forget plot]
  table[row sep=crcr]{%
2.89914810562862	-7.25987739729132\\
2.84061868074977	-7.06863329562188\\
};
\addplot [color=red,solid,line width=1.2pt,forget plot]
  table[row sep=crcr]{%
2.89915635168557	-7.25990175266186\\
2.84063956859419	-7.06865378248535\\
};
\addplot [color=red,solid,line width=1.2pt,forget plot]
  table[row sep=crcr]{%
2.89911121323759	-7.25992511845069\\
2.84061227917064	-7.06867168785916\\
};
\addplot [color=red,solid,line width=1.2pt,forget plot]
  table[row sep=crcr]{%
2.89910712984608	-7.25993435359939\\
2.84056599955965	-7.06869383471077\\
};
\addplot [color=red,solid,line width=1.2pt,forget plot]
  table[row sep=crcr]{%
2.89912906070309	-7.25991571111098\\
2.84060144131974	-7.06867105688682\\
};
\addplot [color=red,solid,line width=1.2pt,forget plot]
  table[row sep=crcr]{%
2.89926996019774	-7.26014587252472\\
2.84077268559138	-7.06889193435971\\
};
\addplot [color=red,solid,line width=1.2pt,forget plot]
  table[row sep=crcr]{%
2.63881701644306	-7.95179862149003\\
2.83032504212476	-7.89413861715862\\
};
\addplot [color=red,solid,line width=1.2pt,forget plot]
  table[row sep=crcr]{%
2.63880101835579	-7.95185202771596\\
2.83020373774004	-7.89384341618013\\
};
\addplot [color=red,solid,line width=1.2pt,forget plot]
  table[row sep=crcr]{%
2.63874547282385	-7.95176153239992\\
2.83017984236646	-7.89385745518748\\
};
\addplot [color=red,solid,line width=1.2pt,forget plot]
  table[row sep=crcr]{%
2.63874189675002	-7.95177474136705\\
2.83020874355061	-7.89397814478696\\
};
\addplot [color=red,solid,line width=1.2pt,forget plot]
  table[row sep=crcr]{%
2.63868168957045	-7.9517402402366\\
2.83016106861443	-7.89398517638129\\
};
\addplot [color=red,solid,line width=1.2pt,forget plot]
  table[row sep=crcr]{%
2.63858098906912	-7.95175913030229\\
2.8300676264902	-7.89402813612093\\
};
\addplot [color=red,solid,line width=1.2pt,forget plot]
  table[row sep=crcr]{%
2.63845667516735	-7.95175497887704\\
2.82995103446665	-7.89404960342075\\
};
\addplot [color=red,solid,line width=1.2pt,forget plot]
  table[row sep=crcr]{%
2.63841166260522	-7.95174317535656\\
2.82991597591482	-7.89407084244188\\
};
\addplot [color=red,solid,line width=1.2pt,forget plot]
  table[row sep=crcr]{%
2.63851001474971	-7.95182060869606\\
2.83002259753792	-7.89417574219618\\
};
\addplot [color=red,solid,line width=1.2pt,forget plot]
  table[row sep=crcr]{%
2.63846951643784	-7.9517844136676\\
2.8299899636036	-7.89416568131931\\
};
\addplot [color=red,solid,line width=1.2pt,forget plot]
  table[row sep=crcr]{%
2.63846340292862	-7.95176951640953\\
2.82999122767183	-7.89417531228156\\
};
\addplot [color=red,solid,line width=1.2pt,forget plot]
  table[row sep=crcr]{%
2.63844150303847	-7.95173762577653\\
2.82997581277602	-7.89416499170999\\
};
\addplot [color=red,solid,line width=1.2pt,forget plot]
  table[row sep=crcr]{%
2.63845065885175	-7.95171819157275\\
2.8300044124898	-7.89421028370635\\
};
\addplot [color=red,solid,line width=1.2pt,forget plot]
  table[row sep=crcr]{%
2.63845136258096	-7.9516760660425\\
2.83001449293303	-7.89419940043187\\
};
\addplot [color=red,solid,line width=1.2pt,forget plot]
  table[row sep=crcr]{%
2.63840496421512	-7.95163248793757\\
2.82998200920028	-7.89420221861248\\
};
\addplot [color=red,solid,line width=1.2pt,forget plot]
  table[row sep=crcr]{%
2.63839823697578	-7.95156918596573\\
2.82998757105791	-7.89417992681416\\
};
\addplot [color=red,solid,line width=1.2pt,forget plot]
  table[row sep=crcr]{%
2.63837217585942	-7.95153406778512\\
2.82997213134947	-7.89418027927721\\
};
\addplot [color=red,solid,line width=1.2pt,forget plot]
  table[row sep=crcr]{%
2.63835111345246	-7.95153535438121\\
2.82996138363237	-7.89421603511117\\
};
\addplot [color=red,solid,line width=1.2pt,forget plot]
  table[row sep=crcr]{%
2.63830504167505	-7.95150449580283\\
2.8299319582162	-7.89424085238984\\
};
\addplot [color=red,solid,line width=1.2pt,forget plot]
  table[row sep=crcr]{%
2.63826854410806	-7.9514067615963\\
2.82990677487444	-7.89418099372262\\
};
\addplot [color=red,solid,line width=1.2pt,forget plot]
  table[row sep=crcr]{%
1.71554710414907	-8.23080182555135\\
1.90687442155057	-8.17254500092157\\
};
\addplot [color=red,solid,line width=1.2pt,forget plot]
  table[row sep=crcr]{%
1.71552411883683	-8.23072580717981\\
1.90687705865665	-8.1725531983135\\
};
\addplot [color=red,solid,line width=1.2pt,forget plot]
  table[row sep=crcr]{%
1.71551231243806	-8.2306361504327\\
1.90688490931881	-8.17252824078935\\
};
\addplot [color=red,solid,line width=1.2pt,forget plot]
  table[row sep=crcr]{%
1.71548502059696	-8.23060524798266\\
1.90687804521584	-8.17256465764076\\
};
\addplot [color=red,solid,line width=1.2pt,forget plot]
  table[row sep=crcr]{%
1.71544133689879	-8.2305284725194\\
1.90685415641361	-8.17255319739815\\
};
\addplot [color=red,solid,line width=1.2pt,forget plot]
  table[row sep=crcr]{%
1.71541990722863	-8.23054810236918\\
1.90682837049961	-8.17255844649642\\
};
\addplot [color=red,solid,line width=1.2pt,forget plot]
  table[row sep=crcr]{%
1.71543342051126	-8.23059125141397\\
1.90682968529618	-8.17256134682695\\
};
\addplot [color=red,solid,line width=1.2pt,forget plot]
  table[row sep=crcr]{%
1.71542505993658	-8.23060762399561\\
1.906815699428	-8.17255916910606\\
};
\addplot [color=red,solid,line width=1.2pt,forget plot]
  table[row sep=crcr]{%
1.71540801580703	-8.2306033158377\\
1.90679935819447	-8.17255717850598\\
};
\addplot [color=red,solid,line width=1.2pt,forget plot]
  table[row sep=crcr]{%
1.71541220220406	-8.2306137494235\\
1.90679967656746	-8.17255485986473\\
};
\addplot [color=red,solid,line width=1.2pt,forget plot]
  table[row sep=crcr]{%
1.7154195752042	-8.23062790528133\\
1.90680146083421	-8.17255059600568\\
};
\addplot [color=red,solid,line width=1.2pt,forget plot]
  table[row sep=crcr]{%
1.71540350503085	-8.23063527717648\\
1.90677807066139	-8.1725338518091\\
};
\addplot [color=red,solid,line width=1.2pt,forget plot]
  table[row sep=crcr]{%
1.71540704942553	-8.23065469087057\\
1.90678059964211	-8.17254992103621\\
};
\addplot [color=red,solid,line width=1.2pt,forget plot]
  table[row sep=crcr]{%
1.71540118303375	-8.23067709713823\\
1.90677841558989	-8.17258445681993\\
};
\addplot [color=red,solid,line width=1.2pt,forget plot]
  table[row sep=crcr]{%
1.71538916920588	-8.23067957634387\\
1.90676390448759	-8.17257870977669\\
};
\addplot [color=red,solid,line width=1.2pt,forget plot]
  table[row sep=crcr]{%
1.71540201295659	-8.23072514955943\\
1.90676900838504	-8.17259879528972\\
};
\addplot [color=red,solid,line width=1.2pt,forget plot]
  table[row sep=crcr]{%
1.71541745549328	-8.23077120116736\\
1.90678562014077	-8.17264869640842\\
};
\addplot [color=red,solid,line width=1.2pt,forget plot]
  table[row sep=crcr]{%
1.71543003810235	-8.23081662098642\\
1.9067932457188	-8.1726777977205\\
};
\addplot [color=red,solid,line width=1.2pt,forget plot]
  table[row sep=crcr]{%
1.7154435723787	-8.23080615897317\\
1.90680519431729	-8.17266211672437\\
};
\addplot [color=red,solid,line width=1.2pt,forget plot]
  table[row sep=crcr]{%
1.71539745852962	-8.23083662934024\\
1.90674499788662	-8.17264625915736\\
};
\addplot [color=red,solid,line width=1.2pt,forget plot]
  table[row sep=crcr]{%
0.795230264601168	-8.49646391031669\\
0.986734993788372	-8.43879295836568\\
};
\addplot [color=red,solid,line width=1.2pt,forget plot]
  table[row sep=crcr]{%
0.795256299115465	-8.49651519605684\\
0.986748637973336	-8.43880311622304\\
};
\addplot [color=red,solid,line width=1.2pt,forget plot]
  table[row sep=crcr]{%
0.795245214275452	-8.49655769050558\\
0.986722114977436	-8.43879441066828\\
};
\addplot [color=red,solid,line width=1.2pt,forget plot]
  table[row sep=crcr]{%
0.795238421259363	-8.49654279888744\\
0.986718174083607	-8.43878897426632\\
};
\addplot [color=red,solid,line width=1.2pt,forget plot]
  table[row sep=crcr]{%
0.795266544285382	-8.49655671480438\\
0.98675070251071	-8.43881749807314\\
};
\addplot [color=red,solid,line width=1.2pt,forget plot]
  table[row sep=crcr]{%
0.795294466946626	-8.49663751109818\\
0.986778593666066	-8.4388981898821\\
};
\addplot [color=red,solid,line width=1.2pt,forget plot]
  table[row sep=crcr]{%
0.795298916178087	-8.49667068350727\\
0.986779456942856	-8.43891947132578\\
};
\addplot [color=red,solid,line width=1.2pt,forget plot]
  table[row sep=crcr]{%
0.795300927448194	-8.49666517533121\\
0.986773194028063	-8.43888653627863\\
};
\addplot [color=red,solid,line width=1.2pt,forget plot]
  table[row sep=crcr]{%
0.795321188464123	-8.49665313295348\\
0.986795130424181	-8.43888004622379\\
};
\addplot [color=red,solid,line width=1.2pt,forget plot]
  table[row sep=crcr]{%
0.795310674791571	-8.49661833660989\\
0.986789552501103	-8.43886161067171\\
};
\addplot [color=red,solid,line width=1.2pt,forget plot]
  table[row sep=crcr]{%
0.795307986646661	-8.4966323335002\\
0.986781596125176	-8.43885814485796\\
};
\addplot [color=red,solid,line width=1.2pt,forget plot]
  table[row sep=crcr]{%
0.795309574036338	-8.49667464303075\\
0.986782333402843	-8.43889763704587\\
};
\addplot [color=red,solid,line width=1.2pt,forget plot]
  table[row sep=crcr]{%
0.795302407908802	-8.4966912416309\\
0.986772548392573	-8.43890555739055\\
};
\addplot [color=red,solid,line width=1.2pt,forget plot]
  table[row sep=crcr]{%
0.795297048663001	-8.4966738865333\\
0.986762910941454	-8.43887402855797\\
};
\addplot [color=red,solid,line width=1.2pt,forget plot]
  table[row sep=crcr]{%
0.79528850525457	-8.49670550083093\\
0.986754074986965	-8.43890467378625\\
};
\addplot [color=red,solid,line width=1.2pt,forget plot]
  table[row sep=crcr]{%
0.795286442609426	-8.49668902791516\\
0.986753517832279	-8.43889318805098\\
};
\addplot [color=red,solid,line width=1.2pt,forget plot]
  table[row sep=crcr]{%
0.795322405039681	-8.49669154479819\\
0.986788194207171	-8.4388914446385\\
};
\addplot [color=red,solid,line width=1.2pt,forget plot]
  table[row sep=crcr]{%
0.795302425387753	-8.49668739987179\\
0.986776152137793	-8.43891359988901\\
};
\addplot [color=red,solid,line width=1.2pt,forget plot]
  table[row sep=crcr]{%
0.795261789027139	-8.49670173679918\\
0.986733905515367	-8.4389226003612\\
};
\addplot [color=red,solid,line width=1.2pt,forget plot]
  table[row sep=crcr]{%
0.794110891712912	-8.49675779742601\\
0.985665483867772	-8.43925268265884\\
};
\addplot [color=red,solid,line width=1.2pt,forget plot]
  table[row sep=crcr]{%
-0.158586982667629	-8.7833654671018\\
0.0329502153223381	-8.72580244263164\\
};
\addplot [color=red,solid,line width=1.2pt,forget plot]
  table[row sep=crcr]{%
-0.158723990511111	-8.78345125178162\\
0.0327931281049975	-8.72582145673615\\
};
\addplot [color=red,solid,line width=1.2pt,forget plot]
  table[row sep=crcr]{%
-0.158762086304151	-8.78348658314014\\
0.032760001138832	-8.72587330323453\\
};
\addplot [color=red,solid,line width=1.2pt,forget plot]
  table[row sep=crcr]{%
-0.158727158710954	-8.78348409693103\\
0.0327980771851422	-8.72588128436997\\
};
\addplot [color=red,solid,line width=1.2pt,forget plot]
  table[row sep=crcr]{%
-0.158724628667087	-8.78350239742951\\
0.0327973189717892	-8.72588865277937\\
};
\addplot [color=red,solid,line width=1.2pt,forget plot]
  table[row sep=crcr]{%
-0.158766619822833	-8.78348211537432\\
0.032764261921166	-8.72589807819069\\
};
\addplot [color=red,solid,line width=1.2pt,forget plot]
  table[row sep=crcr]{%
-0.158800070322492	-8.7834144381862\\
0.0327447383429215	-8.72587674381694\\
};
\addplot [color=red,solid,line width=1.2pt,forget plot]
  table[row sep=crcr]{%
-0.158847037063923	-8.78340437736429\\
0.032706144326691	-8.72589456342672\\
};
\addplot [color=red,solid,line width=1.2pt,forget plot]
  table[row sep=crcr]{%
-0.158874846830897	-8.78342364252373\\
0.0326781704603521	-8.72591328200826\\
};
\addplot [color=red,solid,line width=1.2pt,forget plot]
  table[row sep=crcr]{%
-0.158943688318448	-8.78341517244565\\
0.0326224260904623	-8.72594845329211\\
};
\addplot [color=red,solid,line width=1.2pt,forget plot]
  table[row sep=crcr]{%
-0.158993037377651	-8.78336467500203\\
0.0325823899104246	-8.72592900960681\\
};
\addplot [color=red,solid,line width=1.2pt,forget plot]
  table[row sep=crcr]{%
-0.159003265159628	-8.78334467696898\\
0.032579362710215	-8.72593303443469\\
};
\addplot [color=red,solid,line width=1.2pt,forget plot]
  table[row sep=crcr]{%
-0.159005541535318	-8.7833040121837\\
0.0325916612917359	-8.72594102874621\\
};
\addplot [color=red,solid,line width=1.2pt,forget plot]
  table[row sep=crcr]{%
-0.159039691913963	-8.78327898286845\\
0.0325728913089776	-8.72596739633805\\
};
\addplot [color=red,solid,line width=1.2pt,forget plot]
  table[row sep=crcr]{%
-0.159050548745233	-8.78326636974079\\
0.0325745590511098	-8.725996673897\\
};
\addplot [color=red,solid,line width=1.2pt,forget plot]
  table[row sep=crcr]{%
-0.159072128441098	-8.78327178210202\\
0.0325631792210035	-8.72603622621888\\
};
\addplot [color=red,solid,line width=1.2pt,forget plot]
  table[row sep=crcr]{%
-0.159047655531669	-8.78327677494797\\
0.0325923001726967	-8.72605678388165\\
};
\addplot [color=red,solid,line width=1.2pt,forget plot]
  table[row sep=crcr]{%
-0.159007150581759	-8.78333137889996\\
0.03261849174047	-8.7260634716168\\
};
\addplot [color=red,solid,line width=1.2pt,forget plot]
  table[row sep=crcr]{%
-0.159075662087309	-8.78321468705762\\
0.0326292284636099	-8.72621262622768\\
};
\addplot [color=red,solid,line width=1.2pt,forget plot]
  table[row sep=crcr]{%
-1.14088343899012	-9.08729772186349\\
-0.949642709556767	-9.0287572793831\\
};
\addplot [color=red,solid,line width=1.2pt,forget plot]
  table[row sep=crcr]{%
-1.14091135749211	-9.08719782962034\\
-0.94960096225347	-9.02888545808877\\
};
\addplot [color=red,solid,line width=1.2pt,forget plot]
  table[row sep=crcr]{%
-1.14094251225567	-9.0871412062247\\
-0.949598049537676	-9.02894072008273\\
};
\addplot [color=red,solid,line width=1.2pt,forget plot]
  table[row sep=crcr]{%
-1.14098083544279	-9.08708853206953\\
-0.949612783778286	-9.02896565531577\\
};
\addplot [color=red,solid,line width=1.2pt,forget plot]
  table[row sep=crcr]{%
-1.14099973738316	-9.08705289532762\\
-0.949610933099811	-9.02899838994697\\
};
\addplot [color=red,solid,line width=1.2pt,forget plot]
  table[row sep=crcr]{%
-1.14097608235601	-9.08701590802185\\
-0.94957090916798	-9.02901539356846\\
};
\addplot [color=red,solid,line width=1.2pt,forget plot]
  table[row sep=crcr]{%
-1.14098685595602	-9.08702482915056\\
-0.949570191620919	-9.029062249715\\
};
\addplot [color=red,solid,line width=1.2pt,forget plot]
  table[row sep=crcr]{%
-1.1410197861969	-9.08700622034246\\
-0.949598278491764	-9.0290596381491\\
};
\addplot [color=red,solid,line width=1.2pt,forget plot]
  table[row sep=crcr]{%
-1.14102013983317	-9.08696765239596\\
-0.949603854039527	-9.029003822873\\
};
\addplot [color=red,solid,line width=1.2pt,forget plot]
  table[row sep=crcr]{%
-1.14101430798519	-9.08698096868782\\
-0.949603241477032	-9.02899990610521\\
};
\addplot [color=red,solid,line width=1.2pt,forget plot]
  table[row sep=crcr]{%
-1.14100392365397	-9.08701976507541\\
-0.949593500340254	-9.02903657917755\\
};
\addplot [color=red,solid,line width=1.2pt,forget plot]
  table[row sep=crcr]{%
-1.14101960085492	-9.08708918092601\\
-0.949612924693375	-9.02909362660684\\
};
\addplot [color=red,solid,line width=1.2pt,forget plot]
  table[row sep=crcr]{%
-1.14101334534332	-9.08711418892877\\
-0.949613888548518	-9.02909481338358\\
};
\addplot [color=red,solid,line width=1.2pt,forget plot]
  table[row sep=crcr]{%
-1.14101844988115	-9.08712540211427\\
-0.949621771529783	-9.02909686158407\\
};
\addplot [color=red,solid,line width=1.2pt,forget plot]
  table[row sep=crcr]{%
-1.14102932078646	-9.0871514977416\\
-0.949635428583175	-9.02911376839872\\
};
\addplot [color=red,solid,line width=1.2pt,forget plot]
  table[row sep=crcr]{%
-1.141002482095	-9.08713714744252\\
-0.949612706691659	-9.02908584365821\\
};
\addplot [color=red,solid,line width=1.2pt,forget plot]
  table[row sep=crcr]{%
-1.14098452802236	-9.08713760996402\\
-0.949599691911518	-9.02907002428297\\
};
\addplot [color=red,solid,line width=1.2pt,forget plot]
  table[row sep=crcr]{%
-1.14098767105348	-9.08714358410581\\
-0.949602979443051	-9.02907552216828\\
};
\addplot [color=red,solid,line width=1.2pt,forget plot]
  table[row sep=crcr]{%
-1.14100132265796	-9.08712346932235\\
-0.949612276702183	-9.02906976067131\\
};
\addplot [color=red,solid,line width=1.2pt,forget plot]
  table[row sep=crcr]{%
-1.14119791447704	-9.08695653604098\\
-0.949717803704495	-9.02920389819183\\
};
\addplot [color=red,solid,line width=1.2pt,forget plot]
  table[row sep=crcr]{%
-2.01634319660556	-9.33851160415356\\
-1.82483733093892	-9.28084442618143\\
};
\addplot [color=red,solid,line width=1.2pt,forget plot]
  table[row sep=crcr]{%
-2.01636707036756	-9.33841411327546\\
-1.82481335672003	-9.28090607220432\\
};
\addplot [color=red,solid,line width=1.2pt,forget plot]
  table[row sep=crcr]{%
-2.01635234286196	-9.3384748121267\\
-1.82484242539008	-9.28082109143224\\
};
\addplot [color=red,solid,line width=1.2pt,forget plot]
  table[row sep=crcr]{%
-2.01634108161912	-9.33849010298106\\
-1.82485482614759	-9.28075784193497\\
};
\addplot [color=red,solid,line width=1.2pt,forget plot]
  table[row sep=crcr]{%
-2.01634297138764	-9.33850698876846\\
-1.82486706148042	-9.28074042469199\\
};
\addplot [color=red,solid,line width=1.2pt,forget plot]
  table[row sep=crcr]{%
-2.01635845072249	-9.33852040786126\\
-1.82488835094439	-9.28073458874477\\
};
\addplot [color=red,solid,line width=1.2pt,forget plot]
  table[row sep=crcr]{%
-2.01636686069995	-9.33854058958485\\
-1.82490435053517	-9.28072962867429\\
};
\addplot [color=red,solid,line width=1.2pt,forget plot]
  table[row sep=crcr]{%
-2.01635238254061	-9.33848315583804\\
-1.82490103926246	-9.28063522452678\\
};
\addplot [color=red,solid,line width=1.2pt,forget plot]
  table[row sep=crcr]{%
-2.01635743387405	-9.33848886710206\\
-1.82492139060166	-9.28059032362979\\
};
\addplot [color=red,solid,line width=1.2pt,forget plot]
  table[row sep=crcr]{%
-2.01637475809494	-9.33846105164149\\
-1.82494661650878	-9.28053638842394\\
};
\addplot [color=red,solid,line width=1.2pt,forget plot]
  table[row sep=crcr]{%
-2.01639666347181	-9.33845324076413\\
-1.82497446017003	-9.28050895646368\\
};
\addplot [color=red,solid,line width=1.2pt,forget plot]
  table[row sep=crcr]{%
-2.01640898546917	-9.33843228877678\\
-1.82499256843647	-9.28046889265161\\
};
\addplot [color=red,solid,line width=1.2pt,forget plot]
  table[row sep=crcr]{%
-2.01643392937435	-9.33844535755303\\
-1.82502813813684	-9.28044688273911\\
};
\addplot [color=red,solid,line width=1.2pt,forget plot]
  table[row sep=crcr]{%
-2.01645279649656	-9.33841509795157\\
-1.82505392176994	-9.28039380226297\\
};
\addplot [color=red,solid,line width=1.2pt,forget plot]
  table[row sep=crcr]{%
-2.0164519655988	-9.33838077774379\\
-1.82506919311658	-9.28030639098838\\
};
\addplot [color=red,solid,line width=1.2pt,forget plot]
  table[row sep=crcr]{%
-2.01646962179443	-9.33833487434563\\
-1.82508868949216	-9.28025442366629\\
};
\addplot [color=red,solid,line width=1.2pt,forget plot]
  table[row sep=crcr]{%
-2.01654163619716	-9.3382586481534\\
-1.82514704980329	-9.28022320812645\\
};
\addplot [color=red,solid,line width=1.2pt,forget plot]
  table[row sep=crcr]{%
-3.32752785923869	-9.72733456818213\\
-3.13595659328878	-9.66988502456038\\
};
\addplot [color=red,solid,line width=1.2pt,forget plot]
  table[row sep=crcr]{%
-3.32744465442763	-9.72735588006285\\
-3.13589935234763	-9.66981982831198\\
};
\addplot [color=red,solid,line width=1.2pt,forget plot]
  table[row sep=crcr]{%
-3.32744783923918	-9.72743560481703\\
-3.13589979964104	-9.66990866742148\\
};
\addplot [color=red,solid,line width=1.2pt,forget plot]
  table[row sep=crcr]{%
-3.32741865019896	-9.72742175306575\\
-3.13587172601239	-9.66989110180347\\
};
\addplot [color=red,solid,line width=1.2pt,forget plot]
  table[row sep=crcr]{%
-3.32738023265364	-9.72740298606872\\
-3.13582310900785	-9.66990630459449\\
};
\addplot [color=red,solid,line width=1.2pt,forget plot]
  table[row sep=crcr]{%
-3.32740453159379	-9.72744758087448\\
-3.13583441600465	-9.66999420139888\\
};
\addplot [color=red,solid,line width=1.2pt,forget plot]
  table[row sep=crcr]{%
-3.32737590017784	-9.72748613584558\\
-3.13579825680187	-9.67005786268235\\
};
\addplot [color=red,solid,line width=1.2pt,forget plot]
  table[row sep=crcr]{%
-3.32734679501029	-9.72749725145627\\
-3.13575601433672	-9.670112821853\\
};
\addplot [color=red,solid,line width=1.2pt,forget plot]
  table[row sep=crcr]{%
-3.32735210149304	-9.72750310133501\\
-3.13574754444811	-9.67016468733867\\
};
\addplot [color=red,solid,line width=1.2pt,forget plot]
  table[row sep=crcr]{%
-3.32734718643417	-9.7275223771886\\
-3.13573360317405	-9.67021413423684\\
};
\addplot [color=red,solid,line width=1.2pt,forget plot]
  table[row sep=crcr]{%
-3.32734774590384	-9.72752987768841\\
-3.1357199482509	-9.67026918290844\\
};
\addplot [color=red,solid,line width=1.2pt,forget plot]
  table[row sep=crcr]{%
-3.32736509438594	-9.72755906738566\\
-3.13572542844145	-9.6703381058708\\
};
\addplot [color=red,solid,line width=1.2pt,forget plot]
  table[row sep=crcr]{%
-3.32737870984207	-9.72757598100511\\
-3.13573016369898	-9.67038476873224\\
};
\addplot [color=red,solid,line width=1.2pt,forget plot]
  table[row sep=crcr]{%
-3.32737692785994	-9.72755616652083\\
-3.13571687001105	-9.67040354431186\\
};
\addplot [color=red,solid,line width=1.2pt,forget plot]
  table[row sep=crcr]{%
-3.32739441567569	-9.72754397373657\\
-3.13572451581603	-9.67042436687701\\
};
\addplot [color=red,solid,line width=1.2pt,forget plot]
  table[row sep=crcr]{%
-3.32739645172985	-9.72754301641227\\
-3.13570850878838	-9.67048398968671\\
};
\addplot [color=red,solid,line width=1.2pt,forget plot]
  table[row sep=crcr]{%
-3.32743732602615	-9.72750515279243\\
-3.13573814913932	-9.67048387973656\\
};
\addplot [color=red,solid,line width=1.2pt,forget plot]
  table[row sep=crcr]{%
-3.32705516863571	-9.72492182566993\\
-3.13530642664407	-9.66806745077562\\
};
\addplot [color=red,solid,line width=1.2pt,forget plot]
  table[row sep=crcr]{%
-4.41595362552859	-9.13139020731111\\
-4.35893560629249	-9.32309035202256\\
};
\addplot [color=red,solid,line width=1.2pt,forget plot]
  table[row sep=crcr]{%
-4.4158877191837	-9.13182450080318\\
-4.35883688900054	-9.32351488338773\\
};
\addplot [color=red,solid,line width=1.2pt,forget plot]
  table[row sep=crcr]{%
-4.41591217953841	-9.13189576163177\\
-4.35882797216219	-9.32357620734345\\
};
\addplot [color=red,solid,line width=1.2pt,forget plot]
  table[row sep=crcr]{%
-4.41602730469752	-9.13191842669757\\
-4.35877650517617	-9.32354918089472\\
};
\addplot [color=red,solid,line width=1.2pt,forget plot]
  table[row sep=crcr]{%
-4.4160208155735	-9.13193893669551\\
-4.35879428625274	-9.32357694008184\\
};
\addplot [color=red,solid,line width=1.2pt,forget plot]
  table[row sep=crcr]{%
-4.41602940777603	-9.13191530891778\\
-4.35887650186871	-9.3235752821682\\
};
\addplot [color=red,solid,line width=1.2pt,forget plot]
  table[row sep=crcr]{%
-4.41598733219684	-9.13193401886643\\
-4.35885301063291	-9.32359953297742\\
};
\addplot [color=red,solid,line width=1.2pt,forget plot]
  table[row sep=crcr]{%
-4.41601942201915	-9.13200101818541\\
-4.35875661424225	-9.32362818443671\\
};
\addplot [color=red,solid,line width=1.2pt,forget plot]
  table[row sep=crcr]{%
-4.41600895386915	-9.13198233698291\\
-4.35871978103311	-9.32360162274445\\
};
\addplot [color=red,solid,line width=1.2pt,forget plot]
  table[row sep=crcr]{%
-4.41598073009733	-9.13197959283936\\
-4.3587370277511	-9.32361246721559\\
};
\addplot [color=red,solid,line width=1.2pt,forget plot]
  table[row sep=crcr]{%
-4.41600192852932	-9.13201157082308\\
-4.35876087452016	-9.3236452362786\\
};
\addplot [color=red,solid,line width=1.2pt,forget plot]
  table[row sep=crcr]{%
-4.41595609566212	-9.13201282507232\\
-4.3586977947811	-9.32364133803482\\
};
\addplot [color=red,solid,line width=1.2pt,forget plot]
  table[row sep=crcr]{%
-4.41592877159624	-9.13202811909315\\
-4.35865948400954	-9.32365334890205\\
};
\addplot [color=red,solid,line width=1.2pt,forget plot]
  table[row sep=crcr]{%
-4.4159141424124	-9.13201388435163\\
-4.35864819491511	-9.32364011235092\\
};
\addplot [color=red,solid,line width=1.2pt,forget plot]
  table[row sep=crcr]{%
-4.41589706966262	-9.13199536979547\\
-4.3585999249992	-9.32361227200772\\
};
\addplot [color=red,solid,line width=1.2pt,forget plot]
  table[row sep=crcr]{%
-4.66456801137207	-8.1524538042568\\
-4.60756747793075	-8.34415914895904\\
};
\addplot [color=red,solid,line width=1.2pt,forget plot]
  table[row sep=crcr]{%
-4.66450500105956	-8.15247946831613\\
-4.60749429213264	-8.34418178720573\\
};
\addplot [color=red,solid,line width=1.2pt,forget plot]
  table[row sep=crcr]{%
-4.66441817768252	-8.15247153522894\\
-4.60740698236025	-8.34417370946785\\
};
\addplot [color=red,solid,line width=1.2pt,forget plot]
  table[row sep=crcr]{%
-4.66434148985698	-8.15248151799119\\
-4.60733819066583	-8.34418604032002\\
};
\addplot [color=red,solid,line width=1.2pt,forget plot]
  table[row sep=crcr]{%
-4.66436668151238	-8.15246193519959\\
-4.6073546209042	-8.34416385210499\\
};
\addplot [color=red,solid,line width=1.2pt,forget plot]
  table[row sep=crcr]{%
-4.66438552198046	-8.15249691334235\\
-4.60735571519229	-8.34419355164805\\
};
\addplot [color=red,solid,line width=1.2pt,forget plot]
  table[row sep=crcr]{%
-4.66441195665681	-8.15256753970241\\
-4.60737296122259	-8.34426144414346\\
};
\addplot [color=red,solid,line width=1.2pt,forget plot]
  table[row sep=crcr]{%
-4.66451220198393	-8.15260832430797\\
-4.60748088459844	-8.34430451320403\\
};
\addplot [color=red,solid,line width=1.2pt,forget plot]
  table[row sep=crcr]{%
-4.66455418709878	-8.1526207609718\\
-4.60752399344637	-8.34431728418484\\
};
\addplot [color=red,solid,line width=1.2pt,forget plot]
  table[row sep=crcr]{%
-4.66452055651912	-8.15261240283567\\
-4.60750402854877	-8.344312991093\\
};
\addplot [color=red,solid,line width=1.2pt,forget plot]
  table[row sep=crcr]{%
-4.66452755102244	-8.15259706971393\\
-4.60752623117288	-8.34430218058845\\
};
\addplot [color=red,solid,line width=1.2pt,forget plot]
  table[row sep=crcr]{%
-4.6645279710037	-8.15258459101473\\
-4.60755061963834	-8.34429682701282\\
};
\addplot [color=red,solid,line width=1.2pt,forget plot]
  table[row sep=crcr]{%
-4.66449920880767	-8.15261015504951\\
-4.60755036848499	-8.34433086229289\\
};
\addplot [color=red,solid,line width=1.2pt,forget plot]
  table[row sep=crcr]{%
-4.66456556093337	-8.15260122123448\\
-4.60763697958208	-8.34432794505077\\
};
\addplot [color=red,solid,line width=1.2pt,forget plot]
  table[row sep=crcr]{%
-4.66451233107542	-8.15254961306123\\
-4.60760660069318	-8.34428312043452\\
};
\addplot [color=red,solid,line width=1.2pt,forget plot]
  table[row sep=crcr]{%
-4.66433994442215	-8.15248385118448\\
-4.60741371697149	-8.34421127391851\\
};
\addplot [color=red,solid,line width=1.2pt,forget plot]
  table[row sep=crcr]{%
-4.98701651659064	-7.26024179773908\\
-4.92972162367693	-7.45185937325345\\
};
\addplot [color=red,solid,line width=1.2pt,forget plot]
  table[row sep=crcr]{%
-4.9869936492682	-7.26028263480463\\
-4.92974319568361	-7.45191349235205\\
};
\addplot [color=red,solid,line width=1.2pt,forget plot]
  table[row sep=crcr]{%
-4.98695661273152	-7.26035016031587\\
-4.92975707292032	-7.4519962211817\\
};
\addplot [color=red,solid,line width=1.2pt,forget plot]
  table[row sep=crcr]{%
-4.98702411148529	-7.26035485741696\\
-4.92982747224405	-7.45200178397595\\
};
\addplot [color=red,solid,line width=1.2pt,forget plot]
  table[row sep=crcr]{%
-4.9870452099033	-7.26043028803154\\
-4.92985029994267	-7.45207773068235\\
};
\addplot [color=red,solid,line width=1.2pt,forget plot]
  table[row sep=crcr]{%
-4.98709381887643	-7.26046716223265\\
-4.92990056618676	-7.45211509946856\\
};
\addplot [color=red,solid,line width=1.2pt,forget plot]
  table[row sep=crcr]{%
-4.98712840032673	-7.26052991720394\\
-4.92994979738857	-7.45218222573697\\
};
\addplot [color=red,solid,line width=1.2pt,forget plot]
  table[row sep=crcr]{%
-4.98713818729849	-7.26055489647553\\
-4.92995388792973	-7.45220550541196\\
};
\addplot [color=red,solid,line width=1.2pt,forget plot]
  table[row sep=crcr]{%
-4.98711045784442	-7.2605215096937\\
-4.92993038307671	-7.45217337910686\\
};
\addplot [color=red,solid,line width=1.2pt,forget plot]
  table[row sep=crcr]{%
-4.98718136964581	-7.26056892643704\\
-4.93000609949922	-7.45222222926187\\
};
\addplot [color=red,solid,line width=1.2pt,forget plot]
  table[row sep=crcr]{%
-4.98719902942209	-7.26059466021562\\
-4.93002219923789	-7.45224749763288\\
};
\addplot [color=red,solid,line width=1.2pt,forget plot]
  table[row sep=crcr]{%
-4.98721042656213	-7.26063618699951\\
-4.93001424580111	-7.4522832503904\\
};
\addplot [color=red,solid,line width=1.2pt,forget plot]
  table[row sep=crcr]{%
-4.98722811974266	-7.26067900679061\\
-4.9300392991576	-7.4523282666381\\
};
\addplot [color=red,solid,line width=1.2pt,forget plot]
  table[row sep=crcr]{%
-4.98725428664337	-7.26074370971963\\
-4.93006084403907	-7.45239159027944\\
};
\addplot [color=red,solid,line width=1.2pt,forget plot]
  table[row sep=crcr]{%
-4.98766934003874	-7.26083425105971\\
-4.93048717186334	-7.45248549588415\\
};
\addplot [color=red,solid,line width=1.2pt,forget plot]
  table[row sep=crcr]{%
-5.24763869877198	-6.303616059929\\
-5.1903206720893	-6.49522671677565\\
};
\addplot [color=red,solid,line width=1.2pt,forget plot]
  table[row sep=crcr]{%
-5.24767487029397	-6.30363101426126\\
-5.19033238856363	-6.49523435397381\\
};
\addplot [color=red,solid,line width=1.2pt,forget plot]
  table[row sep=crcr]{%
-5.24762937610189	-6.30357438092964\\
-5.19039052877081	-6.49520870550697\\
};
\addplot [color=red,solid,line width=1.2pt,forget plot]
  table[row sep=crcr]{%
-5.24762587140914	-6.30357968296511\\
-5.19039492447814	-6.49521636711869\\
};
\addplot [color=red,solid,line width=1.2pt,forget plot]
  table[row sep=crcr]{%
-5.2475741728509	-6.3035642325722\\
-5.19040694078569	-6.49521993318662\\
};
\addplot [color=red,solid,line width=1.2pt,forget plot]
  table[row sep=crcr]{%
-5.2475731718004	-6.3035757030774\\
-5.19042000282694	-6.49523559788143\\
};
\addplot [color=red,solid,line width=1.2pt,forget plot]
  table[row sep=crcr]{%
-5.24761055730465	-6.30360991312775\\
-5.19046891650589	-6.49527324526727\\
};
\addplot [color=red,solid,line width=1.2pt,forget plot]
  table[row sep=crcr]{%
-5.24763546881598	-6.30367023591983\\
-5.19049742536702	-6.49533464052023\\
};
\addplot [color=red,solid,line width=1.2pt,forget plot]
  table[row sep=crcr]{%
-5.24763944488561	-6.303687934365\\
-5.19049103004009	-6.49534924679045\\
};
\addplot [color=red,solid,line width=1.2pt,forget plot]
  table[row sep=crcr]{%
-5.24762266707228	-6.30371630225748\\
-5.19049231580581	-6.49538299984516\\
};
\addplot [color=red,solid,line width=1.2pt,forget plot]
  table[row sep=crcr]{%
-5.24759705466464	-6.30373423400032\\
-5.1904766578088	-6.49540389843116\\
};
\addplot [color=red,solid,line width=1.2pt,forget plot]
  table[row sep=crcr]{%
-5.24762365719637	-6.30380486548214\\
-5.19050138834166	-6.49547397201965\\
};
\addplot [color=red,solid,line width=1.2pt,forget plot]
  table[row sep=crcr]{%
-5.24768127723934	-6.3038925177623\\
-5.19056533865547	-6.49556351076772\\
};
\addplot [color=red,solid,line width=1.2pt,forget plot]
  table[row sep=crcr]{%
-5.24766132105748	-6.30391638902466\\
-5.19052727199268	-6.49558198436577\\
};
\addplot [color=red,solid,line width=1.2pt,forget plot]
  table[row sep=crcr]{%
-5.24765654893132	-6.30392912356338\\
-5.1905270057624	-6.49559606201994\\
};
\addplot [color=red,solid,line width=1.2pt,forget plot]
  table[row sep=crcr]{%
-5.2476492155911	-6.30393769541689\\
-5.19052431210592	-6.4956060167478\\
};
\addplot [color=red,solid,line width=1.2pt,forget plot]
  table[row sep=crcr]{%
-5.24764475730174	-6.30396594151902\\
-5.19051627970818	-6.49563319758495\\
};
\addplot [color=red,solid,line width=1.2pt,forget plot]
  table[row sep=crcr]{%
-5.24764617437695	-6.30396063205809\\
-5.19052267527094	-6.49562937194512\\
};
\addplot [color=red,solid,line width=1.2pt,forget plot]
  table[row sep=crcr]{%
-5.24763479025448	-6.30394199293802\\
-5.19049755243636	-6.49560663770725\\
};
\addplot [color=red,solid,line width=1.2pt,forget plot]
  table[row sep=crcr]{%
-5.2475526292077	-6.30371555003786\\
-5.19043245780606	-6.49538528165725\\
};
\addplot [color=red,solid,line width=1.2pt,forget plot]
  table[row sep=crcr]{%
-5.49645365964395	-5.37702366638372\\
-5.43927648253291	-5.56867640030007\\
};
\addplot [color=red,solid,line width=1.2pt,forget plot]
  table[row sep=crcr]{%
-5.49648850723497	-5.37706748496317\\
-5.43930700395605	-5.5687189281687\\
};
\addplot [color=red,solid,line width=1.2pt,forget plot]
  table[row sep=crcr]{%
-5.4964707995634	-5.37713287066184\\
-5.43931516010439	-5.5687920287473\\
};
\addplot [color=red,solid,line width=1.2pt,forget plot]
  table[row sep=crcr]{%
-5.49644918991281	-5.3772024043106\\
-5.43930494459434	-5.5688649599311\\
};
\addplot [color=red,solid,line width=1.2pt,forget plot]
  table[row sep=crcr]{%
-5.49638765326852	-5.37722366088311\\
-5.4392540938706	-5.56888940218949\\
};
\addplot [color=red,solid,line width=1.2pt,forget plot]
  table[row sep=crcr]{%
-5.49636508224928	-5.37721884497556\\
-5.43922997414891	-5.56888412462308\\
};
\addplot [color=red,solid,line width=1.2pt,forget plot]
  table[row sep=crcr]{%
-5.49640342223906	-5.37722689800731\\
-5.43927350875209	-5.56889372608398\\
};
\addplot [color=red,solid,line width=1.2pt,forget plot]
  table[row sep=crcr]{%
-5.49638271189287	-5.37724639324035\\
-5.43927706643259	-5.56892045318734\\
};
\addplot [color=red,solid,line width=1.2pt,forget plot]
  table[row sep=crcr]{%
-5.49636984938369	-5.37727179935781\\
-5.43928691869243	-5.56895262527343\\
};
\addplot [color=red,solid,line width=1.2pt,forget plot]
  table[row sep=crcr]{%
-5.49636056164902	-5.37726086748763\\
-5.43928613504151	-5.56894422573091\\
};
\addplot [color=red,solid,line width=1.2pt,forget plot]
  table[row sep=crcr]{%
-5.49625476641383	-5.37722760107159\\
-5.43919238967001	-5.56891454679389\\
};
\addplot [color=red,solid,line width=1.2pt,forget plot]
  table[row sep=crcr]{%
-5.49620723440626	-5.37724164214891\\
-5.43916734231392	-5.56893527978484\\
};
\addplot [color=red,solid,line width=1.2pt,forget plot]
  table[row sep=crcr]{%
-5.49621357196378	-5.37725414048752\\
-5.43917999772016	-5.5689496579338\\
};
\addplot [color=red,solid,line width=1.2pt,forget plot]
  table[row sep=crcr]{%
-5.49617004198528	-5.37725299898946\\
-5.43915347876469	-5.56895357676249\\
};
\addplot [color=red,solid,line width=1.2pt,forget plot]
  table[row sep=crcr]{%
-5.4961382929632	-5.37727316149312\\
-5.43914620694651	-5.56898101770008\\
};
\addplot [color=red,solid,line width=1.2pt,forget plot]
  table[row sep=crcr]{%
-5.49613984314913	-5.37727704219662\\
-5.43916857887716	-5.56899108718878\\
};
\addplot [color=red,solid,line width=1.2pt,forget plot]
  table[row sep=crcr]{%
-5.49610961912168	-5.377301604005\\
-5.43917151061518	-5.56902549870699\\
};
\addplot [color=red,solid,line width=1.2pt,forget plot]
  table[row sep=crcr]{%
-5.49605552832217	-5.37729864438994\\
-5.4391399194	-5.56902921957559\\
};
\addplot [color=red,solid,line width=1.2pt,forget plot]
  table[row sep=crcr]{%
-5.49594604216122	-5.37726646041241\\
-5.439130425637	-5.56902669017597\\
};
\addplot [color=red,solid,line width=1.2pt,forget plot]
  table[row sep=crcr]{%
-5.8005189961269	-4.35720730376648\\
-5.74332812199356	-4.54885595081588\\
};
\addplot [color=red,solid,line width=1.2pt,forget plot]
  table[row sep=crcr]{%
-5.80034644015642	-4.35723882328402\\
-5.74319546866001	-4.54889937336415\\
};
\addplot [color=red,solid,line width=1.2pt,forget plot]
  table[row sep=crcr]{%
-5.79989358188771	-4.35806187673049\\
-5.74265884246256	-4.54969742824152\\
};
\addplot [color=red,solid,line width=1.2pt,forget plot]
  table[row sep=crcr]{%
-5.79990557676975	-4.35804953646097\\
-5.74269611175777	-4.5496926347271\\
};
\addplot [color=red,solid,line width=1.2pt,forget plot]
  table[row sep=crcr]{%
-5.79988122198054	-4.35797406803259\\
-5.74267697452769	-4.54961872377168\\
};
\addplot [color=red,solid,line width=1.2pt,forget plot]
  table[row sep=crcr]{%
-5.79989651115963	-4.35779245529079\\
-5.74265885691806	-4.54942713622543\\
};
\addplot [color=red,solid,line width=1.2pt,forget plot]
  table[row sep=crcr]{%
-5.79991565828766	-4.35762676631195\\
-5.74267697577008	-4.54926114011699\\
};
\addplot [color=red,solid,line width=1.2pt,forget plot]
  table[row sep=crcr]{%
-5.79988045195375	-4.35756437656467\\
-5.74265114472013	-4.5492015503947\\
};
\addplot [color=red,solid,line width=1.2pt,forget plot]
  table[row sep=crcr]{%
-5.79982760545595	-4.35756344325485\\
-5.74261601127776	-4.54920590590754\\
};
\addplot [color=red,solid,line width=1.2pt,forget plot]
  table[row sep=crcr]{%
-5.79975253582727	-4.35753588169107\\
-5.74255084587714	-4.54918130080368\\
};
\addplot [color=red,solid,line width=1.2pt,forget plot]
  table[row sep=crcr]{%
-5.79969895943629	-4.35749405996998\\
-5.74252321118684	-4.54914722016338\\
};
\addplot [color=red,solid,line width=1.2pt,forget plot]
  table[row sep=crcr]{%
-5.79968360724336	-4.35747706587573\\
-5.74251599993876	-4.549132654563\\
};
\addplot [color=red,solid,line width=1.2pt,forget plot]
  table[row sep=crcr]{%
-5.79971457827167	-4.35750699501685\\
-5.74255089680998	-4.54916375467261\\
};
\addplot [color=red,solid,line width=1.2pt,forget plot]
  table[row sep=crcr]{%
-5.79969593737782	-4.35746116237674\\
-5.74255440741759	-4.54912452756113\\
};
\addplot [color=red,solid,line width=1.2pt,forget plot]
  table[row sep=crcr]{%
-5.79969410792734	-4.35742922932434\\
-5.74256491304886	-4.54909627159436\\
};
\addplot [color=red,solid,line width=1.2pt,forget plot]
  table[row sep=crcr]{%
-5.79971175981249	-4.35745307313846\\
-5.74260908884801	-4.54912801925485\\
};
\addplot [color=red,solid,line width=1.2pt,forget plot]
  table[row sep=crcr]{%
-5.79971606191595	-4.35744481019571\\
-5.74261944619115	-4.54912156014931\\
};
\addplot [color=red,solid,line width=1.2pt,forget plot]
  table[row sep=crcr]{%
-5.79971527490004	-4.35743703573444\\
-5.74262702885664	-4.5491162786474\\
};
\addplot [color=red,solid,line width=1.2pt,forget plot]
  table[row sep=crcr]{%
-5.79975707472067	-4.35743455101877\\
-5.74267728499505	-4.54911631229219\\
};
\addplot [color=red,solid,line width=1.2pt,forget plot]
  table[row sep=crcr]{%
-5.79974046043009	-4.35746920069501\\
-5.7426955067867	-4.54916133215539\\
};
\addplot [color=red,solid,line width=1.2pt,forget plot]
  table[row sep=crcr]{%
-5.79971299431574	-4.35745003315192\\
-5.74269273064808	-4.54914951028024\\
};
\addplot [color=red,solid,line width=1.2pt,forget plot]
  table[row sep=crcr]{%
-5.79973598697605	-4.35745351600336\\
-5.74273471815228	-4.5491586420498\\
};
\addplot [color=red,solid,line width=1.2pt,forget plot]
  table[row sep=crcr]{%
-5.79970332844237	-4.35742695145188\\
-5.74271263847005	-4.54913522267793\\
};
\addplot [color=red,solid,line width=1.2pt,forget plot]
  table[row sep=crcr]{%
-5.7996360892021	-4.35739317917558\\
-5.7426603004963	-4.54910587959303\\
};
\addplot [color=red,solid,line width=1.2pt,forget plot]
  table[row sep=crcr]{%
-5.79966628806817	-4.35738069129516\\
-5.74268683487688	-4.54909230261284\\
};
\addplot [color=red,solid,line width=1.2pt,forget plot]
  table[row sep=crcr]{%
-5.79963220632006	-4.35740283038463\\
-5.74269613126711	-4.54912732896936\\
};
\addplot [color=red,solid,line width=1.2pt,forget plot]
  table[row sep=crcr]{%
-5.79951230925879	-4.35748614329488\\
-5.74270567343213	-4.54924903367271\\
};
\addplot [color=red,solid,line width=1.2pt,forget plot]
  table[row sep=crcr]{%
-6.03708311355265	-3.54265296350567\\
-5.9800277405785	-3.73434199400763\\
};
\addplot [color=red,solid,line width=1.2pt,forget plot]
  table[row sep=crcr]{%
-6.03694603649151	-3.54264494512629\\
-5.97992755419894	-3.73434495210869\\
};
\addplot [color=red,solid,line width=1.2pt,forget plot]
  table[row sep=crcr]{%
-6.03699530373915	-3.54263930171975\\
-5.97997425283126	-3.73433854468476\\
};
\addplot [color=red,solid,line width=1.2pt,forget plot]
  table[row sep=crcr]{%
-6.03845018891307	-3.53793938699972\\
-5.98108959872153	-3.72953730633309\\
};
\addplot [color=red,solid,line width=1.2pt,forget plot]
  table[row sep=crcr]{%
-6.03843254333033	-3.53770865214203\\
-5.98120367716587	-3.7293459576896\\
};
\addplot [color=red,solid,line width=1.2pt,forget plot]
  table[row sep=crcr]{%
-6.03843247695291	-3.53776934369636\\
-5.98121963034112	-3.72941143245187\\
};
\addplot [color=red,solid,line width=1.2pt,forget plot]
  table[row sep=crcr]{%
-6.03846822110657	-3.53778647451278\\
-5.98123508587818	-3.72942250513314\\
};
\addplot [color=red,solid,line width=1.2pt,forget plot]
  table[row sep=crcr]{%
-6.0384573581897	-3.53777292742291\\
-5.98121347505896	-3.72940574779592\\
};
\addplot [color=red,solid,line width=1.2pt,forget plot]
  table[row sep=crcr]{%
-6.03845102628901	-3.53776788514808\\
-5.9811637092167	-3.72938772572483\\
};
\addplot [color=red,solid,line width=1.2pt,forget plot]
  table[row sep=crcr]{%
-6.03840526277167	-3.53778684933573\\
-5.9811375926724	-3.72941256254093\\
};
\addplot [color=red,solid,line width=1.2pt,forget plot]
  table[row sep=crcr]{%
-6.03845684417342	-3.53787153401891\\
-5.98113315526017	-3.72948049698614\\
};
\addplot [color=red,solid,line width=1.2pt,forget plot]
  table[row sep=crcr]{%
-6.03847076647595	-3.53790465433893\\
-5.98112645979361	-3.72950744787577\\
};
\addplot [color=red,solid,line width=1.2pt,forget plot]
  table[row sep=crcr]{%
-6.03850143087536	-3.53792697902316\\
-5.98109351522638	-3.72951072369071\\
};
\addplot [color=red,solid,line width=1.2pt,forget plot]
  table[row sep=crcr]{%
-6.03854236187619	-3.53797194163052\\
-5.98108509472644	-3.72954089123271\\
};
\addplot [color=red,solid,line width=1.2pt,forget plot]
  table[row sep=crcr]{%
-6.03853526640814	-3.5379886943295\\
-5.98104927429572	-3.72954902610738\\
};
\addplot [color=red,solid,line width=1.2pt,forget plot]
  table[row sep=crcr]{%
-6.03853932772687	-3.53803270304673\\
-5.98100835417623	-3.72957953043448\\
};
\addplot [color=red,solid,line width=1.2pt,forget plot]
  table[row sep=crcr]{%
-6.03855070648098	-3.53804475089476\\
-5.98102592749901	-3.72959343870826\\
};
\addplot [color=red,solid,line width=1.2pt,forget plot]
  table[row sep=crcr]{%
-6.03857932527315	-3.53811248697549\\
-5.9810383416677	-3.7296563075642\\
};
\addplot [color=red,solid,line width=1.2pt,forget plot]
  table[row sep=crcr]{%
-6.0385700173346	-3.53812647958291\\
-5.98102338570002	-3.7296686033769\\
};
\addplot [color=red,solid,line width=1.2pt,forget plot]
  table[row sep=crcr]{%
-6.03856829393403	-3.53815680619753\\
-5.98101377933958	-3.72969656146989\\
};
\addplot [color=red,solid,line width=1.2pt,forget plot]
  table[row sep=crcr]{%
-6.03859460669606	-3.53818537268695\\
-5.98103609293176	-3.72972392622969\\
};
\addplot [color=red,solid,line width=1.2pt,forget plot]
  table[row sep=crcr]{%
-6.03854775754926	-3.53817120019811\\
-5.98097741196051	-3.72970619780636\\
};
\addplot [color=red,solid,line width=1.2pt,forget plot]
  table[row sep=crcr]{%
-6.24548543157721	-2.72228327317993\\
-6.188436666443	-2.91397427035158\\
};
\addplot [color=red,solid,line width=1.2pt,forget plot]
  table[row sep=crcr]{%
-6.24553278082112	-2.72222348604173\\
-6.18845658970872	-2.91390631888834\\
};
\addplot [color=red,solid,line width=1.2pt,forget plot]
  table[row sep=crcr]{%
-6.24549016961628	-2.72231263664429\\
-6.18841694128649	-2.91399635167511\\
};
\addplot [color=red,solid,line width=1.2pt,forget plot]
  table[row sep=crcr]{%
-6.2455621935134	-2.72235267791357\\
-6.18848166510427	-2.9140342192172\\
};
\addplot [color=red,solid,line width=1.2pt,forget plot]
  table[row sep=crcr]{%
-6.24559929924251	-2.72238158397293\\
-6.18849963022906	-2.91405742438729\\
};
\addplot [color=red,solid,line width=1.2pt,forget plot]
  table[row sep=crcr]{%
-6.245641726457	-2.72235271768835\\
-6.18854564803467	-2.91402962769263\\
};
\addplot [color=red,solid,line width=1.2pt,forget plot]
  table[row sep=crcr]{%
-6.24566434250815	-2.72231770580952\\
-6.18860441298808	-2.91400538001724\\
};
\addplot [color=red,solid,line width=1.2pt,forget plot]
  table[row sep=crcr]{%
-6.24569464224661	-2.7223258245763\\
-6.18864641135647	-2.91401698074242\\
};
\addplot [color=red,solid,line width=1.2pt,forget plot]
  table[row sep=crcr]{%
-6.24566139282213	-2.72237245643391\\
-6.1886431948305	-2.91407254797741\\
};
\addplot [color=red,solid,line width=1.2pt,forget plot]
  table[row sep=crcr]{%
-6.2456716583737	-2.72235053949922\\
-6.18867563598336	-2.91405722543292\\
};
\addplot [color=red,solid,line width=1.2pt,forget plot]
  table[row sep=crcr]{%
-6.24567203999328	-2.72236110239431\\
-6.18868582670132	-2.91407070438295\\
};
\addplot [color=red,solid,line width=1.2pt,forget plot]
  table[row sep=crcr]{%
-6.24563803911133	-2.72236021728372\\
-6.18864518923468	-2.91406784640405\\
};
\addplot [color=red,solid,line width=1.2pt,forget plot]
  table[row sep=crcr]{%
-6.24563483705869	-2.72233834408104\\
-6.18862672809002	-2.91404143615787\\
};
\addplot [color=red,solid,line width=1.2pt,forget plot]
  table[row sep=crcr]{%
-5.87691905379923	-1.79389342138702\\
-6.06819828339422	-1.85230794291673\\
};
\addplot [color=red,solid,line width=1.2pt,forget plot]
  table[row sep=crcr]{%
-5.8769070747584	-1.79375805864819\\
-6.06819326539233	-1.85214978128727\\
};
\addplot [color=red,solid,line width=1.2pt,forget plot]
  table[row sep=crcr]{%
-5.87666183516648	-1.79356793740244\\
-6.06795819339539	-1.85192634149607\\
};
\addplot [color=red,solid,line width=1.2pt,forget plot]
  table[row sep=crcr]{%
-5.87616137976573	-1.79336451881134\\
-6.06744401084976	-1.85176790095288\\
};
\addplot [color=red,solid,line width=1.2pt,forget plot]
  table[row sep=crcr]{%
-5.87598140258847	-1.79332839144284\\
-6.06725776051603	-1.85175231546346\\
};
\addplot [color=red,solid,line width=1.2pt,forget plot]
  table[row sep=crcr]{%
-5.87597924064313	-1.79336767503715\\
-6.06725609705802	-1.8517899670158\\
};
\addplot [color=red,solid,line width=1.2pt,forget plot]
  table[row sep=crcr]{%
-5.875949548456	-1.7933652757865\\
-6.06722301789352	-1.85179865571043\\
};
\addplot [color=red,solid,line width=1.2pt,forget plot]
  table[row sep=crcr]{%
-5.87598436677686	-1.79334173546986\\
-6.06725241725393	-1.85179285065767\\
};
\addplot [color=red,solid,line width=1.2pt,forget plot]
  table[row sep=crcr]{%
-5.87596891250316	-1.79335263688802\\
-6.06722840734079	-1.85183174114289\\
};
\addplot [color=red,solid,line width=1.2pt,forget plot]
  table[row sep=crcr]{%
-5.87595227998569	-1.79335896270176\\
-6.06719884490325	-1.8518803383455\\
};
\addplot [color=red,solid,line width=1.2pt,forget plot]
  table[row sep=crcr]{%
-5.87597638142641	-1.79339159677351\\
-6.06721807198024	-1.85192889936008\\
};
\addplot [color=red,solid,line width=1.2pt,forget plot]
  table[row sep=crcr]{%
-5.87595571001762	-1.79338528528192\\
-6.06718645320523	-1.85195834103209\\
};
\addplot [color=red,solid,line width=1.2pt,forget plot]
  table[row sep=crcr]{%
-5.87599541858679	-1.79334436937248\\
-6.06720631513739	-1.8519821817491\\
};
\addplot [color=red,solid,line width=1.2pt,forget plot]
  table[row sep=crcr]{%
-5.87603859753112	-1.79334385661663\\
-6.06724050886691	-1.85201096070008\\
};
\addplot [color=red,solid,line width=1.2pt,forget plot]
  table[row sep=crcr]{%
-5.87606801581939	-1.7933172828343\\
-6.06726577917817	-1.85199790388247\\
};
\addplot [color=red,solid,line width=1.2pt,forget plot]
  table[row sep=crcr]{%
-5.87608462693836	-1.79328634589927\\
-6.06727087942163	-1.85200445947084\\
};
\addplot [color=red,solid,line width=1.2pt,forget plot]
  table[row sep=crcr]{%
-5.87610654708541	-1.7933006272405\\
-6.06728740497333	-1.85203630274728\\
};
\addplot [color=red,solid,line width=1.2pt,forget plot]
  table[row sep=crcr]{%
-5.87610887622293	-1.7932780290316\\
-6.0672825429776	-1.852037106114\\
};
\addplot [color=red,solid,line width=1.2pt,forget plot]
  table[row sep=crcr]{%
-5.87611934736073	-1.79327786733991\\
-6.0672927734012	-1.85203772758439\\
};
\addplot [color=red,solid,line width=1.2pt,forget plot]
  table[row sep=crcr]{%
-4.14177980961902	-1.26221328458549\\
-4.33312461901523	-1.32041263094889\\
};
\addplot [color=red,solid,line width=1.2pt,forget plot]
  table[row sep=crcr]{%
-4.14180487466164	-1.2622118436602\\
-4.33313660703043	-1.32045416666297\\
};
\addplot [color=red,solid,line width=1.2pt,forget plot]
  table[row sep=crcr]{%
-4.14183192037879	-1.26221607129482\\
-4.33315688023955	-1.32048063800804\\
};
\addplot [color=red,solid,line width=1.2pt,forget plot]
  table[row sep=crcr]{%
-4.14188096788694	-1.26222965762824\\
-4.33320172710578	-1.32050801632104\\
};
\addplot [color=red,solid,line width=1.2pt,forget plot]
  table[row sep=crcr]{%
-4.14191722076249	-1.26222934070331\\
-4.33323705456552	-1.32051073733706\\
};
\addplot [color=red,solid,line width=1.2pt,forget plot]
  table[row sep=crcr]{%
-4.14192611255018	-1.2622740695602\\
-4.33324828005085	-1.32054780483473\\
};
\addplot [color=red,solid,line width=1.2pt,forget plot]
  table[row sep=crcr]{%
-4.14193348171024	-1.26230986762083\\
-4.33326372250727	-1.32055709035764\\
};
\addplot [color=red,solid,line width=1.2pt,forget plot]
  table[row sep=crcr]{%
-4.14197130059466	-1.26226026746719\\
-4.33330579949236	-1.32049350137676\\
};
\addplot [color=red,solid,line width=1.2pt,forget plot]
  table[row sep=crcr]{%
-4.14200014460949	-1.26220337073935\\
-4.33334152305748	-1.32041399600333\\
};
\addplot [color=red,solid,line width=1.2pt,forget plot]
  table[row sep=crcr]{%
-4.14201244836325	-1.26219819419945\\
-4.33336497125881	-1.32037217447847\\
};
\addplot [color=red,solid,line width=1.2pt,forget plot]
  table[row sep=crcr]{%
-4.14202948816208	-1.2621836904748\\
-4.33338337676291	-1.32035317833004\\
};
\addplot [color=red,solid,line width=1.2pt,forget plot]
  table[row sep=crcr]{%
-4.14205199933916	-1.26219707215684\\
-4.33341089959101	-1.32035007117154\\
};
\addplot [color=red,solid,line width=1.2pt,forget plot]
  table[row sep=crcr]{%
-4.14204447849842	-1.2621869893698\\
-4.33340722365935	-1.32032733478047\\
};
\addplot [color=red,solid,line width=1.2pt,forget plot]
  table[row sep=crcr]{%
-4.14206834696778	-1.26218109342219\\
-4.33343745790449	-1.32030048243562\\
};
\addplot [color=red,solid,line width=1.2pt,forget plot]
  table[row sep=crcr]{%
-4.14207259822446	-1.26218116399235\\
-4.33344932552852	-1.32027546875958\\
};
\addplot [color=red,solid,line width=1.2pt,forget plot]
  table[row sep=crcr]{%
-4.14208102825887	-1.26217578146751\\
-4.3334547263154	-1.32028006437479\\
};
\addplot [color=red,solid,line width=1.2pt,forget plot]
  table[row sep=crcr]{%
-4.14210297537367	-1.26215959278129\\
-4.33347931460413	-1.32025517594169\\
};
\addplot [color=red,solid,line width=1.2pt,forget plot]
  table[row sep=crcr]{%
-4.1421465415666	-1.26212314967118\\
-4.33352613087874	-1.3202080254541\\
};
\addplot [color=red,solid,line width=1.2pt,forget plot]
  table[row sep=crcr]{%
-2.60084962758494	-0.805342933208116\\
-2.79211593719603	-0.863799744682542\\
};
\addplot [color=red,solid,line width=1.2pt,forget plot]
  table[row sep=crcr]{%
-2.60080924433171	-0.805446487708351\\
-2.79209569306987	-0.863837364813808\\
};
\addplot [color=red,solid,line width=1.2pt,forget plot]
  table[row sep=crcr]{%
-2.60080260149506	-0.805487841015705\\
-2.79209641367825	-0.863854590286366\\
};
\addplot [color=red,solid,line width=1.2pt,forget plot]
  table[row sep=crcr]{%
-2.60085180446007	-0.805482319012892\\
-2.79215090335896	-0.863831738551338\\
};
\addplot [color=red,solid,line width=1.2pt,forget plot]
  table[row sep=crcr]{%
-2.60089469285239	-0.805494612204315\\
-2.79219486695633	-0.863840506556871\\
};
\addplot [color=red,solid,line width=1.2pt,forget plot]
  table[row sep=crcr]{%
-2.60080336691282	-0.80556467956525\\
-2.79214411014952	-0.863777392761141\\
};
\addplot [color=red,solid,line width=1.2pt,forget plot]
  table[row sep=crcr]{%
-2.60079389737204	-0.805529039087841\\
-2.79212555738544	-0.863771599784596\\
};
\addplot [color=red,solid,line width=1.2pt,forget plot]
  table[row sep=crcr]{%
-2.60081468328879	-0.805498613617959\\
-2.79214407090884	-0.86374863879336\\
};
\addplot [color=red,solid,line width=1.2pt,forget plot]
  table[row sep=crcr]{%
-2.60081963301712	-0.805531084594\\
-2.79215859074311	-0.863749666304078\\
};
\addplot [color=red,solid,line width=1.2pt,forget plot]
  table[row sep=crcr]{%
-2.60081685824972	-0.80554086734571\\
-2.79217161720052	-0.863707492027571\\
};
\addplot [color=red,solid,line width=1.2pt,forget plot]
  table[row sep=crcr]{%
-2.60080002672064	-0.805570217154702\\
-2.79216313569297	-0.863709365105472\\
};
\addplot [color=red,solid,line width=1.2pt,forget plot]
  table[row sep=crcr]{%
-2.60078109579149	-0.805612977949965\\
-2.7921612575987	-0.863695967427506\\
};
\addplot [color=red,solid,line width=1.2pt,forget plot]
  table[row sep=crcr]{%
-2.60075912009354	-0.805656463069209\\
-2.79214262513525	-0.863728435637321\\
};
\addplot [color=red,solid,line width=1.2pt,forget plot]
  table[row sep=crcr]{%
-2.60073811010843	-0.805653959346988\\
-2.79212437924506	-0.863716821711504\\
};
\addplot [color=red,solid,line width=1.2pt,forget plot]
  table[row sep=crcr]{%
-2.60075732260315	-0.805630591225667\\
-2.79214085779533	-0.863702464428992\\
};
\addplot [color=red,solid,line width=1.2pt,forget plot]
  table[row sep=crcr]{%
-2.60076198790357	-0.805616532844702\\
-2.79214886356363	-0.86367739595471\\
};
\addplot [color=red,solid,line width=1.2pt,forget plot]
  table[row sep=crcr]{%
-2.60075227187224	-0.805686194938763\\
-2.79215479207808	-0.863695463671034\\
};
\addplot [color=red,solid,line width=1.2pt,forget plot]
  table[row sep=crcr]{%
-2.6007491585158	-0.805715462205455\\
-2.7921643742599	-0.863682825278487\\
};
\addplot [color=red,solid,line width=1.2pt,forget plot]
  table[row sep=crcr]{%
-2.60075150653743	-0.805715691080662\\
-2.79217209897048	-0.863665296709835\\
};
\addplot [color=red,solid,line width=1.2pt,forget plot]
  table[row sep=crcr]{%
-2.60072802071299	-0.805782132117014\\
-2.79217238708533	-0.863653148906889\\
};
\addplot [color=red,solid,line width=1.2pt,forget plot]
  table[row sep=crcr]{%
-1.9078385341836	-0.596923481076589\\
-2.09910376464486	-0.65538382333184\\
};
\addplot [color=red,solid,line width=1.2pt,forget plot]
  table[row sep=crcr]{%
-1.90779753386629	-0.596907284646544\\
-2.09907613971635	-0.655323848599385\\
};
\addplot [color=red,solid,line width=1.2pt,forget plot]
  table[row sep=crcr]{%
-1.90777854658952	-0.596937360018682\\
-2.0990637062249	-0.655332460012321\\
};
\addplot [color=red,solid,line width=1.2pt,forget plot]
  table[row sep=crcr]{%
-1.90777964477982	-0.596947644368509\\
-2.09906733490785	-0.655334454569808\\
};
\addplot [color=red,solid,line width=1.2pt,forget plot]
  table[row sep=crcr]{%
-1.90777551547104	-0.596915895720791\\
-2.09906553511052	-0.655295073398501\\
};
\addplot [color=red,solid,line width=1.2pt,forget plot]
  table[row sep=crcr]{%
-1.90777127564403	-0.596919528524124\\
-2.09906625938862	-0.655282437928627\\
};
\addplot [color=red,solid,line width=1.2pt,forget plot]
  table[row sep=crcr]{%
-1.90777691333912	-0.596949075534335\\
-2.09907625187145	-0.655297709426589\\
};
\addplot [color=red,solid,line width=1.2pt,forget plot]
  table[row sep=crcr]{%
-1.90775272077581	-0.596980256295585\\
-2.09905335452556	-0.655324643574427\\
};
\addplot [color=red,solid,line width=1.2pt,forget plot]
  table[row sep=crcr]{%
-1.90775352140466	-0.5969903603774\\
-2.09905542159145	-0.655330595078026\\
};
\addplot [color=red,solid,line width=1.2pt,forget plot]
  table[row sep=crcr]{%
-1.90777923312261	-0.596995921675084\\
-2.09908483903217	-0.65532400365606\\
};
\addplot [color=red,solid,line width=1.2pt,forget plot]
  table[row sep=crcr]{%
-1.90778297607088	-0.597017863962268\\
-2.09909714652872	-0.655317848370752\\
};
\addplot [color=red,solid,line width=1.2pt,forget plot]
  table[row sep=crcr]{%
-1.90778928160097	-0.597014590377973\\
-2.09910347508951	-0.655314499210059\\
};
\addplot [color=red,solid,line width=1.2pt,forget plot]
  table[row sep=crcr]{%
-1.90779864034416	-0.596963510399305\\
-2.09911592760388	-0.655253265893577\\
};
\addplot [color=red,solid,line width=1.2pt,forget plot]
  table[row sep=crcr]{%
-1.9078311074077	-0.596911756105664\\
-2.09914794012935	-0.655203003453404\\
};
\addplot [color=red,solid,line width=1.2pt,forget plot]
  table[row sep=crcr]{%
-1.9078270304753	-0.596917799756295\\
-2.09914462615588	-0.655206542949144\\
};
\addplot [color=red,solid,line width=1.2pt,forget plot]
  table[row sep=crcr]{%
-1.90781789846474	-0.596951065733778\\
-2.09913871720205	-0.655229229034302\\
};
\addplot [color=red,solid,line width=1.2pt,forget plot]
  table[row sep=crcr]{%
-1.9078426534741	-0.596880711745149\\
-2.09916987742777	-0.655137843275832\\
};
\addplot [color=red,solid,line width=1.2pt,forget plot]
  table[row sep=crcr]{%
-1.90783687083536	-0.596865967541851\\
-2.09915659315949	-0.655147730125173\\
};
\addplot [color=red,solid,line width=1.2pt,forget plot]
  table[row sep=crcr]{%
-0.954182888779111	-0.299168875478846\\
-1.14547657419724	-0.357536040213408\\
};
\addplot [color=red,solid,line width=1.2pt,forget plot]
  table[row sep=crcr]{%
-0.954205600686339	-0.299189297370718\\
-1.14549817740994	-0.357560095638742\\
};
\addplot [color=red,solid,line width=1.2pt,forget plot]
  table[row sep=crcr]{%
-0.954114644807212	-0.299069577729956\\
-1.14540639830866	-0.357443073790591\\
};
\addplot [color=red,solid,line width=1.2pt,forget plot]
  table[row sep=crcr]{%
-0.954129951521545	-0.299029144628173\\
-1.14542209943298	-0.35740134817944\\
};
\addplot [color=red,solid,line width=1.2pt,forget plot]
  table[row sep=crcr]{%
-0.954127174220883	-0.29907798270791\\
-1.14541968465462	-0.35744899822038\\
};
\addplot [color=red,solid,line width=1.2pt,forget plot]
  table[row sep=crcr]{%
-0.954136059227195	-0.299104031367782\\
-1.14542849034859	-0.357475306800697\\
};
\addplot [color=red,solid,line width=1.2pt,forget plot]
  table[row sep=crcr]{%
-0.954164591448214	-0.299108095580081\\
-1.14546119157638	-0.35746570673239\\
};
\addplot [color=red,solid,line width=1.2pt,forget plot]
  table[row sep=crcr]{%
-0.954183582252558	-0.299132322256385\\
-1.14549356820167	-0.357446036562278\\
};
\addplot [color=red,solid,line width=1.2pt,forget plot]
  table[row sep=crcr]{%
-0.954181741833527	-0.299145548693145\\
-1.14549027796528	-0.357464019207423\\
};
\addplot [color=red,solid,line width=1.2pt,forget plot]
  table[row sep=crcr]{%
-0.954220681864436	-0.299072481912237\\
-1.14553200705443	-0.35738180238545\\
};
\addplot [color=red,solid,line width=1.2pt,forget plot]
  table[row sep=crcr]{%
-0.954235101264655	-0.299069872536668\\
-1.1455480378124	-0.357373905926131\\
};
\addplot [color=red,solid,line width=1.2pt,forget plot]
  table[row sep=crcr]{%
-0.954254318570722	-0.29905153780097\\
-1.14556781376117	-0.357353738085465\\
};
\addplot [color=red,solid,line width=1.2pt,forget plot]
  table[row sep=crcr]{%
-0.954285055031727	-0.299054039480061\\
-1.14559955838958	-0.357352931450144\\
};
\addplot [color=red,solid,line width=1.2pt,forget plot]
  table[row sep=crcr]{%
-0.954301310817259	-0.299084882288078\\
-1.14561132135449	-0.357398515927585\\
};
\addplot [color=red,solid,line width=1.2pt,forget plot]
  table[row sep=crcr]{%
-0.954329434644527	-0.299051442042071\\
-1.1456404502172	-0.3573617783518\\
};
\addplot [color=red,solid,line width=1.2pt,forget plot]
  table[row sep=crcr]{%
-0.954338797212607	-0.299042767216656\\
-1.14565270168379	-0.357343624465851\\
};
\addplot [color=red,solid,line width=1.2pt,forget plot]
  table[row sep=crcr]{%
-0.954347440246023	-0.299067607878249\\
-1.14565992811298	-0.357373113502983\\
};
\addplot [color=red,solid,line width=1.2pt,forget plot]
  table[row sep=crcr]{%
-0.954114792025703	-0.298958379356081\\
-1.14542897724656	-0.357258315318988\\
};
\addplot [color=red,solid,line width=1.2pt,forget plot]
  table[row sep=crcr]{%
0.953894753058549	-1.00940475016104\\
0.895396358029211	-0.818151154694207\\
};
\addplot [color=red,solid,line width=1.2pt,forget plot]
  table[row sep=crcr]{%
0.953886549434455	-1.00941721175104\\
0.895408084826718	-0.818157521337401\\
};
\addplot [color=red,solid,line width=1.2pt,forget plot]
  table[row sep=crcr]{%
0.953893342463819	-1.00940625864359\\
0.895430585491253	-0.81814176625892\\
};
\addplot [color=red,solid,line width=1.2pt,forget plot]
  table[row sep=crcr]{%
0.953848346095661	-1.00943183582064\\
0.895391957740573	-0.81816539689191\\
};
\addplot [color=red,solid,line width=1.2pt,forget plot]
  table[row sep=crcr]{%
0.953799869504776	-1.00938834297871\\
0.895353003369247	-0.81811899405177\\
};
\addplot [color=red,solid,line width=1.2pt,forget plot]
  table[row sep=crcr]{%
0.953756687576866	-1.00940392024942\\
0.895327960169505	-0.818129029546128\\
};
\addplot [color=red,solid,line width=1.2pt,forget plot]
  table[row sep=crcr]{%
0.953732405628018	-1.00940969554317\\
0.895320641851771	-0.818129623783068\\
};
\addplot [color=red,solid,line width=1.2pt,forget plot]
  table[row sep=crcr]{%
0.953768803477467	-1.00939870567515\\
0.895359407218332	-0.818117910955335\\
};
\addplot [color=red,solid,line width=1.2pt,forget plot]
  table[row sep=crcr]{%
0.953763919381306	-1.00942519249644\\
0.895354291195568	-0.81814446859776\\
};
\addplot [color=red,solid,line width=1.2pt,forget plot]
  table[row sep=crcr]{%
0.953773265420328	-1.00944185591297\\
0.895363525941821	-0.818161165998764\\
};
\addplot [color=red,solid,line width=1.2pt,forget plot]
  table[row sep=crcr]{%
0.953777628322827	-1.00944521979103\\
0.895372468118542	-0.818163131603161\\
};
\addplot [color=red,solid,line width=1.2pt,forget plot]
  table[row sep=crcr]{%
0.953764924601682	-1.00946857163941\\
0.895357739596657	-0.818187101706256\\
};
\addplot [color=red,solid,line width=1.2pt,forget plot]
  table[row sep=crcr]{%
0.953735645684969	-1.0094722301958\\
0.895334459075245	-0.818188928774649\\
};
\addplot [color=red,solid,line width=1.2pt,forget plot]
  table[row sep=crcr]{%
0.953743017368194	-1.00950427011414\\
0.895352304460605	-0.818217771254073\\
};
\addplot [color=red,solid,line width=1.2pt,forget plot]
  table[row sep=crcr]{%
0.953713879000767	-1.00950070698307\\
0.895335377520774	-0.818210480978315\\
};
\addplot [color=red,solid,line width=1.2pt,forget plot]
  table[row sep=crcr]{%
0.953715939329663	-1.00948339150621\\
0.895353830401283	-0.818188163543173\\
};
\addplot [color=red,solid,line width=1.2pt,forget plot]
  table[row sep=crcr]{%
0.953699493498397	-1.00949507956869\\
0.895337171710172	-0.818199916547024\\
};
\addplot [color=red,solid,line width=1.2pt,forget plot]
  table[row sep=crcr]{%
0.953693647467452	-1.00951976901699\\
0.895331124261703	-0.818224667445996\\
};
\addplot [color=red,solid,line width=1.2pt,forget plot]
  table[row sep=crcr]{%
0.953758203315656	-1.00951708155415\\
0.89540214882537	-0.818220006552139\\
};
\addplot [color=red,solid,line width=1.2pt,forget plot]
  table[row sep=crcr]{%
0.953753023117638	-1.0095517813728\\
0.895411481911331	-0.818250279628375\\
};
\end{axis}
\end{tikzpicture}%\\ \vspace*{1em}
    {\footnotesize {(a)~Measurement \#1}} \\
  \end{minipage}%
  %
  \begin{minipage}{.49\columnwidth}

    \tikzsetnextfilename{tikz-roomscan-2}
    %

    \setlength{\figurewidth}{1.15\columnwidth}
    \setlength{\figureheight}{0.7505\figurewidth}
    %
    \centering
    % This file was created by matlab2tikz.
%
%The latest updates can be retrieved from
%  http://www.mathworks.com/matlabcentral/fileexchange/22022-matlab2tikz-matlab2tikz
%where you can also make suggestions and rate matlab2tikz.
%
\begin{tikzpicture}

\begin{axis}[%
width=0.694\figurewidth,
height=\figureheight,
at={(0\figurewidth,0\figureheight)},
scale only axis,
xmin=-5.11653877330595,
xmax=4.21731973252433,
ymin=-9.55650052172439,
ymax=0.537534543793351,
hide axis,
axis x line*=bottom,
axis y line*=left
]
\addplot [color=white!50!blue,solid,forget plot]
  table[row sep=crcr]{%
-1.0845094939447e-07	-4.69013673921806e-06\\
-4.5025631541665e-05	7.89038681676387e-05\\
-8.66840316665647e-05	7.75643416763777e-05\\
-0.000106261566991335	7.78824993474689e-05\\
-0.000105641459581046	7.10899578312942e-05\\
-0.000190179459542946	0.000160144686751866\\
-0.000189047895359037	0.000213707526442867\\
-0.000237490701802056	0.000232197825491909\\
-0.000265102566238762	0.000258094464798419\\
-0.000268191279401646	0.000229548837163551\\
-0.000315455263206736	0.000194086540976717\\
-0.000352655833436704	0.000216807205863249\\
-0.000429165432560625	0.000211936359477502\\
-0.000474315540183687	0.000150641328478139\\
-0.000467312375999315	0.000106136728585754\\
-0.000557045354565887	0.000121931459036191\\
-0.00063728739273045	9.30627152551189e-05\\
-0.000664754011622885	5.22200179466066e-05\\
-0.000706401272943509	2.52617842403309e-05\\
-0.000767889397934542	-2.28212520975189e-05\\
-0.00081875203414985	-4.66315613353271e-05\\
-0.00091194211830687	-3.71427690900849e-05\\
-0.000945456504787495	-2.37746759462622e-05\\
-0.000992167695057343	-2.03460087504504e-05\\
-0.000917418176624676	1.12288798598743e-05\\
0.00473070172372042	0.00141724315893301\\
0.0196867219816721	0.00405339052382379\\
0.036217173497543	-0.00790758584091455\\
0.0650802218432041	-0.0377967374318233\\
0.1078720669014	-0.072654103654428\\
0.164185995043526	-0.12447760950853\\
0.237023551939821	-0.193389512562798\\
0.344686304348216	-0.27667443273648\\
0.469933523303147	-0.3902412842939\\
0.600395700397094	-0.543454067863816\\
0.732595655950097	-0.730373096416718\\
0.845320212307866	-0.891740761210319\\
0.949797373333972	-1.01793101557941\\
1.07620607104888	-1.12965713609728\\
1.23164013715837	-1.22981735688014\\
1.38204362981913	-1.32362967375412\\
1.49531626878177	-1.40135074475518\\
1.59915940900102	-1.46837771145708\\
1.7081802835299	-1.54213888626628\\
1.82839559981185	-1.59839890852276\\
1.94895477197297	-1.61933873514238\\
2.08045054492035	-1.58789640827224\\
2.21476179521977	-1.51685376823527\\
2.32822531566931	-1.44545081617062\\
2.4236597187588	-1.38198257760629\\
2.5009314949609	-1.31904216354006\\
2.55710909480917	-1.26076088308768\\
2.5972844388212	-1.2103581156\\
2.62806185410918	-1.16259769161135\\
2.65415228209222	-1.12188746518879\\
2.67118185519947	-1.09073702138568\\
2.6837550079973	-1.06377654097039\\
2.69248893864002	-1.04414532770111\\
2.69761799177434	-1.03135270974452\\
2.6999656443158	-1.02455413903276\\
2.70023665580449	-1.02339953596701\\
2.70013648147481	-1.02339466579114\\
2.70000702581906	-1.0234376059427\\
2.69994653007453	-1.02350342425396\\
2.69954551330609	-1.02368855869035\\
2.69763943243404	-1.02433421722771\\
2.69437127108888	-1.02546223230046\\
2.69188234857918	-1.02630829473458\\
2.69031021667742	-1.0268792656565\\
2.68927329608691	-1.02727271438986\\
2.68842991738501	-1.02760049223299\\
2.68768485131909	-1.02791689520351\\
2.6870242825006	-1.02821595728613\\
2.68667126075198	-1.02837158322332\\
2.6864577184736	-1.02843902556152\\
2.68621896781131	-1.02851524669503\\
2.68609662209887	-1.02864802780284\\
2.68526084176117	-1.03097496270755\\
2.67226522028994	-1.0473245970002\\
2.62026517674523	-1.08882271050714\\
2.54140608450213	-1.16462116657765\\
2.45779991804525	-1.25673872205458\\
2.40921719688069	-1.36562532312634\\
2.36777455465199	-1.47362159053833\\
2.33099829408951	-1.5475023461504\\
2.30011545488694	-1.59840227951706\\
2.27626414548027	-1.6375532598715\\
2.26537395037815	-1.67892248984167\\
2.26471819026498	-1.73911693239151\\
2.26932948076471	-1.8504712431202\\
2.27397492562231	-2.02875317671548\\
2.27526828883633	-2.25090151858972\\
2.2829500539869	-2.46147102254154\\
2.34042727114776	-2.66137191266566\\
2.44094412169302	-2.84613372411952\\
2.5593865926307	-2.98074503675612\\
2.69248096212947	-3.07256373115602\\
2.8320488226518	-3.13913602374471\\
2.94333182365358	-3.16383145827246\\
3.00912253767297	-3.16369809491989\\
3.04774014056703	-3.15915731027249\\
3.06936528459634	-3.15597773924306\\
3.07745118301085	-3.15721617863168\\
3.08505551335047	-3.15530471240286\\
3.09117886622799	-3.15657034280813\\
3.09591517059979	-3.15804404837235\\
3.09632204314056	-3.15870115025173\\
3.09636952119898	-3.15870794641379\\
3.0967537675808	-3.15954328487385\\
3.09700600756874	-3.15971252445794\\
3.09719424587017	-3.16008142247683\\
3.09721329208068	-3.16008072104982\\
3.09712786586598	-3.16014212129796\\
3.09702502715049	-3.16020080039762\\
3.09697306323425	-3.16024549702763\\
3.09692215114164	-3.16028504710926\\
3.09684527824763	-3.16032306476383\\
3.09673736326175	-3.16037182907913\\
3.09668364336411	-3.16039226330973\\
3.09664896930709	-3.1604257402297\\
3.09662131521083	-3.16044051438213\\
3.09655731180817	-3.16042989568142\\
3.09654336993824	-3.16050032261232\\
3.09647780492738	-3.16053204349722\\
3.09641855343031	-3.16055197285338\\
3.0963888618645	-3.16057986062829\\
3.09636115499291	-3.16058869030682\\
3.09636439855373	-3.16062664014847\\
3.09635501470139	-3.16063908152151\\
3.09633955916303	-3.16063979961763\\
3.09631654540382	-3.16063011753614\\
3.09631398925087	-3.16068838548574\\
3.09629863871614	-3.16073526062508\\
3.09628850549483	-3.16074778478469\\
3.09626105121525	-3.16076918282847\\
3.09624737721909	-3.16082546867302\\
3.09622883178555	-3.16082431677823\\
3.09622996271751	-3.16087108363459\\
3.09607010657696	-3.16123253862895\\
3.08081346641556	-3.16842972729698\\
3.01797772654142	-3.20577887416306\\
2.9381492211869	-3.26761074447709\\
2.8605693853463	-3.38502580160394\\
2.76806939329008	-3.57196987254929\\
2.69213901027043	-3.77037107850989\\
2.64778044684533	-3.96410420727068\\
2.62785062932047	-4.16850951362278\\
2.63199040372792	-4.39744157946429\\
2.64587643393545	-4.61153012475297\\
2.6914032085329	-4.80497406044664\\
2.76505385615446	-4.98221655889331\\
2.88186271740196	-5.10361530931841\\
3.03668969850811	-5.17918444355811\\
3.19881599260018	-5.22683215628281\\
3.3408343792643	-5.24556206589066\\
3.43714423450751	-5.24444231622144\\
3.50534608033019	-5.24951682976781\\
3.5528728040448	-5.25558000261972\\
3.57502492874832	-5.26256467220453\\
3.58365597063898	-5.26748143453138\\
3.58609074059624	-5.27417944946824\\
3.59475486487803	-5.27842958676564\\
3.59897464503583	-5.2809587832696\\
3.6008987377148	-5.28367888607931\\
3.60144190175655	-5.28515274462855\\
3.60138510069777	-5.28537170405735\\
3.6013198773224	-5.28527919392908\\
3.60130087081619	-5.28537566750118\\
3.60125192945229	-5.28539982728188\\
3.60122420192849	-5.28542972393675\\
3.60120741749615	-5.28549225137569\\
3.6011891069211	-5.28549373327576\\
3.60112616854157	-5.28551407149089\\
3.60107713188522	-5.28552044447737\\
3.6010461544579	-5.28554920621349\\
3.60103579005149	-5.28558203072221\\
3.60098993501591	-5.28559286450473\\
3.60095547585574	-5.28559688803739\\
3.60094100265969	-5.28562812086841\\
3.60087231587568	-5.28564271927413\\
3.60076781072682	-5.28567537316098\\
3.60072943621338	-5.28569233473091\\
3.60079619809301	-5.28565732972213\\
3.6008181099194	-5.28565468566984\\
3.60089776898448	-5.28566006364419\\
3.60091857307328	-5.28566671198897\\
3.60093479438434	-5.28565446123955\\
3.6009986217259	-5.28565606139261\\
3.6010912144966	-5.28564375925444\\
3.60107964866293	-5.28566555604489\\
3.60110987571993	-5.28568522988338\\
3.60118999568044	-5.28568593913605\\
3.60118421751218	-5.28570037514783\\
3.60128119546376	-5.28593938739317\\
3.60104415432134	-5.286067041381\\
3.58600866978029	-5.2892073278329\\
3.52195553350814	-5.32809051013229\\
3.41744086830072	-5.41432724422123\\
3.29440117297983	-5.54703888180838\\
3.19948072209993	-5.72570623421672\\
3.10232597518319	-5.93015271218998\\
3.03329598935823	-6.12117423450422\\
3.01623977931219	-6.28837514824324\\
3.03509377172675	-6.4705020003454\\
3.08824840251687	-6.65413655598819\\
3.17159154640825	-6.82187225521669\\
3.27930663937035	-6.97237262279837\\
3.42079323380615	-7.08753271232051\\
3.59151905599081	-7.16692994218235\\
3.77024516888127	-7.21742543145489\\
3.91619427154905	-7.23693289041281\\
4.01454357045008	-7.23907070763327\\
4.07134702932552	-7.23171841187004\\
4.09742334995872	-7.22626342984269\\
4.11173795269158	-7.22133736575868\\
4.12163623645945	-7.21732906922904\\
4.13103359137039	-7.21651080486666\\
4.13755408710622	-7.2162963269873\\
4.13775882470071	-7.21613084571394\\
4.13776358481127	-7.21611243556726\\
4.13773466112525	-7.21598734052995\\
4.13774771607687	-7.21588203833666\\
4.13776454868316	-7.21587957203047\\
4.13775397053063	-7.21586508921776\\
4.13778150585411	-7.21586601112307\\
4.13778080271376	-7.21585359395919\\
4.13778549649053	-7.21583762273456\\
4.13782170684517	-7.21583374958595\\
4.13784094348758	-7.21586191459516\\
4.13784489709201	-7.21588616752403\\
4.1378206912737	-7.21587601229689\\
4.1378093214384	-7.21590747986465\\
4.13777955323495	-7.21592710808564\\
4.13776566929323	-7.21592898929352\\
4.1377531819969	-7.215933164975\\
4.1377670306418	-7.21594856524859\\
4.13778754911529	-7.21597533569055\\
4.13779016507636	-7.21593128231003\\
4.13777352814126	-7.21592630773295\\
4.13777855111538	-7.21593725513197\\
4.13778893655978	-7.21595921220047\\
4.13780666470246	-7.21593655341368\\
4.13783616566131	-7.21592221243457\\
4.13780778997929	-7.21592444801687\\
4.13784108488695	-7.21592630456141\\
4.13781888562751	-7.21591242663677\\
4.13665620927826	-7.215628964142\\
4.12270630067275	-7.21945410957291\\
4.07235444997053	-7.22642904309729\\
3.97358906738038	-7.2418168212575\\
3.84948845005095	-7.26554208071252\\
3.69437867242907	-7.31340138227667\\
3.51094279145316	-7.36852787709735\\
3.3127574005518	-7.42572241122704\\
3.10718275823866	-7.4816604087633\\
2.98082783809748	-7.55552718861798\\
2.91196037771693	-7.64065935330123\\
2.89155023358737	-7.73191081637619\\
2.89723475474142	-7.80392683674904\\
2.91914978906921	-7.8656450193147\\
2.93281083699811	-7.90647703071255\\
2.93942812256812	-7.91985882811354\\
2.93904694243987	-7.92635279610397\\
2.93636258006762	-7.92896418957231\\
2.9325497430958	-7.93182361542879\\
2.92960792259403	-7.93466763940134\\
2.92907529155575	-7.93457064607335\\
2.92909813976788	-7.93461528699171\\
2.9291110680521	-7.93458672316289\\
2.92913853810565	-7.93457837302118\\
2.92917239534459	-7.93460263523934\\
2.92918926736441	-7.93462937525734\\
2.92920476201152	-7.93468691207223\\
2.92920729177712	-7.93470678331676\\
2.92924180214378	-7.93471587235855\\
2.92921877709689	-7.93470013211925\\
2.92921064132136	-7.9347132398917\\
2.92920404123479	-7.93469256516322\\
2.92924677490911	-7.93465520650064\\
2.92924407191942	-7.93464273548596\\
2.92925126590097	-7.93463885975374\\
2.92925404621192	-7.93462906149656\\
2.92923998592334	-7.93462657918816\\
2.92924432673008	-7.93465430954631\\
2.92928274431497	-7.93468828257559\\
2.92930410398434	-7.93469926040925\\
2.92934877846475	-7.9346737606239\\
2.92934521795331	-7.93467308986478\\
2.92946714575931	-7.93421893027096\\
2.91598554238368	-7.92460418536541\\
2.88091178978578	-7.88539918067506\\
2.80366019150825	-7.83200083949079\\
2.69122102378401	-7.79375549479296\\
2.53503801371546	-7.74854646444452\\
2.37465590483859	-7.69535503394131\\
2.2043943481948	-7.65858596771171\\
2.01801892959731	-7.6506592506079\\
1.81769306018271	-7.67456130239071\\
1.62657540440522	-7.71792463995313\\
1.46047174988014	-7.770667257204\\
1.3084881485382	-7.85063249564817\\
1.19069312000225	-7.94631547467188\\
1.11869424496883	-8.07058661171614\\
1.0820945632131	-8.20693530547731\\
1.07140564888849	-8.32543298337674\\
1.06959204772267	-8.39635624997542\\
1.07947480984888	-8.43011783258701\\
1.08156919061721	-8.44119023459224\\
1.08241086062983	-8.44463448596511\\
1.08479392193019	-8.44849594355168\\
1.08541948173186	-8.44901539972338\\
1.08535177547114	-8.44897089741297\\
1.08526180947984	-8.44902429776241\\
1.08522308604505	-8.44900348778397\\
1.08520403090444	-8.44901527755426\\
1.08519394743313	-8.44900996452474\\
1.08519502683355	-8.44896595009374\\
1.08526454656893	-8.44890364816324\\
1.08531511937334	-8.4489140376018\\
1.08531071680857	-8.4489654773698\\
1.08526644031403	-8.44900931368755\\
1.08522688965296	-8.44899886834496\\
1.08516950637368	-8.4490453378777\\
1.08515478635606	-8.44906095160768\\
1.0851545929522	-8.44903497125083\\
1.0848201453538	-8.44907357045382\\
1.08063976142872	-8.44639857617181\\
1.04096166497165	-8.41191925013342\\
0.943900032605557	-8.35752475292756\\
0.781766410148253	-8.29372780188584\\
0.593617587104313	-8.24537145150214\\
0.404554723238866	-8.22284878956944\\
0.21738807411732	-8.23601439309626\\
0.0319486891135122	-8.27080697389188\\
-0.142212367347958	-8.30965753245309\\
-0.308860107021345	-8.36568301573065\\
-0.460795680798517	-8.44404994004487\\
-0.574825103878121	-8.54697853585917\\
-0.650487752001873	-8.6614197241668\\
-0.700362289951746	-8.78183423983257\\
-0.726958558494037	-8.87243343462837\\
-0.740235279769985	-8.92119394547754\\
-0.748244647660282	-8.94752207661193\\
-0.746029998144798	-8.95918348228956\\
-0.746651398196545	-8.96234849286695\\
-0.74549654267234	-8.96243471119719\\
-0.744781115599851	-8.96265833312471\\
-0.744662990417552	-8.96262732518935\\
-0.744672171674341	-8.9626197757918\\
-0.744625310703393	-8.96257301045399\\
-0.744575139653484	-8.96252396702296\\
-0.744587573604365	-8.96251432371972\\
-0.744579311605879	-8.9625333199117\\
-0.744610206779275	-8.96251013233094\\
-0.744642311962121	-8.96250942948563\\
-0.744704346326908	-8.96249764950456\\
-0.744683451226037	-8.96251037932797\\
-0.74467850653943	-8.96249349447544\\
-0.744677445348676	-8.9624813967747\\
-0.74474031957193	-8.96247854436995\\
-0.744775231714493	-8.96251210320161\\
-0.7447600556106	-8.96254876067852\\
-0.744781098758446	-8.96257678562765\\
-0.744754785650624	-8.96261299795111\\
-0.744827471923965	-8.96260045551048\\
-0.747208225026696	-8.96224818932011\\
-0.750413335641132	-8.95814919480851\\
-0.758568589578043	-8.91303835509637\\
-0.80363155428631	-8.8006919683615\\
-0.878364074441795	-8.62804245177909\\
-0.962681142087916	-8.41482490887318\\
-1.04391983013048	-8.1694818217312\\
-1.12296067095747	-7.93383168781567\\
-1.20400661003926	-7.74169636375938\\
-1.3092385600979	-7.59381033854482\\
-1.41347211759225	-7.46677087719331\\
-1.51417436596064	-7.34427622034318\\
-1.61617829117919	-7.23915786686695\\
-1.7319654918651	-7.18197691578199\\
-1.88054660455744	-7.16928210486591\\
-2.04982814118951	-7.19776919727884\\
-2.20080356107445	-7.25321647437876\\
-2.34824510530425	-7.33030571661406\\
-2.50871129200622	-7.4107490893529\\
-2.63111057978071	-7.49914965722791\\
-2.6919642474997	-7.6120953200017\\
-2.75791578056702	-7.76737617394206\\
-2.7924846921949	-7.98394821287764\\
-2.8233425898344	-8.18810678546929\\
-2.86482632575049	-8.36832105512196\\
-2.91871361009407	-8.53204822051507\\
-2.96292394743726	-8.66890170820825\\
-2.9976006874147	-8.77658778381997\\
-3.02916632144108	-8.86174290537892\\
-3.05325877376709	-8.91045015686092\\
-3.06557860415637	-8.91936765767972\\
-3.07333212693839	-8.92415861213703\\
-3.08011798494595	-8.91706900609633\\
-3.08421616502547	-8.9126754087075\\
-3.08803961071227	-8.91307289565824\\
-3.08824229543247	-8.91242256469492\\
-3.08805650775991	-8.91243568478849\\
-3.08802798678241	-8.91240817599605\\
-3.08800439542001	-8.91245031097112\\
-3.08800499292131	-8.91244858784271\\
-3.08803895746175	-8.91247106226615\\
-3.08799689537963	-8.91251131545206\\
-3.08796608485326	-8.91255029391463\\
-3.08793741777034	-8.91255614507232\\
-3.08794217664437	-8.91256154962057\\
-3.08801955755162	-8.91257534900155\\
-3.08801652320818	-8.91258476203081\\
-3.08800746870621	-8.91257533285075\\
-3.08798787252715	-8.91257471172961\\
-3.0879833994319	-8.91257635178266\\
-3.08798000086722	-8.9125754588781\\
-3.08794762513459	-8.91260981576636\\
-3.08795242546595	-8.91262440746279\\
-3.08796339345137	-8.9126444519341\\
-3.08791159188506	-8.9126493211139\\
-3.08793734186741	-8.91264465997108\\
-3.08568440897466	-8.91180062419479\\
-3.0588238569741	-8.88224094552195\\
-3.01532651464924	-8.80028424292202\\
-2.97656633607377	-8.68867217727396\\
-2.93779295829764	-8.53348761867131\\
-2.89283202095912	-8.34118880024107\\
-2.86592717694149	-8.14971808877431\\
-2.85336581828205	-7.96735680471576\\
-2.86191880282282	-7.777921048031\\
-2.88135235117695	-7.58535446718644\\
-2.90041175644225	-7.40779135247733\\
-2.94668711120946	-7.25129236134895\\
-3.07254455562255	-7.16499694380593\\
-3.23527617691505	-7.12420337697998\\
-3.37130699729052	-7.09298831298093\\
-3.46912058509938	-7.06881632520075\\
-3.53182494574523	-7.05853676494215\\
-3.57140618523245	-7.05199036926462\\
-3.58462081930814	-7.05290646085753\\
-3.58551271547087	-7.0568744991221\\
-3.58447405086534	-7.05963105752266\\
-3.58341118242967	-7.06074585751812\\
-3.58328768369053	-7.0608697444658\\
-3.58324772320743	-7.0608557228293\\
-3.58325192451842	-7.06091829946132\\
-3.58318567262482	-7.06087631231156\\
-3.58312075532224	-7.06090240255843\\
-3.5831224157823	-7.06090725810831\\
-3.58312438291711	-7.06088409458084\\
-3.58312918508493	-7.06087482534353\\
-3.58308402154774	-7.06094351477611\\
-3.58314529581024	-7.06091529057835\\
-3.5831726670857	-7.060921443872\\
-3.58317778071365	-7.06096413191539\\
-3.58318560290499	-7.06095698543886\\
-3.58310290830537	-7.06096524003486\\
-3.58310923809808	-7.0609285623816\\
-3.58315158769183	-7.06059615015081\\
-3.58297828491386	-7.059133813642\\
-3.5750737364357	-7.06331665752708\\
-3.54477994264647	-7.07788769516303\\
-3.48778805530207	-7.10648551545158\\
-3.38557044211571	-7.13645416185736\\
-3.24426143236967	-7.13763666991011\\
-3.10877408880863	-7.11208544691229\\
-2.99295589513601	-7.06145436303498\\
-2.87031785670545	-6.99197677095986\\
-2.74968859892979	-6.89439326873472\\
-2.67692400091333	-6.77257151980182\\
-2.65250263309432	-6.64252248151621\\
-2.66310763353198	-6.50602346376105\\
-2.71085228594871	-6.34899451231929\\
-2.77594993682736	-6.17974702554479\\
-2.85608514023522	-6.03493317388114\\
-2.95338776855926	-5.89074791749063\\
-3.06112163433565	-5.71745169537546\\
-3.18677016364407	-5.58028434684102\\
-3.32883065842815	-5.48237468237284\\
-3.4846031224664	-5.38970045341827\\
-3.6465141513642	-5.33411650264275\\
-3.79692565574691	-5.31018120693198\\
-3.91892917638422	-5.29963397665148\\
-3.99926482521786	-5.3062546878944\\
-4.03348123370373	-5.31801348054086\\
-4.04181871275186	-5.32949417344047\\
-4.0521524627728	-5.33547393333179\\
-4.05377346752522	-5.34241271151994\\
-4.05214182856628	-5.34568358239941\\
-4.05115986780981	-5.34757016905601\\
-4.05101060366129	-5.34796693831705\\
-4.05100251638869	-5.34794757865838\\
-4.05101832337835	-5.34799131178833\\
-4.0510243336165	-5.3479941373626\\
-4.05102857063416	-5.34806249410379\\
-4.05101800029647	-5.34808839740503\\
-4.05097120790632	-5.34806142125131\\
-4.05094600999829	-5.34806143433673\\
-4.050927293499	-5.34807260482706\\
-4.05094968842662	-5.34805893565393\\
-4.0509751569074	-5.34806641693826\\
-4.05098962137643	-5.34810587384375\\
-4.05099307027738	-5.34815586680886\\
-4.05100382058782	-5.34818568721046\\
-4.05102795504509	-5.34821369925259\\
-4.05102509810052	-5.34822097413384\\
-4.05104526284059	-5.34817576045602\\
-4.05063783152739	-5.34802814537331\\
-4.03513881500406	-5.35089921265884\\
-3.99339390746934	-5.33030356046734\\
-3.9463596593246	-5.27698595121143\\
-3.90631220574816	-5.17783246358344\\
-3.86205491259032	-5.03559921459117\\
-3.82721930030131	-4.88152206950329\\
-3.82356563113668	-4.72282670145851\\
-3.8639391128173	-4.5630810803471\\
-3.92319328746419	-4.40168465691473\\
-3.96934052564929	-4.23983637052006\\
-4.02598801966558	-4.07005146923612\\
-4.10437796095042	-3.92163974962234\\
-4.20009934218167	-3.77762483822862\\
-4.3070601095658	-3.64341250824798\\
-4.41498621082928	-3.52219742252885\\
-4.49111575888122	-3.41253373557952\\
-4.5584688094744	-3.32249711051709\\
-4.60106466106603	-3.26318075142101\\
-4.61317626836878	-3.2301339023159\\
-4.61904953751019	-3.22975816329148\\
-4.62374632771222	-3.23325033591702\\
-4.62416701192276	-3.23349191175595\\
-4.62403881789663	-3.23389354215218\\
-4.62409241279381	-3.23374265131992\\
-4.62419229037057	-3.23348714194118\\
-4.62416850960203	-3.2334620706985\\
-4.62415034997578	-3.23349019386611\\
-4.62415281166241	-3.23347878174556\\
-4.62413176399485	-3.23345753659332\\
-4.62409412260415	-3.23347013634358\\
-4.62409310263067	-3.23347271643537\\
-4.62411715551666	-3.23348709029789\\
-4.6241094415722	-3.23345457224132\\
-4.62406401658523	-3.23347786707271\\
-4.62412343021001	-3.23349885956571\\
-4.62410368126929	-3.23357921708985\\
-4.62408600594914	-3.23363214870891\\
-4.6240549810958	-3.2336522066133\\
-4.62401028284625	-3.23365248990911\\
-4.62403692901591	-3.23374000686453\\
-4.62405187984398	-3.23380921945674\\
-4.62408160875594	-3.2338530899622\\
-4.62410245720655	-3.23389570321251\\
-4.6241256969725	-3.23388234351461\\
-4.62412042723173	-3.23390804410421\\
-4.62407510247883	-3.23391538002581\\
-4.62409726342291	-3.23391156786489\\
-4.62389978047175	-3.23380231488184\\
-4.62327006476929	-3.23132763883221\\
-4.61777219542217	-3.2347405203544\\
-4.58039412989751	-3.2613044485987\\
-4.52068240543676	-3.29609778041758\\
-4.43032122907212	-3.35619254754721\\
-4.33096640465874	-3.42998841143125\\
-4.20021421381801	-3.48672156769664\\
-4.04972438959154	-3.51654674962392\\
-3.86368382056462	-3.52605343633859\\
-3.63887303553933	-3.51845131200727\\
-3.4400169187942	-3.48651376975192\\
-3.26947342701856	-3.41276006463135\\
-3.11201968154016	-3.29571364673193\\
-2.99222895093419	-3.17275812317652\\
-2.91802840009516	-3.05266667374788\\
-2.85831782076857	-2.90802138930924\\
-2.82439071412671	-2.75428620043189\\
-2.84421047793641	-2.62664629377016\\
-2.91027974927297	-2.50447792604537\\
-2.99605131267278	-2.36082184627314\\
-3.08102137120963	-2.23364758794367\\
-3.1874667587488	-2.1247663099538\\
-3.32684247002172	-2.01405063768635\\
-3.48800434536597	-1.91311171852746\\
-3.60991145354463	-1.82901259196896\\
-3.72265415254701	-1.73924663219597\\
-3.85958676238809	-1.69273993670661\\
-3.99291076481149	-1.65306538069271\\
-4.11260460038706	-1.59674859331337\\
-4.22027911858994	-1.53573864516344\\
-4.30633364232759	-1.46907079057745\\
-4.36715785557358	-1.37764326377509\\
-4.40430172187116	-1.28648186289371\\
-4.42898332943934	-1.23165358609694\\
-4.44199199305619	-1.21326375424391\\
-4.44032910205679	-1.20698810418347\\
-4.43975095408596	-1.20113668879457\\
-4.4377043299887	-1.19697376444509\\
-4.43712547343213	-1.19740400025126\\
-4.43756401245872	-1.19744233806741\\
-4.43751686018717	-1.19745633047342\\
-4.4377055041275	-1.1974815529893\\
-4.43770819317397	-1.19745405992972\\
-4.43770034324698	-1.19740548244145\\
-4.43767224134298	-1.19739661604649\\
-4.43769544620774	-1.19743382032837\\
-4.43772815624535	-1.19740888197717\\
-4.43773600260719	-1.19737551359586\\
-4.437742457897	-1.19734671136325\\
-4.43774407783613	-1.19736361267334\\
-4.43776157904086	-1.19737565011638\\
-4.43778872394688	-1.19741165623958\\
-4.43779137112685	-1.19741512071561\\
-4.43780817078348	-1.19739868950138\\
-4.43788207642832	-1.19742396696602\\
-4.43795072149761	-1.1973497381378\\
-4.43796518070769	-1.19737588075968\\
-4.43801021640024	-1.19732426407822\\
-4.43805695400435	-1.19729293098054\\
-4.43806678010482	-1.19731815764376\\
-4.43720700747154	-1.19802762176576\\
-4.43211189271169	-1.20456469159793\\
-4.41769362129831	-1.25285667506835\\
-4.39851366431377	-1.35120154772625\\
-4.33859984073226	-1.43654052495184\\
-4.21596104092054	-1.51056565622647\\
-4.05933110778421	-1.57497828848098\\
-3.94262920703529	-1.63712422869567\\
-3.84605379717815	-1.69247989796078\\
-3.73880180928362	-1.74866480064005\\
-3.59576827858267	-1.81362088200938\\
-3.43569925167978	-1.88291910359199\\
-3.32008818397026	-1.95462020444112\\
-3.22218128436486	-2.01500055901816\\
-3.08936204584284	-2.05221106683769\\
-2.92647266769235	-2.08265477717385\\
-2.79123311013704	-2.0796573233427\\
-2.66022548063651	-2.02875755419811\\
-2.53054394046914	-1.94980793810065\\
-2.43250439833051	-1.82725745853814\\
-2.37169157383042	-1.66929535240289\\
-2.31012077390094	-1.47963788067926\\
-2.26159484578106	-1.28798335680584\\
-2.23739707272631	-1.10940581914623\\
-2.22486501369122	-0.940045964083772\\
-2.2209581471568	-0.798015790679558\\
-2.23154434219285	-0.704319323164935\\
-2.24182422663853	-0.652962356984639\\
-2.24506062444687	-0.628030222440938\\
-2.24616343953864	-0.61825898645004\\
-2.25144957673805	-0.612608374175943\\
-2.25515974111602	-0.606618380156241\\
-2.2583959858612	-0.601989166262031\\
-2.25824406908004	-0.598348461071159\\
-2.2581597693494	-0.596189575669265\\
-2.25880102759306	-0.59616679116634\\
-2.25926290120729	-0.59643758096422\\
-2.25922247192067	-0.596488822403136\\
-2.25925627374285	-0.596488266224322\\
-2.25927757012258	-0.596516884477357\\
-2.25927512658456	-0.596509063825905\\
-2.25924560351275	-0.596525831135625\\
-2.25925226839525	-0.59649156739435\\
-2.25929109472362	-0.596500175933524\\
-2.2592961730961	-0.59651559853612\\
-2.25932566513749	-0.596508199009433\\
-2.25930863668447	-0.596508534769874\\
-2.25932878292352	-0.596384127959545\\
-2.25934551068732	-0.596365072799113\\
-2.25933111391716	-0.596337928892897\\
-2.2593380300808	-0.596308527092353\\
-2.25935203739377	-0.596260374816083\\
-2.25935337502908	-0.596274737429395\\
-2.25936870737023	-0.596268262721458\\
-2.25941828709231	-0.596295051832078\\
-2.25944799245295	-0.596324257098331\\
-2.25945463972544	-0.59629299750202\\
-2.25944305643448	-0.596313167561111\\
-2.25945426189876	-0.596282042761074\\
-2.25944305382333	-0.596265862240364\\
-2.25947864305001	-0.596264773899672\\
-2.25951429679402	-0.596199735642718\\
-2.25951471187017	-0.596203844938587\\
-2.25894331648181	-0.59614402756279\\
-2.25770073508461	-0.596300411359037\\
-2.25582100671564	-0.610255840252318\\
-2.23738204798055	-0.655707609752763\\
-2.19495100918343	-0.743077874865171\\
-2.14791021794325	-0.860268123382947\\
-2.09427268017415	-1.02194791783587\\
-2.02680285097262	-1.1894493977759\\
-1.94223298337462	-1.36240537937171\\
-1.85328802972351	-1.51285960574421\\
-1.75216865490459	-1.62801263035211\\
-1.64672767928816	-1.7022259751548\\
-1.51076025997859	-1.73676427502533\\
-1.33777021624871	-1.74691441549107\\
-1.16268604265173	-1.73843051619381\\
-0.996618129305375	-1.7169417063536\\
-0.819407079111855	-1.68145175509965\\
-0.659278771819412	-1.63356911993599\\
-0.527350101021355	-1.55624591895666\\
-0.388196904897751	-1.4626585711493\\
-0.278990248022106	-1.35312680374\\
-0.216692229533358	-1.2198043176716\\
-0.196121232024057	-1.06864643046508\\
-0.222706484241068	-0.909258474540044\\
-0.296726020372785	-0.741766313350587\\
-0.37531095526224	-0.591711319125143\\
-0.429825472159253	-0.46555728629568\\
-0.471153032954475	-0.362278046540697\\
-0.503586421500576	-0.282943371412542\\
-0.526666277020833	-0.218590296143464\\
-0.543188437042285	-0.17688028304434\\
-0.552517266932435	-0.156341612660352\\
-0.560225237760001	-0.148592980391451\\
-0.560186280643418	-0.143495898991843\\
-0.56080544329053	-0.142174730278586\\
-0.561093863343358	-0.141967053804509\\
-0.561178295115557	-0.141973000083923\\
-0.561199493467991	-0.141963747335944\\
-0.561229261271086	-0.141919526031978\\
-0.561271469662159	-0.141848755510669\\
-0.561310072863099	-0.141807142003229\\
-0.561328356929992	-0.141780549377351\\
-0.561345175113018	-0.141785398748756\\
-0.561355532052744	-0.141809239407975\\
-0.561366639633146	-0.141822357575848\\
-0.561384140520462	-0.141843933781324\\
-0.561384252609418	-0.141885265519554\\
-0.561395970436353	-0.141873951261116\\
-0.561415812595908	-0.14186401060515\\
-0.56142349372551	-0.141899946520709\\
-0.561426040380096	-0.14193574523073\\
-0.561429626272698	-0.141990041145356\\
-0.561424335282278	-0.142084907158131\\
-0.561433939866494	-0.142128969321269\\
-0.561456221239266	-0.14211281319743\\
-0.561469255229069	-0.142138338326608\\
-0.561458788937653	-0.142192694534786\\
-0.561446757998703	-0.14225590018851\\
-0.561447171053516	-0.142315288594067\\
-0.561207147994464	-0.143229930991969\\
-0.56215554611545	-0.161810860140824\\
-0.51748661034323	-0.231770820729473\\
-0.429661387215723	-0.34389026837063\\
-0.342164926005648	-0.465363708058938\\
-0.227227930203122	-0.590492342293279\\
-0.0630220547628063	-0.713086583257441\\
0.122965880028692	-0.820837681023754\\
0.276244811083041	-0.882701649742535\\
0.420662129377777	-0.89060096817114\\
0.577384928695626	-0.847821723823166\\
0.736299462514455	-0.775155435470144\\
0.873191091847658	-0.685542355839662\\
0.993840725170808	-0.570977635728779\\
1.07835098771227	-0.440013214169852\\
1.11788632314886	-0.287642215658588\\
1.14200482002401	-0.121766070831012\\
1.15195349043892	0.0214942560805639\\
1.14944584840883	0.133574403015123\\
1.14217508646044	0.21275343030631\\
1.12735425288602	0.259444012883619\\
1.10723455394527	0.281027441756232\\
1.08546478945114	0.285641558075472\\
1.07709321823963	0.286560040555912\\
1.07077621916483	0.286526351654111\\
1.06633847552473	0.28480643070962\\
1.06023951552352	0.285197342506307\\
1.05896004672209	0.284816556720657\\
1.05891594114098	0.284859397519187\\
1.05903493012331	0.284946299258387\\
1.05946866473151	0.285042077803546\\
1.05947793358343	0.285053334176337\\
1.05949172647013	0.285005691476558\\
1.05943734043386	0.284971460093315\\
1.05936366314974	0.284942984959696\\
1.05928812610006	0.284956613718397\\
1.05926815987223	0.285010163326066\\
1.05925934071413	0.284977744636758\\
1.05925356343839	0.284959097808369\\
1.05922637064015	0.284937550773857\\
1.05918829147109	0.284972481512312\\
1.05913801884951	0.284931097211475\\
1.05913940340568	0.284856040214729\\
1.05911467386058	0.284799038453627\\
1.05909246369862	0.284804281020889\\
1.05905242360662	0.284775264276177\\
1.05902637701412	0.284756237227569\\
1.05900413626516	0.284773128460808\\
1.05898640682092	0.284767171529028\\
1.05896448370196	0.284770282772091\\
1.05898675493381	0.284770974772103\\
1.0589641074852	0.284718241136997\\
1.05965791328966	0.284352399545354\\
1.06209085990315	0.272016854935998\\
1.07876948588791	0.226447506084413\\
1.11708249579741	0.165183136938901\\
1.17600134617277	0.0730678166511503\\
1.25847347035794	-0.0597870951240671\\
1.34144404812182	-0.231786718419311\\
1.42328563673162	-0.424673951374366\\
1.50192898273642	-0.617715522148227\\
1.5836909431026	-0.79967981735925\\
1.6851908380361	-0.966937677188305\\
1.79837052667051	-1.11834610056196\\
1.90163540423691	-1.24435977298166\\
2.02439118012216	-1.35411767510539\\
2.17256674378988	-1.45122065502446\\
2.32451853308128	-1.51541792041483\\
2.44827528849398	-1.53450294838075\\
2.52918198690221	-1.5364648025928\\
2.5694051649922	-1.53680883558865\\
2.59373136069919	-1.53084046209085\\
2.60654309142361	-1.5276408965704\\
2.61159001600643	-1.53018967877829\\
2.61701888713114	-1.52986103650336\\
2.619688738992	-1.52916841356118\\
2.62100420647787	-1.52873237019968\\
2.62103180834738	-1.52865277966111\\
2.62107965253656	-1.52863516859555\\
2.6210268743399	-1.5286206965384\\
2.6209645244154	-1.52862338236526\\
2.62102241872495	-1.52859418287895\\
2.62105942166974	-1.5285704527493\\
2.62106713764979	-1.52856761424981\\
2.62105042635783	-1.52857514319593\\
2.62105105022155	-1.52859161657372\\
2.62102125937833	-1.52861000015313\\
2.62103559164124	-1.52859996479589\\
2.62107679168877	-1.52859816792482\\
2.62106830132006	-1.52861481826801\\
2.62107843702177	-1.52860416783618\\
2.62108802577415	-1.52861479038832\\
2.62113517189342	-1.5286065113835\\
2.62116400582043	-1.52859553541131\\
2.62114666956295	-1.52860382058468\\
2.62113493053023	-1.52861393345721\\
2.62050288741817	-1.52908427473965\\
2.60939875053724	-1.5289474274176\\
2.53997140446977	-1.52080469460461\\
2.42781849983398	-1.47403236658933\\
2.31095910053997	-1.39250857396558\\
};
\addplot [color=black,solid,line width=1.2pt,forget plot]
  table[row sep=crcr]{%
-5.11653877330595	-1.37896103233387\\
-4.32654160652105	-1.16601707943085\\
-3.53654443973615	-0.953073126527823\\
-2.74654727295124	-0.740129173624798\\
-1.95655010616634	-0.527185220721773\\
-1.16655293938144	-0.314241267818748\\
-0.376555772596531	-0.101297314915723\\
0.413441394188372	0.111646637987301\\
1.20343856097328	0.324590590890326\\
1.99343572775818	0.537534543793351\\
};
\node[align=center, rotate=-164.91444417756, text=black]
at (axis cs:-1.692,0.062) {7.36 m};
\addplot [color=black,solid,line width=1.2pt,forget plot]
  table[row sep=crcr]{%
1.99343572775818	0.537534543793351\\
2.24053395050997	-0.36370278407821\\
2.48763217326177	-1.26494011194977\\
2.73473039601356	-2.16617743982133\\
2.98182861876536	-3.0674147676929\\
3.22892684151715	-3.96865209556446\\
3.47602506426894	-4.86988942343602\\
3.72312328702074	-5.77112675130758\\
3.97022150977253	-6.67236407917915\\
4.21731973252433	-7.57360140705071\\
};
\node[align=center, rotate=105.332383772726, text=black]
at (axis cs:3.588,-3.386) {8.41 m};
\addplot [color=black,solid,line width=1.2pt,forget plot]
  table[row sep=crcr]{%
-2.89924402617534	-9.55650052172439\\
-2.10851471965316	-9.33617839787176\\
-1.31778541313097	-9.11585627401913\\
-0.527056106608786	-8.8955341501665\\
0.263673199913399	-8.67521202631387\\
1.05440250643558	-8.45488990246123\\
1.84513181295777	-8.2345677786086\\
2.63586111947996	-8.01424565475597\\
3.42659042600214	-7.79392353090334\\
4.21731973252433	-7.57360140705071\\
};
\node[align=center, rotate=15.5695138451748, text=black]
at (axis cs:0.793,-9.047) {7.39 m};
\addplot [color=black,solid,line width=1.2pt,forget plot]
  table[row sep=crcr]{%
-5.11653877330595	-1.37896103233387\\
-4.87017269029144	-2.28757653115504\\
-4.62380660727693	-3.19619202997621\\
-4.37744052426242	-4.10480752879738\\
-4.1310744412479	-5.01342302761855\\
-3.88470835823339	-5.92203852643972\\
-3.63834227521888	-6.83065402526088\\
-3.39197619220437	-7.73926952408205\\
-3.14561010918986	-8.64788502290322\\
-2.89924402617534	-9.55650052172439\\
};
\node[align=center, rotate=-74.8293237960073, text=black]
at (axis cs:-4.49,-5.599) {8.47 m};
\addplot [color=red,solid,line width=1.2pt,forget plot]
  table[row sep=crcr]{%
0.0799152873406196	-0.05997397574134\\
-0.080005338603703	0.0601317834776752\\
};
\addplot [color=red,solid,line width=1.2pt,forget plot]
  table[row sep=crcr]{%
0.0931616043835164	-0.0360438518492603\\
-0.0933349724468495	0.0361989805326131\\
};
\addplot [color=red,solid,line width=1.2pt,forget plot]
  table[row sep=crcr]{%
0.0967441390975613	-0.0248219145101791\\
-0.096956662231544	0.024977679508874\\
};
\addplot [color=red,solid,line width=1.2pt,forget plot]
  table[row sep=crcr]{%
0.0990337109305127	-0.0130204648647566\\
-0.0992449938496747	0.0131626447804192\\
};
\addplot [color=red,solid,line width=1.2pt,forget plot]
  table[row sep=crcr]{%
0.0998070740127255	-0.000581001724498147\\
-0.100187432931811	0.000901291098001879\\
};
\addplot [color=red,solid,line width=1.2pt,forget plot]
  table[row sep=crcr]{%
0.0997974214763522	0.00185868247453578\\
-0.10017551726707	-0.00143126742165005\\
};
\addplot [color=red,solid,line width=1.2pt,forget plot]
  table[row sep=crcr]{%
0.0997592079451808	0.0010447614872998\\
-0.100234189348785	-0.000580365836315983\\
};
\addplot [color=red,solid,line width=1.2pt,forget plot]
  table[row sep=crcr]{%
0.0997347458670169	0.00043220169006134\\
-0.100264950999494	8.39872395354983e-05\\
};
\addplot [color=red,solid,line width=1.2pt,forget plot]
  table[row sep=crcr]{%
0.0997312049180951	-0.000117956754252573\\
-0.100267587476898	0.000577054428579676\\
};
\addplot [color=red,solid,line width=1.2pt,forget plot]
  table[row sep=crcr]{%
0.0996818345934159	-0.000542134438480813\\
-0.100312745119829	0.000930307520434248\\
};
\addplot [color=red,solid,line width=1.2pt,forget plot]
  table[row sep=crcr]{%
0.0996426480480695	-0.000752316958133432\\
-0.100347959714943	0.00118593136985993\\
};
\addplot [color=red,solid,line width=1.2pt,forget plot]
  table[row sep=crcr]{%
0.0995647094381839	-0.000894855551353677\\
-0.100423040303305	0.00131872827030868\\
};
\addplot [color=red,solid,line width=1.2pt,forget plot]
  table[row sep=crcr]{%
0.0995188133315219	-0.0010216122475353\\
-0.100467444411889	0.00132289490449157\\
};
\addplot [color=red,solid,line width=1.2pt,forget plot]
  table[row sep=crcr]{%
0.0995258147391587	-0.0010662666712135\\
-0.100460439491157	0.00127854012838501\\
};
\addplot [color=red,solid,line width=1.2pt,forget plot]
  table[row sep=crcr]{%
0.0994372326675304	-0.000947817502032408\\
-0.100551323376662	0.00119168042010479\\
};
\addplot [color=red,solid,line width=1.2pt,forget plot]
  table[row sep=crcr]{%
0.0993588498170254	-0.000785881610450294\\
-0.100633424602486	0.000972007040960532\\
};
\addplot [color=red,solid,line width=1.2pt,forget plot]
  table[row sep=crcr]{%
0.0993334488660729	-0.000547297480703347\\
-0.100662956889319	0.00065173751659656\\
};
\addplot [color=red,solid,line width=1.2pt,forget plot]
  table[row sep=crcr]{%
0.0992933023189854	-0.00021821619497646\\
-0.100706104864872	0.000268739763457122\\
};
\addplot [color=red,solid,line width=1.2pt,forget plot]
  table[row sep=crcr]{%
0.0992319035415636	0.000180678274933691\\
-0.100767682337433	-0.000226320779128729\\
};
\addplot [color=red,solid,line width=1.2pt,forget plot]
  table[row sep=crcr]{%
0.0991786904445345	0.000668559128076853\\
-0.100816194512834	-0.000761822250747507\\
};
\addplot [color=red,solid,line width=1.2pt,forget plot]
  table[row sep=crcr]{%
0.099079553430059	0.00126701136334029\\
-0.100903437666673	-0.00134129690152046\\
};
\addplot [color=red,solid,line width=1.2pt,forget plot]
  table[row sep=crcr]{%
0.0990353186525175	0.00193699021164656\\
-0.100926231662093	-0.00198453956353908\\
};
\addplot [color=red,solid,line width=1.2pt,forget plot]
  table[row sep=crcr]{%
0.0989719678847428	0.00265763789398995\\
-0.100956303274857	-0.00269832991149085\\
};
\addplot [color=red,solid,line width=1.2pt,forget plot]
  table[row sep=crcr]{%
0.0990215058086805	0.00350572063967229\\
-0.10085634216193	-0.00348326287995254\\
};
\addplot [color=red,solid,line width=1.2pt,forget plot]
  table[row sep=crcr]{%
2.73338006779721	-1.11774737241004\\
2.66709324381177	-0.929051699523977\\
};
\addplot [color=red,solid,line width=1.2pt,forget plot]
  table[row sep=crcr]{%
2.7315152049086	-1.11834399815144\\
2.66875775804101	-0.928445333430852\\
};
\addplot [color=red,solid,line width=1.2pt,forget plot]
  table[row sep=crcr]{%
2.73022063029496	-1.11876408759693\\
2.66979342134317	-0.928111124288475\\
};
\addplot [color=red,solid,line width=1.2pt,forget plot]
  table[row sep=crcr]{%
2.72913915558133	-1.11914750727021\\
2.67075390456774	-0.927859341237707\\
};
\addplot [color=red,solid,line width=1.2pt,forget plot]
  table[row sep=crcr]{%
2.72778642008683	-1.11961796598656\\
2.67130460652536	-0.927759151394138\\
};
\addplot [color=red,solid,line width=1.2pt,forget plot]
  table[row sep=crcr]{%
2.72548062486558	-1.12038039364296\\
2.66979824000249	-0.928288040812459\\
};
\addplot [color=red,solid,line width=1.2pt,forget plot]
  table[row sep=crcr]{%
2.72216842183659	-1.12152116426574\\
2.66657412034117	-0.92940330033519\\
};
\addplot [color=red,solid,line width=1.2pt,forget plot]
  table[row sep=crcr]{%
2.71964630043502	-1.12237682744611\\
2.66411839672335	-0.930239762023046\\
};
\addplot [color=red,solid,line width=1.2pt,forget plot]
  table[row sep=crcr]{%
2.7180495602735	-1.12295490677775\\
2.66257087308134	-0.930803624535253\\
};
\addplot [color=red,solid,line width=1.2pt,forget plot]
  table[row sep=crcr]{%
2.71697300609514	-1.12335978982387\\
2.66157358607868	-0.931185638955859\\
};
\addplot [color=red,solid,line width=1.2pt,forget plot]
  table[row sep=crcr]{%
2.71609547628602	-1.12369740607883\\
2.66076435848399	-0.93150357838715\\
};
\addplot [color=red,solid,line width=1.2pt,forget plot]
  table[row sep=crcr]{%
2.71533552627702	-1.12401809277342\\
2.66003417636116	-0.931815697633597\\
};
\addplot [color=red,solid,line width=1.2pt,forget plot]
  table[row sep=crcr]{%
2.71465514029503	-1.12432285452844\\
2.65939342470617	-0.932109060043822\\
};
\addplot [color=red,solid,line width=1.2pt,forget plot]
  table[row sep=crcr]{%
2.71428427967025	-1.12448360737336\\
2.6590582418337	-0.932259559073272\\
};
\addplot [color=red,solid,line width=1.2pt,forget plot]
  table[row sep=crcr]{%
2.71404309908673	-1.12455898590498\\
2.65887233786048	-0.932319065218056\\
};
\addplot [color=red,solid,line width=1.2pt,forget plot]
  table[row sep=crcr]{%
2.71378258628846	-1.12464144986798\\
2.65865534933416	-0.932389043522088\\
};
\addplot [color=red,solid,line width=1.2pt,forget plot]
  table[row sep=crcr]{%
3.12339254551153	-3.25496738473394\\
3.06925154076959	-3.06243491576952\\
};
\addplot [color=red,solid,line width=1.2pt,forget plot]
  table[row sep=crcr]{%
3.1233387707352	-3.2550025962104\\
3.06940027166277	-3.06241329661717\\
};
\addplot [color=red,solid,line width=1.2pt,forget plot]
  table[row sep=crcr]{%
3.12376676987551	-3.25648630289264\\
3.07062172186484	-3.06367654206103\\
};
\addplot [color=red,solid,line width=1.2pt,forget plot]
  table[row sep=crcr]{%
3.12374480390832	-3.25649689645696\\
3.07068178025305	-3.06366454564268\\
};
\addplot [color=red,solid,line width=1.2pt,forget plot]
  table[row sep=crcr]{%
3.12375170464421	-3.25653284288832\\
3.07050402708775	-3.0637513997076\\
};
\addplot [color=red,solid,line width=1.2pt,forget plot]
  table[row sep=crcr]{%
3.12353409367879	-3.25662314942417\\
3.07051596062219	-3.06377845137106\\
};
\addplot [color=red,solid,line width=1.2pt,forget plot]
  table[row sep=crcr]{%
3.12347148433919	-3.25667077213196\\
3.07047464212932	-3.0638202219233\\
};
\addplot [color=red,solid,line width=1.2pt,forget plot]
  table[row sep=crcr]{%
3.12339840262076	-3.25671641185961\\
3.07044589966252	-3.06385368235892\\
};
\addplot [color=red,solid,line width=1.2pt,forget plot]
  table[row sep=crcr]{%
3.12329510832541	-3.25676167989728\\
3.07039544816985	-3.06388444963039\\
};
\addplot [color=red,solid,line width=1.2pt,forget plot]
  table[row sep=crcr]{%
3.12318928519336	-3.25680987046387\\
3.07028544133014	-3.0639337876944\\
};
\addplot [color=red,solid,line width=1.2pt,forget plot]
  table[row sep=crcr]{%
3.12313619769094	-3.256830131233\\
3.07023108903727	-3.06395439538646\\
};
\addplot [color=red,solid,line width=1.2pt,forget plot]
  table[row sep=crcr]{%
3.12308984885561	-3.25686680974213\\
3.07020808975857	-3.06398467071727\\
};
\addplot [color=red,solid,line width=1.2pt,forget plot]
  table[row sep=crcr]{%
3.12306383283588	-3.25688113477441\\
3.07017879758578	-3.06399989398984\\
};
\addplot [color=red,solid,line width=1.2pt,forget plot]
  table[row sep=crcr]{%
3.12297347361855	-3.25687773855625\\
3.07014114999778	-3.06398205280659\\
};
\addplot [color=red,solid,line width=1.2pt,forget plot]
  table[row sep=crcr]{%
3.12295145011465	-3.2569503786071\\
3.07013528976182	-3.06405026661754\\
};
\addplot [color=red,solid,line width=1.2pt,forget plot]
  table[row sep=crcr]{%
3.12279599227871	-3.25701041322952\\
3.0700411145819	-3.06409353247723\\
};
\addplot [color=red,solid,line width=1.2pt,forget plot]
  table[row sep=crcr]{%
3.122817215136	-3.25702436354762\\
3.069960508593	-3.06413535770896\\
};
\addplot [color=red,solid,line width=1.2pt,forget plot]
  table[row sep=crcr]{%
3.12276815371606	-3.25703904239761\\
3.06995415626976	-3.06413833821604\\
};
\addplot [color=red,solid,line width=1.2pt,forget plot]
  table[row sep=crcr]{%
3.12275939909845	-3.25708027640053\\
3.06996939800902	-3.06417300389641\\
};
\addplot [color=red,solid,line width=1.2pt,forget plot]
  table[row sep=crcr]{%
3.12272954621929	-3.25709831688657\\
3.06998048318349	-3.06417984615646\\
};
\addplot [color=red,solid,line width=1.2pt,forget plot]
  table[row sep=crcr]{%
3.12267708624686	-3.25710914535558\\
3.07000203207919	-3.06417045387968\\
};
\addplot [color=red,solid,line width=1.2pt,forget plot]
  table[row sep=crcr]{%
3.12271962612546	-3.2570815422452\\
3.06991346468218	-3.06417869282709\\
};
\addplot [color=red,solid,line width=1.2pt,forget plot]
  table[row sep=crcr]{%
3.12267129580539	-3.25715232896304\\
3.06995668269635	-3.06422444200844\\
};
\addplot [color=red,solid,line width=1.2pt,forget plot]
  table[row sep=crcr]{%
3.12265818751298	-3.25719859141573\\
3.06993908991929	-3.06427192983442\\
};
\addplot [color=red,solid,line width=1.2pt,forget plot]
  table[row sep=crcr]{%
3.12265645922004	-3.25720881845293\\
3.06992055176962	-3.06428675111645\\
};
\addplot [color=red,solid,line width=1.2pt,forget plot]
  table[row sep=crcr]{%
3.12264537047178	-3.25722574143062\\
3.06987673195871	-3.06431262422632\\
};
\addplot [color=red,solid,line width=1.2pt,forget plot]
  table[row sep=crcr]{%
3.12262728293139	-3.25728323442887\\
3.06986747150679	-3.06436770291717\\
};
\addplot [color=red,solid,line width=1.2pt,forget plot]
  table[row sep=crcr]{%
3.1226097519939	-3.25728180507727\\
3.0698479115772	-3.06436682847918\\
};
\addplot [color=red,solid,line width=1.2pt,forget plot]
  table[row sep=crcr]{%
3.12253239351241	-3.25735000430787\\
3.06992753192261	-3.06439216296132\\
};
\addplot [color=red,solid,line width=1.2pt,forget plot]
  table[row sep=crcr]{%
3.62792691163132	-5.3817850447959\\
3.57484328976422	-5.1889583633188\\
};
\addplot [color=red,solid,line width=1.2pt,forget plot]
  table[row sep=crcr]{%
3.62792714708389	-5.38167449047326\\
3.57471260756091	-5.18888389738491\\
};
\addplot [color=red,solid,line width=1.2pt,forget plot]
  table[row sep=crcr]{%
3.6279488717784	-5.38175971205585\\
3.57465286985398	-5.1889916229465\\
};
\addplot [color=red,solid,line width=1.2pt,forget plot]
  table[row sep=crcr]{%
3.62788376358526	-5.38178834013852\\
3.57462009531932	-5.18901131442525\\
};
\addplot [color=red,solid,line width=1.2pt,forget plot]
  table[row sep=crcr]{%
3.62784220349022	-5.38182205761978\\
3.57460620036677	-5.18903739025373\\
};
\addplot [color=red,solid,line width=1.2pt,forget plot]
  table[row sep=crcr]{%
3.62781538805546	-5.38188735448116\\
3.57459944693684	-5.18909714827021\\
};
\addplot [color=red,solid,line width=1.2pt,forget plot]
  table[row sep=crcr]{%
3.6277818917774	-5.38189302680837\\
3.5745963220648	-5.18909443974315\\
};
\addplot [color=red,solid,line width=1.2pt,forget plot]
  table[row sep=crcr]{%
3.62771413001586	-5.38191469547559\\
3.57453820706727	-5.18911344750618\\
};
\addplot [color=red,solid,line width=1.2pt,forget plot]
  table[row sep=crcr]{%
3.62765085627271	-5.38192499401824\\
3.57450340749773	-5.1891158949365\\
};
\addplot [color=red,solid,line width=1.2pt,forget plot]
  table[row sep=crcr]{%
3.62759955102723	-5.38195935677157\\
3.57449275788858	-5.18913905565541\\
};
\addplot [color=red,solid,line width=1.2pt,forget plot]
  table[row sep=crcr]{%
3.62758085843915	-5.38199447465085\\
3.57449072166384	-5.18916958679356\\
};
\addplot [color=red,solid,line width=1.2pt,forget plot]
  table[row sep=crcr]{%
3.62753064453154	-5.38200650844793\\
3.57444922550027	-5.18917922056153\\
};
\addplot [color=red,solid,line width=1.2pt,forget plot]
  table[row sep=crcr]{%
3.62748964493185	-5.38201233219154\\
3.57442130677963	-5.18918144388325\\
};
\addplot [color=red,solid,line width=1.2pt,forget plot]
  table[row sep=crcr]{%
3.62745955253568	-5.38204786216902\\
3.5744224527837	-5.18920837956781\\
};
\addplot [color=red,solid,line width=1.2pt,forget plot]
  table[row sep=crcr]{%
3.62737948504115	-5.38206558991608\\
3.57436514671021	-5.18921984863219\\
};
\addplot [color=red,solid,line width=1.2pt,forget plot]
  table[row sep=crcr]{%
3.62725735797315	-5.38210308643203\\
3.5742782634805	-5.18924765988992\\
};
\addplot [color=red,solid,line width=1.2pt,forget plot]
  table[row sep=crcr]{%
3.62721237957942	-5.3821218619032\\
3.57424649284734	-5.18926280755862\\
};
\addplot [color=red,solid,line width=1.2pt,forget plot]
  table[row sep=crcr]{%
3.62728961565884	-5.38208397969859\\
3.57430278052717	-5.18923067974568\\
};
\addplot [color=red,solid,line width=1.2pt,forget plot]
  table[row sep=crcr]{%
3.62731282480527	-5.3820809791956\\
3.57432339503353	-5.18922839214408\\
};
\addplot [color=red,solid,line width=1.2pt,forget plot]
  table[row sep=crcr]{%
3.62740383353142	-5.38208323794495\\
3.57439170443754	-5.18923688934344\\
};
\addplot [color=red,solid,line width=1.2pt,forget plot]
  table[row sep=crcr]{%
3.6274350066295	-5.38208703532103\\
3.57440213951707	-5.18924638865691\\
};
\addplot [color=red,solid,line width=1.2pt,forget plot]
  table[row sep=crcr]{%
3.62747143591182	-5.38206922492459\\
3.57439815285686	-5.18923969755451\\
};
\addplot [color=red,solid,line width=1.2pt,forget plot]
  table[row sep=crcr]{%
3.62754435837246	-5.38206832132822\\
3.57445288507935	-5.189243801457\\
};
\addplot [color=red,solid,line width=1.2pt,forget plot]
  table[row sep=crcr]{%
3.62764319964908	-5.38205429853548\\
3.57453922934412	-5.18923321997339\\
};
\addplot [color=red,solid,line width=1.2pt,forget plot]
  table[row sep=crcr]{%
3.62764067322917	-5.38207360536646\\
3.57451862409668	-5.18925750672331\\
};
\addplot [color=red,solid,line width=1.2pt,forget plot]
  table[row sep=crcr]{%
3.62767371998607	-5.38209250231551\\
3.57454603145379	-5.18927795745124\\
};
\addplot [color=red,solid,line width=1.2pt,forget plot]
  table[row sep=crcr]{%
3.62775445894366	-5.38209304100897\\
3.57462553241722	-5.18927883726312\\
};
\addplot [color=red,solid,line width=1.2pt,forget plot]
  table[row sep=crcr]{%
3.62774431769951	-5.38210867913685\\
3.57462411732485	-5.18929207115881\\
};
\addplot [color=red,solid,line width=1.2pt,forget plot]
  table[row sep=crcr]{%
3.62781952170295	-5.38235368737208\\
3.57474286922457	-5.18952508741427\\
};
\addplot [color=red,solid,line width=1.2pt,forget plot]
  table[row sep=crcr]{%
4.16399672928579	-7.31262733457744\\
4.11152092011563	-7.11963435685043\\
};
\addplot [color=red,solid,line width=1.2pt,forget plot]
  table[row sep=crcr]{%
4.16403973151436	-7.31259851805815\\
4.11148743810819	-7.11962635307638\\
};
\addplot [color=red,solid,line width=1.2pt,forget plot]
  table[row sep=crcr]{%
4.16405162470092	-7.31246229804824\\
4.11141769754958	-7.11951238301165\\
};
\addplot [color=red,solid,line width=1.2pt,forget plot]
  table[row sep=crcr]{%
4.16407912588852	-7.31235305397002\\
4.11141630626522	-7.11941102270331\\
};
\addplot [color=red,solid,line width=1.2pt,forget plot]
  table[row sep=crcr]{%
4.16409316604282	-7.31235134980993\\
4.1114359313235	-7.11940779425101\\
};
\addplot [color=red,solid,line width=1.2pt,forget plot]
  table[row sep=crcr]{%
4.1640898423293	-7.31233488685725\\
4.11141809873195	-7.11939529157828\\
};
\addplot [color=red,solid,line width=1.2pt,forget plot]
  table[row sep=crcr]{%
4.16413795539897	-7.31233018876114\\
4.11142505630924	-7.119401833485\\
};
\addplot [color=red,solid,line width=1.2pt,forget plot]
  table[row sep=crcr]{%
4.16416309110244	-7.31231070805336\\
4.11139851432507	-7.11939647986501\\
};
\addplot [color=red,solid,line width=1.2pt,forget plot]
  table[row sep=crcr]{%
4.16417603078786	-7.31229248108551\\
4.1113949621932	-7.11938276438362\\
};
\addplot [color=red,solid,line width=1.2pt,forget plot]
  table[row sep=crcr]{%
4.16420896961224	-7.3122895029844\\
4.11143444407811	-7.1193779961875\\
};
\addplot [color=red,solid,line width=1.2pt,forget plot]
  table[row sep=crcr]{%
4.16423714736973	-7.31231522154008\\
4.11144473960544	-7.11940860765023\\
};
\addplot [color=red,solid,line width=1.2pt,forget plot]
  table[row sep=crcr]{%
4.16424811613721	-7.31233755436762\\
4.11144167804681	-7.11943478068044\\
};
\addplot [color=red,solid,line width=1.2pt,forget plot]
  table[row sep=crcr]{%
4.1642370217055	-7.31232380898773\\
4.1114043608419	-7.11942821560605\\
};
\addplot [color=red,solid,line width=1.2pt,forget plot]
  table[row sep=crcr]{%
4.16421237535127	-7.31235891191247\\
4.11140626752554	-7.11945604781684\\
};
\addplot [color=red,solid,line width=1.2pt,forget plot]
  table[row sep=crcr]{%
4.16418552775142	-7.3123777405866\\
4.11137357871849	-7.11947647558467\\
};
\addplot [color=red,solid,line width=1.2pt,forget plot]
  table[row sep=crcr]{%
4.16416769567455	-7.31238070261661\\
4.11136364291191	-7.11947727597042\\
};
\addplot [color=red,solid,line width=1.2pt,forget plot]
  table[row sep=crcr]{%
4.16415702400074	-7.31238438128374\\
4.11134933999305	-7.11948194866626\\
};
\addplot [color=red,solid,line width=1.2pt,forget plot]
  table[row sep=crcr]{%
4.16417284550737	-7.31239924145817\\
4.11136121577622	-7.11949788903901\\
};
\addplot [color=red,solid,line width=1.2pt,forget plot]
  table[row sep=crcr]{%
4.16419201185779	-7.31242638206787\\
4.11138308637278	-7.11952428931323\\
};
\addplot [color=red,solid,line width=1.2pt,forget plot]
  table[row sep=crcr]{%
4.16419963876387	-7.31238095674985\\
4.11138069138885	-7.1194816078702\\
};
\addplot [color=red,solid,line width=1.2pt,forget plot]
  table[row sep=crcr]{%
4.1641930543362	-7.31237322907125\\
4.11135400194632	-7.11947938639466\\
};
\addplot [color=red,solid,line width=1.2pt,forget plot]
  table[row sep=crcr]{%
4.16419270507346	-7.31238564791629\\
4.1113643971573	-7.11948886234765\\
};
\addplot [color=red,solid,line width=1.2pt,forget plot]
  table[row sep=crcr]{%
4.16419852102833	-7.31240885630659\\
4.11137935209122	-7.11950956809434\\
};
\addplot [color=red,solid,line width=1.2pt,forget plot]
  table[row sep=crcr]{%
4.16422708841059	-7.31238322889336\\
4.11138624099433	-7.119489877934\\
};
\addplot [color=red,solid,line width=1.2pt,forget plot]
  table[row sep=crcr]{%
4.16425408079999	-7.31236957507205\\
4.11141825052262	-7.11947484979709\\
};
\addplot [color=red,solid,line width=1.2pt,forget plot]
  table[row sep=crcr]{%
4.16422144261734	-7.31237297809544\\
4.11139413734125	-7.1194759179383\\
};
\addplot [color=red,solid,line width=1.2pt,forget plot]
  table[row sep=crcr]{%
4.16419247858778	-7.31239186344137\\
4.11148969118612	-7.11946074568144\\
};
\addplot [color=red,solid,line width=1.2pt,forget plot]
  table[row sep=crcr]{%
4.1641348578812	-7.31238765456779\\
4.11150291337383	-7.11943719870574\\
};
\addplot [color=red,solid,line width=1.2pt,forget plot]
  table[row sep=crcr]{%
2.83273515885158	-7.96137696679332\\
3.02541542425993	-7.90776432535338\\
};
\addplot [color=red,solid,line width=1.2pt,forget plot]
  table[row sep=crcr]{%
2.83276626867061	-7.9614512816798\\
3.02543001086515	-7.90777929230362\\
};
\addplot [color=red,solid,line width=1.2pt,forget plot]
  table[row sep=crcr]{%
2.83278890213492	-7.9614575317295\\
3.02543323396928	-7.90771591459629\\
};
\addplot [color=red,solid,line width=1.2pt,forget plot]
  table[row sep=crcr]{%
2.83281314400673	-7.96143760703277\\
3.02546393220456	-7.90771913900959\\
};
\addplot [color=red,solid,line width=1.2pt,forget plot]
  table[row sep=crcr]{%
2.8328468956159	-7.96146149042752\\
3.02549789507327	-7.90774378005117\\
};
\addplot [color=red,solid,line width=1.2pt,forget plot]
  table[row sep=crcr]{%
2.83286562475353	-7.9614948898471\\
3.02551290997528	-7.90776386066759\\
};
\addplot [color=red,solid,line width=1.2pt,forget plot]
  table[row sep=crcr]{%
2.8328770564261	-7.96153785501002\\
3.02553246759694	-7.90783596913444\\
};
\addplot [color=red,solid,line width=1.2pt,forget plot]
  table[row sep=crcr]{%
2.83288213541696	-7.96156686991736\\
3.02553244813727	-7.90784669671617\\
};
\addplot [color=red,solid,line width=1.2pt,forget plot]
  table[row sep=crcr]{%
2.83291535286984	-7.96157132190372\\
3.02556825141773	-7.90786042281338\\
};
\addplot [color=red,solid,line width=1.2pt,forget plot]
  table[row sep=crcr]{%
2.83289275225864	-7.96155710400478\\
3.02554480193515	-7.90784316023373\\
};
\addplot [color=red,solid,line width=1.2pt,forget plot]
  table[row sep=crcr]{%
2.8328893533749	-7.96158719546835\\
3.02553192926782	-7.90783928431504\\
};
\addplot [color=red,solid,line width=1.2pt,forget plot]
  table[row sep=crcr]{%
2.83288722740314	-7.96158255165114\\
3.02552085506643	-7.90780257867531\\
};
\addplot [color=red,solid,line width=1.2pt,forget plot]
  table[row sep=crcr]{%
2.83293387951975	-7.96155922444814\\
3.02555967029846	-7.90775118855314\\
};
\addplot [color=red,solid,line width=1.2pt,forget plot]
  table[row sep=crcr]{%
2.83293450752099	-7.96155867510143\\
3.02555363631786	-7.90772679587049\\
};
\addplot [color=red,solid,line width=1.2pt,forget plot]
  table[row sep=crcr]{%
2.83294410514408	-7.9615633985046\\
3.02555842665786	-7.90771432100288\\
};
\addplot [color=red,solid,line width=1.2pt,forget plot]
  table[row sep=crcr]{%
2.83294844670749	-7.96155918411219\\
3.02555964571635	-7.90769893888094\\
};
\addplot [color=red,solid,line width=1.2pt,forget plot]
  table[row sep=crcr]{%
2.83293933748666	-7.96157440119063\\
3.02554063436001	-7.90767875718569\\
};
\addplot [color=red,solid,line width=1.2pt,forget plot]
  table[row sep=crcr]{%
2.83294582077536	-7.96160978674275\\
3.0255428326848	-7.90769883234987\\
};
\addplot [color=red,solid,line width=1.2pt,forget plot]
  table[row sep=crcr]{%
2.83298545239582	-7.96164809654082\\
3.02558003623411	-7.90772846861036\\
};
\addplot [color=red,solid,line width=1.2pt,forget plot]
  table[row sep=crcr]{%
2.83301061594553	-7.96167265769842\\
3.02559759202315	-7.90772586312007\\
};
\addplot [color=red,solid,line width=1.2pt,forget plot]
  table[row sep=crcr]{%
2.8330617031707	-7.9616700405772\\
3.02563585375881	-7.90767748067059\\
};
\addplot [color=red,solid,line width=1.2pt,forget plot]
  table[row sep=crcr]{%
2.83305876645876	-7.96167159461202\\
3.02563166944787	-7.90767458511754\\
};
\addplot [color=red,solid,line width=1.2pt,forget plot]
  table[row sep=crcr]{%
0.989058567856044	-8.47574692188387\\
1.18178039560768	-8.42228387756288\\
};
\addplot [color=red,solid,line width=1.2pt,forget plot]
  table[row sep=crcr]{%
0.988991320624244	-8.47570407421069\\
1.18171223031804	-8.42223772061524\\
};
\addplot [color=red,solid,line width=1.2pt,forget plot]
  table[row sep=crcr]{%
0.988908935962349	-8.47578478663917\\
1.18161468299732	-8.42226380888565\\
};
\addplot [color=red,solid,line width=1.2pt,forget plot]
  table[row sep=crcr]{%
0.988874165629141	-8.47577820598396\\
1.18157200646096	-8.42222876958398\\
};
\addplot [color=red,solid,line width=1.2pt,forget plot]
  table[row sep=crcr]{%
0.98885608900654	-8.47579351670547\\
1.18155197280234	-8.42223703840306\\
};
\addplot [color=red,solid,line width=1.2pt,forget plot]
  table[row sep=crcr]{%
0.98884321331392	-8.47577815524973\\
1.18154468155234	-8.42224177379974\\
};
\addplot [color=red,solid,line width=1.2pt,forget plot]
  table[row sep=crcr]{%
0.988840664158221	-8.47572107658252\\
1.18154938950888	-8.42221082360496\\
};
\addplot [color=red,solid,line width=1.2pt,forget plot]
  table[row sep=crcr]{%
0.988907982099653	-8.47565084397288\\
1.1816211110382	-8.4221564523536\\
};
\addplot [color=red,solid,line width=1.2pt,forget plot]
  table[row sep=crcr]{%
0.988956618478957	-8.47565425647501\\
1.18167362026772	-8.42217381872858\\
};
\addplot [color=red,solid,line width=1.2pt,forget plot]
  table[row sep=crcr]{%
0.988946872959977	-8.47568643541257\\
1.18167456065717	-8.42224451932703\\
};
\addplot [color=red,solid,line width=1.2pt,forget plot]
  table[row sep=crcr]{%
0.988899903882687	-8.47572055956313\\
1.18163297674538	-8.42229806781196\\
};
\addplot [color=red,solid,line width=1.2pt,forget plot]
  table[row sep=crcr]{%
0.988855325932502	-8.47569197056396\\
1.18159845337341	-8.42230576612596\\
};
\addplot [color=red,solid,line width=1.2pt,forget plot]
  table[row sep=crcr]{%
0.988794249968762	-8.47572510460992\\
1.1815447627786	-8.42236557114548\\
};
\addplot [color=red,solid,line width=1.2pt,forget plot]
  table[row sep=crcr]{%
0.988775914823935	-8.47572765598213\\
1.18153365788819	-8.42239424723324\\
};
\addplot [color=red,solid,line width=1.2pt,forget plot]
  table[row sep=crcr]{%
0.988767218806724	-8.4756709263743\\
1.18154196709768	-8.42239901612736\\
};
\addplot [color=red,solid,line width=1.2pt,forget plot]
  table[row sep=crcr]{%
0.988405723102323	-8.47561145247018\\
1.18123456760527	-8.42253568843745\\
};
\addplot [color=red,solid,line width=1.2pt,forget plot]
  table[row sep=crcr]{%
-0.840951327095443	-8.98961910578651\\
-0.64837465373966	-8.93563554459219\\
};
\addplot [color=red,solid,line width=1.2pt,forget plot]
  table[row sep=crcr]{%
-0.840969336085729	-8.98958004519561\\
-0.648375007262953	-8.93565950638799\\
};
\addplot [color=red,solid,line width=1.2pt,forget plot]
  table[row sep=crcr]{%
-0.840934268749582	-8.98949111960114\\
-0.648316352657205	-8.93565490130685\\
};
\addplot [color=red,solid,line width=1.2pt,forget plot]
  table[row sep=crcr]{%
-0.840889549896285	-8.98942256140225\\
-0.648260729410683	-8.93562537264368\\
};
\addplot [color=red,solid,line width=1.2pt,forget plot]
  table[row sep=crcr]{%
-0.840917109007914	-8.98935870132354\\
-0.648258038200816	-8.9356699461159\\
};
\addplot [color=red,solid,line width=1.2pt,forget plot]
  table[row sep=crcr]{%
-0.840898757630167	-8.98941387640843\\
-0.64825986558159	-8.93565276341496\\
};
\addplot [color=red,solid,line width=1.2pt,forget plot]
  table[row sep=crcr]{%
-0.840930621539613	-8.98938721737371\\
-0.648289792018937	-8.93563304728816\\
};
\addplot [color=red,solid,line width=1.2pt,forget plot]
  table[row sep=crcr]{%
-0.840967393048533	-8.9893697860301\\
-0.648317230875709	-8.93564907294116\\
};
\addplot [color=red,solid,line width=1.2pt,forget plot]
  table[row sep=crcr]{%
-0.841028746585764	-8.98936044747446\\
-0.648379946068053	-8.93563485153465\\
};
\addplot [color=red,solid,line width=1.2pt,forget plot]
  table[row sep=crcr]{%
-0.841010535887705	-8.98936354972894\\
-0.648356366564368	-8.93565720892699\\
};
\addplot [color=red,solid,line width=1.2pt,forget plot]
  table[row sep=crcr]{%
-0.841004010965351	-8.98935233281763\\
-0.648353002113508	-8.93563465613326\\
};
\addplot [color=red,solid,line width=1.2pt,forget plot]
  table[row sep=crcr]{%
-0.841003118829623	-8.98933962881707\\
-0.648351771867729	-8.93562316473233\\
};
\addplot [color=red,solid,line width=1.2pt,forget plot]
  table[row sep=crcr]{%
-0.841068693194555	-8.98932709061029\\
-0.648411945949306	-8.9356299981296\\
};
\addplot [color=red,solid,line width=1.2pt,forget plot]
  table[row sep=crcr]{%
-0.841103217097013	-8.98936204234763\\
-0.648447246331972	-8.93566216405559\\
};
\addplot [color=red,solid,line width=1.2pt,forget plot]
  table[row sep=crcr]{%
-0.841089620986318	-8.98939303072866\\
-0.648430490234881	-8.93570449062837\\
};
\addplot [color=red,solid,line width=1.2pt,forget plot]
  table[row sep=crcr]{%
-0.841110281151794	-8.98942242995646\\
-0.648451916365098	-8.93573114129884\\
};
\addplot [color=red,solid,line width=1.2pt,forget plot]
  table[row sep=crcr]{%
-0.841094376088217	-8.98942126746477\\
-0.648415195213031	-8.93580472843746\\
};
\addplot [color=red,solid,line width=1.2pt,forget plot]
  table[row sep=crcr]{%
-0.841171216271133	-8.9893937928384\\
-0.648483727576798	-8.93580711818256\\
};
\addplot [color=red,solid,line width=1.2pt,forget plot]
  table[row sep=crcr]{%
-3.11466645731798	-8.81597691332251\\
-3.06181813354695	-9.00886821606733\\
};
\addplot [color=red,solid,line width=1.2pt,forget plot]
  table[row sep=crcr]{%
-3.11445689571369	-8.81598352298851\\
-3.06165611980612	-9.00888784658847\\
};
\addplot [color=red,solid,line width=1.2pt,forget plot]
  table[row sep=crcr]{%
-3.11433752662194	-8.81593119369752\\
-3.06171844694289	-9.00888515829459\\
};
\addplot [color=red,solid,line width=1.2pt,forget plot]
  table[row sep=crcr]{%
-3.11429454716138	-8.81596804357749\\
-3.06171424367865	-9.00893257836475\\
};
\addplot [color=red,solid,line width=1.2pt,forget plot]
  table[row sep=crcr]{%
-3.11428290277799	-8.81596298553043\\
-3.06172708306463	-9.00893419015499\\
};
\addplot [color=red,solid,line width=1.2pt,forget plot]
  table[row sep=crcr]{%
-3.11431271617608	-8.81598432948375\\
-3.06176519874743	-9.00895779504854\\
};
\addplot [color=red,solid,line width=1.2pt,forget plot]
  table[row sep=crcr]{%
-3.11419578195424	-8.81600422585626\\
-3.06179800880502	-9.00901840504785\\
};
\addplot [color=red,solid,line width=1.2pt,forget plot]
  table[row sep=crcr]{%
-3.11415479228543	-8.81604044155213\\
-3.06177737742108	-9.00906014627714\\
};
\addplot [color=red,solid,line width=1.2pt,forget plot]
  table[row sep=crcr]{%
-3.11410439974035	-8.81604039996003\\
-3.06177043580032	-9.00907189018461\\
};
\addplot [color=red,solid,line width=1.2pt,forget plot]
  table[row sep=crcr]{%
-3.11411265467411	-8.81604675241468\\
-3.06177169861462	-9.00907634682647\\
};
\addplot [color=red,solid,line width=1.2pt,forget plot]
  table[row sep=crcr]{%
-3.1141845750344	-8.8160590713064\\
-3.06185454006884	-9.00909162669669\\
};
\addplot [color=red,solid,line width=1.2pt,forget plot]
  table[row sep=crcr]{%
-3.11419979706504	-8.81607343538913\\
-3.06183324935132	-9.00909608867248\\
};
\addplot [color=red,solid,line width=1.2pt,forget plot]
  table[row sep=crcr]{%
-3.11428261400083	-8.81608897765093\\
-3.06173232341158	-9.00906168805058\\
};
\addplot [color=red,solid,line width=1.2pt,forget plot]
  table[row sep=crcr]{%
-3.11427089064877	-8.81609050080151\\
-3.06170485440552	-9.00905892265771\\
};
\addplot [color=red,solid,line width=1.2pt,forget plot]
  table[row sep=crcr]{%
-3.11426399164279	-8.81609148005116\\
-3.06170280722101	-9.00906122351417\\
};
\addplot [color=red,solid,line width=1.2pt,forget plot]
  table[row sep=crcr]{%
-3.11425835360647	-8.81608997718623\\
-3.06170164812798	-9.00906094056997\\
};
\addplot [color=red,solid,line width=1.2pt,forget plot]
  table[row sep=crcr]{%
-3.11422449788094	-8.81612393100246\\
-3.06167075238824	-9.00909570053026\\
};
\addplot [color=red,solid,line width=1.2pt,forget plot]
  table[row sep=crcr]{%
-3.11424133858575	-8.81614180256958\\
-3.06166351234616	-9.00910701235601\\
};
\addplot [color=red,solid,line width=1.2pt,forget plot]
  table[row sep=crcr]{%
-3.11425401911344	-8.81616231367889\\
-3.0616727677893	-9.00912659018931\\
};
\addplot [color=red,solid,line width=1.2pt,forget plot]
  table[row sep=crcr]{%
-3.11420596827169	-8.81616820498003\\
-3.06161721549843	-9.00913043724778\\
};
\addplot [color=red,solid,line width=1.2pt,forget plot]
  table[row sep=crcr]{%
-3.11404154886432	-8.81611191745818\\
-3.06183313487049	-9.00917740248399\\
};
\addplot [color=red,solid,line width=1.2pt,forget plot]
  table[row sep=crcr]{%
-3.60949388806684	-6.96436464173963\\
-3.55708147931421	-7.15737484719196\\
};
\addplot [color=red,solid,line width=1.2pt,forget plot]
  table[row sep=crcr]{%
-3.60944199293146	-6.96434738001053\\
-3.5570534534834	-7.15736406564806\\
};
\addplot [color=red,solid,line width=1.2pt,forget plot]
  table[row sep=crcr]{%
-3.60942424427379	-6.96440400166756\\
-3.55707960476305	-7.15743259725508\\
};
\addplot [color=red,solid,line width=1.2pt,forget plot]
  table[row sep=crcr]{%
-3.60932962494082	-6.96435432643633\\
-3.55704172030882	-7.1573982981868\\
};
\addplot [color=red,solid,line width=1.2pt,forget plot]
  table[row sep=crcr]{%
-3.60924661533309	-6.96437551802001\\
-3.55699489531138	-7.15742928709685\\
};
\addplot [color=red,solid,line width=1.2pt,forget plot]
  table[row sep=crcr]{%
-3.60923208516095	-6.9643759928885\\
-3.55701274640365	-7.15743852332812\\
};
\addplot [color=red,solid,line width=1.2pt,forget plot]
  table[row sep=crcr]{%
-3.60922492857148	-6.96435036204892\\
-3.55702383726274	-7.15741782711276\\
};
\addplot [color=red,solid,line width=1.2pt,forget plot]
  table[row sep=crcr]{%
-3.60923315071102	-6.96434201755989\\
-3.55702521945885	-7.15740763312716\\
};
\addplot [color=red,solid,line width=1.2pt,forget plot]
  table[row sep=crcr]{%
-3.60918118111629	-6.96440886678662\\
-3.55698686197919	-7.15747816276559\\
};
\addplot [color=red,solid,line width=1.2pt,forget plot]
  table[row sep=crcr]{%
-3.60923757101502	-6.96437932228343\\
-3.55705302060547	-7.15745125887327\\
};
\addplot [color=red,solid,line width=1.2pt,forget plot]
  table[row sep=crcr]{%
-3.60925695565767	-6.96438331726032\\
-3.55708837851372	-7.15745957048369\\
};
\addplot [color=red,solid,line width=1.2pt,forget plot]
  table[row sep=crcr]{%
-3.60926009679975	-6.96442547236604\\
-3.55709546462756	-7.15750279146473\\
};
\addplot [color=red,solid,line width=1.2pt,forget plot]
  table[row sep=crcr]{%
-3.60925972541308	-6.96441611256404\\
-3.55711148039691	-7.15749785831369\\
};
\addplot [color=red,solid,line width=1.2pt,forget plot]
  table[row sep=crcr]{%
-3.6091823680352	-6.96442580881528\\
-3.55702344857554	-7.15750467125444\\
};
\addplot [color=red,solid,line width=1.2pt,forget plot]
  table[row sep=crcr]{%
-3.60917787960933	-6.96438620934544\\
-3.55704059658683	-7.15747091541776\\
};
\addplot [color=red,solid,line width=1.2pt,forget plot]
  table[row sep=crcr]{%
-4.07722175536392	-5.25146317918601\\
-4.02479945195866	-5.44447069744809\\
};
\addplot [color=red,solid,line width=1.2pt,forget plot]
  table[row sep=crcr]{%
-4.07718187558584	-5.25143519006487\\
-4.02482315719154	-5.4444599672519\\
};
\addplot [color=red,solid,line width=1.2pt,forget plot]
  table[row sep=crcr]{%
-4.07719881127637	-5.25147922936637\\
-4.02483783548033	-5.44450339421028\\
};
\addplot [color=red,solid,line width=1.2pt,forget plot]
  table[row sep=crcr]{%
-4.07719542961225	-5.25147950772346\\
-4.02485323762075	-5.44450876700175\\
};
\addplot [color=red,solid,line width=1.2pt,forget plot]
  table[row sep=crcr]{%
-4.07719010892683	-5.25154527328719\\
-4.0248670323415	-5.4445797149204\\
};
\addplot [color=red,solid,line width=1.2pt,forget plot]
  table[row sep=crcr]{%
-4.07717579179151	-5.25157016107581\\
-4.02486020880143	-5.44460663373424\\
};
\addplot [color=red,solid,line width=1.2pt,forget plot]
  table[row sep=crcr]{%
-4.07711985259453	-5.2515407064748\\
-4.02482256321811	-5.44458213602783\\
};
\addplot [color=red,solid,line width=1.2pt,forget plot]
  table[row sep=crcr]{%
-4.07708557257887	-5.25153825956458\\
-4.0248064474177	-5.44458460910889\\
};
\addplot [color=red,solid,line width=1.2pt,forget plot]
  table[row sep=crcr]{%
-4.07706076961898	-5.25154778197881\\
-4.02479381737901	-5.44459742767531\\
};
\addplot [color=red,solid,line width=1.2pt,forget plot]
  table[row sep=crcr]{%
-4.07707319426528	-5.25153141396868\\
-4.02482618258797	-5.44458645733919\\
};
\addplot [color=red,solid,line width=1.2pt,forget plot]
  table[row sep=crcr]{%
-4.07708754978551	-5.25153588840879\\
-4.0248627640293	-5.44459694546773\\
};
\addplot [color=red,solid,line width=1.2pt,forget plot]
  table[row sep=crcr]{%
-4.07708683456653	-5.25157124035032\\
-4.02489240818633	-5.44464050733718\\
};
\addplot [color=red,solid,line width=1.2pt,forget plot]
  table[row sep=crcr]{%
-4.07708127912935	-5.25161879952952\\
-4.02490486142541	-5.4446929340882\\
};
\addplot [color=red,solid,line width=1.2pt,forget plot]
  table[row sep=crcr]{%
-4.07707907845988	-5.25164512098617\\
-4.02492856271576	-5.44472625343474\\
};
\addplot [color=red,solid,line width=1.2pt,forget plot]
  table[row sep=crcr]{%
-4.07708504374305	-5.25166822743445\\
-4.02497086634713	-5.44475917107072\\
};
\addplot [color=red,solid,line width=1.2pt,forget plot]
  table[row sep=crcr]{%
-4.07708459318192	-5.25167615181746\\
-4.02496560301911	-5.44476579645022\\
};
\addplot [color=red,solid,line width=1.2pt,forget plot]
  table[row sep=crcr]{%
-4.07710406064076	-5.25163074993134\\
-4.02498646504042	-5.4447207709807\\
};
\addplot [color=red,solid,line width=1.2pt,forget plot]
  table[row sep=crcr]{%
-4.65043567834519	-3.13744041489101\\
-4.59764195744807	-3.33034666941335\\
};
\addplot [color=red,solid,line width=1.2pt,forget plot]
  table[row sep=crcr]{%
-4.65044652501288	-3.13699509479041\\
-4.59793805572826	-3.32997918909195\\
};
\addplot [color=red,solid,line width=1.2pt,forget plot]
  table[row sep=crcr]{%
-4.65039949044647	-3.13696369950012\\
-4.59793752875759	-3.32996044189687\\
};
\addplot [color=red,solid,line width=1.2pt,forget plot]
  table[row sep=crcr]{%
-4.65035482542913	-3.13698462166318\\
-4.59794587452243	-3.32999576606904\\
};
\addplot [color=red,solid,line width=1.2pt,forget plot]
  table[row sep=crcr]{%
-4.65035016352318	-3.1369712755324\\
-4.59795545980165	-3.32998628795872\\
};
\addplot [color=red,solid,line width=1.2pt,forget plot]
  table[row sep=crcr]{%
-4.65033581781955	-3.13695184990516\\
-4.59792771017014	-3.32996322328147\\
};
\addplot [color=red,solid,line width=1.2pt,forget plot]
  table[row sep=crcr]{%
-4.65031218927534	-3.13696825563639\\
-4.59787605593296	-3.32997201705077\\
};
\addplot [color=red,solid,line width=1.2pt,forget plot]
  table[row sep=crcr]{%
-4.65032006111981	-3.13697325193732\\
-4.59786614414152	-3.32997218093341\\
};
\addplot [color=red,solid,line width=1.2pt,forget plot]
  table[row sep=crcr]{%
-4.65034184325442	-3.13698700867587\\
-4.5978924677789	-3.32998717191991\\
};
\addplot [color=red,solid,line width=1.2pt,forget plot]
  table[row sep=crcr]{%
-4.6503246804797	-3.13695192331901\\
-4.5978942026647	-3.32995722116363\\
};
\addplot [color=red,solid,line width=1.2pt,forget plot]
  table[row sep=crcr]{%
-4.65028165334781	-3.13697586956703\\
-4.59784637982266	-3.32997986457839\\
};
\addplot [color=red,solid,line width=1.2pt,forget plot]
  table[row sep=crcr]{%
-4.65035013729007	-3.13699932673919\\
-4.59789672312996	-3.32999839239223\\
};
\addplot [color=red,solid,line width=1.2pt,forget plot]
  table[row sep=crcr]{%
-4.65033027105456	-3.13707965238495\\
-4.59787709148402	-3.33007878179475\\
};
\addplot [color=red,solid,line width=1.2pt,forget plot]
  table[row sep=crcr]{%
-4.65030699434325	-3.13713106183609\\
-4.59786501755503	-3.33013323558174\\
};
\addplot [color=red,solid,line width=1.2pt,forget plot]
  table[row sep=crcr]{%
-4.65027544746334	-3.13715097789849\\
-4.59783451472826	-3.33015343532811\\
};
\addplot [color=red,solid,line width=1.2pt,forget plot]
  table[row sep=crcr]{%
-4.65024252422312	-3.13715446136787\\
-4.59777804146939	-3.33015051845035\\
};
\addplot [color=red,solid,line width=1.2pt,forget plot]
  table[row sep=crcr]{%
-4.65027068654089	-3.1372423904892\\
-4.59780317149093	-3.33023762323986\\
};
\addplot [color=red,solid,line width=1.2pt,forget plot]
  table[row sep=crcr]{%
-4.65028900463004	-3.13731251856515\\
-4.59781475505792	-3.33030592034833\\
};
\addplot [color=red,solid,line width=1.2pt,forget plot]
  table[row sep=crcr]{%
-4.65031738133959	-3.13735602142155\\
-4.59784583617228	-3.33035015850285\\
};
\addplot [color=red,solid,line width=1.2pt,forget plot]
  table[row sep=crcr]{%
-4.65033892520491	-3.13739882374499\\
-4.59786598920818	-3.33039258268004\\
};
\addplot [color=red,solid,line width=1.2pt,forget plot]
  table[row sep=crcr]{%
-4.65035061340256	-3.13738232404189\\
-4.59790078054243	-3.33038236298734\\
};
\addplot [color=red,solid,line width=1.2pt,forget plot]
  table[row sep=crcr]{%
-4.6503429102606	-3.13740736336153\\
-4.59789794420285	-3.33040872484689\\
};
\addplot [color=red,solid,line width=1.2pt,forget plot]
  table[row sep=crcr]{%
-4.65028077090324	-3.13741013176215\\
-4.59786943405441	-3.33042062828946\\
};
\addplot [color=red,solid,line width=1.2pt,forget plot]
  table[row sep=crcr]{%
-4.65029935476247	-3.13740534832732\\
-4.59789517208334	-3.33041778740246\\
};
\addplot [color=red,solid,line width=1.2pt,forget plot]
  table[row sep=crcr]{%
-4.65007671230344	-3.13728926788881\\
-4.59772284864006	-3.33031536187487\\
};
\addplot [color=red,solid,line width=1.2pt,forget plot]
  table[row sep=crcr]{%
-4.34104132606132	-1.17130097217673\\
-4.53408669885611	-1.22358370395809\\
};
\addplot [color=red,solid,line width=1.2pt,forget plot]
  table[row sep=crcr]{%
-4.34098742650607	-1.17133989057043\\
-4.53404629386827	-1.2235727703764\\
};
\addplot [color=red,solid,line width=1.2pt,forget plot]
  table[row sep=crcr]{%
-4.34117735436168	-1.17136036804703\\
-4.53423365389331	-1.22360273793158\\
};
\addplot [color=red,solid,line width=1.2pt,forget plot]
  table[row sep=crcr]{%
-4.3411755120133	-1.17134962605472\\
-4.53424087433463	-1.22355849380472\\
};
\addplot [color=red,solid,line width=1.2pt,forget plot]
  table[row sep=crcr]{%
-4.34115725475568	-1.17133956476028\\
-4.53424343173827	-1.22347140012262\\
};
\addplot [color=red,solid,line width=1.2pt,forget plot]
  table[row sep=crcr]{%
-4.34112514416865	-1.17134555028692\\
-4.53421933851732	-1.22344768180605\\
};
\addplot [color=red,solid,line width=1.2pt,forget plot]
  table[row sep=crcr]{%
-4.34114855962748	-1.17138197410476\\
-4.53424233278799	-1.22348566655198\\
};
\addplot [color=red,solid,line width=1.2pt,forget plot]
  table[row sep=crcr]{%
-4.34118197337663	-1.17135442797254\\
-4.53427433911407	-1.2234633359818\\
};
\addplot [color=red,solid,line width=1.2pt,forget plot]
  table[row sep=crcr]{%
-4.34119245458588	-1.17131129798409\\
-4.53427955062851	-1.22343972920762\\
};
\addplot [color=red,solid,line width=1.2pt,forget plot]
  table[row sep=crcr]{%
-4.34120154878111	-1.17127272304007\\
-4.5342833670129	-1.22342069968643\\
};
\addplot [color=red,solid,line width=1.2pt,forget plot]
  table[row sep=crcr]{%
-4.34120607473532	-1.17127886699107\\
-4.53428208093693	-1.22344835835561\\
};
\addplot [color=red,solid,line width=1.2pt,forget plot]
  table[row sep=crcr]{%
-4.34122594209118	-1.17128214903434\\
-4.53429721599055	-1.22346915119842\\
};
\addplot [color=red,solid,line width=1.2pt,forget plot]
  table[row sep=crcr]{%
-4.34125952264954	-1.17129435743403\\
-4.53431792524421	-1.22352895504514\\
};
\addplot [color=red,solid,line width=1.2pt,forget plot]
  table[row sep=crcr]{%
-4.34126510291528	-1.17128698367623\\
-4.53431763933842	-1.22354325775498\\
};
\addplot [color=red,solid,line width=1.2pt,forget plot]
  table[row sep=crcr]{%
-4.34128394352549	-1.17126301364927\\
-4.53433239804148	-1.22353436535349\\
};
\addplot [color=red,solid,line width=1.2pt,forget plot]
  table[row sep=crcr]{%
-4.34135824555729	-1.17128682722234\\
-4.53440590729935	-1.2235611067097\\
};
\addplot [color=red,solid,line width=1.2pt,forget plot]
  table[row sep=crcr]{%
-4.34143087415584	-1.17119789175707\\
-4.53447056883938	-1.22350158451852\\
};
\addplot [color=red,solid,line width=1.2pt,forget plot]
  table[row sep=crcr]{%
-4.34144740722502	-1.17121638149178\\
-4.53448295419037	-1.22353538002758\\
};
\addplot [color=red,solid,line width=1.2pt,forget plot]
  table[row sep=crcr]{%
-4.3414953040314	-1.17115421076464\\
-4.53452512876908	-1.2234943173918\\
};
\addplot [color=red,solid,line width=1.2pt,forget plot]
  table[row sep=crcr]{%
-4.34154407931811	-1.17111536387053\\
-4.53456982869058	-1.22347049809056\\
};
\addplot [color=red,solid,line width=1.2pt,forget plot]
  table[row sep=crcr]{%
-4.34155259278643	-1.17114543049754\\
-4.53458096742321	-1.22349088478998\\
};
\addplot [color=red,solid,line width=1.2pt,forget plot]
  table[row sep=crcr]{%
-2.16222045041634	-0.570240123192765\\
-2.35538160476978	-0.622093459139915\\
};
\addplot [color=red,solid,line width=1.2pt,forget plot]
  table[row sep=crcr]{%
-2.1627038828467	-0.570430736225988\\
-2.35582191956788	-0.622444425702451\\
};
\addplot [color=red,solid,line width=1.2pt,forget plot]
  table[row sep=crcr]{%
-2.16266688724011	-0.570469232341139\\
-2.35577805660123	-0.622508412465134\\
};
\addplot [color=red,solid,line width=1.2pt,forget plot]
  table[row sep=crcr]{%
-2.1626938626638	-0.570494021349707\\
-2.35581868482189	-0.622482511098937\\
};
\addplot [color=red,solid,line width=1.2pt,forget plot]
  table[row sep=crcr]{%
-2.16270886483057	-0.570546032388436\\
-2.35584627541459	-0.622487736566279\\
};
\addplot [color=red,solid,line width=1.2pt,forget plot]
  table[row sep=crcr]{%
-2.16270763106234	-0.570533713807748\\
-2.35584262210678	-0.622484413844062\\
};
\addplot [color=red,solid,line width=1.2pt,forget plot]
  table[row sep=crcr]{%
-2.1626811868494	-0.570539037673046\\
-2.3558100201761	-0.622512624598203\\
};
\addplot [color=red,solid,line width=1.2pt,forget plot]
  table[row sep=crcr]{%
-2.16269042135241	-0.570495227350498\\
-2.35581411543809	-0.622487907438201\\
};
\addplot [color=red,solid,line width=1.2pt,forget plot]
  table[row sep=crcr]{%
-2.16272856946701	-0.570506355205008\\
-2.35585361998024	-0.622493996662039\\
};
\addplot [color=red,solid,line width=1.2pt,forget plot]
  table[row sep=crcr]{%
-2.16273431638988	-0.570519294386318\\
-2.35585802980231	-0.622511902685922\\
};
\addplot [color=red,solid,line width=1.2pt,forget plot]
  table[row sep=crcr]{%
-2.16276798864112	-0.57049637266786\\
-2.35588334163385	-0.622520025351005\\
};
\addplot [color=red,solid,line width=1.2pt,forget plot]
  table[row sep=crcr]{%
-2.16275243121385	-0.570491248494875\\
-2.35586484215509	-0.622525821044873\\
};
\addplot [color=red,solid,line width=1.2pt,forget plot]
  table[row sep=crcr]{%
-2.16276319679716	-0.57040168049838\\
-2.35589436904989	-0.62236657542071\\
};
\addplot [color=red,solid,line width=1.2pt,forget plot]
  table[row sep=crcr]{%
-2.16277442990096	-0.570403055239725\\
-2.35591659147368	-0.6223270903585\\
};
\addplot [color=red,solid,line width=1.2pt,forget plot]
  table[row sep=crcr]{%
-2.16275902580651	-0.570379658574017\\
-2.35590320202782	-0.622296199211777\\
};
\addplot [color=red,solid,line width=1.2pt,forget plot]
  table[row sep=crcr]{%
-2.16276571794203	-0.570351090235611\\
-2.35591034221957	-0.622265963949095\\
};
\addplot [color=red,solid,line width=1.2pt,forget plot]
  table[row sep=crcr]{%
-2.16277892439506	-0.570305917670188\\
-2.35592515039248	-0.622214831961978\\
};
\addplot [color=red,solid,line width=1.2pt,forget plot]
  table[row sep=crcr]{%
-2.16278392822716	-0.570306642690452\\
-2.355922821831	-0.622242832168339\\
};
\addplot [color=red,solid,line width=1.2pt,forget plot]
  table[row sep=crcr]{%
-2.16280009504621	-0.57029706494751\\
-2.35593731969425	-0.622239460495406\\
};
\addplot [color=red,solid,line width=1.2pt,forget plot]
  table[row sep=crcr]{%
-2.16285051265322	-0.57032073875394\\
-2.3559860615314	-0.622269364910216\\
};
\addplot [color=red,solid,line width=1.2pt,forget plot]
  table[row sep=crcr]{%
-2.16287970429251	-0.570351854018102\\
-2.35601628061339	-0.62229666017856\\
};
\addplot [color=red,solid,line width=1.2pt,forget plot]
  table[row sep=crcr]{%
-2.16288414952926	-0.570328783220276\\
-2.35602512992163	-0.622257211783764\\
};
\addplot [color=red,solid,line width=1.2pt,forget plot]
  table[row sep=crcr]{%
-2.16287425874945	-0.570342659024157\\
-2.35601185411951	-0.622283676098064\\
};
\addplot [color=red,solid,line width=1.2pt,forget plot]
  table[row sep=crcr]{%
-2.16288617639385	-0.570308886196736\\
-2.35602234740368	-0.622255199325412\\
};
\addplot [color=red,solid,line width=1.2pt,forget plot]
  table[row sep=crcr]{%
-2.16287665829653	-0.570286423159413\\
-2.35600944935014	-0.622245301321314\\
};
\addplot [color=red,solid,line width=1.2pt,forget plot]
  table[row sep=crcr]{%
-2.16291034832943	-0.570292395210768\\
-2.35604693777058	-0.622237152588577\\
};
\addplot [color=red,solid,line width=1.2pt,forget plot]
  table[row sep=crcr]{%
-2.1629476948108	-0.570221063978941\\
-2.35608089877724	-0.622178407306494\\
};
\addplot [color=red,solid,line width=1.2pt,forget plot]
  table[row sep=crcr]{%
-2.16295009156482	-0.570217808201821\\
-2.35607933217551	-0.622189881675354\\
};
\addplot [color=red,solid,line width=1.2pt,forget plot]
  table[row sep=crcr]{%
-2.16237932531927	-0.570155653031814\\
-2.35550730764434	-0.622132402093766\\
};
\addplot [color=red,solid,line width=1.2pt,forget plot]
  table[row sep=crcr]{%
-0.464602519404963	-0.116028452566248\\
-0.657754070826152	-0.167917547601598\\
};
\addplot [color=red,solid,line width=1.2pt,forget plot]
  table[row sep=crcr]{%
-0.464625692840134	-0.116011848902669\\
-0.657773294095847	-0.167915645769219\\
};
\addplot [color=red,solid,line width=1.2pt,forget plot]
  table[row sep=crcr]{%
-0.464657482079649	-0.115960106473745\\
-0.657801040462523	-0.167878945590211\\
};
\addplot [color=red,solid,line width=1.2pt,forget plot]
  table[row sep=crcr]{%
-0.464704468715861	-0.115871566897623\\
-0.657838470608457	-0.167825944123714\\
};
\addplot [color=red,solid,line width=1.2pt,forget plot]
  table[row sep=crcr]{%
-0.464745641683104	-0.115820402483157\\
-0.657874504043093	-0.1677938815233\\
};
\addplot [color=red,solid,line width=1.2pt,forget plot]
  table[row sep=crcr]{%
-0.464765408907853	-0.11578829920016\\
-0.657891304952131	-0.167772799554542\\
};
\addplot [color=red,solid,line width=1.2pt,forget plot]
  table[row sep=crcr]{%
-0.464782284685096	-0.11579293460614\\
-0.65790806554094	-0.167777862891371\\
};
\addplot [color=red,solid,line width=1.2pt,forget plot]
  table[row sep=crcr]{%
-0.464793008815628	-0.115815411177383\\
-0.657918055289859	-0.167803067638567\\
};
\addplot [color=red,solid,line width=1.2pt,forget plot]
  table[row sep=crcr]{%
-0.464800946049796	-0.115840309489398\\
-0.657932333216496	-0.167804405662297\\
};
\addplot [color=red,solid,line width=1.2pt,forget plot]
  table[row sep=crcr]{%
-0.464815468315888	-0.115872958662061\\
-0.657952812725037	-0.167814908900587\\
};
\addplot [color=red,solid,line width=1.2pt,forget plot]
  table[row sep=crcr]{%
-0.464811016352606	-0.115931267005144\\
-0.65795748886623	-0.167839264033965\\
};
\addplot [color=red,solid,line width=1.2pt,forget plot]
  table[row sep=crcr]{%
-0.464818698775647	-0.115934972896998\\
-0.657973242097059	-0.167812929625235\\
};
\addplot [color=red,solid,line width=1.2pt,forget plot]
  table[row sep=crcr]{%
-0.464837769069135	-0.11592790626119\\
-0.657993856122681	-0.16780011494911\\
};
\addplot [color=red,solid,line width=1.2pt,forget plot]
  table[row sep=crcr]{%
-0.464842528842012	-0.115974722853929\\
-0.658004458609008	-0.167825170187489\\
};
\addplot [color=red,solid,line width=1.2pt,forget plot]
  table[row sep=crcr]{%
-0.46484379026772	-0.116015309986886\\
-0.658008290492473	-0.167856180474574\\
};
\addplot [color=red,solid,line width=1.2pt,forget plot]
  table[row sep=crcr]{%
-0.464846400416098	-0.116073241899827\\
-0.658012852129297	-0.167906840390885\\
};
\addplot [color=red,solid,line width=1.2pt,forget plot]
  table[row sep=crcr]{%
-0.464837872454851	-0.116180174029105\\
-0.658010798109705	-0.167989640287158\\
};
\addplot [color=red,solid,line width=1.2pt,forget plot]
  table[row sep=crcr]{%
-0.46484458845387	-0.116235008757297\\
-0.658023291279119	-0.168022929885241\\
};
\addplot [color=red,solid,line width=1.2pt,forget plot]
  table[row sep=crcr]{%
-0.464863118895648	-0.116232848366787\\
-0.658049323582884	-0.167992778028072\\
};
\addplot [color=red,solid,line width=1.2pt,forget plot]
  table[row sep=crcr]{%
-0.464875787895959	-0.116259735803147\\
-0.65806272256218	-0.168016940850069\\
};
\addplot [color=red,solid,line width=1.2pt,forget plot]
  table[row sep=crcr]{%
-0.464861833404299	-0.116327115444281\\
-0.658055744471007	-0.168058273625291\\
};
\addplot [color=red,solid,line width=1.2pt,forget plot]
  table[row sep=crcr]{%
-0.464844914926771	-0.116408580886387\\
-0.658048601070634	-0.168103219490632\\
};
\addplot [color=red,solid,line width=1.2pt,forget plot]
  table[row sep=crcr]{%
-0.464840238120612	-0.116486999659307\\
-0.65805410398642	-0.168143577528826\\
};
\addplot [color=red,solid,line width=1.2pt,forget plot]
  table[row sep=crcr]{%
1.15550148660375	0.310767550823876\\
0.962330395678208	0.258951244214498\\
};
\addplot [color=red,solid,line width=1.2pt,forget plot]
  table[row sep=crcr]{%
1.15558012727077	0.311004405631611\\
0.962489732975854	0.258888192885163\\
};
\addplot [color=red,solid,line width=1.2pt,forget plot]
  table[row sep=crcr]{%
1.15599162991896	0.311182414286128\\
0.962945699544066	0.258901741320964\\
};
\addplot [color=red,solid,line width=1.2pt,forget plot]
  table[row sep=crcr]{%
1.15599268316292	0.31122398784811\\
0.962963184003935	0.258882680504565\\
};
\addplot [color=red,solid,line width=1.2pt,forget plot]
  table[row sep=crcr]{%
1.15600886652768	0.311167527710081\\
0.962974586412567	0.258843855243036\\
};
\addplot [color=red,solid,line width=1.2pt,forget plot]
  table[row sep=crcr]{%
1.155965474948	0.31109270139653\\
0.962909205919724	0.258850218790099\\
};
\addplot [color=red,solid,line width=1.2pt,forget plot]
  table[row sep=crcr]{%
1.1558931343696	0.311059286114581\\
0.962834191929867	0.258826683804812\\
};
\addplot [color=red,solid,line width=1.2pt,forget plot]
  table[row sep=crcr]{%
1.15582389731482	0.311049618067674\\
0.962752354885297	0.258863609369121\\
};
\addplot [color=red,solid,line width=1.2pt,forget plot]
  table[row sep=crcr]{%
1.15580912904083	0.31108392929853\\
0.962727190703624	0.258936397353602\\
};
\addplot [color=red,solid,line width=1.2pt,forget plot]
  table[row sep=crcr]{%
1.15580271406914	0.311042607215227\\
0.962715967359115	0.258912882058289\\
};
\addplot [color=red,solid,line width=1.2pt,forget plot]
  table[row sep=crcr]{%
1.15579413972928	0.311034318409971\\
0.962712987147494	0.258883877206768\\
};
\addplot [color=red,solid,line width=1.2pt,forget plot]
  table[row sep=crcr]{%
1.15576363153165	0.311025043170926\\
0.962689109748655	0.258850058376789\\
};
\addplot [color=red,solid,line width=1.2pt,forget plot]
  table[row sep=crcr]{%
1.15572390552668	0.311066067293301\\
0.962652677415501	0.258878895731324\\
};
\addplot [color=red,solid,line width=1.2pt,forget plot]
  table[row sep=crcr]{%
1.15567118129471	0.31103375125906\\
0.962604856404305	0.25882844316389\\
};
\addplot [color=red,solid,line width=1.2pt,forget plot]
  table[row sep=crcr]{%
1.15567094524244	0.310964686878578\\
0.96260786156892	0.258747393550881\\
};
\addplot [color=red,solid,line width=1.2pt,forget plot]
  table[row sep=crcr]{%
1.1556476595401	0.310902346209528\\
0.962581688181072	0.258695730697727\\
};
\addplot [color=red,solid,line width=1.2pt,forget plot]
  table[row sep=crcr]{%
1.15562526139994	0.310908283931492\\
0.962559665997304	0.258700278110285\\
};
\addplot [color=red,solid,line width=1.2pt,forget plot]
  table[row sep=crcr]{%
1.15558389554326	0.310884169380487\\
0.962520951669983	0.258666359171867\\
};
\addplot [color=red,solid,line width=1.2pt,forget plot]
  table[row sep=crcr]{%
1.15555654045071	0.310869979718934\\
0.962496213577529	0.258642494736204\\
};
\addplot [color=red,solid,line width=1.2pt,forget plot]
  table[row sep=crcr]{%
1.15553406150112	0.310887751451398\\
0.962474211029207	0.258658505470217\\
};
\addplot [color=red,solid,line width=1.2pt,forget plot]
  table[row sep=crcr]{%
1.15551527987282	0.310885683495266\\
0.962457533769025	0.258648659562789\\
};
\addplot [color=red,solid,line width=1.2pt,forget plot]
  table[row sep=crcr]{%
1.15549141315344	0.310895976842675\\
0.962437554250487	0.258644588701506\\
};
\addplot [color=red,solid,line width=1.2pt,forget plot]
  table[row sep=crcr]{%
1.15551123882362	0.310905702813664\\
0.962462271044003	0.258636246730543\\
};
\addplot [color=red,solid,line width=1.2pt,forget plot]
  table[row sep=crcr]{%
1.15548890091653	0.310851825909072\\
0.962439314053871	0.258584656364923\\
};
\addplot [color=red,solid,line width=1.2pt,forget plot]
  table[row sep=crcr]{%
2.64738161413397	-1.62511877229587\\
2.59468200256079	-1.43218678702636\\
};
\addplot [color=red,solid,line width=1.2pt,forget plot]
  table[row sep=crcr]{%
2.64746728262253	-1.62509082150634\\
2.59469202245058	-1.43217951568477\\
};
\addplot [color=red,solid,line width=1.2pt,forget plot]
  table[row sep=crcr]{%
2.64744007236263	-1.62506935111799\\
2.59461367631717	-1.43217204195882\\
};
\addplot [color=red,solid,line width=1.2pt,forget plot]
  table[row sep=crcr]{%
2.64737863711825	-1.62507178644807\\
2.59455041171256	-1.43217497828245\\
};
\addplot [color=red,solid,line width=1.2pt,forget plot]
  table[row sep=crcr]{%
2.64744563720569	-1.62504009272027\\
2.59459920024422	-1.43214827303764\\
};
\addplot [color=red,solid,line width=1.2pt,forget plot]
  table[row sep=crcr]{%
2.64747909014662	-1.62501733511245\\
2.59463975319286	-1.43212357038615\\
};
\addplot [color=red,solid,line width=1.2pt,forget plot]
  table[row sep=crcr]{%
2.64749638078906	-1.62501187331734\\
2.59463789451051	-1.43212335518227\\
};
\addplot [color=red,solid,line width=1.2pt,forget plot]
  table[row sep=crcr]{%
2.64747051092017	-1.62502191157994\\
2.59463034179548	-1.43212837481192\\
};
\addplot [color=red,solid,line width=1.2pt,forget plot]
  table[row sep=crcr]{%
2.64746526320659	-1.62503999319241\\
2.59463683723651	-1.43214323995503\\
};
\addplot [color=red,solid,line width=1.2pt,forget plot]
  table[row sep=crcr]{%
2.64743567008849	-1.62505832262084\\
2.59460684866817	-1.43216167768542\\
};
\addplot [color=red,solid,line width=1.2pt,forget plot]
  table[row sep=crcr]{%
2.64745396937311	-1.62504720072319\\
2.59461721390936	-1.4321527288686\\
};
\addplot [color=red,solid,line width=1.2pt,forget plot]
  table[row sep=crcr]{%
2.64749113960526	-1.62504650758988\\
2.59466244377228	-1.43214982825976\\
};
\addplot [color=red,solid,line width=1.2pt,forget plot]
  table[row sep=crcr]{%
2.64748687935806	-1.62506199932817\\
2.59464972328206	-1.43216763720786\\
};
\addplot [color=red,solid,line width=1.2pt,forget plot]
  table[row sep=crcr]{%
2.6474959597178	-1.62505163796685\\
2.59466091432575	-1.43215669770552\\
};
\addplot [color=red,solid,line width=1.2pt,forget plot]
  table[row sep=crcr]{%
2.64750052756238	-1.62506363563432\\
2.59467552398591	-1.43216594514232\\
};
\addplot [color=red,solid,line width=1.2pt,forget plot]
  table[row sep=crcr]{%
2.64753446433187	-1.62505897303559\\
2.59473587945497	-1.4321540497314\\
};
\addplot [color=red,solid,line width=1.2pt,forget plot]
  table[row sep=crcr]{%
2.64756437951102	-1.62504770111534\\
2.59476363212984	-1.43214336970728\\
};
\addplot [color=red,solid,line width=1.2pt,forget plot]
  table[row sep=crcr]{%
2.64753170087078	-1.6250601844125\\
2.59476163825512	-1.43214745675686\\
};
\addplot [color=red,solid,line width=1.2pt,forget plot]
  table[row sep=crcr]{%
2.6475191397203	-1.6250705221664\\
2.59475072134015	-1.43215734474802\\
};
\end{axis}
\end{tikzpicture}%\\ \vspace*{1em}
    {\footnotesize {(b)~Measurement \#2}} \\
  \end{minipage}
  %
  \tikzexternalenable
  %
  \caption{Two examples of measuring the wall placements of an indoor space with an iPad Pro. The walls (the equations of the planes) are a part of the state variable. Points where the phone is stationary (against the wall) are visualized in red. The true size of the room is 7.30~m $\times$ 8.45~m.}
  \label{fig:roomscan}
  \vspace*{-1em}
\end{figure}




\section{Discussion and Conclusions}
\label{sec:discussion}
\noindent
In this paper we have presented a general framework for inertial navigation using the limited quality data provided by standard handheld smartphones. Up till now, this has been regarded challenging, and we are not aware of any prior published work where the same would have been accomplished.

We presented a probabilistic approach building on extended Kalman filtering for continuous estimation of the position, velocity, and orientation of the mobile device. Furthermore, the IMU sensor biases and scale errors were estimated as a part of the system state. Our approach differentiates itself from prior models by directly employing the Bayesian (fully probabilistic) interpretation of non-linear state estimation (in the spirit of \cite{Sarkka:2013}), and handling the non-additive process noise inside the dynamic model. The estimation scheme avoids unnecessary approximations or error state transformations. Furthermore, we do not assume the sensor sampling rate to be fixed, but use the actual observation timestamps of the sensor events. This helps mitigate problems with missing samples and other unexpected issues with the inputs.

In order to work, the dynamic model needs to be fused with observations. We presented several types of alternative measurements that can be combined with the model. These were position fixes (see Fig.~\ref{fig:intro}), position loop-closures and barometric air pressure data (see Figs.~\ref{fig:intro} and \ref{fig:states}(b)), zero-velocity updates (all examples), and plane tangent observations (Fig.~\ref{fig:roomscan}). We also introduced constraining the speed estimate from exploding by introducing a pseudo-update for the speed. Even though many of these constraints are not general enough to fit  all applications, they still cover many potential use cases.

The presented method has many strong sides. It is general and does not requiring any steps to be detected, specific orientation to be held in, or field of vision to cover any visual features. This differentiates it from conventional PDR and odometry methods for mobile phones. The method is also not limited to estimation in a two-dimensional plane. All these aspects were covered in the experiments, where the phone was held in the pocket, in a bag, on a baby pushchair/stroller, and in an elevator. The last experiment demonstrated how the very same algorithm can be used as a measuring tool for estimating the shape and size of an indoor space.
In the PDR experiments we chose to show what the method is capable of as such. While there exists a multitude of well-tailored methods for all of the isolated test scenarios, there are no exact competing methods for mobile phones which could cover all of these scenarios. Implementing separate methods for comparison with respect to each use case was not viable, and we rather chose to put our focus on providing a convincing set of application examples.

In this paper, the data was collected using the mobile device, but the path was calculated off-line. However, the method is lightweight and capable of running in real-time on an iPhone or iPad (even older models). The computational efficiency comes from the sequential nature of the data processing, which scales linearly in the number of sensor samples.

The method still has some challenges and room for improvement. This kind of inertial navigation systems either work very well or fail miserably (\ie\ diverge)---there is no middle ground. Therefore handling of the noise scales and biases are crucial for success. Estimation of the sensor biases requires some auxiliary information to be fused with the model---be that ZUPTs, loop-closures, position fixes, or something else. Even though ZUPTs can be implemented to be performed subtly in the background (\eg\ when the user places the phone on the table), there are use cases which might be problematic. Even though, it has been argued that estimating the sensor biases as a part of the state would not be useful \cite{Nilsson+Zachariah+Skog+Handel:2013}, our experiences are quite the contrary. However, this requires the model to be both derived and implemented in a stable way avoiding unnecessary approximations in the error propagation.

The model is also sensitive to the noise scale parameters. The results in this paper benefit from the good sensors (\eg\ good dynamic range) in the Apple devices. High-end Android phones show comparable results. On Android devices the sampling rate can be set higher, which benefits the modelling (conventional strapdown INS use thousands of Hz, see \cite{Titterton+Weston:2004}).


In indoor positioning and tracking the INS presented in this paper could serve as a PDR replacement. The requirement for the zero-velocity updates could perhaps be loosened if the model would receive external position estimates based on Wi-Fi, BLE, RFID, or magnetic field anomalies.

Supplementary material for this paper available on: \\
\mbox{\url{https://aaltovision.github.io/handheld-INS/}}








\section*{Acknowledgments}
\noindent
Academy of Finland grants 277685, 295081, 308640, and 310325. We thank Manon Kok for helpful comments.



\bibliographystyle{IEEEtran}

{\small \bibliography{bibliography}}


\end{document}

