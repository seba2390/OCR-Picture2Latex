\documentclass[11pt]{article}

%****************** Useful Packages *****************************
\usepackage{amssymb,amsmath,amsthm,amsfonts}
%\usepackage{showkeys}
\usepackage{subfigure, hyperref, multirow, bm, color, stmaryrd, marginnote, graphicx}
%\usepackage{chngcntr}
%\counterwithin{table}{section}
%\counterwithin{figure}{section}
%*****************************************************************

%********************** Page Formatting**************************
\topmargin=0cm
\oddsidemargin0mm
\textheight23.5cm
\textwidth16cm
\headsep0mm
\headheight0mm
\parskip 0pt
\setlength{\parindent}{0pt}
\numberwithin{equation}{section}
\allowdisplaybreaks

%\renewcommand{\chaptermark}[1]{\markboth{\thechapter.\ #1}{}}
%\renewcommand{\sectionmark}[1]{\markright{\thesection.\ #1}{}}


%%Commands
\newcommand{\R}{\mathbb{R}}
\newcommand{\Z}{\mathbb{Z}}
\newcommand{\N}{\mathbb{N}}
\newcommand{\F}{\mathbb{F}}
\newcommand{\C}{\mathbb{C}}
\newcommand{\T}{\mathbb{T}}
\newcommand{\A}{\mathcal{A}}
\newcommand{\E}{\mathcal{E}}
\newcommand{\D}{\mathcal{D}}
\newcommand{\sR}{\mathcal{R}}
\newcommand{\barR}{\overline{\bm{R}}}
\newcommand{\barx}{\overline{\bm{x}}}
\newcommand{\bars}{\overline s}
\newcommand{\bu}{\bm{u}}
\newcommand{\bw}{\bm{w}}
\newcommand{\bx}{\bm{x}}
\newcommand{\X}{\bm{X}}
\newcommand{\be}{\bm{e}}
\newcommand{\bv}{\bm{v}}
\newcommand{\p}{\partial}
\newcommand{\ts}{\thinspace}
\newcommand{\dive}{{\rm{div}}}
\newcommand{\SB}{{\rm SB}}
%
\newcommand{\abs}[1]{\left\lvert #1 \right\rvert}
\newcommand{\norm}[1]{\left\lVert #1 \right\rVert}
\newcommand{\wh}[1]{\widehat{#1}}
\newcommand{\wt}[1]{\widetilde{#1}}
\newcommand{\mc}[1]{\mathcal{#1}}

\newtheorem{theorem}{Theorem}[section]
\newtheorem{lemma}[theorem]{Lemma}
\newtheorem{proposition}[theorem]{Proposition}
\newtheorem{corollary}[theorem]{Corollary}
\newtheorem{definition}[theorem]{Definition}
\theoremstyle{definition}
\newtheorem{remark}[theorem]{Remark}

\allowdisplaybreaks


\begin{document}
\title{Theoretical justification and error analysis for slender body theory}
\author{Yoichiro Mori, Laurel Ohm, Daniel Spirn
\footnote{This research was supported in part by NSF grant DMS-1620316 and DMS-1516978, awarded to Y.M., by NSF GRF grant 00039202 and a Torske Kubben Fellowship, awarded to L.O., and by NSF grant DMS-1516565, awarded to D.S. The authors thank the IMA where most of this work was performed. The authors also thank the anonymous referees whose detailed comments greatly improved the paper.}\\ \textit{\small School of Mathematics, University of Minnesota, Minneapolis, MN 55455}}
\date{\today}

\maketitle

%%%%%%%%%
\begin{abstract}
Slender body theory facilitates computational simulations of thin fibers immersed in a viscous fluid by approximating each fiber using only the geometry of the fiber centerline curve and the line force density along it. However, it has been unclear how well slender body theory actually approximates Stokes flow about a thin but truly three-dimensional fiber, in part due to the fact that simply prescribing data along a one-dimensional curve does not result in a well-posed boundary value problem for the Stokes equations in $\R^3$. Here, we introduce a PDE problem to which slender body theory (SBT) provides an approximation, thereby placing SBT on firm theoretical footing. The slender body PDE is a new type of boundary value problem for Stokes flow where partial Dirichlet and partial Neumann conditions are specified everywhere along the fiber surface. Given only a 1D force density along a closed fiber, we show that the flow field exterior to the thin fiber is uniquely determined by imposing a {\em fiber integrity condition}: the surface velocity field on the fiber must be constant along cross sections orthogonal to the fiber centerline. Furthermore, a careful estimation of the residual, together with stability estimates provided by the PDE well-posedness framework, allow us to establish error estimates between the slender body approximation and the exact solution to the above problem. The error is bounded by an expression proportional to the fiber radius (up to logarithmic corrections) under mild regularity assumptions on the 1D force density and fiber centerline geometry.
\end{abstract}

%%%%%%%%
\tableofcontents
%%%%%%%%

\section{Introduction}
Describing the motion of thin filaments immersed in a viscous fluid presents an important modeling problem in mathematical biology, engineering, and physics. Numerical simulations of slender fibers have been used to help explain the role of cilia in embryonic development \cite{smith2011mathematical} and mucous transport \cite{smith2007discrete}, simulate microtubules forming the mitotic spindle during cell division \cite{shelley2016dynamics}, understand the rheology of fiber suspensions used in creating composite materials \cite{fan1998direct, hamalainen2011papermaking, petrie1999rheology}, and explore the dynamics of swimming microorganisms \cite{gueron1997cilia, lauga2009hydrodynamics, nguyen2011action, rodenborn2013propulsion, saintillan2011emergence, spagnolie2011comparative}. Models describing the interaction between thin structures and a viscous fluid may also aid in the design and optimization of microfluidic devices \cite{avron2008geometric, becker2003self, buchmann2015flow, dreyfus2005microscopic}. \\

To handle the simulation of the large numbers of thin fibers arising in these models, many existing numerical methods rely on a classical approximation known as \emph{slender body theory}. In essence, slender body theory reduces computational costs by exploiting the thin geometry of the objects being modeled. \\

To begin, we assume that the slender fibers are immersed in low Reynolds number flow, typified by any of the following: high viscosity, very slow (creeping) flow, or flow over very small length scales. Such flows are governed by the Stokes equations \eqref{stokes}, where $\bu$ represents the fluid velocity, $p$ is the pressure, and $\mu$ is the viscosity: 
\begin{equation}
\left.
\begin{aligned}
-\mu \Delta \bu +\nabla p &= 0 \\
\dive \ts \bu &=0
\end{aligned}
\right\rbrace
\label{stokes}
 \end{equation}
 accompanied by appropriate boundary conditions. Stokes flow around solid objects in unbounded or semi-bounded domains can be represented succinctly via \emph{boundary integral equations} over the surface of the object \cite{pozrikidis1992boundary}. However, despite this explicit boundary integral representation of a solution to the Stokes system, solving integral equations over moving surfaces remains a computationally intensive task, especially when simulating tens or hundreds of individual objects. Furthermore, from a modeling perspective, specifying the surface traction at each point along the entire surface of a fiber with complicated geometry can quickly become cumbersome.  \\

Instead of treating a filament as a three-dimensional object and solving equations for its surface velocity, slender body theory approximates a thin filament with a one-dimensional force density $\bm{f}(s)$ defined along the filament centerline. The idea of modeling a thin fiber with a line distribution of fundamental singularities originated with Hancock \cite{hancock1953self}, Cox \cite{cox1970motion}, Batchelor \cite{batchelor1970slender}, Lighthill \cite{lighthill1975mathematical}, and Keller and Rubinow \cite{keller1976slender}. Later, Johnson \cite{johnson1980improved} introduced doublet corrections along the fiber centerline to come up with the integral expression \eqref{SBT2} that we regard as classical slender body theory. Since then, slender body theory has formed the basis for many numerical methods developed to model thin fibers in Stokes flow \cite{bouzarth2011modeling, bringley2008validation, cortez2005method, cortez2012slender, gotz2000interactions, shelley2000stokesian, tornberg2006numerical, tornberg2004simulating}. \\

Despite the many numerical results relying on this theory, there is a lack of rigorous error analysis for slender body theory itself. The theory is built on the assumption that, given only a force density curve $\bm{f}(s)$ along the centerline of a thin but inherently three-dimensional object, we can (approximately) solve for the fiber velocity. However, it is not possible to solve for Stokes flow in three dimensions using only data specified along a one-dimensional curve. In particular, it is not immediately obvious how to rigorously compare the slender body approximation to the actual PDE solution for Stokes flow about a 3D fiber, as it remains unclear what this ``true'' solution should be. Ideally, we should be able to define a unique notion of true solution to the slender body problem given only the force density $\bm{f}(s)$ and the fiber geometry, as this is the only information needed to build a slender body approximation.  \\

Many of the foundational papers in slender body theory compute some notion of asymptotic accuracy of the slender body approximation \cite{gotz2000interactions, johnson1980improved, keller1976slender, sellier1999stokes}. Previous studies \cite{bouzarth2011modeling} have also numerically verified the convergence of the slender body approximation as the slender body radius tends to zero, but to what exactly the approximation is converging remains unclear. Recently, Koens and Lauga \cite{koens2018boundary} derived the slender body expression as an asymptotic limit of the full boundary integral equations. However, this formulation of the slender body problem requires specifying the full two-dimensional surface traction at each point on the slender body surface in order to obtain a ``true'' solution. This notion of true solution, then, is not well-defined without specifying additional force data beyond the force-per-unit-length $\bm{f}(s)$. The question remains: is there a well-posed PDE for which slender body theory is an approximation that requires only the line force density $\bm{f}(s)$ and the fiber geometry as data? \\

In this paper, we address this question by giving meaning to a solution to the Stokes equations about a slender fiber in $\R^3$, given only one-dimensional force data $\bm{f}(s)$ and a ``fiber integrity condition" (see Section \ref{SBT_def}) common to most slender body theories. We prove well-posedness of the slender body PDE problem using only this data. Furthermore, we obtain a rigorous error estimate between the true solution and the slender body approximation both within the bulk fluid and along the fiber centerline. Note that, although many of the applications listed above deal with the dynamic problem of a fiber moving with the local fluid velocity, we consider only the static problem here. Making sense of such a solution in the static case is an important first step toward truly understanding slender body theory in the dynamic case. \\
%We consider here the case of a closed loop; see \cite{mori2018theoretical} for a treatment of fibers with free ends.

Beyond serving as a theoretical justification for the use of slender body theory in modeling and simulation of thin fibers, our PDE framework can be applied more widely to construct slender body theories for different types of fluids. In particular, our formulation makes sense for the full Navier-Stokes equations and may serve as a first step toward a rigorous justification for models such as \cite{lim2008dynamics}. Our framework can also be used to study the case of near-intersection for multiple fibers, a regime where existing slender body theories break down due to nearby fibers introducing strong angular dependence into the velocity field near the opposing fiber centerline. 

%%%%%
\subsection{Slender body geometry}\label{geometric_constraints}
Before we can introduce the slender body approximation, we must precisely describe the slender geometries under consideration. \\

\begin{figure}[!h]
\centering
\includegraphics[scale=0.7]{SB_geometry}\\
\caption{The geometry of the fiber is specified via a $C^1$ orthonormal frame $\be_t(s)$, $\be_{n_1}(s)$, $\be_{n_2}(s)$. Local coordinates $\rho,\theta,s$ uniquely specify the location of a point $\bx$ in a neighborhood $\mathcal{O}$ of the slender body.}
\label{fig:coord_sys}
\end{figure}

Let $\X: \T \equiv \R/\Z \to \R^3$ denote the coordinates of a closed curve $\Gamma_0\in \R^3$, parameterized by arclength $s$ with the length of $\X$ normalized to 1. Let $C^k(\T)$, $k\in \N$, denote the space of $k$-continuously differentiable functions defined on $\T$ (we will use the same notation, without confusion, for scalar or $\R^3$-valued functions). We assume that $\X(s)\in C^2(\T)$ so that its curvature $\kappa(s) = \big| \frac{d^2\X}{d s^2} \big|$ is well-defined. \\

We assume that $\Gamma_0$ is non-self-intersecting; in particular, 
\begin{equation}\label{non_intersecting}
\inf_{s\neq t}\frac{|\X(s)-\X(t)|}{|s-t|} \ge c_{\Gamma}
\end{equation}
for some constant $c_{\Gamma}>0$. \\

For computational purposes, it will be convenient to consider a $C^1$ orthonormal frame along the slender body centerline $\Gamma_0$, periodic with respect to the arclength variable $s$. Such frames are commonly used in describing Kirchhoff rod dynamics (see \cite{antman2005nonlinear, goriely1997nonlinear} for a longer exposition). We begin by defining the tangent vector
\[\be_t(s)=\frac{d \X}{ds}. \]

We then choose a pair $\{\be_{n_1}(s),\be_{n_2}(s)\}$ of orthonormal vectors spanning the plane normal to $\be_t(s)$ at each $s\in \T$. By orthonormality, the vectors $\{\be_t,\be_{n_1},\be_{n_2}\}$ satisfy the ODE 
\begin{equation}\label{moving_ODE}
\frac{d}{ds}\begin{pmatrix}
\be_t(s) \\
\be_{n_1}(s) \\
\be_{n_2}(s) 
\end{pmatrix} = \begin{pmatrix}
0 & \kappa_1(s) & \kappa_2(s) \\
-\kappa_1(s) & 0 & \kappa_3(s) \\
-\kappa_2(s) & -\kappa_3(s) & 0 
\end{pmatrix} \begin{pmatrix}
\be_t(s) \\
\be_{n_1}(s) \\
\be_{n_2}(s) 
\end{pmatrix},
\end{equation}
where $\kappa_j$, $j=1,2,3$ are continuous functions of $s$. Note that if $\X$ is $C^3$ and the curvature $\kappa(s)$ is non-vanishing everywhere on $\T$, we can then use the simpler Frenet frame, where $\be_{n_1}(s) = \be_t'(s)/\kappa(s)$, $\kappa_1(s)=\kappa(s)$, $\kappa_2\equiv 0$, and $\kappa_3=\tau(s)$, the torsion of the curve $\X(s)$. This is useful because the ODE satisfied by the basis vectors is simpler and the coefficients have a clear geometric meaning. However, to allow for more general $C^2$ curves with possibly vanishing curvature at some points, we must refer to a frame that is well-defined when $\kappa(s)=0$.  \\

Although the geometric meaning of the general orthonormal frame coefficients $\kappa_j$ is less clear than for the Frenet frame, we note that the curvature $\kappa(s)$ of the fiber centerline always satisfies
\begin{equation}\label{kappa12}
\kappa(s)=\sqrt{\kappa_1^2(s)+\kappa_2^2(s)}.
\end{equation}

Furthermore, we may choose this orthonormal frame to satisfy the following lemma.  
 \begin{lemma}\label{lemmaorthonormal}
The coefficient $\kappa_3$ in \eqref{moving_ODE} may be made to satisfy
\begin{equation}\label{kappa3}
\kappa_3 \text{ does not depend on } s \text{ and } |\kappa_3| \le \pi.
\end{equation}
\end{lemma}
The proof of this statement is contained in Appendix \ref{moving_frame_pf}. In this construction, the orthonormal frame is almost the same as the Bishop frame \cite{bishop1975there} for open curves, except that $\kappa_3$ cannot necessarily be made to vanish for a closed curve. \\

We define
\begin{equation}\label{kappamax}
\kappa_{\max}=\max_{s\in\T} \abs{\kappa(s)}
\end{equation}
and note that, since $\X$ is a $C^2$ closed loop of length 1, we have $2\pi\le \kappa_{\max}<\infty$. \\

We also define the following cylindrical unit vectors with respect to the moving frame: 
\begin{align*}
\be_{\rho}(s,\theta) &:= \cos\theta \be_{n_1}(s) + \sin\theta\be_{n_2}(s) \\
\be_{\theta}(s,\theta) &:= -\sin\theta \be_{n_1}(s) + \cos\theta\be_{n_2}(s).
\end{align*}

Since the slender body is non-self-intersecting with $C^2$ centerline, there exists
\begin{equation}\label{rmax}
r_{\max} = r_{\max}(\kappa_{\max},c_\Gamma)
\end{equation}
such that points $\bx$ with ${\rm dist}(\bx,\X)< r_{\max}$ may be uniquely parameterized as a tube about the fiber centerline (see Figure \ref{fig:coord_sys}):
\begin{equation}\label{coordinates}
 \bx = \X(s)+\rho\be_{\rho}(s,\theta). 
 \end{equation}
 In fact, we claim that $r_{\max} \sim c_\Gamma/\kappa_{\max}$ should suffice, but we do not prove this here. \\

For $\epsilon < r_{\max}/4$, we then define a slender body $\Sigma_\epsilon$ with uniform radius $\epsilon$ by
\begin{equation}\label{slender_body}
\Sigma_{\epsilon} = \big\{\bx \in \R^3 \ts : \ts \bx= \X(s) + \rho \be_{\rho}(s,\theta), \quad \rho < \epsilon \big\}
\end{equation}
 
 We parameterize the surface of the slender body, $\Gamma_{\epsilon}=\p \Sigma_{\epsilon}$, as 
\begin{equation}\label{gamma_epsilon}
 \Gamma_{\epsilon}(s,\theta)= \X(s) + \epsilon\be_{\rho}(s,\theta). 
 \end{equation}

The surface element on $\Gamma_{\epsilon}$ is then given by
\begin{equation}\label{surface_element}
dS = \mathcal{J}_{\epsilon}(s,\theta) \ts d\theta ds, 
 \end{equation}

where we define
\begin{equation}\label{Jeps_def}
\mathcal{J}_{\epsilon}(s,\theta) := \epsilon\big(1-\epsilon(\kappa_1(s)\cos\theta+\kappa_2(s)\sin\theta) \big).
\end{equation}

We also define the neighborhood
\begin{equation}\label{region_O}
\mathcal{O} = \bigg\{\bx \in \Omega_{\epsilon} \ts : \ts \bx= \X(s) + \rho \be_{\rho}(s,\theta), \quad \epsilon < \rho<r_{\max} \bigg\}
\end{equation}
of the slender body to refer to fluid points $\bx$ near to the slender body. 


%%%%%%%%%%%%%%%%%%%%%%%%%%%%%%%%%%%%%%%%%%%%%%%%%%%%%%%%%%
%%%%%%%%%%%%%%%%%%%%%%%%%%%%%%%%%%%%%%%%%%%%%%%%%%%%%%%%%%
%%%%%%%%%%%%%%%%%%%%%%%%%%%%%%%%%%%%%%%%%%%%%%%%%%%%%%%%%%

\subsection{Classical slender body theory}\label{SBT_def}
With the geometric constraints specified above, we now define the corresponding slender body approximation to Stokes flow about the thin fiber. \\
 
The essential building block of slender body theory is the Stokeslet, the free-space Green's function for the Stokes equations \eqref{stokes}. The Stokeslet represents the Stokes flow in $\R^3$ resulting from a point source at $\bx_0$ of strength ${\bm g}$:
\begin{equation}
\begin{aligned}
-\mu \Delta \bu +\nabla p &= {\bm g}\delta(\bx-\bx_0) \\
\dive \ts \bu &=0 \\
|\bu| &\to 0 \quad \text{ as } |\bx| \to \infty,
\end{aligned}
\label{Stokes_Green}
\end{equation}
where $\delta(\bx)$ denotes the Dirac delta. We define the Stokeslet and its associated pressure tensor as 
\begin{align*}
\mathcal{S}(\widehat\bx) &= \frac{{\bf I}}{|\widehat\bx|} + \frac{{\widehat\bx}{\widehat\bx}^{\rm T}}{|\widehat\bx|^3}, \quad p^{S}({\widehat\bx}) = \nabla \left(\frac{1}{|\widehat\bx|}\right) = \frac{\widehat\bx}{|\widehat\bx|^3},
\end{align*}
where ${\bf I}$ is the identity matrix and $\widehat \bx = \bx-\bx_0$ (see \cite{pozrikidis1992boundary,childress1981mechanics} for a derivation). The solution to \eqref{Stokes_Green} is then given by 
\[ \bu = \frac{1}{8\pi\mu}\mc{S}(\wh \bx) \bm{g}, \quad p = \frac{1}{4\pi} p^S(\wh\bx)\cdot\bm{g}.\]

Since the singularly forced Stokes system \eqref{Stokes_Green} is linear, additional solutions may constructed by differentiating the Stokeslet and taking linear combinations of the Stokeslet and its higher-order derivatives -- dipoles, quadrupoles, octupoles, etc. Inclusion of these higher-order multipole terms in the expression of solutions to \eqref{Stokes_Green} can be useful especially in solving exterior problems, and is sometimes referred to as the method of singularities \cite{pozrikidis1992boundary}. \\

The higher-order term that plays the most important role in slender body theory, known as the doublet, is given by 
\[ \mathcal{D}( \widehat \bx) = \frac{1}{2}\Delta\mathcal{S}({\widehat \bx}) = \frac{{\bf I}}{|\widehat\bx|^3}-3\frac{{\widehat\bx}{\widehat\bx}^{\rm T}}{|\widehat\bx|^5}.\]


%%%%%%%%%%%%%%%%%%%%%%%%%%%%%%%%%%%%%%%%%%%%%%%%%%%%%%%%%%
The idea of slender body theory is to approximate the velocity field around a thin filament in Stokes flow by integrating a superposition of Stokeslets, doublets, and possibly higher-order multipole terms along the centerline of the fiber. The slender body ansatz is given by the integral expression
\begin{equation}\label{SB_ansatz}
\bu^{\SB}(\bx) = \bu_{\infty}(\bx) + \frac{1}{8\pi\mu} \int_{\T} \bigg(\mathcal{S}(\bx-\X(t)) {\bm g}_1(t) +\mathcal{D}(\bx-\X(t)) {\bm g}_2(t)+\cdots \bigg)\ts dt,
\end{equation}
where $\bu_{\infty}$ is the undisturbed background fluid velocity, and the dots indicate the possibility of including higher-order multipole terms. The coefficients ${\bm g}_i$ of the higher-order terms are chosen to best preserve the structural integrity of the fiber (see below). \\

The simplest prescription for $\bm{g}_i$, $i=1,2,\dots$ would be to set $\bm{g}_1(t)=\bm{f}(t)$, $\bm{g}_i=0 \text{ for } i\ge 1$, where $\bm{f}(t)$ is the prescribed force density along the fiber centerline. The problem with this choice is that the surface velocity $\bu^{\SB}\big|_{\Gamma_\epsilon}(s,\theta)$ has a strong $\theta$-dependence on each constant-$s$ cross section (see left image of Figure \ref{fig:fiber_integ}). If the no-slip condition is satisfied on the fiber interface, this will lead to an instantaneous deformation of the fiber cross sectional geometry, destroying the structural integrity of the fiber. Setting $\bm{g}_2(t)=\frac{\epsilon^2}{2}\bm{g}_1(t)$ eliminates this $\theta$-dependence to leading order, so that the surface velocity is almost constant along cross sections (see right image of Figure \ref{fig:fiber_integ}). We term this $\theta$-independence constraint the {\em fiber integrity condition}. Note that the fiber integrity condition is a key feature of most slender body theories -- see, for example, \cite{tornberg2004simulating} and \cite{cox1970motion}. \\

We note that the fiber integrity constraint ignores torque and does not allow the fiber to simply rotate about its centerline. The additional consideration of torque along the fiber (explored in \cite{keller1976slender}; see also \cite{lim2004simulations}) is an extension to the classical slender body approximation \eqref{SBT2} that will be addressed in future work. \\

\begin{figure}[!h]
\centering
\includegraphics[scale=0.5]{vf_nofix}
\includegraphics[scale=0.5]{vf_SB}
\caption{A sketch of the reasoning behind the fiber integrity condition. If the fiber surface velocity $\bu^{\SB}\big|_{\Gamma_{\epsilon}}$ depends strongly on the angle $\theta$, the cross sectional shape of the fiber will deform in the next time instant (left image). When $\theta$-independence is imposed on the surface of each cross section (right image), we ensure the structural integrity of the fiber over time.}
\label{fig:fiber_integ}
\end{figure}

The classical (non-local) slender body approximation to the fluid velocity at a point $\bx$ away from the centerline is thus given by 
\begin{equation}\label{SBT2}
\begin{aligned}
8\pi\mu\bu^{\SB}(\bx) &= \int_{\T}\bigg( \mathcal{S}(\bm{R}) +\frac{\epsilon^2}{2}\mathcal{D}(\bm{R}) \bigg){\bm f}(t) \ts dt; \; \bm{R} = \bx-\X(t), \\
\mc{S}(\bm{R}) &= \frac{{\bf I}}{\abs{\bm{R}}} + \frac{\bm{R}\bm{R}^{\rm T}}{|\bm{R}|^3}, \quad \mc{D}(\bm{R}) = \frac{{\bf I}}{|\bm{R}|^3} - 3\frac{\bm{R}\bm{R}^{\rm T} }{|\bm{R}|^5}.
\end{aligned}
\end{equation}

The corresponding slender body approximation to the pressure in the fluid is given by 
\begin{equation}\label{SB_press0}
p^{\SB}(\bx) = \frac{1}{4\pi}\int_{\T} \frac{\bm{R}\cdot {\bm f}(t)}{|\bm{R}|^3} \ts dt. 
\end{equation}

To approximate the velocity of the slender body itself, a centerline expression $\bu^{\SB}_C(s)$ is often formulated following the matched asymptotics approach of Keller-Rubinow \cite{keller1976slender}. The expression \eqref{SBT2} is evaluated at $\rho=\epsilon$ and the resulting integral kernel $\mc{S}(s,\theta,t;\epsilon) + \frac{\epsilon^2}{2}\mc{D}(s,\theta,t;\epsilon)$ is expanded asymptotically about $\epsilon=0$ to obtain an integral equation on $\X(s)$ approximating $\bm{f}(s)$ given $\bu(s)$. For a periodic filament, the Keller-Rubinow formula (see \cite{shelley2000stokesian, cortez2012slender} for periodization of the original formula) is given by 

\begin{equation}\label{SBT_asymp}
\begin{aligned}
8\pi \mu \ts \bu^{\SB}_C(s) &= \big[({\bf I}- 3\be_t\be_t^{\rm T})-2({\bf I}+\be_t\be_t^{\rm T}) \log(\pi\epsilon/4) \big]{\bm f}(s) \\
&\qquad + \int_{\T} \left[ \left(\frac{{\bf I}}{|\bm{R}_0|}+ \frac{\bm{R}_0\bm{R}_0^{\rm T}}{|\bm{R}_0|^3}\right){\bm f}(t) - \frac{{\bf I}+\be_t(s)\be_t(s)^{\rm T} }{|\sin (\pi(s-t))/\pi|} {\bm f}(s)\right] \ts dt.
\end{aligned}
\end{equation}

Here $\bm{R}_0(s,t) := \X(s) -\X(t)$. The centerline expression \eqref{SBT_asymp} is typically used in numerical simulations to update the position of the fiber centerline. \\

Our aim is to establish a rigorous error estimate for the slender body approximation \eqref{SBT2} as well as the centerline approximation \eqref{SBT_asymp}.


%%%%%%%%%%%%%%%%%%%%%%%%%%%%%%%%%%%%%%%%%%%%%%%%%%%%%%%%%%
\subsection{Slender body PDE formulation}
We must first determine a well-posed PDE for reconstructing a Stokes flow in $\R^3$ given only one-dimensional force data $\bm{f}(s)$. Since this total force alone is not sufficient information to uniquely solve a Stokes boundary value problem, we also impose a fiber integrity condition: the surface velocity of the fiber at each $s$ cross section must be independent of the angle $\theta$. 

We formulate the slender body problem as a boundary value problem for the Stokes system over the fluid domain $\Omega_{\epsilon}=\R^3\backslash \overline{\Sigma_{\epsilon}}$. Note that by rescaling, we can take the viscosity $\mu\equiv 1$. Let $\bm{\sigma}= \nabla \bu+(\nabla\bu)^{\rm T} -p{\bf I}$ denote the stress tensor and ${\bm n}=-\cos\theta\be_{n_1}(s)-\sin\theta\be_{n_2}(s)=-\be_{\rho}(s,\theta)$ denote the unit normal vector pointing into the slender body at each point $(s,\theta)\in \Gamma_{\epsilon}$. We define the slender body PDE as follows: 
\begin{equation}\label{exterior_stokes}
\begin{aligned}
-\Delta \bu +\nabla p &= 0, \; \dive \ts \bu = 0 \quad \text{in } \Omega_{\epsilon} = \R^3 \backslash \Sigma_{\epsilon}, \\
\int_0^{2\pi} (\bm{\sigma} {\bm n}) \ts \mathcal{J}_{\epsilon}(s,\theta) \ts d\theta &= {\bm f}(s) \hspace{1.5cm} \text{ on } \Gamma_{\epsilon}, \\
\bu|_{\Gamma_{\epsilon}} &= \bu(s) \quad \text{(unknown but independent of }\theta), \\
|\bu| \to 0 & \text{ as } |\bx|\to \infty.
\end{aligned}
\end{equation}
Here we use the expression for the Jacobian factor $\mathcal{J}_{\epsilon}(s,\theta)$ given by \eqref{Jeps_def}. In this formulation, the boundary data is specified as partial Neumann and partial Dirichlet information everywhere along the boundary $\Gamma_\epsilon$. Fiber movements are constrained by the partial Dirichlet condition $\bu\big|_{\Gamma_\epsilon}=\bu(s)$, so the fiber may bend along its centerline, but cross sections maintain their circular shape and radius $\epsilon$ over time. Since the expression for $\bu\big|_{\Gamma_\epsilon}$ is not specified beyond the $\theta$-independence, an infinite family of flows $\bu$ satisfy this constraint. The only given data in the above system is $\bm{f}:\T\to\R^3$, the one-dimensional force density along the fiber centerline. We define $\bm{f}$ to be the total surface force $(\bm{\sigma}\bm{n})\big|_{\Gamma_\epsilon}$ acting on the body over each cross section, weighted by the surface area of the fiber via $\mc{J}_\epsilon(s,\theta)$: greater surface area contributes more to the total force along the centerline; smaller surface area contributes less. To close the system, we require that the velocity $\bu$ decays to 0 as $\abs{\bx}\to\infty$.\\

Note that the boundary integral formulation in \cite{koens2018boundary} may be a more familiar representation of Stokes flow about a three-dimensional object, but assumes knowledge of the surface traction at each point over the slender body surface. In our formulation, the only data specified is the line force density $\bm{f}(s)$. Notice that the fiber integrity condition, common to all  slender body theories, then plays an essential role, allowing us to obtain a unique velocity field given only this one-dimensional force data. \\

\begin{figure}[!h]
\centering
\includegraphics[scale=0.67]{SB_fandu}
\caption{In the slender body problem, we specify a line force density $\bm{f}(s)$ everywhere along $\Gamma_\epsilon$ and also require that the (unknown) fiber surface velocity $\bu\big|_{\Gamma_\epsilon}$ is independent of the angle $\theta$.}
\label{fig:fandu}
\end{figure}

As far as we know, this type of elliptic boundary value problem has not been explored in the literature. However, this formulation appears to be the natural PDE interpretation of the slender body problem, as any smooth enough solution to \eqref{exterior_stokes} satisfies the identity
\begin{align*}
\int_{\Omega_{\epsilon}} 2 \ts |\E(\bu)|^2\ts d\bx &= \int_{\Gamma_{\epsilon}} \bu(s)\cdot(\bm{\sigma}{\bm n}) \ts \mathcal{J}_{\epsilon} \ts d\theta ds \\
&= \int_{\T} \bu(s)\cdot {\bm f}(s)\ts ds, \hspace{1cm} \E(\bu) = \frac{\nabla\bu+(\nabla\bu)^{\rm T}}{2},
 \end{align*}
 where $\E(\bu)$ is the strain rate tensor, or symmetric gradient. This expression has a natural physical interpretation: the dissipation per unit time due to viscosity (left hand side) balances the power exerted by the slender body (right hand side). As we will see in Section \ref{PDE_stokes}, this identity is also the basis for our well-posedness theory.  \\
 
%%%%%%%

We show that the PDE \eqref{exterior_stokes} is well-posed in the homogeneous Sobolev space $D^{1,2}(\Omega_{\epsilon})$ (see \eqref{D12_definition} for a definition). Using the definition of weak solution given by Definition \ref{weak_sol_def} and \eqref{weak_exterior_p}, we show the following theorem: 
\begin{theorem}\emph{(Well-posedness of slender body PDE)}\label{stokes_theorem} 
Let $\Omega_{\epsilon}= \R^3\backslash \overline{\Sigma_{\epsilon}}$ for $\Sigma_{\epsilon}$ with $C^2$ centerline $\X(s)$ satisfying the geometric constraints in Section \ref{geometric_constraints}. Given ${\bm f}\in L^2(\T)$, there exists a unique weak solution $(\bu,p)\in D^{1,2}(\Omega_{\epsilon})\times L^2(\Omega_{\epsilon})$ to \eqref{exterior_stokes} satisfying 
\begin{equation}\label{stokes_est}
\|\bu\|_{D^{1,2}(\Omega_{\epsilon})} + \|p\|_{L^2(\Omega_{\epsilon})} \le |\log\epsilon|^{1/2}c_{\kappa} \|{\bm f}\|_{L^2(\T)},
\end{equation}
where the constant $c_{\kappa}$ depends only on the constants $c_{\Gamma}$ and $\kappa_{\max}$ characterizing the shape of the fiber centerline. 

%Furthermore, if the slender body centerline $\X(s)$ is at least $C^4$ and the force ${\bm f}(s)$ is in $H^{1/2}(\T)$, then $(\bu,p)$ is a strong solution to \eqref{exterior_stokes}; i.e. $(\bu,p)$ is in $D^{2,2}(\Omega_{\epsilon})\times H^1(\Omega_{\epsilon})$ and satisfies \eqref{exterior_stokes} pointwise almost everywhere. Furthermore, the strong solution pair $(\bu,p)$ satisfies the estimate 
%\begin{equation}\label{stokes_est_epsilon}
%\|\bu\|_{D^{2,2}(\Omega_{\epsilon})} + \|p\|_{H^1(\Omega_{\epsilon})} \le \epsilon^{-1}|\log\epsilon|^{1/2}c_{\kappa} \| {\bm f}\|_{H^{1/2}(\T)}
%\end{equation} 
%where $c_{\kappa}$ depends on $c_{\Gamma}$, $\kappa_{\max}$, and the first and second derivatives of the moving frame coefficients $\kappa_1(s)$ and $\kappa_2(s)$ in \eqref{moving_ODE}. 
\end{theorem}

%Comparing \eqref{stokes_est} and \eqref{stokes_est_epsilon} suggests that it may be possible to relax the regularity assumption on $\bm{f}$ in \eqref{stokes_est} to $\bm{f}\in H^{-1/2}(\T)$; however, we do not explore this here. Furthermore, the higher regularity theory behind \eqref{stokes_est_epsilon} gives rise to a solution satisfying the slender body PDE \eqref{exterior_stokes} in a classical sense (pointwise almost everywhere), and allows us to give meaning to the surface force $\bm{\sigma}{\bm n}|_{\Gamma_{\epsilon}}$ as a function in $H^{1/2}(\Gamma_{\epsilon})$. \\

The explicit $\epsilon$-dependence of the constant $c_{\kappa}|\log\epsilon|^{1/2}$ is determined by the various inequalities used in the well-posedness theory for \eqref{exterior_stokes}, which will be summarized in Section \ref{constants0}. We are ultimately interested in using the solution theory framework established for Theorem \ref{stokes_theorem} to estimate the error between the true solution and the slender body approximation in terms of the slender body radius $\epsilon$. For this, it is important to be able to characterize and control the $\epsilon$-dependence in any constants arising in the solution theory. From a numerical analysis perspective, determining the $\epsilon$-dependence in the well-posedness theory for the slender body PDE is analogous to establishing the stability of a numerical algorithm. We thus verify the $\epsilon$-dependence of the Korn inequality, trace inequality, and pressure estimate. These are each classical inequalities, but their dependence on the size of the radius in the exterior of a thin, flexible fiber may not have been well known previously. In particular, our trace inequality (Lemma \ref{Trace_inequality}) is genuinely new, as we rely on the fiber integrity constraint in an essential way. The Korn and pressure inequalities shown here (Lemmas \ref{korn_eps} and \ref{divv_p_lem}) apply to more general boundary value problems in the exterior of thin domains, but their dependence on the radius of the thin domain appears to not be well documented. \\

We now state our main result comparing this true solution $\bu$ of \eqref{exterior_stokes} to the slender body approximation $\bu^{\SB}$, defined by \eqref{SBT2}. From this we may also compare the actual slender body velocity $\bu\big|_{\Gamma_\epsilon}(s)$ to the centerline approximation $\bu^{\SB}_C(s)$ \eqref{SBT_asymp}.

\begin{theorem}\emph{(Slender body theory error estimate)}\label{stokes_err_theorem} 
Let $\Omega_{\epsilon}= \R^3\backslash \overline{\Sigma_{\epsilon}}$ for $\Sigma_{\epsilon}$ with $C^2$ centerline $\X(s)$ satisfying the geometric constraints in Section \ref{geometric_constraints}. Given a force ${\bm f}(s)\in C^1(\T)$, let $\bu$ be the true solution to the slender body PDE \eqref{exterior_stokes} and let $\bu^{\SB}$ be the corresponding slender body approximation \eqref{SBT2}. Then the difference $\bu^{\SB}- \bu$, $p^{\SB}-p$ satisfies 
\begin{equation}\label{err_stokes_thm}
\|\bu^{\SB}-\bu\|_{D^{1,2}(\Omega_{\epsilon})}+ \|p^{\SB}-p\|_{L^2(\Omega_{\epsilon})} \le \epsilon|\log\epsilon| \ts c_{\kappa} \ts \|{\bm f}\|_{C^1(\T)}.
\end{equation}
Furthermore, the difference between the true velocity ${\rm Tr}(\bu)(s)$ of the slender body itself and the centerline approximation $\bu^{\SB}_C(s)$, given by \eqref{SBT_asymp}, satisfies 
\begin{equation}\label{center_err_thm}
\norm{{\rm Tr}(\bu) - \bu^{\SB}_C}_{L^2(\T)} \le \epsilon|\log\epsilon|^{3/2} \ts c_{\kappa} \ts \|{\bm f}\|_{C^1(\T)} .
\end{equation}
Here the constants $c_{\kappa}$ depend only on $c_{\Gamma}$ and $\kappa_{\max}$. 
\end{theorem}

In particular, asymptotic calculations by Johnson \cite{johnson1980improved} show that the doublet correction in \eqref{SBT2} for a curved centerline $\X(s)\in C^2(\T)$ allows the surface velocity $\bu^{\SB}\big|_{\Gamma_{\epsilon}}$ to satisfy the $\theta$-independence condition up to $O(\epsilon\abs{\log\epsilon})$, where ``$O$'' is the usual order symbol. We are able to rigorously verify this claim in Proposition \ref{ur_and_derivs}. \\

Although the slender body PDE is well-posed for rough ${\bm f}$, in order to obtain an error estimate, the force must be more regular. It is not clear that ${\bm f}\in C^1(\T)$ is optimal; however, some additional regularity on ${\bm f}$ is required in order for slender body theory to actually be an approximation to the slender body PDE. We will see that this is due to the fact that the error depends crucially on the change in the total force distribution along the fiber centerline. The other sources of error stem from the nonzero curvature of the fiber centerline as well as the finite length of the fiber. These error sources are identified in Section \ref{residual_calc} by calculating the residual between the slender body approximation and the true force and velocity along $\Gamma_{\epsilon}$. Although slender body theory is a continuous approximation to a continuous problem, this step can be considered from a numerical analysis point of view as establishing the consistency of the slender body approximation. The exact form of the error estimates in Theorem \ref{stokes_err_theorem} is derived in Section \ref{error_est_section} by combining the estimates of the residuals from Section \ref{residual_calc} with the stability estimates of Section \ref{PDE_stokes}. 
 

%%%%%%%%%%%%%%%%%%%%%%%%%%%%%%%%%%%%%%%%%%%%%%%%%%%%%%%%%%
%%%%%%%%%%%%%%%%%%%%%%%%%%%%%%%%%%%%%%%%%%%%%%%%%%%%%%%%%%
%%%%%%%%%%%%%%%%%%%%%%%%%%%%%%%%%%%%%%%%%%%%%%%%%%%%%%%%%%
%%%%%%%%%%%%%%%%%%%%%%%%%%%%%%%%%%%%%%%%%%%%%%%%%%%%%%%%%%

\section{Well-posedness of slender body PDE}\label{PDE_stokes}
In this section we prove Theorem \ref{stokes_theorem}. We begin by defining our notion of a weak solution to the slender body PDE \eqref{exterior_stokes} and, in Section \ref{constants0}, state the important inequalities arising in the well-posedness theory, as well as their dependence on $\epsilon$. Then, in Section \ref{EandU_stokes}, we show existence and uniqueness results for the weak solution to \eqref{exterior_stokes}, as well as the estimate \eqref{stokes_est}. \\

We must first define the function space $D^{1,2}(\Omega_{\epsilon})$ for which the well-posedness result is stated. We seek a solution $\bu$ to \eqref{exterior_stokes} defined over the exterior domain $\Omega_{\epsilon}=\R^3\backslash{\overline{\Sigma_{\epsilon}}}$ such that $\bu$ decays to 0 as $|\bx|\to \infty$. However, we do not expect this decay to be especially fast. In particular, we expect that $\bu$ solving \eqref{exterior_stokes} around a thin filament behaves like the Stokeslet far away from the slender body. Thus we expect $|\bu|$ to decay like $\frac{1}{|\bx|}$ as $|\bx| \to \infty$; as such, we do not expect $\bu$ to be in $L^2(\Omega_{\epsilon})$. Nevertheless, we do expect $\nabla \bu\in L^2(\Omega_{\epsilon})$, so we will consider functions in the homogeneous Sobolev space on $\Omega_{\epsilon} = \R^3\backslash \overline{\Sigma_{\epsilon}}$:
\begin{equation}\label{D12_definition}
 D^{1,2}(\Omega_{\epsilon}) = \{ \bu\in L^6(\Omega_{\epsilon}) \ts : \ts \nabla \bu\in L^2(\Omega_{\epsilon}) \}, 
 \end{equation}
 explored in detail in \cite{galdi2011introduction}, Chapter II.6 - II.10. By the Sobolev inequality 
\begin{equation}\label{sobolev_ineq}
\|\bu\|_{L^6(\Omega_{\epsilon})} \le c_S\|\nabla \bu\|_{L^2(\Omega_{\epsilon})}, \qquad c_S>0,
\end{equation}
valid in the exterior domain $\Omega_{\epsilon}\subset \R^3$, we have that
\begin{equation}\label{D12_norm}
\|\bu\|_{D^{1,2}(\Omega_{\epsilon})} \equiv \|\nabla \bu \|_{L^2(\Omega_{\epsilon})}
\end{equation}
is a norm on $D^{1,2}(\Omega_{\epsilon})$, and hence $D^{1,2}(\Omega_{\epsilon})$ is a Hilbert space arising naturally in the exterior domain $\Omega_{\epsilon}$. Letting $C_0^{\infty}(\Omega_{\epsilon})$ denote the space of smooth, compactly supported test functions in $\Omega_\epsilon$, we also define $D^{1,2}_0(\Omega_{\epsilon})$ as the closure of $C_0^{\infty}(\Omega_{\epsilon})$ in $D^{1,2}(\Omega_{\epsilon})$. \\
%We denote the dual of $D^{1,2}_0(\Omega_{\epsilon})$ by $D^{-1,2}(\Omega_{\epsilon})$. \\
%
%Since we are also interested in higher regularity, we define the spaces
%\[ D^{k,2}(\Omega_{\epsilon}) = \{ \bu\in L^6(\Omega_{\epsilon}) \ts | \ts \nabla^\ell \bu\in L^2(\Omega_{\epsilon}), \ts \ell=1,\dots,k \} \]
%along with the norm 
%\[\|\bu\|_{D^{k,2}(\Omega_{\epsilon})} = \sum_{\ell=1}^k\|\nabla^l \bu\|_{L^2(\Omega_{\epsilon})},\] 
%where $\nabla^\ell$ denotes derivatives of order $l\ge 1$. \\

With this definition of the space $D^{1,2}(\Omega_\epsilon)$, we may define the notion of a weak solution to the slender body Stokes PDE. We begin by considering the variational formulation of \eqref{exterior_stokes}. We define the space
\[ \A_{\epsilon}^{\dive}= \{\bv\in D^{1,2}(\Omega_{\epsilon}) \ts : \ts \dive \ts\bv = 0, \bv|_{\Gamma_{\epsilon}}=\bv(s) \}, \]
where the value of the function $\bv(s)$ on the boundary $\Gamma_{\epsilon}$ is unspecified but independent of the surface angle $\theta$; $\bv\in\A_{\epsilon}^{\dive}$ is such that for any $\varphi\in C_0^\infty(\Gamma_\epsilon)$, we have 
\begin{equation}\label{theta_indep} 
 \int_{\Gamma_\epsilon} \bv \frac{\p\varphi}{\p\theta} \ts dS =0.
 \end{equation} 
Note, then, that the trace operator on $\A_{\epsilon}^{\dive}$ is a function defined on both $\Gamma_\epsilon$ and $\T$, as any $\bv\in \A_{\epsilon}^{\dive}$ satisfies 
\begin{align*}
 \|{\rm Tr}(\bv)\|_{L^2(\Gamma_{\epsilon})}^2 &= \int_{\T}\int_0^{2\pi} |\big({\rm Tr}(\bv)\big)(s)|^2 \ts \mathcal{J}_{\epsilon}(s,\theta) \ts d\theta ds \\
 &= 2\pi\epsilon \int_{\T}|\big({\rm Tr}(\bv)\big)(s)|^2 \ts ds = 2\pi\epsilon \|{\rm Tr}(\bv)\|_{L^2(\T)}^2.
 \end{align*}
Here we used that $\mathcal{J}_{\epsilon}(s,\theta)= \epsilon \big(1-\epsilon(\kappa_1(s)\cos\theta+\kappa_2(s)\sin\theta) \big)$. We will make a slight abuse of notation: the trace operator $\rm{Tr}$, when applied to $\A_{\epsilon}^{\dive}$ functions, will be considered as both a function on $\Gamma_\epsilon$ and on $\T$. We then have the following trace inequality for functions $\bv\in \A_{\epsilon}^{\dive}$:
\begin{equation}\label{trace_Adiv}
\frac{1}{\sqrt{2\pi\epsilon}}\|{\rm Tr}(\bv)\|_{L^2(\Gamma_{\epsilon})}=\|{\rm Tr}(\bv)\|_{L^2(\T)} \le c_T \|\nabla\bv\|_{L^2(\Omega_{\epsilon})},
\end{equation}
where the $\epsilon$-dependence of the constant $c_T$ will be specified in Section \ref{constants0}. The set $\A_{\epsilon}^{\dive}$ is nontrivial, as can be seen, for example, by taking any constant function on the surface $\Gamma_{\epsilon}$ and solving the corresponding Stokes boundary value problem in $\Omega_{\epsilon}$ with this boundary data (see \cite{galdi2011introduction}, Chapter V.2 for treatment of the Stokes Dirichlet boundary value problem). Furthermore, taking a sequence $\bv_k\in\A_{\epsilon}^{\dive}$ such that $\bv_k\to\bv$ in $L^2$, we immediately see that $\bv$ satisfies the $\theta$-independence condition \eqref{theta_indep} as well; hence $\A_{\epsilon}^{\dive}$ is a closed subspace of $D^{1,2}(\Omega_{\epsilon})$. \\
%, which can be shown by taking a sequence $\bu_k\in \A_{\epsilon}^{\dive}$ converging strongly to $\bu$ in $D^{1,2}(\Omega_{\epsilon})$. By \eqref{trace_Adiv}, we have strong convergence of the trace ${\rm Tr}(\bu_k)$ in $L^2(\Gamma_{\epsilon})$, and there exists a subsequence ${\rm Tr}(\bu_{k_j})\to {\rm Tr}(\bu)$ pointwise almost everywhere. Then the limit $\bu$ satisfies the $\theta$-independence condition $\bu|_{\Gamma_{\epsilon}}=\bu(s)$. Hence $\A_{\epsilon}^{\dive}$ is a Hilbert space with norm $\|\nabla\bu\|_{L^2(\Omega_{\epsilon})}$.  \\

We can then define a weak solution to \eqref{exterior_stokes} as follows:
\begin{definition}{(Weak solution to slender body Stokes PDE)}\label{weak_sol_def} 
A weak solution $\bu\in \A_{\epsilon}^{\dive}$ to \eqref{exterior_stokes} satisfies
\begin{equation}\label{weak_exterior}
\int_{\Omega_{\epsilon}} 2\ts \mathcal{E}(\bu):\mathcal{E}(\bv)\ts d\bx - \int_{\T} \bv(s)\cdot{\bm f}(s)\ts ds =0
\end{equation}
for any $\bv \in \A_{\epsilon}^{\dive}$. 
\end{definition} 

\begin{remark}
To use the language of finite element analysis, we note that the partial Dirichlet data, given by the fiber integrity condition $\bu\big|_{\Gamma_{\epsilon}}=\bu(s)$, is enforced as part of the function space $\A_{\epsilon}^{\dive}$ (an \emph{essential} boundary condition), whereas the partial Neumann data -- the total force per fiber cross section equals ${\bm f}(s)$ -- arises out of the variational formulation \eqref{weak_exterior} itself (a \emph{natural} boundary condition). 
\end{remark}

To formally verify that weak solutions of the slender body PDE \eqref{exterior_stokes} satisfy \eqref{weak_exterior}, we first note that away from $\Gamma_{\epsilon}$, the Stokes equations can be rewritten in terms of the stress tensor $\bm{\sigma} = 2 \ts\E(\bu) - p{\bf I}$ as $\dive \ts \bm{\sigma} = 0$ in $\Omega_{\epsilon}$. Assume $\bu \in \A_{\epsilon}^{\dive}\cap C_0^{\infty}(\overline\Omega_{\epsilon})$ satisfies the slender body PDE \eqref{exterior_stokes}, where $C^{\infty}_0(\overline \Omega_{\epsilon})$ denotes smooth functions uniformly continuous up to $\Gamma_\epsilon$ that vanish outside of some ball containing $\Sigma_\epsilon$. Note that this differs from the function space $C^{\infty}_0(\Omega_{\epsilon})$, which includes only functions that vanish on $\Gamma_\epsilon$. The stress tensor corresponding to $\bu$ then satisfies $\dive\ts\bm{\sigma} = 0$ in $\Omega_{\epsilon}$. Multiplying this equation by any $\bv \in \A_{\epsilon}^{\dive}\cap C^\infty_0(\overline \Omega_\epsilon)$ and integrating by parts, we have 
\begin{align*}
0 &= -\int_{\Omega_{\epsilon}} \dive \ts \bm{\sigma} \cdot \bv \ts d\bx = \int_{\Omega_{\epsilon}} \bm{\sigma} : \nabla \bv \ts d\bx - \int_{\Gamma_{\epsilon}} \bv \cdot (\bm{\sigma}{\bm n}) \ts dS \\
&= \int_{\Omega_{\epsilon}} \big(2\ts \E(\bu): \nabla \bv - p\ts \dive\ts \bv\big) \ts d\bx - \int_{\T}\int_0^{2\pi}\bv(s) \cdot(\bm{\sigma}{\bm n}) \ts \mathcal{J}_{\epsilon}(s,\theta) \ts d\theta ds \\
&= \int_{\Omega_{\epsilon}} \big(\nabla \bu: \nabla \bv+\nabla\bu^{\rm T}:\nabla \bv\big) \ts d\bx - \int_{\T}\bv(s)\cdot \int_0^{2\pi} (\bm{\sigma}{\bm n}) \ts \mathcal{J}_{\epsilon}(s,\theta) \ts d\theta ds \\
&= \int_{\Omega_{\epsilon}} 2 \ts\E(\bu): \E(\bv) \ts d\bx - \int_{\T} \bv(s) \cdot {\bm f}(s) \ts ds. 
\end{align*}
By density, this computation then holds for any $\bv \in \A_{\epsilon}^{\dive}$. Note that in the second line, we have rewritten the integral over $\Gamma_\epsilon$ in terms of the moving frame coordinates $(s,\theta)$, so the surface element becomes $dS= \mathcal{J}_{\epsilon}(s,\theta) \ts d\theta ds$. In the third line, we use that $\bv\in \A_{\epsilon}^{\dive}$ to pull the boundary term $\bv(s)$ out of the $\theta$-integral. The remaining integral in $\theta$ is exactly the force density $\bm{f}(s)$ that we defined in \eqref{exterior_stokes}. \\

Using this definition of a weak solution, we verify the existence and uniqueness claim of Theorem \ref{stokes_theorem}. Additionally, we show that the following is an equivalent definition of weak solution to \eqref{exterior_stokes} that includes a corresponding weak pressure $p\in L^2(\Omega_{\epsilon})$: 
\begin{definition}{\emph{(Weak solution with pressure)}}\label{pressure_exist}
Given a weak solution $\bu$ satisfying \eqref{weak_exterior}, there exists a unique corresponding pressure $p\in L^2(\Omega_{\epsilon})$ satisfying
\begin{equation}\label{weak_exterior_p}
 \int_{\Omega_{\epsilon}}\big(2 \ts\E(\bu):\E(\bv) - p\ts\dive\ts \bv\big) \ts d\bx - \int_{\T} \bv(s)\cdot{\bm f}(s) \ts ds = 0
\end{equation}
for any $\bv\in \A_{\epsilon}= \{\bv\in D^{1,2}(\Omega_{\epsilon}) \ts : \ts \bv|_{\Gamma_{\epsilon}}=\bv(s) \}$, where we have removed the divergence-free restriction on $\bv$. 
\end{definition}
We show that Definitions \ref{weak_sol_def} and \ref{pressure_exist} are equivalent in Section \ref{EandU_stokes}. Note that if $(\bu,p)\in (\A_{\epsilon}^{\dive}\cap C_0^{\infty}(\overline\Omega_{\epsilon}))\times C_0^{\infty}(\overline\Omega_{\epsilon})$ satisfies \eqref{weak_exterior_p}, then, integrating by parts, 
\begin{align*}
 0 &= -\int_{\Omega_{\epsilon}}\left(2\ts\dive(\E(\bu))\cdot \bv - \nabla p\cdot \bv\right) \ts d\bx + \int_{\Gamma_{\epsilon}}\left(2\ts \E(\bu){\bm n} - p\ts{\bm n}\right)\cdot \bv \ts dS  - \int_{\T} \bv(s)\cdot{\bm f}(s) \ts ds \\
 &= -\int_{\Omega_{\epsilon}} (\Delta\bu-\nabla p) \cdot\bv \ts d\bx + \int_{\T}\int_0^{2\pi}(\bm{\sigma}{\bm n})\cdot \bv(s) \ts \mathcal{J}_{\epsilon}(s,\theta)\ts d\theta ds  - \int_{\T} \bv(s)\cdot{\bm f}(s) \ts ds \\
 &= \int_{\Omega_{\epsilon}} (-\Delta\bu+\nabla p) \cdot\bv \ts d\bx + \int_{\T}\bv(s)\cdot\bigg(\int_0^{2\pi}(\bm{\sigma}{\bm n}) \ts \mathcal{J}_{\epsilon}(s,\theta)\ts d\theta - {\bm f}(s)\bigg) ds.
\end{align*}

Since this holds for any $\bv\in \A_{\epsilon}\cap C^\infty_0(\overline \Omega_\epsilon)$, and thus, by density, for any $\bv\in  \A_{\epsilon}$, the pair $(\bu,p)$ in fact satisfies equation \eqref{exterior_stokes} pointwise almost everywhere. Therefore, any smooth enough solution pair $(\bu,p)$ satisfying the weak formulation \ref{weak_exterior_p} is a classical solution of \eqref{exterior_stokes}. \\
%The higher regularity result of Theorem \ref{stokes_theorem} justifies this type of calculation and shows that, given ${\bm f}$ regular enough, equation \eqref{exterior_stokes} is indeed satisfied in a classical sense. \\

We begin by stating the $\epsilon$-dependence of the inequalities arising in the well-posedness theory for \eqref{exterior_stokes}, the proofs of which are given in Appendix \ref{constants}. Using these inequalities, we show the existence and uniqueness of weak solutions to \eqref{weak_exterior} and hence to \eqref{weak_exterior_p}, as well as the estimate \eqref{stokes_est} from Theorem \ref{stokes_theorem}. 

%%%%%%%%%%%%%%%%%%%%%%%%%%%%%%%
\subsection{Dependence of key inequalities on $\epsilon$}\label{constants0}
In this section we collect the key inequalities used in the well-posedness theory for \eqref{weak_exterior} and note their explicit dependence on the slender body radius $\epsilon$. This will allow us to prove the $\epsilon$-dependence in the constant arising in the estimate \eqref{stokes_est} of Theorem \ref{stokes_theorem}. As noted in the introduction, it will be important to characterize how constants in the well-posedness framework depend on $\epsilon$, as we are ultimately interested in proving the error estimate in Theorem \ref{stokes_err_theorem}. In addition, the explicit $\epsilon$-dependence in some of these inequalities is either completely new, as in the case of the trace inequality (Lemma \ref{Trace_inequality}), or not well-documented, as in the case of the Korn inequality (Lemma \ref{korn_eps}). The proofs of each inequality appear in Appendix \ref{constants}. \\

First, since we are working in the function space $D^{1,2}(\Omega_\epsilon)$ \eqref{D12_definition}, it will be useful to verify the $\epsilon$-independence of the Sobolev inequality \eqref{sobolev_ineq} on $\Omega_\epsilon$. 
 \begin{lemma}\emph{(Sobolev inequality)}\label{sobo_ineq}
Let $\Omega_{\epsilon}=\R^3\backslash\overline{\Sigma_{\epsilon}}$, the exterior of a slender body of radius $\epsilon$. For any $\bu\in D^{1,2}(\Omega_{\epsilon})$, we have
\begin{equation}\label{sobolev_const}
\| \bu\|_{L^6(\Omega_{\epsilon})} \le c_S\|\nabla\bu\|_{L^2(\Omega_{\epsilon})}
\end{equation}
with a constant $c_S$ that is bounded independent of $\epsilon$ as $\epsilon\to 0$. 
\end{lemma}
The proof of Lemma \ref{sobo_ineq} appears in Section \ref{Sob_ineq}. \\

%%%%%
We will also need to establish the $\epsilon$-dependence in the $\A_\epsilon$ trace inequality, which is the same as the $\A_\epsilon^\dive$ trace inequality \eqref{trace_Adiv}. Even though the slender body surface $\Gamma_{\epsilon}$ is codimension 1 and, for $\bu\in D^{1,2}(\Omega_{\epsilon})$, satisfies an $H^{1/2}(\Gamma_{\epsilon})$ trace inequality, the trace estimate needed for our existence theory and error bound is essentially a codimension 2 trace inequality, which appears to introduce an additional $1/\sqrt{\epsilon}$ that we must bound. However, we can show that the constant in the $L^2$ trace inequality grows only like $|\log\epsilon|^{1/2}$ as $\epsilon\to 0$.
\begin{lemma}\emph{(Trace inequality)}\label{Trace_inequality}
Let $\Omega_\epsilon = \R^3\backslash \overline{\Sigma_\epsilon}$ be as in Section \ref{geometric_constraints}. For $\bu\in \A_{\epsilon}$, the $\theta$-independent trace of $\bu$ on $\Gamma_{\epsilon}$ satisfies 
\begin{equation}\label{Trace_ineq} 
\|{\rm Tr}(\bu)\|_{L^2(\T)} \le c_{\kappa} |\log\epsilon|^{1/2} \| \nabla \bu\|_{L^2(\Omega_{\epsilon})}, 
\end{equation}
where ${\rm Tr} : D^{1,2}(\Omega_{\epsilon}) \to L^2(\T)$ is the trace operator and the constant $c_{\kappa}$ depends on the constants $\kappa_{\max}$ and $c_{\Gamma}$ but is independent of the fiber radius $\epsilon$. 
\end{lemma}

This $\epsilon$-dependence in the trace inequality is not surprising, as we expect that in the limit as $\epsilon\to 0$ the true solution will look something like the Stokeslet, which has unbounded velocity along the fiber centerline. In fact, this $\epsilon$ dependence should be optimal for the $L^2(\T)$ trace. The proof of Lemma \ref{Trace_inequality} is shown in Section \ref{trace_sec}. \\

%%%%%%

Next, in order to show estimate \eqref{stokes_est}, we will need a Korn inequality bounding $\nabla \bu$ by $\E(\bu)$, the symmetric part of the gradient. We show in Section \ref{korn_proof} that the constant in the Korn inequality is bounded independently of $\epsilon$. 
\begin{lemma}\emph{(Korn inequality)}\label{korn_eps}
Let $\Omega_{\epsilon}=\R^3 \backslash \overline{\Sigma_{\epsilon}}$ be as in Section \ref{geometric_constraints}. Then any $\bu\in D^{1,2}(\Omega_{\epsilon})$ satisfies 
\begin{equation}\label{korn_ineq}
 \|\nabla \bu\|_{L^2(\Omega_{\epsilon})} \le c_K\|\E(\bu)\|_{L^2(\Omega_{\epsilon})}, 
 \end{equation}
 where the constant $c_K$ depends only on $\kappa_{\max}$ and $c_{\Gamma}$.
\end{lemma}


%%%%%%%%%%%%%%%%%%%%%%%%%%%%%%

Finally, the $\epsilon$-dependence in the estimate \eqref{stokes_est} of Theorem \ref{stokes_theorem} relies on the $\epsilon$-independence of the following inequality, which is intimately tied to the pressure estimate \eqref{press_est} that will be used to show \eqref{stokes_est}. 

\begin{lemma}{\emph{(Solution to $\dive\ts \bv=p$)}}\label{divv_p_lem} 
Let $\Omega_{\epsilon}=\R^3\backslash \overline{\Sigma_{\epsilon}}$ be as in Section \ref{geometric_constraints}. There exists a function $\bv\in D^{1,2}_0(\Omega_{\epsilon})$ satisfying 
\begin{align*}
\dive\ts \bv &= p \quad \text{ in }\Omega_{\epsilon}; \\
\|\bv\|_{D^{1,2}(\Omega_{\epsilon})} &\le c_P\| p\|_{L^2(\Omega_{\epsilon})}, 
\end{align*}
where the constant $c_P$ depends on $\kappa_{\max}$ and $c_{\Gamma}$ but not on $\epsilon$. 
\end{lemma}
For fixed $\epsilon$, the existence of such a $\bv$ is guaranteed by \cite{galdi2011introduction}, Theorem III.3.6, which follows the original construction by Bogovskii \cite{bogovskii1980solutions}. In Section \ref{pressure_const}, we reiterate the proof of this theorem to determine the dependence of the constant $c_P$ on the slender body radius $\epsilon$. 


%%%%%%%%%%%%%%%%%%%%%%%%%%%%%%%
%%%%%%%%%%%%%%%%%%%%%%%%%%%%%%%
%%%%%%%%%%%%%%%%%%%%%%%%%%%%%%%
%%%%%%%%%%%%%%%%%%%%%%%%%%%%%%%
%%%%%%%%%%%%%%%%%%%%%%%%%%%%%%%

\subsection{Existence and uniqueness}\label{EandU_stokes}
We now use the inequalities outlined in the previous section to prove Theorem \ref{stokes_theorem}. We begin by verifying the existence and uniqueness of a weak solution $\bu$ to \eqref{weak_exterior}. 

\begin{proof}[Proof of existence and uniqueness assertion in Theorem \ref{stokes_theorem}:] 
To show existence of a weak solution $\bu \in \A_{\epsilon}^{\dive}$ to \eqref{weak_exterior}, we first show that the bilinear form 
\[ \mathcal{B}[\bu,\bv]:=\int_{\Omega_{\epsilon}} 2 \ts \E(\bu):\E(\bv) \ts d\bx \]
is coercive on $\A_{\epsilon}^{\dive}$. Using the Korn inequality \eqref{korn_ineq}, for any $\bu\in \A_{\epsilon}^{\dive}$ we have
\begin{align*}
\mathcal{B}[\bu,\bu] &= \int_{\Omega_{\epsilon}} 2 \ts |\E(\bu)|^2 \ts d\bx \ge \int_{\Omega_{\epsilon}} \frac{2}{c_K^2} \ts |\nabla\bu|^2 \ts d\bx = \frac{2}{c_K^2}\|\nabla\bu\|_{L^2(\Omega_{\epsilon})}^2 ,
\end{align*}
so $\mathcal{B}[\cdot,\cdot]$ is coercive on $\A_{\epsilon}^{\dive}$. Also, $\mathcal{B}[\cdot,\cdot]$ is bounded, since
\begin{align*}
\abs{\mathcal{B}[\bu,\bv]} &\le \int_{\Omega_{\epsilon}} 2 |\E(\bu)| |\E(\bv)| \ts d\bx \le 2 \|\E(\bu)\|_{L^2(\Omega_{\epsilon})}\|\E(\bv)\|_{L^2(\Omega_{\epsilon})} \le 2 \|\nabla \bu\|_{L^2(\Omega_{\epsilon})}\|\nabla\bv\|_{L^2(\Omega_{\epsilon})}.
\end{align*}

Furthermore, for ${\bm f}\in L^2(\T)$ and $\bv\in \A_{\epsilon}^{\dive}$, the linear functional
\[ \ell({\bm f}) := \int_{\T} {\bm f}(s)\cdot \bv(s) \ts ds \]
is bounded: using Cauchy-Schwarz and the trace inequality (Lemma \ref{Trace_inequality}) in $\A_{\epsilon}^{\dive}$,
\begin{align*}
\int_{\T} \bv(s)\cdot{\bm f}(s) \ts ds &\le \|\bv\|_{L^2(\T)}\|{\bm f}\|_{L^2(\T)} \le c_T\|\nabla\bv\|_{L^2(\Omega_{\epsilon})}\|{\bm f}\|_{L^2(\T)} .
\end{align*}

Since the form $\mathcal{B}[\cdot,\cdot]$ is bounded and coercive on $\A_{\epsilon}^{\dive}$ and the functional $\ell(\cdot)$ is bounded on $\A_{\epsilon}^{\dive}$, by the Lax-Milgram theorem there exists a unique solution $\bu\in \A_{\epsilon}^{\dive}$ to \eqref{weak_exterior}. Furthermore, taking $\bv=\bu$ in \eqref{weak_exterior} and using the Korn inequality \eqref{korn_ineq}, we have that this solution $\bu$ satisfies
\begin{align*}
\|\nabla \bu\|_{L^2(\Omega_{\epsilon})}^2 &\le c_{K}^2\|\E(\bu)\|_{L^2(\Omega_{\epsilon})}^2 \le \frac{c_K^2}{2}\|{\bm f}\|_{L^2(\T)}\|\bu\|_{L^2(\T)} \\
&\le \frac{c_K^2}{2}\left(\frac{1}{4\delta}\|{\bm f}\|_{L^2(\T)}^2+\delta\|\bu\|_{L^2(\T)}^2\right) \le \frac{c_K^2}{2}\left(\frac{1}{4\delta}\|{\bm f}\|_{L^2(\T)}^2+\delta c_T^2\|\nabla\bu\|_{L^2(\Omega_{\epsilon})}^2\right).
\end{align*}

Taking $\delta=\frac{1}{c_T^2c_K^2}$, we obtain
\begin{equation}\label{u_est}
\|\nabla \bu\|_{L^2(\Omega_{\epsilon})} \le \frac{1}{2}c_K^2c_T\|{\bm f}\|_{L^2(\T)}.
\end{equation} 
\end{proof}

%%%%%%%%%%%%%%%%%%%%%%%%%%%%%%%%%%%%%%%%%%%%%%%%%%%%%%%%%%
The existence of a unique velocity $\bu\in D^{1,2}(\Omega_{\epsilon})$ satisfying \eqref{weak_exterior} can be used to show the equivalence of Definitions \ref{weak_sol_def} and \ref{pressure_exist}, the characterization of a weak solution to \eqref{exterior_stokes} without and with the unique corresponding pressure $p\in L^2(\Omega_{\epsilon})$. The existence of the pressure relies on the following lemma, the proof of which can be found in \cite{galdi2011introduction}, Corollary III.5.1:
\begin{lemma}\emph{(de Rham Theorem)}\label{de_rham}
Let $\Omega_{\epsilon}=\R^3 \backslash\overline{\Sigma_{\epsilon}}$. Any bounded linear functional $\ell$ on $D^{1,2}_0(\Omega_{\epsilon})$ identically vanishing on the divergence-free subspace $D^{1,2}_{0,\dive}(\Omega_{\epsilon})$ is of the form 
\[ \ell(\bw)=\int_{\Omega_{\epsilon}} p \ts \dive\ts \bw \ts d\bx \qquad  \bw \in D_0^{1,2}(\Omega_{\epsilon})\]
for some uniquely determined $p\in L^2(\Omega_{\epsilon})$. 
\end{lemma}

 
 \begin{proof}[Proof of equivalence of Definitions \ref{weak_sol_def} and \ref{pressure_exist}:]
We begin by considering \eqref{weak_exterior} away from $\Gamma_{\epsilon}$. Recall the definition of $D^{1,2}_{0,\dive}(\Omega_{\epsilon})$ in Lemma \ref{pressure_exist}. Since $\bu$ is a weak solution to \eqref{weak_exterior}, we have
\[ \int_{\Omega_{\epsilon}} 2\ts\E(\bu):\E(\bw) \ts d\bx = 0 \qquad \text{ for all } \bw \in D^{1,2}_{0,\dive}(\Omega_\epsilon). \]

Using Lemma \ref{de_rham}, we then have
\begin{equation}\label{pressure_eqn}
\int_{\Omega_{\epsilon}} 2\ts\E(\bu):\E(\bw) \ts d\bx = \int_{\Omega_{\epsilon}} p \ts\dive\ts \bw \ts d\bx \qquad \text{ for all } \bw \in D_0^{1,2}(\Omega_{\epsilon}).
\end{equation}

Thus, removing the divergence-free restriction on $\bw\in D_0^{1,2}(\Omega_{\epsilon})$, we recover $p$ in $\Omega_{\epsilon}$ away from the slender body surface $\Gamma_{\epsilon}$. We now must show that this $p$ satisfies the correct boundary conditions for the total surface force over $\Gamma_{\epsilon}$ when integrated against arbitrary $\bv \in \A_{\epsilon}$. \\

Consider a solution $\bu \in \A_{\epsilon}^{\dive}$ satisfying \eqref{weak_exterior}. For any $\bv\in \A_{\epsilon}$ we write $\bv$ as 
\[ \bv = \bw+{\bm \psi} \]
where ${\bm \psi}$ is the unique (weak) solution to the classical exterior Stokes boundary value problem
\begin{equation}\label{classical_stokes}
\begin{aligned}
-\Delta {\bm \psi} + \nabla \pi &= 0, \quad \dive \ts {\bm \psi} = 0 \quad \text{ in }\Omega_\epsilon \\
{\bm \psi} \big|_{\Gamma_{\epsilon}} &= \bv(s) \\
{\bm \psi} &\to 0 \quad \text{as }|\bx| \to \infty
\end{aligned}
\end{equation}
in the space $D^{1,2}_{\dive}(\Omega_{\epsilon})$. Again the subscript ``div'' denotes the divergence-free subspace of $D^{1,2}(\Omega_{\epsilon})$. We refer to \cite{galdi2011introduction}, Chapter V.2 for details on the existence and uniqueness results for \eqref{classical_stokes}. \\

Thus ${\bm \psi}$ is in $\A_\epsilon^{\dive}$, so by Definition \ref{weak_sol_def} we have
\begin{equation}\label{div_free_part}
\int_{\Omega_{\epsilon}} 2 \ts \E(\bu):\E({\bm \psi}) \ts d\bx = \int_{\T} {\bm f}(s)\cdot\bv(s) \ts ds.
\end{equation}

Furthermore, we have that $\bw\in D_0^{1,2}(\Omega_{\epsilon})$ satisfies
\begin{equation}\label{trace_free_part}
\int_{\Omega_{\epsilon}} 2 \ts \E(\bu):\E(\bw) \ts d\bx = \int_{\Omega_{\epsilon}} p\ts \dive\ts\bw \ts d\bx,
\end{equation}
by equation \eqref{pressure_eqn}. \\

Adding \eqref{div_free_part} and \eqref{trace_free_part} we therefore have  
\begin{align*}
\int_{\Omega_{\epsilon}} 2 \ts \E(\bu):\E(\bv) \ts d\bx &= \int_{\Omega_{\epsilon}} 2 \ts \E(\bu):\E(\bw) \ts d\bx+\int_{\Omega_{\epsilon}} 2 \ts \E(\bu):\E({\bm \psi}) \ts d\bx \\
&= \int_{\Omega_{\epsilon}} p\ts \dive\ts\bw \ts d\bx + \int_{\T} {\bm f}(s)\cdot\bv(s) \ts ds.
\end{align*}

Hence the pressure $p$ from Lemma \ref{de_rham} satisfies the desired boundary condition on $\Gamma_{\epsilon}$, and therefore $(\bu,p)\in \A_{\epsilon}^{\dive}\times L^2(\Omega_{\epsilon})$ satisfies
\begin{equation}\label{weakstokes_pressure} 
\int_{\Omega_{\epsilon}} \bigg(2 \ts\mathcal{E}(\bu):\mathcal{E}(\bv) - p\ts\dive\ts \bv \bigg)\ts d\bx - \int_{\T} \bv(s)\cdot{\bm f}(s) \ts ds = 0
\end{equation}
for all $\bv \in \A_{\epsilon}$. We have thus removed the divergence-free constraint on $\bv$ to show the existence of a unique corresponding pressure $p\in L^2(\Omega_{\epsilon})$.
\end{proof} 

Finally, from \eqref{weakstokes_pressure}, we derive the energy estimate \eqref{stokes_est} in Theorem \ref{stokes_theorem}. For this we will need to use the $\epsilon$-independence of the constant $c_P$ established in Lemma \ref{divv_p_lem}. 

\begin{proof}[Proof of estimate \eqref{stokes_est}:]
Following \cite{galdi2011introduction}, we first show that for $(\bu,p)$ satisfying \eqref{weakstokes_pressure}, we have
\begin{equation}\label{press_est}
\|p\|_{L^2(\Omega_{\epsilon})} \le \tilde c_P\|\E(\bu)\|_{L^2(\Omega_{\epsilon})},
\end{equation}
for some constant $\tilde c_P>0$. To show \eqref{press_est}, we consider $\bv\in D^{1,2}_0(\Omega_{\epsilon})$ satisfying
\begin{equation}\label{divv_p}
\begin{aligned}
\dive\ts \bv &= p \qquad \text{ in }\Omega_{\epsilon}; \\
\|\bv\|_{D^{1,2}(\Omega_{\epsilon})} &\le c_P\| p\|_{L^2(\Omega_{\epsilon})}. 
\end{aligned}
\end{equation}
By Lemma \ref{divv_p_lem}, such a $\bv$ exists and the constant $c_P$ depends only on $c_\Gamma$ and $\kappa_{\max}$. \\

%%%%%%%%%%%%%%%%%%%%%%%%%%%%%
 Now, substituting $\bv$ satisfying \eqref{divv_p} into \eqref{weakstokes_pressure}, we have 
\begin{align*}
\int_{\Omega_{\epsilon}} p^2 \ts d\bx &= \int_{\Omega_{\epsilon}} 2 \ts\E(\bu):\E(\bv) \ts d\bx \le 2\|\E(\bu)\|_{L^2(\Omega_{\epsilon})}\|\E(\bv)\|_{L^2(\Omega_{\epsilon})} \le 2\|\E(\bu)\|_{L^2(\Omega_{\epsilon})}\|\nabla\bv\|_{L^2(\Omega_{\epsilon})} \\
& \le \frac{1}{\eta}\|\E(\bu)\|_{L^2(\Omega_{\epsilon})}^2+ \eta\|\nabla\bv\|_{L^2(\Omega_{\epsilon})}^2 \le \frac{1}{\eta}\|\E(\bu)\|_{L^2(\Omega_{\epsilon})}^2+ \eta c_P^2\|p\|_{L^2(\Omega_{\epsilon})}^2, \quad \eta\in\R_+. 
\end{align*}
Taking $\eta=\frac{1}{2c_P^2}$, we obtain \eqref{press_est}, with $\tilde c_P= 2c_P$. \\

Combining the pressure estimate \eqref{press_est} with the velocity estimate \eqref{u_est} and noting the $\epsilon$-dependence of the constants $c_T$, $c_K$, and $c_P$ established in Section \ref{constants0}, we obtain 
\begin{align*}
\|\bu\|_{D^{1,2}(\Omega_{\epsilon})} + \|p\|_{L^2(\Omega_{\epsilon})} &\le \frac{1}{2}c_K^2c_T(1+2c_P) \|{\bm f}\|_{L^2(\T)} \le c_{\kappa}|\log\epsilon|^{1/2}\|{\bm f}\|_{L^2(\T)}.
\end{align*}

\end{proof}

%%%%%%%%%%%%%%%%%%%%%%%%%%%%%%%%%%%%%%%%%%%%%%%%%%%%%%%%%%%%%%%%

%We now proceed to show the higher regularity result and estimate \eqref{stokes_est_epsilon} of Theorem \ref{stokes_theorem}, again tracking constants that appear throughout but not verifying the explicit $\epsilon$-dependence in \eqref {stokes_est_epsilon} until Section \ref{constants}.
%
%
%%%%%%%%%%%%%%%%%%%%%%%%%%%%%%%%%%%%%%%%%%%%%%%%%%%%%%%%%%%
%%%%%%%%%%%%%%%%%%%%%%%%%%%%%%%%%%%%%%%%%%%%%%%%%%%%%%%%%%%
%%%%%%%%%%%%%%%%%%%%%%%%%%%%%%%%%%%%%%%%%%%%%%%%%%%%%%%%%%%
%%%%%%%%%%%%%%%%%%%%%%%%%%%%%%%%%%%%%%%%%%%%%%%%%%%%%%%%%%%
%%%%%%%%%%%%%%%%%%%%%%%%%%%%%%%%%%%%%%%%%%%%%%%%%%%%%%%%%%%
%\subsection{Higher regularity}\label{high_reg}
%
%We have shown the first half of Theorem \ref{stokes_theorem}: given ${\bm f}\in L^2(\T)$, there exists a unique solution $(\bu, p)\in D^{1,2}(\Omega_{\epsilon})\times L^2(\Omega_{\epsilon})$ to the variational slender body Stokes problem \eqref{weak_exterior}. Furthermore, this solution pair satisfies the estimate \eqref{stokes_est}. We now show that if, in addition to satisfying the geometric constraints of Section \ref{geometric_constraints}, the fiber centerline is at least $C^4$ and the force ${\bm f}(s)$ is in $H^{1/2}(\T)$, we in fact have that $(\bu,p)\in D^{2,2}(\Omega_{\epsilon})\times H^1(\Omega_{\epsilon})$ and $(\bu,p)$ satisfies the estimate \eqref{stokes_est_epsilon}. The proof proceeds in three steps: 1. show higher regularity for $\bu$ and $p$ away from $\Gamma_{\epsilon}$; 2. show higher tangential regularity up to $\Gamma_{\epsilon}$; and 3. show higher regularity up to $\Gamma_{\epsilon}$ in the normal direction. 
%
%\subsubsection{Regularity away from $\Gamma_{\epsilon}$}\label{int_reg}
%We begin by showing higher regularity for $(\bu,p)$ away from the slender body surface $\Gamma_{\epsilon}$. The following arguments closely follow \cite{boyer2012mathematical}. \\
%
%We first make note of the following lemma, a version of a relation sometimes known as the Ne\v{c}as inequality, valid in the homogeneous function space $D^{1,2}(\Omega_{\epsilon})$. The proof relies on the discussion surrounding the solution to the problem $\dive \ts\bv=p$ \eqref{divv_p} at the end of Section \ref{EandU_stokes}. 
%\begin{lemma}[Generalized Poincar\'e inequality]\label{D12_necas}
%Let $\Omega_{\epsilon} =\R^3\backslash\overline{\Sigma_{\epsilon}}$. For $p\in L^1_{\text{loc}}(\Omega_{\epsilon})$ with $\nabla p\in D^{-1,2}(\Omega_{\epsilon})$, we have
%\begin{equation}
%\| p \|_{L^2(\Omega_{\epsilon})} \le c_P \|\nabla p\|_{D^{-1,2}(\Omega_{\epsilon})}
%\end{equation}
%where the constant $c_P$ is the same constant arising in the $\dive\ts\bv=p$ estimate \eqref{divv_p}. \\
%\end{lemma}
%
%\begin{proof}
%Choose $f\in C_0^{\infty}(\Omega_{\epsilon})$, and, by the discussion at the end of Section \ref{EandU_stokes}, let $\bv\in C_0^{\infty}(\Omega_{\epsilon})$ be a solution to 
%\begin{align*}
%\dive\ts \bv &= f \\
%\|\nabla \bv\|_{L^2(\Omega_{\epsilon})} &\le c_P\|f\|_{L^2(\Omega_{\epsilon})}
%\end{align*}
%for some constant $c_P>0$. The $\epsilon$-dependence of this constant will be explored in Section \ref{pressure_const}. Then
%\begin{align*}
%\bigg|\int_{\Omega_{\epsilon}} p f \ts d\bx \bigg| &= \bigg|\int_{\Omega_{\epsilon}}p\ts\dive\ts\bv\bigg| \\
%&\le \|\nabla p \|_{D^{-1,2}(\Omega_{\epsilon})}\|\bv\|_{D_0^{1,2}(\Omega_{\epsilon})} \\
%&\le c_P\|\nabla p \|_{D^{-1,2}(\Omega_{\epsilon})}\| f\|_{L^2(\Omega_{\epsilon})},
%\end{align*}
%and therefore $\|p\|_{L^2(\Omega_{\epsilon})}\le c_P\|\nabla p \|_{D^{-1,2}(\Omega_{\epsilon})}$.
%\end{proof}
%
%\begin{remark}\label{necas_remark}
%Lemma \ref{D12_necas} holds in the exterior of the slender body, $\Omega_{\epsilon}=\R^3\backslash\overline{\Sigma_{\epsilon}}$, with a constant $c_P$ that is possibly $\epsilon$-dependent. In the following section, it will also be useful to note that the same type of inequality holds over all of $\R^3$, with a constant that clearly does not depend on $\epsilon$. In particular, for $q\in L^1_{\text{loc}}(\R^3)$ with $\nabla q\in D^{-1,2}(\R^3)$, the generalized Poincar\'e inequality
%\begin{equation}\label{necas_R3}
%\| q \|_{L^2(\R^3)} \le c_q \|\nabla q\|_{D^{-1,2}(\R^3)}
%\end{equation}
%follows from solving the Poisson problem over $\R^3$. \\
%
%We again seek $\bv\in C_0^{\infty}(\R^3)$ satisfying 
%\begin{align*}
%\dive\ts\bv &= f \\
%\|\nabla \bv\|_{L^2(\R^3)} &\le c_q\| f \|_{L^2(\R^3)}
%\end{align*}
%for some $f\in C_0^{\infty}$. In $\R^3$, we can simply let $\bv=\nabla \psi$, where $\psi\in C_0^{\infty}(\R^3)$ is the solution to $\Delta\psi = q$ in $\R^3$. Then by standard Poisson solution theory (see \cite{galdi2011introduction}, Chapter II.11 for details), 
%\[\|\nabla \bv\|_{L^2(\R^3)}=\|\nabla^2\psi \|_{L^2(\R^3)} \le c_q\| q \|_{L^2(\R^3)}.\]
%Then, following the proof of Lemma \ref{D12_necas}, we obtain \eqref{necas_R3}. We leave the generalized Poicar\'e inequality over $\R^3$ as a separate remark since the constant $c_q$ is clearly independent of the slender body radius $\epsilon$, whereas the $\epsilon$-dependence of the constant $c_P$ will need to be analyzed in greater detail in Section \ref{constants}.
%\end{remark}
%
%Lemma \ref{D12_necas} and Remark \ref{necas_remark} will be useful for estimates pertaining to the higher regularity of the pressure. \\
%
%We now recall the local coordinates $(\rho,\theta,s)$ valid in the region $\mathcal{O}$ of the fiber centerline (see \eqref{region_O}). Within $\mathcal{O}$, we define the tubular region 
%\begin{equation}\label{O_double}
%\mathcal{O}^r = \{\bx(\rho,\theta,s) \ts:\ts \rho < r \}. 
%\end{equation}
%for some $r>\epsilon$. Let
%\[\Omega_r = \R^3\backslash \overline{\mathcal{O}^r} \subset \Omega_{\epsilon}. \]
%
%We want to show that $(\bu, p)\in D^{2,2}(\Omega_r)\times H^1(\Omega_r)$ for any $r>\epsilon$;  i.e. our solution $(\bu,p)$ is in fact more regular away from the slender body surface. \\ .
%
%Following (\cite{boyer2012mathematical}, Proposition III.2.3), it suffices to show that $\overline{\bu\phi} \in D^{2,2}(\R^3)$ and $\overline{p\phi}\in H^1(\R^3)$ for any $\phi \in C_0^{\infty}(\Omega_{\epsilon})$ supported away from the slender body surface $\Gamma_{\epsilon}$. Here the notation $\overline{(\cdot)}$ denotes the extension by zero to the interior of the slender body. \\
%
%Let $\bv=\overline{\bu\phi}$ and $q=\overline{p\phi}$. We thus aim to show $(\bv,q)\in D^{2,2}(\R^3)\times H^1(\R^3)$. For any vector $\bm{\psi}\in C_0^{\infty}(\R^3)$, we have
%\begin{align*}
%\int_{\R^3}(-\Delta \bv +\nabla q)\cdot \bm{\psi} &= \int_{\R^3} -\bv \Delta \bm{\psi} -q\ts\dive\ts \bm{\psi} = \int_{\Omega_{\epsilon}} - \bu\phi \Delta \bm{\psi} - p\ts\phi\dive\ts \bm{\psi} \\
%&= \int_{\Omega_{\epsilon}} - \bu(-\bm{\psi}\Delta\phi - 2\nabla\phi\nabla\bm{\psi} + \Delta(\phi\bm{\psi}))- p(-\nabla\phi\bm{\psi} +\dive(\phi\bm{\psi}))\\
%&= \int_{\Omega_{\epsilon}} \bu(\bm{\psi}\Delta\phi + 2\nabla\phi\nabla\bm{\psi})+ p\nabla\phi\bm{\psi} \\
%&= \int_{\Omega_{\epsilon}} (\bu\bm{\psi}\Delta\phi - 2\nabla\bu\nabla\phi\bm{\psi} - 2\bu\Delta\phi\bm{\psi})+ p\nabla\phi\bm{\psi} \\
%&= \int_{\Omega_{\epsilon}} (-\bu\Delta\phi - 2\nabla\bu\nabla\phi + p\nabla\phi)\cdot\bm{\psi} \\
%&= \int_{\R^3} (-\overline{\bu\Delta\phi} - 2\overline{\nabla\bu\nabla\phi} + \overline{p\nabla\phi})\cdot\bm{\psi}. 
%\end{align*}
%
%Here we have used that 
%\[\displaystyle \int_{\Omega_{\epsilon}} -\bu \Delta(\phi\bm{\psi}) - p \ts \dive(\phi\bm{\psi}) =0\]
%since $\phi\bm{\psi}\in C_0^{\infty}(\Omega_{\epsilon})$ and $(\bu, p)$ solves Stokes distributionally in $\Omega_{\epsilon}$. Thus, in the sense of distributions, we have 
%\[ -\Delta \bv +\nabla q = -\overline{\bu\Delta\phi} - 2\overline{\nabla\bu\nabla\phi} + \overline{p\nabla\phi} \equiv {\bm m}. \]
%
%Note that ${\bm m}\in L^2(\R^3)$ since $\phi\in C_0^{\infty}(\Omega_{\epsilon})$. In particular, ${\bm m}$ satisfies 
%\[ \|{\bm m}\|_{L^2(\R^3)} \le \|\Delta\phi\|_{L^3(\Omega_{\epsilon})}\|\bu\|_{L^6(\Omega_{\epsilon})} + 2 \|\nabla\phi\|_{L^{\infty}(\Omega_{\epsilon})}\|\nabla \bu\|_{L^2(\Omega_{\epsilon})}+\|\nabla\phi\|_{L^{\infty}(\Omega_{\epsilon})}\|p\|_{L^2(\Omega_{\epsilon})}. \]
%
%Similarly, we have 
%\begin{align*}
%\dive \ts \bv &= \overline{\nabla \phi \cdot \bu}+\overline{\phi \ts\dive\ts \bu} \\
%&= \overline{\nabla \phi \cdot \bu} \\
%&=: G(\bx),
%\end{align*}
%where $G(\bx)\in D^{1,2}(\R^3)$ satisfies
%\[ \|\nabla G\|_{L^2(\R^3)} \le \|\nabla^2\phi\|_{L^3(\Omega_{\epsilon})}\|\bu\|_{L^6(\Omega_{\epsilon})} + \|\nabla \phi\|_{L^{\infty}(\Omega_{\epsilon})}\|\nabla\bu\|_{L^2(\Omega_{\epsilon})}. \]
%
%We now show that $(\bv,q)\in D^{2,2}(\R^3)\times H^1(\R^3)$ using finite difference operators. For a vector ${\bm h}\in \R^3$ and a function ${\bm g}$ defined on $\R^3$, we define the translation operator $\tau_h$ by
%\[ \tau_h {\bm g}(\bx) = {\bm g}(\bx+ {\bf h}) \]
%and the difference operator $\delta_h$ by
%\[ \delta_h {\bm g}(\bx) = \tau_h {\bm g} - {\bm g} = {\bm g}(\bx+ {\bm h}) - {\bm g}(\bx).\] 
%
%Clearly, for a function ${\bm g}\in D^{k,2}(\R^3)$, we have $\delta_h {\bm g}\in D^{k,2}(\R^3)$ also. Furthermore, we have that $\delta_h$ commutes with differentiation: $\delta_h (\nabla \bv) = \nabla \bv(\bx+{\bm h}) - \nabla \bv(\bx) = \nabla (\delta_h \bv)$. We state two additional useful properties of finite difference operators, referring to \cite{boyer2012mathematical} for proof. We note that these lemmas have been adapted to the $D^{k,2}$ setting, which follows easily from $\bu\in D^{k,2} \Longrightarrow \nabla \bu \in H^{k-1}$. For ease of exposition, we define $D^{0,2}(\R^3):= L^2(\R^3)$ where applicable. 
%
%\begin{proposition}{(Properties of finite difference operators)}\label{fin_diff_prop} 
%\begin{enumerate}
%\item (\cite{boyer2012mathematical}, Lemma III.2.31): For ${\bm g}\in D^{k,2}(\R^3)$, $k\ge 0$, and any ${\bm h}\in \R^3$, we have 
%\[\|\delta_h {\bm g} \|_{D^{k-1,2}} \le |{\bm h}| \|\nabla {\bm g}\|_{D^{k-1,2}} \le |{\bm h}| \|{\bm g}\|_{D^{k,2}}.\]
%\item (\cite{boyer2012mathematical}, Proposition III.2.32): Let $(\be_1,\be_2,\be_3)$ be the canonical basis of $\R^3$. For ${\bm g}\in D^{k,2}(\R^3)$, $k\ge 0$, we define the norm 
%\[ ||| {\bm g} |||_{D^{k+1,2}} = \|{\bm g}\|_{D^{k,2}}+ \sum_{i=1}^3 \sup_{0<h<1}\frac{1}{h}\|\delta_{h\be_i} {\bm g}\|_{D^{k,2}}, \]
%where $\delta_{h\be_i}\bm{g} = \bm{g}(\bx+ h\be_i) - \bm{g}(\bx)$. The following equality holds:
%\[ D^{k+1,2}(\R^3) = \{ {\bm g}\in D^{k,2}(\R^3) \ts : \ts ||| {\bm g} |||_{D^{k+1,2}} < \infty \}, \]
%and
%\[ \|\nabla {\bm g}\|_{D^{k,2}} \le ||| {\bm g} |||_{D^{k+1,2}} \quad \forall {\bm g}\in D^{k+1,2}(\R^3).\]
%\item By linearity of $\delta_h$, 
%\[\delta_h\mathcal{E}(\bu) = \frac{1}{2}(\delta_h\nabla\bu+\delta_h(\nabla\bu)^{\rm T}) = \frac{1}{2}(\nabla(\delta_h\bu)+(\nabla(\delta_h\bu))^{\rm T})=\E(\delta_h\bu),\]
%and thus for the stress tensor $\bm{\sigma}$ we have $\delta_h\bm{\sigma}=\nabla (\delta_h\bu)+(\nabla(\delta_h\bu))^{\rm T}-(\delta_h p){\bf I}$. \\
%\end{enumerate}
%\end{proposition}
%
%Therefore we have that $(\bv,q)\in D^{1,2}(\R^3)\times L^2(\R^3)$ satisfies 
%\begin{equation}\label{finite_stokes}
%\begin{aligned}
%-\Delta \delta_h \bv + \nabla \delta_h q &= \delta_h {\bm m} \\
%\dive\ts\delta_h\bv &= \delta_h G 
%\end{aligned}
%\end{equation}
%on $\R^3$ in the weak sense; i.e.
%\begin{equation}\label{fin_diff_stokes}
%\int_{\R^3} 2\ts \mathcal{E}(\delta_h \bv):\mathcal{E}(\bw) \ts d\bx= \int_{\R^3}(\delta_h q)\dive \ts \bw \ts d\bx + \int_{\R^3} (\delta_h\bm{m})\bw \ts d\bx
%\end{equation}
%for all $\bw\in D^{1,2}_0(\R^3)$ with support away from the slender body $\Sigma_{\epsilon}$. Since $\phi\in C_0^{\infty}$ is supported away from $\Sigma_{\epsilon}$, by construction, $\delta_h\bv \in D^{1,2}_0(\R^3)$ is a suitable test function for \eqref{fin_diff_stokes}. We then have 
%\begin{align*}
%\|\nabla(\delta_h\bv)\|_{L^2(\R^3)}^2 &= \frac{1}{2}\bigg(\int_{\R^3} (\delta_h q)(\delta_hG) \ts d\bx+\int_{\R^3} (\delta_h{\bm m})(\delta_h\bv) \ts d\bx \bigg)\\
%&\le \frac{1}{2}\bigg(\|\delta_h q\|_{L^2(\R^3)}\|\delta_h G\|_{L^2(\R^3)} +  \|\delta_h{\bm m}\|_{D^{-1,2}(\R^3)} \|\delta_h\bv\|_{D^{1,2}(\R^3)}\bigg)\\
%&\le \frac{1}{2}\bigg(|{\bm h}|\|\delta_h q\|_{L^2(\R^3)}\|\nabla G\|_{L^2(\R^3)} +  |{\bm h}|\|{\bm m}\|_{L^2(\R^3)} \|\nabla(\delta_h\bv)\|_{L^2(\R^3)}\bigg).
%\end{align*}
%Now, by Remark \ref{necas_remark}, $\delta_h q\in L^2(\R^3)$ satisfies
%\[ \|\delta_h q\|_{L^2(\R^3)} \le c_q \|\nabla (\delta_h q)\|_{D^{-1,2}(\R^3)}. \]
%Using the equation \eqref{finite_stokes}, we have 
%\begin{align*}
%\|\nabla (\delta_h q)\|_{D^{-1,2}(\R^3)} &=\|\Delta(\delta_h \bv)+\delta_h{\bm m}\|_{D^{-1,2}(\R^3)} \\
%&\le \|\Delta(\delta_h \bv)\|_{D^{-1,2}(\R^3)} + \|\delta_h{\bm m}\|_{D^{-1,2}(\R^3)} \\
%&\le \|\nabla(\delta_h \bv)\|_{L^2(\R^3)} + |{\bm h}|\|{\bm m}\|_{L^2(\R^3)}.
%\end{align*}
%Thus
%\[ \|\delta_h q\|_{L^2(\R^3)} \le c_q\left(\| \nabla(\delta_h\bv)\|_{L^2(\R^3)} + |{\bm h}|\|{\bm m}\|_{L^2(\R^3)}\right), \]
%
%and therefore
%\begin{align*}
%\|\nabla(\delta_h\bv)\|_{L^2(\R^3)}^2 &\le \frac{c_q}{2}\left(|{\bm h}| \| \nabla(\delta_h\bv)\|_{L^2(\R^3)} + |{\bm h}|^2\|{\bm m}\|_{L^2(\R^3)}\right)\|\nabla G\|_{L^2(\R^3)} \\
%& \hspace{3cm} + \frac{1}{2}|{\bm h}|\|{\bm m}\|_{L^2(\R^3)} \|\nabla(\delta_h\bv)\|_{L^2(\R^3)} \\
%&\le \frac{c_q}{2}|{\bm h}|^2\|{\bm m}\|_{L^2(\R^3)}\|\nabla G\|_{L^2(\R^3)} + \frac{|{\bm h}|^2}{8\eta}\left(\|{\bm m}\|_{L^2(\R^3)}^2+ c_q^2\|\nabla G\|_{L^2(\R^3)}^2 \right) \\
%& \hspace{6cm}  + \frac{\eta}{2} \|\nabla(\delta_h\bv)\|_{L^2(\R^3)}^2
%\end{align*}
%for any $\eta\in \R_+$ Taking $\eta=1$, we have
%\begin{equation}\label{int_ineq}
%\|\nabla(\delta_h\bv)\|_{L^2(\R^3)}^2 \le \frac{3 |{\bm h}|^2}{8}\left(\|{\bm m}\|_{L^2(\R^3)}^2+ c_q^2\|\nabla G\|_{L^2(\R^3)}^2\right).
%\end{equation}
%
%Since this inequality holds for arbitrary increment ${\bm h}$, by Proposition  \ref{fin_diff_prop}.2, we have $\nabla \bv \in H^1(\R^3)$, and hence $\bv\in D^{2,2}(\R^3)$. \\
%
%The pressure term $q$ then satisfies
%\begin{align*}
% \|\delta_h q\|_{L^2(\R^3)} &\le c_q\left(\|\nabla(\delta_h\bv)\|_{L^2(\R^3)} + |{\bm h}|\|{\bm m}\|_{L^2(\R^3)}\right) \\
% &\le c_q|{\bm h}|\left(\|\nabla^2 \bv\|_{L^2(\R^3)} + \|{\bm m}\|_{L^2(\R^3)}\right)
% \end{align*}
%
%for any increment ${\bm h}$, and thus $q\in H^1(\R^3)$. Recalling that $\bv = \overline{\bu \phi}$ and $q= \overline{p\phi}$, we therefore have $(\bu, p)\in D^{2,2}(\Omega_r)\times H^1(\Omega_r)$ for any $r>\epsilon$. \\
%
%In total, we have
%\begin{align*}
%\| \nabla^2\bu \|_{L^2(\Omega_r)} + \|\nabla p\|_{L^2(\Omega_r)} &\le c_{\phi}(\| \nabla^2\bv \|_{L^2(\R^3)} + \|\nabla p\|_{L^2(\R^3)}) \\
% &\le c_{\phi}c_q\big( \|{\bm m}\|_{L^2(\R^3)} + c_q\|\nabla G\|_{L^2(\R^3)} \big) \\
% &\le \tilde c_q \big( \|\nabla\bu\|_{L^2(\Omega_{\epsilon})} + \|p\|_{L^2(\Omega_{\epsilon})} \big) \\
% &\le \tilde c_q c_K^2 c_T (1+2c_P) \|{\bm f}\|_{L^2(\T)},
%\end{align*}
%where the constant $\tilde c_q$ is independent of $\epsilon$ since $\phi\in C_0^{\infty}(\Omega_{\epsilon})$ was arbitrary. \\
%
%
%%%%%%%%%%%%%%%%%%%%%%%%%%%%%%%%%%%%%%%%%%%%%%%%%%%%%%%%%%%
%\subsubsection{Tangential regularity up to $\Gamma_{\epsilon}$}\label{tang_reg}
%We proceed with step 2 of the proof of higher regularity for Theorem \ref{stokes_theorem}: we show higher tangential regularity up to the slender body surface $\Gamma_{\epsilon}$. Recall that in the region $\mathcal{O}$ \eqref{region_O} near the slender body surface, a point $\bx$ in space is uniquely specified as 
%\[\bx(\rho,\theta,s) = \X(s) + \rho\be_{\rho}(s,\theta),\]
%In addition, we define the region $\mathcal{O}' \subset \mathcal{O}$ of $\Gamma_{\epsilon}$ as
%\begin{equation}\label{nbd_O2}
%\mathcal{O}' = \bigg\{\bx \in \Omega_{\epsilon} \ts : \ts \bx= \bx(\rho,\theta,s), \ts \epsilon < \rho< \frac{1}{4 \kappa_{\max}}\bigg\}. 
%\end{equation}
%
%For any $g\in H^k(\Omega_{\epsilon})$, we denote 
%\begin{align*}
%\nabla_N g &= (\nabla g \cdot \be_{\rho})\be_{\rho} \quad \text{ in } \mathcal{O}, \text{ and} \\
%\nabla_T g &= \nabla g - \nabla_N g \quad \text{ in } \mathcal{O}.
%\end{align*}
%
%By Theorem III.3.14 in \cite{boyer2012mathematical}, since $\bu\in D^{1,2}(\Omega_{\epsilon})\cap D^{2,2}(\Omega_r)$ for any $r>\epsilon$, we will have that $\bu\in D^{2,2}(\Omega_{\epsilon})$ -- in particular, up to $\Gamma_{\epsilon}$ -- if and only if $\nabla_T \bu \in H^1(\mathcal{O})$ and $\nabla_N \bu\in H^1(\mathcal{O})$. We begin by showing $\nabla_T \bu \in H^1(\mathcal{O})$. To do so, we must make use of the additional regularity of the prescribed force ${\bm f}\in H^{1/2}(\T)$. In contrast to more traditional boundary value problems with Dirichlet, Neumann, or Robin boundary data specified pointwise along the surface, the data ${\bm f}(s)$ is defined only in an average sense over each cross section of the slender body surface $\Gamma_{\epsilon}$. Thus we cannot rely on the often-used method of mapping open subsets of the surface $\Gamma_{\epsilon}$ to the plane in $\R^3$ and using the flat finite difference operators from Section \ref{int_reg} in the upper half-space $\R^3_+$ to show higher tangential regularity. Instead, each cross section $s$ of $\Gamma_{\epsilon}$ must be considered as a complete circle, $0\le \theta<2\pi$, the entirety of which is needed to define and hence use the the data ${\bm f}(s)$. In order to show that $\nabla_T \bu \in H^1(\mathcal{O})$ while leaving each cross section $s$ of $\Gamma_{\epsilon}$ intact, we follow the construction in Boyer-Fabrie \cite{boyer2012mathematical} and define translation operators $\tau^s$ and $\tau^{\theta}$ acting tangent to the surface $\Gamma_{\epsilon}$. \\
%
%We note that, within the neighborhood $\mathcal{O}$, any vector field $\bm{\gamma} \in C^2(\overline\Omega_{\epsilon})$ with $\bm{\gamma} \cdot \be_{\rho} = 0$ can be written as a linear combination of unit vectors in the $s$ and $\theta$ directions: $\be_t(s)$ and $\be_{\theta}(s) = -\sin\theta\be_{n_1}(s)+\cos\theta\be_{n_2}(s)$. To track possible dependence of any resulting constants on the slender body radius $\epsilon$, we decompose the tangential translation operators into the $\theta$ and $s$ directions and exploit the moving frame geometry of the slender body.  \\
%
%For $\bx\in \mathcal{O}'$, we define the tangential translation operators
%\begin{align*}
%\bar\tau_h^{\theta}\big(\bx(\rho,\theta,s)\big) &= \bx(\rho,\theta+2\pi h,s)  \\
%\bar\tau_h^s\big(\bx(\rho,\theta,s) \big) &= \bx(\rho,\theta,s+h)
%\end{align*}
%for $h\in[0,1)$. To extend this definition of translation operator globally throughout $\Omega_{\epsilon}$, beyond the region $\mathcal{O}'$, for $j=\theta,s$ we define 
%\begin{equation}\label{tang_trans_def}
%\tau_h^{j}(\bx) = \bar \tau_{\phi(\rho) h}^{j}(\bx)
%\end{equation}
%where $\phi(\rho)$ is a smooth cutoff function equal to 1 for $\rho<\frac{1}{8 \kappa_{\max}}$ and equal to zero beyond $\rho=\frac{1}{4 \kappa_{\max}}$ with $|\phi'(\rho)| \le c\kappa_{\max} = c_{\kappa}$. In other words, the translation ``step size" varies smoothly from $h$ near the slender body surface to 0 in $\R^3\backslash\mathcal{O}'$; in particular, we have 
%\begin{align*}
%\tau_h^{j}(\bx) &= \begin{cases}
%\bar\tau_h^{j}(\bx), & \quad \epsilon < \rho \le \frac{1}{8\kappa_{\max}}, \\
%\bx, & \quad \bx\in \R^3\backslash \mathcal{O}'. \\
%\end{cases} 
%\end{align*}
% Note that due to the periodicity in both the $s$ and $\theta$ directions, the tangential difference operators $\tau_h^{\theta}$ and $\tau_h^s$ are bijections preserving $\Gamma_{\epsilon}$ and the regions $\mathcal{O}$, $\mathcal{O}' \subset\Omega_{\epsilon}$. \\
%
%Letting $\tau_h^j g = g\circ \tau_h^j(\bx)$, we define the tangential finite difference operator 
%\[\delta_h^j g = \tau_h^j g - g, \qquad j = s,\theta\]
%and note that by definition of $\tau_h^j$ this quantity vanishes outside of the region $\mathcal{O}'$ and hence outside of $\mathcal{O}$ as well. In particular, for $g\in D^{1,2}(\Omega_{\epsilon})$, we have $\delta_h^j g\in H^1(\mathcal{O})$. \\
%
%The tangential translation and difference operators $\tau^{j}_h$ and $\delta_h^j$, $j=\theta,s$, behave similarly to the affine translation operator defined previously, but now translation does not commute with differentiation. For any function $g$ and differential operator $D$, we define the commutator
%\begin{equation}\label{comm2}
%[D, \tau^{\gamma}_h]g = D(\tau^{j}_h g) - \tau^{j}_h(Dg) =D(\delta^{j}_h g) - \delta^{j}_h(Dg),
%\end{equation}
%and for any two functions $g_1$ and $g_2$, we define
%\begin{equation}\label{comm1}
%\begin{aligned}
%\{g_1,g_2\}_h^{j} &= (\tau^{j}_h g_1)g_2 - g_1(\tau^{j}_{-h}g_2) = (\delta^{j}_h g_1)g_2 - g_1(\delta^{j}_{-h}g_2) \\
%\end{aligned}
%\end{equation}
%for $j=\theta, s$.\\
%
%For $g\in H^k$, $k\ge 0$, we also define the norm
%\begin{equation}\label{tang_norm}
% |||g|||_{T, H^{k+1}} = \sup_{0<h<1} \frac{1}{h}\bigg\|\frac{1}{\rho}\delta^{\theta}_h g \bigg\|_{H^k}+ \sup_{0<h<1} \frac{1}{h}\|\delta^{s}_h g\|_{H^k} . 
% \end{equation}
% 
%See Appendix \ref{appendix} for additional properties and estimates related to the tangential finite difference operators.\\
%
%The following higher regularity estimates will depend on $C^4$ regularity of the fiber centerline $\X(s)$; in particular, we will require bounds on the first and second derivatives of the moving frame coefficients $\kappa_1(s)$ and $\kappa_2(s)$ from \eqref{moving_ODE}. We therefore define
%\begin{equation}\label{kap_deriv_consts}
%\begin{aligned}
%m_{\kappa,1} &:= \max_{s\in\T^1}(\abs{\kappa_1'(s)}+ \abs{\kappa_2'(s)})\\
%m_{\kappa,2} &:= \max_{s\in\T^1}(\abs{\kappa_1''(s)}+ \abs{\kappa_2''(s)}),
%\end{aligned}
%\end{equation}
%and note that these constants will be used to show the commutator estimates in the Appendix \ref{appendix}.\\
% 
%In addition to the commutator estimates from the Appendix, we have the following property: 
%\begin{proposition}\label{tang_A}
%For any $g \in \A_{\epsilon}$, $\delta^{j}_h g$ is also in $\A_{\epsilon}$ for $j=\theta,s$.
%\end{proposition}
%\begin{proof} 
%By Proposition \ref{trans_ests}.1, $\delta_h^{j}g$ preserves the regularity of $g$, so it remains to check that $\delta_h^{j}g|_{\Gamma_{\epsilon}}$, $j=\theta,s$, is independent of $\theta$. Since $g\in \A_{\epsilon}$, we have that for each $\bx\in \Gamma_{\epsilon}$, $g(\bx)$ is independent of $\theta$. But tangential translation preserves boundaries; i.e. if $\bx\in\Gamma_{\epsilon}$, then $\tau^{j}_h(\bx)\in \Gamma_{\epsilon}$. Thus for each $\bx \in \Gamma_{\epsilon}$, $g(\tau^{j}_h(\bx)) = \tau_h^{j}g$ is independent of $\theta$, and therefore $\delta_h^{j}g=\tau_h^{j}g - g$ is independent of $\theta$ for all $\bx\in\Gamma_{\epsilon}$.
%\end{proof}
%
%Since tangential translation does not commute with differentiation, taking $\bv\in \A_{\epsilon}^{\dive}$ does not imply that $\tau^j_h \bv\in \A_{\epsilon}^{\dive}$ for $j=s$ or $\theta$. However, the existence of a unique pressure in $L^2(\Omega_{\epsilon})$ corresponding to each solution $\bu\in \A_{\epsilon}^{\dive}$ allows us to make sense of the weak Stokes equation using test functions in $\A_{\epsilon}= \{\bv \in D^{1,2}(\Omega_{\epsilon}) \ts :\ts \bv\big|_{\Gamma_{\epsilon}} = \bv(s) \}$, via \eqref{weakstokes_pressure}. For $\bv \in \A_{\epsilon}$, we do have $\tau_h^j \bv \in \A_{\epsilon}$ due to Lemma \ref{tang_A}.\\
%
%Thus we can use $\delta^j_{-h}\delta^j_{h} \bu$, $j=\theta,s$, as a test function in \eqref{weakstokes_pressure}. We have
%\begin{equation}\label{tang_op_stokes}
%\int_{\Omega_{\epsilon}}2\E(\bu):\E(\delta_{-h}^j\delta_h^j\bu) \ts d\bx = \int_{\Omega_{\epsilon}} p\ts\dive(\delta_{-h}^j\delta_h^j\bu) \ts d\bx + \int_{\T} {\bm f}(s)\delta_{-h}^j\delta_h^j\bu(s) \ts ds. 
%\end{equation}
% 
%Using the commutators \eqref{comm2} and \eqref{comm1}, we can use \eqref{tang_op_stokes} to estimate $\|\E(\delta_h^j\bu)\|_{L^2(\Omega_{\epsilon})}$. Recall that by definition of $\delta_h^j\bu = \tau_h^j \bu - \bu$, $j=s,\theta$ (see \eqref{tang_trans_def}), we have that $\delta_h^j\bu$ vanishes outside of $\mathcal{O}$. We begin by rewriting the left hand side: 
%\begin{align*}
%\int_{\Omega_{\epsilon}}2\ts \E(\bu): &\ts \E(\delta_{-h}^j\delta_h^j\bu) \ts d\bx = \int_{\mathcal{O}}2 \ts \E(\bu):\E(\delta_{-h}^j\delta_h^j\bu) \ts d\bx \\
%&= \int_{\mathcal{O}}2 \ts \E(\bu):[\nabla,\tau_{-h}^j](\delta_h^j\bu) \ts d\bx + \int_{\mathcal{O}}2 \ts \E(\bu):\left([\nabla,\tau_{-h}^j](\delta_h^j\bu)\right)^{\rm T} \ts d\bx \\
%&\hspace{2cm} + \int_{\mathcal{O}}2 \ts \E(\bu):\delta_{-h}^j\E(\delta_h^j\bu) \ts d\bx \\
%&= \int_{\mathcal{O}}2 \ts \E(\bu):[\nabla,\tau_{-h}^j](\delta_h^j\bu) \ts d\bx + \int_{\mathcal{O}}2 \ts \E(\bu):\left([\nabla,\tau_{-h}^j](\delta_h^j\bu)\right)^{\rm T} \ts d\bx \\
%& \qquad + \int_{\mathcal{O}}2 \ts \delta_{h}^j\E(\bu):\E(\delta_h^j\bu) \ts d\bx - \int_{\mathcal{O}}2 \ts \bigg\{\E(\bu),\E(\delta_h^j\bu) \bigg\}_{h} \ts d\bx \\
%&= \int_{\mathcal{O}}2 \ts \E(\bu):[\nabla,\tau_{-h}^j](\delta_h^j\bu) \ts d\bx + \int_{\mathcal{O}}2 \ts \E(\bu):\left([\nabla,\tau_{-h}^j](\delta_h^j\bu)\right)^{\rm T} \ts d\bx \\
%& \qquad - \int_{\mathcal{O}}2 \ts \left([\nabla,\tau_{h}^j]\bu \right):\E(\delta_h^j\bu) \ts d\bx - \int_{\mathcal{O}}2 \ts \left([\nabla,\tau_{h}^j]\bu \right)^{\rm T}:\E(\delta_h^j\bu) \ts d\bx \\
%& \qquad + \int_{\mathcal{O}}2 \ts \E(\delta_{h}^j\bu):\E(\delta_h^j\bu) \ts d\bx - \int_{\mathcal{O}}2 \ts \bigg\{\E(\bu),\E(\delta_h^j\bu) \bigg\}_{h} \ts d\bx. 
%\end{align*}
%
%We can then write \eqref{tang_op_stokes} as
%\begin{equation}\label{tang_op_stokes2}
%\begin{aligned}
%\int_{\mathcal{O}} \E(\delta_{h}^j\bu): \E(\delta_h^j\bu) \ts d\bx &= \frac{1}{2 }\int_{\T} {\bm f}(s)\delta_{-h}^j\delta_h^j\bu(s) \ts ds + \frac{1}{2 }\int_{\mathcal{O}} p\ts\dive(\delta_{-h}^j\delta_h^j\bu) \ts d\bx \\
%& \quad +\int_{\mathcal{O}} \big\{\E(\bu),\E(\delta_h^j\bu) \big\}_{h} \ts d\bx - \int_{\mathcal{O}} \E(\bu):[\nabla,\tau_{-h}^j](\delta_h^j\bu) \ts d\bx \\
%& \quad  - \int_{\mathcal{O}} \E(\bu):\left([\nabla,\tau_{-h}^j](\delta_h^j\bu)\right)^{\rm T} \ts d\bx + \int_{\mathcal{O}} \left([\nabla,\tau_{h}^j]\bu \right):\E(\delta_h^j\bu) \ts d\bx\\
%& \quad  + \int_{\mathcal{O}} \left([\nabla,\tau_{h}^j]\bu \right)^{\rm T}:\E(\delta_h^j\bu) \ts d\bx. 
%\end{aligned}
%\end{equation}
%
%We now proceed to bound each term on the right hand side of \eqref{tang_op_stokes2} using Proposition \ref{trans_ests} to obtain an estimate for $\|\E(\delta_{h}^j\bu)\|_{L^2(\mathcal{O})}$. Throughout, we will label any constants depending only on the geometry of $\X(s)$ as $c_{j,k}$, $j=s,\theta$, $k=0,-1$. The constants $c_{\theta,0}$ depend only on $\kappa_{\max}$ and $c_{\Gamma}$, while the constants $c_{\theta,-1}$ and $c_{s,0}$ depend on $m_{\kappa,1}$ as well. The constants $c_{s,-1}$ depend on $\kappa_{\max}$, $c_{\Gamma}$, and both $m_{\kappa,1}$ and $m_{\kappa,2}$. \\
%
%We begin by recalling the Sobolev-Slobodeckij characterization of the space $H^{1/2}$. Let $W\subset \R^n$ be any domain. A function ${\bm g}\in L^2(W)$ is in $H^{1/2}(W)$ if the seminorm
%\begin{equation}\label{H12_seminorm}
%|{\bm g}|_{H^{1/2}}^2 = \int_{W} \int_{W} \frac{|{\bm g}(\bx)-{\bm g}({\bm y})|^2}{|\bx-{\bm y}|^{1+n}} \ts d\bx d{\bm y}
%\end{equation}
%is finite. \\
%
%Using this definition, we note that for a prescribed force ${\bm f}\in H^{1/2}(\T)$, we have that $\delta^s_h {\bm f}\in H^{1/2}(\T)$ satisfies
%\begin{align*}
%\|\delta_h^s {\bm f}(s)\|_{L^2(\T)}^2 &= \int_{\T} (\delta_h^s {\bm f}(s))^2 \ts ds = \int_{\T} |{\bm f}(s+h)-{\bm f}(s)|^2 \ts ds \\
%& = h^2\int_{\T} \frac{|{\bm f}(s+h)- {\bm f}(s)|^2}{h^2} \ts ds \\
%&\le h^2\int_{\T}\int_{\T} \frac{|{\bm f}(s+h)- {\bm f}(s)|^2}{h^2} \ts ds\ts dh = h^2 |{\bm f} |_{H^{1/2}(\T)}^2,
%\end{align*}
%since the integrand $\frac{|{\bm f}(s+h)-{\bm f}(s)|^2}{h^2}$ is clearly nonnegative. \\
%
%We use this estimate to bound the first term in \eqref{tang_op_stokes2} for $j=s$. Note that this first term vanishes for $j=\theta$, by Proposition \ref{trans_ests}.2. For $j=s$, by \eqref{est_2.5_s}, we have 
%\begin{align*}
%\frac{1}{2 }\int_{\T} {\bm f}(s)\delta_{-h}^s\delta_h^s\bu(s) \ts ds &= \frac{1}{2 }\int_{\T} \delta_{h}^s{\bm f}(s)\delta_h^s\bu(s) \ts ds - \frac{1}{2 }\int_{\T} \{{\bm f},\delta_h^s\bu\}_h \ts ds \\
%&\le \frac{1}{2 } \|\delta_{h}^s{\bm f}\|_{L^2(\T)}\|\delta_{h}^s{\bu}\|_{L^2(\T)} + c_{s,0}\frac{|h|}{2 } \|{\bm f}\|_{L^2(\Gamma_{\epsilon})}\|\delta_h^s\bu\|_{L^2(\T)} \\
%&\le \frac{|h|}{2 } |{\bm f}|_{H^{1/2}(\T)}\|\delta_{h}^s{\bu}\|_{L^2(\T)} + c_{s,0}\frac{|h|}{2 } \|{\bm f}\|_{L^2(\T)}\|\delta_h^s\bu\|_{L^2(\T)} \\
%&\le (1+c_{s,0})|h| \|{\bm f}\|_{H^{1/2}(\T)}\|\delta_h^s\bu\|_{L^2(\T)} \\
%&\le c_T (1+c_{s,0})|h| \|{\bm f}\|_{H^{1/2}(\T)}\|\nabla \delta_h^s\bu\|_{L^2(\mathcal{O})} \\
%&\le c_T^2 (1+c_{s,0})^2\frac{|h|^2}{\eta}\|{\bm f}\|_{H^{1/2}(\T)}^2 + \eta\|\nabla \delta_{h}^s{\bu}\|_{L^2(\mathcal{O})}^2, \quad 0<\eta\in \R. 
%\end{align*}
% 
%Using Proposition \ref{trans_ests}, equation \eqref{est_2}, for $j=s,\theta$, the third term on the right hand side of \eqref{tang_op_stokes2} can be estimated as 
%\begin{align*}
%\int_{\mathcal{O}} \big\{\E(\bu),\E(\delta_h^j\bu) \big\}_{h} \ts d\bx &\le  c_{j,0}|h| \|\E(\bu)\|_{L^2(\mathcal{O})}\|\E(\delta_h^j\bu)\|_{L^2(\mathcal{O})} \\
%&\le \frac{ c_{j,0}^2}{4\eta}|h|^2\|\E(\bu)\|_{L^2(\mathcal{O})}^2 +\eta\|\E(\delta_h^j\bu)\|_{L^2(\mathcal{O})}^2 \\
%&\le \frac{ c_{j,0}^2}{4\eta}|h|^2\|\nabla\bu\|_{L^2(\mathcal{O})}^2 + \eta\|\nabla(\delta_h^j\bu)\|_{L^2(\mathcal{O})}^2. 
%\end{align*} 
%
%By \eqref{est_3}, for $j=s,\theta$, the fourth and fifth terms on the right hand side of \eqref{tang_op_stokes2} satisfy
%\begin{align*}
%\int_{\mathcal{O}} \E(\bu):[\nabla,\tau_{-h}^j](\delta_h^j\bu) \ts d\bx + &\int_{\mathcal{O}} \E(\bu):\left([\nabla,\tau_{-h}^j](\delta_h^j\bu)\right)^{\rm T} \ts d\bx \\
%&\le 2\|\E(\bu)\|_{L^2(\mathcal{O})}\|[\nabla,\tau_{-h}^j](\delta_h^j\bu)\|_{L^2(\mathcal{O})} \\
%&\le 2 c_{j,0}|h| \|\E(\bu)\|_{L^2(\mathcal{O})}\|\nabla(\delta_h^j\bu)\|_{L^2(\mathcal{O})} \\
%&\le \frac{ c_{j,0}^2}{\eta} |h|^2\|\E(\bu)\|_{L^2(\mathcal{O})}^2+ \eta\|\nabla(\delta_h^j\bu)\|_{L^2(\mathcal{O})}^2 \\ 
%&\le \frac{ c_{j,0}^2}{\eta} |h|^2\|\nabla\bu\|_{L^2(\mathcal{O})}^2+ \eta\|\nabla(\delta_h^j\bu)\|_{L^2(\mathcal{O})}^2.
%\end{align*}
%
%Again by \eqref{est_3}, the last two terms on the right hand side of \eqref{tang_op_stokes} can be estimated as
%\begin{align*}
%\int_{\mathcal{O}} \left([\nabla,\tau_{h}^j]\bu \right):\E(\delta_h^j\bu) \ts d\bx + &\int_{\mathcal{O}} \left([\nabla,\tau_{h}^j]\bu \right)^{\rm T}:\E(\delta_h^j\bu) \ts d\bx \\
%&\le 2\|[\nabla,\tau_{h}^j]\bu\|_{L^2(\mathcal{O})} \|\E(\delta_h^j\bu)\|_{L^2(\mathcal{O})} \\
%&\le  c_{j,0}|h|\|\nabla\bu\|_{L^2(\mathcal{O})} \|\E(\delta_h^j\bu)\|_{L^2(\mathcal{O})} \\
%&\le \frac{ c_{j,0}^2}{\eta}|h|^2\|\nabla\bu\|_{L^2(\mathcal{O})}^2 + \eta\|\E(\delta_h^j\bu)\|_{L^2(\mathcal{O})}^2\\
%&\le \frac{ c_{j,0}^2}{\eta}|h|^2\|\nabla\bu\|_{L^2(\mathcal{O})}^2 + \eta\|\nabla(\delta_h^j\bu)\|_{L^2(\mathcal{O})}^2.
%\end{align*}
%
%Finally, to estimate the remaining term involving the pressure, we note that by Lemma \ref{D12_necas} we have 
%\begin{equation}\label{pressure_fun}
% \|\delta^j_h p \|_{L^2(\mathcal{O})}=\|\delta^j_h p \|_{L^2(\Omega_{\epsilon})} \le c_P \|\nabla(\delta^j_h p)\|_{D^{-1,2}(\Omega_{\epsilon})} \le c_P c_{\kappa}\|\nabla(\delta^j_h p)\|_{H^{-1}(\mathcal{O})}. 
% \end{equation}
%
%\begin{remark}
%The final inequality in \eqref{pressure_fun} follows from the fact that $\delta^j_h p$ is supported only within the region $\mathcal{O}'$. We note that for any function $g\in D^{-1,2}(\Omega_{\epsilon})$ with supp$(g)\subset \mathcal{O}'$, we have, for each $\psi\in C_0^{\infty}(\Omega_{\epsilon})$,
%\[\bigg| \int_{\Omega_{\epsilon}} g \ts \nabla \psi \ts d\bx \bigg| = \bigg| \int_{\mathcal{O'}} g \ts \nabla \psi \ts d\bx \bigg| =\bigg| \int_{\mathcal{O'}} g \ts \nabla (\phi \psi) \ts d\bx \bigg|, \]
%where we define $\phi$ to be the smooth cutoff equal to 1 in the region 
%\[ \mathcal{O}'' := \bigg\{\bx \in \Omega_{\epsilon} \ts : \ts \bx= \bx(\rho,\theta,s), \ts \epsilon < \rho< \frac{3}{8\kappa_{\max}}\bigg\} \supset \mathcal{O}' \]
%and equal to zero outside of $\mathcal{O}$. Since $\kappa_{\max}$ depends only on $\kappa$, the centerline curvature, the norm of the gradient of $\phi$ is bounded independent of $\epsilon$. We thus have $\|\phi\psi\|_{H^1_0(\mathcal{O})}\le \|\nabla\psi\|_{L^2(\Omega_{\epsilon})}+ \|\nabla \phi\|_{L^3(\Omega_{\epsilon})}\|\psi\|_{L^6(\Omega_{\epsilon})} \le c_{\kappa}\|\psi\|_{D^{1,2}(\Omega_{\epsilon})}$. \\
%
%Then, provided $\nabla(\phi\psi)$ is non-vanishing, we have
%\begin{align*}
%\frac{\big| \int_{\Omega_{\epsilon}} g \ts \nabla \psi \ts d\bx \big|}{\|\psi\|_{D^{1,2}(\Omega_{\epsilon})}} = \frac{ \big| \int_{\mathcal{O'}} g \ts \nabla (\phi \psi) \ts d\bx \big|}{\|\psi\|_{D^{1,2}(\Omega_{\epsilon})}} \le c_{\kappa}\frac{\big| \int_{\mathcal{O'}} g \ts \nabla (\phi \psi) \ts d\bx \big|}{\|\phi\psi\|_{H^1_0(\mathcal{O})}} \le c_{\kappa}\|g\|_{H^{-1}(\mathcal{O})}, 
%\end{align*}
%since functions of the form $\phi\psi$ are a subset of $H^1_0(\mathcal{O})$. Taking the supremum over $\psi\in C_0^{\infty}(\Omega_{\epsilon})$, we obtain
%\[ \|g\|_{D^{-1,2}(\Omega_{\epsilon})} \le c_{\kappa} \|g\|_{H^{-1}(\mathcal{O})}. \]
%\end{remark}
%
%%%%%%%%%%%%%%%%%%%
%
%Making use of the fact that $(\bu,p)$ is a weak solution to the Stokes equations \eqref{stokes} and using Propositions \ref{trans_ests} and \ref{trans_ests2}, we have 
%\begin{align*}
% \|\nabla(\delta^j_h p)\|_{H^{-1}(\mathcal{O})} &\le \| [\nabla, \tau_h^j] p \|_{H^{-1}(\mathcal{O})} +\| \delta^j_h(\nabla p) \|_{H^{-1}(\mathcal{O})} \\
% &\le  c_{j,-1}|h|\|p \|_{L^2(\mathcal{O})} +\|\delta^j_h( \Delta \bu) \|_{H^{-1}(\mathcal{O})} \\
% &\le  c_{j,-1}|h|\|p \|_{L^2(\mathcal{O})} + \big(\|\Delta(\delta^j_h\bu) \|_{H^{-1}(\mathcal{O})} +\|[\dive,\tau_h^j]\nabla\bu \|_{H^{-1}(\mathcal{O})} \\
% &\hspace{5cm}+\| \dive([\nabla,\tau_h^j]\bu)\|_{H^{-1}(\mathcal{O})} \big) \\
% &\le  c_{j,-1}|h|\|p \|_{L^2(\mathcal{O})} + \big(\|\nabla(\delta^j_h\bu) \|_{L^2(\mathcal{O})} +  c_{j,-1}|h|\|\nabla\bu \|_{L^2(\mathcal{O})} +\| [\nabla,\tau_h^j]\bu\|_{L^2(\mathcal{O})} \big) \\ 
% &\le c_{j,-1} |h|\left(\|p\|_{L^2(\mathcal{O})}+\|\nabla \bu\|_{L^2(\mathcal{O})}\right) +  \|\nabla(\delta^j_h \bu)\|_{L^2(\mathcal{O})}.
%\end{align*}
%
%Thus, using \eqref{pressure_fun}, we have
%\begin{equation}\label{press_tang_est}
%\|\delta^j_h p \|_{L^2(\mathcal{O})} \le c_P c_{j,-1}\left(|h|\left(\|p\|_{L^2(\mathcal{O})}+\|\nabla \bu\|_{L^2(\mathcal{O})}\right) + \|\nabla(\delta^j_h \bu)\|_{L^2(\mathcal{O})}\right).
%\end{equation}
%
%Using Proposition \ref{trans_ests2} and \eqref{press_tang_est}, the pressure term on the right hand side of \eqref{tang_op_stokes2} can be estimated as:
%\begin{equation}\label{pressure_tang}
%\begin{aligned}
%\frac{1}{2 }\int_{\mathcal{O}} p\ts\dive &(\delta_{-h}^j\delta_h^j\bu) \ts d\bx 
%=\frac{1}{2 }\int_{\mathcal{O}} \left(\delta_h^j p \ts[\dive,\tau_h^j]\bu - \{ p, \dive(\delta_h^j \bu)\}_h + p[\dive,\tau_h^j]\delta_h^j \bu \right) \ts d\bx \\
%&\le \frac{1}{2 }\bigg(\|\delta_h^j p\|_{L^2(\mathcal{O})}\|[\dive,\tau_h^j] \bu \|_{L^2(\mathcal{O})} +c_{j,0}|h| \|p\|_{L^2(\mathcal{O})}\|\dive(\delta_h^j\bu)\|_{L^2(\mathcal{O})} \\
%&\hspace{3cm} + \|p\|_{L^2(\mathcal{O})}\|[\dive,\tau_h^j]\delta_h^j\bu\|_{L^2(\mathcal{O})} \bigg) \\
%&\le c_{j,0}\bigg(|h|\|\delta_h^j p\|_{L^2(\mathcal{O})}\|\nabla \bu \|_{L^2(\mathcal{O})} + |h| \|p\|_{L^2(\mathcal{O})}\|\nabla(\delta_h^j\bu)\|_{L^2(\mathcal{O})}\bigg) \\
%&\le c_{j,-1}\bigg( |h|\|\nabla(\delta_h^j \bu)\|_{L^2(\mathcal{O})}\left(c_P\|\nabla \bu \|_{L^2(\mathcal{O})} + \|p\|_{L^2(\mathcal{O})}\right)  \\
%&\hspace{2cm} + c_P |h|^2 \|\nabla \bu\|_{L^2(\mathcal{O})}\left(\|\nabla \bu\|_{L^2(\mathcal{O})}+ \|p\|_{L^2(\mathcal{O})} \right) \bigg) \\
%&\le \frac{c_{j,-1}(c_P+1)^2}{\eta}|h|^2\left(\|\nabla \bu \|_{L^2(\mathcal{O})} + \|p\|_{L^2(\mathcal{O})}\right)^2 + \eta\|\nabla(\delta_h^j\bu)\|_{L^2(\mathcal{O})}^2 \\
%&\hspace{2cm}+ \eta (c_P+1)^2 |h|^2 \|\nabla \bu\|_{L^2(\mathcal{O})}^2.
%\end{aligned}
%\end{equation}
%
%Together, we have that $ \|\E(\delta_{h}^j\bu)\|_{L^2(\mathcal{O})}^2$ satisfies
%\begin{align*}
%  \|\E(\delta_{h}^{\theta}\bu)\|_{L^2(\mathcal{O})}^2 &\le \left(\frac{c_{\theta,-1}}{\eta}+\eta \right)(c_P+1)^2|h|^2 \big(\|\nabla \bu\|_{L^2(\mathcal{O})}^2+\|p\|_{L^2(\mathcal{O})}^2 \big)+ 5\eta \|\nabla(\delta_{h}^j\bu)\|_{L^2(\mathcal{O})}^2 \\
%\text{and} \hspace{2.2cm} & \\
% \|\E(\delta_{h}^s\bu)\|_{L^2(\mathcal{O})}^2 &\le  \left(\frac{c_{s,-1}}{\eta}+\eta \right)(c_P+1)^2 |h|^2 \big(\|\nabla \bu\|_{L^2(\mathcal{O})}^2+\|p\|_{L^2(\mathcal{O})}^2 \big)\\
% &\hspace{2cm}+ c_T^2 c_{s,0}\frac{|h|^2}{\eta}\|{\bm f}\|_{H^{1/2}(\T)}^2 +5\eta \|\nabla(\delta_{h}^j\bu)\|_{L^2(\mathcal{O})}^2.
% \end{align*}
%
%Using Korn's inequality on $\Omega_{\epsilon}$ \eqref{exterior_korn} and taking the parameter $\eta=\frac{1}{10}$, we obtain 
%\begin{equation}\label{u_reg_est}
%\begin{aligned}
% \|\nabla(\delta_{h}^{\theta}\bu)\|_{L^2(\mathcal{O})}^2 &\le c_K^2 (c_P+1)^2 c_{\theta,-1}|h|^2 \big(\|\nabla \bu\|_{L^2(\mathcal{O})}^2+\|p\|_{L^2(\mathcal{O})}^2 \big); \\
%  \|\nabla(\delta_{h}^s\bu)\|_{L^2(\mathcal{O})}^2 &\le c_K^2c_T^2 c_{s,0}|h|^2\|{\bm f}\|_{H^{1/2}(\T)}^2 \\
%  &\hspace{1cm}+ c_K^2(c_P+1)^2 c_{s,-1}|h|^2 \big(\|\nabla \bu\|_{L^2(\mathcal{O})}^2+\|p\|_{L^2(\mathcal{O})}^2 \big).
% \end{aligned}
% \end{equation}
%
%Since any tangent vector field $\bm{\gamma}$ with $\bm{\gamma}\cdot \be_{\rho}=0$ can be decomposed into $\theta$ and $s$ directions in $\mathcal{O}$, Theorem III.3.20 from \cite{boyer2012mathematical} gives us that $\nabla_T\bu \in H^1(\mathcal{O})$. \\
%
%Furthermore, using \eqref{u_reg_est} in \eqref{press_tang_est}, the pressure satisfies 
%\begin{align*}
%\|\delta^{\theta}_h p \|_{L^2(\mathcal{O})} &\le c_{\theta,-1}c_K(c_P+1)|h|\left(\|p \|_{L^2(\mathcal{O})} + \|\nabla \bu\|_{L^2(\mathcal{O})} \right)  \qquad \text{and} \\
%\|\delta^{s}_h p \|_{L^2(\mathcal{O})} &\le c_{s,-1}c_K|h|\left((c_P+1)(\|p \|_{L^2(\mathcal{O})} + \|\nabla \bu\|_{L^2(\mathcal{O})})+ c_T\|{\bm f}\|_{L^2(\mathcal{O})} \right) 
%\end{align*}
%
%Since this holds for any $h\in [-1,1]$, by Theorem III.3.20 in \cite{boyer2012mathematical}, we obtain $\nabla_Tp\in L^2(\mathcal{O})$ as well. 
%
%%%%%%%%%%%%%%%%%%%%%%%%%%%%%%%%%%%%%%%%%%%%%%%%%%%%%%%%%%%
%\subsubsection{Normal regularity up to $\Gamma_{\epsilon}$}
%We complete the proof of higher regularity for Theorem \ref{stokes_theorem} by using the interior and tangential regularity results of Sections \ref{int_reg} and \ref{tang_reg} to show that $\nabla_N \bu \in H^1(\mathcal{O})$ and $\nabla_N p \in L^2(\mathcal{O})$ and hence $(\bu,p) \in D^{2,2}(\Omega_{\epsilon})\times H^1(\Omega_{\epsilon})$. \\
%
%In the region $\mathcal{O}$, we can rewrite the Stokes equations with respect to the moving frame basis $\be_t$, $\be_{\rho}$, $\be_{\theta}$. We decompose $\bu = u_{\rho} \be_{\rho} + u_{\theta} \be_{\theta} + u_s \be_t$, where $u_{\rho} = \bu\cdot\be_{\rho}$, $u_{\theta} = \bu\cdot\be_{\theta}$, and $u_s = \bu\cdot\be_t$. Then, using the gradient and divergence with respect to the moving frame, we write the Stokes equations as: 
%\begin{align*}
%- \Delta \bu +\nabla p &= -\Delta \bu + \frac{\p p}{\p \rho}\be_{\rho} + \frac{1}{\rho}\frac{\p p}{\p\theta}\be_{\theta} +
%\frac{1}{1-\rho\wh\kappa}\bigg(\frac{\p p}{\p s}-\kappa_3\frac{\p p}{\p\theta}\bigg)\be_t =0 \\
%\dive\ts \bu &= \frac{1}{1-\rho\wh\kappa}\bigg(\frac{1}{\rho}\frac{\p(\rho (1-\rho\wh\kappa) u_{\rho})}{\p \rho}+\frac{1}{\rho}\frac{\p ((1-\rho\wh\kappa) u_{\theta})}{\p\theta} + \frac{\p u_s}{\p s} \bigg) = 0.
%\end{align*}
%Here $\wh\kappa(s,\theta) = \kappa_1(s)\cos\theta+ \kappa_2(s)\sin\theta$ satisfies that each of $\|1-\rho\wh\kappa\|_{L^{\infty}(\mathcal{O})}$, $\big\|\wh\kappa\big\|_{L^{\infty}(\mathcal{O})}$, and $\big\|\frac{\p \wh\kappa}{\p \theta}\big\|_{L^{\infty}(\mathcal{O})}$ are bounded by $c_{\kappa}$, a constant depending only on the fiber centerline geometry. Furthermore, by \eqref{denom_bound} and \eqref{denom_bound2}, both $\|1/(1-\rho\wh\kappa)\|_{L^{\infty}(\mathcal{O})}\le c_{\kappa}$ and $\big\|\frac{\p}{\p s}\frac{1}{1-\rho\wh\kappa}\big\|_{L^{\infty}(\mathcal{O})}\le c_{\kappa}$, where the latter constant also depends on the first derivatives of the coefficients $\kappa_1$ and $\kappa_2$. \\
%
%From the divergence-free condition on $\bu$, after multiplying through by $\rho (1-\rho\wh\kappa)$ and differentiating once with respect to $\rho$, we obtain
%\begin{align*}
%\bigg\| \frac{\p^2 u_{\rho}}{\p^2 \rho}\bigg\|_{L^2(\mathcal{O})} &\le c_{\kappa}\bigg(\bigg\|\frac{1}{\rho} \nabla \bu \bigg\|_{L^2(\mathcal{O})} + \bigg\|\frac{1}{\rho}\bigg\|_{L^{\infty}(\mathcal{O})}|\mathcal{O}|^{1/3}\big\|\bu \big\|_{L^6(\mathcal{O})} \\
%&\hspace{2cm}+ \bigg\|\frac{1}{\rho}\frac{\p}{\p \theta}\bigg(\frac{\p u_{\theta}}{\p\rho}\bigg)\bigg\|_{L^2(\mathcal{O})} +\bigg\|\frac{\p}{\p s}\bigg(\frac{\p u_s}{\p \rho}\bigg)\bigg\|_{L^2(\mathcal{O})} \bigg)\\
%&\le c_{\kappa,2}\bigg( \frac{1}{\epsilon} \|\nabla \bu\|_{L^2(\mathcal{O})} + c_Kc_T \|{\bm f}\|_{H^{1/2}(\T)} \\
%  &\hspace{2cm}+ c_K(c_P+1) \big(\|\nabla \bu\|_{L^2(\mathcal{O})}+\|p\|_{L^2(\mathcal{O})} \big)\bigg),
%\end{align*}
%using the tangential regularity properties of $\bu$ and that $|\mathcal{O}|$ is bounded by a constant depending only on $\kappa_{\max}$ and not on $\epsilon$. Here we use $c_{\kappa,2}$ to denote a constant depending on $\kappa_{\max}$, $c_{\Gamma}$, $m_{\kappa,1}$ and $m_{\kappa,2}$. Then, using the estimate \eqref{weak_stokes_est}, we have
%\[ \bigg\| \frac{\p^2 u_{\rho}}{\p^2 \rho}\bigg\|_{L^2(\mathcal{O})} \le c_{\kappa,2} \frac{c_T}{\epsilon}c_K(1+c_P)^2(1+c_K)^2 \|{\bm f}\|_{H^{1/2}(\T)}. \] 
%
%\begin{remark}  
%We note that the factor of $\frac{1}{\epsilon}$ in the above bound is necessary. As a heuristic, we consider an infinite straight cylinder of radius $\epsilon$ and take $\bu = (\frac{1}{\rho}-\frac{1}{\epsilon}) \be_{\theta}$, where $\be_{\theta}$ is now the (constant) angular vector in straight cylindrical coordinates, and $p\equiv$ constant. Ignoring decay conditions toward infinity along the cylinder, $(\bu,p)$ solves the Stokes equations with $\bu=0$ on the cylinder surface. Then
%\begin{align*}
%|\nabla^2\bu| =\bigg| \frac{\p^2}{\p \rho^2} \frac{1}{\rho}\bigg| = \bigg|\frac{2}{\rho^3}\bigg| = \frac{2}{\rho}\big| \nabla \bu \big|,
%\end{align*}
%and within the region $\epsilon < \rho \le 2 \epsilon$, we have $|\nabla^2\bu| \ge \frac{1}{\epsilon}|\nabla \bu|$. 
%\end{remark}
%
%Furthermore, using the $\be_{\rho}$ component of $-\Delta \bu +\nabla p=0$, we have 
%\begin{align*} 
%\frac{\p p}{\p \rho} &= (\Delta\bu) \cdot\be_{\rho} \\
%&= \frac{1}{\rho (1-\rho\wh\kappa)}\frac{\p}{\p \rho}\left(\rho (1-\rho\wh\kappa)\frac{\p \bu}{\p \rho}\right)\cdot\be_{\rho} +\frac{1}{\rho^2(1-\rho\wh\kappa)}\frac{\p}{\p \theta}\bigg((1-\rho\wh\kappa)\frac{\p \bu}{\p\theta}\bigg)\cdot\be_{\rho} \\
%&\hspace{4cm} +\frac{1}{1-\rho\wh\kappa} \frac{\p}{\p s}\bigg( \frac{1}{1-\rho\wh\kappa}\bigg[ \frac{\p \bu}{\p s}- \kappa_3\frac{\p \bu}{\p \theta} \bigg] \bigg)\cdot\be_{\rho} \\
%%
%&= \frac{1}{\rho (1-\rho\wh\kappa)}\frac{\p}{\p \rho}\left(\rho (1-\rho\wh\kappa)\frac{\p u_{\rho}}{\p \rho}\right) +\frac{1}{\rho^2(1-\rho\wh\kappa)}\frac{\p}{\p \theta}\bigg((1-\rho\wh\kappa)\frac{\p \bu}{\p\theta}\bigg)\cdot\be_{\rho} \\
%&\hspace{4cm} +\frac{1}{1-\rho\wh\kappa} \frac{\p}{\p s}\bigg( \frac{1}{1-\rho\wh\kappa}\bigg[ \frac{\p \bu}{\p s}- \kappa_3\frac{\p \bu}{\p \theta} \bigg] \bigg)\cdot\be_{\rho}, 
%\end{align*}
%since $\be_{\rho}(s,\theta)$ does not vary with $\rho$. Therefore, using the tangential regularity of $\bu$ and $p$ along with the the bound on $\frac{\p^2 u_{\rho}}{\p \rho^2}$, we have
%\[ \|\nabla p\|_{L^2(\mathcal{O})} \le c_{\kappa,2} \frac{c_T}{\epsilon}c_K(1+c_P)^2(1+c_K)^2 \|{\bm f}\|_{H^{1/2}(\T)}, \]
%where again $c_{\kappa,2}$ denotes a constant depending on $\kappa_{\max}$, $c_{\Gamma}$, $m_{\kappa,1}$ and $m_{\kappa,2}$. Thus $p\in H^1(\mathcal{O})$. \\
%
%Finally, to estimate $\frac{\p^2 u_{j}}{\p \rho^2}$, $j=\theta,s$, we again use that 
%\begin{align*} 
% \nabla p\cdot\be_j &= (\Delta \bu)\cdot \be_j(s,\theta) \\
%&= \frac{1}{\rho (1-\rho\wh\kappa)}\frac{\p}{\p \rho}\bigg(\rho (1-\rho\wh\kappa)\frac{\p u_j}{\p \rho}\bigg) +\frac{1}{\rho^2(1-\rho\wh\kappa)}\frac{\p}{\p \theta}\bigg((1-\rho\wh\kappa)\frac{\p \bu}{\p\theta}\bigg)\cdot\be_j  \\
%&\hspace{4cm}+\frac{1}{1-\rho\wh\kappa} \frac{\p}{\p s}\bigg( \frac{1}{1-\rho\wh\kappa}\bigg[ \frac{\p \bu}{\p s}- \kappa_3\frac{\p \bu}{\p \theta} \bigg] \bigg)\cdot\be_j , \quad j=\theta,s,
%\end{align*}
%since each of the basis vectors $\be_t(s)$, $\be_{\rho}(s,\theta)$ and $\be_{\theta}(s,\theta)$ are independent of $\rho$. Then we have
%\begin{align*}
% \bigg\|\frac{\p^2 u_j}{\p \rho^2}\bigg\|_{L^2(\mathcal{O})} &\le c_{\kappa} \bigg( \bigg\|\frac{1}{\rho}\bigg\|_{L^{\infty}(\mathcal{O})} \| \nabla \bu\|_{L^2(\mathcal{O})} + \bigg\|\frac{\p^2\bu}{\p s^2}\bigg\|_{L^2(\mathcal{O})} \\
% &\hspace{2cm} + \bigg\|\frac{\p^2\bu}{\p s\p\theta}\bigg\|_{L^2(\mathcal{O})} +\bigg\|\frac{\p^2\bu}{\p \theta^2}\bigg\|_{L^2(\mathcal{O})}+\|\nabla p\|_{L^2(\mathcal{O})} \bigg) \\
%&\le c_{\kappa,2} \frac{c_T}{\epsilon}c_K(1+c_P)^2(1+c_K)^2 \|{\bm f}\|_{H^{1/2}(\T)}, \quad j=\theta, s, 
% \end{align*}
% where $c_{\kappa,2}$ depends only on $\kappa_{\max}$, $c_{\Gamma}$, $m_{\kappa,1}$, and $m_{\kappa,2}$.\\
%
%Therefore, combining the interior, tangential, and normal estimates, we have that $(\bu,p)\in D^{2,2}(\Omega_{\epsilon})\times H^1(\Omega_{\epsilon})$ and satisfies the estimate 
%\begin{equation}\label{regular_est_stokes}
%\begin{aligned}
%\|\nabla^2\bu\|_{L^2(\Omega_{\epsilon})} + \|\nabla p\|_{L^2(\Omega_{\epsilon})} &\le c_{\kappa,2} \frac{c_T}{\epsilon}c_K(1+c_P)^2(1+c_K)^2 \|{\bm f}\|_{H^{1/2}(\T)},
%\end{aligned}
%\end{equation}
%where $c_{\kappa,2}$ depends only on $\kappa_{\max}$, $c_{\Gamma}$, $m_{\kappa,1}$, and $m_{\kappa,2}$.\\
%
%Thus we have shown the higher regularity estimate \eqref{stokes_est_epsilon} of Theorem \ref{stokes_theorem}, except for the explicit $\epsilon$-dependence of the constants. In the next section, we show that, of each of the constants $c_T$, $c_K$, and $c_P$, only $c_T$ depends on $\epsilon$ as $\epsilon\to 0$, growing as $|\log \epsilon|^{1/2}$.  \\
 
 
%%%%%%%%%%%%%%%%%%%%%%%%%%%%%%%%%%%%%%%%%%%%%%%%%%%%%%%%%%
%%%%%%%%%%%%%%%%%%%%%%%%%%%%%%%%%%%%%%%%%%%%%%%%%%%%%%%%%%
%%%%%%%%%%%%%%%%%%%%%%%%%%%%%%%%%%%%%%%%%%%%%%%%%%%%%%%%%%
%
%\section{Geometry revisited}\label{constants}
%So far we have proved the existence, uniqueness, and higher regularity claims in Theorem \ref{stokes_theorem} for a fixed slender body radius $\epsilon>0$. We now aim to prove the $\epsilon$-dependence in the estimate \eqref{stokes_est_epsilon} as $\epsilon \to 0$, which will eventually allow us to derive the error estimate for slender body theory in Theorem \ref{stokes_err_theorem}. In particular, we show that the Korn constant $c_K$ and pressure constant $c_P$ are independent of $\epsilon$, while the constant $c_T$ in the trace inequality has a $|\log\epsilon|^{1/2}$ dependence as $\epsilon\to 0$. This $\epsilon$-dependence in the trace inequality is not surprising, as we expect that in the limit as $\epsilon\to 0$ the true solution will look something like the Stokeslet, which has unbounded velocity along the fiber centerline. \\
%
%Throughout the following sections, we again use $c_{\kappa}$ to denote any constant depending only on $\kappa_{\max}$ and $c_{\Gamma}$ via the moving frame coefficients $\kappa_i(s)$, $i=1,2,3$ \eqref{moving_ODE}. In particular, the value of the constant $c_{\kappa}$ may change throughout the course of a computation, but is always independent of $\epsilon$. 
%
%%%%%%%
%\subsection{Trace inequality}\label{trace_sec}
%We must first establish the $\epsilon$-dependence in the $D^{1,2}(\Omega_{\epsilon})$ trace inequality 
%\[ \|{\rm Tr}(\bu)\|_{L^2(\T)} \le c_T \|\nabla \bu\|_{L^2(\Omega_{\epsilon})}. \]
%Even though the slender body surface $\Gamma_{\epsilon}$ is codimension 1 and for $\bu\in D^{1,2}(\Omega_{\epsilon})$ satisfies an $H^{1/2}(\Gamma_{\epsilon})$ trace inequality, the trace inequality we need for our existence theory and error estimate is essentially a codimension 2 trace inequality. Indeed, for $\A_{\epsilon}=\{\bu\in D^{1,2}(\Omega_{\epsilon}) \ts : \ts \bu\big|_{\Gamma_{\epsilon}} = \bu(s)\}$ we have 
%\begin{align*}
% \|{\rm Tr}(\bu)\|_{L^2(\Gamma_{\epsilon})}^2 &= \int_{\T}\int_0^{2\pi} |\big({\rm Tr}(\bu)\big)(s)|^2 \ts \mathcal{J}_{\epsilon}(s,\theta) \ts d\theta ds \\
% &= 2\pi\epsilon \int_{\T}|\big({\rm Tr}(\bu)\big)(s)|^2 \ts ds = 2\pi\epsilon \|{\rm Tr}(\bu)\|_{L^2(\T)}^2, 
% \end{align*}
%since $\mathcal{J}_{\epsilon}(s,\theta)= \epsilon \big(1-\epsilon(\kappa_1(s)\cos\theta+\kappa_2(s)\sin\theta) \big)$. Thus the trace estimate on $\T$ appears to introduce an additional $1/\sqrt{\epsilon}$ that we must bound. However, we can show that the constant in the $L^2$ trace inequality grows only like $|\log\epsilon|^{1/2}$ as $\epsilon\to 0$.
%\begin{lemma}\emph{(Trace inequality)}\label{Trace_inequality}
%Let $\Omega_\epsilon = \R^3\backslash \Gamma_\epsilon$ be as in Section \ref{geometric_constraints}. For $\bu\in \A_{\epsilon}$, the $\theta$-independent trace of $\bu$ on $\Gamma_{\epsilon}$ satisfies 
%\begin{equation}\label{Trace_ineq} 
%\|{\rm Tr}(\bu)\|_{L^2(\T)} \le c_{\kappa} |\log\epsilon|^{1/2} \| \nabla \bu\|_{L^2(\Omega_{\epsilon})}, 
%\end{equation}
%where ${\rm Tr} : D^{1,2}(\Omega_{\epsilon}) \to L^2(\T)$ is the trace operator and the constant $c_{\kappa}$ depends on the constants $\kappa_{\max}$ and $c_{\Gamma}$ but is independent of the fiber radius $\epsilon$. 
%\end{lemma}
%
%\begin{proof}
%Since the fiber centerline is $C^2$ -- and hence the surface $\Gamma_{\epsilon}$ is $C^2$ -- and the fiber does not self-intersect \eqref{non_intersecting}, we can cover the slender body by finitely many open neighborhoods $W_j$ where 
%\[ W_j = \{ \X(s)+\rho\be_\rho(s,\theta) \ts : \ts 0\le \theta < 2\pi, \ts 0 \le \rho < 1/(2\kappa_{\max}), \ts a_j < s < b_j\}, \quad j=1,\dots,N <\infty.\]
%Here $a_j$ and $b_j$ are chosen such that over each $W_j$, the fiber centerline can be considered as the graph of a $C^2$ function. Note that this choice of $a_j$ and $b_j$ depends only on the shape of the fiber centerline -- in particular, $\kappa_{\max}$ and $c_{\Gamma}$ -- and not on the fiber radius. \\
%
%Then, using a partition of unity $\{\phi_j\}_{j=1}^N$ subordinate to the cover $\{W_j\}$, there exist $\epsilon$-independent $C^2$ diffeomorphisms $\psi_j$, $j=1,\dots,N$ taking the curvature $\kappa$ of the fiber centerline to zero on the set $W_j$ while leaving the radius $\epsilon$ intact. \\ 
%
%\begin{figure}[!h]
%\centering
%\includegraphics[scale=0.5]{diffeo_SB.png}\\
%\caption{ The slender body centerline can be straightened via $\epsilon$-independent diffeomorphisms $\psi_j$; thus it suffices to consider functions $\bu$ around a straight cylinder supported within the truncated cylindrical shell $C_{\epsilon,a}$.}
%\label{fig:diffeo_SB}
%\end{figure}
%
%Let $D_{\rho}\subset \R^2$ denote the open disk of radius $\rho$ in $\R^2$. Define the straight cylindrical surface $\Gamma_{\epsilon,a}:= \p D_{\epsilon}\times [-a,a]$ and the cylindrical shell $C_{\epsilon,a}:= (D_1\backslash \overline{D_{\epsilon}})\times [-a,a]$ for some $a<\infty$, parameterized in cylindrical coordinates $(\rho,\theta,s)$. We define the function space 
%\[\A_S:= \{ \bv\in D^{1,2}(C_{\epsilon,a}) \ts : \ts \bv|_{\Gamma_{\epsilon,a}} = \bv(s); \ts \bv|_{\p C_{\epsilon,a}\backslash\Gamma_{\epsilon,a}} = 0 \}. \]
%
%Then $\psi_j^*(\phi_j\bu)\in \A_S$, and to show Lemma \ref{Trace_inequality} it suffices to prove the $\abs{\log\epsilon}^{1/2}$ dependence in the trace constant about a straight cylinder.
%\begin{lemma}\label{trace_straight_cylinder}
%Let $\bu\in \A_S$. Then the $\theta$-independent trace of $\bu$ on the straight cylinder $\Gamma_{\epsilon,a}$ satisfies
%\begin{equation}\label{cylinder_trace}
%\|{\rm Tr}(\bu)\|_{L^2(-a,a)} \le \frac{1}{2\pi}\abs{\log\epsilon}^{1/2}\|\nabla \bu\|_{L^2(C_{\epsilon,a})}.
%\end{equation}
%\end{lemma}
%
%\begin{proof}
%We show the inequality \eqref{trace_straight_cylinder} for $\bu\in C^1(C_{\epsilon,a})\cap C^0(\overline{C_{\epsilon,a}})\cap \A_S$; the proof for $\bu\in \A_S$ then follows by density. \\
%
%First note that for any $\bu\in C^1(C_{\epsilon,a})\cap C^0(\overline{C_{\epsilon,a}})$ and any $\bx = s\be_t +\epsilon\be_{\rho} + \theta\be_\theta \in \Gamma_{\epsilon,a}$, we may use the fundamental theorem of calculus to write
%\[ \bu(s,\theta,\epsilon) = - \int_{\epsilon}^1 \frac{\p \bu}{\p \rho} \ts d\rho. \]
%Then
%\begin{align*}
%|\bu(s,\theta,\epsilon)| &\le \int_{\epsilon}^{1} \bigg|\frac{\p \bu}{\p\rho} \bigg|\ts d\rho = \int_{\epsilon}^{1} \frac{1}{\sqrt{\rho}}\sqrt{\rho} \bigg|\frac{\p \bu}{\p\rho} \bigg|\ts d\rho \\
%&\le \left(\int_{\epsilon}^{1} \frac{1}{\rho} \ts d\rho\right)^{\frac{1}{2}} \left(\int_{\epsilon}^{1} \bigg|\frac{\p \bu}{\p\rho} \bigg|^2 \ts \rho \ts d\rho\right)^{\frac{1}{2}} \\
%&= \sqrt{|\log\epsilon|}\left(\int_{\epsilon}^{1} \bigg|\frac{\p \bu}{\p\rho} \bigg|^2 \ts \rho \ts d\rho\right)^{\frac{1}{2}}.
%\end{align*}
%
%Therefore ${\rm Tr}(\bu)$ obeys
%\begin{equation}\label{surface_ineq}
%\big|{\rm Tr}(\bu)\big|^2 \le |\log \epsilon| \int_{\epsilon}^1\bigg|\frac{\p \bu}{\p\rho} \bigg|^2 \ts \rho \ts d\rho.
%\end{equation}
%
%This holds for arbitrary $\bu \in C^1(C_{\epsilon,a})\cap C^0(\overline{C_{\epsilon,a}})$, but if $\bu$ also belongs to $\A_S$, by the $\theta$-independence of ${\rm Tr}(\bu)$, we have
%\[\|{\rm Tr}(\bu)\|_{L^2(-a,a)}^2 = \frac{1}{2\pi}\int_{-a}^a \int_0^{2\pi} |{\rm Tr}(\bu)|^2 \ts d\theta \ts ds.\]
%
%Then, using \eqref{surface_ineq}, we have that this $\bu$ satisfies
%\begin{align*}
%\|{\rm Tr}(\bu)\|_{L^2(-a,a)}^2 = \frac{1}{2\pi}\int_{-a}^a\int_0^{2\pi} |{\rm Tr}(\bu)|^2 \ts d\theta ds &\le \frac{1}{2\pi}|\log \epsilon| \int_{-a}^a\int_0^{2\pi}\int_{\epsilon}^1 \bigg| \frac{\p \bu}{\p \rho} \bigg|^2 \ts \rho \ts d\rho \ts d\theta\ts ds \\
%& \le \frac{1}{2\pi}|\log \epsilon| \|\nabla \bu\|_{L^2(C_{\epsilon,a})}^2.
%\end{align*}
%\end{proof}
%
%This estimate holds for $\bu$ defined around a straight cylinder; to return to a curved centerline, the diffeomorphisms $\psi_j^{-1}$ result in an additional constant on each set $W_j$  depending on $\psi_j$ but not $\epsilon$. Summing over the $\phi_j$, we obtain the trace constant for any slender body $\Sigma_{\epsilon}$ satisfying the geometric constraints in Section \ref{geometric_constraints}:
%\begin{equation}\label{trace_const} 
%c_T = c_{\kappa}|\log\epsilon|^{1/2}, 
%\end{equation}
%where $c_{\kappa}$ depends on the shape of the fiber centerline -- in particular, on the constants $\kappa_{\max}$ and $c_\Gamma$ -- but not on $\epsilon$. 
%\end{proof}
%
%%%%%%%
%
%\subsection{Korn inequality}\label{extension}
%The estimate \eqref{weak_stokes_est} for the Stokes problem relies on a Korn inequality \eqref{korn_ineq_UB} bounding $\nabla \bu$ by $\E(\bu)$, the symmetric part of the gradient. We show that the constant in the Korn inequality is bounded independent of $\epsilon$ as $\epsilon\to 0$. 
%\begin{lemma}\emph{(Korn inequality)}\label{korn_eps}
%Let $\Omega_{\epsilon}=\R^3 \backslash \overline{\Sigma_{\epsilon}}$ be as in Section \ref{geometric_constraints}. There exists a constant $c_K$ depending only on $\kappa_{\max}$ and $c_{\Gamma}$ such that for all $\bu\in D^{1,2}(\Omega_{\epsilon})$, the Korn inequality holds:
%\begin{equation}\label{korn_ineq}
% \|\nabla \bu\|_{L^2(\Omega_{\epsilon})} \le c_K\|\mathcal{E}(\bu)\|_{L^2(\Omega_{\epsilon})}. 
% \end{equation}
%\end{lemma}
%
%The proof of Lemma \ref{korn_eps} essentially relies on the existence of a linear operator $T_{\epsilon}$ extending $\bu$ to the interior of the slender body such that $\E(T_{\epsilon}\bu)$ is bounded independent of $\epsilon$ as $\epsilon\to 0$. We can then use the simple proof of the Korn inequality over all of $\R^3$ used in showing \eqref{korn_ineq_UB}. The key is thus to show the following lemma:
%\begin{lemma}\emph{(Extension operator)}\label{extension_eps}
%Let $\Omega_{\epsilon}=\R^3 \backslash \overline{\Sigma_{\epsilon}}$ be as in Section \ref{geometric_constraints}. For $\bu\in D^{1,2}(\Omega_{\epsilon})$, there exists a bounded linear operator $T_{\epsilon}: D^{1,2}(\Omega_{\epsilon})\to D^{1,2}(\R^3)$ extending $\bu$ to the interior of the slender body and satisfying 
%\begin{enumerate}
%\item $T_{\epsilon}\bu|_{\Omega_{\epsilon}} = \bu$ 
%\item $\|\E(T_{\epsilon}\bu) \|_{L^2(\R^3)} \le c_E \| \E(\bu)\|_{L^2(\Omega_{\epsilon})}$, where the constant $c_E$ is independent of the slender body radius $\epsilon$ as $\epsilon\to 0$.
%\end{enumerate}
%\end{lemma}
%
%Note that property 2 implies $\|T_{\epsilon}\bu\|_{D^{1,2}(\R^3)} \le \sqrt{2} c_E \|\bu\|_{D^{1,2}(\Omega_{\epsilon})}$, since
%\begin{align*}
%\|T_{\epsilon}\bu\|_{D^{1,2}(\R^3)} &= \|\nabla(T_{\epsilon}\bu)\|_{L^2(\R^3)} \le \sqrt{2} \|\E(T_{\epsilon}\bu) \|_{L^2(\R^3)} \le \sqrt{2} c_E \| \E(\bu)\|_{L^2(\Omega_{\epsilon})} \\
%& \le 2\sqrt{2} c_E \|\nabla \bu \|_{L^2(\Omega_{\epsilon})} = 2\sqrt{2} c_E \| \bu \|_{D^{1,2}(\Omega_{\epsilon})} .
%\end{align*}
%
%In order to prove Lemma \ref{extension_eps}, we will need to show three additional lemmas. The first is an analogue of Lemma 3.1.2(1) in \cite{mazya1997differentiable}, adapted to use the symmetric gradient rather than the full gradient. 
%\begin{lemma}\emph{(Extension-by-reflection scaling)}\label{extension_ineq}
%Let $\D$, $\D_2$ be bounded $C^2$ domains in $\R^3$ with $\overline \D_2\subset \D$, and let $\D_H=\D\backslash \overline \D_2$ be a bounded $C^2$ domain with a hole. For the rescaled domains $\D_{H,\epsilon}= \epsilon \D_H$, $\D_{\epsilon}=\epsilon\D$ ($\epsilon\in \R_+$), there exists a linear extension operator $T: H^1(\D_{H,\epsilon}) \to H^1(\D_{\epsilon})$ satisfying 
%\begin{equation}\label{extension_symm}
%\| T \bu \|_{L^2(\D_{\epsilon})} \le c\|\bu\|_{L^2(\D_{H,\epsilon})}
%\end{equation}
%as well as the estimate
%\begin{equation}\label{extension_symm}
%\| \E(T \bu) \|_{L^2(\D_{\epsilon})} \le c\bigg( \epsilon^{-1}\|\bu\|_{L^2(\D_{H,\epsilon})} + \|\E(\bu)\|_{L^2(\D_{H,\epsilon})} \bigg).
%\end{equation}
%\end{lemma}
%
%%%%%%%%%
%\begin{proof}
%For a function $\bv$ defined in the upper half-space $\R^3_+$, we recall the standard extension-by-reflection $E:\R^3_+\to \R^3$ across the boundary $x_3=0$ (see \cite{mazya1997differentiable} or \cite{evans2010pde}):
%\[ E\bv(\bx) = \begin{cases}
%\bv(\bx), & \bv\in \R^3_+ \\
%\bv(x_1,x_2,-x_3) & \bv \not\in \R^3_+.
%\end{cases} \]
%For the domain-with-hole $\D_H\subset \R^3$, we cover a neighborhood of the inner boundary $\p\D_2$ with finitely many balls $B^H_i$ centered at points on $\p\D_2$, choosing the cover such that $\D_H\cap B^H_i$ can be mapped via $C^2$ diffeomorphism, denoted by $\Phi^{-1}$, to the half-ball $B_i\cap\R^3_+$, where $B_i$ is a ball in $\R^3$. We then choose open sets $U_j\subset\D_H$ such that $\{ B^H_i\} \cup \{U_i\}$ cover $\D_H$.  We define a partition of unity $\{\varphi_i\}\cup \{\varphi_j\}$ subordinate to this cover, and define the usual extension operator $T:\D_H\to \D$ by
%\[T\bu = \sum_i E(\varphi_i \bu\circ \Phi_i)\circ \Phi_i^{-1} + \sum_j \varphi_j \bu.\]
%
%From this extension operator $T$, we can directly estimate $\|\E(T\bu)\|_{L^2(\D)}$: 
%\begin{align*}
%\|\E(T\bu)\|_{L^2(\D)} &\le c\sum_i\bigg\|\nabla\Phi^{-1}_i\big(\varphi_i \nabla \bu \nabla \Phi_i + \nabla\varphi_i \bu\big)+ \nabla\Phi^{-{\rm T}}_i\big(\varphi_i \nabla \bu \nabla \Phi_i + \nabla\varphi_i \bu\big)^{\rm T} \bigg\|_{L^2(\D_H)} \\
%&\hspace{2cm}+\sum_j \|\varphi_j\E(\bu) \|_{L^2(\D_H)} + \sum_j \|\nabla \varphi_j \bu^{\rm T} \|_{L^2(\D_H)}\\
%&\le c_{\Phi} \|\nabla\bu+ \nabla\bu^{\rm T} \|_{L^2(\D_H)} + c_{\Phi,\phi}\|\bu\|_{L^2(\D_H)} + \|\E(\bu)\|_{L^2(\D_H)} + c_{\phi}\|\bu\|_{L^2(\D_H)} \\
%&\le c\big( \|\bu\|_{L^2(\D_H)} + \|\E(\bu)\|_{L^2(\D_H)} \big)
%\end{align*}
%
%The above inequality, coupled with a scaling argument ($\bx\to \epsilon\bx$) results in the desired $\epsilon$-dependent inequality \eqref{extension_symm}.
% \end{proof}
%
%Again, let $\D$ be a bounded, $C^2$ domain. On $\D$, we define the space of rigid motions
%\[ \mathcal{R} = \{ \bv\in H^1(\D) \ts : \ts \bv = \bm{A}\bx + \bm{b} \text{ for some } \bm{A}= -\bm{A}^{\rm T} \in \R^{3\times 3} \text{ and } \bm{b}\in \R^3\}.\]
%For $\bu\in H^1(\D)$, let $P_{\mathcal{R}}\bu$ be the $L^2$ projection of $\bu$ onto the space of rigid motions, i.e. 
%\[ P_{\mathcal{R}}\bu = \bv\in \mathcal{R} \text{ such that } \|\bu-\bv\|_{L^2(\D)} \le \|\bu - \bw\|_{L^2(\D)}\quad  \forall \bw\in \mathcal{R}. \]
%
%\begin{lemma}\emph{(Korn's inequality for pure strain)}\label{korn_nonrigid}
%Let $\D$ be a bounded Lipschitz domain and let $\sR$ be the space of rigid motions on $\D$. For any $\bw\in H^1(\D)$ with $\bw\perp \sR$, Korn's inequality holds:
%\[ \|\nabla\bw\|_{L^2(\D)} \le c\|\E(\bw)\|_{L^2(\D)}.\] 
%\end{lemma}
%
%\begin{proof}
%The proof of Lemma \ref{korn_nonrigid} relies on the following Korn-type inequality for the bounded domain $\D$: 
%\begin{equation}\label{korn_bdd}
%\|\bu\|_{H^1(\D)} \le c (\|\mathcal{E}(\bu)\|_{L^2(\D)}+\|\bu\|_{L^2(\D)}).
%\end{equation}
%Since the domain dependence of the constant $c$ does not need to be specified in Lemma \ref{korn_nonrigid}, we refer to \cite{duvaut1976inequalities} for a proof of \eqref{korn_bdd}. \\
%
%Now, assume Lemma \ref{korn_nonrigid} does not hold. Then there exists a sequence of functions $\{\bw_k\}\subset H^1(\D)$, $k=1,2,3,\dots$, such that $\bw_k\perp \sR$ and 
%\[ \|\nabla\bw_k\|_{L^2(\D)} > k \|\E(\bw_k)\|_{L^2(\D)}.\] 
%Without loss of generality, $\|\bw_k\|_{L^2(\D)}=1$, so by \eqref{korn_bdd},
%\[\|\E(\bw_k)\|_{L^2(\D)} < \frac{1}{k}\|\nabla\bw_k\|_{L^2(\D)} \le \frac{1}{k}\|\bw_k\|_{H^1(\D)} \le \frac{c}{k}(\|\E(\bw_k)\|_{L^2(\D)}+1).\]
%Taking $k$ sufficiently large (in particular, $k>c$), we have 
%\[ \bigg(1-\frac{c}{k}\bigg)\|\E(\bw_k)\|_{L^2(\D)} < \frac{c}{k},\]
%and thus $\|\E(\bw_k)\|_{L^2(\D)}\to 0$ as $k\to\infty$. Again by the inequality \eqref{korn_bdd}, 
%\[ \|\bw_k\|_{H^1(\D)} \le c\bigg(\frac{c}{k-c}+1\bigg), \]
%so there exists a subsequence $\{\bw_{k_j}\}$ such that $\bw_{k_j}\rightharpoonup \bw$ in $H^1$ for some $\bw\in H^1(\D)$. By Rellich compactness, $\bw_{k_j} \to \bw$ in $L^2$. Furthermore, $\liminf_k \norm{\E(\bw_{k_j})}_{L^2(\D)}\ge \norm{\E(\bw)}_{L^2(\D)}$, so $\E(\bw)=0$. Thus $\bw\in \sR$, but $\bw_k\perp\sR$ for all $k$, and $\bw_{k_j} \to \bw$ in $L^2$, so $\bw\equiv 0$. Thus $\bw_{k_j} \to 0$ in $L^2$, which contradicts $\|\bw_k\|_{L^2(\D)}=1$ for all $k$.
%\end{proof}
%
%\begin{lemma}\emph{(Korn-Poincar\'e inequality)}\label{korn_poincare}
%Let $\D$ be a bounded, Lipschitz domain in $\R^3$. For any $\bu\in H^1(\D)$, we have 
%\begin{equation}\label{KP_ineq}
%\| \bu - P_{\mathcal{R}}\bu\|_{L^2(\D)} \le c\| \E(\bu)\|_{L^2(\D)}
%\end{equation}
%for some constant $c>0$.
%\end{lemma}
%
%\begin{proof}
%Assume that inequality \eqref{KP_ineq} does not hold. Then for each $k=1,2,3,\dots$ there exists a sequence $\{\bu_k\}\subset H^1(\D)$ such that
%\[ \|\bu_k - P_{\sR}\bu_k \|_{L^2(\D)} > k\|\E(\bu_k)\|_{L^2(\D)}. \]
%Define $\bw_k=\bu_k - P_{\sR}\bu_k$, so $\bw_k \perp \sR$ for each $k=1,2,3,\dots$ and $\E(\bw_k)=\E(\bu_k)$. Without loss of generality $\|\bw_k\|_{L^2(\D)} = 1$. Then
%\[ 1= \|\bw_k \|_{L^2(\D)} > k\|\E(\bw_k)\|_{L^2(\D)},\] 
%so $\|\E(\bw_k)\|_{L^2(\D)} <\frac{1}{k} \to 0$ as $k\to \infty$. Furthermore, since $\bw_k \perp \sR$ for each $k$, by Korn's inequality for pure strain (Lemma \ref{korn_nonrigid}) we have $\|\nabla \bw_k\|_{L^2(\D)} < \frac{c}{k}$. Thus $\bw_k$ is uniformly bounded in $H^1$ and there exists a subsequence $\{\bw_{k_l}\}$ such that $\bw_{k_l}\rightharpoonup \bw$ in $H^1$ for some $\bw\in H^1(\D)$. By compactness, $\bw_{k_l}\to \bw$ in $L^2$. Then, since $\liminf_k \|\E(\bw_k)\|_{L^2(\D)} \ge \|\E(\bw)\|_{L^2(\D)}$, we have that the limit $\bw$ satisfies $\E(\bw)=0$, so $\bw\in \sR$. But $\bw_{k_l}\to \bw$ in $L^2$ and $\bw_{k} \perp \sR$ for each $k$, so we must have $\bw\perp \sR$ as well. Thus $\bw\equiv 0$, so $\bw_{k_l} \to 0$ in $L^2$, which contradicts $\|\bw_{k_l} \|_{L^2(\D)}=1$. 
%\end{proof}
%
%With Lemmas \ref{extension_ineq} and \ref{korn_poincare}, we are equipped to prove Lemma \ref{extension_eps}.
%
%\begin{proof}[Proof of Lemma \ref{extension_eps}]
%Let $D_{r}$ denote the disk in $\R^2$ of radius $r$. Using the diffeomorphisms $\psi_j$ defined in Section \ref{trace_sec}, it suffices to consider $\bu\in D^{1,2}(\R^2\backslash D_{\epsilon}\times \R)$ with supp$(\bu)\subset \R^2\backslash D_{\epsilon}\times [-a,a]$ for $a<\infty$ and show that there exists an extension operator into the interior of the infinite cylinder $D_{\epsilon}\times\R \subset\R^3$ with symmetric gradient that is bounded independent of $\epsilon$ as $\epsilon\to 0$. \\
%
%First we define 
%\[ S_{\epsilon} = D_{2\epsilon}\times \R \quad \text{and} \quad G_{\epsilon} = (D_{2\epsilon}\backslash \overline{D_{\epsilon}}) \times \R \subset \R^3. \]
%Since $\bu\in D^{1,2}(\R^2\backslash D_{\epsilon}\times \R)$ with supp$(\bu)\subset \R^2\backslash D_{\epsilon}\times [-a,a]$, we have $\bu\in H^1(G_{\epsilon})$. We show that we can in fact construct a linear extension operator extending $\bu\in H^1(G_{\epsilon})$ to $H^1(S_{\epsilon})$ whose symmetric gradient is bounded independent of $\epsilon$. \\
%
%Following \cite{mazya1997differentiable}, we begin by defining a cover $\{Q_j\}$ of $\R$:
%\[ Q_j = \{s \in \R \ts:\ts |s-j| <1 \}, \quad j\in \Z. \] 
%Let $\{\eta_j\}$ denote a smooth partition of unity subordinate to $Q_j$, where $\eta_j$ can be written as $\eta_j= \phi(s-j)$, translates of the same smooth cutoff function, such that $|\nabla \eta_j|\le c$ for each $j$. We define a sequence of cylinders and cylindrical layers 
%\[ S_{2}^{(j)} = D_{2}\times Q_j \quad \text{and} \quad G^{(j)}_{2} = (D_{2}\backslash \overline{D_1}) \times Q_j \subset \R^3. \]
%and set $S^{(j)}_{\epsilon} = \epsilon S^{(j)}_{2}$ and $G^{(j)}_{\epsilon} = \epsilon G^{(j)}_{2}$.  Then by Lemma \ref{extension_ineq}, there exists a linear extension operator $T_{\epsilon}^{(j)}: H^1(G^{(j)}_{\epsilon}) \to H^1(S^{(j)}_{\epsilon})$ satisfying
% \begin{equation}\label{ext_est_seq1}
% \|\E (T_{\epsilon}^{(j)}\bu) \|_{L^2(S^{(j)}_{\epsilon})} \le c\left(\epsilon^{-1}\|\bu\|_{L^2(G^{(j)}_{\epsilon})} + \|\E(\bu)\|_{L^2(G^{(j)}_{\epsilon})} \right)
% \end{equation}
%and
% \begin{equation}\label{ext_est_seq2}
%  \| T_{\epsilon}^{(j)}\bu \|_{L^2(S^{(j)}_{\epsilon})} \le c\|\bu\|_{L^2(G^{(j)}_{\epsilon})}.
%  \end{equation}
% 
%Let $P_{\sR}^{(j)}\bu$ denote the projection of $\bu\big|_{G^{(j)}_{\epsilon}}\in H^1(G^{(j)}_{\epsilon})$ onto $\sR$, the space of rigid motions on each $G^{(j)}_{\epsilon}$. Then, since $\E(\bw)= 0$ for any $\bw\in \sR$, we have 
% \[ \|\E(\bu - P_{\sR}^{(j)}\bu)\|_{L^2(G^{(j)}_{\epsilon})} =\|\E(\bu)\|_{L^2(G^{(j)}_{\epsilon})}. \]
%By the Korn-Poincar\'e inequality (Lemma \ref{korn_poincare}) and a scaling argument we also have 
%\begin{equation}\label{poincare_est}
%  \|\bu - P_{\sR}^{(j)}\bu\|_{L^2(G^{(j)}_{\epsilon})} \le c\epsilon\|\E(\bu)\|_{L^2(G^{(j)}_{\epsilon})}.
%\end{equation}
%
%Since $P^{(j)}_{\sR}\bu\in \sR$ on each cylindrical shell $G_{\epsilon}^{(j)}$, we can write $P^{(j)}_{\sR}\bu= \bm{A}_j\bx+\bm{b}_j$ for $\bx\in G_{\epsilon}^{(j)}$. We then define the extension to each of the cylinders $S_{\epsilon}^{(j)}$ by 
%\begin{equation}\label{PR_on_S}
%\overline P_{\sR}^{(j)}\bu =\bm{A}_j\bx+ \bm{b}_j, \qquad \bx\in S_{\epsilon}^{(j)}.
%\end{equation}
%  
%With these tools in mind, we now define an extension operator from the cylindrical shell $G_{\epsilon}$ to the cylinder $S_{\epsilon}$. We take 
% \begin{equation}\label{ext_operator}
%  T_{\epsilon}\bu(\bx) = \bv(\bx)+ \bw(\bx) 
%  \end{equation}
% where, for $\bx=\bx(\rho,\theta,s)\in S_{\epsilon}$ and $\bu_j = \bu|_{G^{(j)}_{\epsilon}}$, 
% \begin{align*}
% \bv(\rho,\theta,s) &= \sum_{j\in \Z} \eta_j(s/\epsilon)\bigg(\overline P_{\sR}^{(j)}\bu\bigg)(\bx) \\
% \bw(\rho,\theta,s) &= \sum_{j\in \Z} \eta_j(s/\epsilon)\left(T_{\epsilon}^{(j)}\big(\bu_j -P_{\sR}^{(j)}\bu\big)\right)(\bx).
% \end{align*}
% 
%Note that $T_{\epsilon}\bu \big|_{G_{\epsilon}}=\bu$. Furthermore, we show
%\begin{equation}\label{ext_est_1}
% \|\E(T_{\epsilon}\bu)\|_{L^2(S_{\epsilon})} \le c\| \E(\bu)\|_{L^2(G_{\epsilon})} 
% \end{equation}
%where the constant $c$ does not depend on $\epsilon$ as $\epsilon \to 0$. \\
%
%We begin by estimating $\bv$. Let
%\[\tilde Q_j = \{s\in\R \ts:\ts 0 <s-j<1\}, \quad j\in \Z.\]
%Note that for each $j$ we have $\tilde Q_j\subset Q_j$ and $\tilde Q_j\subset Q_{j+1}$; in particular, $\eta_j(s)+\eta_{j+1}(s)=1$ on $\tilde Q_j$. Define
%\[ \tilde S_{\epsilon}^{(j)} = \epsilon\left(D_{2}\times \tilde Q_j \right) \quad \text{and}\quad\tilde G_{\epsilon}^{(j)} = \epsilon\left((D_{2}\backslash \overline{D_1})\times \tilde Q_j \right).\]
%
%On each $\tilde S_{\epsilon}^{(j)}$, $\bv$ can be rewritten as
%\[ \bv(\rho,\theta,s) = \overline P_{\sR}^{(j)}\bu +\eta_{j+1}(s/\epsilon) (\overline P_{\sR}^{(j+1)}\bu - \overline P_{\sR}^{(j)}\bu). \]
%
%Then, by the definition \eqref{PR_on_S}, we can bound the norm of $\overline P_{\sR}^{(j)}\bu$ on each cylinder $\tilde S_{\epsilon}^{(j)}$ by its norm over the shell $\tilde G_{\epsilon}^{(j)}$: $\|\overline P_{\sR}^{(j)}\bu\|_{L^2(\tilde S_{\epsilon}^{(j)})} \le c\|P_{\sR}^{(j)}\bu\|_{L^2(\tilde G_{\epsilon}^{(j)})}$. Using this, we bound the symmetric gradient of $\bv$: 
%\begin{align*}
%\|\E(\bv)\|_{L^2(\tilde S_{\epsilon}^{(j)})} &= \|\nabla \eta_{j+1}(s/\epsilon)(\overline P_{\sR}^{(j+1)}\bu - \overline P_{\sR}^{(j)}\bu)^{\rm T} +  (\overline P_{\sR}^{(j+1)}\bu - \overline P_{\sR}^{(j)}\bu)\nabla \eta_{j+1}(s/\epsilon)^{\rm T} \|_{L^2(\tilde S_{\epsilon}^{(j)})} \\
%&\le c\epsilon^{-1}\|P_{\sR}^{(j+1)}\bu - P_{\sR}^{(j)}\bu \|_{L^2(\tilde G_{\epsilon}^{(j)})} \\
%&\le c\epsilon^{-1}\left(\|\bu-P_{\sR}^{(j+1)}\bu\|_{L^2(G_{\epsilon}^{(j+1)})}+\|\bu-P_{\sR}^{(j)}\bu\|_{L^2(G_{\epsilon}^{(j)})} \right),
%\end{align*}
% where in the last step we have used that $\tilde G_{\epsilon}^{(j)} \subset G_{\epsilon}^{(j+1)}$ and $\tilde G_{\epsilon}^{(j)} \subset G_{\epsilon}^{(j)}$. Finally, using \eqref{poincare_est}, we have
%\[\|\E( \bv)\|_{L^2(\tilde S_{\epsilon}^{(j)})} \le c\left(\|\E(\bu)\|_{L^2(G_{\epsilon}^{(j)})}+\|\E(\bu)\|_{L^2(G_{\epsilon}^{(j+1)})}\right).\]
%
%Summing over $j$, we then have
%\[ \|\E(\bv)\|_{L^2(S_{\epsilon})} \le c \|\E(\bu)\|_{L^2(G_{\epsilon})} \]
%where $c$ is bounded independent of $\epsilon$ as $\epsilon\to 0$. \\
%
%We now bound the symmetric gradient of $\bw$. On each $\tilde S_{\epsilon}^{(j)}$ we have
%\begin{align*}
% \|\E(\bw)\|_{L^2(\tilde S_{\epsilon}^{(j)})} &\le  \|\E\big(T_{\epsilon}^{(j)}(\bu_j -P_{\sR}^{(j)}\bu)\big) \|_{L^2(\tilde S_{\epsilon}^{(j)})}+  \|\E\big(T_{\epsilon}^{(j+1)}(\bu_{j+1} -P_{\sR}^{(j+1)}\bu)\big) \|_{L^2(\tilde S_{\epsilon}^{(j)})} \\
% &\quad + 2c\epsilon^{-1} \|T_{\epsilon}^{(j)}(\bu_j-P_{\sR}^{(j)}\bu)\|_{L^2(\tilde S_{\epsilon}^{(j)})} + 2c\epsilon^{-1} \|T_{\epsilon}^{(j+1)}(\bu_{j+1}-P_{\sR}^{(j+1)}\bu)\|_{L^2(\tilde S_{\epsilon}^{(j)})}. 
% \end{align*}
%
%Using the inequalities \eqref{ext_est_seq1}, \eqref{ext_est_seq2}, and \eqref{poincare_est}, we have
%\begin{align*}
%\|T_{\epsilon}^{(j)}(\bu_j -P_{\sR}^{(j)}\bu)\|_{L^2(\tilde S_{\epsilon}^{(j)})} &\le c \|\bu_j -P_{\sR}^{(j)}\bu\|_{L^2(\tilde G_{\epsilon}^{(j)})} \\
%&\le c\epsilon\|\E(\bu)\|_{L^2(\tilde G_{\epsilon}^{(j)})}
%\end{align*}
%and
%\begin{align*}
%\|\E\big(T_{\epsilon}^{(j)}(\bu_j-P_{\sR}^{(j)}\bu)\big)\|_{L^2(\tilde S_{\epsilon}^{(j)})} &\le c\left(\epsilon^{-1}\|\bu_j-P_{\sR}^{(j)}\bu\|_{L^2(\tilde G_{\epsilon}^{(j)})}+\|\E\big(\bu_j -P_{\sR}^{(j)}\bu)\|_{L^2(\tilde G_{\epsilon}^{(j)}\big)} \right) \\
%&\le c\| \E(\bu) \|_{L^2(\tilde G_{\epsilon}^{(j)})} .
%\end{align*}
%
%Summing over $j$, we have
%\[ \|\E(\bw)\|_{L^2(S_{\epsilon})} \le c\| \E(\bu) \|_{L^2(G_{\epsilon})}. \]
%
%Therefore the extension operator $T_{\epsilon}: G_{\epsilon}\to S_{\epsilon}$ \eqref{ext_operator} is bounded independent of $\epsilon$ as $\epsilon\to 0$. Defining $T_{\epsilon}\bu=\bu$ in $\R^3\backslash S_{\epsilon}$ then gives the desired extension on all of $\R^3$. 
%\end{proof}
%
%\begin{proof}[Proof of Lemma \ref{korn_eps}]
%Using the extension operator $T_{\epsilon}$ established in Lemma \ref{extension_eps} to extend $\bu\in D^{1,2}(\Omega_{\epsilon})$ to all of $\R^3$, the proof of Lemma \ref{korn_eps} is immediate. We refer to the proof of the Korn inequality over $\R^3$ (see Lemma \ref{korn_ineq_UB}) to show that the Korn inequality \eqref{korn_eps} holds independent of the slender body radius $\epsilon$ as $\epsilon \to 0$. 
%\end{proof}
%
%%%%%%%%%%%%%%%%%%%%%%%%%%%%%%%
%%%%%%%%%%%%%%%%%%%%%%%%%%%%%%%
%%%%%%%%%%%%%%%%%%%%%%%%%%%%%%%
%
%\subsection{Sobolev inequality}\label{Sob_ineq}
%Using the extension operator defined in the previous section, we show that the Sobolev inequality holds with bounds independent of $\epsilon$ as $\epsilon\to 0$. We prove the following lemma:
% 
%\begin{lemma}\emph{(Sobolev inequality)}\label{sobo_ineq}
%Let $\Omega_{\epsilon}=\R^3\backslash\overline{\Sigma_{\epsilon}}$, the exterior of a slender body of radius $\epsilon$. For any $\bu\in D^{1,2}(\Omega_{\epsilon})$, we have
%\begin{equation}\label{sobolev_const}
%\| \bu\|_{L^6(\Omega_{\epsilon})} \le c_S\|\nabla\bu\|_{L^2(\Omega_{\epsilon})}
%\end{equation}
%with a constant $c_S$ that is bounded independent of $\epsilon$ as $\epsilon\to 0$. 
%\end{lemma}
%
%\begin{proof}
%We have
%\begin{align*}
%\| \bu\|_{L^6(\Omega_{\epsilon})} &\le \| T_{\epsilon}\bu\|_{L^6(\R^3)} \le  c_R\| \nabla (T_{\epsilon}\bu)\|_{L^2(\R^3)} \\
%&\le  c_R c_{E}\| \nabla \bu\|_{L^2(\Omega_{\epsilon})}, \quad \text{by Lemma \ref{extension_eps},}
%\end{align*}
%where $c_R$ is the constant in the Sobolev inequality on $\R^3$. Taking $c_S=c_Rc_E$, we obtain the desired result. 
%\end{proof}
%
%
%%%%%%%%%%%%%%%%%%%%%%%%%%%%%%%
%%%%%%%%%%%%%%%%%%%%%%%%%%%%%%%
%%%%%%%%%%%%%%%%%%%%%%%%%%%%%%%
%
%\subsection{Pressure estimate}\label{pressure_const}
%Finally, to complete the proof of the $\epsilon$-dependence in the estimate \eqref{stokes_est} of Theorem \ref{stokes_theorem}, we verify the $\epsilon$-independence of the constant $c_P$ in the pressure inequality \eqref{press_est}, which we recall here:
%\[ \|p\|_{L^2(\Omega_{\epsilon})} \le c_P\|\E(\bu)\|_{L^2(\Omega_{\epsilon})}.\]
%
%Following \cite{galdi2011introduction}, Chapter III.3, we show the following lemma. 
%\begin{lemma}{\emph{(Solution to $\dive\ts\bv=p$)}}\label{divv_p_lem} 
%Let $\Omega_{\epsilon}=\R^3\backslash \overline{\Sigma_{\epsilon}}$, the exterior of a slender body of radius $\epsilon$. There exists a function $\bv\in D^{1,2}_0(\Omega_{\epsilon})$ satisfying 
%\begin{align*}
%\dive\ts \bv &= p \quad \text{ in }\Omega_{\epsilon}; \\
%\|\bv\|_{D^{1,2}(\Omega_{\epsilon})} &\le c_P\| p\|_{L^2(\Omega_{\epsilon})}, 
%\end{align*}
%where the constant $c_P$ depends on $\kappa_{\max}$ and $c_{\Gamma}$ but not on $\epsilon$.
%\end{lemma}
%
%\begin{proof}[Proof of Lemma \ref{divv_p_lem}]
%We begin by taking a sequence $\{p_m\}\subset C_0^{\infty}(\Omega_{\epsilon})$ approximating $p$ in $L^2(\Omega_{\epsilon})$. For each $m\in \N$, let $\psi_m$ be the solution to the Poisson problem $\Delta\psi_m = \overline{p_m}$ in $\R^3$, where $\overline{p_m}$ denotes the extension by zero of $p_m$ to the interior of $\Sigma_{\epsilon}$; i.e. to all of $\R^3$. Then by standard solution theory for the Poisson problem (\cite{galdi2011introduction}, Chapter II.11), we have the estimate
%\begin{equation}\label{poisson_est}
% \|\nabla^2\psi_m\|_{L^2(\Omega_{\epsilon})} \le \|\nabla^2\psi_m\|_{L^2(\R^3)} \le c_q\|\overline{p_m}\|_{L^2(\R^3)} = c_q\|p_m\|_{L^2(\Omega_{\epsilon})} 
%\end{equation}
%where $\nabla^2$ denotes the matrix of second partial derivatives and the constant $c_q$ is independent of $\epsilon$. \\
%
%We define
%\[ \bv_m :=\nabla \psi_m+\bw_m \]
%where $\bw_m\in D^{1,2}(\Omega_{\epsilon})$ is supported only within the neighborhood $\mathcal{O}$ \eqref{region_O} of $\Gamma_{\epsilon}$, and serves to correct for $\nabla \psi_m\neq 0$ on $\Gamma_{\epsilon}$. To this end, $\bw_m$ can be considered as a function in $H^1(\mathcal{O})$ satisfying
%\begin{equation}\label{w_equation}
%\begin{aligned}
%\dive\ts\bw_m &= 0 \quad \text{in }\mathcal{O} \\
%\bw_m &= - \nabla \psi_m \quad \text{on }\Gamma_{\epsilon} \\
%\bw_m &=0 \quad \text{on } \p \mathcal{O} \backslash\Gamma_{\epsilon}, 
%\end{aligned}
%\end{equation}
%which is then extended by zero to all of $\Omega_{\epsilon}$. For each $m\in \N$, such a function $\bw_m$ exists since $\Delta \psi_m=0$ within $\Sigma_{\epsilon}$ and therefore
%\[ \int_{\Gamma_{\epsilon}} \nabla \psi_m\cdot{\bm n}=0. \]
%A solution to \eqref{w_equation} can be constructed by considering the function ${\bm \Psi}_m = - \phi\nabla \psi_m$ where $\phi\in C^{\infty}(\Omega_{\epsilon})$ is a cutoff function satisfying 
%\[ \phi(\rho)=\begin{cases}
%1, & \rho \le r_{\max}/2 \\
%0 & \rho > r_{\max}.
%\end{cases} \] 
%Then by \cite{galdi2011introduction}, Theorem III.3.1, there exists a solution $\bw_m-{\bm \Psi}_m\in H^1_0(\mathcal{O})$ satisfying
%\begin{equation}\label{new_w_equation}
%\begin{aligned}
%\dive(\bw_m-{\bm \Psi}_m) &= -\dive \ts {\bm \Psi}_m \quad \text{in }\mathcal{O}; \\
%\|\nabla(\bw_m-{\bm \Psi}_m)\|_{L^2(\mathcal{O})} &\le c_B\|\dive \ts {\bm \Psi}_m\|_{L^2(\mathcal{O})}.
%\end{aligned}
%\end{equation}
%Since the slender body surface $\Gamma_{\epsilon}$ satisfies the geometric constraints in Section \ref{geometric_constraints}, the region $\mathcal{O}$ satisfies an interior sphere condition with uniform radius $r_{\max}/2$. Then $\mathcal{O}$ can be considered as the infinite union of balls of radius $r_{\max}/2$. Following the construction in the proof of Lemma 2, Chapter 1.1.9 of \cite{maz2013sobolev}, there exist a finite number of domains $\mathcal{O}_k$, star-shaped with respect to balls of radius $r_{\max}/4$, such that
%\[\mathcal{O} = \bigcup_{k=1}^N \mathcal{O}_k. \]
%Here $N$ depends only on $\kappa_{\max}$ and $c_\Gamma$. Then the domain dependence of the constant $c_B$ in \eqref{new_w_equation} has an explicit formula (\cite{galdi2011introduction}, equation III.3.27): 
%\[ c_B \le c_0 \bigg(\frac{\delta(\mathcal{O})}{r_{\max}} \bigg)^3\bigg(1+ \frac{\delta(\mathcal{O})}{r_{\max}} \bigg) \]
%where $\delta(\mathcal{O})$ is the diameter of the region $\mathcal{O}$ and $c_0$ depends on the diameter of the domains $\mathcal{O}_k$, each of which are bounded independent of $\epsilon$ as $\epsilon\to 0$. \\
%
%Then, from \eqref{new_w_equation}, we have
%\begin{equation}\label{w_est1}
%\begin{aligned}
%\|\nabla\bw_m\|_{L^2(\Omega_{\epsilon})} &\le c_B\|\dive\ts {\bm \Psi}_m\|_{L^2(\Omega_{\epsilon})} + \|\nabla{\bm \Psi}_m\|_{L^2(\Omega_{\epsilon})} \\
%&= c_B\|\dive(\phi\nabla \psi_m) \|_{L^2(\Omega_{\epsilon})} + \|\nabla(\phi\nabla \psi_m)\|_{L^2(\Omega_{\epsilon})} .
%\end{aligned}
%\end{equation}
%
%Therefore, using \eqref{poisson_est} and \eqref{w_est1}, we have
%\begin{align*}
%\|\nabla \bw_m\|_{L^2(\Omega_{\epsilon})} &\le (c_B+1)(c_q\|p_m\|_{L^2(\Omega_{\epsilon})}+ c_{\phi}\|\nabla \psi_m\|_{L^2(\mathcal{O})}), 
%\end{align*}
%where $c_{\phi}$ depends on $\nabla\phi$ but is independent of $\epsilon$. We then use the Sobolev inequality on $\R^3$ to obtain 
%\begin{align*}
% \|\nabla \psi_m\|_{L^2(\mathcal{O})} &\le  |\mathcal{O}|^{1/3} \|\nabla\psi_m\|_{L^6(\mathcal{O})} \\
% &\le |\mathcal{O}|^{1/3} \|\nabla\psi_m\|_{L^6(\Omega_{\epsilon})} \\
% &\le |\mathcal{O}|^{1/3} c_S\|\nabla^2\psi_m\|_{L^2(\Omega_{\epsilon})}  \\
% &\le |\mathcal{O}|^{1/3} c_S c_q \|p_m\|_{L^2(\Omega_{\epsilon})}, \quad \text{using }\eqref{poisson_est}.
% \end{align*}
%Now, $|\mathcal{O}|\le c_{\kappa}r_{\max}^2$ is bounded independent of $\epsilon$, and the Sobolev constant $c_S$ is independent of $\epsilon$ (see Section \ref{Sob_ineq}). Thus
%\[ \|\nabla \bw_m\|_{L^2(\Omega_{\epsilon})} \le c_W\|p_m\|_{L^2(\Omega_{\epsilon})} \]
%for $c_W$ independent of $\epsilon$, and 
%\[ \|\nabla \bv_m\|_{L^2(\Omega_{\epsilon})} \le \|\nabla^2\psi_m\|_{L^2(\Omega_{\epsilon})}+ \|\nabla \bw_m\|_{L^2(\Omega_{\epsilon})} \le (c_q+c_W)\|p_m\|_{L^2(\Omega_{\epsilon})}. \]
%
% Passing to the limit we obtain the desired solution to the $\dive \ts\bv=p$ problem of Lemma \eqref{divv_p_lem}, as the constant $c_P=c_q+c_W$ is independent of $\epsilon$.
% \end{proof}
%
%Therefore we arrive at the final form of the estimate \eqref{stokes_est} in Theorem \ref{stokes_theorem}. Plugging the newly-verified $\epsilon$-dependence of the constants $c_K$, $c_T$, and $c_P$ into \eqref{weak_stokes_est}, we have 
%\begin{align*}
%\|\bu\|_{D^{1,2}(\Omega_{\epsilon})} + \|p\|_{L^2(\Omega_{\epsilon})} &\le \frac{1}{2}c_K^2c_T(1+4c_P) \|{\bm f}\|_{L^2(\T)} \\
%&\le c_{\kappa}|\log\epsilon|^{1/2}\|{\bm f}\|_{L^2(\T)}.
%\end{align*}
%
%%The final form of the higher regularity estimate \eqref{stokes_est_epsilon} also follows from tracking these same constants. Again using the form of the constants $c_K$, $c_T$, and $c_P$ in \eqref{regular_est_stokes}, we have 
%\begin{align*}
%\|\nabla^2\bu\|_{L^2(\Omega_{\epsilon})} + \|\nabla p\|_{L^2(\Omega_{\epsilon})} &\le c_{\kappa} \frac{|\log\epsilon|^{1/2}}{\epsilon} \|{\bm f}\|_{H^{1/2}(\T)}.
%\end{align*}
%
%We thus complete the proof of the estimates in Theorem \ref{stokes_theorem}. \\

 %%%%%%%%%%%%%%%%%%%%%%%%%%%%%%%%%%%%%%%%%%%%%%%%%%%%%%%%%%
%%%%%%%%%%%%%%%%%%%%%%%%%%%%%%%%%%%%%%%%%%%%%%%%%%%%%%%%%%
%%%%%%%%%%%%%%%%%%%%%%%%%%%%%%%%%%%%%%%%%%%%%%%%%%%%%%%%%%
%%%%%%%%%%%%%%%%%%%%%%%%%%%%%%%%%%%%%%%%%%%%%%%%%%%%%%%%%%
%%%%%%%%%%%%%%%%%%%%%%%%%%%%%%%%%%%%%%%%%%%%%%%%%%%%%%%%%%
%%%%%%%%%%%%%%%%%%%%%%%%%%%%%%%%%%%%%%%%%%%%%%%%%%%%%%%%%%
%%%%%%%%%%%%%%%%%%%%%%%%%%%%%%%%%%%%%%%%%%%%%%%%%%%%%%%%%%

\section{Slender body residual calculations}\label{residual_calc}
Now that we have proved Theorem \ref{stokes_theorem}, we may proceed to the main aim of the paper: to compare the slender body approximation to the true solution and derive an error estimate in terms of the slender body radius $\epsilon$. In this section, we calculate the residual for the slender body force and velocity approximations, which will then be used in the next section to prove the error bounds in Theorem \ref{stokes_err_theorem}. 

\subsection{Slender body calculations: setup}
The proof of Theorem \ref{stokes_err_theorem} requires knowledge of two expressions: the total surface force ${\bm f}^{\SB}(s)$ exerted by the slender body approximation at each cross section $s$ along the true surface $\Gamma_{\epsilon}$, and the $\theta$-dependence in the slender body velocity $\bu^{\SB}\big|_{\Gamma_\epsilon}(s,\theta)$. Although the true surface velocity $\bu\big|_{\Gamma_{\epsilon}}(s)$ is unknown, we can measure the degree to which $\bu^{\SB}$ fails to satisfy the $\theta$-independence condition along $\Gamma_\epsilon$. In analogy with finite element analysis, the $\theta$-dependence in $\bu^{\SB}\big|_{\Gamma_{\epsilon}}(s,\theta)$ can be considered as the {\it non-conforming} residual, as the slender body approximation $\bu^{\SB}$ therefore does not belong to the function space $\A_{\epsilon}^{\dive}$ required by the well-posedness theory. The force residual ${\bm f}^{\SB}(s)-{\bm f}(s)$, on the other hand, can be considered as the {\it conforming} residual, as the slender body force approximation ${\bm f}^{\SB}$ belongs to the same function space as the prescribed force ${\bm f}$. To show the centerline estimate \eqref{center_err_thm} of Theorem \ref{stokes_err_theorem}, we will also need to consider the centerline residual $|\bu^{\SB}\big|_{\Gamma_\epsilon}(s,\theta)-\bu^{\SB}_C(s)|$ between the slender body approximation on the fiber surface and the centerline slender body approximation \eqref{SBT_asymp}. \\

In this section we will state and prove a few useful lemmas regarding integral estimates along $\T$. The estimates needed to bound both the conforming and non-conforming residuals can be summarized into Lemmas \ref{Rintest0}, \ref{Rintest1}, and \ref{Rintest2}. In addition, we show Lemma \ref{center_est_lem}, which will be used to bound the centerline residual $|\bu^{\SB}\big|_{\Gamma_\epsilon}(s,\theta)-\bu^{\SB}_C(s)|$. These bounds will then be used in Section \ref{SB_vel} to prove a series of propositions leading to Proposition \ref{ur_and_derivs}, which states a bound for the $\theta$-dependence in $\bu^{\SB}\big|_{\Gamma_\epsilon}$ and its derivatives. We will also use Lemma \ref{center_est_lem} to show the centerline residual bound in Proposition \ref{centerline_prop}. In Section \ref{SBforce_res}, we use Lemmas \ref{Rintest0} - \ref{Rintest2} as well as an additional Lemma \ref{theta_int} to estimate the slender body approximation $\bm{f}^{\SB}(s)$ to the force. Ultimately we show Proposition \ref{fSB_est} bounding the residual $\bm{f}^{\SB}(s) - \bm{f}(s)$. Throughout these sections, we will use $c_\kappa$ to denote any constant depending only on the fiber centerline shape through $c_\Gamma$ and $\kappa_{\max}$. \\

We assume that the slender body satisfies the geometric constraints in Section \ref{geometric_constraints}. Although a solution to the slender body PDE \eqref{exterior_stokes} is guaranteed so long as ${\bm f}$ is in $L^2(\T)$, some additional smoothness on $\bm{f}$ is required for the slender body approximation to actually approximate the slender body PDE. Here we will require ${\bm f}$ to be in $C^1(\T)$. We recall that the slender body approximation is given by 
\begin{align}\label{stokes_SB}
\bu^{\SB}(\bx) &=\frac{1}{8\pi}\int_{\T} \bigg( \mc{S}(\bm{R})+\frac{\epsilon^2}{2}\mc{D}(\bm{R}) \bigg)\bm{f}(t) \ts dt; \; 
\bm{R}=\bm{x}-\bm{X}(t),\\
\label{SD}
\mc{S}(\bm{R})&=\frac{{\bf I}}{\abs{\bm{R}}}+\frac{\bm{R}\bm{R}^{\rm T}}{\abs{\bm{R}}^3}, \; 
\mc{D}(\bm{R})=\frac{{\bf I}}{\abs{\bm{R}}^3}-\frac{3\bm{R}\bm{R}^{\rm T}}{\abs{\bm{R}}^5},
\end{align}
with the corresponding slender body pressure given by
\begin{equation}\label{SB_pressure}
p^{\SB}(\bx) = \frac{1}{4\pi}\int_{\T} \frac{\bm{R}\cdot {\bm f}(t)}{|\bm{R}|^3} \ts dt. 
\end{equation}

Recall that within the neighborhood $\mathcal{O}$ \eqref{region_O}, any point $\bx$ can be written
\[\bx(\rho,\theta,s) = \X(s)+\rho \be_{\rho}(s,\theta).\]
Then, for $\bx\in \mathcal{O}$, $\bm{R}$ has the form 
\begin{align*}
\bm{R}(\rho,\theta,s;t) &= \X(s)- \X(t) + \rho \be_{\rho}(s,\theta).
\end{align*}

Before we begin calculations to estimate $\bu^{\SB}$ and $\bm{f}^{\SB}$, we note some useful facts. Using the moving frame ODE \eqref{moving_ODE}, we have 
\begin{align}
\frac{\p \bm{R}}{\p \rho}&=\be_\rho(s,\theta), \label{rhoderiv}\\
\frac{1}{\rho}\frac{\p \bm{R}}{\p \theta}&=\be_\theta(s,\theta),\label{thetaderiv}\\
\frac{1}{1-\rho\widehat{\kappa}}\bigg(\frac{\p \bm{R}}{\p s}-\kappa_3 \frac{\p \bm{R}}{\p \theta}\bigg) &=\be_t(s), \label{sderiv}
\end{align}
where 
\begin{equation}\label{kappahat}
\widehat{\kappa}(s,\theta)=\kappa_1(s)\cos\theta+\kappa_2(s)\sin\theta.
\end{equation}

Next we note that, since $\X$ is a $C^2$ function, for $s,t\in \T$ we have
\begin{equation}\label{CQ}
\X(s)-\X(t)=(s-t)\be_t(s)+(s-t)^2\bm{Q}(s,t), \quad \abs{\bm{Q}(s,t)}\le \frac{\kappa_{\max}}{2}.
\end{equation}

Then, on the slender body surface $\Gamma_{\epsilon}$, we have 
\begin{equation}\label{Reps}
\bm{R}=-\bars \be_t(s)+\epsilon \be_\rho(s,\theta)+ \bars^2\bm{Q}, \quad \abs{\bm{Q}}\le \frac{\kappa_{\max}}{2}, \quad \bars = -(s-t),
\end{equation}
where we have set $\rho=\epsilon$. It will often be convenient to view $\bm{R}$ as a function of $\bars$ and $s$ rather than $t$ and $s$. We may use this expression for $\bm{R}$ to obtain the following two simple estimates.
\begin{lemma}\label{absRests}
Let $\bm{R}$ be as in \eqref{Reps}. Then, for sufficiently small $\epsilon$, we have:
\begin{align}
\label{RQ}
\abs{\abs{\bm{R}}-\sqrt{\bars^2+\epsilon^2}}&\le \frac{\kappa_{\max}}{2} \bars^2,\\
\label{Rlb}
\abs{\bm{R}}&\ge c_\kappa\sqrt{\bars^2+\epsilon^2},
\end{align}
where $\abs{\bars}\leq 1/2$ and $c_\kappa$ depends only on $c_\Gamma$ and $\kappa_{\max}$.
\end{lemma}

\begin{proof}
Note that
\begin{equation}
\abs{\bars \be_t +\epsilon\be_\rho}=\sqrt{\bars^2+\epsilon^2}.
\end{equation}
Inequality \eqref{RQ} then follows from the triangle inequality applied to \eqref{Reps}. To obtain \eqref{Rlb}, note from \eqref{RQ} that, if $\abs{\bars}\le 1/\kappa_{\max}$,
\[\abs{\bm{R}}\ge \sqrt{\bars^2+\epsilon^2}- \frac{\kappa_{\max}}{2}\bars^2\ge \frac{1}{2}\sqrt{\bars^2+\epsilon^2}
+\frac{\abs{\bars}}{2}- \frac{\kappa_{\max}}{2}\bars^2\ge \frac{1}{2}\sqrt{\bars^2+\epsilon^2}.\]
If $\kappa_{\max}\le 2$ we are done. Otherwise, suppose $1/\kappa_{\max}<\abs{\bars}\le 1/2$. Then we have
\begin{equation}
\abs{\bm{R}}\ge \abs{\bm{X}(s)-\bm{X}(t)}-\epsilon \ge c_\Gamma |\bars| -\epsilon\ge  
\frac{c_\Gamma}{\kappa_{\max}}-\epsilon\ge \frac{c_\Gamma}{2\kappa_{\max}},
\end{equation}
where we have used \eqref{non_intersecting} in the second inequality and have taken $\epsilon\le c_\Gamma/(2\kappa_{\max})$ in the last inequality. The above two estimates together imply \eqref{Rlb}.
\end{proof}

%%%%%
We will now make note of some integral estimates that will be used throughout the following section to bound integrals arising from the slender body expression \eqref{stokes_SB} in terms of the prescribed force $\bm{f}\in C^1(\T)$. We first note the following simple but useful calculus result, whose proof we omit. 
\begin{lemma}\label{defints}
Let $m,n$ be integers such that $m\geq 0$ and $n>0$. Then, for $\epsilon$ sufficiently small, we have
\begin{equation}
\int_{-1/2}^{1/2}\frac{\abs{\bars}^{m}}{(\bars^2+\epsilon^2)^{n/2}}d\bars \le 
\begin{cases}
3\abs{\log \epsilon}&\text{ if } n=m+1\\
\pi \epsilon^{m+1-n} &\text{ if } n\geq m+2
\end{cases}
\end{equation}
\end{lemma}

%%%%%
The following integral estimate then follows immediately from Lemmas \ref{absRests} and \ref{defints}. 
\begin{lemma}\label{Rintest0}
Let $\bm{R}$ be as in \eqref{Reps}. Suppose $m,n$ are integers such that $m\geq 0$ and $n>0$. For $\epsilon$ sufficiently small, we have
\begin{equation}
\int_{-1/2}^{1/2} \frac{\abs{\bars}^{m}}{\abs{\bm{R}}^n}d\bars \le 
\begin{cases}
c_\kappa\abs{\log \epsilon} &\text{ if } n=m+1,\\
c_\kappa \epsilon^{m+1-n} &\text{ if } n\ge m+2,
\end{cases}
\end{equation}
where the constants $c_\kappa$ depend only on $n$, $c_\Gamma$ and $\kappa_{\max}$.
\end{lemma}

%%%%%
For the next lemma, we will use the notation 
\begin{equation}
\norm{\bm{g}}_{C^1(\T)}=\norm{\bm{g}}_{C(\T)}+\norm{\bm{g}'}_{C(\T)}, \; 
\norm{\bm{g}}_{C(\T)}=\max_{s\in \T}\abs{\bm{g}(s)}.
\end{equation}

We show the following estimate.
\begin{lemma}\label{Rintest1}
Let $\bm{R}$ be as in \eqref{Reps}. Suppose $m>0$ is an odd integer and $n\ge m+2$, and 
let $\bm{g}\in C^1(\T)$. Then, for sufficiently small $\epsilon$, we have
\begin{equation}\label{Intest_ineq}
\abs{\int^{1/2}_{-1/2}\frac{\bars^m}{\abs{\bm{R}}^n}\bm{g}(\bars)d\bars} \le \begin{cases}
c_\kappa \norm{\bm{g}}_{C^1(\T)}\abs{\log \epsilon} &\text{ if } n=m+2,\\
c_\kappa \norm{\bm{g}}_{C^1(\T)}\epsilon^{m+2-n} &\text{ if } n\ge m+3,
\end{cases}
\end{equation}
where the constants $c_\kappa$ depend only on $n$, $c_\Gamma$ and $\kappa_{\max}$.
\end{lemma}

\begin{proof}
First, we observe that
\begin{equation}
\begin{split}
\int^{1/2}_{-1/2}&\frac{\bars^m}{\abs{\bm{R}}^n}\bm{g}(\bars)d\bars = \bm{I}_1+\bm{I}_2, \\
&\bm{I}_1=\int_{-1/2}^{1/2}\frac{\bars^m}{\abs{\bm{R}}^n}(\bm{g}(\bars)-\bm{g}(0)) \ts d\bars,\\
&\bm{I}_2=\int_{-1/2}^{1/2}\bars^m\bigg(\frac{1}{\abs{\bm{R}}^n}-\frac{1}{(\bars^2+\epsilon^2)^{n/2}}\bigg) \bm{g}(0) \ts d\bars,
\end{split}
\end{equation}
where we used the fact that $m$ is odd in the last equality. We first estimate $\bm{I}_1$. Note that
\[ \abs{\bm{g}(\bars)-\bm{g}(0)}\le \abs{\bars}\norm{\bm{g}'}_{C(\T)}. \]
We have
\begin{equation}\label{I1inRestint1}
\abs{\bm{I}_1}\le \int_{-1/2}^{1/2}\frac{\bars^{m+1}}{\abs{\bm{R}}^n}\norm{\bm{g}'}_{C(\T)}d\bars \le
\begin{cases}
c_\kappa \norm{\bm{g}'}_{C(\T)}\abs{\log \epsilon} &\text{ if } n=m+2,\\
c_\kappa \norm{\bm{g}'}_{C(\T)}\epsilon^{m+2-n} &\text{ if } n\ge m+3,
\end{cases}
\end{equation}
where we used Lemma \ref{Rintest0}. We turn to $\bm{I}_2$. Note that
\[ \frac{1}{\abs{\bm{R}}^n}-\frac{1}{(\sqrt{\bars^2+\epsilon^2})^n}=
\bigg(\frac{1}{\abs{\bm{R}}}-\frac{1}{\sqrt{\bars^2+\epsilon^2}}\bigg) \sum_{l=0}^{n-1}\frac{1}{\abs{\bm{R}}^{l}\big(\sqrt{\bars^2+\epsilon^2}\big)^{n-1-l}}. \]
Using Lemma \ref{absRests}, we have
\[ \abs{\frac{1}{\abs{\bm{R}}}-\frac{1}{\sqrt{\bars^2+\epsilon^2}}}
=\frac{\abs{\abs{\bm{R}}-\sqrt{\bars^2+\epsilon^2}}}{\abs{\bm{R}}\sqrt{\bars^2+\epsilon^2}}
\le \frac{c_\kappa\bars^2}{c_\kappa(\bars^2+\epsilon^2)}. \]
Then, using Lemma \ref{absRests} again, we have
\begin{equation}\label{Rnsigman}
 \abs{\frac{1}{\abs{\bm{R}}^n}-\frac{1}{(\sqrt{\bars^2+\epsilon^2})^n}} \le \frac{c_\kappa\bars^2}{(\bars^2+\epsilon^2)^{(n+1)/2}}.
 \end{equation}
 Thus,
\begin{equation}\label{I2inRestint1}
\begin{split}
\abs{\bm{I}_2}&\le c_\kappa\norm{\bm{g}}_{C(\T)}\int_{-1/2}^{1/2}\frac{\bars^{m+2}}{(\bars^2+\epsilon^2)^{(n+1)/2}}d\bars \\
&\le \begin{cases}
3 c_\kappa\norm{\bm{g}}_{C(\T)}\abs{\log \epsilon}&\text{ if } n=m+2,\\
\pi c_\kappa\norm{\bm{g}}_{C(\T)}\epsilon^{m+2-n} &\text{ if } n\ge m+3,
\end{cases}
\end{split}
\end{equation}
where we used Lemma \ref{defints} in the last inequality. Combining \eqref{I1inRestint1} and \eqref{I2inRestint1}, we obtain the inequality \eqref{Intest_ineq}.
\end{proof}

%%%%%
The final integral we estimate is the following.
\begin{lemma}\label{Rintest2}
Suppose $m\ge 0$ is an even integer, $n$ is an integer such that $n\ge m+3$ and let $\bm{g}\in C^1(\T)$. Then, for sufficiently small $\epsilon$, we have
\begin{equation}\label{epsdmn}
\begin{split}
\abs{\int_{-1/2}^{1/2} \frac{\bars^m}{\abs{\bm{R}}^n}\bm{g}(\bars)d\bars -\epsilon^{m+1-n}d_{mn} \bm{g}(0)} 
&\le c_\kappa\norm{\bm{g}}_{C^1(\T)}\epsilon^{m+2-n},\\
d_{mn}&=\int_{-\infty}^\infty \frac{\tau^m}{(\tau^2+1)^{n/2}}d\tau,
\end{split}
\end{equation}
where the constant $c_\kappa$ depends only on $n$, $c_\Gamma$ and $\kappa_{\max}$. For odd $n$, we have
\begin{equation}
d_{mn}=\sum_{k=0}^{m/2}(-1)^k\begin{pmatrix} m/2\\ 
k \end{pmatrix} d_{0,n-k},\;
d_{0n}=2\frac{(n-3)!!}{(n-2)!!}.
\end{equation}
For certain values of $m$ and $n$, this yields
\begin{equation}
d_{03}=2, \; d_{05}=\frac{4}{3}, \; d_{07}= \frac{16}{15}, \; d_{25}=\frac{2}{3}, \; d_{27}=\frac{4}{15}.
\end{equation}
\end{lemma}

Note that Lemma \ref{Rintest2} immediately implies that, for $\bm{g}\in C^1(\Gamma_\epsilon)$, we have  
\begin{equation}\label{epsdmn_theta}
\begin{aligned}
&\abs{\int_0^{2\pi}\int_{-1/2}^{1/2} \frac{\bars^m}{\abs{\bm{R}}^n}\bm{g}(\bars,\theta) d\bars \ts d\theta -\epsilon^{m+1-n}d_{mn} \int_0^{2\pi}\bm{g}(0,\theta)\ts d\theta} \\
&\hspace{6cm} \le c_\kappa\max_{0\le\theta<2\pi}\norm{\bm{g}(\cdot,\theta)}_{C^1(\T)}\epsilon^{m+2-n}.
\end{aligned}
\end{equation}

\begin{proof}[Proof of Lemma \ref{Rintest2}:]
First, note that
\begin{equation}
\begin{split}
\int^{1/2}_{-1/2}& \frac{\bars^m \bm{g}(\bars)}{\abs{\bm{R}}^n}d\bars - \bm{g}(0)\int_{-\infty}^{\infty} \frac{\bars^m}{(\bars^2+\epsilon^2)^{n/2}}d\bars =\bm{I}_1+\bm{I}_2+\bm{I}_3, \\
\bm{I}_1 &=\int_{-1/2}^{1/2}\frac{\bars^m (\bm{g}(\bars)-\bm{g}(0))}{\abs{\bm{R}}^n}d\bars,\\
\bm{I}_2 &=\int_{-1/2}^{1/2}\bars^m\bigg(\frac{1}{\abs{\bm{R}}^n}-\frac{1}{(\bars^2+\epsilon^2)^{n/2}}\bigg)\bm{g}(0) \ts d\bars,\\
\bm{I}_3 &=2\bm{g}(0)\int_{1/2}^\infty \frac{\bars^m}{(\bars^2+\epsilon^2)^{n/2}}d\bars.
\end{split}
\end{equation}
We may estimate $\bm{I}_1$ and $\bm{I}_2$ in exactly the same way as in the proof of Lemma \ref{Rintest1}. We find that
\begin{align*}
\abs{\bm{I}_1}&\le c_\kappa \norm{\bm{g}'}_{C(\T)}\epsilon^{m+2-n}, \quad \abs{\bm{I}_2}\le c_\kappa\norm{\bm{g}}_{C(\T)}\epsilon^{m+2-n}.
\end{align*}
For $\bm{I}_3$, a simple estimation yields
\begin{align*}
\abs{\bm{I}_3}&\le 2\norm{\bm{g}}_{C(\T)}\int_{1/2}^\infty \frac{\bars^m d\bars}{(\bars^2+\epsilon^2)^{n/2}} \le 2\norm{\bm{g}}_{C(\T)}\int_{1/2}^\infty \bars^{m-n} d\bars = \frac{2^{n-m}}{n-m-1}\norm{\bm{g}}_{C(\T)}.
\end{align*}
Finally, we have
\[ \int_{-\infty}^{\infty} \frac{\bars^m}{(\bars^2+\epsilon^2)^{n/2}}d\bars = \epsilon^{m+1-n}\int_{-\infty}^{\infty} \frac{\tau^m}{(\tau^2+1)^{n/2}} d\tau \equiv \epsilon^{m+1-n}d_{nm}. \]
Combining all of the above, we obtain \eqref{epsdmn}. Note that, since $m$ is even,
\[d_{mn}=\int_{-\infty}^\infty\frac{(\tau^2+1-1)^{m/2}d\tau}{(\tau^2+1)^{n/2}} =\sum_{k=0}^{m/2}(-1)^k\begin{pmatrix} m/2\\ k\end{pmatrix} d_{0,n-k}. \]
For $n$ odd, we have
\begin{align*}
d_{0n}&=\int_{-\pi/2}^{\pi/2} \cos^{n-2}\varphi d\varphi=
\frac{n-3}{n-2}\int_{-\pi/2}^{\pi/2}\cos^{n-4}\varphi d\varphi\\
&=\cdots=\frac{(n-3)(n-5)\cdots 4\cdot 2}{(n-2)(n-4)\cdot 3}\int_{-\pi/2}^{\pi/2}\cos \varphi d\varphi
=2\frac{(n-3)!!}{(n-2)!!}.
\end{align*}
\end{proof}

Finally, we make note of the following lemma, which will be useful for estimating the centerline expression \eqref{SBT_asymp} to obtain the estimate \eqref{center_err_thm} of Theorem \ref{stokes_err_theorem}. 
Recalling the notation $\bm{R}_0(s,\bars) = \X(s) - \X(s+\bars)$, we show:
\begin{lemma}\label{center_est_lem}
Let $\bm{R}$ be as in \eqref{Reps} and suppose $n= 1$ or $n=3$. Then for $\bm{g}\in C^1(\T)$ and $\epsilon$ sufficiently small, we have 
\begin{equation}\label{cent_lem_eq}
\begin{aligned}
\bigg|\int_{-1/2}^{1/2} \frac{\bars^{n-1}}{\abs{\bm{R}}^n}\bm{g}(\bars)d\bars - &\int_{-1/2}^{1/2} \bigg(\frac{\bars^{n-1}}{\abs{\bm{R}_0}^n} \bm{g}(\bars)-\frac{\bm{g}(0)}{\abs{\bars} } \bigg)d\bars + \bm{g}(0)\log(\epsilon^2) + (n-1)\bm{g}(0) \bigg| \\
& \le \epsilon \abs{\log\epsilon} c_\kappa\norm{\bm{g}}_{C^1(\T)},
\end{aligned}
\end{equation}
where the constant $c_\kappa$ depends only on $n$, $c_\Gamma$, and $\kappa_{\max}$. 
\end{lemma} 

\begin{proof}
We begin by considering 
\begin{equation}\label{bmJ}
\bm{J} = \int_{-1/2}^{1/2} \bigg[\bigg(\frac{\bars^{n-1}}{\abs{\bm{R}}^n} - \frac{\bars^{n-1}}{\abs{\bm{R}_0}^n} \bigg) \bm{g}(\bars)+ \frac{\epsilon^2 \bm{g}(0)}{\abs{\bars}\sqrt{\bars^2+\epsilon^2} (\abs{\bars}+\sqrt{\bars^2+\epsilon^2})} + (n-1)\bm{g}(0)\bigg]d\bars.
\end{equation}

We may estimate $\bm{J}$ as 
\begin{align*}
\bm{J} &= \bm{J}_1 + \bm{J}_2+\bm{J}_3; \\
\bm{J}_1 &:= \int_{-1/2}^{1/2} \bigg(\frac{\bars^{n-1}}{\abs{\bm{R}}^n} - \frac{\bars^{n-1}}{\abs{\bm{R}_0}^n} \bigg) (\bm{g}(\bars)-\bm{g}(0))d\bars \\
\bm{J}_2 &:= \int_{-1/2}^{1/2} \bigg(\frac{1}{\abs{\bm{R}}} - \frac{1}{\abs{\bm{R}_0}} +\frac{\epsilon^2}{\abs{\bars}\sqrt{\bars^2+\epsilon^2} (\abs{\bars}+\sqrt{\bars^2+\epsilon^2})}  \bigg) \bm{g}(0) d\bars \\
\bm{J}_3 &:= \int_{-1/2}^{1/2} \bigg(\frac{\bars^{n-1}}{\abs{\bm{R}}^n} -\frac{1}{\abs{\bm{R}}} - \frac{\bars^{n-1}}{\abs{\bm{R}_0}^n} +\frac{1}{\abs{\bm{R}_0}} \bigg) \bm{g}(0) d\bars + (n-1) \bm{g}(0). 
\end{align*}

To estimate each $\bm{J}_i$, it will be convenient to define 
\begin{equation}\label{IR_def}
I_R := \frac{1}{\abs{\bm{R}}} - \frac{1}{\abs{\bm{R}_0}} = \frac{-\epsilon^2 - 2\epsilon\bars^2\bm{Q}\cdot\be_\rho}{\abs{\bm{R}}\abs{\bm{R}_0}(\abs{\bm{R}_0}+\abs{\bm{R}})},
\end{equation}
where we have used \eqref{CQ} and \eqref{Reps}. \\

Note that, using \eqref{IR_def} along with \eqref{Rlb} and \eqref{non_intersecting}, we have
\[ \abs{\frac{\bars^{n-1}}{\abs{\bm{R}}^n} - \frac{\bars^{n-1}}{\abs{\bm{R}_0}^n}} \le c_\kappa\abs{I_R}  \le c_\kappa\bigg( \frac{\epsilon^2}{\abs{\bars}(\bars^2+\epsilon^2)} + \frac{\epsilon \abs{\bars}}{\bars^2+\epsilon^2}\bigg). \]
Therefore, using that $\bm{g}\in C^1(\T)$, we can estimate $\bm{J}_1$ as 
\[ \abs{\bm{J}_1} \le c_\kappa \norm{\bm{g}'}_{C(\T)} \int_{-1/2}^{1/2} \frac{\epsilon^2+\epsilon\bars^2}{\bars^2+\epsilon^2} d\bars \le c_\kappa \epsilon \norm{\bm{g}'}_{C(\T)}.  \]

Furthermore, using the notation \eqref{IR_def}, the integrand of $\bm{J}_2$ satisfies
\begin{align*}
\abs{I_R +\frac{\epsilon^2}{\abs{\bars}\sqrt{\bars^2+\epsilon^2} (\abs{\bars}+\sqrt{\bars^2+\epsilon^2})} } &\le \abs{\frac{1}{\sqrt{\bars^2+\epsilon^2}} - \frac{1}{\abs{\bm{R}}}} \frac{\epsilon^2}{\abs{\bm{R}_0} (\abs{\bm{R}_0}+\abs{\bm{R}})}  \\
&\quad + \abs{\frac{1}{\abs{\bars}} - \frac{1}{\abs{\bm{R}_0}} } \frac{\epsilon^2}{\sqrt{\bars^2+\epsilon^2} (\abs{\bm{R}_0}+\abs{\bm{R}})}  \\ 
&\quad +\abs{\frac{1}{(\abs{\bars}+\sqrt{\bars^2+\epsilon^2})}- \frac{1}{(\abs{\bm{R}_0}+\abs{\bm{R}})}}\frac{\epsilon^2}{\abs{\bars}\sqrt{\bars^2+\epsilon^2}} \\
&\quad + \frac{c_\kappa\epsilon \bars^2}{\abs{\bm{R}}\abs{\bm{R}_0}(\abs{\bm{R}_0}+\abs{\bm{R}})} \\
%
%
&\le c_\kappa \frac{\epsilon^2+\epsilon\abs{\bars}}{\bars^2+\epsilon^2} ,
\end{align*}
where we have used \eqref{RQ} and \eqref{CQ} along with the triangle inequality to bound the difference expressions and \eqref{Rlb} and \eqref{non_intersecting} to bound each of the denominators. Then $\bm{J}_2$ satisfies
\begin{align*}
\abs{\bm{J}_2} \le c_\kappa \int_{-1/2}^{1/2} \frac{(\epsilon^2+\epsilon\abs{\bars})|\bm{g}(0)|}{\bars^2+\epsilon^2} d\bars \le c_\kappa \epsilon\abs{\log\epsilon} \norm{\bm{g}}_{C(\T)},
\end{align*}
by Lemma \ref{defints}. \\

If $n=1$, we are done. For $n=3$, we must also estimate $\bm{J}_3$. We have that $\bm{J}_3$ satisfies 
\begin{align*}
\abs{\bm{J}_3} &\le \norm{\bm{g}}_{C(\T)}\int_{-1/2}^{1/2} \bigg( \bigg|\frac{\bars^2+\epsilon^2-\abs{\bm{R}}^2}{\abs{\bm{R}}^3} - \frac{\bars^2-\abs{\bm{R}_0}^2}{\abs{\bm{R}_0}^3}\bigg| +\bigg|\frac{\epsilon^2}{\abs{\bm{R}}^3} - \frac{\epsilon^2}{\sqrt{\bars^2+\epsilon^2}^3} \bigg|  \bigg) d\bars \\
&\qquad + \norm{\bm{g}}_{C(\T)}\bigg|\int_{-1/2}^{1/2} \frac{\epsilon^2}{\sqrt{\bars^2+\epsilon^2}^3}d\bars - 2 \bigg| \\
%
&\le \norm{\bm{g}}_{C(\T)}\int_{-1/2}^{1/2} \bigg|\frac{2\bars^3\bm{Q}\cdot\be_t-\bars^4\abs{\bm{Q}}^2-2\epsilon\bars^2\bm{Q}\cdot\be_\rho}{\abs{\bm{R}}^3} - \frac{2\bars^3\bm{Q}\cdot\be_t-\bars^4\abs{\bm{Q}}^2}{\abs{\bm{R}_0}^3} \bigg| d\bars \\
&\qquad +c_\kappa \norm{\bm{g}}_{C(\T)}\int_{-1/2}^{1/2} \frac{\epsilon^2\bars^2}{(\bars^2+\epsilon^2)^2} d\bars  + \norm{\bm{g}}_{C(\T)}\bigg|\frac{2}{\sqrt{1+4\epsilon^2}} - 2 \bigg| \\
%
&\le c_\kappa\norm{\bm{g}}_{C(\T)}\int_{-1/2}^{1/2} (|\bars|^3+\bars^4)\bigg(\abs{I_R} \sum_{\ell=0}^2\frac{1}{\abs{\bm{R}_0}^\ell\abs{\bm{R}}^{2-\ell}}  \bigg) d\bars+c_\kappa \epsilon \abs{\log\epsilon} \norm{\bm{g}}_{C(\T)} \\
%
&\le c_\kappa\norm{\bm{g}}_{C(\T)}\int_{-1/2}^{1/2} \frac{\epsilon^2+\epsilon\bars^2 + \epsilon^2|\bars|+\epsilon|\bars|^3}{\abs{\bm{R}}^2} d\bars+c_\kappa \epsilon \abs{\log\epsilon} \norm{\bm{g}}_{C(\T)} \\ 
%
&\le c_\kappa \epsilon \abs{\log\epsilon} \norm{\bm{g}}_{C(\T)},
\end{align*}
using \eqref{Reps}, \eqref{CQ}, and \eqref{Rnsigman} in the second inequality, definition \ref{IR_def} along with Lemmas \ref{defints} and \ref{Rintest0} in the third inequality, and \eqref{non_intersecting} in the fourth inequality. \\

Finally, we show that the expression for $\bm{J}$ \eqref{bmJ} closely matches the expression on the left hand side of \eqref{cent_lem_eq}. We evaluate
\begin{align*}
\int_{-1/2}^{1/2}\bigg(\frac{\epsilon^2}{\abs{\bars}\sqrt{\bars^2+\epsilon^2} (\abs{\bars}+\sqrt{\bars^2+\epsilon^2})} -\frac{1}{\abs{\bars}} \bigg) d\bars &= -\int_{-1/2}^{1/2}\frac{\abs{\bars}\sqrt{\bars^2+\epsilon^2}+\bars^2}{\abs{\bars}\sqrt{\bars^2+\epsilon^2} (\abs{\bars}+\sqrt{\bars^2+\epsilon^2})} d\bars \\
&= -\int_{-1/2}^{1/2}\frac{1}{\sqrt{\bars^2+\epsilon^2} } d\bars \\
&= \log\bigg(\frac{\epsilon^2}{\frac{1}{2}+(\epsilon^2+\frac{1}{4})^{1/2}+\epsilon^2} \bigg).
\end{align*}

Using that 
\[ \bigg|\log\bigg(\frac{\epsilon^2}{\frac{1}{2}+(\epsilon^2+\frac{1}{4})^{1/2}+\epsilon^2}\bigg)  - \log(\epsilon^2)\bigg| \le c \epsilon^2, \]
we obtain Lemma \ref{center_est_lem}.
\end{proof}

%%%%%%%%%%%%%%%%%%%%%%%%%%%%%%%%%
%%%%%%%%%%%%%%%%%%%%%%%%%%%%%%%%%
%%%%%%%%%%%%%%%%%%%%%%%%%%%%%%%%%
\subsection{Slender body velocity residual}\label{SB_vel}
We will now use Lemmas  \ref{Rintest0}, \ref{Rintest1}, and \ref{Rintest2} to obtain an estimate on the non-conforming error -- the degree to which $\bu^{\SB}\big|_{\Gamma_\epsilon}(s,\theta)$ fails to satisfy the $\theta$-independence condition along the fiber surface $\Gamma_\epsilon$. We establish some estimates on $\bu^{\SB}$ and its derivatives along $\Gamma_{\epsilon}$. The derivative estimates will be needed in Section \ref{error_est_section} to obtain an actual error estimate between the slender body approximation $\bu^{\SB}$ and the true solution $\bu$. \\

We show the following proposition. 
\begin{proposition}\label{prop:uSBtheta}
Consider $\bu^{\SB}(\bx)$ for $\bx \in \Gamma_\epsilon$. For sufficiently small $\epsilon$, we have
\begin{equation}
\abs{\frac{1}{\epsilon}\frac{\p\bu^{\SB}}{\p \theta}}\le c_\kappa\norm{\bm{f}}_{C^1(\T)}\abs{\log \epsilon} 
\end{equation}
where the constant $c_\kappa$ depends only on $c_\Gamma$ and $\kappa_{\max}$.
\end{proposition}

\begin{proof} 
Write $\bx=\X(s)+\epsilon \be_\rho$. Using \eqref{stokes_SB}, we have:
\begin{equation}\label{ISDtheta}
\begin{split}
\frac{8\pi}{\epsilon}\frac{\p\bu^{\SB}}{\p \theta} &=\bm{I}_{\mc{S}}+\bm{I}_{\mc{D}}; \\
\bm{I}_\mc{S}&=\frac{1}{\epsilon}\frac{\p}{\p\theta}\int_{-1/2}^{1/2}\mc{S}(\bm{R})\bm{f}(s+\bars) d\bars,\\
\bm{I}_\mc{D}&=\frac{1}{\epsilon}\frac{\p}{\p\theta}\int_{-1/2}^{1/2}\frac{\epsilon^2}{2}\mc{D}(\bm{R})\bm{f}(s+\bars) d\bars.
\end{split}
\end{equation}
We first consider $\bm{I}_{\mc{S}}$. Using \eqref{SD} and \eqref{thetaderiv}, we have
\begin{align*}
\bm{I}_{\mc{S}} &=\bm{I}_{\mc{S},1}+\bm{I}_{\mc{S},2};\\
\bm{I}_{\mc{S},1} &=-\int_{-1/2}^{1/2} \bigg(\frac{\bm{R}\cdot\be_\theta}{\abs{\bm{R}}^3}\bm{f}
+3\frac{(\bm{R}\cdot\be_\theta)(\bm{R}\cdot\bm{f})}{\abs{\bm{R}}^5}\bm{R} \bigg)d\bars,\\
\bm{I}_{\mc{S},2}&=\int_{-1/2}^{1/2} \bigg(\frac{(\bm{R}\cdot \bm{f})\be_\theta+(\be_\theta\cdot \bm{f})\bm{R}}{\abs{\bm{R}}^3}\bigg)d\bars.
\end{align*}
We estimate $\bm{I}_{\mc{S},1}$. First, note from \eqref{CQ} that
\[\abs{\bm{R}\cdot \be_\theta}=\bars^2\abs{\bm{Q}\cdot \be_\theta}\le \frac{\kappa_{\max}}{2}\bars^2. \]
Applying Lemma \ref{Rintest0}, we then have
\begin{equation}\label{IS1}
\abs{\bm{I}_{\mc{S},1}}\le \int_{-1/2}^{1/2}4\frac{\abs{\bm{R}\cdot\be_\theta}}{\abs{\bm{R}}^3}\norm{\bm{f}}_{C(\T)} d\bars \le c_\kappa\norm{\bm{f}}_{C(\T)}\abs{\log \epsilon}.
\end{equation}
Turning to $\bm{I}_{\mc{S},2}$, we note that $(\bm{R}\cdot \bm{f})\be_\theta+(\be_\theta\cdot \bm{f})\bm{R}=\epsilon \bm{g}_0+\bars\bm{g}_1+\bars^2\bm{g}_2$, where
\begin{equation}\label{g012def}
\begin{split}
\bm{g}_0(\bars;s) &=\be_\rho(s)(\be_\theta(s)\cdot \bm{f}(s+\bars))+ \be_\theta(s)(\be_\rho(s)\cdot \bm{f}(s+\bars)),\\
\bm{g}_1(\bars;s) &=\be_t(s)(\be_\theta(s)\cdot \bm{f}(s+\bars))+ \be_\theta(s)(\be_t(s)\cdot \bm{f}(s+\bars)),\\
\bm{g}_2 &=\bm{Q}(\be_\theta\cdot \bm{f})+\be_\theta(\bm{Q}\cdot \bm{f}),
\end{split}
\end{equation}
and we have written out the explicit dependence of $\bm{g}_0$ and $\bm{g}_1$ on $\bars$ and $s$. Applying Lemma \ref{Rintest0} and \eqref{CQ}, we have
\[\abs{\int_{-1/2}^{1/2} \frac{\bars^2 \bm{g}_2}{\abs{\bm{R}}^3}d\bars}
\le \norm{\bm{g}_2(\cdot;s)}_{C(\T)}\int_{-1/2}^{1/2} \frac{\bars^2}{\abs{\bm{R}}^3}d\bars
\le c_\kappa\norm{\bm{f}}_{C(\T)}\abs{\log \epsilon}. \]

Using Lemma \ref{Rintest1}, we have
\[ \abs{\int_{-1/2}^{1/2} \frac{\bars \bm{g}_1(\bars;s)}{\abs{\bm{R}}^3}d\bars}
\le c_\kappa\norm{\bm{g}_1(\cdot;s)}_{C^1(\T)}\abs{\log \epsilon}\le c_\kappa\norm{\bm{f}}_{C^1(\T)}\abs{\log \epsilon}. \]

Finally, using Lemma \ref{Rintest2} with $m=0$, $n=3$, we have 
\begin{equation}\label{hdef}
\begin{split}
\abs{\int_{-1/2}^{1/2} \frac{\epsilon \bm{g}_0(\bars;s)}{\abs{\bm{R}}^3}d\bars - \frac{2}{\epsilon}\bm{g}_0(0; s)}
&\le c_\kappa\norm{\bm{g}_0(\cdot;s)}_{C^1(\T)}\le c_\kappa\norm{\bm{f}}_{C^1(\T)};\\
\bm{g}_0(0;s)&=\be_\rho(s)(\be_\theta(s)\cdot \bm{f}(s))+\be_\theta(s)(\be_\rho(s)\cdot \bm{f}(s)) =: \bm{h}(s).
\end{split}
\end{equation}

Combining the above estimates, we obtain
\begin{equation}\label{IS2}
\abs{\bm{I}_{\mc{S},2}-\frac{2}{\epsilon}\bm{h}(s)}\le c_\kappa\norm{\bm{f}}_{C^1(\T)}\abs{\log\epsilon}.
\end{equation}

Finally, combining \eqref{IS1} and \eqref{IS2}, we have
\begin{equation}\label{ISh}
\abs{\bm{I}_{\mc{S}}-\frac{2}{\epsilon}\bm{h}(s)}\le 
\abs{\bm{I}_{\mc{S},1}}+\abs{\bm{I}_{\mc{S},2}-\frac{2}{\epsilon}\bm{h}(s)}\le
c_\kappa\norm{\bm{f}}_{C^1(\T)}\abs{\log\epsilon}.
\end{equation}

We next consider $\bm{I}_{\mc{D}}$ in \eqref{ISDtheta}. We write
\begin{equation}\label{IDdef}
\begin{split}
\bm{I}_{\mc{D}}&=\frac{3\epsilon^2}{2}\big(\bm{I}_{\mc{D},1}+\bm{I}_{\mc{D},2} \big),\\
\bm{I}_{\mc{D},1}&=\int_{-1/2}^{1/2} \bigg(-\frac{\bm{R}\cdot\be_\theta}{\abs{\bm{R}}^5}\bm{f}+5\frac{(\bm{R}\cdot\be_\theta)(\bm{R}\cdot\bm{f})}{\abs{\bm{R}}^7}\bm{R} \bigg)d\bars,\\
\bm{I}_{\mc{D},2} &= -\int_{-1/2}^{1/2} \bigg(\frac{(\bm{R}\cdot \bm{f})\be_\theta+(\be_\theta \cdot \bm{f})\bm{R}}{\abs{\bm{R}}^5}\bigg) d\bars.
\end{split}
\end{equation}
Following the same steps used to estimate $\bm{I}_{\mc{S},1}$ and $\bm{I}_{\mc{S},2}$ in \eqref{IS1} and \eqref{IS2}, we obtain
\begin{align*}
\abs{\bm{I}_{\mc{D},1}}&\le c_\kappa\norm{\bm{f}}_{C(\T)}\epsilon^{-2},\\
\abs{\bm{I}_{\mc{D},2}+\frac{4}{3\epsilon^3}\bm{h}(s)}&\le c_\kappa\norm{\bm{f}}_{C^1(\T)}\epsilon^{-2},
\end{align*}
where $\bm{h}(s)$ was given in \eqref{hdef}. In particular, in the second estimate, we used Lemma \ref{Rintest2} with $m=0$ and $n=5$. \\

Combining the above, we have
\begin{equation}\label{IDh}
\abs{\bm{I}_{\mc{D}}+\frac{2}{\epsilon}\bm{h}(s)}\le \frac{3\epsilon^2}{2}\bigg(\abs{\bm{I}_{\mc{D},1}}+\abs{\bm{I}_{\mc{D},2}+\frac{4}{3\epsilon^3}\bm{h}(s)}\bigg) \le c_\kappa\norm{\bm{f}}_{C^1(\T)}.
\end{equation}

We finally estimate \eqref{ISDtheta} as 
\begin{align*}
\abs{\frac{1}{\epsilon}\frac{\p\bu^{\SB}}{\p \theta}}&\le \frac{1}{8\pi}\bigg(\abs{\bm{I}_{\mc{S}} -\frac{2}{\epsilon}\bm{h}(s)}+\abs{\bm{I}_{\mc{D}}+\frac{2}{\epsilon}\bm{h}(s)} \bigg) \le c_\kappa\norm{\bm{f}}_{C^1(\T)}\abs{\log \epsilon},
\end{align*}
where we used \eqref{ISh} and \eqref{IDh} in the last inequality.
\end{proof}

%%%%%
We next show the following proposition.
\begin{proposition}\label{prop:uSBstheta}
Consider $\bu^{\SB}(\bx)$ for $\bx\in \Gamma_\epsilon$. The following estimate holds for sufficiently small $\epsilon$:
\begin{equation}
\abs{\frac{\p}{\p\theta}\bigg(\frac{\p\bu^{\SB}}{\p s} - \kappa_3\frac{\p\bu^{\SB}}{\p\theta} \bigg)}\le c_\kappa\norm{\bm{f}}_{C^1(\T)},
\end{equation}
where the constant $c_\kappa$ depends only on the constants $c_\Gamma$ and $\kappa_{\max}$.
\end{proposition}

\begin{proof}
First, note that 
\begin{equation}\label{uSBstheta}
\begin{split}
\frac{\p}{\p\theta}&\bigg(\frac{\p\bu^{\SB}}{\p s}-\kappa_3\frac{\p\bu^{\SB}}{\p\theta}\bigg) = \frac{1}{8\pi}\bigg((1-\epsilon\wh\kappa)\frac{\p \bm{I}^{\SB}}{\p \theta} - \epsilon\frac{\p \wh{\kappa}}{\p \theta}\bm{I}^{\SB} \bigg);\\
&\bm{I}^{\SB}=\frac{8\pi}{1-\epsilon\wh{\kappa}}\bigg(\frac{\p \bu^{\SB}}{\p s} -\kappa_3\frac{\p\bu^{\SB}}{\p \theta}\bigg) = \bm{I}_{\mc{S}}+\frac{3\epsilon^2}{2}\bm{I}_{\mc{D}},\\
&\bm{I}_{\mc{S}}=\int_{-1/2}^{1/2}\bigg(\frac{-\bm{R}\cdot\be_t}{\abs{\bm{R}}^3}\bm{f} + \frac{(\bm{R}\cdot \bm{f})\be_t +(\be_t\cdot \bm{f})\bm{R}}{\abs{\bm{R}}^3} - 3\frac{(\bm{R}\cdot\be_t)(\bm{R}\cdot\bm{f})}{\abs{\bm{R}}^5}\bm{R} \bigg)d\bars,\\
&\bm{I}_{\mc{D}}= \int_{-1/2}^{1/2}\bigg(\frac{-\bm{R}\cdot\be_t}{\abs{\bm{R}}^5}\bm{f} -\frac{(\bm{R}\cdot \bm{f})\be_t +(\be_t\cdot \bm{f})\bm{R}}{\abs{\bm{R}}^5} +5\frac{(\bm{R}\cdot\be_t)(\bm{R}\cdot\bm{f})}{\abs{\bm{R}}^7}\bm{R} \bigg)d\bars,
\end{split}
\end{equation}
where we used \eqref{stokes_SB} and \eqref{sderiv} to obtain the expression for $\bm{I}_{\mc S}$ and $\bm{I}_{\mc D}$.\\

Let us estimate $\bm{I}^{\SB}$. We have
\begin{align*}
\abs{\bm{I}_{\mc{S}}} &\le \int_{-1/2}^{1/2}\abs{\frac{-\bm{R}\cdot\be_t}{\abs{\bm{R}}^3}\bm{f} +\frac{(\bm{R}\cdot \bm{f})\be_t+(\be_t\cdot \bm{f})\bm{R}}{\abs{\bm{R}}^3} - 3\frac{(\bm{R}\cdot\be_t)(\bm{R}\cdot\bm{f})}{\abs{\bm{R}}^5}\bm{R}} d\bars \\
&\le \int_{-1/2}^{1/2} \frac{6\norm{\bm{f}}_{C(\T)}}{\abs{\bm{R}}^2}d\bars
\le c_\kappa \norm{\bm{f}}_{C(\T)}\epsilon^{-1},
\end{align*}
where we used Lemma \ref{Rintest0} in the last inequality. Likewise,
\begin{align*}
\abs{\bm{I}_{\mc{D}}} &\le \int_{-1/2}^{1/2}\abs{\frac{-\bm{R}\cdot\be_t}{\abs{\bm{R}}^5}\bm{f} -\frac{(\bm{R}\cdot \bm{f})\be_t +(\be_t\cdot \bm{f})\bm{R}}{\abs{\bm{R}}^5} +5\frac{(\bm{R}\cdot\be_t)(\bm{R}\cdot\bm{f})}{\abs{\bm{R}}^7}\bm{R}} d\bars \\
&\le \int_{-1/2}^{1/2} \frac{8\norm{\bm{f}}_{C(\T)}}{\abs{\bm{R}}^4}d\bars \le c_\kappa \norm{\bm{f}}_{C(\T)}\epsilon^{-3},
\end{align*}
where we again used Lemma \ref{Rintest0} in the last inequality. Using the above estimates, we have
\begin{equation}\label{CI}
\abs{\bm{I}^{\SB}} \le \abs{\bm{I}_{\mc S}}+\frac{3\epsilon^2}{2}\abs{\bm{I}_{\mc D}}
\le c_\kappa\norm{\bm{f}}_{C(\T)}\epsilon^{-1}.
\end{equation}

We now estimate $\p\bm{I}^{\SB}/\p \theta$. We have
\begin{equation}\label{ISstheta}
\begin{split}
\frac{\p \bm{I}_{\mc S}}{\p \theta} &= \epsilon\big(\bm{I}_{\mc{S},1}+\bm{I}_{\mc{S},2}+\bm{I}_{\mc{S},3}+\bm{I}_{\mc{S},4}\big);\\
%
\bm{I}_{\mc{S},1} &= 3\int_{-1/2}^{1/2}\bigg(\frac{(\bm{R}\cdot\be_\theta)}{\abs{\bm{R}}^5} ((\bm{R}\cdot\be_t)\bm{f}-(\bm{R}\cdot \bm{f})\be_t -(\be_t\cdot \bm{f})\bm{R}) \bigg) d\bars,\\
\bm{I}_{\mc{S},2} &= \int_{-1/2}^{1/2}\frac{(\be_\theta\cdot \bm{f})\be_t +(\be_t\cdot \bm{f})\be_\theta}{\abs{\bm{R}}^3}d\bars,\\
\bm{I}_{\mc{S},3} &= -3\int_{-1/2}^{1/2}\frac{(\bm{R}\cdot\be_t)}{\abs{\bm{R}}^5} \big((\bm{R}\cdot\bm{f})\bm{e}_\theta+(\bm{e}_\theta\cdot\bm{f})\bm{R} \big) d\bars,\\
\bm{I}_{\mc{S},4} &= 15\int_{1/2}^{1/2} \frac{(\bm{R}\cdot\be_t)(\bm{R}\cdot\bm{f})(\bm{R}\cdot\be_\theta)}{\abs{\bm{R}}^7}\bm{R}d\bars.
\end{split}
\end{equation}
We estimate each term in turn. Using \eqref{Reps} and Lemma \ref{Rintest0}, we have that $\bm{I}_{\mc{S},1}$ satisfies
\begin{equation}\label{IS1stheta}
\abs{\bm{I}_{\mc{S},1}} \le c_\kappa\int_{-1/2}^{1/2}\frac{\bars^2}{\abs{\bm{R}}^4}\norm{\bm{f}}_{C(\T)}d\bars
\le c_\kappa\norm{\bm{f}}_{C(\T)}\epsilon^{-1}.
\end{equation}

%The integrand of $\bm{I}_{\mc{S},1}$ satisfies
%\[\abs{\frac{(\bm{R}\cdot\be_\theta)}{\abs{\bm{R}}^5}\big( (\bm{R}\cdot\be_t)\bm{f}-(\bm{R}\cdot \bm{f})\be_t - (\be_t\cdot \bm{f})\bm{R} \big)} \le \frac{3 c_Q\bars^2}{\abs{\bm{R}}^4}\norm{\bm{f}}_{C(\T)}, \]
%where we used \eqref{Reps}. We thus have
%\begin{equation}\label{IS1stheta}
%\abs{\bm{I}_{\mc{S},1}} \le \int_{-1/2}^{1/2}\frac{9 c_Q\bars^2}{\abs{\bm{R}}^4}\norm{\bm{f}}_{C(\T)}d\bars
%\le c_\kappa\norm{\bm{f}}_{C(\T)}\epsilon^{-1},
%\end{equation}
%where we used Lemma \ref{Rintest0}. \\

Next, to estimate $\bm{I}_{\mc{S},2}$, we define $\bm{g}_1$ as in \eqref{g012def}. Using Lemma \ref{Rintest2} with $m=0, n=3$, we have
\begin{equation}\label{hsdef}
\begin{split}
\abs{\bm{I}_{\mc{S},2}-\frac{2}{\epsilon^2}\bm{h}(s)} &\le c_\kappa\norm{\bm{g}_1(\cdot;s)}_{C^1(\T)}\epsilon^{-1}
\le c_\kappa\norm{\bm{f}}_{C^1(\T)}\epsilon^{-1}; \\
\bm{h}(s)&=\bm{g}_1(0;s) = (\be_\theta(s,\theta)\cdot \bm{f}(s))\be_t(s)+(\be_t(s)\cdot \bm{f}(s))\be_\theta(s,\theta).
\end{split}
\end{equation}

To estimate $\bm{I}_{\mc{S},3}$, let
\begin{align*}
\bm{I}_{\mc{S},3}&=\bm{I}_{\mc{S},31}+\bm{I}_{\mc{S},32}; \\
\bm{I}_{\mc{S},31} &= -3\int_{-1/2}^{1/2}\frac{\bars^2(\bm{Q}\cdot\be_t)}{\abs{\bm{R}}^5} \big((\bm{R}\cdot\bm{f})\be_\theta+(\be_\theta\cdot\bm{f})\bm{R} \big) d\bars,\\
\bm{I}_{\mc{S},32}&= -3\int_{-1/2}^{1/2}\frac{\bars}{\abs{\bm{R}}^5} \big((\bm{R}\cdot\bm{f})\be_\theta+(\be_\theta\cdot\bm{f})\bm{R} \big) d\bars.
\end{align*}
Using Lemma \ref{Rintest0}, $\bm{I}_{\mc{S},31}$ may be estimated as
\[ \abs{\bm{I}_{\mc{S},31}}\le c_\kappa\int_{-1/2}^{1/2}\frac{\bars^2}{\abs{\bm{R}}^4}\norm{\bm{f}}_{C(\T)}d\bars \le c_\kappa\norm{\bm{f}}_{C(\T)}\epsilon^{-1}. \]

To estimate $\bm{I}_{\mc{S},32}$, define $\bm{g}_0$, $\bm{g}_1$, and $\bm{g}_2$ as in \eqref{g012def}. We first have
\[\abs{\int_{-1/2}^{1/2} \frac{\bars^3 \bm{g}_2}{\abs{\bm{R}}^5}d\bars} \le c_\kappa\int_{-1/2}^{1/2} \frac{|\bars|^3\norm{\bm{f}}_{C(\T)}}{\abs{\bm{R}}^5}d\bars \le c_\kappa\norm{\bm{f}}_{C(\T)}\epsilon^{-1}, \]
where we used \eqref{Reps} and Lemma \ref{Rintest0}. Next, we have
\[ \abs{\int_{-1/2}^{1/2} \frac{\bars^2 \bm{g}_1}{\abs{\bm{R}}^5}d\bars - \frac{2}{3\epsilon^2}\bm{h}(s)} \le c_\kappa\norm{\bm{g}_1(\cdot;s)}_{C^1(\T)}\epsilon^{-1} \le c_\kappa\norm{\bm{f}}_{C^1(\T)}\epsilon^{-1}, \]
where we used Lemma \ref{Rintest2} with $m=2, n=5$ and $\bm{h}(s)$ as defined in \eqref{hsdef}. For $\bm{g}_2$, we have
\[\abs{\int_{-1/2}^{1/2} \frac{\epsilon \bars \bm{g}_0}{\abs{\bm{R}}^5}d\bars} \le c_\kappa\norm{\bm{g}_0(\cdot;s)}_{C^1(\T)}\epsilon^{-1}\le c_\kappa\norm{\bm{f}}_{C^1(\T)}\epsilon^{-1}, \]
where we used Lemma \ref{Rintest1}. Combining the above estimates, we have
\begin{equation}\label{IS3stheta}
\abs{\bm{I}_{\mc{S},3}+\frac{2}{\epsilon^2}\bm{h}(s)}\le c_\kappa\norm{\bm{f}}_{C^1(\T)}\epsilon^{-1}.
\end{equation}

Finally, we estimate $\bm{I}_{\mc{S},4}$ as 
\begin{equation}\label{IS4stheta}
\abs{\bm{I}_{\mc{S},4}}\le c_\kappa\int_{-1/2}^{1/2}\frac{\bars^2}{\abs{\bm{R}}^4}\norm{\bm{f}}_{C(\T)}d\bars \le c_\kappa\norm{\bm{f}}_{C(\T)}\epsilon^{-1}.
\end{equation}

Using the estimates \eqref{IS1stheta}, \eqref{hsdef}, \eqref{IS3stheta} and \eqref{IS4stheta} in \eqref{ISstheta}, 
we obtain 
\begin{equation}\label{ISthetaest}
\abs{\frac{\p \bm{I}_{\mc{S}}}{\p \theta}}\le c_\kappa\norm{\bm{f}}_{C^1(\T)}.
\end{equation}

We may estimate $\partial \bm{I}_{\mc{D}}/\partial \theta$ in exactly the same way. We have
\begin{equation}\label{IDstheta}
\begin{split}
\frac{\p \bm{I}_{\mc D}}{\p \theta} &= \epsilon \big(\bm{I}_{\mc{D},1}+\bm{I}_{\mc{D},2} +\bm{I}_{\mc{D},3}+\bm{I}_{\mc{D},4}\big); \\
\bm{I}_{\mc{D},1} &= 5\int_{-1/2}^{1/2}\bigg(\frac{(\bm{R}\cdot\bm{e}_\theta)}{\abs{\bm{R}}^7} \big((\bm{R}\cdot\be_t)\bm{f}+(\bm{R}\cdot \bm{f})\be_t +(\be_t\cdot \bm{f})\bm{R} \big) \bigg) d\bars, \\
\bm{I}_{\mc{D},2} &= -\int_{-1/2}^{1/2}\frac{(\be_\theta\cdot \bm{f})\be_t +(\be_t\cdot \bm{f})\be_\theta}{\abs{\bm{R}}^5}d\bars,\\
\bm{I}_{\mc{D},3} &= 5\int_{-1/2}^{1/2}\frac{(\bm{R}\cdot\be_t)}{\abs{\bm{R}}^7} \big((\bm{R}\cdot\bm{f})\be_\theta+(\be_\theta\cdot\bm{f})\bm{R} \big) d\bars,\\
\bm{I}_{\mc{D},4}&=-35\int_{1/2}^{1/2} \frac{(\bm{R}\cdot\be_t)(\bm{R}\cdot\bm{f})(\bm{R}\cdot\be_\theta)}{\abs{\bm{R}}^9}\bm{R} \ts d\bars.
\end{split}
\end{equation}

The estimation of $\bm{I}_{\mc{D},1}$ follows the same pattern as that for $\bm{I}_{\mc{S},1}$ obtained in \eqref{IS1stheta}:
\[\abs{\bm{I}_{\mc{D},1}}\le c_\kappa\norm{f}_{C(\T)}\epsilon^{-3}. \]

The estimation of $\bm{I}_{\mc{D},2}$ is similar to \eqref{hsdef}:
\[ \abs{\bm{I}_{\mc{D},2}+\frac{4}{3\epsilon^4}\bm{h}(s)}\le c_\kappa\norm{\bm{f}}_{C^1(\T)}\epsilon^{-3}, \]
where we used Lemma \ref{Rintest2} with $m=0, n=5$. We estimate $\bm{I}_{\mc{D},3}$ following the steps of estimate \eqref{IS3stheta}. We obtain
\[ \abs{\bm{I}_{\mc{D},3}-\frac{4}{3\epsilon^4}\bm{h}(s)}\le c_\kappa\norm{\bm{f}}_{C^1(\T)}\epsilon^{-3}, \]
where we used Lemma \ref{Rintest2} with $m=2, n=7$. Finally, the estimation of $\bm{I}_{\mc{D},4}$ is similar to \eqref{IS4stheta}, yielding
\[ \abs{\bm{I}_{\mc{D},4}}\le c_\kappa\norm{\bm{f}}_{C(\T)}\epsilon^{-3}. \]
Combining the above estimates, we obtain
\begin{equation}\label{IDthetaest}
\abs{\frac{\p \bm{I}_{\mc{D}}}{\p \theta}}\le c_\kappa\norm{\bm{f}}_{C^1(\T)}\epsilon^{-2}.
\end{equation}

Combining \eqref{ISthetaest} and \eqref{IDthetaest} and recalling the definition of $\bm{I}^{\SB}$ in \eqref{uSBstheta}, we have
\begin{equation}\label{CItheta}
\abs{\frac{\p \bm{I}^{\SB}}{\p \theta}}\le \abs{\frac{\p \bm{I}_{\mc S}}{\p \theta}} +\frac{3\epsilon^2}{2}\abs{\frac{\p \bm{I}_{\mc D}}{\p\theta}} \le c_\kappa \norm{\bm{f}}_{C^1(\T)}.
\end{equation}
We may finally use \eqref{CI} and \eqref{CItheta} together in \eqref{uSBstheta} to obtain
\begin{equation}\label{mixed_est}
\begin{split}
\abs{\frac{\p}{\p \theta}\bigg(\frac{\p \bu^{\SB}}{\p s}-\kappa_3\frac{\p \bu^{\SB}}{\p \theta} \bigg) }
&\le \frac{1}{8\pi}\bigg((1+\epsilon|\wh\kappa|)\abs{\frac{\p \bm{I}^{\SB}}{\p \theta}}+\epsilon\abs{\frac{\p \wh{\kappa}}{\p \theta}}\abs{\bm{I}^{\SB}} \bigg) \le c_\kappa \norm{\bm{f}}_{C^1(\T)},
\end{split}
\end{equation}
where, in the last inequality, we used that
\[ \epsilon\abs{\wh\kappa}\le 2\epsilon\kappa_{\max} \le \frac{1}{4}, \]
by \eqref{kappahat} and \eqref{slender_body}, and
\[\abs{\frac{\p \wh{\kappa}}{\p \theta}} = \abs{-\kappa_1\sin \theta+\kappa_2\cos\theta} \le 2\sqrt{\kappa_1^2+\kappa_2^2} = 2\kappa\le  2\kappa_{\max}, \]
by \eqref{kappahat} and \eqref{kappa12}. 
\end{proof}

With Propositions \ref{prop:uSBtheta} and \ref{prop:uSBstheta}, we are finally equipped to estimate the degree to which $\bu^{\SB}$ fails to satisfy the $\theta$-independence condition along $\Gamma_{\epsilon}$. We define the residual $\bu^{\rm r}(s,\theta)$ as 
\begin{equation}\label{ur}
\bu^{\rm r}(\theta,s) = \bu^{\rm SB}(\epsilon,\theta,s) - \frac{1}{2\pi}\int_0^{2\pi} \bu^{\SB}(\epsilon,\varphi,s) \ts d\varphi.
\end{equation}
Note that the function $\bu^{\rm r}$ measures the deviation of $\bu^{\SB}$ from a $\theta$-independent function. We show the following estimates for $\bu^{\rm r}$.
\begin{proposition}\label{ur_and_derivs}
Consider the residual $\bu^{\rm r}$ defined in \eqref{ur}. For sufficiently small $\epsilon$, we have 
\begin{align}
\label{urest}
\abs{\bu^{\rm r}}&\le c_\kappa \norm{\bm{f}}_{C^1(\T)}\epsilon\abs{\log\epsilon},\\
\label{urtheta}
\abs{\frac{1}{\epsilon}\frac{\p \bu^{\rm r}}{\p \theta}}&\le c_\kappa \norm{\bm{f}}_{C^1(\T)}\abs{\log\epsilon},\\
\label{urs}
\abs{\frac{\p \bu^{\rm r}}{\p s}}&\le c_\kappa \norm{\bm{f}}_{C^1(\T)},
\end{align}
where the constants $c_\kappa$ depend only on $c_\Gamma$ and $\kappa_{\max}$.
\end{proposition}

Note that the estimate \eqref{urest} provides a rigorous proof of the asymptotic calculations done by Johnson in \cite{johnson1980improved}. 

\begin{proof}
Let $\bu^{\rm r}=(u^{\rm r}_1,u^{\rm r}_2, u^{\rm r}_3)$ and likewise for $\bm{u}^{\rm SB}$. We work component-wise. For each fixed $s$, we can find $\theta_0$ satisfying
\[ u^{\SB}_k(\epsilon,\theta_0,s)=\frac{1}{2\pi}\int_0^{2\pi} u^{\SB}_k(\epsilon,\varphi,s) \ts d\varphi. \]
Thus we can write
\[ u^{\rm r}_k(\theta,s)=u^{\SB}_k(\epsilon,\theta,s) - u^{\SB}_k(\epsilon,\theta_0,s) = \int_{\theta_0}^\theta \frac{\p u^{\SB}_k}{\p \theta}(\epsilon,\varphi,s)\ts d\varphi. \]
Using Proposition \ref{prop:uSBtheta}, we have
\begin{equation}
\begin{split}
\abs{u^{\rm r}_k(\theta,s)}&\le \int_{\theta_0}^\theta \abs{\frac{\p u^{\SB}_k}{\p \theta}(\epsilon,\varphi,s)} \ts d\varphi 
\le c_\kappa\abs{\theta-\theta_0}\norm{\bm{f}}_{C^1(\T)}\epsilon \abs{\log\epsilon}\\
&\le c_\kappa\pi\norm{\bm{f}}_{C^1(\T)}\epsilon \abs{\log\epsilon},
\end{split}
\end{equation}
where, in the last equality, we used the fact that $\theta$ and $\theta_0$ are at most $\pi$ apart. This establishes \eqref{urest}. \\

The estimate \eqref{urtheta} is a direct consequence of Proposition \ref{prop:uSBtheta}. \\

We finally establish \eqref{urs}. For each fixed $s$, we find a $\theta_1$ satisfying
\[ \frac{\p u^{\SB}_k}{\p s}(\epsilon,\theta_1,s) = \frac{1}{2\pi}\int_{0}^{2\pi} \frac{\p u^{\SB}_k}{\p s}(\epsilon,\varphi,s)\ts d\varphi. \]
Then we can write
\begin{align*}
\frac{\p u^{\rm r}_k}{\p s}(\theta,s)&=\frac{\p u^{\SB}_k}{\p s}(\epsilon,\theta,s) - \frac{\p u^{\SB}_k}{\p s}(\epsilon,\theta_1,s)
=\int_{\theta_1}^{\theta} \frac{\p}{\p \theta}\bigg(\frac{\p u^{\SB}_k}{\p s}\bigg)(\epsilon,\varphi,s)\ts d\varphi\\
&=\int_{\theta_1}^{\theta} \frac{\p }{\p \theta}\bigg(\frac{\p u^{\SB}_k}{\p s}-\kappa_3\frac{\p u^{\SB}_k}{\p \theta} \bigg)(\epsilon,\varphi,s) \ts d\varphi\\
&\quad +\kappa_3\bigg(\frac{\p u^{\SB}_k}{\p \theta}(\epsilon,\theta,s) - \frac{\p u^{\SB}_k}{\p \theta}(\epsilon,\theta_1,s) \bigg)
\end{align*}
Thus, using Proposition \ref{prop:uSBstheta} and Proposition \ref{prop:uSBtheta}, we have
\begin{equation}
\abs{\frac{\p u^{\rm r}_k}{\p s}(\theta,s)} \le c_\kappa\abs{\theta-\theta_1}\norm{\bm{f}}_{C^1(\T)} + 2\abs{\kappa_3} c_\kappa\norm{\bm{f}}_{C^1(\T)}\epsilon\abs{\log\epsilon}.
\end{equation}
Noting that $\abs{\theta-\theta_1}\le \pi$ and $\abs{\kappa_3}\leq \pi$ by Lemma \ref{lemmaorthonormal}, we obtain the desired estimate.
\end{proof}

%%%%%%%%
Finally, using Lemma \ref{center_est_lem}, we show the following residual estimate for the difference $\bu^{\SB}(s,\theta)-\bu^{\SB}_C(s)$ between the slender body approximation \eqref{stokes_SB} on the fiber surface and the asymptotic centerline expression \eqref{SBT_asymp}.
\begin{proposition}\label{centerline_prop}
Let $\bu^{\SB}(s,\theta)$ be \eqref{stokes_SB} evaluated on the slender body surface $\Gamma_\epsilon$, and let $\bu^{\SB}_C(s)$ be the centerline equation \eqref{SBT_asymp}. Then the difference $\bu^{\SB}(s,\theta)-\bu^{\SB}_C(s)$ satisfies
\begin{equation}\label{centerline_resid}
\abs{\bu^{\SB}(s,\theta)-\bu^{\SB}_C(s)} \le c_\kappa\epsilon\abs{\log\epsilon}\norm{\bm{f}}_{C^1(\T)},
\end{equation}
where $c_\kappa$ depends only on $c_\Gamma$ and $\kappa_{\max}$.
\end{proposition}

\begin{proof}
We begin by writing the Stokeslet term of $\bu^{SB}(s,\theta)$ as
\begin{equation}
\begin{aligned}
\int_{-1/2}^{1/2}&\mc{S}(\bm{R})\bm{f}(s+\bars)d\bars = \mc{S}_1 + \mc{S}_2; \\
\mc{S}_1 &:= \int_{-1/2}^{1/2} \frac{\bm{f}(s+\bars)}{\abs{\bm{R}}} d\bars, \quad \mc{S}_2 := \int_{-1/2}^{1/2} \frac{\bm{R}\bm{R}^{\rm T}}{\abs{\bm{R}}^3} \bm{f}(s+\bars) d\bars.
\end{aligned}
\end{equation}

Now, letting 
\begin{equation}\label{cent_int1}
\bm{J}_{\mc{S},1} = \int_{-1/2}^{1/2} \bigg(\frac{ \bm{f}(s+\bars)}{\abs{\bm{R}_0}}- \frac{\bm{f}(s)}{\abs{\bars}} \bigg) d\bars - \bm{f}(s)\log(\epsilon^2),
\end{equation}
a direct application of Lemma \ref{center_est_lem} yields
\begin{align*}
\abs{\mc{S}_1 - \bm{J}_{\mc{S},1}} &\le \epsilon \abs{\log\epsilon}c_\kappa \norm{\bm{f}}_{C^1(\T)}.
\end{align*}

Furthermore, letting 
\begin{equation}\label{cent_int2}
\bm{J}_{\mc{S},2} = \int_{-1/2}^{1/2} \bigg(\frac{ \bm{R}_0\bm{R}_0^{\rm T}}{\abs{\bm{R}_0}^3}\bm{f}(s+\bars) - \frac{\be_t(s)\be_t(s)^{\rm T}}{\abs{\bars}}\bm{f}(s) \bigg) d\bars - \big[\log(\epsilon^2) + 2\big]\be_t(\be_t\cdot\bm{f}(s))
\end{equation}
and using \eqref{Reps} and \eqref{CQ} along with Lemma \ref{center_est_lem}, we have
 \begin{align*}
\bigg|\mc{S}_2- \bm{J}_{\mc{S},2} - \int_{-1/2}^{1/2} \frac{\epsilon^2\be_\rho\be_{\rho}^{\rm T}}{\abs{\bm{R}}^3} \bm{f}(s+\bars)d\bars \bigg|  &\le c_\kappa\epsilon\abs{\log\epsilon}\norm{\bm{f}}_{C^1(\T)} + c_\kappa\int_{-1/2}^{1/2} \frac{\epsilon\bars^2+\epsilon \abs{\bars}}{\abs{\bm{R}}^3}\abs{\bm{f}}d\bars \\
&\qquad + c_\kappa\norm{\bm{f}}_{C(\T)}\int_{-1/2}^{1/2} \big(|\bars|^3 +\bars^4 \big)\bigg|\frac{1}{\abs{\bm{R}}^3} - \frac{1}{\abs{\bm{R}_0}^3} \bigg| d\bars  \\
&\le c_\kappa\epsilon\abs{\log\epsilon}\norm{\bm{f}}_{C^1(\T)} \\
&\qquad + c_\kappa\norm{\bm{f}}_{C(\T)}\int_{-1/2}^{1/2}\frac{\epsilon^2+\epsilon\bars^2+ \epsilon^2\abs{\bars}+\epsilon|\bars|^3}{\bars^2+\epsilon^2} d\bars \\
&\le c_\kappa\epsilon\abs{\log\epsilon}\norm{\bm{f}}_{C^1(\T)},
\end{align*}
where we have used Lemmas \ref{Rintest0} and \ref{Rintest1} in the second inequality, and \eqref{IR_def}, \eqref{Rlb}, and \eqref{non_intersecting} in the third inequality. By Lemma \ref{Rintest2}, we then have
\begin{align*}
\big|\mc{S}_2- \bm{J}_{\mc{S},2} - 2\be_\rho(\be_{\rho}\cdot\bm{f}(s)) \big|  &\le c_\kappa\epsilon\abs{\log\epsilon}\norm{\bm{f}}_{C^1(\T)}.
\end{align*}

Together, the Stokeslet terms satisfy 
\begin{equation}\label{stokeslet_terms}
\abs{\int_{-1/2}^{1/2}\mc{S}(\bm{R})\bm{f}(s+\bars)d\bars - \bm{J}_{\mc{S},1}-\bm{J}_{\mc{S},2}  - 2\be_\rho(\be_{\rho}\cdot\bm{f}(s))} \le c_\kappa\epsilon\abs{\log\epsilon}\norm{\bm{f}}_{C^1(\T)}.
\end{equation}

%%%%%%%

Similarly, we may write the doublet term of \eqref{stokes_SB} as
\begin{equation}
\begin{aligned}
\int_{-1/2}^{1/2}&\mc{D}(\bm{R})\bm{f}(s+\bars)d\bars = \mc{D}_1 + \mc{D}_2; \\
\mc{D}_1&:= \int_{-1/2}^{1/2} \frac{\bm{f}(s+\bars)}{\abs{\bm{R}}^3} \ts d\bars, \quad \mc{D}_2 := -3\int_{-1/2}^{1/2} \frac{\bm{R}\bm{R}^{\rm T}}{\abs{\bm{R}}^5} \bm{f}(s+\bars) \ts d\bars.
\end{aligned}
\end{equation}

Using Lemma \ref{Rintest2}, we have
\begin{align*}
\abs{\mc{D}_1- \epsilon^{-2}2 \bm{f}(s)} \le c_\kappa \epsilon^{-1}\norm{\bm{f}}_{C^1(\T)}.
\end{align*}

Furthermore, using \eqref{Reps} along with Lemma \ref{Rintest0}, the second term $\D_2$ satisfies 
\begin{align*}
\abs{\mc{D}_2+ \epsilon^{-2}(2\be_t\be_t^{\rm T}+4\be_\rho\be_\rho^{\rm T})\bm{f}(s)} &\le \abs{3\int_{-1/2}^{1/2} \frac{\bars^2\be_t\be_t^{\rm T}}{\abs{\bm{R}}^5} \bm{f}(s+\bars)d\bars + \epsilon^{-2}2\be_t(\be_t\cdot\bm{f}(s))} \\
&\quad +\abs{3\epsilon^2\int_{-1/2}^{1/2} \frac{\be_\rho\be_\rho^{\rm T}}{\abs{\bm{R}}^5} \bm{f}(s+\bars)d\bars + \epsilon^{-2}4\be_\rho(\be_\rho\cdot\bm{f}(s))} + c_\kappa \epsilon^{-1}\norm{\bm{f}}_{C(\T)} \\
&\le  c_\kappa \epsilon^{-1}\norm{\bm{f}}_{C^1(\T)}, 
\end{align*}
where we have used Lemma \ref{Rintest2} in the second inequality. Letting
\begin{equation}\label{cent_int3}
\bm{J}_{\mc{D},1} = ({\bf I}-\be_t\be_t^{\rm T})\bm{f}(s),
\end{equation}
the doublet terms together yield
\begin{equation}\label{doublet_terms}
\abs{\frac{\epsilon^2}{2}\int_{-1/2}^{1/2}\mc{D}(\bm{R})\bm{f}(s+\bars)d\bars -\bm{J}_{\mc{D},1}  + 2\be_\rho(\be_\rho\cdot\bm{f}(s))} \le  c_\kappa \epsilon\norm{\bm{f}}_{C^1(\T)}. 
\end{equation}

Combining \eqref{stokeslet_terms} and \eqref{doublet_terms}, we obtain the following estimate for $\bu^{\SB}$ along $\Gamma_\epsilon$:
\begin{equation}\label{uSB_center0}
\abs{\bu^{\SB}(s,\theta) - \bm{J}_{\mc{S},1}- \bm{J}_{\mc{S},2}- \bm{J}_{\mc{D},1}} \le c_\kappa\epsilon\abs{\log\epsilon}\norm{\bm{f}}_{C^1(\T)}.
\end{equation}

Now, recalling the periodic expression \eqref{SBT_asymp} for $\bu^{\SB}_C(s)$ as well as the identity
\[ \int_{-1/2}^{1/2}\bigg(\frac{1}{\abs{\sin(\pi\bars)/\pi}}-\frac{1}{\abs{\bars}} \bigg)d\bars = 2\log(4/\pi),\]
we notice that
\begin{align*}
\bu^{\SB}_C(s)& - \bm{J}_{\mc{S},1}- \bm{J}_{\mc{S},2}- \bm{J}_{\mc{D},1} \\
&= -({\bf I}+\be_t\be_t^{\rm T})\bm{f}(s)\int_{-1/2}^{1/2}\bigg(\frac{1}{\abs{\sin(\pi\bars)/\pi}} -\frac{1}{\abs{\bars}} \bigg)d\bars + 2\log(4/\pi) ({\bf I}+\be_t\be_t^{\rm T})\bm{f}(s) =0,
\end{align*}
and therefore \eqref{uSB_center0} implies Proposition \ref{centerline_prop}.

\end{proof}

%%%%%%%%%%%%%%%%%%%%%%%%%%%%%%%%%%%%%%%%%%%%%%%%%%%%%%%%%%%%%
%%%%%%%%%%%%%%%%%%%%%%%%%%%%%%%%%%%%%%%%%%%%%%%%%%%%%%%%%%%%%
%%%%%%%%%%%%%%%%%%%%%%%%%%%%%%%%%%%%%%%%%%%%%%%%%%%%%%%%%%%%%

%%%%%%%%%%%%%%%%%%%%%%%%%%%%%%%%%%%%%%%%%%%%%%%%%%%%%%%%%%%%%%%%
\subsection{Slender body force residual}\label{SBforce_res}
It remains to calculate the slender body approximation to the total force at each cross section $s\in \T$, given by
\begin{equation}\label{fSB_expr}
{\bm f}^{\SB}(s)= \int_0^{2\pi}\bigg(-p^{\SB}{\bf I}+2\E(\bu^{\SB}) \bigg) {\bm n} \ts \mc{J}_\epsilon(s,\theta)\ts d\theta.
\end{equation}

The estimation of the slender body force expression \eqref{fSB_expr} will proceed similarly to the calculations for the velocity residual in the previous section, relying on Lemmas \ref{Rintest0} - \ref{Rintest2} to bound the resulting integral terms. Because of the structure of \eqref{fSB_expr}, we will also be able to use a stronger bound (Lemma \ref{theta_int}) relying on $\theta$ integration to remove the $\log\epsilon$ dependence in the force residual estimate.  \\

From \eqref{fSB_expr}, calculating the slender body force requires two main components: the force due to the slender body pressure \eqref{SB_pressure} and the force due to the surface strain rate $\E(\bu^{\SB})\bm{n}\big|_{\Gamma_\epsilon}$. Recalling that ${\bm n}=-\be_\rho$, we can express the surface strain rate with respect to the moving frame basis $\be_t(s)$, $\be_\rho(s,\theta)$, $\be_\theta(s,\theta)$ as
\begin{equation}\label{SB_strain}
\begin{aligned}
2\E(\bu){\bm n} &= -\frac{\p\bu}{\p\rho} -\left(\frac{\p \bu}{\p \rho}\cdot\be_{\rho}\right)\be_{\rho} - \frac{1}{\epsilon}\left(\frac{\p \bu}{\p \theta}\cdot\be_{\rho} \right)\be_{\theta}- \frac{1}{1-\epsilon\wh\kappa}\left(\left(\frac{\p \bu}{\p s}-\kappa_3\frac{\p \bu}{\p\theta}\right)\cdot \be_{\rho}\right) \be_{t}.
\end{aligned}
\end{equation}

%Evaluating the slender body surface strain at $\rho=\epsilon$, we have that each of the terms in the expression \eqref{SB_strain} can be written as
%\begin{equation}\label{CL_uSB_derivs}
%\begin{aligned}
%\frac{\p \bu^{\SB}}{\p\rho} &= \frac{1}{8\pi}\int_{\T} \bigg[\frac{\widehat\bx\cdot\be_{\rho}+\epsilon}{|\bm{R}|^3}{\bm f}(t)- \frac{\be_{\rho}\bm{R}^{\rm T}+\bm{R}\be_{\rho}^{\rm T}}{|\bm{R}|^3}{\bm f}(t) +\frac{3\bm{R}\bm{R}^{\rm T}(\widehat\bx\cdot\be_{\rho}+\epsilon)}{|\bm{R}|^5}{\bm f}(t) \\
%&\hspace{2cm} +\frac{3\epsilon^2}{2}\bigg(\frac{\widehat\bx\cdot\be_{\rho}+\epsilon }{|\bm{R}|^5}{\bm f}(t)+ \frac{\be_{\rho}\bm{R}^{\rm T}+\bm{R}\be_{\rho}^{\rm T}}{|\bm{R}|^5}{\bm f}(t) -\frac{5\bm{R}\bm{R}^{\rm T}(\widehat\bx\cdot\be_{\rho}+\epsilon )}{|\bm{R}|^7}{\bm f}(t) \bigg)\bigg]\ts dt, \\
%\frac{1}{\epsilon}\frac{\p \bu^{\SB}}{\p \theta} &= \frac{1}{8\pi}\int_{\T} \bigg[\frac{\widehat\bx\cdot\be_{\theta}}{|\bm{R}|^3}{\bm f}(t)- \frac{\be_{\theta}\bm{R}^{\rm T}+\bm{R}\be_{\theta}^{\rm T}}{|\bm{R}|^3}{\bm f}(t) +\frac{3\bm{R}\bm{R}^{\rm T}(\widehat\bx\cdot\be_{\theta})}{|\bm{R}|^5}{\bm f}(t) \\
%&\hspace{2cm} +\frac{3\epsilon^2}{2}\bigg(\frac{\widehat\bx\cdot\be_{\theta}}{|\bm{R}|^5}{\bm f}(t)+ \frac{\be_{\theta}\bm{R}^{\rm T}+\bm{R}\be_{\theta}^{\rm T}}{|\bm{R}|^5}{\bm f}(t) -\frac{5\bm{R}\bm{R}^{\rm T}(\widehat\bx\cdot\be_{\theta})}{|\bm{R}|^7}{\bm f}(t) \bigg)\bigg]\ts dt, \\
%\frac{1}{1-\epsilon\wh\kappa}&\left(\frac{\p \bu^{\SB}}{\p s}-\kappa_3\frac{\p \bu^{\SB}}{\p\theta}\right) \\
%&= \frac{1}{8\pi}\int_{\T} \bigg[\frac{\widehat\bx\cdot\be_t}{|\bm{R}|^3}{\bm f}(t) - \frac{\be_t\bm{R}^{\rm T}+\bm{R}\be_t^{\rm T}}{|\bm{R}|^3}{\bm f}(t) +\frac{3\bm{R}\bm{R}^{\rm T}(\widehat\bx\cdot\be_t)}{|\bm{R}|^5}{\bm f}(t) \\
%&\hspace{2cm} +\frac{3\epsilon^2}{2}\bigg(\frac{\widehat\bx\cdot\be_t}{|\bm{R}|^5}{\bm f}(t) + \frac{\be_t\bm{R}^{\rm T}+\bm{R}\be_t^{\rm T}}{|\bm{R}|^5}{\bm f}(t) -\frac{5\bm{R}\bm{R}^{\rm T}(\widehat\bx\cdot\be_t)}{|\bm{R}|^7}{\bm f}(t) \bigg)\bigg]\ts dt. \\
%\end{aligned}
%\end{equation}
%%%%%%%%%%%%%%%%

\begin{remark} 
Before we estimate $\bm{f}^{\SB}$, we consider the (purely heuristic) slender body approximation about an infinitely long fiber with a straight centerline and constant total force $\bm{f}^c$ over each cross section. In this case, although the slender body velocity approximation diverges logarithmically at infinity, the velocity does exactly satisfy the $\theta$-independence condition on the the slender body surface due to the doublet correction with coefficient $\frac{\epsilon^2}{2}$. This is essentially the scenario for which slender body theory is designed to work.  \\
 
Indeed, in the straight centerline/constant force scenario, the slender body force expression \eqref{fSB_expr} also exactly recovers the prescribed force $\bm{f}^c$. When $\kappa\equiv0$, we have $\bm{R}= (s-t)\be_t+\epsilon\be_{\rho}(\theta)$, where the basis vectors no longer depend on the cross section $s$. We can then directly integrate the slender body approximation \eqref{SBT2} in $t$ to obtain: 
\begin{equation}\label{straight_center}
-\frac{\p \bu^{\SB}_{\text{str}}}{\p\rho} = \frac{1}{\epsilon2\pi} \bigg[\bm{f}^c - \be_{\rho}(\be_{\rho}\cdot\bm{f}^c)\bigg], \quad \bigg(\frac{\p \bu^{\SB}_{\text{str}}}{\p\rho}\cdot\be_{\rho}\bigg)\be_{\rho}  = 0, \quad
\frac{1}{\epsilon}\frac{\p \bu^{\SB}_{\text{str}}}{\p \theta} =0, \quad
\frac{\p \bu^{\SB}_{\text{str}}}{\p s} = 0.
\end{equation}
Additionally, the slender body pressure contribution to the total force is given by
\begin{equation}\label{pressure_exact}
\begin{aligned}
p^{\SB}_{\text{str}}(s,\theta) &= \frac{1}{ 2\pi \epsilon} \be_{\rho}\cdot \bm{f}^c.
\end{aligned}
\end{equation}

Thus the slender body approximation to the constant force $\bm{f}^c$ prescribed along an infinite straight cylinder is given by
\begin{equation}
\begin{aligned}
{\bm f}^{\SB}_{\text{str}} &= \int_0^{2\pi} \bigg[-p^{\SB}_{\text{str}}{\bm n} + 2\E(\bu^{\SB}_{\text{str}}){\bm n} \bigg] \epsilon \ts d\theta \\
&= \int_0^{2\pi} \bigg[\frac{1}{2\pi} (\be_{\rho}\cdot\bm{f}^c)\be_{\rho} + \frac{1}{2\pi}\big(\bm{f}^c - \be_{\rho}(\be_{\rho}\cdot\bm{f}^c)\big) \bigg] \ts d\theta \\
&= \int_0^{2\pi} \frac{1}{2\pi}\bm{f}^c \ts d\theta = \bm{f}^c,
\end{aligned}
\end{equation}
so we exactly recover the force $\bm{f}^c$ at each cross section along the fiber. \\

Again, the straight centerline/constant force calculations are purely heuristic, but serve to show that the error in the slender body approximation to the total force, as well as the $\theta$-dependence in the slender body surface velocity, will arise due to the curvature of the fiber centerline, the finite fiber length, and variations in the prescribed force along the centerline. \\
\end{remark}


%%%%%%%%%%%%%%%%%%%%%%%%%%%%%%%%%%%%%%%%%%%%%%%%%%%%%%%%%%%%%%%%
%%%%%%%%%%%%%%%%%%%%%%%%%%%%%%%%%%%%%%%%%%%%%%%%%%%%%%%%%%%%%%%%
%%%%%%%%%%%%%%%%%%%%%%%%%%%%%%%%%%%%%%%%%%%%%%%%%%%%%%%%%%%%%%%%

Given a curved centerline and non-constant prescribed force ${\bm f}(s)$, we compute the slender body approximation to the force, ${\bm f}^{\SB}(s)$ using essentially the same perturbative argument as in the velocity estimation, where we relied on the straight centerline integrand to derive integral bounds for the curved centerline. \\

Although Lemmas \ref{Rintest0}, \ref{Rintest1}, and \ref{Rintest2} are actually enough to obtain an $O(\epsilon\abs{\log\epsilon})$ bound on the residual $\bm{f}^{\SB}-\bm{f}$, it turns out that we can use the $\theta$-integration in the slender body force expression \eqref{fSB_expr} to obtain a slightly stronger $O(\epsilon)$ bound. In particular, for $m=n+2$, we can improve upon the $\abs{\log\epsilon}$ bound guaranteed by Lemma \ref{Rintest1} by relying on cancellation upon integration in $\theta$, rather than symmetry cancellation due to $m$ being odd. For $m=n+1$, we gain an additional $\epsilon$ factor over the Lemma \ref{Rintest0} bound.

\begin{lemma}\label{theta_int}
Let $\bm{R}$ be as in \eqref{Reps}. Suppose $m$ is a non-negative integer and $n= m+1$ or $m+2$. Furthermore, assume $g\in C(\T)$. For $k\in \Z$, $k\neq 0$, $\theta_0\in \R$ and $\epsilon>0$ sufficiently small, we have
\begin{equation}\label{theta_int_eq}
\abs{\int_0^{2\pi}\int_{-1/2}^{1/2}\frac{\bars^m g(\bars)}{\abs{\bm{R}}^n}\cos(k(\theta+\theta_0)) \ts d\bars d\theta}
\le \begin{cases}
c_\kappa \epsilon\abs{\log\epsilon}\norm{g}_{C(\T)}, & n=m+1, \\
c_\kappa \norm{g}_{C(\T)}, & n=m+2,
\end{cases} 
\end{equation}
where the constants $c_\kappa$ depend only on $c_\Gamma, \kappa_{\max}$, and $n$. 
\end{lemma}

\begin{remark}\label{theta_int_rmk}
Note that by plugging in the correct values of $k$ and $\theta_0$, Lemma \ref{theta_int} also covers integrands of the form $\bars^m g(\bars)/\abs{\bm{R}}^n$ integrated against $\sin\theta$ or agains odd triples $\sin^j\theta\cos^k\theta$, $k+j=3$, $k,j\ge0$, via the trigonometric identities  
\begin{align*}
\cos^3\theta &= \frac{1}{4}(3 \cos\theta + \cos(3\theta)), \quad \sin\theta\cos^2\theta = \frac{1}{4}(\sin\theta + \sin(3 \theta)), \\
\sin^3\theta &= \frac{1}{4}(3 \sin\theta - \sin(3 \theta)), \quad \sin^2\theta\cos\theta = \frac{1}{4}(\cos\theta - \cos(3\theta)).
\end{align*}

Note in particular that Lemma \ref{theta_int} applies to integrands of the form $\frac{\bars^m}{\abs{\bm{R}}^n}\be_\rho(\bm{A}(\bars)\cdot\be_\rho)(\bm{B}(\bars)\cdot\be_\rho)$ and $\frac{\bars^m}{\abs{\bm{R}}^n}\be_\theta(\bm{A}\cdot\be_\rho)(\bm{B}\cdot\be_\theta)$, where $\bm{A}=(a_1,a_2,a_3)^{\rm T}$ and $\bm{B}=(b_1,b_2,b_3)^{\rm T}$ are vector-valued functions that do not depend on $\theta$. We can expand these quantities as
\begin{align*}
\be_\rho(\bm{A}\cdot\be_\rho)(\bm{B}\cdot\be_\rho) &= \big(a_2b_2\cos^3\theta + (a_2b_3+b_2a_3)\cos^2\theta\sin\theta+b_3a_3\sin^2\theta\cos\theta \big)\be_{n_1}(s) \\
&\qquad+\big(a_3b_3\sin^3\theta + (a_2b_3+b_2a_3)\sin^2\theta\cos\theta+b_2a_2\cos^2\theta\sin\theta \big)\be_{n_2}(s), \\
%
\be_\theta(\bm{A}\cdot\be_\rho)(\bm{B}\cdot\be_\theta) &= \big(a_3b_2\sin^3\theta + (a_2b_2-b_3a_3)\sin^2\theta\cos\theta-b_3a_2\cos^2\theta\sin\theta \big)\be_{n_1}(s) \\
&\qquad+\big(a_2b_3\cos^3\theta + (a_3b_3-b_2a_2)\sin\theta\cos^2\theta-b_2a_3\cos\theta\sin^2\theta \big)\be_{n_2}(s),
\end{align*}
and, using the above trigonometric identities, apply Lemma \ref{theta_int} to each term.
\end{remark}

%%%%%%%

\begin{proof}[Proof of Lemma \ref{theta_int}]
First note that, for $\bm{R}_0(s,\bars)=\X(s) - \X(s+\bars)$, we may write
\begin{equation}
\begin{aligned}
I &=\int_0^{2\pi}\int_{-1/2}^{1/2}\frac{\bars^m g(\bars)}{\abs{\bm{R}}^n}\cos(k(\theta+\theta_0))d\bars \ts d\theta \\
&= \int_0^{2\pi}\int_{-1/2}^{1/2}\bigg(\frac{1}{\abs{\bm{R}}^n}-\frac{1}{(\abs{\bm{R}_0}^2+\epsilon^2)^{n/2}} \bigg)\bars^m g(\bars)\cos(k(\theta+\theta_0))d\bars \ts d\theta,
\end{aligned}
\end{equation}
where we have used that the second term integrates to zero in $\theta$. Then
\begin{align*}
\abs{I} &\le \int_0^{2\pi}\int_{-1/2}^{1/2}\frac{\norm{g}_{C(\T)}\abs{\bars}^m\abs{\abs{\bm{R}}^2-(\abs{\bm{R}_0}^2+\epsilon^2)}}{\abs{\bm{R}}(\abs{\bm{R}_0}^2+\epsilon^2)^{1/2}(\abs{\bm{R}}+(\abs{\bm{R}_0}^2+\epsilon^2)^{1/2})} \sum_{j=0}^{n-1}\frac{1}{\abs{\bm{R}}^j(\abs{\bm{R}_0}^2+\epsilon^2)^{(n-1-j)/2}} \ts d\bars d\theta.
\end{align*}

Now, by \eqref{CQ} and \eqref{Reps}, we have
\[ \abs{\abs{\bm{R}}^2-\abs{\bm{R}_0}^2-\epsilon^2}=2\epsilon\bars^2\abs{\be_\rho\cdot\bm{Q}}\le \epsilon \kappa_{\max}\bars^2,\]
while by Lemma \ref{absRests} and \eqref{non_intersecting} we have
\[ \abs{\bm{R}}\ge c_\kappa\sqrt{\bars^2+\epsilon^2}, \quad \abs{\bm{R}_0}\ge c_\Gamma\abs{\bars}.\]

Thus
\begin{align*}
\abs{I}&\le c_\kappa \epsilon \norm{g}_{C(\T)}\int_0^{2\pi}\int_{-1/2}^{1/2}\frac{\abs{\bars}^{m+2}}{(\bars^2+\epsilon^2)^{(n+2)/2}} \ts d\bars d\theta \le \begin{cases}
 c_\kappa \epsilon\abs{\log\epsilon}\norm{g}_{C(\T)}, & n=m+1 \\
 c_\kappa \norm{g}_{C(\T)}, & n=m+2,
 \end{cases}
\end{align*}
by Lemma \ref{defints}.

\end{proof}


%%%%%%%%%%%%%%%%%%%%%%%%%%%%%%%%%%%%%%%%%%%%%%%%%%%%%%%%%%%%%%%%
%%%%%%%%%%%%%%%%%%%%%%%%%%%%%%%%%%%%%%%%%%%%%%%%%%%%%%%%%%%%%%%%
%%%%%%%%%%%%%%%%%%%%%%%%%%%%%%%%%%%%%%%%%%%%%%%%%%%%%%%%%%%%%%%%
%%%%%%%%%%%%%%%%%%%%%%%%%%%%%%%%%%%%%%%%%%%%%%%%%%%%%%%%%%%%%%%%
%%%%%%%%%%%%%%%%%%%%%%%%%%%%%%%%%%%%%%%%%%%%%%%%%%%%%%%%%%%%%%%%


We now proceed to estimate the slender body force \eqref{fSB_expr} for a fiber satisfying the geometric constraints of Section \ref{geometric_constraints} given a true force ${\bm f}(s)$ in $C^1(\T)$. Since the stress tensor $\bm{\sigma}^{\SB} = -p^{\SB}{\bf I}+ 2\E(\bu^{\SB})$ with $\E(\bu^{\SB})$ given by \eqref{SB_strain} essentially consists of five distinct terms, each of which in turn consists of derivatives of the slender body expression \eqref{SBT2}, it will be convenient to estimate each of the components of $\bm{f}^{\SB}$ separately. We label the five components of the $\bm{f}^{\SB}$ expression as follows. 
\begin{equation}\label{force_components}
\begin{aligned}
\bm{f}^{\SB} &= \bm{f}^{\SB}_p + \bm{f}^{\SB}_1 + \bm{f}^{\SB}_2 + \bm{f}^{\SB}_3 + \bm{f}^{\SB}_4; \\
\bm{f}^{\SB}_p &:= \int_0^{2\pi} -p^{\SB}\bm{n} \ts \mc{J}_\epsilon \ts d\theta \\ 
\bm{f}^{\SB}_1 &:= -\int_0^{2\pi}  \frac{\p\bu^{\SB}}{\p\rho} \ts \mc{J}_\epsilon d\theta \\ 
\bm{f}^{\SB}_2 &:=  -\int_0^{2\pi} \left(\frac{\p \bu^{\SB}}{\p \rho}\cdot\be_{\rho}\right)\be_{\rho} \ts \mc{J}_\epsilon d\theta  \\ 
\bm{f}^{\SB}_3 &:= -\int_0^{2\pi} \frac{1}{\epsilon}\left(\frac{\p \bu^{\SB}}{\p \theta}\cdot\be_{\rho} \right)\be_{\theta} \ts \mc{J}_\epsilon d\theta \\ 
\bm{f}^{\SB}_4 &:= -\int_0^{2\pi} \frac{1}{1-\epsilon\wh\kappa}\left(\left(\frac{\p \bu^{\SB}}{\p s}-\kappa_3\frac{\p \bu^{\SB}}{\p\theta}\right)\cdot \be_{\rho}\right) \be_{t}\ts \mc{J}_\epsilon d\theta
\end{aligned}
\end{equation}

%%%%%

We begin by estimating $\bm{f}^{\SB}_p$, the contribution of the slender body pressure $p^{\SB}$ to the total force. We show the following proposition:
\begin{proposition}\label{fSBp_est}
Let the slender body $\Sigma_\epsilon$ be as in Section \ref{geometric_constraints}. Given $\bm{f}\in C^1(\T)$, let $\bm{f}^{\SB}_p(s)$ be the pressure component of the slender body force, defined in \eqref{force_components}. Then $\bm{f}^{\SB}_p$ satisfies
 \begin{equation}
 \abs{\bm{f}^{\SB}_p(s)- \frac{1}{2} \big((\bm{f}(s)\cdot\be_{n_1}(s))\be_{n_1}(s) + (\bm{f}(s)\cdot\be_{n_2}(s))\be_{n_2}(s) \big)} \le \epsilon c_\kappa \norm{\bm{f}}_{C^1(\T)},
 \end{equation}
 where the constant $c_\kappa$ depends only on $c_{\Gamma}$ and $\kappa_{\max}$. 
\end{proposition}

\begin{proof}
 As in the velocity residual computation, we will view $\bm{R}=\bm{R}_0+\epsilon\be_\rho(s,\theta)$ as a function of $\theta$, $s$, and $\bars=-(s-t)$, rather than as a function of $\theta$, $s$, and $t$. Then, using the expression \eqref{SB_pressure} for the pressure, along with \eqref{Jeps_def} and \eqref{Reps}, we calculate
 \begin{equation}\label{fSBp}
\begin{aligned}
\bm{f}^{\SB}_p(s) &= \frac{1}{4\pi}\big(\bm{F}_1 +\bm{F}_2 +\bm{F}_3\big); \\
\bm{F}_1 &= \int_0^{2\pi}\int_{-1/2}^{1/2} \frac{\epsilon\be_{\rho}\cdot \bm{f}(s+\bars)}{\abs{\bm{R}}^3} \be_\rho \ts \epsilon \ts d\bars d\theta \\
\bm{F}_2 &= \int_0^{2\pi}\int_{-1/2}^{1/2} \frac{-\bars\be_t\cdot \bm{f}(s+\bars) + \bars^2\bm{Q}\cdot\bm{f}(s+\bars)}{\abs{\bm{R}}^3} \be_\rho \ts \epsilon\ts d\bars d\theta \\
\bm{F}_3 &= -\int_0^{2\pi}\int_{-1/2}^{1/2} \frac{\bm{R}\cdot \bm{f}(s+\bars)}{\abs{\bm{R}}^3} \be_\rho \ts \epsilon^2\wh\kappa \ts d\bars d\theta \\
\end{aligned}
\end{equation}

First note that, using Lemma \ref{Rintest0}, and recalling that $\abs{\wh\kappa}\le 2\kappa_{\max}$, we have that $\bm{F}_3$ satisfies 
\begin{align*}
\abs{\bm{F}_3} &\le 2\pi \norm{\bm{f}}_{C^(\T)} \int_{-1/2}^{1/2} \frac{1}{\abs{\bm{R}}^2} \epsilon^2\abs{\wh\kappa} \ts d\bars \le \epsilon c_\kappa \norm{\bm{f}}_{C(\T)}.
\end{align*}

Next we estimate $\bm{F}_2$. Recalling that $\be_{\rho}(s,\theta)= \cos\theta\be_{n_1}(s)+\sin\theta \be_{n_2}(s)$ while $\bm{f}(s+\bars)$, $\be_t(s)$, and $\bm{Q}(s,\bars)$ are all independent of $\theta$, we can use Lemma \ref{theta_int} to show 
\begin{align*}
\abs{\bm{F}_2} &\le \epsilon c_\kappa \norm{\bm{f}}_{C(\T)}.
\end{align*}

Finally, using Lemma \ref{Rintest2} with $m=0$ and $n=3$, we have that $\bm{F}_1$ satisfies
\begin{equation}\label{hf_def}
\begin{aligned}
\abs{\bm{F}_1 - 2\bm{h}_f(s)} &\le \epsilon c_\kappa\norm{\bm{f}}_{C^1(\T)}; \\
\bm{h}_f(s) &:= \int_0^{2\pi}\be_{\rho}(s,\theta)(\be_{\rho}(s,\theta)\cdot\bm{f}(s)) \ts d\theta \\
&= \pi \big((\bm{f}(s)\cdot\be_{n_1}(s))\be_{n_1}(s) + (\bm{f}(s)\cdot\be_{n_2}(s))\be_{n_2}(s) \big).
\end{aligned}
\end{equation}
 
 Combining these estimates, we obtain 
 \begin{equation}\label{fSBp_est0}
 \abs{\bm{f}^{\SB}_p(s)- \frac{1}{2\pi}\bm{h}_f(s)} \le \frac{1}{4\pi}\big(\abs{\bm{F}_1- 2\bm{h}_f(s)} +\abs{\bm{F}_2} +\abs{\bm{F}_3}\big) \le \epsilon c_\kappa\norm{\bm{f}}_{C^1(\T)}.
 \end{equation}
Recalling the definition of $\bm{h}_f(s)$ \eqref{hf_def}, we obtain Proposition \ref{fSBp_est}.
 \end{proof}

%%%%%%%%%%%%%%%%%%%%%%%%%%%%%%%%%%%%%%%%%%%%%%%%%%%%%%%%%%%%%%%%
%%%%%%%%%%%%%%%%%%%%%%%%%%%%%%%%%%%%%%%%%%%%%%%%%%%%%%%%%%%%%%%%
%%%%%%%%%%%%%%%%%%%%%%%%%%%%%%%%%%%%%%%%%%%%%%%%%%%%%%%%%%%%%%%%
We now proceed to estimate $\bm{f}^{\SB}_1(s)$, the next term in the expression \eqref{force_components} for $\bm{f}^{\SB}$. In particular, we show the following:
\begin{proposition}\label{fSB1_est}
Let $\bm{f}^{\SB}_1(s)$ be as defined in \eqref{force_components}. Then $\bm{f}^{\SB}_1$ satisfies 
\begin{equation}
\abs{\bm{f}^{\SB}_1 - \frac{1}{2}\bigg(\bm{f}(s) + (\bm{f}\cdot\be_t(s))\be_t(s)\bigg) } \le \epsilon c_\kappa\norm{\bm{f}}_{C^1(\T)}
\end{equation}
where the constant $c_\kappa$ depends only on $c_{\Gamma}$ and $\kappa_{\max}$.
\end{proposition}

\begin{proof}
Using the expression \eqref{force_components} for $\bm{f}^{\SB}_1(s)$ and recalling the slender body approximation \eqref{SBT2}, we consider $\bm{f}^{\SB}_1(s)$ as the sum of a Stokeslet and a doublet term. Again considering $\bm{R}$ as a function of $\theta$, $s$, and $\bars$, we can write
\begin{equation}\label{force_1}
\begin{aligned}
\bm{f}^{\SB}_1 &= \frac{1}{8\pi}\bigg(\bm{F}_{\mc{S},1} +\frac{\epsilon^2}{2}\bm{F}_{\mc{D},1}\bigg); \\
\bm{F}_{\mc{S},1} &:= -\int_0^{2\pi}\int_{-1/2}^{1/2} \frac{\p}{\p\rho}\mc{S}(\bm{R})\bm{f}(s+\bars) \ts d\bars \ts \epsilon(1-\epsilon\wh\kappa)d\theta \\
\bm{F}_{\mc{D},1}&:=  -\int_0^{2\pi}\int_{-1/2}^{1/2} \frac{\p}{\p\rho}\mc{D}(\bm{R})\bm{f}(s+\bars) \ts d\bars \ts \epsilon(1-\epsilon\wh\kappa) d\theta.
\end{aligned}
\end{equation}

We begin by estimating $\bm{F}_{\mc{S},1}$. Recalling the notation $\bm{R}_0(s,\bars):= \X(s)-\X(s+\bars)$, we have 
\begin{equation}\label{FS1}
\begin{aligned}
\bm{F}_{\mc{S},1} &= \bm{F}_{\mc{S},11} + \bm{F}_{\mc{S},12} + \bm{F}_{\mc{S},13}+ \bm{F}_{\mc{S},14}; \\
\bm{F}_{\mc{S},11}&= \int_0^{2\pi}\int_{-1/2}^{1/2} \bigg[\frac{\epsilon{\bm f}}{|\bm{R}|^3} +\frac{3\epsilon\bm{R}(\bm{R}\cdot\bm{f})}{|\bm{R}|^5} \bigg]  d\bars \ts \epsilon \ts d\theta\\
\bm{F}_{\mc{S},12}&= \int_0^{2\pi}\int_{-1/2}^{1/2} \bigg[\frac{\bm{R}_0\cdot\be_{\rho}}{|\bm{R}|^3}{\bm f} +\frac{3\bm{R}(\bm{R}\cdot\bm{f})(\bm{R}_0\cdot\be_{\rho})}{|\bm{R}|^5} \bigg]  d\bars \ts \epsilon \ts d\theta\\
\bm{F}_{\mc{S},13}&= -\int_0^{2\pi}\int_{-1/2}^{1/2} \frac{\be_{\rho}(\bm{R}\cdot\bm{f})+\bm{R}(\be_{\rho}\cdot\bm{f}) }{|\bm{R}|^3} d\bars \ts \epsilon \ts d\theta\\
\bm{F}_{\mc{S},14}&= -\int_0^{2\pi}\int_{-1/2}^{1/2} \bigg[\frac{\bm{R}_0\cdot\be_{\rho}+\epsilon}{|\bm{R}|^3}{\bm f} - \frac{\be_{\rho}(\bm{R}\cdot\bm{f})+\bm{R}(\be_{\rho}\cdot\bm{f})}{|\bm{R}|^3} \\
&\hspace{5cm} +\frac{3\bm{R}(\bm{R}\cdot\bm{f})(\bm{R}_0\cdot\be_{\rho}+\epsilon)}{|\bm{R}|^5}\bigg] \ts d\bars \ts \epsilon^2\wh\kappa \ts d\theta.
\end{aligned}
\end{equation}

We estimate each of these terms in turn, relying on Lemmas \ref{Rintest0}, \ref{Rintest1}, \ref{Rintest2}, and \ref{theta_int} accordingly, as we did in the proof of Proposition \ref{fSBp_est}. \\

Using Lemma \ref{Rintest0}, we have
\begin{equation}\label{FS14_est}
\abs{\bm{F}_{\mc{S},14}} \le \epsilon c_\kappa\norm{\bm{f}}_{C(\T)},
\end{equation}
while by Lemmas \ref{theta_int} and \ref{Rintest2} we can show
\begin{equation}\label{FS13_est}
 \abs{\bm{F}_{\mc{S},13} + 4\bm{h}_f(s)} \le \epsilon c_\kappa \norm{\bm{f}}_{C^1(\T)},
 \end{equation}
 where $\bm{h}_f(s)$ is as in \eqref{hf_def}. Similarly, using \eqref{Rlb} along with Lemmas \ref{Rintest0} and \ref{theta_int}, we have
 \begin{equation}\label{FS12_est}
  \abs{\bm{F}_{\mc{S},12}} \le \epsilon c_\kappa \norm{\bm{f}}_{C(\T)}.
 \end{equation}
Finally, by Lemmas \ref{Rintest0}, \ref{Rintest1}, and \ref{Rintest2}, we obtain
 \begin{equation}\label{FS11_est}
 \begin{aligned}
\abs{\bm{F}_{\mc{S},11}- 2\bm{h}_a(s)-4\bm{h}_f(s)} &\le \epsilon c_\kappa\norm{\bm{f}}_{C^1(\T)}; \\
\bm{h}_a(s) &:= 2\pi \big(\bm{f}(s) + \be_t(s)(\be_t(s)\cdot\bm{f}(s))\big),
\end{aligned}
\end{equation}
where $\bm{h}_f(s)$ is again as in \eqref{hf_def}. \\


%%%%%
%First note that, using $(\bm{R}_0\cdot\be_{\rho}) = \bars^2(\bm{Q}\cdot\be_\rho)$ by \eqref{CQ}, the integrand of $\bm{F}_{\mc{S},14}$ satisfies the bound 
%\begin{align*}
%\epsilon^2 \abs{\wh\kappa}\bigg| \frac{\bm{R}_0\cdot\be_{\rho}+\epsilon}{|\bm{R}|^3}{\bm f} - &\frac{\be_{\rho}(\bm{R}\cdot\bm{f})+\bm{R}(\be_{\rho}\cdot\bm{f})}{|\bm{R}|^3}+\frac{3\bm{R}(\bm{R}\cdot\bm{f})(\bm{R}_0\cdot\be_{\rho}+\epsilon)}{|\bm{R}|^5} \bigg| \\
%&\le 2\epsilon^2\kappa_{\max} \bigg(\frac{4(c_Q \bars^2 + \epsilon)}{\abs{\bm{R}}^3}+ \frac{2}{\abs{\bm{R}}^2}\bigg) \norm{\bm{f}}_{C(\T)},
%\end{align*}
%and therefore, using Lemma \ref{Rintest0}, we have
%\begin{equation}\label{FS14_est}
%\abs{\bm{F}_{\mc{S},14}} \le \epsilon 4\pi \kappa_{\max} \big(c_3(1+ 4c_Q\epsilon\abs{\log\epsilon}) + 2\epsilon c_2 \big)\norm{\bm{f}}_{C(\T)}.
%\end{equation}
%
%Next, using \eqref{Reps}, we can rewrite $\bm{F}_{\mc{S},13}$ as 
%\begin{align*}
%\bm{F}_{\mc{S},13} &= \bm{F}_{\mc{S},13a}+ \bm{F}_{\mc{S},13b}; \\
%\bm{F}_{\mc{S},13a}&:= -\int_0^{2\pi}\int_{-1/2}^{1/2} \frac{\be_{\rho}\big((-\bars\be_t+\bars^2\bm{Q})\cdot\bm{f}\big) +(-\bars\be_t+\bars^2\bm{Q})(\be_{\rho}\cdot\bm{f}) }{|\bm{R}|^3} d\bars \ts \epsilon \ts d\theta \\
%\bm{F}_{\mc{S},13b}&:= -\int_0^{2\pi}\int_{-1/2}^{1/2}  \epsilon \frac{2\be_{\rho}(\be_{\rho}\cdot\bm{f}) }{|\bm{R}|^3} d\bars \ts \epsilon \ts d\theta. 
%\end{align*}
%Since $\be_t(s)$, $\bm{Q}(s,\bars)$, and $\bm{f}(s+\bars)$ are all independent of $\theta$, and $\be_{\rho}= \cos\theta\be_{n_1}(s) +\sin\theta\be_{n_2}(s)$, we can use Lemma \ref{theta_int} to obtain the estimate 
%\begin{align*}
%\abs{\bm{F}_{\mc{S},13a}} &\le \epsilon 4\bar c_3(1+\sqrt{\epsilon}c_Q)\norm{\bm{f}}_{C(\T)}.
%\end{align*}
%
%Furthermore, using Lemma \ref{Rintest2} with $m=0$ and $n=3$, we have
%\begin{align*}
%\abs{\bm{F}_{\mc{S},13b} + 4\bm{h}_f(s)} &\le \epsilon c_{0,3}\norm{\bm{f}}_{C^1(\T)},
%\end{align*}
%where $\bm{h}_f(s)$ was defined in \eqref{hf_def}. Therefore
%\begin{equation}\label{FS13_est}
% \abs{\bm{F}_{\mc{S},13} + 4\bm{h}_f(s)} \le \epsilon\big(4\bar c_3(1+\sqrt{\epsilon}c_Q)+c_{0,3} \big)\norm{\bm{f}}_{C^1(\T)}.
% \end{equation}
% 
% Now we estimate $\bm{F}_{\mc{S},12}$. Using \eqref{Reps} and the expansion \eqref{CQ} of $\bm{R}_0=\X(s)-\X(s+\bars)$, we can write
% \begin{align*}
% \bm{F}_{\mc{S},12}&= \bm{F}_{\mc{S},12a}+ \bm{F}_{\mc{S},12b}+ \bm{F}_{\mc{S},12c};\\
% %
% \bm{F}_{\mc{S},12a}&:= \int_0^{2\pi}\int_{-1/2}^{1/2} \epsilon\bigg[\frac{\bars^2(\bm{Q}\cdot\be_{\rho})}{|\bm{R}|^3}{\bm f}+ 3\frac{\bars^4\be_t(\be_t\cdot\bm{f})(\bm{Q}\cdot\be_{\rho})}{|\bm{R}|^5}  \bigg]  d\bars \ts d\theta\\
% %
%\bm{F}_{\mc{S},12b} &:= \int_0^{2\pi}\int_{-1/2}^{1/2} 3\epsilon^2 \frac{\bars^2(\bm{Q}\cdot\be_{\rho})[-\bars\be_t(\be_\rho\cdot\bm{f})- \bars\be_\rho(\be_t\cdot\bm{f})+ \epsilon\be_\rho(\be_\rho\cdot\bm{f})]}{|\bm{R}|^5} d\bars \ts d\theta\\
%%
%\bm{F}_{\mc{S},12c} &:= \int_0^{2\pi}\int_{-1/2}^{1/2} 3\epsilon\bigg[ \frac{\big[-\bars^5(\be_t(\bm{Q}\cdot\bm{f})+ \bm{Q}(\be_t\cdot\bm{f})) + \bars^6\bm{Q}(\bm{Q}\cdot\bm{f})\big](\bm{Q}\cdot\be_{\rho})}{|\bm{R}|^5} \\
%&\hspace{5cm} + \frac{\epsilon\bars^4[\be_\rho(\bm{Q}\cdot\bm{f})+ \bm{Q}(\be_\rho\cdot\bm{f})](\bm{Q}\cdot\be_{\rho})}{|\bm{R}|^5}  \bigg]  d\bars \ts d\theta.
% \end{align*}
% 
% First, we have that $\bm{F}_{\mc{S},12c}$ satisfies 
% \begin{align*}
%\abs{ \bm{F}_{\mc{S},12c}} &\le 2\pi \norm{\bm{f}}_{C(\T)}\int_{-1/2}^{1/2} 3\epsilon c_Q^2 \frac{2|\bars|^5+ c_Q \bars^6+ 2\epsilon\bars^4 }{|\bm{R}|^5} d\bars \\
%&\le \epsilon 3\pi c_Q^2\norm{\bm{f}}_{C(\T)}\bigg(c_R^{-5}(4 +c_Q)+\int_{-1/2}^{1/2}\frac{4\epsilon \bars^4 }{|\bm{R}|^5} d\bars \bigg) \\
%&\le \epsilon 3\pi c_Q^2\norm{\bm{f}}_{C(\T)}\big(c_R^{-5}(4+c_Q)+ 4c_5\epsilon\abs{\log\epsilon} \big),
% \end{align*}
%where we have used equation \eqref{Rlb} and the fact that $\abs{\bars} \le \frac{1}{2}$ to bound the first two terms, and we have used Lemma \eqref{Rintest0} to bound the third term. \\ 
% 
%Next, using Lemma \ref{Rintest0}, we have that $\bm{F}_{\mc{S},12b}$ satisfies
% \begin{align*}
% \abs{\bm{F}_{\mc{S},12b}} &\le 2\pi\norm{\bm{f}}_{C(\T)}\int_{-1/2}^{1/2} 6\epsilon^2 \frac{c_Q(2\abs{\bars}^3+ \epsilon \bars^2)}{|\bm{R}|^5} d\bars \le \epsilon 36\pi c_Qc_5\norm{\bm{f}}_{C(\T)}.
% \end{align*}
% 
%Finally, we use Lemma \ref{theta_int} to show that $\bm{F}_{\mc{S},12a}$ satisfies
%  \begin{align*}
%   \abs{\bm{F}_{\mc{S},12a}} &\le \epsilon^{3/2}c_Q(\bar c_3+ 3\bar c_5)\norm{\bm{f}}_{C(\T)}.
%  \end{align*}
%  
% Combining the above estimates, we obtain  
% \begin{equation}\label{FS12_est}
%  \abs{\bm{F}_{\mc{S},12}} \le \epsilon \bigg(3\pi c_Q^2\big(c_R^{-5}(4+c_Q)+ 4c_5\epsilon\abs{\log\epsilon} \big) +36\pi c_Qc_5 + \epsilon^{1/2}c_Q(\bar c_3+ 3\bar c_5)\bigg)\norm{\bm{f}}_{C(\T)}.
% \end{equation}
% 
% %%%%%
% 
%It remains to estimate $\bm{F}_{\mc{S},11}$. We have
% \begin{align*}
%\bm{F}_{\mc{S},11} &= \bm{F}_{\mc{S},11a} + \bm{F}_{\mc{S},11b} + \bm{F}_{\mc{S},11c}; \\
%\bm{F}_{\mc{S},11a} &:= \int_0^{2\pi}\int_{-1/2}^{1/2} \epsilon^2\bigg[\frac{{\bm f}}{|\bm{R}|^3} +\frac{3[\bars^2\be_t(\be_t\cdot\bm{f})+ \epsilon^2\be_\rho(\be_\rho\cdot\bm{f})]}{|\bm{R}|^5} \bigg]  d\bars \ts d\theta\\
% \bm{F}_{\mc{S},11b} &:= -\int_0^{2\pi}\int_{-1/2}^{1/2} 3\epsilon^3\frac{\bars[\be_t(\be_{\rho}\cdot\bm{f})+ \be_{\rho}(\be_t\cdot\bm{f})]}{|\bm{R}|^5} d\bars \ts d\theta\\
%\bm{F}_{\mc{S},11c} &:= \int_0^{2\pi}\int_{-1/2}^{1/2} 3\epsilon^2\bigg[\frac{\bars^3[\be_t(\bm{Q}\cdot\bm{f})+\bm{Q}(\be_t\cdot\bm{f})]+ \bars^4\bm{Q}(\bm{Q}\cdot\bm{f})}{|\bm{R}|^5} \\
%&\hspace{5cm}+\frac{\epsilon\bars^2[\be_\rho(\bm{Q}\cdot\bm{f})+\bm{Q}(\be_{\rho}\cdot\bm{f})]}{|\bm{R}|^5} \bigg]  d\bars \ts d\theta. 
% \end{align*}
% 
%First, using Lemma \ref{Rintest0}, we can bound $\bm{F}_{\mc{S},11c}$ as
%\begin{align*}
%\abs{\bm{F}_{\mc{S},11c}} &\le 2\pi \norm{\bm{f}}_{C(\T)} \int_{-1/2}^{1/2} 3\epsilon^2\frac{2c_Q|\bars|^3+ c_Q^2\bars^4+2\epsilon c_Q\bars^2}{|\bm{R}|^5} d\bars \\
%&\le \epsilon 6\pi c_5(4c_Q+ \epsilon\abs{\log\epsilon}c_Q^2) \norm{\bm{f}}_{C(\T)}
%\end{align*}
%
%For $\bm{F}_{\mc{S},11b}$, we use Lemma \ref{Rintest1} to obtain 
%\begin{align*}
%\abs{\bm{F}_{\mc{S},11b}} &\le \epsilon 12\pi c_{1,5}\norm{\bm{f}}_{C^1(\T)}.
%\end{align*}
%
%Finally, to estimate $\bm{F}_{\mc{S},11a}$, we use Lemma \ref{Rintest2} to show
%\begin{equation}\label{ha_def}
%\begin{aligned}
%\abs{\bm{F}_{\mc{S},11a} - 2\bm{h}_a(s)-4\bm{h}_f(s)} &\le \epsilon 2\pi(c_{0,3}+ 6c_{0,5}) \norm{\bm{f}}_{C^1(\T)}; \\
%\bm{h}_a(s) &:= \int_0^{2\pi} \bigg(\bm{f}(s) + \be_t(s)(\be_t(s)\cdot\bm{f}(s))\bigg) d\theta \\
%&= 2\pi \big(\bm{f}(s) + \be_t(s)(\be_t(s)\cdot\bm{f}(s))\big),
%\end{aligned}
%\end{equation}
%where $\bm{h}_f(s)$ was defined in \eqref{hf_def}. Together, we have
%\begin{equation}\label{FS11_est}
%\abs{\bm{F}_{\mc{S},11}- 2\bm{h}_a(s)-4\bm{h}_f(s)} \le \epsilon2\pi \big(3c_5(4c_Q+ \epsilon\abs{\log\epsilon}c_Q^2)+ 6 c_{1,5}+ c_{0,3}+ 6c_{0,5} \big)\norm{\bm{f}}_{C^1(\T)}.
%\end{equation}
%
Combining the estimates \eqref{FS14_est}, \eqref{FS13_est}, \eqref{FS12_est}, and \eqref{FS11_est}, we obtain
\begin{equation}\label{FS1_est}
\abs{\bm{F}_{\mc{S},1}- 2\bm{h}_a(s)} \le \epsilon c_\kappa \norm{\bm{f}}_{C^1(\T)}.
\end{equation}

%%%%%%%%%%%
%%%%%%%%%%%
%%%%%%%%%%%
Now we estimate $\bm{F}_{\mc{D},1}$. Following the same outline as in the $\bm{F}_{\mc{S},1}$ estimate, we write 
\begin{equation}\label{FD1}
\begin{aligned}
 \bm{F}_{\mc{D},1} &= 3(\bm{F}_{\mc{D},11} + \bm{F}_{\mc{D},12} + \bm{F}_{\mc{D},13}+ \bm{F}_{\mc{D},14}); \\
\bm{F}_{\mc{D},11}&= \int_0^{2\pi}\int_{-1/2}^{1/2} \bigg[\frac{\epsilon{\bm f}}{|\bm{R}|^5} -\frac{5\epsilon\bm{R}(\bm{R}\cdot\bm{f})}{|\bm{R}|^7} \bigg] d\bars \ts \epsilon \ts d\theta\\
\bm{F}_{\mc{D},12}&= \int_0^{2\pi}\int_{-1/2}^{1/2} \bigg[\frac{\bm{R}_0\cdot\be_{\rho}}{|\bm{R}|^5}{\bm f} -\frac{5\bm{R}(\bm{R}\cdot\bm{f})(\bm{R}_0\cdot\be_{\rho})}{|\bm{R}|^7} \bigg] d\bars \ts \epsilon \ts d\theta\\
\bm{F}_{\mc{D},13}&= \int_0^{2\pi}\int_{-1/2}^{1/2} \frac{\be_{\rho}(\bm{R}\cdot\bm{f})+\bm{R}(\be_{\rho}\cdot\bm{f}) }{|\bm{R}|^5}  d\bars \ts \epsilon \ts d\theta\\
\bm{F}_{\mc{D},14}&= -\int_0^{2\pi}\int_{-1/2}^{1/2} \bigg[\frac{\bm{R}_0\cdot\be_{\rho}+\epsilon}{|\bm{R}|^5}{\bm f} + \frac{\be_{\rho}(\bm{R}\cdot\bm{f})+\bm{R}(\be_{\rho}\cdot\bm{f})}{|\bm{R}|^5} \\
&\hspace{5cm} -\frac{5\bm{R}(\bm{R}\cdot\bm{f})(\bm{R}_0\cdot\be_{\rho}+\epsilon)}{|\bm{R}|^7}\bigg] \ts d\bars \ts \epsilon^2\wh\kappa \ts d\theta.
\end{aligned}
\end{equation}

Now, using Lemma \ref{Rintest0}, we can show
\begin{equation}\label{FD14_est}
\abs{\bm{F}_{\mc{D},14}} \le \epsilon^{-1} c_\kappa\norm{\bm{f}}_{C(\T)}. 
\end{equation}

Furthermore, by Lemmas \ref{Rintest0}, \ref{Rintest1}, and \ref{Rintest2}, we have
\begin{equation}\label{FD13_est}
\abs{\bm{F}_{\mc{D},13} - \epsilon^{-2}\frac{8}{3} \bm{h}_f(s)} \le \epsilon^{-1}c_\kappa\norm{\bm{f}}_{C^1(\T)}.
\end{equation}
where $\bm{h}_f(s)$ is as in \eqref{hf_def}. Then, via \ref{Rintest0}, we can show
\begin{equation}\label{FD12_est}
\abs{\bm{F}_{\mc{D},12}} \le \epsilon^{-1}c_\kappa\norm{\bm{f}}_{C(\T)},
\end{equation}
while Lemmas \ref{Rintest0}, \ref{Rintest1}, and \ref{Rintest2} yield
\begin{equation}\label{FD11_est}
\abs{\bm{F}_{\mc{D},11}+\epsilon^{-2} \frac{8}{3}\bm{h}_f(s)} \le \epsilon^{-1}c_\kappa\norm{\bm{f}}_{C^1(\T)}.
\end{equation}


%%%%%
%Again using Lemma \ref{Rintest0}, we can show
%\begin{equation}\label{FD14_est}
%\begin{aligned}
%\abs{\bm{F}_{\mc{D},14}} &\le \epsilon^2 4\pi\kappa_{\max}\norm{\bm{f}}_{C(\T)} \int_{-1/2}^{1/2} \bigg[\frac{6(c_Q\bars^2+\epsilon) }{|\bm{R}|^5} + \frac{2}{|\bm{R}|^4} \bigg] \ts d\bars \\
%& \le \epsilon^{-1} 4\pi\kappa_{\max}(6c_5(1+\epsilon c_Q)+2c_4)\norm{\bm{f}}_{C(\T)}. 
%\end{aligned}
%\end{equation}
%
%To estimate $\bm{F}_{\mc{D},13}$, we use \eqref{Reps} to write
%\begin{align*}
%\bm{F}_{\mc{D},13} &= \bm{F}_{\mc{D},13a} + \bm{F}_{\mc{D},13b} + \bm{F}_{\mc{D},13c}; \\
%\bm{F}_{\mc{D},13a} &:= \int_0^{2\pi}\int_{-1/2}^{1/2} \epsilon^2\frac{2\be_{\rho}(\be_{\rho}\cdot\bm{f})}{|\bm{R}|^5}  d\bars \ts d\theta\\
%\bm{F}_{\mc{D},13b} &:= -\int_0^{2\pi}\int_{-1/2}^{1/2}\epsilon \frac{\bars\big[\be_{\rho}(\be_t\cdot\bm{f})+\be_t(\be_{\rho}\cdot\bm{f})\big] }{|\bm{R}|^5}  d\bars \ts d\theta\\
%\bm{F}_{\mc{D},13c} &:= \int_0^{2\pi}\int_{-1/2}^{1/2} \epsilon\frac{\bars^2\big[\be_{\rho}(\bm{Q}\cdot\bm{f})+\bm{Q}(\be_{\rho}\cdot\bm{f})\big] }{|\bm{R}|^5}  d\bars  \ts d\theta.
%\end{align*}
%
%We then have
%\begin{align*}
%\abs{\bm{F}_{\mc{D},13c}} \le \epsilon 2\pi\norm{\bm{f}}_{C(\T)}\int_{-1/2}^{1/2} \frac{2c_Q\bars^2}{|\bm{R}|^5} d\bars \le \epsilon^{-1}4\pi c_Qc_5\norm{\bm{f}}_{C(\T)},
%\end{align*}
%by Lemma \ref{Rintest0}. Next, using Lemma \ref{Rintest1}, we can bound $\bm{F}_{\mc{D},13b}$ as
%\begin{align*}
%\abs{\bm{F}_{\mc{D},13b}} \le \epsilon^{-1} c_{1,5} 4\pi \norm{\bm{f}}_{C^1(\T)}.
%\end{align*}
%
%Finally, using Lemma \ref{Rintest2}, we can show
%\begin{align*}
%\abs{\bm{F}_{\mc{D},13a} - \epsilon^{-2}\frac{8}{3} \bm{h}_f(s)} \le \epsilon^{-1}2\pi c_{0,5} \norm{\bm{f}}_{C^1(\T)},
%\end{align*}
%where $\bm{h}_f(s)$ was defined in \eqref{hf_def}. Combining these three estimates, we obtain
%\begin{equation}\label{FD13_est}
%\abs{\bm{F}_{\mc{D},13} - \epsilon^{-2}\frac{8}{3} \bm{h}_f(s)} \le \epsilon^{-1}2\pi\big(2c_Qc_5+2c_{1,5}+c_{0,5} \big)\norm{\bm{f}}_{C^1(\T)}.
%\end{equation}
%
%%%%
%
%Next we estimate $\bm{F}_{\mc{D},12}$. Using that $\bm{R}_0\cdot\be_{\rho}=\bars^2\bm{Q}$, by Lemma \ref{Rintest0} we have
%\begin{equation}\label{FD12_est}
%\abs{\bm{F}_{\mc{D},12}} \le \epsilon 2\pi \norm{\bm{f}}_{C(\T)} \int_{-1/2}^{1/2} \frac{6c_Q\bars^2}{|\bm{R}|^5} d\bars \le \epsilon^{-1}12\pi c_Qc_5\norm{\bm{f}}_{C(\T)}.
%\end{equation}
%
%Finally we estimate $\bm{F}_{\mc{D},11}$. Using \eqref{Reps}, we write
%\begin{align*}
%\bm{F}_{\mc{D},11}&=  \bm{F}_{\mc{D},11a}+  \bm{F}_{\mc{D},11b}+ \bm{F}_{\mc{D},11c}; \\
% \bm{F}_{\mc{D},11a}&:=  \int_0^{2\pi}\int_{-1/2}^{1/2} \epsilon^2\bigg[\frac{{\bm f}}{|\bm{R}|^5} -5\frac{\bars^2\be_t(\be_t\cdot\bm{f})+\epsilon^2\be_\rho(\be_\rho\cdot\bm{f})}{|\bm{R}|^7} \bigg] d\bars \ts d\theta \\
% \bm{F}_{\mc{D},11b}&:=  \int_0^{2\pi}\int_{-1/2}^{1/2} 5\epsilon^3\frac{\bars[\be_t(\be_\rho\cdot\bm{f})+\be_\rho(\be_t\cdot\bm{f})]}{|\bm{R}|^7} d\bars \ts d\theta\\
% \bm{F}_{\mc{D},11c} &:=  -\int_0^{2\pi}\int_{-1/2}^{1/2} 5\epsilon^2 \bigg[\frac{-\bars^3[\be_t(\bm{Q}\cdot\bm{f})+\bm{Q}(\be_t\cdot\bm{f})] + \bars^4\bm{Q}(\bm{Q}\cdot\bm{f}) }{|\bm{R}|^7} \\
% &\hspace{5cm} + \frac{\epsilon \bars^2[\be_\rho(\bm{Q}\cdot\bm{f})+\bm{Q}(\be_\rho\cdot\bm{f})] }{|\bm{R}|^7}\bigg] d\bars \ts d\theta.
%\end{align*}
%
%Using Lemma \ref{Rintest0}, we obtain the bound
%\begin{align*}
%\abs{ \bm{F}_{\mc{D},11c}}&\le 2\pi\norm{\bm{f}}_{C(\T)} \int_{-1/2}^{1/2} 5c_Q\epsilon^2 \frac{2\bars^3+ c_Q\bars^4+ 2\epsilon\bars^2}{|\bm{R}|^7} d\bars \le \epsilon^{-1}10\pi c_Q c_7(4+ \epsilon c_Q )\norm{\bm{f}}_{C(\T)}.
%\end{align*}
%
%Furthermore, by Lemma \ref{Rintest1}, we have
%\begin{align*}
%\abs{ \bm{F}_{\mc{D},11b}}&\le \epsilon^{-1}20\pi c_{1,7}\norm{\bm{f}}_{C^1(\T)}.
%\end{align*}
%
%To estimate $\bm{F}_{\mc{D},11a}$, we use Lemma \ref{Rintest2} with $m= 0,2,0$ and $n=5,7,7$ for the first, second, and third term, respectively. Using that 
%\begin{align*}
%\int_0^{2\pi}&\bigg(\frac{4}{3}\bm{f}(s)-\frac{4}{3}\be_t(s)(\be_t(s)\cdot\bm{f}(s))- \frac{16}{3}\be_{\rho}(s,\theta)(\be_\rho(s,\theta)\cdot\bm{f}) \bigg) \ts d\theta \\
%&= \frac{8\pi}{3}\bm{f}(s)-\frac{8\pi}{3}\be_t(s)(\be_t(s)\cdot\bm{f}(s)) - \frac{16\pi}{3}\big((\bm{f}(s)\cdot\be_{n_1}(s))\be_{n_1}(s) + (\bm{f}(s)\cdot\be_{n_2}(s))\be_{n_2}(s) \big) \\
%&= -\frac{8\pi}{3}\big((\bm{f}(s)\cdot\be_{n_1}(s))\be_{n_1}(s) + (\bm{f}(s)\cdot\be_{n_2}(s))\be_{n_2}(s) \big) = -\frac{8}{3}\bm{h}_f(s),
%\end{align*}
%where $\bm{h}_f(s)$ was defined in \eqref{hf_def}, we can show
%\begin{align*}
%\abs{ \bm{F}_{\mc{D},11a} + \epsilon^{-2} \frac{8}{3}\bm{h}_f(s)} &\le \epsilon^{-1}2\pi(c_{0,5}+10c_{0,7}) \norm{\bm{f}}_{C^1(\T)}.
%\end{align*}
%
%Together, we obtain an estimate for $\bm{F}_{\mc{D},11}$:
%\begin{equation}\label{FD11_est}
%\abs{\bm{F}_{\mc{D},11}+\epsilon^{-2} \frac{8}{3}\bm{h}_f(s)} \le \epsilon^{-1}2\pi(5c_Qc_7(4+\epsilon c_Q)+10c_{1,7}+c_{0,5}+10c_{0,7})\norm{\bm{f}}_{C^1(\T)}.
%\end{equation}

Combining estimates \eqref{FD14_est}, \eqref{FD13_est}, \eqref{FD12_est}, and \eqref{FD11_est}, we obtain
\begin{equation}\label{FD1_est}
\begin{aligned}
\abs{\bm{F}_{\mc{D},1}} &\le \abs{3\bm{F}_{\mc{D},11}+\epsilon^{-2} 8\bm{h}_f(s)} + \abs{3\bm{F}_{\mc{D},12}}+\abs{3\bm{F}_{\mc{D},13}-\epsilon^{-2} 8\bm{h}_f(s)} + \abs{3\bm{F}_{\mc{D},14}} \\
&\le \epsilon^{-1} c_\kappa\norm{\bm{f}}_{C^1(\T)}.
\end{aligned}
\end{equation}

Finally, using the estimates \eqref{FS1_est} and \eqref{FD1_est}, as well as the expression \eqref{force_1} for $\bm{f}_1^{\SB}$, we obtain
\begin{equation}\label{f1sb_est0}
\abs{\bm{f}^{\SB}_1 - \frac{1}{4\pi}\bm{h}_a(s)} \le \frac{1}{8\pi}\bigg(\abs{\bm{F}_{\mc{S},1}- 2\bm{h}_a(s)} +\frac{\epsilon^2}{2}\abs{\bm{F}_{\mc{D},1}}\bigg) \le \epsilon c_\kappa \norm{\bm{f}}_{C^1(\T)},
\end{equation}
from which, using the form of $\bm{h}_a(s)$ in \eqref{FS11_est}, we obtain Proposition \ref{fSB1_est}.
\end{proof}

%%%%%%%%%%%%%%%%%%%%%%%%%%%%%%%%%%%%%%%%%%%%%%%%%%%%%%%%%%%%%%%%
%%%%%%%%%%%%%%%%%%%%%%%%%%%%%%%%%%%%%%%%%%%%%%%%%%%%%%%%%%%%%%%%
%%%%%%%%%%%%%%%%%%%%%%%%%%%%%%%%%%%%%%%%%%%%%%%%%%%%%%%%%%%%%%%%
%%%%%%%%%%%%%%%%%%%%%%%%%%%%%%%%%%%%%%%%%%%%%%%%%%%%%%%%%%%%%%%%
%%%%%%%%%%%%%%%%%%%%%%%%%%%%%%%%%%%%%%%%%%%%%%%%%%%%%%%%%%%%%%%%

Next we show the following bound for the component $\bm{f}^{\SB}_2(s)$ of the slender body force, given by \eqref{force_components}.
\begin{proposition}\label{fSB2_est}
Let the slender body $\Sigma_\epsilon$ be as in Section \ref{geometric_constraints}. Given $\bm{f}\in C^1(\T)$, let $\bm{f}^{\SB}_2(s)$ be defined as in \eqref{force_components}. Then $\bm{f}^{\SB}_2$ satisfies 
\begin{equation}
\abs{\bm{f}^{\SB}_2 } \le \epsilon c_\kappa\norm{\bm{f}}_{C^1(\T)},
\end{equation}
where the constant $c_\kappa$ depends only on $c_{\Gamma}$ and $\kappa_{\max}$.
\end{proposition}

\begin{proof}
Using the $\bm{f}^{\SB}_1$ computation as a guide, we again use \eqref{SBT2} to consider $\bm{f}^{\SB}_2$ as the sum of a Stokeslet and doublet term: 
\begin{equation}\label{force_2}
\begin{aligned}
\bm{f}^{\SB}_2 &= \frac{1}{8\pi} \bigg(\bm{F}_{\mc{S},2} + \frac{\epsilon^2}{2}\bm{F}_{\mc{D},2} \bigg); \\
\bm{F}_{\mc{S},2} &:= -\int_0^{2\pi}\int_{-1/2}^{1/2} \be_\rho \bigg(\frac{\p}{\p\rho}\mc{S}(\bm{R})\bm{f}(s+\bars)\bigg)\cdot\be_{\rho}  \ts d\bars \ts \epsilon(1-\epsilon\wh\kappa)d\theta \\
\bm{F}_{\mc{D},2}&:=  -\int_0^{2\pi}\int_{-1/2}^{1/2} \be_\rho \bigg(\frac{\p}{\p\rho}\mc{D}(\bm{R})\bm{f}(s+\bars)\bigg)\cdot\be_{\rho}  \ts d\bars \ts \epsilon(1-\epsilon\wh\kappa)d\theta  .
\end{aligned}
\end{equation}

As we did for $\bm{f}^{\SB}_1$, we begin by estimating the Stokeslet term $\bm{F}_{\mc{S},2}$. We write $\bm{F}_{\mc{S},2}$ as 
\begin{equation}\label{FS2}
\begin{aligned}
\bm{F}_{\mc{S},2} &= \bm{F}_{\mc{S},21} + \bm{F}_{\mc{S},22} + \bm{F}_{\mc{S},23}+ \bm{F}_{\mc{S},24}; \\
%
\bm{F}_{\mc{S},21} &:= \int_0^{2\pi}\int_{-1/2}^{1/2} \bigg[\frac{\epsilon}{|\bm{R}|^3}({\bm f}\cdot\be_\rho)\be_\rho +\frac{3\epsilon\be_\rho(\bm{R}\cdot\be_{\rho})(\bm{R}\cdot\bm{f})}{|\bm{R}|^5} \bigg] d\bars \ts \epsilon \ts d\theta \\
%
\bm{F}_{\mc{S},22} &:= \int_0^{2\pi}\int_{-1/2}^{1/2} \bigg[\frac{\bm{R}_0\cdot\be_{\rho}}{|\bm{R}|^3}({\bm f}\cdot\be_\rho)\be_\rho +\frac{3\be_\rho(\bm{R}\cdot\be_{\rho})(\bm{R}\cdot\bm{f})(\bm{R}_0\cdot\be_{\rho})}{|\bm{R}|^5}\bigg] d\bars \ts \epsilon \ts d\theta \\
%
\bm{F}_{\mc{S},23} &:= -\int_0^{2\pi}\int_{-1/2}^{1/2} \frac{\be_{\rho}(\bm{R}\cdot\bm{f})+\be_\rho(\bm{R}\cdot\be_{\rho})(\be_{\rho}\cdot\bm{f})}{|\bm{R}|^3} d\bars \ts \epsilon \ts d\theta \\
%
\bm{F}_{\mc{S},24} &:= -\int_0^{2\pi}\int_{-1/2}^{1/2} \bigg[\frac{\bm{R}_0\cdot\be_{\rho}+\epsilon}{|\bm{R}|^3}({\bm f}\cdot\be_\rho)\be_\rho - \frac{\be_{\rho}(\bm{R}\cdot\bm{f})+\be_\rho(\bm{R}\cdot\be_{\rho})(\be_{\rho}\cdot\bm{f})}{|\bm{R}|^3} \\
&\hspace{4cm}+\frac{3\be_\rho(\bm{R}\cdot\be_{\rho})(\bm{R}\cdot\bm{f})(\bm{R}_0\cdot\be_{\rho}+\epsilon)}{|\bm{R}|^5}\bigg] d\bars \ts\epsilon^2\wh\kappa \ts d\theta. 
\end{aligned}
\end{equation}

We again rely on Lemmas \ref{Rintest0}, \ref{Rintest1}, \ref{Rintest2}, and \ref{theta_int} to estimate each of the above components of $\bm{F}_{\mc{S},2}$ in the same way as in the proofs of Propositions \ref{fSBp_est} and \ref{fSB1_est}. By Lemma \ref{Rintest0}, we have
\begin{equation}\label{FS24_est}
\abs{\bm{F}_{\mc{S},24}} \le \epsilon c_\kappa \norm{\bm{f}}_{C(\T)}.
\end{equation}
Additionally, by Lemmas \ref{theta_int} and \ref{Rintest2}, we have
\begin{equation}\label{FS23_est}
 \abs{\bm{F}_{\mc{S},23} + 4\bm{h}_f(s)} \le \epsilon c_\kappa\norm{\bm{f}}_{C^1(\T)}
 \end{equation}
for $\bm{h}_f(s)$ as in \eqref{hf_def}.
From equation \eqref{Rlb} and Lemmas \ref{Rintest0} and \ref{theta_int}, we also obtain
 \begin{equation}\label{FS22_est}
 \abs{\bm{F}_{\mc{S},22}}\le \epsilon c_\kappa\norm{\bm{f}}_{C(\T)}.
 \end{equation}
Finally, using Lemmas \ref{Rintest0}, \ref{Rintest1}, and \ref{Rintest2}, we can show
\begin{equation}\label{FS21_est}
  \abs{\bm{F}_{\mc{S},21}- 6 \bm{h}_f(s)} \le \epsilon c_\kappa \norm{\bm{f}}_{C^1(\T)}.
 \end{equation} 

%%%%%
%As before, we estimate $\bm{F}_{\mc{S},24}$ via Lemma \ref{Rintest0}. Using that $\bm{R}_0\cdot\be_{\rho}=\bars^2\bm{Q}\cdot\be_{\rho}$, we have
%\begin{equation}\label{FS24_est}
%\begin{aligned}
%\abs{\bm{F}_{\mc{S},24}} &\le 2\pi \norm{\bm{f}}_{C(\T)} \int_{-1/2}^{1/2} \epsilon^2\abs{\wh\kappa}\bigg[4\frac{\bars^2c_Q+\epsilon}{|\bm{R}|^3} + \frac{2}{|\bm{R}|^2} \bigg] d\bars \\
%&\le \epsilon 8\pi \kappa_{\max} (2c_3(1+\epsilon\abs{\log\epsilon} c_Q) + c_2)\norm{\bm{f}}_{C(\T)}.
%\end{aligned}
%\end{equation}
%
%Next we bound $\bm{F}_{\mc{S},23}$. Following the same steps as in the $\bm{F}_{\mc{S},13}$ estimate, we rewrite $\bm{F}_{\mc{S},23}$ as
%\begin{align*}
%\bm{F}_{\mc{S},23} &= \bm{F}_{\mc{S},23a}+ \bm{F}_{\mc{S},23b}; \\
%\bm{F}_{\mc{S},23a}&:= -\int_0^{2\pi}\int_{-1/2}^{1/2} \frac{\be_{\rho}\big((-\bars\be_t+\bars^2\bm{Q})\cdot\bm{f}\big) + \bars^2\be_\rho(\bm{Q}\cdot\be_\rho)(\be_{\rho}\cdot\bm{f}) }{|\bm{R}|^3} d\bars \ts \epsilon \ts d\theta \\
%\bm{F}_{\mc{S},23b}&:= -\int_0^{2\pi}\int_{-1/2}^{1/2}  \epsilon \frac{2\be_{\rho}(\be_{\rho}\cdot\bm{f}) }{|\bm{R}|^3} d\bars \ts \epsilon \ts d\theta. 
%\end{align*}
%
%As in the $\bm{F}_{\mc{S},13a}$ estimate, to bound $\bm{F}_{\mc{S},23a}$ we rely on the $\theta$-independence of $\bm{Q}(s,\bars)$, $\bm{f}(s+\bars)$, and $\be_t(s)$, and use Lemma \ref{theta_int} -- in particular, the remark about triple copies of $\be_\rho$ -- to show  
%\begin{align*}
%\abs{\bm{F}_{\mc{S},23a}} &\le \epsilon 2\bar c_3(1+ 6\sqrt{\epsilon}c_Q)\norm{\bm{f}}_{C(\T)}.
%\end{align*}
%
%Noting that $\bm{F}_{\mc{S},23b}= \bm{F}_{\mc{S},13b}$, via Lemma \ref{Rintest2} we again have 
%\begin{align*}
%\abs{\bm{F}_{\mc{S},23b} + 4\bm{h}_f(s)} &\le \epsilon c_{0,3}\norm{\bm{f}}_{C^1(\T)},
%\end{align*}
%where $\bm{h}_f(s)$ was defined in \eqref{hf_def}. Altogether we obtain the estimate
%\begin{equation}\label{FS23_est}
% \abs{\bm{F}_{\mc{S},23} + 4\bm{h}_f(s)} \le \epsilon 2\big(\bar c_3(1+6\sqrt{\epsilon}c_Q)+c_{0,3} \big)\norm{\bm{f}}_{C^1(\T)}.
% \end{equation}
% 
%Next we estimate $\bm{F}_{\mc{S},22}$. We again decompose $\bm{F}_{\mc{S},22}$ as
% \begin{align*}
% \bm{F}_{\mc{S},22}&= \bm{F}_{\mc{S},22a}+ \bm{F}_{\mc{S},22b}+ \bm{F}_{\mc{S},22c};\\
% %
% \bm{F}_{\mc{S},22a}&:= \int_0^{2\pi}\int_{-1/2}^{1/2} \epsilon \frac{\bars^2(\bm{Q}\cdot\be_{\rho})}{|\bm{R}|^3}({\bm f}\cdot\be_\rho)\be_\rho  d\bars \ts d\theta\\
% %
%\bm{F}_{\mc{S},22b} &:= \int_0^{2\pi}\int_{-1/2}^{1/2} 3\epsilon^2 \frac{\bars^2(\bm{Q}\cdot\be_{\rho})[-\bars\be_\rho(\be_t\cdot\bm{f})+ \epsilon\be_\rho(\be_\rho\cdot\bm{f})]}{|\bm{R}|^5} d\bars \ts d\theta\\
%%
%\bm{F}_{\mc{S},22c} &:= \int_0^{2\pi}\int_{-1/2}^{1/2} 3\epsilon\bigg[ \frac{\big[-\bars^5(\be_t\cdot\bm{f})+ \bars^6(\bm{Q}\cdot\bm{f})\big](\bm{Q}\cdot\be_{\rho})^2\be_\rho}{|\bm{R}|^5} \\
%&\hspace{3cm} + \frac{\epsilon\bars^4[(\bm{Q}\cdot\bm{f})+ (\bm{Q}\cdot\be_\rho)(\be_\rho\cdot\bm{f})](\bm{Q}\cdot\be_{\rho})\be_\rho}{|\bm{R}|^5}  \bigg]  d\bars \ts d\theta.
% \end{align*}
%
%First, we have that $\bm{F}_{\mc{S},22c}$ satisfies 
%\begin{align*}
%\abs{\bm{F}_{\mc{S},22c}} &\le 2\pi c_Q^2 \norm{\bm{f}}_{C(\T)} \int_{-1/2}^{1/2} 3\epsilon \frac{|\bars|^5 + c_Q\bars^6 + 2\epsilon\bars^4}{|\bm{R}|^5} \ts d\bars \\
%&\le \epsilon3\pi c_Q^2\norm{\bm{f}}_{C(\T)} \big(c_R^{-5}(4+c_Q)+4c_5\epsilon\abs{\log\epsilon} \big).
%\end{align*}
%This is the same bound as in the $\bm{F}_{\mc{S},12c}$ estimate, where we relied on \eqref{Rlb} as well as $\abs{\bars}\le \frac{1}{2}$ to estimate the first two terms, and used Lemma \ref{Rintest0} to bound the third term. \\
%
%Next we use Lemma \ref{Rintest0} to show
%\begin{align*}
%\abs{\bm{F}_{\mc{S},22b}} \le 2\pi \norm{\bm{f}}_{C(\T)} \int_{-1/2}^{1/2} 3c_Q \epsilon^2 \frac{|\bars|^3+ \epsilon\bars^2}{|\bm{R}|^5} d\bars \le \epsilon 12\pi c_Qc_5 \norm{\bm{f}}_{C(\T)}.
%\end{align*}
%
%Furthermore, by Lemma \ref{theta_int}, using the remark about integration against triples of the form $(\bm{A}(\bars)\cdot\be_\rho)(\bm{B}(\bars)\cdot\be_\theta)\be_\theta$, we show
%\begin{align*}
% \abs{\bm{F}_{\mc{S},22a}} \le \epsilon^{3/2}6 c_Q \bar c_3\norm{\bm{f}}_{C(\T)}. 
%\end{align*}
%
%The above three estimates together give
%\begin{equation}\label{FS22_est}
% \abs{\bm{F}_{\mc{S},22}}\le \epsilon\big( 3\pi c_Q^2(c_R^{-5}(4+c_Q)+4c_5\epsilon\abs{\log\epsilon})+ 12\pi c_Qc_5+\epsilon^{1/2}6 c_Q \bar c_3 \big)\norm{\bm{f}}_{C(\T)}.
% \end{equation}
% 
%Finally we estimate $\bm{F}_{\mc{S},21}$. Using \eqref{Reps}, we write 
%\begin{align*}
%\bm{F}_{\mc{S},21} &= \bm{F}_{\mc{S},21a} + \bm{F}_{\mc{S},21b} + \bm{F}_{\mc{S},21c}; \\
%\bm{F}_{\mc{S},21a} &:= \int_0^{2\pi}\int_{-1/2}^{1/2} \epsilon^2\bigg[\frac{({\bm f}\cdot\be_\rho)\be_\rho}{|\bm{R}|^3} +\frac{3\epsilon^2\be_\rho(\be_\rho\cdot\bm{f})}{|\bm{R}|^5} \bigg]  d\bars \ts d\theta\\
% \bm{F}_{\mc{S},21b} &:= -\int_0^{2\pi}\int_{-1/2}^{1/2} 3\epsilon^3\frac{\bars \be_{\rho}(\be_t\cdot\bm{f})}{|\bm{R}|^5} d\bars \ts d\theta\\
%\bm{F}_{\mc{S},21c} &:= \int_0^{2\pi}\int_{-1/2}^{1/2} 3\epsilon^2\bigg[\frac{\bars^3[-(\be_t\cdot\bm{f})+ \bars(\bm{Q}\cdot\bm{f})](\bm{Q}\cdot\be_\rho)\be_\rho}{|\bm{R}|^5} \\
%&\hspace{3cm}+\frac{\epsilon\bars^2[(\bm{Q}\cdot\bm{f})+(\bm{Q}\cdot\be_\rho)(\be_{\rho}\cdot\bm{f})]\be_\rho}{|\bm{R}|^5} \bigg]  d\bars \ts d\theta. 
% \end{align*}
% 
%We first estimate $\bm{F}_{\mc{S},21c}$ using Lemma \ref{Rintest0}. We have
% \begin{align*}
% \abs{\bm{F}_{\mc{S},21c}} &\le 2\pi \norm{\bm{f}}_{C(\T)} \int_{-1/2}^{1/2} 3c_Q\epsilon^2 \frac{|\bars|^3 + c_Q\bars^4 +2\epsilon\bars^2}{|\bm{R}|^5} d\bars \le \epsilon 6\pi c_Qc_5(3+ \epsilon\abs{\log\epsilon}c_Q) \norm{\bm{f}}_{C(\T)}. 
% \end{align*}
% 
%Next, using Lemma \ref{Rintest1}, we can estimate $\bm{F}_{\mc{S},21b}$ as 
% \begin{align*}
% \abs{\bm{F}_{\mc{S},21b}} &\le \epsilon6\pi c_{1,5} \norm{\bm{f}}_{C^1(\T)}.
% \end{align*}
% 
% Lastly, by Lemma \ref{Rintest2}, we obtain the following estimate for $\bm{F}_{\mc{S},21a}$:
% \begin{align*}
% \abs{\bm{F}_{\mc{S},21a}- 6 \bm{h}_f(s)} &\le \epsilon2\pi(c_{0,3}+3c_{0,5})\norm{\bm{f}}_{C^1(\T)}, 
% \end{align*}
% where $\bm{h}_f(s)$ was defined in \eqref{hf_def}. Altogether, we obtain the estimate
% \begin{equation}\label{FS21_est}
%  \abs{\bm{F}_{\mc{S},21}- 6 \bm{h}_f(s)} \le \epsilon2\pi(3c_Qc_5(3+\epsilon\abs{\log\epsilon}c_Q)+3c_{1,5}+ c_{0,3}+ 3c_{0,5})\norm{\bm{f}}_{C^1(\T)}.
% \end{equation} 
 
Combining estimates \eqref{FS24_est}, \eqref{FS23_est}, \eqref{FS22_est}, and \eqref{FS21_est}, we obtain the bound
\begin{equation}\label{FS2_est}
\abs{\bm{F}_{\mc{S},2}- 2 \bm{h}_f(s)} \le \epsilon c_\kappa \norm{\bm{f}}_{C^1(\T)}.
\end{equation}

%%%%%%%%%%%%%%%%%%%%%%%%%%%
%%%%%%%%%%%%%%%%%%%%%%%%%%%
%%%%%%%%%%%%%%%%%%%%%%%%%%%

Now we estimate the doublet term of the expression \eqref{force_2} for $\bm{f}^{\SB}_2$. We have that $\bm{F}_{\mc{D},2}$ can be expressed as
\begin{equation}\label{FD2}
\begin{aligned}
\bm{F}_{\mc{D},2} &= 3(\bm{F}_{\mc{D},21} + \bm{F}_{\mc{D},22} + \bm{F}_{\mc{D},23}+ \bm{F}_{\mc{D},24}); \\
%
\bm{F}_{\mc{D},21} &:= \int_0^{2\pi}\int_{-1/2}^{1/2} \bigg[\frac{\epsilon({\bm f}\cdot\be_\rho)\be_\rho}{|\bm{R}|^5} - \frac{5\epsilon\be_\rho(\bm{R}\cdot\be_{\rho})(\bm{R}\cdot\bm{f})}{|\bm{R}|^7} \bigg] d\bars \ts \epsilon \ts d\theta \\
%
\bm{F}_{\mc{D},22} &:= \int_0^{2\pi}\int_{-1/2}^{1/2} \bigg[\frac{\bm{R}_0\cdot\be_{\rho}}{|\bm{R}|^5}({\bm f}\cdot\be_\rho)\be_\rho -\frac{5\be_\rho(\bm{R}\cdot\be_{\rho})(\bm{R}\cdot\bm{f})(\bm{R}_0\cdot\be_{\rho})}{|\bm{R}|^7}\bigg] d\bars \ts \epsilon \ts d\theta \\
%
\bm{F}_{\mc{D},23} &:= \int_0^{2\pi}\int_{-1/2}^{1/2} \frac{\be_{\rho}(\bm{R}\cdot\bm{f})+\be_\rho(\bm{R}\cdot\be_{\rho})(\be_{\rho}\cdot\bm{f})}{|\bm{R}|^5} d\bars \ts \epsilon \ts d\theta \\
%
\bm{F}_{\mc{D},24} &:= -\int_0^{2\pi}\int_{-1/2}^{1/2} \bigg[\frac{\bm{R}_0\cdot\be_{\rho}+\epsilon}{|\bm{R}|^5}({\bm f}\cdot\be_\rho)\be_\rho + \frac{\be_{\rho}(\bm{R}\cdot\bm{f})+\be_\rho(\bm{R}\cdot\be_{\rho})(\be_{\rho}\cdot\bm{f})}{|\bm{R}|^5} \\
&\hspace{4cm} -\frac{5\be_\rho(\bm{R}\cdot\be_{\rho})(\bm{R}\cdot\bm{f})(\bm{R}_0\cdot\be_{\rho}+\epsilon)}{|\bm{R}|^7}\bigg] d\bars \ts\epsilon^2\wh\kappa \ts d\theta 
\end{aligned}
\end{equation}

We estimate each of these terms following the same procedure as in the Stokeslet term estimate. In particular, using Lemma \ref{Rintest0}, we can show
\begin{equation}\label{FD24_est}
\abs{\bm{F}_{\mc{D},24}} \le \epsilon^{-1}c_\kappa \norm{\bm{f}}_{C(\T)}.
\end{equation}
Furthermore, by Lemmas \ref{Rintest0}, \ref{Rintest1}, and \ref{Rintest2}, we obtain
\begin{equation}\label{FD23_est}
\abs{\bm{F}_{\mc{D},23}- \epsilon^{-2}\frac{8}{3}\bm{h}_f(s)} \le \epsilon^{-1}c_\kappa \norm{\bm{f}}_{C^1(\T)}
\end{equation}
for $\bm{h}_f(s)$ as in \eqref{hf_def}. Next, by Lemma \ref{Rintest0}, we have
\begin{equation}\label{FD22_est}
\abs{\bm{F}_{\mc{D},22}} \le \epsilon^{-1}c_\kappa\norm{\bm{f}}_{C(\T)}.
\end{equation}
Finally, by Lemmas \ref{Rintest0}, \ref{Rintest1}, and \ref{Rintest2}, we can show
\begin{equation}\label{FD21_est}
\abs{\bm{F}_{\mc{D},21} +\epsilon^{-2}4\bm{h}_f(s)} \le \epsilon^{-1}c_\kappa \norm{\bm{f}}_{C^1(\T)}.
\end{equation}

%%%%%
%We estimate $\bm{F}_{\mc{D},24}$ exactly as we did for $\bm{F}_{\mc{D},14}$. Using Lemma \ref{Rintest0}, we have
%\begin{equation}\label{FD24_est}
%\begin{aligned}
%\abs{\bm{F}_{\mc{D},24}} &\le 2\pi \norm{\bm{f}}_{C(\T)}\int_{-1/2}^{1/2} \epsilon^2\abs{\wh\kappa}\bigg[\frac{6c_Q\bars^2 +\epsilon}{|\bm{R}|^5} + \frac{2}{|\bm{R}|^4}\bigg] d\bars\\ 
%&\le \epsilon^{-1}4\pi\kappa_{\max}\big(c_5(1+ \epsilon 6c_Q)+2c_4\big) \norm{\bm{f}}_{C(\T)}.
%\end{aligned}
%\end{equation}
%
%As in the $\bm{F}_{\mc{D},13}$ estimate, we write $\bm{F}_{\mc{D},23}$ as
%\begin{align*}
%\bm{F}_{\mc{D},23} &= \bm{F}_{\mc{D},23a} + \bm{F}_{\mc{D},23b} + \bm{F}_{\mc{D},23c}; \\
%\bm{F}_{\mc{D},23a} &:= \int_0^{2\pi}\int_{-1/2}^{1/2} \epsilon^2\frac{2\be_{\rho}(\be_{\rho}\cdot\bm{f})}{|\bm{R}|^5}  d\bars \ts d\theta\\
%\bm{F}_{\mc{D},23b} &:= -\int_0^{2\pi}\int_{-1/2}^{1/2}\epsilon \frac{\bars\be_{\rho}(\be_t\cdot\bm{f}) }{|\bm{R}|^5} d\bars \ts d\theta\\
%\bm{F}_{\mc{D},23c} &:= \int_0^{2\pi}\int_{-1/2}^{1/2} \epsilon\frac{\bars^2\big[\be_{\rho}(\bm{Q}\cdot\bm{f})+(\bm{Q}\cdot\be_\rho)(\be_{\rho}\cdot\bm{f})\be_\rho\big] }{|\bm{R}|^5}  d\bars  \ts d\theta.
%\end{align*}
%
%First we estimate $\bm{F}_{\mc{D},23c}$. Lemma \ref{Rintest0} gives
%\begin{align*}
%\abs{\bm{F}_{\mc{D},23c}} \le 2\pi \norm{\bm{f}}_{C(\T)}\int_{-1/2}^{1/2} \epsilon\frac{2c_Q\bars^2}{|\bm{R}|^5} d\bars \le \epsilon^{-1}4\pi c_Q c_5\norm{\bm{f}}_{C(\T)}.
%\end{align*}
%
%Furthermore, using Lemma \ref{Rintest1}, we obtain the bound
%\begin{align*}
%\abs{\bm{F}_{\mc{D},23b}} &\le \epsilon^{-1}2\pi c_{1,5} \norm{\bm{f}}_{C^1(\T)}.
%\end{align*}
%
%Finally, noting that $\bm{F}_{\mc{D},23a}=\bm{F}_{\mc{D},13a}$, by Lemma \ref{Rintest2}, we have
%\begin{align*}
%\abs{\bm{F}_{\mc{D},23a}- \epsilon^{-2}\frac{8}{3}\bm{h}_f(s)} &\le \epsilon^{-1} 4\pi c_{0,5}\norm{\bm{f}}_{C^1(\T)},
%\end{align*}
%where $\bm{h}_f(s)$ was defined in \eqref{hf_def}. Altogether we obtain the estimate
%\begin{equation}\label{FD23_est}
%\abs{\bm{F}_{\mc{D},23}- \epsilon^{-2}\frac{8}{3}\bm{h}_f(s)} \le \epsilon^{-1}2\pi (2c_Qc_5+c_{1,5}+2c_{0,5})\norm{\bm{f}}_{C^1(\T)}.
%\end{equation}
%
%Next, by Lemma \ref{Rintest0}, we have
%\begin{equation}\label{FD22_est}
%\abs{\bm{F}_{\mc{D},22}} \le 2\pi \norm{\bm{f}}_{C(\T)}\int_{-1/2}^{1/2} \epsilon\frac{6c_Q\bars^2}{|\bm{R}|^5} d\bars \le \epsilon^{-1}12\pi c_Q c_5\norm{\bm{f}}_{C(\T)}.
%\end{equation}
%
%Lastly we bound $\bm{F}_{\mc{D},21}$, following the same steps as in the estimate of $\bm{F}_{\mc{D},11}$. We first write 
%\begin{align*}
%\bm{F}_{\mc{D},21}&=  \bm{F}_{\mc{D},21a}+  \bm{F}_{\mc{D},21b}+ \bm{F}_{\mc{D},21c}; \\
% \bm{F}_{\mc{D},21a}&:=  \int_0^{2\pi}\int_{-1/2}^{1/2} \epsilon^2\bigg[\frac{({\bm f}\cdot\be_\rho)\be_\rho}{|\bm{R}|^5} -\frac{5\epsilon^2\be_\rho(\be_\rho\cdot\bm{f})}{|\bm{R}|^7} \bigg] d\bars \ts d\theta \\
% \bm{F}_{\mc{D},21b}&:=  \int_0^{2\pi}\int_{-1/2}^{1/2} 5\epsilon^3\frac{\bars\be_\rho(\be_t\cdot\bm{f})}{|\bm{R}|^7} d\bars \ts d\theta\\
% \bm{F}_{\mc{D},21c} &:=  -\int_0^{2\pi}\int_{-1/2}^{1/2} 5\epsilon^2 \bigg[\frac{\bars^3[-(\be_t\cdot\bm{f}) + \bars(\bm{Q}\cdot\bm{f})](\bm{Q}\cdot\be_\rho)\be_\rho }{|\bm{R}|^7} \\
% &\hspace{3cm} + \frac{\epsilon \bars^2[\be_\rho(\bm{Q}\cdot\bm{f})+(\be_\rho\cdot\bm{Q})(\be_\rho\cdot\bm{f})\be_\rho] }{|\bm{R}|^7}\bigg] d\bars \ts d\theta.
%\end{align*}
%
%We use Lemma \ref{Rintest0} to estimate $\bm{F}_{\mc{D},21c}$:
%\begin{align*}
%\abs{ \bm{F}_{\mc{D},21c}} &\le  2\pi \norm{\bm{f}}_{C(\T)} \int_{-1/2}^{1/2} 5c_Q\epsilon^2 \frac{|\bars|^3+ c_Q\bars^4+ 2\epsilon \bars^2}{|\bm{R}|^7} d\bars \le \epsilon^{-1}10\pi c_Qc_7(3+\epsilon c_Q)\norm{\bm{f}}_{C(\T)}.
%\end{align*}
%
%Next, by Lemma \ref{Rintest1}, we have that $\bm{F}_{\mc{D},21b}$ satisfies
%\begin{align*}
%\abs{ \bm{F}_{\mc{D},21b}}&\le \epsilon^{-1}10\pi c_{1,7} \norm{\bm{f}}_{C^1(\T)}.
%\end{align*}
%
%Finally, using Lemma \ref{Rintest2}, we can estimate $\bm{F}_{\mc{D},21a}$ as
%\begin{align*}
%\abs{\bm{F}_{\mc{D},21a} +\epsilon^{-1}4\bm{h}_f(s)} &\le \epsilon^{-1}2\pi (c_{0,5}+5c_{0,7})\norm{\bm{f}}_{C^1(\T)},
%\end{align*}
%where $\bm{h}_f(s)$ was defined in \eqref{hf_def}. In total we have that $\bm{F}_{\mc{D},21}$ satisfies
%\begin{equation}\label{FD21_est}
%\abs{\bm{F}_{\mc{D},21a} +\epsilon^{-2}4\bm{h}_f(s)} \le \epsilon^{-1}2\pi (5c_Qc_7(3+\epsilon c_Q)+5c_{1,7}+c_{0,5}+ 5c_{0,7})\norm{\bm{f}}_{C^1(\T)}.
%\end{equation}

Combining the estimates \eqref{FD24_est}, \eqref{FD23_est}, \eqref{FD22_est}, and \eqref{FD21_est}, we have that $\bm{F}_{\mc{D},2}$ satisfies
\begin{equation}\label{FD2_est}
\abs{\bm{F}_{\mc{D},2} +\epsilon^{-2}4\bm{h}_f(s)} \le \epsilon^{-1} c_\kappa \norm{\bm{f}}_{C^1(\T)}.
\end{equation}

Then, using the expression \eqref{force_2} for $\bm{f}^{\SB}_2$, along with the estimates \eqref{FS2_est} and \eqref{FD2_est}, we obtain the bound
\begin{equation}\label{f2sb_est0}
\abs{\bm{f}^{\SB}_2} \le \frac{1}{8\pi}\bigg( \abs{\bm{F}_{\mc{S},2} - 2\bm{h}_f(s)} + \frac{\epsilon^2}{2}\abs{\bm{F}_{\mc{D},2} +\epsilon^{-2}4\bm{h}_f(s)}\bigg) \le \epsilon c_\kappa \norm{\bm{f}}_{C^1(\T)}.
\end{equation}
\end{proof}


%%%%%%%%%%%%%%%%%%%%%%%%%%%%%%%%%%%%%%%%%%%%%%%%%%%%%%%%%%%%%%%%
%%%%%%%%%%%%%%%%%%%%%%%%%%%%%%%%%%%%%%%%%%%%%%%%%%%%%%%%%%%%%%%%
%%%%%%%%%%%%%%%%%%%%%%%%%%%%%%%%%%%%%%%%%%%%%%%%%%%%%%%%%%%%%%%%
%%%%%%%%%%%%%%%%%%%%%%%%%%%%%%%%%%%%%%%%%%%%%%%%%%%%%%%%%%%%%%%%
%%%%%%%%%%%%%%%%%%%%%%%%%%%%%%%%%%%%%%%%%%%%%%%%%%%%%%%%%%%%%%%%
%%%%%%%%%%%%%%%%%%%%%%%%%%%%%%%%%%%%%%%%%%%%%%%%%%%%%%%%%%%%%%%%

A similar bound to Proposition \ref{fSB2_est} also holds for the next force component $\bm{f}^{\SB}_3(s)$. 
\begin{proposition}\label{fSB3_est}
Let the slender body $\Sigma_\epsilon$ be as in Section \ref{geometric_constraints}. Given $\bm{f}\in C^1(\T)$, let $\bm{f}^{\SB}_3(s)$ be defined as in \eqref{force_components}. Then $\bm{f}^{\SB}_3$ satisfies the bound
\begin{equation}
\abs{\bm{f}^{\SB}_3 } \le \epsilon c_\kappa\norm{\bm{f}}_{C^1(\T)}
\end{equation}
where the constant $c_\kappa$ depends only on $c_{\Gamma}$ and $\kappa_{\max}$.
\end{proposition}

%%%
\begin{proof}
Following the same steps as in the calculations of $\bm{f}^{\SB}_1$ and $\bm{f}^{\SB}_2$, we use \eqref{SBT2} in the expression \eqref{force_components} for $\bm{f}^{\SB}_3$ to consider $\bm{f}^{\SB}_3$ as the sum 
\begin{equation}\label{force_3}
\begin{aligned}
\bm{f}^{\SB}_3 &=\frac{1}{8\pi}\bigg(\bm{F}_{\mc{S},3} +\frac{\epsilon^2}{2}\bm{F}_{\mc{D},3}\bigg); \\
\bm{F}_{\mc{S},3} &:= -\int_0^{2\pi}\int_{-1/2}^{1/2} \be_\theta \bigg(\frac{\p}{\p\theta}\mc{S}(\bm{R})\bm{f}(s+\bars)\bigg)\cdot\be_\rho \ts d\bars \ts (1-\epsilon\wh\kappa)d\theta \\
\bm{F}_{\mc{D},3}&:=  -\int_0^{2\pi}\int_{-1/2}^{1/2} \be_\theta \bigg(\frac{\p}{\p\theta}\mc{D}(\bm{R})\bm{f}(s+\bars)\bigg)\cdot\be_\rho \ts d\bars \ts (1-\epsilon\wh\kappa) d\theta.
\end{aligned}
\end{equation}

As before, we begin by estimating $\bm{F}_{\mc{S},3}$. We write
\begin{equation}\label{FS3}
\begin{aligned}
\bm{F}_{\mc{S},3} &= \bm{F}_{\mc{S},31} + \bm{F}_{\mc{S},32} + \bm{F}_{\mc{S},33} ; \\
%
\bm{F}_{\mc{S},31} &= \int_0^{2\pi}\int_{-1/2}^{1/2} \bigg[\frac{\bm{R}_0\cdot\be_{\theta}}{|\bm{R}|^3}({\bm f}\cdot\be_{\rho})\be_\theta +\frac{3(\bm{R}\cdot\be_\rho)(\bm{R}\cdot\bm{f})(\bm{R}_0\cdot\be_{\theta})\be_\theta}{|\bm{R}|^5} \bigg] d\bars \ts \epsilon \ts d\theta \\
%
\bm{F}_{\mc{S},32} &= -\int_0^{2\pi}\int_{-1/2}^{1/2} \frac{\be_\theta(\bm{R}\cdot\be_\rho)(\be_{\theta}\cdot\bm{f})}{|\bm{R}|^3} d\bars \ts \epsilon \ts d\theta \\
%
\bm{F}_{\mc{S},33} &= -\int_0^{2\pi}\int_{-1/2}^{1/2} \bigg[\frac{\bm{R}_0\cdot\be_{\theta}}{|\bm{R}|^3}({\bm f}\cdot\be_{\rho})\be_\theta - \frac{\be_\theta(\bm{R}\cdot\be_\rho)(\be_{\theta}\cdot\bm{f})}{|\bm{R}|^3} \\
&\hspace{5cm} +\frac{3(\bm{R}\cdot\be_\rho)(\bm{R}\cdot\bm{f})(\bm{R}_0\cdot\be_{\theta})\be_\theta}{|\bm{R}|^5} \bigg] d\bars \ts \epsilon^2\wh\kappa \ts d\theta.
\end{aligned}
\end{equation}

As in the previous estimates of $\bm{F}_{\mc{S},1}$ and $\bm{F}_{\mc{S},2}$, we first use Lemma \ref{Rintest0} to show
\begin{equation}\label{FS33_est}
\abs{\bm{F}_{\mc{S},33}}  \le \epsilon c_\kappa\norm{\bm{f}}_{C(\T)}.
\end{equation}
Next, by Lemmas \ref{theta_int} and \ref{Rintest2}, we have
\begin{equation}\label{FS32_est}
\begin{aligned}
\abs{\bm{F}_{\mc{S},32} + 2\bm{h}_b(s)} &\le \epsilon c_\kappa\norm{\bm{f}}_{C^1(\T)}; \\
\bm{h}_b(s) &:=  \pi \big((\bm{f}(s)\cdot\be_{n_1}(s))\be_{n_1}(s) + (\bm{f}(s)\cdot\be_{n_2}(s))\be_{n_2}(s) \big).
\end{aligned}
\end{equation}
Finally, using \eqref{Rlb} along with Lemmas \ref{Rintest0} and \ref{theta_int}, we have
\begin{equation}\label{FS31_est}
\abs{\bm{F}_{\mc{S},31}} \le \epsilon c_\kappa \norm{\bm{f}}_{C(\T)}.
\end{equation}

%%%%%
%First, noting that $\bm{R}_0\cdot\be_{\theta}=\bars^2\bm{Q}\cdot\be_\theta$, we can use Lemma \ref{Rintest0} to bound $\bm{F}_{\mc{S},33}$ as
%\begin{equation}\label{FS33_est}
%\begin{aligned}
%\abs{\bm{F}_{\mc{S},33}} &\le 2\pi\norm{\bm{f}}_{C(\T)} \int_{-1/2}^{1/2} \epsilon^2 \abs{\wh\kappa} \bigg[\frac{4c_Q \bars^2}{|\bm{R}|^3}+ \frac{1}{|\bm{R}|^2}\bigg] d\bars\\
%& \le \epsilon4\pi\kappa_{\max} (c_2+\epsilon\abs{\log\epsilon}4c_Qc_3)\norm{\bm{f}}_{C(\T)}.
%\end{aligned}
%\end{equation}
%
%Next we estimate $\bm{F}_{\mc{S},32}$. Using \eqref{Reps}, we write
%\begin{align*}
%\bm{F}_{\mc{S},32} &= \bm{F}_{\mc{S},32a}+\bm{F}_{\mc{S},32b}; \\
%\bm{F}_{\mc{S},32a}&:= -\int_0^{2\pi}\int_{-1/2}^{1/2} \epsilon^2 \frac{\be_\theta(\be_{\theta}\cdot\bm{f})}{|\bm{R}|^3} d\bars \ts d\theta \\
%\bm{F}_{\mc{S},32b} &:= -\int_0^{2\pi}\int_{-1/2}^{1/2} \epsilon \frac{\bars^2\be_\theta(\bm{Q}\cdot\be_\rho)(\be_{\theta}\cdot\bm{f})}{|\bm{R}|^3} d\bars \ts d\theta.
%\end{align*}
%
%Since $\bm{Q}$, $\be_t$, and $\bm{f}$ are all independent of $\theta$, by Lemma \ref{theta_int} and the remark about integration against triples of the form $(\bm{A}(\bars)\cdot\be_\rho)(\bm{B}(\bars)\cdot\be_\theta)\be_\theta$, we have
%\begin{align*}
%\abs{\bm{F}_{\mc{S},32b}} &\le \epsilon^{3/2}6c_Q\bar c_3 \norm{\bm{f}}_{C(\T)}.
%\end{align*}
%
%Also, by Lemma \ref{Rintest2}, we obtain the following estimate for $\bm{F}_{\mc{S},32a}$:
%\begin{equation}\label{hb_def}
%\begin{aligned}
%\abs{\bm{F}_{\mc{S},32a} + 2\bm{h}_b(s)} &\le \epsilon 2\pi c_{0,3}\norm{\bm{f}}_{C^1(\T)}; \\
%\bm{h}_b(s) &:= \int_0^{2\pi} \be_\theta(s,\theta)\big(\be_{\theta}(s,\theta)\cdot\bm{f}(s)\big) \ts d\theta \\
%&= \pi \big((\bm{f}(s)\cdot\be_{n_1}(s))\be_{n_1}(s) + (\bm{f}(s)\cdot\be_{n_2}(s))\be_{n_2}(s) \big).
%\end{aligned}
%\end{equation}
%We note that in fact $\bm{h}_b(s)=\bm{h}_f(s)$, but we will not need to make use of this observation. Together, we have that $\bm{F}_{\mc{S},32}$ satisfies
%\begin{equation}\label{FS32_est}
%\abs{\bm{F}_{\mc{S},32} + 2\bm{h}_b(s)} \le \epsilon (2\pi c_{0,3}+\sqrt{\epsilon}6c_Q\bar c_3)\norm{\bm{f}}_{C^1(\T)}.
%\end{equation}
%
%Lastly, to estimate $\bm{F}_{\mc{S},31}$, using \eqref{Reps} and that $(\bm{R}_0\cdot\be_\theta)=\bars^2(\bm{Q}\cdot\be_\theta)$, we write
%\begin{align*}
%\bm{F}_{\mc{S},31} &= \bm{F}_{\mc{S},31a} +\bm{F}_{\mc{S},31b} +\bm{F}_{\mc{S},31c}; \\
%\bm{F}_{\mc{S},31a} &:= \int_0^{2\pi}\int_{-1/2}^{1/2} \epsilon\bigg[\frac{\bars^2(\bm{Q}\cdot\be_{\theta})({\bm f}\cdot\be_{\rho})\be_\theta}{|\bm{R}|^3} +\frac{3\epsilon^2\bars^2(\bm{Q}\cdot\be_{\theta})({\bm f}\cdot\be_{\rho})\be_\theta}{|\bm{R}|^5} \bigg] d\bars \ts d\theta \\
%%
%\bm{F}_{\mc{S},31b}&:= -\int_0^{2\pi}\int_{-1/2}^{1/2} 3\epsilon^2 \frac{\bars^3(\be_t\cdot\bm{f})(\bm{Q}\cdot\be_{\theta})\be_\theta}{|\bm{R}|^5}  d\bars \ts d\theta \\
%%
%\bm{F}_{\mc{S},31c}&:=\int_0^{2\pi}\int_{-1/2}^{1/2} 3\epsilon \bigg[\frac{\bars^5\big(-(\bm{Q}\cdot\be_\rho)(\be_t\cdot\bm{f}) +\bars(\bm{Q}\cdot\bm{f})(\bm{Q}\cdot\be_\rho)\big)(\bm{Q}\cdot\be_{\theta})\be_\theta}{|\bm{R}|^5}  \\
%&\hspace{4cm} +\frac{\epsilon\bars^4[(\bm{Q}\cdot\bm{f})+(\bm{Q}\cdot\be_\rho)(\be_\rho\cdot\bm{f})](\bm{Q}\cdot\be_{\theta})\be_\theta}{|\bm{R}|^5} \bigg] d\bars \ts d\theta.
%\end{align*}
%
%We estimate $\bm{F}_{\mc{S},31c}$ in the same way that we estimated $\bm{F}_{\mc{S},12c}$ and $\bm{F}_{\mc{S},22c}$. In particular, we have
%\begin{align*}
%\abs{\bm{F}_{\mc{S},31c}}&\le 2\pi \norm{\bm{f}}_{C(\T)} \int_{-1/2}^{1/2} 3c_Q^2\epsilon \frac{|\bars|^5 + c_Q\bars^6 +2\epsilon \bars^4}{|\bm{R}|^5} d\bars \ts d\theta \\
%&\le \epsilon 3\pi c_Q^2\norm{\bm{f}}_{C(\T)}\big(c_R^{-5}(2 + c_Q) + \epsilon\abs{\log\epsilon}2c_5 \big),
%\end{align*}
%where we have estimated the first two terms using the lower bound \eqref{Rlb} on $\bm{R}$ along with $\abs{\bars}\le \frac{1}{2}$, and we have used Lemma \ref{Rintest0}to bound the third term. \\
%
%We estimate $\bm{F}_{\mc{S},31b}$ via Lemma \ref{Rintest0}, obtaining the bound
%\begin{align*}
%\abs{\bm{F}_{\mc{S},31b}}&\le 2\pi \norm{\bm{f}}_{C(\T)} \int_{-1/2}^{1/2} 3c_Q\epsilon^2 \frac{|\bars|^3}{|\bm{R}|^5}  d\bars \le \epsilon 6\pi c_Q c_5\norm{\bm{f}}_{C(\T)}.
%\end{align*}
%
%Finally, we can bound $\bm{F}_{\mc{S},31a}$ using Lemma \ref{theta_int}; in particular, the remark about integration against triples of the form $(\bm{A}(\bars)\cdot\be_\rho)(\bm{B}(\bars)\cdot\be_\theta)\be_\theta$. We then have
%\begin{align*}
%\abs{\bm{F}_{\mc{S},31a}} &\le \epsilon^{3/2} 6c_Q(\bar c_3+3\bar c_5)\norm{\bm{f}}_{C(\T)}.
%\end{align*}
%
%Altogether we can estimate $\bm{F}_{\mc{S},31}$ as
%\begin{equation}\label{FS31_est}
%\abs{\bm{F}_{\mc{S},31}} \le \epsilon \big(3\pi c_Q^2(c_R^{-5}(2 + c_Q) + \epsilon\abs{\log\epsilon}2c_5)+6\pi c_Qc_5+\sqrt{\epsilon}6c_Q(\bar c_3+3\bar c_5) \big)\norm{\bm{f}}_{C(\T)}.
%\end{equation}

Combining the estimates \eqref{FS33_est}, \eqref{FS32_est}, and \eqref{FS31_est}, we have that $\bm{F}_{\mc{S},3}$ satisfies
\begin{equation}\label{FS3_est}
\abs{\bm{F}_{\mc{S},3}+2\bm{h}_b(s)} \le \epsilon c_\kappa \norm{\bm{f}}_{C^1(\T)},
\end{equation}
where $\bm{h}_b(s)$ is as in \eqref{FS32_est}. \\

%%%%%%%%
Now we estimate the doublet term $\bm{F}_{\mc{D},3}$ in the expression \eqref{force_3} for $\bm{f}^{\SB}_3$. As we did for the Stokeslet term, we decompose $\bm{F}_{\mc{D},3}$ as
\begin{equation}\label{FD3}
\begin{aligned}
\bm{F}_{\mc{D},3} &= 3(\bm{F}_{\mc{D},31} + \bm{F}_{\mc{D},32} + \bm{F}_{\mc{D},33}) ; \\
%
\bm{F}_{\mc{D},31} &= \int_0^{2\pi}\int_{-1/2}^{1/2} \bigg[\frac{\bm{R}_0\cdot\be_{\theta}}{|\bm{R}|^5}({\bm f}\cdot\be_{\rho})\be_\theta -\frac{5(\bm{R}\cdot\be_\rho)(\bm{R}\cdot\bm{f})(\bm{R}_0\cdot\be_{\theta})\be_\theta}{|\bm{R}|^7} \bigg] d\bars \ts \epsilon \ts d\theta \\
%
\bm{F}_{\mc{D},32} &= \int_0^{2\pi}\int_{-1/2}^{1/2} \frac{\be_\theta(\bm{R}\cdot\be_\rho)(\be_{\theta}\cdot\bm{f})}{|\bm{R}|^5} d\bars \ts \epsilon \ts d\theta \\
%
\bm{F}_{\mc{D},33} &= -\int_0^{2\pi}\int_{-1/2}^{1/2} \bigg[\frac{\bm{R}_0\cdot\be_{\theta}}{|\bm{R}|^5}({\bm f}\cdot\be_{\rho})\be_\theta + \frac{\be_\theta(\bm{R}\cdot\be_\rho)(\be_{\theta}\cdot\bm{f})}{|\bm{R}|^5} \\
&\hspace{5cm} -\frac{5(\bm{R}\cdot\be_\rho)(\bm{R}\cdot\bm{f})(\bm{R}_0\cdot\be_{\theta})\be_\theta}{|\bm{R}|^7} \bigg] d\bars \ts \epsilon^2\wh\kappa \ts d\theta.
\end{aligned}
\end{equation}

We estimate each of these integrals using the same procedure as each of the previous force term estimates. By Lemma \ref{Rintest0}, we have
\begin{equation}\label{FD33_est}
\abs{\bm{F}_{\mc{D},33}} \le \epsilon^{-1}c_\kappa\norm{\bm{f}}_{C(\T)},
\end{equation}
while Lemmas \ref{Rintest0} and \ref{Rintest2} give
\begin{equation}\label{FD32_est}
\abs{\bm{F}_{\mc{D},32} - \epsilon^{-2}\frac{4}{3}\bm{h}_b(s)} \le \epsilon^{-1}c_\kappa \norm{\bm{f}}_{C^1(\T)}
\end{equation}
for $\bm{h}_b(s)$ as in \eqref{FS32_est}. Finally, by Lemma \ref{Rintest0}, we have
\begin{equation}\label{FD31_est}
\abs{\bm{F}_{\mc{D},31}} \le \epsilon^{-1} c_\kappa \norm{\bm{f}}_{C(\T)}.
\end{equation}

%%%%%
%As before, we can immediately estimate $\bm{F}_{\mc{D},33}$ via Lemma \ref{Rintest0}. We have
%\begin{equation}\label{FD33_est}
%\begin{aligned}
%\abs{\bm{F}_{\mc{D},33}} &\le 2\pi \norm{\bm{f}}_{C(\T)}\int_{-1/2}^{1/2} \epsilon^2\abs{\wh\kappa}\bigg[\frac{6c_Q\bars^2}{|\bm{R}|^5} + \frac{1}{|\bm{R}|^4} \bigg] d\bars \\
%&\le \epsilon^{-1}4\pi\kappa_{\max}(c_4 + \epsilon 6c_Qc_5)\norm{\bm{f}}_{C(\T)}.
%\end{aligned}
%\end{equation}
%
%Next, as usual, we rewrite $\bm{F}_{\mc{D},32}$ as
%\begin{align*}
%\bm{F}_{\mc{D},32} &= \bm{F}_{\mc{D},32a} + \bm{F}_{\mc{D},32b}; \\
%\bm{F}_{\mc{D},32a} &:= \int_0^{2\pi}\int_{-1/2}^{1/2}\epsilon^2 \frac{\be_\theta(\be_{\theta}\cdot\bm{f})}{|\bm{R}|^5} d\bars \ts d\theta \\
%\bm{F}_{\mc{D},32b} &:= \int_0^{2\pi}\int_{-1/2}^{1/2} \epsilon \frac{\bars^2\be_\theta(\bm{Q}\cdot\be_\rho)(\be_{\theta}\cdot\bm{f})}{|\bm{R}|^5} d\bars \ts d\theta.
%\end{align*}
%
%Using Lemma \ref{Rintest0}, we have
%\begin{align*}
%\abs{\bm{F}_{\mc{D},32b}} &\le 2\pi \norm{\bm{f}}_{C(\T)}\int_{-1/2}^{1/2} \epsilon c_Q \frac{\bars^2}{|\bm{R}|^5} d\bars \le \epsilon^{-1}2\pi c_Q c_5\norm{\bm{f}}_{C(\T)}.
%\end{align*}
%
%Also, by Lemma \ref{Rintest2}, we have that $\bm{F}_{\mc{D},32a}$ satisfies
%\begin{align*}
%\abs{\bm{F}_{\mc{D},32a} - \epsilon^{-2}\frac{4}{3}\bm{h}_b(s)} &\le \epsilon^{-1}2\pi c_{0,5} \norm{\bm{f}}_{C^1(\T)},
%\end{align*}
%where $\bm{h}_b(s)$ was defined in \eqref{hb_def}. Together, we obtain
%\begin{equation}\label{FD32_est}
%\abs{\bm{F}_{\mc{D},32} - \epsilon^{-2}\frac{4}{3}\bm{h}_b(s)} \le \epsilon^{-1}2\pi(c_Qc_5+c_{0,5}) \norm{\bm{f}}_{C^1(\T)}.
%\end{equation}
%
%Finally, we estimate $\bm{F}_{\mc{D},31}$. Using Lemma \ref{Rintest0}, we have
%\begin{equation}\label{FD31_est}
%\abs{\bm{F}_{\mc{D},31}} \le 2\pi \norm{\bm{f}}_{C(\T)} \int_{-1/2}^{1/2} \epsilon \frac{6c_Q\bars^2}{|\bm{R}|^5} d\bars \le \epsilon^{-1} 12\pi c_Q\norm{\bm{f}}_{C(\T)}.
%\end{equation}

Altogether, the estimates \eqref{FD33_est}, \eqref{FD32_est}, and \eqref{FD31_est} yield
\begin{equation}\label{FD3_est}
\abs{\bm{F}_{\mc{D},3} - \epsilon^{-2}4\bm{h}_b(s)} \le \epsilon^{-1}c_\kappa \norm{\bm{f}}_{C^1(\T)},
\end{equation}
where $\bm{h}_b(s)$ is as in \eqref{FS32_est}. \\

Using the estimates \eqref{FS3_est} and \eqref{FD3_est}, together with the expression \eqref{force_3} for $\bm{f}^{\SB}_3$, we obtain the bound
\begin{equation}\label{f3sb_est0}
\abs{\bm{f}^{\SB}_3} \le \frac{1}{8\pi}\bigg( \abs{\bm{F}_{\mc{S},3} + 2\bm{h}_b(s)} + \frac{\epsilon^2}{2}\abs{\bm{F}_{\mc{D},3} -\epsilon^{-2}4\bm{h}_b(s)}\bigg) \le \epsilon c_\kappa \norm{\bm{f}}_{C^1(\T)}.
\end{equation}

\end{proof}


%%%%%%%%%%%%%%%%%%%%%%%%%%%%%%%%%%%%%%%%%%%%%%%%%%%%%%%%%%%%%%%%
%%%%%%%%%%%%%%%%%%%%%%%%%%%%%%%%%%%%%%%%%%%%%%%%%%%%%%%%%%%%%%%%
%%%%%%%%%%%%%%%%%%%%%%%%%%%%%%%%%%%%%%%%%%%%%%%%%%%%%%%%%%%%%%%%
%%%%%%%%%%%%%%%%%%%%%%%%%%%%%%%%%%%%%%%%%%%%%%%%%%%%%%%%%%%%%%%%
%%%%%%%%%%%%%%%%%%%%%%%%%%%%%%%%%%%%%%%%%%%%%%%%%%%%%%%%%%%%%%%%
%%%%%%%%%%%%%%%%%%%%%%%%%%%%%%%%%%%%%%%%%%%%%%%%%%%%%%%%%%%%%%%%
It remains to estimate the final term $\bm{f}^{\SB}_4(s)$ of the slender body force expression \eqref{force_components}. We show that $\bm{f}^{\SB}_4(s)$ satisfies the following proposition.

\begin{proposition}\label{fSB4_est}
Let the slender body $\Sigma_\epsilon$ be as in Section \ref{geometric_constraints}. Given $\bm{f}\in C^1(\T)$, let $\bm{f}^{\SB}_4(s)$ be as defined in \eqref{force_components}. We have that $\bm{f}^{\SB}_4$ satisfies the estimate
\begin{equation}
\abs{\bm{f}^{\SB}_4 } \le \epsilon c_\kappa \norm{\bm{f}}_{C^1(\T)},
\end{equation}
where the constant $c_\kappa$ depends only on $c_{\Gamma}$ and $\kappa_{\max}$.
\end{proposition}

\begin{proof}
This estimate follows quickly from Proposition \ref{prop:uSBstheta}. Noting that $\be_\rho(s,\theta) = \cos\theta\be_{n_1}(s)+\sin\theta\be_{n_2}(s) = -\p/\p\theta \be_\theta(s,\theta)$, we have that, using the expression for $\bm{f}^{\SB}_4(s)$ in \eqref{force_components} and integrating by parts in $\theta$, $\bm{f}^{\SB}_4(s)$ can be written 
\begin{equation}\label{force_4}
\begin{aligned}
\bm{f}^{\SB}_4 & =-\frac{1}{8\pi} \int_0^{2\pi}\int_{-1/2}^{1/2} \be_t(s) \bigg(\frac{\p\bu^{\SB}}{\p s} - \kappa_3 \frac{\p \bu^{\SB}}{\p\theta}\bigg)\cdot\be_\rho(s,\theta) \ts d\bars \ts \epsilon \ts d\theta \\
&= -\frac{1}{8\pi} \int_0^{2\pi}\int_{-1/2}^{1/2} \be_t(s) \frac{\p}{\p\theta}\bigg(\frac{\p\bu^{\SB}}{\p s} - \kappa_3 \frac{\p \bu^{\SB}}{\p\theta}\bigg)\cdot\be_\theta(s,\theta) \ts d\bars \ts \epsilon \ts d\theta.
\end{aligned}
\end{equation} 
Then, by Proposition \ref{prop:uSBstheta}, we have
\begin{align*}
\abs{\bm{f}^{\SB}_4} &\le \frac{\epsilon}{8\pi}\int_0^{2\pi}\int_{-1/2}^{1/2} \bigg|\frac{\p}{\p\theta}\bigg(\frac{\p\bu^{\SB}}{\p s} - \kappa_3 \frac{\p \bu^{\SB}}{\p\theta}\bigg)\bigg| \ts d\bars d\theta \le \epsilon c_\kappa \norm{\bm{f}}_{C^1(\T)}.
\end{align*}

\end{proof}
%As with the previous slender body force components, we use the expression \eqref{force_components} for $\bm{f}^{\SB}_4(s)$ and \eqref{SBT2} to write $\bm{f}^{\SB}_4(s)$ as the sum of a Stokeslet and a doublet term:
%\begin{equation}\label{force_4}
%\begin{aligned}
%\bm{f}^{\SB}_4 & =\frac{1}{8\pi}\bigg(\bm{F}_{\mc{S},4} +\frac{\epsilon^2}{2}\bm{F}_{\mc{D},4}\bigg); \\
%\bm{F}_{\mc{S},4} &:= \int_0^{2\pi}\int_{-1/2}^{1/2} \frac{\be_t}{1-\epsilon\wh\kappa} \bigg(\bigg(\frac{\p \mc{S}(\bm{R})}{\p s} - \kappa_3 \frac{\p \mc{S}(\bm{R})}{\p\theta}\bigg)\bm{f}(s+\bars)\bigg)\cdot\be_\rho \ts d\bars \ts \epsilon(1-\epsilon\wh\kappa)d\theta \\
%\bm{F}_{\mc{D},4} &:= \int_0^{2\pi}\int_{-1/2}^{1/2} \frac{\be_t}{1-\epsilon\wh\kappa} \bigg(\bigg(\frac{\p \mc{D}(\bm{R})}{\p s} - \kappa_3 \frac{\p \mc{D}(\bm{R})}{\p\theta}\bigg)\bm{f}(s+\bars)\bigg)\cdot\be_\rho \ts d\bars \ts \epsilon(1-\epsilon\wh\kappa)d\theta.
%\end{aligned}
%\end{equation}
%
%Following the same outline as in the previous calculations, we begin by estimating the Stokeslet term $\bm{F}_{\mc{S},4}$. We again decompose $\bm{F}_{\mc{S},4}$ into three terms:
%\begin{equation}\label{FS4}
%\begin{aligned}
%\bm{F}_{\mc{S},4} &= -(\bm{F}_{\mc{S},41} + \bm{F}_{\mc{S},42} + \bm{F}_{\mc{S},43}) ; \\
%%
%\bm{F}_{\mc{S},41} &= \int_0^{2\pi}\int_{-1/2}^{1/2} \bigg[\frac{\bm{R}_0\cdot\be_t}{|\bm{R}|^3}({\bm f}\cdot\be_\rho)\be_t +\frac{3\be_t(\bm{R}\cdot\be_\rho)(\bm{R}\cdot\bm{f})(\bm{R}_0\cdot\be_t)}{|\bm{R}|^5} \bigg] d\bars \ts \epsilon \ts d\theta \\
%%
%\bm{F}_{\mc{S},42} &= -\int_0^{2\pi}\int_{-1/2}^{1/2} \frac{(\bm{R}\cdot\be_\rho)(\be_t\cdot\bm{f})\be_t}{|\bm{R}|^3} d\bars \ts \epsilon \ts d\theta \\
%%
%\bm{F}_{\mc{S},43} &= -\int_0^{2\pi}\int_{-1/2}^{1/2} \bigg[\frac{\bm{R}_0\cdot\be_t}{|\bm{R}|^3}({\bm f}\cdot\be_\rho)\be_t - \frac{(\bm{R}\cdot\be_\rho)(\be_t\cdot\bm{f})\be_t}{|\bm{R}|^3} \\
%&\hspace{5cm} +\frac{3\be_t(\bm{R}\cdot\be_\rho)(\bm{R}\cdot\bm{f})(\bm{R}_0\cdot\be_t)}{|\bm{R}|^5} \bigg] d\bars \ts \epsilon^2 \wh\kappa \ts d\theta.
%\end{aligned}
%\end{equation}
%
%Again, using Lemma \ref{Rintest0} along with the identity $\bm{R}_0\cdot\be_t = -\bars+\bars^2(\bm{Q}\cdot\be_t)$, we obtain
%\begin{equation}\label{FS43_est}
%\begin{aligned}
%\abs{\bm{F}_{\mc{S},43}} &\le 2\pi \norm{\bm{f}}_{C(\T)}\int_{-1/2}^{1/2} \epsilon^2\abs{\wh\kappa}\bigg[4\frac{|\bars|+c_Q\bars^2}{|\bm{R}|^3}+ \frac{1}{|\bm{R}|^2} \bigg] d\bars \\
%&\le \epsilon4\pi\kappa_{\max}(4c_3(1+ \epsilon\abs{\log\epsilon}c_Q)+c_2)\norm{\bm{f}}_{C(\T)}.
%\end{aligned}
%\end{equation}
%
%For $\bm{F}_{\mc{S},42}$, we again use \eqref{Reps} to rewrite the expression as
%\begin{align*}
%\bm{F}_{\mc{S},42} &= \bm{F}_{\mc{S},42a}+\bm{F}_{\mc{S},42b}; \\
%\bm{F}_{\mc{S},42a} &:= -\int_0^{2\pi}\int_{-1/2}^{1/2} \epsilon^2 \frac{(\be_t\cdot\bm{f})\be_t}{|\bm{R}|^3} d\bars \ts d\theta \\
%\bm{F}_{\mc{S},42b} &:= -\int_0^{2\pi}\int_{-1/2}^{1/2} \epsilon \frac{\bars^2(\bm{Q}\cdot\be_\rho)(\be_t\cdot\bm{f})\be_t}{|\bm{R}|^3} d\bars \ts d\theta.
%\end{align*}
%
%Now, noting the $\theta$-independence of $\bm{Q}(s,\bars)$, $\bm{f}(s+\bars)$, and $\be_t(s)$, as well as the single copy of $\be_\rho(s,\theta)=\cos\theta\be_{n_1}(s)+\sin\theta\be_{n_2}(s)$ in $\bm{F}_{\mc{S},42b}$, we use Lemma \ref{theta_int} to obtain the estimate 
%\begin{align*}
%\abs{\bm{F}_{\mc{S},42b}} &\le \epsilon^{3/2} 8 c_Q \bar c_3 \norm{\bm{f}}_{C(\T)} .
%\end{align*}
%
%Next, using Lemma \ref{Rintest2}, we have
%\begin{equation}\label{hc_def}
%\begin{aligned}
%\abs{\bm{F}_{\mc{S},42a} + 2\bm{h}_c(s)} &\le \epsilon 2\pi c_{0,3}\norm{\bm{f}}_{C^1(\T)}; \\
%\bm{h}_c(s) &:= \int_0^{2\pi}\be_t(s)\big(\be_t(s)\cdot\bm{f}(s)\big) \ts d\theta \\
%&= 2\pi \be_t(s)\big(\be_t(s)\cdot\bm{f}(s)\big).
%\end{aligned}
%\end{equation}
%
%Together, we obtain the estimate
%\begin{equation}\label{FS42_est}
%\abs{\bm{F}_{\mc{S},42} + 2\bm{h}_c(s)} \le \epsilon 2(\pi c_{0,3}+\sqrt{\epsilon}4c_Q\bar c_3)\norm{\bm{f}}_{C^1(\T)}.
%\end{equation}
%
%Lastly, we use $(\bm{R}_0\cdot\be_t)=-\bars + \bars^2(\bm{Q}\cdot\be_t)$ and \eqref{Reps} to rewrite $\bm{F}_{\mc{S},41}$ as 
%\begin{align*}
%\bm{F}_{\mc{S},41} &= \bm{F}_{\mc{S},41a} + \bm{F}_{\mc{S},41b} + \bm{F}_{\mc{S},41c}+ \bm{F}_{\mc{S},41d}; \\
%%
%\bm{F}_{\mc{S},41a} &:= -\int_0^{2\pi}\int_{-1/2}^{1/2} \epsilon\bigg[\frac{\bars({\bm f}\cdot\be_\rho)\be_t}{|\bm{R}|^3} + 3\frac{\epsilon^2\bars(\be_\rho\cdot\bm{f})\be_t }{|\bm{R}|^5} \bigg] d\bars \ts d\theta \\
%%
%\bm{F}_{\mc{S},41b} &:= \int_0^{2\pi}\int_{-1/2}^{1/2} 3\epsilon^2 \frac{\bars^2(\be_t\cdot\bm{f})\be_t }{|\bm{R}|^5} d\bars \ts d\theta \\
%%
%\bm{F}_{\mc{S},41c} &:= \int_0^{2\pi}\int_{-1/2}^{1/2} \epsilon \bigg[\frac{\bars^2(\bm{Q}\cdot\be_t)({\bm f}\cdot\be_\rho)\be_t}{|\bm{R}|^3} + \frac{3\bars^4(\bm{Q}\cdot\be_\rho)(\be_t\cdot\bm{f})\be_t}{|\bm{R}|^5} \bigg] d\bars \ts d\theta \\
%%
%\bm{F}_{\mc{S},41d} &:= \int_0^{2\pi}\int_{-1/2}^{1/2} 3\epsilon \bigg[\frac{\big(\bars^6(\bm{Q}\cdot\be_\rho)(\bm{Q}\cdot\bm{f}) -\bars^5(\be_t\cdot\bm{f})\big)(\bm{Q}\cdot\be_t)\be_t-\bars^5(\bm{Q}\cdot\be_\rho)(\bm{Q}\cdot\bm{f})\be_t}{|\bm{R}|^5} \\
%&\hspace{2cm} -\frac{\epsilon \bars^3[(\bm{Q}\cdot\be_\rho)(\be_\rho\cdot\bm{f})+(\bm{Q}\cdot\bm{f})]\be_t}{|\bm{R}|^5} +\frac{\epsilon\bars^2[-\bars(\be_t\cdot\bm{f})+\epsilon(\be_\rho\cdot\bm{f}) ](\bm{Q}\cdot\be_t)\be_t}{|\bm{R}|^5} \\
%&\hspace{3cm} +\frac{\epsilon\bars^4[(\bm{Q}\cdot\be_\rho)(\be_\rho\cdot\bm{f})+(\bm{Q}\cdot\bm{f})](\bm{Q}\cdot\be_t)\be_t}{|\bm{R}|^5} \bigg] d\bars \ts d\theta.
%\end{align*}
%
%
%As in the estimates for $\bm{F}_{\mc{S},12c}$, $\bm{F}_{\mc{S},22c}$, and $\bm{F}_{\mc{S},31c}$, we make use of the lower bound \eqref{Rlb} and the fact that $\abs{\bars}\le\frac{1}{2}$ to obtain
%\begin{align*}
%\abs{\bm{F}_{\mc{S},41d}} &\le 6c_Q\pi \norm{\bm{f}}_{C(\T)} \int_{-1/2}^{1/2} \epsilon \frac{c_Q^2\bars^6 +(1+ c_Q)|\bars|^5+ 2c_Q\epsilon\bars^4+ 3\epsilon |\bars|^3+ \epsilon^2\bars^2}{|\bm{R}|^5} d\bars \\
%&\le \epsilon3c_Q\pi \norm{\bm{f}}_{C(\T)}\bigg(c_R^{-5}\big(c_Q^2+ 2(1+c_Q) \big) + \int_{-1/2}^{1/2} \frac{2c_Q\epsilon\bars^4+ 3\epsilon |\bars|^3+ \epsilon^2\bars^2}{|\bm{R}|^5} d\bars \bigg) \\
%&\le \epsilon3c_Q\pi \norm{\bm{f}}_{C(\T)}\bigg(c_R^{-5}\big(c_Q^2+ 2(1+c_Q) \big) + c_5(\epsilon\abs{\log\epsilon}2c_Q+ 4) \bigg)
%\end{align*}
%where we also used Lemma \ref{Rintest0} in the last inequality. Next, noting the $\theta$-independence of $\bm{Q}$, $\bm{f}$, and $\be_t$, and recalling $\be_\rho=\cos\theta\be_{n_1}(s)+\sin\theta\be_{n_2}(s)$, we use Lemma \ref{theta_int} to estimate $\bm{F}_{\mc{S},41c}$:
%\begin{align*}
%\abs{\bm{F}_{\mc{S},41c}} &\le \epsilon^{3/2}2c_Q(\bar c_3 +3\bar c_5) \norm{\bm{f}}_{C(\T)}.
%\end{align*}
%
%Furthermore, using Lemma \ref{Rintest2}, we can estimate $\bm{F}_{\mc{S},41b}$ as
%\begin{align*}
%\abs{\bm{F}_{\mc{S},41b} - 2\bm{h}_c(s)} &\le \epsilon 6\pi c_{0,5}\norm{\bm{f}}_{C^1(\T)}.
%\end{align*}
%
%Finally, by Lemma \ref{Rintest1}, we have the bound 
%\begin{align*}
%\abs{\bm{F}_{\mc{S},41a}} &\le \epsilon 2\pi(c_{1,3}+3c_{1,5}) \norm{\bm{f}}_{C^1(\T)}.
%\end{align*}
%
%Combining the above estimates, we obtain the following estimate for $\bm{F}_{\mc{S},41}$:
%\begin{equation}\label{FS41_est}
%\begin{aligned}
%\abs{\bm{F}_{\mc{S},41} - 2\bm{h}_c(s)} &\le \epsilon \bigg(c_Q3\pi\big(c_R^{-5}\big(c_Q^2+ 2(1+c_Q) \big) + c_5(\epsilon\abs{\log\epsilon}2c_Q+ 4) \big) \\
%&\hspace{2cm}+\epsilon^{1/2}2c_Q(\bar c_3 +3\bar c_5) +  6\pi c_{0,5}+2\pi(c_{1,3}+3c_{1,5}) \bigg) \norm{\bm{f}}_{C^1(\T)}.
%\end{aligned}
%\end{equation}
%
%Using the estimates \eqref{FS43_est}, \eqref{FS42_est}, and \eqref{FS41_est}, we can bound $\bm{F}_{\mc{S},4}$ as 
%\begin{equation}\label{FS4_est}
%\begin{aligned}
%\abs{\bm{F}_{\mc{S},4}} &\le \epsilon c_{\mc{S},4} \norm{\bm{f}}_{C^1(\T)},
%\end{aligned}
%\end{equation}
%where the constant $c_{\mc{S},4}$ depends only on $c_{\Gamma}$ and $\kappa_{\max}$. \\
%
%We conclude with an estimate of the doublet component $\bm{F}_{\mc{D},4}$ of $\bm{f}^{\SB}_4(s)$, defined in \eqref{force_4}. As in previous doublet computations, we write $\bm{F}_{\mc{D},4}$ as 
%\begin{equation}\label{FD4}
%\begin{aligned}
%\bm{F}_{\mc{D},4} &= -3(\bm{F}_{\mc{D},31} + \bm{F}_{\mc{D},32} + \bm{F}_{\mc{D},33}) ; \\
%%
%\bm{F}_{\mc{D},41} &= \int_0^{2\pi}\int_{-1/2}^{1/2} \bigg[\frac{\bm{R}_0\cdot\be_t}{|\bm{R}|^5}({\bm f}\cdot\be_\rho)\be_t -\frac{5\be_t(\bm{R}\cdot\be_\rho)(\bm{R}\cdot\bm{f})(\bm{R}_0\cdot\be_t)}{|\bm{R}|^7} \bigg] d\bars \ts \epsilon \ts d\theta \\
%%
%\bm{F}_{\mc{D},42} &= \int_0^{2\pi}\int_{-1/2}^{1/2} \frac{(\bm{R}\cdot\be_\rho)(\be_t\cdot\bm{f})\be_t}{|\bm{R}|^5} d\bars \ts \epsilon \ts d\theta \\
%%
%\bm{F}_{\mc{D},43} &= -\int_0^{2\pi}\int_{-1/2}^{1/2} \bigg[\frac{\bm{R}_0\cdot\be_t}{|\bm{R}|^5}({\bm f}\cdot\be_\rho)\be_t + \frac{(\bm{R}\cdot\be_\rho)(\be_t\cdot\bm{f})\be_t}{|\bm{R}|^5} \\
%&\hspace{4cm} -\frac{5\be_t(\bm{R}\cdot\be_\rho)(\bm{R}\cdot\bm{f})(\bm{R}_0\cdot\be_t)}{|\bm{R}|^7} \bigg] d\bars \ts \epsilon^2 \wh\kappa \ts d\theta.
%\end{aligned}
%\end{equation}
%
%Just as in the $\bm{F}_{\mc{S},43}$ estimate, a bound for $\bm{F}_{\mc{D},43}$ follows immediately from Lemma \ref{Rintest0}:
%\begin{equation}\label{FD43_est}
%\begin{aligned}
%\abs{\bm{F}_{\mc{D},43}} &\le 2\pi \norm{\bm{f}}_{C(\T)}\int_{-1/2}^{1/2} \epsilon^2\abs{\wh\kappa} \bigg[6\frac{\abs{\bars}+c_Q\bars^2}{|\bm{R}|^5} + \frac{1}{|\bm{R}|^4}\bigg] d\bars \\
%&\le \epsilon^{-1} 4\pi\kappa_{\max}\big(6c_5(1 + \epsilon\abs{\log\epsilon}c_Q) + c_4\big) \norm{\bm{f}}_{C(\T)}.
%\end{aligned}
%\end{equation}
%
%To estimate $\bm{F}_{\mc{D},42}$, we first use \eqref{Reps} to write
%\begin{align*}
%\bm{F}_{\mc{D},42} &= \bm{F}_{\mc{D},42a} + \bm{F}_{\mc{D},42b}; \\
%\bm{F}_{\mc{D},42a} &:= \int_0^{2\pi}\int_{-1/2}^{1/2} \epsilon^2\frac{(\be_t\cdot\bm{f})\be_t}{|\bm{R}|^5} d\bars \ts d\theta \\
%\bm{F}_{\mc{D},42b} &:= \int_0^{2\pi}\int_{-1/2}^{1/2}\epsilon \frac{\bars^2(\bm{Q}\cdot\be_\rho)(\be_t\cdot\bm{f})\be_t}{|\bm{R}|^5} d\bars \ts d\theta.
%\end{align*}
%
%Again by Lemma \ref{Rintest0}, we have that $\bm{F}_{\mc{D},42b}$ satisfies the bound
%\begin{align*}
%\abs{\bm{F}_{\mc{D},42b}} &\le 2\pi \norm{\bm{f}}_{C(\T)} \int_{-1/2}^{1/2}c_Q \epsilon \frac{\bars^2}{|\bm{R}|^5} d\bars \le \epsilon^{-1}2\pi c_Q c_5 \norm{\bm{f}}_{C(\T)}. 
%\end{align*}
%
%Furthermore, using Lemma \ref{Rintest2}, we estimate $\bm{F}_{\mc{D},42a}$ as
%\begin{align*}
%\abs{\bm{F}_{\mc{D},42a}- \epsilon^{-2}\frac{4}{3}\bm{h}_c(s)} &\le \epsilon^{-1}2\pi c_{0,5} \norm{\bm{f}}_{C^1(\T)}.
%\end{align*}
%where $\bm{h}(s)$ was defined in \eqref{hc_def}. Putting both of the above estimates together, we obtain
%\begin{equation}\label{FD42_est}
%\abs{\bm{F}_{\mc{D},42}- \epsilon^{-2}\frac{4}{3}\bm{h}_c(s)} \le \epsilon^{-1}2\pi(c_Qc_5+c_{0,5})\norm{\bm{f}}_{C^1(\T)}.
%\end{equation}
%
%Finally we estimate $\bm{F}_{\mc{D},41}$. Noting that $(\bm{R}_0\cdot\be_t) = -\bars + \bars^2(\bm{Q}\cdot\be_t)$, we rewrite the expression for $\bm{F}_{\mc{D},41}$ as
%\begin{align*}
%\bm{F}_{\mc{D},41} &= \bm{F}_{\mc{D},41a} + \bm{F}_{\mc{D},41b} + \bm{F}_{\mc{D},41c}+ \bm{F}_{\mc{D},41d}; \\
%%
%\bm{F}_{\mc{D},41a} &:= -\int_0^{2\pi}\int_{-1/2}^{1/2} \epsilon\bigg[\frac{\bars({\bm f}\cdot\be_\rho)\be_t}{|\bm{R}|^5} - 5\frac{\epsilon^2\bars(\be_\rho\cdot\bm{f})\be_t }{|\bm{R}|^7} \bigg] d\bars \ts d\theta \\
%%
%\bm{F}_{\mc{D},41b} &:= -\int_0^{2\pi}\int_{-1/2}^{1/2} 5\epsilon^2 \frac{\bars^2(\be_t\cdot\bm{f})\be_t }{|\bm{R}|^7} d\bars \ts d\theta \\
%%
%\bm{F}_{\mc{D},41c} &:= \int_0^{2\pi}\int_{-1/2}^{1/2} 5\epsilon \bigg[ \frac{\bars^5(\bm{Q}\cdot\be_\rho)(\bm{Q}\cdot\bm{f})\be_t - \bars^4(\bm{Q}\cdot\be_\rho)(\be_t\cdot\bm{f})\be_t}{|\bm{R}|^7} \\
%&\hspace{3cm} +\frac{\epsilon \bars^3[(\bm{Q}\cdot\be_\rho)(\be_\rho\cdot\bm{f})+(\bm{Q}\cdot\bm{f})]\be_t}{|\bm{R}|^7} \bigg] d\bars \ts d\theta \\
%%
%\bm{F}_{\mc{D},41d} &:= \int_0^{2\pi}\int_{-1/2}^{1/2} \epsilon \bigg[\frac{\bars^2(\bm{Q}\cdot\be_t)({\bm f}\cdot\be_\rho)\be_t}{|\bm{R}|^5} -\frac{5\bars^2\be_t(\bm{R}\cdot\be_\rho)(\bm{R}\cdot\bm{f})(\bm{Q}\cdot\be_t)}{|\bm{R}|^7} \bigg] d\bars \ts d\theta.
%\end{align*}
%
%First we bound $\bm{F}_{\mc{D},41d}$. By \ref{Rintest0}, we have 
%\begin{align*}
%\abs{\bm{F}_{\mc{D},41d}} &\le 2\pi \norm{\bm{f}}_{C(\T)} \int_{-1/2}^{1/2} 6c_Q\epsilon \frac{\bars^2}{|\bm{R}|^5} d\bars \le \epsilon^{-1} 12\pi c_Q c_5 \norm{\bm{f}}_{C(\T)}.
%\end{align*}
%
%To bound $\bm{F}_{\mc{D},41c}$, we again use Lemma \ref{Rintest0} to show
%\begin{align*}
%\abs{\bm{F}_{\mc{D},41c}} &\le 2\pi\norm{\bm{f}}_{C(\T)}\int_{-1/2}^{1/2} 5\epsilon c_Q\frac{c_Q|\bars|^5 + \bars^4+ \epsilon2\abs{\bars}^3}{|\bm{R}|^7} d\bars \le \epsilon^{-1}10\pi c_7c_Q(3+ \epsilon c_Q) \norm{\bm{f}}_{C(\T)}.
%\end{align*}
%
%We next estimate $\bm{F}_{\mc{D},41b}$. Using Lemma \ref{Rintest2}, we have 
%\begin{align*}
%\abs{\bm{F}_{\mc{D},41b} + \epsilon^{-2}\frac{4}{3}\bm{h}_c(s)} &\le \epsilon^{-1}10\pi c_{0,7} \norm{\bm{f}}_{C^1(\T)}, 
%\end{align*}
%with $\bm{h}_c(s)$ as defined in \eqref{hc_def}. \\
%
%Lastly, using Lemma \ref{Rintest1}, we can bound $\bm{F}_{\mc{D},41a}$ as
%\begin{align*}
%\abs{\bm{F}_{\mc{D},41a}} &\le \epsilon^{-1} 2\pi(c_{1,5}+ 5c_{1,7}) \norm{\bm{f}}_{C^1(\T)}.
%\end{align*}
%
%Then, in total, the term $\bm{F}_{\mc{D},41}$ satisfies the estimate
%\begin{equation}\label{FD41_est}
%\abs{\bm{F}_{\mc{D},41} + \epsilon^{-2}\frac{4}{3}\bm{h}_c(s)} \le \epsilon^{-1}2\pi \big(6c_Qc_5+ 5c_Qc_7(3+\epsilon c_Q)+5 c_{0,7} +c_{1,5}+5c_{1,7} \big)\norm{\bm{f}}_{C^1(\T)}.
%\end{equation}
%
%Combining \eqref{FD43_est}, \eqref{FD42_est}, and \eqref{FD41_est}, we thus obtain the following estimate for $\bm{F}_{\mc{D},4}$:
%\begin{equation}\label{FD4_est}
%\abs{\bm{F}_{\mc{D},4}} \le \epsilon^{-1} c_{\mc{D},4} \norm{\bm{f}}_{C^1(\T)},
%\end{equation}
%where the constant $c_{\mc{D},4}$ depends only on $c_{\Gamma}$ and $\kappa_{\max}$. \\
%
%Altogether, using the bounds \eqref{FS4_est} and \eqref{FD4_est} in the expression \eqref{force_4} for $\bm{f}^{\SB}_4$, we obtain the estimate 
%\begin{equation}\label{f4sb_est0}
%\abs{\bm{f}^{\SB}_4} \le \frac{1}{8\pi}\bigg( \abs{\bm{F}_{\mc{S},4}} + \frac{\epsilon^2}{2}\abs{\bm{F}_{\mc{D},4}}\bigg) \le \epsilon\bigg(c_{\mc{S},4} + \frac{1}{2}c_{\mc{D},4}\bigg)\norm{\bm{f}}_{C^1(\T)},
%\end{equation}
%from which follows Proposition \ref{fSB4_est}. 
%\end{proof}
%
%%%%%%%%%%%%%%%%%%%%%%%%%%%%%%%%%%%%%%%%%%%%%%%%%%%%%%%%%%%%%%%%
%%%%%%%%%%%%%%%%%%%%%%%%%%%%%%%%%%%%%%%%%%%%%%%%%%%%%%%%%%%%%%%%
%%%%%%%%%%%%%%%%%%%%%%%%%%%%%%%%%%%%%%%%%%%%%%%%%%%%%%%%%%%%%%%%

Finally, we sum the estimates for the five force components defined in \eqref{force_components}, resulting the in the following estimate for the total slender body force $\bm{f}^{\SB}(s)$. 
\begin{proposition}\label{fSB_est}
Let the slender body $\Sigma_\epsilon$ be as in Section \ref{geometric_constraints}. Given $\bm{f}\in C^1(\T)$, let $\bm{f}^{\SB}(s)$ be the corresponding slender body approximation, given by \eqref{fSB_expr}. Then $\bm{f}^{\SB}$ satisfies 
\begin{equation}
\abs{\bm{f}^{\SB}(s) - \bm{f}(s)} \le \epsilon c_\kappa \norm{\bm{f}}_{C^1(\T)},
\end{equation}
where the constant $c_\kappa$ depends only on $c_{\Gamma}$ and $\kappa_{\max}$.
\end{proposition}

\begin{proof}
First, we introduce some notation. Let $f_t(s) :=(\bm{f}(s)\cdot\be_t(s))$, $f_{n_1}(s) :=(\bm{f}(s)\cdot\be_{n_1}(s))$, and $f_{n_2}(s) :=(\bm{f}(s)\cdot\be_{n_2}(s))$. Using the expression \eqref{force_components} for $\bm{f}^{\SB}$, together with Propositions \ref{fSBp_est}, \ref{fSB1_est}, \ref{fSB2_est}, \ref{fSB3_est}, and \ref{fSB4_est}, we have
\begin{align*}
\abs{\bm{f}^{\SB}(s) - \bm{f}(s)} &= \abs{\bm{f}^{\SB}(s) - \frac{1}{2}\bm{f}(s) - \frac{1}{2}\big(f_t(s)\be_t(s) + f_{n_1}(s)\be_{n_1}(s) + f_{n_2}(s)\be_{n_2}(s)\big)} \\
%
&\le \abs{\bm{f}^{\SB}_p(s)- \frac{1}{2} \big(f_{n_1}(s)\be_{n_1}(s) + f_{n_2}(s)\be_{n_2}(s) \big)} \\
&\quad  +\abs{\bm{f}^{\SB}_1(s) - \frac{1}{2}\bigg(\bm{f}(s) + f_t(s)\be_t(s)\bigg) } +\abs{\bm{f}^{\SB}_2(s)} +\abs{\bm{f}^{\SB}_3(s)}+ \abs{\bm{f}^{\SB}_4(s)} \\
%
&\le \epsilon c_\kappa \norm{\bm{f}}_{C^1(\T)}.
\end{align*}

\end{proof}

%%%%%%%%%%%%%%%%%%%%%%%%%%%%%%%%%%%%%%%%%%%%%%%%%%%%%%%%%%%%%%%%
%%%%%%%%%%%%%%%%%%%%%%%%%%%%%%%%%%%%%%%%%%%%%%%%%%%%%%%%%%%%%%%%
%%%%%%%%%%%%%%%%%%%%%%%%%%%%%%%%%%%%%%%%%%%%%%%%%%%%%%%%%%%%%%%%
%%%%%%%%%%%%%%%%%%%%%%%%%%%%%%%%%%%%%%%%%%%%%%%%%%%%%%%%%%%%%%%%


\section{Error estimate}\label{error_est_section}
Using the residual calculations for the surface velocity $\bu^{\SB}\big|_{\Gamma_{\epsilon}}$ and the total surface force ${\bm f}^{\SB}$, we proceed to prove the error estimate \eqref{err_stokes_thm} in Theorem \ref{stokes_err_theorem}. \\

Let $\bu^{\text{e}}=\bu^{\SB}-\bu$, $p^{\text{e}}=p^{\SB}-p$, and $\bm{\sigma}_e= -p^{\text{e}}{\bf I}+2\E(\bu^{\text{e}})=\bm{\sigma}^{\SB}-\bm{\sigma}$, where $\bu$, $p$, and $\bm{\sigma}=-p{\bf I}+\nabla \bu +(\nabla \bu)^{\rm T}$ correspond to the true solution to \eqref{exterior_stokes}. Then the difference $\bu^{\text{e}}$ satisfies, in the weak sense,
\begin{equation}\label{err_PDE_stokes}
\begin{aligned}
-\Delta \bu^{\text{e}} + \nabla p^{\text{e}} &= 0 \\
\dive\ts \bu^{\text{e}} &= 0 \qquad \text{ in }\Omega_{\epsilon} \\
\int_0^{2\pi} (\bm{\sigma}_e {\bm n}) \ts \mathcal{J}_{\epsilon}(s,\theta)\ts d\theta &= {\bm f}^{\text{e}}(s) \quad \text{on } \Gamma_{\epsilon} \\
\bu^{\text{e}}|_{\Gamma_{\epsilon}} &= \bar \bu^{\text{e}}(s)+\bu^{\rm r}(s,\theta) \\
\bu^{\text{e}} \to 0 &\text{ as }|\bx| \to \infty
\end{aligned}
\end{equation}

where the boundary value $\bar \bu^{\text{e}}(s)=\big(\bu^{\SB}-\bu^{\rm r}\big)\big|_{\Gamma_\epsilon}(s)-\bu\big|_{\Gamma_\epsilon}(s)$ is unknown (since $\bu(s)$ is unknown) but independent of $\theta$. Note that ${\bm f}^{\text{e}}(s)={\bm f}^{\SB}-{\bm f}$ and $\bu^{\rm r}(s,\theta)= \bu^{\rm SB}(\epsilon,\theta,s) - \frac{1}{2\pi}\int_0^{2\pi} \bu^{\SB}(\epsilon,\varphi,s) \ts d\varphi$ \eqref{ur} are both completely known functions along $\Gamma_{\epsilon}$. \\

%%%%%%%%

More precisely, for arbitrary $\bw\in D^{1,2}(\Omega_{\epsilon})$, the error $\bu^{\text{e}}$ satisfies
\begin{equation}\label{variational_err_eqn}
\int_{\Omega_{\epsilon}} \bigg(2 \ts \E(\bu^{\text{e}}):\E(\bw) - p^{\text{e}} \ts\dive\ts\bw\bigg) \ts d\bx = \int_{\Gamma_{\epsilon}} (\bm{\sigma}_e {\bm n})\cdot \bw \ts dS.
\end{equation}
Now, unless $\bw\in \A_{\epsilon} = \big\{ \bw\in D^{1,2}(\Omega_{\epsilon}) \ts: \ts \bw|_{\Gamma_{\epsilon}}=\bw(s) \big\}$, i.e. $\bw$ additionally satisfies the $\theta$-independence condition on the slender body surface $\Gamma_{\epsilon}$, we cannot make use of the known expression ${\bm f}^{\text{e}}(s)$ for the error in the total force. Note in particular that the function $\bu^{\text{e}}$ itself does not belong to the set $\A_{\epsilon}$. \\

However, since $(\bu^{\text{e}},p^{\text{e}})$ satisfies \eqref{variational_err_eqn}, we can exactly follow the proof of the pressure estimate \eqref{press_est} to show that the pressure error $p^{\text{e}}$ satisfies 
\begin{equation}\label{press_err_est}
\|p^{\text{e}}\|_{L^2(\Omega_{\epsilon})} \le c_P\|\E(\bu^{\text{e}})\|_{L^2(\Omega_{\epsilon})}
\end{equation}
where $c_P$ is independent of the slender body radius $\epsilon$.\\

To derive a $D^{1,2}(\Omega_{\epsilon})$ bound for $\bu^{\text{e}}$, we use \eqref{variational_err_eqn} with a very specific choice of $\bw$. In particular, we take 
\begin{equation}
\bw= \widetilde\bu^{\text{e}}:= \bu^{\text{e}}- \widetilde\bv,
\end{equation}
where $\widetilde\bv\in D^{1,2}(\Omega_{\epsilon})$ with $\widetilde\bv\big|_{\Gamma_{\epsilon}} = \bu^{\rm r}(s,\theta)$. We explicitly construct such a $\widetilde\bv$ in Section \ref{v_construct} that we can bound in terms of ${\bm f}(s)$, the true prescribed force. \\

We then have that $\widetilde\bu^{\text{e}}\big|_{\Gamma_{\epsilon}}=\bar\bu^{\text{e}}(s)$, where $\bar \bu^{\text{e}}(s)$ is unknown but independent of $\theta$, so $\widetilde\bu^{\text{e}}\in \A_{\epsilon}$. Thus, using $\widetilde\bu^{\text{e}}$ in place of $\bw$ in \eqref{variational_err_eqn}, we obtain
\begin{equation}\label{variational_err_ee}
\int_{\Omega_{\epsilon}} \bigg(2 \ts \E(\bu^{\text{e}}):\E(\widetilde\bu^{\text{e}}) - p^{\text{e}} \ts\dive\ts\widetilde\bu^{\text{e}} \bigg)\ts d\bx = \int_{\T} {\bm f}^{\text{e}}(s)\cdot \bar\bu^{\text{e}}(s) \ts ds.
\end{equation}

From \eqref{variational_err_ee} we will derive a $D^{1,2}(\Omega_{\epsilon})$ estimate for $\bu^{\text{e}}$ in terms of the prescribed force ${\bm f}(s)$. 

%%%%%%%%%%%%%%%%%%
\subsection{Construction of {\boldmath $\widetilde v$}}\label{v_construct}
In order to use \eqref{variational_err_ee} to obtain an estimate for $\bu^{\text{e}}$ in terms of ${\bm f}(s)$, we must construct the function $\widetilde\bv\in D^{1,2}(\Omega_{\epsilon})$ with $\widetilde\bv\big|_{\Gamma_{\epsilon}} = \bu^{\rm r}(s,\theta)$. We first define 
\[ \bu^{\SB}_{\text{ext}}(\rho,\theta,s) = \begin{cases}
\bu^{\rm r}(\theta,s) & \qquad \text{if } \rho < 4\epsilon, \\
0 & \qquad \text{otherwise}.
\end{cases} \]

Since $\bu^{\rm r}$ is at least $C^1$, $\nabla \bu^{\SB}_{\text{ext}}$ is continuous within $\rho < 4\epsilon$. \\

Let $\phi(\rho)$ be a smooth cutoff function equal to 1 for $\rho < 2$ and equal to 0 for $\rho > 4$ with smooth decay between. We require this decay to satisfy 
\begin{equation}\label{phi_estimate}
\abs{\frac{\p\phi}{\p\rho}}\le c_\phi
\end{equation}
for some constant $c_{\phi}>0$. Let $\phi_\epsilon(\rho) := \phi(\rho/\epsilon)$. \\

We define
\begin{equation}\label{corr_stokes}
\widetilde\bv(\rho,\theta,s) = \phi_\epsilon(\rho) \bu_{\text{ext}}^{\SB}(\rho,\theta,s). 
\end{equation}
Note that $\widetilde\bv(\rho,\theta,s)$ is supported in the region 
\begin{equation}\label{v_support}
\mathcal{O}_{\epsilon} := \big\{ s\be_t(s)+ \rho\be_{\rho}(s,\theta)+ \theta\be_{\theta}(s,\theta) \ts : \ts s\in\T, \ts \epsilon\le \rho\le 4\epsilon, \ts 0\le\theta<2\pi \big\}
\end{equation}
with $|\mathcal{O}_{\epsilon}| = c_{\mc{O}}^2\epsilon^2$. \\

Now, obtaining a $D^{1,2}(\Omega_{\epsilon})$ estimate for $\bu^{\text{e}}$ from \eqref{variational_err_ee} will require an $L^2(\Omega_{\epsilon})$ bound for $\nabla \widetilde\bv$, so we consider
\begin{equation}\label{grad_eSB_stokes}
\nabla \widetilde\bv = \phi_\epsilon \nabla \bu_{\text{ext}}^{\SB}+ (\nabla\phi_\epsilon) (\bu_{\text{ext}}^{\SB})^{\rm T}.
\end{equation}

We have 
\begin{align*}
\phi \nabla \bu_{\text{ext}}^{\SB} &= \frac{\phi_\epsilon}{\epsilon} \frac{\p \bu^{\rm r}}{\p\theta} \be_{\theta}^{\rm T} + \frac{\phi_\epsilon}{1-\rho\wh\kappa}\bigg(\frac{\p\bu^{\rm r}}{\p s}-\kappa_3 \frac{\p\bu^{\rm r}}{\p\theta}\bigg) \be_t^{\rm T} 
\end{align*}
and
\begin{align*}
 (\nabla\phi_\epsilon) (\bu_{\text{ext}}^{\SB})^{\rm T} &= \frac{\p \phi_\epsilon}{\p \rho} \be_{\rho} (\bu^{\rm r})^{\rm T}.
\end{align*}

Finally, notice that
\begin{equation}\label{denom_bound}
\begin{aligned}
|1-\rho\wh\kappa| & \ge 1- \rho |\kappa|(\cos\theta+\sin\theta) &\ge 1- \frac{1}{2 \kappa_{\max}}|\kappa|\sqrt{2} \ge 1- \frac{\sqrt{2}}{2} \ge \frac{1}{4}.
\end{aligned}
\end{equation}

Then, using Proposition \ref{ur_and_derivs} along with \eqref{phi_estimate}, \eqref{denom_bound}, and Lemma \ref{lemmaorthonormal}, we have 
\begin{equation}\label{grad_vSB_est}
\begin{aligned}
\|\nabla \widetilde\bv\|_{L^2(\Omega_{\epsilon})} &\le \|\nabla \widetilde\bv\|_{C(\mathcal{O}_{\epsilon})} \sqrt{|\mathcal{O}_{\epsilon}|} \\
&\le \epsilon c_\kappa\bigg(\bigg\|\frac{1}{\epsilon} \frac{\p \bu^{\rm r}}{\p\theta}\bigg\|_{C(\Gamma_{\epsilon})} + \bigg\|\frac{1}{1-\rho\wh\kappa}\bigg(\frac{\p\bu^{\rm r}}{\p s}-\kappa_3 \frac{\p\bu^{\rm r}}{\p\theta}\bigg)\bigg\|_{C(\Gamma_{\epsilon})}+ \frac{1}{\epsilon}\|\bu^{\rm r}\|_{C(\Gamma_{\epsilon})}\bigg) \\
&\le  c_\kappa \epsilon\abs{\log\epsilon}\|{\bm f}\|_{C^1(\T)}.
\end{aligned}
\end{equation}


%%%%%%%%%%%%%%%%%%
%%%%%%%%%%%%%%%%%%
%%%%%%%%%%%%%%%%%%
\subsection{Estimating the error}
We now use \eqref{variational_err_ee} to obtain a $D^{1,2}(\Omega_{\epsilon})$ bound for the error $\bu^{\text{e}}$.
Recalling that $\widetilde\bu^{\text{e}}=\bu^{\text{e}}-\widetilde\bv$ and thus $\dive\ts \widetilde\bu^{\text{e}}=-\dive\ts \widetilde\bv$, we rewrite \eqref{variational_err_ee} as
\begin{equation}\label{variational_err_2}
\begin{aligned}
\int_{\Omega_{\epsilon}} 2 \ts |\E(\bu^{\text{e}})|^2 \ts d\bx &= \int_{\Omega_{\epsilon}} \bigg(2 \ts \E(\bu^{\text{e}}):\E(\widetilde\bv) - p^{\text{e}} \ts\dive\ts\widetilde\bv \bigg)\ts d\bx + \int_{\T} {\bm f}^{\text{e}}(s) \cdot\bar\bu^{\text{e}}(s) \ts ds \\
&\le \bigg|\int_{\Omega_{\epsilon}} 2 \ts \E(\bu^{\text{e}}):\E(\widetilde\bv)\ts d\bx\bigg| + \bigg|\int_{\Omega_{\epsilon}} p^{\text{e}} \ts\dive\ts\widetilde\bv \ts d\bx \bigg| + \bigg|\int_{\T} {\bm f}^{\text{e}}(s) \cdot\bar\bu^{\text{e}}(s) \ts ds\bigg|.
\end{aligned}
\end{equation}

Using Cauchy Schwarz, the first term on the right hand side of \eqref{variational_err_2} satisfies
\begin{align*}
 \bigg|\int_{\Omega_{\epsilon}} 2 \ts \E(\bu^{\text{e}}):\E(\widetilde\bv) \ts d\bx \bigg| &\le 2 \|\E(\bu^{\text{e}})\|_{L^2(\Omega_{\epsilon})}\|\E(\widetilde\bv)\|_{L^2(\Omega_{\epsilon})} \le \eta \|\E(\bu^{\text{e}}) \|_{L^2(\Omega_{\epsilon})}^2 + \frac{1}{\eta}\|\E(\widetilde\bv)\|_{L^2(\Omega_{\epsilon})}^2 \\
&\le \eta \|\E(\bu^{\text{e}}) \|_{L^2(\Omega_{\epsilon})}^2 + \frac{1}{\eta}\|\nabla \widetilde\bv\|_{L^2(\Omega_{\epsilon})}^2 
\end{align*}
for any $\eta\in \R_+$. \\

By \eqref{press_err_est} and Cauchy Schwarz, the second term on the right hand side of \eqref{variational_err_2} satisfies
\begin{align*}
\bigg| \int_{\Omega_{\epsilon}} p^{\text{e}} \ts\dive\ts\widetilde\bv \ts d\bx \bigg| &\le \|p^{\text{e}}\|_{L^2(\Omega_{\epsilon})}\|\nabla \widetilde\bv\|_{L^2(\Omega_{\epsilon})} \le c_P\|\E(\bu^{\text{e}})\|_{L^2(\Omega_{\epsilon})}\|\nabla \widetilde\bv\|_{L^2(\Omega_{\epsilon})} \\
 &\le \eta \|\E(\bu^{\text{e}}) \|_{L^2(\Omega_{\epsilon})}^2 + \frac{c_P^2}{4\eta}\|\nabla \widetilde\bv\|_{L^2(\Omega_{\epsilon})}^2.
\end{align*}

Finally, the third term on the right hand side of \eqref{variational_err_2} can be estimated using the trace inequality \eqref{Trace_ineq} on the admissible set $\A_{\epsilon}$, the Korn inequality \eqref{korn_ineq}, and Cauchy Schwarz. We have 
\begin{align*}
\bigg| \int_{\T} {\bm f}^{\text{e}}(s)\cdot\bar\bu^{\text{e}}(s) \ts ds \bigg| &\le \|{\bm f}^{\text{e}}\|_{L^2(\T)}\| \bar\bu^{\text{e}}\|_{L^2(\T)} \le c_T\|\nabla \widetilde\bu^{\text{e}}\|_{L^2(\Omega_{\epsilon})}\|{\bm f}^{\text{e}}\|_{L^2(\T)}\\
&\le c_Tc_K\|\E(\widetilde\bu^{\text{e}})\|_{L^2(\Omega_{\epsilon})}\|{\bm f}^{\text{e}}\|_{L^2(\T)} \le \eta\|\E(\widetilde\bu^{\text{e}})\|_{L^2(\Omega_{\epsilon})}^2 + \frac{c_T^2c_K^2}{4\eta}\|{\bm f}^{\text{e}}\|_{L^2(\T)}^2\\
&\le \eta\|\E(\bu^{\text{e}})\|_{L^2(\Omega_{\epsilon})}^2 + \eta\|\nabla \widetilde\bv\|_{L^2(\Omega_{\epsilon})}^2 + \frac{c_T^2c_K^2}{4\eta}\|{\bm f}^{\text{e}}\|_{L^2(\T)}^2,
\end{align*}
again for any $\eta\in \R_+$.  \\

Taking $\eta=\frac{1}{3}$, we obtain the following estimate from \eqref{variational_err_2}:
\begin{equation}\label{weak_err_3}
\|\mathcal{E}(\bu^{\text{e}})\|_{L^2(\Omega_{\epsilon})}^2 \le \frac{3c_T^2c_K^2}{4}\|{\bm f}^{\text{e}}\|_{L^2(\T)}^2 + \bigg(6+ \frac{3c_P^2}{4}\bigg) \|\nabla \widetilde\bv\|_{L^2(\Omega_{\epsilon})}^2.
\end{equation}

Then using the Korn inequality \eqref{korn_ineq}, we have
\begin{equation}\label{weak_err_4}
\|\nabla\bu^{\text{e}} \|_{L^2(\Omega_{\epsilon})}^2 \le \frac{3c_T^2c_K^4}{4}\|{\bm f}^{\text{e}}\|_{L^2(\T)}^2 + c_K^2\bigg(6+ \frac{3c_P^2}{4}\bigg) \|\nabla \widetilde\bv\|_{L^2(\Omega_{\epsilon})}^2.
\end{equation}

Recall that the Korn constant $c_K$ \eqref{korn_ineq} and the pressure constant $c_P$ \eqref{press_est} are both independent of $\epsilon$, while the trace constant $c_T$ \eqref{Trace_ineq} satisfies $c_T=c_{\kappa}|\log\epsilon|^{1/2}$. Also, from \eqref{grad_vSB_est} and Proposition \ref{fSB_est}, we have
\begin{align*}
\|\nabla\widetilde\bv\|_{L^2(\Omega_{\epsilon})} &\le \epsilon|\log\epsilon| c_\kappa  \|{\bm f}\|_{C^1(\T)} \\ 
\|{\bm f}^{\text{e}}\|_{L^2(\T)} &\le \epsilon c_\kappa  \|{\bm f}\|_{C^1(\T)}.
\end{align*}

Therefore we have 
\begin{equation}\label{err_stokes}
\begin{aligned}
\|\bu^{\text{e}}\|_{D^{1,2}(\Omega_{\epsilon})} &\le \epsilon(|\log\epsilon|^{1/2}+|\log\epsilon|)c_{\kappa} \|{\bm f}\|_{C^1(\T)} \\ 
&\le \epsilon |\log\epsilon| c_{\kappa}  \|{\bm f}\|_{C^1(\T)}
\end{aligned}
\end{equation}
where the constant $c_{\kappa}$ depends only on the shape of the fiber centerline through $\kappa_{\max}$ and $c_\Gamma$. Since the pressure error $p^{\text{e}}$ satisfies \eqref{press_err_est}, we also obtain
\begin{equation}\label{err_stokes}
\|\bu^{\text{e}}\|_{D^{1,2}(\Omega_{\epsilon})}+ \|p^{\text{e}}\|_{L^2(\Omega_{\epsilon})} \le \epsilon|\log\epsilon| c_{\kappa} \|{\bm f}\|_{C^1(\T)},
\end{equation}
where again, by Lemma \ref{divv_p_lem}, $c_{\kappa}$ depends only on $\kappa_{\max}$ and $c_\Gamma$.\\

Furthermore, using the $D^{1,2}(\Omega_{\epsilon})$ bound on the error $\bu^{\text{e}}=\bu^{\SB}-\bu$ throughout the fluid domain $\Omega_{\epsilon}$. We first write
%, we can obtain an $L^2$ bound for the trace of the error ${\rm Tr}(\bu^{\text{e}})$ along the slender body surface $\Gamma_{\epsilon}$, scaled by the square root of the slender body surface area $|\Gamma_{\epsilon}|^{1/2}$. We scale the trace on $\Gamma_{\epsilon}$ by $|\Gamma_{\epsilon}|^{-1/2}=\frac{1}{\sqrt{\epsilon}}$ to distinguish the actual error from the fact that the surface area vanishes as $\epsilon\to 0$, so the $L^2$ trace on $\Gamma_{\epsilon}$ always scales like $\sqrt{\epsilon}$. We first write
\begin{align*}
\| {\rm Tr} (\bu^{\text{e}})\|_{L^2(\Gamma_{\epsilon})} \le \|\bar\bu^{\text{e}}(s)\|_{L^2(\Gamma_{\epsilon})}+ \|\bu^{\rm r}\|_{L^2(\Gamma_{\epsilon})}.
\end{align*}

Then, using the estimate \eqref{urest} for $\bu^{\rm r}$, we have 
\begin{align*}
\|\bu^{\rm r}\|_{L^2(\Gamma_{\epsilon})} &= \bigg(\int_{\T}\int_0^{2\pi} |\bu^{\rm r}(s,\theta)|^2 \ts \mathcal{J}_{\epsilon}(s,\theta) \ts d\theta ds\bigg)^{1/2}\\
& \le \sqrt{2}\abs{\Gamma_\epsilon}^{1/2} \|\bu^{\rm r}\|_{C(\Gamma_{\epsilon})} \le \epsilon|\log\epsilon|\ts c_\kappa \abs{\Gamma_\epsilon}^{1/2} \|{\bm f}\|_{C^1(\T)},
\end{align*}
where $|\Gamma_{\epsilon}|$ denotes the fiber surface area. \\ 

Moreover, using the trace inequality \eqref{Trace_ineq} and \eqref{err_stokes}, we have 
\begin{align*}
\|\bar\bu^{\text{e}}(s)\|_{L^2(\Gamma_{\epsilon})} &\le \abs{\Gamma_\epsilon}^{1/2}\|\bar\bu^{\text{e}}\|_{L^2(\T)} \le \abs{\Gamma_\epsilon}^{1/2} \ts c_T\|\nabla \widetilde\bu^{\text{e}}\|_{L^2(\Omega_{\epsilon})} \\
&\le \abs{\Gamma_\epsilon}^{1/2} \ts c_T \big(\|\nabla \bu^{\text{e}}\|_{L^2(\Omega_{\epsilon})} + \|\nabla \widetilde\bv\|_{L^2(\Omega_{\epsilon})} \big)  \\
&= c_{\kappa} \abs{\Gamma_\epsilon}^{1/2}|\log\epsilon|^{1/2}\big(\|\bu^{\text{e}}\|_{D^{1,2}(\Omega_{\epsilon})} + \|\nabla \widetilde\bv\|_{L^2(\Omega_{\epsilon})} \big) \\
&\le \epsilon|\log\epsilon|^{3/2} \ts \abs{\Gamma_\epsilon}^{1/2}c_{\kappa} \ts \|{\bm f}\|_{C^1(\T)},
\end{align*}
where the constant $c_\kappa$ depends only on $\kappa_{\max}$ and $c_{\Gamma}$. \\

In total, scaling by $|\Gamma_{\epsilon}|^{-1/2}$, we obtain
\begin{equation}\label{trace_err_bound}
\frac{1}{|\Gamma_{\epsilon}|^{1/2}}\| {\rm Tr} (\bu^{\text{e}})\|_{L^2(\Gamma_{\epsilon})} \le \epsilon |\log\epsilon|^{3/2} \ts c_{\kappa} \ts \|{\bm f}\|_{C^1(\T)}.
\end{equation}

Using \eqref{trace_err_bound}, we may finally estimate the error in the slender body centerline velocity approximation \eqref{SBT_asymp}, allowing us to obtain the estimate \eqref{center_err_thm} in Theorem \ref{stokes_err_theorem}. We first note that, by Proposition \ref{centerline_prop}, the difference between the surface velocity approximation $\bu^{\SB}(s,\theta)$ and the centerline velocity approximation $\bu^{\SB}_C(s)$ satisfies
\begin{align*}
\bigg(\int_{\T}\int_0^{2\pi}\abs{\bu^{\SB}(s,\theta)-\bu^{\SB}_C(s)}^2 \mc{J}_\epsilon(s,\theta) d\theta ds \bigg)^{1/2} &\le c_\kappa\epsilon\abs{\log\epsilon}\norm{\bm{f}}_{C^1(\T)} \bigg(\int_{\T}\int_0^{2\pi} \mc{J}_\epsilon(s,\theta) d\theta ds \bigg)^{1/2} \\
&= c_\kappa\epsilon\abs{\log\epsilon} \norm{\bm{f}}_{C^1(\T)} \abs{\Gamma_\epsilon}^{1/2}.
\end{align*}

Using the above estimate along with \eqref{trace_err_bound}, we then have that the difference between the true fiber velocity ${\rm Tr}(\bu)(s)$ and the centerline approximation $\bu^{\SB}_C(s)$ satisfies
\begin{align*}
\norm{{\rm Tr}(\bu) - \bu^{\SB}_C}_{L^2(\T)} &= \frac{1}{\abs{\Gamma_\epsilon}^{1/2}}\norm{{\rm Tr}(\bu) - \bu^{\SB}_C}_{L^2(\Gamma_\epsilon)} \\
& \le \frac{1}{\abs{\Gamma_\epsilon}^{1/2}} \bigg(\norm{{\rm Tr}(\bu) - {\rm Tr}(\bu^{\SB})}_{L^2(\Gamma_\epsilon)} + \norm{{\rm Tr}(\bu^{\SB}) - \bu^{\SB}_C}_{L^2(\Gamma_\epsilon)} \bigg)  \\
&\le (\epsilon\abs{\log\epsilon}^{3/2}+\epsilon\abs{\log\epsilon}) c_\kappa \norm{\bm{f}}_{C^1(\T)}.
\end{align*}


%%%%%%%%%%%%%%%%%%%%%%%%%%%%%%%%%%%%%%%%%%%%%%%%%%%%%%%%%%
%%%%%%%%%%%%%%%%%%%%%%%%%%%%%%%%%%%%%%%%%%%%%%%%%%%%%%%%%%
%%%%%%%%%%%%%%%%%%%%%%%%%%%%%%%%%%%%%%%%%%%%%%%%%%%%%%%%%%
%%%%%%%%%%%%%%%%%%%%%%%%%%%%%%%%%%%%%%%%%%%%%%%%%%%%%%%%%%
%%%%%%%%%%%%%%%%%%%%%%%%%%%%%%%%%%%%%%%%%%%%%%%%%%%%%%%%%%
%%%%%%%%%%%%%%%%%%%%%%%%%%%%%%%%%%%%%%%%%%%%%%%%%%%%%%%%%%
%%%%%%%%%%%%%%%%%%%%%%%%%%%%%%%%%%%%%%%%%%%%%%%%%%%%%%%%%%
\appendix
\section{Appendix}\label{appendix}
\subsection{Proof of Lemma \ref{lemmaorthonormal}}\label{moving_frame_pf}
Here we show the bound \eqref{kappa3} on the moving frame coefficient $\kappa_3$.  
\begin{proof}
Let $\be_t(s)$, $\widetilde\be_1(s)$, and $\widetilde\be_2(s)$ define a $C^1$ orthonormal frame satisfying
%\[\Gamma_t=\lbrace \bm{p}\in \mathbb{S}^2 \ts : \ts \bm{p}=\be_t(s) \text{ or } \bm{p}=-\bm{e}_t(s), \ts s\in \T \rbrace,\]
%where $\mathbb{S}^2\subset \R^3$ is the two-sphere. The above is the trajectory of the Gauss map and its reflection through the origin. Note that $\Gamma_t$ is a proper subset of $\mathbb{S}^2$. \\
%
%Pick a point $\bm{q}\in \mathbb{S}^2\backslash \Gamma_t$. Since $\be_t(s)$ is a continuous function and $\bm{q}$ is never equal to $\pm\be_t(s)$,
%\begin{equation}\label{etq<1}
%\max_{s\in \T} \big|\be_t(s)\cdot \bm{q}\big|<1.
%\end{equation}
%Let
%\[ \widetilde\be_{1}(s)=\frac{\bm{q}-(\be_t(s)\cdot \bm{q})\be_t(s)}{|\bm{q}-(\be_t(s)\cdot \bm{q})\be_t(s)|}, \; 
%\widetilde\be_{2}(s)=\be_t(s)\times \widetilde\be_1(s). \]
%By \eqref{etq<1}, the denominator in the expression for $\widetilde{\be}_1(s)$ never vanishes. Thus, $\be_t$, $\widetilde\be_1$, and $\widetilde\be_2$ define an orthonormal frame with $C^1$ dependence on $s$. \\

\begin{equation}\label{kappat12wh}
\frac{d}{d s}\begin{pmatrix} \be_t \\ \widetilde\be_1 \\ \widetilde\be_2 \end{pmatrix}
=\begin{pmatrix}
0 & \widetilde{\kappa}_1 & \widetilde{\kappa}_2 \\
-\widetilde{\kappa}_1 & 0 & \widetilde{\kappa}_3 \\
-\widetilde{\kappa}_2 & -\widetilde{\kappa}_3 & 0
\end{pmatrix}
\begin{pmatrix} \be_t \\ \widetilde\be_1 \\ \widetilde\be_2 \end{pmatrix}.
\end{equation}
Take
\[ \overline{\widetilde{\kappa}_3}=\int_0^1 \widetilde{\kappa}_3(s) \ts ds \]
and let $k$ be the closest integer to $\overline{\widetilde{\kappa}_3}/{2\pi}$. Define 
\begin{equation}\label{kappa3_def}
\kappa_3=\overline{\widetilde{\kappa}_3}-2\pi k; \qquad \varphi(s)=\int_0^s (\widetilde{\kappa}_3(\tau)-\kappa_3) \ts d\tau.
\end{equation}
Note that, by construction,
\[ |\kappa_3| \le \pi. \]
Define
\[\begin{pmatrix}
\be_{n_1}(s) \\ \be_{n_2}(s)
\end{pmatrix}
=\begin{pmatrix}
\cos\varphi(s) & -\sin\varphi(s)\\ \sin\varphi(s) &\cos\varphi(s)
\end{pmatrix}
\begin{pmatrix}
\widetilde{\be}_1(s) \\ \widetilde{\be}_2(s)
\end{pmatrix}. \]
Since $\varphi(1)=2\pi k$, $\be_{n_1}(s)$ and $\be_{n_2}(s)$ are both in $C^1(\T)$. It is also clear that $\be_t(s), \be_{n_1}(s)$ and $\be_{n_2}(s)$ define an orthonormal basis. A straightforward calculation shows that $\be_t(s), \be_{n_1}(s)$ and $\be_{n_2}(s)$ satisfy 
\eqref{moving_ODE} with $\kappa_3$ as in \eqref{kappa3_def} and
\[ \begin{pmatrix}
\kappa_1(s) \\ \kappa_2(s)
\end{pmatrix}
=\begin{pmatrix}
\cos\varphi(s) & -\sin\varphi(s)\\ \sin\varphi(s) &\cos\varphi(s)
\end{pmatrix}
\begin{pmatrix}
\widetilde{\kappa}_1(s) \\ \widetilde{\kappa}_2(s)
\end{pmatrix}. \]

\end{proof}


%%%%%%%%%%%%%%%%%%%%%%%%%%%%%%%%%%%%%%%%%%%%%%%%%%%%%%%%%%%%%%%
\subsection{Proof of $\epsilon$-dependence in well-posedness constants}\label{constants}
In this appendix, we prove the $\epsilon$-dependence claims for each of the inequalities stated in Section \ref{constants0}.

%%%%%%
\subsubsection{Trace inequality}\label{trace_sec}
We begin by proving the trace inequality for $\A_\epsilon$ functions stated in Lemma \ref{Trace_inequality}. We show that the trace constant $c_T$ is proportional to $\abs{\log\epsilon}^{1/2}$. 

\begin{proof}[Proof of Lemma \ref{Trace_inequality}]
Since the fiber centerline is $C^2$ and the fiber does not self-intersect \eqref{non_intersecting}, we can cover the slender body by finitely many open neighborhoods $W_j$ where 
\[ W_j = \{ \X(s)+\rho\be_\rho(s,\theta) \ts : \ts 0\le \theta < 2\pi, \ts 0 \le \rho < r_{\max}/2, \ts a_j < s < b_j\}, \quad j=1,\dots,N <\infty.\]
Here $a_j$ and $b_j$ are chosen such that over each $W_j$, the fiber centerline can be considered as the graph of a $C^2$ function. Note that this choice of $a_j$ and $b_j$ depends only on the shape of the fiber centerline -- in particular, $\kappa_{\max}$ and $c_{\Gamma}$ -- and not on the fiber radius. \\

Then, using a partition of unity $\{\phi_j\}_{j=1}^N$ subordinate to the cover $\{W_j\}$, there exist $\epsilon$-independent $C^2$ diffeomorphisms $\psi_j$, $j=1,\dots,N$ taking the curvature $\kappa$ of the fiber centerline to zero on the set $W_j$ while leaving the radius $\epsilon$ intact. \\ 

\begin{figure}[!h]
\centering
\includegraphics[scale=0.5]{diffeo_SB.png}\\
\caption{ The slender body centerline can be straightened via $\epsilon$-independent diffeomorphisms $\psi_j$; thus it suffices to consider functions $\bu$ around a straight cylinder supported within the truncated cylindrical shell $C_{\epsilon,a}$.}
\label{fig:diffeo_SB}
\end{figure}

Let $D_{\rho}\subset \R^2$ denote the open disk of radius $\rho$ in $\R^2$ centered at the origin. Define the straight cylindrical surface $\Gamma_{\epsilon,a}:= \p D_{\epsilon}\times [-a,a]$ and the cylindrical shell $C_{\epsilon,a}:= (D_1\backslash \overline{D_{\epsilon}})\times [-a,a]$ for some $a<\infty$, parameterized in cylindrical coordinates $(\rho,\theta,s)$. We define the function space 
\[\A_S:= \big\{ \bv\in D^{1,2}(C_{\epsilon,a}) \ts : \ts \bv|_{\Gamma_{\epsilon,a}} = \bv(s); \ts \bv|_{\p C_{\epsilon,a}\backslash\Gamma_{\epsilon,a}} = 0 \big\}. \]

Then $\psi_j^*(\phi_j\bu)\circ \psi_j\in \A_S$, and to show Lemma \ref{Trace_inequality} it suffices to prove the $\abs{\log\epsilon}^{1/2}$ dependence in the trace constant about a straight cylinder.
\begin{lemma}\label{trace_straight_cylinder}
Let $\bu\in \A_S$. Then the $\theta$-independent trace of $\bu$ on the straight cylinder $\Gamma_{\epsilon,a}$ satisfies
\begin{equation}\label{cylinder_trace}
\|{\rm Tr}(\bu)\|_{L^2(-a,a)} \le \frac{1}{2\pi}\abs{\log\epsilon}^{1/2}\|\nabla \bu\|_{L^2(C_{\epsilon,a})}.
\end{equation}
\end{lemma}

\begin{proof}
We show the inequality \eqref{trace_straight_cylinder} for $\bu\in C^1(C_{\epsilon,a})\cap C^0(\overline{C_{\epsilon,a}})\cap \A_S$; the proof for $\bu\in \A_S$ then follows by density. \\

First note that for any $\bu\in C^1(C_{\epsilon,a})\cap C^0(\overline{C_{\epsilon,a}})$ and any $\bx = s\be_t +\epsilon\be_{\rho} + \theta\be_\theta \in \Gamma_{\epsilon,a}$, we may use the fundamental theorem of calculus to write
\[ \bu(s,\theta,\epsilon) = - \int_{\epsilon}^1 \frac{\p \bu}{\p \rho} \ts d\rho. \]
Then
\begin{align*}
|\bu(s,\theta,\epsilon)| &\le \int_{\epsilon}^{1} \bigg|\frac{\p \bu}{\p\rho} \bigg|\ts d\rho = \int_{\epsilon}^{1} \frac{1}{\sqrt{\rho}}\sqrt{\rho} \bigg|\frac{\p \bu}{\p\rho} \bigg|\ts d\rho \\
&\le \left(\int_{\epsilon}^{1} \frac{1}{\rho} \ts d\rho\right)^{\frac{1}{2}} \left(\int_{\epsilon}^{1} \bigg|\frac{\p \bu}{\p\rho} \bigg|^2 \ts \rho \ts d\rho\right)^{\frac{1}{2}} = \sqrt{|\log\epsilon|}\left(\int_{\epsilon}^{1} \bigg|\frac{\p \bu}{\p\rho} \bigg|^2 \ts \rho \ts d\rho\right)^{\frac{1}{2}}.
\end{align*}

Therefore ${\rm Tr}(\bu)$ obeys
\begin{equation}\label{surface_ineq}
\big|{\rm Tr}(\bu)\big|^2 \le |\log \epsilon| \int_{\epsilon}^1\bigg|\frac{\p \bu}{\p\rho} \bigg|^2 \ts \rho \ts d\rho.
\end{equation}

This holds for arbitrary $\bu \in C^1(C_{\epsilon,a})\cap C^0(\overline{C_{\epsilon,a}})$, but if $\bu$ also belongs to $\A_S$, by the $\theta$-independence of ${\rm Tr}(\bu)$, we have
\[\|{\rm Tr}(\bu)\|_{L^2(-a,a)}^2 = \frac{1}{2\pi}\int_{-a}^a \int_0^{2\pi} |{\rm Tr}(\bu)|^2 \ts d\theta \ts ds.\]

Then, using \eqref{surface_ineq}, we have that this $\bu$ satisfies
\begin{align*}
\|{\rm Tr}(\bu)\|_{L^2(-a,a)}^2 = \frac{1}{2\pi}\int_{-a}^a\int_0^{2\pi} |{\rm Tr}(\bu)|^2 \ts d\theta ds &\le \frac{1}{2\pi}|\log \epsilon| \int_{-a}^a\int_0^{2\pi}\int_{\epsilon}^1 \bigg| \frac{\p \bu}{\p \rho} \bigg|^2 \ts \rho \ts d\rho \ts d\theta\ts ds \\
& \le \frac{1}{2\pi}|\log \epsilon| \|\nabla \bu\|_{L^2(C_{\epsilon,a})}^2.
\end{align*}
\end{proof}

This estimate holds for $\bu$ defined around a straight cylinder; to return to a curved centerline, the diffeomorphisms $\psi_j^{-1}$ result in an additional constant on each set $W_j$  depending on $\psi_j$ but not $\epsilon$. Note that any constants arising from the use of cutoffs $\phi_j$ also depend on the Sobolev constant $c_S$ in $\Omega_\epsilon$, but by Lemma \ref{sobo_ineq}, $c_S$ is independent of $\epsilon$. \\

Summing over the neighborhoods $W_j$, we obtain the following trace inequality for any slender body $\Sigma_{\epsilon}$ satisfying the geometric constraints in Section \ref{geometric_constraints}:
\begin{equation}\label{trace_const0} 
\|{\rm Tr}(\bu)\|_{L^2(\T)} \le c_{\kappa}|\log\epsilon|^{1/2}\norm{\nabla\bu}_{L^2(\cup W_j)} \le c_{\kappa}|\log\epsilon|^{1/2}\norm{\nabla\bu}_{L^2(\Omega_\epsilon)},
\end{equation}
where $c_{\kappa}$ depends on the shape of the fiber centerline -- in particular, on the constants $\kappa_{\max}$ and $c_\Gamma$ -- but not on $\epsilon$. 
\end{proof}

%%%%%%
\subsubsection{Extension operator}\label{extension}
The proof of the Korn inequality (Lemmas \ref{korn_eps}) essentially relies on the existence of a linear operator $T_{\epsilon}$ extending $\bu$ to the interior of the slender body such that $\E(T_{\epsilon}\bu)$ is bounded independent of $\epsilon$ as $\epsilon\to 0$. In this section we prove the existence of such an extension. In particular, we show the following lemma:
\begin{lemma}\emph{(Extension operator)}\label{extension_eps}
Let $\Omega_{\epsilon}=\R^3 \backslash \overline{\Sigma_{\epsilon}}$ be as in Section \ref{geometric_constraints}. For $\bu\in D^{1,2}(\Omega_{\epsilon})$, there exists a bounded linear operator $T_{\epsilon}: D^{1,2}(\Omega_{\epsilon})\to D^{1,2}(\R^3)$ extending $\bu$ to the interior of the slender body and satisfying 
\begin{enumerate}
\item $T_{\epsilon}\bu|_{\Omega_{\epsilon}} = \bu$ 
\item $\|\E(T_{\epsilon}\bu) \|_{L^2(\R^3)} \le c_E \| \E(\bu)\|_{L^2(\Omega_{\epsilon})}$, where the constant $c_E$ is independent of the slender body radius $\epsilon$ as $\epsilon\to 0$.
\end{enumerate}
\end{lemma}

Note that property 2 implies $\|T_{\epsilon}\bu\|_{D^{1,2}(\R^3)} \le \sqrt{2} c_E \|\bu\|_{D^{1,2}(\Omega_{\epsilon})}$, since
\begin{align*}
\|T_{\epsilon}\bu\|_{D^{1,2}(\R^3)} &= \|\nabla(T_{\epsilon}\bu)\|_{L^2(\R^3)} \le \sqrt{2} \|\E(T_{\epsilon}\bu) \|_{L^2(\R^3)} \le \sqrt{2} c_E \| \E(\bu)\|_{L^2(\Omega_{\epsilon})} \\
& \le 2\sqrt{2} c_E \|\nabla \bu \|_{L^2(\Omega_{\epsilon})} = 2\sqrt{2} c_E \| \bu \|_{D^{1,2}(\Omega_{\epsilon})} .
\end{align*}

In order to prove Lemma \ref{extension_eps}, we will need a few additional lemmas. The first is an important result from elasticity theory concerning the symmetric gradient. The proof can be found in \cite{duvaut1976inequalities}.  
\begin{lemma}\emph{(Rigid motion)}\label{rigid_motion}
Let $\Omega \subset \R^3$ be any domain. If $\bu: \Omega \to \R^3$ with $\nabla \bu \in L^2(\Omega)$ satisfies $\nabla \bu +(\nabla \bu)^{\rm T}=0$, then $\bu$ is a rigid body motion: $\bu(\bx) = \bm{A}\bx+{\bm b}$ for some constant, antisymmetric $\bm{A}\in \R^{3\times3}$ and constant ${\bm b}\in \R^3$. 
\end{lemma}
%\begin{proof}
%We prove the result for smooth functions; Lemma \ref{rigid_motion} follows by density. In coordinates, let $u_{i,j} = -u_{j,i}$, $i,j=1,2,3$, where the subscript ``$,i$'' denotes differentiation with respect to $x_i$. Note that the diagonal elements $u_{i,i}$ (no summation) vanish. Then each entry satisfies
%\[ u_{i,jj} = -u_{j,ij} = -u_{j,ji} =0 \]
%for each $i,j=1,2,3$ (again, no summation implied). Thus we in fact have that $u_{i,j}=u_{i,j}(x_k)$, a function of $x_k$ only, for each combination of $i,j,k=1,2,3$. In other words,
%\[\nabla \bu = \begin{pmatrix}
%0 & u_{1,2}(x_3) & u_{1,3}(x_2) \\
%-u_{1,2}(x_3) & 0 & u_{2,3}(x_1) \\
%-u_{1,3}(x_2) & -u_{2,3}(x_1) & 0 \\
%\end{pmatrix}, \]
%and therefore
%\begin{align*}
%u_1 &= a_1 x_2x_3+ b_1 x_2 +c_1 x_3 +d_1\\
%u_2 &= a_2 x_1x_3+ b_2 x_1 +c_2 x_3 +d_2\\
%u_3 &= a_3 x_1x_2+ b_3 x_1 +c_3 x_2 +d_3.
%\end{align*}
%But the antisymmetry of $\nabla \bu$ implies $a_1=-a_2=a_3=-a_1=0$, so $\bu= \bm{A}\bx+\bm{b}$ for some $\bm{A}=-\bm{A}^{\rm T}\in \R^{3\times3}$ and constant ${\bm b}\in \R^3$.
% \end{proof}

The fact that the symmetric gradient $\E(\cdot)$ exactly vanishes for rigid motions will be used repeatedly throughout the following construction. \\

Again, let $\D$ be a bounded, $C^2$ domain in $\R^3$. Let $H^1(\D)$ denote the Sobolev space $\{\bv\in L^2(\D) \ts :\ts \nabla\bv\in L^2(\D)\}$ with norm $\norm{\bv}_{H^1(\D)}^2:= \norm{\bv}_{L^2(\D)}^2+\norm{\nabla\bv}_{L^2(\D)}^2$. On $\D$, we define the space of rigid motions 
\[ \mathcal{R} = \{ \bv\in H^1(\D) \ts : \ts \bv = \bm{A}\bx + \bm{b} \text{ for some } \bm{A}= -\bm{A}^{\rm T} \in \R^{3\times 3} \text{ and } \bm{b}\in \R^3\}.\]
Note that $\sR$ is a closed subspace of $L^2(\D)$. For $\bu\in H^1(\D)$, let $P_{\mathcal{R}}\bu$ be the $L^2$ projection of $\bu$ onto the space of rigid motions, i.e. 
\[ P_{\mathcal{R}}\bu = \bv\in \mathcal{R} \text{ such that } \|\bu-\bv\|_{L^2(\D)} \le \|\bu - \bw\|_{L^2(\D)}\quad  \text{for all } \bw\in \mathcal{R}. \]

\begin{lemma}\emph{(Korn inequality for pure strain)}\label{korn_nonrigid}
Let $\D$ be a bounded Lipschitz domain and let $\sR$ be the space of rigid motions on $\D$. For any $\bw\in H^1(\D)$ with $\bw\perp \sR$ in $L^2$, the Korn inequality holds:
\[ \|\nabla\bw\|_{L^2(\D)} \le c\|\E(\bw)\|_{L^2(\D)}.\] 
\end{lemma}

\begin{proof}
The proof of Lemma \ref{korn_nonrigid} relies on the following Korn-type inequality for the bounded domain $\D$: 
\begin{equation}\label{korn_bdd}
\|\bu\|_{H^1(\D)} \le c (\|\mathcal{E}(\bu)\|_{L^2(\D)}+\|\bu\|_{L^2(\D)}).
\end{equation}
Since the domain dependence of the constant $c$ does not need to be specified in Lemma \ref{korn_nonrigid}, we refer to \cite{duvaut1976inequalities} for a proof of \eqref{korn_bdd}. \\

Now, assume Lemma \ref{korn_nonrigid} does not hold. Then there exists a sequence of functions $\{\bw_k\}\subset H^1(\D)$, $k=1,2,3,\dots$, such that $\bw_k\perp \sR$ and 
\[ \|\nabla\bw_k\|_{L^2(\D)} > k \|\E(\bw_k)\|_{L^2(\D)}.\] 
Without loss of generality, $\|\bw_k\|_{L^2(\D)}=1$, so by \eqref{korn_bdd},
\[\|\E(\bw_k)\|_{L^2(\D)} < \frac{1}{k}\|\nabla\bw_k\|_{L^2(\D)} \le \frac{1}{k}\|\bw_k\|_{H^1(\D)} \le \frac{c}{k}(\|\E(\bw_k)\|_{L^2(\D)}+1).\]
Taking $k$ sufficiently large (in particular, $k>c$), we have 
\[ \bigg(1-\frac{c}{k}\bigg)\|\E(\bw_k)\|_{L^2(\D)} < \frac{c}{k},\]
and thus $\|\E(\bw_k)\|_{L^2(\D)}\to 0$ as $k\to\infty$. Again by the inequality \eqref{korn_bdd}, 
\[ \|\bw_k\|_{H^1(\D)} \le c\bigg(\frac{c}{k-c}+1\bigg), \]
so there exists a subsequence $\{\bw_{k_j}\}$ such that $\bw_{k_j}\rightharpoonup \bw$ in $H^1$ for some $\bw\in H^1(\D)$. By Rellich compactness, $\bw_{k_j} \to \bw$ in $L^2$. Furthermore, $\liminf_k \norm{\E(\bw_{k_j})}_{L^2(\D)}\ge \norm{\E(\bw)}_{L^2(\D)}$, so $\E(\bw)=0$. Thus $\bw\in \sR$, but $\bw_k\perp\sR$ for all $k$, and $\bw_{k_j} \to \bw$ in $L^2$, so $\bw\equiv 0$. Thus $\bw_{k_j} \to 0$ in $L^2$, which contradicts $\|\bw_k\|_{L^2(\D)}=1$ for all $k$.
\end{proof}

\begin{remark}\label{korn_rmk}
Note that Lemma \ref{korn_nonrigid} remains true if we replace the orthogonality condition $\bw\perp \sR$ in $L^2(\D)$ with the condition that $\bw$ vanishes on an open set of $\p\D$ containing four points not in a plane. The proof is exactly as above, except that now the sequence $\bw_k\not\in\sR$ due to the vanishing condition on $\p\D$. In the last two lines we then conclude that the limit $\bw\in \sR$ but each $\bw_k\not\in\sR$, so $\bw\equiv 0$, yielding the same contradiction. Note that under the domain rescaling $\D\to \epsilon\D$, the constant in Lemma \ref{korn_nonrigid} remains unchanged.
\end{remark}

Using Lemma \ref{korn_nonrigid}, we can show the following inequality.
\begin{lemma}\emph{(Korn-Poincar\'e inequality)}\label{korn_poincare}
Let $\D$ be a bounded, Lipschitz domain in $\R^3$. For any $\bu\in H^1(\D)$, we have 
\begin{equation}\label{KP_ineq}
\| \bu - P_{\mathcal{R}}\bu\|_{L^2(\D)} \le c\| \E(\bu)\|_{L^2(\D)}
\end{equation}
for some constant $c>0$.
\end{lemma}

\begin{proof}
Assume that inequality \eqref{KP_ineq} does not hold. Then for each $k=1,2,3,\dots$ there exists a sequence $\{\bu_k\}\subset H^1(\D)$ such that
\[ \|\bu_k - P_{\sR}\bu_k \|_{L^2(\D)} > k\|\E(\bu_k)\|_{L^2(\D)}. \]
Define $\bw_k=\bu_k - P_{\sR}\bu_k$, so $\bw_k \perp \sR$ for each $k=1,2,3,\dots$ and $\E(\bw_k)=\E(\bu_k)$. Without loss of generality $\|\bw_k\|_{L^2(\D)} = 1$. Then
\[ 1= \|\bw_k \|_{L^2(\D)} > k\|\E(\bw_k)\|_{L^2(\D)},\] 
so $\|\E(\bw_k)\|_{L^2(\D)} <\frac{1}{k} \to 0$ as $k\to \infty$. Furthermore, since $\bw_k \perp \sR$ for each $k$, by the Korn inequality for pure strain (Lemma \ref{korn_nonrigid}) we have $\|\nabla \bw_k\|_{L^2(\D)} < \frac{c}{k}$. Thus $\bw_k$ is uniformly bounded in $H^1$ and there exists a subsequence $\{\bw_{k_l}\}$ such that $\bw_{k_l}\rightharpoonup \bw$ in $H^1$ for some $\bw\in H^1(\D)$. By compactness, $\bw_{k_l}\to \bw$ in $L^2$. Then, since $\liminf_k \|\E(\bw_k)\|_{L^2(\D)} \ge \|\E(\bw)\|_{L^2(\D)}$, we have that the limit $\bw$ satisfies $\E(\bw)=0$, so $\bw\in \sR$. But $\bw_{k_l}\to \bw$ in $L^2$ and $\bw_{k} \perp \sR$ for each $k$, so we must have $\bw\perp \sR$ as well. Thus $\bw\equiv 0$, so $\bw_{k_l} \to 0$ in $L^2$, which contradicts $\|\bw_{k_l} \|_{L^2(\D)}=1$. 
\end{proof}

Finally, we show an analogue of Lemma 3.1.2(1) in \cite{mazya1997differentiable}, adapted to use the symmetric gradient rather than the full gradient. 
\begin{lemma}\emph{(Extension-by-reflection scaling)}\label{extension_ineq}
Let $\D_1$, $\D_2$ be bounded $C^2$ domains in $\R^2$ with $\overline \D_2\subset \D_1$, and let $\D=\D_1\times[-1,1] \subset\R^3$ and $\D_H=(\D_1\backslash \overline \D_2)\times [-1,1]\subset\R^3$. For the rescaled domains $\D_{H,\epsilon}= \epsilon \D_H$, $\D_{\epsilon}=\epsilon\D$ ($\epsilon>0$), there exists a linear extension operator $T: H^1(\D_{H,\epsilon}) \to H^1(\D_{\epsilon})$ satisfying 
\begin{equation}\label{extension_symm}
\| T \bu \|_{L^2(\D_{\epsilon})} \le c\|\bu\|_{L^2(\D_{H,\epsilon})}
\end{equation}
as well as the estimate
\begin{equation}\label{extension_symm}
\| \E(T \bu) \|_{L^2(\D_{\epsilon})} \le c\bigg( \epsilon^{-1}\|\bu\|_{L^2(\D_{H,\epsilon})} + \|\E(\bu)\|_{L^2(\D_{H,\epsilon})} \bigg).
\end{equation}
\end{lemma}

%%%%%%%%
\begin{proof}
For a function $\bv$ defined in the upper half-space $\R^3_+$, we recall the standard extension-by-reflection $E:\R^3_+\to \R^3$ across the boundary $x_3=0$ (see \cite{mazya1997differentiable} or \cite{evans2010pde}):
\[ E\bv(\bx) = \begin{cases}
\bv(\bx), & \bv\in \R^3_+ \\
\bv(x_1,x_2,-x_3) & \bv \not\in \R^3_+.
\end{cases} \]
For the domain-with-hole $\D_H\subset \R^3$, we cover a neighborhood of the inner boundary $\p\D_2\times [-1,1]$ with finitely many balls $B^H_i$, $i=1, \dots,N$, centered at points on $\p\D_2$, choosing the cover such that $\D_H\cap B^H_i$ can be mapped via $C^2$ diffeomorphism, denoted by $\Phi^{-1}_i$, to the half-ball $B\cap\R^3_+$, where $B$ is a ball in $\R^3$. We then choose open sets $U_j\subset\D_H$, $j=1,\dots,M$, such that $\{ B^H_i\} \cup \{U_j\}$ cover $\D_H$.  We define a partition of unity $\{\varphi_i\}\cup \{\varphi_j\}$ subordinate to this cover, and define the usual extension operator $T:\D_H\to \D$ by
\[T\bu = \sum_i \bigg(E\big((\varphi_i \bu)\circ \Phi_i\big)\bigg)\circ \Phi_i^{-1} + \sum_j \varphi_j \bu.\]

From this extension operator $T$, we can directly estimate $\|\E(T\bu)\|_{L^2(\D)}$. First, note that $\varphi_i \bu$ vanishes on $\p B_i^H \cap \D_H\subset \p(B_i^H\cap\D_H)$. Since $\p B_i^H$ is curved, we may use Remark \ref{korn_rmk} to estimate: 
\begin{align*}
\|\E(T\bu)\|_{L^2(\D)} &\le c\sum_i\norm{\nabla\bigg(\big(E\big((\varphi_i \bu)\circ \Phi_i\big)\big)\circ \Phi_i^{-1}\bigg) }_{L^2(\D)} \\
&\hspace{2cm} +\sum_j \|\varphi_j\E(\bu) \|_{L^2(\D_H)} + \sum_j \|\nabla \varphi_j \bu^{\rm T} \|_{L^2(\D_H)}\\
%
&\le c\sum_i\big\|\nabla(\varphi_i \bu)\big\|_{L^2(\D_H)} + \|\E(\bu)\|_{L^2(\D_H)} + c_{\phi}\|\bu\|_{L^2(\D_H)} \\
&\le c\sum_i\big\|\E(\varphi_i \bu) \big\|_{L^2(\D_H)} + \|\E(\bu)\|_{L^2(\D_H)} + c_{\phi}\|\bu\|_{L^2(\D_H)} \\
&\le c\big( \|\bu\|_{L^2(\D_H)} + \|\E(\bu)\|_{L^2(\D_H)} \big).
\end{align*}

The above inequality, coupled with a scaling argument ($\bx\to \epsilon\bx$) results in the desired $\epsilon$-dependent inequality \eqref{extension_symm}.
 \end{proof}
 
%%%%%
With Lemmas \ref{korn_poincare} and \ref{extension_ineq}, we are equipped to prove Lemma \ref{extension_eps}.

\begin{proof}[Proof of Lemma \ref{extension_eps}]
Let $D_{r}$ denote the disk in $\R^2$ of radius $r$. Using the diffeomorphisms $\psi_j$ defined in Section \ref{trace_sec}, it suffices to consider $\bu\in D^{1,2}((\R^2\backslash D_{\epsilon})\times \R)$ with supp$(\bu)\subset (\R^2\backslash D_{\epsilon})\times [-a,a]$ for $a<\infty$ and show that there exists an extension operator into the interior of the infinite cylinder $D_{\epsilon}\times\R \subset\R^3$ with symmetric gradient that is bounded independently of $\epsilon$ as $\epsilon\to 0$. \\

First we define 
\[ S_{\epsilon} = D_{2\epsilon}\times \R \quad \text{and} \quad G_{\epsilon} = (D_{2\epsilon}\backslash \overline{D_{\epsilon}}) \times \R \subset \R^3. \]
Since $\bu\in D^{1,2}((\R^2\backslash D_{\epsilon})\times \R)$ with supp$(\bu)\subset (\R^2\backslash D_{\epsilon})\times [-a,a]$, we have $\bu\in H^1(G_{\epsilon})$. We show that we can in fact construct a linear extension operator extending $\bu\in H^1(G_{\epsilon})$ to $H^1(S_{\epsilon})$ whose symmetric gradient is bounded independent of $\epsilon$. \\

Following \cite{mazya1997differentiable}, we begin by defining a cover $\{Q_j\}$ of $\R$:
\[ Q_j = \{s \in \R \ts:\ts |s-j| <1 \}, \quad j\in \Z. \] 
Let $\{\eta_j\}$ denote a smooth partition of unity subordinate to $Q_j$, where $\eta_j$ can be written as $\eta_j= \phi(s-j)$, translates of the same smooth cutoff function, such that $|\nabla \eta_j|\le c$ for each $j$. We define a sequence of cylinders and cylindrical layers 
\[ S_{2}^{(j)} = D_{2}\times Q_j \quad \text{and} \quad G^{(j)}_{2} = (D_{2}\backslash \overline{D_1}) \times Q_j \subset \R^3. \]
and set $S^{(j)}_{\epsilon} = \epsilon S^{(j)}_{2}$ and $G^{(j)}_{\epsilon} = \epsilon G^{(j)}_{2}$. Then by Lemma \ref{extension_ineq}, there exists a linear extension operator $T_{\epsilon}^{(j)}: H^1(G^{(j)}_{\epsilon}) \to H^1(S^{(j)}_{\epsilon})$ satisfying
 \begin{equation}\label{ext_est_seq1}
 \|\E (T_{\epsilon}^{(j)}\bu) \|_{L^2(S^{(j)}_{\epsilon})} \le c\left(\epsilon^{-1}\|\bu\|_{L^2(G^{(j)}_{\epsilon})} + \|\E(\bu)\|_{L^2(G^{(j)}_{\epsilon})} \right)
 \end{equation}
and
 \begin{equation}\label{ext_est_seq2}
  \| T_{\epsilon}^{(j)}\bu \|_{L^2(S^{(j)}_{\epsilon})} \le c\|\bu\|_{L^2(G^{(j)}_{\epsilon})}.
  \end{equation}
 
Let $P_{\sR}^{(j)}\bu$ denote the projection of $\bu\big|_{G^{(j)}_{\epsilon}}\in H^1(G^{(j)}_{\epsilon})$ onto $\sR$, the space of rigid motions on each $G^{(j)}_{\epsilon}$. Then, since $\E(\bw)= 0$ for any $\bw\in \sR$, we have 
 \[ \|\E(\bu - P_{\sR}^{(j)}\bu)\|_{L^2(G^{(j)}_{\epsilon})} =\|\E(\bu)\|_{L^2(G^{(j)}_{\epsilon})}. \]
By the Korn-Poincar\'e inequality (Lemma \ref{korn_poincare}) and a scaling argument we also have 
\begin{equation}\label{poincare_est}
  \|\bu - P_{\sR}^{(j)}\bu\|_{L^2(G^{(j)}_{\epsilon})} \le c\epsilon\|\E(\bu)\|_{L^2(G^{(j)}_{\epsilon})}.
\end{equation}

Since $P^{(j)}_{\sR}\bu\in \sR$ on each cylindrical shell $G_{\epsilon}^{(j)}$, we can write $P^{(j)}_{\sR}\bu= \bm{A}_j\bx+\bm{b}_j$ for $\bx\in G_{\epsilon}^{(j)}$. We then define the extension to each of the cylinders $S_{\epsilon}^{(j)}$ by 
\begin{equation}\label{PR_on_S}
\overline P_{\sR}^{(j)}\bu =\bm{A}_j\bx+ \bm{b}_j, \qquad \bx\in S_{\epsilon}^{(j)}.
\end{equation}
  
With these tools in mind, we now define an extension operator from the cylindrical shell $G_{\epsilon}$ to the cylinder $S_{\epsilon}$. We take 
 \begin{equation}\label{ext_operator}
  T_{\epsilon}\bu(\bx) = \bv(\bx)+ \bw(\bx) 
  \end{equation}
 where, for $\bx=\bx(\rho,\theta,s)\in S_{\epsilon}$ and $\bu_j = \bu|_{G^{(j)}_{\epsilon}}$, 
 \begin{align*}
 \bv(\rho,\theta,s) &= \sum_{j\in \Z} \eta_j(s/\epsilon)\bigg(\overline P_{\sR}^{(j)}\bu\bigg)(\bx) \\
 \bw(\rho,\theta,s) &= \sum_{j\in \Z} \eta_j(s/\epsilon)\left(T_{\epsilon}^{(j)}\big(\bu_j -P_{\sR}^{(j)}\bu\big)\right)(\bx).
 \end{align*}
 
Note that $T_{\epsilon}\bu \big|_{G_{\epsilon}}=\bu$. Furthermore, we show
\begin{equation}\label{ext_est_1}
 \|\E(T_{\epsilon}\bu)\|_{L^2(S_{\epsilon})} \le c\| \E(\bu)\|_{L^2(G_{\epsilon})} 
 \end{equation}
where the constant $c$ does not depend on $\epsilon$ as $\epsilon \to 0$. \\

We begin by estimating $\bv$. Let
\[\tilde Q_j = \{s\in\R \ts:\ts 0 <s-j<1\}, \quad j\in \Z.\]
Note that for each $j$ we have $\tilde Q_j\subset Q_j$ and $\tilde Q_j\subset Q_{j+1}$; in particular, $\eta_j(s)+\eta_{j+1}(s)=1$ on $\tilde Q_j$. Define
\[ \tilde S_{\epsilon}^{(j)} = \epsilon\left(D_{2}\times \tilde Q_j \right) \quad \text{and}\quad\tilde G_{\epsilon}^{(j)} = \epsilon\left((D_{2}\backslash \overline{D_1})\times \tilde Q_j \right).\]

On each $\tilde S_{\epsilon}^{(j)}$, $\bv$ can be rewritten as
\[ \bv(\rho,\theta,s) = \overline P_{\sR}^{(j)}\bu +\eta_{j+1}(s/\epsilon) (\overline P_{\sR}^{(j+1)}\bu - \overline P_{\sR}^{(j)}\bu). \]

Then, by the definition \eqref{PR_on_S}, we can bound the norm of $\overline P_{\sR}^{(j)}\bu$ on each cylinder $\tilde S_{\epsilon}^{(j)}$ by its norm over the shell $\tilde G_{\epsilon}^{(j)}$: $\|\overline P_{\sR}^{(j)}\bu\|_{L^2(\tilde S_{\epsilon}^{(j)})} \le c\|P_{\sR}^{(j)}\bu\|_{L^2(\tilde G_{\epsilon}^{(j)})}$. Using this, we bound the symmetric gradient of $\bv$: 
\begin{align*}
\|\E(\bv)\|_{L^2(\tilde S_{\epsilon}^{(j)})} &= \|\nabla \eta_{j+1}(s/\epsilon)(\overline P_{\sR}^{(j+1)}\bu - \overline P_{\sR}^{(j)}\bu)^{\rm T} +  (\overline P_{\sR}^{(j+1)}\bu - \overline P_{\sR}^{(j)}\bu)\nabla \eta_{j+1}(s/\epsilon)^{\rm T} \|_{L^2(\tilde S_{\epsilon}^{(j)})} \\
&\le c\epsilon^{-1}\|P_{\sR}^{(j+1)}\bu - P_{\sR}^{(j)}\bu \|_{L^2(\tilde G_{\epsilon}^{(j)})} \\
&\le c\epsilon^{-1}\left(\|\bu-P_{\sR}^{(j+1)}\bu\|_{L^2(G_{\epsilon}^{(j+1)})}+\|\bu-P_{\sR}^{(j)}\bu\|_{L^2(G_{\epsilon}^{(j)})} \right),
\end{align*}
 where in the last step we have used that $\tilde G_{\epsilon}^{(j)} \subset G_{\epsilon}^{(j+1)}$ and $\tilde G_{\epsilon}^{(j)} \subset G_{\epsilon}^{(j)}$. Finally, using \eqref{poincare_est}, we have
\[\|\E( \bv)\|_{L^2(\tilde S_{\epsilon}^{(j)})} \le c\left(\|\E(\bu)\|_{L^2(G_{\epsilon}^{(j)})}+\|\E(\bu)\|_{L^2(G_{\epsilon}^{(j+1)})}\right).\]

Summing over $j$, we then have
\[ \|\E(\bv)\|_{L^2(S_{\epsilon})} \le c \|\E(\bu)\|_{L^2(G_{\epsilon})} \]
where $c$ is bounded independent of $\epsilon$ as $\epsilon\to 0$. \\

We now bound the symmetric gradient of $\bw$. On each $\tilde S_{\epsilon}^{(j)}$ we have
\begin{align*}
 \|\E(\bw)\|_{L^2(\tilde S_{\epsilon}^{(j)})} &\le  \|\E\big(T_{\epsilon}^{(j)}(\bu_j -P_{\sR}^{(j)}\bu)\big) \|_{L^2(\tilde S_{\epsilon}^{(j)})}+  \|\E\big(T_{\epsilon}^{(j+1)}(\bu_{j+1} -P_{\sR}^{(j+1)}\bu)\big) \|_{L^2(\tilde S_{\epsilon}^{(j)})} \\
 &\quad + 2c\epsilon^{-1} \|T_{\epsilon}^{(j)}(\bu_j-P_{\sR}^{(j)}\bu)\|_{L^2(\tilde S_{\epsilon}^{(j)})} + 2c\epsilon^{-1} \|T_{\epsilon}^{(j+1)}(\bu_{j+1}-P_{\sR}^{(j+1)}\bu)\|_{L^2(\tilde S_{\epsilon}^{(j)})}. 
 \end{align*}

Using the inequalities \eqref{ext_est_seq1}, \eqref{ext_est_seq2}, and \eqref{poincare_est}, we have
\begin{align*}
\|T_{\epsilon}^{(j)}(\bu_j -P_{\sR}^{(j)}\bu)\|_{L^2(\tilde S_{\epsilon}^{(j)})} &\le c \|\bu_j -P_{\sR}^{(j)}\bu\|_{L^2(\tilde G_{\epsilon}^{(j)})} \le c\epsilon\|\E(\bu)\|_{L^2(\tilde G_{\epsilon}^{(j)})}
\end{align*}
and
\begin{align*}
\|\E\big(T_{\epsilon}^{(j)}(\bu_j-P_{\sR}^{(j)}\bu)\big)\|_{L^2(\tilde S_{\epsilon}^{(j)})} &\le c\left(\epsilon^{-1}\|\bu_j-P_{\sR}^{(j)}\bu\|_{L^2(\tilde G_{\epsilon}^{(j)})}+\|\E\big(\bu_j -P_{\sR}^{(j)}\bu)\|_{L^2(\tilde G_{\epsilon}^{(j)}\big)} \right) \\
&\le c\| \E(\bu) \|_{L^2(\tilde G_{\epsilon}^{(j)})} .
\end{align*}

Summing over $j$, we have
\[ \|\E(\bw)\|_{L^2(S_{\epsilon})} \le c\| \E(\bu) \|_{L^2(G_{\epsilon})}. \]

Therefore the extension operator $T_{\epsilon}: G_{\epsilon}\to S_{\epsilon}$ \eqref{ext_operator} is bounded independent of $\epsilon$ as $\epsilon\to 0$. Defining $T_{\epsilon}\bu=\bu$ in $\R^3\backslash S_{\epsilon}$ then gives the desired extension on all of $\R^3$. 
\end{proof}

%%%%%%%%%%%%%%%%%%%%%%%%%
\subsubsection{Korn inequality}\label{korn_proof}
Using the extension operator $T_\epsilon$ defined in Section \ref{extension}, we can now prove $\epsilon$-independence of the Korn constant (Lemma \ref{korn_eps}). \\

We first note that the proof of the Korn inequality for function in $D^{1,2}(\R^3)$ is very simple. We first consider $\bv\in C_0^{\infty}(\R^3)$, then take the closure to show the result for $D^{1,2}(\R^3)$. We have that $\bv\in C_0^{\infty}(\R^3)$ satisfies 
\begin{align*}
\int_{\R^3} |\mathcal{E}(\bv)|^2 \ts d\bx &= \int_{\R^3}\bigg( \frac{1}{2}|\nabla \bv|^2 + \frac{1}{2}\nabla\bv:(\nabla\bv)^{\rm T}\bigg) \ts d\bx = \int_{\R^3} \frac{1}{2}|\nabla \bv|^2 \ts d\bx - \frac{1}{2}\int_{\R^3} \bv \cdot \nabla( \dive \ts\bv) \ts d\bx \\
&= \int_{\R^3} \frac{1}{2}|\nabla \bv|^2 \ts d\bx + \frac{1}{2}\int_{\R^3} |\dive \ts \bv|^2 \ts d\bx \ge \int_{\R^3} \frac{1}{2}|\nabla \bv|^2 \ts d\bx,
\end{align*}
where we have used integration by parts twice, as well as the fact that $\bv$ vanishes at $\infty$. \\

Now, using the extension operator $T_{\epsilon}$ established in Lemma \ref{extension_eps} to extend $\bu\in D^{1,2}(\Omega_{\epsilon})$ to all of $\R^3$, the proof of the Korn inequality (Lemma \ref{korn_eps}) is immediate.

\begin{proof}[Proof of Lemma \ref{korn_eps}]
Let $\bu \in D^{1,2}(\Omega_{\epsilon})$ and let $T_\epsilon\bu$ be the extension of $\bu$ to $\R^3$ defined in Lemma \ref{extension_eps}. Using properties of the extension operator $T_{\epsilon}$ and the Korn inequality on $\R^3$, we then have
\begin{align*}
\|\nabla \bu\|_{L^2(\Omega_{\epsilon})} &\le \|\nabla (T_{\epsilon}\bu)\|_{L^2(\R^3)} \le \sqrt{2}\|T_{\epsilon}\mathcal{E}(\bu)\|_{L^2(\R^3)} \le \sqrt{2}c_E \| \mathcal{E}(\bu) \|_{L^2(\Omega_{\epsilon})}.
\end{align*}
 Taking $c_K=\sqrt{2}c_E$, we obtain \eqref{korn_ineq}. 
\end{proof}

%%%%%%%%%%%%%%%%%%%%%%%%%%%%%%
%%%%%%%%%%%%%%%%%%%%%%%%%%%%%%
%%%%%%%%%%%%%%%%%%%%%%%%%%%%%%

\subsubsection{Sobolev inequality}\label{Sob_ineq}
Using the extension operator defined in Section \ref{extension}, we also immediately obtain the $\epsilon$-independence of the Sobolev inequality stated in Lemma \ref{sobo_ineq}.
\begin{proof}[Proof of Lemma \ref{sobo_ineq}]
We have
\begin{align*}
\| \bu\|_{L^6(\Omega_{\epsilon})} &\le \| T_{\epsilon}\bu\|_{L^6(\R^3)} \le  c_R\| \nabla (T_{\epsilon}\bu)\|_{L^2(\R^3)} \\
&\le  c_R c_{E}\| \nabla \bu\|_{L^2(\Omega_{\epsilon})}, \quad \text{by Lemma \ref{extension_eps},}
\end{align*}
where $c_R$ is the constant in the Sobolev inequality on $\R^3$. Taking $c_S=c_Rc_E$, we obtain the desired result. 
\end{proof}


%%%%%%%%%%%%%%%%%%%%%%%%%%%%%%
%%%%%%%%%%%%%%%%%%%%%%%%%%%%%%
%%%%%%%%%%%%%%%%%%%%%%%%%%%%%%

\subsubsection{Pressure estimate}\label{pressure_const}
Finally, we prove the $\epsilon$-independence claim for the problem $\dive\ts\bv=p$ stated in Lemma \ref{divv_p_lem}. The proof closely follows \cite{galdi2011introduction}, Chapter III.3, with additional attention paid to the domain dependence.

\begin{proof}[Proof of Lemma \ref{divv_p_lem}]
We begin by taking a sequence $\{p_m\}\subset C_0^{\infty}(\Omega_{\epsilon})$ approximating $p$ in $L^2(\Omega_{\epsilon})$. For each $m\in \N$, let $\psi_m$ be the solution to the Poisson problem $\Delta\psi_m = \overline{p_m}$ in $\R^3$, where $\overline{p_m}$ denotes the extension by zero of $p_m$ to the interior of $\Sigma_{\epsilon}$; i.e. to all of $\R^3$. Then by standard solution theory for the Poisson problem (\cite{galdi2011introduction}, Chapter II.11), we have the estimate
\begin{equation}\label{poisson_est}
 \|\nabla^2\psi_m\|_{L^2(\Omega_{\epsilon})} \le \|\nabla^2\psi_m\|_{L^2(\R^3)} \le c_q\|\overline{p_m}\|_{L^2(\R^3)} = c_q\|p_m\|_{L^2(\Omega_{\epsilon})} 
\end{equation}
where $\nabla^2$ denotes the matrix of second partial derivatives and the constant $c_q$ is independent of $\epsilon$. \\

We define
\[ \bv_m :=\nabla \psi_m+\bw_m \]
where $\bw_m\in D^{1,2}(\Omega_{\epsilon})$ is supported only within the neighborhood $\mathcal{O}$ \eqref{region_O} of $\Gamma_{\epsilon}$, and serves to correct for $\nabla \psi_m\neq 0$ on $\Gamma_{\epsilon}$. To this end, $\bw_m$ can be considered as a function in $H^1(\mathcal{O})$ satisfying
\begin{equation}\label{w_equation}
\begin{aligned}
\dive\ts\bw_m &= 0 \quad \text{in }\mathcal{O} \\
\bw_m &= - \nabla \psi_m \quad \text{on }\Gamma_{\epsilon} \\
\bw_m &=0 \quad \text{on } \p \mathcal{O} \backslash\Gamma_{\epsilon}, 
\end{aligned}
\end{equation}
which is then extended by zero to all of $\Omega_{\epsilon}$. For each $m\in \N$, such a function $\bw_m$ exists since $\Delta \psi_m=0$ within $\Sigma_{\epsilon}$ and therefore
\[ \int_{\Gamma_{\epsilon}} \nabla \psi_m\cdot{\bm n}=0. \]
A solution to \eqref{w_equation} can be constructed by considering the function ${\bm \Psi}_m = - \phi\nabla \psi_m$ where $\phi\in C^{\infty}(\Omega_{\epsilon})$ is a cutoff function satisfying 
\[ \phi(\rho)=\begin{cases}
1, & \rho \le r_{\max}/2 \\
0 & \rho > r_{\max}.
\end{cases} \] 
Then by \cite{galdi2011introduction}, Theorem III.3.1, there exists a solution $\bw_m-{\bm \Psi}_m\in H^1_0(\mathcal{O})$ satisfying
\begin{equation}\label{new_w_equation}
\begin{aligned}
\dive(\bw_m-{\bm \Psi}_m) &= -\dive \ts {\bm \Psi}_m \quad \text{in }\mathcal{O}; \\
\|\nabla(\bw_m-{\bm \Psi}_m)\|_{L^2(\mathcal{O})} &\le c_B\|\dive \ts {\bm \Psi}_m\|_{L^2(\mathcal{O})}.
\end{aligned}
\end{equation}
Since the slender body surface $\Gamma_{\epsilon}$ satisfies the geometric constraints in Section \ref{geometric_constraints}, the region $\mathcal{O}$ satisfies an interior sphere condition with uniform radius $r_{\max}/2$. Then $\mathcal{O}$ can be considered as the infinite union of balls of radius $r_{\max}/2$. Following the construction in the proof of Lemma 2, Chapter 1.1.9 of \cite{maz2013sobolev}, there exist a finite number of domains $\mathcal{O}_k$, star-shaped with respect to balls of radius $r_{\max}/4$, such that
\[\mathcal{O} = \bigcup_{k=1}^N \mathcal{O}_k. \]
Here $N$ depends only on $\kappa_{\max}$ and $c_\Gamma$. Then the domain dependence of the constant $c_B$ in \eqref{new_w_equation} has an explicit formula (\cite{galdi2011introduction}, equation III.3.27): 
\[ c_B \le c_0 \bigg(\frac{\delta(\mathcal{O})}{r_{\max}} \bigg)^3\bigg(1+ \frac{\delta(\mathcal{O})}{r_{\max}} \bigg) \]
where $\delta(\mathcal{O})$ is the diameter of the region $\mathcal{O}$ and $c_0$ depends on the diameter of the domains $\mathcal{O}_k$, each of which are bounded independent of $\epsilon$ as $\epsilon\to 0$. \\

Then, from \eqref{new_w_equation}, we have
\begin{equation}\label{w_est1}
\begin{aligned}
\|\nabla\bw_m\|_{L^2(\Omega_{\epsilon})} &\le c_B\|\dive\ts {\bm \Psi}_m\|_{L^2(\Omega_{\epsilon})} + \|\nabla{\bm \Psi}_m\|_{L^2(\Omega_{\epsilon})} \\
&= c_B\|\dive(\phi\nabla \psi_m) \|_{L^2(\Omega_{\epsilon})} + \|\nabla(\phi\nabla \psi_m)\|_{L^2(\Omega_{\epsilon})} .
\end{aligned}
\end{equation}

Therefore, using \eqref{poisson_est} and \eqref{w_est1}, we have
\begin{align*}
\|\nabla \bw_m\|_{L^2(\Omega_{\epsilon})} &\le (c_B+1)(c_q\|p_m\|_{L^2(\Omega_{\epsilon})}+ c_{\phi}\|\nabla \psi_m\|_{L^2(\mathcal{O})}), 
\end{align*}
where $c_{\phi}$ depends on $\nabla\phi$ but is independent of $\epsilon$. We then use the Sobolev inequality on $\R^3$ to obtain 
\begin{align*}
 \|\nabla \psi_m\|_{L^2(\mathcal{O})} &\le  |\mathcal{O}|^{1/3} \|\nabla\psi_m\|_{L^6(\mathcal{O})} \le |\mathcal{O}|^{1/3} \|\nabla\psi_m\|_{L^6(\Omega_{\epsilon})} \\
 &\le |\mathcal{O}|^{1/3} c_S\|\nabla^2\psi_m\|_{L^2(\Omega_{\epsilon})}  \le |\mathcal{O}|^{1/3} c_S c_q \|p_m\|_{L^2(\Omega_{\epsilon})}, \quad \text{using }\eqref{poisson_est}.
 \end{align*}
Now, $|\mathcal{O}|\le c_{\kappa}r_{\max}^2$ is bounded independent of $\epsilon$, and by Lemma \ref{sobo_ineq} the Sobolev constant $c_S$ is independent of $\epsilon$. Thus
\[ \|\nabla \bw_m\|_{L^2(\Omega_{\epsilon})} \le c_W\|p_m\|_{L^2(\Omega_{\epsilon})} \]
for $c_W$ independent of $\epsilon$, and 
\[ \|\nabla \bv_m\|_{L^2(\Omega_{\epsilon})} \le \|\nabla^2\psi_m\|_{L^2(\Omega_{\epsilon})}+ \|\nabla \bw_m\|_{L^2(\Omega_{\epsilon})} \le (c_q+c_W)\|p_m\|_{L^2(\Omega_{\epsilon})}. \]

 Passing to the limit we obtain the desired solution to the $\dive \ts\bv=p$ problem of Lemma \eqref{divv_p_lem}, as the constant $c_P=c_q+c_W$ is independent of $\epsilon$.
 \end{proof}

%The final form of the higher regularity estimate \eqref{stokes_est_epsilon} also follows from tracking these same constants. Again using the form of the constants $c_K$, $c_T$, and $c_P$ in \eqref{regular_est_stokes}, we have 
%\begin{align*}
%\|\nabla^2\bu\|_{L^2(\Omega_{\epsilon})} + \|\nabla p\|_{L^2(\Omega_{\epsilon})} &\le c_{\kappa} \frac{|\log\epsilon|^{1/2}}{\epsilon} \|{\bm f}\|_{H^{1/2}(\T)}.
%\end{align*}
%
%We thus complete the proof of the estimates in Theorem \ref{stokes_theorem}. \\

%%%%%%%%%%%%%%%%%%%%%%%%%%%%%%%%%%%%%
%\subsection{Moving frame gradient and tangential commutator estimates}
%To show higher tangential regularity, we will need estimates of the commutators \eqref{comm2} and \eqref{comm1} that track dependence of constants on the slender body radius $\epsilon$. Before we derive such estimates, we will need an expression for the gradient $\nabla_x$ using the moving frame valid in the region $\mathcal{O}$ \eqref{region_O}. We write $\bx=\bx(\rho,\theta,s)$ as $\bx=\X(s)+\rho\be_{\rho}(s,\theta)$, where $\be_{\rho}(s,\theta)=\cos\theta\be_{n_1}(s)+\sin\theta\be_{n_2}(s)$. Then any function $g$ defined in $\mathcal{O}$ can be written
%\[ g(\rho,\theta,s) = g(\bx) = g(\X(s)+\rho\be_{\rho}(s,\theta)). \]
%
%Let $\nabla_M = \be_{\rho}\frac{\p}{\p \rho}+ \be_{\theta}\frac{1}{\rho}\frac{\p}{\p \theta}+ \be_t\frac{\p}{\p s}$ denote the gradient with respect to the cylindrical moving frame basis vectors $\be_{\rho}(s,\theta)$, $\be_{\theta}(s,\theta) = -\sin\theta\be_{n_1}(s)+\cos\theta\be_{n_2}(s)$, $\be_t(s)$. We then have
%\begin{align*}
%\nabla_M g(\rho,\theta,s) &= (\nabla_M \bx)\nabla_x g(\bx) \\
%&= \big[\be_{\rho}\be_{\rho}^{\rm T}+ \be_{\theta}\be_{\theta}^{\rm T} + (1-\rho\wh\kappa)\be_t\be_t^{\rm T}+ \kappa_3\rho \be_t\be_{\theta}^{\rm T} \big] \nabla_x g(\bx),
%\end{align*}
%where 
%\begin{equation}\label{J_def}
%\wh\kappa(\theta,s):= \kappa_1(s)\cos\theta+\kappa_2(s)\sin\theta,
%\end{equation}
%and therefore 
%\begin{equation}\label{frenet_grad}
%\begin{aligned}
%\nabla_x g(\bx) &= \bigg[\be_{\rho}\be_{\rho}^{\rm T}+ \be_{\theta}\be_{\theta}^{\rm T} + \frac{1}{1-\rho\wh\kappa} \big(\be_t\be_t^{\rm T} - \kappa_3 \rho \be_t\be_{\theta}^{\rm T}\big) \bigg] \nabla_M g(\rho,\theta,s) \\
%&= \bm{A}_s \nabla_M g.
%\end{aligned}
%\end{equation}
%
%Here the matrix 
%\begin{equation}\label{matrix_As} 
%\bm{A}_s := (\nabla_M \bx)^{-1}= \be_{\rho}\be_{\rho}^{\rm T}+ \be_{\theta}\be_{\theta}^{\rm T} + \frac{1}{1-\rho\wh\kappa} \big(\be_t\be_t^{\rm T} - \kappa_3 \rho \be_t\be_{\theta}^{\rm T}\big) 
%\end{equation}
%is defined with respect to the basis vectors $\be_{\rho}(s,\theta)$, $\be_{\theta}(s,\theta)$, $\be_t(s)$ at the point $s$ on the fiber centerline. By definition of $\kappa_{\max}$ \eqref{kappamax}, within the region $\mathcal{O}$ we have
%\begin{equation}\label{denom_bound}
%\begin{aligned}
%|1-\rho\wh\kappa| & \ge 1- \rho |\kappa|(\cos\theta+\sin\theta) \\
%&\ge 1- \frac{1}{2 \kappa_{\max}}|\kappa|\sqrt{2} \ge 1- \frac{\sqrt{2}}{2} \ge \frac{1}{4}.
%\end{aligned}
%\end{equation}
%
%Therefore 
%\[ \|\bm{A}_s\|_{L^{\infty}(\mathcal{O})}\le c_{\kappa} \]
%for $c_{\kappa}$ independent of the slender body radius $\epsilon$. \\
%
%For the higher regularity proof, we will require that the fiber centerline $\X(s)$ is at least $C^4$, as we need bounds on both the first and second derivatives of the moving frame coefficients $\kappa_1(s)$ and $\kappa_2(s)$ \eqref{moving_ODE}. Before we introduce the following proposition concerning tangential translation estimates, we recall the bounds \eqref{kap_deriv_consts}:
%\[ m_{\kappa,1} := \max_{s\in\T^1}(\abs{\kappa_1'(s)}+ \abs{\kappa_2'(s)}), \; m_{\kappa,2} := \max_{s\in\T^1}(\abs{\kappa_1''(s)}+ \abs{\kappa_2''(s)}).\] 
%
%We now show the following: 
%\begin{proposition}\emph{(Estimates for translation operator Jacobian)}\label{grad_ests}
%We have
%\begin{align}
%\| \nabla_x \tau_h^{\theta} - {\bf I}\|_{L^{\infty}(\Omega_{\epsilon})} &= \| \nabla_x \tau_h^{\theta} - {\bf I}\|_{L^{\infty}(\mathcal{O})} \le |h| c_{\theta,a}  \label{theta_ests1}\\
%\| \det(\nabla_x \tau_h^{\theta}) - 1\|_{L^{\infty}(\Omega_{\epsilon})} &= \| \det(\nabla_x \tau_h^{\theta}) - 1\|_{L^{\infty}(\mathcal{O})} \le |h| c_{\theta,b}  \label{theta_ests2}\\
%\|\nabla_x \det(\nabla_x \tau_h^{\theta})\|_{L^{\infty}(\Omega_{\epsilon})} &= \|\nabla_x \det(\nabla_x \tau_h^{\theta})\|_{L^{\infty}(\mathcal{O})} \le |h| c_{\theta,c},  \label{theta_ests3}
%\end{align}
%where $c_{\theta,a}$ and $c_{\theta,b}$ depend only on $\kappa_{\max}$ and $c_\Gamma$, while $c_{\theta,c}$ also depends on $m_{\kappa,1}$. Also,
%\begin{align}
%\| \nabla_x \tau_h^{s} - {\bf I}\|_{L^{\infty}(\Omega_{\epsilon})} &=\| \nabla_x \tau_h^{s} - {\bf I}\|_{L^{\infty}(\mathcal{O})} \le |h| c_{s,a}  \label{s_ests1}\\
%\| \det(\nabla_x \tau_h^{s}) - 1\|_{L^{\infty}(\Omega_{\epsilon})} &=\| \det(\nabla_x \tau_h^{s}) - 1\|_{L^{\infty}(\mathcal{O})} \le |h| c_{s,b}  \label{s_ests2}\\
%\|\nabla_x \det(\nabla_x \tau_h^{s})\|_{L^{\infty}(\Omega_{\epsilon})} &= \|\nabla_x \det(\nabla_x \tau_h^{s})\|_{L^{\infty}(\mathcal{O})} \le |h| c_{s,c}  \label{s_ests3},
%\end{align}
%where $c_{s,a}$ and $c_{s,b}$ depend only on $\kappa_{\max}$, $c_\Gamma$, and $m_{\kappa,1}$, while $c_{s,c}$ also depends on $m_{\kappa,2}$. 
%\end{proposition}
%
%To prove Proposition \ref{grad_ests}, we will need $C^4$ regularity of the slender body centerline $\X(s)$. 
%%%%%%%%
%
%\begin{proof} 
%To show the inequalities \eqref{theta_ests1} - \eqref{s_ests3}, we follow the approach used in \cite{boyer2012mathematical}. Defining the $C^2$ vector fields 
%\begin{align*}
%\Theta^{\theta}(\bx(\rho,\theta,s)) &=  \phi(\rho)\big(2\pi \rho \ts \be_{\theta}(s,\theta)\big), \\
%\Theta^s(\bx(\rho,\theta,s)) &= \phi(\rho)\big((1-\rho\wh\kappa)\be_t(s) + \rho \kappa_3\be_{\theta}(s,\theta)\big),
%\end{align*}
%we use the characterization of the tangential translation operators $\tau_h^{\theta}$ and $\tau_h^s$ as solutions to the ODE
%\begin{align*}
%\frac{d}{dh}\tau_h^j(\bx) &= \Theta^j(\tau_h^j(\bx)), \quad j=\theta, s \\
%\tau_0^j(\bx) &= \bx.
%\end{align*}
%
%Note that for the vector field $\Theta^s$ to be $C^2$, we need the coefficients $\kappa_j(s)\in C^2(\T)$, $j=1,2$, and therefore we need the fiber centerline $\X(s)\in C^4(\T)$. \\
%
%We then have that $\nabla_x\tau_h(\bx)$ satisfies the ODE 
%\begin{align*}
%\frac{d}{dh}(\nabla_x\tau_h^j(\bx)) &= (\nabla_x\Theta^j)\big|_{\tau_h^j(\bx)}^{\rm T} (\nabla_x\tau_h^j(\bx)), \quad j=\theta, s \\
%\nabla_x\tau_0^j(\bx) &= {\bf I}.
%\end{align*}
%
%Within the region $\mathcal{O}$, we have
%\begin{align*}
%|\nabla_x\Theta^{\theta}| &= |\bm{A}_s \nabla_M\Theta^{\theta}| \le c_{\kappa} |\nabla_M (\rho\be_{\theta}(s,\theta))|, \quad \text{by \eqref{denom_bound}}, \\
%&= c_{\kappa}\big| \be_{\rho}\be_{\theta}^{\rm T} - \be_{\theta}\be_{\rho}^{\rm T} + \rho \be_t \big((\kappa_1\sin\theta-\kappa_2\cos\theta)\be_t -\kappa_3\be_{\rho}\big)^{\rm T} \big| \le c_{\theta,\beta},
%\end{align*}
%where $c_{\theta,\beta}$ depends on $\kappa_{\max}$ and $c_\Gamma$ but not on $\epsilon$, and 
%\begin{align*}
%|\nabla_x\Theta^{s}| &= |\bm{A}_s \nabla_M\Theta^{s}| \\
%&\le c_{\kappa}\big| \be_{\rho}\big((-\kappa_1\cos\theta-\kappa_2\sin\theta)\be_t+ \kappa_3\be_{\theta}\big)^{\rm T} + \be_{\theta}\big((\kappa_1\sin\theta-\kappa_2\cos\theta)\be_t - \kappa_3\be_{\rho}\big)^{\rm T} \\
%&\hspace{1cm} + \be_t\big(-\rho(\kappa_1'\cos\theta+\kappa_2'\sin\theta)\be_t + (1-\rho\wh\kappa)[\kappa_1\be_{n_1}+\kappa_2\be_{n_2}] \big)^{\rm T} \\
%&\hspace{1cm} + \be_t \rho \kappa_3[(\kappa_1\sin\theta-\kappa_2\cos\theta)\be_t-\kappa_3\be_{\rho}]^{\rm T}  \big| \\
%&\le c_{s,\beta},
%\end{align*}
%where $c_{s,\beta}$ depends only on $\kappa_{\max}$, $c_\Gamma$, and $m_{\kappa,1}$. Thus the difference $|\nabla_x\tau_h^j(\bx)-{\bf I}|$ satisfies the differential inequality 
%\begin{align*}
%\bigg|\frac{d}{dh}(\nabla_x\tau_h^j(\bx)-{\bf I})\bigg| &\le c_{j,\beta} \big|\nabla_x\tau_h^j(\bx)- {\bf I}\big| + c_{j,\beta}, \quad j=\theta, s \\
%|\nabla_x\tau_0^j(\bx)-{\bf I}| &= 0.
%\end{align*}
%Using a Gr\"onwall inequality, since $|h|<1$, we obtain 
%\begin{align*}
%\big|\nabla_x \tau_h^j(\bx)-{\bf I}\big| &\le \int_0^h c_{j,\beta} \ts dh' \ts e^{c_{j,\beta}|h|} \le c_{j,\beta}|h|,
%\end{align*}
%giving \eqref{theta_ests1} and \eqref{s_ests1}. \\
%
%Similarly, by \cite{boyer2012mathematical}, the Jacobian determinant of $\tau_h^j$, $j=\theta,s$, satisfies the ODE 
%\begin{align*}
%\frac{d}{dh}\det \nabla_x \tau_h^j(\bx) &= \dive \ts \Theta^j \big|_{\tau^j_h(\bx)}\det\nabla_x\tau_h^j(\bx), \\
%\det\nabla_x\tau_0^j(\bx) &=1.
%\end{align*}
%
%Applying $\nabla_x$, we have
%\begin{align*}
%\frac{d}{dh}\nabla_x\det \nabla_x \tau_h^j(\bx) &= \dive \ts \Theta^j \big|_{\tau^j_h(\bx)}\nabla_x\det\nabla_x\tau_h^j(\bx) \\
%&\hspace{1cm} + \nabla_x \dive\ts\Theta^j \big|_{\tau^j_h(\bx)}\cdot \nabla_x\tau_h^j(\bx) \ts \det\nabla_x(\tau_h^j(\bx)), \\
%\nabla_x\det\nabla_x\tau_0^j(\bx) &=0.
%\end{align*}
%
%Now, for any vector field $\Theta$ defined in the region $\mathcal{O}$, $\dive \ts \Theta$ has the form 
%\begin{align*}
%\dive\ts\Theta &= \frac{1}{1-\rho\wh\kappa}\bigg(\frac{1}{\rho}\frac{\p(\rho (1-\rho\wh\kappa)\Theta_{\rho})}{\p \rho}+ \frac{1}{\rho}\frac{\p((1-\rho\wh\kappa)\Theta_{\theta})}{\p \theta} + \frac{\p\Theta_s}{\p s} \bigg),
%\end{align*}
%where $\Theta_{\rho}=\Theta\cdot\be_{\rho}$, $\Theta_{\theta}=\Theta\cdot\be_{\theta}$, $\Theta_{s}=\Theta\cdot\be_{t}$. \\
%
%Then
%\begin{align*}
%\dive\ts\Theta^{\theta} &= \dive(\phi(\rho)2\pi\rho\be_{\theta}) 
%= \phi(\rho)2\pi \frac{\rho(\kappa_1\sin\theta-\kappa_2\cos\theta)}{1-\rho\wh\kappa}, \\
%\dive\ts\Theta^s &= \dive\big(\phi(\rho)((1-\rho\wh\kappa)\be_t + \rho \kappa_3\be_{\theta})\big) 
%=\phi(\rho)\frac{\rho\big((\kappa_1\kappa_3-\kappa_2')\sin\theta-(\kappa_2\kappa_3+\kappa_1')\cos\theta\big)}{1-\rho\wh\kappa}.
%\end{align*}
%
%Note that, using \eqref{denom_bound}, we have 
%\begin{equation}\label{denom_bound2}
%\begin{aligned}
%\bigg|\nabla_x \bigg(\frac{1}{1-\rho\wh\kappa(\theta,s)}\bigg) \bigg| &= \bigg|\bm{A}_s\frac{1}{(1-\rho\wh\kappa)^2}\bigg( (\kappa_1\cos\theta+\kappa_2\sin\theta) \be_{\rho} + (-\kappa_1\sin\theta+\kappa_2\cos\theta) \be_{\theta} \\
%&\hspace{1cm} + \frac{\rho(\kappa_1'\cos\theta+\kappa_2'\sin\theta) + \kappa_3\rho(\kappa_1\sin\theta-\kappa_2\cos\theta)}{1-\rho\wh\kappa} \be_t \bigg)\bigg| \le c_{\kappa,1},
%\end{aligned}
%\end{equation}
%where $c_{\kappa,1}$ depends on $\kappa_{\max}$, $c_{\Gamma}$, and $m_{\kappa,1}$. \\
%
%Furthermore, the cutoff function $\phi(\rho)$ by definition (see \eqref{tang_trans_def}) satisfies
%\begin{align*}
%|\nabla_x \phi(\rho)| & = |\phi'(\rho)\be_{\rho}| \le c_{\kappa}.
%\end{align*}
% 
%Therefore, within the region $\mathcal{O}$, we can bound 
%\begin{align*}
%|\nabla_x\dive\ts \Theta^{\theta}| \le c_{\theta,\gamma}, \quad |\nabla_x \dive\ts\Theta^s| \le c_{s,\gamma},
%\end{align*}
%where the constant $c_{\theta,\gamma}$ depends only on $\kappa_{\max}$, $c_{\Gamma}$, and $m_{\kappa,1}$, while $c_{s,\gamma}$ also depends on $m_{\kappa,2}$. \\
%
%Thus we have that $\det \nabla_x \tau_h^j(\bx)$, $j=s,\theta$, satisfies the following differential inequalities: 
%\begin{align*}
%\bigg|\frac{d}{dh}(\det \nabla_x \tau_h^j(\bx) - 1)\bigg| &\le c_{j,\beta}|\det\nabla_x\tau_h^j(\bx)-1| + c_{j,\beta}, \\
%|\det\nabla_x\tau_0^j(\bx)-1| &=0,
%\end{align*}
%and 
%\begin{align*}
%\bigg|\frac{d}{dh}\nabla_x\det \nabla_x \tau_h^j(\bx)\bigg| &\le c_{j,\gamma}\big|\nabla_x\det\nabla_x\tau_h^j(\bx)\big| + c_{j,\gamma} |\nabla_x\tau_h^j(\bx) \ts\det\nabla_x(\tau_h^j(\bx))|, \\ 
%\nabla_x\det\nabla_x\tau_0^j(\bx) &=0.
%\end{align*}
%
%Again using a Gr\"onwall inequality, from the first differential expression we obtain \eqref{theta_ests2} and \eqref{s_ests2}, and from the second we obtain \eqref{theta_ests3} and \eqref{s_ests3}.  
%\end{proof}
%
%%%%%%%%%%
%Using Proposition \ref{grad_ests}, we can derive the estimates for the commutators given in the following proposition. The proof of these statements follows (\cite{boyer2012mathematical}, Proposition III.3.19), relying on the estimates \eqref{theta_ests1} - \eqref{s_ests3} for our specific tangential translation operators.  
%
%\begin{proposition}\emph{(Tangential translation estimates)}\label{trans_ests} 
%
%The tangential translation operators $\tau_h^j$, $j=\theta,s$, both satisfy the following properties: 
%\begin{enumerate}
%\item For $g\in H^{k}(\mathcal{O})$, $k=0$ or $k=-1$, we have 
%\begin{equation}\label{est_1}
%\sup_{0<h<1} \| \tau^j_h g\|_{H^k(\mathcal{O})} \le c_{j,k}\|g\|_{H^k(\mathcal{O})}.
%\end{equation} 
%\item For any $g_1,g_2\in L^2(\mathcal{O})$, we have
%\begin{equation}\label{est_2}
%\bigg|\int_{\mathcal{O}} \{g_1,g_2\}_{h}^j \ts d\bx\bigg| \le c_{j,0}|h| \|g_1\|_{L^2(\mathcal{O})}\|g_2\|_{L^2(\mathcal{O})}.
%\end{equation}
%Furthermore, for $g_1\in L^2(\T)$ and $g_2\in \A_{\epsilon}$, we have that the commutator along the fiber centerline satisfies
%\begin{equation}\label{est_2.5_gamma}
%\int_{\T} \{g_1,g_2\}_{h}^{\theta} \ts ds =0,
%\end{equation}
%and 
%\begin{equation}\label{est_2.5_s}
%\bigg|\int_{\T} \{g_1,g_2\}_{h}^{s}\bigg| \ts ds \le c_{s,0}|h| \|g_1\|_{L^2(\T)}\|g_2\|_{L^2(\T)}.
%\end{equation}
%\item For $g\in H^{k}(\mathcal{O})$, $k=0$ or $k= -1$, we have
%\begin{equation}\label{est_3}
%\sup_{0<h<1}\| [\nabla,\tau_h^{j}]g\|_{H^{k}(\mathcal{O})} \le c_{j,k}|h| \|\nabla g\|_{H^{k}(\mathcal{O})}. 
%\end{equation}
%\item For any $g\in H^k(\mathcal{O})$,  $k=0$ or $k=-1$, we have
%\begin{equation}\label{est_4}
%||| g |||_{T,H^{k+1}(\mathcal{O})} \le c_{s,k}\|g\|_{H^{k+1}(\mathcal{O})}.
%\end{equation}
%\end{enumerate}
%In each estimate, the constants $c_{\theta,0}$ depend only on $\kappa_{\max}$ and $c_\Gamma$, the constants $c_{\theta,-1}$ and $c_{s,0}$ depend on $\kappa_{\max}$, $c_\Gamma$, and $m_{\kappa,1}$, and the constants $c_{s,-1}$ depend on $\kappa_{\max}$, $c_\Gamma$, $m_{\kappa,1}$, and $m_{\kappa,2}$. 
%\end{proposition}
%
%%%%%%%%%%%%%%%%%
%
%\begin{proof} 
%1. For $k =0$, using \eqref{theta_ests2}, we have
%\begin{align*}
%\int_{\mathcal{O}} g^2(\tau_h^{\theta}(\bx))\ts d\bx &= \int_{\mathcal{O}} g^2(\bx)|\det(\nabla_x\tau_h^{\theta}(\bx))|\ts d\bx \\
%&\le (1+|h|c_{\theta,b}) \|g\|_{L^2(\mathcal{O})}^2.
%\end{align*}
%Using \eqref{s_ests2}, a similar calculation holds for $\tau_h^s$. \\
%
%For the $H^{-1}$ case, we proceed by duality, and show the result first for $g$ smooth. The inequality for $g\in H^{-1}(\mathcal{O})$ then follows by density.  For any $\psi\in H^1_0(\mathcal{O})$, by \eqref{theta_ests2} and \eqref{theta_ests3}, we have
%\begin{align*}
%\langle \tau_h^{\theta}g, \psi \rangle_{H^{-1},H^1_0} &= \int_{\mathcal{O}} g(\bx) \psi(\tau_{-h}^{\theta}(\bx)) |\det\nabla_x\tau_h^{\theta}(\bx)| \ts d\bx \\
%&\le \|\nabla_x\det\nabla_x\tau_h^{\theta}(\bx)\|_{L^{\infty}(\mathcal{O})}\|g\|_{H^{-1}(\mathcal{O})}\| \psi\|_{L^2(\mathcal{O})} \\
%&\quad + \|\det\nabla_x\tau_h^{\theta}(\bx)\|_{L^{\infty}(\mathcal{O})}\|g\|_{H^{-1}(\mathcal{O})}\|\nabla_x \psi\|_{L^2(\mathcal{O})}\\
%&\le (c_{\theta,a}+c_{\theta,b})\|g\|_{H^{-1}(\mathcal{O})}\| \psi\|_{H^1_0(\mathcal{O})}.
%\end{align*}
%A similar computation holds for $\tau_h^s g$ using \eqref{s_ests2} and \eqref{s_ests3}. \\
%
%2. By a change of variables $\bx\to \tau_{-h}(\bx)$, we have
%\[\int_{\mathcal{O}} g_1(\tau_h ^{\theta}(\bx))g_2(\bx) \ts d\bx = \int_{\mathcal{O}} g_1(\bx)g_2(\tau_{-h} ^{\theta}(\bx)) |\det(\nabla_x \tau_h^{\theta}(\bx))| \ts d\bx. \]
%Therefore
%\begin{align*}
%\int_{\mathcal{O}} \{g_1,g_2\}_h^\theta \ts d\bx &= \int_{\mathcal{O}} g_1(\bx)g_2(\tau_{-h} ^{\theta}(\bx)) |\det(\nabla_x \tau_h^{\theta}(\bx))-1| \ts d\bx \\
%&\le \|\det(\nabla_x \tau_h^{\theta}(\bx))-1\|_{L^{\infty}(\mathcal{O})} \|g_1\|_{L^2(\mathcal{O})} \|\tau^{\theta}_{-h} g_2\|_{L^2(\mathcal{O})} \\
%&\le c_{\theta,b}(1+|h|c_{\theta,b})^{1/2}|h| \|g_1\|_{L^2(\mathcal{O})} \| g_2\|_{L^2(\mathcal{O})},
%\end{align*}
%where we have used \eqref{theta_ests2} and \eqref{est_1}. A similar calculation using \eqref{s_ests2} and \eqref{est_1} gives the result for $\tau_h^s$. \\
%
%For $g_2\in \A_{\epsilon}$ and $\bx\in \Gamma_{\epsilon}$, we have that $g_2(\tau_h^{\theta}(\bx)) = g_2(\bx)$ and thus clearly 
%\[ \int_{\Gamma_{\epsilon}} \{g_1,g_2\}_h^\theta \ts d\bx =0. \]
%
%Also, by \eqref{s_ests2},
%\begin{align*}
%\int_{\T} \{g_1,g_2\}_h^s \ts ds &= \int_{\T} g_1(s) g_2(s-h) |\det(\nabla_x \tau_h^{s}(\bx))-1| \ts ds \\
%&\le \|\det(\nabla_x \tau_h^{s}(\bx))-1\|_{L^{\infty}} \|g_1(s)\|_{L^2(\T)} \| g_2(s-h)\|_{L^2(\T)} \\
%&\le c_{s,b}|h|\|g_1\|_{L^2(\T)} \| g_2\|_{L^2(\T)}. \\
%\end{align*}
%
%3. We begin with $k=0$. We have that 
%\begin{align*}
%\nabla_x \big(g(\tau_h^\theta(\bx))\big)&= \big(\nabla_x \tau_h^{\theta}(\bx)\big)\nabla_x g\big|_{\tau_h^{\theta}(\bx)}
%\end{align*}
%and therefore
%\begin{align*}
%\big| [\nabla,\tau_h^{\theta}] g \big| &=  \big|(\nabla_x \tau_h^{\theta}(\bx) -{\bf I})\nabla_x g\big|_{\tau_h^{\theta}(\bx)} \big|  \le |h| c_{\theta,a},
% \end{align*}
% by \eqref{theta_ests1}. Thus we have
%\begin{align*}
%\sup_{0<h<1} \frac{1}{h} \|[\nabla,\tau_h^{\theta}] g \|_{L^2} &\le c_{\theta,a} \|\nabla_x g\|_{L^2}.
%\end{align*}
%
%The argument for $g(\tau_h^s(\bx)$ is similar, using \eqref{s_ests1}.\\
%
%For $g\in H^{-1}(\mathcal{O})$, we proceed by duality. For any $\bm{\psi}\in H^1_0(\mathcal{O})$, we have
%\begin{align*}
%\langle \nabla (\tau^j_h g) - \tau_h^j (\nabla g), \psi\rangle_{H^{-1},H^1_0} &= -\int_{\mathcal{O}} (\tau_h^j g) \ts \dive\ts \bm{\psi} \ts d\bx - \langle \nabla g, (\tau_{-h}^j \bm{\psi}) \det\nabla_x\tau^j_h \rangle_{H^{-1},H^1_0} \\
%&= -\int_{\mathcal{O}} g(\bx) \ts (\tau_{-h}^j(\dive\ts \bm{\psi})) \det \nabla_x \tau^j_{-h} \ts d\bx \\
%&\hspace{2cm} + \int_{\mathcal{O}} g(\bx) \dive\big((\tau_{-h}^j \bm{\psi}) \det\nabla_x\tau^j_h\big) \ts d\bx.
%\end{align*}
%
%Thus, using \eqref{theta_ests2} and \eqref{theta_ests3} for $j=\theta$ and \eqref{s_ests2} and \eqref{s_ests3} for $j=s$, we have
%\begin{align*}
%\langle \nabla (\tau^j_h g) - &\tau_h^j (\nabla g), \bm{\psi}\rangle_{H^{-1},H^1_0} \\
%&\le \|g\|_{L^2(\mathcal{O})} \bigg(\|\tau_{-h}^j(\dive\ts \bm{\psi}) \|_{L^2(\mathcal{O})}\|\det \nabla_x \tau^j_{-h}\|_{L^{\infty}(\mathcal{O})} + \\
%& \qquad \|\dive(\tau_{-h}^j \bm{\psi})\|_{L^2(\mathcal{O})}\|\det\nabla_x\tau^j_h\|_{L^{\infty}(\mathcal{O})} + \|\tau_{-h}^j \bm{\psi}\|_{L^2(\mathcal{O})}\|\nabla_x\det\nabla_x\tau^j_h\|_{L^{\infty}(\mathcal{O})}\bigg) \\
%&\le |h| \|g\|_{L^2(\mathcal{O})} \bigg(c_{j,b}(\|\tau_{-h}^j(\dive\ts \bm{\psi}) \|_{L^2(\mathcal{O})} + \|\dive(\tau_{-h}^j \bm{\psi})\|_{L^2(\mathcal{O})}) + c_{j,c}\|\tau_{-h}^j \bm{\psi}\|_{L^2(\mathcal{O})}\bigg) \\
%&\le c_{j,-1}|h| \|g\|_{L^2(\mathcal{O})} \bigg(c_{j,b}(\|\dive\ts \bm{\psi} \|_{L^2(\mathcal{O})} + \|\tau_{-h}^j \bm{\psi}\|_{H^1(\mathcal{O})}) + c_{j,c}\| \bm{\psi}\|_{L^2(\mathcal{O})}\bigg) \\
%&\le c_{j,-1}(2c_{j,b}+c_{j,c})|h| \|g\|_{L^2(\mathcal{O})} \| \bm{\psi}\|_{H^1(\mathcal{O})}, 
%\end{align*}
%where we have also used \eqref{est_1}.\\
%
%4. We show the inequality for $k=0$ and $g$ sufficiently smooth. The $k=-1$ case can be shown by a duality argument similar to the ones above. \\
%
%By Taylor's theorem, 
%\begin{align*}
%|\delta_h^{\theta}g| &= | g(\rho,\theta+2\pi h,s) - g(\rho,\theta,s) | \\
% &\le |h| \int_{\T} \bigg| \nabla_M g(\tau_{th}^{\theta}(\bx))\cdot(\rho\be_{\theta}(s,\theta+th))\bigg| \ts dt,
%\end{align*}
%and thus
%\[ |\delta_h^{\theta}g|^2 \le |h|^2 \rho^2 \int_{\T} | \nabla_M g(\tau_{th}^{\theta}(\bx))|^2\ts dt. \]
%
%Integrating over $\mathcal{O}$, we have, by \eqref{theta_ests2},
%\begin{align*}
%\bigg\|\frac{1}{\rho}\delta_h^{\theta}g \bigg\|_{L^2(\mathcal{O})}^2 &\le |h|^2\int_{\mathcal{O}} \int_{\T} | \nabla_M g(\tau_{th}^{\theta}(\bx))|^2\ts dt \ts d\bx \\
%&\le |h|^2\int_{\T}\int_{\mathcal{O}} | \nabla_M g(\bx)|^2|\det\nabla_x\tau_{-th}^{\theta}(\bx)|\ts d\bx \ts dt \\
%&\le |h|^2 c_{\theta,b}\int_{\T}\int_{\mathcal{O}} | \nabla_M g(\bx)|^2\ts d\bx \ts dt \\
%&= |h|^2 c_{\theta,b}\int_{\mathcal{O}} |\bm{A}_s^{-1} \bm{A}_s \nabla_M g(\bx)|^2\ts d\bx \\
%&\le |h|^2 c_{\theta,b}\int_{\mathcal{O}} |\nabla_x g(\bx)|^2\ts d\bx. 
%\end{align*}
%
%Similarly, for $\delta_h^s g$, we have
%\begin{align*}
%|\delta_h^s g| &= |g(\rho,\theta,s+h) - g(\rho,\theta,s)| \\
%&\le |h| \int_{\T} \bigg| \nabla_M g(\tau_{th}^s(\bx))\big((1-\rho\wh\kappa(s+th,\theta))\be_t(s+th) + \rho \kappa_3(s+th)\be_{\theta}(s+th,\theta)\big) \bigg| \ts dt. 
%\end{align*}
%Thus
%\[ |\delta_h^s g|^2 \le |h|^2 c_{\kappa} \int_{\T} |\nabla_M g(\tau_{th}^s(\bx))|^2 \ts dt. \]
%
%Integrating over $\mathcal{O}$, we have, by \eqref{s_ests2},
%\begin{align*}
%\|\delta_h^s g\|_{L^2(\mathcal{O})}^2 &\le  |h|^2 c_{\kappa} \int_{\mathcal{O}} \int_{\T} |\nabla_M g(\tau_{th}^s(\bx))|^2 \ts dt \ts d\bx \\
%&\le  |h|^2 c_{\kappa} \int_{\T} \int_{\mathcal{O}} |\nabla_M g(\bx)|^2|\det\nabla_x\tau_{-th}^s(\bx)| \ts d\bx \ts dt \\
%&\le  |h|^2 c_{\kappa}c_{s,b} \int_{\T} \int_{\mathcal{O}} |\nabla_M g(\bx)|^2 \ts d\bx \ts dt \\
%&= |h|^2 c_{\kappa}c_{s,b} \int_{\mathcal{O}} |\bm{A}_s^{-1}\bm{A}_s\nabla_M g(\bx)|^2 \ts d\bx \\
%&\le |h|^2 c_{\kappa}c_{s,b} \int_{\mathcal{O}} |\nabla_x g(\bx)|^2 \ts d\bx. 
%\end{align*}
%
%Adding the estimates for $\|\frac{1}{\rho}\delta_h^{\theta}g \|_{L^2(\mathcal{O})}$ and $\|\delta_h^s g\|_{L^2(\mathcal{O})}$ we obtain \ref{trans_ests}.4. \\
%\end{proof}
%
%From Proposition \ref{trans_ests} we easily obtain the following properties:
%\begin{proposition}\emph{(Additional tangential translation properties)}\label{trans_ests2} 
%For $g\in L^2(\mathcal{O})$ and $j=s,\theta$, we have
%\begin{align}
%\|\delta^j_h g\|_{H^{-1}} &\le c_{j,-1} |h| \|g\|_{L^2} \label{p_properties1}\\
%\| [{\rm{div}},\tau_h^j] \nabla g\|_{H^{-1}} &\le c_{j,-1} |h|\| g\|_{L^2}  \label{p_properties2}\\
%\|{\rm{div}}([\nabla,\tau_h^j]g) \|_{H^{-1}} &\le \|[\nabla,\tau^j_h] g\|_{L^2} \le c_{j,0}|h|\|\nabla g\|_{L^2},  \label{p_properties3}
%\end{align}
%where each $c_{\theta,-1}$ depends only on $\kappa_{\max}$, $c_\Gamma$, and $m_{\kappa,1}$, while each $c_{s,-1}$ also depends on $m_{\kappa,2}$.
%\end{proposition}

%%%%%%%%%%%%%%%%%%%%%%%%%%%%%%%%%%%%%%%%%%%%%%%%%%%%%%%%%%
%%%%%%%%%%%%%%%%%%%%%%%%%%%%%%%%%%%%%%%%%%%%%%%%%%%%%%%%%%
%%%%%%%%%%%%%%%%%%%%%%%%%%%%%%%%%%%%%%%%%%%%%%%%%%%%%%%%%%

\bibliography{SBT_bib.bib}{}
\bibliographystyle{abbrv}
%\bibliographystyle{siam}

%%%%%%%%%%%%%%%%%%%%%%%%%%%%%%%%%%%%%%%%%%%%%%%%%%%%%%%%%%

\end{document}