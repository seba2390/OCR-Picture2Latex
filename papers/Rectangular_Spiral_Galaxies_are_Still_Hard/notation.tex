In Spiral Galaxies, a \defn{board} consists of an $m\times n$ grid of unit squares called \defn{cells}.
A \defn{puzzle} consists of a board and a set of \defn{centers} placed at cell centers, edge midpoints, or grid vertices. Either all centers have the same color, or a subset of the centers may be colored black.
A solution to the puzzle consists of a partition of the grid into polyominoes, called \defn{galaxies}, such that each galaxy contains a single center and is $180^\circ$ rotationally symmetric about its center.

Our reductions are from the \textsc{Planar Positive 1-in-3 SAT} problem, a well-known NP-complete, ASP-complete, and \#P-complete problem~\cite{dyer1986planar, mulzer08, hunt1998complexity}:
%\iffalse
\begin{definition}
  An instance of the \textsc{Planar Positive 1-in-3 SAT} problem is a \defn{formula} $F = (\mathcal{C},\mathcal{V})$ consisting of a set $\mathcal{C} = \{C_1, C_2, \dots, C_k\}$ of $k$ \defn{clauses} over $\ell$ \defn{variables} $\mathcal{V} = \{x_1, x_2, \dots, x_\ell\}$.
  Each clause in $F$ is a set of at most three variables, and in particular contains variables only in their positive form (no negation).  %(e.g. `$x_1$' or `$\neg x_7$'). 
  The variable--clause incidence graph $G$ is planar. %$and it is sufficient to consider formulae where $G$ has a rectilinear embedding.
  A clause is satisfied if and only if it contains exactly one \textsc{true} variable, and the formula $F$ is satisfied if and only if all its clauses are satisfied.
  The goal is to find a Boolean assignment to the variables that satisfies the formula~$F$.
\end{definition}
%\fi
%It is sufficient to consider formulae where $G$ has a rectilinear embedding.
