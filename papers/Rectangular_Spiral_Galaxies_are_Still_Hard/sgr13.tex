In this section, we show that solving %Spiral Galaxies 
Rectangular Galaxies puzzles is NP-complete and ASP-complete, even with the promise that every solution uses only $1\times 1$, $1\times 3$, and $3\times 1$ galaxies, and that the corresponding problem of counting the number of solutions is \#P-complete. %, see~\cite{v-aa-10} Chapter 28 and~\cite{y-ccfas-03,ys-ccfas-03} for definitions of \#P-completeness and ASP-completeness, respectively.
%For our reduction, we again use the \textsc{Planar 1-in-3 SAT} problem.
\iffalse
\begin{theorem}\label{th:sgr13}
	Determining whether a Spiral Galaxies puzzle is solvable with only  $1\times 1$, $1\times 3$, and $3\times 1$ galaxies is NP-complete and ASP-complete, and counting the number of solutions is \#P-complete.
\end{theorem}
\fi
%\begin{proof}
	%The proof is by reduction from \textsc{Planar 1-in-3 SAT}. 
Again we give a reduction from \textsc{Planar Positive 1-in-3 SAT}.
Given an instance $F$ of \textsc{Planar Positive 1-in-3 SAT} with incidence graph $G$, we show how to turn a rectilinear planar embedding of $G$ into a Rectangular Galaxies puzzle $P$ such that a solution to $P$ yields a solution to $F$, thereby showing NP-completeness. Furthermore, there will be a one-to-one correspondence between solutions of $P$ and solutions of $F$, showing \#P-completeness and ASP-completeness. 

In this section, all galaxies will be $1\times 1$, $1\times 3$, or $3\times 1$.
To ease the description, our figures draw these galaxies as edges of length 2 ($1\times 3$ and $3\times 1$) or nonexisting edges ($1\times 1$).
%Hence, in this proof, squares in the figures do not indicate centers for galaxies. %, but the centers will in fact always be placed between two disks.
Instead of drawing the galaxy centers (dots), we draw small boxes (squares) that reflect the endpoints of these edges.
Boxes at distance 2 can be connected by a rectilinear edge, and centers will be located at the middle of each potential edge; see Figure~\ref{fig:3-1-disks-and-centers}.
Hence, any $1\times 3$ or $3\times 1$ galaxy will cover both boxes, denoted by an edge between these two boxes; any $1\times 1$ galaxy will not extend over the boxes, and is shown by a nonexisting edge between the boxes.

\begin{figure}
\centering
\comicII{.226\textwidth}{3-1-disks-centers}
  \caption{\small We place centers (dots) on the middle point of potential edges (light gray) between any pair of boxes (squares). In the remainder of this section, we show only the boxes, not the centers (dots). }
  \label{fig:3-1-disks-and-centers}
\end{figure}

%\todo{explain what the vertex representation means}

\textbf{Overview and filler gadget.}
At a high level, our reduction consists of two main gadgets:
``variable'' gadgets representing the variables of~$F$;
%``negation'' gadgets to force two variables to have opposite values;
and ``clause'' gadgets to form the clauses of~$F$.
%Throughout the discussion, the constructed gadgets have a single solution for any given variable assignment, which will make this reduction parsimonious.

As in Section~\ref{sec:sgr}, we fill any region not covered by these gadgets
with a \defn{filler gadget}, which places a center at every cell of the region.
Figure~\ref{fig:rect-face} shows the filler gadget for a $2 \times 2$ region.
The centers of a filler gadget must be contained in $1 \times 1$ galaxies,
provided every uncovered region is the union of $2 \times 2$ squares.

%The \defn{variable gadget}, shown in Figure~\ref{fig:3-1-var}, has two possible solutions (shown in Figure~\ref{fig:3-1-var}(b) and (c)), each corresponding to one truth assignment for the variable (\textsc{true} and \textsc{false}). We extend the size of the variable gadget to connect to other gadgets---to build \defn{corridors}.
\textbf{Variable gadget.}
Figure~\ref{fig:3-1-var} illustrates the \defn{variable gadget},
which is an alternating pattern of boxes.
While the figure shows the gadget in a simple rectangular form,
variables can also turn arbitrarily, as long as each box has
distance $2$ to exactly two other boxes (i.e., the variable does not
get too close to itself).

%\todo{CS: added lemma and proof here}
\begin{lemma}\label{var-1-3-only-2-sols}
Each variable gadget has exactly two possible solutions (shown in Figure~\ref{fig:3-1-var}(b) and (c)).
\end{lemma}
\begin{proof}
%\begin{proofof}{Lemma~\ref{var-1-3-only-2-sols}}
Because of the filler gadgets, no galaxy centered at a center
of the variable gadget can extend beyond the grid cells
covered by the potential edges of the variable gadget.
Hence, every rectangular galaxy must have a width or height of 1,
which we label as ``vertical'' or ``horizontal'' respectively.
Because the centers are placed with distance 2 and galaxies must be
$180^\circ$ rotationally symmetric about their center,
horizontal and vertical galaxies have a width and height respectively
in $\{1,3\}$.
Because galaxies may not overlap (we aim to decompose the grid),
$1\times 3$/$3\times 1$ galaxies must alternate with $1 \times 1$ galaxies.
There exist exactly two possible solutions that fulfill these conditions.%\hspace*{2.2cm}$\square$
\end{proof}
%\end{proofof}%$\square$%\qed

Each of the two possible solutions corresponds to one truth assignment
for the variable, \textsc{true} and \textsc{false}.
We route each variable gadget to interact with every clause that contains
the variable.  Equivalently, we can imagine each variable gadget as having
a ``tentacle'' to visit every incident clause.


\begin{figure}
\centering
\hspace*{.1\textwidth}
\comic{.226\textwidth}{3-1-var}{(a)}\hfill
\comic{.226\textwidth}{3-1-var-a}{(b)}\hfill
\comic{.226\textwidth}{3-1-var-b}{(c)}
\hspace*{.1\textwidth}
  \caption{\small (a) Variable gadget with two possible states (b) and (c) corresponding to a truth assignment of \textsc{true} and \textsc{false} of the corresponding variable. }
  \label{fig:3-1-var}
\end{figure}

%%%%%%%%%%%%%%
% NEGATION TAKEN OUT
%%%%%%%%%%%%%%
\iffalse
Negating a variable corresponds to inserting a \defn{negation gadget} into the corridor, and to continue with another variable corridor as in Figure~\ref{fig:3-1-neg}.


\begin{figure}
\centering
\hspace*{.1\textwidth}
\comic{.35\textwidth}{3-1-negation-a}{(a)}\hfill
\comic{.35\textwidth}{3-1-negation-b}{(b)}
\hspace*{.1\textwidth}
  \caption{\small Negation gadget in gray, with two black variable gadgets. The incoming variable gadget on the left, has a different truth assignment than the outgoing variable gadget on the right, two cases shown in (a)/(b).
%(a) and (b) show this for the two possible assignments of the incoming variable gadget.
}
  \label{fig:3-1-neg}
\end{figure}
\fi
%%%%%%%%%%%%%%
% END NEGATION TAKEN OUT
%%%%%%%%%%%%%%

\begin{figure}
\centering
\comic{.638\textwidth}{3-1-clause-a}{(a)}\hfill
\comic{.638\textwidth}{3-1-clause-b-2-n}{(b)}\\
\comic{.638\textwidth}{3-1-clause-c-n}{(c)}\hfill
\comic{.638\textwidth}{3-1-clause-d-2-n}{(d)}\\
\comic{.638\textwidth}{3-1-clause-e-n}{(e)}\vspace*{-.2cm}
  \caption{\small (a) Clause gadget in gray, with the three incoming variable gadgets in black.  (b) The three possible states of the clause gadget in red, green, and blue. (c)--(e) Each state of the clause gadget with the corresponding assignments of the variable gadgets; the (only) true variable is shown in the same color as the clause edges.}
  \label{fig:3-1-clause}
\end{figure}

\textbf{Clause gadget.}
Figure~\ref{fig:3-1-clause} shows the clause gadget,
where the clause gadget itself is drawn in gray,
and the three incident variable gadgets are drawn in black.
There are three possible states of the clause gadget,
drawn in red, green, and blue in Figure~\ref{fig:3-1-clause}(b).
Each state forces exactly one of the variables' truth assignments
to be \textsc{true} and all the other variables to be \textsc{false},
as shown in Figure~\ref{fig:3-1-clause}(c)--(e).
%giving a solution to the instance $F$.

\textbf{Putting pieces together.}
For the global construction, we start from an planar embedding of $G$
with edges routed orthogonally on an $O(n) \times O(n)$ grid
\cite{biedl98}, scaled up by a constant factor.
Then we locally replace each clause by a single clause gadget,
and replace each variable by a sufficiently large variable gadget
that extends (via ``tentacles'') to all clauses in which it appears.
%We must ensure that all our gadgets can be connected with each other.
All variable gadgets use one parity
(e.g., all boxes have even $x$ and even $y$ coordinates),
while all clause gadgets 
use the other parity (e.g., all boxes have odd $x$ and odd $y$ coordinates).
Hence, we can always place all gadgets and connect them together as desired.
Because each gadget has a unique solution corresponding to a Boolean
assignment, this reduction is parsimonious.

A solution to an $m\times n$ Rectangular Galaxies puzzle can %obviously 
be verified in polynomial time. We conclude:
%\end{proof}

\begin{theorem}\label{th:sgr13}
	Solving a Rectangular Galaxies puzzle whose solutions have only $1\times 1$, $1\times 3$, and $3\times 1$ galaxies is NP-complete and ASP-complete, and counting the number of solutions is \#P-complete.
\end{theorem}
