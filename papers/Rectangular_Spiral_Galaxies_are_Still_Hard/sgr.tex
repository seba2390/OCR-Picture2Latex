In this section, we show that solving Spiral Galaxies puzzles promised to have solutions with only rectangular galaxies is NP-complete and ASP-complete, and that the corresponding problem of counting the number of solutions is \#P-complete. %, see~\cite{v-aa-10} Chapter 28 and~\cite{y-ccfas-03,ys-ccfas-03} for definitions of \#P-completeness and ASP-completeness, respectively.

% New proof based on https://cocreate.csail.mit.edu/r/S4MNDiTZKoNEjhmd4#R6p5moLX4xwsRoc3t

\iffalse
\begin{theorem}
	Determining whether a Spiral Galaxies puzzle is solvable with only rectangular galaxies is NP-complete.
\end{theorem}
\fi

%\begin{proof}
%The proof is by reduction from \textsc{Planar 1-in-3 SAT}. 
We give a reduction from \textsc{Planar Positive 1-in-3 SAT}. Given an instance $F$ of \textsc{Planar Positive 1-in-3 SAT} with incidence graph $G$, we show how to turn a rectilinear planar embedding of $G$ into a Spiral Galaxies puzzle $P$ such that a solution to $P$ yields a solution to $F$, thereby showing NP-completeness.
Furthermore, there will be a one-to-one correspondence between solutions of $P$ and solutions of $F$, showing \#P-completeness and ASP-completeness. 

\textbf{Overview and filler gadget.}
At a high level, our reduction consists of several gadgets:
``variable'' gadgets representing the variables of~$F$;
``wire'' gadgets to connect variables to clauses;
%``negation'' gadgets to force two variables to have opposite values;
and ``clause'' gadgets to form the clauses of~$F$.
Refer ahead to Figure~\ref{fig:global} for a complete example of the reduction.

 \begin{figure}
\centering
\hspace*{.25\textwidth}
\comic{.1\textwidth}{rect-face-a}{(a)}\hfill
\comic{.1\textwidth}{rect-face-b}{(b)}
\hspace*{.25\textwidth}
  \caption{\small (a) Filler gadget to fill the space in between all other gadgets. (b) Forced solution.}
  \label{fig:rect-face}
\end{figure}

For each region of the board that is not part of these gadgets,
we fill the region with a \defn{filler gadget},
which has a center in every cell of the region.
Figure~\ref{fig:rect-face} shows the filler gadget for a $2 \times 2$ region.
More generally, in every $2 \times 2$ square within the region,
we can argue locally that the four corresponding galaxies
must each consist of a single cell (the one containing the center):
the edges between cells must be galaxy boundaries
to separate the centers into separate galaxies,
and then $180^\circ$ rotational symmetry forces galaxy boundaries
around the $2 \times 2$ square.
As long as the region is the union of such $2 \times 2$ squares,
the filler gadget must consist entirely of single-cell galaxies,
without any interaction with the other gadgets in the construction.
On the other hand, if the region has a width-1 row or a
height-1 column (``thickness~1''), then the galaxy at the center of that cell
might includes cells surrounding the filler gadget.
We must therefore guarantee that every region between other gadgets
is the union of $2 \times 2$ squares (``thickness~2''),
so that the filler gadget has a forced solution of single-cell galaxies.

\begin {figure}
  \centering
  \includegraphics {new-loop}
  \caption 
  {Overall construction of a variable gadget as a sequence of bumps on the top and bottom, where each bump can have a single connection to an incident wire on that side (Figure~\ref{fig:variable-wire}). Bumps without such a connection, such as the one in the bottom right, use a single center.
  }
  \label{fig:variable-loop-fix}
\end {figure}

\textbf{Variable gadget and loop.}
Figure~\ref{fig:variable-loop-fix} shows the overall plan for a variable
gadget: a thickness-$1$ ``variable loop'' that follows a long horizontal
rectangle with regularly spaced bumps on the top and bottom sides,
where each bump has zero or one connection to an incident wire
(which has thickness~$2$).
%We build up this gadget via a sequence of subgadgets.

In more detail, a \defn{variable loop} is a thickness-$1$ loop
built out of the subgadgets in Figures~\ref{fig:variable-straight}
and~\ref{fig:variable-corner}.
Every center is at a cell center, spaced modulo $3$
along the thickness-$1$ loop.
Each center in the middle of a straight piece can choose an $x \times 1$
rectangular galaxy for $x \in \{1,3,5\}$,
which then forces the next galaxy along the straight part to be $6-x \times 1$,
etc., as in Figure~\ref{fig:variable-straight}(b--d).
Each corner subgadget of Figure~\ref{fig:variable-corner}
forbids the center at distance $2$ from the corner
from having a $1 \times 1$ galaxy,
as then the galaxy centered at distance $1$ from the corner
fails to be $180^\circ$ rotationally symmetric;
see Figure~\ref{fig:variable-corner}(d).
Our ``bumpy rectangle'' design from Figure~\ref{fig:variable-loop-fix}
guarantees that every straight portion of a variable loop
is adjacent to at least one corner, and
we can further arrange that all corners have the same alignment.
These properties forbid one global pattern
(subfigure~(d) with blue $5 \times 1$ galaxies)
and leave two possible solution patterns:
all galaxies are $3 \times 1$ or $1 \times 3$ (subfigure~(b)),
and galaxies alternate between blue $1 \times 1$
and red $5 \times 1$/$1 \times 5$ (subfigure~(c)).
These two solutions to the variable gadget correspond to
setting the variable \textsc{false} or \textsc{true}, respectively.

\begin {figure}
  \centering
  \includegraphics {new-straight}
  \caption 
  { (a) A straight piece of a variable gadget.
    (b--c) Two intended valid states.
    (d) Undesired but valid state.
  }
  \label{fig:variable-straight}
%\end {figure}
\medskip
%\begin {figure}
  \centering
  \includegraphics {new-corner}
  \caption 
  { (a) A corner of a variable gadget.
    (b--c) Two valid states.
    (d) The third state is no longer valid.
  }
  \label{fig:variable-corner}
\end {figure}

%\begin{figure}
%\centering
%%\hspace*{.05\textwidth}
%\comic{.75\textwidth}{rect-varloop-a}{(a)}\\
%\comic{.75\textwidth}{rect-varloop-b}{(b)}\\
%%\hspace*{.05\textwidth}\\
%%\hspace*{.05\textwidth}
%\comic{.75\textwidth}{rect-varloop-c}{(c)}
%%\hspace*{.05\textwidth}
%  \caption{\small (a) Variable gadget with two possible solutions (b) and (c) corresponding to a truth assignment of \textsc{true} and \textsc{false}, respectively, of the corresponding variable.}
%  \label{fig:rect-varloop}
%\end{figure}
%

%\begin{lemma}\label{var-2-sols}
%Each variable gadget has exactly two possible solutions.
%\end{lemma}
%\begin{proof}
%By the filler gadgets, the galaxies for centers within the variable gadget
%cannot extend outside the variable gadget.
%We claim that all galaxies within the variable gadget must be rectangles,
%with the following properties:
%%
%\begin{enumerate}
%\item The galaxy from the center in a width-1 corridor has width $1$
%  and height either $2$ or $6$.
%\item The galaxies from the centers in a height-2 corridor
%  all have height $2$, and either all have width $3$,
%  or they alternate between widths $5$ and $1$
%  with the first and last having width~$5$.
%\end{enumerate}
%
%To start, we look at a topmost height-2 corridor,
%which must be connected to downward width-1 corridors,
%forming a ``U-turn''; see Figure~\ref{fig:rec-u-turn}.
%If the leftmost (or rightmost) galaxy in the height-2 corridor
%includes a grid cell from the top row (red) but not the bottom row (blue),
%then this pattern must continue throughout the height-2 corridor (as drawn).
%Then, at the other end, we obtain a galaxy that includes a grid cell from
%the bottom row (red) but not the top row (green), cutting off the top cell
%(green) from being in any galaxy, a contradiction.
%Thus every galaxy from the height-2 corridor must include matching cells
%from both rows.
%By $180^\circ$ rotational symmetry, the leftmost and rightmost galaxies
%also cannot extend beyond the height-2 corridor.
%Thus all galaxies are rectangles.
%The width of the leftmost galaxy rectangle could be $1$, $3$, or $5$
%(any larger would also include the second center).
%By $180^\circ$ rotational symmetry,
%each subsequent galaxy rectangle must have width $5$ minus the previous width.
%
%Moreover, centers are located at edge midpoints. Because galaxies must be $180^\circ$ rotationally symmetric about their center, each galaxy must at least contain the two cells incident to the edge with the center.
%
%Each variable gadget includes a U-turn, that is, height-2 corridors for which both adjacent width-1 corridors are attached to the bottom (or both to the top). These enforce that the galaxies in height-2 corridors must end with a vertical edge of length two; see Figure~\ref{fig:rec-u-turn}. Without loss of generality, we consider a height-2 corridor where both width-1 corridors connect at the bottom. If the leftmost or rightmost galaxy in the height-2 corridor encloses a grid cell in the top height-one row only (red), this enforces the pattern throughout the complete height-2 corridor. Without loss of generality, let this be the leftmost galaxy. Thus, the rightmost galaxy cuts off a set of grid cell(s) in the top row that cannot be caught by any center (green). Consequently, all galaxies in the height-2 corridors must be rectangles. 
%%Because the width-1 corridors have width one only, this is then also enforced for these: t
%
%\begin{figure}
%\centering
%%\hspace*{.25\textwidth}
%\comicII{.5\textwidth}{rect-variable-loop-cont}
%  \caption{\small The topmost height-2 corridor is a U-turn. If only the topmost grid cell of the leftmost column is assigned to the galaxy centered at the leftmost center of the height-2 corridor, then the green grid cell cannot be assigned to any center. (Shaded cells could be assigned to either galaxy.)}
%  \label{fig:rec-u-turn}
%\end{figure}
%
%The galaxies with centers in the width-1 corridors cannot include any grid cells that are not in the same column of cells because both adjacent height-2 corridors can be attached to the same side, thus, extending beyond this column would contradict the galaxies being $180^\circ$ rotationally symmetric about their center. Hence, also all galaxies with centers in the width-1 corridors must be rectangles. If the galaxies with centers in the width-1 corridors extend beyond the two grid cells incident to the edge on which its center is located, they must contain the complete column.
%
%Thus, we have the following properties:
%\begin{itemize}
%\item All galaxies in the variable gadget are rectangles. 
%\item Each galaxy contains the two grid cells incident to the edge on which its center is located.
%\item Each galaxy with a center in a width-1 corridor contains either two or six grid cells. 
%\end{itemize}
%Starting with such a galaxy of two grid cells around a center in a width-1 corridor enforces the galaxy around the first center in each adjacent height-2 corridor to have width 5. This again enforces the next galaxy in the corridor to have width 1, and so on. Analogously, starting with any galaxy around a center in a width-1 corridor with six grid cells enforces the galaxy around the first center in each adjacent height-2 corridor to have width 3. This again enforces the next galaxy in the corridor to have width 3, and so on. Thus, we have exactly two possible solutions for the variable gadgets.
%%
%\end{proof}

%Each variable gadget has two possible solutions, corresponding to setting the variable as \textsc{true} or \textsc{false}, and the rectangle placement is completely determined by the variable assignment. 

%The two possible solutions of the variable gadget correspond to setting the variable as \textsc{true} or \textsc{false},
%and the rectangle placement is completely determined by the variable assignment; see Figure~\ref{fig:rect-varloop}(b) and (c).
%
%From each variable gadget, we can propagate the variable value, creating a \defn{corridor gadget}, shown in
%Figure~\ref{fig:rect-var-corr}. The corridor gadgets are shown in blue and have a center distance of 5 on edge midpoints. %The variable gadgets are shown in black. 
%Again, only two truth assignments are possible, each corresponding to setting the variable to \textsc{true} (Figure~\ref{fig:rect-var-corr}(b)) or \textsc{false} (Figure~\ref{fig:rect-var-corr}(c)). If the corridor does not pick up the correct signal from the variable gadget, then the variable gadget cannot be solved with Rectangular Galaxies; see, e.g., Figure~\ref{fig:rect-var-corr}(d). For the \defn{negation gadget}, we simply connect corridors with centers shifted as shown in Figure~\ref{fig:rect-var-corr}(e). Again, only two truth assignments are possible, each altering the truth assignment from that in the variable gadget Figure~\ref{fig:rect-var-corr}(e)--(i).
%
%\begin{figure}
%\centering
%%\hspace*{.05\textwidth}
%\comic{.286\textwidth}{rect-variable-connection-a}{(a)}\hspace*{.1\textwidth}
%\comic{.286\textwidth}{rect-variable-connection-b}{(b)}\hfill\\
%%\hspace*{.05\textwidth}
%\comic{.286\textwidth}{rect-variable-connection-c}{(c)}\hspace*{.1\textwidth}
%\comic{.286\textwidth}{rect-variable-connection-d}{(d)}\hfill\\
%\comic{.286\textwidth}{rect-variable-connection-i}{(e)}\hspace*{.1\textwidth}
%\comic{.286\textwidth}{rect-variable-connection-e}{(f)}\hfill\\
%\comic{.286\textwidth}{rect-variable-connection-f}{(g)}\hspace*{.1\textwidth}
%\comic{.286\textwidth}{rect-variable-connection-g}{(h)}\hfill\\
%\comic{.286\textwidth}{rect-variable-connection-h}{(i)}\hfill
%%\hspace*{.05\textwidth}
%  \caption{\small (a) Variable gadgets (black centers) and connecting variable corridors (blue centers). (b)/(c) %show the t
%Two possible truth assignments of the connecting corridor depending on the truth assignment in the variable gadget. (d) If the corridor does not pick up the correct signal from the variable gadget the variable gadget cannot be completed. (e)--(i) Negating the truth assignment: (f) and (g) are feasible truth assignments of the connecting corridor depending on the truth assignment in the variable gadget, (h) and (i) show that, if the corridor does not alter the signal from the variable gadget, the variable gadget cannot be completed.}
%  \label{fig:rect-var-corr}
%\end{figure}

\textbf{Wire gadget.}
While thickness $1$ was useful to force the variable gadget to have exactly
two solutions, it seems difficult to copy that truth value onto multiple
thickness-$1$ wires.
We thus introduce thickness-$2$ \defn{wire gadgets}
as shown in Figure~\ref{fig:wire}.
In their simplest form, a wire gadget is a vertical width-$2$ rectangle
with a center at edge midpoints in the even rows modulo~$2$.
The two intended solutions use alternating $2 \times 1$ and $2 \times 3$
galaxies as shown in Figure~\ref{fig:wire}(b--c),
corresponding to \textsc{false} and \textsc{true} signals respectively,

Unfortunately, the basic wire also allows two nonrectangular solutions,
as shown in Figure~\ref{fig:wire}(d--e).
Each of these undesired solutions is prevented by a corresponding
\defn{shift gadget} shown in Figures~\ref{fig:shift-right}
and~\ref{fig:shift-left} respectively.
In either gadget, covering the corner pixels in the middle row
forces one side of the wire to use rectangles.
By guaranteeing that each wire shifts at least once to the right
and at least once to the left (including possibly one shift
canceling out the other), we avoid the undesired solutions,
forcing one of the two intended solutions in Figure~\ref{fig:wire}(b--c).

%Thus we add the U-turn of Figure~\ref{fig:wire-U} to every wire gadget.
%The corner pixels of these gadgets must be captured by the galaxy of
%one of the two nearby centers, which by local case analysis forces
%rectangular galaxies within this gadget, which then propagate into the
%adjacent wire gadgets.
%This addition preserves the two intended rectangular solutions,
%as shown in Figure~\ref{fig:wire-U}(b--c).

\begin {figure}
  \centering
  \includegraphics {new-wire}
  \caption 
  { (a) A wire gadget for connecting a variable to a clause.
    (b--c) Intended valid rectangular states.
    (d--e) Undesired but valid states.
  }
  \label{fig:wire}
\end {figure}
\medskip
\begin {figure}
%  \centering
%  \includegraphics {new-U}
%  \caption 
%  { (a) A U-turn in a wire gadget.
%    (b--c) Two compatible valid states.
%    (d--e) The other two states are no longer valid,
%    failing to cover some corner cells.
%  }
%  \label{fig:wire-U}
%\end {figure}
%\begin {figure}
  \centering
  \includegraphics {new-shift-right}
  \caption 
  { (a) A shift to the right.
    (b---c, e) Three states are still valid.
    (d) The undesired wire state from Figure~\ref{fig:wire}(d) is no longer valid.
  }
  \label{fig:shift-right}
%\end {figure}
\medskip
%\begin {figure}
  \centering
  \includegraphics {new-shift-left}
  \caption 
  { (a) A shift to the left.
    (b---d) Three states are still valid.
    (e) The undesired wire state from Figure~\ref{fig:wire}(e) is no longer valid.
  }
  \label{fig:shift-left}
\end {figure}

Each shift gadget can shift the wire horizontally
by an arbitrary amount $\geq 3$,
as shown in Figure~\ref{fig:shift-var}.
The height-1 transition row makes it easy to argue that the same
solutions in Figures~\ref{fig:shift-right} and~\ref{fig:shift-left}
are forced.
Thus a wire can approximately follow any desired $y$-monotone path
that starts and ends with a vertical segment.
The $x$ coordinates of the start and end vertical segments
can be specified exactly;
the only approximations are that the shifts need $\Omega(1)$ room to navigate,
and we always use at least two shifts.

\begin {figure}
  \centering
  \includegraphics {new-shift-var}
  \caption 
  { Shift gadgets from Figures~\ref{fig:shift-right} and~\ref{fig:shift-left}
    can shift by an arbitrary horizontal amount $\geq 3$.
    (a, c) An even shift places the middle center at a cell center.
    (b) An odd shift places the middle center at an edge midpoint.
  }
  \label{fig:shift-var}
\end {figure}

\textbf{Connecting variables to wires.}
Figure~\ref{fig:variable-wire} shows how to communicate the value of a
variable loop into the signal of a wire.
The vertical thickness-$2$ wire attaches to two consecutive empty cells
of a horizontal segment of the variable loop.
If the wire's galaxies capture those two cells, then the two adjacent
centers of the variable loop must have $1 \times 1$ galaxies,
forcing alternation between $1 \times 1$ and $1 \times 5$ thereafter,
as in Figure~\ref{fig:variable-wire}(c).
Conversely, if the wire's galaxies do not capture those two cells,
then the two adjacent centers of the variable loop must capture them,
forcing the one other solution of the variable loop with
$1 \times 3$ rectangles, as in Figure~\ref{fig:variable-wire}(b).

\begin {figure}
  \centering
  \includegraphics {new-split}
  \caption 
  { (a) A place where a wire from the variable loop is connected.
    (b--c) Two compatible valid states.
  }
  \label{fig:variable-wire}
\end {figure}

Figure~\ref{fig:wires-need-U-turns} shows that the unintended solutions of
wire gadgets from Figure~\ref{fig:wire}(d--e) can propagate into the
variable gadget.  Thus connections to variable gadgets are insufficient to
fix these problems, which is why we needed to introduce the shift gadgets of
Figure~\ref{fig:shift-right} and~\ref{fig:shift-left}
to fix wire gadgets locally.

\begin {figure}
  \centering
  \includegraphics {new-split-problem}
  \caption 
  { Why we needed U-turns in wire gadgets: connections to variables
    are insufficient to force the intended rectangular solutions.
  }
  \label{fig:wires-need-U-turns}
\end {figure}

If we connected multiple wires to the same straight segment
of a variable gadget, then the straight segment between the two connections
would not be incident to any variable corners,
so it could be in the third state of Figure~\ref{fig:variable-straight}(d).
This is why we limit each bump of the variable loop to
having at most one connection to a wire,
so that every connection gadget from Figure~\ref{fig:variable-wire}
is surrounded on either side
by a corner gadget from Figure~\ref{fig:variable-corner}.
%This problem cannot arise if we guarantee (as mentioned above)
%that every straight segment of the variable gadget is incident to
%at least one corner from Figure~\ref{fig:variable-corner}.
%In particular, we can add a corner from Figure~\ref{fig:variable-corner}
%before and after every connection gadget from Figure~\ref{fig:variable-wire}.

Each connection between variable and wire
removes one alternation from the variable gadget,
as indicated by the coloring in Figure~\ref{fig:variable-wire}.
If the total number of connections from a variable gadget is odd,
then we would have a parity mismatch around the variable loop.
Figure~\ref{fig:variable-loop-fix} illustrates a parity fix
by placing a single center in the middle of a bump of the variable gadget
that has no connection to a wire (similar to Figure~\ref{fig:shift-var}).
For simplicity, we can use this construction for every bump
with no connection to a wire.
Because the two adjacent corners turn in the same direction,
this center can be covered in exactly two ways:
with a rectangle that is either the full horizontal width
or $2$ smaller.
% Another fix: \url{https://cocreate.csail.mit.edu/r/SdhgBEBQhzHgTWJkK}


%We show the \defn{corridor bend} gadget in Figure~\ref{fig:rect-corridor}. Moreover, we use a \defn{corridor shift} gadget, shown in Figure~\ref{fig:rect-shift5s}, which allows us to shift the location of the corridor by any number $\geq 3$. These are used to connect to the clause gadgets in the appropriate places.
%
%\begin{figure}
%\centering
%%\hspace*{.05\textwidth}
%\comic{.3\textwidth}{rect-corridor-a}{(a)}\hfill
%\comic{.3\textwidth}{rect-corridor-b}{(b)}\hfill
%%\hspace*{.05\textwidth}\\
%%\hspace*{.05\textwidth}
%\comic{.3\textwidth}{rect-corridor-c}{(c)}
%%\hspace*{.05\textwidth}
%  \caption{\small (a) Variable corridors with bend. (b)/(c) %show the 
%Two possible covers with rectangular galaxies.  }
%  \label{fig:rect-corridor}
%\end{figure}
%
%
%
%
%\begin{figure}
%\centering
%%\hspace*{.05\textwidth}
%\comic{.286\textwidth}{rect-shifts5s-a}{(a)}\hfill
%\comic{.286\textwidth}{rect-shifts5s-b}{(b)}\hfill
%%\hspace*{.05\textwidth}\\
%%\hspace*{.05\textwidth}
%\comic{.286\textwidth}{rect-shifts5s-c}{(c)}
%%\hspace*{.05\textwidth}
%  \caption{\small (a) Variable corridors can be shifted by any number $\geq 3$ (here shown for a shift of 3 as indicated in green). (b) and (c) depict the different variable assignments.  }
%  \label{fig:rect-shift5s}
%\end{figure}

\begin {figure}
  \centering
  \includegraphics {new-clause}
  \caption 
  { The clause gadget connects three wire gadgets (green).
    (b--d) Valid solutions with exactly one true signal.
    (e) Impossibility with zero true signals.
    (f--h) Impossibility with two true signals.
    (i) Impossibility with three true signals.
  }
  \label{fig:clause}
\end {figure}

\textbf{Clause gadget.}
The last gadget is the clause gadget shown in Figure~\ref{fig:clause},
which implements a 1-in-3 constraint between three incident wire gadgets
via a $1 \times 22$ rectangle.
Although we draw all three wire gadgets as below the clause gadget,
each wire could in fact be either above or below the clause gadget.
Figure~\ref{fig:clause}(b--d) illustrates that the clause gadget has a
(unique) solution when exactly one of the wires carries a true signal.
Indeed, any true signal represented by a wire with a galaxy covering
two cells of the clause gadget locally forces the galaxies of the
one or two adjacent centers within the clause, which then propagates to force
the galaxies of the other centers within the clause.
This forced solution turns out to be inconsistent with
any of the other wires carrying a true signal, causing overlapping galaxies
or uncovered cells as shown in Figure~\ref{fig:clause}(e--i).
Therefore the clause gadget has a (unique) solution if and only if
exactly one of the incident wires carries a true signal.

%To combine the corridor gadgets into clauses, we use the \defn{clause gadget} shown in Figure~\ref{fig:rect-clause}. At least one of the three variables' truth assignments is forced to be true (see Figure~\ref{fig:rect-clause}(b)--(d)), more than one (see Figure~\ref{fig:rect-clause}(f)--(h)) or none (see Figure~\ref{fig:rect-clause}(e)) cannot be set to true, hence, it forces exactly one true assignment, giving a solution to the instance $F$.
%
%\begin{figure}
%\centering
%%\hspace*{.05\textwidth}
%\comic{.43\textwidth}{rect-clause-n-a}{(a)}\hfill
%\comic{.43\textwidth}{rect-clause-n-b}{(b)}\\%\vspace{.2cm}\\
%%\hspace*{.05\textwidth}\\
%%\hspace*{.05\textwidth}
%\comic{.43\textwidth}{rect-clause-n-c}{(c)}\hfill
%\comic{.43\textwidth}{rect-clause-n-d}{(d)}\\
%\comic{.43\textwidth}{rect-clause-n-e}{(e)}\hfill
%\comic{.43\textwidth}{rect-clause-n-f}{(f)}\\
%\comic{.43\textwidth}{rect-clause-n-g}{(g)}\hfill
%\comic{.43\textwidth}{rect-clause-n-h}{(h)}
%%\hspace*{.05\textwidth}
%  \caption{\small (a) Clause gadget shown in black and the connecting variable corridors shown in blue. (b)--(d) The clause gadget can be filled in with rectangle galaxies iff exactly one of the three connecting variables has a truth assignment that fulfills the clause. %These three cases are shown in (b), (c) and (d). 
%%(e) If all variables do not fulfill the clause, the clause cannot be completed. (f)--(h) If more than one variable fulfills the clause, the clause cannot be completed.
%The clause cannot be completed if all variables do not fulfill the clause (e), or if more than one variable fulfills the clause (f)--(h).}
%  \label{fig:rect-clause}
%\end{figure}

\textbf{Putting pieces together.}
For the global construction, we construct an orthogonal planar embedding of $G$
where each vertex is represented by a horizontal line segment
with endpoints on the grid,
and each edge is represented by a polygonal line on the grid
that is $y$-monotone and starts and ends with a vertical segment.
Such an embedding can be obtained by
constructing a planar straight-line embedding of $G$
with vertices on an $O(n) \times O(n)$ grid \cite{schnyder90};
applying the transformation of \cite[Theorem~5]{biedl14}
into a $y$-monotone orthogonal drawing where vertices are horizontal segments;
and then staggering vertices with equal $y$ coordinates.
The resulting grid size is $O(n) \times O(n^2)$.

Figure~\ref{fig:global}(a) shows a particularly structured planar embedding
of $G$
%an instance of \textsc{Planar Positive 1-in-3 SAT},
where the variables all lie on a horizontal line,
every clause is either above or below this line,
and variable--clause connections do not cross the line or each other.
Then it is easy to draw each variable--clause connection with a $y$-monotone
path that starts and ends with a vertical segment.
Mulzer and Rote \cite{mulzer08} proved that \textsc{Planar Positive 1-in-3 SAT}
is NP-complete in this form.  Their reduction has also been observed to be
the end of a chain of parsimonious reductions from 3SAT
\cite{6.890-planar-1-in-3},
so it establishes ASP- and \#P-completeness too.
We describe the more general reduction in the previous paragraph
so as to not rely on this unpublished observation,
but use the simpler form for figures.

\begin {figure}
  \centering
  \includegraphics[width=\linewidth]{new-global-both}
  \caption 
  { (a) An embedded instance of Planar Positive 1-in-3 SAT, with all variables on a common horizontal line and all clauses either above or below the line.
    (b) The corresponding gadgets of Spiral Galaxies, excluding filler.
    (c) Full Spiral Galaxies puzzle with filler gadgets.
  }
  \label{fig:global}
\end {figure}

As illustrated in Figure~\ref{fig:global}(b),
we scale up the orthogonal planar embedding by a constant factor;
replace each variable or clause vertex with a variable gadget or clause gadget
respectively,
and replace each $y$-monotone edge with a wire gadget
(including at least one right shift and at least one left shift).
Finally, we place the filler gadget in the regions between other gadgets,
as shown in Figure~\ref{fig:global}(c).
Because each gadget has a unique solution corresponding to a Boolean
assignment, this reduction is parsimonious.

Finally we analyze alignment issues in this construction.
The shift gadgets of Figure~\ref{fig:shift-var} provide exact
horizontal positioning, so there are no constraints on horizontal
positioning of gadgets.
However, vertical positioning controls the parity of the \textsc{false}
and \textsc{true} values of a wire, as in the yellow and green coloring
of Figure~\ref{fig:wire}.
We place all variable loops to have the same vertical positioning modulo~$4$.
Thus we obtain consistent heights modulo $4$
for the row of a wire between a \textsc{false} (yellow) and \textsc{true}
(yellow) center that needs to be aligned with a clause:
for wires extending upward, this is the row above a \textsc{true} center, while
for wires extending downward, this is the row below a \textsc{true} center.
Observe in Figure~\ref{fig:variable-loop-fix} that these rows have the
same height modulo $4$ within a single variable loop,
and that wire gadgets of Figure~\ref{fig:wire} and shift gadgets of
Figures~\ref{fig:shift-right}--\ref{fig:shift-var}
preserve this height modulo~$4$.
We can then place all clause gadgets at such $y$ coordinates.

A solution to an $m\times n$ Spiral Galaxies puzzle can be verified in polynomial time. We conclude: 

\begin{theorem}
	Solving a Spiral Galaxies puzzle whose solutions have only rectangular galaxies is NP-complete and ASP-complete, and counting the number of solutions is \#P-complete.
\end{theorem}
%\end{proof}
