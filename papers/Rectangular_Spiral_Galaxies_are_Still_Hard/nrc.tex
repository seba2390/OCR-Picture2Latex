In this section, we are given a shape $\mathcal{S}$ on a Spiral Galaxies board, and we aim to find the minimum number of centers such that there exist galaxies with these centers that exactly cover the given shape $\mathcal{S}$. We show that this problem is NP-complete by another reduction from \textsc{Planar Positive 1-in-3 SAT}.

\begin{theorem}\label{th:nrc}
	It is NP-complete to minimize the number of centers on a Spiral Galaxies board such that galaxies with these centers exactly cover a given shape~$\mathcal{S}$.
\end{theorem}

%\todo{expand proof?}
%\begin{proof}
%
For our reduction, the input is a \textsc{Planar Positive 1-in-3 SAT} formula, and the output is a ``picture'' --- a subset of cells that should be ``colored black''. In the remainder of the section, we focus on the cells that are part of the picture; we will say that the cells which are not part of the picture are \defn{forbidden}.

First we will define some \defn{low-level gadgets}, which each have a constant size.
Then we will build \defn{high-level gadgets}, which consist of multiple low-level gadgets, and which will be used to encode the 1-in-3-SAT formula.

\subsection {Low-level gadgets}

\textbf{Local center gadgets.}
First we introduce \defn{local center gadgets}: thin constructions with unique shapes, which ensure that there must be at least one center in each of them; see Figure~\ref {fig:center-gadget}. 
The idea will now be to construct a shape that can be covered with exactly this set of centers if and only if the 1-in-3-SAT instance is satisfiable. 

\begin{figure}
\centering
\includegraphics [scale=0.75]{center-gadget}
  \caption{\small The local center gadget must have at least one center. (a)--(d) Different shapes ensure we cannot include them in larger galaxies.}
  \label{fig:center-gadget}
\end{figure}

Specifically, each local center gadget consists of a single cell with a \defn{desired galaxy center}. Its four direct neighbors are all empty, and its four diagonal neighbors are all forbidden. The cells left and right of the center will serve as connections to other low-level gadgets (we describe the gadgets as if they are vertically oriented; in the actual construction they may be rotated by $90^\circ$). The cells above and below the center are extended by symmetrically shaped paths of empty cells which are surrounded by forbidden cells.

We can immediately observe that, if there is \emph{not} a center at the location of the desired galaxy center in a local center gadget, then either the entire gadget will be part of a larger galaxy with a center outside the gadget, or there must be at least two centers placed inside the gadget. In the first case, because of the symmetry of the galaxies, there must be a second copy of the local center gadget that is also part of the same galaxy. For this reason, we use not one but several different versions of the gadget.
As we will see later, $O(1)$ different version will be sufficient.



 \begin{figure}
\centering
\includegraphics [scale=0.75]{block-gadget}
  \caption{\small (a) The block gadget. (b) A straight block. (c) A corner block. (d) A clause block.}
  \label{fig:block-gadget}
\end{figure}

\textbf{Block gadgets.}
The second low-level gadget we use is the \defn{block gadget}.
A block gadget is simply a $5\times5$ room surrounded by forbidden cells, except for possibly in the centers of its four sides. 
The idea is that a block gadget will not contain galaxy centers in an optimal solution, but rather will be connected to local center gadgets and be part of their galaxies. 
Depending on which of the sides of a block gadget are connected to other gadgets, we distinguish \defn{straight blocks}, \defn{corner blocks}, and \defn{clause blocks}; see Figure~\ref {fig:block-gadget}.

\begin {observation} \label {obs:desire}
  Suppose we place local center gadgets and block gadgets in such a way that, for each block gadget $B$, all local center gadgets that $B$ is connected to have different shapes.
  Then we must place at least one galaxy center inside each local center gadget.
\end {observation}

\begin{figure}
\centering
\includegraphics [scale=0.75]{end-gadget}
  \caption{\small An end gadget has one center, whether or not a block is used by the galaxy next to it.}
  \label{fig:end-gadget}
\end{figure}

\textbf{End gadgets.}
Finally, our last low-level gadget is the \defn{end gadget}. 
An end gadget is similar to a bock gadget, but it is a $7\times5$ room and it has only one connection, at the center of one of its sides of length $5$; see Figure~\ref {fig:end-gadget}.

The idea is that, if we place an end gadget at the end of a chain that satisfies the conditions of Observation~\ref {obs:desire}, then each end gadget will require at least one galaxy center to be placed inside it, because the center in the local center gadget next to it has only a $5 \times 5$ room on the other side (unless two end gadgets are directly connected, which is not useful). Depending on where we place it, more or less of the room is available to be included in neighboring galaxies.

\begin {observation} \label {obs:end}
  If an end gadget is connected to a local center gadget, and on the other side of the local center gadget is not another end gadget, then we must place at least one galaxy center inside the end gadget.
\end {observation}

\subsection {High-level gadgets}

With these low-level gadgets in place, we now turn to constructing high-level gadgets that will encode the variables and clauses of our 1-in-3-SAT instance.

\textbf{Fix gadgets.}
First, we define the \defn{fix gadget}: an alternating sequence of four local center gadgets and three block gadgets making a U-turn as in Figure~\ref {fig:fix-gadget}.
The purpose of this gadget is to ensure that, in an optimal solution, each $5\times5$ block will be completely included in a single galaxy, and not split among several galaxies.

\begin {lemma} \label {lem:fix}
In any chain of blocks that contains a fix gadget, an optimal solution requires that each block must completely belong to a single galaxy.
\end {lemma}
\begin {proof}
First, by Observation~\ref {obs:desire}, an optimal solution has exactly one center in each local center gadgets, and none in the block gadgets.
This implies that if {\em any} block gadget in the chain does not completely belong to a single galaxy, then, by symmetry, {\em every} block gadget in the chain does not completely belong to a single galaxy.

Now, suppose we have a fix gadget consisting of blocks $A$, $B$, and $C$, where $A$ and $C$ are corner blocks but $B$ is a straight block, and suppose $A$ is split among multiple galaxies, say a red one at the top and a blue one at the right; refer to Figure~\ref {fig:fix-gadget}(b). Then, by symmetry of the blue galaxy, the center left pixel of $B$ must also be blue and the bottom center pixel of $B$ must also be red. Then, the top center pixel of $C$ must be red (a priori, this could be the same or a different red galaxy --- by connectivity it must be a new galaxy) and the center right pixel of $C$ must be blue (again, either the same or a different blue galaxy) by symmetry of the (second) red galaxy. But when both the center left and top center pixels of $C$ are red then $C$ must be completely red, a contradiction.
  %\hspace*{2.2cm}$\square$
\end {proof}

 \begin{figure}
\centering
\includegraphics [scale=0.75]{block-argument}
  \caption{\small (a) The fix gadget. (b) If the left block is split among multiple galaxies: contradiction.}
  \label{fig:fix-gadget}
\end{figure}


 \begin{figure}
\centering
\includegraphics [scale=0.75]{variable-chain}
  \caption{\small A variable chain and its two possible truth assignments.}
  \label{fig:variable-chain}
\end{figure}

\textbf{Variable gadgets.}
A \defn{variable chain} consists of alternating local center gadgets and block gadgets starting and ending at end gadgets; see Figure~\ref{fig:variable-chain} for a variable chain.
Each variable chain must include at least one fix gadget.
Then, by Lemma~\ref {lem:fix}, each block gadget on a variable chain must belong to exactly one of its two neighboring local center gadgets; this choice propagates throughout the entire chain and encodes the value \textsc{true} or \textsc{false} of one of the variables in our 1-in-3-SAT instance.
Specifically, we show:

\begin {lemma} \label {lem:variable}
Consider a variable chain which
\begin {itemize}
  \item has $k>1$ local center gadgets and $k-1$ block gadgets;
  \item contains at least one fix gadget; and
  \item has the property that every three consecutive local center gadgets have three different shapes.
\end {itemize}
Then there exists no set of $k+1$ or fewer galaxy centers that form a valid puzzle, and there are exactly two sets of $k+2$ galaxy centers that form a valid puzzle, both of which include all $k$ \emph{desired} galaxy centers.
\end {lemma}
\begin {proof}
By Observation~\ref {obs:end} and because $k > 1$, we must place one galaxy center inside each end gadget.
By Observation~\ref {obs:desire} and because we vary the shapes of the local center gadgets, we must place at least one galaxy center at each of the $k$ local center gadgets.
Indeed, a galaxy centered in a local center gadget cannot include both adjacent local center gadgets, since they have different shapes, and a galaxy centered outside a local center gadget cannot include both incident local center gadgets, since they have different shapes.
So, we need at least $k+2$ centers.
This is also sufficient, as shown by the two solutions with $k+2$ centers in Figure~\ref {fig:variable-chain}.
Finally, by Lemma~\ref {lem:fix}, we cannot have more than two possible shapes for each galaxy.
\end {proof}

Note that within one variable chain, we can realize the conditions of Lemma~\ref {lem:variable} with only three different local center gadget shapes, which we alternate cyclicly.

 \begin{figure}
\centering
\includegraphics [scale=0.75]{split-gadget}
  \caption{\small The split gadget is essentially three end gadgets, but can only be solved with a single center if either all three blocks are used or all three are not used.}
  \label{fig:split-gadget}
\end{figure}

\textbf{Split gadgets.}
In order to connect variables to multiple clauses, we need to be able to split them.
Rather than splitting chains in the usual sense, our \defn{split gadget} takes three independent variable chains and forces them to have the same state by combining three end gadgets into a single larger end gadget; see Figure~\ref {fig:split-gadget}. This effectively creates a ``split'' chain with three remaining ends that have the same state; by repeating this we may create as many copies of a variable as required.

The idea behind the split gadget is simple:
by combining three end gadgets into one large $7\times15$ room, we no longer know that we need three separate galaxy centers; in fact, applying Observation~\ref {obs:end} to the three adjacent local center gadgets, we potentially only require to place one galaxy center instead of three.
Observe that this is indeed achievable, but \emph{only} when all three chains are in the same state, because otherwise the remaining shape of the room is not symmetric by $180^\circ$ rotation (and thus cannot be covered by a single additional galaxy).

\begin{observation}
  A split gadget requires only one galaxy center if and only if either all three adjacent local center gadgets do use a $5\times5$ area of its room, or if none of them do.
\end{observation}

\textbf{Clause gadgets.}
A clause of our 1-in-3-SAT formula is now simply encoded by a single clause block where three variable chains meet: it can be solved without using an additional center if and only if precisely one of the three chains needs to use the block.

In order to avoid that galaxies centered on clause blocks include local center gadgets, the three local center gadgets directly incident to the clause block must have distinct shapes.
To achieve this, we may start with an edge coloring of the variable--clause incidence graph and use three unique shapes for each color. Every planar graph of maximal degree $3$ admits an edge coloring with 4 colors, which results in 12 distinct shapes for our local center gadgets.
It is possible, but not necessary, to reduce the number of shapes further.

\subsection{Global construction}

With our high-level gadgets now also in place, we can describe the global construction of our reduction.
%
We start from an planar embedding of $G$
with edges routed orthogonally on an $O(n) \times O(n)$ grid
\cite{biedl98}, scaled up by a constant factor.
We locally replace each clause by a single clause block, each variable by a sufficient number of split gadgets connected by variable chains, and each edge by variable chains connecting variables to clauses.

%Note that in this reduction we do not need negation gadgets: by the symmetric nature of galaxies, each chain already contains an alternating sequence of ``positive'' and ``negative'' instances of its variable.

\textbf{Distance considerations.}
We must make sure that all our gadgets are aligned with the grid, and can be connected to each other. Furthermore, we need to ensure that the number of block gadgets in each variable chain
%is odd or even depending on the required state of the variable in the respective clause.
has the same parity so that all clauses use positive forms of the variable.

We observe that we can achieve all of these requirements by simply adjusting the distance between adjacent block gadgets and allowing a slightly richer class of local center gadgets: instead of a width of $3$ we may give them a width of any desired integer $w$. If $w$ is even, we must place a galaxy center on an edge and not in the center of a cell, and we correspondingly also adjust the vertical arms of the gadget by giving them a width of $2$ rather than $1$.

This concludes the proof of Theorem~\ref{th:nrc}.
%
%\end{proof}










