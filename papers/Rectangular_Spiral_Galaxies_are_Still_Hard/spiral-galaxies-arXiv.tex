% for section-numbered lemmas etc., use "numberwithinsect"
%\documentclass[a4paper,english,numberwithinsect]{eurocg21-submission}
\documentclass{elsarticle}

\usepackage{lineno,hyperref}
\modulolinenumbers[5]

% the recommended bibstyle
\bibliographystyle{elsarticle-num}

% -------------------------------------------------------------------
\usepackage{microtype} % if unwanted, comment out
\usepackage [usenames] {xcolor}
\usepackage {placeins}

%helpful if your graphic files are in another directory
\graphicspath{{./figures/}}

% Author macros::begin %%%%%%%%%%%%%%%%%%%%%%%%%%%%%%%%%%%%%%%%%%%%%%%%
\newcommand\polylog{{\rm polylog}}

\usepackage{todonotes}
\usepackage{amssymb}
\usepackage{amsmath}



\newcommand{\rmk}[3]{\textcolor{blue}{\textsc{#1 #2:}}
\textcolor{red}{\textsf{#3}}}
\newcommand{\maarten}[2][says]{\rmk{Maarten}{#1}{#2}}
\newcommand{\erik}[2][says]{\rmk{Erik}{#1}{#2}}


\newcommand{\bigo}{\operatorname{o}}
\newcommand{\bigO}{\operatorname{O}}
\newcommand{\bigOmega}{\operatorname{\Omega}}
\newcommand{\bigTheta}{\operatorname{\Theta}}
\newcommand{\interior}{\operatorname{int}}
\newcommand{\N}{\mathds{N}}
\newcommand{\NP}{\textit{NP}}
\newcommand{\opt}{\operatorname{OPT}}
\newcommand{\Q}{\mathds{Q}}
\newcommand{\R}{\mathds{R}}
\newcommand{\tgp}{\operatorname{TGP}}
\newcommand{\agp}{\operatorname{AGP}}
\newcommand{\atgp}{\operatorname{ATGP}}
\newcommand{\V}{\operatorname{\mathcal{V}}}
\newcommand{\VK}{\operatorname{VK}}
\newcommand{\Z}{\mathbb{Z}}
\newcommand{\A}{\mathcal{A}}
\newcommand{\defn}[1]{\textbf{\textit{\boldmath #1}}}

\newtheorem{theorem}{Theorem}
\newtheorem{corollary}{Corollary}
\newproof{proof}{Proof}
\newtheorem{lemma}{Lemma}
\newtheorem{observation}{Observation}
\newtheorem{definition}{Definition}

% Redefinition of elsevier proof environment to put a \qed at the end
% of every proof.
\let\realendproof=\endproof
\def\endproof{\hspace*{\fill}\qed\realendproof}

\newcommand{\proofofref}{}
\newproof{zproofof}{Proof of \proofofref}
\newenvironment{proofof}[1]
 {\renewcommand{\proofofref}{#1}\zproofof}
 {\endzproofof}

\hyphenation{rec-tan-gu-lar}



\def\comic#1#2#3{\parbox{#1}{\centering\includegraphics[width=#1]{#2}\\\vspace*{-.15cm}{\footnotesize #3}\vspace*{.25cm}}}
\def\comicII#1#2{\parbox{#1}{\centering\includegraphics[width=#1]{#2}}}
\def\comicIII#1#2#3{\parbox{#1}{\centering\includegraphics[width=#1]{#3}\\{\footnotesize \##2}}}

%\graphicspath{{./graphics/}}%helpful if your graphic files are in another directory

%\bibliographystyle{plainurl}% the recommended bibstyle

% Put figures and text together
\def\textfraction{0.01}
\def\topfraction{0.99}
\def\dbltopfraction{0.99}
\def\bottomfraction{0.99}
\def\floatpagefraction{0.99}
\def\dblfloatpagefraction{0.99}
\def\dbltopnumber{100}


% Author metadata::begin %%%%%%%%%%%%%%%%%%%%%%%%%%%%%%%%%%%%%%%%%%%%%%%%
\title{Rectangular Spiral Galaxies are Still Hard\footnote{This research was performed in part at the 30th and the 33rd Bellairs Winter Workshop
on Computational Geometry. We thank all other participants for a fruitful
atmosphere.}\hspace*{.05cm}\footnote{A preliminary extended abstract appeared as a preprint in the Proceedings of the 37th European Workshop on Computational Geometry~\cite{dls-rsgsh-21}.}}
%optional, in case that the title is too long;
%the running title should fit into the top page column
%\titlerunning{A Contribution to EuroCG 2021}

\author{Erik D. Demaine}
\address{Computer Science and Artificial Intelligence Laboratory,\\
Massachusetts Institute of Technology\\
  \texttt{edemaine@mit.edu}}

\author{Maarten L{\"o}ffler}
\address{Department of Information and Computing Sciences, 
Universiteit Utrecht\\
  \texttt{m.loffler@uu.nl}}

\author{Christiane Schmidt}
\address{Department of Science and Technology, Link\"oping University\\
  \texttt{christiane.schmidt@liu.se}}



% Author macros::end %%%%%%%%%%%%%%%%%%%%%%%%%%%%%%%%%%%%%%%%%%%%%%%%%


\begin{document}

\begin{abstract}
Spiral Galaxies is a pencil-and-paper puzzle played on a grid of unit squares: given a set of points called \emph{centers}, the goal is to partition the grid into polyominoes such that each polyomino contains exactly one center and is $180^\circ$ rotationally symmetric about its center.
We show that this puzzle is NP-complete, ASP-complete, and \#P-complete even if
(a)~all solutions to the puzzle have rectangles for polyominoes; or
(b)~the polyominoes are required to be rectangles and
all solutions to the puzzle have just
$1\times1$, $1\times3$, and $3\times1$ rectangles.
The proof for the latter variant also implies NP/ASP/\#P-completeness of finding a noncrossing perfect matching in distance-$2$ grid graphs where edges connect vertices of Euclidean distance~$2$.
Moreover, we prove NP-completeness of the design problem of minimizing the number of centers such that there exists a set of galaxies that exactly cover a given shape.
\end{abstract}

\maketitle

\section{Introduction}\label{sec:introduction}





Spiral Galaxies is a pencil-and-paper puzzle published by Nikoli since 2001~\cite{n-sg-01} under the name ``Tentai Show''~\cite{gsp-18}. It is played on a grid of unit squares with given ``centers'' --- points located at grid vertices, cell centers, or edge midpoints. The goal is to decompose the grid into polyominoes called ``galaxies'' such that each galaxy contains exactly one center and is $180^\circ$ rotationally symmetric about its center; see Figure~\ref{fig:sg-ex}(a). The solution for a Spiral Galaxies puzzle may not be unique, but typically puzzles are designed to have a unique solution.

\begin{figure}[h]
\centering
\includegraphics[height=7em]{sg-ex.pdf}
\hfill
\includegraphics[height=7em]{example-rectangles.pdf}
\hfill
\includegraphics[height=7em]{sg-ex-bw.pdf}\\\bigskip
\includegraphics[height=7em]{sg-ex-sol.pdf}
\hfill
\includegraphics[height=7em]{example-rectangles-solution.pdf}
\hfill
\includegraphics[height=7em]{sg-ex-bw-sol.pdf}\\
\mbox{}\hfill(a)\hfill\mbox{}\hfill(b)\hfill\mbox{}\hfill(c)\hfill\mbox{}
  \caption
  { \small Several styles of Spiral Galaxies puzzles (top) and their solutions (bottom).
    (a) Classic Spiral Galaxies puzzle.
    (b) Rectangular Galaxies puzzle.
    (c) Spiral Galaxies puzzle with black and white centers such that the polyominoes containing the black centers in the solution yield a picture.
  }
  \label{fig:sg-ex}
\end{figure}


Friedman~\cite{f-sgpnp-02} proved Spiral Galaxies NP-complete for general polyomino galaxies.
His proof uses galaxies of area up to $22$.
Fertin et al.~\cite{fjk-tagsg-15} proved NP-completeness for galaxies of area at most $7$, while the puzzle becomes solvable in polynomial time if galaxies are restricted to have area at most $2$ or to be squares.
How much simpler can we make the galaxy shapes and still have Spiral Galaxies puzzles be NP-complete?

In this paper, we consider Spiral Galaxies with rectangular galaxies,
in two possible senses:
%
\begin{enumerate}
\item In \defn{Rectangular Galaxies}, the polyomino galaxies are
  required to be rectangles.  This is an additional restriction on the puzzle
  beyond $180^\circ$ rotational symmetry around the unique center
  contained in each rectangle.
\item Spiral Galaxies puzzles designed so that all their solutions
  (if any) have rectangular galaxies.  In other words, we have a
  \emph{promise} that all solutions are rectangular.
\end{enumerate}
%
Figure~\ref{fig:sg-ex}(b) shows an example of a Spiral Galaxies puzzle where
the unique solution is rectangular, so it is a valid input to either
Rectangular Galaxies or rectangular-promise Spiral Galaxies.

We prove that both of these puzzle variants are NP-complete and ASP-complete,
and that the corresponding problem of counting the number of solutions is
\#P-complete.
(Refer to~\cite[Chapter 28]{v-aa-10} and~\cite{y-ccfas-03,ys-ccfas-03} for definitions of \#P-completeness and ASP-completeness, respectively.)
In fact, hardness of rectangular-promise Spiral Galaxies implies
hardness of Rectangular Galaxies.
For Rectangular Galaxies,
we further show that the problem remains NP/ASP/\#P-complete
even when all solutions use only $1\times1$, $1\times3$, and $3\times1$
rectangles (a stronger promise),
and hence also if the puzzle is restricted to use only such rectangles.
This special case is in some sense the smallest interesting scenario:
if we restrict to only $1\times1$, $1\times2$, and $2\times1$ rectangles,
then each center determines the shape of its containing polyomino
(of area $1$ or $2$ depending on whether the center is at a cell center
or edge midpoint, respectively), making puzzles easy to solve.
Our proof for $1\times1$, $1\times3$, and $3\times1$ rectangles
also establishes NP-completeness of finding a noncrossing matching in ``distance-2 grid graphs'', whose vertices are a subset of the integer lattice and whose edges connect vertices at Euclidean distance exactly~$2$. %in grid graphs with edge length $2$ 

 
In another variant of the puzzle, a subset of the centers is colored black, and the polyominoes that contain these centers reveals a picture once solved~\cite{gsp-18}; see Figure~\ref{fig:sg-ex}(c). Logic puzzles whose solutions yield pictures are a popular genre; one famous example is the nonogram~\cite{ueda96}. Such Spiral Galaxies puzzles are also the subject of a mathematical fonts~\cite{add-sgf-17}. Constructing an interesting puzzle such that its solution is a given target shape is nontrivial: while a valid puzzle trivially exists, by simply placing a center in every grid cell, the resulting puzzle is clearly not interesting.
We are thus also interested in finding the minimum number of centers such that there exist galaxies that exactly cover a given shape. We prove this puzzle design problem to be NP-complete. %This variant
%Black and white centers are used by Anderson et al.~\cite{add-sgf-17} to create a Spiral Galaxies font.
 



%\paragraph{Roadmap.} %In the remainder of this section we review related work. 


%\subsection{Related work}\label{sec:rw}



\subsection{Notation and Preliminaries}\label{sec:prob}

\section{Notation}
\label{sec:notation}

Let bold functions $\mathbf{f}$ represent the row-wise vectorized versions of their scalar counterparts $f$: $\mathbf{f}(\mathbf{x}) = [f(x_1), f(x_2) \dots f(x_n)]^T$. 

Let $I_a(z)$ be the indicator function for $z=a$.

Let $P(Z^*)$ be a distribution that places probability $\frac{1}{n}$ on each element of  a set of samples $\{z_1, z_2 \dots z_n\}$. Let $z^*$ denote a sample from $Z^*$ and $\bm{z}^*$ denote a set of $n$ such samples (a bootstrap sample).

\subsection{Notation for Observational data}

Observational datasets are represented by a set of $n$ tuples of observations $(x_i,w_i,y_i)$. For each subject $i$, $x_i \in \mathcal{R}^{p}$ is a vector of observed pre-treatment covariates, $w_i \in \{0,1\}^n$ is a binary indicator of treatment status, and $y_i \in \mathcal{R}^n$ is a real-valued outcome. Let capital letters $X$, $W$, and $Y$ represent the corresponding random variables. Let $\mathcal{S}_0$ be the set of indices of untreated subjects, and $\mathcal{S}_1$ the set of indices of treated subjects: $\mathcal{S}_w = \{i | w_i = w\}$.  Denote a series of I.I.D. realizations of a random variable $z \sim P(Z)$ as $\bm{z}$ so that the full observational dataset can be written as $d = (\mathbf{x}, \mathbf{w}, \mathbf{y})$ where $(x_i,w_i,y_i) \overset{\text{I.I.D.}}{\sim} P(X,W,Y)$. 

The average treatment effect is defined as 

\begin{equation}
\tau = E_{X,Y}[Y|X,W=1] - E_{X,Y}[Y|X,W=0]
\label{eq:effect}
\end{equation}

which is the expected difference between what the outcome would have been had a subject received the treatment and what the outcome would have been had a subject not received the treatment, averaged over all subjects in the population. %The outcomes under each of these conditions are referred to as potential outcomes. Let those be denoted as $f_w(x,u) = f(x,u,w) = E[Y_w | X, U]$. %Note that $f_w(\mathbf{x}, \mathbf{u}) + \bm{\eta} \in \mathcal{R}^n$, but $\mathbf{y}_w \in \mathcal{R}^{n_w}$. The latter does not include potential outcomes under treatment $w$ that were not observed.

\subsection{Organization}

The remainder of this article is organized as follows.
In Section~\ref{sec:sgr}, we prove that Spiral Galaxies is NP/ASP/\#P-complete when all solutions use rectangular galaxies.
In Section~\ref{sec:sgr13}, we show that Rectangular Galaxies is NP/ASP/\#P-complete when all solutions use rectangular galaxies of dimensions $1\times1$, $1\times3$, and $3\times1$.
In Section~\ref{sec:match}, we relate this problem to finding noncrossing perfect matchings in distance-2 grid graphs, proving NP/ASP/\#P-completeness of the latter.
In Section~\ref{sec:nrc}, we show that minimizing the number of centers when generating a puzzle with a given output shape is NP-hard.
In Section~\ref{sec:puzz}, we discuss the possibility of allowing multiple solutions in Spiral Galaxies puzzles, and we investigate a small puzzle related to a Spiral Galaxies font.
Finally, we conclude in Section~\ref {sec:conclusion}.

\section{Rectangular-Promise Spiral Galaxies}\label{sec:sgr}
In this section, we show that solving Spiral Galaxies puzzles promised to have solutions with only rectangular galaxies is NP-complete and ASP-complete, and that the corresponding problem of counting the number of solutions is \#P-complete. %, see~\cite{v-aa-10} Chapter 28 and~\cite{y-ccfas-03,ys-ccfas-03} for definitions of \#P-completeness and ASP-completeness, respectively.

% New proof based on https://cocreate.csail.mit.edu/r/S4MNDiTZKoNEjhmd4#R6p5moLX4xwsRoc3t

\iffalse
\begin{theorem}
	Determining whether a Spiral Galaxies puzzle is solvable with only rectangular galaxies is NP-complete.
\end{theorem}
\fi

%\begin{proof}
%The proof is by reduction from \textsc{Planar 1-in-3 SAT}. 
We give a reduction from \textsc{Planar Positive 1-in-3 SAT}. Given an instance $F$ of \textsc{Planar Positive 1-in-3 SAT} with incidence graph $G$, we show how to turn a rectilinear planar embedding of $G$ into a Spiral Galaxies puzzle $P$ such that a solution to $P$ yields a solution to $F$, thereby showing NP-completeness.
Furthermore, there will be a one-to-one correspondence between solutions of $P$ and solutions of $F$, showing \#P-completeness and ASP-completeness. 

\textbf{Overview and filler gadget.}
At a high level, our reduction consists of several gadgets:
``variable'' gadgets representing the variables of~$F$;
``wire'' gadgets to connect variables to clauses;
%``negation'' gadgets to force two variables to have opposite values;
and ``clause'' gadgets to form the clauses of~$F$.
Refer ahead to Figure~\ref{fig:global} for a complete example of the reduction.

 \begin{figure}
\centering
\hspace*{.25\textwidth}
\comic{.1\textwidth}{rect-face-a}{(a)}\hfill
\comic{.1\textwidth}{rect-face-b}{(b)}
\hspace*{.25\textwidth}
  \caption{\small (a) Filler gadget to fill the space in between all other gadgets. (b) Forced solution.}
  \label{fig:rect-face}
\end{figure}

For each region of the board that is not part of these gadgets,
we fill the region with a \defn{filler gadget},
which has a center in every cell of the region.
Figure~\ref{fig:rect-face} shows the filler gadget for a $2 \times 2$ region.
More generally, in every $2 \times 2$ square within the region,
we can argue locally that the four corresponding galaxies
must each consist of a single cell (the one containing the center):
the edges between cells must be galaxy boundaries
to separate the centers into separate galaxies,
and then $180^\circ$ rotational symmetry forces galaxy boundaries
around the $2 \times 2$ square.
As long as the region is the union of such $2 \times 2$ squares,
the filler gadget must consist entirely of single-cell galaxies,
without any interaction with the other gadgets in the construction.
On the other hand, if the region has a width-1 row or a
height-1 column (``thickness~1''), then the galaxy at the center of that cell
might includes cells surrounding the filler gadget.
We must therefore guarantee that every region between other gadgets
is the union of $2 \times 2$ squares (``thickness~2''),
so that the filler gadget has a forced solution of single-cell galaxies.

\begin {figure}
  \centering
  \includegraphics {new-loop}
  \caption 
  {Overall construction of a variable gadget as a sequence of bumps on the top and bottom, where each bump can have a single connection to an incident wire on that side (Figure~\ref{fig:variable-wire}). Bumps without such a connection, such as the one in the bottom right, use a single center.
  }
  \label{fig:variable-loop-fix}
\end {figure}

\textbf{Variable gadget and loop.}
Figure~\ref{fig:variable-loop-fix} shows the overall plan for a variable
gadget: a thickness-$1$ ``variable loop'' that follows a long horizontal
rectangle with regularly spaced bumps on the top and bottom sides,
where each bump has zero or one connection to an incident wire
(which has thickness~$2$).
%We build up this gadget via a sequence of subgadgets.

In more detail, a \defn{variable loop} is a thickness-$1$ loop
built out of the subgadgets in Figures~\ref{fig:variable-straight}
and~\ref{fig:variable-corner}.
Every center is at a cell center, spaced modulo $3$
along the thickness-$1$ loop.
Each center in the middle of a straight piece can choose an $x \times 1$
rectangular galaxy for $x \in \{1,3,5\}$,
which then forces the next galaxy along the straight part to be $6-x \times 1$,
etc., as in Figure~\ref{fig:variable-straight}(b--d).
Each corner subgadget of Figure~\ref{fig:variable-corner}
forbids the center at distance $2$ from the corner
from having a $1 \times 1$ galaxy,
as then the galaxy centered at distance $1$ from the corner
fails to be $180^\circ$ rotationally symmetric;
see Figure~\ref{fig:variable-corner}(d).
Our ``bumpy rectangle'' design from Figure~\ref{fig:variable-loop-fix}
guarantees that every straight portion of a variable loop
is adjacent to at least one corner, and
we can further arrange that all corners have the same alignment.
These properties forbid one global pattern
(subfigure~(d) with blue $5 \times 1$ galaxies)
and leave two possible solution patterns:
all galaxies are $3 \times 1$ or $1 \times 3$ (subfigure~(b)),
and galaxies alternate between blue $1 \times 1$
and red $5 \times 1$/$1 \times 5$ (subfigure~(c)).
These two solutions to the variable gadget correspond to
setting the variable \textsc{false} or \textsc{true}, respectively.

\begin {figure}
  \centering
  \includegraphics {new-straight}
  \caption 
  { (a) A straight piece of a variable gadget.
    (b--c) Two intended valid states.
    (d) Undesired but valid state.
  }
  \label{fig:variable-straight}
%\end {figure}
\medskip
%\begin {figure}
  \centering
  \includegraphics {new-corner}
  \caption 
  { (a) A corner of a variable gadget.
    (b--c) Two valid states.
    (d) The third state is no longer valid.
  }
  \label{fig:variable-corner}
\end {figure}

%\begin{figure}
%\centering
%%\hspace*{.05\textwidth}
%\comic{.75\textwidth}{rect-varloop-a}{(a)}\\
%\comic{.75\textwidth}{rect-varloop-b}{(b)}\\
%%\hspace*{.05\textwidth}\\
%%\hspace*{.05\textwidth}
%\comic{.75\textwidth}{rect-varloop-c}{(c)}
%%\hspace*{.05\textwidth}
%  \caption{\small (a) Variable gadget with two possible solutions (b) and (c) corresponding to a truth assignment of \textsc{true} and \textsc{false}, respectively, of the corresponding variable.}
%  \label{fig:rect-varloop}
%\end{figure}
%

%\begin{lemma}\label{var-2-sols}
%Each variable gadget has exactly two possible solutions.
%\end{lemma}
%\begin{proof}
%By the filler gadgets, the galaxies for centers within the variable gadget
%cannot extend outside the variable gadget.
%We claim that all galaxies within the variable gadget must be rectangles,
%with the following properties:
%%
%\begin{enumerate}
%\item The galaxy from the center in a width-1 corridor has width $1$
%  and height either $2$ or $6$.
%\item The galaxies from the centers in a height-2 corridor
%  all have height $2$, and either all have width $3$,
%  or they alternate between widths $5$ and $1$
%  with the first and last having width~$5$.
%\end{enumerate}
%
%To start, we look at a topmost height-2 corridor,
%which must be connected to downward width-1 corridors,
%forming a ``U-turn''; see Figure~\ref{fig:rec-u-turn}.
%If the leftmost (or rightmost) galaxy in the height-2 corridor
%includes a grid cell from the top row (red) but not the bottom row (blue),
%then this pattern must continue throughout the height-2 corridor (as drawn).
%Then, at the other end, we obtain a galaxy that includes a grid cell from
%the bottom row (red) but not the top row (green), cutting off the top cell
%(green) from being in any galaxy, a contradiction.
%Thus every galaxy from the height-2 corridor must include matching cells
%from both rows.
%By $180^\circ$ rotational symmetry, the leftmost and rightmost galaxies
%also cannot extend beyond the height-2 corridor.
%Thus all galaxies are rectangles.
%The width of the leftmost galaxy rectangle could be $1$, $3$, or $5$
%(any larger would also include the second center).
%By $180^\circ$ rotational symmetry,
%each subsequent galaxy rectangle must have width $5$ minus the previous width.
%
%Moreover, centers are located at edge midpoints. Because galaxies must be $180^\circ$ rotationally symmetric about their center, each galaxy must at least contain the two cells incident to the edge with the center.
%
%Each variable gadget includes a U-turn, that is, height-2 corridors for which both adjacent width-1 corridors are attached to the bottom (or both to the top). These enforce that the galaxies in height-2 corridors must end with a vertical edge of length two; see Figure~\ref{fig:rec-u-turn}. Without loss of generality, we consider a height-2 corridor where both width-1 corridors connect at the bottom. If the leftmost or rightmost galaxy in the height-2 corridor encloses a grid cell in the top height-one row only (red), this enforces the pattern throughout the complete height-2 corridor. Without loss of generality, let this be the leftmost galaxy. Thus, the rightmost galaxy cuts off a set of grid cell(s) in the top row that cannot be caught by any center (green). Consequently, all galaxies in the height-2 corridors must be rectangles. 
%%Because the width-1 corridors have width one only, this is then also enforced for these: t
%
%\begin{figure}
%\centering
%%\hspace*{.25\textwidth}
%\comicII{.5\textwidth}{rect-variable-loop-cont}
%  \caption{\small The topmost height-2 corridor is a U-turn. If only the topmost grid cell of the leftmost column is assigned to the galaxy centered at the leftmost center of the height-2 corridor, then the green grid cell cannot be assigned to any center. (Shaded cells could be assigned to either galaxy.)}
%  \label{fig:rec-u-turn}
%\end{figure}
%
%The galaxies with centers in the width-1 corridors cannot include any grid cells that are not in the same column of cells because both adjacent height-2 corridors can be attached to the same side, thus, extending beyond this column would contradict the galaxies being $180^\circ$ rotationally symmetric about their center. Hence, also all galaxies with centers in the width-1 corridors must be rectangles. If the galaxies with centers in the width-1 corridors extend beyond the two grid cells incident to the edge on which its center is located, they must contain the complete column.
%
%Thus, we have the following properties:
%\begin{itemize}
%\item All galaxies in the variable gadget are rectangles. 
%\item Each galaxy contains the two grid cells incident to the edge on which its center is located.
%\item Each galaxy with a center in a width-1 corridor contains either two or six grid cells. 
%\end{itemize}
%Starting with such a galaxy of two grid cells around a center in a width-1 corridor enforces the galaxy around the first center in each adjacent height-2 corridor to have width 5. This again enforces the next galaxy in the corridor to have width 1, and so on. Analogously, starting with any galaxy around a center in a width-1 corridor with six grid cells enforces the galaxy around the first center in each adjacent height-2 corridor to have width 3. This again enforces the next galaxy in the corridor to have width 3, and so on. Thus, we have exactly two possible solutions for the variable gadgets.
%%
%\end{proof}

%Each variable gadget has two possible solutions, corresponding to setting the variable as \textsc{true} or \textsc{false}, and the rectangle placement is completely determined by the variable assignment. 

%The two possible solutions of the variable gadget correspond to setting the variable as \textsc{true} or \textsc{false},
%and the rectangle placement is completely determined by the variable assignment; see Figure~\ref{fig:rect-varloop}(b) and (c).
%
%From each variable gadget, we can propagate the variable value, creating a \defn{corridor gadget}, shown in
%Figure~\ref{fig:rect-var-corr}. The corridor gadgets are shown in blue and have a center distance of 5 on edge midpoints. %The variable gadgets are shown in black. 
%Again, only two truth assignments are possible, each corresponding to setting the variable to \textsc{true} (Figure~\ref{fig:rect-var-corr}(b)) or \textsc{false} (Figure~\ref{fig:rect-var-corr}(c)). If the corridor does not pick up the correct signal from the variable gadget, then the variable gadget cannot be solved with Rectangular Galaxies; see, e.g., Figure~\ref{fig:rect-var-corr}(d). For the \defn{negation gadget}, we simply connect corridors with centers shifted as shown in Figure~\ref{fig:rect-var-corr}(e). Again, only two truth assignments are possible, each altering the truth assignment from that in the variable gadget Figure~\ref{fig:rect-var-corr}(e)--(i).
%
%\begin{figure}
%\centering
%%\hspace*{.05\textwidth}
%\comic{.286\textwidth}{rect-variable-connection-a}{(a)}\hspace*{.1\textwidth}
%\comic{.286\textwidth}{rect-variable-connection-b}{(b)}\hfill\\
%%\hspace*{.05\textwidth}
%\comic{.286\textwidth}{rect-variable-connection-c}{(c)}\hspace*{.1\textwidth}
%\comic{.286\textwidth}{rect-variable-connection-d}{(d)}\hfill\\
%\comic{.286\textwidth}{rect-variable-connection-i}{(e)}\hspace*{.1\textwidth}
%\comic{.286\textwidth}{rect-variable-connection-e}{(f)}\hfill\\
%\comic{.286\textwidth}{rect-variable-connection-f}{(g)}\hspace*{.1\textwidth}
%\comic{.286\textwidth}{rect-variable-connection-g}{(h)}\hfill\\
%\comic{.286\textwidth}{rect-variable-connection-h}{(i)}\hfill
%%\hspace*{.05\textwidth}
%  \caption{\small (a) Variable gadgets (black centers) and connecting variable corridors (blue centers). (b)/(c) %show the t
%Two possible truth assignments of the connecting corridor depending on the truth assignment in the variable gadget. (d) If the corridor does not pick up the correct signal from the variable gadget the variable gadget cannot be completed. (e)--(i) Negating the truth assignment: (f) and (g) are feasible truth assignments of the connecting corridor depending on the truth assignment in the variable gadget, (h) and (i) show that, if the corridor does not alter the signal from the variable gadget, the variable gadget cannot be completed.}
%  \label{fig:rect-var-corr}
%\end{figure}

\textbf{Wire gadget.}
While thickness $1$ was useful to force the variable gadget to have exactly
two solutions, it seems difficult to copy that truth value onto multiple
thickness-$1$ wires.
We thus introduce thickness-$2$ \defn{wire gadgets}
as shown in Figure~\ref{fig:wire}.
In their simplest form, a wire gadget is a vertical width-$2$ rectangle
with a center at edge midpoints in the even rows modulo~$2$.
The two intended solutions use alternating $2 \times 1$ and $2 \times 3$
galaxies as shown in Figure~\ref{fig:wire}(b--c),
corresponding to \textsc{false} and \textsc{true} signals respectively,

Unfortunately, the basic wire also allows two nonrectangular solutions,
as shown in Figure~\ref{fig:wire}(d--e).
Each of these undesired solutions is prevented by a corresponding
\defn{shift gadget} shown in Figures~\ref{fig:shift-right}
and~\ref{fig:shift-left} respectively.
In either gadget, covering the corner pixels in the middle row
forces one side of the wire to use rectangles.
By guaranteeing that each wire shifts at least once to the right
and at least once to the left (including possibly one shift
canceling out the other), we avoid the undesired solutions,
forcing one of the two intended solutions in Figure~\ref{fig:wire}(b--c).

%Thus we add the U-turn of Figure~\ref{fig:wire-U} to every wire gadget.
%The corner pixels of these gadgets must be captured by the galaxy of
%one of the two nearby centers, which by local case analysis forces
%rectangular galaxies within this gadget, which then propagate into the
%adjacent wire gadgets.
%This addition preserves the two intended rectangular solutions,
%as shown in Figure~\ref{fig:wire-U}(b--c).

\begin {figure}
  \centering
  \includegraphics {new-wire}
  \caption 
  { (a) A wire gadget for connecting a variable to a clause.
    (b--c) Intended valid rectangular states.
    (d--e) Undesired but valid states.
  }
  \label{fig:wire}
\end {figure}
\medskip
\begin {figure}
%  \centering
%  \includegraphics {new-U}
%  \caption 
%  { (a) A U-turn in a wire gadget.
%    (b--c) Two compatible valid states.
%    (d--e) The other two states are no longer valid,
%    failing to cover some corner cells.
%  }
%  \label{fig:wire-U}
%\end {figure}
%\begin {figure}
  \centering
  \includegraphics {new-shift-right}
  \caption 
  { (a) A shift to the right.
    (b---c, e) Three states are still valid.
    (d) The undesired wire state from Figure~\ref{fig:wire}(d) is no longer valid.
  }
  \label{fig:shift-right}
%\end {figure}
\medskip
%\begin {figure}
  \centering
  \includegraphics {new-shift-left}
  \caption 
  { (a) A shift to the left.
    (b---d) Three states are still valid.
    (e) The undesired wire state from Figure~\ref{fig:wire}(e) is no longer valid.
  }
  \label{fig:shift-left}
\end {figure}

Each shift gadget can shift the wire horizontally
by an arbitrary amount $\geq 3$,
as shown in Figure~\ref{fig:shift-var}.
The height-1 transition row makes it easy to argue that the same
solutions in Figures~\ref{fig:shift-right} and~\ref{fig:shift-left}
are forced.
Thus a wire can approximately follow any desired $y$-monotone path
that starts and ends with a vertical segment.
The $x$ coordinates of the start and end vertical segments
can be specified exactly;
the only approximations are that the shifts need $\Omega(1)$ room to navigate,
and we always use at least two shifts.

\begin {figure}
  \centering
  \includegraphics {new-shift-var}
  \caption 
  { Shift gadgets from Figures~\ref{fig:shift-right} and~\ref{fig:shift-left}
    can shift by an arbitrary horizontal amount $\geq 3$.
    (a, c) An even shift places the middle center at a cell center.
    (b) An odd shift places the middle center at an edge midpoint.
  }
  \label{fig:shift-var}
\end {figure}

\textbf{Connecting variables to wires.}
Figure~\ref{fig:variable-wire} shows how to communicate the value of a
variable loop into the signal of a wire.
The vertical thickness-$2$ wire attaches to two consecutive empty cells
of a horizontal segment of the variable loop.
If the wire's galaxies capture those two cells, then the two adjacent
centers of the variable loop must have $1 \times 1$ galaxies,
forcing alternation between $1 \times 1$ and $1 \times 5$ thereafter,
as in Figure~\ref{fig:variable-wire}(c).
Conversely, if the wire's galaxies do not capture those two cells,
then the two adjacent centers of the variable loop must capture them,
forcing the one other solution of the variable loop with
$1 \times 3$ rectangles, as in Figure~\ref{fig:variable-wire}(b).

\begin {figure}
  \centering
  \includegraphics {new-split}
  \caption 
  { (a) A place where a wire from the variable loop is connected.
    (b--c) Two compatible valid states.
  }
  \label{fig:variable-wire}
\end {figure}

Figure~\ref{fig:wires-need-U-turns} shows that the unintended solutions of
wire gadgets from Figure~\ref{fig:wire}(d--e) can propagate into the
variable gadget.  Thus connections to variable gadgets are insufficient to
fix these problems, which is why we needed to introduce the shift gadgets of
Figure~\ref{fig:shift-right} and~\ref{fig:shift-left}
to fix wire gadgets locally.

\begin {figure}
  \centering
  \includegraphics {new-split-problem}
  \caption 
  { Why we needed U-turns in wire gadgets: connections to variables
    are insufficient to force the intended rectangular solutions.
  }
  \label{fig:wires-need-U-turns}
\end {figure}

If we connected multiple wires to the same straight segment
of a variable gadget, then the straight segment between the two connections
would not be incident to any variable corners,
so it could be in the third state of Figure~\ref{fig:variable-straight}(d).
This is why we limit each bump of the variable loop to
having at most one connection to a wire,
so that every connection gadget from Figure~\ref{fig:variable-wire}
is surrounded on either side
by a corner gadget from Figure~\ref{fig:variable-corner}.
%This problem cannot arise if we guarantee (as mentioned above)
%that every straight segment of the variable gadget is incident to
%at least one corner from Figure~\ref{fig:variable-corner}.
%In particular, we can add a corner from Figure~\ref{fig:variable-corner}
%before and after every connection gadget from Figure~\ref{fig:variable-wire}.

Each connection between variable and wire
removes one alternation from the variable gadget,
as indicated by the coloring in Figure~\ref{fig:variable-wire}.
If the total number of connections from a variable gadget is odd,
then we would have a parity mismatch around the variable loop.
Figure~\ref{fig:variable-loop-fix} illustrates a parity fix
by placing a single center in the middle of a bump of the variable gadget
that has no connection to a wire (similar to Figure~\ref{fig:shift-var}).
For simplicity, we can use this construction for every bump
with no connection to a wire.
Because the two adjacent corners turn in the same direction,
this center can be covered in exactly two ways:
with a rectangle that is either the full horizontal width
or $2$ smaller.
% Another fix: \url{https://cocreate.csail.mit.edu/r/SdhgBEBQhzHgTWJkK}


%We show the \defn{corridor bend} gadget in Figure~\ref{fig:rect-corridor}. Moreover, we use a \defn{corridor shift} gadget, shown in Figure~\ref{fig:rect-shift5s}, which allows us to shift the location of the corridor by any number $\geq 3$. These are used to connect to the clause gadgets in the appropriate places.
%
%\begin{figure}
%\centering
%%\hspace*{.05\textwidth}
%\comic{.3\textwidth}{rect-corridor-a}{(a)}\hfill
%\comic{.3\textwidth}{rect-corridor-b}{(b)}\hfill
%%\hspace*{.05\textwidth}\\
%%\hspace*{.05\textwidth}
%\comic{.3\textwidth}{rect-corridor-c}{(c)}
%%\hspace*{.05\textwidth}
%  \caption{\small (a) Variable corridors with bend. (b)/(c) %show the 
%Two possible covers with rectangular galaxies.  }
%  \label{fig:rect-corridor}
%\end{figure}
%
%
%
%
%\begin{figure}
%\centering
%%\hspace*{.05\textwidth}
%\comic{.286\textwidth}{rect-shifts5s-a}{(a)}\hfill
%\comic{.286\textwidth}{rect-shifts5s-b}{(b)}\hfill
%%\hspace*{.05\textwidth}\\
%%\hspace*{.05\textwidth}
%\comic{.286\textwidth}{rect-shifts5s-c}{(c)}
%%\hspace*{.05\textwidth}
%  \caption{\small (a) Variable corridors can be shifted by any number $\geq 3$ (here shown for a shift of 3 as indicated in green). (b) and (c) depict the different variable assignments.  }
%  \label{fig:rect-shift5s}
%\end{figure}

\begin {figure}
  \centering
  \includegraphics {new-clause}
  \caption 
  { The clause gadget connects three wire gadgets (green).
    (b--d) Valid solutions with exactly one true signal.
    (e) Impossibility with zero true signals.
    (f--h) Impossibility with two true signals.
    (i) Impossibility with three true signals.
  }
  \label{fig:clause}
\end {figure}

\textbf{Clause gadget.}
The last gadget is the clause gadget shown in Figure~\ref{fig:clause},
which implements a 1-in-3 constraint between three incident wire gadgets
via a $1 \times 22$ rectangle.
Although we draw all three wire gadgets as below the clause gadget,
each wire could in fact be either above or below the clause gadget.
Figure~\ref{fig:clause}(b--d) illustrates that the clause gadget has a
(unique) solution when exactly one of the wires carries a true signal.
Indeed, any true signal represented by a wire with a galaxy covering
two cells of the clause gadget locally forces the galaxies of the
one or two adjacent centers within the clause, which then propagates to force
the galaxies of the other centers within the clause.
This forced solution turns out to be inconsistent with
any of the other wires carrying a true signal, causing overlapping galaxies
or uncovered cells as shown in Figure~\ref{fig:clause}(e--i).
Therefore the clause gadget has a (unique) solution if and only if
exactly one of the incident wires carries a true signal.

%To combine the corridor gadgets into clauses, we use the \defn{clause gadget} shown in Figure~\ref{fig:rect-clause}. At least one of the three variables' truth assignments is forced to be true (see Figure~\ref{fig:rect-clause}(b)--(d)), more than one (see Figure~\ref{fig:rect-clause}(f)--(h)) or none (see Figure~\ref{fig:rect-clause}(e)) cannot be set to true, hence, it forces exactly one true assignment, giving a solution to the instance $F$.
%
%\begin{figure}
%\centering
%%\hspace*{.05\textwidth}
%\comic{.43\textwidth}{rect-clause-n-a}{(a)}\hfill
%\comic{.43\textwidth}{rect-clause-n-b}{(b)}\\%\vspace{.2cm}\\
%%\hspace*{.05\textwidth}\\
%%\hspace*{.05\textwidth}
%\comic{.43\textwidth}{rect-clause-n-c}{(c)}\hfill
%\comic{.43\textwidth}{rect-clause-n-d}{(d)}\\
%\comic{.43\textwidth}{rect-clause-n-e}{(e)}\hfill
%\comic{.43\textwidth}{rect-clause-n-f}{(f)}\\
%\comic{.43\textwidth}{rect-clause-n-g}{(g)}\hfill
%\comic{.43\textwidth}{rect-clause-n-h}{(h)}
%%\hspace*{.05\textwidth}
%  \caption{\small (a) Clause gadget shown in black and the connecting variable corridors shown in blue. (b)--(d) The clause gadget can be filled in with rectangle galaxies iff exactly one of the three connecting variables has a truth assignment that fulfills the clause. %These three cases are shown in (b), (c) and (d). 
%%(e) If all variables do not fulfill the clause, the clause cannot be completed. (f)--(h) If more than one variable fulfills the clause, the clause cannot be completed.
%The clause cannot be completed if all variables do not fulfill the clause (e), or if more than one variable fulfills the clause (f)--(h).}
%  \label{fig:rect-clause}
%\end{figure}

\textbf{Putting pieces together.}
For the global construction, we construct an orthogonal planar embedding of $G$
where each vertex is represented by a horizontal line segment
with endpoints on the grid,
and each edge is represented by a polygonal line on the grid
that is $y$-monotone and starts and ends with a vertical segment.
Such an embedding can be obtained by
constructing a planar straight-line embedding of $G$
with vertices on an $O(n) \times O(n)$ grid \cite{schnyder90};
applying the transformation of \cite[Theorem~5]{biedl14}
into a $y$-monotone orthogonal drawing where vertices are horizontal segments;
and then staggering vertices with equal $y$ coordinates.
The resulting grid size is $O(n) \times O(n^2)$.

Figure~\ref{fig:global}(a) shows a particularly structured planar embedding
of $G$
%an instance of \textsc{Planar Positive 1-in-3 SAT},
where the variables all lie on a horizontal line,
every clause is either above or below this line,
and variable--clause connections do not cross the line or each other.
Then it is easy to draw each variable--clause connection with a $y$-monotone
path that starts and ends with a vertical segment.
Mulzer and Rote \cite{mulzer08} proved that \textsc{Planar Positive 1-in-3 SAT}
is NP-complete in this form.  Their reduction has also been observed to be
the end of a chain of parsimonious reductions from 3SAT
\cite{6.890-planar-1-in-3},
so it establishes ASP- and \#P-completeness too.
We describe the more general reduction in the previous paragraph
so as to not rely on this unpublished observation,
but use the simpler form for figures.

\begin {figure}
  \centering
  \includegraphics[width=\linewidth]{new-global-both}
  \caption 
  { (a) An embedded instance of Planar Positive 1-in-3 SAT, with all variables on a common horizontal line and all clauses either above or below the line.
    (b) The corresponding gadgets of Spiral Galaxies, excluding filler.
    (c) Full Spiral Galaxies puzzle with filler gadgets.
  }
  \label{fig:global}
\end {figure}

As illustrated in Figure~\ref{fig:global}(b),
we scale up the orthogonal planar embedding by a constant factor;
replace each variable or clause vertex with a variable gadget or clause gadget
respectively,
and replace each $y$-monotone edge with a wire gadget
(including at least one right shift and at least one left shift).
Finally, we place the filler gadget in the regions between other gadgets,
as shown in Figure~\ref{fig:global}(c).
Because each gadget has a unique solution corresponding to a Boolean
assignment, this reduction is parsimonious.

Finally we analyze alignment issues in this construction.
The shift gadgets of Figure~\ref{fig:shift-var} provide exact
horizontal positioning, so there are no constraints on horizontal
positioning of gadgets.
However, vertical positioning controls the parity of the \textsc{false}
and \textsc{true} values of a wire, as in the yellow and green coloring
of Figure~\ref{fig:wire}.
We place all variable loops to have the same vertical positioning modulo~$4$.
Thus we obtain consistent heights modulo $4$
for the row of a wire between a \textsc{false} (yellow) and \textsc{true}
(yellow) center that needs to be aligned with a clause:
for wires extending upward, this is the row above a \textsc{true} center, while
for wires extending downward, this is the row below a \textsc{true} center.
Observe in Figure~\ref{fig:variable-loop-fix} that these rows have the
same height modulo $4$ within a single variable loop,
and that wire gadgets of Figure~\ref{fig:wire} and shift gadgets of
Figures~\ref{fig:shift-right}--\ref{fig:shift-var}
preserve this height modulo~$4$.
We can then place all clause gadgets at such $y$ coordinates.

A solution to an $m\times n$ Spiral Galaxies puzzle can be verified in polynomial time. We conclude: 

\begin{theorem}
	Solving a Spiral Galaxies puzzle whose solutions have only rectangular galaxies is NP-complete and ASP-complete, and counting the number of solutions is \#P-complete.
\end{theorem}
%\end{proof}


\section{Rectangular Galaxies with $1\times1$, $1\times3$, and $3\times1$ Rectangles}\label{sec:sgr13}
In this section, we show that solving %Spiral Galaxies 
Rectangular Galaxies puzzles is NP-complete and ASP-complete, even with the promise that every solution uses only $1\times 1$, $1\times 3$, and $3\times 1$ galaxies, and that the corresponding problem of counting the number of solutions is \#P-complete. %, see~\cite{v-aa-10} Chapter 28 and~\cite{y-ccfas-03,ys-ccfas-03} for definitions of \#P-completeness and ASP-completeness, respectively.
%For our reduction, we again use the \textsc{Planar 1-in-3 SAT} problem.
\iffalse
\begin{theorem}\label{th:sgr13}
	Determining whether a Spiral Galaxies puzzle is solvable with only  $1\times 1$, $1\times 3$, and $3\times 1$ galaxies is NP-complete and ASP-complete, and counting the number of solutions is \#P-complete.
\end{theorem}
\fi
%\begin{proof}
	%The proof is by reduction from \textsc{Planar 1-in-3 SAT}. 
Again we give a reduction from \textsc{Planar Positive 1-in-3 SAT}.
Given an instance $F$ of \textsc{Planar Positive 1-in-3 SAT} with incidence graph $G$, we show how to turn a rectilinear planar embedding of $G$ into a Rectangular Galaxies puzzle $P$ such that a solution to $P$ yields a solution to $F$, thereby showing NP-completeness. Furthermore, there will be a one-to-one correspondence between solutions of $P$ and solutions of $F$, showing \#P-completeness and ASP-completeness. 

In this section, all galaxies will be $1\times 1$, $1\times 3$, or $3\times 1$.
To ease the description, our figures draw these galaxies as edges of length 2 ($1\times 3$ and $3\times 1$) or nonexisting edges ($1\times 1$).
%Hence, in this proof, squares in the figures do not indicate centers for galaxies. %, but the centers will in fact always be placed between two disks.
Instead of drawing the galaxy centers (dots), we draw small boxes (squares) that reflect the endpoints of these edges.
Boxes at distance 2 can be connected by a rectilinear edge, and centers will be located at the middle of each potential edge; see Figure~\ref{fig:3-1-disks-and-centers}.
Hence, any $1\times 3$ or $3\times 1$ galaxy will cover both boxes, denoted by an edge between these two boxes; any $1\times 1$ galaxy will not extend over the boxes, and is shown by a nonexisting edge between the boxes.

\begin{figure}
\centering
\comicII{.226\textwidth}{3-1-disks-centers}
  \caption{\small We place centers (dots) on the middle point of potential edges (light gray) between any pair of boxes (squares). In the remainder of this section, we show only the boxes, not the centers (dots). }
  \label{fig:3-1-disks-and-centers}
\end{figure}

%\todo{explain what the vertex representation means}

\textbf{Overview and filler gadget.}
At a high level, our reduction consists of two main gadgets:
``variable'' gadgets representing the variables of~$F$;
%``negation'' gadgets to force two variables to have opposite values;
and ``clause'' gadgets to form the clauses of~$F$.
%Throughout the discussion, the constructed gadgets have a single solution for any given variable assignment, which will make this reduction parsimonious.

As in Section~\ref{sec:sgr}, we fill any region not covered by these gadgets
with a \defn{filler gadget}, which places a center at every cell of the region.
Figure~\ref{fig:rect-face} shows the filler gadget for a $2 \times 2$ region.
The centers of a filler gadget must be contained in $1 \times 1$ galaxies,
provided every uncovered region is the union of $2 \times 2$ squares.

%The \defn{variable gadget}, shown in Figure~\ref{fig:3-1-var}, has two possible solutions (shown in Figure~\ref{fig:3-1-var}(b) and (c)), each corresponding to one truth assignment for the variable (\textsc{true} and \textsc{false}). We extend the size of the variable gadget to connect to other gadgets---to build \defn{corridors}.
\textbf{Variable gadget.}
Figure~\ref{fig:3-1-var} illustrates the \defn{variable gadget},
which is an alternating pattern of boxes.
While the figure shows the gadget in a simple rectangular form,
variables can also turn arbitrarily, as long as each box has
distance $2$ to exactly two other boxes (i.e., the variable does not
get too close to itself).

%\todo{CS: added lemma and proof here}
\begin{lemma}\label{var-1-3-only-2-sols}
Each variable gadget has exactly two possible solutions (shown in Figure~\ref{fig:3-1-var}(b) and (c)).
\end{lemma}
\begin{proof}
%\begin{proofof}{Lemma~\ref{var-1-3-only-2-sols}}
Because of the filler gadgets, no galaxy centered at a center
of the variable gadget can extend beyond the grid cells
covered by the potential edges of the variable gadget.
Hence, every rectangular galaxy must have a width or height of 1,
which we label as ``vertical'' or ``horizontal'' respectively.
Because the centers are placed with distance 2 and galaxies must be
$180^\circ$ rotationally symmetric about their center,
horizontal and vertical galaxies have a width and height respectively
in $\{1,3\}$.
Because galaxies may not overlap (we aim to decompose the grid),
$1\times 3$/$3\times 1$ galaxies must alternate with $1 \times 1$ galaxies.
There exist exactly two possible solutions that fulfill these conditions.%\hspace*{2.2cm}$\square$
\end{proof}
%\end{proofof}%$\square$%\qed

Each of the two possible solutions corresponds to one truth assignment
for the variable, \textsc{true} and \textsc{false}.
We route each variable gadget to interact with every clause that contains
the variable.  Equivalently, we can imagine each variable gadget as having
a ``tentacle'' to visit every incident clause.


\begin{figure}
\centering
\hspace*{.1\textwidth}
\comic{.226\textwidth}{3-1-var}{(a)}\hfill
\comic{.226\textwidth}{3-1-var-a}{(b)}\hfill
\comic{.226\textwidth}{3-1-var-b}{(c)}
\hspace*{.1\textwidth}
  \caption{\small (a) Variable gadget with two possible states (b) and (c) corresponding to a truth assignment of \textsc{true} and \textsc{false} of the corresponding variable. }
  \label{fig:3-1-var}
\end{figure}

%%%%%%%%%%%%%%
% NEGATION TAKEN OUT
%%%%%%%%%%%%%%
\iffalse
Negating a variable corresponds to inserting a \defn{negation gadget} into the corridor, and to continue with another variable corridor as in Figure~\ref{fig:3-1-neg}.


\begin{figure}
\centering
\hspace*{.1\textwidth}
\comic{.35\textwidth}{3-1-negation-a}{(a)}\hfill
\comic{.35\textwidth}{3-1-negation-b}{(b)}
\hspace*{.1\textwidth}
  \caption{\small Negation gadget in gray, with two black variable gadgets. The incoming variable gadget on the left, has a different truth assignment than the outgoing variable gadget on the right, two cases shown in (a)/(b).
%(a) and (b) show this for the two possible assignments of the incoming variable gadget.
}
  \label{fig:3-1-neg}
\end{figure}
\fi
%%%%%%%%%%%%%%
% END NEGATION TAKEN OUT
%%%%%%%%%%%%%%

\begin{figure}
\centering
\comic{.638\textwidth}{3-1-clause-a}{(a)}\hfill
\comic{.638\textwidth}{3-1-clause-b-2-n}{(b)}\\
\comic{.638\textwidth}{3-1-clause-c-n}{(c)}\hfill
\comic{.638\textwidth}{3-1-clause-d-2-n}{(d)}\\
\comic{.638\textwidth}{3-1-clause-e-n}{(e)}\vspace*{-.2cm}
  \caption{\small (a) Clause gadget in gray, with the three incoming variable gadgets in black.  (b) The three possible states of the clause gadget in red, green, and blue. (c)--(e) Each state of the clause gadget with the corresponding assignments of the variable gadgets; the (only) true variable is shown in the same color as the clause edges.}
  \label{fig:3-1-clause}
\end{figure}

\textbf{Clause gadget.}
Figure~\ref{fig:3-1-clause} shows the clause gadget,
where the clause gadget itself is drawn in gray,
and the three incident variable gadgets are drawn in black.
There are three possible states of the clause gadget,
drawn in red, green, and blue in Figure~\ref{fig:3-1-clause}(b).
Each state forces exactly one of the variables' truth assignments
to be \textsc{true} and all the other variables to be \textsc{false},
as shown in Figure~\ref{fig:3-1-clause}(c)--(e).
%giving a solution to the instance $F$.

\textbf{Putting pieces together.}
For the global construction, we start from an planar embedding of $G$
with edges routed orthogonally on an $O(n) \times O(n)$ grid
\cite{biedl98}, scaled up by a constant factor.
Then we locally replace each clause by a single clause gadget,
and replace each variable by a sufficiently large variable gadget
that extends (via ``tentacles'') to all clauses in which it appears.
%We must ensure that all our gadgets can be connected with each other.
All variable gadgets use one parity
(e.g., all boxes have even $x$ and even $y$ coordinates),
while all clause gadgets 
use the other parity (e.g., all boxes have odd $x$ and odd $y$ coordinates).
Hence, we can always place all gadgets and connect them together as desired.
Because each gadget has a unique solution corresponding to a Boolean
assignment, this reduction is parsimonious.

A solution to an $m\times n$ Rectangular Galaxies puzzle can %obviously 
be verified in polynomial time. We conclude:
%\end{proof}

\begin{theorem}\label{th:sgr13}
	Solving a Rectangular Galaxies puzzle whose solutions have only $1\times 1$, $1\times 3$, and $3\times 1$ galaxies is NP-complete and ASP-complete, and counting the number of solutions is \#P-complete.
\end{theorem}


\section{Noncrossing Matching in Distance-2 Grid Graphs}\label{sec:match}
In this section, we prove an equivalence between Spiral Galaxies with
$1 \times 1$, $1 \times 3$, and $3 \times 1$ galaxies and
``noncrossing perfect matchings'' in ``distance-2 grid graphs''.
First we define the quoted terms.
A \defn{distance-2 grid graph} $G = (V, E)$ has vertices $V \subset \Z \times \Z$ at a finite subset of the integer lattice, and edges $E = \{(v,w) \mid v, w \in V, \|v-w\| = 2\}$ between all pairs of vertices at Euclidean distance exactly $2$.
A \defn{noncrossing perfect matching} in a distance-2 grid graph $G = (V, E)$
is a partition of the vertices $V$ into $|V|/2$ pairs
such that the length-$2$ segments connecting paired vertices
do not intersect (including at endpoints).

Next we observe that distance-2 grid graphs can always be decomposed into two independent components.
Call a vertex $v = (x,y) \in V$ \defn{even} if $x+y$ is even,
and \defn{odd} if $x+y$ is odd.
Because edges connect vertices of distance exactly $2$,
which is even, vertices of different parity cannot be connected together.
Thus even/odd forms a partition of any distance-2 grid graph.
In the remainder, we restrict to \defn{even} distance-2 grid graphs
whose vertices are all even.

Call a Spiral Galaxies puzzle \defn{even} if all centers are at cell centers and all \emph{empty} cells are at even positions.
We claim that there is a one-to-one correspondence between noncrossing perfect matchings in even distance-2 grid graphs and solutions to even Spiral Galaxies puzzles restricted to $1\times 1$, $1\times 3$, and $3\times 1$ galaxies:

\begin {lemma} \label {lem:equiv}
  Given an even distance-2 grid graph, we can efficiently construct an even Spiral Galaxies puzzle restricted to $1\times 1$, $1\times 3$, and $3\times 1$ galaxies and correspond one-to-one with noncrossing perfect matchings in the distance-2 grid graph. Vice versa, we can reduce from a restricted even Spiral Galaxies puzzle to noncrossing perfect matching in an even distance-2 grid graph.
\end {lemma}

\begin {proof}
  Given an even distance-2 grid graph $G = (V, E)$,
  we construct an even Spiral Galaxies puzzle as follows.
  The board is the bounding box of the vertices $V$ enlarged by $1\over2$ on all sides, aligned so that every vertex of $V$ is placed at the center of a board cell.
  %We assume for ease of exposition that the board has its bottom-left corner at the origin.
  For every \emph{even} cell $(x, y)$ of the board, we place a galaxy center at its center if and only if $(x, y) \notin V$; that is, the vertices of $G$ correspond to empty cells of the board.
  For every \emph{odd} cell $(x, y)$ of the board, we always place a galaxy center at its center.
  
  Similarly, from a given even Spiral Galaxies puzzle, we can construct an even distance-2 grid graph by creating a vertex for every empty board cell. (The edges are implied by the vertices by the definition of distance-2 grid graph.)
  
  We claim that noncrossing perfect matchings of $G$ correspond one-to-one with solutions to the Spiral Galaxies instance restricted to only $1\times1$, $1\times3$, and $3\times1$ galaxies.
  Indeed, all empty cells in the puzzle must be captured by a galaxy, and can be captured only in pairs on opposite sides of a galaxy center (in an odd cell), which in $G$ corresponds to an edge connecting two vertices.
  %\hspace*{2.2cm}$\square$
\end {proof}


Combining Lemma~\ref {lem:equiv} and Theorem~\ref{th:sgr13}
(or equivalently, interpreting the boxes in figures as graph vertices),
we obtain hardness for noncrossing perfect matching:
\begin{corollary}
Noncrossing perfect matching in distance-2 grid graphs
% where edges connect vertices of distance 2 %grid graphs with edge length 2 
is NP-complete and ASP-complete, and counting the number of solutions is \#P-complete.
\end{corollary}




\section{Minimizing Centers in Spiral Galaxies for a Given Shape}\label{sec:nrc}
In this section, we are given a shape $\mathcal{S}$ on a Spiral Galaxies board, and we aim to find the minimum number of centers such that there exist galaxies with these centers that exactly cover the given shape $\mathcal{S}$. We show that this problem is NP-complete by another reduction from \textsc{Planar Positive 1-in-3 SAT}.

\begin{theorem}\label{th:nrc}
	It is NP-complete to minimize the number of centers on a Spiral Galaxies board such that galaxies with these centers exactly cover a given shape~$\mathcal{S}$.
\end{theorem}

%\todo{expand proof?}
%\begin{proof}
%
For our reduction, the input is a \textsc{Planar Positive 1-in-3 SAT} formula, and the output is a ``picture'' --- a subset of cells that should be ``colored black''. In the remainder of the section, we focus on the cells that are part of the picture; we will say that the cells which are not part of the picture are \defn{forbidden}.

First we will define some \defn{low-level gadgets}, which each have a constant size.
Then we will build \defn{high-level gadgets}, which consist of multiple low-level gadgets, and which will be used to encode the 1-in-3-SAT formula.

\subsection {Low-level gadgets}

\textbf{Local center gadgets.}
First we introduce \defn{local center gadgets}: thin constructions with unique shapes, which ensure that there must be at least one center in each of them; see Figure~\ref {fig:center-gadget}. 
The idea will now be to construct a shape that can be covered with exactly this set of centers if and only if the 1-in-3-SAT instance is satisfiable. 

\begin{figure}
\centering
\includegraphics [scale=0.75]{center-gadget}
  \caption{\small The local center gadget must have at least one center. (a)--(d) Different shapes ensure we cannot include them in larger galaxies.}
  \label{fig:center-gadget}
\end{figure}

Specifically, each local center gadget consists of a single cell with a \defn{desired galaxy center}. Its four direct neighbors are all empty, and its four diagonal neighbors are all forbidden. The cells left and right of the center will serve as connections to other low-level gadgets (we describe the gadgets as if they are vertically oriented; in the actual construction they may be rotated by $90^\circ$). The cells above and below the center are extended by symmetrically shaped paths of empty cells which are surrounded by forbidden cells.

We can immediately observe that, if there is \emph{not} a center at the location of the desired galaxy center in a local center gadget, then either the entire gadget will be part of a larger galaxy with a center outside the gadget, or there must be at least two centers placed inside the gadget. In the first case, because of the symmetry of the galaxies, there must be a second copy of the local center gadget that is also part of the same galaxy. For this reason, we use not one but several different versions of the gadget.
As we will see later, $O(1)$ different version will be sufficient.



 \begin{figure}
\centering
\includegraphics [scale=0.75]{block-gadget}
  \caption{\small (a) The block gadget. (b) A straight block. (c) A corner block. (d) A clause block.}
  \label{fig:block-gadget}
\end{figure}

\textbf{Block gadgets.}
The second low-level gadget we use is the \defn{block gadget}.
A block gadget is simply a $5\times5$ room surrounded by forbidden cells, except for possibly in the centers of its four sides. 
The idea is that a block gadget will not contain galaxy centers in an optimal solution, but rather will be connected to local center gadgets and be part of their galaxies. 
Depending on which of the sides of a block gadget are connected to other gadgets, we distinguish \defn{straight blocks}, \defn{corner blocks}, and \defn{clause blocks}; see Figure~\ref {fig:block-gadget}.

\begin {observation} \label {obs:desire}
  Suppose we place local center gadgets and block gadgets in such a way that, for each block gadget $B$, all local center gadgets that $B$ is connected to have different shapes.
  Then we must place at least one galaxy center inside each local center gadget.
\end {observation}

\begin{figure}
\centering
\includegraphics [scale=0.75]{end-gadget}
  \caption{\small An end gadget has one center, whether or not a block is used by the galaxy next to it.}
  \label{fig:end-gadget}
\end{figure}

\textbf{End gadgets.}
Finally, our last low-level gadget is the \defn{end gadget}. 
An end gadget is similar to a bock gadget, but it is a $7\times5$ room and it has only one connection, at the center of one of its sides of length $5$; see Figure~\ref {fig:end-gadget}.

The idea is that, if we place an end gadget at the end of a chain that satisfies the conditions of Observation~\ref {obs:desire}, then each end gadget will require at least one galaxy center to be placed inside it, because the center in the local center gadget next to it has only a $5 \times 5$ room on the other side (unless two end gadgets are directly connected, which is not useful). Depending on where we place it, more or less of the room is available to be included in neighboring galaxies.

\begin {observation} \label {obs:end}
  If an end gadget is connected to a local center gadget, and on the other side of the local center gadget is not another end gadget, then we must place at least one galaxy center inside the end gadget.
\end {observation}

\subsection {High-level gadgets}

With these low-level gadgets in place, we now turn to constructing high-level gadgets that will encode the variables and clauses of our 1-in-3-SAT instance.

\textbf{Fix gadgets.}
First, we define the \defn{fix gadget}: an alternating sequence of four local center gadgets and three block gadgets making a U-turn as in Figure~\ref {fig:fix-gadget}.
The purpose of this gadget is to ensure that, in an optimal solution, each $5\times5$ block will be completely included in a single galaxy, and not split among several galaxies.

\begin {lemma} \label {lem:fix}
In any chain of blocks that contains a fix gadget, an optimal solution requires that each block must completely belong to a single galaxy.
\end {lemma}
\begin {proof}
First, by Observation~\ref {obs:desire}, an optimal solution has exactly one center in each local center gadgets, and none in the block gadgets.
This implies that if {\em any} block gadget in the chain does not completely belong to a single galaxy, then, by symmetry, {\em every} block gadget in the chain does not completely belong to a single galaxy.

Now, suppose we have a fix gadget consisting of blocks $A$, $B$, and $C$, where $A$ and $C$ are corner blocks but $B$ is a straight block, and suppose $A$ is split among multiple galaxies, say a red one at the top and a blue one at the right; refer to Figure~\ref {fig:fix-gadget}(b). Then, by symmetry of the blue galaxy, the center left pixel of $B$ must also be blue and the bottom center pixel of $B$ must also be red. Then, the top center pixel of $C$ must be red (a priori, this could be the same or a different red galaxy --- by connectivity it must be a new galaxy) and the center right pixel of $C$ must be blue (again, either the same or a different blue galaxy) by symmetry of the (second) red galaxy. But when both the center left and top center pixels of $C$ are red then $C$ must be completely red, a contradiction.
  %\hspace*{2.2cm}$\square$
\end {proof}

 \begin{figure}
\centering
\includegraphics [scale=0.75]{block-argument}
  \caption{\small (a) The fix gadget. (b) If the left block is split among multiple galaxies: contradiction.}
  \label{fig:fix-gadget}
\end{figure}


 \begin{figure}
\centering
\includegraphics [scale=0.75]{variable-chain}
  \caption{\small A variable chain and its two possible truth assignments.}
  \label{fig:variable-chain}
\end{figure}

\textbf{Variable gadgets.}
A \defn{variable chain} consists of alternating local center gadgets and block gadgets starting and ending at end gadgets; see Figure~\ref{fig:variable-chain} for a variable chain.
Each variable chain must include at least one fix gadget.
Then, by Lemma~\ref {lem:fix}, each block gadget on a variable chain must belong to exactly one of its two neighboring local center gadgets; this choice propagates throughout the entire chain and encodes the value \textsc{true} or \textsc{false} of one of the variables in our 1-in-3-SAT instance.
Specifically, we show:

\begin {lemma} \label {lem:variable}
Consider a variable chain which
\begin {itemize}
  \item has $k>1$ local center gadgets and $k-1$ block gadgets;
  \item contains at least one fix gadget; and
  \item has the property that every three consecutive local center gadgets have three different shapes.
\end {itemize}
Then there exists no set of $k+1$ or fewer galaxy centers that form a valid puzzle, and there are exactly two sets of $k+2$ galaxy centers that form a valid puzzle, both of which include all $k$ \emph{desired} galaxy centers.
\end {lemma}
\begin {proof}
By Observation~\ref {obs:end} and because $k > 1$, we must place one galaxy center inside each end gadget.
By Observation~\ref {obs:desire} and because we vary the shapes of the local center gadgets, we must place at least one galaxy center at each of the $k$ local center gadgets.
Indeed, a galaxy centered in a local center gadget cannot include both adjacent local center gadgets, since they have different shapes, and a galaxy centered outside a local center gadget cannot include both incident local center gadgets, since they have different shapes.
So, we need at least $k+2$ centers.
This is also sufficient, as shown by the two solutions with $k+2$ centers in Figure~\ref {fig:variable-chain}.
Finally, by Lemma~\ref {lem:fix}, we cannot have more than two possible shapes for each galaxy.
\end {proof}

Note that within one variable chain, we can realize the conditions of Lemma~\ref {lem:variable} with only three different local center gadget shapes, which we alternate cyclicly.

 \begin{figure}
\centering
\includegraphics [scale=0.75]{split-gadget}
  \caption{\small The split gadget is essentially three end gadgets, but can only be solved with a single center if either all three blocks are used or all three are not used.}
  \label{fig:split-gadget}
\end{figure}

\textbf{Split gadgets.}
In order to connect variables to multiple clauses, we need to be able to split them.
Rather than splitting chains in the usual sense, our \defn{split gadget} takes three independent variable chains and forces them to have the same state by combining three end gadgets into a single larger end gadget; see Figure~\ref {fig:split-gadget}. This effectively creates a ``split'' chain with three remaining ends that have the same state; by repeating this we may create as many copies of a variable as required.

The idea behind the split gadget is simple:
by combining three end gadgets into one large $7\times15$ room, we no longer know that we need three separate galaxy centers; in fact, applying Observation~\ref {obs:end} to the three adjacent local center gadgets, we potentially only require to place one galaxy center instead of three.
Observe that this is indeed achievable, but \emph{only} when all three chains are in the same state, because otherwise the remaining shape of the room is not symmetric by $180^\circ$ rotation (and thus cannot be covered by a single additional galaxy).

\begin{observation}
  A split gadget requires only one galaxy center if and only if either all three adjacent local center gadgets do use a $5\times5$ area of its room, or if none of them do.
\end{observation}

\textbf{Clause gadgets.}
A clause of our 1-in-3-SAT formula is now simply encoded by a single clause block where three variable chains meet: it can be solved without using an additional center if and only if precisely one of the three chains needs to use the block.

In order to avoid that galaxies centered on clause blocks include local center gadgets, the three local center gadgets directly incident to the clause block must have distinct shapes.
To achieve this, we may start with an edge coloring of the variable--clause incidence graph and use three unique shapes for each color. Every planar graph of maximal degree $3$ admits an edge coloring with 4 colors, which results in 12 distinct shapes for our local center gadgets.
It is possible, but not necessary, to reduce the number of shapes further.

\subsection{Global construction}

With our high-level gadgets now also in place, we can describe the global construction of our reduction.
%
We start from an planar embedding of $G$
with edges routed orthogonally on an $O(n) \times O(n)$ grid
\cite{biedl98}, scaled up by a constant factor.
We locally replace each clause by a single clause block, each variable by a sufficient number of split gadgets connected by variable chains, and each edge by variable chains connecting variables to clauses.

%Note that in this reduction we do not need negation gadgets: by the symmetric nature of galaxies, each chain already contains an alternating sequence of ``positive'' and ``negative'' instances of its variable.

\textbf{Distance considerations.}
We must make sure that all our gadgets are aligned with the grid, and can be connected to each other. Furthermore, we need to ensure that the number of block gadgets in each variable chain
%is odd or even depending on the required state of the variable in the respective clause.
has the same parity so that all clauses use positive forms of the variable.

We observe that we can achieve all of these requirements by simply adjusting the distance between adjacent block gadgets and allowing a slightly richer class of local center gadgets: instead of a width of $3$ we may give them a width of any desired integer $w$. If $w$ is even, we must place a galaxy center on an edge and not in the center of a cell, and we correspondingly also adjust the vertical arms of the gadget by giving them a width of $2$ rather than $1$.

This concludes the proof of Theorem~\ref{th:nrc}.
%
%\end{proof}












\section{Multiple Solutions and the Font Puzzle}\label{sec:puzz}
\begin{figure}
\centering
\comicII{.8\textwidth}{puzzle-letters}
  \caption{\small Puzzle that can be solved for letters A, B, H, R, S, Z (E for disconnected galaxies). 
}
  \label{fig:letter-puzzle}
\end{figure}

Standard Spiral Galaxies puzzles are designed to have a unique solution, with the intent that the puzzle solver finds that one solution.
In this section, we briefly explore the possibility of creating a single puzzle with \emph{multiple} solutions, such that each solution forms a different desired picture.
As a case study, we ask for a single puzzle such that, for every letter of the alphabet, there exists a solution to the puzzle that resembles that letter.
In Figure~\ref{fig:letter-puzzle}, we provide a puzzle which has a solution for the black letters A, B, H, P, R, S, Z (and when we allow for \emph{disconnected} galaxies also for the letter E). 
See Figure~\ref{fig:letter-puzzle-sol} for the solutions.

\begin{figure}
\centering
\comic{.45\textwidth}{puzzle-letters-A}{(a)}
\comic{.45\textwidth}{puzzle-letters-B}{(b)}
\comic{.45\textwidth}{puzzle-letters-H}{(c)}
\comic{.45\textwidth}{puzzle-letters-R}{(d)}
\comic{.45\textwidth}{puzzle-letters-S}{(e)}
\comic{.45\textwidth}{puzzle-letters-Z}{(f)}
\comic{.45\textwidth}{puzzle-letters-E}{(g)}
\comic{.45\textwidth}{puzzle-letters-}{(h)}
 \caption{\small (a--f) One puzzle with solutions drawing the letters A, B, H, R, S, and Z. (g) When we allow disconnected galaxies, we can also make the letter E. The three dark gray areas together form a single galaxy, and the three light gray areas as well. (h) The puzzle also has many ``nonsense'' solutions.}
  \label{fig:letter-puzzle-sol}
\end{figure}

While our construction has solutions for several letters, for most letters of the alphabet it does not, and it also has many additional solutions which do not resemble letters at all.
We leave as an open problem to construct puzzles which either solve for more letters, or for fewer nonletters, or both.
More generally, this raises the question of whether it is possible to create puzzles with multiple solutions in such a way that \emph{every} valid solution forms a meaningful picture.

%\todo{ The P we had does not work, and the A is new as the old one did not work}




\section{Conclusion and Discussion}\label{sec:conclusion}
We showed that both Rectangular Galaxies and rectangular-promise Spiral Galaxies are NP/ASP-complete even if the polyominoes are restricted to be rectangles of arbitrary size---and that counting the number of solutions is \#P-complete. Moreover, we proved that Rectangular Galaxies remains NP/ASP/\#P-complete even when only $1\times1$, $1\times3$, and $3\times1$ rectangles are possible. With the proof of the latter variant, we also showed that finding a noncrossing matching in distance-2 grid graphs is NP-complete. %Moreover, we show that the problem of counting the number of solutions for the latter variant is \#P-complete and ASP-complete. 
Finally, we proved NP-completeness of the design problem of finding the minimum number of centers such that there exist galaxies that exactly cover a given shape. The complexity of the latter problem when we aim for unique solutions remains open.

\section*{Acknowledgements}
We thank the anonymous referees for helpful comments.
C. S. was
partially supported by Jubileumsanslaget fr\r{a}n Knut och
Alice Wallenbergs Stiftelse.

%%
%% Bibliography
%%

%% Please use bibtex, 

\bibliography{lit}

%\newpage
%\section*{Appendix}
%\label{sec:app}
%\appendix

\section{Experimental details and more results}
\label{sec:app_exp}
We run all the experiments on Nvidia RTX 2080 Ti GPUs and V100 GPUs. Table~\ref{tab:app_testbed} summarizes the set of images used in each figure or table in the main paper.  

\captionsetup[table]{font=small}
\begin{table}[H]
    \small
    \centering
    \begin{tabular}{|p{2.5cm}|p{10cm}|}
    \toprule
         {\bf Figure/Table} & {\bf Comments}	\\
    \midrule
        Figure~\ref{fig:BN_var}a & We’ve tuned hyperparams for the attack (see Appendix~\ref{sec:app_hyperparam}) and carried out evaluations on the whole CIFAR-subset. The first sampled batch of size 16 from CIFAR-subset was used in Figure~\ref{fig:BN_var}a to demonstrate the quality of recovery for low-resolution images when BatchNorm statistics are not assumed to be known.  \\
        \midrule
        Figure~\ref{fig:BN_var}b & We’ve tuned hyperparams for the attack (see Appendix~\ref{sec:app_hyperparam}) and carried out evaluations on the whole ImageNet-subset. The best-reconstructed image in ImageNet-subset was used in Figure 1b to demonstrate the quality of recovery for high-resolution images when BatchNorm statistics are not assumed to be known.\\
        \midrule
        Figure~\ref{fig:batch_label_dist} & Percentages of class labels per batch were evaluated over the entire CIFAR10 dataset, for a random seed.	\\
        \midrule
        Figure~\ref{fig:reconstructed_labels} & The first sampled batch of size 16 was used in Figure~\ref{fig:reconstructed_labels} to demonstrate the quality of recovery when labels are not assumed to be known.	\\
        \midrule
        Table~\ref{tab:exp_summary} and Figure~\ref{fig:vis_recon} & We’ve tuned hyperparams for the attack and carried out evaluations on the whole CIFAR-subset. Table~\ref{tab:exp_summary} summarizes the performance of the attack on the whole CIFAR-subset and  Figure~\ref{fig:vis_recon} shows example images.\\
    \bottomrule
    \end{tabular}
    \caption{Summary of experimental testbed for each evaluation.}
    \label{tab:app_testbed}
\end{table}


\subsection{Hyper-parameters}
\label{sec:app_hyperparam}



\paragraph{Training.} For all experiments, we train ResNet-18 for 200 epochs, with a batch size of 128. We use SGD with momentum 0.9 as the optimizer. The initial learning rate is set to 0.1 by default, except for gradient pruning with $p=0.99$ and $p=0.999$. where we set the initial learning rate to 0.02. We decay the learning rate by a factor of 0.1 every 50 epochs.

\paragraph{The attack.}  We report the performance under different $\alpha_{\rm TV}$'s (Figure~\ref{fig:BN_tv_tune}) and $\alpha_{\rm BN}$'s (Figure~\ref{fig:BN_reg_tune}).

\begin{figure}[H]
\captionsetup[subfigure]{labelfont=scriptsize, textfont=tiny}
    \centering
    \subfloat[Original]{\includegraphics[width=0.12\textwidth]{imgs/appendix/TV/original.png}}
    \subfloat[$\alpha_{\rm TV}$=0]{\includegraphics[width=0.12\textwidth]{imgs/appendix/TV/tv_0.png}}
    \subfloat[$\alpha_{\rm TV}$=1e-3]{\includegraphics[width=0.12\textwidth]{imgs/appendix/TV/tv_1e-3.png}}
    \subfloat[$\alpha_{\rm TV}$=5e-3]{\includegraphics[width=0.12\textwidth]{imgs/appendix/TV/tv_5e-3.png}}
    \subfloat[$\alpha_{\rm TV}$=1e-2]{\includegraphics[width=0.12\textwidth]{imgs/appendix/TV/tv_1e-2.png}}
    \subfloat[$\alpha_{\rm TV}$=5e-2]{\includegraphics[width=0.12\textwidth]{imgs/appendix/TV/tv_5e-2.png}}
    \subfloat[$\alpha_{\rm TV}$=1e-1]{\includegraphics[width=0.12\textwidth]{imgs/appendix/TV/tv_1e-1.png}}
    \subfloat[$\alpha_{\rm TV}$=5e-1]{\includegraphics[width=0.12\textwidth]{imgs/appendix/TV/tv_5e-1.png}}
    
    \caption{Attacking a single CIFAR-10 images in $\rm BN_{exact}$ setting, with different coefficients for the total variation regularizer ($\alpha_{\rm TV}$'s). $\alpha_{\rm TV}$=1e-2 gives the best reconstruction.}
    \label{fig:BN_tv_tune}
\end{figure}


\begin{figure}[H]
\vspace{-5mm}
\captionsetup[subfigure]{labelfont=scriptsize, textfont=tiny}
    \centering
    \subfloat[Original]{\includegraphics[width=0.16\textwidth]{imgs/assumptions/BN/original.png}}
    \subfloat[$\alpha_{\rm BN}$=0]{\includegraphics[width=0.16\textwidth]{imgs/assumptions/BN/reconstructed_train_train_bn=0.png}}
    \subfloat[$\alpha_{\rm BN}$=5e-4]{\includegraphics[width=0.16\textwidth]{imgs/assumptions/BN/reconstructed_train_train_bn=5e-4.png}}
    \subfloat[$\alpha_{\rm BN}$=1e-3]{\includegraphics[width=0.16\textwidth]{imgs/assumptions/BN/reconstructed_train_train_bn=1e-3.png}}
    \subfloat[$\alpha_{\rm BN}$=5e-3]{\includegraphics[width=0.16\textwidth]{imgs/assumptions/BN/reconstructed_train_train_bn=5e-3.png}}
    \subfloat[$\alpha_{\rm BN}$=1e-2]{\includegraphics[width=0.16\textwidth ]{imgs/assumptions/BN/reconstructed_train_train_bn=1e-2.png}}
    \caption{Attacking a batch of 16 CIFAR-10 images in $\rm BN_{infer}$ setting, with different coefficients for the BatchNorm regularizer ($\alpha_{\rm BN}$'s). $\alpha_{\rm TV}$=1e-3 gives the best reconstruction.}
    \label{fig:BN_reg_tune}
\end{figure}


\subsection{Details and more results for Section~\ref{sec:assumption}}

\paragraph{Attacking a single ImageNet image.} We launched the attack on ImageNet using the objective function in Eq.~\ref{eq:objective}, where $\alpha_{\rm TV}=0.1$, $\alpha_{\rm BN}=0.001$. We run the attack for 24,000 iterations using Adam optimizer, with initial learning rate 0.1, and decay the learning rate by a factor of $0.1$ at 
$3/8,5/8,7/8$ of training. We rerun the attack 5 times and present the best results measured by LPIPS in Figure~\ref{fig:BN_var}.

\paragraph{Qualitative and quantitative results for a more realistic attack.} We also present results of a more realistic attack in Table~\ref{tab:exp_summary_realistic} and Figure~\ref{fig:vis_recon_realistic}, where the attacker does {\em not} know BatchNorm statistics but knows the private labels. We assume the private labels to be known in this evaluation, because for those batches whose distribution of labels is uniform, the restoration of labels should still be quite accurate~\citep{yin2021see}.
As shown, in the evaluated setting, the attack is no longer effective when the batch size is 32 and Intra-InstaHide with $k=4$ is applied. The accuracy loss to stop the realistic attack is only around $3\%$ (compared to around $7\%$ to stop the strongest attack) .


\begin{figure}[H]
\captionsetup[subfigure]{font=small}
  \centering
  \subfloat{\includegraphics[width=\textwidth]{imgs/Compare_16_32.png}}
  \caption{Reconstruction results under different defenses for a more realistic setting (when the attacker knows private labels but does not know BatchNorm statistics). We also present the results for the strongest attack from Figure~\ref{fig:vis_recon} for comparison. Using Intra-InstaHide with $k=4$ and batch size 32 seems to stop the realistic attack.}
  \label{fig:vis_recon_realistic}
\end{figure}

\captionsetup[table]{font=small}
\begin{table}[H] 
  \scriptsize
  \setlength{\tabcolsep}{2.6pt}
  \renewcommand{\arraystretch}{0.95}
  \begin{tabular}{|l|c|c|c|c|c|c|c|c|c|c|c|c|c|c|c|c|}
  \toprule
   &  \multirow{2}{*}{\bf None} & \multicolumn{6}{c|}{\multirow{2}{*}{\bf GradPrune ($p$)}} & \multicolumn{2}{c|}{\multirow{2}{*}{\bf MixUp ($k$)}} & \multicolumn{2}{c|}{\multirow{2}{*}{\bf Intra-InstaHide ($k$)}} & \multicolumn{2}{c|}{\bf GradPrune ($p=0.9$)}\\
   & & \multicolumn{6}{c|}{} & \multicolumn{2}{c|}{} & \multicolumn{2}{c|}{} & {\bf  + MixUp } & {\bf  + Intra-InstaHide}\\
  \midrule
   {\bf Parameter}  & - & 0.5 & 0.7 & 0.9 & 0.95 & 0.99 & 0.999 & 4 & 6 & 4 & 6 & $k=4$ & $k=4$ \\
   \midrule
   {\bf Test Acc.} & 93.37 & 93.19 & 93.01 & 90.57 & 89.92 & 88.61 & 83.58 &  92.31 & 90.41 & 90.04 & 88.20 & 91.37 & 86.10 \\
   \midrule
  {\bf Time (train)} & $1\times$ & \multicolumn{6}{c|}{$1.04\times$} & \multicolumn{2}{c|}{$1.06\times$} & \multicolumn{2}{c|}{$1.06\times$} & \multicolumn{2}{c|}{$1.10\times$} \\
  \midrule
  \multicolumn{14}{|c|}{\bf Attack batch size $= 16$, the strongest attack} \\
  \midrule
  {\bf Avg. LPIPS $\downarrow$}  & 0.41  & 0.41  & 0.42  & 0.46  & 0.48  & 0.50  & 0.55         & 0.50  & 0.49  & 0.69  & 0.69  & 0.62  & \best{0.73}\\
  {\bf Best LPIPS $\downarrow$}  & 0.21  & 0.22  & 0.27  & 0.29  & 0.30  & 0.29  & 0.48         & 0.31  & 0.28  & 0.56  & 0.56  & 0.37  & \best{0.65}\\
  {(LPIPS std.)}                 & 0.09  & 0.08  & 0.07  & 0.06  & 0.06  & 0.06  & 0.04         & 0.10  & 0.10  & 0.06  & 0.07  & 0.10  & 0.05\\
  \midrule
   \multicolumn{14}{|c|}{\bf Attack batch size $= 16$, attacker knows private labels but does not know BatchNorm statistics} \\
   \midrule
   {\bf Avg. LPIPS $\downarrow$}  & 0.49 & 0.51 & 0.48 & 0.51 & 0.52 & 0.56 & 0.60 & 0.71 & 0.71 & \best{0.75} & \best{0.75} & 0.74 &  0.74\\
   {\bf Best LPIPS $\downarrow$}  & 0.30 & 0.33 & 0.31 & 0.33 & 0.34 & 0.39 & 0.44 & 0.48 & 0.53 & \best{0.65} & 0.63 & 0.61 &  0.63\\
   {(LPIPS std.)}                 & 0.08 & 0.09 & 0.08 & 0.08 & 0.07 & 0.07 & 0.05 & 0.08 & 0.07 & 0.04 & 0.05 & 0.08 &  0.05\\
   \midrule
   \multicolumn{14}{|c|}{\bf Attack batch size $= 32$, the strongest attack} \\
  \midrule
  {\bf Avg. LPIPS $\downarrow$}  & 0.45  & 0.46  & 0.48  & 0.52  & 0.54  & 0.58  & 0.63         & 0.50  & 0.49  & 0.69  & 0.69  & 0.62  & \best{0.73}\\
   {\bf Best LPIPS $\downarrow$}  & 0.18  & 0.18  & 0.22  & 0.31  & 0.43  & 0.48  & 0.54         & 0.31  & 0.28  & 0.56  & 0.56  & 0.37  & \best{0.65}\\
   {(LPIPS std.)}                 & 0.11  & 0.11  & 0.09  & 0.07  & 0.05  & 0.04  & 0.04         & 0.10  & 0.10  & 0.06  & 0.07  & 0.10  & 0.05\\
    \midrule
   \multicolumn{14}{|c|}{\bf Attack batch size $= 32$, attacker knows private labels but does not know BatchNorm statistics} \\
   \midrule
   {\bf Avg. LPIPS $\downarrow$}  & 0.48 & 0.50 & 0.53 & 0.53 & 0.55 & 0.60 & 0.63 & 0.73 & 0.72 & 0.76 & 0.76 & 0.76 & \best{0.77} \\
   {\bf Best LPIPS $\downarrow$}  & 0.29 & 0.32 & 0.32 & 0.31 & 0.40 & 0.41 & 0.55 & 0.63 & 0.60 & \best{0.68} & 0.63 & 0.66 & 0.65\\
   {(LPIPS std.)}                 & 0.08 & 0.07 & 0.07 & 0.08 & 0.08 & 0.06 & 0.04 & 0.06 & 0.06 & 0.04 & 0.05 & 0.06 & 0.05\\
  \bottomrule
  \end{tabular}
  \vspace{2mm}
%   \subfloat{\includegraphics[width=0.98\textwidth]{imgs/Compare_16_32.png}}
  \caption{\small Utility-security trade-off of different defenses for a more realistic setting (when the attacker knows private labels but does not know BatchNorm statistics). We also present the results for the strongest attack from Table~\ref{tab:exp_summary} for comparison. We evaluate the attack on 50 CIFAR-10 images and report the LPIPS score ($\downarrow$: lower values suggest more privacy leakage).
  We mark the least-leakage defense measured by the metric in \best{green}.} 
  \label{tab:exp_summary_realistic}
\end{table}

\iffalse
\paragraph{Qualitative and quantitative results for private labels unknown.} Apart from the example in Figure~\ref{fig:reconstructed_labels} with batch size being 16, we provide another example for how unknown labels affect reconstruction quality in Figure~\ref{fig:assumption2_app}, with batch size being 32. We also provide quantitative measurements in Figure~\ref{tab:assumption2_app1} and~\ref{tab:assumption2_app2}.

\begin{figure}[H]
    \centering
    \subfloat[Reconstructions with and without private labels]{
    \includegraphics[width=0.95\textwidth]{imgs/assumptions/label_known_unknown_32.png}
    \label{fig:assumption2_app}
    }\\
    \subfloat[Batch size = 16]{
        \setlength{\tabcolsep}{4pt}
        \small
        \begin{tabular}[b]{|c|c|c|}
                \toprule
                  & {\bf Labels known} &  {\bf Labela unknown} \\
                \midrule
                %  {\bf Avg. PSNR $\uparrow$} & 12.45 & 12.01  \\
                %  {\bf Best PSNR $\uparrow$} & 17.42 & 14.85    \\
                 {\bf Avg. LPIPS $\downarrow$} & 0.44 & 0.58 \\
                 {\bf Best LPIPS $\downarrow$} & 0.25 & 0.32    \\
                \bottomrule
            \end{tabular}
        \label{tab:assumption2_app1}
        }
        \subfloat[Batch size = 32]{
        \setlength{\tabcolsep}{4pt}
        \small
        \begin{tabular}[b]{|c|c|c|}
                \toprule
                  & {\bf Labels known} &  {\bf Labels unknown} \\
                \midrule
                %  {\bf Avg. PSNR $\uparrow$} & 13.01 & 12.16 \\
                %  {\bf Best PSNR $\uparrow$} & 17.09 & 14.62    \\
                 {\bf Avg. LPIPS $\downarrow$} & 0.41 & 0.62 \\
                 {\bf Best LPIPS $\downarrow$} & 0.21 & 0.39    \\
                \bottomrule
            \end{tabular}
        \label{tab:assumption2_app2}
        }
    \caption{A reconstructed batch of 32 images with and without private labels known (a). We also provide quantitative measurements of reconstructions with batch size 16 (b) and 32 (c) ($\downarrow$: lower values suggest more leakage). The gradient inversion attack is weakened when private labels are not available.}
\end{figure}
\fi


% \iffalse
\subsection{More results for the strongest attack}

\paragraph{Full version of Figure~\ref{fig:vis_recon}.} Figure~\ref{fig:vis_recon_full} provides more examples for reconstructed images by the strongest attack under different defenses and batch sizes. 



\begin{figure}[H]
\captionsetup[subfigure]{font=small}
  \centering
  \vspace{-12mm}
  \subfloat[Batch size $=1$]{\includegraphics[width=\linewidth]{imgs/recon_vis_bs=1_BN_exact_small.png}}\\
  \vspace{-3mm}
  \subfloat[Batch size $=16$]{\includegraphics[width=\linewidth]{imgs/recon_vis_bs=16_BN_exact_small.png}}\\
  \vspace{-3mm}
  \subfloat[Batch size $=32$]{\includegraphics[width=\linewidth]{imgs/recon_vis_bs=32_BN_exact_small.png}}\\
  \vspace{-2mm}
  \caption{Reconstruction results under different defenses with batch size 1 (a), 16 (b) and 32 (c). Full version of Figure~\ref{fig:vis_recon}.}
  \label{fig:vis_recon_full}
  \vspace{-2mm}
\end{figure}


\paragraph{Results with MNIST dataset.} We’ve repeated our main evaluation of defenses and attacks (Table~\ref{tab:exp_summary}) on MNIST dataset~\citep{deng2012mnist} with a simple 6-layer ConvNet model. Note that the simple ConvNet does not contain BatchNorm layers. We evaluate the following defenses on the MNIST dataset with a 6-layer ConvNet architecture against the strongest attack (private labels known):

\begin{itemize}
    \item GradPrune (gradient pruning): gradient pruning sets gradients of small magnitudes to zero. We vary the pruning ratio $p$ in \{0.5, 0.7, 0.9, 0.95, 0.99, 0.999, 0.9999\}.
    \item MixUp: we vary $k$ in \{4,6\}, and set the upper bound of a single coefficient to 0.65 (coefficients sum to 1).
    \item Intra-InstaHide: we vary $k$ in \{4,6\}, and set the upper bound of a single coefficient to 0.65 (coefficients sum to 1). 
    \item A combination of GradPrune and MixUp/Intra-InstaHide.
\end{itemize}

We run the evaluation against the strongest attack and batch size 1 to estimate the upper bound of privacy leakage. Specifically, we assume the attacker knows private labels, as well as the indices of mixed images and mixing coefficients for MixUp and Intra-InstaHide. 

\begin{figure}[t]
    \centering
    \includegraphics[width=0.95\linewidth]{imgs/appendix/recon_vis_MNIST.png}
    \caption{Reconstruction results of MNIST digits under different defenses with the strongest atttack and batch size 1.}
    \label{fig:vis_recon_MNIST}
    \vspace{-5mm}
\end{figure}

For MNIST with a simple 6-layer ConvNet, defending the strongest attack with gradient pruning may require the pruning ratio $p\geq 0.9999$. MixUp with $k=4$ or $k=6$ are not sufficient to defend the gradient inversion attack. Combining MixUp ($k=4$) with gradient pruning ($p=0.99$) improves the defense, however, the reconstructed digits are still highly recognizable. Intra-InstaHide alone ($k=4$ or $k=6$) gives a bit better defending performance than MixUp and GradPrune. Combining InstaHide ($k=4$) with gradient pruning ($p=0.99$) further improves the defense and makes the reconstruction almost unrecognizable. 




\subsection{More results for encoding-based defenses}
We visualize the whole reconstructed dataset under MixUp and Intra-InstaHide defenses with different batch sizes in Figure~\ref{fig:encode_bs1}, \ref{fig:encode_bs16} and \ref{fig:encode_bs32}.  Sample results of the original and the reconstructed batches are provided in Figure~\ref{fig:mixup_vs_instahide}.

\begin{figure}[H]
    \centering
    \includegraphics[width=0.95\textwidth]{imgs/appendix/mixup_vs_instahide.png}
    \caption{Original and reconstructed batches of 16 images under MixUp and Intra-InstaHide defenses. We visualize both the original and the absolute images for the Intra-InstaHide defense. Intra-InstaHide makes pixel-wise matching harder.}
    \label{fig:mixup_vs_instahide}
    \vspace{-5mm}
\end{figure}

\begin{figure}[H]
\captionsetup[subfigure]{labelfont=scriptsize, textfont=tiny}
    \centering
    \subfloat[Original]{\includegraphics[width=0.23\textwidth]{imgs/decode_res/InstaHide/bs1_k4/originals.png}} \hspace{1mm}
    \subfloat[MixUp, $k$=4]{\includegraphics[width=0.23\textwidth]{imgs/decode_res/Mixup/bs1_k4/grad_decode.png}} \hspace{1mm}
    \subfloat[MixUp, $k$=6]{\includegraphics[width=0.23\textwidth]{imgs/decode_res/Mixup/bs1_k6/grad_decode.png}} \hspace{1mm}
    \subfloat[MixUp+GradPrune, $k$=4, $p$=0.9]{\includegraphics[width=0.23\textwidth]{imgs/decode_res/Mixup/bs1_k4_gradprune/grad_decode.png}}
    
    \subfloat[Original]{\includegraphics[width=0.23\textwidth]{imgs/decode_res/InstaHide/bs1_k4/originals.png}} \hspace{1mm}
    \subfloat[InstaHide, $k$=4]{\includegraphics[width=0.23\textwidth]{imgs/decode_res/InstaHide/bs1_k4/grad_decode.png}} \hspace{1mm}
    \subfloat[InstaHide, $k$=6]{\includegraphics[width=0.23\textwidth]{imgs/decode_res/InstaHide/bs1_k6/grad_decode.png}} \hspace{1mm}
    \subfloat[InstaHide+GradPrune, $k$=4, $p$=0.9]{\includegraphics[width=0.23\textwidth]{imgs/decode_res/InstaHide/bs1_k4_gradprune/grad_decode.png}}
    \caption{Reconstrcuted dataset under MixUp and Intra-InstaHide against the strongest attack (batch size is 1).}
    \label{fig:encode_bs1}
    \vspace{-10mm}
\end{figure}


\begin{figure}[H]
\captionsetup[subfigure]{labelfont=scriptsize, textfont=tiny}
    \centering
    \subfloat[Original]{\includegraphics[width=0.23\textwidth]{imgs/decode_res/InstaHide/bs1_k4/originals.png}} \hspace{1mm}
    \subfloat[MixUp, $k$=4]{\includegraphics[width=0.23\textwidth]{imgs/decode_res/Mixup/bs16_k4/grad_decode.png}} \hspace{1mm}
    \subfloat[MixUp, $k$=6]{\includegraphics[width=0.23\textwidth]{imgs/decode_res/Mixup/bs16_k6/grad_decode.png}} \hspace{1mm}
    \subfloat[MixUp+GradPrune, $k$=4, p=0.9]{\includegraphics[width=0.23\textwidth]{imgs/decode_res/Mixup/bs16_k4_gradprune/grad_decode.png}}

    
    \subfloat[Original]{\includegraphics[width=0.23\textwidth]{imgs/decode_res/InstaHide/bs1_k4/originals.png}} \hspace{1mm}
    \subfloat[InstaHide, $k$=4]{\includegraphics[width=0.23\textwidth]{imgs/decode_res/InstaHide/bs16_k4/grad_decode.png}} \hspace{1mm}
    \subfloat[InstaHide, $k$=6]{\includegraphics[width=0.23\textwidth]{imgs/decode_res/InstaHide/bs16_k6/grad_decode.png}} \hspace{1mm}
    \subfloat[InstaHide+GradPrune, $k$=4, $p$=0.9]{\includegraphics[width=0.23\textwidth]{imgs/decode_res/InstaHide/bs16_k4_gradprune/grad_decode.png}}
    \caption{Reconstrcuted dataset under MixUp and Intra-InstaHide against the strongest attack (batch size is 16).}
    \label{fig:encode_bs16}
\end{figure}



\begin{figure}[H]
\captionsetup[subfigure]{labelfont=scriptsize, textfont=tiny}
    \centering
    \subfloat[Original]{\includegraphics[width=0.23\textwidth]{imgs/decode_res/InstaHide/bs1_k4/originals.png}} \hspace{1mm}
    \subfloat[MixUp, $k$=4]{\includegraphics[width=0.23\textwidth]{imgs/decode_res/Mixup/bs32_k4/grad_decode.png}} \hspace{1mm}
    \subfloat[MixUp, $k$=6]{\includegraphics[width=0.23\textwidth]{imgs/decode_res/Mixup/bs32_k6/grad_decode.png}} \hspace{1mm}
    \subfloat[MixUp+GradPrune, $k$=4, $p$=0.9]{\includegraphics[width=0.23\textwidth]{imgs/decode_res/Mixup/bs32_k4_gradprune/grad_decode.png}}

    
    \subfloat[Original]{\includegraphics[width=0.23\textwidth]{imgs/decode_res/InstaHide/bs1_k4/originals.png}} \hspace{1mm}
    \subfloat[InstaHide, $k$=4]{\includegraphics[width=0.23\textwidth]{imgs/decode_res/InstaHide/bs32_k4/grad_decode.png}} \hspace{1mm}
    \subfloat[InstaHide, $k$=6]{\includegraphics[width=0.23\textwidth]{imgs/decode_res/InstaHide/bs32_k6/grad_decode.png}} \hspace{1mm}
    \subfloat[InstaHide+GradPrune, $k$=4, $p$=0.9]{\includegraphics[width=0.23\textwidth]{imgs/decode_res/InstaHide/bs32_k4_gradprune/grad_decode.png}}
    \caption{Reconstrcuted dataset under MixUp and Intra-InstaHide against the strongest attack (batch size is 32).}
    \label{fig:encode_bs32}
\end{figure}






We briefly recall the framework of statistical inference via empirical risk minimization.
Let $(\bbZ, \calZ)$ be a measurable space.
Let $Z \in \bbZ$ be a random element following some unknown distribution $\Prob$.
Consider a parametric family of distributions $\calP_\Theta := \{P_\theta: \theta \in \Theta \subset \reals^d\}$ which may or may not contain $\Prob$.
We are interested in finding the parameter $\theta_\star$ so that the model $P_{\theta_\star}$ best approximates the underlying distribution $\Prob$.
For this purpose, we choose a \emph{loss function} $\score$ and minimize the \emph{population risk} $\risk(\theta) := \Expect_{Z \sim \Prob}[\score(\theta; Z)]$.
Throughout this paper, we assume that
\begin{align*}
     \theta_\star = \argmin_{\theta \in \Theta} L(\theta)
\end{align*}
uniquely exists and satisfies $\theta_\star \in \text{int}(\Theta)$, $\nabla_\theta L(\theta_\star) = 0$, and $\nabla_\theta^2 L(\theta_\star) \succ 0$.

\myparagraph{Consistent loss function}
We focus on loss functions that are consistent in the following sense.

\begin{customasmp}{0}\label{asmp:proper_loss}
    When the model is \emph{well-specified}, i.e., there exists $\theta_0 \in \Theta$ such that $\Prob = P_{\theta_0}$, it holds that $\theta_0 = \theta_\star$.
    We say such a loss function is \emph{consistent}.
\end{customasmp}

In the statistics literature, such loss functions are known as proper scoring rules \citep{dawid2016scoring}.
We give below two popular choices of consistent loss functions.

\begin{example}[Maximum likelihood estimation]
    A widely used loss function in statistical machine learning is the negative log-likelihood $\score(\theta; z) := -\log{p_\theta(z)}$ where $p_\theta$ is the probability mass/density function for the discrete/continuous case.
    When $\Prob = P_{\theta_0}$ for some $\theta_0 \in \Theta$,
    we have $L(\theta) = \Expect[-\log{p_\theta(Z)}] = \kl(p_{\theta_0} \Vert p_\theta) - \Expect[\log{p_{\theta_0}(Z)}]$ where $\kl$ is the Kullback-Leibler divergence.
    As a result, $\theta_0 \in \argmin_{\theta \in \Theta} \kl(p_{\theta_0} \Vert p_\theta) = \argmin_{\theta \in \Theta} L(\theta)$.
    Moreover, if there is no $\theta$ such that $p_\theta \txtover{a.s.}{=} p_{\theta_0}$, then $\theta_0$ is the unique minimizer of $L$.
    We give in \Cref{tab:glms} a few examples from the class of generalized linear models (GLMs) proposed by \citet{nelder1972generalized}.
\end{example}

\begin{example}[Score matching estimation]
    Another important example appears in \emph{score matching} \citep{hyvarinen2005estimation}.
    Let $\bbZ = \reals^\tau$.
    Assume that $\Prob$ and $P_\theta$ have densities $p$ and $p_\theta$ w.r.t the Lebesgue measure, respectively.
    Let $p_\theta(z) = q_\theta(z) / \Lambda(\theta)$ where $\Lambda(\theta)$ is an unknown normalizing constant. We can choose the loss
    \begin{align*}
        \score(\theta; z) := \Delta_z \log{q_\theta(z)} + \frac12 \norm{\nabla_z \log{q_\theta(z)}}^2 + \text{const}.
    \end{align*}
    Here $\Delta_z := \sum_{k=1}^p \partial^2/\partial z_k^2$ is the Laplace operator.
    Since \cite[Thm.~1]{hyvarinen2005estimation}
    \begin{align*}
        L(\theta) = \frac12 \Expect\left[ \norm{\nabla_z q_\theta(z) - \nabla_z p(z)}^2 \right],
    \end{align*}
    we have, when $p = p_{\theta_0}$, that $\theta_0 \in \argmin_{\theta \in \Theta} L(\theta)$.
    In fact, when $q_\theta > 0$ and there is no $\theta$ such that $p_\theta \txtover{a.s.}{=} p_{\theta_0}$, the true parameter $\theta_0$ is the unique minimizer of $L$ \cite[Thm.~2]{hyvarinen2005estimation}.
\end{example}

\myparagraph{Empirical risk minimization}
Assume now that we have an i.i.d.~sample $\{Z_i\}_{i=1}^n$ from $\Prob$.
To learn the parameter $\theta_\star$ from the data, we minimize the empirical risk to obtain the \emph{empirical risk minimizer}
\begin{align*}
    \theta_n \in \argmin_{\theta \in \Theta} \left[ L_n(\theta) := \frac1n \sum_{i=1}^n \score(\theta; Z_i) \right].
\end{align*}
This applies to both maximum likelihood estimation and score matching estimation. 
In \Cref{sec:main_results}, we will prove that, with high probability, the estimator $\theta_n$ exists and is unique under a generalized self-concordance assumption.

\begin{figure}
    \centering
    \includegraphics[width=0.45\textwidth]{graphs/logistic-dikin} %0.4
    \caption{Dikin ellipsoid and Euclidean ball.}
    \label{fig:logistic_dikin}
\end{figure}

\myparagraph{Confidence set}
In statistical inference, it is of great interest to quantify the uncertainty in the estimator $\theta_n$.
In classical asymptotic theory, this is achieved by constructing an asymptotic confidence set.
We review here two commonly used ones, assuming the model is well-specified.
We start with the \emph{Wald confidence set}.
It holds that $n(\theta_n - \theta_\star)^\top H_n(\theta_n) (\theta_n - \theta_\star) \rightarrow_d \chi_d^2$, where $H_n(\theta) := \nabla^2 L_n(\theta)$.
Hence, one may consider a confidence set $\{\theta: n(\theta_n - \theta)^\top H_n(\theta_n) (\theta_n - \theta) \le q_{\chi_d^2}(\delta) \}$ where $q_{\chi_d^2}(\delta)$ is the upper $\delta$-quantile of $\chi_d^2$.
The other is the \emph{likelihood-ratio (LR) confidence set} constructed from the limit $2n [L_n(\theta_\star) - L_n(\theta_n)] \rightarrow_d \chi_d^2$, which is known as the Wilks' theorem \citep{wilks1938large}.
These confidence sets enjoy two merits: 1) their shapes are an ellipsoid (known as the \emph{Dikin ellipsoid}) which is adapted to the optimization landscape induced by the population risk; 2) they are asymptotically valid, i.e., their coverages are exactly $1 - \delta$ as $n \rightarrow \infty$.
However, due to their asymptotic nature, it is unclear how large $n$ should be in order for it to be valid.

Non-asymptotic theory usually focuses on developing finite-sample bounds for the \emph{excess risk}, i.e., $\Prob(L(\theta_n) - L(\theta_\star) \le C_n(\delta)) \ge 1 - \delta$.
To obtain a confidence set, one may assume that the population risk is twice continuously differentiable and $\lambda$-strongly convex.
Consequently, we have $\lambda \norm{\theta_n - \theta_\star}_2^2 / 2 \le L(\theta_n) - L(\theta_\star)$ and thus we can consider the confidence set $\calC_{\text{finite}, n}(\delta) := \{\theta: \norm{\theta_n - \theta}_2^2 \le 2C_n(\delta)/\lambda\}$.
Since it originates from a finite-sample bound, it is valid for fixed $n$, i.e., $\Prob(\theta_\star \in \calC_{\text{finite}, n}(\delta)) \ge 1 - \delta$ for all $n$; however, it is usually conservative, meaning that the coverage is strictly larger than $1 - \delta$.
Another drawback is that its shape is a Euclidean ball which remains the same no matter which loss function is chosen.
We illustrate this phenomenon in \Cref{fig:logistic_dikin}.
Note that a similar observation has also been made in the bandit literature \citep{faury2020improved}.

We are interested in developing finite-sample confidence sets.
However, instead of using excess risk bounds and strong convexity, we construct in \Cref{sec:main_results} the Wald and LR confidence sets in a non-asymptotic fashion, under a generalized self-concordance condition.
These confidence sets have the same shape as their asymptotic counterparts while maintaining validity for fixed $n$.
These new results are achieved by characterizing the critical sample size enough to enter the asymptotic regime.







\end{document}
