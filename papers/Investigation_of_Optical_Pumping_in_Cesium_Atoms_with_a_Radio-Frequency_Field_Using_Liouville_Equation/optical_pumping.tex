\documentclass[a4paper]{article}
\usepackage{qcircuit}
\usepackage{graphicx}
%\usepackage{amsmath}
%% Language and font encodings
%\usepackage[english]{babel}
\usepackage[utf8x]{inputenc}
%\usepackage[utf8]{inputenc}
\usepackage[T1]{fontenc}
\usepackage{authblk}
\usepackage{subfig}
%% Sets page size and margins
\usepackage[a4paper,top=3cm,bottom=2cm,left=3cm,right=3cm,marginparwidth=1.75cm]{geometry}
\input{amssym}
%% Useful packages
%\usepackage{graphicx}
\usepackage[colorinlistoftodos]{todonotes}
\usepackage[colorlinks=true, allcolors=blue]{hyperref}
\usepackage{authblk}
%\usepackage{xcolor}
%\usepackage{ulem}


\date{}
%\usepackage[english]{babel}

\title{\bf Investigation of Optical Pumping in Cesium
atoms with a Radio-Frequency field, Using Liouville equation}
\author[1]{H. Davoodi Yeganeh\thanks{h.yeganeh@ut.ac.ir }}
\author[1]{Z. Shaterzadeh-Yazdi\thanks{zahra.shaterzadeh@ut.ac.ir }}

\affil[1]{School of Engineering Science, College of Engineering, University of Tehran, Tehran, Iran, 143-9955961}
%affil[2]{Department of Physics, Faculty of Basic Sciences, ,University of Air Shahid Sattri Tehran, Iran\bf{Please check whether this affiliation is addressed properly}}

\begin{document}
\maketitle

\begin{abstract}
Optical pumping is a technique for engineering atomic-sublevel population of desired atoms. We investigate the population evolution of Cesium atoms by employing Liouville equation. For this purpose, we apply a circularly polarized light at a frequency suitable for electronic transition from ground states to excited states and calculate the relaxation rate, repopulation, and population evolution of the
Cesium Zeeman sublevels. For engineering the sublevel population after optical pumping, we employ a radiofrequency (RF) field and consider the effect of RF field in Liouville equation. With this approach, we are able to prepare desired distribution of the population in the atomic sublevels with high efficiency, which can be employed in different optical experiments.
\end{abstract}
{\bf Keyword}: Optical pumping, Radio frequency field, Alkali atoms, Cesium atom

\section{Introduction}
Optical pumping is a process in which photons of light are interacting with the constituent atoms of a matter. In an isolated collection of atoms in the form of a gas, the atoms occupy their energy states at a given temperature, in a way predicted by standard statistical mechanics. Assuming that the atoms are exposed to a stream of photons, these photons play a key role in redistribution of the states, occupied by the atoms~\cite{r1,r2}. Among different methods that are used for engineering the atomic-sublevel population, such as chirped laser pulse population transfer~\cite{r3} and laser-induced population transfer~\cite{r4}, optical pumping is the most practical one. This method has application in many fields, such as  magneto-optical trapping and laser cooling~\cite{r5}. 

Population evolution of atomic sub-level states, caused by optical pumping, can be described by evolution of its density matrix, by employing Liouville equation. In this method, it is assumed that the system of interest is a closed system, therefore interaction between the atomic system and its surrounding environment is neglected. However, in most cases, interaction between the atomic system and the environment can be characterized phenomenologically as repopulation and relaxation of the atomic system, regardless of explicitly considering the presence of the environment. 

There are other methods to describe the population evolution of the sublevel states in the optical pumping process. One of the methods is the rate equations~\cite{r6}, in which the fast-transient processes in the excited states can be neglected. Another method is the Lindblad master equation by which the optical pumping is described by means of open quantum system~\cite{r7}. Alkali atoms are usually used in optical pumping, because they have a simpler structure than the other atoms~\cite{r8,r9,r10,r11,r12,r13}. Generally speaking, alkali atoms, such as $^{7}$Li, $^{23}$Na, $^{39}$K, $^{87}$Rb, and$^{133}$Cs, possess spin-half nuclei. They have hyperfine-doublet ground states $n\   ^{2}$S$_{1/2}$ given by $F=I-J,\ I+J$ and fine-doublet excited states $n\   ^{2}$P$_{1/2,\ 3/2}$.
The corresponding excited states $n\   ^{2}$P$_{1/2}$ and $n\   ^{2}$P$_{3/2}$ further have the hyperfine
structure for $F= I-J\  ...\  I+J$. Therefore, there are two hyperfine levels for $J=1/2$ and four hyperfine levels  for $J=3/2$. 

In this paper, we investigate the optical pumping of $^{133}$Cs atoms by employing polarized light, and engineer their zeeman sublevel populations, using Liouville equation. We use a semiclassical approach for describing the interaction between the optical light and the Cs atoms. In this process, first, the Cs atoms are pumped to their excited states, and then radiofrequency~(RF) field is added to the system to engineer the sublevel populations. Manipulation of the atomic states is performed by employing a combination of an appropriate laser field, a magnetic field, and a RF radiation~\cite{r18,r16,r15,r14,r17}. The results show that by using the RF field, the excited sublevel populations return to the ground states,  and the population gets distributed evenly between the sublevels' ground states. 

This paper is structured as follows: in Sec.~\ref{sec2}, we introduce a theoretical model for the process of optical pumping, using the Liouville equation approach. In Sec.~\ref{sec3}, based on the introduced model, we present the results of calculating the population evolution of the Cesium sublevels, both at the presence and in the absence of the RF field, in the process of optical pumping. Finally, in Sec.~\ref{sec4} we end up with the concluding remarks.

\section{Theoretical Model of Optical Pumping  with Liouville Equation Approach}\label{sec2}
The state of an atomic system that is evolving in time is described by its density matrix. The time evolution of an atomic system is dominated by primary conditions of the system, structure of the atoms, and external applied fields such as the static electric and magnetic fields, or the optical electromagnetic field. Furthermore, usually the atomic system is not completely isolated from its surrounding environment, and the interaction with the environment needs to be modeled by including phenomenological terms into the evolution equations. These interactions often lead to relaxation and repopulation phenomenon, such as radiative decay and collisions~\cite{r5,r19,r20}. 

We aim to model a system that is composed of an ensemble of thermally distributed atoms with a range of velocities, located in a vapor cell. The time evolution of the density matrix $\rho$, associated with the system of interest, is governed by~\cite{r20}
\begin{equation}\label{eq1}
i \frac{d}{dt}\rho=[H,\rho]-i \frac{1}{2}\{\Gamma, \rho\}+i \Lambda,
\end{equation}
where $H$ is the total Hamiltonian of the system of interest given by $H=H_0+H_I+H_B$, in which $H_0$ is the Hamiltonian of the desired atoms, $H_I$ is the light-atom interaction Hamiltonian and $H_B$ is the magnetic field–atom interaction Hamiltonian. Parameter $\Gamma $ is the diagonal-form relaxation matrix, which shows  the effect of relaxation on the time evolution of the density matrix. In the desired atomic system, each basis state $|n\rangle$ relaxes with the rate $\Gamma_n$.  The density matrix $\rho$ of the system has the condition $Tr(\rho)=1$,  which is indicating that the number of atoms in the system is conserved.  Hence, there must be repopulation corresponding to the relaxation processes, in order to replenish the atoms. Repopulation is represented by the repopulation matrix $\Lambda$ in Eq.~\ref{eq1}. 

In this research work, we consider all the important steps in optical pumping and employ polarized light for the light-matter interaction. Also, we consider selection rules governing atomic transitions.  For the interaction Hamiltonian ($H_I$), we consider the interaction of the atomic system with static electric and magnetic fields, and a radiofrequency field~(RF). All the states of the atomic system is considered by using the density matrix associated with the system. By considering the selection rules, we can compute repopulation and relaxation rates of the desired system.

It worth to note that in the Liouville aproach, which we used for modeling the optical pumping, the effect of external field, e.g. RF fields, can easily be added in the Hamiltonian, whereas in the other methods, such as the rate equation, this is not simple and have mathematical limitations. Therefore, we employ Liouville equation for optical pumping of Cesium atoms. By obtaining the time evolution of the density matrix, we achieve more information about the polpulation and coherent transition of the desired atomic system. Therefore, by using this method, we can model the optical pumping, engineer the states of the system, and use the optical pumping of alkali atoms efficiently.


\section{Results: Population evolution of the Cesium sublevels }\label{sec3}

Cesium atoms have been employed in  various quantum optics experiments, such as cold atoms prepared by laser cooling and trapping~\cite{r22,r21}. In Cs atoms, the two transitions $6 ^2 S_{1/2}\rightarrow 6 ^2 P_{3/2} $ and $6 ^2 S_{1/2}\rightarrow 6 ^2 P_{1/2} $ are the components of a fine-structure doublet, and each of these transitions additionally have hyperfine structures. The ground state of Cs has $J=1/2$ and $I=7/2$, so it has two hyperfine groundstate levels ($F=3,4$) and two excited hyperfine levels ($f=3,4$) in D1 transition line. In addition, each of these hyperfine levels has $2F+1 $ Zeeman sublevels~\cite{r23}. Fine structure, hyperfine structure and Zeeman splitting of D1 lines in Cs atom are shown schematically in Fig.~\ref{f1}. 
\begin{figure}[h!]
\centering
\includegraphics[width=11.5cm]{p1}
\caption{Schematic view of fine structure, hyperfine structure and Zeeman splitting of D1 line of Cs atom, employed in  optical pumping
process.}\label{f1}
\end{figure}

Cesium atom has 16 ground-state sublevels, i.e.~$\{|F=3,M=3\rangle, \ldots, |3,-3\rangle, \ldots|4,4\rangle, \ldots, |4,-4\rangle\}$ and 16 excited-state sublevels, i.e.~$\{|f=3,m=3\rangle, \ldots, |3,-3\rangle\ldots|4,4\rangle ...\ldots |4,-4\rangle\}$.  The states $\{|F,M\rangle\}$ and $\{|f,m\rangle\}$ are the basis states of the system's Hilbert state. Consequently, the density matrix associated with the Cs atom is represented by a $32\times 32$ matrix.
%\begin{equation}
%|F=4,M=4\rangle=
%\pmatrix{
%1 \cr 0 \cr 0 \cr. \cr .  \cr . \cr 0  \cr 0 
%},
%|4,3\rangle=
%\pmatrix{
%0 \cr 1 \cr 0 \cr. \cr .   \cr . \cr 0  \cr 0 
%},
%\ldots
%,
%|f=3,m=-2\rangle=
%\pmatrix{
%0 \cr 0 \cr 0 \cr. \cr .  \cr . \cr 1  \cr 0 
%},
%|f=3,m=-3\rangle=
%\pmatrix{
%0 \cr 0 \cr 0 \cr. \cr .  \cr . \cr 0  \cr 1
%}
%\end{equation}

Assuming the energy of the ground-state sublevels to be zero and the energy of the excited-state sublevels to be $\hbar\omega_0$, then all the elements of the matrix $H_0$ are zero, except those diagonal elements that represents the excited-state sublevels, i.e.
\begin{equation}
|f,m\rangle \langle f,m|=|4,4\rangle \langle 4,4|= \ldots=|3,-3\rangle \langle 3,-3| =\hbar\omega_0.
\end{equation}
Furthermore, we assume that the right-circular polarized light $\sigma^+ $ and the left-circular polarized light $\sigma^- $ are interacting with the Cs atom. Therefore, the light–atom interaction Hamiltonian is given  by
\begin{equation}
H_I =-\bf{E}.\bf{\hat d},
\end{equation}
where $\bf{E}$ is the optical electric field and  $\bf{\hat d}$ is the dipole operator representing the electric dipole moment corresponding to the Cs atom.

The electric field associated with the right-circular polarized light and  the left-circular polarized light are assumed to be $E^+=(E_0 e^{i\omega t},i E_0 e^{i\omega t},0)$ and $E^-=(E_0 e^{i\omega t},-i E_0 e^{i\omega t},0)$, respectively. Also, we choose $d_x= \frac{1}{\sqrt{2}}(d_{-1}-d_{+1})$ where $d_{-1}$ and $d_{+1}$ are the matrix elements of the dipole operator for the light  $\sigma^+ $ and $\sigma^- $, respectively. Using the Wigner–Eckart theorem, these  matrix elements are given by,
\begin{equation}
\langle F_1 m_1|d_{\pm}|F_2m_2\rangle=(-1)^{F_1-m_1}\langle F_1 m_1|d|F_2m_2\rangle \pmatrix{ F_1 & 1& F_2 \cr -m_1 & \pm 1 & m_2},
\end{equation}
where $\pmatrix{ F_1 & 1& F_2 \cr -m_1 & \pm 1 & m_2}$  is Clebsch-Gordan coefficients describing how individual angular momentum states
may be coupled to yield the total angular momentum state of a system; these coefficients are also known as  3j symbol coefficients. 

We are interested in investegating the optical pumping between $F=4$ and $f=3$ in the Cs atom. Optical pumping between these levels is used in many experiments and practical applications such as quantum memories~\cite{r25}. Hence, we assume that the polarized light interact with these sublevels. In general, we can consider the polarized light to interact with the whole atomic system.
% For  x-polarized  light we have 
%\begin{equation}
%H_I^x =-E.\textbf{d}=\Omega_R cos\omega t\  M_x
%\end{equation}
%where $\Omega_R= E_0 \langle F||d||F'\rangle\pmatrix{ F & 1& F' \cr -m_1 & \pm 1 & m_2} $ is the optical Rabi frequency and $M$ is s $32\times 32$ matrix that all of %the elements M are zero expect elements shows coupling between sublevels $F=4$ and $f=3$ according selection rules.
%The schematics  of transitions between sublevels $F=4$ and $f=3$ with  x-polarized light has been shown in figure \ref{f2}. 
%\begin{figure}[h!]
%\centering
%\includegraphics[width=11.5cm]{p2}
%\caption{   The schematics  of transitions between sublevels $F=4$ and $f=3$ with  x-polarized light  in the cesium D1 line that  is employed in  optical pumping
%process  }\label{f2}
%\end{figure}
Also, one should note that there is similar formulation for the right-circular polarized light and the left-circular polarized light. For $\sigma^+$, the light-atom interaction Hamiltonian is given by
\begin{equation}\label{E55}
\hat{H}_I^+ =-\textbf{E}.\hat\textbf{d}=\Omega_R cos\omega t\  M^+,
\end{equation}
where $\Omega_R$ is the optical Rabi frequency given by
\begin{equation}\label{E6}
\Omega_R= E_0 \langle F||d||F'\rangle\pmatrix{ F & 1& F' \cr -m_1 & \pm 1 & m_2}.
\end{equation}
In Eq.~\ref{E55} , $M^+$ is a $32\times 32$ matrix that, according to the selection rules, all its elements are zero except those elements that show coupling between sublevels $F=4$ and $f=3$. Similarly, for the light source $\sigma^-$, we have
\begin{equation}
\hat{H}_I^- =-\textbf{E}.\hat\textbf{d}=\Omega_R cos\omega t\  M^-.
\end{equation}
Transitions between sublevels $F=4$ and $f=3$ with $\sigma^+$ and $\sigma^-$ has been shown schematically in Fig.~\ref{f3}. 
\begin{figure}[h!]
\centering
\includegraphics[width=11.5cm]{p3}
\caption{Schematic view of transitions between sublevels $F=4$ and $f=3$ with polarized light $\sigma^+$  and (b) polarized light $\sigma^-$ in the D1 line of Cs atoms, which is employed in optical pumping processes.}\label{f3}
\end{figure}

Finally, we consider applying the $z$ component of a magnetic field for causing the Zeeman effect. The corresponding interaction Hamiltonian is
\begin{equation}
\hat{H}_B=\hat \mu .\textbf{B}=g\mu_0F_z B_z=\hbar\Omega_L F_z,
\end{equation}
where $g$ is  the $g$-factor, $\mu$ is the magnetic dipole moment, $\Omega_L=g\mu_0 B_z/ \hbar $  is the Larmor frequency and $F_z$ is the total angular momentum matrix for the $z$ direction. 

The constituent terms of the total Hamiltonian of the system, i.e.~$H=H_0+H_I+H_B$, are now determined. Therefore, we can solve Eq.~\ref{eq1}, in order to study the dynamic of the system of interest. The Hamiltonian has a time dependence at the optical frequency. We use rotating wave approximation and neglect the quickly oscillating terms in the Hamiltonian and only keep terms that are showing the detuning frequency, i.e.~$\Delta =\omega-\omega_0 $. 
%In general, rotating wave approximation Hamiltonian can be considered as 
%\begin{equation}
%H_{RWA}=U^\dagger H U-i\hbar U ^\dagger \frac{\partial U}{\partial t},
%\end{equation}
%where $U$ is diagonal transformation matrix that can be constructed by Wigner $D$-function . 
 
Considering relaxation at a rate $\gamma$ and spontaneous decay at a rate $\Gamma_s$,  the relaxation matrix is then given by a diagonal matrix ($32\times 32$), in which the first 16 diagonal elements are $\gamma$ and the other 16 diagonal elements are $\gamma +\Gamma_s$. Repopulation matrix $\Lambda$, in Eq.~\ref{eq1}, is given by a $32\times 32$ matrix that describes  a process in which atoms leave the region of interest, other
atoms may be entering and these atoms may be polarized or
unpolarized  .  In addition, in repopulation matrix, transition rate between various pairs of upper- and lower-state sublevels can be  considered in  different values and coherences between between sublevels as well as  can be transferred by spontaneous decay. By numerical calculation of Eq.~\ref{eq1}, time evolution of the density matrix of the system is obtained. Then, by applying 
$Tr(\rho |F,M\rangle\langle F,M|)$, the population of Zeeman sublevels is extracted. It should be noted that rotating wave approximation does not affect the population of Zeeman sublevels in our calculations.

Figures~\ref{f4} and~\ref{f5} demonstrate numerical results for the time evolution of the Zeeman sublevels population $F=4$, caused by applying the polarized light $\sigma^+$ and $\sigma^-$, respectively. Here, we consider $\Omega_R=11\times 10^3$~Hz, $\Gamma=613$~MHZ, $\Omega_L=0.05\times \Gamma$~MHz, $\Delta=0.5\times \Gamma$~MHz, $\gamma=0.05\times \Gamma$~MHz and $\omega_0=3351.21$~MHz.
The values of these parameters are chosen based on the experimental results obtained for the Cs atom~\cite{r23}.
In Fig.~\ref{f4}, the population of sublevels $F=4$ driven by the field associated with $\sigma^+$ is demonstrated. It can be seen that the population of  Zeeman  sublevels $m_f=4$ and $m_f=3$  are increased, compared to other sublevels, during optical pumping. The reason for the increase of population in these sublevels is the use of $\sigma^+$ light which is obtained by applying atomic transition rules\cite{r24}. Also, population of sublevels $F=4$ caused by the field $\sigma^-$ is ploted in Fig.~\ref{f5}.  It can be seen the populations of  Zeeman  sublevels $m_f=-4$ and $m_f=-3$ , compared to other sublevels, are increased during optical pumping. The reason for population increment in these sublevels is the use of $\sigma^-$ light.
In Fig. ~\ref{f4}, order of sublevels population is $P_{4,3}>P_{4,2}>P_{4,4}>P_{4,1}$ also, in Fig. ~\ref{f5}, order of sublevels population is $P_{4,-3}>P_{4,-2}>P_{4,-4}>P_{4,-1}$ which indicates the employ of $\sigma^+$ light  and $\sigma^-$ light   respectively and this behavior is caused by the energy difference between the sublevels of the cesium atoms in two cases.
\begin{figure}[h!]
\centering
\includegraphics[width=9.6cm]{sigmap0}
\caption{Time evolution of the population for the Zeeman sublevels $F=4$. The applied polarized light, used for optical pumping is  $\sigma^+$. }\label{f4}
\end{figure}
\begin{figure}[h!]
\centering
\includegraphics[width=8.5cm]{sigmam0}
\caption{Time evolution of the population for the Zeeman sublevels $F=4$. The applied polarized light, used for optical pumping is  $\sigma^-$. }\label{f5}
\end{figure}

\newpage
In addition, we investigate the effect of a magnetic-resonance RF field on the time evolution of the population of Zeeman sublevels. Before pumping, the atoms are distributed evenly between the ground-states Zeeman sublevels. After absorbing photons provided by a laser beam, atoms are raised to Zeeman sublevels of the excited states and then decay spontaneously and return back to the ground states sublevels. During this optical process, all of the  atoms are distributed in the ground states sublevels, but the populations of sublevels are not equal. Pumping can be removed by a RF field; the RF field interacts with all the Zeeman sublevels and by a relaxation mechanism equalizes all the sublevel populations~\cite{r24}. 
We consider applying a RF field to Cs atoms, in the $x$ direction and with the associated Rabi frequency $\Omega_{RF}=g\mu_B B_{0RF}$, 
\begin{equation}
\begin{array}{ccl}
 B_{RF} = B_{0RF}\cos(\omega_{RF}t),\\
 \hat{H}_{RF} =\boldmath\mu .\hat{\textbf{B}}=\Omega_{RF}\cos(\omega_{RF}t)F_{RF},
\end{array}
\end{equation}
where $F_{RF}$ is a  ($32\times 32$) matrix, which has elements describing the magnetic dipole transitions $\Delta m=0,\ \pm 1$. Assuming optical pumping between $F=4$ and $f=3$, we consider magnetic dipole transitions between these sublevels. According to Eq.~\ref{eq1},  the Hamiltonian associated with RF field is added to the total Hamiltonian of the desired system. We employ rotating wave approximation for the Hamiltonian of the RF field, and keep only the terms that are showing RF detuning ($\Delta_{RF}=\omega_{RF}-\Omega_L$). 



Figures~\ref{f6} and \ref{f7} show the results of applying a RF field in the process of optical pumping, and its effect on the time evolution of the Zeeman sublevels population. We choose $\Omega_{RF}=2000$~MHz and $\omega_{RF}=\Omega_L$ for our numerical simulation. In fact, we consider resonance case that is used in experiments~\cite{r23,r25,r24}.  In Figure~\ref{f6}, populations of sublevels $F=4$ are ploted for the case of driving with $\sigma^+$ light, at the presence of applied RF field. As it is seen, by increasing the time steps, the populations of Zeeman sublevels become equal. 
Similarly, in Fig.~\ref{f7} populations of sublevels $F=4$ are equalized for $\sigma^-$ light by applying the RF field.
In both cases, by applying the RF field, the population of sub-levels will be equal. In the graph, the difference between the populations of the subscales can be seen at the level of 0.01, which is consistent with the experiment results
Therefore, according to the figures, it can be concluded that the role of the RF field is to equalize all the sublevel populations.
\begin{figure}[h!]
\centering
\includegraphics[width=10cm]{sigmaporf2000}
\caption{Time evolution of the population of Zeeman-sublevels $F=4$, for the applied polarized light $\sigma^+$, at the presence of a magnetic-resonance RF field. }\label{f6}
\end{figure}
\begin{figure}[h!]
\centering
\includegraphics[width=10cm]{sigmamorf2000}
\caption{Time evolution of the population of Zeeman sublevels $F=4$, for the applied polarized light $\sigma^-$, at the presence of a magnetic-resonance RF field.}\label{f7}
\end{figure}
Furthermore, we consider the time evolution of the population of Zeeman sublevels in terms of the Rabi frequency of the RF field. The results are inllustrated in Figs.~\ref{f8} and~\ref{f9}, for the applied polarized light $\sigma^+$ and  $\sigma^-$, respectively.
In Fig.~\ref{f8}, by increasing the value of RF frequency,  population of Zeeman sublevels is equalized faster. Similarly, this behavior can be seen in Fig.~\ref{f9}. 
Hence, according to the results, it can be seen that with the increase of the Rabi frequency, the population of Zeeman sublevels becomes equal.
\begin{figure}[h!]
\centering
\includegraphics[width=10cm]{rfp}
\caption{Time evolution of the population of Zeeman sublevels $F=4$ for $\sigma^+$ light, as a function of RF Rabi frequency.}\label{f8}
\end{figure}
\begin{figure}[h!]
\centering
\includegraphics[width=10cm]{rfm}
\caption{Time evolution of the population of Zeeman sublevels $F=4$ for $\sigma^-$ light as a function of RF Rabi frequency.}\label{f9}
\end{figure}
%\newpage
\begin{figure}[h!]
\centering
\includegraphics[width=10cm]{pol}
\caption{Time evolution of the atomic polarization for the applied light $\sigma^+$ and $\sigma^-$.}\label{f10}
\end{figure}
\newpage
In Fig.~\ref{f10} we consider the time evolution of atomic polarization for $\sigma^+$ and $\sigma^-$, in terms of the Rabi frequency of the RF field. Atomic polarization can be obtained from $\langle F_Z\rangle=1/M \sum_M p_M M$, where $M$ is the Zeeman sublevel projection number (i.e. different $ m_f$) and $p_M$ is its population. It can be seen that with the increase of the Rabi frequency, the atomic polarization associated with the $\sigma^+$ ($\sigma^-$) light decreases (increases). This result is indicating that by inducing imbalance between Zeeman sublevels population by applying a RF field, the polarization would be changed.


%%%%%%%%%%%%%%%%%%%%%%%%
\newpage
\section{Conclusion}\label{sec4}
In this research study, we investigated the optical pumping of Cs atoms by employing Liouville equation approach. Also, we engineered the population of Cesium Zeeman-sublevel states by applying a RF field. For this purpose, a circular polarized light was applied to Cesium atoms and the effect of relaxation and repopulation were studied. Then, by using a RF field and considering its effect in the Liouville equation, engineering of population in Cesium Zeeman sublevels were performed. Finally, the time evolution of the population of Zeeman sublevels was investigated at the presence and in the absence of the RF field, for optical pumping with both polarized lights, $\sigma^+$ and $\sigma^-$. Also the time evolution of the atomic polarization for $\sigma^+$  and $\sigma^-$ was considered. This approach can be used in all alkali atoms and it has many applications in different optical experiments.

%\newpage

\bibliographystyle{unsrt}
\bibliography{re2}



\end{document}
