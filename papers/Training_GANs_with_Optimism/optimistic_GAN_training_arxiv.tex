\documentclass{article} % For LaTeX2e
\usepackage{iclr2018_conference_arxiv,times}
\iclrfinaltrue

\usepackage{graphicx}
\usepackage{epstopdf}
\usepackage{hyperref}
\usepackage{url}

%% packages
\usepackage{xspace}
\usepackage{graphicx}
\usepackage{amsmath,amssymb,amsthm}
\usepackage{booktabs} % For formal tables
%\usepackage{color}
\usepackage{nicefrac}
%\usepackage{graphicx}
%% HERE arxiv
\usepackage[font=small]{caption}
\usepackage{subcaption}
\usepackage{multirow}
\usepackage{algorithm}
\usepackage[noend]{algpseudocode}
\usepackage{bbm}
\usepackage{mathtools}

\usepackage{enumitem}


%\usepackage[]{color-edits}
\usepackage[suppress]{color-edits}
\addauthor{vs}{blue}
\addauthor{cd}{red}
\addauthor{hz}{brown}
\addauthor{ai}{pink}



\definecolor{amber}{rgb}{1.0, 0.01, 0.5}

\newcount\Comments 
\Comments=1
\newcommand{\kibitz}[2]{\ifnum\Comments=1{\color{#1}{#2}}\fi}
\newcommand{\zf}[1]{\kibitz{amber}{[ZF: #1]}}
\newcommand{\cp}[1]{\kibitz{red}{[CP: #1]}}
\newcommand{\vs}[1]{\kibitz{blue}{[VS: #1]}}
\newcommand{\todo}[1]{\kibitz{blue}{[TODO: #1]}}
\newcommand{\E}{\mathbb{E}}
\DeclareMathOperator*{\argmin}{arg\,min}
\DeclareMathOperator*{\argmax}{arg\,max}
\DeclareMathOperator*{\argsup}{arg\,sup}
\DeclareMathOperator*{\arginf}{arg\,inf}
\newcommand{\1}{\mathbbm{1}}
\newcommand{\st}{\sum{t=1}^{T}}
\newcommand{\eps}{\epsilon}
\newcommand{\G}{\mathcal{G}^\eps}
\newcommand{\kk}{\textbf{k}}
\newcommand{\R}{\mathbb{R}}
\newcommand{\B}{\mathcal{B}}
\newcommand{\D}{\mathcal D}
\newcommand{\mbf}{\mathbf}
\newcommand{\p}{\mathbf{p}}
\newcommand{\s}{\mathbf{s}}
\newcommand{\g}{\textbf{g}}
\newcommand{\w}{\mbf{w}}
\newcommand{\winexp}{\textsc{WIN-EXP}}
\newcommand{\winexpG}{\textsc{WIN-EXP-G}}
\renewcommand{\Pr}{\ensuremath{\mathrm{Pr}}}
\newcommand{\opt}{\ensuremath{\textsc{OPT}}}
\newcommand{\twopartdef}[4]
{
	\left\{
		\begin{array}{ll}
			#1 & \mbox{if } #2 \\
			#3 & \mbox{if } #4
		\end{array}
	\right.
}

%% HERE arXiv
\theoremstyle{plain}
\newtheorem{theorem}{Theorem}[]
\newtheorem{corollary}[theorem]{Corollary}
\newtheorem{lemma}[theorem]{Lemma}
\newtheorem{definition}[theorem]{Definition}
\newtheorem{example}[]{Example}
\newenvironment{prevproof}[2]{\noindent {\em {Proof of {#1}~\ref{#2}:}}}{$\hfill\qed$\vskip \belowdisplayskip}
\newtheorem{remark}{Remark}
\newtheorem{proposition}{Proposition}
\newtheorem{claim}{Claim}



\newcommand{\sgamma}{\ensuremath{\sqrt{\gamma}}}
\newcommand{\inner}[3]{\ensuremath{\langle {#1}, {#2} \rangle_{{#3}}}}
\newcommand{\innerab}[3]{\ensuremath{\inner{{#1}}{{#2}}{AM_{{#3}}^T}}}
\newcommand{\innerac}[3]{\ensuremath{\inner{{#1}}{{#2}}{A^TN_{{#3}}^T}}}
\newcommand{\inneraccostas}[3]{\ensuremath{\inner{{#1}}{{#2}}{A^TN_{{#3}}}}}

\newcommand{\norm}[2]{\ensuremath{\left|\left|{#1}\right|\right|^2_{{#2}}}}

\newcommand{\norma}[1]{\ensuremath{\left|\left|{#1}\right|\right|^2_{A^TA}}}
\newcommand{\normat}[1]{\ensuremath{\left|\left|{#1}\right|\right|^2_{AA^T}}}
\newcommand{\normab}[2]{\ensuremath{\norm{{#1}}{AM_{{#2}}^T}}}
\newcommand{\normac}[2]{\ensuremath{\norm{{#1}}{A^TN_{{#2}}^T}}}

\newcommand{\normabb}[2]{\ensuremath{\normm{{#1}}{AM_{{#2}}^T}}}
\newcommand{\normacc}[2]{\ensuremath{\normm{{#1}}{A^TN_{{#2}}^T}}}
\newcommand{\normm}[2]{\ensuremath{\left|\left|{#1}\right|\right|_{{#2}}}}


\newcommand{\normlt}[1]{\ensuremath{\left|\left|{#1}\right|\right|^2_2}}

\newcommand{\x}[3]{\ensuremath{{#1}x_{t{#2}} - 2\eta{#1}Ay_{t{#2}} +
\eta{#1}Ay_{t{#3}}}}
\newcommand{\xto}{\ensuremath{\x{}{-1}{-2}}}
\newcommand{\xtt}{\ensuremath{\x{}{-2}{-3}}}
\newcommand{\xtao}{\ensuremath{\x{A^T}{-1}{-2}}}

\newcommand{\y}[3]{\ensuremath{{#1}y_{t{#2}} + 2\eta{#1}A^Tx_{t{#2}} -
\eta{#1}A^Tx_{t{#3}}}}
\newcommand{\yto}{\ensuremath{\y{}{-1}{-2}}}
\newcommand{\ytt}{\ensuremath{\y{}{-2}{-3}}}
\newcommand{\ytao}{\ensuremath{\y{A}{-1}{-2}}}

\DeclarePairedDelimiter\floor{\lfloor}{\rfloor}

\def\[#1\]{\begin{align*}#1\end{align*}}
\def\(#1\){\ensuremath{\left(#1\right)}}


\begin{document}
% \nipsfinalcopy is no longer used


\title{Training GANs with Optimism}

% The \author macro works with any number of authors. There are two
% commands used to separate the names and addresses of multiple
% authors: \And and \AND.
%
% Using \And between authors leaves it to LaTeX to determine where to
% break the lines. Using \AND forces a line break at that point. So,
% if LaTeX puts 3 of 4 authors names on the first line, and the last
% on the second line, try using \AND instead of \And before the third
% author name.
\newcommand*\samethanks[1][\value{footnote}]{\footnotemark[#1]}
\author{Constantinos Daskalakis\thanks{These authors contribute equally to this work.}\\
MIT, EECS\\
\texttt{costis@mit.edu}\\
\And  
Andrew Ilyas\samethanks\\
MIT, EECS\\
\texttt{ailyas@mit.edu}\\
\And
Vasilis Syrgkanis\samethanks\\
Microsoft Research\\
\texttt{vasy@microsoft.com}\\
\AND	
Haoyang Zeng\samethanks\\
MIT, EECS\\
\texttt{haoyangz@mit.edu}}

\footnotetext[1]{Code for our models is available at \url{https://github.com/vsyrgkanis/optimistic_GAN_training}}

\date{\today}
\maketitle

\begin{abstract}
We address the issue of limit cycling behavior in training Generative Adversarial Networks and propose the use of Optimistic Mirror Decent (OMD) for training Wasserstein GANs. Recent theoretical results have shown that optimistic mirror decent (OMD) can enjoy faster regret rates in the context of zero-sum games. WGANs is exactly a context of solving a zero-sum game with simultaneous no-regret dynamics.  Moreover, we show that optimistic mirror decent addresses the limit cycling problem in training WGANs. We formally show that in the case of bi-linear zero-sum games the last iterate of OMD dynamics converges to an equilibrium, in contrast to GD dynamics which are bound to cycle. We also portray the huge qualitative difference between GD and OMD dynamics with toy examples, even when GD is modified with many adaptations proposed in the recent literature, such as gradient penalty or momentum. We apply OMD WGAN training to a bioinformatics problem of generating DNA sequences. We observe that models trained with OMD achieve consistently smaller KL divergence with respect to the true underlying distribution, than models trained with GD variants. Finally, we introduce a new algorithm, Optimistic Adam, which is an optimistic variant of Adam. We apply it to WGAN training on CIFAR10 and observe improved performance in terms of inception score as compared to Adam.
\end{abstract}

\section{Introduction}
\section{Introduction}  \label{sec:introduction}

\newcommand\inexpIntro[3]{#1?(#2,#3).}
\newcommand\rinexpIntro[3]{*#1?(#2,#3).}
\newcommand\outexpIntro[3]{#1!(#2,#3).}
\newcommand\outatomIntro[3]{#1!(#2,#3)}

We propose a fully automated method for proving termination of \(\pi\)-calculus processes.
Although there have been a lot of studies on termination analysis for the \(\pi\)-calculus
and related calculi~\cite{Deng06IC,Demangeon07,SangiorgiTermination,KobayashiHybrid,Yoshida04IC,DBLP:journals/jlp/DemangeonHS10,Venet98SAS}, most of them have been rather theoretical,
and there have been surprisingly little efforts in developing  fully automated termination
verification methods and tools based on them. To our knowledge,
Kobayashi's \typical{}~\cite{TyPiCal,KobayashiHybrid} is the only exception that
can prove termination of \(\pi\)-calculus processes (extended with natural numbers)
fully automatically, but its termination analysis is quite limited (see Section~\ref{sec:relatedwork}).

Our method is based on a reduction to termination analysis for sequential programs:
we translate a \(\pi\)-calculus process \(P\) to a sequential program \(S_P\), so that
if \(S_P\) is terminating, so is \(P\). The reduction allows us to use
powerful, mature methods and tools
for termination analysis of sequential programs~\cite{heizmann2016ultimate,freqterm,DBLP:conf/lics/PodelskiR04,Kuwahara2014Termination,DBLP:journals/cacm/CookPR11}.

The idea of the translation is to convert a chain of communications on replicated input
channels to a chain of recursive function calls of the target sequential program.
Let us consider the following Fibonacci process:
\begin{align*}
    & \rinexpIntro{\fib}{n}{r}
        \ifexp{n<2}{ \soutatom{r}{1} \\ &\quad}
                   { \nuexp{s_1} \nuexp{s_2} (\outatomIntro{\fib}{n-1}{s_1} \PAR \outatomIntro{\fib}{n-2}{s_2} \PAR \sinexp{s_1}{x}\sinexp{s_2}{y}\soutatom{r}{x+y}) \\}
    & \PAR \outatomIntro{\fib}{m}{r}
\end{align*}
Here, the process
$\rinexpIntro{\fib}{n}{r} \ldots$ is a function server that computes the \(n\)-th Fibonacci number
in parallel and returns the result to \(r\),
and $\outatom{\fib}{m}{r}$ sends a request for computing the \(m\)-th Fibonacci number;
those who are not familiar with the syntax of the \(\pi\)-calculus may wish to consult
Section~\ref{sec:targetlanguage} first.
To prove that the process above is terminating for any integer \(m\),
it suffices to show that there is no infinite chain of communications on $\fib$:
\[
    \fib(m,r) \to \fib(m_1,r_1) \to \fib(m_2,r_2) \to \cdots.
\]
We convert the process above to the following program:\footnote{The actual translation
  given later is a little more complex.}
\begin{verbatim}
 let rec fib(n) = if n<2 then () else (fib(n-1) [] fib(n-2)) in
 fib(m)
\end{verbatim}
Here, \texttt{[]} represents the non-deterministic choice.
Note that, although the calculation of Fibonacci numbers is not preserved,
for each chain of communications on \texttt{fib}, there is a corresponding
sequence of recursive calls:
\[
\mathtt{fib}(m) \to \mathtt{fib}(m_1) \to \mathtt{fib}(m_2) \to \cdots.
\]
Thus, the termination of the sequential program above implies the termination of
the original process.
As shown in the example above, (i) each communication on a replicated input channel
is converted to a function call, (ii) each communication on a non-replicated input
channel is just removed (or, in the actual translation, replaced by a call of
a trivial function defined by \(f(\seq{x})=(\,)\)), and (iii) parallel composition
is replaced by a non-deterministic choice.
We formalize the translation outlined above and prove its correctness.

The basic translation sketched above sometimes loses too much information.
For example, consider the following process:
\begin{align*}
    & \rinexpIntro{\pre}{n}{r} \soutatom{r}{n-1} \\
    & \PAR \rinexpIntro{f}{n}{r} \ifexp{n<0}{ \soutatom{r}{1} }
                                       { \nuexp{s} (\outatomIntro{\pre}{n}{s} \PAR \sinexp{s}{x}\outatomIntro{f}{x}{r}) } \\
    & \PAR \outatomIntro{f}{m}{r}
\end{align*}
The translation sketched above would yield:
\begin{verbatim}
  let pred(n) = n-1 in
  let rec f(n) = if n<0 then () else (pred(n) [] f(*)) in
  f(m)
\end{verbatim}
Here, \texttt{*} represents a non-deterministic integer: since we have removed
the input $\sinatom{s}{x}$, we do not have information about the value of \( x \).
As a result, the sequential program above is non-terminating, although the original
process is terminating.
To remedy this problem, we also refine the basic translation above by using a refinement
type system for the \(\pi\)-calculus. Using the refinement type system,
we can infer that the value of \(x\) in the original process is less than \(n\),
so that we can refine the definition of \texttt{f} to:
\begin{verbatim}
 let rec f(n) = ... else (pred(n) [] let x=* in assume(x<n);f(x))
\end{verbatim}
The target program is now terminating, from which
we can deduce that the original process is also terminating.
We have implemented an automated tool based on the refined translation above.

The contributions of this paper are summarized as follows.
\begin{itemize}
\item The formalization of the basic translation from the \(\pi\)-calculus
  (extended with integers) to sequential programs, and a proof of its correctness.
\item The formalization of a refined translation based on a refinement type system.
\item An implementation of the refined translation, including automated refinement type
  inference based on CHC solving, and experiments to evaluate the effectiveness of
  our method.
\end{itemize}

The rest of this paper is structured as follows.
Section~\ref{sec:targetlanguage} introduces the source and target languages
of our translation.
Section~\ref{sec:approach} 
formalizes the basic translation, and proves its correctness.
Section~\ref{sec:refinement} refines the basic translation by using a refinement type system.
Section~\ref{sec:implementation} reports an implementation and experiments.
Section~\ref{sec:relatedwork} discusses related work,
and Section~\ref{sec:conclusion} concludes the paper.


\section{Preliminaries: WGANs and Optimistic Mirror Descent}
\section{Preliminaries}\label{chpt:preliminiaries}
In this chapter we will introduce some of the mathematical background and notation needed for this thesis. In particular, we will shortly introduce the differential geometric description of spacetime in Section \ref{sec:spacetime_geometry} and give an introduction to the notion of global hyperbolicity and its connection to Green- and normally-hyperbolic operators in Section \ref{sec:global_hyperbolicity}. In a bit more detail, we will introduce the notion of differential forms and give explicit definitions, also in terms of an index based notation, in Section \ref{sec:differential_forms}. For completeness, in Section \ref{sec:cat-theory}, we present basic definitions of category theory. The reader familiar with these topics can safely skip this chapter and refer to it when interested in the chosen conventions.
%
%
%
%
%%%%%%
%%SPACTIME GEOMETRY
%%%%%
%
%
%
\subsection{Spacetime geometry}\label{sec:spacetime_geometry}
In GR, the universe is mathematically described as a four dimensional \emph{spacetime}, consisting of a smooth, four dimensional manifold \gls{M} (assumed to be Hausdorff, connected, oriented, time-oriented and para-compact) and a Lorentzian metric $g$. We will assume the signature of the Lorentzian metric $g$ to be $(-,+,+,+)$. The Levi-Civita connection on $(\M,g)$ is as usual denoted by \gls{nabla}.
Throughout this thesis, we treat spacetime as fixed, implementing a gravitational background determined classically by Einstein's field equations. Hence, we neglect any back-reaction of the fields on the metric, both in the quantum and the classical case. In that sense, we treat the fields as \emph{test fields}.\par
For the basic mathematical theory regarding Lorentzian manifolds, we refer to the literature: An introduction to the topic with an emphasis on the physical application in GR is for example given in \cite{wald_GR} and \cite{carroll_spacetime-and-gr}.
Here, we will shortly recap the notion of a tangent space and tangent bundle and generalize to the notion of a vector bundle which we will use in the general description of normally hyperbolic operators and differential forms.
In the following, we generalize the setting to an arbitrary smooth manifold $\N$ of dimension $N$ with either Lorentzian or Riemannian metric $k$.\par
%
%
A \emph{tangent vector} $v_x$ at point $x \in \N$ is a linear map $v_x : C^\infty(\N , \IR) \to \IR$ that obeys the Leibniz rule, that is, for $f,g \in C^\infty (\N,\IR)$ it holds $v_x(fg) = f(x)v_x(g) + v_x(f)g(x)$.
We define the \emph{tangent space} \gls{TxN} of $\N$ at $x$ as the real $N$-dimensional vector space of all tangent vectors at point $x$.
The disjoint union of all tangent spaces is called the \emph{tangent bundle} \gls{TN} of $\N$ and is itself a manifold of dimension $2N$. A \emph{vector field} is a map $v: \N \to T\N$ such that $v(x) \in T_x\N$.
The respective dual spaces, that is the space of all linear functionals, the \emph{co-tangent space} and the \emph{co-tangent bundle}, are denoted by \gls{TsxN} and \gls{TsN} respectively.\par
%
For Lorentzian manifolds, we call a tangent vector $v$ at $x \in \N$ \emph{timelike} if $k_{\mu \nu} v^\mu v^\nu < 0$, \emph{spacelike} if $k_{\mu \nu} v^\mu v^\nu > 0$ and \emph{null} (or lightlike) if $k_{\mu \nu} v^\mu v^\nu = 0$. At every point $x \in \N$, we define the set of all \emph{causal}, that is, either timelike or null, tangent vectors in the tangent space at $x$. This set is called the \emph{light cone} at $x$ and it is split up into two distinct parts, one that we call the future light cone, and one that we call the past light cone at $x$. Since we assume the manifold to be time orientable, there exists a smooth vector field $t$ that is timelike at every $x \in \N$. Given this time orientation, we identify the future (past) light cone with the set of tangent vectors $v \in T_x\N$ such that $k_{\mu\nu} v^\mu t^\nu < 0$ (respectively $> 0$). Therefore, a tangent vector $v$ at $x$ is called \emph{future directed} (past directed) if it lies in the future (past) light cone at $x$.\\
Accordingly, a curve $\gamma : I \to \N$ is called timelike (spacelike, null, causal, future or past directed) if its tangent vector $\dot{\gamma}$ is timelike (spacelike, null, causal, future or past directed) at every $x \in \N$.  For every point $x \in \N$ we define the \emph{causal future/past} \gls{causalfuturepast} of $x$ as the set of all points $q \in \N$ that can be reached by a future directed causal curve originating in $x$. For any subset $S \in \N$ we define $J^\pm (S) = \bigcup_{x \in S} J^\pm(x)$ and $J(S) = J^+(S) \cup J^- (S)$. Finally, the future/past domain of dependence $\gls{futurepastdomainofdependence}$ of a set $S \subset \N$ is the set of all points $x \in \N$ such that every inextendible causal curve through $x$ intersects $S$. The \emph{domain of dependence} \gls{domainofdependence} of $S$ is the union of the future and past domain of dependence of the set $S$.
For more details on the causal structure of spacetime we refer to for example \cite[Chapter 8]{wald_GR}.\par
%
%
%
The notion of tangent bundles can be generalized to the notion of a vector bundle. Instead of ``attaching'' the vector spaces $T_x \N$ to every point $x$ of the manifold, we allow for the occurrence of arbitrary vector spaces, called the fibres of the vector bundle. A vector bundle then consists of the base manifold, in our case $\N$, the total space and a map $\pi$ from the total space to the base manifold, that can be locally trivialized. At each point of the base manifold, the pre-image of $\pi$ is the fibre of the vector bundle. To be precise we define, following \cite{rudolph_schmidt}:
\begin{definition}[Vector bundle]
	A smooth \emph{vector bundle} over $\N$ is a tuple $\gls{vectorbundle} = (E,\N, \pi)$, where $E$ is a smooth manifold and $\pi : E \to \N$ is a smooth surjective map satisfying:
	\begin{enumerate}
		\item For every $x \in \N$, $\pi^{-1}(x)$ is a vector space, called the fibre of the bundle at point $x$.
		\item There exists a finite dimensional vector space $F$, an open covering $\left\{ U_\alpha\right\}_\alpha$ of $\N$ and a family of diffeomorphisms $\chi_\alpha : \pi^{-1}(U_\alpha) \to U_\alpha \times F$ such that for all $\alpha$ it holds $\chi_\alpha \comp \text{pr}_1 =  \restr{\pi}{\pi^{-1}(U_\alpha)}$ and for every $x \in \N$ the map $\text{pr}_2 \comp \restr{\chi_\alpha}{\pi^{-1}(x)} : \pi^{-1}(x) \to F$ is linear.
	\end{enumerate}
\end{definition}
Here, the maps $\text{pr}_1$ and $\text{pr}_2$ denote the projection onto the first respectively second component of an element in $U_\alpha \times F$. The properties graphically mean that \emph{locally}, the vector bundle ``looks like" the product of the base manifold with the fibre. The tuples $(U_\alpha, \chi_\alpha)$ are called \emph{local trivializations} of the vector bundle. Like for vector spaces, we can define the sum and product of vector bundles, by using the according vector space definitions on the fibres of the bundle.\par
Let $\mathfrak{X}, \mathfrak{Y}$ be vector bundles over $\N$ with fibres $X_x$ and $Y_x$ at $x \in \N$. We denote by \gls{whitneysum} the \emph{Whitney sum} of the two vector bundles - the vector bundle over $\N$ whose fibres are given by the direct sum $X_x \oplus Y_x$. Similarly, one obtains the local trivializations of the Whitney sum from the trivializations of $\mathfrak{X}, \mathfrak{Y}$ and direct sums.\par
Accordingly, let $\mathfrak{X}, \mathfrak{Y}$ be vector bundles over $\N$ and $\widetilde{\N}$, with fibres $X_x$ and $Y_{\tilde{x}}$ at $x \in \N$, $\tilde{x} \in \widetilde{\N}$ respectively. We denote by \gls{outerproductbundle} the \emph{outer product} of the two vector bundles - the vector bundle over $\N \times \widetilde{\N}$ whose fibres are given by the tensor products $X_x \otimes Y_x$. Similarly, one obtains the local trivializations of the outer product from the trivializations of $\mathfrak{X}, \mathfrak{Y}$ and tensor products. \par
%
Finally, we generalize the notion of vector fields:
\begin{definition}[Sections of vector bundles]
Let $\mathfrak{X}=(E,\N,\pi)$ be a vector bundle with fibres $X_x=\pi^{-1}(x)$ at $x \in \N$. A \emph{smooth section} of the vector bundle is a smooth map $\gamma : \N \to E$ such that $\gamma(x) \in X_x$ for all $x \in \N$. The \emph{vector space of smooth sections} of $\mathfrak{X}$ is denoted by \gls{gammax}, the one with compactly supported sections is as usual denoted by \gls{gammaxzero}.
\end{definition}
In this language, a vector field $v$ is just a smooth section of the tangent bundle of a manifold, $v \in \Gamma(T\N)$. One may therefore identify the physical notion of fields with smooth sections of vector bundles. This point of view will be used to define the notion of differential forms in Section \ref{sec:differential_forms}.\par
In this thesis, we usually are interested in complex valued functions (or sections in general). Therefore, we view all occurring vector bundles as complex, in the sense that we take two distinct copies of the vector bundle, one representing the real, one the imaginary part of the bundle. A section of that complex vector bundle is just a pair of two sections of the real vector bundle under consideration. From now, if not specified explicitly, we will view all vector bundles, including the tangent bundle $T\N$, as complex vector bundles. Accordingly, smooth sections of those bundles will in general be complex valued.
%
%
%
%
%
%
%
%
%%%%%%%
%%PARTIAL DIFFERENTIAL OPERATORS AND GLOBAL HYPERBOLICITY
%%%%%%%
%
%
%
\subsection{Partial differential operators and global hyperbolicity}\label{sec:global_hyperbolicity}
When dealing with field theories, whether classical or quantum, one is, of course, interested in the dynamics of the fields. These are usually described by some partial differential equation, often of second order. In the following, we give a short introduction to the theory of certain partial differential operators acting on smooth sections of a vector bundle over the spacetime $(\M,g)$.\par
%
As we have seen, these smooth sections are generalizations of the notion of a field.  In the following, let $\mathfrak{X}$ denote a vector bundle over the manifold $\M$ and let $P: \Gamma(\mathfrak{X}) \to \Gamma(\mathfrak{X})$ be a partial differential operator acting on smooth sections of the bundle. As in the case of flat spacetime, we are interested in basic questions regarding the differential equation $Pf = j$, for example: Can we formulate a (globally) well posed initial value problem? Does the differential equation possess (unique) solutions? To answer these questions, we will now restrict to the case where $P$ is linear and of second order, as it is often the case in physical applications. One can show that for a certain class of such operators, namely normally hyperbolic partial differential operators of second order, we can rigorously treat these questions.\par
Choosing local coordinates $x=(x_\mu)$ on $\M$ and a local trivialization of $\mathfrak{X}$, a linear partial differential operator of second order is called \emph{normally hyperbolic} if it takes the form
\begin{align}
	P = - \sum_{\mu,\nu} g^{\mu \nu} \partial_\mu \partial_\nu + \sum_{\alpha} A_\alpha (x) \partial_\alpha + B(x) \formspace,
\end{align}
where $A_\alpha$ and $B$ are matrix-valued coefficients depending smoothly on the coordinate $x$ (see. \cite[Chapter 1.5]{baer_ginoux_pfaeffle}). One can also formulate a coordinate independent definition in terms of the principal symbol, which we will not present here (see for example \cite[Section 1.5]{baer_ginoux_pfaeffle} ). \par
%
Normally hyperbolic operators possess unique fundamental solutions (see for example the fundamental solutions to the wave operator as noted in Lemma \ref{lem:fundamental_solution_wave_operator}). These fundamental solutions fulfill certain physically important properties, such as a finite propagation speed smaller than the speed of light. Furthermore, specifying the initial data on some space-like hypersurface $X \in  \M$ specifies a unique solution on the domain of dependence $D(X)$ of $X$. Due to these properties, one often calls normally hyperbolic operators just \emph{wave operators}. But to state a \emph{globally} well posed initial value problem for a wave equation, we need to restrict the class of spacetimes $\M$ under consideration to those that possess space-like hypersurfaces $X$ whose domain of dependence is all of the spacetime, $D(X) = \M$. This leads to the notion of \emph{globally hyperbolic} spacetimes:
\begin{definition}[Global Hyperbolicity]
	A spacetime $\M$ is called \emph{globally hyperbolic} if there exists a Cauchy surface $\gls{sigma}$ in $\M$.
\end{definition}
\noindent Here, a Cauchy surface is a space-like hypersurface $\Sigma \subset \M$ such that every inextendible causal curve $\gamma$ intersects $\Sigma$ exactly once. One can show that Cauchy surfaces fulfill the desired property mentioned above, that is,  $D(\Sigma) = \M$. Furthermore, one can show that any globally hyperbolic spacetime $\M$ is foliated by a one-parameter family $\left\{ \Sigma_t \right\}_t$ of Cauchy surfaces (see for example \cite[Theorem 8.3.14]{wald_GR}). \par
In physical applications, one often finds the dynamics of a theory to be described by wave operators. Most prominently, the Klein-Gordon operator $(\square + m^2)$ acting on scalar fields, or its generalization, the wave operator acting on differential forms introduced in Section \ref{sec:differential_forms}, is normally hyperbolic. But there are also important physical field theories that are not described by wave operators, such as the Proca field treated in this thesis. It turns out that the Proca operator (see Definition \ref{def:proca_operator}) is a so called \emph{Green-hyperbolic} operator. These are again partial differential operators $P$ of second order acting on smooth sections of some vector bundle, such that $P$ (and its dual $P'$) posses fundamental solutions. Obviously, normally hyperbolic operators are Green-hyperbolic, but the opposite is not true. One can generalize some results obtained by studying normally hyperbolic operators to Green-hyperbolic operators. An introduction to this topic is given in \cite{baer_green-hyperbolic}, where it is also shown that the Proca operator is Green-hyperbolic but not normally hyperbolic.\par
For our application, the notion of Green-hyperbolicity is not of vast importance, but it is worth mentioning that there exists a more detailed mathematical background on the treatment of such operators.
A very detailed description of normally hyperbolic operators on Lorentzian manifolds, including proofs of the above statements regarding the initial value problem and the existence of fundamental solutions, is given in \cite{baer_ginoux_pfaeffle}, also with an overview of quantization. A shorter introduction to the topic is for example treated in \cite{baer-ginoux_classical-and-quantum-fields}, also with a description of quantization.
%
%
%
%
%
%
%%%
%
%
%
%%
%%%%%%%%%
%%%DIFFERENTIAL FORMS
%%%%%%%%
%
%
%
\subsection{Differential forms}\label{sec:differential_forms}
%
%
Differential forms provide an elegant, coordinate independent description of calculus on smooth manifolds. In particular, they generalize the notion of line- and volume-integrals that are known from analysis. Differential forms play a remarkable role in physics, as one can argue that they indeed describe fundamental physical entities. As an example, instead of viewing a classical force as a vector, one can think of it, more closely related to experiments, as a differential one-form that assigns a scalar to a tangent vector of a curve. This scalar is the (infinitesimal) work associated with the force along the curve. Also, differential forms allow for an elegant geometric description of field theories, for example the Maxwell and Proca field theories that we encounter in this thesis. In Maxwell's classical theory of electromagnetism, instead of viewing the electric and magnetic field (which are conceptually just forces) as the fundamental physical entities, one introduces the \emph{vector potential}, a one-form, consisting of the scalar electric potential and the vector potential associated with the magnet field. Experiments like the Aharonov-Bohm experiment allow for an interpretation of the vector potential as the fundamental physical object, rather than the associated electromagnetic field. \\
Even more fundamentally, the two main theories of physics, General Relativity and the Standard Model of particle physics, are field theories. They are deeply connected to a geometric interpretation and can be elegantly described using differential forms. \par
%
%
Despite of all this, differential forms are usually not part of the standard curriculum of physicists. We shall therefore introduce the basic aspects and definitions regarding differential forms that are used in this thesis. For a more detailed introduction we refer to the literature: For example \cite[Chapter 2 and 4]{rudolph_schmidt} or \cite[Appendix B]{wald_GR} provide introductions to the topic.\par
%
%
In the following, let $\N$ denote a smooth $N$-dimensional manifold, assumed to be Hausdorff, connected, oriented and para-compact, with either Lorentzian or Riemannian metric $k$ and Levi-Civita connection $\nabla$. For a Lorentzian manifold we use the sign convention $(-,+,\dots,+)$ of the metric $k$. The number of negative eigenvalues of $k$ is denoted by $s$, so $s=0$ for a Riemannian manifold and, in our convention, $s=1$ for a Lorentzian manifold.
Later, we will specify to a four dimensional (globally hyperbolic) spacetime consisting of a four dimensional manifold $\M$ with Lorentzian metric $g$ and Cauchy surface $\Sigma$ with induced Riemannian metric $h$.
%
We define:
\begin{definition}[Differential form]
	Let $p\in \{0,1,\dots,N\}$. A \emph{differential form} $\omega$ of degree $p$, or $p$-form for short, on the manifold $\N$ is an anti-symmetric tensor field of rank $(0,p)$. That is, at every point $x \in \N$, $\omega_x$ is an anti-symmetric multi-linear map
	\begin{align}
	\omega_x : \underbrace{T_x \N \times T_x \N \times \cdots \times T_x \N}_{p\text{-times}} \to \IR \formspace.
	\end{align}
	We denote the vector space\footnote{Naturally, addition and scalar multiplication are defined point-wise.} of $p$-forms on $\N$ by $\gls{omegap}$, the space with compactly supported ones by \gls{omegapz}.
\end{definition}
As an example, a zero-form $f \in \Omega^0(\N)$ is just a $C^\infty$-function from $\N$ to $\IR$, hence we can identify $\Omega^0(\N) = C^\infty (\N, \IR)$. A one-form $A \in \Omega^1(\N)$ is nothing more than a co-vector field and in a physical context usually denoted in local coordinates by $A_\mu$. Note, that alternatively one can directly define a $p$-form as a smooth section of the $p$-th exterior product of the co-tangent bundle and hence identify $\Omega^p(\N) = \Gamma \big( \largewedge^k T^*\N\big)$. As mentioned in Section \ref{sec:spacetime_geometry}, we view the tangent bundle as a complex bundle. Therefore, the sections of that bundle will be complex valued functionals. In that fashion, we will usually view the spaces $\Omega^p(\N)$ as complex valued differential forms.\par
%
Next we define the basic operations, besides addition and scalar multiplication, that one can perform on differential forms.
%
\begin{definition}[Exterior product]
	Let $A \in \Omega^p(\N)$ be a $p$-form and  $B\in \Omega^q(\N)$ a $q$-form on $\N$. \\
	The \emph{exterior product} $\gls{wedge}:\Omega^p(\N) \times \Omega^q(\N) \to \Omega^{p+q} (\N)$ is defined by
	\begin{align}
	(A \wedge B)_{\mu_1\dots\mu_p \nu_1\dots\nu_q} = \frac{(p+q)!}{p!q!}\, A_{[\mu_1 \dots \mu_p} B_{\nu_1\dots\nu_q]} \formspace,
	\end{align}
	where the anti-symmetrization of a tensor $T$ is given through
	\begin{align}
	T_{[\mu_1\dots\mu_p]} = \frac{1}{p!} \sum\limits_{\sigma\in S_N }\textrm{sgn}(\sigma) T_{\sigma(\mu_1)\dots\sigma(\mu_p)} \formspace.
	\end{align}
\end{definition}
Here, $S_N$ denotes the symmetric group\footnote{Usually the symmetric group is defined as the set of permutations of $\{1,2,\dots,N\}$ but we chose the index to run over $\{0,1,\dots,N-1\}$, identifying the time component with zero rather then one.} of degree $N$, consisting of permutations of the set $\{0,1,\dots,N-1\}$.
With this notion of multiplication, point-wise addition and scalar multiplication, the space $\gls{omega} \coloneqq \bigoplus_{p = 0}^\infty \Omega^p(\N) = \bigoplus_{p = 0}^N \Omega^p(\N)$ becomes an algebra, usually called the Grassmann- or \emph{exterior algebra} of differential forms on $\N$. We have used that obviously $\Omega^k(\N) =0$ for $k >N$ due to the anti-symmetrization.\par
Furthermore, we find a notion of how to \emph{pullback} differential forms on manifolds to another manifold, for example the pullback of a differential form on the spacetime $\M$ to differential forms on its Cauchy surface $\Sigma$. Given a $C^\infty$-map $\psi: \widetilde{\N} \to \N$, where $\N, \widetilde{\N}$ are manifolds, we can naturally define the pullback of a function $f \in \Omega^0(\N)$ to a function $(\psi^* f) \in \Omega^0(\widetilde{\N})$ by composing $f$ with $\psi$:
\begin{align}
\psi^* f \coloneqq f \comp \psi \formspace.
\end{align}
\newpage
With the pullback of functions defined, we can define how to \emph{push forward}, or carry along, vector fields on $\widetilde{\N}$ to vector fields on $\N$: Let $f\in \Omega^0(\N)$ and $\tilde{v} \in \Gamma(T\widetilde{\N})$ and $\tilde{x} \in \widetilde{\N}$. Then
\begin{align}
(\psi_* \tilde{v})_{\psi(\tilde{x})} (f) \coloneqq \tilde{v}_{\tilde{x}}(\psi^* f)
\end{align}
defines the vector field $(\psi_* v) \in \Gamma(T\N)$. With these basic operations at hand, we can generalize to define the pullback of differential forms:
\begin{definition}[Pullback]\label{def:pullback}
	Let $\N, \widetilde{\N}$ be manifolds of dimension $N,\widetilde{N}$ respectively, and let $\psi: \widetilde{\N} \to \N$ be a smooth map. Then, $\psi$ defines an algebra homomorphism $\psi^* : \Omega(\N) \to  \Omega(\widetilde{\N})$,
	called the \emph{pullback} of differential forms. For $\omega \in \Omega^p(\N)$, $\tilde{x} \in \widetilde{\N}$ and $\tilde{v}_i \in T_x \widetilde{\N}$, $i=1,2,\dots,p$, it is defined by
	\begin{align}
	\left( \psi^* \omega \right)_{\tilde{x}}  (\tilde{v}_1,\tilde{v}_2,\dots,\tilde{v}_p) \coloneqq \omega_{\psi(\tilde{x})} (\psi_* \tilde{v}_1, \dots , \psi_* \tilde{v}_p) \formspace.
	\end{align}
\end{definition}
%
%
%
%
On the exterior algebra we find a duality, provided by the Hodge operator:
\begin{definition}[Hodge dual]
	The hodge star operator $\gls{hodge}: \Omega^p(\N) \to \Omega^{N-p}(\N)$ is defined through
	\begin{align}
	B \wedge *A = \frac{1}{p!} B^{\mu_1\dots\mu_p}A_{\mu_1\dots\mu_p} \dvolk \formspace,
	\end{align}
	which yields the coordinate representation
	\begin{align}
	(*A)_{\mu_{p+1}\dots\mu_N} = \frac{\detk}{p!} \, \epsilon_{\mu_1\dots\mu_N} A^{\mu_1\dots\mu_p} \formspace.
	\end{align}
\end{definition}
Here, \gls{levicivita} denotes the fully antisymmetric tensor of rank $N$ (Levi-Civita symbol) satisfying $\epsilon_{12,\dots,N} =1$ and the \emph{volume element} \gls{dvolk} is defined by
\begin{align}
\left( \gls{dvolk} \right)_{\alpha_1\dots\alpha_N} = \detk \, \epsilon_{\alpha_1\dots\alpha_N} \formspace.
\end{align}
In a sense, the volume element describes how the curvature of the manifold deforms a unit volume.
The duality follows from the important property of the Hodge operator as stated in the following lemma:
\begin{lemma}
	Let $*$ denote the Hodge star operator on the exterior algebra $\Omega(\N) $. It holds that
	\begin{align}
	** = (-1)^{s+p(N-p)} \, \mathbbm{1} \formspace,
	\end{align}
	which is trivially equivalent to $*^{-1} = (-1)^{s+p(N-p)} \, *$.
\end{lemma}
\begin{proof}
	Let $A \in \Omega^p(\N)$ be a $p$-form on $\N$. Then:
	\begin{align}
	(*{*A})_{\mu_1 \dots \mu_p}
	&= \frac{\detk \, \detk}{p! \, (N-p)!} \; \epsilon_{\alpha_{p+1}\dots\alpha_N \mu_1 \dots \mu_p}\;\epsilon^{\alpha_{1}\dots\alpha_N}\;A_{\alpha_1\dots\alpha_p} \notag\\
	&= (-1)^{p(N-p)} \frac{\detk \, \detk}{p! \, (N-p)!} \; \epsilon_{\alpha_{p+1}\dots\alpha_N \mu_1 \dots \mu_p}\;\epsilon^{\alpha_{p+1}\dots\alpha_{N}\alpha_1\dots\alpha_p}\;A_{\alpha_1\dots\alpha_p}  \notag\\
	&= (-1)^{s+p(N-p)} \delta\indices{^{[\alpha_{1}}_{\mu_{1}}}\, \dots \, \delta\indices{^{\alpha_p ] }_{\mu_p}} \;A_{\alpha_1\dots\alpha_p} \notag\\
	&=  (-1)^{s+p(N-p)}\;A_{\mu_1\dots\mu_p} \formspace
	\end{align}
	We have used Lemma \ref{lem:epsilon_contraction} and, in the last step, that the anti-symmetrization is absorbed by contraction because $A$ is antisymmetric.
\end{proof}
%
%
%
%
%
Furthermore, we can equip the exterior algebra with a differentiable structure, introducing the notion of the exterior derivative.
\begin{definition}[Exterior derivative]
	The \emph{exterior derivative} $\gls{d}:\Omega^p(\N) \to \Omega^{p+1} (\N)$ is defined by the following properties:
	\begin{enumerate}
		\item $d$ is linear
		\item $d$ obeys a graded Leibniz rule: Let $A \in \Omega^p(\N)$ and  $B\in \Omega^q(\N)$, then
		\begin{align}
		d(A \wedge B) = dA \wedge B + (-1)^p \, A \wedge dB
		\end{align}
		\item $d$ is nilpotent, that is,  $d^2 = 0$.
	\end{enumerate}
	In local coordinates, this is equivalent to the representation
	\begin{align}
	(dA)_{\mu \alpha_1\dots\alpha_p} = (p+1)\, \nabla_{[\mu}A_{\alpha_1\dots\alpha_p]} \formspace.
	\end{align}
\end{definition}
An important property of the exterior derivative is that it commutes (or rather intertwines its action) with pullbacks (see \cite[Proposition 4.1.7]{rudolph_schmidt}).
A $p$-form $\omega \in \Omega^p(\N)$ is called \emph{exact} if there is a $(p-1)$-form $\alpha \in \Omega^{p-1}(\N)$ such that $\omega = d\alpha$. We call $\omega$ \emph{closed} if $d \omega =0$. Accordingly, the space of closed $p$-forms is denoted by \gls{omegapd}, the space of exact ones by \gls{domegap}. As usual, the ones with compact support are denoted by a subscript zero. Note, that every exact form is closed, using that $d$ is by definition nilpotent, but the reverse is in general not true. It does hold, however, on certain manifolds with trivial topology, such as Minkowski spacetime. This is expressed in the so called Poincar\'e-Lemma (see for example \cite[Chapter 4]{bott_tu}) based on the study of de Rham cohomology.\par
%
Moreover, $N$-forms can naturally be integrated. Using local coordinates and a partition of unity, we define the integral of $N$-forms via the well known integration on $\IR^N$:
\begin{definition}[Integration on manifolds]
	Let $\left\{U_\alpha, \psi_\alpha\right\}_\alpha$ be an atlas of the manifold $\N$ and $\left\{\chi_\alpha\right\}_\alpha$ a partition of unity subordinate to the locally finite open cover $\left\{U_\alpha\right\}_\alpha$. Let $x^\mu_{(\alpha)}$ be a coordinate basis of $\psi$ on $U_\alpha$. For any $N$-form $\omega \in \Omega^N_0(\M)$ we define the integral
	\begin{align}
	\int\limits_{\N} \omega &\coloneqq \sum_{\alpha} \int\limits_{\psi_\alpha (U_\alpha)} w(x_{(\alpha)}^0,\dots,x_{(\alpha)}^1)\; dx_{(\alpha)}^0 \cdots dx_{(\alpha)}^{N-1} \formspace,
	\end{align}
	where $w$ are the components of $\omega$ in the coordinates $x_{(\alpha)}^\mu$, that is $\omega = w dx_{(\alpha)}^0 \wedge \cdots \wedge dx_{(\alpha)}^{N-1}$.
	This definition is independent of the choice of the atlas and the partition of unity (see \cite[Proposition 3.3]{bott_tu}).
\end{definition}
With integration at our disposal, we present an important theorem regarding the integration of exact differential forms:
\begin{theorem}[Stoke's Theorem]\label{thm:stokes}
	Let $\N$ be an oriented manifold of dimension $N$ and let its boundary $\partial \N$ be endowed with the induced orientation. Let $\gls{inclusionmap} : \partial \N \hookrightarrow \N$ be the inclusion operator.
	Let $\omega \in \Omega^{N-1}_0(\N)$ be a compactly supported $(N-1)$-form on $\N$. Then it holds
	\begin{align}
	\int\limits_\N d\omega = \int\limits_{\partial \N} i^*\omega \formspace.
	\end{align}
\end{theorem}
\begin{proof}
	A proof is given in most of the introductory literature on differential geometry (see for example \cite[Chapter 17, Theorem 2.1]{lang}).
	Note that one can equivalently formulate Stoke's theorem on a \emph{compact} manifold but for {arbitrary} (that is, in general not compactly supported) $(N-1)$-forms on the manifold (see for example \cite[Theorem 4.2.14]{rudolph_schmidt}). This will be of importance in later calculations.
\end{proof}
%
Furthermore, we can define a bilinear map on $\Omega^p(\N)$ using the integration of $N$-forms:
\begin{definition}
	Let $A,B \in \Omega^p(\N)$ such that their supports have a compact intersection. Define the bilinear map $\gls{innerprod} : \Omega^p(\N) \times \Omega^p(\N) \to \IC$ by
	\begin{align}
	\langle A, B \rangle_\N \coloneqq  \int_{\N } A \wedge * B = \int_{\N } A_{\mu_1 \dots \mu_p}B^{\mu_1 \dots \mu_p}\,\dvolk \formspace.
	\end{align}
\end{definition}
Since by definition $A \wedge * B$ is a compactly supported $N$-form, this is well defined. We may sometimes refer to $\langle \cdot , \cdot \rangle_\N$ as an inner product for simplicity, even though it is not positive definite.
%
%
%
%
%
Using the exterior derivative, we define the interior or co-derivative:
\begin{definition}[Interior derivative]
	The \emph{interior derivative} $\gls{delta} : \Omega^p(\N) \to \Omega^{p-1}(\N)$ is defined by
	\begin{align}
	\delta \coloneqq (-1)^{s+1+N(p-1)}\, {*{d*}} \formspace.
	\end{align}
	From the defining properties of $d$ and $*$ it follows $\delta^2 =0$.
\end{definition}
Here, $s$ again denotes the number of negative eigenvalues of the metric $k$ of $\N$. In accordance with our nomenclature, we call a $p$-form $\omega$ co-exact if there exists a $\alpha \in \Omega^{p+1}(\N)$ such that $\omega = \delta \alpha$ and co-closed if $\delta \omega = 0$. Accordingly, the spaces of co-closed and co-exact $p$-forms are denoted by \gls{omegapdelta} and \gls{deltaomegap} respectively.\par
Using the exterior and interior derivative we define the partial differential operator:
\begin{definition}[D'Alembert Operator]
	The d'Alembert (or Laplace - de Rham) operator $\gls{dalembert}: \Omega^p(\N) \to \Omega^{p}(\N)$ is defined by
	\begin{align}
	\square \coloneqq \delta d +d \delta \formspace.
	\end{align}
\end{definition}
By definition of the exterior and interior derivative, it is easy to show that $\square$ commutes with both $d$ and $\delta$:
\begin{align}
\square d &= (\delta d + d \delta )d \notag \\
&= d \delta d \notag \\
&= d (\delta d + d \delta) \formspace,
\end{align}
and analogously for $\delta$.
The d'Alembert operator, and its generalization to $(\square + m^2)$ for some constant $m > 0$, are important examples for a normally hyperbolic differential operators (see Section \ref{sec:global_hyperbolicity}) and we may therefore sometimes just refer to them as \emph{wave operators}.\par
The sign convention in the definition of the exterior derivative is chosen such that on any Lorentzian or Riemannian manifold the interior derivative is formally adjoint to the exterior derivative, that is,  for $A \in \Omega^{p}(\N)$ and $B \in \Omega^{p+1}(\N)$ it holds that
\begin{align}
\langle dA , B \rangle_{\N} = \langle A , \delta B \rangle_\N \formspace,
\end{align}
which leads to a representation in local coordinates of the Manifold given by:
\begin{align}
(\delta A)_{\mu_2\dots\mu_p} = - \nabla^{\mu_1}A_{\mu_1\dots\mu_p} \formspace.
\end{align}
To see that this is consistent, let $A \in \Omega^{p-1}(\N)$ and $B \in \Omega^{p}(\N)$ such that their supports have compact intersection.
We obtain, using Stoke's Theorem \ref{thm:stokes}:
\begin{align}
0 &= \int \limits_{\partial \N} i^* (A \wedge *B) \notag\\
&= \int \limits_{\N} d(A \wedge *B)  \notag\\
&= \int \limits_{\N} dA \wedge *B + (-1)^{p-1} A \wedge d{*B} \notag\\
&= \int \limits_{\N} dA \wedge *B + (-1)^{p-1} A \wedge *{*^{-1}}\underbrace{d{*B}}_{\textrm{is a } (N-p+1) \textrm{ form.}} \notag\\
&= \int \limits_{\N} dA \wedge *B + (-1)^{p-1}(-1)^{s+(N-p+1)(N-N+p-1)} A \wedge *{*d{*B}} \notag\\
&= \int \limits_{\N} dA \wedge *B + (-1)^{p+(1-p)(p-1)} A \wedge *\delta B \formspace.
\end{align}
It can easily be proven by induction that $\big(p+(1-p)(p-1)\big)$ is odd for any $p \in \IN$, which yields the result
\begin{align}
\langle dA , B \rangle_{\N} = \langle A , \delta B \rangle_\N \formspace.
\end{align}
The definitions stated above thus fulfill the requirement of formal adjointness of the exterior and interior derivate on an arbitrary Lorentzian or Riemannian manifold $\N$.
In local coordinates we use a partial integration to obtain
\begin{align}
\langle dA , B \rangle_\N &= \int \limits_{\N} dA \wedge * B \notag\\
%&= \int \limits_{\N} \frac{1}{p!} (dA)^{\alpha_1\dots\alpha_p}\,B_{\alpha_1 \dots \alpha_p} \, \dvolk \notag\\
&= \int \limits_{\N}  \frac{p}{p!} \nabla^{[\alpha_1}A^{\alpha_2\dots\alpha_p]}\,B_{\alpha_1 \dots \alpha_p} \, \dvolk \notag\\
&= \int \limits_{\N}  \frac{1}{(p-1)!} \nabla^{\alpha_1}A^{\alpha_2\dots\alpha_p}\,B_{\alpha_1 \dots \alpha_p} \, \dvolk \notag\\
&= - \int \limits_{\N}  \frac{1}{(p-1)!} A^{\alpha_2\dots\alpha_p}\, \nabla^{\alpha_1}B_{\alpha_1 \dots \alpha_p} \, \dvolk \notag\\
&= \langle A, \delta B \rangle_\N \formspace,
\end{align}
which yields
\begin{align}
-\nabla^{\alpha_1}B_{\alpha_1 \dots \alpha p} = (\delta B)_{\alpha_2 \dots \alpha_p}\formspace.
\end{align}
On the four dimensional spacetime $(\M,g)$ the definitions of the Hodge star operator and the interior derivative simplify, such that
\begin{align}
*_{(\M)}*_{(\M)} &= (-1)^{p+1} \mathbbm{1} \\
\delta_{(\M)} &= *_{(\M)}{d_{(\M)}*_{(\M)}} \formspace ,
\end{align}
holds on the spacetime $(\M,g)$ and
\begin{align}
*_{(\Sigma)}*_{(\Sigma)} &= \mathbbm{1} \\
\delta_{(\Sigma)} &= (-1)^p *_{(\Sigma)}{d_{(\Sigma)}*_{(\Sigma)}}
\end{align}
holds on  $(\Sigma,h)$. In the following we will drop the subscript ${(\M)}$, since we will perform all the calculations on a four dimensional spacetime, except when explicitly noted (for example with a subscript $(\Sigma)$).
%
%
%
%
%
%
%
%
%%%%%%
%%CATEGORY THEORY
%%%%%%
\subsection{Category theory}\label{sec:cat-theory}
The description of Quantum Field Theory on Curved Spacetimes (QFTCS) in the framework of \name{Brunetti}, \name{Fredenhagen} and \name{Verch} \cite{Brunetti_Fredenhagen_Verch} is based on category theory. In this thesis, we will not go into detail on those categorical aspects, however we will need some basic definitions to formulate the theory rigorously, that is namely the notion of a category and that of covariant functors, since, in the used framework, the generally covariant QFTCS is a functor.\par
Here, we present definitions given in \cite[Appendix A.1]{baer_ginoux_pfaeffle} and refer to the appropriate literature for details. We define:
\begin{definition}[Category]
	A \emph{category} $\mathsf{Cat}$ consists of the following:
	\begin{enumerate}
		\item a class $\mathsf{Obj}_\mathsf{Cat}$ whose members are called \emph{objects},
		\item a set $\mathsf{Mor}_\mathsf{Cat}(A,B)$, for any two objects $A,B \in \mathsf{Obj}_\mathsf{Cat}$, whose elements are called \emph{morphisms},
		\item for any three objects $A,B,C \in \mathsf{Obj}_\mathsf{Cat}$ there is a map
		\begin{align}
\mathsf{Mor}_\mathsf{Cat}(B,C) \times \mathsf{Mor}_\mathsf{Cat}(A,B) &\to \mathsf{Mor}_\mathsf{Cat}(A,C) \notag\\
(\psi,\phi) &\mapsto \psi \comp \phi
		\end{align}
		called the composition of morphisms subject to the relations:\vspace{4mm}
		\begin{enumerate}[label=(\arabic*)]
			\item for non equal pairs $(A,B)$, $(A',B')$ of objects, the sets $\mathsf{Mor}_\mathsf{Cat}(A,B)$ and $\mathsf{Mor}_\mathsf{Cat}(A',B')$ are disjoint,
			\item for every object $A$ there exists a morphism $\text{id}_A \in \mathsf{Mor}_\mathsf{Cat}(A,A)$ such that it holds for all objects $B$, morphisms $\psi \in \mathsf{Mor}_\mathsf{Cat}(B,A)$ and $\phi \in \mathsf{Mor}_\mathsf{Cat}(A,B)$
			\begin{align}
				\text{id}_A \comp \psi &= \psi \quad \text{and}\\
				\phi \comp \text{id}_A &= \phi \quad,
			\end{align}
			\item the composition law is associative, that is for an objects $A,B,C,D$ and any morphisms $\psi \in \mathsf{Mor}_\mathsf{Cat}(A,B)$, $\phi \in \mathsf{Mor}_\mathsf{Cat}(B,C)$ and $\chi \in \mathsf{Mor}_\mathsf{Cat}(C,D)$ it holds
			\begin{align}
				(\chi \comp \phi) \comp \psi = \chi \comp (\phi \comp \psi) \formspace.
			\end{align}
		\end{enumerate}
	\end{enumerate}
\end{definition}
%
%
%
\begin{definition}[Functor]
	Let $\mathsf{Cat1}$ and $\mathsf{Cat2}$ be categories. A \emph{covariant functor} $\mathscr{A}: \mathsf{Cat1} \to \mathsf{Cat2}$ consists of the map $\mathscr{A} : \mathsf{Obj}_\mathsf{Cat1} \to \mathsf{Obj}_\mathsf{Cat2}$ and maps $\mathscr{A}: \mathsf{Mor}_\mathsf{Cat1}(A,B) \to \mathsf{Mor}_\mathsf{Cat2}\big(\mathscr{A}(A),\mathscr{A}(B)\big)$ for any two objects $A,B \in \mathsf{Obj}_\mathsf{Cat1}$ such that
	\begin{enumerate}
		\item {the composition is preserved, that is for all objects $A,B,C \in \mathsf{Obj}_\mathsf{Cat1}$ and for any morphisms $\psi \in \mathsf{Mor}_\mathsf{Cat1}(A,B)$ and $\phi \in \mathsf{Mor}_\mathsf{Cat1}(B,C)$ it holds
		\begin{align}
			\mathscr{A}(\phi \comp \psi) = \mathscr{A}(\phi) \comp \mathscr{A}(\psi) \formspace,
		\end{align}}
		\item{
			$\mathscr{A}$ maps identities to identities, that is for any object $A \in \mathsf{Obj}_\mathsf{Cat1}$ it holds
			\begin{align}
				\mathscr{A}(\text{id}_\mathsf{A}) = \text{id}_{\mathscr{A}(A)} \formspace.
			\end{align}
			}
	\end{enumerate}
\end{definition}
%
%
%
%
%
%
%
%
%
%
%
%
%%%%%%
%%SIGN CONVENTIONS
%%%%%%
%
%
\subsection{Sign conventions}\label{sec:sign_conventions}
At certain points throughout this chapter we have had a freedom of choice regarding the signs of some entities, in particular the sign of the signature of the Lorentzian metric $g$ and that of the interior derivative $\delta$. Though at this stage the choice can be made arbitrarily, we want to make it in a way that in the end allows us to make certain physical interpretations on some parameters. More precisely, we want to interpret the parameter $m$ of the Klein-Gordon equation\footnote{or its generalization on $p$-forms} $(\square + m^2) f = 0$ for a zero-form $f \in \Omega^0(\M)$ as a mass in the physical sense. With the chosen sign convention for $\delta$ we find, using ${\delta}f = 0$:
\begin{align}
	\square f
	&= (\delta d + d \delta) f \notag\\
	&= \delta d f \notag\\
	&= - \nabla^\mu \nabla_\mu f \formspace.
\end{align}
In the following heuristic (local) argument we see
\begin{align}
	\square + m^2
	&= -\nabla^\mu \nabla_\mu + m^2 \notag\\
	&\sim \partial_t^2 + \sum_i \partial_i^2 + m^2\notag\\
	&\sim -E^2 + \abs{\vector{p}}^2 + m^2
\end{align}
which yields the correct relativistic relation of energy, momentum and mass according to $E^2 = \abs{\vector{p}}^2 + m^2$.
A similar calculation holds for the Klein-Gordon operator generalized to act on one-forms. If we had found a ``wrong'' relation between energy, momentum and mass, we would have had to adapt the chosen signs. Usually one chooses the sign of the metric and the interior derivative such that they are in some sense mathematically convenient (although one might disagree with another one's choice). We have made the choice of the metric, such that the Cauchy surfaces become Riemannian rather that ``anti-Riemannian'' (with an all minus signature), which seems more natural to some. Also, a lot of the used references on spacetime geometry (in particular the book by \name{Wald} \cite{wald_GR}) use this sign convention, which makes the application of certain formulas easier. As mentioned, the sign of the interior derivative was chosen such that it is formally adjoint to the exterior derivative (with respect the specified inner product) on all Lorentzian and Riemannian manifolds. It seemed convenient for the actual calculations to fix the sign regardless of the signature of the metric of the underlying manifold. One could equivalently have fixed the opposite sign, yielding the two derivatives to be skew-adjoint, which is also done in the literature. However, in the end, one has one freedom left to make the energy-momentum-mass relation work: that is the sign in front of the mass in the Klein-Gordon equation and all other wave equations accordingly. Hence, one regularly also finds the Klein-Gordon equation to be defined with a flipped sign of the mass term. But for our case, we want the mass $m$ in any wave equation to appear with a positive sign.
%
%


\section{An Illustrative Example: Learning the Mean of a Distribution}\label{sec:illustrative}

We consider the following very simple WGAN example: The data are generated by a multivariate normal distribution, i.e. $Q \triangleq N(v, I)$ for some $v\in \mathbb{R}^d$. The goal is for the generator to learn the unknown parameter $v$. In  Appendix \ref{sec:covariance} we also consider a more complex example where the generator is trying to learn a co-variance matrix. 

We consider a WGAN, where the discriminator is a linear function and the generator is a simple additive displacement of the input noise $z$, which is drawn from $F\triangleq N(0, I)$, i.e:
\begin{equation}
\begin{aligned}
D_w(x) =~& \langle w, x\rangle\\
G_{\theta}(z) =~& z + \theta
\end{aligned}
\end{equation}
The goal of the generator is to figure out the true distribution, i.e. to converge to $\theta = v$. The WGAN loss then takes the simple form:
\begin{equation}
L(\theta, w) =  \mathbb{E}_{x\sim N(v, I)}\left[ \langle w, x \rangle \right] - \mathbb{E}_{z\sim N(0,I)}\left[\langle w, z + \theta \rangle\right] 
\end{equation}
We first consider the case where we optimize the true expectations above rather than assuming that we only get samples of $x$ and samples of $z$. Due to linearity of expectation, the expected zero-sum game takes the form:
\begin{equation}
\inf_{\theta} \sup_{w}~\langle w, v-\theta \rangle
\end{equation}
We see here that the unique equilibrium of the above game is for the generator to choose $\theta=v$ and for the discriminator to choose $w = 0$. For this simple zero sum game, we have $\nabla_{w, t}=v-\theta_t$ and $\nabla_{\theta, t}=-w_t$. Hence, the GD dynamics take the form:
\begin{equation}
\begin{aligned}
w_{t+1} =& w_{t} + \eta (v - \theta_{t})\\
\theta_{t+1} =& \theta_{t} + \eta w_t  
\end{aligned}\tag{GD Dynamics for Learning Means}
\end{equation}
while the OMD dynamics take the form:
\begin{equation}
\begin{aligned}
w_{t+1} =& w_{t} + 2\eta\cdot (v - \theta_{t}) - \eta \cdot (v-\theta_{t-1})\\
\theta_{t+1} =& \theta_{t} + 2\eta\cdot w_t - \eta\cdot  w_{t-1} 
\end{aligned}\tag{OMD Dynamics for Learning Means}
\end{equation}

We simulated simultaneous training in this zero-sum game under the GD and under OMD dynamics and we find that GD dynamics always lead to a limit cycle irrespective of the step size or other modifications. In Figure \ref{fig:gd} we present the behavior of the GD vs OMD dynamics in this game for $v = (3, 4)$. We see that even though GD dynamics leads to a limit cycle (whose average does indeed equal to the true vector), the OMD dynamics converge to $v$ in terms of the last iterate. In Figure \ref{fig:sampling} we see that the stability of OMD even carries over to the case of Stochastic Gradients, as long as the batch size is of decent size. 

\begin{figure}[htpb]
    \centering
    \begin{subfigure}[b]{.49\textwidth}
        \centering
        \includegraphics[height=.7in]{sgd_example.png}
        \caption{GD dynamics.}
    \end{subfigure}
    ~ 
    \begin{subfigure}[b]{.49\textwidth}
        \centering
        \includegraphics[height=.7in]{omd_example.png}
        \caption{OMD dynamics.}
    \end{subfigure}
    \caption{Training GAN with GD converges to a limit cycle that oscilates around the equilibrium (we applied weight-clipping at $10$ for the discriminator). On the contrary training with OMD converges to equilibrium in terms of last-iterate convergence.}\label{fig:gd}
\end{figure}

In the appendix we also portray the behavior of the GD dynamics even when we add gradient penalty \citep{Gulrajani2017} to the game loss (instead of weight clipping), adding Nesterov momentum to the GD update rule \citep{Nesterov} or when we train the discriminator multiple times in between a train iteration of the generator. We see that even though these modifications do improve the stability of the GD dynamics, in the sense that they narrow the band of the limit cycle, they still lead to a non-vanishing limit cycle, unlike OMD. 

\begin{figure}[htpb]
    \centering
    \begin{subfigure}[b]{.49\textwidth}
        \centering
        \includegraphics[height=.7in]{stochastic_omd_batch_50.png}
        \caption{Stochastic OMD dynamics with mini-batch of $50$.}
    \end{subfigure}
    ~ 
    \begin{subfigure}[b]{.49\textwidth}
        \centering
        \includegraphics[height=.7in]{stochastic_omd_batch_200.png}
        \caption{Stochastic OMD dynamics with mini-batch of $200$.}
    \end{subfigure}
    \caption{Robustness of last-iterate convergence of OMD to stochastic gradients.}\label{fig:sampling}
\end{figure}

In the next section, we will in fact prove formally that for a large class of zero-sum games including the one presented in this section, OMD dynamics converge to  equilibrium in the sense of last-iterate convergence, as opposed to average-iterate convergence.

\section{Last-Iterate Convergence of Optimistic Adversarial Training}
\label{sec:main:proof OMD converges}
In this section, we show that Optimistic Mirror Descent  
exhibits final-iterate, rather than only average-iterate convergence to min-max solutions for
bilinear functions. 
%In particular, we show that the $\ell_2$ norms of the gradients used by the dynamics shrinks in time.
More precisely, we consider the problem $\min_x \max_y x^T A y$, for some matrix $A$, where $x$ and $y$ are unconstrained. In Appendix \ref{sec:appendix:last-iterate}, we also show that our convergence result appropriately extends to the general case, where the bi-linear game
also contains terms that are linear in the players' individual strategies, i.e.~games of the form:
\begin{equation}
    \inf_{x} \sup_{y} \left(x^TAy + b^Tx + c^Ty + d\right). \label{eq:inf sup problem general}
\end{equation}

In the simpler $\min_{x}\max_{y} x^T Ay$ problem, Optimistic Mirror Descent takes the following form, for all $t \ge 1$:
\begin{align}
    x_{t} &= \xto \label{eq:OGD bilinear x repeat}\\
    y_{t} &= \yto  \label{eq:OGD bilinear y repeat}
\end{align}
\noindent {\em Initialization:} For the above iteration to be meaningful we need to specify $x_0,x_{-1},y_0,y_{-1}$. We choose any $x_0 \in
\mathcal{R}(A)$, and $y_0\in\mathcal{R}(A^T)$, and set $x_{-1}=2x_0$ and $y_{-1}=2y_{0}$, where ${\cal R}(\cdot)$ represents the column space of $A$. In particular, our initialization means that the first step taken by the dynamics gives $x_1=x_0$ and $y_1=y_0$.

\smallskip We will analyze Optimistic Mirror Descent under the assumption $\lambda_{\infty} \le 1$, where $\lambda_{\infty}=\max\{||A||,||A^T||\}$ and $||\cdot||$ denotes spectral norm of matrices. We can always enforce that $\lambda_{\infty} \le 1$ by appropriately scaling $A$. Scaling $A$ by some positive factor clearly does not change the min-max solutions $(x^*,y^*)$, only scales the optimal value $x^{*T}Ay^*$ by the same factor.

We remark that the set of equilibrium solutions of this minimax problem  are pairs $(x,y)$ such that $x$ is in the null space of $A^T$ and $y$ is in the null space of $A$. 
%In particular, finding a solution to $\min_x \max_y x^T A y$ is a trivial problem. So 
In this section we rigorously show that Optimistic Mirror Descent converges to the set of such min-max solutions. This is interesting in light of the fact that Gradient Descent actually diverges, even in the special case where $A$ is the identity matrix, as per the following proposition whose proof is provided in Appendix~\ref{appendix:omitted proofs}.

\begin{proposition} \label{prop:gradient descent unstable}
Gradient descent applied to the problem $\min_x \max_y x^T y$ diverges starting from any initialization $x_0, y_0$ such that $x_0,y_0 \neq 0$.
\end{proposition}

Next, we state our main result of this section, whose proof can be found in Appendix~\ref{sec:appendix:last-iterate}, where we also state its appropriate generalization to the general case~\eqref{eq:inf sup problem general}.

\begin{theorem}[Last Iterate Convergence of OMD]\label{thm:convergence of OGD-main}
Consider the dynamics of Eq.~\eqref{eq:OGD bilinear x repeat} and~\eqref{eq:OGD bilinear y repeat} and any initialization ${1 \over 2}x_{-1}=x_0 \in
\mathcal{R}(A)$, and ${1 \over 2}y_{-1}=y_0\in\mathcal{R}(A^T)$. Let also
$$\gamma = \vsedit{\left\|\(AA^T\)^{+}\right\|},$$
%\max\(\left|\left|\(AA^T\)^{+}\right|\right|,
%	    \left|\left|\(A^TA\)^{+}\right|\right|\),$$
where for a matrix $X$ we denote by	$X^{+}$ its generalized inverse and by $||X||$ its spectral norm. Suppose that \vsedit{$\|A\|\equiv \lambda_{\infty}\le 1$} 
%$\max\{||A||,||A^T||\}\equiv \lambda_{\infty}\le 1$ 
and that $\eta$ is a small enough constant satisfying $\eta <1/(3\gamma^2)$. Letting $\Delta_t = \normlt{A^Tx_t} + \normlt{Ay_t}$, the OMD dynamics satisfy the following:
\begin{align*}
&\Delta_1=\Delta_0 \ge {1 \over (1+\eta)^2} \Delta_2\\
    \forall t\ge 3:~~&\Delta_t \leq \left(1-{\eta^2 \over \gamma^2}\right)\Delta_{t-1} + 16\eta^3 \Delta_0. 
\end{align*}
In particular, $\Delta_t \rightarrow O(\eta \gamma^2 \Delta_0)$, as $t \rightarrow +\infty$, and for large enough $t$, the last iterate of OMD is within $O(\sqrt{\eta} \cdot \gamma \sqrt{\Delta_0})$ distance from the space of equilibrium points of the game, where $\sqrt{\Delta_0}$ is the distance of the initial point $(x_0,y_0)$ from the equilibrium space, and where both distances are taken with respect to the norm $\sqrt{x^T A A^T x + y^T A^T A y}$.
\end{theorem}





\section{Experimental Results for Generating DNA Sequences}
\section{Experiments}

\label{sec:experiments}
In this section we compare the tiered vector to some widely used C++ standard library containers. 
We also compare different variants of the tiered vector. 
We consider how the different representations of the data
structure listed in Section~\ref{sec:implementation}, 
and also how the height of tree and the capacity of the leaves affects the running time.
The following describes the test setup:

\subparagraph{Environment}

All experiments have been performed on a Intel Core i7-4770 CPU @ 3.40GHz with
32 GB RAM. The code has been compiled with GNU GCC version 5.4.0 with flags
``-O3''. The reported times are an average over 10 test runs.
 
 \subparagraph{Procedure}
%have been added to the data structure in
 
In all tests $10^8$ 32-bit integers 
are inserted in the data structure as a preliminary step
to simulate that it has already been
used\footnote{In order to minimize the overall running time of the experiments,
the elements were not added randomly, but we show this does not give our data
structure any benefits}.
For all the access and successor operations $10^9$ elements have been accessed
and the time reported is the average time per element.
For range access, 10.000 consecutive elements are accessed.
For insertion/deletion $10^6$ elements
have been (semi-)randomly\footnote{In order to not impact timing, a simple
access pattern has been used instead of a normal pseudo-random generator.}
added/deleted, though in the case of ``vector'' only 10.000 elements were
inserted/deleted to make the experiments terminate in reasonable time. 

\subsection{Comparison to C++ STL Data Structures}

In the following we have compared our best performing tiered vector (see the next sections) to the vector and
the multiset class from the C++ standard library.
The vector data structure directly supports the
operations of a dynamic array. The multiset class is implemented as a red-black
tree and is therefore interesting to compare with our data structure.
Unfortunately, multiset does not directly support the operations of a dynamic
array (in particular it has no notion of positions of elements). To simulate an
access operation we instead find the successor of an element in the multiset.
This requires a root-to-leaf traversal of the red-black tree, just as an access
operation in a dynamic array implemented as a red-black tree would. Insertion
is simulated as an insertion into the multiset, which again requires the same
computations as a dynamic array implemented as a red-black tree would.

Besides the random access, range access and insertion,
we have also tested the operations \textit{data dependent access},
insertion in the end, deletion, and \textit{successor} queries. In the
\textit{data dependent access} tests, the next index to lookup depends on the values of the prior
lookups. This ensures that the CPU cannot successfully pipeline
consecutive lookups, but must perform them in sequence. We test insertion in the end, since
this is a very common use case. Deletion is performed by deleting elements at
random positions. The $successor$ queries returns the successor of an element
and is not actually part of the
dynamic array problem, but is included since it is a commonly used operation on
a multiset in C++. It is simply implemented as a binary search over the elements in
both the vector and tiered vector tests where the elements are now inserted in sorted order. 

The results are summarized in Table~\ref{tab:test_comp} which shows that the vector performs slightly better than the tiered vector on all access and successor tests. As expected from the $\Theta(n)$ running time, it performs extremely poor on random insertion and deletion. For insertion in the end of the sequence, vector is also slightly faster than the tiered vector. The interesting part is that even though the tiered vector requires several extra memory lookups and computations, we have managed to get the running time down to less than the double of the vector for access, even less for data dependent access and only a few percent slowdown for range access. As discussed earlier,
this is most likely because the entire tree structure (without the elements)
fits within the CPU cache, and because the computations required has been minimized.

Comparing our tiered vector to multiset, we would expect access operations to be
faster since they run in $O(1)$ time compared to $O(\log n)$. On the other
hand, we would expect insertion/deletion to be significantly slower since it
runs in $O(n^{1/l})$ time compared to $O(\log n)$ (where $l = 4$ in these tests). We
see our expectations hold for the access operations where the tiered vector is faster by more than an order of magnitude.
In random insertions however,  the tiered vector is only $8\%$ slower -- even when operating on 100.000.000 elements. Both the tiered
vector and set requires $O(\log n)$ time for the successor operation. In our
experiments the tiered vector is 3 times faster for the successor operation.

Finally, we see that the memory usage of vector and tiered vector is almost identical.
This is expected since in both cases the space usage is dominated by the space taken by the actual elements.
The multiset uses more than 10 times as much space, so this is also a considerable drawback of the red-black tree behind this structure. 

To sum up, the tiered vectors performs better than multiset on all tests
but insertion, where it performs only slightly worse.

%\caption{Figures (a) through (e) show the performance of \textit{Tiered Arrays} (\protect\purple) compared
%to the \textit{set} (\protect\green) and \textit{vector} (\protect\blue) data structures from the C++ standard library.} \label{fig:animals}
\begin{table}
	\centering
	\begin{tabular}{|l|r|r|r|r|r|}
		\hline
		& \multicolumn{1}{l|}{\textit{tiered vector}} & \multicolumn{1}{l|}{\textit{set}} & \multicolumn{1}{l|}{\textit{set / tiered}} & \multicolumn{1}{l|}{\textit{vector}} & \multicolumn{1}{l|}{\textit{vector / tiered}} \\ \hline
		access     & $34.07$ ns                                  & $1432.05$ ns                      & 42.03                                      & $21.63$ ns                           & 0.63                                          \\ \hline
		dd-access    & $99.09$ ns                                  & $1436.67$ ns                      & 14.50                                      & $79.37$ ns                           & 0.80                                          \\ \hline
		range access   & $0.24$ ns                                   & $13.02$ ns                        & 53.53                                      & $0.23$ ns                            & 0.93                                          \\ \hline
		insert   & $1.79$ $\mu$s                               & $1.65$ $\mu$s                     & 0.92                                       & $21675.49$ $\mu$s                     & 12082.33                                      \\ \hline
		insertion in end     & $7.28$ ns                               & $242.90$ ns                     & 33.38                                       & $2.93$ ns                     & 0.40                                      \\ \hline
		successor & $0.55$ $\mu$s                               & $1.53$ $\mu$s                     & 2.75                                       & $0.36$ $\mu$s                        & 0.65                                          \\ \hline
		delete     & $1.92$ $\mu$s                               & $1.78$ $\mu$s                     & 0.93                                       & $21295.25$ $\mu$s                     & 11070.04                                      \\ \hline
		memory     & $408$ MB                               & $4802$ MB                     & 11.77                                       & $405$ MB                    & 0.99                                      \\ \hline
	\end{tabular}
	\caption{The table summarizes the performance of the implicit tiered vector
		compared to the performance of multiset and vector from the C++ standard library.\
		dd-access refers to data dependent access.}
\label{tab:test_comp}
\end{table}


\definecolor{cpurple}{RGB}{131,24,197}
\definecolor{cgreen}{RGB}{70,156,118}
\definecolor{cblue}{RGB}{11,178,228}
\definecolor{cdblue}{RGB}{11,112,173}
\definecolor{corange}{RGB}{219,162,55}
\definecolor{cyellow}{RGB}{238,228,98}
\definecolor{cred}{RGB}{110,55,38}
\newcommand{\purple}{\raisebox{2pt}{\tikz{\draw[cpurple,solid,line width=1.9pt](0,0) -- (3mm,0);}}}
\newcommand{\green}{\raisebox{2pt}{\tikz{\draw[cgreen,solid,line width=1.9pt](0,0) -- (3mm,0);}}}
\newcommand{\blue}{\raisebox{2pt}{\tikz{\draw[cblue,solid,line width=1.9pt](0,0) -- (3mm,0);}}}
\newcommand{\dblue}{\raisebox{2pt}{\tikz{\draw[cdblue,solid,line width=1.9pt](0,0) -- (3mm,0);}}}
\newcommand{\orange}{\raisebox{2pt}{\tikz{\draw[corange,solid,line width=1.9pt](0,0) -- (3mm,0);}}}
\newcommand{\yellow}{\raisebox{2pt}{\tikz{\draw[cyellow,solid,line width=1.9pt](0,0) -- (3mm,0);}}}
\newcommand{\red}{\raisebox{2pt}{\tikz{\draw[cred,solid,line width=1.9pt](0,0) -- (3mm,0);}}}


\begin{figure}[ht]
	\centering
	\begin{subfigure}[b]{0.3\textwidth}
		\includegraphics[width=\textwidth]{layout_test_get}
		\caption{\textit{access}}
	\end{subfigure}
	\begin{subfigure}[b]{0.3\textwidth}
		\includegraphics[width=\textwidth]{layout_test_random}
		\caption{\textit{insert}}
	\end{subfigure}
        \caption{Figures (a) and (b) show the performance of the
            \textit{original} (\protect\purple), \textit{optimized original}
            (\protect\green), \textit{lazy} (\protect\blue) \textit{packed
            lazy} (\protect\orange),
            \textit{implicit} (\protect\yellow)
            and \textit{packed implicit} (\protect\dblue) layouts.}
\label{fig:test_representation}
\end{figure}
\subsection{Tiered Vector Variants}

In this test we compare the performance
of the implementations listed in Section~\ref{sec:implementation} to that 
or the original data structure as described in~\ref{thm:pointer}.

%\paragraph{Optimized Original}
\subparagraph*{Optimized Original}
By co-locating the child offset and child pointer, the two memory lookups are at
adjacent memory locations. Due to the cache lines in modern processors,
the second memory lookup will then often be answered directly by the fast
L1-cache.
As can be seen on Figure~\ref{fig:test_representation}, this small change in the memory layout results in a significant improvement in performance for both access and insertion. In the latter case, the running time is more than halved.

%\paragraph{Lazy and Packed Lazy}
\subparagraph*{Lazy and Packed Lazy}

Figure~\ref{fig:test_representation} shows
how the fewer memory probes required by the
\textit{lazy} implementation in comparison to the \text{original}
and \text{optimized original} results in better performance.
Packing the offset and pointer in the leaves results in even better performance
for both access and insertion even though it requires a few extra instructions
to do the actual packing and unpacking.

%\paragraph{Implicit}
\subparagraph*{Implicit}
From Figure~\ref{fig:test_representation}, we see the implicit
data structure is the fastest.
This is as expected because it requires fewer
memory accesses than the other structures except
for the packed lazy which instead has a slight
computational overhead due to the packing and unpacking.

As shown in Theorem~\ref{thm:implicit} the implicit data structure has a
bigger memory overhead than the lazy data structure.
Therefore the packed lazy representation might be beneficial in some
settings.

%\paragraph{Packed Implicit}
\subparagraph*{Packed Implicit}

Packing the offsets array could lead to 
better cache performance due to the smaller memory footprint and therefore
yield better overall performance.
As can be seen on Figure~\ref{fig:test_representation},
the smaller memory footprint
did not improve the performance in practice.
The simple reason for this,
is that the strategy we used for packing the offsets required
extra computation. This clearly dominated the possible gain from the
hypothesized better cache performance. We tried a few strategies to minimize
the extra computations needed at the expense of slightly worse memory usage,
but none of these led to better results than when not packing the offsets at
all.

\subsection{Width Experiments}

\begin{figure}
	\centering
	\begin{subfigure}[b]{0.3\textwidth}
		\includegraphics[width=\textwidth]{width_test_get}
		\caption{\textit{access}}
	\end{subfigure}
	\begin{subfigure}[b]{0.3\textwidth}
		\includegraphics[width=\textwidth]{width_test_sum}
		\caption{\textit{range access}}
	\end{subfigure}
	\begin{subfigure}[b]{0.3\textwidth}
		\includegraphics[width=\textwidth]{width_test_random}
		\caption{\textit{insert}}
	\end{subfigure}
	\caption{Figures (a), (b) and (c) show the performance of the \textit{implicit} (\protect\purple) and
		the \textit{optimized original} tiered vector (\protect\green) for different tree widths.}
\label{fig:test_width}
\end{figure}

This experiment was performed to determine the best capacity ratio between the leaf nodes and the internal nodes.
The six different width configurations we have tested are: 32-32-32-4096, 32-32-64-2048, 32-64-64-1024, 64-64-64-512, 64-64-128-256, and 64-128-128-128.
All configurations have a constant height 4 and a capacity of approximately 130 mio.

We expect the performance of access operations to remain unchanged, since the
amount of work required only depends on the height of the tree,
and not the widths. We expect range access to perform better when the leaf size
is increased, since more elements will be located in consecutive memory
locations. For $insertion$ there is not a clearly expected behavior as the time
used to physically move elements in a leaf will increase with leaf size, but
then less operations on the internal nodes of the tree has to be performed.

On Figure~\ref{fig:test_width} we see access times are actually decreasing
slightly when leaves get bigger. This was not expected, but is most likely
due to small changes in the memory layout that results in slightly better cache
performance. The same is the case for range access, but this was expected. For
insertion, we see there is a tipping point. For our particular instance, the
best performance is achieved when the leaves have size 512.

%Based on this, we have performed the remaining tests with the 64-64-64-512 configuration (unless otherwise specified).

\subsection{Height Experiments}

\begin{figure}
	\centering
	\begin{subfigure}[b]{0.3\textwidth}
		\includegraphics[width=\textwidth]{height_get}
		\caption{\textit{access(i)}}
	\end{subfigure}
	\begin{subfigure}[b]{0.3\textwidth}
		\includegraphics[width=\textwidth]{height_sum}
		\caption{\textit{access(i, m)}}
	\end{subfigure}
	\begin{subfigure}[b]{0.3\textwidth}
		\includegraphics[width=\textwidth]{height_random}
		\caption{\textit{insert}}
	\end{subfigure}
	\caption{Figures (a),(b) and (c) show the performance of the \textit{implicit} (\protect\purple) and
		the \textit{optimized original} tiered vector (\protect\green) for different tree heights.}
\label{fig:test_height}
\end{figure}

In these tests we have studied how different heights affect the performance of
access and insertion operations. We have tested the configurations 8196-16384,
512-512-512, 64-64-64-512, 16-16-32-32-512, 8-8-16-16-16-512. All resulting in
the same capacity, but with heights in the range 2-6.

We expect the access operations to perform better for lower trees, since
the number of operations that must be performed is linear in the height. On the
other hand we expect insertion to perform significantly better with higher
trees, since its running time is $O(n^{1/l})$ where $l$ is the height plus one. 

On Figure~\ref{fig:test_height} we see the results follow our expectations. However, the access operations only perform slightly worse on higher trees.
This is most likely because all internal nodes fit within the L3-cache. Therefore the running time is dominated by the lookup of the element itself.
(It is highly unlikely that the element requested by an access 
to a random position would be among the small fraction of elements that
fit in the L3-cache).

Regarding insertion, we see significant improvements up until a height of 4. After that, increasing the height does not change the running time noticeably. This is most likely due to the hidden constant in $O(n^{1/l})$ increasing rapidly with the height.



\subsection{Configuration Experiments}

\begin{figure}
    \centering
    \begin{subfigure}[b]{0.3\textwidth}
        \includegraphics[width=\textwidth]{small_get}
        \caption{\textit{access}}
    \end{subfigure}
    \begin{subfigure}[b]{0.3\textwidth}
        \includegraphics[width=\textwidth]{small_sum}
        \caption{\textit{range access}}
    \end{subfigure}
    \begin{subfigure}[b]{0.3\textwidth}
        \includegraphics[width=\textwidth]{small_random}
        \caption{\textit{insert(i,x)}}
    \end{subfigure}
    \caption{Figures (a) and (b) show the performance of the
    \textit{base} (\protect\purple),
    \textit{rotated} (\protect\green), 
    \textit{non-aligned sizes} (\protect\blue),
    \textit{non-templated} (\protect\orange)
    layouts.}
\label{fig:test_minor}
\end{figure}

In these experiments, we test a few hypotheses about how different changes
impact the running time. The results are shown on
Figure~\ref{fig:test_minor}, the leftmost result (base) is
the implicit 64-64-64-512 configuration of the tiered vector 
to which we compare our hypotheses.
%our final and best

\textit{Rotated}: 
As already mentioned, the insertions performed as a
preliminary step to the tests are not done at random positions.
This means that all offsets are zero when our real operations
start. The purpose of this test is the ensure that
there are no significant performance gains in starting
from such a configuration which could otherwise
lead to misleading results.
To this end, we have randomized all
offsets (in a way such that the data structure is still valid, but the
order of elements change) after doing the preliminary insertions
but before timing the operations. As can be seen on
Figure~\ref{fig:test_minor}, the difference between this and the normal
procedure is insignificant, thus we find our approach gives a fair picture.


\textit{Non-Aligned Sizes}: In all our previous tests, we have ensured all
nodes had an out-degree that was a power of 2. This was chosen in order to let the
compiler simplify some calculations, i.e.\ replacing multiplication/division
instructions by shift/and instructions. As Figure~\ref{fig:test_minor} shows,
using sizes that are not powers of 2 results in significantly worse performance.
Besides showing that powers of 2 should always be used, this also indicates that not only
the number of memory accesses during an operation is critical for our
performance, but also the amount of computation we make.

\textit{Non-Templated}
The non-templated results 
in Figure~\ref{fig:test_representation} the
show that the change to templated recursion
has had a major impact on the running time. It should be noted that some
improvements have not been implemented in the non-templated version,
but it gives a good indication that this has been quite useful.

\vspace{-.1in}
\section{Generating Images from CIFAR10 with Optimistic Adam}\label{sec:cifar10}
\vspace{-.1in}
In this section we applying optimistic WGAN training to generating images, after training on CIFAR10. Given the success of Adam on training image WGANs we will use an optimistic version of the Adam algorithm, rather than vanilla OMD. We denote the latter by \emph{Optimistic Adam}. Optimistic Adam could be of independent interest even beyond training WGANs. We present Optimistic Adam for (G) but the analog is also used for training (D).
\begin{algorithm}[h]
\begin{algorithmic}
\State Parameters: stepsize $\eta$, exponential decay rates for moment estimates $\beta_1, \beta_2\in [0,1)$, stochastic loss as a function of weights $\ell_t(\theta)$, initial parameters $\theta_0$
%\State Initialize parameters to $\theta_0$
\For{each iteration $t\in \{1,\ldots, T\}$}
\State Compute stochastic gradient: $\nabla_{\theta,t} = \nabla_{\theta} \ell_t(\theta)$
\State Update biased estimate of first moment: $m_t = \beta_1 m_{t-1} + (1-\beta_1) \cdot \nabla_{\theta,t}$
\State Update biased estimate of second moment: $v_t = \beta_2 v_{t-1} + (1-\beta_2) \cdot \nabla_{\theta,t}^2$
\State Compute bias corrected first moment: $\hat{m}_t = m_t/(1 - \beta_1^t)$
\State Compute bias corrected second moment: $\hat{v}_t = v_t/(1 - \beta_2^t)$
\State Perform \emph{optimistic gradient step}: $\theta_t = \theta_{t-1} - 2 \eta \cdot \frac{\hat{m}_t}{\sqrt{\hat{v}_t}+\epsilon} + \eta \frac{\hat{m}_{t-1}}{\sqrt{\hat{v}_{t-1}}+\epsilon}$ 
\EndFor
\State Return $\theta_T$
\end{algorithmic}
\caption{\emph{Optimistic ADAM}, proposed algorithm for training WGANs on images. }\label{alg:opt-adam}
\end{algorithm}
\vspace{-.1in}
We trained on CIFAR10 images with Optimistic Adam with the hyper-parameters matched to \cite{Gulrajani2017}, and we observe that it outperforms Adam in terms of inception score (see Figure \ref{fig:optimistic-Adam}), a standard metric of quality of WGANs \citep{Gulrajani2017, salimans2016improved}. In particular we see that optimistic Adam achieves high numbers of inception scores after very few epochs of training. We observe that for Optimistic Adam, training the discriminator once after one iteration of the generator training, which matches the intuition behind the use of optimism, outperforms the 1:5 generator-discriminator training scheme. We see that vanilla Adam performs poorly when the discriminator is trained only once in between iterations of the generator training. Moreover, even if we use vanilla Adam and train $5$ times (D) in between a training of (G), as proposed by \cite{arjovsky2017wasserstein}, then performance is again worse than Optimistic Adam with a 1:1 ratio of training. The same learning rate $0.0001$ and betas ($\beta_1=0.5, \beta_2=0.9$) as in Appendix B of \cite{Gulrajani2017}  were used for all the methods compared. We also matched other hyper-parameters such as gradient penalty coefficient $\lambda$ and batch size. For a larger sample of images see Appendix \ref{sec:appendix-cifar10}. 
\vspace{-.1in}
\begin{figure}[htpb]
    \centering
    \begin{subfigure}[b]{.67\textwidth}
        \centering
    		\includegraphics[height=1.7in]{optimAdam-eps-converted-to.pdf}        
		\caption{Inception score on CIFAR10, when training with Adam and Optimistic Adam. ``ratio1" means we performed $1$ iteration of training of (D) in between $1$ iteration of (G). Otherwise we performed $5$ iterations. We further test (averaging over 35 trials) the two top-performing optimizers, Adam (ratio 5) and Optimistic Adam with ratio 1, in Appendix~\ref{sec:appendix-errorbars}.}
    \end{subfigure}
    ~~
    \begin{subfigure}[b]{.3\textwidth}
        \centering
    		\includegraphics[height=1.7in]{optimAdam_v0_1e-04_ratio1_epoch93-eps-converted-to.pdf}
    	\caption{Sample of images from Generator of Epoch $94$, which had the highest inception score.}
    \end{subfigure}
    \caption{Comparison of Adam and Optimistic Adam on CIFAR10.}\label{fig:optimistic-Adam}
\end{figure}


\bibliographystyle{iclr2018_conference}
\bibliography{agt}

\newpage

\appendix

\onecolumn


% \tableofcontents{}

% \newpage

\section*{Supplementary Material}
\addcontentsline{toc}{section}{Supplementary Material}


Throughout this discussion, 
we will make frequently use 
of the following standard results
concerning the exponential concentration 
of random variables:

\begin{lemma}[Hoeffding's inequality for independent RVs~\citep{hoeffding1994probability}] Let $Z_1, Z_2, \ldots, Z_n$ be independent bounded random variables with $Z_i \in [a,b]$ for all $i$, then 
    \begin{align*}
        \prob\left( \frac{1}{n} \sum_{i=1}^n (Z_i - \Expo{Z_i}) \ge t \right) \le \exp{\left( -\frac{2nt^2}{(b-a)^2} \right) }
    \end{align*} 
    and 
    \begin{align*}
        \prob\left( \frac{1}{n} \sum_{i=1}^n (Z_i - \Expo{Z_i}) \le -t \right) \le \exp{\left( -\frac{2nt^2}{(b-a)^2} \right) }
    \end{align*} 
    for all $t \ge 0$. 
\end{lemma}

\begin{lemma}[Hoeffding's inequality for sampling with replacement~\citep{hoeffding1994probability}] \label{lem:hoeffding_sampling} Let $\calZ = (Z_1, Z_2, \ldots, Z_N)$ be a finite population of $N$ points with $Z_i \in [a.b]$ for all $i$. Let $X_1, X_2, \ldots X_n$ be a random sample drawn without replacement from $\calZ$. Then for all $t \ge 0$, we have 
    \begin{align*}
        \prob\left( \frac{1}{n} \sum_{i=1}^n (X_i - \mu ) \ge t \right) \le \exp{\left( -\frac{2nt^2}{(b-a)^2} \right) }
    \end{align*} 
    and 
    \begin{align*}
        \prob\left( \frac{1}{n} \sum_{i=1}^n (X_i - \mu ) \le -t \right) \le \exp{\left( -\frac{2nt^2}{(b-a)^2} \right) } \,,
    \end{align*} 
    where $\mu = \frac{1}{N} \sum_{i=1}^{N} Z_i$. 
\end{lemma}

We now discuss one condition that generalizes the exponential concentration to dependent random variables.
\begin{condition}[Bounded difference inequality] \label{cond:BDC} Let $\calZ$ be some set and $\phi: \calZ^n \to \Real$. We say that $\phi$ satisfies the bounded difference assumption if 
there exists $c_1, c_2, \ldots c_n \ge 0$ s.t. for all $i$, we have 
\begin{align*}
    \sup_{Z_1,Z_2, \ldots,Z_n, Z_i^\prime \in \calZ^{n+1} } \abs{\phi (Z_1, \ldots, Z_i, \ldots, Z_n ) - \phi (Z_1, \ldots, Z_i^\prime, \ldots, Z_n ) } \le c_i \,.
\end{align*} 
\end{condition}

\begin{lemma}[McDiarmid’s inequality~\citep{mcdiarmid1989}] \label{lem:McDiarmid} Let $Z_1, Z_2, \ldots, Z_n$ be independent random variables on set $\calZ$ and $\phi : \calZ^n \to \Real$ satisfy bounded difference inequality (\codref{cond:BDC}). Then for all $t>0$, we have 
    \begin{align*}
        \prob\left( \phi(Z_1, Z_2, \ldots, Z_n) - \Expo{\phi(Z_1, Z_2, \ldots, Z_n)} \ge t \right) \le \exp{\left( -\frac{2t^2}{\sum_{i=1}^n c_i^2} \right) } 
    \end{align*} 
    and 
    \begin{align*}
        \prob\left( \phi(Z_1, Z_2, \ldots, Z_n) - \Expo{\phi(Z_1, Z_2, \ldots, Z_n)} \le -t \right) \le \exp{\left( -\frac{2t^2}{\sum_{i=1}^n c_i^2} \right) } \,.
    \end{align*} 
\end{lemma}


\section{Proofs from \secref{sec:ERM_training}}\label{app:proof_erm}

\textbf{Additional notation {} {}} Let $m_1$ be the number of mislabeled points ($\wt S_M$) and $m_2$ be the number of correctly labeled points ($\wt S_C$). Note $m_1 + m_2 = m$. 


\subsection{Proof of \thmref{thm:error_ERM}}


\begin{proof}[Proof of \lemref{lem:fit_mislabeled}] 
    The main idea of our proof is to regard 
    the clean portion of the data 
    ($S \cup \wt S_C$) as fixed.   
    Then, there exists an (unknown) classifier $f^*$ 
    that minimizes the expected risk
    calculated on the (fixed) clean data
    and (random draws of) the mislabeled data $\wt S_M$. 
    % 
    % 
    Formally, 
    \begin{align}
    f^* \defeq \argmin_{f \in \calF} \error_{\widecheck {\calD}} (f) \,, \label{eq:modified_ERM}
    \end{align}
    where $$\widecheck \calD = \frac{n}{m+n} \calS + \frac{m_2}{m+n} \wt \calS_C  + \frac{m_1}{m+n}\calDm \,.$$ 
    Note here that $\widecheck \calD$ is a combination 
    of the \emph{empirical distribution} 
    over correctly labeled data $S \cup \wt S_C$
    and the (population) distribution 
    over mislabeled data $\calDm$.
    Recall that 
    \begin{align}
    \wh f \defeq \argmin_{f \in \calF} \error_{\calS \cup \wt S} (f) \,. \label{eq:orig_ERM}
    \end{align}
    % 
    % 
    Since, $\widehat f$ minimizes 0-1 error 
    on $S \cup \wt S$, using ERM optimality on \eqref{eq:orig_ERM},  
    we have 
    \begin{align}
        \error_{\calS \cup \wt \calS}(\widehat f) \le \error_{
            \calS \cup \wt \calS}(f^*) \,.    \label{eq:step1}
    \end{align}
    Moreover, since $f^*$ is independent of $\wt S_M$, using Hoeffding's bound,
    % \footnote{For a fully rigorous argument,
    % refer to the complete proof in App.~\ref{app:proof_erm}.} 
    we have with probability at least $1-\delta$ that
    \begin{align}
      \error_{\wt \calS_M}(f^*) \le \error_{ \calDm}(f^*) +  \sqrt{\frac{\log(1/\delta)}{2 m_1}} \,. \label{eq:step2} 
    \end{align}
    %$ 
    %for some constant $c_1\le 1/2$. 
    Finally, since $f^*$ is the optimal classifier on $\widecheck \calD$, 
    we have 
    \begin{align}
        \error_{\widecheck \calD}(f^*) \le \error_{\widecheck \calD}(\widehat f) \,. \label{eq:step3}
    \end{align}
    Now to relate \eqref{eq:step1} and \eqref{eq:step3}, we multiply \eqref{eq:step2} by $\frac{m_1}{m+n}$ and add $\frac{n}{m+n} \error_{\calS} (f)  + \frac{m_2}{m+n} \error_{\wt \calS_C} (f)$ both the sides. Hence, 
    we can rewrite \eqref{eq:step2} as follows: 
    \begin{align}
        \error_{\calS \cup \wt\calS}(f^*) \le \error_{ \widecheck \calD}(f^*) +  \frac{m_1}{m+n}\sqrt{\frac{\log(1/\delta)}{2 m_1}} \,. \label{eq:step4} 
    \end{align}
    Now we combine equations \eqref{eq:step1}, \eqref{eq:step4}, and \eqref{eq:step3}, to get 
    \begin{align}
        \error_{\calS \cup \wt \calS}(\wh f) \le \error_{\widecheck \calD}(\wh f) +  \frac{m_1}{m+n}\sqrt{\frac{\log(1/\delta)}{2 m_1}} \,, 
    \end{align}
    which implies 
    \begin{align}
        \error_{ \wt \calS_M}(\wh f) \le \error_{\calDm}(\wh f) + \sqrt{\frac{\log(1/\delta)}{2 m_1}} \,. \label{eq:lemma1_final}
    \end{align}
    Since $\wt S$ is obtained by randomly labeling an unlabeled dataset, we assume $2m_1 \approx m$ \footnote{Formally, with probability at least $1-\delta$, we have  $(m - 2m_1)\le \sqrt{m\log(1/\delta)/2}$.}. Moreover, using $\error_{\calDm} = 1 - \error_{\calD}$ we obtain the desired result.   
    % Combining the above steps and using the fact 
    % that $\error_\calD = 1- \error_{\calDm} $, 
    % we obtain the desired result.
\end{proof}

\begin{proof}[Proof of \lemref{lem:mislabeled_error}]
    Recall $\error_{\wt S} (f) = \frac{m_1}{m} \error_{\wt S_M}(f) + \frac{m_2}{m} \error_{\wt S_C}(f)$. Hence, we have 
    \begin{align}
        2\error_{\wt S}(f) - \error_{\wt S_M}(f) - \error_{\wt S_C}(f) &= \left(\frac{2m_1}{m} \error_{\wt S_M}(f) - \error_{\wt S_M}(f)\right) + \left(\frac{2m_2}{m} \error_{\wt S_C}(f) - \error_{\wt S_C}(f)\right) \\ &= \left(\frac{2m_1}{m} - 1\right) \error_{\wt S_M}(f) + \left(\frac{2m_2}{m} - 1 \right)\error_{\wt S_C} (f) \,.
    \end{align} 
    Since the dataset is labeled uniformly at random, with probability at least $1-\delta$, we have  $\left(\frac{2m_1}{m} - 1\right) \le \sqrt{\frac{\log(1/\delta)}{2m}}$. Similarly, we have with probability at least $1-\delta$, $\left(\frac{2m_2}{m} - 1\right) \le \sqrt{\frac{\log(1/\delta)}{2m}}$. Using union bound, with probability at least $1-\delta$, we have
    % \begin{align}
    %     2\error_{\wt S} - \error_{\wt S_M}(f) - \error_{\wt S_C}(f) \le \sqrt{\frac{\log(2/\delta)}{2m}} \left(\error_{\wt S_M}(f) + \error_{\wt S_C}(f) \right) \le 2\sqrt{\frac{\log(2/\delta)}{2m}} \,. \label{eq:lemma2_final}
    % \end{align}
    \begin{align}
        2\error_{\wt S} - \error_{\wt S_M}(f) - \error_{\wt S_C}(f) \le \sqrt{\frac{\log(2/\delta)}{2m}} \left(\error_{\wt S_M}(f) + \error_{\wt S_C}(f) \right) \,. \label{eq:lemma2_prefinal}
    \end{align}
    With re-arranging $\error_{\wt S_M}(f) + \error_{\wt S_C}(f)$ and using the inequality $ 1- a\le \frac{1}{1+a} $, we have  
    \begin{align}
        2\error_{\wt S} - \error_{\wt S_M}(f) - \error_{\wt S_C}(f) \le 2\error_{\wt \calS} \sqrt{\frac{\log(2/\delta)}{2m}}  \,. \label{eq:lemma2_final}
    \end{align}

    % We obtain the desired result by using 
\end{proof}

\begin{proof}[Proof of \lemref{lem:clear_error}]
% Recall 0-1 error on each point  $(x,y) \in S \cup \wt S$ is given by $\I{ f(x)\ne y}$.
In the set of correctly labeled points $S \cup \wt S_C$, we have $S$ as a random subset of $S \cup \wt S_C$. Hence, using Hoeffding's inequality for sampling without replacement (\lemref{lem:hoeffding_sampling}), we have with probability at least $1-\delta$
\begin{align}
    \error_{\wt \calS_C} (\wh f)- \error_{\calS \cup \wt \calS_C}( \wh f) \le  \sqrt{\frac{\log(1/\delta)}{2m_2}} \,.
\end{align}
Re-writing $\error_{\calS \cup \wt \calS_C}( \wh f)$ as $\frac{m_2}{m_2 + n} \error_{\wt \calS_C }(\wh f) + \frac{n}{m_2 + n} \error_{\calS }(\wh f)$, we have with probability at least $1-\delta$
\begin{align}
   \left(\frac{n}{n+m_2}\right) \left(\error_{\wt \calS_C} (\wh f)- \error_{\calS}( \wh f) \right) \le  \sqrt{\frac{\log(1/\delta)}{2m_2}} \,.
\end{align}
As before, assuming $2m_2 \approx m$, we have with probability at least $1-\delta$ 
\begin{align}
    \error_{\wt \calS_C} (\wh f)- \error_{\calS}( \wh f) \le \left(1+\frac{m_2}{n}\right)  \sqrt{\frac{\log(1/\delta)}{m}} \le \left(1 + \frac{m}{2n}\right) \sqrt{\frac{\log(1/\delta)}{m}} \,. \label{eq:lemma3_final}
\end{align} 
\end{proof}

\begin{proof}[Proof of \thmref{thm:error_ERM}] 
    Having established these core intermediate results, we can now combine above three lemmas to prove the main result. 
    In particular, we bound the population error on clean data ($\error_\calD(\wh f)$) as follows:  
    \begin{enumerate}[(i)]
        \item First, use \eqref{eq:lemma1_final}, to obtain an upper bound on the population error on clean data, i.e., with probability at least $1-\delta/4$, we have
        \begin{align}
            \error_{ \calD} (\wh f) \le 1 - \error_{ \wt \calS_M}(\wh f) + \sqrt{\frac{\log(4/\delta)}{m}} \,. 
        \end{align}
        \item  Second, use \eqref{eq:lemma2_final}, to relate the error on the mislabeled fraction with error on clean portion of randomly labeled data and error on whole randomly labeled dataset, i.e., with probability at least $1-\delta/2$, we have 
        \begin{align}
            - \error_{\wt S_M}(f) \le \error_{\wt S_C}(f) - 2\error_{\wt S}  + 2\error_{\wt S} \sqrt{\frac{\log(4/\delta)}{2m}}  \,. 
        \end{align} 
        \item Finally, use \eqref{eq:lemma3_final} to relate the error on the clean portion of randomly labeled data and error on clean training data, i.e., with probability $1-\delta/4$, we have 
        \begin{align}
            \error_{\wt \calS_C} (\wh f)\le - \error_{\calS}( \wh f) + \left(1 + \frac{m}{2n} \right) \sqrt{\frac{\log(4/\delta)}{m}} \,. 
        \end{align} 
    \end{enumerate}

    Using union bound on the above three steps, we have with probability at least $1-\delta$: 
    \begin{align}
        \error_\calD (\wh f) \le \error_{\calS}(\wh f)   + 1 - 2\error_{\wt \calS}(\wh f)   + \left(\sqrt{2} \error_{\wt S} + 2 + \frac{m}{2n}\right)  \sqrt{\frac{\log(4/\delta)}{m}} \,.
    \end{align}
    % Note that $(1/\sqrt{2} + 2.5)$ is a loose constant. In experiments, we use the ratio $\frac{m}{n}$
    %  the exact error $\error_{\wt \calS}(\wh f)$ 
    % to evaluate R.H.S.    
\end{proof}

\subsection{Proof of \propref{prop:rademacher}}

\begin{proof}[Proof of \propref{prop:rademacher}]
    For a classifier $ f: \calX \to \{-1, 1\}$, we have $1 - 2\,\indict{ f(x) \ne y} = y \cdot f(x)$. Hence, by definition of $\error$, we have 
    \begin{align}
        1 -2\error_{\wt \calS}(f) = \frac{1}{m}\sum_{i=1}^m y_i \cdot f(x_i) \le \sup_{f \in \calF} \, \frac{1}{m} \sum_{i=1}^m y_i \cdot f(x_i)  \,. \label{eq:error_rademacher}
    \end{align}
    Note that for fixed inputs $(x_1, x_2, \ldots, x_m)$ in $\wt S$, $(y_1, y_2, \ldots y_m)$ are random labels. Define $\phi_1 (y_1, y_2, \ldots, y_m) \defeq \sup_{f \in \calF} \, \frac{1}{m} \sum_{i=1}^m y_i \cdot f(x_i)$. We have the following bounded difference condition on $\phi_1$. For all i, 
    \begin{align}
        \sup_{y_1, \ldots y_m, y_i^\prime \in \{-1, 1\}^{m+1} } \abs{ \phi_1 (y_1,\ldots, y_i, \ldots, y_m) - \phi_1 (y_1,\ldots, y_i^\prime, \ldots, y_m)  } \le 1/m \,. \label{cond1_rademacher}
    \end{align} 
    
    Similarly, we define $\phi_2 (x_1, x_2, \ldots, x_m) \defeq \Expt{ y_i \sim_U \{-1, 1\}  }{ \sup_{f \in \calF} \, \frac{1}{m}  \sum_{i=1}^m y_i \cdot f(x_i)}$. We have the following bounded difference condition on $\phi_2$. 
    For all i,
    \begin{align}
        \sup_{x_1, \ldots x_m, x_i^\prime \in \calX^{m+1} } \abs{ \phi_2 (x_1,\ldots, x_i, \ldots, x_m) - \phi_1 (x_1,\ldots, x_i^\prime, \ldots, x_m)  } \le 1/m \,. \label{cond2_rademacher}
    \end{align}
    Using McDiarmid’s inequality (\lemref{lem:McDiarmid}) twice 
    with Condition \eqref{cond1_rademacher} and \eqref{cond2_rademacher}, 
    with probability at least $1-\delta$, we have
    \begin{align}
        \sup_{f \in \calF} \, \frac{1}{m} \sum_{i=1}^m y_i \cdot f(x_i)  - \Expt{x,y}{\sup_{f \in \calF} \, \frac{1}{m} \sum_{i=1}^m y_i \cdot f(x_i) } \le \sqrt{\frac{2\log(2/\delta)}{m}} \,. \label{eq:final_rademacher}
    \end{align} 
    Combining \eqref{eq:error_rademacher} and \eqref{eq:final_rademacher}, we obtain the desired result. 
\end{proof}


\subsection{Proof of \thmref{thm:error_regularized_ERM}}

Proof of \thmref{thm:error_regularized_ERM} follows similar to the proof of \thmref{thm:error_ERM}. Note that the same results in \lemref{lem:fit_mislabeled}, \lemref{lem:mislabeled_error}, and \lemref{lem:clear_error} hold in the regularized ERM case. However, the arguments in the proof of \lemref{lem:fit_mislabeled} change slightly. Hence, we state the lemma for regularized ERM and prove it here for completeness. 

\begin{lemma} \label{lem:lemma1_reg}
    Assume the same setup as \thmref{thm:error_regularized_ERM}. 
    Then for any $\delta >0$, with probability at least  $1-\delta$ 
    over the random draws of mislabeled data $\wt S_M$, we have 
    \begin{align}
        \error_\calD(\widehat f)  \le 1 -\error_{\wt \calS_M}(\widehat f) + \sqrt{\frac{\log(1/\delta)}{m}}\,. 
    \end{align} 
\end{lemma}
\begin{proof}
    The main idea of the proof remains the same, i.e. regard 
    the clean portion of the data 
    ($S \cup \wt S_C$) as fixed.   
    Then, there exists a classifier $f^*$ 
    that is optimal over draws 
    of the mislabeled data $\wt S_M$. 

    
    Formally, 
    \begin{align}
    f^* \defeq \argmin_{f \in \calF} \error_{\widecheck {\calD}} (f)  + \lambda R(f) \,, \label{eq:modified_ERM_reg}
    \end{align}
    where $$\widecheck \calD = \frac{n}{m+n} \calS + \frac{m_1}{m+n} \wt \calS_C  + \frac{m_2}{m+n}\calDm \,.$$ That is, $\widecheck \calD$ a combination of 
    the \emph{empirical distribution} 
    over correctly labeled data $S \cup \wt S_C$
    % in $S\cup \wt S$ 
    and the (population) distribution 
    over mislabeled data $\calDm$.
    Recall that 
    \begin{align}
    \wh f \defeq \argmin_{f \in \calF} \error_{\calS \cup \wt S} (f) + \lambda R(f) \,. \label{eq:orig_ERM_reg}
    \end{align}
    % 
    % 
    Since, $\widehat f$ minimizes 0-1 error 
    on $S \cup \wt S$, using ERM optimality on \eqref{eq:orig_ERM},  
    we have 
    \begin{align}
        \error_{\calS \cup \wt \calS}(\widehat f) + \lambda R(\wh f) \le \error_{
            \calS \cup \wt \calS}(f^*) + \lambda R(f^*) \,.    \label{eq:step1_reg}
    \end{align}
    Moreover, since $f^*$ is independent of $\wt S_M$, using Hoeffding's bound,
    % \footnote{For a fully rigorous argument,
    % refer to the complete proof in App.~\ref{app:proof_erm}.} 
    we have with probability at least $1-\delta$ that
    \begin{align}
      \error_{\wt \calS_M}(f^*) \le \error_{ \calDm}(f^*) +  \sqrt{\frac{\log(1/\delta)}{2 m_1}} \,. \label{eq:step2_reg} 
    \end{align}
    %$ 
    %for some constant $c_1\le 1/2$. 
    Finally, since $f^*$ is the optimal classifier on $\widecheck \calD$, 
    we have 
    \begin{align}
        \error_{\widecheck \calD}(f^*) + \lambda R(f^*) \le \error_{\widecheck \calD}(\widehat f) + \lambda R(\wh f) \,. \label{eq:step3_reg}
    \end{align}
     Now to relate \eqref{eq:step1_reg} and \eqref{eq:step3_reg}, we can re-write the \eqref{eq:step2_reg} as follows: 
    \begin{align}
        \error_{\calS \cup \wt\calS}(f^*) \le \error_{ \widecheck \calD}(f^*) +  \frac{m_1}{m+n}\sqrt{\frac{\log(1/\delta)}{2 m_1}} \,. \label{eq:step4_reg} 
    \end{align}
    After adding $\lambda R(f^*)$ on both sides in \eqref{eq:step4_reg}, we combine equations \eqref{eq:step1_reg}, \eqref{eq:step4_reg}, and \eqref{eq:step3_reg}, to get 
    \begin{align}
        \error_{\calS \cup \wt \calS}(\wh f) \le \error_{\widecheck \calD}(\wh f) +  \frac{m_1}{m+n}\sqrt{\frac{\log(1/\delta)}{2 m_1}} \,, 
    \end{align}
    which implies 
    \begin{align}
        \error_{ \wt \calS_M}(\wh f) \le \error_{\calDm}(\wh f) + \sqrt{\frac{\log(1/\delta)}{2 m_1}} \,. \label{eq:lemma_reg_final}
    \end{align}
    Similar as before, since $\wt S$ is obtained by randomly labeling an unlabeled dataset, we assume 
    $2m_1 \approx m$. Moreover, using $\error_{\calDm} = 1 - \error_{\calD}$ we obtain the desired result. 
\end{proof}
% \begin{proof}[Proof of ]
    
% \end{proof}

\subsection{Proof of \thmref{thm:multiclass_ERM}}

To prove our results in the multiclass case,
we first state and prove lemmas
parallel to those
% We first state and prove lemmas 
% parallel 
% to the three lemmas 
used in the proof of balanced binary case. 
We then combine these results 
% in the three lemmas 
to obtain the result in \thmref{thm:multiclass_ERM}. 

Before stating the result, 
we define mislabeled distribution $\calDm$ for any $\calD$.
While $\calDm$ and $\calD$ share 
the same marginal distribution over inputs $\calX$,
the conditional distribution over labels $y$ 
given an input $x\sim \calD_\calX$ is changed as follows:
For any $x$, the Probability Mass Function (PMF) over $y$ is defined as:  
$p_{\calDm} (\cdot \vert x) \defeq \frac{1 - p_{\calD}(\cdot \vert x)}{k - 1}$, where $ p_{\calD}(\cdot \vert x)$ is the PMF over $y$ for the distribution $\calD$. 

\begin{lemma} \label{lem:fit_mislabeled_multi}
    Assume the same setup as \thmref{thm:multiclass_ERM}. 
    Then for any $\delta >0$, with probability at least  $1-\delta$ 
    over the random draws of mislabeled data $\wt S_M$, we have 
    \begin{align}
        \error_\calD(\widehat f)  \le (k-1)\left(1 -\error_{\wt \calS_M}(\widehat f)\right) + (k-1)\sqrt{\frac{\log(1/\delta)}{m}}\,. \label{eq:lemma1_multi}
    \end{align}   
\end{lemma} 

\begin{proof}
   
    The main idea of the proof remains the same.
    We begin by regarding the clean portion of the data 
    ($S \cup \wt S_C$) as fixed. 
    Then, there exists a classifier $f^*$ 
    that is optimal over draws 
    of the mislabeled data $\wt S_M$. 
    
    However, in the multiclass case,
    we cannot as easily relate the population error on mislabeled data 
    to the population accuracy on clean data.   
    While for binary classification, 
    % we could upper bound $\error_{\wt \calS_M}$ 
    % with $1-\error_\calD$ 
    we could lower bound the population accuracy $1-\error_\calD$
    with the empirical error on mislabeled data $\error_{\wt \calS_M}$ 
    (in the proof of \lemref{lem:fit_mislabeled}), 
    for multiclass classification, 
    error on the mislabeled data 
    and accuracy on the clean data 
    in the population 
    are not so directly related.  
    To establish \eqref{eq:lemma1_multi},
    we break the error on the 
    (unknown) mislabeled data 
    into two parts: one term corresponds 
    to predicting the true label on mislabeled data, 
    and the other corresponds to predicting 
    neither the true label 
    nor the assigned (mis-)label.  
    Finally, we relate these errors to their
    population counterparts to establish \eqref{eq:lemma1_multi}. 
    
    Formally, 
    \begin{align}
    f^* \defeq \argmin_{f \in \calF} \error_{\widecheck {\calD}} (f)  + \lambda R(f) \,, \label{eq:modified_ERM_reg2}
    \end{align}
    where $$\widecheck \calD = \frac{n}{m+n} \calS + \frac{m_1}{m+n} \wt \calS_C  + \frac{m_2}{m+n}\calDm \,.$$ 
    That is, $\widecheck \calD$ is a combination 
    of the \emph{empirical distribution} 
    over correctly labeled data $S \cup \wt S_C$
    % in $S\cup \wt S$ 
    and the (population) distribution 
    over mislabeled data $\calDm$.
    Recall that 
    \begin{align}
    \wh f \defeq \argmin_{f \in \calF} \error_{\calS \cup \wt S} (f) + \lambda R(f) \,. \label{eq:orig_ERM_reg2}
    \end{align}
    % 
    % 
    Following the exact steps from the proof of \lemref{lem:lemma1_reg}, 
    with probability at least $1-\delta$, we have  
    \begin{align}
        \error_{ \wt \calS_M}(\wh f) \le \error_{\calDm}(\wh f) + \sqrt{\frac{\log(1/\delta)}{2 m_1}} \,. \label{eq:lemma1_final_multi_prev}
    \end{align}
    Similar to before, since $\wt S$ is obtained 
    by randomly labeling an unlabeled dataset, 
    we assume 
    $\frac{k}{k-1} m_1 \approx m$. 
    
    Now we will relate $\error_{\calDm} (\wh f)$ with $\error_{\calD}(\wh f)$. 
    Let $y^T$ denote the (unknown) true label 
    for a mislabeled point $(x, y)$ 
    (i.e., label before replacing it with a mislabel). 
    \begin{align*}    
         \Expt{(x, y) \in \sim \calDm}{\indict{ \wh f(x) \ne y }}  &= \underbrace{\Expt{(x, y) \in \sim \calDm}{\indict{ \wh f(x) \ne y \land \wh f(x) \ne y^T}}}_{\RN{1}} \\ &\qquad \qquad + \underbrace{\Expt{(x, y) \in \sim \calDm}{\indict{ \wh f(x) \ne y \land \wh f(x) = y^T}}}_{\RN{2}} \,. \numberthis \label{eq:excess_term}
    \end{align*}
    Clearly, term 2 is one minus the accuracy 
    on the clean unseen data, i.e.,
    \begin{align}
        \RN{2} = 1 - \Expt{{x,y} \sim \calD}{ \indict{ \wh f(x) \ne y}} = 1- \error_{\calD}(\wh f) \,. \label{eq:term1}    
    \end{align}
    Next, we relate term 1 with the error on the unseen clean data. 
    We show that term 1 is equal to the error on the unseen clean data 
    scaled by $\frac{k-2}{k-1}$,
    where $k$ is the number of labels.
    Using the definition of mislabeled distribution $\calDm$,  
    we have 
    \begin{align}
        \RN{1} = \frac{1}{k-1} \left( \Expt{(x, y) \in \sim \calD}{ \sum_{i \in \calY \land i\ne y}  \indict{ \wh f(x) \ne i \land \wh f(x) \ne y}} \right) = \frac{k-2}{k-1} \error_{\calD}(\wh f) \,.\label{eq:term2}
    \end{align}    

    Combining the result in \eqref{eq:term1}, \eqref{eq:term2} and \eqref{eq:excess_term}, we have 
    \begin{align}
        \error_{\calDm}(\wh f) = 1- \frac{1}{k-1} \error_{\calD}(\wh f) \,.\label{eq:combine_terms}
    \end{align}
    Finally, combining the result in \eqref{eq:combine_terms} 
    with equation \eqref{eq:lemma1_final_multi_prev}, 
    we have with probability $1-\delta$, 
    \begin{align}
      \error_{\calD}(\wh f) \le  (k-1) \left( 1- \error_{ \wt \calS_M}(\wh f) \right)  + (k-1) \sqrt{\frac{k \log(1/\delta)}{ 2(k-1)m}} \,. \label{eq:lemma1_final_multi}
    \end{align}
\end{proof}

\begin{lemma} \label{lem:mislabeled_error_multi}
    Assume the same setup as \thmref{thm:multiclass_ERM}. 
    Then for any $\delta >0$, 
    with probability at least $1-\delta$ 
    over the random draws of $\wt S$, we have  
    % \begin{align}
        $$\abs{k\error_{\wt \calS}(\widehat f) - \error_{\wt \calS_C}(\widehat f) -  (k-1)\error_{\wt \calS_M}(\widehat f) } \le  2k\sqrt{\frac{\log(4/\delta)}{2m}}\,. $$ % \label{eq:lemma2}
    % \end{align}   
    %  for some constant $c_3 \le 1.0\,$.
\end{lemma} 


\begin{proof}
    Recall $\error_{\wt S} (f) = \frac{m_1}{m} \error_{\wt S_M}(f) + \frac{m_2}{m} \error_{\wt S_C}(f)$. Hence, we have 
    \begin{align*}
        k\error_{\wt S}(f) - (k-1)\error_{\wt S_M}(f) - \error_{\wt S_C}(f) &= (k-1)\left(\frac{k m_1}{(k-1) m} \error_{\wt S_M}(f) - \error_{\wt S_M}(f)\right) \\ & \qquad \qquad + \left(\frac{km_2}{m} \error_{\wt S_C}(f) - \error_{\wt S_C}(f)\right) \\ &= k \left[ \left(\frac{m_1}{m} - \frac{k-1}{k}\right) \error_{\wt S_M}(f) + \left(\frac{m_2}{m} - \frac{1}{k} \right) \error_{\wt S_C} (f) \right] \,.
    \end{align*} 
    Since the dataset is randomly labeled, 
    we have with probability at least $1-\delta$, 
    $\left(\frac{m_1}{m} - \frac{k-1}{k}\right) \le \sqrt{\frac{\log(1/\delta)}{2m}}$. 
    Similarly, we have with probability at least $1-\delta$, 
    $\left(\frac{m_2}{m} - \frac{1}{k}\right) \le \sqrt{\frac{\log(1/\delta)}{2m}}$. 
    Using union bound, we have with probability at least $1-\delta$
    % \begin{align}
    %     2\error_{\wt S} - \error_{\wt S_M}(f) - \error_{\wt S_C}(f) \le \sqrt{\frac{\log(2/\delta)}{2m}} \left(\error_{\wt S_M}(f) + \error_{\wt S_C}(f) \right) \le 2\sqrt{\frac{\log(2/\delta)}{2m}} \,. \label{eq:lemma2_final}
    % \end{align}
    \begin{align}
        k\error_{\wt S}(f) - (k-1)\error_{\wt S_M}(f) - \error_{\wt S_C}(f)  \le k \sqrt{\frac{\log(2/\delta)}{2m}} \left(\error_{\wt S_M}(f) + \error_{\wt S_C}(f) \right) \,. \label{eq:lemma2_final_multi}
    \end{align}

    % We obtain the desired result by using 
\end{proof}

\begin{lemma} \label{lem:clear_error_multi}
    Assume the same setup as \thmref{thm:multiclass_ERM}. 
    Then for any $\delta >0$, with probability at least $1-\delta$ 
    over the random draws of $\wt S_C$ and $S$, we have 
    % \begin{align}
        $$\abs{\error_{\wt \calS_C}(\widehat f) - \error_{\calS}(\widehat f) } \le 1.5 \sqrt{\frac{k\log(2/\delta)}{2m}}\,.$$ %\label{eq:lemma3}
    % \end{align}   
    % for some constant $c_2 \le 1.2\,$.
\end{lemma} 
\begin{proof}
    % Recall 0-1 error on each point  $(x,y) \in S \cup \wt S$ is given by $\I{ f(x)\ne y}$.
    In the set of correctly labeled points $S \cup \wt S_C$,
    we have $S$ as a random subset of $S \cup \wt S_C$. 
    Hence, using Hoeffding's inequality 
    for sampling without replacement 
    (\lemref{lem:hoeffding_sampling}), 
    we have with probability at least $1-\delta$
    \begin{align}
        \error_{\wt \calS_c} (\wh f)- \error_{\calS \cup \wt \calS_C}( \wh f) \le  \sqrt{\frac{\log(1/\delta)}{2m_2}} \,.
    \end{align}
    Re-writing $\error_{\calS \cup \wt \calS_C}( \wh f)$ 
    as $\frac{m_2}{m_2 + n} \error_{\wt \calS_C }(\wh f) + \frac{n}{m_2 + n} \error_{\calS }(\wh f)$, 
    we have with probability at least $1-\delta$
    \begin{align}
       \left(\frac{n}{n+m_2}\right) \left(\error_{\wt \calS_c} (\wh f)- \error_{\calS}( \wh f) \right) \le  \sqrt{\frac{\log(1/\delta)}{2m_2}} \,.
    \end{align}
    As before, assuming $km_2 \approx m$, 
    we have with probability at least $1-\delta$ 
    \begin{align}
        \error_{\wt \calS_c} (\wh f)- \error_{\calS}( \wh f) \le \left(1+\frac{m_2}{n}\right)  \sqrt{\frac{k\log(1/\delta)}{2m}} \le \left( 1 + \frac{1}{k}\right) \sqrt{\frac{k\log(1/\delta)}{2m}} \,. \label{eq:lemma3_final_multi}
    \end{align} 
\end{proof}

\begin{proof}[Proof of \thmref{thm:multiclass_ERM}] 
    Having established these core intermediate results, 
    we can now combine above three lemmas. 
    In particular, we bound the population error 
    on clean data ($\error_\calD(\wh f)$) as follows:  
    \begin{enumerate}[(i)]
        \item First, use \eqref{eq:lemma1_final_multi}, 
        to obtain an upper bound on the population error on clean data, 
        i.e., with probability at least $1-\delta/4$, we have
        \begin{align}
            \error_{ \calD} (\wh f) \le (k-1)\left(1 - \error_{ \wt \calS_M}(\wh f) \right) + (k-1) \sqrt{\frac{k\log(4/\delta)}{2(k-1)m}} \,. 
        \end{align}
        \item  Second, use \eqref{eq:lemma2_final_multi}
        to relate the error on the mislabeled fraction 
        with error on clean portion of randomly labeled data 
        and error on whole randomly labeled dataset, 
        i.e., with probability at least $1-\delta/2$, we have 
        \begin{align}
            - (k-1)\error_{\wt S_M}(f) \le \error_{\wt S_C}(f) - k\error_{\wt S}  + k\sqrt{\frac{\log(4/\delta)}{2m}}  \,. 
        \end{align} 
        \item Finally, use \eqref{eq:lemma3_final_multi} 
        to relate the error on the clean portion of randomly labeled data 
        and error on clean training data, 
        i.e., with probability $1-\delta/4$, we have 
        \begin{align}
            \error_{\wt \calS_C} (\wh f)\le - \error_{\calS}( \wh f) + \left(1 + \frac{m}{kn} \right) \sqrt{\frac{k\log(4/\delta)}{2m}} \,. 
        \end{align} 
    \end{enumerate}

    Using union bound on the above three steps, 
    we have with probability at least $1-\delta$: 
    \begin{align}
        \error_\calD (\wh f) \le \error_{\calS}(\wh f) + (k-1) - k\error_{\wt \calS}(\wh f)   + (\sqrt{k(k-1)} + k + \sqrt{k} + \frac{m}{n\sqrt{k}})  \sqrt{\frac{\log(4/\delta)}{2m}} \,.\label{eq:multiclass_ERM_final}
    \end{align}
    Simplifying the term in RHS of \eqref{eq:multiclass_ERM_final}, 
    we get the desired result. 
    % Note that since $\frac{m}{n\sqrt{k}}$ 
    % is much smaller than the sum of the other terms
    % the other terms in summation, 
    % we ignore $\frac{m}{n\sqrt{k}}$  
    % Z: ??? --- great
    % that 
    % them
    in the final bound. 
    % we ignore that in the final bound. 
    % Note that $(1/\sqrt{2} + 2.5)$ is a loose constant. In experiments, we use the ratio $\frac{m}{n}$
    %  the exact error $\error_{\wt \calS}(\wh f)$ 
    % to evaluate R.H.S.    
\end{proof}

\newpage
\section{Proofs from \secref{sec:linear_models}}\label{app:proof_gd}
We suppose that the parameters of the linear function 
are obtained via gradient descent on 
the following $L_2$ regularized problem: 
\begin{align}
    % n in denominator is avoided deliberately
    \calL_S(w; \lambda) \defeq \sum_{i=1}^n{(w^Tx_i - y_i)^2} + \lambda \norm{w}{2}^2 \,, \label{eq:l2_MSE_app}   
\end{align}
where $\lambda\ge0$ is a regularization parameter. 
We assume access to a clean dataset 
$S = \{(x_i, y_i)\}_{i=1}^n \sim \calD^n$ 
and randomly labeled dataset 
$\wt S = \{(x_i, y_i)\}_{i=n+1}^{n+m} \sim \wt \calD^m$. 
Let $\bX = [x_1, x_2, \cdots, x_{m+n}]$ 
and $\by = [y_1, y_2, \cdots, y_{m+n}]$. 
Fix a positive learning rate $\eta$ such that 
$\eta \le 1/\left(\norm{\bX^T\bX}{\text{op}} + \lambda^2\right)$ 
and an initialization $w_0 = 0$. 
% \todos{Assumption made for simplicty}. 
Consider the following gradient descent iterates 
to minimize objective \eqref{eq:l2_MSE_app} on $S \cup \wt S$:
\begin{align}
w_t = w_{t-1} - \eta \grad_w \calL_{S \cup \wt S} (w_{t-1}; \lambda) \quad \forall t=1,2,\ldots \label{eq:GD_iterates_app}
\end{align} 
Then we have $\{ w_t\}$ converge to the limiting solution 
$\wh w = \left( \bX^T\bX+\lambda \boldsymbol{I}\right)^{-1}\bX^T\by$. Define $\widehat f (x) \defeq f(x ; \wh w) $.  

% \subsection{\textcolor{red}{Errata}}

% We wish to correct the following error in the body:
% \codref{cond:error_stability} is not enough 
% to guarantee the result in \thmref{thm:linear}. 
% We now present a slightly stronger condition 
% called \emph{hypothesis stability} 
% under which we obtain a result 
% similar to \thmref{thm:linear}. 

% This error doesn't change the main arguments of the proof,
% where we show that the empirical train error 
% is less than or equal to the leave-one-out error.
% We need a stronger condition to relate leave-one-out error 
% with the population error of the original classifier. 
% Specifically, while \codref{cond:error_stability} 
% relates the average population error of leave-one-out classifiers 
% with the population error of the original classifier, 
% we need the new condition to show the concentration 
% of the empirical leave-one-out error 
% and average population error of leave-one-out classifiers. 
% main takeaway 

% Note that the new condition, 
% while being stronger than the previous one, 
% still doesn't imply generalization \citep{bousquet2002stability,elisseeff2003leave,abou2019exponential}. 
% Overall, the main results in \secref{sec:ERM_training} 
% and takeaways of the paper remain unaffected by the error.  

% We now present the new condition 
% and a corrected statement of \thmref{thm:linear}. 
% Recall, for a given training set $S \sim \calD^n $, 
% we use $S_{(i)}$ to denote the training set $S$ 
% with the $i^{\text{th}}$ point removed.

% \begin{condition}[Hypothesis Stability] 
%     \label{cond:hypothesis_stability}
%     We have $\beta$ hypothesis stability 
%     if our training algorithm $\calA$ satisfies the following: 
%     \begin{align*}
%     % ${\sum_{i=1}^n \frac{\error_{\calD}( f(\calA, S_{(i)}))}{n} - \error_\calD(f(\calA, S))} \le \beta\,$.
%     \forall i \in \{1,2,\ldots, n\}, \quad  \Expt{\calS, (x,y) \in \calD}{ \abs{\error\left( f(x) ,y  \right) - \error\left( f_{(i)}(x), y \right) }} \le \frac{\beta}{n} \,,
%     \end{align*}
%     where $f_{(i)} \defeq f(\calA, S_{(i)})$ and $ f \defeq f(\calA, S)$.
% \end{condition}

% \begin{theorem}[Correct statement of \thmref{thm:linear}] \label{thm:new_linear}
%     Assume that this gradient descent algorithm satisfies \codref{cond:hypothesis_stability}
%     with $\beta=\calO(1)$.  
%     Then for any $\delta >0$, with probability at least $1-\delta$ 
%     over the random draws of datasets $\wt S$ and $S$, we have:
%     \begin{align}
%         \error_\calD(\widehat f) \le \error_\calS(\widehat f) + 1 - 2 \error_{\wt\calS}(\widehat f) + \left(\frac{1}{\sqrt{2}} + 1.5 \right) \sqrt{\frac{\log(4/\delta)}{m}} + \sqrt{\frac{4}{\delta}\left(\frac{1}{m} +\frac{3\beta}{m+n} \right)}  \,. \label{eq:gd_error}
%     \end{align} 
%     % for some constant $c\le 3.2$.
% \end{theorem}

\subsection{Proof of \thmref{thm:linear}}
We use a standard result from linear algebra, 
namely the Shermann-Morrison formula 
\citep{sherman1950adjustment} for matrix inversion:  

\begin{lemma}[\citet{sherman1950adjustment}] \label{lem:sherman}
    Suppose $\bA \in \Real^{n \times n}$ 
    is an invertible square matrix 
    and $u,v \in \Real^n$ are column vectors. 
    Then $\bA + uv^T$ is invertible iff $1 + v^T \bA u \ne 0$ 
    and in particular
    \begin{align}
        (\bA + u v^T)^{-1} = \bA^{-1}  - \frac{\bA^{-1} uv^T \bA^{-1} }{ 1 + v^T \bA^{-1} u} \,.
    \end{align}   
\end{lemma}
\newcommand\byy[1]{\by_{\left(#1\right)}}
\newcommand\bXX[1]{\bX_{\left(#1\right)}}
\newcommand\ff[1]{\wh f_{\left(#1\right)}}

For a given training set $S \cup \wt S_C$, 
define leave-one-out error 
on mislabeled points in the training data 
as $$\error_{\text{LOO}(\wt S_M) } = \frac{\sum_{(x_i, y_i) \in \wt S_M} \error( f_{(i)}( x_i), y_i)}{ \abs{\wt S_M }} \,, $$
where $f_{(i)} \defeq f(\calA, (S \cup \wt S)_{(i)})$. 
To relate empirical leave-one-out error and population error 
with hypothesis stability condition, 
we use the following lemma:   

\begin{lemma}[\citet{bousquet2002stability}] \label{lem:stability_error}
    For the leave-one-out error, we have
    \begin{align}
        \Expo{ \left( \error_{\calDm}(\wh f) -\error_{\text{LOO}(\wt S_M) } \right)^2 } \le \frac{1}{2m_1}+  \frac{3\beta}{n + m}\,.
    \end{align}   
    % where $ f \defeq f(\calA, S \cup \wt S) $.
\end{lemma}

Proof of the above lemma is similar 
to the proof of Lemma 9 in \citet{bousquet2002stability} 
and can be found in \appref{app:proof_lem_error}. 
% 
% Before presenting the result, we introduce some notation. 
Before presenting the proof of \thmref{thm:linear}, 
we introduce some more notation. 
Let $\bX_{(i)}$ denote the matrix of covariates 
with the $i^{\text{th}}$ point removed. 
Similarly, let $\by_{(i)}$ be the array of responses 
with the $i^{\text{th}}$ point removed. 
Define the corresponding regularized GD solution 
as $\wh w_{(i)} = \left( \bXX{i}^T\bXX{i}+\lambda \boldsymbol{I}\right)^{-1}\bXX{i}^T\byy{i}$. 
Define $\ff{i}(x) \defeq f(x ; \wh w_{(i)}) $.

\begin{proof}[Proof of \thmref{thm:linear}]
    Because squared loss minimization does not imply 0-1 error minimization, 
    we cannot use arguments from \lemref{lem:fit_mislabeled}. 
    This is the main technical difficulty. 
    To compare the 0-1 error at a train point with an unseen point, 
    we use the closed-form expression for $\widehat{w}$ 
    and Shermann-Morrison formula 
    to upper bound training error 
    with leave-one-out cross validation error. 
    
    The proof is divided into three parts: 
    In part one, we show that 0-1 error 
    on mislabeled points in the training set 
    is lower than the error obtained 
    by leave-one-out error at those points. 
    In part two, we relate this leave-one-out error 
    with the population error on mislabeled distribution
    using \codref{cond:hypothesis_stability}.
    While the empirical leave-one-out error is an unbiased estimator 
    of the average population error of leave-one-out classifiers, 
    we need hypothesis stability 
    to control the variance 
    of empirical leave-one-out error. 
    Finally, in part three, we show 
    that the error on the mislabeled training points 
    can be estimated with just the randomly labeled 
    and clean training data (as in proof of \thmref{thm:error_ERM}).  

    \textbf{Part 1 {} {}} First we relate training error with leave-one-out error.        
    For any training point $(x_i, y_i)$ in $\wt S \cup S$, we have 
    \begin{align}
        \error(\wh f(x_i), y_i ) &= \indict{ y_i \cdot x_i^T \wh w < 0 } = \indict{ y_i \cdot x_i^T \left( \bX^T\bX+\lambda \boldsymbol{I}\right)^{-1}\bX^T\by < 0 } \\
        &= \indict{ y_i \cdot x_i^T \underbrace{\left( \bXX{i}^T\bXX{i} + x_i ^T x_i +\lambda \boldsymbol{I}\right)^{-1}}_{\RN{1}} (\bXX{i}^T\byy{i} + y_i \cdot x_i) < 0 } \,.
    \end{align}
    Letting $\bA = \left(\bXX{i}^T\bXX{i} +\lambda \boldsymbol{I}\right)$ 
    and using \lemref{lem:sherman} on term 1, we have 
    \begin{align}
        \error(\wh f(x_i), y_i ) &= \indict{ y_i \cdot x_i^T \left[\bA^{-1} -  \frac{\bA^{-1} x_i x_i^T \bA^{-1}}{ 1 + x_i ^T \bA^{-1} x_i } \right] (\bXX{i}^T\byy{i} + y_i \cdot x_i) < 0 } \\
        &= \indict{ y_i \cdot\left[ \frac{ x_i^T \bA^{-1} ( 1 + x_i ^T \bA^{-1} x_i ) -  x_i^T \bA^{-1} x_i x_i^T \bA^{-1}}{ 1 + x_i ^T \bA ^{-1}x_i } \right] (\bXX{i}^T\byy{i} + y_i \cdot x_i) < 0 } \\
        &= \indict{ y_i \cdot\left[ \frac{ x_i^T \bA^{-1}}{ 1 + x_i ^T \bA ^{-1}x_i } \right] (\bXX{i}^T\byy{i} + y_i \cdot x_i) < 0 } \,.
    \end{align}

    Since $1 + x_i^T \bA^{-1} x_i > 0$, we have 
    \begin{align}
        \error(\wh f(x_i), y_i ) &= \indict{ y_i \cdot x_i^T \bA^{-1} (\bXX{i}^T\byy{i} + y_i \cdot x_i) < 0 } \\
        &= \indict{ x_i^T \bA^{-1} x_i +  y_i \cdot x_i^T \bA^{-1} (\bXX{i}^T\byy{i}) < 0 } \\
        &\le \indict{ y_i \cdot x_i^T \bA^{-1} (\bXX{i}^T\byy{i}) < 0 } = \error(\ff{i}(x_i), y_i ) \,.\label{eq:LOO_error}
    \end{align}

    Using \eqref{eq:LOO_error}, we have 
    \begin{align}
        \error_{\wt \calS_M } (\wh f) \le \error_{\text{LOO} (\wt S_M)} \defeq \frac{\sum_{(x_i, y_i) \in \wt S_M} \error(\ff{i}(x_i), y_i ) }{\abs{\wt \calS_M}}\label{eq:LOO_error_final} \,.
    \end{align}
    \textbf{Part 2 {}{}} We now relate RHS in \eqref{eq:LOO_error_final} 
    with the population error on mislabeled distribution. 
    To do this, we leverage \codref{cond:hypothesis_stability} 
    and \lemref{lem:stability_error}. 
    In particular, we have 

    \begin{align}
        \Expt{\calS \cup \wt \calS_M }{ \left(\error_{\calDm}(\wh f) - \error_{\text{LOO} (\wt S_M)}\right)^2 } \le \frac{1}{2m_1} + \frac{3\beta}{m+n} \,.
    \end{align}

    Using Chebyshev's inequality, with probability at least $1-\delta$, we have 
    \begin{align}
        \error_{\text{LOO} (\wt S_M)} \le  \error_{\calDm}(\wh f)   + \sqrt{\frac{1}{\delta}\left(\frac{1}{2m_1} +\frac{3\beta}{m+n} \right)} \,. \label{eq:final_mislabeled_linear}
    \end{align}
    

    \textbf{Part 3 {}{}} Combining \eqref{eq:final_mislabeled_linear} and \eqref{eq:LOO_error_final}, we have 

    \begin{align}
        \error_{\wt \calS_M } (\wh f) \le \error_{\calDm}(\wh f)   + \sqrt{\frac{1}{\delta}\left(\frac{1}{2m_1} +\frac{3\beta}{m+n} \right)} \,. \label{eq:linear_parallel_lem1}
    \end{align}

    Compare \eqref{eq:linear_parallel_lem1} with \eqref{eq:lemma1_final} 
    in the proof of \lemref{lem:fit_mislabeled}. 
    We obtain a similar relationship 
    between $\error_{\wt \calS_M }$ and $\error_{\calDm}$ 
    but with a polynomial concentration 
    instead of exponential concentration. 
    In addition, since we just use concentration arguments 
    to relate mislabeled error to the errors
    on the clean and unlabeled portions 
    of the randomly labeled data, 
    we can directly use the results 
    in \lemref{lem:mislabeled_error} and \lemref{lem:clear_error}. 
    Therefore, combining results in \lemref{lem:mislabeled_error}, \lemref{lem:clear_error}, and \eqref{eq:linear_parallel_lem1} with union bound, 
    we have with probability at least $1-\delta$
    \begin{align}
        \error_\calD(\widehat f) \le \error_\calS(\widehat f) + 1 - 2 \error_{\wt\calS}(\widehat f) + \left(\sqrt{2}\error_{\wt\calS}(\widehat f) + 1 + \frac{m}{2n} \right) \sqrt{\frac{\log(4/\delta)}{m}} + \sqrt{\frac{4}{\delta}\left(\frac{1}{m} +\frac{3\beta}{m+n} \right)}  \,.
    \end{align}
    

       
\end{proof}

\subsection{Extension to multiclass classification} \label{app:multiclass_linear}
For multiclass problems with squared loss minimization, as standard practice, we consider one-hot encoding for the underlying label, i.e., a class label $c \in [k]$ is treated as $(0, \cdot, 0,1,0, \cdot, 0) \in \Real^k$ (with $c$-th coordinate being 1).  As before, we suppose that the parameters of the linear function 
are obtained via gradient descent on the following $L_2$ regularized problem: 
\begin{align}
    % n in denominator is avoided deliberately
    \calL_S(w; \lambda) \defeq \sum_{i=1}^n\norm{w^Tx_i - y_i}{2}^2 + \lambda \sum_{j=1}^k \norm{w_j}{2}^2 \,, \label{eq:l2_multiclass_MSE_app}   
\end{align}
where $\lambda\ge0$ is a regularization parameter. 
We assume access to a clean dataset 
$S = \{(x_i, y_i)\}_{i=1}^n \sim \calD^n$ 
and randomly labeled dataset 
$\wt S = \{(x_i, y_i)\}_{i=n+1}^{n+m} \sim \wt \calD^m$. 
Let $\bX = [x_1, x_2, \cdots, x_{m+n}]$ 
and $\by = [e_{y_1}, e_{y_2}, \cdots, e_{y_{m+n}}]$. 
Fix a positive learning rate $\eta$ such that 
$\eta \le 1/\left(\norm{\bX^T\bX}{\text{op}} + \lambda^2\right)$ 
and an initialization $w_0 = 0$. 
% \todos{Assumption made for simplicty}. 
Consider the following gradient descent iterates 
to minimize objective \eqref{eq:l2_MSE_app} on $S \cup \wt S$:
\begin{align}
{w_j}^t = {w_j}^{t-1} - \eta \grad_{w_j} \calL_{S \cup \wt S} (w^{t-1}; \lambda) \quad \forall t=1,2,\ldots \text{ and } j=1,2,\ldots,k  \,. \label{eq:GD_multi_iterates_app}
\end{align} 
Then we have $\{ {w_j}^t\}$ for all $j =1,2,\cdots, k$ converge to the limiting solution 
$\wh w_j = \left( \bX^T\bX+\lambda \boldsymbol{I}\right)^{-1}\bX^T\by_j$. Define $\widehat f (x) \defeq f(x ; \wh w) $.  

\begin{theorem}\label{thm:multi_linear}
    Assume that this gradient descent algorithm satisfies \codref{cond:hypothesis_stability}
    with $\beta=\calO(1)$.  
    Then for a multiclass classification problem wth $k$ classes, for any $\delta >0$, with probability at least $1-\delta$, we have:
    \begin{align*}
        \error_\calD(\widehat f) \le \error_\calS(\widehat f) &+ (k-1)\left(1 - \frac{k}{k-1} \error_{\wt\calS}(\widehat f) \right) \\ &+ \left(k + \sqrt{k} + \frac{m}{n\sqrt{k}} \right) \sqrt{\frac{\log(4/\delta)}{2m}} + \sqrt{k(k-1)} \sqrt{\frac{4}{\delta}\left(\frac{1}{m} +\frac{3\beta}{m+n} \right)}  \,. \numberthis \label{eq:gd_multi_error}
    \end{align*} 
    % for some constant $c\le 3.2$.
\end{theorem}
\begin{proof}
    The proof of this theorem is divided into two parts. In the first part, we relate the error on the mislabeled samples with the population error on the mislabeled data. Similar to the proof of \thmref{thm:linear}, we use Shermann-Morrison formula to upper bound training error with leave-one-out error on each $\wh w^j$. Second part of the proof follows entirely from the proof of \thmref{thm:multiclass_ERM}. In essence, the first part derives an equivalent of \eqref{eq:lemma1_final_multi_prev} for GD training with squared loss and then the second part follows from the proof  of \thmref{thm:multiclass_ERM}. 
    
    \textbf{Part-1:} Consider a training point $(x_i,y_i)$ in $\wt S \cup S $. For simplicity, we use $c_i$ to denote the class of $i$-th point and use $y_i$ as the corresponding one-hot embedding. Recall error in multiclass point is given by $\error(\wh f(x_i), y_i ) = \indict{ c_i \not \in \argmax x_i^T \wh w }$. Thus, there exists a $j \ne c_i \in [k]$, such that we have
     \begin{align}
        \error(\wh f(x_i), y_i ) &= \indict{ c_i \not \in \argmax x_i^T \wh w } = \indict{ x_i^T \wh w_{c_i} < x_i^T \wh w_{j}  } \\ &= \indict{ x_i^T \left( \bX^T\bX+\lambda \boldsymbol{I}\right)^{-1}\bX^T\by_{c_i} < x_i^T \left( \bX^T\bX+\lambda \boldsymbol{I}\right)^{-1}\bX^T\by_{j} } \\
        &= \indict{ x_i^T \underbrace{\left( \bXX{i}^T\bXX{i} + x_i ^T x_i +\lambda \boldsymbol{I}\right)^{-1}}_{\RN{1}} \left(\bXX{i}^T{\by_{c_i}}_{(i)} + x_i - \bXX{i}^T{\by_{j}}_{(i)}\right) < 0 } \,.
    \end{align}
    Letting $\bA = \left(\bXX{i}^T\bXX{i} +\lambda \boldsymbol{I}\right)$ 
    and using \lemref{lem:sherman} on term 1, we have 
    \begin{align}
        \error(\wh f(x_i), y_i ) &= \indict{ x_i^T \left[\bA^{-1} -  \frac{\bA^{-1} x_i x_i^T \bA^{-1}}{ 1 + x_i ^T \bA^{-1} x_i } \right]  \left(\bXX{i}^T{\by_{c_i}}_{(i)} + x_i - \bXX{i}^T{\by_{j}}_{(i)}\right) < 0 } \\
        &= \indict{ \left[ \frac{ x_i^T \bA^{-1} ( 1 + x_i ^T \bA^{-1} x_i ) -  x_i^T \bA^{-1} x_i x_i^T \bA^{-1}}{ 1 + x_i ^T \bA ^{-1}x_i } \right]  \left(\bXX{i}^T{\by_{c_i}}_{(i)} + x_i - \bXX{i}^T{\by_{j}}_{(i)}\right) < 0 } \\
        &= \indict{ \left[ \frac{ x_i^T \bA^{-1}}{ 1 + x_i ^T \bA ^{-1}x_i } \right]  \left(\bXX{i}^T{\by_{c_i}}_{(i)} + x_i - \bXX{i}^T{\by_{j}}_{(i)}\right) < 0} \,.
    \end{align}
    Since $1 + x_i^T \bA^{-1} x_i > 0$, we have 
    \begin{align}
        \error(\wh f(x_i), y_i ) &= \indict{ x_i^T \bA^{-1}  \left(\bXX{i}^T{\by_{c_i}}_{(i)} + x_i - \bXX{i}^T{\by_{j}}_{(i)}\right) < 0 } \\
        &= \indict{ x_i^T \bA^{-1} x_i +  x_i^T \bA^{-1}  \bXX{i}^T{\by_{c_i}}_{(i)}  - x_i^T\bA^{-1}  \bXX{i}^T{\by_{j}}_{(i)} < 0 } \\
        &\le \indict{  x_i^T \bA^{-1}  \bXX{i}^T{\by_{c_i}}_{(i)}  - x_i^T\bA^{-1}  \bXX{i}^T{\by_{j}}_{(i)} < 0  } = \error(\ff{i}(x_i), y_i ) \,.\label{eq:LOO_error_multi}
    \end{align}
    Using \eqref{eq:LOO_error_multi}, we have 
    \begin{align}
        \error_{\wt \calS_M } (\wh f) \le \error_{\text{LOO} (\wt S_M)} \defeq \frac{\sum_{(x_i, y_i) \in \wt S_M} \error(\ff{i}(x_i), y_i ) }{\abs{\wt \calS_M}}\label{eq:LOO_error_multi_final} \,.
    \end{align}
    
    We now relate RHS in \eqref{eq:LOO_error_final} 
    with the population error on mislabeled distribution. 
    Similar as before, to do this, we leverage \codref{cond:hypothesis_stability} 
    and \lemref{lem:stability_error}. Using  \eqref{eq:final_mislabeled_linear} and \eqref{eq:LOO_error_multi_final}, we have 
    \begin{align}
        \error_{\wt \calS_M } (\wh f) \le \error_{\calDm}(\wh f)   + \sqrt{\frac{1}{\delta}\left(\frac{1}{2m_1} +\frac{3\beta}{m+n} \right)} \,. \label{eq:linear_multi_parallel_lem1}
    \end{align}
    
    We have now derived a parallel to \eqref{eq:lemma1_final_multi_prev}. Using the same arguments in the proof of \lemref{lem:fit_mislabeled_multi}, we have 
    \begin{align}
      \error_{\calD}(\wh f) \le  (k-1) \left( 1- \error_{ \wt \calS_M}(\wh f) \right)  + (k-1)\sqrt{\frac{k}{\delta(k-1)}\left(\frac{1}{2m_1} +\frac{3\beta}{m+n} \right)}  \,. \label{eq:lemma1_linear_final_multi}
    \end{align}
    
    \textbf{Part-2:} We now combine the results in \lemref{lem:mislabeled_error_multi} and \lemref{lem:clear_error_multi} to obtain the final inequality in terms of quantities that can be computed from just the randomly labeled and clean data. Similar to the binary case, we obtained a polynomial concentration instead of exponential concentration. Combining \eqref{eq:lemma1_linear_final_multi} with \lemref{lem:mislabeled_error_multi} and \lemref{lem:clear_error_multi}, we have with probability at least $1-\delta$
    \begin{align*}
        \error_\calD(\widehat f) \le \error_\calS(\widehat f) &+ (k-1)\left(1 - \frac{k}{k-1} \error_{\wt\calS}(\widehat f) \right) \\ &+ \left(k + \sqrt{k} + \frac{m}{n\sqrt{k}} \right) \sqrt{\frac{\log(4/\delta)}{2m}} + \sqrt{k(k-1)} \sqrt{\frac{4}{\delta}\left(\frac{1}{m} +\frac{3\beta}{m+n} \right)}  \,. \numberthis \label{eq:gd_multi_error_proof}
    \end{align*} 
\end{proof}

\subsection{Discussion on \codref{cond:hypothesis_stability}} \label{app:discuss_cond1}
The quantity in LHS of \codref{cond:hypothesis_stability} 
measures how much the function learned by the algorithm 
(in terms of error on unseen point) will change 
when one point in the training set is removed. 
% Discussion on exponential concentration and stronger condition. 
% Notice that hypothesis stability implies error stability, i.e., \codref{cond:error_stability} \citep{bousquet2002stability}.  
% In summary, while error stability allowed us 
% to relate the average population error 
% of the leave-one-out classifiers 
% with the population error of the original classifier, 
We need hypothesis stability condition 
to control the variance of the empirical leave-one-out error to show concentration of average leave-one-error with the population error. 

Additionally, we note that while the dominating term in the RHS of \thmref{thm:linear} matches with the dominating term in ERM bound in \thmref{thm:error_ERM}, there is a polynomial concentration term 
(dependence on $1/\delta$ instead of $\log(\sqrt{1/\delta})$) 
in \thmref{thm:linear}. 
Since with hypothesis stability, 
we just bound the variance, 
the polynomial concentration is due 
to the use of Chebyshev's inequality 
instead of an exponential tail inequality
(as in \lemref{lem:fit_mislabeled}).
Recent works have highlighted that 
a slightly stronger condition than hypothesis stability 
can be used to obtain an exponential concentration 
for leave-one-out error \citep{abou2019exponential},
but we leave this for future work for now. 
% We leave 
% However, the constants 

% we also want to highlight  

\subsection{Formal statement and proof of \propref{prop:early_stop}} \label{app:formal_early_stop}

Before formally presenting the result, 
we will introduce some notation.  
By $\calL_{S}(w)$, we denote 
the objective in \eqref{eq:l2_MSE_app} with $\lambda=0$. 
Assume Singular Value Decomposition (SVD) of $\bX$
as $\sqrt{n} \bU \bS^{1/2} \bV^T$. 
Hence $\bX^T \bX = \bV \bS \bV^T$.
Consider the GD iterates defined in \eqref{eq:GD_iterates_app}. 
% 
We now derive closed form expression 
for the $t^\text{th}$ iterate of gradient descent:  
% 
\begin{align}
    w_t = w_{t-1} + \eta \cdot \bX^T (\by - \bX w_{t-1}) = (\bI - \eta \bV \bS \bV^T )w_{k-1} + \eta \bX^T \by \,.
\end{align}
Rotating by $\bV^T$, we get 
\begin{align}
    \wt w_t = (\bI - \eta\bS )\wt w_{k-1} + \eta \wt \by \label{eq:GD_recur},
\end{align}
where $\wt w_t = \bV^T w_t $ and $\wt \by = \bV^T \bX^T \by$. 
Assuming the initial point $w_0 = 0$ 
and applying the recursion in \eqref{eq:GD_recur}, we get
\begin{align}
    \wt w_t = \bS ^{-1} ( \bI - (\bI - \eta \bS)^k ) \wt \by \,, 
\end{align} 
Projecting solution back to the original space, we have 
\begin{align}
     w_t = \bV \bS ^{-1} ( \bI - (\bI - \eta \bS)^k ) \bV^T \bX^T \by \,. 
\end{align} 
% We will work with this GD solution at any iterate $t$ in the next proposition. 
Define $f_t(x) \defeq f(x;w_t)$ 
as the solution at the $t^{\text{th}}$ iterate. 
Let $\wt w_{\lambda} = \argmin_{w} \calL_\calS (w;\lambda) = (\bX^T \bX + \lambda \bI)^{-1} \bX^T \by = \bV (\bS + \lambda \bI )^{-1} \bV^T \bX^T \by $. 
% ) \,,$ for all $t=1,2,\ldots\,.$ 
and define $\wt f_\lambda(x) \defeq f(x;\wt w_\lambda)$ as the regularized solution. 
Assume $\kappa$ be the condition number 
of the population covariance matrix 
and let $s_\text{min}$ be the minimum positive 
singular value of the empirical covariance matrix. 
Our proof idea is inspired from recent work 
on relating gradient flow solution 
and regularized solution 
for regression problems \citep{ali2018continuous}. 
We will use the following lemma in the proof: 
\begin{lemma} \label{lem:ineq_soln}
    For all $x \in [0,1]$ and for all $ k \in \mathbb{N}$, 
    we have (a) $ \frac{kx}{1+kx} \le 1- (1-x)^k$ 
    and (b) $ 1- (1-x)^k \le 2 \cdot \frac{kx}{kx+1} $.
    %  where $g(c)$ is a constant dependent on $c$. For $c = 1$, $g(c) = 2.0$.   
\end{lemma}
\begin{proof}
    % [Proof of \lemref{lem:ineq_soln}]
    % Part (a) is easy. 
    Using $ (1-x)^k \le \frac{1}{1+kx}$, we have part (a). 
    For part (b), we numerically maximize 
    $\frac{ (1+kx ) (1 - (1-x)^k) }{kx}$ 
    for all $k\ge 1$ and for all $x \in [0, 1]$.  
\end{proof}

% 
% Next, 

\begin{prop}[Formal statement of \propref{prop:early_stop}] \label{prop:formal_early_stop}
Let $\lambda = \frac{1}{t\eta}$. 
For a training point $x$, we have 
\begin{align*}
    \Expt{x \sim \calS}{(f_t(x) - \wt f_\lambda(x))^2} &\le c(t,\eta) \cdot \Expt{x \sim \calS}{f_t(x)^2} \,, %\label{eq:early_stop}
\end{align*}
where $c(t, \eta) \defeq \min( 0.25, \frac{1}{s_\text{min}^2 t^2 \eta^2})$. 
Similarly for a test point, we have 
\begin{align*}
    \Expt{x \sim \calD_\calX}{(f_t(x) - \wt f_\lambda(x))^2} &\le \kappa \cdot c(t,\eta) \cdot \Expt{x \sim \calD_\calX}{f_t(x)^2} \,. %\label{eq:early_stop}
\end{align*}
\end{prop} 

\begin{proof}
    %%%%%%%%%%%%% 
    We want to analyze the expected squared difference output 
    of regularized linear regression 
    with regularization constant $\lambda = \frac{1}{\eta t}$ 
    and the gradient descent solution at the $t^\text{th}$ iterate. 
    We separately expand the algebraic expression 
    for squared difference at a training point and a test point. 
    % We start by considering the difference  
    Then the main step is to show that 
    $\left[ \bS ^{-1} ( \bI - (\bI - \eta \bS)^k )  - (\bS + \lambda \bI )^{-1}\right] \preceq c(\eta, t) \cdot \bS ^{-1} ( \bI - (\bI - \eta \bS)^k ) $.

    %%%%%%%%%%%%%
    
   \textbf{Part 1 {} {}} 
    First, we will analyze the squared difference 
    of the output at a training point 
    (for simplicity, we refer to $S \cup \wt S$ as $S$), i.e., 
    \begin{align}
        \Expt{ x \sim \calS }{\left(f_t(x) - \wt f_\lambda (x)\right)^2} &= \norm{\bX w_t - \bX \wt w_\lambda}{2}^2\\ &=   \norm{\bX \bV \bS ^{-1} ( \bI - (\bI - \eta \bS)^t ) \bV^T \bX^T \by - \bX \bV (\bS + \lambda \bI )^{-1} \bV^T \bX^T \by }{2}^2 \\
        &= \norm{\bX \bV \left(\bS ^{-1} ( \bI - (\bI - \eta \bS)^t ) - (\bS + \lambda \bI )^{-1} \right) \bV^T \bX^T \by  }{2} \\
        &=  \by^T \bV \bX \left( \underbrace{\bS ^{-1} ( \bI - (\bI - \eta \bS)^t ) - (\bS + \lambda \bI )^{-1}}_{\RN{1}} \right)^2 \bS \bV^T \bX^T \by \label{eq:train_GD_rel} \,.
        %  (\bX \bV \bS ^{-1} ( \bI - (\bI - \eta \bS)^k ) \bV^T \bX^T \by)^T \bX \bV \bS ^{-1} ( \bI - (\bI - \eta \bS)^k ) \bV^T \bX^T \by
    \end{align}
    We now separately consider term 1. 
    Substituting $\lambda = \frac{1}{t \eta}$, 
    we get
    \begin{align}
        \bS ^{-1} ( \bI - (\bI - \eta \bS)^t ) - (\bS + \lambda \bI )^{-1} &= \bS^{-1} \left( ( \bI - (\bI - \eta \bS)^t ) - (\bI + \bS^{-1} \lambda )^{-1}\right) \\
        &= \underbrace{\bS^{-1} \left( ( \bI - (\bI - \eta \bS)^t ) - (\bI + ( \bS t \eta)^{-1}  )^{-1}\right)}_{\bA} \,.
    \end{align}

    We now separately bound the diagonal entries in matrix $\bA$. 
    With $s_i$, we denote $i^{\text{th}}$ diagonal entry of $\bS$.
    Note that since $ \eta\le 1/\norm{S}{\text{op}}$, 
    for all $i$, $\eta s_i  \le 1$.  
    Consider $i^{\text{th}}$ diagonal term (which is non-zero) 
    of the diagonal matrix $\bA$, we have 
    \begin{align}
        \bA_{ii} = \frac{1}{s_i} \left(  1 - (1 - s_i \eta)^t - \frac{t \eta s_i}{1 + t \eta s_i } \right) &=  \frac{1 - (1 - s_i \eta)^t}{s_i} \left( \underbrace{ 1 - \frac{t \eta s_i}{(1 + t \eta s_i)(1 - (1 - s_i \eta)^t)}}_{\RN{2}} \right) \\ 
         &\le \frac{1}{2}\left[ \frac{1 - (1 - s_i \eta)^t}{ s_i} \right] \tag*{(Using \lemref{lem:ineq_soln} (b))} \,.
    \end{align} 
    Additionally, we can also show the following upper bound on term 2: 
    \begin{align}
         1 - \frac{t \eta s_i}{(1 + t \eta s_i)(1 - (1 - s_i \eta)^t)} &= \frac{(1 + t \eta s_i)(1 - (1 - s_i \eta)^t) - t \eta s_i }{(1 + t \eta s_i)(1 - (1 - s_i \eta)^t)} \\
         & \le  \frac{ 1 -  (1 - s_i \eta)^t - t \eta s_i (1 - s_i \eta)^t}{(1 + t \eta s_i)(1 - (1 - s_i \eta)^t)} \\
         & \le \frac{1}{t\eta s_i} \,. \tag{Using \lemref{lem:ineq_soln} (a)}
        %  &\le \frac{1}{2}\left[ \frac{1 - (1 - s_i \eta)^t}{ s_i} \right] \tag*{(Using \lemref{lem:ineq_soln})} \,.
    \end{align} 

    Combining both the upper bounds 
    on each diagonal entry $\bA_{ii}$, we have 
    \begin{align}
    \bA \preceq c_1(\eta, t) \cdot \bS^{-1} ( \bI - (\bI - \eta \bS)^t ) \,, \label{eq:upperbound_diagonal}
    \end{align}
    where $c_1(\eta, t ) = \min(0.5, \frac{1}{t s_i \eta })$. Plugging this into \eqref{eq:train_GD_rel}, we have 
    \begin{align}
        \Expt{ x \sim \calS }{\left(f_t(x) - \wt f_\lambda (x)\right)^2} &\le c(\eta, t) \cdot \by^T \bV \bX  \left( \bS^{-1} ( \bI - (\bI - \eta \bS)^t ) \right)^2 \bS \bV^T \bX^T \by \\
        &=   c(\eta, t) \cdot \by^T \bV \bX  \left( \bS^{-1} ( \bI - (\bI - \eta \bS)^t ) \right) \bS \left( \bS^{-1} ( \bI - (\bI - \eta \bS)^t ) \right) \bV^T \bX^T \by \\
        & =  c(\eta, t) \cdot \norm{\bX w_t}{2}^2 \\
        &= c(\eta, t) \cdot  \Expt{ x \sim \calS }{\left(f_t(x) \right)^2} \,,
    \end{align}
    where $c(\eta, t ) = \min(0.25, \frac{1}{t^2 s^2_i \eta^2 })$.

    \textbf{Part 2 {} {}} With $\bSigma$, 
    we denote the underlying true covariance matrix. 
    We now consider the squared difference of output at an unseen point: 
    \begin{align}
        \Expt{ x \sim \calD_{\calX} }{\left(f_t(x) - \wt f_\lambda (x)\right)^2} &= \Expt{x \sim \calD_{\calX}}{\norm{x^T w_t - x^T \wt w_\lambda}{2}} \\
        &=   \norm{x^T \bV \bS ^{-1} ( \bI - (\bI - \eta \bS)^t ) \bV^T \bX^T \by - x^T \bV (\bS + \lambda \bI )^{-1} \bV^T \bX^T \by }{2} \\
        &= \norm{x^T \bV \left(\bS ^{-1} ( \bI - (\bI - \eta \bS)^t ) - (\bS + \lambda \bI )^{-1} \right) \bV^T \bX^T \by  }{2} \\
        &= \by^T \bV \bX \left( \bS ^{-1} ( \bI - (\bI - \eta \bS)^t ) - (\bS + \lambda \bI )^{-1} \right) \bV^T \bSigma \bV \\ &\qquad \qquad \qquad \qquad \qquad \left( (\bI - (\bI - \eta \bS)^t ) - (\bS + \lambda \bI )^{-1} \right) \bV^T \bX^T \by \\
        &\le \sigma_{\text{max}} \cdot \by^T \bV \bX \left( \underbrace{\bS ^{-1} ( \bI - (\bI - \eta \bS)^t ) - (\bS + \lambda \bI )^{-1}}_{\RN{1}} \right)^2 \bV^T \bX^T \by \,, \label{eq:test_GD_rel}
        %  (\bX \bV \bS ^{-1} ( \bI - (\bI - \eta \bS)^k ) \bV^T \bX^T \by)^T \bX \bV \bS ^{-1} ( \bI - (\bI - \eta \bS)^k ) \bV^T \bX^T \by
    \end{align}
    where $\sigma_{\text{max}}$ is the maximum eigenvalue 
    of the underlying covariance matrix $\bSigma$. 
    Using the upper bound on term 1 in \eqref{eq:upperbound_diagonal}, 
    we have 
    \begin{align}
        \Expt{ x \sim \calD_{\calX} }{\left(f_t(x) - \wt f_\lambda (x)\right)^2} &\le \sigma_{\text{max}} \cdot c(\eta, t) \cdot \by^T \bV \bX  \left( \bS^{-1} ( \bI - (\bI - \eta \bS)^t ) \right)^2 \bV^T \bX^T \by \\
        &=   \kappa \cdot c(\eta, t) \cdot \sigma_{\text{min}}\cdot \norm{\bV \left( \bS^{-1} ( \bI - (\bI - \eta \bS)^t ) \right) \bV^T \bX^T \by}{2}^2 \\
        &\le \kappa \cdot c(\eta, t) \cdot \left[ \bV \left( \bS^{-1} ( \bI - (\bI - \eta \bS)^t ) \right) \bV^T \bX^T \right]^T \bSigma \\
        &\qquad \qquad \qquad \qquad \qquad \left[ \bV \left( \bS^{-1} ( \bI - (\bI - \eta \bS)^t ) \right) \bV^T \bX^T \right] \by \\
        & = \kappa \cdot c(\eta, t) \cdot \Expt{x \sim \calD_{\calX}}{\norm{x^T w_t}{2}} \,.
    \end{align}
% 
% 
    % Since $ \eta\le 1/\norm{S}{\text{op}}$, invoking \lemref{lem:ineq_soln} to upper bound term 1 with
\end{proof}

\subsection{Extension to deep learning} \label{appsubsec:ext_DL}
Under \asmpref{appsubsec:justifying_assumption1}, we present the formal result parallel to \thmref{thm:multiclass_ERM}. 
\begin{theorem} \label{thm:multiclass_ERM_algoA}
    Consider a multiclass classification problem 
    with $k$ classes. Under \asmpref{asmp:deep_models}, 
    for any $\delta >0$, with probability at least $1-\delta$,
    we have
    \vspace{-10pt}
    \begin{align*}
        \error_\calD(\widehat f)  \le \error_\calS(\widehat f) + (k-1) \left(1 - \tfrac{k}{k-1} \error_{\wt\calS}(\widehat f)\right) + c\sqrt{\frac{\log(\frac{4}{\delta})}{2m}} \,,\numberthis \label{eq:multiclass_ERM_deep}
    % \vspace{-20pt}
    \end{align*}
    for some constant $c \le ((c+1) k+\sqrt{k} + \frac{m}{n\sqrt{k}})$.
\end{theorem}

The proof follows exactly as in step (i) to (iii) in \thmref{thm:multiclass_ERM}.  

\subsection{Justifying~\asmpref{asmp:deep_models}} \label{appsubsec:justifying_assumption1}

Motivated by the analysis on linear models, we now discuss alternate (and weaker) conditions that imply \asmpref{asmp:deep_models}. 
We need hypothesis stability (\codref{cond:hypothesis_stability}) and the following assumption relating training error and leave-one-error: 

\begin{assumption} \label{asmp:loo_error}
Let $\wh f$ be a model obtained by training with algorithm $\calA$ on a mixture of clean $S$ and randomly labeled data $\wt S$. Then we assume we have 
\begin{align*}
    \error_{\wt \calS_M} (\wh f) \le  \error_{\text{LOO} (\wt S_M)} \,, 
\end{align*}
for all $(x_i, y_i) \in  \wt S_M$ where $\wh f_{(i)} \defeq f(\calA, S \cup {{}\wt S_M}_{(i)})$ and  $\error_{\text{LOO} (\wt S_M)} \defeq  \frac{\sum_{(x_i, y_i) \in \wt S_M} \error(\ff{i}(x_i), y_i ) }{\abs{\wt \calS_M}}$.  
\end{assumption}

% we assume this to extend our result (parallel to \thmref{thm:multi_linear}) for deep models. 
Intuitively, this assumption states that the error on a (mislabeled) datum $(x,y)$ included in the training set is less than the error on that datum $(x,y)$ obtained by a model trained on the training set $S - \{(x,y)\}$. We proved this for linear models trained with GD in the proof of \thmref{thm:multi_linear}. 
% 
\codref{cond:hypothesis_stability} with $\beta = \calO(1)$ and \asmpref{asmp:loo_error} together with \lemref{lem:stability_error} implies \asmpref{asmp:deep_models} with a polynomial residual term (instead of logarithmic in $1/\delta$): 
\begin{align}
     \error_{\calS_M} (\wh f) \le  \error_{\calDm}(\wh f)   + \sqrt{\frac{1}{\delta}\left(\frac{1}{m} +\frac{3\beta}{m+n} \right)} \,.
\end{align}
% Note that this  

\newpage 
\section{Additional experiments and details}\label{app:exp}
\newcommand\tab[1][1cm]{\hspace*{#1}}

\subsection{Datasets} \label{sec:app_dataset}

\textbf{Toy Dataset {} {}} Assume fixed constants $\mu$ and $\sigma$. For a given label $y$, we simulate features $x$ in our toy classification setup as follows: 
\begin{align*}
    x \defeq \texttt{concat} \left[ x_1, x_2\right] \quad \text{where} \quad  x_1 \sim  \calN( y \cdot \mu, \sigma^2 I_{d \times d}) \ \  \text{and} \ \  x_1 \sim  \calN( 0, \sigma^2 I_{d \times d}) \,.
\end{align*}  
% where $y$ is the true label and $x$ is the corresponding feature vector. 
In experiements throughout the paper, we fix dimention $d=100$, $\mu = 1.0 $, and $\sigma = \sqrt{d}$. Intuitively, $x_1$ carries the information about the underlying label and $x_2$ is additional noise independent of the underlying label. 

\textbf{CV datasets {} {}} We use MNIST~\citep{lecun1998mnist} and CIFAR10~\cite{krizhevsky2009learning}. 
% For binary tasks, 
We produce a binary variant from the multiclass classification problem by mapping classes $\{0,1,2,3,4\}$ to label $1$ and $\{ 5,6,7,8,9\}$ to label $-1$. For CIFAR dataset, we also use the standard data augementation of random crop and horizontal flip. PyTorch code is as follows: 

\texttt{(transforms.RandomCrop(32, padding=4),\\
\tab transforms.RandomHorizontalFlip())}

\textbf{NLP dataset {} {}} We use IMDb Sentiment analysis~\citep{maas2011learning} corpus.  

\subsection{Architecture Details} 

All experiments were run on NVIDIA GeForce RTX 2080 Ti GPUs. We used PyTorch~\citep{NEURIPS2019a9015} and Keras with Tensorflow~\citep{abadi2016tensorflow} backend for experiments. 
% , ELMo embeddings~\citep{Peters:2018}, and Hugging Face Transformers~\citep{wolf-etal-2020-transformers}. 

\textbf{Linear model {} {}} For the toy dataset, we simulate a linear model with scalar output and the same number of parameters as the number of dimensions.   

\textbf{Wide nets {} {}} To simulate the NTK regime, we experiment with $2-$layered wide nets. The PyTorch code for 2-layer wide MLP is as follows: 


\texttt{ nn.Sequential( \\
\tab     nn.Flatten(),\\
\tab    nn.Linear(input\_dims, 200000, bias=True),\\
\tab    nn.ReLU(),\\
\tab    nn.Linear(200000, 1, bias=True)\\
\tab     )}


We experiment both (i) with the second layer fixed at random initialization; (ii)  and updating both layers' weights.     

\textbf{Deep nets for CV tasks {} {}} We consider a 4-layered MLP. The PyTorch code for 4-layer MLP is as follows: 

\texttt{ nn.Sequential(nn.Flatten(), \\
\tab        nn.Linear(input\_dim, 5000, bias=True),\\
\tab        nn.ReLU(),\\
\tab        nn.Linear(5000, 5000, bias=True),\\
\tab        nn.ReLU(),\\
\tab        nn.Linear(5000, 5000, bias=True),\\
\tab        nn.ReLU(),\\
% \tab        nn.Linear(5000, 5000, bias=True),\\
% \tab        nn.ReLU(),\\
\tab        nn.Linear(1024, num\_label, bias=True)\\
\tab        )}

For MNIST, we use $1000$ nodes instead of $5000$ nodes in the hidden layer. 
% 
We also experiment with convolutional nets. In particular, we use ResNet18 \citep{he2016deep}. Implementation adapted from:  \url{https://github.com/kuangliu/pytorch-cifar.git}. 

\textbf{Deep nets for NLP {} {}} We use a simple LSTM model with embeddings intialized with ELMo embeddings~\citep{Peters:2018}. Code adapted from: \url{https://github.com/kamujun/elmo_experiments/blob/master/elmo_experiment/notebooks/elmo_text_classification_on_imdb.ipynb} 

We also evaluate our bounds with a BERT model. In particular, we fine-tune an off-the-shelf uncased BERT model~\citep{devlin2018bert}. Code adapted from Hugging Face Transformers~\citep{wolf-etal-2020-transformers}: \url{https://huggingface.co/transformers/v3.1.0/custom_datasets.html}. 


\subsection{Additonal experiments}

\textbf{Results with SGD on underparameterized linear models {} {}} 

\begin{figure*}[h]
    \centering 
    % \vspace{-15pt}
    % \includegraphics[width=0.9\linewidth]{example-image-a}
    \includegraphics[width=0.3\linewidth]{figures/lowdim-Gaussian-SGD.pdf}
    % \includegraphics[width=0.9\linewidth]{figures/{CIFAR10_rn=0.1_lr=0.2_wd=0.005}.png}
    \vspace{-5pt}
    \caption{ 
    % Predicted lower bound 
    % on different
    We plot the accuracy and corresponding bound 
    (RHS in \eqref{eq:erm}) at $\delta = 0.1$
    for toy binary classification task. 
    Results aggregated over $3$ seeds. 
    % i.e., $1-\error$ where $\error$ is the term in the RHS of \eqref{eq:erm}
    Accuracy vs fraction of unlabeled data (w.r.t clean data) 
    in the toy setup with a linear model trained with SGD. Results parallel to \figref{fig:error_binary}(a) with SGD.  }
    \label{fig:error_binary_linear}
    \vspace{-5pt}
\end{figure*}

\textbf{Results with wide nets on binary MNIST {} {}}

\begin{figure*}[h]
    \centering 
    % \vspace{-15pt}
    % \includegraphics[width=0.9\linewidth]{example-image-a}
    \subfigure[GD with MSE loss]{\includegraphics[width=0.3\linewidth]{figures/MNIST-GD_MSE.pdf}} \hfil
    \subfigure[SGD with CE loss]{\includegraphics[width=0.3\linewidth]{figures/MNIST-SGD_CE.pdf}}
    \subfigure[SGD with MSE loss]{\includegraphics[width=0.3\linewidth]{figures/MNIST-SGD_MSE-first-layer.pdf}}
    % \includegraphics[width=0.9\linewidth]{figures/{CIFAR10_rn=0.1_lr=0.2_wd=0.005}.png}
    \vspace{-5pt}
    \caption{ 
    % Predicted lower bound 
    % on different
    We plot the accuracy and corresponding bound 
    (RHS in \eqref{eq:erm}) at $\delta = 0.1$ 
    for binary MNIST classification. 
    Results aggregated over $3$ seeds. 
    % i.e., $1-\error$ where $\error$ is the term in the RHS of \eqref{eq:erm}
    Accuracy vs fraction of unlabeled data 
    for a 2-layer wide network on binary MNIST with both the layers training in (a,b) and only first layer training in (c). 
    Results parallel to \figref{fig:error_binary}(b) .  }
    \label{fig:error_binary_MNIST}
    \vspace{-5pt}
\end{figure*}

% \begin{figure*}[h]
%     \centering 
%     % \vspace{-15pt}
%     % \includegraphics[width=0.9\linewidth]{example-image-a}
%     \subfigure[GD with MSE loss]{\includegraphics[width=0.3\linewidth]{figures/MNIST.pdf}} \hfil
    
%     \subfigure[SGD with CE loss]{\includegraphics[width=0.3\linewidth]{figures/MNIST.pdf}}
%     % \includegraphics[width=0.9\linewidth]{figures/{CIFAR10_rn=0.1_lr=0.2_wd=0.005}.png}
%     \vspace{-5pt}
%     \caption{ 
%     % Predicted lower bound 
%     % on different
%     We plot the accuracy and corresponding bound 
%     (RHS in \eqref{eq:erm}) at $\delta = 0.1$
%     for binary MNIST classification. 
%     Results aggregated over $3$ seeds. 
%     % i.e., $1-\error$ where $\error$ is the term in the RHS of \eqref{eq:erm}
%     Accuracy vs fraction of unlabeled data 
%     for a 2-layer wide network on binary MNIST with just the first layer training. 
%     Results parallel to \figref{fig:error_binary}(b) with only the first layer training.  }
%     \label{fig:error_binary_MNIST}
%     \vspace{-5pt}
% \end{figure*}

\textbf{Results on CIFAR 10 and MNIST {} {}} 
% 
We plot epoch wise error curve for results in \tabref{table:multiclass}(\figref{fig:error_epoch_CIFAR10} and \figref{fig:error_epoch_MNIST}). We observe the same trend as in \figref{fig:error_CIFAR10}. Additionally, we plot an \emph{oracle bound} obtained by tracking the error on mislabeled data which nevertheless were predicted as true label. To obtain an exact emprical value of the oracle bound, we need underlying true labels for the randomly labeled data. 
% Note that our bound in \thmref{thm:multiclass_ERM}, lower bounds the accuracy as predicted by the oracle bound. 
While with just access to extra unlabeled data we cannot calculate oracle bound, we note that the oracle bound is very tight and never violated in practice underscoring an importamt aspect of generalization in multiclass problems. This highlight that even a stronger conjecture may hold in multiclass classification, i.e., error on mislabeled data (where nevertheless true label was predicted) lower bounds the population error on the distribution of mislabeled data and hence, the error on (a specific) mislabeled portion predicts the population accuracy on clean data. 
% 
On the other hand, the dominating term of in \thmref{thm:multiclass_ERM} is loose when compared with the oracle bound. The main reason, we believe is the pessimistic upper bound in \eqref{eq:lemma1_final_multi_prev} in the proof of \lemref{lem:fit_mislabeled_multi}. We leave an investigation on this gap for future. 
% of fit 

% However, oracle bound highlights two . One,  



\begin{figure}[h]
    \centering 
    % \vspace{-15pt}
    % \includegraphics[width=0.9\linewidth]{example-image-a}
    \subfigure[MLP]{\includegraphics[width=0.3\linewidth]{figures/CIFAR10-FNN.pdf}} \hfil
    \subfigure[ResNet]{\includegraphics[width=0.3\linewidth]{figures/CIFAR10-Resnet.pdf}}
    % \includegraphics[width=0.9\linewidth]{figures/{CIFAR10_rn=0.1_lr=0.2_wd=0.005}.png}
    % \vspace{-10pt}
    \caption{ Per epoch curves for CIFAR10 corresponding results in \tabref{table:multiclass}. As before, we just plot the dominating term in the RHS of \eqref{eq:multiclass_ERM} as predicted bound. Additionally, we also plot the predicted lower bound by the error on mislabeled data which nevertheless were predicted as true label. We refer to this as ``Oracle bound''. See text for more details. 
    % 
    % except for the stopping point. 
    % The bound predicted by RATT (RHS in \eqref{eq:multiclass_ERM}) is vacuous. 
    }\label{fig:error_epoch_CIFAR10}
    % \vspace{-15pt}
\end{figure}


\begin{figure}[h]
    \centering 
    % \vspace{-15pt}
    % \includegraphics[width=0.9\linewidth]{example-image-a}
    \subfigure[MLP]{\includegraphics[width=0.3\linewidth]{figures/MNIST-FNN.pdf}} \hfil
    \subfigure[ResNet]{\includegraphics[width=0.3\linewidth]{figures/MNIST-Resnet.pdf}}
    % \includegraphics[width=0.9\linewidth]{figures/{CIFAR10_rn=0.1_lr=0.2_wd=0.005}.png}
    % \vspace{-10pt}
    \caption{ Per epoch curves for MNIST corresponding results in \tabref{table:multiclass}. As before, we just plot the dominating term in the RHS of \eqref{eq:multiclass_ERM} as predicted bound. Additionally, we also plot the predicted lower bound by the error on mislabeled data which nevertheless were predicted as true label. We refer to this as ``Oracle bound''. See text for more details. 
    % 
    % except for the stopping point. 
    % The bound predicted by RATT (RHS in \eqref{eq:multiclass_ERM}) is vacuous. 
    }\label{fig:error_epoch_MNIST}
    % \vspace{-15pt}
\end{figure}

\textbf{Results on CIFAR 100 {} {}} 
% 
On CIFAR100, our bound in \eqref{eq:multiclass_ERM} yields vacous bounds. However, the oracle bound as explained above yields tight guarantees in the initial phase of the learning (i.e., when learning rate is less than $0.1$) (\figref{fig:error_CIFAR100}).  

\begin{figure}[h]
    \centering 
    % \vspace{-15pt}
    % \includegraphics[width=0.9\linewidth]{example-image-a}
    \includegraphics[width=0.3\linewidth]{figures/CIFAR100-Resnet.pdf}
    % \includegraphics[width=0.9\linewidth]{figures/{CIFAR10_rn=0.1_lr=0.2_wd=0.005}.png}
    % \vspace{-10pt}
    \caption{ Predicted lower bound by the error on mislabeled data which nevertheless were predicted as true label with ResNet18 on CIFAR100. We refer to this as ``Oracle bound''. See text for more details. 
    % 
    % except for the stopping point. 
    The bound predicted by RATT (RHS in \eqref{eq:multiclass_ERM}) is vacuous. 
    }\label{fig:error_CIFAR100}
    % \vspace{-15pt}
\end{figure}


% \paragraph{Experiments on CIFAR100} 


% \subsection{Model Selection using RATT}


\subsection{Hyperparameter Details}


\textbf{\figref{fig:error_CIFAR10} {} {}} We use clean training dataset of size $40,000$. We fix the amount of unlabeled data at $20\%$ of the clean size, i.e. we include additional $8,000$ points with randomly assigned labels. We use test set of $10,000$ points. For both MLP and ResNet, we use SGD with an initial learning rate of $0.1$ and momentum $0.9$. We fix the weight decay parameter at $5\times 10^{-4}$. After $100$ epochs, we decay the learning rate to $0.01$. We use SGD batch size of $100$. 

\textbf{\figref{fig:error_binary} (a) {} {}} We obtain a toy dataset according to the process described in \secref{sec:app_dataset}. We fix $d=100$ and create a dataset of $50,000$ points with balanced classes. Moreover, we sample additional covariates with the same procedure to create randomly labeled dataset. For both SGD and GD training, we use a fixed learning rate $0.1$.    

\textbf{\figref{fig:error_binary} (b) {} {}} Similar to binary CIFAR, we use clean training dataset of size $40,000$ and fix the amount of unlabeled data at $20\%$ of the clean dataset size. To train wide nets, we use a fixed learning of $0.001$ with GD and SGD. We decide the weight decay parameter and the early stopping point that maximizes our generalization bound (i.e. without peeking at unseen data ).  We use SGD batch size of $100$. 

\textbf{\figref{fig:error_binary} (c) {} {}} With IMDb dataset, we use a clean dataset of size $20,000$ and as before, fix the amount of unlabeled data at $20\%$ of the clean data. To train ELMo model, we use Adam optimizer with a fixed learning rate $0.01$ and weight decay $10^{-6}$ to minimize cross entropy loss. We train with batch size $32$ for 3 epochs. To fine-tune BERT model, we use Adam optimizer with learning rate $5\times 10^{-5}$ to minimize cross entropy loss. We train with a batch size of $16$ for 1 epoch.    

\textbf{\tabref{table:multiclass} {} {}} For multiclass datasets, we train both MLP and ResNet with the same hyperparameters as described before. We sample a clean training dataset of size $40,000$ and fix the amount of unlabeled data at $20\%$ of the clean size. We use SGD with an initial learning rate of $0.1$ and momentum $0.9$. We fix the weight decay parameter at $5\times 10^{-4}$. After $30$ epochs for ResNet and after $50$ epochs for MLP, we decay the learning rate to $0.01$.  We use SGD with batch size $100$. 
For \figref{fig:error_CIFAR100}, we use the same hyperparameters as 
CIFAR10 training, except we now decay learning rate after $100$ epochs. 


In all experiments, to identify the best possible accuracy on just the clean data, we use the exact same set of hyperparamters except the stopping point. We choose a stopping point that maximizes test performance. 

\subsection{Summary of experiments }

\begin{center}
    \begin{table}[H] 
        \centering
        \begin{tabular}{|c|c|c|c|} 
        \hline
        Classification type & Model category & Model & Dataset  \\ [0.5ex] 
        \hline
        \hline
        \multirow{10}{*}{Binary} & Low dimensional & Linear model & Toy Gaussain dataset  \\
                        \cline{2-4}
                         & Overparameterized 
                        %  & Linear model & Toy Gaussain dataset \\
                        %  \cline{3-4}
                        %  & & 2-layer wide net& Toy Gaussain dataset \\
                        %  \cline{3-4}
                         & \multirow{2}{*}{2-layer wide net} & \multirow{2}{*}{Binary MNIST} \\
                         & linear nets & &  
                         \\
                         \cline{2-4}                 
                         & \multirow{6}{*}{Deep nets} & \multirow{2}{*}{MLP} & Binary MNIST \\
                         \cline{4-4}
                         & &  & Binary CIFAR \\
                         \cline{3-4}
                         &  & \multirow{2}{*}{ResNet} & Binary MNIST \\
                         \cline{4-4}
                         & &  & Binary CIFAR \\
                         \cline{3-4}
                         &  & ELMo-LSTM model & IMDb Sentiment Analysis \\
                         \cline{3-4}
                         & & BERT pre-trained model & IMDb Sentiment Analysis \\
        \hline
        \multirow{5}{*}{Multiclass} & \multirow{5}{*}{Deep nets} & \multirow{2}{*}{MLP} & MNIST \\
                        \cline{4-4} 
                        & & & CIFAR10 \\                   
                        \cline{3-4}
                         &   & \multirow{3}{*}{ResNet} & MNIST \\
                         \cline{4-4}
                         &   & & CIFAR10 \\
                         \cline{4-4}
                         &   & & CIFAR100 \\
        \hline
        \end{tabular}
        % \caption{Summary of experiments performed} \label{table:experiments}
    \end{table}    
    % \footnotetext[6]{We use both MSE loss and cross-entropy loss.}
    % \footnotetext[6]{We try 2 variants: one with a fixed first layer and the other with both layers trainable.}
\end{center}

\newpage
\section{Proof of \lemref{lem:stability_error}} \label{app:proof_lem_error}

\begin{proof}[Proof of \lemref{lem:stability_error}]
    Recall, we have a training set $S \cup \wt S_C$. We defined leave-one-out error on mislabeled points as $$\error_{\text{LOO}(\wt S_M) } = \frac{\sum_{(x_i, y_i) \in \wt S_M} \error( f_{(i)}( x_i), y_i)}{ \abs{\wt S_M }} \,, $$
    where $f_{(i)} \defeq f(\calA, (S \cup \wt S)_{(i)})$. Define $S^\prime \defeq S \cup \wt S$. Assume $(x,y)$ and $(x^\prime,y^\prime)$ as i.i.d. samples from ${\calDm}$. 
    Using Lemma 25 in \citet{bousquet2002stability}, we have
    \begin{align*}
        \Expo{ \left( \error_{\calDm}(\wh f) -\error_{\text{LOO}(\wt S_M) } \right)^2 } \le & \Expt{ S^\prime, (x,y), (x^\prime,y^\prime) }{ \error(\wh f(x), y ) \error(\wh f(x^\prime), y^\prime )} - 2 \Expt{ S^\prime, (x,y) }{ \error(\wh f(x), y ) \error(f_{(i)}(x_i), y_i )} \\
        & + \frac{m_1-1}{m_1}\Expt{ S^\prime }{  \error(f_{(i)}(x_i), y_i )  \error(f_{(j)}(x_j), y_j )} + \frac{1}{m_1} \Expt{ S^\prime }{  \error(f_{(i)}(x_i), y_i ) } \,. \numberthis \label{eq:main_reln}
    \end{align*}
    We can rewrite the equation above as : 
    \begin{align*}
        \Expo{ \left( \error_{\calDm}(\wh f) -\error_{\text{LOO}(\wt S_M) } \right)^2 } \le &  \, \underbrace{\Expt{ S^\prime, (x,y), (x^\prime,y^\prime) }{ \error(\wh f(x), y ) \error(\wh f(x^\prime), y^\prime ) - \error(\wh f(x), y ) \error(f_{(i)}(x_i), y_i )}}_{\RN{1}} \\
        & + \underbrace{\Expt{ S^\prime }{  \error(f_{(i)}(x_i), y_i )  \error(f_{(j)}(x_j), y_j ) -  \error(\wh f(x), y ) \error(f_{(i)}(x_i), y_i )}}_{\RN{2}} \\ &+ \underbrace{\frac{1}{m_1} \Expt{ S^\prime }{  \error(f_{(i)}(x_i), y_i ) - \error(f_{(i)}(x_i), y_i )  \error(f_{(j)}(x_j), y_j ) }}_{\RN{3}} \,. \numberthis \label{eq:main_reln2}
    \end{align*}
    
    We will now bound term $\RN{3}$.  Using Cauchy-Schwarz's inequality, we have
    
    \begin{align}
        \Expt{ S^\prime }{  \error(f_{(i)}(x_i), y_i ) - \error(f_{(i)}(x_i), y_i )  \error(f_{(j)}(x_j), y_j ) }^2 &\le  \Expt{ S^\prime }{  \error(f_{(i)}(x_i), y_i ) }^2 \Expt{S^\prime}{1 -   \error(f_{(j)}(x_j), y_j ) }^2 \\
        &\le \frac{1}{4} \,.\label{eq:term1_lem12}
    \end{align}
    
    Note that since $(x_i,y_i)$, $(x_j ,y_j )$, $(x,y)$, and $(x^\prime, y^\prime)$ are all from same distribution $\calDm$, we directly incorporate the bounds on term $\RN{1}$ and $\RN{2}$ from the proof of Lemma 9 in \citet{bousquet2002stability}. Combining that with \eqref{eq:term1_lem12} and our definition of hypothesis stability in \codref{cond:hypothesis_stability}, we have the required claim. 
    
    
    % We now re-write term $\RN{1}$ as
    % \begin{align*}
    %         &\Expt{S^\prime, (x,y), (x^\prime,y^\prime) }{ \error(\wh f(x), y ) \error(\wh f(x^\prime), y^\prime ) - \error(\wh f(x), y ) \error(f_{(i)}(x_i), y_i )} \\ & \qquad = \Expt{ S^\prime, (x,y), (x^\prime,y^\prime) }{ \error(\wh f(x), y ) \error(\wh f  (x^\prime), y^\prime ) - \error(\wh f ^\prime(x), y ) \error(f_{(i)}(x^\prime), y^\prime )} \tag{Exchanging $(x_i, y_i)$ with $(x^\prime, y^\prime)$ in the second term} \\
    %         & \qquad = \Expt{ S^\prime, (x,y), (x^\prime,y^\prime) }{  \left(\error(\wh f(x), y )-  \error(f_{(i)}(x), y ) \right) \error(\wh f  (x^\prime), y^\prime )  } \\
    %         & \qquad  + \Expt{ S^\prime, (x,y), (x^\prime,y^\prime) }{  \left(\error(f_{(i)}(x), y ) -\error(\wh f ^\prime(x), y ) \right) \error(\wh f  (x^\prime), y^\prime )}  \\
    %         & \qquad +\Expt{ S^\prime, (x,y), (x^\prime,y^\prime) }{  \left( \error(\wh f  (x^\prime), y^\prime ) -  \error(f_{(i)}(x^\prime), y^\prime ) \right) \error(\wh f ^\prime(x), y ) }  \,, \numberthis \label{eq:term1_final}
    % \end{align*}
    % where $\wh f^\prime$ is the classifier obtained by training on $ S^\prime_{(i)} \cup \{ (x^\prime, y^\prime) \} $. Similarly we can re-write term $\RN{2}$ as 
    % \begin{align*}
    %     & \Expt{ S^\prime }{  \error(f_{(i)}(x_i), y_i )  \error(f_{(j)}(x_j), y_j ) -  \error(\wh f(x), y ) \error(f_{(i)}(x_i), y_i )} \\
    %     &\quad  = \Expt{ S^\prime, (x,y), (x^\prime,y^\prime)}{  \error(f^{\prime\prime}_{(i)}(x), y )  \error(f_{(j)}^{\prime}(x^\prime), y^\prime ) -  \error(\wh f(x), y ) \error(f_{(i)}(x_i), y_i )} \tag{Exchanging $(x_i, y_i)$ with $(x, y)$ and $(x_j, y_j)$ with $(x^\prime, y^\prime)$ in the first term}\\
    %     &\quad = \Expt{ S^\prime, (x,y), (x^\prime,y^\prime)}{  \error(f^{\prime\prime}_{(j)}(x), y )  \error(f_{(i)}^{\prime}(x^\prime), y^\prime ) -  \error(\wh f^\prime (x), y ) \error(f^\prime_{(j)}(x^\prime), y^\prime )} \tag{Exchanging $(x_i, y_i)$ and $(x_j, y_j)$ and then replacing $(x_j, y_j)$ with $(x^\prime, y^\prime)$ in the second term} \\
    %     & \quad = \Expt{ S^\prime, (x,y), (x^\prime,y^\prime) }{  \left( \error(f_{(i)}^{\prime}(x^\prime), y^\prime )   -  \error(\wh f^{\prime\prime}  (x^\prime), y^\prime ) \right)  \error(f^{\prime\prime}_{(j)}(x), y )   } \\
    %     & \quad  + \Expt{ S^\prime, (x,y), (x^\prime,y^\prime) }{  \left( \error(f^{\prime\prime}_{(j)}(x), y )  -\error(\wh f ^\prime(x), y ) \right) \error(\wh f^{\prime\prime}  (x^\prime), y^\prime )  }  \\
    %     & \quad+ \Expt{ S^\prime, (x,y), (x^\prime,y^\prime) }{  \left( \error(\wh f^{\prime\prime}  (x^\prime), y^\prime )  -  \error(f^\prime_{(j)}(x^\prime), y^\prime ) \right)  \error(\wh f^\prime (x), y ) }   \\
    %     & \quad = \Expt{ S^\prime, (x,y), (x^\prime,y^\prime) }{  \left( \error(f_{(i)}^{\prime}(x^\prime), y^\prime )   -  \error(\wh f (x^\prime), y^\prime ) \right)  \error(f_{(i)}(x_j), y_j )   } \\
    %     & \quad  + \Expt{ S^\prime, (x,y), (x^\prime,y^\prime) }{  \left( \error(f^{\prime\prime}_{(j)}(x), y )  -\error(\wh f (x), y ) \right) \error(\wh f^{\prime\prime}  (x_j), y_j )  }  \\
    %     & \quad+ \Expt{ S^\prime, (x,y), (x^\prime,y^\prime) }{  \left( \error(\wh f^{\prime\prime}  (x^\prime), y^\prime )  -  \error(f^\prime_{(j)}(x^\prime), y^\prime ) \right)  \error(\wh f^\prime (x^\prime), y^\prime ) }  \,, \numberthis \label{eq:term2_final}
    % \end{align*}
    % where $f^{\prime\prime}_{(j)}$ is trained on $S^\prime_{(j,i)} \cup {(x,y)}$, $f^{\prime}_{(i)}$ is trained on $S^\prime_{(j,i)} \cup {(x^\prime,y^\prime)}$, and $\wh f^{\prime\prime} $ is trained on $S^\prime_{(j)} \cup {(x,y)}$. Note in the last line we replaced $(x,y)$ by $(x_j, y_j)$ in the first term, replaced $(x^\prime,y^\prime)$ by $(x_j, y_j)$ in the second term and exchanged $(x_i,y_i)$ with $(x_j,y_j)$ and also $(x,y)$ and $(x^\prime, y^\prime)$
    
    
\end{proof}


% 
% 16th Century Version Control 
% 

% \onecolumn

% \section*{Supplementary Material}
% We will be using the following standard results
% on exponential concentration of random variables 
% all throughout the discussion:

% \begin{lemma}[Hoeffding's inequality for independent RVs~\citep{hoeffding1994probability}] Let $Z_1, Z_2, \ldots, Z_n$ be independent bounded random variables with $Z_i \in [a,b]$ for all $i$, then 
%     \begin{align*}
%         \prob\left( \frac{1}{n} \sum_{i=1}^n (Z_i - \Expo{Z_i}) \ge t \right) \le \exp{\left( -\frac{2nt^2}{(b-a)^2} \right) }
%     \end{align*} 
%     and 
%     \begin{align*}
%         \prob\left( \frac{1}{n} \sum_{i=1}^n (Z_i - \Expo{Z_i}) \le -t \right) \le \exp{\left( -\frac{2nt^2}{(b-a)^2} \right) }
%     \end{align*} 
%     for all $t \ge 0$. 
% \end{lemma}

% \begin{lemma}[Hoeffding's inequality for sampling with replacement~\citep{hoeffding1994probability}] \label{lem:hoeffding_sampling} Let $\calZ = (Z_1, Z_2, \ldots, Z_N)$ be a finite population of $N$ points with $Z_i \in [a.b]$ for all $i$. Let $X_1, X_2, \ldots X_n$ be a random sample drawn without replacement from $\calZ$. Then for all $t \ge 0$, we have 
%     \begin{align*}
%         \prob\left( \frac{1}{n} \sum_{i=1}^n (X_i - \mu ) \ge t \right) \le \exp{\left( -\frac{2nt^2}{(b-a)^2} \right) }
%     \end{align*} 
%     and 
%     \begin{align*}
%         \prob\left( \frac{1}{n} \sum_{i=1}^n (X_i - \mu ) \le -t \right) \le \exp{\left( -\frac{2nt^2}{(b-a)^2} \right) } \,,
%     \end{align*} 
%     where $\mu = \frac{1}{N} \sum_{i=1}^{N} Z_i$. 
% \end{lemma}

% We now discuss one condition that generalizes the exponential concentration to dependent random variables.
% \begin{condition}[Bounded difference inequality] \label{cond:BDC} Let $\calZ$ be some set and $\phi: \calZ^n \to \Real$. We say that $\phi$ satisfies the bounded difference assumption if 
% there exists $c_1, c_2, \ldots c_n \ge 0$ s.t. for all $i$, we have 
% \begin{align*}
%     \sup_{Z_1,Z_2, \ldots,Z_n, Z_i^\prime in \calZ^{n+1} } \abs{\phi (Z_1, \ldots, Z_i, \ldots, Z_n ) - \phi (Z_1, \ldots, Z_i^\prime, \ldots, Z_n ) } \le c_i \,.
% \end{align*} 
% \end{condition}

% \begin{lemma}[McDiarmid’s inequality~\citep{mcdiarmid1989}] \label{lem:McDiarmid} Let $Z_1, Z_2, \ldots, Z_n$ be independent random variables on set $\calZ$ and $\phi : \calZ^n \to \Real$ satisfy bounded difference assumption (\codref{cond:BDC}). Then for all $t>0$, we have 
%     \begin{align*}
%         \prob\left( \phi(Z_1, Z_2, \ldots, Z_n) - \Expo{\phi(Z_1, Z_2, \ldots, Z_n)} \ge t \right) \le \exp{\left( -\frac{2t^2}{\sum_{i=1}^n c_i^2} \right) } 
%     \end{align*} 
%     and 
%     \begin{align*}
%         \prob\left( \phi(Z_1, Z_2, \ldots, Z_n) - \Expo{\phi(Z_1, Z_2, \ldots, Z_n)} \le -t \right) \le \exp{\left( -\frac{2t^2}{\sum_{i=1}^n c_i^2} \right) } \,
%     \end{align*} 
% \end{lemma}


% \section{Proofs from \secref{sec:ERM_training}}\label{app:proof_erm}

% \textbf{Additional notation {} {}} Let $m_1$ be the number of mislabeled points ($\wt S_M$) and $m_2$ be the number of correctly labeled points ($\wt S_C$). Note $m_1 + m_2 = m$. 


% \subsection{Proof of \thmref{thm:error_ERM}}


% \begin{proof}[Proof of \lemref{lem:fit_mislabeled}] 
%     The main idea of our proof is to regard 
%     the clean portion of the data 
%     ($S \cup \wt S_C$) as fixed.   
%     Then, there exists a classifier $f^*$ 
%     that is optimal over draws 
%     of the mislabeled data $\wt S_M$. 
% % 
%     % 
%     Formally, 
%     \begin{align}
%     f^* \defeq \argmin_{f \in \calF} \error_{\widecheck {\calD}} (f) \,, \label{eq:modified_ERM}
%     \end{align}
%     where $$\widecheck \calD = \frac{n}{m+n} \calS + \frac{m_1}{m+n} \wt \calS_C  + \frac{m_2}{m+n}\calDm \,.$$ That is, $\widecheck \calD$ a combination of 
%     the \emph{empirical distribution} 
%     over correctly labeled data $S \cup \wt S_C$
%     % in $S\cup \wt S$ 
%     and the (population) distribution 
%     over mislabeled data $\calDm$.
%     Recall that 
%     \begin{align}
%     \wh f \defeq \argmin_{f \in \calF} \error_{\calS \cup \wt S} (f) \,. \label{eq:orig_ERM}
%     \end{align}
%     % 
%     % 
%     Since, $\widehat f$ minimizes 0-1 error 
%     on $S \cup \wt S$, using ERM optimality on \eqref{eq:orig_ERM},  
%     we have 
%     \begin{align}
%         \error_{\calS \cup \wt \calS}(\widehat f) \le \error_{
%             \calS \cup \wt \calS}(f^*) \,.    \label{eq:step1}
%     \end{align}
%     Moreover, since $f^*$ is independent of $\wt S_M$, using Hoeffding's bound,
%     % \footnote{For a fully rigorous argument,
%     % refer to the complete proof in App.~\ref{app:proof_erm}.} 
%     we have with probability at least $1-\delta$ that
%     \begin{align}
%       \error_{\wt \calS_M}(f^*) \le \error_{ \calDm}(f^*) +  \sqrt{\frac{\log(1/\delta)}{2 m_1}} \,. \label{eq:step2} 
%     \end{align}
%     %$ 
%     %for some constant $c_1\le 1/2$. 
%     Finally, since $f^*$ is the optimal classifier on $\widecheck \calD$, 
%     we have 
%     \begin{align}
%         \error_{\widecheck \calD}(f^*) \le \error_{\widecheck \calD}(\widehat f) \label{eq:step3}
%     \end{align}
%      Now to relate \eqref{eq:step1} and \eqref{eq:step3}, we can re-write the \eqref{eq:step2} as follows: 
%     \begin{align}
%         \error_{\calS \cup \wt\calS}(f^*) \le \error_{ \widecheck \calD}(f^*) +  \frac{m_1}{m+n}\sqrt{\frac{\log(1/\delta)}{2 m_1}} \,. \label{eq:step4} 
%     \end{align}
%     Now we combine equations \eqref{eq:step1}, \eqref{eq:step4}, and \eqref{eq:step3}, to get 
%     \begin{align}
%         \error_{\calS \cup \wt \calS}(\wh f) \le \error_{\widecheck \calD}(\wh f) +  \frac{m_1}{m+n}\sqrt{\frac{\log(1/\delta)}{2 m_1}} \,, 
%     \end{align}
%     which implies 
%     \begin{align}
%         \error_{ \wt \calS_M}(\wh f) \le \error_{\calDm}(\wh f) + \sqrt{\frac{\log(1/\delta)}{2 m_1}} \,. \label{eq:lemma1_final}
%     \end{align}
%     Since $\wt S$ is obtained by randomly labeling an unlabeled dataset, we assume $2m_1 \approx m$ \footnote{Formally, with probability at least $1-\delta$, we have  $(m - 2m_1)\le \sqrt{m\log(1/\delta)/2}$ }. Moreover, using $\error_{\calDm} = 1 - \error_{\calD}$ we obtain the desired result.   
%     % Combining the above steps and using the fact 
%     % that $\error_\calD = 1- \error_{\calDm} $, 
%     % we obtain the desired result.
% \end{proof}

% \begin{proof}[Proof of \lemref{lem:mislabeled_error}]
%     Recall $\error_{\wt S} (f) = \frac{m_1}{m} \error_{\wt S_M}(f) + \frac{m_2}{m} \error_{\wt S_C}(f)$. Hence, we have 
%     \begin{align}
%         2\error_{\wt S}(f) - \error_{\wt S_M}(f) - \error_{\wt S_C}(f) &= \left(\frac{2m_1}{m} \error_{\wt S_M}(f) - \error_{\wt S_M}(f)\right) + \left(\frac{2m_2}{m} \error_{\wt S_C}(f) - \error_{\wt S_C}(f)\right) \\ &= \left(\frac{2m_1}{m} - 1\right) \error_{\wt S_M}(f) + \left(\frac{2m_2}{m} - 1 \right)\error_{\wt S_C} (f) \,.
%     \end{align} 
%     Since the dataset is randomly labeled, with probability at least $1-\delta$, we have  $\left(\frac{2m_1}{m} - 1\right) \le \sqrt{\frac{\log(1/\delta)}{2m}}$. Similarly, we have with probability at least $1-\delta$, $\left(\frac{2m_2}{m} - 1\right) \le \sqrt{\frac{\log(1/\delta)}{2m}}$. Using union bound, we have with probability at least $1-\delta$
%     % \begin{align}
%     %     2\error_{\wt S} - \error_{\wt S_M}(f) - \error_{\wt S_C}(f) \le \sqrt{\frac{\log(2/\delta)}{2m}} \left(\error_{\wt S_M}(f) + \error_{\wt S_C}(f) \right) \le 2\sqrt{\frac{\log(2/\delta)}{2m}} \,. \label{eq:lemma2_final}
%     % \end{align}
%     \begin{align}
%         2\error_{\wt S} - \error_{\wt S_M}(f) - \error_{\wt S_C}(f) \le \sqrt{\frac{\log(2/\delta)}{2m}} \left(\error_{\wt S_M}(f) + \error_{\wt S_C}(f) \right) \,. \label{eq:lemma2_prefinal}
%     \end{align}
%     With re-arranging $\error_{\wt S_M}(f) + \error_{\wt S_C}(f)$ and using the inequality $ 1- a\le \frac{1}{1+a} $, we have  
%     \begin{align}
%         2\error_{\wt S} - \error_{\wt S_M}(f) - \error_{\wt S_C}(f) \le 2\error_{\wt \calS} \sqrt{\frac{\log(2/\delta)}{2m}}  \,. \label{eq:lemma2_final}
%     \end{align}

%     % We obtain the desired result by using 
% \end{proof}

% \begin{proof}[Proof of \lemref{lem:clear_error}]
% % Recall 0-1 error on each point  $(x,y) \in S \cup \wt S$ is given by $\I{ f(x)\ne y}$.
% In the set of correctly labeled points $S \cup \wt S_C$, we have $S$ as a random subset of $S \cup \wt S_C$. Hence, using Hoeffding's inequality for sampling without replacement (\lemref{lem:hoeffding_sampling}), we have with probability at least $1-\delta$
% \begin{align}
%     \error_{\wt \calS_c} (\wh f)- \error_{\calS \cup \wt \calS_C}( \wh f) \le  \sqrt{\frac{\log(1/\delta)}{2m_2}} \,.
% \end{align}
% Re-writing $\error_{\calS \cup \wt \calS_C}( \wh f)$ as $\frac{m_2}{m_2 + n} \error_{\wt \calS_C }(\wh f) + \frac{n}{m_2 + n} \error_{\calS }(\wh f)$, we have with probability at least $1-\delta$
% \begin{align}
%   \left(\frac{n}{n+m_2}\right) \left(\error_{\wt \calS_c} (\wh f)- \error_{\calS}( \wh f) \right) \le  \sqrt{\frac{\log(1/\delta)}{2m_2}} \,.
% \end{align}
% As before, assuming $2m_2 \approx m$, we have with probability at least $1-\delta$ 
% \begin{align}
%     \error_{\wt \calS_c} (\wh f)- \error_{\calS}( \wh f) \le \left(1+\frac{m_2}{n}\right)  \sqrt{\frac{\log(1/\delta)}{m}} \le 1.5 \sqrt{\frac{\log(1/\delta)}{m}} \,. \label{eq:lemma3_final}
% \end{align} 
% \end{proof}

% \begin{proof}[Proof of \thmref{thm:error_ERM}] 
%     Having established these core intermediate results, we can now combine above three lemmas to prove the main result. 
%     In particular, we bound the population error on clean data ($\error_\calD(\wh f)$) as follows:  
%     \begin{enumerate}[(i)]
%         \item First, use \eqref{eq:lemma1_final}, to obtain an upper bound on the population error on clean data, i.e., with probability at least $1-\delta/4$, we have
%         \begin{align}
%             \error_{ \calD} (\wh f) \le 1 - \error_{ \wt \calS_M}(\wh f) + \sqrt{\frac{\log(4/\delta)}{m}} \,. 
%         \end{align}
%         \item  Second, use \eqref{eq:lemma2_final}, to relate the error on the mislabeled fraction with error on clean portion of randomly labeled data and error on whole randomly labeled dataset, i.e., with probability at least $1-\delta/2$, we have 
%         \begin{align}
%             - \error_{\wt S_M}(f) \le \error_{\wt S_C}(f) - 2\error_{\wt S}  + \sqrt{\frac{\log(4/\delta)}{2m}}  \,. 
%         \end{align} 
%         \item Finally, use \eqref{eq:lemma3_final} to relate the error on the clean portion of randomly labeled data and error on clean training data, i.e., with probability $1-\delta/4$, we have 
%         \begin{align}
%             \error_{\wt \calS_C} (\wh f)\le - \error_{\calS}( \wh f) + \left(1 + \frac{m}{2n} \right) \sqrt{\frac{\log(4/\delta)}{m}} \,. 
%         \end{align} 
%     \end{enumerate}

%     Using union bound on the above three steps, we have with probability at least $1-\delta$: 
%     \begin{align}
%         \error_\calD (\wh f) \le \error_{\calS}(\wh f)   + 1 - 2\error_{\wt \calS}(\wh f)   + (1/\sqrt{2} + 2.5)  \sqrt{\frac{\log(4/\delta)}{m}} \,.
%     \end{align}
%     Note that $(1/\sqrt{2} + 2.5)$ is a loose constant. In experiments, we use the ratio $\frac{m}{n}$
%     %  the exact error $\error_{\wt \calS}(\wh f)$ 
%     to evaluate R.H.S.    
% \end{proof}

% \subsection{Proof of \propref{prop:rademacher}}

% \begin{proof}[Proof of \propref{prop:rademacher}]
%     For a classifier $ f: \calX \to \{-1, 1\}$, we have $1 - 2\,\indict{ f(x) \ne y} = y \cdot f(x)$. Hence, by definition of $\error$, we have 
%     \begin{align}
%         1 -2\error_{\wt \calS}(f) = \frac{1}{m}\sum_{i=1}^m y_i \cdot f(x_i) \le \sup_{f \in \calF} \, \frac{1}{m} \sum_{i=1}^m y_i \cdot f(x_i)  \,. \label{eq:error_rademacher}
%     \end{align}
%     Note that for fixed inputs $(x_1, x_2, \ldots, x_m)$ in $\wt S$, $(y_1, y_2, \ldots y_m)$ are random labels. Define $\phi_1 (y_1, y_2, \ldots, y_m) \defeq \sup_{f \in \calF} \, \frac{1}{m} \sum_{i=1}^m y_i \cdot f(x_i)$. We have the following bounded difference condition on $\phi_1$. For all i, 
%     \begin{align}
%         \sup_{y_1, \ldots y_m, y_i^\prime \in \{-1, 1\}^{m+1} } \abs{ \phi_1 (y_1,\ldots, y_i, \ldots, y_m) - \phi_1 (y_1,\ldots, y_i^\prime, \ldots, y_m)  } \le 1/m \,. \label{cond1_rademacher}
%     \end{align} 
    
%     Similarly define $\phi_2 (x_1, x_2, \ldots, x_m) \defeq \Expt{ y_i \sim_U \{-1, 1\}  }{ \sup_{f \in \calF} \, \frac{1}{m}  \sum_{i=1}^m y_i \cdot f(x_i)}$. We have the following bounded difference condition on $\phi_2$. For all i,
%     \begin{align}
%         \sup_{x_1, \ldots x_m, x_i^\prime \in \calX^{m+1} } \abs{ \phi_2 (x_1,\ldots, x_i, \ldots, x_m) - \phi_1 (x_1,\ldots, x_i^\prime, \ldots, x_m)  } \le 1/m \,. \label{cond2_rademacher}
%     \end{align}
%     Using McDiarmid’s inequality (\lemref{lem:McDiarmid}) twice with Condition \eqref{cond1_rademacher} and \eqref{cond2_rademacher}, with probability at least $1-\delta$, we have
%     \begin{align}
%         \sup_{f \in \calF} \, \frac{1}{m} \sum_{i=1}^m y_i \cdot f(x_i)  - \Expt{x,y}{\sup_{f \in \calF} \, \frac{1}{m} \sum_{i=1}^m y_i \cdot f(x_i) } \le \sqrt{\frac{2\log(2/\delta)}{m}} \label{eq:final_rademacher}
%     \end{align} 
%     Combining \eqref{eq:error_rademacher} and \eqref{eq:final_rademacher}, we obtain the desired result. 
% \end{proof}


% \subsection{Proof of \thmref{thm:error_regularized_ERM}}

% Proof of \thmref{thm:error_regularized_ERM} follows similar to the proof of \thmref{thm:error_ERM}. Note that the same results in \lemref{lem:fit_mislabeled}, \lemref{lem:mislabeled_error}, and \lemref{lem:clear_error} hold in the regularized ERM case. However, the arguments in the proof of \lemref{lem:fit_mislabeled} changes slightly. Hence, we state and prove a lemma parallel to \lemref{lem:fit_mislabeled} for completeness. 

% \begin{lemma} \label{lem:lemma1_reg}
%     Assume the same setup as \thmref{thm:error_regularized_ERM}. 
%     Then for any $\delta >0$, with probability at least  $1-\delta$ 
%     over the random draws of mislabeled data $\wt S_M$, we have 
%     \begin{align}
%         \error_\calD(\widehat f)  \le 1 -\error_{\wt \calS_M}(\widehat f) + \sqrt{\frac{\log(1/\delta)}{m}}\,. 
%     \end{align} 
% \end{lemma}
% \begin{proof}
%     The main idea of the proof remains the same, i.e. regard 
%     the clean portion of the data 
%     ($S \cup \wt S_C$) as fixed.   
%     Then, there exists a classifier $f^*$ 
%     that is optimal over draws 
%     of the mislabeled data $\wt S_M$. 

    
%     Formally, 
%     \begin{align}
%     f^* \defeq \argmin_{f \in \calF} \error_{\widecheck {\calD}} (f)  + \lambda R(f) \,, \label{eq:modified_ERM_reg}
%     \end{align}
%     where $$\widecheck \calD = \frac{n}{m+n} \calS + \frac{m_1}{m+n} \wt \calS_C  + \frac{m_2}{m+n}\calDm \,.$$ That is, $\widecheck \calD$ a combination of 
%     the \emph{empirical distribution} 
%     over correctly labeled data $S \cup \wt S_C$
%     % in $S\cup \wt S$ 
%     and the (population) distribution 
%     over mislabeled data $\calDm$.
%     Recall that 
%     \begin{align}
%     \wh f \defeq \argmin_{f \in \calF} \error_{\calS \cup \wt S} (f) + \lambda R(f) \,. \label{eq:orig_ERM_reg}
%     \end{align}
%     % 
%     % 
%     Since, $\widehat f$ minimizes 0-1 error 
%     on $S \cup \wt S$, using ERM optimality on \eqref{eq:orig_ERM},  
%     we have 
%     \begin{align}
%         \error_{\calS \cup \wt \calS}(\widehat f) + \lambda R(\wh f) \le \error_{
%             \calS \cup \wt \calS}(f^*) + \lambda R(f^*) \,.    \label{eq:step1_reg}
%     \end{align}
%     Moreover, since $f^*$ is independent of $\wt S_M$, using Hoeffding's bound,
%     % \footnote{For a fully rigorous argument,
%     % refer to the complete proof in App.~\ref{app:proof_erm}.} 
%     we have with probability at least $1-\delta$ that
%     \begin{align}
%       \error_{\wt \calS_M}(f^*) \le \error_{ \calDm}(f^*) +  \sqrt{\frac{\log(1/\delta)}{2 m_1}} \,. \label{eq:step2_reg} 
%     \end{align}
%     %$ 
%     %for some constant $c_1\le 1/2$. 
%     Finally, since $f^*$ is the optimal classifier on $\widecheck \calD$, 
%     we have 
%     \begin{align}
%         \error_{\widecheck \calD}(f^*) + \lambda R(f^*) \le \error_{\widecheck \calD}(\widehat f) + \lambda R(\wh f) \label{eq:step3_reg}
%     \end{align}
%      Now to relate \eqref{eq:step1_reg} and \eqref{eq:step3_reg}, we can re-write the \eqref{eq:step2_reg} as follows: 
%     \begin{align}
%         \error_{\calS \cup \wt\calS}(f^*) \le \error_{ \widecheck \calD}(f^*) +  \frac{m_1}{m+n}\sqrt{\frac{\log(1/\delta)}{2 m_1}} \,. \label{eq:step4_reg} 
%     \end{align}
%     After adding $\lambda R(f^*)$ on both sides in \eqref{eq:step4_reg}, we combine equations \eqref{eq:step1_reg}, \eqref{eq:step4_reg}, and \eqref{eq:step3_reg}, to get 
%     \begin{align}
%         \error_{\calS \cup \wt \calS}(\wh f) \le \error_{\widecheck \calD}(\wh f) +  \frac{m_1}{m+n}\sqrt{\frac{\log(1/\delta)}{2 m_1}} \,, 
%     \end{align}
%     which implies 
%     \begin{align}
%         \error_{ \wt \calS_M}(\wh f) \le \error_{\calDm}(\wh f) + \sqrt{\frac{\log(1/\delta)}{2 m_1}} \,. \label{eq:lemma_reg_final}
%     \end{align}
%     Similar as before, since $\wt S$ is obtained by randomly labeling an unlabeled dataset, we assume 
%     $2m_1 \approx m$. Moreover, using $\error_{\calDm} = 1 - \error_{\calD}$ we obtain the desired result. 
% \end{proof}
% % \begin{proof}[Proof of ]
    
% % \end{proof}

% \subsection{Proof of \thmref{thm:multiclass_ERM}}

% We first state and prove lemmas parallel to three lemmas used in the proof of balanced binary case. Then we combine the results in the three lemmas to obtain the result in \thmref{thm:multiclass_ERM}. 

% Before stating the result, we define mislabeled distribution $\calDm$ for any $\calD$. While $\calDm$ and $\calD$ share 
% the same marginal distribution over $\calX$, 
% the distribution over labels $y$ 
% given an input $x\sim \calD_\calX$ is changed.
% In particular, for any $x$, the pdf over $y$ is changed to:  
% $p_{\calDm} (\cdot \vert x) \defeq \frac{1 - p_{\calD}(\cdot \vert x)}{k - 1}$.

% \begin{lemma} \label{lem:fit_mislabeled_multi}
%     Assume the same setup as \thmref{thm:multiclass_ERM}. 
%     Then for any $\delta >0$, with probability at least  $1-\delta$ 
%     over the random draws of mislabeled data $\wt S_M$, we have 
%     \begin{align}
%         \error_\calD(\widehat f)  \le (k-1)\left(1 -\error_{\wt \calS_M}(\widehat f)\right) + (k-1)\sqrt{\frac{\log(1/\delta)}{m}}\,. \label{eq:lemma1_multi}
%     \end{align}   
% \end{lemma} 

% \begin{proof}
%     The main idea of the proof remains the same, i.e. regard 
%     the clean portion of the data 
%     ($S \cup \wt S_C$) as fixed. 
%     Then, there exists a classifier $f^*$ 
%     that is optimal over draws 
%     of the mislabeled data $\wt S_M$. 
    
%     However, we need to be careful while relating population error on mislabeled data with population accuracy on clean data.   
%     While for binary classification,  we could upper bound $\error_{\wt \calS_M}$ 
%     with $1-\error_\calD$  (in the proof of \lemref{lem:fit_mislabeled}), 
%     for multiclass classification, 
%     error on the mislabeled data 
%     and accuracy on the clean data 
%     in the population 
%     are not so directly related.  
%     To establish \eqref{eq:lemma1_multi},
%     we break the error on the 
%     (unknown) mislabeled data 
%     into two parts: one term corresponds 
%     to predicting the true label on mislabeled data, 
%     and the other corresponds to predicting 
%     neither the true label 
%     nor the assigned (mis-)label.  
%     Finally, we relate these errors to their
%     population counterparts to establish \eqref{eq:lemma1_multi}. 
    
%     Formally, 
%     \begin{align}
%     f^* \defeq \argmin_{f \in \calF} \error_{\widecheck {\calD}} (f)  + \lambda R(f) \,, \label{eq:modified_ERM_reg2}
%     \end{align}
%     where $$\widecheck \calD = \frac{n}{m+n} \calS + \frac{m_1}{m+n} \wt \calS_C  + \frac{m_2}{m+n}\calDm \,.$$ That is, $\widecheck \calD$ a combination of 
%     the \emph{empirical distribution} 
%     over correctly labeled data $S \cup \wt S_C$
%     % in $S\cup \wt S$ 
%     and the (population) distribution 
%     over mislabeled data $\calDm$.
%     Recall that 
%     \begin{align}
%     \wh f \defeq \argmin_{f \in \calF} \error_{\calS \cup \wt S} (f) + \lambda R(f) \,. \label{eq:orig_ERM_reg2}
%     \end{align}
%     % 
%     % 
%     Following the exact steps from the proof of \lemref{lem:lemma1_reg}, with probability at least $1-\delta$, we have  
%     \begin{align}
%         \error_{ \wt \calS_M}(\wh f) \le \error_{\calDm}(\wh f) + \sqrt{\frac{\log(1/\delta)}{2 m_1}} \,. \label{eq:lemma1_final_multi_prev}
%     \end{align}
%     Similar to before, since $\wt S$ is obtained by randomly labeling an unlabeled dataset, we assume 
%     $\frac{k}{k-1} m_1 \approx m$. 
    
%     Now we will relate $\error_\calDm (\wh f)$ with $\error_{\calD}(\wh f)$. Let $y^T$ denote the (unknown) true label for a mislabeled point $(x, y)$ (i.e., label before replacing it with a mislabel). 
%     \begin{align}    
%          \Expt{(x, y) \in \sim \calDm}{\indict{ \wh f(x) \ne y }}  &= \underbrace{\Expt{(x, y) \in \sim \calDm}{\indict{ \wh f(x) \ne y \land \wh f(x) \ne y^T}}}_{\RN{1}} + \underbrace{\Expt{(x, y) \in \sim \calDm}{\indict{ \wh f(x) \ne y \land \wh f(x) = y^T}}}_{\RN{2}} \,. \label{eq:excess_term}
%     \end{align}
%     Clearly, term 2 is one minus the accuracy on the clean unseen data, i.e. 
%     \begin{align}
%         \RN{2} = 1 - \Expt{{x,y} \sim \calD}{ \indict{ \wh f(x) \ne y}} = 1- \error_{\calD}(\wh f) \,. \label{eq:term1}    
%     \end{align}
%     Next, we  relate term 1 with the error on the unseen clean data. We show that term 1 is equal to the error on the unseen clean data scaled by $\frac{k-2}{k-1}$ where $k$ is the number of labels. Using the definition of mislabeled distribution $\calDm$,  we have 
%     \begin{align}
%         \RN{1} = \frac{1}{k-1} \left( \Expt{(x, y) \in \sim \calD}{ \sum_{i \in \calY \land i\ne y}  \indict{ \wh f(x) \ne i \land \wh f(x) \ne y}} \right) = \frac{k-2}{k-1} \error_{\calD}(\wh f) \,.\label{eq:term2}
%     \end{align}    

%     Combining the result in \eqref{eq:term1}, \eqref{eq:term2} and \eqref{eq:excess_term}, we have 
%     \begin{align}
%         \error_{\calDm}(\wh f) = 1- \frac{1}{k-1} \error_{\calD}(\wh f) \,.\label{eq:combine_terms}
%     \end{align}
%     Finally, combining the result in \eqref{eq:combine_terms} with equation \eqref{eq:lemma1_final_multi_prev}, we have with probability $1-\delta$, 
%     \begin{align}
%       \error_{\calD}(\wh f) \le  (k-1) \left( 1- \error_{ \wt \calS_M}(\wh f) \right)  + (k-1) \sqrt{\frac{k \log(1/\delta)}{ 2(k-1)m}} \,. \label{eq:lemma1_final_multi}
%     \end{align}
% \end{proof}

% \begin{lemma} \label{lem:mislabeled_error_multi}
%     Assume the same setup as \thmref{thm:multiclass_ERM}.  Then for any $\delta >0$, with probability at least $1-\delta$ over the random draws of $\wt S$, we have  
%     % \begin{align}
%         $$\abs{k\error_{\wt \calS}(\widehat f) - \error_{\wt \calS_C}(\widehat f) -  (k-1)\error_{\wt \calS_M}(\widehat f) } \le  2k\sqrt{\frac{\log(4/\delta)}{2m}}\,. $$ % \label{eq:lemma2}
%     % \end{align}   
%     %  for some constant $c_3 \le 1.0\,$.
% \end{lemma} 


% \begin{proof}
%     Recall $\error_{\wt S} (f) = \frac{m_1}{m} \error_{\wt S_M}(f) + \frac{m_2}{m} \error_{\wt S_C}(f)$. Hence, we have 
%     \begin{align}
%         k\error_{\wt S}(f) - (k-1)\error_{\wt S_M}(f) - \error_{\wt S_C}(f) &= (k-1)\left(\frac{k m_1}{(k-1) m} \error_{\wt S_M}(f) - \error_{\wt S_M}(f)\right) + \left(\frac{km_2}{m} \error_{\wt S_C}(f) - \error_{\wt S_C}(f)\right) \\ &= k \left[ \left(\frac{m_1}{m} - \frac{k-1}{k}\right) \error_{\wt S_M}(f) + \left(\frac{m_2}{m} - \frac{1}{k} \right) \error_{\wt S_C} (f) \right] \,.
%     \end{align} 
%     Since the dataset is randomly labeled, we have with probability at least $1-\delta$, $\left(\frac{m_1}{m} - \frac{k-1}{k}\right) \le \sqrt{\frac{\log(1/\delta)}{2m}}$. Similarly, we have with probability at least $1-\delta$, $\left(\frac{m_2}{m} - \frac{1}{k}\right) \le \sqrt{\frac{\log(1/\delta)}{2m}}$. Using union bound, we have with probability at least $1-\delta$
%     % \begin{align}
%     %     2\error_{\wt S} - \error_{\wt S_M}(f) - \error_{\wt S_C}(f) \le \sqrt{\frac{\log(2/\delta)}{2m}} \left(\error_{\wt S_M}(f) + \error_{\wt S_C}(f) \right) \le 2\sqrt{\frac{\log(2/\delta)}{2m}} \,. \label{eq:lemma2_final}
%     % \end{align}
%     \begin{align}
%         k\error_{\wt S}(f) - (k-1)\error_{\wt S_M}(f) - \error_{\wt S_C}(f)  \le k \sqrt{\frac{\log(2/\delta)}{2m}} \left(\error_{\wt S_M}(f) + \error_{\wt S_C}(f) \right) \,. \label{eq:lemma2_final_multi}
%     \end{align}

%     % We obtain the desired result by using 
% \end{proof}

% \begin{lemma} \label{lem:clear_error_multi}
%     Assume the same setup as \thmref{thm:multiclass_ERM}. 
%     Then for any $\delta >0$, with probability at least $1-\delta$ 
%     over the random draws of $\wt S_C$ and $S$, we have 
%     % \begin{align}
%         $$\abs{\error_{\wt \calS_C}(\widehat f) - \error_{\calS}(\widehat f) } \le 1.5 \sqrt{\frac{k\log(2/\delta)}{2m}}\,.$$ %\label{eq:lemma3}
%     % \end{align}   
%     % for some constant $c_2 \le 1.2\,$.
% \end{lemma} 
% \begin{proof}
%     % Recall 0-1 error on each point  $(x,y) \in S \cup \wt S$ is given by $\I{ f(x)\ne y}$.
%     In the set of correctly labeled points $S \cup \wt S_C$, we have $S$ as a random subset of $S \cup \wt S_C$. Hence, using Hoeffding's inequality for sampling without replacement (\lemref{lem:hoeffding_sampling}), we have with probability at least $1-\delta$
%     \begin{align}
%         \error_{\wt \calS_c} (\wh f)- \error_{\calS \cup \wt \calS_C}( \wh f) \le  \sqrt{\frac{\log(1/\delta)}{2m_2}} \,.
%     \end{align}
%     Re-writing $\error_{\calS \cup \wt \calS_C}( \wh f)$ as $\frac{m_2}{m_2 + n} \error_{\wt \calS_C }(\wh f) + \frac{n}{m_2 + n} \error_{\calS }(\wh f)$, we have with probability at least $1-\delta$
%     \begin{align}
%       \left(\frac{n}{n+m_2}\right) \left(\error_{\wt \calS_c} (\wh f)- \error_{\calS}( \wh f) \right) \le  \sqrt{\frac{\log(1/\delta)}{2m_2}} \,.
%     \end{align}
%     As before, assuming $km_2 \approx m$, we have with probability at least $1-\delta$ 
%     \begin{align}
%         \error_{\wt \calS_c} (\wh f)- \error_{\calS}( \wh f) \le \left(1+\frac{m_2}{n}\right)  \sqrt{\frac{k\log(1/\delta)}{2m}} \le \left( 1 + \frac{1}{k}\right) \sqrt{\frac{k\log(1/\delta)}{2m}} \,. \label{eq:lemma3_final_multi}
%     \end{align} 
% \end{proof}

% \begin{proof}[Proof of \thmref{thm:multiclass_ERM}] 
%     Having established these core intermediate results, we can now combine above three lemmas. 
%     In particular, we bound the population error on clean data ($\error_\calD(\wh f)$) as follows:  
%     \begin{enumerate}[(i)]
%         \item First, use \eqref{eq:lemma1_final_multi}, to obtain an upper bound on the population error on clean data, i.e., with probability at least $1-\delta/4$, we have
%         \begin{align}
%             \error_{ \calD} (\wh f) \le (k-1)\left(1 - \error_{ \wt \calS_M}(\wh f) \right) + (k-1) \sqrt{\frac{k\log(4/\delta)}{2(k-1)m}} \,. 
%         \end{align}
%         \item  Second, use \eqref{eq:lemma2_final_multi}, to relate the error on the mislabeled fraction with error on clean portion of randomly labeled data and error on whole randomly labeled dataset, i.e., with probability at least $1-\delta/2$, we have 
%         \begin{align}
%             - (k-1)\error_{\wt S_M}(f) \le \error_{\wt S_C}(f) - k\error_{\wt S}  + k\sqrt{\frac{\log(4/\delta)}{2m}}  \,. 
%         \end{align} 
%         \item Finally, use \eqref{eq:lemma3_final_multi} to relate the error on the clean portion of randomly labeled data and error on clean training data, i.e., with probability $1-\delta/4$, we have 
%         \begin{align}
%             \error_{\wt \calS_C} (\wh f)\le - \error_{\calS}( \wh f) + \left(1 + \frac{m}{kn} \right) \sqrt{\frac{k\log(4/\delta)}{2m}} \,. 
%         \end{align} 
%     \end{enumerate}

%     Using union bound on the above three steps, we have with probability at least $1-\delta$: 
%     \begin{align}
%         \error_\calD (\wh f) \le \error_{\calS}(\wh f) + (k-1) - k\error_{\wt \calS}(\wh f)   + (\sqrt{k(k-1)} + k + \sqrt{k} + \frac{m}{n\sqrt{k}})  \sqrt{\frac{\log(4/\delta)}{2m}} \,.
%     \end{align}
%     % Note that $\frac{m}{n\sqrt{k}}$ is much smaller than the other terms in addition. Hence, we ignore this in the final bound. 
%     % Note that $(1/\sqrt{2} + 2.5)$ is a loose constant. In experiments, we use the ratio $\frac{m}{n}$
%     %  the exact error $\error_{\wt \calS}(\wh f)$ 
%     % to evaluate R.H.S.    
% \end{proof}

% \newpage
% \section{Proofs from \secref{sec:linear_models}}\label{app:proof_gd}

% We suppose that the parameters of the linear function 
% are obtained via gradient descent on 
% the following $L_2$ regularized problem: 
% \begin{align}
%     % n in denominator is avoided deliberately
%     \calL_S(w; \lambda) \defeq \sum_{i=1}^n{(w^Tx_i - y_i)^2} + \lambda \norm{w}{2}^2 \,, \label{eq:l2_MSE_app}   
% \end{align}
% where $\lambda\ge0$ is a regularization parameter. 
% We assume access to a clean dataset 
% $S = \{(x_i, y_i)\}_{i=1}^n \sim \calD^n$ 
% and randomly labeled dataset 
% $\wt S = \{(x_i, y_i)\}_{i=n+1}^{n+m} \sim \wt \calD^m$. 
% Let $\bX = [x_1, x_2, \cdots, x_{m+n}]$ 
% and $\by = [y_1, y_2, \cdots, y_{m+n}]$. 
% Fix a positive learning rate $\eta$ such that 
% $\eta \le 1/\left(\norm{\bX^T\bX}{\text{op}} + \lambda^2\right)$ 
% and an initialization $w_0 = 0$. 
% % \todos{Assumption made for simplicty}. 
% Consider the following gradient descent iterates 
% to minimize objective \eqref{eq:l2_MSE_app} on $S \cup \wt S$:
% \begin{align}
% w_t = w_{t-1} - \eta \grad_w \calL_{S \cup \wt S} (w_{t-1}; \lambda) \quad \forall t=1,2,\ldots \label{eq:GD_iterates_app}
% \end{align} 
% Then we have $\{ w_t\}$ converge to the limiting solution 
% $\wh w = \left( \bX^T\bX+\lambda \boldsymbol{I}\right)^{-1}\bX^T\by$. Define $\widehat f (x) \defeq f(x ; \wh w) $.  

% \subsection{\textcolor{red}{Errata}}

% We wish to correct the following error in the body: \codref{cond:error_stability} is not enough to guarantee the result in \thmref{thm:linear}. We now present a slightly stronger condition called \emph{hypothesis stability} under which we obtain a result similar to \thmref{thm:linear}. 

% This error doesn't change the main arguments of the proof where we show that the empirical train error is less than or equal to the leave-one-out error. We need a stronger condition to relate leave-one-out error with the population error of the original classifier. Specifically, while \codref{cond:error_stability} relates the average population error of leave-one-out classifiers with the population error of the original classifier, we need the new condition to show the concentration of the empirical leave-one-out error and  average population error of leave-one-out classifiers. 
% % main takeaway 

% Note that the new condition, while being stronger than the previous one, still doesn't imply generalization~\cite{bousquet2002stability,elisseeff2003leave,abou2019exponential}. Overall, the main results in \secref{sec:ERM_training} and takeaways of the paper remain unaffected by the error.  

% We now present the new condition and a corrected statement of \thmref{thm:linear}. Recall, for a given training set $S \sim \calD^n $, 
% we use $S_{(i)}$ to denote the training set $S$ 
% with the $i^{\text{th}}$ point removed.

% \begin{condition}[Hypothesis Stability] 
%     \label{cond:hypothesis_stability}
%     We have $\beta$ hypothesis stability 
%     if our training algorithm $\calA$ satisfies the following: 
%     \begin{align*}
%     % ${\sum_{i=1}^n \frac{\error_{\calD}( f(\calA, S_{(i)}))}{n} - \error_\calD(f(\calA, S))} \le \beta\,$.
%     \forall i \in \{1,2,\ldots, n\}, \quad  \Expt{\calS, (x,y) \in \calD}{ \abs{\error\left( f(x) ,y  \right) - \error\left( f_{(i)}(x), y \right) }} \le \frac{\beta}{n} \,,
%     \end{align*}
%     where $f_{(i)} \defeq f(\calA, S_{(i)})$ and $ f \defeq f(\calA, S)$.
% \end{condition}

% \begin{theorem}[Correct statement of \thmref{thm:linear}] \label{thm:new_linear}
%     Assume that this gradient descent algorithm satisfies \codref{cond:hypothesis_stability}
%     with $\beta=\calO(1)$.  
%     Then for any $\delta >0$, with probability at least $1-\delta$ 
%     over the random draws of datasets $\wt S$ and $S$, we have:
%     \begin{align}
%         \error_\calD(\widehat f) \le \error_\calS(\widehat f) + 1 - 2 \error_{\wt\calS}(\widehat f) + \left(\frac{1}{\sqrt{2}} + 1.5 \right) \sqrt{\frac{\log(4/\delta)}{m}} + \sqrt{\frac{4}{\delta}\left(\frac{1}{m} +\frac{3\beta}{m+n} \right)}  \,. \label{eq:gd_error}
%     \end{align} 
%     % for some constant $c\le 3.2$.
% \end{theorem}

% \subsection{Proof of \thmref{thm:new_linear}}
% We use a standard result from linear algebra, namely Shermann-Morrison formula~\citep{sherman1950adjustment} for matrix inversion:  

% \begin{lemma}[\citet{sherman1950adjustment}] \label{lem:sherman}
%     Suppose $\bA \in \Real^{n \times n}$ is an invertible square matrix and $u,v \in \Real^n$ are column vectors. Then $\bA + uv^T$ is invertible iff $1 + v^T \bA u \ne 0$ and in particular
%     \begin{align}
%         (\bA + u v^T)^{-1} = \bA^{-1}  - \frac{\bA^{-1} uv^T \bA^{-1} }{ 1 + v^T \bA^{-1} u} \,.
%     \end{align}   
% \end{lemma}
% \newcommand\byy[1]{\by_{\left(#1\right)}}
% \newcommand\bXX[1]{\bX_{\left(#1\right)}}
% \newcommand\ff[1]{\wh f_{\left(#1\right)}}

% For a given training set $S \cup \wt S_C$, define leave-one-out error on mislabeled points in the training data as $$\error_{\text{LOO}(\wt S_M) } = \frac{\sum_{(x_i, y_i) \in \wt S_M} \error( f_{(i)}( x_i), y_i)}{ \abs{\wt S_M }} \,, $$
% where $f_{(i)} \defeq f(\calA, (S \cup \wt S)_{(i)})$. To relate empirical leave-one-out error and population error with hypothesis stability condition, we use the following lemma:   

% \begin{lemma}[\citet{bousquet2002stability}] \label{lem:stability_error}
%     For the leave-one-out error, we have
%     \begin{align}
%         \Expo{ \left( \error_{\calDm}(\wh f) -\error_{\text{LOO}(\wt S_M) } \right)^2 } \le \frac{1}{2m_1}+  \frac{3\beta}{n + m}\,.
%     \end{align}   
%     % where $ f \defeq f(\calA, S \cup \wt S) $.
% \end{lemma}

% Proof of the above lemma is similar to the proof of  Lemma 9 in \citet{bousquet2002stability} and can be found in \appref{app:proof_lem_error}. 
% % 
% % Before presenting the result, we introduce some notation. 
% Before presenting the proof of \thmref{thm:new_linear}, we introduce some more notation. Let $\bX_{(i)}$ denote the matrix of covariates with $i^{\text{th}}$ point removed. Similarly let $\by_{(i)}$ be the array of responses with $i^{\text{th}}$ point removed. Define the corresponding regularized GD solution as $\wh w_{(i)} = \left( \bXX{i}^T\bXX{i}+\lambda \boldsymbol{I}\right)^{-1}\bXX{i}^T\byy{i}$. Define $\ff{i}(x) \defeq f(x ; \wh w_{(i)}) $.

% \begin{proof}[Proof of \thmref{thm:new_linear}]
%     Because squared loss minimization does not imply 0-1 error minimization, we cannot use arguments from \lemref{lem:fit_mislabeled}. This is the main technical difficulty. To compare the 0-1 error at a train point with an unseen point, 
%     we use the closed-form expression for $\widehat{w}$ and Shermann-Morrison formula to upper bound training error with leave-one-out cross validation error. 
    
%     The proof is divided into three parts: In part one, we show that 0-1 error on mislabeled points in the training set is lower than the error obtained by leave-one-out error at those points. In part two, we relate this leave-one-out error with the population error on mislabeled distribution using \codref{cond:hypothesis_stability}. While the empirical leave-one-out error is unbiased estimator of the average population error of leave-one-out classifiers, we need hypothesis stability to control the variance of empirical leave-one-out error. Finally in part three, we show that the error on the mislabeled training points can be estimated with just the randomly labeled and  clean training data (as in proof of \thmref{thm:error_ERM}).  

%     \textbf{Part 1 {} {}} First we relate training error with leave-one-out error.        
%     For any 
%     training point $(x_i, y_i)$ in $\wt S \cup S$, we have 
%     \begin{align}
%         \error(\wh f(x_i), y_i ) &= \indict{ y_i \cdot x_i^T \wh w < 0 } = \indict{ y_i \cdot x_i^T \left( \bX^T\bX+\lambda \boldsymbol{I}\right)^{-1}\bX^T\by < 0 } \\
%         &= \indict{ y_i \cdot x_i^T \underbrace{\left( \bXX{i}^T\bXX{i} + x_i ^T x_i +\lambda \boldsymbol{I}\right)^{-1}}_{\RN{1}} (\bXX{i}^T\byy{i} + y \cdot x_i) < 0 }
%     \end{align}
%     Letting $\bA = \left(\bXX{i}^T\bXX{i} +\lambda \boldsymbol{I}\right)$ and using \lemref{lem:sherman} on term 1, we have 
%     \begin{align}
%         \error(\wh f(x_i), y_i ) &= \indict{ y_i \cdot x_i^T \left[\bA^{-1} -  \frac{\bA^{-1} x_i x_i^T \bA^{-1}}{ 1 + x_i ^T \bA^{-1} x_i } \right] (\bXX{i}^T\byy{i} + y \cdot x_i) < 0 } \\
%         &= \indict{ y_i \cdot\left[ \frac{ x_i^T \bA^{-1} ( 1 + x_i ^T \bA^{-1} x_i ) -  x_i^T \bA^{-1} x_i x_i^T \bA^{-1}}{ 1 + x_i ^T \bA ^{-1}x_i } \right] (\bXX{i}^T\byy{i} + y \cdot x_i) < 0 } \\
%         &= \indict{ y_i \cdot\left[ \frac{ x_i^T \bA^{-1}}{ 1 + x_i ^T \bA ^{-1}x_i } \right] (\bXX{i}^T\byy{i} + y \cdot x_i) < 0 } \,.
%     \end{align}

%     Since $1 + x_i^T \bA^{-1} x_i > 0$, we have 
%     \begin{align}
%         \error(\wh f(x_i), y_i ) &= \indict{ y_i \cdot x_i^T \bA^{-1} (\bXX{i}^T\byy{i} + y \cdot x_i) < 0 } \\
%         &= \indict{ x_i^T \bA^{-1} x_i +  y_i \cdot x_i^T \bA^{-1} (\bXX{i}^T\byy{i}) < 0 } \\
%         &\le \indict{ y_i \cdot x_i^T \bA^{-1} (\bXX{i}^T\byy{i}) < 0 } = \error(\ff{i}(x_i), y_i ) \,.\label{eq:LOO_error}
%     \end{align}

%     Using \eqref{eq:LOO_error}, we have 
%     \begin{align}
%         \error_{\wt \calS_M } (\wh f) \le \error_{\text{LOO} (S_M)} \defeq \frac{\sum_{(x_i, y_i) \in \wt S_M} \error(\ff{i}(x_i), y_i ) }{\abs{\wt \calS_M}}\label{eq:LOO_error_final}
%     \end{align}
%     \textbf{Part 2 {}{}} We now relate RHS in \eqref{eq:LOO_error_final} with the population error on mislabeled distribution. To do this, we leverage \codref{cond:hypothesis_stability} and \lemref{lem:stability_error}. In particular, we have 

%     \begin{align}
%         \Expt{\calS \cup \wt \calS_M }{ \left(\error_{\calDm}(\wh f) - \error_{\text{LOO} (S_M)}\right)^2 } \le \frac{1}{2m_1} + \frac{3\beta}{m+n} \,.
%     \end{align}

%     Using Chebyshev's inequality, with probability at least $1-\delta$, we have 
%     \begin{align}
%         \error_{\text{LOO} (S_M)} \le  \error_{\calDm}(\wh f)   + \sqrt{\frac{1}{\delta}\left(\frac{1}{2m_1} +\frac{3\beta}{m+n} \right)} \,. \label{eq:final_mislabeled_linear}
%     \end{align}
    

%     \textbf{Part 3 {}{}} Combining \eqref{eq:final_mislabeled_linear} and \eqref{eq:LOO_error_final}, we have 

%     \begin{align}
%         \error_{\wt \calS_M } (\wh f) \le \error_{\calDm}(\wh f)   + \sqrt{\frac{1}{\delta}\left(\frac{1}{2m_1} +\frac{3\beta}{m+n} \right)} \,. \label{eq:linear_parallel_lem1}
%     \end{align}

%     Compare \eqref{eq:linear_parallel_lem1}, with \eqref{eq:lemma1_final} in the proof of \lemref{lem:fit_mislabeled}. We obtain a similar relationship between $\error_{\wt \calS_M }$ and $\error_{\calDm}$ but with a polynomial concentration instead of exponential concentration. 
%     In addition, since we just use concentration arguments to relate mislabeled error with the error on clean portion and unlabeled portion, we can directly use the results in \lemref{lem:mislabeled_error} and \lemref{lem:clear_error}. Therefore, combining results in \lemref{lem:mislabeled_error}, \lemref{lem:clear_error}, and \eqref{eq:linear_parallel_lem1} with union bound, we have with probability at least $1-\delta$

%     \begin{align}
%         \error_\calD(\widehat f) \le \error_\calS(\widehat f) + 1 - 2 \error_{\wt\calS}(\widehat f) + \left(\frac{1}{\sqrt{2}} + 1.5 \right) \sqrt{\frac{\log(4/\delta)}{m}} + \sqrt{\frac{4}{\delta}\left(\frac{1}{m} +\frac{3\beta}{m+n} \right)}  \,.
%     \end{align}
    

       
% \end{proof}

% \subsection{Discussion on \codref{cond:hypothesis_stability}}

% The quantity in LHS of \codref{cond:hypothesis_stability} measures how much the function learned by the algorithm (in terms of error on unseen point) will change when one point in the training set is removed. 
% % Discussion on exponential concentration and stronger condition. 
% Notice that hypothesis stability implies error stability, i.e., \codref{cond:error_stability} ~\cite{bousquet2002stability}.  In summary, while error stability allowed us to relate the average population error of the leave-one-out classifiers with the population error of the original classifier, we need hypothesis stability condition to control the variance of the empirical leave-one-out error. 

% Additionally, we note that while the dominating term in the RHS of \thmref{thm:new_linear} matches with the dominating term in ERM bound in \thmref{thm:error_ERM}, there is a polynomial concentration term (dependence on $1/\delta$ instead of $\log(\sqrt{1/\delta})$) in  \thmref{thm:new_linear}. 
% Since with hypothesis stability, we just bound the variance,  the polynomial concentration is due to the use of Chebyshev's inequality instead of an exponential tail inequality (as in \lemref{lem:fit_mislabeled}).
% Recent works have highlighted that slightly stronger condition than hypothesis stability can be used to obtained an exponential concentration for leave-one-out error~\citep{abou2019exponential}, but we leave this for future work for now. 
% % We leave 
% % However, the constants 

% % we also want to highlight  

% \subsection{Formal statement and proof of  of \propref{prop:early_stop}}

% Before formally presenting the result, we will introduce some notation.  By $\calL_{S}(w)$, we denote 
% the objective in \eqref{eq:l2_MSE_app} with $\lambda=0$. 
% Assume Singular Value Decomposition (SVD) of $\bX$  as $\sqrt{n} \bU \bS^{1/2} \bV^T$. Hence $\bX^T \bX = \bV \bS \bV^T$.
% Consider the GD iterates defined in \eqref{eq:GD_iterates_app}. 
% % 
% We now derive closed form expression for the $t^\text{th}$ iterate of gradient descent:  
% % 
% \begin{align}
%     w_t = w_{t-1} + \eta \cdot \bX^T (\by - \bX w_{t-1}) = (\bI - \eta \bV \bS \bV^T )w_{k-1} + \eta \bX^T \by \,.
% \end{align}
% Rotating by $\bV^T$, we get 
% \begin{align}
%     \wt w_t = (\bI - \eta\bS )\wt w_{k-1} + \eta \wt \by \,, \label{eq:GD_recur}
% \end{align}
% where $\wt w_t = \bV^T w_t $ and $\wt \by = \bV^T \bX^T \by$. Assuming the initial point $w_0 = 0$ and applying the recursion in \eqref{eq:GD_recur}, we get
% \begin{align}
%     \wt w_t = \bS ^{-1} ( \bI - (\bI - \eta \bS)^k ) \wt \by \,, 
% \end{align} 
% Projecting solution back to the original space, we have 
% \begin{align}
%      w_t = \bV \bS ^{-1} ( \bI - (\bI - \eta \bS)^k ) \bV^T \bX^T \by \,, 
% \end{align} 
% % We will work with this GD solution at any iterate $t$ in the next proposition. 
% Define $f_t(x) \defeq f(x;w_t)$ as the solution at the $t^{\text{th}}$ iterate. 
% Let $\wt w_{\lambda} = \argmin_{w} \calL_\calS (w;\lambda) = (\bX^T \bX + \lambda \bI)^{-1} \bX^T \by = \bV (\bS + \lambda \bI )^{-1} \bV^T \bX^T \by $. 
% % ) \,,$ for all $t=1,2,\ldots\,.$ 
% and define $\wt f_\lambda(x) \defeq f(x;\wt w_\lambda)$ as the regularized solution. 
% Assume $\kappa$ be the condition number of the population covariance matrix 
% and 
% let $s_\text{min}$ be the minimum positive singular value of the empirical covariance matrix. Our proof idea is inspired from recent work on relating gradient flow solution and regularized solution for regression problems \citep{ali2018continuous}. We will use the following lemma in the proof: 
% \begin{lemma} \label{lem:ineq_soln}
%     For all $x \in [0,1]$ and for all $ k \in \mathbb{N}$, we have (a) $ \frac{kx}{1+kx} \le 1- (1-x)^k$ and (b) $ 1- (1-x)^k \le 2 \cdot \frac{kx}{kx+1} $.
%     %  where $g(c)$ is a constant dependent on $c$. For $c = 1$, $g(c) = 2.0$.   
% \end{lemma}
% \begin{proof}
%     % [Proof of \lemref{lem:ineq_soln}]
%     % Part (a) is easy. 
%     Using $ (1-x)^k \le \frac{1}{1+kx}$, we have part (a). For part (b), we numerically maximize $\frac{ (1+kx ) (1 - (1-x)^k) }{kx}$ for all $k\ge 1$ and for all $x \in [0, 1]$.  
% \end{proof}

% % 
% % Next, 

% \begin{prop}[Formal statement of \propref{prop:early_stop}] \label{prop:formal_early_stop}
% Let $\lambda = \frac{1}{t\eta}$. For a training point $x$, we have 
% \begin{align*}
%     \Expt{x \sim \calS}{(f_t(x) - \wt f_\lambda(x))^2} &\le c(t,\eta) \cdot \Expt{x \sim \calS}{f_t(x)^2} \,, %\label{eq:early_stop}
% \end{align*}
% where $c(t, \eta) \defeq \min( 0.25, \frac{1}{s_\text{min}^2 t^2 \eta^2})$. Similarly for a test point, we have 
% \begin{align*}
%     \Expt{x \sim \calD_\calX}{(f_t(x) - \wt f_\lambda(x))^2} &\le \kappa \cdot c(t,\eta) \cdot \Expt{x \sim \calD_\calX}{f_t(x)^2} \,. %\label{eq:early_stop}
% \end{align*}
% \end{prop} 

% \begin{proof}
%     %%%%%%%%%%%%% 
%     We want to analyze the expected squared difference output of regularized linear regression with regularization constant $\lambda = \frac{1}{\eta t}$ and gradient descent solution at $t^\text{th}$ iterate. We separately expand the algebraic expression for squared difference at a training point and a test point. 
%     % We start by considering the difference  
%     Then the main step is to show that  $\left[ \bS ^{-1} ( \bI - (\bI - \eta \bS)^k )  - (\bS + \lambda \bI )^{-1}\right] \preceq c(\eta, t) \cdot \bS ^{-1} ( \bI - (\bI - \eta \bS)^k ) $.

%     %%%%%%%%%%%%%
    
%   \textbf{Part 1 {} {}} 
%     First, we will analyze the squared difference of output at a training point (for simplicity, we refer to $S \cup \wt S$ as $S$), i.e. 
%     \begin{align}
%         \Expt{ x \sim \calS }{\left(f_t(x) - \wt f_\lambda (x)\right)^2} &= \norm{\bX w_t - \bX \wt w_\lambda}{2}^2 =   \norm{\bX \bV \bS ^{-1} ( \bI - (\bI - \eta \bS)^t ) \bV^T \bX^T \by - \bX \bV (\bS + \lambda \bI )^{-1} \bV^T \bX^T \by }{2}^2 \\
%         &= \norm{\bX \bV \left(\bS ^{-1} ( \bI - (\bI - \eta \bS)^t ) - (\bS + \lambda \bI )^{-1} \right) \bV^T \bX^T \by  }{2} \\
%         &=  \by^T \bV \bX \left( \underbrace{\bS ^{-1} ( \bI - (\bI - \eta \bS)^t ) - (\bS + \lambda \bI )^{-1}}_{\RN{1}} \right)^2 \bS \bV^T \bX^T \by \label{eq:train_GD_rel}
%         %  (\bX \bV \bS ^{-1} ( \bI - (\bI - \eta \bS)^k ) \bV^T \bX^T \by)^T \bX \bV \bS ^{-1} ( \bI - (\bI - \eta \bS)^k ) \bV^T \bX^T \by
%     \end{align}
%     We now separately consider term 1. Substituting $\lambda = \frac{1}{t \eta}$, we get
%     \begin{align}
%         \bS ^{-1} ( \bI - (\bI - \eta \bS)^t ) - (\bS + \lambda \bI )^{-1} &= \bS^{-1} \left( ( \bI - (\bI - \eta \bS)^t ) - (\bI + \bS^{-1} \lambda )^{-1}\right) \\
%         &= \underbrace{\bS^{-1} \left( ( \bI - (\bI - \eta \bS)^t ) - (\bI + ( \bS t \eta)^{-1}  )^{-1}\right)}_{\bA}
%     \end{align}

%     We now separately bound the diagonal entries in matrix $\bA$. 
%     With $s_i$, we denote $i^{\text{th}}$ diagonal entry of $\bS$. Note that since $ \eta\le 1/\norm{S}{\text{op}}$, for all $i$, $\eta s_i  \le 1$.  Consider $i^{\text{th}}$ diagonal term (which is non-zero) of the diagonal matrix $\bA$, we have 
%     \begin{align}
%         \bA_{ii} = \frac{1}{s_i} \left(  1 - (1 - s_i \eta)^t - \frac{t \eta s_i}{1 + t \eta s_i } \right) &=  \frac{1 - (1 - s_i \eta)^t}{s_i} \left( \underbrace{ 1 - \frac{t \eta s_i}{(1 + t \eta s_i)(1 - (1 - s_i \eta)^t)}}_{\RN{2}} \right) \\ 
%          &\le \frac{1}{2}\left[ \frac{1 - (1 - s_i \eta)^t}{ s_i} \right] \tag*{(Using \lemref{lem:ineq_soln} (b))} \,.
%     \end{align} 
%     Additionally, we can also show the following upper bound on term 2: 
%     \begin{align}
%          1 - \frac{t \eta s_i}{(1 + t \eta s_i)(1 - (1 - s_i \eta)^t)} &= \frac{(1 + t \eta s_i)(1 - (1 - s_i \eta)^t) - t \eta s_i }{(1 + t \eta s_i)(1 - (1 - s_i \eta)^t)} \\
%          & \le  \frac{ 1 -  (1 - s_i \eta)^t - t \eta s_i (1 - s_i \eta)^t}{(1 + t \eta s_i)(1 - (1 - s_i \eta)^t)} \\
%          & \le \frac{1}{t\eta s_i} \,. \tag{Using \lemref{lem:ineq_soln} (a)}
%         %  &\le \frac{1}{2}\left[ \frac{1 - (1 - s_i \eta)^t}{ s_i} \right] \tag*{(Using \lemref{lem:ineq_soln})} \,.
%     \end{align} 

%     Combining both the upper bounds on each diagonal entry $\bA_{ii}$, we have 
%     \begin{align}
%     \bA \preceq c_1(\eta, t) \cdot \bS^{-1} ( \bI - (\bI - \eta \bS)^t ) \,, \label{eq:upperbound_diagonal}
%     \end{align}
%     where $c_1(\eta, t ) = \min(0.5, \frac{1}{t s_i \eta })$. Plugging this into \eqref{eq:train_GD_rel}, we have 
%     \begin{align}
%         \Expt{ x \sim \calS }{\left(f_t(x) - \wt f_\lambda (x)\right)^2} &\le c(\eta, t) \cdot \by^T \bV \bX  \left( \bS^{-1} ( \bI - (\bI - \eta \bS)^t ) \right)^2 \bS \bV^T \bX^T \by \\
%         &=   c(\eta, t) \cdot \by^T \bV \bX  \left( \bS^{-1} ( \bI - (\bI - \eta \bS)^t ) \right) \bS \left( \bS^{-1} ( \bI - (\bI - \eta \bS)^t ) \right) \bV^T \bX^T \by \\
%         & =  c(\eta, t) \cdot \norm{\bX w_t}{2}^2 \\
%         &= c(\eta, t) \cdot  \Expt{ x \sim \calS }{\left(f_t(x) \right)^2} \,,
%     \end{align}
%     where $c(\eta, t ) = \min(0.25, \frac{1}{t^2 s^2_i \eta^2 })$.

%     \textbf{Part 2 {} {}} With $\bSigma$, we denote the underlying true covariance matrix. We now consider the squared difference of output at an unseen point: 
%     \begin{align}
%         \Expt{ x \sim \calD_{\calX} }{\left(f_t(x) - \wt f_\lambda (x)\right)^2} &= \Expt{x \sim \calD_{\calX}}{\norm{x^T w_t - x^T \wt w_\lambda}{2}} \\
%         &=   \norm{x^T \bV \bS ^{-1} ( \bI - (\bI - \eta \bS)^t ) \bV^T \bX^T \by - x^T \bV (\bS + \lambda \bI )^{-1} \bV^T \bX^T \by }{2} \\
%         &= \norm{x^T \bV \left(\bS ^{-1} ( \bI - (\bI - \eta \bS)^t ) - (\bS + \lambda \bI )^{-1} \right) \bV^T \bX^T \by  }{2} \\
%         &= \by^T \bV \bX \left( \bS ^{-1} ( \bI - (\bI - \eta \bS)^t ) - (\bS + \lambda \bI )^{-1} \right) \bV^T \bSigma \bV \\ &\qquad \qquad \qquad \qquad \qquad \left( (\bI - (\bI - \eta \bS)^t ) - (\bS + \lambda \bI )^{-1} \right) \bV^T \bX^T \by \\
%         &\le \sigma_{\text{max}} \cdot \by^T \bV \bX \left( \underbrace{\bS ^{-1} ( \bI - (\bI - \eta \bS)^t ) - (\bS + \lambda \bI )^{-1}}_{\RN{1}} \right)^2 \bV^T \bX^T \by \,, \label{eq:test_GD_rel}
%         %  (\bX \bV \bS ^{-1} ( \bI - (\bI - \eta \bS)^k ) \bV^T \bX^T \by)^T \bX \bV \bS ^{-1} ( \bI - (\bI - \eta \bS)^k ) \bV^T \bX^T \by
%     \end{align}
%     where $\sigma_{\text{max}}$ is the maximum eigenvalue of the underlying covariance matrix $\bSigma$. Using the upper bound on term 1 in \eqref{eq:upperbound_diagonal}, we have 
%     \begin{align}
%         \Expt{ x \sim \calD_{\calX} }{\left(f_t(x) - \wt f_\lambda (x)\right)^2} &\le \sigma_{\text{max}} \cdot c(\eta, t) \cdot \by^T \bV \bX  \left( \bS^{-1} ( \bI - (\bI - \eta \bS)^t ) \right)^2 \bV^T \bX^T \by \\
%         &=   \kappa \cdot c(\eta, t) \cdot \sigma_{\text{min}}\cdot \norm{\bV \left( \bS^{-1} ( \bI - (\bI - \eta \bS)^t ) \right) \bV^T \bX^T \by}{2}^2 \\
%         &\le \kappa \cdot c(\eta, t) \cdot \left[ \bV \left( \bS^{-1} ( \bI - (\bI - \eta \bS)^t ) \right) \bV^T \bX^T \right]^T \bSigma \\
%         &\qquad \qquad \qquad \qquad \qquad \left[ \bV \left( \bS^{-1} ( \bI - (\bI - \eta \bS)^t ) \right) \bV^T \bX^T \right] \by \\
%         & = \kappa \cdot c(\eta, t) \cdot \Expt{x \sim \calD_{\calX}}{\norm{x^T w_t}{2}} \,.
%     \end{align}
% % 
% % 
%     % Since $ \eta\le 1/\norm{S}{\text{op}}$, invoking \lemref{lem:ineq_soln} to upper bound term 1 with
% \end{proof}


% \newpage
% \section{Additional experiments and details}\label{app:exp}
% \newcommand\tab[1][1cm]{\hspace*{#1}}

% \subsection{Datasets} \label{sec:app_dataset}

% \textbf{Toy Dataset {} {}} Assume fixed constants $\mu$ and $\sigma$. For a given label $y$, we simulate features $x$ in our toy classification setup as follows: 
% \begin{align*}
%     x \defeq \texttt{concat} \left[ x_1, x_2\right] \quad \text{where} \quad  x_1 \sim  \calN( y \cdot \mu, \sigma^2 I_{d \times d}) \ \  \text{and} \ \  x_1 \sim  \calN( 0, \sigma^2 I_{d \times d}) \,.
% \end{align*}  
% % where $y$ is the true label and $x$ is the corresponding feature vector. 
% In experiements throughout the paper, we fix dimention $d=100$, $\mu = 1.0 $, and $\sigma = \sqrt{d}$. Intuitively, $x_1$ carries the information about the underlying label and $x_2$ is additional noise independent of the underlying label. 

% \textbf{CV datasets {} {}} We use MNIST~\citep{lecun1998mnist} and CIFAR10~\cite{krizhevsky2009learning}. 
% % For binary tasks, 
% We produce a binary variant from the multiclass classification problem by mapping classes $\{0,1,2,3,4\}$ to label $1$ and $\{ 5,6,7,8,9\}$ to label $-1$. For CIFAR dataset, we also use the standard data augementation of random crop and horizontal flip. PyTorch code is as follows: 

% \texttt{(transforms.RandomCrop(32, padding=4),\\
% \tab transforms.RandomHorizontalFlip())}

% \textbf{NLP dataset {} {}} We use IMDb Sentiment analysis~\citep{maas2011learning} corpus.  

% \subsection{Architecture Details} 

% All experiments were run on NVIDIA GeForce RTX 2080 Ti GPUs. We used PyTorch~\citep{NEURIPS2019a9015} and Keras with Tensorflow~\citep{abadi2016tensorflow} backend for experiments. 
% % , ELMo embeddings~\citep{Peters:2018}, and Hugging Face Transformers~\citep{wolf-etal-2020-transformers}. 

% \textbf{Linear model {} {}} For the toy dataset, we simulate a linear model with scalar output and the same number of parameters as the number of dimensions.   

% \textbf{Wide nets {} {}} To simulate the NTK regime, we experiment with $2-$layered wide nets. The PyTorch code for 2-layer wide MLP is as follows: 


% \texttt{ nn.Sequential( \\
% \tab     nn.Flatten(),\\
% \tab    nn.Linear(input\_dims, 200000, bias=True),\\
% \tab    nn.ReLU(),\\
% \tab    nn.Linear(200000, 1, bias=True)\\
% \tab     )}


% We experiment both (i) with the first layer fixed at random initialization; (ii)  and updating both layers' weights.     

% \textbf{Deep nets for CV tasks {} {}} We consider a 4-layered MLP. The PyTorch code for 4-layer MLP is as follows: 

% \texttt{ nn.Sequential(nn.Flatten(), \\
% \tab        nn.Linear(input\_dim, 5000, bias=True),\\
% \tab        nn.ReLU(),\\
% \tab        nn.Linear(5000, 5000, bias=True),\\
% \tab        nn.ReLU(),\\
% \tab        nn.Linear(5000, 5000, bias=True),\\
% \tab        nn.ReLU(),\\
% % \tab        nn.Linear(5000, 5000, bias=True),\\
% % \tab        nn.ReLU(),\\
% \tab        nn.Linear(1024, num\_label, bias=True)\\
% \tab        )}

% For MNIST, we use $1000$ nodes instead of $5000$ nodes in the hidden layer. 
% % 
% We also experiment with convolutional nets. In particular, we use ResNet18 \citep{he2016deep}. Implementation adapted from:  \url{https://github.com/kuangliu/pytorch-cifar.git}. 

% \textbf{Deep nets for NLP {} {}} We use a simple LSTM model with embeddings intialized with ELMo embeddings~\citep{Peters:2018}. Code adapted from: \url{https://github.com/kamujun/elmo_experiments/blob/master/elmo_experiment/notebooks/elmo_text_classification_on_imdb.ipynb} 

% We also evaluate our bounds with a BERT model. In particular, we fine-tune an off-the-shelf uncased BERT model~\citep{devlin2018bert}. Code adapted from Hugging Face Transformers~\citep{wolf-etal-2020-transformers}: \url{https://huggingface.co/transformers/v3.1.0/custom_datasets.html}. 


% \subsection{Additonal experiments}

% 1. SGD with linear models on cross entropy and MSE loss. 

% 2. CE loss and SGD. GD with MSE loss 

% 3. Binary MNIST with MLP. multiclass MNIST  

% \textbf{Results on CIFAR 10 {} {}} 
% % 
% We plot epoch wise error curve for results in \tabref{table:multiclass}. We observe the same trend as in \figref{fig:error_CIFAR10}. Additionally, we plot an \emph{oracle bound} obtained by tracking the error on mislabeled data which nevertheless were predicted as true label. To obtain an exact emprical value of the oracle bound, we need underlying true labels for the randomly labeled data. 
% % Note that our bound in \thmref{thm:multiclass_ERM}, lower bounds the accuracy as predicted by the oracle bound. 
% While with just access to extra unlabeled data we cannot calculate oracle bound, we note that the oracle bound is very tight and never violated in practice underscoring an importamt aspect of generalization in multiclass problems. This highlight that even a stronger conjecture may hold in multiclass classification, i.e., error on mislabeled data (where nevertheless true label was predicted) lower bounds the population error on the distribution of mislabeled data and hence, the error on (a specific) mislabeled portion predicts the population accuracy on clean data. 
% % 
% On the other hand, the dominating term of in \thmref{thm:multiclass_ERM} is loose when compared with the oracle bound. The main reason, we believe is the pessimistic upper bound in \eqref{eq:lemma1_final_multi_prev} in the proof of \lemref{lem:fit_mislabeled_multi}. We leave an investigation on this gap for future. 
% % of fit 

% % However, oracle bound highlights two . One,  



% \begin{figure}[h]
%     \centering 
%     % \vspace{-15pt}
%     % \includegraphics[width=0.9\linewidth]{example-image-a}
%     \includegraphics[width=0.4\linewidth]{figures/CIFAR10-FNN.pdf} \hfil
%     \includegraphics[width=0.4\linewidth]{figures/CIFAR10-Resnet.pdf}
%     % \includegraphics[width=0.9\linewidth]{figures/{CIFAR10_rn=0.1_lr=0.2_wd=0.005}.png}
%     % \vspace{-10pt}
%     \caption{ Per epoch curves for CIFAR10 corresponding results in \tabref{table:multiclass}. As before, we just plot the dominating term in the RHS of \eqref{eq:multiclass_ERM} as predicted bound. Additionally, we also plot the predicted lower bound by the error on mislabeled data which nevertheless were predicted as true label. We refer to this as ``Oracle bound''. See text for more details. 
%     % 
%     % except for the stopping point. 
%     % The bound predicted by RATT (RHS in \eqref{eq:multiclass_ERM}) is vacuous. 
%     }\label{fig:error_epoch_CIFAR10}
%     % \vspace{-15pt}
% \end{figure}


% \textbf{Results on CIFAR 100 {} {}} 
% % 
% On CIFAR100, our bound in \eqref{eq:multiclass_ERM} yields vacous bounds. However, the oracle bound as explained above yields tight guarantees in the initial phase of the learning (i.e., when learning rate is less than $0.1$). 

% \begin{figure}[h]
%     \centering 
%     % \vspace{-15pt}
%     % \includegraphics[width=0.9\linewidth]{example-image-a}
%     \includegraphics[width=0.4\linewidth]{figures/CIFAR100-Resnet.pdf}
%     % \includegraphics[width=0.9\linewidth]{figures/{CIFAR10_rn=0.1_lr=0.2_wd=0.005}.png}
%     % \vspace{-10pt}
%     \caption{ Predicted lower bound by the error on mislabeled data which nevertheless were predicted as true label with ResNet18 on CIFAR100. We refer to this as ``Oracle bound''. See text for more details. 
%     % 
%     % except for the stopping point. 
%     The bound predicted by RATT (RHS in \eqref{eq:multiclass_ERM}) is vacuous. 
%     }\label{fig:error_CIFAR100}
%     % \vspace{-15pt}
% \end{figure}


% % \paragraph{Experiments on CIFAR100} 



% \subsection{Hyperparameter Details}


% \textbf{\figref{fig:error_CIFAR10} {} {}} We use clean training dataset of size $40,000$. We fix the amount of unlabeled data at $20\%$ of the clean size, i.e. we include additional $8,000$ points with randomly assigned labels. We use test set of $10,000$ points. For both MLP and ResNet, we use SGD with an initial learning rate of $0.1$ and momentum $0.9$. We fix the weight decay parameter at $5\times 10^{-4}$. After $100$ epochs, we decay the learning rate to $0.01$. We use SGD batch size of $100$. 

% \textbf{\figref{fig:error_binary} (a) {} {}} We obtain a toy dataset according to the process described in \secref{sec:app_dataset}. We fix $d=100$ and create a dataset of $50,000$ points with balanced classes. Moreover, we sample additional covariates with the same procedure to create randomly labeled dataset. For both SGD and GD training, we use a fixed learning rate $0.1$.    

% \textbf{\figref{fig:error_binary} (b) {} {}} Similar to binary CIFAR, we use clean training dataset of size $40,000$ and fix the amount of unlabeled data at $20\%$ of the clean dataset size. To train wide nets, we use a fixed learning of $0.001$ with GD and SGD. We decide the weight decay parameter and the early stopping point that maximizes our generalization bound (i.e. without peeking at unseen data ).  We use SGD batch size of $100$. 

% \textbf{\figref{fig:error_binary} (c) {} {}} With IMDb dataset, we use a clean dataset of size $20,000$ and as before, fix the amount of unlabeled data at $20\%$ of the clean data. To train ELMo model, we use Adam optimizer with a fixed learning rate $0.01$ and weight decay $10^{-6}$ to minimize cross entropy loss. We train with batch size $32$ for 3 epochs. To fine-tune BERT model, we use Adam optimizer with learning rate $5\times 10^{-5}$ to minimize cross entropy loss. We train with a batch size of $16$ for 1 epoch.    

% \textbf{\tabref{table:multiclass} {} {}} For multiclass datasets, we train both MLP and ResNet with the same hyperparameters as described before. We sample a clean training dataset of size $40,000$ and fix the amount of unlabeled data at $20\%$ of the clean size. We use SGD with an initial learning rate of $0.1$ and momentum $0.9$. We fix the weight decay parameter at $5\times 10^{-4}$. After $30$ epochs for ResNet and after $50$ epochs for MLP, we decay the learning rate to $0.01$.  We use SGD with batch size $100$. 
% For \figref{fig:error_CIFAR100}, we use the same hyperparameters as 
% CIFAR10 training, except we now decay learning rate after $100$ epochs. 


% In all experiments, to identify the best possible accuracy on just the clean data, we use the exact same set of hyperparamters except the stopping point. We choose a stopping point that maximizes test performance. 

% \subsection{Summary of experiments }

% \begin{center}
%     \begin{table}[H] 
%         \centering
%         \begin{tabular}{|c|c|c|c|} 
%         \hline
%         Classification type & Model category & Model & Dataset  \\ [0.5ex] 
%         \hline
%         \hline
%         \multirow{9}{*}{Binary} & Low dimensional & Linear model & Toy Gaussain dataset  \\
%                         \cline{2-4}
%                          & \multirow{1}{*}{Overparameterized linear nets} 
%                         %  & Linear model & Toy Gaussain dataset \\
%                         %  \cline{3-4}
%                         %  & & 2-layer wide net& Toy Gaussain dataset \\
%                         %  \cline{3-4}
%                          & 2-layer wide net & Binary MNIST \\
%                          \cline{2-4}                 
%                          & \multirow{6}{*}{Deep nets} & \multirow{2}{*}{MLP} & Binary MNIST \\
%                          \cline{4-4}
%                          & &  & Binary CIFAR \\
%                          \cline{3-4}
%                          &  & \multirow{2}{*}{ResNet} & Binary MNIST \\
%                          \cline{4-4}
%                          & &  & Binary CIFAR \\
%                          \cline{3-4}
%                          &  & ELMo-LSTM model & IMDb Sentiment Analysis \\
%                          \cline{3-4}
%                          & & BERT pre-trained model & IMDb Sentiment Analysis \\
%         \hline
%         \multirow{5}{*}{Multiclass} & \multirow{5}{*}{Deep nets} & \multirow{2}{*}{MLP} & MNIST \\
%                         \cline{4-4} 
%                         & & & CIFAR10 \\                   
%                         \cline{3-4}
%                          &   & \multirow{3}{*}{ResNet} & MNIST \\
%                          \cline{4-4}
%                          &   & & CIFAR10 \\
%                          \cline{4-4}
%                          &   & & CIFAR100 \\
%         \hline
%         \end{tabular}
%         % \caption{Summary of experiments performed} \label{table:experiments}
%     \end{table}    
%     % \footnotetext[6]{We use both MSE loss and cross-entropy loss.}
%     % \footnotetext[6]{We try 2 variants: one with a fixed first layer and the other with both layers trainable.}
% \end{center}

% \newpage
% \section{Proof of \lemref{lem:stability_error}} \label{app:proof_lem_error}

% \begin{proof}[Proof of \lemref{lem:stability_error}]
%     Recall, we have a training set $S \cup \wt S_C$. We defined leave-one-out error on mislabeled points as $$\error_{\text{LOO}(\wt S_M) } = \frac{\sum_{(x_i, y_i) \in \wt S_M} \error( f_{(i)}( x_i), y_i)}{ \abs{\wt S_M }} \,, $$
%     where $f_{(i)} \defeq f(\calA, (S \cup \wt S)_{(i)})$. Define $S^\prime \defeq S \cup \wt S$. Assume $(x,y)$ and $(x^\prime,y^\prime)$ as i.i.d. samples from ${\calDm}$. 
%     Using Lemma 25 in \citet{bousquet2002stability}, we have
%     \begin{align*}
%         \Expo{ \left( \error_{\calDm}(\wh f) -\error_{\text{LOO}(\wt S_M) } \right)^2 } \le & \Expt{ S^\prime, (x,y), (x^\prime,y^\prime) }{ \error(\wh f(x), y ) \error(\wh f(x^\prime), y^\prime )} - 2 \Expt{ S^\prime, (x,y) }{ \error(\wh f(x), y ) \error(f_{(i)}(x_i), y_i )} \\
%         & + \frac{m_1-1}{m_1}\Expt{ S^\prime }{  \error(f_{(i)}(x_i), y_i )  \error(f_{(j)}(x_j), y_j )} + \frac{1}{m_1} \Expt{ S^\prime }{  \error(f_{(i)}(x_i), y_i ) } \,. \numberthis \label{eq:main_reln}
%     \end{align*}
%     We can rewrite the equation above as : 
%     \begin{align*}
%         \Expo{ \left( \error_{\calDm}(\wh f) -\error_{\text{LOO}(\wt S_M) } \right)^2 } \le &  \, \underbrace{\Expt{ S^\prime, (x,y), (x^\prime,y^\prime) }{ \error(\wh f(x), y ) \error(\wh f(x^\prime), y^\prime ) - \error(\wh f(x), y ) \error(f_{(i)}(x_i), y_i )}}_{\RN{1}} \\
%         & + \underbrace{\Expt{ S^\prime }{  \error(f_{(i)}(x_i), y_i )  \error(f_{(j)}(x_j), y_j ) -  \error(\wh f(x), y ) \error(f_{(i)}(x_i), y_i )}}_{\RN{2}} \\ &+ \underbrace{\frac{1}{m_1} \Expt{ S^\prime }{  \error(f_{(i)}(x_i), y_i ) - \error(f_{(i)}(x_i), y_i )  \error(f_{(j)}(x_j), y_j ) }}_{\RN{3}} \,. \numberthis \label{eq:main_reln2}
%     \end{align*}
    
%     We will now bound term $\RN{3}$.  Using Schwarz's inequality, we have
    
%     \begin{align}
%         \Expt{ S^\prime }{  \error(f_{(i)}(x_i), y_i ) - \error(f_{(i)}(x_i), y_i )  \error(f_{(j)}(x_j), y_j ) }^2 &\le  \Expt{ S^\prime }{  \error(f_{(i)}(x_i), y_i ) }^2 \Expt{S^\prime}{1 -   \error(f_{(j)}(x_j), y_j ) }^2 \\
%         &\le \frac{1}{4} \label{eq:term1_lem12}
%     \end{align}
    
%     Note that since $(x_i,y_i)$, $(x_j ,y_j )$, $(x,y)$, and $(x^\prime, y^\prime)$ are all from same distribution $\calDm$, we directly incorporate the bounds on term $\RN{1}$ and $\RN{2}$ from proof of Lemma 9 in \citet{bousquet2002stability}. Combining that with \eqref{eq:term1_lem12} and our definition of hypothesis stability in \codref{cond:hypothesis_stability}, we have the required claim. 
    
    
%     % We now re-write term $\RN{1}$ as
%     % \begin{align*}
%     %         &\Expt{S^\prime, (x,y), (x^\prime,y^\prime) }{ \error(\wh f(x), y ) \error(\wh f(x^\prime), y^\prime ) - \error(\wh f(x), y ) \error(f_{(i)}(x_i), y_i )} \\ & \qquad = \Expt{ S^\prime, (x,y), (x^\prime,y^\prime) }{ \error(\wh f(x), y ) \error(\wh f  (x^\prime), y^\prime ) - \error(\wh f ^\prime(x), y ) \error(f_{(i)}(x^\prime), y^\prime )} \tag{Exchanging $(x_i, y_i)$ with $(x^\prime, y^\prime)$ in the second term} \\
%     %         & \qquad = \Expt{ S^\prime, (x,y), (x^\prime,y^\prime) }{  \left(\error(\wh f(x), y )-  \error(f_{(i)}(x), y ) \right) \error(\wh f  (x^\prime), y^\prime )  } \\
%     %         & \qquad  + \Expt{ S^\prime, (x,y), (x^\prime,y^\prime) }{  \left(\error(f_{(i)}(x), y ) -\error(\wh f ^\prime(x), y ) \right) \error(\wh f  (x^\prime), y^\prime )}  \\
%     %         & \qquad +\Expt{ S^\prime, (x,y), (x^\prime,y^\prime) }{  \left( \error(\wh f  (x^\prime), y^\prime ) -  \error(f_{(i)}(x^\prime), y^\prime ) \right) \error(\wh f ^\prime(x), y ) }  \,, \numberthis \label{eq:term1_final}
%     % \end{align*}
%     % where $\wh f^\prime$ is the classifier obtained by training on $ S^\prime_{(i)} \cup \{ (x^\prime, y^\prime) \} $. Similarly we can re-write term $\RN{2}$ as 
%     % \begin{align*}
%     %     & \Expt{ S^\prime }{  \error(f_{(i)}(x_i), y_i )  \error(f_{(j)}(x_j), y_j ) -  \error(\wh f(x), y ) \error(f_{(i)}(x_i), y_i )} \\
%     %     &\quad  = \Expt{ S^\prime, (x,y), (x^\prime,y^\prime)}{  \error(f^{\prime\prime}_{(i)}(x), y )  \error(f_{(j)}^{\prime}(x^\prime), y^\prime ) -  \error(\wh f(x), y ) \error(f_{(i)}(x_i), y_i )} \tag{Exchanging $(x_i, y_i)$ with $(x, y)$ and $(x_j, y_j)$ with $(x^\prime, y^\prime)$ in the first term}\\
%     %     &\quad = \Expt{ S^\prime, (x,y), (x^\prime,y^\prime)}{  \error(f^{\prime\prime}_{(j)}(x), y )  \error(f_{(i)}^{\prime}(x^\prime), y^\prime ) -  \error(\wh f^\prime (x), y ) \error(f^\prime_{(j)}(x^\prime), y^\prime )} \tag{Exchanging $(x_i, y_i)$ and $(x_j, y_j)$ and then replacing $(x_j, y_j)$ with $(x^\prime, y^\prime)$ in the second term} \\
%     %     & \quad = \Expt{ S^\prime, (x,y), (x^\prime,y^\prime) }{  \left( \error(f_{(i)}^{\prime}(x^\prime), y^\prime )   -  \error(\wh f^{\prime\prime}  (x^\prime), y^\prime ) \right)  \error(f^{\prime\prime}_{(j)}(x), y )   } \\
%     %     & \quad  + \Expt{ S^\prime, (x,y), (x^\prime,y^\prime) }{  \left( \error(f^{\prime\prime}_{(j)}(x), y )  -\error(\wh f ^\prime(x), y ) \right) \error(\wh f^{\prime\prime}  (x^\prime), y^\prime )  }  \\
%     %     & \quad+ \Expt{ S^\prime, (x,y), (x^\prime,y^\prime) }{  \left( \error(\wh f^{\prime\prime}  (x^\prime), y^\prime )  -  \error(f^\prime_{(j)}(x^\prime), y^\prime ) \right)  \error(\wh f^\prime (x), y ) }   \\
%     %     & \quad = \Expt{ S^\prime, (x,y), (x^\prime,y^\prime) }{  \left( \error(f_{(i)}^{\prime}(x^\prime), y^\prime )   -  \error(\wh f (x^\prime), y^\prime ) \right)  \error(f_{(i)}(x_j), y_j )   } \\
%     %     & \quad  + \Expt{ S^\prime, (x,y), (x^\prime,y^\prime) }{  \left( \error(f^{\prime\prime}_{(j)}(x), y )  -\error(\wh f (x), y ) \right) \error(\wh f^{\prime\prime}  (x_j), y_j )  }  \\
%     %     & \quad+ \Expt{ S^\prime, (x,y), (x^\prime,y^\prime) }{  \left( \error(\wh f^{\prime\prime}  (x^\prime), y^\prime )  -  \error(f^\prime_{(j)}(x^\prime), y^\prime ) \right)  \error(\wh f^\prime (x^\prime), y^\prime ) }  \,, \numberthis \label{eq:term2_final}
%     % \end{align*}
%     % where $f^{\prime\prime}_{(j)}$ is trained on $S^\prime_{(j,i)} \cup {(x,y)}$, $f^{\prime}_{(i)}$ is trained on $S^\prime_{(j,i)} \cup {(x^\prime,y^\prime)}$, and $\wh f^{\prime\prime} $ is trained on $S^\prime_{(j)} \cup {(x,y)}$. Note in the last line we replaced $(x,y)$ by $(x_j, y_j)$ in the first term, replaced $(x^\prime,y^\prime)$ by $(x_j, y_j)$ in the second term and exchanged $(x_i,y_i)$ with $(x_j,y_j)$ and also $(x,y)$ and $(x^\prime, y^\prime)$
    
    
% \end{proof}

\section{Another Example: Learning a Co-Variance Matrix}\label{sec:covariance}

We demonstrate the benefits of using OMD over GD in another simple illustrative example. In this case, the example is does not boil down to a bi-linear game and therefore, the simulation results portray that the theoretical results we provided for bi-linear games, carry over qualitatively beyond the linear case.

Consider the case where the data distribution is a mean zero multi-variate normal with an unknown co-variance matrix, i.e., $x \sim N(0, \Sigma)$. We will consider the case where the discriminator is the set of all quadratic functions:
\begin{equation}
D_W(x) = \sum_{ij} W_{ij} x_i x_j = x^T W x 
\end{equation}
The generator is a linear function of the random input noise $z\sim N(0, I)$, of the form:
\begin{equation}
G_V(z) = V z
\end{equation}
The parameters $W$ and $V$ are both $d\times d$ matrices. The WGAN game loss associated with these functions is then:
\begin{equation}
L(V, W)= \mathbb{E}_{x\sim N(0, \Sigma)}\left[ x^T W x \right] - \mathbb{E}_{z\sim N(0,I)}\left[z^T V^T W V z \right] 
\end{equation}
Expanding the latter we get:
\begin{align*}
L(V, W)=& \mathbb{E}_{x\sim N(0, \Sigma)}\left[ \sum_{ij} W_{ij} x_i x_j \right] - \mathbb{E}_{z\sim N(0,I)}\left[ \sum_{ij} W_{ij} \sum_{k} V_{ik} z_k  \sum_{m} V_{jm} z_m\right] \\
=&
\mathbb{E}_{x\sim N(0, \Sigma)}\left[ \sum_{ij} W_{ij} x_i x_j \right] - \mathbb{E}_{z\sim N(0,I)}\left[ \sum_{ijkm} W_{ij} V_{ik} V_{jm} z_k z_m \right]\\
=& \sum_{ij} W_{ij} \mathbb{E}_{x\sim N(0, \Sigma)}\left[ x_i x_j \right] - \sum_{ijkm} W_{ij} V_{ik} V_{jm} \mathbb{E}_{z\sim N(0,I)}\left[ z_k z_m \right]\\
=& \sum_{ij} W_{ij} \Sigma_{ij} - \sum_{ijkm} W_{ij} V_{ik} V_{jm} 1\{k=m\}\\
=& \sum_{ij} W_{ij} \Sigma_{ij} - \sum_{ijk} W_{ij} V_{ik} V_{jk}\\
=& \sum_{ij} W_{ij} \left(\Sigma_{ij} - \sum_{k} V_{ik} V_{jk}\right)
\end{align*}
Given that the covariance matrix is symmetric positive definite, we can write it as $\Sigma = U U^T$. Then the loss simplifies to:
\begin{align}
L(V, W) = \sum_{ij} W_{ij} \left(\Sigma_{ij} - \sum_{k} V_{ik} V_{jk}\right) =& \sum_{ijk} W_{ij} \left(U_{ik} U_{jk} -  V_{ik} V_{jk}\right)
\end{align}
The equilibrium of this game is for the generator to choose $V_{ik} = U_{ik}$ for all $i,k$, and for the discriminator to pick $W_{ij}=0$. For instance, in the case of a single dimension we have $L(V,W) = W\cdot (\sigma^2 - V^2)$, where $\sigma^2$ is the variance of the Gaussian. Hence, the equilibrium is for the generator to pick $V=\sigma$ and the discriminator to pick $W=0$.

\paragraph{Dynamics without sampling noise.} For the mean GD dynamics the update rules are as follows:
\begin{equation}
\begin{aligned}
W_{ij}^t =& W_{ij}^{t-1} + \eta \left(\Sigma_{ij} - \sum_{k} V_{ik}^{t-1} V_{jk}^{t-1}\right) \\
V_{ij}^t =& V_{ij}^{t-1} + \eta \sum_{k} \left(W_{ik}^{t-1} + W_{ki}^{t-1}\right) V_{kj}^{t-1} 
\end{aligned}
\end{equation}
We can write the latter updates in a simpler matrix form:
\begin{equation}
\begin{aligned}
W_t =& W_{t-1} + \eta \left(\Sigma - V_{t-1} V_{t-1}^T\right)\\
V_t =& V_{t-1} + \eta (W_{t-1} + W_{t-1}^T) V_{t-1}
\end{aligned}\tag{GD for Covariance}
\end{equation}
Similarly the OMD dynamics are:
\begin{equation}
\begin{aligned}
W_t =& W_{t-1} + 2\eta \left(\Sigma - V_{t-1} V_{t-1}^T\right) - \eta \left(\Sigma - V_{t-2} V_{t-2}^T\right)\\
V_t =& V_{t-1} + 2\eta (W_{t-1} + W_{t-1}^T) V_{t-1} - \eta (W_{t-2} + W_{t-2}^T) V_{t-2}
\end{aligned}\tag{OMD for Covariance}
\end{equation}

Due to the non-convexity of the generators problem and because there might be multiple optimal solutions (e.g. if $\Sigma$ is not strictly positive definite), it is helpful in this setting to also help dynamics by adding $\ell_2$ regularization to the loss of the game. The latter simply adds an extra $2\lambda W_{t}$ at each gradient term $\nabla_W L(V_t, W_t)$ for the discriminator and a $2\lambda V_{t}$ at each gradient term $\nabla_{V} L(V_t, W_t)$ for the generator. In Figures \ref{fig:covariance} and \ref{fig:covariance2d} we give the weights and the implied covariance matrix $\Sigma^G=VV^T$ of the generator's distribution for each of the dynamics for an example setting of the step-size and regularization parameters and for two and three dimensional gaussians respectively. We again see how OMD can stabilize the dynamics to converge pointwise.

\paragraph{Stochastic dynamics.} In Figure \ref{fig:stoch_covariance} and \ref{fig:stoch_covariance2} we also portray the instability of GD and the robustness of the stability of OMD under stochastic dynamics. In the case of stochastic dynamics the gradients are replaced with unbiased estimates or with averages of unbiased estimates over a small minibatch. In the case of a mini-batch of one, the unbiased estimates of the gradients in this setting take the following form:
\begin{equation}
\begin{aligned}
\hat{\nabla}_{W, t} = x_{t} x_{t}^T - V_{t}z_{t} z_{t}^T  V_{t}^T\\
\hat{\nabla}_{V, t} = - (W_{t} + W_{t}^T) V_{t} z_t z_t^T
\end{aligned}\tag{Stochastic Gradients}
\end{equation}
where $x_t, z_t$ are samples drawn from the true distribution and from the random noise distribution respectively. Hence, the stochastic dynamics simply follow by replacing gradients with unbiased estimates:
\begin{equation}
\begin{aligned}
W_t =& W_{t-1} + \eta \hat{\nabla}_{W, t-1}\\
V_t =& V_{t-1} - \eta \hat{\nabla}_{V, t-1}
\end{aligned}\tag{SGD for Covariance}
\end{equation}
\begin{equation}
\begin{aligned}
W_t =& W_{t-1} + 2\eta \hat{\nabla}_{W, t-1} - \eta \hat{\nabla}_{W, t-2}\\
V_t =& V_{t-1} - 2\eta \hat{\nabla}_{V, t-1} + \eta \hat{\nabla}_{V, t-2}
\end{aligned}\tag{SOMD for Covariance}
\end{equation}

\newpage

\begin{figure}[htpb]
    \centering
    \begin{subfigure}[b]{1\textwidth}
        \centering
    		\begin{subfigure}[b]{.3\textwidth}
    		\includegraphics[height=1.7in]{2d_covariance_gd_disc.png}
			\end{subfigure}        
    		\begin{subfigure}[b]{.3\textwidth}
    		\includegraphics[height=1.7in]{2d_covariance_gd_gen_V.png}
			\end{subfigure}        
    		\begin{subfigure}[b]{.3\textwidth}
    		\includegraphics[height=1.7in]{2d_covariance_gd_gen_Sigma.png}
			\end{subfigure}        
        \caption{GD dynamics. $\eta=0.1$, $T=500$, $\lambda=0.3$.}
    \end{subfigure}
    \begin{subfigure}[b]{1\textwidth}
        \centering
    		\begin{subfigure}[b]{.3\textwidth}
    		\includegraphics[height=1.7in]{2d_covariance_omd_disc.png}
			\end{subfigure}        
    		\begin{subfigure}[b]{.3\textwidth}
    		\includegraphics[height=1.7in]{2d_covariance_omd_gen_V.png}
			\end{subfigure}        
    		\begin{subfigure}[b]{.3\textwidth}
    		\includegraphics[height=1.7in]{2d_covariance_omd_gen_Sigma.png}
			\end{subfigure}        
        \caption{OMD dynamics. $\eta=0.1$, $T=500$, $\lambda=0.3$.}
    \end{subfigure}
    \caption{Stability of OMD vs GD in the co-variance learning problem for a two-dimensional gaussian ($d=2$). Weight clipping in $[-1,1]$ was applied in both dynamics.}\label{fig:covariance2d}
\end{figure}


\begin{figure}[H]
    \centering
    \begin{subfigure}[b]{1\textwidth}
        \centering
    		\begin{subfigure}[b]{.3\textwidth}
    		\includegraphics[height=1.7in]{covariance_gd_disc.png}
			\end{subfigure}        
    		\begin{subfigure}[b]{.3\textwidth}
    		\includegraphics[height=1.7in]{covariance_gd_gen_V.png}
			\end{subfigure}        
    		\begin{subfigure}[b]{.3\textwidth}
    		\includegraphics[height=1.7in]{covariance_gd_gen_Sigma.png}
			\end{subfigure}        
        \caption{GD dynamics. $\eta=0.1$, $T=500$, $\lambda=0.3$.}
    \end{subfigure}
    \begin{subfigure}[b]{1\textwidth}
        \centering
    		\begin{subfigure}[b]{.3\textwidth}
    		\includegraphics[height=1.7in]{covariance_omd_disc.png}
			\end{subfigure}        
    		\begin{subfigure}[b]{.3\textwidth}
    		\includegraphics[height=1.7in]{covariance_omd_gen_V.png}
			\end{subfigure}        
    		\begin{subfigure}[b]{.3\textwidth}
    		\includegraphics[height=1.7in]{covariance_omd_gen_Sigma.png}
			\end{subfigure}        
        \caption{OMD dynamics. $\eta=0.1$, $T=500$, $\lambda=0.3$.}
    \end{subfigure}
    \caption{Stability of OMD vs GD in the co-variance learning problem for a three-dimensional gaussian ($d=3$). Weight clipping in $[-1,1]$ was applied in both dynamics.}\label{fig:covariance}
\end{figure}

\newpage

\begin{figure}[H]
    \begin{subfigure}[b]{1\textwidth}
        \centering
    		\begin{subfigure}[b]{.3\textwidth}
    		\includegraphics[height=1.7in]{covariance_stoch_gd_disc.png}
			\end{subfigure}        
    		\begin{subfigure}[b]{.3\textwidth}
    		\includegraphics[height=1.7in]{covariance_stoch_gd_gen_V.png}
			\end{subfigure}        
    		\begin{subfigure}[b]{.3\textwidth}
    		\includegraphics[height=1.7in]{covariance_stoch_gd_gen_Sigma.png}
			\end{subfigure}        
        \caption{Stochastic GD dynamics with mini-batch size $50$. $\eta=0.02$, $T=1000$, $\lambda=0.1$.}
    \end{subfigure}
    \begin{subfigure}[b]{1.01\textwidth}
    		\begin{subfigure}[b]{.19\textwidth}
    		\hspace{-.3in}     	
    		\includegraphics[height=1.3in]{covariance_gd_true.png}
    		\caption{True Distribution}
			\end{subfigure}        
    		\begin{subfigure}[b]{.19\textwidth}
    		\hspace{-.3in}     	
    		\includegraphics[height=1.3in]{covariance_gd_iterate_minus_50.png}
    		\caption{Iterate $T-50$}
			\end{subfigure}   
    		\begin{subfigure}[b]{.19\textwidth}
    		\hspace{-.3in}     	
    		\includegraphics[height=1.3in]{covariance_gd_iterate_minus_35.png}
    		\caption{Iterate $T-35$}
			\end{subfigure}   
    		\begin{subfigure}[b]{.19\textwidth}
    		\hspace{-.3in}     	
    		\includegraphics[height=1.3in]{covariance_gd_iterate_minus_20.png}
    		\caption{Iterate $T-20$}
			\end{subfigure}  
    		\begin{subfigure}[b]{.19\textwidth}
    		\hspace{-.3in}     	
    		\includegraphics[height=1.3in]{covariance_gd_last_iterate.png}
    		\caption{Iterate $T$}
			\end{subfigure}  
    	\caption{Comparison of true distribution and distribution of generator at various points closer to the end of training.}
    \end{subfigure}
    \caption{Stochastic GD dynamics for covariance learning of a two-dimensional gaussian ($d=2$). Weight clipping in $[-1,1]$ was applied to the discriminator weights.}\label{fig:stoch_covariance}
\end{figure}

\begin{figure}[H]
    \centering
    \begin{subfigure}[b]{1\textwidth}
        \centering
    		\begin{subfigure}[b]{.3\textwidth}
    		\includegraphics[height=1.7in]{covariance_stoch_omd_disc.png}
			\end{subfigure}        
    		\begin{subfigure}[b]{.3\textwidth}
    		\includegraphics[height=1.7in]{covariance_stoch_omd_gen_V.png}
			\end{subfigure}        
    		\begin{subfigure}[b]{.3\textwidth}
    		\includegraphics[height=1.7in]{covariance_stoch_omd_gen_Sigma.png}
			\end{subfigure}        
        \caption{Stochastic OMD dynamics with mini-batch size $50$. $\eta=0.02$, $T=1000$, $\lambda=0.1$.}
    \end{subfigure}
    \begin{subfigure}[b]{1.01\textwidth}
    		\begin{subfigure}[b]{.19\textwidth}   
    		\hspace{-.3in}     	
    		\includegraphics[height=1.3in]{covariance_omd_true.png}
    		\caption{True Distribution}
			\end{subfigure}        
    		\begin{subfigure}[b]{.19\textwidth}
    		\hspace{-.3in}     	
    		\includegraphics[height=1.3in]{covariance_omd_iterate_minus_50.png}
    		\caption{Iterate $T-50$}
			\end{subfigure}   
    		\begin{subfigure}[b]{.19\textwidth}
    		\hspace{-.3in}     	
    		\includegraphics[height=1.3in]{covariance_omd_iterate_minus_35.png}
    		\caption{Iterate $T-35$}
			\end{subfigure}   
    		\begin{subfigure}[b]{.19\textwidth}
    		\hspace{-.3in}     	
    		\includegraphics[height=1.3in]{covariance_omd_iterate_minus_20.png}
    		\caption{Iterate $T-20$}
			\end{subfigure}  
    		\begin{subfigure}[b]{.19\textwidth}
    		\hspace{-.3in}     	
    		\includegraphics[height=1.3in]{covariance_omd_last_iterate.png}
    		\caption{Iterate $T$}
			\end{subfigure}  
    	\caption{Comparison of true distribution and distribution of generator at various points closer to the end of training.}
    \end{subfigure}
    \caption{Stability of OMD with stochastic gradients for covariance learning of a two-dimensional gaussian ($d=2$). Weight clipping in $[-1,1]$ was applied to the discriminator weights.}\label{fig:stoch_covariance2}
\end{figure}




\section{Last Iterate Convergence of OMD in Bilinear Case}\label{sec:appendix:last-iterate}
 \label{sec:bilinear OMD convergence}

The goal of this section is to show that Optimistic Mirror Descent exhibits last iterate convergence to min-max solutions for bilinear functions. In Section~\ref{app:proof of special case minmax}, we provide the proof of Theorem~\ref{thm:convergence of OGD-main}, that OMD exhibits last iterate convergence to min-max solutions of the following min-max problem
\begin{align}
 \min_x \max_y x^T A y, \label{eq:our minmax}
\end{align}
where $A$ is an abitrary matrix and $x$ and $y$ are unconstrained. In Section~\ref{app:proof of general case minmax}, we state the appropriate extension of our theorem to the general case:
\begin{align}
    \inf_{x} \sup_{y} \left(x^TAy + b^Tx + c^Ty + d\right). \label{eq:general inf sup}
\end{align}

\subsection{Proof of Theorem~\ref{thm:convergence of OGD-main}} \label{app:proof of special case minmax}

\smallskip As stated in Section~\ref{sec:main:proof OMD converges}, for the min-max problem~\eqref{eq:our minmax}
%
%In this section, we show that Optimistic Gradient Descent  
%exhibits final-iterate, rather than only average-iterate convergence to min-max solutions for
%bilinear functions. 
%%In particular, we show that the $\ell_2$ norms of the gradients used by the dynamics shrinks in time.
%More precisely, we consider the problem $\min_x \max_y x^T A y$, for some matrix $A$, where $x$ and $y$ are unconstrained. 
Optimistic Mirror Descent takes the following form, for all $t \ge 1$:
\begin{align}
    x_{t} &= \xto \label{eq:OGD bilinear x}\\
    y_{t} &= \yto  \label{eq:OGD bilinear y}
\end{align}
where for the above iterations to be meaningful we need to specify $x_0,x_{-1},y_0,y_{-1}$. 

\smallskip As stated in Section~\ref{sec:main:proof OMD converges} we allow any initialization $x_0 \in \mathcal{R}(A)$, and $y_0\in\mathcal{R}(A^T)$, and set $x_{-1}=2x_0$ and $y_{-1}=2y_{0}$, where ${\cal R}(\cdot)$ represents column space. In particular, our initialization means that the first step taken by the dynamics gives $x_1=x_0$ and $y_1=y_0$.

Before giving our proof of Theorem~\ref{thm:convergence of OGD-main}, we need some further notation.
%\smallskip We will analyze Optimistic Gradient Descent under the assumption $\lambda_{\infty} \le 1$, where $\lambda_{\infty}=\max\{||A||,||A^T||\}$ and $||\cdot||$ denotes spectral norm of matrices. We can always enforce that $\lambda_{\infty} \le 1$ by appropriately scaling $A$. Scaling $A$ by some positive factor clearly does not change the min-max solutions $(x^*,y^*)$, only scales the optimal value $x^{*T}Ay^*$ by the same factor.
%
%\begin{remark}
%We remark that $(x,y)=(0,0)$ is always a solution to $\min_x \max_y x^T A y$. More generally, the solutions to the problem are pairs $(x,y)$ such that $x$ is in the null space of $A^T$ and $y$ is in the null space of $A$. In particular, finding a solution to $\min_x \max_y x^T A y$ is a trivial problem. So this section only serves the purpose of rigorously showing that Optimistic Gradient Descent converges to a min-max solution. This is interesting in light of the fact that Gradient Descent actually diverges, even in the special case where $A$ is the identity matrix, as per the following proposition whose proof is provided in Appendix~\ref{appendix:omitted proofs}.
%
%
%{\begin{proposition} \label{prop:gradient descent unstable}
%Gradient descent applied to problem $\min_x \max_y x^T A y$ diverges from any initialization $x_0, y_0$ such that $x_0,y_0 \neq 0$.
%\end{proposition}}
%\end{remark}
%
%
%\paragraph{Notation.} We start with some notation that will be handy later on. 
For all $i \in \mathbb{N}$, we set:
\begin{align*}
      &~~~~~~~~~~M_i = A^j(A^TA)^k, N_i = \(A^T\)^j \(AA^T\)^k \\ 
    &~~~~~~~~~~\Delta^i_t = \normlt{N_iAy_t} + \normlt{M_iA^Tx_t}\\
		&\text{where $k \in \mathbb{Z}$ and $j \in \{0, 1\}$ are such that:~}   i = 2k + j.
\end{align*}
\noindent With this notation, $\Delta_t^0 =
\normlt{A^Tx_t} + \normlt{Ay_t}$, $\Delta^1_t = \normlt{AA^Tx_t} +
\normlt{A^TAy_t}$, $\Delta^2_t = \normlt{A^TAA^Tx_t} +
\normlt{AA^TAy_t}$, etc.

We also use the notation $\langle u, v \rangle_X = u^TXX^Tv$, for vectors $u, v
\in \mathbb{R}^d$ and square $d \times d$ matrices $X$. We similarly define the
norm notation $||u||_X=\sqrt{\langle u, u \rangle_X}$. Given our notation, we
have the following claim, shown in Appendix~\ref{appendix:omitted proofs}.
\begin{claim} \label{claim:pushing A's around}
For all matrices $A$ and vectors $u,v$ of the appropriate dimensions:\\
$$\innerab{Au}{Av}{i} = \innerac{u}{v}{i+1};~~\innerac{A^Tu}{A^Tv}{i} = \innerab{u}{v}{i+1};~~\innerab{u}{Av}{i} = \innerac{v}{A^Tu}{i}.$$
\end{claim}


With our notation in place, we show (through iterated expansion of the update rule), the
following lemma, proved in Appendix~\ref{appendix:omitted proofs}:
\begin{lemma}  \label{lemma:first bound}
For the dynamics of Eq.~\eqref{eq:OGD bilinear x} and~\eqref{eq:OGD bilinear y} and any initialization ${1 \over 2}x_{-1}=x_0 \in
\mathcal{R}(A)$, and ${1 \over 2}y_{-1}=y_0\in\mathcal{R}(A^T)$ we have the following for all $i, t \in \mathbb{N}$ such that $i\ge 0$ and $t \ge 2$:
$$\Delta^i_t - \Delta^i_{t-1} = 4\eta^2\Delta^{i+1}_{t-1} -
5\eta^2\Delta^{i+1}_{t-2} - 2\eta^3\(\innerab{x_{t-2}}{Ay_{t-4}}{i+1} -
\innerac{y_{t-2}}{A^Tx_{t-4}}{i+1}\).$$
\end{lemma}


\medskip We are ready to prove Theorem~\ref{thm:convergence of OGD-main}. Its proof is implied by the following stronger theorem, and Corollary~\ref{cor:gradient becomes small}.

%\begin{theorem}\label{thm:convergence of OGD}
%Consider the dynamics of Eq.~\eqref{eq:OGD bilinear x} and~\eqref{eq:OGD bilinear y} and any initialization ${1 \over 2}x_{-1}=x_0 \in
%\mathcal{R}(A)$, and ${1 \over 2}y_{-1}=y_0\in\mathcal{R}(A^T)$. Let $\gamma = \max\(\left|\left|\(AA^T\)^{+}\right|\right|,
	    %\left|\left|\(A^TA\)^{+}\right|\right|\)$, where for a matrix $X$ we denote by	$X^{+}$ its generalized inverse and by $||X||$ its
    %spectral norm. Suppose that $\max\{||A||,||A^T||\}\equiv \lambda_{\infty}\le 1$ and $\eta<\gamma$. Then, for all $i \in \mathbb{N}$:
		%\begin{align}
		%\Delta^i_1 = \Delta^i_{0}, \label{eq:target condition 1}
		%\end{align}
		%and, for all $i,t\in \mathbb{N}$ such that $t \ge 2$, the following condition holds:
%\begin{align}
    %H(i,t):~~\Delta^i_t \leq \left(1-{\eta}\right)\Delta^i_{t-1} + O(\eta^3 \Delta^0_0). \label{eq:target condition 2}
%\end{align}
%\end{theorem}

\begin{theorem}\label{thm:convergence of OGD}
Consider the dynamics of Eq.~\eqref{eq:OGD bilinear x} and~\eqref{eq:OGD bilinear y} and any initialization ${1 \over 2}x_{-1}=x_0 \in
\mathcal{R}(A)$, and ${1 \over 2}y_{-1}=y_0\in\mathcal{R}(A^T)$. Let $$\gamma = \max\(\left|\left|\(AA^T\)^{+}\right|\right|,
	    \left|\left|\(A^TA\)^{+}\right|\right|\),$$ where for a matrix $X$ we denote by	$X^{+}$ its generalized inverse and by $||X||$ its
    spectral norm. Suppose that $\max\{||A||,||A^T||\}\equiv \lambda_{\infty}\le 1$ and $\eta$ is a small enough constant satisfying $\eta <1/(3\gamma^2)$. Then, for all $i \in \mathbb{N}$:
		\begin{align}
		\Delta^i_1 = \Delta^i_{0}, \label{eq:target condition 1}\\
		\Delta^i_2 \le (1+\eta)^2\Delta^i_{0}, \label{eq:target condition 1.5}
		\end{align}
		and, for all $i,t\in \mathbb{N}$ such that $t \ge 3$, the following condition holds:
\begin{align}
    H(i,t):~~\Delta^i_t \leq \left(1-{\eta^2 \over \gamma^2}\right)\Delta^i_{t-1} + 16\eta^3 \Delta^0_0. \label{eq:target condition 2}
\end{align}
\end{theorem}

\begin{proof}
Eq.~\eqref{eq:target condition 1} holds trivially as under our initialization $x_1=x_0$ and $y_1=y_0$. Eq.~\eqref{eq:target condition 1.5} is also easy to show by noticing the following. Given our initialization:
\begin{align*}
x_2=x_0-\eta A y_0\\
y_2=y_0+\eta A x_0
\end{align*}
Hence (using $j=i \mod 2$):
\begin{align}
M_iA^Tx_2=M_iA^Tx_0 - \eta M_iA^TA y_0\\~~~~~~~~~~~~\Rightarrow~~||M_iA^Tx_2||_2 &\le ||M_iA^Tx_0||_2+\eta ||M_iA^TA y_0||_2\\
&= ||M_iA^Tx_0||_2+\eta ||A^j(A^T)^{1-j}N_iA y_0||_2\\
&\le ||M_iA^Tx_0||_2+\eta \lambda_{\infty} ||N_iA y_0||_2\\
&\le ||M_iA^Tx_0||_2+\eta  ||N_iA y_0||_2 \label{eq:kourasi1}
\end{align}
Similarly:
\begin{align}
||N_iAy_2||_2 &\le ||N_iAy_0||_2+\eta  ||M_iA^T x_0||_2 \label{eq:kourasi2}
\end{align}
It follows from~\eqref{eq:kourasi1} and~\eqref{eq:kourasi2} that
$$\Delta^i_2 \le (1+\eta)^2 \Delta^i_0.$$

We use induction on $t$ to prove~\eqref{eq:target condition 2}. We start our proof by showing the inductive step, and postpone establishing the basis of our induction to the end of this proof. For the inductive step, we assume that $H(i,\tau)$ holds for all $i \ge 0$ and $1\le \tau < t$, for some $t>3$. Assuming this, we show next that $H(i,t)$ holds for all $i$. To do this, we make use of a few lemmas, whose proofs are given in Appendix~\ref{appendix:omitted proofs}. 
	
    \begin{lemma} \label{lemma:restated bound} Under the conditions of the theorem, for all $i \ge 0, t \ge 2$:
		%$$\Delta^i_t - \Delta^i_{t-1} \leq 4\eta^2\Delta_{t-1}^{i+1} -
    %5\eta^2\Delta_{t-2}^{i+1} +2\eta^3(9\Delta_{t-2}^{i+1} + 8\eta^2 \Delta_{t-3}^{i+1} + 8\eta^2 \Delta_{t-4}^{i+1} + 8\eta^2 \Delta_{t-5}^{i+1}).$$
$$\Delta^i_t - \Delta^i_{t-1} \leq 4\eta^2\Delta^{i+1}_{t-1} - 5\eta^2\Delta^{i+1}_{t-2} +
    \eta^3(\Delta_{t-2}^{i+1} + \Delta_{t-4}^{i+1}).$$

		
    %$$\Delta^i_t - \Delta^i_{t-1} \leq 4\eta^2\Delta_{t-1}^{i+1} -
    %5\eta^2\Delta_{t-2}^{i+1} +
    %\eta^3\((1+\lambda_\infty)\Delta_{t-2}^{i+1} +
    %4\eta\lambda^2_\infty\Delta_{0}^{i+1}\).$$
    \end{lemma}
		
		\begin{lemma} \label{lemma:semi-trivial upper bound}
		Under the conditions of the theorem, for all $i, t \ge 0$: $\Delta_t^{i+1} \le \Delta_t^{i}$.
		\end{lemma}

    \begin{lemma} \label{lemma:lower bound}  Under the conditions of the theorem, for all $i \ge 0, t \ge 0$:
	$$\Delta_{t}^{i+2} \geq \frac{1}{\gamma^2}\Delta_{t}^{i}.$$
    \end{lemma}
    
    Given these lemmas, we show our inductive step. So for $t \ge 4$:
		  \begin{align}
	\Delta^i_t - \Delta^i_{t-1} &\leq 4\eta^2\Delta^{i+1}_{t-1} - 5\eta^2\Delta^{i+1}_{t-2} +
    \eta^3(\Delta_{t-2}^{i+1} + \Delta_{t-4}^{i+1}) \label{eq:derivation1}\\
	&\leq -\eta^2 \Delta_{t-1}^{i+1} + \eta^3(\Delta_{t-2}^{i+1} + \Delta_{t-4}^{i+1})+ 80 \eta^5 \Delta_0^0 \\
	%&\leq \eta^2(2\eta-1)\Delta^{i+1}_{t-2} +
	%4\eta^4\Delta_0^0 +O(\eta^5 \Delta^0_0)\\
	%&= -\eta^2\Delta^{i+1}_{t-2} + 2\eta^3\Delta_{t-2}^{i+1} +O(\eta^5\Delta^0_0)\\
&\leq -\frac{1}{\gamma^2}\eta^2\Delta^{i-1}_{t-1} + \eta^3(\Delta_{t-2}^{i+1} + \Delta_{t-4}^{i+1})+ 80 \eta^5 \Delta_0^0\\
&\leq -\frac{1}{\gamma^2}\eta^2\Delta^{i-1}_{t-1} + \eta^3(\Delta_{t-2}^{0} + \Delta_{t-4}^{0})+ 80 \eta^5 \Delta_0^0 \\
&\leq -\frac{1}{\gamma^2}\eta^2\Delta^{i-1}_{t-1} + \(2\eta^3\Delta_{2}^{0} +2\eta^3 {\gamma^2 \over \eta^2}16\eta^3 \Delta^0_0\)+ 80 \eta^5 \Delta_0^0 \\
&\leq -\frac{1}{\gamma^2}\eta^2\Delta^{i-1}_{t-1} + \(2\eta^3(1+\eta)^2 +32\eta^3 {\gamma^2 \over \eta^2}\eta^3  + 80 \eta^5\)\Delta_0^0 \\
&\leq -\frac{1}{\gamma^2}\eta^2\Delta^{i-1}_{t-1} + \(2\eta^3(1+\eta)^2 +11\eta^3 + 80 \eta^5 \)\Delta_0^0 \\
	&\leq -\frac{1}{\gamma^2}\eta^2\Delta^{i-1}_{t-1} + 16\eta^3\Delta_0^0\\
	&\leq -\frac{1}{\gamma^2}\eta^2\Delta^{i}_{t-1} + 16\eta^3\Delta_0^0	\label{eq:derivation6}
    \end{align}
		
		
		  %\begin{align}
	%\Delta^i_t - \Delta^i_{t-1} &\leq 4\eta^2\Delta_{t-1}^{i+1} -
	%5\eta^2\Delta_{t-2}^{i+1} +
%2\eta^3(9\Delta_{t-2}^{i+1} + 8\eta^2 \Delta_{t-3}^{i+1} + 8\eta^2 \Delta_{t-4}^{i+1} + 8\eta^2 \Delta_{t-5}^{i+1}) \label{eq:derivation1}\\
	%&\leq \eta^2(18\eta - 1)\Delta_{t-2}^{i+1} +O(\eta^5\Delta^0_0) \\
	%%&\leq \eta^2(2\eta-1)\Delta^{i+1}_{t-2} +
	%%4\eta^4\Delta_0^0 +O(\eta^5 \Delta^0_0)\\
	%&= -\eta^2\Delta^{i+1}_{t-2} + 2\eta^3\Delta_{t-2}^{i+1} +O(\eta^5\Delta^0_0)\\
	%&\leq -\frac{1}{\gamma}\eta^2\Delta^{i}_{t-2} + 2\eta^3\Delta_0^0 +O(\eta^5\Delta^0_0)\\
	%&\leq -\frac{1}{\gamma}\eta^2\Delta^{i}_{t-1} + O(\eta^3\Delta_0^0)	\label{eq:derivation6}
    %\end{align}
    %%\begin{align}
	%\Delta^i_t - \Delta^i_{t-1} &\leq 4\eta^2\Delta_{t-1}^{i+1} -
	%5\eta^2\Delta_{t-2}^{i+1} +
	%\eta^3\((1+\lambda_\infty)\Delta_{t-2}^{i+1} +
	%4\eta\lambda^2_\infty\Delta_{0}^{i+1}\) \label{eq:derivation1}\\
	%&\leq \eta^2((1+\lambda_\infty)\eta - 1)\Delta_{t-2}^{i+1} +
	%4\eta^4\lambda_\infty^2\Delta_0^{i+1} +O(\eta^5\Delta^0_0) \\
	%&\leq \eta^2(2\eta-1)\Delta^{i+1}_{t-2} +
	%4\eta^4\Delta_0^0 +O(\eta^5 \Delta^0_0)\\
	%&= -\eta^2\Delta^{i+1}_{t-2} + \eta^3(2\Delta_{t-2}^{i+1}+4\eta\Delta_0^0) +O(\eta^5\Delta^0_0)\\
	%&\leq -\frac{1}{\gamma}\eta^2\Delta^{i}_{t-2} + 6\eta^3\Delta_0^0 +O(\eta^5\Delta^0_0)\\
	%&\leq -\frac{1}{\gamma}\eta^2\Delta^{i}_{t-1} + 6\eta^3\Delta_0^0 +O(\eta^4\Delta^0_0)	\label{eq:derivation6}
    %\end{align}
where for the first inequality we used Lemma~\ref{lemma:restated bound}, 
for the second inequality we used that $\Delta_{t-1}^{i+1} \le \Delta_{t-2}^{i+1}+16\eta^3\Delta^0_0$ (which is implied by the induction hypothesis), 
for the third inequality we used Lemma~\ref{lemma:lower bound},
for the fourth inequality we used Lemma~\ref{lemma:semi-trivial upper bound},
for the fifth inequality we applied the induction hypothesis iteratively, for the sixth inequality we used Eq.~\eqref{eq:target condition 1.5}, for the seventh and eighth inequality we used that $\eta$ is small enough, and for the last inequality we used Lemma~\ref{lemma:semi-trivial upper bound}.
%
 %and that $\Delta_{0}^{i+1} \le \Delta_{0}^{0}$ (which also easily follows from the fact that $\lambda_\infty \le 1$), 
%for the fourth inequality we used Lemma~\ref{lemma:lower bound} and that $\Delta^{i+1}_{t-2} \le \Delta_0^0 + O(\eta^2\Delta^0_0)$, which follows by first noting that $\Delta^{i+1}_{t-2} \le \Delta^{0}_{t-2}$ (which easily follows from the fact that $\lambda_\infty \le 1$) and then noting that $\Delta^{0}_{t-2}\le \Delta^{0}_{0}+O(\eta^2\Delta^0_0)$ (which follows by iteratively applying the inductive hypothesis and using~\eqref{eq:target condition 1}), and for the last inequality we used that $\Delta^{i}_{t-1}\le \Delta^{i}_{t-2}+O(\eta^3\Delta^0_0)$, which follows from the inductive hypothesis, and that $\eta < \gamma$. 
Hence:
$$\Delta^i_t \leq \(1-{\eta^2 \over \gamma^2 }\)\Delta^i_{t-1} + 16\eta^3\Delta^0_0.$$
This completes the proof of our inductive step.

It remains to show the basis of the induction, namely that $H(i,3)$ holds for all $i \in \mathbb{N}$. From Lemma~\ref{lemma:restated bound} we have:
 \begin{align}
	\Delta^i_3 - \Delta^i_{2} &\leq 4\eta^2\Delta^{i+1}_{2} - 5\eta^2\Delta^{i+1}_{1} +
    \eta^3(\Delta_{1}^{i+1} + \Delta_{-1}^{i+1})\\
 &\leq 4\eta^2\Delta^{i+1}_{2} - 5\eta^2\Delta^{i+1}_{0}+5\eta^3 \Delta^{i+1}_0\\
&= -\eta^2\Delta^{i+1}_{2}+ 5\eta^2(\Delta^{i+1}_{2}-\Delta^{i+1}_{0})+5\eta^3 \Delta^{i+1}_0\\
&\le -\eta^2\Delta^{i+1}_{2}+ 5\eta^3(2+\eta)\Delta^{i+1}_{0}+5\eta^3 \Delta^{i+1}_0\\
&= -\eta^2\Delta^{i+1}_{2}+ 5\eta^3(3+\eta)\Delta^{i+1}_{0}\\
 &\le -\eta^2\Delta^{i+1}_{2}+15\eta^3(1+\eta/3)\Delta^0_{0}\\
 &\leq -{\eta^2 \over \gamma^2}\Delta^{i-1}_{2}+15\eta^3(1+\eta/3)\Delta^0_{0}\\
 &\leq -{\eta^2 \over \gamma^2}\Delta^{i}_{2}+15\eta^3(1+\eta/3)\Delta^0_{0},
	\end{align}
where for the second equality we used that $0.5x_{-1}=x_0=x_1$ and $0.5y_{-1}=y_0=y_1$ (which follow from our initialization),
for the third inequality we used that~\eqref{eq:target condition 1.5},
for the fourth inequality we used Lemma~\ref{lemma:semi-trivial upper bound}, 
for the fifth inequality we used Lemma~\ref{lemma:lower bound}, 
and for the last inequality we used Lemma~\ref{lemma:semi-trivial upper bound}. Hence, for small enough $\eta$, we have:
$$\Delta^i_3 \le \(1-{\eta^2 \over \gamma^2}\) \Delta^i_{2} + 16 \eta^3 \Delta^0_0.$$
%To do this, we follow the  derivation in lines~\eqref{eq:derivation1}-\eqref{eq:derivation6} above, noticing that what we needed for this derivation to go through holds for $t=2$. In particular, the first inequality uses Lemma~\ref{lemma:restated bound}, which holds for $t=2$. The second inequality goes through because $\Delta_{1}^{i+1} = \Delta_{0}^{i+1}$, for all $i$, given~\eqref{eq:target condition 1}. The third inequality goes through for the same reasons that were used in the induction step. The fourth inequality goes through since $\Delta_{0}^{i+1} \geq \frac{1}{\gamma}\Delta_{0}^{i}$, which holds from Lemma~\ref{lemma:lower bound}, and $\Delta^{i+1}_{0}\le \Delta^{0}_{0}$, which holds since $\lambda_{\infty}\le 1$. The last inequality follows from the fact that $\Delta^{i}_{1} = \Delta^{i}_{0}$.
    \end{proof}

\begin{corollary} \label{cor:gradient becomes small}
Under the conditions of Theorem~\ref{thm:convergence of OGD}, $\Delta^0_t \equiv \normlt{A^Tx_t} + \normlt{Ay_t} \rightarrow O(\eta \gamma^2 \Delta^0_0)$ as $t \rightarrow +\infty$. In particular, for large enough $t$, the last iterate of OMD is within $O\(\sqrt{\eta} \cdot \gamma \sqrt{\Delta^0_0}\)$ distance from the space of equilibrium points of the game, where $\sqrt{\Delta^0_0}$ is the distance of the initial point $(x_0,y_0)$ from the equilibrium space, and where both distances are taken with respect to the norm $\sqrt{x^T A A^T x + y^T A^T A y}$.
%In particular, for large enough $t$, $x_t$ and $y_t$ are approximately optimal in the following sense:
%\begin{align}
%x_t^T A y_t \le \min_x x^T A y_t + O(\eta^2\Delta^0_0);\\
%x_t^T A y_t \ge \max_y x^T A y + O(\eta^2\Delta^0_0).
%\end{align}
\end{corollary}
\begin{prevproof}{Corollary}{cor:gradient becomes small}
It follows from~\eqref{eq:target condition 1},~\eqref{eq:target condition 1.5} and~\eqref{eq:target condition 2} that:
\begin{align*}
\Delta^0_t &\le \(1-{\eta^2 \over \gamma^2}\)^{t-2}  (1+\eta)^2 \Delta^0_0 + 16 \sum_{t=0}^{\infty}\(1-{\eta^2 \over \gamma^2}\)^t\eta^3 \Delta^0_0 \\&= \(1-{\eta^2 \over \gamma^2}\)^{t-2}  (1+\eta)^2 \Delta^0_0 + O\(\eta \gamma^2 \Delta^0_0\),
\end{align*}
which shows the first part of our claim. For the second part of our claim recall that the solutions to~\eqref{eq:our minmax} are all pairs $(x,y)$ such that $x$ is in the null space of $A^T$ and $y$ is in the null space of $A$. 
%For our second claim let us pick a $t$ such that $(1-\eta)^t=\eta^2$. For such $t$, $\Delta^0_t = O\(\eta^2 \Delta^0_0\)$.
\end{prevproof}

\subsection{General Bilinear Case} \label{app:proof of general case minmax}

\begin{theorem}\label{theorem:general}
    Consider OMD for the min-max problem~\eqref{eq:general inf sup}:
		\begin{align*}
    \inf_{x} \sup_{y} \left(x^TAy + b^Tx + c^Ty + d\right). \label{eq:general inf sup}
\end{align*}
Under the same conditions as Corollary~\ref{cor:gradient becomes small} and whenever~\eqref{eq:general inf sup} is finite, OMD exhibits last iterate convergence in the same sense as in Corollary~\ref{cor:gradient becomes small}. In particular, for large enough $t$, the last iterate of OMD is within $O\(\sqrt{\eta} \cdot \gamma \sqrt{\Delta^0_0}\)$ distance from the space of equilibrium points of the game, where $\sqrt{\Delta_0}$ is the distance of the point $(x_0+(A^T)^+c,y_0+A^+b)$ from the equilibrium space, and where both distances are taken with respect to the norm $\sqrt{x^T A A^T x + y^T A^T A y}$. Whenever~\eqref{eq:general inf sup} is infinite or undefined, the OMD dynamics travels to infinity and we characterize its motion.
\end{theorem}
\begin{prevproof}{Theorem}{theorem:general}
Trivially, we need only consider functions of the form $x^TAy + b^Tx + c^Ty$. We consider the following decompositions
of $b$ and $c$: 
\[
    b &= b_1 + b_2 &\text{where}~b_1 \in \mathcal{R}(A), b_2 \in \mathcal{N}(A^T)  \\ 
    c &= c_1 + c_2 &\text{where}~c_1 \in \mathcal{R}(A^T), c_2 \in \mathcal{N}(A) 
		\]
		Given the above we can also define $b_3$ and $c_3$ as follows:
\[
    Ac_3 &= b_1 &\text{ feasible since } b_1 \in \mathcal{R}(A) \\
    A^Tb_3 &= c_1&\text{ feasible since } c_1 \in \mathcal{R}(A^T)  
\]

Then, we can make the following variable substition:
\[
    \alpha_t &= x_t + \eta t b_2 + b_3 \\
    \beta_t &= y_t - \eta t c_2 + c_3 \\
    \text{so that: }& \\
    A^T\alpha_t &= A^Tx_t + \eta t A^T b_2 + A^T b_3 \\
    &= A^Tx_t + c_1~~~~\text{since $b_2 \in \mathcal{N}(A^T)$} \\
    A\beta_t &= Ay_t - \eta t A c_2 + A c_3  \\
    &= Ay_t + b_1~~~~\text{since $c_2 \in \mathcal{N}(A)$}  \\
\]

We also state the OMD dynamics for $x_t$ and $y_t$ for problem~\eqref{eq:general inf sup}:
\[
    x_t &= x_{t-1} - 2\eta (A y_{t-1} + b)  + \eta (A y_{t-2} + b) \\ 
    &= x_{t-1} - 2\eta A y_{t-1} + \eta A y_{t-2} - \eta b \\ 
    y_t &= y_{t-1} + 2\eta (A^T x_{t-1} + c)  - \eta (A^T x_{t-2} + c) \\
    &= y_{t-1} + 2\eta A^T x_{t-1}  - \eta A^T x_{t-2} + \eta c 
\]

Note that given this update step:
\[
    x_{t+1} &= x_{t} - 2\eta A y_t + \eta A y_{t-1} - \eta b \\
    x_{t+1} &= x_{t} - \eta b_2 - 2\eta A y_t + \eta A y_{t-1} - \eta A c_3 \\
    x_{t+1} &= x_t - \eta b_2 - 2\eta A (y_t + c_3) + \eta A (y_{t-1} + c_3) \\
    x_{t+1} &= x_t - \eta b_2 - 2\eta A (y_t - \eta c_2 t + c_3) + \eta A (y_{t-1} - \eta c_2 (t-1) + c_3) \\
    x_{t+1} + \eta b_2 (t+1) &= x_t + \eta b_2 t - 2\eta A (y_t - \eta c_2 t + c_3) + \eta A (y_{t-1} - \eta c_2 (t-1) + c_3) \\
    x_{t+1} + \eta b_2 (t+1) + b_3  &= x_t + \eta b_2 t + b_3 - 2\eta A (y_t -
    \eta c_2 t + c_3) + \eta A (y_{t-1} - \eta c_2 (t-1) + c_3) \\
    \alpha_{t+1} &= \alpha_t - 2\eta A \beta_t + \eta A \beta_{t-1} \\
    \text{Analogously: }& \\
    \beta_{t+1} &= \beta_t + 2\eta A^T \alpha_t - \eta A^T \alpha_{t-1}
\]
Note that these are precisely the dynamics for which we proved convergence in
Theorem~\ref{thm:convergence of OGD-main}. Thus, by invoking Theorem~\ref{thm:convergence of OGD} and Corollary~\ref{cor:gradient becomes small} on the sequence $(\alpha_t,\beta_t)$ and then substituting back $(x_t,y_t)$, we have that for all large enough $t$:
\[
    x_t &= -\eta b_2 t - b_3 + \epsilon_x(t) \\
    y_t &= \eta c_2 t - c_3 + \epsilon_y(t) \\
    &\text{such that } ||A^T\epsilon_x(t)||_2, ||A\epsilon_y(t)||_2 \in O\(\sqrt{\eta} \cdot \gamma \sqrt{\Delta^0_0}\),
\]
where $\Delta^0_0 = ||A^T(x_0+b_3)||_2^2 + ||A (y_0+c3)||_2^2$.

In particular, this shows that, whenever~\eqref{eq:general inf sup} is finite (i.e.~$b_2=c_2=0$), OMD exhibits last iterate convergence. For large enough $t$, the last iterate of OMD is within $O\(\sqrt{\eta} \cdot \gamma \sqrt{\Delta^0_0}\)$ distance from the space of equilibrium points of the game, where $\sqrt{\Delta^0_0}$ is the distance of $(x_0+b_3,y_0+c_3)$ from the equilibrium space in the norm $\sqrt{x^T A A^T x + y^T A^T A y}$. Whenever~\eqref{eq:general inf sup} is infinite or undefined, the OMD dynamics travels to infinity linearly, with fluctuations around the divergence specified as above. 
\end{prevproof}
%\begin{corollary} \label{cor:general bilinear functions}
%    OGD converges as in the last corollary for functions of the form:
%    $$f(x,y) = (x+b)^TA(x+c)$$
%\end{corollary}
%\begin{prevproof}{Corollary}{cor:general bilinear functions}
%    Let $\alpha_t = x_t + b$, and $\beta_t = y_t + c$. Then, note that the
%    function can be written simply as $f(\alpha, \beta) = \alpha^TA\beta$. Thus,
%    by the main result, we have that both $\alpha, \beta \rightarrow 0$, which
%    in turn implies that $x_t \rightarrow b$, $y_t \rightarrow c$ as required.
%\end{prevproof}

%\begin{remark}{}
%    The above corrollary actually holds for any function $f$ of the form:
%    $$f(x, y) = x^TAy + bx + cy + r$$
%    If $b \in \mathcal{R}(A), c \in \mathcal{R}(A^T)$. This is because if these
%    two hold, we can write $d, e$ s.t. $Ae = b$, $A^Td = c$, then:
%    $$f(x, y) = (x+d)^TA(y+e) - (d^Te + r)$$
%    The constant term can be disregarded since it is not present in any of the
%    gradients, and thus this reduces to the corollary above. Note that if one
%    of these conditions does not hold, the optimum may be ill-defined. For
%    example, $A = [1,0;0,0], b = [1,1], c = [1,1]$ is a configuration such that
%    $f(x,y) = x_1y_1 + x_1 + x_2 + y_1 + y_2 + r$; here, $x_2$ and $y_2$ can be
%    set to $+\infty$, $-\infty$ respectively, so the optimum itself is not
%    well-defined.
%\end{remark}

\subsection{Omitted Proofs} \label{appendix:omitted proofs}

\begin{prevproof}{Proposition}{prop:gradient descent unstable}
    To show this, we consider the $\ell_2$ distance of the solution at time $t$.
    First, recall the GD update step in the special case of $f(x, y) = x^Ty$:
    \[
	x_{t} = x_{t-1} - \eta y_{t-1} \\
	y_{t} = y_{t-1} + \eta x_{t-1}
    \]

    Then, note that the squared $\ell_2$ distance of the running iterate $(x_t,y_t)$ to the unique equilibrium solution $(0,0)$  is given
    by $d(t) := ||x_t||_2^2 + ||y_t||_2^2$, which we can calculate:
    \[
	||x_t||_2^2 &= ||x_{t-1}||^2_2 - 2\eta x_{t-1}^Ty_{t-1} +
	\eta^2||y_{t-1}||_2^2 \\
	||y_{t-1}||_2^2 &= ||y_{t-1}||^2_2 + 2\eta y_{t-1}^T x_{t-1} +
	\eta^2||x_{t-1}||_2^2 \\
	\therefore d(t) &= d(t-1) + \eta^2 d(t-1) \\
	&= (1+\eta^2)d(t-1)
    \]
    This indicates that for any value of $\eta>0$, the running iterate of 
    GD \textit{diverges} from the equilibrium.
\end{prevproof}

\begin{prevproof}{Claim}{claim:pushing A's around}
For our first claim, observe that:
\[
    \innerab{Au}{Av}{i} &= u^TA^TAM_i^TM_iA^TAv \\
			&= u^TA^TA(A^TA)^k(A^j)^TA^j(A^TA)^kA^TAv \\
   &= u^TA^T(AA^T)^kA(A^j)^TA^jA^T(AA^T)^kAv \\
   &= u^TA^T[(AA^T)^kA(A^j)^T][A^jA^T(AA^T)^k]Av \\
   &= u^TA^TN_{i+1}^TN_{i+1}Av \\
   &= \innerac{u}{v}{i+1}
\]
Our second claim, $\innerac{A^Tu}{A^Tv}{i} = \innerab{u}{v}{i+1}$, is proven analogously.

For our third claim:
\[
    \innerab{u}{Av}{i} &= u^TAM_i^TM_iA^TAv \\
		       &= u^TA(A^TA)^{k}(A^TA)^j(A^TA)^kA^TAv \\
    \text{if $j = 0$: }&  \\
		       &= u^TA(A^TA)^{k}(A^TA)^kA^TAv \\
	 &= u^TA[A^T(AA^T)^k][(AA^T)^kA]v \\
  &= u^TA [A^T N_{i}^T][N_{i}A]v \\
  &= \innerac{v}{A^Tu}{i}\\
    \text{otherwise: }& \\
		      &= u^TA(A^TA)^kA^TA(A^TA)^kA^TAv \\
					&= u^TA[(A^TA)^kA^TA][(A^TA)^kA^TA]v \\
					&= u^TA[A^T(AA^T)^kA][A^T(AA^T)^kA]v \\
										&= u^TA[A^TN_i^T][N_iA]v \\
 &= \innerac{v}{A^Tu}{i} 
\]
\end{prevproof}

\begin{prevproof}{Lemma}{lemma:first bound}
First, we note the following scaled update rule:
\[
    M_{i}A^Tx_t &= M_i\(\xtao\) \\
    N_{i}Ay_t &= N_i\(\ytao\)
\]

\noindent Then, taking the norm of both sides, and using the statements of Claim~\ref{claim:pushing A's around}:
\[
    \normab{x_t}{i} &= \normab{x_{t-1}}{i} + 4\eta^2\normac{y_{t-1}}{i+1} +
    \eta^2\normac{y_{t-2}}{i+1} - 4\eta\innerab{x_{t-1}}{Ay_{t-1}}{i} \\
    &\qquad+ 2\eta\innerab{x_{t-1}}{Ay_{t-2}}{i} -
    4\eta^2\innerac{y_{t-1}}{y_{t-2}}{i+1} \\[2ex]
    \normac{y_t}{i} &= \normac{y_{t-1}}{i} + 4\eta^2\normab{x_{t-1}}{i+1} +
    \eta^2\normab{x_{t-2}}{i+1} + 4\eta\innerac{y_{t-1}}{A^Tx_{t-1}}{i} \\
    &\qquad+ 2\eta\innerac{y_{t-1}}{A^T x_{t-2}}{i} -
    4\eta^2\innerab{x_{t-1}}{x_{t-2}}{i+1} \\[2ex]
    \therefore \Delta_t^i &= \normab{x_t}{i} + \normac{y_t}{i} \\
    &= \Delta^i_{t-1} + 4\eta^2\Delta^{i+1}_{t-1} +
    \eta^2\Delta^{i+1}_{t-2} + 2\eta(\innerab{x_{t-1}}{Ay_{t-2}}{i} -
    \innerac{y_{t-1}}{Ax_{t-2}}{i}) \\
    &\qquad\qquad - 4\eta^2(\innerab{x_{t-1}}{x_{t-2}}{i+1} +
    \innerac{y_{t-1}}{y_{t-2}}{i+1})
\]

\noindent Expanding the first pair of inner products above and using Claim~\ref{claim:pushing A's around} again:
\[
   \innerab{x_{t-1}}{Ay_{t-2}}{i} - \innerac{y_{t-1}}{A^T x_{t-2}}{i} &=
    \innerab{\xtt}{Ay_{t-2}}{i} \\
    &\qquad - \innerac{\ytt}{A^T x_{t-2}}{i} \\[2ex]
    = -2\eta(\normac{y_{t-2}}{i+1}+&\normab{x_{t-2}}{i+1}) + \eta(\innerab{x_{t-2}}{x_{t-3}}{i+1} +
    \innerac{y_{t-2}}{y_{t-3}}{i+1})
\]

Then, multiplying by $2\eta$ and substituting into the previous derivation yields:
\[
    \Delta^i_t - \Delta^i_{t-1} &= 4\eta^2\Delta^{i+1}_{t-1} -
    3\eta^2\Delta^{i+1}_{t-2} + 2\eta^2(\innerab{x_{t-2}}{x_{t-3}}{i+1} +
    \innerac{y_{t-2}}{y_{t-3}}{i+1}) \\
    &\qquad\qquad - 4\eta^2(\innerab{x_{t-1}}{x_{t-2}}{i+1} +
    \innerac{y_{t-1}}{y_{t-2}}{i+1})
\]

\noindent Now, consider the following inner product:
\[
    \innerab{x_{t-2}}{x_{t-1}}{i+1} + \innerac{y_{t-2}}{y_{t-1}}{i+1} &=
    \innerab{x_{t-2}}{\xtt}{i+1} \\
    &\qquad + \innerac{y_{t-2}}{\ytt}{i+1} \\[2ex]
    &= \Delta^{i+1}_{t-2} + \eta\(\innerab{x_{t-2}}{Ay_{t-3}}{i+1} -
    \innerac{y_{t-2}}{A^Tx_{t-3}}{i+1}\)
\]

\noindent Once again, we multiply by $-4\eta^2$ and substitute:
\[
    \Delta^i_t - \Delta^i_{t-1} &= 4\eta^2\Delta^{i+1}_{t-1} -
    7\eta^2\Delta^{i+1}_{t-2} + 2\eta^2(\innerab{x_{t-2}}{x_{t-3}}{i+1} +
    \innerac{y_{t-2}}{y_{t-3}}{i+1}) \\
    &\qquad + 4\eta^3(\innerac{y_{t-2}}{A^Tx_{t-3}}{i+1} -
    \innerab{x_{t-2}}{Ay_{t-3}}{i+1})
\]

Now, we use the update step for time $t-2$. For all $t \geq 1$, this is
well-defined, since $x_{-1}$ and $y_{-1}$ are defined. To ensure that this step
is sound for $t = 0$ requires we define the following, where $X^+$ denotes the
generalized inverse:
\[
    x_{-2} = 4x_0 + \frac{1}{\eta}(A^T)^{+} y_0 \\
    y_{-2} = 4y_0 - \frac{1}{\eta}A^+ x_0
\]
We define these such that: $A^Tx_{-2} = 4A^Tx_0 + \frac{y_0}{\eta}$ and
$Ay_{-2} = 4Ay_0 - \frac{x_0}{\eta}$ (since $x_0 \in R(A)$ and $y_0 \in R(A^T)$,
and thus the following equalities hold:
\[
    x_0 = x_{-1} - 2\eta Ay_{-1} + \eta Ay_{-2} \\
    y_0 = y_{-1} + 2\eta A^T x_{-1} - \eta A^Tx_{-2}
\]

\noindent This allows us to use the following expansion freely for all $t \geq 2$:
\[
    x_{t-2} &= \x{}{-3}{-4} &&\implies x_{t-3} - 2\eta Ay_{t-3} = x_{t-2} - \eta Ay_{t-4} \\
    y_{t-2} &= \y{}{-3}{-4} &&\implies y_{t-3} + 2\eta A^Tx_{t-3} = y_{y-2} + \eta A^Tx_{t-4}
\]

\noindent We can gather the inner product terms and use this update rule to get
our final desired result:
\[
    \Delta^i_t - \Delta^i_{t-1} &= 4\eta^2\Delta^{i+1}_{t-1} -
    7\eta^2\Delta^{i+1}_{t-2} + 2\eta^2(\innerab{x_{t-2}}{x_{t-3} - 2\eta
    Ay_{t-3}}{i+1} + \innerac{y_{t-2}}{y_{t-3} + 2\eta A^Tx_{t-3}}{i+1}) \\[1ex]
    &= 4\eta^2\Delta^{i+1}_{t-1} - 5\eta^2\Delta^{i+1}_{t-2} -
    2\eta^3(\innerab{x_{t-2}}{Ay_{t-4}}{i+1} -
    \innerac{y_{t-2}}{A^Tx_{t-4}}{i+1})
\]
\end{prevproof}

\begin{prevproof}{Lemma}{lemma:restated bound}
\noindent To prove this, first consider the following trivial inequality:
\[
    \normac{y_{t-2} - A^Tx_{t-4}}{i+1} &+
    \normab{x_{t-2} + Ay_{t-4}}{i+1} \\
    &= \normac{y_{t-2}}{i+1} -
    2\innerac{y_{t-2}}{A^Tx_{t-4}}{i+1} + \normac{A^Tx_{t-4}}{i+1} \\
    &\qquad+ \normab{x_{t-2}}{i+1} +
    2\innerab{x_{t-2}}{Ay_{t-4}}{i+1} + \normab{Ay_{t-4}}{i+1} \\
    &\geq 0
\]

\noindent Rearranging:
\[
    2\innerac{y_{t-2}}{A^Tx_{t-4}}{i+1} - 2\innerab{x_{t-2}}{Ay_{t-4}}{i+1}
    &\leq \Delta_{t-2}^{i+1} + \(\normac{A^Tx_{t-4}}{i+1} +
    \normab{Ay_{t-4}}{i+1}\) \\
				&\leq \Delta_{t-2}^{i+1} + \lambda_\infty^2\(\normab{x_{t-4}}{i+1} +
    \normac{y_{t-4}}{i+1}\) \\
    &\leq \Delta_{t-2}^{i+1} +
    \(\lambda_\infty^2\Delta_{t-4}^{i+1}\)  \\
		    &\leq \Delta_{t-2}^{i+1} + \Delta_{t-4}^{i+1}
\]

%\vscomment{Why is the above last inequality correct? Also can we maybe also write an intermediate observation that $\|y\|_{A^T N_{i+1}^T}^2 = \|N_{i+1} A y\|_2^2$ and similarly for the $x$.}
%
%\noindent Now, using the update rule we know that:
%\[
    %\normab{x_{t-2}}{i} &= \normab{x_{t} + 2\eta Ay_{t-1} + \eta Ay_{t-2} - \eta
    %Ay_{t-3}}{i}\\
    %&\leq 8\normab{x_{t}}{i} + 8\eta^2\normab{ Ay_{t-1}}{i}+8\eta^2\normab{ Ay_{t-2}}{i} + 8\eta^2\normab{Ay_{t-3}}{i}\\
		%&\leq 8\normab{x_{t}}{i} + 8\eta^2\lambda_{\infty}^2\normac{ y_{t-1}}{i}+8\eta^2\lambda_{\infty}^2\normac{ y_{t-2}}{i} + 8\eta^2\lambda_{\infty}^2\normac{y_{t-3}}{i}
%\]
%Similarly:
%\[
%\normac{y_{t-2}}{i} \leq 8\normac{y_{t}}{i} + 8\eta^2\lambda_{\infty}^2\normab{ x_{t-1}}{i}+8\eta^2\lambda_{\infty}^2\normab{ x_{t-2}}{i} + 8\eta^2\lambda_{\infty}^2\normab{x_{t-3}}{i}
%\]	
%Hence, recalling that $\lambda_{\infty}\le 1$:
%\[	
    %\Delta_{t-2}^i &\leq 8\Delta_{t}^i + 8\eta^2 \Delta_{t-1}^i + 8\eta^2 \Delta_{t-2}^i + 8\eta^2 \Delta_{t-3}^i
		%\\
		   %&\leq \Delta_{t}^i + 4\eta(1-\eta^2)^{t-3}\Delta_{0}^{i+1} \\
    %&\leq \Delta_{t}^i + 4\eta\lambda_\infty\Delta_{0}^{i}
%\]
%%
%%
%\noindent Plugging the above inequality (for time $t-4$ and index $i+1$ on the left hand side)\footnote{If needed (for time $t=2$), we use the same construction of $x_{-2},y_{-2}$ found in the Proof of Lemma~\ref{lemma:first bound}.} in to our earlier bound we get:
%%Applying this for $t-4$ 
%\[
    %2\innerac{y_{t-2}}{A^Tx_{t-4}}{i+1} - 2\innerab{x_{t-2}}{Ay_{t-4}}{i+1}
    %&\leq \Delta_{t-2}^{i+1} +
    %\lambda_\infty^2\(8\Delta_{t-2}^{i+1} + 8\eta^2 \Delta_{t-3}^{i+1} + 8\eta^2 \Delta_{t-4}^{i+1} + 8\eta^2 \Delta_{t-5}^{i+1}\)  \\
		%&\leq 9\Delta_{t-2}^{i+1} + 8\eta^2 \Delta_{t-3}^{i+1} + 8\eta^2 \Delta_{t-4}^{i+1} + 8\eta^2 \Delta_{t-5}^{i+1}
    %%&\leq (1+\lambda_\infty^2)\Delta_{t-2}^{i+1} +
    %%4\eta\lambda^2_\infty\Delta_0^{i+1}
%\]

\noindent Now, we can apply this bound to the result of Lemma~\ref{lemma:first bound}:
\[
    \Delta^i_{t} - \Delta^i_{t-1} &= 4\eta^2\Delta^{i+1}_{t-1} -
    5\eta^2\Delta^{i+1}_{t-2} - 2\eta^3(\innerab{x_{t-2}}{Ay_{t-4}}{i+1} -
    \innerac{y_{t-2}}{A^Tx_{t-4}}{i+1}) \\
    &\leq 4\eta^2\Delta^{i+1}_{t-1} - 5\eta^2\Delta^{i+1}_{t-2} +
    \eta^3(\Delta_{t-2}^{i+1} + \Delta_{t-4}^{i+1})
\]

%\[
    %\Delta^i_{t} - \Delta^i_{t-1} &= 4\eta^2\Delta^{i+1}_{t-1} -
    %5\eta^2\Delta^{i+1}_{t-2} - 2\eta^3(\innerab{x_{t-2}}{Ay_{t-4}}{i+1} -
    %\innerac{y_{t-2}}{A^Tx_{t-4}}{i+1}) \\
    %&\leq 4\eta^2\Delta^{i+1}_{t-1} - 5\eta^2\Delta^{i+1}_{t-2} +
    %2\eta^3(9\Delta_{t-2}^{i+1} + 8\eta^2 \Delta_{t-3}^{i+1} + 8\eta^2 \Delta_{t-4}^{i+1} + 8\eta^2 \Delta_{t-5}^{i+1}) \\
%\]

\noindent Which is what we sought out to prove.
\end{prevproof}

%\begin{prevproof}{Lemma}{lemma:lower bound}
%Now, note that choosing $x_0 \in \mathcal{R}(A) \implies x_t \in
%\mathcal{R}(A) = \mathcal{R}(AA^T)\ \forall\ t$, due to the update step.
%Similarly, $y_t \in \mathcal{R}(A^T)$. Thus, $x_t = AA^T(AA^T)^{+}x_t$ and
%$y_t = A^TA(A^TA)^{+}y_t$, for all $t$. Letting $Q = (AA^T)^{+}$ and $P =
%(A^TA)^{+}$, and recalling key properties of the generalized inverse: 
%\[
    %\Delta_t^{i} &= \norm{M_{i}A^Tx_t}{2} + \norm{N_{i}Ay_t}{2} \\
       %&= \norm{M_iA^TAA^TQx_t}{2} +
    %\norm{N_iAA^TAPy_t}{2}\\
    %&= \norm{M_{i+2}A^TQx_t}{2} +
	%\norm{N_{i+2}APy_t}{2} \\ 
    %&= \norm{(A^T)^jQ(AA^T)^{k+1}x_t}{2} +
	%\norm{A^jP(A^TA)^{k+1}y_t}{2} \\
    %&= \begin{cases}
	%\norm{QM_{i+1}A^Tx}{2} + \norm{PN_{i+1}Ay_t}{2} &\text{ if $j = 0$}\\[0.5ex]
	%\norm{PM_{i+1}A^Tx}{2} + \norm{QN_{i+1}Ay_t}{2} &\text{ if $j = 1$}
    %\end{cases} \\
    %&\leq \max \(||Q||, ||P||\)\cdot \Delta_t^{i+1}
%\]
%\end{prevproof}

		\begin{prevproof}{Lemma}{lemma:semi-trivial upper bound}
		Suppose $j=i \mod 2$ and $k=(i-j)/2$. Notice the following identities:
		\begin{align*}
		&M_i = A^j(A^TA)^k, N_i = \(A^T\)^j \(AA^T\)^k\\
		&M_{i+1} = (A^T)^jA(A^TA)^k, N_{i+1} = A^j A^T\(AA^T\)^k
		\end{align*}
		Now:
		\begin{align*}
		\Delta^{i+1}_t &= \normlt{N_{i+1}Ay_t} + \normlt{M_{i+1}A^Tx_t}\\
		&=\normlt{A^j A^T\(AA^T\)^kAy_t} + \normlt{(A^T)^jA(A^TA)^kA^Tx_t}\\
		&\le \lambda_{\infty}^2\(\normlt{(A^T)^j\(AA^T\)^kAy_t} + \normlt{A^j(A^TA)^kA^Tx_t}\)\\
		&\le \lambda_{\infty}^2\(\normlt{N_iAy_t} + \normlt{M_iA^Tx_t}\)\\
		&\le \Delta^{i}_t,
		\end{align*}
		where for the last inequality we used that $\lambda_{\infty}\le 1$.
		\end{prevproof}
		
\begin{prevproof}{Lemma}{lemma:lower bound}
Given our initialization, $x_0 \in \mathcal{R}(A)$. This implies $x_t \in
\mathcal{R}(A), \forall\ t$, due to the update step of the dynamics. Recalling key properties of the matrix pseudoinverse, this implies: $x_t \equiv A A^+ x_t = A A^T (A A^T)^+ x_t$, for all $t$.
Similarly, given our initialization, $y_t \in \mathcal{R}(A^T)$, for all $t$, which implies $y_t \equiv A^T (A^T)^+ y_t = A^T A (A^T A)^+ y_t$, for all $t$. Letting $Q = (AA^T)^{+}$ and $P =
(A^TA)^{+}$, we get the following (where $j = i \mod 2$ and $k= (i-j)/2$): 
\[
    \Delta_t^{i} &= \norm{M_{i}A^Tx_t}{2} + \norm{N_{i}Ay_t}{2} \\
       &= \norm{M_iA^TAA^TQx_t}{2} +
    \norm{N_iAA^TAPy_t}{2}\\
    &= \norm{M_{i+2}A^TQx_t}{2} +
	\norm{N_{i+2}APy_t}{2} \\ 
	&= \norm{A^j (A^TA)^{k+1}A^T (AA^T)^{+} x_t}{2} +
	\norm{(A^T)^j (AA^T)^{k+1}A(A^TA)^+y_t}{2} \\ 
    &= \norm{A^j (A^TA)^{+} A^T (AA^T)^{k+1}x_t}{2} +
	\norm{(A^T)^j (AA^T)^{+} A(A^TA)^{k+1}y_t}{2} \\
    &= \begin{cases}
	\norm{(A^TA)^{+} M_{i+2}A^Tx_t}{2} + \norm{(AA^T)^{+}N_{i+2}Ay_t}{2} &\text{ if $j = 0$}\\[0.5ex]
	\norm{(AA^T)^{+}M_{i+2}A^Tx_t}{2} + \norm{(A^TA)^{+} N_{i+2}Ay_t}{2} &\text{ if $j = 1$}
    \end{cases} \\
    &\leq \max \(||Q||, ||P||\)^2\cdot \Delta_t^{i+2},
\]
where for the fourth and fifth equality we used the following key property of pseudo-inverses: $A^+=(A^TA)^+A^T=A^T(AA^T)^+$.
\end{prevproof}



\section{DNA-Generation WGAN Architecture}\label{sec:appendix-dna-arch}\label{sec:apdxdna}
\begin{table}[H]
\label{table:arch}
\begin{center}
\begin{tabular}{lllll}
\multicolumn{1}{c}{\bf Operation}  &\multicolumn{1}{c}{\bf Kernel} &\multicolumn{1}{c}{\bf Output Shape} &\multicolumn{1}{c}{\bf BatchNorm?} &\multicolumn{1}{c}{\bf Nonlinearity}
\\ \hline \\
Length of DNA sequence: $L = 6$ \\
Gradient penalty: $\lambda=1e^{-4}$\\
Batch size: $512$\\
$G(z):$ \\
$z$ & - & 50 & - & - \\
Fully connected         & - & 128 & no & tanh\\
Fully connected             & - & $16\times \frac{L}{2}$  & yes & tanh \\
Reshape  & - & $ 16 \times 1 \times \frac{L}{2}$ & - & -\\
Upsampling by 2 & - & $16 \times 1 \times L$ & - & - \\
 Convolution             & $[1\times 3]\times 4$ & $4\times1\times L$ & no & tanh  \\
 $D(x):$ \\
$x$   & - & $4 \times  1 \times L$  &- &-\\
  Convolution             & $[1\times 3] \times 16$ &  $16\times1\times L$ & no & tanh  \\
  Fully connected             & - & 32  & no & tanh \\
  Fully connected             & - & 1  & no & linear \\
\end{tabular}
\end{center}
\end{table}

\section{CIFAR10 WGAN Architecture}\label{sec:appendix-cifar10-arch}
\begin{table}[H]
\label{table:arch}
\begin{center}
\begin{tabular}{lllll}
\multicolumn{1}{c}{\bf Operation}  &\multicolumn{1}{c}{\bf Kernel} &\multicolumn{1}{c}{\bf Output Shape} &\multicolumn{1}{c}{\bf BatchNorm?} &\multicolumn{1}{c}{\bf Nonlinearity}
\\ \hline \\
Gradient penalty: $\lambda=10$\\
Batch size: 64 \\
$G(z):$ \\
$z$ & - & 100 & - & - \\
Fully connected         & - & 1024 & no & LeakyReLU\\
Fully connected             & - & $8192$  & yes & LeakyReLU \\
Reshape  & - & $ 128 \times 8 \times 8 $ & - & -\\
TransposedConv             & $[5\times 5]\times 128$ & $128\times16\times 16$ & yes & LeakyReLU  \\
Convolution             & $[5\times 5]\times 64$ & $64\times16\times 16$ & yes & LeakyReLU  \\
TransposedConv             & $[5\times 5]\times 64$ & $64\times32\times 32$ & yes & LeakyReLU  \\
Convolution             & $[5\times 5]\times 3$ & $3\times32\times 32$ & no & tanh  \\
 
 $D(x):$ \\
$x$   & - & $3 \times  32 \times 32$  &- &-\\
  Convolution             & $[5\times 5] \times 64$ &  $64\times32\times32$ & no & LeakyReLU  \\
   Convolution             & $[5\times 5] \times 128$ &  $128\times14\times 14$ & no & LeakyReLU  \\
    Convolution             & $[5\times 5] \times 128$ &  $128\times 7\times 7$ & no & LeakyReLU  \\
  Fully connected             & - & 1024  & no & LeakyReLU \\
  Fully connected             & - & 1  & no & linear \\
\end{tabular}
\end{center}
\end{table}

\newpage

\section{CIFAR10 Generator Image Samples}\label{sec:appendix-cifar10}
\begin{figure}[H]
    \centering  
    \begin{subfigure}[b]{1\textwidth}
        \centering
    		\includegraphics[]{optimAdam_v0_1e-04_ratio1_epoch93-eps-converted-to.pdf}
    	\caption{Sample of images from Generator of Epoch $94$, which had the highest inception score.}
    \end{subfigure}
    \caption{Samples of images from Generator trained via Optimistic Adam on CIFAR10.}\label{fig:optimistic-Adam}
\end{figure}

\subsection{Comparison of Early Epoch Images of Optimistic Adam vs Adam}
Below we give samples of images from an early epoch $19$ Generator trained via Optimistic Adam with 1:1 training ratio, Adam with 1:1 and Adam with 5:1 ratio on CIFAR10. We see that Optimistic Adam has already achieved visually appealing results unlike the latter two vanilla Adam based versions.

\begin{figure}[H]
        \centering
    		\includegraphics[height=4in]{optimAdam_v0_1e-04_ratio1_epoch19-eps-converted-to.pdf}
    	\caption{Sample of images from Generator of Epoch $19$ trained via Optimistic Adam and 1:1 training ratio.}
\end{figure}
\begin{figure}[H]
        \centering
    		\includegraphics[height=4in]{adam_v0_1e-04_ratio1_epoch19-eps-converted-to.pdf}
    	\caption{Sample of images from Generator of Epoch $19$ trained via  Adam and 1:1 training ratio.}
\end{figure} 
\begin{figure}[H]
        \centering
    		\includegraphics[height=4in]{adam_v0_1e-04_epoch19-eps-converted-to.pdf}
    	\caption{Sample of images from Generator of Epoch $19$ trained via  Adam and 5:1 training ratio.}
\end{figure}
    

\section{CIFAR10 Adam vs. Optimistic Adam Comparison}\label{sec:appendix-errorbars}
\begin{figure}[H]
    \centering  
    \includegraphics[width=0.9\textwidth]{cifar10-error-bars.pdf}
    \caption{The inception scores across epochs for GANs trained with Optimistic Adam (ratio 1) and Adam (ratio 5) on CIFAR10 (the two top-performing optimizers found in Section~\ref{sec:cifar10}, with 10\%-90\% confidence intervals. The GANs were trained for 30 epochs and results gathered across 35 runs.}\label{fig:optimistic-Adam}
\end{figure}


\end{document}
