\documentclass{article} % For LaTeX2e
\usepackage{iclr2018_conference_arxiv,times}
\iclrfinaltrue

\usepackage{graphicx}
\usepackage{epstopdf}
\usepackage{hyperref}
\usepackage{url}

%% packages
\usepackage{xspace}
\usepackage{graphicx}
\usepackage{amsmath,amssymb,amsthm}
\usepackage{booktabs} % For formal tables
%\usepackage{color}
\usepackage{nicefrac}
%\usepackage{graphicx}
%% HERE arxiv
\usepackage[font=small]{caption}
\usepackage{subcaption}
\usepackage{multirow}
\usepackage{algorithm}
\usepackage[noend]{algpseudocode}
\usepackage{bbm}
\usepackage{mathtools}

\usepackage{enumitem}


%\usepackage[]{color-edits}
\usepackage[suppress]{color-edits}
\addauthor{vs}{blue}
\addauthor{cd}{red}
\addauthor{hz}{brown}
\addauthor{ai}{pink}



\definecolor{amber}{rgb}{1.0, 0.01, 0.5}

\newcount\Comments 
\Comments=1
\newcommand{\kibitz}[2]{\ifnum\Comments=1{\color{#1}{#2}}\fi}
\newcommand{\zf}[1]{\kibitz{amber}{[ZF: #1]}}
\newcommand{\cp}[1]{\kibitz{red}{[CP: #1]}}
\newcommand{\vs}[1]{\kibitz{blue}{[VS: #1]}}
\newcommand{\todo}[1]{\kibitz{blue}{[TODO: #1]}}
\newcommand{\E}{\mathbb{E}}
\DeclareMathOperator*{\argmin}{arg\,min}
\DeclareMathOperator*{\argmax}{arg\,max}
\DeclareMathOperator*{\argsup}{arg\,sup}
\DeclareMathOperator*{\arginf}{arg\,inf}
\newcommand{\1}{\mathbbm{1}}
\newcommand{\st}{\sum{t=1}^{T}}
\newcommand{\eps}{\epsilon}
\newcommand{\G}{\mathcal{G}^\eps}
\newcommand{\kk}{\textbf{k}}
\newcommand{\R}{\mathbb{R}}
\newcommand{\B}{\mathcal{B}}
\newcommand{\D}{\mathcal D}
\newcommand{\mbf}{\mathbf}
\newcommand{\p}{\mathbf{p}}
\newcommand{\s}{\mathbf{s}}
\newcommand{\g}{\textbf{g}}
\newcommand{\w}{\mbf{w}}
\newcommand{\winexp}{\textsc{WIN-EXP}}
\newcommand{\winexpG}{\textsc{WIN-EXP-G}}
\renewcommand{\Pr}{\ensuremath{\mathrm{Pr}}}
\newcommand{\opt}{\ensuremath{\textsc{OPT}}}
\newcommand{\twopartdef}[4]
{
	\left\{
		\begin{array}{ll}
			#1 & \mbox{if } #2 \\
			#3 & \mbox{if } #4
		\end{array}
	\right.
}

%% HERE arXiv
\theoremstyle{plain}
\newtheorem{theorem}{Theorem}[]
\newtheorem{corollary}[theorem]{Corollary}
\newtheorem{lemma}[theorem]{Lemma}
\newtheorem{definition}[theorem]{Definition}
\newtheorem{example}[]{Example}
\newenvironment{prevproof}[2]{\noindent {\em {Proof of {#1}~\ref{#2}:}}}{$\hfill\qed$\vskip \belowdisplayskip}
\newtheorem{remark}{Remark}
\newtheorem{proposition}{Proposition}
\newtheorem{claim}{Claim}



\newcommand{\sgamma}{\ensuremath{\sqrt{\gamma}}}
\newcommand{\inner}[3]{\ensuremath{\langle {#1}, {#2} \rangle_{{#3}}}}
\newcommand{\innerab}[3]{\ensuremath{\inner{{#1}}{{#2}}{AM_{{#3}}^T}}}
\newcommand{\innerac}[3]{\ensuremath{\inner{{#1}}{{#2}}{A^TN_{{#3}}^T}}}
\newcommand{\inneraccostas}[3]{\ensuremath{\inner{{#1}}{{#2}}{A^TN_{{#3}}}}}

\newcommand{\norm}[2]{\ensuremath{\left|\left|{#1}\right|\right|^2_{{#2}}}}

\newcommand{\norma}[1]{\ensuremath{\left|\left|{#1}\right|\right|^2_{A^TA}}}
\newcommand{\normat}[1]{\ensuremath{\left|\left|{#1}\right|\right|^2_{AA^T}}}
\newcommand{\normab}[2]{\ensuremath{\norm{{#1}}{AM_{{#2}}^T}}}
\newcommand{\normac}[2]{\ensuremath{\norm{{#1}}{A^TN_{{#2}}^T}}}

\newcommand{\normabb}[2]{\ensuremath{\normm{{#1}}{AM_{{#2}}^T}}}
\newcommand{\normacc}[2]{\ensuremath{\normm{{#1}}{A^TN_{{#2}}^T}}}
\newcommand{\normm}[2]{\ensuremath{\left|\left|{#1}\right|\right|_{{#2}}}}


\newcommand{\normlt}[1]{\ensuremath{\left|\left|{#1}\right|\right|^2_2}}

\newcommand{\x}[3]{\ensuremath{{#1}x_{t{#2}} - 2\eta{#1}Ay_{t{#2}} +
\eta{#1}Ay_{t{#3}}}}
\newcommand{\xto}{\ensuremath{\x{}{-1}{-2}}}
\newcommand{\xtt}{\ensuremath{\x{}{-2}{-3}}}
\newcommand{\xtao}{\ensuremath{\x{A^T}{-1}{-2}}}

\newcommand{\y}[3]{\ensuremath{{#1}y_{t{#2}} + 2\eta{#1}A^Tx_{t{#2}} -
\eta{#1}A^Tx_{t{#3}}}}
\newcommand{\yto}{\ensuremath{\y{}{-1}{-2}}}
\newcommand{\ytt}{\ensuremath{\y{}{-2}{-3}}}
\newcommand{\ytao}{\ensuremath{\y{A}{-1}{-2}}}

\DeclarePairedDelimiter\floor{\lfloor}{\rfloor}

\def\[#1\]{\begin{align*}#1\end{align*}}
\def\(#1\){\ensuremath{\left(#1\right)}}


\begin{document}
% \nipsfinalcopy is no longer used


\title{Training GANs with Optimism}

% The \author macro works with any number of authors. There are two
% commands used to separate the names and addresses of multiple
% authors: \And and \AND.
%
% Using \And between authors leaves it to LaTeX to determine where to
% break the lines. Using \AND forces a line break at that point. So,
% if LaTeX puts 3 of 4 authors names on the first line, and the last
% on the second line, try using \AND instead of \And before the third
% author name.
\newcommand*\samethanks[1][\value{footnote}]{\footnotemark[#1]}
\author{Constantinos Daskalakis\thanks{These authors contribute equally to this work.}\\
MIT, EECS\\
\texttt{costis@mit.edu}\\
\And  
Andrew Ilyas\samethanks\\
MIT, EECS\\
\texttt{ailyas@mit.edu}\\
\And
Vasilis Syrgkanis\samethanks\\
Microsoft Research\\
\texttt{vasy@microsoft.com}\\
\AND	
Haoyang Zeng\samethanks\\
MIT, EECS\\
\texttt{haoyangz@mit.edu}}

\footnotetext[1]{Code for our models is available at \url{https://github.com/vsyrgkanis/optimistic_GAN_training}}

\date{\today}
\maketitle

\begin{abstract}
We address the issue of limit cycling behavior in training Generative Adversarial Networks and propose the use of Optimistic Mirror Decent (OMD) for training Wasserstein GANs. Recent theoretical results have shown that optimistic mirror decent (OMD) can enjoy faster regret rates in the context of zero-sum games. WGANs is exactly a context of solving a zero-sum game with simultaneous no-regret dynamics.  Moreover, we show that optimistic mirror decent addresses the limit cycling problem in training WGANs. We formally show that in the case of bi-linear zero-sum games the last iterate of OMD dynamics converges to an equilibrium, in contrast to GD dynamics which are bound to cycle. We also portray the huge qualitative difference between GD and OMD dynamics with toy examples, even when GD is modified with many adaptations proposed in the recent literature, such as gradient penalty or momentum. We apply OMD WGAN training to a bioinformatics problem of generating DNA sequences. We observe that models trained with OMD achieve consistently smaller KL divergence with respect to the true underlying distribution, than models trained with GD variants. Finally, we introduce a new algorithm, Optimistic Adam, which is an optimistic variant of Adam. We apply it to WGAN training on CIFAR10 and observe improved performance in terms of inception score as compared to Adam.
\end{abstract}

\section{Introduction}
% \leavevmode
% \\
% \\
% \\
% \\
% \\
\section{Introduction}
\label{introduction}

AutoML is the process by which machine learning models are built automatically for a new dataset. Given a dataset, AutoML systems perform a search over valid data transformations and learners, along with hyper-parameter optimization for each learner~\cite{VolcanoML}. Choosing the transformations and learners over which to search is our focus.
A significant number of systems mine from prior runs of pipelines over a set of datasets to choose transformers and learners that are effective with different types of datasets (e.g. \cite{NEURIPS2018_b59a51a3}, \cite{10.14778/3415478.3415542}, \cite{autosklearn}). Thus, they build a database by actually running different pipelines with a diverse set of datasets to estimate the accuracy of potential pipelines. Hence, they can be used to effectively reduce the search space. A new dataset, based on a set of features (meta-features) is then matched to this database to find the most plausible candidates for both learner selection and hyper-parameter tuning. This process of choosing starting points in the search space is called meta-learning for the cold start problem.  

Other meta-learning approaches include mining existing data science code and their associated datasets to learn from human expertise. The AL~\cite{al} system mined existing Kaggle notebooks using dynamic analysis, i.e., actually running the scripts, and showed that such a system has promise.  However, this meta-learning approach does not scale because it is onerous to execute a large number of pipeline scripts on datasets, preprocessing datasets is never trivial, and older scripts cease to run at all as software evolves. It is not surprising that AL therefore performed dynamic analysis on just nine datasets.

Our system, {\sysname}, provides a scalable meta-learning approach to leverage human expertise, using static analysis to mine pipelines from large repositories of scripts. Static analysis has the advantage of scaling to thousands or millions of scripts \cite{graph4code} easily, but lacks the performance data gathered by dynamic analysis. The {\sysname} meta-learning approach guides the learning process by a scalable dataset similarity search, based on dataset embeddings, to find the most similar datasets and the semantics of ML pipelines applied on them.  Many existing systems, such as Auto-Sklearn \cite{autosklearn} and AL \cite{al}, compute a set of meta-features for each dataset. We developed a deep neural network model to generate embeddings at the granularity of a dataset, e.g., a table or CSV file, to capture similarity at the level of an entire dataset rather than relying on a set of meta-features.
 
Because we use static analysis to capture the semantics of the meta-learning process, we have no mechanism to choose the \textbf{best} pipeline from many seen pipelines, unlike the dynamic execution case where one can rely on runtime to choose the best performing pipeline.  Observing that pipelines are basically workflow graphs, we use graph generator neural models to succinctly capture the statically-observed pipelines for a single dataset. In {\sysname}, we formulate learner selection as a graph generation problem to predict optimized pipelines based on pipelines seen in actual notebooks.

%. This formulation enables {\sysname} for effective pruning of the AutoML search space to predict optimized pipelines based on pipelines seen in actual notebooks.}
%We note that increasingly, state-of-the-art performance in AutoML systems is being generated by more complex pipelines such as Directed Acyclic Graphs (DAGs) \cite{piper} rather than the linear pipelines used in earlier systems.  
 
{\sysname} does learner and transformation selection, and hence is a component of an AutoML systems. To evaluate this component, we integrated it into two existing AutoML systems, FLAML \cite{flaml} and Auto-Sklearn \cite{autosklearn}.  
% We evaluate each system with and without {\sysname}.  
We chose FLAML because it does not yet have any meta-learning component for the cold start problem and instead allows user selection of learners and transformers. The authors of FLAML explicitly pointed to the fact that FLAML might benefit from a meta-learning component and pointed to it as a possibility for future work. For FLAML, if mining historical pipelines provides an advantage, we should improve its performance. We also picked Auto-Sklearn as it does have a learner selection component based on meta-features, as described earlier~\cite{autosklearn2}. For Auto-Sklearn, we should at least match performance if our static mining of pipelines can match their extensive database. For context, we also compared {\sysname} with the recent VolcanoML~\cite{VolcanoML}, which provides an efficient decomposition and execution strategy for the AutoML search space. In contrast, {\sysname} prunes the search space using our meta-learning model to perform hyperparameter optimization only for the most promising candidates. 

The contributions of this paper are the following:
\begin{itemize}
    \item Section ~\ref{sec:mining} defines a scalable meta-learning approach based on representation learning of mined ML pipeline semantics and datasets for over 100 datasets and ~11K Python scripts.  
    \newline
    \item Sections~\ref{sec:kgpipGen} formulates AutoML pipeline generation as a graph generation problem. {\sysname} predicts efficiently an optimized ML pipeline for an unseen dataset based on our meta-learning model.  To the best of our knowledge, {\sysname} is the first approach to formulate  AutoML pipeline generation in such a way.
    \newline
    \item Section~\ref{sec:eval} presents a comprehensive evaluation using a large collection of 121 datasets from major AutoML benchmarks and Kaggle. Our experimental results show that {\sysname} outperforms all existing AutoML systems and achieves state-of-the-art results on the majority of these datasets. {\sysname} significantly improves the performance of both FLAML and Auto-Sklearn in classification and regression tasks. We also outperformed AL in 75 out of 77 datasets and VolcanoML in 75  out of 121 datasets, including 44 datasets used only by VolcanoML~\cite{VolcanoML}.  On average, {\sysname} achieves scores that are statistically better than the means of all other systems. 
\end{itemize}


%This approach does not need to apply cleaning or transformation methods to handle different variances among datasets. Moreover, we do not need to deal with complex analysis, such as dynamic code analysis. Thus, our approach proved to be scalable, as discussed in Sections~\ref{sec:mining}.

\section{Preliminaries: WGANs and Optimistic Mirror Descent}
%!TEX root = hopfwright.tex
%

In this section we systematically recast the Hopf bifurcation problem in Fourier space. 
We introduce appropriate scalings, sequence spaces of Fourier coefficients and convenient operators on these spaces. 
To study Equation~\eqref{eq:FourierSequenceEquation} we consider Fourier sequences $ \{a_k\}$ and fix a Banach space in which these sequences reside. It is indispensable for our analysis that this space have an algebraic structure. 
The Wiener algebra of absolutely summable Fourier series is a natural candidate, which we use with minor modifications. 
In numerical applications, weighted sequence spaces with algebraic and geometric decay have been used to great effect to study periodic solutions which are $C^k$ and analytic, respectively~\cite{lessard2010recent,hungria2016rigorous}. 
Although it follows from Lemma~\ref{l:analytic} that the Fourier coefficients of any solution decay exponentially, we choose to work in a space of less regularity. 
The reason is that by working in a space with less regularity, we are better able to connect our results with the global estimates in \cite{neumaier2014global}, see Theorem~\ref{thm:UniqunessNbd2}.


%
%
%\begin{remark}
%	Although it follows from Lemma~\ref{l:analytic} that the Fourier coefficients of any solution decay exponentially, we choose to work in a space of less regularity, namely summable Fourier coefficients. This allows us to draw SOME MORE INTERESTING CONCLUSION LATER.
%	EXPLAIN WHY WE CHOOSE A NORM WITH ALMOST NO DECAY!
%	% of s Periodic solutions to Wright's equation are known to be real analytic and so their  Fourier coefficients must decay geometrically [Nussbaum].
%	% We do not use such a strong result;  any periodic solution must be continuously differentiable, by which it follows that $ \sum | c_k| < \infty$.
%\end{remark}


\begin{remark}\label{r:a0}
There is considerable redundancy in Equation~\eqref{eq:FourierSequenceEquation}. First, since we are considering real-valued solutions $y$, we assume $\c_{-k}$ is the complex conjugate of $\c_k$. This symmetry implies it suffices to consider Equation~\eqref{eq:FourierSequenceEquation} for $k \geq 0$.
Second, we may effectively ignore the zeroth Fourier coefficient of any periodic solution \cite{jones1962existence}, since it is necessarily equal to $0$. 
%In \cite{jones1962existence}, it is shown that if $y \not\equiv -1$ is a periodic solution of~\eqref{eq:Wright} with frequency $\omega$, then $ \int_0^{2\pi/\omega} y(t) dt =0$. 
		The self contained argument is as follows. 
		As mentioned in the introduction, any periodic solution to Wright's equation must satisfy $ y(t) > -1$ for all $t$. 
	By dividing Equation~\eqref{eq:Wright} by $(1+y(t))$, which never vanishes, we obtain
	\[
	\frac{d}{dt} \log (1 + y(t)) = - \alpha y(t-1).
	\]  
	Integrating over one period $L$ we derive the condition 
	$0=\int_0^L y(t) dt $.
	Hence $a_0=0$ for any periodic solution. 
	It will be shown in Theorem~\ref{thm:FourierEquivalence1} that a related argument implies that we do not need to consider Equation~\eqref{eq:FourierSequenceEquation} for $k=0$.
\end{remark}

%%%
%%%
%%%\begin{remark}\label{r:c0} 
%%%In \cite{jones1962existence}, it is shown that if $y \not\equiv -1$ is a periodic solution of~\eqref{eq:Wright} with frequency $\omega$, then $ \int_0^{2\pi/\omega} y(t) dt =0$. 
%%%PERHAPS TOO MUCH DETAIL HERE. The self contained argument is as follows.
%%%If $y \not\equiv -1$ then $y(t) \neq -1$ for all $t$, since if $y(t_0)=-1$ for some $t_0 \in \R$ then $y'(t_0)=0$ by~\eqref{eq:Wright} and in fact by differentiating~\eqref{eq:Wright} repeatedly one obtains that all derivatives of $y$ vanish at $t_0$. Hence $y \equiv -1$ by Lemma~\ref{l:analytic}, a contradiction. Now divide~\eqref{eq:Wright} by $(1+y(t))$, which never vanishes, to obtain
%%%\[
%%%  \frac{d}{dt} \log |1 + y(t)| = - \alpha y(t-1).
%%%\]  
%%%Integrating over one period we obtain $\int_0^L y(t) dt =0$.
%%%\end{remark}



%Furthermore, the condition that $y(t)$ is real forces $\c_{-k} = \overline{\c}_{k}$.  
%
We define the spaces of absolutely summable Fourier series
\begin{alignat*}{1}
	\ell^1 &:= \left\{ \{ \c_k \}_{k \geq 1} : 
    \sum_{k \geq 1} | \c_k| < \infty  \right\} , \\
	\ell^1_\bi &:= \left\{ \{ \c_k \}_{k \in \Z} : 
    \sum_{k \in \Z} | \c_k| < \infty  \right\} .
\end{alignat*} 
We identify any semi-infinite sequence $ \{ \c_k \}_{k \geq 1} \in \ell^1$ with the bi-infinite sequence $ \{ \c_k \}_{k \in \Z} \in \ell^1_\bi$ via the conventions (see Remark~\ref{r:a0})
\begin{equation}
  \c_0=0 \qquad\text{ and }\qquad \c_{-k} = \c_{k}^*. 
\end{equation}
In other word, we identify $\ell^1$ with the set
\begin{equation*}
   \ell^1_\sym := \left\{ \c \in \ell^1_\bi : 
	\c_0=0,~\c_{-k}=\c_k^* \right\} .
\end{equation*}
On $\ell^1$ we introduce the norm
\begin{equation}\label{e:lnorm}
  \| \c \| = \| \c \|_{\ell^1} := 2 \sum_{k = 1}^\infty |\c_k|.
\end{equation}
The factor $2$ in this norm is chosen to have a Banach algebra estimate.
Indeed, for $\c, \tilde{\c} \in \ell^1 \cong \ell^1_\sym$ we define
the discrete convolution 
\[
\left[ \c * \tilde{\c} \right]_k = \sum_{\substack{k_1,k_2\in\Z\\ k_1 + k_2 = k}} \c_{k_1} \tilde{\c}_{k_2} .
\]
Although $[\c*\tilde{\c}]_0$ does not necessarily vanish, we have $\{\c*\tilde{\c}\}_{k \geq 1} \in \ell^1 $ and 
\begin{equation*}
	\| \c*\tilde{\c} \| \leq \| \c \| \cdot  \| \tilde{\c} \| 
	\qquad\text{for all } \c , \tilde{\c} \in \ell^1, 
\end{equation*}
hence $\ell^1$ with norm~\eqref{e:lnorm} is a Banach algebra.

By Lemma~\ref{l:analytic} it is clear that any periodic solution of~\eqref{eq:Wright} has a well-defined Fourier series $\c \in \ell^1_\bi$. 
The next theorem shows that in order to study periodic orbits to Wright's equation we only need to study Equation~\eqref{eq:FourierSequenceEquation} 
for $k \geq 1$. For convenience we introduce the notation 
\[
G(\alpha,\omega,\c)_k=
( i \omega k + \alpha e^{ - i \omega k}) \c_k + \alpha \sum_{k_1 + k_2 = k} e^{- i \omega k_1} \c_{k_1} \c_{k_2} \qquad \text{for } k \in \N.
\]
We note that we may interpret the trivial solution $y(t)\equiv 0$ as a periodic solution of arbitrary period.
\begin{theorem}
\label{thm:FourierEquivalence1}
Let $\alpha>0$ and $\omega>0$.
If $\c \in \ell^1 \cong \ell^1_{\sym}$ solves
$G(\alpha,\omega,\c)_k =0$  for all $k \geq 1$,
then $y(t)$ given by~\eqref{eq:FourierEquation} is a periodic solution of~\eqref{eq:Wright} with period~$2\pi/\omega$.
Vice versa, if $y(t)$ is a periodic solution of~\eqref{eq:Wright} with period~$2\pi/\omega$ then its Fourier coefficients $\c \in \ell^1_\bi$ lie in $\ell^1_\sym \cong \ell^1$ and solve $G(\alpha,\omega,\c)_k =0$ for all $k \geq 1$.
\end{theorem}

\begin{proof}	
	If $y(t)$ is a periodic solution of~\eqref{eq:Wright} then it is real analytic by Lemma~\ref{l:analytic}, hence its Fourier series $\c$ is well-defined and $\c \in \ell^1_{\sym}$ by Remark~\ref{r:a0}.
	Plugging the Fourier series~\eqref{eq:FourierEquation} into~\eqref{eq:Wright} one easily derives that $\c$ solves~\eqref{eq:FourierSequenceEquation} for all $k \geq 1$.

To prove the reverse implication, assume that $\c \in \ell^1_\sym$ solves
Equation~\eqref{eq:FourierSequenceEquation} for all $k \geq 1$. Since $\c_{-k}
= \c_k^*$, Equation \eqref{eq:FourierSequenceEquation} is also satisfied for
all $k \leq -1$. It follows from the Banach algebra property and
\eqref{eq:FourierSequenceEquation} that $\{k \c_k\}_{k \in \Z} \in \ell^1_\bi$,
hence $y$, given by~\eqref{eq:FourierEquation}, is continuously differentiable.
% (and by bootstrapping one infers that $\{k^m c_k \} \in \ell^1_\bi$, 
% hence $y \in C^m$ for any $m \geq 1$).
	Since~\eqref{eq:FourierSequenceEquation} is satisfied for all $k \in \Z \setminus \{0\}$ (but not necessarily for $k=0$) one may perform the inverse Fourier transform on~\eqref{eq:FourierSequenceEquation} to conclude that
	$y$ satisfies the delay equation 
\begin{equation}\label{eq:delaywithK}
   	y'(t) = - \alpha y(t-1) [ 1 + y(t)] + C
\end{equation}
	for some constant $C \in \R$. 
   Finally, to prove that $C=0$ we argue by contradiction.
   Suppose $C \neq 0$. Then $y(t) \neq -1$ for all $t$.
   Namely, at any point where $y(t_0) =-1$ one would have $y'(t_0) = C$
   which has fixed sign,   hence it would follow that $y$ is not periodic
   ($y$ would not be able to cross $-1$ in the opposite direction, 
   preventing $y$  from being periodic).  
  We may thus divide~\eqref{eq:delaywithK} through by $1 + y(t)$ and obtain 
\begin{equation*}
	\frac{d}{dt} \log | 1 + y(t) | = - \alpha y(t-1) + \frac{C}{1+y(t)} .
\end{equation*}
	By integrating both sides of the equation over one period $L$ and by using that $\c_0=0$, we 
	obtain
	\[
	 C \int_0^L \frac{1}{1+y(t)} dt =0.
	\]
	Since the integrand is either strictly negative or strictly positive, this implies that $C=0$. Hence~\eqref{eq:delaywithK} reduces to~\eqref{eq:Wright},
	and $y$ satisfies Wright's equation. 
\end{proof}






To efficiently study Equation~\eqref{eq:FourierSequenceEquation}, we introduce the following linear operators on $ \ell^1$:
\begin{alignat*}{1}
   [K \c ]_k &:= k^{-1} \c_k  , \\ 
   [ U_\omega \c ]_k &:= e^{-i k \omega} \c_k  .
\end{alignat*}
The map $K$ is a compact operator, and it has a densely defined inverse $K^{-1}$. The domain of $K^{-1}$ is denoted by
\[
  \ell^K := \{ \c \in \ell^1 : K^{-1} \c \in \ell^1 \}.  
\]
The map $U_{\omega}$ is a unitary operator on $\ell^1$, but
it is discontinuous in $\omega$. 
With this notation, Theorem~\ref{thm:FourierEquivalence1} implies that our problem of finding a SOPS to~\eqref{eq:Wright} is equivalent to finding an $\c \in \ell^1$ such that
\begin{equation}
\label{e:defG}
  G(\alpha,\omega,\c) :=
  \left( i \omega K^{-1} + \alpha U_\omega \right) \c + \alpha \left[U_\omega \, \c \right] * \c  = 0.
\end{equation}


%In order for the solutions of Equation \ref{eq:FHat} to be isolated we need to impose a phase condition. 
%If there is a sequence $ \{ c_k \} $ which satisfies  Equation \ref{eq:FHat}, then $ y( t + \tau) = \sum_{k \in \Z} c_k e^{ i k \omega (t + \tau)}$ satisfies Wright's equation at parameter $\alpha$. 
%Fix $ \tau = - Arg[c_1] / \omega$ so that $ c_1  e^{ i \omega \tau} $ is a nonnegative real number. 
%By Proposition \ref{thm:FourierEquivalence1} it follows that $\{ c'_k \} =  \{c_k e^{ i \omega k \tau }   \}$ is a solution to Equation \ref{eq:FHat}, and furthermore that $ c'_1 = \epsilon$ for some $ \epsilon \geq 0$. 


Periodic solutions are invariant under time translation: if $y(t)$ solves Wright's equation, then so does $ y(t+\tau)$ for any $\tau \in \R$. 
We remove this degeneracy by adding a phase condition. 
Without loss of generality, if $\c \in \ell^1$ solves Equation~\eqref{e:defG}, we may assume that $\c_1 = \epsilon$ for some 
\emph{real non-negative}~$\epsilon$:
\[
  \ell^1_{\epsilon} := \{\c \in \ell^1 : \c_1 = \epsilon \} 
  \qquad \text{where } \epsilon \in \R,  \epsilon \geq 0.
\]
In the rest of our analysis, we will split elements $\c \in \ell^1$ into two parts: $\c_1$ and $\{\c_{k}\}_{k \geq 2}$.  
We define the basis elements $\e_j \in \ell^1$ for $j=1,2,\dots$ as
\[
  [\e_j]_k = \begin{cases}
  1 & \text{if } k=j, \\
  0 & \text{if } k \neq j.
  \end{cases}
\]
We note that $\| \e_j \|=2$. 
Then we can decompose
% We define
% \[
%   \tilde{\epsilon} := (\epsilon,0,0,0,\dots) \in \ell^1
% \]
% and
% For clarity when referring to sequences $\{c_{k}\}_{k \geq 2}$, we make the following definition:
% \[
% \ell^1_0  := \{ \tc \in \ell^1 : \tc_1 = 0 \}.
% \]
% With the
any $\c \in \ell^1_\epsilon$ uniquely as
\begin{equation}\label{e:aepsc}
  \c= \epsilon \e_1 + \tc \qquad \text{with}\quad 
  \tc \in \ell^1_0 := \{ \tc \in \ell^1 : \tc_1 = 0 \}.
\end{equation}
We follow the classical approach in studying Hopf bifurcations and consider 
$\c_1 = \epsilon$ to be a parameter, and then find periodic solutions with Fourier modes in $\ell^1_{\epsilon}$.
This approach rewrites the function $G: \R^2 \times \ell^K \to \ell^1$ as a function $\tilde{F}_\epsilon : \R^2 \times \ell^K_0 \to \ell^1$, where 
we denote 
\[
\ell^K_0 := \ell^1_0 \cap \ell^K.
\]
% I AM ACTUALLY NOT SURE IF YOU WANT TO DEFINE THIS WITH RANGE IN $\ell^1$
% OR WITH DOMAIN IN $\ell^1_0$ ?? IT SEEMS TO DEPEND ON WHICH GLOBAL STATEMENT YOU WANT/NEED TO MAKE!?
\begin{definition}
We define the $\epsilon$-parameterized family of  functions $\tilde{F}_\epsilon: \R^2 \times \ell^K_0  \to \ell^1$ 
by 
\begin{equation}
\label{eq:fourieroperators}
\tilde{F}_{\epsilon}(\alpha,\omega, \tc) := 
\epsilon [i \omega + \alpha e^{-i \omega}] \e_1 + 
( i \omega K^{-1} + \alpha U_{\omega}) \tc + 
\epsilon^2 \alpha e^{-i \omega}  \e_2  +
\alpha \epsilon L_\omega \tc + 
\alpha  [ U_{\omega} \tc] * \tc ,
\end{equation}
where
$L_\omega : \ell^1_0 \to \ell^1$ is given by
\[
   L_{\omega} := \sigma^+( e^{- i \omega} I + U_{\omega}) + \sigma^-(e^{i \omega} I + U_{\omega}),
\]
with $I$ the identity and  $\sigma^\pm$ the shift operators on $\ell^1$:
\begin{alignat*}{2}
\left[ \sigma^- a \right]_k &:=  a_{k+1}  , \\
\left[ \sigma^+ a \right]_k &:=  a_{k-1}  &\qquad&\text{with the convention } \c_0=0.
\end{alignat*}
The operator $ L_\omega$ is discontinuous in $\omega$ and $ \| L_\omega \| \leq 4$. 
\end{definition} 

%The maps $ \sigma^{+}$ and $ \sigma^-$ are shift up and shift down operators respectively. 
We reformulate Theorem~\ref{thm:FourierEquivalence1}  in terms of the map  $\tilde{F}$. 
We note that it follows from Lemma~\ref{l:analytic} and 
%\marginpar{Reformulate}
%one's choice of  
Equation~\eqref{eq:FourierSequenceEquation}  
%or Equation ~\eqref{eq:fourieroperators},
that the Fourier coefficients of any periodic solution of~\eqref{eq:Wright} lie in $\ell^K$.
These observations are summarized in the following theorem.
\begin{theorem}
\label{thm:FourierEquivalence2}
	Let $ \epsilon \geq 0$,  $\tc \in \ell^K_0$, $\alpha>0$ and $ \omega >0$. 
	Define $y: \R\to \R$ as 
\begin{equation}\label{e:ytc}
	y(t) = 
	\epsilon \left( e^{i \omega t }  + e^{- i \omega t }\right) 
	+  \sum_{k = 2}^\infty   \tc_k e^{i \omega k t }  + \tc_k^* e^{- i \omega k t } .
\end{equation}
%	and suppose that $ y(t) > -1$. 
	Then $y(t)$ solves~\eqref{eq:Wright} if and only if $\tilde{F}_{\epsilon}( \alpha , \omega , \tc) = 0$. 
	Furthermore, up to time translation, any periodic solution of~\eqref{eq:Wright} with period $2\pi/\omega$ is described by a Fourier series of the form~\eqref{e:ytc} with $\epsilon \geq 0$ and $\tc \in \ell^K_0$.
\end{theorem}


%We note that for $\epsilon>0$ such solutions are truly periodic, while for $\epsilon=0$ a zero of $\tilde{F}_\epsilon$ may either correspond to a periodic solution or to the trivial solution $y(t) \equiv 0$. 



% \begin{proof}
%  By Proposition \ref{thm:FourierEquivalence1}, it suffices to show that $\tilde{F}(\alpha,\omega,c) =0$ is equivalent to Equation \ref{eq:FourierSequenceEquation} being satisfied for $k \geq 1$.
%  Since Equation \ref{eq:FourierSequenceEquation} is equivalent to Equation \ref{eq:FHat}, we expand  Equation \ref{eq:FHat} by writing $ \hat{c} = \hat{\epsilon } + c$  where $ \hat{\epsilon} := (\epsilon,0,0,\dots) \in \ell^1$ as below:
%  \begin{equation}
%  0=  \left( i \omega K^{-1} + \alpha U_\omega \right) (\hat{\epsilon}+ c) + \alpha \left[U_\omega \, (\hat{\epsilon}+ c) \right] * (\hat{\epsilon}+ c) \label{eq:Intial}
%  \end{equation}
%  The RHS of Equation \ref{eq:Intial} is $ \tilde{F}(\alpha,\omega,c)$, so the theorem is proved.
% \end{proof}



Since we want to analyze a Hopf bifurcation, we will want to solve $\tilde{F}_\epsilon = 0$ for small values of~$\epsilon$. 
However, at the bifurcation point, $ D \tilde{F}_0(\pp  ,\pp , 0)$ is not invertible.
In order for our asymptotic analysis to be non-degenerate,
we work with a rescaled version of the problem. To this end, for any $\epsilon >0$, we rescale both $\tc$ and $\tilde{F}$ as follows. Let $\tc = \epsilon c$ and 
\begin{equation}\label{e:changeofvariables}
  \tilde{F}_\epsilon (\alpha,\omega,\epsilon c) = \epsilon F_\epsilon (\alpha,\omega,c).
\end{equation}
For $\epsilon>0$ the problem then reduces to finding zeros of 
\begin{equation}
\label{eq:FDefinition}
	F_\epsilon(\alpha,\omega, c) := 
	[i \omega + \alpha e^{-i \omega}] \e_1 + 
	( i \omega K^{-1} + \alpha U_{\omega}) c + 
	\epsilon \alpha e^{-i \omega} \e_2  +
	\alpha \epsilon L_\omega c + 
	\alpha \epsilon [ U_{\omega} c] * c.
\end{equation}
We denote the triple $(\alpha,\omega,c) \in \R^2 \times \ell^1_0$ by $x$.
To pinpoint the components of $x$ we use the projection operators
\[
   \pi_\alpha x = \alpha, \quad \pi_\omega x = \omega, \quad 
  \pi_c x = c \qquad\text{for any } x=(\alpha,\omega,c).
\]

After the change of variables~\eqref{e:changeofvariables} we now have an invertible Jacobian $D F_0(\pp  ,\pp , 0)$ at the bifurcation point.
On the other hand, for $\epsilon=0$ the zero finding problems for $\tilde{F}_\epsilon$ and $F_\epsilon$ are not equivalent. 
However, it follows from the following lemma that any nontrivial periodic solution having $ \epsilon=0$ must have a relatively large size when $ \alpha $ and $ \omega $ are close to the bifurcation point. 

\begin{lemma}\label{lem:Cone}
	Fix $ \epsilon \geq 0$ and $\alpha,\omega >0$. 
	Let
	\[
	b_* :=  \frac{\omega}{\alpha} - \frac{1}{2} - \epsilon  \left(\frac{2}{3}+ \frac{1}{2}\sqrt{2 + 2 |\omega-\pp| } \right).
	\]
Assume that $b_*> \sqrt{2} \epsilon$. 
Define
% \begin{equation*}%\label{e:zstar}
% 	z^{\pm}_* :=b_* \pm \sqrt{(b_*)^2- \epsilon^2 } .
% \end{equation*}
% \note[J]{Proposed change to match Lemma E.4}
\begin{equation}\label{e:zstar}
z^{\pm}_* :=b_* \pm \sqrt{(b_*)^2- 2 \epsilon^2 } .
\end{equation}
If there exists a $\tc \in \ell^1_0$ such that $\tilde{F}_\epsilon(\alpha, \omega,\tc) = 0$, then \\
\mbox{}\quad\textup{(a)} either $ \|\tc\| \leq  z_*^-$ or $ \|\tc\| \geq z_*^+  $.\\
\mbox{}\quad\textup{(b)} 
$ \| K^{-1} \tc \| \leq (2\epsilon^2+ \|\tc\|^2) / b_*$. 
\end{lemma}
\begin{proof}
	The proof follows from Lemmas~\ref{lem:gamma} and~\ref{lem:thecone} in Appendix~\ref{appendix:aprioribounds}, combined with the observation that
$\frac{\omega}{\alpha} - \gamma \geq b_*$,
% \[
%   \frac{\omega}{\alpha} - \gamma \geq b_*
%  \qquad\text{for all }
% | \alpha - \pp| \leq r_\alpha \text{ and } 
%   | \omega - \pp| \leq r_\omega.
% \]
with $\gamma$ as defined in Lemma~\ref{lem:gamma}.
\end{proof}

\begin{remark}\label{r:smalleps}
We note that for $\alpha < 2\omega$
\begin{alignat*}{1}
z^+_* &\geq   \frac{2 \omega - \alpha}{\alpha} 
- \epsilon \left(4/3+\sqrt{2 + 2 |\omega-\pp| } \, \right) + \cO(\epsilon^2)
\\[1mm]
z^-_* & \leq   \cO(\epsilon^2)
\end{alignat*}
for small $\epsilon$. 
Hence Lemma~\ref{lem:Cone} implies that for values of $(\alpha,\omega)$ near $(\pp,\pp)$ any solution has either $\|\tc\|$ of order 1 or $\|\tc\| =  \cO(\epsilon^2)$. 
The asymptotically small term bounding $z_*^-$ is explicitly calculated in Lemma~\ref{lem:ZminusBound}. 
A related consequence is that for $\epsilon=0$ there are no nontrivial solutions 
of $\tilde{F}_0(\alpha,\omega,\tc)=0$ with 
$\| \tc \| < \frac{2 \omega - \alpha}{\alpha} $. 
\end{remark}

\begin{remark}\label{r:rhobound}
In Section~\ref{s:contraction} we will work on subsets of $\ell^K_0$ of the form
\[
  \ell_\rho := \{ c \in \ell^K_0 : \|K^{-1} c\| \leq \rho \} .
\]
Part (b) of Lemma~\ref{lem:Cone} will be used in Section~\ref{s:global} to guarantee that we are not missing any solutions by considering $\ell_\rho$ (for some specific choice of $\rho$) rather than the full space $\ell^K_0$.
In particular, we infer from Remark~\ref{r:smalleps} that  small solutions (meaning roughly that $\|\tc\| \to 0$ as $\epsilon \to 0$)
satisfy $\| K^{-1} \tc \| = \cO(\epsilon^2)$.
\end{remark}

The following theorem guarantees that near the bifurcation point the problem of finding all periodic solutions is equivalent to considering the rescaled problem $F_\epsilon(\alpha,\omega,c)=0$.
\begin{theorem}
\label{thm:FourierEquivalence3}
\textup{(a)} Let $ \epsilon > 0$,  $c \in \ell^K_0$, $\alpha>0$ and $ \omega >0$. 
	Define $y: \R\to \R$ as 
\begin{equation}\label{e:yc}
	y(t) = 
	\epsilon \left( e^{i \omega t }  + e^{- i \omega t }\right) 
	+ \epsilon  \sum_{k = 2}^\infty   c_k e^{i \omega k t }  + c_k^* e^{- i \omega k t } .
\end{equation}
%	and suppose that $ y(t) > -1$. 
	Then $y(t)$ solves~\eqref{eq:Wright} if and only if $F_{\epsilon}( \alpha , \omega , c) = 0$.\\
\textup{(b)}
Let $y(t) \not\equiv 0$ be a periodic solution of~\eqref{eq:Wright} of period $2\pi/\omega$
 with Fourier coefficients $\c$.
Suppose $\alpha < 2\omega$ and $\| \c \| < \frac{2 \omega - \alpha}{\alpha} $.
Then, up to time translation, $y(t)$ is described by a Fourier series of the form~\eqref{e:yc} with $\epsilon > 0$ and $c \in \ell^K_0$.
\end{theorem}

\begin{proof}
Part (a) follows directly from Theorem~\ref{thm:FourierEquivalence2} and the  change of variables~\eqref{e:changeofvariables}.
To prove part (b) we need to exclude the possibility that there is a nontrivial solution with $\epsilon=0$. The asserted bound on the ratio of $\alpha$ and $\omega$ guarantees, by Lemma~\ref{lem:Cone} (see also Remark~\ref{r:smalleps}), that indeed $\epsilon>0$ for any nontrivial solution. 
\end{proof}

We note that in practice (see Section~\ref{s:global}) a bound on $\| \c \|$ is derived from a bound on $y$ or $y'$ using Parseval's identity.

\begin{remark}\label{r:cone}
It follows from Theorem~\ref{thm:FourierEquivalence3} and Remark~\ref{r:smalleps} that for values of $(\alpha,\omega)$ near $(\pp,\pp)$ any reasonably bounded solution satisfies $\| c\| =  O(\epsilon)$ as well as $\|K^{-1} c \| = O(\epsilon)$ asymptotically (as $\epsilon \to 0$).
These bounds will be made explicit (and non-asymptotic) for specific choices of the parameters in Section~\ref{s:global}.
\end{remark}

% We are able to rule out such large amplitude solutions using global estimates such as those in \cite{neumaier2014global}.
% Hence, near the bifurcation point, the problem of describing periodic solutions of~\eqref{eq:Wright} reduces to studying the family of zeros finding problems $F_\epsilon=0$.





%Specifically, if a solution having $ \epsilon = 0$ does in fact correspond to a nontrivial periodic solution and $\alpha  < 2\omega $, then $ \| \tilde{c} \| > 2 \omega \alpha^{-1} -1$. 
%%PERHAPS THIS NEEDS A FORMULATION AS A THEOREM AS WELL?
%%IN OTHER WORDS: ARE WE SURE WE HAVE FOUND ALL ZEROS OF $\tilde{F}_0$, I.E. ALL SOLUTIONS WITH $\epsilon=0$ NEAR THE BIFURCATION POINT? AFTER RESCALING THESE ARE INVISIBLE?
%%THERE SHOULD BE A STATEMENT ABOUT THIS SOMEWHERE! EITHER HERE OR SOME





We finish this section by defining a curve of approximate zeros $\bx_\epsilon$ of $F_\epsilon$ 
(see \cite{chow1977integral,hassard1981theory}). 
%(see \cite{chow1977integral,morris1976perturbative,hassard1981theory}). 


\begin{definition}\label{def:xepsilon}
Let
\begin{alignat*}{1}
	\balpha_\epsilon &:= \pp + \tfrac{\epsilon^2}{5} ( \tfrac{3\pi}{2} -1)  \\
	\bomega_\epsilon &:= \pp -  \tfrac{\epsilon^2}{5} \\
	\bc_\epsilon 	 &:= \left(\tfrac{2 - i}{5}\right) \epsilon \,  \e_2 \,.
\end{alignat*}
We define the approximate solution 
$ \bx_\epsilon := \left( \balpha_\epsilon , \bomega_\epsilon  , \bc_\epsilon \right)$
for all $\epsilon \geq 0$.
\end{definition}

We leave it to the reader to verify that both 
 $F_\epsilon(\pp,\pp,\bc_{\epsilon})=\cO(\epsilon^2)$ and $F_\epsilon(\bx_\epsilon)=\cO(\epsilon^2)$.
%%%	
%%%	
%%%	}{Better like this?}
%%%\annote[J]{ $F_\epsilon(\bx_0)=\cO(\epsilon^2)$ and $F_\epsilon(\bx_\epsilon)=\cO(\epsilon^2)$.}{I think we'd still need the $ \bar{c}_\epsilon$ term in $\bar{x}_0$ to be of order $ \epsilon$.}
%%%\remove[JB]{We show in Proposition A.1
%%%%\ref{prop:ApproximateSolutionWorks} 
%%% that any $ x \in \R^2 \times \ell^1_0$ which is $ \cO(\epsilon^2)$ close to $ \bar{x}_\epsilon $ will yield the estimate $F_\epsilon(x) = \cO(\epsilon^2)$.
%%%Hence choosing $\{ \pp , \pp, \bar{c}_\epsilon\}$ as our approximate solution would also have been a natural choice for performing an $\cO(\epsilon^2)$ analysis and would have simplified several of our calculations.
%%%However,} 
%%%
We choose to use the more accurate approximation 
for the $ \alpha$ and $ \omega $ components to improve our final quantitative results. 














%
% Values for $ (\alpha, \omega,c)$ which approximately solve $\tilde{F}(\alpha,\omega,c) = 0$  are computed in  \cite{chow1977integral,morris1976perturbative,hassard1981theory} and are as follows:
%  \begin{eqnarray}
%  \tilde{\alpha}( \epsilon) &:=& \pi /2 + \tfrac{\epsilon^2}{5} ( \tfrac{3\pi}{2} -1) \nonumber \\
%  \tilde{\omega}( \epsilon) &:=& \pi /2 -  \tfrac{\epsilon^2}{5} \label{eq:ScaleApprox} \\
%  \tc(\epsilon) 	  &:=& \{ \left(\tfrac{2 - i}{5}\right)  \epsilon^2 , 0,0, \dots \} \nonumber
%  \end{eqnarray}
% In Appendix \ref{sec:OperatorNorms} we illustrate an alternative method for deriving this approximation.
%
%
%
%
% We want to solve $ \tilde{F}(\alpha , \omega, \hat{c}) =0$ for small values of $ \epsilon$.
% However $ D \tilde{F}(\alpha , \omega , c)$ is not invertible at $ ( \pp , \pp , 0)$ when $ \epsilon = 0$.
% In order for our asymptotic analysis to be non-degenerate, we need to make the change of variables $ c \mapsto \epsilon c$.
% Under this change of variables, we define the function $ F$ below so that $ \tilde{F}(\alpha , \omega , \epsilon c) =\epsilon  F( \alpha , \omega , c)$.
%
%
%
% \begin{definition}
% Construct an $\epsilon$-parameterized family of densely defined functions  $F : \R^2 \oplus \ell^1 / \C \to \ell^1$ by:
% \begin{equation}
% \label{eq:FDefinition}
% 	F(\alpha,\omega, c) :=
% 	[i \omega + \alpha e^{-i \omega}]_1 +
% 	( i \omega K^{-1} + \alpha U_{\omega}) c +
% 	[\epsilon \alpha e^{-i \omega}]_2  +
% 	\alpha \epsilon L_\omega c +
% 	\alpha \epsilon [ U_{\omega} c] * c.
% \end{equation}
% \end{definition}

%%
%%
%%\begin{corollary}
%%	\label{thm:FourierEquivalence3}
%%	Fix $ \epsilon > 0$, and $ c \in \ell^1 / \C $, and $ \omega >0$. Define $y: \R\to \R$ as 
%%	\[
%%	y(t) = 
%%	\epsilon \left( e^{i \omega t }  + e^{- i \omega t }\right) 
%%	+  \epsilon  \left( \sum_{k = 2}^\infty   c_k e^{i \omega k t }  + \overline{c}_k e^{- i \omega k t } \right) 
%%	\]
%%	and suppose that $ y(t) > -1$. 
%%	Then $y(t)$ solves Wright's equation at parameter $ \alpha > 0 $ if and only if $ F( \alpha , \omega , c) = 0$ at parameter $ \epsilon$. 
%%	
%%	
%%	
%%\end{corollary}
%%
%%
%%\begin{proof}
%%	Since $ \tilde{F}(\alpha,\omega, \epsilon c) = \epsilon F( \alpha , \omega , c)$, the result follows from Theorem \ref{thm:FourierEquivalence2}.
%%\end{proof}

% If we can find $(\alpha , \omega, c)$ for which $ F( \alpha , \omega,c)=0$ at parameter $\epsilon$, then $ \tilde{F}(\alpha ,\omega, c)=0$.
% By Theorem \ref{thm:FourierEquivalence2} this amounts to finding a periodic solution to Wright's equation.
% Lastly, because we have performed the change of variables $ c \mapsto \epsilon c$, we need to  apply this change of variables to our approximate solution as well.
%
% \begin{definition}
% 	Define the approximate solution $ x( \epsilon) = \left\{ \alpha(\epsilon ) , \omega ( \epsilon ) , c(\epsilon) \right\}$ as below,  where $c(\epsilon) = \{ c_2( \epsilon) , 0 ,0 , \dots\} $.
% 	We may also write $ x_\epsilon = x(\epsilon) $.
% 	\begin{eqnarray}
% 	\alpha( \epsilon) &:=& \pi /2 + \tfrac{\epsilon^2}{5} ( \tfrac{3\pi}{2} -1) \nonumber \\
% 	\omega( \epsilon) &:=& \pi /2 -  \tfrac{\epsilon^2}{5} \label{eq:Approx} \\
% 	c_2(\epsilon) 	  &:=& \left(\tfrac{2 - i}{5}\right) \epsilon \nonumber
% 	\end{eqnarray}
%
% \end{definition}


\section{An Illustrative Example: Learning the Mean of a Distribution}\label{sec:illustrative}

We consider the following very simple WGAN example: The data are generated by a multivariate normal distribution, i.e. $Q \triangleq N(v, I)$ for some $v\in \mathbb{R}^d$. The goal is for the generator to learn the unknown parameter $v$. In  Appendix \ref{sec:covariance} we also consider a more complex example where the generator is trying to learn a co-variance matrix. 

We consider a WGAN, where the discriminator is a linear function and the generator is a simple additive displacement of the input noise $z$, which is drawn from $F\triangleq N(0, I)$, i.e:
\begin{equation}
\begin{aligned}
D_w(x) =~& \langle w, x\rangle\\
G_{\theta}(z) =~& z + \theta
\end{aligned}
\end{equation}
The goal of the generator is to figure out the true distribution, i.e. to converge to $\theta = v$. The WGAN loss then takes the simple form:
\begin{equation}
L(\theta, w) =  \mathbb{E}_{x\sim N(v, I)}\left[ \langle w, x \rangle \right] - \mathbb{E}_{z\sim N(0,I)}\left[\langle w, z + \theta \rangle\right] 
\end{equation}
We first consider the case where we optimize the true expectations above rather than assuming that we only get samples of $x$ and samples of $z$. Due to linearity of expectation, the expected zero-sum game takes the form:
\begin{equation}
\inf_{\theta} \sup_{w}~\langle w, v-\theta \rangle
\end{equation}
We see here that the unique equilibrium of the above game is for the generator to choose $\theta=v$ and for the discriminator to choose $w = 0$. For this simple zero sum game, we have $\nabla_{w, t}=v-\theta_t$ and $\nabla_{\theta, t}=-w_t$. Hence, the GD dynamics take the form:
\begin{equation}
\begin{aligned}
w_{t+1} =& w_{t} + \eta (v - \theta_{t})\\
\theta_{t+1} =& \theta_{t} + \eta w_t  
\end{aligned}\tag{GD Dynamics for Learning Means}
\end{equation}
while the OMD dynamics take the form:
\begin{equation}
\begin{aligned}
w_{t+1} =& w_{t} + 2\eta\cdot (v - \theta_{t}) - \eta \cdot (v-\theta_{t-1})\\
\theta_{t+1} =& \theta_{t} + 2\eta\cdot w_t - \eta\cdot  w_{t-1} 
\end{aligned}\tag{OMD Dynamics for Learning Means}
\end{equation}

We simulated simultaneous training in this zero-sum game under the GD and under OMD dynamics and we find that GD dynamics always lead to a limit cycle irrespective of the step size or other modifications. In Figure \ref{fig:gd} we present the behavior of the GD vs OMD dynamics in this game for $v = (3, 4)$. We see that even though GD dynamics leads to a limit cycle (whose average does indeed equal to the true vector), the OMD dynamics converge to $v$ in terms of the last iterate. In Figure \ref{fig:sampling} we see that the stability of OMD even carries over to the case of Stochastic Gradients, as long as the batch size is of decent size. 

\begin{figure}[htpb]
    \centering
    \begin{subfigure}[b]{.49\textwidth}
        \centering
        \includegraphics[height=.7in]{sgd_example.png}
        \caption{GD dynamics.}
    \end{subfigure}
    ~ 
    \begin{subfigure}[b]{.49\textwidth}
        \centering
        \includegraphics[height=.7in]{omd_example.png}
        \caption{OMD dynamics.}
    \end{subfigure}
    \caption{Training GAN with GD converges to a limit cycle that oscilates around the equilibrium (we applied weight-clipping at $10$ for the discriminator). On the contrary training with OMD converges to equilibrium in terms of last-iterate convergence.}\label{fig:gd}
\end{figure}

In the appendix we also portray the behavior of the GD dynamics even when we add gradient penalty \citep{Gulrajani2017} to the game loss (instead of weight clipping), adding Nesterov momentum to the GD update rule \citep{Nesterov} or when we train the discriminator multiple times in between a train iteration of the generator. We see that even though these modifications do improve the stability of the GD dynamics, in the sense that they narrow the band of the limit cycle, they still lead to a non-vanishing limit cycle, unlike OMD. 

\begin{figure}[htpb]
    \centering
    \begin{subfigure}[b]{.49\textwidth}
        \centering
        \includegraphics[height=.7in]{stochastic_omd_batch_50.png}
        \caption{Stochastic OMD dynamics with mini-batch of $50$.}
    \end{subfigure}
    ~ 
    \begin{subfigure}[b]{.49\textwidth}
        \centering
        \includegraphics[height=.7in]{stochastic_omd_batch_200.png}
        \caption{Stochastic OMD dynamics with mini-batch of $200$.}
    \end{subfigure}
    \caption{Robustness of last-iterate convergence of OMD to stochastic gradients.}\label{fig:sampling}
\end{figure}

In the next section, we will in fact prove formally that for a large class of zero-sum games including the one presented in this section, OMD dynamics converge to  equilibrium in the sense of last-iterate convergence, as opposed to average-iterate convergence.

\section{Last-Iterate Convergence of Optimistic Adversarial Training}
\label{sec:main:proof OMD converges}
In this section, we show that Optimistic Mirror Descent  
exhibits final-iterate, rather than only average-iterate convergence to min-max solutions for
bilinear functions. 
%In particular, we show that the $\ell_2$ norms of the gradients used by the dynamics shrinks in time.
More precisely, we consider the problem $\min_x \max_y x^T A y$, for some matrix $A$, where $x$ and $y$ are unconstrained. In Appendix \ref{sec:appendix:last-iterate}, we also show that our convergence result appropriately extends to the general case, where the bi-linear game
also contains terms that are linear in the players' individual strategies, i.e.~games of the form:
\begin{equation}
    \inf_{x} \sup_{y} \left(x^TAy + b^Tx + c^Ty + d\right). \label{eq:inf sup problem general}
\end{equation}

In the simpler $\min_{x}\max_{y} x^T Ay$ problem, Optimistic Mirror Descent takes the following form, for all $t \ge 1$:
\begin{align}
    x_{t} &= \xto \label{eq:OGD bilinear x repeat}\\
    y_{t} &= \yto  \label{eq:OGD bilinear y repeat}
\end{align}
\noindent {\em Initialization:} For the above iteration to be meaningful we need to specify $x_0,x_{-1},y_0,y_{-1}$. We choose any $x_0 \in
\mathcal{R}(A)$, and $y_0\in\mathcal{R}(A^T)$, and set $x_{-1}=2x_0$ and $y_{-1}=2y_{0}$, where ${\cal R}(\cdot)$ represents the column space of $A$. In particular, our initialization means that the first step taken by the dynamics gives $x_1=x_0$ and $y_1=y_0$.

\smallskip We will analyze Optimistic Mirror Descent under the assumption $\lambda_{\infty} \le 1$, where $\lambda_{\infty}=\max\{||A||,||A^T||\}$ and $||\cdot||$ denotes spectral norm of matrices. We can always enforce that $\lambda_{\infty} \le 1$ by appropriately scaling $A$. Scaling $A$ by some positive factor clearly does not change the min-max solutions $(x^*,y^*)$, only scales the optimal value $x^{*T}Ay^*$ by the same factor.

We remark that the set of equilibrium solutions of this minimax problem  are pairs $(x,y)$ such that $x$ is in the null space of $A^T$ and $y$ is in the null space of $A$. 
%In particular, finding a solution to $\min_x \max_y x^T A y$ is a trivial problem. So 
In this section we rigorously show that Optimistic Mirror Descent converges to the set of such min-max solutions. This is interesting in light of the fact that Gradient Descent actually diverges, even in the special case where $A$ is the identity matrix, as per the following proposition whose proof is provided in Appendix~\ref{appendix:omitted proofs}.

\begin{proposition} \label{prop:gradient descent unstable}
Gradient descent applied to the problem $\min_x \max_y x^T y$ diverges starting from any initialization $x_0, y_0$ such that $x_0,y_0 \neq 0$.
\end{proposition}

Next, we state our main result of this section, whose proof can be found in Appendix~\ref{sec:appendix:last-iterate}, where we also state its appropriate generalization to the general case~\eqref{eq:inf sup problem general}.

\begin{theorem}[Last Iterate Convergence of OMD]\label{thm:convergence of OGD-main}
Consider the dynamics of Eq.~\eqref{eq:OGD bilinear x repeat} and~\eqref{eq:OGD bilinear y repeat} and any initialization ${1 \over 2}x_{-1}=x_0 \in
\mathcal{R}(A)$, and ${1 \over 2}y_{-1}=y_0\in\mathcal{R}(A^T)$. Let also
$$\gamma = \vsedit{\left\|\(AA^T\)^{+}\right\|},$$
%\max\(\left|\left|\(AA^T\)^{+}\right|\right|,
%	    \left|\left|\(A^TA\)^{+}\right|\right|\),$$
where for a matrix $X$ we denote by	$X^{+}$ its generalized inverse and by $||X||$ its spectral norm. Suppose that \vsedit{$\|A\|\equiv \lambda_{\infty}\le 1$} 
%$\max\{||A||,||A^T||\}\equiv \lambda_{\infty}\le 1$ 
and that $\eta$ is a small enough constant satisfying $\eta <1/(3\gamma^2)$. Letting $\Delta_t = \normlt{A^Tx_t} + \normlt{Ay_t}$, the OMD dynamics satisfy the following:
\begin{align*}
&\Delta_1=\Delta_0 \ge {1 \over (1+\eta)^2} \Delta_2\\
    \forall t\ge 3:~~&\Delta_t \leq \left(1-{\eta^2 \over \gamma^2}\right)\Delta_{t-1} + 16\eta^3 \Delta_0. 
\end{align*}
In particular, $\Delta_t \rightarrow O(\eta \gamma^2 \Delta_0)$, as $t \rightarrow +\infty$, and for large enough $t$, the last iterate of OMD is within $O(\sqrt{\eta} \cdot \gamma \sqrt{\Delta_0})$ distance from the space of equilibrium points of the game, where $\sqrt{\Delta_0}$ is the distance of the initial point $(x_0,y_0)$ from the equilibrium space, and where both distances are taken with respect to the norm $\sqrt{x^T A A^T x + y^T A^T A y}$.
\end{theorem}





\section{Experimental Results for Generating DNA Sequences}
\section{Experiments}

\label{sec:experiments}
In this section we compare the tiered vector to some widely used C++ standard library containers. 
We also compare different variants of the tiered vector. 
We consider how the different representations of the data
structure listed in Section~\ref{sec:implementation}, 
and also how the height of tree and the capacity of the leaves affects the running time.
The following describes the test setup:

\subparagraph{Environment}

All experiments have been performed on a Intel Core i7-4770 CPU @ 3.40GHz with
32 GB RAM. The code has been compiled with GNU GCC version 5.4.0 with flags
``-O3''. The reported times are an average over 10 test runs.
 
 \subparagraph{Procedure}
%have been added to the data structure in
 
In all tests $10^8$ 32-bit integers 
are inserted in the data structure as a preliminary step
to simulate that it has already been
used\footnote{In order to minimize the overall running time of the experiments,
the elements were not added randomly, but we show this does not give our data
structure any benefits}.
For all the access and successor operations $10^9$ elements have been accessed
and the time reported is the average time per element.
For range access, 10.000 consecutive elements are accessed.
For insertion/deletion $10^6$ elements
have been (semi-)randomly\footnote{In order to not impact timing, a simple
access pattern has been used instead of a normal pseudo-random generator.}
added/deleted, though in the case of ``vector'' only 10.000 elements were
inserted/deleted to make the experiments terminate in reasonable time. 

\subsection{Comparison to C++ STL Data Structures}

In the following we have compared our best performing tiered vector (see the next sections) to the vector and
the multiset class from the C++ standard library.
The vector data structure directly supports the
operations of a dynamic array. The multiset class is implemented as a red-black
tree and is therefore interesting to compare with our data structure.
Unfortunately, multiset does not directly support the operations of a dynamic
array (in particular it has no notion of positions of elements). To simulate an
access operation we instead find the successor of an element in the multiset.
This requires a root-to-leaf traversal of the red-black tree, just as an access
operation in a dynamic array implemented as a red-black tree would. Insertion
is simulated as an insertion into the multiset, which again requires the same
computations as a dynamic array implemented as a red-black tree would.

Besides the random access, range access and insertion,
we have also tested the operations \textit{data dependent access},
insertion in the end, deletion, and \textit{successor} queries. In the
\textit{data dependent access} tests, the next index to lookup depends on the values of the prior
lookups. This ensures that the CPU cannot successfully pipeline
consecutive lookups, but must perform them in sequence. We test insertion in the end, since
this is a very common use case. Deletion is performed by deleting elements at
random positions. The $successor$ queries returns the successor of an element
and is not actually part of the
dynamic array problem, but is included since it is a commonly used operation on
a multiset in C++. It is simply implemented as a binary search over the elements in
both the vector and tiered vector tests where the elements are now inserted in sorted order. 

The results are summarized in Table~\ref{tab:test_comp} which shows that the vector performs slightly better than the tiered vector on all access and successor tests. As expected from the $\Theta(n)$ running time, it performs extremely poor on random insertion and deletion. For insertion in the end of the sequence, vector is also slightly faster than the tiered vector. The interesting part is that even though the tiered vector requires several extra memory lookups and computations, we have managed to get the running time down to less than the double of the vector for access, even less for data dependent access and only a few percent slowdown for range access. As discussed earlier,
this is most likely because the entire tree structure (without the elements)
fits within the CPU cache, and because the computations required has been minimized.

Comparing our tiered vector to multiset, we would expect access operations to be
faster since they run in $O(1)$ time compared to $O(\log n)$. On the other
hand, we would expect insertion/deletion to be significantly slower since it
runs in $O(n^{1/l})$ time compared to $O(\log n)$ (where $l = 4$ in these tests). We
see our expectations hold for the access operations where the tiered vector is faster by more than an order of magnitude.
In random insertions however,  the tiered vector is only $8\%$ slower -- even when operating on 100.000.000 elements. Both the tiered
vector and set requires $O(\log n)$ time for the successor operation. In our
experiments the tiered vector is 3 times faster for the successor operation.

Finally, we see that the memory usage of vector and tiered vector is almost identical.
This is expected since in both cases the space usage is dominated by the space taken by the actual elements.
The multiset uses more than 10 times as much space, so this is also a considerable drawback of the red-black tree behind this structure. 

To sum up, the tiered vectors performs better than multiset on all tests
but insertion, where it performs only slightly worse.

%\caption{Figures (a) through (e) show the performance of \textit{Tiered Arrays} (\protect\purple) compared
%to the \textit{set} (\protect\green) and \textit{vector} (\protect\blue) data structures from the C++ standard library.} \label{fig:animals}
\begin{table}
	\centering
	\begin{tabular}{|l|r|r|r|r|r|}
		\hline
		& \multicolumn{1}{l|}{\textit{tiered vector}} & \multicolumn{1}{l|}{\textit{set}} & \multicolumn{1}{l|}{\textit{set / tiered}} & \multicolumn{1}{l|}{\textit{vector}} & \multicolumn{1}{l|}{\textit{vector / tiered}} \\ \hline
		access     & $34.07$ ns                                  & $1432.05$ ns                      & 42.03                                      & $21.63$ ns                           & 0.63                                          \\ \hline
		dd-access    & $99.09$ ns                                  & $1436.67$ ns                      & 14.50                                      & $79.37$ ns                           & 0.80                                          \\ \hline
		range access   & $0.24$ ns                                   & $13.02$ ns                        & 53.53                                      & $0.23$ ns                            & 0.93                                          \\ \hline
		insert   & $1.79$ $\mu$s                               & $1.65$ $\mu$s                     & 0.92                                       & $21675.49$ $\mu$s                     & 12082.33                                      \\ \hline
		insertion in end     & $7.28$ ns                               & $242.90$ ns                     & 33.38                                       & $2.93$ ns                     & 0.40                                      \\ \hline
		successor & $0.55$ $\mu$s                               & $1.53$ $\mu$s                     & 2.75                                       & $0.36$ $\mu$s                        & 0.65                                          \\ \hline
		delete     & $1.92$ $\mu$s                               & $1.78$ $\mu$s                     & 0.93                                       & $21295.25$ $\mu$s                     & 11070.04                                      \\ \hline
		memory     & $408$ MB                               & $4802$ MB                     & 11.77                                       & $405$ MB                    & 0.99                                      \\ \hline
	\end{tabular}
	\caption{The table summarizes the performance of the implicit tiered vector
		compared to the performance of multiset and vector from the C++ standard library.\
		dd-access refers to data dependent access.}
\label{tab:test_comp}
\end{table}


\definecolor{cpurple}{RGB}{131,24,197}
\definecolor{cgreen}{RGB}{70,156,118}
\definecolor{cblue}{RGB}{11,178,228}
\definecolor{cdblue}{RGB}{11,112,173}
\definecolor{corange}{RGB}{219,162,55}
\definecolor{cyellow}{RGB}{238,228,98}
\definecolor{cred}{RGB}{110,55,38}
\newcommand{\purple}{\raisebox{2pt}{\tikz{\draw[cpurple,solid,line width=1.9pt](0,0) -- (3mm,0);}}}
\newcommand{\green}{\raisebox{2pt}{\tikz{\draw[cgreen,solid,line width=1.9pt](0,0) -- (3mm,0);}}}
\newcommand{\blue}{\raisebox{2pt}{\tikz{\draw[cblue,solid,line width=1.9pt](0,0) -- (3mm,0);}}}
\newcommand{\dblue}{\raisebox{2pt}{\tikz{\draw[cdblue,solid,line width=1.9pt](0,0) -- (3mm,0);}}}
\newcommand{\orange}{\raisebox{2pt}{\tikz{\draw[corange,solid,line width=1.9pt](0,0) -- (3mm,0);}}}
\newcommand{\yellow}{\raisebox{2pt}{\tikz{\draw[cyellow,solid,line width=1.9pt](0,0) -- (3mm,0);}}}
\newcommand{\red}{\raisebox{2pt}{\tikz{\draw[cred,solid,line width=1.9pt](0,0) -- (3mm,0);}}}


\begin{figure}[ht]
	\centering
	\begin{subfigure}[b]{0.3\textwidth}
		\includegraphics[width=\textwidth]{layout_test_get}
		\caption{\textit{access}}
	\end{subfigure}
	\begin{subfigure}[b]{0.3\textwidth}
		\includegraphics[width=\textwidth]{layout_test_random}
		\caption{\textit{insert}}
	\end{subfigure}
        \caption{Figures (a) and (b) show the performance of the
            \textit{original} (\protect\purple), \textit{optimized original}
            (\protect\green), \textit{lazy} (\protect\blue) \textit{packed
            lazy} (\protect\orange),
            \textit{implicit} (\protect\yellow)
            and \textit{packed implicit} (\protect\dblue) layouts.}
\label{fig:test_representation}
\end{figure}
\subsection{Tiered Vector Variants}

In this test we compare the performance
of the implementations listed in Section~\ref{sec:implementation} to that 
or the original data structure as described in~\ref{thm:pointer}.

%\paragraph{Optimized Original}
\subparagraph*{Optimized Original}
By co-locating the child offset and child pointer, the two memory lookups are at
adjacent memory locations. Due to the cache lines in modern processors,
the second memory lookup will then often be answered directly by the fast
L1-cache.
As can be seen on Figure~\ref{fig:test_representation}, this small change in the memory layout results in a significant improvement in performance for both access and insertion. In the latter case, the running time is more than halved.

%\paragraph{Lazy and Packed Lazy}
\subparagraph*{Lazy and Packed Lazy}

Figure~\ref{fig:test_representation} shows
how the fewer memory probes required by the
\textit{lazy} implementation in comparison to the \text{original}
and \text{optimized original} results in better performance.
Packing the offset and pointer in the leaves results in even better performance
for both access and insertion even though it requires a few extra instructions
to do the actual packing and unpacking.

%\paragraph{Implicit}
\subparagraph*{Implicit}
From Figure~\ref{fig:test_representation}, we see the implicit
data structure is the fastest.
This is as expected because it requires fewer
memory accesses than the other structures except
for the packed lazy which instead has a slight
computational overhead due to the packing and unpacking.

As shown in Theorem~\ref{thm:implicit} the implicit data structure has a
bigger memory overhead than the lazy data structure.
Therefore the packed lazy representation might be beneficial in some
settings.

%\paragraph{Packed Implicit}
\subparagraph*{Packed Implicit}

Packing the offsets array could lead to 
better cache performance due to the smaller memory footprint and therefore
yield better overall performance.
As can be seen on Figure~\ref{fig:test_representation},
the smaller memory footprint
did not improve the performance in practice.
The simple reason for this,
is that the strategy we used for packing the offsets required
extra computation. This clearly dominated the possible gain from the
hypothesized better cache performance. We tried a few strategies to minimize
the extra computations needed at the expense of slightly worse memory usage,
but none of these led to better results than when not packing the offsets at
all.

\subsection{Width Experiments}

\begin{figure}
	\centering
	\begin{subfigure}[b]{0.3\textwidth}
		\includegraphics[width=\textwidth]{width_test_get}
		\caption{\textit{access}}
	\end{subfigure}
	\begin{subfigure}[b]{0.3\textwidth}
		\includegraphics[width=\textwidth]{width_test_sum}
		\caption{\textit{range access}}
	\end{subfigure}
	\begin{subfigure}[b]{0.3\textwidth}
		\includegraphics[width=\textwidth]{width_test_random}
		\caption{\textit{insert}}
	\end{subfigure}
	\caption{Figures (a), (b) and (c) show the performance of the \textit{implicit} (\protect\purple) and
		the \textit{optimized original} tiered vector (\protect\green) for different tree widths.}
\label{fig:test_width}
\end{figure}

This experiment was performed to determine the best capacity ratio between the leaf nodes and the internal nodes.
The six different width configurations we have tested are: 32-32-32-4096, 32-32-64-2048, 32-64-64-1024, 64-64-64-512, 64-64-128-256, and 64-128-128-128.
All configurations have a constant height 4 and a capacity of approximately 130 mio.

We expect the performance of access operations to remain unchanged, since the
amount of work required only depends on the height of the tree,
and not the widths. We expect range access to perform better when the leaf size
is increased, since more elements will be located in consecutive memory
locations. For $insertion$ there is not a clearly expected behavior as the time
used to physically move elements in a leaf will increase with leaf size, but
then less operations on the internal nodes of the tree has to be performed.

On Figure~\ref{fig:test_width} we see access times are actually decreasing
slightly when leaves get bigger. This was not expected, but is most likely
due to small changes in the memory layout that results in slightly better cache
performance. The same is the case for range access, but this was expected. For
insertion, we see there is a tipping point. For our particular instance, the
best performance is achieved when the leaves have size 512.

%Based on this, we have performed the remaining tests with the 64-64-64-512 configuration (unless otherwise specified).

\subsection{Height Experiments}

\begin{figure}
	\centering
	\begin{subfigure}[b]{0.3\textwidth}
		\includegraphics[width=\textwidth]{height_get}
		\caption{\textit{access(i)}}
	\end{subfigure}
	\begin{subfigure}[b]{0.3\textwidth}
		\includegraphics[width=\textwidth]{height_sum}
		\caption{\textit{access(i, m)}}
	\end{subfigure}
	\begin{subfigure}[b]{0.3\textwidth}
		\includegraphics[width=\textwidth]{height_random}
		\caption{\textit{insert}}
	\end{subfigure}
	\caption{Figures (a),(b) and (c) show the performance of the \textit{implicit} (\protect\purple) and
		the \textit{optimized original} tiered vector (\protect\green) for different tree heights.}
\label{fig:test_height}
\end{figure}

In these tests we have studied how different heights affect the performance of
access and insertion operations. We have tested the configurations 8196-16384,
512-512-512, 64-64-64-512, 16-16-32-32-512, 8-8-16-16-16-512. All resulting in
the same capacity, but with heights in the range 2-6.

We expect the access operations to perform better for lower trees, since
the number of operations that must be performed is linear in the height. On the
other hand we expect insertion to perform significantly better with higher
trees, since its running time is $O(n^{1/l})$ where $l$ is the height plus one. 

On Figure~\ref{fig:test_height} we see the results follow our expectations. However, the access operations only perform slightly worse on higher trees.
This is most likely because all internal nodes fit within the L3-cache. Therefore the running time is dominated by the lookup of the element itself.
(It is highly unlikely that the element requested by an access 
to a random position would be among the small fraction of elements that
fit in the L3-cache).

Regarding insertion, we see significant improvements up until a height of 4. After that, increasing the height does not change the running time noticeably. This is most likely due to the hidden constant in $O(n^{1/l})$ increasing rapidly with the height.



\subsection{Configuration Experiments}

\begin{figure}
    \centering
    \begin{subfigure}[b]{0.3\textwidth}
        \includegraphics[width=\textwidth]{small_get}
        \caption{\textit{access}}
    \end{subfigure}
    \begin{subfigure}[b]{0.3\textwidth}
        \includegraphics[width=\textwidth]{small_sum}
        \caption{\textit{range access}}
    \end{subfigure}
    \begin{subfigure}[b]{0.3\textwidth}
        \includegraphics[width=\textwidth]{small_random}
        \caption{\textit{insert(i,x)}}
    \end{subfigure}
    \caption{Figures (a) and (b) show the performance of the
    \textit{base} (\protect\purple),
    \textit{rotated} (\protect\green), 
    \textit{non-aligned sizes} (\protect\blue),
    \textit{non-templated} (\protect\orange)
    layouts.}
\label{fig:test_minor}
\end{figure}

In these experiments, we test a few hypotheses about how different changes
impact the running time. The results are shown on
Figure~\ref{fig:test_minor}, the leftmost result (base) is
the implicit 64-64-64-512 configuration of the tiered vector 
to which we compare our hypotheses.
%our final and best

\textit{Rotated}: 
As already mentioned, the insertions performed as a
preliminary step to the tests are not done at random positions.
This means that all offsets are zero when our real operations
start. The purpose of this test is the ensure that
there are no significant performance gains in starting
from such a configuration which could otherwise
lead to misleading results.
To this end, we have randomized all
offsets (in a way such that the data structure is still valid, but the
order of elements change) after doing the preliminary insertions
but before timing the operations. As can be seen on
Figure~\ref{fig:test_minor}, the difference between this and the normal
procedure is insignificant, thus we find our approach gives a fair picture.


\textit{Non-Aligned Sizes}: In all our previous tests, we have ensured all
nodes had an out-degree that was a power of 2. This was chosen in order to let the
compiler simplify some calculations, i.e.\ replacing multiplication/division
instructions by shift/and instructions. As Figure~\ref{fig:test_minor} shows,
using sizes that are not powers of 2 results in significantly worse performance.
Besides showing that powers of 2 should always be used, this also indicates that not only
the number of memory accesses during an operation is critical for our
performance, but also the amount of computation we make.

\textit{Non-Templated}
The non-templated results 
in Figure~\ref{fig:test_representation} the
show that the change to templated recursion
has had a major impact on the running time. It should be noted that some
improvements have not been implemented in the non-templated version,
but it gives a good indication that this has been quite useful.

\vspace{-.1in}
\section{Generating Images from CIFAR10 with Optimistic Adam}\label{sec:cifar10}
\vspace{-.1in}
In this section we applying optimistic WGAN training to generating images, after training on CIFAR10. Given the success of Adam on training image WGANs we will use an optimistic version of the Adam algorithm, rather than vanilla OMD. We denote the latter by \emph{Optimistic Adam}. Optimistic Adam could be of independent interest even beyond training WGANs. We present Optimistic Adam for (G) but the analog is also used for training (D).
\begin{algorithm}[h]
\begin{algorithmic}
\State Parameters: stepsize $\eta$, exponential decay rates for moment estimates $\beta_1, \beta_2\in [0,1)$, stochastic loss as a function of weights $\ell_t(\theta)$, initial parameters $\theta_0$
%\State Initialize parameters to $\theta_0$
\For{each iteration $t\in \{1,\ldots, T\}$}
\State Compute stochastic gradient: $\nabla_{\theta,t} = \nabla_{\theta} \ell_t(\theta)$
\State Update biased estimate of first moment: $m_t = \beta_1 m_{t-1} + (1-\beta_1) \cdot \nabla_{\theta,t}$
\State Update biased estimate of second moment: $v_t = \beta_2 v_{t-1} + (1-\beta_2) \cdot \nabla_{\theta,t}^2$
\State Compute bias corrected first moment: $\hat{m}_t = m_t/(1 - \beta_1^t)$
\State Compute bias corrected second moment: $\hat{v}_t = v_t/(1 - \beta_2^t)$
\State Perform \emph{optimistic gradient step}: $\theta_t = \theta_{t-1} - 2 \eta \cdot \frac{\hat{m}_t}{\sqrt{\hat{v}_t}+\epsilon} + \eta \frac{\hat{m}_{t-1}}{\sqrt{\hat{v}_{t-1}}+\epsilon}$ 
\EndFor
\State Return $\theta_T$
\end{algorithmic}
\caption{\emph{Optimistic ADAM}, proposed algorithm for training WGANs on images. }\label{alg:opt-adam}
\end{algorithm}
\vspace{-.1in}
We trained on CIFAR10 images with Optimistic Adam with the hyper-parameters matched to \cite{Gulrajani2017}, and we observe that it outperforms Adam in terms of inception score (see Figure \ref{fig:optimistic-Adam}), a standard metric of quality of WGANs \citep{Gulrajani2017, salimans2016improved}. In particular we see that optimistic Adam achieves high numbers of inception scores after very few epochs of training. We observe that for Optimistic Adam, training the discriminator once after one iteration of the generator training, which matches the intuition behind the use of optimism, outperforms the 1:5 generator-discriminator training scheme. We see that vanilla Adam performs poorly when the discriminator is trained only once in between iterations of the generator training. Moreover, even if we use vanilla Adam and train $5$ times (D) in between a training of (G), as proposed by \cite{arjovsky2017wasserstein}, then performance is again worse than Optimistic Adam with a 1:1 ratio of training. The same learning rate $0.0001$ and betas ($\beta_1=0.5, \beta_2=0.9$) as in Appendix B of \cite{Gulrajani2017}  were used for all the methods compared. We also matched other hyper-parameters such as gradient penalty coefficient $\lambda$ and batch size. For a larger sample of images see Appendix \ref{sec:appendix-cifar10}. 
\vspace{-.1in}
\begin{figure}[htpb]
    \centering
    \begin{subfigure}[b]{.67\textwidth}
        \centering
    		\includegraphics[height=1.7in]{optimAdam-eps-converted-to.pdf}        
		\caption{Inception score on CIFAR10, when training with Adam and Optimistic Adam. ``ratio1" means we performed $1$ iteration of training of (D) in between $1$ iteration of (G). Otherwise we performed $5$ iterations. We further test (averaging over 35 trials) the two top-performing optimizers, Adam (ratio 5) and Optimistic Adam with ratio 1, in Appendix~\ref{sec:appendix-errorbars}.}
    \end{subfigure}
    ~~
    \begin{subfigure}[b]{.3\textwidth}
        \centering
    		\includegraphics[height=1.7in]{optimAdam_v0_1e-04_ratio1_epoch93-eps-converted-to.pdf}
    	\caption{Sample of images from Generator of Epoch $94$, which had the highest inception score.}
    \end{subfigure}
    \caption{Comparison of Adam and Optimistic Adam on CIFAR10.}\label{fig:optimistic-Adam}
\end{figure}


\bibliographystyle{iclr2018_conference}
\bibliography{agt}

\newpage

\appendix

\chapter{Supplementary Material}
\label{appendix}

In this appendix, we present supplementary material for the techniques and
experiments presented in the main text.

\section{Baseline Results and Analysis for Informed Sampler}
\label{appendix:chap3}

Here, we give an in-depth
performance analysis of the various samplers and the effect of their
hyperparameters. We choose hyperparameters with the lowest PSRF value
after $10k$ iterations, for each sampler individually. If the
differences between PSRF are not significantly different among
multiple values, we choose the one that has the highest acceptance
rate.

\subsection{Experiment: Estimating Camera Extrinsics}
\label{appendix:chap3:room}

\subsubsection{Parameter Selection}
\paragraph{Metropolis Hastings (\MH)}

Figure~\ref{fig:exp1_MH} shows the median acceptance rates and PSRF
values corresponding to various proposal standard deviations of plain
\MH~sampling. Mixing gets better and the acceptance rate gets worse as
the standard deviation increases. The value $0.3$ is selected standard
deviation for this sampler.

\paragraph{Metropolis Hastings Within Gibbs (\MHWG)}

As mentioned in Section~\ref{sec:room}, the \MHWG~sampler with one-dimensional
updates did not converge for any value of proposal standard deviation.
This problem has high correlation of the camera parameters and is of
multi-modal nature, which this sampler has problems with.

\paragraph{Parallel Tempering (\PT)}

For \PT~sampling, we took the best performing \MH~sampler and used
different temperature chains to improve the mixing of the
sampler. Figure~\ref{fig:exp1_PT} shows the results corresponding to
different combination of temperature levels. The sampler with
temperature levels of $[1,3,27]$ performed best in terms of both
mixing and acceptance rate.

\paragraph{Effect of Mixture Coefficient in Informed Sampling (\MIXLMH)}

Figure~\ref{fig:exp1_alpha} shows the effect of mixture
coefficient ($\alpha$) on the informed sampling
\MIXLMH. Since there is no significant different in PSRF values for
$0 \le \alpha \le 0.7$, we chose $0.7$ due to its high acceptance
rate.


% \end{multicols}

\begin{figure}[h]
\centering
  \subfigure[MH]{%
    \includegraphics[width=.48\textwidth]{figures/supplementary/camPose_MH.pdf} \label{fig:exp1_MH}
  }
  \subfigure[PT]{%
    \includegraphics[width=.48\textwidth]{figures/supplementary/camPose_PT.pdf} \label{fig:exp1_PT}
  }
\\
  \subfigure[INF-MH]{%
    \includegraphics[width=.48\textwidth]{figures/supplementary/camPose_alpha.pdf} \label{fig:exp1_alpha}
  }
  \mycaption{Results of the `Estimating Camera Extrinsics' experiment}{PRSFs and Acceptance rates corresponding to (a) various standard deviations of \MH, (b) various temperature level combinations of \PT sampling and (c) various mixture coefficients of \MIXLMH sampling.}
\end{figure}



\begin{figure}[!t]
\centering
  \subfigure[\MH]{%
    \includegraphics[width=.48\textwidth]{figures/supplementary/occlusionExp_MH.pdf} \label{fig:exp2_MH}
  }
  \subfigure[\BMHWG]{%
    \includegraphics[width=.48\textwidth]{figures/supplementary/occlusionExp_BMHWG.pdf} \label{fig:exp2_BMHWG}
  }
\\
  \subfigure[\MHWG]{%
    \includegraphics[width=.48\textwidth]{figures/supplementary/occlusionExp_MHWG.pdf} \label{fig:exp2_MHWG}
  }
  \subfigure[\PT]{%
    \includegraphics[width=.48\textwidth]{figures/supplementary/occlusionExp_PT.pdf} \label{fig:exp2_PT}
  }
\\
  \subfigure[\INFBMHWG]{%
    \includegraphics[width=.5\textwidth]{figures/supplementary/occlusionExp_alpha.pdf} \label{fig:exp2_alpha}
  }
  \mycaption{Results of the `Occluding Tiles' experiment}{PRSF and
    Acceptance rates corresponding to various standard deviations of
    (a) \MH, (b) \BMHWG, (c) \MHWG, (d) various temperature level
    combinations of \PT~sampling and; (e) various mixture coefficients
    of our informed \INFBMHWG sampling.}
\end{figure}

%\onecolumn\newpage\twocolumn
\subsection{Experiment: Occluding Tiles}
\label{appendix:chap3:tiles}

\subsubsection{Parameter Selection}

\paragraph{Metropolis Hastings (\MH)}

Figure~\ref{fig:exp2_MH} shows the results of
\MH~sampling. Results show the poor convergence for all proposal
standard deviations and rapid decrease of AR with increasing standard
deviation. This is due to the high-dimensional nature of
the problem. We selected a standard deviation of $1.1$.

\paragraph{Blocked Metropolis Hastings Within Gibbs (\BMHWG)}

The results of \BMHWG are shown in Figure~\ref{fig:exp2_BMHWG}. In
this sampler we update only one block of tile variables (of dimension
four) in each sampling step. Results show much better performance
compared to plain \MH. The optimal proposal standard deviation for
this sampler is $0.7$.

\paragraph{Metropolis Hastings Within Gibbs (\MHWG)}

Figure~\ref{fig:exp2_MHWG} shows the result of \MHWG sampling. This
sampler is better than \BMHWG and converges much more quickly. Here
a standard deviation of $0.9$ is found to be best.

\paragraph{Parallel Tempering (\PT)}

Figure~\ref{fig:exp2_PT} shows the results of \PT sampling with various
temperature combinations. Results show no improvement in AR from plain
\MH sampling and again $[1,3,27]$ temperature levels are found to be optimal.

\paragraph{Effect of Mixture Coefficient in Informed Sampling (\INFBMHWG)}

Figure~\ref{fig:exp2_alpha} shows the effect of mixture
coefficient ($\alpha$) on the blocked informed sampling
\INFBMHWG. Since there is no significant different in PSRF values for
$0 \le \alpha \le 0.8$, we chose $0.8$ due to its high acceptance
rate.



\subsection{Experiment: Estimating Body Shape}
\label{appendix:chap3:body}

\subsubsection{Parameter Selection}
\paragraph{Metropolis Hastings (\MH)}

Figure~\ref{fig:exp3_MH} shows the result of \MH~sampling with various
proposal standard deviations. The value of $0.1$ is found to be
best.

\paragraph{Metropolis Hastings Within Gibbs (\MHWG)}

For \MHWG sampling we select $0.3$ proposal standard
deviation. Results are shown in Fig.~\ref{fig:exp3_MHWG}.


\paragraph{Parallel Tempering (\PT)}

As before, results in Fig.~\ref{fig:exp3_PT}, the temperature levels
were selected to be $[1,3,27]$ due its slightly higher AR.

\paragraph{Effect of Mixture Coefficient in Informed Sampling (\MIXLMH)}

Figure~\ref{fig:exp3_alpha} shows the effect of $\alpha$ on PSRF and
AR. Since there is no significant differences in PSRF values for $0 \le
\alpha \le 0.8$, we choose $0.8$.


\begin{figure}[t]
\centering
  \subfigure[\MH]{%
    \includegraphics[width=.48\textwidth]{figures/supplementary/bodyShape_MH.pdf} \label{fig:exp3_MH}
  }
  \subfigure[\MHWG]{%
    \includegraphics[width=.48\textwidth]{figures/supplementary/bodyShape_MHWG.pdf} \label{fig:exp3_MHWG}
  }
\\
  \subfigure[\PT]{%
    \includegraphics[width=.48\textwidth]{figures/supplementary/bodyShape_PT.pdf} \label{fig:exp3_PT}
  }
  \subfigure[\MIXLMH]{%
    \includegraphics[width=.48\textwidth]{figures/supplementary/bodyShape_alpha.pdf} \label{fig:exp3_alpha}
  }
\\
  \mycaption{Results of the `Body Shape Estimation' experiment}{PRSFs and
    Acceptance rates corresponding to various standard deviations of
    (a) \MH, (b) \MHWG; (c) various temperature level combinations
    of \PT sampling and; (d) various mixture coefficients of the
    informed \MIXLMH sampling.}
\end{figure}


\subsection{Results Overview}
Figure~\ref{fig:exp_summary} shows the summary results of the all the three
experimental studies related to informed sampler.
\begin{figure*}[h!]
\centering
  \subfigure[Results for: Estimating Camera Extrinsics]{%
    \includegraphics[width=0.9\textwidth]{figures/supplementary/camPose_ALL.pdf} \label{fig:exp1_all}
  }
  \subfigure[Results for: Occluding Tiles]{%
    \includegraphics[width=0.9\textwidth]{figures/supplementary/occlusionExp_ALL.pdf} \label{fig:exp2_all}
  }
  \subfigure[Results for: Estimating Body Shape]{%
    \includegraphics[width=0.9\textwidth]{figures/supplementary/bodyShape_ALL.pdf} \label{fig:exp3_all}
  }
  \label{fig:exp_summary}
  \mycaption{Summary of the statistics for the three experiments}{Shown are
    for several baseline methods and the informed samplers the
    acceptance rates (left), PSRFs (middle), and RMSE values
    (right). All results are median results over multiple test
    examples.}
\end{figure*}

\subsection{Additional Qualitative Results}

\subsubsection{Occluding Tiles}
In Figure~\ref{fig:exp2_visual_more} more qualitative results of the
occluding tiles experiment are shown. The informed sampling approach
(\INFBMHWG) is better than the best baseline (\MHWG). This still is a
very challenging problem since the parameters for occluded tiles are
flat over a large region. Some of the posterior variance of the
occluded tiles is already captured by the informed sampler.

\begin{figure*}[h!]
\begin{center}
\centerline{\includegraphics[width=0.95\textwidth]{figures/supplementary/occlusionExp_Visual.pdf}}
\mycaption{Additional qualitative results of the occluding tiles experiment}
  {From left to right: (a)
  Given image, (b) Ground truth tiles, (c) OpenCV heuristic and most probable estimates
  from 5000 samples obtained by (d) MHWG sampler (best baseline) and
  (e) our INF-BMHWG sampler. (f) Posterior expectation of the tiles
  boundaries obtained by INF-BMHWG sampling (First 2000 samples are
  discarded as burn-in).}
\label{fig:exp2_visual_more}
\end{center}
\end{figure*}

\subsubsection{Body Shape}
Figure~\ref{fig:exp3_bodyMeshes} shows some more results of 3D mesh
reconstruction using posterior samples obtained by our informed
sampling \MIXLMH.

\begin{figure*}[t]
\begin{center}
\centerline{\includegraphics[width=0.75\textwidth]{figures/supplementary/bodyMeshResults.pdf}}
\mycaption{Qualitative results for the body shape experiment}
  {Shown is the 3D mesh reconstruction results with first 1000 samples obtained
  using the \MIXLMH informed sampling method. (blue indicates small
  values and red indicates high values)}
\label{fig:exp3_bodyMeshes}
\end{center}
\end{figure*}

\clearpage



\section{Additional Results on the Face Problem with CMP}

Figure~\ref{fig:shading-qualitative-multiple-subjects-supp} shows inference results for reflectance maps, normal maps and lights for randomly chosen test images, and Fig.~\ref{fig:shading-qualitative-same-subject-supp} shows reflectance estimation results on multiple images of the same subject produced under different illumination conditions. CMP is able to produce estimates that are closer to the groundtruth across different subjects and illumination conditions.

\begin{figure*}[h]
  \begin{center}
  \centerline{\includegraphics[width=1.0\columnwidth]{figures/face_cmp_visual_results_supp.pdf}}
  \vspace{-1.2cm}
  \end{center}
	\mycaption{A visual comparison of inference results}{(a)~Observed images. (b)~Inferred reflectance maps. \textit{GT} is the photometric stereo groundtruth, \textit{BU} is the Biswas \etal (2009) reflectance estimate and \textit{Forest} is the consensus prediction. (c)~The variance of the inferred reflectance estimate produced by \MTD (normalized across rows).(d)~Visualization of inferred light directions. (e)~Inferred normal maps.}
	\label{fig:shading-qualitative-multiple-subjects-supp}
\end{figure*}


\begin{figure*}[h]
	\centering
	\setlength\fboxsep{0.2mm}
	\setlength\fboxrule{0pt}
	\begin{tikzpicture}

		\matrix at (0, 0) [matrix of nodes, nodes={anchor=east}, column sep=-0.05cm, row sep=-0.2cm]
		{
			\fbox{\includegraphics[width=1cm]{figures/sample_3_4_X.png}} &
			\fbox{\includegraphics[width=1cm]{figures/sample_3_4_GT.png}} &
			\fbox{\includegraphics[width=1cm]{figures/sample_3_4_BISWAS.png}}  &
			\fbox{\includegraphics[width=1cm]{figures/sample_3_4_VMP.png}}  &
			\fbox{\includegraphics[width=1cm]{figures/sample_3_4_FOREST.png}}  &
			\fbox{\includegraphics[width=1cm]{figures/sample_3_4_CMP.png}}  &
			\fbox{\includegraphics[width=1cm]{figures/sample_3_4_CMPVAR.png}}
			 \\

			\fbox{\includegraphics[width=1cm]{figures/sample_3_5_X.png}} &
			\fbox{\includegraphics[width=1cm]{figures/sample_3_5_GT.png}} &
			\fbox{\includegraphics[width=1cm]{figures/sample_3_5_BISWAS.png}}  &
			\fbox{\includegraphics[width=1cm]{figures/sample_3_5_VMP.png}}  &
			\fbox{\includegraphics[width=1cm]{figures/sample_3_5_FOREST.png}}  &
			\fbox{\includegraphics[width=1cm]{figures/sample_3_5_CMP.png}}  &
			\fbox{\includegraphics[width=1cm]{figures/sample_3_5_CMPVAR.png}}
			 \\

			\fbox{\includegraphics[width=1cm]{figures/sample_3_6_X.png}} &
			\fbox{\includegraphics[width=1cm]{figures/sample_3_6_GT.png}} &
			\fbox{\includegraphics[width=1cm]{figures/sample_3_6_BISWAS.png}}  &
			\fbox{\includegraphics[width=1cm]{figures/sample_3_6_VMP.png}}  &
			\fbox{\includegraphics[width=1cm]{figures/sample_3_6_FOREST.png}}  &
			\fbox{\includegraphics[width=1cm]{figures/sample_3_6_CMP.png}}  &
			\fbox{\includegraphics[width=1cm]{figures/sample_3_6_CMPVAR.png}}
			 \\
	     };

       \node at (-3.85, -2.0) {\small Observed};
       \node at (-2.55, -2.0) {\small `GT'};
       \node at (-1.27, -2.0) {\small BU};
       \node at (0.0, -2.0) {\small MP};
       \node at (1.27, -2.0) {\small Forest};
       \node at (2.55, -2.0) {\small \textbf{CMP}};
       \node at (3.85, -2.0) {\small Variance};

	\end{tikzpicture}
	\mycaption{Robustness to varying illumination}{Reflectance estimation on a subject images with varying illumination. Left to right: observed image, photometric stereo estimate (GT)
  which is used as a proxy for groundtruth, bottom-up estimate of \cite{Biswas2009}, VMP result, consensus forest estimate, CMP mean, and CMP variance.}
	\label{fig:shading-qualitative-same-subject-supp}
\end{figure*}

\clearpage

\section{Additional Material for Learning Sparse High Dimensional Filters}
\label{sec:appendix-bnn}

This part of supplementary material contains a more detailed overview of the permutohedral
lattice convolution in Section~\ref{sec:permconv}, more experiments in
Section~\ref{sec:addexps} and additional results with protocols for
the experiments presented in Chapter~\ref{chap:bnn} in Section~\ref{sec:addresults}.

\vspace{-0.2cm}
\subsection{General Permutohedral Convolutions}
\label{sec:permconv}

A core technical contribution of this work is the generalization of the Gaussian permutohedral lattice
convolution proposed in~\cite{adams2010fast} to the full non-separable case with the
ability to perform back-propagation. Although, conceptually, there are minor
differences between Gaussian and general parameterized filters, there are non-trivial practical
differences in terms of the algorithmic implementation. The Gauss filters belong to
the separable class and can thus be decomposed into multiple
sequential one dimensional convolutions. We are interested in the general filter
convolutions, which can not be decomposed. Thus, performing a general permutohedral
convolution at a lattice point requires the computation of the inner product with the
neighboring elements in all the directions in the high-dimensional space.

Here, we give more details of the implementation differences of separable
and non-separable filters. In the following, we will explain the scalar case first.
Recall, that the forward pass of general permutohedral convolution
involves 3 steps: \textit{splatting}, \textit{convolving} and \textit{slicing}.
We follow the same splatting and slicing strategies as in~\cite{adams2010fast}
since these operations do not depend on the filter kernel. The main difference
between our work and the existing implementation of~\cite{adams2010fast} is
the way that the convolution operation is executed. This proceeds by constructing
a \emph{blur neighbor} matrix $K$ that stores for every lattice point all
values of the lattice neighbors that are needed to compute the filter output.

\begin{figure}[t!]
  \centering
    \includegraphics[width=0.6\columnwidth]{figures/supplementary/lattice_construction}
  \mycaption{Illustration of 1D permutohedral lattice construction}
  {A $4\times 4$ $(x,y)$ grid lattice is projected onto the plane defined by the normal
  vector $(1,1)^{\top}$. This grid has $s+1=4$ and $d=2$ $(s+1)^{d}=4^2=16$ elements.
  In the projection, all points of the same color are projected onto the same points in the plane.
  The number of elements of the projected lattice is $t=(s+1)^d-s^d=4^2-3^2=7$, that is
  the $(4\times 4)$ grid minus the size of lattice that is $1$ smaller at each size, in this
  case a $(3\times 3)$ lattice (the upper right $(3\times 3)$ elements).
  }
\label{fig:latticeconstruction}
\end{figure}

The blur neighbor matrix is constructed by traversing through all the populated
lattice points and their neighboring elements.
% For efficiency, we do this matrix construction recursively with shared computations
% since $n^{th}$ neighbourhood elements are $1^{st}$ neighborhood elements of $n-1^{th}$ neighbourhood elements. \pg{do not understand}
This is done recursively to share computations. For any lattice point, the neighbors that are
$n$ hops away are the direct neighbors of the points that are $n-1$ hops away.
The size of a $d$ dimensional spatial filter with width $s+1$ is $(s+1)^{d}$ (\eg, a
$3\times 3$ filter, $s=2$ in $d=2$ has $3^2=9$ elements) and this size grows
exponentially in the number of dimensions $d$. The permutohedral lattice is constructed by
projecting a regular grid onto the plane spanned by the $d$ dimensional normal vector ${(1,\ldots,1)}^{\top}$. See
Fig.~\ref{fig:latticeconstruction} for an illustration of the 1D lattice construction.
Many corners of a grid filter are projected onto the same point, in total $t = {(s+1)}^{d} -
s^{d}$ elements remain in the permutohedral filter with $s$ neighborhood in $d-1$ dimensions.
If the lattice has $m$ populated elements, the
matrix $K$ has size $t\times m$. Note that, since the input signal is typically
sparse, only a few lattice corners are being populated in the \textit{slicing} step.
We use a hash-table to keep track of these points and traverse only through
the populated lattice points for this neighborhood matrix construction.

Once the blur neighbor matrix $K$ is constructed, we can perform the convolution
by the matrix vector multiplication
\begin{equation}
\ell' = BK,
\label{eq:conv}
\end{equation}
where $B$ is the $1 \times t$ filter kernel (whose values we will learn) and $\ell'\in\mathbb{R}^{1\times m}$
is the result of the filtering at the $m$ lattice points. In practice, we found that the
matrix $K$ is sometimes too large to fit into GPU memory and we divided the matrix $K$
into smaller pieces to compute Eq.~\ref{eq:conv} sequentially.

In the general multi-dimensional case, the signal $\ell$ is of $c$ dimensions. Then
the kernel $B$ is of size $c \times t$ and $K$ stores the $c$ dimensional vectors
accordingly. When the input and output points are different, we slice only the
input points and splat only at the output points.


\subsection{Additional Experiments}
\label{sec:addexps}
In this section, we discuss more use-cases for the learned bilateral filters, one
use-case of BNNs and two single filter applications for image and 3D mesh denoising.

\subsubsection{Recognition of subsampled MNIST}\label{sec:app_mnist}

One of the strengths of the proposed filter convolution is that it does not
require the input to lie on a regular grid. The only requirement is to define a distance
between features of the input signal.
We highlight this feature with the following experiment using the
classical MNIST ten class classification problem~\cite{lecun1998mnist}. We sample a
sparse set of $N$ points $(x,y)\in [0,1]\times [0,1]$
uniformly at random in the input image, use their interpolated values
as signal and the \emph{continuous} $(x,y)$ positions as features. This mimics
sub-sampling of a high-dimensional signal. To compare against a spatial convolution,
we interpolate the sparse set of values at the grid positions.

We take a reference implementation of LeNet~\cite{lecun1998gradient} that
is part of the Caffe project~\cite{jia2014caffe} and compare it
against the same architecture but replacing the first convolutional
layer with a bilateral convolution layer (BCL). The filter size
and numbers are adjusted to get a comparable number of parameters
($5\times 5$ for LeNet, $2$-neighborhood for BCL).

The results are shown in Table~\ref{tab:all-results}. We see that training
on the original MNIST data (column Original, LeNet vs. BNN) leads to a slight
decrease in performance of the BNN (99.03\%) compared to LeNet
(99.19\%). The BNN can be trained and evaluated on sparse
signals, and we resample the image as described above for $N=$ 100\%, 60\% and
20\% of the total number of pixels. The methods are also evaluated
on test images that are subsampled in the same way. Note that we can
train and test with different subsampling rates. We introduce an additional
bilinear interpolation layer for the LeNet architecture to train on the same
data. In essence, both models perform a spatial interpolation and thus we
expect them to yield a similar classification accuracy. Once the data is of
higher dimensions, the permutohedral convolution will be faster due to hashing
the sparse input points, as well as less memory demanding in comparison to
naive application of a spatial convolution with interpolated values.

\begin{table}[t]
  \begin{center}
    \footnotesize
    \centering
    \begin{tabular}[t]{lllll}
      \toprule
              &     & \multicolumn{3}{c}{Test Subsampling} \\
       Method  & Original & 100\% & 60\% & 20\%\\
      \midrule
       LeNet &  \textbf{0.9919} & 0.9660 & 0.9348 & \textbf{0.6434} \\
       BNN &  0.9903 & \textbf{0.9844} & \textbf{0.9534} & 0.5767 \\
      \hline
       LeNet 100\% & 0.9856 & 0.9809 & 0.9678 & \textbf{0.7386} \\
       BNN 100\% & \textbf{0.9900} & \textbf{0.9863} & \textbf{0.9699} & 0.6910 \\
      \hline
       LeNet 60\% & 0.9848 & 0.9821 & 0.9740 & 0.8151 \\
       BNN 60\% & \textbf{0.9885} & \textbf{0.9864} & \textbf{0.9771} & \textbf{0.8214}\\
      \hline
       LeNet 20\% & \textbf{0.9763} & \textbf{0.9754} & 0.9695 & 0.8928 \\
       BNN 20\% & 0.9728 & 0.9735 & \textbf{0.9701} & \textbf{0.9042}\\
      \bottomrule
    \end{tabular}
  \end{center}
\vspace{-.2cm}
\caption{Classification accuracy on MNIST. We compare the
    LeNet~\cite{lecun1998gradient} implementation that is part of
    Caffe~\cite{jia2014caffe} to the network with the first layer
    replaced by a bilateral convolution layer (BCL). Both are trained
    on the original image resolution (first two rows). Three more BNN
    and CNN models are trained with randomly subsampled images (100\%,
    60\% and 20\% of the pixels). An additional bilinear interpolation
    layer samples the input signal on a spatial grid for the CNN model.
  }
  \label{tab:all-results}
\vspace{-.5cm}
\end{table}

\subsubsection{Image Denoising}

The main application that inspired the development of the bilateral
filtering operation is image denoising~\cite{aurich1995non}, there
using a single Gaussian kernel. Our development allows to learn this
kernel function from data and we explore how to improve using a \emph{single}
but more general bilateral filter.

We use the Berkeley segmentation dataset
(BSDS500)~\cite{arbelaezi2011bsds500} as a test bed. The color
images in the dataset are converted to gray-scale,
and corrupted with Gaussian noise with a standard deviation of
$\frac {25} {255}$.

We compare the performance of four different filter models on a
denoising task.
The first baseline model (`Spatial' in Table \ref{tab:denoising}, $25$
weights) uses a single spatial filter with a kernel size of
$5$ and predicts the scalar gray-scale value at the center pixel. The next model
(`Gauss Bilateral') applies a bilateral \emph{Gaussian}
filter to the noisy input, using position and intensity features $\f=(x,y,v)^\top$.
The third setup (`Learned Bilateral', $65$ weights)
takes a Gauss kernel as initialization and
fits all filter weights on the train set to minimize the
mean squared error with respect to the clean images.
We run a combination
of spatial and permutohedral convolutions on spatial and bilateral
features (`Spatial + Bilateral (Learned)') to check for a complementary
performance of the two convolutions.

\label{sec:exp:denoising}
\begin{table}[!h]
\begin{center}
  \footnotesize
  \begin{tabular}[t]{lr}
    \toprule
    Method & PSNR \\
    \midrule
    Noisy Input & $20.17$ \\
    Spatial & $26.27$ \\
    Gauss Bilateral & $26.51$ \\
    Learned Bilateral & $26.58$ \\
    Spatial + Bilateral (Learned) & \textbf{$26.65$} \\
    \bottomrule
  \end{tabular}
\end{center}
\vspace{-0.5em}
\caption{PSNR results of a denoising task using the BSDS500
  dataset~\cite{arbelaezi2011bsds500}}
\vspace{-0.5em}
\label{tab:denoising}
\end{table}
\vspace{-0.2em}

The PSNR scores evaluated on full images of the test set are
shown in Table \ref{tab:denoising}. We find that an untrained bilateral
filter already performs better than a trained spatial convolution
($26.27$ to $26.51$). A learned convolution further improve the
performance slightly. We chose this simple one-kernel setup to
validate an advantage of the generalized bilateral filter. A competitive
denoising system would employ RGB color information and also
needs to be properly adjusted in network size. Multi-layer perceptrons
have obtained state-of-the-art denoising results~\cite{burger12cvpr}
and the permutohedral lattice layer can readily be used in such an
architecture, which is intended future work.

\subsection{Additional results}
\label{sec:addresults}

This section contains more qualitative results for the experiments presented in Chapter~\ref{chap:bnn}.

\begin{figure*}[th!]
  \centering
    \includegraphics[width=\columnwidth,trim={5cm 2.5cm 5cm 4.5cm},clip]{figures/supplementary/lattice_viz.pdf}
    \vspace{-0.7cm}
  \mycaption{Visualization of the Permutohedral Lattice}
  {Sample lattice visualizations for different feature spaces. All pixels falling in the same simplex cell are shown with
  the same color. $(x,y)$ features correspond to image pixel positions, and $(r,g,b) \in [0,255]$ correspond
  to the red, green and blue color values.}
\label{fig:latticeviz}
\end{figure*}

\subsubsection{Lattice Visualization}

Figure~\ref{fig:latticeviz} shows sample lattice visualizations for different feature spaces.

\newcolumntype{L}[1]{>{\raggedright\let\newline\\\arraybackslash\hspace{0pt}}b{#1}}
\newcolumntype{C}[1]{>{\centering\let\newline\\\arraybackslash\hspace{0pt}}b{#1}}
\newcolumntype{R}[1]{>{\raggedleft\let\newline\\\arraybackslash\hspace{0pt}}b{#1}}

\subsubsection{Color Upsampling}\label{sec:color_upsampling}
\label{sec:col_upsample_extra}

Some images of the upsampling for the Pascal
VOC12 dataset are shown in Fig.~\ref{fig:Colour_upsample_visuals}. It is
especially the low level image details that are better preserved with
a learned bilateral filter compared to the Gaussian case.

\begin{figure*}[t!]
  \centering
    \subfigure{%
   \raisebox{2.0em}{
    \includegraphics[width=.06\columnwidth]{figures/supplementary/2007_004969.jpg}
   }
  }
  \subfigure{%
    \includegraphics[width=.17\columnwidth]{figures/supplementary/2007_004969_gray.pdf}
  }
  \subfigure{%
    \includegraphics[width=.17\columnwidth]{figures/supplementary/2007_004969_gt.pdf}
  }
  \subfigure{%
    \includegraphics[width=.17\columnwidth]{figures/supplementary/2007_004969_bicubic.pdf}
  }
  \subfigure{%
    \includegraphics[width=.17\columnwidth]{figures/supplementary/2007_004969_gauss.pdf}
  }
  \subfigure{%
    \includegraphics[width=.17\columnwidth]{figures/supplementary/2007_004969_learnt.pdf}
  }\\
    \subfigure{%
   \raisebox{2.0em}{
    \includegraphics[width=.06\columnwidth]{figures/supplementary/2007_003106.jpg}
   }
  }
  \subfigure{%
    \includegraphics[width=.17\columnwidth]{figures/supplementary/2007_003106_gray.pdf}
  }
  \subfigure{%
    \includegraphics[width=.17\columnwidth]{figures/supplementary/2007_003106_gt.pdf}
  }
  \subfigure{%
    \includegraphics[width=.17\columnwidth]{figures/supplementary/2007_003106_bicubic.pdf}
  }
  \subfigure{%
    \includegraphics[width=.17\columnwidth]{figures/supplementary/2007_003106_gauss.pdf}
  }
  \subfigure{%
    \includegraphics[width=.17\columnwidth]{figures/supplementary/2007_003106_learnt.pdf}
  }\\
  \setcounter{subfigure}{0}
  \small{
  \subfigure[Inp.]{%
  \raisebox{2.0em}{
    \includegraphics[width=.06\columnwidth]{figures/supplementary/2007_006837.jpg}
   }
  }
  \subfigure[Guidance]{%
    \includegraphics[width=.17\columnwidth]{figures/supplementary/2007_006837_gray.pdf}
  }
   \subfigure[GT]{%
    \includegraphics[width=.17\columnwidth]{figures/supplementary/2007_006837_gt.pdf}
  }
  \subfigure[Bicubic]{%
    \includegraphics[width=.17\columnwidth]{figures/supplementary/2007_006837_bicubic.pdf}
  }
  \subfigure[Gauss-BF]{%
    \includegraphics[width=.17\columnwidth]{figures/supplementary/2007_006837_gauss.pdf}
  }
  \subfigure[Learned-BF]{%
    \includegraphics[width=.17\columnwidth]{figures/supplementary/2007_006837_learnt.pdf}
  }
  }
  \vspace{-0.5cm}
  \mycaption{Color Upsampling}{Color $8\times$ upsampling results
  using different methods, from left to right, (a)~Low-resolution input color image (Inp.),
  (b)~Gray scale guidance image, (c)~Ground-truth color image; Upsampled color images with
  (d)~Bicubic interpolation, (e) Gauss bilateral upsampling and, (f)~Learned bilateral
  updampgling (best viewed on screen).}

\label{fig:Colour_upsample_visuals}
\end{figure*}

\subsubsection{Depth Upsampling}
\label{sec:depth_upsample_extra}

Figure~\ref{fig:depth_upsample_visuals} presents some more qualitative results comparing bicubic interpolation, Gauss
bilateral and learned bilateral upsampling on NYU depth dataset image~\cite{silberman2012indoor}.

\subsubsection{Character Recognition}\label{sec:app_character}

 Figure~\ref{fig:nnrecognition} shows the schematic of different layers
 of the network architecture for LeNet-7~\cite{lecun1998mnist}
 and DeepCNet(5, 50)~\cite{ciresan2012multi,graham2014spatially}. For the BNN variants, the first layer filters are replaced
 with learned bilateral filters and are learned end-to-end.

\subsubsection{Semantic Segmentation}\label{sec:app_semantic_segmentation}
\label{sec:semantic_bnn_extra}

Some more visual results for semantic segmentation are shown in Figure~\ref{fig:semantic_visuals}.
These include the underlying DeepLab CNN\cite{chen2014semantic} result (DeepLab),
the 2 step mean-field result with Gaussian edge potentials (+2stepMF-GaussCRF)
and also corresponding results with learned edge potentials (+2stepMF-LearnedCRF).
In general, we observe that mean-field in learned CRF leads to slightly dilated
classification regions in comparison to using Gaussian CRF thereby filling-in the
false negative pixels and also correcting some mis-classified regions.

\begin{figure*}[t!]
  \centering
    \subfigure{%
   \raisebox{2.0em}{
    \includegraphics[width=.06\columnwidth]{figures/supplementary/2bicubic}
   }
  }
  \subfigure{%
    \includegraphics[width=.17\columnwidth]{figures/supplementary/2given_image}
  }
  \subfigure{%
    \includegraphics[width=.17\columnwidth]{figures/supplementary/2ground_truth}
  }
  \subfigure{%
    \includegraphics[width=.17\columnwidth]{figures/supplementary/2bicubic}
  }
  \subfigure{%
    \includegraphics[width=.17\columnwidth]{figures/supplementary/2gauss}
  }
  \subfigure{%
    \includegraphics[width=.17\columnwidth]{figures/supplementary/2learnt}
  }\\
    \subfigure{%
   \raisebox{2.0em}{
    \includegraphics[width=.06\columnwidth]{figures/supplementary/32bicubic}
   }
  }
  \subfigure{%
    \includegraphics[width=.17\columnwidth]{figures/supplementary/32given_image}
  }
  \subfigure{%
    \includegraphics[width=.17\columnwidth]{figures/supplementary/32ground_truth}
  }
  \subfigure{%
    \includegraphics[width=.17\columnwidth]{figures/supplementary/32bicubic}
  }
  \subfigure{%
    \includegraphics[width=.17\columnwidth]{figures/supplementary/32gauss}
  }
  \subfigure{%
    \includegraphics[width=.17\columnwidth]{figures/supplementary/32learnt}
  }\\
  \setcounter{subfigure}{0}
  \small{
  \subfigure[Inp.]{%
  \raisebox{2.0em}{
    \includegraphics[width=.06\columnwidth]{figures/supplementary/41bicubic}
   }
  }
  \subfigure[Guidance]{%
    \includegraphics[width=.17\columnwidth]{figures/supplementary/41given_image}
  }
   \subfigure[GT]{%
    \includegraphics[width=.17\columnwidth]{figures/supplementary/41ground_truth}
  }
  \subfigure[Bicubic]{%
    \includegraphics[width=.17\columnwidth]{figures/supplementary/41bicubic}
  }
  \subfigure[Gauss-BF]{%
    \includegraphics[width=.17\columnwidth]{figures/supplementary/41gauss}
  }
  \subfigure[Learned-BF]{%
    \includegraphics[width=.17\columnwidth]{figures/supplementary/41learnt}
  }
  }
  \mycaption{Depth Upsampling}{Depth $8\times$ upsampling results
  using different upsampling strategies, from left to right,
  (a)~Low-resolution input depth image (Inp.),
  (b)~High-resolution guidance image, (c)~Ground-truth depth; Upsampled depth images with
  (d)~Bicubic interpolation, (e) Gauss bilateral upsampling and, (f)~Learned bilateral
  updampgling (best viewed on screen).}

\label{fig:depth_upsample_visuals}
\end{figure*}

\subsubsection{Material Segmentation}\label{sec:app_material_segmentation}
\label{sec:material_bnn_extra}

In Fig.~\ref{fig:material_visuals-app2}, we present visual results comparing 2 step
mean-field inference with Gaussian and learned pairwise CRF potentials. In
general, we observe that the pixels belonging to dominant classes in the
training data are being more accurately classified with learned CRF. This leads to
a significant improvements in overall pixel accuracy. This also results
in a slight decrease of the accuracy from less frequent class pixels thereby
slightly reducing the average class accuracy with learning. We attribute this
to the type of annotation that is available for this dataset, which is not
for the entire image but for some segments in the image. We have very few
images of the infrequent classes to combat this behaviour during training.

\subsubsection{Experiment Protocols}
\label{sec:protocols}

Table~\ref{tbl:parameters} shows experiment protocols of different experiments.

 \begin{figure*}[t!]
  \centering
  \subfigure[LeNet-7]{
    \includegraphics[width=0.7\columnwidth]{figures/supplementary/lenet_cnn_network}
    }\\
    \subfigure[DeepCNet]{
    \includegraphics[width=\columnwidth]{figures/supplementary/deepcnet_cnn_network}
    }
  \mycaption{CNNs for Character Recognition}
  {Schematic of (top) LeNet-7~\cite{lecun1998mnist} and (bottom) DeepCNet(5,50)~\cite{ciresan2012multi,graham2014spatially} architectures used in Assamese
  character recognition experiments.}
\label{fig:nnrecognition}
\end{figure*}

\definecolor{voc_1}{RGB}{0, 0, 0}
\definecolor{voc_2}{RGB}{128, 0, 0}
\definecolor{voc_3}{RGB}{0, 128, 0}
\definecolor{voc_4}{RGB}{128, 128, 0}
\definecolor{voc_5}{RGB}{0, 0, 128}
\definecolor{voc_6}{RGB}{128, 0, 128}
\definecolor{voc_7}{RGB}{0, 128, 128}
\definecolor{voc_8}{RGB}{128, 128, 128}
\definecolor{voc_9}{RGB}{64, 0, 0}
\definecolor{voc_10}{RGB}{192, 0, 0}
\definecolor{voc_11}{RGB}{64, 128, 0}
\definecolor{voc_12}{RGB}{192, 128, 0}
\definecolor{voc_13}{RGB}{64, 0, 128}
\definecolor{voc_14}{RGB}{192, 0, 128}
\definecolor{voc_15}{RGB}{64, 128, 128}
\definecolor{voc_16}{RGB}{192, 128, 128}
\definecolor{voc_17}{RGB}{0, 64, 0}
\definecolor{voc_18}{RGB}{128, 64, 0}
\definecolor{voc_19}{RGB}{0, 192, 0}
\definecolor{voc_20}{RGB}{128, 192, 0}
\definecolor{voc_21}{RGB}{0, 64, 128}
\definecolor{voc_22}{RGB}{128, 64, 128}

\begin{figure*}[t]
  \centering
  \small{
  \fcolorbox{white}{voc_1}{\rule{0pt}{6pt}\rule{6pt}{0pt}} Background~~
  \fcolorbox{white}{voc_2}{\rule{0pt}{6pt}\rule{6pt}{0pt}} Aeroplane~~
  \fcolorbox{white}{voc_3}{\rule{0pt}{6pt}\rule{6pt}{0pt}} Bicycle~~
  \fcolorbox{white}{voc_4}{\rule{0pt}{6pt}\rule{6pt}{0pt}} Bird~~
  \fcolorbox{white}{voc_5}{\rule{0pt}{6pt}\rule{6pt}{0pt}} Boat~~
  \fcolorbox{white}{voc_6}{\rule{0pt}{6pt}\rule{6pt}{0pt}} Bottle~~
  \fcolorbox{white}{voc_7}{\rule{0pt}{6pt}\rule{6pt}{0pt}} Bus~~
  \fcolorbox{white}{voc_8}{\rule{0pt}{6pt}\rule{6pt}{0pt}} Car~~ \\
  \fcolorbox{white}{voc_9}{\rule{0pt}{6pt}\rule{6pt}{0pt}} Cat~~
  \fcolorbox{white}{voc_10}{\rule{0pt}{6pt}\rule{6pt}{0pt}} Chair~~
  \fcolorbox{white}{voc_11}{\rule{0pt}{6pt}\rule{6pt}{0pt}} Cow~~
  \fcolorbox{white}{voc_12}{\rule{0pt}{6pt}\rule{6pt}{0pt}} Dining Table~~
  \fcolorbox{white}{voc_13}{\rule{0pt}{6pt}\rule{6pt}{0pt}} Dog~~
  \fcolorbox{white}{voc_14}{\rule{0pt}{6pt}\rule{6pt}{0pt}} Horse~~
  \fcolorbox{white}{voc_15}{\rule{0pt}{6pt}\rule{6pt}{0pt}} Motorbike~~
  \fcolorbox{white}{voc_16}{\rule{0pt}{6pt}\rule{6pt}{0pt}} Person~~ \\
  \fcolorbox{white}{voc_17}{\rule{0pt}{6pt}\rule{6pt}{0pt}} Potted Plant~~
  \fcolorbox{white}{voc_18}{\rule{0pt}{6pt}\rule{6pt}{0pt}} Sheep~~
  \fcolorbox{white}{voc_19}{\rule{0pt}{6pt}\rule{6pt}{0pt}} Sofa~~
  \fcolorbox{white}{voc_20}{\rule{0pt}{6pt}\rule{6pt}{0pt}} Train~~
  \fcolorbox{white}{voc_21}{\rule{0pt}{6pt}\rule{6pt}{0pt}} TV monitor~~ \\
  }
  \subfigure{%
    \includegraphics[width=.18\columnwidth]{figures/supplementary/2007_001423_given.jpg}
  }
  \subfigure{%
    \includegraphics[width=.18\columnwidth]{figures/supplementary/2007_001423_gt.png}
  }
  \subfigure{%
    \includegraphics[width=.18\columnwidth]{figures/supplementary/2007_001423_cnn.png}
  }
  \subfigure{%
    \includegraphics[width=.18\columnwidth]{figures/supplementary/2007_001423_gauss.png}
  }
  \subfigure{%
    \includegraphics[width=.18\columnwidth]{figures/supplementary/2007_001423_learnt.png}
  }\\
  \subfigure{%
    \includegraphics[width=.18\columnwidth]{figures/supplementary/2007_001430_given.jpg}
  }
  \subfigure{%
    \includegraphics[width=.18\columnwidth]{figures/supplementary/2007_001430_gt.png}
  }
  \subfigure{%
    \includegraphics[width=.18\columnwidth]{figures/supplementary/2007_001430_cnn.png}
  }
  \subfigure{%
    \includegraphics[width=.18\columnwidth]{figures/supplementary/2007_001430_gauss.png}
  }
  \subfigure{%
    \includegraphics[width=.18\columnwidth]{figures/supplementary/2007_001430_learnt.png}
  }\\
    \subfigure{%
    \includegraphics[width=.18\columnwidth]{figures/supplementary/2007_007996_given.jpg}
  }
  \subfigure{%
    \includegraphics[width=.18\columnwidth]{figures/supplementary/2007_007996_gt.png}
  }
  \subfigure{%
    \includegraphics[width=.18\columnwidth]{figures/supplementary/2007_007996_cnn.png}
  }
  \subfigure{%
    \includegraphics[width=.18\columnwidth]{figures/supplementary/2007_007996_gauss.png}
  }
  \subfigure{%
    \includegraphics[width=.18\columnwidth]{figures/supplementary/2007_007996_learnt.png}
  }\\
   \subfigure{%
    \includegraphics[width=.18\columnwidth]{figures/supplementary/2010_002682_given.jpg}
  }
  \subfigure{%
    \includegraphics[width=.18\columnwidth]{figures/supplementary/2010_002682_gt.png}
  }
  \subfigure{%
    \includegraphics[width=.18\columnwidth]{figures/supplementary/2010_002682_cnn.png}
  }
  \subfigure{%
    \includegraphics[width=.18\columnwidth]{figures/supplementary/2010_002682_gauss.png}
  }
  \subfigure{%
    \includegraphics[width=.18\columnwidth]{figures/supplementary/2010_002682_learnt.png}
  }\\
     \subfigure{%
    \includegraphics[width=.18\columnwidth]{figures/supplementary/2010_004789_given.jpg}
  }
  \subfigure{%
    \includegraphics[width=.18\columnwidth]{figures/supplementary/2010_004789_gt.png}
  }
  \subfigure{%
    \includegraphics[width=.18\columnwidth]{figures/supplementary/2010_004789_cnn.png}
  }
  \subfigure{%
    \includegraphics[width=.18\columnwidth]{figures/supplementary/2010_004789_gauss.png}
  }
  \subfigure{%
    \includegraphics[width=.18\columnwidth]{figures/supplementary/2010_004789_learnt.png}
  }\\
       \subfigure{%
    \includegraphics[width=.18\columnwidth]{figures/supplementary/2007_001311_given.jpg}
  }
  \subfigure{%
    \includegraphics[width=.18\columnwidth]{figures/supplementary/2007_001311_gt.png}
  }
  \subfigure{%
    \includegraphics[width=.18\columnwidth]{figures/supplementary/2007_001311_cnn.png}
  }
  \subfigure{%
    \includegraphics[width=.18\columnwidth]{figures/supplementary/2007_001311_gauss.png}
  }
  \subfigure{%
    \includegraphics[width=.18\columnwidth]{figures/supplementary/2007_001311_learnt.png}
  }\\
  \setcounter{subfigure}{0}
  \subfigure[Input]{%
    \includegraphics[width=.18\columnwidth]{figures/supplementary/2010_003531_given.jpg}
  }
  \subfigure[Ground Truth]{%
    \includegraphics[width=.18\columnwidth]{figures/supplementary/2010_003531_gt.png}
  }
  \subfigure[DeepLab]{%
    \includegraphics[width=.18\columnwidth]{figures/supplementary/2010_003531_cnn.png}
  }
  \subfigure[+GaussCRF]{%
    \includegraphics[width=.18\columnwidth]{figures/supplementary/2010_003531_gauss.png}
  }
  \subfigure[+LearnedCRF]{%
    \includegraphics[width=.18\columnwidth]{figures/supplementary/2010_003531_learnt.png}
  }
  \vspace{-0.3cm}
  \mycaption{Semantic Segmentation}{Example results of semantic segmentation.
  (c)~depicts the unary results before application of MF, (d)~after two steps of MF with Gaussian edge CRF potentials, (e)~after
  two steps of MF with learned edge CRF potentials.}
    \label{fig:semantic_visuals}
\end{figure*}


\definecolor{minc_1}{HTML}{771111}
\definecolor{minc_2}{HTML}{CAC690}
\definecolor{minc_3}{HTML}{EEEEEE}
\definecolor{minc_4}{HTML}{7C8FA6}
\definecolor{minc_5}{HTML}{597D31}
\definecolor{minc_6}{HTML}{104410}
\definecolor{minc_7}{HTML}{BB819C}
\definecolor{minc_8}{HTML}{D0CE48}
\definecolor{minc_9}{HTML}{622745}
\definecolor{minc_10}{HTML}{666666}
\definecolor{minc_11}{HTML}{D54A31}
\definecolor{minc_12}{HTML}{101044}
\definecolor{minc_13}{HTML}{444126}
\definecolor{minc_14}{HTML}{75D646}
\definecolor{minc_15}{HTML}{DD4348}
\definecolor{minc_16}{HTML}{5C8577}
\definecolor{minc_17}{HTML}{C78472}
\definecolor{minc_18}{HTML}{75D6D0}
\definecolor{minc_19}{HTML}{5B4586}
\definecolor{minc_20}{HTML}{C04393}
\definecolor{minc_21}{HTML}{D69948}
\definecolor{minc_22}{HTML}{7370D8}
\definecolor{minc_23}{HTML}{7A3622}
\definecolor{minc_24}{HTML}{000000}

\begin{figure*}[t]
  \centering
  \small{
  \fcolorbox{white}{minc_1}{\rule{0pt}{6pt}\rule{6pt}{0pt}} Brick~~
  \fcolorbox{white}{minc_2}{\rule{0pt}{6pt}\rule{6pt}{0pt}} Carpet~~
  \fcolorbox{white}{minc_3}{\rule{0pt}{6pt}\rule{6pt}{0pt}} Ceramic~~
  \fcolorbox{white}{minc_4}{\rule{0pt}{6pt}\rule{6pt}{0pt}} Fabric~~
  \fcolorbox{white}{minc_5}{\rule{0pt}{6pt}\rule{6pt}{0pt}} Foliage~~
  \fcolorbox{white}{minc_6}{\rule{0pt}{6pt}\rule{6pt}{0pt}} Food~~
  \fcolorbox{white}{minc_7}{\rule{0pt}{6pt}\rule{6pt}{0pt}} Glass~~
  \fcolorbox{white}{minc_8}{\rule{0pt}{6pt}\rule{6pt}{0pt}} Hair~~ \\
  \fcolorbox{white}{minc_9}{\rule{0pt}{6pt}\rule{6pt}{0pt}} Leather~~
  \fcolorbox{white}{minc_10}{\rule{0pt}{6pt}\rule{6pt}{0pt}} Metal~~
  \fcolorbox{white}{minc_11}{\rule{0pt}{6pt}\rule{6pt}{0pt}} Mirror~~
  \fcolorbox{white}{minc_12}{\rule{0pt}{6pt}\rule{6pt}{0pt}} Other~~
  \fcolorbox{white}{minc_13}{\rule{0pt}{6pt}\rule{6pt}{0pt}} Painted~~
  \fcolorbox{white}{minc_14}{\rule{0pt}{6pt}\rule{6pt}{0pt}} Paper~~
  \fcolorbox{white}{minc_15}{\rule{0pt}{6pt}\rule{6pt}{0pt}} Plastic~~\\
  \fcolorbox{white}{minc_16}{\rule{0pt}{6pt}\rule{6pt}{0pt}} Polished Stone~~
  \fcolorbox{white}{minc_17}{\rule{0pt}{6pt}\rule{6pt}{0pt}} Skin~~
  \fcolorbox{white}{minc_18}{\rule{0pt}{6pt}\rule{6pt}{0pt}} Sky~~
  \fcolorbox{white}{minc_19}{\rule{0pt}{6pt}\rule{6pt}{0pt}} Stone~~
  \fcolorbox{white}{minc_20}{\rule{0pt}{6pt}\rule{6pt}{0pt}} Tile~~
  \fcolorbox{white}{minc_21}{\rule{0pt}{6pt}\rule{6pt}{0pt}} Wallpaper~~
  \fcolorbox{white}{minc_22}{\rule{0pt}{6pt}\rule{6pt}{0pt}} Water~~
  \fcolorbox{white}{minc_23}{\rule{0pt}{6pt}\rule{6pt}{0pt}} Wood~~ \\
  }
  \subfigure{%
    \includegraphics[width=.18\columnwidth]{figures/supplementary/000010868_given.jpg}
  }
  \subfigure{%
    \includegraphics[width=.18\columnwidth]{figures/supplementary/000010868_gt.png}
  }
  \subfigure{%
    \includegraphics[width=.18\columnwidth]{figures/supplementary/000010868_cnn.png}
  }
  \subfigure{%
    \includegraphics[width=.18\columnwidth]{figures/supplementary/000010868_gauss.png}
  }
  \subfigure{%
    \includegraphics[width=.18\columnwidth]{figures/supplementary/000010868_learnt.png}
  }\\[-2ex]
  \subfigure{%
    \includegraphics[width=.18\columnwidth]{figures/supplementary/000006011_given.jpg}
  }
  \subfigure{%
    \includegraphics[width=.18\columnwidth]{figures/supplementary/000006011_gt.png}
  }
  \subfigure{%
    \includegraphics[width=.18\columnwidth]{figures/supplementary/000006011_cnn.png}
  }
  \subfigure{%
    \includegraphics[width=.18\columnwidth]{figures/supplementary/000006011_gauss.png}
  }
  \subfigure{%
    \includegraphics[width=.18\columnwidth]{figures/supplementary/000006011_learnt.png}
  }\\[-2ex]
    \subfigure{%
    \includegraphics[width=.18\columnwidth]{figures/supplementary/000008553_given.jpg}
  }
  \subfigure{%
    \includegraphics[width=.18\columnwidth]{figures/supplementary/000008553_gt.png}
  }
  \subfigure{%
    \includegraphics[width=.18\columnwidth]{figures/supplementary/000008553_cnn.png}
  }
  \subfigure{%
    \includegraphics[width=.18\columnwidth]{figures/supplementary/000008553_gauss.png}
  }
  \subfigure{%
    \includegraphics[width=.18\columnwidth]{figures/supplementary/000008553_learnt.png}
  }\\[-2ex]
   \subfigure{%
    \includegraphics[width=.18\columnwidth]{figures/supplementary/000009188_given.jpg}
  }
  \subfigure{%
    \includegraphics[width=.18\columnwidth]{figures/supplementary/000009188_gt.png}
  }
  \subfigure{%
    \includegraphics[width=.18\columnwidth]{figures/supplementary/000009188_cnn.png}
  }
  \subfigure{%
    \includegraphics[width=.18\columnwidth]{figures/supplementary/000009188_gauss.png}
  }
  \subfigure{%
    \includegraphics[width=.18\columnwidth]{figures/supplementary/000009188_learnt.png}
  }\\[-2ex]
  \setcounter{subfigure}{0}
  \subfigure[Input]{%
    \includegraphics[width=.18\columnwidth]{figures/supplementary/000023570_given.jpg}
  }
  \subfigure[Ground Truth]{%
    \includegraphics[width=.18\columnwidth]{figures/supplementary/000023570_gt.png}
  }
  \subfigure[DeepLab]{%
    \includegraphics[width=.18\columnwidth]{figures/supplementary/000023570_cnn.png}
  }
  \subfigure[+GaussCRF]{%
    \includegraphics[width=.18\columnwidth]{figures/supplementary/000023570_gauss.png}
  }
  \subfigure[+LearnedCRF]{%
    \includegraphics[width=.18\columnwidth]{figures/supplementary/000023570_learnt.png}
  }
  \mycaption{Material Segmentation}{Example results of material segmentation.
  (c)~depicts the unary results before application of MF, (d)~after two steps of MF with Gaussian edge CRF potentials, (e)~after two steps of MF with learned edge CRF potentials.}
    \label{fig:material_visuals-app2}
\end{figure*}


\begin{table*}[h]
\tiny
  \centering
    \begin{tabular}{L{2.3cm} L{2.25cm} C{1.5cm} C{0.7cm} C{0.6cm} C{0.7cm} C{0.7cm} C{0.7cm} C{1.6cm} C{0.6cm} C{0.6cm} C{0.6cm}}
      \toprule
& & & & & \multicolumn{3}{c}{\textbf{Data Statistics}} & \multicolumn{4}{c}{\textbf{Training Protocol}} \\

\textbf{Experiment} & \textbf{Feature Types} & \textbf{Feature Scales} & \textbf{Filter Size} & \textbf{Filter Nbr.} & \textbf{Train}  & \textbf{Val.} & \textbf{Test} & \textbf{Loss Type} & \textbf{LR} & \textbf{Batch} & \textbf{Epochs} \\
      \midrule
      \multicolumn{2}{c}{\textbf{Single Bilateral Filter Applications}} & & & & & & & & & \\
      \textbf{2$\times$ Color Upsampling} & Position$_{1}$, Intensity (3D) & 0.13, 0.17 & 65 & 2 & 10581 & 1449 & 1456 & MSE & 1e-06 & 200 & 94.5\\
      \textbf{4$\times$ Color Upsampling} & Position$_{1}$, Intensity (3D) & 0.06, 0.17 & 65 & 2 & 10581 & 1449 & 1456 & MSE & 1e-06 & 200 & 94.5\\
      \textbf{8$\times$ Color Upsampling} & Position$_{1}$, Intensity (3D) & 0.03, 0.17 & 65 & 2 & 10581 & 1449 & 1456 & MSE & 1e-06 & 200 & 94.5\\
      \textbf{16$\times$ Color Upsampling} & Position$_{1}$, Intensity (3D) & 0.02, 0.17 & 65 & 2 & 10581 & 1449 & 1456 & MSE & 1e-06 & 200 & 94.5\\
      \textbf{Depth Upsampling} & Position$_{1}$, Color (5D) & 0.05, 0.02 & 665 & 2 & 795 & 100 & 654 & MSE & 1e-07 & 50 & 251.6\\
      \textbf{Mesh Denoising} & Isomap (4D) & 46.00 & 63 & 2 & 1000 & 200 & 500 & MSE & 100 & 10 & 100.0 \\
      \midrule
      \multicolumn{2}{c}{\textbf{DenseCRF Applications}} & & & & & & & & &\\
      \multicolumn{2}{l}{\textbf{Semantic Segmentation}} & & & & & & & & &\\
      \textbf{- 1step MF} & Position$_{1}$, Color (5D); Position$_{1}$ (2D) & 0.01, 0.34; 0.34  & 665; 19  & 2; 2 & 10581 & 1449 & 1456 & Logistic & 0.1 & 5 & 1.4 \\
      \textbf{- 2step MF} & Position$_{1}$, Color (5D); Position$_{1}$ (2D) & 0.01, 0.34; 0.34 & 665; 19 & 2; 2 & 10581 & 1449 & 1456 & Logistic & 0.1 & 5 & 1.4 \\
      \textbf{- \textit{loose} 2step MF} & Position$_{1}$, Color (5D); Position$_{1}$ (2D) & 0.01, 0.34; 0.34 & 665; 19 & 2; 2 &10581 & 1449 & 1456 & Logistic & 0.1 & 5 & +1.9  \\ \\
      \multicolumn{2}{l}{\textbf{Material Segmentation}} & & & & & & & & &\\
      \textbf{- 1step MF} & Position$_{2}$, Lab-Color (5D) & 5.00, 0.05, 0.30  & 665 & 2 & 928 & 150 & 1798 & Weighted Logistic & 1e-04 & 24 & 2.6 \\
      \textbf{- 2step MF} & Position$_{2}$, Lab-Color (5D) & 5.00, 0.05, 0.30 & 665 & 2 & 928 & 150 & 1798 & Weighted Logistic & 1e-04 & 12 & +0.7 \\
      \textbf{- \textit{loose} 2step MF} & Position$_{2}$, Lab-Color (5D) & 5.00, 0.05, 0.30 & 665 & 2 & 928 & 150 & 1798 & Weighted Logistic & 1e-04 & 12 & +0.2\\
      \midrule
      \multicolumn{2}{c}{\textbf{Neural Network Applications}} & & & & & & & & &\\
      \textbf{Tiles: CNN-9$\times$9} & - & - & 81 & 4 & 10000 & 1000 & 1000 & Logistic & 0.01 & 100 & 500.0 \\
      \textbf{Tiles: CNN-13$\times$13} & - & - & 169 & 6 & 10000 & 1000 & 1000 & Logistic & 0.01 & 100 & 500.0 \\
      \textbf{Tiles: CNN-17$\times$17} & - & - & 289 & 8 & 10000 & 1000 & 1000 & Logistic & 0.01 & 100 & 500.0 \\
      \textbf{Tiles: CNN-21$\times$21} & - & - & 441 & 10 & 10000 & 1000 & 1000 & Logistic & 0.01 & 100 & 500.0 \\
      \textbf{Tiles: BNN} & Position$_{1}$, Color (5D) & 0.05, 0.04 & 63 & 1 & 10000 & 1000 & 1000 & Logistic & 0.01 & 100 & 30.0 \\
      \textbf{LeNet} & - & - & 25 & 2 & 5490 & 1098 & 1647 & Logistic & 0.1 & 100 & 182.2 \\
      \textbf{Crop-LeNet} & - & - & 25 & 2 & 5490 & 1098 & 1647 & Logistic & 0.1 & 100 & 182.2 \\
      \textbf{BNN-LeNet} & Position$_{2}$ (2D) & 20.00 & 7 & 1 & 5490 & 1098 & 1647 & Logistic & 0.1 & 100 & 182.2 \\
      \textbf{DeepCNet} & - & - & 9 & 1 & 5490 & 1098 & 1647 & Logistic & 0.1 & 100 & 182.2 \\
      \textbf{Crop-DeepCNet} & - & - & 9 & 1 & 5490 & 1098 & 1647 & Logistic & 0.1 & 100 & 182.2 \\
      \textbf{BNN-DeepCNet} & Position$_{2}$ (2D) & 40.00  & 7 & 1 & 5490 & 1098 & 1647 & Logistic & 0.1 & 100 & 182.2 \\
      \bottomrule
      \\
    \end{tabular}
    \mycaption{Experiment Protocols} {Experiment protocols for the different experiments presented in this work. \textbf{Feature Types}:
    Feature spaces used for the bilateral convolutions. Position$_1$ corresponds to un-normalized pixel positions whereas Position$_2$ corresponds
    to pixel positions normalized to $[0,1]$ with respect to the given image. \textbf{Feature Scales}: Cross-validated scales for the features used.
     \textbf{Filter Size}: Number of elements in the filter that is being learned. \textbf{Filter Nbr.}: Half-width of the filter. \textbf{Train},
     \textbf{Val.} and \textbf{Test} corresponds to the number of train, validation and test images used in the experiment. \textbf{Loss Type}: Type
     of loss used for back-propagation. ``MSE'' corresponds to Euclidean mean squared error loss and ``Logistic'' corresponds to multinomial logistic
     loss. ``Weighted Logistic'' is the class-weighted multinomial logistic loss. We weighted the loss with inverse class probability for material
     segmentation task due to the small availability of training data with class imbalance. \textbf{LR}: Fixed learning rate used in stochastic gradient
     descent. \textbf{Batch}: Number of images used in one parameter update step. \textbf{Epochs}: Number of training epochs. In all the experiments,
     we used fixed momentum of 0.9 and weight decay of 0.0005 for stochastic gradient descent. ```Color Upsampling'' experiments in this Table corresponds
     to those performed on Pascal VOC12 dataset images. For all experiments using Pascal VOC12 images, we use extended
     training segmentation dataset available from~\cite{hariharan2011moredata}, and used standard validation and test splits
     from the main dataset~\cite{voc2012segmentation}.}
  \label{tbl:parameters}
\end{table*}

\clearpage

\section{Parameters and Additional Results for Video Propagation Networks}

In this Section, we present experiment protocols and additional qualitative results for experiments
on video object segmentation, semantic video segmentation and video color
propagation. Table~\ref{tbl:parameters_supp} shows the feature scales and other parameters used in different experiments.
Figures~\ref{fig:video_seg_pos_supp} show some qualitative results on video object segmentation
with some failure cases in Fig.~\ref{fig:video_seg_neg_supp}.
Figure~\ref{fig:semantic_visuals_supp} shows some qualitative results on semantic video segmentation and
Fig.~\ref{fig:color_visuals_supp} shows results on video color propagation.

\newcolumntype{L}[1]{>{\raggedright\let\newline\\\arraybackslash\hspace{0pt}}b{#1}}
\newcolumntype{C}[1]{>{\centering\let\newline\\\arraybackslash\hspace{0pt}}b{#1}}
\newcolumntype{R}[1]{>{\raggedleft\let\newline\\\arraybackslash\hspace{0pt}}b{#1}}

\begin{table*}[h]
\tiny
  \centering
    \begin{tabular}{L{3.0cm} L{2.4cm} L{2.8cm} L{2.8cm} C{0.5cm} C{1.0cm} L{1.2cm}}
      \toprule
\textbf{Experiment} & \textbf{Feature Type} & \textbf{Feature Scale-1, $\Lambda_a$} & \textbf{Feature Scale-2, $\Lambda_b$} & \textbf{$\alpha$} & \textbf{Input Frames} & \textbf{Loss Type} \\
      \midrule
      \textbf{Video Object Segmentation} & ($x,y,Y,Cb,Cr,t$) & (0.02,0.02,0.07,0.4,0.4,0.01) & (0.03,0.03,0.09,0.5,0.5,0.2) & 0.5 & 9 & Logistic\\
      \midrule
      \textbf{Semantic Video Segmentation} & & & & & \\
      \textbf{with CNN1~\cite{yu2015multi}-NoFlow} & ($x,y,R,G,B,t$) & (0.08,0.08,0.2,0.2,0.2,0.04) & (0.11,0.11,0.2,0.2,0.2,0.04) & 0.5 & 3 & Logistic \\
      \textbf{with CNN1~\cite{yu2015multi}-Flow} & ($x+u_x,y+u_y,R,G,B,t$) & (0.11,0.11,0.14,0.14,0.14,0.03) & (0.08,0.08,0.12,0.12,0.12,0.01) & 0.65 & 3 & Logistic\\
      \textbf{with CNN2~\cite{richter2016playing}-Flow} & ($x+u_x,y+u_y,R,G,B,t$) & (0.08,0.08,0.2,0.2,0.2,0.04) & (0.09,0.09,0.25,0.25,0.25,0.03) & 0.5 & 4 & Logistic\\
      \midrule
      \textbf{Video Color Propagation} & ($x,y,I,t$)  & (0.04,0.04,0.2,0.04) & No second kernel & 1 & 4 & MSE\\
      \bottomrule
      \\
    \end{tabular}
    \mycaption{Experiment Protocols} {Experiment protocols for the different experiments presented in this work. \textbf{Feature Types}:
    Feature spaces used for the bilateral convolutions, with position ($x,y$) and color
    ($R,G,B$ or $Y,Cb,Cr$) features $\in [0,255]$. $u_x$, $u_y$ denotes optical flow with respect
    to the present frame and $I$ denotes grayscale intensity.
    \textbf{Feature Scales ($\Lambda_a, \Lambda_b$)}: Cross-validated scales for the features used.
    \textbf{$\alpha$}: Exponential time decay for the input frames.
    \textbf{Input Frames}: Number of input frames for VPN.
    \textbf{Loss Type}: Type
     of loss used for back-propagation. ``MSE'' corresponds to Euclidean mean squared error loss and ``Logistic'' corresponds to multinomial logistic loss.}
  \label{tbl:parameters_supp}
\end{table*}

% \begin{figure}[th!]
% \begin{center}
%   \centerline{\includegraphics[width=\textwidth]{figures/video_seg_visuals_supp_small.pdf}}
%     \mycaption{Video Object Segmentation}
%     {Shown are the different frames in example videos with the corresponding
%     ground truth (GT) masks, predictions from BVS~\cite{marki2016bilateral},
%     OFL~\cite{tsaivideo}, VPN (VPN-Stage2) and VPN-DLab (VPN-DeepLab) models.}
%     \label{fig:video_seg_small_supp}
% \end{center}
% \vspace{-1.0cm}
% \end{figure}

\begin{figure}[th!]
\begin{center}
  \centerline{\includegraphics[width=0.7\textwidth]{figures/video_seg_visuals_supp_positive.pdf}}
    \mycaption{Video Object Segmentation}
    {Shown are the different frames in example videos with the corresponding
    ground truth (GT) masks, predictions from BVS~\cite{marki2016bilateral},
    OFL~\cite{tsaivideo}, VPN (VPN-Stage2) and VPN-DLab (VPN-DeepLab) models.}
    \label{fig:video_seg_pos_supp}
\end{center}
\vspace{-1.0cm}
\end{figure}

\begin{figure}[th!]
\begin{center}
  \centerline{\includegraphics[width=0.7\textwidth]{figures/video_seg_visuals_supp_negative.pdf}}
    \mycaption{Failure Cases for Video Object Segmentation}
    {Shown are the different frames in example videos with the corresponding
    ground truth (GT) masks, predictions from BVS~\cite{marki2016bilateral},
    OFL~\cite{tsaivideo}, VPN (VPN-Stage2) and VPN-DLab (VPN-DeepLab) models.}
    \label{fig:video_seg_neg_supp}
\end{center}
\vspace{-1.0cm}
\end{figure}

\begin{figure}[th!]
\begin{center}
  \centerline{\includegraphics[width=0.9\textwidth]{figures/supp_semantic_visual.pdf}}
    \mycaption{Semantic Video Segmentation}
    {Input video frames and the corresponding ground truth (GT)
    segmentation together with the predictions of CNN~\cite{yu2015multi} and with
    VPN-Flow.}
    \label{fig:semantic_visuals_supp}
\end{center}
\vspace{-0.7cm}
\end{figure}

\begin{figure}[th!]
\begin{center}
  \centerline{\includegraphics[width=\textwidth]{figures/colorization_visuals_supp.pdf}}
  \mycaption{Video Color Propagation}
  {Input grayscale video frames and corresponding ground-truth (GT) color images
  together with color predictions of Levin et al.~\cite{levin2004colorization} and VPN-Stage1 models.}
  \label{fig:color_visuals_supp}
\end{center}
\vspace{-0.7cm}
\end{figure}

\clearpage

\section{Additional Material for Bilateral Inception Networks}
\label{sec:binception-app}

In this section of the Appendix, we first discuss the use of approximate bilateral
filtering in BI modules (Sec.~\ref{sec:lattice}).
Later, we present some qualitative results using different models for the approach presented in
Chapter~\ref{chap:binception} (Sec.~\ref{sec:qualitative-app}).

\subsection{Approximate Bilateral Filtering}
\label{sec:lattice}

The bilateral inception module presented in Chapter~\ref{chap:binception} computes a matrix-vector
product between a Gaussian filter $K$ and a vector of activations $\bz_c$.
Bilateral filtering is an important operation and many algorithmic techniques have been
proposed to speed-up this operation~\cite{paris2006fast,adams2010fast,gastal2011domain}.
In the main paper we opted to implement what can be considered the
brute-force variant of explicitly constructing $K$ and then using BLAS to compute the
matrix-vector product. This resulted in a few millisecond operation.
The explicit way to compute is possible due to the
reduction to super-pixels, e.g., it would not work for DenseCRF variants
that operate on the full image resolution.

Here, we present experiments where we use the fast approximate bilateral filtering
algorithm of~\cite{adams2010fast}, which is also used in Chapter~\ref{chap:bnn}
for learning sparse high dimensional filters. This
choice allows for larger dimensions of matrix-vector multiplication. The reason for choosing
the explicit multiplication in Chapter~\ref{chap:binception} was that it was computationally faster.
For the small sizes of the involved matrices and vectors, the explicit computation is sufficient and we had no
GPU implementation of an approximate technique that matched this runtime. Also it
is conceptually easier and the gradient to the feature transformations ($\Lambda \mathbf{f}$) is
obtained using standard matrix calculus.

\subsubsection{Experiments}

We modified the existing segmentation architectures analogous to those in Chapter~\ref{chap:binception}.
The main difference is that, here, the inception modules use the lattice
approximation~\cite{adams2010fast} to compute the bilateral filtering.
Using the lattice approximation did not allow us to back-propagate through feature transformations ($\Lambda$)
and thus we used hand-specified feature scales as will be explained later.
Specifically, we take CNN architectures from the works
of~\cite{chen2014semantic,zheng2015conditional,bell2015minc} and insert the BI modules between
the spatial FC layers.
We use superpixels from~\cite{DollarICCV13edges}
for all the experiments with the lattice approximation. Experiments are
performed using Caffe neural network framework~\cite{jia2014caffe}.

\begin{table}
  \small
  \centering
  \begin{tabular}{p{5.5cm}>{\raggedright\arraybackslash}p{1.4cm}>{\centering\arraybackslash}p{2.2cm}}
    \toprule
		\textbf{Model} & \emph{IoU} & \emph{Runtime}(ms) \\
    \midrule

    %%%%%%%%%%%% Scores computed by us)%%%%%%%%%%%%
		\deeplablargefov & 68.9 & 145ms\\
    \midrule
    \bi{7}{2}-\bi{8}{10}& \textbf{73.8} & +600 \\
    \midrule
    \deeplablargefovcrf~\cite{chen2014semantic} & 72.7 & +830\\
    \deeplabmsclargefovcrf~\cite{chen2014semantic} & \textbf{73.6} & +880\\
    DeepLab-EdgeNet~\cite{chen2015semantic} & 71.7 & +30\\
    DeepLab-EdgeNet-CRF~\cite{chen2015semantic} & \textbf{73.6} & +860\\
  \bottomrule \\
  \end{tabular}
  \mycaption{Semantic Segmentation using the DeepLab model}
  {IoU scores on the Pascal VOC12 segmentation test dataset
  with different models and our modified inception model.
  Also shown are the corresponding runtimes in milliseconds. Runtimes
  also include superpixel computations (300 ms with Dollar superpixels~\cite{DollarICCV13edges})}
  \label{tab:largefovresults}
\end{table}

\paragraph{Semantic Segmentation}
The experiments in this section use the Pascal VOC12 segmentation dataset~\cite{voc2012segmentation} with 21 object classes and the images have a maximum resolution of 0.25 megapixels.
For all experiments on VOC12, we train using the extended training set of
10581 images collected by~\cite{hariharan2011moredata}.
We modified the \deeplab~network architecture of~\cite{chen2014semantic} and
the CRFasRNN architecture from~\cite{zheng2015conditional} which uses a CNN with
deconvolution layers followed by DenseCRF trained end-to-end.

\paragraph{DeepLab Model}\label{sec:deeplabmodel}
We experimented with the \bi{7}{2}-\bi{8}{10} inception model.
Results using the~\deeplab~model are summarized in Tab.~\ref{tab:largefovresults}.
Although we get similar improvements with inception modules as with the
explicit kernel computation, using lattice approximation is slower.

\begin{table}
  \small
  \centering
  \begin{tabular}{p{6.4cm}>{\raggedright\arraybackslash}p{1.8cm}>{\raggedright\arraybackslash}p{1.8cm}}
    \toprule
    \textbf{Model} & \emph{IoU (Val)} & \emph{IoU (Test)}\\
    \midrule
    %%%%%%%%%%%% Scores computed by us)%%%%%%%%%%%%
    CNN &  67.5 & - \\
    \deconv (CNN+Deconvolutions) & 69.8 & 72.0 \\
    \midrule
    \bi{3}{6}-\bi{4}{6}-\bi{7}{2}-\bi{8}{6}& 71.9 & - \\
    \bi{3}{6}-\bi{4}{6}-\bi{7}{2}-\bi{8}{6}-\gi{6}& 73.6 &  \href{http://host.robots.ox.ac.uk:8080/anonymous/VOTV5E.html}{\textbf{75.2}}\\
    \midrule
    \deconvcrf (CRF-RNN)~\cite{zheng2015conditional} & 73.0 & 74.7\\
    Context-CRF-RNN~\cite{yu2015multi} & ~~ - ~ & \textbf{75.3} \\
    \bottomrule \\
  \end{tabular}
  \mycaption{Semantic Segmentation using the CRFasRNN model}{IoU score corresponding to different models
  on Pascal VOC12 reduced validation / test segmentation dataset. The reduced validation set consists of 346 images
  as used in~\cite{zheng2015conditional} where we adapted the model from.}
  \label{tab:deconvresults-app}
\end{table}

\paragraph{CRFasRNN Model}\label{sec:deepinception}
We add BI modules after score-pool3, score-pool4, \fc{7} and \fc{8} $1\times1$ convolution layers
resulting in the \bi{3}{6}-\bi{4}{6}-\bi{7}{2}-\bi{8}{6}
model and also experimented with another variant where $BI_8$ is followed by another inception
module, G$(6)$, with 6 Gaussian kernels.
Note that here also we discarded both deconvolution and DenseCRF parts of the original model~\cite{zheng2015conditional}
and inserted the BI modules in the base CNN and found similar improvements compared to the inception modules with explicit
kernel computaion. See Tab.~\ref{tab:deconvresults-app} for results on the CRFasRNN model.

\paragraph{Material Segmentation}
Table~\ref{tab:mincresults-app} shows the results on the MINC dataset~\cite{bell2015minc}
obtained by modifying the AlexNet architecture with our inception modules. We observe
similar improvements as with explicit kernel construction.
For this model, we do not provide any learned setup due to very limited segment training
data. The weights to combine outputs in the bilateral inception layer are
found by validation on the validation set.

\begin{table}[t]
  \small
  \centering
  \begin{tabular}{p{3.5cm}>{\centering\arraybackslash}p{4.0cm}}
    \toprule
    \textbf{Model} & Class / Total accuracy\\
    \midrule

    %%%%%%%%%%%% Scores computed by us)%%%%%%%%%%%%
    AlexNet CNN & 55.3 / 58.9 \\
    \midrule
    \bi{7}{2}-\bi{8}{6}& 68.5 / 71.8 \\
    \bi{7}{2}-\bi{8}{6}-G$(6)$& 67.6 / 73.1 \\
    \midrule
    AlexNet-CRF & 65.5 / 71.0 \\
    \bottomrule \\
  \end{tabular}
  \mycaption{Material Segmentation using AlexNet}{Pixel accuracy of different models on
  the MINC material segmentation test dataset~\cite{bell2015minc}.}
  \label{tab:mincresults-app}
\end{table}

\paragraph{Scales of Bilateral Inception Modules}
\label{sec:scales}

Unlike the explicit kernel technique presented in the main text (Chapter~\ref{chap:binception}),
we didn't back-propagate through feature transformation ($\Lambda$)
using the approximate bilateral filter technique.
So, the feature scales are hand-specified and validated, which are as follows.
The optimal scale values for the \bi{7}{2}-\bi{8}{2} model are found by validation for the best performance which are
$\sigma_{xy}$ = (0.1, 0.1) for the spatial (XY) kernel and $\sigma_{rgbxy}$ = (0.1, 0.1, 0.1, 0.01, 0.01) for color and position (RGBXY)  kernel.
Next, as more kernels are added to \bi{8}{2}, we set scales to be $\alpha$*($\sigma_{xy}$, $\sigma_{rgbxy}$).
The value of $\alpha$ is chosen as  1, 0.5, 0.1, 0.05, 0.1, at uniform interval, for the \bi{8}{10} bilateral inception module.


\subsection{Qualitative Results}
\label{sec:qualitative-app}

In this section, we present more qualitative results obtained using the BI module with explicit
kernel computation technique presented in Chapter~\ref{chap:binception}. Results on the Pascal VOC12
dataset~\cite{voc2012segmentation} using the DeepLab-LargeFOV model are shown in Fig.~\ref{fig:semantic_visuals-app},
followed by the results on MINC dataset~\cite{bell2015minc}
in Fig.~\ref{fig:material_visuals-app} and on
Cityscapes dataset~\cite{Cordts2015Cvprw} in Fig.~\ref{fig:street_visuals-app}.


\definecolor{voc_1}{RGB}{0, 0, 0}
\definecolor{voc_2}{RGB}{128, 0, 0}
\definecolor{voc_3}{RGB}{0, 128, 0}
\definecolor{voc_4}{RGB}{128, 128, 0}
\definecolor{voc_5}{RGB}{0, 0, 128}
\definecolor{voc_6}{RGB}{128, 0, 128}
\definecolor{voc_7}{RGB}{0, 128, 128}
\definecolor{voc_8}{RGB}{128, 128, 128}
\definecolor{voc_9}{RGB}{64, 0, 0}
\definecolor{voc_10}{RGB}{192, 0, 0}
\definecolor{voc_11}{RGB}{64, 128, 0}
\definecolor{voc_12}{RGB}{192, 128, 0}
\definecolor{voc_13}{RGB}{64, 0, 128}
\definecolor{voc_14}{RGB}{192, 0, 128}
\definecolor{voc_15}{RGB}{64, 128, 128}
\definecolor{voc_16}{RGB}{192, 128, 128}
\definecolor{voc_17}{RGB}{0, 64, 0}
\definecolor{voc_18}{RGB}{128, 64, 0}
\definecolor{voc_19}{RGB}{0, 192, 0}
\definecolor{voc_20}{RGB}{128, 192, 0}
\definecolor{voc_21}{RGB}{0, 64, 128}
\definecolor{voc_22}{RGB}{128, 64, 128}

\begin{figure*}[!ht]
  \small
  \centering
  \fcolorbox{white}{voc_1}{\rule{0pt}{4pt}\rule{4pt}{0pt}} Background~~
  \fcolorbox{white}{voc_2}{\rule{0pt}{4pt}\rule{4pt}{0pt}} Aeroplane~~
  \fcolorbox{white}{voc_3}{\rule{0pt}{4pt}\rule{4pt}{0pt}} Bicycle~~
  \fcolorbox{white}{voc_4}{\rule{0pt}{4pt}\rule{4pt}{0pt}} Bird~~
  \fcolorbox{white}{voc_5}{\rule{0pt}{4pt}\rule{4pt}{0pt}} Boat~~
  \fcolorbox{white}{voc_6}{\rule{0pt}{4pt}\rule{4pt}{0pt}} Bottle~~
  \fcolorbox{white}{voc_7}{\rule{0pt}{4pt}\rule{4pt}{0pt}} Bus~~
  \fcolorbox{white}{voc_8}{\rule{0pt}{4pt}\rule{4pt}{0pt}} Car~~\\
  \fcolorbox{white}{voc_9}{\rule{0pt}{4pt}\rule{4pt}{0pt}} Cat~~
  \fcolorbox{white}{voc_10}{\rule{0pt}{4pt}\rule{4pt}{0pt}} Chair~~
  \fcolorbox{white}{voc_11}{\rule{0pt}{4pt}\rule{4pt}{0pt}} Cow~~
  \fcolorbox{white}{voc_12}{\rule{0pt}{4pt}\rule{4pt}{0pt}} Dining Table~~
  \fcolorbox{white}{voc_13}{\rule{0pt}{4pt}\rule{4pt}{0pt}} Dog~~
  \fcolorbox{white}{voc_14}{\rule{0pt}{4pt}\rule{4pt}{0pt}} Horse~~
  \fcolorbox{white}{voc_15}{\rule{0pt}{4pt}\rule{4pt}{0pt}} Motorbike~~
  \fcolorbox{white}{voc_16}{\rule{0pt}{4pt}\rule{4pt}{0pt}} Person~~\\
  \fcolorbox{white}{voc_17}{\rule{0pt}{4pt}\rule{4pt}{0pt}} Potted Plant~~
  \fcolorbox{white}{voc_18}{\rule{0pt}{4pt}\rule{4pt}{0pt}} Sheep~~
  \fcolorbox{white}{voc_19}{\rule{0pt}{4pt}\rule{4pt}{0pt}} Sofa~~
  \fcolorbox{white}{voc_20}{\rule{0pt}{4pt}\rule{4pt}{0pt}} Train~~
  \fcolorbox{white}{voc_21}{\rule{0pt}{4pt}\rule{4pt}{0pt}} TV monitor~~\\


  \subfigure{%
    \includegraphics[width=.15\columnwidth]{figures/supplementary/2008_001308_given.png}
  }
  \subfigure{%
    \includegraphics[width=.15\columnwidth]{figures/supplementary/2008_001308_sp.png}
  }
  \subfigure{%
    \includegraphics[width=.15\columnwidth]{figures/supplementary/2008_001308_gt.png}
  }
  \subfigure{%
    \includegraphics[width=.15\columnwidth]{figures/supplementary/2008_001308_cnn.png}
  }
  \subfigure{%
    \includegraphics[width=.15\columnwidth]{figures/supplementary/2008_001308_crf.png}
  }
  \subfigure{%
    \includegraphics[width=.15\columnwidth]{figures/supplementary/2008_001308_ours.png}
  }\\[-2ex]


  \subfigure{%
    \includegraphics[width=.15\columnwidth]{figures/supplementary/2008_001821_given.png}
  }
  \subfigure{%
    \includegraphics[width=.15\columnwidth]{figures/supplementary/2008_001821_sp.png}
  }
  \subfigure{%
    \includegraphics[width=.15\columnwidth]{figures/supplementary/2008_001821_gt.png}
  }
  \subfigure{%
    \includegraphics[width=.15\columnwidth]{figures/supplementary/2008_001821_cnn.png}
  }
  \subfigure{%
    \includegraphics[width=.15\columnwidth]{figures/supplementary/2008_001821_crf.png}
  }
  \subfigure{%
    \includegraphics[width=.15\columnwidth]{figures/supplementary/2008_001821_ours.png}
  }\\[-2ex]



  \subfigure{%
    \includegraphics[width=.15\columnwidth]{figures/supplementary/2008_004612_given.png}
  }
  \subfigure{%
    \includegraphics[width=.15\columnwidth]{figures/supplementary/2008_004612_sp.png}
  }
  \subfigure{%
    \includegraphics[width=.15\columnwidth]{figures/supplementary/2008_004612_gt.png}
  }
  \subfigure{%
    \includegraphics[width=.15\columnwidth]{figures/supplementary/2008_004612_cnn.png}
  }
  \subfigure{%
    \includegraphics[width=.15\columnwidth]{figures/supplementary/2008_004612_crf.png}
  }
  \subfigure{%
    \includegraphics[width=.15\columnwidth]{figures/supplementary/2008_004612_ours.png}
  }\\[-2ex]


  \subfigure{%
    \includegraphics[width=.15\columnwidth]{figures/supplementary/2009_001008_given.png}
  }
  \subfigure{%
    \includegraphics[width=.15\columnwidth]{figures/supplementary/2009_001008_sp.png}
  }
  \subfigure{%
    \includegraphics[width=.15\columnwidth]{figures/supplementary/2009_001008_gt.png}
  }
  \subfigure{%
    \includegraphics[width=.15\columnwidth]{figures/supplementary/2009_001008_cnn.png}
  }
  \subfigure{%
    \includegraphics[width=.15\columnwidth]{figures/supplementary/2009_001008_crf.png}
  }
  \subfigure{%
    \includegraphics[width=.15\columnwidth]{figures/supplementary/2009_001008_ours.png}
  }\\[-2ex]




  \subfigure{%
    \includegraphics[width=.15\columnwidth]{figures/supplementary/2009_004497_given.png}
  }
  \subfigure{%
    \includegraphics[width=.15\columnwidth]{figures/supplementary/2009_004497_sp.png}
  }
  \subfigure{%
    \includegraphics[width=.15\columnwidth]{figures/supplementary/2009_004497_gt.png}
  }
  \subfigure{%
    \includegraphics[width=.15\columnwidth]{figures/supplementary/2009_004497_cnn.png}
  }
  \subfigure{%
    \includegraphics[width=.15\columnwidth]{figures/supplementary/2009_004497_crf.png}
  }
  \subfigure{%
    \includegraphics[width=.15\columnwidth]{figures/supplementary/2009_004497_ours.png}
  }\\[-2ex]



  \setcounter{subfigure}{0}
  \subfigure[\scriptsize Input]{%
    \includegraphics[width=.15\columnwidth]{figures/supplementary/2010_001327_given.png}
  }
  \subfigure[\scriptsize Superpixels]{%
    \includegraphics[width=.15\columnwidth]{figures/supplementary/2010_001327_sp.png}
  }
  \subfigure[\scriptsize GT]{%
    \includegraphics[width=.15\columnwidth]{figures/supplementary/2010_001327_gt.png}
  }
  \subfigure[\scriptsize Deeplab]{%
    \includegraphics[width=.15\columnwidth]{figures/supplementary/2010_001327_cnn.png}
  }
  \subfigure[\scriptsize +DenseCRF]{%
    \includegraphics[width=.15\columnwidth]{figures/supplementary/2010_001327_crf.png}
  }
  \subfigure[\scriptsize Using BI]{%
    \includegraphics[width=.15\columnwidth]{figures/supplementary/2010_001327_ours.png}
  }
  \mycaption{Semantic Segmentation}{Example results of semantic segmentation
  on the Pascal VOC12 dataset.
  (d)~depicts the DeepLab CNN result, (e)~CNN + 10 steps of mean-field inference,
  (f~result obtained with bilateral inception (BI) modules (\bi{6}{2}+\bi{7}{6}) between \fc~layers.}
  \label{fig:semantic_visuals-app}
\end{figure*}


\definecolor{minc_1}{HTML}{771111}
\definecolor{minc_2}{HTML}{CAC690}
\definecolor{minc_3}{HTML}{EEEEEE}
\definecolor{minc_4}{HTML}{7C8FA6}
\definecolor{minc_5}{HTML}{597D31}
\definecolor{minc_6}{HTML}{104410}
\definecolor{minc_7}{HTML}{BB819C}
\definecolor{minc_8}{HTML}{D0CE48}
\definecolor{minc_9}{HTML}{622745}
\definecolor{minc_10}{HTML}{666666}
\definecolor{minc_11}{HTML}{D54A31}
\definecolor{minc_12}{HTML}{101044}
\definecolor{minc_13}{HTML}{444126}
\definecolor{minc_14}{HTML}{75D646}
\definecolor{minc_15}{HTML}{DD4348}
\definecolor{minc_16}{HTML}{5C8577}
\definecolor{minc_17}{HTML}{C78472}
\definecolor{minc_18}{HTML}{75D6D0}
\definecolor{minc_19}{HTML}{5B4586}
\definecolor{minc_20}{HTML}{C04393}
\definecolor{minc_21}{HTML}{D69948}
\definecolor{minc_22}{HTML}{7370D8}
\definecolor{minc_23}{HTML}{7A3622}
\definecolor{minc_24}{HTML}{000000}

\begin{figure*}[!ht]
  \small % scriptsize
  \centering
  \fcolorbox{white}{minc_1}{\rule{0pt}{4pt}\rule{4pt}{0pt}} Brick~~
  \fcolorbox{white}{minc_2}{\rule{0pt}{4pt}\rule{4pt}{0pt}} Carpet~~
  \fcolorbox{white}{minc_3}{\rule{0pt}{4pt}\rule{4pt}{0pt}} Ceramic~~
  \fcolorbox{white}{minc_4}{\rule{0pt}{4pt}\rule{4pt}{0pt}} Fabric~~
  \fcolorbox{white}{minc_5}{\rule{0pt}{4pt}\rule{4pt}{0pt}} Foliage~~
  \fcolorbox{white}{minc_6}{\rule{0pt}{4pt}\rule{4pt}{0pt}} Food~~
  \fcolorbox{white}{minc_7}{\rule{0pt}{4pt}\rule{4pt}{0pt}} Glass~~
  \fcolorbox{white}{minc_8}{\rule{0pt}{4pt}\rule{4pt}{0pt}} Hair~~\\
  \fcolorbox{white}{minc_9}{\rule{0pt}{4pt}\rule{4pt}{0pt}} Leather~~
  \fcolorbox{white}{minc_10}{\rule{0pt}{4pt}\rule{4pt}{0pt}} Metal~~
  \fcolorbox{white}{minc_11}{\rule{0pt}{4pt}\rule{4pt}{0pt}} Mirror~~
  \fcolorbox{white}{minc_12}{\rule{0pt}{4pt}\rule{4pt}{0pt}} Other~~
  \fcolorbox{white}{minc_13}{\rule{0pt}{4pt}\rule{4pt}{0pt}} Painted~~
  \fcolorbox{white}{minc_14}{\rule{0pt}{4pt}\rule{4pt}{0pt}} Paper~~
  \fcolorbox{white}{minc_15}{\rule{0pt}{4pt}\rule{4pt}{0pt}} Plastic~~\\
  \fcolorbox{white}{minc_16}{\rule{0pt}{4pt}\rule{4pt}{0pt}} Polished Stone~~
  \fcolorbox{white}{minc_17}{\rule{0pt}{4pt}\rule{4pt}{0pt}} Skin~~
  \fcolorbox{white}{minc_18}{\rule{0pt}{4pt}\rule{4pt}{0pt}} Sky~~
  \fcolorbox{white}{minc_19}{\rule{0pt}{4pt}\rule{4pt}{0pt}} Stone~~
  \fcolorbox{white}{minc_20}{\rule{0pt}{4pt}\rule{4pt}{0pt}} Tile~~
  \fcolorbox{white}{minc_21}{\rule{0pt}{4pt}\rule{4pt}{0pt}} Wallpaper~~
  \fcolorbox{white}{minc_22}{\rule{0pt}{4pt}\rule{4pt}{0pt}} Water~~
  \fcolorbox{white}{minc_23}{\rule{0pt}{4pt}\rule{4pt}{0pt}} Wood~~\\
  \subfigure{%
    \includegraphics[width=.15\columnwidth]{figures/supplementary/000008468_given.png}
  }
  \subfigure{%
    \includegraphics[width=.15\columnwidth]{figures/supplementary/000008468_sp.png}
  }
  \subfigure{%
    \includegraphics[width=.15\columnwidth]{figures/supplementary/000008468_gt.png}
  }
  \subfigure{%
    \includegraphics[width=.15\columnwidth]{figures/supplementary/000008468_cnn.png}
  }
  \subfigure{%
    \includegraphics[width=.15\columnwidth]{figures/supplementary/000008468_crf.png}
  }
  \subfigure{%
    \includegraphics[width=.15\columnwidth]{figures/supplementary/000008468_ours.png}
  }\\[-2ex]

  \subfigure{%
    \includegraphics[width=.15\columnwidth]{figures/supplementary/000009053_given.png}
  }
  \subfigure{%
    \includegraphics[width=.15\columnwidth]{figures/supplementary/000009053_sp.png}
  }
  \subfigure{%
    \includegraphics[width=.15\columnwidth]{figures/supplementary/000009053_gt.png}
  }
  \subfigure{%
    \includegraphics[width=.15\columnwidth]{figures/supplementary/000009053_cnn.png}
  }
  \subfigure{%
    \includegraphics[width=.15\columnwidth]{figures/supplementary/000009053_crf.png}
  }
  \subfigure{%
    \includegraphics[width=.15\columnwidth]{figures/supplementary/000009053_ours.png}
  }\\[-2ex]




  \subfigure{%
    \includegraphics[width=.15\columnwidth]{figures/supplementary/000014977_given.png}
  }
  \subfigure{%
    \includegraphics[width=.15\columnwidth]{figures/supplementary/000014977_sp.png}
  }
  \subfigure{%
    \includegraphics[width=.15\columnwidth]{figures/supplementary/000014977_gt.png}
  }
  \subfigure{%
    \includegraphics[width=.15\columnwidth]{figures/supplementary/000014977_cnn.png}
  }
  \subfigure{%
    \includegraphics[width=.15\columnwidth]{figures/supplementary/000014977_crf.png}
  }
  \subfigure{%
    \includegraphics[width=.15\columnwidth]{figures/supplementary/000014977_ours.png}
  }\\[-2ex]


  \subfigure{%
    \includegraphics[width=.15\columnwidth]{figures/supplementary/000022922_given.png}
  }
  \subfigure{%
    \includegraphics[width=.15\columnwidth]{figures/supplementary/000022922_sp.png}
  }
  \subfigure{%
    \includegraphics[width=.15\columnwidth]{figures/supplementary/000022922_gt.png}
  }
  \subfigure{%
    \includegraphics[width=.15\columnwidth]{figures/supplementary/000022922_cnn.png}
  }
  \subfigure{%
    \includegraphics[width=.15\columnwidth]{figures/supplementary/000022922_crf.png}
  }
  \subfigure{%
    \includegraphics[width=.15\columnwidth]{figures/supplementary/000022922_ours.png}
  }\\[-2ex]


  \subfigure{%
    \includegraphics[width=.15\columnwidth]{figures/supplementary/000025711_given.png}
  }
  \subfigure{%
    \includegraphics[width=.15\columnwidth]{figures/supplementary/000025711_sp.png}
  }
  \subfigure{%
    \includegraphics[width=.15\columnwidth]{figures/supplementary/000025711_gt.png}
  }
  \subfigure{%
    \includegraphics[width=.15\columnwidth]{figures/supplementary/000025711_cnn.png}
  }
  \subfigure{%
    \includegraphics[width=.15\columnwidth]{figures/supplementary/000025711_crf.png}
  }
  \subfigure{%
    \includegraphics[width=.15\columnwidth]{figures/supplementary/000025711_ours.png}
  }\\[-2ex]


  \subfigure{%
    \includegraphics[width=.15\columnwidth]{figures/supplementary/000034473_given.png}
  }
  \subfigure{%
    \includegraphics[width=.15\columnwidth]{figures/supplementary/000034473_sp.png}
  }
  \subfigure{%
    \includegraphics[width=.15\columnwidth]{figures/supplementary/000034473_gt.png}
  }
  \subfigure{%
    \includegraphics[width=.15\columnwidth]{figures/supplementary/000034473_cnn.png}
  }
  \subfigure{%
    \includegraphics[width=.15\columnwidth]{figures/supplementary/000034473_crf.png}
  }
  \subfigure{%
    \includegraphics[width=.15\columnwidth]{figures/supplementary/000034473_ours.png}
  }\\[-2ex]


  \subfigure{%
    \includegraphics[width=.15\columnwidth]{figures/supplementary/000035463_given.png}
  }
  \subfigure{%
    \includegraphics[width=.15\columnwidth]{figures/supplementary/000035463_sp.png}
  }
  \subfigure{%
    \includegraphics[width=.15\columnwidth]{figures/supplementary/000035463_gt.png}
  }
  \subfigure{%
    \includegraphics[width=.15\columnwidth]{figures/supplementary/000035463_cnn.png}
  }
  \subfigure{%
    \includegraphics[width=.15\columnwidth]{figures/supplementary/000035463_crf.png}
  }
  \subfigure{%
    \includegraphics[width=.15\columnwidth]{figures/supplementary/000035463_ours.png}
  }\\[-2ex]


  \setcounter{subfigure}{0}
  \subfigure[\scriptsize Input]{%
    \includegraphics[width=.15\columnwidth]{figures/supplementary/000035993_given.png}
  }
  \subfigure[\scriptsize Superpixels]{%
    \includegraphics[width=.15\columnwidth]{figures/supplementary/000035993_sp.png}
  }
  \subfigure[\scriptsize GT]{%
    \includegraphics[width=.15\columnwidth]{figures/supplementary/000035993_gt.png}
  }
  \subfigure[\scriptsize AlexNet]{%
    \includegraphics[width=.15\columnwidth]{figures/supplementary/000035993_cnn.png}
  }
  \subfigure[\scriptsize +DenseCRF]{%
    \includegraphics[width=.15\columnwidth]{figures/supplementary/000035993_crf.png}
  }
  \subfigure[\scriptsize Using BI]{%
    \includegraphics[width=.15\columnwidth]{figures/supplementary/000035993_ours.png}
  }
  \mycaption{Material Segmentation}{Example results of material segmentation.
  (d)~depicts the AlexNet CNN result, (e)~CNN + 10 steps of mean-field inference,
  (f)~result obtained with bilateral inception (BI) modules (\bi{7}{2}+\bi{8}{6}) between
  \fc~layers.}
\label{fig:material_visuals-app}
\end{figure*}


\definecolor{city_1}{RGB}{128, 64, 128}
\definecolor{city_2}{RGB}{244, 35, 232}
\definecolor{city_3}{RGB}{70, 70, 70}
\definecolor{city_4}{RGB}{102, 102, 156}
\definecolor{city_5}{RGB}{190, 153, 153}
\definecolor{city_6}{RGB}{153, 153, 153}
\definecolor{city_7}{RGB}{250, 170, 30}
\definecolor{city_8}{RGB}{220, 220, 0}
\definecolor{city_9}{RGB}{107, 142, 35}
\definecolor{city_10}{RGB}{152, 251, 152}
\definecolor{city_11}{RGB}{70, 130, 180}
\definecolor{city_12}{RGB}{220, 20, 60}
\definecolor{city_13}{RGB}{255, 0, 0}
\definecolor{city_14}{RGB}{0, 0, 142}
\definecolor{city_15}{RGB}{0, 0, 70}
\definecolor{city_16}{RGB}{0, 60, 100}
\definecolor{city_17}{RGB}{0, 80, 100}
\definecolor{city_18}{RGB}{0, 0, 230}
\definecolor{city_19}{RGB}{119, 11, 32}
\begin{figure*}[!ht]
  \small % scriptsize
  \centering


  \subfigure{%
    \includegraphics[width=.18\columnwidth]{figures/supplementary/frankfurt00000_016005_given.png}
  }
  \subfigure{%
    \includegraphics[width=.18\columnwidth]{figures/supplementary/frankfurt00000_016005_sp.png}
  }
  \subfigure{%
    \includegraphics[width=.18\columnwidth]{figures/supplementary/frankfurt00000_016005_gt.png}
  }
  \subfigure{%
    \includegraphics[width=.18\columnwidth]{figures/supplementary/frankfurt00000_016005_cnn.png}
  }
  \subfigure{%
    \includegraphics[width=.18\columnwidth]{figures/supplementary/frankfurt00000_016005_ours.png}
  }\\[-2ex]

  \subfigure{%
    \includegraphics[width=.18\columnwidth]{figures/supplementary/frankfurt00000_004617_given.png}
  }
  \subfigure{%
    \includegraphics[width=.18\columnwidth]{figures/supplementary/frankfurt00000_004617_sp.png}
  }
  \subfigure{%
    \includegraphics[width=.18\columnwidth]{figures/supplementary/frankfurt00000_004617_gt.png}
  }
  \subfigure{%
    \includegraphics[width=.18\columnwidth]{figures/supplementary/frankfurt00000_004617_cnn.png}
  }
  \subfigure{%
    \includegraphics[width=.18\columnwidth]{figures/supplementary/frankfurt00000_004617_ours.png}
  }\\[-2ex]

  \subfigure{%
    \includegraphics[width=.18\columnwidth]{figures/supplementary/frankfurt00000_020880_given.png}
  }
  \subfigure{%
    \includegraphics[width=.18\columnwidth]{figures/supplementary/frankfurt00000_020880_sp.png}
  }
  \subfigure{%
    \includegraphics[width=.18\columnwidth]{figures/supplementary/frankfurt00000_020880_gt.png}
  }
  \subfigure{%
    \includegraphics[width=.18\columnwidth]{figures/supplementary/frankfurt00000_020880_cnn.png}
  }
  \subfigure{%
    \includegraphics[width=.18\columnwidth]{figures/supplementary/frankfurt00000_020880_ours.png}
  }\\[-2ex]



  \subfigure{%
    \includegraphics[width=.18\columnwidth]{figures/supplementary/frankfurt00001_007285_given.png}
  }
  \subfigure{%
    \includegraphics[width=.18\columnwidth]{figures/supplementary/frankfurt00001_007285_sp.png}
  }
  \subfigure{%
    \includegraphics[width=.18\columnwidth]{figures/supplementary/frankfurt00001_007285_gt.png}
  }
  \subfigure{%
    \includegraphics[width=.18\columnwidth]{figures/supplementary/frankfurt00001_007285_cnn.png}
  }
  \subfigure{%
    \includegraphics[width=.18\columnwidth]{figures/supplementary/frankfurt00001_007285_ours.png}
  }\\[-2ex]


  \subfigure{%
    \includegraphics[width=.18\columnwidth]{figures/supplementary/frankfurt00001_059789_given.png}
  }
  \subfigure{%
    \includegraphics[width=.18\columnwidth]{figures/supplementary/frankfurt00001_059789_sp.png}
  }
  \subfigure{%
    \includegraphics[width=.18\columnwidth]{figures/supplementary/frankfurt00001_059789_gt.png}
  }
  \subfigure{%
    \includegraphics[width=.18\columnwidth]{figures/supplementary/frankfurt00001_059789_cnn.png}
  }
  \subfigure{%
    \includegraphics[width=.18\columnwidth]{figures/supplementary/frankfurt00001_059789_ours.png}
  }\\[-2ex]


  \subfigure{%
    \includegraphics[width=.18\columnwidth]{figures/supplementary/frankfurt00001_068208_given.png}
  }
  \subfigure{%
    \includegraphics[width=.18\columnwidth]{figures/supplementary/frankfurt00001_068208_sp.png}
  }
  \subfigure{%
    \includegraphics[width=.18\columnwidth]{figures/supplementary/frankfurt00001_068208_gt.png}
  }
  \subfigure{%
    \includegraphics[width=.18\columnwidth]{figures/supplementary/frankfurt00001_068208_cnn.png}
  }
  \subfigure{%
    \includegraphics[width=.18\columnwidth]{figures/supplementary/frankfurt00001_068208_ours.png}
  }\\[-2ex]

  \subfigure{%
    \includegraphics[width=.18\columnwidth]{figures/supplementary/frankfurt00001_082466_given.png}
  }
  \subfigure{%
    \includegraphics[width=.18\columnwidth]{figures/supplementary/frankfurt00001_082466_sp.png}
  }
  \subfigure{%
    \includegraphics[width=.18\columnwidth]{figures/supplementary/frankfurt00001_082466_gt.png}
  }
  \subfigure{%
    \includegraphics[width=.18\columnwidth]{figures/supplementary/frankfurt00001_082466_cnn.png}
  }
  \subfigure{%
    \includegraphics[width=.18\columnwidth]{figures/supplementary/frankfurt00001_082466_ours.png}
  }\\[-2ex]

  \subfigure{%
    \includegraphics[width=.18\columnwidth]{figures/supplementary/lindau00033_000019_given.png}
  }
  \subfigure{%
    \includegraphics[width=.18\columnwidth]{figures/supplementary/lindau00033_000019_sp.png}
  }
  \subfigure{%
    \includegraphics[width=.18\columnwidth]{figures/supplementary/lindau00033_000019_gt.png}
  }
  \subfigure{%
    \includegraphics[width=.18\columnwidth]{figures/supplementary/lindau00033_000019_cnn.png}
  }
  \subfigure{%
    \includegraphics[width=.18\columnwidth]{figures/supplementary/lindau00033_000019_ours.png}
  }\\[-2ex]

  \subfigure{%
    \includegraphics[width=.18\columnwidth]{figures/supplementary/lindau00052_000019_given.png}
  }
  \subfigure{%
    \includegraphics[width=.18\columnwidth]{figures/supplementary/lindau00052_000019_sp.png}
  }
  \subfigure{%
    \includegraphics[width=.18\columnwidth]{figures/supplementary/lindau00052_000019_gt.png}
  }
  \subfigure{%
    \includegraphics[width=.18\columnwidth]{figures/supplementary/lindau00052_000019_cnn.png}
  }
  \subfigure{%
    \includegraphics[width=.18\columnwidth]{figures/supplementary/lindau00052_000019_ours.png}
  }\\[-2ex]




  \subfigure{%
    \includegraphics[width=.18\columnwidth]{figures/supplementary/lindau00027_000019_given.png}
  }
  \subfigure{%
    \includegraphics[width=.18\columnwidth]{figures/supplementary/lindau00027_000019_sp.png}
  }
  \subfigure{%
    \includegraphics[width=.18\columnwidth]{figures/supplementary/lindau00027_000019_gt.png}
  }
  \subfigure{%
    \includegraphics[width=.18\columnwidth]{figures/supplementary/lindau00027_000019_cnn.png}
  }
  \subfigure{%
    \includegraphics[width=.18\columnwidth]{figures/supplementary/lindau00027_000019_ours.png}
  }\\[-2ex]



  \setcounter{subfigure}{0}
  \subfigure[\scriptsize Input]{%
    \includegraphics[width=.18\columnwidth]{figures/supplementary/lindau00029_000019_given.png}
  }
  \subfigure[\scriptsize Superpixels]{%
    \includegraphics[width=.18\columnwidth]{figures/supplementary/lindau00029_000019_sp.png}
  }
  \subfigure[\scriptsize GT]{%
    \includegraphics[width=.18\columnwidth]{figures/supplementary/lindau00029_000019_gt.png}
  }
  \subfigure[\scriptsize Deeplab]{%
    \includegraphics[width=.18\columnwidth]{figures/supplementary/lindau00029_000019_cnn.png}
  }
  \subfigure[\scriptsize Using BI]{%
    \includegraphics[width=.18\columnwidth]{figures/supplementary/lindau00029_000019_ours.png}
  }%\\[-2ex]

  \mycaption{Street Scene Segmentation}{Example results of street scene segmentation.
  (d)~depicts the DeepLab results, (e)~result obtained by adding bilateral inception (BI) modules (\bi{6}{2}+\bi{7}{6}) between \fc~layers.}
\label{fig:street_visuals-app}
\end{figure*}


\section{Another Example: Learning a Co-Variance Matrix}\label{sec:covariance}

We demonstrate the benefits of using OMD over GD in another simple illustrative example. In this case, the example is does not boil down to a bi-linear game and therefore, the simulation results portray that the theoretical results we provided for bi-linear games, carry over qualitatively beyond the linear case.

Consider the case where the data distribution is a mean zero multi-variate normal with an unknown co-variance matrix, i.e., $x \sim N(0, \Sigma)$. We will consider the case where the discriminator is the set of all quadratic functions:
\begin{equation}
D_W(x) = \sum_{ij} W_{ij} x_i x_j = x^T W x 
\end{equation}
The generator is a linear function of the random input noise $z\sim N(0, I)$, of the form:
\begin{equation}
G_V(z) = V z
\end{equation}
The parameters $W$ and $V$ are both $d\times d$ matrices. The WGAN game loss associated with these functions is then:
\begin{equation}
L(V, W)= \mathbb{E}_{x\sim N(0, \Sigma)}\left[ x^T W x \right] - \mathbb{E}_{z\sim N(0,I)}\left[z^T V^T W V z \right] 
\end{equation}
Expanding the latter we get:
\begin{align*}
L(V, W)=& \mathbb{E}_{x\sim N(0, \Sigma)}\left[ \sum_{ij} W_{ij} x_i x_j \right] - \mathbb{E}_{z\sim N(0,I)}\left[ \sum_{ij} W_{ij} \sum_{k} V_{ik} z_k  \sum_{m} V_{jm} z_m\right] \\
=&
\mathbb{E}_{x\sim N(0, \Sigma)}\left[ \sum_{ij} W_{ij} x_i x_j \right] - \mathbb{E}_{z\sim N(0,I)}\left[ \sum_{ijkm} W_{ij} V_{ik} V_{jm} z_k z_m \right]\\
=& \sum_{ij} W_{ij} \mathbb{E}_{x\sim N(0, \Sigma)}\left[ x_i x_j \right] - \sum_{ijkm} W_{ij} V_{ik} V_{jm} \mathbb{E}_{z\sim N(0,I)}\left[ z_k z_m \right]\\
=& \sum_{ij} W_{ij} \Sigma_{ij} - \sum_{ijkm} W_{ij} V_{ik} V_{jm} 1\{k=m\}\\
=& \sum_{ij} W_{ij} \Sigma_{ij} - \sum_{ijk} W_{ij} V_{ik} V_{jk}\\
=& \sum_{ij} W_{ij} \left(\Sigma_{ij} - \sum_{k} V_{ik} V_{jk}\right)
\end{align*}
Given that the covariance matrix is symmetric positive definite, we can write it as $\Sigma = U U^T$. Then the loss simplifies to:
\begin{align}
L(V, W) = \sum_{ij} W_{ij} \left(\Sigma_{ij} - \sum_{k} V_{ik} V_{jk}\right) =& \sum_{ijk} W_{ij} \left(U_{ik} U_{jk} -  V_{ik} V_{jk}\right)
\end{align}
The equilibrium of this game is for the generator to choose $V_{ik} = U_{ik}$ for all $i,k$, and for the discriminator to pick $W_{ij}=0$. For instance, in the case of a single dimension we have $L(V,W) = W\cdot (\sigma^2 - V^2)$, where $\sigma^2$ is the variance of the Gaussian. Hence, the equilibrium is for the generator to pick $V=\sigma$ and the discriminator to pick $W=0$.

\paragraph{Dynamics without sampling noise.} For the mean GD dynamics the update rules are as follows:
\begin{equation}
\begin{aligned}
W_{ij}^t =& W_{ij}^{t-1} + \eta \left(\Sigma_{ij} - \sum_{k} V_{ik}^{t-1} V_{jk}^{t-1}\right) \\
V_{ij}^t =& V_{ij}^{t-1} + \eta \sum_{k} \left(W_{ik}^{t-1} + W_{ki}^{t-1}\right) V_{kj}^{t-1} 
\end{aligned}
\end{equation}
We can write the latter updates in a simpler matrix form:
\begin{equation}
\begin{aligned}
W_t =& W_{t-1} + \eta \left(\Sigma - V_{t-1} V_{t-1}^T\right)\\
V_t =& V_{t-1} + \eta (W_{t-1} + W_{t-1}^T) V_{t-1}
\end{aligned}\tag{GD for Covariance}
\end{equation}
Similarly the OMD dynamics are:
\begin{equation}
\begin{aligned}
W_t =& W_{t-1} + 2\eta \left(\Sigma - V_{t-1} V_{t-1}^T\right) - \eta \left(\Sigma - V_{t-2} V_{t-2}^T\right)\\
V_t =& V_{t-1} + 2\eta (W_{t-1} + W_{t-1}^T) V_{t-1} - \eta (W_{t-2} + W_{t-2}^T) V_{t-2}
\end{aligned}\tag{OMD for Covariance}
\end{equation}

Due to the non-convexity of the generators problem and because there might be multiple optimal solutions (e.g. if $\Sigma$ is not strictly positive definite), it is helpful in this setting to also help dynamics by adding $\ell_2$ regularization to the loss of the game. The latter simply adds an extra $2\lambda W_{t}$ at each gradient term $\nabla_W L(V_t, W_t)$ for the discriminator and a $2\lambda V_{t}$ at each gradient term $\nabla_{V} L(V_t, W_t)$ for the generator. In Figures \ref{fig:covariance} and \ref{fig:covariance2d} we give the weights and the implied covariance matrix $\Sigma^G=VV^T$ of the generator's distribution for each of the dynamics for an example setting of the step-size and regularization parameters and for two and three dimensional gaussians respectively. We again see how OMD can stabilize the dynamics to converge pointwise.

\paragraph{Stochastic dynamics.} In Figure \ref{fig:stoch_covariance} and \ref{fig:stoch_covariance2} we also portray the instability of GD and the robustness of the stability of OMD under stochastic dynamics. In the case of stochastic dynamics the gradients are replaced with unbiased estimates or with averages of unbiased estimates over a small minibatch. In the case of a mini-batch of one, the unbiased estimates of the gradients in this setting take the following form:
\begin{equation}
\begin{aligned}
\hat{\nabla}_{W, t} = x_{t} x_{t}^T - V_{t}z_{t} z_{t}^T  V_{t}^T\\
\hat{\nabla}_{V, t} = - (W_{t} + W_{t}^T) V_{t} z_t z_t^T
\end{aligned}\tag{Stochastic Gradients}
\end{equation}
where $x_t, z_t$ are samples drawn from the true distribution and from the random noise distribution respectively. Hence, the stochastic dynamics simply follow by replacing gradients with unbiased estimates:
\begin{equation}
\begin{aligned}
W_t =& W_{t-1} + \eta \hat{\nabla}_{W, t-1}\\
V_t =& V_{t-1} - \eta \hat{\nabla}_{V, t-1}
\end{aligned}\tag{SGD for Covariance}
\end{equation}
\begin{equation}
\begin{aligned}
W_t =& W_{t-1} + 2\eta \hat{\nabla}_{W, t-1} - \eta \hat{\nabla}_{W, t-2}\\
V_t =& V_{t-1} - 2\eta \hat{\nabla}_{V, t-1} + \eta \hat{\nabla}_{V, t-2}
\end{aligned}\tag{SOMD for Covariance}
\end{equation}

\newpage

\begin{figure}[htpb]
    \centering
    \begin{subfigure}[b]{1\textwidth}
        \centering
    		\begin{subfigure}[b]{.3\textwidth}
    		\includegraphics[height=1.7in]{2d_covariance_gd_disc.png}
			\end{subfigure}        
    		\begin{subfigure}[b]{.3\textwidth}
    		\includegraphics[height=1.7in]{2d_covariance_gd_gen_V.png}
			\end{subfigure}        
    		\begin{subfigure}[b]{.3\textwidth}
    		\includegraphics[height=1.7in]{2d_covariance_gd_gen_Sigma.png}
			\end{subfigure}        
        \caption{GD dynamics. $\eta=0.1$, $T=500$, $\lambda=0.3$.}
    \end{subfigure}
    \begin{subfigure}[b]{1\textwidth}
        \centering
    		\begin{subfigure}[b]{.3\textwidth}
    		\includegraphics[height=1.7in]{2d_covariance_omd_disc.png}
			\end{subfigure}        
    		\begin{subfigure}[b]{.3\textwidth}
    		\includegraphics[height=1.7in]{2d_covariance_omd_gen_V.png}
			\end{subfigure}        
    		\begin{subfigure}[b]{.3\textwidth}
    		\includegraphics[height=1.7in]{2d_covariance_omd_gen_Sigma.png}
			\end{subfigure}        
        \caption{OMD dynamics. $\eta=0.1$, $T=500$, $\lambda=0.3$.}
    \end{subfigure}
    \caption{Stability of OMD vs GD in the co-variance learning problem for a two-dimensional gaussian ($d=2$). Weight clipping in $[-1,1]$ was applied in both dynamics.}\label{fig:covariance2d}
\end{figure}


\begin{figure}[H]
    \centering
    \begin{subfigure}[b]{1\textwidth}
        \centering
    		\begin{subfigure}[b]{.3\textwidth}
    		\includegraphics[height=1.7in]{covariance_gd_disc.png}
			\end{subfigure}        
    		\begin{subfigure}[b]{.3\textwidth}
    		\includegraphics[height=1.7in]{covariance_gd_gen_V.png}
			\end{subfigure}        
    		\begin{subfigure}[b]{.3\textwidth}
    		\includegraphics[height=1.7in]{covariance_gd_gen_Sigma.png}
			\end{subfigure}        
        \caption{GD dynamics. $\eta=0.1$, $T=500$, $\lambda=0.3$.}
    \end{subfigure}
    \begin{subfigure}[b]{1\textwidth}
        \centering
    		\begin{subfigure}[b]{.3\textwidth}
    		\includegraphics[height=1.7in]{covariance_omd_disc.png}
			\end{subfigure}        
    		\begin{subfigure}[b]{.3\textwidth}
    		\includegraphics[height=1.7in]{covariance_omd_gen_V.png}
			\end{subfigure}        
    		\begin{subfigure}[b]{.3\textwidth}
    		\includegraphics[height=1.7in]{covariance_omd_gen_Sigma.png}
			\end{subfigure}        
        \caption{OMD dynamics. $\eta=0.1$, $T=500$, $\lambda=0.3$.}
    \end{subfigure}
    \caption{Stability of OMD vs GD in the co-variance learning problem for a three-dimensional gaussian ($d=3$). Weight clipping in $[-1,1]$ was applied in both dynamics.}\label{fig:covariance}
\end{figure}

\newpage

\begin{figure}[H]
    \begin{subfigure}[b]{1\textwidth}
        \centering
    		\begin{subfigure}[b]{.3\textwidth}
    		\includegraphics[height=1.7in]{covariance_stoch_gd_disc.png}
			\end{subfigure}        
    		\begin{subfigure}[b]{.3\textwidth}
    		\includegraphics[height=1.7in]{covariance_stoch_gd_gen_V.png}
			\end{subfigure}        
    		\begin{subfigure}[b]{.3\textwidth}
    		\includegraphics[height=1.7in]{covariance_stoch_gd_gen_Sigma.png}
			\end{subfigure}        
        \caption{Stochastic GD dynamics with mini-batch size $50$. $\eta=0.02$, $T=1000$, $\lambda=0.1$.}
    \end{subfigure}
    \begin{subfigure}[b]{1.01\textwidth}
    		\begin{subfigure}[b]{.19\textwidth}
    		\hspace{-.3in}     	
    		\includegraphics[height=1.3in]{covariance_gd_true.png}
    		\caption{True Distribution}
			\end{subfigure}        
    		\begin{subfigure}[b]{.19\textwidth}
    		\hspace{-.3in}     	
    		\includegraphics[height=1.3in]{covariance_gd_iterate_minus_50.png}
    		\caption{Iterate $T-50$}
			\end{subfigure}   
    		\begin{subfigure}[b]{.19\textwidth}
    		\hspace{-.3in}     	
    		\includegraphics[height=1.3in]{covariance_gd_iterate_minus_35.png}
    		\caption{Iterate $T-35$}
			\end{subfigure}   
    		\begin{subfigure}[b]{.19\textwidth}
    		\hspace{-.3in}     	
    		\includegraphics[height=1.3in]{covariance_gd_iterate_minus_20.png}
    		\caption{Iterate $T-20$}
			\end{subfigure}  
    		\begin{subfigure}[b]{.19\textwidth}
    		\hspace{-.3in}     	
    		\includegraphics[height=1.3in]{covariance_gd_last_iterate.png}
    		\caption{Iterate $T$}
			\end{subfigure}  
    	\caption{Comparison of true distribution and distribution of generator at various points closer to the end of training.}
    \end{subfigure}
    \caption{Stochastic GD dynamics for covariance learning of a two-dimensional gaussian ($d=2$). Weight clipping in $[-1,1]$ was applied to the discriminator weights.}\label{fig:stoch_covariance}
\end{figure}

\begin{figure}[H]
    \centering
    \begin{subfigure}[b]{1\textwidth}
        \centering
    		\begin{subfigure}[b]{.3\textwidth}
    		\includegraphics[height=1.7in]{covariance_stoch_omd_disc.png}
			\end{subfigure}        
    		\begin{subfigure}[b]{.3\textwidth}
    		\includegraphics[height=1.7in]{covariance_stoch_omd_gen_V.png}
			\end{subfigure}        
    		\begin{subfigure}[b]{.3\textwidth}
    		\includegraphics[height=1.7in]{covariance_stoch_omd_gen_Sigma.png}
			\end{subfigure}        
        \caption{Stochastic OMD dynamics with mini-batch size $50$. $\eta=0.02$, $T=1000$, $\lambda=0.1$.}
    \end{subfigure}
    \begin{subfigure}[b]{1.01\textwidth}
    		\begin{subfigure}[b]{.19\textwidth}   
    		\hspace{-.3in}     	
    		\includegraphics[height=1.3in]{covariance_omd_true.png}
    		\caption{True Distribution}
			\end{subfigure}        
    		\begin{subfigure}[b]{.19\textwidth}
    		\hspace{-.3in}     	
    		\includegraphics[height=1.3in]{covariance_omd_iterate_minus_50.png}
    		\caption{Iterate $T-50$}
			\end{subfigure}   
    		\begin{subfigure}[b]{.19\textwidth}
    		\hspace{-.3in}     	
    		\includegraphics[height=1.3in]{covariance_omd_iterate_minus_35.png}
    		\caption{Iterate $T-35$}
			\end{subfigure}   
    		\begin{subfigure}[b]{.19\textwidth}
    		\hspace{-.3in}     	
    		\includegraphics[height=1.3in]{covariance_omd_iterate_minus_20.png}
    		\caption{Iterate $T-20$}
			\end{subfigure}  
    		\begin{subfigure}[b]{.19\textwidth}
    		\hspace{-.3in}     	
    		\includegraphics[height=1.3in]{covariance_omd_last_iterate.png}
    		\caption{Iterate $T$}
			\end{subfigure}  
    	\caption{Comparison of true distribution and distribution of generator at various points closer to the end of training.}
    \end{subfigure}
    \caption{Stability of OMD with stochastic gradients for covariance learning of a two-dimensional gaussian ($d=2$). Weight clipping in $[-1,1]$ was applied to the discriminator weights.}\label{fig:stoch_covariance2}
\end{figure}




\section{Last Iterate Convergence of OMD in Bilinear Case}\label{sec:appendix:last-iterate}
 \label{sec:bilinear OMD convergence}

The goal of this section is to show that Optimistic Mirror Descent exhibits last iterate convergence to min-max solutions for bilinear functions. In Section~\ref{app:proof of special case minmax}, we provide the proof of Theorem~\ref{thm:convergence of OGD-main}, that OMD exhibits last iterate convergence to min-max solutions of the following min-max problem
\begin{align}
 \min_x \max_y x^T A y, \label{eq:our minmax}
\end{align}
where $A$ is an abitrary matrix and $x$ and $y$ are unconstrained. In Section~\ref{app:proof of general case minmax}, we state the appropriate extension of our theorem to the general case:
\begin{align}
    \inf_{x} \sup_{y} \left(x^TAy + b^Tx + c^Ty + d\right). \label{eq:general inf sup}
\end{align}

\subsection{Proof of Theorem~\ref{thm:convergence of OGD-main}} \label{app:proof of special case minmax}

\smallskip As stated in Section~\ref{sec:main:proof OMD converges}, for the min-max problem~\eqref{eq:our minmax}
%
%In this section, we show that Optimistic Gradient Descent  
%exhibits final-iterate, rather than only average-iterate convergence to min-max solutions for
%bilinear functions. 
%%In particular, we show that the $\ell_2$ norms of the gradients used by the dynamics shrinks in time.
%More precisely, we consider the problem $\min_x \max_y x^T A y$, for some matrix $A$, where $x$ and $y$ are unconstrained. 
Optimistic Mirror Descent takes the following form, for all $t \ge 1$:
\begin{align}
    x_{t} &= \xto \label{eq:OGD bilinear x}\\
    y_{t} &= \yto  \label{eq:OGD bilinear y}
\end{align}
where for the above iterations to be meaningful we need to specify $x_0,x_{-1},y_0,y_{-1}$. 

\smallskip As stated in Section~\ref{sec:main:proof OMD converges} we allow any initialization $x_0 \in \mathcal{R}(A)$, and $y_0\in\mathcal{R}(A^T)$, and set $x_{-1}=2x_0$ and $y_{-1}=2y_{0}$, where ${\cal R}(\cdot)$ represents column space. In particular, our initialization means that the first step taken by the dynamics gives $x_1=x_0$ and $y_1=y_0$.

Before giving our proof of Theorem~\ref{thm:convergence of OGD-main}, we need some further notation.
%\smallskip We will analyze Optimistic Gradient Descent under the assumption $\lambda_{\infty} \le 1$, where $\lambda_{\infty}=\max\{||A||,||A^T||\}$ and $||\cdot||$ denotes spectral norm of matrices. We can always enforce that $\lambda_{\infty} \le 1$ by appropriately scaling $A$. Scaling $A$ by some positive factor clearly does not change the min-max solutions $(x^*,y^*)$, only scales the optimal value $x^{*T}Ay^*$ by the same factor.
%
%\begin{remark}
%We remark that $(x,y)=(0,0)$ is always a solution to $\min_x \max_y x^T A y$. More generally, the solutions to the problem are pairs $(x,y)$ such that $x$ is in the null space of $A^T$ and $y$ is in the null space of $A$. In particular, finding a solution to $\min_x \max_y x^T A y$ is a trivial problem. So this section only serves the purpose of rigorously showing that Optimistic Gradient Descent converges to a min-max solution. This is interesting in light of the fact that Gradient Descent actually diverges, even in the special case where $A$ is the identity matrix, as per the following proposition whose proof is provided in Appendix~\ref{appendix:omitted proofs}.
%
%
%{\begin{proposition} \label{prop:gradient descent unstable}
%Gradient descent applied to problem $\min_x \max_y x^T A y$ diverges from any initialization $x_0, y_0$ such that $x_0,y_0 \neq 0$.
%\end{proposition}}
%\end{remark}
%
%
%\paragraph{Notation.} We start with some notation that will be handy later on. 
For all $i \in \mathbb{N}$, we set:
\begin{align*}
      &~~~~~~~~~~M_i = A^j(A^TA)^k, N_i = \(A^T\)^j \(AA^T\)^k \\ 
    &~~~~~~~~~~\Delta^i_t = \normlt{N_iAy_t} + \normlt{M_iA^Tx_t}\\
		&\text{where $k \in \mathbb{Z}$ and $j \in \{0, 1\}$ are such that:~}   i = 2k + j.
\end{align*}
\noindent With this notation, $\Delta_t^0 =
\normlt{A^Tx_t} + \normlt{Ay_t}$, $\Delta^1_t = \normlt{AA^Tx_t} +
\normlt{A^TAy_t}$, $\Delta^2_t = \normlt{A^TAA^Tx_t} +
\normlt{AA^TAy_t}$, etc.

We also use the notation $\langle u, v \rangle_X = u^TXX^Tv$, for vectors $u, v
\in \mathbb{R}^d$ and square $d \times d$ matrices $X$. We similarly define the
norm notation $||u||_X=\sqrt{\langle u, u \rangle_X}$. Given our notation, we
have the following claim, shown in Appendix~\ref{appendix:omitted proofs}.
\begin{claim} \label{claim:pushing A's around}
For all matrices $A$ and vectors $u,v$ of the appropriate dimensions:\\
$$\innerab{Au}{Av}{i} = \innerac{u}{v}{i+1};~~\innerac{A^Tu}{A^Tv}{i} = \innerab{u}{v}{i+1};~~\innerab{u}{Av}{i} = \innerac{v}{A^Tu}{i}.$$
\end{claim}


With our notation in place, we show (through iterated expansion of the update rule), the
following lemma, proved in Appendix~\ref{appendix:omitted proofs}:
\begin{lemma}  \label{lemma:first bound}
For the dynamics of Eq.~\eqref{eq:OGD bilinear x} and~\eqref{eq:OGD bilinear y} and any initialization ${1 \over 2}x_{-1}=x_0 \in
\mathcal{R}(A)$, and ${1 \over 2}y_{-1}=y_0\in\mathcal{R}(A^T)$ we have the following for all $i, t \in \mathbb{N}$ such that $i\ge 0$ and $t \ge 2$:
$$\Delta^i_t - \Delta^i_{t-1} = 4\eta^2\Delta^{i+1}_{t-1} -
5\eta^2\Delta^{i+1}_{t-2} - 2\eta^3\(\innerab{x_{t-2}}{Ay_{t-4}}{i+1} -
\innerac{y_{t-2}}{A^Tx_{t-4}}{i+1}\).$$
\end{lemma}


\medskip We are ready to prove Theorem~\ref{thm:convergence of OGD-main}. Its proof is implied by the following stronger theorem, and Corollary~\ref{cor:gradient becomes small}.

%\begin{theorem}\label{thm:convergence of OGD}
%Consider the dynamics of Eq.~\eqref{eq:OGD bilinear x} and~\eqref{eq:OGD bilinear y} and any initialization ${1 \over 2}x_{-1}=x_0 \in
%\mathcal{R}(A)$, and ${1 \over 2}y_{-1}=y_0\in\mathcal{R}(A^T)$. Let $\gamma = \max\(\left|\left|\(AA^T\)^{+}\right|\right|,
	    %\left|\left|\(A^TA\)^{+}\right|\right|\)$, where for a matrix $X$ we denote by	$X^{+}$ its generalized inverse and by $||X||$ its
    %spectral norm. Suppose that $\max\{||A||,||A^T||\}\equiv \lambda_{\infty}\le 1$ and $\eta<\gamma$. Then, for all $i \in \mathbb{N}$:
		%\begin{align}
		%\Delta^i_1 = \Delta^i_{0}, \label{eq:target condition 1}
		%\end{align}
		%and, for all $i,t\in \mathbb{N}$ such that $t \ge 2$, the following condition holds:
%\begin{align}
    %H(i,t):~~\Delta^i_t \leq \left(1-{\eta}\right)\Delta^i_{t-1} + O(\eta^3 \Delta^0_0). \label{eq:target condition 2}
%\end{align}
%\end{theorem}

\begin{theorem}\label{thm:convergence of OGD}
Consider the dynamics of Eq.~\eqref{eq:OGD bilinear x} and~\eqref{eq:OGD bilinear y} and any initialization ${1 \over 2}x_{-1}=x_0 \in
\mathcal{R}(A)$, and ${1 \over 2}y_{-1}=y_0\in\mathcal{R}(A^T)$. Let $$\gamma = \max\(\left|\left|\(AA^T\)^{+}\right|\right|,
	    \left|\left|\(A^TA\)^{+}\right|\right|\),$$ where for a matrix $X$ we denote by	$X^{+}$ its generalized inverse and by $||X||$ its
    spectral norm. Suppose that $\max\{||A||,||A^T||\}\equiv \lambda_{\infty}\le 1$ and $\eta$ is a small enough constant satisfying $\eta <1/(3\gamma^2)$. Then, for all $i \in \mathbb{N}$:
		\begin{align}
		\Delta^i_1 = \Delta^i_{0}, \label{eq:target condition 1}\\
		\Delta^i_2 \le (1+\eta)^2\Delta^i_{0}, \label{eq:target condition 1.5}
		\end{align}
		and, for all $i,t\in \mathbb{N}$ such that $t \ge 3$, the following condition holds:
\begin{align}
    H(i,t):~~\Delta^i_t \leq \left(1-{\eta^2 \over \gamma^2}\right)\Delta^i_{t-1} + 16\eta^3 \Delta^0_0. \label{eq:target condition 2}
\end{align}
\end{theorem}

\begin{proof}
Eq.~\eqref{eq:target condition 1} holds trivially as under our initialization $x_1=x_0$ and $y_1=y_0$. Eq.~\eqref{eq:target condition 1.5} is also easy to show by noticing the following. Given our initialization:
\begin{align*}
x_2=x_0-\eta A y_0\\
y_2=y_0+\eta A x_0
\end{align*}
Hence (using $j=i \mod 2$):
\begin{align}
M_iA^Tx_2=M_iA^Tx_0 - \eta M_iA^TA y_0\\~~~~~~~~~~~~\Rightarrow~~||M_iA^Tx_2||_2 &\le ||M_iA^Tx_0||_2+\eta ||M_iA^TA y_0||_2\\
&= ||M_iA^Tx_0||_2+\eta ||A^j(A^T)^{1-j}N_iA y_0||_2\\
&\le ||M_iA^Tx_0||_2+\eta \lambda_{\infty} ||N_iA y_0||_2\\
&\le ||M_iA^Tx_0||_2+\eta  ||N_iA y_0||_2 \label{eq:kourasi1}
\end{align}
Similarly:
\begin{align}
||N_iAy_2||_2 &\le ||N_iAy_0||_2+\eta  ||M_iA^T x_0||_2 \label{eq:kourasi2}
\end{align}
It follows from~\eqref{eq:kourasi1} and~\eqref{eq:kourasi2} that
$$\Delta^i_2 \le (1+\eta)^2 \Delta^i_0.$$

We use induction on $t$ to prove~\eqref{eq:target condition 2}. We start our proof by showing the inductive step, and postpone establishing the basis of our induction to the end of this proof. For the inductive step, we assume that $H(i,\tau)$ holds for all $i \ge 0$ and $1\le \tau < t$, for some $t>3$. Assuming this, we show next that $H(i,t)$ holds for all $i$. To do this, we make use of a few lemmas, whose proofs are given in Appendix~\ref{appendix:omitted proofs}. 
	
    \begin{lemma} \label{lemma:restated bound} Under the conditions of the theorem, for all $i \ge 0, t \ge 2$:
		%$$\Delta^i_t - \Delta^i_{t-1} \leq 4\eta^2\Delta_{t-1}^{i+1} -
    %5\eta^2\Delta_{t-2}^{i+1} +2\eta^3(9\Delta_{t-2}^{i+1} + 8\eta^2 \Delta_{t-3}^{i+1} + 8\eta^2 \Delta_{t-4}^{i+1} + 8\eta^2 \Delta_{t-5}^{i+1}).$$
$$\Delta^i_t - \Delta^i_{t-1} \leq 4\eta^2\Delta^{i+1}_{t-1} - 5\eta^2\Delta^{i+1}_{t-2} +
    \eta^3(\Delta_{t-2}^{i+1} + \Delta_{t-4}^{i+1}).$$

		
    %$$\Delta^i_t - \Delta^i_{t-1} \leq 4\eta^2\Delta_{t-1}^{i+1} -
    %5\eta^2\Delta_{t-2}^{i+1} +
    %\eta^3\((1+\lambda_\infty)\Delta_{t-2}^{i+1} +
    %4\eta\lambda^2_\infty\Delta_{0}^{i+1}\).$$
    \end{lemma}
		
		\begin{lemma} \label{lemma:semi-trivial upper bound}
		Under the conditions of the theorem, for all $i, t \ge 0$: $\Delta_t^{i+1} \le \Delta_t^{i}$.
		\end{lemma}

    \begin{lemma} \label{lemma:lower bound}  Under the conditions of the theorem, for all $i \ge 0, t \ge 0$:
	$$\Delta_{t}^{i+2} \geq \frac{1}{\gamma^2}\Delta_{t}^{i}.$$
    \end{lemma}
    
    Given these lemmas, we show our inductive step. So for $t \ge 4$:
		  \begin{align}
	\Delta^i_t - \Delta^i_{t-1} &\leq 4\eta^2\Delta^{i+1}_{t-1} - 5\eta^2\Delta^{i+1}_{t-2} +
    \eta^3(\Delta_{t-2}^{i+1} + \Delta_{t-4}^{i+1}) \label{eq:derivation1}\\
	&\leq -\eta^2 \Delta_{t-1}^{i+1} + \eta^3(\Delta_{t-2}^{i+1} + \Delta_{t-4}^{i+1})+ 80 \eta^5 \Delta_0^0 \\
	%&\leq \eta^2(2\eta-1)\Delta^{i+1}_{t-2} +
	%4\eta^4\Delta_0^0 +O(\eta^5 \Delta^0_0)\\
	%&= -\eta^2\Delta^{i+1}_{t-2} + 2\eta^3\Delta_{t-2}^{i+1} +O(\eta^5\Delta^0_0)\\
&\leq -\frac{1}{\gamma^2}\eta^2\Delta^{i-1}_{t-1} + \eta^3(\Delta_{t-2}^{i+1} + \Delta_{t-4}^{i+1})+ 80 \eta^5 \Delta_0^0\\
&\leq -\frac{1}{\gamma^2}\eta^2\Delta^{i-1}_{t-1} + \eta^3(\Delta_{t-2}^{0} + \Delta_{t-4}^{0})+ 80 \eta^5 \Delta_0^0 \\
&\leq -\frac{1}{\gamma^2}\eta^2\Delta^{i-1}_{t-1} + \(2\eta^3\Delta_{2}^{0} +2\eta^3 {\gamma^2 \over \eta^2}16\eta^3 \Delta^0_0\)+ 80 \eta^5 \Delta_0^0 \\
&\leq -\frac{1}{\gamma^2}\eta^2\Delta^{i-1}_{t-1} + \(2\eta^3(1+\eta)^2 +32\eta^3 {\gamma^2 \over \eta^2}\eta^3  + 80 \eta^5\)\Delta_0^0 \\
&\leq -\frac{1}{\gamma^2}\eta^2\Delta^{i-1}_{t-1} + \(2\eta^3(1+\eta)^2 +11\eta^3 + 80 \eta^5 \)\Delta_0^0 \\
	&\leq -\frac{1}{\gamma^2}\eta^2\Delta^{i-1}_{t-1} + 16\eta^3\Delta_0^0\\
	&\leq -\frac{1}{\gamma^2}\eta^2\Delta^{i}_{t-1} + 16\eta^3\Delta_0^0	\label{eq:derivation6}
    \end{align}
		
		
		  %\begin{align}
	%\Delta^i_t - \Delta^i_{t-1} &\leq 4\eta^2\Delta_{t-1}^{i+1} -
	%5\eta^2\Delta_{t-2}^{i+1} +
%2\eta^3(9\Delta_{t-2}^{i+1} + 8\eta^2 \Delta_{t-3}^{i+1} + 8\eta^2 \Delta_{t-4}^{i+1} + 8\eta^2 \Delta_{t-5}^{i+1}) \label{eq:derivation1}\\
	%&\leq \eta^2(18\eta - 1)\Delta_{t-2}^{i+1} +O(\eta^5\Delta^0_0) \\
	%%&\leq \eta^2(2\eta-1)\Delta^{i+1}_{t-2} +
	%%4\eta^4\Delta_0^0 +O(\eta^5 \Delta^0_0)\\
	%&= -\eta^2\Delta^{i+1}_{t-2} + 2\eta^3\Delta_{t-2}^{i+1} +O(\eta^5\Delta^0_0)\\
	%&\leq -\frac{1}{\gamma}\eta^2\Delta^{i}_{t-2} + 2\eta^3\Delta_0^0 +O(\eta^5\Delta^0_0)\\
	%&\leq -\frac{1}{\gamma}\eta^2\Delta^{i}_{t-1} + O(\eta^3\Delta_0^0)	\label{eq:derivation6}
    %\end{align}
    %%\begin{align}
	%\Delta^i_t - \Delta^i_{t-1} &\leq 4\eta^2\Delta_{t-1}^{i+1} -
	%5\eta^2\Delta_{t-2}^{i+1} +
	%\eta^3\((1+\lambda_\infty)\Delta_{t-2}^{i+1} +
	%4\eta\lambda^2_\infty\Delta_{0}^{i+1}\) \label{eq:derivation1}\\
	%&\leq \eta^2((1+\lambda_\infty)\eta - 1)\Delta_{t-2}^{i+1} +
	%4\eta^4\lambda_\infty^2\Delta_0^{i+1} +O(\eta^5\Delta^0_0) \\
	%&\leq \eta^2(2\eta-1)\Delta^{i+1}_{t-2} +
	%4\eta^4\Delta_0^0 +O(\eta^5 \Delta^0_0)\\
	%&= -\eta^2\Delta^{i+1}_{t-2} + \eta^3(2\Delta_{t-2}^{i+1}+4\eta\Delta_0^0) +O(\eta^5\Delta^0_0)\\
	%&\leq -\frac{1}{\gamma}\eta^2\Delta^{i}_{t-2} + 6\eta^3\Delta_0^0 +O(\eta^5\Delta^0_0)\\
	%&\leq -\frac{1}{\gamma}\eta^2\Delta^{i}_{t-1} + 6\eta^3\Delta_0^0 +O(\eta^4\Delta^0_0)	\label{eq:derivation6}
    %\end{align}
where for the first inequality we used Lemma~\ref{lemma:restated bound}, 
for the second inequality we used that $\Delta_{t-1}^{i+1} \le \Delta_{t-2}^{i+1}+16\eta^3\Delta^0_0$ (which is implied by the induction hypothesis), 
for the third inequality we used Lemma~\ref{lemma:lower bound},
for the fourth inequality we used Lemma~\ref{lemma:semi-trivial upper bound},
for the fifth inequality we applied the induction hypothesis iteratively, for the sixth inequality we used Eq.~\eqref{eq:target condition 1.5}, for the seventh and eighth inequality we used that $\eta$ is small enough, and for the last inequality we used Lemma~\ref{lemma:semi-trivial upper bound}.
%
 %and that $\Delta_{0}^{i+1} \le \Delta_{0}^{0}$ (which also easily follows from the fact that $\lambda_\infty \le 1$), 
%for the fourth inequality we used Lemma~\ref{lemma:lower bound} and that $\Delta^{i+1}_{t-2} \le \Delta_0^0 + O(\eta^2\Delta^0_0)$, which follows by first noting that $\Delta^{i+1}_{t-2} \le \Delta^{0}_{t-2}$ (which easily follows from the fact that $\lambda_\infty \le 1$) and then noting that $\Delta^{0}_{t-2}\le \Delta^{0}_{0}+O(\eta^2\Delta^0_0)$ (which follows by iteratively applying the inductive hypothesis and using~\eqref{eq:target condition 1}), and for the last inequality we used that $\Delta^{i}_{t-1}\le \Delta^{i}_{t-2}+O(\eta^3\Delta^0_0)$, which follows from the inductive hypothesis, and that $\eta < \gamma$. 
Hence:
$$\Delta^i_t \leq \(1-{\eta^2 \over \gamma^2 }\)\Delta^i_{t-1} + 16\eta^3\Delta^0_0.$$
This completes the proof of our inductive step.

It remains to show the basis of the induction, namely that $H(i,3)$ holds for all $i \in \mathbb{N}$. From Lemma~\ref{lemma:restated bound} we have:
 \begin{align}
	\Delta^i_3 - \Delta^i_{2} &\leq 4\eta^2\Delta^{i+1}_{2} - 5\eta^2\Delta^{i+1}_{1} +
    \eta^3(\Delta_{1}^{i+1} + \Delta_{-1}^{i+1})\\
 &\leq 4\eta^2\Delta^{i+1}_{2} - 5\eta^2\Delta^{i+1}_{0}+5\eta^3 \Delta^{i+1}_0\\
&= -\eta^2\Delta^{i+1}_{2}+ 5\eta^2(\Delta^{i+1}_{2}-\Delta^{i+1}_{0})+5\eta^3 \Delta^{i+1}_0\\
&\le -\eta^2\Delta^{i+1}_{2}+ 5\eta^3(2+\eta)\Delta^{i+1}_{0}+5\eta^3 \Delta^{i+1}_0\\
&= -\eta^2\Delta^{i+1}_{2}+ 5\eta^3(3+\eta)\Delta^{i+1}_{0}\\
 &\le -\eta^2\Delta^{i+1}_{2}+15\eta^3(1+\eta/3)\Delta^0_{0}\\
 &\leq -{\eta^2 \over \gamma^2}\Delta^{i-1}_{2}+15\eta^3(1+\eta/3)\Delta^0_{0}\\
 &\leq -{\eta^2 \over \gamma^2}\Delta^{i}_{2}+15\eta^3(1+\eta/3)\Delta^0_{0},
	\end{align}
where for the second equality we used that $0.5x_{-1}=x_0=x_1$ and $0.5y_{-1}=y_0=y_1$ (which follow from our initialization),
for the third inequality we used that~\eqref{eq:target condition 1.5},
for the fourth inequality we used Lemma~\ref{lemma:semi-trivial upper bound}, 
for the fifth inequality we used Lemma~\ref{lemma:lower bound}, 
and for the last inequality we used Lemma~\ref{lemma:semi-trivial upper bound}. Hence, for small enough $\eta$, we have:
$$\Delta^i_3 \le \(1-{\eta^2 \over \gamma^2}\) \Delta^i_{2} + 16 \eta^3 \Delta^0_0.$$
%To do this, we follow the  derivation in lines~\eqref{eq:derivation1}-\eqref{eq:derivation6} above, noticing that what we needed for this derivation to go through holds for $t=2$. In particular, the first inequality uses Lemma~\ref{lemma:restated bound}, which holds for $t=2$. The second inequality goes through because $\Delta_{1}^{i+1} = \Delta_{0}^{i+1}$, for all $i$, given~\eqref{eq:target condition 1}. The third inequality goes through for the same reasons that were used in the induction step. The fourth inequality goes through since $\Delta_{0}^{i+1} \geq \frac{1}{\gamma}\Delta_{0}^{i}$, which holds from Lemma~\ref{lemma:lower bound}, and $\Delta^{i+1}_{0}\le \Delta^{0}_{0}$, which holds since $\lambda_{\infty}\le 1$. The last inequality follows from the fact that $\Delta^{i}_{1} = \Delta^{i}_{0}$.
    \end{proof}

\begin{corollary} \label{cor:gradient becomes small}
Under the conditions of Theorem~\ref{thm:convergence of OGD}, $\Delta^0_t \equiv \normlt{A^Tx_t} + \normlt{Ay_t} \rightarrow O(\eta \gamma^2 \Delta^0_0)$ as $t \rightarrow +\infty$. In particular, for large enough $t$, the last iterate of OMD is within $O\(\sqrt{\eta} \cdot \gamma \sqrt{\Delta^0_0}\)$ distance from the space of equilibrium points of the game, where $\sqrt{\Delta^0_0}$ is the distance of the initial point $(x_0,y_0)$ from the equilibrium space, and where both distances are taken with respect to the norm $\sqrt{x^T A A^T x + y^T A^T A y}$.
%In particular, for large enough $t$, $x_t$ and $y_t$ are approximately optimal in the following sense:
%\begin{align}
%x_t^T A y_t \le \min_x x^T A y_t + O(\eta^2\Delta^0_0);\\
%x_t^T A y_t \ge \max_y x^T A y + O(\eta^2\Delta^0_0).
%\end{align}
\end{corollary}
\begin{prevproof}{Corollary}{cor:gradient becomes small}
It follows from~\eqref{eq:target condition 1},~\eqref{eq:target condition 1.5} and~\eqref{eq:target condition 2} that:
\begin{align*}
\Delta^0_t &\le \(1-{\eta^2 \over \gamma^2}\)^{t-2}  (1+\eta)^2 \Delta^0_0 + 16 \sum_{t=0}^{\infty}\(1-{\eta^2 \over \gamma^2}\)^t\eta^3 \Delta^0_0 \\&= \(1-{\eta^2 \over \gamma^2}\)^{t-2}  (1+\eta)^2 \Delta^0_0 + O\(\eta \gamma^2 \Delta^0_0\),
\end{align*}
which shows the first part of our claim. For the second part of our claim recall that the solutions to~\eqref{eq:our minmax} are all pairs $(x,y)$ such that $x$ is in the null space of $A^T$ and $y$ is in the null space of $A$. 
%For our second claim let us pick a $t$ such that $(1-\eta)^t=\eta^2$. For such $t$, $\Delta^0_t = O\(\eta^2 \Delta^0_0\)$.
\end{prevproof}

\subsection{General Bilinear Case} \label{app:proof of general case minmax}

\begin{theorem}\label{theorem:general}
    Consider OMD for the min-max problem~\eqref{eq:general inf sup}:
		\begin{align*}
    \inf_{x} \sup_{y} \left(x^TAy + b^Tx + c^Ty + d\right). \label{eq:general inf sup}
\end{align*}
Under the same conditions as Corollary~\ref{cor:gradient becomes small} and whenever~\eqref{eq:general inf sup} is finite, OMD exhibits last iterate convergence in the same sense as in Corollary~\ref{cor:gradient becomes small}. In particular, for large enough $t$, the last iterate of OMD is within $O\(\sqrt{\eta} \cdot \gamma \sqrt{\Delta^0_0}\)$ distance from the space of equilibrium points of the game, where $\sqrt{\Delta_0}$ is the distance of the point $(x_0+(A^T)^+c,y_0+A^+b)$ from the equilibrium space, and where both distances are taken with respect to the norm $\sqrt{x^T A A^T x + y^T A^T A y}$. Whenever~\eqref{eq:general inf sup} is infinite or undefined, the OMD dynamics travels to infinity and we characterize its motion.
\end{theorem}
\begin{prevproof}{Theorem}{theorem:general}
Trivially, we need only consider functions of the form $x^TAy + b^Tx + c^Ty$. We consider the following decompositions
of $b$ and $c$: 
\[
    b &= b_1 + b_2 &\text{where}~b_1 \in \mathcal{R}(A), b_2 \in \mathcal{N}(A^T)  \\ 
    c &= c_1 + c_2 &\text{where}~c_1 \in \mathcal{R}(A^T), c_2 \in \mathcal{N}(A) 
		\]
		Given the above we can also define $b_3$ and $c_3$ as follows:
\[
    Ac_3 &= b_1 &\text{ feasible since } b_1 \in \mathcal{R}(A) \\
    A^Tb_3 &= c_1&\text{ feasible since } c_1 \in \mathcal{R}(A^T)  
\]

Then, we can make the following variable substition:
\[
    \alpha_t &= x_t + \eta t b_2 + b_3 \\
    \beta_t &= y_t - \eta t c_2 + c_3 \\
    \text{so that: }& \\
    A^T\alpha_t &= A^Tx_t + \eta t A^T b_2 + A^T b_3 \\
    &= A^Tx_t + c_1~~~~\text{since $b_2 \in \mathcal{N}(A^T)$} \\
    A\beta_t &= Ay_t - \eta t A c_2 + A c_3  \\
    &= Ay_t + b_1~~~~\text{since $c_2 \in \mathcal{N}(A)$}  \\
\]

We also state the OMD dynamics for $x_t$ and $y_t$ for problem~\eqref{eq:general inf sup}:
\[
    x_t &= x_{t-1} - 2\eta (A y_{t-1} + b)  + \eta (A y_{t-2} + b) \\ 
    &= x_{t-1} - 2\eta A y_{t-1} + \eta A y_{t-2} - \eta b \\ 
    y_t &= y_{t-1} + 2\eta (A^T x_{t-1} + c)  - \eta (A^T x_{t-2} + c) \\
    &= y_{t-1} + 2\eta A^T x_{t-1}  - \eta A^T x_{t-2} + \eta c 
\]

Note that given this update step:
\[
    x_{t+1} &= x_{t} - 2\eta A y_t + \eta A y_{t-1} - \eta b \\
    x_{t+1} &= x_{t} - \eta b_2 - 2\eta A y_t + \eta A y_{t-1} - \eta A c_3 \\
    x_{t+1} &= x_t - \eta b_2 - 2\eta A (y_t + c_3) + \eta A (y_{t-1} + c_3) \\
    x_{t+1} &= x_t - \eta b_2 - 2\eta A (y_t - \eta c_2 t + c_3) + \eta A (y_{t-1} - \eta c_2 (t-1) + c_3) \\
    x_{t+1} + \eta b_2 (t+1) &= x_t + \eta b_2 t - 2\eta A (y_t - \eta c_2 t + c_3) + \eta A (y_{t-1} - \eta c_2 (t-1) + c_3) \\
    x_{t+1} + \eta b_2 (t+1) + b_3  &= x_t + \eta b_2 t + b_3 - 2\eta A (y_t -
    \eta c_2 t + c_3) + \eta A (y_{t-1} - \eta c_2 (t-1) + c_3) \\
    \alpha_{t+1} &= \alpha_t - 2\eta A \beta_t + \eta A \beta_{t-1} \\
    \text{Analogously: }& \\
    \beta_{t+1} &= \beta_t + 2\eta A^T \alpha_t - \eta A^T \alpha_{t-1}
\]
Note that these are precisely the dynamics for which we proved convergence in
Theorem~\ref{thm:convergence of OGD-main}. Thus, by invoking Theorem~\ref{thm:convergence of OGD} and Corollary~\ref{cor:gradient becomes small} on the sequence $(\alpha_t,\beta_t)$ and then substituting back $(x_t,y_t)$, we have that for all large enough $t$:
\[
    x_t &= -\eta b_2 t - b_3 + \epsilon_x(t) \\
    y_t &= \eta c_2 t - c_3 + \epsilon_y(t) \\
    &\text{such that } ||A^T\epsilon_x(t)||_2, ||A\epsilon_y(t)||_2 \in O\(\sqrt{\eta} \cdot \gamma \sqrt{\Delta^0_0}\),
\]
where $\Delta^0_0 = ||A^T(x_0+b_3)||_2^2 + ||A (y_0+c3)||_2^2$.

In particular, this shows that, whenever~\eqref{eq:general inf sup} is finite (i.e.~$b_2=c_2=0$), OMD exhibits last iterate convergence. For large enough $t$, the last iterate of OMD is within $O\(\sqrt{\eta} \cdot \gamma \sqrt{\Delta^0_0}\)$ distance from the space of equilibrium points of the game, where $\sqrt{\Delta^0_0}$ is the distance of $(x_0+b_3,y_0+c_3)$ from the equilibrium space in the norm $\sqrt{x^T A A^T x + y^T A^T A y}$. Whenever~\eqref{eq:general inf sup} is infinite or undefined, the OMD dynamics travels to infinity linearly, with fluctuations around the divergence specified as above. 
\end{prevproof}
%\begin{corollary} \label{cor:general bilinear functions}
%    OGD converges as in the last corollary for functions of the form:
%    $$f(x,y) = (x+b)^TA(x+c)$$
%\end{corollary}
%\begin{prevproof}{Corollary}{cor:general bilinear functions}
%    Let $\alpha_t = x_t + b$, and $\beta_t = y_t + c$. Then, note that the
%    function can be written simply as $f(\alpha, \beta) = \alpha^TA\beta$. Thus,
%    by the main result, we have that both $\alpha, \beta \rightarrow 0$, which
%    in turn implies that $x_t \rightarrow b$, $y_t \rightarrow c$ as required.
%\end{prevproof}

%\begin{remark}{}
%    The above corrollary actually holds for any function $f$ of the form:
%    $$f(x, y) = x^TAy + bx + cy + r$$
%    If $b \in \mathcal{R}(A), c \in \mathcal{R}(A^T)$. This is because if these
%    two hold, we can write $d, e$ s.t. $Ae = b$, $A^Td = c$, then:
%    $$f(x, y) = (x+d)^TA(y+e) - (d^Te + r)$$
%    The constant term can be disregarded since it is not present in any of the
%    gradients, and thus this reduces to the corollary above. Note that if one
%    of these conditions does not hold, the optimum may be ill-defined. For
%    example, $A = [1,0;0,0], b = [1,1], c = [1,1]$ is a configuration such that
%    $f(x,y) = x_1y_1 + x_1 + x_2 + y_1 + y_2 + r$; here, $x_2$ and $y_2$ can be
%    set to $+\infty$, $-\infty$ respectively, so the optimum itself is not
%    well-defined.
%\end{remark}

\subsection{Omitted Proofs} \label{appendix:omitted proofs}

\begin{prevproof}{Proposition}{prop:gradient descent unstable}
    To show this, we consider the $\ell_2$ distance of the solution at time $t$.
    First, recall the GD update step in the special case of $f(x, y) = x^Ty$:
    \[
	x_{t} = x_{t-1} - \eta y_{t-1} \\
	y_{t} = y_{t-1} + \eta x_{t-1}
    \]

    Then, note that the squared $\ell_2$ distance of the running iterate $(x_t,y_t)$ to the unique equilibrium solution $(0,0)$  is given
    by $d(t) := ||x_t||_2^2 + ||y_t||_2^2$, which we can calculate:
    \[
	||x_t||_2^2 &= ||x_{t-1}||^2_2 - 2\eta x_{t-1}^Ty_{t-1} +
	\eta^2||y_{t-1}||_2^2 \\
	||y_{t-1}||_2^2 &= ||y_{t-1}||^2_2 + 2\eta y_{t-1}^T x_{t-1} +
	\eta^2||x_{t-1}||_2^2 \\
	\therefore d(t) &= d(t-1) + \eta^2 d(t-1) \\
	&= (1+\eta^2)d(t-1)
    \]
    This indicates that for any value of $\eta>0$, the running iterate of 
    GD \textit{diverges} from the equilibrium.
\end{prevproof}

\begin{prevproof}{Claim}{claim:pushing A's around}
For our first claim, observe that:
\[
    \innerab{Au}{Av}{i} &= u^TA^TAM_i^TM_iA^TAv \\
			&= u^TA^TA(A^TA)^k(A^j)^TA^j(A^TA)^kA^TAv \\
   &= u^TA^T(AA^T)^kA(A^j)^TA^jA^T(AA^T)^kAv \\
   &= u^TA^T[(AA^T)^kA(A^j)^T][A^jA^T(AA^T)^k]Av \\
   &= u^TA^TN_{i+1}^TN_{i+1}Av \\
   &= \innerac{u}{v}{i+1}
\]
Our second claim, $\innerac{A^Tu}{A^Tv}{i} = \innerab{u}{v}{i+1}$, is proven analogously.

For our third claim:
\[
    \innerab{u}{Av}{i} &= u^TAM_i^TM_iA^TAv \\
		       &= u^TA(A^TA)^{k}(A^TA)^j(A^TA)^kA^TAv \\
    \text{if $j = 0$: }&  \\
		       &= u^TA(A^TA)^{k}(A^TA)^kA^TAv \\
	 &= u^TA[A^T(AA^T)^k][(AA^T)^kA]v \\
  &= u^TA [A^T N_{i}^T][N_{i}A]v \\
  &= \innerac{v}{A^Tu}{i}\\
    \text{otherwise: }& \\
		      &= u^TA(A^TA)^kA^TA(A^TA)^kA^TAv \\
					&= u^TA[(A^TA)^kA^TA][(A^TA)^kA^TA]v \\
					&= u^TA[A^T(AA^T)^kA][A^T(AA^T)^kA]v \\
										&= u^TA[A^TN_i^T][N_iA]v \\
 &= \innerac{v}{A^Tu}{i} 
\]
\end{prevproof}

\begin{prevproof}{Lemma}{lemma:first bound}
First, we note the following scaled update rule:
\[
    M_{i}A^Tx_t &= M_i\(\xtao\) \\
    N_{i}Ay_t &= N_i\(\ytao\)
\]

\noindent Then, taking the norm of both sides, and using the statements of Claim~\ref{claim:pushing A's around}:
\[
    \normab{x_t}{i} &= \normab{x_{t-1}}{i} + 4\eta^2\normac{y_{t-1}}{i+1} +
    \eta^2\normac{y_{t-2}}{i+1} - 4\eta\innerab{x_{t-1}}{Ay_{t-1}}{i} \\
    &\qquad+ 2\eta\innerab{x_{t-1}}{Ay_{t-2}}{i} -
    4\eta^2\innerac{y_{t-1}}{y_{t-2}}{i+1} \\[2ex]
    \normac{y_t}{i} &= \normac{y_{t-1}}{i} + 4\eta^2\normab{x_{t-1}}{i+1} +
    \eta^2\normab{x_{t-2}}{i+1} + 4\eta\innerac{y_{t-1}}{A^Tx_{t-1}}{i} \\
    &\qquad+ 2\eta\innerac{y_{t-1}}{A^T x_{t-2}}{i} -
    4\eta^2\innerab{x_{t-1}}{x_{t-2}}{i+1} \\[2ex]
    \therefore \Delta_t^i &= \normab{x_t}{i} + \normac{y_t}{i} \\
    &= \Delta^i_{t-1} + 4\eta^2\Delta^{i+1}_{t-1} +
    \eta^2\Delta^{i+1}_{t-2} + 2\eta(\innerab{x_{t-1}}{Ay_{t-2}}{i} -
    \innerac{y_{t-1}}{Ax_{t-2}}{i}) \\
    &\qquad\qquad - 4\eta^2(\innerab{x_{t-1}}{x_{t-2}}{i+1} +
    \innerac{y_{t-1}}{y_{t-2}}{i+1})
\]

\noindent Expanding the first pair of inner products above and using Claim~\ref{claim:pushing A's around} again:
\[
   \innerab{x_{t-1}}{Ay_{t-2}}{i} - \innerac{y_{t-1}}{A^T x_{t-2}}{i} &=
    \innerab{\xtt}{Ay_{t-2}}{i} \\
    &\qquad - \innerac{\ytt}{A^T x_{t-2}}{i} \\[2ex]
    = -2\eta(\normac{y_{t-2}}{i+1}+&\normab{x_{t-2}}{i+1}) + \eta(\innerab{x_{t-2}}{x_{t-3}}{i+1} +
    \innerac{y_{t-2}}{y_{t-3}}{i+1})
\]

Then, multiplying by $2\eta$ and substituting into the previous derivation yields:
\[
    \Delta^i_t - \Delta^i_{t-1} &= 4\eta^2\Delta^{i+1}_{t-1} -
    3\eta^2\Delta^{i+1}_{t-2} + 2\eta^2(\innerab{x_{t-2}}{x_{t-3}}{i+1} +
    \innerac{y_{t-2}}{y_{t-3}}{i+1}) \\
    &\qquad\qquad - 4\eta^2(\innerab{x_{t-1}}{x_{t-2}}{i+1} +
    \innerac{y_{t-1}}{y_{t-2}}{i+1})
\]

\noindent Now, consider the following inner product:
\[
    \innerab{x_{t-2}}{x_{t-1}}{i+1} + \innerac{y_{t-2}}{y_{t-1}}{i+1} &=
    \innerab{x_{t-2}}{\xtt}{i+1} \\
    &\qquad + \innerac{y_{t-2}}{\ytt}{i+1} \\[2ex]
    &= \Delta^{i+1}_{t-2} + \eta\(\innerab{x_{t-2}}{Ay_{t-3}}{i+1} -
    \innerac{y_{t-2}}{A^Tx_{t-3}}{i+1}\)
\]

\noindent Once again, we multiply by $-4\eta^2$ and substitute:
\[
    \Delta^i_t - \Delta^i_{t-1} &= 4\eta^2\Delta^{i+1}_{t-1} -
    7\eta^2\Delta^{i+1}_{t-2} + 2\eta^2(\innerab{x_{t-2}}{x_{t-3}}{i+1} +
    \innerac{y_{t-2}}{y_{t-3}}{i+1}) \\
    &\qquad + 4\eta^3(\innerac{y_{t-2}}{A^Tx_{t-3}}{i+1} -
    \innerab{x_{t-2}}{Ay_{t-3}}{i+1})
\]

Now, we use the update step for time $t-2$. For all $t \geq 1$, this is
well-defined, since $x_{-1}$ and $y_{-1}$ are defined. To ensure that this step
is sound for $t = 0$ requires we define the following, where $X^+$ denotes the
generalized inverse:
\[
    x_{-2} = 4x_0 + \frac{1}{\eta}(A^T)^{+} y_0 \\
    y_{-2} = 4y_0 - \frac{1}{\eta}A^+ x_0
\]
We define these such that: $A^Tx_{-2} = 4A^Tx_0 + \frac{y_0}{\eta}$ and
$Ay_{-2} = 4Ay_0 - \frac{x_0}{\eta}$ (since $x_0 \in R(A)$ and $y_0 \in R(A^T)$,
and thus the following equalities hold:
\[
    x_0 = x_{-1} - 2\eta Ay_{-1} + \eta Ay_{-2} \\
    y_0 = y_{-1} + 2\eta A^T x_{-1} - \eta A^Tx_{-2}
\]

\noindent This allows us to use the following expansion freely for all $t \geq 2$:
\[
    x_{t-2} &= \x{}{-3}{-4} &&\implies x_{t-3} - 2\eta Ay_{t-3} = x_{t-2} - \eta Ay_{t-4} \\
    y_{t-2} &= \y{}{-3}{-4} &&\implies y_{t-3} + 2\eta A^Tx_{t-3} = y_{y-2} + \eta A^Tx_{t-4}
\]

\noindent We can gather the inner product terms and use this update rule to get
our final desired result:
\[
    \Delta^i_t - \Delta^i_{t-1} &= 4\eta^2\Delta^{i+1}_{t-1} -
    7\eta^2\Delta^{i+1}_{t-2} + 2\eta^2(\innerab{x_{t-2}}{x_{t-3} - 2\eta
    Ay_{t-3}}{i+1} + \innerac{y_{t-2}}{y_{t-3} + 2\eta A^Tx_{t-3}}{i+1}) \\[1ex]
    &= 4\eta^2\Delta^{i+1}_{t-1} - 5\eta^2\Delta^{i+1}_{t-2} -
    2\eta^3(\innerab{x_{t-2}}{Ay_{t-4}}{i+1} -
    \innerac{y_{t-2}}{A^Tx_{t-4}}{i+1})
\]
\end{prevproof}

\begin{prevproof}{Lemma}{lemma:restated bound}
\noindent To prove this, first consider the following trivial inequality:
\[
    \normac{y_{t-2} - A^Tx_{t-4}}{i+1} &+
    \normab{x_{t-2} + Ay_{t-4}}{i+1} \\
    &= \normac{y_{t-2}}{i+1} -
    2\innerac{y_{t-2}}{A^Tx_{t-4}}{i+1} + \normac{A^Tx_{t-4}}{i+1} \\
    &\qquad+ \normab{x_{t-2}}{i+1} +
    2\innerab{x_{t-2}}{Ay_{t-4}}{i+1} + \normab{Ay_{t-4}}{i+1} \\
    &\geq 0
\]

\noindent Rearranging:
\[
    2\innerac{y_{t-2}}{A^Tx_{t-4}}{i+1} - 2\innerab{x_{t-2}}{Ay_{t-4}}{i+1}
    &\leq \Delta_{t-2}^{i+1} + \(\normac{A^Tx_{t-4}}{i+1} +
    \normab{Ay_{t-4}}{i+1}\) \\
				&\leq \Delta_{t-2}^{i+1} + \lambda_\infty^2\(\normab{x_{t-4}}{i+1} +
    \normac{y_{t-4}}{i+1}\) \\
    &\leq \Delta_{t-2}^{i+1} +
    \(\lambda_\infty^2\Delta_{t-4}^{i+1}\)  \\
		    &\leq \Delta_{t-2}^{i+1} + \Delta_{t-4}^{i+1}
\]

%\vscomment{Why is the above last inequality correct? Also can we maybe also write an intermediate observation that $\|y\|_{A^T N_{i+1}^T}^2 = \|N_{i+1} A y\|_2^2$ and similarly for the $x$.}
%
%\noindent Now, using the update rule we know that:
%\[
    %\normab{x_{t-2}}{i} &= \normab{x_{t} + 2\eta Ay_{t-1} + \eta Ay_{t-2} - \eta
    %Ay_{t-3}}{i}\\
    %&\leq 8\normab{x_{t}}{i} + 8\eta^2\normab{ Ay_{t-1}}{i}+8\eta^2\normab{ Ay_{t-2}}{i} + 8\eta^2\normab{Ay_{t-3}}{i}\\
		%&\leq 8\normab{x_{t}}{i} + 8\eta^2\lambda_{\infty}^2\normac{ y_{t-1}}{i}+8\eta^2\lambda_{\infty}^2\normac{ y_{t-2}}{i} + 8\eta^2\lambda_{\infty}^2\normac{y_{t-3}}{i}
%\]
%Similarly:
%\[
%\normac{y_{t-2}}{i} \leq 8\normac{y_{t}}{i} + 8\eta^2\lambda_{\infty}^2\normab{ x_{t-1}}{i}+8\eta^2\lambda_{\infty}^2\normab{ x_{t-2}}{i} + 8\eta^2\lambda_{\infty}^2\normab{x_{t-3}}{i}
%\]	
%Hence, recalling that $\lambda_{\infty}\le 1$:
%\[	
    %\Delta_{t-2}^i &\leq 8\Delta_{t}^i + 8\eta^2 \Delta_{t-1}^i + 8\eta^2 \Delta_{t-2}^i + 8\eta^2 \Delta_{t-3}^i
		%\\
		   %&\leq \Delta_{t}^i + 4\eta(1-\eta^2)^{t-3}\Delta_{0}^{i+1} \\
    %&\leq \Delta_{t}^i + 4\eta\lambda_\infty\Delta_{0}^{i}
%\]
%%
%%
%\noindent Plugging the above inequality (for time $t-4$ and index $i+1$ on the left hand side)\footnote{If needed (for time $t=2$), we use the same construction of $x_{-2},y_{-2}$ found in the Proof of Lemma~\ref{lemma:first bound}.} in to our earlier bound we get:
%%Applying this for $t-4$ 
%\[
    %2\innerac{y_{t-2}}{A^Tx_{t-4}}{i+1} - 2\innerab{x_{t-2}}{Ay_{t-4}}{i+1}
    %&\leq \Delta_{t-2}^{i+1} +
    %\lambda_\infty^2\(8\Delta_{t-2}^{i+1} + 8\eta^2 \Delta_{t-3}^{i+1} + 8\eta^2 \Delta_{t-4}^{i+1} + 8\eta^2 \Delta_{t-5}^{i+1}\)  \\
		%&\leq 9\Delta_{t-2}^{i+1} + 8\eta^2 \Delta_{t-3}^{i+1} + 8\eta^2 \Delta_{t-4}^{i+1} + 8\eta^2 \Delta_{t-5}^{i+1}
    %%&\leq (1+\lambda_\infty^2)\Delta_{t-2}^{i+1} +
    %%4\eta\lambda^2_\infty\Delta_0^{i+1}
%\]

\noindent Now, we can apply this bound to the result of Lemma~\ref{lemma:first bound}:
\[
    \Delta^i_{t} - \Delta^i_{t-1} &= 4\eta^2\Delta^{i+1}_{t-1} -
    5\eta^2\Delta^{i+1}_{t-2} - 2\eta^3(\innerab{x_{t-2}}{Ay_{t-4}}{i+1} -
    \innerac{y_{t-2}}{A^Tx_{t-4}}{i+1}) \\
    &\leq 4\eta^2\Delta^{i+1}_{t-1} - 5\eta^2\Delta^{i+1}_{t-2} +
    \eta^3(\Delta_{t-2}^{i+1} + \Delta_{t-4}^{i+1})
\]

%\[
    %\Delta^i_{t} - \Delta^i_{t-1} &= 4\eta^2\Delta^{i+1}_{t-1} -
    %5\eta^2\Delta^{i+1}_{t-2} - 2\eta^3(\innerab{x_{t-2}}{Ay_{t-4}}{i+1} -
    %\innerac{y_{t-2}}{A^Tx_{t-4}}{i+1}) \\
    %&\leq 4\eta^2\Delta^{i+1}_{t-1} - 5\eta^2\Delta^{i+1}_{t-2} +
    %2\eta^3(9\Delta_{t-2}^{i+1} + 8\eta^2 \Delta_{t-3}^{i+1} + 8\eta^2 \Delta_{t-4}^{i+1} + 8\eta^2 \Delta_{t-5}^{i+1}) \\
%\]

\noindent Which is what we sought out to prove.
\end{prevproof}

%\begin{prevproof}{Lemma}{lemma:lower bound}
%Now, note that choosing $x_0 \in \mathcal{R}(A) \implies x_t \in
%\mathcal{R}(A) = \mathcal{R}(AA^T)\ \forall\ t$, due to the update step.
%Similarly, $y_t \in \mathcal{R}(A^T)$. Thus, $x_t = AA^T(AA^T)^{+}x_t$ and
%$y_t = A^TA(A^TA)^{+}y_t$, for all $t$. Letting $Q = (AA^T)^{+}$ and $P =
%(A^TA)^{+}$, and recalling key properties of the generalized inverse: 
%\[
    %\Delta_t^{i} &= \norm{M_{i}A^Tx_t}{2} + \norm{N_{i}Ay_t}{2} \\
       %&= \norm{M_iA^TAA^TQx_t}{2} +
    %\norm{N_iAA^TAPy_t}{2}\\
    %&= \norm{M_{i+2}A^TQx_t}{2} +
	%\norm{N_{i+2}APy_t}{2} \\ 
    %&= \norm{(A^T)^jQ(AA^T)^{k+1}x_t}{2} +
	%\norm{A^jP(A^TA)^{k+1}y_t}{2} \\
    %&= \begin{cases}
	%\norm{QM_{i+1}A^Tx}{2} + \norm{PN_{i+1}Ay_t}{2} &\text{ if $j = 0$}\\[0.5ex]
	%\norm{PM_{i+1}A^Tx}{2} + \norm{QN_{i+1}Ay_t}{2} &\text{ if $j = 1$}
    %\end{cases} \\
    %&\leq \max \(||Q||, ||P||\)\cdot \Delta_t^{i+1}
%\]
%\end{prevproof}

		\begin{prevproof}{Lemma}{lemma:semi-trivial upper bound}
		Suppose $j=i \mod 2$ and $k=(i-j)/2$. Notice the following identities:
		\begin{align*}
		&M_i = A^j(A^TA)^k, N_i = \(A^T\)^j \(AA^T\)^k\\
		&M_{i+1} = (A^T)^jA(A^TA)^k, N_{i+1} = A^j A^T\(AA^T\)^k
		\end{align*}
		Now:
		\begin{align*}
		\Delta^{i+1}_t &= \normlt{N_{i+1}Ay_t} + \normlt{M_{i+1}A^Tx_t}\\
		&=\normlt{A^j A^T\(AA^T\)^kAy_t} + \normlt{(A^T)^jA(A^TA)^kA^Tx_t}\\
		&\le \lambda_{\infty}^2\(\normlt{(A^T)^j\(AA^T\)^kAy_t} + \normlt{A^j(A^TA)^kA^Tx_t}\)\\
		&\le \lambda_{\infty}^2\(\normlt{N_iAy_t} + \normlt{M_iA^Tx_t}\)\\
		&\le \Delta^{i}_t,
		\end{align*}
		where for the last inequality we used that $\lambda_{\infty}\le 1$.
		\end{prevproof}
		
\begin{prevproof}{Lemma}{lemma:lower bound}
Given our initialization, $x_0 \in \mathcal{R}(A)$. This implies $x_t \in
\mathcal{R}(A), \forall\ t$, due to the update step of the dynamics. Recalling key properties of the matrix pseudoinverse, this implies: $x_t \equiv A A^+ x_t = A A^T (A A^T)^+ x_t$, for all $t$.
Similarly, given our initialization, $y_t \in \mathcal{R}(A^T)$, for all $t$, which implies $y_t \equiv A^T (A^T)^+ y_t = A^T A (A^T A)^+ y_t$, for all $t$. Letting $Q = (AA^T)^{+}$ and $P =
(A^TA)^{+}$, we get the following (where $j = i \mod 2$ and $k= (i-j)/2$): 
\[
    \Delta_t^{i} &= \norm{M_{i}A^Tx_t}{2} + \norm{N_{i}Ay_t}{2} \\
       &= \norm{M_iA^TAA^TQx_t}{2} +
    \norm{N_iAA^TAPy_t}{2}\\
    &= \norm{M_{i+2}A^TQx_t}{2} +
	\norm{N_{i+2}APy_t}{2} \\ 
	&= \norm{A^j (A^TA)^{k+1}A^T (AA^T)^{+} x_t}{2} +
	\norm{(A^T)^j (AA^T)^{k+1}A(A^TA)^+y_t}{2} \\ 
    &= \norm{A^j (A^TA)^{+} A^T (AA^T)^{k+1}x_t}{2} +
	\norm{(A^T)^j (AA^T)^{+} A(A^TA)^{k+1}y_t}{2} \\
    &= \begin{cases}
	\norm{(A^TA)^{+} M_{i+2}A^Tx_t}{2} + \norm{(AA^T)^{+}N_{i+2}Ay_t}{2} &\text{ if $j = 0$}\\[0.5ex]
	\norm{(AA^T)^{+}M_{i+2}A^Tx_t}{2} + \norm{(A^TA)^{+} N_{i+2}Ay_t}{2} &\text{ if $j = 1$}
    \end{cases} \\
    &\leq \max \(||Q||, ||P||\)^2\cdot \Delta_t^{i+2},
\]
where for the fourth and fifth equality we used the following key property of pseudo-inverses: $A^+=(A^TA)^+A^T=A^T(AA^T)^+$.
\end{prevproof}



\section{DNA-Generation WGAN Architecture}\label{sec:appendix-dna-arch}\label{sec:apdxdna}
\begin{table}[H]
\label{table:arch}
\begin{center}
\begin{tabular}{lllll}
\multicolumn{1}{c}{\bf Operation}  &\multicolumn{1}{c}{\bf Kernel} &\multicolumn{1}{c}{\bf Output Shape} &\multicolumn{1}{c}{\bf BatchNorm?} &\multicolumn{1}{c}{\bf Nonlinearity}
\\ \hline \\
Length of DNA sequence: $L = 6$ \\
Gradient penalty: $\lambda=1e^{-4}$\\
Batch size: $512$\\
$G(z):$ \\
$z$ & - & 50 & - & - \\
Fully connected         & - & 128 & no & tanh\\
Fully connected             & - & $16\times \frac{L}{2}$  & yes & tanh \\
Reshape  & - & $ 16 \times 1 \times \frac{L}{2}$ & - & -\\
Upsampling by 2 & - & $16 \times 1 \times L$ & - & - \\
 Convolution             & $[1\times 3]\times 4$ & $4\times1\times L$ & no & tanh  \\
 $D(x):$ \\
$x$   & - & $4 \times  1 \times L$  &- &-\\
  Convolution             & $[1\times 3] \times 16$ &  $16\times1\times L$ & no & tanh  \\
  Fully connected             & - & 32  & no & tanh \\
  Fully connected             & - & 1  & no & linear \\
\end{tabular}
\end{center}
\end{table}

\section{CIFAR10 WGAN Architecture}\label{sec:appendix-cifar10-arch}
\begin{table}[H]
\label{table:arch}
\begin{center}
\begin{tabular}{lllll}
\multicolumn{1}{c}{\bf Operation}  &\multicolumn{1}{c}{\bf Kernel} &\multicolumn{1}{c}{\bf Output Shape} &\multicolumn{1}{c}{\bf BatchNorm?} &\multicolumn{1}{c}{\bf Nonlinearity}
\\ \hline \\
Gradient penalty: $\lambda=10$\\
Batch size: 64 \\
$G(z):$ \\
$z$ & - & 100 & - & - \\
Fully connected         & - & 1024 & no & LeakyReLU\\
Fully connected             & - & $8192$  & yes & LeakyReLU \\
Reshape  & - & $ 128 \times 8 \times 8 $ & - & -\\
TransposedConv             & $[5\times 5]\times 128$ & $128\times16\times 16$ & yes & LeakyReLU  \\
Convolution             & $[5\times 5]\times 64$ & $64\times16\times 16$ & yes & LeakyReLU  \\
TransposedConv             & $[5\times 5]\times 64$ & $64\times32\times 32$ & yes & LeakyReLU  \\
Convolution             & $[5\times 5]\times 3$ & $3\times32\times 32$ & no & tanh  \\
 
 $D(x):$ \\
$x$   & - & $3 \times  32 \times 32$  &- &-\\
  Convolution             & $[5\times 5] \times 64$ &  $64\times32\times32$ & no & LeakyReLU  \\
   Convolution             & $[5\times 5] \times 128$ &  $128\times14\times 14$ & no & LeakyReLU  \\
    Convolution             & $[5\times 5] \times 128$ &  $128\times 7\times 7$ & no & LeakyReLU  \\
  Fully connected             & - & 1024  & no & LeakyReLU \\
  Fully connected             & - & 1  & no & linear \\
\end{tabular}
\end{center}
\end{table}

\newpage

\section{CIFAR10 Generator Image Samples}\label{sec:appendix-cifar10}
\begin{figure}[H]
    \centering  
    \begin{subfigure}[b]{1\textwidth}
        \centering
    		\includegraphics[]{optimAdam_v0_1e-04_ratio1_epoch93-eps-converted-to.pdf}
    	\caption{Sample of images from Generator of Epoch $94$, which had the highest inception score.}
    \end{subfigure}
    \caption{Samples of images from Generator trained via Optimistic Adam on CIFAR10.}\label{fig:optimistic-Adam}
\end{figure}

\subsection{Comparison of Early Epoch Images of Optimistic Adam vs Adam}
Below we give samples of images from an early epoch $19$ Generator trained via Optimistic Adam with 1:1 training ratio, Adam with 1:1 and Adam with 5:1 ratio on CIFAR10. We see that Optimistic Adam has already achieved visually appealing results unlike the latter two vanilla Adam based versions.

\begin{figure}[H]
        \centering
    		\includegraphics[height=4in]{optimAdam_v0_1e-04_ratio1_epoch19-eps-converted-to.pdf}
    	\caption{Sample of images from Generator of Epoch $19$ trained via Optimistic Adam and 1:1 training ratio.}
\end{figure}
\begin{figure}[H]
        \centering
    		\includegraphics[height=4in]{adam_v0_1e-04_ratio1_epoch19-eps-converted-to.pdf}
    	\caption{Sample of images from Generator of Epoch $19$ trained via  Adam and 1:1 training ratio.}
\end{figure} 
\begin{figure}[H]
        \centering
    		\includegraphics[height=4in]{adam_v0_1e-04_epoch19-eps-converted-to.pdf}
    	\caption{Sample of images from Generator of Epoch $19$ trained via  Adam and 5:1 training ratio.}
\end{figure}
    

\section{CIFAR10 Adam vs. Optimistic Adam Comparison}\label{sec:appendix-errorbars}
\begin{figure}[H]
    \centering  
    \includegraphics[width=0.9\textwidth]{cifar10-error-bars.pdf}
    \caption{The inception scores across epochs for GANs trained with Optimistic Adam (ratio 1) and Adam (ratio 5) on CIFAR10 (the two top-performing optimizers found in Section~\ref{sec:cifar10}, with 10\%-90\% confidence intervals. The GANs were trained for 30 epochs and results gathered across 35 runs.}\label{fig:optimistic-Adam}
\end{figure}


\end{document}
