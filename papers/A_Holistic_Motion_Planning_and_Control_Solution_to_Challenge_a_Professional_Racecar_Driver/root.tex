% \documentclass[letterpaper, 10 pt, conference]{ieeeconf}  % Comment this line out if you need a4paper
% \IEEEoverridecommandlockouts                              % This command is only needed if 
% \overrideIEEEmargins

\documentclass[letterpaper, 10 pt, journal, twoside]{ieeetran}
% Use this command for final RAL version


% Please add the following lines, customizing the underline italics fields as necessary:
% Paper headers
\markboth{IEEE Robotics and Automation Letters. Preprint Version. Accepted July, 2021}
{Srinivasan \MakeLowercase{\textit{et al.}}: A Holistic Motion Planning and Control Solution to Challenge a Professional Racecar Driver} 
% Use only for final RAL version

% \documentclass[journal]{IEEEtran}
% \documentclass[letterpaper, 10 pt, conference]{ieeeconf}
% \IEEEoverridecommandlockouts % This command is only needed if you want to use the \thanks command
\usepackage{mathtools}
\usepackage{bbding}
% \overrideIEEEmargins                                      % Needed to meet printer requirements.

% *** CITATION PACKAGES *** %
\usepackage{cite}

% *** GRAPHICS RELATED PACKAGES *** %
\usepackage[pdftex]{graphicx}
% declare the path(s) where your graphic files are
\graphicspath{{figures/}}
\DeclareGraphicsExtensions{.pdf,.jpeg,.jpg,.png}

% *** MATH PACKAGES *** %
\usepackage{amsmath}

% *** PDF, URL AND HYPERLINK PACKAGES *** %
\usepackage{url}

% Add all sections as separate files
\usepackage{subfiles}

% *** Do not adjust lengths that control margins, column widths, etc. ***
% *** Do not use packages that alter fonts (such as pslatex).         ***

\usepackage{xcolor}
\usepackage{booktabs}
\usepackage{tabularx,ragged2e}

% Clever references
\usepackage{hyperref}
\usepackage[nameinlink,noabbrev,capitalise]{cleveref}

% Comments
\usepackage{comment}
% Toggle next lines to display and hide comments
\newcommand{\sir}[1]{\textcolor{blue}{#1}}
\newcommand{\seb}[1]{\textcolor{blue}{#1}}
\newcommand{\alex}[1]{\textcolor{blue}{#1}}
\usepackage[normalem]{ulem}

% correct bad hyphenation here
\hyphenation{op-tical net-works semi-conduc-tor}


% Toggle to hide TODOs or make visible
% \newcommand{\mycommenttodo}[1] {{}}
\newcommand{\mycommenttodo}[1] {{#1}}
\definecolor{CommentRed}{rgb}{1.0,0,0}
\newcommand{\TODO}[1] {{\color{CommentRed} {\mycommenttodo{TODO: \textbf{#1}}}}}

%
% paper title
% Titles are generally capitalized except for words such as a, an, and, as,
% at, but, by, for, in, nor, of, on, or, the, to and up, which are usually
% not capitalized unless they are the first or last word of the title.
% Linebreaks \\ can be used within to get better formatting as desired.
% Do not put math or special symbols in the title.
\title{A Holistic Motion Planning and Control Solution \\to Challenge a Professional Racecar Driver}

% author names and IEEE memberships
% note positions of commas and nonbreaking spaces ( ~ ) LaTeX will not break
% a structure at a ~ so this keeps an author's name from being broken across
% two lines.
% use \thanks{} to gain access to the first footnote area
% a separate \thanks must be used for each paragraph as LaTeX2e's \thanks
% was not built to handle multiple paragraphs
%
\author{Sirish~Srinivasan$^{*}$, Sebastian~Nicolas~Giles$^{*}$, Alexander~Liniger%
\thanks{Manuscript received: February, 24, 2021; Revised June, 11, 2021; Accepted July, 14, 2021. This paper was recommended for publication by Editor Stephen J. Guy upon evaluation of the Associate Editor and Reviewers' comments.}
\thanks{Sirish Srinivasan and Sebastian Nicolas Giles are with AMZ Driverless, ETH Z\"urich, 8092 Z\"urich, Switzerland (e-mail: sirishs@ethz.ch; sgiles@ethz.ch)}%
\thanks{Alexander Liniger is with the Computer Vision Lab, ETH Z\"urich, 8092 Z\"urich, Switzerland (e-mail: alex.liniger@vision.ee.ethz.ch)}
\thanks{$^*$ The authors contributed equally to this work.}
\thanks{A supplementary video providing a high level overview of the approach, along with a demonstration of the experimental results is available.}
\thanks{Digital Object Identifier (DOI): see top of this page.}
}%

% note the % following the last \IEEEmembership and also \thanks - 
% these prevent an unwanted space from occurring between the last author name
% and the end of the author line. i.e., if you had this:
% 
% \author{....lastname \thanks{...} \thanks{...} }
%                     ^------------^------------^----Do not want these spaces!
%
% a space would be appended to the last name and could cause every name on that
% line to be shifted left slightly. This is one of those "LaTeX things". For
% instance, "\textbf{A} \textbf{B}" will typeset as "A B" not "AB". To get
% "AB" then you have to do: "\textbf{A}\textbf{B}"
% \thanks is no different in this regard, so shield the last } of each \thanks
% that ends a line with a % and do not let a space in before the next \thanks.
% Spaces after \IEEEmembership other than the last one are OK (and needed) as
% you are supposed to have spaces between the names. For what it is worth,
% this is a minor point as most people would not even notice if the said evil
% space somehow managed to creep in.

\begin{document}

% make the title area
\maketitle
\begin{abstract}
We present a holistically designed three layer control architecture capable of outperforming a professional driver racing the same car. Our approach focuses on the co-design of the motion planning and control layers, extracting the full potential of the connected system. First, a high-level planner computes an optimal trajectory around the track, then in real-time a mid-level nonlinear model predictive controller follows this path using the high-level information as guidance. Finally a high frequency, low-level controller tracks the states predicted by the mid-level controller. Tracking the predicted behavior has two advantages: it reduces the mismatch between the model used in the upper layers and the real car, and allows for a torque vectoring command to be optimized by the higher level motion planners. The tailored design of the low-level controller proved to be crucial for bridging the gap between planning and control, unlocking unseen performance in autonomous racing. The proposed approach was verified on a full size racecar, considerably improving over the state-of-the-art results achieved on the same vehicle. Finally, we also show that the proposed co-design approach outperforms a professional racecar driver.

\end{abstract}

% Note that keywords are not normally used for peerreview papers.
\begin{IEEEkeywords}
Motion and Path Planning, Field Robots, Intelligent Transportation Systems.
\end{IEEEkeywords}

\IEEEpeerreviewmaketitle

\section{Introduction}
\label{sec:intro}
\subfile{sections/introduction.tex}

\section{Low Level Control Design}
\label{sec:llc_formulation}
\subfile{sections/llc_formulation.tex}

\section{Model}
\label{sec:model}
\subfile{sections/vehicle_model.tex}

\section{High Level Control Formulation}
\label{sec:hlc_formulation}
\subfile{sections/hlc_formulation.tex}

\section{Results and Discussion}
\label{sec:results}
\subfile{sections/results.tex}

\section{Conclusion}
\label{sec:conclusion}
\subfile{sections/conclusion.tex}


% use section* for acknowledgment
\section*{Acknowledgment}

We would like to thank the entire AMZ Driverless team, this work would not have been possible without the effort of every single member, and we are glad for having the opportunity to work with such amazing people. We would also like to thank the numerous alumni for the insightful discussions.

% references section

\bibliographystyle{IEEEtran}
\bibliography{root.bbl}

\end{document}
