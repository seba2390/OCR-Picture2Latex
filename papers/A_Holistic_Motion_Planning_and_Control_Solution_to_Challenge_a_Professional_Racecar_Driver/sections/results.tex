
We first discuss our robotic platform - the autonomous racecar \emph{pilatus driverless}, followed by implementation details. Thereafter, we benchmark our control approach in simulation using a realistic vehicle simulator to study the low-level adaptations. Finally, we present experimental results including an in-depth comparison with a professional racing driver. 

\subsection{pilatus driverless}
\label{sec:car_specs}
All our experiments are performed using the autonomous racecar \emph{pilatus driverless} (shown in \cref{fig:pilatus_driving}). \emph{pilatus} is a lightweight single seater race car, with an all-wheel drive electric powertrain. The racecar can produce $\pm375$\,Nm of torque at each wheel by means of four independent $38.4$\,kW motors. Our racecar can accelerate from $0-100$\,km/h in $2.1$\,s and can reach lateral accelerations of over 20\,m/s$^2$. \emph{pilatus} is equipped with a complete sensor suite including two LiDARs, three cameras, an optical ground-velocity sensor and two IMUs. The low-level control as well as the state estimation are deployed on an ETAS ES900 real-time embedded system; the remainder of the Autonomous System (AS), including mapping and localization, runs on an Intel Xeon E3 processor, see \cite{valls2018design,gosala2019redundant,Kabzan2019_AMZ,srinivasan2020end,andresen2020accurate} for more details. 

\subsection{TRO and MPC-Curv Implementation Details} \label{sec:TRO_imp}
TRO uses a spatial discretization of $\Delta s = 0.5$\,m. Given this discretization, the optimization problem defined in \eqref{eq:TRO} is solved in $\sim 5$\,s on an Intel Xeon E3 processor. We run MPC-Curv at a frequency of 40Hz, with a time scaling $\sigma = 1.5$. We use a prediction horizon of $T=40$ which results in a time horizon of 1.5\,s, using the time scaling. 

\subsection{Simulation Study}
\label{sec:results_sim}

To highlight the benefits achieved by co-designing the high and low-level controllers, we compare our full pipeline against a modified version of it that cannot optimize over torque vectoring and uses the more basic LLC as in \cite{vazquez2020optimization}. Note that the second method is similar to \cite{vazquez2020optimization}. We perform a simulation on the Formula Student Germany racetrack from 2018, using a high fidelity vehicle dynamics simulator. As an upper performance bound we also include the TRO solution, which uses the dynamic bicycle model \eqref{eq:dynamics} with no mismatch. \cref{fig:sim_vx_progress} shows the longitudinal velocity $v_x$ against the progress along the track. From the comparison, it is clearly visible that our approach can reach higher speeds. The lap times of the three methods confirm this: our full system completes one lap in 19.9\,s whilst without the low-level adaptations the lap time is 22.1\,s, for reference, the TRO optimal lap time is 18.0\,s.

\begin{figure}[h]
\begin{center}
\includegraphics[width=0.8\columnwidth] {sim_progress_vx}
\end{center}
	\caption{Simulated longitudinal velocity against track position, comparing our approach with and w/o the proposed low-level adaptations, against TRO.}
	\label{fig:sim_vx_progress}
\end{figure}

\subsection{Experimental Comparison}
\label{sec:results_exp}

For our comparison with a professional racing driver, we setup a race track compliant with the Formula Student Driverless regulations \cite{fsg-rules}, composed of sharp turns, straights and chicanes.
We tried to keep the comparison between the human driver and our proposed solution as fair as possible, however, there are some differences. First, in autonomous mode, the car is lighter due to the absence of a driver. Second, for safety, the AS speed was limited to 18\,m/s, whereas the human driver was unrestricted and reaches speeds up to 22\,m/s. Third, the AS can only use regenerative braking, since the hydraulic brakes are exclusively used for an emergency braking system. Since this is not necessary for the human driver, the hydraulic brakes can also be actuated, allowing higher deceleration. We would also like to note that the steering system used by the AS is slower than the steering actuated by a driver. 

The experiments were run consecutively, with one run for each of the driving modes. The duration was of $12$ and $18$ laps for the human and AS respectively. In both cases, the car remained within the track and did not hit any cones.

\subsubsection{Lap-time comparison}
\label{sec:laptime_comp}
All of the lap times recorded during the experiment are shown in \cref{tab:laptime_comp} and \cref{fig:laptime_comp}, no data was discarded.
Our proposed controller achieves both the lowest average and minimum lap times. In \cref{fig:laptime_comp} we can see that the autonomous system achieves six laps that are faster than the fastest lap by the professional human driver. Note that the variability of lap times in the autonomous mode comes from a few instances where an emergency planner is triggered and automatically slows down the car to avoid leaving the track. This can occur during large combined slip (corner exit), an issue we want to tackle in future work.

\begin{table}[h]
\centering
\caption{Lap-time comparison between human and autonomous driver}
\label{tab:laptime_comp}
\renewcommand\baselinestretch{1.0}
\begin{tabularx}{0.32\textwidth}{l c c}
    \toprule
    & Human & Autonomous \\
    \midrule
    Best lap-time(s) & 13.62 & 13.39 \\
    Mean lap-time(s) & 14.19 & 13.95 \\
    \bottomrule
\end{tabularx}
\end{table}

\begin{figure}[h]
\centering
\includegraphics[width=0.95\columnwidth] {times}
	\caption{Lap-time distribution for autonomous system and human driver.}
	\label{fig:laptime_comp}
\end{figure}
  
\subsubsection{Driving comparison}
\label{sec:drive_comp}

\begin{figure}[h]
\centering
\includegraphics[width=0.8\columnwidth] {track}
	\caption{GPS trajectory comparison of the best autonomous and human driven laps. The driving direction is clockwise.}
	\label{fig:path_comp}
\end{figure}

\cref{fig:path_comp} shows the comparison of the driven GPS paths for the autonomous and human modes, and \cref{fig:lap_comp} compares the longitudinal velocity of all driven laps. One major difference between the driven paths can be seen at the increasing radius curve on the left extreme. The AS does not follow the intuitive inner radius of the curve but goes wide, which allows later braking and a faster and straighter curve exit, which we can see in \cref{fig:lap_comp} at 100\,m. A similar difference can also be seen in the curve at the right extreme, where especially the last chicane before the curve entry and the curve exit are driven faster. The offline TRO problem can efficiently perform such trade-offs between travelled distance and speed, while even a professional driver needs significant track time to evaluate such trade-offs. 

\begin{figure}[h]
\centering
% \vspace{0.2cm}
\includegraphics[width=0.8\columnwidth] {lap_analysis}
	\caption{Comparison of the vehicle dynamics on the best lap. The velocity in all of the laps is also shown in a lighter color.}
	\label{fig:lap_comp}
\end{figure}

The other significant difference we can see in \cref{fig:lap_comp} is the effect of the 18\,m/s speed limiter for the AS. However, our controller is able to brake later and accelerate earlier at several locations along the lap, which in total results in lower lap-times, even with the top speed disadvantage. 

\subsubsection{Inputs comparison}
\label{sec:veh_comp}
The low-level inputs are shown in \cref{fig:lap_comp}. The AS in contrast to the human driver, makes limited use of the steering. 
This is compensated using more torque vectoring, which provides a faster response of the lateral dynamics. This would not be possible without the strong coupling at the core of our holistic architecture. Further, the high-frequency component of the yaw moment is the intended effect of the LLC tracking to compensate for model mismatch.

The final difference is the relative total torque input, where we can see that the AS demands less torque. This is due to the speed limit, and the fact that the friction ellipse constraint limits the torque before the traction controller. The human driver on the other hand relies on the traction controller in certain situations, e.g., at 100\,m. 

\subsubsection{Maximum performance}
\label{sec:gg}

\cref{fig:gg} compares the longitudinal and lateral accelerations recorded by the car in autonomous and human driven modes. We can see that the highest lateral, positive longitudinal and combined accelerations are achieved by the AS. The human driver has a higher negative longitudinal acceleration, due to the availability of hydraulic brakes, which are not used by the AS. Finally, we can also see the drastic performance difference between our approach and \cite{vazquez2020optimization}, while using the same car.

\begin{figure}[h]
% \vspace{0.2cm}
\begin{center}
\includegraphics[width=0.9\columnwidth] {g-g_all}
\end{center}
	\caption{Longitudinal and lateral acceleration comparison between our system, the professional human driver and Vazquez et.al \cite{vazquez2020optimization}}
	\label{fig:gg}
\end{figure}


