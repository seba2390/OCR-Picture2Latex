\newcommand*\target[1]{\bar{#1}}

Our fully autonomous racecar (see \cref{sec:car_specs} for complete specifications) is equipped with four wheelhub motors that can be controlled independently. Computing the references for each motor creates the need for a LLC, but also allows for significant freedom in its design.
Going beyond basic wheel torque distribution and stabilization\cite{Kabzan2019_AMZ, roborace_llc}, we define a novel input interface for our LLC, consisting of the desired short-term trajectories of key vehicle states, namely the longitudinal acceleration $\bar{a}_x$, yaw rate $\bar r$ and steering angle $\bar \delta$.
Note that a bar on top of a variable denotes a target, while no bar refers to the actual variable.
Importantly, the drive force and torque vectoring commands used by the NMPC and described in Section \ref{sec:curv_model} are purely virtual and are not directly passed to the LLC.
This is the fundamental idea for our high-level to low-level coupling: the state predictions by the NMPC are processed and passed to the LLC which appropriately commands the individual actuators to reproduce them.
The low-level feedback loop actively makes the car behave more like the NMPC model, mitigating the effect of model mismatch in the NMPC.

\subsection{Wheel Torque Controller}

The LLC operates at 200Hz, significantly faster than the 40Hz of the upper level NMPC which sends the target trajectories. To exploit the higher bandwidth, we generate in-between target values using linear up-sampling of the MPC target trajectories. We correct for the delay in the robotic system by slightly shifting the target points in time. 

The LLC computes the torques of the four wheels in a two step approach. First, the required yaw moment and total torque are determined. Second, the individual wheel torques are computed fulfilling these requirements.

The yaw moment is computed using a proportional controller to track the target yaw rate $\tau_{z} = K_P (\target{r}-r)$. The total torque demand is computed using a PID-controller for the target longitudinal acceleration as $\tau_{\text{total}} = PID(\target{a}_x-a_x) + m \target{a}_x + q$, where the feed forward part is designed such that $m$ approximates the inertia of the vehicle including the drive train and $q$ the effect of drag. The computation of $\tau_{z}$ and $\tau_{\text{total}}$ is the main difference to the LLC used in \cite{vazquez2020optimization} and by the human driver in Section \ref{sec:results_exp} where the steering angle is used to compute the target yaw rate \cite{milliken}, and the driver command is directly mapped to the total torque demand.

An initial torque distribution is then computed by splitting the total torque $\tau_{\text{total}}$ equally between the left and right sides of the car. Further, the individual torques are scaled proportional to the normal force $F_z$ on each wheel, resulting in $\tau'=[\tau'_{FL}, \tau'_{FR},\tau'_{RL},\tau'_{RR}]$. The torque vectoring algorithm then determines the torque differences $\Delta\tau_F$ and $\Delta\tau_R$, which adjusts the wheel torques to $\tau=\tau' + [ \Delta\tau_F, -\Delta\tau_F, \Delta\tau_R, -\Delta\tau_R]$. The torque differences $\Delta\tau_F$ and $\Delta\tau_R$ are computed such that the desired yaw moment $\tau_z$ is produced, accounting for the effect of the drive force of the angled front wheels. Since the torque difference neglects the load distribution, in a final step, $\Delta\tau_F$ and $\Delta\tau_R$ are distributed proportional to the vertical tire load on each axle. This allows, for example, to use mainly the rear wheels for torque vectoring during a corner exit. 

The resulting wheel torques are finally tracked by a drive motor controller operating at 1kHz which also implements a traction controller to adapt the reference torque if a slip-ratio-based wheel speed range is violated \cite{ev_traction_control}. 

\subsection{Steering delay compensation module}
The position tracking delay of a servo steering system, due to mechanical compliance and limited power, can severely hinder driving.
To mitigate this we propose a virtual target point $\target{\delta}_{\text{act}}$ as a reference for the steering positioning actuator, based on an external steering shaft sensor measurement $\delta$.
The virtual target point is estimated as 
$\target{\delta}_{\text{act}} = \target{\delta} + K_P (\target{\delta}-\delta) + K_D (\dot{\target{\delta}} - \dot{\delta})$, where $\dot{\target{\delta}}(t)$ is the steering rate from the MPC predictions. 
