Given the LLC, we introduce the vehicle model used in the higher level motion planners. Similar to \cite{vazquez2020optimization} we formulate a dynamic bicycle model in curvilinear coordinates, but we modify the interface with our LLC (See \cref{sec:llc_formulation}). 

\subsection{Curvilinear Dynamic Bicycle Model}
\label{sec:curv_model}
Curvilinear coordinates describe a coordinate frame (Frenet frame) formulated locally with respect to a reference path, drastically simplifies path following formulations. In our case the reference path can be the center line or the track-level optimized path. The kinematic states in the curvilinear setting describe the state relative to the reference path and are the progress along the path $s$, the deviation orthogonal to the path $n$, and the local heading $\mu$. Note that the dynamic states are not influenced by the change in the coordinate system, and in our model we consider the longitudinal $v_x$ and lateral velocities $v_y$, and yaw rate $r$. A visualization of the curvilinear coordinates as well as the other states is shown in \cref{fig:model}. 

\begin{figure}[h]
    \centering
    % \vspace{0.2cm}
    \includegraphics[width=0.35\textwidth]{CurvilinearModel}
    \caption{Visualization of the curvilinear coordinates (blue), as well as the dynamic states (green) and the forces (red).}
    \label{fig:model}
\end{figure}

Since the LLC handles traction control and considering load changes, we can use a relatively simple model in our higher levels. We follow the popular modeling approach successfully used in \cite{Liniger2015,Kabzan2019_AMZ,vazquez2020optimization}, and uses a dynamic bicycle model with Pacejka tire models. Compared to \cite{vazquez2020optimization} we include some important differences - first we assume that the force of the motors $F_M$ can be directly controlled, and for simplicity assume that the same motor force is applied at the front and rear wheels. This approach is better aligned with our LLC, which is designed to track an acceleration target and does not expect a driver command as in \cite{vazquez2020optimization}. Second, we introduce the torque vectoring moment $M_{tv}$ as an input. This is in contrast to \cite{Kabzan2019_AMZ,vazquez2020optimization} where the torque vectoring was determined by a simple P-controller. This input is fundamental, as it allows the higher level controllers to fully utilize the torque vectoring capabilities, going beyond a simple rule-based method designed for human drivers. We use a state lifting technique to consider input rates, but do not lift $M_{tv}$ to allow the high level controllers to use the torque vectoring for highly transient situations. The state is given by $\mathbf{\tilde{x}} = [s, n, \mu, v_x, v_y, r, F_M, \delta]^T$ and the input as $\mathbf{u} = [\Delta F_M,\Delta \delta,  M_{tv}]^T$. Resulting in the following system dynamics,
\begin{align}
\label{eq:dynamics}
    &\dot{s} = \dfrac{v_x\cos{\mu}-v_y\sin{\mu}}{1-n \kappa(s)}\,, \nonumber\\
    &\dot{n} = v_x\sin{\mu}+v_y\cos{\mu}\,, \nonumber\\
    &\dot{\mu} = r - \kappa(s) \dot{s}\,, \\
    &\dot{v}_x = \tfrac{1}{m}(F_M (1+\cos{\delta}) - F_{y,F} \sin{\delta} + m v_y r - F_{\text{fric}})\,, \nonumber\\
    &\dot{v}_y = \tfrac{1}{m}(F_{y,R} + F_M  \sin{\delta} + F_{y,F} \cos{\delta} - m v_x r)\,,\nonumber\\
    &\dot{r} = \tfrac{1}{I_z}((F_M  \sin{\delta} + F_{y,F} \cos{\delta})l_F - F_{y,R}l_R + M_{tv})\,,\nonumber \\
    &\dot{F}_M = \Delta F_M\,, \nonumber \\
    &\dot{\delta} = \Delta \delta\,, \nonumber
\end{align}
where $l_F$ and $l_R$ are the distances from the Center of Gravity (CoG) to the front and rear wheels respectively, $m$ is the mass of the vehicle and $I_z$ the moment of inertia. Finally, $\kappa(s)$ is the curvature of the reference path at the progress $s$. We denote the dynamics in \eqref{eq:dynamics} as $\mathbf{\dot{\tilde{x}}} = f^c_t(\mathbf{\tilde{x}}, \mathbf{u})$, where the superscript $c$ highlights that it is a continuous model and the subscript $t$ that it is a time-domain model.

The lateral forces at the front $F_{y,F}$ and rear $F_{y,R}$ tires are modeled using a simplified Pacejka tire model \cite{pacejka1992magic},
\begin{align}
\label{eq:lateral_forces}
\begin{split}
    F_{y,F} & = F_{N,F} D\sin{(C\arctan{(B\alpha_F)})}\,, \\
    F_{y,R} & = F_{N,R} D\sin{(C\arctan{(B\alpha_R)})}\,, \\
\end{split}
\end{align}
where $\alpha_F = \arctan{( \frac{v_y + l_F r}{v_x} )} - \delta$ and $\alpha_R = \arctan{( \frac{v_y - l_R r}{v_x})}$ are the slip angles at the front and rear wheels respectively, and $B$, $C$ and $D$ are the parameters of the simplified Pacejka tire model. The net normal load $F_{N,net} = m g + C_l v_x^2$, where $C_l$ is a lumped lift coefficient. Compared to \cite{vazquez2020optimization} we also consider the aerodynamic downforce, which is important since we push the car to the limit of friction. The resulting normal loads on the front and rear tires are given by $F_{N,F} = F_{N,net} l_R /(l_F + l_R) $ and $F_{N,R} = F_{N,net} l_F /(l_F + l_R)$. Finally, the friction force $F_{\text{fric}}$ is a combination of a static rolling resistance $C_r$ and the aerodynamic drag term $C_d v_x^2$.

\subsection{Constraints}
\label{sec:constraints}
Similar to \cite{vazquez2020optimization} we impose constraints to ensure that the car remains within the track, and that we do not demand inputs that violate friction ellipse or input constraints. More precisely we have a track constraint $\mathbf{\tilde{x}} \in \mathcal{X}_{\text{Track}}$, which is a heading-dependent constraint on the lateral deviation $n$, and ensures that the whole car remains inside the track \cite{liniger2020safe}, 
\begin{align}
\label{eq:boundary}
\begin{split}
    n + L_c\sin{|\mu|} + W_c\cos{\mu} & \leq \mathcal{N}_L(s) \,,\\
    -n + L_c\sin{|\mu|} + W_c\cos{\mu} & \leq \mathcal{N}_R(s)\,,
\end{split}
\end{align}
where $L_{c}$ and $W_c$ are the distances from the CoG to the furthest apart corner point of the car, and $\mathcal{N}_{R/L}(s)$ are the left and right track width at the progress $s$. The tire models used \eqref{eq:lateral_forces} do not consider combined slip. To prevent the high level layers to demand unrealistic accelerations from the LLC, we limit the combined forces to remain within a friction ellipse, 
\begin{align}
\label{eq:ellipse}
\begin{split}
(\rho_{long}F_M)^2 + F_{y,F/R}^2 & \leq (\lambda D_{F/R})^2\,, \\
\end{split}
\end{align}
where $\rho_{long}$ defines the shape of the ellipse, and $\lambda$ determines the maximum combined force. We denote the friction ellipse constraints in \eqref{eq:ellipse} by $\mathbf{\tilde{x}} \in \mathcal{X}_\text{FE}$.

Finally, we consider box constraints for both the physical inputs and their rates. We introduce a compact notation for all these inputs $\mathbf{a} = [F_M, \delta, \Delta F_M, \Delta \delta, M_{tv}]^T$, and constrain them to their physical limits by the box constraint $\mathbf{a} \in \mathcal{A}$.
