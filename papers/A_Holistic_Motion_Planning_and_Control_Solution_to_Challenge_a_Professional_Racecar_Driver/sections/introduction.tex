% \IEEEPARstart{A}{utonomous} 
\IEEEPARstart{A}{utonomous} driving has progressed massively over past few decades, from its humble beginnings in the 1980s \cite{dickmanns1987,Pomerleau1989}, over the DARPA challenges \cite{buehler2005,buehler2009}, to the self-driving car companies of today. One goal for autonomous driving has always been to surpass human driving capabilities. This is especially true for autonomous racing, where professional racecar drivers are a challenging benchmark.
However, most existing methods fall short of this goal \cite{hermansdorfer2020benchmarking}.
In the last years, several motion planning methods for autonomous racing have emerged \cite{Gerdes2012,Liniger2015}. In this paper, we introduce a holistic view-point on motion planning and control of autonomous racecars, and show that the co-design of all layers from track-level trajectory planning to low-level control of the vehicle dynamics results in an unseen performance on a full-sized autonomous racecar. In fact our proposed approach achieves higher driving performance and lower lap times than a professional racecar driver, both driving the same Formula Student Driverless car developed by AMZ Racing, from ETH Z\"urich.

Most autonomous racing motion planners and controllers can be divided into three levels. The first level is track-level planning, where the race line around the track is determined. This can be done using either lap-time optimization methods \cite{lot2014curvilinear,Rucco2015,vazquez2020optimization}, or using the center line \cite{Liniger2015,Kabzan2019_AMZ,Rosolia2017}. The mid-level is tasked to follow the track-level path, and is normally based on two common approaches - static feedback controllers \cite{Gerdes2012,Betz2019,TALVALA2011137} or online optimization-based methods \cite{Liniger2015,funke2016collision,caporale2019towards,williams2016aggressive} such as Nonlinear Model Predictive Control (NMPC). The last level is the Low-Level vehicle Control (LLC) which handles steering and stability control, and typically hosts torque vectoring algorithms which are also beneficial for a human driver \cite{human_llc}.
Even though this layer is fundamental, it is often under-explored, especially the coupling with the mid-level.
There exist several works that make use of a sophisticated LLC. However, these are either designed for human drivers \cite{Kabzan2019_AMZ,vazquez2020optimization} or do not exploit that the level above is an automatic controller \cite{TALVALA2011137,Chatzikomis_2018}. 
In this work, we show that co-designing the LLC to work in harmony with the higher levels allows improving the performance of the autonomous racecar drastically. We achieve this by tracking parts of the mid-level NMPC state trajectory with the LLC. This reinforces recent work which showed that better coupling the track and mid-level controllers \cite{vazquez2020optimization,Novi2019} can improve the performance, and \cite{Chatzikomis_2018} that highlighted the benefits of torque vectoring for autonomous cars. 

\begin{figure}[t]
\begin{center}
\includegraphics[width=0.8\columnwidth] {Pilatus_driving}
\end{center}
	\caption{\emph{pilatus} driverless, the formula student race car \copyright FSG - Schulz.}
	\label{fig:pilatus_driving}
\end{figure}


A different view point supporting the coupling of different levels is based on model mismatch. Several papers discovered model mismatch as a crucial issue in autonomous racing - the problem arises due to the relatively simple models used in most autonomous racing stacks. Solutions range from using complex models \cite{Novi2019}, stochastic MPC \cite{carrau2016efficient}, to NMPC with model learning \cite{Kabzan2019_learning,Brunner2017} to learn the model mismatch. All these methods tackle the problem in the mid-level and come with drawbacks in terms of the computational load. However, using our co-design approach, we can use the LLC to make the real-car behave closer to the model of the NMPC.

\begin{figure*}[t]
% \vspace{0.2cm}
\centering
\includegraphics[width=0.8\textwidth] {full_diagram}
\caption{Full planning and control architecture.
The feedback from the state estimator is omitted for simplicity. 
}
\label{fig:tv}
\end{figure*}

In this paper we extend the approach proposed in \cite{vazquez2020optimization} and highlight the importance of properly coupled high and low-level controllers. Our contributions are threefold: 
\begin{itemize}
    \item We propose a new low-level controller (LLC) designed to distribute the motor torques of our all-wheel drive race car to reduce the model mismatch between the real car and the model used in the higher level layers.
    \item We propose to directly optimize over a torque vectoring input in the higher level controllers, while interfacing it with the LLC in terms of a yaw-rate target trajectory. This allows to extract the full potential out of our LLC.
    \item We show that the proposed framework can drastically improve the performance over the current state-of-the-art autonomous racing systems, both in simulation and experiments. In fact our method is the first which performs on par with a professional racecar driver, even outperforming the driver in our experimental setup.
\end{itemize}
In this work, attention is given to the LLC formulation and its coupling with the upper control layers, complementing \cite{vazquez2020optimization} which focused on the high and mid-level and their coupling. 
This is also the main difference to other NMPC-based methods \cite{Liniger2015,Rosolia2017,williams2016aggressive} which do not focus on the LLC or torque vectoring. 
With respect to other works that considered LLC, ours is distinguished by the novel interface between the levels:
in \cite{Chatzikomis_2018,roborace_llc,vazquez2020optimization,Kabzan2019_AMZ} the low-level targets are mainly determined by the steering angle, in contrast, we incorporate the yaw rate of the mid-level NMPC. 
In \cite{TALVALA2011137, Gerdes2012}, the focus is on traction control which is a subtask of our LLC.

\section{Co-designed Controller Architecture}

As discussed in the introduction, we build upon \cite{vazquez2020optimization} but introduce several fundamental changes highlighted in our first two contributions. However, we keep a similar architecture for the track and mid-level layers, which are the focus of \cite{vazquez2020optimization}. Our full architecture is shown in \cref{fig:tv}. We assume a reference path from a mapping run and offline compute an optimal trajectory around the track using our Trajectory Optimization (TRO) module. This path is then followed by the MPC-Curv module, using the terminal velocity constraint from the TRO module. However, in contrast to \cite{vazquez2020optimization}, the optimal solution from MPC-Curv is then translated to set point trajectories for the acceleration, yaw rate and steering, which are tracked using our new LLC. 
At the same time, the redeveloped vehicle model makes the motion planning levels aware of the torque vectoring capability of the car and enables optimization over a new yaw moment command. In addition to these two main differences from \cite{vazquez2020optimization}, we also modify the TRO and MPC-Curv modules, resulting in a drastically improved driving performance.
