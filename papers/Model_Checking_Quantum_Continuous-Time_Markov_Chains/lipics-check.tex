\documentclass[a4paper,UKenglish,cleveref,autoref,thm-restate,authorcolumns]{lipics-v2019}
%This is a template for producing LIPIcs articles. 
%See lipics-manual.pdf for further information.
%for A4 paper format use option "a4paper", for US-letter use option "letterpaper"
%for british hyphenation rules use option "UKenglish", for american hyphenation rules use option "USenglish"
%for section-numbered lemmas etc., use "numberwithinsect"
%for enabling cleveref support, use "cleveref"
%for enabling autoref support, use "autoref"
%for anonymousing the authors (e.g. for double-blind review), add "anonymous"
%for enabling thm-restate support, use "thm-restate"

%\graphicspath{{./graphics/}}%helpful if your graphic files are in another directory

\usepackage{amsmath,amssymb}
\usepackage{bm,upgreek}
\usepackage[noend]{algpseudocode}
\usepackage{algorithm,algorithmicx}
\usepackage{pgf,tikz}
\usetikzlibrary{shapes,arrows,automata}
\usepackage{textcomp}
\usepackage{hyperref}
\usepackage{array}
\usepackage[braket,qm]{qcircuit}
\usepackage{dsfont}
\usepackage{lineno}
\usepackage{multicol}
\usepackage{graphics}

\newcommand{\QC}{\mathfrak{Q}}
\newcommand{\h}{\mathcal{H}}
\renewcommand{\L}{\mathcal{L}}
\newcommand{\D}{\mathcal{D}}
\newcommand{\I}{\mathcal{I}}
\newcommand{\II}{\mathbb{I}}
\renewcommand{\P}{\mathds{P}}
\newcommand{\M}{\mathds{M}}
\newcommand{\HH}{\mathbf{H}}
\newcommand{\LL}{\mathbf{L}}
\newcommand{\PP}{\mathbf{P}}
\newcommand{\T}{\mathrm{T}}
\newcommand{\dd}{\mathrm{d}}
\newcommand{\id}{\mathbf{I}}
\newcommand{\TT}{\mathbb{T}}
\newcommand{\J}{\mathcal{J}}
\newcommand{\JJ}{\mathbb{J}}
\newcommand{\vl}{\mathrm{V2L}}
\newcommand{\lv}{\mathrm{L2V}}
\newcommand{\cq}{\mathrm{cq}}
\newcommand{\e}{\mathrm{e}}
\newcommand{\tr}{\mathrm{tr}}
\newcommand{\spn}{\mathrm{span}}
\newcommand{\supp}{\mathrm{supp}}
\newcommand{\diag}{\mathrm{diag}}
\newcommand{\BigO}{\mathcal{O}}
\newcommand{\x}{\mathbf{x}}
\newcommand{\ntl}{\mathrm{U}\,}
\newcommand{\mnt}{\mathrm{mnt}}
\newcommand{\true}{\texttt{true}}
\newcommand{\false}{\texttt{false}}
\renewcommand{\l}{\mathcal{L}}
\renewcommand{\e}{\mathcal{E}}

\newtheorem{conjecture}[theorem]{Conjecture}
\renewcommand{\algorithmicrequire}{\textbf{Input:}}
\renewcommand{\algorithmicensure}{\textbf{Output:}}
\algnewcommand\algorithmiccom{\textbf{Complexity:}}
\algnewcommand\Com{\item[\algorithmiccom]}

\newcommand{\xm}[1]{\textcolor{blue}{\textbf{XM:} #1{}}}
\newcommand{\ny}[1]{{\color{yellow} [NY: #1]}}
%\renewcommand{\xm}[1]{}
\newcommand{\jg}[1]{{\color{red} {JG: {#1} :JG} \color{red} }}
\newcommand{\mjy}[1]{{\color{orange} {\textbf{MJY:} #1{}}}}

\bibliographystyle{plainurl}% the mandatory bibstyle

\title{Model Checking Quantum Continuous-Time Markov Chains}
%TODO Please add

\titlerunning{Model Checking Quantum CTMCs}
%TODO optional, please use if title is longer than one line

\author{Ming Xu}{Shanghai Key Lab of Trustworthy Computing,
East China Normal University, China}{mxu@cs.ecnu.edu.cn}{}{}
\author{Jingyi Mei}{Shanghai Key Lab of Trustworthy Computing,
East China Normal University, China}{mjyecnu@163.com}{}{}
\author{Ji Guan}{State Key Lab of Computer Science, Institute of Software,
Chinese Academy of Sciences, China}{guanji1992@gmail.com}{}{}
\author{Nengkun Yu}{Centre for Quantum Software and Information,
University of Technology Sydney, Australia}{nengkunyu@gmail.com}{}{}
%TODO mandatory, please use full name; only 1 author per \author macro;
%first two parameters are mandatory, other parameters can be empty.
%Please provide at least the name of the affiliation and the country.
%The full address is optional

\authorrunning{M.~Xu, J.~Mei, J.~Guan, and N.~Yu}
%TODO mandatory. First: Use abbreviated first/middle names.
%Second (only in severe cases): Use first author plus 'et al.'

\Copyright{TBD}
%TODO mandatory, please use full first names. LIPIcs license is "CC-BY";
% http://creativecommons.org/licenses/by/3.0/

\ccsdesc[500]{Theory of computation~Verification by model checking}
\ccsdesc[500]{Computing methodologies~Symbolic and algebraic algorithms}
\ccsdesc[300]{Theory of computation~Quantum computation theory}
%TODO mandatory: Please choose ACM 2012 classifications from
% https://dl.acm.org/ccs/ccs_flat.cfm 

\keywords{Model Checking, Formal Logic, Quantum Computing, Computer Algebra}
%TODO mandatory; please add comma-separated list of keywords

%\category{} %optional, e.g. invited paper

%\relatedversion{} %optional, e.g. full version hosted on arXiv, HAL,
%or other respository/website
%\relatedversion{A full version of the paper is available at \url{...}.}

%\supplement{}%optional, e.g. related research data, source code
% ... hosted on a repository like zenodo, figshare, GitHub, ...

%\funding{(Optional) general funding statement \dots}%optional,
%to capture a funding statement, which applies to all authors.
%Please enter author specific funding statements as fifth argument of the \author macro.

%\acknowledgements{I want to thank \dots}%optional

\nolinenumbers %uncomment to disable line numbering

%\hideLIPIcs  %uncomment to remove references to LIPIcs series (logo, DOI, ...),
%e.g. when preparing a pre-final version to be uploaded to arXiv
%or another public repository

%Editor-only macros:: begin (do not touch as author)%%%%%%%%%%%%%%%%%%%%%%%%%%%%%%%%%%
\EventEditors{John Q. Open and Joan R. Access}
\EventNoEds{2}
\EventLongTitle{42nd Conference on Very Important Topics (CVIT 2016)}
%\EventShortTitle{CONCUR 2021}
\EventAcronym{CVIT}
\EventYear{2016}
\EventDate{December 24--27, 2016}
\EventLocation{Little Whinging, United Kingdom}
\EventLogo{}
\SeriesVolume{42}
\ArticleNo{23}
%%%%%%%%%%%%%%%%%%%%%%%%%%%%%%%%%%%%%%%%%%%%%%%%%%%%%%

\begin{document}
	
\maketitle
	%TODO mandatory: add short abstract of the document

\begin{abstract}
	Verifying quantum systems has attracted a lot of interests in the last decades.
	In this paper, we initialised the model checking of quantum continuous-time Markov chain (QCTMC).
	As a real-time system,
	we specify the temporal properties on QCTMC by signal temporal logic (STL).
	To effectively check the atomic propositions in STL,
	we develop a state-of-art real root isolation algorithm under Schanuel's conjecture;
	further, we check the general STL formula by interval operations with a bottom-up fashion,
	whose query complexity turns out to be linear in the size of the input formula
	by calling the real root isolation algorithm.
	A running example of an open quantum walk is provided to demonstrate our method.
\end{abstract}


%\linenumbers
\section{Introduction}
Aiming to study nature,
physicists use different mechanics depending on the scale of the objects
they are interested in.
Classical mechanics describes nature at macroscopic scale
(far larger than $10^{-9}$ meters),
while quantum mechanics is applied at microscopic scale (near or less than $10^{-9}$ meters). 
A particle at this level can be mathematically represented
by a normalised complex vector $\ket{s}$ in a Hilbert space $\h$. 
The time evolution of a single particle \emph{closed} system is described by
the Schr\"odinger equation 
\begin{equation}\label{eq:schrodinger}
	\frac{\dd\ket{s(t)}}{\dd t} = -\imath \HH\ket{s(t)}
\end{equation}
with some \emph{Hamiltonian} $\HH$ (a Hermitian matrix on $\h$),
where $\ket{s(t)}$ is the state of the system at time $t$. 
More practically,
an \emph{open} quantum system interacting with the surrounding environment need to be considered.
Suffering noises from the environment,
the state of the system cannot be completely known.
Thus a \emph{density operator} $\rho$ (positive semidefinite matrix with unit trace) on $\h$
is introduced to describe the uncertainty of the possible states: 
$\rho = \sum_{i\in I} p_i\op{s_i}{s_i}$,
where $\{(p_i,\ket{s_i})\}_{i\in I}$ is a mixed state
or an ensemble expressing that the quantum state is at $\ket{s_i}$ with probability $p_i$,
and $\bra{s_i}$ is the complex conjugate and transpose of $\ket{s_i}$.
In this case, the evolution is described by the Lindblad's master equation:
\begin{equation}\label{Lind}
	\frac{\dd \rho(t)}{\dd t}=\L(\rho(t))
\end{equation}
where $\rho(t)$ stands for the (possibly mixed) state of the system,
and $\L$ is a linear function of $\rho(t)$ (to be formally described in Subsection~\ref{S22}),
which is generally irreversible.

To reveal physical phenomenon,
physicists have intensively studied the properties of closed and open quantum systems
case by case in the last decades,
such as long-term behaviors~(e.g. \cite{GLT+10})
and stabilities~(e.g. \cite{CiT15}).
In recent years,
the computer science community has stepped into this field
and adopted model checking technique to study quantum systems~\cite{GNP06,GNP08}.
Specifically, the quantum systems can be simulated by some mathematical models
and a bulk of physical properties can be reformulated as formulas
in some temporal logic with atomic propositions of quantum interpretation.
In particular, a \emph{quantum discrete-time Markov chain} (QDTMC) $(\h,\e)$
has been introduced as quantum generalisations of classical discrete-time Markov chains (DTMC)
to model the evolution in Eq.~\eqref{Lind} in a single time unit:
\begin{equation}\label{eq:evolution}
	\rho(t+1)=\e(\rho(t))
\end{equation}
where $\e$ is a discretised quantum operation obtained from the Lindblad's master equation,
usually called a \emph{super-operator} in the field of quantum information and computation.
Several fundamental model checking-related problems
for QDTMCs have been studied in the literature,
including limiting states~\cite{Wol12,GFY18},
reachability~\cite{YFY+13,XHF21},
repeated reachability~\cite{FHT+17},
linear time properties~\cite{LiF15},
and persistence based on irreducible and periodic decomposition techniques~\cite{BaN11,GFY18}.
These techniques were equipped to solve real-world problems in several different areas.
For example, \cite{FYY13} proposed algorithms to
model checking quantum cryptographic protocols
against a quantum extension of probabilistic computation tree logic (PCTL);
and linear temporal logic (LTL) was adopted to
specify a bulk of properties of quantum many-body and statistical systems,
and corresponding model checking algorithm was developed~\cite{GFT+19}.
See~\cite{YiF19,YiF21} for the comprehensive review of this research line. 

However, to the best of our knowledge,
there is no work on model checking the quantum continuous-time system of Eq.~\eqref{Lind}.
In contrast, there are fruitful results in the classical counterpart,
which is usually modelled by a continuous-time Markov chain (CTMC). 

The seminal work on verifying CTMCs is Aziz et~al.'s paper~\cite{ASS+96,ASS+00}.
The authors introduced continuous stochastic logic (CSL) interpreted on CTMCs.
Roughly speaking, the syntax of CSL amounts to
that of PCTL plus the multiphase until formula
$\Phi_1 \ntl^{\I_1} \Phi_2 \cdots \ntl^{\I_K} \Phi_{K+1}$,
for some $K \ge 1$.
Because \cite{ASS+96} restricts probabilities in $\Pr_{>\texttt{c}}$
to $\texttt{c} \in \mathbb{Q}$,
they can show decidability of model checking for CSL
using number-theoretic analysis.
An approximate model checking algorithm for a reduced version of CSL
was provided by Baier et~al.~\cite{BKH99},
who restrict path formulas to binary until: $\Phi_1 \ntl^\I \Phi_2$.
Under this logic, they successfully applied efficient numerical techniques
for \emph{transient analysis}~\cite{BHH+03}
using \emph{uniformisation}~\cite{Ste94}.
The approximate algorithms have been extended
for multiphase until formulas
using \emph{stratification}~\cite{ZJN+11,ZJN+12}.
Xu~\textit{et~al.} considered the multi-phase until formulas
over the CTMC with rewards~\cite{XZJ+16}.
An integral-style algorithm was proposed to attack this problem,
whose effectiveness is ensured by number-theoretic results and algebraic methods.
Recently, continuous linear logic was introduced to specified on CTMCS,
whose decidability was established~\cite{GuY20}.
Most the above algorithms have been implemented in probabilistic model checkers,
like \textsl{PRISM}~\cite{KNP11}, \textsl{Storm}~\cite{DJK+17},
and \textsl{ePMC}~\cite{HLS+14}.

Unfortunately, these results from CTMCs cannot directly tackle the problem of
automatically verifying quantum continuous-time systems.
The main obstacle is that the state space of classical case is finite,
while the space of quantum states in even a finite-dimensional Hilbert space
is continuum (i.e., infinite).
In this paper, we cross the difficulty
by introducing quantum continuous-time Markov chains (QCTMC)
to model the evolution of quantum continuous-time systems in Eq.~\eqref{Lind}
and converting them into a distribution transformer
that preserves the laws of quantum mechanics.
Then, we consider a wide logic, signal temporal logic (STL),
to specify real-time properties against QCTMC.
The STL is more expressible than LTL and CTL.
Finally we present an exact method to decide the STL formula
using real root isolation and interval operations,
whose query complexity turns out to be linear in the size of the input formula
by calling the real root isolation routine.

The key contributions of the present paper are three-fold:
\begin{enumerate}
	\item In the field of formal verification,
		the model checking on DTMC, QDTMC and CTMC has been well studied in the past decades,
		but no work on checking QCTMCs as far as we know.
		The first contribution is filling this blank.
	\item In order to solve the atomic propositions in STL,
		we develop a state-of-the-art real root isolation algorithm
		for a rich class of real-valued functions based on Schanuel's conjecture.
	\item We provide a running example---%
		open quantum walk equipped with an absorbing boundary,
		which drops the restriction on the underlying graph being symmetric/Hermitian.
		The non-Hermitian structure brings real technical hardness.
		Fortunately, it is overcome by Eq.~\eqref{Lind}
		employed for describing the dynamical system of QCTMC.
\end{enumerate}
 
%	quantum automata~\cite{MQL12,WLY21}
%	hardness~\cite{BJKP05},
%	CLL~\cite{GuY20}

%For discrete-time Markov chain (DTMC), early work dates back to 1980s.
%Based on computation tree logic (CTL)~\cite{CES86},
%Hansson and Jonsson introduced probabilistic CTL (PCTL)
%by adding the probabilistic quantifier $\Pr_{>\texttt{c}}$
%with a probability threshold $\texttt{c}$,
%and further gave an algorithm for checking the validity of PCTL formulas~\cite{HaJ89}.
%Like CTL, PCTL consists of state formulas and path formulas.
%The syntax of PCTL path formulas allows
%neither Boolean combinations of path formulas
%nor nesting them.
%The former can be used to express a conditional probability~\cite{AvR08},
%while the latter can generate something similar to multiphase until formulas.
%Both operations on path formulas are allowed
%in linear temporal logic (LTL)~\cite{Pnu77}.
%A natural extension of PCTL is PCTL$^*$, introduced
%by Aziz et~al.\@~\cite{ASB+95},
%which subsumes both PCTL and LTL.
%The decidability of PCTL$^*$ formulas over DTMCs follows from the fact in~\cite{Eme90}
%that a set of paths satisfying a formula in probabilistic LTL is measurable.

%Quantum hardware has been rapidly developed in the last decades.
%In the meantime,
%quantum software will be crucial in harnessing the power of quantum computers,
%such as the BB84 protocol for quantum key distribution~\cite{BeB84},
%Shor's algorithm for integer factorization~\cite{Sho94},
%Grover's algorithm for unstructured search~\cite{Gro96},
%and the HHL algorithm for solving linear equations~\cite{HHL09}.
%To ensure the reliability of quantum software,
%verification technologies are urgent to be developed for quantum systems,
%including quantum Markov chains.
%Gay~\textit{et~al.}~\cite{GNP06,GNP08}
%restricted the quantum operations to the Clifford group gates
%(including Hadamard, CNOT and phase gates),
%restricted the state space of quantum discrete-time Markov chain (QDTMC)
%to a set of finitely describable states, called stabilizers,
%that is closed under those Clifford group gates,
%and applied \textsc{PRISM} to check the quantum protocols---superdense coding,
%quantum teleportation,
%and quantum error correction.
%Whereas, Feng~\textit{et~al.} proposed
%the model of super-operator weighted QDTMC~\cite{FYY13},
%which gave rise to an alternative way to finitely describable states.
%The model was shown to be able to describe some hybrid systems~\cite{LiF15}.
%Under the model, the authors considered the reachability probability~\cite{YFY+13},
%the repeated reachability probability~\cite{FHT+17},
%and the model checking of linear time properties~\cite{LiF15}
%and a quantum analogy of computation tree logic (QCTL)~\cite{FYY13}.

\subparagraph{Organisation}
The rest of the paper is structured as follows.
Section~\ref{S2} review some notions and notations
in quantum computing, the Lindblad's master equation and number theory.
In Sections~\ref{S3} and~\ref{S4},
we introduce the model of quantum continuous-time Markov chains
and the signal temporal logic (STL), respectively.
We solve the atomic propositions in STL in Section~\ref{S5},
and decide the general STL formulas in Section~\ref{S6}.
Section~\ref{S7} is the conclusion.


\section{Preliminaries}\label{S2}
Here we will recall some useful notions and notations from quantum computing~\cite{NiC00},
the Lindblad's master equation and number theory.

\subsection{Quantum Computing}
Let $\h$ be a Hilbert space with dimension $d$.
We employ the Dirac notations that are standard in quantum computing:
\begin{itemize}
	\item $\ket{s}$ stands for a unit column vector in $\h$ labelled with $s$;
	\item $\bra{s}:=\ket{s}^\dag$ is the Hermitian adjoint
	(complex conjugate and transpose) of $\ket{s}$;
	\item $\ip{s_1}{s_2}:=\bra{s_1}\ket{s_2}$
	is the inner product of $\ket{s_1}$ and $\ket{s_2}$;
	\item $\op{s_1}{s_2}:=\ket{s_1} \otimes \bra{s_2}$ is the outer product,
	where $\otimes$ denotes tensor product; and
	\item $\ket{s_1,s_2}:=\ket{s_1}\ket{s_2}$ is a shorthand of
    the product state $\ket{s_1}\otimes\ket{s_2}$.
\end{itemize}

A linear operator $\gamma$ is \emph{Hermitian} if $\gamma=\gamma^\dag$;
and it is \emph{positive}
if $\bra{s}\gamma\ket{s} \ge 0$ holds for any $\ket{s}\in\h$.
A \emph{projector} $\PP$ is a positive operator of
the form $\sum_{i=1}^m \op{s_i}{s_i}$ with $m\le d$,
where $\ket{s_i}$ are orthonormal.
Clearly, there is a bijective map
between projectors $\PP=\sum_{i=1}^m \op{s_i}{s_i}$
and subspaces of $\h$ that are spanned by $\{\ket{s_i}:1 \le i \le m\}$.
In sum, positive operators are Hermitian ones
whose eigenvalues are nonnegative;
and projectors are positive operators whose eigenvalues are $0$ or $1$.
Besides, a linear operator $\mathbf{U}$ is \emph{unitary}
if $\mathbf{U}\mathbf{U}^\dag=\mathbf{U}^\dag\mathbf{U}=\id$
where $\id$ is the \emph{identity} operator.

The \emph{trace} of a linear operator $\gamma$ is defined as
$\tr(\gamma):=\sum_{i=1}^d \bra{s_i}\gamma\ket{s_i}$
for any orthonormal basis $\{\ket{s_i}:1 \le i \le d\}$ of $\h$.
A \emph{density operator} $\rho$ on $\h$
is a positive operator with unit trace.
It gives rise to a generic way to describe quantum states:
if a density operator $\rho$ is $\op{s}{s}$ for some $\ket{s}\in \h$,
$\rho$ is said to be a \emph{pure} state;
otherwise it is a \emph{mixed} one,
i.e. $\rho=\sum_{i=1}^m p_i \op{s_i}{s_i}$ with $m\ge 2$ by spectral decomposition,
where $p_i$ are positive eigenvalues
(interpreted as the \emph{probabilities} of taking the pure states $\ket{s_i}$)
and their sum is $1$.
In other words, a pure state indicates the system state which we completely know;
a mixed state gives all possible system states, with total probability $1$, which we know.
We denote by $\D_\h$ the set of density operators on $\h$.
The subscript $\h$ of $\D_\h$ will be omitted if it is clear from the context. 

The system evolution between pure states is characterised by some unitary operator $\mathbf{U}$,
i.e. $\op{s(t)}{s(t)}=\mathbf{U}(t)\op{s(0)}{s(0)}\mathbf{U}^\dag(t)$
where $\mathbf{U}(t)$ comes from $\exp(-\imath \HH t)$
for the Hermitian operator $\HH$ in Eq.~\eqref{eq:schrodinger};
the system evolution between density operators (pure or mixed states) is characterised
by some completely positive operator-sum, a.k.a. \emph{super-operator},
i.e. $\rho(t)=\sum_{j=1}^m \LL_j(t) \rho(0) \LL_j^\dag(t)$
where $\LL_j$ are linear operators satisfying
the trace preservation $\sum_{j=1}^m \LL_j^\dag \LL_j=\id$.
The latter dynamical system is obtained in such a way:
an enlarged unitary operator acts on the purified composite state
$(\sum_{i=1}^m \sqrt{p_i}\ket{s_i,env_i})(\sum_{i=1}^m \sqrt{p_i}\bra{s_i,env_i})$
with $\rho(0)=\sum_{i=1}^m p_i \op{s_i}{s_i}$
for some orthonormal environment states $\ket{env_i}$~\cite[Section~2.5]{NiC00},
then the environment in the resulting composite state is traced out,
which turns out to be the aforementioned operator-sum form.
We call it an \emph{open} system as it interacts with the environment.
Whereas, the former dynamical system, independent from the environment,
is called a \emph{closed} system.
Open systems are more common than closed systems in practice.


\subsection{Lindblad's Master Equation}\label{S22}
To characterise state evolution of the continuous-time open system with the \emph{memoryless} property,
we employ the Lindblad's master equation~\cite{Lin76,GKS76} that is
\begin{equation}\label{eq:Lindblad}
	%&=-\imath[\HH,\rho]+
	%\sum_j\left(\LL_j\rho \LL_j^\dag-\tfrac{1}{2}\{\LL_j^\dag\LL_j,\rho\}\right) \\
	\rho'=\L(\rho)
	=-\imath\HH\rho+\imath\rho\HH
	+\sum_{j=1}^m \left(\LL_j\rho \LL_j^\dag
	-\tfrac{1}{2}\LL_j^\dag\LL_j\rho-\tfrac{1}{2}\rho\LL_j^\dag\LL_j\right),
\end{equation}
where $\HH$ is a Hermitian operator and $\LL_j$ are linear operators.
%$[\mathbf{A},\mathbf{B}]:=\mathbf{A}\mathbf{B}-\mathbf{B}\mathbf{A}$
%is the \emph{commutator} between two linear operators $\mathbf{A}$ and $\mathbf{B}$,
%and $\{\mathbf{A},\mathbf{B}\}:=\mathbf{A}\mathbf{B}+\mathbf{B}\mathbf{A}$
%is the \emph{anti-commutator}.
The terms $-\imath\HH\rho+\imath\rho\HH$ describe the evolution of the internal system;
the terms $\LL_j\rho \LL_j^\dag
-\tfrac{1}{2}\LL_j^\dag\LL_j\rho-\tfrac{1}{2}\rho\LL_j^\dag\LL_j$ 
describe the interaction between system and environment.
In other words,
to characterise the evolution of an open system,
it is necessary to use those linear operators $\LL_j$ besides the Hermitian operator $\HH$.
It is known to be the most general type of Markovian and time-homogeneous master equation
describing (in general non-unitary) evolution of the system state
that preserves the laws of quantum mechanics
(i.e., completely positive and trace-preserving for any initial condition).

In the following, we will derive the solution of Eq.~\eqref{eq:Lindblad}.
We first define two useful functions:
\begin{itemize}
	\item $\lv(\gamma):=\sum_{i=1}^d \sum_{j=1}^d \bra{i}\gamma\ket{j} \ket{i,j}$
	that rearranges entries of the linear operator $\gamma$
	on the Hilbert space $\h$ with dimension $d$
	as a column vector; and
	\item $\vl(\mathbf{v}):=\sum_{i=1}^d \sum_{j=1}^d \bra{i,j} \mathbf{v} \op{i}{j}$
	that rearranges entries of the column vector $\mathbf{v}$ as a linear operator.
\end{itemize}
Here, $\lv$ and $\vl$ are read as
``linear operator to vector'' and ``vector to linear operator'', respectively.
They are mutually inverse functions,
so that
if a linear operator (resp.~its vectorisation) is determined,
its vectorisation (resp.~the original linear operator) is determined.
Using the fact that $\mathbf{D}=\mathbf{A}\mathbf{B}\mathbf{C} \Longleftrightarrow
\lv(\mathbf{D})=(\mathbf{A} \otimes \mathbf{C}^\T)\lv(\mathbf{B})$
holds for any linear operators $\mathbf{A},\mathbf{B},\mathbf{C},\mathbf{D}$,
we can reformulate Eq.~\eqref{eq:Lindblad} as the linear ordinary differential equation
\begin{equation}\label{eq:ODE}
	\begin{aligned}
		\lv(\rho') &=\left[-\imath\HH\otimes\id+\imath\id\otimes\HH^\T
		+\sum_{j=1}^m \left(\LL_j\otimes\LL_j^{*}
		-\tfrac{1}{2}\LL_j^\dag\LL_j\otimes\id-\tfrac{1}{2}\id\otimes \LL_j^\T\LL_j^*\right)\right]
		\lv(\rho) \\
		&= \M \cdot\lv(\rho),
	\end{aligned}
\end{equation}
where $*$ denotes entry-wise complex conjugate.
We call $\M=-\imath\HH\otimes\id+\imath\id\otimes\HH^\T
+\sum_{j=1}^m \big( \LL_j\otimes\LL_j^{*}
-\tfrac{1}{2}\LL_j^\dag\LL_j\otimes\id \linebreak[0]
-\tfrac{1}{2}\id\otimes \LL_j^\T\LL_j^* \big)$
the \emph{governing matrix} of Eq.~\eqref{eq:ODE}.
As a result,
we get the desired solution $\lv(\rho(t))=\exp(\M\cdot t)\cdot\lv(\rho(0))$
or equivalently $\rho(t)=\vl(\exp(\M\cdot t)\cdot\lv(\rho(0)))$ in a closed form.
It can be obtained in polynomial time by the standard method~\cite{Kai80}.


\subsection{Number Theory}
\begin{definition}
	A number $\alpha$ is \emph{algebraic},
	denoted by $\alpha \in \mathbb{A}$,
	if there is a nonzero $\mathbb{Q}$-polynomial $f_\alpha(z)$ of least degree,
	satisfying $f_\alpha(\alpha)=0$;
	otherwise $\alpha$ is \emph{transcendental}.
\end{definition}
In the above definition, such a polynomial $f_\alpha(z)$ is called
the \emph{minimal polynomial} of $\alpha$.
The \emph{degree} $D$ of $\alpha$ is $\deg_z(f_\alpha)$.
The standard encoding of $\alpha$ is the minimal polynomial $f_\alpha$
plus an isolation disk in the complex plane
that distinguishes $\alpha$ from other roots of $f_\alpha$.

\begin{definition}
Let $\mu_1,\ldots,\mu_m$ be irrational complex numbers.
Then the \emph{field extension} $\mathbb{Q}(\mu_1,\ldots,\mu_m):\mathbb{Q}$
is the smallest set
that contains $\mu_1,\ldots,\mu_m$ and is closed under arithmetic operations,
i.e. addition, substraction, multiplication and division.
\end{definition}
Here those irrational complex numbers $\mu_1,\ldots,\mu_m$ are called
the generators of the field extension.
A field extension is \emph{simple} if it has only one generator.
For instance, the field extension $\mathbb{Q}(\sqrt{2}):\mathbb{Q}$
is exactly the set $\{a+b\sqrt{2} : a,b \in \mathbb{Q}\}$.

\begin{lemma}[{\cite[Algorithm~2]{Loo83}}]\label{lem:simple}
	Let $\alpha_1$ and $\alpha_2$ be two algebraic numbers of
	degrees $D_1$ and $D_2$, respectively.
	There is an algebraic number $\mu$ of degree at most $D_1 D_2$,
	such that the field extension $\mathbb{Q}(\mu):\mathbb{Q}$
	is exactly $\mathbb{Q}(\alpha_1,\alpha_2):\mathbb{Q}$.
\end{lemma}
For a collection of algebraic numbers $\alpha_1,\ldots,\alpha_m$
appearing in the input instance,
by repeatedly applying this lemma,
we can obtain a simple field extension $\mathbb{Q}(\mu):\mathbb{Q}$
that can span all $\alpha_1,\ldots,\alpha_m$.
\begin{lemma}[{\cite[Corollary~4.1.5]{Coh96}}]\label{lem:closed}
	Let $\alpha$ be an algebraic number of degree $D$,
	and $g(z)$ an $\mathbb{A}$-polynomial with degree $D_g$
	and coefficients taken from $\mathbb{Q}(\alpha):\mathbb{Q}$.
	There is a $\mathbb{Q}$-polynomial $f(z)$ of degree at most $DD_g$,
	such that roots of $g(z)$ are those of $f(z)$.
\end{lemma}
The above lemma entails that
roots of any $\mathbb{A}$-polynomial are also algebraic.

\begin{theorem}[Lindemann (1882)~{\cite[Theorem~1.4]{Bak75}}]\label{Lindemann}
	For any nonzero algebraic numbers $\beta_1,\ldots,\beta_m$ and
	any distinct algebraic numbers $\lambda_1,\ldots,\lambda_m$,
	the sum $\sum_{i=1}^m \beta_i \mathrm{e}^{\lambda_i}$ with $m \ge 1$ is nonzero.
\end{theorem}

\begin{conjecture}[Schanuel (1960s)~\cite{Ax71}]\label{Schanuel}
	Let $\lambda_1,\ldots,\lambda_m$ be
	$\mathbb{Q}$-linearly independent complex numbers.
	Then the field extension
	$\mathbb{Q}(\lambda_1,\mathrm{e}^{\lambda_1},\ldots,\lambda_m,\mathrm{e}^{\lambda_m}):\mathbb{Q}$
	has transcendence degree at least $m$.
\end{conjecture}

Let $\mathbb{A}[z_1,\ldots,z_m]$ denote the ring
that contains all $\mathbb{A}$-polynomials in variables $z_1,\ldots,z_m$.
Assuming Schanuel's conjecture, we could get:
\begin{corollary}[{\cite[Proposition~5]{COW16}}]\label{cor:common}
	Let $\lambda_1,\ldots,\lambda_m$ be
	$\mathbb{Q}$-linearly independent algebraic numbers.
	Then two co-prime elements $\varphi_1$ and $\varphi_2$
	in the ring $\mathbb{A}[t,\exp(\lambda_1 t),\ldots,\exp(\lambda_m t)]$
	have no common root except for $0$.
\end{corollary}
	
	
\section{Quantum Continuous-Time Markov Chain}\label{S3}
In this section, we propose the model of quantum continuous-time Markov chain (QCTMC).
We will reveal that it extends the classical continuous-time Markov chain (CTMC).
To show the practical usefulness,
an example is further provided for modelling open quantum walk.

For the sake of clarity,
we start with the QCTMC without classical states:

\begin{definition}\label{def:QCTMC1}
	A \emph{quantum continuous-time Markov chain} $\QC$ is a pair $(\h,\L)$,
	in which
	\begin{itemize}
		\item $\h$ is the Hilbert space,
		\item $\L$ is the transition generator function given by
		a Hermitian operator $\HH$
		and a finite set of linear operators $\LL_j$ on $\h$.
	\end{itemize}
	Usually, a density operator $\rho(0) \in \D$
	is appointed as the initial state of $\QC$.
\end{definition}

In the model,
the transition generator function $\L$ gives rise to a \emph{universal} way
to describe the bahavior of the QCTMC,
following the generality of the Lindblad's master equation.
Thus the state $\rho(t)$ is given by the closed-form solution to Eq.~\eqref{eq:ODE},
i.e., $\vl(\exp(\M\cdot t)\cdot\lv(\rho(0)))$, 
where $\M$ is the governing matrix for $\L$,
is a computable function from $\mathbb{R}_{\ge 0}$ to $\mathbb{C}^{d \times d}$.
We notice that $0 \le \tr(\PP \rho(t)) \le 1$ holds for any projector $\PP$ on $\h$,
as $\rho(t)$ is a density operator on $\D$.
Considering computability, the entries of $\HH$, $\LL_j$ and $\rho(0)$ are supposed to be algebraic.

Next, we equip the QCTMC in Definition~\ref{def:QCTMC1}
with finitely many classical states.
\begin{definition}\label{def:QCTMC2}
	A \emph{quantum continuous-time Markov chain} $\QC$	with a finite set $S$ of classical states
	is a pair $(\h_\cq,\L)$, in which
	\begin{itemize}
		\item $\h_\cq:=\mathcal{C} \otimes \h$ is the classical--quantum system
		with $\mathcal{C}=\spn(\{\ket{s}: s\in S\})$, and
		\item $\L$ is the transition generator function given by
		a Hermitian operator $\HH$
		and a finite set of linear operators $\LL_j$ on the enlarged $\h_\cq$.
	\end{itemize}
	Usually, a density operator $\rho(0) \in \D_{\h_\cq}$
	is appointed as the initial state of $\QC$.
\end{definition}

In fact, the models in Definitions~\ref{def:QCTMC1} and~\ref{def:QCTMC2}
have the same expressibility:
The QCTMC in Definition~\ref{def:QCTMC1} can be obtained
by setting the singleton state set $S=\{s\}$ of the QCTMC in Definition~\ref{def:QCTMC2};
conversely, the QCTMC in Definition~\ref{def:QCTMC2} can be obtained
by setting the Hilbert space as $\h_\cq$ of the QCTMC in Definition~\ref{def:QCTMC1}.
Hence, we can freely choose one of the two definitions for convenience.
As an immediate result, using Definition~\ref{def:QCTMC2},
we can easily see that the QCTMC extends the CTMC by the following lemma:
\begin{lemma}
	Given a CTMC $\mathfrak{C}=(S,\mathbf{Q})$,
	it can be modelled by a QCTMC $\QC=(\mathcal{C} \otimes \h,\L)$
	with $\mathcal{C}=\spn(\{\ket{s}:s\in S\})$ and $\dim(\h)=1$.
\end{lemma}
\begin{proof}
    It suffices to show that
    the states of a CTMC $\mathfrak{C}$ can be obtained by those of some QCTMC $\QC$.
	The state $\x=(x_s)_{s \in S}$ of $\mathfrak{C}$ is given by
	the dynamical system $\x'(t)=\x(t) \cdot \mathbf{Q}$
	or equivalently its closed-form solution
	$\x(t)=\x(0) \cdot \exp(\mathbf{Q}\cdot t)$,
	where $\x(0)$ is a row vector interpreted as the initial state.
	We construct the QCTMC by setting Hermitian operator $\HH=0$
	and linear operators $\LL_{s,t}=\op{t}{s}\otimes \mathbf{Q}[s,t]$ in $Q$
	for each pair $s,t \in S$.
	It is not hard to validate that the state $\rho(t)$ of $\QC$
	is $\diag(\x(t))$ of $\mathfrak{C}$,
	thus the lemma follows.
\end{proof}
	
\begin{example}[Open Quantum Walk~\cite{Pel14,SiP15}]\label{ex1}
	Open quantum walk (OQW) is a quantum analogy of random walk,
	whose system evolution interacts with environment.
	For the sake of clarity, we suppose that
	a particle walking along the $2$-dimensional hypercubic shown in Figure~\ref{fig:QW}.
	The position set is $S = \{s_{00}, s_{01}, s_{10}, s_{11}\}$,
	where $s_{00}$ denotes the starting position, a.k.a. the entrance,
	$s_{01}$ and $s_{10}$ denote transient positions,
	and $s_{11}$ denotes the exit, an absorbing boundary.
	The direction set is $\{\mathrm{F},\mathrm{S}\}$,
	where $\mathrm{F}$ means the particle takes the external transition along the first coordinate,
	while $\mathrm{S}$ means the particle takes the external transition along the second coordinate.
	The particle will choose a direction at every moment by the inner quantum ``coin-tossing''
	before being absorbed.
	This action is implemented by the Hadamard operator
	$H = \op{+}{\mathrm{F}} + \op{-}{\mathrm{S}}$
	with $\ket{\pm} = (\ket{\mathrm{F}}\pm\ket{\mathrm{S}})/\sqrt{2}$,
	which denotes a fair selection between $\mathrm{F}$ and $\mathrm{S}$
	as the probability amplitudes of both directions are $\tfrac{1}{2}=(\pm1/\sqrt{2})^2$.
	
	The OQW is modelled by a QCTMC $\QC_1=(\mathcal{C}\otimes\h,\L)$
	with $\mathcal{C}=\spn(\{\ket{s}:s\in S\})$,
	where each position in $S$ represents a classical state,
	and the transition function $\L$ is given by the Hermitian operator $\HH=0$
	and the unique linear operator
	\[
	\begin{aligned}
		\LL = & \op{s_{01}}{s_{00}}\otimes\op{\mathrm{S}}{-}
			+ \op{s_{10}}{s_{00}}\otimes\op{\mathrm{F}}{+}
			+ \op{s_{00}}{s_{01}}\otimes\op{\mathrm{S}}{-} \ + \\
			& \op{s_{11}}{s_{01}}\otimes\op{\mathrm{F}}{+}
			+ \op{s_{00}}{s_{10}}\otimes\op{\mathrm{F}}{+} 
			+ \op{s_{11}}{s_{10}}\otimes\op{\mathrm{S}}{-}.
	\end{aligned}
	\]
	For instance, when the particle is in the position $\ket{s_{00}}$,
	we first apply the quantum coin-tossing $H$ to the state in $\h$,
	then we get the result $\mathrm{F}$ or $\mathrm{S}$
	leading to the position $\ket{s_{10}}$ or $\ket{s_{01}}$.
	The composite operations are $(\op{\mathrm{F}}{\mathrm{F}}) H = \op{\mathrm{F}}{+}$
	and $(\op{\mathrm{S}}{\mathrm{S}}) H = \op{\mathrm{S}}{-}$,
	which makes up the first two terms in the above $\LL$.
	
	From $\LL$,
	we could get the governing matrix
	$\M = \LL\otimes\LL^* - \tfrac{1}{2}\LL^\dag\LL\otimes\id - \tfrac{1}{2}\id\otimes\LL^\T\LL^*$.
	Let $\rho(0) = \op{s_{00}}{s_{00}}\otimes\op{\mathrm{F}}{\mathrm{F}}$ be an initial state.
	Then the states $\rho(t)$ of the OQW could be computed
	as $\vl(\exp(\M\cdot t)\cdot\lv(\rho(0)))$ (we omit the detailed value due to space limit). \qed
		
		\begin{multicols}{2}
		\begin{itemize}
		\item[] 
		\begin{minipage}{\linewidth}
		\captionsetup{type=figure}
		\begin{tikzpicture}[->,>=stealth',auto,node distance=2.5cm,semithick,inner sep=2pt]
			\node[state](s0){$s_{01}$};
			\node[state,initial,initial text={entrance}](s1)[below of=s0]{$s_{00}$};
			\node[state](s2)[right of=s1]{$s_{10}$};
		\node[state,accepting,label={right:exit}](s3)[right of=s0]{$s_{11}$};
			\draw(s1)edge[bend left]node[]{S}(s0);
			\draw(s0)edge[bend left]node[]{S}(s1);
			\draw(s2)edge[bend left]node[]{F}(s1);
			\draw(s1)edge[bend left]node[]{F}(s2);
			\draw(s2)edge[bend left]node[]{S}(s3);
			\draw[dashed](s3)edge[bend left]node[]{S}(s2);
			\draw(s0)edge[bend left]node[]{F}(s3);
			\draw[dashed](s3)edge[bend left]node[]{F}(s0);
		\end{tikzpicture}
    	\captionof{figure}{A sample OQW with an absorbing boundary}
    	\label{fig:QW}
		\end{minipage}
		\begin{minipage}{\linewidth}
		\item[]
		\null\vfill\null\vfill
		As an absorbing boundary $s_{11}$ is introduced here,
		two transitions from it (in dashed line) do not exist anymore.
		Absorbing boundaries makes it impossible
		to characterise the system evolution by a closed system,
		i.e., using some Hermitian operator $\HH$ only,
		as usual in~\cite{SiP15}. 
		Fortunately, we could characterise it by an open system,
		i.e., using the linear operators $\LL_j$.
		\vfill\null\vfill \null
		\end{minipage}
		\end{itemize}
		\end{multicols}
\end{example}


\section{Signal Temporal Logic}\label{S4}
Here we recall the signal temporal logic (STL)~\cite{MaN04},
which is widely used to express real-time properties.
Using it we could specify richer properties of QCTMC
than linear temporal logic (LTL) and computation tree logic (CTL) in the time-bounded fragment.

\begin{definition}\label{def:syntax}
	The syntax of the STL formulas are defined as follows:
	\[
	\phi := \Phi \mid \neg\phi \mid \phi_1 \wedge \phi_2 \mid \phi_1 \ntl^\I \phi_2
	\]
	in which the atomic propositions $\Phi$,
	interpreted as \emph{signals},
	are of the form $p(\x) \in \II$
	where $p$ is a $\mathbb{Q}$-polynomial in $\x=(x_s)_{s\in S}$
	and $\II$ is a rational interval,
	and $\I$ is a finite time interval.
	Here $\ntl$ is called the until operator,
	and $\phi_1 \ntl^\I \phi_2$ is the until formula.
\end{definition}
	
\begin{definition}\label{def:semantics}
	The semantics of the STL formulas interpreted on a QCTMC $\QC$ in Definition~\ref{def:QCTMC2}
	are given by the satisfaction relation $\models$:
	\begin{align*}
		\rho(t) & \models \Phi
		&& \textup{if } p(\x) \in \II
		\textup{ holds with }x_s = \tr(\PP_s(\rho(t))), \\
		\rho(t) & \models \neg\phi
		&& \textup{if } \rho(t) \not\models \phi, \\
		\rho(t) & \models \phi_1 \wedge \phi_2
		&& \textup{if } \rho(t) \models \phi_1 \wedge \rho(t) \models \phi_2, \\
		\rho(t) & \models \phi_1 \ntl^\I \phi_2
		&& \textup{if there exists a real number }t' \in \I, \\
		&&& \textup{such that }\forall\,t_1 \in [t,t+t') \,:\, \rho(t_1) \models \phi_1
		\textup{ and }\rho(t+t') \models \phi_2,
	\end{align*}
	where $\PP_s$ is the projector $\op{s}{s} \otimes \id$ onto classical state $s$.
\end{definition}

From the semantics, we can see that
$\Phi_1 \ntl^{\I_1} (\Phi_2 \ntl^{\I_2} \Phi_3)$
and $(\Phi_1 \ntl^{\I_1} \Phi_2) \ntl^{\I_2} \Phi_3$ have different meanings:
The former formula requires there exist $t'\in \I_1$ and $t''\in \I_2$
such that $\forall\,t_1\in[t,t+t'): \rho(t_1) \models \Phi_1$,
$\forall\,t_2\in[t+t',t+t'+t''): \rho(t_2) \models \Phi_2$ and $\rho(t+t'+t'') \models \Phi_3$;
while the latter formula requires there exists a $t''\in \I_2$
such that $\rho(t+t'') \models \Phi_3$
and for each $t_1\in[t,t+t'')$, there exists a $t'\in \I_1$
such that $\forall\,t_1\in[t_1,t_1+t'): \rho(t_1) \models \Phi_1$
and $\rho(t_1+t') \models \Phi_2$.
We usually use the \emph{parse tree}
to clarify the structure of an STL formula $\phi$.

The logic is very generic.
STL has more expressive atomic propositions than LTL,
as $\true \equiv p(\x) \in (-\infty,+\infty)$ and $\false \equiv p(\x) \in \emptyset$.
CTL has a two-stage syntax consisting of state and path formulas.
Negation and conjunction are allowed in only state formulas, not path ones.
Whereas, STL allows negation and conjunction in any subformulas.
Besides the standard Boolean calculus,
we can easily obtain a few derivations:
$\Diamond^\I \phi \equiv  \true \ntl^\I \phi$,
$\Box^\I \phi \equiv \neg(\Diamond^\I \neg \phi)$,
and $\phi_1 \mathrm{R}^\I \phi_2 \equiv \neg (\neg\phi_1 \ntl^\I \neg\phi_2)$
where $\mathrm{R}$ is the release operator.
	
\begin{example}\label{ex2}
    Consider the open quantum walk $\QC_1$ described in Example~\ref{ex1}.
    It is not hard to get
    the probabilities of the particle staying respectively in $s_{00},s_{01},s_{10},s_{11}$ as follows:
	\begin{align*}
		x_{0,0} &= \tr(\PP_{s_{0,0}}(\rho(t)))
			= \tfrac{1}{2}\exp(-\tfrac{2+\sqrt{2}}{2}t)
			+ \tfrac{1}{2}\exp(-\tfrac{2-\sqrt{2}}{2}t), \\
		x_{0,1} &= \tr(\PP_{s_{0,1}}(\rho(t)))
			= -\tfrac{\sqrt{2}}{4}\exp(-\tfrac{2+\sqrt{2}}{2}t)
			+ \tfrac{\sqrt{2}}{4}\exp(-\tfrac{2-\sqrt{2}}{2}t) \\
			&\quad + \tfrac{1}{4}[\exp(-\tfrac{3}{2}t)-\exp(-\tfrac{1}{2}t)]\cos(\tfrac{1}{2}t) 
			+\tfrac{1}{4}[\exp(-\tfrac{3}{2}t)+\exp(-\tfrac{1}{2}t)]\sin(\tfrac{1}{2}t), \\
		x_{1,0} &= \tr(\PP_{s_{1,0}}(\rho(t)))
			= -\tfrac{\sqrt{2}}{4}\exp(-\tfrac{2+\sqrt{2}}{2}t)
			+\tfrac{\sqrt{2}}{4}\exp(-\tfrac{2-\sqrt{2}}{2}t) \\
			&\quad -\tfrac{1}{4}[\exp(-\tfrac{3}{2}t)-\exp(-\tfrac{1}{2}t)]\cos(\tfrac{1}{2}t)
			- \tfrac{1}{4}[\exp(-\tfrac{3}{2}t)+\exp(-\tfrac{1}{2}t)]\sin(\tfrac{1}{2}t), \\
		x_{1,1} &= \tr(\PP_{s_{1,1}}(\rho(t)))
    	    = 1	+\tfrac{-1+\sqrt{2}}{2}\exp(-\tfrac{2+\sqrt{2}}{2}t)
    	    -\tfrac{1+\sqrt{2}}{2}\exp(-\tfrac{2-\sqrt{2}}{2}t).
	\end{align*}
    We can see that
    the probability of exiting via $s_{11}$ would be one as $t$ approaches infinity.
    A further question is asking
    how fast the particle would reach the exit.
    This is actually a question about convergence performance.
    It is worth monitoring the critical moment
    at which the probability of exiting via $s_{11}$
    equals that of staying at transient positions $s_{01},s_{10}$.
    A requirement to be studied is like the following property.
	\begin{quote}
		\textbf{Property A:} At each moment $t\in[0,5]$,
		whenever the probability of staying at $s_{01}$ or $s_{10}$
		is greater than $\tfrac{1}{5}$,
		the probability of exiting via $s_{11}$ would overweight
		than that of staying at $s_{01}$ or $s_{10}$
		within the coming one unit of time.
	\end{quote}
	We formally specify it with the rich STL formula
	\[
		\phi_1 \equiv \Box^{\I_1}\, (\Phi_1 \rightarrow \Diamond^{\I_2} \Phi_2)
		\equiv \neg(\true\,\ntl^{\I_1}\, (\Phi_1 \wedge \neg(\true\,\ntl^{\I_2} \Phi_2))),
	\]
	where $\I_1=[0,5],\I_2=[0,1]$ are time intervals
	and $\Phi_1 \equiv x_{0,1}+x_{1,0}>\tfrac{1}{5},
	\Phi_2 \equiv x_{1,1} \ge x_{0,1}+x_{1,0}$ are atomic propositions,
	a.k.a. signals. \qed
\end{example}


\section{Solving Atomic Propositions}\label{S5}
As a basic step to decide the STL formula,
we need to solve the atomic proposition $\Phi$.
That is,
we will compute all solutions w.r.t. $t$, in which $\rho(t) \models \Phi$ holds.
We achieve it by a reduction to the real root isolation
for a class of real-valued functions, \emph{exponential polynomials}.
Although real roots of exponential polynomials have been studied
in many existing literature~\cite{AMW08,COW16,HLX+18},
the ones to be isolated in this paper involve the complicated complex exponents.
So we develop a state-of-the-art real root isolation for them,
whose completeness is established on Conjecture~\ref{Schanuel}.

Given an atomic proposition $\Phi\equiv p(\x) \in \II$
(assuming that $\II$ is bounded),
we would like to determine the algebraic structure of
\begin{equation}\label{eq:EP1}
	\varphi(t)=(p(\x(t))-\inf\II)(p(\x(t))-\sup\II),
\end{equation}
with which we will design an algorithm for solving $\Phi\equiv p(\x) \in \II$.
The structure of $\varphi(t)$ depends on that of $x_s(t)=\tr(\PP_s(\rho(t)))$.
We claim that
each entry of $\rho(t)$ is of the exponential polynomial form
\begin{equation}\label{eq:EP}
	\beta_1(t) \exp(\alpha_1 t) + \beta_2(t) \exp(\alpha_2 t) + \cdots
	+ \beta_m(t) \exp(\alpha_m t),
\end{equation}
where $\beta_1(t),\ldots,\beta_m(t)$ are nonzero $\mathbb{A}$-polynomials
and $\alpha_1,\ldots,\alpha_m$ are distinct algebraic numbers.
It follows the facts:
\begin{enumerate}
	\item The governing matrix $\M$
		in the state $\rho(t)=\vl(\exp(\M\cdot t)\cdot\lv(\rho(0)))$ of the QCTMC
		takes algebraic numbers  as entries. 
	\item The characteristic polynomial of $\M$ is an $\mathbb{A}$-polynomial.
	    The eigenvalues $\alpha_1,\ldots,\alpha_m$ of $\M$ are algebraic,
	    as they are roots of that $\mathbb{A}$-polynomial by Lemma~\ref{lem:closed}.
		Those eigenvalues make up all exponents in~\eqref{eq:EP}.
	\item The entries of the matrix exponential $\exp(\M\cdot t)$ are in the form~\eqref{eq:EP}.
\end{enumerate}
The same structure holds for $x_s(t)$,
as $x_s(t)=\tr(\PP_s(\rho(t)))$ is simply a sum of some entries of $\rho(t)$.
Furthermore, $\varphi(t)$ is also of the exponential polynomial form~\eqref{eq:EP},
since it is a $\mathbb{Q}$-polynomial in $\x(t)=(x_s(t))_{s \in S}$.
If $\II$ is unbounded from below (resp.~above),
the left (resp.~right) factor could be removed from~\eqref{eq:EP1} for further consideration.

Next, we will isolate all real roots $\lambda_1,\ldots,\lambda_n$ of $\varphi(t)$
in a bounded interval $\mathcal{B}$ (to be specified in the next section).
Before stating the core isolation algorithm---Algorithm~\ref{isol},
an overview of the isolation procedure is provided in Fig.~\ref{flow}.
The instances to be treated can be roughly divided into two classes: 
one is trivial that can be solved by the classical methods, e.g.,~\cite{CoL76},
for ordinary polynomials; 
the other is nontrivial that can be solved by Algorithm~\ref{isol}
but should meet three requirements of the input in Algorithm~\ref{isol}.
After the preprocesses \textbf{Basis Finding}, \textbf{Polynomialisation} and \textbf{Factoring},
the two classes of instances can be separated and solved by the corresponding methods.
% In the end, the \textbf{Refinement} process outputs the pairwise disjoint isolation intervals.

\begin{figure}
	\includegraphics[scale=0.4]{flow2.pdf}
	\caption{Flow chart of the whole isolation procedure}
	\label{flow}
\end{figure}

The technical details along the flow chart in Fig.~\ref{flow} are described below.
\begin{itemize}
	\item \textbf{Basis Finding}
	For a given set of algebraic numbers $\{\alpha_1,\ldots,\alpha_m\}$
	extracted from the exponents of the input exponential polynomial $\varphi(t)$,
	we can compute a simple extension
	$\mathbb{Q}(\mu):\mathbb{Q}=\mathbb{Q}(\alpha_1,\ldots,\alpha_m):\mathbb{Q}$
	by Lemma~\ref{lem:simple},
	such that each $\alpha_i=q_i(\mu)$ with $q_i\in \mathbb{Z}[x]$ and $\deg(q_i)<\deg(\mu)$;
	and further construct a $\mathbb{Q}$-linearly independent basis $\{\mu_1,\ldots,\mu_k\}$
	of those exponents $\{\alpha_1,\ldots,\alpha_m\}$ by~\cite[Section~3]{HLX+18},
	such that each $\alpha_i$ can be $\mathbb{Z}^+$-linearly expressed by $\{\mu_1,\ldots,\mu_k\}$.
	\item \textbf{Polynomialisation}
	Thus we can get a polynomial representation
	$f(t,\exp(\mu_1 t),\ldots,\linebreak[0]\exp(\mu_k t))$
	of $\varphi(t)$, where $f$ is a $(k+1)$-variate polynomial with algebraic coefficients.
	That is, $\varphi(t)$ is obtained by
	substituting $t,\exp(\mu_1 t),\ldots,\exp(\mu_k t)$ (as $k+1$ variables) into $f$.
	\item \textbf{Factoring}
	Factoring $\varphi(t)$ into irreducible factors $\varphi_i(t)$ ($1 \le i \le \ell$)
	corresponds to
	factoring the $(k+1)$-variate $\mathbb{A}$-polynomial $f$
	into irreducible factors $f_i$ ($1 \le i \le \ell$),
	which has been implemented in polynomial time, e.g.~\cite{Kal85}.
	\item \textbf{Univariate?}
	If the irreducible exponential polynomial $\varphi_i(t)$ corresponds
	a univariate polynomial $f_i$,
	isolating the real roots of $\varphi_i(t)$ can be treated by classical methods~\cite{CoL76};
	otherwise we will resort to Algorithm~\ref{isol},
	for which we can infer:
	\begin{enumerate}
		\item The exponential polynomial $\varphi(t)$ in the form~\eqref{eq:EP}
	    is plainly an \emph{analytic} function that is infinitely differentiable;
	    and it is a real-valued function,
	    or equivalently the imaginary part of $\varphi(t)$ is identically zero,
	    since each variable $x_s(t)$ is exactly the real-valued function $\tr(\PP_s \rho(t))$.
	    The same holds for any factor $\varphi_i(t)$ of $\varphi(t)$,
	    which ensures the first requirement of the input in Algorithm~\ref{isol}.
        \item By Theorem~\ref{Lindemann},
        thanks to the irreducibility of $\varphi_i(t)$,
	    we have $\varphi_i(\lambda) \ne 0$ holds for any $\lambda\in\mathbb{A}\setminus\{0\}$,
	    which ensures the second requirement of the input in Algorithm~\ref{isol}. 
	    \item By Conjucture~\ref{Schanuel} and Corollary~\ref{cor:common},
        each irreducible factor $\varphi_i(t)$ and its derivative $\varphi_i'(t)$
		are co-prime,
		and thus have no common real root except for $0$,
		which ensures the last requirement of the input in Algorithm~\ref{isol}.
	\end{enumerate}
	\item \textbf{Refinement}
		After performing Algorithm~\ref{isol}
		with each individual irreducible factor $\varphi_i(t)$ of $\varphi(t)$,
		we would obtain a list of disjoint isolation intervals $\I_{i,1},\ldots,\I_{i,n_i}$.
		The isolation intervals of different irreducible factors may be overlapping.
		However, by Corollary~\ref{cor:common} again,
		we have that
		each pair of co-prime factors of $\varphi(t)$ has no common real root except for $0$.
        So all these isolation intervals $\I_{i,1},\ldots,\I_{i,n_i}$ ($1 \le i \le \ell$)
		can be further refined to be pairwise disjoint.
		That would be the complete list of isolation intervals $\I_1,\ldots,\I_n$ for $\varphi(t)$,
		and thereby completes the whole isolation procedure.
\end{itemize}

\begin{algorithm}[ht!]
	\caption{\textsf{Real Root Isolation for a Real-valued Function}}\label{isol}
	\begin{algorithmic}[1]
		\item[] $$\{\I_1,\ldots,\I_n\} \Leftarrow {\sf Isolate}(\varphi,\I)$$
		\Require $\varphi(t)$ is a real-valued function
			defined on a rational interval $\I=[l,u]$,
			satisfying:
			\begin{enumerate}
			    \item $\varphi(t)$ is twice-differentiable,
				\item $\varphi(t)$ has no rational root in $\I$, and
				\item $\varphi(t)$ and $\varphi'(t)$ have no common real root in $\I$.
			\end{enumerate}
		\Ensure $\I_1,\ldots,\I_n$ are finitely many disjoint intervals,
			such that each contains exactly one real root of $\varphi$ in $\I$,
			and together contain all.
		\State compute an upper bound $M$ of $\{|\varphi'(t)|\,:\,t\in\I\}$;
		\State compute an upper bound $M'$ of $\{|\varphi''(t)|\,:\,t\in\I\}$;
		\State $i \gets 0$, $N \gets 2$ and $\delta \gets (u-l)/N$;
		\Comment{Here $N>1$ is a free parameter to indicate the number of subintervals to be split.
			We predefine it simply as $2$.}
		\While{$i \le N$}
		\If{$|\varphi(l+i\delta)|>M\delta$}\label{ln:unsat1}
			$i \gets i+1$;
		\Comment{$\varphi$ has no local real root}
		\ElsIf{$|\varphi(l+i\delta+\delta)|>M\delta$}\label{ln:unsat2}
			$i \gets i+2$;
		\Comment{$\varphi$ has no local real root}
		\ElsIf{$|\varphi'(l+i\delta)| \ge M'\delta$}\label{ln:dec1}
		\Comment{$\varphi$ is locally monotonic}
		\If{$\varphi(l+i\delta)\varphi(l+i\delta+\delta)<0$}\label{ln:sat1}
			\textbf{output} $(l+i\delta,l+i\delta+\delta)$;
		\EndIf
		\State $i \gets i+1$;
		\ElsIf{$|\varphi'(l+i\delta+\delta)| \ge M'\delta$}\label{ln:dec2}
		\Comment{$\varphi$ is locally monotonic}
		\If{$\varphi(l+i\delta)\varphi(l+i\delta+2\delta)<0$}\label{ln:sat2}
			\textbf{output} $(l+i\delta,l+i\delta+2\delta)$;
		\EndIf
		\State $i \gets i+2$;
		\Else
		\State\label{ln:rec} ${\sf Isolate}(\varphi,[l+i\delta,l+i\delta+\delta])$;
		\State $i \gets i+1$.
		\EndIf
		\EndWhile
	\end{algorithmic}
\end{algorithm}

% \begin{algorithm}[ht!]
% 	\caption{\textsf{Real Root Isolation for a Real-valued Function}}\label{isol}
% 	\begin{algorithmic}[1]
% 		\item[] $$\{\I_1,\ldots,\I_n\} \Leftarrow {\sf Isolate}_N(\varphi,\I)$$
% 		\Require $\varphi(t)$ is a twice-differentiable real-valued function
% 		defined on a rational interval $\I=[l,u]$,
% 		satisfying:
% 		\begin{enumerate}
% 			\item $\varphi(t)$ has no rational root in $\I$, and
% 			\item $\varphi(t)$ and $\varphi'(t)$ have no common real root in $\I$.
% 		\end{enumerate}
% 		An integer $N\ge 2$ indicates the number of subintervals that $\I$ will be divided into.
% 		\Ensure $\I_1,\ldots,\I_n$ are finitely many disjoint intervals,
% 		such that each contains exactly one real root of $\varphi$ in $\I$,
% 		and together contain all.
% 		\State compute an upper bound $M$ of $\{|\varphi'(t)|\,:\,t\in\I\}$;
% 		\State compute an upper bound $M'$ of $\{|\varphi''(t)|\,:\,t\in\I\}$;
% 		\State $i \gets 0$ and $\delta \gets (u-l)/N$;
% 		\For{$i$ \textbf{from} $0$ \textbf{to} $N-1$}
% 		\State $a \gets l+i\delta$ and $b \gets l+(i+1)\delta$;
% 		\If{$ \max(|\varphi(a)|, |\varphi(b)|)>M\delta$ }\label{ln:unsat1}
% 		\State \textbf{continue};
% 		\ElsIf{$\max(|\varphi'(a)|, |\varphi'(b)|) \ge M'\delta$}\label{ln:dec1}
% 		\If{$\varphi(a)\varphi(b)<0$}\label{ln:sat1}
% 			\textbf{output} $(a,b)$;
% 		\EndIf
% 		\Else
% 		\State\label{ln:rec} ${\sf Isolate}_N(\varphi,[a,b])$;
% 		\EndIf
% 		\EndFor
% 	\end{algorithmic}
% \end{algorithm}	

\subparagraph*{Soundness of Algorithm~\ref{isol}}
We justify three groups of treatment in the loop body in turn.
\begin{enumerate}
	\item If the condition in Line~\ref{ln:unsat1} (resp.~\ref{ln:unsat2}) holds,
		$\varphi$ has no real root in the neighborhood
		centered at $l+i\delta$ (resp.~$l+(i+1)\delta$) with radius $\delta$.
		So we exclude the neighborhood.
	\item If the condition in Line~\ref{ln:dec1} (resp.~\ref{ln:dec2}) holds,
		$\varphi$ is monotonic in the neighborhood
		centered at $l+i\delta$ (resp.~$l+(i+1)\delta$) with radius $\delta$.
		Then, if the condition in Line~\ref{ln:sat1} (resp.~\ref{ln:sat2}) holds,
		i.e. $\varphi$ has different signs at endpoints of the neighborhood,
		the unique real root in the neighborhood exists and should be output;
		otherwise we just exclude the neighborhood.
	\item If it is not in the two decisive cases listed above,
		we perform Algorithm~\ref{isol} recursively
		in a subinterval $[l+i\delta,l+(i+1)\delta]$ of $[l,u]$. \qed
\end{enumerate}
	
% \begin{enumerate}
% 	\item If the condition in Line~\ref{ln:unsat1} holds,
% 		the absolute value of the derivative of $\varphi(x)$ is not sufficient large to make its value reach zero within the step $\delta$,
% 		which implies $\varphi$ has no real root in $[l+i\delta, l+(i+1)\delta]$.
% 		So we exclude the subinterval.
% 	\item If the condition in Line~\ref{ln:dec1} holds,
% 		$\varphi$ is monotonous in $[l+i\delta, l+(i+1)\delta]$.
% 		Then, if the condition in Line~\ref{ln:sat1} holds,
% 		i.e. $\varphi$ has different signs at endpoints of subinterval,
% 		the unique real root in the subinterval exists and should be output;
% 		otherwise we just exclude the subinterval.
% 	\item If it is not in the two decisive cases listed above,
% 		we perform Algorithm~\ref{isol} recursively
% 		in the subinterval $[l+i\delta,l+(i+1)\delta]$ of $[l,u]$. \qed
% \end{enumerate}
	
\subparagraph*{Completeness of Algorithm~\ref{isol}}
The termination of Algorithm~\ref{isol} entails the completeness.
In other words, it suffices to show that
for any real-valued function $\varphi$ that satisfies the requirements,
Algorithm~\ref{isol} can always output all real roots of $\varphi$
within a finitely many times of recursion.
Since $\varphi$ and $\varphi'$ are real-valued functions defined
on the closed and bounded interval $\I$
and they have no common real root in $\I$,
there is a positive constant $\tau$,
such that either $|\varphi| \ge \tau$ or $|\varphi'| \ge \tau$ holds everywhere of $\I$.
Then, at any point $c$ in $\I$,
we can get a neighborhood with constant radius $rad:=\min(\tau/M,\tau/M')$,
in which either $\varphi$ has no real root or is monotonic.
So the two decisive cases must take place,
provided that the subinterval has length not greater than $rad$.
It implies that the recursion depth of Algorithm~\ref{isol} is bounded
by $\lceil \log_2(\|\I\|/rad) \rceil$, where $\|\I\| = \sup\I-\inf\I$.
Hence the termination is guaranteed. \qed

After obtaining all real roots $\lambda_1,\ldots,\lambda_n$ of $\varphi(t)$
in the bounded interval $\mathcal{B}$ by Algorithm~\ref{isol},
we have that
on each interval $\J_i$ ($0 \le i \le n$) of the $n+1$ intervals
in $\mathcal{B}\setminus\{\lambda_1,\ldots,\lambda_n\}$:
\begin{equation}\label{eq:interval}
	[\inf\mathcal{B},\lambda_1),(\lambda_1,\lambda_2),\ldots,
	(\lambda_{n-1},\lambda_n),(\lambda_n,\sup\mathcal{B}],
\end{equation}
$p(\x(t)) \in \II$ holds everywhere of $t\in \J_i$ or nowhere.
Finally, we can obtain the desired solution set $ \JJ$ (to be used in the next section)
of $p(\x(t)) \in \II$ by a finite union as follows
\begin{equation}
   \bigcup_{\substack{0 \le i \le n \\ p(\x(\J_i)) \subseteq \II}} \J_i  \cup
    \bigcup_{\substack{1 \le i \le n \\ p(\x(\lambda_i)) \in \II}} \{\lambda_i\}.
\end{equation}

\begin{example}\label{ex3}
Reconsider Example~\ref{ex2}. 
The exponential polynomial extracted from $\Phi_1$ is
\[
	\varphi_1(t) = x_{0,1}(t) + x_{1,0}(t) - \tfrac{1}{5}
	= -\tfrac{1}{5}	- \tfrac{\sqrt{2}}{2}\exp(-\tfrac{2+\sqrt{2}}{2}t)
	+ \tfrac{\sqrt{2}}{2}\exp(-\tfrac{2-\sqrt{2}}{2}t),
\]
and the exponential polynomial extracted from $\Phi_2$ is
\[
	\varphi_2(t) = x_{1,1}(t) - x_{0,1}(t) - x_{1,0}(t) 
	= 1+(\sqrt{2}-\tfrac{1}{2})\exp(-\tfrac{2+\sqrt{2}}{2}t)
	- (\sqrt{2}+\tfrac{1}{2})\exp(-\tfrac{2-\sqrt{2}}{2}t).
\] 
To solve $\Phi_1$ and $\Phi_2$ in a bounded interval, say $\mathcal{B}=[0,6]$,
we need to determine the real roots of $\varphi_1$ and $\varphi_2$.
Since both $\varphi_1$ and $\varphi_2$ are irreducible
and their corresponding polynomial representations are bivariate,
they have no rational root, repeated real root, nor common real root. 
After invoking Algorithm~\ref{isol} on $\varphi_1$ with $\mathcal{B}$,
we obtain two isolation intervals $[0,\tfrac{25}{64}]$
(containing real root $\lambda_1 \approx 0.352097$)
and $[\tfrac{275}{64},\tfrac{575}{128}]$ (containing $\lambda_2 \approx 4.49181$),
which could be easily refined up to any precision.
For $\varphi_2$,
as $\varphi_2(0) = 0$, 
we make a slight shift on the left endpoint of $\mathcal{B}$ in the consideration,
e.g., $[\tfrac{1}{1000},6]$.
After invoking Algorithm~\ref{isol} on $\varphi_2$ with $[\tfrac{1}{1000},6]$,
we could get the unique isolation intervals	$[\tfrac{125059}{64000},\tfrac{275117}{128000}]$,
which contains the real root $\lambda_3 \approx 2.14897$.
The three isolation intervals are pairwise disjoint.

Finally, we have that
the solution set of $\Phi_1$ is $(\lambda_1,\lambda_2)$ in $\mathcal{B}$,
and the solution set of $\Phi_2$ is $\{0\} \cup [\lambda_3,6]$. \qed
\end{example}

Under Conjecture~\ref{Schanuel}, we get the following computability result
while the complexity is still an open problem as left in existing literature~\cite{COW16,HLX+18}.
\begin{lemma}
	The atomic propositions in STL are solvable over QCTMCs.
\end{lemma}

In fact, Conjecture~\ref{Schanuel} is a powerful tool
to treat roots of the general exponential polynomial.
For some special subclasses of exponential polynomials,
there are solid theorems to treat them:
one is Theorem~\ref{Lindemann} that has been employed~\cite{AMW08}
for the exponential polynomials in the ring $\mathbb{Q}[t,\exp(t)]$,
the other is the Gelfond--Schneider theorem employed~\cite[Subsection~4.1]{HLX+18}
for the exponential polynomials in $\mathbb{Q}[\exp(\mu_1 t),\exp(\mu_2 t)]$
where $\mu_1$ and $\mu_2$ are
\emph{two} $\mathbb{Q}$-linear independent \emph{real} algebraic numbers.
In Example~\ref{ex3}, the Gelfond--Schneider theorem is sufficient to
treat roots of the exponential polynomials $\varphi_1$ and $\varphi_2$,
thus the termination is guaranteed.
Algorithm~\ref{isol} can isolate roots of elements in
$\mathbb{Q}[t,\exp(\mu_1 t),\ldots,\exp(\mu_k t)]$
for \emph{arbitrarily many} $\mathbb{Q}$-linear independent
\emph{complex} algebraic numbers $\mu_1,\ldots,\mu_k$.
Additionally, Conjecture~\ref{Schanuel} has been employed
to isolate simple roots of more expressible functions
than exponential polynomials~\cite{Str08,Str09},
but fails to find repeated roots.
Hence, this paper makes a trade-off
between the expressibility of functions and the completeness of methodologies.

	
\section{Checking STL Formulas}\label{S6}
In the previous section, we have solved atomic propositions.
Now we consider the general STL formula $\phi$.
For a given formula $\phi$,
we compute the so-called \emph{post-monitoring period} $\mnt(\phi)$,
independent on the initial time $t_0$,
such that the truth of $\rho(t_0)\models\phi$ could be affected
by those $\rho(t)$ with $t \in [t_0,t_0+\mnt(\phi)]$.
Then we decide $\phi$ with a bottom-up fashion.
The complexity turns out to be linear in the size $\|\phi\|$ of the input STL formula $\phi$,
which is defined as the number of logic connectives in $\phi$ as standard.

Given an STL formula $\phi$,
we need to post-monitor a time period to decide the truth of $\rho(t_0)\models\phi$
at an initial time $t_0$,
especially for the until formula $\phi_1 \ntl^\I \phi_2$.
For example, according to the semantics of STL,
to decide $\rho(t_0)\models \phi_2$
with $\phi_2 \equiv \Phi_1 \ntl^{\I_1} (\Phi_2 \ntl^{\I_2} \Phi_3)$,
we have to monitor the states $\rho(t)$ from time $t_0$ to $t_0+\sup\I_1+\sup\I_2$.
It inspires us
to calculate the post-monitoring period $\mnt(\phi)$ as
\begin{equation}\label{mnt}
	\mnt(\phi)=\begin{cases}
		0 & \textup{if }\phi=\Phi, \\
		\mnt(\phi_1) & \textup{if }\phi=\neg\phi_1, \\
		\max(\mnt(\phi_1),\mnt(\phi_2))	& \textup{if }\phi=\phi_1 \wedge \phi_2, \\
		\sup\I+\max(\mnt(\phi_1),\mnt(\phi_2)) & \textup{if }\phi=\phi_1 \ntl^\I \phi_2.
	\end{cases}
\end{equation}

\begin{lemma}\label{lem:mnt}
	Given an STL formula $\phi$,
	the satisfaction $\rho(t_0) \models \phi$ is entirely determined by
	the states $\rho(t)$ with $t_0 \le t \le t_0+\mnt(\phi)$.
\end{lemma}
\begin{proof}
	We discuss it upon the syntactical structure of the STL formula $\phi$.
	
	For the atomic proposition $\Phi$,
	$\rho(t_0) \models \Phi$ is plainly determined by $\rho(t_0)$.
	    	
	For the negation $\neg\phi_1$,
	if $\rho(t_0) \models \phi_1$
	is determined by $\rho(t)$ with $t_0 \le t \le t_0+\mnt(\phi_1)$,
	so is $\rho(t_0) \models \neg\phi_1$.
			
	For the conjunction $\phi_1 \wedge \phi_2$,
	if $\rho(t_0) \models \phi_1$ (resp.~$\rho(t_0) \models \phi_2$)
	is determined by $\rho(t)$ with $t_0 \le t \le t_0+\mnt(\phi_1)$
	(resp.~$t_0 \le t \le t_0+\mnt(\phi_2)$),
	$\rho(t_0) \models \phi_1 \wedge \phi_2$ is determined by the union of those states,
	i.e. $\rho(t)$ with $t_0 \le t \le t_0+\max(\mnt(\phi_1),\mnt(\phi_2))$.
			
	For the until formula $\phi_1 \ntl^\I \phi_2$,
	if $\rho(t_0) \models \phi_1$ (resp.~$\rho(t_0) \models \phi_2$)
	is determined by $\rho(t)$ with $t_0 \le t \le t_0+\mnt(\phi_1)$
	(resp.~$t_0 \le t \le t_0+\mnt(\phi_2)$),
	$\rho(t_0) \models \phi_1 \ntl^\I \phi_2$ is determined by
	$\rho(t)$ with $t_0 \le t \le t_0+\sup\I+\max(\mnt(\phi_1),\mnt(\phi_2))$,
	where $\sup\I$ is caused by the admissible transition at the latest time in the until formula,
	since then we have to determine $\rho(t) \models \phi_1$ from time $t_0$ to $t_0+\sup\I$
	and determine $\rho(t_0+\sup\I) \models \phi_2$.
\end{proof}
	
To decide $\rho(0) \models \phi$,
our method is based on the parse tree $\mathcal{T}$ of $\phi$ as follows.

Basically, for each leaf of $\mathcal{T}$
that represents an atomic proposition $\Phi$,
we compute the solution set $\JJ$ (possibly a union of maximal solution intervals $\J$)
of $\Phi$ within the monitoring interval $\mathcal{B}:=[0,\mnt(\phi)]$
by Algorithm~\ref{isol}.

Inductively, for each intermediate node of $\mathcal{T}$
that represents the subformula $\psi$ of $\phi$,
we tackle it into three classes.
\begin{itemize}
	\item If $\psi=\neg \phi_1$,
		supposing that $\JJ_1$ is the solution set of $\phi_1$,
		the solution set $\JJ'$ of $\psi$
		is $\mathcal{B} \setminus \JJ_1$.
	\item If $\psi=\phi_1 \wedge \phi_2$,
		supposing that $\JJ_1$ (resp.~$\JJ_2$) is the solution set of
		$\phi_1$ (resp.~$\phi_2$),
		the solution set $\JJ'$ of $\psi$ is $\JJ_1 \cap \JJ_2$.
	\item If $\psi=\phi_1 \ntl^\I \phi_2$,
		supposing that $\JJ_1$ (resp.~$\JJ_2$) is the solution set of
		$\phi_1$ (resp.~$\phi_2$),
		the solution set $\JJ'$ of $\psi$ is
		\[
		\{t : (t'\in\I) \wedge ([t,t+t') \subseteq \JJ_1 )\wedge (t+t'\in\JJ_2)\}.
		\]
		directly from the semantics of the until formula $\phi_1 \ntl^\I \phi_2$
		in Definition~\ref{def:semantics}, as
		\begin{itemize}
			\item $[t,t+t') \subseteq \JJ_1$
			if and only if $\forall\,t_1 \in [t,t+t') \,:\, \rho(t_1) \models \phi_1$, and 
			\item $t+t'\in\JJ_2$ if and only if $\rho(t+t') \models \phi_2$.
		\end{itemize}
    \end{itemize}
Note that the inductive steps of the above procedure
do not generally produce all solutions of the subformula $\psi$ in $\mathcal{B}$.
Since by Lemma~\ref{lem:mnt}
$\rho(t_0) \models \psi$ is entirely determined
by $\rho(t)$ with $t_0 \le t \le t_0+\mnt(\psi)$,
the resulting solution set $\JJ'$ contains all solutions of $\psi$ in $[0,\mnt(\phi)-\mnt(\psi)]$
and possibly misses some solutions in the right subinterval
$(\mnt(\phi)-\mnt(\psi),\mnt(\phi)]$.
Anyway, the subinterval $[0,\mnt(\phi)-\mnt(\psi)]$ has the left-closed endpoint $0$,
which suffices to decide $\rho(0) \models \phi$.

With a bottom-up fashion,
we could eventually get the solution set $\JJ$ of $\phi$,
by which $\rho(0) \models \phi$ can be decided to be true if and only if $0 \in \JJ$.
Overall, the procedure costs $\|\phi\|$ times of the interval operations
and at most $\|\phi\|$ times of calling Algorithm~\ref{isol}
for getting the solution set of an atomic proposition.
That is, the query complexity of model checking STL formulas is linear in $\|\phi\|$
by calling Algorithm~\ref{isol}.
The query complexity addresses the issue of the number of calls to a black box routine
with unknown complexities and is commonly used in quantum computing,
where the routine is usually called an oracle.
For example,
oracles can be a procedure of encoding an entry of a matrix into a quantum state~\cite{HHL09}
or a quantum circuit of preparing a specific quantum state~\cite{GWY21}.

\begin{example}
	For the STL formula
	$\phi_1 \equiv \neg(\true\,\ntl^{\I_1}\,(\Phi_1 \wedge \neg(\true\,\ntl^{\I_2} \Phi_2)))$,
	the post-monitoring period $\mnt(\phi_1)$ is $\sup\I_1+\sup\I_2=6$ by Eq.~\eqref{mnt},
	implying $\mathcal{B}=[0,6]$ as used in Example~\ref{ex3}.
	We construct the parse tree of $\phi_1$
	in Figure~\ref{fig:parse}.
	Based on it,
	we could calculate the solution sets of all nodes in a bottom-up fashion:
	\begin{itemize}
		\item Basically, the solution set for $\Phi_1$ is $(\lambda_1,\lambda_2)$
		where $\lambda_1 \approx 0.352097$ and $\lambda_2 \approx 4.49181$,
		the solution set for $\Phi_2$ is $\{0\}\cup [\lambda_3,6]$ where $\lambda_3 \approx 2.14897$,
		and the solution set for $\true$ is plainly $[0,6]$.
		The post-monitoring periods of the three leaves are $0$. 
		Since $\mathcal{B}$ is $[0,6]$, 
		we have got all the solutions in $[0,6]$.
		\item The solution set for $\true\,\ntl^{\I_2} \Phi_2$ is $[\lambda_3-1,6]$,
		and that for the negation is $[0,\lambda_3-1)$.
		The post-monitoring periods of the two nodes are $1$,
		thus we have got all solutions in $[0,5]$.
		\item The solution set for $\Phi_1 \wedge \neg(\true\,\ntl^{\I_2} \Phi_2)$
		is $(\lambda_1,\lambda_3-1)$.
		Its post-monitoring period is $1$, 
		thus we have got all solutions in $[0,5]$.
		\item Finally, the solution set for
		$\true\,\ntl^{\I_1}\, (\Phi_1 \wedge \neg(\true\,\ntl^{\I_2} \Phi_2))$ is $[0,\lambda_3-1)$,
		and that for the negation (the root, representing the whole STL formula $\phi_1$)
		is $[\lambda_3-1,6]$.
		Their post-monitoring periods are $6$,
		thus we have got the solution in $[0,0]$.
	\end{itemize}
    Since $0\notin [\lambda_3-1,6]$, 
    we can decide $\rho(0) \models \phi_1$ to be false.
    Hence the particle walking along Figure~\ref{fig:QW} does not satisfy
    the desired convergence performance---Property A. \qed
		
\begin{figure}[ht]
	\begin{tikzpicture}[->,>=stealth',auto,node distance=1.8cm,semithick,inner sep=2pt]
		\node[state,rectangle](s1){$\neg$};
		\node[state,rectangle](s2)[right of=s1]{$\ntl^{\I_1}$};
		\node[state,rectangle](s3)[right of=s2]{$\wedge$};
		\node[state,rectangle](s4)[below of=s3]{$\true$};
		\node[state,rectangle](s5)[right of=s3]{$\neg$};
		\node[state,rectangle](s6)[below of=s5]{$\Phi_1$};
		\node[state,rectangle](s7)[right of=s5]{$\ntl^{\I_2}$};
		\node[state,rectangle](s8)[right of=s7]{$\Phi_2$};
		\node[state,rectangle](s9)[below of=s8]{$\true$};
	
		\draw[->](s1)edge[]node{}(s2);
		\draw[->](s2)edge[]node{}(s3);
		\draw[->](s2)edge[]node{}(s4);
		\draw[->](s3)edge[]node{}(s5);
		\draw[->](s3)edge[]node{}(s6);
		\draw[->](s5)edge[]node{}(s7);
		\draw[->](s7)edge[]node{}(s8);
		\draw[->](s7)edge[]node{}(s9);
	\end{tikzpicture}
	\caption{Parse tree of the STL formula $\phi_1$}\label{fig:parse}
\end{figure}

%The solution set of the subformula $\psi \equiv \true\,\ntl^{\I_2} \Phi_2$ here
%is $[\lambda_3-1,6]$,
%which reports all solutions in the interval $[0,\mnt(\phi_1)-\mnt(\psi)]=[0,5]$. \qed
\end{example}

Finally, under Conjecture~\ref{Schanuel}, we obtain the main result:
\begin{theorem}
	The STL formulas are decidable over QCTMCs.
\end{theorem}


\section{Concluding Remarks}\label{S7}
In this paper, we introduced the model of QCTMC that extends CTMC,
and established the decidability of the STL formulas over it.
To this goal, we firstly solved the atomic propositions in STL
by real root isolation of a wide class of exponential polynomials,
whose completeness was based on Schanuel's conjecture.
Then we decided the general STL formula using interval operations with a bottom-up fashion,
whose (query) complexity turns out to be linear in the size of the input formula
by calling the developed state-of-art real root isolation routine.
We demonstrated our method by a running example of an open quantum walk.

For future work,
we would like to explore the following aspects:
\begin{itemize}
	\item how to apply the proposed method to verify non-Markov models in the real world~\cite{Pel14};
    \item how to design an efficient numerical approximation of the exact method in this paper;
    \item and checking other formal logics, e.g.~\cite{ASS+96}, over the QCTMC.
\end{itemize}


\bibliography{ref}
	
%\newpage
%\appendix
%\chapter{Supplementary Material}
\label{appendix}

In this appendix, we present supplementary material for the techniques and
experiments presented in the main text.

\section{Baseline Results and Analysis for Informed Sampler}
\label{appendix:chap3}

Here, we give an in-depth
performance analysis of the various samplers and the effect of their
hyperparameters. We choose hyperparameters with the lowest PSRF value
after $10k$ iterations, for each sampler individually. If the
differences between PSRF are not significantly different among
multiple values, we choose the one that has the highest acceptance
rate.

\subsection{Experiment: Estimating Camera Extrinsics}
\label{appendix:chap3:room}

\subsubsection{Parameter Selection}
\paragraph{Metropolis Hastings (\MH)}

Figure~\ref{fig:exp1_MH} shows the median acceptance rates and PSRF
values corresponding to various proposal standard deviations of plain
\MH~sampling. Mixing gets better and the acceptance rate gets worse as
the standard deviation increases. The value $0.3$ is selected standard
deviation for this sampler.

\paragraph{Metropolis Hastings Within Gibbs (\MHWG)}

As mentioned in Section~\ref{sec:room}, the \MHWG~sampler with one-dimensional
updates did not converge for any value of proposal standard deviation.
This problem has high correlation of the camera parameters and is of
multi-modal nature, which this sampler has problems with.

\paragraph{Parallel Tempering (\PT)}

For \PT~sampling, we took the best performing \MH~sampler and used
different temperature chains to improve the mixing of the
sampler. Figure~\ref{fig:exp1_PT} shows the results corresponding to
different combination of temperature levels. The sampler with
temperature levels of $[1,3,27]$ performed best in terms of both
mixing and acceptance rate.

\paragraph{Effect of Mixture Coefficient in Informed Sampling (\MIXLMH)}

Figure~\ref{fig:exp1_alpha} shows the effect of mixture
coefficient ($\alpha$) on the informed sampling
\MIXLMH. Since there is no significant different in PSRF values for
$0 \le \alpha \le 0.7$, we chose $0.7$ due to its high acceptance
rate.


% \end{multicols}

\begin{figure}[h]
\centering
  \subfigure[MH]{%
    \includegraphics[width=.48\textwidth]{figures/supplementary/camPose_MH.pdf} \label{fig:exp1_MH}
  }
  \subfigure[PT]{%
    \includegraphics[width=.48\textwidth]{figures/supplementary/camPose_PT.pdf} \label{fig:exp1_PT}
  }
\\
  \subfigure[INF-MH]{%
    \includegraphics[width=.48\textwidth]{figures/supplementary/camPose_alpha.pdf} \label{fig:exp1_alpha}
  }
  \mycaption{Results of the `Estimating Camera Extrinsics' experiment}{PRSFs and Acceptance rates corresponding to (a) various standard deviations of \MH, (b) various temperature level combinations of \PT sampling and (c) various mixture coefficients of \MIXLMH sampling.}
\end{figure}



\begin{figure}[!t]
\centering
  \subfigure[\MH]{%
    \includegraphics[width=.48\textwidth]{figures/supplementary/occlusionExp_MH.pdf} \label{fig:exp2_MH}
  }
  \subfigure[\BMHWG]{%
    \includegraphics[width=.48\textwidth]{figures/supplementary/occlusionExp_BMHWG.pdf} \label{fig:exp2_BMHWG}
  }
\\
  \subfigure[\MHWG]{%
    \includegraphics[width=.48\textwidth]{figures/supplementary/occlusionExp_MHWG.pdf} \label{fig:exp2_MHWG}
  }
  \subfigure[\PT]{%
    \includegraphics[width=.48\textwidth]{figures/supplementary/occlusionExp_PT.pdf} \label{fig:exp2_PT}
  }
\\
  \subfigure[\INFBMHWG]{%
    \includegraphics[width=.5\textwidth]{figures/supplementary/occlusionExp_alpha.pdf} \label{fig:exp2_alpha}
  }
  \mycaption{Results of the `Occluding Tiles' experiment}{PRSF and
    Acceptance rates corresponding to various standard deviations of
    (a) \MH, (b) \BMHWG, (c) \MHWG, (d) various temperature level
    combinations of \PT~sampling and; (e) various mixture coefficients
    of our informed \INFBMHWG sampling.}
\end{figure}

%\onecolumn\newpage\twocolumn
\subsection{Experiment: Occluding Tiles}
\label{appendix:chap3:tiles}

\subsubsection{Parameter Selection}

\paragraph{Metropolis Hastings (\MH)}

Figure~\ref{fig:exp2_MH} shows the results of
\MH~sampling. Results show the poor convergence for all proposal
standard deviations and rapid decrease of AR with increasing standard
deviation. This is due to the high-dimensional nature of
the problem. We selected a standard deviation of $1.1$.

\paragraph{Blocked Metropolis Hastings Within Gibbs (\BMHWG)}

The results of \BMHWG are shown in Figure~\ref{fig:exp2_BMHWG}. In
this sampler we update only one block of tile variables (of dimension
four) in each sampling step. Results show much better performance
compared to plain \MH. The optimal proposal standard deviation for
this sampler is $0.7$.

\paragraph{Metropolis Hastings Within Gibbs (\MHWG)}

Figure~\ref{fig:exp2_MHWG} shows the result of \MHWG sampling. This
sampler is better than \BMHWG and converges much more quickly. Here
a standard deviation of $0.9$ is found to be best.

\paragraph{Parallel Tempering (\PT)}

Figure~\ref{fig:exp2_PT} shows the results of \PT sampling with various
temperature combinations. Results show no improvement in AR from plain
\MH sampling and again $[1,3,27]$ temperature levels are found to be optimal.

\paragraph{Effect of Mixture Coefficient in Informed Sampling (\INFBMHWG)}

Figure~\ref{fig:exp2_alpha} shows the effect of mixture
coefficient ($\alpha$) on the blocked informed sampling
\INFBMHWG. Since there is no significant different in PSRF values for
$0 \le \alpha \le 0.8$, we chose $0.8$ due to its high acceptance
rate.



\subsection{Experiment: Estimating Body Shape}
\label{appendix:chap3:body}

\subsubsection{Parameter Selection}
\paragraph{Metropolis Hastings (\MH)}

Figure~\ref{fig:exp3_MH} shows the result of \MH~sampling with various
proposal standard deviations. The value of $0.1$ is found to be
best.

\paragraph{Metropolis Hastings Within Gibbs (\MHWG)}

For \MHWG sampling we select $0.3$ proposal standard
deviation. Results are shown in Fig.~\ref{fig:exp3_MHWG}.


\paragraph{Parallel Tempering (\PT)}

As before, results in Fig.~\ref{fig:exp3_PT}, the temperature levels
were selected to be $[1,3,27]$ due its slightly higher AR.

\paragraph{Effect of Mixture Coefficient in Informed Sampling (\MIXLMH)}

Figure~\ref{fig:exp3_alpha} shows the effect of $\alpha$ on PSRF and
AR. Since there is no significant differences in PSRF values for $0 \le
\alpha \le 0.8$, we choose $0.8$.


\begin{figure}[t]
\centering
  \subfigure[\MH]{%
    \includegraphics[width=.48\textwidth]{figures/supplementary/bodyShape_MH.pdf} \label{fig:exp3_MH}
  }
  \subfigure[\MHWG]{%
    \includegraphics[width=.48\textwidth]{figures/supplementary/bodyShape_MHWG.pdf} \label{fig:exp3_MHWG}
  }
\\
  \subfigure[\PT]{%
    \includegraphics[width=.48\textwidth]{figures/supplementary/bodyShape_PT.pdf} \label{fig:exp3_PT}
  }
  \subfigure[\MIXLMH]{%
    \includegraphics[width=.48\textwidth]{figures/supplementary/bodyShape_alpha.pdf} \label{fig:exp3_alpha}
  }
\\
  \mycaption{Results of the `Body Shape Estimation' experiment}{PRSFs and
    Acceptance rates corresponding to various standard deviations of
    (a) \MH, (b) \MHWG; (c) various temperature level combinations
    of \PT sampling and; (d) various mixture coefficients of the
    informed \MIXLMH sampling.}
\end{figure}


\subsection{Results Overview}
Figure~\ref{fig:exp_summary} shows the summary results of the all the three
experimental studies related to informed sampler.
\begin{figure*}[h!]
\centering
  \subfigure[Results for: Estimating Camera Extrinsics]{%
    \includegraphics[width=0.9\textwidth]{figures/supplementary/camPose_ALL.pdf} \label{fig:exp1_all}
  }
  \subfigure[Results for: Occluding Tiles]{%
    \includegraphics[width=0.9\textwidth]{figures/supplementary/occlusionExp_ALL.pdf} \label{fig:exp2_all}
  }
  \subfigure[Results for: Estimating Body Shape]{%
    \includegraphics[width=0.9\textwidth]{figures/supplementary/bodyShape_ALL.pdf} \label{fig:exp3_all}
  }
  \label{fig:exp_summary}
  \mycaption{Summary of the statistics for the three experiments}{Shown are
    for several baseline methods and the informed samplers the
    acceptance rates (left), PSRFs (middle), and RMSE values
    (right). All results are median results over multiple test
    examples.}
\end{figure*}

\subsection{Additional Qualitative Results}

\subsubsection{Occluding Tiles}
In Figure~\ref{fig:exp2_visual_more} more qualitative results of the
occluding tiles experiment are shown. The informed sampling approach
(\INFBMHWG) is better than the best baseline (\MHWG). This still is a
very challenging problem since the parameters for occluded tiles are
flat over a large region. Some of the posterior variance of the
occluded tiles is already captured by the informed sampler.

\begin{figure*}[h!]
\begin{center}
\centerline{\includegraphics[width=0.95\textwidth]{figures/supplementary/occlusionExp_Visual.pdf}}
\mycaption{Additional qualitative results of the occluding tiles experiment}
  {From left to right: (a)
  Given image, (b) Ground truth tiles, (c) OpenCV heuristic and most probable estimates
  from 5000 samples obtained by (d) MHWG sampler (best baseline) and
  (e) our INF-BMHWG sampler. (f) Posterior expectation of the tiles
  boundaries obtained by INF-BMHWG sampling (First 2000 samples are
  discarded as burn-in).}
\label{fig:exp2_visual_more}
\end{center}
\end{figure*}

\subsubsection{Body Shape}
Figure~\ref{fig:exp3_bodyMeshes} shows some more results of 3D mesh
reconstruction using posterior samples obtained by our informed
sampling \MIXLMH.

\begin{figure*}[t]
\begin{center}
\centerline{\includegraphics[width=0.75\textwidth]{figures/supplementary/bodyMeshResults.pdf}}
\mycaption{Qualitative results for the body shape experiment}
  {Shown is the 3D mesh reconstruction results with first 1000 samples obtained
  using the \MIXLMH informed sampling method. (blue indicates small
  values and red indicates high values)}
\label{fig:exp3_bodyMeshes}
\end{center}
\end{figure*}

\clearpage



\section{Additional Results on the Face Problem with CMP}

Figure~\ref{fig:shading-qualitative-multiple-subjects-supp} shows inference results for reflectance maps, normal maps and lights for randomly chosen test images, and Fig.~\ref{fig:shading-qualitative-same-subject-supp} shows reflectance estimation results on multiple images of the same subject produced under different illumination conditions. CMP is able to produce estimates that are closer to the groundtruth across different subjects and illumination conditions.

\begin{figure*}[h]
  \begin{center}
  \centerline{\includegraphics[width=1.0\columnwidth]{figures/face_cmp_visual_results_supp.pdf}}
  \vspace{-1.2cm}
  \end{center}
	\mycaption{A visual comparison of inference results}{(a)~Observed images. (b)~Inferred reflectance maps. \textit{GT} is the photometric stereo groundtruth, \textit{BU} is the Biswas \etal (2009) reflectance estimate and \textit{Forest} is the consensus prediction. (c)~The variance of the inferred reflectance estimate produced by \MTD (normalized across rows).(d)~Visualization of inferred light directions. (e)~Inferred normal maps.}
	\label{fig:shading-qualitative-multiple-subjects-supp}
\end{figure*}


\begin{figure*}[h]
	\centering
	\setlength\fboxsep{0.2mm}
	\setlength\fboxrule{0pt}
	\begin{tikzpicture}

		\matrix at (0, 0) [matrix of nodes, nodes={anchor=east}, column sep=-0.05cm, row sep=-0.2cm]
		{
			\fbox{\includegraphics[width=1cm]{figures/sample_3_4_X.png}} &
			\fbox{\includegraphics[width=1cm]{figures/sample_3_4_GT.png}} &
			\fbox{\includegraphics[width=1cm]{figures/sample_3_4_BISWAS.png}}  &
			\fbox{\includegraphics[width=1cm]{figures/sample_3_4_VMP.png}}  &
			\fbox{\includegraphics[width=1cm]{figures/sample_3_4_FOREST.png}}  &
			\fbox{\includegraphics[width=1cm]{figures/sample_3_4_CMP.png}}  &
			\fbox{\includegraphics[width=1cm]{figures/sample_3_4_CMPVAR.png}}
			 \\

			\fbox{\includegraphics[width=1cm]{figures/sample_3_5_X.png}} &
			\fbox{\includegraphics[width=1cm]{figures/sample_3_5_GT.png}} &
			\fbox{\includegraphics[width=1cm]{figures/sample_3_5_BISWAS.png}}  &
			\fbox{\includegraphics[width=1cm]{figures/sample_3_5_VMP.png}}  &
			\fbox{\includegraphics[width=1cm]{figures/sample_3_5_FOREST.png}}  &
			\fbox{\includegraphics[width=1cm]{figures/sample_3_5_CMP.png}}  &
			\fbox{\includegraphics[width=1cm]{figures/sample_3_5_CMPVAR.png}}
			 \\

			\fbox{\includegraphics[width=1cm]{figures/sample_3_6_X.png}} &
			\fbox{\includegraphics[width=1cm]{figures/sample_3_6_GT.png}} &
			\fbox{\includegraphics[width=1cm]{figures/sample_3_6_BISWAS.png}}  &
			\fbox{\includegraphics[width=1cm]{figures/sample_3_6_VMP.png}}  &
			\fbox{\includegraphics[width=1cm]{figures/sample_3_6_FOREST.png}}  &
			\fbox{\includegraphics[width=1cm]{figures/sample_3_6_CMP.png}}  &
			\fbox{\includegraphics[width=1cm]{figures/sample_3_6_CMPVAR.png}}
			 \\
	     };

       \node at (-3.85, -2.0) {\small Observed};
       \node at (-2.55, -2.0) {\small `GT'};
       \node at (-1.27, -2.0) {\small BU};
       \node at (0.0, -2.0) {\small MP};
       \node at (1.27, -2.0) {\small Forest};
       \node at (2.55, -2.0) {\small \textbf{CMP}};
       \node at (3.85, -2.0) {\small Variance};

	\end{tikzpicture}
	\mycaption{Robustness to varying illumination}{Reflectance estimation on a subject images with varying illumination. Left to right: observed image, photometric stereo estimate (GT)
  which is used as a proxy for groundtruth, bottom-up estimate of \cite{Biswas2009}, VMP result, consensus forest estimate, CMP mean, and CMP variance.}
	\label{fig:shading-qualitative-same-subject-supp}
\end{figure*}

\clearpage

\section{Additional Material for Learning Sparse High Dimensional Filters}
\label{sec:appendix-bnn}

This part of supplementary material contains a more detailed overview of the permutohedral
lattice convolution in Section~\ref{sec:permconv}, more experiments in
Section~\ref{sec:addexps} and additional results with protocols for
the experiments presented in Chapter~\ref{chap:bnn} in Section~\ref{sec:addresults}.

\vspace{-0.2cm}
\subsection{General Permutohedral Convolutions}
\label{sec:permconv}

A core technical contribution of this work is the generalization of the Gaussian permutohedral lattice
convolution proposed in~\cite{adams2010fast} to the full non-separable case with the
ability to perform back-propagation. Although, conceptually, there are minor
differences between Gaussian and general parameterized filters, there are non-trivial practical
differences in terms of the algorithmic implementation. The Gauss filters belong to
the separable class and can thus be decomposed into multiple
sequential one dimensional convolutions. We are interested in the general filter
convolutions, which can not be decomposed. Thus, performing a general permutohedral
convolution at a lattice point requires the computation of the inner product with the
neighboring elements in all the directions in the high-dimensional space.

Here, we give more details of the implementation differences of separable
and non-separable filters. In the following, we will explain the scalar case first.
Recall, that the forward pass of general permutohedral convolution
involves 3 steps: \textit{splatting}, \textit{convolving} and \textit{slicing}.
We follow the same splatting and slicing strategies as in~\cite{adams2010fast}
since these operations do not depend on the filter kernel. The main difference
between our work and the existing implementation of~\cite{adams2010fast} is
the way that the convolution operation is executed. This proceeds by constructing
a \emph{blur neighbor} matrix $K$ that stores for every lattice point all
values of the lattice neighbors that are needed to compute the filter output.

\begin{figure}[t!]
  \centering
    \includegraphics[width=0.6\columnwidth]{figures/supplementary/lattice_construction}
  \mycaption{Illustration of 1D permutohedral lattice construction}
  {A $4\times 4$ $(x,y)$ grid lattice is projected onto the plane defined by the normal
  vector $(1,1)^{\top}$. This grid has $s+1=4$ and $d=2$ $(s+1)^{d}=4^2=16$ elements.
  In the projection, all points of the same color are projected onto the same points in the plane.
  The number of elements of the projected lattice is $t=(s+1)^d-s^d=4^2-3^2=7$, that is
  the $(4\times 4)$ grid minus the size of lattice that is $1$ smaller at each size, in this
  case a $(3\times 3)$ lattice (the upper right $(3\times 3)$ elements).
  }
\label{fig:latticeconstruction}
\end{figure}

The blur neighbor matrix is constructed by traversing through all the populated
lattice points and their neighboring elements.
% For efficiency, we do this matrix construction recursively with shared computations
% since $n^{th}$ neighbourhood elements are $1^{st}$ neighborhood elements of $n-1^{th}$ neighbourhood elements. \pg{do not understand}
This is done recursively to share computations. For any lattice point, the neighbors that are
$n$ hops away are the direct neighbors of the points that are $n-1$ hops away.
The size of a $d$ dimensional spatial filter with width $s+1$ is $(s+1)^{d}$ (\eg, a
$3\times 3$ filter, $s=2$ in $d=2$ has $3^2=9$ elements) and this size grows
exponentially in the number of dimensions $d$. The permutohedral lattice is constructed by
projecting a regular grid onto the plane spanned by the $d$ dimensional normal vector ${(1,\ldots,1)}^{\top}$. See
Fig.~\ref{fig:latticeconstruction} for an illustration of the 1D lattice construction.
Many corners of a grid filter are projected onto the same point, in total $t = {(s+1)}^{d} -
s^{d}$ elements remain in the permutohedral filter with $s$ neighborhood in $d-1$ dimensions.
If the lattice has $m$ populated elements, the
matrix $K$ has size $t\times m$. Note that, since the input signal is typically
sparse, only a few lattice corners are being populated in the \textit{slicing} step.
We use a hash-table to keep track of these points and traverse only through
the populated lattice points for this neighborhood matrix construction.

Once the blur neighbor matrix $K$ is constructed, we can perform the convolution
by the matrix vector multiplication
\begin{equation}
\ell' = BK,
\label{eq:conv}
\end{equation}
where $B$ is the $1 \times t$ filter kernel (whose values we will learn) and $\ell'\in\mathbb{R}^{1\times m}$
is the result of the filtering at the $m$ lattice points. In practice, we found that the
matrix $K$ is sometimes too large to fit into GPU memory and we divided the matrix $K$
into smaller pieces to compute Eq.~\ref{eq:conv} sequentially.

In the general multi-dimensional case, the signal $\ell$ is of $c$ dimensions. Then
the kernel $B$ is of size $c \times t$ and $K$ stores the $c$ dimensional vectors
accordingly. When the input and output points are different, we slice only the
input points and splat only at the output points.


\subsection{Additional Experiments}
\label{sec:addexps}
In this section, we discuss more use-cases for the learned bilateral filters, one
use-case of BNNs and two single filter applications for image and 3D mesh denoising.

\subsubsection{Recognition of subsampled MNIST}\label{sec:app_mnist}

One of the strengths of the proposed filter convolution is that it does not
require the input to lie on a regular grid. The only requirement is to define a distance
between features of the input signal.
We highlight this feature with the following experiment using the
classical MNIST ten class classification problem~\cite{lecun1998mnist}. We sample a
sparse set of $N$ points $(x,y)\in [0,1]\times [0,1]$
uniformly at random in the input image, use their interpolated values
as signal and the \emph{continuous} $(x,y)$ positions as features. This mimics
sub-sampling of a high-dimensional signal. To compare against a spatial convolution,
we interpolate the sparse set of values at the grid positions.

We take a reference implementation of LeNet~\cite{lecun1998gradient} that
is part of the Caffe project~\cite{jia2014caffe} and compare it
against the same architecture but replacing the first convolutional
layer with a bilateral convolution layer (BCL). The filter size
and numbers are adjusted to get a comparable number of parameters
($5\times 5$ for LeNet, $2$-neighborhood for BCL).

The results are shown in Table~\ref{tab:all-results}. We see that training
on the original MNIST data (column Original, LeNet vs. BNN) leads to a slight
decrease in performance of the BNN (99.03\%) compared to LeNet
(99.19\%). The BNN can be trained and evaluated on sparse
signals, and we resample the image as described above for $N=$ 100\%, 60\% and
20\% of the total number of pixels. The methods are also evaluated
on test images that are subsampled in the same way. Note that we can
train and test with different subsampling rates. We introduce an additional
bilinear interpolation layer for the LeNet architecture to train on the same
data. In essence, both models perform a spatial interpolation and thus we
expect them to yield a similar classification accuracy. Once the data is of
higher dimensions, the permutohedral convolution will be faster due to hashing
the sparse input points, as well as less memory demanding in comparison to
naive application of a spatial convolution with interpolated values.

\begin{table}[t]
  \begin{center}
    \footnotesize
    \centering
    \begin{tabular}[t]{lllll}
      \toprule
              &     & \multicolumn{3}{c}{Test Subsampling} \\
       Method  & Original & 100\% & 60\% & 20\%\\
      \midrule
       LeNet &  \textbf{0.9919} & 0.9660 & 0.9348 & \textbf{0.6434} \\
       BNN &  0.9903 & \textbf{0.9844} & \textbf{0.9534} & 0.5767 \\
      \hline
       LeNet 100\% & 0.9856 & 0.9809 & 0.9678 & \textbf{0.7386} \\
       BNN 100\% & \textbf{0.9900} & \textbf{0.9863} & \textbf{0.9699} & 0.6910 \\
      \hline
       LeNet 60\% & 0.9848 & 0.9821 & 0.9740 & 0.8151 \\
       BNN 60\% & \textbf{0.9885} & \textbf{0.9864} & \textbf{0.9771} & \textbf{0.8214}\\
      \hline
       LeNet 20\% & \textbf{0.9763} & \textbf{0.9754} & 0.9695 & 0.8928 \\
       BNN 20\% & 0.9728 & 0.9735 & \textbf{0.9701} & \textbf{0.9042}\\
      \bottomrule
    \end{tabular}
  \end{center}
\vspace{-.2cm}
\caption{Classification accuracy on MNIST. We compare the
    LeNet~\cite{lecun1998gradient} implementation that is part of
    Caffe~\cite{jia2014caffe} to the network with the first layer
    replaced by a bilateral convolution layer (BCL). Both are trained
    on the original image resolution (first two rows). Three more BNN
    and CNN models are trained with randomly subsampled images (100\%,
    60\% and 20\% of the pixels). An additional bilinear interpolation
    layer samples the input signal on a spatial grid for the CNN model.
  }
  \label{tab:all-results}
\vspace{-.5cm}
\end{table}

\subsubsection{Image Denoising}

The main application that inspired the development of the bilateral
filtering operation is image denoising~\cite{aurich1995non}, there
using a single Gaussian kernel. Our development allows to learn this
kernel function from data and we explore how to improve using a \emph{single}
but more general bilateral filter.

We use the Berkeley segmentation dataset
(BSDS500)~\cite{arbelaezi2011bsds500} as a test bed. The color
images in the dataset are converted to gray-scale,
and corrupted with Gaussian noise with a standard deviation of
$\frac {25} {255}$.

We compare the performance of four different filter models on a
denoising task.
The first baseline model (`Spatial' in Table \ref{tab:denoising}, $25$
weights) uses a single spatial filter with a kernel size of
$5$ and predicts the scalar gray-scale value at the center pixel. The next model
(`Gauss Bilateral') applies a bilateral \emph{Gaussian}
filter to the noisy input, using position and intensity features $\f=(x,y,v)^\top$.
The third setup (`Learned Bilateral', $65$ weights)
takes a Gauss kernel as initialization and
fits all filter weights on the train set to minimize the
mean squared error with respect to the clean images.
We run a combination
of spatial and permutohedral convolutions on spatial and bilateral
features (`Spatial + Bilateral (Learned)') to check for a complementary
performance of the two convolutions.

\label{sec:exp:denoising}
\begin{table}[!h]
\begin{center}
  \footnotesize
  \begin{tabular}[t]{lr}
    \toprule
    Method & PSNR \\
    \midrule
    Noisy Input & $20.17$ \\
    Spatial & $26.27$ \\
    Gauss Bilateral & $26.51$ \\
    Learned Bilateral & $26.58$ \\
    Spatial + Bilateral (Learned) & \textbf{$26.65$} \\
    \bottomrule
  \end{tabular}
\end{center}
\vspace{-0.5em}
\caption{PSNR results of a denoising task using the BSDS500
  dataset~\cite{arbelaezi2011bsds500}}
\vspace{-0.5em}
\label{tab:denoising}
\end{table}
\vspace{-0.2em}

The PSNR scores evaluated on full images of the test set are
shown in Table \ref{tab:denoising}. We find that an untrained bilateral
filter already performs better than a trained spatial convolution
($26.27$ to $26.51$). A learned convolution further improve the
performance slightly. We chose this simple one-kernel setup to
validate an advantage of the generalized bilateral filter. A competitive
denoising system would employ RGB color information and also
needs to be properly adjusted in network size. Multi-layer perceptrons
have obtained state-of-the-art denoising results~\cite{burger12cvpr}
and the permutohedral lattice layer can readily be used in such an
architecture, which is intended future work.

\subsection{Additional results}
\label{sec:addresults}

This section contains more qualitative results for the experiments presented in Chapter~\ref{chap:bnn}.

\begin{figure*}[th!]
  \centering
    \includegraphics[width=\columnwidth,trim={5cm 2.5cm 5cm 4.5cm},clip]{figures/supplementary/lattice_viz.pdf}
    \vspace{-0.7cm}
  \mycaption{Visualization of the Permutohedral Lattice}
  {Sample lattice visualizations for different feature spaces. All pixels falling in the same simplex cell are shown with
  the same color. $(x,y)$ features correspond to image pixel positions, and $(r,g,b) \in [0,255]$ correspond
  to the red, green and blue color values.}
\label{fig:latticeviz}
\end{figure*}

\subsubsection{Lattice Visualization}

Figure~\ref{fig:latticeviz} shows sample lattice visualizations for different feature spaces.

\newcolumntype{L}[1]{>{\raggedright\let\newline\\\arraybackslash\hspace{0pt}}b{#1}}
\newcolumntype{C}[1]{>{\centering\let\newline\\\arraybackslash\hspace{0pt}}b{#1}}
\newcolumntype{R}[1]{>{\raggedleft\let\newline\\\arraybackslash\hspace{0pt}}b{#1}}

\subsubsection{Color Upsampling}\label{sec:color_upsampling}
\label{sec:col_upsample_extra}

Some images of the upsampling for the Pascal
VOC12 dataset are shown in Fig.~\ref{fig:Colour_upsample_visuals}. It is
especially the low level image details that are better preserved with
a learned bilateral filter compared to the Gaussian case.

\begin{figure*}[t!]
  \centering
    \subfigure{%
   \raisebox{2.0em}{
    \includegraphics[width=.06\columnwidth]{figures/supplementary/2007_004969.jpg}
   }
  }
  \subfigure{%
    \includegraphics[width=.17\columnwidth]{figures/supplementary/2007_004969_gray.pdf}
  }
  \subfigure{%
    \includegraphics[width=.17\columnwidth]{figures/supplementary/2007_004969_gt.pdf}
  }
  \subfigure{%
    \includegraphics[width=.17\columnwidth]{figures/supplementary/2007_004969_bicubic.pdf}
  }
  \subfigure{%
    \includegraphics[width=.17\columnwidth]{figures/supplementary/2007_004969_gauss.pdf}
  }
  \subfigure{%
    \includegraphics[width=.17\columnwidth]{figures/supplementary/2007_004969_learnt.pdf}
  }\\
    \subfigure{%
   \raisebox{2.0em}{
    \includegraphics[width=.06\columnwidth]{figures/supplementary/2007_003106.jpg}
   }
  }
  \subfigure{%
    \includegraphics[width=.17\columnwidth]{figures/supplementary/2007_003106_gray.pdf}
  }
  \subfigure{%
    \includegraphics[width=.17\columnwidth]{figures/supplementary/2007_003106_gt.pdf}
  }
  \subfigure{%
    \includegraphics[width=.17\columnwidth]{figures/supplementary/2007_003106_bicubic.pdf}
  }
  \subfigure{%
    \includegraphics[width=.17\columnwidth]{figures/supplementary/2007_003106_gauss.pdf}
  }
  \subfigure{%
    \includegraphics[width=.17\columnwidth]{figures/supplementary/2007_003106_learnt.pdf}
  }\\
  \setcounter{subfigure}{0}
  \small{
  \subfigure[Inp.]{%
  \raisebox{2.0em}{
    \includegraphics[width=.06\columnwidth]{figures/supplementary/2007_006837.jpg}
   }
  }
  \subfigure[Guidance]{%
    \includegraphics[width=.17\columnwidth]{figures/supplementary/2007_006837_gray.pdf}
  }
   \subfigure[GT]{%
    \includegraphics[width=.17\columnwidth]{figures/supplementary/2007_006837_gt.pdf}
  }
  \subfigure[Bicubic]{%
    \includegraphics[width=.17\columnwidth]{figures/supplementary/2007_006837_bicubic.pdf}
  }
  \subfigure[Gauss-BF]{%
    \includegraphics[width=.17\columnwidth]{figures/supplementary/2007_006837_gauss.pdf}
  }
  \subfigure[Learned-BF]{%
    \includegraphics[width=.17\columnwidth]{figures/supplementary/2007_006837_learnt.pdf}
  }
  }
  \vspace{-0.5cm}
  \mycaption{Color Upsampling}{Color $8\times$ upsampling results
  using different methods, from left to right, (a)~Low-resolution input color image (Inp.),
  (b)~Gray scale guidance image, (c)~Ground-truth color image; Upsampled color images with
  (d)~Bicubic interpolation, (e) Gauss bilateral upsampling and, (f)~Learned bilateral
  updampgling (best viewed on screen).}

\label{fig:Colour_upsample_visuals}
\end{figure*}

\subsubsection{Depth Upsampling}
\label{sec:depth_upsample_extra}

Figure~\ref{fig:depth_upsample_visuals} presents some more qualitative results comparing bicubic interpolation, Gauss
bilateral and learned bilateral upsampling on NYU depth dataset image~\cite{silberman2012indoor}.

\subsubsection{Character Recognition}\label{sec:app_character}

 Figure~\ref{fig:nnrecognition} shows the schematic of different layers
 of the network architecture for LeNet-7~\cite{lecun1998mnist}
 and DeepCNet(5, 50)~\cite{ciresan2012multi,graham2014spatially}. For the BNN variants, the first layer filters are replaced
 with learned bilateral filters and are learned end-to-end.

\subsubsection{Semantic Segmentation}\label{sec:app_semantic_segmentation}
\label{sec:semantic_bnn_extra}

Some more visual results for semantic segmentation are shown in Figure~\ref{fig:semantic_visuals}.
These include the underlying DeepLab CNN\cite{chen2014semantic} result (DeepLab),
the 2 step mean-field result with Gaussian edge potentials (+2stepMF-GaussCRF)
and also corresponding results with learned edge potentials (+2stepMF-LearnedCRF).
In general, we observe that mean-field in learned CRF leads to slightly dilated
classification regions in comparison to using Gaussian CRF thereby filling-in the
false negative pixels and also correcting some mis-classified regions.

\begin{figure*}[t!]
  \centering
    \subfigure{%
   \raisebox{2.0em}{
    \includegraphics[width=.06\columnwidth]{figures/supplementary/2bicubic}
   }
  }
  \subfigure{%
    \includegraphics[width=.17\columnwidth]{figures/supplementary/2given_image}
  }
  \subfigure{%
    \includegraphics[width=.17\columnwidth]{figures/supplementary/2ground_truth}
  }
  \subfigure{%
    \includegraphics[width=.17\columnwidth]{figures/supplementary/2bicubic}
  }
  \subfigure{%
    \includegraphics[width=.17\columnwidth]{figures/supplementary/2gauss}
  }
  \subfigure{%
    \includegraphics[width=.17\columnwidth]{figures/supplementary/2learnt}
  }\\
    \subfigure{%
   \raisebox{2.0em}{
    \includegraphics[width=.06\columnwidth]{figures/supplementary/32bicubic}
   }
  }
  \subfigure{%
    \includegraphics[width=.17\columnwidth]{figures/supplementary/32given_image}
  }
  \subfigure{%
    \includegraphics[width=.17\columnwidth]{figures/supplementary/32ground_truth}
  }
  \subfigure{%
    \includegraphics[width=.17\columnwidth]{figures/supplementary/32bicubic}
  }
  \subfigure{%
    \includegraphics[width=.17\columnwidth]{figures/supplementary/32gauss}
  }
  \subfigure{%
    \includegraphics[width=.17\columnwidth]{figures/supplementary/32learnt}
  }\\
  \setcounter{subfigure}{0}
  \small{
  \subfigure[Inp.]{%
  \raisebox{2.0em}{
    \includegraphics[width=.06\columnwidth]{figures/supplementary/41bicubic}
   }
  }
  \subfigure[Guidance]{%
    \includegraphics[width=.17\columnwidth]{figures/supplementary/41given_image}
  }
   \subfigure[GT]{%
    \includegraphics[width=.17\columnwidth]{figures/supplementary/41ground_truth}
  }
  \subfigure[Bicubic]{%
    \includegraphics[width=.17\columnwidth]{figures/supplementary/41bicubic}
  }
  \subfigure[Gauss-BF]{%
    \includegraphics[width=.17\columnwidth]{figures/supplementary/41gauss}
  }
  \subfigure[Learned-BF]{%
    \includegraphics[width=.17\columnwidth]{figures/supplementary/41learnt}
  }
  }
  \mycaption{Depth Upsampling}{Depth $8\times$ upsampling results
  using different upsampling strategies, from left to right,
  (a)~Low-resolution input depth image (Inp.),
  (b)~High-resolution guidance image, (c)~Ground-truth depth; Upsampled depth images with
  (d)~Bicubic interpolation, (e) Gauss bilateral upsampling and, (f)~Learned bilateral
  updampgling (best viewed on screen).}

\label{fig:depth_upsample_visuals}
\end{figure*}

\subsubsection{Material Segmentation}\label{sec:app_material_segmentation}
\label{sec:material_bnn_extra}

In Fig.~\ref{fig:material_visuals-app2}, we present visual results comparing 2 step
mean-field inference with Gaussian and learned pairwise CRF potentials. In
general, we observe that the pixels belonging to dominant classes in the
training data are being more accurately classified with learned CRF. This leads to
a significant improvements in overall pixel accuracy. This also results
in a slight decrease of the accuracy from less frequent class pixels thereby
slightly reducing the average class accuracy with learning. We attribute this
to the type of annotation that is available for this dataset, which is not
for the entire image but for some segments in the image. We have very few
images of the infrequent classes to combat this behaviour during training.

\subsubsection{Experiment Protocols}
\label{sec:protocols}

Table~\ref{tbl:parameters} shows experiment protocols of different experiments.

 \begin{figure*}[t!]
  \centering
  \subfigure[LeNet-7]{
    \includegraphics[width=0.7\columnwidth]{figures/supplementary/lenet_cnn_network}
    }\\
    \subfigure[DeepCNet]{
    \includegraphics[width=\columnwidth]{figures/supplementary/deepcnet_cnn_network}
    }
  \mycaption{CNNs for Character Recognition}
  {Schematic of (top) LeNet-7~\cite{lecun1998mnist} and (bottom) DeepCNet(5,50)~\cite{ciresan2012multi,graham2014spatially} architectures used in Assamese
  character recognition experiments.}
\label{fig:nnrecognition}
\end{figure*}

\definecolor{voc_1}{RGB}{0, 0, 0}
\definecolor{voc_2}{RGB}{128, 0, 0}
\definecolor{voc_3}{RGB}{0, 128, 0}
\definecolor{voc_4}{RGB}{128, 128, 0}
\definecolor{voc_5}{RGB}{0, 0, 128}
\definecolor{voc_6}{RGB}{128, 0, 128}
\definecolor{voc_7}{RGB}{0, 128, 128}
\definecolor{voc_8}{RGB}{128, 128, 128}
\definecolor{voc_9}{RGB}{64, 0, 0}
\definecolor{voc_10}{RGB}{192, 0, 0}
\definecolor{voc_11}{RGB}{64, 128, 0}
\definecolor{voc_12}{RGB}{192, 128, 0}
\definecolor{voc_13}{RGB}{64, 0, 128}
\definecolor{voc_14}{RGB}{192, 0, 128}
\definecolor{voc_15}{RGB}{64, 128, 128}
\definecolor{voc_16}{RGB}{192, 128, 128}
\definecolor{voc_17}{RGB}{0, 64, 0}
\definecolor{voc_18}{RGB}{128, 64, 0}
\definecolor{voc_19}{RGB}{0, 192, 0}
\definecolor{voc_20}{RGB}{128, 192, 0}
\definecolor{voc_21}{RGB}{0, 64, 128}
\definecolor{voc_22}{RGB}{128, 64, 128}

\begin{figure*}[t]
  \centering
  \small{
  \fcolorbox{white}{voc_1}{\rule{0pt}{6pt}\rule{6pt}{0pt}} Background~~
  \fcolorbox{white}{voc_2}{\rule{0pt}{6pt}\rule{6pt}{0pt}} Aeroplane~~
  \fcolorbox{white}{voc_3}{\rule{0pt}{6pt}\rule{6pt}{0pt}} Bicycle~~
  \fcolorbox{white}{voc_4}{\rule{0pt}{6pt}\rule{6pt}{0pt}} Bird~~
  \fcolorbox{white}{voc_5}{\rule{0pt}{6pt}\rule{6pt}{0pt}} Boat~~
  \fcolorbox{white}{voc_6}{\rule{0pt}{6pt}\rule{6pt}{0pt}} Bottle~~
  \fcolorbox{white}{voc_7}{\rule{0pt}{6pt}\rule{6pt}{0pt}} Bus~~
  \fcolorbox{white}{voc_8}{\rule{0pt}{6pt}\rule{6pt}{0pt}} Car~~ \\
  \fcolorbox{white}{voc_9}{\rule{0pt}{6pt}\rule{6pt}{0pt}} Cat~~
  \fcolorbox{white}{voc_10}{\rule{0pt}{6pt}\rule{6pt}{0pt}} Chair~~
  \fcolorbox{white}{voc_11}{\rule{0pt}{6pt}\rule{6pt}{0pt}} Cow~~
  \fcolorbox{white}{voc_12}{\rule{0pt}{6pt}\rule{6pt}{0pt}} Dining Table~~
  \fcolorbox{white}{voc_13}{\rule{0pt}{6pt}\rule{6pt}{0pt}} Dog~~
  \fcolorbox{white}{voc_14}{\rule{0pt}{6pt}\rule{6pt}{0pt}} Horse~~
  \fcolorbox{white}{voc_15}{\rule{0pt}{6pt}\rule{6pt}{0pt}} Motorbike~~
  \fcolorbox{white}{voc_16}{\rule{0pt}{6pt}\rule{6pt}{0pt}} Person~~ \\
  \fcolorbox{white}{voc_17}{\rule{0pt}{6pt}\rule{6pt}{0pt}} Potted Plant~~
  \fcolorbox{white}{voc_18}{\rule{0pt}{6pt}\rule{6pt}{0pt}} Sheep~~
  \fcolorbox{white}{voc_19}{\rule{0pt}{6pt}\rule{6pt}{0pt}} Sofa~~
  \fcolorbox{white}{voc_20}{\rule{0pt}{6pt}\rule{6pt}{0pt}} Train~~
  \fcolorbox{white}{voc_21}{\rule{0pt}{6pt}\rule{6pt}{0pt}} TV monitor~~ \\
  }
  \subfigure{%
    \includegraphics[width=.18\columnwidth]{figures/supplementary/2007_001423_given.jpg}
  }
  \subfigure{%
    \includegraphics[width=.18\columnwidth]{figures/supplementary/2007_001423_gt.png}
  }
  \subfigure{%
    \includegraphics[width=.18\columnwidth]{figures/supplementary/2007_001423_cnn.png}
  }
  \subfigure{%
    \includegraphics[width=.18\columnwidth]{figures/supplementary/2007_001423_gauss.png}
  }
  \subfigure{%
    \includegraphics[width=.18\columnwidth]{figures/supplementary/2007_001423_learnt.png}
  }\\
  \subfigure{%
    \includegraphics[width=.18\columnwidth]{figures/supplementary/2007_001430_given.jpg}
  }
  \subfigure{%
    \includegraphics[width=.18\columnwidth]{figures/supplementary/2007_001430_gt.png}
  }
  \subfigure{%
    \includegraphics[width=.18\columnwidth]{figures/supplementary/2007_001430_cnn.png}
  }
  \subfigure{%
    \includegraphics[width=.18\columnwidth]{figures/supplementary/2007_001430_gauss.png}
  }
  \subfigure{%
    \includegraphics[width=.18\columnwidth]{figures/supplementary/2007_001430_learnt.png}
  }\\
    \subfigure{%
    \includegraphics[width=.18\columnwidth]{figures/supplementary/2007_007996_given.jpg}
  }
  \subfigure{%
    \includegraphics[width=.18\columnwidth]{figures/supplementary/2007_007996_gt.png}
  }
  \subfigure{%
    \includegraphics[width=.18\columnwidth]{figures/supplementary/2007_007996_cnn.png}
  }
  \subfigure{%
    \includegraphics[width=.18\columnwidth]{figures/supplementary/2007_007996_gauss.png}
  }
  \subfigure{%
    \includegraphics[width=.18\columnwidth]{figures/supplementary/2007_007996_learnt.png}
  }\\
   \subfigure{%
    \includegraphics[width=.18\columnwidth]{figures/supplementary/2010_002682_given.jpg}
  }
  \subfigure{%
    \includegraphics[width=.18\columnwidth]{figures/supplementary/2010_002682_gt.png}
  }
  \subfigure{%
    \includegraphics[width=.18\columnwidth]{figures/supplementary/2010_002682_cnn.png}
  }
  \subfigure{%
    \includegraphics[width=.18\columnwidth]{figures/supplementary/2010_002682_gauss.png}
  }
  \subfigure{%
    \includegraphics[width=.18\columnwidth]{figures/supplementary/2010_002682_learnt.png}
  }\\
     \subfigure{%
    \includegraphics[width=.18\columnwidth]{figures/supplementary/2010_004789_given.jpg}
  }
  \subfigure{%
    \includegraphics[width=.18\columnwidth]{figures/supplementary/2010_004789_gt.png}
  }
  \subfigure{%
    \includegraphics[width=.18\columnwidth]{figures/supplementary/2010_004789_cnn.png}
  }
  \subfigure{%
    \includegraphics[width=.18\columnwidth]{figures/supplementary/2010_004789_gauss.png}
  }
  \subfigure{%
    \includegraphics[width=.18\columnwidth]{figures/supplementary/2010_004789_learnt.png}
  }\\
       \subfigure{%
    \includegraphics[width=.18\columnwidth]{figures/supplementary/2007_001311_given.jpg}
  }
  \subfigure{%
    \includegraphics[width=.18\columnwidth]{figures/supplementary/2007_001311_gt.png}
  }
  \subfigure{%
    \includegraphics[width=.18\columnwidth]{figures/supplementary/2007_001311_cnn.png}
  }
  \subfigure{%
    \includegraphics[width=.18\columnwidth]{figures/supplementary/2007_001311_gauss.png}
  }
  \subfigure{%
    \includegraphics[width=.18\columnwidth]{figures/supplementary/2007_001311_learnt.png}
  }\\
  \setcounter{subfigure}{0}
  \subfigure[Input]{%
    \includegraphics[width=.18\columnwidth]{figures/supplementary/2010_003531_given.jpg}
  }
  \subfigure[Ground Truth]{%
    \includegraphics[width=.18\columnwidth]{figures/supplementary/2010_003531_gt.png}
  }
  \subfigure[DeepLab]{%
    \includegraphics[width=.18\columnwidth]{figures/supplementary/2010_003531_cnn.png}
  }
  \subfigure[+GaussCRF]{%
    \includegraphics[width=.18\columnwidth]{figures/supplementary/2010_003531_gauss.png}
  }
  \subfigure[+LearnedCRF]{%
    \includegraphics[width=.18\columnwidth]{figures/supplementary/2010_003531_learnt.png}
  }
  \vspace{-0.3cm}
  \mycaption{Semantic Segmentation}{Example results of semantic segmentation.
  (c)~depicts the unary results before application of MF, (d)~after two steps of MF with Gaussian edge CRF potentials, (e)~after
  two steps of MF with learned edge CRF potentials.}
    \label{fig:semantic_visuals}
\end{figure*}


\definecolor{minc_1}{HTML}{771111}
\definecolor{minc_2}{HTML}{CAC690}
\definecolor{minc_3}{HTML}{EEEEEE}
\definecolor{minc_4}{HTML}{7C8FA6}
\definecolor{minc_5}{HTML}{597D31}
\definecolor{minc_6}{HTML}{104410}
\definecolor{minc_7}{HTML}{BB819C}
\definecolor{minc_8}{HTML}{D0CE48}
\definecolor{minc_9}{HTML}{622745}
\definecolor{minc_10}{HTML}{666666}
\definecolor{minc_11}{HTML}{D54A31}
\definecolor{minc_12}{HTML}{101044}
\definecolor{minc_13}{HTML}{444126}
\definecolor{minc_14}{HTML}{75D646}
\definecolor{minc_15}{HTML}{DD4348}
\definecolor{minc_16}{HTML}{5C8577}
\definecolor{minc_17}{HTML}{C78472}
\definecolor{minc_18}{HTML}{75D6D0}
\definecolor{minc_19}{HTML}{5B4586}
\definecolor{minc_20}{HTML}{C04393}
\definecolor{minc_21}{HTML}{D69948}
\definecolor{minc_22}{HTML}{7370D8}
\definecolor{minc_23}{HTML}{7A3622}
\definecolor{minc_24}{HTML}{000000}

\begin{figure*}[t]
  \centering
  \small{
  \fcolorbox{white}{minc_1}{\rule{0pt}{6pt}\rule{6pt}{0pt}} Brick~~
  \fcolorbox{white}{minc_2}{\rule{0pt}{6pt}\rule{6pt}{0pt}} Carpet~~
  \fcolorbox{white}{minc_3}{\rule{0pt}{6pt}\rule{6pt}{0pt}} Ceramic~~
  \fcolorbox{white}{minc_4}{\rule{0pt}{6pt}\rule{6pt}{0pt}} Fabric~~
  \fcolorbox{white}{minc_5}{\rule{0pt}{6pt}\rule{6pt}{0pt}} Foliage~~
  \fcolorbox{white}{minc_6}{\rule{0pt}{6pt}\rule{6pt}{0pt}} Food~~
  \fcolorbox{white}{minc_7}{\rule{0pt}{6pt}\rule{6pt}{0pt}} Glass~~
  \fcolorbox{white}{minc_8}{\rule{0pt}{6pt}\rule{6pt}{0pt}} Hair~~ \\
  \fcolorbox{white}{minc_9}{\rule{0pt}{6pt}\rule{6pt}{0pt}} Leather~~
  \fcolorbox{white}{minc_10}{\rule{0pt}{6pt}\rule{6pt}{0pt}} Metal~~
  \fcolorbox{white}{minc_11}{\rule{0pt}{6pt}\rule{6pt}{0pt}} Mirror~~
  \fcolorbox{white}{minc_12}{\rule{0pt}{6pt}\rule{6pt}{0pt}} Other~~
  \fcolorbox{white}{minc_13}{\rule{0pt}{6pt}\rule{6pt}{0pt}} Painted~~
  \fcolorbox{white}{minc_14}{\rule{0pt}{6pt}\rule{6pt}{0pt}} Paper~~
  \fcolorbox{white}{minc_15}{\rule{0pt}{6pt}\rule{6pt}{0pt}} Plastic~~\\
  \fcolorbox{white}{minc_16}{\rule{0pt}{6pt}\rule{6pt}{0pt}} Polished Stone~~
  \fcolorbox{white}{minc_17}{\rule{0pt}{6pt}\rule{6pt}{0pt}} Skin~~
  \fcolorbox{white}{minc_18}{\rule{0pt}{6pt}\rule{6pt}{0pt}} Sky~~
  \fcolorbox{white}{minc_19}{\rule{0pt}{6pt}\rule{6pt}{0pt}} Stone~~
  \fcolorbox{white}{minc_20}{\rule{0pt}{6pt}\rule{6pt}{0pt}} Tile~~
  \fcolorbox{white}{minc_21}{\rule{0pt}{6pt}\rule{6pt}{0pt}} Wallpaper~~
  \fcolorbox{white}{minc_22}{\rule{0pt}{6pt}\rule{6pt}{0pt}} Water~~
  \fcolorbox{white}{minc_23}{\rule{0pt}{6pt}\rule{6pt}{0pt}} Wood~~ \\
  }
  \subfigure{%
    \includegraphics[width=.18\columnwidth]{figures/supplementary/000010868_given.jpg}
  }
  \subfigure{%
    \includegraphics[width=.18\columnwidth]{figures/supplementary/000010868_gt.png}
  }
  \subfigure{%
    \includegraphics[width=.18\columnwidth]{figures/supplementary/000010868_cnn.png}
  }
  \subfigure{%
    \includegraphics[width=.18\columnwidth]{figures/supplementary/000010868_gauss.png}
  }
  \subfigure{%
    \includegraphics[width=.18\columnwidth]{figures/supplementary/000010868_learnt.png}
  }\\[-2ex]
  \subfigure{%
    \includegraphics[width=.18\columnwidth]{figures/supplementary/000006011_given.jpg}
  }
  \subfigure{%
    \includegraphics[width=.18\columnwidth]{figures/supplementary/000006011_gt.png}
  }
  \subfigure{%
    \includegraphics[width=.18\columnwidth]{figures/supplementary/000006011_cnn.png}
  }
  \subfigure{%
    \includegraphics[width=.18\columnwidth]{figures/supplementary/000006011_gauss.png}
  }
  \subfigure{%
    \includegraphics[width=.18\columnwidth]{figures/supplementary/000006011_learnt.png}
  }\\[-2ex]
    \subfigure{%
    \includegraphics[width=.18\columnwidth]{figures/supplementary/000008553_given.jpg}
  }
  \subfigure{%
    \includegraphics[width=.18\columnwidth]{figures/supplementary/000008553_gt.png}
  }
  \subfigure{%
    \includegraphics[width=.18\columnwidth]{figures/supplementary/000008553_cnn.png}
  }
  \subfigure{%
    \includegraphics[width=.18\columnwidth]{figures/supplementary/000008553_gauss.png}
  }
  \subfigure{%
    \includegraphics[width=.18\columnwidth]{figures/supplementary/000008553_learnt.png}
  }\\[-2ex]
   \subfigure{%
    \includegraphics[width=.18\columnwidth]{figures/supplementary/000009188_given.jpg}
  }
  \subfigure{%
    \includegraphics[width=.18\columnwidth]{figures/supplementary/000009188_gt.png}
  }
  \subfigure{%
    \includegraphics[width=.18\columnwidth]{figures/supplementary/000009188_cnn.png}
  }
  \subfigure{%
    \includegraphics[width=.18\columnwidth]{figures/supplementary/000009188_gauss.png}
  }
  \subfigure{%
    \includegraphics[width=.18\columnwidth]{figures/supplementary/000009188_learnt.png}
  }\\[-2ex]
  \setcounter{subfigure}{0}
  \subfigure[Input]{%
    \includegraphics[width=.18\columnwidth]{figures/supplementary/000023570_given.jpg}
  }
  \subfigure[Ground Truth]{%
    \includegraphics[width=.18\columnwidth]{figures/supplementary/000023570_gt.png}
  }
  \subfigure[DeepLab]{%
    \includegraphics[width=.18\columnwidth]{figures/supplementary/000023570_cnn.png}
  }
  \subfigure[+GaussCRF]{%
    \includegraphics[width=.18\columnwidth]{figures/supplementary/000023570_gauss.png}
  }
  \subfigure[+LearnedCRF]{%
    \includegraphics[width=.18\columnwidth]{figures/supplementary/000023570_learnt.png}
  }
  \mycaption{Material Segmentation}{Example results of material segmentation.
  (c)~depicts the unary results before application of MF, (d)~after two steps of MF with Gaussian edge CRF potentials, (e)~after two steps of MF with learned edge CRF potentials.}
    \label{fig:material_visuals-app2}
\end{figure*}


\begin{table*}[h]
\tiny
  \centering
    \begin{tabular}{L{2.3cm} L{2.25cm} C{1.5cm} C{0.7cm} C{0.6cm} C{0.7cm} C{0.7cm} C{0.7cm} C{1.6cm} C{0.6cm} C{0.6cm} C{0.6cm}}
      \toprule
& & & & & \multicolumn{3}{c}{\textbf{Data Statistics}} & \multicolumn{4}{c}{\textbf{Training Protocol}} \\

\textbf{Experiment} & \textbf{Feature Types} & \textbf{Feature Scales} & \textbf{Filter Size} & \textbf{Filter Nbr.} & \textbf{Train}  & \textbf{Val.} & \textbf{Test} & \textbf{Loss Type} & \textbf{LR} & \textbf{Batch} & \textbf{Epochs} \\
      \midrule
      \multicolumn{2}{c}{\textbf{Single Bilateral Filter Applications}} & & & & & & & & & \\
      \textbf{2$\times$ Color Upsampling} & Position$_{1}$, Intensity (3D) & 0.13, 0.17 & 65 & 2 & 10581 & 1449 & 1456 & MSE & 1e-06 & 200 & 94.5\\
      \textbf{4$\times$ Color Upsampling} & Position$_{1}$, Intensity (3D) & 0.06, 0.17 & 65 & 2 & 10581 & 1449 & 1456 & MSE & 1e-06 & 200 & 94.5\\
      \textbf{8$\times$ Color Upsampling} & Position$_{1}$, Intensity (3D) & 0.03, 0.17 & 65 & 2 & 10581 & 1449 & 1456 & MSE & 1e-06 & 200 & 94.5\\
      \textbf{16$\times$ Color Upsampling} & Position$_{1}$, Intensity (3D) & 0.02, 0.17 & 65 & 2 & 10581 & 1449 & 1456 & MSE & 1e-06 & 200 & 94.5\\
      \textbf{Depth Upsampling} & Position$_{1}$, Color (5D) & 0.05, 0.02 & 665 & 2 & 795 & 100 & 654 & MSE & 1e-07 & 50 & 251.6\\
      \textbf{Mesh Denoising} & Isomap (4D) & 46.00 & 63 & 2 & 1000 & 200 & 500 & MSE & 100 & 10 & 100.0 \\
      \midrule
      \multicolumn{2}{c}{\textbf{DenseCRF Applications}} & & & & & & & & &\\
      \multicolumn{2}{l}{\textbf{Semantic Segmentation}} & & & & & & & & &\\
      \textbf{- 1step MF} & Position$_{1}$, Color (5D); Position$_{1}$ (2D) & 0.01, 0.34; 0.34  & 665; 19  & 2; 2 & 10581 & 1449 & 1456 & Logistic & 0.1 & 5 & 1.4 \\
      \textbf{- 2step MF} & Position$_{1}$, Color (5D); Position$_{1}$ (2D) & 0.01, 0.34; 0.34 & 665; 19 & 2; 2 & 10581 & 1449 & 1456 & Logistic & 0.1 & 5 & 1.4 \\
      \textbf{- \textit{loose} 2step MF} & Position$_{1}$, Color (5D); Position$_{1}$ (2D) & 0.01, 0.34; 0.34 & 665; 19 & 2; 2 &10581 & 1449 & 1456 & Logistic & 0.1 & 5 & +1.9  \\ \\
      \multicolumn{2}{l}{\textbf{Material Segmentation}} & & & & & & & & &\\
      \textbf{- 1step MF} & Position$_{2}$, Lab-Color (5D) & 5.00, 0.05, 0.30  & 665 & 2 & 928 & 150 & 1798 & Weighted Logistic & 1e-04 & 24 & 2.6 \\
      \textbf{- 2step MF} & Position$_{2}$, Lab-Color (5D) & 5.00, 0.05, 0.30 & 665 & 2 & 928 & 150 & 1798 & Weighted Logistic & 1e-04 & 12 & +0.7 \\
      \textbf{- \textit{loose} 2step MF} & Position$_{2}$, Lab-Color (5D) & 5.00, 0.05, 0.30 & 665 & 2 & 928 & 150 & 1798 & Weighted Logistic & 1e-04 & 12 & +0.2\\
      \midrule
      \multicolumn{2}{c}{\textbf{Neural Network Applications}} & & & & & & & & &\\
      \textbf{Tiles: CNN-9$\times$9} & - & - & 81 & 4 & 10000 & 1000 & 1000 & Logistic & 0.01 & 100 & 500.0 \\
      \textbf{Tiles: CNN-13$\times$13} & - & - & 169 & 6 & 10000 & 1000 & 1000 & Logistic & 0.01 & 100 & 500.0 \\
      \textbf{Tiles: CNN-17$\times$17} & - & - & 289 & 8 & 10000 & 1000 & 1000 & Logistic & 0.01 & 100 & 500.0 \\
      \textbf{Tiles: CNN-21$\times$21} & - & - & 441 & 10 & 10000 & 1000 & 1000 & Logistic & 0.01 & 100 & 500.0 \\
      \textbf{Tiles: BNN} & Position$_{1}$, Color (5D) & 0.05, 0.04 & 63 & 1 & 10000 & 1000 & 1000 & Logistic & 0.01 & 100 & 30.0 \\
      \textbf{LeNet} & - & - & 25 & 2 & 5490 & 1098 & 1647 & Logistic & 0.1 & 100 & 182.2 \\
      \textbf{Crop-LeNet} & - & - & 25 & 2 & 5490 & 1098 & 1647 & Logistic & 0.1 & 100 & 182.2 \\
      \textbf{BNN-LeNet} & Position$_{2}$ (2D) & 20.00 & 7 & 1 & 5490 & 1098 & 1647 & Logistic & 0.1 & 100 & 182.2 \\
      \textbf{DeepCNet} & - & - & 9 & 1 & 5490 & 1098 & 1647 & Logistic & 0.1 & 100 & 182.2 \\
      \textbf{Crop-DeepCNet} & - & - & 9 & 1 & 5490 & 1098 & 1647 & Logistic & 0.1 & 100 & 182.2 \\
      \textbf{BNN-DeepCNet} & Position$_{2}$ (2D) & 40.00  & 7 & 1 & 5490 & 1098 & 1647 & Logistic & 0.1 & 100 & 182.2 \\
      \bottomrule
      \\
    \end{tabular}
    \mycaption{Experiment Protocols} {Experiment protocols for the different experiments presented in this work. \textbf{Feature Types}:
    Feature spaces used for the bilateral convolutions. Position$_1$ corresponds to un-normalized pixel positions whereas Position$_2$ corresponds
    to pixel positions normalized to $[0,1]$ with respect to the given image. \textbf{Feature Scales}: Cross-validated scales for the features used.
     \textbf{Filter Size}: Number of elements in the filter that is being learned. \textbf{Filter Nbr.}: Half-width of the filter. \textbf{Train},
     \textbf{Val.} and \textbf{Test} corresponds to the number of train, validation and test images used in the experiment. \textbf{Loss Type}: Type
     of loss used for back-propagation. ``MSE'' corresponds to Euclidean mean squared error loss and ``Logistic'' corresponds to multinomial logistic
     loss. ``Weighted Logistic'' is the class-weighted multinomial logistic loss. We weighted the loss with inverse class probability for material
     segmentation task due to the small availability of training data with class imbalance. \textbf{LR}: Fixed learning rate used in stochastic gradient
     descent. \textbf{Batch}: Number of images used in one parameter update step. \textbf{Epochs}: Number of training epochs. In all the experiments,
     we used fixed momentum of 0.9 and weight decay of 0.0005 for stochastic gradient descent. ```Color Upsampling'' experiments in this Table corresponds
     to those performed on Pascal VOC12 dataset images. For all experiments using Pascal VOC12 images, we use extended
     training segmentation dataset available from~\cite{hariharan2011moredata}, and used standard validation and test splits
     from the main dataset~\cite{voc2012segmentation}.}
  \label{tbl:parameters}
\end{table*}

\clearpage

\section{Parameters and Additional Results for Video Propagation Networks}

In this Section, we present experiment protocols and additional qualitative results for experiments
on video object segmentation, semantic video segmentation and video color
propagation. Table~\ref{tbl:parameters_supp} shows the feature scales and other parameters used in different experiments.
Figures~\ref{fig:video_seg_pos_supp} show some qualitative results on video object segmentation
with some failure cases in Fig.~\ref{fig:video_seg_neg_supp}.
Figure~\ref{fig:semantic_visuals_supp} shows some qualitative results on semantic video segmentation and
Fig.~\ref{fig:color_visuals_supp} shows results on video color propagation.

\newcolumntype{L}[1]{>{\raggedright\let\newline\\\arraybackslash\hspace{0pt}}b{#1}}
\newcolumntype{C}[1]{>{\centering\let\newline\\\arraybackslash\hspace{0pt}}b{#1}}
\newcolumntype{R}[1]{>{\raggedleft\let\newline\\\arraybackslash\hspace{0pt}}b{#1}}

\begin{table*}[h]
\tiny
  \centering
    \begin{tabular}{L{3.0cm} L{2.4cm} L{2.8cm} L{2.8cm} C{0.5cm} C{1.0cm} L{1.2cm}}
      \toprule
\textbf{Experiment} & \textbf{Feature Type} & \textbf{Feature Scale-1, $\Lambda_a$} & \textbf{Feature Scale-2, $\Lambda_b$} & \textbf{$\alpha$} & \textbf{Input Frames} & \textbf{Loss Type} \\
      \midrule
      \textbf{Video Object Segmentation} & ($x,y,Y,Cb,Cr,t$) & (0.02,0.02,0.07,0.4,0.4,0.01) & (0.03,0.03,0.09,0.5,0.5,0.2) & 0.5 & 9 & Logistic\\
      \midrule
      \textbf{Semantic Video Segmentation} & & & & & \\
      \textbf{with CNN1~\cite{yu2015multi}-NoFlow} & ($x,y,R,G,B,t$) & (0.08,0.08,0.2,0.2,0.2,0.04) & (0.11,0.11,0.2,0.2,0.2,0.04) & 0.5 & 3 & Logistic \\
      \textbf{with CNN1~\cite{yu2015multi}-Flow} & ($x+u_x,y+u_y,R,G,B,t$) & (0.11,0.11,0.14,0.14,0.14,0.03) & (0.08,0.08,0.12,0.12,0.12,0.01) & 0.65 & 3 & Logistic\\
      \textbf{with CNN2~\cite{richter2016playing}-Flow} & ($x+u_x,y+u_y,R,G,B,t$) & (0.08,0.08,0.2,0.2,0.2,0.04) & (0.09,0.09,0.25,0.25,0.25,0.03) & 0.5 & 4 & Logistic\\
      \midrule
      \textbf{Video Color Propagation} & ($x,y,I,t$)  & (0.04,0.04,0.2,0.04) & No second kernel & 1 & 4 & MSE\\
      \bottomrule
      \\
    \end{tabular}
    \mycaption{Experiment Protocols} {Experiment protocols for the different experiments presented in this work. \textbf{Feature Types}:
    Feature spaces used for the bilateral convolutions, with position ($x,y$) and color
    ($R,G,B$ or $Y,Cb,Cr$) features $\in [0,255]$. $u_x$, $u_y$ denotes optical flow with respect
    to the present frame and $I$ denotes grayscale intensity.
    \textbf{Feature Scales ($\Lambda_a, \Lambda_b$)}: Cross-validated scales for the features used.
    \textbf{$\alpha$}: Exponential time decay for the input frames.
    \textbf{Input Frames}: Number of input frames for VPN.
    \textbf{Loss Type}: Type
     of loss used for back-propagation. ``MSE'' corresponds to Euclidean mean squared error loss and ``Logistic'' corresponds to multinomial logistic loss.}
  \label{tbl:parameters_supp}
\end{table*}

% \begin{figure}[th!]
% \begin{center}
%   \centerline{\includegraphics[width=\textwidth]{figures/video_seg_visuals_supp_small.pdf}}
%     \mycaption{Video Object Segmentation}
%     {Shown are the different frames in example videos with the corresponding
%     ground truth (GT) masks, predictions from BVS~\cite{marki2016bilateral},
%     OFL~\cite{tsaivideo}, VPN (VPN-Stage2) and VPN-DLab (VPN-DeepLab) models.}
%     \label{fig:video_seg_small_supp}
% \end{center}
% \vspace{-1.0cm}
% \end{figure}

\begin{figure}[th!]
\begin{center}
  \centerline{\includegraphics[width=0.7\textwidth]{figures/video_seg_visuals_supp_positive.pdf}}
    \mycaption{Video Object Segmentation}
    {Shown are the different frames in example videos with the corresponding
    ground truth (GT) masks, predictions from BVS~\cite{marki2016bilateral},
    OFL~\cite{tsaivideo}, VPN (VPN-Stage2) and VPN-DLab (VPN-DeepLab) models.}
    \label{fig:video_seg_pos_supp}
\end{center}
\vspace{-1.0cm}
\end{figure}

\begin{figure}[th!]
\begin{center}
  \centerline{\includegraphics[width=0.7\textwidth]{figures/video_seg_visuals_supp_negative.pdf}}
    \mycaption{Failure Cases for Video Object Segmentation}
    {Shown are the different frames in example videos with the corresponding
    ground truth (GT) masks, predictions from BVS~\cite{marki2016bilateral},
    OFL~\cite{tsaivideo}, VPN (VPN-Stage2) and VPN-DLab (VPN-DeepLab) models.}
    \label{fig:video_seg_neg_supp}
\end{center}
\vspace{-1.0cm}
\end{figure}

\begin{figure}[th!]
\begin{center}
  \centerline{\includegraphics[width=0.9\textwidth]{figures/supp_semantic_visual.pdf}}
    \mycaption{Semantic Video Segmentation}
    {Input video frames and the corresponding ground truth (GT)
    segmentation together with the predictions of CNN~\cite{yu2015multi} and with
    VPN-Flow.}
    \label{fig:semantic_visuals_supp}
\end{center}
\vspace{-0.7cm}
\end{figure}

\begin{figure}[th!]
\begin{center}
  \centerline{\includegraphics[width=\textwidth]{figures/colorization_visuals_supp.pdf}}
  \mycaption{Video Color Propagation}
  {Input grayscale video frames and corresponding ground-truth (GT) color images
  together with color predictions of Levin et al.~\cite{levin2004colorization} and VPN-Stage1 models.}
  \label{fig:color_visuals_supp}
\end{center}
\vspace{-0.7cm}
\end{figure}

\clearpage

\section{Additional Material for Bilateral Inception Networks}
\label{sec:binception-app}

In this section of the Appendix, we first discuss the use of approximate bilateral
filtering in BI modules (Sec.~\ref{sec:lattice}).
Later, we present some qualitative results using different models for the approach presented in
Chapter~\ref{chap:binception} (Sec.~\ref{sec:qualitative-app}).

\subsection{Approximate Bilateral Filtering}
\label{sec:lattice}

The bilateral inception module presented in Chapter~\ref{chap:binception} computes a matrix-vector
product between a Gaussian filter $K$ and a vector of activations $\bz_c$.
Bilateral filtering is an important operation and many algorithmic techniques have been
proposed to speed-up this operation~\cite{paris2006fast,adams2010fast,gastal2011domain}.
In the main paper we opted to implement what can be considered the
brute-force variant of explicitly constructing $K$ and then using BLAS to compute the
matrix-vector product. This resulted in a few millisecond operation.
The explicit way to compute is possible due to the
reduction to super-pixels, e.g., it would not work for DenseCRF variants
that operate on the full image resolution.

Here, we present experiments where we use the fast approximate bilateral filtering
algorithm of~\cite{adams2010fast}, which is also used in Chapter~\ref{chap:bnn}
for learning sparse high dimensional filters. This
choice allows for larger dimensions of matrix-vector multiplication. The reason for choosing
the explicit multiplication in Chapter~\ref{chap:binception} was that it was computationally faster.
For the small sizes of the involved matrices and vectors, the explicit computation is sufficient and we had no
GPU implementation of an approximate technique that matched this runtime. Also it
is conceptually easier and the gradient to the feature transformations ($\Lambda \mathbf{f}$) is
obtained using standard matrix calculus.

\subsubsection{Experiments}

We modified the existing segmentation architectures analogous to those in Chapter~\ref{chap:binception}.
The main difference is that, here, the inception modules use the lattice
approximation~\cite{adams2010fast} to compute the bilateral filtering.
Using the lattice approximation did not allow us to back-propagate through feature transformations ($\Lambda$)
and thus we used hand-specified feature scales as will be explained later.
Specifically, we take CNN architectures from the works
of~\cite{chen2014semantic,zheng2015conditional,bell2015minc} and insert the BI modules between
the spatial FC layers.
We use superpixels from~\cite{DollarICCV13edges}
for all the experiments with the lattice approximation. Experiments are
performed using Caffe neural network framework~\cite{jia2014caffe}.

\begin{table}
  \small
  \centering
  \begin{tabular}{p{5.5cm}>{\raggedright\arraybackslash}p{1.4cm}>{\centering\arraybackslash}p{2.2cm}}
    \toprule
		\textbf{Model} & \emph{IoU} & \emph{Runtime}(ms) \\
    \midrule

    %%%%%%%%%%%% Scores computed by us)%%%%%%%%%%%%
		\deeplablargefov & 68.9 & 145ms\\
    \midrule
    \bi{7}{2}-\bi{8}{10}& \textbf{73.8} & +600 \\
    \midrule
    \deeplablargefovcrf~\cite{chen2014semantic} & 72.7 & +830\\
    \deeplabmsclargefovcrf~\cite{chen2014semantic} & \textbf{73.6} & +880\\
    DeepLab-EdgeNet~\cite{chen2015semantic} & 71.7 & +30\\
    DeepLab-EdgeNet-CRF~\cite{chen2015semantic} & \textbf{73.6} & +860\\
  \bottomrule \\
  \end{tabular}
  \mycaption{Semantic Segmentation using the DeepLab model}
  {IoU scores on the Pascal VOC12 segmentation test dataset
  with different models and our modified inception model.
  Also shown are the corresponding runtimes in milliseconds. Runtimes
  also include superpixel computations (300 ms with Dollar superpixels~\cite{DollarICCV13edges})}
  \label{tab:largefovresults}
\end{table}

\paragraph{Semantic Segmentation}
The experiments in this section use the Pascal VOC12 segmentation dataset~\cite{voc2012segmentation} with 21 object classes and the images have a maximum resolution of 0.25 megapixels.
For all experiments on VOC12, we train using the extended training set of
10581 images collected by~\cite{hariharan2011moredata}.
We modified the \deeplab~network architecture of~\cite{chen2014semantic} and
the CRFasRNN architecture from~\cite{zheng2015conditional} which uses a CNN with
deconvolution layers followed by DenseCRF trained end-to-end.

\paragraph{DeepLab Model}\label{sec:deeplabmodel}
We experimented with the \bi{7}{2}-\bi{8}{10} inception model.
Results using the~\deeplab~model are summarized in Tab.~\ref{tab:largefovresults}.
Although we get similar improvements with inception modules as with the
explicit kernel computation, using lattice approximation is slower.

\begin{table}
  \small
  \centering
  \begin{tabular}{p{6.4cm}>{\raggedright\arraybackslash}p{1.8cm}>{\raggedright\arraybackslash}p{1.8cm}}
    \toprule
    \textbf{Model} & \emph{IoU (Val)} & \emph{IoU (Test)}\\
    \midrule
    %%%%%%%%%%%% Scores computed by us)%%%%%%%%%%%%
    CNN &  67.5 & - \\
    \deconv (CNN+Deconvolutions) & 69.8 & 72.0 \\
    \midrule
    \bi{3}{6}-\bi{4}{6}-\bi{7}{2}-\bi{8}{6}& 71.9 & - \\
    \bi{3}{6}-\bi{4}{6}-\bi{7}{2}-\bi{8}{6}-\gi{6}& 73.6 &  \href{http://host.robots.ox.ac.uk:8080/anonymous/VOTV5E.html}{\textbf{75.2}}\\
    \midrule
    \deconvcrf (CRF-RNN)~\cite{zheng2015conditional} & 73.0 & 74.7\\
    Context-CRF-RNN~\cite{yu2015multi} & ~~ - ~ & \textbf{75.3} \\
    \bottomrule \\
  \end{tabular}
  \mycaption{Semantic Segmentation using the CRFasRNN model}{IoU score corresponding to different models
  on Pascal VOC12 reduced validation / test segmentation dataset. The reduced validation set consists of 346 images
  as used in~\cite{zheng2015conditional} where we adapted the model from.}
  \label{tab:deconvresults-app}
\end{table}

\paragraph{CRFasRNN Model}\label{sec:deepinception}
We add BI modules after score-pool3, score-pool4, \fc{7} and \fc{8} $1\times1$ convolution layers
resulting in the \bi{3}{6}-\bi{4}{6}-\bi{7}{2}-\bi{8}{6}
model and also experimented with another variant where $BI_8$ is followed by another inception
module, G$(6)$, with 6 Gaussian kernels.
Note that here also we discarded both deconvolution and DenseCRF parts of the original model~\cite{zheng2015conditional}
and inserted the BI modules in the base CNN and found similar improvements compared to the inception modules with explicit
kernel computaion. See Tab.~\ref{tab:deconvresults-app} for results on the CRFasRNN model.

\paragraph{Material Segmentation}
Table~\ref{tab:mincresults-app} shows the results on the MINC dataset~\cite{bell2015minc}
obtained by modifying the AlexNet architecture with our inception modules. We observe
similar improvements as with explicit kernel construction.
For this model, we do not provide any learned setup due to very limited segment training
data. The weights to combine outputs in the bilateral inception layer are
found by validation on the validation set.

\begin{table}[t]
  \small
  \centering
  \begin{tabular}{p{3.5cm}>{\centering\arraybackslash}p{4.0cm}}
    \toprule
    \textbf{Model} & Class / Total accuracy\\
    \midrule

    %%%%%%%%%%%% Scores computed by us)%%%%%%%%%%%%
    AlexNet CNN & 55.3 / 58.9 \\
    \midrule
    \bi{7}{2}-\bi{8}{6}& 68.5 / 71.8 \\
    \bi{7}{2}-\bi{8}{6}-G$(6)$& 67.6 / 73.1 \\
    \midrule
    AlexNet-CRF & 65.5 / 71.0 \\
    \bottomrule \\
  \end{tabular}
  \mycaption{Material Segmentation using AlexNet}{Pixel accuracy of different models on
  the MINC material segmentation test dataset~\cite{bell2015minc}.}
  \label{tab:mincresults-app}
\end{table}

\paragraph{Scales of Bilateral Inception Modules}
\label{sec:scales}

Unlike the explicit kernel technique presented in the main text (Chapter~\ref{chap:binception}),
we didn't back-propagate through feature transformation ($\Lambda$)
using the approximate bilateral filter technique.
So, the feature scales are hand-specified and validated, which are as follows.
The optimal scale values for the \bi{7}{2}-\bi{8}{2} model are found by validation for the best performance which are
$\sigma_{xy}$ = (0.1, 0.1) for the spatial (XY) kernel and $\sigma_{rgbxy}$ = (0.1, 0.1, 0.1, 0.01, 0.01) for color and position (RGBXY)  kernel.
Next, as more kernels are added to \bi{8}{2}, we set scales to be $\alpha$*($\sigma_{xy}$, $\sigma_{rgbxy}$).
The value of $\alpha$ is chosen as  1, 0.5, 0.1, 0.05, 0.1, at uniform interval, for the \bi{8}{10} bilateral inception module.


\subsection{Qualitative Results}
\label{sec:qualitative-app}

In this section, we present more qualitative results obtained using the BI module with explicit
kernel computation technique presented in Chapter~\ref{chap:binception}. Results on the Pascal VOC12
dataset~\cite{voc2012segmentation} using the DeepLab-LargeFOV model are shown in Fig.~\ref{fig:semantic_visuals-app},
followed by the results on MINC dataset~\cite{bell2015minc}
in Fig.~\ref{fig:material_visuals-app} and on
Cityscapes dataset~\cite{Cordts2015Cvprw} in Fig.~\ref{fig:street_visuals-app}.


\definecolor{voc_1}{RGB}{0, 0, 0}
\definecolor{voc_2}{RGB}{128, 0, 0}
\definecolor{voc_3}{RGB}{0, 128, 0}
\definecolor{voc_4}{RGB}{128, 128, 0}
\definecolor{voc_5}{RGB}{0, 0, 128}
\definecolor{voc_6}{RGB}{128, 0, 128}
\definecolor{voc_7}{RGB}{0, 128, 128}
\definecolor{voc_8}{RGB}{128, 128, 128}
\definecolor{voc_9}{RGB}{64, 0, 0}
\definecolor{voc_10}{RGB}{192, 0, 0}
\definecolor{voc_11}{RGB}{64, 128, 0}
\definecolor{voc_12}{RGB}{192, 128, 0}
\definecolor{voc_13}{RGB}{64, 0, 128}
\definecolor{voc_14}{RGB}{192, 0, 128}
\definecolor{voc_15}{RGB}{64, 128, 128}
\definecolor{voc_16}{RGB}{192, 128, 128}
\definecolor{voc_17}{RGB}{0, 64, 0}
\definecolor{voc_18}{RGB}{128, 64, 0}
\definecolor{voc_19}{RGB}{0, 192, 0}
\definecolor{voc_20}{RGB}{128, 192, 0}
\definecolor{voc_21}{RGB}{0, 64, 128}
\definecolor{voc_22}{RGB}{128, 64, 128}

\begin{figure*}[!ht]
  \small
  \centering
  \fcolorbox{white}{voc_1}{\rule{0pt}{4pt}\rule{4pt}{0pt}} Background~~
  \fcolorbox{white}{voc_2}{\rule{0pt}{4pt}\rule{4pt}{0pt}} Aeroplane~~
  \fcolorbox{white}{voc_3}{\rule{0pt}{4pt}\rule{4pt}{0pt}} Bicycle~~
  \fcolorbox{white}{voc_4}{\rule{0pt}{4pt}\rule{4pt}{0pt}} Bird~~
  \fcolorbox{white}{voc_5}{\rule{0pt}{4pt}\rule{4pt}{0pt}} Boat~~
  \fcolorbox{white}{voc_6}{\rule{0pt}{4pt}\rule{4pt}{0pt}} Bottle~~
  \fcolorbox{white}{voc_7}{\rule{0pt}{4pt}\rule{4pt}{0pt}} Bus~~
  \fcolorbox{white}{voc_8}{\rule{0pt}{4pt}\rule{4pt}{0pt}} Car~~\\
  \fcolorbox{white}{voc_9}{\rule{0pt}{4pt}\rule{4pt}{0pt}} Cat~~
  \fcolorbox{white}{voc_10}{\rule{0pt}{4pt}\rule{4pt}{0pt}} Chair~~
  \fcolorbox{white}{voc_11}{\rule{0pt}{4pt}\rule{4pt}{0pt}} Cow~~
  \fcolorbox{white}{voc_12}{\rule{0pt}{4pt}\rule{4pt}{0pt}} Dining Table~~
  \fcolorbox{white}{voc_13}{\rule{0pt}{4pt}\rule{4pt}{0pt}} Dog~~
  \fcolorbox{white}{voc_14}{\rule{0pt}{4pt}\rule{4pt}{0pt}} Horse~~
  \fcolorbox{white}{voc_15}{\rule{0pt}{4pt}\rule{4pt}{0pt}} Motorbike~~
  \fcolorbox{white}{voc_16}{\rule{0pt}{4pt}\rule{4pt}{0pt}} Person~~\\
  \fcolorbox{white}{voc_17}{\rule{0pt}{4pt}\rule{4pt}{0pt}} Potted Plant~~
  \fcolorbox{white}{voc_18}{\rule{0pt}{4pt}\rule{4pt}{0pt}} Sheep~~
  \fcolorbox{white}{voc_19}{\rule{0pt}{4pt}\rule{4pt}{0pt}} Sofa~~
  \fcolorbox{white}{voc_20}{\rule{0pt}{4pt}\rule{4pt}{0pt}} Train~~
  \fcolorbox{white}{voc_21}{\rule{0pt}{4pt}\rule{4pt}{0pt}} TV monitor~~\\


  \subfigure{%
    \includegraphics[width=.15\columnwidth]{figures/supplementary/2008_001308_given.png}
  }
  \subfigure{%
    \includegraphics[width=.15\columnwidth]{figures/supplementary/2008_001308_sp.png}
  }
  \subfigure{%
    \includegraphics[width=.15\columnwidth]{figures/supplementary/2008_001308_gt.png}
  }
  \subfigure{%
    \includegraphics[width=.15\columnwidth]{figures/supplementary/2008_001308_cnn.png}
  }
  \subfigure{%
    \includegraphics[width=.15\columnwidth]{figures/supplementary/2008_001308_crf.png}
  }
  \subfigure{%
    \includegraphics[width=.15\columnwidth]{figures/supplementary/2008_001308_ours.png}
  }\\[-2ex]


  \subfigure{%
    \includegraphics[width=.15\columnwidth]{figures/supplementary/2008_001821_given.png}
  }
  \subfigure{%
    \includegraphics[width=.15\columnwidth]{figures/supplementary/2008_001821_sp.png}
  }
  \subfigure{%
    \includegraphics[width=.15\columnwidth]{figures/supplementary/2008_001821_gt.png}
  }
  \subfigure{%
    \includegraphics[width=.15\columnwidth]{figures/supplementary/2008_001821_cnn.png}
  }
  \subfigure{%
    \includegraphics[width=.15\columnwidth]{figures/supplementary/2008_001821_crf.png}
  }
  \subfigure{%
    \includegraphics[width=.15\columnwidth]{figures/supplementary/2008_001821_ours.png}
  }\\[-2ex]



  \subfigure{%
    \includegraphics[width=.15\columnwidth]{figures/supplementary/2008_004612_given.png}
  }
  \subfigure{%
    \includegraphics[width=.15\columnwidth]{figures/supplementary/2008_004612_sp.png}
  }
  \subfigure{%
    \includegraphics[width=.15\columnwidth]{figures/supplementary/2008_004612_gt.png}
  }
  \subfigure{%
    \includegraphics[width=.15\columnwidth]{figures/supplementary/2008_004612_cnn.png}
  }
  \subfigure{%
    \includegraphics[width=.15\columnwidth]{figures/supplementary/2008_004612_crf.png}
  }
  \subfigure{%
    \includegraphics[width=.15\columnwidth]{figures/supplementary/2008_004612_ours.png}
  }\\[-2ex]


  \subfigure{%
    \includegraphics[width=.15\columnwidth]{figures/supplementary/2009_001008_given.png}
  }
  \subfigure{%
    \includegraphics[width=.15\columnwidth]{figures/supplementary/2009_001008_sp.png}
  }
  \subfigure{%
    \includegraphics[width=.15\columnwidth]{figures/supplementary/2009_001008_gt.png}
  }
  \subfigure{%
    \includegraphics[width=.15\columnwidth]{figures/supplementary/2009_001008_cnn.png}
  }
  \subfigure{%
    \includegraphics[width=.15\columnwidth]{figures/supplementary/2009_001008_crf.png}
  }
  \subfigure{%
    \includegraphics[width=.15\columnwidth]{figures/supplementary/2009_001008_ours.png}
  }\\[-2ex]




  \subfigure{%
    \includegraphics[width=.15\columnwidth]{figures/supplementary/2009_004497_given.png}
  }
  \subfigure{%
    \includegraphics[width=.15\columnwidth]{figures/supplementary/2009_004497_sp.png}
  }
  \subfigure{%
    \includegraphics[width=.15\columnwidth]{figures/supplementary/2009_004497_gt.png}
  }
  \subfigure{%
    \includegraphics[width=.15\columnwidth]{figures/supplementary/2009_004497_cnn.png}
  }
  \subfigure{%
    \includegraphics[width=.15\columnwidth]{figures/supplementary/2009_004497_crf.png}
  }
  \subfigure{%
    \includegraphics[width=.15\columnwidth]{figures/supplementary/2009_004497_ours.png}
  }\\[-2ex]



  \setcounter{subfigure}{0}
  \subfigure[\scriptsize Input]{%
    \includegraphics[width=.15\columnwidth]{figures/supplementary/2010_001327_given.png}
  }
  \subfigure[\scriptsize Superpixels]{%
    \includegraphics[width=.15\columnwidth]{figures/supplementary/2010_001327_sp.png}
  }
  \subfigure[\scriptsize GT]{%
    \includegraphics[width=.15\columnwidth]{figures/supplementary/2010_001327_gt.png}
  }
  \subfigure[\scriptsize Deeplab]{%
    \includegraphics[width=.15\columnwidth]{figures/supplementary/2010_001327_cnn.png}
  }
  \subfigure[\scriptsize +DenseCRF]{%
    \includegraphics[width=.15\columnwidth]{figures/supplementary/2010_001327_crf.png}
  }
  \subfigure[\scriptsize Using BI]{%
    \includegraphics[width=.15\columnwidth]{figures/supplementary/2010_001327_ours.png}
  }
  \mycaption{Semantic Segmentation}{Example results of semantic segmentation
  on the Pascal VOC12 dataset.
  (d)~depicts the DeepLab CNN result, (e)~CNN + 10 steps of mean-field inference,
  (f~result obtained with bilateral inception (BI) modules (\bi{6}{2}+\bi{7}{6}) between \fc~layers.}
  \label{fig:semantic_visuals-app}
\end{figure*}


\definecolor{minc_1}{HTML}{771111}
\definecolor{minc_2}{HTML}{CAC690}
\definecolor{minc_3}{HTML}{EEEEEE}
\definecolor{minc_4}{HTML}{7C8FA6}
\definecolor{minc_5}{HTML}{597D31}
\definecolor{minc_6}{HTML}{104410}
\definecolor{minc_7}{HTML}{BB819C}
\definecolor{minc_8}{HTML}{D0CE48}
\definecolor{minc_9}{HTML}{622745}
\definecolor{minc_10}{HTML}{666666}
\definecolor{minc_11}{HTML}{D54A31}
\definecolor{minc_12}{HTML}{101044}
\definecolor{minc_13}{HTML}{444126}
\definecolor{minc_14}{HTML}{75D646}
\definecolor{minc_15}{HTML}{DD4348}
\definecolor{minc_16}{HTML}{5C8577}
\definecolor{minc_17}{HTML}{C78472}
\definecolor{minc_18}{HTML}{75D6D0}
\definecolor{minc_19}{HTML}{5B4586}
\definecolor{minc_20}{HTML}{C04393}
\definecolor{minc_21}{HTML}{D69948}
\definecolor{minc_22}{HTML}{7370D8}
\definecolor{minc_23}{HTML}{7A3622}
\definecolor{minc_24}{HTML}{000000}

\begin{figure*}[!ht]
  \small % scriptsize
  \centering
  \fcolorbox{white}{minc_1}{\rule{0pt}{4pt}\rule{4pt}{0pt}} Brick~~
  \fcolorbox{white}{minc_2}{\rule{0pt}{4pt}\rule{4pt}{0pt}} Carpet~~
  \fcolorbox{white}{minc_3}{\rule{0pt}{4pt}\rule{4pt}{0pt}} Ceramic~~
  \fcolorbox{white}{minc_4}{\rule{0pt}{4pt}\rule{4pt}{0pt}} Fabric~~
  \fcolorbox{white}{minc_5}{\rule{0pt}{4pt}\rule{4pt}{0pt}} Foliage~~
  \fcolorbox{white}{minc_6}{\rule{0pt}{4pt}\rule{4pt}{0pt}} Food~~
  \fcolorbox{white}{minc_7}{\rule{0pt}{4pt}\rule{4pt}{0pt}} Glass~~
  \fcolorbox{white}{minc_8}{\rule{0pt}{4pt}\rule{4pt}{0pt}} Hair~~\\
  \fcolorbox{white}{minc_9}{\rule{0pt}{4pt}\rule{4pt}{0pt}} Leather~~
  \fcolorbox{white}{minc_10}{\rule{0pt}{4pt}\rule{4pt}{0pt}} Metal~~
  \fcolorbox{white}{minc_11}{\rule{0pt}{4pt}\rule{4pt}{0pt}} Mirror~~
  \fcolorbox{white}{minc_12}{\rule{0pt}{4pt}\rule{4pt}{0pt}} Other~~
  \fcolorbox{white}{minc_13}{\rule{0pt}{4pt}\rule{4pt}{0pt}} Painted~~
  \fcolorbox{white}{minc_14}{\rule{0pt}{4pt}\rule{4pt}{0pt}} Paper~~
  \fcolorbox{white}{minc_15}{\rule{0pt}{4pt}\rule{4pt}{0pt}} Plastic~~\\
  \fcolorbox{white}{minc_16}{\rule{0pt}{4pt}\rule{4pt}{0pt}} Polished Stone~~
  \fcolorbox{white}{minc_17}{\rule{0pt}{4pt}\rule{4pt}{0pt}} Skin~~
  \fcolorbox{white}{minc_18}{\rule{0pt}{4pt}\rule{4pt}{0pt}} Sky~~
  \fcolorbox{white}{minc_19}{\rule{0pt}{4pt}\rule{4pt}{0pt}} Stone~~
  \fcolorbox{white}{minc_20}{\rule{0pt}{4pt}\rule{4pt}{0pt}} Tile~~
  \fcolorbox{white}{minc_21}{\rule{0pt}{4pt}\rule{4pt}{0pt}} Wallpaper~~
  \fcolorbox{white}{minc_22}{\rule{0pt}{4pt}\rule{4pt}{0pt}} Water~~
  \fcolorbox{white}{minc_23}{\rule{0pt}{4pt}\rule{4pt}{0pt}} Wood~~\\
  \subfigure{%
    \includegraphics[width=.15\columnwidth]{figures/supplementary/000008468_given.png}
  }
  \subfigure{%
    \includegraphics[width=.15\columnwidth]{figures/supplementary/000008468_sp.png}
  }
  \subfigure{%
    \includegraphics[width=.15\columnwidth]{figures/supplementary/000008468_gt.png}
  }
  \subfigure{%
    \includegraphics[width=.15\columnwidth]{figures/supplementary/000008468_cnn.png}
  }
  \subfigure{%
    \includegraphics[width=.15\columnwidth]{figures/supplementary/000008468_crf.png}
  }
  \subfigure{%
    \includegraphics[width=.15\columnwidth]{figures/supplementary/000008468_ours.png}
  }\\[-2ex]

  \subfigure{%
    \includegraphics[width=.15\columnwidth]{figures/supplementary/000009053_given.png}
  }
  \subfigure{%
    \includegraphics[width=.15\columnwidth]{figures/supplementary/000009053_sp.png}
  }
  \subfigure{%
    \includegraphics[width=.15\columnwidth]{figures/supplementary/000009053_gt.png}
  }
  \subfigure{%
    \includegraphics[width=.15\columnwidth]{figures/supplementary/000009053_cnn.png}
  }
  \subfigure{%
    \includegraphics[width=.15\columnwidth]{figures/supplementary/000009053_crf.png}
  }
  \subfigure{%
    \includegraphics[width=.15\columnwidth]{figures/supplementary/000009053_ours.png}
  }\\[-2ex]




  \subfigure{%
    \includegraphics[width=.15\columnwidth]{figures/supplementary/000014977_given.png}
  }
  \subfigure{%
    \includegraphics[width=.15\columnwidth]{figures/supplementary/000014977_sp.png}
  }
  \subfigure{%
    \includegraphics[width=.15\columnwidth]{figures/supplementary/000014977_gt.png}
  }
  \subfigure{%
    \includegraphics[width=.15\columnwidth]{figures/supplementary/000014977_cnn.png}
  }
  \subfigure{%
    \includegraphics[width=.15\columnwidth]{figures/supplementary/000014977_crf.png}
  }
  \subfigure{%
    \includegraphics[width=.15\columnwidth]{figures/supplementary/000014977_ours.png}
  }\\[-2ex]


  \subfigure{%
    \includegraphics[width=.15\columnwidth]{figures/supplementary/000022922_given.png}
  }
  \subfigure{%
    \includegraphics[width=.15\columnwidth]{figures/supplementary/000022922_sp.png}
  }
  \subfigure{%
    \includegraphics[width=.15\columnwidth]{figures/supplementary/000022922_gt.png}
  }
  \subfigure{%
    \includegraphics[width=.15\columnwidth]{figures/supplementary/000022922_cnn.png}
  }
  \subfigure{%
    \includegraphics[width=.15\columnwidth]{figures/supplementary/000022922_crf.png}
  }
  \subfigure{%
    \includegraphics[width=.15\columnwidth]{figures/supplementary/000022922_ours.png}
  }\\[-2ex]


  \subfigure{%
    \includegraphics[width=.15\columnwidth]{figures/supplementary/000025711_given.png}
  }
  \subfigure{%
    \includegraphics[width=.15\columnwidth]{figures/supplementary/000025711_sp.png}
  }
  \subfigure{%
    \includegraphics[width=.15\columnwidth]{figures/supplementary/000025711_gt.png}
  }
  \subfigure{%
    \includegraphics[width=.15\columnwidth]{figures/supplementary/000025711_cnn.png}
  }
  \subfigure{%
    \includegraphics[width=.15\columnwidth]{figures/supplementary/000025711_crf.png}
  }
  \subfigure{%
    \includegraphics[width=.15\columnwidth]{figures/supplementary/000025711_ours.png}
  }\\[-2ex]


  \subfigure{%
    \includegraphics[width=.15\columnwidth]{figures/supplementary/000034473_given.png}
  }
  \subfigure{%
    \includegraphics[width=.15\columnwidth]{figures/supplementary/000034473_sp.png}
  }
  \subfigure{%
    \includegraphics[width=.15\columnwidth]{figures/supplementary/000034473_gt.png}
  }
  \subfigure{%
    \includegraphics[width=.15\columnwidth]{figures/supplementary/000034473_cnn.png}
  }
  \subfigure{%
    \includegraphics[width=.15\columnwidth]{figures/supplementary/000034473_crf.png}
  }
  \subfigure{%
    \includegraphics[width=.15\columnwidth]{figures/supplementary/000034473_ours.png}
  }\\[-2ex]


  \subfigure{%
    \includegraphics[width=.15\columnwidth]{figures/supplementary/000035463_given.png}
  }
  \subfigure{%
    \includegraphics[width=.15\columnwidth]{figures/supplementary/000035463_sp.png}
  }
  \subfigure{%
    \includegraphics[width=.15\columnwidth]{figures/supplementary/000035463_gt.png}
  }
  \subfigure{%
    \includegraphics[width=.15\columnwidth]{figures/supplementary/000035463_cnn.png}
  }
  \subfigure{%
    \includegraphics[width=.15\columnwidth]{figures/supplementary/000035463_crf.png}
  }
  \subfigure{%
    \includegraphics[width=.15\columnwidth]{figures/supplementary/000035463_ours.png}
  }\\[-2ex]


  \setcounter{subfigure}{0}
  \subfigure[\scriptsize Input]{%
    \includegraphics[width=.15\columnwidth]{figures/supplementary/000035993_given.png}
  }
  \subfigure[\scriptsize Superpixels]{%
    \includegraphics[width=.15\columnwidth]{figures/supplementary/000035993_sp.png}
  }
  \subfigure[\scriptsize GT]{%
    \includegraphics[width=.15\columnwidth]{figures/supplementary/000035993_gt.png}
  }
  \subfigure[\scriptsize AlexNet]{%
    \includegraphics[width=.15\columnwidth]{figures/supplementary/000035993_cnn.png}
  }
  \subfigure[\scriptsize +DenseCRF]{%
    \includegraphics[width=.15\columnwidth]{figures/supplementary/000035993_crf.png}
  }
  \subfigure[\scriptsize Using BI]{%
    \includegraphics[width=.15\columnwidth]{figures/supplementary/000035993_ours.png}
  }
  \mycaption{Material Segmentation}{Example results of material segmentation.
  (d)~depicts the AlexNet CNN result, (e)~CNN + 10 steps of mean-field inference,
  (f)~result obtained with bilateral inception (BI) modules (\bi{7}{2}+\bi{8}{6}) between
  \fc~layers.}
\label{fig:material_visuals-app}
\end{figure*}


\definecolor{city_1}{RGB}{128, 64, 128}
\definecolor{city_2}{RGB}{244, 35, 232}
\definecolor{city_3}{RGB}{70, 70, 70}
\definecolor{city_4}{RGB}{102, 102, 156}
\definecolor{city_5}{RGB}{190, 153, 153}
\definecolor{city_6}{RGB}{153, 153, 153}
\definecolor{city_7}{RGB}{250, 170, 30}
\definecolor{city_8}{RGB}{220, 220, 0}
\definecolor{city_9}{RGB}{107, 142, 35}
\definecolor{city_10}{RGB}{152, 251, 152}
\definecolor{city_11}{RGB}{70, 130, 180}
\definecolor{city_12}{RGB}{220, 20, 60}
\definecolor{city_13}{RGB}{255, 0, 0}
\definecolor{city_14}{RGB}{0, 0, 142}
\definecolor{city_15}{RGB}{0, 0, 70}
\definecolor{city_16}{RGB}{0, 60, 100}
\definecolor{city_17}{RGB}{0, 80, 100}
\definecolor{city_18}{RGB}{0, 0, 230}
\definecolor{city_19}{RGB}{119, 11, 32}
\begin{figure*}[!ht]
  \small % scriptsize
  \centering


  \subfigure{%
    \includegraphics[width=.18\columnwidth]{figures/supplementary/frankfurt00000_016005_given.png}
  }
  \subfigure{%
    \includegraphics[width=.18\columnwidth]{figures/supplementary/frankfurt00000_016005_sp.png}
  }
  \subfigure{%
    \includegraphics[width=.18\columnwidth]{figures/supplementary/frankfurt00000_016005_gt.png}
  }
  \subfigure{%
    \includegraphics[width=.18\columnwidth]{figures/supplementary/frankfurt00000_016005_cnn.png}
  }
  \subfigure{%
    \includegraphics[width=.18\columnwidth]{figures/supplementary/frankfurt00000_016005_ours.png}
  }\\[-2ex]

  \subfigure{%
    \includegraphics[width=.18\columnwidth]{figures/supplementary/frankfurt00000_004617_given.png}
  }
  \subfigure{%
    \includegraphics[width=.18\columnwidth]{figures/supplementary/frankfurt00000_004617_sp.png}
  }
  \subfigure{%
    \includegraphics[width=.18\columnwidth]{figures/supplementary/frankfurt00000_004617_gt.png}
  }
  \subfigure{%
    \includegraphics[width=.18\columnwidth]{figures/supplementary/frankfurt00000_004617_cnn.png}
  }
  \subfigure{%
    \includegraphics[width=.18\columnwidth]{figures/supplementary/frankfurt00000_004617_ours.png}
  }\\[-2ex]

  \subfigure{%
    \includegraphics[width=.18\columnwidth]{figures/supplementary/frankfurt00000_020880_given.png}
  }
  \subfigure{%
    \includegraphics[width=.18\columnwidth]{figures/supplementary/frankfurt00000_020880_sp.png}
  }
  \subfigure{%
    \includegraphics[width=.18\columnwidth]{figures/supplementary/frankfurt00000_020880_gt.png}
  }
  \subfigure{%
    \includegraphics[width=.18\columnwidth]{figures/supplementary/frankfurt00000_020880_cnn.png}
  }
  \subfigure{%
    \includegraphics[width=.18\columnwidth]{figures/supplementary/frankfurt00000_020880_ours.png}
  }\\[-2ex]



  \subfigure{%
    \includegraphics[width=.18\columnwidth]{figures/supplementary/frankfurt00001_007285_given.png}
  }
  \subfigure{%
    \includegraphics[width=.18\columnwidth]{figures/supplementary/frankfurt00001_007285_sp.png}
  }
  \subfigure{%
    \includegraphics[width=.18\columnwidth]{figures/supplementary/frankfurt00001_007285_gt.png}
  }
  \subfigure{%
    \includegraphics[width=.18\columnwidth]{figures/supplementary/frankfurt00001_007285_cnn.png}
  }
  \subfigure{%
    \includegraphics[width=.18\columnwidth]{figures/supplementary/frankfurt00001_007285_ours.png}
  }\\[-2ex]


  \subfigure{%
    \includegraphics[width=.18\columnwidth]{figures/supplementary/frankfurt00001_059789_given.png}
  }
  \subfigure{%
    \includegraphics[width=.18\columnwidth]{figures/supplementary/frankfurt00001_059789_sp.png}
  }
  \subfigure{%
    \includegraphics[width=.18\columnwidth]{figures/supplementary/frankfurt00001_059789_gt.png}
  }
  \subfigure{%
    \includegraphics[width=.18\columnwidth]{figures/supplementary/frankfurt00001_059789_cnn.png}
  }
  \subfigure{%
    \includegraphics[width=.18\columnwidth]{figures/supplementary/frankfurt00001_059789_ours.png}
  }\\[-2ex]


  \subfigure{%
    \includegraphics[width=.18\columnwidth]{figures/supplementary/frankfurt00001_068208_given.png}
  }
  \subfigure{%
    \includegraphics[width=.18\columnwidth]{figures/supplementary/frankfurt00001_068208_sp.png}
  }
  \subfigure{%
    \includegraphics[width=.18\columnwidth]{figures/supplementary/frankfurt00001_068208_gt.png}
  }
  \subfigure{%
    \includegraphics[width=.18\columnwidth]{figures/supplementary/frankfurt00001_068208_cnn.png}
  }
  \subfigure{%
    \includegraphics[width=.18\columnwidth]{figures/supplementary/frankfurt00001_068208_ours.png}
  }\\[-2ex]

  \subfigure{%
    \includegraphics[width=.18\columnwidth]{figures/supplementary/frankfurt00001_082466_given.png}
  }
  \subfigure{%
    \includegraphics[width=.18\columnwidth]{figures/supplementary/frankfurt00001_082466_sp.png}
  }
  \subfigure{%
    \includegraphics[width=.18\columnwidth]{figures/supplementary/frankfurt00001_082466_gt.png}
  }
  \subfigure{%
    \includegraphics[width=.18\columnwidth]{figures/supplementary/frankfurt00001_082466_cnn.png}
  }
  \subfigure{%
    \includegraphics[width=.18\columnwidth]{figures/supplementary/frankfurt00001_082466_ours.png}
  }\\[-2ex]

  \subfigure{%
    \includegraphics[width=.18\columnwidth]{figures/supplementary/lindau00033_000019_given.png}
  }
  \subfigure{%
    \includegraphics[width=.18\columnwidth]{figures/supplementary/lindau00033_000019_sp.png}
  }
  \subfigure{%
    \includegraphics[width=.18\columnwidth]{figures/supplementary/lindau00033_000019_gt.png}
  }
  \subfigure{%
    \includegraphics[width=.18\columnwidth]{figures/supplementary/lindau00033_000019_cnn.png}
  }
  \subfigure{%
    \includegraphics[width=.18\columnwidth]{figures/supplementary/lindau00033_000019_ours.png}
  }\\[-2ex]

  \subfigure{%
    \includegraphics[width=.18\columnwidth]{figures/supplementary/lindau00052_000019_given.png}
  }
  \subfigure{%
    \includegraphics[width=.18\columnwidth]{figures/supplementary/lindau00052_000019_sp.png}
  }
  \subfigure{%
    \includegraphics[width=.18\columnwidth]{figures/supplementary/lindau00052_000019_gt.png}
  }
  \subfigure{%
    \includegraphics[width=.18\columnwidth]{figures/supplementary/lindau00052_000019_cnn.png}
  }
  \subfigure{%
    \includegraphics[width=.18\columnwidth]{figures/supplementary/lindau00052_000019_ours.png}
  }\\[-2ex]




  \subfigure{%
    \includegraphics[width=.18\columnwidth]{figures/supplementary/lindau00027_000019_given.png}
  }
  \subfigure{%
    \includegraphics[width=.18\columnwidth]{figures/supplementary/lindau00027_000019_sp.png}
  }
  \subfigure{%
    \includegraphics[width=.18\columnwidth]{figures/supplementary/lindau00027_000019_gt.png}
  }
  \subfigure{%
    \includegraphics[width=.18\columnwidth]{figures/supplementary/lindau00027_000019_cnn.png}
  }
  \subfigure{%
    \includegraphics[width=.18\columnwidth]{figures/supplementary/lindau00027_000019_ours.png}
  }\\[-2ex]



  \setcounter{subfigure}{0}
  \subfigure[\scriptsize Input]{%
    \includegraphics[width=.18\columnwidth]{figures/supplementary/lindau00029_000019_given.png}
  }
  \subfigure[\scriptsize Superpixels]{%
    \includegraphics[width=.18\columnwidth]{figures/supplementary/lindau00029_000019_sp.png}
  }
  \subfigure[\scriptsize GT]{%
    \includegraphics[width=.18\columnwidth]{figures/supplementary/lindau00029_000019_gt.png}
  }
  \subfigure[\scriptsize Deeplab]{%
    \includegraphics[width=.18\columnwidth]{figures/supplementary/lindau00029_000019_cnn.png}
  }
  \subfigure[\scriptsize Using BI]{%
    \includegraphics[width=.18\columnwidth]{figures/supplementary/lindau00029_000019_ours.png}
  }%\\[-2ex]

  \mycaption{Street Scene Segmentation}{Example results of street scene segmentation.
  (d)~depicts the DeepLab results, (e)~result obtained by adding bilateral inception (BI) modules (\bi{6}{2}+\bi{7}{6}) between \fc~layers.}
\label{fig:street_visuals-app}
\end{figure*}
	
		
\end{document}
