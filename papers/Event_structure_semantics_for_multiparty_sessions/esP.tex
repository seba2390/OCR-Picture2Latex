
% !TEX root =cdgS.tex

\section{Event Structure Semantics of Processes}\mylabel{sec:process-ES}

In this section, we define an event structure semantics for processes,
and show that the obtained event structures are PESs.  This semantics
will be the basis for defining the ES semantics for networks
in~\refToSection{sec:netS-ES}.  We start by introducing process
events, which are non-empty sequences of actions.


\begin{definition}[Process event]
  \mylabel{proceventP} 
{\em Process events} $\procev,
  \procev'$,  also called \emph{p-events},  are defined by: 
\[\procev \ \quad ::= \pi ~~ \mid~~
    \pastpref{\pi}{\procev} \qquad\qquad \pi \in \!
    \set{\sendL{\pp}{\la}, \rcvL{\pp}{\la} 
      \mid \pp\in \Participants, \la \in \Messages}\]
%$}
We denote by $\Procev$ the set of  p-events, and by
$\cardin{\procev}$ the length of the sequence of actions in
the p-event $\procev$. 
\end{definition}
%

Let $\actseq$ denote a (possibly empty) sequence of actions, and
$\sqsubseteq$ denote the prefix ordering on such sequences.  Each
 p-event  $\procev$ may be written either in the form $\procev = \concat
{\pi}{\actseq}$ or in the form $\procev =
\concat{\actseq}{\pi}$. We shall feel free to use any of these forms. 
When a  p-event  is written as $\procev = \concat{\actseq}{\pi} $,
then $\actseq$ may be viewed as the \emph{causal history} of
$\procev$, namely the sequence of past actions that must have happened
in the process for $\procev$ to be able to happen. 

We define the  \emph{action}  of a  p-event   to be
its last action: 
%\\
%as  the rightmost action:\\  %follows
% \centerline{$ 
\[
\act{\concat{\actseq}{\pi}} = \pi
\]
%$}


\begin{definition}[Causality and conflict relations on process events] %\text{~}\\[5pt]
\mylabel{procevent-relations}
The \emph{causality} relation $\leq$ and the \emph{conflict} relation
$\gr$ on the set of  p-events  $\Procev$ are defined by:
\begin{enumerate}
\item \mylabel{ila--esp2} 
%the $\precP$ relation on $\Procev$ is  given by:\\
$\procev \sqsubseteq \procev' \ \impl \ \procev\precP\procev'$;\smallskip
\item \mylabel{ila--esp3} 
%the $\gr$ relation on $\Procev$ is given by: \\
  $\pi\neq\pi'  \impl
  \concat{\concat{\actseq}{\pi}}{\actseq'}\,\grr\,\,\concat{\concat{\actseq}{\pi'}}{\actseq''}$.
\end{enumerate}
\end{definition}
%

% \bigskip 
% \noindent
% \bcomila green Io penso che le relazioni di causalità e conflitto dovrebbero
% essere definite sugli eventi in generale, come nel mio lavoro con
% Gérard su CCS (RR-1484 in PAPERS), e non solo sugli eventi delle PES
% di processo, come facciamo ora. Idem per gli eventi delle reti sincrone
% in seguito.  La nostra definizione sintattica degli eventi ci permette
% di farlo. Il motivo per cui lo propongo è che questo ci facilita la
% vita nelle prove, perché ci permette di usare liberamente la stessa
% notazione $\leq, \precN, \grr$ per tutte le PES di processo e FES di
% rete senza specificare a quale PES o FES ci riferiamo. In realtà, in
% molte delle nostre prove attuali usiamo già (impropriamente) queste
% relazioni come se fossero definite in generale, senza far riferimento
% ad una particolare PES o FES. Se invece diamo la definizione in
% generale, questo uso diventa corretto. In conclusione, io darei la
% definizione di causalità e conflitto sugli eventi di processo in
% questo punto qui, cioè subito prima la \refToDef{esp}. green\ecomila
% \bigskip

\begin{definition} [Event structure of a process] \mylabel{esp}
The {\em event structure of process} $\PP$ is the triple
%\\
% \centerline{$
\[
 \ESP{\PP} = (\ES(\PP), \precP_\PP , \gr_\PP)
 \]
% $} 
 where:
\begin{enumerate}
%\item  \mylabel{esp1} $\ESPlus(\PP)$ is defined by:
%\begin{enumerate}
%\item \mylabel{esp1a} $\ESPlus(\inp\pp{i}{I}{\la}{\PP} )\bpa = \epa \bigcup_{i\in I} \set{  \rcvL{\pp}{\la_i}   } \cup \bigcup_{i\in I}
%\set{{ \rcvL{\pp}{\la_i} }\cdot \procev_i \sep \procev_i
%  \in\ESPlus(\PP_i)}$;
%\item \mylabel{esp1b} $\ESPlus(\oup\pp{i}{I}{\la}{\PP} )\bpa = \epa  \bigcup_{i\in I} \set{  \sendL{\pp}{\la_i}   } \cup \bigcup_{i\in I}
%\set{{ \sendL{\pp}{\la_i} }\cdot \procev_i \sep \procev_i
%  \in\ESPlus(\PP_i)}$;
%\item  $\ESPlus( \inact ) =  
%\emptyset$;
%\end{enumerate}
%and $\ES(\PP)=\{\procev\ |\ \procev\in\ESPlus(\PP)\}$ 
\item \mylabel{ila-esp1} 
%$\ES(\PP) \subseteq \Procev $ is the set of
%  non-empty traces
%  of $\PP$ (cf \refToDef{def:trace}); \\
   $\ES(\PP) \subseteq \Procev $ is the set of non-empty sequences
  of labels along the nodes and edges of a path from the root to an
  edge in the tree of $\PP$; 
%  \epa \bcomila  cambiare con ''a path from the root to an
%  edge in the tree of $\PP$'' ? \ecomila
%\\
% \bcomila Prima era ``the set containing all 
% the sequences of labels of nodes and edges found on the way from the
% root of the tree of $\PP$ to a node (omitting the label of the node
% itself).'' \ecomila
%\\
\item \mylabel{ila-esp2} 
$\precP_\PP$ is the restriction of
  $\precP$ to the set %  $(\ES(\PP) \times \ES(\PP))$; 
$\ES(\PP)$;
   %  $\procev \sqsubseteq \procev' \ \impl \ \procev
% \precP\procev'$; 
\item \mylabel{ila-esp3} 
$\gr_\PP$ is the restriction of
  $\grr$ to the set 
% $(\ES(\PP) \times \ES(\PP))$. 
$\ES(\PP)$.
% the  $\gr$ relation on the set of events $\ES(\PP)$ is
%   given by: \\
%   $\pi\neq\pi' \impl  \concat{\concat{\actseq}{\pi}}{\actseq'}\,\grr\,\,\concat{\concat{\actseq}{\pi'}}{\actseq''}  $ .
\end{enumerate}
\end{definition}

It is easy to see that $\gr_\PP = (\ES(\PP)\times \ES(\PP))
\,\backslash \, (\precP_\PP \cup \geq_\PP)$.
%
In the following we shall feel free to drop the subscript in
$\precP_\PP$ and $\gr_\PP$.
\bigskip

%%%%%%%%%%%%%%%%%%%%%%%
Note that the set $\ES(\PP)$ may be denumerable, as shown by the following example.  
\begin{example}\mylabel{ex:rec1}
If  $\PP=\sendL{\q}\la;\PP \oplus\sendL{\q}{\la'}$, 
then 
$ \ES(\PP)  = \begin{array}[t]{l}
\set{\underbrace{\sendL{\q}\la\cdot\ldots\cdot\sendL{\q}\la}_n \mid
  n\geq 1} \quad \cup \\
\set{\underbrace{\sendL{\q}\la\cdot\ldots\cdot\sendL{\q}\la}_n\cdot\sendL{\q}{\la'}\mid
  n\geq 0} 
\end{array} $
\end{example}


\begin{proposition}\mylabel{basta10}
  Let $\PP$ be a process. Then $\ESP{\PP}$ is a prime event structure.
\end{proposition}
\begin{proof} We show that $\precP$ and $\gr$ satisfy Properties
  \ref{pes2} and \ref{pes3} of \refToDef{pes}.  Reflexivity,
  transitivity and antisymmetry of $\precP$ follow from the
  corresponding properties of $\sqsubseteq$. As for irreflexivity and
  symmetry of $\gr$, they follow from Clause \ref{ila--esp3} of
  Definition \ref{procevent-relations} and the corresponding
  properties of inequality.  To show conflict hereditariness, suppose
  that $\procev \grr \procev'\precP \procev''$.  From Clause
  \ref{ila--esp3} of Definition \ref{procevent-relations} there are
  $\pi$, $\pi'$, $\actseq$, $\actseq'$ and $\actseq$ such that
  $\pi\neq\pi'$ and $\procev=\concat{\concat{\actseq}{\pi}}{\actseq'}$
  and $\procev'=\concat{\concat{\actseq}{\pi'}}{\actseq''}$.  From
  $\procev'\precP \procev''$ we derive that
  $\procev''=\concat{\concat{\actseq}{\pi'}}{\concat{\actseq''}{\actseq_1}}$
  for some $\actseq_1$.  Therefore $\procev \grr \procev''$, again
  from Clause \ref{ila--esp3}.
 \end{proof}

%