% !TEX root =cdgS.tex

%

\subsection{Relating Transition Sequences of Global Types and Proving
  Sequences of their ESs}\label{bullets}

 In this subsection, we relate the traces that label the
transition sequences of global types with the configurations of their
PESs.  As for n-events, we need retrieval and residual operators
for g-events. The first  operator was already introduced in
\refToDef{causal-path}, so we only need to define the second, which is
given next. 


\begin{definition}[Residual of g-events after communications]\text{~}\\[-10pt]\mylabel{def:PostPreGl}
\begin{enumerate} 
\item\mylabel{def:PostPreGl1}  The {\em  residual operator} $\bullet$ applied to a 
communication  and  a g-event is defined by:   
%\\
%\centerline{$
\[
\postG{\eqclass\comseq}{{\alpha}}=\begin{cases}
 \eqclass{\comseq'}      & \text{if }  \comseq\sim\concat{\alpha}{\comseq'}\text { and } \comseq'\neq\emptyseq\\
 \eqclass{\comseq}    & \text{if  $\participant{\alpha}
   \cap \participant{\comseq} = \emptyset$ } 
\end{cases}
   \]
%$}
\item\mylabel{def:PostPreGl2} The operator %s $\preG\cdot\cdot$ and 
$\bullet$ naturally extends to 
nonempty
traces: 
%\\
%\centerline{$
\[
\postG{\comocc}{(\concat\alpha\comseq)}=\postG{(\postG{\comocc}\alpha)}\comseq \qquad  \comseq\not=\emptyseq
\]
%$}
  \end{enumerate}
\end{definition} 

 The operator $\bullet$
gives the global event obtained by erasing the
communication,  if it occurs in head position (modulo
$\sim$) in  
the event and leaves the event unchanged if the participants of the global event and of the communication are disjoint. 
Note that the  operator
$\postG{\eqclass\comseq}{\alpha}$ is undefined whenever either 
$\eqclass\comseq=\{\alpha\}$ or  
 one of the participants of $\alpha$ occurs in $\comseq$
but its first communication is different from $\alpha$. 

\bigskip

 The following lemma %, proved in the Appendix, 
 gives some simple properties
of the retrieval and residual operators for g-events. 
The first five  statements  correspond to those of \refToLemma{prop:prePostNet}
for n-events. 
The last three  statements  
%points 
give properties 
%useful 
 that are relevant  only for the operators $\circ$ and $\bullet$.
\begin{lemma}[Properties of 
  retrieval and residual for g-events]\text{~}\\[-10pt]\mylabel{prop:prePostGl}
%Let $\Nt\stackred\alpha\Nt'$.
%The following hold: 
\begin{enumerate}
\item \mylabel{ppg1a} If $\postG{\comocc}{\alpha}$ is defined, then $\preG{(\postG{\comocc}{\alpha})}{\alpha}=\comocc$;
\item \mylabel{ppg1b} 
$\postG{(\preG{\comocc}{\alpha})}{\alpha}=\comocc$;
 \item  \mylabel{ppg4a}  If  $\comocc_1< \comocc_2$, 
then $\preG{\comocc_1}{\alpha}< \preG{\comocc_2}{\alpha}$;
\item \mylabel{ppg4b} If  $\comocc_1<\comocc_2$ and  both $\postG{\comocc_1}{\alpha}$ and
  $\postG{\comocc_2}{\alpha}$ %is %
  are 
  defined, then
  $\postG{\comocc_1}{\alpha}< \postG{\comocc_2}{\alpha}$;
 \item \mylabel{prop:prePostGl5}  If $\comocc_1\gr \comocc_2$, 
then $\preG{\comocc_1}{\alpha}\gr \preG{\comocc_2}{\alpha}$;
  \item \mylabel{ppg3} If $\comocc<\preG{\comocc'}{\alpha}$, then either $\comocc=\eqclass\alpha$ or $\postG{\comocc}{\alpha}<{\comocc'}$;
 \item \mylabel{prop:prePostGl6}  If $\participant{\alpha_1}\cap\participant{\alpha_2}=\emptyset$, then $\preG{(\preG{\comocc}{\alpha_2})}{\alpha_1}=\preG{(\preG{\comocc}{\alpha_1})}{\alpha_2}$;
 \item \mylabel{prop:prePostGl7} If $\participant{\alpha_1}\cap\participant{\alpha_2}=\emptyset$ and both $\postG{(\preG{\comocc}{\alpha_1})}{\alpha_2}$, $\postG{\comocc}{\alpha_2}$ are defined, then $\preG{(\postG{\comocc}{\alpha_2})}{\alpha_1}= \postG{(\preG{\comocc}{\alpha_1})}{\alpha_2}$.
\end{enumerate}
\end{lemma}


 The next lemma relates the retrieval and residual operator with the global types which are branches of choices. 

\begin{lemma}\label{paldf}
 The following hold: \\[-10pt]
\begin{enumerate}
\item\label{paldf1} If $\comocc\in\EGG( \G)$, then $\preG\comocc{\Comm\pp{\la}\q}\in \EGG( \gt\pp\q i I \la \G)$, where $\la=\la_k$ and $\G=\G_k$ for some $k\in I$;
\item\label{paldf2} If $\comocc\in\EGG( \gt\pp\q i I \la \G)$ and $\postG\comocc{\Comm\pp{\la_k}\q}$ is defined, then $\postG\comocc{\Comm\pp{\la_k}\q}\in\EGG( \G_k)$, where $k\in I$.
\end{enumerate}
\end{lemma}
\begin{proof}
(\ref{paldf1}) By \refToDef{eg}(\ref{eg1a}) $\comocc\in\EGG( \G)$ implies $\comocc=\ev\comseq$ for some $\comseq\in\FPaths\G$. Since $\preG\comocc{\Comm\pp{\la}\q}=\ev{\concat{\Comm\pp{\la}\q}\comseq}$ by \refToDef{causal-path} and $\concat{\Comm\pp{\la}\q}\comseq\in\FPaths{ \gt\pp\q i I \la \G}$ we conclude $\preG\comocc{\Comm\pp{\la}\q}\in \EGG( \gt\pp\q i I \la \G)$ by \refToDef{eg}(\ref{eg1a}).

(\ref{paldf2}) By \refToDef{eg}(\ref{eg1a}) $\comocc\in\EGG( \gt\pp\q i I \la \G)$ implies $\comocc=\ev\comseq$ for some $\comseq\in\FPaths{\gt\pp\q i I \la \G}$. We get $\comseq=\concat{\Comm\pp{\la_h}\q}{\comseq'}$ with $\comseq'\in\FPaths{\G_h}$ or $\comseq'=\ee$ for some $h\in I$. The hypothesis $\postG\comocc{\Comm\pp{\la_k}\q}$ defined implies either $h=k$ and $\comseq'\not=\ee$ or $\participant{\comseq'}\cap\set{\pp,\q}=\emptyset$ and $\postG\comocc{\Comm\pp{\la_k}\q}=\ev{\comseq'}$ by \refToDef{def:PostPreGl}(\ref{def:PostPreGl1}). In the first case $\comseq'\in\FPaths{\G_k}$.
In the second case $\comseq''\in\FPaths{\G_k}$ for some $\comseq''\sim\comseq'$ by definition of projection,  which prescribes the same behaviours to all participants different from $\pp,\q$,  see \refToFigure{fig:proj}. We conclude  $\postG\comocc{\Comm\pp{\la_k}\q}\in\EGG( \G_k)$ by \refToDef{eg}(\ref{eg1a}).
\end{proof}

 
The following lemma %, proved in the Appendix,  
plays the role  %is the analogue 
of  \refToLemma{epp} for n-events.

  
\begin{lemma}\label{paltr}
Let $\G\stackred\alpha \G'$.
\begin{enumerate}
\item\label{paltr1} If $\comocc\in\EGG(\G')$, then $\preG\comocc{\alpha}\in \EGG(\G)$;
\item\label{paltr2} If $\comocc\in\EGG(\G)$ and $\postG\comocc{\alpha}$ is defined, then $\postG\comocc{\alpha}\in\EGG( \G')$.
\end{enumerate}
\end{lemma}



 We show next that each trace gives rise to a sequence of
g-events,  compare with \refToDef{nec}.  %and we establish some simple properties of such sequence of g-events.  



\begin{definition}[Building sequences of g-events from traces]
%[Global events from communications]
\label{gecdef}
We define the {\em sequence of global events corresponding to a
   trace  $\comseq$} by
   %\\
%{\em global proving sequence corresponding to $\comseq$} by\\
%  \centerline{$
\[
  \gec{\comseq}=\Seq{\comocc_1;\cdots}{\comocc_n}
  \]
 % $} 
  where
   $\comocc_i=\ev{\range\comseq1i}$ for all $i$,  $1\leq i\leq n$. 
  \end{definition}
  
  We show that $\gec\cdot$ has similar properties as  $\nec\cdot$, see \refToLemma{ecn}(\ref{ecn2}).  The proof is straightforward.  
  
  \begin{lemma}\mylabel{ecg}
Let  $\gec{\comseq}=\Seq{\comocc_1;\cdots}{\comocc_n}$. 

\begin{enumerate}
\item \mylabel{ecg1} $\comm{\comocc_i}=\at\comseq{i}$  for all $i$,  $1\leq i\leq n$. 
\item \mylabel{ecg2} If $1\leq h,k\leq n$, then $\neg (\comocc_h\gr\comocc_k)$;
 \end{enumerate}
 
  \end{lemma}
  
 We may now prove the correspondence between the traces labelling
the transition sequences of a global type and the proving sequences of
its PES. Let us stress the difference between the set of traces
$\FPaths{\G}$ of a global type $\G$ as defined at page
\pageref{G-traces} and the set of traces that label the transition
sequences of $\G$, which is a larger set due to the internal Rule
\rulename{Icomm} of the LTS for global types given in \refToFigure{ltgt}.   


  
\begin{theorem}\mylabel{uf13}
 If %Let 
$\G\stackred\comseq \G'$,
then
 $\gec{\comseq}$ is a proving sequence in $ \ESG{ \G}$. 
\end{theorem}
\begin{proof}
By induction on ${\comseq}$.\\
 \emph{Base case.} 
Let $\comseq=\alpha$,  then $\gec{\alpha}=\eqclass\alpha$.   We use a further induction on 
%The proof is by induction on the reduction 
the inference of the transition  $ \G\stackred\alpha \G'$.\\
 Let $ \G=\gt\pp\q i I \la \G$, $\G'= \G_h$  and $\alpha=\Comm\pp{\la_h}\q$ for some $h\in I$. By \refToDef{eg}(\ref{eg1a}) $\eqclass{\Comm\pp{\la_h}\q}\in \EGG( \G)$.\\
Let $\G= \gt\pp\q i I \la \G$ and $\G'= \gt\pp\q i I \la {\G'}$ and $ \G_i\stackred{\alpha}\G_i'$  for all $ i \in I$ and $\participant\alpha\cap\set{\pp,\q}=\emptyset$. By induction $\eqclass{\alpha}\in\EGG( \G_i)$ for all $ i \in I$. 
By \refToLemma{paldf}(\ref{paldf1}) $\preG{\eqclass{\alpha}}{\Comm\pp{\la_i}\q}\in\EGG(\G)$ for all $ i \in I$.
By \refToDef{eg}(\ref{eg1a}) $\preG{\eqclass{\alpha}}{\Comm\pp{\la_i}\q}=\eqclass{\alpha}$, since $\participant\alpha{\cap}\set{\pp,\q}{=}\emptyset$. We conclude $\eqclass{\alpha}\in\EGG(\G)$. \\
%The proof in the case $\G= \G_1\parG\G_2$ is similar and simpler. \\
%%%%%%%%%%%%%
\emph{Inductive case.}   Let
$\comseq=\concat{\alpha}{\comseq'}$ with $\comseq'\not=\ee$.  %where
  %$\comseq'=\alpha_2\,\cdots\,\alpha_{n}$ and $n>1$.  
  From
 $\G\stackred{\comseq}\G'$ we get    $\G\stackred{\alpha}\G_0\stackred{\comseq'}\G'$ for some $\G_0$.  Let $\gec{\comseq}=\Seq{\comocc_1;\cdots}{\comocc_{n}}$ and $\gec{\comseq'}=\Seq{\comocc'_2;\cdots}{\comocc'_{n}}$.
  By induction 
    $\gec{\comseq'}$  is a proving
    sequence in $\ESG{\G_0}$. 
   By Definitions~\ref{gecdef} and~\ref{causal-path}  $\comocc_i=\preG{\comocc'_i}{\alpha}$, which implies 
    $\postG{\comocc_i}{\alpha}=\comocc'_i$ by \refToLemma{prop:prePostGl}(\ref{ppg1b}) for  all $i$,  $2\leq i\leq n$.\\
    We can show that $\comocc_1=\eqclass\alpha\in\EGG(\G)$ as in the proof of the base case. By \refToLemma{paltr}(\ref{paltr1}) $\comocc_i\in\EGG(\G)$ since $\comocc_i'\in\EGG(\G_0)$ and 
    $\postG{\comocc_i}{\alpha}=\comocc'_i$ for  all $i$,  $2\leq i\leq n$.
We  prove that   $\gec{\comseq}$  is a
proving sequence in  $\ESG{\G}$. 
Let
$\comocc<  \comocc_k$ for some $k$, $1\leq k\leq n$.  Note that this implies 
$k>1$. 
 Since $\comocc_k=\preG{\comocc'_k}{\alpha}$ by \refToLemma{prop:prePostGl}(\ref{ppg3})
    either $\comocc=\eqclass\alpha$ or $\postG{\comocc}{\alpha}<\comocc'_h$. %\\
    If $\comocc=\eqclass\alpha=\comocc_1$ we are done. Otherwise $\postG{\comocc}{\alpha}\in\EGG(\G_0)$ by \refToLemma{paldf}(\ref{paldf2}).  
 Since  $\gec{\comseq'}$  is a proving
    sequence in $\ESG{\G_0}$, there is $h<k$ such that $\postG{\comocc}{\alpha}=\comocc_h'$ and this implies $\comocc=\preG{(\postG{\comocc}{\alpha})}{\alpha}=\preG{\comocc_h'}{\alpha}=\comocc_h$  by \refToLemma{prop:prePostGl}(\ref{ppg1a}). 
\end{proof}


%\input{lemma4}
\begin{theorem}\mylabel{uf14}
  If $\Seq{\comocc_1;\cdots}{\comocc_n}$ is a proving sequence in $
  \ESG{ \G}$, then $ \G\stackred\comseq \G'$, where
  $\comseq=\concat{\concat{\comm{\comocc_1}}\cdots}{\comm{\comocc_n}}$.
\end{theorem}

\begin{proof} 
The proof is by induction on the length $n$ of the proving sequence. Let $\comm{\comocc_1}=\alpha$ and $\set{\pp,\q}=\participant\alpha$.\\
 {\it Case $n=1$.}   Since $\globev_1$ is the
first event of a proving sequence, we have $\globev_1 =\eqclass{\alpha}$. We show this case by induction on $d=\weight(\G,\pp)=\weight(\G,\q)$.\\
{\it Case $d=1$.} Let $\alpha = \Comm\pp\la\q$ and $\G=\gt\pp\q i I \la \G$ and $\la=\la_h$ for some $h\in I$. Then $\G\stackred\alpha\G_h$ by rule \rulename{Ecomm}.\\
{\it Case $d>1$.} Let  $\G= \gt\pr\ps i I \la \G$ and $\set{\pr,\ps}\cap\set{\pp,\q}=\emptyset$. By \refToDef{def:PostPreGl}(\ref{def:PostPreGl1})  $\postG{\comocc_1}{\Comm\pr{\la_i}\ps}$
is defined for all $i\in I$ since $\set{\pr,\ps}\cap\set{\pp,\q}=\emptyset$. This implies $\postG{\comocc_1}{\Comm\pr{\la_i}\ps}\in\EGG(\G_i)$  for all $i\in I$ by \refToLemma{paldf}(\ref{paldf2}).
By induction hypothesis $\G_i\stackred\alpha\G'_i$ for all $i\in I$. Then we can apply rule  \rulename{Icomm} to derive $\G\stackred\alpha\gtp\pr\ps i I \la \G$.\\
{\it Case $n>1$.}    
Let $\G\stackred\alpha\G''$
be the transition as obtained from the base case. 
We show that $\postG{\comocc_j}{\alpha}$ is defined for all $j$, $2\leq j\leq n$. 
If $\postG{\comocc_k}{\alpha}$ were undefined
for some $k$, $2\leq k\leq n$, then by
\refToDef{def:PostPreGl}(\ref{def:PostPreGl1}) either
$\comocc_k=\comocc_1$ or $\comocc_k=\eqclass\comseq$ with
$\comseq\not\sim\concat\alpha{\comseq'}$ and
$\participant\alpha\cap\participant\comseq\not=\emptyset$. In the second case
$\pro\alpha{\pp}\grr\pro\comseq{\pp}$ or $\pro\alpha{\q}\grr\pro\comseq{\q}$, which
implies $\comocc_k\grr\comocc_1$.  So both cases are impossible.  If
$\postG{\comocc_j}{\alpha}$ is defined, by
\refToLemma{paltr}(\ref{paltr2}) we get $
\postG{\comocc_j}{\alpha}\in\EGG(\G'')$ for all $j$,
$2\leq j\leq n$.\\
 We show that $\comocc'_2 ;\cdots ;\comocc'_n$ is a
proving sequence in $\ESG{\G''}$ where $\comocc'_j =
\postG{\comocc_j}{\alpha}$ for all $j$,
$2\leq j\leq n$.
By \refToLemma{prop:prePostGl}(\ref{ppg1a}) $\comocc_j =
\preG{\comocc'_j}{\alpha}$ for all $j$, $2\leq j\leq n$. 
Then by
\refToLemma{prop:prePostGl}(\ref{prop:prePostGl5}) no two
events in  the sequence $\comocc'_2 ;\cdots ;\comocc'_n$  can be in conflict. 
Let $\comocc\in\EGG(\G'')$ and $\comocc< \comocc'_h$
for some $h$, $2\leq h\leq n$.  
By  \refToLemma{paltr}(\ref{paldf1})  $\preG{\comocc}\alpha$ and $\preG{\comocc'_h}\alpha$  belong to $\EGG(\G)$. By
\refToLemma{prop:prePostGl}(\ref{ppg4a})
$\preG{\comocc}\alpha<\preG{\comocc'_h}{\alpha}$. 
By \refToLemma{prop:prePostGl}(\ref{ppg1a})
$\preG{\comocc'_h}{\alpha}=\comocc_h$.
Let $\comocc' = \preG{\comocc}\alpha$. 
Then
$\comocc' < \comocc_h$ implies, by \refToDef{provseq} 
and the fact that $\ESG{\G}$ is a PES, that there is $k<h$ 
such that $\comocc'
=\comocc_k$.  By \refToLemma{prop:prePostGl}(\ref{ppg1a})
we get $\comocc=\postG{\comocc'}{\alpha} = \postG{\comocc_k}\alpha
= \comocc'_k$.\\
Since $\comocc'_2 ;\cdots ;\comocc'_n$ is a proving
sequence in $\ESG{\G''}$, by induction
$\G''\stackred{\comseq'}\G'$  where $\comseq' = 
\concat{\concat{\comm{\comocc'_2}}\ldots}{\comm{\comocc'_n}}$.  Let
$\comseq=\concat{\concat{\comm{\comocc_1}}\ldots}{\comm{\comocc_n}}$. Since
$\comm{\comocc'_j} = \comm{\comocc_j}$ for all $j, 2\leq j\leq n$, we
    have $\comseq = \concat{\alpha}{\comseq'}$. Hence
$\G\stackred\alpha\G''\stackred{\comseq'}\G'$
is the required transition sequence. 

\end{proof}


 The last ingredient required  to prove our main theorem is
the following separation result from~\cite{BC91} (Lemma 2.8 p. 12):

%
\begin{lemma}[Separation~\cite{BC91}]
\mylabel{separation}
%
Let $S=(E,\prec, \gr)$ be a flow event structure and $\ESet, \ESet' \in \Conf{S}$
be such that $\ESet \subset \ESet'$.
Then there exist $e \in \ESet'\backslash \ESet$ such that $\ESet \cup \set{e} \in \Conf{S}$.
\end{lemma}

We may now  finally  show the
correspondence between the configurations of the  FES of a network
and the configurations of the PES  of its global type.
Let $\simeq$ denote isomorphism on domains of configurations.

%
\begin{theorem}[Isomorphism]\mylabel{iso}
 If $\derN\Nt\RG$, then $\CD{\ESN{\Nt}} \simeq \CD{\ESG{\RG}}$.
\end{theorem}
\begin{proof}
  By \refToTheorem{uf12} if $\Seq{{\netev_1};\cdots}{\netev_n}$ is a
  proving sequence of $\ESN{\Nt}$, then $\Nt\stackred\comseq\Nt'$
  where $\comseq=\comm{\netev_1}\cdots\comm{\netev_n}$. By  applying iteratively  Subject
  Reduction (\refToTheorem{sr}) $\G\stackred\comseq\G'$ and
  $\derN{\Nt'}{\RG'}$.  By \refToTheorem{uf13} %(\ref{uf131}) 
  $\gec\comseq$ is a
  proving sequence of $\ESG{\RG}$.
  
  By \refToTheorem{uf14} if $\Seq{{\globev_1};\cdots}{\globev_n}$ is a
  proving sequence of $\ESG{\RG}$, then $\G\stackred\comseq\G'$ where
  $\comseq=\comm{\globev_1}\cdots\comm{\globev_n}$. By  applying iteratively  Session
  Fidelity (\refToTheorem{sf}) $\Nt\stackred\comseq\Nt'$ and
  $\derN{\Nt'}{\RG'}$.  By \refToTheorem{uf10} %(\ref{uf101}) 
  $\nec\comseq$ is a
  proving sequence of $\ESN{\Nt}$.
  
  
  
Therefore we have a bijection between $\CD{\ESN{\Nt}}$ and
$\CD{\ESG{\RG}}$, given by $\nec\comseq \leftrightarrow \gec\comseq$
for any $\comseq$ generated by the (bisimilar) LTSs of $\Nt$ and $\RG$.
 

   We show now that this bijection preserves inclusion of
  configurations.  By
  \refToLemma{separation} it is enough to prove that if 
  $\Seq{{\netev_1};\cdots}{\netev_n} \in
  \Conf{\ESN{\Nt}}$ is mapped to $\Seq{{\globev_1};\cdots}{\globev_n} \in \Conf{\ESG{\RG}}$, then
  $\Seq{\Seq{{\netev_1};\cdots}{\netev_n}}\netev \in
  \Conf{\ESN{\Nt}}$  iff  %and only if
  $\Seq{\Seq{{\globev_1};\cdots}{\globev_n}}\globev \in
  \Conf{\ESG{\RG}}$,  where
  $\Seq{\Seq{{\globev_1};\cdots}{\globev_n}}\globev$ is the image of
$\Seq{\Seq{{\netev_1};\cdots}{\netev_n}}\netev$ under the bijection. I.e. let $\nec{\concat{\comseq}{\alpha}} =
\Seq{{\netev_1};\cdots}{\netev_n;\netev}$ and $\gec{\concat{\comseq}{\alpha}} =
\Seq{{\globev_1};\cdots}{\globev_n;\globev}$. This implies $\comseq=\comm{\netev_1}\cdots\comm{\netev_n}=
\comm{\globev_1}\cdots\comm{\globev_n}$ and $\alpha=\comm\netev=\comm\globev$ by  Lemmas~\ref{ecn} and~\ref{ecg}. 

%By \refToLemma{snb} $\GE(\Nt)$ is balanced. 
By \refToTheorem{uf12}, if
$\Seq{{\netev_1};\cdots}{\netev_n;\netev}$ is a proving sequence of
$\ESN{\Nt}$, then $\Nt\stackred\comseq\Nt_0\stackred\alpha\Nt'$.
By  applying iteratively  Subject Reduction (\refToTheorem{sr})
$\G\stackred\comseq\G_0\stackred\alpha\G'$ and $\derN{\Nt'}{\RG'}$.
By \refToTheorem{uf13} %(\ref{uf131}) 
$\gec{\concat\comseq\alpha}$ is a proving
sequence of $\ESG{\RG}$.

By \refToTheorem{uf14}, if
  $\Seq{{\globev_1};\cdots}{\globev_n;\globev}$ is a proving sequence
  of $\ESG{\RG}$, then $\G\stackred\comseq\G_0\stackred\alpha\G'$.
By  applying iteratively  Session Fidelity (\refToTheorem{sf})
  $\Nt\stackred\comseq\Nt_0\stackred\alpha\Nt'$ and
  $\derN{\Nt'}{\RG'}$.   By \refToTheorem{uf10} %(\ref{uf101})
  $\nec{\concat\comseq\alpha}$ is a proving sequence of
  $\ESN{\Nt}$. 
  \end{proof}

