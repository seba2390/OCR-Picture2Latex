% !TEX root =cdgS.tex


\subsection{Further Properties}
\mylabel{subsec:further-props}

In this subsection, we first prove two properties of the
conflict relation in network ESs: non disjoint n-events are always in
conflict, and conflict induced by Clause (\ref{c22}) of
\refToDef{netevent-relations} is semantically inherited.  
 We then discuss the relationship between causal sets and prime
configurations and prove two further properties of causal sets, which
are shared with prime configurations: finiteness, and the existence 
of a causal set for each event in a configuration.
Finally,  observing that the FES of a network may be viewed as the product
of the PESs of its processes, we proceed to prove a classical property
for ES products, namely that their projections on their components
preserve configurations.  To this end, we define a projection function
from n-events to participants, yielding p-events, and we show that
configurations of a network ES project down to configurations of the 
PESs of its processes. 

\bigskip

Let us start with the conflict properties. By definition, two
n-events intersect each other if and only if they share a located
event $\locev{\pp}{\procev}$. Otherwise, the two n-events are
disjoint. 
 Note that if 
$\locev{\pp}{\procev}\in(\netev\cap\netev')$, then
$\loc{\netev} = \loc{\netev'} = \set{\pp, \q}$, where $\q =\ptone{\act{\procev}}$.
The next proposition establishes that two  distinct  intersecting n-events
in $\DE$ are in conflict. 
\begin{proposition}[Sharing of located events implies conflict]\mylabel{prop:conf}
  If $\netev, \netev' \in  \DE $ and $\netev\neq \netev'$ and
  $(\netev\cap\netev') \neq \emptyset$, then $\netev\gr \netev'$.
\end{proposition} 
\begin{proof}
  Let  $\locev{\pp}{\procev}\in(\netev\cap\netev')$ and
  $\loc{\netev}=\loc{\netev'} =\set{\pp,\q}$. 
%$\q = \participant{\act{\procev}}$.  
  Then there must exist $\procev_0, \procev'_0$ such that
  $\locev{\q}{\procev_0}\in \netev$ and $\locev{\q}{\procev'_0}\in
  \netev'$.   From
  $\dualevS{\locev{\pp}{\procev}}{\locev{\q}{\procev_0}}$ and
  $\dualevS{\locev{\pp}{\procev}}{\locev{\q}{\procev'_0}}$ it follows that
  $\projb{\procev_0}{\pp} = \projb{\procev'_0}{\pp}$. This, in conjunction
  with the fact that $\ptone{\act{\procev_0}} =
  \ptone{\act{\procev'_0}} = \pp$, implies that  
  neither $\procev_0 < \procev'_0$ nor $\procev'_0 < \procev_0$.
  Thus $\procev_0 \grr \procev'_0$ and therefore $\netev \grr \netev'$
  by \refToDef{netevent-relations}.
\end{proof}



Although conflict is not hereditary in FESs, we prove that a conflict
due to incompatible mutual projections (i.e., a conflict derived by
Clause (\ref{c22}) of \refToDef{netevent-relations}) is semantically
inherited.  Let $\tr\bm n$ denote the prefix of length $n$ of $\bm$.

\begin{proposition}[Semantic conflict hereditariness]
\mylabel{sem-conf}
Let $\locev{\pp}{\procev}\in \netev$ and $\locev{\q}{\procev'}\in\netev'$ with $\pp\not=\q$.
Let $n=min \set{\eh{\projb\procev\q},\eh{\projb{\procev'}\pp}}$. 
If $\neg(\dualev{\tr{(\projb\procev\q)}n}{\tr{(\projb{\procev'}\pp)}n})$,
then there exists no configuration $\ESet$ such that $\netev,
\netev'\in \ESet$. 
\end{proposition}
\begin{proof}
  Suppose ad absurdum that $\ESet$ is a configuration such that
  $\netev, \netev'\in \ESet$. If $\cardin{\projb\procev\q} =
  \cardin{\projb{\procev'}\pp}$ then $\netev \grr\netev '$ by
  \refToDef{netevent-relations}(\ref{c22}) and we reach immediately a
  contradiction.  So, assume $\cardin{\projb\procev\q} >
  \cardin{\projb{\procev'}\pp} = n$. This means that $\cardin{\procev}
  > 1$ and thus there exists a non-empty causal set $E_\netev$ of
  $\netev$ such that $E_\netev \subseteq\ESet$.  Let $\procev_0 <
  \procev$ be such that $\cardin{\projb{\procev_0}\q}
  =\cardin{\projb{\procev'}\pp} =n$.  By definition of causal set,
  there exists $\netev_0 \in E_\netev$ such that
  $\locev{\pp}{\procev_0}\in \netev_0$. By
  \refToDef{netevent-relations}(\ref{c22}) we have then $\netev_0\grr
  \netev'$, contradicting the fact that $\ESet$ is conflict-free. 
\end{proof}


%%%%%%%%%% INIZIO PEZZO PRESO DA ESPN1 %%%


We prove now two further properties of causal sets. 
For the reader familiar with ESs, the notion of causal set may be
reminiscent of that of \emph{prime configuration}~\cite{Winskel80},
which similarly consists of a complete set of causes for a given
event\footnote{In PESs, the prime configuration associated with an
  event is unique, while it is not unique in FESs and more generally
  in Stable ESs, just like a causal set.}. However, there are some
important differences: the first is that a causal set does not include
the event it causes, unlike a prime configuration. The second is that
a causal set only contains direct causes of an event, and thus it is
not downward-closed up to conflicts, as opposed to a prime
configuration. The last difference is that, while a prime
configuration uniquely identifies its caused event, a causal set may
cause different events, as shown in \refToExample{ex:causal-sets}.


A common feature of prime configurations and causal sets is that they
are both finite. For causal sets, this is implied by minimality
together with Clause (\ref{cs2}) of \refToDef{cs}, as shown by the
following lemma. 
\\
%

\begin{lemma}
\label{cs-prop}
 Let $\netev\in \EvSet \subseteq \DE$. If $E$ is a causal set of $\netev$ in $\EvSet$,
  then $E$ is finite. 
\end{lemma} 
\begin{proof}
  Suppose
  $\netev=\set{\locev\pp\procev,\locev\q{\procev'}}$.  We show that
  $\cardin{E} \leq \cardin{\procev} + \cardin{\procev'} - 2$, where
  $\cardin{E}$ is the cardinality of $E$.  By Condition (\ref{cs2}) of \refToDef{cs}, for
  each $\procev_0 < \procev$ and $\procev'_0 < \procev'$ there must be
  $\netev_0, \netev'_0\in E$ such that $\locev\pp\procev_0 \in
  \netev_0$ and $\locev\q\procev'_0 \in \netev'_0$. Note that
  $\netev_0$ and $\netev'_0$ could possibly coincide. Moreover, there
  cannot be $\netev'\in E$ such that $\locev\pp\procev_0
  \in \netev' \neq \netev_0$ or $\locev\q\procev'_0 \in \netev'
  \neq \netev'_0$, since this would contradict the
  minimality of
  $E$  (and also its conflict-freeness, since by
  \refToProp{prop:conf} we would have $\netev' \grr \netev_0$). 
Hence the number of events in $E$ is at most $(\cardin{\procev}
  -1) + (\cardin{\procev'} - 1)$.
\end{proof}



 A key property of causal sets, which is again shared with prime
configurations, is that each configuration includes a unique causal
set for each n-event in the configuration.



\begin{lemma}\label{csl}
  If $\ESet$ is a configuration of $\ESN{\Nt}$ and $\netev\in\ESet$, then there is a
 unique causal set $E$ of $\netev$ such that $E\subseteq\ESet$.
\end{lemma}
\begin{proof} 
  By \refToDef{def:narrowing}, if $\netev\in \GE(\Nt)$, then $\netev$
  has at least one causal set included in $\GE(\Nt)$.  Let
  $E'=\set{\netev'\in\ESet\mid \netev'\prec\netev}$. By
  \refToDef{configF}, $E'\cup \set{\netev}$ is conflict-free.
  Moreover, if $\locev{\pp}{\procev}\in \netev$ and $\procev' <
  \procev$, then by \refToProp{prop:conf} there is at most one
 $\netev'' \in E'$ such that $\locev{\pp}{\procev'}\in \netev''$. 
  Therefore, $E'\subseteq E$ for some causal set $E$ of $\netev$ by
  \refToDef{cs}.  We show that $E\subseteq E'$.  Assume ad absurdum
  that $\netev_0\in E\backslash E'$.  By definition of causal set,
  $\netev_0\precN\netev$. By definition of $E'$, $\netev_0\not\in E'$
  implies $\netev_0\not\in \ESet$.  By \refToDef{configF} this implies
  $\netev_0\gr\netev_1\prec\netev$ for some $\netev_1\in\ESet$. Then
  $\netev_1\in E'$ by definition of $E'$, and thus $\netev_1\in E$.
  Hence $\netev_0,\netev_1\in E$ and $\netev_0\gr\netev_1$,
  contradicting \refToDef{cs}.
\end{proof}



In the remainder of this section we show that projections of n-event
configurations give p-event configurations. We start by formalising
the projection function of n-events to p-events and showing that it is
downward surjective.

\begin{definition}[Projection of n-events to p-events]
\label{def:proj-network-n-event}
\[
\projnet{\pp}{\netev}=\begin{cases}
\procev     & \text{if } \locev{\pp}{\procev}\in\netev, \\
 undefined   & \text{otherwise}.
\end{cases}
\]
The projection function $\projnet{\pp}{\cdot}$ is extended to sets of
n-events in the obvious way:
%\\
%\centerline{$
\[
\projnet{\pp}{X}  = \set{ \procev \mid 
    \exists\netev \in X  \, .\, \projnet{\pp}{\netev} = \procev}
    \]
%$}  
\end{definition}

\begin{example}\mylabel{ex:narrowing-ila}
  Let $\set{\netev_1,\netev_2,\netev_3}$ be the configuration defined in \refToExample{ex1}. We get
%  \\
%  \centerline{$
\[
\projnet{\q}{\set{\netev_1,\netev_2,\netev_3}} =
    \set{\rcvL\pp{\la_1},
      \concat{\rcvL\pp{\la_1}}{\sendL\pr{\la_2}}}
      \]
%$}
\end{example}

\begin{example}
Let  $\Nt=\pP{\pp}{\rcvL{\pr}\la;\rcvL{\q}{\la'}} \parN \pP{\q}{\sendL{\pp}{\la'}}$.  Then
%\\
%\centerline{$
\[ 
\GE(\Nt) =\nr{\set{\set{\locev\pp{\concat{\rcvL{\pr}\la}{\rcvL{\q}{\la'}}},\locev\q{\sendL{\pp}{\la'}}}}}=\emptyset
\]
%$}
Note that if we did not apply narrowing the set of events of 
$\ESN{\Nt}$ 
%$\Nt$
would be the singleton
$\set{\locev\pp{\concat{\rcvL{\pr}\la}{\rcvL{\q}{\la'}}},\locev\q{\sendL{\pp}{\la'}}}$,
which would also be a configuration $\ESet$ of $\ESN{\Nt}$. However,
$\projnet{\pp}{\netev}=\set{\concat{\rcvL{\pr}\la}{\rcvL{\q}{\la'}}}$
would not be configuration in $\ES(\PP)$, since it would contain the
event $\concat{\rcvL{\pr}\la}{\rcvL{\q}{\la'}}$ without its cause
$\rcvL{\pr}\la$. 
\end{example}



Narrowing ensures that each projection of the set of n-events of a
network  FES  on one of its participants is downward
surjective (according to~\refToDef{down-onto}):


\begin{lemma}[Downward surjectivity of projections]
\label{prop:down-onto1}
Let $\ESN{\Nt} = (\GE(\Nt), \precN_\Nt , \grr_\Nt)$ and $\ESP{\PP} =
(\ES(\PP), \precP_\PP , \gr_\PP)$ and $\pP{\pp}{\PP}\in\Nt$. Then the
partial function $\projnetfun{\pp}:\GE(\Nt) \rightarrow_* \ES(\PP)$ is
downward surjective.
\end{lemma}
\begin{proof}
  As mentioned already in~\refToSection{sec:eventStr}, any PES $S =
  (E, \leq, \gr)$ may be viewed as a FES, with $\prec$ given by $<$
  (the strict ordering underlying $\leq$). Let $\procev \in \ES(\PP)$
  and $\netev \in \GE(\Nt)$.  Then the property we need to show is:
\[\procev <_\PP \projnet{\pp}{\netev} \implies \exists \netev' \in
\GE(\Nt)~.~\procev = \projnet{\pp}{\netev'} \] Note that $\procev
<_\PP \projnet{\pp}{\netev}$ implies $\projnet{\pp}{\netev} =
\concat{\procev}{\procev'}$ for some $\procev'$. Recall that $\GE(\Nt)
= \nr{\DE(\Nt)}$, where $\nr{\cdot}$ is the narrowing function
(\refToDef{def:narrowing}).\\  By definition of narrowing,
$\occ{\locev{\pp}{\concat{\procev}{\procev'}}}{\GE(\Nt)}$ implies 
that there is $E\subseteq\GE(\Nt)$ such that $E$ is a causal set of
$\netev$ in $\GE(\Nt)$.  Therefore
$\locev{\pp}{\concat{\procev}{\procev'}}\in\netev$
implies $\occ{\locev{\pp}{\procev}}{E}$ and so
$\occ{\locev{\pp}{\procev}}{\GE(\Nt)}$, which is what we wanted to show.
\end{proof}

\begin{theorem}[Projection preserves configurations]
\mylabel{config-preservation}
  If $\pP{\pp}{\PP}\in\Nt$, then $\ESet \in \Conf{\ESN{\Nt}}$ implies
  $\projnet{\pp}{\ESet} \in \Conf{\ESP{\PP}}$.
\end{theorem}

\begin{proof}  Clearly, $\projnet{\pp}{\ESet}$ is
  conflict-free. We show that it is also downward-closed. 
 If $\netev\in\ESet$, by \refToLemma{csl} there is a
  causal set $E$ of $\netev$ such that $E \subseteq\ESet$. If
  $\locev\pp\procev\in\netev$ and $\procev'<\procev$, by \refToDef{cs}
  there is $\netev'\in E$ such that
  $\locev\pp{\procev'}\in\netev'$. We conclude that $\netev'\in\ESet$,
   and therefore $\procev' \in \projnet{\pp}{\ESet}$. 
\end{proof}

 The reader may wonder why our ES semantics for sessions is not
cast in categorical terms, like classical ES semantics for process
calculi~\cite{Win80, CZ97}, where process constructions arise as
categorical constructions (e.g., parallel composition arises
as a categorical product).  In fact, a categorical formulation of our
semantics would not be possible, due to our two-level syntax for
processes and networks, which does not allow networks to be further
composed in parallel. However, it should be clear that our
construction of a network FES from the process PESs of its
components is a form of parallel composition, and the properties
expressed by \refToLemma{prop:down-onto1} and
\refToTheorem{config-preservation} give some evidence that this
construction enjoys the properties usually required for a categorical
product of ESs.  



