% !TEX root =cdgS.tex

\section{Introduction}
\label{sec:intro}

Session types were proposed in the mid-nineties~\cite{THK94,HVK98}, as
a tool for specifying and analysing web services and communication
protocols. They were first introduced in a variant of the
$\pi$-calculus to describe binary interactions between processes.
Such binary interactions may often be viewed as client-server 
protocols. 
%
Subsequently, session types were extended to {\em multiparty
  sessions}~\cite{CHY08,CHY16}, where several participants may
interact with each other.  A multiparty session is an interaction
among peers, and there is no need to distinguish one of the
participants as representing the server.  All one needs is an abstract
specification of the protocol that guides the interaction.  This is
called the \emph{global type} of the session. The global type
describes the behaviour of the whole session, as opposed to the local
types that describe the behaviours of single participants. In a
multiparty session, local types may be retrieved as projections from
the global type.
%

Typical safety properties ensured by session types are
\emph{communication safety} (absence of communication errors),
\emph{session fidelity} (agreement with the protocol) and
\emph{deadlock-freedom}~\cite{CHY16}.  When dealing with multiparty
sessions, the type system is often enhanced so as to guarantee also
the liveness property known as \emph{progress} (no participant gets
stuck)~\cite{H2016}.
\\
Some simple examples of sessions not satisfying the above properties
are: 1) a sender emitting a message while the receiver expects a
different message (communication error); 2) two participants both
waiting to receive a message from the other one (deadlock due to a
protocol violation); 3) a three-party session where the first
participant waits to receive a message from the second participant,
which keeps interacting forever with the third participant
(starvation, although the session is not deadlocked).


What makes session types particularly attractive is that they offer
several advantages at once: 1) static safety guarantees, 2) automatic
check of protocol implementation correctness, based on local types,
and 3) a strong connection with linear
logics~\cite{CP10,TCP11,Wadler14,PCPT14,CPT16}, and with concurrency
models such as communicating automata~\cite{DY12}, graphical
choreographies~\cite{LTY15,TuostoG18} and message-sequence
charts~\cite{CHY16}.


In this paper we further investigate the relationship between
multiparty session types and concurrency models, by focussing on Event
Structures~\cite{Win88}.  We consider a standard multiparty session
calculus where sessions are described as networks of sequential
processes~\cite{DY12}.  Each process implements a participant in the
session.  We propose an interpretation of such networks as \emph{Flow
  Event Structures} (FESs)~\cite{BC88a,BC94} (a subclass of Winskel's
Stable Event Structures~\cite{Win88}), which allows concurrency
between session communications to be explicitly represented. We then
introduce global types for these networks, and define an
interpretation of them as \emph{Prime Event Structures}
(PESs)~\cite{Win80,NPW81}. Since the syntax of global types does not
allow all the concurrency among communications to be expressed, the
events of the associated PES need to be defined as equivalence classes
of communication sequences up to \emph{permutation equivalence}. We
show that when a network is typable by a global type, the FES
semantics of the former is equivalent, in a precise technical sense,
to the PES semantics of the latter.  In a companion
paper~\cite{CDG21}, we investigated a similar Event Structure
semantics for a session calculus with asynchronous communication,
which led to a quite different treatment as it made use of a new
notion of asynchronous global type. A detailed comparison
with~\cite{CDG21} will be given in \refToSection{sec:related}.

 This paper is an expanded and amended version
of~\cite{CDG-LNCS19}. The main novelty is that we use a coinductive
definition for processes and global types, which simplifies several
definitions and proofs, and a more stringent definition for network
events. This definition relies on the new notion of causal set, 
which is crucial for the correctness of our ES semantics.
% (as will be explained in more detail in \refToSection{sec:related}). 
% As a consequence, also the definition of conflict on events and some
% examples have been modified
Finally, the present paper includes all proofs
of results, some of which require ingenuity. 

The paper is organised as follows. Section~\ref{sec:calculus}
introduces our multiparty session calculus.  In
Section~\ref{sec:eventStr} we recall the definitions of PESs and FESs,
which will be used to interpret processes
(Section~\ref{sec:process-ES}) and networks
(Section~\ref{sec:netS-ES}), respectively.  PESs are also used to
interpret global types (Section~\ref{sec:events}), which are defined
in Section~\ref{sec:types}. In Section~\ref{sec:results} we prove the
equivalence between the FES semantics of a network and the PES
semantics of its global type.  Section~\ref{sec:related} discusses
related work %in some detail %at length,
and sketches directions for future work.  The Appendix contains some technical proofs. 






 
 

