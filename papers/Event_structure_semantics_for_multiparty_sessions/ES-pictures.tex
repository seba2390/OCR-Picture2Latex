% !TEX root =cdgS.tex



%\afterpage{
\begin{figure}[!h]

\begin{center}

\bigskip

$\Nt = \pP\pp{\q!\la_1 ; \pr!\la\oplus \q!\la_2 ;
  \pr!\la} \parN\pP\q{\pp?\la_1 ; \ps!\la' +
  \pp?\la_2  ; \ps!\la' }\parN\pP\pr{\pp?\la  ; \ps!\la'' } \parN\pP\ps{\q?\la'  ; \pr?\la'' }  $


%%%%%%%%%%%%%%%%%%%%%%%%%%%%%%%%%%%%%%%%%%%%%%


\setlength{\unitlength}{1mm}
\begin{picture}(100,70)
                          

\put(8,60){{\small $\netev_1 = \set{\locev{\pp}{\,\q!\la_1}, \locev{\q}{\,\pp?\la_1}}$}} 
\put(5,35){{\small $\netev''_1 =\set{\locev{\q}{\,\pp?\la_1\cdot\ps!\la'},
        \locev{\ps}{\,\q?\la'}}$}} 
\put(29,10){{\small $\netev =\set{\locev{\pr}{\pp?\la\cdot \ps! \la''}, \locev{\ps}{\q?\la'\cdot\pr?\la''}}$}}
\put(62,60){{\small $\netev_2 =\set{\locev{\pp}{\,\q!\la_2}, \locev{\q}{\,\pp?\la_2}}$}}
\put(56,35){{\small $\netev''_2 =\set{\locev{\q}{\,\pp?\la_2\cdot\ps!\la'},
        \locev{\ps}{\,\q?\la'}}$}} 
\put(-15,25) {{\small $\netev'_1 =\set{\locev{\pp}{\,\q!\la_1\cdot\pr!\la},
      \locev{\pr}{\,\pp?\la}}$}}
\put(80,25) {{\small $\netev'_2 =\set{\locev{\pp}{\,\q!\la_2{\cdot}\pr!\la}\,, \locev{\pr}{\,\pp?\la}}$}}
\put(50,47.5){{$\gr$}}

\thicklines
\linethickness{0.3mm}
\put(32,57){\vector(0,-1){17}}
\put(72,57){\vector(0,-1){17}}
%
\multiput(51.5,64)(0,-1){13}{\bf{$\cdot$}}
\multiput(51.5,43)(0,-1){28}{\bf{$\cdot$}}
\put(33,31){\vector(1,-2){7}}
\put(71,31){\vector(-1,-2){7}}
\put(21,57){\vector(-1,-1){25}}
\put(83,57){\vector(1,-1){25}}
\put(13,21){\vector(2,-1){12}}
\put(93,21){\vector(-2,-1){12}}

\end{picture}
\end{center}

\caption{FES of the network $\Nt$.}
\mylabel{fig:network-FES}
\end{figure}



%%%%%% DRAWING of type PES %%%%%%%%%%%


\begin{figure}[!h]

\begin{center}

\bigskip

$\G = 
\gtCom\pp\q{}:(\Seq{\la_1}{\Seq{\Seq{\gtCom\pp\pr{\la}}{\gtCom\q\ps{\la'}}}}{\gtCom\pr\ps{\la''}}~\GlSyB~
\Seq{\la_2}{\Seq{\Seq{\gtCom\pp\pr{\la}}{\gtCom\q\ps{\la'}}}}{\gtCom\pr\ps{\la''}})$


\setlength{\unitlength}{1mm}
\begin{picture}(100,70)
                          

\put(9,60){{\small $\globev_1 = \eqclass{\pp\q\la_1}$}}   
\put(21,35){{\small $\globev''_1 = \eqclass{\pp\q\la_1 \cdot \q\ps\la'}$}}
\put(-2,10){{\small $\globev = \eqclass{\pp\q\la_1\cdot \pp\pr\la \cdot
\q\ps\la' \cdot \pr\ps\la''}$}}
\put(65,10){{\small $\globev' = \eqclass{\pp\q\la_2\cdot \pp\pr\la \cdot
\q\ps\la' \cdot \pr\ps\la''}$}}
\put(75,60){{\small $ \globev_2 = \eqclass{\pp\q\la_2}$}}
\put(-12,35){{\small $\globev'_1 = \eqclass{\pp\q\la_1 \cdot
      \pp\pr\la}$}}
\put(56,35){{\small $\globev''_2 = \eqclass{\pp\q\la_2 \cdot \q\ps\la'}$}}
\put(90,35){{\small $\globev'_2 = \eqclass{\pp\q\la_2\cdot \pp\pr\la}$}}
\put(50,60){{$\gr$}}


\thicklines
\linethickness{0.3mm}
\put(20,57){\vector(1,-1){15}}
\put(35,31){\vector(-1,-1){15}}
\put(1,31){\vector(1,-1){15}}
\put(102,31){\vector(-1,-1){15}}
\put(68,31){\vector(1,-1){15}}
%
\multiput(30,60)(1,0){16}{\bf{$\cdot$}}
\multiput(56,60)(1,0){16}{\bf{$\cdot$}}
%
\put(15,57){\vector(-1,-1){15}}
\put(87,57){\vector(1,-1){15}}
\put(83,57){\vector(-1,-1){15}}
\end{picture}
\end{center}

\caption{PES of the type $\G$.}
\mylabel{fig:type-PES}
\end{figure}

%} endAfterPage

We conclude this section with two pictures that summarise the
features of our ES semantics and illustrate the difference between the
FES of a network and the PES of its type.  In general these two ESs
are not isomorphic, unless the FES of the network is itself a PES.

Consider the network FES pictured in \refToFigure{fig:network-FES},
where the arrows represent the flow relation and all the n-events on
the left of the dotted line are in conflict with all the n-events on
the right of the line. In particular, notice that the conflicts
between n-events with a common location are deduced by Clause
(\ref{c21}) of \refToDef{netevent-relations}, while the conflicts
between n-events with disjoint sets of locations, such as $\netev'_1$
and $\netev''_2$, are deduced by Clause (\ref{c22}) of
\refToDef{netevent-relations}. Observe also that the n-event $\netev$
has two different causal sets in $\GE(\Nt)$, namely $\set{\netev'_1,
  \netev''_1}$ and $\set{\netev'_2, \netev''_2}$. The reader familiar
with ESs will have noticed that there are also two prime
configurations\footnote{A prime configuration is a
  configuration with a unique maximal element, its \emph{culminating}
  event.} whose maximal element is $\netev$, namely
$\set{\netev_1,\netev'_1, \netev''_1, \netev}$ and
$\set{\netev_2,\netev'_2, \netev''_2, \netev}$.  It is easy to see
that the network $\Nt$ can be typed with the global type 
shown in \refToFigure{fig:type-PES}. 
% $\G =
% \gtCom\pp\q{}:(\Seq{\la_1}{\Seq{\Seq{\gtCom\pp\pr{\la}}{\gtCom\q\ps{\la'}}}}{\gtCom\pr\ps{\la''}}~\GlSyB~
% \Seq{\la_2}{\Seq{\Seq{\gtCom\pp\pr{\la}}{\gtCom\q\ps{\la'}}}}{\gtCom\pr\ps{\la''}})$.

Consider now the PES of the type $\G$ pictured in
\refToFigure{fig:type-PES}, where the arrows represent the covering
relation of the partial order of causality and inherited conflicts are
not shown.  Note that while the FES of $\Nt$ has a unique maximal
n-event $\netev$, the PES of its type $\G$ has two maximal g-events
$\globev$ and $\globev'$. This is because an n-event only records the
computations that occurred at its locations, while a g-event records
the global computation and keeps a record of each choice, including
those involving locations that are disjoint from those of its last
communication.  Indeed, g-events correspond exactly to prime
configurations.  

Note that the FES of a network may be easily recovered from the
PES of its global type by using the following function
$\gn{\cdot}$ that maps g-events to n-events:
%\\
%\centerline{$
\[
\gn{\globev} = \set{\locev{\pp}{\projS{\comseq}\pp},
  \locev{\q}{\projS{\comseq}\q}} \quad \mbox{if }
\globev = \eqclass{\comseq}\mbox{ with}~~\partcomm{{\sf cm}(\globev)} = \set{\pp, \q}
\]
%$}

On the other hand, the inverse construction is not as direct. First of
all, an n-event in the network FES may give rise to several g-events
in the type PES, as shown by the n-event $\netev$ in
\refToFigure{fig:network-FES}, which gives rise to the pair of
g-events $\globev$ and $\globev'$ in
\refToFigure{fig:type-PES}. Moreover, the local information contained
in an n-event is not sufficient to reconstruct the corresponding
g-events: for each n-event, we need to consider all the prime
configurations that culminate with that event, and then map each of
these configurations to a g-event. Hence, we need a function ${\sf
  ng}(\cdot)$ that maps n-events to sets of prime configurations of
the FES, and then maps each such configuration to a g-event. We will
not explicitly define this function here, since we miss another
important ingredient to compare the FES of a network and the PES of
its type, namely a structural characterisation of the FESs that
represent typable networks.  Indeed, if we started from the FES of a
non typable network, this construction would not be correct. Consider
for instance the network $\Nt'$ obtained from $\Nt$ 
by
omitting the output $\pr! \la$ from the second branch of the process
of $\pp$. 
Then
the FES of $\Nt'$ would not contain the n-event  $\netev'_2$ and the
event $\netev$ would have the unique causal set
$\set{\netev'_1,\netev''_1}$, and the unique prime configuration
culminating with $\netev$ would be
$\set{\netev_1,\netev'_1,\netev''_1, \netev}$.  Then our construction would
give a PES that differs from that of type $\G$ only for the absence of
the g-events $\globev'_2$ and $\globev'$. However, the network $\Nt'$
is not typable and thus we would expect the construction to fail.
Note that in the FES of $\Nt'$, the n-event  $\netev''_2$
is a cause of
$\netev$ but does not belong to any causal set of $\netev$. Thus a
possible well-formedness property to require for FESs to be images of
a typable network would be that each cause of each n-event belong to
some causal set of that event. However, this would still not be enough to
exclude the FES of the non typable network $\Nt''$ obtained from $\Nt'$ 
by omitting the
output $\ps! \la'$ from the second branch of the process of $\q$. 


\label{wf-discussion}

To conclude, 
%in the absence of such characterisation, 
in the absence of a semantic counterpart for the well-formedness
properties of global types,
which eludes us for the time being, we will follow
another approach here, namely we will compare the FESs of networks and
the PESs of their types at a more operational level, by looking at their
configuration domains and by relating their configurations to the
transition sequences of the underlying networks and types.
