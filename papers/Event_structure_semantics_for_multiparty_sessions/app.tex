% !TEX root =cdgS.tex

\section{}
This Appendix contains the proofs of Lemmas~\ref{pf},  \ref{keysr}, \ref{prop:prePostNet},  \ref{epp}, \ref{prop:prePostGl}, and \ref{paltr}.
\begin{lemmaa}{\ref{pf}}{If $\GP$ is bounded, then $\proj\GP\pr$ is a partial function for all $\pr$.  }\end{lemmaa}
\begin{proof}
  We redefine the projection   $\downarrow_\pr$    as the
  largest relation between global types and processes such that $\prR\G\PP\pr$
  implies: %\bmc commentato per compilare \emc  \bcompa\ Fatto\ecompa
\begin{enumerate}[label=\roman*)]%[i)]
\item if $\pr\not\in  \participant{\G}$, then $\PP=\inact$;
\item if $\G= \gt\pr\pp i I \la \G $, then $\PP= \oupP\q{i}{I}{\M}{\PP_i}$ and $\prR{\G_i}{\PP_i}\pr$ for all $i\in I$;
\item if $\G= \gt\pp\pr i I \la \G $, then $\PP= \inpP\pp{i}{I}{\M}{\PP_i}$ and $\prR{\G_i}{ \PP_i}\pr$ for all $i\in I$;
\item if $\G= \gt{\pp}\q i I \la \G $ and
  $\pr\not\in\set{\pp,\q}$ and $\pr\in   \participant{\G_i}$, then
  $\prR{\G_i}{\PP}\pr$ for all $i\in I$. 
\end{enumerate}
The equality $\mathcal E$ of processes is the largest symmetric binary relation   $\RR$   
%$\mathcal E$
on processes such that 
%$\iR\PP\Q{\mathcal E}$ implies:
$\iR\PP\Q{  \RR  }$ implies: %\bmc commentato per compilare \emc  \bcompa\ Fatto\ecompa
\begin{enumerate}[label=(\alph*)]%[(a)]
\item\label{cbf} if $\PP=\oupP\pp{i}{I}{\M}{\PP_i}$ , then $\Q=\oupP\pp{i}{I}{\M}{\Q_i}$ and $\iR{\PP_i}{\Q_i}{  \RR  }$ for all $i\in I$;
\item\label{caf}  if $\PP=\inpP\pp{i}{I}{\M}{\PP_i}$ , then $\Q=\inpP\pp{i}{I}{\M}{\Q_i}$ and $\iR{\PP_i}{\Q_i}{  \RR  }$ for all $i\in I$.
\end{enumerate}
It is then enough to show that the relation
$\RR_\pr  =\set{(\PP,\Q)\mid  \exists\, \G
  \, . \ \prR\G\PP\pr\text { and } \prR\G\Q\pr}$ 
%implies $\iR\PP\Q\mathcal E$,  %is a bisimulation, 
%i.e. $\RR$  
satisfies Clauses~\ref{cbf} and~\ref{caf}   (with $\RR$ replaced by
$\RR_\pr$),   since this will imply $ \RR_\pr \subseteq
\mathcal{E}$.  Note first that $(\inact, \inact) \in \RR_\pr$ because
$(\End, \inact) \in \downarrow_\pr$, and that $(\inact, \inact) \in
\mathcal E$ because Clauses~\ref{cbf} and~\ref{caf} are vacuously
satisfied by the pair   $(\inact, \inact)$.
The proof is by induction on $d=\weight(\G,\pr)$. We only consider
Clause~\ref{caf}, the proof   for   Clause~\ref{cbf}
being similar.  So, assume 
  $(\PP,\Q)\in \RR_\pr$ and   
$\PP=\inpP\pp{i}{I}{\M}{\PP_i}$. \\[3pt]
{\it Case $d=1$.}  In this case  $\G= \gt\pp\pr i I \la \G $ and $\PP= \inpP\pp{i}{I}{\M}{\PP_i}$ and $\prR{\G_i}{  \PP_i}\pr$ for all $i\in I$. From $\prR\G\Q\pr$ we get 
$\Q= \inpP\pp{i}{I}{\M}{\Q_i}$ and $\prR{\G_i}{ \Q_i}\pr$ for all $i\in I$. 
  Hence  $\Q$ has the required form and
   $\iR{\PP_i}{\Q_i}\RR_\pr$ for all $i\in I$.\\
   {\it Case $d>1$.}  In this case  $\G= \gt\pp\q j  J {\la'} \G $ and $\pr\not\in \set{\pp,\q}$ and $\prR{\G_j}{\PP}\pr$ for all $j\in J$. From
$\prR\G\Q\pr$ we get $\prR{\G_j}{\Q}\pr$ for all $j\in J$.   Then   $\iR{\PP}{\Q}{{  \RR_\pr }}$.
\end{proof}
\begin{lemmaa}{\ref{keysr}}{Let $\G$ be a well-formed global type. 
\begin{enumerate}
\item
If $\proj\G\pp=\oup\q{i}{I}{\M}{\PP}$ and $\proj\G\q=\inp\pp{j}{J}{\M'}{\Q}$, then $I=J$, $\M_i=\M_i'$, $\G\stackred{\Comm\pp{\la_i}\q}\G_i$, $\proj{\G_i}\pp=\PP_i$ and $\proj{\G_i}\q=\Q_i$ for all $i\in I$.
\item If $\G\stackred{\Comm\pp{\la}\q}\G'$, then
  $\proj\G\pp=\oup\q{i}{I}{\M}{\PP}$,
  $\proj\G\q=\inp\pp{i}{I}{\M}{\Q}$, where $\la_i=\la$
for some $i\in I$, and 
$\proj{\G'}\pr=\proj\G\pr$ for all $\pr\not\in\set{\pp,\q}$.
\end{enumerate}}\end{lemmaa}
\begin{proof}
  (\ref{keysr1}). The proof is by induction on $d=\weight(\G,\pp)$.\\
 If $d=1$, then by definition of projection (see
  Figure~\ref{fig:proj}) $\proj\G\pp=\oup\q{i}{I}{\M}{\PP}$ implies
  $\G=\gt\pp\q i I {\la} {\G}$ with $\proj{\G_i}\pp=\PP_i$. By the
  same definition it follows that $J = I$ and $\la'_j=\la_j$ and $\Q_j
  = \proj{\G_j}\q$ for all $j\in J$.  Moreover
  $\G\stackred{\Comm\pp{\la_i}\q}\G_i$ by Rule
  \rulename{Ecomm} for all $i\in I$. 
  \\
  If $d>1$, then $\G=\gt\pr\ps h H {\la''} {\G'}$ with
  $\set{\pp,\q}\cap\set{\pr,\ps}=\emptyset$.  By definition of
  projection $\proj\G\pp=\proj{\G'_h}\pp$ and
  $\proj\G\q=\proj{\G'_h}\q$ for all $h\in H$. By Proposition~\ref{dd}
  $\weight(\G,\pp)>\weight(\G'_h,\pp)$ for all $h\in H$.  Then by
  induction $I=J$, $\M_i=\M_i'$,
  \mbox{$\G'_h\stackred{\Comm\pp{\la_i}\q}\G^i_h$,} $\proj{\G^i_h}\pp=\PP_i$
  and $\proj{\G^i_h}\q=\Q_i$ for all $i\in I$ and all $h\in H$. Let
  $\G_i=\gt\pr\ps h H {\la''} {\G^i}$.  By Rule \rulename{Icomm}
  $\G\stackred{\Comm\pp{\la_i}\q}\G_i$ for all $i\in I$.
  By definition of projection $\proj{\G_i}\pp=\PP_i$ and $\proj{\G_i}\q=\Q_i$ for all $i\in I$.\\
  (\ref{keysr2}). The proof is by induction on  the  transition rules of \refToFigure{ltgt}.  \\
  The interesting case is: \prooftree
  \G_h\stackred{\Comm\pp{\la}\q}\G_h' \quad h \in H
  \quad\set{\pp,\q}\cap\set{\ps,\pt}=\emptyset \justifies \gt\ps\pt h
  H {\la'} \G \stackred{\Comm\pp{\la}\q}\gt\ps\pt h H {\la'} {\G'}
  \using ~~~\rulename{Icomm}
  \endprooftree\\
  with $\G=\gt\ps\pt h H {\la'} \G$ and $\G'=\gt\ps\pt h H {\la'} {\G'} $. By induction $\proj{\G_h}\pp=\oup\q{i}{I}{\M}{\PP}$, $\proj{\G_h}\q=\inp\pp{i}{I}{\M}{\Q}$, $\M=\M_i$ for some $i\in I$ and 
$\proj{\G'_h}\pr=\proj{\G_h}\pr$ for all $\pr\not\in\set{\pp,\q}$ and all $h\in H$. By definition of projection $\proj\G\pp=\proj{\G_h}\pp$ and $\proj\G\q=\proj{\G_h}\q$ for all $h\in H$. For $\pr\not\in\set{\pp,\q,\ps,\pt}$ we get $\proj{\G'}\pr=\proj{\G_h'}\pr=\proj{\G_h}\pr=\proj{\G}\pr$. Moreover $\proj{\G'}\ps=\oupp\pt{h}{H}{\M'}{\proj{\G'_h}\ps}=\oupp\pt{h}{H}{\M'}{\proj{\G_h}\ps}=\proj{\G}\ps$ and $\proj{\G'}\pt=\inpp\pt{h}{H}{\M'}{\proj{\G'_h}\pt}=\inpp\ps{h}{H}{\M'}{\proj{\G_h}\ps}=\proj{\G}\pt$.
\end{proof}
\begin{lemmaa}{\ref{prop:prePostNet}}{\begin{enumerate}
\item  If $\post{\netev}{\alpha}$ is defined, then
  $\preP{\post{\netev}{\alpha}}{\alpha}=\netev$;
\item 
$\postP{\pre{\netev}{\alpha}}{\alpha}=\netev$;
 \item   If  $\netev\precN \netev'$, 
then $\pre{\netev}{\alpha}\precN \pre{\netev'}{\alpha}$;
\item  If  $\netev\precN \netev'$ and both $\post{\netev}{\alpha}$ and
  $\post{\netev'}{\alpha}$ are defined, then
  $\post{\netev}{\alpha}\precN \post{\netev'}{\alpha}$; 
\item If  $\netev\grr \netev'$, then $\pre{\netev}{\alpha}\grr
  \pre{\netev'}{\alpha}$;
\item  If  $\netev\grr
  \netev'$ and both $\post{\netev}{\alpha}$ and $\post{\netev'}{\alpha}$
  are defined, then $\post{\netev}{\alpha}\grr\post{\netev'}{\alpha}$;
  \item  If  $\pre{\netev}{\alpha}\grr
  \pre{\netev'}{\alpha}$, then $\netev\grr \netev'$.
\end{enumerate}}\end{lemmaa}

\begin{proof}
 For (\ref{ppn1}) and (\ref{ppn1b})  it  is enough to show
  the corresponding properties for
  located events.
  
  (\ref{ppn1}) Since $\postP{\locev{\pp}\event}{\alpha}$ is defined,
  we have $\event=\concat{(\projS{\alpha}{\pp})}\event'$ and
  $\postP{\locev{\pp}\event}{\alpha}= \locev{\pp}{\event'}$ for some
  $\event'$. Then $\preP{ \postP{\locev{\pp}\event}{\alpha}}{\alpha} =
  \preP{\locev{\pp}{\event'}}{\alpha}=
  \locev{\pp}{\concat{(\projS{\alpha}{\pp})}{\event'}}=
  \locev{\pp}{\event}$.
 
 
  (\ref{ppn1b}) Since
  $\preP{\locev{\pp}{\event}}\alpha=\locev{\pp}{\concat{(\projS{\alpha}{\pp})}\event}
  \,$ is always defined, we immediately get $
  \postP{\preP{\locev{\pp}{\event}}\alpha}{\alpha} =
  \postP{\locev{\pp}{\concat{(\projS{\alpha}{\pp})}\event}}{\alpha} =
  \locev{\pp}{\event}$.
 
  (\ref{ppn2b}) Let $\netev\precN \netev'$. By
  \refToDef{netevent-relations}(\ref{c1}), there are
  $\locev{\pp}{\event}\in\netev$ and $\locev{\pp}{\event'}\in\netev'$
  such that $\event<\event'$.  Then
  $\preP{\locev{\pp}{\event}}\alpha=\locev{\pp}{\concat{(\projS{\alpha}{\pp})}\event}
  \, \in \pre{\netev}{\alpha}$ and
  $\preP{\locev{\pp}{\event'}}\alpha=\locev{\pp}{\concat{(\projS{\alpha}{\pp})}\event'}
  \, \in \pre{\netev'}{\alpha}$. Since $\event<\event'$ implies
  $\concat{(\projS{\alpha}{\pp})}{\event} <
  \concat{(\projS{\alpha}{\pp})}{\event'}$, we conclude that
  $\pre{\netev}{\alpha}\precN \pre{\netev'}{\alpha}$.

  (\ref{ppn2}) % Again by \refToDef{netevent-relations}, if
% $\netev\precN \netev'$ then 
As in the previous case,
there are $\locev{\pp}{\event}\in\netev$ and
$\locev{\pp}{\event'}\in\netev'$ such that $\event<\event'$.  Since
both $\post{\netev}{\alpha}$ and $\post{\netev'}{\alpha}$ are defined,
% also $\postP{\locev{\pp}\event}{\alpha}$ and
% $\postP{\locev{\pp}\event'}{\alpha}$ are defined and thus 
there exist $\event_0$ and $\event'_0$ such that
$\event=\concat{(\projS{\alpha}{\pp})}\event_0$ and $\event'
=\concat{(\projS{\alpha}{\pp})}\event'_0$ and
$\postP{\locev{\pp}\event}{\alpha}= \locev{\pp}{\event_0}$ and
$\postP{\locev{\pp}\event'}{\alpha}= \locev{\pp}{\event'_0}$. Since
$\event<\event'$ implies $\event_0<\event'_0$,
%and $\postP{\locev{\pp}\event}{\alpha} \in \post{\netev}{\alpha}$ 
%and $\postP{\locev{\pp}\event'}{\alpha} \in \post{\netev'}{\alpha}$,
we conclude that $\post{\netev}{\alpha}\precN \post{\netev'}{\alpha}$.

(\ref{ppn3b}) Let $\netev\grr\netev'$. If Clause (\ref{c21}) of
\refToDef{netevent-relations} applies, then there are
$\locev{\pp}{\procev}\in \netev$ and $\locev{\pp}{\procev'}\in\netev'$
such that $\procev \grr \procev'$.  From
$\pre{(\locev{\pp}{\procev})}{\alpha}=\locev{\pp}{\concat{(\projS{\alpha}{\pp})}\event}$
and
$\pre{(\locev{\pp}{\procev'})}{\alpha}=\locev{\pp}{\concat{(\projS{\alpha}{\pp})}{\event'}}$
we get
$\concat{(\projS{\alpha}{\pp})}\event\grr\concat{(\projS{\alpha}{\pp})}{\event'}$. If
Clause (\ref{c22}) of \refToDef{netevent-relations} applies, then there
are $\locev{\pp}{\procev}\in \netev$ and
$\locev{\q}{\procev'}\in\netev'$ with $\pp \neq \q$ such that
$\cardin{\proj\procev\q} = \cardin{\proj{\procev'}\pp}$ and
$\neg(\dualev{\proj\procev\q}{\proj{\procev'}\pp})$. Let
$\event_0 =\concat{(\projS{\alpha}{\pp})}\event$ and $\event'_0
=\concat{(\projS{\alpha}{\q})}\event'$.  If
$\participant{\alpha}\neq\set{\pp,\q}$, then
$\proj{(\projS{\alpha}{\pp})}{\q} = \ee =
\proj{(\projS{\alpha}{\q})}{\pp}$ and thus
$\proj{\procev_0}\q=\proj{\procev}\q$ and
$\proj{\procev'_0}\pp=\proj{\procev'}\pp$. If
$\participant{\alpha}=\set{\pp,\q}$, say $\alpha=\Comm\pp\la\q$, then
$\procev_0=\sendL{\q}{\la}\cdot\procev$ and
$\procev'_0=\rcvL{\pp}{\la}\cdot\procev'$, which implies
$\cardin{\proj{\procev_0}\q} = \cardin{\proj{\procev}\q} +1 =
\cardin{\proj{\procev'}\pp} +1 = \cardin{\proj{\procev'_0}\pp}$ and
$\neg(\dualev{\proj{\procev_0}\q}{\proj{\procev'_0}\pp})$.  In both
cases we conclude that $\pre\netev\alpha\grr\pre{\netev'}\alpha$.

(\ref{ppn3}) The proof is similar to that of Point (\ref{ppn3b}),
considering that $\post{\netev}{\alpha}$ and $\post{\netev'}{\alpha}$
are defined.

(\ref{ppn7}) Let $\pre\netev\alpha\grr\pre{\netev'}\alpha$. If Clause
(\ref{c21}) of \refToDef{netevent-relations} applies, then there are
$\locev{\pp}{\procev}\in \netev$ and $\locev{\pp}{\procev'}\in
\netev'$ such that
$\concat{(\projS{\alpha}{\pp})}\event\grr\concat{(\projS{\alpha}{\pp})}{\event'}$.
Therefore $\procev \grr \procev'$ and thus $\netev \grr \netev'$.  If
Clause (\ref{c22}) of \refToDef{netevent-relations} applies, then there
are
$\locev{\pp}{\procev_0}=\pre{(\locev{\pp}{\procev})}{\alpha}\in\pre\netev\alpha
$ and
$\locev{\q}{\procev'_0}=\pre{(\locev{\q}{\procev'})}{\alpha}\in\pre{\netev'}\alpha$
with $\pp \neq \q$ such that $\cardin{\proj{\procev_0}\q} =
\cardin{\proj{\procev'_0}\pp}$ and
$\neg(\dualev{\proj{\procev_0}\q}{\proj{\procev'_0}\pp})$.  It follows
that $\event_0 =\concat{(\projS{\alpha}{\pp})}\event$ and $\event'_0
=\concat{(\projS{\alpha}{\q})}\event'$ and $\locev{\pp}{\procev}\in
\netev$ and $\locev{\q}{\procev'}\in \netev'$.  If
$\participant{\alpha}\neq\set{\pp,\q}$, then
$\proj{(\projS{\alpha}{\pp})}{\q} = \ee =
\proj{(\projS{\alpha}{\q})}{\pp}$ and thus
$\proj{\procev}\q=\proj{\procev_0}\q$ and
$\proj{\procev'}\pp=\proj{\procev'_0}\pp$.  If
$\participant{\alpha}=\set{\pp,\q}$, say $\alpha=\Comm\pp\la\q$, then
$\procev_0=\sendL{\q}{\la}\cdot\procev$ and
$\procev'_0=\rcvL{\pp}{\la}\cdot\procev'$, and thus
$\cardin{\proj{\procev}\q} = \cardin{\proj{\procev_0}\q} - 1 =
\cardin{\proj{\procev'_0}\pp} -1 = \cardin{\proj{\procev'}\pp}$ and
$\neg(\dualev{\proj{\procev}\q}{\proj{\procev'}\pp})$. In both cases
we conclude that $\netev \grr \netev'$.
\end{proof}

\begin{lemmaa}{\ref{epp}}{
Let $\Nt\stackred{\alpha}\Nt'$. Then 
\begin{enumerate}
\item $\set{\nec\alpha}\cup\set{\pre\netev\alpha\mid\netev\in\GE(\Nt')}\subseteq\GE(\Nt)$; 
\item $\set{\post\netev\alpha\mid\netev\in\GE(\Nt)\text{ and }\post\netev\alpha\text{ defined} }\subseteq\GE(\Nt')$. 
\end{enumerate}
}\end{lemmaa}
\begin{proof}
  Let $\alpha=\Comm{\pp}{\la}{\q}$. From $\Nt\stackred{\alpha}\Nt'$ we get
%  \\
%  \centerline{$
\[
  \Nt=\pP{\pp}{\textstyle{\oupp\q{i}{I}{\la}{\PP}}}\parN
    \pP{\q}{\inp\pp{j}{J}{\la}{\Q}}\parN\Nt_0
    \]
%$} 
    where
  for some $k\in (I \cap J)$ we have $\la_k = \la$ and 
%  \\
%  \centerline{$
\[
  \Nt'=\pP{\pp}{\PP_k}\parN \pP{\q}{\Q_k}\parN\Nt_0
  \]
  %$}
  
  (\ref{epp1})   Let
  $\RT=\set{\nec\alpha}\cup\set{\pre\netev\alpha\mid\netev\in\GE(\Nt')}$.
  We first show that $\RT\subseteq\DE(\Nt)$.  By
  \refToDef{netev-relations}(\ref{netev-relations1})
  $\nec{\alpha}\in\DE(\Nt)$.  Let $\netev=\set{\locev{\pr}{\event},
    \locev{\ps}{\event'}} \in \GE(\Nt')$. We want to prove
  that $\pre\netev\alpha\in\DE(\Nt)$.  By
  \refToDef{netev-relations}(\ref{netev-relations1}) there are $R,S$
  such that $\pP\pr R\in\Nt'$ and $\pP\ps S\in\Nt'$ and $\procev\in
  \ES(R)$ and $\procev'\in \ES(S)$.
%
There are two possible cases:
\begin{itemize}
\item
$\set{\pr,\ps} \cap \set{\pp,\q} = \emptyset$. Then
$\pP\pr R\in\Nt$ and $\pP\ps S\in\Nt$ and thus $\pre\netev\alpha = \netev \in\DE(\Nt)$;
%Since $\pre\netev\alpha = \netev$, we are done;
\item $\set{\pr,\ps} \cap \set{\pp,\q} \neq \emptyset$. 
Suppose $\pr = \pp$. 
Then $\procev\in\ES(\PP_k)$ and 
$\locev\pp {\sendL\q{\la_k}\cdot\procev}\in{\pre\netev\alpha}$ and
$\sendL\q{\la_k}\cdot\procev\in\ES(\oup\q{i}{I}{\la}{\PP})$. 
There are two subcases:
\begin{itemize}
\item \mylabel{case1} $\ps = \q$. Then $\procev'\in\ES(\Q_k)$ and
  $\locev\q {\rcvL\pp{\la_k}\cdot\procev'}\in{\pre\netev\alpha}$ and
  $\sendL\q{\la_k}\cdot\procev'\in\ES(\inp\pp{j}{J}{\la}{\Q})$.  In
  this case we have $\pre\netev\alpha = \set{\locev\pp
      {\sendL\q{\la_k}\cdot\procev}, \locev\q
      {\rcvL\pp{\la_k}\cdot\procev'}} \in \DE(\Nt)$;
%
\item \mylabel{case2} $\ps \neq \q$. Then $\pre{\locev{\ps}{\event'}}{\alpha} =
\locev{\ps}{\event'}$, and thus $\pre{\netev}{\alpha} =
\set{\locev{\pp}{\sendL\q{\la_k}\cdot\procev}, \locev{\ps}{\event'}} \in \DE(\Nt)$.
\end{itemize}
\end{itemize}
Therefore, $\pre\netev\alpha\in\DE(\Nt)$. 
% We conclude that 
Hence
$\RT\subseteq\DE(\Nt)$.  We want now to show that
$\RT\subseteq\GE(\Nt)$.  



Recall from \refToSection{sec:netS-ES} that 
$\GE(\Nt)$ is the greatest fixed point of the function
%\\
%\centerline{ $ 
\[
f_{\DE(\Nt)}(X) = \set{\netev_0\in \DE(\Nt) \mid
    \exists E_0 \subseteq X. \, E_0 ~\text{is a causal set of }
    \netev_0 \text{ in } X}
    \]
%    $ } 
%@
Then $\GE(\Nt)$ is also the greatest post-fixed point of
$f_{\DE(\Nt)}(X)$, namely the greatest $X$ such that $X\subseteq
f_{\DE(\Nt)}(X)$.
Therefore, to show that $\RT \subseteq \GE(\Nt)$, it is enough to
show that $\RT$ is also a post-fixed point of
$f_{\DE(\Nt)}(X)$, namely that $\RT \subseteq f_{\DE(\Nt)}(\RT)$.  


Consider first the event $\nec{\alpha}$. Since the only 
causal set of $\nec{\alpha}$ in any set is $\emptyset$,
it is immediate that $\nec{\alpha}\in f_{\RT}(\RT)$. 
Consider now $\pre\netev\alpha \in \RT$ for some $\netev\in \GE(\Nt')$  with $\loc\netev=\set{\pr,\ps}$.  Define
%\\
%\centerline{$
\[
\causpre{\alpha}{E}{\netev}= \begin{cases}
    \Xi    & \text{if }\set{\pr,\ps} \cap \set{\pp,\q} = \emptyset\\
    \set{\nec\alpha}\cup\Xi & \text{otherwise}
\end{cases}
\]
%$}
where $\Xi=
\set{\pre{\netev'}\alpha\mid\netev'
  \in E\text{ and $E$ is a causal set of $\netev$ in $\GE(\Nt')$}}$. 

%$\causpre{\alpha}{E}{\netev}= \set{\pre{\netev'}\alpha\mid\netev'
%  \in E\text{ where $E$ is a causal set of $\netev$ in $\GE(\Nt')$}}$ if $\set{\pr,\ps} \cap \set{\pp,\q} = \emptyset$, and
%$\causpre{\alpha}{E}{\netev}= \set{\pre{\netev'}\alpha\mid\netev'
%  \in E\text{ where $E$ is a causal set of $\netev$ in $\GE(\Nt')$}}$ otherwise. 
  We show that $\causpre{\alpha}{E}{\netev}$ is a causal set of
$\pre\netev\alpha$ in $\RT$, namely that it is a minimal 
subset of $\RT$ satisfying Conditions (\ref{cs1}) and
(\ref{cs2}) of \refToDef{cs}.\\ % and it is minimal. 
{\em Condition }(\ref{cs1}) If $\nec{\alpha} \in \causpre{\alpha}{E}{\netev}$, then
$\set{\pr,\ps} \cap \set{\pp,\q} \neq \emptyset$. A conflict between $\nec{\alpha}$ and any other event of
$\causpre{\alpha}{E}{\netev} \cup \set{\pre\netev\alpha}$ can only be derived
by Clause (\ref{c21}) of \refToDef{netevent-relations}, since $\nec{\alpha}
=\set{\locev{\pp}{\sendL{\q}{\la}},
  \locev{\q}{\rcvL{\pp}{\la}}}$ and
$\proj{(\projS{\alpha}{\pp})}{\pt} = \proj{(\projS{\alpha}{\q})}{\pt}
= \ee$ for all $\pt\not\in\set{\pp,\q}$.  Suppose $\pr
= \pp$. Then $\locev{\pp}{\concat{\sendL{\q}\la}{\procev}}\in
\pre{\netev}{\alpha}$.  Since $\sendL{\q}\la <
\concat{\sendL{\q}\la}{\procev}$,  Clause (\ref{c21}) cannot be
used to derive a conflict $\nec{\alpha} \grr\pre{\netev}{\alpha}$.
Similarly, if $ \pre{\netev_1}\alpha\in \causpre{\alpha}{E}{\netev}$ and
$\locev{\pp}{\procev_1}\in \netev_1$, then
$\locev{\pp}{\concat{\sendL{\q}\la}{\procev_1}}\in \netev_1$.
Then $\sendL{\q}\la <
\concat{\sendL{\q}\la}{\procev_1}$, hence  Clause (\ref{c21})  
cannot be used to derive $\nec{\alpha}\grr \pre{\netev_1}\alpha$.\\
Suppose now $\pre{\netev_1}\alpha\in \causpre{\alpha}{E}{\netev}$ and  
$\pre{\netev_2}\alpha\in \causpre{\alpha}{E}{\netev}$. Since $E$ is a causal
set, we have $\neg(\netev_1\grr\netev_2)$. Thus 
$\neg(\pre{\netev_1}\alpha\grr\pre{\netev_2}{\alpha})$ by
\refToLemma{prop:prePostNet}(\ref{ppn7}). \\
{\em Condition }(\ref{cs2}) Let $\netev=\set{\locev{\pr}{\event},
  \locev{\ps}{\event'}}$, we have 
$\pre\netev\alpha =\set{\locev{\pr}{\concat{(\projS{\alpha}{\pr})}{\procev}},
  \locev{\ps}{\concat{(\projS{\alpha}{\ps})}{\procev'}}}$.  We 
show that if $\procev_0 < \concat{(\projS{\alpha}{\pr})}{\procev}$,
then $\locev{\pr}{\procev_0} \in
\netev_0$ for some $\netev_0 \in \causpre{\alpha}{E}{\netev}$.
From $\procev_0 < \concat{(\projS{\alpha}{\pr})}{\procev}$ we derive
%If $\pr \in \set{\pp,\q}$, then $\projS{\alpha}{\pr}\neq \ee$ and
$\procev_0 = \concat{(\projS{\alpha}{\pr})}{\actseq}$ for some
$\actseq$ such that $\actseq < \procev$. If $\actseq \neq \ee$, then
$\actseq = \procev'_0 < \procev$. Since $E$ is a causal set,
$\procev'_0 < \procev_0$ implies $\occ{\locev{\pr}{\procev'_0}}{E}$.
Hence $\occ{\locev{\pr}{\procev_0}}{\causpre{\alpha}{E}{\netev}}$. If
instead $\actseq = \ee$, then it must be $\procev_0 =
\projS{\alpha}{\pr}\neq \ee$ and thus $\pr \in \set{\pp,\q}$. In this
case $\set{\nec{\alpha}} \in \causpre{\alpha}{E}{\netev}$ and thus
$\occ{\locev{\pr}{\procev_0}}{\causpre{\alpha}{E}{\netev}}$.\\
As for {\em minimality }, we first show that $\netev'\prec
\pre\netev\alpha $ for all $\netev'\in\causpre{\alpha}{E}{\netev}$.
If $\nec{\alpha} \in \causpre{\alpha}{E}{\netev}$, then $\set{\pr,\ps}
\cap \set{\pp,\q} \neq \emptyset$. Then $\nec{\alpha} \prec
\pre\netev\alpha$.  If $ \netev_1\in \causpre{\alpha}{E}{\netev}$ and
$ \netev_1\neq \nec{\alpha}$, then there exists $\netev'_1 \in E$ such
that $\netev_1 =\pre{\netev'_1}\alpha$. Since $E$ is a causal set for
$\netev$, we have $\netev'_1 \prec \netev$. Therefore $\netev_1 =
\pre{\netev'_1}\alpha\prec\pre\netev\alpha$ by
\refToLemma{prop:prePostNet}(\ref{ppn2b}). Assume now that
$\causpre{\alpha}{E}{\netev}$ is not minimal.  Then there is
$E'\subset\causpre{\alpha}{E}{\netev}$ that verifies Condition
(\ref{cs2}) of \refToDef{cs} for $\pre\netev\alpha$. Let $\netev'\in
\causpre{\alpha}{E}{\netev}\setminus E'$. Then $\netev'\prec
\pre\netev\alpha=\set{\locev{\pr}{\procev_{\pr}},
  \locev{\ps}{\procev_{\ps}}}$. Assume that
$\locev{\pr}{\procev'_{\pr}}\in\netev'$ with
$\procev'_{\pr}<\procev_{\pr}$ (the proof is similar for $\ps$).  By
Condition (\ref{cs2}), there is $\netev''\in E'$ such that
$\locev{\pr}{\procev'_{\pr}}\in\netev''$.  But then
$\netev'\grr\netev''$ by \refToProp{prop:conf}, contradicting the fact
that $\causpre{\alpha}{E}{\netev}$ verifies Condition
(\ref{c1}). Therefore $\causpre{\alpha}{E}{\netev}$ is minimal.

(\ref{epp2}) Let $\RS=\set{\post\netev\alpha\mid\netev\in
  \GE(\Nt)\text{ and }\post\netev\alpha\text{ defined} }$.  We first
show that $\RS\subseteq\DE(\Nt')$.  Let
$\netev=\set{\locev{\pr}{\event}, \locev{\ps}{\event'}} \in \GE(\Nt)$
be such that $\post\netev\alpha$ is defined. We want to prove that
$\post\netev\alpha\in\DE(\Nt')$.  By
\refToDef{netev-relations}(\ref{netev-relations1}) there are $R,S$
such that $\pP\pr R\in\Nt$ and $\pP\ps S\in\Nt$ and $\procev\in
\ES(R)$ and $\procev'\in \ES(S)$.
%
There are two possible cases:
\begin{itemize}
\item
$\set{\pr,\ps} \cap \set{\pp,\q} = \emptyset$. Then
$\pP\pr R\in\Nt'$ and $\pP\ps S\in\Nt'$ and thus $\post\netev\alpha = \netev \in\DE(\Nt')$;
\item $\set{\pr,\ps} \cap \set{\pp,\q} \neq \emptyset$. 
Suppose $\pr = \pp$. 
Then $\procev\in\ES(\oup\q{i}{I}{\la}{\PP})$ and since  $\post\netev\alpha$ is defined we have that
$\procev=\sendL\q{\la_k}\cdot\procev_{k}$ where $\procev_k\in\ES(\PP_k)$.
There are two subcases:
\begin{itemize}
\item \mylabel{Case1} $\ps = \q$. Then $\procev'\in\ES(\inp\pp{j}{J}{\la}{\Q})$ 
and since  $\post\netev\alpha$ is defined
  $\procev'={\rcvL\pp{\la_k}\cdot\procev'_{k}}$ where $\procev'_k\in\ES(\Q_k)$. In
  this case we have $\post\netev\alpha = \set{\locev\pp
      {\procev_{k}}, \locev\q
      {\procev'_{k}}} \in \DE(\Nt')$;
%
\item \mylabel{Case2} $\ps \neq \q$. Then $\post{\locev{\ps}{\event'}}{\alpha} =
\locev{\ps}{\event'}$, and thus $\post{\netev}{\alpha} =
\set{\locev{\pp}{\procev_k}, \locev{\ps}{\event'}} \in \DE(\Nt')$.
\end{itemize}
\end{itemize}
Therefore $\RS\subseteq\DE(\Nt')$. We want now to show that
$\RS\subseteq\GE(\Nt')$.

%%%%%% FINE PEZZO NUOVO %%%%%%%%%%%%%%%%%%%%%%%%%%%


We proceed as in the proof of Statement (\ref{epp1}). We know that
$\GE(\Nt')$ is the greatest post-fixed point of the function
%\\
%\centerline{ $
\[ 
f_{\DE({\Nt'})}(X) = \set{\netev_0\in \DE(\Nt') \mid
    \exists E_0 \subseteq X. \, E_0 ~\text{is a causal set of }
    \netev_0 \text{ in } X}
    \]
%    $ } 
%
Then, in order to obtain $\RS \subseteq \GE(\Nt')$ it is enough to
show that $\RS$ is a post-fixed point of
$f_{\DE({\Nt'})}(X)$, namely that $\RS \subseteq f_{\DE({\Nt'})}(\RS)$.  

%%%%%% FINE PEZZO NUOVO %%%%%%%%%%%%%%%%%%%%%%%%%%%




Let $\post\netev\alpha \in \RS$ for some $\netev\in \GE(\Nt)$. Define
%\\
%\centerline{$
\[
\causpost{\alpha}{E}{\netev}=\set{\post{\netev'}\alpha\mid\netev'
    \in E\text{ and $E$ is a causal set of $\netev$ in $\GE(\Nt)$}}
    \]
%$}
We show that $\causpost{\alpha}{E}{\netev}$ is a causal set of
$\post\netev\alpha$ in $\RS$, namely that it is a minimal subset
of $\RS$ satisfying Conditions (\ref{cs1}) and (\ref{cs2}) of
\refToDef{cs}.\\ % and it is minimal. \\
{\em Condition }(\ref{cs1}) Suppose $\post{\netev_1}\alpha\in
\causpost{\alpha}{E}{\netev}$ and $\post{\netev_2}\alpha\in
\causpost{\alpha}{E}{\netev}$. Since $E$ is a causal set  and
$\netev_1, \netev_2 \in E$, we have
$\neg(\netev_1\grr\netev_2)$. Thus
$\neg(\post{\netev_1}\alpha\grr\post{\netev_2}{\alpha})$ by
\refToLemma{prop:prePostNet}(\ref{ppn3b}) and (\ref{ppn1}). \\
{\em Condition }(\ref{cs2}) Since $\netev=\set{\locev{\pr}{\event},
  \locev{\ps}{\event'}}$ and $\post\netev\alpha$ is defined,
 we have $\procev={\concat{(\projS{\alpha}{\pr})}{\procev_{\pr}}}$ and
$\procev'=\concat{(\projS{\alpha}{\ps})}{\procev_{\ps}}$ and  $\post\netev\alpha
=\set{\locev{\pr}{\procev_{\pr}}, \locev{\ps}{\procev_{\ps}}}$.
Let $\procev_0 < \procev_{\pr}$. Then
$\concat{(\projS{\alpha}{\pr})}{\procev_0} <
\concat{(\projS{\alpha}{\pr})}{\procev_\pr} = \procev$. 
Since $E$ is a causal set for $\netev$ in $\GE(\Nt)$, this implies 
$\occ{\locev{\pr}{\concat{(\projS{\alpha}{\pr})}{\procev_0}}}{E}$. 
Hence $\occ{\locev{\pr}{\procev_0}}{\causpost{\alpha}{E}{\netev}}$. 
\\
As for {\em minimality}, we first show that $\netev'\prec
\post\netev\alpha $ for all $\netev'\in\causpost{\alpha}{E}{\netev}$.
If $ \netev_1\in \causpost{\alpha}{E}{\netev}$, then there exists
$\netev'_1 \in E$ such that $\netev_1 =\post{\netev'_1}\alpha$. Since
$E$ is a causal set for $\netev$, we have $\netev'_1 \prec
\netev$. Therefore $\netev_1 =
\pre{\netev'_1}\alpha\prec\pre\netev\alpha$ by
\refToLemma{prop:prePostNet}(\ref{ppn2b}). Assume now that
$\causpost{\alpha}{E}{\netev}$ is not minimal.  Then there is
$E'\subset\causpost{\alpha}{E}{\netev}$ that verifies Condition
(\ref{cs2}) of \refToDef{cs} for $\post\netev\alpha$.
%
Let $\netev'\in \causpost{\alpha}{E}{\netev}\setminus E'$. Then
$\netev'\prec \post\netev\alpha=\set{\locev{\pr}{\procev_{\pr}},
  \locev{\ps}{\procev_{\ps}}}$. Assume that
$\locev{\pr}{\procev'_{\pr}}\in\netev'$ with
$\procev'_{\pr}<\procev_{\pr}$ (the proof is similar for $\ps$).  By
Condition (\ref{cs2}), there is $\netev''\in E'$ such that
$\locev{\pr}{\procev'_{\pr}}\in\netev''$.  But then
$\netev'\grr\netev''$ by \refToProp{prop:conf}, contradicting the fact
that $\causpost{\alpha}{E}{\netev}$ verifies Condition
(\ref{c1}). Therefore $\causpost{\alpha}{E}{\netev}$ is minimal.
\end{proof}
\begin{lemmaa}{\ref{prop:prePostGl}}{\begin{enumerate}
\item  If $\postG{\comocc}{\alpha}$ is defined, then $\preG{(\postG{\comocc}{\alpha})}{\alpha}=\comocc$;
\item 
$\postG{(\preG{\comocc}{\alpha})}{\alpha}=\comocc$;
 \item   If  $\comocc_1< \comocc_2$, 
then $\preG{\comocc_1}{\alpha}< \preG{\comocc_2}{\alpha}$;
\item  If  $\comocc_1<\comocc_2$ and  both $\postG{\comocc_1}{\alpha}$ and
  $\postG{\comocc_2}{\alpha}$ %is %
  are 
  defined, then
  $\postG{\comocc_1}{\alpha}< \postG{\comocc_2}{\alpha}$;
 \item   If $\comocc_1\gr \comocc_2$, 
then $\preG{\comocc_1}{\alpha}\gr \preG{\comocc_2}{\alpha}$;
  \item  If $\comocc<\preG{\comocc'}{\alpha}$, then either $\comocc=\eqclass\alpha$ or $\postG{\comocc}{\alpha}<{\comocc'}$;
 \item  If $\participant{\alpha_1}\cap\participant{\alpha_2}=\emptyset$, then $\preG{(\preG{\comocc}{\alpha_2})}{\alpha_1}=\preG{(\preG{\comocc}{\alpha_1})}{\alpha_2}$;
 \item  If $\participant{\alpha_1}\cap\participant{\alpha_2}=\emptyset$ and both $\postG{(\preG{\comocc}{\alpha_1})}{\alpha_2}$, $\postG{\comocc}{\alpha_2}$ are defined, then $\preG{(\postG{\comocc}{\alpha_2})}{\alpha_1}= \postG{(\preG{\comocc}{\alpha_1})}{\alpha_2}$.
\end{enumerate}}\end{lemmaa}
\begin{proof}
  (\ref{ppg1a}) If $\postG{\eqclass\comseq}{\alpha}$ is defined, then
  in case $\participant\alpha\cap\participant\comseq=\emptyset$ we get
  $\postG{\eqclass\comseq}{\alpha}=\eqclass\comseq$ and also
  $\preG{\eqclass\comseq}\alpha=\eqclass\comseq$, so
  $\preG{(\postG{\eqclass\comseq}{\alpha})}{\alpha}=\eqclass\comseq$.
  Instead if $\participant\alpha\cap\participant\comseq\not=\emptyset$, then
  $\postG{\eqclass\comseq}{\alpha}=\eqclass{\comseq'}$ where
  $\comseq\sim\concat\alpha{\comseq'}$ and
  $\comseq'\neq\emptyseq$.   From $\participant\alpha\cap\participant\comseq\not=\emptyset$ we get 
  $\preG{\eqclass{\comseq'}}\alpha=\eqclass{\concat{\alpha}{\comseq'}}$ by \refToDef{causal-path}. This 
  implies $\preG{(\postG{\eqclass\comseq}{\alpha})}{\alpha}=\eqclass\comseq$. 
%  By \refToDef{def:glEvent} $\comseq$ is a
%  pointed sequence, so
%  $\participant\alpha\cap\participant\comseq\not=\emptyset$.  By
%  \refToDef{causal-path}
%  $\preG{\eqclass{\comseq'}}\alpha=\eqclass{\concat{\alpha}{\comseq'}}$,
%  which implies
%  $\preG{(\postG{\eqclass\comseq}{\alpha})}{\alpha}=\eqclass\comseq$.
  
  (\ref{ppn1b}) By \refToDef{causal-path} either
  $\preG{\eqclass\comseq}\alpha=\eqclass{\concat{\alpha}{\comseq}}$
  if $\participant\alpha\cap\participant\comseq\not=\emptyset$, or
  $\preG{\comseq}\alpha=\eqclass{\comseq}$.  In the first
  case
  $\postG{\eqclass{\concat{\alpha}{\comseq}}}{\alpha}=\eqclass\comseq$
  and in the second
  $\postG{\eqclass{\comseq}}{\alpha}=\eqclass\comseq$, which proves the result. 
  
  (\ref{ppg4a})
   Let $\comocc_1=\eqclass{\comseq}$ and
  $\comocc_2=\eqclass{\concat{\comseq}{\comseq'}}$.  If  %Suppose
  $\participant{\alpha} \cap \participant{\comseq} \neq \emptyset$,
  then %also 
  $\participant{\alpha}
  \cap \participant{\concat{\comseq}{\comseq'}} \neq \emptyset$, and
  we have
  $\preG{\comocc_1}{\alpha}=\eqclass{\concat{\alpha}{\comseq}}$ and
  $\preG{\comocc_2}{\alpha}=\eqclass{\concat{\alpha}{\concat{\comseq}{\comseq'}}}$.
Whence $\preG{\comocc_1}{\alpha} \leq \preG{\comocc_2}{\alpha}$.
  Suppose now $\participant{\alpha} \cap \participant{\comseq} =
  \emptyset$. Then $\preG{\comocc_1}{\alpha}=\eqclass{\comseq} =
  \comocc_1$. Now, if also  $\participant{\alpha} \cap \participant{\comseq'} =
  \emptyset$, then $\preG{\comocc_2}{\alpha}=\eqclass{\concat{\comseq}{\comseq}} =
  \comocc_2$ and we are done. If instead $\participant{\alpha} \cap \participant{\comseq'} \neq
  \emptyset$, then $\preG{\comocc_2}{\alpha}=\eqclass{\concat{\alpha}{\concat{\comseq}{\comseq'}}} =
\eqclass{\concat{\comseq}{\concat{\alpha}{\comseq'}}}$, whence $\comocc_1 \leq \preG{\comocc_2}{\alpha}$.

   (\ref{ppg4b}) Let $\comocc_1=\eqclass{\comseq}$ and
  $\comocc_2=\eqclass{\concat{\comseq}{\comseq'}}$. If
  $\participant{\alpha} \cap \participant{\comseq}= \participant{\alpha} \cap \participant{\concat{\comseq}{\comseq'}}= \emptyset$, then 
  $\postG{\comocc_1}{\alpha}=\comocc_1$ and $\postG{\comocc_2}{\alpha}=\comocc_2$. If $\participant{\alpha} \cap \participant{\comseq} \neq
  \emptyset$, then $\comseq\sim\concat\alpha{\comseq_0}$, which implies $\postG{\comocc_1}{\alpha}=\eqclass{\comseq_0}$ and $\postG{\comocc_2}{\alpha}=\eqclass{\concat{\comseq_0}{\comseq'}}$. If
  $\participant{\alpha} \cap \participant{\comseq}=\emptyset$ and $\participant{\alpha} \cap \participant{\concat{\comseq}{\comseq'}} \neq
  \emptyset$, then $\postG{\comocc_1}{\alpha}=\eqclass{\comseq}$ and $\comseq'\sim\concat\alpha{\comseq_0}$, which implies $\postG{\comocc_2}{\alpha}=\eqclass{\concat{\comseq}{\comseq_0}}$.
     
  
  (\ref{prop:prePostGl5}) Let $\comocc_1=\eqclass{\comseq}$ and
  $\comocc_2=\eqclass{\comseq'}$ and $\pro{\comseq}\pp\gr\pro{\comseq'}\pp$ for some $\pp$. The only interesting case is  $\participant\alpha\cap\participant{\comseq}=\emptyset$ and  $\participant\alpha\cap\participant{\comseq'}\not=\emptyset$. This implies $\preG{\comocc_1}{\alpha}=\eqclass{\comseq}$ and  $\preG{\comocc_2}{\alpha}=\eqclass{\concat\alpha{\comseq'}}$.
  We get $\pro{(\concat\alpha{\comseq'})}\pp=\pro{\comseq'}\pp$ since $\participant\alpha\cap\participant{\comseq}=\emptyset$ implies $\pp\not\in\participant\alpha$. We conclude  $\preG{\comocc_1}{\alpha}\gr \preG{\comocc_2}{\alpha}$.
  
  (\ref{ppg3})  Let $\comocc=\eqclass{\comseq}$ and $\preG{\comocc'}{\alpha}=\eqclass{\concat{\comseq}{\comseq'}}$.
  If $\postG{\comocc}{\alpha}$ is defined by Point~\ref{ppg4b}
  $\postG{\comocc}{\alpha}<\postG{(\preG{\comocc'}{\alpha})}{\alpha}$
  and by Point~\ref{ppg1b}
  $\postG{(\preG{\comocc'}{\alpha})}{\alpha}=\comocc'$. Otherwise
  either $\comocc=\eqclass\alpha$, in which case we are done, or 
  $\participant{\alpha} \cap \participant{\comseq} \neq
  \emptyset$ and $\comseq\not\sim\concat\alpha{\comseq_0}$. 
  This last case is impossible, since $\participant{\alpha} \cap \participant{\concat{\comseq}{\comseq'}} \neq
  \emptyset$ and $\concat{\comseq}{\comseq'}\not\sim\concat\alpha{\comseq_1}$  contradict the definition of $\circ$ (\refToDef{causal-path}(\ref{causal-path1})).

  
  
  (\ref{prop:prePostGl6}) Let $\comocc=\eqclass{\comseq}$. By \refToDef{causal-path}(\ref{causal-path1}) we have four cases: 
%\bmc commentato per compilare \emc  \bcompa\ Fatto\ecompa
  \begin{enumerate}[label=(\alph*)]%[(a)]
  \item $\preG{(\preG{\comseq}{\alpha_2})}{\alpha_1}=\eqclass{\concat{\alpha_1}{(\concat{\alpha_2}\comseq)}}=\eqclass{\concat{\alpha_2}{(\concat{\alpha_1}\comseq)}}=\preG{(\preG{\comseq}{\alpha_1})}{\alpha_2}$ if $\participant{\alpha_1}\cap\participant{\comseq}\not=\emptyset$ and $\participant{\alpha_2}\cap\participant{\comseq}\not=\emptyset$, since $\participant{\alpha_1}\cap\participant{\alpha_2}=\emptyset$;
  \item $\preG{(\preG{\comseq}{\alpha_2})}{\alpha_1}=\eqclass{\concat{\alpha_1}\comseq}=\preG{(\preG{\comseq}{\alpha_1})}{\alpha_2}$ if $\participant{\alpha_1}\cap\participant{\comseq}\not=\emptyset$ and $\participant{\alpha_2}\cap\participant{\comseq}=\emptyset$;
   \item $\preG{(\preG{\comseq}{\alpha_2})}{\alpha_1}=\eqclass{\concat{\alpha_2}\comseq}=\preG{(\preG{\comseq}{\alpha_1})}{\alpha_2}$ if $\participant{\alpha_1}\cap\participant{\comseq}=\emptyset$ and $\participant{\alpha_2}\cap\participant{\comseq}\not=\emptyset$;
    \item $\preG{(\preG{\comseq}{\alpha_2})}{\alpha_1}=\eqclass{\comseq}=\preG{(\preG{\comseq}{\alpha_1})}{\alpha_2}$ if $\participant{\alpha_1}\cap\participant{\comseq}=\emptyset$ and $\participant{\alpha_2}\cap\participant{\comseq}=\emptyset$.
  \end{enumerate}
  
   (\ref{prop:prePostGl7}) Let $\comocc=\eqclass{\comseq}$. By Definitions~\ref{causal-path}(\ref{causal-path1}) and~\ref{def:PostPreGl}(\ref{def:PostPreGl1}) we have four cases:
%\bmc commentato per compilare \emc  \bcompa\ Fatto\ecompa
  \begin{enumerate}[label=(\alph*)]%[(a)]
\item $\preG{(\postG{\comseq}{\alpha_2})}{\alpha_1}=\eqclass{\concat{\alpha_1}{\comseq'}}=\postG{(\preG{\comseq}{\alpha_1})}{\alpha_2}$ if $\participant{\alpha_1}\cap\participant{\comseq}\not=\emptyset$ and $\comseq\sim\concat{\alpha_2}{\comseq'}$, which implies $\concat{\alpha_1}\comseq=\concat{\alpha_1}{(\concat{\alpha_2}{\comseq'})}\sim\concat{\alpha_2}{(\concat{\alpha_1}{\comseq'})}$, since $\participant{\alpha_1}\cap\participant{\alpha_2}=\emptyset$;
\item $\preG{(\postG{\comseq}{\alpha_2})}{\alpha_1}=\eqclass{\concat{\alpha_1}{\comseq}}=\postG{(\preG{\comseq}{\alpha_1})}{\alpha_2}$ if $\participant{\alpha_1}\cap\participant{\comseq}\not=\emptyset$ and $\participant{\alpha_2}\cap\participant{\comseq}=\emptyset$;
\item $\preG{(\postG{\comseq}{\alpha_2})}{\alpha_1}=\eqclass{\comseq'}=\postG{(\preG{\comseq}{\alpha_1})}{\alpha_2}$ if $\participant{\alpha_1}\cap\participant{\comseq}=\emptyset$ and $\comseq\sim\concat{\alpha_2}{\comseq'}$;
\item $\preG{(\postG{\comseq}{\alpha_2})}{\alpha_1}=\eqclass{\comseq}=\postG{(\preG{\comseq}{\alpha_1})}{\alpha_2}$ if $\participant{\alpha_1}\cap\participant{\comseq}=\emptyset$ and $\participant{\alpha_2}\cap\participant{\comseq}=\emptyset$. \qedhere
  \end{enumerate}
  \end{proof}
\begin{lemmaa}{\ref{paltr}}{Let $\G\stackred\alpha \G'$.
\begin{enumerate}
\item If $\comocc\in\EGG(\G')$, then $\preG\comocc{\alpha}\in \EGG(\G)$;
\item If $\comocc\in\EGG(\G)$ and $\postG\comocc{\alpha}$ is defined, then $\postG\comocc{\alpha}\in\EGG( \G')$.
\end{enumerate}
}\end{lemmaa}
\begin{proof}
Both proofs are by induction on the inference of the transition
$\G\stackred\alpha \G'$, see \refToFigure{ltgt}.

(\ref{paltr1}) For rule \rulename{Ecomm} we get $\G=\gt\pp\q i I \la \G$ and $\G'=\G_k$ and $\alpha=\Comm\pp{\la_k}\q$ for some $k\in I$. We conclude $\preG\comocc{\alpha}\in \EGG(\G)$ by \refToLemma{paldf}(\ref{paldf1}).\\
For rule \rulename{Icomm} we get $\G=\gt\pp\q i I \la \G$ and $\G'=\gtp\pp\q i I \la \G$ and $\G_i\stackred\alpha \G'_i$ for all $i\in I$ and $\participant{\alpha}\cap\set{\pp,\q}=\emptyset$. By \refToDef{eg}(\ref{eg1a}) $\comocc\in\EGG( \G')$ implies $\comocc=\ev\comseq$ for some $\comseq\in\FPaths{\G'}$. This implies $\comseq=\concat{\Comm\pp{\la_k}\q}{\comseq'}$ and $\comocc=\eqclass{\comseq_0}$ 
with either $\comseq_0\sim \concat{\Comm\pp{\la_k}\q}{\comseq'_0}$ for some $k\in I$ or $\participant{\comseq_0}\cap\set{\pp,\q}=\emptyset$ by \refToDef{causal-path}. Then $\postG\comocc{\Comm\pp{\la_k}\q}$ is defined unless $\comseq_0=\Comm\pp{\la_k}\q$ by \refToDef{def:PostPreGl}(\ref{def:PostPreGl1}). We consider two cases.\\
If $\comseq_0=\Comm\pp{\la_k}\q$, then $\preG\comocc{\alpha}=\eqclass{\Comm\pp{\la_k}\q}$ since $\participant{\alpha}\cap\set{\pp,\q}=\emptyset$. We conclude $\preG\comocc{\alpha}\in \EGG(\G)$ by \refToDef{eg}(\ref{eg1a}). Otherwise let $\comocc'=\postG\comocc{\Comm\pp{\la_k}\q}$. By \refToLemma{paldf}(\ref{paldf2}) $\comocc'\in \EGG(\G_k')$. By induction $\preG{\comocc'}{\alpha}\in \EGG(\G_k)$.
By \refToLemma{paldf}(\ref{paldf1}) $\preG{(\preG{\comocc'}{\alpha})}{\Comm\pp{\la_k}\q}\in \EGG(\G)$. 
%Rimpiazza precedente
 We now show that $\preG{(\preG{\comocc'}{\alpha})}{\Comm\pp{\la_k}\q}=\preG\comocc{\alpha}$. 
 By  \refToLemma{prop:prePostGl}(\ref{prop:prePostGl6}) and $\participant{\alpha}\cap\set{\pp,\q}=\emptyset$ we get
 $\preG{(\preG{\comocc'}{\alpha})}{\Comm\pp{\la_k}\q}=\preG{(\preG{\comocc'}{{\Comm\pp{\la_k}\q}})}\alpha$ and
  by \refToLemma{prop:prePostGl}(\ref{ppg1a}) we have
 $\preG{\comocc'}{\Comm\pp{\la_k}\q}=\preG{(\postG\comocc{\Comm\pp{\la_k}\q})}{\Comm\pp{\la_k}\q}=\comocc$. Therefore
 $\preG{(\preG{\comocc'}{\alpha})}{\Comm\pp{\la_k}\q}=\preG\comocc{\alpha}\in \EGG(\G)$. 
 
%Precedente
%\refToLemma{prop:prePostGl}(\ref{prop:prePostGl6}) gives $\preG{(\preG{\comocc'}{\alpha})}{\Comm\pp{\la_k}\q}=\preG{(\preG{\comocc'}{\Comm\pp{\la_k}\q})}{\alpha}$ since $\participant{\alpha}\cap\set{\pp,\q}=\emptyset$. We conclude $\preG\comocc{\alpha}\in \EGG(\G)$ since $\preG{\comocc'}{\Comm\pp{\la_k}\q}=\preG{(\postG\comocc{\Comm\pp{\la_k}\q})}{\Comm\pp{\la_k}\q}=\comocc$ by \refToLemma{prop:prePostGl}(\ref{ppg1a}). 
%

(\ref{paltr2}) For rule \rulename{Ecomm} we get $\G=\gt\pp\q i I \la \G$ and $\G'=\G_k$ and $\alpha=\Comm\pp{\la_k}\q$ for some $k\in I$. We conclude $\postG\comocc{\alpha}\in \EGG(\G')$ by \refToLemma{paldf}(\ref{paldf2}).\\
For rule \rulename{Icomm} we get $\G=\gt\pp\q i I \la \G$ and $\G=\gtp\pp\q i I \la \G$ and $\G_i\stackred\alpha \G'_i$ for all $i\in I$ and $\participant{\alpha}\cap\set{\pp,\q}=\emptyset$. By \refToDef{eg}(\ref{eg1a}) $\comocc\in\EGG( \G)$ implies $\comocc=\ev\comseq$ for some $\comseq\in\FPaths{\G}$. This implies $\comseq=\concat{\Comm\pp{\la_k}\q}{\comseq'}$ and $\comocc=\eqclass{\comseq_0}$ 
with either $\comseq_0\sim \concat{\Comm\pp{\la_k}\q}{\comseq'_0}$ for some $k\in I$ or $\participant{\comseq_0}\cap\set{\pp,\q}=\emptyset$ by \refToDef{causal-path}. Then $\postG\comocc{\Comm\pp{\la_k}\q}$ is defined unless $\comseq_0=\Comm\pp{\la_k}\q$ by \refToDef{def:PostPreGl}(\ref{def:PostPreGl1}). We consider two cases.\\
If $\comseq_0=\Comm\pp{\la_k}\q$, then $\postG\comocc{\alpha}=\eqclass{\Comm\pp{\la_k}\q}$ since $\participant{\alpha}\cap\set{\pp,\q}=\emptyset$. We conclude $\postG\comocc{\alpha}\in \EGG(\G')$ by \refToDef{eg}(\ref{eg1a}). Otherwise let $\comocc'=\postG\comocc{\Comm\pp{\la_k}\q}$. By \refToLemma{paldf}(\ref{paldf2}) $\comocc'\in \EGG(\G_k)$. We first show that $\postG{\comocc'}{\alpha}$ is defined. Since $\postG{\comocc}{\alpha}$ and $\postG\comocc{\Comm\pp{\la_k}\q}$ are defined, by \refToDef{def:PostPreGl}(\ref{def:PostPreGl1}) we have four cases:
%\bmc commentato per compilare \emc  \bcompa\ Fatto\ecompa
\begin{enumerate}[label=(\alph*)]%[(a)]
\item\label{ca} $\comseq_0\sim\concat\alpha{\comseq_1}$ for some $\comseq_1$ and $\comseq_0\sim \concat{\Comm\pp{\la_k}\q}{\comseq'_0}$;
\item\label{cb} $\comseq_0\sim\concat\alpha{\comseq_1}$ and  $\participant{\comseq_0}\cap\set{\pp,\q}=\emptyset$;
\item\label{cc}  $\participant{\alpha}\cap\participant{\comseq_0}=\emptyset$ and $\comseq_0\sim \concat{\Comm\pp{\la_k}\q}{\comseq'_0}$;
\item\label{cd}  $\participant{\alpha}\cap\participant{\comseq_0}=\emptyset$ and $\participant{\comseq_0}\cap\set{\pp,\q}=\emptyset$.
 \end{enumerate}
 In case \ref{ca} $\comseq_0\sim\concat\alpha{\concat{\Comm\pp{\la_k}\q}{\comseq'_1}}\sim \concat{\Comm\pp{\la_k}\q}{\concat\alpha{\comseq'_1}}$ for some $\comseq'_1$  since  $\participant{\alpha}\cap\set{\pp,\q}=\emptyset$. Notice that $\comseq'_1\not=\ee$ since $\comseq_0$ is pointed and $\participant{\alpha}\cap\set{\pp,\q}=\emptyset$. We get $\comocc'=\postG\comocc{\Comm\pp{\la_k}\q}=\eqclass{\concat\alpha{\comseq'_1}}$ and $\postG{\comocc'}{\alpha}=\eqclass{\comseq'_1}$.\\
 In case \ref{cb} $\comocc'=\comocc$ and $\postG{\comocc'}{\alpha}=\eqclass{\comseq_1}$.\\
 In case \ref{cc} $\comocc'=\eqclass{\comseq_0'}$ and $\postG{\comocc'}{\alpha}=\eqclass{\comseq_0'}$, since $\participant{\alpha}\cap\participant{\comseq_0}=\emptyset$ implies $\participant{\alpha}\cap\participant{\comseq_0'}=\emptyset$.\\
 In case \ref{cd} $\comocc'=\comocc$ and $\postG{\comocc'}{\alpha}=\comocc$.\\
 By induction $\postG{\comocc'}{\alpha}\in\EGG(\G'_k)$. By \refToLemma{paldf}(\ref{paldf1}) $\preG{(\postG{\comocc'}{\alpha})}{\Comm\pp{\la_k}\q}\in \EGG(\G')$. 
 
 We now show that $\preG{(\postG{\comocc'}{\alpha})}{\Comm\pp{\la_k}\q}=\postG\comocc{\alpha}$. From $\comocc'=\postG\comocc{\Comm\pp{\la_k}\q}$
 and \refToLemma{prop:prePostGl}(\ref{ppg1a})  
 $\preG{\comocc'}{\Comm\pp{\la_k}\q}=\comocc$. Therefore from $\postG{\comocc}{\alpha}$ defined
 we have $\postG{(\preG{\comocc'}{\Comm\pp{\la_k}\q})}{\alpha}$ defined.
 Since $\postG{\comocc'}{\alpha}$ is also defined and $\participant{\alpha}\cap\set{\pp,\q}=\emptyset$, by 
 \refToLemma{prop:prePostGl}(\ref{prop:prePostGl7}) we get 
 $\preG{(\postG{\comocc'}{\alpha})}{\Comm\pp{\la_k}\q}=\postG{(\preG{\comocc'}{\Comm\pp{\la_k}\q})}{\alpha}$.
 Therefore $\preG{(\postG{\comocc'}{\alpha})}{\Comm\pp{\la_k}\q}=\postG{\comocc}{\alpha}\in \EGG(\G')$.
\end{proof}

