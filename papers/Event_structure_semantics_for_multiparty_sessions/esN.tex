% !TEX root =cdgS.tex

\section{Event Structures}\label{sec:eventStr}



We recall now the definitions of \emph{Prime Event Structure} (PES)
from~\cite{Win80,NPW81} and \emph{Flow Event Structure} (FES)
from~\cite{BC88a}. The class of FESs is more general than that of
PESs: for a precise comparison of various classes of event structures,
we refer the reader to~\cite{BC91}. As we shall see in
Sections~\ref{sec:process-ES} and~\ref{sec:netS-ES}, while PESs are sufficient to interpret
processes, the  greater  generality of FESs is needed to interpret networks.
%
\begin{definition}[Prime Event Structure] \mylabel{pes} A prime event structure {\rm (PES)} is a
    tuple 
$S=(E,\leq, \gr)$ where:
\begin{enumerate}
\item \mylabel{pes1} $E$ is a denumerable set of events;
\item \mylabel{pes2}    $\leq\,\subseteq (E\times E)$ is a partial order relation,
called the \emph{causality} relation;
\item \mylabel{pes3}  $\gr\subseteq (E\times E)$ is an irreflexive symmetric relation, called the
\emph{conflict} relation, satisfying the property: $\forall e, e', e''\in E: e \grr e'\leq
e''\Rightarrow e \grr e''$ (\emph{conflict hereditariness}).
\end{enumerate}
\end{definition}



\begin{definition}[Flow Event Structure]\mylabel{fes} A flow event structure {\rm (FES)}
    is a  tuple $S=(E,\prec,\gr)$ where:
    \begin{enumerate}
\item\mylabel{fes1} $E$ is a denumerable set of events;
\item\mylabel{fes2} $\prec\,\subseteq (E\times E)$ is an irreflexive relation,
called the \emph{flow} relation;
\item\mylabel{fes3} $\gr\subseteq (E\times E)$ is a symmetric relation, called the
\emph{conflict} relation.
\end{enumerate}
\end{definition}
%
Note that the flow relation is not required to be transitive, nor
acyclic (its reflexive and transitive closure is just a preorder, not
necessarily a partial order).  Intuitively, the flow relation
represents a possible {\sl direct causality} between two events.
%Observe also that 
 Moreover,  in a FES the conflict relation is not required to be
irreflexive nor hereditary; indeed, FESs may exhibit self-conflicting
events, as well as disjunctive causality (an event may have
conflicting causes).




Any PES $S = (E , \leq, \gr)$ may be regarded as a FES, with $\prec$
given by $<$ (the strict ordering) % $<$, or by
or by the covering relation of $\leq$.

\bigskip

We now recall the definition of {\sl configuration}\/ for event
structures. Intuitively, a configuration is a set of events having
occurred at some stage of the computation.   Thus, the semantics
of an event structure $S$ is given by its poset of configurations ordered
by set inclusion, where $\ESet_1 \subset \ESet_2$ means that $S$ may evolve
from $\ESet_1$ to $\ESet_2$. 
%
\begin{definition}[PES configuration] \mylabel{configP}
  Let $S=(E,\leq, \gr)$ be a prime event structure. A
    configuration of $S$ is a finite subset $\ESet$ of $E$ such that:
\begin{enumerate}
    \item \mylabel{configP1} $\ESet$ is downward-closed: \ $e'\leq e \in \ESet \, \ \impl \ \ e'\in \ESet$; 
 \item \mylabel{configP2} $\ESet$ is conflict-free: $\forall e, e' \in \ESet, \neg (e \gr
e')$.
\end{enumerate}
\end{definition}
%
The definition of
configuration for FESs is slightly more elaborated.
%
% Since flow event structures are rather general, the definition of
% configuration is slightly more elaborated than for prime event
% structures. 
%
% Let $\Cfree_S$ be the set of conflict-free, or consistent,
% sets of events: $X \in \Cfree_S$ iff $\forall e, e' \in X, \neg (e \gr
% e')$.  Obviously, an event $e$ is inconsistent, i.e. $e \gr e$, if and
% only if $e \not\in X$ for any $X \in \Cfree_S$. 
For a subset $\ESet$ of
$E$, let $\prec_\ESet$ be the restriction of the flow relation to $\ESet$ and
$ \prec_\ESet^*$ be its transitive and reflexive closure.
% $\leq_X =_{def} {\prec_X}^*$ be the preorder 
% %\PGComm{cosa vuoi dire? la chiusura transitiva? o sia riflessiva che
% %transitiva?} 
% generated by $\prec_X$.  
% Here we shall only deal with {\sl finite} configurations,
% which are enough to determine the whole domain associated with an
% event structure, see~\cite{BC91}. We recall the definition:
%
\begin{definition}[FES configuration]
\mylabel{configF}
Let $S=(E,\prec, \gr)$ be a flow event
structure. A configuration of $S$ is a finite
subset $\ESet$ of $E$ such that:
\begin{enumerate}
\item\mylabel{configF1} $\ESet$ is downward-closed up to conflicts: 
\ $e'\prec e \in \ESet, \ e'\notin \ESet\, \  \impl \ 
\,\exists \, e''\in \ESet.\,\, e'\gr \,e''\prec e$;
\item\mylabel{configF2} $\ESet$ is conflict-free: $\forall e, e' \in \ESet, \neg (e \gr
e')$;
\item\mylabel{configF3} $\ESet$ has no causality cycles: the relation $ \prec_\ESet^*$  is a partial order.
\end{enumerate}
\end{definition}
%
Condition (\ref{configF2}) is the same as for prime event
structures. Condition (\ref{configF1}) is adapted to account for the more general
-- non-hereditary -- conflict relation. 
It states that any event appears in a configuration with a ``complete
set of causes''.  Condition (\ref{configF3}) ensures that any event in a
configuration is actually reachable at some stage of the computation.

\bigskip

If $S$ is a prime or flow event structure, we denote by $\Conf{S}$ its
set of 
%finite 
 configurations.   Then, the \emph{domain of
  configurations} of $S$ is defined as follows:
%
\begin{definition}[ES configuration domain]
\mylabel{configDom}
Let $S$ be a prime or flow event
structure with set of configurations $\Conf{S}
 $. The \emph{domain of configurations} of $S$ is the partially ordered set
\mbox{$\CD{S} \eqdef (\Conf{S}, \subseteq)$.}
\end{definition}
%\bma
We recall from~\cite{BC91} a useful characterisation for
configurations of FESs, which is based on the notion of proving
sequence, defined as follows:

%\newpage

\begin{definition}[Proving sequence]
  \mylabel{provseq} Given a flow event structure $S=(E,\prec, \gr)$, a
  \emph{proving sequence} in $S$ is a sequence $\Seq{e_1;
    \cdots}{e_n}$ of distinct non-conflicting events (i.e. $i\not=j\
  \impl\ e_i\not=e_j$ and $\neg(e_i\gr e_j)$ for all $i,j$) satisfying:
 % \\ \centerline{$
  \[
  \forall i\leq n\,\forall e \in E \,: \quad e\prec e_i\
    \ \impl\ \ \exists j<i\,. \ \ \text{ either } \ e = e_j\ \text{ or
    } \ e\grr e_j \prec e_i 
    %$}
    \]
\end{definition}

Note that any prefix of a proving sequence is itself a proving
sequence. 

\bigskip

We have the following characterisation of configurations of FESs  
in terms of proving sequences.
%
\begin{proposition}[Representation of FES configurations as proving
  sequences~\cite{BC91}]
  \mylabel{provseqchar} Given a flow event structure $S=(E,\prec,
  \gr)$, a subset $\ESet$ of $E$ is a configuration of $S$ if and only
  if it can be enumerated as a proving sequence $\Seq{e_1;
    \cdots}{e_n}$.
\end{proposition}
%
Since PESs may be viewed as particular FESs, we may use
\refToDef{provseq} and \refToProp{provseqchar} both for the FESs
associated with networks (see Sections~\ref{sec:netS-ES})
%and~\ref{sec:netA-ES}) 
and for the PESs associated with global types
(see \refToSection{sec:events}).  Note that for a PES
the condition of \refToDef{provseq} simplifies to
%\\
%\centerline{$
\[
\forall i\leq n\,\forall e \in E \,: \quad e < e_i\ \
  \impl\ \ \exists j<i\,. \ \ e = e_j 
%  $}
\]

\medskip

To conclude this section, we recall from~\cite{CZ97} the definition of
\emph{downward surjectivity} (or \emph{downward-onto}, as it was
called there), a property that is required for partial functions
between two FESs in order to ensure that they preserve configurations.
We will make use of this property in \refToSection{sec:netS-ES}.


\begin{definition}[Downward surjectivity]
  \mylabel{down-onto} Let $S_i=(E_i,\prec_i, \gr_i)$, be a flow event
  structure, $i=0,1$. 
%Let $e,e'$ range over $E_0$ and $\procev, \procev'$ range over $E_1$. 
Let $e_i,e'_i$ range over $E_i$, $i=0,1$.  
A partial function $f: E_0
  \rightarrow_* E_1$ is \emph{downward surjective} if it satisfies the
  condition:
\[e_1 \prec_1 f(e_0) \implies \exists e'_0 \in E_0~.~e_1 = f(e'_0) \]
%\[\procev \prec_1 f(e) \implies \exists e' \in E_0~.~\procev = f(e') \]
 \end{definition}

