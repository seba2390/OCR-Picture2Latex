% !TEX root =cdgS.tex

\section{Event Structure Semantics of Global Types}
\mylabel{sec:events}

 We define now the event structure associated with a global type, 
 which will be a PES whose events
are equivalence classes of particular traces. 


 We recall that a trace $\comseq \in \Comseq$ is a finite sequence of 
communications (see \refToDef{traces}). 
We will use the following notational conventions:

\begin{itemize}
\item We denote by 
  $\at{\comseq}{i}$ the $i$-th
  element of  $\comseq$,  $i > 0$. 
\item If $i \leq j$, we define $\range{\comseq}{i}{j} =
  \at{\comseq}{i} \cdots \at{\comseq}{j}$ to be the subtrace of
  $\comseq$ consisting of the $(j-i+1)$ elements starting from the
  $i$-th one and ending with the $j$-th one.  If $i > j$, we convene
  $\range{\comseq}{i}{j}$ to be the empty trace $\ee$.
\end{itemize}
If not otherwise stated we assume that $\comseq$ has $n$
elements, so $\comseq=\range{\comseq}{1}{n}$. 

\bigskip

We start by defining an equivalence relation on $\Comseq$ which allows
swapping of communications with disjoint participants.


\begin{definition}[Permutation equivalence]\mylabel{def:permEq}
The permutation equivalence on
$\Comseq$ is the least equivalence $\sim$ such that
%\\
%\centerline{$
\[
\concat{\concat{\concat{\comseq}{\alpha}}{\alpha'}}{\comseq'}\,\sim\,\,
\concat{\concat{\concat{\comseq}{\alpha'}}{\alpha}}{\comseq'}
\quad\text{if}\quad \participant{\alpha}\cap\participant{\alpha'}=\emptyset
\]
%$}
We denote by $\eqclass{\comseq}$ the equivalence class of the  trace 
$\comseq$, and by $\quotient$ the set of equivalence classes on
$\Comseq$. Note that $\eqclass{\emptyseq} = \set{\emptyseq}\in
\quotient$, and $\eqclass{\alpha}= \set{\alpha} \in \quotient$ for any
$\alpha$.  Moreover $\eh{\comseq'}=\eh{\comseq}\,$ for all
$\comseq'\in\eqclass{\comseq}$. 
\end{definition}
%
The events associated with a global type, 
called  \emph{
  g-events}  and denoted by $\globev, \globev'$, are equivalence classes of
particular  traces  that we call \emph{pointed}.
Intuitively, in a pointed  trace all communications  but the last one
are causes of
%the last one.
 some subsequent communication. 
Formally:
\begin{definition}[Pointed  trace]\mylabel{pcs}
A  trace   $\comseq = \range{\comseq}{1}{n}$ is said to be \emph{pointed} if
%\\
%\centerline{
\[
 \mbox{~~for all $i$, $1\leq i<n$,
$\,\participant{\at{\comseq}i}\cap\participant{\range\comseq{(i+1)}n}\not=\emptyset
$}
\]
%}
\end{definition}
%
Note that the condition of \refToDef{pcs} must be satisfied only by the 
$\at{\comseq}i$ with $i < n$, thus it is vacuously satisfied by any
 trace   of length 1. 
\begin{example}\mylabel{exg6}
  Let $\alpha_1= \Comm\pp{\la_1}\q, \,\alpha_2= \Comm\pr{\la_2}\ps$ and
  $\alpha_3= \Comm\pr{\la_3}\pp$.  Then $\sigma_1 = \alpha_1$ 
  and $\sigma_3 =
  \concat{\concat{\alpha_1}{\alpha_2}}{\alpha_3}\,$ are pointed
   traces,  while  $\sigma_2 = \concat{\alpha_1}{\alpha_2}\,$ is
  \textit{not} a pointed trace.
\end{example}

We use $\last{\comseq}$ to denote the last communication of
$\comseq$.  


\begin{lemma}\mylabel{ma1}
Let $\comseq$ be a pointed  trace.  If  $\comseq \sim \comseq'$, then $\comseq'$ is a
 pointed  trace   and $\last{\comseq}=\last{\comseq'}$.  
\end{lemma}
\begin{proof}
  Let $\comseq \sim \comseq'$. By \refToDef{def:permEq} $\comseq'$ is
  obtained from $\comseq$ by $m$ swaps of adjacent communications. The
  proof is by induction on such a number 
 $m$. \\
  If $m=0$ the result is obvious.\\
  If $m>0$, then there exists $\comseq_0$ obtained from $\comseq$ by
  $m-1$ swaps of adjacent communications and there are  
$\comseq_1$,  $\comseq_2$,  $\alpha$ and $\alpha'$ such that
%  \\
%  \centerline{$
\[
  \comseq_0 =
    \concat{\concat{\concat{\comseq_1}{\alpha}}{\alpha'}}{\comseq_2}\,\sim\,\,
    \concat{\concat{\concat{\comseq_1}{\alpha'}}{\alpha}}{\comseq_2}
    = \comseq'\ \ \mbox{and}\
    \ \participant{\alpha}\cap\participant{\alpha'}=\emptyset
    \]
  %  $.}  
   By
  induction hypothesis $\comseq_0$ is a pointed  trace  
  and $\last{\comseq}=\last{\comseq_0}$.  Therefore
  $\comseq_2\neq\emptyseq$ since otherwise $\alpha'$ would be the
  last communication of $\comseq_0$ and it cannot be
  $\participant{\alpha}\cap \participant{\alpha'}=\emptyset$.\
  This implies $\last{\comseq}=\last{\comseq'}$.\\
  To show that $\comseq'$ is pointed, since all the communications in
  $\comseq_1$ and $\comseq_2$ have the same successors in
  $\comseq_0$ and $\comseq'$, all we have to prove is  that the
  required property holds for the two swapped communications
  $\alpha'$ and $\alpha$ in $\comseq'$, namely:
\[
\begin{array}{l}
 \participant{\alpha'}\cap(\participant{\alpha}\cup\participant{\comseq_2})\not=\emptyset\\[2pt]
 \participant{\alpha}\cap\participant{\comseq_2}\not=\emptyset\\[2pt]
\end{array}
\]
Since $\participant{\alpha}\cap\participant{\alpha'}=\emptyset$, these
two statements are respectively equivalent to:
\[
\begin{array}{l}
\participant{\alpha'}\cap\participant{\comseq_2}\not=\emptyset \\[2pt]
\participant{\alpha}\cap(\participant{\alpha'}\cup\participant{\comseq_2})\not=\emptyset\\[2pt]
\end{array}
\]
The last two statements are known to hold since $\comseq_0$ is pointed
by induction hypothesis.
\end{proof}



\begin{definition}[Global event]\mylabel{def:glEvent}
Let $\comseq = \concat{\comseq'}{\alpha}\,$ be a pointed
 trace.  
Then $\gamma =
\eqclass{\comseq}$ is a \emph{global event}, also called \emph{g-event}, with communication
$\alpha$, notation  $\comm\gamma=\alpha$.\\
 We denote by $\GEs$ the set of g-events.\end{definition}

Notice that  ${\sf cm}(\cdot)$  is well defined due
to \refToLemma{ma1}.  
 
 \bigskip

 We now introduce an operator called ``retrieval'', which
 applied to a communication $\alpha$ and a g-event $\comocc$, yields
 the g-event corresponding to $\comocc$ before the communication
 $\alpha$ is executed. 
  
  
  \begin{definition}[Retrieval of g-events before communications]\text{~}\\[-15pt]
  \mylabel{causal-path} 
    \begin{enumerate}
    \item\mylabel{causal-path1} The  {\em retrieval operator} $\circ$ applied to a
  communication and  a g-event  
       is defined by
%       \\
%      \centerline{$
\[
      \cau{\alpha}\eqclass{\comseq}=\begin{cases}
          \eqclass{\concat{\alpha}{\comseq}} & \text{if~
            $\participant{\alpha}\cap\participant{\comseq} \neq \emptyset$}\\
  %$\eqclass{\concat{\Comm{\pp}{\M}{\q}}\comseq}$ is a global event}, \\
   \eqclass{\comseq}  & \text{otherwise}
\end{cases}
\]
%$}
\item\mylabel{causal-path2} The operator $\circ$ naturally extends to  nonempty traces 
%\\
%\centerline{$
\[
\cau{(\concat\alpha\comseq)}\comocc=\cau\alpha{(\cau\comseq\comocc)}\qquad\comseq\not=\emptyseq
\]
%$}
  \end{enumerate}
\end{definition} 

Using the retrieval, we can define the mapping ${\sf ev}(\cdot)$  which,
 applied to a trace $\comseq$, gives the g-event representing
the communication $\last\comseq$ prefixed by its causes occurring in
$\comseq$. 
\begin{definition}
\mylabel{causal-path3}    The {\em g-event generated by a  trace  } 
 %$\concat\comseq\alpha$, notation $\ev{\concat\comseq\alpha}$, 
 is defined by:
% \\
% \centerline{$
\[
 \ev{\concat\comseq\alpha}=\cau\comseq{\eqclass\alpha}
 \]
 %$}
 \end{definition}
 Clearly $\comm{\ev\comseq}=\last\comseq$.

\bigskip

 We proceed now to define the causality and conflict relations on
g-events. 
To define the conflict relation, % between g-events,
 it is handy to define the projection of a  trace  on a participant, which 
gives the sequence of the participant's actions  in the trace.  

\begin{definition}[Projection]
\mylabel{def:projection}
\begin{enumerate}
\item The {\em projection} of $\alpha$ onto $\pr$, %notation 
$\projS\alpha\pr$, is defined by:
%\\
%\centerline{$
\[
\projS{\Comm\pp\la\q}\pr=\begin{cases}
      \sendL\q\la & \text{if }\pr=\pp\\
      \rcvL\pp\la & \text{if }\pr=\q\\
      \ee& \text{if }\pr\not\in\set{\pp,\q}
  \end{cases}
  \]
  %$} 
\item The projection of a  trace  $\comseq$ onto $\pr$, $\projS\comseq\pr$, is defined by:
%\\
%\centerline{$
\[
\projS{\ee}\pr=\ee\quad\quad
\projS{(\alpha\cdot\comseq)}\pr=\projS\alpha\pr\cdot\projS{\comseq}\pr
\]
%$} 
\end{enumerate}
\end{definition}



\begin{definition}[Causality and conflict relations on g-events] \label{sgeo}
The {\em causality} relation $\precP$ and the {\em conflict} relation $\gr$ on the set of g-events $\GEs$ are defined by:
\begin{enumerate}
\item\mylabel{sgeo1} $\comocc\precP\comocc'$
~if~$\comocc=\eqclass\comseq$ and $\comocc'=\eqclass{\concat\comseq{\comseq'}}$ for some $\comseq,\comseq'$;
 %$\eqclass{\comseq}\,\leq \,\eqclass{\comseq'}$ 
\item\mylabel{sgeo2}
  $\eqclass{\comseq}\grr\eqclass{\comseq'}$~if~$\projS\comseq\pp\grr\projS{\comseq'}\pp$
  for some $\pp$.
\end{enumerate}
\end{definition}
\noindent If
$\comocc=\eqclass{\concat{\concat\comseq{\alpha}}{\concat{\comseq'}{\alpha'}}}$,
then the communication $\alpha$ must be done before the communication
$\alpha'$. This is expressed by the causality
$\eqclass{\concat\comseq{\alpha}}\precP\comocc$.  An example is
$\eqclass{\Comm\pp\la\q}\precP\eqclass{\concat{\Comm\pr{\la'}\ps}{\concat{\Comm\pp\la\q}{\Comm\ps{\la''}\q}}}$. \\
As regards conflict, note that if $\comseq\sim\comseq'$ then
$\projS\comseq\pp=\projS{\comseq'}\pp$ for all $\pp$, because $\sim$
does not swap communications which share some participant.  Hence,
conflict is well defined, since it does not depend on the trace chosen
in the equivalence class. The condition
$\projS\comseq\pp\gr\projS{\comseq'}\pp$ states that participant $\pp$
does the same actions in both traces up to some point, after which it
performs two different actions in $\comseq$ and $\comseq'$.  For example $\eqclass{\concat{\concat{\Comm\pp\la\q}{\Comm\pr{\la_1}\pp}}{\Comm\q{\la'}\pp}}\gr  \eqclass{\concat{\Comm\pp\la\q}{\Comm\pr{\la_2}\pp}}$, since 
$\projS{(\concat{\concat{\Comm\pp\la\q}{\Comm\pr{\la_1}\pp}}{\Comm\q{\la'}\pp})}\pp=
   \concat{\concat{\sendL\q\la}{\rcvL\pr{\la_1}}}{\rcvL\q{\la'}}\gr\concat{\sendL\q\la}{\rcvL\pr{\la_2}}=
\projS{(\concat{\Comm\pp\la\q}{\Comm\pr{\la_2}\pp})}\pp$.

\begin{definition}[Event structure of a global type] \mylabel{eg}The {\em
    event structure of  the global type
    } 
    $\GP$ is the triple
%    \\
% \centerline{$
\[
 \ESG{\GP} = (\EGG(\GP), \precP_\GP , \grr_\GP)
 \]
% $} 
 where:
\begin{enumerate}
\item\mylabel{eg1a} 
 $\EGG(\GP) =
\set{ \ev{\comseq}\ |\ \comseq\in\FPaths{\G}}$  
\item\mylabel{eg2} $\precP_\GP$ is the restriction of $\precP$ to the set $\EGG(\GP)$;
\item\mylabel{eg3} $\gr_\GP$ is the restriction of $\gr$ to the set $\EGG(\GP)$.
\end{enumerate}
\end{definition}
 Note that, %due to Clause~\ref{eg1d} of  \refToDef{eg},  
 in case the tree of $\G$ is infinite, the set $\EGG(\G)$  is  
 denumerable.  
\begin{example}\mylabel{ex:eventsGT}
  Let
  $\G_1=\Seq{\gtCom\pp\q{\la_1}}{\Seq{\gtCom\pr\ps{\la_2}}{\gtCom\pr\pp{\la_3}}}$
  and
  $\G_2=\Seq{\gtCom\pr\ps{\la_2}}{\Seq{\gtCom\pp\q{\la_1}}{\gtCom\pr\pp{\la_3}}}$.
  Then $\EGG(\G_1)=\EGG(\G_2)=\set{\comocc_1,\comocc_2,\comocc_3}$
  where
   %\\ \centerline{$
  \[
  \comocc_1=\{\Comm\pp{\la_1}\q\}\qquad
    \comocc_2=\{\Comm\pr{\la_2}\ps\}\qquad
    \comocc_3=\{\concat{\concat{\Comm\pp{\la_1}\q}{\Comm\pr{\la_2}\ps}}{\Comm\pr{\la_3}\pp},
    \concat{\concat{\Comm\pr{\la_2}\ps}{\Comm\pp{\la_1}\q}}{\Comm\pr{\la_3}\pp}
    \}
    \]
    %$} 
    with $\comocc_1\precP\comocc_3$ and
  $\comocc_2\precP\comocc_3$. The configurations are
  $\set{\comocc_1}$, $\set{\comocc_2}$,   $\set{\comocc_1,\comocc_2}$    and
  $\set{\comocc_1,\comocc_2,\comocc_3}$, and the proving sequences
  are
 % \\ \centerline{$
 \[
  \comocc_1\qquad \comocc_2\qquad
    \Seq{\comocc_1}{\comocc_2}\qquad
    \Seq{\comocc_2}{\comocc_1}\qquad
    \Seq{\Seq{\comocc_1}{\comocc_2}}{\comocc_3}\qquad
    \Seq{\Seq{\comocc_2}{\comocc_1}}{\comocc_3}
    \]
%    $}
  If $\G'$ is as in \refToExample{exg}, then
  $\EGG(\G')=\set{\comocc_1,\comocc_2,\comocc_3}$ where
%  \\
%  \centerline{$
\[
  \comocc_1=\{\Comm\pp{\la_1}\q\}\qquad
    \comocc_2=\{\concat{\Comm\pp{\la_1}\q}{\Comm\q{\la_2}\pr}\}\qquad
    \comocc_3=\{\concat{\concat{\Comm\pp{\la_1}\q}{\Comm\q{\la_2}\pr}}{\Comm{ \pr }{\la_3}{ \ps }}\}
    \]
%$} 
with
  $\comocc_1\precP\comocc_2\precP\comocc_3$. The configurations are
  $\set{\comocc_1}$, $\set{\comocc_1,\comocc_2}$ and
  $\set{\comocc_1,\comocc_2,\comocc_3}$,  and there is a unique
   proving sequence
  corresponding to each configuration. \end{example}
%
\begin{theorem}\mylabel{basta12}
Let $\GP$ be a  global type. 
Then  $\ESG{\GP}$  is a prime event structure.
\end{theorem}

\begin{proof}
  We show that $\precP$ and $\gr$ satisfy Properties (\ref{pes2}) and
  (\ref{pes3}) of \refToDef{pes}. Reflexivity and transitivity of
  $\precP$ follow from the properties of concatenation and of
  permutation equivalence. As for antisymmetry, by
  \refToDef{sgeo}(\ref{sgeo1}) $\eqclass{\comseq}\,\leq
  \,\eqclass{\comseq'}$ implies $\comseq' \sim
  \concat{\comseq}{\comseq_1}$ for some $\comseq_1$ and
  $\eqclass{\comseq'}\,\leq \,\eqclass{\comseq}$ implies $\comseq \sim
  \concat{\comseq'}{\comseq_2}$ for some $\comseq_2$.  Hence $\comseq
  \sim \concat{\concat{\comseq}{\comseq_1}}{\comseq_2}$, which implies
  $\comseq_1 = \comseq_2 = \ee$.  Irreflexivity and symmetry of $\gr$
  follow from the corresponding properties of $\gr$ on p-events. \\ 
    As for conflict hereditariness, suppose that $\eqclass{\comseq} \grr
  \eqclass{\comseq'}\precP \eqclass{\comseq''}$.  By
  \refToDef{sgeo}(\ref{sgeo1}) and (\ref{sgeo2}) we have respectively
  that $\concat{\comseq'}{\comseq_1}\sim\comseq''$ for some
  ${\comseq_1}$ and $\projS\comseq\pp\grr\projS{\comseq'}\pp$ for some
  $\pp$.  Hence also
  $\projS\comseq\pp\grr\projS{(\concat{\comseq'}{\comseq_1})}\pp$,
  whence by \refToDef{sgeo}(\ref{sgeo2}) we conclude that
  $\eqclass{\comseq} \grr \eqclass{\comseq''}$. 
\end{proof}



Observe that while our interpretation of networks as FESs exactly
reflects the concurrency expressed by the syntax of networks, our
interpretation of global types as PESs exhibits more concurrency than
that given by the syntax of global types.

