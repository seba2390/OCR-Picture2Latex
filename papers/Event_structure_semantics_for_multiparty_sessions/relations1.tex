% !TEX root =cdgS.tex

\section{Equivalence of the two Event Structure Semantics}\label{sec:results}




\begin{figure}[h]
\centering\footnotesize
\begin{minipage}{8cm}
\xymatrix{
&&\ar@{.>}[dl]_{\raisebox{1ex}{\small Th.8.8}}
\netev_1;\ldots;\netev_n=\nec{\comseq}&&\\
\Nt\ar@{-}[r] &\ar@{.>}[d]_{\raisebox{1ex}{\small SR}}{\comseq=\comm{\netev_1}\cdot\ldots\cdot\comm{\netev_n}}\ar[rrr]&&\ar@{.>}[ul]_{\raisebox{1ex}{\small Th.8.7}}& \Nt'\\ 
\G\ar@{-}[rrr] &\ar@{.>}[dr]_{\raisebox{1ex}{\small Th.8.15~~~~}}&&{\comseq=\comm{\globev_1}\cdot\ldots\cdot\comm{\globev_n}}\ar[r]\ar@{.>}[u]_{\raisebox{1ex}{\small SF}}& \G'\\ 
&&\gec{\comseq}=\globev_1;\ldots;\globev_n \ar@{.>}[ur]_{\raisebox{1ex}{\small ~~~~Th.8.16}}&&
}
\end{minipage}
\caption{Isomorphism proof in a nutshell.}\label{ipn}\end{figure}


In this section we establish our main result for typable networks,
namely the isomorphism between the domain of configurations of the FES
of such a network and the domain of configurations of the PES of its
global type.   To do so, we will first relate the transition
sequences of networks and global types to the configurations of their
respective ESs. Then, we will exploit our results of Subject Reduction
(\refToTheorem{sr}) and Session Fidelity (\refToTheorem{sf}), which
relate the transition sequences of networks and their global types, to
derive a similar relation between the configurations of their
respective ESs. The schema of our proof is described by the 
diagram in \refToFigure{ipn}. 
%where starting from the top line we prove the bottom line and
%viceversa, using the results mentioned on the various arrows. 
Here, SR
stands for Subject Reduction and SF for Session Fidelity,  $\netev_1;\ldots;\netev_n$ and $\globev_1;\ldots;\globev_n$ are proving sequences of $\ESN\Nt$ and $\ESG\G$, respectively.
Finally 
$\nec{\comseq}$ and $\gec{\comseq}$ denote   the proving sequence
of n-events and the proving sequence of g-events corresponding to the trace
$\comseq$ (as given in \refToDef{nec} and \refToDef{gecdef}).  
\refToTheorem{uf12} says that, if
$\Seq{{\netev_1};\cdots}{\netev_n}$ is a proving sequence of
$\ESN{\Nt}$, then $\Nt\stackred\comseq\Nt'$, where $\comseq=\comm{\netev_1}\cdot\ldots\cdot\comm{\netev_n}$. By Subject Reduction (\refToTheorem{sr})
$\G\stackred\comseq\G'$. This implies  
that $\gec{\comseq}$ is a proving
sequence of $\ESG{\RG}$ by \refToTheorem{uf13}. Dually, \refToTheorem{uf14}  says that, if
$\Seq{{\globev_1};\cdots}{\globev_n}$ is a proving sequence of
$\ESG{\G}$, then $\G\stackred\comseq\G'$, where $\comseq=\comm{\globev_1}\cdot\ldots\cdot\comm{\globev_n}$.  By Session Fidelity (\refToTheorem{sf})  $\Nt\stackred\comseq\Nt'$. Lastly $\nec{\comseq}$ is a proving
sequence of $\ESN{\Nt}$ by \refToTheorem{uf10}. The equalities in the top and bottom lines are proved in Lemmas~\ref{ecn}(\ref{ecn22}) and~\ref{ecg}(\ref{ecg1}). 
 


 This section is divided in two subsections: \refToSection{lozenges},
 which handles the upper part of the above diagram, and
 \refToSection{bullets}, which handles the lower part of the diagram and
 then connects the two parts using both SR and SF within
 \refToTheorem{iso}, our closing result.


\subsection{Relating Transition Sequences of Networks and Proving
  Sequences of their ESs}\label{lozenges}

 The aim of this subsection is to relate the traces that label
the transition sequences of networks with the configurations of their
FESs.  We start by showing how network communications affect
n-events in the associated ES.  To this end  we define two
partial operators $\lozenge$ and $\blacklozenge$, which applied to a
communication $\alpha$ and an n-event $\netev$ yield another n-event
$\netev'$ (when defined), which represents the event $\netev$ before
the communication $\alpha$ or after the communication $\alpha$,
respectively. We call ``retrieval'' the $\lozenge$ operator (in
agreement with \refToDef{causal-path}) and ``residual'' the
$\blacklozenge$ operator.


Formally, the operators $\lozenge$ and $\blacklozenge$ are defined as follows. 


  
\begin{definition}[Retrieval and residual of n-events with
  respect to communications]\text{~}\\[-15pt]\mylabel{def:PostPre}
\begin{enumerate}
\item\mylabel{def:PostPre1} The {\em retrieval operator} $\lozenge$ applied to a
  communication and a  located event  returns the located event
  obtained
  by  prefixing the process event by the projection of the communication:
%  \\
%  \centerline{$
\[
  \preP{\locev{\pp}{\event}}\alpha=\locev{\pp}{\concat{(\projS{\alpha}{\pp})}\event}
  \]
%$}
\item\mylabel{def:PostPre2} The {\em residual operator} $\blacklozenge$ applied to a
  communication and a located event returns the located event obtained
  by erasing
  from the process event the projection of the communication (if possible):
%  \\
%  \centerline{$
\[
  \postP{\locev{\pp}\event}\alpha=%\begin{cases}
    \locev{\pp}{\event'}
      %& 
      \quad
      \text{if }\event=
        \concat{(\projS{\alpha}{\pp})}{\event'}
        \]
 % \end{cases}
 % $} 


  \item\mylabel{def:PostPre3} The operators $\lozenge$ and
    $\blacklozenge$ naturally extend to n-events  and to traces:
%    \\
% \centerline{$
\[
\begin{array}{c}
 \preP{\set{\locev{\pp}{\event},\locev{\q}{\event'}}}{\alpha}=
  \set{\preP{\locev{\pp}{\event}}{\alpha},\preP{\locev{\q}{\event'}}{\alpha}}
  %$}
  \\
%\centerline{$
\postP{\set{\locev{\pp}{\event},\locev{\q}{\event'}}}{\alpha}=
  \set{\postP{\locev{\pp}\event}{\alpha},\postP{\locev{\q}{\event'}}{\alpha}}
 % $} 
 \\
 % \centerline{$ 
  \pre\netev{\epsilon}=\netev\qquad
  \pre\netev{(\concat\alpha\comseq)}=\preP{\pre\netev\comseq}\alpha\qquad\post\netev{(\concat\alpha\comseq)}=\postP{\post\netev\alpha}\comseq \qquad \comseq\not=\emptyseq
  \end{array}
  \]
  %$}
    \end{enumerate}
\end{definition}
\noindent
Note that the operator $\lozenge$ is always defined. Instead
$\post{\locev{\pr}\event}{\Comm\pp\la\q}$ is undefined if $\pr \in
\set{\pp,\q}$ and either $\event$ is just one atomic action or
$\projS{\Comm\pp\la\q}\pr$ is not the first atomic action of $\event$.

\bigskip

 The retrieval and residual operators are inverse of each other. Moreover they preserve the flow and conflict relations. %The proof is in the Appendix. 

\begin{lemma}[Properties of 
  retrieval and residual for n-events]\text{~}\\[-10pt]
\mylabel{prop:prePostNet}
\begin{enumerate}
\item \mylabel{ppn1} If $\post{\netev}{\alpha}$ is defined, then
  $\preP{\post{\netev}{\alpha}}{\alpha}=\netev$;
\item \mylabel{ppn1b} 
$\postP{\pre{\netev}{\alpha}}{\alpha}=\netev$;
 \item  \mylabel{ppn2b}  If  $\netev\precN \netev'$, 
then $\pre{\netev}{\alpha}\precN \pre{\netev'}{\alpha}$;
\item \mylabel{ppn2} If  $\netev\precN \netev'$ and both $\post{\netev}{\alpha}$ and
  $\post{\netev'}{\alpha}$ are defined, then
  $\post{\netev}{\alpha}\precN \post{\netev'}{\alpha}$; 
\item \mylabel{ppn3b} If  $\netev\grr \netev'$, then $\pre{\netev}{\alpha}\grr
  \pre{\netev'}{\alpha}$;
\item \mylabel{ppn3} If  $\netev\grr
  \netev'$ and both $\post{\netev}{\alpha}$ and $\post{\netev'}{\alpha}$
  are defined, then $\post{\netev}{\alpha}\grr\post{\netev'}{\alpha}$;
  \item \mylabel{ppn7} If  $\pre{\netev}{\alpha}\grr
  \pre{\netev'}{\alpha}$, then $\netev\grr \netev'$.
%\item \mylabel{ppn8} If  $\post{\netev}{\alpha}\grr\post{\netev'}{\alpha}$, then $\netev\grr\netev'$.
\end{enumerate}
\end{lemma}


Starting from the trace $\comseq \neq \ee$  that labels a transition
sequence  in a network, one can
reconstruct the corresponding sequence of n-events in its 
FES.  Recall
that  $\range\comseq1{i}$ is the prefix of length $i$ of 
%$\range\comseq1{i-1}$ is the prefix of length $(i-1)$ of
$\comseq$ and $\range{\comseq}{i}{j}$ is the empty trace if $i\geq
j$.

%\newpage

\begin{definition}[Building sequences of n-events from traces]
%[n-events from  communications]
  \mylabel{nec} If $\comseq$ is a trace with
  $\at\comseq{i}=\Comm{\pp_i}{\la_i}{\q_i}$, $1\leq i\leq n$, we
  define
  the {\em sequence of n-events corresponding to $\comseq$} by
%  \\
%  \centerline{$
  \[
  \nec{\comseq}=\Seq{\netev_1;\cdots}{\netev_n}
  \]
 % $} 
 where
  $\netev_i=\pre{\set{\locev{\pp_i}{\sendL{\q_i}{\la_i}},\locev{\q_i}{\rcvL{\pp_i}
        {\la_i}}}}{\range\comseq1{i-1}}$ for $1\leq i\leq n$.
  \end{definition}
   
  It is immediate to see that, if $\comseq=\Comm{\pp}{\la}{\q}$, then
  $\nec{\comseq}$ is the event
  $\set{\locev{\pp}{\sendL{\q}{\la}},\locev{\q}{\rcvL{\pp}{\la}}}$.

\bigskip
%
% We prove next some expected properties of the sequence
%$\nec{\comseq}$. 
 We show now that two n-events occurring in $\nec{\comseq}$ cannot be in conflict  and that from $\nec{\comseq}$ we can recover $\comseq$. Moreover we relate the retrieval and residual operators with the mapping $\nec{\cdot}$. 

\begin{lemma}[Properties of $\nec{\cdot}$]\mylabel{ecn}\text{~}\\[-10pt]
\begin{enumerate}
\item\mylabel{ecn2} Let
  $\nec{\comseq}=\Seq{\netev_1;\cdots}{\netev_n}$.  Then 
\begin{enumerate}
 \item\mylabel{ecn22} $\comm{\netev_i}=\at\comseq{i}$  for all $i$, $1\leq i\leq n$;
\item\mylabel{ecn21} If $1\leq h,k\leq n$, then $\neg(\netev_h\gr\netev_k)$.
  \end{enumerate}
\item\mylabel{ecn1} $\neg(\nec\alpha\grr\pre\netev\alpha)$ for all $\netev$.
\item\mylabel{ecn3} Let $\comseq=\alpha\cdot\comseq'$ and
  $\comseq'\neq \ee$.  If
  $\nec{\comseq}=\Seq{\netev_1;\cdots}{\netev_n}$ and
  $\nec{\comseq'}=\Seq{\netev'_2;\cdots}{\netev'_n}$, then
  $\pre{\netev'_i}\alpha=\netev_i$ and
  $\post{\netev_i}\alpha=\netev_i'$ for all $i$, $2\leq i\leq n$.
 \end{enumerate}
  \end{lemma}
%
  \begin{proof}
  (\ref{ecn22}) 
 Immediate  from \refToDef{nec},  since
$\comm{\pre{\netev}{\sigma}}=\comm{\netev}$ for any event $\netev$. 

(\ref{ecn21}) We show that neither Clause (\ref{c21}) nor
Clause (\ref{c22}) of \refToDef{netevent-relations} can be used to
derive $\netev_h\grr\netev_k$.  Notice that $\netev_i=
\set{\locev{\pp_i}{\projS{{\range\comseq1{i}}}{\pp_i}},
  \locev{\q_i}{\projS{{\range\comseq1{i}}}{\q_i}}}$. So if
$\locev{\pp}\procev\in{\netev_h}$ and
$\locev{\pp}{\procev'}\in{\netev_k}$  with $h<k$,  then
 either  $\procev < \procev'$  or $\procev =
\procev'$. 
%otherwise $\procev'  <  \procev$. 
Therefore Clause (\ref{c21}) 
%of \refToDef{netevent-relations} 
does not apply.  If
$\locev{\pp}{\procev}\in{\netev_h}$ and $\locev{\q}{\procev'}\in{\netev_k}$ and 
  $\pp \neq \q$ and $\cardin{\proj\procev\q} =
  \cardin{\proj{\procev'}\pp}$,
  then it must be
  $\dualev{\proj\procev\q =
    \proj{(\projS{\range\comseq1{h}}{\pp})}{\q}}
{\proj{(\projS{\range\comseq1{k}}{\q})}{\pp}
= \proj{\procev'}\pp}$.  Therefore Clause (\ref{c22}) 
%of \refToDef{netevent-relations} 
cannot be used.

     (\ref{ecn1}) We show that neither Clause (\ref{c21}) nor
    Clause (\ref{c22}) of \refToDef{netevent-relations} can be used to
    derive $\nec\alpha\grr\pre\netev\alpha$.  Let
    $\participant{\alpha} = \set{\pp, \q}$. Then $\nec{\alpha}
    =\set{\locev{\pp}{\projS{\alpha}{\pp}},
      \locev{\q}{\projS{\alpha}{\q}}}$. Note that
    $\locev{\pp}{\procev}\in \pre{\netev}{\alpha}$ iff $\procev =
    \concat{(\projS{\alpha}{\pp})}{\procev'}$ and
    $\locev{\pp}{\procev'}\in \netev$. Since $\projS{\alpha}{\pp} <
    \concat{(\projS{\alpha}{\pp})}{\procev'}$, Clause (\ref{c21})
    of \refToDef{netevent-relations}
    cannot be used. Now suppose $\locev{\pr}{\procev}\in
    \pre{\netev}{\alpha}$ for some $\pr \notin \set{\pp, \q}$.  In
    this case $\proj{(\projS{\alpha}{\pp})}{\pr} =
    \proj{(\projS{\alpha}{\q})}{\pr} = \ee$. Therefore,  since
    $\dualev{\ee}{\ee}$, Clause
    (\ref{c22}) of \refToDef{netevent-relations} does not
    apply.  
  

 

(\ref{ecn3}) Notice that $\at{\comseq}{i}=\at{\comseq'}{i-1}$ for all
$i$, $2\leq i\leq n$. Then,
by \refToDef{nec}
%\\
%\centerline{$
\[
\begin{array}{lll}\netev_i &=&\pre{\nec{\at{\comseq}{i}}}{\range\comseq1{i-1}}=
\pre{(\pre{\nec{\at{\comseq}{i}}}{\range\comseq2{i-1}})}\alpha=\\
&&\pre{(\pre{\nec{\at{\comseq'}{i-1}}}{\range{\comseq'}1{i-2}})}\alpha=\pre{\netev'_i}\alpha
%\text{for all $i$, $2\leq i\leq n$}
\end{array}
\]
%$} 
for all $i$, $2\leq i\leq n$.
\\
By
\refToLemma{prop:prePostNet}(\ref{ppn1b})
$\pre{\netev'_i}\alpha=\netev_i$ implies
$\post{\netev_i}\alpha=\netev_i'$ for all $i$, $2\leq i\leq n$.
  \end{proof}
  
%%%%%%%%%%%%%%%
  
 
  
 It is handy to notice that if $\post{\netev}{\alpha}$ is undefined and $\netev$ is an event of a network with communication $\alpha$, then either $\netev=\nec{\alpha}$ or $\netev\grr\nec{\alpha}$. 

\begin{lemma}\mylabel{dc} 
If $\Nt\stackred\alpha\Nt'$ and $\netev\in\GE(\Nt)$, then $\netev=\nec{\alpha}$ or $\netev\grr\nec{\alpha}$ or $\post{\netev}{\alpha}$ is defined.
\end{lemma}


\begin{proof}
Let
  %  $\participant{\alpha} = \set{\pp, \q}$. Then 
$\nec{\alpha} =\set{\locev{\pp}{\projS{\alpha}{\pp}},
  \locev{\q}{\projS{\alpha}{\q}}}$ and
$\netev=\set{\locev{\pr}{\event}, \locev{\ps}{\event'}}$.  By
\refToDef{def:PostPre}(\ref{def:PostPre3}) $\post{\netev}{\alpha}$ is
defined iff $\event= \concat{(\projS{\alpha}{\pr})}{\event_0}$ and
$\event'= \concat{(\projS{\alpha}{\ps})}{\event'_0}$ for some
$\event_0, \event'_0$.\\
There are 2 possibilities:
\begin{itemize}
\item
$\set{\pr,\ps} \cap \set{\pp,\q} = \emptyset$. Then
$\projS{\alpha}{\pr} = \projS{\alpha}{\ps} = \ee$ and
$\post{\netev}{\alpha} = \netev$;
\item $\set{\pr,\ps} \cap \set{\pp,\q} \neq \emptyset$. 
Suppose $\pr = \pp$. There are three possible subcases:
\begin{enumerate}
\item \mylabel{ila1} $\event = \concat{\pi}{\actseq}$ with $\pi \neq \projS{\alpha}{\pp}$.
Then $\locev{\pr}{\event} \grr \locev{\pp}{\projS{\alpha}{\pp}}$ and
thus $\netev\grr\nec{\alpha}$;
%
\item \mylabel{ila2} 
$\event = \projS{\alpha}{\pp}$. Then either
$\event'= \projS{\alpha}{\q}$ and $\netev=\nec{\alpha}$, or
$\event' \neq \projS{\alpha}{\q}$ 
and $\netev\grr\nec{\alpha}$ by
\refToProp{prop:conf};
% \bcomila Qui ho tolto la frase ``(which can be used since
% $\nec{\alpha} \in\GE(\Nt)$ and $\netev\in\GE(\Nt)$)'' visto che ora 
% la \refToProp{prop:conf} vale per due n-eventi qualunque. \ecomila 
\item \mylabel{ila3} 
$\event = \concat{(\projS{\alpha}{\pp})}{\event_0}$. Then 
$\post{\locev{\pp}{\event}}{\alpha} = \locev{\pp}{\event_0}$. Now, if
$\ps \neq \q$ we have $\post{\locev{\ps}{\event'}}{\alpha} =
\locev{\ps}{\event'}$, and thus $\post{\netev}{\alpha} =
\set{\locev{\pp}{\event_0}, \locev{\ps}{\event'}}$.
Otherwise,
$\netev=\set{\locev{\pp}{\concat{(\projS{\alpha}{\pp})}{\event_0}},
  \locev{\q}{\event'}}$. By \refToDef{n-event} $\dualevS{\locev{\pp}{\concat{(\projS{\alpha}{\pp})}{\event_0}}}{\locev{\q}{\event'}}$, which implies 
  $\event'=\concat{(\projS{\alpha}{\q})}{\event'_0}$ for some $\event'_0$.
\end{enumerate}
\end{itemize} 
\end{proof}

 The following lemma, which is technically quite challenging, %and
%therefore is proved in the Appendix, %will be crucial to prove the two forthcoming theorems. 
  relates the n-events of two networks which differ for one communication by means of the retrieval and residual operators. 

\begin{lemma}\label{epp}
Let $\Nt\stackred{\alpha}\Nt'$. Then 
\begin{enumerate}
\item\label{epp1} $\set{\nec\alpha}\cup\set{\pre\netev\alpha\mid\netev\in\GE(\Nt')}\subseteq\GE(\Nt)$; 
\item\label{epp2}  $\set{\post\netev\alpha\mid\netev\in\GE(\Nt)\text{ and }\post\netev\alpha\text{ defined} }\subseteq\GE(\Nt')$. 
\end{enumerate}
\end{lemma}


 We may now prove the correspondence between the traces labelling
the transition sequences of a network and the proving sequences of its
FES.  

\begin{theorem}\mylabel{uf10}  If $\Nt\stackred\comseq\Nt'$, 
  then $\nec{\comseq}$ is a proving sequence in $\ESN{\Nt}$.
\end{theorem}
\begin{proof}
  The proof is by  induction on ${\comseq}$.\\
 \emph{Base case.}  Let $\comseq=\alpha$. 
From
  $\Nt\stackred\alpha\Nt'$ and  \refToLemma{epp}(\ref{epp1})
$\nec{\alpha}\in\GE(\Nt)$. Since $\nec{\alpha}$ has no causes, 
by \refToDef{provseq} we conclude that $\nec{\alpha}$
is a proving sequence in $\ESN{\Nt}$.
%\ecompa
\\
\emph{Inductive case.}
%
\noindent
Let
  $\comseq=\concat{\alpha}{\comseq'}$.
From
$\Nt\stackred{\comseq}\Nt'$ we get
$\Nt\stackred{\alpha}\Nt''\stackred{\comseq'}\Nt'$ for some
$\Nt''$. Let $\nec{\comseq}=\Seq{\netev_1;\cdots}{\netev_{n}}$ and
$\nec{\comseq'}=\Seq{\netev'_2;\cdots }{\netev'_{n}}$.  By induction
$\nec{\comseq'}$ is a proving sequence in 
$\ESN{\Nt''}$.  \\
We show that $\nec{\comseq}$ is a proving sequence in 
$\ESN{\Nt}$.  By
\refToLemma{ecn}(\ref{ecn21}) $\nec{\comseq'}$ is conflict free.  By
\refToLemma{ecn}(\ref{ecn3}) $\netev_i=\pre{\netev_i'}{\alpha}$ for
all $i$, $2\leq i\leq n$. This implies $\netev_i\in\GE(\Nt)$ for all
$i$, $2\leq i\leq n$ by \refToLemma{epp}(\ref{epp1}) and
$\neg{(\netev_1\gr\netev_j)}$ for all $i,j$, $2\leq i,j\leq n$ by
\refToLemma{prop:prePostNet}(\ref{ppn7}).   Finally, since
$\netev_1=\nec{\alpha}$,  by \refToLemma{ecn}(\ref{ecn1}) we
obtain $\neg{(\netev_1\gr\netev_i)}$ for all $i$, $2\leq i\leq n$.  We
conclude that $\nec{\comseq}$ is conflict-free  and included in $\GE(\Nt)$.    Let
$\netev\in\GE(\Nt)$ and $\netev\precN \netev_k$ for some $k$, $1\leq
k\leq n$.  This implies $k > 1$  since $\nec{\alpha}$ has no
causes. Hence $\netev_k=\pre{\netev_k'}{\alpha}$.  By \refToLemma{dc},  we know that 
$\netev=\nec{\alpha}$ or $\netev\gr\nec{\alpha}$ or
$\post{\netev}{\alpha}$ is defined.  We consider the three
cases. Let $\participant\alpha=\set{\pp,\q}$. \\  
 {\em Case $\netev = \nec{\alpha}$}. In this case
we conclude immediately since $\nec{\alpha} = \netev_1$ and $1<k$. \\
{\em Case $\netev\gr\nec{\alpha}$}.  Since $\nec{\alpha}=
\netev_1$,  if
$\netev_1\prec\netev_k$  we are done. 
If $\netev_1\not\prec\netev_k$,  then $\loc{\netev_k} \cap \set{\pp,\q} =
\emptyset$   
% 
otherwise $\netev_1\grr\netev_k$. 
% 
We get %and thus 
$\netev_k=\pre{\netev_k'}{\alpha}=\netev_k'$.  Since
$\netev\precN \netev_k$, there exists
%$\pr\not\in\set{\pp,\q}$ such that 
$\locev\pr\procev\in \netev$ and $\locev\pr{\procev'}\in
\netev_k=\netev_k'$ such that $\procev<\procev'$,
where $\pr \notin \set{\pp,\q}$
because $\pr \in \loc{\netev_k}$. 
 Since
$\nec{\comseq'}$ is a proving sequence in  
$\ESN{\Nt''}$,  by
\refToLemma{csl} there is $\netev_h'  \in \GE(\Nt'')$ such that $\locev\pr\procev\in
\netev_h'$.  Since $\pre{\locev\pr\procev}\alpha=\locev\pr\procev$ we
get $\locev\pr\procev\in \netev_h$. This implies
$\netev_h\prec\netev_k$,  where $\netev_h\gr\netev$  by
\refToProp{prop:conf}. 
\\
{\em Case $\post{\netev}{\alpha}$ defined}. We get
$\post{\netev}{\alpha}\precN{\netev'_k}$ by
\refToLemma{prop:prePostNet}(\ref{ppn2}).  Since $\nec{\comseq'}$ is a
proving sequence in $\ESN{\Nt''}$, there is $h<k$ such that either
$\post{\netev}{\alpha}=\netev_h'$ or
$\post{\netev}{\alpha}\grr\netev_h'\prec\netev_k'$.  In the first case
$\netev=\preP{\post{\netev}{\alpha}}{\alpha}=\pre{\netev_h'}{\alpha}=\netev_h$
by \refToLemma{prop:prePostNet}(\ref{ppn1}).  %and (\ref{ppn1b}).
In the second case: \begin{itemize}\item from
  $\post{\netev}{\alpha}\grr\netev_h'$ we get
  $(\preP{\post{\netev}{\alpha}}{\alpha})\grr(\pre{\netev_h'}{\alpha})$
  by
  \refToLemma{prop:prePostNet}(\ref{ppn3b}),
  which implies $\netev\gr\netev_h$ by
  \refToLemma{prop:prePostNet}(\ref{ppn1}), and
\item from $\netev_h'\prec\netev_k'$ we get
  $(\pre{\netev_h'}{\alpha})\prec(\pre{\netev_k'}{\alpha})$ by
  \refToLemma{prop:prePostNet}(\ref{ppn2b}),  namely 
  $\netev_h\prec\netev_k$. \qedhere
\end{itemize}
\end{proof}



\begin{theorem}\mylabel{uf12}
If  
$\Seq{\netev_1;\cdots}{\netev_n}$ is a proving sequence in $\ESN{\Nt}$, 
then $\Nt\stackred\comseq\Nt'$, where $\comseq=\comm{\netev_1}\cdots\comm{\netev_n}$.
\end{theorem}


\begin{proof}
  The proof is by induction on $n$.\\
  Case $n=1$.  Let
  $\netev_1=\set{\locev{\pp}{\actseq\cdot\sendL{\q}{\la}},
\locev{\q}{\actseq'\cdot\rcvL{\pp}{\la}}}$.
   Then  $\comm{\netev_1}=\Comm{\pp}{\la}{\q}$.  We first show
  that $\actseq=\actseq'=\ee$. Assume ad absurdum that
  $\actseq\neq\ee$  or  $\actseq'\neq\ee$. By narrowing, this
  implies that there is $\netev\in\GE(\Nt)$ such that
  $\netev\prec\netev_1$, 
%and $\neg(\netev\gr\netev_1)$. 
 contradicting the fact that $\netev_1$ is a proving sequence.\\ 
By
  \refToDef{netev-relations}(\ref{netev-relations1}) we have
  $\Nt=\pP{\pp}{\PP}\parN \pP{\q}{\Q}\parN\Nt_0$ with
  $\sendL{\q}{\la}\in \ES(\PP)$ and $\rcvL{\pp}{\la}\in\ES(\Q)$.
  Whence by \refToDef{esp}(\ref{ila-esp1}) we get
  $\PP=\oup\q{i}{I}{\la}{\PP}$ and $\Q=\inp\pp{j}{J}{\la}{\Q}$ where
  $\la = \la_k$ for some $k\in
  I \cap J$. Therefore 
%  \\
%  \centerline{$
\[
  \Nt\stackred{\Comm{\pp}{\la}{\q}}\pP{\pp}{\PP_k}\parN
    \pP{\q}{\Q_k}\parN\Nt_0
    \]
    %$.}  
    Case $n>1$. Let $\netev_1$ and $\Nt$
  be as in the basic case, $\Nt''=\pP{\pp}{\PP_k}\parN
  \pP{\q}{\Q_k}\parN\Nt_0 $ and $\alpha=\Comm{\pp}{\la}{\q}$.
 Since $\Seq{\netev_1;\cdots}{\netev_n}$ is a proving sequence, we have
$\neg(\netev_l\grr\netev_{l'})$ for all $l, l'$ such that $1\leq l, l'\leq n$. 
Moreover, for all $l$, $2\leq l\leq n$ we have
$\netev_l\not=\netev_1=\nec\alpha$, thus
$\post{\netev_l}{\alpha}$ is defined by \refToLemma{dc}.  Let  $\netev_l'=\post{\netev_l}{\alpha}$ for all $l$, $2\leq l\leq n$,  then
$\netev_l'\in\GE(\Nt'')$ by \refToLemma{epp}(\ref{epp2}).  
% \bcomila Ho rigirato un po' le cose nel pezzo sopra,
% aggiungendo il fatto che $\neg(\netev_l\grr\netev_{l'})$ per tutti gli
% $l,l'$ tali che $1\leq l, l'\leq n$, che serve sotto. \ecomila 
\\
 We show  that
$ \Seq{\netev_2';\cdots}{\netev_n'}$   is a
proving sequence in $\ESN{\Nt''}$.  First notice that for all
$l$, $2\leq l\leq n$, $\neg(\netev_l\grr\netev_{l'})$ implies
$\neg(\netev_l'\grr\netev_{l'}')$ by 
\refToLemma{prop:prePostNet}(\ref{ppn3b}) and (\ref{ppn1}). Let now 
$\netev\prec\netev_h'$ for some $h$, $2\leq h\leq n$.  By
\refToLemma{prop:prePostNet}(\ref{ppn2b}) and (\ref{ppn1})
$\pre\netev\alpha\prec\preP{\post{\netev_h}{\alpha}}{\alpha}=\netev_h$. This
implies by \refToDef{provseq} that there is $h'<h$ such that either
$\pre\netev\alpha=\netev_{h'}$ or
$\pre\netev\alpha\gr\netev_{h'}\prec\netev_h$. Therefore, since
$\netev_l'$ is defined for all $l$, $2\leq l\leq n$, we get either $\netev=\netev_{h'}'$  by
\refToLemma{prop:prePostNet}(\ref{ppn1b})  or
$\netev\gr\netev_{h'}'\prec\netev_h'$  by
\refToLemma{prop:prePostNet}(\ref{ppn3}) and
(\ref{ppn2}). \\
By induction $\Nt''\stackred{\comseq'}\Nt'$ where
$\comseq'=\comm{\netev_2'}\cdots\comm{\netev_n'}$.
Since $\comm{\netev_l}=\comm{\netev_l'}$ for all $l$,
$2\leq l\leq n$ we get $\comseq=\alpha\cdot\comseq'$.  Hence
$\Nt\stackred{\alpha} \Nt''\stackred{\comseq'}\Nt'$ is the required
transition sequence.
\end{proof}