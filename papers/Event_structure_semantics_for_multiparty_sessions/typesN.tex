% !TEX root =cdgS.tex

\section{Global Types}
\mylabel{sec:types}

This section is devoted to our type system for multiparty sessions.
Global types describe the communication protocols involving all
session participants.  Usually, global types are projected into local
types and typing rules are used to derive local types for
processes~\cite{CHY08,Coppo2016,CHY16}. The simplicity of our calculus
allows us to project directly global types into processes and to have
exactly one typing rule, see \refToFigure{fig:typing}.
This section is split  in two subsections.\\
The first subsection presents the projection of global types onto
processes, together with the proof of its soundness. Moreover it
introduces a \emph{boundedness} condition on global types, which is
crucial for our type system to ensure  progress.\\
The second subsection presents the type system, as well as an LTS for
global types.  Lastly, the properties of Subject Reduction, Session
Fidelity and Progress are shown.

\subsection{Well-formed Global Types}\label{wfgt}
Global types are built from choices among communications.
 

\begin{definition}[Global types]
\mylabel{def:GlobalTypes} 
Global types $\G$  are defined by: 
%\\[4pt]
%\centerline{ $
\[
\begin{array}{lll}
      \G~~& \coDefGr & 
    \gt\pp\q i I \la \G 
 %         ~\mid 
%          \G_1\parG\dots\parG\G_n 
%          ~\mid~\mu\ty.\G ~\mid ~\ty
      ~\mid ~\End
  \end{array}
  \]
%$
%}
%\noindent
where $I$ is not empty, $\la_h\not=\la_k\,$ for all $h,k\in I$, $h\neq k$, i.e. messages
in choices are all different.  
\end{definition}  

As for processes, $ \coDefGr$ indicates that global types are 
defined coinductively. Again, we focus on \emph{regular} terms. 
%coinductively defined \emph{regular} terms. 
%

Sequential composition ($;$) has higher precedence than choice
($\GlSy$).  %
When $I$ is a singleton, a choice $ \gt\pp\q i I \la \G$ will be
rendered simply as $\gtCom{\pp}{\q}{\la} \, ; \G$.  In writing
global types,
we omit the final $\End$.

Given a global type, the sequences of decorations of nodes and edges on
the path from the root to an edge in the tree of the global type are traces, in
the sense of \refToDef{traces}.  We denote by $\FPaths{\G}$ the set of
traces of $\G$. By definition, $\FPaths{\End} = \emptyset$ and each
trace in $\FPaths{\G}$ is non-empty.
\label{G-traces}

%%%%%%%%%%% Precedente definizione %%%%%%%%%%%%%%%%%%%%%%
% The set of {\em participants of a global type $\GP$}, $\participant{\GP}$, is the smallest set such that: 
% % \\
% %\centerline{$
% \[
% \begin{array}{l}
% \participant{\gt\pp\q i I \M \RG}=\set{\pp,\q}\cup\bigcup_{i\in I}\participant{\RG_i}\\
% \participant{\End}=\emptyset 
% \end{array}
% \]
% %$}
% \bcomila
% Come la definizione di $play(\GP)$ nel lavoro asincrono (FI p.~7), la
% definizione di $\participant{\GP}$ potrebbe essere semplificata nel
% modo seguente: 
% \ecomila
%%%%%%%%%%%%%%%%%%%%%%%%%%%%%%%%%%%%%%%%%%%%%%%%%%


The set of {\em participants of a global type $\GP$},
$\participant{\GP}$, 
is defined to be the union of the sets of participants of all its traces, namely\\
\centerline{$\participant{\GP} = \bigcup_{\comseq \in
    \FPaths{\G}} \participant{\comseq} $}  Note that the
regularity assumption ensures that the set of participants is finite.



\begin{figure}[t]
% \centerline{
% $
\[
 \begin{array}{c}
 \\[-1pt]
\proj\G{\pr} = \inact \text{ if }\pr\not\in\participant\G \\\\
\proj{(\gt\pp\q i I \M \RG)}\pr=\begin{cases}
  \inpP\pp{i}{I}{\M}{\proj{\RG_i}\pr}    & \text{if }\pr=\q, \\
    \oupP\q{i}{I}{\M}{\proj{\RG_i}\pr}    & \text{if }\pr=\pp, \\
      \proj{\RG_1}\pr  & \text{if } \pr\not\in\set{\pp, \q}\text{ and } \pr\in\participant{\RG_1} \text{ and }\\
      &\proj{\RG_i}\pr=\proj{\RG_1}\pr\text{ for all } i \in I
\end{cases}\\[13pt]
\end{array}
\]
%$
%}
\caption{Projection of  global types onto participants.} \mylabel{fig:proj}
\end{figure}
 
\bigskip

The projection of a global type onto participants is given in
\refToFigure{fig:proj}. As usual, projection is defined only when it
is defined on all participants. Because of the simplicity of our
calculus, the projection of a global type, when defined, is simply a
process.  The definition is coinductive, so a global type with an
infinite (regular) tree produces a process with a regular tree. The
projection of a choice type on the sender produces an output choice,
i.e. a process sending one of its possible messages to the receiver
and then acting according to the projection of the corresponding
branch.  Similarly for the projection on the receiver, which produces
a process which is an input choice.  Projection of a choice type on
the other participants is defined only if it produces the same process
for all the branches of the choice. This is a
standard condition for multiparty session types~\cite{CHY08}.\\
% We need to show that projection is well defined, i.e. that it is a
% partial function. The proof uses the  notion of  depth of
% participants in global types (\refToDef{depth}), see \refToLemma{pf}.
 
 
Our coinductive definition of global types is more permissive than that based
on the standard $\mu$-notation used in \cite{CHY08}, because it
allows more global types to be projected, as shown by  the
following example.
 \begin{example}\label{de}
   The global type $\G=
   \gtCom\pp\q{}:(\Seq{\la_1}{\gtCom\q\pr{\la_3}}~\GlSyB~\Seq{\la_2}{\G})$
   is projectable and
\begin{itemize}
  \item $\proj{\G}{\pp}=\PP=\sendL{\q}{\la_1} \oplus\Seq{\sendL{\q}{\la_2}}{\PP}$
  \item $\proj{\G}{\q}=\Q=\Seq{\rcvL{\pp}{\la_1}}{\sendL{\pr}{\la_3}} +\Seq{\rcvL{\pp}{\la_2}}{\Q}$
  \item $\proj{\G}{\pr}=\rcvL{\q}{\la_3}$
\end{itemize}
 On the other hand, the corresponding global type based on the
$\mu$-notation 
\[\G'=\mu\ty.  \,\gtCom\pp\q{}:  (\Seq{\la_1}{\gtCom\q\pr{\la_3}}~\GlSyB~\Seq{\la_2}{\ty})\]
 is not projectable because $\proj{\G'}{\pr}$
is not defined. 
\end{example}

% We want to ensure by typing that each participant wishing to
% communicate can eventually do so.  To this aim, we require global
% types to satisfy the property of boundedness, which states that for
% each participant, the depth of the first occurrence of that
% participant in all traces of any subtree of the type is bounded.  This
% is formalised by the following definition of {\em depth}.

 To achieve progress, we need to ensure that each network
participant occurs in every computation, whether finite or
infinite. This means that each type participant must occur in every
path of the tree of the type. Projectability already ensures that each
participant of a choice type occurs in all its branches. This implies
that if one branch of the choice gives rise to an infinite path,
either the participant occurs at some finite depth in this path, or
this path crosses infinitely many branching points in which the
participant occurs in all branches. In the latter case, since the
depth of the participant increases when crossing each branching point,
there is no bound on the depth of the participant over all paths of
the type. Hence, to ensure that all type participants occur in all 
paths, it is enough to require the existence of such bounds. This
motivates the following definition of depth and boundedness.  

 \begin{definition}[Depth  and boundedness]\label{depth}$\;$\\
 Let the two functions $\weight(\comseq,\pp)$ and $\weight(\G,\pp)$ be defined by:
   % Let $\weight(\concat\comseq\alpha,\pp)=\cardin\comseq + 1 $ if
   % $\pp\not\in\participant{\comseq}$ and $\pp\in\participant{\alpha}$.
   % \\ \bcomila Mi sembra che con questa definizione
   % $\weight(\comseq,\pp)$ sia indefinita per tutte le tracce in cui
   % non occorre $\pp$ e anche per tutte le tracce della forma
   % $\concat{\comseq}{\concat{\alpha}{\comseq'}}$ dove $\comseq$ e
   % $\alpha$ sono come sopra. E in questo
   % caso anche $\weight(\G,\pp)$ viene indefinita! 
   % Mi sembra che dovremmo usare la stessa definizione che in FI.
   %\\
   %\ecomila 
\[\weight(\comseq,\pp)=\begin{cases} n &\text{ if }\comseq =
\concat{\concat{\comseq_1}{\alpha}}{\comseq_2}\text { and }\cardin{\comseq_1} = n-1\text { and }\pp \notin
\participant{\comseq_1}\text { and }\pp \in \participant{\alpha}\\
  0   & \text{otherwise }
\end{cases}
\]
Then

$\weight(\G,\pp)=
       \sup  \{\weight(\comseq,\pp)\ |\ \comseq\in\FPaths{\G}\}$\\
    
% \[
%    \weight(\G,\pp)=\begin{cases}
%        \sup  \{\weight(\comseq,\pp)\ |\ \comseq\in\FPaths{\G}\}   & \text{if }\pp\in\participant{\G}\\
%        0   & \text{otherwise }\\
%      \end{cases}
%      \]
%      \bcomila Mi sembra che la seconda clausola di $\weight(\G,\pp)$
%      si potrebbe eliminare perché se $\pp\notin\participant{\G}$
%      allora $\weight(\comseq,\pp) =0$ per ogni
%      $\comseq\in\FPaths{\G}$ e quindi per la prima clausola abbiamo
%      già $\sup \{\weight(\comseq,\pp)\ |\
%      \comseq\in\FPaths{\G}\} =0$. \\
%      \ecomila
We say that a global type $\GP$ is {\em bounded}
   if $\weight(\G',\pp)$ is finite for all subtrees $\G'$ of $\GP$ and
   for all participants $\pp$.
 \end{definition}
If $\weight(\G,\pp)$ is finite, then there are  no paths in
the tree of $\G$ in which $\pp$ is delayed indefinitely. 
Note that if  
 $\weight(\G,\pp)$ is finite, $\G$ may have subtrees $\G'$ for which
 $\weight(\G',\pp)$ is  infinite 
%not finite 
as the following example shows.
 \begin{example}
Consider $\G'= \Seq{\gtCom\q\pr{\la}}{\G}$  where
 $\G$ is as defined in \refToExample{de}. 
 Then we have: 
% \\
% \centerline{$
\[
 \weight(\G',\pp)=2\quad\quad \weight(\G',\q)=1\quad\quad\weight(\G',\pr)=1
 \]
% $}
 whereas
% \\
% \centerline{$
\[
 \weight(\G,\pp)=1\quad\quad \weight(\G,\q)=1\quad\quad\weight(\G,\pr)=\infty
 \]
% $ }
  since\\ \centerline{$\FPaths{\G}=\{\underbrace{\Comm\pp{\la_2}\q\cdots \Comm\pp{\la_2}\q}_n\cdot\Comm\pp{\la_1}\q\cdot\Comm\q{\la_3}\pr\ |\ n\geq 0\}
 \cup\{\Comm\pp{\la_2}\q\cdots\Comm\pp{\la_2}\q\cdots\} $ }  and
  $\sup\{2,3,\ldots\} = \infty$. 
 \end{example}
 
The depths of the participants  in  $\G$
which are not participants of its root communication decrease in the immediate subtrees of $\G$.  

\begin{proposition}\label{dd}
If $\G=\gt\pp\q i I \la \G$ and $\pr\in\participant{\G}{\setminus}\set{\pp,\q}$, then $\weight(\G,\pr)>\weight(\G_i,\pr)$ for all $i\in I$.
\end{proposition}
\begin{proof}
Each trace $\comseq\in\FPaths{\G}$ is of the shape $\concat{\Comm\pp{\la_i}\q}{\comseq'}$ where $i\in I$ and $\comseq'\in\FPaths{\G_i}$. 
\end{proof}

% \bcomila Nella riga precedente ho sostituito $\Comm\pp\q{\la_i}$ con
% $\Comm\pp{\la_i}\q$. Verificare che non abbiamo fatto questo typo
% anche nel seguito. \ecomila
We can now show that the definition of projection given in
\refToFigure{fig:proj} is sound  for bounded global types. 
%The proof is given in the Appendix.
% We need to show that projection is well defined, i.e. that it is a
% partial function. The proof uses the  notion of  depth of
% participants in global types (\refToDef{depth}), see \refToLemma{pf}.
 
\begin{lemma}\label{pf}
If $\GP$ is bounded, then $\proj\GP\pr$ is a partial function for all $\pr$.  
\end{lemma}

Boundedness and projectability single out the global types we want to use in
our type system.

 \begin{definition} [Well-formed global types] \label{wfs}We say that  the global type 
 $\GP$ is {\em well formed} 
if $\GP$ is bounded 
and $\proj{\GP}{\pp}$ is defined for all  $\pp$.
 \end{definition}
Clearly it is sufficient to check that
  $\proj{\GP}{\pp}$ is defined  for all $\pp\in\participant{\GP}$, since otherwise 
 $\proj{\GP}{\pp}=\inact$.  
 

 \subsection{Type System}
 \begin{figure}[h]
% \centerline{
% $%\;\\
{\small \[
 \begin{array}{c}
\inact\subt\inact~\rulename{ $\subt$ -$\inact$}\quad \cSinferrule{\PP_i\subt\Q_i ~~~~~i\in I}{\inp\pp{i}{I\cup J}{\M}{\PP}\subt \inp\pp{i}{I}{\M}{\Q}}{\rulename{ $\subt$-In}}
\quad \cSinferrule{\PP_i\subt\Q_i ~~~~~i\in
  I}{\oupTx\pp{i}{I}{\M}{\PP}\subt \oup\pp{i}{I}{\M}{\Q}}{\rulename{
    $\subt$-Out}}\\ \\
\inferrule{\PP_i\subt\proj\GP{\pp_i}~~~~~i\in I~~~~~~\participant\GP\subseteq \set{\pp_i\mid i\in I}}
{\derN{\PiB_{i\in I}\pP{\pp_i}{\PP_i}}\GP} ~\rulename{Net}
\\[13pt]
\end{array}
\]}
%$}
\caption{Preorder on processes and  network typing rule.} \mylabel{fig:typing}
\end{figure}
The definition of  well-typed network is given in \refToFigure{fig:typing}.
 We first define a preorder on processes, $\PP\leq\Q$, %saying when a 
meaning that {\em process $\PP$ can be used where we expect process $\Q$}.  
More precisely, $\PP\leq\Q$ if either $\PP$ is equal to $\Q$,   or we
are in one of two situations: 
either both $\PP$ and $\Q$ are output processes with the same
receiver and choice of   messages,  and their continuations after the send
are two processes $\PP'$ and $\Q'$ such that $\PP'\leq\Q'$; or they
are both input processes with the same sender and choice of messages,
and $\PP$ may receive more  messages  than $\Q$ (and thus have more
behaviours) but whenever it receives the same message as $\Q$ their
continuations are two processes $\PP'$ and $\Q'$ such that
$\PP'\leq\Q'$.
 The rules are interpreted coinductively, since the processes may have
infinite (regular) trees.\\ 
 A network is well typed
if all its participants have associated processes that behave as
specified by the projections of a global type.
In Rule \rulename{Net}, the condition
$\participant{\GP}\subseteq\set{\pp_i\mid i\in I}$ ensures that all
participants of the  global type
appear in the network.  Moreover it permits additional participants
that do not appear in the global type,
allowing the typing of sessions containing $\pP{\pp}{\inact}$ for a
fresh $\pp$ --- a property required to guarantee invariance of types
under structural congruence of networks.


\begin{example}\mylabel{exg}
The  first network of   \refToExample{ex:rec2} and the network of \refToExample{ex1}  
%%\ref{ex3-variant}  and \ref{ex-prec-gr}  
can be typed respectively by
%\\ \centerline{$
\[
\begin{array}{lll}
\G&=&   \gtCom\pp\q{}:(\Seq\la\G~\GlSyB~\la') \\
\G'&=& \Seq{\Seq{\gtCom\pp\q{\la_1}}{\gtCom\q\pr{\la_2}}}{\gtCom\pr\ps{\la_3}}\\
\end{array}
\]
%$}
%\noindent
%The second network of \refToExample{ex:rec2} cannot be typed
%because its candidate global type: 
%%\\
%%\centerline{$
%\[
%\G'' = \gtCom\pp\q{}:(\Seq\la{\G''}~\GlSyB~\Seq{\Seq{\la'}{\gtCom\pp\pr{\la}}}{\gtCom\pr\ps{\la'}} ) 
%\]
%%$}
%is not bounded, given that $\weight(\G'',\pr)$ and $\weight(\G'',\ps)$
%are not finite.\\
%The network of \refToExample{ex2} cannot be typed because 
% we cannot define a global type whose projections are greater than or equal to  the
%processes associated with  the network  participants.  
\end{example}
 


 

\begin{figure}
 %\centerline{$%\;\\
 \[
\begin{array}{c}
 \gt\pp\q i I \la \G \stackred{\Comm\pp{\la_j}\q}\G_j~~~~~~j\in I{~~~\rulename{Ecomm}}
 \\ \\
 \prooftree
 \G_i\stackred\alpha\G_i' \quad 
 \text{ for all }
i \in I \quad\participant{\alpha}\cap\set{\pp,\q}=\emptyset
 \justifies
 \gt\pp\q i I \la \G \stackred\alpha\gt\pp\q i I \la {\G'} 
 \using ~~~\rulename{Icomm}
  \endprooftree\\ \\
\end{array}
%$}
\]
\caption{
LTS for global types.
}\mylabel{ltgt}
\end{figure}

It is handy to define the LTS for  global types
given in \refToFigure{ltgt}.  Rule \rulename{Icomm} is justified by
the fact that in a projectable global type
$\gt\pp\q i I \la \G$, the behaviours of the participants different
from $\pp$ and $\q$ are the same in all branches, and hence they are
independent from the choice and may be executed before it.  This LTS
respects well-formedness of global types,  as shown in Proposition
\ref{prop:wfs}.

We start with a lemma %, proved in the Appendix,  
relating the projections of a well-formed global type
with its  transitions. 


\begin{lemma}\label{keysr}  Let $\G$ be a well-formed global type. 
\begin{enumerate}
\item\label{keysr1}
If $\proj\G\pp=\oup\q{i}{I}{\M}{\PP}$ and $\proj\G\q=\inp\pp{j}{J}{\M'}{\Q}$, then $I=J$, $\M_i=\M_i'$, $\G\stackred{\Comm\pp{\la_i}\q}\G_i$, $\proj{\G_i}\pp=\PP_i$ and $\proj{\G_i}\q=\Q_i$ for all $i\in I$.
\item\label{keysr2} If $\G\stackred{\Comm\pp{\la}\q}\G'$, then
  $\proj\G\pp=\oup\q{i}{I}{\M}{\PP}$,
  $\proj\G\q=\inp\pp{i}{I}{\M}{\Q}$, where $\la_i=\la$
for some $i\in I$, and 
$\proj{\G'}\pr=\proj\G\pr$ for all $\pr\not\in\set{\pp,\q}$.
\end{enumerate}
\end{lemma}

\begin{proposition}\label{prop:wfs}
If $\G$ is a well-formed global type and $\G\stackred{\Comm\pp{\la}\q}\G'$, then $\G'$ is a well-formed global type.
\end{proposition}
\begin{proof}
If  $\G\stackred{\Comm\pp{\la}\q}\G'$, by  \refToLemma{keysr}(\ref{keysr1}) and (\ref{keysr2})
$\proj{\G'}\pr$ is defined for all $\pr$. 
The proof that $\weight(\G'',\pr)$ for all $\pr$ and $\G''$ subtree
of $\G'$ is easy by induction on the  transition   rules
of \refToFigure{ltgt}. 
\end{proof}
\noindent
Given the previous proposition, we will focus on {\bf well-formed global types  from now on.} \\


 We end this section with the expected proofs of Subject Reduction, Session Fidelity \cite{CHY08,CHY16}  and Progress \cite{Coppo2016,Padovani15}, which use Inversion and Canonical Form lemmas.

\begin{lemma}[Inversion]\mylabel{lemma:InvSync}
If $\derN{\Nt}{\G}$, then $\PP\subt\proj\G{\pp}$ for all  $\pP{\pp}{\PP}\in\Nt$. \end{lemma}

\begin{lemma}[Canonical Form]\mylabel{lemma:CanSync}
If  $\derN\Nt\G$ and
$\pp\in\participant\G$, then $\pP{\pp}{\PP}\in\Nt$ and $\PP\subt\proj\G{\pp}$.
\end{lemma}

\begin{theorem}[Subject Reduction]\mylabel{sr}
If $\derN\Nt\G$ and $\Nt\stackred\alpha\Nt'$, then $\G\stackred\alpha\G'$ and \mbox{$\derN{\Nt'}{\G'}$.}
\end{theorem}
\begin{proof}
Let  $\alpha=\Comm\pp{\la}\q$. 
By Rule
\rulename{Com} of \refToFigure{fig:netred}, $\Nt\equiv \pP{\pp}{\PP}\parN \pP{\q}{\Q}\parN\Nt''$ where
$\PP=\oup\q{i}{I}{\la}{\PP}$ and $\Q=\inp\pp{j}{J}{\la}{\Q}$  and 
$\Nt'\equiv\pP{\pp}{\PP_h}\parN \pP{\q}{\Q_h}\parN\Nt''$ and $\la=\la_h$ for some $h\in I\cap J$.
From \refToLemma{lemma:InvSync} we get
\begin{enumerate}
\item \label{psr1} $\proj\G{\pp}=\oup\q{i}{I}{\la}{\PP'}$ with
  $\PP_i\subt\PP'_i$ for all $i\in I$, from Rule 
\rulename{ $\subt$ -Out} of \refToFigure{fig:typing}, and
\item \label{psr2} $\proj\G{\q}=\inp\pp{j}{J'}{\la}{\Q'}$ with
  $\Q_j\subt\Q'_j$ for all $j\in J'\subseteq J$, from Rule \rulename{
    $\subt$ -In} of \refToFigure{fig:typing}, and
\item \label{psr3}  $\R\subt\proj\G{\pr}$ for all   $\pP{\pr}{\R}\in\Nt''$.
\end{enumerate}
By Lemma~\ref{keysr}(\ref{keysr1}) $\G\stackred{\Comm\pp{\la_h}\q}\G_h$ and $\proj{\G_h}\pp=\PP_h'$ and $\proj{\G_h}\q=\Q'_h$. By Lemma~\ref{keysr}(\ref{keysr2}) $\proj{\G_h}\pr=\proj{\G}\pr$  for all $\pr\not\in\set{\pp,\q}$. We can then choose $\G'=\G_h$.
\end{proof}
\begin{theorem}[Session Fidelity]\mylabel{sf}
If $\derN\Nt\G$ and $\G\stackred\alpha\G'$, then $\Nt\stackred\alpha\Nt'$ and $\derN{\Nt'}{\G'}$. 
\end{theorem}
\begin{proof}
Let  $\alpha=\Comm\pp{\la}\q$. By Lemma~\ref{keysr}(\ref{keysr2}) $\proj\G\pp=\oup\pp{i}{I}{\M}{\PP}$ and $\proj\G\q=\inp\pp{i}{I}{\M}{\Q}$ and $\M=\M_i$ for some $i\in I$ and 
$\proj{\G'}\pr=\proj\G\pr$ for all $\pr\not\in\set{\pp,\q}$. By Lemma~\ref{keysr}(\ref{keysr1})  $\proj{\G'}\pp=\PP_i$ and $\proj{\G'}\q=\Q_i$. From \refToLemma{lemma:CanSync} and \refToLemma{lemma:InvSync} we get \mbox{$\Nt\equiv \pP{\pp}{\PP}\parN \pP{\q}{\Q}\parN\Nt''$} and 
\begin{enumerate}
\item \label{sfc1} $\PP=\oup\q{i}{I}{\la}{\PP'}$ with $\PP'_i\subt\PP_i$ for $i\in I$, from Rule \rulename{ $\subt$ -Out} of \refToFigure{fig:typing}, and
\item \label{sfc2}  $\Q=\inp\pp{j}{J}{\la}{\Q'}$ with $\Q'_j\subt\Q_j$ for $j\in I\subseteq J$, from Rule \rulename{ $\subt$ -In} of \refToFigure{fig:typing},  and 
\item \label{sfc3}  $\R\subt\proj\G{\pr}$ for all   $\pP{\pr}{\R}\in\Nt''$.
\end{enumerate}
We can then choose $\Nt'=\pP{\pp}{\PP'_i}\parN \pP{\q}{\Q'_i}\parN\Nt''$.
\end{proof} 


We are now able to prove that in a typable network, every participant
whose process is not terminated may eventually perform a communication. This
property is generally referred to as progress.
\begin{theorem}[Progress]\mylabel{pr}
If $\derN\Nt\G$ and $\pP\pp\PP\in\Nt$, then $\Nt\stackred{\comseq\cdot\alpha}\Nt'$ and $\pp\in\participant\alpha$.
\end{theorem}
\begin{proof}
We prove by induction on $d=\weight(\G,\pp)$ that:   if $\derN\Nt\G$ and $\pP\pp\PP\in\Nt$, then   $\G\stackred{\concat{\comseq}\alpha}\G'$ with $\pp\in\participant\alpha$. This will imply  $\Nt\stackred{\concat{\comseq}\alpha}\Nt'$  by Session Fidelity  (\refToTheorem{sf}). \\
%Let $\G\equiv\G\parG\GP_0$ with $\pp\in \participant \G$.\\
{\it Case $d=1$.}  In this case  $\G= \gt\q\pr i I \la \G $ and $\pp\in\set{\q,\pr}$ and $\G\stackred{\Comm\q{\la_h} \pr}\G_h$ for some $h\in I$ by Rule \rulename{Ecomm}.\\ 
   {\it Case $d>1$.}  In this case  $\G= \gt\q\pr i I {\la} \G $  and $\pp\not\in\set{\q,\pr}$. By \refToLemma{dd} this implies $\weight(\G_i,\pp)<d$ for all $i\in I$. Using Rule \rulename{Ecomm} we get $\G\stackred{\Comm\q{\la_i} \pr}\G_i$ for all $i\in I$.   By Session Fidelity, $\Nt\stackred{\Comm\q{\la_i}\pr}\Nt_i$ and 
   $\derN{\Nt_i}{\G_i}$ for all $i\in I$. Moreover, since  $\pp\not\in\set{\q,\pr}$ we also have  $\pP\pp\PP\in\Nt_i$ for all $i\in I$.
   By induction $\G_i\stackred{\concat{\comseq_i}{\alpha_i}}\G_i'$ with $\pp\in\participant{\alpha_i}$  for all $i\in I$. We conclude $\G\stackred{\Comm\q{\la_i} \pr\cdot\concat{\comseq_i}{\alpha_i}}\G_i'$ for all $i\in I$. 
\end{proof} 
The proof of the progress theorem shows that the execution 
%reduction 
strategy which uses only Rule \rulename{EComm} is fair, since there are no infinite  transition sequences where some participant is stuck.  This is due to the boundedness condition on global types.


\begin{example}
  The second network of \refToExample{ex:rec2} and the network of
  \refToExample{ex2} cannot be typed because they do not enjoy
  progress. Notice that the candidate global type for the second
  network of \refToExample{ex:rec2}:
\[
\G'' = \gtCom\pp\q{}:(\Seq\la{\G''}~\GlSyB~\Seq{\Seq{\la'}{\gtCom\pp\pr{\la}}}{\gtCom\pr\ps{\la'}} ) 
\]
is not bounded, given that $\weight(\G'',\pr)$ and $\weight(\G'',\ps)$
are not finite.\\
Moreover we cannot define a global type whose projections are greater
than or equal to the processes associated with the network of
\refToExample{ex2}.  
\end{example} 



