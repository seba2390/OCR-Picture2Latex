% !TEX root =cdgS.tex


\section{Event Structure Semantics of Networks}\mylabel{sec:netS-ES}

In this section we define the ES semantics of networks and show that
the resulting ESs, which we call \emph{network ESs}, are FESs. We also
show that when the network is binary, namely when it has only two
participants, then the obtained FES is a PES.  The formal treatment
involves defining the set of potential events of network ESs, which we
call \emph{network events}, as well as introducing the notion of
\emph{causal set} of a network event and the notion of
\emph{narrowing} of a set of network events.  This will be the subject
of \refToSection{subsec:main-defs-props}.

In \refToSection{subsec:further-props}, we first prove some properties
of the conflict relation in network ESs.  Then, we come back to
causal sets and we show that they are always finite and that each
configuration includes a unique causal set for each of its
n-events. We also discuss the relationship between causal sets and
prime configurations, which are specific configurations that are in
1-1 correspondence with events in ESs. Finally, we define a notion of
projection from n-events to p-events, and prove that this projection
(extended to sets of n-events) is downward surjective  and
preserves configurations.



\subsection{Definitions and Main Properties}
\mylabel{subsec:main-defs-props}

We start by defining network events, the potential events
of network ESs.  Since these events represent communications between
two network participants $\pp$ and $\q$, they should be pairs of \emph
{dual  p-events}, namely, of  p-events  emanating
respectively from $\pp$ and $\q$, which have both dual actions and
dual causal histories.


Formally, to define network events we need to specify the
\emph{location} of  p-events, namely the participant to which they
belong: 
\begin{definition}[Located event]
\mylabel{proceventL} We call \emph{located event} a  p-event 
$\procev$ pertaining to a participant $\pp$, written $\pp::\procev$.
 \end{definition}
 As hinted above, network events should be pairs of dual located
 events $\locev{\pp}{\actseq\cdot\pi}$ and
 $\locev{\q}{\actseq'\cdot\pi'}$ with matching actions $\pi$ and
 $\pi'$ and matching histories $\actseq$ and $\actseq'$. To formalise
 the matching condition, we first define the projections of process
 events on participants, which yield sequences of \emph{undirected
   actions} of the form $!\la$ and $?\la$, or the empty sequence
 $\emptyseq$.  Then we introduce a notion of duality between located
 events, based on a notion of duality between undirected actions.
 

Let $\bm$ range over $!\la$  and $?\la$, and $\bms$ range over 
 (possibly empty)  sequences of $\bm$'s. 

\begin{definition}[Projection of  p-events]\label{pd}
The projection of a  p-event  $\procev$ on a participant
$\pp$, written $\projb{\procev}{\pp}$, is defined by: \\
\[
\projb{\sendL{\q}{\la}}\pp=\begin{cases}
\sendL{}{\la}     & \text{if }\pp=\q \\
   \emptyseq   & \text{otherwise}
\end{cases}\quad
\projb{\rcvL{\q}{\la}}\pp=\begin{cases}
\rcvL{}{\la}     & \text{if }\pp=\q \\
   \emptyseq   & \text{otherwise}
\end{cases}
\]
\[
\projb{(\concat{\pi}{\procev})}{\pp}=
\concat{\projb{\pi}\pp} {\projb\procev\pp} 
\]
\end{definition}


\begin{definition}[Duality of undirected action sequences]\label{dualProcEv}
The \emph{duality of undirected action sequences}, written $\dualev\bms{\bms'}$, is the
symmetric relation induced by: 
\[ \dualev{\ee}{\ee}
\qquad\quad
\dualev\bms{\bms'} ~\impl ~ \dualev{\,\concat{\sendL{}{\la}} \bms\,}
 { \,\concat{\rcvL{}{\la}} \bms'}
 \]
\end{definition} 


\begin{definition}[Duality of located events]\label{dualLocEv}
Two located events $\locev{\pp}{\procev}, \locev{\q}{\procev'}$ are \emph{dual}, written  $\dualevS{
  \locev{\pp}{\procev}}{\locev{\q}{\procev'}}$,  if $\dualev{\projb\procev\q}{\projb{\procev'}\pp}$
  and 
$\ptone{\act{\procev}} = \q\,$ and
$\,\ptone{\act{\procev'}} = \pp$. 
\end{definition}



Dual located events may be sequences of actions of different length. For instance 
$\dualevS{\locev{\pp}{\concat{\sendL{\q}{\la}}{\sendL{\pr}{\la'}}}}{\locev{\pr}{\rcvL{\pp}{\la'}}}$ and
$\dualevS{\locev{\pp}{{\sendL{\q}{\la}}}}{\locev{\q}{\concat{\sendL{\pr}{\la'}}{\rcvL{\pp}{\la}}}}$. 



\begin{definition}[Network event]
%\mylabel{proceventN}
\mylabel{n-event} 
{\em Network events} $\netev, \netev'$, also called
\emph{n-events}, are unordered
pairs of dual % 
located events, namely:
%\\
%\centerline{$ 
\[
\netev ::= \set{\locev{\pp}{\procev},
    \locev{\q}{\procev'}} \qquad
  \text{where}~~~\dualevS{\locev{\pp}{\procev}}{\locev{\q}{\procev'}}
  %$}
\]  
We denote by $\DE$ the set of n-events.
 \end{definition}


 We define {\em the communication of the event} $\netev$, notation
 $\comm\netev$, by $\comm\netev=\Comm\pp{\M}\q$ if
 $\netev=\set{\locev{\pp}{\concat\actseq{\sendL\q\M}},
   \locev{\q}{\concat{\actseq'}{\rcvL\pp\M}}}$ and we say that the
 n-event $\netev$ \emph{represents} the communication
 $\Comm\pp{\M}\q$.  We also define the set of \emph{locations} of an
 n-event to be $\loc{\set{\locev{\pp}{\procev}, \locev{\q}{\procev'}}}
 = \set{\pp,\q}$.

 It is handy to have a notion of occurrence of a located event in
a set of network events: 
 
 \begin{definition}\label{olesne}
   {\em A located event $\locev{\pp}{\procev}$ occurs in a set $E$ of
     n-events,} notation $\occ{\locev{\pp}{\procev}}E$, if
   $\locev{\pp}{\procev}\in\netev$ and $\netev\in E$ for some
   $\netev$.
 \end{definition} 

We define now the flow and conflict relations on network
events. While the flow relation is the expected one (a network event
inherits the causality from its constituent processes), the conflict
relation is more subtle, as it can arise also between network
events with disjoint  sets of locations. 
%participants.

 In the following definition we use $\eh\bms$ to denote the length of
 the sequence $\bms$. 
\begin{definition}[Flow and conflict relations on n-events] %\text{~}\\[5pt]
\mylabel{netevent-relations}
The \emph{flow} relation $\precN$ and the \emph{conflict} relation
$\gr$ on the set of n-events $\DE$ are defined by:

\begin{enumerate}
\item\mylabel{c1} 
$\netev \precN\netev '$ if 
$
\locev{\pp}{\procev}\in \netev 
~\&~\locev{\pp}{\procev'}\in\netev' ~\&~ \procev  <  \procev' $; 
\item\mylabel{c2} 
  $\netev \grr\netev '$ if 
 \begin{enumerate}
\item \label{c21} either
  $  \locev{\pp}{\procev}\in \netev 
~\&~\locev{\pp}{\procev'}\in\netev'~\&~\procev \grr \procev' $;
\item\label{c22}
 or $\locev{\pp}{\procev}\in \netev 
~\&~\locev{\q}{\procev'}\in\netev' ~\&~\pp \neq \q ~\&~
\cardin{\projb\procev\q} = \cardin{\projb{\procev'}\pp}   ~\&~
\neg(\dualev{\projb\procev\q}{\projb{\procev'}\pp})$. 
\end{enumerate}
\end{enumerate}
\end{definition}
%
Two n-events are in conflict if they share a participant
with conflicting p-events (Clause (\ref{c21}))
or if some of their participants have communicated with each other in the past
in incompatible ways (Clause (\ref{c22})).  Note that the two
clauses are not exclusive, as shown in the following example. 


\begin{example}\mylabel{ex:conflict} This example illustrates the use
  of 
%Clause (\ref{c2}) of
  \refToDef{netevent-relations} in various cases. It also shows
that the flow and conflict relations may be overlapping on n-events.
%
\begin{enumerate}
\item \mylabel{ex:conflict1} Let
  $\netev=\set{\locev\pp{\q!\la_1\cdot\pr!\la},\locev\pr{\pp?\la}}$
  and $\netev'=\set{\locev\pp{\q!\la_2}, \locev\q{\pp?\la_2}}$.  Then
  $\netev \grr\netev '$ by Clause (\ref{c21}) 
since
  $\q!\la_1\cdot\pr!\la\grr\q!\la_2$. Note that $\netev \grr\netev '$
  can be also deduced by Clause (\ref{c22}), since
  $\projb{(\q!\la_1\cdot\pr!\la)}{\q} = \,!\la_1$ and
  $\projb{\pp?\la_2}\pp = \,?\la_2$ and $\cardin{!\la_1} =
  \cardin{?\la_2}$ and $\neg(\dualev{!\la_1}{?\la_2})$.
%
\item \mylabel{ex:conflict2} 
Let $\netev$ be as in (\ref{ex:conflict1})
and $\netev'=\set{\locev\pp{\q!\la_2\cdot \q!\la},
  \locev\q{\pp?\la_2\cdot \pp?\la}}$. Again, we can deduce
  $\netev \grr\netev '$ using Clause (\ref{c21}) 
  since $\q!\la_1\cdot\pr!\la\grr\q!\la_2\cdot \q!\la$.  On the other
  hand, Clause (\ref{c22}) does not apply in this case since
  $\projb{(\q!\la_1\cdot\pr!\la)}{\q} = \,!\la_1$ and
  $\projb{(\pp?\la_2\cdot\pp?\la)}\pp = \,?\la_2\cdot?\la$ and thus
  $\cardin{!\la_1} \neq \cardin{?\la_2\cdot?\la}$.
%
\item \mylabel{ex:conflict3} Let $\netev$ be as in
  (\ref{ex:conflict1}) and
  $\netev'=\set{\locev\q{\pp?\la_2\cdot \ps!\la},
    \locev\ps{\q?\la}}$. Here $\loc{\netev} \cap \loc{\netev'}
  =\emptyset$, so clearly Clause (\ref{c21}) does not apply.  On the
  other hand, $\netev \grr\netev '$ can be deduced by Clause
  (\ref{c22}) since $\projb{(\q!\la_1\cdot\pr!\la)}{\q} = \,!\la_1$
  and $\projb{(\pp?\la_2\cdot\ps!\la)}\pp = \,?\la_2$ and
  $\cardin{!\la_1} = \cardin{?\la_2} $ and
  $\neg(\dualev{!\la_1}{?\la_2})$.  
%


\item \mylabel{ex:conflict5} Let $\netev$ be as in
  (\ref{ex:conflict1}) and $\netev'=\set{\locev\pp{\q!\la_2
\cdot\pr!\la \cdot\pr!\la'}, \locev\pr{\pp?\la \cdot\pp?\la'}}$.
In this case we have both $\netev\prec\netev'$ by Clause (\ref{c1})
and $\netev\grr\netev'$ by Clause (\ref{c21}), namely, causality
is inherited from participant $\pr$ and conflict from participant $\pp$.
 %
\end{enumerate}
\end{example}


We introduce now the notion of \emph{causal set} of an n-event
$\netev$ in a given set of events $\EvSet$. Intuitively, a causal set
of $\netev$ in $\EvSet$ is a \emph{complete set of non-conflicting
direct causes} of $\netev$ which is included in $\EvSet$.

\begin{definition}[Causal set]
\mylabel{cs}
Let $\netev\in \EvSet \subseteq \DE$. 
A set of n-events $E$ is a {\em causal set} of $\netev$ in
$\EvSet$ if $E$ is a minimal subset of $\EvSet$ such that
\begin{enumerate}
\item \mylabel{cs1} $E \cup \set{\netev}$ is conflict-free and  
\item \mylabel{cs2} $\locev{\pp}{\procev}\in\netev$ and $\procev' <
  \procev$ 
imply $\occ{\locev{\pp}{\procev'}}{E}$.
\end{enumerate}
\end{definition} 

Note that in the above definition, the conjunction of minimality and
Clause (\ref{cs2}) implies that, if $\netev'\in E$, then
$\netev'\prec\netev$. Thus $E$ is a set of direct causes of $\netev$.
Moreover, a causal set of an n-event cannot be included in another
causal set of the same n-event, as this would contradict the
minimality of the larger set. Hence, \refToDef{cs} indeed formalises
the idea that causal sets should be complete sets of compatible direct
causes of a given n-event.


\begin{example}\mylabel{ex:causal-sets}
  Let $\netev_1=\set{\locev\pp{\q!\la_1\cdot
      \pr!\la},\locev\pr{\pp?\la}}$ and
  $\netev_2=\set{\locev\pp{\q!\la_2\cdot
      \pr!\la},\locev\pr{\pp?\la}}$.  Then both $\set{\netev_1}$ and
  $\set{\netev_2}$ are causal sets of $\netev=\set{\locev\pr{\pp?\la
      \cdot\ps!  \la'},\locev\ps{\pr?\la'}}$ in $\EvSet =
  \set{\netev_1, \netev_2, \netev}$. Note that $\netev_1 \grr
  \netev_2$ and that neither $\netev_1$ nor $\netev_2$ has a causal
  set in $\EvSet$.

Let us now consider also 
$\netev'_1=\set{\locev\pp{\q!\la_1},\locev\q{\pp?\la_1}}$ and
$\netev'_2=\set{\locev\pp{\q!\la_2},\locev\q{\pp?\la_2}}$.  Then
$\netev$ still has the same causal sets $\set{\netev_1}$ and
$\set{\netev_2}$ in $\EvSet' = \set{\netev'_1, \netev'_2, \netev_1,
  \netev_2, \netev}$, while each $\netev_i$, $i=1,2$, has the unique
causal set $\set{\netev'_i}$ in $\EvSet'$, and each $\netev'_i$,
$i=1,2$, has the empty causal set in $\EvSet'$.

Finally, $\netev$ has infinitely many causal sets in $\DE$.  For
instance, if for every natural number $n$ we let
$\netev_n=\set{\locev\pp{\q!\la_n\cdot \pr!\la},\locev\pr{\pp?\la}}$,
then each $\set{\netev_n}$ is a causal set of $\netev$ in $\DE$.
Symmetrically, a causal set may cause infinitely many events in
$\DE$. For instance, the above causal sets $\set{\netev_1}$ and
$\set{\netev_2}$ of $\netev$ could also act as causal sets for any
n-event $\netev''_n=\set{\locev\pr{\pp?\la \cdot\ps!
    \la_n},\locev\ps{\pr?\la_n}}$ or, assuming the set of participants
to be denumerable, for any event $%\widehat{\netev}_n
\netev'''_n=\set{\locev\pr{\pp?\la \cdot \ps_n!
    \la'},\locev{\ps_n}{\pr?\la'}}$.  
\end{example}


When defining the set of events of a network ES, we want to prune out
all the n-events that do not have
%without 
a causal set in the set itself.  The reason is that such
n-events cannot happen.  This pruning is achieved by means of the
following narrowing function.

\begin{definition}[Narrowing of a set of n-events]
\mylabel{def:narrowing}
The \emph{narrowing} of a set $E$ of n-events, denoted by $\nr{E}$,
is the greatest fixpoint of the function $f_E$ on sets of 
n-events defined by:
%\\ \centerline{$
\[
\begin{array}{lll}
f_E(X) &=& \set{\netev\in E \mid \exists E' \subseteq X . \, E' \text{is a causal
    set of } \netev \text{ in $X$ }  }
 %f_E(X) &=& \set{\netev\in E \mid \exists E' \subseteq  X. \, E' \text{is a causal set of } \netev}
 \end{array}
 %$}
\] 
\end{definition}

Note that we could not have taken $\nr{E}$ to be the least
fixpoint of $f_E$ rather than its greatest fixpoint.
Indeed, the least fixpoint of $f_E$ would be the empty set. 


\begin{example} 
\label{ex-narrowing}
The following two examples illustrate the notions of causal set
and narrowing. 

Let $\netev_1=\set{\locev\pr{\ps?\la_1}, \locev\ps{\pr!\la_1}}$,
$\netev_2=\set{\locev\pr{\ps?\la_2}, \locev\ps{\pr!\la_2}}$,
$\netev_3=\set{\locev\pp{\pr?\la_1},\locev\pr{\ps?\la_1\cdot\pp!\la_1}}$,
$\netev_4=\set{\locev\q{\ps?\la_2},\locev\ps{\pr!\la_2\cdot\q!\la_2}}$,
$\netev_5=\set{\locev\pp{\pr?\la_1\cdot\q!\la},
  \locev\q{\ps?\la_2\cdot\pp?\la}}$.  Then 
$\nr{\set{\netev_1,\ldots,\netev_5}}= \set{\netev_1,\ldots,\netev_4}$,
 because a causal set for $\netev_5$ would need to contain both
$\netev_3$ and $\netev_4$, but this is not possible since
$\netev_3 \grr \netev_4$   
by Clause (\ref{c22}) of
\refToDef{netevent-relations}. In fact $\projb{(\ps?\la_1 \cdot\pp!\la_1)}\ps = \,?\la_1$ and
  $\projb{(\pr!\la_2  \cdot\q!\la_2)}\pr = \,!\la_2$ and $\cardin{?\la_1} =
  \cardin{!\la_2}$ and $\neg(\dualev{?\la_1}{!\la_2})$.


  Let $\netev_1=\set{\locev\pr{\ps?\la_1}, \locev\ps{\pr!\la_1}}$,
  $\netev_2=\set{\locev\pr{\ps?\la_2}, \locev\ps{\pr!\la_2}}$,
  $\netev_3=\set{\locev\pp{\pr?\la_1},\locev\pr{\ps?\la_1\cdot\pp!\la_1}}$,
  $\netev_4=\set{\locev\pp{\pr?\la_1\cdot\ps?\la_2},\locev\ps{\pr!\la_2\cdot\pp!\la_2}}$,
  $\netev_5=\set{\locev\pp{\pr?\la_1\cdot\ps?\la_2\cdot\q!\la},\locev\q{\pp?\la}}$.
  Here $\nr{\set{\netev_1,\ldots,\netev_5}}=
  \set{\netev_1,\netev_2,\netev_3}$. Indeed, a causal set for
  $\netev_4$ would need to contain both $\netev_2$ and $\netev_3$, but
  this is not possible since $\netev_2 \grr \netev_3$  by Clause
  (\ref{c21}) of \refToDef{netevent-relations}.  In fact
  $\ps?\la_2\grr\ps?\la_1  \cdot\pp!\la_1$.  Then, 
  %although  
  $\netev_5$   %is not in conflict with  $\netev_3$, it 
  will  also
  be pruned by the narrowing since any causal set for
  $\netev_5$ should contain $\netev_4$.
\end{example}

We can now  finally  define the event structure associated with a  network:

\begin{definition}[Event structure of a
   network] \mylabel{netev-relations} The {\em  event
     structure of network} $\Nt $
   is the triple
%   \\
%   \centerline{$
\[
   \ESN{\Nt} = (\GE(\Nt), \precN_\Nt , \grr_\Nt)
%   $} 
\]
where:
\begin{enumerate}
\item\mylabel{netev-relations1} 
$\begin{array}[t]{ll}\GE(\Nt) =\nr{\DE(\Nt)} ~~\text{ with }\\
\DE(\Nt) = \set{\set{\locev{\pp}{\procev},\locev{\q}{\procev'}} \sep\pP{\pp}{\PP}{\in}\Nt,\pP{\q}{\Q}{\in}\Nt,
\procev{\in}\ES(\PP),  \procev' {\in}\ES(\Q),
  \dualevS{\locev{\pp}{\procev}}{\locev{\q}{\procev'}}}
\end{array}$ 
\smallskip
\item \mylabel{c1SE} 
$\precN _\Nt$ is the restriction of
  $\precN$ to the set  $\GE(\Nt)$;
\item \mylabel{c2SE} 
$\gr _\Nt$ is the restriction of
  $\gr$ to the set  $\GE(\Nt)$.  \end{enumerate}
  \end{definition}



 The set of n-events of a network ES can be infinite, as  shown
 by  the following example. 
\begin{example}\mylabel{ex:rec2}
Let  $\PP$ be as in \refToExample{ex:rec1}, $\Q=\rcvL{\pp}\la;\Q +\rcvL{\pp}{\la'}$ 
and $\Nt=\pP{\pp}{\PP} \parN \pP{\q}{\Q}$. Then
$$ \GE(\Nt) =
\begin{array}[t]{l}
 \set{\set{\locev\pp{\underbrace{\sendL{\q}\la\cdot\ldots\cdot\sendL{\q}\la}_n},
\locev\q{\underbrace{\rcvL{\pp}\la\cdot\ldots\cdot\rcvL{\pp}\la}_n}}\mid
n\geq 1} \quad \cup  \\[15pt]
\set{\set{\locev\pp{\underbrace{\sendL{\q}\la\cdot\ldots\cdot\sendL{\q}\la}_n\cdot\sendL{\q}{\la'}},
\locev\q{\underbrace{\rcvL{\pp}\la\cdot\ldots\cdot\rcvL{\pp}\la}_n\cdot\rcvL{\pp}{\la'}}}\mid
n\geq 0} 
\end{array}
$$
 A simple variation of this example shows that even within the
events of a network ES, an n-event $\netev$ may have an infinite
number of causal sets. Let $\netev=\set{\locev\pr{\pp?\la \cdot\ps!
   \la'},\locev\ps{\pr?\la'}}$ be as in \refToExample{ex:causal-sets}.  
%
% Let $\PP'=\sendL{\q}\la;\PP' \oplus\sendL{\q}{\la'}; \sendL{\pr}\la$
% and $\Q$ %$\Q=\rcvL{\pp}\la;\Q +\rcvL{\pp}{\la'}$
% be as above. Let moreover $\R = \rcvL{\pp}\la;\sendL{\ps}{\la'}$ and
% $S = \rcvL{\pr}\la'$.
%
Consider the network $\Nt'=\pP{\pp}{\PP'} \parN \pP{\q}{\Q} \parN
\pP{\pr}{\R} \parN \pP{\ps}{S}$, where
$\PP'=\sendL{\q}\la;\PP'
\oplus\sendL{\q}{\la'}; \sendL{\pr}\la$,
$\Q$ %$\Q=\rcvL{\pp}\la;\Q +\rcvL{\pp}{\la'}$
is as above, $\R = \rcvL{\pp}\la;\sendL{\ps}{\la'}$ and
$S = \rcvL{\pr}\la'$.

Then $\netev$ has an infinite number of causal sets $E_n =
\set{\netev_n}$ in $\GE(\Nt')$, where
%\\
%\centerline{ $
\[
\netev_n =
\set{\locev\pp{\underbrace{\sendL{\q}\la\cdot\ldots\cdot\sendL{\q}\la}_n
\cdot\,\sendL{\q}{\la'}\cdot\sendL{\pr}{\la},
\locev{\pr}{\rcvL{\pp}\la}}}
%$}
\]
On the other hand, a causal set may only cause a finite number of
events in a network ES, since the number of branches in any choice is
finite, as well as the number of participants in the network.



\end{example}



%
\begin{theorem}\mylabel{nf}
Let $\Nt$ be a  network. Then $\ESN{\Nt}$ is a flow event
structure with an irreflexive conflict relation. 
\end{theorem}
\begin{proof}
  The relation  $\precN _\Nt$  is irreflexive since $ \procev < \procev'$
  implies $\netev\not=\netev'$, where $\procev,\procev ',\netev,
  \netev'$ are as in  \refToDef{netevent-relations}(\ref{c1}).
  As for the conflict relation, note first that a conflict between an
  n-event and itself could not be derived by Clause (\ref{c22}) of
  \refToDef{netevent-relations}, since the two located events of
  an n-event are dual by construction. Then,  symmetry and
  irreflexivity of the conflict relation follow from the corresponding
  properties of conflict between  p-events. 
\end{proof}

Notably, n-events with disjoint sets of 
%participants 
 locations  may 
%also 
be
related by the transitive closure of the flow relation, as illustrated
by the following example, which also shows how  n-events 
inherit the flow relation from the causality relation of their 
p-events. 


\begin{example}\mylabel{ex1}

  Let $\Nt$ be the network 
%  \\
%  \centerline{ $
\[
 \pP{\pp}{\sendL{\q}{\la_1}}\parN
    \pP{\q}{\Seq{\rcvL{\pp}{\la_1}}{\sendL{\pr}{\la_2}}} \parN
    \pP{\pr}{\Seq{\rcvL{\q}{\la_2}}{\sendL{\ps}{\la_3}}}\parN
    \pP{\ps}{\rcvL{\pr}{\la_3}}
%$ }
\]
Then $\ESN{\Nt}$ has three network  events
%\\
%  \centerline{$
\[
\begin{array}{c}
\netev_1 {=}
    \set{\locev{\pp}{\sendL{\q}{\la_1}},\locev{\q}{\rcvL{\pp}{\la_1}}}\quad
    \netev_2 {=}
    \set{\locev{\q}{\concat{{\rcvL{\pp}{\la_1}}}{\sendL{\pr}{\la_2}}},\locev{\pr}{\rcvL{\q}{\la_2}}}
%\\
\quad
  \netev_3 {=} 
    \set{\locev{\pr}{\concat{\rcvL{\q}{\la_2}}{\sendL{\ps}{\la_3}}},
      \locev{\ps}{\rcvL{\pr}{\la_3}}}
%      $} 
\end{array}
  \]    
The flow relation obtained by
  \refToDef{netev-relations} is: $\netev_1 \precN \netev_2$ and
  $\netev_2 \precN \netev_3$.  Note that each time the flow relation
  is inherited from the causality within a different participant, $\q$
  in the first case and $\pr$ in the second case.  
  The nonempty configurations are $\set{\netev_1}, \set{\netev_1,
    \netev_2}$ and $\set{\netev_1, \netev_2, \netev_3}$.
%
Note that $\ESN{\Nt}$ has only one proving sequence per configuration
(which is  the one  given by the numbering of events).
\end{example}



If  a network is  binary, %If $\Nt$ is a binary network
then its FES may be turned into a PES by replacing $\precN$ with
$\precN^*$.  To prove this result, we first show  a
property of n-events of binary networks. We say that an n-event
$\netev$ is {\em binary} if  the participants occurring in 
the p-events of $\netev$ are contained in $\loc\netev$. 


\begin{lemma}
  \mylabel{bnc} Let $\netev$ and $\netev'$ be binary n-events with
  $\loc\netev=\loc{\netev'}$. Then $\netev\grr\netev'$ iff
  $\locev\pp\procev\in\netev$ and $\locev\pp{\procev'}\in\netev'$
  imply $\procev\grr\procev'$.
  \end{lemma} 
%
\begin{proof}
 The ``if'' direction holds  by
\refToDef{netevent-relations}(\ref{c21}).  We show the
``only-if'' direction.  First observe that  for  any
n-event $\netev=\set{\locev\pp{\procev_1},\locev\q{\procev_2}}$ the
condition $\dualevS{\locev\pp{\procev_1}}
%\widehat\Join
{\locev\q{\procev_2}}$ of
\refToDef{n-event} implies
$\projb{\procev_1}{\q}\Join\projb{\procev_2}{\pp}$ by
\refToDef{dualLocEv}, which in turn implies
$\cardin{\projb{\procev_1}{\q}}=\cardin{\projb{\procev_2}{\pp}}$ by
\refToDef{dualProcEv}.   If $\netev$ is a binary event, 
%In a binary n-event 
we also have $\cardin{\procev_1}=\cardin{\projb{\procev_1}{\q}}$ and
$\cardin{\procev_2}=\cardin{\projb{\procev_2}{\pp}}$ by \refToDef{pd}, 
since  all the actions of $\procev_1$ involve $\q$ and all the
actions of $\procev_2$ involve $\pp$, and thus 
the projections do not erase actions. \\
Assume  now 
$\netev'=\set{\locev\pp{\procev_1'},\locev\q{\procev_2'}}$.
We consider two cases  (the others being symmetric):  
\begin{itemize}[label=--]
%Let $\netev\grr\netev'$ since $\procev_1\grr\procev_1'$.
\item $\netev\grr\netev'$  because 
  $\procev_1\grr\procev_1'$.
   Then 
$\projb{\procev_1}{\q}\Join\projb{\procev_2}{\pp}$ and
  $\projb{\procev_1'}{\q}\Join\projb{\procev_2'}{\pp}$
  imply $\procev_2\grr\procev_2'$; 
 \item $\netev\grr\netev'$  because 
  $\cardin{\projb{\procev_1}{\q}}=\cardin{\projb{\procev'_2}{\pp}}$
  and $\neg(\projb{\procev_1}{\q}\Join\projb{\procev'_2}{\pp})$. 
   As argued before, 
%for all n-events  %By duality, 
  we have $\cardin{\projb{\procev_2}{\pp}}
  =\cardin{\projb{\procev_1}{\q}}$ and
  $\cardin{\projb{\procev'_2}{\pp}} =\cardin{\projb{\procev'_1}{\q}}$.
  Then, from
  $\cardin{\projb{\procev_1}{\q}}=\cardin{\projb{\procev'_2}{\pp}}$
  and the above remark about binary events, we get
  $\cardin{\procev_2}=\cardin{\procev_1} =\cardin{\procev'_2} =
  \cardin{\procev'_1}$. From
  $\neg(\projb{\procev_1}{\q}\Join\projb{\procev'_2}{\pp})$ it follows
  that $\procev_1 \neq \procev'_1$ and $\procev_2
  \neq \procev'_2$. Then we may conclude, since $\cardin{\procev_i} =
  \cardin{\procev'_i}$ and $\procev_i \neq \procev'_i$ imply
  $\procev_i \grr \procev'_i$ for $i=1,2$.  \qedhere 
  \end{itemize} 
\end{proof} 

\begin{theorem}
%\mylabel{basta11}
\mylabel{binary-network-FES}
Let $\Nt = \pP{\pp_1}{\PP_1} \parN \pP{\pp_2}{\PP_2}\,$ and $\,\ESN{\Nt}
=  (\GE(\Nt), \precN _\Nt, \grr)$. Then  $\nr{\DE(\Nt)} =
  \DE(\Nt)$ and   the structure $\ESNstar{\Nt} \eqdef (\GE(\Nt), \precN^* _\Nt , \grr)$
is a prime event structure.
\end{theorem}
\begin{proof} 
  We first show that $\nr{\DE(\Nt)} = \DE(\Nt)$. By
  \refToDef{netev-relations}(\ref{netev-relations1}) 
%  \\
%  \centerline{$
\[
  \DE(\Nt) =
    \set{\set{\locev{\pp_1}{\procev_1},\locev{\pp_2}{\procev_2}} \sep
      \procev_1 \in \ES(\PP_1), \procev_2 \in \ES(\PP_2),
      \dualevS{\locev{\pp_1}{\procev_1}}{\locev{\pp_2}{\procev_2}}}
%      $}
\]
Let $\set{\locev{\pp_1}{\procev_1},\locev{\pp_2}{\procev_2}} \in
  \DE(\Nt)$. Since
  $\dualevS{\locev{\pp_1}{\procev_1}}{\locev{\pp_2}{\procev_2}}$ and
  all the actions in $\procev_1$ involve $\pp_2$ and all the actions
  in $\procev_2$ involve $\pp_1$, we know that $\procev_1$ and
  $\procev_2$ have the same length $n \geq 1$ and for each $i, 1\leq i
  \leq n$, the prefixes of length $i$ of $\procev_1$ and $\procev_2$,
  written $\procev^i_1$ and $\procev^i_2$, must themselves be
  dual. Then
  $\set{\locev{\pp_1}{\procev^i_1},\locev{\pp_2}{\procev^i_2}} \in
  \DE(\Nt)$ for each $i, 1\leq i \leq n$, hence
  $\set{\locev{\pp_1}{\procev_1},\locev{\pp_2}{\procev_2}}$ has a
  causal
  set in $\DE(\Nt)$.\\
  We prove now that the reflexive and transitive closure $\precN^*
  _\Nt$ of $\precN _\Nt$ is a partial order.  Since by definition $\precN^*
  _\Nt$ is a preorder, we only need to show that it is antisymmetric.
  Define the length of an n-event $\netev =
  \set{\locev{\pp_1}{\procev_1},\locev{\pp_2}{\procev_2}}$ to be
  $\length{\netev} \eqdef \, \cardin{\procev_1} + \cardin{\procev_2}$
  (where $\cardin{\procev}$ is the length of $\procev$, as
%defined in 
 given by 
\refToDef{proceventP}). 
 Let now $\netev, \netev'\in \GE(\Nt)$, with  $\netev =
\set{\locev{\pp_1}{\procev_1},\locev{\pp_2}{\procev_2}}$ and $\netev'
= \set{\locev{\pp_1}{\procev'_1},\locev{\pp_2}{\procev'_2}}$.  By
definition $\netev \precN _\Nt \netev'$ implies $\procev_i < \procev'_i$
for some $i=1,2$, which in turn implies $\cardin{\procev_i} <
\cardin{\procev'_i}$.  
 As observed above, 
$\procev_1$ and $\procev_2$ must have the
same length, and so must $\procev'_1$ and $\procev'_2$ . 
This means that if $\netev \precN _\Nt \netev'$ then
 $\length{\netev} = \cardin{\procev_1}+ \cardin{\procev_2} < 
\cardin{\procev'_1} + \cardin{\procev'_2} = \length{\netev'}$. 
From this we can conclude
that if $\netev \precN^* _\Nt \netev'$ and $\netev' \precN^* _\Nt \netev$, then
necessarily $ \netev = \netev'$.
\\
Finally we show that the relation $\grr$ satisfies the required
properties. By \refToTheorem{nf} we only need to prove that $\grr$ is
hereditary. Let $\netev$ and $\netev'$ be as above. If
$\netev\grr\netev'$, then by \refToLemma{bnc}
$\procev_1\grr\procev_1'$ and $\procev_2\grr\procev_2'$.  Let now
$\netev'' =
\set{\locev{\pp_1}{\procev''_1},\locev{\pp_2}{\procev''_2}}$. If
$\netev'\precN^* _\Nt\netev''$, this means that there exist $\netev_1,
\ldots, \netev_n$ such that $\netev'\precN  _\Nt\netev_1 \ldots
\precN _\Nt\netev_n=\netev''$. We prove by induction on $n$ that
$\netev\grr\netev''$. For $n=1$ we have $\netev'\precN _\Nt\netev''$.  Then
by Clause (\ref{c1}) of \refToDef{netev-relations} we have $\procev'_j
< \procev''_j$ for some $j\in\set{1,2}$.
 Since $\procev_i \grr \procev'_i$ for all $i\in\set{1,2}$ 
and $\grr$ is hereditary on p-events, we deduce
$\procev_j\grr\procev''_j$, which implies $\netev\grr\netev''$. 
Suppose now $n >1$. By induction $\netev\grr\netev_{n-1}$. Since
$\netev_{n-1} \precN _\Nt\netev_n=\netev''$ we then obtain
$\netev\grr\netev''$ by the same argument as in the base case.
\end{proof}


If  a network  %$\Nt$ 
has more than two participants, then the
duality requirement on its  n-events  is not sufficient to ensure the
absence of circular dependencies\footnote{This is a well-known issue
  in multiparty session types, which motivated the introduction of
  global types in \cite{CHY08}, see \refToSection{sec:types}.}.  For
instance, in the following ternary network (which may be viewed as
representing the 3-philosopher deadlock) the relation $\precN^*$ is
not a partial order.
%
\begin{example}\mylabel{ex2}
Let $\Nt$ be the network 
%\\
%\centerline{$
\[
\pP{\pp}{\Seq{\rcvL{\pr}{\la}}{\sendL{\q}{\la'}}} \parN
\pP{\q}{\Seq{\rcvL{\pp}{\la'}}{\sendL{\pr}{\la''}}} \parN
\pP{\pr}{\Seq{\rcvL{\q}{\la''}}{\sendL{\pp}{\la}}}
\]
%$}
Then  $\ESN{\Nt}$  has three  n-events
%\\
%\centerline{$
\[
\begin{array}{c}
\netev_1 = \set{\locev{\pp}{\rcvL{\pr}{\la}},\locev{\pr}{\concat{{\rcvL{\q}{\la''}}}{\sendL{\pp}{\la}}}}\qquad
\netev_2 =
\set{\locev{\pp}{\concat{{\rcvL{\pr}{\la}}}{\sendL{\q}{\la'}}},
\locev{\q}{\rcvL{\pp}{\la'}}}\\
\netev_3 = \set{\locev{\q}{\concat{{\rcvL{\pp}{\la'}}}{\sendL{\pr}{\la''}}},\locev{\pr}{\rcvL{\q}{\la''}}}
\end{array}
\]
%$}
By \refToDef{netev-relations}(\ref{c1}) 
we have
$\netev_1 \precN \netev_2 \precN \netev_3$ and  $\netev_3 \precN \netev_1$.
The only configuration of  $\ESN{\Nt}$  is the empty
configuration, because the only set of n-events that satisfies
%left-closure 
downward-closure up to conflicts
is $X =\set{\netev_1, \netev_2, \netev_3}$, 
but this is not a configuration because 
$\prec_X^*$ is not  a partial order (recall that $\prec_X$ is the
restriction of $\precN$ to $X$) and hence the condition (\ref{configF3}) of
\refToDef{configF} is not satisfied. 
\end{example}