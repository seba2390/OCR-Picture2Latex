% !TEX root =cdgS.tex

\section{Related Work and Conclusions}
\mylabel{sec:related}

Event Structures (ESs) were introduced in Winskel's PhD
Thesis~\cite{Win80} and in the seminal paper by Nielsen, Plotkin and
Winskel~\cite{NPW81}, roughly in the same frame of time as Milner's
calculus CCS~\cite{Mil80}. It is therefore not surprising that the
relationship between these two approaches for modelling concurrent
computations started to be investigated very soon afterwards. The
first interpretation of CCS into ESs was proposed by Winskel
in~\cite{Win82}. This interpretation made use of Stable ESs, because
PESs, the simplest form of ESs, appeared not to be flexible enough to
account for CCS parallel composition. Indeed, since CCS parallel
composition allows for two concurrent complementary actions to either
synchronise or occur independently in any order, each pair of such
actions gives rise to two forking computations: this requires
duplication of the same continuation process for these forking
computations in PESs, while the continuation process may be shared by
the forking computations in Stable ESs, which allow for disjunctive
causality.  Subsequently, ESs (as well as other nonsequential
``denotational models'' for concurrency such as Petri Nets) have been
used as the touchstone for assessing noninterleaving operational
semantics for CCS: for instance, the pomset semantics for CCS by
Boudol and Castellani~\cite{BC87,BC88a} and the semantics based on
``concurrent histories'' proposed by Degano, De Nicola and
Montanari~\cite{DM87,DDM88,DDM90}, were both shown to agree with an
interpretation of CCS processes into some class of ESs (PESs
for~\cite{DDM88,DDM90}, PESs with non-hereditary conflict
for~\cite{BC87}, and FESs for~\cite{BC88a}).  Among the early
interpretations of process calculi into ESs, we should also mention
the PES semantics for TCSP (Theoretical CSP~\cite{BHR84,Old86}),
proposed by Goltz and Loogen~\cite{LG91} and later generalised by
Baier and Majster-Cederbaum~\cite{BM94}, and the Bundle ES semantics
for LOTOS, proposed by Langerak~\cite{Lan93} and extended by
Katoen~\cite{Kat96}.  Like FESs, Bundle ESs are a subclass of Stable
ESs. We recall the relationships between the above
classes of ESs (the reader is referred to~\cite{BC94} for separating examples):
%\\[2pt]
%\centerline{$
\[
Prime~ESs \subset Bundle~ESs \subset Flow~ESs \subset
  Stable~ESs \subset General~ESs 
  \]
%$}

More sophisticated ES semantics for CCS, based on FESs and designed to
be robust under action refinement~\cite{AH89,DD93,GGR96}, were 
subsequently  proposed by Goltz and van
Glabbeek~\cite{GG04}. Importantly, all the above-mentioned classes of
ESs, except General ESs, give rise to the same \emph{prime algebraic
  domains} of configurations, from which one can recover a PES by
selecting the complete prime elements.

More recently, ES semantics have been investigated for the
$\pi$-calculus by Crafa, Varacca and Yoshida~\cite{CVY07,VY10,CVY12}
and by Cristescu, Krivine and Varacca~\cite{Cri15,CKV15,CKV16}. 
Previously,  other
causal models for the $\pi$-calculus had already been put forward by
Jategaonkar and Jagadeesan~\cite{JJ95}, by Montanari and
Pistore~\cite{MP95}, by Cattani and Sewell~\cite{CS04} and by Bruni,
Melgratti and Montanari~\cite{BMM06}.  The main new issue, when
addressing causality-based semantics for the $\pi$-calculus, is the
implicit causality induced by scope extrusion. Two alternative views
of such implicit causality had been proposed in  early  work on
noninterleaving operational semantics for the $\pi$-calculus,
respectively by Boreale and Sangiorgi~\cite{BS98} and by Degano and
Priami~\cite{DP99}.  Essentially, in~\cite{BS98} an \emph{extruder}
(that is, an output of a private name) is considered to cause any
action that uses the extruded name, whether in subject or object
position, while in~\cite{DP99} it is considered to cause only the
actions that use the extruded name in subject position. Thus, for
instance, in the process $P = \nu a \,(\overline{b} \langle a\rangle
\pc \overline{c} \langle a\rangle \pc a)$, the two parallel extruders
are considered to be causally dependent in the former approach, and
independent in the latter. All the causal models for the
$\pi$-calculus mentioned above, including the ES-based ones, take one
or the other of these two stands.  Note that opting for the second one
leads necessarily to a non-stable ES model, where there may be causal
ambiguity within the configurations themselves: for instance, in the
above example the maximal configuration contains three events, the
extruders $\overline{b}\langle a\rangle$, $\overline{c} \langle
a\rangle$ and the input on $a$, and one does not know which of the two
extruders enabled the input. Indeed, the paper~\cite{CVY12} uses
non-stable ESs.  The use of non-stable ESs (General ESs) to express
situations where a computational step can merge parts of the state is
advocated for instance by Baldan, Corradini and Gadducci
in~\cite{BCG17}. These ESs give rise to configuration domains that are
not prime algebraic, hence the classical representation theorems have
to be adjusted.

In our simple setting, where we deal only with single sessions and do
not consider session interleaving nor delegation, we can dispense with
channels altogether, and therefore the question of parallel extrusion
does not arise. In this sense, our notion of causality is closer to
that of CCS than to the more complex one of the
$\pi$-calculus. However, even in a more general setting, where
participants would be paired with the channel name of the session they
pertain to, the issue of parallel extrusion would not
arise: indeed, in the above example $b$ and $c$ should be equal,
because participants can only delegate their own channel, but then
they could not be in parallel because of linearity, one of the
distinguishing features enforced by session types. Hence
we believe that in a session-based framework
the two above views of implicit causality should collapse into just
one.

 
We now briefly discuss our design choices. 
\begin{itemize}
\item  The calculus
considered in the present paper  uses synchronous communication -
rather than asynchronous, buffered communication - because this is how
communication is  classically  modelled in ESs, when they
are used to give semantics to process calculi.   We should
mention however that after first proposing the present study
in~\cite{CDG-LNCS19}, we also considered a calculus with asynchronous
communication in the companion paper~\cite{CDG21}.  In that work too,
networks are interpreted as FESs, and their associated global types,
which we called \emph{asynchronous types} as they split communications
into outputs and inputs, are interpreted as PESs.  The key result is
again an isomorphism between the configuration domain of the FES of a
typed network and that of the PES of its type.  
%
\item Concerning the choice operator, we adopted here the basic (and
  most restrictive) variant for it, as it was originally proposed for
  multiparty session calculi in \cite{CHY08}.
% where the sender may send different messages to the same
% receiver, 
% We did many choices to shorten the description of the calculus
% and of the types. 
This is essentially a simplifying assumption, and we do not foresee any difficulty in extending our results
%Our results hold also if the communication is asynchronous
%\cite{CDG17}, and the syntax of the global types allows
to a more general choice operator,
% allowing for different receivers,
where the projection is rendered more flexible through the use of  a merge operator~\cite{DY11}. 
% \cite{CHY16}.  
%
\item 
%Finally, concerning 
 As regards the preorder on processes, which is akin to a subtyping relation, we envisaged to use the
standard subtyping,  in which a process with fewer outputs
 can be used in place of 
%smaller than 
 a process with more outputs.  However, in that case Session Fidelity
 would become weaker, since a transition in the LTS of a global type
 would only ensure a transition in the LTS of the corresponding
 network, but not necessarily with the same labelling communication.
 The main drawback would be that \refToTheorem{iso} would no longer
 hold: more precisely, the domains of network configurations would
 only be embedded in (and not isomorphic to) the domains of their
 global type configurations. Notably, typability is independent from
 the use of our preorder or of the standard one, as proved
 in~\cite{BDLT21}.
\end{itemize}

As regards future work, we plan to define an asynchronous
transition system (ATS)~\cite{Bed88} for our calculus, along the lines
of~\cite{BC94}, and show that it provides a noninterleaving
operational semantics for networks that is equivalent to their FES
semantics. This would enable us also to investigate the issue of
reversibility, jointly on our networks and on their FES
representations, since the ATS semantics would give us the handle to
unwind networks, while the corresponding FESs could be unrolled
following one of the methods proposed in existing work on reversible
event structures~\cite{PU2016,CKV16,GPY19,GPY21,Gra21}.

As mentioned at the end of \refToSection{sec:events}, the quest
for a semantic counterpart of our well-formedness conditions on global
types -- namely, for properties that characterise the FESs obtained
from typable networks -- is still open.  By way of comparison, such
semantic well-formedness conditions have been proposed in~\cite{TuostoG18} for
\emph{graphical choreographies}, a truly concurrent
graphical model for global specifications with two kinds of forking
nodes, representing respectively choice and parallel
composition. In~\cite{TuostoG18}, those well-formedness conditions, called
\emph{well-sequencing} and \emph{well-branchedness}, were shown to be
sufficient to ensure projectability on local specifications. In our
case, the property corresponding to well-sequencing is automatically
ensured by our ES semantics, and we conjecture that the well-branchedness
condition for choice nodes (corresponding to projectability) could amount in our
simpler setting\footnote{Our choice operator for global types is less
general than that of~\cite{TuostoG18}.} to the following semantic
condition:

Let $\netev_1, \netev_2\in \GE(N)$ and $\locev{\pp}{\actseq\cdot\pi}\in\netev_1$
and $\locev{\pp}{\actseq\cdot\pi'}\in\netev_2$ with
$\pi\neq\pi'$ and 
$\q = \ptone{\pi}=\ptone{\pi'}$.
If $\netev_1\precN^*\netev'_1$ for some $\netev'_1\in \GE(N)$
such that $\pr\in\loc{\netev'_1}$ with $\pr\not\in \set{\pp, \q}$, then
$\netev_2\precN^*\netev'_2$ for some $\netev'_2\in \GE(N)$ such that
$\pr\in\loc{\netev'_2}$.

This condition would allow us to rule out the FESs of both networks
$\Nt'$ and $\Nt''$ discussed at page~\pageref{wf-discussion}. However,
it should be completed with a condition corresponding to boundedness, and the
conjunction of these two conditions might still not be sufficient in general to ensure typability.
We plan to further investigate this question in the near future.







