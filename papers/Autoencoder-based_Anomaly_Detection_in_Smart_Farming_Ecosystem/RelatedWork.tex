\section{Related Work}
\label{sec:related}
The exponential rise in number of internet connected devices has raised security concerns, especially in the agriculture sector, as farmers will not be able to bear the potential loss and damage to crops. 
Gupta et al. \cite{gupta2020security} developed a comprehensive survey of issues in the security and privacy of the IoT in 2020. The paper covers the architecture of a smart farm environment in-depth and potential real world attack scenarios. It focuses on smart farming in its mature form, in which devices are interconnected with one another and also connect to the internet.  Jeba et al. collected pH and moisture data from soil sensors connected to an Arduino then visualized the data using ThingSpeak. Thingspeak provides a way to visualize and monitor farm conditions through the use of WiFi. In our deployed architecture, we collected data locally using a raspberry pie, acting as a smart edge based system \cite{jeba2018anomaly}. Sontowski et al. \cite{sontowski2020cyber} described various types of network attacks that could be orchestrated on smart farms. The authors also targeted a smart farming test bed with a Denial of Service (DoS) attack. This attack and the others described in the paper are ways an attacker can manipulate or harm the system. The approach presented in our work will be able to identify certain behavior in the form of anomalies and later potentially report the findings to the user. %Specific research into sensor behavior without internet connectivity is what we aim to achieve from our paper.

A seminal work by Chandola et al. \cite{chandola2009anomaly} titled ``Anomaly Detection: A Survey" offers a comprehensive study of types of data, anomalies, and techniques for detection. It also details all domains in which anomaly detection has been used. This detailed survey provided the authors of this paper with much of their fundamental understanding of the topic. Several papers \cite{park2021anomaly,hasan2019attack, cook2019anomaly, luo2018distributed, chukkapalli2020ontologies, sedjelmaci2016lightweight,kimmell2021analyzing,gupta2021detecting} offered specifics on anomalies in smart ecosystems. In smart farming, the sensors used in a connected farm are often exposed to harsh environments. The conditions make the devices prove to failure, malfunction, attacks, tampering, etc. Any of these conditions could cause abnormal device readings, which would be considered anomalous compared to normal data \cite{gaddam2020detecting}. The low cost of IoT devices in general and the associated implications are further elaborated in later sections of this paper. 
%Some research works offered insight or new solutions for anomaly detection for IoT data. 
Kotevska et al. \cite{kotevska2019kensor} from Oak Ridge National Laboratory created an algorithm called "Kensor" which aims to achieve detection of normal and abnormal behavior, except instead of individual sensor data they explored co-located/coordinated sensors. These sensors are located in close proximity to one another and the combined collected data from each is used to paint a picture of normal or abnormal. This algorithm offers a solution to more interconnected systems than ours.  However, the inclusion of co-located or coordinated sensors aids the anomaly detection tool in its performance and will be considered in future work. Guo et al. \cite{guo2018multidimensional} proposed a model called "GRU-based Gaussian Mixture VAE system" for anomaly detection in multivariate time-series data. GRU (Gated Recurrent Unit) cells are used to discover correlations among time sequences. Gaussian mixture means that the model combines several Gaussian distributions rather than the more common distribution, Gaussian single-modal. The authors found that this model outperformed traditional Variational Autoencoders (VAEs) in tests on accuracy and F1 score.  *** Yin et al. \cite{8986829} tested a Convolutional Neural Network (CNN) for anomaly detection on time-series data. The authors found that the resulting metrics were promising and achieved a desirable result in anomaly detection. BigClue Analytics \cite{hurubigclue} is a middle-ware solution that offers data approximation, sampling, parallel processing, and anomaly detection in a low-latency scenario. Our focus is not currently in low-latency solutions. However, this would be useful for systems already "online". They tested statistical algorithms, as well as linear regression, signal decomposition, and other methods of anomaly detection. The authors also mention several downfalls of statistical methods for time-series anomaly detection including: missed subtle outliers, inability to detect multiple consecutive outliers, lack of predicted values, etc. In our work, we chose not to use a statistical machine learning method for our system for these and other reasons mentioned in Section \ref{sec:ml-model}. The BigClue Analytics paper chose to use a smart greenhouse as their use case as well. Their interval of sampling was much longer than ours at every 15 minutes, and we believe that this amount of time is too long to wait between samples. They also only evaluated temperature and humidity data, whereas we look at several more sensor outputs. Lastly, ARIMA \cite{zhang2003time} was suggested as a popular time-series analysis technique that we intend to explore in our future work. All of these related works offered solutions that helped us refine our choices involving model selection. We will consider some of the approaches in our future work.

Throughout our literature review, we found that there is an overall lack in studies done on the behavior of specific sensors. We wanted to contribute to the understanding of IoT sensors for smart farming by creating a working environment and collecting a large set of data. We will also be publishing the dataset for public use. Since smart farming is a relatively new domain, we found that efforts to improve the systems and/or detect anomalies were lacking. This is what we hope to offer with this paper and future work.