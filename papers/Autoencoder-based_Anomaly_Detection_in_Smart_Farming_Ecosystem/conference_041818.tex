\documentclass[conference]{IEEEtran}
\IEEEoverridecommandlockouts
% The preceding line is only needed to identify funding in the first footnote. If that is unneeded, please comment it out.
\usepackage{cite}
\usepackage{gensymb}
\usepackage{amsmath,amssymb,amsfonts}
\usepackage{algorithmic}
\usepackage{graphicx}
\usepackage{soul}
\usepackage{textcomp}
\usepackage{xcolor}
\def\BibTeX{{\rm B\kern-.05em{\sc i\kern-.025em b}\kern-.08em
    T\kern-.1667em\lower.7ex\hbox{E}\kern-.125emX}}
\begin{document}

\title{Autoencoder-based Anomaly Detection \\in Smart Farming Ecosystem}
%Using Autoencoder for Anomaly Detection \\in Smart Farming Ecosystem
%Deep Autoencoder-based Anomaly Detection \\in Smart Farming Ecosystem
%Smart Farming Anomaly Detection
% Using Autoencoder for Anomaly Detection in Smart Farming 


\author{\IEEEauthorblockN{Mary Adkisson\IEEEauthorrefmark{1}, Jeffrey C Kimmell\IEEEauthorrefmark{2}, Maanak Gupta\IEEEauthorrefmark{3}, and Mahmoud Abdelsalam\IEEEauthorrefmark{4}}
\IEEEauthorblockA{\IEEEauthorrefmark{1}\IEEEauthorrefmark{2}\IEEEauthorrefmark{3}{Dept. of Computer Science},
{Tennessee Technological University},
Cookeville, Tennessee 38505, USA \\\IEEEauthorrefmark{4}Dept. of Computer Science, North Carolina A\&T State University,
Greensboro, NC, USA\\}
\IEEEauthorrefmark{1}maadkisson42@tntech.edu,
\IEEEauthorrefmark{2}jckimmell42@tntech.edu, 
\IEEEauthorrefmark{3}mgupta@tntech.edu
\IEEEauthorrefmark{4}mabdelsalam01@ncat.edu}

\iffalse
\author{\IEEEauthorblockN{1\textsuperscript{st} Mary Adkisson}
\IEEEauthorblockA{\textit{Tennessee Tech University} \\
\textit{Computer Science Dept.}\\
Cookeville, TN USA \\
maadkisson42@tntech.edu}
\and
\IEEEauthorblockN{2\textsuperscript{nd} Jeffrey Kimmell}
\IEEEauthorblockA{\textit{Tennessee Tech University} \\
\textit{Computer Science Dept.}\\
Cookeville, TN USA \\
jckimmell42@tntech.edu}
\and
\IEEEauthorblockN{3\textsuperscript{rd} Maanak Gupta}
\IEEEauthorblockA{\textit{Tennessee Tech University} \\
\textit{Computer Science Dept.}\\
Cookeville, TN USA \\
mgupta@tntech.edu}

}
\fi

\maketitle



%that another person, despite having different goals, is analogous to us - we both require food, exert energy and feel similar degrees of pain and pleasure. 
%agents are generally similar - e.g. 
%incur time step penalties, avoid dangerous features, or 
%such that the corresponding optimal action, when viewed through the agent’s own action-value function, matches the action taken by the other agent. 
%our architecture produces an empathetic representation of the other agent's observations that a


% Stephan
\iffalse
We can usually assume others have goals analogous to our own, and this assumption is useful for human cognitive processes like empathy where we can build on similarities between us and others to better understand the behaviour of others. Inspired by empathy we apply this process to multi-agent games by designing a simple and interpretable architecture to model another agent's action-value function. This involves learning an \emph{Imagination Network} to transform the other agent's observed state in order to produce a human-interpretable \emph{empathetic state} which, when presented to the learning agent, produces behaviours that mimic the other agent. Our approach is applicable to multi-agent scenarios consisting of a single learning agent and other (independent) agents acting according to fixed policies. This architecture is particularly beneficial for (but not limited to) algorithms using a composite value or reward function. We show our method produces better performance in multi-agent games, where it robustly estimates the other's model in different environment configurations. Additionally, we show that the empathetic states are human interpretable, and thus verifiable.
\fi

We can usually assume others have goals analogous to our own. This assumption can also, at times, be applied to multi-agent games - e.g. Agent 1's attraction to green pellets is analogous to Agent 2's attraction to red pellets. This ``analogy'' assumption is tied closely to the cognitive process known as empathy. Inspired by empathy, we design a simple and explainable architecture to model another agent's action-value function. This involves learning an \emph{Imagination Network} to transform the other agent's observed state in order to produce a human-interpretable \emph{empathetic state} which, when presented to the learning agent, produces behaviours that mimic the other agent. Our approach is applicable to multi-agent scenarios consisting of a single learning agent and other (independent) agents acting according to fixed policies. This architecture is particularly beneficial for (but not limited to) algorithms using a composite value or reward function. We show our method produces better performance in multi-agent games, where it robustly estimates the other's model in different environment configurations. Additionally, we show that the empathetic states are human interpretable, and thus verifiable.


%Getting along with someone else isn't always easy, but it is vital for a stable and productive society. Communication and understanding are key for smooth social interactions, leading to better cooperation. Luckily for us evolution has bestowed humans and animals the ability to empathise. Empathy tries to understand the feelings and goals of another, by using ourself as a point of reference. How would we feel if we were in a similar situation? What fears and goals do we share? What objects or goals, though not exactly the same, do we feel similarly towards? Despite our biological ability to empathise, understanding another is really tricky and we humans rely on a vast set of information sources to understand the other. These are not limited to verbal and written streams, but also encompass behaviours, body language and subtle verbal and phsyical cues, each of which are small clues to the others hidden state. Now if you thought empathy between humans was hard, try between agents! Or between a human and an agent! With the increasing prevalence of artifically intelligent agents, particularly in the form of robots or virtual agents, their presence in shared environments with other agents or humans will necessesitate the need to understand others. In this work we integrate an empathy-based architecture for modeling `the other' in artificial agents. We hone in on settings where two analagous agents share a space. Of these two, the learning agent (the one we train) tries to models the other `independent agent' by referencing its own rewards and value function. As there is yet no evidence to believe agents have `feelings', we instead utilise our architecture to understand the other's goals (articulate as a reward function). Our proposed architecture is a two stage neural network, the first of which constructs an empathetic representation of the independent agents state, before sending this representation into a copy of its own value function (the second network). In this way, our significant contribution is the ability to generate an empathetic state that is human interpretable, allowing a user to clearly see what features both agents share (those that elicit empathy). In turn, the model can be used to infer the underlying rewards  of the other agent which can be used to elicit cooperative behaviour in the learning agent.

\iffalse
Smooth social interactions are 
%driven by our ability to understand how others feel and behave. This is
enabled via mechanisms such as empathy, which allows us to understand how another is feeling by referencing our own emotions from similar situations. 
%Artificially intelligent systems that learn from interactions with other agents and humans are becoming increasingly common. 
As artificially intelligent systems become more prevalent in applications involving interactions with other agents and humans, it becomes increasingly important that they too exhibit such empathetic qualities. In this work we propose a novel empathy-based approach to enable a learning agent to understand the rewards and values of another agent (the independent agent) by referencing its own reward and value function. 
The empathy mechanism is realised through a two stage neural network architecture, the first of which reconstructs the independent agent's state from the learning agent's perspective, following which it is passed through the learning agent's value function network. 
%We train our learning agent to model the independent agent via an empathy-based architecture. 
%Applied to symmetric games (games in which the goals of each agent are similar, but not necessarily the same), our empathy mechanism allows us to model the independent agent's behaviours by grounding them in the learning agent's own reward and value functions. This in turn enables it to interact with the independent agent in a more context-informed manner.
Through this empathy-based approach, we (1) learn an estimate of the independent agent's reward function that is consistent with the scale of the learning agent's own reward function and (2) produce an empathetic representation of the independent agent's state that is grounded in the learning agent's own value function and experiences. %These aspects allow the learning agent to effectively account for the presence of other agents in its environment.
%the benefits of empathy are the ability to (1) infer a more comparable reward and value function for the independent agent, and (2) the ability to produce an empathetic representation of the independent agent's state as imagined by the learning agent.
%\thommen{Not necessarily human understandable, right? Should we say empathetic state representations?}
%\thommen{Points 1 and 2 are relative to the sympathy paper}\manisha{I don't understand what you mean here}\thommen{I mean that when you say 'more comparable reward and value function..', it automatically implies that this contribution is in relation to the sympathy paper - and at this point, the reader has no context of that paper. Maybe a better way to say it is that with empathy, we can model the independent agent's behaviors by grounding them in the learning agent's own reward and value functions, which may enable it to interact with the independent agent in a more context-informed manner. Specifically, our empathy-based framework (1) learns an estimate of the independent agent's reward function that is consistent with the scale of its own reward function and (2) produces representations of the independent agent's states that are grounded in its own value function and experiences(?). These aspects allow the learning agent to effectively account for the presence of other agents in its environment.} 
%interpretable (can we also say more accurate?) to the learning agent.}
We demonstrate the benefits of these estimates and representations in a variety of environments and show that our learning agent is able to behave considerately towards the independent agent whilst still completing its own task. 
\fi

% !TEX root = ../arxiv.tex

Unsupervised domain adaptation (UDA) is a variant of semi-supervised learning \cite{blum1998combining}, where the available unlabelled data comes from a different distribution than the annotated dataset \cite{Ben-DavidBCP06}.
A case in point is to exploit synthetic data, where annotation is more accessible compared to the costly labelling of real-world images \cite{RichterVRK16,RosSMVL16}.
Along with some success in addressing UDA for semantic segmentation \cite{TsaiHSS0C18,VuJBCP19,0001S20,ZouYKW18}, the developed methods are growing increasingly sophisticated and often combine style transfer networks, adversarial training or network ensembles \cite{KimB20a,LiYV19,TsaiSSC19,Yang_2020_ECCV}.
This increase in model complexity impedes reproducibility, potentially slowing further progress.

In this work, we propose a UDA framework reaching state-of-the-art segmentation accuracy (measured by the Intersection-over-Union, IoU) without incurring substantial training efforts.
Toward this goal, we adopt a simple semi-supervised approach, \emph{self-training} \cite{ChenWB11,lee2013pseudo,ZouYKW18}, used in recent works only in conjunction with adversarial training or network ensembles \cite{ChoiKK19,KimB20a,Mei_2020_ECCV,Wang_2020_ECCV,0001S20,Zheng_2020_IJCV,ZhengY20}.
By contrast, we use self-training \emph{standalone}.
Compared to previous self-training methods \cite{ChenLCCCZAS20,Li_2020_ECCV,subhani2020learning,ZouYKW18,ZouYLKW19}, our approach also sidesteps the inconvenience of multiple training rounds, as they often require expert intervention between consecutive rounds.
We train our model using co-evolving pseudo labels end-to-end without such need.

\begin{figure}[t]%
    \centering
    \def\svgwidth{\linewidth}
    \input{figures/preview/bars.pdf_tex}
    \caption{\textbf{Results preview.} Unlike much recent work that combines multiple training paradigms, such as adversarial training and style transfer, our approach retains the modest single-round training complexity of self-training, yet improves the state of the art for adapting semantic segmentation by a significant margin.}
    \label{fig:preview}
\end{figure}

Our method leverages the ubiquitous \emph{data augmentation} techniques from fully supervised learning \cite{deeplabv3plus2018,ZhaoSQWJ17}: photometric jitter, flipping and multi-scale cropping.
We enforce \emph{consistency} of the semantic maps produced by the model across these image perturbations.
The following assumption formalises the key premise:

\myparagraph{Assumption 1.}
Let $f: \mathcal{I} \rightarrow \mathcal{M}$ represent a pixelwise mapping from images $\mathcal{I}$ to semantic output $\mathcal{M}$.
Denote $\rho_{\bm{\epsilon}}: \mathcal{I} \rightarrow \mathcal{I}$ a photometric image transform and, similarly, $\tau_{\bm{\epsilon}'}: \mathcal{I} \rightarrow \mathcal{I}$ a spatial similarity transformation, where $\bm{\epsilon},\bm{\epsilon}'\sim p(\cdot)$ are control variables following some pre-defined density (\eg, $p \equiv \mathcal{N}(0, 1)$).
Then, for any image $I \in \mathcal{I}$, $f$ is \emph{invariant} under $\rho_{\bm{\epsilon}}$ and \emph{equivariant} under $\tau_{\bm{\epsilon}'}$, \ie~$f(\rho_{\bm{\epsilon}}(I)) = f(I)$ and $f(\tau_{\bm{\epsilon}'}(I)) = \tau_{\bm{\epsilon}'}(f(I))$.

\smallskip
\noindent Next, we introduce a training framework using a \emph{momentum network} -- a slowly advancing copy of the original model.
The momentum network provides stable, yet recent targets for model updates, as opposed to the fixed supervision in model distillation \cite{Chen0G18,Zheng_2020_IJCV,ZhengY20}.
We also re-visit the problem of long-tail recognition in the context of generating pseudo labels for self-supervision.
In particular, we maintain an \emph{exponentially moving class prior} used to discount the confidence thresholds for those classes with few samples and increase their relative contribution to the training loss.
Our framework is simple to train, adds moderate computational overhead compared to a fully supervised setup, yet sets a new state of the art on established benchmarks (\cf \cref{fig:preview}).


\section{Related work}\label{sect:related}

\paragraph{{Recovery}} {The works most closely most closely related to ours are those based on the \emph{recovery} notion, that is, the type system of Gordon et al. \cite{GordonEtAl12} and the Pony language  \cite{ClebschEtAl15}.} Indeed, the capsule property has many variants in the literature, such as \emph{isolated} \cite{GordonEtAl12}, \emph{uniqueness} \cite{Boyland10} and \emph{external uniqueness}~\cite{ClarkeWrigstad03}, \emph{balloon} \cite{Almeida97,ServettoEtAl13a}, \emph{island} \cite{DietlEtAl07}. 
%The fact that aliasing can be controlled by using \emph{lent} (\emph{borrowed}) references is well-known~\cite{Boyland01,NadenEtAl12}.
However, before the work of Gordon et al. \cite{GordonEtAl12}, the capsule property was only ensured in simple situations, such as using a primitive deep clone operator, or composing subexpressions with the same property.

The important novelty of the type system of Gordon et al. \cite{GordonEtAl12} has been \emph{recovery}, that is, the ability to ensure properties (e.g., capsule or immutability) by keeping into account not only the expression itself but the way the surrounding context is used. {Notably,} an expression which does not use external mutable references is recognized to be a capsule. 
{In the Pony language \cite{ClebschEtAl15}  the ideas of Gordon et al. \cite{GordonEtAl12} are extended to a richer set of reference immutability permissions. In their terminology \texttt{value} is immutable, \texttt{ref} is mutable, \texttt{box} is similar to \emph{readonly} as often found in literature, different from our $\readable$ since it can be aliased. An ephemeral isolated reference \lstinline{iso^} is similar to a $\capsule$ reference in our calculus, whereas non ephemeral \texttt{iso} references offer destructive reads and are more
similar to isolated fields \cite{GordonEtAl12}. Finally, \texttt{tag} only allows object identity checks and \texttt{trn} (transition) is a subtype of \texttt{box} that can be converted to \texttt{value}, providing a way to create values without using isolated references. The last two qualifiers have no equivalent in our
work or in  \cite{GordonEtAl12}.}

Our {type system greatly enhances the recovery mechanism used in such previous work \cite{GordonEtAl12,ClebschEtAl15} by using lent references, and rules \rn{t-swap} and \rn{t-unrst}.} For instance, the examples in \refToFigure{TypingOne} and \refToFigure{TypingTwo} would be ill-typed in \cite{GordonEtAl12}. 

{A minor difference with the type systems of Gordon et al. \cite{GordonEtAl12} and Pony \cite{GordonEtAl12,ClebschEtAl15} is that we only allow fields to be $\mutable$ or $\imm$.
Allowing \emph{readonly} fields means holding a reference that is useful for observing but non making remote modifications. However, our type system supports the $\readable$ modifier rather than the \emph{readonly}, and the $\readable$ qualifier includes the $\lent$ restriction. Since something which is $\lent$ cannot be saved as part of a $\mutable$ object, $\lent$ fields are not compatible with the current design where objects are born $\mutable$. The motivation for supporting $\readable$ rather than \emph{readonly} is discussed in a specific point later.
Allowing $\capsule$ fields means that programs can store an externally unique object graph into the heap and decide later whether to unpack
 permanently or freeze the reachable objects.  This can be useful, but, as for $\readable$ versus \emph{readonly}, our opinion is that this power is hard to use for good, since
it requires destructive reads, as discussed in a specific point later. 
In most cases, the same expressive power can be achieved by having the
field as $\mutable$ and recovering the $\capsule$ property for the outer object.}

\paragraph{Capabilities}
 {In other proposals \cite{HallerOdersky10,CastegrenWrigstad16} types are compositions of one or more \emph{capabilities}. The modes of the capabilities in a type control how resources of that
type can be aliased. The compositional aspect of capabilities is an important difference
from type qualifiers, as accessing different parts of an object through different capabilities in the same type gives different properties. 
By using capabilities it is possible to obtain an expressivity which looks similar to our type system, even though with different sharing notions and syntactic constructs. For instance, the \emph{full encapsulation} notion in \cite{HallerOdersky10}\footnote{{This paper includes a very good survey of work in this area, notably explaining the difference between \emph{external uniqueness}~\cite{ClarkeWrigstad03} and \emph{full encapsulation}.}}, apart from the fact that sharing of immutable objects is not allowed, is equivalent to the guarantee of our $\capsule$ qualifier, while
our $\lent$ and their \Q|@transient| achieve similar results in different ways.}
Their model has a higher syntactic/logic overhead to explicitly  track regions.
As for all work before~\cite{GordonEtAl12}, objects need to be born \Q|@unique| and the type system 
permits to manipulate data preserving their uniqueness. With recovery~\cite{GordonEtAl12},
instead, we can forget about uniqueness, use normal code designed to work on conventional shared data, and then
recover the aliasing encapsulation property.

\paragraph{Destructive reads} Uniqueness can be enforced by destructive reads, i.e., assigning a copy of 
the unique reference to a variable an destroying the original reference, see
\cite{GordonEtAl12,Boyland10}. Traditionally, borrowing/fractional permissions~\cite{NadenEtAl12} are related to uniqueness  in the opposite way: a unique reference can be borrowed,
it is possible to track when all borrowed aliases are buried~\cite{Boyland01}, and then uniqueness can be recovered.
These techniques offers a sophisticate alternative to destructive reads. 
We also wish to avoid destructive reads. In our work, we ensure uniqueness by linearity, that is, by allowing at most
one use of a $\capsule$ reference.

In our opinion, programming with destructive reads is involved and hurts the correctness of the program, since it leads to the style of programming outlined below, where \Q@a.f@ is a unique/isolated field with destructive read.
\begin{lstlisting}
a.f=c.doStuff(a.f)//style suggested by destructive reads
\end{lstlisting}
The object referenced by \lstinline{a}{} has an \emph{unique/isolated} field \lstinline{f} containing an object \lstinline{b}.
This object \lstinline{b}{} is passed to a client \lstinline{c}{}, which can use (potentially modifying) it. A typical pattern is that the result of such computation is a reference to \lstinline{b}{}, which \lstinline{a}{} can then recover. This approach allows \emph{isolated} fields, as shown above, but has  a serious drawback:
an \emph{isolated} field can become unexpectedly not available (in the example, during execution of \lstinline{doStuff}{}), hence any object contract
involving such field can be broken.
{This can cause {subtle} bugs if \Q@a@ is in the reachable object graph of \Q@c@.}

In our approach, the  $\capsule$ qualifier cannot be applied to fields. Indeed, the ``only once'' use of capsule variables 
makes no sense on fields.
{However, we support the same level of control of the reachable object graph by passing mutable objects to clients as $\lent$, in order to control aliasing behaviour.
That is, the previous code can be rewritten} as follows:
\begin{lstlisting}
c.doStuff(a.f())//our suggested style
\end{lstlisting}
{where \Q@a.f()@ is a getter returning the field as $\lent$.
Note how, during the execution of \Q@doStuff@, \Q@a.f()@ is still available, and,} after the execution of \Q@doStuff@, the aliasing relation {for this field is the same as it was
before \Q@doStuff@ was called.}

\paragraph{Ownership} A {closely related} stream of research is that on \emph{ownership} (see an overview in~\cite{ClarkeEtAl13}) which, however, offers an {opposite} approach. In the ownership approach, it is provided a way to express and prove the ownership invariant\footnote{{Ownership invariant (owner-as-dominator):
Object $o_1$ is owned by object  $o_2$ if in the object graph $o_2$
is a dominator node for $o_1$;
that is, all paths from the roots of the graph (the stack variables)
to $o_1$ pass throw $o_2$.
Ownership invariant (owner-as-modifier):
Object $o_1$ is owned by object  $o_2$ if any field update over $o_1$
is initiated by $o_2$, that is, a call of a method of $o_2$ is present
in the stack trace.}}, which, however, is expected to be guaranteed by defensive cloning, as explained below. In our approach, instead, the capsule concept models an efficient \emph{ownership transfer}. In other words, when an object $\x$ is ``owned'' by another object $\y$, it remains always true that $\y$ can be only accessed only through $\x$, whereas the capsule notion is more dynamic: a capsule can be ``opened'', that is, assigned to a standard reference and modified, and then we can recover the original capsule guarantee. 

For example, assuming a graph with a list of nodes, and a constructor taking in input such list,
the following code establishes the ownership invariant using $\capsule$, and ensures that it cannot be violated using $\lent$.
\begin{lstlisting}
class Graph{
  private final NodeList nodes;
  private Graph(NodeList nodes){this.nodes=nodes; }

  static Graph factory(capsule NodeList nodes){
    return new Graph(nodes);
    }
  
  lent NodeList borrowNodes(mut){return nodes;}
}
\end{lstlisting}
Requiring the parameter of the \lstinline{factory}{} method to be a $\capsule$ guarantees that the list of nodes provided as argument is not referred from the external environment. 
The factory \emph{moves} an isolated portion of store as local store of the newly created object. 
Cloning, if needed, becomes responsibility of the client which provides the list of nodes to the graph. The getter tailors the exposure level of the private store. 

Without aliasing control ($\capsule$ qualifier),  in order to ensure ownership of its list of nodes, the {factory method} should clone the argument, since it comes from an external client environment.
This solution, called  \label{cloning} \emph{defensive cloning}~\cite{Bloch08}, is very popular in the Java community, but inefficient,
since it requires to duplicate the reachable object
graph of the parameter, until immutable nodes are
reached.\footnote{{In most languages, for owner-as-modifier defensive cloning is needed
only when new data is saved inside of an object, while for owner-as-dominator it is needed also when internal data are exposed.}}
Indeed, many programmers prefer to write {unsafe}
 code instead of using defensive cloning for efficiency reasons.
However, this unsafe approach is only possible when programmers have control of the client code, that is, they are not 
working in a library setting.
Indeed many important Java libraries (including the standard Java libraries) today
use defensive cloning to ensure ownership of their internal state.

As mentioned above, our approach is the opposite of the one offered by many ownership approaches, which provide a formal way to express  and prove the ownership invariant that, however, are expected to be guaranteed by defensive cloning. 
We, instead, model an efficient \emph{ownership transfer} through the capsule concept, then, 
duplication of memory, if needed, is performed on the client side\footnote{
Other work in literature supports ownership transfer, see for example~\cite{MullerRudich07, ClarkeWrigstad03}.
In literature it is however applied to uniquess/external uniqueness, thus not {the whole} reachable object graph is transfered.
}.

Moreover, depending on how we expose the owned data, we can closely model
both \emph{owners-as-dominators} (by providing no getter)
and \emph{owners-as-qualifiers} (by providing a \Q@read@ getter). In the example, the method \lstinline{borrowNodes}{} is an example of a $\lent$ getter, a third variant besides the two described on page \pageref{exposer}.  This variant is particularly useful in the case of a field which is owned, indeed, \Q@Graph@ instances can release the mutation control of their nodes without permanently {losing} the aliasing control.

In our approach all properties are deep. On the opposite side, most ownership approaches allows one to distinguish
subparts of the reachable object graph that are referred but not logically owned. This viewpoint has many advantages, for example the Rust language\footnote{\texttt{rust-lang.org}} leverages on ownership to control object deallocation without a garbage collector.
Rust employs a form of uniqueness that can be seen as a restricted ``owners-as-dominators" discipline.  
Rust lifetime parameters behave like additional ownership parameters~\cite{OstlundEtAl08}.

However, in most ownership based approaches 
it is not trivial to encode the concept of full encapsulation, while supporting (open) sub-typing and avoiding defensive cloning.
This depends on how any specific ownership approach entangles subtyping with 
 gaining extra ownership parameters
and extra references to global ownership domains.

\paragraph{Readable notion} Our $\readable$ qualifier is different from \emph{readonly} as used, e.g., in \cite{BirkaErnst04}.
 An object cannot be modified through a readable/readonly reference. However, 
$\readable$ also prevents aliasing.
As discussed in \cite{Boyland06}, readonly semantics can be easily misunderstood by
programmers. Indeed, some wrongly believe it means immutable, whereas the object denoted by a readonly reference can be modified through other references, while others do not realize that readonly data can still be saved in fields, and thus used as a secondary window to observe the change in the object state.
Our proposal addresses both problems, since we explicitly support the $\imm$ qualifier, hence it is more difficult for programmers to confuse the two concepts, and our $\readable$ (readonly + lent) data  cannot be saved in client's fields.

Javari~\cite{TschantzErnst05} also supports the \emph{readonly} type qualifier, and makes a huge design effort to support \emph{assignable} and \emph{mutable} fields, to have fine-grained readonly constraints.  The need of such flexibility is motivated by performance reasons.
In our design philosophy, we do not offer any way of breaking the invariants enforced by the type system. Since our invariants are very strong, we expect compilers to be able to perform optimization, thus recovering most of the efficiency lost to properly use immutable and readable objects.




\section{Methodology}
\label{sec:Methodology}

We begin by formally defining our problem.  
\subsection{Problem Statement}
\label{subsec:ProblemStatement}

\noindent \textbf{Input:}
\begin{enumerate}
    \item $\Sigma$: a finite alphabet. $\Sigma^+$ denotes the set of all non-empty strings over $\Sigma$. In this paper, we focus on strings that are job descriptions expressed as free form text.
    \item $\mathcal{Y}$: a finite set of labels. In this paper, SOC codes are treated as labels.
    \item $\mathcal{D} = \{(x_i, y_i): 1 \leq i \leq n \}$: a labeled dataset of size $n \in \mathbb{N}$, where $x_i \in \Sigma^+$ is a job description, and $y_i \in \mathcal{Y}$ is its corresponding SOC code.
\end{enumerate}

\noindent \textbf{Output:}
A function $f: \Sigma^+ \rightarrow \mathcal{Y}$ which maps a job description $x$ to an SOC code $y = f(x)$ such that $f$ minimizes the expected error with respect to some loss function.

From a pragmatic standpoint, we want such a function $f$ to be available as a web service (i.e., web API) which accepts a request containing description $x$ to produce a response containing the predicted SOC code $y = f(x)$.

\subsection{Approach}
\label{subsec:Approach}

Our approach may be described as a sequence of steps as follows.

\subsubsection{Text Vectorization}
Since a majority of machine learning algorithms assume inputs to be real valued vectors, predictive modeling based on text often requires vectorizing the text, i.e., computing real valued vector representation of text. We consider two different vectorization techniques, which are as follows.
\paragraph{TF-IDF $n$-grams} An $n$-gram ($n \in \mathbb{N}$) is a sequence of $n$ tokens. Given $n_{\mathrm{min}}, n_{\mathrm{max}} \in \mathbb{N}$ ($n_{\mathrm{min}} \leq n_{\mathrm{max}}$), a corpus of text in $\Sigma^+$ can be used to compute the vocabulary of all $n$-grams where $n_{\mathrm{min}} \leq n \leq  n_{\mathrm{max}}$. Subsequently, any string $x \in \Sigma^+$ may be represented as a vector of counts, i.e., term frequencies (TF) of $n$-grams present in $x$. Such a vector representation of a string is typically sparse, i.e., most of its components are zero, since most $n$-grams in the vocabulary are typically absent in it. To offset the effect of highly frequent $n$-grams with little semantic value, the vectors are weighted by inverse document frequencies (IDF), resulting in TF-IDF $n$-gram representations.
While TF-IDF representations have been found to achieve high accuracy in text categorization \cite{DBLP:conf/ecml/Joachims98}, the high dimensionality of the sparse vectors generally entails high computational costs for training predictive models.
\paragraph{Doc2vec} An alternative approach that addresses the issue of dimensionality consists of using neural architectures for vectorizing words \cite{mikolov2013efficient} and strings \cite{DBLP:conf/icml/LeM14}, using contextual similarity to predict semantic similarity. The resulting representations are known as word embeddings and document embeddings, respectively, and the above neural architectures are referred to as word2vec and doc2vec, respectively. Embeddings computed by word2vec and doc2vec are typically of lower dimensionality compared to TF-IDF $n$-gram representations. Therefore, such embeddings are considered dense vector representations. Since job descriptions are strings of arbitrary length, we use doc2vec to compute dense vector representations of such descriptions.

\subsubsection{Predictive Modeling}
For each type of vectorization, we train a set of standard classifiers for predicting SOC code, namely, $k$-nearest neighbors (KNN), Gaussian na\"ive Bayes (GNB), logistic regression (LR), linear support vector machine (LinearSVC), support vector machine with radial basis function (SVC-RBF), decision tree (DT), and random forest (RF).

\subsubsection{Evaluation and Model Selection}
To evaluate the models, we use $n$-fold cross validation. The dataset is first divided into $n$ slices (or folds) of (roughly) equal size. In each round of cross validation, a different slice is held out for testing while the remaining $n - 1$ slices are used for training. Several metrics are recorded in each round. At the end of $n$ rounds of training and testing, these metrics are averaged and reported. These scores help identify models best suited to the problem.

\subsubsection{Deployment}
Once a model has been selected, we deploy it as a web service which can accept a \texttt{POST} request whose body contains a job description in free form text and produce a response containing the predicted SOC code.

The next section presents our empirical evaluation.



\section{Experimental Results And Discussion}
\label{sec:results}
The results presented in this section test the performance of the Autoencoder model. We evaluate our model using the performance metrics: accuracy, precision, recall, and F1 score, defined as follow: 

\vspace{-5mm}
\begin{align*}
    Accuracy &= \frac{TP+TN}{TP+TN+FP+FN}
\end{align*}
\vspace{-3mm}
\begin{align*}
    Precision &= \frac{TP}{TP+FP}
\end{align*}
\vspace{-3mm}
\begin{align*}
    Recall &= \frac{TP}{TP+FN}
\end{align*}
\vspace{-3mm}
\begin{align*}
    F1 ~Score &= 2 \times \frac{Precision \times Recall}{Precision + Recall}
\end{align*}

In our experiments, a \textit{positive} outcome means an abnormal activity was detected, whereas a negative outcome means a normal activity was detected.
True Positive (TP) refers to an abnormal activity that was correctly classified as abnormal. 
True Negative (TN) refers to a normal activity that was correctly classified as normal.
False Positive (FP) refers to a normal activity that was misclassified as abnormal.
and False Negative (FN) refers to an abnormal activity that was misclassified as normal.

The success of our model is based on measuring the reconstruction error that is produced by any given data point. Figure \ref{fig:recon} shows an example of reconstructed data overlaid the original data that was inserted into the model. %To be clear, the values shown in this graph are not measurements of the reconstruction loss that are shown in Figure \ref{fig:thresh}. This figure only shows the normalized temperature measurement from each data point, that is why they are not being represented in degrees.
In this figure, extremely severe dips in temperature denoted by the blue line (representing our original data) can be noticed. The data reconstructed by the model, represented by the red line, does not dip as much as the original data. This is because our model was not able to reconstruct these points accurately due to the fact that they are anomalies. The reconstruction loss (i.e. different between the original and the reconstructed data), where the model recognizes normal or abnormal behavior, is shown in Figure \ref{fig:thresh}. The figure shows a visualization of the mean-squared-error (MSE) generated by the model after it was given each data point within the test data set. The dotted red line denotes the threshold determined as mentioned in Section \ref{sec:ml-model}. Each data point's actual label is represented either by blue color to denote a normal behavior or red color to denote an anomaly and every data point that lies above the threshold was classified as anomalous. This figure illustrates our model's capability to detect the majority of anomalies by measuring the MSE produced by each data point.

Overall, as shown in Figure \ref{fig:aeresults}, our model was able to attain high performance with over $90\%$ in all metrics. The precision is lower than the recall metric which shows that the model produced slightly more false positives than false negatives. In a smart farming environment, a higher rate of false positives would not have a dramatic affect on the productivity of day to day operations and would ensure a higher number of anomalous situations are detected. A rather problematic situation would be if there were more false negatives than positives. A user would much prefer receiving an alert when nothing was wrong than not receiving an alert and enabling potential harm to occur to the crops and hardware. In the future, we hope to further decrease the number of false positives and negatives in order to fine-tune an overall more accurate model. This can be done by using more training samples.


% \begin{table}[!t]
%     \caption{Results}
%     \centering
%     \begin{tabular}{| c | c | c | c |}
%     \hline
%     Accuracy & Precision & Recall & F1\\ [0.5ex] % inserts table %heading
%     \hline
    
%     98.98\% & 90\% & 92.95\% & 91.45\% \\
    
%     \hline
%     \end{tabular}
%     \label{table:results}
% \end{table}

\begin{figure}[t!]
    \centering
    \includegraphics[width=8cm]{figures/aeresults-v2.png}
    \caption{Performance metrics for Autoencoder Model}
    \label{fig:aeresults}
\end{figure}


\section{Conclusion and Future Work}
\label{sec:conclusion}
Our approach has shown that smart farming anomaly detection can be done at an extremely accurate level by using an Autoencoder. Our approach would allow vast scalability by only requiring non-anomalous data for training. Greenhouses provide controlled environments that create consistent conditions for crops and data collection. Environments such as this are a perfect use case for our approach since the performance of an Autoencoder can drastically improve when provided with large amounts of non-anomalous data. Our approach shows that it may not be entirely necessary for machine learning professionals that are working on anomaly detection within smart farming to be highly concerned with developing models that are trained using labeled data that contains both normal and anomalous data. 

In the future, we will explore more anomaly detection models in order to optimize the system's performance. Once the best model has been selected, the architecture could be brought online to be used and tested with the added interactions of Internet connectivity. By bringing the system online we will have the ability to alert users of potential threats or anomalous behavior. These alerts could be coupled with actuators such as fertilization, watering, video monitoring, etc. The introduction of cameras can be ``used to calculate biomass development and fertilization status of crops" \cite{Walter6148}. They can also be used to allow the system-user to monitor their property from afar. We plan to introduce photo and video monitoring as one of our next steps to improve security and broaden our scope.

\section{Acknowledgements}
\label{sec:ack}
We thank TTU Shipley Farms for allowing to use greenhouse, and setup smart farm testbed. Dr. Brian Leckie and his group were instrumental in our system and early stages of data collection. We are thankful to Ms. Deepti Gupta to provide helpful guidance on dealing with time-series, correlated data and gave input on our model selection. This research is partially supported by the NSF Grant 2025682 at TTU.
% \begin{abstract}
% This document is a model and instructions for \LaTeX.
% This and the IEEEtran.cls file define the components of your paper [title, text, heads, etc.]. *CRITICAL: Do Not Use Symbols, Special Characters, Footnotes, 
% or Math in Paper Title or Abstract.
% \end{abstract}

% \begin{IEEEkeywords}
% component, formatting, style, styling, insert
% \end{IEEEkeywords}

% \section{Introduction}
% This document is a model and instructions for \LaTeX.
% Please observe the conference page limits. 

% \section{Ease of Use}

% \subsection{Maintaining the Integrity of the Specifications}

% The IEEEtran class file is used to format your paper and style the text. All margins, 
% column widths, line spaces, and text fonts are prescribed; please do not 
% alter them. You may note peculiarities. For example, the head margin
% measures proportionately more than is customary. This measurement 
% and others are deliberate, using specifications that anticipate your paper 
% as one part of the entire proceedings, and not as an independent document. 
% Please do not revise any of the current designations.

% \section{Prepare Your Paper Before Styling}
% Before you begin to format your paper, first write and save the content as a 
% separate text file. Complete all content and organizational editing before 
% formatting. Please note sections \ref{AA}--\ref{SCM} below for more information on 
% proofreading, spelling and grammar.

% Keep your text and graphic files separate until after the text has been 
% formatted and styled. Do not number text heads---{\LaTeX} will do that 
% for you.

% \subsection{Abbreviations and Acronyms}\label{AA}
% Define abbreviations and acronyms the first time they are used in the text, 
% even after they have been defined in the abstract. Abbreviations such as 
% IEEE, SI, MKS, CGS, ac, dc, and rms do not have to be defined. Do not use 
% abbreviations in the title or heads unless they are unavoidable.

% \subsection{Units}
% \begin{itemize}
% \item Use either SI (MKS) or CGS as primary units. (SI units are encouraged.) English units may be used as secondary units (in parentheses). An exception would be the use of English units as identifiers in trade, such as ``3.5-inch disk drive''.
% \item Avoid combining SI and CGS units, such as current in amperes and magnetic field in oersteds. This often leads to confusion because equations do not balance dimensionally. If you must use mixed units, clearly state the units for each quantity that you use in an equation.
% \item Do not mix complete spellings and abbreviations of units: ``Wb/m\textsuperscript{2}'' or ``webers per square meter'', not ``webers/m\textsuperscript{2}''. Spell out units when they appear in text: ``. . . a few henries'', not ``. . . a few H''.
% \item Use a zero before decimal points: ``0.25'', not ``.25''. Use ``cm\textsuperscript{3}'', not ``cc''.)
% \end{itemize}

% \subsection{Equations}
% Number equations consecutively. To make your 
% equations more compact, you may use the solidus (~/~), the exp function, or 
% appropriate exponents. Italicize Roman symbols for quantities and variables, 
% but not Greek symbols. Use a long dash rather than a hyphen for a minus 
% sign. Punctuate equations with commas or periods when they are part of a 
% sentence, as in:
% \begin{equation}
% a+b=\gamma\label{eq}
% \end{equation}

% Be sure that the 
% symbols in your equation have been defined before or immediately following 
% the equation. Use ``\eqref{eq}'', not ``Eq.~\eqref{eq}'' or ``equation \eqref{eq}'', except at 
% the beginning of a sentence: ``Equation \eqref{eq} is . . .''

% \subsection{\LaTeX-Specific Advice}

% Please use ``soft'' (e.g., \verb|\eqref{Eq}|) cross references instead
% of ``hard'' references (e.g., \verb|(1)|). That will make it possible
% to combine sections, add equations, or change the order of figures or
% citations without having to go through the file line by line.

% Please don't use the \verb|{eqnarray}| equation environment. Use
% \verb|{align}| or \verb|{IEEEeqnarray}| instead. The \verb|{eqnarray}|
% environment leaves unsightly spaces around relation symbols.

% Please note that the \verb|{subequations}| environment in {\LaTeX}
% will increment the main equation counter even when there are no
% equation numbers displayed. If you forget that, you might write an
% article in which the equation numbers skip from (17) to (20), causing
% the copy editors to wonder if you've discovered a new method of
% counting.

% {\BibTeX} does not work by magic. It doesn't get the bibliographic
% data from thin air but from .bib files. If you use {\BibTeX} to produce a
% bibliography you must send the .bib files. 

% {\LaTeX} can't read your mind. If you assign the same label to a
% subsubsection and a table, you might find that Table I has been cross
% referenced as Table IV-B3. 

% {\LaTeX} does not have precognitive abilities. If you put a
% \verb|\label| command before the command that updates the counter it's
% supposed to be using, the label will pick up the last counter to be
% cross referenced instead. In particular, a \verb|\label| command
% should not go before the caption of a figure or a table.

% Do not use \verb|\nonumber| inside the \verb|{array}| environment. It
% will not stop equation numbers inside \verb|{array}| (there won't be
% any anyway) and it might stop a wanted equation number in the
% surrounding equation.

% \subsection{Some Common Mistakes}\label{SCM}
% \begin{itemize}
% \item The word ``data'' is plural, not singular.
% \item The subscript for the permeability of vacuum $\mu_{0}$, and other common scientific constants, is zero with subscript formatting, not a lowercase letter ``o''.
% \item In American English, commas, semicolons, periods, question and exclamation marks are located within quotation marks only when a complete thought or name is cited, such as a title or full quotation. When quotation marks are used, instead of a bold or italic typeface, to highlight a word or phrase, punctuation should appear outside of the quotation marks. A parenthetical phrase or statement at the end of a sentence is punctuated outside of the closing parenthesis (like this). (A parenthetical sentence is punctuated within the parentheses.)
% \item A graph within a graph is an ``inset'', not an ``insert''. The word alternatively is preferred to the word ``alternately'' (unless you really mean something that alternates).
% \item Do not use the word ``essentially'' to mean ``approximately'' or ``effectively''.
% \item In your paper title, if the words ``that uses'' can accurately replace the word ``using'', capitalize the ``u''; if not, keep using lower-cased.
% \item Be aware of the different meanings of the homophones ``affect'' and ``effect'', ``complement'' and ``compliment'', ``discreet'' and ``discrete'', ``principal'' and ``principle''.
% \item Do not confuse ``imply'' and ``infer''.
% \item The prefix ``non'' is not a word; it should be joined to the word it modifies, usually without a hyphen.
% \item There is no period after the ``et'' in the Latin abbreviation ``et al.''.
% \item The abbreviation ``i.e.'' means ``that is'', and the abbreviation ``e.g.'' means ``for example''.
% \end{itemize}
% An excellent style manual for science writers is \cite{b7}.

% \subsection{Authors and Affiliations}
% \textbf{The class file is designed for, but not limited to, six authors.} A 
% minimum of one author is required for all conference articles. Author names 
% should be listed starting from left to right and then moving down to the 
% next line. This is the author sequence that will be used in future citations 
% and by indexing services. Names should not be listed in columns nor group by 
% affiliation. Please keep your affiliations as succinct as possible (for 
% example, do not differentiate among departments of the same organization).

% \subsection{Identify the Headings}
% Headings, or heads, are organizational devices that guide the reader through 
% your paper. There are two types: component heads and text heads.

% Component heads identify the different components of your paper and are not 
% topically subordinate to each other. Examples include Acknowledgments and 
% References and, for these, the correct style to use is ``Heading 5''. Use 
% ``figure caption'' for your Figure captions, and ``table head'' for your 
% table title. Run-in heads, such as ``Abstract'', will require you to apply a 
% style (in this case, italic) in addition to the style provided by the drop 
% down menu to differentiate the head from the text.

% Text heads organize the topics on a relational, hierarchical basis. For 
% example, the paper title is the primary text head because all subsequent 
% material relates and elaborates on this one topic. If there are two or more 
% sub-topics, the next level head (uppercase Roman numerals) should be used 
% and, conversely, if there are not at least two sub-topics, then no subheads 
% should be introduced.

% \subsection{Figures and Tables}
% \paragraph{Positioning Figures and Tables} Place figures and tables at the top and 
% bottom of columns. Avoid placing them in the middle of columns. Large 
% figures and tables may span across both columns. Figure captions should be 
% below the figures; table heads should appear above the tables. Insert 
% figures and tables after they are cited in the text. Use the abbreviation 
% ``Fig.~\ref{fig}'', even at the beginning of a sentence.

% \begin{table}[htbp]
% \caption{Table Type Styles}
% \begin{center}
% \begin{tabular}{|c|c|c|c|}
% \hline
% \textbf{Table}&\multicolumn{3}{|c|}{\textbf{Table Column Head}} \\
% \cline{2-4} 
% \textbf{Head} & \textbf{\textit{Table column subhead}}& \textbf{\textit{Subhead}}& \textbf{\textit{Subhead}} \\
% \hline
% copy& More table copy$^{\mathrm{a}}$& &  \\
% \hline
% \multicolumn{4}{l}{$^{\mathrm{a}}$Sample of a Table footnote.}
% \end{tabular}
% \label{tab1}
% \end{center}
% \end{table}

% \begin{figure}[htbp]
% %\centerline{\includegraphics{fig1.png}}
% \caption{Example of a figure caption.}
% \label{fig}
% \end{figure}

% Figure Labels: Use 8 point Times New Roman for Figure labels. Use words 
% rather than symbols or abbreviations when writing Figure axis labels to 
% avoid confusing the reader. As an example, write the quantity 
% ``Magnetization'', or ``Magnetization, M'', not just ``M''. If including 
% units in the label, present them within parentheses. Do not label axes only 
% with units. In the example, write ``Magnetization (A/m)'' or ``Magnetization 
% \{A[m(1)]\}'', not just ``A/m''. Do not label axes with a ratio of 
% quantities and units. For example, write ``Temperature (K)'', not 
% ``Temperature/K''.

% \section*{Acknowledgment}

% The preferred spelling of the word ``acknowledgment'' in America is without 
% an ``e'' after the ``g''. Avoid the stilted expression ``one of us (R. B. 
% G.) thanks $\ldots$''. Instead, try ``R. B. G. thanks$\ldots$''. Put sponsor 
% acknowledgments in the unnumbered footnote on the first page.

% \section*{References}

% Please number citations consecutively within brackets \cite{IEEEhowto:IEEEtranpage}. The 
% sentence punctuation follows the bracket \cite{b2}. Refer simply to the reference 
% number, as in \cite{b3}---do not use ``Ref. \cite{b3}'' or ``reference \cite{b3}'' except at 
% the beginning of a sentence: ``Reference \cite{b3} was the first $\ldots$''

% Number footnotes separately in superscripts. Place the actual footnote at 
% the bottom of the column in which it was cited. Do not put footnotes in the 
% abstract or reference list. Use letters for table footnotes.

% Unless there are six authors or more give all authors' names; do not use 
% ``et al.''. Papers that have not been published, even if they have been 
% submitted for publication, should be cited as ``unpublished'' \cite{b4}. Papers 
% that have been accepted for publication should be cited as ``in press'' \cite{b5}. 
% Capitalize only the first word in a paper title, except for proper nouns and 
% element symbols.

% For papers published in translation journals, please give the English 
% citation first, followed by the original foreign-language citation \cite{b6}.

\bibliographystyle{./bibliography/IEEEtran}
\bibliography{./bibliography.bib}

\vspace{12pt}
%\color{red}
%IEEE conference templates contain guidance text for composing and formatting conference papers. Please ensure that all template text is removed from your conference paper prior to submission to the conference. Failure to remove the template text from your paper may result in your paper not being published.

\end{document}
