\begin{abstract}
The inclusion of Internet of Things (IoT) devices is growing rapidly in all application domains. Smart Farming supports devices connected, and with the support of Internet, cloud or edge computing infrastructure provide remote control of watering and fertilization, real time monitoring of farm conditions, and provide solutions to more sustainable practices. This could involve using irrigation systems only when the detected soil moisture level is low or stop when the plant reaches a sufficient level of soil moisture content. These improvements to efficiency and ease of use come with added risks to security and privacy. 
%This connectivity between devices and a critical system, such as a farm, allows an adversary seeking to cause harm a way to do so. 
Cyber attacks in large coordinated manner can disrupt economy of agriculture-dependent nations. %could  halting food production or destroying property. 
To the sensors in the system, an attack may appear as anomalous behaviour. In this context, there are possibilities of anomalies generated due to faulty hardware, issues in network connectivity (if present), or simply abrupt changes to the environment due to weather, human accident, or other unforeseen circumstances.
%Anomalous data can also be caused by faults in the hardware, issues in network connectivity (if present), or simply abrupt changes to the environment due to weather, human accident, or other unforeseen circumstances. 
To make such systems more secure, it is imperative to detect such data discrepancies, and trigger appropriate mitigation mechanisms. 

%It is important to be able to identify these data discrepancies and determine if they are harmful or not.  If harmful, it is imperative to detect quickly and accurately so that the farm user can be alerted. 

%To accomplish this goal, this work aims to use anomaly detection as a solution to this problem. 
In this paper, we propose an anomaly detection model for Smart Farming using an unsupervised Autoencoder machine learning model. We chose to use an Autoencoder because it encodes and decodes data and attempts to ignore outliers. When it encounters anomalous data the result will be a high reconstruction loss value, signaling that this data was not like the rest. Our model was trained and tested on data collected from our designed greenhouse test-bed. Proposed Autoencoder model based anomaly detection achieved 98.98\% and took 262 seconds to train and has a detection time of .0585 seconds.

%The accuracy for this model was found to be 98.98\% and took 262 seconds to train and has a detection time of .0585 seconds. %Future steps involve the addition of more sensors, cameras to record pictures and videos of the environment, and comparison of multiple machine learning methods to optimize performance parameters.

\emph{Keywords: Smart Farming, Anomaly Detection, Autoencoder, Time-Series Data, Grove Sensors, Unsupervised Learning}
\end{abstract}