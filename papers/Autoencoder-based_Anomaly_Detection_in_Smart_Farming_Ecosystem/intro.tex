\section{Introduction}
Smart farming is the implementation of IoT technology in a traditional farm environment. Farms provide food, jobs, and commerce across the globe. The addition of technology to this system has the potential to reduce soil depletion by monitoring crop growth patterns and reduce the amount of fertilizer and water used by optimizing schedules for each to match weather patterns and specific crop needs. Smart farms also have the potential to improve crop yields, as well as increase levels of sustainability \cite{iotfood}. Figure \ref{fig:smart_farming_interaction}, shows an end to end interaction among various entities involved in the smart farming ecosystem. The result of a successful smart farm would be decreased waste and increased output, all while making the process easier for the farmer. However, farming is a particularly critical sector due to the world's dependency on its physical output. A disruption in food supply would have negative consequences on even a small farm and the people who depend on its output. These consequences and dependence become greater the larger the farm is. Disruptions could come in the way of device failure, natural disaster, or attack on the system.  

There are different IoT devices that can be used within a smart farming system including, but not limited to, sensors for soil moisture level, temperature, humidity, etc., actuators to control light level, air circulation, watering, fertilizer, and many more. Smart farming devices are often exposed to harsh conditions such as extreme heat and light, as well as condensation build up or even flooding. Although the price of IoT devices is low, making them affordable for any level of farmer to use, they have inherent flaws and limitations. Mostly have minimal or no security protocols and overall low cost of hardware parts. This means that these devices are easy to replace, but also sacrifice consistency in readings, and as mentioned before, little or no privacy and security mechanisms. IoT devices are highly susceptible to failure and manipulation by attackers. An example of an attack could be for an adversary to target a smart farming infrastructure to disrupt food production simply to cause harm or to gain financially by placing holding the systems hostage and demanding a ransom before relinquishing control. Simply by deploying IoT devices for smart farming purposes would inherit different types of security risks that the farmers and community previously would not have to worry about. Further, with the exponential rise in the number of IoT devices in the world has introduced new types/variations or degree of risks in security and privacy. These devices often have unsatisfactory security practices such as weak/guessable passwords, insecure network services, etc. 

A list of 10 of the biggest areas of insecurity in IoT devices can be found in the OWASP IoT Top 10 document \cite{owasp}. To make IoT devices' security flaws even worse, these devices are also deployed on a massive scale, which means that the vulnerability of a single sensor could be exacerbated by using 10s or 100s of the same sensor on a large farm \cite{tawalbeh2020iot}. IoT devices often have Bluetooth, WiFi, or other network connectivity capabilities. This is what makes these devices captivating and innovative. They allow the user to monitor farm or greenhouse conditions remotely and, in advanced scenarios, control actuation to ensure that optimal conditions are upheld. However, network connectivity offers hackers a pathway to perform attacks and directly opens a farm up to the dangers of the Internet. Apparently the aspects of smart farming that make it beneficial, internet connectivity and use of cheap IoT devices are also what put the system at risk. This is because IoT devices make it possible for cyber attacks to move past cyberspace and into the physical world \cite{fu2021hawatcher}. In addition, in connected systems like smart farming where data reading can result in actuation of other devices, it is critical to identify anomalous behaviour timely. 

\begin{figure}[t!]
    \centering
    \includegraphics[width=\columnwidth]{figures/SF-interaction-v2.png}
    \caption{Smart Farming Conceptual Architecture \cite{gupta2020security}.}
    \label{fig:smart_farming_interaction}
\end{figure}

In this paper, we focus on detecting anomalous behaviour of different sensors deployed in a smart farming ecosystem. Determining anomaly in this critical and time sensitive domain is imperative to curtail and limit the cyber risk and provide an opportunity for activating/deploying relevant security mitigation solutions. In addition, some sensor readings result in action from other devices, for example, a low moisture reading will result in actuate a water sprinkler. Therefore, it is critical to timely identify even a slightest anomalous behaviour (which could be because of a faulty sensor, or cyber attack) to prevent large scale damage to ecosystem. Our goal is to implement machine learning to accurately and quickly detect anomalies that could be the result of device failure, accidental interference, or by attacks. We envision that by exploring anomalous behavior and mitigation techniques, we can make smart farming safer to use and expand the work being done in this domain. The entire overview of our approach is as follows:
The first step was to select sensors, set up a test environment, and collect data. The data we collected from sensors deployed in a greenhouse were independent from any network connectivity. This was done to focus on collecting data that represents normal, healthy conditions within the greenhouse. Although we do not study the effects of potential attacks or anomalies due to network connectivity, our model would still be able to detect anomalies because we trained solely on non-anomalous data. At this point, deployment in commercial farm settings is beyond the scope of our research, and we are only focused on developing a model which can detect anomalies. Therefore, we leave the possibility of connectivity up to future work. After data collection, we processed the data to prepare it for ingestion into a machine learning model. We chose to use Autoencoder\footnote{https://blog.keras.io/building-autoencoders-in-keras.html}, because it uses a neural network to encode data into a low dimension and then decode it, attempting to minimize reconstruction loss. It's able to perform anomaly detection by checking the magnitude of the reconstruction loss \cite{adwithauto}. In other words, the Autoencoder's inability to reconstruct particular data implies that the data is anomalous. Using this method we were able to achieve high accuracy and train and predict quickly as elaborated in the later sections. Our success using this model encourages us to test other deep learning models in the future and compare their metrics.

The main contributions of this paper as as follows:
\begin{itemize}
    \item We developed a scalable smart farming environment and collected data from different sensors. We also highlight some challenges faced and solutions we designed.
    \item We designed and injected anomalous scenarios along with some natural anomalies encountered in our smart farming environment.  
    \item We trained and tested an Autoencoder model which is an unsupervised artificial neural network.
    \item We demonstrated how an Autoencoder model can perform well with promising results.  
\end{itemize}
%(1) created scalable smart farming environment, (2) collected data from our sensors to train and test model, (3) ran Autoencoder algorithm and produced promising results.

The remainder of the paper is organized as follows: Section \ref{sec:related}  discusses the related works in anomaly detection and security issues in smart farming. We introduced the deployed smart farming architecture in Section \ref{sec:arch} elaborating on the different sensors, hardware and software used to conduct our experiments. Section \ref{sec:data} highlights the entire data collection and processing stages along with the different anomalies injected into the system. Description of the Autoencoder machine learning model and its architecture is done in Section \ref{sec:ml-model}. The results generated by the model are discussed in Section \ref{sec:results} followed by conclusion and future work in Section \ref{sec:conclusion}. 
