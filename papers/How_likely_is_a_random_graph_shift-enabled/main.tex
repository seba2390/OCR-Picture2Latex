% !TeX document-id = {b34d089e-9655-494d-bb55-e92e5e170dae}
% !TEX TS-program = pdflatex
% !BIB TS-program = bibtex
%
\documentclass[journal]{IEEEtran}
%
% If IEEEtran.cls has not been installed into the LaTeX system files,
% manually specify the path to it like:
% \documentclass[journal]{../sty/IEEEtran}


\usepackage[pdftex]{graphicx}
%\usepackage{epstopdf}
\usepackage{wrapfig}
%\usepackage{caption}
\usepackage{subfigure}
\usepackage[symbol]{footmisc}
\usepackage{bm}
\usepackage{bbm}
\usepackage{amsmath}
\usepackage{cases}
\usepackage{color}
\usepackage{outlines}
\usepackage{url}
%\usepackage{mathrsfs}
%\def\correspondingauthor{\footnote{Corresponding author.}}

% Some very useful LaTeX packages include:
% (uncomment the ones you want to load)


% *** MISC UTILITY PACKAGES ***
%
%\usepackage{ifpdf}
% Heiko Oberdiek's ifpdf.sty is very useful if you need conditional
% compilation based on whether the output is pdf or dvi.
% usage:
% \ifpdf
%   % pdf code
% \else
%   % dvi code
% \fi
% The latest version of ifpdf.sty can be obtained from:
% http://www.ctan.org/pkg/ifpdf
% Also, note that IEEEtran.cls V1.7 and later provides a builtin
% \ifCLASSINFOpdf conditional that works the same way.
% When switching from latex to pdflatex and vice-versa, the compiler may
% have to be run twice to clear warning/error messages.




\usepackage{amssymb}


% *** CITATION PACKAGES ***
%
\usepackage{cite}
% cite.sty was written by Donald Arseneau
% V1.6 and later of IEEEtran pre-defines the format of the cite.sty package
% \cite{} output to follow that of the IEEE. Loading the cite package will
% result in citation numbers being automatically sorted and properly
% "compressed/ranged". e.g., [1], [9], [2], [7], [5], [6] without using
% cite.sty will become [1], [2], [5]--[7], [9] using cite.sty. cite.sty's
% \cite will automatically add leading space, if needed. Use cite.sty's
% noadjust option (cite.sty V3.8 and later) if you want to turn this off
% such as if a citation ever needs to be enclosed in parenthesis.
% cite.sty is already installed on most LaTeX systems. Be sure and use
% version 5.0 (2009-03-20) and later if using hyperref.sty.
% The latest version can be obtained at:
% http://www.ctan.org/pkg/cite
% The documentation is contained in the cite.sty file itself.




% *** GRAPHICS RELATED PACKAGES ***
%
\ifCLASSINFOpdf
  % \usepackage[pdftex]{graphicx}
  % declare the path(s) where your graphic files are
  % \graphicspath{{../pdf/}{../jpeg/}}
  % and their extensions so you won't have to specify these with
  % every instance of \includegraphics
  % \DeclareGraphicsExtensions{.pdf,.jpeg,.png}
\else
  % or other class option (dvipsone, dvipdf, if not using dvips). graphicx
  % will default to the driver specified in the system graphics.cfg if no
  % driver is specified.
  % \usepackage[dvips]{graphicx}
  % declare the path(s) where your graphic files are
  % \graphicspath{{../eps/}}
  % and their extensions so you won't have to specify these with
  % every instance of \includegraphics
  % \DeclareGraphicsExtensions{.eps}
\fi
% graphicx was written by David Carlisle and Sebastian Rahtz. It is
% required if you want graphics, photos, etc. graphicx.sty is already
% installed on most LaTeX systems. The latest version and documentation
% can be obtained at: 
% http://www.ctan.org/pkg/graphicx
% Another good source of documentation is "Using Imported Graphics in
% LaTeX2e" by Keith Reckdahl which can be found at:
% http://www.ctan.org/pkg/epslatex
%
% latex, and pdflatex in dvi mode, support graphics in encapsulated
% postscript (.eps) format. pdflatex in pdf mode supports graphics
% in .pdf, .jpeg, .png and .mps (metapost) formats. Users should ensure
% that all non-photo figures use a vector format (.eps, .pdf, .mps) and
% not a bitmapped formats (.jpeg, .png). The IEEE frowns on bitmapped formats
% which can result in "jaggedy"/blurry rendering of lines and letters as
% well as large increases in file sizes.
%
% You can find documentation about the pdfTeX application at:
% http://www.tug.org/applications/pdftex





% *** MATH PACKAGES ***
%
%\usepackage{amsmath}
% A popular package from the American Mathematical Society that provides
% many useful and powerful commands for dealing with mathematics.
%
% Note that the amsmath package sets \interdisplaylinepenalty to 10000
% thus preventing page breaks from occurring within multiline equations. Use:
%\interdisplaylinepenalty=2500
% after loading amsmath to restore such page breaks as IEEEtran.cls normally
% does. amsmath.sty is already installed on most LaTeX systems. The latest
% version and documentation can be obtained at:
% http://www.ctan.org/pkg/amsmath





% *** SPECIALIZED LIST PACKAGES ***
%
%\usepackage{algorithmic}
% algorithmic.sty was written by Peter Williams and Rogerio Brito.
% This package provides an algorithmic environment fo describing algorithms.
% You can use the algorithmic environment in-text or within a figure
% environment to provide for a floating algorithm. Do NOT use the algorithm
% floating environment provided by algorithm.sty (by the same authors) or
% algorithm2e.sty (by Christophe Fiorio) as the IEEE does not use dedicated
% algorithm float types and packages that provide these will not provide
% correct IEEE style captions. The latest version and documentation of
% algorithmic.sty can be obtained at:
% http://www.ctan.org/pkg/algorithms
% Also of interest may be the (relatively newer and more customizable)
% algorithmicx.sty package by Szasz Janos:
% http://www.ctan.org/pkg/algorithmicx




% *** ALIGNMENT PACKAGES ***
%
%\usepackage{array}
% Frank Mittelbach's and David Carlisle's array.sty patches and improves
% the standard LaTeX2e array and tabular environments to provide better
% appearance and additional user controls. As the default LaTeX2e table
% generation code is lacking to the point of almost being broken with
% respect to the quality of the end results, all users are strongly
% advised to use an enhanced (at the very least that provided by array.sty)
% set of table tools. array.sty is already installed on most systems. The
% latest version and documentation can be obtained at:
% http://www.ctan.org/pkg/array


% IEEEtran contains the IEEEeqnarray family of commands that can be used to
% generate multiline equations as well as matrices, tables, etc., of high
% quality.




% *** SUBFIGURE PACKAGES ***
%\ifCLASSOPTIONcompsoc
%  \usepackage[caption=false,font=normalsize,labelfont=sf,textfont=sf]{subfig}
%\else
%  \usepackage[caption=false,font=footnotesize]{subfig}
%\fi
% subfig.sty, written by Steven Douglas Cochran, is the modern replacement
% for subfigure.sty, the latter of which is no longer maintained and is
% incompatible with some LaTeX packages including fixltx2e. However,
% subfig.sty requires and automatically loads Axel Sommerfeldt's caption.sty
% which will override IEEEtran.cls' handling of captions and this will result
% in non-IEEE style figure/table captions. To prevent this problem, be sure
% and invoke subfig.sty's "caption=false" package option (available since
% subfig.sty version 1.3, 2005/06/28) as this is will preserve IEEEtran.cls
% handling of captions.
% Note that the Computer Society format requires a larger sans serif font
% than the serif footnote size font used in traditional IEEE formatting
% and thus the need to invoke different subfig.sty package options depending
% on whether compsoc mode has been enabled.
%
% The latest version and documentation of subfig.sty can be obtained at:
% http://www.ctan.org/pkg/subfig




% *** FLOAT PACKAGES ***
%
%\usepackage{fixltx2e}
% fixltx2e, the successor to the earlier fix2col.sty, was written by
% Frank Mittelbach and David Carlisle. This package corrects a few problems
% in the LaTeX2e kernel, the most notable of which is that in current
% LaTeX2e releases, the ordering of single and double column floats is not
% guaranteed to be preserved. Thus, an unpatched LaTeX2e can allow a
% single column figure to be placed prior to an earlier double column
% figure.
% Be aware that LaTeX2e kernels dated 2015 and later have fixltx2e.sty's
% corrections already built into the system in which case a warning will
% be issued if an attempt is made to load fixltx2e.sty as it is no longer
% needed.
% The latest version and documentation can be found at:
% http://www.ctan.org/pkg/fixltx2e


%\usepackage{stfloats}
% stfloats.sty was written by Sigitas Tolusis. This package gives LaTeX2e
% the ability to do double column floats at the bottom of the page as well
% as the top. (e.g., "\begin{figure*}[!b]" is not normally possible in
% LaTeX2e). It also provides a command:
%\fnbelowfloat
% to enable the placement of footnotes below bottom floats (the standard
% LaTeX2e kernel puts them above bottom floats). This is an invasive package
% which rewrites many portions of the LaTeX2e float routines. It may not work
% with other packages that modify the LaTeX2e float routines. The latest
% version and documentation can be obtained at:
% http://www.ctan.org/pkg/stfloats
% Do not use the stfloats baselinefloat ability as the IEEE does not allow
% \baselineskip to stretch. Authors submitting work to the IEEE should note
% that the IEEE rarely uses double column equations and that authors should try
% to avoid such use. Do not be tempted to use the cuted.sty or midfloat.sty
% packages (also by Sigitas Tolusis) as the IEEE does not format its papers in
% such ways.
% Do not attempt to use stfloats with fixltx2e as they are incompatible.
% Instead, use Morten Hogholm'a dblfloatfix which combines the features
% of both fixltx2e and stfloats:
%
% \usepackage{dblfloatfix}
% The latest version can be found at:
% http://www.ctan.org/pkg/dblfloatfix




%\ifCLASSOPTIONcaptionsoff
%  \usepackage[nomarkers]{endfloat}
% \let\MYoriglatexcaption\caption
% \renewcommand{\caption}[2][\relax]{\MYoriglatexcaption[#2]{#2}}
%\fi
% endfloat.sty was written by James Darrell McCauley, Jeff Goldberg and 
% Axel Sommerfeldt. This package may be useful when used in conjunction with 
% IEEEtran.cls'  captionsoff option. Some IEEE journals/societies require that
% submissions have lists of figures/tables at the end of the paper and that
% figures/tables without any captions are placed on a page by themselves at
% the end of the document. If needed, the draftcls IEEEtran class option or
% \CLASSINPUTbaselinestretch interface can be used to increase the line
% spacing as well. Be sure and use the nomarkers option of endfloat to
% prevent endfloat from "marking" where the figures would have been placed
% in the text. The two hack lines of code above are a slight modification of
% that suggested by in the endfloat docs (section 8.4.1) to ensure that
% the full captions always appear in the list of figures/tables - even if
% the user used the short optional argument of \caption[]{}.
% IEEE papers do not typically make use of \caption[]'s optional argument,
% so this should not be an issue. A similar trick can be used to disable
% captions of packages such as subfig.sty that lack options to turn off
% the subcaptions:
% For subfig.sty:
% \let\MYorigsubfloat\subfloat
% \renewcommand{\subfloat}[2][\relax]{\MYorigsubfloat[]{#2}}
% However, the above trick will not work if both optional arguments of
% the \subfloat command are used. Furthermore, there needs to be a
% description of each subfigure *somewhere* and endfloat does not add
% subfigure captions to its list of figures. Thus, the best approach is to
% avoid the use of subfigure captions (many IEEE journals avoid them anyway)
% and instead reference/explain all the subfigures within the main caption.
% The latest version of endfloat.sty and its documentation can obtained at:
% http://www.ctan.org/pkg/endfloat
%
% The IEEEtran \ifCLASSOPTIONcaptionsoff conditional can also be used
% later in the document, say, to conditionally put the References on a 
% page by themselves.




% *** PDF, URL AND HYPERLINK PACKAGES ***
%
%\usepackage{url}
% url.sty was written by Donald Arseneau. It provides better support for
% handling and breaking URLs. url.sty is already installed on most LaTeX
% systems. The latest version and documentation can be obtained at:
% http://www.ctan.org/pkg/url
% Basically, \url{my_url_here}.




% *** Do not adjust lengths that control margins, column widths, etc. ***
% *** Do not use packages that alter fonts (such as pslatex).         ***
% There should be no need to do such things with IEEEtran.cls V1.6 and later.
% (Unless specifically asked to do so by the journal or conference you plan
% to submit to, of course. )


% correct bad hyphenation here
\hyphenation{op-tical net-works semi-conduc-tor}

\usepackage{amsmath,amssymb,amsthm}

\begin{document}

%
% paper title
% Titles are generally capitalized except for words such as a, an, and, as,
% at, but, by, for, in, nor, of, on, or, the, to and up, which are usually
% not capitalized unless they are the first or last word of the title.
% Linebreaks \\ can be used within to get better formatting as desired.
% Do not put math or special symbols in the title.
\title{How likely is a random graph shift-enabled?}
%
%
% author names and IEEE memberships
% note positions of commas and nonbreaking spaces ( ~ ) LaTeX will not break
% a structure at a ~ so this keeps an author's name from being broken across
% two lines.
% use \thanks{} to gain access to the first footnote area
% a separate \thanks must be used for each paragraph as LaTeX2e's \thanks
% was not built to handle multiple paragraphs
%

\author{Liyan~Chen,
Samuel~Cheng$^1$,~\IEEEmembership{Senior~Member,~IEEE,}
                Vladimir~Stankovic,~\IEEEmembership{Senior~Member,~IEEE,}
        and~Lina~Stankovic,~\IEEEmembership{Senior~Member,~IEEE}
 %Lina~Stankovic,~\IEEEmembership{Senior~Member,~IEEE,}
% and        ~Vladimir~Stankovic,~\IEEEmembership{Senior~Member,~IEEE} 
% <-this % stops a space
\thanks{L. Chen is with 
Key Laboratory of Oceanographic Big Data Mining \& Application of Zhejiang Province, Zhejiang Ocean University, Zhoushan, Zhejiang 316022, China
and the Department
of Computer Science and Technology, Tongji University, Shanghai,
 201804 China  (e-mail: chenliyan@tongji.edu.cn).}% <-this % stops %a space
\thanks{S. Cheng is with
% the Department of Computer Science and Technology, Tongji University, Shanghai, 201804 China and 
the School of Electrical and Computer Engineering, University of Oklahoma, OK 74105, USA (email: samuel.cheng@ou.edu).}
% \thanks{L. Stankovic, and V. Stankovic are with Department of Electronic and Electrical Engineering,
%          University of Strathclye, Glasgow, G1 1XW U.K.
%         (e-mail:\~lina.stankovic,~vladimir.stankovic\}@strath.ac.uk).}% <-this % stops a space
\thanks{V. Stankovic and L. Stankovic are with Department of Electronic and Electrical Engineering,
         University of Strathclyde, Glasgow, G1 1XW U.K.
        (e-mail:\{vladimir.stankovic,~lina.stankovic\}@strath.ac.uk).}% <-this % stops a space
\thanks{$^1$ Corresponding author.}
\thanks{%Manuscript received April 19, 2005; revised August 26, 2015.
}
}

% note the % following the last \IEEEmembership and also \thanks - 
% these prevent an unwanted space from occurring between the last author name
% and the end of the author line. i.e., if you had this:
% 
% \author{....lastname \thanks{...} \thanks{...} }
%                     ^------------^------------^----Do not want these spaces!
%
% a space would be appended to the last name and could cause every name on that
% line to be shifted left slightly. This is one of those "LaTeX things". For
% instance, "\textbf{A} \textbf{B}" will typeset as "A B" not "AB". To get
% "AB" then you have to do: "\textbf{A}\textbf{B}"
% \thanks is no different in this regard, so shield the last } of each \thanks
% that ends a line with a % and do not let a space in before the next \thanks.
% Spaces after \IEEEmembership other than the last one are OK (and needed) as
% you are supposed to have spaces between the names. For what it is worth,
% this is a minor point as most people would not even notice if the said evil
% space somehow managed to creep in.



% The paper headers
\markboth{}%
%\markboth{gJournal of \LaTeX\ Class Files,~Vol.~14, No.~8, August~2015}%
{Shell \MakeLowercase{\textit{et al.}}: Bare Demo of IEEEtran.cls for IEEE Journals}
% The only time the second header will appear is for the odd numbered pages
% after the title page when using the twoside option.
% 
% *** Note that you probably will NOT want to include the author's ***
% *** name in the headers of peer review papers.                   ***
% You can use \ifCLASSOPTIONpeerreview for conditional compilation here if
% you desire.




% If you want to put a publisher's ID mark on the page you can do it like
% this:
%\IEEEpubid{0000--0000/00\$00.00~\copyright~2015 IEEE}
% Remember, if you use this you must call \IEEEpubidadjcol in the second
% column for its text to clear the IEEEpubid mark.



% use for special paper notices
%\IEEEspecialpapernotice{(Invited Paper)}




% make the title area
\maketitle

% As a general rule, do not put math, special symbols or citations
% in the abstract or keywords.
\begin{abstract}
The shift-enabled property of an underlying graph is essential in designing distributed filters. This article discusses when a random graph is shift-enabled. In particular, popular graph models Erdős–Rényi (ER), Watts–Strogatz (WS), Barabási–Albert (BA) random graph are used, weighted and unweighted, as well as signed graphs. Our results show that the considered unweighted connected random graphs are shift-enabled with high probability when the number of edges is moderately high. However, very dense graphs, as well as fully connected graphs, are not shift-enabled. Interestingly, this behaviour is not observed for weighted connected graphs, which are always shift-enabled unless the number of edges in the graph is very low. 
%The proposed research activity will fill the gap of linear filter design in graph signal processing.


\end{abstract} %

\begin{IEEEkeywords}
graph signal processing, shift-enabled graphs, undirected graph, random graph.
\end{IEEEkeywords}

% For peer review papers, you can put extra information on the cover
% page as needed:
% \ifCLASSOPTIONpeerreview
% \begin{center} \bfseries EDICS Category: 3-BBND \end{center}
% \fi
%
% For peerreview papers, this IEEEtran command inserts a page break and
% creates the second title. It will be ignored for other modes.
\IEEEpeerreviewmaketitle



\newtheorem{myDef}{Definition}
\newtheorem{Thm}{Theorem}
\newtheorem{Rem}{Remark}
\newtheorem{Lem}{Lemma}
\newtheorem{Cor}{Corollary}

\section{Introduction}
\label{sec:intro}
Graph signal processing (GSP) extends classical digital signal processing to signals on graphs and provides a promising solution to numerous real-world problems that involve data defined on topologically complicated domains~\cite{Liyanchen2018}. For large graphs, graph signals need to be processed in a distributed rather than centralized manner \cite{sandryhaila_2014_big_data}. That is, a graph node may only have access to graph signals acquired by nodes in its physical proximity. Furthermore, for large graphs with millions of nodes, a centralised implementation of
the graph filter \cite{Shuman_2013_The_emerging_field}, \cite{ortega2018graph} through direct matrix multiplication is computationally intractable~\cite{segarra2017optimal,coutino2019advances}.
Thus to make the graph filtering feasible, it is necessary to perform
the filtering operation locally \cite{sandryhaila_2014_big_data}. 
%Therefore, in
%practice, we expect that a node can only impose direct influence on an adjacent node.
For practical design purposes,
it is necessary to be in a position to decompose graph filters in
a form of polynomial of a shift matrix, of graph shift operator, $S$, that uniquely defines graph topology
(for example, graph adjacency or Laplacian matrix)
%(see example in Figure~\ref{fig_local}) 
\cite{sandryhaila_2013_discrete}, \cite{segarra2017network}. 
However, not all graph filters can be represented as polynomials of the shift matrix\footnote{The importance of this polynomial representation has been reiterated in a recent survey paper \cite{ortega2018graph}.}.
%can view $S$ as the generator of a filter. But we will call it the ``shift" operator as it is more commonly referred to in the literature.
%It is known to be a ``shift" since it is analogous to the $z$ operator in $z$-transform of classical 1-D DSP.  Please refer to Section \ref{sect:basicconcepts} for more details. 


% \begin{figure}[htb]
% 	\centering
% 	\includegraphics[width=2in]{shift}
% 	\caption{$x$: graph signal;  $S$: shift-enabled shift matrix (Laplacian matrix).}        \label{fig_shift}
% \end{figure}

% \begin{figure}[htb]
% 	\centering
% 	\includegraphics[width=3.5in]{local}
% 	\caption{Graph filter $H$ is a polynomial in $S$. Note that, different colors  represent different values in matrices. As the shift matrix $S$ captures the local structure of the graph, the computational complexity is greatly reduced.}
% 	\label{fig_local}
% \end{figure}

Given a graph, necessary conditions for a graph filter to be representable as a polynomial of the graph shift matrix is discussed in \cite{Liyanchen2018} and \cite{Liyanchen2021}, where the notion of \emph{shift-enabled} graph is introduced as a graph where any shift-invariant filter $H$ can be represented as a polynomial of the shift matrix.
It is shown in \cite{Liyanchen2018} and \cite{Liyanchen2021} that the shift-enabled condition \cite{Liyanchen2018} is important for both directed and undirected graphs, and hence it needs to be taken into account.  %性质这么重要,那么在现实的应用中,

%For non shift-enabled graph, the shift matrix $S$ can be converted into a new shift matrix $\tilde{S}$ such that $\tilde{S}$ is shift-enabled and shift-invariant filter $H$ is a polynomial of $\tilde{S}$ \cite{sandryhaila_2014_discrete_frequency}. However, it is proved in~\cite{Liyanchen2018} that for directed graphs, such conversion is not always possible. Furthermore, it is shown in \cite{Liyanchen2021} that for undirected graphs, although it is possible to convert the non-shift-enabled shift matrix $S$ to shift-enabled $\tilde{S}$, such that graph filter $H$ is a polynomial of $\tilde{S}$, the converted $\tilde{S}$ will no longer maintain the topological structure of the original graph, which makes $\tilde{S}$ unable to reflect the local properties of the graph. As a result, the computational complexity of  $H=h(\tilde{S})$ is increased.
% 对无向图来说虽然可以通过将shift matrix $S$的转换为$\tilde{S}$, 然后将$H$ 表示为$\tilde{S}$ 的多项式,但是转换后的$\tilde{S}$ 将不再保持原来图的拓扑结构,
% 这使得$\tilde{S}$ 不能反映图的局部性质,从而加大了$H=h(\tilde{S})$的计算复杂度.

% As a result, in this article, we will focus on the properties of 
% graph filters, where we assume that each node only has access to the signals of the current
% and its neighboring nodes at each time instance. And we will say a filter that only takes its
% signals at the current location and its neighboring locations as local. Of course, it would be
% a significant constraint on the choice of filters if we only allow them to be local. However,
% this constraint can be overcome as we ``cascade" multiple local filters.

% As a consequence, inherently it is interesting to decompose filters in a form of polynomial
% of some local matrix S. One can view S as the generator of a filter. But we will call it the
% ``shift" operator as it is more commonly referred to in the literature. It is known to be a
% ``shift" since it is analogous to the z operator in z-transform of classical 1-D DSP.  Please see
% Section $\cdots$ for more details. 

This paper focuses on finding a likelihood for a random graph to be shift enabled. %That is, we 
%the question if almost all graphs shift-enabled, are almost no graphs shift-enabled, or is neither true?
%How likely is a graph shift-enabled? 
This problem has received relatively little attention in the research community, since most researchers currently assume that the shift-enabled
condition simply holds or ignore the condition completely.
%according to the conclusion in\cite{sandryhaila_2014_discrete_frequency}.
%No one has studied this issue so far. 
To illustrate ``how likely is a graph shift-enabled'',
we discuss the probability that some classic random graphs are shift-enabled. 
%First of all, it discusses unweighted random graphs, including ER random graph, Watts graph and BA graph. Then,  weighted random graphs are discussed, including exponential weight , Gaussian weight and signed graphs. % 这一问题至今没有人研究过, 本文对一问题进行了讨论. 首先,本文首先对未加权的随机图进行讨论,包括ER random graph, Watts graph 和 BA graph. 然后, 加权的三种随机图进行讨论,包括 exponential weight and Gaussian weight.
In particular, the main contribution of this paper is characterising the bahaviour of the probability that:
\begin{itemize}
	\item an unweighted random Erdős–Rényi (ER) graph, Watts–Strogatz (WS) graph and Barabási–Albert (BA) graph is shift-enabled as a function of graph parameters;
	\item the above three weighted random graphs are shift-enabled, where the weights are generated based on exponential and Gaussian distribution;
	\item a random signed graph is shift-enabled; 
	\item the analysis of the above results.
\end{itemize}
Our results show that the considered unweighted connected random graphs are shift-enabled with high probability when the number of edges is moderately high. However, very dense graphs, as well as fully connected graphs, are not shift-enabled. Interestingly, this behaviour is not observed for weighted connected graphs, which are always shift-enabled unless the number of edges in the graph is very low.

The outline of the paper is as follows. Section
\ref{sect:basicconcepts} describes the basic concepts and fundamental properties of a shift-enabled graph. Section~\ref{sec:method} provides the main results of the paper for unweighted graphs, that is, the characterisation of the behaviour of the probability that a graph is shift enabled. Section~\ref{sec:weights} extends the results to weighted and signed graphs.
Section~\ref{sec:discussion} concludes the paper.

%\textbf{Notation:}



% \textbf{Experiment:}
% To illustrate how likely is a random graph shift-enabled, we present numerical experiments. In order to assess the change rules of shift-enabled, %we generate 1000 random graphs for each  experimental value with $N=20$ nodes.
% each experimental value comes from 1000  times testing, and 100 random graphs are generated each time, and the average value is taken as the experimental value.
%%%%%%%%%%%%%%%%%%%%%%%%%%%%%%%%%%%%%%%%%%
\section{Basic concepts and properties of shift-enabled graphs}
\label{sect:basicconcepts}
In this section, we introduce our notation and briefly review the concepts of shift-enabled graphs and their properties relevant to this article. For more details, see \cite{sandryhaila_2014_big_data,sandryhaila_2014_discrete_frequency, Shuman_2013_The_emerging_field,sandryhaila_2013_discrete}.

% $L$, $\mathcal{G}^C$, and $Spec(X)$ denote the combinatorial Laplacian matrix, the complement of $\mathcal{G}$, and the spectrum of the matrix $X$, %$L_{\mathcal{G}}$,
% respectively. 
%For a graph $\mathcal{G}$, $N$ and $M$ will denote the number of vertices and edges, respectively, and $p$ is the probability that a random graph is shift enabled. 

Let $\mathcal{G}=(V,E,W)$ be a graph, where $V=\{v_1,v_2,\cdots,v_{N}\}$ is the vertex set, and $E\subset\{1,\cdots,N\}\times\{1,\cdots, N\}$ is the edge set in which $(i,j)\in E$ if vertex $v_i$  and vertex $v_j$ have a link. $W=(w)_{i,j=1}^N$ is the weighted adjacency matrix, in which  $w_{i,j}$ represents the weight of the edge $(i,j)\in E$.
%each edge of a
%graph has an associated numerical
%value, called a weight. 
%In particular, for an unweighted graph $W$ is equal to the adjacency  matrix $A$ of the graph. Then, the graph is denoted by $\mathcal{G}=(V,E,A)$. 
 Throughout this article, a graph $\mathcal{G}$ is assumed to be simple, i.e. a finite, graph without loops and/or multiple edges. 
%{\color{red} Let $\bm{x}=(x_1,x_2,\cdots,x_{N})^T$ be a {\em graph signal}, where each sample $x_i \in \bm{x}$ corresponds to a vertex $v_i \in V$, and 
%DO WE NEED TO DEFINE IT HERE? IT DOESN"T SEEM TO BE USED ANYWHERE -sam}\textcolor{blue} { It used in Figure 1, Should we delete it?-Liyan }
Let $D=diag(D_1,\cdots,D_n)$, with $D_i=\sum_{j=1}^{N}w_{i,j}$, be the degree matrix of $\mathcal{G}$\cite{poignard2018spectra}.
%\end{myDef}).
% use $S$ to denotes general shift matrix. 
%The combinatorial Laplacian matrix is defined as $L=D-W$
%\begin{myDef}[Laplacian matrix]\label{def:Laplacian_matrix}
%	\begin{equation*}
%	L=D-W
%	\end{equation*}

% Specially, $L=D-A$ for an unweighted graph. 


The graph is shift-enabled if its shift matrix $S$ (see Remark 2 below) complies with the following definition.

% \begin{myDef}[linear shift-invariant filter (LSI)] 
% A filter $H$ is linear shift-invariant if $HS=SH$.
% \end{myDef}
% That is, the shifted and filtered operations should coummute. In other words,  the shifted filtered output should be the same as filtered output of a shifted input. 

\begin{myDef}[Shift-enabled graph {\cite{Liyanchen2018,sandryhaila_2013_discrete}}]
	\label{def:shift_enable}
	A graph ${\mathcal G}$ is shift-enabled if its corresponding shift matrix $S$ satisfies $p_S(\lambda) =m_S(\lambda)$, where $p_S(\lambda)$ and $m_S(\lambda)$ are the minimum polynomial and the characteristic polynomials of $S$, respectively. We also say that $S$ is shift-enabled when the above condition is satisfied. Otherwise, $S$ and the corresponding graph, are non-shift-enabled.
\end{myDef}

\begin{Rem}
The shift-enabled condition ($p_S(\lambda) =m_S(\lambda)$) is equivalent to the condition that each Jordan block of the Jordan normal form of the shift matrix has a distinct eigenvalue (see Proposition 6.6.2 in \cite{lancaster_1985_matrix_theory}).
Consequently, for real symmetric matrices, the above condition naturally degenerates to the condition that all eigenvalues have to be distinct (see details in Lemma  ~\ref{Lem_undirected_shift}).
\end{Rem}

\begin{Rem}
The graph adjacency matrix $A$, %combinatorial
Laplacian $L=D-W$,  the normalized Laplacian matrix $\mathcal{L}=L^{-1/2} D L^{-1/2}$, the signless Laplacian matrix $L^{+}=D+W$  and the probability transition matrix $T=D^{-1}W$ are generally chosen as the graph shift operator or graph shift matrix~\cite{sandryhaila_2014_big_data, sandryhaila_2014_discrete_frequency,Shuman_2013_The_emerging_field,Marques_2017}. Here,  we use $S$ to denote the general
shift matrix 
and select Laplacian matrix as the shift matrix for the specific discussion in Section~\ref{sec:method} and Section~\ref{sec:weights}, since
%select Laplacian matrix as shift matrix which is one of most popular choices [add refference],
Laplacian matrix is one of most popular shift matrix~\cite{Marques_2017,dong2020graph,Dittrich20200signedgraph}. %the results can be readily extended to other shift matrices.
Most of the conclusions in this paper, however, apply to other shift matrices (see Figure~\ref{fig:different shift matrix}).
%(see discussion in Section~\ref{sec:discussion}). 
\end{Rem}

%The common shift matrices include
%the Laplacian matrix, the normalized Laplacian matrix, the diffusion matrix, and so on \cite{Marques_2017}. 

% It was shown in  \cite{sandryhaila_2013_discrete} that any filter commuting with shift matrix $S$ can be represented as a polynomial in $S$ provided that $S$ is shift-enabled.




% As both shift matrix $S$ and filter matrix $H$ are symmetric, we can obtain the following lemma.


% \begin{Lem}{\label{Lem_diag}}
% If shift matrix $S$ and filter matrix $H$ are diagonalizable (this condition always holds for symmetric matrix) then $S$ and $H$ are simultaneously diagonalizable (by an invertible matrix) if and only if $HS=SH$~\rm{(see Theorem 1.3.12 in \cite{horn_2012_matrix_analysis})}.
% \end{Lem}

For shift-enabled graphs, we have the following result.
\begin{Thm}{\label{Thm_shift_commute}}
	The shift matrix $S$ is shift-enabled if and only if every matrix $H$ commuting with $S$ is a polynomial in $S$\rm~{\cite{sandryhaila_2013_discrete}}.
\end{Thm}
A graph filter $H$ is linear shift-invariant (LSI) if $H$ commutes with shift matrix $S$  ($HS=SH$).
That is, the shifted and filtered operations are commuted, i.e., the shifted filtered output is the same as filtered output of a shifted input. 
Theorem~\ref{Thm_shift_commute} implies that 
an LSI filter naively designed cannot always be represented as a polynomial of shift
operators as is the case in classical DSP; that is, as mentioned in Section~\ref{sec:intro}, whether the graph is directed or undirected, the shift-enabled condition is important. Thus, it is interesting to investigate ``how likely is a graph shift-enabled?''. The next section gives the answers for commonly used random graphs.

\section{Unweighted random graphs}
\label{sec:method}
In this section we focus on classic random unweighted graphs, namely,
we will consider ER, WS (small world model), and BA graphs (scale-free model), and calculate probability $p$ that the random graph is shift enabled. We examine how $p$ depends on the parameters used to generate the graph.

\subsection{Generic Random Graphs Models}
\begin{figure}[htb]
    \centering
    \includegraphics[width=1\linewidth]{ERWSBAGRAPH50nodes}
    \caption{Examples of generic random graph models with $N$=50 nodes. (a) ER graph with the probability of generating edges $P=0.5$. 
    ER random graph has no central node, and most nodes are evenly connected.
    %没有中心节点,大部分节点都均匀的连在一起
    (b) WS graph with average degree $K=4$ and rewriting probability $\beta=0.5$. WS graph has small-world properties, including short average path lengths and high clustering. (c) BA graph with the initial number of nodes $m_0=5$ and the number of edges for each node addition $m=4$.
    BA graph is a scale-free network, in which the connections between nodes are severely unevenly distributed: a few nodes in the network called Hub points have extremely many connections, while most nodes have only a small number of connections.
    %各节点之间的连接状况(度数)具有严重的不均匀分布性:网络中少数称之为Hub点的节点拥有极其多的连接,而大多数节点只有很少量的连接。
    }
    \label{fig:ERWSBA}
\end{figure}
\subsubsection{Erdős–Rényi random graph (Figure~\ref{fig:ERWSBA} (a))}
The Erdős–Rényi (ER) random graph is often used to model many real-world inference problems, and it has been used in the context of graph filter design, e.g., in \cite{segarra2017network}, \cite{mei2016signal} and \cite{nassif2018distributed}. The ER graph topology can be defined in two ways~\cite{frieze2015introduction} as:
\begin{itemize}
	\item 
	$\mathcal{G}(N,P)$ where $N$ and $P$ denote  the number of nodes and the probability that an edge is present, respectively.
	\item
	$\mathcal{G}(N,M)$ where $M$ is the number of edges, which are randomly distributed in the graph.
\end{itemize}
The relationship between the two models
is $P=M/\binom{N}{2}$, %and for very large $N$, 
and the two models are asymptotic equivalent as $N$ increase (Theorem 1.4 in Ref.~\cite{frieze2015introduction}).	
\subsubsection{Watts-Strogatz random graph (Figure~\ref{fig:ERWSBA} (b))}
The Watts-Strogatz (WS) model is a classic random graph generation model which produces graphs with small-world properties \cite{Watts1998Collective}.
WS graph has three parameters, $N$, $K$ and $\beta$, and is denoted as $\mathcal{G}=WS(N, K,\beta)$ where $N$ is the number of nodes, $K$ is the average degree of nodes, and $\beta$ is the rewriting probability. If $\beta=0$ WS graph is a regular ring lattice in which each node is connected to the nearest $K$ nodes, and if $\beta=1$, the WS graph becomes an ER graph.



\subsubsection{Barabási–Albert random graph (Figure~\ref{fig:ERWSBA} (c))}
The Barabási–Albert (BA) model is an algorithm for generating random scale-free networks using a preferential attachment mechanism. It can model many practical scale-free networks 
such as the World Wide Web, citation networks and social networks. In contrast to the ER and the WS models, the BA model is scale-free, and the connections between the nodes are severely unevenly distributed: a few nodes in the network, called Hub points, have extremely many connections, while most other nodes have only a small number of connections. The
BA random model has three parameters: the number of nodes $N$, the initial number of nodes $m_0$, and the number of edges for each node addition $m$, $m\leq m_0$, and is denoted by $\mathcal{G}=BA(N,m_0, m)$. New nodes are added according to the priority strategy $-$ ``the more connections between nodes, the greater the possibility of receiving a new link" ~\cite{barabasi1999emergence,albert2002statistical,barabasi2013network}.

\subsection{%The General Law of Probability Change of Shift-enabled Property of Unweighted Graph
Shift-enabled properties of ER graphs}\label{sec: ER graph}
 
Figure~\ref{fig:the shift-enabled probability of ER graphs} shows the probability $p$ that an ER graph $\mathcal{G}(N,M)$ is shift-enabled as a function of the number of edges $M$. The simulation results are provided for $N$=50 nodes and the results are averaged over $10^5$ runs. From the figure, we can identify three distinct regions: Region 1: a very small $M$ where $p$ is zero; Region 2: the flat region when $p$ reached the maximum close to 1; Region 3, where $p$ drops to 0 for very large $M$, following a symmetric trend to Region 1.

Next, based on these heuristic findings, we separately treat the three regions. We fix the number of nodes $N$ and change the number of edges $M$, and theoretically analyse the behaviour of the probability $p$ that the resulting graph is shift-enabled. 

\begin{figure}[htb]
	\centering
	\includegraphics[width=2.5in]{ERLaplacianRandom50Vertices}
	\caption{The shift-enabled probability of the unweighted ER graph with $N$=50 nodes: probability that the graph is shift-enabled $p$ vs. the number of edges $M$.}        \label{fig:the shift-enabled probability of ER graphs}
\end{figure}



%\begin{Thm}{\label{Thm:general law}}
%As Figure \ref{fig:compare the shift-enabled probability of different graphs} shown,  %the changing law of shift-enabled tends to stabilize with the number of vertices increases. Furthermore,


\subsubsection{Region 1 $-$ $M$ is small}\label{sec: Region 1}

In this region, based on the following theorem, the probability of a graph being shift-enabled is always zero.

\begin{Thm}{\label{Thm_p_tends_to_zero}}
	If a graph $\mathcal{G}$ is unconnected then the shift matrix is not shift-enabled.
\end{Thm}
\begin{proof}
	Assume $\mathcal{G}$ has two unconnected components ${\mathcal{G}}_1$ and ${\mathcal{G}}_2$, with corresponding Laplacian matrices $L_1$ and $L_2$, respectively. Since both $L_1$ and $L_2$ have an eigenvalue equal to zero, $L$, the Laplacian matrix of $\mathcal{G}$ will have the eigenvalue 0 with at least the multiplicity of two. Therefore, based on Lemma~\ref{Lem_undirected_shift}, $\mathcal{G}$ is non-shift-enabled.
\end{proof}
Since an $N$-node connected graph has at least  $N-1$ edges, we have the following corollary.
\begin{Cor} {\label{Cor:non shift for small edges}}
	If $M\leq N-2$ then the shift matrix $L$ of graph $\mathcal{G}$ is non-shift-enabled. 
\end{Cor}
According to Corollary~\ref{Cor:non shift for small edges} the probability of a graph being shift-enabled is $p=0$ when $M$ is small with respect to $N$, that is, when $M\leq N-2$.

%\end{Thm}

%\subsubsection{The proof of Law 1 (The probability of shift-enabled graph $P=0$ as the number of edges $M$ is relatively small.)}

\subsubsection{Region 2 $-$ Moderate $M$}\label{sec: Region 2}

%\subsubsection{The proof of Law 2. That is, the probability of shift-enabled graph increases in the first half. And the probability of shift-enabled graph is close to $1$ in the first half, when the number of vertices $N$ and the number of edges $M$ are large enough.}

The following theorem shows that when $N$ is sufficiently large, and the probability that an edge is present $P$ is far from 0 and 1 (alternatively, $M$ is moderately large) then the eigenvalues of Laplacian matrix are distinct. The uniqueness of the eigenvalues guarantees the shift-enable property by Lemma~\ref{Lem_undirected_shift}.
% \begin{Thm} [Simple Spectrum, Theorem 1.3 in Ref. \cite{tao2017random}]
%  Let $M_N$ be a random matrix  whose upper triangular entries have non-trivial distribution for some fixed $\mu>0$. Then for every fixed constant $c>0$ and $N$ sufficiently
% large (depending on $c,\mu$), the spectrum of $M_N$ is simple with probability at
% least $1-N^{-c}$.
% \end{Thm}
\begin{Thm}[Distinct eigenvalue, Theorem 1.3 in Ref. \cite{tao2017random}]{\label{Thm: simple spectrum}}
	Let $\mathcal{G}$ be a connected graph and $X_N=(x_{i,j})_{1\leq i,j \leq N}$ be an $N\times N$ real symmetric random matrix in which the upper-triangular entries $x_{i,j} (i<j)$ are independent ((see Remark~\ref{remark:upper triangular} (a))
	%whose upper triangular entries 
	and have non-trivial distribution  for some fixed $\mu>0$ (see Remark ~\ref{remark:upper triangular} (b)). %{\color{red} of [17]? -sam})~{\color{blue} Partly from [17], it mainly explains why the condition of our random graph is true.-Liyan} %{\color{red} biggest problem to me is that is no (a) and (b) in Remark 1. -sam} ~{\color{blue}Sorry, I have modified it.}
	%and diagonal entries $m_{i,i}$ are independent. 
	Furthermore, 
	$x_{i,i}$ are independent of the upper diagonal entries $x_{i,j} (1\leq i<j\leq N)$ (see details in Remark~\ref{remark:upper triangular} (c)).
	Then for every fixed constant $c>0$ and $N$ sufficiently
large (depending on $c$ and $\mu$), the eigenvalues of $X_N$ are distinct with probability at
least $1-N^{-c}$. 
That is, for sufficiently large $N$, the probability that the eigenvalues of $X_N$ are distinct tends to 1.
\end{Thm}

%\newline
\begin{figure}[htb]
	\centering
	\subfigure[]{
	\includegraphics[width=0.48\linewidth]{ERConnectShiftNode5}}
	\subfigure[]{
	\includegraphics[width=0.48\linewidth]{ERConnectShiftNode10}}
	\subfigure[]{
	\includegraphics[width=0.48\linewidth]{ERConnectShiftNode15}}
	\caption{For an ER graph, the relationship between the probability that the graph is connected (red line) and the shift-enabled probability $p$ (blue line). The horizontal axis shows the number of the edges $M$ and the vertical axis is the probability $p$. (a) $N$=5 nodes; (b) $N$=10 nodes; (c) $N$=15 nodes.}
	\label{fig:connectshiftproblapnode5}
\end{figure}

\begin{Rem}\label{remark:upper triangular}
	(a) The upper-triangular entries are independent since we focus on randomly generated graphs. %Any random variable independent of $n$ is non-trivial if it does not take a single value almost surely.
	(b) Non-trivial distribution\footnote{A real-valued random variable $\xi$ is non-trivial if there is a fixed $\mu > 0$
such that $Pr\{\xi=x\}\leq 1-\mu$ (see Equation (1) in Ref.~\cite{tao2017random}).} - elements in Laplacian matrix of an ER graph are non-trivial if $P$ (the probability that an edge is present) stays bounded away from both 0 and 1. If $P=0$ or $P=1$, the entries in the graph Laplacian matrix have trivial distribution, and Theorem~\ref{Thm: simple spectrum} is not applicable.
%, that is, Region~\ref{sec: Region 1} of $M$ is small and Region~\ref{sec: Region 1} of $M$ is very large  in Figure~\ref{fig:the shift-enabled probability of ER graphs}. 
(c) Since $x_{i,i}=0$ for $1\leq i \leq N$, diagonal entries $x_{i,i}$ are independent of the upper diagonal entries.
\end{Rem}
As a result, the eigenvalues of Laplacian matrix are distinct (the graph is shift-enabled, hence $p=1$) when the number of nodes $N$ is large enough, and the number of edges $M$ is moderately large (very large $M$ implies $P$ close to zero, where Theorem~\ref{Thm: simple spectrum} does not hold).

Our experiments show that the behaviour of $p$ is very similar to the probability of a random graph being connected (see Figure  {\ref{fig:connectshiftproblapnode5}}).
Combining Theorem~\ref{Thm_p_tends_to_zero}, Theorem~\ref{Thm: simple spectrum} and the relation between shift-enabled and connected graphs, we claim that $p$ is close to 1 in Region 2, when the number of nodes $N$ is large and the probability of an edge is bounded away from 0 and 1. When $N$ is very small, Theorem~\ref{Thm: simple spectrum} does not hold; indeed, it can be seen from Figure  {\ref{fig:connectshiftproblapnode5}} that $p$ does not reach 1 for $N$=5 and 10. 

\subsubsection{Region 3 $-$ Very large $M$}\label{sec: Region 3}

%\subsubsection{The proof of Theorem~\ref{Thm:general law}.Law 3 (the symmetry of the shift-enabled property)}

%Consider $\mathcal{G}$ is a  simple undirected graph with $N$ vertices,

We analyse Region 3, that is, the case of very large $M$, by looking graph complement and showing that a graph ${\mathcal{G}}$ and its complement ${\mathcal{G^C}}$ have the same shift-enabled probability. Let $L_{\mathcal{G}}$ and $L_{\mathcal{G^C}}$ denote the Laplacian matrix of ${\mathcal{G}}$ and its complement ${\mathcal{G}^C}$, respectively. 
%Then, their spectrum $Spec(L_{\mathcal{G}})$ and $Spec(L_{\mathcal{G^C}})$ have the following properties.
The following theorem gives the condition that  $\mathcal{G}$ and its complement have the same shift-enabled property.
\begin{Thm}{\label{Thm_complement_shift}}
	If for an $N$-node random graph $\mathcal{G}$, $N$ is not an eigenvalue of $L$ then $\mathcal{G}$ is a shift-enabled graph if and only if $\mathcal{G}^C$ is a shift enabled graph.
\end{Thm}
\begin{proof}
By Lemma~\ref{Lem_complement_connect}, if $N$ is not an eigenvalue of $L$ then $\mathcal{G}^C$ is a connected graph.
	To prove the sufficiency, let $Spec(L_{\mathcal{G}})=(\lambda_1, \lambda_2, \cdots, \lambda_{N-1},\lambda_N=0)$. %, and $N$ is not the eigenvalue of $L$.
	Then, $Spec(L_{\mathcal{G^C}})=(N-\lambda_1, N-\lambda_2,  \cdots,N-\lambda_{N-1} , 0)$ by  Lemma~\ref{Lem_spec_complement}.
	
	If $\mathcal{G}$ is a shift-enabled graph, then $\lambda_i \neq \lambda_j$  %$\lambda_k \neq 0=\lambda_N$ ($1\leq i<j\leq N$, $1\leq k<N$)
	according to Lemma~\ref{Lem_undirected_shift}.
	If in addition $N$ is not the eigenvalue of $L$, it can be readily concluded that
	%It follows that %$\lambda_i \neq \lambda_j$ and $\lambda_k \neq 0$  if and only if 
	$N-\lambda_i \neq N-\lambda_j$ and $N-\lambda_k \neq 0$  ($1\leq i<j\leq N-1$, $1\leq k\leq N-1$), i.e., all eigenvalues in $L_{\mathcal{G}^C}$ are distinct. Consequently,  $\mathcal{G}^C$ is a shift-enabled graph
	by Lemma~\ref{Lem_undirected_shift}.
	
	The necessity can be easily proven in the  same way.
	To sum up, $\mathcal{G}$ is a shift-enabled graph if and only if $\mathcal{G}^C$ is a shift-enabled graph.
\end{proof}

\begin{Cor}
\label{Cor:R3}
If $N$ is not an eigenvalue of $L$, then, for very large number of edges relative to the number of nodes $N$ ($M>N(N-1)/2-N+2$) the graph is non-shift-enabled.
\end{Cor}


Based on Theorem~\ref{Thm_complement_shift}, the behaviour of $p$ for $\mathcal{G}$ and $\mathcal{G}^C$ is symmetric. Since based on Corollary~\ref{Cor:non shift for small edges}, $\mathcal{G}^C$ is non-shift-enabled when its number of edges is smaller or equal to $N-2$, then, in Region 3, for a very large number of edges, $\mathcal{G}$ is also non-shift-enabled. Note that this implies that fully connected ER graphs are non-shift-enabled. Indeed, as we know, the eigenvalues of a fully connected graph are ${\{-1\}}^{N-1}$ (the multiplicity of $-1$ is $N-1$) and $N-1$, which ensures that the graph is not shift enabled.

% \begin{proof}
% Let $Spec(L_{\mathcal{G}})=(\lambda_1, \lambda_2, \cdots, \lambda_{n-1},\lambda_n=0)$, and $n$ is not the eigenvalue of $L$. By Lemma~\ref{Lem_undirected_shift}, all eigenvalues in $L_{\mathcal{G}}$ are distinct as $\mathcal{G}$ is a shift-enabled graph.
% That is, $\lambda_i \neq \lambda_j$  for $1\leq i<j\leq n$. It follows Lemma~\ref{Lem_spec_complement} that $n-\lambda_i \neq n-\lambda_j$ and $n-\lambda_k \neq 0$ for $1\leq i<j<n$ and  $1\leq k<n$, i.e., all eigenvalues in $L_{\mathcal{G}^C}$ are distinct. Consequently,  $\mathcal{G}^C$ is a shift-enabled graph. 

% The necessity can be easily proved in the same way.
% To sum up, $\mathcal{G}$ is a shift-enabled graph if and only if $\mathcal{G}^C$ is a shift-enabled graph.
% \end{proof}
% \subsection{The Effect of $N$ on ER Graph}
% ER graph has two parameters the number of vertices and the number of edges. With the increase of the number of vertices, the law of probability change of shift-enabled tends to be stable (see Figure~\ref{fig:ER_change_nodes}). And the law of change with the increase of edges is indicated in Theorem~\ref{Thm:general law}.
% \begin{figure}[htb]
% 	\centering
% 	\includegraphics[width=1\linewidth]{ERLapChangeNodeShift}
% 	\caption{X-axis: the number of edges, Y-axis: the shift-enabled probability of the graph. With the increase of the number of vertices, the law of probability change of shift-enabled tends to be stable. }
% 	\label{fig:ER_change_nodes}
% \end{figure}

The conclusions can be extended to other commonly used shift matrices.
\begin{Rem}
		Commonly used shift matrices of $\mathcal{G}$ are: adjacency matrix $A$, %combinatorial 
		Laplacian matrix $L$, normalized Laplacian matrix $\mathcal{L}$, signless Laplacian matrix $|L|$ and probability transition matrix $T$ which have similar shift-enabled property rules (see Figure~\ref{fig:different shift matrix} for an illustration).
	\begin{figure}[htb]
		\centering
		\includegraphics[width=0.7\linewidth]{ERDifferentShiftMatrix50nodes}
		\caption{$p$ vs. the number of edges $M$ in an ER random graph with $N$=50 nodes. Adjacency matrix $A$, Laplacian matrix $L$, normal Laplacian matrix $\mathcal{L}$,  signless Laplacian matrix $|L|$ and probability transition matrix $T$ have similar shift-enabled properties.}
		 	\label{fig:different shift matrix}
	\end{figure}
\end{Rem}
\subsection{Shift-enabled properties of WS graphs}
In the previous subsection we discussed how, for fixed $N$, $p$ changes as the number of edges $M$ varies in ER graphs. For an $\mathcal{G}=WS(N, K,\beta)$ graph, in this subsection, we investigate how $p$ depends on WS parameters, namely, $K$, the average degree of nodes and $\beta$ the rewriting probability. Note that the number of edges is $M=N\dot K$. We separately discuss two cases. 
\subsubsection{The number of nodes $N$ fixed, and the average degree $K$ and rewriting probability $\beta$ vary}


\begin{figure}[htb]
	\centering
	\includegraphics[width=0.7\linewidth]{WSNode50ChangeBeta}
	\caption{The effect of $\beta$ on the shift-enabled probability $p$ of WS graphs. Except for $\beta=0$ where WS graph is a regular ring lattice, the similar conclusions as for ER graphs can be taken. 
%		(a) $\beta=0$. (b) $\beta=0.3$. 
%		(b) $\beta=05$. 
%		(b) $\beta=0.7$. 
%		(b) $\beta=1$. 
	}
	\label{fig:nodes20changebetalap}
\end{figure}

As shown in Figure~\ref{fig:nodes20changebetalap}, except for $\beta=0$ where the WS graph is a regular ring lattice (see Theorem~\ref{Lem:ring lattice}), the discussions  related to the ER graphs in Section~\ref{sec: Region 1} and Section~\ref{sec: Region 2}
%in the previous subsection 
apply to WS graphs as well.
\begin{Thm}{\label{Lem:ring lattice}}
Regular ring lattice is non-shift-enabled when the shift matrix is Laplacian matrix.  
\end{Thm}
\begin{proof}
% For $\beta=0$, WS graph is a regular ring lattice which is non-shift-enabled, 
% since 
Regular ring lattice is a special circulant graph in which each node is connected to the nearest $K$ nodes.
Let $A_{cir}$ denote an $N\times N$
circulant matrix whose first row is $[0,1,0,\cdots, 0]$ and the adjacency  matrix of regular ring lattice is $A_{ring}$ whose first row is $[0,\underbrace { {1,1, \cdots ,1} }_K,0,\cdots,0,\underbrace { {1,1, \cdots ,1} }_K]$. Then,
\begin{equation*}
    A_{ring}={\sum_{j=1}^{K}A_{cir}^{j}}+{\sum_{j={N-K}}^{N-1}A_{cir}^{j}}.
\end{equation*}

Since the eigenvalues of $A_{ring}$ are $1,\omega,\omega^{N-1}$, where $\omega=exp(2\pi i/N)=\cos\frac{2\pi }{N}+i\sin\frac{2\pi }{N},$ %~\footnote{$i$ is an imaginary unit.}, 
it can be easily concluded that the eigenvalues of $A_{ring}$ are
\begin{flalign*}
 \lambda_j&=\omega^j+%\omega^{2j}+
 \cdots+\omega^{K j}+\omega^{(N-K)j}+\cdots+\omega^{(N-1)j}\\   
 &=2[\cos\frac{2\pi j}{N}+\cos\frac{4\pi j }{N}+\cdots+\cos\frac{2K\pi j} {N}],
\end{flalign*}
 where $j=0,1,\cdots,N-1$~\cite{toddh}.
It follows that 
\begin{flalign*}
        &~~\lambda_{N-1}\\
        &=2[\cos\frac{2(N-1)\pi }{N}+\cos\frac{4(N-1)\pi  }{N}+\cdots+\cos\frac{2K(N-1)\pi } {N}]\\
      &=2[\cos\frac{2\pi }{N}+\cos\frac{4\pi  }{N}+\cdots+\cos\frac{2K\pi } {N}]\\
      &=\lambda_1.
\end{flalign*}
Generally, 
$\lambda_0=2K$ and $\lambda_j =\lambda_{N-j}$, for $j=1,2,\cdots,\lfloor \frac{N-1}{2} \rfloor$~\footnote{$\lfloor \frac{N-1}{2} \rfloor$ denotes rounding down $\frac{N-1}{2}$.}.

 Furthermore, the Laplacian matrix of the regular ring lattice is $L_{ring}=D_{ring}-A_{ring}$ and the degree matrix $D_{ring}=2K\times I_N$ \footnote{$I_N$ is an identity matrix of size $N \times N$.}. Therefore, the eigenvalues of $L_{ring}$ are $2K-\lambda_j$ for $i=0,1,\cdots,N$ and $2K-\lambda_j =2K-\lambda_{N-j}$ for $j=1,2,\cdots,\lfloor \frac{N-1}{2} \rfloor$.
Then, according to Lemma~\ref{Lem_undirected_shift}, the regular ring lattice is not shift enabled.
% Assume $(a_1,a_2,\cdots,a_n)^T$ is an eigenvector of Laplacian matrix $L$.
% Then, $(a_n,a_1,\cdots,a_{n-1})^T$ is also an eigenvector of $L$ since the Laplacian matrix of a regular ring lattice is a circulant symmetric matrix.


% The Laplacian matrix of a regular ring lattice is a circulant symmetric matrix which has multiple identical eigenvalues. Otherwise, by the property of the cyclic matrix,  if $(a_1,a_2,\cdots,a_n)$ is an eigenvector of $L$ then $(a_n,a_1,\cdots,a_{n-1})$ is also an eigenvector of $L$. has multiple eigenvalues
% Unless all eigenvectors are scalar multiple of its rotational shift (which is impossible unless the eigenvector is $(1,1,\cdots,1)$, but  that is also impossible since Laplacian matrix is symmetric and so eigenvectors span the entire space), we must have at least one eigenvector such that one of its rotational shifts is linearly independent of the eigenvector itself. And so the respective eigenvalue is not distinct, i.e., shift matrix $L$ is non-shift-enabled.

%{\color{red} SO WHAT? DON"T GET DISTRACXTED FROM WHAT YOU ARE TRYING TO SHOW}~{\color{blue} Now I finished the proof.-Liyan}
\end{proof}
\subsubsection{Rewriting probability $\beta$ fixed and the number of nodes $N$ and the average node degree $K$ vary}

As shown in Figure~\ref{fig:Watt_change_node}, similarly to ER graphs, as the number of nodes increases, the shift-enabled property tends to be stable and has the properties discussed in Section~\ref{sec: Region 1} and Section~\ref{sec: Region 2}. %Section~\ref{sec: Region 3}.
\begin{Rem}
%{\color{blue} 
Note that, the symmetry in Section~\ref{sec: Region 3} is not as obvious in Figure~\ref{fig:nodes20changebetalap} and  Figure~\ref{fig:Watt_change_node} as it is in Figure~\ref{fig:the shift-enabled probability of ER graphs}. This is because the horizontal axis in WS graph is the average $K$ which is a non-negative even number, while the horizontal axis in Figure~\ref{fig:nodes20changebetalap}
is the number of edges $M$. The symmetry of $p$ of $\mathcal{G}$ and $\mathcal{G}^C$ in Theorem~\ref{Thm_complement_shift} is discussed based on $M$. $K$ and $M$ are linearly dependent via $K=M/N$.
%}
\end{Rem}
%值得注意的是,Region 3的对称性在图6中表现不如图3(ER 图)那么明显。这是因为WS的横坐标是图的平均度K(K是非负偶数),而图3的横坐标是图的边数m=Nk。定理4中的G和G^c的shift性质的对称性结论是基于边的变化进行讨论的。k的变化反应M的变化,但是不是完全一致。
%{\color{red} SHOULD COMMENT WHY THE GRAPH IS NOT SYMMETRIC DESPITE THM4}

\begin{figure}[htb]
	\centering
	\subfigure[]{
		\label{fig:WattFixBeta10node}
		
		\includegraphics[width=1.5in]{WSNode10Fixbeta5}}
	\centering
	\subfigure[]{
		\label{fig:WattFixBeta20node} 
		\includegraphics[width=1.5in]{WSNode20Fixbeta5}}
	\centering
	\subfigure[]{
		\label{fig:Watt_fix_beta_50node} 
		\includegraphics[width=1.5in]{WSNode50Fixbeta5}}
	\caption{The shift-enabled probability $p$ of WS graph as a function of $K$. With the increase of the number of nodes, $p$ follows the same trends as for ER graphs. (a) $N=$10 nodes. 
		(b) $N=$20 nodes. (c) $N=$50 nodes. 
	}
	\label{fig:Watt_change_node}
\end{figure}



\subsection{Shift-enabled properties of BA graphs}
\begin{figure}[htb]
	\centering
	\subfigure[]{
		\includegraphics[width=1.5in]{BA50nodeschangem01}}
	\quad

	\centering
	\subfigure[]{
		\includegraphics[width=1.5in]{BA50nodeschangem025}
	}

	\caption{The effect of $m_0$ on the shift-enabled probability $p$ in a BA graph with $N=50$.
		(a) $m_0=1$; 
		(b) $m_0=25$;
	}
	\label{fig:BA_fix_change_m0}
\end{figure}
	\begin{figure}[htb]
	    \centering
	    \includegraphics[width=2in]{Barabasi_Albert_generated_network}
	    \caption{The BA graph with $m_0=1$ and $N=50$. As $m=1$ the graph is a tree graph, in which case the  main components of the random model contain only a small number of nodes.} %~\cite{enwiki}.}
	    \label{fig:BA graph with m0_1}
	\end{figure}
\begin{figure}[htb]
    \centering
    \includegraphics[width=2.5in]{BAChangeNm01Lap}
    \caption{BA graph: $p$ vs. the number of nodes $N$ for $m_0=m=1$.}
    \label{fig:BAChangeN_m0_1}
\end{figure}
In Figure~\ref{fig:BA_fix_change_m0}, we fixed the total number of nodes $N$, and change the initial number of nodes $m_0$ from 1 to $N$ to study the probability $p$ of a graph being shift-enabled for a range of $m_0$ values.
 The BA graph used in the experiment was generated by $gsp\_barabasi\_albert.m$ function available in GSPBox~\cite{perraudin2014gspbox}. 
 Note that the total number of edges is $M=(N-m_0)\times m$.  Since $m\leq m_0$, the maximum total number of edges is $|M|_{max}=(N-m_0)\times m_0$. For fixed $N$, $|M|_{max}$ is $m_0=\frac{N}{2}$. We provide the results for $m_0=\frac{N}{2}=25$.
 
 Based on Figure~\ref{fig:BA_fix_change_m0}, we make the following conclusions:
 \begin{itemize}
     \item  As can be seen from Figure~\ref{fig:BA_fix_change_m0} (a), when $m_0=1$, the number of edges for each node addition, $m$, can only be equal to 1 (since $m\leq m_0$). In this case the graph is a tree graph (Figure~\ref{fig:BA graph with m0_1}),  in which few nodes have many connections, while most nodes have very few connections, which reflects the scale-free feature of the BA graph. As shown in Figure~\ref{fig:BAChangeN_m0_1}, as the number of nodes $N$ increases, the connectivity of the vertex is increasingly concentrated, and the probability $p$ tends to 0.


     %, and hence according to Section~\ref{sec: Region 1}, the graph is non-shift-enabled. 
%	~\cite{enwiki:998323343}

 
 \item When $m_0=25$ and $m=1$, $M=25<N-1=49$, the graph is unconnected and hence $p=0$. As $m$ increases, the probability that the graph is connected increases, and as a result, $p$ increases.
 %{\color{red} DIDN"T DESCRIBE FIG 8b}~{\color{blue} I deleted it. Liyan}
\end{itemize}



\section{Extension to Weighted and Signed graphs}
\label{sec:weights}
In this section we extend the results to random weighted and signed graphs. 

\subsection{Weighted graphs}
Many practical scenarios are modelled by weighted graphs, where Gaussian distribution or Exponential distribution~\cite{spiegel2001probability} are often used to define the edge weights. Figures~\ref{fig:graph compare with with exponentially distributed weights} and \ref{fig:graph compare with Gaussian distributed weights} show how $p$ varies with the number of edges $M$ for the three considered random graph models with $N$=50 nodes.

%Indeed, many random variables approximately follow normal distribution, such as measurement error, temperature, blood pressure, and so on~\cite{heckert2002handbook}, while exponential distribution is commonly employed in the formation of models of lifetime distribution~\cite{balakrishnan2018exponential}. It can be used to represent the time interval of independent random events, such as the time interval of such as the time interval of customers entering the bank, the time interval of new entries (wikipedia) and so on. %(维基百科).

%In contrast to unweighted graphs, the Laplacian matrix of the weighted graph is formulated as follows.
 
%\begin{myDef}{\label{Def: Laplacian matrix of weight graph}}[Laplacian matrix of the weighted graph]
%	$L=D-W$, where $W$ is the weighted adjacency matrix representing the weights of the edges, i.e., $W=(w)_{i,j=1}^N$ and $D=diag(D_1, D_2, \cdots, D_N)$ with $D_i={\sum_{j=1}^N}w_{i,j}$ \cite{poignard2018spectra}.
%\end{myDef}

%\subsubsection{the general law of shift-enabled probability of weighted graph}

\begin{figure}[htb]
	\centering
	\subfigure[]{
    		\includegraphics[width=1.5in]{ERExp50nodesLamda}}
	\subfigure[]{
		\includegraphics[width=1.5in]{WSExp50NodesBeta5}}
	\subfigure[]{
		\includegraphics[width=1.5in]{BAExp50nodesm010}}
	\caption{The shift-enabled probability $p$ of weighted graphs with exponentially distributed weights with $\lambda=10$. (a){ ER graph; (b) WS graph;} (C) BA graph.}
	\label{fig:graph compare with with exponentially distributed weights}
\end{figure}


\begin{figure}[htb]
	\centering
	\subfigure[]{
		\includegraphics[width=1.5in]{ERGaussian50nodes}}
	\subfigure[]{
		\includegraphics[width=1.5in]{WSGaussianNode50Fixbeta}}
	\subfigure[]{
		\includegraphics[width=1.5in]{BAGau50nodesm010}}
	\caption{The shift-enabled probability $p$ of weighted graphs with Gaussian distributed weights with $\mu=100$ and $\sigma=10$. {(a) ER graph; (b) WS graph;} (C) BA graph.}
	\label{fig:graph compare with Gaussian distributed weights}
\end{figure}

\begin{Thm}{\label{Thm:general law_weighted_graph}}
%As Fig. {\ref{fig:graph compare with with exponentially distributed weights}} and Fig. {\ref{fig:graph compare with Gaussian distributed weights}} shown, the probability of shift-enabled have the following laws.
Theorem \ref{Thm_p_tends_to_zero}, Corollary \ref{Cor:non shift for small edges} and Theorem~\ref{Thm: simple spectrum} still hold for weighted graphs. Therefore, the probability $p$ of the graph being shift-enabled has the following properties:
\begin{itemize}
	\item {Region 1.} The probability $p$ is $0$, when the number of edges $M$ is relatively small which in accordance with Section~\ref{sec: Region 1} for an unweighted graph.
	\item {Region 2.}
	The probability $p$ is close to 1 when the number of edges increases which is in accordance to Section~\ref{sec: Region 2}.
	\end{itemize}
\end{Thm}

In the last region, $p$ remains close $1$, which is different from the shift-enabled probability behaviour of the unweighted graph. This is due to the following theorem.
%\begin{itemize}
\begin{Thm} {\label{Thm: weighted graph with large edges}}
In the weighted graph,  the probability $p$ is close to $1$ and does not drop to zero for a large number of edges $M$.
\end{Thm}
%	\end{itemize}
\begin{proof}
The results in Section~\ref{sec: Region 3} do no longer hold for the weighted graph since Theorem~\ref{Thm_complement_shift} does not hold. Furthermore, the complements of weighted graph are generally connected as the original graph is connected (Complement of undirected weighted graph are defined in detail in Section 2.7.1 of Ref.~\cite{networks}), 
which implies that Theorem~\ref{Thm: simple spectrum} still holds for the complement graph. So, the probability $p$ remains close to $1$.
% Theorem~\ref{Thm  simple spectrum} holds for
\end{proof}
%\subsubsection{the probability of the complement  and original graph have the same shift-enable}
%\begin{Rem}
%	Note that, for a weighted BA graph the probability is almost equal to 1, since BA graph is connected. %Specifically, the total number of edge $|E|=(m_0-1)+(N-m_0)\times m$, and as $m = 1$, it takes the minimum value $N-1$.
%\end{Rem}

%\subsubsection{The effect of the parameter on exponential weighted ER graph}
Exponential distribution function $f(x)=\lambda e^{-\lambda x},(x>0)$ has one parameter $\lambda$. As Figure \ref{fig:erexpnode20mu100lap} shows for $\lambda=0.1, 1, 10, 100$, and the parameter  $\lambda$ has almost no effect on the shift-enabled properties.
\begin{figure}[htb]
	\centering
	\includegraphics[width=0.5\linewidth]{ERExp50NodesChangeLamda}
	\caption{ER graphs with the weights defined using exponential distribution. The influence of parameter $\lambda$ on the shift-enabled probability $p$, for an ER weighted graph with $N$=50 nodes. The horizontal axis shows the number of edges $M$.
	}. 
	%\color{red} You probably want to increase the font size (legends and labels) in the figures. And I think it is better to keep all plots to have approximately the same sizes. And please check typo in graph labels (e.g., Everage -> Average). 
	\label{fig:erexpnode20mu100lap}
\end{figure}

%\subsubsection{The effect of parameter on Gaussian weighted ER graph}
In Gaussian distribution, the distribution function $ f(x)={\frac {1}{\sigma {\sqrt {2\pi }}}}e^{-{\frac {1}{2}}\left({\frac {x-\mu }{\sigma }}\right)^{2}} $, in which $\mu$ and $\sigma>0$ are mean and standard deviation of random variable. The weight may be negative, and the graphs with negative weights are called sign graphs which is discussed in Section~\ref{section: signed graph}. As Figure \ref{fig:ergaussiannode20sigama10lap} shows, (for positive weights) the parameters of Gaussian distribution have almost no effect on the shift-enabled properties.
\begin{figure}[htb]
	\centering
	\subfigure[]{
	\includegraphics[width=0.48\linewidth]{ERGaussian50NodesChangeMu}}
	\centering
	\subfigure[]{
	\includegraphics[width=0.48\linewidth]{ERGaussian50NodesChangesigama}}
	\caption{ER graphs with the weights defined using Gaussian distribution. Probability $p$ vs the number of edges $M$ for three different values of (a) $\mu$ (b) $\sigma$.}
	\label{fig:ergaussiannode20sigama10lap}
\end{figure}

\subsection{Signed Graphs}{\label{section: signed graph}}
Up to now, only graphs with unweighted or positively weighted edges are discussed. Some applications, however, involve signed graphs in which case negative weights are admitted.
In these cases, the negative edges are usually used to measure the dissimilarity between nodes. 

Signed graphs are widely used in social networks, in which case person support/oppose each other, recommendation networks (likes/dislikes), and biological networks (promote and inhibit relationships between neurons)~\cite{cheung2018robust,parisien2008solving,Dittrich20200signedgraph}. This article studies the most simplest signed graphs with weights equal to $1$ or $-1$ introduced by Harary~\cite{harary1953notion} in 1953, to deal with social relations including  disliking, indifference, and liking. The results from the previous section, can easily be extended to signed graphs.

The Laplacian matrix of signed graphs is defined as $L_{sign}=D_{sign}-W$, where $D_{sign}=\sum_j|w_{ij}|$, i.e., the sum of the absolute weights of incident edges~\cite{Dittrich20200signedgraph,liu2020generalized}. 
Signed graphs have two categories: balanced and unbalanced graphs. %Balance and unbalance signed graph have the different shift-enabled properties. Next, the two cases are discussed separately.
A signed graph is balanced if the product of edge weights around each cycle is positive~\cite{harary1953notion}. Otherwise, the graph is unbalanced. 
%Whether a graph is balance can be determined by the following theorems.
% \textcolor{red}{Is the process of creating an unbalance and balance graphs necessary?
% Theorem \ref{Thm:balance _graph_condition} is ready for generating balance graph?}
% \begin{Thm}\label{Thm:balance _graph_condition}
% 	A connected graph $\mathcal{G}$ with nonzero signed edge weights is a signed graph. Then the following conditions are equivalent~\cite{zaslavsky1982signed,hou2003laplacian,Dittrich20200signedgraph}:
% 	\begin{itemize}
% 		\item [(1)] $\mathcal{G}$ is balanced. 
% 		\item [(2)] $L$ is positive definite, i.e., all the eigenvalues of $L$ are positive.
% 		\item [(3)] The vertices of $\mathcal{G}$ can divided into two subsets: $V_1$ and $V_2$, such that the edges in $V_1$ or $V_2$ are $+1$, whereas the edges between $V_1$ and $V_2$ are $-1$.
% 	\end{itemize}
% \end{Thm}
Balanced and unbalanced signed graphs have the distinct shift-enabled properties. Next, the two cases are considered separately.

For balanced graphs, Laplacian matrix $L_{sign}$ has the following property.
\begin{Thm}[\cite{kunegis2011spectral}]{\label{Thm balance_spectral}}
	The eigenvalues of the Laplacian matrices of a balanced graph $\mathcal{G}_{sign}$ and the corresponding unsigned graph $\mathcal{G}$ %\footnote{Let $w_{i,j}$ and $(w_{i,j})_{sign}$ denote the weight of $\mathcal{G}$ and $\mathcal{G}_{sign}$, respectively. Then $w_{i,j}=|(w_{i,j})_{sign}|$.}
		% If $L$ is the signed Laplacian matrix of the balanced graph $G$, the Laplacian matrix $\tilde{L}$ of $\tilde{\mathcal{G}}$ of the corresponding unsigned graph $\tilde{\mathcal{G}}$ of $G$, 
	are identical. 
\end{Thm}

% For balance graph, its spectrum of the Laplacian matrix $\tilde{L}$ is equivalent to that of the corresponding unsigned graph $L$~\cite{kunegis2011spectral}.
Based on this theorem it is easy to show the following shift-enabled properties the signed graphs.

\begin{Thm}
  Balanced graphs and unbalanced graphs 
  have the same shift-enabled properties as the corresponding unsigned graphs and general random weighted graphs, respectively.
\end{Thm}
\begin{proof}
First of all, based on Lemma~\ref{Lem_undirected_shift} and  Theorem~\ref{Thm balance_spectral}, it is obvious that a balanced signed graph has the same shift-enabled properties as the corresponding unsigned graph. In the considered case, the signed graph has weights $1$ and $-1$. So, its shift-enabled properties are in accordance with the corresponding unweighted graph (see Figure~\ref{fig:balance_unbalance}(a) for a simulation example).

For unbalanced graphs, it is easy to show that Theorem~\ref{Thm:general law_weighted_graph} and Theorem~\ref{Thm: weighted graph with large edges} still hold, so its shift-enabled properties are in accordance with those of random weighted graphs (see Figure~\ref{fig:balance_unbalance}(b)).
\end{proof}

\begin{Rem}
Random balanced graphs and unbalanced graphs are generated differently. 
A usual way to generate a random unbalanced graph, is to start from  an ER graph, and then randomly replace all weights that are 1 by -1, i.e., to convert the unsigned ER graph into a signed. Because balanced graphs require that the product of edge weights around each cycle is positive, which is difficult to satisfy when the number of edges is high, this method is not practical for generating balanced graphs. 

% Note that the graph $\mathcal{G}$ is balanced if all the eigenvalues of Laplacian matrix $L$ are positive by Lemma~\ref{Lem:balance_graph_condition}. Otherwise, $\mathcal{G}$ is unbalanced.

 According to Lemma~\ref{Lem:balance_graph_condition}(3), a balanced signed graph can be generated in the following way. Firstly, divide randomly the vertices into two sets $V_1$ and $V_2$; then the weights of edges connecting two vertices that are both in $V_1$ or both in $V_2$ are set to $1$; edges connecting one vertex from $V_1$ with a vertex to $V_2$ are set to $-1$. 
\end{Rem}
%According to Theorem~\ref{Thm balance _graph_condition}.(3), we generate balance graph by the following steps. Firstly, divide the vertices into  two parts $V_1$ and $V_2$ randomly, then the weights of edges whose vertices are within $V_1$ or $V_2$ are $1$, and whose vertices between $V_1$ and $V_2$ are $-1$. 
%
%The first thing, we need construct random signed graphs. Firstly, we generate ER graph, then randomly replace the weight 1 by -1, i,e, we convert unsigned graph into signed. Then the graph $\mathcal{G}$ is balance as all the eigenvalues of laplacian matrix $L$ are positive by Theorem~\ref{Thm balance _graph_condition}.(2). Otherwise, $\mathcal{G}$ is unbalance. 
%
%As the edges increasing, the probability of shift-enabled trends to zero for balance graph  since as the edges increasing the random graph are almost unbalance graph.
%This strongly suggests that balanced graphs are much rarer. The result seems obvious and even trivial as a balanced graph requires each cycle to be "balanced". And for a single cycle (or cycle graph), the probability of balance or imbalance should be about $1/2$. For non-cycle graphs with many cycles, the probability of being balanced should be approximately $(1/2)^k$ where $k$ is the number of cycles and thus decrease exponentially with the number of cycles.

%According to Theorem~\ref{Thm balance _graph_condition}.(3), we generate balance graph by the following steps. Firstly, divide the vertices into  two parts $V_1$ and $V_2$ randomly, then the weights of edges whose vertices are within $V_1$ or $V_2$ are $1$, and whose vertices between $V_1$ and $V_2$ are $-1$. 
%In contrast, for the unbalance graph, there is no correlation between its eigenvalues of Laplacian matrix and $L$. As the Figure shows that it can be regarded as a general weighted graph with weight $1$ and $-1$, and its  changing trend of shift probability is consistent with the weighted graph.

\begin{figure}[htb]
	\centering
	\subfigure[]{
		\includegraphics[width=1.5in]{Balance50Nodes}}
	\subfigure[]{
		\includegraphics[width=1.5in]{unBalance50Nodes}}
	\caption{The shift-enabled properties of signed  graph.
	Balanced graph has the same shift-enabled properties as the corresponding unsigned  graph.
	Unbalanced graph`s shift-enabled properties are in accordance with those of general weighted  graphs. $p$ vs. $M$, with $N$=50 nodes, for a signed
		 (a) Balanced; (b) Unbalanced  graph.}
	\label{fig:balance_unbalance}
\end{figure}

% \section{Discussion}{\label{sec:discussion}}
% % \begin{figure}[htb]
% % 	\centering
% % 	\subfigure[]{
% % 		\includegraphics[width=2in]{ERLaplacianRandom50Vertices}}
% % 	\centering
% % 	\subfigure[]{
% % 		\includegraphics[width=2in]{fixbeta0.5Node50}}
% % 	\subfigure[]{
% % 		\includegraphics[width=2in]{BANode50M025}}
% % 	\caption{The shift-enabled probability of the unweighted ER graph with 50 vertices. 
% % 	%They have the similar patterns of change.
% % 	%	(a) ER graph.
% % 	%	(b) WS graph (x-axis denotes the average degree of the nodes $K$. The number of edges is $M=N\dot K$). (c) BA graph (x-axis is the number of edges for each node addition $m$; the number of edges is then $M=m\dot(N-m_0$), where $m_0$ is the initial number of nodes.
% % 	}
% % 	\label{fig:compare the shift-enabled probability of different graphs}
% % \end{figure}

% \begin{Rem}
% 		Commonly used shift matrices of $\mathcal{G}$ are: adjacency matrix $A$, combinatorial Laplacian matrix $L$, normalized Laplacian matrix $\mathcal{L}$, signless Laplacian matrix $|L|$ and probability transition matrix $T$ which have similar shift-enabled property rules (see Figure~\ref{fig:different shift matrix} for an illustration).
% 	\begin{figure}[htb]
% 		\centering
% 		\includegraphics[width=1\linewidth]{different_shift_matrix_ER}
% 		\caption{Adjacency matrix $A$, Laplacian matrix $L$, normal Laplacian matrix $\mathcal{L}$,  signless Laplacian matrix $|L|$ and probability transition matrix $T$ have similar shift-enabled change rules.}
% 		 	\label{fig:different shift matrix}
% 	\end{figure}
% \end{Rem}

\section{Conclusion}
\label{sec:discussion}
Recognising the importance of shift-enable property, the paper discusses when a random graph is shift-enabled. The behaviour of the shift-enabled property for weighted and unweighted graphs was discussed and the the design guidelines that ensure the shift-enable property were given. 
A future direction is to extend the findings to directed graphs and how to transform shift matrix when the shift-enabled condition for $S$ is not satisfied.
%rules and structures that may be used to identify the shift-enabled graphs.
It is also interesting to study if one may decompose non-shift-enabled graphs into shift-enabled subgraphs, to optimize the design of the GSP filters.

\appendices


\section{}
%Appendix two text goes here.
It is easily determined whether a graph is shift-enabled by the following lemma.

\begin{Lem}{\label{Lem_undirected_shift}}
	If shift matrix $L$ is a real symmetric matrix, then $L$ is shift-enabled, if and only if all eigenvalues of $L$ are distinct {\rm\cite{ortega2018graph,Liyanchen2021}}.
\end{Lem}

Lemma~\ref{Lem_undirected_shift} indicates that an undirected graph, which shift matrix is real symmetric, is shift-enabled if and only if its eigenvalues are all distinct.
% \begin{Lem}{\label{Lem_shift_enable}}{\rm{\cite{Liyanchen2018}}}
%	For an undirected graph,  $\mathcal{G}$ is shift-enabled if and only if all eigenvalues in $L_{\mathcal{G}}$ are distinct.
%\end{Lem}
\begin{Lem}{\label{Lem_spec_complement}}{\rm{\cite{Merris1994Laplacian}}}
	Assume 
	\begin{equation*}
	Spec(L_{\mathcal{G}})=(\lambda_1, \lambda_2, \cdots, \lambda_{N-1},\lambda_N=0),
	\end{equation*}
	%where we assume the eigenvalues to be arranged in nonincreasing order:  $\lambda_1 \geq \lambda_2  \geq \cdots \geq \lambda_n=0$,
	then
	\begin{equation*}
	Spec(L_{\mathcal{G^C}})=(N-\lambda_1, N-\lambda_2,  \cdots,N-\lambda_{N-1}, 0).
	\end{equation*} 
\end{Lem}
\begin{proof}
Since $\lambda_i \neq 0$ for $i=1,\cdots, N-1$,
$u=[1,1,\cdots,1]^T$ is the only eigenvector of $L_{\mathcal{G}}$ with eigenvalue $0$. Let $x_1,x_2,\cdots,x_{N-1}$ be the remaining eigenvectors of $L_{\mathcal{G}}$. As $L_{\mathcal{G}}$ is symmetric and $\lambda_N \neq \lambda_i$ for $i\neq N$, $u$ is orthogonal to $x_1,x_2,\cdots,x_{N-1}$.

	Let $L_{K_N}$ be the Laplacian matrix of a complete graph with $N$ vertices. It is easily seen that
	\begin{equation}{\label{Equ_Laplacain_complement}}
	L_{\mathcal{G}}+L_{\mathcal{G}^C}=L_{K_n}=NI_{N}-J_{N}, 
	\end{equation}
	where $I_N$ and $J_{N}$ are identity matrix and all-1s matrix with $N$ vertices, respectively. 
%	{\color{red} It is known that 0 is an eigenvalue of $L_{K_N}$ and $u=[1,1,\cdots,1]^T$ is one of the corresponding eigenvectors. }



%Assume the remaining $N-1$ eigenvectors are $x_i (i=1,2,\cdots,N-1)$, then $x_i$  is orthogonal to $u$ if $L_{\mathcal{G}}$ is symmetric. 
% {\color{red} Then, $u$ and $x_i (i=1,2,\cdots,N-1)$ are linearly independent eigenvectors of $L_{K_N}$ which span the entire $N$ dimensional space.
% 	THIS IS IRRELEVANT AS WELL - sam} {\color{blue} I revised this paragraph and the next.-liyan}
	
Therefore, according to Equation (\ref{Equ_Laplacain_complement}), 
% 	$L_{\mathcal{G}^C}=L_{K_N}-L_{\mathcal{G}}=NI_N-J_N-L_{\mathcal{G}}$. Note that, $J_Nx=0$ for $x_i$ orthogonal to $u$. Then,
	\begin{equation*}
	L_{\mathcal{G}^C}x_i=(NI_N-J_N-L_{\mathcal{G}})x_i=Nx_i-J_Nx_i-{\lambda}_i x_i=(N-{\lambda}_i)x_i,
	\end{equation*} 
	where $J_N x_i = 0$ as $x_i$ is orthogonal to $u$. Thus,  
	$(N-{\lambda}_i) \in Spec(L_{\mathcal{G}^C})$ for $i=1,\cdots, N-1$.
	
		Finally, $0 \in Spec(L_{\mathcal{G^C}})$ as $u=[1,1,\cdots,1]^T$ is still an eigenvector of $L_{\mathcal{G}^C}$ with eigenvalue $0$. 
% 	$L_{\mathcal{G}^C}u=L_{\mathcal{G}}u=L_{K_N}u=0u$, thus 
% 	$0\in Spec(L_{\mathcal{G}^C})$.
% 	{\color{red}Assume the remaining $N-1$ eigenvectors are $x_i (i=1,2,\cdots,N-1)$, then $x_i$  is orthogonal to $u$ if $L_{\mathcal{G}}$ is symmetric.}
% 	, and 
Hence,
	$Spec(L_{\mathcal{G^C}})=(N-\lambda_1, N-\lambda_2,  \cdots,N-\lambda_{N-1} , 0)$.
\end{proof}
\begin{Lem}{\label{Lem_complement_connect}}{\rm{\cite{Merris1994Laplacian}}}
	If $N$ is not the eigenvalue of $L_{\mathcal{G}}$, %is not the eigenvalue of $\mathcal{G}^C$, 
	then $\mathcal{G}^C$ is a connected graph.
\end{Lem}

\begin{Lem}\label{Lem:balance_graph_condition}
    %Let $\mathcal{G}$ is a connected signed graph 
	%
	Let $\mathcal{G}$ be a connected signed graph, i.e., a graph whose non-zero edge weights can take positive and negative values. 
	%Then, $\mathcal{G}$ is balanced if and only if $L$ is positive definite, i.e., all the eigenvalues of $L$ are positive.
	The following conditions are equivalent~\cite{zaslavsky1982signed,hou2003laplacian,Dittrich20200signedgraph}:
	\begin{itemize}
		\item [(1)] $\mathcal{G}$ is balanced. 
		\item [(2)] $L$ is positive definite, i.e., all the eigenvalues of $L$ are positive.
		\item [(3)] The vertices of $\mathcal{G}$ can be divided into two subsets: $V_1$ and $V_2$, such that all edges connecting vertices that are both in $V_1$ and all edges connecting the vertices that are both in $V_2$ are $+1$, whereas the edges connecting a vertex in $V_1$ with a vertex in $V_2$ are $-1$.
	\end{itemize}
\end{Lem}


% use section* for acknowledgment
\section*{Acknowledgment}
The authors thank Bochao Zhao and Kanghang He for helpful discussions.
This research was supported by the European Union's Horizon 2020 research and innovation programme under the Marie Sklodowska-Curie grant agreement  number 734331, the National Key Research and Development Project under grant 2017YFE0119300 and general scientific research projects of Zhejiang Education Department in 2020 under grant Y202044676.

% Can use something like this to put references on a page
% by themselves when using endfloat and the captionsoff option.
\ifCLASSOPTIONcaptionsoff
  \newpage
\fi



% trigger a \newpage just before the given reference
% number - used to balance the columns on the last page
% adjust value as needed - may need to be readjusted if
% the document is modified later
%\IEEEtriggeratref{8}
% The "triggered" command can be changed if desired:
%\IEEEtriggercmd{\enlargethispage{-5in}}

% references section

% can use a bibliography generated by BibTeX as a .bbl file
% BibTeX documentation can be easily obtained at:
% http://mirror.ctan.org/biblio/bibtex/contrib/doc/
% The IEEEtran BibTeX style support page is at:
% http://www.michaelshell.org/tex/ieeetran/bibtex/

\bibliographystyle{IEEEtran}
 %argument is your BibTeX string definitions and bibliography database(s)
\bibliography{reference}
%
% <OR> manually copy in the resultant .bbl file
% set second argument of \begin to the number of references
% (used to reserve space for the reference number labels box)
%\begin{thebibliography}{1}
%%
%\bibitem{IEEEhowto:kopka}
%H.~Kopka and P.~W. Daly, \emph{A Guide to \LaTeX}, 3rd~ed.\hskip 1em plus
%  0.5em minus 0.4em\relax Harlow, England: Addison-Wesley, 1999.
%
%\end{thebibliography}
%
%% biography section
%% 
%% If you have an EPS/PDF photo (graphicx package needed) extra braces are
%% needed around the contents of the optional argument to biography to prevent
%% the LaTeX parser from getting confused when it sees the complicated
%% \includegraphics command within an optional argument. (You could create
%% your own custom macro containing the \includegraphics command to make things
%% simpler here.)
%%\begin{IEEEbiography}[{\includegraphics[width=1in,height=1.25in,clip,keepaspectratio]{mshell}}]{Michael Shell}
%% or if you just want to reserve a space for a photo:
%
%\begin{IEEEbiography}{Michael Shell}
%Biography text here.
%\end{IEEEbiography}
%
%% if you will not have a photo at all:
%\begin{IEEEbiographynophoto}{John Doe}
%Biography text here.
%\end{IEEEbiographynophoto}
%
%% insert where needed to balance the two columns on the last page with
%% biographies
%%\newpage
%
%\begin{IEEEbiographynophoto}{Jane Doe}
%Biography text here.
%\end{IEEEbiographynophoto}

% You can push biographies down or up by placing
% a \vfill before or after them. The appropriate
% use of \vfill depends on what kind of text is
% on the last page and whether or not the columns
% are being equalized.

%\vfill

% Can be used to pull up biographies so that the bottom of the last one
% is flush with the other column.
%\enlargethispage{-5in}



% that's all folks


\end{document}

\begin{Rem}
As opposed to unweighted graphs, the complement of a weighted graph has a more complicated structure.
\begin{myDef}[The complement of edge-weighted graph] {\label{Def:complement of edge weighted graph}}
	Let  $\mathcal{G}=(V,E,W)$ be a weighted graph. Then $\mathcal{G}^C=(V,E^C,W^C)$ is the complement of $\mathcal{G}$, in which the weight of edges is defined as follows.
	\begin{equation*}
	%E_1^C =\{(\{u,v\},1)\lvert u\in V, v\in V \wedge (\{u,v\},1)\notin E\}.  
	E_1^C =\{(\{u,v\},w=1)\lvert u\in V, v\in V \wedge u\neq v \wedge \{u,v\} \notin E\}.  
	\end{equation*}
	
	\begin{equation*}
	E_2^C =\{(\{u,v\},w^{-1})\lvert u\in V, v\in V \wedge (\{u,v\},w)\in E\wedge w\neq 1 \}. 
	\end{equation*}
	$E^C=E_1^C \cup E_2^C$.
\end{myDef}
(cite by http://www.xatlantis.ch/index.php/education/zeus-framework/15-graph-theory)
\end{Rem}

Figure~\ref{fig:complementweightedgraph} is an example of complement of a weighted graph.
\begin{figure}[htb]
	\centering
	\includegraphics[width=0.7\linewidth]{complementWeightedGraph}
	\caption{An illustrative example of complement of weighted graph. The left  is the original graph, and the right is complement of left.}
	\label{fig:complementweightedgraph}
\end{figure}


