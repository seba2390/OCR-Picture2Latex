%\documentclass[prl,floatfix,twocolumn,showpacs]{revtex4}
\documentclass[pre,floatfix,twocolumn,showpacs,a4paper,nofootinbib]{revtex4}
%\topmargin=.1in
\usepackage{amsmath}
\usepackage{amsfonts}
\usepackage{amssymb}
\usepackage{calrsfs}
\usepackage{mathrsfs}
\usepackage{epsfig}
\usepackage{graphicx}
\usepackage{dcolumn}
\usepackage{color}
%\usepackage{url}
%\usepackage[bookmarks=false]{hyperref}


\begin {document}

\title
{Characterization of the community structure in a large-scale production network in Japan}
\author
{
Abhijit Chakraborty, Hazem Krichene, Hiroyasu Inoue, Yoshi Fujiwara  
}
\affiliation
{
\begin {tabular}{c}
Graduate School of Simulation Studies, University of Hyogo, Kobe 650-0047, Japan 
\end{tabular}
}
\begin{abstract}
Inter-firm organizations, which play a driving role in the economy of a country, can be represented in the form of
a customer-supplier network. Such a network exhibits a  heavy-tailed degree distribution, disassortative mixing and 
a prominent community structure. We analyze a large-scale data set of customer-supplier relationships 
containing data from one million Japanese firms. Using a directed network framework,
we show that the production network exhibits the characteristics listed above.
We conduct detailed
investigations to characterize the communities in the network. The topology within smaller communities  
is found to be very close to a tree-like structure but becomes denser as the community size increases. 
A large fraction $(\sim 40\%)$ of firms with relatively small in- or out-degrees have customers or suppliers solely from within their own communities,
indicating interactions of a highly local nature. 
The interaction strengths between communities as measured by the inter-community link weights follow 
a highly heterogeneous distribution. We further present the statistically significant over-expressions
of different prefectures and sectors within different communities. 


\end{abstract}
\pacs {89.65.Gh, %Economics; econophysics, financial markets,business and management
       89.75.Hc, %Networks and genealogical trees
%       05.65.+b, %Self-organized systems
       89.65.-s  %Social and economic systems
       }

\maketitle

\section {Introduction}

At the mesoscopic level, many real-world networks exhibit community structures,
i.e., structures consisting of groups of nodes that are highly connected 
among themselves and sparsely linked with the other nodes in the network~\cite{girvan2002community,krause2003compartments}.  
These communities are the building blocks of the network and play pivotal roles in its functional activities or dynamic 
processes. For example, in metabolic networks, the communities represent functional groupings of metabolites based on their pathways~\cite{guimera2005functional},
whereas the community structures in air traffic networks represent actual travel patterns~\cite{rosvall2014memory}.
Community structure has diverse applications. It can be used as the basis of an efficient mirror server for achieving
better performance on the World Wide Web~\cite{krishnamurthy2000network}, in the design of powerful recommendation systems~\cite{reddy2002graph} and also to identify
the characteristic features of nodes in subnetworks.  

Over the past decade, extensive studies have been conducted on algorithms for detecting communities~\cite{fortunato2010community, newman2012communities} in which the
number of communities is not required to be fixed a priori, as in graph partitioning problems~\cite{kernighan1970efficient}. 
Many algorithms have been proposed for community detection, and two recent popular techniques are based on modularity optimization~\cite{newman2004finding, newman2004fast}
and the diffusion dynamics in a network~\cite{rosvall2007information,rosvall2008maps}. The modularity is the fraction of intra-community links minus the expected fraction 
given a random distribution. The Newman–Girvan community detection algorithm~\cite{newman2004finding} is a modified divisive clustering technique that 
finds the best partitioning scheme with the optimal modularity using the edge betweenness of the network as a metric.  
This algorithm is very effective in detecting communities, but it is numerically expensive and has limited applicability to small-scale networks. 
Another algorithm~\cite{clauset2004finding}, using an agglomerative hierarchical clustering approach with greedy optimization of the modularity,
has subsequently been proposed for detecting communities in a large-scale network with one million nodes and links. Whereas community detection via modularity 
maximization is based solely on the topology of the network, another popular technique for community detection known as 
``Infomap"~\cite{rosvall2007information,rosvall2008maps} uses the diffusion dynamics in the network. A simple random walk is used to model the diffusion process or
information flow in the network. It is observed that a random walker spends a longer
time within certain subunits of a network. These subunits are the communities of the network. 
Note that the definition of a community in this method is different from that in the modularity maximization method. This technique can capture more complex structures in the network
than can techniques based solely on the network topology. Initially, Infomap was proposed for identifying the two-level partitioning of a network that
minimizes the two-level map equation. Later, the two-level map equation was generalized to the hierarchical map equation, which provides a hierarchical partitioning of a
network~\cite{rosvall2011multilevel}. Whereas modularity-maximization-based techniques are more 
applicable to networks that contain links that represent pairwise relationships, Infomap is more suitable for networks with flows between
nodes~\cite{rosvall2008maps}. 
%---------------------------------------------------------------------------
\begin{figure*}[t]
\begin{center}
\includegraphics[width=0.99\linewidth,clip]{Figure01}
\end{center}
\caption{(color online)
{\bf Structural properties of the production network.
Probability density distributions $P$ of (a) the nodal in-degrees $k_{in}$ and (b) the nodal out-degrees $k_{out}$.
Variation of (c) the clustering coefficient $C(k) $ as a function of degree $k$
and (d) the average nearest neighbor degree $\langle k_{nn}(k) \rangle$ as a function of degree $k$.}
Logarithmic binning of the horizontal axis is used in all plots. 
%The red line in each plot represents the best power-law fit to the data. 
}
\label{fig1}
\end{figure*}
%---------------------------------------------------------------------------

Although substantial efforts have been devoted to community detection, only a few studies~\cite{lancichinetti2010characterizing, tumminello2011community, tumminello2012identification, vitali2014community, marotta2015bank, marotta2016backbone} 
have characterized the community structure. Here, we characterize the communities in a large-scale production network with one million nodes and links.
Like many social and technological networks, a production network exhibits a prominent community structure~\cite{fujiwara2010large, iino2010community, iino2015community}. 
A production network represents the interactions between firms. It constitutes the backbone of an economy 
in the form of flows of goods and services. Each firm buys intermediate goods from upstream firms for its production, and 
thus, relationships between firms develop through customer-supplier connections.
%A complex network framework is very useful for the analysis of the customer-supplier relationships between firms.
%Firms are the nodes and customer-supplier relationship represents the links of the network. 
The typical topological characteristics of a production network include
a scale-free degree distribution, a small-world nature and disassortative mixing~\cite{fujiwara2010large, mizuno2014structure}. 
The structure of a production network plays an important role in the origin of  business cycles and aggregate fluctuations, as shown by  Acemoglu et al. considering the input-output network of sectors~\cite{acemoglu2012network}. They showed that the asymmetry characterizing the  degree distribution
of the network is the main cause of  large-scale fluctuations in economic systems. Although they found a scale-free nature of 
the degree distribution characterized by a power law in the input-output network of  U.S. industries, such scale-free nature can
disappear when one considers higher levels of coarse-grained input-output networks~\cite{cerina2015world}, which in turn can give an incorrect impression about 
the underlying network topology. In our work, we study the network structure at the firm level, which might be of further importance
in macroeconomic phenomena.
Previous studies on Japanese production networks 
based on modularity maximization have shown that the subcommunities are characterized by geographical regions and sectors~\cite{iino2010community, iino2015community}.
It has also been observed that directed and undirected versions of the same network yield similar results. 
Because the links in a production network represent the flow of goods and services from one firm to another, it is more appropriate to use the Infomap algorithm for community detection.
Moreover, a rigorous, statistically significant characterization of the communities in production networks remains to be conducted.

In this paper, we analyze nationwide inter-firm relationship data from Japan. 
This data set contains data representing one million firms and their customer-supplier relationships. We construct a directed network in which firms are 
represented by nodes and directed links are present from each supplier firm to its customers.
We calculate the standard structural metrics of the network to capture the global properties of the system.
%As the link in this network indicates flow of capital from one firm to another, we use Infomap algorithm to unmask community structure.
By employing the Infomap algorithm, we reveal the community structure of the production network. 
The community structure of the directed network of these firms is found to be distinctly different from that of its undirected counterpart.
The topological features within and among communities indicate a non-trivial local structure of the network. By defining a weighted
directed coarse-grained network in which the nodes represent the identified communities and the numbers of customer-supplier relationships
between communities are treated as link weights, we observe a high heterogeneity of the inter-community interactions. 
Furthermore, in this study, we propose a deeper characterization of the properties of the communities (prefectures and sectors) derived using a 
rigorous statistical procedure, as prescribed in~\cite{tumminello2011community}. 


\section {Data}   
\label{data}

Our data consist of $N_o = 1,247,521$ firms and $L_o = 5,488,484$ distinct links representing  customer-supplier relationships between firms throughout Japan 
for the year 2016. The data set is commercially available from Tokyo Shoko Research (TSR), Inc., one of the leading credit research agencies in Japan. 
TSR collects firms' credit information through investigations of financial statements and corporate documents as well as through 
oral surveys at branch offices located across the country. The data set contains the precise geographic locations and sectorwise classifications of the 
firms. 

We use a prefecture-based division of all regions to analyze our data. All $47$ prefectures 
are listed in Appendix A. 
To see whether the sectors to which different firms belong play any role in community formation, we use the Japan Standard Industrial Classification\footnote{\url{http://www.soumu.go.jp/toukei_toukatsu/index/seido/sangyo/}}, which
divides all the firms into 98 major groups. All 98 major sectors 
are listed in Appendix B. 




%---------------------------------------------------------------------------
\begin{figure}[t]
\begin{center}
\includegraphics[width=0.99\linewidth,clip]{Figure02}
\end{center}
\caption{
{\bf Distribution $n_x$ of the component sizes $x$ in the network.
The largest weakly connected component contains $\sim 99\%$ of the nodes in the entire network.}
}
\label{fig2}
\end{figure}
%---------------------------------------------------------------------------  

\section{Results}
We reveal the detailed topological features of the Japanese production network. To define the network,
each firm is treated as a node, and a directed link of the form $A \rightarrow B$ indicates that $A$ is a supplier firm
for firm $B$. The structure of the production network exhibits a number of empirical patterns, which we present in the following
subsections. 

\subsection{Structural properties of the production network}
The links in the production network are directed in nature, which allows us to characterize the nodes in terms of their in- and out-degrees.
The in-degree and out-degree of a node represent its numbers of suppliers and customers, respectively. We observe that both
the in-degree and out-degree distributions have a heavy-tailed nature, as shown in Fig.~\ref{fig1}~(a-b). 
%the best fits of these distributions to a power-law decay have a functional form of $P(k_{in/out})\sim k^{-\gamma_{in/out}}$, with $\gamma_{in} = 2.37$ and $\gamma_{out} = 2.21$. 
It is also observed that the in- and out-degrees of the nodes are positively
correlated. 
%Our results reflect the scale-free nature of the production network, with 
The heavy-tailed behavior of the degree distribution reflects the fact that production network has
a large number of firms with few suppliers
or customers and a smaller number of firms with many suppliers or customers.
Both the in-degree and out-degree show a positive and significant correlation with the firm size measured by using either its net 
sales or total numbers of employees~\cite{fujiwara2010large}.  Because large firms  generally have more suppliers and customers 
than small firms, the rich-get-richer principle is the key to such a power law nature of the degree distributions~\cite{barabasi1999emergence, chakraborty2010weighted}. Because of the high asymmetry in the degree 
distributions of the underlying network, aggregate fluctuations can appear in the system due to idiosyncratic shocks to large firms~\cite{acemoglu2012network}. 

We further investigate the cliqueness among neighbors and the mixing properties of the network considering undirected links. 
The cliqueness among the neighbors of a node is measured by the clustering coefficient, which is also a measure of the three-point
correlations among neighboring nodes. As seen from Fig.~\ref{fig1}~(c), the clustering coefficients $C(k)$ for 
the production network decay with $k$ 
%following a power law, with the form $C(k) \simeq k^{-\beta}$ with $\beta = 0.88$, 
implying the presence of a hierarchical structure in the network. The mixing patterns of the nodes are measured by the average degree $\langle k_{nn}(k) \rangle$
of an arbitrary neighbor of a node of degree $k$. Fig.~\ref{fig1}~(d) shows that $\langle k_{nn}(k) \rangle$ decreases 
%following a power law 
as $k$ increases,
%with the form  $\langle k_{nn}(k) \rangle \simeq k^{-\nu}$ with $ \nu = 0.58$, 
indicating a disassortative nature of the
network, i.e., nodes of higher degrees are connected to nodes of lower degrees. This finding reflects the fact that large firms are generally
connected with small firms. 

The component size distribution is another important property of the network. The component size is defined as the number of nodes 
in a subnetwork in which at least one path exists between any arbitrary pair of nodes.
It is observed that $99\%$ of the nodes are contained within the largest weakly connected component of the network, whereas the sizes of the other components are very small, as shown in Fig.~\ref{fig2}. 
More precisely, the largest weakly connected component contains  $ N = 1, 234, 687$ nodes and  $ L = 5, 481, 403$ links. 





%%---------------------------------------------------------------------------
%\begin{figure}[top]
%\begin{center}
%\includegraphics[width=0.99\linewidth,clip]{Pk_meta.eps}
%\end{center}
%\caption{(color online)
%{\bf Degree distributions $P(k)$ and strength distributions $P(s)$ of coarse the grained network.}
%}
%\label{fig4}
%\end{figure}
%---------------------------------------------------------------------------
%---------------------------------------------------------------------------
\begin{figure}[t]
\begin{center}
\includegraphics[width=0.99\linewidth,clip]{Figure03}
\end{center}
\caption{
{\bf Distributions $D(s)$ of community sizes $s$ for the empirical production network with directed links (filled circles) and for its randomized counterpart (open circles).
%(b) rank-size distribution of communities for undirected network of firms.
} 
Communities are detected in the largest weakly connected component of the network.
Logarithmic binning is used for the horizontal axis. 
}
\label{fig3}
\end{figure}
%---------------------------------------------------------------------------  
\subsection{Community structure of the production network}
Although the structural properties discussed above are important for obtaining a general understanding of the system, a complex network such as this one also exhibits a far more
complex pattern, namely, a community structure. We perform a community detection analysis on the largest weakly connected component of the production network, 
considering directed links. We use the hierarchical map equation~\cite{rosvall2011multilevel} to discover the community structure because this method is very efficient for
large-scale complex networks and is also applicable to directed networks. In this procedure, a random walker is used as a proxy for 
the flow of goods and services between firms in the production network, and the algorithm identifies the best hierarchical partitioning with the shortest 
average per-step description length (code length) for the random walker. 

The analysis of the empirical production network reveals $311$ communities and 
$8054$ inter-community links at the top community level with a code length of $12.925$. To determine the statistical significance of our result, we compare it with the result obtained from an identical 
analysis of a `null model', which is a random network with the same nodal degree values $\{k_{in}(i)\}$ and $\{k_{out}(i)\}$.
Following the the method of configuration models~\cite{newman2010networks}, we construct the randomization version of the empirical network using a pair-wise link exchange technique with 
the constraint of no multiple links between any pairs of nodes. 
In stark contrast with the empirical result, the randomized version of the empirical network is found to contain $ 62, 917 \pm 120$ communities and $4, 234, 729 \pm 1, 552$ inter-community connections with
a code length of $14.214 \pm 0.005$. The results are averaged over 10 uncorrelated random networks. 
As seen from Fig.~\ref{fig3}, the community size distributions for the empirical network and the randomized version of the empirical 
network are found to be distinctly different in nature. Whereas the empirical network has a broad distribution of community sizes with a wide range of values 
spanning several orders of magnitude, the randomized version of the empirical network has a comparatively narrow distribution, in which the largest community
size is $\sim 1, 000$. 
We further compare the community structure of the directed network of firms with that of its undirected counterpart. In the undirected network, we find
only $26$ communities with distinct and widely varying sizes (not shown), which is smaller than the number
of communities in the directed network. This difference in the community structure findings between the directed and undirected versions of the production network 
arises because Infomap allows the random walker to move only unidirectionally on directed links and permits two-way movement on undirected links.
A random walker on an undirected network can move from one sub-region  of the network to another sub-region  with a transition probability 
proportional to the total number of connected links between the two sub-regions. For the same movement on the directed version of the network, 
the transition probability is proportional to the total number of outgoing links from one region  to another.  Clearly, the 
transition probability for the movement between sub-regions is always higher on an undirected network than on its directed counterpart. 
For this reason, in many cases, distinctly different communities in directed networks can be merged with the same community in 
an undirected network. As a result, we find more communities of smaller size on directed networks than on the 
undirected version of the network. 
This is not a very special case with the production network.
The only distinct property of our network is the low fraction ($2.68 \%$) of the bi-directional links. 
The directed version of the production network shows different community structure from its undirected  counterpart because of a low fraction of bi-directional links. 


%---------------------------------------------------------------------------
\begin{figure}[t]
\begin{center}
\includegraphics[width=0.99\linewidth,clip]{Figure04}
\end{center}
\caption{
%(color online)
{\bf Scatter plot of the intra-community scaled link density $\bar{\rho}$ versus the community size $s$.}
The dotted lines represent two limiting cases: $\bar{\rho} = s$ (clique) and $\bar{\rho} = 1$ (tree).
}
\label{fig4}
\end{figure}
%---------------------------------------------------------------------------

\subsubsection{Topological features of communities in the production network}
We investigate the topological features within each community in the production network. The link density within a community is
the ratio of the number of internal links to the maximum possible number of links. The link density $\rho$ within a community of
size $s$ can be calculated as  $\rho = e/s(s-1)$, where $e$ is the number of links within the community. The scaled link density $\bar{\rho}$
within a community is defined as $\bar{\rho} = \rho s = e/(s-1)$. A value of $\bar{\rho} = 1$ corresponds to a community with a tree-like structure
with unidirectional links, and $\bar{\rho} = s$ corresponds to a complete graph structure, i.e., a structure in which every node is connected to all other nodes in the community. 
Fig.~\ref{fig4} shows a scatter plot of the scaled link density $\bar{\rho}$ versus the community size $s$. It is evident that the network structures within the
communities are far from being complete graph structures; indeed, they are very close to an ideal tree-like structure when the community size is small $(s < 1000)$.
However, beyond $(s > 1000)$, the scaled link density gradually increases as the community size increases. 

Next, we study the fraction of the neighbors of a node that belong to its own community. We find this fraction separately for the in-degree ($k_{in}^{intra} / k_{in}$) 
and out-degree ($k_{out}^{intra} / k_{out}$)  of the node. 
%In and out degrees of a node here indicate suppliers and customers of a firm respectively.
The probability distributions for the fraction of suppliers ($P(k_{in}^{intra} / k_{in})$) and the fraction of customers ($P(k_{out}^{intra} / k_{out})$) of a node
that belong to its community are shown in Fig.~\ref{fig5}. The two distributions are very similar in nature and have a maximum value of $~0.4$ at
$k_{in}^{intra} / k_{in} = k_{out}^{intra} / k_{out}=1 $, meaning that for $40\%$ of the firms, all suppliers and customers of a firm belong to own community. 
We further observe that these firms have very small ($<100$) in- or out-degrees.  
This result reflects the fact that for a large proportion of firms, their interactions are confined within a local region.  





%---------------------------------------------------------------------------
\begin{figure}[t]
\begin{center}
\includegraphics[width=0.99\linewidth,clip]{Figure05}
\end{center}
\caption{
%(color online)
{\bf Probability distributions $P(k_{in}^{intra} / k_{in})$ and  $P(k_{out}^{intra} / k_{out})$ for the fractions of in-neighbors ($k_{in}^{intra} / k_{in}$) and out-neighbors ($k_{out}^{intra} / k_{out}$) 
of a node that belong to the same community as that node.}
The circle and triangle symbols represent the in- and out-neighbor distributions, respectively. 
}
\label{fig5}
\end{figure}
%---------------------------------------------------------------------------
%---------------------------------------------------------------------------
\begin{figure}[t]
\begin{center}
\includegraphics[width=0.99\linewidth,clip]{Figure06}
\end{center}
\caption{(color online)
{\bf The complementary cumulative distribution function $P(w)$ of the inter-community link weights $w$ for the production network.}
A power-law fit to the data (red line) using the maximum likelihood estimation technique yields $P(w)\sim w^{-\gamma_w +1}$ 
with $\gamma_w =2.00 \pm 0.10$, $w_{min}=642 \pm 301$, and $p~value = 0.07$.
}
\label{fig6}
\end{figure}
%--------------------------------------------------------------------------- 


%------------------------------------------------------------------------------------------------------
\begin{table*}[t] 
\begin {center}
 \scriptsize
\begin{tabular}{|l|l|p{7.5cm}|p{7.5cm}|} \hline
  Rank   & Size      & Over-expression of sectors            &  Over-expression of prefectures  \\ \hline
  1      & 233, 294   & Manufacturing, Electronics, Water transport, Wholesale, etc.   &  Urban prefectures (Tokyo and its neighboring prefectures, Osaka, Aichi, Hyogo, etc.)          \\ 
  2      & 139, 380   & Agriculture, Food, Fisheries, Road freight transport, Cooperative associations, N.E.C., etc.        & Rural prefectures (Aomori, Miyagi, Shizuoka, etc.)             \\
  3      &  59, 906   & Construction, Real estate, Banking, etc.      &  Tokyo and its neighboring prefectures, Osaka          \\
  4      &  47, 849   & Manufacture of textiles, rubber, leather, etc.        &  Tokyo, Osaka, Kyoto, Aichi, etc.           \\ 
  5      &  44, 349   & Medical services, Research institutes, Chemical products, etc.     &  Hokkaido, Tokyo, Hiroshima, etc.     \\                  
  6      &  43, 397   & Retail trade (machinery and equipment), Automobile  maintenance, Transport, Insurance institutions, etc.       & Many (22) prefectures            \\                
  7      &  43, 018   & Multiple sectors        & Hokkaido              \\         
  8      &  38, 819   & Multiple sectors       & Tokyo               \\ 
  9      &  33, 654   & Information services and many others   & Tokyo, Kanagawa, Osaka            \\
  10     &  33, 563   & Construction and others         & Gifu, Aichi, Mie                         \\ \hline
     \end{tabular}
\caption{Brief summary of our results on the over-expression of sectors and prefectures in the ten largest communities}
\label{table1}
\end {center}
\end {table*}
%---------------------------------------------------------------------------------
%---------------------------------------------------------------------------
%\begin{figure}[top]
%\begin{center}
%\includegraphics[width=0.99\linewidth,clip]{Dominant_sec_geo.eps}
%\end{center}
%\caption{
%{\bf Scatter plot between dominant prefecture share and dominant sector share in the communities of the network.}
%Symbol sizes are proportional to the logarithmic of community sizes. 
%}
%\label{fig6}
%\end{figure}
%--------------------------------------------------------------------------- 

%Fig.~\ref{fig5} shows the dominance of geography and sector share over all communities having size more than 10. 
%The size of the circle indicates size of the communities in logarithmic scale. 
%The position of the most of the communities below the diagonal implying a clear geography dominance over the sector share dominance. 
%---------------------------------------------------------------------------
%\begin{figure}[top]
%\begin{center}
%\includegraphics[width=0.99\linewidth,clip]{Demply.eps}
%\end{center}
%\caption{(color online)
%{\bf Distributions $D(e)$ of number of employees e for a firm  in the network (black circle) and for communities in the coarse grained network (red circle).}
%}
%\label{fig7}
%\end{figure}

% The coarse grained version of the production network, in which one considers communities as nodes, is very sparse in nature as the 
%number of inter-modular links and communities are of the same order.
 

An inter-community link indicates that at least one customer-supplier relationship exists between
the members of the two linked communities. These links are directed, and the corresponding in- and out-degree distributions exhibit very different behaviors compared with the original
network. We also investigate the link weights of the inter-community links, where a link weight is defined based on the total number of customer-supplier relationships between
members of the two linked communities. Fig.~\ref{fig6} shows the complementary cumulative distribution function $P(w)$ of the inter-community link weights $w$, which is heavy tailed in nature.
However, A power-law fit to the data using the maximum likelihood estimation technique~\cite{clauset2009power} yields $P(w)\sim w^{-\gamma_w +1}$ 
with $\gamma_w =2.00 \pm 0.10$, $w_{min}=642 \pm 301$, and $p~value = 0.07$. The low $p~value$ implies that the distribution is not a power law.   
%a power-law decay of the functional form $P(w)\sim w^{-\gamma_w+1}$ with $\gamma_w =2.00 \pm 0.10$ 
%is obtained for this distribution via maximum likelihood estimation~\cite{clauset2009power}. 
The distribution of the inter-community link weights indicates highly heterogeneous relationships between the communities in the production network.  


\subsubsection{Over-expression of node attributes in the production network}

Most real-world networks are characterized by a community structure.  However, the factors that influence 
such clustering of nodes are different for different systems. For example,  the community
structure of a social network for mobile phone communications shows that people speaking the same 
language belong to the same community~\cite{blondel2008fast},  biological functions play a key role in forming 
communities of proteins in yeast~\cite{chen2006detecting}, and in stock markets, communities present similarities in the 
economic sectors of the stocks~\cite{onnela2003dynamics}. The communities in a network are mainly formed following
the principle of homophily. We use the over-expression analysis to characterize the communities. 
The study helps one to understand the basis of forming communities that have particular attributes.
We consider geographical and sector attributes; however, in general, one can take any attributes.  



%---------------------------------------------------------------------------
\begin{figure}[t]
\begin{center}
\includegraphics[width=0.99\linewidth,clip]{Figure07}
\end{center}
\caption{
{\bf The location and sectorial attributes of firms.
(a) The fraction of firms belonging to each prefecture, $F_{g_i}$, is plotted versus the corresponding index $g_i$.
(b) The fraction of firms belonging to each sector, $F_{s_i}$, is plotted versus the corresponding index $s_i$.}
The location indices $g_i$ and the sector indices $s_i$ are defined in Appendix A and Appendix B, respectively. 
}
\label{fig7}
\end{figure}
%--------------------------------------------------------------------------- 

%---------------------------------------------------------------------------
\begin{figure}[t]
\begin{center}
\includegraphics[width=0.99\linewidth,clip]{Figure08}
\end{center}
\caption{
{\bf Statistically significant over-expression of (a) prefectures and (b) sectors in individual communities, represented by black square symbols.}
The vertical axis represents the size-wise rank $R_i$ of the $i$-th community.}
\label{fig8}
\end{figure}
%---------------------------------------------------------------------------
We investigate the over-expression of different prefectures or sectors within a community following the procedure used in~\cite{tumminello2011community}.
The probability that $X$ randomly selected elements from a cluster $C$ of size $N_C$ will have attribute $Q$ is measured by the hypergeometric distribution
$$H(X|N, N_C, N_Q)= \frac{\binom{N_C}{X}\binom{N-N_C}{N_Q-X}}{\binom{N}{N_Q}}$$,
where $N_Q$ represents the total number of elements with attribute $Q$ in the system.
If $N_{C,Q}$ is the number of elements in cluster $C$ with attribute $Q$, then one can calculate an associated 
{\em p value:} $$p(N_{C,Q}) = 1- \sum_{X=0}^{N_{C,Q}-1}H(X|N, N_C, N_Q)$$ 
The attribute $Q$ is over-expressed within a community if $p(N_{C,Q})$ is smaller than some threshold value $p_c$.
We must choose $p_c$ appropriately to exclude false positives since we are considering a multiple-hypothesis test.
We set $p_c=0.01/N_A$, as used in~\cite{tumminello2011community}, which is sufficient for Bonferroni correction~\cite{miller1981normal}.  

In our study, $N_A$ denotes the total number of possible attributes and takes on values of $47$ and $98$ for the prefecture- and sector-based attributes, respectively.
Fig.~\ref{fig7}~(a) shows the fraction of firms belonging to each prefecture, $F_{g_i}$. The largest proportion of the firms $(15.0\%)$
belong to Tokyo, the capital of Japan.  
Fig.~\ref{fig7}~(b) shows the fraction of firms belonging to each sector, $F_{s_i}$. 
The largest proportion of the firms $(13.2\%)$ belong to the general construction work sector, which includes both public and private construction work.   
The over-expression of prefectures and sectors within individual communities consisting of at least 10 firms is visualized in Fig.~\ref{fig8}. Prefectures are over-expressed within
$138$ communities in the production network, and sectors are over-expressed within $136$ communities. A brief summary of the over-expression of prefectures and sectors in the ten largest communities is tabulated
in Table~\ref{table1}, and the details are given in Appendix C. Our investigation reveals that the largest community is characterized by the over-expression of manufacturing sectors and urban prefectures.
The agriculture, food, and fisheries sectors and rural prefectures are the defining characteristics of the second-largest community. The third-largest community is characterized by the construction, real estate, and banking sectors and by 
Osaka and Tokyo and its neighboring prefectures. The fourth-largest community is mainly characterized by the manufacturing sectors of textiles, rubber, and leather. Medical
and other health services constitute the most distinctive feature of the fifth-largest community. Retail trade (machinery and equipment) and automobile maintenance services dominate
 the sixth-largest community. Hokkaido is the only over-expressed prefecture in the seventh-largest community, whereas the eighth-largest community is strongly associated with Tokyo.
Only three prefectures --- Tokyo, Kanagawa, and Osaka --- are over-expressed in the ninth-largest community, and three different prefectures --- Gifu, Aichi, and Mie --- are the distinctive
features of the tenth-largest community. It is observed that for a given community, its over-expressed sectors are strongly related to its over-expressed prefectures.
We also note that even if the extent to which an attribute is present in the system as a whole is very small, it may be over-expressed in a particular community. For example, the banking
sector is represented by only $161$ banks in our data set, but this sector is over-expressed in the third- and ninth-largest communities. We conclude that different communities are characterized
by distinct features related to different sectors and prefectures. 


%we investigate the firm size distribution with the number of employees as the measure of firm size. We have only $7\%$ missing data in our data set.
%We have studied the firm size distribution for both the original network and coarse grained network. As can be seen from Fig.~\ref{fig6} distributions 
%of firm size follow the well-known Zipf's law $D(e)\sim e^{-\alpha}$ for both the networks. The slope of the straight portion of the distributions in double
%log scale gives the value of $\alpha = 2.09 \pm 0.02$ and $2.0 \pm 0.06 $ respectively for the original network and its coarse grained version.
%Note that the tail of the distribution in the coarse grained network is shifted up from its intermediate straight portion because of finite size effect.   
%From this above result, we conclude that characteristic of the firm size distributions remain invariant even in the coarse grained version of the network 
%and follows Zipf's law. 
 
\section {Conclusion} 
We have studied a large-scale production network in Japan that exhibits 
a  heavy-tailed degree distribution, hierarchical clustering and disassortative
mixing, consistent with previous studies~\cite{fujiwara2010large, mizuno2014structure}.
We have delved deeper into the data using the Infomap technique to detect the 
communities within the network. This production network
contains many communities of highly heterogeneous sizes and with a wide range of
values of the link weights between them. The community structures identified in the directed and undirected
versions of the network display distinct behaviors. Smaller communities tend to have tree-like structures, whereas larger communities have higher link densities.
A large fraction ($40\%$) of the firms in the production network have suppliers or customers
only from within their own communities. We have also characterized the communities based on the over-expression of geographical locations and sectorial classifications. 

The topological properties of a network play a crucial role in dynamical processes.
Our analysis presents the backbone that can be utilized to study any
agent based modeling to explain economic phenomena. Our analysis 
further shows sectorial and regional attributes of firms have an 
important role in forming communities.  This indicates the network 
structure of firms is neither random nor because of the preferential 
attachment rule~\cite{barabasi1999emergence}. 
A future research in this direction will be to model the production network that shows community structure with homophily in industrial sectors and regions.
This study has an application to understand the business cycle correlations in production network~\cite{krichene2017business}. 

%It is found that 
%geographical dominance is more visible than sectorial influence. It is also
%observed that the firm size distribution in the production network as well as 
%in its coarse grained version follows Zipf's law robustly. 

Our results suggest several interesting topics for future economic research. 
In this study, we have characterized the communities only at the top community level.
One could generalize the applied technique to characterize the communities at all levels.
%The highly heterogeneous  community's number reflects the 
A bottom-up structural analysis of the Japanese economy would be helpful for better understanding its complexity. 
Thus, future economic research seeking to understand GDP
fluctuations, inflation and monetary stability should evaluate the Japanese economy at the microscopic
level. 
%Moreover, the Zipf's law of firm size distribution
%indicates the heterogeneous impact of firms on the Japanese economy. 
Accordingly,
future studies should investigate the roles and risks of the largest firms
to yield an understanding of economic shock and driven systemic risk.


\noindent
\begin{center}
{\bf ACKNOWLEDGMENTS}
\end{center}
This research was supported by MEXT as Exploratory Challenges on
Post-K computer (Studies of Multi-level Spatiotemporal Simulation of
Socioeconomic Phenomena), and the Grant-in-Aid for Scientific Research (KAKENHI) by JSPS Grant Number 17H02041.
We thank Hideaki Aoyama, Hiroshi Iyetomi and Yuichi Kichikawa for useful discussions on Infomap. 

%\noindent
\begin{center}
{\bf APPENDIX A: GEOGRAPHICAL ATTRIBUTES OF FIRMS}
\end{center}
Here we list the $47$ prefectures in Japan with their index, that we have used for our study:

1. Hokkaido, 2. Aomori, 3. Iwate, 4. Miyagi, 5. Akita, 6. Yamagata, 7. Fukushima, 8. Ibaraki, 9. Tochigi, 10. Gunma,
11. Saitama, 12. Chiba, 13. Tokyo, 14. Kanagawa, 15. Niigata, 16. Toyama, 17. Ishikawa, 18. Fukui, 19. Yamanashi, 20. Nagano,
21. Gifu, 22. Shizuoka, 23. Aichi, 24. Mie, 25. Shiga, 26. Kyoto, 27. Osaka, 28. Hyogo, 29. Nara, 30. Wakayama,
31. Tottori, 32. Shimane, 33. Okayama, 34. Hiroshima, 35. Yamaguchi, 36. Tokushima, 37. Kagawa, 38. Ehime, 39. Kochi, 40. Fukuoka,
41. Saga, 42. Nagasaki, 43. Kumamoto, 44. Oita, 45. Miyazaki, 46. Kagoshima, 47. Okinawa.

\begin{center}
{\bf APPENDIX B: SECTORIAL ATTRIBUTES OF FIRMS}
\end{center}
Here we list  the $98$ major sectors in Japan with their index using Japan Standard Industrial Classification:

1. Agriculture, 2. Forestry, 3. Fisheries, except Aquaculture, 4. Aquaculture, 5. Mining and quarrying of stone, 6. Construction work, general including public and private construction work,
7. Construction work by specialist contractor, except equipment installation work, 8. Equipment installation work, 9. Manufacture of food, 10. Manufacture of beverages, tobacco and feed,
11. Manufacture of textile products 12. Manufacture of lumber and wood products, except furniture, 13. Manufacture of furniture and fixtures
14. Manufacture of pulp, paper and paper products, 15. Printing and allied industries, 16. Manufacture of chemical and allied product, 17. Manufacture of petroleum and coal products,
18. Manufacture of plastic products, except otherwise classified, 19. Manufacture of rubber products, 20. Manufacture of leather tanning, leather products and fur skins,
21. Manufacture of ceramic, stone and clay products, 22. Manufacture of iron and steel, 23. Manufacture of non-ferrous metals and products, 24. Manufacture of fabricated metal products,
25. Manufacture of general-purpose machinery, 26. Manufacture of production machinery, 27. Manufacture of business oriented machinery, 28. Electronic parts, devices and electronic circuits,
29. Manufacture of electrical machinery, equipment and supplies, 30. Manufacture of information and communication electronics equipment, 31. Manufacture of transportation equipment,
32. Miscellaneous manufacturing industries, 33. Production, transmission and distribution of electricity, 34. Production and distribution of gas, 35. Heat supply,
36. Collection, purification and distribution of water and sewage collection, processing and disposal, 37. Communications, 38. Broadcasting, 39. Information services,
40. Services incidental to internet, 41. Video picture information, sound information, character information production and distribution, 42. Railway transport, 43. Road Passenger transport,
44. Road freight transport, 45. Water transport,  46. Air transport, 47. Warehousing, 48. Services incidental to transport, 49. Postal services, including mail delivery, 
50. Wholesale trade, general merchandise, 51. Wholesale trade (textile and apparel), 52. Wholesale trade (food and beverages), 53. Wholesale trade (building materials, minerals and metals, etc),
54. Wholesale trade (machinery and equipment), 55. Miscellaneous Wholesale trade, 56. Retail trade, general merchandise, 57. Retail trade (woven fabrics, apparel, apparel accessories and notions),
58. Retail trade (food and beverage), 59. Retail trade (machinery and equipment), 60. Miscellaneous retail trade, 61. Nonstore retailers, 62. Banking, 63. Financial institutions for cooperative organizations,
64. Non-deposit money corporations, including lending and credit card business, 65. Financial products transaction dealers and futures commodity dealers, 66. Financial auxiliaries,
67. Insurance institutions, including insurance agents brokers and services, 68. Real estate agencies, 69. Real estate lessors and managers, 70. Goods rental and leasing,
71. Scientific and development research institutes, 72. Professional services, N.E.C., 73. Advertising, 74. Technical services, N.E.C., 75. Accommodations, 
76. Eating and drinking places, 77. Food take out and delivery services, 78. Laundry, beauty, and bath services, 79. Miscellaneous living-related and personal services, 
80. Services for amusement and recreation, 81. School education, 82. Miscellaneous education, learning support, 83. Medical and other health services, 84. Public health and hygiene,
85. Social insurance, social welfare and care services, 86. Postal services, 87. Cooperative associations, N.E.C, 88. Waste disposal business, 89. Automobile maintenance services,
90. Machine, etc. repair services, except otherwise classified, 91. Employment and worker dispatching services, 92. Miscellaneous business services, 93. Political, business and cultural organizations, 
94. Religion, 95. Miscellaneous services, 96. Foreign governments and international agencies in Japan, 97. National government services, 98. Local government services. 

\begin{center}
{\bf APPENDIX C: DETAIL DESCRIPTIONS OF TEN LARGEST COMMUNITIES}
\end{center}
Here we report the details of over-expression of prefectures and sectors in ten largest communities.
The pair of values within the parentheses indicate the number of occurrence of the attribute within the community 
and in the whole system. 
%--------------------------------------------------------------------------- 
\begin {itemize}
\item {\bf Community rank:} 1, {\bf size:} 233294\\
{\bf Prefecture over-expressions:} Gunma (4436/19893), Saitama (11181/46418), Tokyo (37172/186179), 
Kanagawa (15776/62781), Nagano (4528/21641), Shizuoka(7889/35806), Aichi (19903/70128)
Mie (3771/17747), Osaka ( 26641/87226), Hyogo (10120/41282), Hiroshima (6795/30651), Yamaguchi (2881/12604).\\
{\bf Sector over-expressions:} Equipment installation work(24461/101430), Manufacture of chemical and allied product (2420/5946),
Manufacture of petroleum and coal products (198/518), Manufacture of plastic products except otherwise classified (6030/8922),
Manufacture of rubber products (1530/2060), Manufacture of ceramic, stone and clay products (1846/8051), Manufacture of iron and steel (2602/3210),
Manufacture of non-ferrous metals and products (1960/2291), Manufacture of fabricated metal products (17655/24494),
Manufacture of general-purpose machinery (6676/8125), Manufacture of production machinery (13814/17342),
Manufacture of business oriented machinery (2672/4194), Electronic parts, devices and electronic circuits (3768/4363)
Manufacture of electrical machinery, equipment and supplies (7081/8795), Manufacture of information and communication electronics equipment (1389/2101)
Manufacture of transportation equipment (5049/6623), Miscellaneous manufacturing industries (2585/11270),
Production, transmission and distribution of electricity (376/1306), Water transport (894/1490), Services incidental to transport (1219/4474),
Wholesale trade, general merchandise (1568/4891), Wholesale trade (building materials, minerals and metals, etc) (10858/34380), 
Wholesale trade (machinery and equipment) (19336/40672), Retail trade (machinery and equipment) (11653/41319),
Technical services, N.E.C. (5298/26216), Machine, etc. repair services, except otherwise classified (2490/4670)
Employment and worker dispatching services (1055/4683).
\item {\bf Community rank:} 2, {\bf size:} 139380\\
{\bf Prefecture over-expressions:} Aomori (2447/14681), Iwate(2115/12419), Miyagi (3388/22888), Akita (1804/11705),
Yamagata (1969/13168), Fukushima (2596/20897), Niigata (3153/25586), Nagano (2914/21641), Shizuoka (4713/35806), Wakayama (1341/10004)
Tottori (920/5468), Shimane (1030/8209), Tokushima (1268/9172), Kagawa (2018/12287), Ehime (2271/16097), Kochi (1125/8078),
Saga (1468/7873), Nagasaki (1945/12078), Kumamoto (2356/16358), Oita (2129/12738), Miyazaki (2153/12013), Kagoshima (3251/15423).\\
{\bf Sector over-expressions:} Agriculture (4277/10747), Fisheries, except aquaculture (563/724), Aquaculture (627/771),
Manufacture of food (16020/19833), Manufacture of beverages, tobacco and feed (2675/3935), Road freight transport (3615/29678),
Warehousing (372/1997), Wholesale trade, general merchandise (931/4891), Wholesale trade(food and beverages) (23126/28535),
Miscellaneous wholesale trade (6497/34637), Retail trade, general merchandise (1095/1531), Retail trade (food and beverage) (18082/23948),
Nonstore retailers (297/1203), Financial institutions for cooperative organizations (130/511), Non-deposit money corporations, 
including lending and credit card business (180/755), Real estate lessors and managers (3631/25301), Accommodations (2339/6062),
Eating and drinking places (11276/15725), Food take out and delivery services (280/432), Miscellaneous living-related and personal services (1413/7001),
Services for amusement and recreation (1119/8242), Social insurance, social welfare and care services (1119/7616),
Cooperative associations, N.E.C (2366/7925), Miscellaneous services (128/249).
\item {\bf Community rank:} 3, {\bf size:} 59906\\
{\bf Prefecture over-expressions:} Saitama (4665/46418), Chiba (3833/39828), Tokyo (21293/186179), Kanagawa (7438/62781), Osaka (6139/87226).\\
{\bf Sector over-expressions:} Construction work, general including public and private construction work (9653/163082),
Construction work by specialist contractor, except equipment installation work (15242/125183), Equipment installation work (8306/101430),
Production, transmission and distribution of electricity (142/1306), Heat supply (27/80), Banking (40/161), Non-deposit money corporations, 
including lending and credit card business (109/755), Financial products transaction dealers and futures commodity transaction dealers (240/1263),
Financial auxiliaries (69/262), Real estate agencies (3587/19183), Real estate lessors and managers (3570/25301),
Goods rental and leasing (567/6910), Professional services, N.E.C.(1127/19393), Technical services, N.E.C.(2609/26216),
Miscellaneous business services (1887/19963), Foreign governments and international agencies in Japan (7/20).
\item {\bf Community rank:} 4, {\bf size:} 47849\\
{\bf Prefecture over-expressions:} Tokyo (9248/186179), Ishikawa (659/12307), Fukui (1122/11599), Gifu (1127/20230),
Aichi (2929/70128), Kyoto (2407/24369), Osaka (5657/87226), Hyogo (2010/41282), Nara (535/9516), Wakayama (518/10004),
Okayama (1011/20023), Ehime (721/16097).\\
{\bf Sector over-expressions:} Manufacture of textile products (8893/12402), Manufacture of rubber products (222/2060),
Manufacture of leather tanning, leather products and fur skins (1033/1520), Miscellaneous manufacturing industries (958/11270),
Warehousing(112/1997), Wholesale trade, general merchandise ( 536/4891), Wholesale trade (textile and apparel) (7241/10562),
Miscellaneous wholesale trade (1910/34637), Retail trade (woven fabrics, apparel, apparel accessories and notions)(13082/16255),
Miscellaneous retail trade (3107/63111), Nonstore retailers (114/1203), Miscellaneous living-related and personal services (866/7001),
Services for amusement and recreation (691/8242).
\item {\bf Community rank:} 5, {\bf size:} 44349\\
{\bf Prefecture over-expressions:} Hokkaido (2778/58982), Iwate (774/12419), Tochigi (1258/20653), Tokyo(9954/186179),
 Kyoto(1159/24369), Hiroshima (1673/30651), Tokushima (565/9172),  Kochi (394/8078), Fukuoka (2091/45359), Kumamoto (1174/16358),
 Miyazaki (794/12013).\\
 {\bf Sector over-expressions:} Manufacture of chemical and allied product (739/5946), Manufacture of business oriented machinery (575/4194),
 Wholesale trade (machinery and equipment) (1916/40672), Miscellaneous wholesale trade (1735/34637),  Miscellaneous retail trade (6574/63111),
 Scientific and development research institutes (117/547), Medical and other health services (22622/26123), Public health and hygiene (139/387),
 Social insurance, social welfare and care services (1414/7616).
 \item {\bf Community rank:} 6, {\bf size:} 43397\\
 {\bf Prefecture over-expressions:} Aomori (1028/14681), Miyagi (1514/22888), Akita (759/11705),  Yamagata(726/13168), Ibaraki (1211/26412),
 Tochigi (1011/20653), Gunma (1051/19893), Saitama (1819/46418), Chiba (1716/39828), Kanagawa (2380/62781), Toyama (773/13423), Ishikawa (673/12307),
 Fukui (539/11599), Yamanashi (519/9981), Gifu (857/20230), Shizuoka (1514/35806), Mie (741/17747), Okayama (963/20023), Hiroshima (1209/30651),
 Saga (358/7873), Kumamoto (767/16358), Oita (671/12738), Miyazaki (535/12013).\\
 {\bf Sector over-expressions:} Manufacture of transport equipment (369/6623), Road passenger transport (1199/4492),  Wholesale trade
 (machinery and equipment) (3031/40672), Retail trade (machinery and equipment) (15483/41319), Insurance institutions,
 including insurance agents, brokers and services (4805/6234),  Goods rental and leasing (407/6910), School education (312/3416),
 Automobile maintenance services (8607/17703).
 \item {\bf Community rank:} 7, {\bf size:} 43018\\
 {\bf Prefecture over-expressions:} Hokkaido (37867/58982).\\
 {\bf Sector over-expressions:} Agriculture (1214/10747), Forestry (156/934), Fisheries, except aquaculture (62/724), 
 Mining and quarrying of stone (142/1479), Construction work, general including public and private construction work (7077/163082),
Construction work by specialist contractor, except equipment installation work (6095/125183), Equipment installation work (3960/101430), 
Manufacture of lumber and wood products, except furniture (273/5132), Road passenger transport (222/4492), Road freight transport (1554/29678),
Wholesale trade  (building materials, minerals and metals, etc) (1339/34380), Retail trade (food and beverage) (1203/23948), 
Financial institutions for cooperative organizations (41/511), Real estate lessors and managers (1001/25301), 
Goods rental and leasing (336/6910), Technical services, N.E.C. (1266/26216), Accommodations (267/6062),
Eating and drinking places (683/15725), Social insurance, social welfare and care services (328/7616), 
Cooperative associations, N.E.C (520/7925), Automobile maintenance services (1077/17703), Local government services (223/2112).
\item {\bf Community rank:} 8, {\bf size:} 38819\\
 {\bf Prefecture over-expressions:} Tokyo (18716/186179).\\
 {\bf Sector over-expressions:} Printing and allied industries (1138/10996), Manufacture of information and communication electronics equipment (99/2101),
 Miscellaneous manufacturing industries (1312/11270), Communications (75/865), Broadcasting (358/742), Information services (3096/23741),
 Services incidental to internet (427/1236), Video picture information, sound information, character information production and distribution (4361/7665),
 Wholesale trade, general merchandise (246/4891), Miscellaneous wholesale trade (2373/34637), Miscellaneous retail trade (4586/63111),
 Nonstore retailers (256/1203), Financial products transaction dealers and futures commodity transaction dealers (72/1263), 
 Goods rental and leasing (483/6910), Professional services, N.E.C. (2821/19393), Advertising (2572/6306),  Services for amusement and recreation(1225/8242),
 School education (529/3416), Miscellaneous education, learning support (718/2087), Employment and worker dispatching services (558/4683),
 Miscellaneous business services(1701/19963), Political, business and cultural organizations (411/5288).
\item {\bf Community rank:} 9, {\bf size:} 33654\\
 {\bf Prefecture over-expressions:} Tokyo (13884/186179), Kanagawa (2018/62781), Osaka (2667/87226).\\
 {\bf Sector over-expressions:}  Equipment installation work (4096/101430), Manufacture of information and communication electronics equipment (280/2101),
 Production, transmission and distribution of electricity (61/1306), Communications (494/865), Broadcasting (70/742), Information services (11114/23741),
 Services incidental to internet (380/1236), Video picture information, sound information, character information production and distribution (278/7665),
 Wholesale trade (machinery and equipment) (1801/40672), Retail trade (machinery and equipment) (1838/41319), Nonstore retailers(65/1203),
 Banking (17/161), Financial institutions for cooperative organizations (35/511), Non-deposit money corporations, including lending and credit card business (65/755),
 Financial products transaction dealers and futures commodity transaction dealers (272/1263), Financial auxiliaries (49/262),
 Goods rental and leasing (302/6910), Professional services, N.E.C. (2350/19393), Advertising (230/6306), Technical services, N.E.C. (852/26216),
 Miscellaneous education, learning support (128/2087), Employment and worker dispatching services (676/4683), Miscellaneous business services (1154/19963), 
 Political, business and cultural organizations (293/5288).
 
 \item {\bf Community rank:} 10, {\bf size:} 33563\\
 {\bf Prefecture over-expressions:} Gifu (8843/20230), Aichi (19937/70128), Mie (752/17747) \\
 {\bf Sector over-expressions:} Mining and quarrying of stone (88/1479), Construction work, general including public and private construction work (9573/163082),
 Construction work by specialist contractor, except equipment installation work (7621/125183), Equipment installation work (4717/101430),
 Manufacture of lumber and wood products, except furniture (211/5132), Manufacture of ceramic, stone and clay products (354/8051), 
 Wholesale trade  (building materials, minerals and metals, etc) (1168/34380), Real estate agencies (859/19183), Waste disposal business (318/9078),
 Local government services (102/2112).
\end{itemize}
%% References
%%
%% Following citation commands can be used in the body text:
%% Usage of \cite is as follows:
%%   \cite{key}          ==>>  [#]
%%   \cite[chap. 2]{key} ==>>  [#, chap. 2]
%%   \citet{key}         ==>>  Author [#]

%% References with bibTeX database:

%\bibliographystyle{model1-num-names}
\bibliographystyle{apsrev4-1}
%\bibliography{com}
\documentclass[review]{elsarticle}

\usepackage{lineno,hyperref}
\modulolinenumbers[5]
	

\usepackage{color}
\usepackage[normalem]{ulem}
\usepackage{diagbox}
\usepackage{makecell}

%\journal{Journal of Mathematical Psychology}
\journal{arXiv}

%%%%%%%%%%%%%%%%%%%%%%%
%% Elsevier bibliography styles
%%%%%%%%%%%%%%%%%%%%%%%
%% To change the style, put a % in front of the second line of the current style and
%% remove the % from the second line of the style you would like to use.
%%%%%%%%%%%%%%%%%%%%%%%

%% Numbered
%\bibliographystyle{model1-num-names}

%% Numbered without titles
%\bibliographystyle{model1a-num-names}

%% Harvard
%\bibliographystyle{model2-names.bst}\biboptions{authoryear}

%% Vancouver numbered
%\usepackage{numcompress}\bibliographystyle{model3-num-names}

%% Vancouver name/year
%\usepackage{numcompress}\bibliographystyle{model4-names}\biboptions{authoryear}

%% APA style
%\bibliographystyle{model5-names}\biboptions{authoryear}

%% AMA style
%\usepackage{numcompress}\bibliographystyle{model6-num-names}

%% `Elsevier LaTeX' style
\bibliographystyle{elsarticle-num}
%%%%%%%%%%%%%%%%%%%%%%%

\begin{document}

\begin{frontmatter}

\title{New Empirical Evidence on Disjunction
Effect and Cultural Dependence}
%\tnotetext[mytitlenote]{ }

%% Group authors per affiliation:
\author{Indranil Mukhopadhyay\fnref{email1}}
\address{Human Genomics Unit, Indian Statistical Institute, Calcutta-700035.}
\fntext[email1]{ Email: indranilm100@gmail.com.}

\author{Nithin Nagaraj\fnref{email2a}}
\address{Consciousness Studies Programme, National Institute of Advanced Studies, IISc. Campus, Bengaluru 560012.}
\fntext[email2a]{Email: nithin@nias.iisc.ernet.in.}
\author{Sisir Roy\fnref{email2b}}
\address{Conciousness Studies Programme, National Institute of Advanced Studies,IISC Campus, Bengaluru 560012.}
\fntext[email2b]{Corresponding author, email: sisir.sisirroy@gmail.com.}

% \author{Sisir Roy\fnref{email3}}
% \address{Consciousness Studies Programme, National Institute of Advanced Studies, IISc. Campus, Bengaluru 560012.}
% \fntext[email3]{Email: sisir.sisirroy\@gmail.com}


%% or include affiliations in footnotes:
% \author[mymainaddress,mysecondaryaddress]{Elsevier Inc}
% \ead[url]{www.elsevier.com}

% \author[mysecondaryaddress]{Global Customer Service\corref{mycorrespondingauthor}}
% \cortext[mycorrespondingauthor]{Corresponding author}
% \ead{support@elsevier.com}

% \address[mymainaddress]{1600 John F Kennedy Boulevard, Philadelphia}
% \address[mysecondaryaddress]{360 Park Avenue South, New York}

\begin{abstract}
We perform new experiment using almost the same sample size considered by Tversky and Shafir to test the validity of classical probability theory in decision making. The results clearly indicate that the disjunction effect depends also on culture and more specifically on gender(females rather than males). We did more statistical analysis rather that putting the actual values done by previous authors. We propose different kind of disjunction effect i.e. strong and weak based on our statistical analysis. 
\end{abstract}

\begin{keyword}
Disjunction effect \sep decision making \sep quantum probability \sep cognition
%\MSC[2010] 00-01\sep  99-00
\end{keyword}

\end{frontmatter}

%\linenumbers


\section{Introduction}
Tversky and Shafir (1992)~\cite{tversky1992choice} discovered a phenomenon called the {\it disjunction effect}, while following the process of testing a rational axiom of decision theory. It is also called as the sure thing principle (Savage, 1954)~\cite{savage1954foundations}. First consider the states $A$ and $B$ that belong to the state of the world $X$.  This principle states that, if action A over B is preferred, and under the complementary state of the world, again, action A over B is preferred, it is expected that one should prefer action A over B even when the state of the world is not known. Symbolically, this can be expressed as

 If $${P(A\cap X) > P(B \cap X)} $$ and $${P(A\cap X^C) > P(B \cap X^C)}$$

Then $$ P(A) = P(A \cap (X\cup X^C) > P(B \cap (X \cup X^C) = P(B)$$   
i.e. $$ P (A) > P (B) $$ occurs always. 

With the aim of testing this principle, in their experiment, Tversky and Shafir (1992)~\cite{tversky1992choice} performed the test considering a two stage gamble by presenting $98$ students. 
They adopted a two stage gamble, i.e., it is possible to play the gamble twice. The gamble is done under the following two conditions: 
\begin{itemize}
\item The students are informed that they lost the first gamble.
\item The students remained unaware of the outcome of the first gamble.
\end{itemize}

The gamble to be played, had an equal stake, i.e., of wining ${200}$ or loosing ${100}$ for each stage of taking decision, i.e., whether to play or not to play the gamble. 

Interestingly, the results of these experiments can be described into the following manner: 
	\begin{itemize}
		\item The students who won the first gamble 69\% choose to play at the second stage;
		\item The students who lost then 59 \% choose to play again;	
		\item The students who are unaware whether they won or lost 36\% of them (i.e., less of the majority of the students) choose to play again.
			
	\end{itemize}
	
	Explaining the findings in terms of choice based on reasons, Tversky and Shafer (1992)~\cite{tversky1992choice} did raise some questions for these surprising results: whenever, the persons, related to the play, knew that if they win, then as they would have extra house money, they can play again, because if they lose, they can play again to recover the loss. Now the students who did not know the outcome, then the main issue is why sizable fraction of the students want to play again the game since either they win or lose and cannot be anything else? Thus they arrived at the key result, but faced the problem of explaining the outcome just as either a win or loss. Busemeyer, Wang and Townsand (2006)~\cite{busemeyer2006quantum} originally suggested that this finding could be an example of an interference effect, similar to that found in the double slit experiments conducted in modern particle physics. 

Let us consider the following analogy between the disjunction experiment and the classic double slit type of experiment in physics:
Both the cases involve two possible paths: here, in the disjunction experiment, the two paths are inferring the outcome of either a {\it win} or {\it a loss} with the first gamble; for the double split experiment, the two paths are splitting the photon into the upper or lower channel applying a beam splitter. The path taken can be known (observed) or unknown (unobserved), in both the experiments. Finally in both the cases, the fact is that when the case of gambling for disjunction experiment and hence, the detection at a location for the two slit experiments are considered for the chosen unknown, i.e., unobserved conditions, the resultant probability, meant for observing interference phenomena, are found to be much less than each kind of the probabilities which is observed for the known (observed) cases. Under these circumstances, we can speculate that during the disjunction experiment, under the unknown condition, instead of being definite, so far as the win or loss state is concerned, the student enters a superposition state. In fact, this state prevents finding a reason for choosing the gamble. In double-slit experiments, the law of additivity of probabilities of occurring two mutually exclusive events (particle aspect or wave aspect) is violated i.e. total probability
                                                 $$P_{AB} = P_A + P_B$$
																						
																								 
for two mutually exclusive events A and B. 
This is due to the existence of interference effects, known as Formula of Total Probability (FTP). It has already been established fact that the two slit (interference) experiment is the basic experiment which violates FTP. Feynman (1951)~\cite{feynman1966feynman}, in many of his works in physics, presented his points with detailed arguments about this experiment. There, the results, i.e., the appearance of interference fringes appeared to him, not at all surprising phenomena. He explained it as follows:
In principle, interaction with slits placed on the screen may produce any possible kind of distribution of points on the registration screen. Now, let us try to explain following quantum probabilistic features which appear only when one considers following three kind of different experiments~\cite{conte2009mental}: %(Conte et al., 2009): 

\begin{itemize}
\item When only the first slit open i.e., the case $B=+1$, in an equivalent manner.
\item When only the second slit is open i.e., $B= -1$, in this case.
\item In the particular case, both slits being open, it is the random variable B determining the slit to pass through.
\end{itemize}

At this stage, let us now choose any point at the registration screen.  Then resultant scenario will be as follows: 
\begin{itemize}
\item the random variable A if  $A= +1$.
\item the opposite case happens if a particle hits the screen at this point, i.e., $A= -1$.
\end{itemize}

Now for classical particles, FTP should predict the probability for the  experiment (both slits are open), supposed to be provided by the (1) and (2) experiments. But, it has already been mentioned that, in case of quantum particles, FTP is violated: for the additional cosine-type term appearing in the right-hand side of FTP, it is the interference effect in probabilities which is responsible. Feynman characterized this particular characteristic feature of the two slit experiment as the most profound violation of laws of classical probability theory. He explained it the following way: 

In an ideal experiment, where there is no presence of any other external uncertain disturbances, the probability of an event, called probability amplitude is the absolute square of the complex quantity. But, when there is possibility of having the event in many possible ways, the probability amplitude is the sum of the probability amplitude considered separately. Following their experimental results, Tversky and Shafir (1992)~\cite{tversky1992choice} demand that this violation of classical probability is also possible to be present, happening in their experiments with cognitive systems. Though, due to the possible restrictions present in quantum mechanics, we could not start from Hilbert formalism at the start, in the laboratory, this formalism was justified by experiments.
Let us now try to interpret the meaning of interference effect within the context of the experiments on gambling, described above. We will follow, here, Busemayer's formulation (2011)~\cite{busemeyer2011quantum} which is as follows:

Two different judgment tasks A and B are considered in this case. The task A is considered having $j$ (taking $j= 2$, binary choice) different levels of response variable and B, with $k$ (say, $k= 7$ points of confidence rating) levels of a response measure. Two groups, being randomly chosen, out of the total participants we have: 

\begin{itemize}
\item Group A gets task A only.
\item Group BA gets task B followed by task A. 
\end{itemize}

The response probabilities can be estimated as follows:
\begin{itemize}
\item From  the group A,  let   $p( A = j)$ be estimated;  This denotes  the probability of choosing level j out of the response to task A .
\item And then from the group B, let $p(B=k)$ be estimated; the corresponding probability denoted  by choosing level k from the task B.
\item Now, it is possible to estimate the conditional probability $p (A=j|B=k)$; which can be stated as the probability of responding with level j from the task A, given the person responded with level k on earlier task B.
\end{itemize}

So, we can write the estimated interference for level j to task A (produced, when responding to task B) as,  
                                     $$ A (j) = p (A = j) – p_T (A = j)$$,
where,  $p_T (A = j)$  denotes the total probability for the response to task A.
In defense of using Quantum formalism in case of human judgments, Busemeyer and Truebold did put four following reasons, beautifully, in their famous paper (Busemeyer et al., 2011~\cite{busemeyer2011quantum}):

{\it (a) judgment is not a simple readout from a preexisting or recorded state, but instead it is constructed from the current context and question; from this first point it then follows that (b) making a judgment changes the context which disturbs the cognitive system; and the second point implies that (c) changes in context produced by the first judgment affects the next judgment producing order effects, so that (d) human judgments do not obey the commutative rule of classic probability theory(Busemeyer et al; 2011~\cite{busemeyer2011quantum})}.
\noindent

In fact, the existence of interference term for microscopic entities or quantum entities clearly indicates the existence of three valued or non-Boolean logic. This is popularly known as Quantum logic. It is mathematically shown that a set of propositions which satisfies the different axiomatic structures for the non-Boolean logic generates Hilbert space structures. The quantum probability associated with this type of Quantum logic can be applied to decision making problems in cognitive domain. It is to be noted that, up till now, no quantum mechanical framework is taken as valid description of the anatomical structures and function of the brain. This framework of quantum probability is very abstract and devoid of any material content. So it can be applied to any branch of knowledge like Biology, Social science etc. Of course, it is necessary to understand the issue of contextualization, for example, here, in case of decision making in brain. It is worth mentioning that the decision making may depend also on culture. For example, the students participated in the above mentioned gamble are mainly taken from the west. So far as the authors' knowledge, concern of this kind of experiment has not been performed taking the subjects from the east. We presume that this is an important factor one should consider in this kind of experiment involving gambling since the very concept of gambling may depend on culture. While experimenting to find the effect of culture, we have found an additional interesting fact. It is not only culture but also gender that might play an important role on disjunction effect. It is natural to expect and consider that the gender dependency is closely dependent on culture. Recently, we performed the above mentioned two stage gambling considering almost the same sample size in India. The results clearly show that the violation of classical probability rule depends very much on the variation of gender. In the next section we will describe the experiment and the results.

\section{Materials and Methods}
 In order to get an insight of the above discussion we have performed a gambling experiment in the line of Tvaersky and Shafir (1952). Our main objective is to see whether there is any disjunction effect, especially in Indian context.

\subsection{Participants}
As we believe that there might be a cultural effect, we carefully select the participants. The participants should consist of both males and females as cultural effect with respect to gender is sometimes significantly observable. We have selected the population of experimental objects as a homogeneous groups with respect to age so that there would not be any effect that can mask our objectives. From Raja Rajeswari Engineering College, Bengaluru, India, we select $101$ college students randomly. It is seen that there are $50$ female and $51$ male students participating the experiment. The students belong 1st, 2nd, or 3rd year of their engineering curriculum and naturally belong to a very homogeneous age group. Prior permission is taken from the principal of the college and the studetns gave consents to this experiment. 


\subsection{Design and procedure}
In the experiment we toss a coin and the experiment depends on the outcome. We first divide students into two groups: one consisting $70$ students and the other $31$ students. We perform the experiment similar to that described in Busemeyer. If a students wins the first toss, he/she will receive Rs. $100$; if loses, he/she will lose Rs. $50$. After the first toss, we ask a student whether he/she wants to play again. If he/she wants to play again, he/she will win Rs. $200$ if his/her guess about he outcome matches with the outcome of the second toss; otherwise will lose Rs. $100$. We play this game under two different schemes.

In scheme $1$, each student belonging to the first group ($70$ students), will declare his/her guess about the result of a coin tossing experiment and will be informed the result after the toss. We then ask that student whether he/she wants to play again and record his/her response. In scheme $2$, we perform this experiment for the second group of students ($31$ students) slightly differently than the first scheme. Here, each student will declare his/her guess about the result if the first toss, but he/she will not be informed the result. However, we then ask him/her whether he/she wants to play again and record his/her response. We have used only one coin throughout experiment. Before the beginning of this experiment, we checked whether the coin is unbiased using a binomial experiment and confirmed that it is unbiased. We also make sure that the student who has perfumed the experiment has no way to disclose the result or his/her guess and attitude towards this experimental result. Moreover, there is no exchange of information between the two groups of students for two different schemes.


\begin{itemize}
\item 	We have selected randomly 101 college students from Raja Rajeswari Engineering College, Bengaluru, India. The students belong 1st, 2nd, or 3rd year of their engineering curriculum and naturally belong to a very homogeneous age group.
\item 	We used an unbiased coin for the experiment. Before the experiment we have checked that the coin used is unbiased.
\item We then divide them into two groups: one consisting $70$ students and the other $31$ students.
\item We performed the experiment similar to that described in Busemeyer. If a students wins the first toss, he/she will receive Rs. $100$; if loses, he/she will lose Rs. $50$.
\item Each student belonging to the first group ($70$ students), will declare his/her guess about the result of a coin tossing experiment and will be informed the result after the toss. We record if he/she wants to play again.
\item Each student belonging to the second group ($31$ students) will declare his/her guess about the result if the first toss, but he/she will not be informed the result. We then record if he/she wants to play again.
\end{itemize}
\noindent

{\it {\bf Probability theoretic explanation:}}

We have already discussed that $P(A) > P(B)$ is true always under the conditions that $P(A\cap X) > P(B \cap X)$ and $P(A\cap X^C) > P(B \cap X^C)$. 

Thus, if you prefer A when the event $X$ is known, and if you prefer B when the complementary event $X^C$ is known, then it will imply that you will always prefer A over B irrespective of the events $X$ or $X^C$. Any violation of this is termed as disjunction effect.

We have performed experiment similar to that given in Busemeyer and we observed marked violation of this probabilistic claim and its explanation depends on several factors,  not reported or discussed in literature. Not only the psychological factors, but also other factors like sex, culture, geographical region etc, might have important role to play.

\section{Results}
 We have observed a few interesting facts from the experiment. Instead on looking at the numerical figures of the outcomes of the experiment only as in Tversky and Shafir (1952), we have done a detailed statistical analysis in order to strengthen the interpretation of our observations. As described in the previous section, our experiment consists of two different schemes. Under the first scheme, the students were informed the outcome of the first toss. If they know that they won the first gamble, $76.47\%$ want to play again, i.e. majority want to play again. However we would like to be sure that the result is statistically significant. For this, we perform a test of null hypothesis $H_0:p=0.5$ against an alternative hypothesis $H_1:p>0.5$. The p-value associated with this test is $0.0004$ indicating that the majority wants too play again. On the other hand, if they did no€™t know, whether they won or lost, $58.33\%$ want to play again. Although it seems that majority wants to play again, but the result is not statistically significant (p-value$=0.1215$). Under scheme $2$, when the students did not know the outcome of the first toss, i.e. when they did not know whether they won or lost, $54.84\%$ want to play again. So, in this case, although it seems that majority wants to play again, but the result is not statistically significant since the associated p-value is $0.3601$.

In each case majority wants to play again. But in Busemeyer the corresponding figures are $69\%$, $59\%$, and $36\%$ respectively, while in our experiment the figures are $76.47\%$, $58.33\%$, and $54.84\%$ respectively. The last figure differs widely while the second figure matches surprisingly. We think that probably p-value corresponding to the second figure (i.e. $59\%$) is not significant. So Busemeyer inference needs to be revisited. However, the idea of paying again when they know that whether they won or lost still remains valid if we go only by the actual figures compared to $50\%$, which indicates the state of indifference. So we did an exact test of hypothesis that $H_0: p=0.5$ against $H_1: p>0.5$ i.e. to see absence of disjunction effect. The p-value is $0.3601$ indicating that they are indifferent to the decision. So either disjunction effect is not observed here, or very weak, but they do not prefer to play when they do not know the state.



 It seems clear that the observations by Busemeyer is {\it{not}} matching with ours. We conjecture that this discrepancy may be due to the effect of different cultural settings in which the experiments are conducted. However, we proceed further to examine whether there exists any other factor that might play an implicit role in the experimental results. Indian culture and gender are intermingled always. Hence it would be a wise idea to revisit the experimental results incorporating gender factor. We observed that among females, $75\%$ (p-value$=0.0106$) females want to play again if they know that they won, $52.94\%$ (p-value$=0.3145$) females want to play again if they know that they lost. This result is not at all significant indicating that once lost, females do not want to take one more risk. Moreover, $41.18\%$ (p-value$=0.6855$) females want to play again if they do not the first result.

This picture is markedly different among males. $77.78\%$ (p-value$=0.0038$) males want to play again if they know that they won, $63.16\%$ (p-value$=0.0835$) males want to play again if they know that they lost. However, this result is marginally significant, although not strong. Moreover, $71.43\%$ (p-value$=0.0288$) males want to play again when they do no€™t know the outcome of the first experiment. Since majority wants to play again whether they won or lost or uninformed, there is no disjunction effect observed among males. 



\noindent
\vskip5pt

\section{Discussions}
The above results of our experiment raise the following important issues:

\begin{itemize}
\item 	If we consider only absolute values, males do not show any disjunction effect, but females show strong disjunction effect. Now we have to explain this in Indian context, if possible.
\item	However, if we go by the p-values of the corresponding tests of significance, interesting observations can be made using the combination of tests’ results. For all individuals, males and females taken together, let us first combine the results of the conditions that first result is known to be win and that to be loss. The combined p-value for this is 0.00053 indicating that majority of the individuals want to play again if they know the result. But the p-value for the result when they do not know the first result is 0.2366 indicating that there is disjunction effect.
\item For males, disjunction effect is not observed if irrespective of whether we go by the actual values or by the p-values; combined p-value is 0.0029 whereas p-value when the first result is not known is 0.0288. However, the general effect is moving towards the disjunction effect although not established statistically or by observations.
\item For females, although actual observations suggest only weak disjunction effect, but comparing p-values for the combined p-value (0.0223) to the p-value (0.6855) when the first result is unknown shows clear disjunction effect.
\item Disjunction effect is although a clear concept and is realized in a number of experiments, cultural effects are strong especially among males and females. In Indian context, probably disjunction effect for females is so strong that they overcome the absence of disjunction effect among males.
\item Busemeyer'€™s experimental results are based only on the actual values and not tested statistically. This is highly influenced by specific cases and specific scenarios considered in the experiment. It is not wise to declare that the effect is present or absent whenever the number of observations is less or greater than $50\%$; statistical test must be employed to validate and confirm the findings.
\item The missing part of all other previous experiments is the appropriate statistical analysis of the results. Based on our analysis, we propose different kinds of disjunction effect, e.g. strong and weak. 
\end{itemize}
This can be stated in the following manner.

\begin{table}[!h]
\caption{Categorisation of disjunction effect based on actual observation and statistical tests.}\centering
\vspace{2mm}
\begin{tabular}{|c|c|c|}
\hline
\diaghead{\theadfont ColumnmnHead}
{}{Actual observation}  & significant & Not significant\\
{Statistical test}{} & & \\
\hline
Significant & Strong effect & --- \\
(p-value is small) & & \\
\hline
Not significant & Weak effect & No effect \\
(p-value is large) & & \\
\hline
\end{tabular}
\end{table}

The above categorization triggers to rethink the classification of disjunction effect with respect toothed factor like gender etc. Here we propose a general categorization based on actual observations and results of statistical significance tests in presence of another factor `gender'. This is given in Table 2.

\begin{table}[!h]
\caption{A general categorization of disjunction effect based on actual observation and statistical tests.}\centering
\vspace{2mm}
\begin{tabular}{|c|c|c|c|}
\hline

{}{Actual observation} & Significant in both & Significant in males & Not significant \\
\diaghead{\theadfont ColumnmnHead}
{}{} & males and females & but not in females & in both \\
{Statistical test}{} & & or vice versa & \\
\hline
Significant in both & & & \\
males and females & Strong effect & --- & --- \\
(p-value is small) & & & \\
\hline
Significant in males but & & & \\
not in females or vice & Moderate effect & Moderate effect & --- \\
versa (large p-value) & & & \\
\hline
Not significant & & & \\
in both & Weak effect & Weak effect & No effect \\
(p-value is large) & & & \\
\hline
\end{tabular}
\end{table}

In this manner we may come up with more logical categorization.
\noindent
\vskip5pt



\section*{Acknowledgements}
The authors acknowledge the Rector, RR Group of institutions, the principal R.R.Engineering College and the participating students from R.R.Engineering college for their help and cooperation. We would also like to thank Mr. Nepal Banerjee for his help in conducting the experiments. This work is under the project SB/S4/MS:844/2013 approved by SERB, DST, Government of India.                             
                    

% \section{Bibliography styles}
% There are various bibliography styles available. You can select the style of your choice in the preamble of this document. These styles are Elsevier styles based on standard styles like Harvard and Vancouver. Please use Bib\TeX\ to generate your bibliography and include DOIs whenever available.

% Here are two sample references: \cite{Feynman1963118,Dirac1953888}.

\section*{References}

\bibliography{mybibfile}

\end{document}   
%\begin{thebibliography}{37}%
%\end{thebibliography}%
\end {document}
