% This is samplepaper.tex, a sample chapter demonstrating the
% LLNCS macro package for Springer Computer Science proceedings;
% Version 2.20 of 2017/10/04
%
\documentclass[runningheads]{llncs}
%
\usepackage{graphicx}
\usepackage{amssymb}
\usepackage{amsmath}
\usepackage{microtype}
\usepackage{booktabs}
\usepackage{multirow}
\usepackage{array}
\usepackage{hyperref}
\usepackage{cite}
\usepackage{microtype}
% Used for displaying a sample figure. If possible, figure files should
% be included in EPS format.
%
% If you use the hyperref package, please uncomment the following line
% to display URLs in blue roman font according to Springer's eBook style:
\renewcommand\UrlFont{\color{blue}\rmfamily}

\usepackage{xcolor}
\definecolor{mycolor}{HTML}{FF6600}
\definecolor{mycolor2}{HTML}{6699CC}
\definecolor{mycolor3}{HTML}{CC0000}
\newcommand{\herman}[1]{\textcolor{mycolor}{#1}}
\newcommand{\matthew}[1]{\textcolor{mycolor2}{#1}}


\begin{document}
%
\title{StarGAN-ZSVC: Towards Zero-Shot Voice Conversion in Low-Resource Contexts\thanks{This work is supported in part by the National Research Foundation of South Africa (grant number: 120409) and a Google Faculty Award for HK.}}
%
\titlerunning{StarGAN-ZSVC}
% If the paper title is too long for the running head, you can set
% an abbreviated paper title here
\author{Matthew Baas\orcidID{0000-0003-3001-6292} \and
Herman Kamper\orcidID{0000-0003-2980-3475}}
%
\authorrunning{M. Baas and H. Kamper}

\institute{E\&E Engineering, Stellenbosch University, Stellenbosch, South Africa\\
\email{\{20786379,kamperh\}@sun.ac.za}}

% First names are abbreviated in the running head.
% If there are more than two authors, 'et al.' is used.
\maketitle  

\begin{abstract}
Voice conversion is the task of converting a spoken utterance from a source speaker so that it appears to be said by a different target speaker while retaining the linguistic content of the utterance. Recent advances have led to major improvements in the quality of voice conversion systems. 
However, to be useful in a wider range of contexts, voice conversion systems would need to be (i)~trainable without access to parallel data, (ii) work in a zero-shot setting where both the source and target speakers are unseen during training, and (iii)~run in real time or faster. Recent techniques fulfil one or two of these requirements, but not all three. This paper extends recent voice conversion models based on generative adversarial networks (GANs), to satisfy all three of these conditions. We specifically extend the recent StarGAN-VC model by conditioning it on a speaker embedding (from a potentially unseen speaker). This allows the model to be used in a zero-shot setting, and we therefore call it StarGAN-ZSVC. We compare StarGAN-ZSVC against other voice conversion techniques in a low-resource setting using a small 9-minute training set.
Compared to AutoVC---another recent neural zero-shot approach---we observe that StarGAN-ZSVC gives small improvements in the zero-shot setting, showing that real-time zero-shot voice conversion is possible even for a model trained on very little data. Further work is required to see whether scaling up StarGAN-ZSVC will also improve zero-shot voice conversion quality in high-resource contexts.

\keywords{speech processing \and voice conversion \and generative adversarial networks \and zero-shot.}
\end{abstract}

% --- Content ----

\section{Introduction}

Many problems in econometrics, statistics, causal inference, and finance involve linear functionals of unknown functions:
\begin{equation}
\theta(g)=\E[m(Z; g)]
\end{equation}
where $Z$ denotes a random vector, and $g: \mcX\to \R$ is a function in some space $ \mcG$. A continuous linear functional that is mean square continuous with respect to $\ell_2$ norm can be written in a more benign and useful manner. Formally, for a given linear functional $\theta(\cdot)$, there exists a function $a_0$ such that for any $g\in \mcG$:\footnote{For simplicity of exposition, throughout the paper we consider scalar-valued functions $g$. All our results naturally extend to vector-valued functions $g$, and estimate a vector valued Riesz representer that satisfies that $\theta(g)=\E[a(X)'g(X)]$.}
\begin{equation}
    \theta(g) = \E[a_0(X)\, g(X)]
\end{equation}
This result is known as the Riesz representation theorem, and the function $a_0$ is the Riesz representer of the linear functional. Evaluation of a linear functional $\theta(g)$ can be achieved by simply taking the inner product between $a_0$ and $g$.

Knowing the Riesz representation of a linear functional is a critical building block in a variety of learning problems. For instance, in semi-parametric models, $g_0$ is an unknown regression function and $\theta(g_0)$ is a causal or structural parameter of interest. The Riesz representer $a_0$ of the functional $\theta(\cdot)$ can be used to debias the plug-in estimator and construct semi-parametrically efficient estimators of the parameter $\theta(g_0)$. In asset pricing applications, the Riesz representer corresponds to the stochastic discount factor, which is of primary interest when pricing financial derivatives.

Irrespective of the downstream application, the goal of this paper is to derive an estimator for the Riesz representer of any linear functional, when given access to $n$ samples of the random vector $Z$ and a target function space $\mcA$ that can well approximate the function $a_0$. We propose and analyze an estimator $\hat{a}$, with small mean-squared-error. Formally, with probability (w.p.) $1-\zeta$:
\begin{equation}
    \|\hat{a}-a_0\|_2 = \sqrt{\E\left[\left(\hat{a}(X) - a_0(X)\right)^2\right]} \leq \epsilon_{n,\zeta}
\end{equation}

We consider estimation of the Riesz representer within some function space $\mcA$ and propose an adversarial estimator based on regularized variants of the following min-max criterion:
\begin{equation}
    \hat{a} = \argmin_{a\in \mcA} \max_{f\in \mcF} \frac{1}{n}\sum_{i=1}^n \left(m(Z_i;f) - a(X_i)\cdot f(X_i) - f(X_i)^2\right)
\end{equation}
We derive oracle inequalities for this estimator as a function of the localized Rademacher complexity of the function space $\mcA$ and the approximation error $\epsilon = \min_{a\in \mcA} \|a-a_0\|_{2}$.

We show that as long as the function class $\mcF$ contains the star-hull of differences of functions in $\mcA$, i.e. $\mcF:= \{r(a-a'): a, a'\in \mcA, r\in [0, 1]\}$, then the estimation rate of the adversarial estimator achieves w.p. $1-\zeta$:
\begin{equation}
    \|\hat{a} - a_0\|_2 = O\left(\epsilon + \delta_n + \sqrt{\frac{\log(1/\zeta)}{n}}\right)
\end{equation}
where $\delta_n$ is the critical radius of the function classes $\mcF$ and $m\circ \mcF=\{Z\to m(Z; f): f\in \mcF\}$. The critical radius of a function class is a widely used quantity in statistical learning theory that allows one to argue fast estimation rates that are nearly optimal. For instance, for parametric function classes, the critical radius is of order $n^{-1/2}$, leading to fast parametric rates (as compared to $n^{-1/4}$ which would be achievable via looser uniform deviation bounds).

Moreover, the critical radius has been analyzed and derived for a variety of function spaces of interest, such as neural networks, high-dimensional linear functions, reproducing kernel Hilbert spaces, and VC-subgraph classes. Thus our general theorem allows us to appeal to these characterizations and provide oracle rates for a family of Riesz representer estimators. Prior work on estimating Riesz representers only considered particular high-dimensional parametric classes and derived specialized estimators for the function space of interest. Our adversarial estimator provides a single approach that tackles generic function spaces in a uniform manner.

We also examine the computational aspect of our estimator. We provide examples of how estimation can be achieved in a computationally efficient manner for several function spaces of interest.

Finally, we show how our estimator can be used in the context of estimating causal or structural parameters in semi-parametric models. Specifically, our mean square rate for the Riesz representer is sufficiently fast to achieve semi-parametric efficiency and asymptotic normality of the causal or structural parameter.

\subsection{Applications: Causal Inference and Asset Pricing}\label{sec:intro_examples}

This learning problem arises in two important domains for economic research: causal inference and asset pricing.

\paragraph{Automated De-biasing of Causal Estimates.} In causal inference, a variety of treatment effects and policy effects can be formulated as functionals--i.e., scalar summaries--of an underlying regression \cite{chernozhukov2016locally}. Formally, the causal parameter $\theta_0=\theta(g_0)=\mathbb{E}[m(Z;g_0)]$ is a functional $\theta(\cdot)$ of the nuisance parameter $g_0(x):=\mathbb{E}[Y|X=x]$. In this paper, we consider a variety of treatment and policy effects including
\begin{enumerate}
    \item Average treatment effect (ATE): $\theta_0=\mathbb{E}[g_0(1,W)-g_0(0,W)]$, where $X=(D,W)$ consists of treatment and covariates.
    \item Average policy effect: $\theta_0=\int g_0(x)d\mu(x)$ where $\mu(x)=F_1(x)-F_0(x)$
    \item Policy effect from transporting covariates: $\theta_0=\mathbb{E}[g_0(t(X))-g_0(X)]$
    \item Cross effect: $\theta_0=\mathbb{E}[Dg_0(0,W)]$, where $X=(D,W)$ consists of treatment and covariates.
    \item Regression decomposition: $\mathbb{E}[Y|D=1]-\mathbb{E}[Y|D=0]=\theta_0^{response}+\theta_0^{composition}$
    where
    \begin{align}
        \theta_0^{response}&=\mathbb{E}[g_0(1,W)|D=1]-\mathbb{E}[g_0(0,W)|D=1] \\
        \theta_0^{composition}&=\mathbb{E}[g_0(0,W)|D=1]-\mathbb{E}[g_0(0,W)|D=0]
    \end{align}
    \item Average treatment on the treated (ATT): $\theta_0=\mathbb{E}[g_0(1,W)|D=1]-\mathbb{E}[g_0(0,W)|D=1]$, where $X=(D,W)$ consists of treatment and covariates.
    \item Local average treatment effect (LATE): $\theta_0=\frac{\mathbb{E}[g_0(1,W)-g_0(0,W)]}{\mathbb{E}[h_0(1,W)-h_0(0,W)]}$, where $X=(V,W)$ consists of instrument and covariates and $h_0(x):=\mathbb{E}[D|X=x]$ is a second regression.
\end{enumerate}
More generally, our results extend to parameters defined implicitly by $0=\mathbb{E}[m(Z;g_0;\theta_0)]$, such as partially linear regression and partially linear instrumental variable regression.

    If the regression $g_0$ is learned by a regularized estimator $\hat{g}$, then estimation of the causal parameter $\theta_0$  by a plug-in estimator $\mathbb{E}_n[m(Z;\hat{g})]$ is badly biased. The solution is to use a de-biased formulation of the causal parameter instead: $\theta_0=\mathbb{E}[m(Z;g_0)+a_0(X)\{Y-g_0(X)\}]$. Observe that $a_0$ arises in the bias correction term. We re-visit this example in Section~\ref{sec:debiasing}.

%

\paragraph{Fundamental Asset Pricing Equation.} In asset pricing, a variety of financial models deliver the same fundamental asset pricing equation. This equation is of both theoretical and practical interest. Theoretically, it elucidates why asset prices or returns are what they are. Practically, it can be used to identify trading opportunities when assets are mis-priced. The asset pricing equation follows from two weak assumptions: free portfolio formation, and the law of one price.  In Appendix~\ref{sec:finance}, we review the derivation for a general audience.\footnote{The same asset pricing equation can be derived from either a model of complete markets for contingent claims, or a model of investor utility maximization. Free portfolio formation is a weaker assumption on market structure than the existence of complete markets for contingent claims. The law of one price is a weaker assumption on preference structure than investor utility maximization. We present these additional derivations in Appendix~\ref{sec:finance}.}

Formally, the fundamental asset pricing equation is $p_{t,i}=\mathbb{E}_t[m_{t+1}x_{t+1,i}]$ where $p_{t,i}$ is the price of asset $i$ at time $t$, $x_{t+1,i}$ is payoff of asset $i$ at time $t+1$, and $m_{t+1}$ is the market-wide stochastic discount factor (SDF) at time $t+1$.\footnote{The SDF has many additional names: marginal rate of substitution, state price density, and pricing kernel. Each name corresponds to a different derivation of the asset pricing equation, starting from different first principles.} The expectation is conditional on information $(I_t,I_{t,i})$ known at time $t$:  $I_t$ are macroeconomic conditioning variables that are not asset specific, e.g. inflation rates and market return; $I_{t,i}$ are asset-specific characteristics, e.g. the size or book-to-market ratio of firm $i$ at time $t$. The asset pricing equation encompasses stocks, bonds, and options. We clarify its many instantiations below, where $d_{t+1}$ is dividend, $C$ is the call price, $S_T$ is the stock price at expiration, $K$ is the strike price. 

\begin{table}[H]
       \centering
       \begin{tabular}{|c||c|c|}
        \hline 
            Asset & Price $p_t$ & Payoff $x_{t+1}$ \\
             \hline 
            \hline
            Stock &$p_t$& $p_{t+1}+d_{t+1}$ \\
              Bond &$p_t$&$1$\\
             Option &$C$&$\max\{S_T-K,0\}$ \\
             \hline 
            Return & $1$& $R_{t+1}$ \\
            Excess return &0&$R^e_{t+1}$ \\
            \hline 
       \end{tabular}
       \caption{Generality of asset pricing equation}
       \label{tab:my_label}
   \end{table}
 
 The fundamental asset pricing equation can also be parametrized in terms of returns. If an investor pays one dollar for an asset $i$ today, the gross rate of return $R_{t+1,i}$ is how many dollars the investor receives tomorrow; formally, the price is $p_{t,i}=1$ and the payoff is $x_{t+1,i}=R_{t+1,i}$ by definition. Next consider what happens when an investor borrows a dollar today at the interest rate $R_{t+1}^f$ and buys an asset $i$ that gives the gross rate of return $R_{t+1,i}$ tomorrow. From the perspective of the investor, who paid nothing out-of-pocket, the price is $p_{t,i}=0$ while the payoff is the excess rate of return $R_{t+1,i}^e:=R_{t+1,i}-R_{t+1}^f$, leading to the asset pricing equation: $0=\mathbb{E}_t[m_{t+1}R^e_{t+1,i}]$.
 
 
 Following \cite{chen2019deep}, we focus on the latter excess return parametrization of the asset pricing equation. Taking expectations yields the unconditional moment restriction
$$
0=\mathbb{E}[m_{t+1}R^e_{t+1,i}z(I_t,I_{t,i})]=\mathbb{E}[\mathbb{E}[m_{t+1}|R^e_{t+1,i},I_t,I_{t,i}]R^e_{t+1,i}z(I_t,I_{t,i})],\quad \forall z(\cdot)
$$
Our framework nests this final expression. Specifically,
$$
\theta(g)=0,\quad g(R^e_{t+1,i},I_t,I_{t,i})=R^e_{t+1,i}z(I_t,I_{t,i}),\quad a_0(R^e_{t+1,i},I_t,I_{t,i})=\mathbb{E}[m_{t+1}|R^e_{t+1,i},I_t,I_{t,i}]
$$
By estimating $a_0$, which is the projection of the SDF onto excess returns and other available information, one can pin down the price of any hypothetical asset. 

%
%
%
%

\subsection{Related Work}

\textbf{Classical Semi-parametric Statistics.} Classical semi-parametric statistical theory studies the asymptotic properties of statistical quantities that are functionals of a density or a regression over a low-dimensional domain \cite{levit1976efficiency,hasminskii1979nonparametric,ibragimov1981statistical,pfanzagl1982lecture,klaassen1987consistent,robinson1988root,van1991differentiable,bickel1993efficient,newey1994asymptotic,robins1995semiparametric,vaart,bickel1988estimating,newey1998undersmoothing,ai2003efficient,newey2004twicing,ai2007estimation,tsiatis2007semiparametric,kosorok2007introduction,ai2012semiparametric}. Any continuous linear functional has a Riesz representer. In this classical theory, the Riesz representer appears in the influence function and therefore in the asymptotic variance of semi-parametric estimators \cite{newey1994asymptotic}. We depart from classical theory by considering the high-dimensional setting.

\textbf{De-biased Machine Learning and Targeted Maximum Likelihood.} Because the Riesz representer appears in the asymptotic variance of semi-parametric estimators, it can be incorporated into estimation to ensure semi-parametric efficiency. In practice, this can be achieved by introducing a de-biasing term into the estimating equation \cite{hasminskii1979nonparametric,bickel1988estimating,zhang2014confidence,belloni2011inference,belloni2014inference,belloni2014uniform,belloni2014pivotal,javanmard2014confidence,javanmard2014hypothesis,javanmard2018debiasing,van2014asymptotically,ning2017general,chernozhukov2015valid,neykov2018unified,ren2015asymptotic,jankova2015confidence,jankova2016confidence,jankova2018semiparametric,bradic2017uniform,zhu2017breaking,zhu2018linear}. In doubly robust estimating equations for regression functionals, the de-biasing term is the product between the Riesz representer and the regression residual \cite{robins1995analysis,robins1995semiparametric,van2006targeted,van2011targeted,luedtke2016statistical,toth2016tmle}. The more general principle at play is Neyman orthogonality: the learning problem for the functional of interest becomes orthogonal to the learning problems for both the regression and the Riesz representer \cite{neyman1959,neyman1979c,vaart,robins2008higher,zheng2010asymptotic,belloni2014uniform,belloni2014pivotal,chernozhukov2016locally,belloni2017program,chernozhukov2018double,foster2019orthogonal}.

De-biased machine learning and targeted maximum likelihood combine the algorithmic insight of doubly-robust moment functions with the algorithmic insight of sample splitting \cite{bickel1982adaptive,schick1986asymptotically,klaassen1987consistent,vaart,robins2008higher}.  In doing so, these frameworks facilitate a general analysis of residuals such that the target functional is $\sqrt{n}$-consistent under minimal assumptions on the estimators used for the regression and Riesz representer \cite{scharfstein1999adjusting,rubin2005general,rubin2006extending,van2006targeted,zheng2010asymptotic,van2011targeted,diaz2013targeted,van2014targeted,kennedy2017nonparametric,kennedy2020optimal}. In particular, any machine learning estimators are permitted that satisfy $\sqrt{n}\|\hat{g}-g_0\|_2\cdot\|\hat{a}-a_0\|_2\rightarrow 0$ \cite{chernozhukov2018double,chernozhukov2016locally}.

The Riesz representer may be a difficult object to estimate. Even for simple regression functionals such as policy effects, its closed form involves ratios of densities. In restricted models, where the regression is known to belong to a certain function class, there is the further difficulty of projecting the Riesz representer accordingly. A recent literature explores the possibility of directly estimating the Riesz representer, without estimating its components or even knowing its functional form \cite{robins2007comment,newey2018cross,athey2018approximate,chernozhukov2018global,chernozhukov2018learning,hirshberg2018debiased,hirshberg2019augmented,singh2019biased,rothenhausler2019incremental}. A crucial insight, on which we build, is that the Riesz representer is directly identified from data. 

\cite{hirshberg2019augmented} observe that to debias an average moment, it is sufficient to estimate an empirical analogue of the Riesz representer that approximately satisfies the Riesz representer moment equation on the $n$ samples. They propose a parametric min-max criterion to estimate $n$ parameters corresponding to the $n$ evaluations of the empirical Riesz representer. Unlike \cite{hirshberg2019augmented}, we provide a guarantee on learning the true Riesz representer, we approximate the Riesz representer within non-parametric function spaces, and our result therefore has broader application beyond causal inference. Importantly, \cite{hirshberg2019augmented} require that the same sample used to estimate the $n$ parameters is used in final stage estimation of the causal parameter. As such, the analysis requires that the regression function $g$ lies in a Donsker class--a restriction that precludes many machine learning estimators. By contrast, our adversarial estimator provides fast estimation rates with respect to the true Reisz representer and hence can be used in combination with cross-fitting and sample splitting to eliminate the Donsker assumption.


\textbf{Adversarial Estimation.} Riesz representation theorem can be viewed as a continuum of unconditional moment restrictions. The non-parametric instrumental variable problem, based on a conditional moment restriction, also implies a continuum of unconditional moment restrictions \cite{newey2003instrumental,hall2005nonparametric,blundell2007semi,chen2009efficient,darolles2011nonparametric,chen2012estimation,chen2015sieve,chen2018optimal}. A central insight of this work is that the min-max approach for conditional moment models may be adapted to the problem of learning the Riesz representer. In a min-max approach, the continuum of unconditional moment restrictions is enforced adversarially over a set of test functions \cite{goodfellow2014generative,arjovsky2017wasserstein,dikkala2020minimax}. 

The fundamental advantage of the min-max approach is its unified analysis over arbitrary function classes. In particular, via local Rademacher analysis, one can derive an abstract bound that encompasses sparse linear models, neural networks, and RKHS methods \cite{koltchinskii2000rademacher,bartlett2005local}. As such, the min-max approach is actually a family of algorithms adaptive to a variety of data settings with a unified guarantee \cite{negahban2012,lecue2017regularization,Lecue2018}. 

\textbf{Machine Learning Approaches to Causal Inference and Asset Pricing.} By pursuing a min-max approach, our work relates to previous work that incorporates a variety of machine learning methods into causal inference. Much work on de-biased machine learning focuses on sparse and approximately sparse models \cite{chernozhukov2018global,chernozhukov2018learning,chernozhukov2018plug}. A neural network estimator with mean square rate has been successfully used to learn the nuisance regression in semiparametric estimation \cite{chen1999improved,farrell2018deep} and to learn the structural function in nonparametric instrumental variable regression \cite{deepiv,bennett2019deep,dikkala2020minimax}. A more recent literature incorporates RKHS methods into causal inference due to their convenient closed form solutions and strong performance on smooth designs \cite{nie2017quasi,singh2019kernel,muandet2019dual,singh2020kernel,muandet2020kernel}.

Finally, our works provides a theoretical foundation for a growing literature that incorporates machine learning into asset pricing. We follow the asset pricing literature in framing the problem of learning a stochastic discount factor as the problem of learning a Riesz representer \cite{hansen1997assessing}. Specifically, we propose a deep min-max approach based on free portfolio formation and the law of one price \cite{bansal1993no,chen2019deep}. This approach differs from deep learning approaches that predict asset prices via nonparametric regression \cite{messmer2017deep,feng2018deep,gu2020autoencoder,bianchi2020bond}. Unlike previous work, we prove mean square rates for the stochastic discount factor, and we prove $\sqrt{n}$-consistency and semiparametric efficiency for expected asset prices.
\section{Related Work}
\label{sec:related_work}
% In this section, we review the related work, which includes graph neural networks, and robust graph neural networks. 

\subsection{Graph Neural Networks}
Graph Neural Networks (GNNs) have shown their great power in modeling graph structured data for various applications~\cite{wang2019semi,wang2018cross,zhao2020semi,dai2021say,zhao2021graphsmote}.
To generalize neural networks for graphs, two categories of GNNs are proposed, i.e., spectral-based~\cite{bruna2013spectral,henaff2015deep,kipf2016semi,levie2018cayleynets} and spatial-based~\cite{velivckovic2017graph,hamilton2017inductive,chen2018fastgcn,chiang2019cluster}. \citeauthor{bruna2013spectral} \cite{bruna2013spectral} first propose spectral-based GNNs by defining graph convolution with spectral graph theory. For instance, GCN~\cite{kipf2016semi} simplifies the convolutional operation by using the first order approximation. Spatial-based graph convolution is defined in spatial domain, which updates node representation by aggregating its neighbors' representations \cite{niepert2016learning,gilmer2017neural,hamilton2017inductive}. 
For example, self-attention of neighbor nodes is leveraged in graph attention network (GAT) \cite{velivckovic2017graph}. Moreover, various spatial methods are proposed to solve the scalability issue~\cite{chen2018fastgcn,chiang2019cluster} and learn deeper GNNs~\cite{chen2020simple}.  Recently, to alleviate the problem of lacking labeled nodes, many efforts are taken to explore GNNs using self-supervision, which aims to learn better node representations with pretext tasks~\cite{sun2019multi,li2018deeper,kim2021find,zhu2020self,jin2020self,dai2021towards}. For instance, superGAT~\cite{kim2021find} deploys edge prediction in GAT to guide the learning of attention for better representations. SE-GNN~\cite{dai2021towards} deploys contrastive learning to benefit the similarity modeling for self-explainable GNN.

Inspired by the great success of GNNs, methods that construct graphs and adopt GNNs for data without explicit relational structure are also explored~\cite{henaff2015deep,chen2019multi,jiang2019semi,dai2021nrgnn}. Generally, a graph would be built based on certain rules~\cite{henaff2015deep,chen2019multi} or be learned in an end-to-end model~\cite{jiang2019semi,dai2021nrgnn}. Our RS-GNN is inherently different from these methods as we eliminate/down-weight the noisy edges and predict the missing edges for robust GNNs on noisy graphs with limited labels. 

\subsection{Robust GNNs}
Although GNNs have obtained great achievements, they are vulnerable to adversarial attacks~\cite{wu2019adversarial,dai2018adversarial,zugner2018adversarial,zugner2019adversarial}. Based on the objective, the adversarial attacks on GNNs can be split into two categories, i.e., targeted attack~\cite{dai2018adversarial,zugner2018adversarial} and non-targeted attack~\cite{zugner2019adversarial}. Targeted attack methods aim to degrade the performance of the GNNs on target nodes. 
For instance, \textit{nettack}~\cite{zugner2018adversarial} adds adversarial perturbations to a graph to attack targeted nodes. Non-targeted attack aims to reduce the overall performance of GNNs. For example, \textit{metattack}~\cite{zugner2019adversarial} poisons the graph globally to achieve non-targeted attack with meta-learning. To defend against adversarial attacks, many efforts are taken recently~\cite{zhu2019robust,wu2019adversarial,entezari2020all,jin2020graph,tang2020transferring,zhang2020gnnguard}. \cite{wu2019adversarial} prune the perturbed edges based on Jaccard similarity of node features. Another preprocessing method by low-rank approximation of adjacent matrix is investigated~\cite{entezari2020all}. Pro-GNN~\cite{jin2020graph} is the most similar work to ours, which learns a clean graph structure by low-rank constraint. However, they only tackle the adversarial edges and their computational cost is very large due to the direct learning of the graph and the sparse low-rank constraint.
This work is inherently different from these methods as: (i) we study a novel problem of developing robust GNN for both noisy graphs and label sparsity issues; and (ii) the proposed RS-GNN simultaneously tackles the two issues by learning an link predictor to 
down-weight noisy edges and connecting nodes with high similarity to facilitate message-passing; 
and (iii) RS-GNN uses link predictor instead of direct graph learning to save computational cost. 
\section{StarGAN-ZSVC}
\label{sec:starganzsvc}

While StarGAN-VC and StarGAN-VC2 allows training with non-parallel data and runs sufficiently fast, it is limited in its ability to only perform voice conversion for speakers seen during training: while parallel $X_{\text{src}}$ and $X_{\text{trg}}$ utterance pairs are not required, the model can only synthesize speech for target speaker identities (specified as one-hot vectors) seen during training.
This could preclude the use of these models in many practical situations where zero-shot conversion is required between unseen speakers.
Conversely, AutoVC allows for such zero-shot prediction and is trained on non-parallel data, but it is implemented with a slow vocoder and
its performance suffers when trained on very little data. Combining the strengths of both of these methods, we propose the \textit{StarGAN zero-shot voice conversion} model -- StarGAN-ZSVC.

\subsection{Overcoming the Zero-shot Barrier}
To achieve voice conversion between multiple speakers, the original StarGAN-VC2 model creates an explicit embedding vector for each source-target speaker pairing, which is incorporated
as part of the generator $G$ and discriminator $D$.
This requires that each source-target speaker pairing is seen during training so that the corresponding embedding exists and has been trained -- prohibiting zero-shot voice conversion.
To overcome this hurdle, we instead infer separate source and target speaker embeddings, $\mathbf{s}_{\text{src}}$ and $\mathbf{s}_{\text{trg}}$, using a speaker encoder network $E$ -- similar to the approach followed in AutoVC (Section~\ref{sec:autovc}).

This framework is shown in Figure~\ref{fig:system-diagram}.
Utterances from unseen speakers (i.e.\  $X_{\text{src}}$ and $Y_{\text{trg}}$) are fed to the speaker encoder $E$, yielding embeddings for these new speakers, which are then used to condition $G$ and $D$, 
thereby enabling zero-shot conversion. 
The generator uses these embeddings to produce a converted Mel-spectrogram $X_{\text{src}\rightarrow \text{trg}}$ from a given source utterance's Mel-spectrogram $X_{\text{src}}$.


\begin{figure}[!t]
\includegraphics[width=\textwidth]{figures/system-diagram.pdf}
\caption{The StarGAN-ZSVC system framework. The speaker encoder network $E$ and the WaveGlow vocoder are pretrained on large speech corpora, while the generator $G$ and discriminator $D$ are trained on a 9-minute subset of the VCC dataset. During inference, arbitrary utterances for the source and target speaker are used to obtain source and target speaker embeddings, $\mathbf{s}_{\textrm{src}}$ and $\mathbf{s}_{\textrm{trg}}$. 
} \label{fig:system-diagram}
\end{figure}

$E$ is trained on a large corpus 
using a GE2E loss \cite{GE2E} which aims to simultaneously maximize distances between embeddings from different speakers while minimizing the distances between embeddings from utterances of the same speaker.
NVIDIA's WaveGlow \cite{waveglow} is used, which does not require any speaker information for vocoding and thus readily allows zero-shot conversion.

\subsection{Overcoming the Speed Barrier}
The speed of the full voice conversion system during inference is bounded by 
(a)~the speed of the generator $G$; 
(b)~the speed of converting the utterance between time and frequency domains, consisting of the initial conversion from time-domain waveform to Mel-spectrogram and the speed of the vocoder; 
and (c)~the speed of the speaker encoder $E$.
To ensure that the speed of the full system is at least real-time, each subsystem needs to be
faster than real-time.

\subsubsection{(a) Generator Speed.}
For the generator $G$ to be sufficiently fast, we design it to only include convolution, linear, normalization, and upscaling layers as opposed to a recurrent architecture like those used in AutoVC \cite{autovc}. 
By ensuring that the majority of layers are convolutions, we obtain better-than real-time speeds for the generator.

\subsubsection{(b) Vocoder and Mel-spectrogram Speed.} 
The choice of vocoder greatly affects computational cost. 
Higher-quality methods, such as those from the WaveNet family~\cite{wavenet}, are typically much slower than real-time, while purely convolutional methods such as MelGAN~\cite{MelGAN} are much faster 
but has poorer quality.
Often the main difference between the slower and faster methods is again the presence of traditional recurrent layers in the vocoder architecture. 

We opt for a reasonable middle-ground choice with the WaveGlow vocoder, which does have recurrent connections but does not use any recurrent layers with dense multiplications such as LSTM or Gated Recurrent Unit (GRU, another kind of recurrent cell architecture \cite{gru}) layers. We specifically use a pretrained WaveGlow network, as provided with 
the original paper \cite{waveglow}. 
Furthermore, the speed of the Mel-spectrogram transformation for the input audio is well faster than real-time due to the efficient nature of the fast Fourier transform and the multiplication by Mel-basis filters.

\subsubsection{(c) Speaker Encoder Speed.} 
The majority of research efforts into obtaining speaker embeddings involve models using slower recurrent layers, often making these encoder networks the bottleneck. 
We also make use of a recurrent stacked-GRU network as our speaker embedding network $E$.
However, we only need to obtain a single speaker embedding to perform 
any number of conversions involving that speaker.
We therefore treat this as a preprocessing step where we apply $E$ to a few arbitrary utterances from the target and source speakers, averaging the results to obtain target and source speaker embeddings, and use those same embeddings for all subsequent conversions.

We also design the speaker embeddings to be 256-dimensional vectors of unit length. If we were to use StarGAN-ZSVC downstream for data augmentation (where we want speech from novel speakers), we could then simply sample random unit-length vectors of this dimensionality  to use with the generator.

\begin{figure}[!t]
\includegraphics[width=\textwidth]{figures/architecture.pdf}
\caption{StarGAN-ZSVC's network architectures. The speaker encoder $E$ is a recurrent network similar to that used in the original GE2E paper, while the generator $G$ and discriminator $D$ are modified versions from the original StarGAN-VC2 architecture. Within layers, \texttt{k} and \texttt{s} represent kernel size and stride (for convolutions), \texttt{f} is the scaling factor (for pixel shuffle), and \texttt{h} and \texttt{c} are the height and channels of the output (for reshape layers). A number alongside a layer indicates the number of output channels (for convolutions), or output units (for linear and GRU layers). GLU layers split the input tensor in half along the \textit{channels} dimension. GSP, GLU, and SELU indicate global sum pooling, gated linear units, and scaled exponential linear units, respectively.} \label{fig:architecture}
\end{figure}

\subsection{Architecture}
With the previous considerations in mind, we design the generator $G$, discriminator $D$, and encoder network $E$, as shown in Figure~\ref{fig:architecture}.
The generator and discriminator are adapted from StarGAN-VC2 \cite{stargan-vc2}, while the speaker encoder
is adapted from the original model proposed for speaker identification \cite{GE2E}. 
Specifically, for $E$ 
we use a simple stacked GRU model, while for $D$ we use a projection discriminator \cite{projection_discriminator}. 
For $G$, we use the 2-1-2D generator from StarGAN-VC2 with a modified central set of layers, denoted by the \textit{Conditional Block} in the figure. 

These conditional blocks are intended to provide the network with a way to modulate the channels of an input spectrogram, with modulation factors conditioned on the specific source and target speaker pairing.
They utilize a convolutional layer 
followed by a modified conditional instance normalization layer \cite{CIN} and a gated linear unit \cite{glu}.

The modified conditional instance normalization layer performs the following operation on an input feature vector $\mathbf{f}$:
\begin{equation}
\text{CIN}(\mathbf{f}, \gamma, \beta) 
= \gamma \left( \frac{\mathbf{f} - \mu(\mathbf{f})}{\sigma (\mathbf{f}) } \right) + \beta
\end{equation}
where $\mu(\mathbf{f})$ and $\sigma(\mathbf{f})$ are respectively the scalar mean and standard deviation of vector $\mathbf{f}$, while $\gamma$ and $\beta$ are computed using two linear layers which derive their inputs from the speaker embeddings, as depicted in Figure~\ref{fig:architecture}. The above
is computed separately for each channel when the input feature contains multiple~channels.

For the discriminator, the source and target speaker embeddings are also fed through several linear layers and activation functions to multiply with the pooled output of $D$'s main branch.


\section{Experiments}
\subsection{Datasets and Metrics}

%dialogue task分四种, intent prediction, slot-filling, semantic parsing, and dialogue state tracking. 然后我们完成的数据集任务是relation extraction, emotional recognition, speech act classification,不太确定是不是都能解决

\textbf{DialogRE} \citep{yu-etal-2020-dialogue} is a relation extraction task based on 1,788 dialogues from the Friends transcript. Each pair of arguments can be classified as one of 36 possible relation types. For each of the 10,168 human-annotated entity pairs, the trigger words are also provided. % such as neighbours or siblings. (subject, object, relation type) triplets

\textbf{EmoryNLP} \citep{zahiri:18a} is an emotion detection task based on 12,606 utterances from the Friends transcript. Each utterance can be classified as one of seven emotions, e.g., joyful, scared. 
%The label is human-annotated based on the dialogue context. 

\textbf{DailyDialog} \citep{DailyDialog} is a dialogue database containing 13,118 simple English dialogues. Each utterance can be assigned an emotion label from seven categories (anger, surprise, etc.). 
%targets both emotion detection and act classification. DailyDialog The label is human-annotated based on the dialogue context.  (Inform, Questions, Directives, Commissive) 

\textbf{MELD} \citep{poria-etal-2019-meld} is an emotion detection task based on 13,000 sentences from the Friends transcript. Each utterance can be classified as one of eight emotions, such as sad, disgust. % or neutral. 

%The label is human-annotated based on the multimodal cues of the dialogue, including visual, audio and textual information. 
\begin{table}[t]
  % \setlength{\tabcolsep}{3.5pt}
% \footnotesize
% \scriptsize
  \centering
  \vspace{-0.35cm}
  \begin{tabular}{lllllll}
    \toprule
  \multirow{2}{*}{\textbf{Method}} &\multirow{2}{*}{\textbf{MELD}} & \multirow{2}{*}{\textbf{ENLP}} & \multirow{2}{*}{\textbf{DDialog}} & \multirow{2}{*}{\textbf{MRDA}} & \multicolumn{2}{c}{\textbf{DialogRE-Test}}      \\
    % \textbf{Method} &\textbf{MELD} & \textbf{ENLP} & \textbf{DDialog} &\textbf{MRDA} & \multicolumn{2}{c}{\textbf{DialogRE}}      \\
    %\cmidrule(r){2-5}
    \cmidrule(l){6-7}   
    % \cmidrule(l){2-2}  \cmidrule(l){3-3} \cmidrule(l){4-4} \cmidrule(l){5-5}  \cmidrule(l){6-7}  
      &  & & & & $F1$  & $F1_c$  \\
    \midrule
    % BERT &61.50	& 34.17	& 54.09& 91.0 & 58.5	& 53.2\\
    % +HiDialog & + 1.78	& +0.63 & +5.55 & +0.3 & + &\textbf{59.8}\\
    PHT  &61.90&-&	60.14&	\textbf{92.4}&- &- \\
    DialogXL  & 62.41 & 34.73 & 54.93 & -& - & -  \\
    % \midrule
    RoBERTa$_s$& 64.19&38.03	&61.65	&91.3	&71.3 & 63.7\\
+Intra-turn &\textbf{65.64}\textsubscript{\textcolor{green}{+1.45}}& \textbf{38.13}\textsubscript{\textcolor{green}{+0.1}} &\textbf{61.83}\textsubscript{\textcolor{green}{+0.28}}&91.5\textsubscript{\textcolor{green}{+0.2}} & \textbf{74.4}\textsubscript{\textcolor{green}{+3.1}}&\textbf{66.6}\textsubscript{\textcolor{green}{+2.9}}\\
    \bottomrule
  \end{tabular}
  % \vspace{-0.2cm}
  % \caption{All methods performance on 5 multi-turn dialogue-based understanding datasets: MELD, EmoryNLP (Weighted-F1), DailyDialog (Micro-F1), MRDA (Top-1 Acc.), DialogRE (F1 and F1$_c$), averaged over five runs. Performance gains over the RoBERTa$_s$ are highlighted in green.}
    \caption{All methods performance on 5 multi-turn dialogue-based understanding datasets: MELD, EmoryNLP, DailyDialog, MRDA, DialogRE, averaged over five runs. Performance gains over the RoBERTa$_s$ are highlighted in green.}
  \label{tab:exp-mtr}
  \vspace{-0.5cm}
\end{table}


\begin{minipage}[]{0.48\linewidth}
\footnotesize
\setlength{\tabcolsep}{7pt}
\begin{center}
\begin{tabular}{l c c} 
 \toprule
  {\textbf{Method}} & \textbf{F1} & \textbf{F1$_c$} \\
 \midrule
HiDialog                      & 77.1        & 68.2       \\ 
w/o attention mask & 76.5 (-0.6) & 67.9 (-0.3) \\
w/o special tokens & 75.6 (-1.5) & 67.4 (-0.8) \\
only intra-turn     & 74.4 (-2.7) & 66.6 (-1.6) \\
\bottomrule
\end{tabular}
\end{center}
\captionof{table}{Ablation Study on HiDialog components on DialogRE to evaluate the individual effect of turn-level attention, turn-level special tokens, and graph module. } %\textit{Ablation study. Turn-level} is omitted for brevity.}
\label{tab:ablation_structure}
\end{minipage}
\hspace*{0.1cm}
\begin{minipage}[]{0.48\linewidth}
\footnotesize
\setlength{\tabcolsep}{7pt}
\begin{center}
% \vspace{-0.5cm}
\begin{tabular}{lccc} 
\toprule
\textbf{Method} & \textbf{I} &  \textbf{II} &  \textbf{III}\\
\midrule
BERT & 42.5  & 60.7 & 65.6 \\
% BERT$_s$ & 46.5 & 61.5 & 69.4 \\
GDPNet  & 47.4  &59.8  &68.1 \\
RoBERTa$_s$ & 57.4 & 69.3 & 79.6 \\
TUCORE-GCN  &62.3  &\textbf{71.3}  &79.9 \\
\midrule
HiDialog & \textbf{76.6}  & 70.5	& \textbf{80.9}  \\
w/o graph module &65.5 & 69.9 & 79.4\\
\bottomrule
\end{tabular}
\end{center}
% \vspace{-0.3cm}
\captionof{table}{All methods performance on DialogRE. We break down the performance into three groups (I) asymmetric inverse relations, (II) symmetric inverse relations, and (III) others.}
\label{tab:symmetric}
\vspace{0.5cm}
\end{minipage}

\textbf{MRDA} \citep{MRDA} is a dialogue act task based on 75 hours of real-life meeting transcript. Each sentence is assigned a general dialogue act (topic change, repeat, etc.) and a specific dialogue act (apology, suggestion, etc.).  % explanation sympathy

\textbf{Metrics}. For DialogRE, F1 and F1$_c$ are used as evaluation metrics. F1$_c$ modifies F1 by taking an early part of the dialogue as input \cite{yu-etal-2020-dialogue}. For MELD and EmoryNLP, we use weighted-F1 as metrics. For DailyDialog, the Micro-F1 score excluding the neutral class is used as the metric. 
\subsection{Results and Analysis}
\textbf{Overall Performance}. We first evaluated HiDialog on the Dialogue Relation Extract (DRE) dataset, DialogRE \citep{yu-etal-2020-dialogue} and the Emotion Recognition in Conversation (ERC) dataset, MELD \citep{poria-etal-2019-meld}. We selected BERT \citep{bertbase}, GDPNet \citep{xue2021gdpnet}, RoBERTa$_s$ \citep{yu-etal-2020-dialogue}, SimpleRE \citep{SimpleRE}, and TUCORE-GCN \citep{lee2021graph} as baselines. As reported in Table \ref{tab:exp-re}, HiDialog established new state-of-the-art results on both datasets. 
On the DialogRE test set, HiDialog surpassed the previous SOTA, TUCORE-GCN, by 4\% in F1 and 2.3\% in F1$_c$. On the MELD dataset, HiDialog outperformed TUCORE-GCN by 1.5\% in weighted F1. 
%, surpassing the previous SOTA by 4\% in F1 and 2.3\% in F1$_c$ on the DialogRE test set, by 1.5\% in weighted F1 score on the MELD test set. 


\textbf{Towards Generality}.
% In view of the simplicity and effectiveness of the intra-turn modeling, it is expected to have a general use for further work on dialogue understanding. To validate this idea, we incorporate our intra-turn modeling into the baseline encoder, without any extra components such as the inter-turn module or speaker embeddings. For a fair comparison, only the global $[CLS]$ token embedding from the encoder output is fed into a softmax classifier to make a prediction. 
Our intra-turn modeling's simplicity suggests its potential as a valuable solution for enhancing dialogue understanding without the need for extra pre-training. To assess this claim, we integrated it into the baseline encoder without any additional components, such as an inter-turn module or speaker embeddings. For fair comparisons, only the encoder's global $[CLS]$ token was used in a softmax classifier for prediction.
%Our intra-turn modeling's simplicity and effectiveness suggest its potential as a valuable solution for enhancing dialogue understanding, without additional pre-training. 

% \begin{wraptable}{r}{8.2cm}
%   \setlength{\tabcolsep}{3.5pt}
% % \footnotesize
% \scriptsize
%   \centering
%   \vspace{-0.35cm}
%   \begin{tabular}{lllllll}
%     \toprule
%   \multirow{2}{*}{\textbf{Method}} &\multirow{2}{*}{\textbf{MELD}} & \multirow{2}{*}{\textbf{ENLP}} & \multirow{2}{*}{\textbf{DDialog}} & \multirow{2}{*}{\textbf{MRDA}} & \multicolumn{2}{c}{\textbf{DialogRE}}      \\
%     % \textbf{Method} &\textbf{MELD} & \textbf{ENLP} & \textbf{DDialog} &\textbf{MRDA} & \multicolumn{2}{c}{\textbf{DialogRE}}      \\
%     %\cmidrule(r){2-5}
%     \cmidrule(l){6-7}   
%     % \cmidrule(l){2-2}  \cmidrule(l){3-3} \cmidrule(l){4-4} \cmidrule(l){5-5}  \cmidrule(l){6-7}  
%       &  & & & & $F1$  & $F1_c$  \\
%     \midrule
%     % BERT &61.50	& 34.17	& 54.09& 91.0 & 58.5	& 53.2\\
%     % +HiDialog & + 1.78	& +0.63 & +5.55 & +0.3 & + &\textbf{59.8}\\
%     PHT  &61.90&-&	60.14&	\textbf{92.4}&- &- \\
%     DialogXL  & 62.41 & 34.73 & 54.93 & -& - & -  \\
%     % \midrule
%     RoBERTa$_s$& 64.19&38.03	&61.65	&91.3	&71.3 & 63.7\\
% +Intra-turn &\textbf{65.64}\textsubscript{\textcolor{green}{+1.45}}& \textbf{38.13}\textsubscript{\textcolor{green}{+0.1}} &\textbf{61.83}\textsubscript{\textcolor{green}{+0.28}}&91.5\textsubscript{\textcolor{green}{+0.2}} & \textbf{74.4}\textsubscript{\textcolor{green}{+3.1}}&\textbf{66.6}\textsubscript{\textcolor{green}{+2.9}}\\
%     \bottomrule
%   \end{tabular}
%   \vspace{-0.2cm}
%   % \caption{All methods performance on 5 multi-turn dialogue-based understanding datasets: MELD, EmoryNLP (Weighted-F1), DailyDialog (Micro-F1), MRDA (Top-1 Acc.), DialogRE (F1 and F1$_c$), averaged over five runs. Performance gains over the RoBERTa$_s$ are highlighted in green.}
%     \caption{All methods performance on 5 multi-turn dialogue-based understanding datasets: MELD, EmoryNLP, DailyDialog, MRDA, DialogRE, averaged over five runs. Performance gains over the RoBERTa$_s$ are highlighted in green.}
%   \label{tab:exp-mtr}
%   \vspace{-0.35cm}
% \end{wraptable}



% \begin{minipage}[]{0.5\linewidth}
% % \begin{figure}[h]
% % \vspace{-0.2cm}
% \footnotesize
% \centering
% \includegraphics[width=7cm]{sections/length.pdf}
% \vspace{-0.5cm}
% \captionof{figure}{Analysis of robustness of HiDialog tackling increasing utterance length compared to baseline TUCORE-GCN on DialogRE dataset.}
% \vspace{0.6cm}
% \label{fig:length}
% % \end{figure}
% \end{minipage}



We conducted the experiment on 5 datasets from 3 different tasks: DRE (DialogRE), ERC (MELD, EmoryNLP \citep{zahiri:18a}, DailyDialog \citep{DailyDialog}), and Dialogue Act Classification (MRDA \citep{MRDA}). We chose RoBERTa$_s$, Pretrained Hierarchical Transformer (PHT) \citep{chapuis2020hierarchical}, and DialogXL \citep{DialogXL} as baselines. Compared to PHT and DialogXL, both of which require additional pre-training to address the domain adaption gap, the performance of proposed intra-turn modeling is surprisingly good in all 5 datasets (Table \ref{tab:exp-mtr}). 

% Moreover, we conducted an ablation study and analysis, which further reveals HiDialog is good at handling asymmetric relations and robust against increasing utterance length (see Appendix \ref{ap:ablation}). 

% \begin{wraptable}{r}{7cm}
% % \scriptsize
% % \vspace{-0.5cm}
% % \setlength{\tabcolsep}{3.5pt}
% \begin{center}

% \begin{tabular}{l c c} 
%  \toprule
%   {\textbf{Method}} & \textbf{F1} & \textbf{F1$_c$} \\
%  \midrule
% HiDialog                      & 77.1        & 68.2       \\ 
% w/o attention mask & 76.5 (-0.6) & 67.9 (-0.3) \\
% w/o special tokens & 75.6 (-1.5) & 67.4 (-0.8) \\
% only intra-turn     & 74.4 (-2.7) & 66.6 (-1.6) \\
% \bottomrule
% \end{tabular}
% \end{center}
% % \vspace{-0.2cm}
% \caption{Ablation Study on HiDialog components on DialogRE to evaluate the individual effect of turn-level attention, turn-level special tokens, and graph module. } %\textit{Ablation study. Turn-level} is omitted for brevity.}
% \label{tab:ablation_structure}
% % \vspace{-0.2cm}
% \end{wraptable}



\textbf{Ablation study on components.} We conducted an ablation study on DialogRE to evaluate key components in HiDialog: turn-level attention, turn-level special tokens, and inter-turn module (Table \ref{tab:ablation_structure}). First, after we removed the turn-level attention mask, the performance slightly dropped. In this case, these special tokens are able to aggregate information from the entire sequence, thus they are not context-aware at the turn level. We experimented with removing intra-turn modeling, resulting in only one difference from the final HiDialog: here we used an average of corresponding token embeddings for initialization. The $F1$ score decreases by 1.5\% and the $F1_c$ score declines by 0.8\%.

\textbf{Analysis of relations}. We grouped the test set of DialogRE according to the relation types into three subsets: (I) asymmetric, when a relation type differs from its inversion (e.g. \textit{children} and \textit{parents});  (II) symmetric, when a relation type is the same as its inversion (e.g. \textit{spouse}); (III) other, when a relation type does not have inversion (e.g. \textit{age}). We compared the performance of our model with baselines and report the results in Table \ref{tab:symmetric}. As we can observe, there is a great performance increase in the asymmetric subset while the F1 score drops moderately for symmetric relations. This trend reverses when we remove the graph module in our method (i.e. symmetric $>$ asymmetric). 
% \clearpage

\textbf{Analysis of robustness against increasing utterance length.} With the hierarchical aggregation in HiDialog, each turn-level special token is enforced to capture intra-turn critical information regardless of the whole dialogue. This nature enables our method to handle dialogues of various lengths. Thus, we further divided the samples in the DialogRE test set into six groups according to their lengths and compared HiDialog against the previous SOTA, TUCORE-GCN. As shown in Figure \ref{fig:length}, our method consistently outperforms TUCORE-GCN in all groups, where the largest performance gap can be found in the group with less than 100 tokens. Moreover, TUCORE-GCN shows a great drop with an increase of length (i.e., from $[400,500)$ to $[500,+\infty)$), while HiDialog maintains decent performance for long sequences.

 \begin{figure}[t]
\centering
% \vspace{-0.2cm}
\footnotesize
\centering
\includegraphics[width=7cm]{sections/length.pdf}
% \vspace{-0.5cm}
\captionof{figure}{Analysis of robustness of HiDialog tackling increasing utterance length compared to baseline TUCORE-GCN on DialogRE dataset.}
% \vspace{0.6cm}
\label{fig:length}
\end{figure}


%Note that both PHT and DialogXL are pre-trained methods that require extra computation
%Moreover, it outperforms the current SOTA by 1.3\% in $F1$ and 0.7\% in $F1_c$ on the DialogRE dataset, by 0.3\% in weighted F1 on MELD.
%HiDialog performs well in managing asymmetric relations and handling longer utterances, as revealed in our ablation study (see Appendix \ref{ap:ablation}). It provides an effective solution for bridging the gap between pre-training on general corpora and dialogue understanding without additional computational costs or training data while maintaining good performance. 
%Considering that our turn-level attention is easy to adopt and does not introduce any parameters to the base encoder, we believe it can be used as a strong baseline or plug-in module for future work in the community.
%Our turn-level attention mechanism can be effortlessly integrated into the base encoder without introducing any new parameters. Thus, we anticipate that it will serve as a compelling baseline or plug-in module for upcoming research in this field.

\subsection{Results on YCB-Video}

\begin{table}[b]
    \centering
\begin{tabular}{l|ccc}
    \toprule
                           &    ADD-S  &    ADD(-S) \\\midrule
DenseFusion (per-pixel)    &     91.2  &     82.9   \\ 
DenseFusion (iterative)    &     93.2  &     86.1   \\
CosyPose                   &     89.8  &     84.5   \\
PVN3D                      &     95.5  &     91.8   \\     
FFB6D                      &     96.6  &     92.7   \\     
ES6D                       &     93.6  &     89.0   \\   
SyMFM6D                    & \tb{96.8} & \tb{94.1}  \\ 
\bottomrule
\end{tabular}
    \caption{Single-view results on YCB-Video using the AUC metrics for ADD-S and \mbox{ADD(-S)}. The best results are printed in bold.}
    \label{tab_ycbv_sv}
\end{table}

\cref{tab_ycbv_sv} compares the single-view performance of our SyMFM6D network with all baseline methods using the AUC of ADD-S and \mbox{ADD(-S)} on YCB-Video. Please note that MV6D corresponds to PVN6D in the single-view scenario. The results show that our approach copes very well with the dynamic camera setup of YCB-Video while outperforming all methods significantly. On the symmetry-aware \mbox{ADD(-S)} AUC metric, SyMFM6D outperforms the current state-of-the-art FFB6D by even \SI{1.5}{\%}. 
Please note that unlike DenseFusion (iterative) and CosyPose, our approach does not perform computationally expensive post processing or iterative refinement procedures.


To examine the effect of our symmetry-aware training procedure, we provide an object-wise evaluation of the three best single-view methods on YCB-Video in \cref{fig_ycb_sv_objects}. Please note that in single-view mode, our model architecture is the same as FFB6D except for our novel symmetry-aware loss function. 
The results show that not only most symmetric objects (highlighted in bold) are estimated more accurate but also most non-symmetric objects.
This indicates that there is a synergy effect which improves the keypoint detection for non-symmetric objects due to an improvement of the keypoint detection for symmetric objects.

\begin{table}[tbp]
    \vspace{2mm}
    \centering
    \begin{tabular}{l|ccc}
        \toprule 
        Object class  		   &   PVN3D  &   FFB6D  &  SyMFM6D  \\\midrule
        Master chef can        &    80.5  &    80.6  &\tb{80.7} \\
        Cracker box            &    94.8  &    94.6  &\tb{94.9} \\
        Sugar box              &    96.3  &\tb{96.6} &\tb{96.6} \\
        Tomato soup can        &    88.5  &\tb{89.6} &    87.9  \\
        Mustard bottle         &    96.2  &    97.0  &\tb{97.8} \\
        Tuna fish can          &    89.3  &    88.9  &\tb{92.3} \\
        Pudding box            &\tb{95.7} &    94.6  &    93.3  \\
        Gelatin box            &    96.1  &\tb{96.9} &    96.1  \\
        Potted meat can        &    88.6  &    88.1  &\tb{90.0} \\
        Banana                 &    93.7  &    94.9  &\tb{95.2} \\
        Pitcher base           &    96.5  &    96.9  &\tb{97.5} \\
        Bleach cleanser        &    93.2  &\tb{94.8} &    93.9  \\
        \tb{Bowl}              &    90.2  &    96.3  &\tb{96.4} \\
        Mug                    &    95.4  &    94.2  &\tb{95.7} \\
        Power drill            &    95.1  &    95.9  &\tb{96.4} \\
        \tb{Wood block}        &    90.4  &    92.6  &\tb{95.2} \\
        Scissors               &    92.7  &    95.7  &\tb{95.8} \\
        Large marker           &\tb{91.8} &    89.1  &    90.0  \\
        \tb{Large clamp}       &    93.6  &    96.8  &\tb{96.9} \\
        \tb{Extra large clamp} &    88.4  &\tb{96.0} &    95.3  \\
        \tb{Foam brick}        &    96.8  &    97.3  &\tb{97.6} \\\midrule
        ALL                    &    91.8  &    92.7  &\tb{94.1} \\\bottomrule
    \end{tabular} 
	\caption{Single-view results on YCB-Video evaluated for each object class individually using the \mbox{ADD(-S)} AUC metric. Symmetric objects and the best results are printed in bold.}
	\label{fig_ycb_sv_objects}
\end{table}

\cref{fig_ycbv_sv} shows a visualization of three scenes of YCB-Video with 6D pose ground truth, predictions of FFB6D, and predictions of our SyMFM6D network using only the depicted view. It can be seen that both FFB6D and SyMFM6D estimate very accurate poses as the scenes of YCB-Video contain only a few objects and not many occlusions. However, SyMFM6D predicts even more accurate poses than FFB6D due to our proposed symmetry-aware training procedure.

\begin{figure*}[htbp]
        \vspace{2mm}
	\centering
	\begin{minipage}{0.24\textwidth}
		\centering
		\textbf{Original View}
	\end{minipage}%
	\begin{minipage}{0.24\textwidth}
		\centering
		\textbf{Ground Truth}
	\end{minipage}%
	\begin{minipage}{0.24\textwidth}
		\centering
		\textbf{FFB6D}
	\end{minipage}%
	\begin{minipage}{0.24\textwidth}
		\centering
		\textbf{SyMFM6D}
	\end{minipage}% 
	\setkeys{Gin}{width=0.24\linewidth}
	\includegraphics{figures/ycb_sv/0055_001042_rgb_1800.png}\,%
	\includegraphics{figures/ycb_sv/0055_001042_gt_1800.png}\,%
	\includegraphics{figures/ycb_sv/0055_001042_FFB6D_1view_1800.png}\,%
	\includegraphics{figures/ycb_sv/0055_001042_SyMV6D_16sym_1view.png}\,%
	\vspace{0.7mm}
	
	\includegraphics{figures/ycb_sv/0051_000636_rgb_864.png}\,%
	\includegraphics{figures/ycb_sv/0051_000636_gt_864.png}\,%
	\includegraphics{figures/ycb_sv/0051_000636_FFB6D_1view_864.png}\,%
	\includegraphics{figures/ycb_sv/0051_000636_SyMV6D_16sym_1view_864.png}\,%
	\vspace{0.7mm}
	
	\includegraphics{figures/ycb_sv/0058_000553_rgb_2461.png}\,%
	\includegraphics{figures/ycb_sv/0058_000553_gt_2461.png}\,%
	\includegraphics{figures/ycb_sv/0058_000553_FFB6D_1view_2461.png}\,%
	\includegraphics{figures/ycb_sv/0058_000553_SyMV6D_16sym_1view.png}\,%
	\caption{Comparison of 6D pose predictions on single frames of the YCB-Video dataset.}
	\label{fig_ycbv_sv}
\end{figure*}

\cref{tab_ycbv_mv} compares our multi-view results with all multi-view baseline methods on YCB-Video using three and five input views.
We see that our approach with disabled symmetry training procedure already outperforms all previous multi-view methods significantly. Enabling the symmetry awareness further improves the results slightly. However, 
using more views does not improve the accuracy as most views of YCB-Video are very similar in which case additional views do not provide beneficial information while the learning problem of fusing different views becomes slightly harder.

\begin{table}[!hbt]
    \tabcolsep=1.35mm
    \centering
\begin{tabular}{l|cc|cc}
    \toprule 
                  & \multicolumn{2}{c|}{ADD-S}
                                          & \multicolumn{2}{c}{ADD(-S)} \\
                  &  3 views  &  5 views  &  3 views  &  5 views  \\\midrule
CosyPose          &     92.3  &     93.4  &     87.7  &     88.8  \\
MV6D              &     91.2  &     91.1  &     85.6  &     84.0  \\
SyMFM6D (no sym)  &     95.2  &     95.2  &     91.5  &     91.4  \\
SyMFM6D           & \tb{95.4} & \tb{95.4} & \tb{91.7} & \tb{91.6} \\
\bottomrule
\end{tabular}
    \caption{Quantitative multi-view results on YCB-Video. The best results are printed in bold.}
    \label{tab_ycbv_mv}
\end{table}



\subsection{Results on MV-YCB FixCam, WiggleCam and SymMovCam}

We show the quantitative results on the datasets MV-YCB FixCam, MV-YCB WiggleCam, and MV-YCB SymMovCam in \cref{tab_fixCam_wiggleCam}. It includes a comparison with two modified CosyPose (CP) versions with and without known camera poses as presented by \cite{mv6d}.
Our SyMFM6D network yields the best results on all metrics on all three datasets. This shows that SyMFM6D copes very well with the strong occlusions in the datasets. The results on WiggleCam are just slightly worse than on FixCam which demonstrates that our approach is robust towards inaccurately known camera poses.

On the novel SymMovCam dataset, our method outperforms the baselines by a much larger margin than on FixCam and WiggleCam. This is due to the symmetric objects in the datasets on which the keypoint estimation of the baseline methods is inaccurate. The results also prove that our approach is robust to very dynamic camera setups where the cameras are mounted at varying positions.


\begin{table*}[h]
	\tabcolsep=1.0mm
	\centering
	\begin{tabular}{r|cccccc|cccccc|ccccc}
    \toprule
                           &     \multicolumn{6}{c|}{MV-YCB FixCam}                        &      \multicolumn{6}{c|}{MV-YCB WiggleCam}                    &      \multicolumn{5}{c}{MV-YCB SymMovCam}            \\
                           &  PVN3D   &   FFB6D  &   CP   &    CP    &   MV6D   &   Ours   &  PVN3D   & FFB6D    &   CP   &   CP     &  MV6D    &  Ours    &  PVN3D   & FFB6D    &   Ours   &    MV6D  &  Ours    \\ 
    Number of views        &   1      &     1    &   3    &    3     &   3      &    3     &    1     & 1        &   3    &   3      &   3      &    3     &    1     &    1     &     1    &     3    &    3     \\
    Known cam poses        &\checkmark&\checkmark&$\times$&\checkmark&\checkmark&\checkmark&\checkmark&\checkmark&$\times$&\checkmark&\checkmark&\checkmark&\checkmark&\checkmark&\checkmark&\checkmark&\checkmark\\
    \midrule                                                                                      
    ADD-S AUC              &  81.3    &   82.3   &  90.8  &   91.9   &  96.9    &\tb{97.3} &   80.8   &   81.9   & 90.0   &  91.3    &    96.2  &\tb{96.7} &   75.0   &   79.9   &   80.6   &   92.8   & \tb{94.2}\\
    ADD(-S) AUC            &  74.9    &   76.3   &  82.4  &   84.6   &  94.8    &\tb{95.6} &   74.0   &   75.5   & 81.0   &  83.4    &    93.0  &\tb{94.2} &   68.5   &   75.6   &   76.7   &   88.7   & \tb{91.6}\\
    ADD-S \textless   ~\SI{2}{cm} &  82.1    &   83.6   &  92.9  &   93.0   &  98.8    &\tb{98.9} &   82.0   &   83.4   & 92.3   &  92.6    &    98.7  &\tb{98.8} &   77.2   &   81.1   &   81.9   &   96.3   & \tb{96.6}\\
    ADD(-S) \textless ~\SI{2}{cm} &  73.0    &   74.8   &  80.6  &   82.4   &  96.5    &\tb{96.8} &   72.4   &   74.0   & 78.9   &  81.6    &\tb{96.0} &\tb{96.0} &   64.5   &   74.5   &   76.3   &   91.6   & \tb{93.6}\\
    \bottomrule
	\end{tabular}
	\caption{Quantitative results on the datasets MV-YCB FixCam (left), MV-YCB WiggleCam (middle), and MV-YCB SymMovCam (right). The baseline CosyPose (CP) uses PVN3D as backend network as described in \cite{mv6d}. The best results per dataset are printed in bold.}
	\label{tab_fixCam_wiggleCam}
\end{table*}



\subsection{Keypoint Visualization}

\cref{fig_ycbv_sv_keypoints} shows predicted keypoints of FFB6D and SyMFM6D in a YCB-Video scene. We additionally visualize the keypoint proposals of each object in individual colors.
The resulting predicted keypoints are white, the target keypoints are black. You can see that both FFB6D and SyMFM6D predict very accurate keypoints on all non-symmetric objects. However, FFB6D fails to predict accurate keypoints on the large clamp which has one discrete rotational symmetry. This shortcoming of FFB6D is also apparent on other symmetric objects. We believe that this is caused by the ambiguities of the object poses resulting in ambiguous target keypoints which results in averaging over the multiple solutions given by the symmetry. Therefore, the training loss is minimized when predicting keypoints on the symmetric axis rather than predicting them on the desired target locations. SyMFM6D in contrast overcomes this problem by our novel symmetry-aware training procedure as it can be seen in \cref{fig_ycbv_sv_keypoints_SyMFM6D}.

\begin{figure}[!tbh]
  \centering
\begin{subfigure}[b]{0.49\columnwidth}
  \includegraphics[width=1.0\columnwidth]{figures/0054_000204_FFB6D_1view_keypoints_cropped.jpg}
   \caption{FFB6D}
   \label{fig_ycbv_sv_keypoints_FFB6D}
\end{subfigure}
\begin{subfigure}[b]{0.49\columnwidth}
  \centering
  \includegraphics[width=1.0\columnwidth]{figures/0054_000204_SyMFM6D_16sym_1view_keypoints_BestSym_cropped.jpg}
   \caption{SyMFM6D}
   \label{fig_ycbv_sv_keypoints_SyMFM6D}
   \end{subfigure}
	\caption{Visualization of the predicted keypoints on single frames of the YCB-Video dataset.} 
   \label{fig_ycbv_sv_keypoints}
\end{figure}


\subsection{Implementation Details and Runtime}

We trained our network up to seven days on four NVIDIA Tesla V100 GPUs with \SI{32}{GB} of memory. 
The network architecture of our SyMFM6D approach has 3.5 million trainable parameters and requires about \SI{46}{ms} for processing a single RGB-D image on a single GPU. 
Mean shift clustering and least-squares fitting for computing a 6D pose require additional \SI{14}{ms} per object. 
Please visit our previously mentioned GitHub repository for code, datasets, and further details.

 We propose a novel commonsense reasoning challenge, \textsc{RiddleSense}, which requires complex commonsense skills for reasoning about creative and counterfactual questions, coming with a large multiple-choice QA dataset.  
 We systematically evaluate recent commonsense reasoning methods over the proposed \textsc{RiddleSense} dataset, and find that the best model is still far behind human performance, suggesting that there is still much space for commonsense reasoning methods to improve.
 We hope \textsc{RiddleSense} can serve as a benchmark dataset for future research targeting complex commonsense reasoning and computational creativity.


\section*{Acknowledgements}
This research is supported in part by the Office of the Director of National Intelligence (ODNI), Intelligence Advanced Research Projects Activity (IARPA), via Contract No. 2019-19051600007, the DARPA MCS program under Contract No. N660011924033 with the United States Office Of Naval Research, the Defense Advanced Research Projects Agency with award W911NF-19-20271, and NSF SMA 18-29268. The views and conclusions contained herein are those of the authors and should not be interpreted as necessarily representing the official policies, either expressed or implied, of ODNI, IARPA, or the U.S. Government. We would like to thank all the collaborators in USC INK research lab and the reviewers for their constructive feedback on the work.

% ---- Bibliography ---- 
%
% BibTeX users should specify bibliography style 'splncs04'.
% References will then be sorted and formatted in the correct style.
%
% \bibliographystyle{splncs04}
% \bibliography{mybibliography}
%
% For citations of references, we prefer the use of square brackets
% and consecutive numbers. Citations using labels or the author/year
% convention are also acceptable. The following bibliography provides
% a sample reference list with entries for journal
% articles~\cite{ref_article1}, an LNCS chapter~\cite{ref_lncs1}, a
% book~\cite{ref_book1}, proceedings without editors~\cite{ref_proc1},
% and a homepage~\cite{ref_url1}. Multiple citations are grouped
% \cite{ref_article1,ref_lncs1,ref_book1},
% \cite{ref_article1,ref_book1,ref_proc1,ref_url1}.

\bibliography{References}{}
\bibliographystyle{splncs04}
\end{document}
