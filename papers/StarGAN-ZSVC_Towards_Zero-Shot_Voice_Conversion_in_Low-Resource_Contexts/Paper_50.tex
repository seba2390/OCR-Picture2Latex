% This is samplepaper.tex, a sample chapter demonstrating the
% LLNCS macro package for Springer Computer Science proceedings;
% Version 2.20 of 2017/10/04
%
\documentclass[runningheads]{llncs}
%
\usepackage{graphicx}
\usepackage{amssymb}
\usepackage{amsmath}
\usepackage{microtype}
\usepackage{booktabs}
\usepackage{multirow}
\usepackage{array}
\usepackage{hyperref}
\usepackage{cite}
\usepackage{microtype}
% Used for displaying a sample figure. If possible, figure files should
% be included in EPS format.
%
% If you use the hyperref package, please uncomment the following line
% to display URLs in blue roman font according to Springer's eBook style:
\renewcommand\UrlFont{\color{blue}\rmfamily}

\usepackage{xcolor}
\definecolor{mycolor}{HTML}{FF6600}
\definecolor{mycolor2}{HTML}{6699CC}
\definecolor{mycolor3}{HTML}{CC0000}
\newcommand{\herman}[1]{\textcolor{mycolor}{#1}}
\newcommand{\matthew}[1]{\textcolor{mycolor2}{#1}}


\begin{document}
%
\title{StarGAN-ZSVC: Towards Zero-Shot Voice Conversion in Low-Resource Contexts\thanks{This work is supported in part by the National Research Foundation of South Africa (grant number: 120409) and a Google Faculty Award for HK.}}
%
\titlerunning{StarGAN-ZSVC}
% If the paper title is too long for the running head, you can set
% an abbreviated paper title here
\author{Matthew Baas\orcidID{0000-0003-3001-6292} \and
Herman Kamper\orcidID{0000-0003-2980-3475}}
%
\authorrunning{M. Baas and H. Kamper}

\institute{E\&E Engineering, Stellenbosch University, Stellenbosch, South Africa\\
\email{\{20786379,kamperh\}@sun.ac.za}}

% First names are abbreviated in the running head.
% If there are more than two authors, 'et al.' is used.
\maketitle  

\begin{abstract}
Voice conversion is the task of converting a spoken utterance from a source speaker so that it appears to be said by a different target speaker while retaining the linguistic content of the utterance. Recent advances have led to major improvements in the quality of voice conversion systems. 
However, to be useful in a wider range of contexts, voice conversion systems would need to be (i)~trainable without access to parallel data, (ii) work in a zero-shot setting where both the source and target speakers are unseen during training, and (iii)~run in real time or faster. Recent techniques fulfil one or two of these requirements, but not all three. This paper extends recent voice conversion models based on generative adversarial networks (GANs), to satisfy all three of these conditions. We specifically extend the recent StarGAN-VC model by conditioning it on a speaker embedding (from a potentially unseen speaker). This allows the model to be used in a zero-shot setting, and we therefore call it StarGAN-ZSVC. We compare StarGAN-ZSVC against other voice conversion techniques in a low-resource setting using a small 9-minute training set.
Compared to AutoVC---another recent neural zero-shot approach---we observe that StarGAN-ZSVC gives small improvements in the zero-shot setting, showing that real-time zero-shot voice conversion is possible even for a model trained on very little data. Further work is required to see whether scaling up StarGAN-ZSVC will also improve zero-shot voice conversion quality in high-resource contexts.

\keywords{speech processing \and voice conversion \and generative adversarial networks \and zero-shot.}
\end{abstract}

% --- Content ----

\epigraph{\normalsize ``\textit{ \textbf{The essence of a riddle is to express true facts under impossible combinations.}}"}{\normalsize--- \textit{Aristotle}, \textit{Poetics} (350 BCE)\vspace{0pt}}

\noindent
A \textit{riddle} is a puzzling question about {concepts} in our everyday life.
% , and we which one needs common sense to reason about.
For example, a riddle might ask ``\textit{My life can be measured in hours. I serve by being devoured. Thin, I am quick. Fat, I am slow. Wind is my foe. What am I?}''~
The correct answer ``\textit{candle},'' is reached by considering a collection of \textit{commonsense knowledge}:
{a candle can be lit and burns for a few hours; a candle's life depends upon its diameter; wind can extinguish candles, etc.}
\begin{figure}[t]
	\centering 
	\includegraphics[width=1\linewidth]{riddle_intro_final.pdf}
	\caption{ 
    The top example is a trivial commonsense question from the CommonsenseQA~\cite{Talmor2018CommonsenseQAAQ} dataset. 
    The two bottom examples are from our proposed \textsc{RiddleSense} challenge.
    The right-bottom question is a descriptive riddle that implies multiple commonsense facts about \textit{candle}, and it needs understanding of figurative language such as metaphor;
    The left-bottom one additionally needs counterfactual reasoning ability to address the \textit{`but-no'} cues. 
    These riddle-style commonsense questions  require NLU systems to have higher-order reasoning skills with the understanding of creative language use.
	}
	\label{fig:intro} 
\end{figure}

It is believed that the \textit{riddle} is one of the earliest forms of oral literature,
which can be seen as a formulation of thoughts about common sense, a mode of association between everyday concepts, and a metaphor as higher-order use of natural language~\cite{hirsch2014poet}.
Aristotle stated in his \textit{Rhetoric} (335-330 BCE) that good riddles generally provide satisfactory metaphors for rethinking common concepts in our daily life.
He also pointed out in the \textit{Poetics} (350 BCE): ``the essence of a riddle is to express true facts under impossible combinations,'' which suggests that solving riddles is a nontrivial  reasoning task.

Answering riddles is indeed a challenging cognitive process as it requires \textit{complex} {commonsense reasoning skills}.
% which we refer to \textit{higher-order commonsense reasoning}. 
% A successful riddle-solving model should be able to reason with \textit{multiple pieces} of commonsense facts, as 
A riddle can describe \textit{multiple pieces} of commonsense knowledge with \textit{figurative} devices such as metaphor and personification (e.g., ``wind is my \underline{foe} $\xrightarrow[]{}$ \textit{extinguish}'').
% , as shown by the examples in Figure~\ref{fig:intro}.
%%%
Moreover, \textit{counterfactual thinking} is also necessary for answering many riddles such as ``\textit{what can you hold in your left hand \underline{but not} in your right hand? $\xrightarrow[]{}$ your right elbow.}''
These riddles with \textit{`but-no'} cues require that models use counterfactual reasoning ability to consider possible solutions for situations or objects that are seemingly impossible at face value.
This \textit{reporting bias}~\cite{gordon2013reporting} makes riddles a more difficult type of commonsense question for pretrained language models to learn and reason.
% In addition, the model needs to associate commonsense knowledge with the creative use of language in descriptions, which may have figurative devices such as metaphor and personification (e.g., ``wind is my \underline{foe} $\xrightarrow[]{}$ \textit{extinguish}''). 
%For instance, one needs to know that devour
% Thus, a riddle here can be seen as a complex commonsense question that tests higher-order reasoning ability with creativity.
In contrast, \textit{superficial} commonsense questions such as ``\textit{What home entertainment equipment requires cable?}'' in  CommonsenseQA~\cite{Talmor2018CommonsenseQAAQ} are more straightforward and explicitly stated.
We illustrate this comparison in Figure~\ref{fig:intro}.


In this paper,
we introduce the \textsc{RiddleSense} challenge 
to study the task of answering riddle-style commonsense questions\footnote{We use ``riddle'' and ``riddle-style commonsense question'' interchangeably in this paper.} requiring \textit{creativity}, \textit{counterfactual thinking} and \textit{complex commonsense reasoning}.
\textsc{RiddleSense} is presented as a \textit{multiple-choice question answering} task where a model selects one of five answer choices to a given riddle question as its predicted answer, as shown in Fig.~\ref{fig:intro}.
We construct the dataset by first crawling from several free websites featuring large collections of human-written riddles and then aggregating, verifying, and correcting these examples using a combination of human rating and NLP tools to create a dataset consisting of 5.7k high-quality examples.
Finally, we use \textit{Amazon Mechanical Turk} to crowdsource quality distractors to create a challenging benchmark.
We show that our riddle questions are more challenging than {CommonsenseQA} by analyzing graph-based statistics over ConceptNet~\cite{Speer2017ConceptNet5A}, a large knowledge graph for common sense reasoning.

% The distractors for the training data are automatically generated from ConceptNet and language models while the distractors for the dev and the test sets are crowd-sourced from Amazon Mechanical Turk (AMT).
% Through data analysis based on graph connectivity, 




Recent studies have demonstrated that
 fine-tuning large pretrained language models, such as {BERT}~\cite{Devlin2019}, RoBERTa, and ALBERT~\cite{Lan2020ALBERT}, can achieve strong results on current commonsense reasoning benchmarks.
Developed on top of these language models, graph-based language reasoning models such as KagNet~\cite{kagnet-emnlp19} and MHGRN~\cite{feng2020scalable} show superior performance. 
Most recently, UnifiedQA~\cite{khashabi2020unifiedqa} proposes to unify different QA tasks and train a text-to-text model for learning from all of them, which achieves state-of-the-art performance on many commonsense benchmarks.

To provide a comprehensive benchmarking analysis, we systematically compare the above methods.
Our experiments reveal that while humans achieve 91.33\% accuracy on \textsc{riddlesense}, the best language models can only achieve 68.80\% accuracy, suggesting that there is still much room for improvement in the field of solutions to complex commonsense reasoning questions with language models.
% We also provide error analysis to better understand the limitation of current methods.
We believe the proposed \textsc{RiddleSense} challenge suggests productive future directions for machine commonsense reasoning as well as the understanding of higher-order and creative use of natural language.


% (previous state-of-the-art on \texttt{CommonsenseQA} (56.7\%)).
% However, there still exists a large gap between performance of said baselines and human performance.
% we show that the questions in RiddleSense is significantly more challenging, in terms of the length of the paths from question concepts and answer concepts.


%Apart from that, current pre-trained language models (e.g., BERT~\cite{}, RoBERTa~\cite{}, etc.) and commonsense-reasoning models (e.g., KagNet~\cite{}), can be easily adapted to work for this format with minimal modifications. 


%Note that these auto-generated distractors may be still easy for , which could diminish the testing ability of the dataset.
%We design an ader filtering method to get rid of the false negative   and control the task difficulty. 
% To strengthen the task, we propose an adversarial cross-filtering method to remove the distractors that ineffectively mislead the selected base models.
% Finally, we use human efforts to inspect the distractors and remove false negative ones, to make sure that all distractors either does not make sense or much less plausible than the correct answers.
%Introducing these fine-tuned models is inspired by the adversarial filtering algorithms~\cite{}, which can effectively reduce the  bias inside datasets for creating a more reliable benchmark.  



%Those distractors are explicitly annotated by human experts such that they are close to the meaning of 
%The main idea is to use multiple trainable generative models for learning to generate answers in a cross-validation style. 
%The wrong predictions
%Simply put, for every step, we use a large subset of the riddles and their current options ot learn multiple models for answering the remaining riddles via generation.
%After each step, we consolidate the 


% In the distantly supervised learning, we use the definition of concepts (i.e., glossary) of \textit{Wiktionary}\footnote{\url{https://www.wiktionary.org/}} to create riddles with answers as training data. 
% In the transfer learning setting, we aim to test the transferability of models across relevant datasets, such as CommonsenseQA~\cite{Talmor2018CommonsenseQAAQ}.

% We believe the \textsc{RiddleSense} task can benefit multiple communities in natural language processing. 
% First, the commonsense reasoning community can use \textsc{RiddleSense} as a new space to evaluate their reasoning models. The \textsc{RiddleSense} focuses on more complex and creative commonsense questions, which will encourage them to propose more higher-order commonsense reasoning models. 
% Second, \textsc{RiddleSense} is an NLU 
% task similar to those in the GLUE~\cite{wang2018glue} and SuperGLUE~\cite{wang2019superglue} leaderboard that can serve as a benchmark for testing various pre-trained language models.
% Last but not the least, as our task shares the similar format with many open-domain question answering tasks like \textit{Natural Questions}~\cite{kwiatkowski2019natural}, researchers in QA area may be also interested in \textsc{RiddleSense}. 





\section{Related work}
\label{sec:related_work}

Accessibility is an essential component of computing, which aims to make technology broadly accessible to as many users as possible, including those with differing sets of abilities. Improvements in usability and accessibility falls to the community, to better understand the needs of users with differing abilities, and to design technologies that play to this spectrum of abilities \citep{Wobbrock2011AbilityBasedDC}.
In computing, significant strides have been made to increase the accessibility of web content. For example, various versions of the Web Content Accessibility Guidelines (WCAG) \citep{Chisholm2001WebCA, Caldwell2008WebCA} and the in-progress working draft for WCAG 3.0,\footnote{\href{https://www.w3.org/TR/wcag-3.0/}{https://www.w3.org/TR/wcag-3.0/}} or standards such as ARIA from the W3C's Web Accessibility Initiative (WAI)\footnote{\href{https://www.w3.org/WAI/standards-guidelines/aria/}{https://www.w3.org/WAI/standards-guidelines/aria/}} have been released and used to guide web accessibility design and implementation. Similarly, positive steps have been made to improve the accessibility of user interfaces and user experience \citep{Peissner2012MyUIGA, Peissner2013UserCI, Thompson2014ImprovingTU, Bigham2014MakingTW}, as well as various types of media content \citep{Mirri2017TowardsAG, Nengroo2017AccessibleI, Gleason2020TwitterAA}. 

We take inspiration from accessibility design principles in our effort to make research publications more accessible to users who are blind and low vision. Blindness and low vision are some of the most common forms of disability, affecting an estimated 3--10\% of Americans depending on how visual impairment is defined \citep{CDCVisionLossBurden}. BLV researchers also make up a representative sample of researchers in the United States and worldwide. A recent Nature editorial pushes the scientific community to better support researchers with visual impairments \citep{NatureCareerColumn2020}, since existing tools and resources can be limited. There are many inherent accessibility challenges to performing research. In this paper, we engage with one of these challenges that affects all domains of study, accessing and reading the content of academic publications. 

BLV users interact with papers using screen readers, braille displays, text-to-speech, and other assistive tools. A WebAIM survey of screen reader users found that the vast majority (75.1\%) of respondents indicate that PDF documents are very or somewhat likely to pose significant accessibility issues.\footnote{\href{https://webaim.org/projects/screenreadersurvey8/}{https://webaim.org/projects/screenreadersurvey8/}} Most paper are published in PDF, which is inherently inaccessible, due in large part to its conflation of visual layout information with semantic content \citep{NielsenPDFStillUnfit, Bigham2016AnUT}. 
\citet{Bigham2016AnUT} describe the historical reasons we use PDF as the standard document format for scientific publications, as well as the barriers the format itself presents to accessibility. Prior work on scientific accessibility have made recommendations for how to make PDFs more accessible \cite{Rajkumar2020PDFAO, Darvishy2018PDFAT}, including greater awareness for what constitutes an accessible PDF and better tooling for generating accessible PDFs. Some work has focused on addressing components of paper accessibility, such as the correct way for screen readers to interpret and read mathematical equations \citep{Flores2010MathMLTA, Bates2010SpokenMU, Sorge2014TowardsMM, Mackowski2017MultimediaPF, Ahmetovic2018AxessibilityAL, Ferreira2004EnhancingTA, Sojka2013AccessibilityII}, describe charts and figures \citep{Elzer2008AccessibleBC, Engel2017TowardsAC, Engel2019SVGPlottAA}, automatically generate figure captions \citep{Chen2019NeuralCG, Qian2020AFS}, or automatically classify the content of figures \citep{Kim2018MultimodalDL}. Other work applicable to all types of PDF documents aims to improve automatic text and layout detection of scanned documents \cite{Nazemi2014PracticalSM} and extract table content \cite{Fan2015TableRD, Rastan2019TEXUSAU}. In this work, we focus on the issue of representing overall document structure, and navigation within that structure. Being able to quickly navigate the contents of a paper through skimming and scanning is an essential reading technique \citep{Maxwell1972SkimmingAS}, which is currently under-supported by PDF documents and PDF readers when reading these documents by screen reader. 

There also exists a variety of automatic and manual tools that assess and fix accessibility compliance issues in PDFs, including the Adobe Acrobat Pro Accessibility Checker\footnote{\href{https://www.adobe.com/accessibility/products/acrobat/using-acrobat-pro-accessibility-checker.html}{https://www.adobe.com/accessibility/products/acrobat/using-acrobat-pro-accessibility-checker.html}}, Common Look\footnote{\href{https://monsido.com/monsido-commonlook-partnership}{https://monsido.com/monsido-commonlook-partnership}}, ABBYY FineReader\footnote{\href{https://pdf.abbyy.com/}{https://pdf.abbyy.com/}}, PAVE\footnote{\href{https://pave-pdf.org/faq.html}{https://pave-pdf.org/faq.html}}, and PDFA Inspector\footnote{\href{https://github.com/pdfae/PDFAInspector}{https://github.com/pdfae/PDFAInspector}}. To our knowledge, PAVE and PDFA Inspector are the only non-proprietary, open-source tools for this purpose. Based on our experiences, however, all of these tools require some degree of human intervention to properly tag a scientific document, and tagging and fixing must be performed for each new version of a PDF, regardless of how minor the change may be.

Guidelines and policy changes have been introduced in the past decade to ameliorate some of the issues around scientific PDF accessibility. Some conferences, such as The ACM CHI Virtual Conference on Human Factors in Computing Systems (CHI) and The ACM SIGACCESS Conference on Computers and Accessibility (ASSETS), have released guidelines for creating accessible submissions.\footnote{See \href{http://chi2019.acm.org/authors/papers/guide-to-an-accessible-submission/}{http://chi2019.acm.org/authors/papers/guide-to-an-accessible-submission/} and \href{https://assets19.sigaccess.org/creating_accessible_pdfs.html}{https://assets19.sigaccess.org/creating\_accessible\_pdfs.html}} The ACM Digital Library\footnote{\href{https://dl.acm.org/}{https://dl.acm.org/}} provides some publications in HTML format, which is easier to make accessible than PDF~\cite{Graells2007EstudioDL}. \citet{Ribera2019PublishingAP} conducted a case study on DSAI 2016 (Software Development and Technologies for Enhancing Accessibility and Fighting Infoexclusion). The authors of DSAI were responsible for creating accessible proceedings and identified barriers to creating accessible proceedings, including lack of sufficient tooling and lack of awareness of accessibility. The authors recommended creating a new role in the organizing committee dedicated to accessible publishing. These policy changes have led to improvements in localized communities, but have not been widely adopted by all academic publishers and conference organizers.

Table~\ref{tab:prior_work} lists prior studies that have analyzed PDF accessibility of academic papers, and shows how our study compares. Prior work has primarily focused on papers published in Human-Computer Interaction and related fields, specific to certain publication venues, while our analysis tries to quantify paper accessibility more broadly.
\citet{Brady2015CreatingAP} quantified the accessibility of 1,811 papers from CHI 2010-2016, ASSETS 2014, and W4A, assessing the presence of document tags, headers, and language. They found that compliance improved over time as a response to conference organizers offering to make papers accessible as a service to any author upon request. \citet{Lazar2017MakingTF} conducted a study quantifying accessibility compliance at CHI from 2010 to 2016 as well as ASSETS 2015,
%\jb{Define acronyms in prev para}
confirming the results of \citet{Brady2015CreatingAP}. They found that across 5 accessibility criteria, the rate of compliance was less than 30\% for CHI papers in each of the 7 years that were studied. The study also analyzed papers from ASSETS 2015, an ACM conference explicitly focused on accessibility, and found that those papers had significantly higher rates of compliance, with over 90\% of the papers being tagged for correct reading order and no criteria having less than 50\% compliance. This finding indicates that community buy-in is an important contributor to paper accessibility.
\citet{Nganji2015ThePD} conducted a study of 200 PDFs of papers published in four disability studies journals, finding that accessibility compliance was between 15-30\% for the four journals analyzed, with some publishers having higher adherence than others. To date, no large scale analysis of scientific PDF accessibility has been conducted outside of disability studies and HCI, due in part to the challenge of scaling such an analysis. We believe such an analysis is useful for establishing a baseline and characterizing routes for future improvement. Consequently, as part of this work, we conduct an analysis of scientific PDF accessibility across various fields of study, and report our findings relative to prior work. 


\begin{table}[t!]
\small
    \centering
    \begin{tabularx}{\linewidth}{L{22mm}L{15mm}L{48mm}L{16mm}L{34mm}}
        \toprule
        \textbf{Prior work} & \textbf{PDFs analyzed} & \textbf{Venues} & \textbf{Year} & \textbf{Accessibility checker} \\
        \midrule
        \citet{Brady2015CreatingAP} & 1811 & CHI, ASSETS and W4A & 2011--2014 & PDFA Inspector \\ [0.5mm]
        \hline \\ [-2.5mm]
        \citet{Lazar2017MakingTF} & 465 + 32 & CHI and ASSETS & 2014--2015 & Adobe Acrobat Action Wizard \\ [0.5mm]
        \hline \\ [-2.5mm]
        \citet{Ribera2019PublishingAP} & 59 & DSAI & 2016 & Adobe PDF Accessibility Checker 2.0 \\ [0.5mm]
        \hline \\ [-2.5mm]
        \citet{Nganji2015ThePD} & 200 & \textit{Disability \& Society}, \textit{Journal of Developmental and Physical Disabilities}, \textit{Journal of Learning Disabilities}, and \textit{Research in Developmental Disabilities} & 2009--2013 & Adobe PDF Accessibility Checker 1.3 \\ [0.6mm]
        \hline \\ [-2.5mm]
        \textbf{\textit{Our analysis}} & \numpdfs & Venues across various fields of study & 2010--2019 & Adobe Acrobat Accessibility Plug-in Version 21.001.20145 \\
        \bottomrule
    \end{tabularx}
    \caption{Prior work has investigated PDF accessibility for papers published in specific venues such as CHI, ASSETS, W4A, DSAI, or various disability journals. Several of these works were conducted manually, and were limited to a small number of papers, while the more thorough analysis was conducted for CHI and ASSETS, two conference venues focused on accessibility and HCI. Our study expands on this prior work to investigate accessibility over \numpdfs PDFs sampled from across different fields of study.
    }
    % \Description{
    % Prior work, PDFs analyzed, Venues, Year, Accessibility checker 
    % Brady et al. [7], 1811, CHI, ASSETS and W4A, 2011--2014, PDFA Inspector 
    % Lazar et al. [23], 465 + 32, CHI and ASSETS, 2014--2015, Adobe Acrobat Action Wizard 
    % Ribera et al. [40], 59, DSAI, 2016, Adobe PDF Accessibility Checker 2.0 
    % Nganji [33], 200, Disability & Society, Journal of Developmental and Physical Disabilities, Journal of Learning Disabilities, and Research in Developmental Disabilities, 2009--2013, Adobe PDF Accessibility Checker 1.3
    % Our analysis, 11397, Venues across various fields of study, 2010--2019, Adobe Acrobat Accessibility Plug-in Version 21.001.20145 
    % }
    \label{tab:prior_work}
\end{table}
\section{StarGAN-ZSVC}
\label{sec:starganzsvc}

While StarGAN-VC and StarGAN-VC2 allows training with non-parallel data and runs sufficiently fast, it is limited in its ability to only perform voice conversion for speakers seen during training: while parallel $X_{\text{src}}$ and $X_{\text{trg}}$ utterance pairs are not required, the model can only synthesize speech for target speaker identities (specified as one-hot vectors) seen during training.
This could preclude the use of these models in many practical situations where zero-shot conversion is required between unseen speakers.
Conversely, AutoVC allows for such zero-shot prediction and is trained on non-parallel data, but it is implemented with a slow vocoder and
its performance suffers when trained on very little data. Combining the strengths of both of these methods, we propose the \textit{StarGAN zero-shot voice conversion} model -- StarGAN-ZSVC.

\subsection{Overcoming the Zero-shot Barrier}
To achieve voice conversion between multiple speakers, the original StarGAN-VC2 model creates an explicit embedding vector for each source-target speaker pairing, which is incorporated
as part of the generator $G$ and discriminator $D$.
This requires that each source-target speaker pairing is seen during training so that the corresponding embedding exists and has been trained -- prohibiting zero-shot voice conversion.
To overcome this hurdle, we instead infer separate source and target speaker embeddings, $\mathbf{s}_{\text{src}}$ and $\mathbf{s}_{\text{trg}}$, using a speaker encoder network $E$ -- similar to the approach followed in AutoVC (Section~\ref{sec:autovc}).

This framework is shown in Figure~\ref{fig:system-diagram}.
Utterances from unseen speakers (i.e.\  $X_{\text{src}}$ and $Y_{\text{trg}}$) are fed to the speaker encoder $E$, yielding embeddings for these new speakers, which are then used to condition $G$ and $D$, 
thereby enabling zero-shot conversion. 
The generator uses these embeddings to produce a converted Mel-spectrogram $X_{\text{src}\rightarrow \text{trg}}$ from a given source utterance's Mel-spectrogram $X_{\text{src}}$.


\begin{figure}[!t]
\includegraphics[width=\textwidth]{figures/system-diagram.pdf}
\caption{The StarGAN-ZSVC system framework. The speaker encoder network $E$ and the WaveGlow vocoder are pretrained on large speech corpora, while the generator $G$ and discriminator $D$ are trained on a 9-minute subset of the VCC dataset. During inference, arbitrary utterances for the source and target speaker are used to obtain source and target speaker embeddings, $\mathbf{s}_{\textrm{src}}$ and $\mathbf{s}_{\textrm{trg}}$. 
} \label{fig:system-diagram}
\end{figure}

$E$ is trained on a large corpus 
using a GE2E loss \cite{GE2E} which aims to simultaneously maximize distances between embeddings from different speakers while minimizing the distances between embeddings from utterances of the same speaker.
NVIDIA's WaveGlow \cite{waveglow} is used, which does not require any speaker information for vocoding and thus readily allows zero-shot conversion.

\subsection{Overcoming the Speed Barrier}
The speed of the full voice conversion system during inference is bounded by 
(a)~the speed of the generator $G$; 
(b)~the speed of converting the utterance between time and frequency domains, consisting of the initial conversion from time-domain waveform to Mel-spectrogram and the speed of the vocoder; 
and (c)~the speed of the speaker encoder $E$.
To ensure that the speed of the full system is at least real-time, each subsystem needs to be
faster than real-time.

\subsubsection{(a) Generator Speed.}
For the generator $G$ to be sufficiently fast, we design it to only include convolution, linear, normalization, and upscaling layers as opposed to a recurrent architecture like those used in AutoVC \cite{autovc}. 
By ensuring that the majority of layers are convolutions, we obtain better-than real-time speeds for the generator.

\subsubsection{(b) Vocoder and Mel-spectrogram Speed.} 
The choice of vocoder greatly affects computational cost. 
Higher-quality methods, such as those from the WaveNet family~\cite{wavenet}, are typically much slower than real-time, while purely convolutional methods such as MelGAN~\cite{MelGAN} are much faster 
but has poorer quality.
Often the main difference between the slower and faster methods is again the presence of traditional recurrent layers in the vocoder architecture. 

We opt for a reasonable middle-ground choice with the WaveGlow vocoder, which does have recurrent connections but does not use any recurrent layers with dense multiplications such as LSTM or Gated Recurrent Unit (GRU, another kind of recurrent cell architecture \cite{gru}) layers. We specifically use a pretrained WaveGlow network, as provided with 
the original paper \cite{waveglow}. 
Furthermore, the speed of the Mel-spectrogram transformation for the input audio is well faster than real-time due to the efficient nature of the fast Fourier transform and the multiplication by Mel-basis filters.

\subsubsection{(c) Speaker Encoder Speed.} 
The majority of research efforts into obtaining speaker embeddings involve models using slower recurrent layers, often making these encoder networks the bottleneck. 
We also make use of a recurrent stacked-GRU network as our speaker embedding network $E$.
However, we only need to obtain a single speaker embedding to perform 
any number of conversions involving that speaker.
We therefore treat this as a preprocessing step where we apply $E$ to a few arbitrary utterances from the target and source speakers, averaging the results to obtain target and source speaker embeddings, and use those same embeddings for all subsequent conversions.

We also design the speaker embeddings to be 256-dimensional vectors of unit length. If we were to use StarGAN-ZSVC downstream for data augmentation (where we want speech from novel speakers), we could then simply sample random unit-length vectors of this dimensionality  to use with the generator.

\begin{figure}[!t]
\includegraphics[width=\textwidth]{figures/architecture.pdf}
\caption{StarGAN-ZSVC's network architectures. The speaker encoder $E$ is a recurrent network similar to that used in the original GE2E paper, while the generator $G$ and discriminator $D$ are modified versions from the original StarGAN-VC2 architecture. Within layers, \texttt{k} and \texttt{s} represent kernel size and stride (for convolutions), \texttt{f} is the scaling factor (for pixel shuffle), and \texttt{h} and \texttt{c} are the height and channels of the output (for reshape layers). A number alongside a layer indicates the number of output channels (for convolutions), or output units (for linear and GRU layers). GLU layers split the input tensor in half along the \textit{channels} dimension. GSP, GLU, and SELU indicate global sum pooling, gated linear units, and scaled exponential linear units, respectively.} \label{fig:architecture}
\end{figure}

\subsection{Architecture}
With the previous considerations in mind, we design the generator $G$, discriminator $D$, and encoder network $E$, as shown in Figure~\ref{fig:architecture}.
The generator and discriminator are adapted from StarGAN-VC2 \cite{stargan-vc2}, while the speaker encoder
is adapted from the original model proposed for speaker identification \cite{GE2E}. 
Specifically, for $E$ 
we use a simple stacked GRU model, while for $D$ we use a projection discriminator \cite{projection_discriminator}. 
For $G$, we use the 2-1-2D generator from StarGAN-VC2 with a modified central set of layers, denoted by the \textit{Conditional Block} in the figure. 

These conditional blocks are intended to provide the network with a way to modulate the channels of an input spectrogram, with modulation factors conditioned on the specific source and target speaker pairing.
They utilize a convolutional layer 
followed by a modified conditional instance normalization layer \cite{CIN} and a gated linear unit \cite{glu}.

The modified conditional instance normalization layer performs the following operation on an input feature vector $\mathbf{f}$:
\begin{equation}
\text{CIN}(\mathbf{f}, \gamma, \beta) 
= \gamma \left( \frac{\mathbf{f} - \mu(\mathbf{f})}{\sigma (\mathbf{f}) } \right) + \beta
\end{equation}
where $\mu(\mathbf{f})$ and $\sigma(\mathbf{f})$ are respectively the scalar mean and standard deviation of vector $\mathbf{f}$, while $\gamma$ and $\beta$ are computed using two linear layers which derive their inputs from the speaker embeddings, as depicted in Figure~\ref{fig:architecture}. The above
is computed separately for each channel when the input feature contains multiple~channels.

For the discriminator, the source and target speaker embeddings are also fed through several linear layers and activation functions to multiply with the pooled output of $D$'s main branch.

% !TEX root = ../top.tex
% !TEX spellcheck = en-US

\section{Experiments}
\label{sec:experiments}

In this section, we first evaluate our framework on the SPEED dataset, and then introduce the SwissCube dataset, which contains accurate 3D mesh and physically-modeled astronomical objects, and perform thorough ablation studies on it. We further show results on real images of the same satellite. Finally, to demonstrate the generality of our approach we evaluate it on the standard Occluded-LINEMOD dataset depicting small depth variations. 
% \WJ{"CubeSat" leaks author nationalities and even institution (it's an EPFL project). In my community, this would be perceived negatively during peer review. I'd suggest using a temporary name ("NanoSat") with an asterisk/footnote saying that the dataset name is temporarily anonymized as not to reveal authorship.} \YH{I am not sure if it is in CV, while it does not hurt to change to a temporary name.} \MS{I agree, but I suggest "CubeSat", which is the standard term for this type of satellite.}

% \yh{
We train our model starting from a backbone pre-trained on ImageNet~\cite{Deng09}, and, for any 6D pose dataset, feed it 3M unique training samples obtained via standard online data augmentation strategies, such as random shift, scale, and rotation. To evaluate the accuracy, we will report the individual performance under different depth ranges, using the standard ADI-0.1d~\cite{Hu19a,Hu20a} accuracy metrics, which encodes the percentage of samples whose 3D reconstruction error is below 10\% of the object diameter. On the SPEED dataset, however, we use a different metric, as we do not have access to the 3D SPEED model, making the computation of ADI impossible. Instead, we use the metric from the competition, that is, ${\bf e}_{\bf q}+{\bf e}_{\bf t}$, where ${\bf e}_{\bf q}$ is the angular error between the ground-truth quaternion and the predicted one, and ${\bf e}_{\bf t}$ is the normalized translation error. Furthermore, because the depth distribution of SPEED is not uniform, with only few images depicting the satellite at a large distance from the camera, we only report the average error on the whole test set, as in the competition.
% }
The source code and dataset are publicly available at \href{https://github.com/cvlab-epfl/wide-depth-range-pose}{https://github.com/cvlab-epfl/wide-depth-range-pose}.

\subsection{Evaluation on the SPEED Dataset}
Although the SPEED dataset has several drawbacks, discussed in Section~\ref{sec:related}, it remains a valuable benchmark, and we thus begin by evaluating our method on it. As the test annotations are not publicly available, and the competition is not ongoing, we divide the training set into two parts, 10K images for training and the remaining 2K ones for testing.
We evaluate the two top-performing methods from the competition,~\cite{Chen19DLR} (DLR) 
% \MS{Don't they have a better name?}
and~\cite{Hu19a} (SegDriven-Z), on these new splits using the publicly-available code, and find their errors to be of similar magnitude to the ones reported online during the challenge.
Note that our method, as DLR and SegDriven-Z, uses the 3D model to define the keypoints whose image location we predict. We therefore exploit a method of~\cite{Hartley00} to first reconstruct the satellite from the dataset. 

Table~\ref{tab:speed_stoa} compares our results to those of the two top-performing methods on this dataset. Note that DLR combines the results of 6 pose estimation networks, followed by an additional pose refinement strategy to improve accuracy. We therefore also report the results of our method with and without this pose refinement strategy. Note, however, that we still use a single pose estimation network. Furthermore, for our method, we report the results of two separate networks trained at different input resolutions. 
At the resolution of 960$\times$, we outperform the two state-of-the-art methods, while our architecture is much smaller and much faster. To further speed up our approach, we train a network at a third (640$\times$) of the raw image resolution. This network remains on par with DLR but runs 20+ times faster.
% \MS{What resolution does Chen use? If they use 960, I would tend to turn this the other way around: Say that, at the same resolution, we outperform the two state-of-the-art methods, while our architecture is much smaller and much faster. To further speed up our approach, we train a network at a third of the raw image resolution. This network remains on par with Chen but runs 50 times faster.}
% \YH{As they use two networks, we can not compare the resolution directly. In more detail. they train a 768x768 detector first and resize all the detected bounding box to 768x768 again to feed into the next pose network. And they use the second pose network 6 times for results ensemble.}
% The faster version can already on par with the top performer but runs 10+ times faster. Our slower version performs the best and still runs 5+ times faster than the competitors.
% \WJ{Can you explain the rationale for these versions with different resolutions? Just speed? Wasn't clear from the text.}\YH{Yes, mainly for speed, also for GPU memory consumption, especially during training.}
% \WJ{You could more prominently point out speed as one of the benefits in the introduction. Practical usage in an autonomous satellite will require low-latency low-compute answers.}\YH{Fixed}

% ~\footnote{\href{https://github.com/BoChenYS/satellite-pose-estimation}{https://github.com/BoChenYS/satellite-pose-estimation}}$^{,}$\footnote{\href{https://github.com/cvlab-epfl/segmentation-driven-pose}{https://github.com/cvlab-epfl/segmentation-driven-pose}}

%  
 
% !TEX root = ../top.tex
% !TEX spellcheck = en-US

\begin{figure}[t]
    \begin{center}
    \includegraphics[width=0.6\linewidth]{./fig/swisscube_statistics/swisscube_statistics.pdf}
    % \fbox{\rule{0pt}{2in} \rule{0.25\linewidth}{0pt}}
    \end{center}
    \vspace{-6mm}
    \caption{{\bf The depth distribution of the target object in datasets.} 
    Nearly 80\% of the SPEED dataset is located in the depth range of 1 to 5 times the object diameter. As contrast, our Swisscube dataset is uniformly distributed among the depth range approximately. Here, the depth is in the unit of times of the diameter of the target object, and note that, we do not have the accurate 3D object model of SPEED, its diameter is approximately computed from a 3D reconstruction. \YH{Suppose to remove this figure, no space}
    }
    \label{fig:swisscube_statistics}
\end{figure}


 
% !TEX root = ../top.tex
% !TEX spellcheck = en-US

\begin{table}
    \centering
    \scalebox{0.8}{
    \begin{small}
    % \rowcolors{2}{white}{gray!10}
    \begin{tabular}{cccccc}
        \toprule
        &&	\multicolumn{2}{c}{Accuracy} & \multirow{2}{*}{Model Size} & \multirow{2}{*}{FPS}\\
        && Raw & Refinement & \\
        \midrule
        \multicolumn{2}{l}{SegDriven-Z~\cite{Hu19a}} & 0.022 & - & 89.2 M & 3.1 \\
        \multicolumn{2}{l}{DLR~\cite{Chen19DLR}} & 0.017 & 0.012 & 176.2 M & 0.7 \\
        \midrule
        \multirow{2}{*}{\bf Ours} 
        & 640$\times$ & 0.018 & 0.013 & {\bf 51.5 M} & {\bf 35} \\
        & 960$\times$ & {\bf 0.016} & {\bf 0.010} & {\bf 51.5 M} & 18 \\
        % {\bf Ours} & {\bf 51.5 M} & {\bf $\sim$ 45 ms} \\
        % Chen {\it etc.} & 48.4 + 21.3 $\times$ 6 = 176.2 M & $\sim$ 1500 ms \\
        % SegDriven-Z & 44.6 + 44.6 = 89.2 M & $\sim$ 300 ms \\ 
        % \midrule
        \bottomrule
    \end{tabular}
    \end{small}
    }
    \vspace{-3mm}
    \caption{{\bf Comparison with the state of the art on SPEED.} Our method outperforms the two top-performing methods in the challenge and is much faster and lighter.}
    \label{tab:speed_stoa}
\end{table} 
\subsection{Evaluation on the SwissCube Dataset}
To facilitate the evaluation of 6D object pose estimation methods in the wide-depth-range scenario, we
introduce a novel SwissCube dataset. The renderings in this dataset account for the
precise 3D shape of the satellite and include realistic models of the star backdrop, Sun, Earth,
and target satellite, including the effects of global illumination, mainly
glossy reflection of the Sun and Earth from the satellite's surface.
To create the 3D model of the SwissCube, we modeled every mechanical part from
raw CAD files, including solar panels, antennas, and screws, and we
carefully assigned material parameters to each part.

The renderings feature a space environment based on the relative placement and
sizes of the Earth and Sun. Correct modeling of the Earth is most important, as
it is often directly observed in the images and significantly affects the
appearance of the satellite via inter-reflection. We extract a high-resolution
spectral texture of the Earth's surface and atmosphere from published data
products acquired by the NASA Visible Infrared Imaging Radiometer Suite (VIIRS)
instrument. These images account for typical cloud coverage and provide
accurate spectral color information on 6 wavelength bands. Illumination from
the Sun is also modeled spectrally using the extraterrestrial solar irradiance
spectrum. The spectral simulation performed using the open source Mitsuba 2
renderer~\cite{Nimier19} finally produces an RGB output that
can be ingested by standard computer vision tools.

% !TEX root = ../top.tex
% !TEX spellcheck = en-US

\begin{figure}[t]
    \begin{center}
    \includegraphics[width=0.6\linewidth]{./fig/render_setting/render_setting.pdf}
    % \fbox{\rule{0pt}{2in} \rule{0.25\linewidth}{0pt}}
    \end{center}
    \vspace{-6mm}
    \caption{{\bf Settings for physical rendering of SwissCube.} We physically model the Sun, the Earth, and the complex illumination conditions that can occur in space.}
    \label{fig:render_setting}
\end{figure}

The renderings also include a backdrop of galaxies, nebulae, and star clusters
based on the HYG database star catalog~\cite{hygdatabase} containing around
120K astronomical objects along with information about position and
brightness. The irradiance due to astronomical objects is orders of magnitude
below that of the Sun. To increase the diversity of the dataset, and to ensure
that the network ultimately learns to ignore such details, we boost the
brightness of astronomical objects in renderings to make them more apparent.

Following these steps, we place the SwissCube into its actual orbit located
approximately 700 km above the Earth's surface along with a virtual observer
positioned in a slightly elevated orbit. We render sequences with different
relative velocities, distances and angles. To this end, we use a wide field-of-view (100$^{\circ}$) camera whose distance to the target ranges uniformly between $1d$ to $10d$, where $d$ indicates the diameter of the SwissCube without taking the antennas into accounts.
% \MS{Do you use a wide field-of-view camera? With what angle? Does the diameter $d$ include the antennas, or is it just the cube edge length?}
% \YH{Yes, we use the virtual camera with a FOV of 100. the diameter is computed from only the cube body and does not take the antennas into accounts. And, we treat the Swisscube as an asymmetrical object.}
The high-level
setup is illustrated in Fig.~\ref{fig:render_setting}. Note that the renderings
are essentially black when the SwissCube passes into the earth's shadow, and we
detect and remove such configurations.

We generate 500 scenes each consisting of a 100-frame sequence, for a total of
50K images. We take 40K images from 400 scenes for training and the 10K
image from the remaining 100 scenes for testing. 
%We make the depth range of the %CubeSat dataset approximately uniformly distributed from 1d to 10d, as
We render the images at a 1024$\times$1024
resolution, a few of which are shown in Fig.~\ref{fig:results_demo}. During network processing, we resize the
input to 512$\times$512. 
%Although higher input resolution often means higher
%accuracy, as shown by the SPEED experiments, we will focus on this resolution
%setting for a detailed ablation study in this experiment. 
We report the ADI-0.1d accuracy at three
depth ranges, which we refer to as {\it near}, {\it medium}, and {\it far}, corresponding to the depth ranges [1d-4d],
[4d-7d], and [7d-10d], respectively.

% !TEX root = ../top.tex
% !TEX spellcheck = en-US

\begin{figure*}[t]
    \begin{center}
    % \fbox{\rule{0pt}{1in} \rule{0.25\linewidth}{0pt}}
    \includegraphics[width=0.135\linewidth]{./fig/results_demo/_data_swisscube_20200922_hu_test_seq_000482_000000_rgb_000006.png}
    \includegraphics[width=0.135\linewidth]{./fig/results_demo/_data_swisscube_20200922_hu_test_seq_000401_000000_rgb_000011.png}
    \includegraphics[width=0.135\linewidth]{./fig/results_demo/_data_swisscube_20200922_hu_test_seq_000424_000000_rgb_000082.png}
    \includegraphics[width=0.135\linewidth]{./fig/results_demo/_data_swisscube_20200922_hu_test_seq_000406_000000_rgb_000052.png}
    \includegraphics[width=0.135\linewidth]{./fig/results_demo/_data_swisscube_20200922_hu_test_seq_000410_000000_rgb_000038.png}
    \includegraphics[width=0.135\linewidth]{./fig/results_demo/_data_swisscube_20200922_hu_test_seq_000417_000000_rgb_000058.png} \\
    \includegraphics[width=0.135\linewidth]{./fig/results_demo/_data_swisscube_20200922_hu_test_seq_000482_000000_rgb_000006_pred.png}
    \includegraphics[width=0.135\linewidth]{./fig/results_demo/_data_swisscube_20200922_hu_test_seq_000401_000000_rgb_000011_pred.png}
    \includegraphics[width=0.135\linewidth,trim=0 100 200 100, clip]{./fig/results_demo/_data_swisscube_20200922_hu_test_seq_000424_000000_rgb_000082_pred.png}
    \includegraphics[width=0.135\linewidth,trim=0 100 200 100, clip]{./fig/results_demo/_data_swisscube_20200922_hu_test_seq_000406_000000_rgb_000052_pred.png}
    \includegraphics[width=0.135\linewidth,trim=100 50 200 250, clip]{./fig/results_demo/_data_swisscube_20200922_hu_test_seq_000410_000000_rgb_000038_pred.png}
    \includegraphics[width=0.135\linewidth,trim=250 200 50 100, clip]{./fig/results_demo/_data_swisscube_20200922_hu_test_seq_000417_000000_rgb_000058_pred.png}
    \end{center}
    \vspace{-6mm}
    \caption{{\bf Qualitative results on the SwissCube dataset.} Our method yields accurate pose estimates at all scales.}
    \label{fig:results_demo}
\end{figure*}

\subsubsection{Effect of our Ensemble-Aware Sampling}
We first evaluate the effectiveness of our ensemble-aware sampling strategy, further comparing our approach with the single-scale baseline SegDriven~\cite{Hu19a}, which uses the same backbone as us. Note that the original SegDriven method did not rely on a detector to zoom in on the object, but was extended with a YOLOv3~\cite{Redmon18} one in the SPEED competition, resulting in the SegDriven-Z approach evaluated above. For our comparison on the SwissCube dataset to be fair, we therefore also report the results of SegDriven-Z.
% \MS{Could we also evaluate Chen on this dataset? This would be more convincing, although probably too late.} \YH{We had the result, I will add it back.}
Moreover, we also evaluate the top performer on the SPEED dataset, DLR~\cite{Chen19DLR}, on our dataset.

Fig.~\ref{fig:param_study} demonstrates the effectiveness of our sampling strategy.
Our results with different $\lambda$ values, which controls the ensemble-aware sampling, show that large values, such as $\lambda>10$, yield lower accuracies. With such large values, our sampling strategy degenerates to the one commonly-used in FPN-based object detectors. This therefore evidences the importance of encouraging every pyramid level to produce valid estimates at more than a single object scale. 
%adopts is much inferior to other settings. That big $\lambda$ makes every pyramid level working on unoverlapped training instances, making different pyramid levels uncombinable during inference for a specific instance. On the other hand, the case of 
Note also that $\lambda=0$, which corresponds to distributing every training instance uniformly to all levels, does not yield the best results, suggesting that forcing every level to produce high-accuracy at all the scales is sub-optimal. In other words, each level should perform well in a reasonable scale range, but these ranges should overlap across the pyramid levels. 
%The imposing of large variation difficulties to every pyramid level makes their performance deteriorate, leading to a worse fusion accuracy. 
This is achieved approximately with $\lambda=1$, which we will use in the following experiments.

Table~\ref{tab:parameters_study} summarizes the comparison results with other baselines. Because it does not explicitly handle scale, SegDriven performs poorly on far objects. This is improved by the detector used in SegDiven-Z. However, the performance of this two-stage approach remains much worse than that of our framework.
Our method outperforms DLR as well, even though our method is 20+ times faster than DLR.
% , independently of the hyper-parameter value $\lambda$, controlling the ensemble-aware sampling. 
Fig.~\ref{fig:results_demo} depicts a few rendered images and corresponding poses estimated with our approach. 
% \MS{I would tend to show this at the end of the first subsection, and potentially compare with SegDriven-Z.}

 
% !TEX root = ../top.tex
% !TEX spellcheck = en-US

\begin{table}
    \centering
    \scalebox{0.8}{
    \begin{small}
    % \rowcolors{2}{white}{gray!10}
    \begin{tabular}{lcccc}
    \toprule
    & Near & Medium & Far & All  \\
    \midrule
    SegDriven~\cite{Hu19a} &  41.1 & 22.9 & 7.1 & 21.8 \\ 
    SegDriven-Z~\cite{Hu19a} &  52.6 & 45.4 & 29.4 & 43.2 \\ 
    DLR~\cite{Chen19DLR} & 63.8 & 47.8 & 28.9 & 46.8 \\
    {\bf Ours} & {\bf 65.2} & {\bf 48.7} & {\bf 31.9} & {\bf 47.9} \\
    \bottomrule
    \end{tabular}
    \end{small}
    }
    \vspace{-3mm}
    \caption{\bf Our method outperforms all baselines on SwissCube.}
        % Our multi-scale framework outperforms the single-scale baseline SegDriven~\cite{Hu19a} and its zoomed version (SegDriven-Z) significantly, and also DLR~\cite{Chen19DLR}, the top performer on SPEED dataset.
    \label{tab:parameters_study}
\end{table} 

 
% !TEX root = ../top.tex
% !TEX spellcheck = en-US

\begin{table}
    \centering
    \scalebox{0.8}{
    \begin{small}
    % \rowcolors{2}{white}{gray!10}
    \begin{tabular}{lcccc}
    \toprule
    & Near & Medium & Far & All  \\
    \midrule
    SegDriven~\cite{Hu19a} &  41.1 & 22.9 & 7.1 & 21.8 \\ 
    SegDriven-Z~\cite{Hu19a} &  52.6 & 45.4 & 29.4 & 43.2 \\ 
    DLR~\cite{Chen19DLR} & 63.8 & 47.8 & 28.9 & 46.8 \\
    {\bf Ours} & {\bf 65.2} & {\bf 48.7} & {\bf 31.9} & {\bf 47.9} \\
    \bottomrule
    \end{tabular}
    \end{small}
    }
    \vspace{-3mm}
    \caption{\bf Our method outperforms all baselines on SwissCube.}
        % Our multi-scale framework outperforms the single-scale baseline SegDriven~\cite{Hu19a} and its zoomed version (SegDriven-Z) significantly, and also DLR~\cite{Chen19DLR}, the top performer on SPEED dataset.
    \label{tab:parameters_study}
\end{table} 

 
% !TEX root = ../top.tex
% !TEX spellcheck = en-US

\begin{table}
    \centering
    \scalebox{0.8}{
    \begin{small}
    % \rowcolors{2}{white}{gray!10}
    \begin{tabular}{ccccc}
    \toprule
    & Near & Medium & Far & All \\
    \midrule
    L1 & 0 & 25.2 & \underline{31.8} & 19.5 \\
    L2 & 36.5 & \underline{48.4} & 27.7 & 38.2 \\
    L3 & \underline{62.3} & 47.4 & 19.9 & \underline{42.6} \\
    L4 & 59.2 & 20.2 & 1.7 & 26.3 \\
    L5 & 25.5 & 0.9 & 0 & 8.3 \\
    \midrule
    {\bf Fusion} & {\bf 65.2} & {\bf 48.7} & {\bf 31.9} & {\bf 47.9} \\
    \bottomrule
    \end{tabular}
    \end{small}
    }
    \vspace{-3mm}
    \caption{{\bf Effect of the multi-scale fusion.} Each pyramid level favors a specific depth range, which our multi-scale fusion strategy leverages to outperform every individual level.}
    \label{tab:fusion_effect}
\end{table} 


\subsubsection{Effect of our Multi-Scale Fusion}

To better understand the role of each pyramid level during multi-scale fusion, we study the accuracy obtained using the predictions of each individual pyramid level.
Intuitively, we expect the levels with a larger receptive field (feature maps with low spatial resolution) to perform well for close objects, and those with a small receptive field (feature maps with high spatial resolution) to produce better results far-away ones. While the results in Table~\ref{tab:fusion_effect} confirm this intuition for Levels L1, L2 and L3, we observe that the performance degrades at L4 and L5. We believe this to be due to the very low spatial resolution of the corresponding feature maps, 8$\times$8, and 4$\times$4, respectively, making it difficult for these levels to output precise poses. Nevertheless, the accuracy after multi-scale fusion outperforms every individual level, and we leave the study of a different number of pyramid levels to future work.
% \MS{This suggests that we should probably just stop at L3...}\YH{Although the performance of L4 and L5 alone is bad, we are not sure if the L4 or L5 can contribute to the final loss via ensemble. We need more experiments to verify it, so leave it as it is right now.}

%performance with the results combined only from feature cells within each level's segmentation mask. Table~\ref{tab:fusion_effect} shows the results. In intuition, levels with larger reception fields perform better for closer objects and vice versa. However, we find that this is not always true. The performance of level 4 on near objects can not match the one on level 3, and level 5 becomes even more worse. Note that, the spatial feature dimensions for L1, L2, L3, L4, and L5 are 64$\times$64, 32$\times$32, 16$\times$16, 8$\times$8, and 4$\times$4, respectively. Although L4, especially L5, has larger reception fields, the lower spatial resolution makes them less discriminable against 2D keypoints and introduces more visual noises for each cell. Nevertheless, the accuracy after multi-scale fusion outperforms every single level and we leave the study of a different number of pyramid levels to future work.

\subsubsection{Effect of the 3D Loss}

 
% !TEX root = ../top.tex
% !TEX spellcheck = en-US

\begin{table}
    \centering
    \scalebox{0.8}{
    \begin{small}
    % \rowcolors{2}{white}{gray!10}
    \begin{tabular}{cccccccccc}
    \toprule
    & Near & Medium & Far & All \\
    \midrule
    2D loss & 64.6 & 42.0 & 24.0 & 43.1   \\
    {\bf 3D loss} & {\bf 65.2} & {\bf 48.7} & {\bf 31.9} & {\bf 47.9} \\
    \midrule
    Delta & +0.6 & +6.7 & +7.9 & +4.8 \\
    \bottomrule
    \end{tabular}
    \end{small}
    }
    \vspace{-3mm}
    \caption{{\bf Effect of the 3D loss.} The proposed 3D loss outperforms the 2D one in every depth ranges. The farther the object, the more obvious the advantage of the 3D loss.}
    \label{tab:error_3d_vs_2d}
\end{table} 

% !TEX root = ../top.tex
% !TEX spellcheck = en-US

\begin{figure}[t]
\centering
\includegraphics[width=0.6\linewidth]{./fig/error_vs_positions/error_vs_pos.pdf}
\vspace{-3mm}
\caption{\small {\bf Pose error as a function of the object position.} The performance of the 2D loss clearly degrades for objects near the image center, whereas that of our 3D loss doesn't. See Fig.~\ref{fig:cube_problem}(b) for the underlying geometry. Note that as the object moves closer to the image boundary, it becomes truncated, which degrades the performance of both losses.}
\label{fig:error_vs_positions}
\end{figure}

%The popular 2D reprojection loss has server problems in the wide-depth-range scenarios as discussed in Fig.~\ref{fig:cube_problem}. To fairly compare the proposed 3D loss against the 2D loss, we train our framework two times from the same initial states and with the same other settings except for the adopted regression loss. 
In Table~\ref{tab:error_3d_vs_2d}, we compare the results obtained by training our approach with either the commonly-used 2D reprojection loss or our loss function in 3D space. Note that our 3D loss outperforms the 2D one in all depth ranges, and the farther the object, the larger the gap between the results of the two loss functions.
In Fig.~\ref{fig:error_vs_positions}, we plot the average accuracy as a function of the object image location. The performance of the 2D loss degrades significantly when the object is located near the image center, whereas the accuracy of our 3D loss remains stable for most object positions. Note that, The reason both of them become worse in the right part of the figure is due to the object truncation by image borders.

\subsection{Results on Real Images}

In Fig.~\ref{fig:domain_adaptation}, we illustrate the performance of our approach on real images. Note that these real images were not captured in space but in a lab environment using a mock-up model of the target and an OptiTrack motion capture system to obtain ground-truth pose information for a few images. We then fine-tuned our model pre-trained on our synthetic SwissCube dataset using only 20 real images with pose annotations. Because this procedure only requires small amounts of annotated real data, it would be applicable in an actual mission, where images can be sent to the ground, annotated manually, and the updated network parameters uploaded back to space.
%Although our CubeSat dataset is rendered by a computer, thanks to its high realism, it can be easily adapted to real data. For the real data, we obtain it by capturing a real-size mock-up of the target. We use a simple finetune~\cite{1}, which is a very basic domain adaptation technique, to adapt our model to the read data. shows some real results on two different satellites, CubeSat and VESPA as well. Although the real data is not captured from the ``real'' space and we are sure we can find better domain adaptation methods, it shines a bright light for the preparation of the real launching in the future.

%  
% !TEX root = ../top.tex
% !TEX spellcheck = en-US

\begin{table}
    \centering
    \begin{small}
    % \rowcolors{2}{white}{gray!10}
    \begin{tabular}{lcccc}
        \toprule
        &	Near & Medium & Far & All\\
        \midrule
        {\bf Ours} & {\bf 60.8} & {\bf 51.6} & {\bf 35.1} & {\bf 49.0} \\
        Chen {\it etc.} & 56.7 & 48.2 & 32.8 & 46.1 \\
        SegDriven-Z & 52.6 & 45.4 & 29.4 & 43.2 \\ 
        % \midrule
        \bottomrule
    \end{tabular}
    \end{small}
    \vspace{-3mm}
    \caption{{\bf Comparison with the state of the art on CubeSat.} bla bla bla bla bla bla bla bla bla bla bla bla bla bla bla bla bla bla bla bla bla bla bla bla bla bla bla bla bla bla bla bla bla bla bla bla a bla bla bla bla bla bla bla bla bla bla bla bla bla bla bla bla bla bla bla bla bla bla bla bla bla bla bla bla bla bla bla bla bla bla bla bla bla bla bla bla v, much more faster as shown in Table~\ref{tab:swisscube_stoa}.}
    \label{tab:swisscube_stoa}
\end{table} 
% !TEX root = ../top.tex
% !TEX spellcheck = en-US

\begin{figure}[t]
\centering
\includegraphics[width=0.29\linewidth,trim=450 380 400 400,clip]{fig/real_results/im_36_12_723405.jpg}
\includegraphics[width=0.29\linewidth,trim=450 380 400 400,clip]{fig/real_results/im_409_50_004279.jpg}
\includegraphics[width=0.29\linewidth,trim=450 380 400 400,clip]{fig/real_results/im_880_97_080135.jpg}
% \begin{tabular}{cc}
%     \fbox{\rule{0pt}{1.5in} \rule{0.4\linewidth}{0pt}} &
%     \fbox{\rule{0pt}{1.5in} \rule{0.4\linewidth}{0pt}} \\
%     (a) SwissCube & (b) VESPA\\
% \end{tabular}
\vspace{-3mm}
\caption{\small {\bf Qualitative results on real data.} Our model easily adapts to real data, using as few as 20 annotated images.
}
\label{fig:domain_adaptation}
\end{figure}
 
% !TEX root = ../top.tex
% !TEX spellcheck = en-US

\begin{table}
    \centering
    \scalebox{0.8}{
    % \rowcolors{2}{white}{gray!10}
    \begin{small}
    \begin{tabular}{L{4em}C{3em}C{5em}C{3em}C{3em}}
        \toprule
        & PVNet & SimplePnP & Hybrid & {\bf Ours}\\
        \midrule
        Ape    & 15.8 & 19.2 & 20.9 &  {\bf 22.3} \\
        Can    & 63.3 & 65.1 & 75.3 &  {\bf 77.8} \\
        Cat    & 16.7 & 18.9 & 24.9 &  {\bf 25.1} \\
        Driller& 65.7 & 69.0 & 70.2 &  {\bf 70.6} \\
        Duck   & 25.2 & 25.3 & 27.9 &  {\bf 30.2} \\
    Eggbox$^*$ & 50.2 & 52.0 & 52.4 &  {\bf 52.5} \\
    Glue$^*$   & 49.6 & 51.4 & 53.8 &  {\bf 54.9} \\
       Holepun.& 39.7 & 45.6 & 54.2 &  {\bf 55.6} \\
        \midrule
        Avg.   & 40.8 & 43.3 & 47.5 &  {\bf 48.6} \\
        \bottomrule
    \end{tabular}
    \end{small}
    }
    \vspace{-3mm}
    \caption{{\bf Comparison on Occluded-LINEMOD.} We compare our results with those of PVNet~\cite{Peng19a}, SimplePnP~\cite{Hu20a} and Hybrid~\cite{Song20a}. Symmetry objects are denoted with ``$^*$''.}
    \label{tab:occ_linemod_stoa}
\end{table} 

\subsection{Evaluation on Occluded-LINEMOD}

Finally, to demonstrate that our approach is general, and thus applies to datasets depicting small depth variations, we evaluate it on the standard Occluded-LINEMOD dataset~\cite{Krull15}. Following~\cite{Hu20a}, we use the raw images at resolution 640$\times$480 as input to our network, train our model on the LINEMOD~\cite{Hinterstoisser12b} dataset and test it on Occluded-LINEMOD without overlapped data. Although our framework supports multi-object training, for the evaluation to be fair, we train one model for each object type and compare it with methods not relying on another refinement procedure.
Considering the small depth variations in this dataset, we remove the two pyramid levels with the largest reception fields from our framework, leaving only ${\cal F}_1$, ${\cal F}_2$ and ${\cal F}_3$. As shown in Table~\ref{tab:occ_linemod_stoa}, our model outperforms the state of the art even in this general 6D object pose estimation scenario.

%shows the comparison results of our framework against the state-of-the-art methods. It shows that our multi-scale fusion framework also works pretty well in general 6D object pose estimation.


\section{Methodology}
\label{sec:benchmark}

\subsection{Description of hardware and software}
\label{ssec:supercomp}
The benchmarks reported in this paper were performed on the Intel Xeon Phi systems provided by the Joint Laboratory for System Evaluation (JLSE) and the Theta supercomputer at the Argonne Leadership Computing Facility (ALCF) \cite{alcf}, which is a part of the U.S. Department of Energy (DOE) Office of Science (SC) Innovative and Novel Computational Impact on Theory and Experiment (INCITE) program \cite{incite}. Theta is a 10-petaflop Cray XC40 supercomputer consisting of 3,624 Intel Xeon Phi 7230 processors. Hardware details for the JLSE and Theta system are shown in \Cref{tab:hw}.

The Intel Xeon Phi processor used in this paper has 64 cores each equipped with L1 cache. Each core also has two Vector Processing Units, both of which need to be used to get peak performance. This is possible because the core can execute two instructions per cycle. In practical terms, this can be achieved by using two threads per core. Pairs of cores constitute a tile. Each tile has an L2 cache symmetrically shared by the core pair. The L2 caches between tiles are connected by a two dimensional mesh. The cores themselves operate at 1.3 GHz. Beyond the L1 and L2 cache structure, all the cores in the Intel Xeon Phi processor share 16 GBytes of MCDRAM (also known as High Bandwidth Memory) and 192 GBytes of DDR4. The bandwidth of MCDRAM is approximately 400 GBytes/sec while the bandwidth of DDR4 is approximately 100 GBytes/sec. 

\begin{table}
  \caption{Hardware and software specifications}
  \label{tab:hw}

  \begin{tabularx}{\columnwidth}{XX}
  \toprule
			\multicolumn{2}{c}{\textbf{\intelphi\ node characteristics}} \\
    \midrule 
    \intelphi\ models				&	7210 and 7230 (64~cores, 1.3~GHz, 
    									2,622 GFLOPs) \\
    Memory per node					&	16 GB MCDRAM, \newline 192 GB DDR4 RAM \\
    Compiler						&	Intel Parallel Studio XE 2016v3 \\
    \midrule
    		\multicolumn{2}{c}{\textbf{JLSE \iphi\ cluster (26.2 TFLOPS peak)}} \\
    \midrule
    \# of \intelphi\ nodes	&	10 \\
    Interconnect type				&	Intel Omni-Path\textsuperscript{TM} \\
    \midrule
    		\multicolumn{2}{c}{\textbf{Theta supercomputer (9.65~PFLOPS peak)}} \\
    \midrule
    \# of \intelphi\ nodes				&	3,624 \\
    Interconnect type				&	Aries interconnect with \newline Dragonfly topology \\
  \bottomrule
\end{tabularx}

\end{table}

\begin{table}
\begin{threeparttable}
  \caption{Chemical systems used in benchmarks and their size characteristics}
  \label{tab:chem}

  \begin{tabularx}{\columnwidth}{XYYYYY}

  \toprule

  \multirow{2}{*}{Name}	&	\multirow{2}{*}{\# atoms}	&	\multirow{2}{*}{\# BFs\tnote{a}}	&	\multicolumn{3}{c}{Memory footprint\tnote{b}, GB} \\
  \cmidrule(l){4-6}
  		& & &	{MPI\tnote{c}}	&	{Pr.F.\tnote{d}}	&	{Sh.F.\tnote{e}} \\
  \midrule
  	0.5~nm	&	44			&	660			&	7		&	0.13		&	0.03	\\
	1.0~nm	&	120			&	1800		&	48		&	1			&	0.2	\\
	1.5~nm	&	220			&	3300		&	160		&	3			&	0.8	\\
	2.0~nm	&	356			&	5340		&	417		&	8			&	2	\\
	5.0~nm	&	2016		&	30240		&	9869	&	257			&	52	\\
	\bottomrule
  \end{tabularx}

  \begin{tablenotes}
  	\item [a] BF -- basis function
    \item [b] Estimated using \crefrange{eqn:mem:mpi}{eqn:mem:shr}
 	\item [c] MPI-only SCF code
    \item [d] Private Fock SCF code
    \item [e] Shared Fock SCF code
  \end{tablenotes}
\end{threeparttable}
\end{table}

These two levels of memory can be configured in three different ways (or modes). The modes are referred to as Flat mode, Cache mode, and Hybrid mode. Flat mode treats the two levels of memory as separate entities. The Cache mode treats the MCDRAM as a direct mapped L3 cache to the DDR4 layer. Hybrid mode allows the user to use a fraction of MCDRM as L3 cache allocate the rest of the MCDRAM as part of the DDR4 memory.
In Flat mode, one may choose to run entirely in MCDRAM or entirely in DDR4. The "numactl" utility provides an easy mechanism to select which memory is used. It is also possible to choose the kind of memory used via the "memkind" API, though as expected this requires changes to the source code.

Beyond memory modes, the Intel Xeon Phi processor supports five cluster modes. The motivation for these modes can be understood in the following manner: to maintain cache coherency the Intel Xeon Phi processor employs a distributed tag directory (DTD). This is organized as a set of per-tile tag directories (TDs), which identify the state and the location on the chip of any cache line. For any memory address, the hardware can identify the TD responsible for that address. The most extreme case of a cache miss requires retrieving data from main memory (via a memory controller). It is therefore of interest to have the TD as close as possible to the memory controller. This leads to a concept of locality of the TD and the memory controllers.
It is in the developer's interest to maintain the locality of these messages to achieve the lowest latency and greatest bandwidth of communication with caches. Intel Xeon Phi supports all-to-all, quadrant/hemisphere and sub-NUMA cluster SNC-4/SNC-2 modes of cache operation.

For large problem sizes, different memory and clustering modes were observed to have little impact on the time to solution for the three versions of the GAMESS code. For this reason, we simply chose the mode most easily available to us. In other words, since the choice of mode made little difference in performance, our choice of Quad-Cache mode was ultimately driven by convenience (this being the default choice in our particular environment). Our comments here apply to large problem sizes, so for small problem sizes, the user will have to experiment to find the most suitable mode(s).


\subsection{Description of chemical systems}
\label{ssec:chemical}
For benchmarks, a system consisting of parallel series of graphene sheets was chosen. This system is of interest to researchers in the area of (micro)lubricants \cite{kawai2016superlubricity}. A physical depiction of the configuration is provided in \Cref{fig:graphene}. The graphene-sheet system is ideal for benchmarking, because the size of the system is easily manipulated. Various Fock matrix sizes can be targeted by adjusting the system size.

\begin{figure}
	\includegraphics[width=\columnwidth]{Figure2}
	\caption{Model system of a C$_{2016}$ graphene bilayer. In the text, we refer to this system as 5~nm.
    		 There are two layers with size 5~nm by 5~nm.
             Each graphene layer consists of 1,008 carbon atoms.}
    \label{fig:graphene}
\end{figure}

\subsection{Characteristics of datasets}
\label{ssec:datasets}
In all, five configurations of the graphene sheets system were studied. The datasets for the systems studied are labeled as follows: 0.5~nm, 1.0~nm, 1.5~nm, 2.0~nm, and 5.0~nm.  \Cref{tab:chem} lists size characteristics of these configurations. The same 6-31G(d) basis set (per atom) was used in all calculations. For N basis functions, the density, Fock, AO overlap, one-electron Fock matrices and the matrix of MO coefficients are N$\times$N in size. These are the main data structures of significant size. Therefore, the benchmarks performed in this work process matrices which range from 660$\times$660 to 30,240$\times$30,240. For each of the systems studied, \Cref{tab:chem} lists the memory requirements of the three versions of GAMESS HF code.
Denoting $N_{BF}$ as the number of basis functions, the following equations describe the asymptotic $(N_{BF}\to\infty)$ memory footprint for the studied HF algorithms:
\begin{subequations}
	\label{eqn:mem}
	\begin{align}
		M_{MPI} =& 5/2 \cdot N_{BF}^2 \cdot N_{MPI\_per\_node}, 				\label{eqn:mem:mpi} \\
		M_{PrF} =& (2+N_{threads}) \cdot N_{BF}^2 \cdot N_{MPI\_per\_node}, 	\label{eqn:mem:prv} \\
		M_{ShF} =& 7/2 \cdot N_{BF}^2 \cdot N_{MPI\_per\_node},					\label{eqn:mem:shr}
	\end{align}
\end{subequations}
where $M_{MPI}$, $M_{PrF}$, $M_{ShF}$ denote the memory footprint of MPI-only, private Fock, and shared Fock algorithms respectively; $N_{threads}$ denotes the number of threads per MPI process for the OpenMP code, and $N_{MPI\_per\_node}$ denotes the number of MPI processes per KNL node. For OpenMP runs $N_{MPI\_per\_node}=4$, while for MPI runs the number of MPI ranks was varied from 64 to 256.

If one compares columns MPI versus Pr.F and Sh.F. in \Cref{tab:chem}, you will see that the private Fock code has about a 50 times less footprint compared to the stock MPI code. For the shared Fock code, the difference is even more dramatic with a savings of about 200 times. The ideal difference is 256 times since we compare 256 MPI ranks in the stock MPI code where all data structures are replicated versus 1 MPI rank with 256 threads for the hybrid MPI/OpenMP codes. But we introduced additional replicated structures (see \Cref{fig:buffer}) and many relatively small data structures are replicated also in the MPI/OpenMP codes. This explains the difference between the ideal and observed footprints.

Each of the aforementioned datasets was used to benchmark three versions of the GAMESS code. The first version is the stock GAMESS MPI-only release that is freely available on the GAMESS website~\cite{gamesswebsite}. The second version is a hybrid MPI/OpenMP code, derived from the stock release. This version has a shared density matrix, but a thread-private Fock matrix. The third version of the code is in turn derived from the second version; it has shared density and Fock matrices. A key objective was to see how these incremental changes allow one to manage (i.e., reduce) the memory footprint of the original code while simultaneously driving higher performance.

\section{Results}
\label{sec:results}

\subsection{Single node performance}
\label{ssec:singlenode}
The second generation Intel Xeon Phi processor supports four hardware threads per physical core. Generally, more threads per core can help hide latencies inherent in an application. For example, when one thread is waiting for memory, another can use the processor. The out-of-order execution engine is beneficial in this regard as well. To manipulate the placement of processes and threads, the \verb|I_MPI_DOMAIN| and \verb|KMP_AFFINITY| environment variables were used. 
We examined the performance picture when one thread per core is utilized and when four threads per core are utilized. As expected, the benefit is highest for all versions of GAMESS for two threads (or processes) per core. For three and four threads per core, some gain is observed, albeit at a diminished level. \Cref{fig:afty} shows the scaling curves with respect to the number of hardware threads utilized observed by us.

\begin{figure}
	\includegraphics[width=\columnwidth]{Figure3}
	\caption{Performance dependence on OpenMP thread affinity type for the shared Fock version of the GAMESS code
    		 on a single \intelphireg\ processor using the 1.0 nm benchmark.
             All calculations are performed in quad-cache mode.
             Four MPI ranks were used in all cases.
             The number of threads per MPI rank was varied from 1 to 64.}
    \label{fig:afty}
\end{figure}

\begin{figure}
	\includegraphics[width=\columnwidth]{Figure4}
	\caption{Scalability with respect to the number of hardware threads of the original MPI code
    		and two OpenMP versions on a single \intelphireg\ processor using the 1.0~nm benchmark.}
    \label{fig:singlescaling}
\end{figure}

As a first test, single-node scalability was examined with respect to hardware threads of all three versions of GAMESS. For the MPI-only version of GAMESS, the number of ranks was varied from~4 to~256. For the hybrid versions of GAMESS, the number of ranks times the number of threads per rank is the number of hardware threads targeted. The larger memory requirements of the original MPI-only code restrict the computations to, at most, 128 hardware threads. In contrast, the two hybrid versions can easily utilize all 256 hardware threads available. Finally, in general terms, on cache based memory architectures, it is expected that larger memory footprints potentially lead to more cache capacity and cache line conflict effects. These effects can lead to diminished performance, and this is yet another motivation to look at a hybrid MPI+X approach.

The results of our single-node tests are plotted in \Cref{fig:singlescaling}. It is found that using the private Fock version leads to the best time to solution for the 1.0~nm dataset, for any number of hardware threads. This version of the code is much more memory-efficient than the stock version but, because the Fock matrix data structure is private, it has a much larger memory footprint than the shared Fock version of GAMESS. Nevertheless, because the Fock matrix is private, there is less thread contention than the shared Fock version.

It was mentioned in \Cref{ssec:omp} that shared Fock algorithm introduces additional overhead for thread synchronization. For small numbers of Intel Xeon Phi threads, this overhead is expected to be low. Therefore the shared Fock version is expected to be on par with the other versions. Eventually, as the overhead of the synchronization mechanisms begins to increase, the private Fock version of the code is found to dominate. In the end, the private Fock version outperforms stock GAMESS because of the reduced memory footprint, and outperforms the shared Fock version because of a lower synchronization overhead.
Therefore, on a single node, the private Fock version gives the best time-to-solution of the three codes, but the shared Fock version strikes a (better) balance between memory utilization and performance.

\begin{figure}
	\includegraphics[width=\columnwidth]{Figure5}
	\caption{Time to solution (x axis, time in seconds) for different clustering and memory modes.
    		 Left column displays the small chemical system -- 0.5~nm bilayer graphene and
             right column displays one of the largest molecules bilayer graphene -- 2.0~nm.}
    \label{fig:tts}
\end{figure}

Beyond this, one must consider the choice of memory mode and cluster mode of the Intel Xeon Phi processor. It should be noted that, depending on the compute and memory access patterns of a code, the choice of memory and cluster mode can be a potentially significant performance variable. The performance impact of different memory and cluster modes is examined for the 0.5~nm (small) and~2.0~nm (large) datasets. The results are shown in \Cref{fig:tts}. For both datasets, some variation in performance is apparent when different cluster modes and memory modes are used. The smaller dataset indicates more sensitivity to these variables than the larger dataset. Also, for both data sizes the private Fock version performs best in all cluster and memory modes tested. Also, except in the All-to-All cluster mode, the shared Fock version significantly outperforms the MPI-only stock version. In the All-to-All mode, the MPI-only version actually outperforms the shared Fock version for small datasets, and the two versions are close to parity for large datasets. In total, it is concluded that the quadrant-cache cluster-memory mode is best suited to the design of the GAMESS hybrid codes.

\subsection{Multi-node performance}
It is very important to note that the total number of MPI ranks for GAMESS is actually twice the number of compute ranks because of the DDI. The DDI layer was originally implemented to support one-sided communication using MPI-1. For GAMESS developers, the benefit of DDI is convenience in programming. The downside is that each MPI compute process is complemented by an MPI data server~(DDI) process, which clearly results in increased memory requirements. Because data structures are replicated on a rank-by-rank basis, the impact of DDI on memory requirements is particularly unfavorable to the original version of the GAMESS code. To alleviate some of the limitations of the original implementation, an implementation of DDI based on MPI-3 was developed \cite{pruitt2016private}. Indeed, by leveraging the ``native'' support of one-sided communication in MPI-3, the need for a DDI process alongside each MPI rank was eliminated. For all three versions of the code benchmarked here, no DDI processes were needed.

\begin{figure}
	\includegraphics[width=\columnwidth]{Figure6}
	\caption{Multi-node scalability of the Private Fock and the Shared Fock hybrid MPI-OpenMP
    		 and the MPI-only stock GAMESS codes on the Theta machine with the 2.0~nm dataset.
             The quad-cache cluster-memory mode was used for all data points.}
    \label{fig:2nm}
\end{figure}

\Cref{fig:2nm} shows the multi-node scalability of the MPI-only version of GAMESS versus the private Fock and the shared Fock hybrid versions. It is important to appreciate at the outset that the multi-node scalability of the original MPI-only version of GAMESS is already reasonable. For example, the code scales linearly to 256 Xeon Phi nodes, and it is really the memory footprint bottleneck that limits how well all the Xeon Phi cores on any given node can be used. This pressure is reduced in the private Fock version of the code, and it is essentially eliminated in the shared Fock version. Overall, for the 2~nm dataset, the shared Fock code runs about six times faster than stock GAMESS on 512 Xeon Phi processors. It resulted from the better load balance of the shared Fock algorithm that uses all four shell indices -- two are used in MPI and two are used in OpenMP workload distribution. The actual timings and efficiencies are listed in \Cref{tab:efficiency}.

\begin{table}
\begin{threeparttable}
  \caption{Parallel efficiency of the three different HF algorithms using 2.0~nm dataset}
  \label{tab:efficiency}
  \begin{tabularx}{\columnwidth}{XYYYYYY}

  \toprule
    	\multirow{2}{*}{\# Nodes}		&	\multicolumn{3}{c}{Time-to-solution, s} &
                            \multicolumn{3}{c}{Parallel efficiency, \%} \\
        \cmidrule(rl{0.75em}){2-4} \cmidrule(l){5-7}
  					&	{MPI\tnote{a}} &	{Pr.F.\tnote{a}} &	{Sh.F.\tnote{a}} &
                        {MPI\tnote{a}} &	{Pr.F.\tnote{a}} &	{Sh.F.\tnote{a}} \\

  	\midrule
		4	&	2661	&	1128	&	1318	&	100	&	100	&	100 \\
		16	&	685		&	288		&	332		&	97	&	98	&	99 \\
		64	&	195		&	78		&	85		&	85	&	90	&	97 \\
		128	&	118		&	49		&	43		&	70	&	72	&	96 \\
		256	&	85		&	44		&	23		&	49	&	40	&	90 \\
		512	&	82		&	44		&	13		&	25	&	20	&	79 \\
    \bottomrule
   \end{tabularx}

 	\begin{tablenotes}
 		\item [a] MPI-only SCF code
    	\item [b] Private Fock SCF code
    	\item [c] Shared Fock SCF code
 	\end{tablenotes}
\end{threeparttable}
\end{table}

\begin{figure}
	\includegraphics[width=\columnwidth]{Figure7}
	\caption{Scalability of the Shared Fock hybrid MPI-OpenMP version of GAMESS on the Theta machine
    		 for the 5.0~nm (i.e. large) dataset in quadrant cache mode on 3,000 \intelphireg\ processors.
             The results here are for 4~MPI ranks per node with 64~threads per rank,
             giving full saturation (in terms of hardware threads) on every \intelphireg\ node. For each point in the figure, we show the time in seconds.}
    \label{fig:5nm}
\end{figure}

\Cref{fig:5nm} shows the behavior of the shared Fock version of GAMESS for the 5~nm dataset. It is the largest dataset we could fit in memory on Theta. Since we run on 4~MPI ranks the memory footprint is approximately 208~GB per node. This figure shows good scaling of the code up to 3,000 Xeon Phi nodes, which is equal to 192,000 cores (64~cores per node).
 We propose a novel commonsense reasoning challenge, \textsc{RiddleSense}, which requires complex commonsense skills for reasoning about creative and counterfactual questions, coming with a large multiple-choice QA dataset.  
 We systematically evaluate recent commonsense reasoning methods over the proposed \textsc{RiddleSense} dataset, and find that the best model is still far behind human performance, suggesting that there is still much space for commonsense reasoning methods to improve.
 We hope \textsc{RiddleSense} can serve as a benchmark dataset for future research targeting complex commonsense reasoning and computational creativity.


\section*{Acknowledgements}
This research is supported in part by the Office of the Director of National Intelligence (ODNI), Intelligence Advanced Research Projects Activity (IARPA), via Contract No. 2019-19051600007, the DARPA MCS program under Contract No. N660011924033 with the United States Office Of Naval Research, the Defense Advanced Research Projects Agency with award W911NF-19-20271, and NSF SMA 18-29268. The views and conclusions contained herein are those of the authors and should not be interpreted as necessarily representing the official policies, either expressed or implied, of ODNI, IARPA, or the U.S. Government. We would like to thank all the collaborators in USC INK research lab and the reviewers for their constructive feedback on the work.

% ---- Bibliography ---- 
%
% BibTeX users should specify bibliography style 'splncs04'.
% References will then be sorted and formatted in the correct style.
%
% \bibliographystyle{splncs04}
% \bibliography{mybibliography}
%
% For citations of references, we prefer the use of square brackets
% and consecutive numbers. Citations using labels or the author/year
% convention are also acceptable. The following bibliography provides
% a sample reference list with entries for journal
% articles~\cite{ref_article1}, an LNCS chapter~\cite{ref_lncs1}, a
% book~\cite{ref_book1}, proceedings without editors~\cite{ref_proc1},
% and a homepage~\cite{ref_url1}. Multiple citations are grouped
% \cite{ref_article1,ref_lncs1,ref_book1},
% \cite{ref_article1,ref_book1,ref_proc1,ref_url1}.

\bibliography{References}{}
\bibliographystyle{splncs04}
\end{document}
