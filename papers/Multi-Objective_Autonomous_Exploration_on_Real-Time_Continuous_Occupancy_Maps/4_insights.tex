\vspace{-15pt}
\section{Main Experimental Insights}\label{sec:insights}%\vspace{-10pt}
%=== Real time Bayesian Hilbert Map ===
The quality and computation time of BHM highly rely on the choice of kernel bandwidth and number of hinge points.
To make BHM more applicable for real-time scenarios, we propose a convex-hull-based hinge points selection strategy.
% Before the exploration task starts, a set of hinge points are defined in the whole area which is to be explored.
% For each time step, a convex hull is formed based on the training points.
At each time step, a slightly enlarged convex hull is built based on the training points within the field of view or sensing radius.
This tighter ``bounding box'' will enclose the most relevant hinge points.
We generate a more compact feature vector using these selected hinge points and only predict the occupancy probability locally.
% The hull-restricted hinge points will generate a new feature vector for predicting the local occupancy probability. 

%The hinge points which are not within the area covered by the training data will be discarded for the current local prediction. 
% By only including the important hinge points in the training and prediction process, the computation time is significantly decreased.
% These local prediction values will be fused into a global map in a way that no computation is needed for explored areas.

%=== ParetoMCTS ===
In the child node selection of ParetoMCTS, there is a weight parameter that balances exploitation of the recently discovered most promising child node and exploration of alternatives which may turn out to be a superior choice at later time.
Normalizing the exploitation scores or multiplying the exploration weight by the maximum exploitation score makes the weight parameter insensitive to the scales of objectives. 
% To take advantage of the continuous occupancy prediction property of BHM, we do not need to predict the occupancy values on a predefined grid and plan on this grid.
% Instead, we just need to predict on a specific position when the planner needs the occupancy information.
In addition, when the tree enters unknown areas, everywhere seems equally uncertain or informative, which makes the tree unable to grow asymmetrically, limiting the searching depth.
% We believe changing the tree growth mode from MCTS style to Rapidly-Exploring Random Tree (RRT) style will improve the estimation of the informativeness of unknown area, which is one of tasks that we plan to test and compare. 