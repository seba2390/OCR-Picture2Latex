\vspace{-15pt}
\section{Experimental Plan}\label{sec:experiments}%\vspace{-10pt}

%=== What we already have done so far ===
% We propose an exploration system which integrates Bayesian Hilbert Map and Pareto Monte Carlo Tree Search. 
Our preliminary results show that the proposed system could achieve efficient exploration in unknown environments.
However, making our proposed system work in real environment still requires further efforts due to some stumbling blocks such as high dimensional working space, cluttered real-world environment, and the requirement of a  consistent replanning mechanism.
To validate that our proposed system is able to achieve autonomous exploration in real environments, we schedule the following experiments for the remaining work.

%=== What we will do in the future ===
\begin{itemize}
    %=== Incorporate new informative metrics ===
    \item
    Compare our proposed system with the geometric-frontiers-based benchmark~\cite{cieslewski2017rapid} and the information-theoretic-frontiers-based benchmark~\cite{francis_functional_2020,bai_information-theoretic_2016}. We will compare the map entropy vs. exploration time of different systems and show that our system could outperform baselines. %and achieve a high exploration efficiency. 
    %=== Conduct 2D exploration using Jackal in real environment ===
    \item
    To validate the effectiveness and efficiency of our system in real-world applications, we plan to conduct ground vehicle  experiments with a Clearpath Jackal robot equipped with a Velodyne Lidar in real indoor environments. The localization will be obtained by a Lidar-based SLAM, and the BHM will be built upon the Lidar measurements. We will also compare our exploration method with other exploration and planning algorithms, such as kinodynamic A*~\cite{zhou2019robust}, RRT*~\cite{karaman2011sampling} and our recent Markov Decision Process~(MDP) based decision-theoretic planning framework~\cite{xu2020kernel}.
    %=== Extend our system to 3D environment ===
    %=== Conduct exploration with a quadrotor in real environment ===
    \item
    %To make our system flexibly operate in high dimensional environment, 
    Then we plan to extend our mapping and planning methods to the 3D environment and conduct experiments with our DJI M100 quadrotor equipped with an RGBD camera. We have previously integrated and validated the quadrotor's basic perception, control, and planning functionalities. Particularly, for this task we will use a vision-based odometry~\cite{qin2018vins} for drone localization (we have validated the performance on a ground vehicle), and further incorporate the 3D BHM with advanced 3D reconstruction methods (e,g., \cite{wang2019real}) to accomplish this task in mapping the 3D space. 
    % \item
    % Eventually, we will show that a quadrotor with our proposed methods could efficiently explore an unknown indoor room-like environment. 
\end{itemize}



