\section{Results}
\label{sec:3}
% Always give a unique label
% and use \ref{<label>} for cross-references
% and \cite{<label>} for bibliographic references
% use \sectionmark{}
% to alter or adjust the section heading in the running head

In this section, we demonstrate our proposed exploration system could efficiently explore unknown environments in real time. Our experiments are conducted with a Gazebo simulator and the environment to be mapped is shown in Fig.~\ref{fig:env}(a). In our experiment, for Bayesian Hilbert Mapping, we set hinge points resolution and query points resolution as 0.2 and 0.1, respectively. The kernel lengthscale $\gamma=40.0$. For ParetoMCTS, we set the maximum iteration for each search as 500. \ZC{Explain the parameter settings for ParetoMCTS.} Fig.~\ref{fig:exploration}(b)-(l) show the exploration process in our simulated environment.
% \begin{figure}[htbp]
%     \centering
%     \includegraphics[width=0.6\linewidth]{figs/env.png}
%     \caption{
%     }
%     \label{fig:pareto_mcts}
% \end{figure}

\begin{figure}[htbp]
    \begin{minipage}[tb]{0.33\linewidth}
        \centering
        \includegraphics[width=1\linewidth]{figs/env.png}\\(a)
        \label{fig:env}
    \end{minipage}
    \begin{minipage}[tb]{0.33\linewidth}
        \centering
        \includegraphics[width=1\linewidth]{figs/0.png}\\(b)
        \label{fig:exploration_0}
    \end{minipage}
    \begin{minipage}[tb]{0.33\linewidth}
        \centering
        \includegraphics[width=1\linewidth]{figs/1.png}\\(c)
        \label{fig:exploration_1}
    \end{minipage}
    \begin{minipage}[tb]{0.33\linewidth}
        \centering
        \includegraphics[width=1\linewidth]{figs/2.png}\\(d)
        \label{fig:exploration_2}
    \end{minipage}
    \begin{minipage}[tb]{0.33\linewidth}
        \centering
        \includegraphics[width=1\linewidth]{figs/4.png}\\(e)
        \label{fig:exploration_3}
    \end{minipage}
    \begin{minipage}[tb]{0.33\linewidth}
        \centering
        \includegraphics[width=1\linewidth]{figs/6.png}\\(f)
        \label{fig:exploration_4}
    \end{minipage}
    \begin{minipage}[tb]{0.33\linewidth}
        \centering
        \includegraphics[width=1\linewidth]{figs/8.png}\\(g)
        \label{fig:exploration_5}
    \end{minipage}
    \begin{minipage}[tb]{0.33\linewidth}
        \centering
        \includegraphics[width=1\linewidth]{figs/10.png}\\(h)
        \label{fig:exploration_6}
    \end{minipage}
    \begin{minipage}[tb]{0.33\linewidth}
        \centering
        \includegraphics[width=1\linewidth]{figs/12.png}\\(i)
        \label{fig:exploration_7}
    \end{minipage}
    \begin{minipage}[tb]{0.33\linewidth}
        \centering
        \includegraphics[width=1\linewidth]{figs/13.png}\\(j)
        \label{fig:exploration_8}
    \end{minipage}
    \begin{minipage}[tb]{0.33\linewidth}
        \centering
        \includegraphics[width=1\linewidth]{figs/14.png}\\(k)
        \label{fig:exploration_9}
    \end{minipage}
    \begin{minipage}[tb]{0.33\linewidth}
        \centering
        \includegraphics[width=1\linewidth]{figs/17.png}\\(l)
        \label{fig:exploration_11}
    \end{minipage}
    \caption{\small
         Exploration process on top of Bayesian Hilbert Map with Pareto Monte Carlo Tree Search.
    }
    \label{fig:exploration}
\end{figure}

Our proposed system not only considers the geometric frontiers but also the high uncertainty areas which have high entropy values. In this way, the robot is expe
\section{Experiments Completed or Scheduled}
\section{Main Experimental Insights}

% The experiments setting is shown in Fig.~\ref{fig:test_env}. Our experiments are conducted in a Gazebo simulator, where a quadrotor with a RGBD camera is used for mapping, as shown in Fig.~\ref{fig:gazebo_env}. The depth image from camera is converted to laser scan in a certain 2D plane, like the one shown in Fig.~\ref{fig:scan}. With this scan, traditional occupancy grid mapping~\cite{thrun2002probabilistic} gives us the result like in Fig.~\ref{fig:ogm}. In this paper, we use Bayesina Hilber Mapping, which require training data sampled from sensor measurements. As the robot is moving forward, for each frame of measurements, training data like in Fig.~\ref{fig:training} will be generated and be used for training our occupancy prediction model.

% \subsection{Real-time Bayesian Hilbert Mapping} \label{sec:3.1}

% Two hyperparameters influence the computation time in Bayesian Hilbert Mapping: kernel resolution and kernel bandwidth. The kernel resolution refers to the distribution density of hinge points while the kernel bandwidth is the $\gamma$ in RBF kernel. In this paper, we predefine a fixed kernel resolution before mapping starts. With different kernel resolution and kernel bandwidth, the computation time for training and predicting vary. Apply our proposed convex hull selection strategy, the computation time is saved. Some comparison could be seen in seen in Table.~\ref{tab:reso} and Table.~\ref{tab:band}. Using our training data in Fig.~\ref{fig:training}, our predicted occupancy map and entropy map are obtained as shown in Fig.~\ref{fig:maps}.

% \ZC{TODO--(1) Retrim Fig.~\ref{fig:test_env}; (2) Enlarge the convex hulls; (3) Remeasure the computation time for both local mapping and global mapping; (4) Change the pics in Fig.~\ref{fig:maps}; (5) Change the legend color; (6) Remake the two tables in a more compact way; (7) Explain the local time and global time.}

% \begin{figure}[htbp]
%     \begin{minipage}[t]{0.24\linewidth}
%     \centering
%     \includegraphics[width=1\linewidth]{figs/gazebo_env.png}
%     \label{fig:gazebo_env}
%     \end{minipage}
%     \begin{minipage}[t]{0.24\linewidth}
%     \centering
%     \includegraphics[width=1\linewidth]{figs/scan.png}
%     \label{fig:scan}
%     \end{minipage}
%     \begin{minipage}[t]{0.24\linewidth}
%     \centering
%     \includegraphics[width=1\linewidth]{figs/ogm.png}
%     \label{fig:ogm}
%     \end{minipage}
%     \begin{minipage}[t]{0.24\linewidth}
%     \centering
%     \includegraphics[width=1\linewidth]{figs/training.png}
%     \label{fig:training}
%     \end{minipage}
%     \caption{
%         \small (a)~The simulated environment in Gazebo; (b)~The scan data generated from an RGB-D camera; (c)~The mapping result from traditional Occupancy Grid Mapping; (d)~Training data for BHM. Green points are occupied training data while red points are free training data.
%     }
%     \label{fig:test_env}
% \end{figure}

% \begin{table}
% \parbox{.45\linewidth}{
%     \centering
%     \caption{\label{tab:reso}Comparison of original BHM and ours method on computation time for training and predicting with varying kernel resolutions. Kernel bandwidth stays at 15.0.}
%     \begin{tabular}{||c c c||}
%          \hline
%          \shortstack{Hinge \\ resolution}  & \shortstack{Original \\ BHM~(sec)} & Ours~(sec) \\ [0.5ex] 
%          \hline\hline
%          0.1 & 1.2 & ? \\ 
%          \hline
%          0.2 & 0.38 & ?  \\
%          \hline
%          0.3 & 0.18 & ? \\
%          \hline
%          0.4 & 0.1 & ?\\
%          \hline
%          0.5 & 0.08 & ?\\ [1ex] 
%          \hline
%     \end{tabular}
% }
% \hfill
% \parbox{.45\linewidth}{
%     \centering
%     \caption{\label{tab:band}Comparison of original BHM and ours method on computation time for training and predicting with varying kernel bandwidths. Kernel resolution stays at 0.2m.}
%     \begin{tabular}{||c c c||} 
%          \hline
%          \shortstack{Kernel \\ lengthscale} & \shortstack{Original \\ BHM~(sec)} & Ours~(sec) \\ [0.5ex] 
%          \hline\hline
%          1.0 & 0.2 & ? \\ 
%          \hline
%          5.0 & 0.22 & ?\\
%          \hline
%          10.0 & 0.27 & ?\\
%          \hline
%          15.0 & 0.37 & ? \\
%          \hline
%          18.0 & 0.4 & ?\\ [1ex] 
%          \hline
%     \end{tabular}
% }
% \end{table}

% \begin{figure}[htbp]
%     \begin{minipage}[t]{0.19\linewidth}
%         \centering
%         \includegraphics[width=1\linewidth]{figs/bhm_occ.png}
%         \label{fig:bhm_occ}
%     \end{minipage}
%     \begin{minipage}[t]{0.19\linewidth}
%         \centering
%         \includegraphics[width=1\linewidth]{figs/bhm_entropy.png}
%         \label{fig:bhm_entropy}
%     \end{minipage}
%     \begin{minipage}[t]{0.19\linewidth}
%         \centering
%         \includegraphics[width=1\linewidth]{figs/tradition_ogm.png}
%         \label{fig:tradition_ogm}
%     \end{minipage}
%     \begin{minipage}[t]{0.19\linewidth}
%         \centering
%         \includegraphics[width=1\linewidth]{figs/bhm_global_occ.png}
%         \label{fig:bhm_global_occ}
%     \end{minipage}
%     \begin{minipage}[t]{0.19\linewidth}
%         \centering
%         \includegraphics[width=1\linewidth]{figs/bhm_global_entropy.png}
%         \label{fig:bhm_global_entropy}
%     \end{minipage}
%     \caption{\small
%          (a) Occupancy map from Bayesian Hilbert Mapping.
%          Green indicates free spaces, red indicates obstacles while brown indicates uncertain areas;
%          (b) Entropy map from Bayesian Hilbert Mapping. Green indicates low-entropy values while red indicates high-entropy values;
%          (c) A global map built by traditional Occupancy Grid Mapping.
%          The robot follows a predefined 10m straight path to build the global map;
%          (d) The global occupancy map built by Bayesian Hilbert Mapping;
%          (e) The global entropy map built by Bayesian Hilbert Mapping.
%     }
%     \label{fig:maps}
% \end{figure}


% \subsection{Pareto Monte Carlo Tree Search} \label{sec:3.2}

% With the occupancy map and reward map from Bayesian Hilbert Mapping, Pareto Monte Carlo Tree Search will plan a trajectory that has the highest accumulative reward value. Since our task is to quickly explore the unknown environment and the occupancy map and reward map are built incrementally, hence the treatment to unexplored areas is essential to the behavior of the robot. Specifically, for the occupancy map, we could treat the uncertain areas as either occupied or free spaces; For reward map, from the definition of entropy, we should treat the unexplored areas as high-reward spaces. However, this will cause much space of the map as high-reward areas such that the planning will be distracted. Therefore, we could treat the reward values in unexplored areas as either high or low. Since Pareto Monte Carlo Tree Search has to use occupancy map and reward map simultaneously, we can have four different combinations of occupancy map and reward map, as shown in Fig.~\ref{fig:pmcts}. The red tree is the tree returned from Pareto Monte Carlo Tree Search. The long purple path is the trajectory that has the highest accumulative reward on the tree. The short blue path is the trajectory executed by the robot.

% \begin{figure}[htbp]
%     \centering
%     \includegraphics[width=1\linewidth]{figs/pmcts.png}
%     \caption{\small
%         Planning results from Pareto Monte Carlo Tree Search with different combinations of occupancy map and reward map.
%         If the uncertain areas are regarded as occupied areas, we name the occupancy map as 1, else if the uncertain areas are regarded as free space, we name the occupancy map as 0.
%         Similarly, we name the reward map whose uncertain spaces have high reward values as 1 while the reward map whose uncertain areas have low reward values as 0.
%         We then have four combinations: (a) 1-1; (b) 1-0; (c) 0-1; (d) 0-0.
%     }
%     \label{fig:pmcts}
% \end{figure}

% If uncertain areas on occupancy map are treated as free spaces, then the tree will have much chance to grow into places where we are not sure if there are obstacles or not. This may give robot more courage to explore new areas but add more safety concern as well in the same time. In mobile robots navigation tasks, we have to treat safety as a hard constraints, particularly for flying robots like quadrotors. Therefore, in this paper, we treat the occupancy map in a conservative way where uncertain areas are treated as occupied spaces. The resulting occupancy map is like the left pics in Fig.~\ref{fig:pmcts}(a) or Fig.~\ref{fig:pmcts}(b). 

% In the reward map, if uncertain areas are treated as high-reward spaces, then too much spaces on reward map will be assigned as high rewards. Instead, we want to distinguish the most important areas~(i.e., the frontiers in front of the robot) among those high-reward places such that the robot will be guided in a highly efficient way. A simple way to do this is to set the entropy value at places where the occupancy probability is 0.5 as 0 and then renormalize the entropy value on the whole map. The reshaped reward map is like the right pics in Fig.~\ref{fig:pmcts}(b) or Fig.~\ref{fig:pmcts}(d). 

% \subsection{Exploration} \label{sec:3.3}
% \begin{figure}[htbp]
%     \begin{minipage}[tb]{0.19\linewidth}
%         \centering
%         \includegraphics[width=1\linewidth]{figs/exploration_1.png}\\t=0
%         \label{fig:exploration_1}
%     \end{minipage}
%     \begin{minipage}[tb]{0.19\linewidth}
%         \centering
%         \includegraphics[width=1\linewidth]{figs/exploration_2.png}\\t=5
%         \label{fig:exploration_2}
%     \end{minipage}
%     \begin{minipage}[tb]{0.19\linewidth}
%         \centering
%         \includegraphics[width=1\linewidth]{figs/exploration_3.png}\\t=14
%         \label{fig:exploration_3}
%     \end{minipage}
%     \begin{minipage}[tb]{0.19\linewidth}
%         \centering
%         \includegraphics[width=1\linewidth]{figs/exploration_4.png}\\t=21
%         \label{fig:exploration_4}
%     \end{minipage}
%     \begin{minipage}[tb]{0.19\linewidth}
%         \centering
%         \includegraphics[width=1\linewidth]{figs/exploration_5.png}\\t=25
%         \label{fig:exploration_5}
%     \end{minipage}\\
%     \begin{minipage}[tb]{0.19\linewidth}
%         \centering
%         \includegraphics[width=1\linewidth]{figs/exploration_6.png}\\t=32
%         \label{fig:exploration_6}
%     \end{minipage}
%     \begin{minipage}[tb]{0.19\linewidth}
%         \centering
%         \includegraphics[width=1\linewidth]{figs/exploration_7.png}\\t=36
%         \label{fig:exploration_7}
%     \end{minipage}
%     \begin{minipage}[tb]{0.19\linewidth}
%         \centering
%         \includegraphics[width=1\linewidth]{figs/exploration_8.png}\\t=39
%         \label{fig:exploration_8}
%     \end{minipage}
%     \begin{minipage}[tb]{0.19\linewidth}
%         \centering
%         \includegraphics[width=1\linewidth]{figs/exploration_9.png}\\t=42
%         \label{fig:exploration_9}
%     \end{minipage}
%     \begin{minipage}[tb]{0.19\linewidth}
%         \centering
%         \includegraphics[width=1\linewidth]{figs/exploration_11.png}\\t=48
%         \label{fig:exploration_11}
%     \end{minipage}
%     \caption{\small
%          Exploration process on top of Bayesian Hilbert Map with Pareto Monte Carlo Tree Search.
%          The process is shown as the time order.
%     }
%     \label{fig:exploration}
% \end{figure}

% % \begin{figure} \vspace{-3pt}
% %   \centering
% %   \subfigure[]
% %   	{\label{fig:exploration_1}\includegraphics[width=1.2in]{figs/exploration_1.png}}
% %   \subfigure[]
% %   	{\label{fig:exploration_2}\includegraphics[width=1.2in]{figs/exploration_2.png}}
% %   \subfigure[]
% %   	{\label{fig:exploration_3}\includegraphics[width=1.2in]{figs/exploration_3.png}}
% %   \subfigure[]
% %   	{\label{fig:exploration_4}\includegraphics[width=1.2in]{figs/exploration_4.png}}
% %   \subfigure[]
% %   	{\label{fig:exploration_5}\includegraphics[width=1.2in]{figs/exploration_5.png}}
% %   \subfigure[]
% %   	{\label{fig:exploration_6}\includegraphics[width=1.2in]{figs/exploration_6.png}}
% %   \subfigure[]
% %   	{\label{fig:exploration_7}\includegraphics[width=1.2in]{figs/exploration_7.png}}
% %   \subfigure[]
% %   	{\label{fig:exploration_8}\includegraphics[width=1.2in]{figs/exploration_8.png}}
% %   \subfigure[]
% %   	{\label{fig:exploration_9}\includegraphics[width=1.2in]{figs/exploration_9.png}}
% %   \subfigure[]
% %   	{\label{fig:exploration_11}\includegraphics[width=1.2in]{figs/exploration_11.png}}
% %   \caption{\small } \vspace{-10pt}
% % \label{fig:exploration}  
% % \end{figure}

% Integrating Bayesian Hilbert Mapping and Pareto Monte Carlo Tree Search gives us a system which is able to autonomously explore an unknown environments. In our experiments, for Bayesian Hilbert Mapping, we set $\gamma=18.0$. The resolutions for hinge points and query points are set as 0.25m and 0.2m, respectively. For Pareto Monte Carlo Tree Search, we set the maximum iteration for growing the tree as 200. We assume the localization is perfect and access the odometry information directly from the Gazebo simulator. The space to be mapped is a corridor environment as shown in Fig.~\ref{fig:gazebo_env}. An exploration process in a simulated environment is conducted and the results are shown in Fig.~\ref{fig:exploration}. 