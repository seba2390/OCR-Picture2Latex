\section{Motivation, Problem Statement, Related Work} \label{sec:introduction}
%=== Pull figure ===
\begin{figure}[htbp] \vspace{-15pt}
    \centering
    % Nearest geometric frontier
    \begin{minipage}[t]{0.3\linewidth}
        \centering
        \includegraphics[width=1\linewidth,height=1\linewidth]{figs/ogm_frontiers.png}
        \caption{
            Discrete occupancy grid map with frontiers pointed by arrows.
            % Exploring frontiers pointed by green arrows will uncover larger unknown area than that of the blue ones.
            % Inconsistent occupancy predictions are circled by orange, which motivates us to embrace continuous occupancy mapping.
        }
        \label{fig:pull_figure_1}
    \end{minipage}
    % % Most informative frontier
    % \begin{minipage}[t]{0.3\linewidth}
    %     \centering
    %     \includegraphics[width=1\linewidth,height=1\linewidth]{figs/gp_frontiers.png}
    %     \caption{Informative locations}
    %     \label{fig:pull_figure_2}
    % \end{minipage}
    % Informative area
    \quad \quad 
    \begin{minipage}[t]{0.3\linewidth}
        \centering
        \includegraphics[width=1\linewidth,height=1\linewidth]{figs/informative_area.png}
        \caption{Exploring frontiers that lead to large unknown areas.}
        \label{fig:pull_figure_3}
    \end{minipage} %\vspace{-15pt}
\end{figure}

%=== Autonomous exploration is important ===
Autonomous exploration in unknown environments is a needed capability for many robotic applications.
%=== What problem are we solving? ===
Existing exploration frameworks  rely on either geometric frontiers~\cite{yamauchi1997frontier,burgard2000collaborative,shen2012stochastic,cieslewski2017rapid} or  information-theoretic frontiers~\cite{charrow2015information,zhang2019fsmi,francis_functional_2020,francis_occupancy_2019,bai_information-theoretic_2016,jadidi_mutual_2015}, with which separate planners are required to plan collision-free paths towards the selected frontiers.
However, \textit{just because a frontier itself is informative does not necessarily mean that the robot will be in an informative ``area" after reaching that frontier}.
For example, in Fig.\,\ref{fig:pull_figure_1}, exploring the green arrow pointed open areas will uncover more unknown spaces compared to visiting those indicated by blue arrows.
Thus the frontiers of blue arrows will lead to ``less informative" actions compared with those of the green arrows.
%=== Our solution ===
In this work, we propose to use a multi-objective variant of Monte-Carlo tree search --- ParetoMCTS --- that provides a non-myopic Pareto optimal action sequence and hence leads the robot to frontiers with the greatest extent of unknown areal uncovering (Fig.\,\ref{fig:pull_figure_3} demonstrates that our planner prefers the frontier at the bottom because it leads to a larger uncovered space).
% The red dots in Fig.\,\ref{fig:pull_figure_1} and the bright area in Fig\,\ref{fig:pull_figure_2} are frontiers derived from a discrete occupancy map and a continuous occupancy map, respectively.
%
Also importantly, some ``fake'' frontiers (circled in orange) can present in the discrete occupancy grid map (Fig.\,\ref{fig:pull_figure_1}) due to the underlying inappropriate assumption that grids are independent to each other.
%To avoid these inconsistent results by taking into consideration the inter-dependencies among occupancy values, 
To fix such issue, we need to explicitly model the inter-dependencies among occupancy values, and hereby we adopt the Bayesian Hilbert Map (BHM)~\cite{senanayake2017bayesian} for continuous occupancy mapping which, however, is computationally prohibitive for large complex problems. 
We propose strategies and preliminary results for addressing above issues for real-time environment exploration applications. %further made it more applicable to real-time tasks.
% TODO: no need for two-stage planning
%=== Results and take-home messages ===
% \WZ{Simulation and field experiments to validate the effectiveness and efficiency of our method?}


%%%%%%%%%%%%%%%%%%%%%%%%%%%%%%%%%%%%%%%%%%%%%%%%%%%%%%%%%%%%%%%%%%%%%%%%%%%%%%%%%%%%%%%%%%%%%%%%%%%%%%%%%%%%%%%%%%%%%%%%%%%%%%%%%%
% %=== Introduction to autonomous exploration ===
% Exploration in unknown environments is an essential capability of autonomous systems. 
% %It also serves as a fundamental techniques for many real applications, such as search and rescue, disaster relief, industry inspection and so on. 
% This problem becomes challenging due to the complexity of the realistic environment and the coupling of perception and planning. Prevalent methods on robotic explorations include frontier-based and information-based.

% %=== Geometric-frontier methods ===
% The frontier-based methods explicitly detect frontiers in the current map, then a path is planned to guide the robot to move to the frontiers. The concept of \textit{frontier-based exploration} was firstly proposed by \cite{yamauchi1997frontier}. Frontiers are defined as the boundaries of free spaces and uncertain spaces. Leading robots to frontiers will result in a quick exploration of unknown areas of the environment to be explored. \cite{burgard2000collaborative} proposed a frontier-based target point selection strategy for multi-agent exploration. The proposed strategy simultenously consider the costs of reaching a target point and the utility of target points. A 3D frontier detection method, named Stochastic Differential Equation-based Exploration algorithm~(SDEE) was proposed in \cite{shen2012stochastic}. The proposed method determine regions for further exploration based on a stochastic differential equation. The authors demonstrate their method on quadrotors in 3D map representation. Some extensions of frontier-based exploration on UAV-UGV collaboration were proposed in \cite{butzke20153}~\cite{wang2018collaborative}. \cite{cieslewski2017rapid} proposed an extension of classical frontier-based exploration method, which facilitates exploration at high speeds  by rapidly selecting a goal frontier from a quadrotor's field of view. The local goal is selected such that the change in velocity to reach it is minimized.

% %=== Information-theoretic methods (still frontier :-)) ===
% The information-based methods select path that maximizes certain information-theoretic metrics, i.e., Shannon Mutual Information~(MI) to actively maximize the information gathered by the robot. Although Mutual Information has been successfully demonstrated to navigate robots to explore new areas~\cite{julian2014mutual}, it is computationally costly. To overcome the bottleneck, many variants of MI were proposed, such as CSQMI~\cite{charrow2015information}, FSMI~\cite{zhang2019fsmi}, and FCMI. Those proposed MI-based criteria could be computed quickly and are applicable for real-time applications.

% %=== Occupancy grid map -> continuous occupancy mapping ===

% %=== Our method ===
% % Independent grids -> interdependent grids -> speedup
% % An informative point -> an informative direction
% % global-local planner -> single planner

% %=== Paper organization (optional) ===