\documentclass[10pt,twocolumn,twoside]{IEEEtran}
%\documentclass[10pt,onecolumn]{IEEEtran}
%\usepackage{setspace}
%\doublespacing
\usepackage{amsmath,amsopn,amssymb,stackrel}
\usepackage{graphicx,xspace,color,soul}
\usepackage{epsfig,subfigure}
\usepackage{longtable,multirow}
\usepackage{array,float}
\usepackage{cite,citesort}
\usepackage{url}
% make vector in bold
\renewcommand{\vec}[1]{\mathbf{#1}}
\usepackage{algorithm,algorithmicx,algpseudocode}
\renewcommand{\algorithmicrequire}{\textbf{Input:}}
\renewcommand{\algorithmicensure}{\textbf{Output:}}
\newcommand\floor[1]{\lfloor#1\rfloor}
\newcommand\ceil[1]{\lceil#1\rceil}

\usepackage{epstopdf}
\usepackage{stfloats}


\newcommand{\red}[1] {\textcolor[rgb]{1.0,0.0,0.0}{{#1}}}
\newcommand{\blue}[1] {\textcolor[rgb]{0.0,0.0,1.0}{{#1}}}

\newcounter{mytempeqncnt}
\graphicspath{{figs/}}

\begin{document}
\title{Joint Denoising / Compression of Image Contours via Shape Prior and Context Tree}

\author{
Amin Zheng~\IEEEmembership{Student Member,~IEEE},
Gene Cheung~\IEEEmembership{Senior Member,~IEEE},
Dinei Florencio~\IEEEmembership{Fellow,~IEEE}
\begin{small}
\thanks{A. Zheng is with 
        Department of Electronic and Computer Engineering, 
        The Hong Kong University of Science and Technology, 
        Clear Water Bay, Hong Kong, China
        (e-mail: amzheng@connect.ust.hk).}
        \thanks{G. Cheung is with 
        National Institute of Informatics, 2-1-2, Hitotsubashi, Chiyoda-ku,
        Tokyo, Japan 101--8430 
        (e-mail: cheung@nii.ac.jp).}
        \thanks{D. Florencio is with  
        Microsoft Research,
        Redmond, WA USA 
        (e-mail: dinei@microsoft.com).}
%Phone Number: +1-778-782-7159 Fax Number: +1-778-782-4951
\end{small}
}%
\maketitle
\vspace{0.1in}

\begin{abstract}
With the advent of depth sensing technologies, the extraction of object contours in images---a common and important pre-processing step for later higher-level computer vision tasks like object detection and human action recognition---has become easier. 
However, acquisition noise in captured depth images means that detected contours suffer from unavoidable errors. 
In this paper, we propose to jointly denoise and compress detected contours in an image for bandwidth-constrained transmission to a client, who can then carry out aforementioned application-specific tasks using the decoded contours as input.
We first prove theoretically that in general a joint denoising / compression approach can outperform a separate two-stage approach that first denoises then encodes contours lossily.
Adopting a joint approach, we first propose a burst error model that models typical errors encountered in an observed string $\vec{y}$ of directional edges.
We then formulate a rate-constrained maximum a posteriori (MAP) problem that trades off the posterior probability $P(\hat{\vec{x}} | \vec{y})$ of an estimated string $\hat{\vec{x}}$ given $\vec{y}$ with its code rate $R(\hat{\vec{x}})$. 
We design a dynamic programming (DP) algorithm that solves the posed problem optimally, and propose a compact context representation called total suffix tree (TST) that can reduce complexity of the algorithm dramatically.
Experimental results show that our joint denoising / compression scheme outperformed a competing separate scheme in rate-distortion performance noticeably.
\end{abstract}

\begin{IEEEkeywords}
contour coding, joint denoising / compression, image compression
\end{IEEEkeywords}

\IEEEpeerreviewmaketitle

%\vspace{-0.05in}
\section{Introduction}
\label{sec:intro}
% !TEX root = ../arxiv.tex

Unsupervised domain adaptation (UDA) is a variant of semi-supervised learning \cite{blum1998combining}, where the available unlabelled data comes from a different distribution than the annotated dataset \cite{Ben-DavidBCP06}.
A case in point is to exploit synthetic data, where annotation is more accessible compared to the costly labelling of real-world images \cite{RichterVRK16,RosSMVL16}.
Along with some success in addressing UDA for semantic segmentation \cite{TsaiHSS0C18,VuJBCP19,0001S20,ZouYKW18}, the developed methods are growing increasingly sophisticated and often combine style transfer networks, adversarial training or network ensembles \cite{KimB20a,LiYV19,TsaiSSC19,Yang_2020_ECCV}.
This increase in model complexity impedes reproducibility, potentially slowing further progress.

In this work, we propose a UDA framework reaching state-of-the-art segmentation accuracy (measured by the Intersection-over-Union, IoU) without incurring substantial training efforts.
Toward this goal, we adopt a simple semi-supervised approach, \emph{self-training} \cite{ChenWB11,lee2013pseudo,ZouYKW18}, used in recent works only in conjunction with adversarial training or network ensembles \cite{ChoiKK19,KimB20a,Mei_2020_ECCV,Wang_2020_ECCV,0001S20,Zheng_2020_IJCV,ZhengY20}.
By contrast, we use self-training \emph{standalone}.
Compared to previous self-training methods \cite{ChenLCCCZAS20,Li_2020_ECCV,subhani2020learning,ZouYKW18,ZouYLKW19}, our approach also sidesteps the inconvenience of multiple training rounds, as they often require expert intervention between consecutive rounds.
We train our model using co-evolving pseudo labels end-to-end without such need.

\begin{figure}[t]%
    \centering
    \def\svgwidth{\linewidth}
    \input{figures/preview/bars.pdf_tex}
    \caption{\textbf{Results preview.} Unlike much recent work that combines multiple training paradigms, such as adversarial training and style transfer, our approach retains the modest single-round training complexity of self-training, yet improves the state of the art for adapting semantic segmentation by a significant margin.}
    \label{fig:preview}
\end{figure}

Our method leverages the ubiquitous \emph{data augmentation} techniques from fully supervised learning \cite{deeplabv3plus2018,ZhaoSQWJ17}: photometric jitter, flipping and multi-scale cropping.
We enforce \emph{consistency} of the semantic maps produced by the model across these image perturbations.
The following assumption formalises the key premise:

\myparagraph{Assumption 1.}
Let $f: \mathcal{I} \rightarrow \mathcal{M}$ represent a pixelwise mapping from images $\mathcal{I}$ to semantic output $\mathcal{M}$.
Denote $\rho_{\bm{\epsilon}}: \mathcal{I} \rightarrow \mathcal{I}$ a photometric image transform and, similarly, $\tau_{\bm{\epsilon}'}: \mathcal{I} \rightarrow \mathcal{I}$ a spatial similarity transformation, where $\bm{\epsilon},\bm{\epsilon}'\sim p(\cdot)$ are control variables following some pre-defined density (\eg, $p \equiv \mathcal{N}(0, 1)$).
Then, for any image $I \in \mathcal{I}$, $f$ is \emph{invariant} under $\rho_{\bm{\epsilon}}$ and \emph{equivariant} under $\tau_{\bm{\epsilon}'}$, \ie~$f(\rho_{\bm{\epsilon}}(I)) = f(I)$ and $f(\tau_{\bm{\epsilon}'}(I)) = \tau_{\bm{\epsilon}'}(f(I))$.

\smallskip
\noindent Next, we introduce a training framework using a \emph{momentum network} -- a slowly advancing copy of the original model.
The momentum network provides stable, yet recent targets for model updates, as opposed to the fixed supervision in model distillation \cite{Chen0G18,Zheng_2020_IJCV,ZhengY20}.
We also re-visit the problem of long-tail recognition in the context of generating pseudo labels for self-supervision.
In particular, we maintain an \emph{exponentially moving class prior} used to discount the confidence thresholds for those classes with few samples and increase their relative contribution to the training loss.
Our framework is simple to train, adds moderate computational overhead compared to a fully supervised setup, yet sets a new state of the art on established benchmarks (\cf \cref{fig:preview}).


%\vspace{-0.1in}
\section{Related Work}
\label{sec:related}
\section{Related Work}\label{sec:related}
 
The authors in \cite{humphreys2007noncontact} showed that it is possible to extract the PPG signal from the video using a complementary metal-oxide semiconductor camera by illuminating a region of tissue using through external light-emitting diodes at dual-wavelength (760nm and 880nm).  Further, the authors of  \cite{verkruysse2008remote} demonstrated that the PPG signal can be estimated by just using ambient light as a source of illumination along with a simple digital camera.  Further in \cite{poh2011advancements}, the PPG waveform was estimated from the videos recorded using a low-cost webcam. The red, green, and blue channels of the images were decomposed into independent sources using independent component analysis. One of the independent sources was selected to estimate PPG and further calculate HR, and HRV. All these works showed the possibility of extracting PPG signals from the videos and proved the similarity of this signal with the one obtained using a contact device. Further, the authors in \cite{10.1109/CVPR.2013.440} showed that heart rate can be extracted from features from the head as well by capturing the subtle head movements that happen due to blood flow.

%
The authors of \cite{kumar2015distanceppg} proposed a methodology that overcomes a challenge in extracting PPG for people with darker skin tones. The challenge due to slight movement and low lighting conditions during recording a video was also addressed. They implemented the method where PPG signal is extracted from different regions of the face and signal from each region is combined using their weighted average making weights different for different people depending on their skin color. 
%

There are other attempts where authors of \cite{6523142,6909939, 7410772, 7412627} have introduced different methodologies to make algorithms for estimating pulse rate robust to illumination variation and motion of the subjects. The paper \cite{6523142} introduces a chrominance-based method to reduce the effect of motion in estimating pulse rate. The authors of \cite{6909939} used a technique in which face tracking and normalized least square adaptive filtering is used to counter the effects of variations due to illumination and subject movement. 
The paper \cite{7410772} resolves the issue of subject movement by choosing the rectangular ROI's on the face relative to the facial landmarks and facial landmarks are tracked in the video using pose-free facial landmark fitting tracker discussed in \cite{yu2016face} followed by the removal of noise due to illumination to extract noise-free PPG signal for estimating pulse rate. 

Recently, the use of machine learning in the prediction of health parameters have gained attention. The paper \cite{osman2015supervised} used a supervised learning methodology to predict the pulse rate from the videos taken from any off-the-shelf camera. Their model showed the possibility of using machine learning methods to estimate the pulse rate. However, our method outperforms their results when the root mean squared error of the predicted pulse rate is compared. The authors in \cite{hsu2017deep} proposed a deep learning methodology to predict the pulse rate from the facial videos. The researchers trained a convolutional neural network (CNN) on the images generated using Short-Time Fourier Transform (STFT) applied on the R, G, \& B channels from the facial region of interests.
The authors of \cite{osman2015supervised, hsu2017deep} only predicted pulse rate, and we extended our work in predicting variance in the pulse rate measurements as well.

All the related work discussed above utilizes filtering and digital signal processing to extract PPG signals from the video which is further used to estimate the PR and PRV.  %
The method proposed in \cite{kumar2015distanceppg} is person dependent since the weights will be different for people with different skin tone. In contrast, we propose a deep learning model to predict the PR which is independent of the person who is being trained. Thus, the model would work even if there is no prior training model built for that individual and hence, making our model robust. 

%

%\vspace{-0.05in}
\section{Problem Formulation}
\label{sec:problem}
We briefly recall the framework of statistical inference via empirical risk minimization.
Let $(\bbZ, \calZ)$ be a measurable space.
Let $Z \in \bbZ$ be a random element following some unknown distribution $\Prob$.
Consider a parametric family of distributions $\calP_\Theta := \{P_\theta: \theta \in \Theta \subset \reals^d\}$ which may or may not contain $\Prob$.
We are interested in finding the parameter $\theta_\star$ so that the model $P_{\theta_\star}$ best approximates the underlying distribution $\Prob$.
For this purpose, we choose a \emph{loss function} $\score$ and minimize the \emph{population risk} $\risk(\theta) := \Expect_{Z \sim \Prob}[\score(\theta; Z)]$.
Throughout this paper, we assume that
\begin{align*}
     \theta_\star = \argmin_{\theta \in \Theta} L(\theta)
\end{align*}
uniquely exists and satisfies $\theta_\star \in \text{int}(\Theta)$, $\nabla_\theta L(\theta_\star) = 0$, and $\nabla_\theta^2 L(\theta_\star) \succ 0$.

\myparagraph{Consistent loss function}
We focus on loss functions that are consistent in the following sense.

\begin{customasmp}{0}\label{asmp:proper_loss}
    When the model is \emph{well-specified}, i.e., there exists $\theta_0 \in \Theta$ such that $\Prob = P_{\theta_0}$, it holds that $\theta_0 = \theta_\star$.
    We say such a loss function is \emph{consistent}.
\end{customasmp}

In the statistics literature, such loss functions are known as proper scoring rules \citep{dawid2016scoring}.
We give below two popular choices of consistent loss functions.

\begin{example}[Maximum likelihood estimation]
    A widely used loss function in statistical machine learning is the negative log-likelihood $\score(\theta; z) := -\log{p_\theta(z)}$ where $p_\theta$ is the probability mass/density function for the discrete/continuous case.
    When $\Prob = P_{\theta_0}$ for some $\theta_0 \in \Theta$,
    we have $L(\theta) = \Expect[-\log{p_\theta(Z)}] = \kl(p_{\theta_0} \Vert p_\theta) - \Expect[\log{p_{\theta_0}(Z)}]$ where $\kl$ is the Kullback-Leibler divergence.
    As a result, $\theta_0 \in \argmin_{\theta \in \Theta} \kl(p_{\theta_0} \Vert p_\theta) = \argmin_{\theta \in \Theta} L(\theta)$.
    Moreover, if there is no $\theta$ such that $p_\theta \txtover{a.s.}{=} p_{\theta_0}$, then $\theta_0$ is the unique minimizer of $L$.
    We give in \Cref{tab:glms} a few examples from the class of generalized linear models (GLMs) proposed by \citet{nelder1972generalized}.
\end{example}

\begin{example}[Score matching estimation]
    Another important example appears in \emph{score matching} \citep{hyvarinen2005estimation}.
    Let $\bbZ = \reals^\tau$.
    Assume that $\Prob$ and $P_\theta$ have densities $p$ and $p_\theta$ w.r.t the Lebesgue measure, respectively.
    Let $p_\theta(z) = q_\theta(z) / \Lambda(\theta)$ where $\Lambda(\theta)$ is an unknown normalizing constant. We can choose the loss
    \begin{align*}
        \score(\theta; z) := \Delta_z \log{q_\theta(z)} + \frac12 \norm{\nabla_z \log{q_\theta(z)}}^2 + \text{const}.
    \end{align*}
    Here $\Delta_z := \sum_{k=1}^p \partial^2/\partial z_k^2$ is the Laplace operator.
    Since \cite[Thm.~1]{hyvarinen2005estimation}
    \begin{align*}
        L(\theta) = \frac12 \Expect\left[ \norm{\nabla_z q_\theta(z) - \nabla_z p(z)}^2 \right],
    \end{align*}
    we have, when $p = p_{\theta_0}$, that $\theta_0 \in \argmin_{\theta \in \Theta} L(\theta)$.
    In fact, when $q_\theta > 0$ and there is no $\theta$ such that $p_\theta \txtover{a.s.}{=} p_{\theta_0}$, the true parameter $\theta_0$ is the unique minimizer of $L$ \cite[Thm.~2]{hyvarinen2005estimation}.
\end{example}

\myparagraph{Empirical risk minimization}
Assume now that we have an i.i.d.~sample $\{Z_i\}_{i=1}^n$ from $\Prob$.
To learn the parameter $\theta_\star$ from the data, we minimize the empirical risk to obtain the \emph{empirical risk minimizer}
\begin{align*}
    \theta_n \in \argmin_{\theta \in \Theta} \left[ L_n(\theta) := \frac1n \sum_{i=1}^n \score(\theta; Z_i) \right].
\end{align*}
This applies to both maximum likelihood estimation and score matching estimation. 
In \Cref{sec:main_results}, we will prove that, with high probability, the estimator $\theta_n$ exists and is unique under a generalized self-concordance assumption.

\begin{figure}
    \centering
    \includegraphics[width=0.45\textwidth]{graphs/logistic-dikin} %0.4
    \caption{Dikin ellipsoid and Euclidean ball.}
    \label{fig:logistic_dikin}
\end{figure}

\myparagraph{Confidence set}
In statistical inference, it is of great interest to quantify the uncertainty in the estimator $\theta_n$.
In classical asymptotic theory, this is achieved by constructing an asymptotic confidence set.
We review here two commonly used ones, assuming the model is well-specified.
We start with the \emph{Wald confidence set}.
It holds that $n(\theta_n - \theta_\star)^\top H_n(\theta_n) (\theta_n - \theta_\star) \rightarrow_d \chi_d^2$, where $H_n(\theta) := \nabla^2 L_n(\theta)$.
Hence, one may consider a confidence set $\{\theta: n(\theta_n - \theta)^\top H_n(\theta_n) (\theta_n - \theta) \le q_{\chi_d^2}(\delta) \}$ where $q_{\chi_d^2}(\delta)$ is the upper $\delta$-quantile of $\chi_d^2$.
The other is the \emph{likelihood-ratio (LR) confidence set} constructed from the limit $2n [L_n(\theta_\star) - L_n(\theta_n)] \rightarrow_d \chi_d^2$, which is known as the Wilks' theorem \citep{wilks1938large}.
These confidence sets enjoy two merits: 1) their shapes are an ellipsoid (known as the \emph{Dikin ellipsoid}) which is adapted to the optimization landscape induced by the population risk; 2) they are asymptotically valid, i.e., their coverages are exactly $1 - \delta$ as $n \rightarrow \infty$.
However, due to their asymptotic nature, it is unclear how large $n$ should be in order for it to be valid.

Non-asymptotic theory usually focuses on developing finite-sample bounds for the \emph{excess risk}, i.e., $\Prob(L(\theta_n) - L(\theta_\star) \le C_n(\delta)) \ge 1 - \delta$.
To obtain a confidence set, one may assume that the population risk is twice continuously differentiable and $\lambda$-strongly convex.
Consequently, we have $\lambda \norm{\theta_n - \theta_\star}_2^2 / 2 \le L(\theta_n) - L(\theta_\star)$ and thus we can consider the confidence set $\calC_{\text{finite}, n}(\delta) := \{\theta: \norm{\theta_n - \theta}_2^2 \le 2C_n(\delta)/\lambda\}$.
Since it originates from a finite-sample bound, it is valid for fixed $n$, i.e., $\Prob(\theta_\star \in \calC_{\text{finite}, n}(\delta)) \ge 1 - \delta$ for all $n$; however, it is usually conservative, meaning that the coverage is strictly larger than $1 - \delta$.
Another drawback is that its shape is a Euclidean ball which remains the same no matter which loss function is chosen.
We illustrate this phenomenon in \Cref{fig:logistic_dikin}.
Note that a similar observation has also been made in the bandit literature \citep{faury2020improved}.

We are interested in developing finite-sample confidence sets.
However, instead of using excess risk bounds and strong convexity, we construct in \Cref{sec:main_results} the Wald and LR confidence sets in a non-asymptotic fashion, under a generalized self-concordance condition.
These confidence sets have the same shape as their asymptotic counterparts while maintaining validity for fixed $n$.
These new results are achieved by characterizing the critical sample size enough to enter the asymptotic regime.


%\vspace{-0.05in}
\section{Error Term}
\label{sec:error}
\begin{figure}[tb]

\begin{minipage}[b]{.48\linewidth}
  \centering
  \centerline{\includegraphics[width=4cm]{burst_example-eps-converted-to.pdf}}
%  \vspace{1.5cm}
  \centerline{(a)}\medskip
\end{minipage}
\hfill
\begin{minipage}[b]{0.48\linewidth}
  \centering
  \centerline{\includegraphics[width=3cm]{markov_process-eps-converted-to.pdf}}
%  \vspace{1.5cm}
  \centerline{(b)}\medskip
\end{minipage}

\vspace{-0.2cm}
\caption{{(a) The two red squares are the erred pixels. Observed $\vec{y}$ is composed of black solid edges (good states) and green solid edges (bad states). The ground truth $\vec{x}$ is composed of black solid edges and black dotted edges. (b) A three-state Markov model.}}
\label{fig:burst_example}
\end{figure}


We first rewrite the posterior $P(\vec{x} | \vec{y})$ in (\ref{eq:lagrangian_objective}) using Bayes' Rule: 
\begin{equation}
P(\vec{x}|\vec{y}) = \frac{P(\vec{y}|\vec{x}) P(\vec{x})}{P(\vec{y})}
\end{equation}
where $P(\vec{y}|\vec{x})$ is the likelihood of observing DCC string $\vec{y}$ given ground truth $\vec{x}$, and $P(\vec{x})$ is the prior which describes \textit{a priori} knowledge about the target DCC string.
We next describe an error model for DCC strings, then define likelihood $P(\vec{y} | \vec{x})$ and prior $P(\vec{x})$ in turn.


%\begin{figure*}[!t]
%
%
%% The spacer can be tweaked to stop underfull vboxes.
%
%% ensure that we have normalsize text
%\normalsize
%% Store the current equation number.
%\setcounter{mytempeqncnt}{\value{equation}}
%% Set the equation number to one less than the one
%% desired for the first equation here.
%% The value here will have to changed if equations
%% are added or removed prior to the place these
%% equations are referenced in the main text.
%\setcounter{equation}{8}
%
%\begin{equation}
%\label{eq:straight}
%s(\vec{w})=\underset{1 \leq k \leq L_{\vec{w}}}{\max}\left\{\frac{|(m_k-m_0)(n_{L_{\vec{w}}}-n_0)-(n_k - n_0)(m_{L_{\vec{w}}}-m_0) |}{\sqrt{(m_{L_{\vec{w}}}-m_0)^2+(n_{L_{\vec{w}}}-n_0)^2}}  \right\}
%\end{equation}
%
%% IEEE uses as a separator
%\hrulefill
%
%% Restore the current equation number.
%\setcounter{equation}{\value{mytempeqncnt}}
%
%\vspace*{4pt}
%
%\end{figure*}


\subsection{Error Model for DCC String}
\label{subsec:error_model}

Assuming that pixels in an image are corrupted by a small amount of independent and identically distributed (iid) noise, a detected contour will occasionally be shifted from the true contour by one or two pixels. 
However, the computed DCC string from the detected contour will experience a sequence of wrong symbols---a \textit{burst error}. 
This is illustrated in Fig.\;\ref{fig:burst_example}(a), where the left single erred pixel (in red) resulted in two erred symbols in the DCC string. The right single error pixel also resulted in a burst error in the observed string, which is \textit{longer} than the original string. Based on these observations, we propose our DCC string error model as follows.

We define a \textit{three-state Markov model} as illustrated in Fig.\;\ref{fig:burst_example}(b) to model the probability of observing DCC string $\vec{y}$ given original string $\vec{x}$. 
State \texttt{0} is the good state, and \textit{burst error state} \texttt{1} and \textit{burst length state} \texttt{2} are the bad states. 
$p$, $q_1$ and $q_2$ are the transition probabilities from state \texttt{0} to \texttt{1}, \texttt{1} to \texttt{2}, and \texttt{2} to \texttt{0}, respectively.
%, which can be estimated from training data.
Note that state \texttt{1} cannot transition directly to \texttt{0}, and likewise state \texttt{2} to \texttt{1} and \texttt{0} to \texttt{2}.

Starting at good state \texttt{0}, each journey to state \texttt{1} then to \texttt{2} then back to \texttt{0} is called a \textit{burst error event}.
From state \texttt{0}, each self-loop back to \texttt{0} with probability $1-p$ means that the next observed symbol $y_i$ is the same as $x_i$ in original $\vec{x}$. 
A transition to burst error state \texttt{1} with probability $p$, and each subsequent self-loop with probability $1-q_1$, mean observed $y_i$ is now different from $x_i$.
A transition to burst length state \texttt{2} then models the \textit{length increase} in observed $\vec{y}$ over original $\vec{x}$ due to this burst error event: the number of self-loops taken back to state \texttt{2} is the increase in number of symbols. 
A return to good state \texttt{0} signals the end of this burst error event. 


\subsection{Likelihood Term}
\label{subsec:error_likelihod}

Given the three-state Markov model, we can compute likelihood $P(\vec{y} | \vec{x})$ as follows. 
For simplicity, we assume that $\vec{y}$ starts and ends at good state \texttt{0}. 
Denote by $K$ the total number of burst error events in $\vec{y}$ given $\vec{x}$. 
Further, denote by $l_1(k)$ and $l_2(k)$ the number of visits to state \texttt{1} and \texttt{2} respectively during the $k$-th burst error event.
Similarly, denote by $l_0(k)$ the number of visits to state \texttt{0} \textit{after} the $k$-th burst error event. 
We can then write the likelihood $P(\vec{y}|\vec{x})$ as:

\vspace{-0.1in}
\begin{small}
\begin{equation}
\begin{split}
&(1-p)^{l_0(0)} \prod^{K}_{k=1} 
p(1-q_1)^{l_1(k)-1} q_1(1-q_2)^{l_2(k)-1}q_2 (1-p)^{l_0(k)-1} 
\end{split}
\label{eq:likelihood}
\end{equation}
\end{small}
%\red{missing the last series of good state at the end?} \blue{There are $K+1$ series of good state in total. $(1-p)^{l_0(1)}$ denotes the first series and $(1-p)^{l_0(k+1)-1}$ denotes the following $K$ series.}

\vspace{-0.15in}
For convenience, we define the total number of visits to state \texttt{0}, \texttt{1} and \texttt{2} as $\Gamma=\sum_{k=0}^{K} l_0(k)$, $\Lambda=\sum_{k=1}^K l_1(k)$ and $\Delta=\sum_{k=1}^K l_2(k)$, respectively.
We can then write the negative log of the likelihood as:
\begin{equation}
\label{eq:negaLogLikelihood}
\begin{split}
&-\log{P(\vec{y}|\vec{x})} =\\
& - K(\log{p}+\log{q_1}+\log{q_2}) -(\Gamma-K)\log{(1-p)} \\
&  - (\Lambda-K)\log{(1-q_1)} - (\Delta-K)\log{(1-q_2)}
\end{split}
\end{equation}

Assuming that burst errors are rare events, $p$ is small and $\log(1-p) \approx 0$. Hence:
\begin{equation}
\label{eq:negaLogLikelihoodAppro}
\begin{split}
& -\log{P(\vec{y}|\vec{x})} \\
& \approx  - K(\log{p}+\log{q_1}+\log{q_2}) \\ 
& - (\Lambda-K)\log{(1-q_1)} - (\Delta-K)\log{(1-q_2)} \\
& = - K(\underbrace{\log{p}+\log{\frac{q_1}{1-q_1}}+\log{\frac{q_2}{1-q_2}}}_{-c_0}) \\
& - \Lambda \underbrace{\log(1-q_1)}_{-c_1} -
\Delta \underbrace{\log(1-q_2)}_{-c_2}
\end{split}
\end{equation}
%Since $p$ is small, $-(\log{p}+\log{\frac{q_1}{1-q_1}}+\log{\frac{q_2}{1-q_2}})$ is a positive number. 
Thus $-\log P(\vec{y}|\vec{x})$ simplifies to:
\begin{equation}
\label{eq:negaLogLikelihoodApproSimple}
\begin{split}
& -\log{P(\vec{y}|\vec{x})}  \approx  (c_0 + c_2) K + c_1 \Lambda 
+ c_2 \Delta'
\end{split}
\end{equation}
%\red{the number of visits to state \texttt{2} is not exactly the same as length increase.} \blue{For each burst error, the number of visits to state \texttt{2} equals to 1 + the length increase. We always have state \texttt{2} even if the length doesn't increase.}
where $\Delta' = \Delta - K$ is the length increase in observed $\vec{y}$ compared to $\vec{x}$ due to the $K$ burst error events\footnote{Length increase of observed $\vec{y}$ due to $k$-th burst error event is $l_2(k) - 1$.}.
(\ref{eq:negaLogLikelihoodApproSimple}) states that the negative log of the likelihood is a linear sum of three terms: i) the number of burst error events $K$; ii) the number of error corrupted symbols $\Lambda$; and iii) the length increase $\Delta'$ in observed string $\vec{y}$.
This agrees with our intuition that more error events, more errors and more deviation in DCC length will result in a larger objective value in (\ref{eq:lagrangian_objective}).
We will validate this error model empirically in our experiments in Section \ref{sec:results}. 


\subsection{Prior Term}
\label{subsec:error_prior}

Similarly to \cite{daribo14,zheng17}, we propose a \textit{geometric shape prior} based on the assumption that contours in natural images tend to be more straight than curvy.
%The geometric prior models the underlying signal which is independent of the contour data.
Specifically, we write prior $P(\vec{x})$ as:
\begin{equation}
P(\vec{x}) = \exp\left\{- \beta \sum\limits_{i=D_s+1}^{L_{\vec{x}}} s(\vec{x}_{i-D_s}^{i})\right\}
\label{eq:geometric_prior}
\end{equation}
where $\beta$ and $D_s$ are parameters. 
$s(\vec{x}_{i-D_s}^i)$ measures the \textit{straightness} of DCC sub-string $\vec{x}_{i-D_s}^i$.
Let $\vec{w}$ be a DCC string of length $D_s+1$, \textit{i.e.}, $L_{\vec{w}}=D_s+1$. 
Then $s(\vec{w})$ is defined as the \textit{maximum Euclidean distance} between any coordinates of edge $\vec{e}_{\vec{w}}(k), 0\leq k \leq L_{\vec{w}}$ and the line connecting the first point $(m_0,n_0)$ and the last point $(m_{L_{\vec{w}}}, n_{L_{\vec{w}}})$ of $\vec{w}$ on the 2D grid.
We can write $s(\vec{w})$ as:

\vspace{-0.1in}
\begin{small}
\begin{equation}
\label{eq:straight}
\begin{split}
& \underset{0\! \leq k \leq L_{\vec{w}}}{\max}\left\{\!\frac{|(m_k\!-\!m_0)(n_{L_{\vec{w}}}\!-\!n_0)\!-\!(n_k\! -\! n_0)(m_{L_{\vec{w}}}\!-\!m_0) |}{\sqrt{(m_{L_{\vec{w}}}-m_0)^2+(n_{L_{\vec{w}}}-n_0)^2}} \! \right\}
\end{split}
\end{equation}
\end{small}

Since we compute the sum of a straightness measure for overlapped sub-strings, the length of each sub-string $\vec{x}_{i-D_s}^i$ should not be too large to capture the local contour behavior.
Thus, we choose $D_s$ to be a fixed small number in our implementation. 
Some examples of $s(\vec{w})$ are shown in Fig.\;\ref{fig:prior}. 


%% The previous equation was number six.
%% Account for the double column equations here.
%\addtocounter{equation}{1}

\begin{figure}[h]

\begin{minipage}[b]{.32\linewidth}
  \centering
  \centerline{\includegraphics[width=1.5cm]{prior_a-eps-converted-to.pdf}}
%  \vspace{1.5cm}
  \centerline{(a)}\medskip
\end{minipage}
\hfill
\begin{minipage}[b]{0.32\linewidth}
  \centering
  \centerline{\includegraphics[width=1.3cm]{prior_b-eps-converted-to.pdf}}
%  \vspace{1.5cm}
  \centerline{(b)}\medskip
\end{minipage}
\hfill
\begin{minipage}[b]{0.32\linewidth}
  \centering
  \centerline{\includegraphics[width=2.8cm]{prior_c-eps-converted-to.pdf}}
%  \vspace{1.5cm}
  \centerline{(c)}\medskip
\end{minipage}

\vspace{-0.3cm}
\caption{Three examples of the straightness of $s(\vec{w})$ with $L_{\vec{w}}=4$. (a) $\vec{w}=rrls$ and $s(\vec{w})=4\sqrt{5}/5$. (b) $\vec{w}=lrlr$ and $s(\vec{w})=6\sqrt{13}/13$. (c) $\vec{w}=ssss$ and $s(\vec{w})=0$.}
\label{fig:prior}
\end{figure}

Combining the likelihood and prior terms, we get the negative log of the posterior,
\begin{equation}
\label{eq:posterior}
\begin{split}
 -\log P(\hat{\vec{x}}|\vec{y})= & (c_0 + c_2) K + c_1 \Lambda + c_2 \Delta' \\
& +\beta \sum\limits_{i=D_s+1}^{L_{\hat{\vec{x}}}} s(\hat{\vec{x}}_{i-D_s}^i)
\end{split}
\end{equation}


%\vspace{-0.05in}
\section{Rate Term}
\label{sec:rate}
We losslessly encode a chosen DCC string $\hat{\vec{x}}$ using arithmetic coding \cite{DCC1st1991}. 
Specifically, to implement arithmetic coding using local statistics, each DCC symbol $\hat{x}_i \in \mathcal{A}$ is assigned a conditional probability $P(\hat{x}_i|\hat{\vec{x}}_{1}^{i-1})$ given its all previous symbols $\hat{\vec{x}}_{1}^{i-1}$.
%The conditional probabilities are input to an arithmetic coder for entropy coding.
The rate term $R(\hat{\vec{x}})$ is thus approximated as the summation of negative log of conditional probabilities of all symbols in $\hat{\vec{x}}$:
\begin{equation}
\label{eq:rate}
R(\hat{\vec{x}})= - \sum\limits_{i=1}^N \log_2 P(\hat{x}_i|\hat{\vec{x}}_{1}^{i-1}) 
\end{equation}

To compute $R(\hat{\vec{x}})$, one must assign conditional probabilities $P(\hat{x}_i|\hat{\vec{x}}_{1}^{i-1})$ for all symbols $\hat{x}_i$.
In our implementation, we use the variable-length context tree model \cite{begleiter2004prediction} to compute the probabilities.
Specifically, to code contours in the target image, one context tree is trained using contours in a set of training images\footnote{For coding contours in the target frame of a video, the training images are the earlier coded frames.} which have correlated statistics with the target image.
Next we introduce the context tree model, then discuss our construction of a context tree using \textit{prediction by partial matching} (PPM)\footnote{While the computation of conditional probabilities are from PPM, our construction of a context tree is novel and stands as one key contribution in this paper.} \cite{moffat1990implementing}.

\begin{figure}[t]

\begin{minipage}[b]{1\linewidth}
  \centering
  \centerline{\includegraphics[width=8 cm]{MCT_statespace_simple-eps-converted-to.pdf}}
  %\vspace{0.1cm}
  \centerline{}
  %\medskip
\end{minipage}

\vspace{-0.2cm}
\caption{An example of context tree.
Each node is a sub-string and the root node is an empty sub-string.
The contexts are all the end nodes on $\mathcal{T}$: $\mathcal{T}=\{\texttt{l},\texttt{sl},\texttt{sls},\texttt{slr},\texttt{ss},\texttt{sr},\texttt{rl},\texttt{r},\texttt{rr}\}$.}
\label{fig:MCT_statespace}
\end{figure}


\subsection{Definition of Context Tree}
\label{subsec:contextTree}

We first define notations related to the contours in the set of training images.
Denote by $\vec{x}(m)$, $1 \leq m \leq M$, the $m$-th DCC string in the training set $\mathcal{X} = \{\vec{x}(1),\ldots,\vec{x}(M)\}$, where $M$ denotes the total number of DCC strings in $\mathcal{X}$. The total number of symbols in $\mathcal{X}$ is denoted by $L = \sum^{M}_{m=1}L_{\vec{x}(m)}$. 
Denote by $\vec{u}\vec{v}$ the concatenation of sub-strings $\vec{u}$ and $\vec{v}$.

We now define $N(\vec{u})$ as the number of occurrences of sub-string $\vec{u}$ in the training set $\mathcal{X}$. $N(\vec{u})$ can be computed as:
\begin{equation}
N(\vec{u})=\sum^{M}_{m=1}\sum_{i=1}^{L_{\vec{x}(m)} - |\vec{u}| + 1}
%1_{[\vec{x}(j)^{i+|\vec{u}|-1}_{i}=\vec{u}]} 
\vec{1} \left(
\vec{x}(m)_i^{i+|\vec{u}|-1} = \vec{u}
\right) 
\end{equation}
%where $\vec{u} \in \stackrel [k=1] {+\infty}{\cup} \mathcal{D}^{k}$. 
where $\vec{1}(\vec{c})$ is an indicator function that evaluates to $1$ if the specified binary clause $\vec{c}$ is true and $0$ otherwise.

Denote by $P(x|\vec{u})$ the conditional probability of symbol $x$ occurring given its previous sub-string is $\vec{u}$, where $x\in\mathcal{A}$. Given training data $\mathcal{X}$, $P(x|\vec{u})$ can be estimated using $N(\vec{u})$ as done in \cite{buhlmann1999variable},
\begin{equation}
\begin{array}{cc}
P(x|\vec{u})=\frac{N(x\vec{u})}{\delta +  N(\vec{u})}
\end{array}
\label{eq:cal_prob}
\end{equation}
where $\delta$ is a chosen parameter for different models.
%\red{u mean a parameter? what's the purpose of this parameter?}
%\blue{this parameter is used in PPM.}

Given $\mathcal{X}$, we learn a context model to assign a conditional probability to any symbol given its previous symbols in a DCC string. Specifically, to calculate the conditional probability $P(\hat{x}_i|\hat{\vec{x}}^{i-1}_{1})$, the model determines a \textit{context} $\vec{w}$ to calculate $P(\hat{x}_i|\vec{w})$, where $\vec{w}$ is a \textit{prefix} of the sub-string $\hat{\vec{x}}^{i-1}_{1}$, \textit{i.e.}, $\vec{w} = \hat{\vec{x}}^{i-1}_{i-l}$ for some context length $l$:
\begin{equation}
P(\hat{x}_i|\hat{\vec{x}}^{i-1}_{1}, \text{context model})=P(\hat{x}_i|\vec{w})
\label{eq:conditionalProbContextModel}
\end{equation}
$P(\hat{x}_i|\vec{w})$ is calculated using (\ref{eq:cal_prob}) given $\mathcal{X}$. 
The context model determines a unique context $\vec{w}$ of finite length for every possible past $\hat{\vec{x}}^{i-1}_{1}$. The set of all mappings from $\hat{\vec{x}}^{i-1}_{1}$ to $\vec{w}$ can be represented compactly as a context tree.

Denote by $\mathcal{T}$ the context tree, where $\mathcal{T}$ is a \textit{ternary tree}: each node has at most three children. 
The root node has an empty sub-string, and each child node has a sub-string $\vec{u}x$ that is a concatenation of: i) its parent's sub-string $\vec{u}$ if any, and ii) the symbol $x$ (one of \texttt{l}, \texttt{s} and \texttt{r}) representing the link connecting the parent node and itself in $\mathcal{T}$. 
An example is shown in Fig.\;\ref{fig:MCT_statespace}.
The contexts of the tree $\mathcal{T}$ are the sub-strings of the \textit{context nodes}---nodes that have at most two children, \textit{i.e.}, the end nodes and the intermediate nodes with fewer than three children. 
Note that $\mathcal{T}$ is completely specified by its set of context nodes and vice versa.
For each $\hat{\vec{x}}^{i-1}_{1}$, a context $\vec{w}$ is obtained by traversing $\mathcal{T}$ from the root node to the deepest context node, matching symbols $\hat{x}_{i-1}, \hat{x}_{i-2}, \ldots$ into the past.
We can then rewrite (\ref{eq:conditionalProbContextModel}) as follows:
\begin{equation}
P(\hat{x}_i|\hat{\vec{x}}^{i-1}_{1}, \mathcal{T})=P(\hat{x}_i|\vec{w}).
\label{eq:conditionalProbContextTree}
\end{equation}


\subsection{Construction of Context Tree by Prediction by Partial Matching (PPM)}
\label{subsec:PPM}

The PPM algorithm is considered to be one of the best lossless compression algorithms \cite{begleiter2004prediction}, which is based on the context tree model.
Using PPM, all the possible sub-strings $\vec{w}$ with non-zero occurrences in $\mathcal{X}$, \textit{i.e.}, $N(\vec{w}) > 0$, are contexts on the context tree. 
The key idea of of PPM is to deal with the \textit{zero frequency} problem when estimate $P(\hat{x}_i|\vec{w})$, where sub-string $\hat{x}_i\vec{w}$ does not occur in $\mathcal{X}$, \textit{i.e.}, $N(\hat{x}_i\vec{w})=0$.
In such case, using (\ref{eq:cal_prob}) to estimate $P(\hat{x}_i|\vec{w})$ would result in zero probability, which cannot be used for arithmetic coding.
When $N(\hat{x}_i\vec{w})=0$, $P(\hat{x}_i|\vec{w})$ is estimated instead by reducing the context length by one, \textit{i.e.}, $P(\hat{x}_i|\vec{w}_2^{|\vec{w}|})$.
If sub-string $\hat{x}_i \vec{w}_2^{|\vec{w}|}$ still does not occur in $\mathcal{X}$, the context length is further reduced until symbol $\hat{x}_i$ along with the shortened context occurs in $\mathcal{X}$. 
Let $\mathcal{A}_{\vec{w}}$ be an alphabet in which each symbol along with the context $\vec{w}$ occurs in $\mathcal{X}$, \textit{i.e.}, $\mathcal{A}_{\vec{w}}=\{x|N(x\vec{w}) > 0, x \in \mathcal{A}\}$.
Based on the PPM implemented in \cite{moffat1990implementing}, $P(\hat{x}_i|\vec{w})$ is computed using the following (recursive) equation:
\begin{equation}
P(\hat{x}_i|\vec{w}) = 
\begin{cases}
\frac{N(\hat{x}_i \vec{w})}{|\mathcal{A}_{\vec{w}}| + N(\vec{w})},&\text{if } \hat{x}_i \in \mathcal{A}_{\vec{w}} \\
\frac{|\mathcal{A}_{\vec{w}}|}{|\mathcal{A}_\vec{w}| + N(\vec{w})} \cdot P(\hat{x}_i|\vec{w}_2^{|\vec{w}|}),&\text{otherwise}
\end{cases}.
\label{eq:ppm_prob}
\end{equation}
%where $\frac{|\mathcal{A}_{\vec{w}}|}{|\mathcal{A}_\vec{w}| + N(\vec{w})}$ is a factor. 
%\red{what do u mean by factor?}

To construct $\mathcal{T}$, we traverse the training data $\mathcal{X}$ once to collect statistics for all potential contexts.
Each node in $\mathcal{T}$, \textit{i.e.}, sub-string $\mathbf{u}$, has three counters which store the number of occurrences of sub-strings $l\mathbf{u}$, $s\mathbf{u}$ and $r\mathbf{u}$, \textit{i.e.}, $N(l\mathbf{u})$, $N(s\mathbf{u})$ and $N(r\mathbf{u})$.
To reduce memory requirement, we set an upper bound $D$ on the maximum depth of $\mathcal{T}$.
As done in \cite{rissanen1983universal}, we choose the maximum depth of $\mathcal{T}$ as $D=\ceil{\ln{L}/\ln{3}}$, which ensures a large enough $D$ to capture natural statistics of the training data of length $L$.
$\mathcal{T}$ is constructed as described in Algorithm \ref{al:contextTree}.


\begin{algorithm}
\caption{Construction of the Context Tree}
\label{al:contextTree}
\begin{algorithmic}[1]

\State{Initialize $\mathcal{T}$ to an empty tree with only root node}

\For{each symbol $x(m)_i,\vec{x}(m)\in\mathcal{X}, \;i\geq D+1$, from $k=1$ to $k=D$ in order}

\If{there exist a node $\mathbf{u}=\vec{x}(m)_{i-k}^{i-1}$ on $\mathcal{T}$}
	\State{increase the counter $N(x(m)_i\mathbf{u})$ by $1$}
\Else
	\State{add node $\mathbf{u}=\vec{x}(m)_{i-k}^{i-1}$ to $\mathcal{T}$}
\EndIf

\EndFor


\end{algorithmic}
\end{algorithm}


The complexity of the algorithm is $O(D \, L)$.
To estimate the code rate of symbol $\hat{x}_i$, we first find the matched context $\vec{w}$ given past $\hat{\vec{x}}_{i-D}^{i-1}$ by traversing the context tree $\mathcal{T}$ from the root node to the deepest node, \textit{i.e.}, $\vec{w}=\hat{\vec{x}}^{i-1}_{i-|\vec{w}|}$, and then compute the corresponding conditional probability $P(\hat{x}_i|\vec{w})$ using (\ref{eq:ppm_prob}).
%The total rate of coding DCC string $\hat{\vec{x}}$ is computed as the summation of the negative log of conditional probability of all symbols in $\hat{\vec{x}}$.

In summary, having defined the likelihood, prior and rate terms, our Lagrangian objective (\ref{eq:lagrangian_objective}) can now be rewritten as:
\begin{equation}
\label{eq:objective}
\begin{split}
J(\hat{\vec{x}})=& -\log P(\vec{y}|\hat{\vec{x}})-\beta \log P(\hat{\vec{x}}) + \lambda R(\hat{\vec{x}})\\
\approx &(c_0 + c_2) K + c_1 \Lambda + c_2 \Delta' +\beta \sum\limits_{i=D_s+1}^{L_{\hat{\vec{x}}}} s(\hat{\vec{x}}_{i-D_s}^i)\\
 &- \lambda \sum\limits_{i=D+1}^{L_{\hat{\vec{x}}}} \log_2 P(\hat{x}_i|\hat{\vec{x}}_{i-D}^{i-1}). 
\end{split}
\end{equation}
We describe a dynamic programming algorithm to minimize the objective optimally in Section \ref{sec:algorithm}.

%\vspace{-0.05in}
%\section{Alternative Problem Formulation}
%\label{sec:alternative}
%In this section, we show that it is possible to formulate an alternative \textit{rate-constrained Maximum Likelihood (ML)} problem or \textit{unconstrained Maximum a Posterior (MAP)} problem by combining the prior term and the rate term into a \textit{super rate} term denoted by $R_s(\hat{\vec{x}})$ or a \textit{super prior} denoted by $P_s(\hat{\vec{x}})$ respectively.
\red{I don' get this super rate and super prior. why do we need this definitions? this needs more explanation why these combos are reasonable.}
The corresponding formulations are as follows:
\begin{equation}
\label{eq:alternative_lagrangian_objective}
\underset{\hat{\vec{x}}\in  \mathcal{S}}{\min}\ -\log P(\vec{y}|\hat{\vec{x}})+\lambda_s R_s(\hat{\vec{x}})
\end{equation}
\begin{equation}
\label{eq:alternative_MAP}
\underset{\hat{\vec{x}}\in  \mathcal{S}}{\max}\ P(\vec{y}|\hat{\vec{x}})\cdot P_s(\hat{\vec{x}})
\end{equation}
Different from (\ref{eq:lagrangian_objective}), (\ref{eq:alternative_lagrangian_objective}) performs the rate-constrained ML estimation, \textit{i.e.}, find $\hat{\vec{x}} \in  \mathcal{S}$ that maximize likelihood $P(\vec{y}|\hat{\vec{x}})$ subject to $R_s(\hat{\vec{x}})\leq R_{\max}$.
In contrast, (\ref{eq:alternative_MAP}) performs the unconstrained MAP estimation, \textit{i.e.}, find $\hat{\vec{x}} \in  \mathcal{S}$ that maximizes the product of the likelihood $P(\vec{y}|\hat{\vec{x}})$ and the super prior $P_s(\hat{\vec{x}})$.

In (\ref{eq:objective}), both the prior term and the rate term can be regarded as a modelling of the contour signal.
Specifically, the prior models the underlying statistics of the contour by assuming that contours are more likely to be straight than curvy, which is independent of the signal observations. 
In contrast, the rate term models the symbol probabilities from the training data, which is a data-driven statistical model.
It is possible to combine the two terms into one term as in (\ref{eq:alternative_lagrangian_objective}) and (\ref{eq:alternative_MAP}), but in general they are capturing distinct distributions of the contour.
For example, if the rate constraint is loose and the signal is corrupted by heavy noise, then the reconstructed contours obtained by solving (\ref{eq:objective}) without the prior term are not likely straight.
Similarly, after removing the rate term in (\ref{eq:objective}), the reconstructed contours may not have a small entropy (rate).
\red{point being made here is not crystal clear.}

To find out the relationship between the rate-constrained MAP, rate-constrained ML and unconstrained MAP problems, we rewrite (\ref{eq:alternative_MAP}) as follows:
\begin{equation}
\underset{\hat{\vec{x}}\in  \mathcal{S}}{\min}\ -\log P(\vec{y}|\hat{\vec{x}}) - \beta_s \log P_s(\hat{\vec{x}})
\end{equation}
When the combined super rate term and super prior term are both equal to the sum of the prior term and rate term in the original formulation (\ref{eq:objective}), \textit{i.e.}, 
\begin{equation}
\lambda_s R_s(\hat{\vec{x}}) = -\beta_s \log P_s(\hat{\vec{x}}) = \lambda R(\hat{\vec{x}}) - \beta \log P(\hat{\vec{x}}),
\end{equation}
all the three formulations lead to the same solution.
This provides additional insight into the relationship between the rate-constrained MAP, rate-constrained ML and unconstrained MAP problems.
\red{Idon't get this. why do we have this discussion?}

%\vspace{-0.05in}
\section{Optimization Algorithm}
\label{sec:algorithm}
%% This declares a command \Comment
%% The argument will be surrounded by /* ... */
\SetKwComment{Comment}{/* }{ */}

\begin{algorithm}[t]
\caption{Training Scheduler}\label{alg:TS}
% \KwData{$n \geq 0$}
% \KwResult{$y = x^n$}
\LinesNumbered
\KwIn{Training data $\mathcal{D}_{train}=\{(q_i, a_i, p_i^+)\}_{i=1}^m$, \\
\qquad \quad Iteration number $L$.}
\KwOut{A set of optimal model parameters.}

\For{$l=1,\cdots, L$}{
    Sample a batch of questions $Q^{(l)}$\\
    \For{$q_i\in Q^{(l)}$}{
        $\mathcal{P}_{i}^{(l)} \gets \mathrm{arg\,max}_{p_{i,j}}(\mathrm{sim}(q_i^{en},p_{i,j}),K)$\\
        $\mathcal{P}_{Gi}^{(l)} \gets \mathcal{P}_{i}^{(l)}\cup\{p^+_i\}$\\
        Compute $\mathcal{L}^i_{retriever}$, $\mathcal{L}^i_{postranker}$, $\mathcal{L}^i_{reader}$\\ according to Eq.\ref{eq:retriever}, Eq.\ref{eq:rerank}, Eq.\ref{eq:reader}\\
    }
    % $\mathcal{L}^{(l)}_{retriever} \gets \frac{1}{|Q^{(l)}|}\sum_i\mathcal{L}^i_{retriever}$\\
    % $\mathcal{L}^{(l)}_{retriever} \gets \mathrm{Avg}(\mathcal{L}^i_{retriever})$,
    % $\mathcal{L}^{(l)}_{rerank} \gets \mathrm{Avg}(\mathcal{L}^i_{rerank})$,
    % $\mathcal{L}^{(l)}_{reader} \gets \mathrm{Avg}(\mathcal{L}^i_{reader})$\\
    % Compute $\mathcal{L}^{(l)}_{retriever}$, $\mathcal{L}^{(l)}_{rerank}$, and $\mathcal{L}^{(l)}_{reader}$ by averaging over $Q^{(l)}$\\
    $\mathcal{L}^{(l)} \gets \frac{1}{|Q^{(l)}|}\sum_i(\mathcal{L}^{i}_{retriever} + \mathcal{L}^{i}_{postranker}+ \mathcal{L}^{i}_{reader})$\\
    $\mathcal{P}^{(l)}_K\gets\{\mathcal{P}^{(l)}_i|q_i\in Q^{(l)}\}$,\quad $\mathcal{P}^{(l)}_{KG}\gets\{\mathcal{P}^{(l)}_{Gi}|q_i\in Q^{(l)}\}$\\
    Compute the coefficient $v^{(l)}$ according to Eq.~\ref{eq:v}\\
  \eIf{$ v^{(l)}=1$}{
    $\mathcal{L}^{(l)}_{final} \gets \mathcal{L}^{(l)}(\mathcal{P}_{KG}^{(l)})$\\
  }{
      $\mathcal{L}^{(l)}_{final} \gets \mathcal{L}^{(l)}(\mathcal{P}^{(l)}_{K}),$\\
    }
    Optimize $\mathcal{L}^{(l)}_{final}$
}
\end{algorithm}


%  \eIf{$ \mathcal{L}^{(l-1)}_{retriever}<\lambda$}{
%     $\mathcal{L}^{(l)}_{final} \gets \mathcal{L}^{(l)}(\mathcal{P}_K^{(l)})$\\
%   }{
%       $\mathcal{L}^{(l)}_{final} \gets \mathcal{L}^{(l)}(\mathcal{P}^{(l)}_{KG}),$\\
%     }

%\vspace{-0.05in}
\section{Experimental Results}
\label{sec:results}
%!TEX ROOT = ../../centralized_vs_distributed.tex

\section{{\titlecap{the centralized-distributed trade-off}}}\label{sec:numerical-results}

\revision{In the previous sections we formulated the optimal control problem for a given controller architecture
(\ie the number of links) parametrized by $ n $
and showed how to compute minimum-variance objective function and the corresponding constraints.
In this section, we present our main result:
%\red{for a ring topology with multiple options for the parameter $ n $},
we solve the optimal control problem for each $ n $ and compare the best achievable closed-loop performance with different control architectures.\footnote{
\revision{Recall that small (large) values of $ n $ mean sparse (dense) architectures.}}
For delays that increase linearly with $n$,
\ie $ f(n) \propto n $, 
we demonstrate that distributed controllers with} {few communication links outperform controllers with larger number of communication links.}

\textcolor{subsectioncolor}{Figure~\ref{fig:cont-time-single-int-opt-var}} shows the steady-state variances
obtained with single-integrator dynamics~\eqref{eq:cont-time-single-int-variance-minimization}
%where we compare the standard multi-parameter design 
%with a simplified version \tcb{that utilizes spatially-constant feedback gains
and the quadratic approximation~\eqref{eq:quadratic-approximation} for \revision{ring topology}
with $ N = 50 $ nodes. % and $ n\in\{1,\dots,10\} $.
%with $ N = 50 $, $ f(n) = n $ and $ \tau_{\textit{min}} = 0.1 $.
%\autoref{fig:cont-time-single-int-err} shows the relative error, defined as
%\begin{equation}\label{eq:relative-error}
%	e \doteq \dfrac{\optvarx-\optvar}{\optvar}
%\end{equation}
%where $ \optvar $ and $ \optvarx $ denote the the optimal and sub-optimal scalar variances, respectively.
%The performance gap is small
%and becomes negligible for large $ n $.
{The best performance is achieved for a sparse architecture with  $ n = 2 $ 
in which each agent communicates with the two closest pairs of neighboring nodes. 
This should be compared and contrasted to nearest-neighbor and all-to-all 
communication topologies which induce higher closed-loop variances. 
Thus, 
the advantage of introducing additional communication links diminishes 
beyond}
{a certain threshold because of communication delays.}

%For a linear increase in the delay,
\textcolor{subsectioncolor}{Figure~\ref{fig:cont-time-double-int-opt-var}} shows that the use of approximation~\eqref{eq:cont-time-double-int-min-var-simplified} with $ \tilde{\gvel}^* = 70 $
identifies nearest-neighbor information exchange as the {near-optimal} architecture for a double-integrator model
with ring topology. 
This can be explained by noting that the variance of the process noise $ n(t) $
in the reduced model~\eqref{eq:x-dynamics-1st-order-approximation}
is proportional to $ \nicefrac{1}{\gvel} $ and thereby to $ \taun $,
according to~\eqref{eq:substitutions-4-normalization},
making the variance scale with the delay.

%\mjmargin{i feel that we need to comment about different results that we obtained for CT and DT double-intergrator dynamics (monotonic deterioration of performance for the former and oscillations for the latter)}
\revision{\textcolor{subsectioncolor}{Figures~\ref{fig:disc-time-single-int-opt-var}--\ref{fig:disc-time-double-int-opt-var}}
show the results obtained by solving the optimal control problem for discrete-time dynamics.
%which exhibit similar trade-offs.
The oscillations about the minimum in~\autoref{fig:disc-time-double-int-opt-var}
are compatible with the investigated \tradeoff~\eqref{eq:trade-off}:
in general, 
the sum of two monotone functions does not have a unique local minimum.
Details about discrete-time systems are deferred to~\autoref{sec:disc-time}.
Interestingly,
double integrators with continuous- (\autoref{fig:cont-time-double-int-opt-var}) ad discrete-time (\autoref{fig:disc-time-double-int-opt-var}) dynamics
exhibits very different trade-off curves,
whereby performance monotonically deteriorates for the former and oscillates for the latter.
While a clear interpretation is difficult because there is no explicit expression of the variance as a function of $ n $,
one possible explanation might be the first-order approximation used to compute gains in the continuous-time case.
%which reinforce our thesis exposed in~\autoref{sec:contribution}.

%\begin{figure}
%	\centering
%	\includegraphics[width=.6\linewidth]{cont-time-double-int-opt-var-n}
%	\caption{Steady-state scalar variance for continuous-time double integrators with $ \taun = 0.1n $.
%		Here, the \tradeoff is optimized by nearest-neighbor interaction.
%	}
%	\label{fig:cont-time-double-int-opt-var-lin}
%\end{figure}
}

\begin{figure}
	\centering
	\begin{minipage}[l]{.5\linewidth}
		\centering
		\includegraphics[width=\linewidth]{random-graph}
	\end{minipage}%
	\begin{minipage}[r]{.5\linewidth}
		\centering
		\includegraphics[width=\linewidth]{disc-time-single-int-random-graph-opt-var}
	\end{minipage}
	\caption{Network topology and its optimal {closed-loop} variance.}
	\label{fig:general-graph}
\end{figure}

Finally,
\autoref{fig:general-graph} shows the optimization results for a random graph topology with discrete-time single integrator agents. % with a linear increase in the delay, $ \taun = n $.
Here, $ n $ denotes the number of communication hops in the ``original" network, shown in~\autoref{fig:general-graph}:
as $ n $ increases, each agent can first communicate with its nearest neighbors,
then with its neighbors' neighbors, and so on. For a control architecture that utilizes different feedback gains for each communication link
	(\ie we only require $ K = K^\top $) we demonstrate that, in this case, two communication hops provide optimal closed-loop performance. % of the system.}

Additional computational experiments performed with different rates $ f(\cdot) $ show that the optimal number of links increases for slower rates: 
for example, 
the optimal number of links is larger for $ f(n) = \sqrt{n} $ than for $ f(n) = n $. 
\revision{These results are not reported because of space limitations.}

%\vspace{-0.05in}
\section{Conclusion}
\label{sec:conclude}
\section{Conclusion}
\label{sec:conclude}

We propose an algorithm named \ebdjoin+ for computing edit similarity join, one of the most important operations in database systems.  Different from all previous approaches, we first embed the input strings from the edit space to the Hamming space, and then try to perform a filtering (for reducing candidate pairs) in the Hamming space where efficient tools like locality sensitive hashing are available.  Our experiments have shown that \ebdjoin+ significantly outperforms, at a very small cost of accuracy, all existing algorithms on long strings and large thresholds.
%, which are critical to applications such as the analysis of genome sequences in bioinformatics. 

\bibliographystyle{IEEEbib}
\bibliography{ref}

\end{document}
