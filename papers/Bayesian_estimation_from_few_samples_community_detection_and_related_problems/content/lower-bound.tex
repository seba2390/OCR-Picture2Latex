\section{Lower bounds against low-degree polynomials at the Kesten-Stigum threshold}
\label{sec:lower-bound}

In this section we prove two lower bounds for $k$-community partial recovery algorithms based on low-degree polynomials.

\subsection{Low-degree Fourier spectrum of the k-community block model}
\begin{theorem}
  \label{thm:lower-bound-density}
  Let $d,\e,k$ be constants.
  Let $\mu : \{ 0,1\}^{n \times n} \rightarrow \R$ be the relative density of $SBM(n,d,\e,k)$ with respect to $G(n,\tfrac dn)$.
  Let $\mu^{\leq \ell}$ be the projection of $\mu$ to the degree-$\ell$ polynomials with respect to the norm induced by $G(n,\tfrac dn)$.\footnote{That is, $\|f\| = (\E_{G \sim G(n,\tfrac dn)} f(G)^2)^{1/2}$.}
  For any constant $\delta > 0$ and $\xi > 0$ (allowing $\xi \leq o(1)$),
  \[
  \|\mu^{\leq \ell}\| \text{ is }
  \begin{cases} \geq n^{\Omega(1)} \text{ if $\e^2 d > (1 + \delta) k^2, \ell \geq O(\log n)$}\\
  \leq n^{2 \xi} \text{ if $\e^2 d < (1 - \delta) k^2, \ell < n^{\xi}$ }
  \end{cases} \mper
  \]
\end{theorem}
This proves Theorem~\ref{thm:lower-bound} (see discussion following statement of that theorem).
To prove the theorem we need the following lemmas.
\begin{lemma}
  \label{lem:lb-degree-one}
  Let $\chi_\alpha : \{0,1\}^{n\times n} \rightarrow \R $ be the $\tfrac dn$-biased Fourier character.
  If $\alpha \subseteq \binom{n}{2}$, considered as a graph on $n$ vertices, has any degree-one vertex, then
  \[
  \E_{G \sim SBM(n,d,\e,k)} \chi_\alpha(G) = 0
  \]
\end{lemma}
The proof follows from calculations very similar to those in Section~\ref{sec:mm}, so we omit it.


\begin{proof}[Proof of Theorem~\ref{thm:lower-bound-density}]
  The bound $\|\mu^{\leq \ell}\| \geq n^{\Omega(1)}$ when $\e^2d > (1 + \delta)k^2$ and $\ell \gg \log(n)$, follows from almost identical calculations to Section~\ref{sec:mm},\footnote{The calculations in Section~\ref{sec:mm} are performed for long-armed stars; to prove the present result the analogous calculations should be performed for cycles of logarithmic lengh. Similar calculations also appear in many previous works.} so we omit this argument and focus on the regime $\e^2d < (1 - \delta)k^2$.

  By definition and elementary Fourier analysis,
  \begin{align}
  \|\mu^{\leq \ell}\|^2 = \sum_{\alpha \subseteq \binom{n}{2}, |\alpha| \leq \ell} \widehat{\mu}(\alpha)^2 \label{eq:lb-2}
  \end{align}
  Also by definition,
  \[
    \widehat{\mu}(\alpha) = \E_{G \sim G(n,\tfrac dn)} \mu(G) \chi_\alpha(G) = \E_{G \sim SBM(n,d,\e,k)} \chi_\alpha
  \]
  where $\{\chi_\alpha\}$ are the $\tfrac dn$-biased Fourier characters.
  Thus, using Lemma~\ref{lem:lb-degree-one} we may restrict attion to the contribution of those $\alpha \subseteq \binom{n}{2}$ with $|\alpha| \leq \ell$ and containing no degree-$1$ vertices.

  Fix such an $\alpha$, and suppose it has $C(\alpha)$ connected components and $V_2(\alpha)$ vertices of degree $2$ (considered again as a graph on $[n]$).
  Fact~\ref{fact:character-est} (following this proof) together with routine computations shows that
  \[
    \Paren{\E_{G \sim SBM(n,d,\e,k)} \chi_\alpha(G)}^2 \leq \Paren{(1 + O(\tfrac dn)) \e^2 \tfrac dn}^{|\alpha|} k^{-2 (V(\alpha) - C(\alpha))}
    \leq \Paren{1 + O(\tfrac dn)}^{|\alpha|} \cdot n^{-|\alpha|} \cdot (1 - \delta)^{|\alpha|} \cdot k^{2(|\alpha| - V(\alpha) + C(\alpha))}\mper
  \]
  Let $c(\alpha) = \Paren{1 + O(\tfrac dn)}^{|\alpha|} \cdot n^{-|\alpha|} \cdot (1 - \delta)^{|\alpha|} \cdot k^{2(|\alpha| - V(\alpha) + C(\alpha))}$ be this upper bound on the contribution of $\alpha$ to the right-hand side of \eqref{eq:lb-2}.
  It will be enough to bound
  \[
    (*) \quad \defeq \sum_{\substack{ \alpha \subseteq \binom{n}{2} \\ |\alpha| \leq \ell \\ \alpha \text{ has no degree 1 nodes}}} c(\alpha)
  \]
  Given any $\alpha$ as in the sum, we may partition it into two vertex-disjoint subgraphs, $\alpha_0$ and $\alpha_1$, where $\alpha_0$ is a union of cycles and no connected component of $\alpha_1$ is a cycle, such that $\alpha = \alpha_0 \cup \alpha_1$.
  Thus,
  \[
   (*) \leq \Paren{\sum_{\alpha_0} c(\alpha_0) } \Paren {\sum_{\alpha_1} c(\alpha_1) }
  \]
  where $\alpha_0$ ranges over unions of cycles with $|\alpha_0| \leq \ell$ and $\alpha_1$ ranges over graphs on $[n]$ with at most $\ell$ where all degrees are at least $2$ and containing no connected component which is a cycle.
  Lemmas \ref{lem:lb-no-cycles} and \ref{lem:lb-cycles}, which follow, the terms above as $O(1)$ and $n^{2 \xi}$, respectively, which finishes the proof.
\end{proof}

\begin{fact}\label{fact:character-est}
  Let $U$ be a connected graph on $t$ vertices where all degrees are at least $2$.
  For each vertex $v$ of $U$ let $\sigma_v \in \R^k$ be a uniformly random standard basis vector.
  Let $\tsigma_v = \sigma_v - \tfrac 1k \cdot 1$.
  Then
  \[
    \Abs{\E \prod_{(u,v) \in U} \iprod{\tsigma_v, \tsigma_u}} \leq  t k^{-t + 1}
  \]
\end{fact}
\begin{proof}
  Consider a particular realization of $\sigma_1,\ldots,\sigma_t$.
  Suppose all but $m$ vertices $v$ in $U$ are adjacent to at least $2$ vertices $u_1,u_2$ such that $\sigma_{u_1} \neq \sigma_v$ and $\sigma_{u_2} \neq \sigma_v$.
  In this case,
  \[
  \Abs{\prod_{(u,v) \in U} \iprod{\tsigma_v, \tsigma_u}} \leq k^{-(t-m)}\mper
  \]
  The probability of such a pattern of disagreements is at most $k^{-m}$, unless $m = t$, in which case the probability is at most $k^{-t+1}$.
  The fact follows.
\end{proof}

\begin{lemma}
\label{lem:lb-no-cycles}
  For $\alpha \subseteq \binom{n}{2}$, let $V(\alpha)$ be the number of vertices in $\alpha$, let $C(\alpha)$ be the number of connected components in $\alpha$.
  For constants $\e,d,k$, let $c(\alpha) \defeq \Paren{1 + O(\tfrac dn)}^{|\alpha|} \cdot n^{-|\alpha|} \cdot (1 - \delta)^{|\alpha|} \cdot k^{2(|\alpha| - V(\alpha) + C(\alpha))}$
  Let $\ell \leq n^{0.01}$ and
  \[
  U = \left \{ \alpha \subseteq \binom{n}{2} \, : \, \alpha \text{ has all degrees $\geq 2$, has no connected components which are cycles, $|\alpha| \leq \ell$} \right \}\mper
  \]
  Then
  \[
  \sum_{\alpha \in U} c(\alpha) \leq O(1)\mper
  \]
\end{lemma}
\begin{proof}
  We will use a coding argument to bound the number of $\alpha \in U$ with $V$ vertices, $E$ edges, and $C$ connected components.
  We claim that any such $\alpha$ is uniquely specified by the following encoding.

  To encode $\alpha$, start by picking an arbitrary vertex $v_1$ in $\alpha$.
  List the vertices $v_1,\ldots,v_{|V|}$ of $\alpha$, each requiring $\log n$ bits, starting from $v_1$, using the following rules to pick $v_i$.
  \begin{enumerate}
    \item  If $v_{i-1}$ has a neighbor not yet appearing in the list $v_1,\ldots,v_{i-1}$, let $v_i$ be any such neighbor.
    \item Otherwise, if $v_{i-1}$ has a neighbor $v_j$ which
    \begin{enumerate}
    \item appears in the list $v_1,\ldots,v_{i-1}$ and
    \item for which either $j = 1$ or $v_{j-1}$ is not adjacent to $v_j$ in $\alpha$, and
    \item for which if $j \neq i'$ for $i' \leq i-1$ being the minimal index such that $v_{i'},\ldots,v_{i-1}$ is a path in $\alpha$ (i.e. $v_j,\ldots,v_{i-1}$ are not a cycle in $\alpha$)
    \end{enumerate}
    then reorder the list as follows.
  Remove vertices $v_j,\ldots,v_{j'}$ where $j'$ is the greatest index so that all edges $v_{\ell},v_{\ell+1}$ exist in $\alpha$ for $j \leq \ell \leq j'$.
  Also remove vertices $v_{i'},\ldots,v_{i-1}$ where $i'$ is analogously the minimal index such that edes $v_{\ell},v_{\ell+1}$ exist in $\alpha$ for $i' \leq \ell \leq i-1$.
  Then, append the list $v_{j'},v_{j'-1},\ldots,v_j,v_{i-1},\ldots,v_{i'}$.
  By construction, all of these vertices appear in a path in $\alpha$.
  The new list retains the invariant that every vertex either preceeds a neighbor in $\alpha$ or has no neighbors in $\alpha$ which have not previous appeared in the list.
  \item
  Otherwise, let $v_i$ be an arbitrary vertex in $\alpha$ in the same connected component as $v_{i-1}$, if some such vertices has not yet appeared in the list.
  \item Otherwise, let $v_i$ be an arbitrary vertex of $\alpha$ not yet appearing among $v_1,\ldots,v_{i-1}$.
  \end{enumerate}
  After the list of vertices, append to the encoding the following information.
  First, a list of the $R$ (for \textbf{r}emoved) pairs $v_i,v_{i+1}$ for which there is not an edge $(v_i,v_{i+1})$ in $\alpha$.
  This uses $2 R \log V$ bits.
  Last, a list of the edges in $\alpha$ which are not among the pairs $v_i,v_{i+1}$ (each edge encoded using $2 \log V$ bits).

  We argue that the number $R$ of removed pairs (and hence the length of their list in the encoding) is not too great.
  In particular, we claim $R \leq 2(E - V)$.
  In fact, this is true connected-component-wise in $\alpha$.
  To see it, proceed as follows.

  Fix a connected component $\beta$ of $\alpha$.
  Let $v_t$ be the first vertex in $\beta$ to appear in the list $v_1,\ldots,v_{|V|}$.
  Proceeding in increasing order down the list from $v_t$, let $(v_{r_1},v_{r_1+1}), (v_{r_2},v_{r_2 +1}),\ldots$ be the pairs encountered (before leaving $\beta$) which do not correspond to edges in $\alpha$ (and hence will later appear in the list of removed pairs).

  Construct a sequence of subgraphs $\beta_j$ of $\beta$ as follows.
  The graph $\beta_1$ is the line on vertices $v_t,\ldots,v_{r_1}$.
  To construct the graph $\beta_{j}$, start from $\beta_{j-1}$ and add the line from $v_{r_{j-1}+1}$ to $v_{r_{j}}$ (by definition all these edges appear in $\beta$).
  Since $v_{r_j}$ must have at least degree $2$, it has a neighbor $u_j$ in $\beta$ among the vertices $v_{a}$ for $a < r_j$ aside from $v_{r_j-1}$.
  (If $v_{r_j}$ had a neighbor not yet appearing in the list, then $v_{r_j+1}$ would have been that neighbor, contrary to assumption.)
  Choose any such neighbor and add it to $\beta_j$; this finishes construction of the graph $\beta_j$.
  For later use, note that either adding the edge to $u_j$ turns $\beta_j \setminus b_{j-1}$ into a cycle or $u_j$ is not itself among the $v_r$'s, since otherwise in constructing the list we would have done a reordering operation.

  In each of the graphs $\beta_j$, the number of edges is equal to the number of vertices.
  To obtain $\beta$, we must add $E_\beta - V_\beta$ edges (where $E_\beta$ is the number of edges and $\beta$ and $V_\beta$ is the number of vertices).
  We claim that in so doing at least one half of a distinct such edge must be added per $\beta_j$; we prove this via a charging scheme.
  As noted above, each graph $\beta_j \setminus \beta_{j-1}$ either contains $v_{r_{j-1}}$ as a degree-$1$ vertex or it forms cycle.
  If it contains a degree-1 vertex, by construction this vertex is not $u_{j'}$ for any $j' > j$, otherwise we would have reordered.
  So charge $\beta_j$ to the edge which must be added to fix the degree-1 vertex.

  In the cycle case, either some edge among the $E_\beta - V_\beta$ additional edges is added incident to the cycle (in which case we charge $\beta_j$ to this edge), or some $u_{j'}$ for $j' > j$ is in $\beta_j \setminus \beta_{j-1}$.
  If the latter, then $\beta_{j'} \setminus \beta_{j' - 1}$ contains a degree-1 vertex and $\beta_{j} \setminus \beta{j -1}$ can be charged to the edge which fixes that degree $1$ vertex.
  Every additional edge was charged at most twice.
  Thus, $R \leq 2(E - V)$

  It is not hard to check that $\alpha$ can be uniquely decoded from the encoding previously described.
  The final result of this encoding scheme is that each $\alpha$ can be encoded with at most $V \log n + 6(E-V) \log V$ bits, and so there are at most $n^V \cdot V^{6(E-V)}$ choices for $\alpha$.
  The contribution of such $\alpha$ to $\sum_{\alpha \in U} c(\alpha)$ is thus at most
  \[
  n^{-(E-V)} V^{6(E-V)} (1 - \delta/2)^E k^{2(E-V+C)}
  \]
  We know that $C \leq E-V$.
  So as long as $k,V \leq n^{0.01}$, we obtain that this contributes at most $n^{(E-V)/2} (1 - \delta/2)^E$.
  Summing across all $V,E \leq n^{0.01}$, the lemma follows.
\end{proof}


\begin{lemma}\label{lem:lb-cycles}
  For $\alpha \subseteq \binom{n}{2}$, let $V(\alpha)$ be the number of vertices in $\alpha$, let $C(\alpha)$ be the number of connected components in $\alpha$.
  For constants $1 > \delta > 0$ and $k$, let $c(\alpha) \defeq \Paren{1 + O(\tfrac dn)}^{|\alpha|} \cdot n^{-|\alpha|} \cdot (1 - \delta)^{|\alpha|} \cdot k^{2(|\alpha| - V(\alpha) + C(\alpha))}$
  Let $\ell \leq n^{\xi/k^2}$ for some $\xi > 0$ (allowing $\xi \leq o(1)$) and
  \[
  U = \left \{ \alpha \subseteq \binom{n}{2} \, : \, \alpha \text{ is a union of cycles} \right \}\mper
  \]
  Then
  \[
    \sum_{\alpha \in U} c(\alpha) \leq  n^{2 \xi}
  \] 
\end{lemma}
\begin{proof}
  Let $U_t$ be the set of $\alpha$ which are unions of $t$-cycles (we exclude the empty $\alpha$).
  Let $c_t = \sum_{\alpha \in U_t} c_\alpha$.
  Then
  \[
   \sum_{\alpha \in U} c(\alpha) \leq  \prod_{t \leq \ell} (1 + c_t)\mper
  \]
  Count the $\alpha \in U_t$ which contain exactly $p$ cycles of length $t$ by first choosing a list of $pt$ vertices---there are $n^{pt}$ choices.
  In doing so we will count each alpha $p! t^p$ times, since each of the $p$ cycles can be rotated and the cycles can themselves be exchanged.
  All in all, there are at most $n^{pt}/(p! t^p)$ such $\alpha$, and they contribute at most
  \[
  \frac{c(\alpha) n^{pt}}{p! t^p} \leq \frac{ (1 - \delta/2)^{pt} k^{2p}}{p! t^p} \leq k^{2p}/(p! t^p)\mper
  \]
  for large enough $n$.
  Thus, summing over all $\alpha \in U_t$, we get
  \[
    (1 + c_t) \leq \sum_{p=0}^\ell \frac{ (1 - \delta/2)^p k^{2p}}{p! t^p} \leq \exp( k^2 /t)\mper
  \]
  So,
  \[
    \prod_{t \leq \ell} (1 + c_t) \leq \exp(k^2 \sum_{t=1}^\ell 1/t) \leq \exp(k^2 \log 2\ell) \leq (2\ell)^{k^2} \leq n^{2\xi}\mper
  \]
\end{proof}



\subsection{Lower bound for estimating communities}
\begin{theorem}
  Let $d,\e,k,\delta$ be constants such that $\e^2 d < (1 -\delta)k^2$.
  Let $f : \{0,1\}^{n \times n} \rightarrow \R$ be any function, let $i,j \in [n]$ be distinct.
  Then if $f$ satisfies $\E_{G \sim G(n,\tfrac dn)} f(G) = 0$ and is correlated with the indicator $\Ind_{\sigma_i = \sigma_j}$ that $i$ and $j$ are in the same community in the following sense:
  \[
    \frac{\E_{G \sim SBM(n,d,\e,k)} f(G)(\Ind_{\sigma_i = \sigma_j} - \tfrac 1k)}{(\E_{G \sim G(n,\tfrac dn)} f(G)^2)^{1/2}} \geq \Omega(1)
  \]
  then $\deg f \geq n^{c(d,\e,k)}$ for some $c(d,\e,k) > 0$.
\end{theorem}
\begin{proof}
  Let $g(G) = \mu(G) \E[\Ind_{\sigma_i = \sigma_j} - \frac 1k \, | \, G ]$, where $\mu(G)$ is the relative density of $SBM(n,d,\e,k)$.
  Standard Fourier analysis shows that the optimal degree-$\ell$ choice for such $f$ to maximize the above correlation is $g^{\leq \ell}$, the orthogonal projection of $g$ to the degree-$\ell$ polynomials with respect to the measure $G(n,\tfrac dn)$, and the correlation is at most $\|g^{\leq \ell}\|$.
  It suffices to show that for some constant $c(d,\e,k)$, if $\ell < n^{c(d,\e,k)}$ then $\|g^{\leq \ell}\| \leq o(1)$.

  For this we expand $g$ in the Fourier basis, noting that
  \[
    \widehat{g}(\alpha) = \E_{\sigma,G \sim SBM(n,d,\e,k)} \iprod{\tsigma_i,\tsigma_j} \chi_\alpha(G)
  \]
  where as usual $\tsigma_i = \sigma_i - \tfrac 1k \cdot 1$ is the centered indicator of $i$'s community.
  By-now routine computations show that
  \[
  \widehat{g}(\alpha)^2 \leq \Paren{(1 + O(d/n)) \e^2 \tfrac dn }^{|\alpha|} \cdot \Paren{\E \iprod{\tsigma_i,\tsigma_j} \cdot \prod_{(k,\ell) \in \alpha} \iprod{\tsigma_i,\tsigma_j}}^2
  \]
  We assume that $(i,j) \notin \alpha$; it is not hard to check that such $\alpha$'s dominate the norm $\|g^{\leq \ell}\|$.
  If some vertex aside from $i,j$ in $\alpha$ has degree $1$ then this is zero.
  Similarly, if $i$ or $j$ does not appear in $\alpha$ then this is zero.
  Otherwise,
  \[
  \widehat{g}(\alpha)^2 \leq \Paren{(1 + O(d/n)) }^{|\alpha|} n^{-|\alpha|} (1 - \delta)^{|\alpha|} k^{2(|\alpha| - V(\alpha) + C(\alpha)}
  \]
  where as usual $V(\alpha)$ is the number of vertices in $\alpha$ and $C(\alpha)$ is the number of connected components in $\alpha$.
  Let $\beta(\alpha)$ be the connected component of $\alpha$ containing $i$ and $j$ (if they are not in the same component the arguments are mostly unchanged).
  Then we can bound
  \[
  \|g^{\leq \ell}\|^2 = \sum_{|\alpha| \leq \ell} \widehat{g}(\alpha)^2 \leq \|\mu^{\leq \ell}\|^2 \cdot \sum_{\beta} \Paren{(1 + O(d/n)) }^{|\beta|} n^{-|\beta|} (1 - \delta)^{|\beta|} k^{2(|\beta| - V(\beta) + 1}
  \]
  where $\beta$ ranges over connected graphs with vertices from $[n]$, at most $\ell$ edges, every vertex except $i$ and $j$ having degree at least $2$, and containing $i$ and $j$ with degree at least $1$.
  There are at most $n^{V-2} V^{O(E-V)}$ such graphs containing at $V$ vertices aside from $i$ and $j$ and $E$ edges (by an analogous argument as in Lemma~\ref{lem:lb-no-cycles}).\Snote{}
  The total contribution from such $\beta$ is therefore at most
  \[
    \frac{k^{2(E - V + 1)} V^{O(E - V)}}{n^{E - V + 2}}
  \]
  Summing over $V$ and $E$, we get 
  \[
    \sum_{\beta} \Paren{(1 + O(d/n)) }^{|\beta|} n^{-|\beta|} (1 - \delta)^{|\beta|} k^{2(|\beta| - V(\beta) + 1} \leq n^{-\Omega(1)}
  \]
  so long as $\ell \leq n^c$ for small enough $c$.
  Using Theorem~\ref{thm:lower-bound-density} to bound $\|\mu^{\leq \ell} \|$ finishes the proof.
\end{proof}


\iffalse

\Snote{}
\subsection{Proofs of lemmas}
\begin{proof}[Proof of Lemma~\ref{lem:lb-odd-degrees}]
  As usual, we denote by $\sigma_i \in \R^k$ the indicator vector of vertex $i$'s community and $\tsigma_i = \sigma_i - \tfrac 1k \cdot 1$ its centered version.
  Expanding $\chi_\alpha$, by similar computations as in Section~\ref{sec:mm},
  \[
  \E \chi_\alpha(G) = \Paren{\e \sqrt{\tfrac dn (1 - \tfrac dn)^{-1}}}^{|\alpha|} \E_\sigma \prod_{(a,b) \in \sigma}\iprod{\tsigma_a,\tsigma_b}\mper\qedhere
  \]
\end{proof}





\paragraph{Preliminaries}
In this section we work with a different norm and inner product on functions $f : \{ \pm 1\}^{\binom{n}{2}} \rightarrow \R$, defined by the \Erdos-Renyi distribution $G(n,\tfrac d n)$.
Let $\iprodu{f,g} = \E_{G \sim G(n,\tfrac dn)} f(G) g(G)$.
Let $\Normu{f} = \iprodu{f,f}^{1/2}$.
Here the subscript $u$ stands for ``uniform,'' since $G(n,\tfrac d n)$ is very similar to the uniform distribution on graphs with $\tfrac{dn}{2}$ edges.
The degree parameter $d$ will always be clear from context.

We also define a different projection to low degree functions.
Let $f : \{ \pm 1\}^{\binom{n}{2}} \rightarrow \R$.
We write
\[
  f^{\lequ D} = \argmax_{g} \tfrac{\iprodu{f,g}}{\Normu{g}}\mper
\]
Letting $G^\alpha$ be the $\tfrac d n$-biased characters, standard Fourier analysis shows that
\[
  f^{\lequ D}(G) = \sum_{|\alpha| \leq D} \widehat{f}(\alpha) G^\alpha\mper
\]

We will work with the non-overlapping version of the $k$-community block model.
This is the model of the previous section in the regime $\alpha \rightarrow 0$.
Alternatively, we sample the community membership vectors $\sigma \in \R^k$ as random standard basis vectors.
We denote this model $G(n,d,\epsilon,k)$.


\begin{theorem}[No safe estimators below the computational threshold]\label{thm:block-model-lowerbound}
  Let $\sigma_1,\ldots,\sigma_n, G \sim G(n,d,\epsilon, k)$.
  Let $\tsigma_i = \sigma_i - \tfrac 1 k \cdot 1$ be the centered version of $\sigma_i$.
  Let $M = \{\iprod{\tsigma_i,\tsigma_j}\}_{ij \in [n]^2}$ be the (centered version of) the matrix indicating which pairs of nodes are in the same community.

  Suppose $\epsilon^2 d < (1 - \delta) k^2$ for some $\delta = \Omega(1)$.
  Then for any matrix-valued function $P(G) : \{\pm 1\}^{\binom{n}{2}} \rightarrow \R^{n \times n}$ with $\deg P \leq n^{o(1)}$,
  \[
    \frac{\iprodu{P, M}}{\Normu{P}\cdot \Normu{M}}  = o(1)\mper
  \]
\Snote{}
\end{theorem}

Our first lemma characterizes the behavior of the Fourier characters $G^\alpha$ under the distribution $G(n,d,\epsilon,k)$.
The calculations involved to prove it are quite similar to those in the previous section, with two main differences.
First, rather than just considering $\alpha$ with the ``long-armed star'' shape, we will consider arbitrary $\alpha$.
However, we do not need to carry out the calculation exactly; we will be able to make some approximations.

Let $\mu(G)$ be the relative density of $G(n,d,\epsilon,k)$ with respect to $G(n,\tfrac d n)$.
That is, for any function $g : \{ \pm 1\}^{\binom{n}{2}} \rightarrow \R$,
\[
  \E_{G(n,d,\epsilon,k)}g(G) = \E_{G(n,\tfrac d n)} \mu(G) g(G)\mper
\]

\begin{lemma}\label{lem:best-safe-estimator}
  Let $f(\sigma)$ be a real-valued function of community memberships.
  Let $g(G) = \E_{G(n,d,\epsilon,k)}[f(\sigma) \, | \, G]$.
  The best degree-$D$ safe estimator of $f$ is the function $(\mu \cdot g)^{\lequ D}$.
  That is,
  \[
    (\mu \cdot g)^{\leq D} = \argmax_{\deg h \leq D} \frac{\iprod{h, f}}{\Normu{h}} = \argmax_{\deg h \leq D} \frac{\E_{\sigma, G \sim G(n,d,\epsilon,k)} f(\sigma) h(G)}{(\E_{G(n,\tfrac d n)} h(G)^2)^{1/2}}\mper
  \]
\end{lemma}
\begin{proof}
  We start by rewriting the numerator of the last expression.
  \[
    \E_{\sigma, G \sim G(n,d,\epsilon,k)} f(\sigma) h(G)
    = \E_{G \sim G(n,\tfrac d n)}\Brac{ \mu(G) \cdot h(G) \cdot \E_{G(n,d,\epsilon,k)}[f(\sigma) \, |\,  G]}
    = \E_{G \sim G(n,\tfrac d n)}\Brac{ \mu(G) \cdot h(G) \cdot g(G)}\mper
  \]
  Thus, the best safe estimator of $f$ with degree at most $D$ is the solution to
  \[
    \argmax_{\deg h \leq D} \frac{\iprodu{\mu \cdot g, h}}{\Normu{h}}
  \]
  which is by definition $(\mu \cdot g)^{\lequ D}$.
\end{proof}
By this lemma and elementary Fourier analysis, the quality of the best safe estimator will be given by
\[
  \frac{\iprodu{\mu \cdot g, (\mu \cdot g)^{\lequ D}}}{\Normu{(\mu \cdot g)^{\lequ D}}} = \frac{\Normu{(\mu \cdot g)^{\lequ D}}^2}{\Normu{(\mu \cdot g)^{\lequ D}}} = \Normu{(\mu \cdot g)^{\lequ D}} = \Paren{\sum_{|\alpha| \leq D} (\widehat{\mu \cdot g})(\alpha)^2}^{1/2}
\]
Thus, it will be enough to bound the Fourier coefficients $(\widehat{\mu \cdot g})(\alpha)$.
These are given by
\[
 (\widehat{\mu \cdot g})(\alpha)
 = \E_{G \sim G(n,\tfrac d n)} \mu(G) \cdot g(G) \cdot G^\alpha
 = \E_{G \sim G(n,d,\epsilon,k)} g(G) G^\alpha
 = \E_{\sigma, G \sim G(n,d,\epsilon,k)} f(\sigma) G^\alpha\mper
\]
\begin{lemma}\label{lem:lowerbound-coeffs}
  Let $\sigma_1,\ldots,\sigma_n, G \sim G(n,d,\epsilon, k)$.
  Let $\tsigma_i = \sigma_i - \tfrac 1 k \cdot 1$ be the centered version of $\sigma_i$.

  Suppose $i,j,k \in [n]$ are all distinct.
  Let $\alpha \subseteq {\binom{n}{2}}$ have $V$ vertices and $E$ edges.
  \begin{enumerate}[(a)]
    \item If $\deg_\alpha(i)$, the number of edges in $\alpha$ adjancent to $i$, is $0$, then for every $s \in [k]$,
      \[
        \E_{G(n,d,\e,k)} \tsigma_i(s) \tsigma_j(s) \tsigma_k(s) G^\alpha = 0
      \]
    and the same goes for $\deg_\alpha(j), \deg_\alpha(k)$.
  \item If $c$ is the number of vertices among $\{i,j,k\}$ having degree exactly $1$ or $2$ in $\alpha$, and $V_2(\alpha)$ are those vertices not equal to $i,j,k$ having degree exactly $2$ in $\alpha$, then
    \[
      \Abs{\E_{G(n,d,\e,k)} \tsigma_i(s) \tsigma_j(s) \tsigma_k(s) G^\alpha} \leq 
\Paren{\epsilon \sqrt{\tfrac dn} (1 + O(\tfrac d n))}^{|E(\alpha)|} k^{-|V_2(\alpha)| - c}\mper
    \]
  \end{enumerate}
\end{lemma}
We will need the following fact; the proof is elementary.
\begin{fact}\label{fact:lowerbound-covariance}
  Let $\sigma \in \R^k$ be a random standard basis vector, and let $\tsigma = \sigma - \tfrac 1 k \cdot 1$.
  Then $\E \tsigma \tsigma^\top = \tfrac 1 k \cdot \Pi$, where $\Pi$ is the projector to the subspace orthogonal to the all-$1$s vector.
\end{fact}
\begin{proof}
  Fix $\alpha$.
  Since $\alpha$ may be thought of as a graph, we write $V(\alpha)$ for the vertices indicent to edges in $\alpha$, and $E(\alpha)$ for the edges in $\alpha$.
  Let $V_2(\alpha)$ be the number of vertices in $V(\alpha) \setminus \{i,j,k\}$ with degree $2$, and let $\alpha_2$ be the graph obtained by removing each of the vertices in $V_2$ from $\alpha$ and connecting its neighbors by an edge.
  By similar computations as in the preceding section,
  \begin{align*}
    \E_{G(n,d,\e,k)} & \tsigma_i(s) \tsigma_j(s) \tsigma_k(s) G^\alpha\\
    & = \Paren{\epsilon \sqrt{\tfrac dn} (1 + O(\tfrac d n))}^{|E(\alpha)|} \E_{\sigma}\Brac{ \tsigma_i(s) \tsigma_j(s) \tsigma_k(s)\prod_{(a,b) \in \alpha} \iprod{\tsigma_a, \tsigma_b}}\\
    & = \Paren{\epsilon \sqrt{\tfrac dn} (1 + O(\tfrac d n))}^{|E(\alpha)|} k^{-|V_2(\alpha)|} \E_{\sigma}\Brac{ \tsigma_i(s) \tsigma_j(s) \tsigma_k(s)\prod_{(a,b) \in \alpha_2 } \iprod{\tsigma_a, \tsigma_b}}
  \end{align*}
    In the last step, we have used Fact~\ref{fact:lowerbound-covariance} to take the expectation over all the $\sigma_\ell$'s for $\ell \in V_2(\alpha)$.
    Now, if all the remaining vertices $(V \setminus V_2) \cup \{i,j,k\}$ have degree $2$ or greater in $\alpha_2$ then we are done, because the expression inside the expectation is always bounded by $1$.

    The only other possibility is that one or more of the vertices $\{i,j,k\}$ have degree $1$.
    (If any other vertices have degree $1$, then the whole expression is $0$.)
    In this case, we can use Fact~\ref{fact:lowerbound-covariance} again before bounding what remains inside the expectation by $1$ to conclude the lemma.
\end{proof}

Now we can prove Theorem~\ref{thm:block-model-lowerbound}.
\begin{proof}[Proof of Theorem~\ref{thm:block-model-lowerbound}]
  We will focus first on entries $i,j,k$ of the $3$-tensor $\sum_{s \in [k]} v_s^{\tensor 3}$ with $i \neq j \neq k$.

  From Lemma~\ref{lem:best-safe-estimator} and the discussion following it, we know that to analyze the best safe degree-$D$ estimator of $f(\sigma) = \sum_{s \in [k]}v_s(i) v_s(j) v_s(k)$ it is enough to bound
  \[
    \sum_{|\alpha| \leq D} \Paren{\E_{\sigma, G \sim G(n,d,\epsilon,k)} f(\sigma) G^\alpha}^{2}\mper
  \]
  First,
  consider the contribution to this sum from all $\alpha$ with $\tfrac{E(\alpha)}{V(\alpha)} = 1 + \delta$ for constant $\delta = \Omega(1)$.
  By Lemma~\ref{lem:lowerbound-coeffs}, the contribution of each term like this is at most
  $\Paren{\epsilon \sqrt{\tfrac dn} (1 + O(\tfrac d n))}^{2(1 + \delta)|V(\alpha)|} k^{-2|V_2(\alpha)|}$.
  How many such terms are there?
  To choose such a term with $t$ vertices, first choose $t$ vertices (there are at most $n^t$ choices), and then choose $(1 + \delta)t$ edges (there are at most $\binom{\binom t 2}{(1 + \delta)t} \leq t^{2(1 + \delta)t}$ choices).
  So, the total contribution of such terms with $t$ vertices is at most
  \[
    \Paren{\e^2 d (1 + O(\tfrac d n))}^{(1 + \delta)t} n^{-\delta t} k^{-2|V_2(\alpha)|} t^{{2(1 + \delta)t}}
  \]
For $\delta = \Omega(1)$ and $t,k = n^{o(1)}$, we have $n^{\delta/2 } > kt$, so such terms with $t$ vertices contribute at most $n^{-\delta t/2}$, and $\sum_{t = 1}^n n^{-\delta t/2} \leq O(n^{-\delta/2})$.

Next, we address the contribution from $\alpha$'s with $\tfrac{|E(\alpha)|}{|V(\alpha)|} = 1 + o(1)$.
Suppose every vertex incident to edges in $\alpha$ has degree at least $2$.
Then 
The only vertices incident to edges in $\alpha$ which might have degree $1$ are $i,j,$ and $k$; the rest of the vertices have degree at least $2$.
(Any other $\alpha$'s will contribute $0$.)


  \Snote{} 

\end{proof}
\fi