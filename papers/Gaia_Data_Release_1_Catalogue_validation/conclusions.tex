%+++++++++++++++++++++++++++++++++++++++++++++++++++++++++++++++++++++++++++
\section{Conclusions and recommendations for data usage}\label{sec:conclusions}
%+++++++++++++++++++++++++++++++++++++++++++++++++++++++++++++++++++++++++++

This paper summarizes the results of the validation tests applied to the
first Gaia data release as a final 
quality control before its publication. These tests have both 
confirmed the global quality of the data and shown several shortcomings
due to the preliminary nature of the release, based on a 
limited set of observations and processed using initial 
versions of the processing pipelines, see 
\citet{DPACP-14,DPACP-12,DPACP-8}
for a more detailed discussion on these issues.

We advise the users of {\GDR1} to keep these shortcomings
in mind for its scientific exploitation since they may
have relevant effects on the final results extracted from
them. In the next sections we discuss some of the main
limitations arising from them, but the limitations
for the use of the Gaia data in any specific case
should be carefully assessed as a part of the data analysis.


%----------------------------------------
\subsection{Effect of correlations}\label{sec:effect-correlations}
%----------------------------------------

The astrometric data in DR1 is provided with formal uncertainties for each one of
the parameters (five in the case of TGAS and two in the case of the main
catalogue). Although these standard uncertainties are enough when using each
of the parameters in isolation, they do not contain the complete information 
about the error distribution of the astrometric data. Indeed, the astrometry
of a star in the Gaia catalogue is the result of the Astrometric Global 
Iterative Solution -- \cite{DPACP-14} -- and therefore its parameters
(whether two or five) are obtained from a joint fitting during the Source
Update stage. Thus, strictly speaking, the error distributions of these 
parameters can only be described by a joint distribution of all of them. 

For this reason DR1 provides, in addition to the standard uncertainties, a correlation
matrix for the astrometric parameters: a correlation value is given in
dimensionless units (values in the range $[-1,1]$) for each pair of 
parameters. This matrix should be used for the error analysis when the
astrometric parameters are jointly used. For instance, the calculation of the
transverse spatial velocity of a star requires the use of its parallax (for
the distance) and the proper motions in right ascension and declination; 
therefore the three correlations between them will be needed for the
error analysis, since if the correlations are high the three uncertainties cannot 
be treated as being fully independent. If the correlations are not
included, the dispersion of velocities could be underestimated, for 
instance.

It is also important to note that in Gaia DR1, due to 
the limited timespan and number of observations, the values of these correlations
can be large. For instance, \figref{fig:pmra-pi-correlation}a shows the
histogram of the $\mu_{\alpha *}$ and $\varpi$ correlations in the TGAS dataset. It is
clear that the fraction of stars with high correlations is large. However, although 
this applies to most TGAS stars, the Hipparcos subset is strikingly different, 
\figref{fig:pmra-pi-correlation}b, as the precise first epoch Hipparcos 
positions allowed to better decouple the proper motion from the parallax.

 \begin{figure}
 	\centering
%   \includegraphics[width=0.49\columnwidth]{figures-statistics/TGAS/PmRa-Parallax-Correlation-TGAS.jpg}
%   \includegraphics[width=0.49\columnwidth]{figures-statistics/Hipparcos/AstrometricCorrelationStats_HistogramCorParallaxPmRa_HistogramSAM.png}
   \includegraphics[width=0.49\columnwidth,height=0.4\columnwidth]{figures-statistics/LargeLabels/TGAS/histogram_TGAS_Parallax_RA_ProperMotion_Correlation.png}
   \includegraphics[width=0.49\columnwidth,height=0.4\columnwidth]{figures-statistics/LargeLabels/HIPP/histogram_Hipparcos_Parallax_RA_ProperMotion_Correlation.png}
   \caption{Histogram of the correlations between proper motion in right ascension and parallax 
   for the whole TGAS dataset (left)
   and for its subset of Hipparcos stars alone (right).\label{fig:pmra-pi-correlation}}
 \end{figure}

% assuming that the usage of the covariance matrix has been put into light in the previous paragraph :
The usage of the {\GDR1} covariance matrix between parameters should however be done with some caution. All the tests done against external catalogues using the covariance matrix to compute the residuals $R_\chi$ indicate a much larger number of outliers than when using only each astrometric parameter normalised residuals independently. The abnormally high values of $R_\chi$ can be seen in  \figref{fig:cu9val_wp944_hipPMchi2}a for the Hipparcos catalogue and they \beforeReferee{are most probably explaining} \afterReferee{most probably explain} the bright Gaia sources mis-match with UCAC4 (\figref{fig:wp944_ucac4}) as well as the high number of LMC member stars removed by a $\chi^2$ test. 
Moreover, a strong increase of the $R_\chi$ residual for bright sources has been seen on the Hipparcos proper motions (\figref{fig:cu9val_wp944_hipPMchi2}b) as well as on the ICRF2 QSO positions. This indicates that a censorship using the covariance matrix will induce a censorship on the magnitude too.

And again, beside the correlations between astrometric parameters, there are also
correlations between stars which produce systematics at small 
scales (\secref{sec:summary-astro}).

%----------------------------------------
\subsection{Censorships and truncations, completeness}
%----------------------------------------

As discussed in \secref{sec:completeness}, {\GDR1} is incomplete in several ways. There are global effects, small scale effects and effects related to crowding, angular separation, brightness, colour and position that make the incompleteness of the catalogue very difficult to describe. For this reason the use of {\GDR1} for star count analysis, although not impossible, should be done with great care. Specially in small fields the complex features of the completeness caused by under-scanning and lack of on-board resources, as depicted in \figref{fig:942_holes} should be taken into account.

%----------------------------------------
\subsection{Data transformation and error distributions}
%----------------------------------------

Besides the above described limitations due to the characteristics
of {\GDR1}, related to its preliminary nature, we want to conclude 
this paper with a warning to the user about potential biases
introduced by the use of transformed quantities. We will not discuss
this issue in full since it is not the goal of this text, and 
we rather refer the reader to other texts.

First of all, the TGAS dataset in {\GDR1} provides an unprecedented
set of stellar parallaxes, more than two million. But most frequently 
the users of these data will rather be interested in obtaining 
stellar distances from the parallaxes, and the first obvious idea
will be just to apply the well known relation 

$d = \frac{1}{\varpi}$

\noindent where $d$ is the distance in parsecs and $\varpi$ is the parallax
in arcseconds. Although this relation is formally true, the presence
of observational errors complicates its use for the {\it estimation}
of distances from parallaxes. Notice that we use on purpose the word {\it
estimation} because in practice this is the most we can do to obtain a distance
from a parallax: build an estimator. Due to the observational error
the observed parallax will be a value {\it around} the true parallax,
determined by some statistical distribution describing the error. In
the case of Gaia this distribution is almost gaussian, its width given
by the standard uncertainties in the catalogue and centered (unbiased) in the
true value within the limits of the systematics described in previous
sections. 

A discussion on how to use the observed parallaxes, understood as
these realisations of the error distributions was already presented
at the time of the release of the Hipparcos catalogue in \citet{1997IAUJD..14E...5B}
and a further discussion can be found in \citet{1999ASPC..167...13A}. 
We refer the reader to these papers, which warn about
the truncation of samples based on the relative parallax error and 
the bias in the estimated distances if one just naively inverts the
observed parallaxes.

Solving these problems is not obvious. Simple procedures
can help to some extent, for instance never average distances
obtained from inverting observed parallaxes, but rather first average
the parallaxes and then invert the result -- see \citet{1999ASPC..167...13A} --.
But a proper solution would require a careful analysis of the 
problem in hand to define an unbiased estimator of the distances
needed, for instance using a Bayesian estimator. We refer the reader
to \citet{2015PASP..127..994B} for a discussion of this kind of
methods. Beside distances, another application of parallaxes is
the computation of an absolute magnitude; here again, the formal
expression $M_G = m_G - 10 + 5 \log(\varpi) - A_G$ has to face
the non-linear use of parallaxes having an observational error.

Beyond the problems with the use of trigonometric parallaxes discussed
in the papers cited above we also want to add a word of warning 
about the comparison of the Gaia DR1 parallaxes with parallaxes from
other sources. In this case the properties of the error distribution
in {\em each} catalogue, and their combined effect, should be properly taken
into account when drawing conclusions about the comparison. We will illustrate
this with a couple of examples. First, to compare the Hipparcos and
TGAS parallaxes one can draw a plot of the differences between them versus
the Hipparcos parallaxes. The result can be seen in \figref{fig:pi-Hip-Tgas}a,
and to the unaware reader this figure can suggest a strong systematic difference
between the two sets for small values of the parallax $\varpi<2$ mas.
However, such a behaviour is just what one can expect when drawing this
figure when the two sets of parallaxes have significantly different values
of the uncertainties. Figure \ref{fig:pi-Hip-Tgas}b shows this using simulated data. 
Starting from a set of error-free (simulated) parallaxes imitating
the distribution of the dataset used in the previous figure, two sets of 
parallaxes were generated: one with uncertainties around 1~mas (Hipparcos-like) 
and another one with uncertainties around 0.3~mas (TGAS-like). As can be seen
in the figure, in spite of the simulation being completely bias-free
and therefore without any systematic difference between the two sets
of parallaxes, the figure is similar to the one from real Hipparcos data
and could (wrongly) suggest the presence of systematic effects in one
or another catalogue. In fact, the asymmetric top-tail in these figures
is just an effect of the longer tail of negative parallaxes in the 
Hipparcos data when compared with the TGAS data.

 \begin{figure}
 \centering
   \includegraphics[width=0.49\columnwidth]{figures-conclusions/Pi-Hip-Tgas.jpg}
   \includegraphics[width=0.49\columnwidth]{figures-conclusions/DifPi-Hip.png}
   \caption{TGAS minus Hipparcos parallaxes vs Hipparcos parallaxes, source: L. Lindegren (left).
   Simulation based on completely unbiased sets of Hipparcos-like and TGAS-like 
   parallaxes (right). \label{fig:pi-Hip-Tgas}}
 \end{figure}

% \begin{figure}
% \centering
%   \includegraphics[width=0.8\columnwidth]{figures-conclusions/DifPi-Hip.png}
%   \caption{Simulation of \figref{fig:pi-Hip-Tgas} based on completely unbiased
%            sets of Hipparcos-like and TGAS-like parallaxes. Notice that
%            in spite of the complete absence of biases and therefore any
%            systematic difference, for small values of parallax the 
%            distribution of the figure is not symmetric. \label{fig:DifPi-Hip}}
% \end{figure}

A second example of such effects deriving from the error distributions in the 
parallaxes is present when comparing trigonometric parallaxes versus photometric
or spectroscopic parallaxes. In this case the effect does not come from the
different magnitudes of the errors but from their different distributions, the first
ones being gaussian and the second one (derived from magnitudes or spectra)
being log-normally distributed. Figure \ref{fig:DifPi-Phot} shows another simulation
illustrating this effect. Starting from a set of error-free (simulated) parallaxes 
two sets of parallaxes were generated: one with log-normal errors (photometric-like) 
and another one with normal errors, in both cases with a standard deviation of 0.3~mas. 
Again, the figure could suggest to the unaware reader a systematic effect, making
the TGAS parallaxes smaller than the photometric ones, specially for large parallaxes
(short distances). The linear fit (red line)
added to the figure stresses this effect. However, as stated, the simulation is
completely bias-free and therefore this effect comes purely from the properties
of the error distributions of the two datasets
and the complete (anti)correlation between abscissa and ordinate
\citep[see also][Fig. 4]{1999ASPC..167...13A}.

 \begin{figure}
 \centering
   \includegraphics[width=0.8\columnwidth]{figures-conclusions/DifPi-Phot.png}
   \caption{Simulation comparing photometric parallaxes with TGAS-like parallaxes. Notice that
            in spite of the complete absence of biases and therefore any
            systematic difference, there is an apparent systematic difference
            between the two datasets, specially for large parallaxes. \label{fig:DifPi-Phot}}
 \end{figure}

The discussions presented above about the proper use of the parallaxes
also extend to the case of the $G$ magnitude contained
in Gaia DR1. The archive does not contain, on purpose, standard uncertainties for
these magnitudes. Instead, errors are given for the fluxes from which
these magnitudes are obtained, along with the fluxes themselves. The problem
in this case is again that the obtention of the desired quantity, 
the magnitude $m$, from the observed quantity, the observed flux
$F$ is non-linear:

$m = -2.5 \log(F) + C_0$

\noindent where $C_0$ is the zero point of the photometric band.
As in the case of the parallax this non-linearity will introduce
biases if not properly taken into account, although in this case
the effect is less severe because the relative errors are smaller.
%A discussion on the proper estimation of errors for photometric
%data can be found in {\bf [TODO cite Floor's photometry paper]}. Not sure it will be! 

We recall here however that the flux uncertainty provided in {\GDR1} corresponds to the observed scatter which can be much lower than the systematics and \beforeReferee{is therefore not}\afterReferee{may therefore not be} fully representative of the actual uncertainties, especially for bright stars. 




%----------------------------------------
\subsection{Conclusion}
%----------------------------------------

At the end of this paper, it is needed to recall that the validation, 
by its very nature, has insisted more on the various problems found rather 
than on the intrinsic quality of the Catalogue. The summary about 
the Catalogue completeness can be found in \secref{sec:summary-content},
what was found about astrometry in \secref{sec:summary-astro}, and conclusions 
about photometry are given in \secref{sec:summary-photo}.

It must nevertheless be underlined that the {\GDR1} represents 
a major breakthrough since the Hipparcos Catalogue
on the direct measurement of the solar neighbourhood. With $20\times$ 
more stars than Hipparcos, and a median precision $3\times$ better,
it will provide new basis for studies on stellar physics and galactic
structure, provided the limitations shown above are accounted for.

With the promise of soon being superseded by the {\gaia} DR2 data,
{\GDR1} proves the ESA cornerstone mission concept, the good health of the 
instruments, the capabilities of the on-ground reconstruction, 
and the strong dedication of the community members involved in the project.


