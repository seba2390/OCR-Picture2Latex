
%+++++++++++++++++++++++++++++++++++++++++++++++++++++++++++++++++++++++++++
\section{Astrometric quality of {\GDR1}}\label{astroqual}
%+++++++++++++++++++++++++++++++++++++++++++++++++++++++++++++++++++++++++++

For the majority of the sources included in {\GDR1}, the 1\,140\,622\,719 {\it secondary sources}, the only available astrometric parameter is the position. 
For the 2\,057\,050 {\it primary sources}, the TGAS subset, the complete set of astrometric parameters is available: position, trigonometric parallax and proper motion. 
As a consequence, most tests concerning astrometry have been devoted to TGAS validation and only \secref{sec:DR1pos} deals with tests on secondary sources astrometry.

We study in \secref{sec:accuracy-plx} the accuracy of the TGAS parallaxes, and in \secref{sec:precision-plx} their precision. 
In both cases, we discuss first the estimation done using internal ({\gaia} only) data, then with external data.
Table \ref{tab:cu9val_wp944_summaryplx} gives a summary of the difference between the TGAS parallaxes and those from external catalogues that are presented in this section. 

\begin{table*}
\caption{Summary of the comparison between the TGAS parallaxes and the external catalogues. The number of outliers (at 5$\sigma$) versus the total number of stars is presented. The parallax difference ($\varpi_G-\varpi_E$, in mas) and the extra uncertainty (in mas) that needs to be quadratically added or subtracted to the data to adjust the residuals are indicated in red [green] when they are [not] significant (p-value limit: 0.01). A global estimate of the parallax offset as given by the weighted average of these various tests is $-0.036\pm 0.002$ mas, very similar to the estimate found using quasars, and the median of the extra dispersion is $-0.14\pm 0.08$ mas.
\label{tab:cu9val_wp944_summaryplx}}
\centering
\begin{tabular}{lccc} 
\hline\hline
{\bf Catalogue} & {\bf Outliers} & {\bf \parallax~difference} & {\bf \parallax~extra dispersion} \\
 \hline
 Hipparcos & 0.09\% & \textcolor{red}{$-0.094 \pm 0.004$} & \textcolor{red}{$0.58\afterReferee{0} \pm 0.005$} \\
  \hline
 VLBI & 0 / 9 & \textcolor{green}{$0.083 \pm 0.12$} & - \\
 HST & 2 / 19 & \textcolor{green}{$-0.11 \pm 0.19$} & \textcolor{green}{$0.6 \pm 0.2$} \\
 RECONS & 0 / 13 & \textcolor{green}{$-1.04 \pm 0.58$} & \textcolor{green}{$-0.9 \pm 0.5$} \\
 VLBI \& HST \& RECONS & 2 / 41 &  \textcolor{green}{$-0.08 \pm 0.12$} & \textcolor{green}{$0.42 \pm 0.13$} \\
 \hline
 Cepheids & 0 / 207 & \textcolor{green}{$-0.014 \pm 0.014$} & \textcolor{red}{$-0.18 \pm 0.01$} \\
 RRLyrae & 0 / 130 & \textcolor{red}{$-0.07 \pm 0.02$} & \afterReferee{\textcolor{red}{$-0.16 \pm 0.02$}} \\
 Cepheids \& RRLyrae & 0 / 337 & \textcolor{red}{$-0.034 \pm 0.012$} & \textcolor{red}{$-0.17 \pm 0.01$} \\
 \hline
 RAVE & 47 / 5144 & \textcolor{red}{$0.07\afterReferee{0} \pm 0.005$} & \textcolor{green}{$-0.06 \pm 0.02$} \\
 APOGEE & 0 / 2505 & \textcolor{red}{$-0.06\afterReferee{0} \pm 0.006$} & \textcolor{red}{$-0.12 \pm 0.01$} \\
 LAMOST & 6 / 317 & \textcolor{green}{$-0.01 \pm 0.02$} & \textcolor{red}{$-0.17 \pm 0.02$} \\
 PASTEL & 1 / 218 & \textcolor{green}{$0.05 \pm 0.02$} & \textcolor{green}{$0.1\afterReferee{0} \pm 0.05$} \\
 APOKASC & 0 / 969 & \textcolor{red}{$-0.07\afterReferee{0} \pm 0.009$} & \textcolor{red}{$-0.15 \pm 0.01$} \\
 \hline
 LMC & 2 / 142 & \textcolor{red}{$0.11 \pm 0.02$} &  \textcolor{green}{$-0.14 \pm 0.03$} \\
 SMC & 0 / 58 & \textcolor{red}{$-0.12 \pm 0.05$} &  \textcolor{green}{$-0.09 \pm 0.09$} \\
  \hline
 ICRF2 QSO auxiliary solution & 1 / 2060 & \textcolor{red}{$-0.046 \pm 0.01\afterReferee{0}$} & \textcolor{red}{$-0.17 \pm 0.01$} \\
 \hline
\end{tabular}
\end{table*}


%----------------------------------------
\subsection{TGAS Parallax accuracy}\label{sec:accuracy-plx}
%----------------------------------------

\subsubsection{Parallax accuracy using quasars}\label{sec:QSO-plx}
In the course of the AGIS astrometric solution, about 135\,000 quasars were included
and solved for parallax and positions, with proper motions being constrained with 
a prior near zero{\masyr} \citep[][Sect. 4.2]{2016A&A...586A..26M,DPACP-14} and made available
for validation (and are not part of {\GDR1}).
As the true parallax for quasars can be considered as null, the study of these parallaxes gives a direct 
information on the properties of the parallax errors. Unfortunately, the available
quasars cover part of the sky only, and in particular they can give little insight
inside the galactic plane. 

The median zero-point of the quasar parallaxes is significantly non-zero: $-0.04\afterReferee{0}\pm 0.003$ mas. 
This is close to the value for the ICRF2 QSO subsample, see \tabref{tab:cu9val_wp944_summaryplx},
and corroborated by other all sky external comparisons in this table and discussed in more details below,
and this is what we adopt as average {\GDR1} parallax zero-point. 

We selected random sky regions with 2{\degr} radius, keeping only those possessing at least 20 quasars,
and computed median parallaxes in these regions. 
The map of the median parallaxes in these regions is represented 
\figref{fig:cu9val_942_QSOKsky}. \afterReferee{Outside of the galactic plane
where the lack of objects (see \figref{fig:cu9val_942_QSOrhoplx}) brings little information, 
there are} large scale spatial effects with characteristic 
amplitude of about 0.3 mas (significant at $2\sigma$). In a few (exceptional) small regions, the 
parallax bias may even reach the mas level. 

\begin{figure}
\begin{center}
\columnImage{{figures-942/TGAS_01.00-ecl-qso-reg-MedPlx-r2}.png}
\caption{\afterReferee{Median parallaxes of quasars in 2{\deg} radius regions (mas), ecliptic coordinates. 
There is little insight in the galactic plane, due to the lack of objects.}
\beforeReferee{Due to the lack of objects (see \figref{fig:cu9val_942_QSOrhoplx}), the galactic plane 
bring few information.} Outside of it, local systematics
with about 0.3 mas characteristic amplitude can be seen.}\label{fig:cu9val_942_QSOKsky}
\end{center}
\end{figure}


\begin{figure}
\begin{center}
\includegraphics[width=0.49\columnwidth, height=0.45\columnwidth]{{figures-942/MatchedObservations-Parallax_zoom_}.png}
\includegraphics[width=0.49\columnwidth, height=0.45\columnwidth]{{figures-942/RaParallaxCorr-Parallax_zoom_}.png}
\caption{Median quasar parallaxes (mas) vs number of observations (left)
and vs correlation between right ascension and parallax (right).}\label{fig:cu9val_942_QSOrhoplxsky}
\end{center}
\end{figure}


\begin{figure}
\begin{center}
\includegraphics[width=0.49\columnwidth, height=0.35\columnwidth]{{figures-942/healpix_Quasars_Number_of_Matched_Observations}.png}
\includegraphics[width=0.49\columnwidth, height=0.35\columnwidth]{{figures-942/healpix_Quasars_RA_and_Parallax_Correlation}.png}
\caption{Healpix map in ecliptic coordinates of the number of quasar observations (left)
and of the correlation between right ascension and parallax (right).}\label{fig:cu9val_942_QSOrhoplx}
\end{center}
\end{figure}


\begin{figure}
\begin{center}
\includegraphics[width=0.49\columnwidth, height=0.45\columnwidth]{figures-942/ScanDirectionStrengthK1-Parallax_zoom_.png}
\includegraphics[width=0.49\columnwidth, height=0.45\columnwidth]{figures-942/ScanDirectionStrengthK4-Parallax_zoom_.png}
\caption{Median quasar parallaxes (mas) vs 
scan direction strength K1 (left) and vs K4 (right).}\label{fig:cu9val_942_QSOK1K4}
\end{center}
\end{figure}

The bias variations are directly related to the number of 
measurements (\figref{fig:cu9val_942_QSOrhoplxsky}a, \ref{fig:cu9val_942_QSOrhoplx}a), 
and consequently to the standard uncertainties,
with also a 0.3 mas amplitude. Parallax biases look also related to the  
correlations between right ascension and parallax
(\figref{fig:cu9val_942_QSOrhoplxsky}b, \ref{fig:cu9val_942_QSOrhoplx}b).
In \figref{fig:cu9val_942_QSOKsky} and \figref{fig:cu9val_942_QSOrhoplx}, the regions along 
$\lambda\sim 0$ and 180{\deg} (ecliptic pole scanning law) appear clearly. 

As for the origins of these systematics, possible along-scan measurements 
problems, if \dt{scan\_direction\_strength\_k1}\footnote{The 
``scan direction strength'' fields in the Catalogue quantify the 
distribution of AL scan directions across the source and \dt{scan\_direction\_strength\_k1} 
is the degree of concentration when the sense of direction is taken into account; as for
\dt{scan\_direction\_strength\_k4}, a value near 1 indicates that the scans are concentrated in two
nearly orthogonal directions.} 
is a proxy for this, may be part of the reason (\figref{fig:cu9val_942_QSOK1K4}a), 
with some contribution from possible chromaticity problems. 
The \dt{scan\_direction\_strength\_k4}, associated to small 
numbers of observations,  \afterReferee{also} looks \beforeReferee{also} contributing (\figref{fig:cu9val_942_QSOK1K4}b) with 
here again a 0.3 mas amplitude. 

\afterReferee{It is important to stress that the map illustrating spatial variations of the 
parallax bias of the quasars, \figref{fig:cu9val_942_QSOKsky}, cannot be used to ``correct''
the parallaxes. The quasars are faint, and the TGAS parallaxes, which were 
obtained with a different astrometric solution, may
suffer from supplementary effects due to their bright magnitudes.}

\subsubsection{Parallax accuracy tested with very distant stars}\label{sec:LMCSMC}
%
The zero point of the parallaxes and their precision can also be tested directly by using stars in TGAS (or quasars, see previous subsection) distant enough so that their measured parallaxes can be considered as null according to the catalogue's expected precision. The normalized parallax distribution of those sources should follow a standard normal distribution. For TGAS we have been looking for stars with $\varpi<0.1$~mas. This limit has been chosen to be consistent with TGAS precision (estimated to be of the order of a few tenths of mas). For \GDR1, only the Magellanic Clouds contain enough confirmed members in TGAS for this test. 

 \paragraph{LMC/SMC.}
 A catalogue containing 250 LMC and 79 SMC Tycho-2 stars has been compiled from the literature: Hipparcos 
\citep[Annex 4 of][]{1992ESASP1136.....T}, \cite{1989ESASP1111B.191P}, \cite{2008AcA....58..163S}, \cite{2009AJ....138.1003B}, \cite{2009ApJS..184..172G}, \cite{2012ApJ...749..177N} for the LMC; Hipparcos \citep[Annex 4 of][]{1992ESASP1136.....T}, \cite{1989ESASP1111B.191P}, \cite{2010AcA....60...91S}, \cite{2004MNRAS.353..601E}, \cite{2010AJ....140..416B}, \cite{2010ApJ...719.1784N} for the SMC. 
 For the 46 Hipparcos stars included, the Hipparcos and Simbad information has been confirmed to be fully consistent with LMC/SMC membership.  
 % TODO more about the references? 
 
A mean parallax of 0.11$\pm$0.02~mas has been found for the LMC and -0.12$\pm$0.05~mas for the SMC with a small over-estimation (by 0.14~mas) of the uncertainties. None of these values is consistent with the all-sky zero-point and this indicates local variations of the parallax zero point across the sky, confirming the spatial variations found \secref{sec:QSO-plx}. Further filtering of the sources has been done by comparing the parallaxes and proper motions of the stars with the mean values of the clouds (taken from SIMBAD) through a $\chi^2$ test. Using a limit p-value of 0.01 on this $\chi^2$ test removes 20\% of the LMC stars (3\% of the SMC). The remaining stars still show a significant parallax bias although reduced as expected. A correlation of the parallax residual with magnitude is observed in all cases (with a larger residual for the brighter stars). This dependency with magnitude and the surprisingly large number of outliers indicated by the $\chi^2$ test are similar to the Hipparcos $\chi^2$ test results (Section \ref{sec:wp944_astrom}), suggesting that a filtering based on the covariance matrix is actually hiding Gaia related issues rather than LMC/SMC membership issues. 

\subsubsection{Parallax accuracy tested with distant stars}\label{sec:distantstars}

An estimation of the parallax accuracy can also be obtained with stars distant enough so that their estimated distance through period-luminosity relation or spectrophotometry is known with a precision better than $\sigma_{\varpi_\mathrm{E}}<0.1$~mas, i.e. much more precise than the TGAS parallaxes. A maximum likelihood method \citep[improved from][Sect. 4]{1995A&A...304...52A} has been implemented to estimate the offset and extra-dispersion that should be \beforeReferee{take}\afterReferee{taken} into account for the Gaia parallaxes to be consistent with these external distance estimates. 

Two catalogues have been tested using the period-luminosity relation: 

% \paragraph{Distant Cepheids and RR Lyrae.}
\paragraph{Cepheids.} The catalogue of  \cite{2012ApJ...747...50N} has been used. It provides distance modulus for the Cepheids using the Wesenheit function. The error on the distance modulus has been estimated by adding quadratically the dispersion around the Wesenheit function, the uncertainty on the distance modulus of the LMC used to calibrate this relation, the $I$-magnitude error and the overall dispersion seen by \citet{2012ApJ...747...50N} when comparing their distance modulus to other methods (0.2~mag). The latter was needed in order the get distance moduli consistent with the Hipparcos parallaxes. 
  The catalogue contains 233 Tycho-2 stars with $\sigma_{\varpi_\mathrm{E}}<0.1$~mas. 
  
\paragraph{RRLyrae.} For TGAS we used the catalogue of \cite{2005A&A...442..381M}. We computed the distance modulus using the magnitude independent of extinction \Kjk = $K - \frac{A_K}{A_J - A_K} (J-K)$. The extinction coefficients were computed applying the \cite{FitzpatrickMassa07} extinction curve on the \citet{CastelliKurucz03} SEDs. \MK\ was derived from the period-luminosity relation of \cite{2015ApJ...807..127M} \afterReferee{(assuming a mean metallicity of -1.0~dex with a dispersion of 0.2)} and the colours were derived from \cite{2004ApJS..154..633C} transformed in the 2MASS system using the transformations of \cite{2001AJ....121.2851C}. The catalogue contains 150 Tycho-2 stars with $\sigma_{\varpi_\mathrm{E}}<0.1$~mas.
  
A parallax offset of $-0.034 \pm 0.012$~mas and a small overestimation of the standard uncertainty are significative when the Cepheids and the RR Lyrae samples are combined (Table~\ref{tab:cu9val_wp944_summaryplx}). \\

\begin{figure}
\begin{center}
\includegraphics[width=0.7\columnwidth]{figures-945/tgas-rave-compare-full-1.pdf}
\caption{Distribution of $\varpi_\mathrm{TGAS}/\varpi_\mathrm{RAVE} - 1$ for $\sim 200\,000$ stars matched in the
RAVE catalogue to the TGAS solution. Stars along EPSL, $\lambda \sim 180${\deg}, 
appear to have a systematically overestimated parallax by
up to $\sim 0.3$~mas, with stars with G magnitudes in the range $10 - 11.5$
and colour $1.4 \le ${\bprp}$ \le 1.8$ being the most strongly affected.}
\label{fig:rave2tgas-wp945}
\end{center}
\end{figure}

%\subsubsection{Parallax accuracy using spectrophotometric parallaxes}
For the following catalogues, spectrophotometric distance moduli have been collected or computed. 
\paragraph{RAVE} \citep{2013AJ....146..134K} with distances from \citet{Binney2014MNRAS.437..351B}. 
% WARNING: here we mix 2 WP results, although the xmatch has been done differently, so no detail on the xmatch is done ;-p 
It contains 6850 Tycho-2 stars with $\sigma_{\varpi_\mathrm{E}}<0.1$~mas. A comparison with Hipparcos has shown the presence of 24\% of outliers, mainly due to dwarf/giant mis-classifications. 
Strong outliers are also seen in the comparison with TGAS but they represent only 1\% of the sample. A global parallax offset of 0.07\afterReferee{0}$\pm$0.005~mas is seen with a strong variation with sky position (with 0.3~mas amplitude). 
This is the only catalogue, together with the LMC, that present a significant positive parallax bias (Table \ref{tab:cu9val_wp944_summaryplx}). 
To further study the presence of systematic effects in localized regions on the sky that could affect the RAVE results, another test has been made using this time all the $192\,655$ stars in common between TGAS and RAVE.  
Thanks to their extended sky coverage, we could identify a
systematic difference in the parallaxes in the region with ecliptic
coordinates $\lambda \sim 180${\deg}, as shown in
\figref{fig:rave2tgas-wp945}. The amplitude of this effect is of
order $\sim 0.3$~mas and affects more strongly the fainter and redder
TGAS stars. It appears that this effect is directly correlated with the 
number of observations along-scan (\dt{astrometric\_n\_obs\_al} parameter) 
and the ecliptic scanning law followed early in the mission\afterReferee{, 
and is consistent with the spatial biases found with quasars at \secref{sec:accuracy-plx}}.


\paragraph{APOGEE DR12} \citep{2015AJ....150..148H}. Distance moduli were computed using a Bayesian method on the Padova isochrones \citep[CMD 2.7]{Bressan12} and using the magnitude independent of extinction \Kjk. The prior on the mass distribution used the IMF of \cite{Chabrier01} while the prior on age was chosen flat. Stars too far from the isochrones were rejected using the $\chi^2_{0.99}$ criterion. It led to 3100 Tycho stars with $\sigma_{\varpi_\mathrm{E}}<0.1$~mas. 
  %results
  A global parallax difference of $-0.06\afterReferee{0}\pm 0.006$~mas was found, with a strong variation with magnitude, the brighter the larger the difference. 
  
\paragraph{LAMOST DR1} \citep{LamostDR1}. Same method as for APOGEE. It leads to 451 stars with $\sigma_{\varpi_\mathrm{E}}<0.1$~mas. 
  % results
  No significant parallax difference was detected with this sample. 
 
\paragraph{PASTEL} \citep{2016A&A...591A.118S}. Same method as for APOGEE. It leads to 917 Tycho stars with $\sigma_{\varpi_\mathrm{E}}<0.1$~mas. 
  %results 
  No significant parallax difference was found except for the blue stars ($J$-\Ks$<$0.3), with a difference up to 0.3~mas, most probably linked to the spectro-photometric distance determination that has been less tested on those young massive stars and is more dependent on the age prior. Therefore only stars with $J$-\Ks$>$0.3 are used in the summary \tabref{tab:cu9val_wp944_summaryplx}. 
 
\paragraph{APOKASC} using the distances provided by \cite{2014MNRAS.445.2758R} derived using both Kepler asteroseismologic and APOGEE spectroscopic parameters. It contains 984 Tycho sources with $\sigma_{\varpi_\mathrm{E}}<0.1$~mas. The median $\sigma_{\varpi_\mathrm{E}}$ of this catalogue is 0.02~mas. 
  %results
  A global parallax difference of $-0.07\afterReferee{0}\pm 0.009$~mas is seen, with a strong variation with magnitude, similar to what was found with the APOGEE results. Both use the Padova isochrones, have the Kepler region and its spectroscopic parameters in common, but the distance modulus for APOGEE has been computed by us and the APOKASC has a precision on its distance modulus much increased thanks to the usage of the asteroseismology parameters. The variation of the parallax difference with magnitude could come from a feature of the stellar evolution models. Both the APOKASC and APOGEE catalogues present a correlation between magnitude and colour, but in the APOKASC the brighter stars are bluer than the fainter stars (due to the extinction effect on the red clump population) while in APOGEE it is the opposite (due to the more evolved giants being redder); one therefore does not expect the colour to be able to explain the systematics we see in magnitude. 

%All the catalogues indicated here have been checked to pass the accuracy test on Hipparcos data, with the exception of APOKASC which contains only two Hipparcos stars and LAMOST DR1 which does not contain any Hipparcos star with a $\sigma_{\varpi}<0.1$~mas.  
% global results for the spectroscopic sample:
All those tests with TGAS show significant variations with sky position but with global parallax differences lower than 0.3~mas. 
These tests also show a small correlation with colour ($<$0.2~mas), but not all in the same direction nor with the same amplitude, indicating an expected bias linked to survey parameter correlations and/or stellar isochrones/priors. 




\subsubsection{Parallax accuracy tested using distant clusters}\label{sssec:cu9val_ocpar}

This test aims at assessing the internal consistency of parallaxes within a cluster, and checking the parallaxes against photometric distances in order to verify the zero-point of parallaxes. 

Sky coordinates, ages, extinctions and distances have been obtained for all clusters listed in the \citet{2014A&A...564A..79D} database \citep{1995ASSL..203..127M}. Making use of theoretical isochrones \citep{Bressan12}, we retained 488 clusters with an age/distance/extinction combination allowing them to contain stars reaching magnitude $V=11.5$ (the magnitude at which Tycho-2 becomes strongly incomplete). 

All stars within a radius corresponding to a distance of 3\,pc from the center of the cluster were searched, which means that the angular size of the queried field depends on the cluster distance.
Stars were selected based on their identifier in the Tycho-2 catalogue, avoiding \beforeReferee{binary}\afterReferee{double} stars flagged in \citet{2002A&A...384..180F}. When available, a preliminary knowledge of cluster membership was used, but the final cluster membership was determined from the TGAS data itself. The method used was that of \citet{1999A&A...345..471R}, which makes use of proper motions and parallaxes. 

We limited the statistics to clusters more distant than 1\,000\,pc 
so that the uncertainty of the photometric parallaxes is mostly better than the \beforeReferee{one}\afterReferee{uncertainty} of the {\GDR1} parallaxes.
 For every cluster, we computed the average difference $\Delta\mathbf{P}$ between the measured parallax of each star and the reference value (or \textit{photometric} parallax) $\varpi_\mathrm{ref}$ normalised by the uncertainty. 
 In order to compute those values, we need to take into account the  uncertainties on the parallaxes (i.e. $\sigma_\mathrm{\varpi,ref}$ on the reference value and $\sigma_{\varpi}$ on TGAS parallaxes) and the correlation among parameters of nearby stars. We note $S$=diag($\sigma_{i}$) the diagonal matrix made with the standard errors $\sigma_{i}$:


\begin{equation}
S=
\begin{pmatrix}
  \sigma_{\varpi,1} & 0 & ... & 0 \\
  0 & \sigma_{\varpi,2} & ... & 0 \\
  ... & ... & ... & ... \\
  0 & 0 & ... & \sigma_{\varpi,n} \\
 \end{pmatrix}
\end{equation}

\noindent
and we note $\mathbf{C}$ the correlation matrix, where $C_{ij}$ is the correlation coefficient between the parallaxes of star $i$ and star $j$, constructed as in \citet{2010IAUS..261..320H}. The matrix $\mathbf{\Sigma}=\mathbf{SCS}$ is the covariance matrix of $\mathbf{P}$. Noting $\mathbf{D}$ the design matrix $n$-vector (1,1,...,1), we can compute the mean parallax $\varpi= \sigma^2_{\varpi} (\mathbf{D}^T \mathbf{\Sigma}^{-1} \Delta\mathbf{P}) $ with $\sigma^2_{\varpi}=(\mathbf{D}^T \mathbf{\Sigma}^{-1} \mathbf{D})^{-1}$ the square of its standard error.

Once an average difference to the reference value ($\overline{\Delta\varpi}$) and associated error ($\sigma_{\overline{\Delta\varpi}}$) was established for each cluster, we studied the global distribution of $\Delta_\mathrm{off}$=$\overline{\Delta\varpi} / \sqrt{ \sigma^2_{\overline{\Delta\varpi}} + \sigma^2_{\varpi,ref}}$  which tells us by how many standard errors the average measured parallax differs from the reference parallax. In the absence of systematics, this distribution is expected to be centred on zero, with a dispersion of one sigma. A mean value differing from zero would indicate a global offset. Conservatively, we considered that all photometric distances listed in the \citet{2014A&A...564A..79D} database are affected by uncertainties of 20\%.  No significant global parallax offset was found, but an apparent systematic error varying with sky position (see \figref{fig:cu9val_947_hist_manual_pmean}). Most clusters with overestimated parallaxes appeared to be located in the Galactic regions with $l<200^{\circ}$ (towards the Galactic anticentre), while most of the underestimated parallaxes were at $l<200^{\circ}$ (see \figref{fig:WP947_VAL_010_030_Ocs_dst1000pc}). The parallax offsets were $-0.16\pm 0.04$ mas for $l>200^{\circ}$ and $+0.13\pm 0.04$ mas for $l<200^{\circ}$.


\begin{figure}
\centering
%\includegraphics[width=0.8\columnwidth]{figures-947/cu9val_947_hist_manual_pmean.pngcu9val_947_hist_pmean.pdf}
\includegraphics[width=0.8\columnwidth,height=0.5\columnwidth]{figures-947/cu9val_947_hist_pmean.pdf}
\caption{Distribution of the differences between the mean TGAS parallaxes and the one from photometric distance for the distant open clusters. 
\afterReferee{Red and blue labels are attributed to the clusters defined \figref{fig:WP947_VAL_010_030_Ocs_dst1000pc}}.} \label{fig:cu9val_947_hist_manual_pmean}
\end{figure}

\begin{figure}
 \begin{center}
\includegraphics[width=0.8\columnwidth,height=0.5\columnwidth]{figures-947/cu9val_947_010_030_Ocs_dst1000pc.png}
\end{center}
\caption{Sky distribution of open clusters more distant than 1\,000\,pc. The blue group appears to contain objects with underestimated parallaxes, while the red group contains overestimated parallaxes (\figref{fig:cu9val_947_hist_manual_pmean}).}\label{fig:WP947_VAL_010_030_Ocs_dst1000pc}
\end{figure}


We investigated the possibility that this effect could be caused by uncertainties in the automatic membership procedure applied. We manually inspected the results of the membership determination and discarded a certain number of clusters for which the cluster membership could not be securely established. The final statistics were computed for a sample of 38 distant clusters with secure membership determinations.
The median value of differences to the reference values for these 38 clusters is $+0.004\pm 0.02$ mas, confirming no obvious global parallax offset. Splitting the sample into two groups ($l>200^{\circ}$ and $l<200^{\circ}$), we find respectively a median of $-0.02\pm 0.032$ mas for the $l>200^{\circ}$ sample, and $+0.044\pm0.027$ mas for the $l<200^{\circ}$ sample, which does not show a significant difference. 

Unfortunately, the low number of tracers available in this experiment did not allow us to draw a map of the bias by averaging values in coordinate space. The slight variation in zero-point between the $l>200^{\circ}$ and $l<200^{\circ}$ groups can then be interpreted either as random variations caused by the uncertainties on the reference values, or as local variations of the parallax zero point (of the order of a few tenths of mas on a scale of several degrees).


