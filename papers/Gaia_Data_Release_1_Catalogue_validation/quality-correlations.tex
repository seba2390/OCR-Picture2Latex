\subsection{Quality indicators of the astrometric solution}\label{sec:quality_indic}

As mentioned before, the {\GDR1} astrometric solution applied only a single star
model to all stars; resolved doubles with small magnitude difference or astrometric 
binaries with noticeable orbital or acceleration motion are thus susceptible
to lead to a bad astrometric fit. Second, as described too, the adopted PSFs are not yet
optimal for all stars (and probably not for very blue or very red stars),
\beforeReferee{the attitude}\afterReferee{and the modelling of the satellite attitude}
can still be improved together with the geometric or CCD calibrations. There is
no {\em stricto sensu} goodness of fit metrics in the catalogue, as they would actually 
never be good given the caveat above. However there are both 
\dt{astrometric\_n\_bad\_obs\_al}, \dt{astrometric\_n\_\-bad\_\-obs\_ac} and 
\dt{astrometric\_excess\_noise} and its significance \dt{astrometric\_\-excess\_\-noise\_sig}.
Beside a median floor at about 0.5 mas due to attitude, etc, 
the \dt{astrometric\_excess\_noise} appears, 
as expected, sensitive to calibration problems for bright stars and extreme colours 
(\figref{fig:cu9val_942_excessnoise}). Outside these cases, and outside some regions
(see corresponding Figure in \secref{sec:statistics}), a star with a larger
and significant excess noise is a candidate to being non-single. It is thus
suggested to take advantage of these fields for the selection of ``cleaner'' samples.

\begin{figure}[]
\begin{center}
\smallImage{{figures-942/PhotGMeanMag-AstrometricExcessNoise}.png}
\smallImage{{figures-942/BP-RP-AstrometricExcessNoise}.png}
\caption{Astrometric excess noise (mas) smoothed as a function of $G$ magnitude (left) 
and {\gbp-\grp} (right).}\label{fig:cu9val_942_excessnoise}
\end{center}
\end{figure}





%----------------------------------------
\subsection{Summary of the astrometric validation}\label{sec:summary-astro}
%----------------------------------------

{\GDR1} is the most precise all sky astrometric survey since Hipparcos. 
And indeed, the quoted parallax precision in the catalogue appears correctly 
estimated, or slightly pessimistic only, as found by error deconvolution 
and when compared to external catalogues.
The only exception is the comparison with Hipparcos, which then points to some
underestimation of the errors in Hipparcos itself.

However, the preliminary character of the astrometric solution, and in particular
problems related to imperfect attitude or instrument modelling reveal systematic 
errors of the same order as the random errors.
A global negative parallax zero point (about -0.04 mas) is consistently 
found with many independent estimation methods (quasars, period-luminosity candles,
spectro/astero/photometric parallaxes).
This zero-point is however a consequence of large scale spatial variations
related to the scanning law that may reach at least a 0.3 mas amplitude 
(i.e. comparable to the median precision of stars in the catalogue). 
This is also consistently shown independently with quasars, LMC, SMC or RAVE data. 
In extreme cases, larger local biases may be expected.
Correlation with magnitude is also found towards the bright end.

For the scientific exploitation, the consequence of these systematics is that 
local parallax averages cannot be more precise than about 0.3 mas. 
Any study should take into account that any catalogue parallax is 
$\varpi \pm\sigma_\varpi \mathrm{~(rand.)} \pm 0.3 \mathrm{~(syst.)}$
And because the correlations between parallaxes and the other astrometric parameters 
is frequently very large, systematics must be present as well on the other 
astrometric parameters.

Another consequence of the presence of astrometric systematics is that all  
luminosity or kinematical calibrations must ensure that the 
\beforeReferee{stars are well}\afterReferee{star samples are evenly} distributed, which is in itself another issue, as completeness is difficult 
to ensure, cf. \secref{sec:summary-content}. 

Concerning proper motions, significant differences with Tycho-2 have been found 
which clearly originate from this catalogue, although some correlations 
with Gaia-only parameters may marginally also be interpreted as originating from 
Gaia, but this can only be to a much lesser extent. In particular, several  
components of close double systems have wrong astrometric solutions, due to incorrect 
cross-matches.

In TGAS as well as for the whole catalogue, the astrometric deficiencies
look related to bright stars and small number of observations.
There is no doubt that these problems will be resolved in the next Gaia data 
releases. 



