%+++++++++++++++++++++++++++++++++++++++++++++++++++++++++++++++++++++++++++
\section{Sky coverage and completeness of DR1\label{sec:completeness}}
%+++++++++++++++++++++++++++++++++++++++++++++++++++++++++++++++++++++++++++

The {\GDR1} release is expected to be incomplete in various ways, full detail of these limitations 
being described in \cite{DPACP-14,DPACP-8}: 
\begin{itemize}
\item {\GDR1} is based on 14 months of data only. As a result, some regions, especially at low ecliptic latitudes, have been poorly observed, both in terms of the number of observations and of the coverage in scanning directions, see for example Fig.~2 of \cite{DPACP-8}. Stars with less than 5 focal plane transits have been filtered out; 
\item stars with a low quality astrometry solution for whatever reason have been filtered out; 
\item bright stars or high proper motions stars may be missing; 
\item faint stars are missing in very dense areas (for stellar densities higher than $\sim$ 400\,000 stars per square degree at $G<20$); 
\item stars with extremely blue or red colours have been filtered out during the photometric calibration. 
\end{itemize}

The tests presented in this section aim at a better characterisation of the object content of DR1, including TGAS,
as for the homogeneity of the sky distribution and the small scale completeness of the Catalogue.
These tests have been performed from different points of view, for various populations 
and using various inputs and methods: using the characteristics of Gaia data only (internal tests), using external data (all sky external catalogues, detailed catalogues of specific samples of stars or of specific regions of the sky), or using Galaxy models.


%----------------------------------------
\subsection{Limiting magnitude\label{sec:faintlimit}}
%----------------------------------------

The completeness of {\GDR1} is the result of a complex interplay between high stellar densities implying a possible overlap of the images on the focal plane, scanning law defining the number of times a region was observed, and data processing. Due to limited telemetry resources, the star images sent to ground followed a decision algorithm which is a complex function of the magnitude. In addition, at the end of the data processing a filtering was applied to discard poor solutions both in the astrometry and in the photometry. As a result, the density distribution over the sky in the final Catalogue is not a simple function of the stellar density, as usually expected. 

A first, indirect information about the completeness is brought by the limiting magnitude 
of the Catalogue. Sky variations of the 0.99 quantile of the $G$ magnitude are
shown in \figref{fig:942_limitmag} for TGAS and the whole Catalogue. Concerning the latter,
it appears that Gaia will easily reach at the end of mission \beforeReferee{the} $G>21$  
in a significant fraction of the sky, even if this is still very limited for {\GDR1}; 
it seems however that one magnitude has been 
lost in the under-scanned regions, and two magnitudes in the Baade window. 
The limiting magnitude of TGAS stars \afterReferee{also has an amplitude of two magnitudes} over the sky,
with the brightest regions being also those with some astrometric deficiencies,
as shown below.

\begin{figure}
\centering
\includegraphics[width=0.7\columnwidth]{./figures-942/TGAS-G-Mag-Quantile-99.png}
\includegraphics[width=0.7\columnwidth]{./figures-942/GDR1-G-Mag-Quantile-99.png}
\caption{Limiting magnitude: 99\% percentile of the $G$ distribution in ecliptic coordinates: 
a) TGAS, b) full Catalogue.}
\label{fig:942_limitmag}
\end{figure}



%----------------------------------------
\subsection{Overall large scale coverage and completeness} \label{sec:sky-coverage}
%----------------------------------------

\subsubsection{Overall sky coverage and completeness of TGAS}

The overall TGAS content has been tested with respect to the Tycho-2 \citep{2000A&A...355L..27H} and Hipparcos Catalogues \citep{1997A&A...323L..49P,1997ESASP1200.....E} 
for detection of possible duplicate entries and characterisation of missing entries. 
TGAS contains 79\% of the Hipparcos and  80\% of the Tycho-2 stars. 
One of the reasons for the missing stars is a bad astrometric 
solution, as all sources with a parallax uncertainty above 1 mas were not kept 
in TGAS (validation tests done on preliminary data had indeed shown several problems
associated to these stars).
The sky distribution of the Tycho-2 sources not present in TGAS is presented \figref{fig:wp944_tgas_tycho2}, showing the impact of the Gaia scanning law (the number of observations and the orientation of the scans being correlated with the solution reliability criteria filters applied for {\GDR1}). 

\begin{figure}
    \begin{center}
        \includegraphics[width=0.8\columnwidth]{./figures-944/TGAS_tycho2_notfound.png}
        \caption[Tycho-2 stars not in TGAS]{Sky distribution of Tycho-2 stars not in TGAS, in galactic coordinates.}
        \label{fig:wp944_tgas_tycho2} 
    \end{center}
\end{figure}

The detail of the histogram of \figref{fig:942_tgashist} shows that stars fainter than
10.5~mag have suffered a higher loss than average, a likely reason is the occasional 
source duplication described in \secref{sec:wp942:dupes}, 
which affects these magnitudes more.
The loss is clearer for stars brighter than 6\,mag,
partly due to an insufficient number of bright calibration sources for the
broad band photometers, so no colour was available. 
The $G$ magnitude calibration
includes a colour term \citep{DPACP-9}, so a missing colour means that
no $G$-band photometry was produced, and the source did not enter the release.
Stars brighter than about 5, and a fraction of sources fainter than this, 
were also among the sources not kept in TGAS due to the bad quality of their 
astrometric solution.

TGAS completeness has also been tested with respect to high proper motion stars: a selection of 1\,098 high proper motion (HPM) stars has been made with SIMBAD on stars with a Tycho or HIP identifier and a proper motion larger than 0.5\,arcsec\,yr$^{-1}$ (proper motions mainly from Tycho-2 and Hipparcos). 
40\% of this selection is not found in the TGAS solution, in particular bright stars. All stars with a proper motion larger than 3.5 arcsec\,yr$^{-1}$ are absent from TGAS. 
Stars with a proper motion larger than 1~arcsec\,yr$^{-1}$ in TGAS have been confirmed to have a large proper motion in SIMBAD. 
%impossible to understand without indicating that the tests were checked on Hipparcos beforehand
%The validation of this test on Hipparcos results in the detection of one star, HIP 67593, with a very large proper motion in Hipparcos but with small proper motion in SIMBAD (source = Tycho-2), but this star is not in TGAS. 


\subsubsection{Overall sky coverage of {\GDR1} from external data.}
The overall sky coverage of {\GDR1} has been tested by comparison with two deeper all sky catalogues:  2MASS \citep{2006AJ....131.1163S} and UCAC4 \citep{2013AJ....145...44Z}. 
The tests performed here use the crossmatch between {\GDR1} and these two catalogues provided to the users in the Gaia Archive \citep{DPACP-17}. The variation over the sky of four key parameters are checked: the number of cross-matched sources, the mean number of neighbours (stars which could have been considered as cross-matched, but for which the cross-match was not as good as for the selected source = the {\it best neighbour}), the number of Gaia stars with the same {\it best neighbour}, and the number of Gaia sources without any match. Finally, a random subset of about 5 million sources has been selected in order to check, if any, the different properties in magnitude, colour, proper motion, goodness of fit, etc... of the above four categories of stars.

\begin{figure*}
    \begin{center}
        \includegraphics[width=0.33\textwidth]{./figures-944/UCAC4_notfound.png}
        \includegraphics[width=0.33\textwidth]{./figures-944/UCAC4_duplicated.png}
        \includegraphics[width=0.33\textwidth]{./figures-944/UCAC4_gaiaextra.png}
        \caption[Sky distribution versus UCAC4]{Sky distribution versus UCAC4, in galactic coordinates; a) UCAC4 sources not in Gaia {\GDR1} (5\%); b) UCAC4 sources with multiple matches in \GDR1; c) {\GDR1} sources with \gmag$<14$ not in UCAC4. }
        \label{fig:wp944_ucac4} 
    \end{center}
\end{figure*}

% results for UCAC4
\paragraph{UCAC4.} Only 5\% of the UCAC4 catalogue does not have a match in Gaia DR1. Their sky distribution (\figref{fig:wp944_ucac4}a) shows the footprint of the Gaia scanning law. 
7\% of the UCAC4 sources appear more than once in the cross-match table. 
We will refer to them as multiple-matches, it does not mean that this refer to (or only to) duplicate Gaia entries as discussed in \secref{sec:duplicates}: the Gaia resolution is much better than ground-based instruments so that multiple objects may appear where ground-based catalogues see one object only; those multiple-matches are distributed mainly in high density region, as expected, but their sky distribution also shows the Gaia scanning law footprint (\figref{fig:wp944_ucac4}b). 
258\,605 sources with \gmag$<14$ appear in the Gaia catalogue but not in UCAC4 which is supposed to be complete to about magnitude $R=16$; their sky distribution (\figref{fig:wp944_ucac4}c) follows the Gaia scanning law footprint and recalls the footprint of the Tycho-2 stars not in TGAS (\figref{fig:wp944_tgas_tycho2}). A detailed inspection of those sources indicates that a large portion of them are actually present in the UCAC4 catalogue but that the cross-match could not be done, the positional differences being beyond the astrometric uncertainties. This may be linked to the fact that a large portion of those sources have been measured along uneven scan orientations. % scanDirectionStrenghtK4 close to 1. 

% results for 2MASS
\paragraph{2MASS.} For this test, we selected 2MASS stars with photometric quality flag AAA and magnitude $J<14$ (this limit corresponds roughly to $V<20$ for \av$<5$). As expected, most of the missing sources are located in high extinction regions along the galactic plane, but some extra features are also apparent showing the Gaia scanning law footprint (\figref{fig:wp944_2mass}a).
The 2MASS multiple-matches have a sky pattern (\figref{fig:wp944_2mass}b) similar to the one observed with UCAC4, with the main concentration being as expected along the dense areas added to a smaller Gaia scanning law footprint. 

\begin{figure}
    \begin{center}
        \includegraphics[width=0.33\textwidth]{./figures-944/2mass_notfound.png}
        \includegraphics[width=0.33\textwidth]{./figures-944/2mass_duplicated.png}\\
        \caption[Sky distribution versus 2MASS]{Sky distribution versus 2MASS, in galactic coordinates. a) 2MASS sources with J$<$14 not in {\GDR1}; b) 2MASS multiple-matches in {\GDR1}.}
        \label{fig:wp944_2mass} 
    \end{center}
\end{figure}

\paragraph{Quasars.}
Quasars are essential objects for various reasons and several tests verify that they have been correctly observed by Gaia and identified. \beforeReferee{This}\afterReferee{The} first test compares {\GDR1} quasars with ground-based quasar compilations: GIQC \citep{GIQC}, LQAC3 \citep{2015A&A...583A..75S} and SDSS DR10 \citep{2014A&A...563A..54P} catalogues. It is a check for completeness, duplication and magnitude consistency. 
While the quasars were also affected by the duplicated sources issue (\secref{sec:duplicates}), the filtering seems to have removed them nicely. 
81\% of GIQC, 53\% of LQAC3 and 11\% of SDSS quasars are present in \GDR1, a ratio that reaches 93\% for the LQAC3 sources with a magnitude $B$ brighter than 20. 

\paragraph{Galaxies.} For galaxies, the cross-match has been done with SDSS DR12 \citep{2015ApJS..219...12A} sources with a galaxy spectral classification. The properties of cross-matched galaxies are compared to those of missing galaxies (magnitudes, redshift, axis-ratios and radii). Unfortunately, only $\sim$0.2\% of the SDSS galaxies are present in {\GDR1} due to the different filters applied. Still some large resolved galaxies can have multiple detections associated to them, tracing their shape. 

\subsubsection{Completeness from comparison with a Galaxy model}

\begin{figure*}
\begin{center}
\includegraphics[width=0.24\textwidth]{figures-943/DR1-result-GOG18/count-DR1-GOG18-reldif-12-13.pdf}
\includegraphics[width=0.24\textwidth]{figures-943/DR1-result-GOG18/count-DR1-GOG18-reldif-13-14.pdf}
\includegraphics[width=0.24\textwidth]{figures-943/DR1-result-GOG18/count-DR1-GOG18-reldif-14-15.pdf}
\includegraphics[width=0.24\textwidth]{figures-943/DR1-result-GOG18/count-DR1-GOG18-reldif-15-16.pdf}
\includegraphics[width=0.24\textwidth]{figures-943/DR1-result-GOG18/count-DR1-GOG18-reldif-16-17.pdf}
\includegraphics[width=0.24\textwidth]{figures-943/DR1-result-GOG18/count-DR1-GOG18-reldif-17-18.pdf}
\includegraphics[width=0.24\textwidth]{figures-943/DR1-result-GOG18/count-DR1-GOG18-reldif-18-19.pdf}
\includegraphics[width=0.24\textwidth]{figures-943/DR1-result-GOG18/count-DR1-GOG18-reldif-19-20.pdf}
\caption{Relative star count differences between {\GDR1} and GOG18 simulation in different magnitude bins, 
from $12<G<13$ to $19<G<20$ by step of one magnitude, in galactic coordinates. 
Beside the prominent feature of the Magellanic Clouds (absent from the Galaxy model), 
and inadequacies of the 3D extinction model in the galactic plane, 
the {\gaia} incompleteness around the ecliptic plane due to the scanning law starts 
clearly to appear from $G>16$.}
\label{fig:skymap}
\end{center}
\end{figure*}

\begin{figure*}
\begin{center}
%\includegraphics[width=0.25\textwidth,height=0.15\textwidth]{figures-943/DR1-result-GOG18/count-0-21.pdf}
\includegraphics[width=0.25\textwidth,height=0.15\textwidth]{figures-943/DR1-result-GOG18/count-90-0.pdf}
\includegraphics[width=0.25\textwidth,height=0.15\textwidth]{figures-943/DR1-result-GOG18/count-43-0.pdf}%\includegraphics[width=0.25\textwidth,height=0.15\textwidth]{figures-943/DR1-result-GOG18/count-180-0.pdf}
%\includegraphics[width=0.25\textwidth,height=0.15\textwidth]{figures-943/DR1-result-GOG18/count-313-0.pdf}
\includegraphics[width=0.25\textwidth,height=0.15\textwidth]{figures-943/DR1-result-GOG18/count-90-21.pdf}
%\includegraphics[width=0.25\textwidth,height=0.15\textwidth]{figures-943/DR1-result-GOG18/count-180-19.pdf}
%\includegraphics[width=0.25\textwidth,height=0.15\textwidth]{figures-943/DR1-result-GOG18/count-SGP.pdf}
\includegraphics[width=0.25\textwidth,height=0.15\textwidth]{figures-943/DR1-result-GOG18/count-45--45.pdf}
\includegraphics[width=0.25\textwidth,height=0.15\textwidth]{figures-943/DR1-result-GOG18/count-225-45.pdf}
\includegraphics[width=0.25\textwidth,height=0.15\textwidth]{figures-943/DR1-result-GOG18/count-NGP.pdf}
\caption{Star counts per square degree as a function of magnitude in several directions. Open circles linked with red lines are for {\GDR1} data, filled blue diamonds are simulations from GOG18. Error bars represent the Poisson noise for one square degree field. The bottom row shows regions impacted by the scanning law and the filtering of stars with a low number of observations.}
\label{fig:histograms}
\end{center}
\end{figure*}


Since {\GDR1} only contains $G$ magnitudes and positions, the validation with models consists in the comparison between the distribution of star densities over the sky and a realisation of the Besan\c{c}on Galactic Model \citep[BGM,][]{bgm}, hereafter version 18 of the Gaia Object Generator ~\citep[GOG18, ][]{2014A&A...566A.119L}. The simulation contains 2 billion stars including single stars and multiple systems, and incorporates a model for the expected errors on {\gaia} photometric and astrometric parameters. % after the full 5 year Gaia mission. %A complete description of the models is given in Appendix \ref{app:models}. 

In the validation process, star counts as a function of positions and in magnitude bins have been compared with the model (\figref{fig:skymap}). Systematic differences in Galactic plane fields are mostly due to 3D extinction model problems, but could also be due to other inadequacies of the model (such as local clumps not taken into account in a smooth model). These systematics are seen even in bright magnitude bins. On the other hand, differences at intermediate latitudes in the region of the Magellanic Clouds are not to be considered because these galaxies have not been included in this GOG catalogue. 
There is no other strong difference between data and model that could warn about the quality of the data at magnitudes brighter than 16. However at fainter magnitudes, some regions have significantly less stars than expected from the model. These regions are located specifically around $l=200-250${\deg}, $b=30-60${\deg} and $l=30-80${\deg}, $b=-60;-30${\deg}. At magnitudes fainter than 19, regions all along the ecliptic suffer from this smaller number of sources due to the scanning law and the filtering of objects with a too low number of observations. 
Also at $G>16$ some discrepancies appear in the outer bulge regions, which might be due to incompleteness of the data when the field is crowded (see \secref{sec:underdensity} and \figref{fig:942_holes}).

% in the release due to the scanning law. We confirm what is seen by comparison with ground based catalogues, that {\GDR1} is impacted by the filtering of objects having too low number of observations, and that this is visible in specific regions of the sky.

To estimate in more details the completeness in specific fields, we compared histograms of star counts from {\GDR1} and the GOG18 simulation as a function of magnitude. Figure~\ref{fig:histograms} shows such histograms in some regions of the galactic plane, at intermediate latitudes and at the Galactic poles.  
In the Galactic plane (\figref{fig:histograms}a) the star counts show a drop in the Gaia data at magnitudes brighter than in the model. This could be a priori due to inadequate extinction model or model density laws, or to incompleteness in the Gaia data at faint magnitudes due to undetected or omitted sources. Since the bright magnitude counts are fairly well fitted, the latter hypothesis is most probable. This is also pointed out by comparison with previous catalogues. In the outer Galaxy, GOG18 simulation is probably a too rough model of the Galactic structures, as can be seen in the fields at longitude 180{\deg} 
where the some substructures such as the Monoceros ring or the anticentre overdensity might contribute.
In \figref{fig:histograms}b, the field at longitude 43-47{\deg} and latitude 0{\deg} is for 2 lines of sights, where the model (in blue) gives similar star counts for the two lines while the data (in red) do not. We believe that this is due to varying extinction, which is underestimated in the model for these specific fields.

Over the whole sky, up to magnitude 18, there is a relative difference of a few percent (from less than 3\% at magnitude 12 to 10\% at magnitude 18). Between 18 and 19 the relative difference is 15\%. In the range 19 to 20, the difference is 25\% on the average. 
At high latitudes, and specifically at the Galactic poles, the agreement between the model and the data is also quite good. The regions where the Gaia data seem to suffer from incompleteness are located in the specific regions around $l=225${\deg}, $b=45${\deg} and $l=45${\deg}, $b=-45${\deg}, most probably related to the filtering of sources with a low number of observations. %This will be corrected in the next release which will cover a larger set of observations. 
The data are however probably complete up to $G=16$ in those regions ($l=225${\deg}, $b=45${\deg}), although the incompleteness could also occur at brighter magnitudes in some areas (at $G=14$ in $l=45${\deg}, $b=-45${\deg}). 

These comparisons show that Gaia data have a distribution over the sky and as a function of magnitude which is close to what is expected from a Galaxy model in most regions of the sky. However it points towards an incompleteness at magnitudes fainter than 16 in some specific areas less observed due to the scanning law, and because sources with a small number of observations have been filtered out. The completeness is also reduced in the Galactic plane due to undetected or omitted sources in crowded regions. This is expected to be solved in future releases where a larger number of observations will be available.


%----------------------------------------
\subsection{Small scale completeness of {\GDR1}} 
%----------------------------------------

\subsubsection{Illustrations of under-observed regions\label{sec:underdensity}}

Empty regions due to the threshold on the number of observations are 
illustrated in \figref{fig:942_holes}a near the galactic center; regions under-scanned 
like these ones are not frequent and have a
limited area, below 0.1 square degree \citep[see also][Sect. 6.2]{DPACP-8}. The field shown in \figref{fig:942_holes}b
near the bulge suffered from limited on-board resources, which created holes
in the sky coverage, as shown also for globular clusters in \figref{fig:947_patchy_6GCs}.

\begin{figure}
\centering
\includegraphics[width=0.7\columnwidth]{./figures-942/density-galactic-center.jpg}
\includegraphics[width=0.7\columnwidth]{./figures-942/holes-near-l330-b-3.jpg}
\caption{Regions with under-densities in DR1: a) under-scanned field near $l=354${\deg}, $b=-3${\deg}, 
size $\sim 3$ square degrees;
b) holes created by lack of on-board resources in another dense field near $l=330${\deg}, $b=3${\deg}, size $\sim 200$ square arcmin.}
\label{fig:942_holes}
\end{figure}

\subsubsection{Tests with respect to external catalogues\label{sec:wp944_smallscalecompleteness}}
The small scale completeness of {\GDR1} and its variation with the sky stellar density has been tested in comparison with two catalogues: \beforeReferee{the} Version 1 of the Hubble Space Telescope (HST) Source Catalogue \citep[HSC,][]{2016AJ....151..134W} and a selection of fields observed by OGLE \citep{2008AcA....58...69U}. 

\paragraph{Hubble Source Catalogue.} The HSC is a very non-uniform catalogue based on deep pencil-beam HST observations made using a wide variety of instruments (Wide Field Planetary Camera 2 (WFPC2), Wide Field Camera 3 (WFC3) and the Wide Field Channel of the Advanced Camera for Surveys (ACS) and observing modes. The spatial resolution of Gaia is comparable to that of Hubble and the HSC is therefore an excellent tool to test the completeness of {\GDR1} on specific samples of stars. To check the completeness as a function of \gmag, we computed an approximate \gmag-band magnitude from HST F555W and F814W magnitudes (\ghst) using theoretical colour-colour relations derived following the procedure of \cite{2010A&A...523A..48J}. 
%The announced mean photometric accuracy is better than 0.10 mag and the relative astrometric residuals within 10 mas. The absolute astrometric accuracy is better than 0.1 arcsec in most cases.

The first test was made in a crowded field of one degree radius around Baade's Window. Nearly 13\,000 stars were considered, observed in both the F555W and F814W HST filters with either WFPC2 or WFC3. 

The second test was made on samples of stars observed with one of the three HST cameras, using the red filter F814W and either F555W or F606W. Sources were selected following the recommendations of \cite{2016AJ....151..134W} to reduce the number of artefacts.
Moreover, only stars with an absolute astrometric correction flag in HST set to yes have been selected, leading to a typical absolute astrometric accuracy of about 0.1\arcsec. The size of the resulting samples varies from 1600 stars for ACS-F555W to nearly 120\,000 stars for ACS-F606W, going through 15-23\,000 stars for the four other samples. The completeness of Gaia observations for these samples, position differences and colour-colour relations have been tested. 

% Results
The completeness results of both tests are presented in \figref{fig:cu9val_wp944_HSTcompl}. In Baade's Window, the completeness follows the expectations for DR1: in this very dense area, on-board limitations lead to a brighter effective magnitude limit. The ``all-sky'' result (using here 128\,000 ACS stars with F606W$<20$~mag)
is at first sight more surprising, but in fact bright source observations with HST are quite rare and are done mainly in very dense areas (which need the HST resolution) such as globular clusters, which also suffer from Gaia on-board limitations.  We further checked this interpretation by using individual HST observations and images around a few positions : the test made for a low density area around the dwarf spheroidal galaxy Leo II \citep{2011ApJ...741..100L} leads to a completeness at magnitude 20 of nearly 100\%, while a test for a high density area around the globular cluster NGC 7078 \citep{2014ApJ...797..115B} leads to a completeness worse than the one presented \figref{fig:cu9val_wp944_HSTcompl}.

\begin{figure}
    \begin{center}
        \includegraphics[width=0.49\columnwidth]{./figures-944/BWcompleteness.pdf}
        \includegraphics[width=0.49\columnwidth]{./figures-944/hstaf606w_compl.pdf}
        \caption[{\GDR1} completeness versus HST]{{\GDR1} completeness (in \%) versus the Hubble Source Catalogue as a function of \ghst\ magnitude. The dotted lines correspond to the 1$\sigma$ confidence interval; a) in Baade's Window ($l=1${\deg}, $b=-4${\deg}); b) for all-sky HSC sources observed with the ACS and the F606W filter. }
        \label{fig:cu9val_wp944_HSTcompl} 
    \end{center}
\end{figure}



\begin{figure*}
\centering
\includegraphics[width=0.6\textwidth]{./figures-947/947_completeness_3GCs.png}
\caption{Completeness against density in the field of three chosen GCs, in different magnitude ranges. Fields such as NGC\,1261 have a median of 220 observations, allowing for a much better completeness in the denser regions than NGC\,6752 (40 observations). } \label{fig:947_completeness_3GCs}
\end{figure*}
%220 60 40

\begin{figure*}
\centering
\includegraphics[width=0.75\textwidth]{./figures-947/947_patchy_6GCs.png}
\caption{Stellar distribution for six chosen GCs, colour-coded by number of $G$ observation for each star. \textit{Top row:} examples of holes caused by limited on-board resources or bright stars. \textit{Bottom row:} in some regions patterns are visible corresponding to stripes where no stars had a sufficient number of observations. } \label{fig:947_patchy_6GCs}
\end{figure*}

\begin{figure}
\centering
\includegraphics[width=0.8\columnwidth]{./figures-947/947_completeness_5053.png}
\caption{Completeness of Gaia relative to HST in the area around NGC\,5053 featuring stellar densities under 1 million per square degree.} \label{fig:947_completeness_5053}
\end{figure}

\paragraph{HST observations of Globular Clusters.} We run detailed completeness tests within globular clusters using HST data specifically reduced for the study of those crowded fields. We used 26 globular clusters for which HST photometry is available from the archive of \citet[][see Table \ref{chap:cu9val_T1}]{2007AJ....133.1658S}. %Sarajedini 2007.
The data for all GCs were acquired with the ACS and contain magnitudes in the bands F606W and F814W. The observations cover fields of 3\,arcmin$\times$3\,arcmin size.  For M4 (NGC~6121), data by  \citet{2013AN....334.1062B}, and \citet{2015MNRAS.454.2621M} %Bedin 2013,Malavolta 2015
 taken in the HST project GO-12911 in WFC3/UVIS filters were used. For this test, the photometric transformations HST bands to Gaia $G$-band were adjusted for each cluster to fit a sample of bright stars in order to avoid issues due to variations in metallicity and extinction.

High quality relative positions and relative proper motions are available for these clusters. When artificial star experiments were available in the original HST catalogue (GCs marked  with * in Table \ref{chap:cu9val_T1}), the completeness of HST data has been evaluated by comparing the number of input and recovered artificial stars in each spatial bin. We find the completeness of the HST data to be well above 90\% and close to 100\% in all cases for  stars brighter than $V=21$, but for the very crowded cluster NGC5139 (OmegaCen).
The GCs are chosen to present different level of crowding down to $G\sim 22$. In general, HST data cover the inner core of the clusters, where the stellar densities are above $10^6$ stars per square degree in almost all regions (above 30 million in many cases, and up to 110 million stars per square degree in the core of NGC~104/47~Tuc). In a few cases, lower densities are reached in the external regions. We therefore expect Gaia to be very severely incomplete in most of the regions studied in this test.
The HST magnitudes were converted to Gaia $G$ magnitudes using the same transformations as previously between $G$ and F814W, F606W but on the Vega photometric system. 

\begin{table}
\centering\footnotesize
\caption{GCs used in the completeness test. 
Asterisks denote the ones with artificial star experiments available 
in the original HST catalogue.\label{chap:cu9val_T1}}
\begin{tabular}{lrr}
\hline\hline
\multicolumn{1}{c}{cluster} &\multicolumn{1}{c}{$\alpha$ (J2000)} &\multicolumn{1}{c}{$\delta$ (J2000)} \\
\hline
  LYN07 & 242.7619 & -55.315\\
  NGC~104* & 6.0219 & -72.0804\\
  NGC~288 & 13.1886 & -26.5791\\
  NGC~1261 & 48.0633 & -55.2161\\
  NGC~1851 & 78.5267 & -40.0462\\
  NGC~2298 & 102.2465 & -36.0045\\
  NGC~4147 & 182.5259 & 18.5433\\
  NGC~5053 & 199.1128 & 17.6981\\
  NGC~5139* & 201.6912 & -47.476\\
  NGC~5272 & 205.5475 & 28.3754\\
  NGC~5286 & 206.6103 & -51.3735\\
  NGC~5466 & 211.364 & 28.5342\\
  NGC~5927 & 232.002 & -50.6733\\
  NGC~5986 & 236.5144 & -37.7866\\
  NGC~6121* & 245.8974 & -26.5255\\
  NGC~6205 & 250.4237 & 36.4602\\
  NGC~6366 & 261.9349 & -5.0763\\
  NGC~6397* & 265.1725 & -53.6742\\
  NGC~6656* & 279.1013 & -23.9034\\
  NGC~6752* & 287.7157 & -59.9857\\
  NGC~6779 & 289.1483 & 30.1845\\
  NGC~6809* & 294.998 & -30.9621\\
  NGC~6838* & 298.4425 & 18.7785\\
  NGC~7099 & 325.0919 & -23.1789\\
  PAL~01 & 53.3424 & 79.5809\\
  PAL~02 & 71.5245 & 31.3809\\
\hline
\end{tabular}
\end{table}

For each GC, the total density of stars in square bins of 0.008\,deg = 0.5\,arcmin was evaluated, then in each bin we counted the number of stars present in the HST photometry and in the {\GDR1}, by slice in magnitude. 



The completeness of {\GDR1} is shown in \figref{fig:947_completeness_3GCs} for three clusters, as a function of the stellar density observed in the HST data. Different crowded regions present different degrees of completeness, depending on the number of observations in that region. In addition, holes are found around bright stars (typically for $G<11-12$ mag), and entire stripes are missing, as illustrated in \figref{fig:947_patchy_6GCs}. 

%The completeness of {\GDR1} relative to HST as a function of the Gaia magnitude is shown in \figref{fig:947_completeness_5053} for an area around NGC\,5053 with stellar densities under 1 million per square degree. In this case, due to the higher number of transits and lower star density, the completeness is very high.


%Regions having similar density in HST images, present very different degrees of completeness in the {\GDR1} data.


In less crowded regions, such as in the field around NGC\,5053 where stellar densities are under 1 million per square degree, the completeness is very high, as shown in \figref{fig:947_completeness_5053}.
 
 
%Typical examples can be found in Fig.\ref{Figresult:holes} 
%where the number of CCD transits used in the photometric analysis is shown (see also the discussion in the section \ref{artefacts}.


\paragraph{OGLE catalogues.} To further test the variation of the completeness with sky density, we looked at the completeness versus OGLE data using a few fields in the OGLE-III Disk \citep{2010AcA....60..295S}, OGLE-III Bulge \citep{2011AcA....61...83S} and OGLE-IV LMC \citep{2012AcA....62..219S} surveys. A \gmag-band magnitude was computed from OGLE $V$ and $I$ magnitudes (\goggle) using an empirical relation derived from the matched Gaia/OGLE sources (two relations were derived, one for OGLE-III and one for OGLE-IV due to their different filters). The stellar densities were estimated from the OGLE data themselves, therefore they are certainly slightly under-estimated. As can be seen in \figref{fig:cu9val_wp944_Oglecompl}, the completeness is not only dependent on the sky density, but also on the sky position, linked to the Gaia scanning law, as we saw above. In the bulge fields, the completeness may show a drop around \gmag =15 (as seen in \figref{fig:cu9val_wp944_Oglecompl}b, confirming the feature of \figref{fig:cu9val_wp944_HSTcompl}a). This is due to the fact that the reddest stars have not been kept in {\GDR1} (because of filtering at calibration level) and those missing stars correspond to the reddened red giant branch of the bulge (\figref{fig:cu9val_wp944_Oglecompl}c).  


\begin{figure*}
    \begin{center}
        \includegraphics[height=0.5\columnwidth]{./figures-944/OgleCompl.pdf}
        \hspace{0.15\columnwidth}
        \includegraphics[height=0.5\columnwidth]{./figures-944/OgleBlg100.pdf}
        \includegraphics[height=0.49\columnwidth]{./figures-944/Ogle-G-V-I.png}
        \caption[Completeness versus OGLE]{{\GDR1} completeness versus some OGLE Catalogues. a) Completeness at \gmag=18 of some OGLE fields as a function of the measured density at \gmag=20; b) Completeness in OGLE Bulge field blg100 ($l=-0.3${\deg}, $b=-1.55${\deg}), density: 970\,000 stars/deg$^2$; c) associated color-magnitude diagram, stars in red being missing in {\GDR1}. }
        \label{fig:cu9val_wp944_Oglecompl} 
    \end{center}
\end{figure*}


