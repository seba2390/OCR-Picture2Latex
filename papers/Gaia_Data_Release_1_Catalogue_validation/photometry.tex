%+++++++++++++++++++++++++++++++++++++++++++++++++++++++++++++++++++++++++++
\section{Photometric quality of DR1}\label{photoqual}
%+++++++++++++++++++++++++++++++++++++++++++++++++++++++++++++++++++++++++++

The photometric quality of {\GDR1}, accuracy and precision, 
has been tested using both internal methods 
(using Gaia photometry only) and by comparisons to external catalogues.

%----------------------------------------
\subsection{Internal test of the photometric accuracy}
%----------------------------------------

Using the \gbp/\grp~photometry, a way was found to check internally the variation of the \gmag~magnitude zero point with magnitude, that we will also check below with the external catalogues. 
One should keep in mind however that the {\gaia} photometric data are correlated due to the calibration procedures. 

We randomly selected sources at high galactic latitude ($\vert b \vert>50$\deg) with photometric quoted uncertainties in \gmag, \gbp~and \grp~$<0.02$~mag and a minimum of 10 observations in each band. We re-sampled this selection to have a uniform distribution in magnitude. An empirical robust spline regression was derived which models the global (\gbp-\grp)/(\gmag-\grp) colour relation
and we computed the residuals of the observed \gmag-\grp\ minus the \gmag-\grp $=f$(\gbp-\grp) spline.

The variation of these residuals with magnitude (\figref{fig:wp944_bprpphot}) is consistent with what we observed in the comparison with external catalogues below. First, the variations at bright magnitudes ($G<$12) are most probably linked to the different gate effects and saturation issues. 
%The strongest jump is around G=11 which could be due both to the last gate effect and to the window size change at G=11.5 for \gbp/\grp. 
Second, the window size changes on-board at \gmag~magnitudes 13 and 16. In very preliminary data, this induced  a strong jump at \gmag=13, seen and corrected in the calibration process of the DPAC photometric group \citep{DPACP-9}. In {\GDR1} the jump at \gmag=13 seems nicely corrected but a small jump at \gmag=16 is still visible. The increase in the residual dispersion seen in \figref{fig:wp944_bprpphot} at faint magnitudes is linked to the reduced precision of \gbp/\grp. 

\begin{figure}
\begin{center}
\includegraphics[width=0.7\columnwidth]{./figures-944/bprpbias.png} 
\end{center}
\caption[Gaia \gmag\ versus \gbp\ and \grp~photometry]{Gaia \gmag~versus \gbp\ and \grp~photometry. Residuals of \gmag-\grp\ from a global \gmag-\grp=$f$(\gbp-\grp) spline as a function of \gmag~magnitude. The red line is a smoothed spline fit. The sample contains 10\,000 stars with a uniform distribution in magnitude, therefore the lighter grey scale indicates less dispersion in the residuals. 
}
\label{fig:wp944_bprpphot}
\end{figure}


%----------------------------------------
\subsection{Internal test of the photometric precision}
%----------------------------------------

\def\gbr{$G_\mathrm{BR}$}
\beforeReferee{With only one band and its quoted precision published, validating the photometric 
precision without external comparisons looks also difficult. A method designated as
``internally homogeneous photometric samples'' was however devised to provide
at least some upper limits on the photometric precision.}

\beforeReferee{The basic principle is to compare in samples the observed dispersion of the $G$
magnitude of stars to the one which is predicted by their standard uncertainty 
(i.e. the quoted precision $\sigma_G$ in the Catalogue). 
There is obviously no way of obtaining an unbiased value about $G$ magnitude
by selecting samples based on the $G$ magnitude itself, so the principle 
is to make a selection of samples using other magnitudes, {\gbp} and {\grp}. 
In practice, we use instead what we will call {\gbr}=({\gbp}+{\grp})$/2$ for two reasons: first
this makes a magnitude closer to $G$, second the precision on {\gbr} is better 
than on {\gbp} or {\grp} alone. }

\beforeReferee{Samples were thus selected around some {\gbr} values, with an extremely
thin width around this value (about one {\mmag} width
for the brightest stars), using stars with a very similar colour, choosing e.g.
$0.78 < $ {\bprp} $ < 0.80$ as it provides samples large enough 
(other colour ranges have been used without significant changes).}

\beforeReferee{In such internally homogeneous samples, one can assume that
the $G$ magnitude has some central value with a very small dispersion 
due to the differences between stars of varying type, distance, extinction
(or variability),
and also due to the $G$ measurement errors, which is what we are trying to estimate.
Obviously, the stars in these samples have to be selected with a {\gbr} 
with a precision better or comparable to the small bin width chosen, which 
in turn strongly reduces the sample sizes. We also removed all sources with
an astrometric excess noise significance above 200 which would
be an indication of duplicity or data reduction problems.}

\beforeReferee{The central $G$ value in such subsamples remains unknown (although
it could be modelled by photometric transformations, though probably not to
the required precision), so we subtract the median value in order to keep
the $G$ residuals only, centred on 0. Then, for a given {\gbr} central
value, all the residuals obtained in all subsamples of one {\mmag} 
bin size are collected for each {\mmag} from {\gbr}-0.5 mag to {\gbr}+0.5 mag
and then added to build one single sample to study. One such resulting sample, 
around $BR=13$, is shown in \figref{fig:942-internal-photometry-hist-disp}a. The
resulting dispersion appears indeed small enough to allow the study of the effect
of the measurement errors at the {\mmag} level. }

%\begin{figure}
%\centering
%\includegraphics[width=0.47\columnwidth, height=0.45\columnwidth]{{./figures-942/HistG-mag-13-pm-0.002-sigBPRP-0.001-BPRP-0.78-0.8}.png}
%\includegraphics[width=0.47\columnwidth, height=0.45\columnwidth]{{./figures-942/uwe-pm-0.002-0.002-sigBPRP-0.002-0.001-BPRP-0.78-0.8-30}.png}
%\caption{Histogram of $G$ magnitudes residuals in a sample with {\bprp}$\approx 0.8$ and {\gbr}$\in(12.5,13.5)$ mag 
%collected by steps of one {\mmag} in {\gbr} (left). 
%Resulting estimations of the unit-weight errors of the $G$ magnitude for {\gbr} 
%around magnitudes 10, 11, 12, 13, 14, 15, using samples with {\bprp}$\approx 0.8$ (right).}
%\label{fig:942-internal-photometry-hist-disp}
%\end{figure}

\beforeReferee{The variance around 0 is the sum of some intrinsic variance
(due to the {\gbr} bin width and {\gbr} measurement errors, the small variations
between stars, variability, multiplicity, etc) 
and of the variance due to the measurement errors estimated by the $\sigma_G$ standard 
uncertainty. The observed  dispersion 
(robustly estimated by a MAD - Median Absolute Deviation) 
is clearly varying as expected with the $G$ standard uncertainty
(\figref{fig:942-internal-photometry-bins-precision}).}

\beforeReferee{So, a robust weighted fit has been done by cutting in 
equal size bins of standard uncertainty, estimating in each bin the variance by 
the square of the MAD and fitting the linear model:
observed variance = intrinsic variance + unit-weight variance $\times$ 
standard uncertainty squared. What is of interest here is the unit-weight 
error (square root of the unit-weight variance), representing 
the ratio of the ``external'' errors over the ``internal'' ones. 
The intrinsic variance also estimated through this fit, and which is 
of no interest here, has been subtracted in \figref{fig:942-internal-photometry-bins-precision}
to compare only the variance due to the measurement errors to the published $G$ precisions.}

\beforeReferee{The same estimation of the unit-weight error  
has been done for several values of magnitude, and for various colours,
with consistent results: an upper limit of the underestimation 
of the standard uncertainty of the $G$ magnitude
in the Catalogue as a function of magnitude in the
[10-15] magnitude range shown is a factor 1.7 at magnitude 13 
(\figref{fig:942-internal-photometry-hist-disp}b). 
There are no results for stars brighter than 10
(the number of stars being much too small to permit this
method) nor for the faintest stars (as there is a 
too large range of standard uncertainties
and to few {\gbr} precise enough to build uncontaminated bins
for the analysis).}

\beforeReferee{There is however no reason that a simple straight line 
would be valid for the unit-weight on the whole standard 
uncertainty range.
Splitting instead this range in four different bins
(though not trying to get continuity of the results between bins),
\figref{fig:942-internal-photometry-bins-precision}, 
suggests that the uncertainties might be underestimated for
the most precise stars
($\sigma_G\in(0.3,1.5]$ mmag: unit-weight error=$1.59\pm 0.12$, 
\figref{fig:942-internal-photometry-bins-precision}a), 
slightly overestimated in the medium uncertainty range 
($\sigma_G\in(1.5,4]$: uwe=$0.51\pm 0.09$ and
$\sigma_G\in(4,10]$: uwe=$0.47\pm 0.17$), 
and correctly estimated for the least precise
magnitudes
($\sigma_G\in(10,30]$: uwe=$1\pm 0.2$, 
\figref{fig:942-internal-photometry-bins-precision}b).
For the most precise $G$, one cannot exclude
however that the available {\gbr} was not precise enough to 
avoid an unmodelled dispersion, so that this method would 
provide an upper limit only on the unit-weight error.}
%range; bin size; $\sigma_{BR}$; $0.68<BR<0.7$; $0.78<BR<0.8$; $0.9<BR<0.92$
%0 - 0.0015; 0.001; 0.001; ; $1.59\pm 0.12$;  
%0.0015 - 0.004; 0.002; 0.002; ; $0.51\pm 0.09$;
%0.004 - 0.01; 0\input{../../../../../../Desktop/GAT-GaiaSource-GDR1-public-plots}
%.004; 0.004; ; $0.47\pm 0.17$;
%0.01 - 0.03; 0.01; 0.03; ; $1\pm 0.2$;


%\begin{figure}
%\centering
%\includegraphics[width=0.47\columnwidth, height=0.45\columnwidth]{{./figures-942/Extra-BPpRP-sigG-0-0.0015-pm-0.001-sigBPRP-0.001-BPRP-0.78-0.8}.png}
%\includegraphics[width=0.47\columnwidth, height=0.45\columnwidth]{{./figures-942/Extra-BPpRP-sigG-0.01-0.03-pm-0.01-sigBPRP-0.03-BPRP-0.78-0.8}.png}
%\caption{Observed $G$ variance minus intrinsic variance and fit of the unit-weight
%variance in the $\sigma_G\in]0.3,1.5]$ mmag bin of standard uncertainty of $G$ (left), 
%and the $]10,30]$ mmag bin (right), for samples with {\bprp}$\approx 0.8$, log scale. 
%The unit-weight variance from the fit is in red, the green one representing unit-weight=1.}
%\label{fig:942-internal-photometry-bins-precision}
%\end{figure}

%This method using internally homogeneous 
%photometric samples can be applied only to very large 
%catalogues such as Gaia and shows that for {\GDR1} there is no
%apparent huge over/underestimation of the published
%uncertainties on $G$, at least in the restricted cases studied
%(10-16 magnitude range, good {\gbr} precision, intermediate colours). 
%It should however be noted that possible correlations
%between stars were not estimated nor taken into account, 
%and that the results obtained tell nothing for standard
%uncertainties below $\approx 0.5$ mmag,
%nor for very bright stars, for bluer or redder stars, i.e.
%where we may actually expect photometric (and astrometric) problems.

\afterReferee{With only one band and its quoted precision published, validating the photometric 
precision without external comparisons is difficult. 
We made experiments using {\gbp} and {\grp} in order to check that the
observed variance vary as expected with the quoted precision $\sigma_G$ 
in the Catalogue, viz. 
observed variance = intrinsic variance + unit-weight variance $\times$ 
standard uncertainty squared. For most stars, there was no indication 
that their standard uncertainties were underestimated.

However,} there are about 12 million stars with $G$ standard uncertainties 
better than 0.5 mmag, which are thus difficult to check. 
There are however indications that some of the best precisions may be too optimistic:
the 53 most precise stars (having $\sigma_G< 0.1$ mmag) have a median value
of about 80 observations while the 1000 most precise have about 500 observations 
as median value. While the latter may explain a good precision, the
former cannot, as they would otherwise beat the Poisson noise (note that
a significant fraction of DR1 sources have standard uncertainties below 
Poisson noise). 
The most precise photometry may thus contain a mix of stars with
a large number of observations (\beforeReferee{OK}\afterReferee{as expected}) 
and of stars with very small apparent scatter, 
by chance or due to correlations, and these uncertainties 
should thus not be taken at face value.

%\subsection{Photometric quality of DR1 from the comparison with external data}
%----------------------------------------
\subsection{Photometric accuracy and precision from external catalogues\label{sect:wp944photometry}}
%----------------------------------------

The following tests compare the photometry of Gaia DR1, including TGAS, with external photometry. We check here the distribution of a mixed colour index, Gaia magnitude minus the external catalogue magnitude, versus an external catalogue colour. An empirical robust spline regression was derived which models the global colour-colour relation. The residuals from this model were then analysed as a function of magnitude, colour and sky position.

\paragraph{HST CALSPEC standard stars} \citep{Bohlin07}. The HST CALSPEC standard spectrophotometric database\footnote{\scriptsize\url{http://www.stsci.edu/hst/observatory/crds/calspec.html}} has been used to compute theoretical \gmag-magnitudes by convolving their spectra with the nominal Gaia passband using the pre-launch nominal passband. As this passband has not yet been adapted to the real Gaia response, expected photometric differences are observed, reaching a difference of up to 0.1~mag  at \bmv=1.2. This confirms that the pre-launch filter should not be used blindly by the community working on {\GDR1} data. Instead colour-colour transformations between Gaia and other photometric systems, available in {\GDR1} documentation, should be used. An updated passband will be provided with DR2. 

% \begin{figure}
%  \begin{center}
% \includegraphics[width=0.45\columnwidth]{./figures-944/calspec.png} 
% \hspace{0.05\columnwidth}
% \includegraphics[width=0.45\columnwidth]{./figures-944/landolt.png} 
% \end{center}
% \caption[Gaia versus photometric standards]{a) Difference between the Gaia magnitude and the reference magnitude computed using the Calspec spectra and the pre-launch Gaia passband.
% b) Colour-colour relation with the Landolt standards; the red dotted line is the colour-colour polynomial relation provided in section \ref{chap:cu5phot}}
% \label{fig:wp944_std}
% \end{figure}


\paragraph{$BVRI$ photometric standard stars}  \citep{Landolt92}. 397 stars, mostly within the magnitude range $11.5<V<16.0$ and in the colour range $-0.3<B-V<2.3$, with photometric scatter $<0.02$ mag have been selected for this test. 
The observed dispersion around the colour-colour relation is larger than the quoted errors. This can be explained by an intrinsic stellar variability or by an under-estimation of the errors in one or both catalogues. 

\paragraph{Hipparcos photometry.} The sample of the {\it well behaved} Hipparcos stars (i.e. excluding known or suspected binaries, see \secref{sec:wp944_astrom}) has been used here with extra filters to exclude variable stars (variability flag VA=0) and restrict the sample to stars with good Hipparcos photometry ($\sigma_{Hp}<0.01$~mag and $\sigma_{B-V}<0.02$ or $\sigma_{V-I}<0.03$~mag). Although the {\preDR1} filtering removed the strongest outliers, a number of outliers are still present in the colour-colour relations, but a large fraction of them can be filtered out using their photometric errors, as illustrated in \figref{fig:wp944_hipphot}a where red dots are stars with $\sigma_G>0.01$~mag.

We have further selected a subset of the Hipparcos stars with low extinction ($A_V<0.05$~mag) using the 3D extinction map of \cite{Puspitarini14} or, when the star reaches the limit of the map, the 2D map of \cite{Schlegel98}. This selection ensures a clean colour-colour spline relation \gmhp\ vs \vmi. The residuals versus this global relation show a strong variation with magnitude (\figref{fig:wp944_hipphot}b), with an amplitude up to 0.01~mag\afterReferee{. Such a systematic is }ten times larger than the uncertainties quoted for \gmag~at magnitude 8. This is most likely due saturation effects near gate changes or residual calibration errors linked to this. 

\begin{figure}
 \begin{center}
\includegraphics[width=0.65\columnwidth,height=0.5\columnwidth]{./figures-944/hipBV.png} \\
\vspace{0.5cm}
\includegraphics[width=0.65\columnwidth,height=0.5\columnwidth]{./figures-944/hipphotbias.png} 
\end{center}
\caption[Gaia versus Hipparcos photometry]{{\GDR1} versus Hipparcos photometry. Top: Colour-colour relation \gmhp~as a function of \bmv; in red, stars with $\sigma_G>0.01$~mag; the red dotted lines are the colour-colour polynomial relation provided in the release documentation for dwarfs and giants. 
Bottom: \gmhp\ residuals from a global \gmhp\ $=f(V-I)$ spline relation as a function of \gmag~magnitude for low extinction stars only.
}
\label{fig:wp944_hipphot}
\end{figure}

\paragraph{SDSS photometry.} \afterReferee{Here} we used \beforeReferee{here} the tertiary standard stars of \cite{Betoule13} calibrated to the HST-CALSPEC spectrophotometric standards with a precision of about 0.4\% in {\it griz}. It covers four CFHT Deep fields and the SDSS strip 82. While the CFHT fields are in low extinction regions, for the SDSS strip only areas with a maximum $E(B-V)<0.03$ according to the \cite{Schlegel98} map are selected.
The residuals versus the global colour-colour spline relation (\figref{fig:wp944_sdssphot}) show a strong increase of the residuals at the faint end in all SDSS and CFHT fields, with an amplitude larger than the quoted uncertainties, of the order of 0.01 at \gmag=20. An increase of the bias at $\sim$16~mag is also seen in the SDSS field (the SNLS is too faint to probe this magnitude) which could be due to window class change but also to saturation in the SDSS data. 
We checked that the increase at the faint end is not due to the random errors alone (as the ordinate is correlated with the abscissa in \figref{fig:wp944_sdssphot}) by checking that this increase was visible using also all the SDSS magnitudes, in particular with $z$ that is fully independent from the other magnitudes used for the residual computation. Note that we did similar checks for all the other external catalogues.

A confirmation of this global behaviour has been obtained with the OGLE data which were used for the completeness tests (\secref{sec:wp944_smallscalecompleteness}). To avoid potential zero point issues, we used data from a single CCD at a time. The large extinction of those fields lead to a less well defined colour-colour relation but the increase of the residuals with magnitude is nevertheless also seen in the OGLE data, confirming the $> 0.02$ mag zero point variation with magnitude of the Gaia photometry at its faint end. % OGLE photometry too bright % more on this ?

\begin{figure}
 \begin{center}
%\includegraphics[width=0.45\columnwidth]{./figures-944/sdssCC.png} 
%\hspace{0.05\columnwidth}
\includegraphics[width=0.65\columnwidth,height=0.5\columnwidth]{./figures-944/sdssbias.pdf} 
\end{center}
\caption[Gaia versus SDSS photometry]{{\GDR1} versus SDSS photometry: %(a): Colour-colour relation $G-g$ as a function of $g-i$; the red dotted line is the colour-colour polynomial relation provided in section \ref{chap:cu5phot}. (b): 
$G-r$ residuals from a global $G-r=f(g-i)$ spline relation as a function of \gmag~magnitude.
}
\label{fig:wp944_sdssphot}
\end{figure}

\paragraph{Tycho-2 photometry.} We selected only stars with photometric errors in \bt~and \vt~$<0.05$~mag and at high galactic latitude ($\vert b \vert>40$\deg) to have a low extinction. To obtain clean colour-colour relations, 
%\figref{fig:wp944_tycho2phot}, 
the sample has been roughly separated between dwarfs and giants with a colour cut at \btmvt~=0.9~mag and an absolute magnitude cut at \mg=4.5, taking into account the parallax error at 1~$\sigma$. The residuals show a variation with \gmag~magnitude, confirming the increase seen at $G\sim 8$ with Hipparcos and suggesting an increase at $G\sim 11$ as well. 

%\begin{figure}
% \begin{center}
%\includegraphics[width=0.45\columnwidth]{./figures-944/tycho2dwarfs.png} 
%\hspace{0.05\columnwidth}
%\includegraphics[width=0.45\columnwidth]{./figures-944/tycho2giants.png} 
%%\includegraphics[width=0.45\columnwidth]{./figures-944/tycho2bias.png} 
%\end{center}
%\caption[Gaia versus Tycho-2 photometry]{{\GDR1} versus Tycho-2 photometry. Colour-colour relations for a) dwarfs b) giants. The red dotted lines are the colour-colour polynomial relations provided in the release documentation. 
%}
%\label{fig:wp944_tycho2phot}
%\end{figure}

\paragraph{2MASS photometry.} The comparison with 2MASS is more difficult due to a sharp feature at $J-$\Ks$\sim 0.8$ for the red dwarfs and the unavailability of parallaxes in {\GDR1}. To remove the red dwarf feature, we selected only stars with $J-$\Ks$<0.7$. As for Tycho-2 we selected only stars with photometric errors in $J$ and \Ks$<0.05$~mag and at high galactic latitude ($\vert b \vert>40$\deg). The residuals also show an important variation with \gmag~magnitude. 

% Some more info valid for all the tests above 
\paragraph{} All the tests above also show a correlation of the \gmag~residuals with Gaia \gbp-\grp~which has not been studied in detail as this colour is not part of {\GDR1}, but this variation does not exceed $\sim$0.01~mag.
Those tests also show a significant correlation between the photometric residuals and the astrometric excess noise which measures the disagreement with the astrometric model. This is expected as the astrometry and the photometry share the same PSF model
and the same windows, possibly contaminated by a neighbour.


% \begin{figure}
%  \begin{center}
% \includegraphics[width=0.45\columnwidth]{./figures-944/tmassbias.png} 
% \end{center}
% \caption[Gaia versus 2MASS photometry]{Gaia versus 2MASS photometry. Residuals versus a global $G-J$=$f$($J$-\Ks) spline relation as a function of \gmag~magnitude. 
% }
% \label{fig:wp944_2massphot}
% \end{figure}



%----------------------------------------
\subsection{Testing $G$ photometry using clusters}\label{sssec:cu9val_ocphot}
%----------------------------------------
 
To test the photometric accuracy and precision of {\GDR1} against published photometry of stellar clusters, we made use of a sample of high photometric quality by \cite{2008ApJS..176..262T}. These authors provided high precision photometry in $V$ band (a few  mmag), for 5 open clusters: Hyades, Praesepe, Coma Ber, NGC752 and M67. The photometry in this catalogue is highly homogeneous, both in data reduction and in zero point for all the clusters. In addition, we used M4 HST photometry by \cite{2014MNRAS.442.2381N} in $F606W$ band, where repeated observations allowed to reach a few mmag precision (for the relevant magnitudes, $F606W<$21).

\medskip
For all clusters, the same procedure was adopted, namely:
\begin{itemize}
\item the reference catalogue was checked, removing variable and multiple stars. Variability information was taken from  SIMBAD. Multiplicity information was taken from the Hipparcos catalogue. For the Hyades, we used also \cite{2016A&A...585A...7K} catalogue to remove multiple stars. In the case of M4, variability information is taken from \cite{2014MNRAS.442.2381N}. After this selection, the total number of stars is 40 in M4 (down to $G\sim 14$), and 232 in the open cluster sample.
\item We extracted for each source the {\gaia} data. For the open cluster sample stars, the cross match is straightforward, being all bright stars observed in the Hipparcos catalogue. For M4, at fainter magnitudes and with a high level of crowding, a more sophisticated cross match procedure was followed taking into account proper motions (from L. Bedin, priv. comm.).
\item The difference between $G$ magnitude and a reference magnitude does depend on the apparent colour, \beforeReferee{thus} \afterReferee{and consequently it depends} both on temperature and extinction. In the case of open clusters, to gain statistics yet working with homogeneous extinction levels, we grouped the 5 clusters according to the extinction level \citep[from][]{2008AJ....136.1388T, 2007AJ....133..370T, 2007AJ....134..934T, 2006AJ....132.2453T}. The 3 groups are: Coma and Hyades, ($E(B-V)<0.01$); Praesepe ($E(B-V) \sim 0.1$); M67 and NGC752
($E(B-V) = 0.1$ -- 0.14).  
\item For each group of OCs and for M4, we derived separately the relation between $G$ magnitude and the reference magnitude against colour, using a low order spline.
\item We analysed the residuals of this function against the apparent G magnitude.
\end{itemize}

We show the residuals in \figref{fig:cu9val_W947_T08phot_test}, for the 5 open clusters together and in \figref{fig:cu9val_W947_M4phot_test} for M4. In both figures, we fitted a high order spline. The residuals clearly show systematics at a 10 mmag level related to the presence of gates, as discussed in \secref{sect:wp944photometry} using a comparison with large external catalogues.

%-------------------------

%\figref{fig:cu9val_W947_T08phot_test}
\begin{figure}
\centering
\includegraphics[width=0.9\columnwidth]{./figures-947/cu9val_W947_T08phot_test.png}
\caption[$G-V$ vs magnitude for open clusters]{Residuals of the difference $G-V$ against a low order spline, as a function of  the magnitude\afterReferee{, for 5 different clusters}. The $V$ magnitude is from the \cite{2008ApJS..176..262T} catalogue\beforeReferee{, for 5 different clusters}. Red lines marks the gates position in magnitude. The green curve is a high-order spline fit to the data.} \label{fig:cu9val_W947_T08phot_test}
\end{figure}
%-------------------------

  
%-------------------------

%\figref{fig:cu9val_W947_M4phot_test}
\begin{figure}
\centering
\includegraphics[width=0.9\columnwidth]{./figures-947/cu9val_W947_M4phot_test.png}
\caption[HST $F606W$ photometry for M4]{Same as in \figref{fig:cu9val_W947_T08phot_test} but for M4 using $G$ minus HST $F606W$ photometry by \cite{2014MNRAS.442.2381N}.} \label{fig:cu9val_W947_M4phot_test}
\end{figure}
%-------------------------





%----------------------------------------
\subsection{Photometry for variable stars}\label{sec:variables}
%----------------------------------------


Gaia is particularly interesting for stellar variability studies since it provides a remarkable time-domain survey, which is going to help to better characterise already known variables and even detect new ones. {\GDR1} includes light curves for a selection of Cepheids and RR~Lyrae stars as described in \cite{DPACP-15,DPACP-13}. Several tests were developed to validate the data compared to ground-based surveys.

Additionally, objects with intrinsic or extrinsic variability may also affect the Gaia data analysis \citep{2000ASPC..210..482E}. For instance
% brightness is not used for x-match
%, variations in luminosity can complicate the cross matching of sources, leading to a wrong determination of physical parameters. In the opposite sense, 
the instrument and/or the data processing can also introduce false variability that might be interpreted as real. This aspect has been taken into consideration to implement a set of tests which verify that no significant statistical biases are present in {\GDR1}.


\subsubsection{Testing variable stars light curves.}\label{chap:cu9val_var_lc}

We compared the dataset of Cepheids and RR~Lyrae stars included in the {\GDR1} against the OGLE IV SEP catalogue \citep{2012AcA....62..219S}. We found that reported {\GDR1} periods, average $G$ magnitudes and amplitudes are in agreement with the external catalogue and no particular outlier was found. OGLE also classifies stars depending on their variability, no particular disagreement was found with {\GDR1} classification.

Light curves included in {\GDR1} were also compared to OGLE IV SEP catalogue. Since OGLE uses $V$ and $I$ filters, it was necessary to transform them into $G$ magnitudes, which was possible thanks to the internal work done by the DPAC variability group. Additionally, to ease the comparison task, OGLE light curves were linearly interpolated to match the data points present in the folded Gaia light curves as shown in Fig.~\ref{fig:946_folded_cepheid}. This is a simple approach, the magnitude transformation is not perfect and the interpolation is more difficult in regions with fewer measurements, but it has been shown to be good enough to discard the presence of extreme outliers. 

Considering the whole sample, we found an average RMS of $0.04\pm0.02$ (the average {\gmag} magnitude is $\sim18.99$~mag). After a visual inspection of transformed OGLE and Gaia folded light curves with larger RMS, we did not identify any significant outlier.

\begin{figure}
\centering
\includegraphics[width=0.7\columnwidth]{figures-946/RRLyrae_different_folded_light_curve_5284216255613299840}
\caption{Example of a folded light curve corresponding to a Gaia RR~Lyrae star compared to a magnitude-converted and interpolated OGLE counterpart. The interpolation process hides the real dispersion present in OGLE, which is generally greater than in Gaia.}\label{fig:946_folded_cepheid}
\end{figure}


The determination of the light curves of variable stars is not limited to the presence of accurate photometry, but also it is fundamental to have reliable registered times for each measurement. To validate this aspect, we computed and compared the time separations between the moment of maximum and minimum magnitude in the Gaia and OGLE light curves. As a complementary test, we also computed $v = \frac{\left(t^{\mathrm{max}}_{\mathrm{OGLE}} - t^{\mathrm{max}}_{\mathrm{Gaia}} \right)}{p}$, where $t^{\mathrm{max}}$ are the times of maximum magnitude and $p$ the period, and we considered the decimal part of $v$, which should be close to 0.000 or close to 0.999 if the variable has gone through the full variability cycle an integer number of times. Both validations were executed considering the whole group of variables together, since it is expected that in individual cases there can be variations due to sources not pulsating completely regularly. Based on statistical tests, we did not find any significant discrepancy in the reported times between catalogues.

%
% Error: the following analysis was actually not significant as
% the mean G uncertainty uses a scatter, so it not only includes the random 
% error but ALSO the large amplitude of variables
%
%We used these comparisons to check indirectly the mean $G$ photometric uncertainty 
%the following way. Light curves were available for a sample of Cepheids and RR Lyrae, and the denser
%OGLE data was used to compute the RMS of the difference of epoch $G$ magnitudes.
%As the Gaia epoch photometric uncertainty was not available to us, we estimated it with 
%$\sigma_\mathrm{epoch} = 1.0857 \sqrt{{N_\mathrm{obs}}\over 8}{{\sigma_F}\over{F}}$,
%where $F$ is the mean flux and assuming 8 observations on the average per transit.
%The unit-weight uncertainty (ratio of the observed RMS over the predicted uncertainty)
%vs the mean $G$ uncertainty is shown \figref{fig:946_Ogle}. 
%For the Cepheids, the robust unit-weight is 1.1, whereas it is 1.3 for the fainter RR Lyrae. 
%
%\begin{figure}[]
%\centering
%\includegraphics[width=0.8\columnwidth]{figures-946/Cepheids-RR-Lyrae-uncertainties.png}
%\caption{Unit-weight uncertainty of the magnitude for a sample of Cepheids and RR Lyrae 
%from a comparison with OGLE data.}\label{fig:946_Ogle}
%\end{figure}


\subsubsection{Comparing distributions of variable stars to constant stars.}\label{chap:cu9val_var_dist}

The Hipparcos catalogue and its variability classification was used as main reference for creating two different subsets of Gaia sources with constant and variable stars. Then, these groups were compared to check whether:

    \begin {itemize}
    \item Parallaxes are not affected by variability
    \item No correlation exists between parallaxes or parallax uncertainties and periods, amplitudes, mean $G$ magnitude or colours
    \item Mean $G$ magnitude are within known min/max magnitudes for variable stars.
    \end {itemize}

The cross-matched group formed by constant stars contained 36\,661 sources with a mean {\gmag} magnitude of $8.27\pm1.11$, while the variable stars group was composed by 1\,820 sources with a mean {\gmag} magnitude of $8.26\pm1.10$. Based on statistical tests, we found that the normalized parallax difference distributions between these two groups were consistent and, for periodic stars, that no correlation were identified with periods or amplitudes. Hence, stellar variability does not seem to have a major effect in the reported {\GDR1} parallaxes. 



%----------------------------------------
\subsection{Summary of the photometric validation}\label{sec:summary-photo}
%----------------------------------------

With very precise photometry for (much) more than one billion stars, 
the {\gaia} photometry is on the verge of becoming a standard for
several decades. It is thus extremely important to understand the
properties and limitations of $G$ photometry for {\GDR1}.

It appears that systematics are present at the 10 mmag 
level with a strong variation with magnitude. This
is well above the standard uncertainties for bright stars and
could originate from saturation and gate configuration changes.
These points will be solved for the DR2. 

As for the photometric precision, the standard uncertainties may be 
underestimated for the most precise, \beforeReferee{though by a factor smaller than 1.7, but is}
\afterReferee{but they are} probably correctly estimated for most of the other stars.
