\section{Gaia archive interface validation}\label{sec:betatest}

\subsection{Testing Methodology}\label{sec:wp910:method}

%Earlier 
This section discusses the validation procedures employed in testing the design and interfaces of the archive systems delivering the {\GDR1}  data to the end user community. 

The design of the Gaia Archive was such as to fulfil the set of data access requirements gathered through a community scoping exercise. The Gaia user community were asked to suggest a number of ``Gaia data access scenarios'' and
enter them on the Gaia Data Access wiki pages at {\small\url{http://great.ast.cam.ac.uk/
Greatwiki/GaiaDataAccess}}. All scenarios received to March 2012 were considered and analysed, and presented in the DPAC Gaia data access scenarios scoping document GAIA-C9-TN-LEI-AB-026\footnote{{\scriptsize\url{http://www.rssd.esa.int/doc_fetch.php?id=3125400}}}. The Gaia ESA Archive~\citep{DPACP-19} was designed to take into account these user requirements. 

Within the Catalogue validation exercise a ``Gaia Beta Test Group'' (BTG) was constituted with a remit to perform a range of usage tests on the Gaia Data Archive and associated access clients and interfaces. The BTG is composed of members from across the DPAC, with expertise in all areas of Gaia. In addition the BTG includes members from the astronomical data centres associated with DPAC. 

The BTG generated a range of archive tests, documented the results of these tests, and raised fault reports in cases where the tests failed. These issues were reported through the DPAC ticketing system, with each being assigned to the relevant members of the Gaia Archive team. 

A range of the test queries generated have subsequently been re-used as part of the user documentation associated with the {\GDR1} release, in particular many queries have entered the  {\GDR1} Cookbook\footnote{{\scriptsize\url{https://gaia.ac.uk/science/gaia-data-release-1/adql-cookbook}}}. 

\subsection{Testing the Main {\GDR1}  Archive}\label{sec:wp910:main}

The main website access to the {\GDR1}  data is accessible at {\small\url{http://archives.esac.esa.int/gaia/}}. This was made available to the BTG at an early stage, initially populated with simulation data. Testing commenced early 2016, with an initial focus on the web interfaces to the archive. This included queries constructed via the simple form based archive pages, or more complex queries using ADQL (Astronomical Data Query Language\footnote{Documentation for ADQL available at {\scriptsize\url{http://www.ivoa.net/documents/REC/ADQL/ADQL-20081030.pdf}.}}, a IVOA\footnote{International Virtual Observatory Alliance: {\scriptsize\url{http://www.ivoa.net}}} standard. 

Later testing exercised remote programmatic access utilising the IVOA Table Access Protocol\footnote{see the IVOA Standard definition at {\scriptsize\url{http://www.ivoa.net/documents/TAP/20100327/}}} interface. 

Issues raised included those related to the user interface issues and also to the archive documentation. Functionality issues covered topics such as simplifying bulk data download, to use of server side storage, to inconsistencies in data table schemas. 

At the time of {\GDR1} release to the community, all raised issues classified as high priority have been fixed or resolved. Some lower priority issues will be addressed in upcoming maintenance releases, these being documented at the time of public data release. 

\subsection{Testing the {\GDR1}  Partner Archives}\label{sec:wp910:partner}

The {\GDR1} will also be released through a number or `partner' data centres. These provide alternative access points to the Gaia data, and additionally each provides some specific functionalities not available through the main ESA Gaia archive. 

The Gaia partner archives publishing {\GDR1} data are available at the following access points:

\begin{itemize}
\item Centre de Donn\'{e}es astronomiques de Strasbourg (CDS): {\small\url{http://cdsweb.u-strasbg.fr/gaia#gdr1}}
\item Leibniz-Institute for Astrophysics Potsdam (AIP): {\small\url{https://gaia.aip.de/}}
\item Astronomisches Rechen-Institut (ARI),  Zentrum f\"{u}r Astronomie der Universit\"{a}t Heidelberg: {\small\url{http://gaia.ari.uni-heidelberg.de/}}
\item ASI Science Data Center, Italian Space Agency  (ASDC): {\small\url{http://gaiaportal.asdc.asi.it/}}
\end{itemize}

Each partner data centre was provided with the {\GDR1}  data in early August 2016 in advance of the {\GDR1}  data release. This enabled a range of tests of the interfaces to be carried out by the BTG. All issues found were reported to the operators of these partner data centres. 

%\subsection{Looking ahead to Gaia DR2}\label{sec:wp910:dr2}
%
%The BTG will carry out a full suite of functionality and usability tests of the Gaia archive as it is developed and enhanced to support future Gaia data releases. The first of these, Gaia DR2, will take place in the second half of 2017.  The BTG will also assess the expected feedback from the community in their use of the {\GDR1}  archive and associated services. 
