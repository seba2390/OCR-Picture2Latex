%+++++++++++++++++++++++++++++++++++++++++++++++++++++++++++++++++++++++++++
\section{Data and models}\label{sec:description}
%+++++++++++++++++++++++++++++++++++++++++++++++++++++++++++++++++++++++++++

%----------------------------------------
\subsection{Data used}\label{data}
%----------------------------------------

\subsubsection{Gaia data}\label{gaia-data}
%During the validation preparation, we successively received intermediate 
%Catalogues, 
%%TGAS\_00.00 and TGAS\_00.01, 
%several preliminary releases of TGAS, and a preliminary version of the whole Catalogue. 
%%The data were covering the period 25 July 2014 to 19 May 2015 for TGAS\_00.00 and TGAS\_00.01, and 25 July 2014 to 16 September 2015 for TGAS\_01.00. 
%These preliminary versions of Gaia data were only delivered to exercise the validation software, but also allowed to identify and eliminate a few data inconsistencies at this initial stage. 
%Then, t
Two months before the final go-ahead to publish the {\GDR1} Catalogue, we received the 
official preliminary Catalogue, called {\preDR1} in what follows, which was validated,
then subsequently filtered, as described in \secref{sec:wp942:dupes}, to produce the {\GDR1} Catalogue.
Generally speaking, the validation work has had access to the same fields as
published in {\GDR1} so that any user can reproduce the work indicated below. For 
example we did not have access to any individual transit data or calibration data,
or more generally to the main {\gaia} database, and this
fostered developing methods independent from the work done within the Gaia 
groups producing the data.
A few supplementary fields were however kindly made available for validation
purposes, such as the preliminary {\gbp} and {\grp} magnitudes (in order to study possible 
chromatic effects).

\subsubsection{Simulated Gaia data}\label{simulated-gaia-data}\label{agislab}

In the course of the preparation of the data validation, we also needed 
simulated data, mostly for testing the astrometry of the
TGAS solution. For this purpose we built a simulated catalogue,
called Simu-AGISLab in what follows, which contained astrometric data
for the Tycho-2 stars, on top of which were added simulated TGAS astrometric errors.
Simu-AGISLab used as simulated proper motions the Tycho-2 ones, but they were
``deconvolved'' using the formula indicated in \citet[Eq.~10]{1999ASPC..167...13A}
to avoid a spurious increase of their dispersion with the TGAS astrometric errors added
by the simulation. The simulated parallaxes were a weighted average of
``deconvolved'' Hipparcos parallaxes (for nearby stars) and the photometric parallaxes 
from the \cite{2011yCat.6135....0P} catalogue (for more distant stars). 
The simulated TGAS astrometric errors were produced as described in the 
Tycho-Gaia Astrometric Solution document \citep{2015A&A...574A.115M}, based
on solution algorithms described in \citet[][Sect. 7.2]{2012A&A...538A..78L}.

In addition, global simulations of the Gaia data generated by the 
DPAC group devoted to this purpose
were also used for validation tasks comparing models with data (see 
\secref{models-intro}).

%For fainter objects, we also made use of the Initial Gaia Source List (IGSL) 
%a Catalogue representing the selection before launch of known optical astrometric 
%and photometric information of sources up to magnitude G=21, where possible 
%\citep{2014A&A...570A..87S}.

\subsubsection{External data}\label{external-data}
The comparison of {\GDR1} to external catalogues is a tricky task as the Gaia Catalogue is unique in many ways: it combines the angular resolution of \beforeReferee{Hubble}\afterReferee{the Hubble Space Telescope} with a complete survey all over the sky in optical wavelength, down to a \gmag-magnitude $\simeq$ 21, unprecedented astrometric accuracy and all-sky homogeneous photometric data. 

However, the comparison with external catalogues is one way towards a deeper understanding of many of the parameters describing the performance of the Catalogue: overall sky coverage, spatial resolution, catalogue completeness and, of course, precision and accuracy of the different types of data for the various categories of objects observed by Gaia. Besides the Hipparcos and Tycho-2 catalogues, many other catalogues have been used, especially chosen for each of these tests. They are described in each of the relevant subsections.

%, with details about the properties of the catalogues and the selection criteria given in Appendix \ref{app:catalogues}. 
The cross-match between TGAS and the external catalogues or compilations has been done using directly Tycho-2 or Hipparcos identifiers, either provided in the publications or obtained through SIMBAD queries \citep{2000A&AS..143....9W} using the identifiers given in the original papers. For the full {\GDR1} tests, a positional cross-match has been used. 


%----------------------------------------
\subsection{Data integrity and consistency}\label{sec:wp942:consistent}
%----------------------------------------

{\GDR1} is the combined work of hundreds of people divided into dozens of groups working on several complementary yet independent pipelines. 
In addition to testing the data themselves, therefore, we tested the data \emph{representations} to ensure that all catalogue entries were valid and self-consistent. We checked that catalogue values were \emph{finite}, that data were \emph{present} (or missing) when expected, that all fields were in their expected \emph{ranges}, that observation counts agreed with each other, that source identifiers were \emph{unique}, that correlation coefficients formed a \emph{valid} correlation matrix, that fluxes and magnitudes were related as expected, that the positions obtained from the equatorial, ecliptic, and galactic coordinates agreed, and so on. We also confirmed that the {\GDR1} in different data formats indeed contained the same data.

All data integrity issues were fixed before the data release. 
%It can be noted, however,
%that 16\,678 sources have two-dimensional observations (i.e. positional
%measures both in the AL and AC directions), but do not have an
%\texttt{astrometricWeightAc} field, i.e.\ the weight of the AC data in the
%astrometric solution as one might expect. The reason is that all the AC
%measures in question were considered outliers.
%
%We found that 16,678 sources had two-dimensional observations but did not have a \texttt{astrometricWeightAc} field; all other data integrity issues were fixed before the final version of Gaia~DR1 was released.
%
For TGAS solutions we also checked individual values of proper motions and parallax
looking for e.g.\ \afterReferee{negative parallaxes or} unrealistic tangential velocities. We then 
checked the \emph{uncertainties} of the five astrometric parameters to make sure that
they decreased with the number of observations, or to see if there were Healpix pixels with an
unusually high fraction of large uncertainties. All in all we were particularly
interested in regions on the sky where dubious values occur with higher
frequency than in typical areas, with the aim of excluding if needed such regions from
the release. Although some poorly scanned regions were identified as
problematic, none were finally excluded.

Sources brighter than \afterReferee{about} 12~mag are observed with ``gates'', i.e.\ with
reduced exposure time. We therefore checked that the astrometric standard
uncertainties did not show rapid changes as a function of magnitude.

We found only a few minor issues in the {\GDR1} astrometry as for the data ranges. 
Large values of fields like \dt{astrometric\-\_excess\_noise}\footnote{Roughly speaking, this 
is the noise which should be added to the uncertainty of the observations to obtain a perfect
fit for the astrometric model. The
fields of the Gaia Catalogue are described at {\small\url{https://gaia.esac.esa.int/documentation/GDR1/datamodel/}}} 
and \dt{astrometric\-\_excess\-\_noise\_sig} that statistically were expected for only about a thousand sources are actually present in about 205 million sources, including nearly the entire TGAS sample. These large values reflect the large errors introduced by the preliminary attitude solution for the Gaia spacecraft; a better solution will be used in future releases \citep{DPACP-14} and we expect this problem will be solved. In addition, 4\,288 sources have positions based on only two one-dimensional measurements, providing an astrometric solution with no degrees of freedom. These minimally constrained solutions are expected to go away as more data are collected.

%Because {\GDR1} is a first data release with possibly not all safeguards implemented in
%the data analysis pipelines, we characterised the astrometric data for all sources that 
%passed the release criteria. 
We tested whether sources had enough astrometric measurements 
to allow for a 2- or 5-parameter solution, as appropriate. We then compared the distribution of 
astrometric goodness-of-fit indicators with their expected distributions.

Photometry and astrometry were derived in independent pipelines each of which
could decide to reject or downweight a number of individual observations for a
given source.  We therefore checked if the number of valid observations was
similar in the two pipelines. If more than half of the observations were
rejected, and if the number of valid observations in each pipeline adds up to
less than the total number of observations for the source, there is a problem:
it is not possible to know if the astrometric and
photometric results refer to the same object or e.g.\ to different components
of a binary star. This problem affects less than 9\,000 sources in {\GDR1}
and we expect it to be also solved in future releases\footnote{These stars are not 
flagged, but can be found using \dt{phot\_g\_n\_obs}, \dt{astrometric\_n\_good\_obs\_al}, 
\dt{matched\_observations}}.
% (8708 gaia sources; 6 TGAS sources).

%----------------------------------------
\subsection{Galaxy models}\label{models-intro}
%----------------------------------------

%The Gaia mission provides an unprecedented amount of data. Although many stars observed by Gaia have already been observed in other surveys, the astrometric parameters are most often totally new. Moreover parallaxes have been obtained up to now for a limited number of stars although slightly larger numbers of distances have been estimated using other methods. Hence, the validation of the proper motions and parallaxes of Gaia cannot be uniquely done by comparison to existing data.

Models contain a summary of our present knowledge about the stars in the Milky Way. This knowledge is obviously imperfect and one expects that many of the discrepancies between models and real Gaia data to be due to the models themselves. However, at the level of our current knowledge, if a model performs with a satisfactory accuracy compared to existing data, it can be used for Gaia validation (at the level of this accuracy). This is what we have done in the set of tests based on models. These tests \afterReferee{may} supersede the validation \beforeReferee{with existing}\afterReferee{using external} data in regions of the sky where data are too scarce, or in magnitude ranges where existing data are not accurate enough or incomplete, or in case they do not exist in large portions of sky (such as e.g. parallaxes).

On {\GDR1}, three kinds of tests have been performed: tests on stellar densities, tests on proper motions, and tests on parallaxes. In all tests we analysed the distribution on the sky of the model densities and of the statistical distribution of astrometric parameters (proper motions and parallaxes) and compared them with Gaia data. In order to establish a threshold for test results we compared the model with previous catalogues on portions of sky when available. For this first data release only the Besan\c{c}on Galactic Model~\citep{bgm} has been used for comparisons with Gaia data. 

%A detailed description of the models and of the method to use them is given Appendix \ref{app:models}.


