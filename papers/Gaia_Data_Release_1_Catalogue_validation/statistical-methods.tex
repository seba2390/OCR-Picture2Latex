
%+++++++++++++++++++++++++++++++++++++++++++++++++++++++++++++++++++++++++++
\section{Multidimensional analysis}\label{multidim}
%+++++++++++++++++++++++++++++++++++++++++++++++++++++++++++++++++++++++++++

%----------------------------------------
\subsection{Description of statistical methods}\label{sec:stat}
%----------------------------------------

To understand whether the statistical properties of the {\GDR1} dataset
are consistent with expectations, we compared the distribution of the data  
(and in particular their degree of clustering) to suitable simulations 
for all two-dimensional subspaces. In the case of
TGAS, the comparison data is the simulation designated as
``Simu-AGISLab-CS-DM18.3cor'' (\secref{agislab}), while for {\GDR1} it is GOG18. 

To this end, we use the Kullback-Leibler divergence (KLD):
\begin{equation}
\label{eqn:KLD_wp945}
p_\mathrm{KLD} = -\int {\rm d}^2 x p({\bf x}) \log p({\bf x})/q({\bf x})
\end{equation}
where ${\bf x}$ is a (sub)space of observables, $p({\bf x})$ is the
distribution of the observables in the dataset, and $q({\bf x})$ is
some comparison distribution. When $q({\bf x}) = \Pi_i p_i(x_i)$,
i.e. the product of the marginalized 1D distribution of each of the
observables, the KLD gives the mutual information. This expression
shows that the mutual information is sensitive to clustering or
correlations in the dataset, with a high degree leading to large values
while in their absence $p_\mathrm{KLD}$ would be zero.

We thus computed $p_\mathrm{KLD}$ for more than 300 subspaces for the
data, as well as for the simulations. In both cases, we used a range
for the observables defined by the data after 3-$\sigma$ clipping the
top and bottom regions. Since the simulated and the observed data can
have different distributions without this necessarily implying a
problem in the data, we prefered to work with the relative mutual information rankings. If
the structure is similar in data and simulations, we expect the
rankings to cluster around the one-to-one line, while if a subspace
shows very different rankings this would imply very
different distributions. Such a subspace (or observable) is flagged
for further inspection. This is important since the number of
subspaces is very large.

The comparison to the simulations is sensitive to global issues
(across the whole sky), while there could potentially be systematic
problems in the data restricted to small localized regions of the
sky. Therefore, we also compared the values of the mutual information
obtained for different regions of the sky (e.g. symmetric with respect
to the Galactic plane) and with similar number of observations.

%----------------------------------------
\subsection{Results from the KLD statistical methods}\label{sec:res_stat}
%----------------------------------------

\subsubsection{TGAS and comparison to AGISLab simulations}

Figure \ref{fig:rank_wp945} shows the mutual information ranking of
the two-dimensional subspaces from the TGAS data versus the ranking of
the same subspaces in the AGISLab simulation. Most subspaces with
direct observables (e.g. \dt{ra}, \dt{dec}, etc., black points) show
very similar distributions in the data and in the simulations, as
evidenced by their closeness to the 1:1 line.  Subspaces associated to
errors (blue crosses) and to correlations between observables/errors
(magenta circles), tend to deviate more in general. Examples of the
distributions found for some of the subspaces deviating more strongly
(red hexagons in \figref{fig:rank_wp945}) are given in
\figref{fig:subspace-sims-example}.

\begin{figure}
\begin{center}
\includegraphics[width=0.8\columnwidth]{figures-945/mi_tgas2sim-v2a.pdf}
\caption{Ranking of two-dimensional subspaces according to their mutual
information in the TGAS data (x-axis) vs. the simulation (y-axis). The black squares correspond to subspaces formed only from observables, while the blue crosses are those containing an uncertainty, and the 
magenta circles contain a correlation parameter. The red hexagons correspond to the subspaces shown in \figref{fig:subspace-sims-example}.
}\label{fig:rank_wp945}
\end{center}
\end{figure}

\begin{figure}
\begin{center}
\includegraphics[width=0.8\columnwidth]{figures-945/deviant2.pdf}
\caption{Examples of the subspaces showing a strong deviation from the 1:1 expected relation shown in \figref{fig:rank_wp945}, particularly in the astrometric errors (left) and correlations (right) in TGAS (top) compared to those in the simulations (bottom).}\label{fig:subspace-sims-example}
\end{center}
\end{figure}

\subsubsection{TGAS comparison in different sky regions}

Naively, one might expect regions with similar number of observations
to have similar distributions of errors, and if symmetric with respect
to the Galactic plane or centre, perhaps also in the distribution of several
of the observables. To check for the presence of \beforeReferee{systematicities} \afterReferee{systematics} in
the data, we selected 60 regions with a similar \dt{astrometric\_n\_obs\_al} 
(in the range 60 to 140), of which (20) 40 have a (non-)symmetric 
counterpart.  The left panel of \figref{fig:tgas_sky}
shows their distribution in Galactic coordinates. For these regions we
have computed the mutual information and compared the values to their
counterpart. The normalised deviation from the naively
expected 1:1 line is plotted in the right panel of 
\figref{fig:tgas_sky}, and is defined as $ \sum_i |p_{i,KLD} -
p_{i,KLD}^*|/[0.5*(p_{i,KLD} + p_{i,KLD}^*)]$, where $i$ runs through
the various subspaces and $p$ and $p*$ are the mutual information for
the region and its counterpart. Blue and red points correspond to comparisons between symmetric and 
non-symmetric regions respectively. This plot shows that non-symmetric
regions sometimes have \beforeReferee{more} different distributions. 
By dividing the normalised deviation (whose median value is $\sim 30$) 
by the number of subspaces (780 for TGAS) we obtain an estimate
of the average deviation per region. In this way we found that on
average there are 4\% differences in the mutual information between
different regions. Comparison to the results of AGISLab simulations
does not reveal pairs of regions whose mutual information appear to be
very different for specific subspaces.
\begin{figure}
\begin{center}
\includegraphics[height=0.49\columnwidth, width=0.53\columnwidth]{figures-945/tgas_sky_regions}
\includegraphics[height=0.49\columnwidth, width=0.46\columnwidth]{figures-945/Normalised_Average_tgas}
\caption{Left: Distribution of regions for which the mutual information has been computed, where the inset indicates the number 
of observations inside the regions. The regions are circles in $l-\sin b$ space, with the positive $b$ region in solid and its symmetric counterpart in dashed. Regions that are compared and which are not symmetric are connected by a grey line. Right: average deviation of the mutual information between a region and its counterpart, in (red) blue  for (non) symmetric counterparts.}\label{fig:tgas_sky}
\end{center}
\end{figure}

\subsubsection{{\GDR1} comparison to GOG simulations}

In \figref{fig:rank_dr1_wp945} we show the rankings obtained for the
observables and their errors in the full {\GDR1} Catalogue. Because of the smaller number of
observables, only 21 subspaces exist. The relation of
the mutual information in data and simulations is very close to the
1:1 line, implying similar distributions and hence a good
understanding of the data as far as this global statistic can
test. The observables showing the greater deviations are those related
to uncertainties, and this can be understood from the fact that GOG18 models 
the uncertainties expected at the end of mission, rather than those obtained after 14 months of
observations. 

\begin{figure}
\begin{center}
\includegraphics[width=0.8\columnwidth]{figures-945/gogRanking.pdf}
\caption{Ranking of two-dimensional subspaces according to their mutual
information in the {\GDR1} data (x-axis) vs. the GOG simulation (y-axis).}\label{fig:rank_dr1_wp945}
\end{center}
\end{figure}


