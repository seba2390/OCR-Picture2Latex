\subsection{TGAS astrometric precision}\label{sec:precision-plx}

\subsubsection{Internal estimation of the parallax uncertainty.}\label{sec:wp942:parallaxdeconv}

The quasar analysis in \secref{sec:QSO-plx} \beforeReferee{also} allowed to study the parallax 
dispersion. It was found that the robust unit-weight error (the ratio of the observed
dispersion over the standard uncertainty) decreased with magnitude 
from $\sim 1$ down to about 0.8 at $G=20$. 
It would however be difficult to extrapolate this overestimation of the
uncertainties to the much brighter TGAS sources, so this question was studied differently.

The measured TGAS parallax distribution, at least its small and negative tail, can be used to estimate the parallax uncertainties without referring to the formal uncertainty, following the \afterReferee{deconvolution} procedure of \citet{1995A&A...304...61L}. The procedure models the observed distribution as the convolution of a nonparametric true parallax distribution (subject only to the constraint that all true parallaxes are positive) with a Gaussian error kernel. The Gaussian width parameter that gives the best fit to the observed distribution has been adopted as the parallax uncertainty of the sample.

As noted by \citeauthor{1995A&A...304...61L}, the estimated parallax uncertainty is usually biased, and the process of solving for the true parallax distribution, which resembles Lucy-Richardson deconvolution, suffers from overfitting as the number of iterations increases. Both effects need to be controlled. As the parallax distribution of the TGAS sample differs from that of the Hipparcos sample explored in \citet{1995A&A...304...61L}, we performed simulations to determine the bias correction factor and number of iterations to use for TGAS data. The Simu-AGISLab simulated data (\secref{agislab}) were randomly sampled with new errors to produce a realistic data set large enough for testing. We used cross-validation to test the predictive accuracy of the debiased estimates, including uncertainties in the bias correction factor.

Unlike \citeauthor{1995A&A...304...61L}, we found that 2-3 iterations gave much more accurate results than a few dozen, regardless of the sample being studied; the reasons for this discrepancy are not yet clear. We fit arbitrary (nonlinear) functions to the bias correction factor and the accuracy of the debiased parallax uncertainty, enabling prediction of the bias correction to $\sim 8\%$ and of the accuracy of the final parallax uncertainty (i.e., the uncertainty on the uncertainties) to $\sim 20\%$.
We also found that simulation runs with small ($N \sim 100$) or precise ($\sigma_\varpi \sim 0.1 \textrm{ mas}$) data sets behaved very differently from the trends seen for larger or less precise data; presumably the sharp changes at high precisions are related to the parallax distribution assumed for the TGAS catalogue.

\begin{figure}[tbhp]
\centering
%\includegraphics[width=1\columnwidth]{./figures-942/plxdeconv_hip_before.png}
\includegraphics[width=0.49\columnwidth]{./figures-942/plxdeconv_hip.png}
%\includegraphics[width=1\columnwidth]{./figures-942/plxdeconv_tycho2_before.png}
\includegraphics[width=0.49\columnwidth]{./figures-942/plxdeconv_tycho2.png}
\caption{Best-fit uncertainties from deconvolution of parallaxes versus standard uncertainties for TGAS Hipparcos stars (left) and for Tycho-2 (non-Hipparcos) stars (right), with bisector represented. Error bars include all sources of uncertainty, including bias correction.}\label{fig:wp942:parallaxdeconv}
\end{figure}

When modelling the observed parallax distribution, we first corrected all parallaxes for the $-0.04$~mas bias found \secref{sec:QSO-plx}, though analysis with and without the correction gave indistinguishable results. We analysed the TGAS data in bins of standard uncertainty of 0.05 mas width, and separately for each type of astrometric solution, in case each group had different error properties.

We show in \figref{fig:wp942:parallaxdeconv} the results of modelling the TGAS 
parallaxes dispersion compared to the standard uncertainties. 
As can be seen the TGAS standard uncertainties $\sigma_\varpi$ on parallaxes 
appear accurate. More quantitatively, a weighted fit for Hipparcos stars is 
$(0.980{\scriptstyle\pm 0.135})\sigma_\varpi - 0.003{\scriptstyle\pm 0.062}$,
while for Tycho-2 stars it is 
$(0.973{\scriptstyle\pm 0.024})\sigma_\varpi +  0.011{\scriptstyle\pm 0.011}$,
both being consistent with a unit-weight error = 1. 
Assuming a unit-weight error = 1, and fitting only for an extra dispersion 
(quadratically added) gives $-0.19\pm 0.02$ mas for Hipparcos and 
$-0.11\pm 0.01$ mas for Tycho-2. This is consistent with the median value obtained 
with external estimates, \tabref{tab:cu9val_wp944_summaryplx}, and it shows
that the standard uncertainties appear (except probably for the most precise 
parallaxes) slightly pessimistic. 

%We found only a few minor issues in the Gaia-DR1 astrometry. The values of \texttt{astrometricExcessNoise} and \texttt{astrometricExcessNoiseSig} reach high levels, ones expected of only a thousand sources, in 205 million sources, including nearly the entire TGAS sample. These large values reflect the large errors introduced by the preliminary attitude solution for the Gaia spacecraft; a better solution will be used in future releases \citep{DPACP-14}. In addition, 4288 sources have positions based on only two one-dimensional measurements, providing an astrometric solution with no degrees of freedom. These minimally constrained solutions are expected to go away as more data are collected.
%paragraph moved to 3.2 Data integrity and consistency

\subsubsection{Comparison with external astrometric data}\label{sec:wp944_astrom}

The comparison of Gaia results with external astrometric data is not straightforward as Gaia will provide the most accurate and the most numerous astrometric data ever produced, at least in the optical domain. However the consistency between Gaia data and carefully selected external astrometric data might be important in order to detect any statistical misbehaviour in one or the other source of data, including Gaia.

Only positions from the Hipparcos or Tycho-2 catalogues have been used as priors in TGAS. The parallaxes and/or the proper motions have not been used, so this ensures that the comparison with TGAS parallaxes and proper motions is meaningful, as they are independent from those of Hipparcos and Tycho-2. Note that another independent comparison with those catalogues is presented in Annex C of \cite{DPACP-14}.
%For the validation of DR1, and especially of TGAS, with respect to the catalogues or compilations presented below, we are looking for possible trends of the normalized residuals (Gaia-external) with magnitude, colour, position on the sky or proper motion.
%Each of the catalogue is used to compare, as much as possible, the positions, parallaxes, proper motions and reference frame. 
For the Hipparcos and Tycho-2 proper motion tests, the global rotation between the reference frames of Hipparcos and TGAS derived in \cite{DPACP-14} has been applied. A possible (residual-)rotation has been checked. 
For each catalogue, the distribution of the normalized residuals (Gaia-External) of each parameter $R_\mathcal{N}$, e.g. for the parallax $R_\mathcal{N}$=(\parallax$_G$-\parallax$_{E}$)/$({\sigma_{\text{\parallax}_G}^2+\sigma_{\text{\parallax}_{E}}^2})^{1/2}$, has been checked to be consistent with a normal distribution, and correlations of those residuals with magnitude, colour and sky position have been checked too. 

A $\chi^2$ test has been also performed on combined parameters $X$ ($X$ being the positions, or the proper motions, or the parallaxes and proper motions) using the full covariance matrix of both the external ($\Sigma_E$) and the Gaia ($\Sigma_G$) catalogues to compute the normalized residuals $R_{\chi}=(X_G-X_E)^T (\Sigma_G+\Sigma_E)^{-1} (X_G-X_E)$  and their distribution has been tested to follow a chi-squared distribution with $n$ degrees of freedom, $n$ being the number of parameters tested (e.g. 2 for {\GDR1} positions, 2 for TGAS proper motions and 3 for TGAS parallaxes and proper-motions). Similarly to the one dimensional case, correlations with magnitude and colour, and sky distribution of those residuals have also been tested. 

In all the tests, we used a p-value limit of 0.01 (e.g. we indicate that we find a bias, extra variance or a correlation with a confidence level higher than 99\%). For the normalized residuals using individual parameter ($R_\mathcal{N}$), this level corresponds to $\vert R_\mathcal{N} \vert>2.6$, while for the $\chi^2$ residuals on 2 components this level corresponds to $R_{\chi}>9.21$.

%\subsubsection{TGAS parallaxes and proper motions}
%----------------------------------------

For the validation of TGAS, the following astrometric catalogues have been considered:
\paragraph{Hipparcos new reduction.} %\citep{2007A&A...474..653V}. 
A selection of {\it well behaved} Hipparcos stars has been done using the 5-parameter solution type with a good astrometric solution (goodness of fit $\vert F2 \vert<5$), and without any binary flag indicated in the literature, mainly from WDS \citep{WDS}, CCDM \citep[Catalogue of the Components of Double and Multiple Stars,][]{2000A&A...363..991D} and SB9 \citep[9th Catalogue of Spectroscopic Binary Orbits,][]{2004A&A...424..727P}\beforeReferee{)}. Stars also included in Tycho-2 \beforeReferee{are}\afterReferee{were} kept only if the proper motions from Hipparcos \beforeReferee{are}\afterReferee{were} consistent with those of Tycho-2 (rejection p-value: 0.001). The resulting sample includes 93\,802 well behaved stars, against which both the parallaxes and proper motions of TGAS have been tested. 

A global parallax zero point difference between {\gaia} and Hipparcos of $-0.094\pm 0.004$~mas \beforeReferee{is}\afterReferee{was} found\footnote{If we 
assume, as shown \secref{sec:QSO-plx}, a $-0.04\pm 0.003$~mas zero-point for {\GDR1}, 
an estimate of the Hipparcos zero-point (new reduction) would then be $+0.054\pm 0.005$~mas. This
would also be the zero-point of the first Hipparcos reduction as the average parallax difference between both reductions
is about 0. This value is then marginally consistent with the estimation done two decades ago 
\cite[$-0.02\pm 0.06$~mas,][]{1995A&A...304...52A} with preliminary Hipparcos data, and to what was estimated with the published data, $-0.05\pm 0.05$~mas \cite[][Vol III, Chap. 20.]{1997ESASP1200.....E}}.
The under-estimation of the standard uncertainties for both parallax and proper motions  is significative (extra dispersion of 0.6 mas). Small variations of the parallax and proper motion residuals is seen with sky position  (\figref{fig:cu9val_wp944_hipplx}) and magnitude (smaller than 0.1 mas, most probably due to the gates).

\begin{figure}
    \begin{center}
        \includegraphics[width=0.8\columnwidth]{./figures-944/hipPlxEcl.png}
        \caption[TGAS versus Hipparcos Parallaxes]{Sky variation of the normalized residuals $R_\mathcal{N}$ of the TGAS versus Hipparcos parallaxes in ecliptic coordinates. Although correlation with sky position is significant, no sky region indicate a normalized residual larger than 2.6.}
        \label{fig:cu9val_wp944_hipplx} 
    \end{center}
\end{figure}

The $\chi^2$ test with Hipparcos, using either parallax and proper motions or proper motions only, shows stronger variations across the sky (\figref{fig:cu9val_wp944_hipPMchi2}a), with areas showing a mean residual $R_\chi$ over 9.21 (the p-value 0.01 limit)  while the residuals of parallax or proper motions components individually stay below the p-value limit ($\vert R_\mathcal{N} \vert<2.6$). 11\% of the sources have a $\chi^2$ p-value$<$0.01, e.g. 11 times more than expected.
Moreover a strong correlation between $R_\chi$ and \gmag~magnitude is observed  (\figref{fig:cu9val_wp944_hipPMchi2}b). This behaviour of $R_\chi$ is also seen with the quasar positions (\secref{sec:DR1pos}), indicating potential issues with the covariance matrix. Those could be due to extra correlations introduced by the attitude and calibration models not taken into account in the provided covariance matrix \citep{2012A&A...543A..14H}.

\begin{figure}
    \begin{center}
        \includegraphics[width=0.8\columnwidth]{./figures-944/hipPMchi2Ecl.pdf}
        %\vspace{0.5cm}
        \includegraphics[width=6cm]{./figures-944/hipPMchi2Mag.pdf}
        \caption[TGAS versus Hipparcos astrometry]{TGAS proper motions versus Hipparcos $\chi^2$ test: the residuals $R_\chi$ should follow a $\chi^2$ of 2 degrees of freedom. Sky regions with a significant residual ($R_\chi>9.21$) are highlighted in black.  a) Sky variation in ecliptic coordinates. b) correlation with the \gmag~magnitude; the dotted lines correspond to the 1$\sigma$ confidence interval. }
        \label{fig:cu9val_wp944_hipPMchi2} 
    \end{center}
\end{figure}

\paragraph{Hipparcos and Tycho-2 stars with inconsistent proper motions.}
The second sample includes the 1574 stars previously eliminated because of the inconsistency between Hipparcos and Tycho-2 proper motions. A specific test has been done on those stars: most of them are expected to be long period binaries not detected in Hipparcos, and for which the longer time baseline of Tycho-2 could have provided a more accurate value. 

The TGAS solution also has a long time baseline thanks to its Hipparcos/Tycho input position. It has been therefore tested that the TGAS solution for those stars is globally closer to the Tycho-2 solution than to the Hipparcos solution. 
This is indeed the case with 7\% of those TGAS sources being outliers versus the Tycho-2 solution while 50\% are outliers versus the Hipparcos solution. 

\paragraph{Tycho-2.} %\citep{2000A&A...355L..27H}. 
Only Tycho-2 stars with a normal astrometric treatment (no double star with Tycho-2 separate entries, no close known or suspected double star with {\it photocentre treatment}) have been used in this test. Due to the different priors used for the Hipparcos and Tycho-2 stars (Hipparcos positions at the Hipparcos mean epoch, J1991.25, for Hipparcos stars; Tycho-2 positions at the effective Tycho-2 observation epoch, taken to be the mean of the $\alpha$ and $\delta$ epochs, for Tycho-2 stars) in the TGAS solution, the test has been done once for the Tycho-2 sources not in Hipparcos and once for the Tycho-2 sources in the {\it well behaved} Hipparcos sub-sample described above.  

For the Tycho-2 sources in the Hipparcos well-behaved sub-sample an under-estimation of the standard uncertainties is seen (extra dispersion of 0.6\masyr\ similar to what is found with the Hipparcos sample) and a correlation with magnitude and colour is found with an amplitude smaller than 0.1 mas (the residuals increasing with magnitude and colour). 
 For the Tycho-2 sources not in Hipparcos, a strong variation of the residuals is seen with sky position (\figref{fig:cu9val_wp944_tycPM}) with features parallel to the equatorial declinations which corresponds to the zones of the Astrographic Catalogue used to derive the Tycho-2 proper motions. A very large extra dispersion of 1.8\masyr\ is also observed. We most probably see here the defaults of the Tycho-2 proper motions. A rotation smaller than 0.2\masyr\ is also observed. 
 
\begin{figure}
    \begin{center}
        \includegraphics[width=0.8\columnwidth]{./figures-944/tycPMchi2.png}
        \caption[TGAS versus Tycho-2 proper motions]{Sky variation of the residuals $R_\chi$ of the TGAS versus Tycho-2 parallaxes in equatorial coordinates. }
        \label{fig:cu9val_wp944_tycPM} 
    \end{center}
\end{figure}
 
\paragraph{VLBI compilation.} VLBI data have mainly been obtained from the USA VLBA, the Japanese VERA and the European EVN: 90 proper motions and 44 parallaxes (including respectively 70 and 30 stars in Tycho-2). Over the years, with increasing baseline length and better calibration of the ionospheric and tropospheric delays, astrometric accuracy using VLBI at centimetre wavelengths is approaching ${\sim}$10~$\mu$as for parallaxes and ${\sim}$1~$\mu$as yr$^{-1}$ for proper motions \citep[][and reference therein]{2014ARA&A..52..339R}. For proper motions, only those with a mean epoch $>$ 2000 were considered as calibration techniques improved drastically at that epoch, especially with new detailed maps of ionospheric delay. The compilation covers all stellar sources for which trigonometric parallaxes and proper motions have been obtained from VLBI astrometry \citep[as quoted in the review of][]{2014ARA&A..52..339R}, but also stars with only proper motions obtained from VLBI positions \citep{2007AJ....133..906B}, and VLBI proper motions of X-ray binaries with an estimation of distance obtained by other means \citep{2014PASA...31...16M}.  

36 stars of this compilation are present in TGAS, including 9 with parallax information. All the tests associated to this catalogue pass (parallax and proper motion bias, variance, correlations), with the exception of the full covariance matrix \chitwo test which indicates that half the stars with both parallax and proper-motion information available (assuming no correlation for the VLBI parameters as this information is rarely available) have a \chitwo p-value larger than 0.01. 
%A small rotation of the proper motion reference system smaller than 5~mas is also detected.

\paragraph{HST compilation.} The Fine Guidance Sensors (FGS) on Hubble Space Telescope have produced high accuracy trigonometric parallaxes of astrophysically interesting objects such as Cepheids, RR-Lyrae, novae, cataclysmic variables or cluster members  \citep{2015IAUGA..2257159B, 2007AJ....133.1810B}. The FGS field of view is small and the parallaxes of target stars have been measured with respect to reference stars which have their own parallaxes estimated by spectro-photomometric measurements. The correction to absolute leads to a median error of absolute parallaxes announced to be 0.2 mas. The present compilation covers 69 stars with parallaxes (including 43 in Tycho-2) and, for about a third of them, proper motions, published up to end 2015. 

19 stars of this compilation are present in TGAS, passing all the tests. 

\paragraph{RECONS.} The REsearch Consortium On Nearby Stars (RECONS, www.recons.org) has built a database of all systems estimated to be closer than 25~pc (parallaxes greater than 40 mas with errors smaller than 10 mas). We have used the database as published on 1st April 2015 \citep{2015IAUGA..2253773H}, leading to 348 stars (including 27 in Tycho-2) with trigonometric parallaxes.

13 stars of this compilation are present in TGAS, passing all the tests. 
 



\begin{table*}
\caption{Comparison in the LMC and SMC of the correlations between astrometric
parameters: the median of the standard correlations given in the Catalogue appear consistent with the empirical ones
computed with the astrometric residuals.}\label{tab:test_correl}
\centering\small
\begin{tabular}{lcccccc} 
\hline\hline
 && \multicolumn{2}{c}{LMC} && \multicolumn{2}{c}{SMC} \\
correlation &~~~& predicted & observed &~~~& predicted & observed \\
\hline
$\rho(\varpi,\mu_{\alpha *})$ && +0.7\afterReferee{47} $\pm$ 0.013 & +0.77\afterReferee{4} $\pm$ 0.063  && -0.2\afterReferee{03} $\pm$ 0.023 & -0.3\afterReferee{56} $\pm$ 0.139 \\
$\rho(\varpi,\mu_{\delta})$ && -0.68\afterReferee{0} $\pm$ 0.012 & -0.42\afterReferee{4} $\pm$ 0.09\afterReferee{0} && -0.8\afterReferee{01} $\pm$ 0.005 & -0.6\afterReferee{75} $\pm$ 0.11\afterReferee{0} \\
$\rho(\mu_{\alpha *},\mu_{\delta})$ && -0.31\afterReferee{1} $\pm$ 0.033 & -0.22\afterReferee{0} $\pm$ 0.097 && -0.1\afterReferee{17} $\pm$ 0.028 & -0.17\afterReferee{2} $\pm$ 0.147 \\
\hline
\end{tabular}
\end{table*}

\subsubsection{Validation of the astrometric correlations.}\label{sec:extcorrel}
As shown above and stressed in \secref{sec:effect-correlations}, the correlation between
astrometric parameters should not be neglected when computing covariance matrices.
After having tested the formal uncertainties above, checking whether these correlations
are accurate is also needed.

It is usually difficult to compute these correlations but there are at least two
different local areas, the LMC and SMC, where average proper motions and parallaxes
are already known to a sufficient precision. The astrometric errors can thus be computed
from the residuals between Gaia proper motions and parallaxes and the external estimation.
We used the Tycho-2 stars only (not the Hipparcos ones) as the internal dispersion 
of the proper motions can be neglected compared to the astrometric uncertainties 
($\sim 1$\masyr) in the former case, not in the latter.

Using the star list indicated in \secref{sec:LMCSMC} restricted to Tycho-2 stars, we rejected 
all sources having in absolute value one of these residuals 3 times larger than their formal
uncertainty, to avoid any contamination by field stars.  

In each Magellanic Cloud, we then computed the medians of the formal correlations 
as given in the Catalogue, and we estimated the actual ones computing the
empirical correlation coefficients between residuals. As shown in \tabref{tab:test_correl},
the various estimations are consistent with the predictions at a p-value=0.01.
The expected internal variations of the proper motions inside the Clouds also explain
the large dispersion.

Although this test has been done on two regions only, it is reasonable to consider 
that the correlations between astrometric parameters, as given in the Catalogue, 
are statistically reliable.

There is however an important caveat. We did not discuss explicitly in this paper the 
angular correlations between stars which we know exist (see \figref{fig:cu9val_942_QSOKsky}).
In principle, this section should have compared the full observed covariance-matrix
(of all stars $\times$ 5 astrometric parameters) to the predicted one, but it is much too
difficult to predict the correlations between stars for now. It is thus possible that
the local comparison done here shows an agreement while a whole sky comparison would 
disagree. 



