%----------------------------------------
\subsubsection{Comparisons with proper motion from distant open clusters}\label{sssec:cu9val_ocpm}
%----------------------------------------
The aim of this test was two-fold: assessing the internal consistency of proper motions within stellar clusters, and looking for biases and systematics by testing the proper motions zero-point against literature values.

Following the open cluster selection described in \secref{sssec:cu9val_ocpar}, we computed the difference between the proper motion of each star and the reference value for its cluster listed in the MWSC catalogue \citep{2013A&A...558A..53K} and in \citet{2014A&A...564A..79D}. This procedure is designed to take into account possible small-scale correlations between parameters.
For each cluster, we obtained a mean value $\Delta$ of this difference, and its associated error $\sigma$. We flagged the objects for which the difference to the reference value is too large to be explained by the nominal uncertainties, as well as those with discrepant small or large internal dispersions. The test also looks for trends in proper motions against magnitude and colour.

A global zero-point test was performed from the $\Delta$ values obtained for individual clusters, restraining the sample to objects distant enough so that their internal dispersion in proper motions is negligible compared to the uncertainty on the proper motion of individual stars. The expected all-sky average of this quantity should be zero if no bias is present. A clustering test allows us to verify if outliers are randomly distributed, or if they cluster in problematic areas in the sky.

We retained 20 clusters that are sufficiently distant and present secure membership for more than 10 stars.
Scaling the difference $\Delta$ according to the total uncertainty (standard uncertainties listed in TGAS and uncertainty on the literature value), we found no significant differences in proper motions. In units of uncertainty, the all-sky zero-point of $\mu_{\alpha *}$ is $+0.04\pm0.21$, and for $\mu_{\delta}$: $+0.12\pm0.26$. We also found that outliers appear homogeneously distributed across the sky.


%----------------------------------------
\subsubsection{Specific tests on known double and multiple systems} 
%----------------------------------------

In addition to the above general tests, a specific test has also been done on known double and multiple systems from the Hipparcos new reduction (HIP2) and the TDSC in order to detect any possible bias between single and non-single stars. For non-Hipparcos systems, we use the component designation given in the TDSC, m\_TDSC, to distinguish between primary components (A or Aa), unresolved systems (AB), and secondary components (all other entries in TDSC). For Hipparcos systems, four categories with increasing periods were distinguished: stochastic solutions (short period, solution type Sn = 1 modulo 10 in HIP2), acceleration stars with 7- or 9-parameter solution (intermediate period, Sn = 7 or 9 modulo 10 in HIP2), secondary component (long period, separation $\rho>0$ as provided in the original Hipparcos catalogue), other double stars (the remaining non single stars). 
The characteristics of those Hipparcos and Tycho systems were compared to those of the {\it well behaved} Hipparcos sample described in \secref{sec:wp944_astrom}, adding the extra criterion of passing the $\chi^2$ test comparing the parallax and proper motion between Hipparcos and TGAS. Of course, within these ``{\it single star}'' samples, many unknown unresolved binaries may hide.
%Note: GoF filtre n'enleve plus aucune etoile dans DR1, critere enleve de la description (et du code). 

A difference in behaviour between those different subsets with respect to the {\it single star} samples was looked for, using various parameters: the parallax and proper motion residuals (TGAS-external), and the TGAS errors, goodness of fit and excess noise (source modeling errors). Mainly acceleration solutions are expected to show large discrepancies between their proper motions in TGAS and those from Hipparcos or TDSC. 
Another source of discrepancy may be the fictitious difference created by the comparison of TGAS and Hipparcos proper motions for close systems for which only the photocentre was observed by Hipparcos. 
For example, it was found that the excess noise, which is about 0.5 mas on the average except for very bright stars
(\secref{sec:quality_indic}) did not exhibit significant differences between
single, primaries and secondaries; on the contrary unresolved systems had significantly 
degraded solutions with about 1.2 mas excess noise on the average in the $7\lesssim G\lesssim 12$ mag range. 

Several other tests have also been done on secondary components, checking whether the separation or position angle 
with respect to the primary component had no adverse effect. In the past, during the validation 
of early preliminary Gaia data, it had been found that proper motions of many secondaries 
below 2\arcsec\ separation had a large discrepancy (up to a 80{\masyr} amplitude) compared to TDSC. 
Noting that 2\arcsec\ divided by the time span between Hipparcos
and Gaia (2015-1991) gives about 80{\masyr}, it was deduced that the cross-matching of some close double
stars had been deficient: \afterReferee{most probably} the wrong first epoch position had been used for the Tycho-Gaia astrometric
solution (TGAS), e.g. the Tycho position for the A component was associated to the observations of the B component
because it was closer to it, depending on the position angle of the system, and {\it vice versa}. 

\beforeReferee{Since this preliminary solution, stars with a parallax uncertainty 
larger than 1 mas had no more been considered for TGAS and } 
\afterReferee{Unlike the preliminary Gaia data, the TGAS solution disregarded
stars with a parallax uncertainty larger than 1 mas, which}
received a 2 parameter astrometric solution instead. However, for close 
double stars which remain in TGAS, and as can be seen in
\figref{fig:cu9val_wp944_muTDSC}, there are still several pairs mis-identified,
and it is unclear whether the mis-identification comes from Gaia or Tycho in the
first place.
Using this figure, it should be easy for the user to detect and reject the
bad astrometric solutions for pairs (both components) depending on a) separation 
below 2\arcsec, b) position angle in the bad range, c) proper motions differences 
above uncertainties and possibly d) large excess noise.

\begin{figure}
    \begin{center}
        \includegraphics[width=0.49\columnwidth]{figures-944/TDSC-TGAS-mua.pdf}
        \includegraphics[width=0.49\columnwidth]{figures-944/TDSC-TGAS-mud.pdf}
        \caption{Difference between TGAS and TDSC proper motions ({\masyr}) as a function of position angle $\theta$ (deg) 
        for secondary components of multiple systems with $\rho<2$\arcsec; $\mu_{\alpha *}$ (left) 
        and $\mu_{\delta}$ (right).}
        \label{fig:cu9val_wp944_muTDSC} 
    \end{center}
\end{figure}


%----------------------------------------
\subsection{TGAS validation from the comparison with Galaxy models}
%----------------------------------------

Two Besan\c{c}on Galactic Model simulations have been run for TGAS validation, using slightly modified models, both in density laws and kinematics, in order to verify the dependency of the model inputs to the validation. Both simulations were done with the model described in \cite{2014A&A...564A.102C} where the evolutionary scheme has been updated, as well as the IMF, SFR and evolutionary tracks. Moreover, the thick disc and halo populations have been updated, following \cite{2014A&A...569A..13R}, with new density laws. Concerning the kinematics, we used alternatively the standard model kinematics \citep{bgm}, hereafter BGMBTG2, and a revised kinematics from an analysis of RAVE survey (Robin et al, in prep), hereafter called BGMBTG4. BGMBTG2 and BGMBTG4 also differ by several model parameters such as the extinction model and thin disc scale length. 

The use of two different models allows to evaluate what is due to acceptable model variations in the parallax and proper motion distributions. Model parameters are described in Mor et al. 2015 (internal Gaia documentation GAIA-C9-TN-UB-RMC-001). The simulations contain binary systems where the second component is merged with the primary when the separation is smaller than 0.8 arcsec, the estimated resolution of the Tycho-2 catalogue. We also introduced the uncertainties expected in TGAS after 6 months of Gaia observations, following the recipes published in September 2014 after commissioning phase\footnote{\scriptsize\url{http://www.cosmos.esa.int/web/gaia/science-performance}}.

The validation was done by comparing the proper motion and parallax distributions in TGAS catalogue to simulated ones. The sky \beforeReferee{is}\afterReferee{was} divided in healpix rings with healpixsize 20, giving a solid angle of 8.5943 square degree in each bin, and 4800 bins in total. Then bins were grouped in rings of equal galactic latitudes in order to compare the values between latitude rings. Finally, we consider 5 latitude intervals (-90 to -70\deg, -70 to -20\deg, -20 to 20\deg, 20 to 70\deg, and 70 to 90\deg) in order to analyse the characteristics of the distributions in the plane, at intermediate latitudes, and at the poles separately. For each region of the sky considered, we compared the mean and standard deviation between the model and the data for the parallax and proper motion distributions. 

\subsubsection{Parallaxes}

Figure \ref{fig:Mean_parallax} shows the mean parallax differences between the BGMBTG2 simulation and TGAS data, as a function of latitude rings. Each panel corresponds to a magnitude interval of 0.5 mag width, starting at \vt=9. 

% following Anthony advice, add a word of precaution
From these comparisons we notice that, for bright stars, the mean parallax differences seem to suffer from a slight zero point offset, which also depends slightly on Galactic latitude. 
%This positional dependency is most probably related to the scanning law and the small number of observations near the ecliptic. 
The systematic shift between models and TGAS data is of the order or less than 1 mas depending on the region of the sky, 
%as also seen in the zero point error estimated from quasars (see \secref{sec:QSO-plx}), 
but it is unclear whether this originates from the data or the model.

\begin{figure*}
\begin{center}
\includegraphics[width=3.5cm, height=4cm]{figures-943/Res-40-10-Ring-mean-1.pdf}
\includegraphics[width=3.5cm, height=4cm]{figures-943/Res-40-10-Ring-mean-2.pdf}
\includegraphics[width=3.5cm, height=4cm]{figures-943/Res-40-10-Ring-mean-3.pdf}
\includegraphics[width=3.5cm, height=4cm]{figures-943/Res-40-10-Ring-mean-4.pdf}
\includegraphics[width=3.5cm, height=4cm]{figures-943/Res-40-10-Ring-mean-5.pdf}
\caption{Mean difference in parallax in mas between BGMBTG2 model simulation and TGAS data, in different rings of latitude, for five magnitude bins in \vt\ from left to right, from 9-9.5 (left) to 11-11.5 (right).}
\label{fig:Mean_parallax}
\end{center}
\end{figure*}

In the standard deviation in parallax, the comparison with models shows a good agreement. The dominant factor in the simulation of the parallax standard deviation is the error model assumed to simulate the errors added in the BGM simulations. The good agreement implies that the dependency of the parallax errors on magnitude and latitude is in agreement with the expectations.


\subsubsection{Proper motions}

Figure \ref{fig:Mean_properMotion_l} shows the differences in the mean proper motion along Galactic longitude ($\mu_{l *}$) between the BGMBTG2 and BGMBTG4 simulations and TGAS data, as a function of latitude healpix rings. Each panel corresponds to a magnitude interval of 0.5 mag width, starting at \vt=9. Both models show similar difference distributions with the data. 

Figure \ref{fig:Mean_properMotion_b} shows the differences in the mean proper motion along Galactic latitude ($\mu_b$). The zero point differences between models and data are at the level of the differences between the two models at bright magnitudes. However systematic differences appear in the faintest magnitude bins 
which again can be attributed either to the model or
related to large correlated errors in some regions of the ecliptic plane due to the scanning law. Notice also the higher noise level at the Galactic poles due to the smaller number of sources. 

\begin{figure*}
\begin{center}
\includegraphics[width=3.5cm, height=4cm]{figures-943/TGAS-result-7Jun16/TGAS-BTG2-BTG4-Ring-mul-mean-1.pdf}
\includegraphics[width=3.5cm, height=4cm]{figures-943/TGAS-result-7Jun16/TGAS-BTG2-BTG4-Ring-mul-mean-2.pdf}
\includegraphics[width=3.5cm, height=4cm]{figures-943/TGAS-result-7Jun16/TGAS-BTG2-BTG4-Ring-mul-mean-3.pdf}
\includegraphics[width=3.5cm, height=4cm]{figures-943/TGAS-result-7Jun16/TGAS-BTG2-BTG4-Ring-mul-mean-4.pdf}
\includegraphics[width=3.5cm, height=4cm]{figures-943/TGAS-result-7Jun16/TGAS-BTG2-BTG4-Ring-mul-mean-5.pdf}
\caption{Difference in mean proper motion along Galactic longitude ($\mu_{l *}$) between TGAS data and two models: BGMBTG2 (red), BGMBTG4 (blue), in different magnitude intervals, between $V_T$=9 (left) to $V_T$=11.5 (right) by steps of 0.5 magnitude.  }
\label{fig:Mean_properMotion_l}
\end{center}
\end{figure*}

\begin{figure*}
\begin{center}
\includegraphics[width=3.5cm, height=4cm]{figures-943/TGAS-result-7Jun16/TGAS-BTG2-BTG4-Ring-mub-mean-1.pdf}
\includegraphics[width=3.5cm, height=4cm]{figures-943/TGAS-result-7Jun16/TGAS-BTG2-BTG4-Ring-mub-mean-2.pdf}
\includegraphics[width=3.5cm, height=4cm]{figures-943/TGAS-result-7Jun16/TGAS-BTG2-BTG4-Ring-mub-mean-3.pdf}
\includegraphics[width=3.5cm, height=4cm]{figures-943/TGAS-result-7Jun16/TGAS-BTG2-BTG4-Ring-mub-mean-4.pdf}
\includegraphics[width=3.5cm, height=4cm]{figures-943/TGAS-result-7Jun16/TGAS-BTG2-BTG4-Ring-mub-mean-5.pdf}
\caption{Difference in mean proper motion along Galactic latitude ($\mu_b$) between TGAS data and two models: BGMBTG2 (red), BGMBTG4 (blue), in different magnitude intervals, between $V_T$=9 (left) to $V_T$=11.5 (right) by steps of 0.5 magnitude.}
\label{fig:Mean_properMotion_b}
\end{center}
\end{figure*}


%----------------------------------------
\subsection{{\GDR1} positions and reference frame}\label{sec:DR1pos}
%----------------------------------------

For the billion+ sources of {\GDR1}, the only astrometric parameters available are the two components of the position. The astrometry of the secondary DR1 dataset has been compared with the following catalogues:
 \paragraph{URAT1 star positions} \citep{urat1}. URAT1 is a catalogue containing stellar positions of 228\,276\,482 stars down to $R$=18.5, at epochs ranging from 2012.3 to 2014.6 with typical standard errors of 10--30 mas. Only stars distant enough to have a proper motion smaller than 100\masyr\ even assuming a tangential velocity of 500\kms\ were used. The Gaia-ESO and LAMOST surveys have been used to estimate the spectrophotometric distances of those stars (see method in \secref{sec:distantstars}), leading respectively to samples of 5\,384 and 136\,234 stars. The cross-match between DR1, including TGAS, with URAT1 was done by position, with multiple detections within 0.2\arcsec~removed. 

   %results
Correlations with magnitude, colours, sky positions are seen, but overall this effect stays within an amplitude of 30~mas.

 \paragraph{ICRF2 QSO positions} \citep{2015AJ....150...58F}. The second realisation of the International Celestial Reference Frame (ICRF2) contains very precise positions of 3\,414 compact radio astronomical objects. The positional noise floor is announced to be  of about 40\muas ~and the directional stability of the frame axes of about 10\muas. A least-square method using the covariance matrix of both catalogues allows to estimate the rotation and dipolar deformation between the ICRF2 and the Gaia reference frames. Correlations of differences between {\GDR1} and ICRF2 positions with other parameters such as magnitude and colours were tested, following the same methods as described above for stars. 

   % results
The test has been done both on the auxiliary quasar solution and on the main {\GDR1} secondary solution, with the same conclusions so that only the numbers corresponding to {\GDR1} are provided below \citep[note that the priors used in their astrometric reduction are different,][]{DPACP-14,DPACP-26}.  
2\,292 ICRF2 quasars are found in {\GDR1} within a 0.1\arcsec\ radius. As expected by construction \citep{DPACP-14}, no rotation versus the ICRF2 is found, but a deformation (glide) is detected, lower than 0.2~mas. It should be noted that this deformation is not significant anymore if the cross-match radius is increased from 0.1 to 0.5\arcsec\ which adds 15 sources. The residuals of the position differences normalized using the covariance matrix of both {\GDR1} and the ICRF2 $R_\chi$ show a too large number of outliers (10\% with a p-value $<0.01$, i.e. 10 times more than expected) and $R_\chi$ is correlated both with the magnitude and with the number of observations. This behaviour of $R_\chi$ is the same as the one observed in the comparison with Hipparcos (\secref{sec:wp944_astrom}). 
   
   % Is this figure needed? 
%    \begin{figure}
%     \begin{center}
%         \includegraphics[width=0.5\columnwidth]{cu9/Figures/wp944/qsoPoschi2.png}
%         \caption[QSO residuals]{Variation of the $R_\chi$ of the DR1 versus ICRF2 positions as a fonction of \gmag~magnitude. }
%         \label{fig:wp944_qsochi2} 
%         %TODO fig update 
%     \end{center}
%   \end{figure}

More anecdotally, four known quasars were included in the Hipparcos and Tycho-2 catalogues (HIP  60936 = 3C273, TYC 9365-284-1, TYC 259-212-1, TYC 3017-939-1). Only the first and the last ones are present in TGAS. 3C273 has an astrometry consistent with null parallax and proper motion, but this is not the case for the Tycho-2 AGN, TYC 3017-939-1 ($R_\chi$=25.3).



