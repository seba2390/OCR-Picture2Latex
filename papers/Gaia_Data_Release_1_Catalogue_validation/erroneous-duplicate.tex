%+++++++++++++++++++++++++++++++++++++++++++++++++++++++++++++++++++++++++++
\section{Erroneous or duplicate entries}\label{sec:wp942:dupes}
%+++++++++++++++++++++++++++++++++++++++++++++++++++++++++++++++++++++++++++

The {\preDR1} Catalogue received for validation was subject to several tests
concerning possible erroneous entries. This led to the filtering of a 
significant number of sources (37\,433\,092 sources were removed, 
3.2\% of the input sources).
As this filtering was obviously \beforeReferee{non} \afterReferee{not} perfect (removing actual sources 
while conserving erroneous ones), and had an impact on the Catalogue
content, the rationale, methods used and results are described in this section.

%----------------------------------------
\subsection{Erroneous faint TGAS sources}
%----------------------------------------

\subsubsection{Data before filtering}\label{sec:wp942:photofilter}

As can be seen in \figref{fig:942_tgashist}a, there was a significant number 
of objects (2\,381 sources) in the {\preDR1} version of TGAS 
that had $G\gtrsim 14$~mag,
i.e. clearly fainter than what was expected for Tycho-2. This led to the study
of the $G$ photometry for these stars and, beyond, for the whole catalogue.

A particular concern has been to catch coarse processing errors in the photometry.
For bright sources, the exposure time in each CCD on-board {\gaia} is reduced by activating
special TDI gates on the device as the star image crosses the CCD. This smaller exposure
time is then taken into account when computing the flux. However, in some rare occasions 
the information on gate activation did not
reach the photometric pipeline. The result was artificially low fluxes in that
particular transit, and for reasons beyond the scope of this paper, this
could upset the processing and lead to erroneous $G$ magnitudes. 

We therefore specifically checked if sources appeared much fainter 
in $G$ than in both {\gbp} and {\grp}, the preliminary 
versions of photometry to be published in later releases \citep{DPACP-10}. 
In practice the limit was set
at 3~mag in order not to eliminate diffuse objects with a bright core,
e.g. galaxies, which were expected to be bright in the diaphragm photometry
of {\gbp} and {\grp}; stars with $G-G_{\rm BP} > 3$ and $G-G_{\rm RP} > 3$, 
thus where a problem with $G$ was suspected, were filtered (164\,446 TGAS or 
secondary sources).

While the median number of $G$-band observations per source is 72 in {\GDR1},
it was also found that roughly half of the too faint TGAS sources had fewer 
than 10 CCD observations, and indeed, on the whole catalogue
stars with less than 10 observations clearly behaved incorrectly.
This led to the removal of all sources with less than 10 $G$ observations from {\preDR1} 
(746\,292 TGAS or secondary sources).

\subsubsection{Data after filtering}

Figure~\ref{fig:942_tgashist}b shows the resulting magnitude distribution for TGAS
in {\GDR1}, i.e. after full filtering. There is a remaining tail with 352 sources fainter 
than $G = 13.5$\,mag, and the presence 
of such sources in TGAS calls for an explanation. We have taken a closer look 
at the 60 faintest TGAS stars of which the brightest
has $G = 14.98$\,mag. Of these 60 stars, 25 have a neighbour brighter than $G =
13.5$\,mag and closer than 5\arcsec\ in {\GDR1} suggesting that the wrong star
may have been used in the TGAS solution, which is therefore not valid.  Of the
remaining 35 stars, just over half (18) have from one to four neighbours within
5\arcsec. In these cases we may be dealing with spurious Tycho-2 stars.
Tycho-2 \citep{2000A&A...357..367H} was using an input star list dominated by
photographic catalogues, and a blend of sources may therefore have been seen as
a single bright source. It may then happen that a Tycho-2 solution was derived
from the mixed signal of contaminating sources. We see that as a likely
explanation for most of these cases. For stars that are isolated in {\GDR1},
spurious Tycho-2 stars cannot be excluded, but in at least one case, the faint
Gaia source turns out to be a variable of the R\,CrB type. This star
(HIP\,92207) has $G = 16.57$\,mag in Gaia DR1, but is as bright as $V_T =
10.29$\,mag in Tycho-2. This is in good agreement with available light curves.
It is too early to say if there are more high amplitude variables in the
sample.

\begin{figure}
\centering
\includegraphics[width=0.49\columnwidth]{figures-942/cu9val_942_int_tgasGmagHist.png}
\includegraphics[width=0.49\columnwidth]{figures-942/cu9val_942_int_tgasGmagHist_filtered.png}
\caption{Histogram of $G$ magnitudes for TGAS stars (a) before
and (b) after validation filtering.}
\label{fig:942_tgashist}
\end{figure}



%----------------------------------------
\subsection{Duplicate entries}
%----------------------------------------

\subsubsection{{\GDR1} before filtering}\label{sec:duplicates}

Before launch, a catalogue with known optical astrometric and photometric information 
of sources up to magnitude $G=21$ had been built in order to be used as
Initial {\gaia} Source List \citep[IGSL,][]{2014A&A...570A..87S}.

Stars from IGSL may have initially contained duplicates originating from e.g. overlapping plates.
Automatically generated catalogues such as {\GDR1} may also have multiple copies of a source for a variety of reasons, including poor cross-matching of multiple observations, inconsistent handling of close doubles, or other observational or processing problems, beside the duplicates originating from the IGSL. 
To test for duplicate sources we cross-matched the Gaia catalogue against itself, identifying pairs of sources that could not possibly be real doubles, either because they fell within one pixel (59~mas) of each other or because their positions were consistent to within $5\sigma$. Only reference epoch positions were used, with no corrections for high proper motion stars.

It was found that the {\preDR1} Gaia catalogue contained 71~million sources 
with a counterpart within one pixel or $5\sigma$. 
Most appeared in pairs, but some were clustered in groups of up to eight duplicates. 
Up to one third of sources around $G \sim 11 \textrm{ mag}$ were affected, far more 
than at much brighter or much fainter magnitudes. 

For {\GDR1}, we removed all but one source from each group of close matches, selecting the source with the more precise parallax (if present) and breaking ties by the source with \beforeReferee{the} more observations, followed by the better position or photometric error. Because duplicated sources may have compromised astrometry or photometry (e.g., if a source was duplicated because of a cross-matching problem), the surviving sources were marked with the \dt{duplicated\_source} flag in the final catalogue (35\,951\,041 TGAS or secondary sources).

Two examples of the effect of the filtering of duplicate sources are shown in Figs. \ref{fig:cu9val_942_int_closedoubles} and  \ref{fig:cu9val_942_int_southPoleDuplicates}. The result of the filtering as done for {\GDR1}
is illustrated in Figs. \ref{fig:cu9val_942_int_closedoubles} and \ref{fig:cu9val_942_int_southPoleDuplicates}c. The artefacts in 
Figs. \ref{fig:cu9val_942_int_southPoleDuplicates}a and \ref{fig:cu9val_942_int_southPoleDuplicates}b are
the traces of the overlaps of photographic plates used in some of the surveys from which
the IGSL catalogue was built, causing an excess of duplicate sources in {\GDR1}.

\begin{figure}
\begin{center}
\includegraphics[width=0.8\columnwidth]{{figures-942/l=350,b=0-BeforeAndAlter-filtering}.png}
\caption{\small Number of pairs of sources vs their angular separation in the field 
($l=350$\deg, $b=0$\deg) before (red) and after filtering (green).
The line corresponds to a random distribution up to 10\arcsec\ of the latter.}\label{fig:cu9val_942_int_closedoubles}
\end{center}\end{figure}

\begin{figure}
\begin{center}
\includegraphics[width=0.32\columnwidth]{{figures-942/cu9val_942_int_south-pole-1}.pdf}
\includegraphics[width=0.32\columnwidth]{{figures-942/cu9val_942_int_south-pole-2}.pdf}
\includegraphics[width=0.32\columnwidth]{{figures-942/cu9val_942_int_south-pole-3}.pdf}
\caption{\small Effect of duplicate stars in a field of radius 4\deg\ around the South pole: (a) original density 
map in {\preDR1} before validation filtering, (b) duplicates found, (c) after 
duplicates filtering.}\label{fig:cu9val_942_int_southPoleDuplicates}
\end{center}\end{figure}


%While they were neither removed nor flagged as duplicates, sources with neighbours at 0.1-2\arcsec\ separation should be handled with care. While doubles with these separations were cleanly resolved by the Gaia telescope, the DR1 pipelines did not process truncated source extraction windows (\textbf{TODO: cite a Gaia pipeline paper}) for close sources with $G > 13$, leaving only a single window centred on the brighter component. 
%The result of processing multiple sources in the same window was undefined: the recorded image parameters could correspond to any or none of the astrophysical sources, with no guarantee of consistency. This issue will be resolved in later data releases, but in Gaia~DR1 sources with neighbours within 2\arcsec, particularly with $G > 13$, may be either correctly reduced sources, real sources with incorrect parameters, or spurious sources.

\subsubsection{{\GDR1} after filtering}

Although it is estimated that about 99\% of the duplicates have been removed, 
spurious sources may still remain in {\GDR1}. Formal uncertainties on positions 
of these duplicates may have been underestimated, and the $5\sigma$ criterion 
on positional difference used for rejection may finally not have been large enough. 
This underestimation was suspected the following way: a pair made of one duplicate source and the source
it duplicates actually refers to one single source which dispatched part of its observations
between both (depending on the orientation of the satellite scans). We used this property to
compare the positions and magnitudes in pairs and found that uncertainties were underestimated
by a factor 2 for positions and 4 for magnitudes. While this result cannot be extrapolated
to all normal (not duplicated) stars, this gives at least an upper limit and justifies in any
case the presence of the \dt{duplicated\_source} flag.

A comparison with the Washington Visual Double Star Catalogue \citep[WDS,][]{WDS}
confirms that some duplicates remain, as can be seen with the excess of stars with a near 
zero separation in the bottom left of \figref{fig:wp944_WDS}b.

\beforeReferee{Remaining duplicates may be also present in particular in high density fields; here, however, computing
the probability to get several stars close to each other just by chance (mostly optical doubles)
shows that actual stars may have been removed too so}
\afterReferee{In high density fields, there is a chance to get several stars 
very close to each other by chance only, i.e. optical doubles. 
Trying to remove more duplicates would lead to removing actual stars by mistake.}
The adopted filtering
may \beforeReferee{finally}\afterReferee{actually} have been a reasonable compromise, until the expected improvement
in Gaia DR2.




%
%%----------------------------------------
%\subsection{Spurious detections around bright stars}\label{sec:wp942:spurious}
%%----------------------------------------
%
%Tests have also been devised for the detection of possible spurious sources
%around bright stars, 
%false detections which could be due to noise variations on 
%the extended PSF image generated by these stars.
%A very preliminary and unfiltered Catalogue has been used to run 
%two tests developed to detect whether the number of stars in a given
%small field was roughly uniform as a function of surface and 
%orientation. 

%An example of a very bright star with spurious
%detections on the spikes of the PSF is in shown \figref{fig:942_gamma_pyx}
%as well as the detection of the non-uniformity.
%
%\begin{figure}[]
%\centering
%\includegraphics[width=1\columnwidth]{figures-942/Gamma-pyxidis-field.jpg}
%\includegraphics[width=1\columnwidth]{figures-942/Gamma-pyxidis-tests.jpg}
%\caption{A one arcmin radius field around $\gamma$ Pyxidis in an unfiltered 
%preliminary Catalogue. Spurious detections are detected by the non uniformity
%in radius$^2$ and orientation (counts, average value and $3\sigma$ thresholds).
%Such spurious sources have been filtered in DR1.
%\label{fig:942_gamma_pyx}}
%\end{figure}

%For {\GDR1}, spurious sources around bright stars 
%%like this one have been filtered before DR1. Many of them 
%were already filtered at the end of the data processing
%and did not even reach the {\preDR1} Catalogue,
%so the statistical tests did not find significant problems on a
%subset of bright stars.
%
