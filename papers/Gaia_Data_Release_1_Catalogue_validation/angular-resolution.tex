%----------------------------------------
\subsection{Completeness and angular resolution}
%----------------------------------------

Although there are no doubts about the excellent, spatial angular resolution of  
{\gaia}\footnote{e.g. Pluto and Charon could easily be separated with a 0.36" along-scan 
separation, see {\scriptsize\url{http://www.cosmos.esa.int/web/gaia/iow_20160121}}}, the {\it effective} angular
separation in {\GDR1} can be questioned, e.g. due to possible cross-match problems.

\subsubsection{Distribution of the distances between pairs of sources \label{sec:neighbours}}

A simple way of checking the angular resolution of a catalogue is to look
at the distribution of the distances between pairs of sources. For a random
star field with $\rho$ stars per unit area, a ring of radius $r$, centred on
a given star, will contain $\rho 2 \pi r \Delta r$ stars, where $\Delta r$ is
the width of the ring. For a sample of $N$ stars, we will have
$N \rho \pi r \Delta r$ unique pairs at that separation.

We have looked at two fields, a dense field of radius 2\degr\ centred at $(l,b)
= (330\degr, -4\degr)$ with 400\,000 stars per square degree and a sparse 
field of radius 15\deg ($l=260$\deg, $b=-60$\deg) with 2\,900
stars per square degree, scaled to produce the same number of sources. 
Figure~\ref{fig:942_ghist} shows the distribution of $G$ magnitudes in these two fields.
The difference of slopes comes from the fact that the dense field may integrate  
disk stars on a larger distance, with extinction not that large at $b=-4$\degr,
whereas the sparse field at higher latitude quickly leaves the disk and
integrates the thick disk, less dense. 

\begin{figure}
\centering
\includegraphics[width=0.8\columnwidth]{./figures-942/sparse3_dense.pdf}
\caption{$G$ magnitude histograms for a dense field ($l=330$\deg, $b=-4$\deg, $\rho=2$\deg) 
and a sparse field ($l=260$\deg, $b=-60$\deg, $\rho=15$\deg). The sparse field has
been scaled as to give about the same number of sources as the dense field.
\label{fig:942_ghist}}
\end{figure}


\begin{figure}
\centering
\includegraphics[width=0.49\columnwidth,height=0.4\columnwidth]{./figures-942/pairs_dense.pdf}
\includegraphics[width=0.49\columnwidth,height=0.4\columnwidth]{./figures-942/pairs_sparse3.pdf}
\caption{Distribution of source to source distances in {\GDR1} for a dense
($l=330$\deg, $b=-4$\deg, $\rho=2$\deg, left) 
%and sparse ($l=240$\deg, $b=45$\deg, $\rho=20$\deg, {\em bottom panel}) star field. 
and sparse ($l=260$\deg, $b=-60$\deg, $\rho=15$\deg, right) star field. 
The dashed lines show the
relation corresponding to a random distribution of the sources.
\label{fig:942_pairs}}
\end{figure}

The resulting distributions of distance between sources are shown in
\figref{fig:942_pairs}. For the dense field (left) the distribution is
close to random for separations above 4\arcsec, but drops for smaller
separations with a sharp drop at 2\arcsec. In the shallow field, which is much 
larger and not as uniform, the sharp drop between 2\arcsec\ and 2\farcs5 is also
seen, but not the drop at 3\farcs5. In order to improve the uniformity of the sparse
field, three small areas around galaxies and clusters were left out when deriving 
the distribution.

To better understand these results, we made a simple simulation of a dense,
random field, starting with 500\,000~stars in a square degree. We then removed
sources which had very poor chances of ever getting a clean photometric
observation. The photometric windows are quite large, 2\farcs1 in the across
scan direction and a diagonal size of 4\farcs1. If a source had either a
significantly brighter neighbour within 2\farcs1 or at least two such
neighbours between 2\farcs1 and 4\farcs1, it was removed. We took neighbours 
brighter by more than 0.2~mag. The criterion of two bright
neighbours is very simplistic and is taken to represent the cases where a star
is unlikely to ever get a clean photometric observation, irrespective of the
scanning direction.
Figure \ref{fig:942_simul}a shows the resulting distribution, which reproduces
many of the same characteristics seen in the real data (separations below 4\arcsec) shown in
\figref{fig:942_pairs}a.

\begin{figure}
\centering
\includegraphics[width=0.49\columnwidth,height=0.4\columnwidth]{./figures-942/dense_simulation.pdf}
\includegraphics[width=0.49\columnwidth,height=0.4\columnwidth]{./figures-942/retained.pdf}
\caption{Simulation of the distribution of source to source distances in 
a dense, random field (left) after applying selection criteria similar to
Gaia DR1. The fraction retained is shown in the right panel.
The field has a true source density of 500\,000 stars per square degree,
but only 322\,000 remain after applying the selection criteria.
\label{fig:942_simul}}
\end{figure}

We can therefore expect that the population of pairs closer than 2\arcsec\
consists of sources of similar brightness, where in a given transit either
source had a fair chance of being detected as the brighter and therefore got a
full observation window instead of the truncated window assigned to the fainter
detection in case of overlapping windows. 
For a brief description of the on-board conflict resolution see e.g.
\citet[][Sect.\ 2]{DPACP-7}.
%Fabricius et al.\ (2016, Sect.\ 2, this volume).  
There is of course still the risk, that a few of the closest pairs are in
reality two catalogue instances of the same source (duplicates) as discussed in
\secref{sec:duplicates}. 

We can now further understand the drop between 2\arcsec\ and 4\arcsec\ as being
due to conflicts between the photometric windows for the sources. This drop is
not present in the sparse field, where the chance of having two disturbing
sources in the right distance range is much smaller than in a dense field.

An important lesson from the simulation is illustrated in the second panel
of \figref{fig:942_simul}. Of the original 500\,000 stars in the simulation
only 322\,000 (64\%) survived the selection criteria described above. 
This has a significant impact on the fainter couple of magnitudes.
%{\bf (TODO: warning against uncritical use of Gaia DR1 for star counts, either
%here or later.)} -> seen in so many points that should be later... 

Below 2\arcsec\ separation, the dense field shows the expected small fraction of
field stars of similar magnitude. However, the sparse field shows a peak below
half an arcsecond, suggesting a high frequency of binaries in that area. We
looked in more detail at the 73 pairs brighter than 12\,mag to see if the Tycho 
Double Star Catalogue \citep[TDSC,][]{2002A&A...384..180F} could confirm 
the duplicity. Of the 65 pairs found in Tycho-2, 
% 10 HIP, 30 TDSC, 7 WDS 
47 are listed as doubles in TDSC, while 7 may be doubles missing in TDSC,
and 11 are possibly duplicated {\gaia} sources. 
This small test thus indicates that the majority of the {\GDR1} doubles are
actual double stars. 


%\subsubsection{Number of nearest neighbours as a function  of the \gmag~magnitude}
%{\bf TODO: text with U graph versus \gmag}

\subsubsection{Tests of the angular resolution using the WDS}\label{ssec:angular-resolution-944}

The spatial resolution of the Gaia catalogue has also been tested using the Washington Visual Double Star Catalogue \citep[WDS,][]{WDS}. A selection was made of sources composed of only 2 components, with the magnitudes for both the primary and the secondary brighter than 20 mag and a separation smaller than 10\arcsec. Sources had also to have been observed at least twice with differences between the two observed separations smaller than 2\arcsec and magnitude differences had to be smaller than 3 magnitude, and must not have a note indicating an approximate position (!), a dubious double (X), uncertain identification (I) nor photometry from a blue ($B$) or near-IR band ($K$). 
The resulting selection contains 43\,580 systems. The completeness of {\GDR1} versus the observation of these systems shows the performance of Gaia detection and observation of double systems as a function of the separation and magnitude difference between the components. 

The results are illustrated by a plot of completeness versus separation \beforeReferee{and 2D plots of completeness versus separation and magnitude difference, }presented in \figref{fig:wp944_WDS}a. As discussed in previous section, the angular resolution of {\GDR1} degrades rapidly below 4\arcsec. Although the filtering of {\preDR1} removed most of the duplicated sources, 
the excess of points with a very small Gaia separation and a WDS separation below about 1\arcsec\ 
in \figref{fig:wp944_WDS}b shows that a few duplicates ($\sim$0.5\% of the WDS sample) may still be present.


\begin{figure}
    \begin{center}
        \includegraphics[width=0.48\columnwidth]{./figures-944/WDScompl1D.pdf}
        \hspace{0.05\columnwidth}
        \includegraphics[width=0.42\columnwidth]{./figures-944/WDSsep.pdf}
        \caption[Visual double stars completeness versus WDS]{\beforeReferee{Visual binaries completeness}\afterReferee{Completeness of double stars} versus WDS; a) Completeness as a function of the \beforeReferee{visual binary separation}\afterReferee{separation between components}; 
        b) Separation between the components found with Gaia vs WDS separation (arcsecond).}
        \label{fig:wp944_WDS} 
    \end{center}
\end{figure}

%        \includegraphics[width=0.45\columnwidth]{./figures-944/WDScompl2D.png}
%        \vspace{0.05\columnwidth}
%        b) Completeness as a function of both separation and magnitude difference between the components. 


%----------------------------------------
\subsection{Summary of the Catalogue completeness}\label{sec:summary-content}
%----------------------------------------

A large filtering has been done on the main {\gaia} database to avoid spurious 
stars, for example a minimum of 5 focal plane transits for a star to be published 
in {\GDR1}. Due to the scanning law, and the resulting varying number of observations, 
the consequence is that some sky regions have a poor coverage, or are, locally, 
not covered at all. On the positive side, the filtering has succeeded to avoid 
spurious stars or ghosts which could be produced in the surroundings of bright stars,
or at least our statistical tests did not detect special features due to false
detections.

The limiting magnitude is therefore very inhomogeneous over the sky,
and the completeness as a function of magnitude is as well inhomogeneous:
starting from $G=16$ some sky zones appear clearly incomplete. 
Dense areas are, as expected, more affected due to the window and gate conflicts and the lack of on-board resources \citep{DPACP-18}.
High extinction regions also suffers from an increased, colour dependent, 
completeness issue due to the removal of the very red sources by the photometric pipeline \citep{DPACP-12}.

Duplicate sources which have been one of the main problems of {\preDR1} have mostly
been removed, although not completely, and their effect on the astrometric
or photometric properties of a fraction of bright star is probably still present.

Due to the preliminary nature of this data release the effective
angular resolution of the {\GDR1} data (not the angular resolution of
the Gaia instrument itself which is as expected) is also degraded, with
a deficit of close doubles.  In sparse regions, however, the spatial
capabilities of Gaia may already overcome the ground-based ones.  

As for TGAS, a significant fraction (20\%) of Tycho-2 stars are not present,
also due to the scanning coverage and to calibration problems, in particular at 
the bright end. A large fraction of high proper motion stars are missing, 
and those redder and fainter. 

It thus appears that {\GDR1} is not complete in any sense (magnitude,
colour, volume, resolution, proper motion, duplicity, etc.), so that any
statistical analysis should be careful to produce unbiased results.

The current completeness is however not representative of the future Gaia capabilities. 
That this will be corrected at the next data release triggers another
warning for the users preparing star lists: the \dt{source\_id} list present in DR2 
(and further releases) may be partly different from {\GDR1}. 
On one hand the gains to expect on the 
cross-matching performances (at small angular separations) and the larger number 
of transits (i.e. less stars with not enough observations to be published)
imply that many more stars will be present in DR2. On the other hand, a significant
number of \dt{source\_id} may disappear, caused by both splitting and merging sources.




