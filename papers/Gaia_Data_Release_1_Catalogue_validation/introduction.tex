%+++++++++++++++++++++++++++++++++++++++++++++++++++++++++++++++++++++++++++
\section{Introduction}
%+++++++++++++++++++++++++++++++++++++++++++++++++++++++++++++++++++++++++++


This paper describes the validation of the first data release 
from the European Space Agency mission {\gaia} \citep{DPACP-18}.
In a historical perspective, {\gaia}, following in the footsteps 
of the great astronomical catalogues since the first by Hipparchus of Nicaea, 
describes the state of the sky at the beginning of the $21^\mathrm{st}$ century.
It is the heir of the massive international astronomical projects, 
initiated in the late $19^\mathrm{th}$ century with the Carte du Ciel \citep{2000A&G....41e..16J}, 
and a direct successor of the ESA Hipparcos mission \citep{1997A&A...323L..49P}.

Despite the precautions taken during the acquisition of the satellite 
observations and when building the data processing system, it is a difficult
task to ensure perfect astrometric, photometric, spectroscopic and 
classification data for a one billion source catalogue built from the intricate 
combination of many data items for each entry. 
However, several actions have been undertaken to ensure the quality of the 
{\gaia} Catalogue through both internal and external data validation processes 
before each release.
The results from the external validation work are described in this paper.

%----------------------------------------
\paragraph{The {\GDR1}:}
%----------------------------------------
There is an exhaustive description of the {\gaia} operations and 
instruments in \citet{DPACP-18}, of the {\gaia} processing in \citet{DPACP-8} 
and the astrometric and photometric pre-processing is also detailed in \citet{DPACP-7}.
For this reason we mention here only what is strictly necessary and
invite the reader to refer to the above papers or to the {\gaia}
documentation for details.

The {\gaia} satellite is slowly spinning and measures the fluxes and observation times 
of all sources crossing the focal plane (their {\gaia} {\it transit}), sending to the 
ground small windows of pixels around the sources. These times correspond to 
one-dimensional, along-scan positions ({\it AL} in what follows) 
which are used in an astrometric global iterative solution process 
\citep[AGIS,][]{DPACP-14} which also needs to simultaneously calibrate the 
instruments and reconstruct the attitude of the satellite. A star crossing
the focal plane is measured on 9 \afterReferee{CCDs} in the astrometric instrument so the number
of observations of a star can be up to 9 times the number of its transits.
On-board resources are able to cope with various stellar densities; however, 
for very dense fields above 400\,000 sources per square degree, 
the brighter sources are preferentially selected.

The photometric instrument is composed of two prisms, a Blue Photometer (BP)
and a Red Photometer (RP). This colour information is not present in the
{\GDR1}, only the $G$-band photometry, derived from the fluxes measured in 
the astrometric instrument being given. The CCD dynamic range does
not allow to observe all sources from the brightest up to $G\sim 21$: sources 
brighter than $G\sim 12$ would be saturated. To avoid this, 
Time Delay Integration (TDI) gates are present
on the CCD and can be activated for bright sources, which in practice reduce 
their integration time (but also complicates their calibration).

Astrometry and photometry are then derived on-ground in independent pipelines, 
which are part of the work developed under the responsibility of the
body in charge of the data processing for the Gaia mission, the 
Gaia Data Processing and Analysis Consortium \citep[DPAC,][]{DPACP-8}.

This first data release 
contains preliminary results based on observations collected during the
first 14 months of mission since the start of nominal operations in July
2014.  At the start of nominal operations of the spacecraft on 25 July
2014, a special scanning law was followed, the Ecliptic Pole Scanning
Law (EPSL).  In EPSL mode, the spin axis of the spacecraft always lies
in the ecliptic plane, such that the field-of-view directions pass the
north and south ecliptic poles on each six-hour spin.  Then followed
the Nominal Scanning Law (NSL) with a precession rate of 5.8 revolutions
per year, starting on 22 August 2014.  As we will notice below, the EPSL
mode left some imprints on the Catalogue content and scientific results.

{\GDR1} contains a total of 1 142 679 769 sources, the astrometric part of {\GDR1} 
being built in two parts: the {\it primary} sources contains positions, parallaxes, 
and mean proper motions for 2\,057\,050 of the stars brighter than about
magnitude $V=11.5$ (about 80\% of these stars).
This data set, the Tycho Gaia Astrometric Solution (TGAS), was obtained through the combination 
of the {\gaia} observations with the positions of the sources obtained by Hipparcos
\citep{1997ESASP1200.....E} when available, or Tycho-2 \citep{2000A&A...355L..27H}.
The second part of {\GDR1}, the {\it secondary} sources, contains the positions and $G$ magnitudes
for 1\,140\,622\,719 sources brighter than about magnitude $G=21$. An annex of variable
stars located around the south ecliptic pole \beforeReferee{could also be published} \afterReferee{is also part of the release} thanks to the 
large number of observations made during the EPSL mode. 

%----------------------------------------
\paragraph{The Catalogue Validation:}
%----------------------------------------
In terms of scientific project, the quality of the released data has been controlled 
by two complementary approaches: the {\it verifications} done internally at each
step of the processing development in order to answer the question: are we building
the Catalogue correctly? and the {\it validations} at the end: is the
final Catalogue correct?

It is fundamental to note that the first step of the validations is logically 
represented by the many tests implemented in the {\gaia} DPAC 
groups before producing their own data, and which are described
in dedicated publications, \citet{DPACP-14} for the astrometry,
\citet{DPACP-11} for the photometry, and \citet{DPACP-15}
for the variability. 

To assess the Catalogue properties and as a final check before publication,
the DPAC deemed useful to implement a second and last step: a validation of the Catalogue
as a whole and actually, this must be stressed, a fully independent validation.  

%A large effort has thus been dedicated to this work.
%The main purposes for the validation tests were the following:
%\begin{itemize}
%\item implement sanity checks on the fields of the Catalogue 
%\item check the accuracy and precision of Catalogue parameters
%\item verify the correct distribution of these parameters, in particular the
%absence of large numbers of outliers
%\item study the completeness of the Catalogue
%\item detect, when possible, instrumental or processing problems
%\item and more generally check what would not be covered by the internal 
%DPAC verifications, e.g., cross-checks between data.
%\end{itemize}
%
%While the DPAC is mostly organised in terms of astronomical fields (astrometry,
%photometry, spectroscopy, etc.), the validation areas have been split in terms
%of various methods which can be applied, with the purpose of being able to 
%cross-check the various results obtained with these different methods. 
%
%Each of these methods have their pros and cons. 
%Internal methods, using the {\gaia} data alone, can be applied to any source without any
%cross-match ambiguity. On the other hand, external catalogues provide independent 
%results which can be compared to the {\gaia} data. Where external data are not available,
%galaxy models may sometimes help to explain whether observed features are, or not, 
%artifacts. Also, statistical methods for multidimensional analysis can find 
%data properties inside the observational noise.
%Finally, while this validation development model is transversal, some special objects 
%require a special treatment, such as clusters, and those having an important time 
%dependence such as in variability analysis, and also solar system objects. 
%Dedicated sub-work packages led by experts of the fields above have been 
%included to provide the needed insight into the Catalogue quality. 

The actual Catalogue validation operations began after data 
from the DPAC groups had been collected and a consolidated Catalogue 
had been built before publication. At this step, no re-run of the data processing
\beforeReferee{being} \afterReferee{was} possible, only the rejection of some stars (if strictly needed)
and some cosmetic changes on the data fields could be done.
After the rejection of problematic stars, 
a process labelled as {\it filtering},
the validation was again performed, and most of the catalogue properties 
described in this paper refer to this post-filtering, published, final {\GDR1} data. 

\bigskip
The organisation of this paper is as follows: 
\secref{sec:description} summarises the data and models
used. Section \ref{sec:wp942:dupes} describes the erroneous or duplicate entries
found and partly removed. The main properties of the {\GDR1} Catalogue are discussed, 
\secref{sec:completeness}, for the sky coverage and completeness,
with a multidimensional analysis in \secref{multidim}, the
astrometric quality of {\GDR1} in \secref{astroqual} and the
photometric properties in \secref{photoqual}. As a conclusion, 
recommendations for data usage are given in \secref{sec:conclusions}.
The validation procedures employed in testing the design and interfaces 
of the archive systems are described in Appendix
together with some illustrations of the statistical properties of the Catalogue. 
 

