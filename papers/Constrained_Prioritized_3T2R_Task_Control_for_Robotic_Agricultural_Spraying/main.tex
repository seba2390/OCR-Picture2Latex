\documentclass[letterpaper, 10 pt, conference]{ieeeconf}
\usepackage[utf8]{inputenc}

% paper

\IEEEoverridecommandlockouts                              % This command is only
% needed if you want to
% use the \thanks command
\overrideIEEEmargins
% See the \addtolength command later in the file to balance the column lengths
% on the last page of the document

\pdfminorversion=4

\usepackage{graphicx}  				% Versatile graphics manipulation options

\usepackage[croatian]{babel}  % Croatian typographical rules and hyphenation patterns 
\usepackage[utf8]{inputenc}  	% Encoding of Croatian characters
\usepackage[T1]{fontenc}
\usepackage{ae,aecompl}     	% Type 1 fonts, similar to Computer Modern

\usepackage{microtype}				% Improves spacing

\usepackage{subfig}

\usepackage{amsmath}

\usepackage{tabularx}
\usepackage{booktabs}
%\newcolumntype{C}{>{\centering\arraybackslash}X} % centered version of "X" type
\setlength{\extrarowheight}{1pt}
\usepackage{enumerate}				% Additional options for listing of items in enumerate environment
\usepackage{algorithm2e}			% Writing pseudo-code
\usepackage{todonotes}				% Adding todo items
\usepackage{dirtree}					% Simple display of directory tree
\usepackage{hyperref}					% Managing cross-referencing

\usepackage{lmodern}
\usepackage{nccmath}
%\usepackage[keeplastbox]{flushend}
\usepackage{scalerel,stackengine}
\usepackage{comment}
\usepackage{cite}

\usepackage{amssymb}
\usepackage{makecell}


\parskip 1ex

%calligraphy packages
\usepackage{calrsfs}
\DeclareMathAlphabet{\pazocal}{OMS}{zplm}{m}{n}
\newcommand{\Ca}{\pazocal{C}}
\newcommand{\Oa}{\pazocal{O}}
\newcommand{\Va}{\pazocal{V}}
\newcommand{\Ua}{\pazocal{U}}
\newcommand{\Aa}{\pazocal{A}}
\newcommand{\Ta}{\pazocal{T}}
\newcommand{\La}{\pazocal{L}}

\newcommand{\Ja}{\pazocal{J}}

\graphicspath{{./figures/}}
\usepackage{float}

\newcommand{\norm}[1]{\left\lVert #1 \right\rVert}

\title{\LARGE \bf
	Constrained Prioritized 3T2R Task Control for Robotic Agricultural Spraying  
}
\author{Ivo Vatavuk, Zdenko Kovačić
	\thanks{\hrule}
	\thanks{Ivo Vatavuk, MSc, is a PhD student at the Faculty of Electrical Engineering and Computing,
	University of Zagreb, Unska 3, 10000 Zagreb, Croatia: \tt(ivo.vatavuk@fer.hr)
	}
	\thanks{
	Zdenko Kovačić, PhD, is a full professor at the Faculty of Electrical Engineering and Computing,
	University of Zagreb, Unska 3, 10000 Zagreb, Croatia \tt(zdenko.kovacic@fer.hr)
	}}%

\makeatletter
\newcommand{\removelatexerror}{\let\@latex@error\@gobble}
\newcommand{\mb}[1]{\boldsymbol{#1}}

\makeatother

\begin{document}
	
	\maketitle
	\thispagestyle{empty}
	\pagestyle{empty}
	
	%%%%%%%%%%%%%%%%%%%%%%%%%%%%%%%%%%%%%%%%%%%%%%%%%%%%%%%%%%%%%%%%%%%%%%%%%%%%%%%%
	\begin{abstract}
		In this paper, we present a solution for robot arm-controlled agricultural spraying, handling the spraying task as a constrained prioritized 3T2R task. 3T2R tasks in robot manipulation consist of three translational and two rotational degrees of freedom, and are frequently used when the end-effector is axis-symmetric. The solution presented in this paper introduces a prioritization between the translational and rotational degrees of freedom of the 3T2R task, and we discuss the utility of this kind of approach for both velocity and positional inverse kinematics, which relate to continuous and selective agricultural spraying applications respectively.       
	\end{abstract}
	\begin{keywords}
	Agricultural Automation, Mobile Manipulation, Optimization and Optimal Control 
	\end{keywords}
	\section{Introduction}

Scientific literature is most commonly available in the form of PDFs, which pose challenges for accessibility \citep{NielsenPDFStillUnfit, Bigham2016AnUT}. When researchers, students, and other individuals who are blind or low vision (BLV) interact with scientific PDFs through screen readers, the availability of document structure tags, labeled reading order, labeled headers, and image alt-text are necessary to facilitate these interactions. However, these features must be painstakingly added by authors using proprietary software tools, and as a result, are often missing from papers. Low vision or dyslexic readers who interact with PDFs through screen magnification or text-to-speech may also find the complexity of certain academic paper PDF formats challenging, e.g., non-linear layout can interrupt the flow of text in a magnifying tool. Inaccessible paper PDFs can lead to high cognitive overload, frustration, and abandonment of reading for BLV readers. 

Unfortunately, we find that the majority of scientific PDFs lack basic accessibility features. We estimate based on a sample of \numpdfs PDFs from multiple fields of study that only around \percaccessible of paper PDFs released in the last decade satisfy all of the aforementioned accessibility requirements. 
Accessibility challenges for academic PDFs are largely due to three factors: (1) the complexity of the PDF file format, which make it less amenable to certain accessibility features, (2) the dearth of tools, especially non-proprietary tools, for creating accessible PDFs, and (3) the dependency on volunteerism from the community with minimal support or enforcement \citep{Bigham2016AnUT}. The intent of the PDF file format is to support faithful visual representation of a document for printing, a goal that is inherently divergent from that of document representation for the purposes of accessibility. Though some professional organizations like the Association for Computing Machinery (ACM) have encouraged PDF accessibility through standards and writing guidelines,\footnote{\href{https://www.acm.org/publications/authors/submissions}{https://www.acm.org/publications/authors/submissions}} uptake among academic publishers and disciplines more broadly has been limited. 

While policy changes help, the fact remains that most academic PDFs produced today, and historically, are inaccessible, yet remain as the dominant way to read those papers. A long-range solution will necessitate buy-in from multiple stakeholders---publishers, authors, readers, technologists, granting agencies, and the like. But in the interim, there are technological solutions that can be offered as a sort of ``band-aid'' to the problem. We use this paper to offer an in-depth qualitative and quantitative description of the problem as it stands, and to introduce one such technological solution: the \scially system that automatically extracts semantic information from paper PDFs and re-renders this content in the form of an accessible HTML document. Though the process is imperfect and can introduce errors, we demonstrate the ability of the rendered HTMLs to reduce cognitive load and facilitate in-paper navigation and interactions for BLV users. 

The goals and contributions of this paper are three-fold:

\begin{enumerate}
    \item We characterize the state of academic-paper PDF accessibility by estimating the degree of adherence to accessibility criteria for papers published in the last decade (2010--2019), and describe correlations between year, field of study, PDF typesetting software, and PDF accessibility.
    \item We propose an automated approach for extracting the content of academic PDFs and displaying this content in a more accessible HTML document format. We build a prototype that re-renders 12 million PDFs in HTML, and describe the design decisions, features, and quality of the renders (assessed as faithfulness to the source PDF). We perform expert grading of the rendered HTML and report an error analysis. A demo of our system is available at \href{https://scia11y.org/}{scia11y.org}, which makes available 1.5M HTML renders of open access PDFs.
    \item We conduct an exploratory user study with \numusers BLV scholars to better understand the challenges they experience when reading academic papers and how our proposed tool might augment their current workflow. During the study, we ask users to interact with the prototype and offer feedback for its improvement. We perform open coding of interviews to identify existing reading challenges, coping mechanisms, as well as positive and negative responses to prototype features. We summarize the findings of this user study into a set of design recommendations.
\end{enumerate}

Our analysis reveals that PDF accessibility adherence is low across all fields of study. Of the five accessibility criteria we assess, only \percaccessible of the PDFs we assess demonstrate full compliance. Though compliance for several criteria seems to be increasing over time, author awareness and contribution to accessibility remains low, as Alt-text has the lowest compliance of the five criteria at between 5--10\% (Alt-text is the only criterion of the five that \textit{requires} author intervention in all cases using current tools). We also find that typesetting software is strongly associated with accessibility compliance, with LaTeX and publishing software like Arbortext APP producing low compliance PDFs, while Microsoft Word is generally associated with higher compliance.


\begin{figure}[t!]
    \centering
    \includegraphics[width=\textwidth]{figures/pipeline.png}
    \caption{A schematic for creating the \scially HTML render from a paper PDF. Starting with the raw two-column PDF on the left, S2ORC \citep{lo-wang-2020-s2orc} is used to extract title, authors, abstract, section headers, body text, and references. S2ORC also identifies links between inline citations and references to figures and table objects. DeepFigures \citep{Siegel2018ExtractingSF} is used to extract figures and tables, along with their captions. The output of these two models are merged with metadata from the Semantic Scholar API. Heuristics are used to construct a table of contents, to insert figures and tables in the appropriate places in the text, and to repair broken URLs. We add HTML headers as illustrated (header tags for sections, paragraph tags for body text, and figure tags for figures and tables); highlighted components (table of contents and links in references) are not in the PDF and novel navigational features that we introduce to the HTML render. An example HTML render of parts of a paper document is show to the right (actual render is single column, which is split here for presentation).}
    \label{fig:pipeline}
    \Description{A schematic diagram showing the components of the SciA11y pipeline. An image of a paper PDF is on the left. Red boxes on the PDF image highlight the text components from the paper, with an arrow pointing to a box that says "S2ORC extracts: title, authors, abstract, section headers, body paragraphs, and references." A blue box on the PDF image highlights a figure, with an arrow pointing to a box that says "DeepFigures extracts: figures, figure captions, tables, and table titles/captions." A box below "S2ORC extracts" and "DeepFigures extracts" says "Additional content: metadata from Semantic Scholar API, table of contents, figures and tables inserted at first mention, and links between references and text." Arrows from all three boxes point into a larger box that describes the SciA11y prototype, where HTML tags are inserted around various blocks of text extracted from the PDF. On the right of all this is a screen capture of an example HTML render, showing how the semantic content from the PDF is represented as a single-column HTML page for easy reading.}
\end{figure}

To offset the reading challenges of inaccessible papers for BLV researchers, we propose and test the \scially system for rendering academic PDFs into accessible HTML documents. As shown in Figure~\ref{fig:pipeline}, our prototype integrates several machine learning text and vision models to extract the structure and semantic content of papers. The content is represented as an HTML document with headings and links for navigation, figures and tables, as well as other novel features to assist in document structure understanding. Our evaluation of the \scially system identifies common classes of extraction problems, and finds that though many papers exhibit some extraction errors, the majority (55\%) have no major problems that impact readability, and another 32\% have only some problems that impact readability.

Through our user study, we identify numerous challenges faced by BLV users when reading paper PDFs, including some that affect the whole document or limit navigation, and many that affect the ability of the reader to understand text or various elements of a paper like math content or tables. Responses to \scially were positive; participants especially liked navigation features such as headings, the table of contents, and bidirectional links between inline citations and references. Of the extraction errors in \scially, missed or incorrectly extracted headings were the most problematic, as these impact the user's ability to navigate between sections and fully trust the system. All users reported being likely to use the system in the future. When asked how the system might be integrated into their workflow, one participant replied ``I think it would become the workflow.'' Another participant said, ``for unaccessible PDFs, this is life-changing.'' We condense these findings into a set of recommendations for designing and engineering accessible reading systems (Section~\ref{sec:designrecs}). Most importantly, documents should be structured to match a reader's mental model, objects should be properly tagged, and care should be taken to reduce the reader's cognitive load and increase trust in the system. Features that emulate the external memory that visual layout provides to sighted users can be especially beneficial.

This paper is organized as follows. Following a description of related work in Section \ref{sec:related_work}, we first provide a meta-scientific analysis of the current state of academic PDF accessibility in Section \ref{sec:sos}. In Section \ref{sec:pdf2html}, we document our pipeline for converting PDF to HTML and describe the \scially prototype for rendering papers. An evaluation of HTML render quality and faithfulness is provided in Section \ref{sec:evaluation}. Section \ref{sec:user_study} describes our user study and findings. 
We recognize that no PDF extraction system is perfect, and many open research challenges remain in improving these systems. However, based on our findings, we believe \scially can dramatically improve screen reader navigation of most papers compared to PDFs, and is well-positioned to assist BLV researchers with many of their most common reading use cases. Our hope is that a system such as \scially can improve BLV researcher access to the content of academic papers, and that these design recommendations can be leveraged by others to create better, more faithful, and ultimately more usable tools and systems for scholars in the BLV community.

        \section{Constrained Prioritized Task Space Control for Continuous Spraying}
\label{sec:cont_spr}

As already mentioned, continuous spraying refers to the problem of applying the spraying agent to the entire canopy of the plant. Constrained prioritized task space control is used to select joint velocity commands that follow the commanded spraying frame velocity. Velocity of the spraying frame is controlled as a prioritized 3T2R control task, prioritizing its translational over its rotational component.  

\subsection{Velocity Level Prioritized Task Space Control}

\begin{figure*}[!t]
\centering
\includegraphics[width=0.8\textwidth]{./figures/sideways_comparison_newest.png}
\caption{ Continuous spraying examples, from left to right: slow spraying, positionally constrained slow spraying, fast spraying. 
}
\label{fig:sideways_fig}
\end{figure*}

Joint velocity commands are selected by solving a constrained prioritized task space control problem \cite{deLasa2010}. The general constrained prioritized task space control problem is defined as:
\begin{equation}
	\begin{aligned}
		h_i = & \ \underset{\boldsymbol{x}}{\text{min}} & & E_i(\boldsymbol{x})\\
		& \ \ \text{s.t.} & & E_k(\boldsymbol{x}) = h_k, \forall k < i\\
		& & & \boldsymbol{A}_{eq}\boldsymbol{x} + \boldsymbol{b}_{eq} = 0\\
		& & & \boldsymbol{A}_{ieq}\boldsymbol{x} + \boldsymbol{A}_{ieq} \geq 0
	\end{aligned}
\end{equation}
where $E_i$ is the quadratic cost function of the $i$-th priority, $h_i$ is the optimal value of that cost function, and $\boldsymbol{A}_{eq}$, $\boldsymbol{b}_{eq}$, $\boldsymbol{A}_{ieq}$ and $\boldsymbol{b}_{ieq}$ are the matrices and vectors describing linear equality and inequality constraints, respectively.

Priorities used for continuous agricultural spraying are:
\begin{itemize}
    \item Translational part of the 3T2R task
    \item Rotational part of the 3T2R task
    \item Desired joint positions
\end{itemize}

The cost function of the first priority has a following form:
\begin{equation} \label{eq:priority_1}
	\begin{aligned}
E_1(\dot{\boldsymbol{q}}) = & \norm{ \boldsymbol{v}_c - \boldsymbol{J}_{T}\dot{\boldsymbol{q}} }^2
	\end{aligned}
\end{equation}
where $\boldsymbol{v_c}$ is the commanded linear velocity of the spraying frame, $\boldsymbol{J}_{T}$ is the translational part of the spraying frame Jacobian, and $\dot{\boldsymbol{q}}$ is the joint velocity vector.
$\boldsymbol{v_c}$ is the output of the MPC solver described in \cite{Vatavuk2022}. Generally, there are multiple joint velocity vectors $\dot{\boldsymbol{q}}$ that result in the commanded linear velocity, and the criterion function of the second priority is selected between those solutions, in the null space of the first priority.

The cost function of the second priority, referring to the rotational part of the 3T2R task, has a following form:
\begin{equation} \label{eq:priority_2}
E_2(\dot{\boldsymbol{q}}) = \norm{ \omega^L_{c,x} - \boldsymbol{J}^L_{R,x}\dot{\boldsymbol{q}} }^2 + \norm{ \omega^L_{c,y} - \boldsymbol{J}^L_{R,y}\dot{\boldsymbol{q}} }^2
\end{equation}
where $\omega^L_{c,x}$ and $\omega^L_{c,y}$ are commanded angular velocities of the spraying frame around its local $x$ and $y$ axes respectively, and $\boldsymbol{J}^L_{R,x}$ and $\boldsymbol{J}^L_{R,y}$ are the corresponding Jacobian matrices. Since the spraying nozzle is an axis-symmetric tool, the angular velocity around its local $z$ axis is not directly controlled.

The final priority, which resolves any redundancy remaining after minimizing the first two priorities, favors such joint velocities $\dot{\boldsymbol{q}}$ that move the arm towards a desired configuration:
\begin{equation} \label{eq:priority_3}
E_3(\dot{\boldsymbol{q}}) = \norm{ \dot{\boldsymbol{q}}_c - \dot{\boldsymbol{q}} }^2
\end{equation}

The commanded joint velocities $\dot{\boldsymbol{q}}_c$ that drive the robot arm towards a desired pose $\boldsymbol{q}_d$ are selected by a proportional controller:
\begin{equation}
\dot{\boldsymbol{q}}_{c} = K_{P,q}(\boldsymbol{q}_d - \boldsymbol{q})
\end{equation}
where $K_{P,q}$ is the controller gain and $\boldsymbol{q}$ is a current joint position vector.

Inequality constraints are used to enforce joint velocity and acceleration limits:
\begin{equation}
\underline{\dot{\boldsymbol{q}}} \leq \dot{\boldsymbol{q}} \leq \overline{\dot{\boldsymbol{q}}}
\end{equation}
\begin{equation}
\label{eq:acc_constr}
\underline{\ddot{\boldsymbol{q}}} \leq \ddot{\boldsymbol{q}} \leq \overline{\ddot{\boldsymbol{q}}}
\end{equation}

Since the prioritized task space control problem deals with joint velocities, equation (\ref{eq:acc_constr}) is replaced with the one in the velocity space:
\begin{equation}
\dot{\boldsymbol{q}}_P + \underline{\ddot{\boldsymbol{q}}}\Delta t \leq \dot{\boldsymbol{q}} \leq \dot{\boldsymbol{q}}_P + \overline{\ddot{\boldsymbol{q}}}\Delta t
\end{equation}
where $\Delta t$ is the control time step, and $\dot{\boldsymbol{q}}_P$ are joint velocities in the previous time step.

\subsection{Commands for the rotational part of the 3T2R task}
Commands for the local angular velocities of the spraying frame $\omega^L_{c,x}$ and $\omega^L_{c,y}$ are calculated using the error between the desired and the current approach axis orientation:
\begin{equation}
err_{\alpha} = \arccos( \boldsymbol{app}_{z} \cdot \boldsymbol{app}_{d,z} )
\end{equation}
\begin{equation}
\boldsymbol{err}_{axis} = \boldsymbol{app}_{z} \times \boldsymbol{app}_{d,z}
\end{equation}
where $\boldsymbol{app}_{z}$ and $\boldsymbol{app}_{z,d}$ are the current and the desired approach axis vectors, respectivelly, $err_{\alpha}$ is the angular distance between the two vectors, and $\boldsymbol{err}_{axis}$ is an axis around which $err_{\alpha}$ acts. 

Angular error vector represented in the local frame is:
\begin{equation}
\boldsymbol{\alpha}^L_{err} = {}_L\boldsymbol{R}^B(err_{\alpha} \cdot \boldsymbol{err}_{axis})
\end{equation}
If the $z$ axis of the frame is considered its approach axis, the $z$ component of $\boldsymbol{\alpha}^L_{err}$ is always zero, and the local angular velocities are calculated as:
\begin{equation}
\boldsymbol{\omega}^L_{c} = K_{P,\omega}\boldsymbol{\alpha}^L_{err} = 
\begin{bmatrix}
    \omega^L_{c,x} \\
    \omega^L_{c,y} \\
    0 \\
\end{bmatrix}
\end{equation}
where $K_{P,\omega}$ is the proportional controller gain.
%\begin{equation}
%\boldsymbol{J}^L_{R} = {}^L\boldsymbol{R}_B \boldsymbol{J}_{R}
%\end{equation}

%\begin{equation}
%\boldsymbol{\omega}^L_{c} = K_{P,\omega}\boldsymbol{\alpha}^L_{err}
%\end{equation}

%\todo[inline]{Clamping of orientation err, Kori equals rate, deadbeat}

\subsection{Continuous Spraying Examples}
\begin{figure*}[!ht]
\centering
\includegraphics[width=0.95\textwidth]{./figures/consecutive_spraying_newest}
\caption{
	Spraying frame rotates freely around its approach axis while minimizing the 3T2R task as well as joint movement. 
}
\label{fig:consecutive}
\end{figure*}
The previously described approach was tested on three continuous spraying examples, with different commanded linear velocities and constraints (Fig. \ref{fig:sideways_fig}). In all the examples the spraying frame rotates freely around its approach axis, as a result of 3T2R control (Fig. \ref{fig:consecutive}). Footage of the examples can be seen in the accompanying video\footnote{\url{https://www.youtube.com/watch?v=FRdmGsSCAh4}}. Constrained prioritized task space control solver described in \cite{deLasa2010} by de Lasa et al. was implemented in C++ using OSQP (Operator Splitting Quadratic Program) quadratic programming solver \cite{osqp}. This implementation was used for the experiments, and is available on GitHub\footnote{\url{https://github.com/ivatavuk/ptsc_eigen}}.

First example has a low commanded linear velocity of $0.2$ m/s, resulting in both the linear and rotational velocity being feasible during the entire trajectory. The 3T2R task is followed in its entirety as a result, and the third priority (Eq. (\ref{eq:priority_3})) fully constrains the prioritized optimization problem. 

In the second example, the same commanded linear velocity was used as in the first one, but with an addition of a positional constraint on a nozzle height. The nozzle is not allowed to reach positions lower than $0.3$ m from the robot arm base. During the lower segment of the trajectory, this constraint becomes active, which results in the prioritization between the translational and rotational component of the 3T2R task being noticeable (Fig. \ref{fig:sideways_fig}).    

%\todo[inline]{Positional constraint equations}
%\todo[inline]{Exp comparison pos/orientation errors}

Third example has a large commanded linear velocity of the spraying frame of $0.8$ m/s, which results in joint velocity and acceleration constraints being reached during the execution of the trajectory. As a consequence, the 3T2R task is not achievable in its entirety, and the third priority is disregarded for the most part of the trajectory. To combat this issue, for the fast trajectory only two priorities are used. The first priority is the same as in the previous examples (Eq. \ref{eq:priority_1}), and the second priority is a weighted combination of the rotational component of the 3T2R task and desired joint movement:

\begin{equation} \label{eq:priority_2_2}
E_2(\dot{\boldsymbol{q}}) = \norm{ \omega^L_{x,d} - \boldsymbol{J}^L_{R,x}\dot{\boldsymbol{q}} }^2 + \norm{ \omega^L_{y,d} - \boldsymbol{J}^L_{R,y}\dot{\boldsymbol{q}} }^2 + w\norm{ \dot{\boldsymbol{q}}_d - \dot{\boldsymbol{q}} }^2
\end{equation}

In this example, the position of the spraying frame follows the commanded linear velocity, while its desired orientation is not achievable due to joint velocity and acceleration constraints.

The utility of the presented method resides in the prioritization between the translational and rotational components of the 3T2R tasks. When the 3T2R task velocity is not feasible in its entirety, due to the constraints posed by the robot arm or due to the custom constraints posed by the user, task priorities are utilized to find an optimal spraying angle.

%If the 3T2R task velocity is feasible in its entirety, the results do not differ from the standard velocity level 3T2R task control.


        \section{Prioritized Positional Inverse Kinematics for Selective Spraying}
Selective agricultural spraying refers to the task of spraying a specific part of the plant, for example a cluster of grapes. This task is handled as a prioritized positional inverse kinematics problem. Prioritization between the translational and rotational components of the 3T2R task used for continuous spraying remains for this use case. 

\subsection{Prioritized Positional Inverse Kinematics Solver}

\begin{figure*}[!ht]
\centering
\includegraphics[width=0.9\textwidth]{./figures/pik_all_three_newest.png}
\caption{Prioritized positional inverse kinematics examples for the task of selective agricultural spraying. Transparent blue sphere and arrow represent the desired position and desired approach axis orientation of the spraying frame respectively, and the transparent purple sphere represents the desired elbow position.}
\label{fig:pik_fig}
\end{figure*}

Prioritized positional inverse kinematics solver implementation is similar to standard numerical inverse kinematics, iteratively solving the velocity level problem. The velocity level problem is solved as a constrained prioritized task-space control problem, as described in section \ref{sec:cont_spr}. While the standard positional inverse kinematics solvers aim to achieve a commanded end-effector pose, the presented solver has the ability of handling multiple, potentially conflicting tasks with different priorities. 

Solver pseudoalgorithm is given in Algorithm \ref{alg:pik}. The algorithm requires an initial guess for joint positions $\boldsymbol{q}_{initial}$. Task errors and Jacobians are calculated based on the current joint positions $\boldsymbol{q}$ and the type of the task. Error gradients are updated for each task as a difference between the task error in current and previous iteration of the algorithm. Task Jacobians and clamped errors are used to construct a prioritized task space control problem. Finally, a solution to the prioritized task space problem is used to update the current joint positions. If the sum of all task error norms or error gradient norms reaches a threshold the problem is considered to be solved. 
\begin{comment}
\begin{algorithm}
\caption{Positional prioritized inverse kinematics solver.}\label{alg:pik}
$\boldsymbol{q} \gets \boldsymbol{q}_{initial}$\\
tasks $\gets [ ]$ \\
%$\nabla err \gets 0$\;
\While{$\sum \norm{ \boldsymbol{err_i}} \geq \varepsilon_{e}$ \textbf{and} $\sum \norm{ \nabla \boldsymbol{err_i}} \geq \varepsilon_\nabla$}
{
    \For{$i\gets0$ \KwTo N}{
    $\boldsymbol{J}_i$ \gets $getTaskJacobian(\boldsymbol{q}, tasktype_i)$\\
    $\boldsymbol{err}_i$ \gets $getTaskError(\boldsymbol{q}, tasktype_i)$\\
    $\nabla \boldsymbol{err}_i$ \gets $updateGradient(\boldsymbol{err}_i)$\\
    $\boldsymbol{err}_i$ \gets $clampTaskError(\boldsymbol{err}_i, $tasktype$_i)$\\
    tasks.insert($\boldsymbol{J}_i, \boldsymbol{err}_i$)\\
    }
    $\boldsymbol{q}$ \gets $solvePTSC(tasks, constraints)$\\
    tasks.clear$()$ \\
}

\end{algorithm}
\end{comment}

\begin{algorithm}
\caption{Positional prioritized inverse kinematics solver.}\label{alg:pik}
$\boldsymbol{q} \gets \boldsymbol{q}_{\text{initial}}$\\
tasks $\gets [ ]$ \\
%$\nabla \text{err} \gets 0$\;
\While{$\sum \|\boldsymbol{err}_i\| \geq \varepsilon_{e}$ \textbf{and} $\sum \|\nabla \boldsymbol{err}_i\| \geq \varepsilon_{\nabla}$}
{
    \For{$i\gets0$ \KwTo $N$}{
        $\boldsymbol{J}_i \gets \text{getTaskJacobian}(\boldsymbol{q}, \text{tasktype}_i)$\\
        $\boldsymbol{err}_i \gets \text{getTaskError}(\boldsymbol{q}, \text{tasktype}_i)$\\
        $\nabla \boldsymbol{err}_i \gets \text{updateGradient}(\boldsymbol{err}_i)$\\
        $\boldsymbol{err}_i \gets \text{clampTaskError}(\boldsymbol{err}_i, \text{tasktype}_i)$\\
        tasks.\text{insert}$(\boldsymbol{J}_i, \boldsymbol{err}_i)$\\
    }
    $\boldsymbol{q} \gets \text{solvePTSC}(\text{tasks}, \text{constraints})$\\
    tasks.\text{clear}()\\
}
\end{algorithm}
 
Prioritized inverse kinematics library for ROS is available on GitHub\footnote{\url{https://github.com/ivatavuk/pik_ros}}. MoveIt is used to calculate the Jacobians of the specified frames, which must be present in the URDF file of the MoveIt planning group. 

The Jacobians obtained with MoveIt are modified to support any of the following tasks: 
\begin{itemize}
    \item Frame pose task
    \item Frame position task
    \item Frame orientation task
    \item Frame approach axis vector task
\end{itemize}
These task types correspond to the $tasktype_i$ variable in the pseudoalgorithm \ref{alg:pik}. A frame pose task Jacobian is the standard Jacobian matrix, and frame position and orientation task Jacobians correspond to the first and last three rows of the frame pose Jacobian. The Jacobian and the error for the frame approach axis vector task are calculated as described in section \ref{sec:cont_spr}. 

\begin{comment}
 The framework allows for user defined parameters used by solver, which are:
\begin{itemize}
    \item Change in joint angle constraint
    \item Positional clamp magnitude
    \item Orientational clamp magnitude
    \item Use constrained optimization
    \item Error norm threshold
    \item Maximum execution time
    \item Maximum number of iterations
\end{itemize}   
\end{comment}

\subsection{Selective Spraying Examples}
Tasks used in selective agricultural spraying examples are, with decreasing priorities:
\begin{itemize}
    \item Spraying frame position task
    \item Spraying frame approach axis orientation task
    \item Elbow frame position task
\end{itemize}

Like for the continuous spraying, there is a prioritization of the spraying frame position over its approach axis orientation, which correspond to the translational and rotational components of the 3T2R task. The third priority, which fully constrains the positional inverse kinematics problem is the desired elbow frame position.

The solver was tested on three different examples seen in Fig. \ref{fig:pik_fig}, with desired values for all the tasks given in table \ref{tab:pik_examples}. Tasks are set up in such a way that in the first two examples the desired values of the full 3T2R task are feasible, and the elbow position task fully constrains the problem, and in the last example only the position of the spraying frame is feasible (Fig. \ref{fig:pik_fig}). 

\begin{table}[]
    \centering
    \begin{tabular}{c | c c c c}
         Example & \makecell{ Spraying frame \\ position [m] }  & \makecell{ Spraying frame \\ approach axis \\ vector } & \makecell{ Elbow \\ position [m]} \\
         \hline
         1 & [0.4 1.0 0.2] & [0 1 0] & [0.0 -0.5 0.5] \\
         2 & [0.4 1.0 0.8] & [0.511 0.511 0.69] & [0.0 -0.5 0.5] \\
         3 & [0.4 1.0 0.8] & [0.577 0.577 -0.577] & [0.0 -0.5 0.5]
    \end{tabular}
    \caption{Desired values for prioritized tasks used in the examples.}
    \label{tab:pik_examples}
\end{table}

The description of solver performance for the examples is given in Tab. \ref{tab:pik_results}. All experiments were conducted on a $2.2$\textit{GHz} Intel Core i7 processor. It can be noticed that the third example takes the largest amount of time to be solved, which is due to the solution being close to the robot arm singularity. %This is even more noticable when solution polishing is not used, which results in large movements around the singularity that do not improve the solution. 

\begin{table}[]
    \centering
    \begin{tabular}{c | c c c c}
         Example & \makecell{ Task 1 \\ err [m] }  & \makecell{ Task 2 \\ err [rad]} & \makecell{ Task 3 \\ err [m]} & \makecell{ Time \\ \textnormal{[ms]} } \\
         \hline
         1 & 0.00020 & 0.00019 & 0.3525 & 11.09 \\
         2 & 0.00054 & 0.00067 & 0.7222 & 19.86 \\
         3 & 0.00247 & 0.2052 & 0.6253 & 30.07 
    \end{tabular}
    \caption{Task errors and calculation time for the examples.}
    \label{tab:pik_results}
\end{table}

Parameters for the solver used in the examples are:
\begin{itemize}
    \item Use constrained optimization = $True$
    \item Error norm gradient threshold = $1\times 10^{-3}$
    \item Change in joint angle constraint = $10$ [$^{\circ}$]
    \item Use solution polishing = $True$
    \item Polish error norm gradient threshold = $1\times 10^{-2}$
    \item Polish change in joint angle constraint = $3$ [$^{\circ}$]
    \item Positional clamp magnitude = $0.3$ [m]
    \item Orientational clamp magnitude = $30$  [$^{\circ}$]
    %\item Maximum execution time = $\infty$ [s]
    %\item Maximum number of iterations = $\infty$
\end{itemize}

Solution polishing refers to the usage of smaller change in joint angle constraint when the solver is close to the solution, which is detected as \textit{polish error norm gradient threshold} being reached.  
\begin{comment}
\begin{table}[]
    \centering
    \begin{tabular}{c|c|c|c|c}
        Example &               1 & 2 & 3 & 4\\
        Task 1 error [m] &      0.0015632 & 0.0015632 & 0.0015632 & 5.74\\
        Task 2 error [rad] &    0.0015632 & 0.0015632 & 0.0015632 & 5.74 \\
        Task 3 error [m] &      0.0015632 & 0.0015632 & 0.0015632 & 5.74\\
        Time [ms] &             0.0015632 & 0.00126516 & 0.78838 & 5.74 
    \end{tabular}
    \caption{Caption}
    \label{tab:my_label}
\end{table}
\end{comment}

All three tasks are not feasible in any of the given examples, so the solver considers the positional prioritized IK problem solved once task error gradients reach a specified threshold. For most prioritized inverse kinematics applications the same would be the case, as the main strength of this approach is its ability to handle conflicting, infeasible tasks with clearly defined priorities. 

        %\pagebreak
	\section{Conclusion and Future Work}
For the task of robotic agricultural spraying, and for robotic spraying in general, the position of the spraying frame is more important than its orientation to ensure satisfactory spray coverage. We propose a solution where constrained prioritized optimization is used for velocity and positional level 3T2R task control, which corresponds to continuous and selective agricultural spraying tasks, respectively. Prioritized task space control and prioritized positional inverse kinematics are described in detail. Positional inverse kinematics are solved using iterative constrained prioritized task space control. In the future work, the prioritized IK framework is planned to be expanded to allow for more task types, such as preferred joint positions, manipulability maximization task and others. Voxel based obstacle avoidance is also planned to be included. We plan to explore the applicability of the framework for different robot control tasks. Prioritized positional inverse kinematics could have a number of applications, most interesting ones including high dimensional floating base robotic systems, which could have a high number of prioritized conflicting tasks. The utility of the prioritized optimization described in this paper for trajectory planning also remains to be explored.    

	
	\subsubsection*{Acknowledgments}
	Research work presented in this article has been supported by the project Heterogeneous autonomous robotic system in viticulture and mariculture (HEKTOR) financed by the European Union through the European Regional Development Fund-The Competitiveness and Cohesion Operational Programme (KK.01.1.1.04.0036).
	\nocite{*}
	\bibliographystyle{ieeetr}
	\bibliography{bibliography/asdf}
	
\end{document}
