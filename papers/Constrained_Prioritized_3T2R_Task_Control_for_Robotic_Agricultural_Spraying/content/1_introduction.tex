\section{Introduction}

Agricultural robotics is a rapidly advancing research field that focuses on developing and deploying robotic technology for various agricultural tasks. The goal is to enhance the efficiency and sustainability of different agricultural procedures and address labor shortages. Research presented in this paper is a part of the project HEKTOR \cite{hektor, Goricanec2021}, which aims to introduce heterogeneous robotic systems to the agricultural areas of viticulture and mariculture. A mobile manipulator is envisioned to autonomously perform various viticultural tasks, including monitoring, spraying and suckering.  

Manual agricultural spraying is often performed with a spray wand, a nozzle mounted on the end of a lightweight pole. The nozzle is often mounted at an angle, making it easier for the operator to control both the position and the orientation of the nozzle, and reach high and low areas of the canopy. In the presented work, a spray wand is mounted as the robot arm end-effector (Fig. \ref{fig:intro_fig}), aiming to maintain the advantages of manual spraying while benefiting from increased efficiency and precision of robotic technology.

\begin{figure}[!ht]
\centering
\includegraphics[width=0.65\columnwidth]{./figures/intro_fig_newest.png}
\caption{
The scenario in this paper involves mounting the spray wand for manual vineyard spraying as the end-effector of a mobile manipulator. The nozzle used to apply the spraying agent is an axis-symmetric tool. 
}
\label{fig:intro_fig}
\end{figure}

Our previous work focused on the problem of selecting coordinated control inputs for the vehicle and the robot arm in the same scenario \cite{Vatavuk2022}. The robot arm was controlled in the task space, controlling solely the translational velocity of the spraying frame, depicted in Fig. \ref{fig:intro_fig}, and disregarding its rotation. The reasoning behind this was that, to achieve large enough linear velocities of the spraying frame, and reach high and low areas of the plant, it is not possible to fully control the orientation of the spraying frame. In this paper, a more complete solution to the robot arm control problem is offered, handling the control of the spraying frame as a prioritized 3T2R task.

3T2R tasks, also known as pointing tasks, are frame pose control tasks where all three components of the frame position, and only two components of the frame orientation are considered \cite{Schappler2019}. Since only five degrees of freedom are controlled, a functional redundancy is introduced for robot arms with six degrees of freedom and more. There is extensive research on different approaches to resolving functional redundancies in robot manipulation \cite{From2007, Zlajpah2017, Schappler2019, Zlajpah2021, Zanchettin2011}. 
Tasks performed with axis-symmetric tools, such as robotic welding, paint spraying and drilling, are frequent examples of 3T2R tasks. In robotic drilling for example, both the position and the orientation of the drill bit are important for task execution, but the rotation around the drill bit is not. 

We handle the agricultural spraying task in a similar way, since the rotation around the approach axis of the spraying frame does not effect the application of the spraying agent. However, unlike the drilling task, spraying with a correct spraying frame position and a non ideal approach axis orientation can still be acceptable \cite{From2010, From2011, Zanchettin2011}. 

In \cite{From2011}, From et. al. report that the linear velocity of the paint gun is far more important than its orientation for achieving uniform paint coating. We believe the same to be the case in agricultural spraying, even to a larger extent, since the spraying agent in agricultural spraying is generally less dense than the paint in spray painting applications, and human operators performing the agricultural spraying tasks generally handle the orientation of the nozzle with less care than in the paint spraying applications. 

This insight is handled by introducing a prioritization between translational and rotational components of the 3T2R task. 
Prioritized task space control \cite{deLasa2009, deLasa2010, Wensing2013} replaces commonly used task weighting approach, with hard priorities that are guaranteed to be satisfied. The solution to the lower priority task is found inside the nullspace of a higher priority task. This is performed iteratively, for any number of tasks with different priorities, until a certain task fully constrains the optimization problem. This kind of approach is often referred to as prioritized velocity space inverse kinematics (IK), prioritized instantaneous IK or just prioritized IK \cite{Chiacchio1991, Lillo2019, Moe2016, An2019}. 

In this work, constrained prioritized task space control algorithm presented in \cite{deLasa2010} is used to solve the prioritized velocity space inverse kinematics problem, for the continuous agricultural spraying application. Continuous spraying refers to the task of treating the entire canopy of the plant. Previous robotic approaches to this task mostly used a set of nozzles fixed on the mobile vehicle \cite{Berenstein2010, Berenstein2019, Cantelli2019}, while the robot arm was mostly utilized for selective spraying \cite{Oberti2013, Oberti2016}. Selective spraying refers to the problem of spraying a specific part of the plant, for example a single disease-ridden leaf, or a fruit cluster. A solution to this problem is also presented, handling agricultural selective spraying as a prioritized positional level IK problem. A solver for this purpose based on iterative constrained prioritized task space control is presented. 

\subsection{Contribution}
We present a constrained prioritized 3T2R task control scheme for agricultural spraying, solving the 3T2R control task on both velocity and position levels, prioritizing between its translational and rotational components. Two use cases are discussed in which the velocity and the position level algorithms are applied to continuous and selective agricultural spraying respectively. The implementation of the velocity level prioritized task space control scheme for continuous spraying, and the prioritized positional inverse kinematics solver for selective spraying are discussed in detail. 

\subsection{Paper Organization}
The remainder of this paper is structured as follows: Section II presents the constrained prioritized task space control approach for continuous agricultural spraying. The details of the approach are presented as well as the discussion on the effects of different constrains on the performance. Section III presents the solution to the selective agricultural spraying problem, approached as a prioritized positional inverse kinematics problem. Details of the implementation are presented, and the results and their implications are discussed. Finally, Section IV. concludes the paper with some comments on future work.  