\section{Constrained Prioritized Task Space Control for Continuous Spraying}
\label{sec:cont_spr}

As already mentioned, continuous spraying refers to the problem of applying the spraying agent to the entire canopy of the plant. Constrained prioritized task space control is used to select joint velocity commands that follow the commanded spraying frame velocity. Velocity of the spraying frame is controlled as a prioritized 3T2R control task, prioritizing its translational over its rotational component.  

\subsection{Velocity Level Prioritized Task Space Control}

\begin{figure*}[!t]
\centering
\includegraphics[width=0.8\textwidth]{./figures/sideways_comparison_newest.png}
\caption{ Continuous spraying examples, from left to right: slow spraying, positionally constrained slow spraying, fast spraying. 
}
\label{fig:sideways_fig}
\end{figure*}

Joint velocity commands are selected by solving a constrained prioritized task space control problem \cite{deLasa2010}. The general constrained prioritized task space control problem is defined as:
\begin{equation}
	\begin{aligned}
		h_i = & \ \underset{\boldsymbol{x}}{\text{min}} & & E_i(\boldsymbol{x})\\
		& \ \ \text{s.t.} & & E_k(\boldsymbol{x}) = h_k, \forall k < i\\
		& & & \boldsymbol{A}_{eq}\boldsymbol{x} + \boldsymbol{b}_{eq} = 0\\
		& & & \boldsymbol{A}_{ieq}\boldsymbol{x} + \boldsymbol{A}_{ieq} \geq 0
	\end{aligned}
\end{equation}
where $E_i$ is the quadratic cost function of the $i$-th priority, $h_i$ is the optimal value of that cost function, and $\boldsymbol{A}_{eq}$, $\boldsymbol{b}_{eq}$, $\boldsymbol{A}_{ieq}$ and $\boldsymbol{b}_{ieq}$ are the matrices and vectors describing linear equality and inequality constraints, respectively.

Priorities used for continuous agricultural spraying are:
\begin{itemize}
    \item Translational part of the 3T2R task
    \item Rotational part of the 3T2R task
    \item Desired joint positions
\end{itemize}

The cost function of the first priority has a following form:
\begin{equation} \label{eq:priority_1}
	\begin{aligned}
E_1(\dot{\boldsymbol{q}}) = & \norm{ \boldsymbol{v}_c - \boldsymbol{J}_{T}\dot{\boldsymbol{q}} }^2
	\end{aligned}
\end{equation}
where $\boldsymbol{v_c}$ is the commanded linear velocity of the spraying frame, $\boldsymbol{J}_{T}$ is the translational part of the spraying frame Jacobian, and $\dot{\boldsymbol{q}}$ is the joint velocity vector.
$\boldsymbol{v_c}$ is the output of the MPC solver described in \cite{Vatavuk2022}. Generally, there are multiple joint velocity vectors $\dot{\boldsymbol{q}}$ that result in the commanded linear velocity, and the criterion function of the second priority is selected between those solutions, in the null space of the first priority.

The cost function of the second priority, referring to the rotational part of the 3T2R task, has a following form:
\begin{equation} \label{eq:priority_2}
E_2(\dot{\boldsymbol{q}}) = \norm{ \omega^L_{c,x} - \boldsymbol{J}^L_{R,x}\dot{\boldsymbol{q}} }^2 + \norm{ \omega^L_{c,y} - \boldsymbol{J}^L_{R,y}\dot{\boldsymbol{q}} }^2
\end{equation}
where $\omega^L_{c,x}$ and $\omega^L_{c,y}$ are commanded angular velocities of the spraying frame around its local $x$ and $y$ axes respectively, and $\boldsymbol{J}^L_{R,x}$ and $\boldsymbol{J}^L_{R,y}$ are the corresponding Jacobian matrices. Since the spraying nozzle is an axis-symmetric tool, the angular velocity around its local $z$ axis is not directly controlled.

The final priority, which resolves any redundancy remaining after minimizing the first two priorities, favors such joint velocities $\dot{\boldsymbol{q}}$ that move the arm towards a desired configuration:
\begin{equation} \label{eq:priority_3}
E_3(\dot{\boldsymbol{q}}) = \norm{ \dot{\boldsymbol{q}}_c - \dot{\boldsymbol{q}} }^2
\end{equation}

The commanded joint velocities $\dot{\boldsymbol{q}}_c$ that drive the robot arm towards a desired pose $\boldsymbol{q}_d$ are selected by a proportional controller:
\begin{equation}
\dot{\boldsymbol{q}}_{c} = K_{P,q}(\boldsymbol{q}_d - \boldsymbol{q})
\end{equation}
where $K_{P,q}$ is the controller gain and $\boldsymbol{q}$ is a current joint position vector.

Inequality constraints are used to enforce joint velocity and acceleration limits:
\begin{equation}
\underline{\dot{\boldsymbol{q}}} \leq \dot{\boldsymbol{q}} \leq \overline{\dot{\boldsymbol{q}}}
\end{equation}
\begin{equation}
\label{eq:acc_constr}
\underline{\ddot{\boldsymbol{q}}} \leq \ddot{\boldsymbol{q}} \leq \overline{\ddot{\boldsymbol{q}}}
\end{equation}

Since the prioritized task space control problem deals with joint velocities, equation (\ref{eq:acc_constr}) is replaced with the one in the velocity space:
\begin{equation}
\dot{\boldsymbol{q}}_P + \underline{\ddot{\boldsymbol{q}}}\Delta t \leq \dot{\boldsymbol{q}} \leq \dot{\boldsymbol{q}}_P + \overline{\ddot{\boldsymbol{q}}}\Delta t
\end{equation}
where $\Delta t$ is the control time step, and $\dot{\boldsymbol{q}}_P$ are joint velocities in the previous time step.

\subsection{Commands for the rotational part of the 3T2R task}
Commands for the local angular velocities of the spraying frame $\omega^L_{c,x}$ and $\omega^L_{c,y}$ are calculated using the error between the desired and the current approach axis orientation:
\begin{equation}
err_{\alpha} = \arccos( \boldsymbol{app}_{z} \cdot \boldsymbol{app}_{d,z} )
\end{equation}
\begin{equation}
\boldsymbol{err}_{axis} = \boldsymbol{app}_{z} \times \boldsymbol{app}_{d,z}
\end{equation}
where $\boldsymbol{app}_{z}$ and $\boldsymbol{app}_{z,d}$ are the current and the desired approach axis vectors, respectivelly, $err_{\alpha}$ is the angular distance between the two vectors, and $\boldsymbol{err}_{axis}$ is an axis around which $err_{\alpha}$ acts. 

Angular error vector represented in the local frame is:
\begin{equation}
\boldsymbol{\alpha}^L_{err} = {}_L\boldsymbol{R}^B(err_{\alpha} \cdot \boldsymbol{err}_{axis})
\end{equation}
If the $z$ axis of the frame is considered its approach axis, the $z$ component of $\boldsymbol{\alpha}^L_{err}$ is always zero, and the local angular velocities are calculated as:
\begin{equation}
\boldsymbol{\omega}^L_{c} = K_{P,\omega}\boldsymbol{\alpha}^L_{err} = 
\begin{bmatrix}
    \omega^L_{c,x} \\
    \omega^L_{c,y} \\
    0 \\
\end{bmatrix}
\end{equation}
where $K_{P,\omega}$ is the proportional controller gain.
%\begin{equation}
%\boldsymbol{J}^L_{R} = {}^L\boldsymbol{R}_B \boldsymbol{J}_{R}
%\end{equation}

%\begin{equation}
%\boldsymbol{\omega}^L_{c} = K_{P,\omega}\boldsymbol{\alpha}^L_{err}
%\end{equation}

%\todo[inline]{Clamping of orientation err, Kori equals rate, deadbeat}

\subsection{Continuous Spraying Examples}
\begin{figure*}[!ht]
\centering
\includegraphics[width=0.95\textwidth]{./figures/consecutive_spraying_newest}
\caption{
	Spraying frame rotates freely around its approach axis while minimizing the 3T2R task as well as joint movement. 
}
\label{fig:consecutive}
\end{figure*}
The previously described approach was tested on three continuous spraying examples, with different commanded linear velocities and constraints (Fig. \ref{fig:sideways_fig}). In all the examples the spraying frame rotates freely around its approach axis, as a result of 3T2R control (Fig. \ref{fig:consecutive}). Footage of the examples can be seen in the accompanying video\footnote{\url{https://www.youtube.com/watch?v=FRdmGsSCAh4}}. Constrained prioritized task space control solver described in \cite{deLasa2010} by de Lasa et al. was implemented in C++ using OSQP (Operator Splitting Quadratic Program) quadratic programming solver \cite{osqp}. This implementation was used for the experiments, and is available on GitHub\footnote{\url{https://github.com/ivatavuk/ptsc_eigen}}.

First example has a low commanded linear velocity of $0.2$ m/s, resulting in both the linear and rotational velocity being feasible during the entire trajectory. The 3T2R task is followed in its entirety as a result, and the third priority (Eq. (\ref{eq:priority_3})) fully constrains the prioritized optimization problem. 

In the second example, the same commanded linear velocity was used as in the first one, but with an addition of a positional constraint on a nozzle height. The nozzle is not allowed to reach positions lower than $0.3$ m from the robot arm base. During the lower segment of the trajectory, this constraint becomes active, which results in the prioritization between the translational and rotational component of the 3T2R task being noticeable (Fig. \ref{fig:sideways_fig}).    

%\todo[inline]{Positional constraint equations}
%\todo[inline]{Exp comparison pos/orientation errors}

Third example has a large commanded linear velocity of the spraying frame of $0.8$ m/s, which results in joint velocity and acceleration constraints being reached during the execution of the trajectory. As a consequence, the 3T2R task is not achievable in its entirety, and the third priority is disregarded for the most part of the trajectory. To combat this issue, for the fast trajectory only two priorities are used. The first priority is the same as in the previous examples (Eq. \ref{eq:priority_1}), and the second priority is a weighted combination of the rotational component of the 3T2R task and desired joint movement:

\begin{equation} \label{eq:priority_2_2}
E_2(\dot{\boldsymbol{q}}) = \norm{ \omega^L_{x,d} - \boldsymbol{J}^L_{R,x}\dot{\boldsymbol{q}} }^2 + \norm{ \omega^L_{y,d} - \boldsymbol{J}^L_{R,y}\dot{\boldsymbol{q}} }^2 + w\norm{ \dot{\boldsymbol{q}}_d - \dot{\boldsymbol{q}} }^2
\end{equation}

In this example, the position of the spraying frame follows the commanded linear velocity, while its desired orientation is not achievable due to joint velocity and acceleration constraints.

The utility of the presented method resides in the prioritization between the translational and rotational components of the 3T2R tasks. When the 3T2R task velocity is not feasible in its entirety, due to the constraints posed by the robot arm or due to the custom constraints posed by the user, task priorities are utilized to find an optimal spraying angle.

%If the 3T2R task velocity is feasible in its entirety, the results do not differ from the standard velocity level 3T2R task control.

