






\begin{IEEEbiography}[{\includegraphics[width=1in,height=1.25in,clip,keepaspectratio]{Yu.jpg}}]{Yu Chen}
received the BS degree in mathematics and applied mathematics from Nanjing University of Science and Technology.
He completed  the PhD degree at the same university.
He is now  a researcher at  Motovis Research Australia.
His current research interests are deep learning, autonomous driving and pose estimation in particular. \end{IEEEbiography}





\begin{IEEEbiography}
[{\includegraphics[width=1in,height=1.25in,clip,keepaspectratio]{shen2.jpg}}]
{Chunhua Shen}
is a Professor at School of Computer Science, University of Adelaide.
Before that, he was with the computer vision program at NICTA (National ICT Australia), Canberra Research Laboratory for about six years.
He studied at Nanjing University, at Australian National University, and received his PhD degree from the University of Adelaide.
    From 2012 to 2016, he held an Australian Research Council Future Fellowship.
 \end{IEEEbiography}





\begin{IEEEbiography}
[{\includegraphics[width=1in,height=1.25in,clip,keepaspectratio]{hao_chen.png}}]{Hao Chen} received the master's degree from Zhejiang University, China. He is working towards the PhD degree at School of Computer Science,  The University of Adelaide. His current research interests in deep learning and
 its applications in computer vision and text analysis.
 \end{IEEEbiography}

\begin{IEEEbiography}[{\includegraphics[width=1in,height=1.25in,clip,keepaspectratio]{Wei.jpg}}]{Xiu-Shen Wei} (M'18) received his BS degree in computer science, and his Ph.D. degree in computer science and technology from Nanjing University. He is now the Research Lead of Megvii (Face++) Research Nanjing. He has published more than ten academic papers on the top-tier international journals and conferences, such as IEEE TIP, IEEE TNNLS, Machine Learning Journal, ICCV, IJCAI, etc. He achieved the first place in the Apparent Personality Analysis competition (in association with ECCV 2016) and the first runner-up in the Cultural Event Recognition competition (in association with ICCV 2015) as the team director. He also received the Presidential Special Scholarship (the highest honor for Ph.D. students) in Nanjing University. His research interests are computer vision and machine learning. He is a PC member of ICCV, CVPR, ECCV, NIPS, IJCAI, AAAI, etc..\end{IEEEbiography}


\begin{IEEEbiography}
[{\includegraphics[width=1in,height=1.25in,keepaspectratio]{Lingqiao.JPG}}]
{Lingqiao Liu} received the BS and MS degrees in communication engineering from the University of Electronic Science and Technology of China, Chengdu, in 2006 and 2009, respectively, and the PhD degree from the Australian National University, Canberra, in 2014. He is now a Lecturer at the University of Adelaide. In 2016, he was awarded the Discovery Early Career Researcher Award by the Australian Research Council. His research interests include various topics in computer vision and machine learning.
\end{IEEEbiography}


\begin{IEEEbiography}
[{\includegraphics[width=1in,height=1.25in,clip,keepaspectratio]{Jian.png}}]{Jian Yang} received the PhD degree from Nanjing University of Science and Technology (NUST), on the subject of pattern recognition and intelligence systems in 2002. In 2003, he was a postdoctoral researcher at the University of Zaragoza. From 2004 to 2006, he was a Postdoctoral Fellow at Biometrics Centre of Hong Kong Polytechnic University. From 2006 to 2007, he was a Postdoctoral Fellow at Department of Computer Science of New Jersey Institute of Technology. Now, he is a Chang-Jiang professor in the School of Computer Science and Technology of NUST. He is the author of more than 100 scientific papers in pattern recognition and computer vision. His journal papers have been cited more than 4000 times in the ISI Web of Science, and 9000 times in the Web of Scholar Google. His research interests include pattern recognition, computer vision and machine learning. Currently, he is/was an associate editor of Pattern Recognition Letters, IEEE Trans. Neural Networks and Learning Systems, and Neurocomputing. He is a Fellow of IAPR.\end{IEEEbiography}




