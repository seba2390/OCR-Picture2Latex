\section{Selection}
\label{sec:selection}

%(Finding the $k$-th smallest element.)

%\textbf{Problem:} For selection, can we select the element of rank $k$ where the number of comparisons on this element is sensitive to $k$?

In this section we consider the problem of finding the $k$-th smallest element of an unsorted array.
There is a randomized algorithm that selects the $k$-th smallest element with expected fragile complexity of $O(\log \log n)$ for the selected element~\cite{afshani:fragile-ESA19}.
We consider the question if this complexity can be improved for small $k$.
In this section we define a sampling method that, combined with the algorithm given in~\cite{afshani:fragile-ESA19}, selects the $k$-th smallest element with expected $O(\log \log k)$ comparisons.

Next, we define the filtering method \textsc{ReSet} in a tail-recursive fashion. The idea of this procedure is the following: First, we build a random half size sample $A_1$ from the input set $X$. Later, we continue recursively constructing a random half sample $A_i$ from the previous sample $A_{i-1}$ until we get a random sample $A_{\ell}$ of size $O(k+1)$. Once $A_{\ell}$ is given, then a set $A'_{\ell}$ of size $O(k)$ is given for the previous recursive call. Using such set, a new subset $A'_{\ell-1}$ is given from the previous sample $A_{\ell-1}$ where its expected size is $O(k)$. This process continuous until a final subset $\mathcal{C}$ is given from the input set $X$ such that its expected size $O(k)$ and it contains the $k$-th smallest element of $X$. 
%Later, we show that such filtering method compares the selected element by at most $O(\log^* N)$ in expectation.

\noindent\fbox{\scalebox{0.95}{\begin{minipage}{\textwidth}
\begin{small}
\begin{algorithmic}[1]
	 \Procedure{ReSet}{$X, k$}
	 	\Comment{Returns a small subset $\mathcal{C}$ of $X$ that contains the $k$-th element}
			\State Let $n=|X|$ and $\mathcal{C} = \emptyset$
			\If{$k \geq \frac{n}{2}-1$}
			    \Comment{The set has size $O(k+1)$}
			    \State Let $A'=X$
			\Else
			    \Comment{Recursively construct a sample of expected size $O(k+1)$}
			    \State Sample $A$ uniformly at random from $X$, $|A|= \frac{n}{2}$
			    \State Let $A'=$ \textsc{ReSet($A, k$)}
			\EndIf
			\State Choose the ($k+1$)-th smallest element $z$ from $A'$ (by standard linear time selection) 
			\State Let $\mathcal{C}=\{x \in X : x\leq z\}$
			\State \Return $\mathcal{C}$
	\EndProcedure
\end{algorithmic}
\end{small}
\end{minipage}}
}

%We are ready to prove the result of this section.
In the following theorem we show that the combination of the \textsc{ReSet} procedure and the randomized selection algorithm in~\cite{afshani:fragile-ESA19}, results in expected $O(\log \log k)$ comparisons for the selected element. %In particular we use a sample of size $\frac{n}{k}$ and \textsc{ReSet} as a filtering method.

\begin{theorem}
Randomized selection is possible in expected fragile complexity\\ $O(\log \log k)$ in the selected element.
\end{theorem}
\begin{proof}
Let us show that the following procedure for selecting the $k$-th element in a set $X$ with $|X|=n$, gives an expected fragile complexity $O(\log \log k)$ in the $k$-th element:

\emph{If $k > n^{\frac{1}{100}}$, then let $S'=X$. If $k \leq n^{\frac{1}{100}}$, then sample uniformly at random $S$ from $X$, where $|S|=\frac{n}{k}$. Let $C=$ \textsc{ReSet}($S, k$) and select the $k+1$-th smallest element $z$ from $\mathcal{C}$ by standard linear time selection. Let $S'=\{x \in X: x\leq z\}$. Finally, apply to $S'$ the randomized selection algorithm of~\cite{afshani:fragile-ESA19}.}

Let $x_k$ denote the $k$-th smallest element in $X$ and let $f_k$ denote the \frag of $x_k$. 
Note that if $x_k \in S$, then, before constructing $S'$, $f_k$ is given by the \frag of $x_k$ in $\textsc{ReSet}(S,k)$ plus $O(|\mathcal{C}|)$ when finding the ($k+1$)-th smallest element in $\mathcal{C}$. Otherwise, $x_k$ is not compared until $S'$ is constructed. 
On the other hand, recall that the expected $f_k$ in the algorithm in~\cite{afshani:fragile-ESA19} is $O(\log \log m)$ where $m$ is the size of the input set. Hence, the expected $f_k$ after selecting the $k+1$-th element in  $\mathcal{C}$ is 1 when creating $S'$ plus the expected $f_k$ in the randomized selection algorithm in~\cite{afshani:fragile-ESA19} that is $\sum_{|S'|}O(\log \log |S'|)\mathbb{P}[|S'|]=\mathbb{E}[O(\log \log |S'|)]$. Thus, $\mathbb{E}[f_k]=(\mathbb{E}[f_k\text{ in \textsc{ReSet}}| x_k \in S]+\mathbb{E}[|\mathcal{C}|])\mathbb{P}[x_k \in S] + 0\mathbb{P}[x_k \notin S] + 1 + \mathbb{E}[O(\log \log |S'|)]$. Since the logarithm is a concave function,  $\mathbb{E}[O(\log \log |S'|)]\leq O(\log \log (\mathbb{E}[|S'|]))$.  
Therefore, if we prove that: (i) the expected \frag of $x_k$ before creating $S'$ is $O(1)$ and (ii) $\mathbb{E}[|S'|]=c'k^{c}$ for some constants $c$ and $c'$. Then, we obtain that $\mathbb{E}[f_k]\leq O(1)+1+O(c\log\log k+\log c') = O(\log \log k)$, as desired. In order to prove (i) and (ii) we consider 2 cases: (1) $k > n^{\frac{1}{100}}$, (2) $k \leq n^{\frac{1}{100}}$.
\\
\emph{Case 1)} $S' = X$ and it makes no previous comparisons in any element, proving (i). In addition, $S'$ has size less than $k^{100}$. Thus, (ii) holds. 
\\
\emph{Case 2)} $S$ is a sample of $X$ with size $\frac{n}{k}$ and $S'=\textsc{ReSet}(S, k)$. \\
First, let us show (i). If $x_k \notin S$, then there are no previous comparisons. Hence, the expected \frag of $x_k$ before constructing $S'$ is given by $(\mathbb{E}[f_k\text{ in \textsc{ReSet}}| x_k \in S'] + \mathbb{E}[|\mathcal{C}|])\mathbb{P}[x_k \in S]+ 0$. Since $S$ is an uniform random sample with size $\frac{n}{k}$, $\mathbb{P}[x_k \in S] = \frac{1}{k}$, it suffices to show that $\mathbb{E}[f_k\text{ in \textsc{ReSet}}| x_k \in S'] + \mathbb{E}[|\mathcal{C}|]=O(k)$, which gives an expectation of $O(k)\frac{1}{k}=O(1)$, proving (i). So, let us show that $\mathbb{E}[f_k\text{ in \textsc{ReSet}}| x_k \in S'] + \mathbb{E}[|\mathcal{C}|]=O(k)$. 
Let $A_0=S$ and let $A_1$ be the sample of $A_0$ when passing through line 6 in \textsc{ReSet}. Similarly, denote by $A_i$ to the sample of $A_{i-1}$ in the $i$-th recursive call of \textsc{ReSet} and let $A'_i=\textsc{ReSet}(A_{i}, k)$. Note that by definition $A'_0=\mathcal{C}$. 
Let $\ell+1$ be the number of recursive calls in $\textsc{ReSet}(S, k)$.  

Since $A_i$ is a uniform random sample of size $\tfrac{|A_{i-1}|}{2}$ for all $i\geq 1$, $\mathbb{P}[x \in A_i| x \in A_{i-1}]=2^{-1}$ and $\mathbb{P}[x \in A_i| x \notin A_{i-1}]=0$. Hence, $\mathbb{P}[x_{k} \in A_i]=\mathbb{P}[x_k \in \cap_{i=0}^{i} A_i]=2^{-i}$. Note that the number of comparisons of $x_k$ in \textsc{ReSet} is given by the number of times $x_k$ is compared in lines 8 and 9. Thus, for each $i$-th recursive call: if $x_k \in A_i$, then $x_k$ is compared once in line 9; and if $x_k \in A_i\cap A'_i$, then $x_k$ is compared at most $|A'_i|$ times in line 8. Otherwise, $x_k$ is not compared in that and the next iterations.  
Thus, $\mathbb{E}[f_k\text{ in \textsc{ReSet}}| x_k \in S'] + \mathbb{E}[|\mathcal{C}|] \leq \sum_{i=0}^{\ell} (1+\mathbb{E}[|A'_i|])\mathbb{P}[x_k \in A_i]= \sum_{i=0}^{\ell} 2^{-i}(1+\mathbb{E}[|A'_i|])\leq 2(1+ \mathbb{E}[|A'_i|])$. 
Let us compute $\mathbb{E}[|A'_i|]$.  
Since the ($\ell$$+$$1$)-th iteration $\textsc{ReSet}(A_{\ell},k)$ passes through the if in line 3, there is no new sample from $A_{\ell}$. 
Thus, $A'_{\ell}$ is given by the $k+1$ smallest elements of $A_{\ell}$. Therefore, $\mathbb{E}[|A'_{\ell}|]=k+1$ Denote by $a'^{i}_j$ to the $j$-th smallest element of $A'_i$. For the case of $0\leq i<\ell$, we have $A'_i=\{x\in A_{i+1} : x\leq a'^{i+1}_{k+1}\}$. Hence, $\mathbb{E}[|A'_i|]=\mathbb{E}[|\{x\in A_{i+1} : x\leq a'^{i+1}_{k+1}\}|] =\mathbb{E}[|\{x \in A_{i+1} : x \leq a'^{i+1}_1]\}|] + \sum_{j=1}^{k} \mathbb{E}[|\{x \in A_{i+1} :  a'^{i+1}_{j-1}< x \leq a'^{i+1}_{j}]\}|]\leq \sum_{j=1}^{k+1}\sum_{t=1}^{\infty} t2^{-1}(2^{-(t-1)})=2(k+1)$. 
Therefore, $\mathbb{E}[f_k\text{ in \textsc{ReSet}}| x_k \in S'] + \mathbb{E}[|\mathcal{C}|]=\sum_{i=0}^{\ell} 2^{-i}(1+\mathbb{E}[|A'_i|])\leq 2+ 2\mathbb{E}[|A'_i|]=O(k)$ proving (i). 
Finally, let us show (ii): For simplicity, let $c_j$ denote the $j$-th smallest element of $\mathcal{C}$. Then, $\mathbb{E}[|S'|]=\mathbb{E}[|\{x\in X: x\leq c_1\}|]+\sum_{j=1}^{k}\mathbb{E}[|\{x\in X: c_j \leq x\leq c_{j+1}\}|]\leq \sum_{j=1}^{k+1}\sum_{j=0}^{\infty} j k^{-1}(1-k^{-1})^{j-1}=k(k+1)=O(k^2)$, proving (ii). \qed
\end{proof}