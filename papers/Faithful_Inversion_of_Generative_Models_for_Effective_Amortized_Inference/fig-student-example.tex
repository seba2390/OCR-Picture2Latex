\begin{figure}[t]
    \centering
    \begin{subfigure}[b]{0.32\textwidth}
        \centering
        \includegraphics[scale=0.3]{student-bn.pdf}
        \caption{}
        \label{fig:student-bn}
    \end{subfigure}
    \begin{subfigure}[b]{0.32\textwidth}
        \centering
        \includegraphics[scale=0.3]{student-induced.pdf}
        \caption{}
        \label{fig:student-induced}
    \end{subfigure}
    \begin{subfigure}[b]{0.32\textwidth}
        \centering
        \includegraphics[scale=0.3]{student-inverse.pdf}
        \caption{}
        \label{fig:student-inverse}
    \end{subfigure}
    \caption{
    (a) Student BN example; 
    (b) Induced graph under elimination ordering $D,I,S,G,L,J,H$ (additional edges indicated with a dotted line); 
    (c) Inverse structure.
\vspace{-10pt}}
    \label{fig:student-example}
\end{figure}
