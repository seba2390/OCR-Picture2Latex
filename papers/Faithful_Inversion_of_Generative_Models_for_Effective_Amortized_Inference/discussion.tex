\section{Discussion}
We have presented NaMI, a tractable framework that, given the BN structure for a generative model, produces a natural factorization for its inverse that is a minimal I-map for the model.  We have argued that this should be used to guide the design of the coarse-grain structure of the inference network in amortized inference. Having empirically analyzed the implications of using NaMI, we find that it learns better inference networks than previous heuristic approaches.
We further found that, in the context of VAEs, improved inference networks have a knock-on effect on the generative network, improving the generative networks as well.

Our framework opens new possibilities for learning structured deep generative models that combine traditional Bayesian modeling by probabilistic graphical models with deep neural networks. This allows us to leverage our typically strong knowledge of which variables effect which others, while not overly relying on our weak knowledge of the exact functional form these relationships take.

To see this, note that if we forgo the niceties of making mean-field assumptions,
we can impose arbitrary structure on a generative model simply by
controlling its parameterization.  The only requirement on the generative network
to evaluate the ELBO is that
we can evaluate the network density at a given input.  Recent
advances in normalizing flows \citep{HuangEtAl2018, ChenEtAl2018} mean it is possible
to construct flexible and general purpose distributions that satisfy this requirement
and are amenable to application of dependency constraints from our
graphical model.  This obviates the need to make assumptions such
as conjugacy as done by, for example,~\citet{JohnsonEtAl2016}.

NaMI provides a critical component to constructing such a framework, as it allows one to ensure that the inference
network respects the structural assumptions imposed on the generative network,
without which a tight variational bound cannot be achieved.

%%% Local Variables:
%%% mode: latex
%%% TeX-master: "main"
%%% End:
