\begin{figure*}[t]
     \begin{center}
        \subfigure[]{
            \label{fig:example1-bn}
            \includegraphics[scale=0.5]{example1-bn.pdf}
        }\hspace{1.25cm}
				\subfigure[]{
            \label{fig:example1-brooks}
            \includegraphics[scale=0.5]{example1-brooks.pdf}
        }\hspace{1.25cm}
        \subfigure[]{
					\label{fig:example1-inverse}
					\includegraphics[scale=0.5]{example1-inverse.pdf}
        }\\ \vspace{0.5cm}
				\subfigure[]{
            \label{fig:example2-bn}
            \includegraphics[scale=0.5]{example2-bn.pdf}
        }\hspace{1.25cm}
				\subfigure[]{
            \label{fig:example2-brooks}
            \includegraphics[scale=0.5]{example2-brooks.pdf}
        }\hspace{1.25cm}
        \subfigure[]{
					\label{fig:example2-inverse}
					\includegraphics[scale=0.5]{example2-inverse.pdf}
        }
    \end{center}
    \caption[Simple BN example]{(a,d) Two simple BN structures for a generative model, (b,e) The corresponding inverse BN structures formed by Stuhlm{\"u}ller's Algorithm, (c,f) The inverse BN structure formed by our algorithm. This demonstrate how Stuhlm{\"u}ller's Algorithm can miss many edges and longer-term dependencies.}
   \label{fig:bn-examples}
\end{figure*}
