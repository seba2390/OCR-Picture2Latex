
%% bare_jrnl.tex
%% V1.4b
%% 2015/08/26
%% by Michael Shell
%% see http://www.michaelshell.org/
%% for current contact information.
%%
%% This is a skeleton file demonstrating the use of IEEEtran.cls
%% (requires IEEEtran.cls version 1.8b or later) with an IEEE
%% journal paper.
%%
%% Support sites:
%% http://www.michaelshell.org/tex/ieeetran/
%% http://www.ctan.org/pkg/ieeetran
%% and
%% http://www.ieee.org/

%%*************************************************************************
%% Legal Notice:
%% This code is offered as-is without any warranty either expressed or
%% implied; without even the implied warranty of MERCHANTABILITY or
%% FITNESS FOR A PARTICULAR PURPOSE! 
%% User assumes all risk.
%% In no event shall the IEEE or any contributor to this code be liable for
%% any damages or losses, including, but not limited to, incidental,
%% consequential, or any other damages, resulting from the use or misuse
%% of any information contained here.
%%
%% All comments are the opinions of their respective authors and are not
%% necessarily endorsed by the IEEE.
%%
%% This work is distributed under the LaTeX Project Public License (LPPL)
%% ( http://www.latex-project.org/ ) version 1.3, and may be freely used,
%% distributed and modified. A copy of the LPPL, version 1.3, is included
%% in the base LaTeX documentation of all distributions of LaTeX released
%% 2003/12/01 or later.
%% Retain all contribution notices and credits.
%% ** Modified files should be clearly indicated as such, including  **
%% ** renaming them and changing author support contact information. **
%%*************************************************************************


% *** Authors should verify (and, if needed, correct) their LaTeX system  ***
% *** with the testflow diagnostic prior to trusting their LaTeX platform ***
% *** with production work. The IEEE's font choices and paper sizes can   ***
% *** trigger bugs that do not appear when using other class files.       ***                          ***
% The testflow support page is at:
% http://www.michaelshell.org/tex/testflow/



\documentclass[journal]{IEEEtran}
%
% If IEEEtran.cls has not been installed into the LaTeX system files,
% manually specify the path to it like:
% \documentclass[journal]{../sty/IEEEtran}





% Some very useful LaTeX packages include:
% (uncomment the ones you want to load)


% *** MISC UTILITY PACKAGES ***
%
%\usepackage{ifpdf}
% Heiko Oberdiek's ifpdf.sty is very useful if you need conditional
% compilation based on whether the output is pdf or dvi.
% usage:
% \ifpdf
%   % pdf code
% \else
%   % dvi code
% \fi
% The latest version of ifpdf.sty can be obtained from:
% http://www.ctan.org/pkg/ifpdf
% Also, note that IEEEtran.cls V1.7 and later provides a builtin
% \ifCLASSINFOpdf conditional that works the same way.
% When switching from latex to pdflatex and vice-versa, the compiler may
% have to be run twice to clear warning/error messages.






% *** CITATION PACKAGES ***
%
\usepackage{booktabs}
\usepackage{graphicx}
\usepackage{cite}
\usepackage{amsmath}
\usepackage{amssymb}
\usepackage{multirow}
\usepackage{color}
% cite.sty was written by Donald Arseneau
% V1.6 and later of IEEEtran pre-defines the format of the cite.sty package
% \cite{} output to follow that of the IEEE. Loading the cite package will
% result in citation numbers being automatically sorted and properly
% "compressed/ranged". e.g., [1], [9], [2], [7], [5], [6] without using
% cite.sty will become [1], [2], [5]--[7], [9] using cite.sty. cite.sty's
% \cite will automatically add leading space, if needed. Use cite.sty's
% noadjust option (cite.sty V3.8 and later) if you want to turn this off
% such as if a citation ever needs to be enclosed in parenthesis.
% cite.sty is already installed on most LaTeX systems. Be sure and use
% version 5.0 (2009-03-20) and later if using hyperref.sty.
% The latest version can be obtained at:
% http://www.ctan.org/pkg/cite
% The documentation is contained in the cite.sty file itself.






% *** GRAPHICS RELATED PACKAGES ***
%
\ifCLASSINFOpdf
  % \usepackage[pdftex]{graphicx}
  % declare the path(s) where your graphic files are
  % \graphicspath{{../pdf/}{../jpeg/}}
  % and their extensions so you won't have to specify these with
  % every instance of \includegraphics
  % \DeclareGraphicsExtensions{.pdf,.jpeg,.png}
\else
  % or other class option (dvipsone, dvipdf, if not using dvips). graphicx
  % will default to the driver specified in the system graphics.cfg if no
  % driver is specified.
  % \usepackage[dvips]{graphicx}
  % declare the path(s) where your graphic files are
  % \graphicspath{{../eps/}}
  % and their extensions so you won't have to specify these with
  % every instance of \includegraphics
  % \DeclareGraphicsExtensions{.eps}
\fi
% graphicx was written by David Carlisle and Sebastian Rahtz. It is
% required if you want graphics, photos, etc. graphicx.sty is already
% installed on most LaTeX systems. The latest version and documentation
% can be obtained at: 
% http://www.ctan.org/pkg/graphicx
% Another good source of documentation is "Using Imported Graphics in
% LaTeX2e" by Keith Reckdahl which can be found at:
% http://www.ctan.org/pkg/epslatex
%
% latex, and pdflatex in dvi mode, support graphics in encapsulated
% postscript (.eps) format. pdflatex in pdf mode supports graphics
% in .pdf, .jpeg, .png and .mps (metapost) formats. Users should ensure
% that all non-photo figures use a vector format (.eps, .pdf, .mps) and
% not a bitmapped formats (.jpeg, .png). The IEEE frowns on bitmapped formats
% which can result in "jaggedy"/blurry rendering of lines and letters as
% well as large increases in file sizes.
%
% You can find documentation about the pdfTeX application at:
% http://www.tug.org/applications/pdftex





% *** MATH PACKAGES ***
%
%\usepackage{amsmath}
% A popular package from the American Mathematical Society that provides
% many useful and powerful commands for dealing with mathematics.
%
% Note that the amsmath package sets \interdisplaylinepenalty to 10000
% thus preventing page breaks from occurring within multiline equations. Use:
%\interdisplaylinepenalty=2500
% after loading amsmath to restore such page breaks as IEEEtran.cls normally
% does. amsmath.sty is already installed on most LaTeX systems. The latest
% version and documentation can be obtained at:
% http://www.ctan.org/pkg/amsmath





% *** SPECIALIZED LIST PACKAGES ***
%
%\usepackage{algorithmic}
% algorithmic.sty was written by Peter Williams and Rogerio Brito.
% This package provides an algorithmic environment fo describing algorithms.
% You can use the algorithmic environment in-text or within a figure
% environment to provide for a floating algorithm. Do NOT use the algorithm
% floating environment provided by algorithm.sty (by the same authors) or
% algorithm2e.sty (by Christophe Fiorio) as the IEEE does not use dedicated
% algorithm float types and packages that provide these will not provide
% correct IEEE style captions. The latest version and documentation of
% algorithmic.sty can be obtained at:
% http://www.ctan.org/pkg/algorithms
% Also of interest may be the (relatively newer and more customizable)
% algorithmicx.sty package by Szasz Janos:
% http://www.ctan.org/pkg/algorithmicx




% *** ALIGNMENT PACKAGES ***
%
%\usepackage{array}
% Frank Mittelbach's and David Carlisle's array.sty patches and improves
% the standard LaTeX2e array and tabular environments to provide better
% appearance and additional user controls. As the default LaTeX2e table
% generation code is lacking to the point of almost being broken with
% respect to the quality of the end results, all users are strongly
% advised to use an enhanced (at the very least that provided by array.sty)
% set of table tools. array.sty is already installed on most systems. The
% latest version and documentation can be obtained at:
% http://www.ctan.org/pkg/array


% IEEEtran contains the IEEEeqnarray family of commands that can be used to
% generate multiline equations as well as matrices, tables, etc., of high
% quality.




% *** SUBFIGURE PACKAGES ***
%\ifCLASSOPTIONcompsoc
%  \usepackage[caption=false,font=normalsize,labelfont=sf,textfont=sf]{subfig}
%\else
%  \usepackage[caption=false,font=footnotesize]{subfig}
%\fi
% subfig.sty, written by Steven Douglas Cochran, is the modern replacement
% for subfigure.sty, the latter of which is no longer maintained and is
% incompatible with some LaTeX packages including fixltx2e. However,
% subfig.sty requires and automatically loads Axel Sommerfeldt's caption.sty
% which will override IEEEtran.cls' handling of captions and this will result
% in non-IEEE style figure/table captions. To prevent this problem, be sure
% and invoke subfig.sty's "caption=false" package option (available since
% subfig.sty version 1.3, 2005/06/28) as this is will preserve IEEEtran.cls
% handling of captions.
% Note that the Computer Society format requires a larger sans serif font
% than the serif footnote size font used in traditional IEEE formatting
% and thus the need to invoke different subfig.sty package options depending
% on whether compsoc mode has been enabled.
%
% The latest version and documentation of subfig.sty can be obtained at:
% http://www.ctan.org/pkg/subfig




% *** FLOAT PACKAGES ***
%
%\usepackage{fixltx2e}
% fixltx2e, the successor to the earlier fix2col.sty, was written by
% Frank Mittelbach and David Carlisle. This package corrects a few problems
% in the LaTeX2e kernel, the most notable of which is that in current
% LaTeX2e releases, the ordering of single and double column floats is not
% guaranteed to be preserved. Thus, an unpatched LaTeX2e can allow a
% single column figure to be placed prior to an earlier double column
% figure.
% Be aware that LaTeX2e kernels dated 2015 and later have fixltx2e.sty's
% corrections already built into the system in which case a warning will
% be issued if an attempt is made to load fixltx2e.sty as it is no longer
% needed.
% The latest version and documentation can be found at:
% http://www.ctan.org/pkg/fixltx2e


%\usepackage{stfloats}
% stfloats.sty was written by Sigitas Tolusis. This package gives LaTeX2e
% the ability to do double column floats at the bottom of the page as well
% as the top. (e.g., "\begin{figure*}[!b]" is not normally possible in
% LaTeX2e). It also provides a command:
%\fnbelowfloat
% to enable the placement of footnotes below bottom floats (the standard
% LaTeX2e kernel puts them above bottom floats). This is an invasive package
% which rewrites many portions of the LaTeX2e float routines. It may not work
% with other packages that modify the LaTeX2e float routines. The latest
% version and documentation can be obtained at:
% http://www.ctan.org/pkg/stfloats
% Do not use the stfloats baselinefloat ability as the IEEE does not allow
% \baselineskip to stretch. Authors submitting work to the IEEE should note
% that the IEEE rarely uses double column equations and that authors should try
% to avoid such use. Do not be tempted to use the cuted.sty or midfloat.sty
% packages (also by Sigitas Tolusis) as the IEEE does not format its papers in
% such ways.
% Do not attempt to use stfloats with fixltx2e as they are incompatible.
% Instead, use Morten Hogholm'a dblfloatfix which combines the features
% of both fixltx2e and stfloats:
%
% \usepackage{dblfloatfix}
% The latest version can be found at:
% http://www.ctan.org/pkg/dblfloatfix




%\ifCLASSOPTIONcaptionsoff
%  \usepackage[nomarkers]{endfloat}
% \let\MYoriglatexcaption\caption
% \renewcommand{\caption}[2][\relax]{\MYoriglatexcaption[#2]{#2}}
%\fi
% endfloat.sty was written by James Darrell McCauley, Jeff Goldberg and 
% Axel Sommerfeldt. This package may be useful when used in conjunction with 
% IEEEtran.cls'  captionsoff option. Some IEEE journals/societies require that
% submissions have lists of figures/tables at the end of the paper and that
% figures/tables without any captions are placed on a page by themselves at
% the end of the document. If needed, the draftcls IEEEtran class option or
% \CLASSINPUTbaselinestretch interface can be used to increase the line
% spacing as well. Be sure and use the nomarkers option of endfloat to
% prevent endfloat from "marking" where the figures would have been placed
% in the text. The two hack lines of code above are a slight modification of
% that suggested by in the endfloat docs (section 8.4.1) to ensure that
% the full captions always appear in the list of figures/tables - even if
% the user used the short optional argument of \caption[]{}.
% IEEE papers do not typically make use of \caption[]'s optional argument,
% so this should not be an issue. A similar trick can be used to disable
% captions of packages such as subfig.sty that lack options to turn off
% the subcaptions:
% For subfig.sty:
% \let\MYorigsubfloat\subfloat
% \renewcommand{\subfloat}[2][\relax]{\MYorigsubfloat[]{#2}}
% However, the above trick will not work if both optional arguments of
% the \subfloat command are used. Furthermore, there needs to be a
% description of each subfigure *somewhere* and endfloat does not add
% subfigure captions to its list of figures. Thus, the best approach is to
% avoid the use of subfigure captions (many IEEE journals avoid them anyway)
% and instead reference/explain all the subfigures within the main caption.
% The latest version of endfloat.sty and its documentation can obtained at:
% http://www.ctan.org/pkg/endfloat
%
% The IEEEtran \ifCLASSOPTIONcaptionsoff conditional can also be used
% later in the document, say, to conditionally put the References on a 
% page by themselves.




% *** PDF, URL AND HYPERLINK PACKAGES ***
%
%\usepackage{url}
% url.sty was written by Donald Arseneau. It provides better support for
% handling and breaking URLs. url.sty is already installed on most LaTeX
% systems. The latest version and documentation can be obtained at:
% http://www.ctan.org/pkg/url
% Basically, \url{my_url_here}.




% *** Do not adjust lengths that control margins, column widths, etc. ***
% *** Do not use packages that alter fonts (such as pslatex).         ***
% There should be no need to do such things with IEEEtran.cls V1.6 and later.
% (Unless specifically asked to do so by the journal or conference you plan
% to submit to, of course. )


% correct bad hyphenation here
\hyphenation{op-tical net-works semi-conduc-tor}



\begin{document}
%\tableofcontents
% paper title
% Titles are generally capitalized except for words such as a, an, and, as,
% at, but, by, for, in, nor, of, on, or, the, to and up, which are usually
% not capitalized unless they are the first or last word of the title.
% Linebreaks \\ can be used within to get better formatting as desired.
% Do not put math or special symbols in the title.
\title{Video-based Person Re-identification with Accumulative Motion Context}
%\title{End-to-end Two-stream Recurrent Convolutional Network for Video-Based Person Re-identification}
%
%
% author names and IEEE memberships
% note positions of commas and nonbreaking spaces ( ~ ) LaTeX will not break
% a structure at a ~ so this keeps an author's name from being broken across
% two lines.
% use \thanks{} to gain access to the first footnote area
% a separate \thanks must be used for each paragraph as LaTeX2e's \thanks
% was not built to handle multiple paragraphs
%

\author{Hao~Liu, Zequn Jie, Karlekar Jayashree, Meibin Qi, Jianguo Jiang
	and Shuicheng Yan, \IEEEmembership{Fellow,~IEEE}, Jiashi Feng% <-this % stops a space
	\thanks{Hao Liu, Meibin Qi and Jianguo Jiang are with School of Computer and Information, Hefei University of Technology, P. R. China. Hao Liu is also with Department of Electrical and Computer Engineering, National University of Singapore, Singapore, e-mail: hfut.haoliu@gmail.com; qimeibin@163.com; jgjiang@hfut.edu.cn.}% <-this % stops a space
	\thanks{Karlekar Jayashree is with Panasonic R\&D Center Singapore, Singapore, e-mail: Karlekar.Jayashree@sg.panasonic.com. }
	\thanks{Zequn Jie, Shuicheng Yan and Jiashi Feng are with Department of Electrical and Computer Engineering, National University of Singapore, Singapore, e-mail: zequn.nus@gmail.com, eleyans@nus.edu.sg, elefjia@nus.edu.sg.}}%


% note the % following the last \IEEEmembership and also \thanks - 
% these prevent an unwanted space from occurring between the last author name
% and the end of the author line. i.e., if you had this:
% 
% \author{....lastname \thanks{...} \thanks{...} }
%                     ^------------^------------^----Do not want these spaces!
%
% a space would be appended to the last name and could cause every name on that
% line to be shifted left slightly. This is one of those "LaTeX things". For
% instance, "\textbf{A} \textbf{B}" will typeset as "A B" not "AB". To get
% "AB" then you have to do: "\textbf{A}\textbf{B}"
% \thanks is no different in this regard, so shield the last } of each \thanks
% that ends a line with a % and do not let a space in before the next \thanks.
% Spaces after \IEEEmembership other than the last one are OK (and needed) as
% you are supposed to have spaces between the names. For what it is worth,
% this is a minor point as most people would not even notice if the said evil
% space somehow managed to creep in.



% The paper headers
%\markboth{IEEE Transactions on Circuits and Systems for Video Technology,~Vol.~XX, No.~XX, XXXX~2016}%
%{Shell \MakeLowercase{\textit{et al.}}: Bare Demo of IEEEtran.cls for IEEE Journals}
% The only time the second header will appear is for the odd numbered pages
% after the title page when using the twoside option.
% 
% *** Note that you probably will NOT want to include the author's ***
% *** name in the headers of peer review papers.                   ***
% You can use \ifCLASSOPTIONpeerreview for conditional compilation here if
% you desire.



% If you want to put a publisher's ID mark on the page you can do it like
% this:




% Remember, if you use this you must call \IEEEpubidadjcol in the second
% column for its text to clear the IEEEpubid mark.


% use for special paper notices
%\IEEEspecialpapernotice{(Invited Paper)}




% make the title area
\maketitle

\IEEEpubid{\begin{minipage}{\textwidth}\ \\[12pt] \centering
		 Copyright \copyright  20xx IEEE. Personal use of this material is permitted. However, permission to use this material for any other purposes must be obtained from the IEEE by sending an email to pubs-permissions@ieee.org.		
\end{minipage}}




% As a general rule, do not put math, special symbols or citations
% in the abstract or keywords.
\begin{abstract}
Video based person re-identification plays a central role in realistic security and video surveillance. 
In this paper we propose a novel Accumulative Motion Context  (AMOC)  network for addressing this important problem, which effectively exploits  the long-range motion context for robustly identifying the same person under challenging conditions. Given a video sequence of the same or different persons, the proposed AMOC network jointly learns appearance representation and motion context from a collection of adjacent frames using a two-stream convolutional architecture. Then AMOC accumulates clues from motion context by recurrent aggregation,   allowing effective information flow among adjacent frames and capturing dynamic gist of the persons.  The architecture of AMOC is end-to-end trainable and thus motion context can be adapted to complement appearance clues under unfavorable conditions (\textit{e.g.}, occlusions). Extensive experiments are conduced on three public benchmark datasets,  \textit{i.e.}, the iLIDS-VID, PRID-2011 and MARS datasets, to investigate the performance of AMOC. The experimental results demonstrate that the proposed AMOC network outperforms  state-of-the-arts for video-based re-identification significantly and confirm the advantage of exploiting long-range motion context for video based person re-identification, validating our motivation evidently.
\end{abstract}

% Note that keywords are not normally used for peerreview papers.
\begin{IEEEkeywords}
video surveillance, person re-identification, accumulative motion context
\end{IEEEkeywords}






% For peer review papers, you can put extra information on the cover
% page as needed:
% \ifCLASSOPTIONpeerreview
% \begin{center} \bfseries EDICS Category: 3-BBND \end{center}
% \fi
%
% For peerreview papers, this IEEEtran command inserts a page break and
% creates the second title. It will be ignored for other modes.
\IEEEpeerreviewmaketitle



\section{Introduction}
% The very first letter is a 2 line initial drop letter followed
% by the rest of the first word in caps.
% 
% form to use if the first word consists of a single letter:
% \IEEEPARstart{A}{demo} file is ....
% 
% form to use if you need the single drop letter followed by
% normal text (unknown if ever used by the IEEE):
% \IEEEPARstart{A}{}demo file is ....
% 
% Some journals put the first two words in caps:
% \IEEEPARstart{T}{his demo} file is ....
% 
% Here we have the typical use of a "T" for an initial drop letter
% and "HIS" in caps to complete the first word.
\IEEEPARstart{T}{he} person re-identification (re-id) problem has received increasing attention \cite{farenzena2010person, li2014deepreid, yi2014deep, ahmed2015improved, Chen2016An, Wang2015Cross, liu2016end, liu2015kernelized, yan2016person,mclaughlinrecurrent, wang2016person, you2016top, liu2015spatio, zhu2016video, gao2016temporally}, which associates different tracks of a person moving between non-overlapping cameras distributed at different physical locations. It aims to re-identify the same person captured by one camera in another camera at a new location. This task is essential to many important surveillance applications such as people tracking and forensic search. However, it is still a  challenging problem because pedestrian images captured in different camera views could display large variations in lightings, poses, viewpoints, and cluttered backgrounds. 

\IEEEpubidadjcol

In this work, we investigate the problem of video-based person re-identification. The recent  state-of-the-art methods often solve the person re-id task  by matching  spatial appearance features (\textit{e.g.} color and texture) using a pair of still images captured for persons~\cite{farenzena2010person, li2014deepreid, yi2014deep, ahmed2015improved, liu2016end, liu2015kernelized}.  However, appearance feature representations learned from still images of people are intrinsically limited due to the inherent visual ambiguity caused by, \textit{e.g.}, clothing similarity among people in public places and appearance variance from cross-camera viewing condition variations. In this work, we propose to explore both appearance and  motion information from video frame sequences of people for re-id.  The video setting is indeed more natural for re-id considering  a person is often captured by surveillance cameras in a video rather than only  a single image. By using sequences of persons' video frames, not only the spatial appearance but also the temporal information, such as persons' gait,  can be utilized to discern difficult cases when trying to recognize a person in another different camera. In general, describing person video can naturally be attributed to spatial and temporal cues. The spatial part  carries information about scenes and appearance information of the persons, like clothing color, height and shape, while the temporal part, in the form of motion across frames, conveys the movement of the observer (the cameras) and the person objects, which is complementary to the spatial part. 

Recently, spatial-temporal information has been explored in video-based person re-id works\cite{yan2016person,mclaughlinrecurrent, wang2016person, you2016top, liu2015spatio, zhu2016video, gao2016temporally}. In  \cite{mclaughlinrecurrent}, the authors extract optical flows between consecutive frames of a person to represent the short-term temporal information, then concatenate them with RGB images to construct the spatial-temporal features. However, this method simply uses one CNN to extract spatial and temporal information simultaneously. The limitation of current single-stream CNN architecture is that it cannot take  advantage of valuable temporal information and the performance is consequently limited by only using spatial (appearance) features. Thus the learned spatial-temporal features may not be sufficiently discriminative for identifying different persons in real videos.

\IEEEpubidadjcol

To overcome the limitation of single-stream CNN architecture, in related fields, such as action recognition (\cite{simonyan2014two, feichtenhofer2016convolutional}), the authors decompose spatial and temporal information processing into two individual streams to separately learn more representative features  from video sequences, and then the information from the two streams is fused at certain intermediate layers. For the person re-id task, in the process of discriminating two walking persons, at a specific location such as legs or arms of a person, they may have discriminative motion characteristics except for some spatial features such as trousers' colour or texture. As it can only be obtained between two consecutive frames, this motion is called motion context. Therefore, we design a novel two-stream spatial-temporal architecture, called the Accumulative Motion Context networks (AMOC), that can well capture the spatial features and motion context from sequences of persons. And then the useful appearance and motion information is accumulated  in a recurrent way for providing long-term information. 



Furthermore, to our best knowledge, the motion information in most of related methods~\cite{simonyan2014two, feichtenhofer2016convolutional, mclaughlinrecurrent} is represented by hand-crafted optical flow extracted in an off-line way. Such off-line optical flow extraction has the following disadvantages: (i) The pre-extraction of optical flow is time and resources consuming due to the storage demanding, especially for the large-scale dataset, such as MARS~\cite{zheng2016mars}; (ii) Off-line extracted optical flow may not be optimal for the person re-id task since it is independent of person re-id. Instead, we expect the model to automatically learn motion information from video sequential frames discriminative for person re-id. To encode the motion information from raw video frames, a recent work called FlowNet\cite{dosovitskiy2015flownet} is proposed to estimate high quality optical flow. Basically, the idea of FlowNet is to exploit the ability of convolutional networks to learn strong features at multiple levels of scale and abstraction, which can help finding the actual correspondences between frames. The FlowNet consists of contracting and expanding parts trained as a whole using back-propagation. The optical flow is first learned and spatially compressed in contracting part and then refined to high resolution in expanding part. Inspired by FlowNet, we design a new motion network in our AMOC architecture to perform the end-to-end motion context information learning task. Although our proposed end-to-end AMOC also involves estimating the optical flow (encoding motion information) from raw video frames similar to FlowNet, there are still several distinct differences with that model, which are listed as follows: 
\begin{itemize} 
\item First, They have different targets. FlowNet is for computing low-level optical flow. In contrast, our proposed AMOC model is to utilize the flow information for performing high-level person re-identification tasks. Due to different application targets, the optimization goal of FlowNet is to minimize the endpoint errors (EPE)\cite{dosovitskiy2015flownet} for getting more precise optical flow while our AMOC is to jointly re-identify persons and estimate optical flow towards benefiting person re-id.   

\item Second, the motion net in our model works together with the spatial stream to learn both appearance feature and motion feature which are complementary to each other. By contrast, FlowNet only focuses on learning the motion information encoded by the optical flow.

\item  Last but not least, considering the data in person re-id datasets that we experimented on has much lower resolution than the synthetic data that FlowNet uses, the structure of relevant layers in motion net is essentially modified to allow our model to take in low resolution video frames, especially for the person video frames, as input.

\end{itemize} 


Our experiments and analyses show our end-to-end trainable two-stream approach successfully learns motion features helpful for improving the performance of person re-identification. Moreover, we explore how to fuse the learned appearance feature and motion feature together can make our model perform best. Compared to other state-of-the-art video person re-id methods, our method is also able to achieve  better performance, as clearly validated by extensive experimental results. 

To sum up, we make the following contributions to video-based person re-identification:

\begin{itemize} 

\item We propose a novel two-stream motion context accumulating network model which can directly learn both spatial features and motion context from raw person video frames in an end-to-end manner, instead of pre-extracting optical flow using the off-line algorithm. Then the learned spatial appearance features and temporal motion context information are accumulated in a recurrent way.

\item We quantitatively validate the good performance of our end-to-end two-stream recurrent convolutional network by comparing it with the state-of-the-arts on three benchmark datasets: iLIDS-VID~\cite{wang2014person}, PRID-2011~\cite{hirzer2011person} and MARS~\cite{zheng2016mars}.
\end{itemize}

\section {Related work}
Person re-identification has been extensively studied in recent years. Existing works on person re-identification can be roughly divided into two types: person re-id for still images and  person re-id for video sequences.
\subsection{Person Re-id for Still Images}
Previous works on person re-id for still images focus on the invariant feature representation and distance metric learning. Discriminative features that are invariant to environmental and view-point changes play a determining  role in person re-id performance.~\cite{gray2008viewpoint} combines spatial and color information using an ensemble of discriminant localized features and classifiers selected by boosting to improve viewpoint invariance. Symmetry and asymmetry perceptual attributes are exploited in~\cite{farenzena2010person}, based on the idea that features closer to the body axes of symmetry are more robust against scene clutter.~\cite{cheng2011custom} fits a body configuration composed of chest, head, thighs, and legs in pedestrian images and extracts per-part color information as well as color displacement within the whole body to handle pose variation.~\cite{ma2012local} turns local descriptors into the Fisher Vector to produce a global representation of an image. Kviatkovsky \textit{et al.}~\cite{kviatkovsky2013color} propose a novel illumination-invariant feature representation based on the logchromaticity (log) color space and demonstrate that color as a single cue shows relatively good performance in identifying persons under greatly varying imaging conditions.

After feature extraction, distance metric learning is used in person re-id to emphasize inter-person distance and de-emphasize intra-person distance. Large margin nearest neighbor metric (LMNN)~\cite{weinberger2009distance} is proposed to improve the performance of the traditional kNN classification. Prosser \textit{et al.}~\cite{prosser2010person} formulate person re-identification as a ranking problem, and use ensembled RankSVMs to learn pairwise similarity. Zheng \textit{et al.}~\cite{zheng2013reidentification} propose a soft discriminative scheme termed relative distance comparison (RDC) by large and small distances corresponding to wrong matches and right matches, respectively.~\cite{liao2015person} proposes a high dimensional representation of color and Scale Invariant Local Ternary Pattern (SILTP) histograms. It constructs a histogram of pixel features, and then takes its maximum values within horizontal strips to tackle viewpoint variations while maintaining local discrimination.

\subsection{Person Re-id for Video Sequences}

Recently, some works consider performing person re-id in video sequences. In this scenario, the person matching problem in videos is crucial in exploiting multiple frames in videos to boost the performance.~\cite{simonnet2012re} applies Dynamic Time Warping (DTW) to solve the sequence matching problem in video-based person re-id.~\cite{karaman2012identity} uses a conditional random field (CRF) to ensure similar frames in a video sequence to receive similar labels.~\cite{li2013locally} proposes to learn to map between the appearances in sequences by taking into account the differences between specific camera pairs.~\cite{wang2014person} introduces a pictorial video segmentation approach and deploys a fragment selecting and ranking model for person matching.~\cite{karanam2015sparse} introduces a block sparse model to handle the video-based person re-id problem by the recovery problem in the embedding space.~\cite{liu2015spatio} proposes a spatio-temporal appearance representation method, and feature vectors that encode the spatially and temporally aligned appearance of the person in a walking cycle are extracted.~\cite{you2016top} proposes a top-push distance learning (TDL) model incorporating a top-push constraint to quantify ambiguous video representation for video-based person re-id.

Deep learning methods have also been applied to video-based person re-id to simultaneously solve feature representation and metric learning problem. Usually DNNs are used to learn ranking functions based on pairs~\cite{yi2014deep} or triplets~\cite{ding2015deep} of images. Such methods typically rely on a deep network (\emph{e.g.} Siamese network~\cite{hadsell2006dimensionality}) used for feature mapping from raw images to a feature space where images from the same person are close while images from different persons are widely separated.~\cite{mclaughlinrecurrent} uses a recurrent neural network to learn the interaction between multiple frames in a video and a Siamese network to learn the discriminative video-level features for person re-id.~\cite{yan2016person} uses the Long-Short Term Memory (LSTM) network to aggregate frame-wise person features in a recurrent manner. Unlike the existing deep learning based methods for person re-id in videos, our proposed Accumulative Motion Context (AMOC) networks introduces an end-to-end two-stream architecture that has specialized network streams  for learning spatial appearance and temporal feature representations individually. Spatial appearance information from the raw video frame input and temporal motion information from the optical flow predicted by the motion network are processed respectively and then fused at  higher recurrent layers to form a discriminative video-level representation.



\begin{figure*}[htb]
	\centering
	\includegraphics[height=9cm]{./framework.pdf}
	\vspace{-4.5mm}	
	\caption{The architecture of our proposed  Accumulative Motion Context Network (AMOC). At each time-step, each pair of two consecutive frames is processed by a two-stream network, including spatial network (yellow and green boxes) and motion network (blue box), to learn spatial appearance and temporal motion feature representations. Then these two-stream features are fused in a recurrent way for learning discriminative accumulative motion contexts. The two-steam features are integrated by temporal pooling layer from an arbitrarily long video sequence into a single feature representation. Finally, the two-stream sub-networks for two sequences from two different cameras are constructed following the Siamese network architecture\cite{hadsell2006dimensionality} in which the parameters of Camera A networks and Camera B networks are shared. The whole  AMOC network is end-to-end trained by introducing multi-task losses (classification loss and contrastive loss) to satisfy both the contrastive objective and to predict the person’s identity.}
	
	%\vspace{-5.5mm}	
	\label{fig:framework}
\end{figure*}
\section{Proposed Method}

We propose an end-to-end Accumulative Motion Context Network (AMOC) based architecture that addresses the video person re-identification problem through joint spatial appearance learning and motion context accumulating from raw video frames. We first introduce the overall architecture of our AMOC model (\ref{arch}), which is illustrated in Fig. \ref{fig:framework}. Then for each pair of frames from a person sequence, the details of motion networks and spatial networks  and how they collaborate with each other are described.  Besides, the recurrent fusion layers to integrate two-stream spatial-temporal information and fusion method are elaborated. Finally, implementation details of training and test are introduced for reproducing the results.
% An example of a floating figure using the graphicx package.
% Note that \label must occur AFTER (or within) \caption.
% For figures, \caption should occur after the \includegraphics.
% Note that IEEEtran v1.7 and later has special internal code that
% is designed to preserve the operation of \label within \caption
% even when the captionsoff option is in effect. However, because
% of issues like this, it may be the safest practice to put all your
% \label just after \caption rather than within \caption{}.
%
% Reminder: the "draftcls" or "draftclsnofoot", not "draft", class
% option should be used if it is desired that the figures are to be
% displayed while in draft mode.
%
%\begin{figure}[!t]
%\centering
%\includegraphics[width=2.5in]{myfigure}
% where an .eps filename suffix will be assumed under latex, 
% and a .pdf suffix will be assumed for pdflatex; or what has been declared
% via \DeclareGraphicsExtensions.
%\caption{Simulation results for the network.}
%\label{fig_sim}
%\end{figure}

\subsection{Architecture Overview }\label{arch}
Fig.~\ref{fig:framework} illustrates the architecture of the proposed end-to-end Accumulative Motion Context network (AMOC). In our architecture each two consecutive frames is processed by a two-stream network to learn spatial appearance and temporal motion features representing the person's appearance at a certain time instant. Then these two-stream features are fused in a recurrent way for learning discriminative accumulative motion contexts. After that, the temporal pooling layer integrates the two-steam features in time from an arbitrarily long video sequence into a single feature representation. Finally, the two-stream sub-networks for two sequences from two different cameras are constructed following the Siamese network architecture~\cite{hadsell2006dimensionality} in which the parameters of Camera A networks and Camera B networks are shared. To end-to-end train this network, we adopt multi-task loss functions including  
contrastive loss and classification loss. The contrastive loss decides whether two sequences describe the same person or not while the classification loss predicts the identity of the person in the sequence. In the following, we will give more detailed explanations to each component of our proposed network. 

% Note that the IEEE typically puts floats only at the top, even when this
% results in a large percentage of a column being occupied by floats.
\subsection{End-to-end Two-stream  Networks }
As aforementioned,  each frame in a sequence of the person is processed by  two convolutional network streams jointly. Specifically, one stream uses spatial networks (yellow boxes in Fig. \ref{fig:framework}) to learn spatial features from raw video frames while for the other stream spatial networks (green boxes in Fig.~\ref{fig:framework}) are applied on the motion context features produced by motion networks (blue boxes in Fig.~\ref{fig:framework}) to learn the temporal feature representations at each location between a pair of video frames. In this subsection, we introduce the details of the motion networks and spatial networks, and then describe how they work together. 
\subsubsection{The Motion Networks}\label{motion_net}
As shown in Fig.~\ref{fig:flownet}, at each time-step a pair of consecutive video frames of a person is processed by the motion network within AMOC (corresponding to blue boxes in Fig. \ref{fig:framework}) to predict motion between the adjacent frames. Similar to the structure used in \cite{dosovitskiy2015flownet}, the motion network consists of several convolutional and  deconvolutional layers for up-sizing the high-level coarse feature maps learned by convolutional layers. In particular, it has 6 convolutional layers (corresponding to ``Conv1'', ``Conv\_1'', ``Conv2'', ``Conv2\_1'',``Conv3'' and ``Conv3\_1'' ) with stride of 2 (the simplest form of pooling) in six of them and a tanh non-linearity after each layer.  Taking the concatenated two person frames as inputs with size of $h \times w \times 6$ (\textit{h} is the height of the frame and \textit{w} is the width), the network employs 6 convolutional layers  to learn high-level abstract representations of the frame pair by producing the feature maps with reduced sizes (\textit{w.r.t.} the raw input frames). However, this size shrinking process could result in low resolution and harm the performance of person re-id. So in order to provide dense per-pixel predictions we need to refine the coarse pooled representation.  To perform the refinement, we apply the ``deconvolution" on feature maps, and concatenate them with corresponding feature maps from the ``contractive" part of the network and an upsampled coarser flow prediction (``Pred1'', ``Pred2'', ``Pred3''). In this way, the networks could preserve both the high-level information passed from coarser feature maps and refine local information provided in lower layer feature maps. Each step increases the resolution by a factor of 2. We repeat this process for twice, leading to the final predicted flow (``Pred3'') whose resolution is still as half as that of raw input. Note that our motion networks do not have any fully connected layers, which can take video frames of arbitrary size as input. This motion network can be end-to-end trained by using optical flow generated by several off-the-shelf algorithms as supervision, such as Lucas-Kanade \cite{lucas1981iterative} and EpicFlow\cite{revaud2015epicflow} algorithm. The training details will be described in Sec. \ref{pre_motion}.
\begin{figure*}[htb]
	\centering
	\includegraphics[width=16cm]{./flownet.pdf}
	\vspace{-3.5mm}	
	\caption{The structure of motion networks of our proposed accumulative motion context network at one time-step.  It has 6 convolutional layers (corresponding to ``Conv1'', ``Conv\_1'', ``Conv2'', ``Conv2\_1'',``Conv3'' and ``Conv3\_1'' ) with stride of 2 in six of them and a tanh non-linearity layer after each layer. The inputs are the concatenated two person frames with size of $h \times w \times 6$. To provide dense per-pixel predictions, several deconvolutional layers are applied on output feature maps of convolutional layers and motion predictions to refine the coarse pooled representation. The purple cubes represent the convolutional kernels while the green cubes are the deconvolutional kernels. }
	
	%\vspace{-5.5mm}	
	\label{fig:flownet}
\end{figure*}

\subsubsection{The Spatial Networks}\label{spatial}
\begin{figure}[htb]
	\centering
	\includegraphics[width=8.5cm]{./spatnet.pdf}
	\vspace{-3.5mm}	
	\caption{The structure of spatial networks of our proposed accumulative motion context network at one time-step. It consists of 3 convolutional layers and 3 max-pooling layers with a \textit{tanh} non-linearity layer interpolated after each convolutional layer. And there is a fully-connected layer at the top of last max-pooling layer. The purple cubes are convolutional kernels while the red ones are pooling kernels.}
	
	%\vspace{-5.5mm}	
	\label{fig:spatnet}
\end{figure}
As shown in Fig. \ref{fig:framework}, there are two spatial networks (yellow and green boxes) in both streams to learn spatial feature representations from raw video frames and temporal features at the spatial location between two consecutive frames. Here, both spatial networks lying in the two streams have the same structure, each of which contains  3 convolutional layers and 3 max-pooling layers with a non-linearity layer (in this paper we use \textit{tanh}) interpolated after each convolutional layer. And there is a fully-connected layer at the top of last max-pooling layer. The details of spatial networks are shown in Fig. \ref{fig:spatnet}, where the purple cubes are convolutional kernels and the red ones are pooling kernels. All the strides in the convolutional layers and pooling layers are set as 2. Note that although two spatial networks have the same structure, they play different roles in two streams. The inputs of the network (yellow boxes in Fig.~\ref{fig:framework}) are raw RGB images when serves as a spatial feature extractor. Otherwise, the inputs are the last predictions (``Pred3'') of motion networks (blue boxes in Fig. \ref{fig:framework}). 

%\subsubsection{Spatial-Temporal Fusion}


\subsection{Spatial Fusion and Motion Context Accumulation}

\subsubsection{Spatial fusion}\label{sp_fuse}

Here we consider different fusion methods (orange boxes in Fig. \ref{fig:framework} ) to fuse the two stream networks. Our intention is to fuse the spatial features and motion context information at the spatial location  such that channel responses at the same pixel position are put in correspondence. Because the structures of spatial networks in two streams are the same, the feature map produced by each layer in each stream has the exact location correlation. So the problem is how to fuse the outputs of corresponding layers of two streams. To motivate our fusion strategy, consider for example discriminating two walking persons. If legs move or arms swing periodically at some spatial location then the motion network can recognize that motion and obtain the motion context information from two consecutive frames, and the spatial network can recognize the location (legs or hands) and their combination so as to discriminate the persons.

This spatial correspondence can be easily obtained when the two networks have the same spatial resolution at the layers to be fused. The  simplest way is overlaying (stacking) layers from one network on the other. However, there is also the issue of establishing the correct correspondence between one channel (or channels) in one network and the corresponding channel (or channels) of the other network. Suppose different channels in the spatial network learning spatial feature representation from one video frame are responsible for different body areas (head, arms, legs, \textit{etc.}), and one channel in the spatial network following the motion network is responsible for contextual motion between two neighboring frames in the fields. Then, after the channels are stacked, the filters in the subsequent layers must learn the correspondence between these appropriate channels  in order to best discriminate between these motions from different person samples.
To make this more concrete, we investigate the following 3 ways of fusing layers between two stream networks. Suppose  $\textbf{x}^{A} \in \mathbb{R}^{H \times W \times D}$ and $\textbf{x}^{B} \in \mathbb{R}^{H \times W \times D}$ are two feature maps from two layers and need to be fused, where \textit{W}, \textit{H} and \textit{D} are the width, height and channel number of the respective feature maps. And $\textbf{y} $ represents the fused feature maps. When applied to the feedforward spatial network architecture which is shown in Fig. \ref{fig:spatnet}, which consists of convolutional, max-pooling non-linearity and fully-connected layers, the fusion can be performed at different points in the network to implement \textit{e.g.} early-fusion or late-fusion. 
\begin{itemize}
	\item \textbf{Concatenation fusion}  This fusion operation stacks the two feature maps at the same spatial locations \textit{i}, \textit{j} across the feature channels \textit{d}:

\begin{equation}\label{concf}
y_{i,j,2d}^{\mathrm{cat}}= x_{i,j,d}^{\mathrm{A}} ,  y_{i,j,2d-1}^{\mathrm{cat}} =  x_{i,j,d}^{\mathrm{B}},
\end{equation}		
where $\textbf{x}^{A}, \textbf{x}^{B}\in\mathbb{R}^{H \times W \times D} , \textbf{y}^{\mathrm{cat}} \in \mathbb{R}^{H \times W \times 2D}$ and $1\leq i \leq H,1\leq j \leq W$.
	\\
	\item \textbf{Sum fusion} The sum fusion computes the sum of the two feature maps at the same spatial locations \textit{i}, \textit{j }and channels \textit{d}:
\begin{equation}\label{sumf}
y_{i,j,d}^{\mathrm{sum}}= x_{i,j,d}^{\mathrm{A}} + x_{i,j,d}^{\mathrm{B}},
\end{equation}		
	where $ \textbf{y}^{\mathrm{cat}} \in \mathbb{R}^{H \times W \times 2D}$. 
	\item \textbf{Max fusion}
	Similarly, max fusion takes the maximum of the two feature maps:
	\begin{equation}\label{convf}
	y_{i,j,d}^{\mathrm{max}}=\mathrm{max} \{ x_{i,j,d}^{\mathrm{A}} , x_{i,j,d}^{\mathrm{B}} \} .
	\end{equation}
\end{itemize}	

Now, we briefly introduce where our fusion method should be applied to. Injecting fusion layers can have significant impact on the number of parameters and layers in a two-stream network, especially if only the network which is fused into is kept and the other network tower is truncated. For example, in Fig. \ref{fig:spatnet}, if the fusion operation is performed on the second max-pooling layers  (``Max-pool2'') of two spatial networks in the two-stream network, then the previous parts (``Conv1, Max-pool1, Conv2'' and non-linearity layers) are kept, and the layers (``Conv3, Max-pool3, Fully connected'') after fusion operation share one set of parameters. The illustration is shown in Fig. \ref{fig:fusion}.
	
In the experimental section (Sec. \ref{ana}), we evaluate and compare the performance of each of the three fusion methods in terms of their re-identification accuracy.
\begin{figure}[htb]
	\centering
	\includegraphics[width=7.5cm]{./fusion.pdf}
	%\vspace{-5.5mm}	
	\caption{The illustration of fusion of two spatial networks at the second max-pooling layer (Max-pool2).}
	
	%\vspace{-5.5mm}	
	\label{fig:fusion}
\end{figure}
\subsubsection{Motion Context Accumulation}\label{mo_acc}
We now consider the techniques to combine fused features  $\textbf{f}^{(t)}$ (output of the fully-connected layer in Fig. \ref{fig:fusion}) containing both spatial appearance and motion context information over time \textit{t}. Because the length of a sequence is arbitrary, the motion context is also unfixed for each person. Therefore, we exploit  Recurrent neural networks (RNN) which can process an arbitrarily long time-series using a neural network to address the problem of accumulating motion context information. Specifically, an RNN has feedback connections allowing it to remember information over time and produces an output based on both the current input and information from the previous time-steps at each time-step.  As shown in Fig. \ref{fig:framework},  the recurrent connections of RNN are ``unrolled"  in time to create a very deep feed-forward network.  Given the unrolled network, the lateral connections serve as ``memory'', allowing information to flow between an arbitrary number of time-steps. 

As video-based person re-identification involves recognizing a person from a sequence, accumulating the motion context information of each frame at each instant would be helpful to improve the performance of re-identification. This motion accumulation can be achieved by using the recurrent connections allowing information to be passed between time-steps. To be more clear,  we aim to better capture person spatial appearance features and motion context information present in the video sequence, and then to accumulate them along the time axis. Specifically, given the \textit{p}-dimension output $\textbf{f}^{(t)} \in \mathbb{R}^{p \times 1}$ of the fused spatial networks, the RNN can be defined as follows:
\begin{align}\label{RNN}
\textbf{o}^{(t)}&= M\textbf{f}^{(t)} + N\textbf{r}^{(t-1)},\\
\textbf{r}^{(t)}&= tanh(\textbf{o}^{(t)}).
\end{align}
Here $\textbf{o}^{(t)} \in \mathbb{R}^{q \times 1}  $ is the $q$-dimensional output of RNN at time-step $t$, and $\textbf{r}^{(t-1)} \in \mathbb{R}^{q \times 1}  $  contains the information on the RNN's state at the previous time-step. The $M\in \mathbb{R}^{q \times p}$ and $N\in \mathbb{R}^{q \times q}$ represent the corresponding parameters for $\textbf{f}^{(t)}$ and $\textbf{r}^{(t-1)}$ respectively, where $q$ is the dimension of the output of the last fully-connected layer in fusion part and $p$ is the dimension of the feature embedding-space. 

Although RNNs are able to accumulate the fused information, they still have some limitations. Specifically, some time-steps may be more dominant in the output of the RNN, which could reduce the RNN’s effectiveness when used to accumulate the input information over a full sequence  because discriminative frames may appear anywhere in the sequence. To overcome the drawback, similar to \cite{mclaughlinrecurrent}, we add a temporal pooling layer after RNN to allow for the aggregation of information across all time steps.  The temporal pooling layer aims to capture long-term information present in the whole sequence, which combines with the motion context information accumulated through RNN. In this paper, we can use either average-pooling  or max-pooling over the temporal dimension to produce a single feature vector. $\textbf{u}_{ave}$ represents the person's spatial appearance and motion information averaged over the whole input sequence while $\textbf{u}_{max}$ denotes the feature vector after max-pooling. Given the temporal pooling layer inputs $\left\lbrace \textbf{o}^{(1)}, \textbf{o}^{(2)}, ...,  \textbf{o}^{(T)} \right\rbrace$, the average-pooling and max-pooling methods are as follows:
\begin{align}
\textbf{u}_{ave} &=\dfrac{1}{T}\sum_{t=1}^{T}\textbf{o}^{(t)},\\
\textbf{u}_{max}^i &=\mathrm{max}\left( \left[ \textbf{o}^{(1),i},\textbf{o}^{(2),i},...,\textbf{o}^{(t),i}\right] \right), 
\end{align}
where $T$ is the length of the sequence or time-steps. And $\textbf{u}_{max}^i$  is the i’th element of the vector vs and $\left[ \textbf{o}^{(1),i},\textbf{o}^{(2),i},...,\textbf{o}^{(t),i}\right]$ are \textit{i}th elements of the feature vector across the temporal dimension.
\subsubsection {Multi-task Loss}
Similar to the method suggested in~\cite{mclaughlinrecurrent}, we train the whole  AMOC network to satisfy both the contrastive objective and to predict the person’s identity. Given the sequence feature vector \textbf{u} including accumulative spatial appearance feature and motion context information, output by the  feature representation learning networks of our AMOC, we can predict the identity of  the person in the sequence using the standard softmax function, which is defined as follows:
\begin{equation}
\textit{I }(\mathbf{u}) = P(z=c|\mathbf{u})=\frac{\exp(S_{c}\mathbf{u})}{\sum_{k}\exp(S_{k} \mathbf{u})},
\end{equation}
where there are a total of \textit{K} identities, \textit{z} is the identity of the person, and \textit{S} is the softmax weight matrix while $S_{c}$ and $S_{K}$ represent the \textit{c}th and \textit{k}th column of it, respectively. Then the corresponding softmax loss function (pink boxes in Fig. \ref{fig:framework}) is defined as follows:
\begin{equation}
L_{class} = -\textup{log}(P(z=c|\mathbf{u})).
\end{equation}

Besides, given a pair of sequence feature vectors $(\mathbf{u}_{(a)}, \mathbf{u}_{(b)})$  output by the Siamese network,  the contrastive loss (purple box in Fig.\ref{fig:framework}) function can be defined as
\begin{align}\label{con_loss}
L_{con} = \left \| \mathbf{u}_{(a)} - \mathbf{u}^+_{(b)}  \right \|_{2}^{2}+\mathrm{max}\left\{0, \alpha-  \left \| \mathbf{u}_{(a)} - \mathbf{u}^-_{(b)}  \right \|_{2}^{2} \right\},
\end{align}
where $\mathbf{u}^+_{(b)}$ represents the positive pair of $\mathbf{u}_{(a)}$ while $\mathbf{u}^-_{(b)}$ represents the negative pair of $\mathbf{u}_{(a)}$. The loss consists of two penalties: the first term penalizes a positive pair $(\mathbf{u}_{(a)}, \mathbf{u}^+_{(b)})$ that is too far apart, and the second penalizes a negative pair $(\mathbf{u}_{(a)}, \mathbf{u}^-_{(b)})$ that is closer than a margin $\alpha$. If a negative pair is already separated by $\alpha$, then there is no penalty for this pair and $L_{con}(\mathbf{u}_{(a)},\mathbf{u}^-_{(b)}) = 0$.

Finally, we jointly end-to-end train our architecture with both classification loss and contrastive loss.  We can now define the overall multi-task training loss function  $\textit{L}_{multi}$ for a single pair of person sequences, which jointly optimizes the classification cost and the contrastive cost as follows:
\begin{align}
L_{multi}(\mathbf{u}_{(a)},\mathbf{u}_{(b)}) = &L_{con}(\mathbf{u}_{(a)},\mathbf{u}_{(b)})\notag\\
&+L_{class}(\mathbf{u}_{(a)})+L_{class}(\mathbf{u}_{(b)}).
\end{align}
Here, we give equal weights for the classification cost and contrastive cost terms. The above network can be trained end-to-end using back-propagation-through-time from raw video frames (details of our training parameters can be found in Sec. \ref{impl_details}). During the training phase, all recurrent connections are unrolled to create a deep feed-forward graph, where the weights of AMOC are shared between all time-steps. And in the test phase,  we discard the multi-task loss functions and apply the AMOC on the raw video sequences as a feature extractor, where the feature vectors extracted by it can be directly compared using Euclidean distance.


\subsection{Implementation Details}\label{impl_details}
\subsubsection{Pre-training of the Motion Networks}\label{pre_motion}
As aforementioned in the Sec.~\ref{motion_net}, we use the pre-extracted optical flow as the ground truth to pre-train the motion networks. Our aim is that our proposed architecture can directly learn motion context information from raw consecutive video frames. When the pre-training of motion networks is finished, we use the trained parameters to initialize the motion networks in the whole framework. 

As shown in the Fig.~\ref{fig:flownet}, our motion networks produce three motion maps of optical flow maps at three levels of scale (``Pred1'', ``Pred2'', ``Pred3'') for a pair of person frames. Because the size of each prediction map is 1/8, 1/4 and 1/2 of the input frame size respectively, the pre-extracted optical flow maps are all downsampled to the corresponding sizes, then serve as the ground truth. In this paper, we use the smooth-$L_{1}$  loss function~\cite{girshick2015fast} to compute the losses between the predictions $\textbf{e}^{(l)} \in\mathbb{R}^{h^{(l)}\times w^{(l)} \times 2} $ and ground truth optical flow maps $\textbf{g}^{(l)} \in\mathbb{R}^{h^{(l)}\times w^{(l)} \times 2} $, where $l = 1, 2 ,3$. The loss function is defined as

\begin{align}
L^{(l)}_{(motion)}(\textbf{e}^{(l)}, \textbf{g}^{(l)})&=\sum_{i,j,k}\textup{smooth}_{L_{1}}(e_{i,j,k}^{(l)}- g_{i,j,k}^{(l)}),\\
\textup{smooth}_{L_{1}}(\theta)&=\left\{\begin{matrix}
&0.5\theta^{2}   &\textup{if}\left | \theta \right |<1\\ 
&\left |\theta \right |-0.5&\textup{otherwise}.
\end{matrix}\right.
\end{align} 
Then, the overall cost function can be written as
\begin{equation}
L_{(motion\_all)}=\sum_{l=1}^{3}\omega_{l}L^{(l)}_{(motion)},
\end{equation}
where $\omega_{l}$ represents the weight of each loss between different scale level prediction and ground truth map. In the pre-training phase of the motion network, we set them to (0.01, 0.02, 0.08) respectively. 

 All the input pairs of video frames are resized to $128 \times 64$ and Adam\cite{kingma2014adam} is chosen as the optimization method due to its faster convergence than standard stochastic gradient descent with momentum for our task. As recommended in \cite{kingma2014adam}, we fix the parameters of Adam : $ \beta_{1}= 0.9$ and $\beta_{2} = 0.999$. Considering every pixel is a training sample, we use small mini-batches of 4 frame pairs. We start with learning rate $\lambda= 1e-4$ and then divide it by 2 every 10k iterations after the first 20k. Note, the motion network of our model has a good generalization ability, which is verified in the following Section (Sec. \ref{ana}). Therefore, in the experiments on all three datasets, we use the motion network only pre-trained on the iLIDS-VID to initialize the one in the whole end-to-end AMOC framework.
 
\subsubsection{Training of the Overall Architecture}
After pre-training the motion network, we use it to initialize the one in our AMOC to end-to-end train the whole network. In the end-to-end training process, we set margin $\alpha$ to 2  in the Eqn. \eqref{con_loss}, and the embedding-space dimension is set to 128. For the temporal pooling, we adopt the average-pooling method in the related experiments if not specified. Besides, the learning rate is set to $1e-3$. Note, as we mentioned in  Sec. \ref{motion_net}, the resolution of output final predicted flow map of the motion network (``Pred3'')  is as half as that of input. Therefore, at each time-step, we resize the first frame of a pair to two scales of $64 \times 32$ and $128 \times 64$. For the second frame, we resize it to $128 \times 64$. That is to say, at each time-step, the first-stream spatial network (yellow boxes in Fig.\ref{fig:framework}) takes the first frame within a pair with size of $64 \times 32$ to learn spatial appearance feature representations. And the second-stream networks, including spatial network and motion network (green and blue boxes in Fig.\ref{fig:framework}) take the pair frames with sizes of $128 \times 64$ as input to extract motion context information. This operation can be performed in an on-line way. Furthermore, it guarantees that the feature map generated by each layer of two spatial networks in two streams has the same resolution thus can be fused at an arbitrary layer. 
\subsubsection{Data Augmentation}
To increase diversity of the training sequences to overcome the data imbalance and overfitting issue, data augmentation is applied on all the datasets. Specifically, we artificially augment the data by performing random 2D translation, similar to the processing in \cite{mclaughlinrecurrent}. For all frame images of size $w \times h$ in a sequence, we sample same-sized frame images around the image center, with translation drawn from a uniform distribution in the range $[\text{-0.05}w, \text{0.05}w] \times [\text{-0.05}h, \text{0.05}h]$. Besides, the horizontal flip is also performed to augment the data. At each epoch of training phase, the augmentation is applied once. During the testing phase, data augmentation is also applied, and the similarity scores between sequences are averaged over all the augmented data.
\section{Experiments}
\subsection{Datasets }
 In this paper, we use iLIDS-VID\cite{wang2014person}, PRID-2011\cite{hirzer2011person} and MARS\cite{zhu2016video}, which are three public benchmarks available, to evaluate our proposed AMOC model. 
 
 \noindent\textbf{iLIDS-VID}:$\quad$The iLIDS-VID dataset is created from the pedestrians captured in two non-overlapping camera views at an airport arrival hall under a multi-camera CCTV network. It is very challenging due to large clothing similarities among people, lighting and viewpoint variations across camera views, cluttered background and random occlusions. There are 600 image sequences of 300 distinct individuals in the dataset, with one pair of image sequences from two camera views for each person. The image sequences with an average number of 73 range in length from 23 to 192 image frames. 
 
 \noindent\textbf{PRID-2011}:$\quad$The PRID-2011 dataset  totally contains 749 persons captured by two non-overlapping cameras. Among,  there are 400 image sequences for 200 people from two camera views that are adjacent to each other, with sequences lengths of 5 to 675 frames. Compared with the iLIDS-VID dataset, it is less challenging due to being captured in non-crowded outdoor scenes with relatively simple and clean backgrounds and rare occlusions. Similar to the protocol used in \cite{mclaughlinrecurrent}, we only use the first 200 persons appearing in both cameras for evaluation.
 
 
 \noindent\textbf{MARS}:$\quad$ The MARS containing 1,261 IDs and around 20,000 video sequences is the largest video re-id  benchmark dataset to date.  Each sequence is automatically obtained by the Deformable Part Model (Deformable Part Model)\cite{Felzenszwalb2014Object} detector and the GMMCP\cite{Dehghan2015GMMCP} tracker. These sequences are captured by six cameras at most and two cameras at least, and each identity has 13.2 sequences on average. Additionally, there are also 3,248 distractor sequences are contained in the dataset. The dataset is fixedly split into training and test sets, with 631 and 630 identities, respectively. In testing, 2,009 probes are selected for query.
 \begin{table*}[htbp]
 	\linespread{1.5}\selectfont
 	\centering
 	\caption{Rank1, Rank5, Rank10 and Rank20 recognition rate (in \%) of various  methods on iLIDS-VID and PRID-2011 datasets. }
 	\begin{tabular}{c|cccc|cccc}
 		\hline
 		\textbf{Dataset} &\multicolumn{4}{c|}{\textbf{iLIDS-VID}}&\multicolumn{4}{c}{\textbf{PRID-2011}}\\
 		\hline
 		\textbf{Methods} & \textbf{Rank1} & \textbf{Rank5} & \textbf{Rank10} & \textbf{Rank20} & \textbf{Rank1} & \textbf{Rank5} & \textbf{Rank10} & \textbf{Rank20}\\
 		\hline
 		Baseline + LK-Flow\cite{mclaughlinrecurrent} & 58.0 & 84.0 & 91.0 & 96.0 & 70.0 & 90.0 & 95.0 & 97.0\\
 		Baseline + EpicFlow\cite{revaud2015epicflow} & 59.3 & 87.2 & 92.7 & 98.2 &76.2 & 97.5&98.2 &99.0\\ 
 		AMOC w/o Motion		                                    & 54.2 & 78.3 & 89.1 & 95.8 &65.4 & 88.9 &95.6 &98.5\\ 
 		AMOC + LK-Flow & 63.3 & 85.3 & 95.1 & 96.4 &76.0 & 96.5&97.4 &99.6 \\
 		AMOC + EpicFlow & 65.5& 93.1& 97.2 & 98.7 &82.0 &97.3 &99.3 &99.4\\
 		end-to-end AMOC + LK-Flow & 65.3 & 87.3 & 96.1 & 98.4 &78.0 &97.2 &99.1 &99.7 \\
 		\hline
 		\textbf{end-to-end AMOC + EpicFlow} & \textbf{68.7} & \textbf{94.3} & \textbf{98.3} & \textbf{99.3} & \textbf{83.7} & \textbf{98.3} & \textbf{99.4} & \textbf{100} \\
 		\hline
 	\end{tabular}%
 	
 	\label{tab:lk_epic}%
 \end{table*}%
 \subsection{Experimental Settings and Evaluation Protocol}\label{exp_setting}
 For the experiments performed on iLIDS-VID and PRID-2011 datasets,  half of persons are extracted for training and the other half for testing. All experiments are conducted 10 times with different training/testing splits and the averaged results ensure their stability. For the MARS dataset, we use the provided fixed training and test set, containing 631 and 630 identities respectively, to conduct experiments. Additionally, the training of networks in our architecture, including pre-training of motion networks and end-to-end training of the whole framework, are all implemented by using the Torch\cite{torch} framework on NVIDIA GeForce GTX TITAN X GPU.
 
 As we described in Sec. \ref{arch}, our network is a Siamese-like network, so the positive  and negative sequence pairs are randomly on-line selected during the training phase. However, positive and negative sequence pairs consist of two full sequences of an arbitrary length containing the same person or different persons under different cameras respectively. To guarantee the fairness of experiments, we follow the same sequence length setting in \cite{mclaughlinrecurrent}. Considering the efficiency of  training,  a sub-sequence containing 16 consecutive frames is sampled from the full length sequence of a person during training. We train our model for 1000 epochs on the iLIDS-VID and PRID-2011 datasets. On the MARS, the epoch number is set to 2000. At each epoch, the random selection is performed once. During testing, the sequences under the first camera and second camera are regarded as the probe and  the gallery respectively, as in \cite{mclaughlinrecurrent} and \cite{wang2014person}. And the length of each person sequence is set to 128 for testing. If the length of a full sequence of  a person is smaller than 128, we use the full-length sequence of this person to test. For the iLIDS-VID and PRID-2011 datasets, the training takes approximately 4-5 hours. And the training on the MARS consumes around 1 day. In the test phase, it takes roughly 5 minutes for one split testing on the iLIDS-VID and PRID-2011 datasets. On the MARS with larger scale, the testing time is increased to about 15 minutes.
 
 In experiments, for each pedestrian, the matching of his or her probe sequence (captured by one camera) with the gallery sequence (captured by another camera) is ranked. To reflect the statistics of the ranks of true matches, the Cumulative Match Characteristic (CMC) curve is adopted as the evaluation metric.  Specifically, in the testing phase, the Euclidean distances between probe sequence features and those of gallery sequences are computed firstly. Then, for each probe person, a rank order of all the persons in the gallery is sorted from the one with the smallest distance to the biggest distance. Finally, the percentage of true matches founded among the first $m$ ranked persons is computed and denoted as rank$m$. In addition, for the MARS dataset, the mean average precision (mAP) as in \cite{zheng2016mars} is also employed to evaluate the performance since each query has multiple cross-camera ground truth matches. Note, all the experiments performed on three datasets are under single-query setting. 
 
 
\subsection{Analysis of the Proposed AMOC Model}\label{ana}
 
Before showing the comparison of our method with the state-of-the-arts, we conduct several analytic experiments on iLIDS-VID and PRID-2011 datasets to verify the effectiveness of our model for solving the video-based person re-identification problem. We analyse  and investigate the effect of several factors upon the performance, which include the  generation of motion context information, the selection of spatial fusion method, the location of performing spatial fusion, the sequence lengths of probe and gallery for testing, the temporal pooling method and other parameter settings, such as embedding size of RNN and margin $\alpha$ in Eqn. ~\eqref{con_loss}. In this paper, we regard the method in \cite{mclaughlinrecurrent} as our baseline, in which  the spatial-temporal features are also employed in a recurrent way but without two-stream structure or spatial fusion method, and all the optical flow maps serving as motion information are pre-extracted in off-line way. 
 \subsubsection{Effect of Different Motion Information }
 As described in Sec. \ref{motion_net}, our motion network  is able to end-to-end learn motion information from video sequence frames. Thus except for the Lucas-Kanade optical flow (LK-Flow) algorithm~\cite{lucas1981iterative} used in \cite{mclaughlinrecurrent}, we  also exploit the optical flow maps produced by EpicFlow\cite{revaud2015epicflow} algorithm to investigate the effect of different motion information. The experimental results are shown in Tab. \ref{tab:lk_epic}. The AMOC networks in ``AMOC + LK-Flow'' and ``AMOC + EpicFlow'' methods are the non-end-to-end versions, the motion networks of which are replaced by the corresponding optical flow maps pre-extracted. Note that here we only study the effect of different optical flows carrying motion context information and verify the effectiveness of end-to-end learning. Therefore, all the shown results of our AMOC are achieved by using the ``concatenation fusion'' method introduced in the Sec. \ref{sp_fuse} and fusing two-stream networks at ``Max-pool2'' layer as illustrated in Fig.~\ref{fig:fusion}. In Tab.~\ref{tab:lk_epic}, ``Baseline + LK-FLOW'' is the method from \cite{mclaughlinrecurrent}, and we observe that the performance is boosted when the LK-Flow is replaced by the EpicFlow to produce optical flow in either baseline methods or our AMOC method. As introduced in \cite{revaud2015epicflow}, EpicFlow assumes that contours often coincide with motion discontinuities and computes a dense correspondence field by performing a sparse-to-dense interpolation from an initial sparse set of matches, leveraging contour cues using an edge-aware distance.  So it is more robust to motion boundaries, occlusions and large displacements than the LK-Flow algorithm\cite{lucas1981iterative}, which is beneficial to the extraction of motion information between video frames. In addition, we can see that the performance of our ``end-to-end AMOC'' using different optical flow generation methods is both improved. 
 
 \begin{figure*}[htb]
 	\centering
 	\includegraphics[width=14cm]{./flow.pdf}
 	\vspace{-3.5mm}	
 	\caption{The visualization of output optical flow of the proposed motion networks performed on iLIDIS-VID and PRID-2011 datasets. The first and forth rows are the consecutive raw video frames from iLIDS-VID and PRID-2011 datasets while the optical flow maps computed using EpicFlow\cite{revaud2015epicflow} for the two datasets are shown in the second and fifth rows. The third row and sixth row are the output flow maps of our motion networks. All the produced optical flow maps are encoded exploiting the flow colour coding\cite{baker2011database} method. Different colours represent different directions of motions, shades of which indicate the speeds of motions. }
 	
 	%\vspace{-5.5mm}	
 	\label{fig:flow_demo}
 \end{figure*}
 
 
 \begin{figure}[htb]
 	\begin{center}
 		\begin{tabular}{cc}
 			{\hspace{-3pt}}
 			\includegraphics[height=0.39\linewidth, width=0.46\linewidth]{./dif_len.pdf}	&
 				\includegraphics[height=0.39\linewidth, width=0.46\linewidth]{./dif_pooling.pdf}\\
 			{\footnotesize{(a) Effect of sequence length }} &
 		{\footnotesize{(b) Effect of pooling methods}} \\
 			\includegraphics[height=0.39\linewidth,width=0.46\linewidth]{./diff_margin.pdf}&
 			\includegraphics[height=0.39\linewidth,width=0.46\linewidth]{./dif_emsize.pdf}
 			\\
 		
 				{\footnotesize{(c) Effect of margin}}&
 				 {\footnotesize{(d) Effect of embedding size}}
 			
 			%{\vspace{0.5em}}
 		\end{tabular}
 	\end{center}
 	\caption{The performance of our proposed end-to-end AMOC using different parameter settings on iLIDS-VID dataset. (a) shows comparisons of our proposed end-to-end AMOC with baseline method\cite{mclaughlinrecurrent} using different query and gallery sequence lengths, ranging from 1 to 128, for testing. (b) shows the CMC curves of our method using two different temporal pooling methods (\textit{i.e}.,  average-pooling and max-pooling). (c) gives our method's performance variation depending on different margins in contrastive loss (Eqn.~\eqref{con_loss}). (d) illustrates the effect of different RNN embedding sizes upon the performance of our method.}
 	
 	\label{fig:ana_fig}
 	\vspace{-1em}
 \end{figure}
 
 Moreover, we also visualize the optical flow maps produced by our motion networks in Fig.~\ref{fig:flow_demo}.  The first and forth rows are the consecutive raw video frames from iLIDS-VID and PRID-2011 datasets while the optical flow maps computed using \cite{revaud2015epicflow} for the two datasets are shown in the second and fifth rows. The third row and sixth row are the output flow maps of our motion networks. All the produced optical flow maps are encoded with the flow color coding~\cite{baker2011database} method which is also used  in \cite{revaud2015epicflow}. Different colors represent different directions of motions, and shapes indicate the speeds of motions. The faster the motion is, the darker its color will be.  In this experiment, we only use the sequences of half persons of iLIDS-VID for training, and perform the motion networks on the left half persons' sequences of iLIDS-VID and the whole PRID-2011 dataset. In other words, for the PRID-2011 dataset, there is no need to re-train the motion networks on it. From the results shown in Fig.~\ref{fig:flow_demo}, we can see that our motion networks can well approximate the optical flow maps produced by EpicFlow~\cite{revaud2015epicflow} and successfully capture the motion details of persons, such as the speed and amplitude of legs moving. Especially for the PRID-2011 dataset, our motion network can achieve the good optical flow estimation without using training data from PRID-2011, which means our motion network has a good generalization ability. 
 
 To further illustrate the effectiveness of motion information embedded in the AMOC architecture, we remove the temporal stream (corresponding to the green and blue boxes in Fig. ~\ref{fig:framework}) of AMOC to perform the analytic experiments. Namely, our AMOC degenerates to the single-stream network which only uses appearance features of persons and there is no motion information included and accumulated. The results are reported as ``AMOC w/o Motion'' in Tab. \ref{tab:lk_epic}. Compared with other methods using motion information in Tab. \ref{tab:lk_epic}, the performance of ``AMOC w/o Motion'' is the worst for both datasets. This suggests that the motion information is beneficial for improving the person re-identification performance in video sequence.
  
   
 
 




\subsubsection{Effect of Spatial Fusion Method and Location}
In this part, we investigate how and where to fuse two-stream networks in our end-to-end AMOC network. For these experiments, we use the same spatial network architecture introduced in Sec.~\ref{spatial}. The fusion layer can be injected at any location, such as after ``Max-pool 2'', \textit{i.e.} its input is the output of ``Max-pool 2'' from the two streams. After the fusion layer a single processing stream is used. 

We compare different fusion strategies in Tab.~\ref{tab:fusion},  where we report the rank1, rank5, rank10 and rank20 recognition rate on both iLIDS-VID and PRID-2011 datasets. From the results, we see that ``Concatenation'' fusion method performs considerably higher than ``Sum'' and ``Max'' fusion methods. Compared to the fusion methods, our end-to-end AMOC network shows more sensitiveness to the location of spatial fusion. Specifically, for all three fusion methods, our method can achieve the best performance when the spatial fusion is performed after the ``Max-pool2'' layer. And fusion at FC (Fully-Connected) layers results in an evident drop in the performance. The reason for FC performing worse may be that at this layer spatial correspondences between spatial appearance and motion context information would be collapsed.  


\begin{table*}[htbp]
	\linespread{1.3}\selectfont
	\centering
	\caption{Rank1, Rank5, Rank10 and Rank20 recognition rate (in \%) of various fusion methods on iLIDS-VID and PRID-2011 datasets. }
	\begin{tabular}{c|c|cccc|cccc}
		\hline
		\multirow{2}{*}{\textbf{Fusion Method}} &\textbf{Dataset} &\multicolumn{4}{c|}{\textbf{iLIDS-VID}}&\multicolumn{4}{c}{\textbf{PRID-2011}}\\
		\cline{2-10}
		&\textbf{Fusion Layer} & \textbf{Rank1} & \textbf{Rank5} & \textbf{Rank10} & \textbf{Rank20} & \textbf{Rank1} & \textbf{Rank5} & \textbf{Rank10} & \textbf{Rank20}\\
		\hline
		
		\multirow{7}{*}{Sum} &{Tanh1} & 60.8 & 90.1 & 91.7 & 96.5 &71.5 & 93.4&97.3 &98.1\\
		%\cline{2-10}
		&{Max-pool1} & 61.2 & 89.8 & 90.3 & 95.1 &72.6 & 92.8&96.2 &96.9\\
		&{Tanh2} & 65.5 & 91.8 & 95.6 & 96.5 &79.9 & 95.8&97.6 &97.8\\
		&{Max-pool2} & 67.8 & 93.4 & 96.5 & 98.3 & 80.6 & 96.6 & 97.8 & 99.2 \\
		&{Tanh3} & 63.9 & 91.8 & 96.7 & 97.7 &78.3 & 94.5&97.8 &98.2\\
		&{Max-pool3} & 65.0 & 92.8 & 97.9 & 98.4 &78.8 & 94.9&98.0 &99.1\\
		&{FC} & 60.7 & 88.1 & 93.2 & 94.3 &72.0 & 91.2&93.8 &94.9\\
		\hline
		
		\multirow{7}{*}{Max}&{Tanh1} & 60.6 & 90.0 & 91.9 & 97.2 &73.1 & 94.9 &97.2 &99.5\\
		%\cline{2-10}
		&{Max-pool1} & 61.0 & 91.2 & 93.1 & 97.8 &75.1 & 94.9&99.0 &99.5\\
		&{Tanh2} & 66.9 & 94.2 & 97.1 & 98.9 &81.2 & 97.3&98.5 &99.3\\
		&{Max-pool2} & 68.2 & 95.4 & 97.5 & 98.9 & 81.6 & 98.6 & 98.8 & 99.3 \\
		&{Tanh3} & 64.0 & 91.5& 96.4 & 97.3 &78.1 & 94.1&97.3 &98.2\\
		&{Max-pool3} & 64.3 & 93.1 & 98.2 & 98.3 &78.4 & 94.5&97.8 &99.0\\
		&{FC} & 61.2 & 89.1 & 93.6 & 94.9 &73.5 & 90.2&92.9 &94.7\\
		
		\hline
		\multirow{7}{*}{\textbf{Concatenation}} &{Tanh1} & 64.1 & 92.3 & 94.2 & 98.0 &74.8 & 95.5&98.4 &99.2\\
		%\cline{2-10}
		&{Max-pool1} & 64.8 & 92.0 & 94.1 & 98.3 &75.3 & 95.8&97.6 &99.2\\
		&{Tanh2} & \textbf{69.2} & 93.8 & 97.9 & 98.7 &83.0 & 97.9&99.0 &99.6\\
		&\textbf{Max-pool2} & 68.7 & \textbf{94.3} & \textbf{98.3} & \textbf{99.3} & \textbf{83.7} & \textbf{98.3} & \textbf{99.4} & \textbf{100} \\
		&{Tanh3} & 65.2 & 92.3 & 97.1 & 98.4 &80.0 & 96.3&99.4 &99.6\\
		&{Max-pool3} & 66.1 & 92.8 & 97.9 & 98.4 &80.0 & 96.9&99.8 &99.8\\
		&{FC} & 62.3 & 88.3 & 93.9 & 95.6 &73.2 & 91.3&94.4 &96.7\\
		\hline
	\end{tabular}%
	
	\label{tab:fusion}%
\end{table*}%


\subsubsection{Effect of  Sequence Length and Temporal Pooling Methods}\label{lenandtp}
To further study how re-identification accuracy varies depending on the lengths of the probe and gallery sequences during the test phase, we perform the experiments on the iLIDS-VID dataset. The testing lengths of the probe and gallery sequences are set to the same number and  simultaneously increased from 1 to 128, in steps corresponding with the powers-of-two. Training lengths are set to 16 as indicated in the Sec.~\ref{exp_setting} .

The results shown in Fig.~\ref{fig:ana_fig} (a) prove that increasing both the probe and gallery sequence lengths can bring the improvement of performance. This is reasonable because more appearance and motion information would be exploited if more samples for each person are available. The similar phenomenon is also verified by the baseline method\cite{mclaughlinrecurrent}. Comparing to it, our method can achieve higher rank1 recognition rates by significant margins under different sequence length settings.

As aforementioned in the Sec.~\ref{mo_acc},  the AMOC accumulates the RNN output spatial-temporal features into a single feature vector by apply either average-pooling or max-pooling over the temporal dimension. Now we compare the performance of the two pooling methods. Results are summarized in 
Fig.~\ref{fig:ana_fig} (b). We observe that average-pooling is superior than the max-pooling. One possible reason is that average-pooling consider all the time steps equally important in the decision, whilst  max-pooling only employ the feature value in the temporal step with the largest activation. Therefore, average-pooling over the temporal sequence of features can produce a more robust feature vector to compress and represent the person's appearance and motion information over a period of time. 


%\begin{table}[htbp]
%	\centering
%	\caption{iLIDS-VID CMC re-identification accuracy as the lengths of the probe and gallery sequences are increased simultaneously}
%	\begin{tabular}{c|cccc|cccc}
%	\hline
%	\textbf{Dataset} &\multicolumn{4}{c|}{\textbf{iLIDS-VID}}&\multicolumn{4}{c}{\textbf{PRID-2011}}\\
%		\textbf{Length} & \textbf{Rank1} & \textbf{Rank5} & \textbf{Rank10} & \textbf{Rank20} \\& \textbf{Rank1} & \textbf{Rank5} & \textbf{Rank10} & \textbf{Rank20}\\
%		\hline
%		1 & 25.0    & 51.7  & 69.1 & 82.2 \\
%	2  & 27.9 & 53.2  & 71.5 & 82.3 \\
%	4  & 34.1  & 57.3 & 75.1 & 89.1 \\
%	8  & 41.3  & 68.1 & 78.3 & 89.1 \\
%	16  & 50.4  & 77.6 & 88.5 & 95.3 \\
%	32  & 60.8  & 89.2 & 93.4 & 97.4 \\
%	64  & 65.2  & 90.1 & 95.2 & 98.2 \\
%	128  & 68.7 & 94.3 & 98.3 &99.3 \\
%		\hline
%	\end{tabular}%
%	\label{tab:dif_len}%
%\end{table}%




\subsubsection{Effect of Other Parameter Settings}
In this subsection, we conduct experimental analysis on iLIDS-VID to investigate the effect of other parameter settings on our proposed AMOC: the margin $\alpha$ in Eqn.~\eqref{con_loss} and embedding size of RNN. 

In this paper, we adopt multi-task loss including contrastive loss and classification loss. The margin $\alpha$, in the contrastive loss function, can be set empirically, since the AMOC model can learn to adaptively scale the feature vectors proportional to $\alpha$. But the choice of the margin is still crucial to the performance. Therefore, we investigate a few different margins ranging from 1 to 10. The rank1 recognition rate is used for evaluation and the result curve is shown in Fig.~\ref{fig:ana_fig} (c). We observe that the rank1 recognition rate stays stable when we set margin smaller than 5. And our model can achieve the best rank1 performance (68.7\%) when the margin is set to 2. If we further increase the margin from 5 to 10, the performance would drop considerably. Therefore, 2 is the best choice of margin in our model. 

Additionally, the effect of different embedding sizes (\{64, 128, 256, 512, 1024\}) of RNN on the rank1 performance of our method  is also illustrated in the Fig.~\ref{fig:ana_fig} (d). We find our method can obtain the highest rank1 recognition rate under 128 embedding size setting. When the embedding size is reduced to 64, the rank1 drops slightly. This is probably because of information loss from the reduction of parameters. When we use 256 embedding size, the slight drop can also be observed. Moreover, the performance would be constantly undermined if the embedding size is further increased to 1024, mainly due to over-fitting. Besides, increasing the embedding size can also bring longer training time. So we choose the 128 embedding size of RNN as it gives the best trade-off between performance and computational cost.   


 \subsection{Comparison with State-of-the-Art Methods}
 We further evaluate the performance of end-to-end AMOC, by comparing it with the state-of-the-art methods on iILIDS-VID, PRID-2011 and MARS datasets. The methods includes STA\cite{liu2015spatio}, DVR\cite{wang2016person}, TDL\cite{you2016top}, SI$^{2}$DL\cite{zhu2016video}, PaMM\cite{cho2016improving}, mvRMLLC+ST-Alignment\cite{chen2015person}, TAPR\cite{gao2016temporally}, SRID\cite{karanam2015sparse}, AFDA\cite{Li2015Multi}, DVDL~\cite{Karanam2015Person}, HOG3D\cite{Kl2008A}, KISSME\cite{K2012Large}, GEI\cite{Han2005Individual}, HistLBP\cite{xiong2014person} XQDA\cite{liao2015person}, LOMO\cite{liao2015person}, BoW\cite{zheng2015scalable}, gBiCov\cite{ma2014covariance}, IDE\cite{zheng2016person} and RFA-Net\cite{yan2016person}. Note, as discussed in Sec.~\ref{ana}, the motion network of our model has a good generalization ability. Therefore, for all the experiments of our end-to-end AMOC performed on all three datasets, we only use the  iLIDS-VID dataset to pre-train the motion network. 
 
 \begin{table*}
 	\linespread{1.5}\selectfont
 	\centering
 	\caption{Comparison of our end-to-end AMOC's performance on iLIDS-VID and PRID-2011 datasets to the state-of-the-arts.  }
 	\begin{tabular}{c|cccc|cccc}
 		\hline
 		\textbf{Dataset} &\multicolumn{4}{c|}{\textbf{iLIDS-VID}}&\multicolumn{4}{c}{\textbf{PRID-2011}}\\
 		\hline
 		\textbf{Methods} & \textbf{Rank1} & \textbf{Rank5} & \textbf{Rank10} & \textbf{Rank20} & \textbf{Rank1} & \textbf{Rank5} & \textbf{Rank10} & \textbf{Rank20}\\
 		\hline
 		Baseline \cite{mclaughlinrecurrent} & 58.0 & 84.0 & 91.0 & 96.0 & 70.0 & 90.0 & 95.0 & 97.0\\
 		STA\cite{liu2015spatio} & 44.3 & 71.7 & 83.7 & 91.7 &64.1 & 87.3&89.9 &92.0\\
 		DVR\cite{wang2016person} &39.5&61.1&71.7&81.8&40.0&71.7&84.5&92.2\\
 		TDL\cite{you2016top} &56.3&87.6&95.6&98.3&56.7&80.0&87.6&93.6\\
 		$\mathrm{SI^{2}DL}$\cite{zhu2016video}&48.7&81.1&89.2&97.3&76.7&95.6&96.7&98.9\\
 		PaMM\cite{cho2016improving} &30.3&56.3&70.3&82.7&56.5&85.7&96.3&97.0\\
 		mvRMLLC+ST-Alignment\cite{chen2015person}&\textbf{69.1}&89.9&96.4&98.5&66.8&91.3&96.2&98.8\\
 		TAPR\cite{gao2016temporally}&55.0&87.5&93.8&97.2&73.9&94.6&94.7&98.9\\
 		SRID\cite{karanam2015sparse}&24.9&44.5&55.6&66.2&35.1&59.4&69.8&79.7\\
 		AFDA\cite{Li2015Multi}   &37.5&62.7&73.0&81.8&43.0&72.7&84.6&91.9\\
 		DVDL\cite{Karanam2015Person} &25.9&48.2&57.3&68.9&40.6&69.7&77.8&85.6\\
 		RFA-Net\cite{yan2016person}&49.3&76.8&85.3&90.0&58.2&85.8&93.4&97.9\\
 		\hline
 		AMOC + EpicFlow & 65.5& 93.1& 97.2 & 98.7 &82.0 &97.3 &99.3 &99.4\\
 		\textbf{end-to-end AMOC + EpicFlow} & 68.7& \textbf{94.3} & \textbf{98.3} & \textbf{99.3} & \textbf{83.7} & \textbf{98.3} & \textbf{99.4} & \textbf{100} \\
 		\hline
 	\end{tabular}%
 	\label{tab:comp}%
 \end{table*}%
 
 \subsubsection{Results on iLIDS-VID and PRID-2011}\label{res_ilids}
 
 Comparing the CMC results shown in Tab.~\ref{tab:comp}, we can see that the non-end-to-end version of our AMOC can achieve higher performance than all the compared methods for both iLIDS-VID and PRID-2011 datasets. When the end-to-end AMOC is applied, the performance is further boosted, especially for the Rank1 protocol. The improvements are 3.2\% and 1.7\% for iLIDS-VID and PRID-2011 datasets respectively. Moreover, to our best knowledge, we are the first to introduce a two-stream network structure and end-to-end learning motion information from raw frame pairs  to solve the video-based person re-identification problem. Compared to those methods also using spatial-temporal features, such as STA~\cite{liu2015spatio}, RFA-Net\cite{yan2016person}  and TAPR\cite{gao2016temporally},  our AMOC can achieve performance improvement by a large margin for both datasets, due to the two-stream structure of our AMOC which separately deals with the spatial appearance and motion information from context and then performs feature integration through spatial fusion. From the results, we notice that the second best method, ``mvRMLLC+ST-Alignment\cite{chen2015person}'', can also achieve good performance on iLIDS-VID. However, this method needs the complex pre-processing of person video frames while our end-to-end AMOC can directly learn representations from raw video frames. Note, in Tab. \ref{tab:comp}, ``end-to-end AMOC + EpicFlow'' means our end-to-end AMOC uses the motion networks pre-trained by using EpicFlow optical flow maps as the supervision.  
 
 
% An example of a double column floating figure using two subfigures.
% (The subfig.sty package must be loaded for this to work.)
% The subfigure \label commands are set within each subfloat command,
% and the \label for the overall figure must come after \caption.
% \hfil is used as a separator to get equal spacing.
% Watch out that the combined width of all the subfigures on a 
% line do not exceed the text width or a line break will occur.
%
%\begin{figure*}[!t]
%\centering
%\subfloat[Case I]{\includegraphics[width=2.5in]{box}%
%\label{fig_first_case}}
%\hfil
%\subfloat[Case II]{\includegraphics[width=2.5in]{box}%
%\label{fig_second_case}}
%\caption{Simulation results for the network.}
%\label{fig_sim}
%\end{figure*}
%
% Note that often IEEE papers with subfigures do not employ subfigure
% captions (using the optional argument to \subfloat[]), but instead will
% reference/describe all of them (a), (b), etc., within the main caption.
% Be aware that for subfig.sty to generate the (a), (b), etc., subfigure
% labels, the optional argument to \subfloat must be present. If a
% subcaption is not desired, just leave its contents blank,
% e.g., \subfloat[].


% An example of a floating table. Note that, for IEEE style tables, the
% \caption command should come BEFORE the table and, given that table
% captions serve much like titles, are usually capitalized except for words
% such as a, an, and, as, at, but, by, for, in, nor, of, on, or, the, to
% and up, which are usually not capitalized unless they are the first or
% last word of the caption. Table text will default to \footnotesize as
% the IEEE normally uses this smaller font for tables.
% The \label must come after \caption as always.
%
%\begin{table}[!t]
%% increase table row spacing, adjust to taste
%\renewcommand{\arraystretch}{1.3}
% if using array.sty, it might be a good idea to tweak the value of
% \extrarowheight as needed to properly center the text within the cells
%\caption{An Example of a Table}
%\label{table_example}
%\centering
%% Some packages, such as MDW tools, offer better commands for making tables
%% than the plain LaTeX2e tabular which is used here.
%\begin{tabular}{|c||c|}
%\hline
%One & Two\\
%\hline
%Three & Four\\
%\hline
%\end{tabular}
%\end{table}





% Note that the IEEE does not put floats in the very first column
% - or typically anywhere on the first page for that matter. Also,
% in-text middle ("here") positioning is typically not used, but it
% is allowed and encouraged for Computer Society conferences (but
% not Computer Society journals). Most IEEE journals/conferences use
% top floats exclusively. 
% Note that, LaTeX2e, unlike IEEE journals/conferences, places
% footnotes above bottom floats. This can be corrected via the
% \fnbelowfloat command of the stfloats package.

\subsubsection{Results on MARS}\label{res_mars}
MARS is a large and realistic video-based person re-id dataset since it was captured in the campus of  university with complex environment. Besides, it contains several natural detection/tracking errors as the person videos were collected by applying the automatic DPM detector and GMMCP tracker. Each person is captured by six cameras at most in MARS~\textemdash~compared with iLIDS-VID and PRID-2011, MARS has a much larger scale: 4 times and 30 times larger in the number of identities and total tracklets, respectively. Therefore, the relationships between person pairs are more complicated. 

In Tab. \ref{tab:mars}, we compare our results with the state-of-the-art methods on MARS. The compared methods include 8 descriptors (\textit{i.e.}, SDALF, HOG3D, HistLBP, gBiCov, GEI, LOMO, BoW and IDE) and 3 metric learning methods (\textit{i.e.}, DVR, KISSME and XQDA).    

Among all the recent video re-id methods, the best known rank1 accuracy is 65.3\% on MARS under single query setting, reported in \cite{zheng2016mars}. From Tab. \ref{tab:mars}, we can observe that our AMOC model can achieve the Rank1 recognition rate higher than the current best method (IDE\cite{zheng2016person}+XQDA) by 3\%. Although the rank5 performance of AMOC is slightly lower than ``IDE+XQDA'', it still achieves the mAP as high as 52.9\%. Note, the descriptor ``IDE'' in ``IDE+XQDA'' is obtained by fine-tuning the CaffeNet\cite{Krizhevsky2012ImageNet} on ImageNet- pretrained model. By contrast, our model is simply trained only on MARS from scratch. Besides, in all the compared methods, the extraction of descriptors and metric learning are separated as two individual processes. That is to say, the extraction of descriptors can not update parameters in the process of metric learning. In our proposed method, the motion feature and appearance feature are end-to-end learned jointly with the following feature accumulation part. As a result, the learned features contain enough discriminative information for the ultimate person re-id target.  To conclude, the experimental results show that our model performs comparable to other methods in more complex multi-camera person re-identification tasks, benefiting from the inherent end-to-end trainable appearance and motion information accumulation mechanism.


\begin{table}[htbp]
	\centering
	\caption{Comparison of our end-to-end AMOC method’s performance on MARS dataset to the state-of-the-arts.}
	\begin{tabular}{c|ccc|c}
		\hline
	
		\textbf{Method} & \textbf{Rank1} & \textbf{Rank5}& \textbf{Rank20}& \textbf{mAP}\\
		\hline
		SDALF\cite{farenzena2010person}+DVR\cite{wang2016person}  & 4.1 & 12.3 &25.1 &1.8\\
	    HOG3D\cite{Kl2008A}+KISSME\cite{K2012Large} &2.6&6.4&12.4&0.8\\
	    HistLBP\cite{xiong2014person}+XQDA\cite{liao2015person}	&18.6&33.0&45.9&8.0\\
		gBiCov\cite{ma2014covariance}+XQDA                                &9.2&19.8&33.5&3.7\\
	   GEI\cite{Han2005Individual}+KISSME  &1.2 &2.8 &7.4 &0.4\\
		LOMO\cite{liao2015person}+XQDA &30.7&46.6&60.9&16.4\\
		BoW\cite{zheng2015scalable}+KISSME &30.6&46.2&59.2&15.5\\
		IDE\cite{zheng2016person}  + XQDA  &65.3 &\textbf{82.0} &{89.0} &47.6\\
		IDE + KISSME &65.0	&{81.1}	&{88.9}	&45.6\\
		\hline
		\textbf{end-to-end AMOC+EpicFlow} & \textbf{68.3} & 81.4 &\textbf{90.6}&\textbf{52.9}\\
		\hline
	\end{tabular}%
	\label{tab:mars}%
\end{table}%

\subsection{Discussion}
We here discuss some potential limitations of our proposed model though its practical  effectiveness and superior performance have already been demonstrated in the above experiments.  

One potential issue with our model is the learned flow information may be redundant and not relevant to identifying the person of interest. From the visualization of optical flow maps estimated by the motion net of our model in Fig.~\ref{fig:flow_demo}, we observe that the foreground person and background present different colours (\textit{i.e.}, motion patterns)  in each frame. This is reasonable as the person and background move with different velocities in the video. As we explained in Sec.~\ref{pre_motion}, the motion net within our model is pre-trained using pre-extracted optical flow as supervision. If the supervision optical flow includes too much non-related motion information (\textit{i.e.}, background motion information),the person motion will be dominated by background motion within the computed optical flow. This is quite possible for the dataset including person video sequences automatically obtained by the detector and tracker, such as MARS, because inherent detection or tracking errors can bring more non-related motion information. Although deep networks that our method utilized are mostly powerful for learning robust feature representation, we believe the video person re-id performance would be further boosted if non-related motion information can be effectively suppressed like leveraging object segmentation models. We will study this issue in our future works.  


\section{Conclusion}
In this work, we propose an end-to-end Accumulative Motion Context Network (AMOC) based method addressing video person re-identification problem through joint spatial appearance learning and motion context accumulating from raw video frames. We conducted extensive experiments on three public available video-based person re-identification datasets to validate our method. Experimental results demonstrated that our model outperforms other state-of-the-art methods in most cases, and verified that our accumulative motion context model is beneficial for the recognition accuracy in person matching.

\section*{Acknowledgment}
This work was supported in part by the National Natural Science Foundation of China under Grant 61371155, Grant 61174170, and Grant 61632007, and in part by the China Scholarship Council under Grant 201506690007. 

The work of Jiashi Feng was partially supported by National University of Singapore startup grant R-263-000-C08-133, Ministry of Education of Singapore AcRF Tier One grant R-263-000-C21-112 and NUS IDS grant R-263-000-C67-646.


% if have a single appendix:
%\appendix[Proof of the Zonklar Equations]
% or
%\appendix  % for no appendix heading
% do not use \section anymore after \appendix, only \section*
% is possibly needed

% use appendices with more than one appendix
% then use \section to start each appendix
% you must declare a \section before using any
% \subsection or using \label (\appendices by itself
% starts a section numbered zero.)
%


%\appendices
%\section{Proof of the First Zonklar Equation}
%Appendix one text goes here.

% you can choose not to have a title for an appendix
% if you want by leaving the argument blank
%\section{}
%Appendix two text goes here.


% use section* for acknowledgment
%\section*{Acknowledgment}


%The authors would like to thank...


% Can use something like this to put references on a page
% by themselves when using endfloat and the captionsoff option.
\ifCLASSOPTIONcaptionsoff
  \newpage
\fi



% trigger a \newpage just before the given reference
% number - used to balance the columns on the last page
% adjust value as needed - may need to be readjusted if
% the document is modified later
%\IEEEtriggeratref{8}
% The "triggered" command can be changed if desired:
%\IEEEtriggercmd{\enlargethispage{-5in}}

% references section

% can use a bibliography generated by BibTeX as a .bbl file
% BibTeX documentation can be easily obtained at:
% http://mirror.ctan.org/biblio/bibtex/contrib/doc/
% The IEEEtran BibTeX style support page is at:
% http://www.michaelshell.org/tex/ieeetran/bibtex/
%\bibliographystyle{IEEEtran}
% argument is your BibTeX string definitions and bibliography database(s)
%\bibliography{IEEEabrv,../bib/paper}
%
% <OR> manually copy in the resultant .bbl file
% set second argument of \begin to the number of references
% (used to reserve space for the reference number labels box)
% Generated by IEEEtran.bst, version: 1.14 (2015/08/26)
\begin{thebibliography}{10}
	\providecommand{\url}[1]{#1}
	\csname url@samestyle\endcsname
	\providecommand{\newblock}{\relax}
	\providecommand{\bibinfo}[2]{#2}
	\providecommand{\BIBentrySTDinterwordspacing}{\spaceskip=0pt\relax}
	\providecommand{\BIBentryALTinterwordstretchfactor}{4}
	\providecommand{\BIBentryALTinterwordspacing}{\spaceskip=\fontdimen2\font plus
		\BIBentryALTinterwordstretchfactor\fontdimen3\font minus
		\fontdimen4\font\relax}
	\providecommand{\BIBforeignlanguage}[2]{{%
			\expandafter\ifx\csname l@#1\endcsname\relax
			\typeout{** WARNING: IEEEtran.bst: No hyphenation pattern has been}%
			\typeout{** loaded for the language `#1'. Using the pattern for}%
			\typeout{** the default language instead.}%
			\else
			\language=\csname l@#1\endcsname
			\fi
			#2}}
	\providecommand{\BIBdecl}{\relax}
	\BIBdecl
	
	\bibitem{farenzena2010person}
	M.~Farenzena, L.~Bazzani, A.~Perina, V.~Murino, and M.~Cristani, ``Person
	re-identification by symmetry-driven accumulation of local features,'' in
	\emph{IEEE Conference on Computer Vision and Pattern Recognition}, 2010, pp.
	2360--2367.
	
	\bibitem{li2014deepreid}
	W.~Li, R.~Zhao, T.~Xiao, and X.~Wang, ``Deepreid: Deep filter pairing neural
	network for person re-identification,'' in \emph{IEEE Conference on Computer
		Vision and Pattern Recognition}, 2014, pp. 152--159.
	
	\bibitem{yi2014deep}
	D.~Yi, Z.~Lei, S.~Liao, and S.~Z. Li, ``Deep metric learning for person
	re-identification,'' in \emph{2014 22nd International Conference on Pattern
		Recognition (ICPR)}.\hskip 1em plus 0.5em minus 0.4em\relax IEEE, 2014, pp.
	34--39.
	
	\bibitem{ahmed2015improved}
	E.~Ahmed, M.~Jones, and T.~K. Marks, ``An improved deep learning architecture
	for person re-identification,'' in \emph{IEEE Conference on Computer Vision
		and Pattern Recognition}, 2015, pp. 3908--3916.
	
	\bibitem{Chen2016An}
	Y.~C. Chen, W.~S. Zheng, J.~H. Lai, and P.~C. Yuen, ``An asymmetric distance
	model for cross-view feature mapping in person re-identification,''
	\emph{IEEE Transactions on Circuits and Systems for Video Technology}, pp.
	1--1, 2016.
	
	\bibitem{Wang2015Cross}
	X.~Wang, W.~S. Zheng, X.~Li, and J.~Zhang, ``Cross-scenario transfer person
	reidentification,'' \emph{IEEE Transactions on Circuits and Systems for Video
		Technology}, vol.~26, no.~8, pp. 1--1, 2015.
	
	\bibitem{liu2016end}
	H.~Liu, J.~Feng, M.~Qi, J.~Jiang, and S.~Yan, ``End-to-end comparative
	attention networks for person re-identification,'' \emph{IEEE Transactions on
		Image Processing}, vol.~26, no.~7, pp. 3492--3506, 2017.
	
	\bibitem{liu2015kernelized}
	H.~Liu, M.~Qi, and J.~Jiang, ``Kernelized relaxed margin components analysis
	for person re-identification,'' \emph{IEEE Signal Processing Letters},
	vol.~22, no.~7, pp. 910--914, 2015.
	
	\bibitem{yan2016person}
	Y.~Yan, B.~Ni, Z.~Song, C.~Ma, Y.~Yan, and X.~Yang, ``Person re-identification
	via recurrent feature aggregation,'' in \emph{European Conference on Computer
		Vision}.\hskip 1em plus 0.5em minus 0.4em\relax Springer, 2016, pp. 701--716.
	
	\bibitem{mclaughlinrecurrent}
	N.~McLaughlin, J.~Martinez~del Rincon, and P.~Miller, ``Recurrent convolutional
	network for video-based person re-identification,'' in \emph{IEEE Conference
		on Computer Vision and Pattern Recognition}, 2016, pp. 1325--1334.
	
	\bibitem{wang2016person}
	T.~Wang, S.~Gong, X.~Zhu, and S.~Wang, ``Person re-identification by
	discriminative selection in video ranking,'' \emph{IEEE Transactions on
		Pattern Analysis and Machine Intelligence}, vol.~38, pp. 2501--2514, 2016.
	
	\bibitem{you2016top}
	J.~You, A.~Wu, X.~Li, and W.-S. Zheng, ``Top-push video-based person
	re-identification,'' \emph{arXiv preprint arXiv:1604.08683}, 2016.
	
	\bibitem{liu2015spatio}
	K.~Liu, B.~Ma, W.~Zhang, and R.~Huang, ``A spatio-temporal appearance
	representation for viceo-based pedestrian re-identification,'' in \emph{IEEE
		International Conference on Computer Vision}, 2015, pp. 3810--3818.
	
	\bibitem{zhu2016video}
	X.~Zhu, X.-Y. Jing, F.~Wu, and H.~Feng, ``Video-based person re-identification
	by simultaneously learning intra-video and inter-video distance
	metrics.''\hskip 1em plus 0.5em minus 0.4em\relax IJCAI, 2016.
	
	\bibitem{gao2016temporally}
	C.~Gao, J.~Wang, L.~Liu, J.~G. Yu, and N.~Sang, ``Temporally aligned pooling
	representation for video-based person re-identification,'' in \emph{IEEE
		International Conference on Image Processing}, 2016, pp. 4284--4288.
	
	\bibitem{simonyan2014two}
	K.~Simonyan and A.~Zisserman, ``Two-stream convolutional networks for action
	recognition in videos,'' in \emph{Advances in Neural Information Processing
		Systems}, 2014, pp. 568--576.
	
	\bibitem{feichtenhofer2016convolutional}
	C.~Feichtenhofer, A.~Pinz, and A.~Zisserman, ``Convolutional two-stream network
	fusion for video action recognition,'' \emph{arXiv preprint
		arXiv:1604.06573}, 2016.
	
	\bibitem{zheng2016mars}
	L.~Zheng, Z.~Bie, Y.~Sun, J.~Wang, C.~Su, S.~Wang, and Q.~Tian, ``Mars: A video
	benchmark for large-scale person re-identification,'' in \emph{European
		Conference on Computer Vision}.\hskip 1em plus 0.5em minus 0.4em\relax
	Springer, 2016, pp. 868--884.
	
	\bibitem{dosovitskiy2015flownet}
	A.~Dosovitskiy, P.~Fischer, E.~Ilg, P.~Hausser, C.~Hazirbas, V.~Golkov,
	P.~van~der Smagt, D.~Cremers, and T.~Brox, ``Flownet: Learning optical flow
	with convolutional networks,'' in \emph{IEEE International Conference on
		Computer Vision}, 2015, pp. 2758--2766.
	
	\bibitem{wang2014person}
	T.~Wang, S.~Gong, X.~Zhu, and S.~Wang, ``Person re-identification by video
	ranking,'' in \emph{European Conference on Computer Vision}.\hskip 1em plus
	0.5em minus 0.4em\relax Springer, 2014, pp. 688--703.
	
	\bibitem{hirzer2011person}
	M.~Hirzer, C.~Beleznai, P.~M. Roth, and H.~Bischof, ``Person re-identification
	by descriptive and discriminative classification,'' in \emph{Scandinavian
		conference on Image analysis}.\hskip 1em plus 0.5em minus 0.4em\relax
	Springer, 2011, pp. 91--102.
	
	\bibitem{gray2008viewpoint}
	D.~Gray and H.~Tao, ``Viewpoint invariant pedestrian recognition with an
	ensemble of localized features,'' in \emph{European conference on computer
		vision}.\hskip 1em plus 0.5em minus 0.4em\relax Springer, 2008, pp. 262--275.
	
	\bibitem{cheng2011custom}
	D.~S. Cheng, M.~Cristani, M.~Stoppa, L.~Bazzani, and V.~Murino, ``Custom
	pictorial structures for re-identification.'' in \emph{BMVC}, vol.~1, no.~2,
	2011, p.~6.
	
	\bibitem{ma2012local}
	B.~Ma, Y.~Su, and F.~Jurie, ``Local descriptors encoded by fisher vectors for
	person re-identification,'' in \emph{European Conference on Computer
		Vision}.\hskip 1em plus 0.5em minus 0.4em\relax Springer, 2012, pp. 413--422.
	
	\bibitem{kviatkovsky2013color}
	I.~Kviatkovsky, A.~Adam, and E.~Rivlin, ``Color invariants for person
	reidentification,'' \emph{IEEE Transactions on Pattern Analysis and Machine
		Intelligence}, vol.~35, no.~7, pp. 1622--1634, 2013.
	
	\bibitem{weinberger2009distance}
	K.~Q. Weinberger and L.~K. Saul, ``Distance metric learning for large margin
	nearest neighbor classification,'' \emph{Journal of Machine Learning
		Research}, vol.~10, no. Feb, pp. 207--244, 2009.
	
	\bibitem{prosser2010person}
	B.~Prosser, W.-S. Zheng, S.~Gong, T.~Xiang, and Q.~Mary, ``Person
	re-identification by support vector ranking.'' in \emph{BMVC}, vol.~2, no.~5,
	2010, p.~6.
	
	\bibitem{zheng2013reidentification}
	W.-S. Zheng, S.~Gong, and T.~Xiang, ``Reidentification by relative distance
	comparison,'' \emph{IEEE transactions on pattern analysis and machine
		intelligence}, vol.~35, no.~3, pp. 653--668, 2013.
	
	\bibitem{liao2015person}
	S.~Liao, Y.~Hu, X.~Zhu, and S.~Z. Li, ``Person re-identification by local
	maximal occurrence representation and metric learning,'' in \emph{IEEE
		Conference on Computer Vision and Pattern Recognition}, 2015, pp. 2197--2206.
	
	\bibitem{simonnet2012re}
	D.~Simonnet, M.~Lewandowski, S.~A. Velastin, J.~Orwell, and E.~Turkbeyler,
	``Re-identification of pedestrians in crowds using dynamic time warping,'' in
	\emph{European Conference on Computer Vision}.\hskip 1em plus 0.5em minus
	0.4em\relax Springer, 2012, pp. 423--432.
	
	\bibitem{karaman2012identity}
	S.~Karaman and A.~D. Bagdanov, ``Identity inference: generalizing person
	re-identification scenarios,'' in \emph{European Conference on Computer
		Vision}.\hskip 1em plus 0.5em minus 0.4em\relax Springer, 2012, pp. 443--452.
	
	\bibitem{li2013locally}
	W.~Li and X.~Wang, ``Locally aligned feature transforms across views,'' in
	\emph{IEEE Conference on Computer Vision and Pattern Recognition}, 2013, pp.
	3594--3601.
	
	\bibitem{karanam2015sparse}
	S.~Karanam, Y.~Li, and R.~J. Radke, ``Sparse re-id: Block sparsity for person
	re-identification,'' in \emph{IEEE Conference on Computer Vision and Pattern
		Recognition Workshops}, 2015, pp. 33--40.
	
	\bibitem{ding2015deep}
	S.~Ding, L.~Lin, G.~Wang, and H.~Chao, ``Deep feature learning with relative
	distance comparison for person re-identification,'' \emph{Pattern
		Recognition}, vol.~48, no.~10, pp. 2993--3003, 2015.
	
	\bibitem{hadsell2006dimensionality}
	R.~Hadsell, S.~Chopra, and Y.~LeCun, ``Dimensionality reduction by learning an
	invariant mapping,'' in \emph{IEEE Conference on Computer Vision and Pattern
		Recognition}, 2006, pp. 1735--1742.
	
	\bibitem{lucas1981iterative}
	B.~D. Lucas, T.~Kanade \emph{et~al.}, ``An iterative image registration
	technique with an application to stereo vision.'' in \emph{IJCAI}, vol.~81,
	no.~1, 1981, pp. 674--679.
	
	\bibitem{revaud2015epicflow}
	J.~Revaud, P.~Weinzaepfel, Z.~Harchaoui, and C.~Schmid, ``Epicflow:
	Edge-preserving interpolation of correspondences for optical flow,'' in
	\emph{IEEE Conference on Computer Vision and Pattern Recognition}, 2015, pp.
	1164--1172.
	
	\bibitem{girshick2015fast}
	R.~Girshick, ``Fast r-cnn,'' in \emph{IEEE International Conference on Computer
		Vision}, 2015, pp. 1440--1448.
	
	\bibitem{kingma2014adam}
	D.~Kingma and J.~Ba, ``Adam: A method for stochastic optimization,''
	\emph{arXiv preprint arXiv:1412.6980}, 2014.
	
	\bibitem{Felzenszwalb2014Object}
	P.~F. Felzenszwalb, R.~B. Girshick, D.~Mcallester, and D.~Ramanan, ``Object
	detection with discriminatively trained part-based models,'' \emph{IEEE
		Transactions on Pattern Analysis and Machine Intelligence}, vol.~32, no.~9,
	pp. 1627--45, 2014.
	
	\bibitem{Dehghan2015GMMCP}
	A.~Dehghan, S.~M. Assari, and M.~Shah, ``Gmmcp tracker: Globally optimal
	generalized maximum multi clique problem for multiple object tracking,'' in
	\emph{IEEE International Conference on Computer Vision and Pattern
		Recognition}, 2015, pp. 4091--4099.
	
	\bibitem{torch}
	R.~Collobert, K.~Kavukcuoglu, and C.~Farabet, ``Torch7: A matlab-like
	environment for machine learning,'' in \emph{BigLearn, NIPS Workshop}, 2011.
	
	\bibitem{baker2011database}
	S.~Baker, D.~Scharstein, J.~Lewis, S.~Roth, M.~J. Black, and R.~Szeliski, ``A
	database and evaluation methodology for optical flow,'' \emph{International
		Journal of Computer Vision}, vol.~92, no.~1, pp. 1--31, 2011.
	
	\bibitem{cho2016improving}
	Y.-J. Cho and K.-J. Yoon, ``Improving person re-identification via pose-aware
	multi-shot matching,'' in \emph{IEEE Conference on Computer Vision and
		Pattern Recognition}, 2016, pp. 1354--1362.
	
	\bibitem{chen2015person}
	J.~Chen, Y.~Wang, and Y.~Y. Tang, ``Person re-identification by exploiting
	spatio-temporal cues and multi-view metric learning.''
	
	\bibitem{Li2015Multi}
	Y.~Li, Z.~Wu, S.~Karanam, and R.~J. Radke, ``Multi-shot human re-identification
	using adaptive fisher discriminant analysis,'' in \emph{British Machine
		Vision Conference}, 2015, pp. 73.1--73.12.
	
	\bibitem{Karanam2015Person}
	S.~Karanam, Y.~Li, and R.~J. Radke, ``Person re-identification with
	discriminatively trained viewpoint invariant dictionaries,'' in \emph{IEEE
		International Conference on Computer Vision}, 2015, pp. 4516--4524.
	
	\bibitem{Kl2008A}
	A.~Kläser, M.~Marszalek, and C.~Schmid, ``A spatio-temporal descriptor based
	on 3d-gradients,'' in \emph{British Machine Vision Conference 2008, Leeds,
		September}, 2008.
	
	\bibitem{K2012Large}
	M.~Köstinger, M.~Hirzer, P.~Wohlhart, P.~M. Roth, and H.~Bischof, ``Large
	scale metric learning from equivalence constraints,'' in \emph{Computer
		Vision and Pattern Recognition}, 2012, pp. 2288--2295.
	
	\bibitem{Han2005Individual}
	J.~Han and B.~Bhanu, ``Individual recognition using gait energy image,''
	\emph{IEEE Transactions on Pattern Analysis \& Machine Intelligence}, vol.~28,
	no.~2, pp. 316--322, 2005.
	
	\bibitem{xiong2014person}
	F.~Xiong, M.~Gou, O.~Camps, and M.~Sznaier, ``Person re-identification using
	kernel-based metric learning methods,'' in \emph{European conference on
		computer vision}.\hskip 1em plus 0.5em minus 0.4em\relax Springer, 2014, pp.
	1--16.
	
	\bibitem{zheng2015scalable}
	L.~Zheng, L.~Shen, L.~Tian, S.~Wang, J.~Wang, and Q.~Tian, ``Scalable person
	re-identification: A benchmark,'' in \emph{IEEE International Conference on
		Computer Vision}, 2015, pp. 1116--1124.
	
	\bibitem{ma2014covariance}
	B.~Ma, Y.~Su, and F.~Jurie, ``Covariance descriptor based on bio-inspired
	features for person re-identification and face verification,'' \emph{Image
		and Vision Computing}, vol.~32, no.~6, pp. 379--390, 2014.
	
	\bibitem{zheng2016person}
	L.~Zheng, H.~Zhang, S.~Sun, M.~Chandraker, and Q.~Tian, ``Person
	re-identification in the wild,'' \emph{arXiv preprint arXiv:1604.02531},
	2016.
	
	\bibitem{Krizhevsky2012ImageNet}
	A.~Krizhevsky, I.~Sutskever, and G.~E. Hinton, ``Imagenet classification with
	deep convolutional neural networks,'' in \emph{International Conference on
		Neural Information Processing Systems}, 2012, pp. 1097--1105.
	
\end{thebibliography}

%\begin{IEEEbiography}
%	[{\includegraphics[width=1in,height=1.25in,clip,keepaspectratio]{./author_pics/Hao_Liu.jpg}}]{Hao Liu} received the M.E. degree in Communication and Information Systems from Hefei University of Technology, where he is currently pursuing the Ph.D. degree. His research interests include computer vision, image processing, pattern recognition, and deep learning. He is an reviewer of the ACM Transactions on Multimedia Computing, Communications, and Applications, the IEEE Transactions on Cybernetics and several other journals.
%\end{IEEEbiography}
%
%\begin{IEEEbiography}
%	[{\includegraphics[width=1in,height=1.25in,clip,keepaspectratio]{./author_pics/Zequn_Jie.jpg}}]{Zequn Jie} received the B.E. degree in mechanical engineering from the University of Science and Technology of China. He obtained his Ph.D. degree from National University of Singapore. Currently He is working as a research scientist in Tencent AI Lab. His current research interests mainly include object localization related topics in computer vision, such as object proposal, object detection.
%\end{IEEEbiography}
%
%\begin{IEEEbiography}
%	[{\includegraphics[width=1in,height=1.25in,clip,keepaspectratio]{./author_pics/Karlekar_Jayashree.jpg}}]{Karlekar Jayashree} Received Ph.D degree from Dept. of Electrical and Electronics , Indian Institute of Technology, Mumbai in 2000. She worked as Technical Leader in Hughes Software Systems, New Delhi during 1999-2003, as a faculty in Dept. Electronics and Communications, Shri Guru Gobind Singjji Institute of Eng. \&Tech., Nanded during 2004-2007 and as a Research Fellow in School of Computing and Dept. Electronics and Computer Eng.  in National University of Singapore during 2007-2010. She is currently working in Panasonic R\&D Center, Singapore as a Senior Staff Engineer.
%\end{IEEEbiography}
%
%
%\begin{IEEEbiography}
%	[{\includegraphics[width=1in,height=1.25in,clip,keepaspectratio]{./author_pics/Meibin_Qi.jpg}}]{Meibin Qi} is currently a Professor in the School of Computer and Information at Hefei University of Technology. He received the B.E. degree in radio technology from Chongqing University in 1991, the M.E. and Ph.D. degrees in signal and information processing from Hefei University of Technology in 2001 and 2007. His research interests include pattern recognition, video coding, video surveillance and the application of DSP technology.
%\end{IEEEbiography}
%
%\begin{IEEEbiography}
%	[{\includegraphics[width=1in,height=1.25in,clip,keepaspectratio]{./author_pics/Jianguo_Jiang.jpg}}]{Jianguo Jiang} is currently a professor in the School of Computer and Information at Hefei University of Technology. He received the B.E. degree in radio technology and M.E. degree in signal and information processing from Hefei University of Technology in 1982 and 1989. His research interests include digital image analysis and processing, distributed intelligent system and the application of DSP technology.
%\end{IEEEbiography}
%
%\begin{IEEEbiography}
%	[{\includegraphics[width=1in,height=1.25in,clip,keepaspectratio]{./author_pics/Shuicheng_Yan.jpg}}]{Shuicheng Yan} is chief scientist of Qihoo/360 company, and also the Dean's Chair Associate Professor at National University of Singapore. Dr. Yan's research areas include machine learning, computer vision and multimedia, and he has authored/co-authored hundreds of technical papers over a wide range of research topics, with Google Scholar citation over 20,000 times and H-index 66. He is ISI Highly-cited Researcher of 2014, 2015 and 2016. His team received 7 times winner or honorable-mention prizes in PASCAL VOC and ILSVRC competitions, along with more than 10 times best (student) paper prizes. He is IAPR Fellow.
%\end{IEEEbiography}
%
%\begin{IEEEbiography}
%	[{\includegraphics[width=1in,height=1.25in,clip,keepaspectratio]{./author_pics/Jiashi_Feng.jpg}}]{Jiashi Feng} is currently an Assistant Professor in the Department of Electrical and Computer Engineering at the National University of Singapore. He got his B.E. degree from University of Science and Technology, China in 2007 and Ph.D. degree from National University of Singapore in 2014. He was a postdoc researcher at University of California from 2014 to 2015. His current research interest focuses on machine learning and computer vision techniques for large-scale data analysis. Specifically, he has done work in object recognition, deep learning, machine learning, high-dimensional statistics and big data analysis.
%\end{IEEEbiography}
% biography section
% 
% If you have an EPS/PDF photo (graphicx package needed) extra braces are
% needed around the contents of the optional argument to biography to prevent
% the LaTeX parser from getting confused when it sees the complicated
% \includegraphics command within an optional argument. (You could create
% your own custom macro containing the \includegraphics command to make things
% simpler here.)
%\begin{IEEEbiography}[{\includegraphics[width=1in,height=1.25in,clip,keepaspectratio]{mshell}}]{Michael Shell}
% or if you just want to reserve a space for a photo:

%\begin{IEEEbiography}{Michael Shell}
%Biography text here.
%\end{IEEEbiography}
%
%% if you will not have a photo at all:
%\begin{IEEEbiographynophoto}{John Doe}
%Biography text here.
%\end{IEEEbiographynophoto}

% insert where needed to balance the two columns on the last page with
% biographies
%\newpage
%\bibliography{video_personn}{}
%\bibliographystyle{IEEEtran}
% You can push biographies down or up by placing
% a \vfill before or after them. The appropriate
% use of \vfill depends on what kind of text is
% on the last page and whether or not the columns
% are being equalized.

%\vfill

% Can be used to pull up biographies so that the bottom of the last one
% is flush with the other column.
%\enlargethispage{-5in}



% that's all folks
\end{document}


