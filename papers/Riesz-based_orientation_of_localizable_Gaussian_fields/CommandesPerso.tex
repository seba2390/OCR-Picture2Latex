%%%%%%%%%%%%%%%%%%%%%%%%%%%%%%%%%%%%%%%%
%           Commandes perso            %
%%%%%%%%%%%%%%%%%%%%%%%%%%%%%%%%%%%%%%%%

\newcommand{\alp}{\texorpdfstring{\ensuremath{\upalpha}\xspace}{alpha }}
\newcommand{\bet}{\texorpdfstring{\ensuremath{\upbeta}\xspace}{b\'{e}ta }}
\newcommand{\alpbet}{\texorpdfstring{\ensuremath{\upalpha-\upbeta}\xspace}{alpha-b\'{e}ta}}
\newcommand{\alpt}{\ensuremath{\alpha_2}\xspace}
\newcommand{\strt}{\gls{strt}\xspace}


% Tenseur des déformation cylindrique
\newcommand{\epsrr}{\ensuremath{\varepsilon_{rr}}\xspace}
\newcommand{\epstt}{\ensuremath{\varepsilon_{\theta\theta}}\xspace}
\newcommand{\epszz}{\ensuremath{\varepsilon_{zz}}\xspace}
\newcommand{\epsrt}{\ensuremath{\varepsilon_{r\theta}}\xspace}
\newcommand{\epstz}{\ensuremath{\varepsilon_{\theta z}}\xspace}
\newcommand{\epszr}{\ensuremath{\varepsilon_{zr}}\xspace}

\newcommand{\matlab}{\textsc{Matlab}\texttrademark\xspace}

\renewcommand{\geq}{\geqslant}
\renewcommand{\leq}{\leqslant}


%% Figures centrées, et en position 'here, top, bottom or page'
\newenvironment{figureth}{%
		\begin{figure}[htbp]
			\centering
	}{
		\end{figure}
		}
		
		
%% Tableaux centrés, et en position 'here, top, bottom or page'
\newenvironment{tableth}{%
		\begin{table}[htbp]
			\centering
			%\rowcolors{1}{coleurtableau}{coleurtableau}
	}{
		\end{table}
		}

%% Sous-figures centrées, en position 'top'		
\newenvironment{subfigureth}[1]{%
	\begin{subfigure}[t]{#1}
	\centering
}{
	\end{subfigure}
}

\newcommand{\citationChap}[2]{%
	\epigraph{{\fontfamily{futs}\selectfont \og \textit{#1}\fg{}}}{#2}
}

%% On commence par une page impaire quand on change le style de numérotation de pages 
\let\oldpagenumbering\pagenumbering
\renewcommand{\pagenumbering}[1]{%
	\cleardoublepage
	\oldpagenumbering{#1}
}

\newcommand{\diff}{\mathop{}\mathopen{}\mathrm{d}}
%\newcommand{\abs}[1]{\left\lvert#1\right\rvert}
\newcommand{\norme}[1]{\left\lVert#1\right\rVert}

\newcommand{\eu}{\ensuremath{\mathrm{e}}}
\newcommand{\ii}{\ensuremath{\mathrm{i}}}
\newcommand{\jj}{\ensuremath{\mathrm{j}}}
\newcommand{\kk}{\ensuremath{\mathrm{k}}}
\newcommand{\T}{\ensuremath{\mathsf{T}}}

\newcommand{\expj}[1]{\ensuremath{\eu^{\, \jj #1}}}
\providecommand{\newoperator}[3]{%
	\newcommand*{#1}{\mathop{#2}#3}}
\providecommand{\renewoperator}[3]{%
	\renewcommand*{#1}{\mathop{#2}#3}}
\renewoperator{\Re}%
	{\mathrm{Re}}{\nolimits}
\renewoperator{\Im}%
	{\mathrm{Im}}{\nolimits}
	
\newcommand{\vect}[1]{\bm{#1}}
\newcommand{\mat}[1]{\mathbf{#1}}
\newcommand{\ten}[1]{\bm{\mathsf{#1}}}
\newcommand{\ope}[1]{\mathbf{#1}}
\newcommand{\ip}[2]{\left \langle #1,\, #2\right \rangle}
\newcommand{\RT}{\mathcal{R}}
\newcommand{\RTb}{\bm{\RT}}
