  \RequirePackage{lineno}

\documentclass[floatfix,superscriptaddress,a4paper,
               showpieces,showkeys,nofootinbib,preprint]{revtex4-2}

\textwidth=17cm \textheight=24.5cm \topmargin=-0.5cm
\oddsidemargin=-0.2cm
\usepackage{epsfig}
\usepackage{latexsym}
\usepackage{xspace}
\usepackage{comment}
\usepackage{soul}
\usepackage{inputenc}
\usepackage{indentfirst}
\usepackage{enumerate}
\usepackage{color}
\usepackage{epstopdf}
\usepackage{colortbl}
\usepackage{blindtext}

\usepackage{amsmath}
\usepackage{bm}
\usepackage[english]{babel}
\usepackage{url}
\topmargin= -1cm
\textheight= 22cm
\usepackage{hyperref}

\graphicspath{{plots/}}

\begin{document}
\begin{table}[!htbp]
\centering
\caption{
\label{tab:positionPass}Abbreviations of player positions.
}\s
\begin{tabular}{@{}ll}
\hline
Player position                         & Abbreviation \\
\hline
Center Back                             & CB                   \\
Right Center Back                       & RCB                \\
Left Center Back                        & LCB                   \\
Left Defensive Midfielder               & LDMF                   \\
Right Defensive Midfielder              & RDMF                  \\
Right Center Back (3 at the back)       & RCB3                  \\
Goalkeeper                              & GK                 \\
Defensive Midfielder                    & DMF                  \\
Left Center Midfielder                  & LCMF                  \\
Left Center Back (3 at the back)        & LCB3                  \\
Right Center Midfielder                 & RCMF                  \\
Left Center Midfielder (3 at the back)  & LCMF3                   \\
Right Back                              & RB                   \\
Left Back                               & LB                  \\
Attacking Midfielder                    & AMF                   \\
Right Center Midfielder (3 at the back) & RCMF3                  \\
Left Attacking Midfielder               & LAMF                 \\
Left Wing Forward                       & LWF                  \\
Right Wing Forward                      & RWF                  \\
Left Wing                               & LW                   \\
Right Attacking Midfielder              & RAMF                   \\
Right Wing Back                         & RWB                   \\
Second Striker                          & SS                   \\
Right Wing                              & RW                  \\
Left Wing Back                          & LWB                  \\
Striker                          & CF                   \\
Left Back (5 at the back)               & LB5                 \\
Right Back (5 at the back)              & RB5         \\
\hline
\end{tabular}

\end{table}


\title{
Production locality and spatial diffusion of heavy flavour \\
at high energy densities 
}

\author{M. Gazdzicki}
\affiliation{Goethe-University Frankfurt am Main, Germany}
\affiliation{Jan Kochanowski University, Kielce, Poland}
\author{D. Kiko\l{}a}
\affiliation{Warsaw University of Technology, Warsaw, Poland}
\author{I. Pidhurskyi}
\affiliation{Goethe-University Frankfurt am Main, Germany}
\affiliation{Jan Kochanowski University, Kielce, Poland}
\affiliation{European Organization for Nuclear Research, CERN, Geneva, Switzerland}
\author{L. Tinti}
\affiliation{Jan Kochanowski University, Kielce, Poland}
 
  
\begin{abstract}
% --------------------------
%         Abstract
% --------------------------
Heavy-ion collisions are a unique tool to test the behaviour of matter in extreme conditions. The momentum correlations of charm and bottom hadrons have been considered for testing heavy quarks' thermalisation in the hot, dense medium produced by the collisions.
In addition to these back-to-back correlations due to the quark--anti-quark pair creation dynamics, there are other important sources of momentum correlations, which allow us to explore the rich physics of heavy-ion collisions further.

Here we show significant momentum correlations even in the thermalisation of charm quarks in the expanding medium, and this effect can be measured in collisions at sufficiently low energies.

The momentum correlations depend on the correlations in the configuration space,
more specifically, on the spatial separation between charm and anti-charm hadrons emission points.
The pair production locality and spatial diffusion of the charm quarks give the latter.
Using an example of central Pb+Pb collisions at the CERN SPS energies, we show that future measurements of the azimuthal correlations
of charm and anti-charm hadrons should allow us to distinguish between different assumptions on their spatial correlations. 

This provides a unique window into a poorly understood sector of particle production at high-energy densities. The measurements can help to constrain the diffusion of charm quarks and verify assumptions concerning production locality
of a charm and ant-charm quark pair.

\end{abstract}

\maketitle

\newpage
%\linenumbers

% -----------------------
%        Text 
% -----------------------

\section{Introduction}

Collisions of heavy ions at relativistic energies provide insights into fascinating features of nuclear matter at high energy densities.
This includes the creation of the Quark-Gluon Plasma (QGP) - a state of matter with quark and gluon degrees of freedom expected to exist in the Universe's first moments. Impressive progress has been made in the last three decades in experimental and theoretical studies of the QGP. Still, there are areas where an adequate understanding of the underlying processes is yet to be achieved, for example, in measuring and modelling particle correlations and fluctuations.

The most popular approaches to predict the spectra of final-state particles in heavy-ion collisions are focused on the lowest~\cite{Florkowski:Phenomenology}. For instance,
the relativistic-kinetic theory deals with the one-particle distribution function $f(x,{\bf p})$ but neglects the many-particle distributions. That is the higher orders in the Bogoliubov – Born – Green – Kirkwood – Yvon hierarchy~\cite{Bogoliubov:1946}. 
More generally, the expectation values of operators are considered, for instance, the energy density in hydrodynamics. Still, their fluctuations (e.g. variance and higher orders) and related correlations are more difficult to deal with. Experimentally, measurements of correlations and fluctuations are also significantly more challenging than measurements of one-particle spectra.
Recently the field has been mostly motivated by the search for the critical point of strongly interacting matter; for example, see Ref.~\cite{Gazdzicki:2015ska} and references therein.

Measurements of correlations between the charm meson and its anti-particle have been proposed to test the equilibration of charm~\cite{Zhu:2006er,Cao:2015cba} in momentum space. In a semi-classical picture, the initial back-to-to back momentum correlations between the $c$ and $\bar c$ quarks are reduced by the interactions with the medium and
hadronisation of the quarks (see, for instance, Ref.~\cite{He_2020} and references therein).
This paper is motivated by the fact that, even for a locally thermalised and expanding medium, the momenta of charm and anti-charm hadrons originating from the same $c$ and $\bar{c}$ pair are correlated. 
Depending on the creation points and spatial diffusion properties of the charm in the medium, the charm hadron and the anti-hadron emission points can be either close or distant. In a locally thermalised and expanding system, the charm hadrons have an average momentum dependent on the fluid cell's drift speed (flow). If the emission points of the hadrons are close, they will have a similar drift, and thus their momenta will be correlated. 


Thus the presented here idea
utilises the collective flow of charm hadrons measured
in heavy-ion collisions at 
high energies~\cite{STAR:D0:v1,STAR:D0:v2,STAR:eHF:v2:2014yia,ALICE:D0:v2,ALICE:D0:v2:2013,ALICE:eHF:v2,ALICE:D:v2:2020iug,ATLAS:muHF:v2:2018ofq,ALICE:D:v2:2017pbx,ALICE:beauty:v2:2020hdw,PHENIX:eHF:v2}. We assume that final-state momenta of charm
hadrons are given by the superposition of the flow and a random (thermal) contribution due to statistical hadronisation. Hadronic rescattering 
and final-state interactions are neglected, supported by the recent measurements of interaction parameters of $D$-mesons with hadrons~\cite{ALICE:2022enj,QM2022:ALICE:D:hadron:femotscopy}. The flow contribution depends only on the emission point in the freeze-out hypersurface, whereas the thermal contribution is a random effect, uncorrelated for different hadrons.

Here we stress that the arguments are generally valid, but to directly measure the wanted correlations, one should have no more than one $c\bar{c}$ pair produced per collision. Otherwise, the measured two-particle correlation function includes pairs of $c$- and $\bar{c}$-hadrons 
coming from different and likely independent charm production processes.
The magnitude of this unwanted contribution to the momentum-correlation results strongly depends on the multiplicity distribution of heavy-flavour pairs. This effect is especially important in the heavy-ion collisions at RHIC and the LHC. On average, one expects $\simeq3$ $c\bar{c}$ pairs in the 10\% most central Au+Au collisions at $\mathrm{\sqrt{s_{NN}} = 200}$~GeV at RHIC~\cite{STAR:2018zdy,STAR:2012nbd}, and a few tens at the LHC (for example, $\simeq 30$ $c\bar{c}$ pairs in the 10\% most central Pb+Pb reactions at $\mathrm{\sqrt{s_{NN}} = 5.02}$~TeV~\cite{ALICE:2021dhb,ALICE:2021rxa}). The
multiplicity distribution of heavy-flavour is nowadays difficult to access experimentally, and thus the wanted correlations at very high energies cannot be extracted in a model-independent way. Thus to minimize the bias due to unwanted correlations, the measurements should be performed at sufficiently low collision energies, where the mean multiplicity of  $c\bar{c}$ pairs is below one.
For this reason, we consider an example of central Pb+Pb collisions at the
CERN SPS energies. 
This example can be straightforwardly extended to bottom hadron production at LHC.


The heavy-flavour production and azimuthal correlations in heavy-ion collisions at very high energies were addressed theoretically in the past; for review, see Ref.~\cite{Andronic:2015wma}. In particular, they were considered as a tool for uncovering a mechanism behind the jet suppression~\cite{Attems:2022ubu,Attems:2022otp} and the study of charm energy-loss 
mechanism~\cite{Rohrmoser:2017vsa,PhysRevC.90.024907,Wang:2019vhg,Wang:2021xpv}.
The heavy-quark spatial diffusion in QCD matter was discussed recently in 
Refs.~\cite{Sambataro:2020pge,Capellino:2022nvf,Satapathy:2022xdw}, see also references therein.
The ATLAS experiment measured the azimuthal-angle correlations of muon pairs originating from heavy-flavour decays in 5.02~TeV Pb+Pb collisions~\cite{ATLAS:2022mhn}. One notes that the measured 
muon pairs come from jet-like correlations of high transverse-momentum heavy-flavour hadrons. 



The paper is organised as follows.
The above intuitive expectations are quantified using simple modelling presented in Sec.~\ref{sec:model}. The physics meaning of different assumptions on charm-hadron correlation in the emission volume is also discussed in this section. The feasibility of the corresponding measurements is estimated in Sec.~\ref{sec:experiment}, and the results are summarised in Sec.~\ref{sec:summary}.


%---------------------------------------------------------------------
\section{Quantitative predictions and discussion}
\label{sec:model}

The following assumptions are made to quantify the intuitive expectations for the considered momentum correlation:

\begin{enumerate}[(i)]
\item
The production of charm  hadrons in head-on Pb+Pb collisions is considered.
The collision energy is assumed to be adjusted to have a mean charm multiplicity below one, allowing to neglect production of more than one $c$- and $\bar{c}$ hadron pair in a single collision. This likely corresponds to the top CERN SPS energy ($\sqrt{s_{NN}} \approx 17$~GeV)~\cite{Snoch:2018nnj}.

\item 
The charm hadrons are emitted from the freeze-out hyper-surface of a spherical fireball undergoing a Hubble-like expansion. That is, the three velocity reads $\vec{v} = \vec{r}/t $, with $\vec{r} = {x,y,z}$ being the distance from the centre of the fireball, and four-velocity $u^\mu =x^\mu/\tau=x^\mu/\sqrt{t^2-r^2}$. It was recently demonstrated that Hubble-like expansion is an appropriate approximation of velocity fields in heavy-ion collisions in the energy range of our interest~\cite{Tsegelnik:2022eoz}.

\item The freeze-out hyper-surface is set by the freeze-out time $\tau=\tau_{fo}$ and the maximal radius $r\le R_{max}$. They are set, respectively, to $\tau_{fo}=9$~fm/$c$ and $R_{max}= 6$~fm. 

\item Emission probability of charm hadrons is independent of the fluid cell on the freeze-out hyper-surface, consistent with the method used to predict the spectra within the relativistic hydrodynamics approach. Note that the considered correlations are given by the conditional probability of the charm hadrons to be emitted from the same cell or another one with respect to the anti-charm one.

\item
In the rest frame of the flow, the charm hadron momentum $p$ distribution at
the freeze-out hyper-surface is assumed to be the statistical one:
\begin{equation}
   \frac{d^3N} {dp~d^2\Omega} ~\propto~ p^2 \exp{ \left(  \frac{- \sqrt{m^2 + p^2} }{T_{\rm fo}} \right) }~,
\end{equation}
where $m = 1.869$~\GeVc is the charm hadron mass assumed to be equal to the $D^0$ meson mass, and the temperature parameter is $T_{\rm fo}$~=~150~MeV. 
The statistical momenta of charm hadrons are drawn independently.
\item
To calculate the hadron momentum in the collision rest frame, the obtained statistical four-momentum is boosted with the flow velocity, 
$\vec{v} =  \vec{u}~/~\sqrt{1+u^2}$.
\end{enumerate}
Note that for simplicity, we do not consider correlations between momenta $c$ and $\bar{c}$-hadrons resulting from the energy-momentum conservation and dynamics of the pair creation process. The change of these correlations during the system evolution was discussed in Ref.~\cite{Zhu:2006er,Cao:2015cba} for heavy-ion collision at top RHIC energy ($\sqrt{s_{NN}} = 200$~GeV) and at the LHC. Given that we focus on producing charmed mesons with low \pt\ in low-energy collisions, we expect the back-to-back correlation will not play a significant role in the measurement we consider in this work.

Then the results are calculated for three different space correlations of the $c$- and $\bar{c}$-hadrons. These are
\begin{enumerate}[(a)]
\item
The $c$- and $\bar{c}$-hadrons are emitted from the same fluid cell.
Thus the average of their momenta is set by the  drift velocity of the cell. Their actual momenta are different because of the independence of their momenta in the fluid rest frame. This ansatz is labelled  the \textit{local} emission.
\item
The emission points of charm hadrons are independent of each other.
They don't have a common drift velocity, hence.
This ansatz is labelled the \textit{independent} emission.
\item
The intermediate case is modelled, assuming the correlation function of the emission points to be the 3D Gaussian with $\sigma = \sigma_x = \sigma_y = \sigma_z =2$~fm. 
Note that the points are required to be within the fireball volume. The flow components of $c$- and $\bar{c}$-hadrons are different but correlated, leading to the correlation of their hadron momenta.
Clearly in the limits of $\sigma \rightarrow 0$ and $\sigma \rightarrow \infty$ 
one recovers the local and independent emissions, respectively.
This ansatz is labelled the \textit{correlated} emission. 
\end{enumerate}

Figure~\ref{fig:Sim:Corr} shows the distribution of $c$-$\bar{c}$ hadron pairs 
in the difference of azimuthal angles $\Delta\phi$ (\textit{left}) and transverse momenta $\Delta p_T$ (\textit{right}) for local, independent and correlated emission.
The results are obtained using the Monte Carlo technique with $10^7$ events generated.
The distributions of the pairs in $\Delta \phi$ significantly differ for local, independent and correlated emissions. The differences are smaller in the case of the transverse momentum difference.
The flat distribution in $\Delta\phi$ for the independent emission  is independent of the flow and random momentum contributions modelling. The distributions in $\Delta\phi$ decrease monotonically for the local and correlated emission from $\Delta \phi = 0$ to $\Delta \phi = \pi$, but the quantitative properties of this qualitative behaviour depend on model details. Nonetheless, the effect of correlation at $\Delta \phi \approx 0$ is remarkably different compared to the back-to-back correlations expected for charm pair production in hard parton scatterings~\cite{Zhu:2006er,Cao:2015cba}. Thus, we expect experimental data will allow discrimination between these two different kinds of correlation. 


\begin{figure}[ht]

\includegraphics[width=0.49\textwidth]{plots/CorrDeltaPhiT150MeV-lines.pdf} 
\includegraphics[width=0.49\textwidth]{plots/CorrDeltaPtT150MeV-lines-maxPt2GeV.pdf} 
\includegraphics[width=0.49\textwidth]{plots/CorrDeltaPhiT150MeV-ratio.pdf} 
\includegraphics[width=0.49\textwidth]{plots/CorrDeltaPtT150MeV-ratio-maxPt2GeV.pdf} 
\caption{
The distribution of $c$-$\bar{c}$ pairs 
in the difference of azimuthal angles $\Delta\phi$ (\textit{left}) and transverse momenta $\Delta p_T$ (\textit{right}) for local, independent and correlated 
($\sigma = 2$~fm emission).
}
\label{fig:Sim:Corr}
\end{figure} 

It is clear that in the case of the $\Delta\phi$ distribution
rather limited data statistics (see the next section) should allow us to distinguish between predictions obtained assuming different space correlations between the emitted charm hadrons and different production mechanisms of charm quarks.
Encouraged by this conclusion,
we turn to the standard approach to heavy-ion collisions~\cite{Florkowski:Phenomenology}  and, within it, discuss the implications of different possible outcomes of the experimental measurements.
%%%%%%%

The approach pictures heavy-ion collisions at high energies as a time sequence of the following stages:
\begin{enumerate}[(1)]
\item 
\textit{Initial stage} - a high-density quark-gluon plasma is created.
QCD is assumed to be a valid theory. Charm-anti-charm quark pairs are produced locally and in a limited number because of the high energy threshold.
\item
\textit{Expansion stage} - the plasma expands~\cite{Landau:1953gs}, reaching the hadronisation temperature $T_H \approx 150$~MeV. The pair  of (anti-)charm quarks thermalise with the medium and flow. 
\item
\textit{Hadronisation stage} - the plasma, including the $c$ and $\bar{c}$  quarks, is converted to hadrons and resonances following the statistical rules~\cite{Hagedorn:1980kb, Becattini:2005xt} applied in the rest frame of a plasma fluid element. Thus, the flow and hadronisation (local statistical process) contributions give the momenta of charm hadrons. 
\item
\textit{Free-streaming stage} - resonances decay, and non-interacting hadrons freely stream in the vacuum to a detector.
\end{enumerate}

Many additional details, conceptual and quantitative, can be added~\cite{Florkowski:Phenomenology}, about the hydrodynamic evolution or the rescattering after hadronisation. The stages listed above are the most relevant to this paper aiming for a qualitative discussion of the correlations, the feasibility of their measurements and their phenomenological implications. 

Within the standard heavy-ion approach,
\begin{enumerate}[(A)]
\item
The experimental data consistent with the local emission would imply a small
space separation of $c$- and $\bar{c}$-quarks during the expansion stage.
This should be confronted with the charm-quark spatial diffusion calculated using the QCD-base approaches; for recent examples, see Refs.~\cite{Sambataro:2020pge,Capellino:2022nvf,Satapathy:2022xdw}.
\item
The experimental data consistent with the independent emission would imply a large spatial diffusion of the charm quarks in the plasma. 
Ultimately, for more accurate models, because of the limited expansion time, the experimental results may even be inconsistent with a local production coupled to a semi-classical transport (hence not faster than speed-of-light, 1). This might imply non-local effects in the expansion. 

For the historical record, this paper was motivated by the non-local, indeterministic toy model~\cite{Gazdzicki:2022zej} requiring the \textit{teleportation} transitions in its most symmetric version.
\item
The data consistent with the correlated emission would give a sensitive input for restricting the charm-quark spatial diffusion in the plasma.

\item
It is always wise to leave a door open to the unexpected. The approximations used to compute the spectra  might fail to describe the correlations, and the experimental results could qualitatively disagree with the expectations.  

\end{enumerate}


%%%%%%%%%%%%%%%%%%%%%%%%%%%%%%%
\vspace{1cm}
\section{Required statistics of Pb+Pb collisions for testing the model predictions}
\label{sec:experiment}

In this section, we discuss the feasibility of performing the required measurements of correlations between charm and anti-charm hadrons produced in head-on heavy-ion collisions. The important physics condition is a mean multiplicity of charm being small enough to neglect the production of two and more pairs of charm hadrons. This requirement implies the measurements at relatively small collision energies, probably close to the top SPS energy of $\sqrt{s_{NN}} \approx 20$~GeV. It also suggests collecting data in the fixed target mode, which due to
the Lorentz boost of the center-of-mass, allows for high detection acceptance and efficiency. As shown below,
if the charm yield is significantly lower than one or the reconstruction of open-charm hadrons is too inefficient, the event statistics needed to perform the measurement may be well beyond the capabilities of the now-a-day experiments.

For now, we only consider an analysis of the most abundant open-charm hadrons, namely, $D^0$ and $\bar{D^0}$ mesons. As shown later, this should 
be sufficient to measure the correlations with sufficient precision.
The required event statistics can be derived from the average number of reconstructed $D^0\bar{D^0}$-pairs per event, $\langle D^0 \bar{D^0} \rangle_{\textnormal{rec}}$. 

We perform our feasibility study assuming detector setup and performance similar to \NASixtyOne\ experiment at CERN collecting Pb+Pb collisions at $\sqrt{s_{NN}} = 17.3$~GeV. Assuming that processes that impact the reconstruction of a $D^0$ and a $\bar{D^0}$ mesons within an event are approximately uncorrelated, we estimate the average number of reconstructed pairs as
\begin{equation} \label{eq:avgD0D0yield}
    \langle D^0\bar{D^0} \rangle_{rec} \approx
        \langle c\bar{c} \rangle \cdot
        \left(
            P(c \rightarrow D^0) \cdot
            \textnormal{BR}(D^0 \rightarrow K \pi) \cdot
            P(\textnormal{acc}) \cdot
            P(\textnormal{sel}) \cdot
            P(\textnormal{rec})
        \right)^2,
\end{equation}
where $\langle c\bar{c} \rangle$ is the average number of $c\bar{c}$-pairs per event. The $P(c \rightarrow D^0)$ =  0.31 is a probability for $c$-quark to hadronize into the $D^0$ meson evaluated within the PHSD model \cite{Cassing_2009}, BR$(D^0 \rightarrow K^+ \pi^-)$ =  3.98\% is a branching ratio of decay channel used in the measurements~\cite{Workman:2022ynf}, $P(\textnormal{acc})$ = 0.5 is a probability for $D^0$ to be within an acceptance region of the detector, $P(\textnormal{sel})$ = 0.2 is a probability for $D^0$ to pass background-suppressing selection of charm meson candidates, and $P(\textnormal{rec})$ = 0.9 is a probability of reconstructing the meson.
The value of $P(\textnormal{acc})$ was evaluated using the \GeantFour simulation with the detector setup for November 2022, $P(\textnormal{sel})$ is taken from the pilot analysis of $D^0$ and $\bar{D^0}$ production~\cite{Merzlaya:2771816}, and $P(\textnormal{rec})$ was obtained from a \GeantFour simulation with the setup for November 2022 and reconstruction software used for previous open charm analysis using 2017 and 2018 data ~\cite{mbajda, Merzlaya:2771816}.



Finally, given $\langle D^0\bar{D^0} \rangle_{rec}$, an estimate of the required event statistics can be obtained via
\begin{equation} \label{eq:nevents}
    \{\textnormal{number of head-on events to collect}\} \approx
        \frac{\{\textnormal{number of $D^0\bar{D^0}$ pairs to reconstruct}\}}
             {\langle D^0\bar{D^0} \rangle_{rec}}.
\end{equation}


The value of $\langle c\bar{c} \rangle$ is neither reliably predicted by models nor measured by experiments. However, considering available estimates~\cite{Snoch:2018nnj}, we expect that the value of $\langle c\bar{c} \rangle$ for head-on Pb+Pb at $\sqrt{s_{NN}} \approx 17$~GeV should range from  0.1 up to 1. 

Putting all together, estimates on the run time needed to collect 1000 $D^0\bar{D^0}$-pairs for different event rates of the updated 
\NASixtyOne experiment and for different values of $\langle c\bar{c} \rangle$ are given in Table~\ref{tab:times}. Figure~\ref{fig:dphi-1000} demonstrates the statistical precision of a signal from 1000 $D^0\bar{D^0}$-pairs assuming that the
statistical fluctuations of background pairs can be neglected.


\begin{table}[h]
\begin{tabular}{|c|c|c|c|c|}
    \hline
              & $\langle c\bar{c} \rangle = 0.1$         & $\langle c\bar{c} \rangle = 0.2$          & $\langle c\bar{c} \rangle = 0.5$     & $\langle c\bar{c} \rangle = 1$           \\ \hline
    1 kHz     & $1000$ days & $500$ days   & $200$ days  & $100$ days  \\ \hline
    10 kHz    & $100$ days  & $50$ days    & $20$ days   & $10$ days   \\ \hline
    100 kHz   & $10$ days   & $5$ days     & $2$ days    & $1$ day     \\ \hline
            \hline
    $N_{pair}/N_{comb}$ & 91\% & 83\%  & 66\%             & 50\%             \\ \hline
\end{tabular}
\caption{\label{tab:times} Estimate of the duration of a data-taking period needed to collect 1000 $D^0\bar{D^0}$-pairs (first three rows). 
In the calculations, the duty cycle of 30\% was assumed.
The last row shows the ratio of the produced pairs of $c\bar{c}$ quarks to all combinations of them.   
}
\end{table}


\begin{figure}[ht] \label{fig:dphi-1000}
    \includegraphics[width=0.49\textwidth]{plots/Stat-projections-AllCases.pdf}
    \caption{The projection for statistical precision of measurement of the azimuthal correlation $\Delta \phi$ assuming the experiment registered $N = 1000~~ D^0\bar{D^0}$ pairs. The local, independent, and correlated emission is assumed. }
\label{fig:Sim:Corr2}
\end{figure}

A typical ion beam period at CERN is about four weeks.
Entries in Table~\ref{tab:times} with a data-taking time of 100 days or more correspond to scenarios where the measurement may take longer than a period between
the CERN accelerators' long shutdowns. Moreover, at the moment, the event rate of 100~kHz would require a significant update of the \NASixtyOne detector and its beamline. However, a setup corresponding to 10~kHz should be achievable within the nearest years. Thus
we find that for $\langle c\bar{c} \rangle > 0.2$, it should be possible to perform the measurements of $c\bar{c}$-correlations by \NASixtyOne in the CERN Run~4 period (2028-2032).
The additional possibility for the experimental study would be constructing a new
experiment optimized for charm measurements. The corresponding letter of intent was recently submitted to the CERN SPSC~\cite{Ahdida:2845241}. 
  

The discussed measurement of correlations between $c$ and $\bar{c}$ is only meaningful for the quarks produced as a pair. However, if multiple pairs of $c\bar{c}$ quarks were produced within the same event, we will observe an unavoidable background due to combining $c$- and $\bar{c}$-hadrons originating from different pairs. Quantifying this background suffices taking a ratio between  a multiplicity  of produced $c\bar{c}$ pairs in the event to a number of all possible combinations of $c$- and $\bar{c}$-hadrons that could be observed in the event. To compute this ratio for different values of $\langle c\bar{c} \rangle$, it was assumed that $c\bar{c}$-multiplicity follows a Poisson distribution, parameterized by the given $\langle c\bar{c} \rangle$. This yields a probability of having more than one $c\bar{c}$ pair within the same event, in which case one has unwanted combinations of $c$- and $\bar{c}$-hadrons. Obtained values of the ratio pairs to combinations are given in Table~\ref{tab:times}. Values around $50\%$ indicate that about half of the combinations of $c$- and $\bar{c}$-hadrons are unwanted. For the real-world analysis, background due to the misidentification of open-charm hadrons would likely be significant. Moreover, lower values of $\langle c\bar{c} \rangle$ also imply higher requirements for the event statistics. It suggests that $\langle c\bar{c} \rangle$ should be cautiously picked for reasonable analysis. Realistic estimates of $\langle c\bar{c} \rangle$ for head-on Pb+Pb collisions at the top CERN SPS energies range between $0.1$ and $1$.
This further supports the conclusion that the measurements at the CERN SPS should 
considered.

\clearpage

\section{Summary}
\label{sec:summary}

In this work, we propose to study momentum correlations between pairs of $c$- and $\bar{c}$-hadrons produced in heavy-ion collisions at low collision energies. 
We argue that the correlations are sensitive to the four-velocity of the fluid cells from which charm hadrons are emitted. This relates the momentum correlations of charm hadrons measured in an experiment to the spatial correlations of the charm hadron emission points. The latter depends on production locality and spatial diffusion of charm at high energy densities. The obtained predictions for azimuthal angle correlations for local, independent, and correlated emission of charm hadrons differ 
significantly.

Since the emission of multiple, uncorrelated, pairs of $c$- and $\bar{c}$-hadrons
from a single collision would spoil the wanted correlations, it is mandatory to perform the measurements at sufficiently-low collision energies granting a low production probability of multiple-charm pairs. 
The proposed method can also be used for bottom hadrons.

As a quantitative example, we consider charm hadron measurements in head-on Pb+Pb collisions at the CERN SPS. Assuming typical values of data-taking parameters for the NA61/SHINE experiment at SPS, we show that the required measurements would need
a data-taking rate of 10k~Hz or more. These rates are easily allowed by the current detector technologies. Thus the corresponding measurements should be possible by the upgraded \NASixtyOne and the new NA60++ experiments after the CERN LS3 upgrade period.


\begin{acknowledgments} 
\itshape{
This work is partially supported by
the Polish National Science Centre grant 2018/30/A/ST2/00226, the National Science Centre, Poland, grant no. 2018/30/E/ST2/00089, the German Research Foundation grant GA1480\slash 8-1, by the Polish National Science Centre grant 2020/39/D/ST2/02054.
}

\end{acknowledgments}
\clearpage


\newpage
\bibliographystyle{utphys}
\bibliography{references}

\end{document}
