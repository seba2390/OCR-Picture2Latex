\subsection{Mass Estimation}
\begin{figure*}
\begin{center}
    \includegraphics[width=0.9\textwidth]{newplots/NFWlambda4MassBias.pdf}
     \caption{
     NFW halo Mass Bias for $M=10^{14.8}, 10^{15.0}, 10^{15.2},$ and $
     10^{15.4}$   $M_{\odot}$ respectively, reconstructed using $\lambda=4$.
     The darker grey area indicate a $5\%$ bias and the lighter grey area
     indicate a $20\%$ mass bias. The error bar indicate
    the standard deviation of reconstructed mass with respect to $\frac{a}{c}$ over the range
    $[0.5,1]$.
     }
     \label{NoisyMassBiasNFWlambda4}
\end{center}

\end{figure*}

\begin{figure*}
\begin{center}
\includegraphics[width=0.9\textwidth]{newplots/CUSPYlambda4MassBias.pdf}
\caption{
    Cuspy NFW halo Mass Bias for $M=10^{14.8}, 10^{15.0}, 10^{15.2},$ and $
    10^{15.4}$ $M_{\odot}$ respectively, reconstructed using $\lambda=4$. The
    darker grey area indicate a $5\%$ bias and the lighter grey area indicate a
    $20\%$ mass bias. The error bar indicate
    the standard deviation of reconstructed mass with respect to $\frac{a}{c}$ over the range
    $[0.5,1]$.
    }
    \label{NoisyMassBiasCuspylambda4}
\end{center}
\end{figure*}

Figs.~\ref{NoisyMassBiasNFWlambda4} and \ref{NoisyMassBiasCuspylambda4} show
the mass estimation of halos of masses $10^{14.8}, 10^{15.0}, 10^{15.2}$ and
$10^{15.4}$ $M_{\odot}$, reconstructed using $\lambda=4$. We observe for the $10^{14.8}M_\odot$ halos, while detection at lower redshift with a big $\lambda$ yields $\lessapprox 5\%$ mass
estimation bias, the
performance of \splinv{} decreases drastically as redshift of the halo
increases. With a larger $\lambda$, we see that the mass estimation for larger
mass improves, with performance of reconstructing NFW halos better than that of
cuspy NFW halos.

\subsection{Redshift Estimation}
For redshift estimations, we see a pretty similar result as in Sect. \ref{sec:oneHalo_noisy_z}, where the redshift estimation for halo of masses $10^{15.0}, 10^{15.2}$ and $10^{15.4}$ have consistently less than $5\%$ bias with $z>0.0625$. However, redshift estimation for halo with mass $10^{14.8} M_\odot$  for $z>0.3325$ shows above $40\%$ mass bias. This is probably due to the fact that, at this redshift level, halo with this mass are hard to detect with $\lambda=4$, causing we do not have enough data point to correct estimate the mass. 
\begin{figure*}
\begin{center}
\includegraphics[width=0.9\textwidth]{newplots/NFWlambda4Redshift.pdf}
\caption{
    NFW halo Redshift Estimation for $M= 10^{14.8}, 10^{15.0},
    10^{15.2}$ and $10^{15.4}$ $M_{\odot}$ respectively, reconstructed using
    $\lambda=2$. The darker grey area indicate a $5\%$ bias and the lighter
    grey area indicate a $20\%$ mass bias.
    }
    \label{NoisyRedshiftBiasNFWlambda4}
\end{center}
\end{figure*}

\begin{figure*}

\begin{center}
\includegraphics[width=0.9\textwidth]{newplots/CUSPYlambda4Redshift.pdf}
\caption{
    Cuspy NFW halo Redshift Estimation for $M= 10^{14.8}, 10^{15.0},
    10^{15.2}$ and $10^{15.4}$ $M_{\odot}$ respectively, reconstructed using
    $\lambda=2$. The darker grey area indicate a $5\%$ bias and the lighter
    grey area indicate a $20\%$ mass bias.
    }
    \label{NoisyRedshiftBiasCuspylambda4}
\end{center}
\end{figure*}


% \begin{figure*}
% \begin{center}
% \includegraphics[width=0.9\textwidth]{smooth0/Detectionparam_cuspy_lbd4.pdf}
% \caption{
%     Expected detection number for cuspy NFW halos in a Full sky survey.
%     }
%     \label{expected_detction_number cuspy lbd4}
% \end{center}
% \end{figure*}

% \begin{figure*}
% \begin{center}
% \includegraphics[width=0.9\textwidth]{smooth0/Detectionparam_nfw_lbd4.pdf}
% \caption{
%     Expected detection number for NFW halos in a Full sky survey.
%     }
%     \label{expected_detction_number nfw lbd4}
% \end{center}
% \end{figure*}
