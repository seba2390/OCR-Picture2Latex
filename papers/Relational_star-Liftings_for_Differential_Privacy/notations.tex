% --------------------------------------------------------------------
\usepackage[only,llbracket,rrbracket]{stmaryrd}
\SetSymbolFont{stmry}{bold}{U}{stmry}{m}{n}

\usepackage{amsmath,amsthm,amssymb,mathtools,nicefrac}
\usepackage[bb=boondox]{mathalfa}
\usepackage{xspace}
\usepackage{enumitem}
\usepackage{bbding}
\usepackage[sort&compress]{natbib}

\makeatletter
\def\NAT@spacechar{~}% NEW
\makeatother

% --------------------------------------------------------------------
\newcommand{\notation}[4][0]{\newcommand{#2}[#1]{#3}}

\newcommand{\rname}[1]{[\textsc{#1}]}

% --------------------------------------------------------------------
% TODOs

\usepackage{todonotes}
\usepackage{setspace}

\setlength{\marginparwidth}{3cm}

\newcounter{todocnt}
\newcommand{\todox}[3][]{%
\refstepcounter{todocnt}{%
\setstretch{0.7}%
\todo[color={red!100!green!33},size=\small,#1]{%
  \textbf{[\uppercase{#2}\thetodocnt]:}~#3}}}

\newcommand{\gb}[2][]{\todox[#1]{GB}{#2}}
\newcommand{\py}[2][]{\todox[#1]{PY}{#2}}
\newcommand{\jh}[2][]{\todox[#1]{JH}{#2}}
\newcommand{\te}[2][]{\todox[#1]{TE}{#2}}

% --------------------------------------------------------------------
% English

\def\ie{i.e.\xspace}
\def\eg{e.g.\xspace}

% --------------------------------------------------------------------
% Local labels

\newcounter{localc}

\newcommand*\lclabel[1]{\label{exer:\thelocalc:#1}}
\newcommand*\lcref[1]{\ref{exer:\thelocalc:#1}}
\newcommand*\lceqref[1]{\eqref{exer:\thelocalc:#1}}

\newcommand{\steplocal}{\stepcounter{localc}}

% --------------------------------------------------------------------
% Standard symbols

\newcommand{\eqdef}{\mathrel{\stackrel{\scriptscriptstyle \triangle}{=}}}
\newcommand{\iffdef}{\mathrel{\stackrel{\scriptscriptstyle \triangle}{\iff}}}

\newcommand{\inv}[1]{#1^{\raisebox{.2ex}{$\scriptscriptstyle-\!1$}}}

\newcommand{\proj}[1]{\pi_{#1}}
\newcommand{\fst}{\proj{1}}
\newcommand{\snd}{\proj{2}}

\newcommand{\rR}{\mathrel{\mathcal{R}}}
\newcommand{\rS}{\mathrel{\mathcal{S}}}

\newcommand{\oR}{\mathcal{R}}
\newcommand{\oS}{\mathcal{S}}

\newcommand{\simplies}{\mathrel{\Rightarrow}}

\newcommand{\true}{{\mathrel{\top}}}
\newcommand{\false}{{\mathrel{\perp}}}

\newcommand{\fun}[1]{\lambda\,{#1} .\,}
\newcommand{\card}[1]{{|{#1}|}}
\newcommand{\rcomp}[1]{\overline{#1}}

\newcommand{\relrestr}[3]{\mathrel{{#3}_{|_{{#1} \!\times\! {#2}}}}}

\newcommand{\qt}[2]{{#1}_{/_{\!#2}}}
\newcommand{\qtc}[2]{{[#1]}_{#2}}

\newcommand{\rrefl}[1]{\mathrel{#1^=}}
\newcommand{\rsym}[1]{\mathrel{\inv{#1}}}

\def\ssrc{\mathop{\top}}
\def\sdst{\mathop{\perp}}

\newcommand{\cset}[1]{\mathcal{E}(#1)}

% --------------------------------------------------------------------
% \bigtimes
\DeclareFontFamily{U}{mathx}{\hyphenchar\font45}
\DeclareFontShape{U}{mathx}{m}{n}{
      <5> <6> <7> <8> <9> <10>
      <10.95> <12> <14.4> <17.28> <20.74> <24.88>
      mathx10
      }{}
\DeclareSymbolFont{mathx}{U}{mathx}{m}{n}
\DeclareMathSymbol{\bigtimes}{1}{mathx}{"91}

% --------------------------------------------------------------------
% Standard sets

\newcommand{\rset}[1]{\ensuremath{\mathbb{#1}}}

\newcommand{\BB}{{\{0, 1\}}}
\newcommand{\NN}{\rset{N}}
\newcommand{\ZZ}{\rset{Z}}
\newcommand{\QQ}{\rset{Q}}
\newcommand{\RR}{\rset{R}}
\newcommand{\RRP}{\rset{R}^+}

\renewcommand{\setminus}{\mathrel{-}}

\newcommand{\I}[1]{\mathcal{I}_{#1}}

\newcommand{\intv}[2]{{[{#1}, {#2}]}}
\newcommand{\indicator}[1]{\raisebox{.25ex}{$\chi_{#1}$}}

% ---------------------------------------------------------------------
% Distributions

\newcommand{\DistOp}{{\mathbb{D}}}
\newcommand{\FDistOp}{{\DistOp^{\scriptscriptstyle =1}}}
\newcommand{\Exp}{\mathbb{E}}
\newcommand{\ExpD}{\mathbb{E}}
\newcommand{\PrS}{\mathbb{P}}

\newcommand{\Dist}{\DistOp}
\newcommand{\FDist}{\FDistOp}

\newcommand{\dnull}[1][]{{{\mathbb{0}}^{#1}}}
\newcommand{\dunit}[2][]{{{\mathbb{1}}^{#1}_{#2}}}
\newcommand{\dnormed}[1]{{#1}^{=1}}
\newcommand{\dlet}[3]{\ExpD_{{#1} \sim {#2}} [{#3}]}
\newcommand{\dslet}[2]{\ExpD_{#1} [{#2}]}
\newcommand{\dclet}[4]{\dlet {#1} {\drestr {#2} {#3}} {#4}}
\newcommand{\dsclet}[3]{\dslet {\drestr {#1} {#2}} {#3}}
\newcommand{\dmargin}[2]{{#1} \circ \inv{#2}}
\newcommand{\dlift}[1]{#1^{\sharp}}
\newcommand{\mlift}[1]{{\overline{#1}}} %FIXME
\newcommand{\dproj}[1]{\dlift{\proj{#1}}}
\newcommand{\dfst}{\dproj{1}}
\newcommand{\dsnd}{\dproj{2}}
\newcommand{\drestr}[2]{{#1}_{|{#2}}}
\newcommand{\rdrestr}[2]{\drestr {#2} {#1}}
\newcommand{\dlim}[2]{\lim_{{#1}\infty}\,#2}
\newcommand{\dprod}{\mathbin{\star}}
\newcommand{\dprodeq}{\mathbin{\overline{\star}}}
\newcommand{\djoin}[2]{{#1} \mathbin{\Join} {#2}}
\newcommand{\dswap}[1]{#1^{\leftrightarrow}}
\newcommand{\dleq}{\mathrel{\preceq}}
\newcommand{\dgeq}{\mathrel{\succeq}}
\newcommand{\dlt}{\mathrel{\prec}}
\newcommand{\dgt}{\mathrel{\succ}}
\newcommand{\lap}[1]{\mathcal{L}_{#1}}

\newcommand{\E}[3]{\Exp_{{#1} \sim {#2}} [{#3}]}
\newcommand{\cE}[4]{\Exp_{{#1} \sim {#2}} [{#3} \mid {#4}]}
\newcommand{\sE}[2]{\Exp_{{#1}} [{#2}]}

\renewcommand{\P}[3]{\PrS_{{#1} \sim {#2}} [{#3}]}
\newcommand{\sP}[2]{\PrS_{#1} [{#2}]}
\newcommand{\iP}[1]{\PrS [{#1}]}

\newcommand{\mass}[1]{|{#1}|}
\DeclareMathOperator{\supp}{supp}

\newcommand{\wtn}[2]{\langle {#1, #2} \rangle}

\newcommand{\dcoupled}[3]
  {{#1} \mathrel{\blacktriangleleft} \langle {#2} \mathrel{\&} {#3} \rangle}

\newcommand{\dlifted}[4]
  {{#1} \mathrel{\blacktriangleleft_{#2}} \langle {#3} \mathrel{\&} {#4} \rangle}

\newcommand{\dacoupled}[6]
  {\langle {#1, #2} \rangle \mathrel{\blacktriangleleft}_{#3,#4}
     \langle {#5} \mathrel{\&} {#6} \rangle}

\newcommand{\dalifted}[7]
  {\langle {#1, #2} \rangle \mathrel{\blacktriangleleft_{#3,#4}^{#5}}
     \langle {#6} \mathrel{\&} {#7} \rangle}

% --------------------------------------------------------------------
% Assertions

\newcommand{\pre}{\phi}
\newcommand{\post}{\psi}

% --------------------------------------------------------------------
% Sid'ed

\def\ms{\mspace{-1.5mu}}

\newcommand{\lmark}{{\scriptscriptstyle \vartriangleleft}}
\newcommand{\rmark}{{\scriptscriptstyle \vartriangleright}}
\newcommand{\pmark}{{\scriptscriptstyle \Join}}

\newcommand{\lside}{_{\ms\lmark}}
\newcommand{\rside}{_{\ms\rmark}}

\newcommand{\LPVars}[1][]{\PVars[#1]^{\ms\lmark}}
\newcommand{\RPVars}[1][]{\PVars[#1]^{\ms\rmark}}
\newcommand{\PPVars}[1][]{\PVars[#1]^{\ms\pmark}}

\newcommand{\LSExpr}[1][]{\SExpr[#1]^{\ms\lmark}}
\newcommand{\RSExpr}[1][]{\SExpr[#1]^{\ms\rmark}}
\newcommand{\PSExpr}[1][]{\SExpr[#1]^{\ms\pmark}}

% --------------------------------------------------------------------
% Semantics

\newcommand{\sem}[1]{{\llbracket {#1} \rrbracket}}
\newcommand{\semop}[1]{{\overline{#1}}}

% --------------------------------------------------------------------
% Logics

\newcommand{\lift}[1]{\mathrel{#1^\sharp}}
\newcommand{\alifttoplas}[2]{\mathrel{#1^{(1)}_{#2}}}
\newcommand{\alifticalp}[2]{\mathrel{#1^{(2)}_{#2}}}
\newcommand{\aliftnew}[2]{\mathrel{#1^{(\star)}_{#2}}}

\newcommand{\symlifttoplas}[2]{\mathrel{\overline{#1}^{(1)}_{#2}}}
\newcommand{\symliftnew}[2]{\mathrel{\overline{#1}^{(\star)}_{#2}}}

% --------------------------------------------------------------------
% Refs

% \iflipics
% \theoremstyle{plain}
% \newtheorem{conj}[theorem]{Conjecture}
% \theoremstyle{definition}
% \newtheorem{defprop}[theorem]{Defn.-Lemma}
% \newtheorem{nottion}[theorem]{Notation}
% \else
% \theoremstyle{plain}
% \newtheorem{theorem}{Theorem}
% \newtheorem{lemma}[theorem]{Lemma}
% \newtheorem{corollary}[theorem]{Corollary}
% \newtheorem{conj}[theorem]{Conjecture}
% \theoremstyle{definition}
% \newtheorem{definition}[theorem]{Definition}
% \newtheorem{defprop}[theorem]{Defn.-Lemma}
% \newtheorem{example}[theorem]{Example}
% \newtheorem{nottion}[theorem]{Notation}
% \theoremstyle{remark}
% \newtheorem*{remark}{Remark}
% \fi

\usepackage{hyperref}

\iflipics
\else
\usepackage[capitalise]{cleveref}

\crefname{section}{section}{sections}
\Crefname{section}{Section}{Sections}

\crefname{lemma}{lemma}{lemmas}
\Crefname{lemma}{Lemma}{Lemmas}

\crefname{theorem}{theorem}{theorems}
\Crefname{theorem}{Theorem}{Theorems}

\crefname{definition}{definition}{definitions}
\Crefname{definition}{Definition}{Definitions}
\fi

\renewcommand{\epsilon}{\varepsilon}

%%% Local Variables:
%%% mode: latex
%%% TeX-master: "main"
%%% End:
