We now introduce the hexapod microrobot considered in this paper.
This robot is of particular interest due to the unique challenges that arise when attempting traditional gait design techniques.
The micro-scale of the walker makes it very challenging to obtain an accurate dynamics model.
Moreover, the robot is subject to wear-and-tear, and therefore any learning approach employed must be capable of learning gaits within a limited number of trials.

\subsection{Physical Description}    
    %
	\begin{figure}[t]
	  \centering
	  \includegraphics[width=0.96\linewidth]{fig/leg_diagram.pdf}
	  \caption{Diagram of the robot leg showing the actuation sequence (active motors are shown in red). Each leg has 2 motors, each one independently actuating a single DOF. 
      }
	  \label{fig:leg_diagram}
	%   \vspace{-8pt}
	\end{figure}
	% 
	
	The hexapod microrobot is based on silicon microelectromechanical systems (MEMS) technology. 
	The robot's legs are made using linear motors actuating planar pin-joint linkages~\citep{Contreras2016DurabilityOS}. 
	A tethered single-legged walking robot was previously demonstrated using this technology~\citep{contreras_first_2017}. 
	The hexapodal robot is assembled using three chips.  
	The two chips on the side each have 3 of the leg assemblies, granting six 2 degree-of-freedom (DOF) legs for the whole robot. 
	The top chip acts to hold the leg chips together for support, and to route the signals for off-board power and control.
	Overall, the robot measures \SI{13}{\milli\meter} long by \SI{9.6}{\milli\meter} wide and stands at \SI{8}{\milli\meter} tall with an overall weight of approximately \SI{200}{\milli\gram}. 
 
\subsection{Actuation}
	Each of the robot's legs has 2-DOF in the plane of fabrication, as shown in \fig{fig:leg_diagram}.
	Both DOFs are actuated, thus the leg has 2 motors, one to actuate the vertical DOF to lift the robot's body and a second to actuate the horizontal DOF for the vertical stride. 
	The actuators used for the legs are electrostatic gap-closing inchworm motors~\cite{penskiy_optimized_2013}.
	During a full cycle, each leg moves \SI{0.6}{\milli\meter} vertically with a horizontal stride of \SI{2}{\milli\meter}. 
	For more details on the actuation mechanism used on our microrobot, we refer readers to ~\cite{Contreras2017DynamicsOE}.
	
\subsection{Simulator}
	%
	\begin{wrapfigure}{r}{0.46\linewidth} 
	  \centering
	  \vspace{-10pt}
	  \includegraphics[width=0.98\linewidth]{fig/simulator.pdf}
	  \caption{The simulated micro walker.}
	  \label{fig:vrep}
	  \vspace{-8pt}
	\end{wrapfigure}
	% 
	In our experimental simulations, we used the robotics simulator V-REP~\citep{vrep} for constructing a scaled-up simulated model of the physical microrobot (see \fig{fig:vrep}).
	Since V-REP was not designed with simulation of microrobots in mind, it was not capable of simulating the dynamics of the leg joints accurately and would produce wildly unstable models at the desired scale.
	We chose to scale up the size of the robot in simulation by a factor of $100$ in order to account for the issues with scaling in simulation (all the experimental results are re-normalized to the dimensions of the real robot).
	We believe that this re-scaling still allows meaningful results to be produced for several reasons.
	First, the experiments performed in this paper are meant to demonstrate the validity of the proposed controller, and the learning approach for training an actual physical microrobot.
	The policies trained are not meant to work on the real robot without any re-tuning or modification.
	Second, the simulator still allows to test the basic motion patterns we want to implement on the microrobot.
	Finally, our contribution lends credibility to the potential application of Bayesian-inspired optimization methods to a setting where evaluations can be costly and time consuming.
