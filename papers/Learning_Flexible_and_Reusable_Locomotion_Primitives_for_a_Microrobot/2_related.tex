% Microrobots
There has been an abundance of work published on the design and development of walking~\citep{Ambroggi1997} and flying millimeter-scale microrobots~\citep{4457871,8001934, Kilberg2017MEMSAC}.
Much of this work focuses on hardware considerations such as the design of micro-sized joints and actuators rather than control.
To our knowledge, no previous work has implemented a CPG-based controller for on-board control of a walking microrobot, nor has learning been used for locomotion on a microrobots.

% Hexapod locomotion
While hexapod gaits have been thoroughly studied and tested~\citep{Altendorfer2001,Hoover2008}, much of the work did not easily transfer to our microrobot due to the drastically different leg dynamics. 
Most hexapods make use of rotational joints with higher DOF while our walker uses only two prismatic spring joints per leg, resulting in less control and unique constraints on leg retraction and actuation. 

% Optimization for locomotion
While sufficient for simple controllers with few parameters, manually tuning controller parameters can require an immense amount of domain expertise and time.
As such, automatic gait optimization is an important research field that has been studied with a wide variety of approaches in both the single-objective~\citep{Tedrake2004,Chernova2004,Niehaus2007,Lizotte2007,Tesch2011,Oliveira2011,Oliveira2013,Calandra2015a} and multi-objective setting~\citep{Capi2005,Oliveira2011,Oliveira2013,Tesch2013}.
%
Evolutionary algorithms have been successfully used to train quadrupedal robots~\citep{Chernova2004,Oliveira2011}, but this approach often requires thousands of experiments before producing good results, which is unfeasible on fragile microrobots.

% Bayesian optimization in locomotion 
A more data-efficient approach used before to learn gaits for snake and bipedal robots is Bayesian optimization~\citep{Lizotte2007,Tesch2011,Calandra2015a,Antonova2017}. 
Bayesian optimization has been applied to contextual policy search in the context of robot manipulation~\citep{Metzen2015}.
Our contribution builds off of this work by applying and extending the contextual framework to learning movement trajectories and path planning.
%
% Multi-objective optimization and BO
Another extension of Bayesian optimization related to our work is Multi-objective Bayesian optimization, which has also been previously applied in the context of robotic locomotion~\cite{Tesch2013}.
However, past work is only concerned with using multi-objective optimization to balance the trade-off between various competing goals.
Our main contribution demonstrates an entirely novel application of multi-objective optimization to learning motor primitives that does not involve the trade-off between various goals, but instead uses a multi-objective model to learn over an area of possible trajectories for path planning.
