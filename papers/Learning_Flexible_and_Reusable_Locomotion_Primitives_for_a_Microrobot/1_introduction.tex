\IEEEPARstart{S}{ubstantial} progress has been made in recent years towards the development of fully autonomous microrobots~\citep{saito_miniaturized_2016,Vogtmann2017}.
However, gait design for robot locomotion at the sub-centimeter scale is not a well-studied problem.
Completing more complicated locomotion tasks like navigating complex environments is even more challenging.
These issues become exacerbated when dealing with legged locomotion, where even walking straight is still an active area of study for normal-sized robots.
In this paper, we present a novel approach for the autonomous optimization of locomotion primitives and gaits.

While locomotion on larger-scale robots has been thoroughly investigated, transferring many of these proven approaches to the millimeter scale poses many unique challenges.
One such obstacle is the lack of access to sufficiently accurate simulated models at the millimeter scale.
Even simulation environments designed to simulate dynamics at this scale are generally unequipped for usage in robotics contexts.
Additionally, working with microrobots can place severe limitations on the number of iterations as trials become much more time-consuming and expensive to run.

While microrobot locomotion has been addressed in the past, much of the work is primarily concerned with the mechanical design and manufacturing of microrobots.
Accomplishing more sophisticated locomotion tasks on the sub-centimeter scale remains an open area for research.
Analytical implementations of various gait behaviors have worked on microrobots~\citep{ebefors_walking_1999,hollar_solar_2003}, but these solutions can become unwieldy for robots with higher DOF such as legged walkers (\eg, our micro-hexapod).
Data-driven automatic gait optimization is a viable alternative to analytical gait design and optimization, but using these techniques can be challenging due to the high number of trials that might be necessary to perform in order to learn viable gaits.

%
\begin{figure}[t]
  \centering
  \includegraphics[width=0.94\linewidth]{fig/flik_cad_assembled.pdf}
  \caption{The six-legged micro walker considered in our study as a CAD model (left) and an assembled prototype (right).}
  \label{fig:flik_cad}
 \vspace{-10pt}
\end{figure}
% 

Our contributions are two-fold:
1) we validate the use of both CPG controllers and Bayesian optimization for microrobots on a wide range of single and multi-objective locomotion tasks.
2) we introduce a novel approach to efficiently learn gaits and motor primitives from scratch without the need for prior knowledge (\eg, a dynamics model).
This is accomplished by collecting data on various motor primitives using contextual policy search and using those evaluations to reformulate the problem into a multi-objective optimization task, providing us a model that can map any set of parameters to a predicted trajectory.
Using this model, we can optimize our parameters on various trajectories for subsequent use in path planning.
This approach is not tied exclusively to microrobots, but can be used for any walking robot.

To evaluate our approach, we used a simulated hexapod microrobot modeled after the recently developed microrobot~\citep{contreras_first_2017} shown in \fig{fig:flik_cad}. 
We first validate the use of a CPG controller on our microrobot to reduce the number of parameters during optimization.
Then, we validate the use of Bayesian optimization and existing techniques on a curriculum of progressively more difficult tasks including learning single-objective, contextual, and multi-objective gaits.
As a proof of concept, we evaluated our approach by learning motor primitives from 250 trials and subsequently using them to successfully navigate through a maze.
