%========================================================
% GENERAL
%--------------------------------------------------------

% \newcommand{\citet}[1]{\cite{#1}}
% \newcommand{\citep}[1]{\cite{#1}}

\newcommand{\ie}[0]{i.e.}
\newcommand{\eg}[0]{e.g.}
\newcommand{\wrt}[0]{w.r.t.}
\newcommand{\apriori}[0]{a priori}
\newcommand{\aposteriori}[0]{a posteriori}

\newcommand{\email}[1]{\href{mailto:#1}{\nolinkurl{#1}}}
\newcommand{\mailto}[1]{\href{mailto:#1}{#1}}
\newcommand{\link}[1]{\colora{\url{#1}}}

\renewcommand{\sec}[1]{Section~\ref{#1}}
\newcommand{\fig}[1]{Figure~\ref{#1}}
\newcommand{\eq}[1]{Equation~\eqref{#1}}
\newcommand{\app}[1]{Appendix~\ref{#1}}
\newcommand{\tab}[1]{Table~\ref{#1}}
\newcommand{\alg}[1]{Algorithm~\ref{#1}}
\newcommand{\algline}[1]{Line~\ref{#1}}
\newcommand{\chp}[1]{Chapter~\ref{#1}}

\newcommand{\subfig}[1]{(\subref{#1})}

\newcommand{\newword}[1]{\textbf{#1}}

%\newindex{todo}{tod}{tnd}{TODO List} % start todo list
%\newindex{fixme}{fix}{fnd}{FIXME List} % start fixme list
%\newcommand{\todo}[1]{\textcolor{blue}{TODO: #1}\index[todo]{#1}} % macro for todo entries
%\newcommand{\fixme}[1]{#1}
%\DeclareOption{mydraft}{\renewcommand{\fixme}[1]{\textcolor{red}{\textbf{#1}}}}
%\ProcessOptions

\newcommand{\fixme}[1]{\textcolor{red}{#1}}
%\DeclareOption{mydraft}{\renewcommand{\fixme}[1]{\textcolor{red}{#1}\index[fixme]{#1}}} 		% macro for fixme entries
\DeclareOption{mydraft}{\providecommand{\fixme}[1]{\textcolor{red}{#1}}} 		% macro for fixme entries
\ProcessOptions

% \newcommand{\comment}[1]{#1}
% %\DeclareOption{mydraft}{\ifdef{\comment}{\renewcommand{\comment}[1]{\textcolor{blue}{\textbf{[NdR: #1]}}}}{\newcommand{\comment}[1]{\textcolor{blue}{\textbf{[NdR: #1]}}}}}
% \DeclareOption{mydraft}{\providecommand{\comment}[1]{\textcolor{blue}{\textbf{[NdR: #1]}}}}		% macro for comment

\ProcessOptions

% Colors used in Matlab plots
\definecolor{matlab1}{rgb}{0,0,1}
\definecolor{matlab2}{rgb}{0,0.5,0}
\definecolor{matlab3}{rgb}{1,0,0}
\definecolor{matlab4}{rgb}{0,0.75,0.75}
\definecolor{matlab5}{rgb}{0.75,0,0.75}
\definecolor{matlab6}{rgb}{0.75,0.75,0}
\definecolor{matlab7}{rgb}{0.25,0.25,0.25}


\definecolor{darkgreen}{rgb}{0,0.5,0}		%Olivegreen?
\definecolor{purple}{rgb}{0.75,0,0.75}
\definecolor{pink}{rgb}{1,0.4,0.6}


\newcommand{\specialcell}[2][c]{%
  \begin{tabular}[#1]{@{}c@{}}#2\end{tabular}}
  
\newcommand{\unknown}[0]{\fixme{XXX}}
\newcommand{\missingcitation}[0]{\fixme{\cite{???}}}

\providecommand{\SetAlgoLined}{\SetLine}
\providecommand{\DontPrintSemicolon}{\dontprintsemicolon}


\newcommand{\capitalize}[1]{\expandafter\MakeUppercase\expandafter{#1}}

\newcommand{\colora}[1]{{\usebeamercolor[fg]{framesubtitle}#1}}

\newcommand{\highlight}[1]{
		\begin{center}
			\Large\colora{\textbf{#1}}
		\end{center}
		}
		
\makeatletter
\newcommand*{\compress}{\@minipagetrue}
\makeatother

\newenvironment{compactItemize}{
\compress
\begin{itemize}[itemsep=0pt,topsep=0pt]
}{\end{itemize}}

\newenvironment{superCompactItemize}{
\compress
\begin{itemize}[itemsep=0pt,topsep=0pt,leftmargin=*]
}{\end{itemize}}

% Only for LEGACY, please use compactItemize
\newenvironment{compact_itemize}{
\compress
\begin{itemize}[itemsep=0pt,topsep=0pt]
}{\end{itemize}}

\newenvironment{compactTabular}{\renewcommand{\arraystretch}{1}\begin{tabular}}{\end{tabular}}


\newcommand{\MyCitationOne}[1]{%
  \begin{textblock*}{.96\textwidth}(.1\textwidth,8.4cm)%
    	\scriptsize
	\colora{[#1]}
  \end{textblock*}
  % TDO: use \CitationText: \CitationText{.96\textwidth}(8.4cm){#1}
}
\newcommand{\MyCitationTwo}[1]{%
  \begin{textblock*}{.96\textwidth}(.1\textwidth,8cm)%
    	\scriptsize
	\colora{[#1]}
  \end{textblock*}
    % TDO: use \CitationText
}

\newcommand{\MyCitationThree}[1]{%
  \begin{textblock*}{.96\textwidth}(.1\textwidth,7.2cm)%
    	\scriptsize
	\colora{[#1]}
  \end{textblock*}
    % TDO: use \CitationText
}

\newcommand{\CitationText}[3]{%
  \begin{textblock*}{#1}(.1\textwidth,#2)%
    	\scriptsize
	\colora{[#3]}
  \end{textblock*}
}

\newcommand{\CiteTextShort}[1]{
  \colora{[#1]}
}

% \SetKwInOut{Input}{Input}

\newcommand{\fracpartial}[2]{\frac{\partial #1}{\partial  #2}}

%========================================================
% MATH
%--------------------------------------------------------

\newcommand{\abs}[1]{\left| #1 \right|}	% absolute value

\renewcommand{\vec}[1]{\boldsymbol{#1}}				% Vector
\newcommand{\mat}[1]{\boldsymbol{\capitalize{#1}}}		% Matrix
\DeclareMathOperator{\diag}{\mathrm{diag}} 			% Diagonal matrix

\newcommand{\R}[0]{\mathds{R}}					% Real numbers

\renewcommand{\d}{\mathrm{d}}					% Derivate
\newcommand{\gradient}{\nabla}					% Gradient
\newcommand{\hessian}{H}					% Hessian
%\nomenclature{\hessian}{Hessian}

\newcommand{\inv}[0]{^{-1}} 					% Inverse
\newcommand{\T}[0]{^T} 						% Transpose

\newcommand{\norm}[1]{\left|\left| #1 \right|\right|} 		% Norm
\newcommand{\asin}[0]{\sin^{-1}} 	% 

\newcommand{\E}{\mathds{E}}	 				% expectation operator
\DeclareMathOperator{\var}{\mathrm{var}} 			% variance
\newcommand{\varMat}[0]{\mat{\Sigma}}
\newcommand{\prob}{{p}} 					% probability density function
\newcommand{\normcdf}[1]{\Phi\left( #1 \right)} 		% Normal Cumulative distribution function
\newcommand{\normpdf}[1]{\phi\left( #1 \right)} 		% Normal probability density function
\newcommand{\gaussNo}[0]{\mathcal{N}}	% Gaussian Distribution
\newcommand{\gauss}[2]{\gaussNo \left( #1, #2 \right)}	% Gaussian Distribution
\newcommand{\unifNo}[0]{\mathcal{U}}
\newcommand{\unif}[2]{\unifNo \left( #1,#2 \right)}


\newcommand{\degrees}[0]{\circ}


%========================================================
% MACHINE LEARNING
%--------------------------------------------------------

\newcommand{\D}[0]{\mathds{D}} 				% Dataset
\newcommand{\dataset}[0]{\mathcal{D}}
\newcommand{\trainingset}[0]{\mathcal{D}}

\newcommand{\inputSca}[0]{\theta} % x
\newcommand{\inputVec}[0]{\vec{\inputSca}}
\newcommand{\inputMatrix}[0]{\mat{\inputSca}}

\newcommand{\outputSca}[0]{y}
\newcommand{\outputVec}[0]{\vec{\outputSca}}
\newcommand{\outputMatrix}[0]{\mat{\outputSca}}

\newcommand{\latentMatrix}[0]{\mat{h}}

\newcommand{\testInputVec}[0]{\inputVec^*}

\newcommand{\dimInputs}{D}
\newcommand{\dimLatents}{Q}
\newcommand{\dimTargets}{P}

\newcommand{\inputSpace}[0]{\mathcal{X}}
\newcommand{\inputsSpace}{\mathcal{X}}
\newcommand{\targetsSpace}{\mathcal{Y}}
\newcommand{\latentsSpace}{\mathcal{H}}
\newcommand{\reallatentsSpace}{\mathcal{L}}

\newcommand{\regressionNo}[0]{f}
\newcommand{\regression}[1]{\regressionNo \left( #1 \right)}

\newcommand{\measurementNoise}[0]{\epsilon}
\newcommand{\noise}{\measurementNoise}

\newcommand{\parameterModel}[0]{\gamma}
\newcommand{\parametersModel}[0]{\vec{\parameterModel}}

\newcommand{\parameterReward}[0]{\gamma}
\newcommand{\parametersReward}[0]{\vec{\parameterModel}}

\newcommand{\loss}[0]{\mathcal{L}}

\newcommand{\nDimInput}[0]{\dimparameters}


%========================================================
% GAUSSIAN PROCESSES
%--------------------------------------------------------

% %\newcommand{\GP}[0]{\text{GP}} % Gaussian process
% \newcommand{\GP}[0]{\mathcal{GP}} 	% Gaussian Process
% \newcommand{\coveq}[0]{k} % k ( \vec x_p, \vec x_q )
% \newcommand{\gpkernel}[0]{covariance function}
% \newcommand{\kronecker}[0]{\delta} 
% \newcommand{\hyperparameter}[0]{\theta}
% \newcommand{\hyperparameters}[0]{\vec{\hyperparameter}}

%\newcommand{\GP}[0]{\text{GP}} % Gaussian process
\newcommand{\GP}[0]{\mathcal{GP}} 	% Gaussian Process
\newcommand{\coveq}[0]{k} % k ( \vec x_p, \vec x_q )
\newcommand{\covVec}[0]{\vec{\coveq}} % k ( \vec x_p, \vec x_q )
\newcommand{\covMatrix}[0]{\mat{K}}
\newcommand{\gpkernel}[0]{covariance function}
\newcommand{\gpprior}[0]{m_f}
\newcommand{\kronecker}[0]{\delta} 
\newcommand{\hyperparameter}[0]{\omega}
\newcommand{\hyperparameters}[0]{\vec{\hyperparameter}}

\newcommand{\mean}[0]{\mu}
\newcommand{\std}[0]{\sigma}
\newcommand{\variance}[0]{\std^2}


%========================================================
% OPTIMIZATION
%--------------------------------------------------------

\DeclareMathOperator*{\argmin}{arg\,min}
\DeclareMathOperator*{\minimize}{\text{arg min}}
\newcommand{\maximize}[0]{\text{arg max}} 			% Maximize

\newcommand{\parameter}[0]{\theta} 				% Parameter
\newcommand{\parameters}[0]{\vec{\parameter}} 			% Parameters
\newcommand{\context}[0]{\vec{c}}
\newcommand{\dimparameters}{d}					% Dimensionality parameters
\newcommand{\objfuncNo}[0]{f}					% Objective function
\newcommand{\objfunc}[1]{\objfuncNo\left(#1\right)}		% Objective function (as a function)
\newcommand{\gradientline}{\d \objfunc{\parameters}/\d \parameters}
\newcommand{\objeval}[0]{y}
\newcommand{\objevals}[0]{\vec{\objeval}}
\newcommand{\iteration}[0]{t}

% BO
\newcommand{\respsurfNo}[0]{\hat \objfuncNo}			% Response surface
\newcommand{\respsurf}[1]{\respsurfNo \left( #1 \right)} 	% Response surface (as a function)
\newcommand{\acqfuncNo}[0]{\alpha}				% Acquisition function
\newcommand{\acqfunc}[1]{\acqfuncNo \left( #1 \right)}		% Acquisition function (as a function)
\newcommand{\acqsurf}[1]{\acqfuncNo \left( #1 \right)}		% Acquisition surface
\newcommand{\targetValue}[0]{T} 				% Target value (for PI and EI)

% HDBO
\newcommand{\boundLowDim}{B_\latentSpace}
\newcommand{\boundHighDim}{B_\inputSpace}
\newcommand{\transfMatrix}{\mat{A}}
\newcommand{\nEmbeddings}{k}

% Multi-objective optimization
\newcommand{\numbersubobj}[0]{n}
\newcommand{\mergefuncmooNo}[0]{z}
\newcommand{\mergefuncmoo}[1]{\mergefuncmooNo \left( #1 \right)}
\newcommand{\weightsmoo}[0]{\alpha} % LEGACY
\newcommand{\mooweight}[0]{\alpha}
\newcommand{\mooweights}[0]{\vec{\alpha}}
\newcommand{\subobjfuncNo}[0]{g}
\newcommand{\subobjfunc}[1]{\subobjfuncNo \left( #1 \right)}
\newcommand{\PF}[0]{\mathcal{P}}
\newcommand{\refPoint}[0]{\mathcal{R}}
\newcommand{\hypervolume}[0]{\mathcal{H}}
\newcommand{\HVR}[0]{\text{HVR}}
\newcommand{\dom}[0]{\succ} % MOO domination

% Robustness 
\newcommand{\robustnessNo}[0]{r}
\newcommand{\rNoise}[0]{\robustnessNo_f}
\newcommand{\rPar}[0]{\robustnessNo_p}
\newcommand{\neighborhood}[0]{\xi}


%========================================================
% Neural Networks
%--------------------------------------------------------

\newcommand{\nntfNo}[0]{\sigma} 	% Neural Networks Transfer function
\newcommand{\nntf}[1]{\nntfNo \left( #1 \right)} 	% Neural Networks Transfer function
\newcommand{\nnBh}[0]{\vec{Bh}} 				% Neural Networks Bias of Hidden layer
\newcommand{\nnB}[0]{\vec{B}} 					% Neural Networks Bias
\newcommand{\nnW}[0]{\mat{W}} 					% Neural Networks Weights
\newcommand{\nnE}[0]{E} 					% Neural Networks Error function
\newcommand{\sizeLayer}[0]{q}

%========================================================

\newcommand{\experiment}[1]{\todo[color=green!80,inline]{Experiment: #1}}
\newcommand{\question}[1]{\todo[color=orange!80,inline]{Question: #1}}
\newcommand{\implement}[1]{\todo[color=yellow!80,inline]{Implement: #1}}
\newcommand{\statement}[1]{\framebox{\textbf{#1}}}

%========================================================
% ROBOTICS
%--------------------------------------------------------

\newcommand{\q}{\vec{q}}					% Position
\ifdef{\dq}{\renewcommand{\dq}{\dot{\q}}}{\newcommand{\dq}{\dot{\q}}}
\newcommand{\ddq}{\ddot{\q}}					% Acceleration
\newcommand{\skinInput}{\mathcal{\vec{s}}}			% Skin input

\newcommand{\dof}[0]{DOF}
\newcommand{\gains}[0]{\mat{K}}
\newcommand{\controlSig}[0]{\vec{u}}

\newcommand{\frictionMatrixNo}[0]{\mat{f}}
\newcommand{\inertiaMatrixNo}[0]{\mat{M}}
\newcommand{\inertiaMatrix}[0]{\inertiaMatrixNo \left( \q \right)}
\newcommand{\gravityMatrixNo}[0]{\vec{g}}
\newcommand{\torque}[0]{\tau}
\newcommand{\torques}[0]{\vec{\torque}}
%\newcommand{\forceNo}[0]{\vec F}
\newcommand{\extForce}[0]{\gamma}
\newcommand{\extForces}[0]{\vec{\extForce}}
\newcommand{\ftsForce}{F}
\newcommand{\ftsForces}{\vec{\ftsForce}}
\newcommand{\jtsForce}{\tau}%_text{JTS}}
\newcommand{\jtsForces}{\vec{\jtsForce}}%_text{JTS}}
\newcommand{\Hmatrix}[0]{\vec h \left(\q,\dq \right)}
\newcommand{\jacobian}[0]{\mat{J}}
\newcommand{\coriolis}[0]{C}

\newcommand{\torquesRBD}[0]{\torques_{\textit{RBD}}}
\newcommand{\torquesPD}[0]{\torques_{\textit{PD}}}
\newcommand{\torquesID}[0]{\torques_{\textit{ID}}}
\newcommand{\torquesExt}[0]{\torques_{\textit{ext}}}
%\newcommand{\torquesCM}[0]{\torques_{\textit{CM}}}
\newcommand{\torquesCM}[0]{\torquesExt}


%========================================================
% (PO)MDP
%--------------------------------------------------------

\newcommand{\MDPSetState}{\mathcal{S}}				% Set possible States
\newcommand{\MDPSetAction}{\mathcal{A}}				% Set possible Actions
\newcommand{\MDPprobability}{z}			% Transition function
\newcommand{\MDPreward}{r}
\newcommand{\MDPState}{\vec{s}} 				% State
\newcommand{\MDPAction}{\vec{a}}				% Action
\newcommand{\MDPObservation}{\vec{o}}				% Action
\newcommand{\MDPTime}{t}					% Time

\newcommand{\MDPdimState}[0]{n}
\newcommand{\MDPdimAction}[0]{m}

\newcommand{\policyspace}[0]{\mathcal{P}}
\newcommand{\policyNo}[0]{\pi}					% Policy
\newcommand{\policy}[1]{\policyNo \left( #1 \right)}
% \newtheorem{theorem}{Theorem}
% \newtheorem{lemma}[theorem]{Lemma}

%\newrobustcmd*{\citefirstlastauthor}{\AtNextCite{\DeclareNameAlias{labelname}{f‌​irst-last}}\citeauthor}

%\newcites{A}{References}

\newcommand{\auxreward}[0]{\tilde{\MDPreward}} % Auxiliary reward function = \MDPreward + \rewardshaping
\newcommand{\rewardshaping}[0]{\mathcal{G}} % reward shaping term
\newcommand{\rewardshapingParameters}[0]{{\vec{\theta}_\rewardshaping}}
\newcommand{\learner}[0]{\mathcal{L}}
\newcommand{\metalearner}[0]{M}
\newcommand{\metaparameters}[0]{\Phi}

%========================================================
% ROBOTS
%--------------------------------------------------------
\newcommand{\robotFox}[0]{\textit{Fox}}
%\newcommand{\robotFox2}[0]{\textit{Fox-2}}
\newcommand{\robotIcub}[0]{\textit{iCub}}



%======================================================== 
% TEMP
%--------------------------------------------------------
\newcommand{\foxvel}{0.45}
\newcommand{\newmethodlong}[0]{Manifold Gaussian Processes} 	% 
\newcommand{\newmethodshort}[0]{MGP} 	% (Feature Learning GP = FLGP) (Manifold GP = mGP)

\newcommand{\robot}[0]{\robotFox}
\newcommand{\idyn}[0]{\textit{iDyn}}

\newcommand{\nSensorsIcub}[0]{2000}

\newcommand{\namenewmethod}{REMBO+}



%========================================================
% TEMP (for thesis merging)
%--------------------------------------------------------

\newcommand{\mappingNo}[0]{M}
\newcommand{\gprNo}[0]{G}
\newcommand{\inputs}{\mat X}
\newcommand{\targets}{\mat Y}
\newcommand{\reallatents}{\mat L}
\newcommand{\latents}{\mat H}
\newcommand{\mapping}[1]{{\mappingNo \hspace{-2pt} \left( #1 \right) }}
\newcommand{\nntransformation}[0]{\mathds{T}}
\newcommand{\secExpOne}[0]{Step Function}
\newcommand{\secExpTwo}[0]{Bipedal Robot Locomotion}
\newcommand{\secExpThree}[0]{Raw Visual Data}
%\newcommand{\secExpFour}[0]{Correlated Inputs}
\newcommand{\secExpFour}[0]{Multiple Length-Scales}
\newcommand{\secExpFive}[0]{2-D Discontinuous function}

\newcommand{\marc}[1]{\todo[inline,color=yellow]{Marc: #1}}


\usepackage{array}
\makeatletter
\newcommand{\thickhline}{%
    \noalign {\ifnum 0=`}\fi \hrule height 1pt
    \futurelet \reserved@a \@xhline
}
\newcolumntype{"}{@{\hskip\tabcolsep\vrule width 1pt\hskip\tabcolsep}}
\makeatother
