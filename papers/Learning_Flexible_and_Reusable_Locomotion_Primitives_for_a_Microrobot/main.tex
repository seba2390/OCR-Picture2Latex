\documentclass[letterpaper, 10 pt, journal, twoside]{IEEEtran} 

\IEEEoverridecommandlockouts                              % This command is only needed if 
                                                          % you want to use the \thanks command

\title{Learning Flexible and Reusable Locomotion Primitives for a Microrobot}
\markboth{IEEE Robotics and Automation Letters. Preprint Version. Accepted January, 2018}
{Yang \MakeLowercase{\textit{et al.}}: Learning Flexible and Reusable Locomotion Primitives for a Microrobot}  
% Make room for more info lines in the \author command  
\author{Brian Yang, Grant Wang, Roberto Calandra, Daniel Contreras, Sergey Levine, and Kristofer Pister%
\thanks{Manuscript received: September, 09, 2017; Revised January, 21, 2018; Accepted January, 27, 2018.}%Use only for final RAL version
\thanks{This paper was recommended for publication by Editor Yu Sun upon evaluation of the Associate Editor and Reviewers' comments. 
This work was supported by the Berkeley Sensor and Actuator Center, and by Berkeley DeepDrive. \textit{(Brian Yang and Grant Wang contributed equally to this work.) (Corresponding author: \href{mailto:roberto.calandra@berkeley.edu}{Roberto Calandra})}}
% Equally contributed authors according to IEEE guidelines: https://www.ieee.org/publications_standards/publications/journmag/online_style_manual-10292015.pdf
\thanks{All the authors are with the Department of Electrical Engineering and Computer Sciences, University of California, Berkeley, USA
        {\tt\small \{brianhyang, grant.wang5, roberto.calandra, dscontreras, ksjp\}@berkeley.edu, svlevine@eecs.berkeley.edu}}%
\thanks{Digital Object Identifier (DOI): 10.1109/LRA.2018.2806083}
}


% OLD PREAMBLE:

% \usepackage{jsen}
% \usepackage{cite}
% \usepackage{amsmath,amssymb,amsfonts, bbm, mathtools}
% \usepackage{algorithm,algorithmic}
% \usepackage{graphicx}
% \usepackage{textcomp}
% \usepackage{wrapfig}
% \usepackage{xfrac}
% \usepackage{stackengine}
% \usepackage{subfigure}
% \def\delequal{\mathrel{\ensurestackMath{\stackon[1pt]{=}{\scriptstyle\Delta}}}}



% \usepackage{color, soul}
% \newcommand{\hlt}[1]{\hl{#1}}
% \newcommand{\red}[1]{\textcolor{red}{#1}}

% \def\BibTeX{{\rm B\kern-.05em{\sc i\kern-.025em b}\kern-.08em
%     T\kern-.1667em\lower.7ex\hbox{E}\kern-.125emX}}
% \markboth{\journalname, VOL. XX, NO. XX, XXXX 2017}
% {Author \MakeLowercase{\textit{et al.}}: Preparation of Papers for IEEE TRANSACTIONS and JOURNALS (February 2017)}
% \definecolor{abstractbg}{rgb}{0.89804,0.94510,0.83137}
% \setlength{\fboxrule}{0pt}
% \setlength{\fboxsep}{0pt}

% NEW PREAMBLE:


\usepackage{amsmath,amsfonts,amssymb,bbm, amsthm, xfrac}
\usepackage{algorithmic}
\usepackage{algorithm}
\usepackage{array, multirow}
% \usepackage[caption=false,font=normalsize,labelfont=sf,textfont=sf]{subfig}
\usepackage{caption, subcaption}
\usepackage{textcomp}
\usepackage{stfloats}
\usepackage{url}
\usepackage{verbatim}
\usepackage{graphicx}
\usepackage{cite}
\usepackage{caption}
\usepackage{subcaption}
\hyphenation{}

\theoremstyle{plain}
\newtheorem{theorem}{Theorem}

\usepackage{color, soul}
\newcommand{\hlt}[1]{\hl{#1}}
\newcommand{\red}[1]{\textcolor{red}{#1}}

\newcommand{\citep}[1]{\cite{#1}}

\begin{document}

\maketitle

%%%%%%%%%%%%%%%%%%%%%%%%%%%%%%%%%%%%%%%%%%%%%%%%%%%%%%%%%%%%%%%%%%%%%%%%%%%%%%%%

\begin{abstract}
	\begin{abstract}
%\medskip
%\centering \textcolor{red}{Write the abstract last}
Silicon-compatible short- and mid-wave infrared emitters are highly sought-after for on-chip monolithic integration of electronic and photonic circuits to serve a myriad of applications in sensing and communication. To address this longstanding challenge, GeSn semiconductors have been proposed as versatile building blocks for silicon-integrated optoelectronic devices. In this regard, this work demonstrates light-emitting diodes (LEDs) consisting of a vertical PIN double heterostructure  p-Ge$_{0.94}$Sn$_{0.06}$/i-Ge$_{0.91}$Sn$_{0.09}$/n-Ge$_{0.95}$Sn$_{0.05}$ grown epitaxially on a silicon wafer using germanium interlayer and multiple GeSn buffer layers. The emission from these GeSn LEDs at variable diameters in the 40-120 $\mu$m range is investigated under both DC and AC operation modes. The fabricated LEDs exhibit a room temperature emission in the extended short-wave range centered around 2.5 $\mu$m under an injected current density as low as 45 A/cm$^2$.  By comparing the photoluminescence and electroluminescence signals, it is demonstrated that the LED emission wavelength is not affected by the device fabrication process or heating during the LED operation. Moreover, the measured optical power was found to increase monotonically as the duty cycle increases indicating that the DC operation yields the highest achievable optical power. The LED emission profile and bandwidth are also presented and discussed. 
\end{abstract}
\end{abstract}
\begin{IEEEkeywords}
Learning and Adaptive Systems; Micro/Nano Robots; Legged Robots
\end{IEEEkeywords}

%%%%%%%%%%%%%%%%%%%%%%%%%%%%%%%%%%%%%%%%%%%%%%%%%%%%%%%%%%%%%%%%%%%%%%%%%%%%%%%%

\section{Introduction}

	\section{Introduction}

Scientific literature is most commonly available in the form of PDFs, which pose challenges for accessibility \citep{NielsenPDFStillUnfit, Bigham2016AnUT}. When researchers, students, and other individuals who are blind or low vision (BLV) interact with scientific PDFs through screen readers, the availability of document structure tags, labeled reading order, labeled headers, and image alt-text are necessary to facilitate these interactions. However, these features must be painstakingly added by authors using proprietary software tools, and as a result, are often missing from papers. Low vision or dyslexic readers who interact with PDFs through screen magnification or text-to-speech may also find the complexity of certain academic paper PDF formats challenging, e.g., non-linear layout can interrupt the flow of text in a magnifying tool. Inaccessible paper PDFs can lead to high cognitive overload, frustration, and abandonment of reading for BLV readers. 

Unfortunately, we find that the majority of scientific PDFs lack basic accessibility features. We estimate based on a sample of \numpdfs PDFs from multiple fields of study that only around \percaccessible of paper PDFs released in the last decade satisfy all of the aforementioned accessibility requirements. 
Accessibility challenges for academic PDFs are largely due to three factors: (1) the complexity of the PDF file format, which make it less amenable to certain accessibility features, (2) the dearth of tools, especially non-proprietary tools, for creating accessible PDFs, and (3) the dependency on volunteerism from the community with minimal support or enforcement \citep{Bigham2016AnUT}. The intent of the PDF file format is to support faithful visual representation of a document for printing, a goal that is inherently divergent from that of document representation for the purposes of accessibility. Though some professional organizations like the Association for Computing Machinery (ACM) have encouraged PDF accessibility through standards and writing guidelines,\footnote{\href{https://www.acm.org/publications/authors/submissions}{https://www.acm.org/publications/authors/submissions}} uptake among academic publishers and disciplines more broadly has been limited. 

While policy changes help, the fact remains that most academic PDFs produced today, and historically, are inaccessible, yet remain as the dominant way to read those papers. A long-range solution will necessitate buy-in from multiple stakeholders---publishers, authors, readers, technologists, granting agencies, and the like. But in the interim, there are technological solutions that can be offered as a sort of ``band-aid'' to the problem. We use this paper to offer an in-depth qualitative and quantitative description of the problem as it stands, and to introduce one such technological solution: the \scially system that automatically extracts semantic information from paper PDFs and re-renders this content in the form of an accessible HTML document. Though the process is imperfect and can introduce errors, we demonstrate the ability of the rendered HTMLs to reduce cognitive load and facilitate in-paper navigation and interactions for BLV users. 

The goals and contributions of this paper are three-fold:

\begin{enumerate}
    \item We characterize the state of academic-paper PDF accessibility by estimating the degree of adherence to accessibility criteria for papers published in the last decade (2010--2019), and describe correlations between year, field of study, PDF typesetting software, and PDF accessibility.
    \item We propose an automated approach for extracting the content of academic PDFs and displaying this content in a more accessible HTML document format. We build a prototype that re-renders 12 million PDFs in HTML, and describe the design decisions, features, and quality of the renders (assessed as faithfulness to the source PDF). We perform expert grading of the rendered HTML and report an error analysis. A demo of our system is available at \href{https://scia11y.org/}{scia11y.org}, which makes available 1.5M HTML renders of open access PDFs.
    \item We conduct an exploratory user study with \numusers BLV scholars to better understand the challenges they experience when reading academic papers and how our proposed tool might augment their current workflow. During the study, we ask users to interact with the prototype and offer feedback for its improvement. We perform open coding of interviews to identify existing reading challenges, coping mechanisms, as well as positive and negative responses to prototype features. We summarize the findings of this user study into a set of design recommendations.
\end{enumerate}

Our analysis reveals that PDF accessibility adherence is low across all fields of study. Of the five accessibility criteria we assess, only \percaccessible of the PDFs we assess demonstrate full compliance. Though compliance for several criteria seems to be increasing over time, author awareness and contribution to accessibility remains low, as Alt-text has the lowest compliance of the five criteria at between 5--10\% (Alt-text is the only criterion of the five that \textit{requires} author intervention in all cases using current tools). We also find that typesetting software is strongly associated with accessibility compliance, with LaTeX and publishing software like Arbortext APP producing low compliance PDFs, while Microsoft Word is generally associated with higher compliance.


\begin{figure}[t!]
    \centering
    \includegraphics[width=\textwidth]{figures/pipeline.png}
    \caption{A schematic for creating the \scially HTML render from a paper PDF. Starting with the raw two-column PDF on the left, S2ORC \citep{lo-wang-2020-s2orc} is used to extract title, authors, abstract, section headers, body text, and references. S2ORC also identifies links between inline citations and references to figures and table objects. DeepFigures \citep{Siegel2018ExtractingSF} is used to extract figures and tables, along with their captions. The output of these two models are merged with metadata from the Semantic Scholar API. Heuristics are used to construct a table of contents, to insert figures and tables in the appropriate places in the text, and to repair broken URLs. We add HTML headers as illustrated (header tags for sections, paragraph tags for body text, and figure tags for figures and tables); highlighted components (table of contents and links in references) are not in the PDF and novel navigational features that we introduce to the HTML render. An example HTML render of parts of a paper document is show to the right (actual render is single column, which is split here for presentation).}
    \label{fig:pipeline}
    \Description{A schematic diagram showing the components of the SciA11y pipeline. An image of a paper PDF is on the left. Red boxes on the PDF image highlight the text components from the paper, with an arrow pointing to a box that says "S2ORC extracts: title, authors, abstract, section headers, body paragraphs, and references." A blue box on the PDF image highlights a figure, with an arrow pointing to a box that says "DeepFigures extracts: figures, figure captions, tables, and table titles/captions." A box below "S2ORC extracts" and "DeepFigures extracts" says "Additional content: metadata from Semantic Scholar API, table of contents, figures and tables inserted at first mention, and links between references and text." Arrows from all three boxes point into a larger box that describes the SciA11y prototype, where HTML tags are inserted around various blocks of text extracted from the PDF. On the right of all this is a screen capture of an example HTML render, showing how the semantic content from the PDF is represented as a single-column HTML page for easy reading.}
\end{figure}

To offset the reading challenges of inaccessible papers for BLV researchers, we propose and test the \scially system for rendering academic PDFs into accessible HTML documents. As shown in Figure~\ref{fig:pipeline}, our prototype integrates several machine learning text and vision models to extract the structure and semantic content of papers. The content is represented as an HTML document with headings and links for navigation, figures and tables, as well as other novel features to assist in document structure understanding. Our evaluation of the \scially system identifies common classes of extraction problems, and finds that though many papers exhibit some extraction errors, the majority (55\%) have no major problems that impact readability, and another 32\% have only some problems that impact readability.

Through our user study, we identify numerous challenges faced by BLV users when reading paper PDFs, including some that affect the whole document or limit navigation, and many that affect the ability of the reader to understand text or various elements of a paper like math content or tables. Responses to \scially were positive; participants especially liked navigation features such as headings, the table of contents, and bidirectional links between inline citations and references. Of the extraction errors in \scially, missed or incorrectly extracted headings were the most problematic, as these impact the user's ability to navigate between sections and fully trust the system. All users reported being likely to use the system in the future. When asked how the system might be integrated into their workflow, one participant replied ``I think it would become the workflow.'' Another participant said, ``for unaccessible PDFs, this is life-changing.'' We condense these findings into a set of recommendations for designing and engineering accessible reading systems (Section~\ref{sec:designrecs}). Most importantly, documents should be structured to match a reader's mental model, objects should be properly tagged, and care should be taken to reduce the reader's cognitive load and increase trust in the system. Features that emulate the external memory that visual layout provides to sighted users can be especially beneficial.

This paper is organized as follows. Following a description of related work in Section \ref{sec:related_work}, we first provide a meta-scientific analysis of the current state of academic PDF accessibility in Section \ref{sec:sos}. In Section \ref{sec:pdf2html}, we document our pipeline for converting PDF to HTML and describe the \scially prototype for rendering papers. An evaluation of HTML render quality and faithfulness is provided in Section \ref{sec:evaluation}. Section \ref{sec:user_study} describes our user study and findings. 
We recognize that no PDF extraction system is perfect, and many open research challenges remain in improving these systems. However, based on our findings, we believe \scially can dramatically improve screen reader navigation of most papers compared to PDFs, and is well-positioned to assist BLV researchers with many of their most common reading use cases. Our hope is that a system such as \scially can improve BLV researcher access to the content of academic papers, and that these design recommendations can be leveraged by others to create better, more faithful, and ultimately more usable tools and systems for scholars in the BLV community.


%%%%%%%%%%%%%%%%%%%%%%%%%%%%%%%%%%%%%%%%%%%%%%%%%%%%%%%%%%%%%%%%%%%%%%%%%%%%%%%%

\section{Related Work}
\label{sec:related}

	\section{Related Work}\label{related_work}
% \subsubsection{Multi-Turn Dialogue Understanding}
\textbf{Multi-Turn Dialogue Understanding}. Various tasks and corresponding benchmarks are proposed to evaluate the capacities of dialogue understanding models. Dialogue-based relation extraction (RE) is a classification task that assigns a pair of entities a relation label in a dialogue. Focusing on the word level,  \cite{xue2021gdpnet} constructed a multi-view graph with words in the dialogue as nodes and proposed Dynamic Time Warping Pooling to automatically select words in interest. SimpleRE \citep{SimpleRE} designed a novel input sequence format and utilized a Relation Refinement Gate to filter the semantic representation which is later fed into the classifier. TUCORE-GCN \citep{zahiri:18a} used a heterogeneous dialogue graph to encode the interaction between speakers, arguments, and turns across the dialogues.
% \citet{christopoulou-etal-2019-connecting} constructed an edge-oriented graph model to encode the dialogue as a graph with nodes and edges of different functions and applied an inference mechanism on the graph edges to recognize the internal relationship. By constructing a mention-level graph and an entity-level graph, \citet{zahiri:18a} reasoned the relation between entities by path inference.

Emotion Recognition in Conversation (ERC) has been extensively studied in the research community. It aims to attach an emotional label to every turn in a given dialogue. \cite{kratzwald2018deep} customized the recurrent neural network with bidirectional processing to solve the problem of emotion classification. \cite{majumder2019dialoguernn} leveraged the Recurrent Neural Network to extract the information of the party states and use it to predict the emotion in conversations with two speakers. On top of the recurrent neural network, COSMIC \cite{ghosal2020cosmic} models the commonsense knowledge, mental states, events, and actions to enhance emotion detection in dialogue. 
%Towards solving the context propagation problems in the recurrent neural network-based methods, \citet{ghosal-etal-2019-dialoguegcn} proposed a graph-based method that models the utterances as nodes and the speakers' dependency as edges. 

Deep learning-based methods have been extensively studied in recent works \citep{lee-dernoncourt-2016-sequential, chen2018dialogue, raheja2019dialogue} regarding Dialogue Act classification (DAC). \cite{chen2018dialogue} introduced a relation layer into the shared hierarchical encoder to model the interaction between the tasks of dialog act recognition and sentiment classification.
%Combining recurrent neural networks and convolutional neural networks, \citet{lee-dernoncourt-2016-sequential} incorporated the preceding texts while classifying the act.


% \subsubsection{Context-Aware Representation Learning}
\textbf{Context-Aware Representation Learning}. To address dynamics and semantic changes in multi-turn dialogue, previous works extend pre-trained large language models to learn context-aware representations for turns \citep{lee2021graph, DialogXL, DCM, chapuis2020hierarchical}. TUCORE-GCN \citep{lee2021graph} proposes the turn attention module, masking out distant turns to learn the contextual embeddings. Instead of adding extra modules, DialogXL \citep{DialogXL} targets the encoder and incorporates four self-attention mechanisms to different attention heads to capture diverse dialog-aware information. Similarly, such dialogue-oriented self-attention can also be found in MDFN \citep{MDFN} where it is defined as utterance-aware and speaker-aware channels. However, most of them involve an additional pre-training stage \citep{DialogXL, DCM, chapuis2020hierarchical}. 
%These efforts focus on the turn-level modeling but ignored the gap between the pre-training objective and dialogue understanding tasks. Also, though achieving preferable performance on limited datasets or tasks, these methods have not led to a unified solution to multi-turn dialogue understanding. 

%%%%%%%%%%%%%%%%%%%%%%%%%%%%%%%%%%%%%%%%%%%%%%%%%%%%%%%%%%%%%%%%%%%%%%%%%%%%%%%%

\section{The Hexapod Microrobot}
	
    We now introduce the hexapod microrobot considered in this paper.
This robot is of particular interest due to the unique challenges that arise when attempting traditional gait design techniques.
The micro-scale of the walker makes it very challenging to obtain an accurate dynamics model.
Moreover, the robot is subject to wear-and-tear, and therefore any learning approach employed must be capable of learning gaits within a limited number of trials.

\subsection{Physical Description}    
    %
	\begin{figure}[t]
	  \centering
	  \includegraphics[width=0.96\linewidth]{fig/leg_diagram.pdf}
	  \caption{Diagram of the robot leg showing the actuation sequence (active motors are shown in red). Each leg has 2 motors, each one independently actuating a single DOF. 
      }
	  \label{fig:leg_diagram}
	%   \vspace{-8pt}
	\end{figure}
	% 
	
	The hexapod microrobot is based on silicon microelectromechanical systems (MEMS) technology. 
	The robot's legs are made using linear motors actuating planar pin-joint linkages~\citep{Contreras2016DurabilityOS}. 
	A tethered single-legged walking robot was previously demonstrated using this technology~\citep{contreras_first_2017}. 
	The hexapodal robot is assembled using three chips.  
	The two chips on the side each have 3 of the leg assemblies, granting six 2 degree-of-freedom (DOF) legs for the whole robot. 
	The top chip acts to hold the leg chips together for support, and to route the signals for off-board power and control.
	Overall, the robot measures \SI{13}{\milli\meter} long by \SI{9.6}{\milli\meter} wide and stands at \SI{8}{\milli\meter} tall with an overall weight of approximately \SI{200}{\milli\gram}. 
 
\subsection{Actuation}
	Each of the robot's legs has 2-DOF in the plane of fabrication, as shown in \fig{fig:leg_diagram}.
	Both DOFs are actuated, thus the leg has 2 motors, one to actuate the vertical DOF to lift the robot's body and a second to actuate the horizontal DOF for the vertical stride. 
	The actuators used for the legs are electrostatic gap-closing inchworm motors~\cite{penskiy_optimized_2013}.
	During a full cycle, each leg moves \SI{0.6}{\milli\meter} vertically with a horizontal stride of \SI{2}{\milli\meter}. 
	For more details on the actuation mechanism used on our microrobot, we refer readers to ~\cite{Contreras2017DynamicsOE}.
	
\subsection{Simulator}
	%
	\begin{wrapfigure}{r}{0.46\linewidth} 
	  \centering
	  \vspace{-10pt}
	  \includegraphics[width=0.98\linewidth]{fig/simulator.pdf}
	  \caption{The simulated micro walker.}
	  \label{fig:vrep}
	  \vspace{-8pt}
	\end{wrapfigure}
	% 
	In our experimental simulations, we used the robotics simulator V-REP~\citep{vrep} for constructing a scaled-up simulated model of the physical microrobot (see \fig{fig:vrep}).
	Since V-REP was not designed with simulation of microrobots in mind, it was not capable of simulating the dynamics of the leg joints accurately and would produce wildly unstable models at the desired scale.
	We chose to scale up the size of the robot in simulation by a factor of $100$ in order to account for the issues with scaling in simulation (all the experimental results are re-normalized to the dimensions of the real robot).
	We believe that this re-scaling still allows meaningful results to be produced for several reasons.
	First, the experiments performed in this paper are meant to demonstrate the validity of the proposed controller, and the learning approach for training an actual physical microrobot.
	The policies trained are not meant to work on the real robot without any re-tuning or modification.
	Second, the simulator still allows to test the basic motion patterns we want to implement on the microrobot.
	Finally, our contribution lends credibility to the potential application of Bayesian-inspired optimization methods to a setting where evaluations can be costly and time consuming.


%%%%%%%%%%%%%%%%%%%%%%%%%%%%%%%%%%%%%%%%%%%%%%%%%%%%%%%%%%%%%%%%%%%%%%%%%%%%%%%%

\section{Background}
\label{sec:background}

	\subsection{Central Pattern Generators}
\label{sec:bg:cpg}
	Central pattern generators (CPGs) are neural circuits found in nearly all vertebrates, which produce periodic outputs without sensory input~\citep{Junzhi_Yu_2014}.
	CPGs are also a common choice for designing gaits for robot locomotion~\citep{Ijspeert2008}.
    We chose to use CPGs for our controller because they are capable of reproducing a wide variety of different gaits simply by manipulating the relative coupling phase biases between oscillators. 
	This allows us to easily produce a variety of gait patterns without having to manually program those behaviors. 
	In addition, CPGs are not computationally intensive and can have on-chip hardware implementations using VLSI or FPGA. 
	This makes them well suited to be eventually used in our physical microrobot, where the processing power is limited.
	CPGs can be modeled as a network of coupled non-linear oscillators where the dynamics of the network are determined by the set of differential equations
	%
	\begin{align}
	  \dot{\phi_i} &= \omega_{i} + \displaystyle \sum_{j} (\omega_{ij}r_j\sin(\phi_j - \phi_i - \varphi_{ij}))\,,\\
	  \ddot{r_i} &= a_r (\dfrac{a_r}{4} (R_i - r_i) - \dot{r_i})\,,\\
	  \ddot{x_i} &= a_x (\dfrac{a_x}{4} (X_i - x_i) - \dot{x_i})\,,
	\end{align}
	%
	where $\phi_{i}$ is a state variable corresponding to the phase of the oscillations and $\omega_{i}$ is the target frequency for the oscillations. 
    $\omega_{ij}$ and $\varphi_{ij}$ are the coupling weights and phase biases which change how the oscillators influence each other.
    To implement our desired gaits, we only need to modify the phase biases between the oscillators~$\phi_{ij}$.
    $r_{i}$ and $x_{i}$ are state variables for the amplitude and offset of each oscillator, and $R_{i}$ and $X_{i}$ are control parameters for the desired amplitude and offset. 
    The constants $a_{r}$ and $a_{x}$ are constant positive gains and allow us to control how quickly the amplitude and offset variables can be modulated. 
    A more detailed explanation of the network can be found in Crespi's original work~\citep{Crespi_2007}. 
	One of the foremost benefits of using a CPG controller is a drastic reduction in the number of parameters~$\parameters_i$ we need to optimize.
    Overall, the parameters that we consider during the optimization are $\parameters = \left[\omega, R, X_{l}, X_{r} \right]$ where $\omega$ is the frequency of the oscillators and $R$ is the phase difference between each of the vertical-horizontal oscillator pairs. 
    In order to allow for directional control, $X_{l}$ and $X_{r}$ are the amplitudes of the left and right side oscillators respectively.    
    
\subsection{Bayesian Optimization}
\label{sec:bg:BO}
	%
	Even with a complete CPG network, some amount of parameter tuning is necessary to obtain efficient locomotion. 
	To automate the parameter tuning, we use Bayesian optimization (BO), an approach often used for global optimization of black box functions~\citep{Kushner1964,Jones2001,Calandra2015a}.
	We formulate the tuning of the CPG parameters as the optimization
	%
	\begin{align}
		\parameters^* = \maximize_{\parameters}\, \objfunc{\parameters}\,,
		\label{eq:optimization}
	\end{align}
	%
	where $\parameters$ are the CPG parameters to be optimized w.r.t. the objective function of choice~$\objfuncNo$ (\eg, walking speed, which we investigate in \sec{sec:results:soo}).
	At each iteration, BO learns a model $\tilde{\objfuncNo}: \parameters \rightarrow \objfunc{\parameters}$ from the dataset of the previously evaluated parameters and corresponding objective values measured~$\dataset=\{\parameters, \objfunc{\parameters}\}$.
	Subsequently, the learned model $\tilde{\objfuncNo}$ is used to perform a ``virtual'' optimization through the use of an acquisition function which controls the trade-off between exploration and exploitation.
	Once the model is optimized, the resulting set of parameters $\parameters^*$ is finally evaluated on the real system, and is added to the dataset together with the corresponding measurement $\objfunc{\parameters^*}$  before starting a new iteration.
	A common model used in BO for learning the underlying objective, and the one that we consider, is Gaussian processes~\citep{Rasmussen2006}. 
	For more information regarding BO, we refer the readers to~\citep{Jones2001,Shahriari2016}.

	
\subsection{Multi-objective Bayesian Optimization}
	A special case of the optimization task of \eq{eq:optimization} is multi-objective optimization~\citep{Branke2008a}.
	Often times in robotics\footnote{As well as in nature~\citep{Hoyt1981}.}, there are multiple conflicting objectives that need to be optimized simultaneously, resulting in design trade-offs (e.g., walking speed vs energy efficiency which we investigate in \sec{sec:results:moo}).  
	When multiple objectives are taken into consideration, there is no longer necessarily a single optimum solution, but rather the goal of the optimization became to find the set of Pareto optimal solutions~\citep{Pareto1906}, which also takes the name of Pareto front~(PF).
	Formally, the PF is the set of parameters that are not dominated, where a set of parameters~$\parameters_1$ is said to dominate $\parameters_2$ when 
	%
	\begin{align}
		\left\{
		\begin{array}{l l}
		\forall i \in \{1,\dots ,\numbersubobj\}: &\objfuncNo_i(\parameters_1) \leq \objfuncNo_i(\parameters_2)\\
		\exists j \in \{1,\dots , \numbersubobj\}:  &\objfuncNo_j(\parameters_1) < \objfuncNo_j(\parameters_2)
		\end{array} \right.
	\end{align}
	Intuitively, if $\parameters_1 \dom \parameters_2$, then $\parameters_1$ is preferable to $\parameters_2$ as it never performs worse, but at least in one objective function it performs strictly better. 
	However, different dominant variables are equivalent in terms of optimality as they represent different trade-offs.
	      
	Multi-objective optimization can often be difficult to perform as it might require a significant amount of experiments.
	This is especially true with our microrobot where large number of experiments can wear-and-tear the robot.
	As a result, the number of evaluations allowed to find the Pareto set of solutions is limited. 
	Luckily for us, there exist extensions of BO which address multi-objective optimization.
	In particular, the multi-objective Bayesian optimization algorithm that we consider is ParEGO~\citep{Knowles2006}. 
	The main intuition of ParEGO is that at every iteration, the multiple objectives can be randomly scalarized into a single objective (via the augmented Tchebycheff function), which is subsequently optimized as in the standard Bayesian optimization algorithm (by creating a response surface, and then optimizing its acquisition function).
	For more information about multi-objective Bayesian optimization we refer the reader to~\citep{Wagner2010}.
    
\subsection{Contextual Bayesian Optimization}
	%
	Another special case of the optimization task of \eq{eq:optimization}, is contextual optimization.
	In contextual optimization, we assume that there are multiple correlated, but slightly different, tasks which we want to solve, and that they are identified by a context variable~$\context$.
	An example (which we investigate in \sec{sec:results:context1}) might be walking on inclined slopes, where the contextual variable is the angle of the slope.
	The contextual optimization can hence be formalized as 
	%
	\begin{align}
		\parameters^* = \maximize_{\parameters}\, \objfunc{\parameters,\context}\,,
	\end{align}
	%
	where for each context~$\context$, a potentially different set of parameters~$\parameters^*$ exists.
	The main advantage compared to treating each task independently is that, in contextual optimization, we can exploit the correlation between the tasks to generalize, and as a result quickly learn how to solve a new context.
	Specifically, in this paper we consider contextual Bayesian optimization (cBO)~\citep{Metzen2015} which extends the classic BO framework from \sec{sec:bg:BO}.
	Contextual Bayesian Optimization learns a joint model $\tilde{\objfuncNo}: \{\parameters,\context\} \rightarrow \objfunc{\parameters}$, but now, at every iteration the acquisition function is optimized with a constrained optimization where the context $\context$ is provided by the environment. 
	However, because the model jointly model the context-parameter space, experience learned in one context can be generalized to similar contexts. 
	By utilizing cBO, we will show in \sec{sec:results} that our microrobot can learn to walk (and generalize) to different environmental contexts such as walking uphill and curving.


	
%%%%%%%%%%%%%%%%%%%%%%%%%%%%%%%%%%%%%%%%%%%%%%%%%%%%%%%%%%%%%%%%%%%%%%%%%%%%%%%%

\section{Learning Locomotion Primitives for Path Planning}
\label{sec:approach}

	%
\begin{figure}[t]
	\centering
	\includegraphics[height=2.4cm]{fig/cpg2.pdf}
	\caption{Output of one vertical-horizontal oscillator pair in the CPG network, which corresponds to one leg on the robot. 
	The retraction phase of both motors occurs concurrently and rapidly in order to simulate the physical constraints on the actual physical microrobot.}
	\label{fig:cpg}
\end{figure}
% 
%
\begin{figure}[t]
	\centering
	\includegraphics[width=0.98\linewidth]{fig/gaits.pdf}
	\caption{Contact/swing patterns for different gaits.}
	\label{fig:gaits}
	\vspace{-8pt}
\end{figure}
%

We now present our novel approach to learn motor primitives for path planning.
This approach relies on the possibility of re-using the evaluations collected using cBO to convert the task into a multi-objective optimization problem.
We specifically consider a cBO task where we want to optimize the parameters $\parameters$ to reach different target positions $\context=\left[ \Delta x_\text{des}, \Delta y_\text{des} \right]$ (this setting is evaluated in \sec{sec:results:context2}).
The objective function in this case can be defined as the Euclidean distance 
%
\begin{align}
	\objfuncNo = \sqrt{\left(\Delta x_\text{des} -\Delta x_\text{obs}\right)^{2} + \left(\Delta y_\text{des} -\Delta y_\text{obs}\right)^{2}}\,,
\end{align}
%
where $\Delta x_\text{obs}, \Delta y_\text{obs}$ are the actual positions measured after evaluating a set of parameters.
The cBO model would map $\tilde{\objfuncNo}: \left[\parameters, \Delta x_\text{des}, \Delta y_\text{des}\right] \rightarrow \objfunc{\parameters}$. 
However, in order to compute $f$ it would need to measure $\Delta x_\text{obs}, \Delta y_\text{obs}$, effectively generating data of the form
%
\begin{align}
	\left[\parameters, \Delta x_\text{des}, \Delta y_\text{des}\right] \rightarrow \left[\Delta x_\text{obs}, \Delta y_\text{obs}, \objfunc{\parameters}\right]
\end{align}
%
We can now re-use the data generated from this contextual optimization to learn a motor primitive model in the form $g: \parameters \rightarrow \left[\Delta x_\text{obs}, \Delta y_\text{obs}\right]$.
The purpose of this learned model~$g$ is now to provide an estimate of the final displacement obtained for a set of parameters independently from the optimization process that generated it.
Once such a model is learned, we can use it to compute parameters that lead to the desired displacement $\Delta x^*_\text{obs}, \Delta y^*_\text{obs}$ by optimizing the parameters w.r.t. the output of the model
	%
\begin{align}
	\parameters^* = \maximize_{\parameters}\, z(g(\parameters))\,,
% 
\end{align}
%
where $z$ is a scalarization function of our choice (e.g., the Euclidean distance).
This is equivalent to learning a continuous function that generates motor primitives from the desired displacement.
It should be noted that this optimization is performed on the model $g$ and therefore does not require any physical interaction with the robot.
Moreover, we can optimize the parameters over a series of multiple displacements to obtain a path planning optimization. 
In \sec{sec:results:planning}, when performing path planning using the learned motor primitives we will employ a simple shooting method optimization which randomly samples multiple candidate parameters and selects the best outcome.


	
%%%%%%%%%%%%%%%%%%%%%%%%%%%%%%%%%%%%%%%%%%%%%%%%%%%%%%%%%%%%%%%%%%%%%%%%%%%%%%%%

\section{Experimental Simulation Results}
\label{sec:results}

	%
\begin{figure}[t]
  \centering
  \begin{subfigure}{0.49\linewidth}
	  \includegraphics[width=0.98\linewidth]{fig/tripod_normal.pdf}
	  \caption{Dual Tripod}
	  \label{fig:soo:1}
  \end{subfigure}
  \hfill  
  \begin{subfigure}{0.49\linewidth}
	  \includegraphics[width=0.98\linewidth]{fig/ripple_normal.pdf}
	  \caption{Ripple}
	  \label{fig:soo:2}
  \end{subfigure}
  \\
  \begin{subfigure}{0.49\linewidth}
	  \includegraphics[width=0.98\linewidth]{fig/wave_normal.pdf}
	  \caption{Wave}
	  \label{fig:soo:3}
  \end{subfigure}
  \hfill
  \begin{subfigure}{0.49\linewidth}
	  \includegraphics[width=0.98\linewidth]{fig/fourtwo_normal.pdf}
	  \caption{Four-Two}
	  \label{fig:soo:4}
  \end{subfigure}
  \caption{Learning curve for the four gaits (median and 65th percentile). We can see how, for all the gaits, BO learns to walk from scratch within 50 iterations. After the optimization, Dual Tripod and Ripple are the fastest gaits at $\sim \SI[per-mode=symbol]{1.1}{\centi\meter\per\second}$ and $\sim \SI[per-mode=symbol]{1.2}{\centi\meter\per\second}$ respectively.}
  \label{fig:soo}
\end{figure}
%
In this section, we discuss our controller implementation as well as the performance of our simulated microrobot on various locomotion tasks.
The code used for performing the simulation and videos of the various locomotion tasks are available online at \url{https://sites.google.com/view/learning-locomotion-primitives}.

\subsection{Controller Implementation}
	We built our controller following the setup described in \sec{sec:bg:cpg}, using a network of 12 coupled phase oscillators (one per motor).
	In order to translate the output of each of the oscillators into motor actuation, we calculate the oscillator outputs for each vertical-horizontal motor pair using the piecewise function
	%
	\begin{align}
		    \begin{cases}
			    x_{i} + r_{i}cos(\phi_{i}), x_{j} + r_{j}cos(\phi_{j}) &\text{if }\phi_{i}>\pi,\phi_{j}>\pi\,,\\
			    x_{i} + r_{i}, x_{j} + r_{j}cos(\phi_{j}) &\text{if }\phi_{i}\leq\pi,\phi_{j}>\pi\,,\\
			    x_{i} + r_{i}, x_{j} + r_{j} &\text{if }\phi_{i}\leq\pi,\phi_{j}\leq\pi\,,\\
			    x_{i} + r_{i}cos(\phi_{i}), x_{j} - r_{j} &\text{if }\phi_{i}\leq\pi,\phi_{j}>\pi\,,
		    \end{cases}
	\end{align}
	%
	where the $i$th oscillator outputs to its respective vertical motor and the $j$th oscillator outputs to its respective horizontal motor. 
	This allows us to discard the parts of the oscillator output that are not consistent with the physical constraints of the physical robot, since the actual leg actuators cannot partially retract (see \fig{fig:cpg}).
	We choose to mutually couple all six of the vertical oscillators (with a coupling weight of 4 to ensure quick convergence on stable limit cycles).
	We refer the reader to \cite{Crespi_2007} for a more comprehensive discussion of oscillator coupling in CPGs.
	Each of the horizontal oscillators are also coupled with their respective vertical oscillator in order to encapsulate the dynamics of each leg.
	We chose to implement four different gaits with the CPG -- tripod, ripple, wave, and four-two (see \fig{fig:gaits}). 
	For a more detailed description of these gaits we refer the reader to~\cite{Campos2010}.
	We use the same frequency and phase difference for the whole network in order to reduce the number of parameters and speed up the rate of convergence.
	We use two separate parameters for amplitude, each controlling the left and right set of legs respectively.
	This choice of parameters allows us to control the turning of the robot which is necessary for path planning and corrections for not walking straight.

    
\subsection{Learning to Walk Straight}
\label{sec:results:soo}

	We optimized the four gaits considered (i.e., dual tripod, ripple, wave, and four-two) using as our objective function the walking speed of the robot (measured as the distance traveled after $\SI{1}{\second}$).
	Since some gaits result in curved motions, we also penalized the speed objective with a term proportional to the drift from the axis of locomotion.
	The optimization used the 4 parameters outlined in \sec{sec:bg:cpg} and was repeated 50 times for each of the gaits. 
	In \fig{fig:soo}, we show the median and 65th percentiles of the best solution obtained so far in the trials.
	The results show that the optimizer was able to learn to walk from scratch within 50 iterations.
	Moreover, it can be noted that the optimized tripod and ripple are the fastest gaits at $\sim \SI[per-mode=symbol]{1.1}{\centi\meter\per\second}$ and $\sim \SI[per-mode=symbol]{1.2}{\centi\meter\per\second}$ respectively.
	
\subsection{Multi-objective Gait Optimization}
\label{sec:results:moo}
	%
	\begin{figure}[t]
	  \centering
	  \begin{subfigure}{0.49\linewidth}
		  \includegraphics[width=0.99\linewidth]{fig/Dual_Tripod.pdf}
		  \caption{Dual Tripod}
		  \label{fig:moo:1}
	  \end{subfigure}
	  \hfill  
	  \begin{subfigure}{0.49\linewidth}
		  \includegraphics[width=0.99\linewidth]{fig/Ripple.pdf}
		  \caption{Ripple}
		  \label{fig:moo:2}
	  \end{subfigure}
	  \\
	  \begin{subfigure}{0.49\linewidth}
		  \includegraphics[width=0.99\linewidth]{fig/Wave.pdf}
		  \caption{Wave}
		  \label{fig:moo:3}
	  \end{subfigure}
	  \hfill
	  \begin{subfigure}{0.49\linewidth}
		  \includegraphics[width=0.99\linewidth]{fig/Four-Two.pdf}
		  \caption{Four-Two}
		  \label{fig:moo:4}
	  \end{subfigure}
	  \caption{Performance measured for the four gaits, and the corresponding PFs. ParEGO is able to quickly explore the PF for each of our four gaits.}
	  \label{fig:moo}
	\end{figure}
	%

	In the previous simulation we only considered walking speed as our objective. 
	However, for practical gait design, energy efficiency is another objective of great interest, particularly when it comes to designing gaits for a microrobot with real energy restrictions. 
	For this reason, we now consider a multi-objective optimization setting and compare the different gaits w.r.t. both walking speed, and energy consumption.
	The energy consumption of the robot was computed by measuring the forces exerted by each of the 12 motors along the axis of actuation and calculating the power used to actuate the motors. 
	Since the retraction of the legs is spring powered, the energy input in the cycle is only during motor extension.
	Hence, we only consider the cost of extending the legs.
	With the mass of the robot and the time of each trial being held constant, we quantify the energy efficiency of a gait and estimate the cost of transport.
   
            %
	\begin{wrapfigure}{r}{0.52\linewidth} 
	\vspace{-12pt}
	  \centering
	  \includegraphics[width=\linewidth]{fig/all_pareto.pdf}
	  \caption{Comparison of the PFs obtained for the different gaits.}
	  \label{fig:moo:all}
	  \vspace{-10pt}
	\end{wrapfigure}
	% 
	We optimized the four gaits again with the same 4 parameters as the previous optimization, but this time using multi-objective Bayesian optimization with a budget of 50 iterations.
	    %
	\begin{figure}[t]
	  \centering
	  \includegraphics[width=0.95\linewidth]{fig/discovery2.pdf}
	  \caption{PF of the unrestrained gait optimization versus the best performance of the four nature-inspired gaits. The faster solutions outperform the fastest nature-inspired gaits, albeit with more energy expenditure. However, the inability of the optimizer to match the performance of the gaits at lower speeds within 1250 trials shows that the gait parametrization can help limit the search space to find better solutions easier. \textit{(top)} Pattern for two of the discovered gaits.}
	  \label{fig:moo:new}
	\end{figure}
	%
	In \fig{fig:moo} we can see the performance measured and Pareto fronts obtained for the different gaits.
	To better compare the PF from the different gaits, we also visualized just  the PFs together in \fig{fig:moo:all}. 
	From these results, we can see how the tripod gait dominates the other gaits for speed $<\SI[per-mode=symbol]{0.6}{\centi\meter\per\second}$, while Ripple dominates when the speed is $>\SI[per-mode=symbol]{0.6}{\centi\meter\per\second}$, hence giving a clear indication of which gait is preferable under different circumstances. 
    

\subsection{Discovering New Gaits with Multi-objective Optimization}

	In addition to optimizing the four nature-inspired gaits, we also tested multi-objective optimization on the walker without constraining to using predefined gaits.
	To parametrize the oscillator couplings, we thus discretized each gait into intervals of constant length.
	Within each of these intervals, we assume that each leg steps exactly once, keeping each of the oscillators in the CPG in phase with each other.
	This allows us to parametrize gaits by assigning each leg a point during each interval where it begins stepping.
	While this parametrization excludes certain gaits that cannot be expressed in this form, we leave the study of more sophisticated gait parameterizations for gait discovery to future works.
	
	The resulting multi-objective optimization task had 8 parameters (frequency, phase difference between horizontal and vertical motors, and the six gait coupling parameters).
	Due to the higher parameter dimensionality, and because this training was not intended for on-line training, we ran the optimization for 250 iterations in order to allow a more comprehensive exploration of the optimization space.
	We also repeated the optimization five times for a total of 1250 trials.
	In \fig{fig:moo:new} we can see the Pareto front for the resulting gaits.
	We found that the fastest discovered gaits were actually able to outperform the four nature-inspired gaits implemented by a substantial margin.
	Even while penalizing curved paths, the fastest discovered gait outperformed Ripple (the fastest nature-inspired gait we found) by almost $50\%$.
	However, for low-speed gaits, the nature inspired gaits out-perform the gaits produced by the unconstrained optimization, indicating the optimization did not yet fully converged to the optimal PF.
	
\subsection{Learning to Walk on Inclined Surfaces}
\label{sec:results:context1}
	%
	\begin{figure}[t]
	  \centering
	  \includegraphics[width=0.96\linewidth]{fig/contextual_generalized_mod.pdf}
	  \caption{Performance of the contextual policy (median and 65th percentile) for a wide range of inclines. The policy was trained only at 5, 10 and 15 degrees, but it was capable of generalizing smoothly to unseen inclinations. 
	  }
	  \label{fig:incline}
	\end{figure}
    
	\begin{figure}[t]
	  \centering
	  	\begin{subfigure}{0.49\linewidth}
      \centering
	  \includegraphics[width=\columnwidth]{fig/contextual_vs_normal.pdf}
	  \caption{Inclined surface.}
	  \label{fig:contextual:1}
	\end{subfigure}
\hfill
	\begin{subfigure}{0.49\linewidth}
	  \centering
	  \includegraphics[width=\linewidth]{fig/contextual_vs_normal_turning.pdf}
	  \caption{Curved trajectory.}
	  \label{fig:contextual:2}
	  	\end{subfigure}
\caption{Comparison between the optimization performance of a contextual optimizer and a normal optimizer for two different tasks: (a) walking on inclines (b) walking curved trajectories. In both cases, the contextual optimizer can leverage prior simulations to obtain high-performing gaits in fewer simulations.}
	   \vspace{-10pt}
       \end{figure}
       
	We now consider the case of contextual optimization and specifically the task of gait optimization for slopes with different inclinations.
	We framed learning to walk on inclined terrain as a contextual policy search, where the angle of the inclination is the context. 
	In this simulation, we decided to use Dual Tripod for our gait with mostly the same open parameters as the previous simulations.
    We used a single parameter to represent the amplitude for the entire network in order to keep the number of parameters low with the addition of a contextual variable, leaving us with 3 parameters and 1 contextual parameter.
	To respect real world constraints, where testing randomly sampled incline angles over a continuous interval can be excessively time-consuming, we chose at training time to perform simulations only from a small number of inclines: 5, 10, and 15 degrees.
	
	After optimizing the gaits for these three inclines over 50 iterations, we studied how the contextual optimizer is able to generalize across the context space by testing the performance of the contextual policy for a wide range of inclines.
	In \fig{fig:incline} we can see that the policy performs well on intermediary inclines and seems to smoothly interpolate between the training inclines as is desirable.
	The gradual decrease in performance as the inclines get steeper can be attributed to the increasing physical difficulty for climbing up steeper inclines.
	We also compared cBO against using standard BO to train the robot for an untested incline. 
	As shown in \fig{fig:contextual:1}, the contextual optimization was able to converge on optimal performance significantly faster than standard BO.
	This result demonstrate the ability of cBO to efficiently use data accumulated in previous contexts to quickly reach optimize gaits in new unseen contexts.

	
\subsection{Learning to Curve}
\label{sec:results:context2}
	
	Another useful task that can be framed as contextual optimization is learning motor primitives to walk curved trajectories for use in path planning.
	We used the same parameters as in \sec{sec:results:soo} and the contextual parameters in this case were the target displacements along both the x and y axes from the point of origin.
	In order to train particular trajectories, we selected five evenly spaced target points along the front quadrant of the field of vision.
	Since the primary objective was to reach the desired destination, we chose to use the distance of the final position to the target position as our sole objective function.
	We found that over 10 repetitions, the walker was able to accurately move and turn towards all of the target points within 250 iterations.
	In \fig{fig:contextual:2}, we compared the performance of cBO against standard BO on a previously unseen target position $(4\cos{\pi / 16}, 4\sin{\pi / 16})$.
	We found that, as in the case of inclinations, the contextual policy was able to learn the optimal parameters for a novel trajectory within very few iterations.


\subsection{Learning Motor Primitives for Path Planning}
\label{sec:results:planning}
    %
	\begin{figure}[t]
	  \centering
	  \includegraphics[width=\linewidth]{fig/turning_map_both_hor.pdf}
	  \caption{Comparison of the performances of cBO and our approach for learning motor primitives (using the same data). 
	  With the robot having an initial position of $(0,0)$, we evaluated the error between the desired position (indicated by the element of the grid) and the reached position.
	  Darker color indicates better target accuracy.
	  While cBO accurately learned trajectories near the training targets, it did not generalize well to unseen targets. 
	  In contrast, our approach had a more comprehensive coverage as it could leverage better information about the environment to improve generalization.}
	  \label{fig:pathing1}
	\end{figure}
	% 

	    %
	\begin{wrapfigure}{r}{0.50\linewidth} 
	\vspace{-10pt}
	  \centering
	  \includegraphics[width=\linewidth]{fig/contextual_path.pdf}
	  \caption{Path constructed using the locomotion primitives learned with our approach.}
	  \label{fig:pathing2}
	   \vspace{-8pt}
	\end{wrapfigure}
	% 
	In the previous simulation we learned motor primitives capable of walking curved trajectories.
    While the model handled trajectories near and between the targets quite well, the performance on trajectories well within the physical capabilities of the robot but not in proximity to the targets left much to be desired, as shown in \fig{fig:pathing1}.
	We now demonstrate how our approach presented in \sec{sec:approach} can be used to significantly improve the movement accuracy (compared to cBO using the same data), as well as how such motor primitives can be used to perform path planning.
    First, we reused the data from the previous simulation in order to reformulate the task as a multi-objective optimization as described in \sec{sec:approach}.
    Then, we used our trained model to sample 10,000 trajectories by randomly sampling from the parameter space.
    Out of all these trajectories, we selected the one with the smallest expected error subject to not walking through the wall.
    Evaluating the resulting sequence of motor primitives on the real system (\ie, the simulator) demonstrated that the expected trajectory was capable of navigating the maze, as shown in \fig{fig:pathing2}.



    
    
    



%%%%%%%%%%%%%%%%%%%%%%%%%%%%%%%%%%%%%%%%%%%%%%%%%%%%%%%%%%%%%%%%%%%%%%%%%%%%%%%%

\section{Conclusion}

	In this work, we propose sub-character tokenization and conduct comprehensive experiments to illustrate its advantage over existing tokenization methods. 
%
% The idea of SubChar tokenization is simple: we encode every Chinese character into a short sequence of symbols and then construct the vocabulary on the encoded sequences with sub-word tokenization. 
%
Compared to treating each individual character as a token (CharTokenizer) or directly running sub-word tokenization on the raw Chinese text (sub-word tokenizer), our SubChar tokenizers not only perform competitively on downstream NLU tasks, more importantly, they can be much more efficient and robust.
%
We conduct a series of ablation and analysis to understand the reasons why SubChar tokenizers are more efficient, as well as the impact of linguistic and non-linguistic encoding.
%
Given the advantages of our SubChar tokenizers, we believe that they are better alternatives to all existing Chinese tokenizers, especially in applications where efficiency and robustness are critical. 
%
% We will release well-documented code, tokenizers, and pretrained models for easy usage. 
%
It is possible that our approach can be useful for other morphologically poor languages and more complicated methods could be developed based on SubChar tokenization for even better performance. We leave these interesting directions for future exploration.
%
On a broader level, our work makes an important attempt in developing more tailored methods for a language drastically different from English with promising results. We believe that this is a crucial future direction for the community given the language diversity in the world.
%
We hope that our work can inspire more such work in order to benefit language technology users from different countries and cultures.


% In this paper, we have explored three linguistically informed tokenization methods motivated by the unique linguistic characteristics of the Chinese writing system. Specifically, we find that pronunciation-based and glyph-based tokenizers can match or outperform the conventional Chinese tokenizers and Chinese word segmentation is not a useful addition for the tokenizer.
% Moreover, we find that our glyph-based tokenizers achieve large gains on noisy input as compared to the baselines, while our pronunciation-based tokenizers obtain limited success. This highlights the potential advantage of our proposed methods in real-life scenarios with noisy data.
% We believe that our work sets an important example of exploiting the unique linguistic property of a language beyond English to develop more tailored techniques, which should be an important direction for the global NLP community. 




%%%%%%%%%%%%%%%%%%%%%%%%%%%%%%%%%%%%%%%%%%%%%%%%%%%%%%%%%%%%%%%%%%%%%%%%%%%%%%%%

% \section*{APPENDIX}
% 
% Appendixes should appear before the acknowledgment.

%%%%%%%%%%%%%%%%%%%%%%%%%%%%%%%%%%%%%%%%%%%%%%%%%%%%%%%%%%%%%%%%%%%%%%%%%%%%%%%%

\bibliographystyle{IEEEtran}
\bibliography{paper-nanorobots}  % .bib


\end{document}
