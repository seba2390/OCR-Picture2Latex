\documentclass[letterpaper, 10 pt, journal, twoside]{IEEEtran} 

\IEEEoverridecommandlockouts                              % This command is only needed if 
                                                          % you want to use the \thanks command

\title{Learning Flexible and Reusable Locomotion Primitives for a Microrobot}
\markboth{IEEE Robotics and Automation Letters. Preprint Version. Accepted January, 2018}
{Yang \MakeLowercase{\textit{et al.}}: Learning Flexible and Reusable Locomotion Primitives for a Microrobot}  
% Make room for more info lines in the \author command  
\author{Brian Yang, Grant Wang, Roberto Calandra, Daniel Contreras, Sergey Levine, and Kristofer Pister%
\thanks{Manuscript received: September, 09, 2017; Revised January, 21, 2018; Accepted January, 27, 2018.}%Use only for final RAL version
\thanks{This paper was recommended for publication by Editor Yu Sun upon evaluation of the Associate Editor and Reviewers' comments. 
This work was supported by the Berkeley Sensor and Actuator Center, and by Berkeley DeepDrive. \textit{(Brian Yang and Grant Wang contributed equally to this work.) (Corresponding author: \href{mailto:roberto.calandra@berkeley.edu}{Roberto Calandra})}}
% Equally contributed authors according to IEEE guidelines: https://www.ieee.org/publications_standards/publications/journmag/online_style_manual-10292015.pdf
\thanks{All the authors are with the Department of Electrical Engineering and Computer Sciences, University of California, Berkeley, USA
        {\tt\small \{brianhyang, grant.wang5, roberto.calandra, dscontreras, ksjp\}@berkeley.edu, svlevine@eecs.berkeley.edu}}%
\thanks{Digital Object Identifier (DOI): 10.1109/LRA.2018.2806083}
}



%\usepackage[OT1,T1]{fontenc}

\usepackage[numbers,sort&compress]{natbib}
\renewcommand{\bibfont}{\footnotesize}
%\usepackage{cite}
%\usepackage{mystyle}
%%%%%%%%%%%%%%%%%%%%%%%%%%%%%%%%%%%%
\makeatletter

\usepackage{etex}

%%% Review %%%

\usepackage{zref-savepos}

\newcounter{mnote}%[page]
\renewcommand{\themnote}{p.\thepage\;$\langle$\arabic{mnote}$\rangle$}

\def\xmarginnote{%
  \xymarginnote{\hskip -\marginparsep \hskip -\marginparwidth}}

\def\ymarginnote{%
  \xymarginnote{\hskip\columnwidth \hskip\marginparsep}}

\long\def\xymarginnote#1#2{%
\vadjust{#1%
\smash{\hbox{{%
        \hsize\marginparwidth
        \@parboxrestore
        \@marginparreset
\footnotesize #2}}}}}

\def\mnoteson{%
\gdef\mnote##1{\refstepcounter{mnote}\label{##1}%
  \zsavepos{##1}%
  \ifnum20432158>\number\zposx{##1}%
  \xmarginnote{{\color{blue}\bf $\langle$\arabic{mnote}$\rangle$}}% 
  \else
  \ymarginnote{{\color{blue}\bf $\langle$\arabic{mnote}$\rangle$}}%
  \fi%
}
  }
\gdef\mnotesoff{\gdef\mnote##1{}}
\mnoteson
\mnotesoff








%%% Layout %%%

% \usepackage{geometry} % override layout
% \geometry{tmargin=2.5cm,bmargin=m2.5cm,lmargin=3cm,rmargin=3cm}
% \setlength{\pdfpagewidth}{8.5in} % overrides default pdftex paper size
% \setlength{\pdfpageheight}{11in}

\newlength{\mywidth}

%%% Conventions %%%

% References
\newcommand{\figref}[1]{Fig.~\ref{#1}}
\newcommand{\defref}[1]{Definition~\ref{#1}}
\newcommand{\tabref}[1]{Table~\ref{#1}}
% general
%\usepackage{ifthen,nonfloat,subfigure,rotating,array,framed}
\usepackage{framed}
%\usepackage{subfigure}
\usepackage{subcaption}
\usepackage{comment}
%\specialcomment{nb}{\begingroup \noindent \framed\textbf{n.b.\ }}{\endframed\endgroup}
%%\usepackage{xtab,arydshln,multirow}
% topcaption defined in xtab. must load nonfloat before xtab
%\PassOptionsToPackage{svgnames,dvipsnames}{xcolor}
\usepackage[svgnames,dvipsnames]{xcolor}
%\definecolor{myblue}{rgb}{.8,.8,1}
%\definecolor{umbra}{rgb}{.8,.8,.5}
%\newcommand*\mybluebox[1]{%
%  \colorbox{myblue}{\hspace{1em}#1\hspace{1em}}}
\usepackage[all]{xy}
%\usepackage{pstricks,pst-node}
\usepackage{tikz}
\usetikzlibrary{positioning,matrix,through,calc,arrows,fit,shapes,decorations.pathreplacing,decorations.markings,decorations.text}

\tikzstyle{block} = [draw,fill=blue!20,minimum size=2em]

% allow prefix to scope name
\tikzset{%
	prefix node name/.code={%
		\tikzset{%
			name/.code={\edef\tikz@fig@name{#1 ##1}}
		}%
	}%
}


\@ifpackagelater{tikz}{2013/12/01}{
	\newcommand{\convexpath}[2]{
		[create hullcoords/.code={
			\global\edef\namelist{#1}
			\foreach [count=\counter] \nodename in \namelist {
				\global\edef\numberofnodes{\counter}
				\coordinate (hullcoord\counter) at (\nodename);
			}
			\coordinate (hullcoord0) at (hullcoord\numberofnodes);
			\pgfmathtruncatemacro\lastnumber{\numberofnodes+1}
			\coordinate (hullcoord\lastnumber) at (hullcoord1);
		}, create hullcoords ]
		($(hullcoord1)!#2!-90:(hullcoord0)$)
		\foreach [evaluate=\currentnode as \previousnode using \currentnode-1,
		evaluate=\currentnode as \nextnode using \currentnode+1] \currentnode in {1,...,\numberofnodes} {
			let \p1 = ($(hullcoord\currentnode) - (hullcoord\previousnode)$),
			\n1 = {atan2(\y1,\x1) + 90},
			\p2 = ($(hullcoord\nextnode) - (hullcoord\currentnode)$),
			\n2 = {atan2(\y2,\x2) + 90},
			\n{delta} = {Mod(\n2-\n1,360) - 360}
			in 
			{arc [start angle=\n1, delta angle=\n{delta}, radius=#2]}
			-- ($(hullcoord\nextnode)!#2!-90:(hullcoord\currentnode)$) 
		}
	}
}{
	\newcommand{\convexpath}[2]{
		[create hullcoords/.code={
			\global\edef\namelist{#1}
			\foreach [count=\counter] \nodename in \namelist {
				\global\edef\numberofnodes{\counter}
				\coordinate (hullcoord\counter) at (\nodename);
			}
			\coordinate (hullcoord0) at (hullcoord\numberofnodes);
			\pgfmathtruncatemacro\lastnumber{\numberofnodes+1}
			\coordinate (hullcoord\lastnumber) at (hullcoord1);
		}, create hullcoords ]
		($(hullcoord1)!#2!-90:(hullcoord0)$)
		\foreach [evaluate=\currentnode as \previousnode using \currentnode-1,
		evaluate=\currentnode as \nextnode using \currentnode+1] \currentnode in {1,...,\numberofnodes} {
			let \p1 = ($(hullcoord\currentnode) - (hullcoord\previousnode)$),
			\n1 = {atan2(\x1,\y1) + 90},
			\p2 = ($(hullcoord\nextnode) - (hullcoord\currentnode)$),
			\n2 = {atan2(\x2,\y2) + 90},
			\n{delta} = {Mod(\n2-\n1,360) - 360}
			in 
			{arc [start angle=\n1, delta angle=\n{delta}, radius=#2]}
			-- ($(hullcoord\nextnode)!#2!-90:(hullcoord\currentnode)$) 
		}
	}
}

% circle around nodes

% typsetting math
\usepackage{qsymbols,amssymb,mathrsfs}
\usepackage{amsmath}
\usepackage[standard,thmmarks]{ntheorem}
\theoremstyle{plain}
\theoremsymbol{\ensuremath{_\vartriangleleft}}
\theorembodyfont{\itshape}
\theoremheaderfont{\normalfont\bfseries}
\theoremseparator{}
\newtheorem{Claim}{Claim}
\newtheorem{Subclaim}{Subclaim}
\theoremstyle{nonumberplain}
\theoremheaderfont{\scshape}
\theorembodyfont{\normalfont}
\theoremsymbol{\ensuremath{_\blacktriangleleft}}
\newtheorem{Subproof}{Proof}

\theoremnumbering{arabic}
\theoremstyle{plain}
\usepackage{latexsym}
\theoremsymbol{\ensuremath{_\Box}}
\theorembodyfont{\itshape}
\theoremheaderfont{\normalfont\bfseries}
\theoremseparator{}
\newtheorem{Conjecture}{Conjecture}

\theorembodyfont{\upshape}
\theoremprework{\bigskip\hrule}
\theorempostwork{\hrule\bigskip}
\newtheorem{Condition}{Condition}%[section]


%\RequirePckage{amsmath} loaded by empheq
\usepackage[overload]{empheq} % no \intertext and \displaybreak
%\usepackage{breqn}

\let\iftwocolumn\if@twocolumn
\g@addto@macro\@twocolumntrue{\let\iftwocolumn\if@twocolumn}
\g@addto@macro\@twocolumnfalse{\let\iftwocolumn\if@twocolumn}

%\empheqset{box=\mybluebox}
%\usepackage{mathtools}      % to polish math typsetting, loaded
%                                % by empeq
\mathtoolsset{showonlyrefs=false,showmanualtags}
\let\underbrace\LaTeXunderbrace % adapt spacing to font sizes
\let\overbrace\LaTeXoverbrace
\renewcommand{\eqref}[1]{\textup{(\refeq{#1})}} % eqref was not allowed in
                                       % sub/super-scripts
\newtagform{brackets}[]{(}{)}   % new tags for equations
\usetagform{brackets}
% defined commands:
% \shortintertext{}, dcases*, \cramped, \smashoperator[]{}

\usepackage[Smaller]{cancel}
\renewcommand{\CancelColor}{\color{Red}}
%\newcommand\hcancel[2][black]{\setbox0=\hbox{#2}% colored horizontal cross
%  \rlap{\raisebox{.45\ht0}{\color{#1}\rule{\wd0}{1pt}}}#2}



\usepackage{graphicx,psfrag}
\graphicspath{{figure/}{image/}} % Search path of figures

% for tabular
\usepackage{diagbox} % \backslashbox{}{} for slashed entries
%\usepackage{threeparttable} % threeparttable, \tnote{},
                                % tablenotes, and \item[]
%\usepackage{colortab} % \cellcolor[gray]{0.9},
%\rowcolor,\columncolor,
%\usepackage{colortab} % \LCC \gray & ...  \ECC \\

% typesetting codes
%\usepackage{maple2e} % need to use \char29 for ^
\usepackage{algorithm2e}
\usepackage{listings} 
\lstdefinelanguage{Maple}{
  morekeywords={proc,module,end, for,from,to,by,while,in,do,od
    ,if,elif,else,then,fi ,use,try,catch,finally}, sensitive,
  morecomment=[l]\#,
  morestring=[b]",morestring=[b]`}[keywords,comments,strings]
\lstset{
  basicstyle=\scriptsize,
  keywordstyle=\color{ForestGreen}\bfseries,
  commentstyle=\color{DarkBlue},
  stringstyle=\color{DimGray}\ttfamily,
  texcl
}
%%% New fonts %%%
\DeclareMathAlphabet{\mathpzc}{OT1}{pzc}{m}{it}
\usepackage{upgreek} % \upalpha,\upbeta, ...
%\usepackage{bbold}   % blackboard math
\usepackage{dsfont}  % \mathds

%%% Macros for multiple definitions %%%

% example usage:
% \multi{M}{\boldsymbol{#1}}  % defines \multiM
% \multi ABC.                 % defines \MA \MB and \MC as
%                             % \boldsymbol{A}, \boldsymbol{B} and
%                             % \boldsymbol{C} respectively.
% 
%  The last period '.' is necessary to terminate the macro expansion.
%
% \multi*{M}{\boldsymbol{#1}} % defines \multiM and \M
% \M{A}                       % expands to \boldsymbol{A}

\def\multi@nostar#1#2{%
  \expandafter\def\csname multi#1\endcsname##1{%
    \if ##1.\let\next=\relax \else
    \def\next{\csname multi#1\endcsname}     
    %\expandafter\def\csname #1##1\endcsname{#2}
    \expandafter\newcommand\csname #1##1\endcsname{#2}
    \fi\next}}

\def\multi@star#1#2{%
  \expandafter\def\csname #1\endcsname##1{#2}
  \multi@nostar{#1}{#2}
}

\newcommand{\multi}{%
  \@ifstar \multi@star \multi@nostar}

%%% new alphabets %%%

\multi*{rm}{\mathrm{#1}}
\multi*{mc}{\mathcal{#1}}
\multi*{op}{\mathop {\operator@font #1}}
% \multi*{op}{\operatorname{#1}}
\multi*{ds}{\mathds{#1}}
\multi*{set}{\mathcal{#1}}
\multi*{rsfs}{\mathscr{#1}}
\multi*{pz}{\mathpzc{#1}}
\multi*{M}{\boldsymbol{#1}}
\multi*{R}{\mathsf{#1}}
\multi*{RM}{\M{\R{#1}}}
\multi*{bb}{\mathbb{#1}}
\multi*{td}{\tilde{#1}}
\multi*{tR}{\tilde{\mathsf{#1}}}
\multi*{trM}{\tilde{\M{\R{#1}}}}
\multi*{tset}{\tilde{\mathcal{#1}}}
\multi*{tM}{\tilde{\M{#1}}}
\multi*{baM}{\bar{\M{#1}}}
\multi*{ol}{\overline{#1}}

\multirm  ABCDEFGHIJKLMNOPQRSTUVWXYZabcdefghijklmnopqrstuvwxyz.
\multiol  ABCDEFGHIJKLMNOPQRSTUVWXYZabcdefghijklmnopqrstuvwxyz.
\multitR   ABCDEFGHIJKLMNOPQRSTUVWXYZabcdefghijklmnopqrstuvwxyz.
\multitd   ABCDEFGHIJKLMNOPQRSTUVWXYZabcdefghijklmnopqrstuvwxyz.
\multitset ABCDEFGHIJKLMNOPQRSTUVWXYZabcdefghijklmnopqrstuvwxyz.
\multitM   ABCDEFGHIJKLMNOPQRSTUVWXYZabcdefghijklmnopqrstuvwxyz.
\multibaM   ABCDEFGHIJKLMNOPQRSTUVWXYZabcdefghijklmnopqrstuvwxyz.
\multitrM   ABCDEFGHIJKLMNOPQRSTUVWXYZabcdefghijklmnopqrstuvwxyz.
\multimc   ABCDEFGHIJKLMNOPQRSTUVWXYZabcdefghijklmnopqrstuvwxyz.
\multiop   ABCDEFGHIJKLMNOPQRSTUVWXYZabcdefghijklmnopqrstuvwxyz.
\multids   ABCDEFGHIJKLMNOPQRSTUVWXYZabcdefghijklmnopqrstuvwxyz.
\multiset  ABCDEFGHIJKLMNOPQRSTUVWXYZabcdefghijklmnopqrstuvwxyz.
\multirsfs ABCDEFGHIJKLMNOPQRSTUVWXYZabcdefghijklmnopqrstuvwxyz.
\multipz   ABCDEFGHIJKLMNOPQRSTUVWXYZabcdefghijklmnopqrstuvwxyz.
\multiM    ABCDEFGHIJKLMNOPQRSTUVWXYZabcdefghijklmnopqrstuvwxyz.
\multiR    ABCDEFGHIJKL NO QR TUVWXYZabcd fghijklmnopqrstuvwxyz.
\multibb   ABCDEFGHIJKLMNOPQRSTUVWXYZabcdefghijklmnopqrstuvwxyz.
\multiRM   ABCDEFGHIJKLMNOPQRSTUVWXYZabcdefghijklmnopqrstuvwxyz.
\newcommand{\RRM}{\R{M}}
\newcommand{\RRP}{\R{P}}
\newcommand{\RRe}{\R{e}}
\newcommand{\RRS}{\R{S}}
%%% new symbols %%%

%\newcommand{\dotgeq}{\buildrel \textstyle  .\over \geq}
%\newcommand{\dotleq}{\buildrel \textstyle  .\over \leq}
\newcommand{\dotleq}{\buildrel \textstyle  .\over {\smash{\lower
      .2ex\hbox{\ensuremath\leqslant}}\vphantom{=}}}
\newcommand{\dotgeq}{\buildrel \textstyle  .\over {\smash{\lower
      .2ex\hbox{\ensuremath\geqslant}}\vphantom{=}}}

\DeclareMathOperator*{\argmin}{arg\,min}
\DeclareMathOperator*{\argmax}{arg\,max}

%%% abbreviations %%%

% commands
\newcommand{\esm}{\ensuremath}

% environments
\newcommand{\bM}{\begin{bmatrix}}
\newcommand{\eM}{\end{bmatrix}}
\newcommand{\bSM}{\left[\begin{smallmatrix}}
\newcommand{\eSM}{\end{smallmatrix}\right]}
\renewcommand*\env@matrix[1][*\c@MaxMatrixCols c]{%
  \hskip -\arraycolsep
  \let\@ifnextchar\new@ifnextchar
  \array{#1}}



% sets of number
\newqsymbol{`N}{\mathbb{N}}
\newqsymbol{`R}{\mathbb{R}}
\newqsymbol{`P}{\mathbb{P}}
\newqsymbol{`Z}{\mathbb{Z}}

% symbol short cut
\newqsymbol{`|}{\mid}
% use \| for \parallel
\newqsymbol{`8}{\infty}
\newqsymbol{`1}{\left}
\newqsymbol{`2}{\right}
\newqsymbol{`6}{\partial}
\newqsymbol{`0}{\emptyset}
\newqsymbol{`-}{\leftrightarrow}
\newqsymbol{`<}{\langle}
\newqsymbol{`>}{\rangle}

%%% new operators / functions %%%

\newcommand{\sgn}{\operatorname{sgn}}
\newcommand{\Var}{\op{Var}}
\newcommand{\diag}{\operatorname{diag}}
\newcommand{\erf}{\operatorname{erf}}
\newcommand{\erfc}{\operatorname{erfc}}
\newcommand{\erfi}{\operatorname{erfi}}
\newcommand{\adj}{\operatorname{adj}}
\newcommand{\supp}{\operatorname{supp}}
\newcommand{\E}{\opE\nolimits}
\newcommand{\T}{\intercal}
% requires mathtools
% \abs,\abs*,\abs[<size_cmd:\big,\Big,\bigg,\Bigg etc.>]
\DeclarePairedDelimiter\abs{\lvert}{\rvert} 
\DeclarePairedDelimiter\norm{\lVert}{\rVert}
\DeclarePairedDelimiter\ceil{\lceil}{\rceil}
\DeclarePairedDelimiter\floor{\lfloor}{\rfloor}
\DeclarePairedDelimiter\Set{\{}{\}}
\newcommand{\imod}[1]{\allowbreak\mkern10mu({\operator@font mod}\,\,#1)}

%%% new formats %%%
\newcommand{\leftexp}[2]{{\vphantom{#2}}^{#1}{#2}}


% non-floating figures that can be put inside tables
\newenvironment{nffigure}[1][\relax]{\vskip \intextsep
  \noindent\minipage{\linewidth}\def\@captype{figure}}{\endminipage\vskip \intextsep}

\newcommand{\threecols}[3]{
\hbox to \textwidth{%
      \normalfont\rlap{\parbox[b]{\textwidth}{\raggedright#1\strut}}%
        \hss\parbox[b]{\textwidth}{\centering#2\strut}\hss
        \llap{\parbox[b]{\textwidth}{\raggedleft#3\strut}}%
    }% hbox 
}

\newcommand{\reason}[2][\relax]{
  \ifthenelse{\equal{#1}{\relax}}{
    \left(\text{#2}\right)
  }{
    \left(\parbox{#1}{\raggedright #2}\right)
  }
}

\newcommand{\marginlabel}[1]
{\mbox[]\marginpar{\color{ForestGreen} \sffamily \small \raggedright\hspace{0pt}#1}}


% up-tag an equation
\newcommand{\utag}[2]{\mathop{#2}\limits^{\text{(#1)}}}
\newcommand{\uref}[1]{(#1)}


% Notation table

\newcommand{\Hline}{\noalign{\vskip 0.1in \hrule height 0.1pt \vskip
    0.1in}}
  
\def\Malign#1{\tabskip=0in
  \halign to\columnwidth{
    \ensuremath{\displaystyle ##}\hfil
    \tabskip=0in plus 1 fil minus 1 fil
    &
    \parbox[t]{0.8\columnwidth}{##}
    \tabskip=0in
    \cr #1}}


%%%%%%%%%%%%%%%%%%%%%%%%%%%%%%%%%%%%%%%%%%%%%%%%%%%%%%%%%%%%%%%%%%%
% MISCELLANEOUS

% Modification from braket.sty by Donald Arseneau
% Command defined is: \extendvert{ }
% The "small versions" use fixed-size brackets independent of their
% contents, whereas the expand the first vertical line '|' or '\|' to
% envelop the content
\let\SavedDoubleVert\relax
\let\protect\relax
{\catcode`\|=\active
  \xdef\extendvert{\protect\expandafter\noexpand\csname extendvert \endcsname}
  \expandafter\gdef\csname extendvert \endcsname#1{\mskip-5mu \left.%
      \ifx\SavedDoubleVert\relax \let\SavedDoubleVert\|\fi
     \:{\let\|\SetDoubleVert
       \mathcode`\|32768\let|\SetVert
     #1}\:\right.\mskip-5mu}
}
\def\SetVert{\@ifnextchar|{\|\@gobble}% turn || into \|
    {\egroup\;\mid@vertical\;\bgroup}}
\def\SetDoubleVert{\egroup\;\mid@dblvertical\;\bgroup}

% If the user is using e-TeX with its \middle primitive, use that for
% verticals instead of \vrule.
%
\begingroup
 \edef\@tempa{\meaning\middle}
 \edef\@tempb{\string\middle}
\expandafter \endgroup \ifx\@tempa\@tempb
 \def\mid@vertical{\middle|}
 \def\mid@dblvertical{\middle\SavedDoubleVert}
\else
 \def\mid@vertical{\mskip1mu\vrule\mskip1mu}
 \def\mid@dblvertical{\mskip1mu\vrule\mskip2.5mu\vrule\mskip1mu}
\fi

%%%%%%%%%%%%%%%%%%%%%%%%%%%%%%%%%%%%%%%%%%%%%%%%%%%%%%%%%%%%%%%%

\makeatother

%%%%%%%%%%%%%%%%%%%%%%%%%%%%%%%%%%%%

\usepackage{ctable}
\usepackage{fouridx}
%\usepackage{calc}
\usepackage{framed}
\usetikzlibrary{positioning,matrix}

\usepackage{paralist}
%\usepackage{refcheck}
\usepackage{enumerate}

\usepackage[normalem]{ulem}
\newcommand{\Ans}[1]{\uuline{\raisebox{.15em}{#1}}}



\numberwithin{equation}{section}
\makeatletter
\@addtoreset{equation}{section}
\renewcommand{\theequation}{\arabic{section}.\arabic{equation}}
\@addtoreset{Theorem}{section}
\renewcommand{\theTheorem}{\arabic{section}.\arabic{Theorem}}
\@addtoreset{Lemma}{section}
\renewcommand{\theLemma}{\arabic{section}.\arabic{Lemma}}
\@addtoreset{Corollary}{section}
\renewcommand{\theCorollary}{\arabic{section}.\arabic{Corollary}}
\@addtoreset{Example}{section}
\renewcommand{\theExample}{\arabic{section}.\arabic{Example}}
\@addtoreset{Remark}{section}
\renewcommand{\theRemark}{\arabic{section}.\arabic{Remark}}
\@addtoreset{Proposition}{section}
\renewcommand{\theProposition}{\arabic{section}.\arabic{Proposition}}
\@addtoreset{Definition}{section}
\renewcommand{\theDefinition}{\arabic{section}.\arabic{Definition}}
\@addtoreset{Claim}{section}
\renewcommand{\theClaim}{\arabic{section}.\arabic{Claim}}
\@addtoreset{Subclaim}{Theorem}
\renewcommand{\theSubclaim}{\theTheorem\Alph{Subclaim}}
\makeatother

\newcommand{\Null}{\op{Null}}
%\newcommand{\T}{\op{T}\nolimits}
\newcommand{\Bern}{\op{Bern}\nolimits}
\newcommand{\odd}{\op{odd}}
\newcommand{\even}{\op{even}}
\newcommand{\Sym}{\op{Sym}}
\newcommand{\si}{s_{\op{div}}}
\newcommand{\sv}{s_{\op{var}}}
\newcommand{\Wtyp}{W_{\op{typ}}}
\newcommand{\Rco}{R_{\op{CO}}}
\newcommand{\Tm}{\op{T}\nolimits}
\newcommand{\JGK}{J_{\op{GK}}}

\newcommand{\diff}{\mathrm{d}}

\newenvironment{lbox}{
  \setlength{\FrameSep}{1.5mm}
  \setlength{\FrameRule}{0mm}
  \def\FrameCommand{\fboxsep=\FrameSep \fcolorbox{black!20}{white}}%
  \MakeFramed {\FrameRestore}}%
{\endMakeFramed}

\newenvironment{ybox}{
	\setlength{\FrameSep}{1.5mm}
	\setlength{\FrameRule}{0mm}
  \def\FrameCommand{\fboxsep=\FrameSep \fcolorbox{black!10}{yellow!8}}%
  \MakeFramed {\FrameRestore}}%
{\endMakeFramed}

\newenvironment{gbox}{
	\setlength{\FrameSep}{1.5mm}
\setlength{\FrameRule}{0mm}
  \def\FrameCommand{\fboxsep=\FrameSep \fcolorbox{black!10}{green!8}}%
  \MakeFramed {\FrameRestore}}%
{\endMakeFramed}

\newenvironment{bbox}{
	\setlength{\FrameSep}{1.5mm}
\setlength{\FrameRule}{0mm}
  \def\FrameCommand{\fboxsep=\FrameSep \fcolorbox{black!10}{blue!8}}%
  \MakeFramed {\FrameRestore}}%
{\endMakeFramed}

\newenvironment{yleftbar}{%
  \def\FrameCommand{{\color{yellow!20}\vrule width 3pt} \hspace{10pt}}%
  \MakeFramed {\advance\hsize-\width \FrameRestore}}%
 {\endMakeFramed}

\newcommand{\tbox}[2][\relax]{
 \setlength{\FrameSep}{1.5mm}
  \setlength{\FrameRule}{0mm}
  \begin{ybox}
    \noindent\underline{#1:}\newline
    #2
  \end{ybox}
}

\newcommand{\pbox}[2][\relax]{
  \setlength{\FrameSep}{1.5mm}
 \setlength{\FrameRule}{0mm}
  \begin{gbox}
    \noindent\underline{#1:}\newline
    #2
  \end{gbox}
}

\newcommand{\gtag}[1]{\text{\color{green!50!black!60} #1}}
\let\labelindent\relax
\usepackage{enumitem}

%%%%%%%%%%%%%%%%%%%%%%%%%%%%%%%%%%%%
% fix subequations
% http://tex.stackexchange.com/questions/80134/nesting-subequations-within-align
%%%%%%%%%%%%%%%%%%%%%%%%%%%%%%%%%%%%

\usepackage{etoolbox}

% let \theparentequation use the same definition as equation
\let\theparentequation\theequation
% change every occurence of "equation" to "parentequation"
\patchcmd{\theparentequation}{equation}{parentequation}{}{}

\renewenvironment{subequations}[1][]{%              optional argument: label-name for (first) parent equation
	\refstepcounter{equation}%
	%  \def\theparentequation{\arabic{parentequation}}% we patched it already :)
	\setcounter{parentequation}{\value{equation}}%    parentequation = equation
	\setcounter{equation}{0}%                         (sub)equation  = 0
	\def\theequation{\theparentequation\alph{equation}}% 
	\let\parentlabel\label%                           Evade sanitation performed by amsmath
	\ifx\\#1\\\relax\else\label{#1}\fi%               #1 given: \label{#1}, otherwise: nothing
	\ignorespaces
}{%
	\setcounter{equation}{\value{parentequation}}%    equation = subequation
	\ignorespacesafterend
}

\newcommand*{\nextParentEquation}[1][]{%            optional argument: label-name for (first) parent equation
	\refstepcounter{parentequation}%                  parentequation++
	\setcounter{equation}{0}%                         equation = 0
	\ifx\\#1\\\relax\else\parentlabel{#1}\fi%         #1 given: \label{#1}, otherwise: nothing
}

% hyperlink
\PassOptionsToPackage{breaklinks,letterpaper,hyperindex=true,backref=false,bookmarksnumbered,bookmarksopen,linktocpage,colorlinks,linkcolor=BrickRed,citecolor=OliveGreen,urlcolor=Blue,pdfstartview=FitH}{hyperref}
\usepackage{hyperref}

% makeindex style
\newcommand{\indexmain}[1]{\textbf{\hyperpage{#1}}}
\newcommand{\citep}[1]{\cite{#1}}

\begin{document}

\maketitle

%%%%%%%%%%%%%%%%%%%%%%%%%%%%%%%%%%%%%%%%%%%%%%%%%%%%%%%%%%%%%%%%%%%%%%%%%%%%%%%%

\begin{abstract}
	\begin{abstract}

Visual perception tasks often require vast amounts of labelled data, including 3D poses and image space segmentation masks. The process of creating such training data sets can prove difficult or time-intensive to scale up to efficacy for general use. Consider the task of pose estimation for rigid objects. Deep neural network based approaches have shown good performance when trained on large, public datasets. However, adapting these networks for other novel objects, or fine-tuning existing models for different environments, requires significant time investment to generate newly labelled instances. Towards this end, we propose ProgressLabeller as a method for more efficiently generating large amounts of 6D pose training data from color images sequences for custom scenes in a scalable manner. ProgressLabeller is intended to also support transparent or translucent objects, for which the previous methods based on depth dense reconstruction will fail.
We demonstrate the effectiveness of ProgressLabeller by rapidly create a dataset of over 1M samples with which we fine-tune a state-of-the-art pose estimation network in order to markedly improve the downstream robotic grasp success rates. Progresslabeller is open-source at \href{https://github.com/huijieZH/ProgressLabeller}{https://github.com/huijieZH/ProgressLabeller}

\end{abstract}
\end{abstract}
\begin{IEEEkeywords}
Learning and Adaptive Systems; Micro/Nano Robots; Legged Robots
\end{IEEEkeywords}

%%%%%%%%%%%%%%%%%%%%%%%%%%%%%%%%%%%%%%%%%%%%%%%%%%%%%%%%%%%%%%%%%%%%%%%%%%%%%%%%

\section{Introduction}

	\begin{figure}[t]
\begin{center}
   \includegraphics[width=1.0\linewidth]{figures/nas_comp_v3}
\end{center}
   \vspace{-4mm}
   \caption{The comparison between NetAdaptV2 and related works. The number above a marker is the corresponding total search time measured on NVIDIA V100 GPUs.}
\label{fig:nas_comparison}
\end{figure}

\section{Introduction}
\label{sec:introduction}

Neural architecture search (NAS) applies machine learning to automatically discover deep neural networks (DNNs) with better performance (e.g., better accuracy-latency trade-offs) by sampling the search space, which is the union of all discoverable DNNs. The search time is one key metric for NAS algorithms, which accounts for three steps: 1) training a \emph{super-network}, whose weights are shared by all the DNNs in the search space and trained by minimizing the loss across them, 2) training and evaluating sampled DNNs (referred to as \emph{samples}), and 3) training the discovered DNN. Another important metric for NAS is whether it supports non-differentiable search metrics such as hardware metrics (e.g., latency and energy). Incorporating hardware metrics into NAS is the key to improving the performance of the discovered DNNs~\cite{eccv2018-netadapt, Tan2018MnasNetPN, cai2018proxylessnas, Chen2020MnasFPNLL, chamnet}.


There is usually a trade-off between the time spent for the three steps and the support of non-differentiable search metrics. For example, early reinforcement-learning-based NAS methods~\cite{zoph2017nasreinforcement, zoph2018nasnet, Tan2018MnasNetPN} suffer from the long time for training and evaluating samples. Using a super-network~\cite{yu2018slimmable, Yu_2019_ICCV, autoslim_arxiv, cai2020once, yu2020bignas, Bender2018UnderstandingAS, enas, tunas, Guo2020SPOS} solves this problem, but super-network training is typically time-consuming and becomes the new time bottleneck. The gradient-based methods~\cite{gordon2018morphnet, liu2018darts, wu2018fbnet, fbnetv2, cai2018proxylessnas, stamoulis2019singlepath, stamoulis2019singlepathautoml, Mei2020AtomNAS, Xu2020PC-DARTS} reduce the time for training a super-network and training and evaluating samples at the cost of sacrificing the support of non-differentiable search metrics. In summary, many existing works either have an unbalanced reduction in the time spent per step (i.e., optimizing some steps at the cost of a significant increase in the time for other steps), which still leads to a long \emph{total} search time, or are unable to support non-differentiable search metrics, which limits the performance of the discovered DNNs.

In this paper, we propose an efficient NAS algorithm, NetAdaptV2, to significantly reduce the \emph{total} search time by introducing three innovations to \emph{better balance} the reduction in the time spent per step while supporting non-differentiable search metrics:

\textbf{Channel-level bypass connections (mainly reduce the time for training and evaluating samples, Sec.~\ref{subsec:channel_level_bypass_connections})}: Early NAS works only search for DNNs with different numbers of filters (referred to as \emph{layer widths}). To improve the performance of the discovered DNN, more recent works search for DNNs with different numbers of layers (referred to as \emph{network depths}) in addition to different layer widths at the cost of training and evaluating more samples because network depths and layer widths are usually considered independently. In NetAdaptV2, we propose \emph{channel-level bypass connections} to merge network depth and layer width into a single search dimension, which requires only searching for layer width and hence reduces the number of samples.

\textbf{Ordered dropout (mainly reduces the time for training a super-network, Sec.~\ref{subsec:ordered_droput})}: We adopt the idea of super-network to reduce the time for training and evaluating samples. In previous works, \emph{each} DNN in the search space requires one forward-backward pass to train. As a result, training multiple DNNs in the search space requires multiple forward-backward passes, which results in a long training time. To address the problem, we propose \emph{ordered dropout} to jointly train multiple DNNs in a \emph{single} forward-backward pass, which decreases the required number of forward-backward passes for a given number of DNNs and hence the time for training a super-network.

\textbf{Multi-layer coordinate descent optimizer (mainly reduces the time for training and evaluating samples and supports non-differentiable search metrics, Sec.~\ref{subsec:optimizer}):} NetAdaptV1~\cite{eccv2018-netadapt} and MobileNetV3~\cite{Howard_2019_ICCV}, which utilizes NetAdaptV1, have demonstrated the effectiveness of the single-layer coordinate descent (SCD) optimizer~\cite{book2020sze} in discovering high-performance DNN architectures. The SCD optimizer supports both differentiable and non-differentiable search metrics and has only a few interpretable hyper-parameters that need to be tuned, such as the per-iteration resource reduction. However, there are two shortcomings of the SCD optimizer. First, it only considers one layer per optimization iteration. Failing to consider the joint effect of multiple layers may lead to a worse decision and hence sub-optimal performance. Second, the per-iteration resource reduction (e.g., latency reduction) is limited by the layer with the smallest resource consumption (e.g., latency). It may take a large number of iterations to search for a very deep network because the per-iteration resource reduction is relatively small compared with the network resource consumption. To address these shortcomings,  we propose the \emph{multi-layer coordinate descent (MCD) optimizer} that considers multiple layers per optimization iteration to improve performance while reducing search time and preserving the support of non-differentiable search metrics.

Fig.~\ref{fig:nas_comparison} (and Table~\ref{tab:nas_result}) compares NetAdaptV2 with related works. NetAdaptV2 can reduce the search time by up to $5.8\times$ and $2.4\times$ on ImageNet~\cite{imagenet_cvpr09} and NYU Depth V2~\cite{nyudepth} respectively and discover DNNs with better performance than state-of-the-art NAS works. Moreover, compared to NAS-discovered MobileNetV3~\cite{Howard_2019_ICCV}, the discovered DNN has $1.8\%$ higher accuracy with the same latency.



%%%%%%%%%%%%%%%%%%%%%%%%%%%%%%%%%%%%%%%%%%%%%%%%%%%%%%%%%%%%%%%%%%%%%%%%%%%%%%%%

\section{Related Work}
\label{sec:related}

	\section{Related Work}
Our work draws from, and improves upon, several research threads.

\textbf{Sustainability.}~\citet{srba2016stack} conducted a case study on why StackOverflow, the largest and oldest of the sites in \CQA{StackExchange} network, is failing. They shed some insights into knowledge market failure such as novice and negligent users generating low quality content perpetuating the decline of the market. However, they do not provide a systematic way to understand and prevent failures in these markets.~\citet{wu2016} introduced a framework for understanding the user strategies in a knowledge market---revealing the importance of diverse user strategies for sustainable markets. In this paper, we present an alternative model that provides many interesting insights including knowledge market sustainability.

\textbf{Activity Dynamics.}~\citet{walk2016} modeled user-level activity dynamics in \CQA{StackExchange} using two factors: intrinsic activity decay, and positive peer influence. However, the model proposed there does not reveal the collective platform dynamics, and the eventual success or failure of a platform.~\citet{abufouda2017} developed two models for predicting the interaction decay of community members in online social communities. Similar to~\citet{walk2016}, these models accommodate user-level dynamics, whereas we concentrate on the collective platform dynamics.~\citet{wu2011} proposed a discrete generalized beta distribution (DGBD) model that reveals several insights into the collective platform dynamics, notably the concept of a size-dependent distribution. In this paper, we improve upon the concept of a size-dependent distribution.  

\textbf{Economic Perspective.} \citet{Kumar2010} proposed an economic view of CQA platforms, where they concentrated on the growth of two types of users in a market setting: users who provide questions, and users who provide answers. In this paper, we concentrate on a subsequent problem---the ``relation'' between user growth and content generation in a knowledge market.~\citet{butler2001} proposed a resource-based theory of sustainable social structures. While they treat members as resources, like we do, our model differs in that it concentrates on a market setting, instead of a network setting, and takes the complex content dependency of the platform into consideration. Furthermore, our model provides a systematic way to understand successes and failures of knowledge markets, which none of these models provide.  

\textbf{Scale Study.}~\citet{lin2017} examined Reddit communities to characterize the effect of user growth in voting patterns, linguistic patterns, and community network patterns. Their study reveals that these patterns do not change much after a massive growth in the size of the user community.~\citet{tausczik2017} investigated the effects of crowd size on solution quality in StackExchange communities. Their study uncovers three distinct levels of group size in the crowd that affect solution quality: topic audience size, question audience size, and number of contributors. In this paper, we examine the consequence of scale on knowledge markets from a different perspective by using a set of health metrics.

\textbf{Stability.} Successes and failures of platforms have been studied from the perspective of user retention and stability~\cite{patil2013, garcia2013, kapoor2014, ellis2016}. Notably,~\citet{patil2013} studied the dynamics of group stability based on the average increase or decrease in member growth. Our paper examines stability in a different manner---namely, by considering the relative exchangeability of users as a function of scale.

\textbf{User Growth.} Successes and failures of user communities have also been widely studied from the perspective of user growth~\cite{Kumar2006, Backstrom2006, kairam2012, Ribeiro2014, zang2016}.~\citet{kairam2012} examined diffusion and non-diffusion growth to design models that predict the longevity of social groups.~\citet{Ribeiro2014} proposed a daily active user prediction model which classifies membership based websites as sustainable and unsustainable. While this perspective is important, we argue that studying the successes and failures of communities based on content production can perhaps be more meaningful~\cite{kraut2014, zhu2014, zhu2014niche}.

\textbf{Modeling CQA Websites.} There is a rich body of work that extensively analyzed CQA websites~\cite{Adamic2008, chen2010, anderson2012, wang2013, srba2016}, along with user behavior~\cite{zhang2007, liu2011, pal2012, hanrahan2012, upadhyay2017}, roles~\cite{furtado2013, kumar2016}, and content generation~\cite{baezaYates2015, Yang2015, ferrara2017}. Notably,~\citet{Yang2015} noted the \emph{scalability problem} of CQA---namely, the volume of questions eventually subsumes the capacity of the answerers within the community. Understanding and modeling this phenomenon is one of the goals of this paper.



%%%%%%%%%%%%%%%%%%%%%%%%%%%%%%%%%%%%%%%%%%%%%%%%%%%%%%%%%%%%%%%%%%%%%%%%%%%%%%%%

\section{The Hexapod Microrobot}
	
    We now introduce the hexapod microrobot considered in this paper.
This robot is of particular interest due to the unique challenges that arise when attempting traditional gait design techniques.
The micro-scale of the walker makes it very challenging to obtain an accurate dynamics model.
Moreover, the robot is subject to wear-and-tear, and therefore any learning approach employed must be capable of learning gaits within a limited number of trials.

\subsection{Physical Description}    
    %
	\begin{figure}[t]
	  \centering
	  \includegraphics[width=0.96\linewidth]{fig/leg_diagram.pdf}
	  \caption{Diagram of the robot leg showing the actuation sequence (active motors are shown in red). Each leg has 2 motors, each one independently actuating a single DOF. 
      }
	  \label{fig:leg_diagram}
	%   \vspace{-8pt}
	\end{figure}
	% 
	
	The hexapod microrobot is based on silicon microelectromechanical systems (MEMS) technology. 
	The robot's legs are made using linear motors actuating planar pin-joint linkages~\citep{Contreras2016DurabilityOS}. 
	A tethered single-legged walking robot was previously demonstrated using this technology~\citep{contreras_first_2017}. 
	The hexapodal robot is assembled using three chips.  
	The two chips on the side each have 3 of the leg assemblies, granting six 2 degree-of-freedom (DOF) legs for the whole robot. 
	The top chip acts to hold the leg chips together for support, and to route the signals for off-board power and control.
	Overall, the robot measures \SI{13}{\milli\meter} long by \SI{9.6}{\milli\meter} wide and stands at \SI{8}{\milli\meter} tall with an overall weight of approximately \SI{200}{\milli\gram}. 
 
\subsection{Actuation}
	Each of the robot's legs has 2-DOF in the plane of fabrication, as shown in \fig{fig:leg_diagram}.
	Both DOFs are actuated, thus the leg has 2 motors, one to actuate the vertical DOF to lift the robot's body and a second to actuate the horizontal DOF for the vertical stride. 
	The actuators used for the legs are electrostatic gap-closing inchworm motors~\cite{penskiy_optimized_2013}.
	During a full cycle, each leg moves \SI{0.6}{\milli\meter} vertically with a horizontal stride of \SI{2}{\milli\meter}. 
	For more details on the actuation mechanism used on our microrobot, we refer readers to ~\cite{Contreras2017DynamicsOE}.
	
\subsection{Simulator}
	%
	\begin{wrapfigure}{r}{0.46\linewidth} 
	  \centering
	  \vspace{-10pt}
	  \includegraphics[width=0.98\linewidth]{fig/simulator.pdf}
	  \caption{The simulated micro walker.}
	  \label{fig:vrep}
	  \vspace{-8pt}
	\end{wrapfigure}
	% 
	In our experimental simulations, we used the robotics simulator V-REP~\citep{vrep} for constructing a scaled-up simulated model of the physical microrobot (see \fig{fig:vrep}).
	Since V-REP was not designed with simulation of microrobots in mind, it was not capable of simulating the dynamics of the leg joints accurately and would produce wildly unstable models at the desired scale.
	We chose to scale up the size of the robot in simulation by a factor of $100$ in order to account for the issues with scaling in simulation (all the experimental results are re-normalized to the dimensions of the real robot).
	We believe that this re-scaling still allows meaningful results to be produced for several reasons.
	First, the experiments performed in this paper are meant to demonstrate the validity of the proposed controller, and the learning approach for training an actual physical microrobot.
	The policies trained are not meant to work on the real robot without any re-tuning or modification.
	Second, the simulator still allows to test the basic motion patterns we want to implement on the microrobot.
	Finally, our contribution lends credibility to the potential application of Bayesian-inspired optimization methods to a setting where evaluations can be costly and time consuming.


%%%%%%%%%%%%%%%%%%%%%%%%%%%%%%%%%%%%%%%%%%%%%%%%%%%%%%%%%%%%%%%%%%%%%%%%%%%%%%%%

\section{Background}
\label{sec:background}

	\subsection{Central Pattern Generators}
\label{sec:bg:cpg}
	Central pattern generators (CPGs) are neural circuits found in nearly all vertebrates, which produce periodic outputs without sensory input~\citep{Junzhi_Yu_2014}.
	CPGs are also a common choice for designing gaits for robot locomotion~\citep{Ijspeert2008}.
    We chose to use CPGs for our controller because they are capable of reproducing a wide variety of different gaits simply by manipulating the relative coupling phase biases between oscillators. 
	This allows us to easily produce a variety of gait patterns without having to manually program those behaviors. 
	In addition, CPGs are not computationally intensive and can have on-chip hardware implementations using VLSI or FPGA. 
	This makes them well suited to be eventually used in our physical microrobot, where the processing power is limited.
	CPGs can be modeled as a network of coupled non-linear oscillators where the dynamics of the network are determined by the set of differential equations
	%
	\begin{align}
	  \dot{\phi_i} &= \omega_{i} + \displaystyle \sum_{j} (\omega_{ij}r_j\sin(\phi_j - \phi_i - \varphi_{ij}))\,,\\
	  \ddot{r_i} &= a_r (\dfrac{a_r}{4} (R_i - r_i) - \dot{r_i})\,,\\
	  \ddot{x_i} &= a_x (\dfrac{a_x}{4} (X_i - x_i) - \dot{x_i})\,,
	\end{align}
	%
	where $\phi_{i}$ is a state variable corresponding to the phase of the oscillations and $\omega_{i}$ is the target frequency for the oscillations. 
    $\omega_{ij}$ and $\varphi_{ij}$ are the coupling weights and phase biases which change how the oscillators influence each other.
    To implement our desired gaits, we only need to modify the phase biases between the oscillators~$\phi_{ij}$.
    $r_{i}$ and $x_{i}$ are state variables for the amplitude and offset of each oscillator, and $R_{i}$ and $X_{i}$ are control parameters for the desired amplitude and offset. 
    The constants $a_{r}$ and $a_{x}$ are constant positive gains and allow us to control how quickly the amplitude and offset variables can be modulated. 
    A more detailed explanation of the network can be found in Crespi's original work~\citep{Crespi_2007}. 
	One of the foremost benefits of using a CPG controller is a drastic reduction in the number of parameters~$\parameters_i$ we need to optimize.
    Overall, the parameters that we consider during the optimization are $\parameters = \left[\omega, R, X_{l}, X_{r} \right]$ where $\omega$ is the frequency of the oscillators and $R$ is the phase difference between each of the vertical-horizontal oscillator pairs. 
    In order to allow for directional control, $X_{l}$ and $X_{r}$ are the amplitudes of the left and right side oscillators respectively.    
    
\subsection{Bayesian Optimization}
\label{sec:bg:BO}
	%
	Even with a complete CPG network, some amount of parameter tuning is necessary to obtain efficient locomotion. 
	To automate the parameter tuning, we use Bayesian optimization (BO), an approach often used for global optimization of black box functions~\citep{Kushner1964,Jones2001,Calandra2015a}.
	We formulate the tuning of the CPG parameters as the optimization
	%
	\begin{align}
		\parameters^* = \maximize_{\parameters}\, \objfunc{\parameters}\,,
		\label{eq:optimization}
	\end{align}
	%
	where $\parameters$ are the CPG parameters to be optimized w.r.t. the objective function of choice~$\objfuncNo$ (\eg, walking speed, which we investigate in \sec{sec:results:soo}).
	At each iteration, BO learns a model $\tilde{\objfuncNo}: \parameters \rightarrow \objfunc{\parameters}$ from the dataset of the previously evaluated parameters and corresponding objective values measured~$\dataset=\{\parameters, \objfunc{\parameters}\}$.
	Subsequently, the learned model $\tilde{\objfuncNo}$ is used to perform a ``virtual'' optimization through the use of an acquisition function which controls the trade-off between exploration and exploitation.
	Once the model is optimized, the resulting set of parameters $\parameters^*$ is finally evaluated on the real system, and is added to the dataset together with the corresponding measurement $\objfunc{\parameters^*}$  before starting a new iteration.
	A common model used in BO for learning the underlying objective, and the one that we consider, is Gaussian processes~\citep{Rasmussen2006}. 
	For more information regarding BO, we refer the readers to~\citep{Jones2001,Shahriari2016}.

	
\subsection{Multi-objective Bayesian Optimization}
	A special case of the optimization task of \eq{eq:optimization} is multi-objective optimization~\citep{Branke2008a}.
	Often times in robotics\footnote{As well as in nature~\citep{Hoyt1981}.}, there are multiple conflicting objectives that need to be optimized simultaneously, resulting in design trade-offs (e.g., walking speed vs energy efficiency which we investigate in \sec{sec:results:moo}).  
	When multiple objectives are taken into consideration, there is no longer necessarily a single optimum solution, but rather the goal of the optimization became to find the set of Pareto optimal solutions~\citep{Pareto1906}, which also takes the name of Pareto front~(PF).
	Formally, the PF is the set of parameters that are not dominated, where a set of parameters~$\parameters_1$ is said to dominate $\parameters_2$ when 
	%
	\begin{align}
		\left\{
		\begin{array}{l l}
		\forall i \in \{1,\dots ,\numbersubobj\}: &\objfuncNo_i(\parameters_1) \leq \objfuncNo_i(\parameters_2)\\
		\exists j \in \{1,\dots , \numbersubobj\}:  &\objfuncNo_j(\parameters_1) < \objfuncNo_j(\parameters_2)
		\end{array} \right.
	\end{align}
	Intuitively, if $\parameters_1 \dom \parameters_2$, then $\parameters_1$ is preferable to $\parameters_2$ as it never performs worse, but at least in one objective function it performs strictly better. 
	However, different dominant variables are equivalent in terms of optimality as they represent different trade-offs.
	      
	Multi-objective optimization can often be difficult to perform as it might require a significant amount of experiments.
	This is especially true with our microrobot where large number of experiments can wear-and-tear the robot.
	As a result, the number of evaluations allowed to find the Pareto set of solutions is limited. 
	Luckily for us, there exist extensions of BO which address multi-objective optimization.
	In particular, the multi-objective Bayesian optimization algorithm that we consider is ParEGO~\citep{Knowles2006}. 
	The main intuition of ParEGO is that at every iteration, the multiple objectives can be randomly scalarized into a single objective (via the augmented Tchebycheff function), which is subsequently optimized as in the standard Bayesian optimization algorithm (by creating a response surface, and then optimizing its acquisition function).
	For more information about multi-objective Bayesian optimization we refer the reader to~\citep{Wagner2010}.
    
\subsection{Contextual Bayesian Optimization}
	%
	Another special case of the optimization task of \eq{eq:optimization}, is contextual optimization.
	In contextual optimization, we assume that there are multiple correlated, but slightly different, tasks which we want to solve, and that they are identified by a context variable~$\context$.
	An example (which we investigate in \sec{sec:results:context1}) might be walking on inclined slopes, where the contextual variable is the angle of the slope.
	The contextual optimization can hence be formalized as 
	%
	\begin{align}
		\parameters^* = \maximize_{\parameters}\, \objfunc{\parameters,\context}\,,
	\end{align}
	%
	where for each context~$\context$, a potentially different set of parameters~$\parameters^*$ exists.
	The main advantage compared to treating each task independently is that, in contextual optimization, we can exploit the correlation between the tasks to generalize, and as a result quickly learn how to solve a new context.
	Specifically, in this paper we consider contextual Bayesian optimization (cBO)~\citep{Metzen2015} which extends the classic BO framework from \sec{sec:bg:BO}.
	Contextual Bayesian Optimization learns a joint model $\tilde{\objfuncNo}: \{\parameters,\context\} \rightarrow \objfunc{\parameters}$, but now, at every iteration the acquisition function is optimized with a constrained optimization where the context $\context$ is provided by the environment. 
	However, because the model jointly model the context-parameter space, experience learned in one context can be generalized to similar contexts. 
	By utilizing cBO, we will show in \sec{sec:results} that our microrobot can learn to walk (and generalize) to different environmental contexts such as walking uphill and curving.


	
%%%%%%%%%%%%%%%%%%%%%%%%%%%%%%%%%%%%%%%%%%%%%%%%%%%%%%%%%%%%%%%%%%%%%%%%%%%%%%%%

\section{Learning Locomotion Primitives for Path Planning}
\label{sec:approach}

	%
\begin{figure}[t]
	\centering
	\includegraphics[height=2.4cm]{fig/cpg2.pdf}
	\caption{Output of one vertical-horizontal oscillator pair in the CPG network, which corresponds to one leg on the robot. 
	The retraction phase of both motors occurs concurrently and rapidly in order to simulate the physical constraints on the actual physical microrobot.}
	\label{fig:cpg}
\end{figure}
% 
%
\begin{figure}[t]
	\centering
	\includegraphics[width=0.98\linewidth]{fig/gaits.pdf}
	\caption{Contact/swing patterns for different gaits.}
	\label{fig:gaits}
	\vspace{-8pt}
\end{figure}
%

We now present our novel approach to learn motor primitives for path planning.
This approach relies on the possibility of re-using the evaluations collected using cBO to convert the task into a multi-objective optimization problem.
We specifically consider a cBO task where we want to optimize the parameters $\parameters$ to reach different target positions $\context=\left[ \Delta x_\text{des}, \Delta y_\text{des} \right]$ (this setting is evaluated in \sec{sec:results:context2}).
The objective function in this case can be defined as the Euclidean distance 
%
\begin{align}
	\objfuncNo = \sqrt{\left(\Delta x_\text{des} -\Delta x_\text{obs}\right)^{2} + \left(\Delta y_\text{des} -\Delta y_\text{obs}\right)^{2}}\,,
\end{align}
%
where $\Delta x_\text{obs}, \Delta y_\text{obs}$ are the actual positions measured after evaluating a set of parameters.
The cBO model would map $\tilde{\objfuncNo}: \left[\parameters, \Delta x_\text{des}, \Delta y_\text{des}\right] \rightarrow \objfunc{\parameters}$. 
However, in order to compute $f$ it would need to measure $\Delta x_\text{obs}, \Delta y_\text{obs}$, effectively generating data of the form
%
\begin{align}
	\left[\parameters, \Delta x_\text{des}, \Delta y_\text{des}\right] \rightarrow \left[\Delta x_\text{obs}, \Delta y_\text{obs}, \objfunc{\parameters}\right]
\end{align}
%
We can now re-use the data generated from this contextual optimization to learn a motor primitive model in the form $g: \parameters \rightarrow \left[\Delta x_\text{obs}, \Delta y_\text{obs}\right]$.
The purpose of this learned model~$g$ is now to provide an estimate of the final displacement obtained for a set of parameters independently from the optimization process that generated it.
Once such a model is learned, we can use it to compute parameters that lead to the desired displacement $\Delta x^*_\text{obs}, \Delta y^*_\text{obs}$ by optimizing the parameters w.r.t. the output of the model
	%
\begin{align}
	\parameters^* = \maximize_{\parameters}\, z(g(\parameters))\,,
% 
\end{align}
%
where $z$ is a scalarization function of our choice (e.g., the Euclidean distance).
This is equivalent to learning a continuous function that generates motor primitives from the desired displacement.
It should be noted that this optimization is performed on the model $g$ and therefore does not require any physical interaction with the robot.
Moreover, we can optimize the parameters over a series of multiple displacements to obtain a path planning optimization. 
In \sec{sec:results:planning}, when performing path planning using the learned motor primitives we will employ a simple shooting method optimization which randomly samples multiple candidate parameters and selects the best outcome.


	
%%%%%%%%%%%%%%%%%%%%%%%%%%%%%%%%%%%%%%%%%%%%%%%%%%%%%%%%%%%%%%%%%%%%%%%%%%%%%%%%

\section{Experimental Simulation Results}
\label{sec:results}

	%
\begin{figure}[t]
  \centering
  \begin{subfigure}{0.49\linewidth}
	  \includegraphics[width=0.98\linewidth]{fig/tripod_normal.pdf}
	  \caption{Dual Tripod}
	  \label{fig:soo:1}
  \end{subfigure}
  \hfill  
  \begin{subfigure}{0.49\linewidth}
	  \includegraphics[width=0.98\linewidth]{fig/ripple_normal.pdf}
	  \caption{Ripple}
	  \label{fig:soo:2}
  \end{subfigure}
  \\
  \begin{subfigure}{0.49\linewidth}
	  \includegraphics[width=0.98\linewidth]{fig/wave_normal.pdf}
	  \caption{Wave}
	  \label{fig:soo:3}
  \end{subfigure}
  \hfill
  \begin{subfigure}{0.49\linewidth}
	  \includegraphics[width=0.98\linewidth]{fig/fourtwo_normal.pdf}
	  \caption{Four-Two}
	  \label{fig:soo:4}
  \end{subfigure}
  \caption{Learning curve for the four gaits (median and 65th percentile). We can see how, for all the gaits, BO learns to walk from scratch within 50 iterations. After the optimization, Dual Tripod and Ripple are the fastest gaits at $\sim \SI[per-mode=symbol]{1.1}{\centi\meter\per\second}$ and $\sim \SI[per-mode=symbol]{1.2}{\centi\meter\per\second}$ respectively.}
  \label{fig:soo}
\end{figure}
%
In this section, we discuss our controller implementation as well as the performance of our simulated microrobot on various locomotion tasks.
The code used for performing the simulation and videos of the various locomotion tasks are available online at \url{https://sites.google.com/view/learning-locomotion-primitives}.

\subsection{Controller Implementation}
	We built our controller following the setup described in \sec{sec:bg:cpg}, using a network of 12 coupled phase oscillators (one per motor).
	In order to translate the output of each of the oscillators into motor actuation, we calculate the oscillator outputs for each vertical-horizontal motor pair using the piecewise function
	%
	\begin{align}
		    \begin{cases}
			    x_{i} + r_{i}cos(\phi_{i}), x_{j} + r_{j}cos(\phi_{j}) &\text{if }\phi_{i}>\pi,\phi_{j}>\pi\,,\\
			    x_{i} + r_{i}, x_{j} + r_{j}cos(\phi_{j}) &\text{if }\phi_{i}\leq\pi,\phi_{j}>\pi\,,\\
			    x_{i} + r_{i}, x_{j} + r_{j} &\text{if }\phi_{i}\leq\pi,\phi_{j}\leq\pi\,,\\
			    x_{i} + r_{i}cos(\phi_{i}), x_{j} - r_{j} &\text{if }\phi_{i}\leq\pi,\phi_{j}>\pi\,,
		    \end{cases}
	\end{align}
	%
	where the $i$th oscillator outputs to its respective vertical motor and the $j$th oscillator outputs to its respective horizontal motor. 
	This allows us to discard the parts of the oscillator output that are not consistent with the physical constraints of the physical robot, since the actual leg actuators cannot partially retract (see \fig{fig:cpg}).
	We choose to mutually couple all six of the vertical oscillators (with a coupling weight of 4 to ensure quick convergence on stable limit cycles).
	We refer the reader to \cite{Crespi_2007} for a more comprehensive discussion of oscillator coupling in CPGs.
	Each of the horizontal oscillators are also coupled with their respective vertical oscillator in order to encapsulate the dynamics of each leg.
	We chose to implement four different gaits with the CPG -- tripod, ripple, wave, and four-two (see \fig{fig:gaits}). 
	For a more detailed description of these gaits we refer the reader to~\cite{Campos2010}.
	We use the same frequency and phase difference for the whole network in order to reduce the number of parameters and speed up the rate of convergence.
	We use two separate parameters for amplitude, each controlling the left and right set of legs respectively.
	This choice of parameters allows us to control the turning of the robot which is necessary for path planning and corrections for not walking straight.

    
\subsection{Learning to Walk Straight}
\label{sec:results:soo}

	We optimized the four gaits considered (i.e., dual tripod, ripple, wave, and four-two) using as our objective function the walking speed of the robot (measured as the distance traveled after $\SI{1}{\second}$).
	Since some gaits result in curved motions, we also penalized the speed objective with a term proportional to the drift from the axis of locomotion.
	The optimization used the 4 parameters outlined in \sec{sec:bg:cpg} and was repeated 50 times for each of the gaits. 
	In \fig{fig:soo}, we show the median and 65th percentiles of the best solution obtained so far in the trials.
	The results show that the optimizer was able to learn to walk from scratch within 50 iterations.
	Moreover, it can be noted that the optimized tripod and ripple are the fastest gaits at $\sim \SI[per-mode=symbol]{1.1}{\centi\meter\per\second}$ and $\sim \SI[per-mode=symbol]{1.2}{\centi\meter\per\second}$ respectively.
	
\subsection{Multi-objective Gait Optimization}
\label{sec:results:moo}
	%
	\begin{figure}[t]
	  \centering
	  \begin{subfigure}{0.49\linewidth}
		  \includegraphics[width=0.99\linewidth]{fig/Dual_Tripod.pdf}
		  \caption{Dual Tripod}
		  \label{fig:moo:1}
	  \end{subfigure}
	  \hfill  
	  \begin{subfigure}{0.49\linewidth}
		  \includegraphics[width=0.99\linewidth]{fig/Ripple.pdf}
		  \caption{Ripple}
		  \label{fig:moo:2}
	  \end{subfigure}
	  \\
	  \begin{subfigure}{0.49\linewidth}
		  \includegraphics[width=0.99\linewidth]{fig/Wave.pdf}
		  \caption{Wave}
		  \label{fig:moo:3}
	  \end{subfigure}
	  \hfill
	  \begin{subfigure}{0.49\linewidth}
		  \includegraphics[width=0.99\linewidth]{fig/Four-Two.pdf}
		  \caption{Four-Two}
		  \label{fig:moo:4}
	  \end{subfigure}
	  \caption{Performance measured for the four gaits, and the corresponding PFs. ParEGO is able to quickly explore the PF for each of our four gaits.}
	  \label{fig:moo}
	\end{figure}
	%

	In the previous simulation we only considered walking speed as our objective. 
	However, for practical gait design, energy efficiency is another objective of great interest, particularly when it comes to designing gaits for a microrobot with real energy restrictions. 
	For this reason, we now consider a multi-objective optimization setting and compare the different gaits w.r.t. both walking speed, and energy consumption.
	The energy consumption of the robot was computed by measuring the forces exerted by each of the 12 motors along the axis of actuation and calculating the power used to actuate the motors. 
	Since the retraction of the legs is spring powered, the energy input in the cycle is only during motor extension.
	Hence, we only consider the cost of extending the legs.
	With the mass of the robot and the time of each trial being held constant, we quantify the energy efficiency of a gait and estimate the cost of transport.
   
            %
	\begin{wrapfigure}{r}{0.52\linewidth} 
	\vspace{-12pt}
	  \centering
	  \includegraphics[width=\linewidth]{fig/all_pareto.pdf}
	  \caption{Comparison of the PFs obtained for the different gaits.}
	  \label{fig:moo:all}
	  \vspace{-10pt}
	\end{wrapfigure}
	% 
	We optimized the four gaits again with the same 4 parameters as the previous optimization, but this time using multi-objective Bayesian optimization with a budget of 50 iterations.
	    %
	\begin{figure}[t]
	  \centering
	  \includegraphics[width=0.95\linewidth]{fig/discovery2.pdf}
	  \caption{PF of the unrestrained gait optimization versus the best performance of the four nature-inspired gaits. The faster solutions outperform the fastest nature-inspired gaits, albeit with more energy expenditure. However, the inability of the optimizer to match the performance of the gaits at lower speeds within 1250 trials shows that the gait parametrization can help limit the search space to find better solutions easier. \textit{(top)} Pattern for two of the discovered gaits.}
	  \label{fig:moo:new}
	\end{figure}
	%
	In \fig{fig:moo} we can see the performance measured and Pareto fronts obtained for the different gaits.
	To better compare the PF from the different gaits, we also visualized just  the PFs together in \fig{fig:moo:all}. 
	From these results, we can see how the tripod gait dominates the other gaits for speed $<\SI[per-mode=symbol]{0.6}{\centi\meter\per\second}$, while Ripple dominates when the speed is $>\SI[per-mode=symbol]{0.6}{\centi\meter\per\second}$, hence giving a clear indication of which gait is preferable under different circumstances. 
    

\subsection{Discovering New Gaits with Multi-objective Optimization}

	In addition to optimizing the four nature-inspired gaits, we also tested multi-objective optimization on the walker without constraining to using predefined gaits.
	To parametrize the oscillator couplings, we thus discretized each gait into intervals of constant length.
	Within each of these intervals, we assume that each leg steps exactly once, keeping each of the oscillators in the CPG in phase with each other.
	This allows us to parametrize gaits by assigning each leg a point during each interval where it begins stepping.
	While this parametrization excludes certain gaits that cannot be expressed in this form, we leave the study of more sophisticated gait parameterizations for gait discovery to future works.
	
	The resulting multi-objective optimization task had 8 parameters (frequency, phase difference between horizontal and vertical motors, and the six gait coupling parameters).
	Due to the higher parameter dimensionality, and because this training was not intended for on-line training, we ran the optimization for 250 iterations in order to allow a more comprehensive exploration of the optimization space.
	We also repeated the optimization five times for a total of 1250 trials.
	In \fig{fig:moo:new} we can see the Pareto front for the resulting gaits.
	We found that the fastest discovered gaits were actually able to outperform the four nature-inspired gaits implemented by a substantial margin.
	Even while penalizing curved paths, the fastest discovered gait outperformed Ripple (the fastest nature-inspired gait we found) by almost $50\%$.
	However, for low-speed gaits, the nature inspired gaits out-perform the gaits produced by the unconstrained optimization, indicating the optimization did not yet fully converged to the optimal PF.
	
\subsection{Learning to Walk on Inclined Surfaces}
\label{sec:results:context1}
	%
	\begin{figure}[t]
	  \centering
	  \includegraphics[width=0.96\linewidth]{fig/contextual_generalized_mod.pdf}
	  \caption{Performance of the contextual policy (median and 65th percentile) for a wide range of inclines. The policy was trained only at 5, 10 and 15 degrees, but it was capable of generalizing smoothly to unseen inclinations. 
	  }
	  \label{fig:incline}
	\end{figure}
    
	\begin{figure}[t]
	  \centering
	  	\begin{subfigure}{0.49\linewidth}
      \centering
	  \includegraphics[width=\columnwidth]{fig/contextual_vs_normal.pdf}
	  \caption{Inclined surface.}
	  \label{fig:contextual:1}
	\end{subfigure}
\hfill
	\begin{subfigure}{0.49\linewidth}
	  \centering
	  \includegraphics[width=\linewidth]{fig/contextual_vs_normal_turning.pdf}
	  \caption{Curved trajectory.}
	  \label{fig:contextual:2}
	  	\end{subfigure}
\caption{Comparison between the optimization performance of a contextual optimizer and a normal optimizer for two different tasks: (a) walking on inclines (b) walking curved trajectories. In both cases, the contextual optimizer can leverage prior simulations to obtain high-performing gaits in fewer simulations.}
	   \vspace{-10pt}
       \end{figure}
       
	We now consider the case of contextual optimization and specifically the task of gait optimization for slopes with different inclinations.
	We framed learning to walk on inclined terrain as a contextual policy search, where the angle of the inclination is the context. 
	In this simulation, we decided to use Dual Tripod for our gait with mostly the same open parameters as the previous simulations.
    We used a single parameter to represent the amplitude for the entire network in order to keep the number of parameters low with the addition of a contextual variable, leaving us with 3 parameters and 1 contextual parameter.
	To respect real world constraints, where testing randomly sampled incline angles over a continuous interval can be excessively time-consuming, we chose at training time to perform simulations only from a small number of inclines: 5, 10, and 15 degrees.
	
	After optimizing the gaits for these three inclines over 50 iterations, we studied how the contextual optimizer is able to generalize across the context space by testing the performance of the contextual policy for a wide range of inclines.
	In \fig{fig:incline} we can see that the policy performs well on intermediary inclines and seems to smoothly interpolate between the training inclines as is desirable.
	The gradual decrease in performance as the inclines get steeper can be attributed to the increasing physical difficulty for climbing up steeper inclines.
	We also compared cBO against using standard BO to train the robot for an untested incline. 
	As shown in \fig{fig:contextual:1}, the contextual optimization was able to converge on optimal performance significantly faster than standard BO.
	This result demonstrate the ability of cBO to efficiently use data accumulated in previous contexts to quickly reach optimize gaits in new unseen contexts.

	
\subsection{Learning to Curve}
\label{sec:results:context2}
	
	Another useful task that can be framed as contextual optimization is learning motor primitives to walk curved trajectories for use in path planning.
	We used the same parameters as in \sec{sec:results:soo} and the contextual parameters in this case were the target displacements along both the x and y axes from the point of origin.
	In order to train particular trajectories, we selected five evenly spaced target points along the front quadrant of the field of vision.
	Since the primary objective was to reach the desired destination, we chose to use the distance of the final position to the target position as our sole objective function.
	We found that over 10 repetitions, the walker was able to accurately move and turn towards all of the target points within 250 iterations.
	In \fig{fig:contextual:2}, we compared the performance of cBO against standard BO on a previously unseen target position $(4\cos{\pi / 16}, 4\sin{\pi / 16})$.
	We found that, as in the case of inclinations, the contextual policy was able to learn the optimal parameters for a novel trajectory within very few iterations.


\subsection{Learning Motor Primitives for Path Planning}
\label{sec:results:planning}
    %
	\begin{figure}[t]
	  \centering
	  \includegraphics[width=\linewidth]{fig/turning_map_both_hor.pdf}
	  \caption{Comparison of the performances of cBO and our approach for learning motor primitives (using the same data). 
	  With the robot having an initial position of $(0,0)$, we evaluated the error between the desired position (indicated by the element of the grid) and the reached position.
	  Darker color indicates better target accuracy.
	  While cBO accurately learned trajectories near the training targets, it did not generalize well to unseen targets. 
	  In contrast, our approach had a more comprehensive coverage as it could leverage better information about the environment to improve generalization.}
	  \label{fig:pathing1}
	\end{figure}
	% 

	    %
	\begin{wrapfigure}{r}{0.50\linewidth} 
	\vspace{-10pt}
	  \centering
	  \includegraphics[width=\linewidth]{fig/contextual_path.pdf}
	  \caption{Path constructed using the locomotion primitives learned with our approach.}
	  \label{fig:pathing2}
	   \vspace{-8pt}
	\end{wrapfigure}
	% 
	In the previous simulation we learned motor primitives capable of walking curved trajectories.
    While the model handled trajectories near and between the targets quite well, the performance on trajectories well within the physical capabilities of the robot but not in proximity to the targets left much to be desired, as shown in \fig{fig:pathing1}.
	We now demonstrate how our approach presented in \sec{sec:approach} can be used to significantly improve the movement accuracy (compared to cBO using the same data), as well as how such motor primitives can be used to perform path planning.
    First, we reused the data from the previous simulation in order to reformulate the task as a multi-objective optimization as described in \sec{sec:approach}.
    Then, we used our trained model to sample 10,000 trajectories by randomly sampling from the parameter space.
    Out of all these trajectories, we selected the one with the smallest expected error subject to not walking through the wall.
    Evaluating the resulting sequence of motor primitives on the real system (\ie, the simulator) demonstrated that the expected trajectory was capable of navigating the maze, as shown in \fig{fig:pathing2}.



    
    
    



%%%%%%%%%%%%%%%%%%%%%%%%%%%%%%%%%%%%%%%%%%%%%%%%%%%%%%%%%%%%%%%%%%%%%%%%%%%%%%%%

\section{Conclusion}

	\section{Conclusion and Future Work}\label{sec:conclusion}
In this paper, we report an approach which adopts reinforcement learning algorithms to solve the problem of robustness-guided falsification of CPS systems. We implement our approach in a prototype tool and conduct preliminary evaluations with a widely adopted CPS system. The evaluation results show that our method can reduce the number of episodes to find the falsifying input. As a future work, we plan to extend the current work to explore more reinforcement learning algorithms and evaluate our methods on more CPS benchmarks. 

%%%%%%%%%%%%%%%%%%%%%%%%%%%%%%%%%%%%%%%%%%%%%%%%%%%%%%%%%%%%%%%%%%%%%%%%%%%%%%%%

% \section*{APPENDIX}
% 
% Appendixes should appear before the acknowledgment.

%%%%%%%%%%%%%%%%%%%%%%%%%%%%%%%%%%%%%%%%%%%%%%%%%%%%%%%%%%%%%%%%%%%%%%%%%%%%%%%%

\bibliographystyle{IEEEtran}
\bibliography{paper-nanorobots}  % .bib


\end{document}
