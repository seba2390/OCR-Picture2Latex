Designing controllers for locomotion is a daunting task.
In this paper, we demonstrated on a simulated microrobot that this process can be significantly automated. 
Our main contributions are two-fold: 
1) we introduced a coherent curriculum of increasing challenging tasks, which we use to evaluate the CPG controller of our microrobot using Bayesian optimization.
2) we presented a new approach that enables walking robots to efficiently learn motor primitives from scratch.
By using the data collected from contextual optimization we reformulate the problem into a multi-objective optimization task, and learn a model that can map any set of parameters to a predicted trajectory. 
This model can subsequently be used for path planning.
Our experimental simulation results demonstrate that using this approach a microrobot can successfully learn accurate locomotion primitives within 250 trials, and subsequently use them to navigate through a maze, without any prior knowledge about the environment or its own dynamics. 

The gaits obtained on the simulated microrobot might not yield good results when applied to the real microrobot, due to the low-fidelity of the simulator used. 
However, the methodology used to obtain them is realistically applicable to real microrobots, and is uniquely able to address concerns that exist on the sub-centimeter scale (\eg, lack of a precise physics simulator and budgeting of physical experiments). 
In future work, we plan to evaluate our approach and findings on the physical hexapod microrobot.