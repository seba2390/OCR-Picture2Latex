\section{Introduction}

\subsection{Interval orders}
A \emph{partial order} is a reflexive, antisymmetric, and transitive relation on a set. A set $X$ together with a partial order on $X$ is called a \emph{poset}, with \emph{ground set} $X$. In this paper, we assume $X$ is finite. The elements of $X$ are sometimes called \emph{points}, or \emph{vertices}. To avoid clutter, with a slight abuse of notation, we will often use the same letter to denote the poset, its ground set, and the relation.

Let $\mathcal{I}$ be the set of nonempty closed intervals of the real line.
An \emph{interval representation} of $P$ is a function $f:P\to\mathcal{I}$ such that $x<y$ in $P$ if and only if $f(x)$ is entirely left of (and hence disjoint from) $f(y)$. An \emph{interval order} is a poset that has an interval representation.

Restrictions can be placed on the codomain of $f$ to define subclasses of interval orders. Let $S$ be a nonempty set of nonnegative real numbers. An \emph{$S$-representation} of $P$ is an interval representation of $P$ such that the length of $f(x)$ is in $S$ for all $x\in P$.

Following Fishburn and Graham \cite{FG-85}, we will use the notation $C(S)$ to denote the family of posets that have an $S$-representation. As a special case, $C([\alpha,\beta])$ denotes the family of posets for which there is a representation with intervals of lengths between $\alpha$ and $\beta$ (inclusive). We will use the shorthands $C[\alpha,\beta]=C([\alpha,\beta])$, and $C(\alpha)=C([1,\alpha])$.

Obviously, $C[\alpha, \beta] = C[m\alpha, m\beta]$, for all $m\in\mathbb{R}^+$. It is also clear that the class of interval orders is $C(\mathbb{R}^+\cup\{0\})=C(\mathbb{R}^+)$.

A well-studied subclass of interval order is $C(1)$ which is called the class of \emph{semiorders}. Characterization of interval orders and semiorders via forbidden subposets are classical results of poset theory. Fishburn \cite{F-70} proved that a poset is an interval order, if and only if contains no $\tpt$: a poset consisting of two incomparable chains, each of size $2$. Scott and Suppes \cite{SS-58} proved that a poset is a semiorder if and only if it contains no $\tpt$ and no $\tpo$; this latter is a poset consisting of two incomparable chains, sizes $3$ and $1$.

The \emph{interval count} of the interval order $P$, denoted by $IC(P)$ is the least integer $k$ for which $P$ has an $S$-representation such that $|S|=k$. The interval count problem was suggested by Graham (cf.~\cite{LAP-82}). Semiorders are exactly the posets $P$ for which $IC(P)=1$. However, the problem of deciding whether a poset is of interval count at most $2$ already seems to be a difficult problem. The decision problem is not not known to be either NP-complete or polynomial.

According to a survey by Cerioli et al. \cite{COS-12}, ``The interval count problem is an intriguing problem, in which very intuitive statements are proved not to hold.'' For example, for each $k\geq 2$, Fishburn \cite{F} constructed an infinite family of minimal interval orders whose interval count is greater than $k$, so for $k\geq 2$, the class of posets with interval count $k$ does not have a characterization with a finite set of minimal excluded subposets.

\subsection{Dimension}
A set of linear extensions $\{L_1,\ldots,L_t\}$ of the poset $P$ is called a \emph{realizer}, if $P=\cap L_i$. To verify that a certain set of linear extensions is a realizer, it suffices to verify that for every pair $(x,y)$ of incomparable elements there exists a linear extensions $L_i$ in the set in which $x>y$. We will refer to this property as $(x,y)$ being \emph{reversed} in $L_i$. (In fact it suffices to verify that every so-called \emph{critical pair} is reversed, but we will not make use of this property.)

The minimum cardinality of a realizer is the \emph{dimension} of $P$. An extensive treatise of dimension theory of posets is the monograph by Trotter \cite{T}. Since dimension is a classic way of measuring the complexity of a poset, it is natural to ask if the dimension of (restricted) interval orders is bounded.

It has been well-known that the dimension of the class of interval orders is unbounded. In fact, the dimension of the universal interval order $I_n$, whose representation uses every closed interval with integer endpoints between $1$ and $n$, is well-understood. The rate of growth, $\Theta(\lg\lg n)$, was first determined by F\"uredi et al.~\cite{FHRT-92}, but using further results, the exact value is known within an error of $5$.

On the other hand, Rabinovitch \cite{R-78} proved that the dimension of semiorders is at most $3$. A simple proof of this theorem is provided by Bosek et al.~ \cite{BKKM-12}. Similar techniques to their proof is used in this paper.

\subsection{Motivation}
Allowing open intervals of arbitrary length in a representation does not extend the class of interval orders. However, this is not the case with restricted classes.

\begin{comment}
Let $OC(S)$ denote the class of posets that have a representation with every interval length in $S$, but allowing a mixture of open and closed intervals. Extend all the shorthands used with the $C$-notation to be used with the $OC$-notation. As mentioned above, it is easy to see that $C(\mathbb{R}^+)=OC(\mathbb{R}^+)$. On the other hand, $\tpo$ is in $OC(1)$ but not in $C(1)$.
\end{comment}

In a recent paper, Keller, Trenk, and Young \cite{KTY-22} proved the following result.
\begin{theorem}\label{thm:KTY}
%\begin{itemize}
%\item If $P$ is in $OC(1)$, then $\dim(P)\leq 3$.
If $P$ is in $C(\{0,1\})$, then $\dim(P)\leq 3$.
%\end{itemize}
\end{theorem}

They proposed among others, the following problems (paraphrased).
\begin{problem}\ \label{pr:KTY}
\begin{enumerate}
\item \label{pr:KTY2} Find an upper bound on the dimension for posets in $C(\{r,s\})$ with $r,s>0$.
\item \label{pr:KTYr} More generally, find an upper bound on the dimension for posets in $C(S)$ in terms of $|S|$.
\end{enumerate}
\end{problem}

They offered the easy upper bounds $8$ for (\ref{pr:KTY2}), and $3r+\binom{|S|}{2}$ for (\ref{pr:KTYr}). In Section~\ref{sec:KTYq}, we give a better bounds for both parts. In fact, together with a computer search done by K\'ezdy, we will see that dimension of posets in $C(\{r,s\})$, $r,s>0$ is between $4$ and $5$. Also, using the techniques we develop, we simplify the proof of Theorem~\ref{thm:KTY}, and we prove some other related results.

