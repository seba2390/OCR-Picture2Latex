\section{Dimension bounds for posets of bounded interval count}
\label{sec:KTYq}

Since $C(\{0,s\})=C(\{0,1\})$ for all $s>0$, Theorem~\ref{thm:KTY} provides a perfect answer to dimension bounds on this class. The next natural question, also question (\ref{pr:KTY2}) of Problem~\ref{pr:KTY}, is to find an upper bound for the dimension of posets in the class $C(\{r,s\})$ with $r,s>0$. We give an answer below, using the terminology of interval counts.

\begin{proposition}\label{prop:KTY2}
If $P$ be an interval order with $IC(P)\leq 2$. Then $\dim(P)\leq 5$.
\end{proposition}

\begin{proof}
Consider an $\{r,s\}$-representation, and partition $P$ into $X_1\cup X_2$ with $X_1$ containing points corresponding to intervals of length $r$, and $X_2$ containing points corresponding to intervals of length $s$. Since $\dim(P|_{X_1})\leq 3$ and $\dim(P|_{X_2})\leq 3$, using Theorem~\ref{thm:maxplus2} for $s=2$, we conclude $\dim(P)\leq 5$.
\end{proof}

Andr\'e K\'ezdy (personal communication) ran a computer search using a state-of-the-art dimension algorithm to find a large dimensional poset of interval count $2$.
K\'ezdy found that the interval order represented by the $51$ intervals in the set
\[
\{[a,b]:\quad a,b\in\mathbb{Z};\quad 1\leq a,b\leq 30;\quad|b-a|\in\{1,8\}\}
\]
is $4$-dimensional. Furthermore, he found that it has a $45$-element subposet that is irreducible, i.e.\ the removal of any point reduces the dimension to $3$.\footnote{This poset can be constructed from the set of intervals above by removing the intervals $[8, 9]$, $[9, 10]$, $[10, 11]$, $[20, 21]$, $[21, 22]$, $[22, 23]$.}

This answers another question by Keller, Trenk, and Young, namely, the minimum number of interval lengths required to force a $4$-dimensional interval order.

Using the general version of Theorem~\ref{thm:maxplus2}, we conclude the following result, providing a logarithmic upper bound to question (\ref{pr:KTYr}) of Problem~\ref{pr:KTY}.

\begin{proposition}
Let $P$ be an interval order with $IC(P)\leq r$. Then $\dim(P)\leq 2\lceil\lg r\rceil+3$.
\end{proposition}

Keller, Trenk, and Young conjecture an upper bound of $O(\lg\lg r)$ for the dimension of intervals orders of count at most $r$. They base their conjecture on the dimension of the universal interval orders. However, we note that the universal interval order is a much stronger restriction than a bounded interval count. Even the rate of growth of the best upper bound remains wide open.

