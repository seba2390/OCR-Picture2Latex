\section{The Keller--Trenk--Young Theorem}

This section contains a short proof of Theorem~\ref{thm:KTY}.

Let $P$ be a poset. One can partition the ground set $X$ of $P$ into antichains by subsequent removal of minimal elements. That is, let $A_1=\min(P)$, $A_2=\min(P-A_1)$, \dots, $A_i=\min(P-A_1-\cdots-A_{i-1})$. Then $X=A_1\cup\cdots\cup A_h$ is a partition, $h$ is the height of $P$, and each $A_i$ is an antichain. We will call this the \emph{ranking partition} of $P$, as it defines a rank function on graded posets. (But we may apply this definition for non-graded posets, as well).

The following simple lemma has been in used in many arguments.

\begin{lemma}\label{lem:ranking}
If $P$ has no $\tpo$ subposet, and $A_1,\ldots,A_h$ is a ranking partition, then $A_i<A_{i+2}$ for all $i$.
\end{lemma}

\begin{proof}
Let $x\in A_i$, $y\in A_{i+2}$. Then there exists $z\in A_{i+1}$, and $z<y$; and $w\in A_i$, and $w<z$. Now $x\| w$, because $x,w\in A_i$. Hence $x\not> y$. If $x\|y$, then the set $\{x,y,z,w\}$ induces a $\tpo$. So $x<y$.
\end{proof}

\begin{proof}[Proof of Theorem~\ref{thm:KTY}]
We may assume that $P$ has no duplicated holdings, so it has a twin-free $\{0,1\}$-representation. By Theorem~\ref{thm:representations}, we may consider a distinguishing representation. Partition $P$ into $U\cup Z$, where $U$ contains the points that are represented by unit intervals, and $Z$ contains the ones whose interval length is $0$. Let $A_1,\ldots,A_h$ be a ranking partition of $P|_U$. Let $E=\cup\{A_i:\text{$i$ is even}\}$, and $O=\{A_i:\text{$i$ is odd}\}$.

Define two choice functions:
\begin{align*}
f_1(x)&=
\begin{cases}
l_x&\text{ if $x\in O\cup Z$}\\
r_x&\text{ if $x\in E$}
\end{cases}\\
f_2(x)&=
\begin{cases}
l_x&\text{ if $x\in E\cup Z$}\\
r_x&\text{ if $x\in O$}
\end{cases}
\end{align*}
Let $L_1=L(f_1)$, and $L_2=L(f_2)$. Define a third linear extension:
\[
L_3:(L_1|_{A_1})^d<\cdots<(L_1|_{A_h})^d
\]

The linear extensions $L_1$ and $L_2$ will reverse every incomparable pair for which (1) one is in $U$, the other is in $Z$; (2) one is in $E$, the other is in $O$. The only incomparable pairs left are in the same $A_i$: those are reversed in $L_3$.
\end{proof}
