\section{Choice functions and partitions}

Let $P$ be an interval order with a representation $I$. An injective function $f:\mathbb{R}\to\mathbb{R}$ is a \emph{choice function}, if $f(x)\in I(x)$ for all $x\in P$. Every choice function $f$ defines a linear extension $L(f)$ of $P$ in a natural way: $x<y$ in $L(f)$ if $f(x)<f(y)$.

Choice functions were defined by Kierstead and Trotter \cite{KT-00}. Theorems~\ref{thm:choice} and \ref{thm:partition} appear in their paper, but our proofs in this section are different; in fact we found them before we found their paper. We feel that the proofs here are more transparent and more constructive in the sense, that they lead to easy-to-implement algorithms.

What makes the simple idea of choice functions powerful is the following theorem.

\begin{theorem}\label{thm:choice}
Let $P$ be an interval order with no duplicated holdings, and let $I$ be a distinguishing representation. If $L$ is a linear extension of $P$, then there exists a choice function $f$ with $L(f)=L$.
\end{theorem}

\begin{proof}
Let $|P|=n$, and 
let $\epsilon=\frac{1}{n}\min\{|a-b|:\text{$a$ and $b$ are distinct endpoints}\}$.
Let $L$ be such that $x_1<\cdots<x_n$ in $L$. We will construct $f$ recursively. 

\[
f(x_i)=
\begin{cases}
f(x_{i-1})+\epsilon & \text{if $i\geq 2$ and $f(x_{i-1})\geq l_{x_i}$}\\
l_{x_i} & \text{otherwise}\\
\end{cases}
\]

Obviously, $f(x_{i-1})<f(x_i)$ for all $i$. So we only need to show that $f(x_i)\in I(x_i)$.

If $f(x_i)=l_{x_i}$ this is clear, so we may assume that $f(x_i)\neq l_{x_i}$. Call the points $x_i$ with this property ``forced''.

Let $j$ be the greatest positive integer such that $j\leq i$ and $x_j$ is not forced. %Well-defined, because $x_1$ is not forced. The same implies $i\geq 2$.
Firstly, $r_{x_i}>l_{x_j}$, %distinguishing repr., so can't be equal
secondly, $r_{x_i}\geq l_{x_j}+n\epsilon$.
Hence $f(x_i)=l_{x_j}+(i-j)\epsilon\leq l_{x_j}+n\epsilon\leq r_{x_i}$, and so
$l_{x_i}\leq f(x_{i-1})\leq f(x_i)\leq r_{x_i}$.
\end{proof}

Theorem~\ref{thm:representations} and Theorem~\ref{thm:choice} show that constructing linear extensions for posets in classes $C(S)$ can usually be done by constructing choice functions. This makes choice functions a useful tool for upper bound proofs on the dimension for these classes. We will demonstrate this repeatedly in the balance of the paper.

The following theorem shows how special interval orders are. Nothing remotely close is true for general posets.

\begin{theorem}\label{thm:partition}
Let $P$ be an interval order, and partition the ground set $X$ into $s$ parts: $X=X_1\cup\cdots\cup X_s$. Let $L_i$ be a linear extensions of $P|_{X_i}$. Then there exists a linear extension $L$ of $P$ such that $L|_{X_i}=L_i$.
\end{theorem}

\begin{proof}
We will use induction on $s$. The $s=1$ case is trivial; let $s=2$.

Let $X_1,X_2,L_1,L_2$ be defined as in the lemma. Define the relation $E=L_1\cup L_2\cup  P$, and the directed graph $G=(X,E)$. It is sufficient to show that $G$ has no directed closed walk; indeed, if that is the case, the transitive closure $T$ of $G$ is an extension of the poset $P$, and any linear extension $L$ of $T$ will satisfy the requirements of the conclusion of the lemma.

Suppose for a contradiction that $G$ contains a directed closed walk.
Since neither $G[X_1]$ nor $G[X_2]$ contains a directed closed walk, every directed closed walk in $G$ must have both an $X_1X_2$ and an $X_2X_1$ edge. We will call these edges \emph{cross-edges}. Let $C$ be a directed closed walk in $G$ with the minimum number of cross-edges.

As we noted, $C$ contains at least one $X_1X_2$ edge; let $(a,b)$ be such an edge. Let $(c,d)$ be the first $X_2X_1$ edge that follows $(a,b)$ in $C$. Observe that $c<d$, $a<b$ in $P$, and $b\leq c$ in $L_2$. If $d=a$, then $c<d=a<b$ in $P$, which would contradict $b\leq c$ in $L_2$. If $d>a$ in $L_1$, then we could eliminate the path $ab\ldots cd$ in $C$, replacing it with the single-edge path $ad$, and thereby decreasing the number of cross-edges in $C$, contradicting the minimality of $C$. (See Figure~\ref{fig:cycles}.)

\begin{figure}
\begin{tikzpicture}
    \tikzstyle{oval} = [ellipse, minimum width=2cm, minimum height=4cm, draw]
    \tikzstyle{point} = [circle, minimum size=4pt, inner sep=0pt, fill, draw]

    % Draw ovals
    \node[oval] (leftOval) at (0,0) [label=$X_1$] {};
    \node[oval] (rightOval) at (3,0) [label=$X_2$] {};

    % Points for arrows
    \coordinate (leftTop) at ([yshift=1.5cm]leftOval.center);
    \coordinate (rightTop) at ([yshift=1.5cm]rightOval.center);
    \coordinate (leftBottom) at ([yshift=-1.5cm]leftOval.center);
    \coordinate (rightBottom) at ([yshift=-1.5cm]rightOval.center);

    % Draw arrows
    \draw[-{Latex[length=3mm]}] (leftTop) to[bend left] (rightTop);
    \draw[-{Latex[length=3mm]}] (rightBottom) to[bend left] (leftBottom);

    \draw[-{Latex[length=3mm]}, decorate, decoration={snake, amplitude=.4mm, segment length=2mm, post length=3mm}] (rightTop) to[bend left=20] (rightBottom);
    
    \draw[-{Latex[length=3mm]}, decorate, decoration={snake, amplitude=.4mm, segment length=2mm, post length=3mm}] plot [smooth, tension=1] coordinates {(leftBottom) (3,0) (leftTop)};
    
    \draw[-{Latex[length=3mm]},dashed] (leftBottom) to[bend left=20] (leftTop);
    
    % Add dots at the endpoints of the arrows
    \node[point] at (leftTop) [label=left:$a$] {};
    \node[point] at (rightTop) [label=right:$b$] {};
    \node[point] at (rightBottom) [label=right:$c$] {};
    \node[point] at (leftBottom) [label=left:$d$] {};
\end{tikzpicture}
\caption{Minimal oriented cycles}\label{fig:cycles}
\end{figure}

So we concluded that $d<a$ in $L_1$, and recall that $b\leq c$ in $L_2$. If $b\leq c$ in $P$, then $a<b\leq c<d$ would contradict $d<a$ in $L_1$. (In particular, $b\neq c$.) Obviously, $b\not>c$ in $P$, so $b\|c$ in $P$. Similarly, $d\| a$ in $P$. Hence the set $\{a,b,c,d\}$ induces a $\mathbf{2}+\mathbf{2}$ in $P$, a contradiction.

If $s>2$, then we can apply the hypothesis for the part $X'=X_1\cup\cdots\cup X_{s-1}$, and use the $s=2$ case for $X'$ and $X_s$. This finishes the proof.
\end{proof}

We conclude the section with the following all-important corollary, which is a generalization of a theorem by Kierstead and Trotter~\cite{KT-00}.

\begin{theorem}\label{thm:maxplus2}
Let $P$ be an interval order, and partition the ground set $X$ into $s$ parts: $X=X_1\cup\cdots\cup X_s$. Let $P_i=P|_{X_i}$. Then
\[
\dim(P)\leq \max\{\dim(P_1),\ldots,\dim(P_s)\}+2\lceil \lg s\rceil
\]
\end{theorem}

\begin{proof}
Let $t=\max\{\dim(P_1),\dim(P_2)\}$. We proceed
by induction on $s$. Trivial for $s=1$; let $s=2$. Consider a distinguishing representation of $P$.

By Theorem ~\ref{thm:partition}, there exists a family $\mathcal{R}$ of $t$ linear extensions of $P$, such that the restriction of the linear extensions in $\mathcal{R}$ to each $X_i$ form a realizer of $P_i$. Then define two choice functions $f_1$ and $f_2$, where $f_1(x)=l_x$, $f_2(x)=r_x$ for every $x \in X_1$; $f_1(y)=r_y$, $f_2(y)=l_y$ for every $y \in X_2$. Let $L_1=L(f_1)$, $L_2=L(f_2)$. Clearly, $\mathcal{R} \cup \{L_1, L_2\}$ is a realizer of $P$.

If $s>2$, then we can apply the hypothesis for the parts $X'=X_1\cup\cdots\cup X_{\lfloor s/2\rfloor}$ and $X''=X_{\lfloor s/2\rfloor+1}\cup\cdots\cup X_s$, and use the $s=2$ case for $X'$ and $X''$. A routine calculation finishes the proof.
\end{proof}
