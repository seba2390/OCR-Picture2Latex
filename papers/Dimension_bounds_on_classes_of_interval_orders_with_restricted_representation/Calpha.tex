\section{Posets in $C(t)$}

Recall that $C(t)$ is the class of interval orders representable by intervals of length between $1$ and $t$. In this sections, we use our techniques to find upper bounds for this class.

\begin{theorem}\label{thm:C2}
If $P$ is in $C(2)$, then $\dim(P)\leq 4$.
\end{theorem}

We start with a generalization of Lemma~\ref{lem:ranking}. The proof is essentially the same.

\begin{lemma}\label{lem:genranking}
If $P$ has no $\mathbf{k}+\mathbf{1}$ subposet, and $A_1,\ldots,A_h$ is a ranking partition, then $A_i<A_{i+k-1}$ for all $i$.
\end{lemma}

\begin{proof}[Proof of Theorem~\ref{thm:C2}]
Let $A_1,\ldots,A_h$ be a ranking partition of $P$. Notice that $P$ has no $\mathbf{4}+\mathbf{1}$ subposet, so by Lemma~\ref{lem:genranking}, $A_i<A_{i+3}$ for all $i$. We define three choice functions $f_0,f_1,f_2$ as follows.
\[
f_i(x)=
\begin{cases}
r_x&\text{ if $x\in A_j$ with $j\equiv i\mod 3$}\\
l_x&\text{otherwise}
\end{cases}
\]
Let $L_0=L(f_0)$, $L_1=L(f_1)$, $L_2=L(f_2)$. If $(x,y)$ is an incomparable pair, and $x\in A_i$, $y\in A_j$, $i\neq j$, then they are reversed in $L_k$ for which $k\equiv i\mod 3$.

We still need to reverse incomparable pairs that appear within the same $A_i$. This can be done the usual way:
\[
L_3:(L_1|_{A_1})^d<\cdots<(L_1|_{A_h})^d
\]

Now $\{L_0,L_1,L_2,L_3\}$ is a realizer of $P$.
\end{proof}

A simple generalization of this proof would lead to an upper bound $\lceil t\rceil+2$ for the dimension of the class $C(t)$. However, we can do much better with a divide-and-conquer technique.

\begin{theorem}\label{thm:Ct}
For $t\geq 2$, if $P$ is in $C(t)$, then $\dim(P)\leq 2\lceil\lg\lg t\rceil+4$.
\end{theorem}

\begin{proof}
Let $n=2^{2^{\lceil\lg\lg t\rceil}}$. Since $n\geq t$, we have $P\in C(n)$. We will show, by induction on $k$, that for $n=2^{2^k}$ for some $k\in\mathbb{N}$, and for $P\in C(n)$, we have $\dim(P)\leq 2\lg\lg n+4=2k+4$. This will finish the proof.

For $k=0$, the statement is Theorem~\ref{thm:C2}. Now assume $k\geq 1$. Then $n$ is a perfect square; let $m=\sqrt{n}=2^{2^{k-1}}$. Consider a $[1,n]$-representation of $P$, and partition its ground set into points represented by ``short'' intervals, which are of length at most $m$, and ``long intervals'', which are longer than $m$. Let the subsets be $S$ and $L$, respectively.

Notice that $P|_S\in C(m)$, and $P|_L\in C(m,n)=C(n/m)=C(m)$. Using Theorem~\ref{thm:maxplus2}, we have
\[
\dim(P)\leq\max\{\dim(P|_S),\dim(P|_L)\}+2=2\lg\lg m+4+2=2(k-1)+6=2k+4.
\]
\end{proof}

We note that the universal interval order shows that Theorem~\ref{thm:Ct} is best possible in terms of rate of growth (up to a multiplicative factor).

Again, lower bounds in Theorems~\ref{thm:C2} and \ref{thm:Ct} seem to be difficult to find. The existence of a $4$-dimensional poset in $C(2)$ seems likely, but we have not succeeded in finding one. We could only show a very modest result: just one longer interval is not sufficient to raise the dimension to $4$. This is expressed in the following theorem, which is also relevant for finding lower bounds for Proposition~\ref{prop:KTY2}.

\begin{theorem}
Let $P$ be an interval order with a $\{1,s\}$-representation, such that $0\leq s\leq 2$, and there is only one interval of length $s$. Then $\dim(P)\leq 3$.
\end{theorem}

\begin{proof}
Let $x$ be the point represented by the interval of length $s$. With a shift of the intervals in the representation, we may assume that $r_x=-s/2$, and $l_x=s/2$, i.e.\ the midpoint of the interval of $x$ is $0$.

Partition the poset based on the position of the rest of the intervals in the representation. Let $U$ be the point corresponding to intervals with positive endpoints, $D$ of the same with negative endpoints, and let $M$ be the set of points with intervals containing the number $0$.

Let $U_1,\ldots,U_m$ be a ranking partition of $P|_U$, and let $D_1,\ldots,D_k$ be a ranking partition of $(P|_D)^d$. Now the ground set of $P$ is partitioned into the sets
\[
D_k,D_{k-1},\ldots,D_1,M,U_1,\ldots,U_m.
\]
Reindex this partition so that the same parts, in the same order, are denoted by $S_1,\ldots,S_{m+k+1}$. We claim that $S_i<S_{i+2}$ for all $i$.

Since $P|_D$ and $P|_U$ are semiorders, and hence have no $\tpo$, by Lemma~\ref{lem:ranking}, it is sufficient to prove that $x<y$ whenever (1) $x\in D_2$, $y\in M$; or (2) $x\in D_1$, $y\in U_1$; or (3) $x\in M$, $y\in U_2$. Of these, two is clear, because $x$ has negative right endpoint, and $y$ has positive left endpoint. The proof of (1) and (3) is symmetric, so we assume $x\in M$ and $y\in U_2$.

Since $y\in U_2$, there exists $z\in U_1$ with $z<y$. Hence
\[
r_x=s/2\leq 1<l_z+1=r_z<l_y,
\]
which implies $x<y$.

Now we can finish the proof in the familiar way. We define two choice functions: $f_1(x)=l_x$, $f_2(x)=r_x$ if $x\in S_{2i}$, and $f_1(x)=r_x$, $f_2(x)=l_x$ if $x\in S_{2i+1}$ for some $i$. Construct the two corresponding linear extensions $L_1=L(f_1)$, $L_2=L(f_2)$. Finally, let $L_3$ be such that $(L_1|_{S_1})^d<\cdots<(L_1|_{S_{m+k+1}})^d$ in $L_3$. Then $\{L_1,L_2,L_3\}$ is a realizer.
\end{proof}
