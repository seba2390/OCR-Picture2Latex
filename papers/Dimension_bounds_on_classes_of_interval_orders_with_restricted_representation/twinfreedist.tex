\section{Twin-free and distinguishing representations}

Let $P$ be an interval order, and fix a representation for $P$. Let $x,y\in P$ be such that the same interval is assigned to both. We call $x$ and $y$ a \emph{twin} (of points). If a representation does not have any twins, we call it \emph{twin-free}. A representation of an interval order is \emph{distinguishing}, if every real number occurs at most once as an endpoint of an interval of the representation, i.e.~no two intervals share an endpoint. A distinguishing representation is of course twin-free.

Let $P$ be a poset, and $x,y\in P$. We say $x$ and $y$ have \emph{duplicated holdings}, if $\{z\in P: z>x\}=\{z\in P: z>y\}$ and $\{z\in P: z<x\}=\{z\in P: z<y\}$; in other words the upsets and the downsets of $x$ and $y$ are the same. If $P$ is an interval order with a representation in which $x$ and $y$ are twins, then they have duplicated holdings. So if an interval order has no duplicated holdings, then every representation is twin-free.

One important property of two elements with duplicated holdings is that we may discard one of them without reducing the dimension (as long as the dimension is at least $2$). We will use this property later by assuming that a certain poset, for which we are proving an upper bound for its dimension, has no duplicated holdings.

It is easy to see, that every interval order has a distinguishing $\mathbb{R}^+$-representation. Things get less obvious for other $S$-representations. For example, an antichain of size at least $2$ does not have a distinguishing (or even twin-free) $\{0\}$-representation. We prove that---essentially---this is the only problem case.

We caution the reader that the usual handwaving argument of ``just shake the intervals a little until there are no common endpoints'' will not work here. Since we are not allowed the change of the set of lengths, common endpoints may be essential and necessary. The fact that we use closed intervals plays a crucial role here. E.g. $\tpo$ has a $\{1\}$-representation using open and closed intervals, but there is no distinguishing representation.

In the following proof and the balance of the paper, we use the notation $l_x$ and $r_x$ for the left and right endpoint of the interval corresponding to the point $x$ in a given representation.

\begin{theorem}\label{thm:representations}
Let $S\subseteq\mathbb{R}^+\cup\{0\}$, $S\neq\emptyset$.
\begin{enumerate}
\item\label{part:1} Every poset $P\in C(S)$ that has a twin-free free $S$-representation also has a distinguishing $S$-representation.
\item\label{part:2} If $0\not\in S$, then every poset $P\in C(S)$ has a distinguishing $S$-representation.
\end{enumerate} 
\end{theorem}

\begin{proof}
Let $S\subseteq\mathbb{R}^+\cup\{0\}$, $S\neq\emptyset$, and let $P \in C(S)$. Consider an $S$-representation of $P$; with a slight abuse of notation, the multiset of intervals in this representations will also be referred as $P$. We will define two symmetric operations that we will perform repeatedly. These will be used to decrease the number of common endpoints of the intervals. After this, we enter a second phase, in which we remove twins, if possible.

\subsection*{Left and right compression}

Let $c\in\mathbb{R}$, and $\epsilon>0$. Let $L=\{x\in P: l_x<c\}$, and let $R=P-L$. Define $L'=\{[l_x+\epsilon,r_x+\epsilon]:x\in P\}$. Let $P'=L'\cup R$, a multiset of intervals. The operation that creates $P'$ from $P$ is what we call ``left compression'' with parameters $c$ and $\epsilon$.

We can similarly define right compressions. Let $R=\{x\in P: r_x>c\}$, and let $L=P-R$. Define $R'=\{[l_x-\epsilon,r_x-\epsilon]:x\in P\}$. Let $P'=L\cup R'$ to define the operation of right compression.

\begin{lemma}
Let $P$ be a poset (representation), $c\in\mathbb{R}$, and let $\epsilon=\frac12\min\{|a-b|:\text{$a$ and $b$ are distinct endpoints}\}$. Let $P'$ be the left (right) compression of $P$ with parameters $c$ and $\epsilon$. Then $P$ and $P'$ represent isomorphic posets.
\end{lemma}

\begin{proof}[Proof of lemma]
We will do the proof for left compressions. The argument for right compressions is symmetric.

Notice that if $a$ and $b$ are two endpoints of intervals of $P$, then their relation won't change, unless $a=b$. More precisely, if $a<b$ in $P$ then the corresponding points in $P'$ will maintain this relation. Similarly for $a>b$.

So if $x$ and $y$ are two intervals in $P$ with no common endpoints, then their (poset) relation is maintained in $P'$.

Now suppose that $x$ and $y$ are intervals with some common endpoints. There are a few cases to consider.

If $l_x=l_y$ then either $x,y\in L$ or $x,y\in R$, so either both are shifted, or neither. Therefore $x\|y$ both in $P$ and in $P'$.

Now suppose $l_x\neq l_y$; without loss of generality $l_x<l_y$. Also assume $r_x=r_y$. Then $l_x+\epsilon<l_y$, so $x\|y$ both in $P$ and in $P'$.

The remaining case is, without loss of generality, $r_x=l_y$. Then $r_x+\epsilon<r_y$ (unless $l_y=r_y=r_x$, which was covered in the second case), so, again $x\|y$ both in $P$ and in $P'$.
\end{proof}

Now we return to the proof of the theorem. We will perform left and right compressions until no common endpoints remain except for twins. Let $x$, $y$ be two intervals with a common endpoint, but $x\neq y$. Let $\epsilon=\frac12\min\{|a-b|:\text{$a$ and $b$ are distinct endpoints}\}$, as above.

\begin{itemize}
\item
If $l_x=l_y$ and $r_x\neq r_y$, perform a right compression with $c=\min\{r_x,r_y\}$ and $\epsilon$.
\item
If $r_x=r_y$ and $l_x\neq l_y$, perform a left compression with $c=\max\{l_x,l_y\}$ and $\epsilon$.
\item If $l_x<r_x=l_y<r_y$ (or vice versa) either a left or a right shift will work with $c=r_x=l_y$.
\end{itemize}

Note that even though the definition of $\epsilon$ looks the same in every step, the actual value will change as the representation changes. Indeed, it is easy to see that $\epsilon$ is getting halved in every step.

If $P$ started with a twin-free representation, then we have arrived to a distinguishing representation, so part~(\ref{part:1}) is proven.

If $P$ had twins, those are still present at the representation. Let $x$ and $y$ be identical intervals of the representation, and let $\epsilon=\frac12\min\{|a-b|:\text{$a$ and $b$ are distinct endpoints}\}$ again. If $0\not\in S$, then the length of $x$ (and hence the length of $y$) is positive. Note that this length is at least $\epsilon$. Move $x$ by $\epsilon$ to the right, that is, replace $x$ with the interval $[l_x+\epsilon,r_x+\epsilon]$. The new representation will not have the $x$,$y$ twin and represents the same poset. Repeat this until all twins disappear.
\end{proof}


