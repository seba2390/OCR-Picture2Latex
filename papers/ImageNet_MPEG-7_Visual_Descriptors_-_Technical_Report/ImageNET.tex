%% bare_conf.tex
%% V1.4a
%% 2014/09/17
%% by Michael Shell
%% See:
%% http://www.michaelshell.org/
%% for current contact information.
%%
%% This is a skeleton file demonstrating the use of IEEEtran.cls
%% (requires IEEEtran.cls version 1.8a or later) with an IEEE
%% conference paper.
%%
%% Support sites:
%% http://www.michaelshell.org/tex/ieeetran/
%% http://www.ctan.org/tex-archive/macros/latex/contrib/IEEEtran/
%% and
%% http://www.ieee.org/

%%*************************************************************************
%% Legal Notice:
%% This code is offered as-is without any warranty either expressed or
%% implied; without even the implied warranty of MERCHANTABILITY or
%% FITNESS FOR A PARTICULAR PURPOSE! 
%% User assumes all risk.
%% In no event shall IEEE or any contributor to this code be liable for
%% any damages or losses, including, but not limited to, incidental,
%% consequential, or any other damages, resulting from the use or misuse
%% of any information contained here.
%%
%% All comments are the opinions of their respective authors and are not
%% necessarily endorsed by the IEEE.
%%
%% This work is distributed under the LaTeX Project Public License (LPPL)
%% ( http://www.latex-project.org/ ) version 1.3, and may be freely used,
%% distributed and modified. A copy of the LPPL, version 1.3, is included
%% in the base LaTeX documentation of all distributions of LaTeX released
%% 2003/12/01 or later.
%% Retain all contribution notices and credits.
%% ** Modified files should be clearly indicated as such, including  **
%% ** renaming them and changing author support contact information. **
%%
%% File list of work: IEEEtran.cls, IEEEtran_HOWTO.pdf, bare_adv.tex,
%%                    bare_conf.tex, bare_jrnl.tex, bare_conf_compsoc.tex,
%%                    bare_jrnl_compsoc.tex, bare_jrnl_transmag.tex
%%*************************************************************************


% *** Authors should verify (and, if needed, correct) their LaTeX system  ***
% *** with the testflow diagnostic prior to trusting their LaTeX platform ***
% *** with production work. IEEE's font choices and paper sizes can       ***
% *** trigger bugs that do not appear when using other class files.       ***                          ***
% The testflow support page is at:
% http://www.michaelshell.org/tex/testflow/



\documentclass[conference]{IEEEtran}
% Some Computer Society conferences also require the compsoc mode option,
% but others use the standard conference format.
%
% If IEEEtran.cls has not been installed into the LaTeX system files,
% manually specify the path to it like:
% \documentclass[conference]{../sty/IEEEtran}





% Some very useful LaTeX packages include:
% (uncomment the ones you want to load)


% *** MISC UTILITY PACKAGES ***
%
%\usepackage{ifpdf}
% Heiko Oberdiek's ifpdf.sty is very useful if you need conditional
% compilation based on whether the output is pdf or dvi.
% usage:
% \ifpdf
%   % pdf code
% \else
%   % dvi code
% \fi
% The latest version of ifpdf.sty can be obtained from:
% http://www.ctan.org/tex-archive/macros/latex/contrib/oberdiek/
% Also, note that IEEEtran.cls V1.7 and later provides a builtin
% \ifCLASSINFOpdf conditional that works the same way.
% When switching from latex to pdflatex and vice-versa, the compiler may
% have to be run twice to clear warning/error messages.






% *** CITATION PACKAGES ***
%
%\usepackage{cite}
% cite.sty was written by Donald Arseneau
% V1.6 and later of IEEEtran pre-defines the format of the cite.sty package
% \cite{} output to follow that of IEEE. Loading the cite package will
% result in citation numbers being automatically sorted and properly
% "compressed/ranged". e.g., [1], [9], [2], [7], [5], [6] without using
% cite.sty will become [1], [2], [5]--[7], [9] using cite.sty. cite.sty's
% \cite will automatically add leading space, if needed. Use cite.sty's
% noadjust option (cite.sty V3.8 and later) if you want to turn this off
% such as if a citation ever needs to be enclosed in parenthesis.
% cite.sty is already installed on most LaTeX systems. Be sure and use
% version 5.0 (2009-03-20) and later if using hyperref.sty.
% The latest version can be obtained at:
% http://www.ctan.org/tex-archive/macros/latex/contrib/cite/
% The documentation is contained in the cite.sty file itself.






% *** GRAPHICS RELATED PACKAGES ***
%
\ifCLASSINFOpdf
  % \usepackage[pdftex]{graphicx}
  % declare the path(s) where your graphic files are
  % \graphicspath{{../pdf/}{../jpeg/}}
  % and their extensions so you won't have to specify these with
  % every instance of \includegraphics
  % \DeclareGraphicsExtensions{.pdf,.jpeg,.png}
\else
  % or other class option (dvipsone, dvipdf, if not using dvips). graphicx
  % will default to the driver specified in the system graphics.cfg if no
  % driver is specified.
  % \usepackage[dvips]{graphicx}
  % declare the path(s) where your graphic files are
  % \graphicspath{{../eps/}}
  % and their extensions so you won't have to specify these with
  % every instance of \includegraphics
  % \DeclareGraphicsExtensions{.eps}
\fi
% graphicx was written by David Carlisle and Sebastian Rahtz. It is
% required if you want graphics, photos, etc. graphicx.sty is already
% installed on most LaTeX systems. The latest version and documentation
% can be obtained at: 
% http://www.ctan.org/tex-archive/macros/latex/required/graphics/
% Another good source of documentation is "Using Imported Graphics in
% LaTeX2e" by Keith Reckdahl which can be found at:
% http://www.ctan.org/tex-archive/info/epslatex/
%
% latex, and pdflatex in dvi mode, support graphics in encapsulated
% postscript (.eps) format. pdflatex in pdf mode supports graphics
% in .pdf, .jpeg, .png and .mps (metapost) formats. Users should ensure
% that all non-photo figures use a vector format (.eps, .pdf, .mps) and
% not a bitmapped formats (.jpeg, .png). IEEE frowns on bitmapped formats
% which can result in "jaggedy"/blurry rendering of lines and letters as
% well as large increases in file sizes.
%
% You can find documentation about the pdfTeX application at:
% http://www.tug.org/applications/pdftex





% *** MATH PACKAGES ***
%
%\usepackage[cmex10]{amsmath}
% A popular package from the American Mathematical Society that provides
% many useful and powerful commands for dealing with mathematics. If using
% it, be sure to load this package with the cmex10 option to ensure that
% only type 1 fonts will utilized at all point sizes. Without this option,
% it is possible that some math symbols, particularly those within
% footnotes, will be rendered in bitmap form which will result in a
% document that can not be IEEE Xplore compliant!
%
% Also, note that the amsmath package sets \interdisplaylinepenalty to 10000
% thus preventing page breaks from occurring within multiline equations. Use:
%\interdisplaylinepenalty=2500
% after loading amsmath to restore such page breaks as IEEEtran.cls normally
% does. amsmath.sty is already installed on most LaTeX systems. The latest
% version and documentation can be obtained at:
% http://www.ctan.org/tex-archive/macros/latex/required/amslatex/math/





% *** SPECIALIZED LIST PACKAGES ***
%
%\usepackage{algorithmic}
% algorithmic.sty was written by Peter Williams and Rogerio Brito.
% This package provides an algorithmic environment fo describing algorithms.
% You can use the algorithmic environment in-text or within a figure
% environment to provide for a floating algorithm. Do NOT use the algorithm
% floating environment provided by algorithm.sty (by the same authors) or
% algorithm2e.sty (by Christophe Fiorio) as IEEE does not use dedicated
% algorithm float types and packages that provide these will not provide
% correct IEEE style captions. The latest version and documentation of
% algorithmic.sty can be obtained at:
% http://www.ctan.org/tex-archive/macros/latex/contrib/algorithms/
% There is also a support site at:
% http://algorithms.berlios.de/index.html
% Also of interest may be the (relatively newer and more customizable)
% algorithmicx.sty package by Szasz Janos:
% http://www.ctan.org/tex-archive/macros/latex/contrib/algorithmicx/




% *** ALIGNMENT PACKAGES ***
%
%\usepackage{array}
% Frank Mittelbach's and David Carlisle's array.sty patches and improves
% the standard LaTeX2e array and tabular environments to provide better
% appearance and additional user controls. As the default LaTeX2e table
% generation code is lacking to the point of almost being broken with
% respect to the quality of the end results, all users are strongly
% advised to use an enhanced (at the very least that provided by array.sty)
% set of table tools. array.sty is already installed on most systems. The
% latest version and documentation can be obtained at:
% http://www.ctan.org/tex-archive/macros/latex/required/tools/


% IEEEtran contains the IEEEeqnarray family of commands that can be used to
% generate multiline equations as well as matrices, tables, etc., of high
% quality.




% *** SUBFIGURE PACKAGES ***
%\ifCLASSOPTIONcompsoc
%  \usepackage[caption=false,font=normalsize,labelfont=sf,textfont=sf]{subfig}
%\else
%  \usepackage[caption=false,font=footnotesize]{subfig}
%\fi
% subfig.sty, written by Steven Douglas Cochran, is the modern replacement
% for subfigure.sty, the latter of which is no longer maintained and is
% incompatible with some LaTeX packages including fixltx2e. However,
% subfig.sty requires and automatically loads Axel Sommerfeldt's caption.sty
% which will override IEEEtran.cls' handling of captions and this will result
% in non-IEEE style figure/table captions. To prevent this problem, be sure
% and invoke subfig.sty's "caption=false" package option (available since
% subfig.sty version 1.3, 2005/06/28) as this is will preserve IEEEtran.cls
% handling of captions.
% Note that the Computer Society format requires a larger sans serif font
% than the serif footnote size font used in traditional IEEE formatting
% and thus the need to invoke different subfig.sty package options depending
% on whether compsoc mode has been enabled.
%
% The latest version and documentation of subfig.sty can be obtained at:
% http://www.ctan.org/tex-archive/macros/latex/contrib/subfig/




% *** FLOAT PACKAGES ***
%
%\usepackage{fixltx2e}
% fixltx2e, the successor to the earlier fix2col.sty, was written by
% Frank Mittelbach and David Carlisle. This package corrects a few problems
% in the LaTeX2e kernel, the most notable of which is that in current
% LaTeX2e releases, the ordering of single and double column floats is not
% guaranteed to be preserved. Thus, an unpatched LaTeX2e can allow a
% single column figure to be placed prior to an earlier double column
% figure. The latest version and documentation can be found at:
% http://www.ctan.org/tex-archive/macros/latex/base/


%\usepackage{stfloats}
% stfloats.sty was written by Sigitas Tolusis. This package gives LaTeX2e
% the ability to do double column floats at the bottom of the page as well
% as the top. (e.g., "\begin{figure*}[!b]" is not normally possible in
% LaTeX2e). It also provides a command:
%\fnbelowfloat
% to enable the placement of footnotes below bottom floats (the standard
% LaTeX2e kernel puts them above bottom floats). This is an invasive package
% which rewrites many portions of the LaTeX2e float routines. It may not work
% with other packages that modify the LaTeX2e float routines. The latest
% version and documentation can be obtained at:
% http://www.ctan.org/tex-archive/macros/latex/contrib/sttools/
% Do not use the stfloats baselinefloat ability as IEEE does not allow
% \baselineskip to stretch. Authors submitting work to the IEEE should note
% that IEEE rarely uses double column equations and that authors should try
% to avoid such use. Do not be tempted to use the cuted.sty or midfloat.sty
% packages (also by Sigitas Tolusis) as IEEE does not format its papers in
% such ways.
% Do not attempt to use stfloats with fixltx2e as they are incompatible.
% Instead, use Morten Hogholm'a dblfloatfix which combines the features
% of both fixltx2e and stfloats:
%
% \usepackage{dblfloatfix}
% The latest version can be found at:
% http://www.ctan.org/tex-archive/macros/latex/contrib/dblfloatfix/




% *** PDF, URL AND HYPERLINK PACKAGES ***
%
%\usepackage{url}
% url.sty was written by Donald Arseneau. It provides better support for
% handling and breaking URLs. url.sty is already installed on most LaTeX
% systems. The latest version and documentation can be obtained at:
% http://www.ctan.org/tex-archive/macros/latex/contrib/url/
% Basically, \url{my_url_here}.




% *** Do not adjust lengths that control margins, column widths, etc. ***
% *** Do not use packages that alter fonts (such as pslatex).         ***
% There should be no need to do such things with IEEEtran.cls V1.6 and later.
% (Unless specifically asked to do so by the journal or conference you plan
% to submit to, of course. )

\usepackage{url}
\usepackage{graphicx}

\newcommand{\midtilde}{\raisebox{-1mm}{\textasciitilde}}


\begin{document}
%
% paper title
% Titles are generally capitalized except for words such as a, an, and, as,
% at, but, by, for, in, nor, of, on, or, the, to and up, which are usually
% not capitalized unless they are the first or last word of the title.
% Linebreaks \\ can be used within to get better formatting as desired.
% Do not put math or special symbols in the title.
\title{ImageNet MPEG-7 Visual Descriptors \\ \huge{Technical Report}}



% author names and affiliations
\author{\IEEEauthorblockN{Fr\'{e}d\'{e}ric Rayar
\IEEEauthorblockA{Universit\'{e} Fran\c{c}ois Rabelais of Tours,\\ LI EA-6300, France\\
frederic.rayar@univ-tours.fr}}
}

% conference papers do not typically use \thanks and this command
% is locked out in conference mode. If really needed, such as for
% the acknowledgment of grants, issue a \IEEEoverridecommandlockouts
% after \documentclass

% for over three affiliations, or if they all won't fit within the width
% of the page, use this alternative format:
% 
%\author{\IEEEauthorblockN{Michael Shell\IEEEauthorrefmark{1},
%Homer Simpson\IEEEauthorrefmark{2},
%James Kirk\IEEEauthorrefmark{3}, 
%Montgomery Scott\IEEEauthorrefmark{3} and
%Eldon Tyrell\IEEEauthorrefmark{4}}
%\IEEEauthorblockA{\IEEEauthorrefmark{1}School of Electrical and Computer Engineering\\
%Georgia Institute of Technology,
%Atlanta, Georgia 30332--0250\\ Email: see http://www.michaelshell.org/contact.html}
%\IEEEauthorblockA{\IEEEauthorrefmark{2}Twentieth Century Fox, Springfield, USA\\
%Email: homer@thesimpsons.com}
%\IEEEauthorblockA{\IEEEauthorrefmark{3}Starfleet Academy, San Francisco, California 96678-2391\\
%Telephone: (800) 555--1212, Fax: (888) 555--1212}
%\IEEEauthorblockA{\IEEEauthorrefmark{4}Tyrell Inc., 123 Replicant Street, Los Angeles, California 90210--4321}}




% use for special paper notices
%\IEEEspecialpapernotice{(Invited Paper)}




% make the title area
\maketitle

% As a general rule, do not put math, special symbols or citations
% in the abstract
\begin{abstract}
ImageNet is a large scale and publicly available image database. It currently offers more than 14 millions of images, organised according to the WordNet hierarchy. One of the main objective of the creators is to provide to the research community a relevant database for visual recognition applications such as object recognition, image classification or object localisation. However, only a few visual descriptors of the images are available to be used  by the researchers. Only SIFT-based features have been extracted from a subset of the collection. This technical report presents the extraction of some MPEG-7 visual descriptors from the ImageNet database. These descriptors are made publicly available in an effort towards open research.

\end{abstract}

% no keywords




% For peer review papers, you can put extra information on the cover
% page as needed:
% \ifCLASSOPTIONpeerreview
% \begin{center} \bfseries EDICS Category: 3-BBND \end{center}
% \fi
%
% For peerreview papers, this IEEEtran command inserts a page break and
% creates the second title. It will be ignored for other modes.
\IEEEpeerreviewmaketitle


\section{Introduction}

ImageNet~\cite{Deng:2009} is a large scale image ontology database. The investigators of this project, namely Pr. Li Fei-Fei and her colleagues at Princeton University, have started their efforts in 2009. They believe that such a database is \textit{``a critical resource for developing advanced, large-scale content-based image search and image understanding algorithms, as well as for providing critical training and benchmarking data for such algorithms.''} Such a resource can be useful in visual recognition applications such as object recognition, image classification or object localisation.\\

ImageNet is organised according to the well-known WordNet~\cite{Fellbaum:1998} ontology. It is a \textit{``large lexical database of English. Nouns, verbs, adjectives and adverbs are grouped into sets of cognitive synonyms (synsets), each expressing a distinct concept''}. Hence, ImageNet can be viewed as a tree where each node corresponds to a synset. Each node contains 500 to 1000 images to illustrate the associated synset. Images of each concept have been quality-controlled and human-annotated. More details on the curation process can be found in the creators paper~\cite{Deng:2009}. The database and related resources are publicly available at \url{http://www.image-net.org}. Currently, ImageNet has 14,197,122 images and 21,841 synsets indexed. In addition to the images, one can find SIFT~\cite{Lowe:2004} based features or object attributes, but only for images of small subsets of the 21,841 synsets (1000~\cite{ImageNet:sbow} and 384~\cite{ImageNet:oa} synsets respectively).\\

In this work, we aim at providing to the community two MPEG-7 visual descriptors for the whole ImageNet dataset. The rest of the paper is organises as follows: the MPEG-7 visual description standard is succinctly presented in Section~\ref{sec:mpeg}. The two computed descriptors, namely Colour Layout Descriptor (CLD) and Edge Histogram Descriptor (EHD), are described in subsection~\ref{sec:cld} and~\ref{sec:ehd} respectively. Finally, the provided data sets are presented in Section~\ref{sec:data}.\\


\section{MPEG-7}
\label{sec:mpeg}
Following its previous MPEG version, where the emphasis was put on encoding, MPEG-7~\cite{Manjunath:2002} aims at describing multimedia content, such as image, audio and video.
MPEG-7 specifies, among others, visual descriptors that have been introduced and stressed out in the research community. The objective is to provide a set of
standardised descriptors to characterise multimedia content in order to guaranty interoperability and achieve content based multimedia retrieval efficiently. \\

These visual descriptors~\cite{Sikora:2001} are organised in four categories:
\begin{itemize}
	\item colour descriptors~\cite{Manjunath:2001}
	\item texture descriptors~\cite{Manjunath:2001}
	\item shape descriptors~\cite{Bober:2001}
	\item motion descriptors (for videos)~\cite{Jeannin:2001}
\end{itemize}
One can refer to the references attached to each category to find a detailed description of each descriptors.\\


We describe below the two descriptors that have been computed in this work, namely Colour Layout Descriptor and Edge Histogram Descriptor (colour and texture descriptor respectively). Note that the original definition are presented in this paper. One can find some studies that have improve the basic definition of CLD and EHD (\textit{e.g.} in~\cite{ehd:2002}).\\

\subsection{Colour Layout Descriptor}
\label{sec:cld}
The CLD is designed to capture the spatial distribution of dominant
colours in an image. The colours are expressed in the YCbCr colour space. CLD main advantages are first that it is a very compact descriptor, therefore it fits perfectly for fast browsing and search applications. Second, it is resolution invariant, thanks to its definition.\\

The feature extraction process consists in four stages:
\begin{itemize}
	\item \textit{image partitioning}: the input picture is divided into \\$8\times8=64$ blocks. This guarantee the resolution or scale invariance.
	\item \textit{representative colour detection}: one representative colour is computed for each block. To select the representative colour, any method can be applied, but 
the average colour of the block is usually used. An $8\times8$ icon is obtained (see Figure~\ref{fig:colorgrid}).
	\item \textit{DCT transformation}: each colour channel of the icon is transformed into a series of coefficients by performing a $8\times8$ Discrete Cosine Transform (DCT). Three series of 64 DCT coefficients are computed.
	\item \textit{quantization}: a few low-frequency coefficients are selected using zigzag scanning (see Figure~\ref{fig:zigzag}) and quantised to form the descriptor.\\
\end{itemize}

Hence, we obtained a descriptor with a size of $3\times64=192$. One can find more details on the the CLD computation in~\cite{Manjunath:2001} or~\cite{Manjunath:2002}.

\begin{figure}[!ht]
\center
	\includegraphics[width=0.9\linewidth]{colorgrid.png}
	\caption{CLD: representative colour detection on a $8\times8$ partitioned image.}
	\label{fig:colorgrid}
\end{figure}

\begin{figure}[!ht]
\center
	\includegraphics[width=0.4\linewidth]{zigzag.png}
	\caption{CLD: zig-zag re-ordering.}
	\label{fig:zigzag}
\end{figure}



\subsection{Edge Histogram Descriptor}
\label{sec:ehd}
EHD~\cite{ehd:2000} captures the spatial distribution of five types of edges in an image. In that sense, CLD and EHD can be viewed as alike as two peas in a pod. Figure~\ref{fig:edges} shows the five edges considered.\\

The EHD feature extraction is quite straightforward: first, an image is divided using a $4\times4$ blocks. For each block, an histogram of the distribution of the considered edges is built (see Figure~\ref{fig:zoning}). Therefore, we have $4\times4\times5=80$ values that constitutes the texture descriptor.\\

One can find more details on the the EHD computation in~\cite{Manjunath:2001}.
	
\begin{figure}[!ht]
\center
	\includegraphics[width=0.8\linewidth]{edges.png}
	\caption{EHD: 5 considered types of contour. Illustration from \cite{ehd:2002}.}
	\label{fig:edges}
\end{figure}

\begin{figure}[!ht]
\center
	\includegraphics[width=0.8\linewidth]{zoning.png}
	\caption{EHD: histogram computation for each cell. Illustration from \cite{ehd:2002}.}
	\label{fig:zoning}
\end{figure}

\section{Files Description}
\label{sec:data}

\subsection{Database}
\begin{itemize}
	\item All the 21,841 synsets have been processed
	\item A total of 14,197,060 images have been processed.
	\item The remaining 62 images have not processed. They  mostly correspond to corrupted image files.
\end{itemize}

\subsection{Files format}
\begin{itemize}
	\item There is one file per synset, named \textit{synset\_id.ext}\\ with ext = \{cld, ehd\}.
	\item Each file contains the descriptors for the $n$ images associated with its synset. 
	\item Hence, for a given synset, the associated file contains $n$ lines with the scheme
\begin{verbatim}
	image_id;value_1;value_2;...;value_d;
\end{verbatim}	
with 	
\[
	d = \left\{
		\begin{array}{rc}
			192 & \textrm{if CLD} \\
			 80 & \textrm{if EHD}
		\end{array}
	\right.
\]
\end{itemize}

\subsection{Files availability}
\begin{itemize}
	\item The following archives have been made publicly available:
		\begin{itemize}
			\item \textbf{ImageNet\_MPEG-7\_CLD\_whole.zip}~\footnote{\url{http://rfai.li.univ-tours.fr/PublicData/ImageNetFeatures/ImageNet_MPEG-7_CLD_whole.zip}} (\midtilde 1GB)
			\item \textbf{ImageNet\_MPEG-7\_EHD\_whole.zip}~\footnote{\url{http://rfai.li.univ-tours.fr/PublicData/ImageNetFeatures/ImageNet_MPEG-7_EHD_whole.zip}} (\midtilde 2GB)
		\end{itemize}
		\vspace{1mm}
	\item \textit{Note}: some images \textit{might} not have both CLD and EHD. However, it should not be the case. We just did not checked this.
\end{itemize}


\bibliographystyle{IEEEtran}
\bibliography{ImageNET}




% that's all folks
\end{document}


