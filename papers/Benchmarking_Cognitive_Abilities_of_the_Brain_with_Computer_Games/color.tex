We put a lot of emphasis on what colors to choose for our benchmark. The reason for this is that even the standard test requires a constant focus of 10 minutes, 
which can put a lot of pressure on one’s eyes. To ease this strain as much as we possibly could, we took lots of things into account. Firstly, we tried to maximize the contrast between the background and the figures. This means that we picked some colors that could be easily distinguished and then we ran some manual tests. The result was a significant drop in the overall burden of the eyes.

After this, we thought about how we could make the benchmark available for a wider range of people, namely for those who suffer from 
parachromatism or even
disambiguation. This is rather important as it is said that roughly 8\% of men and 0.5\% of 
women\footnote{\url{http://www.color-blindness.com/2006/04/28/colorblind-population}} suffer from one of these. 
In order for them to be able to comfortably run our benchmark, we tried 
to pick colors that are easily distinguishable even for these people.

Another problem was that we did not target a specific age group. On the contrary, we were especially curious about the results of adults, adolescents, teenagers and children. Therefore, we needed to pick a color scheme that was modern, vivid, yet not too complex and not too abstract. This, too, required a lot of experimentation.