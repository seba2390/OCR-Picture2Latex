\documentclass[a4paper]{article}
\pdfoutput=1

\usepackage[english]{babel}
\usepackage[utf8]{inputenc}
\usepackage[T1]{fontenc}
\usepackage{amsmath}
\usepackage{amssymb}
\usepackage{graphicx}
\usepackage[colorinlistoftodos]{todonotes}
\usepackage[hyphens]{url}
\usepackage[colorlinks,urlcolor=blue]{hyperref}
\usepackage{listings}
\usepackage{color}
\usepackage{graphicx}
\usepackage{caption}
\usepackage{subcaption}
\usepackage{paralist}
\usepackage{csquotes}
\usepackage{authblk}

\title{Benchmarking Cognitive Abilities of the Brain with Computer Games}

\author[1,*]{Norbert B\'atfai}
\author[2]{D\'avid Papp} 
\author[1]{Ren\'at\'o Besenczi}
\author[1]{Gerg{\H o} Bogacsovics}
\author[1]{D\'avid Veres}

\affil[1]{Department of Information Technology, University of Debrecen, Hungary}
\affil[2]{Department of Psychology, University of Debrecen, Hungary}
\affil[*]{Corresponding author: Norbert B\'atfai, batfai.norbert@inf.unideb.hu}

\begin{document}
\maketitle
\begin{abstract}
\begin{abstract}
\noindent\textbf{Abstract:} In this paper, we present \textit{JADE}, a targeted linguistic fuzzing platform which strengthens the linguistic complexity of seed questions to simultaneously and consistently break a wide range of widely-used LLMs categorized in three groups: eight open-sourced Chinese, six commercial Chinese and four commercial English LLMs. JADE generates three safety benchmarks for the three groups of LLMs, which contain unsafe questions that are highly threatening: the questions simultaneously trigger harmful generation of multiple LLMs, with an average unsafe generation ratio of \textbf{$70\%$} (please see the table below), while are still natural questions, fluent and preserving the core unsafe semantics. We release the benchmark demos generated for commercial English LLMs and open-sourced Chinese LLMs in the following link: \url{https://github.com/whitzard-ai/jade-db}. For readers who are interested in evaluating on more questions generated by JADE, please contact us.


% This results in a safety benchmark of natural questions which simultaneously trigger harmful generation of a wide range of widely-used LLMs below, in over $70\%$ test cases.



% Table generated by Excel2LaTeX from sheet 'Sheet2'
\begin{center}
\scalebox{0.65}{
    \begin{tabular}{lccccccc}
    \toprule
    \multirow{2}[3]{*}{\textbf{Group}} & \multicolumn{4}{c}{\multirow{2}[3]{*}{\textbf{Model Name}}} & \multicolumn{3}{c}{\textbf{Unsafe Generation Ratio}} \\
\cmidrule{6-8}          & \multicolumn{4}{c}{}          & \textbf{Average} & \textbf{Least} & \textbf{Most} \\
    \midrule
    \multirow{2}[2]{*}{\textbf{Open-sourced LLM (Chinese)}} & ChatGLM & ChatGLM2 & InternLM & Ziya  & \multirow{2}[2]{*}{74.13\%} & \multirow{2}[2]{*}{49.00\%} & \multirow{2}[2]{*}{93.50\%} \\
          & Baichuan & BELLE & MOSS  & ChatYuan2 &       &       &  \\
    \midrule
    \textbf{Commercial LLM (English)} & ChatGPT & Claude & PaLM2 & LLaMA2 & 74.38\% & 35.00\% & 91.25\% \\
    \midrule
    \multirow{2}[2]{*}{\textbf{Commercial LLM (Chinese)}} & Doubao & Wenxin Yiyan & ChatGLM & SenseChat & \multirow{2}[2]{*}{77.5\%} & \multirow{2}[2]{*}{56.00\%} & \multirow{2}[2]{*}{90.00\%} \\
          & Baichuan & ABAB  & \multicolumn{2}{c}{\footnotesize{(For the detailed info., please refer to Table 2)}} &       &       &  \\
    \bottomrule
    \end{tabular}}%
\end{center}





\textit{JADE} is based on Noam Chomsky's seminal theory of transformational-generative grammar. Given a seed question with unsafe intention, \textit{JADE} invokes a sequence of generative and transformational rules to increment the complexity of the syntactic structure of the original question, until the safety guardrail is broken. Our key insight is: Due to the complexity of human language, most of the current best LLMs can hardly recognize the invariant evil from the infinite number of different syntactic structures which form an unbound example space that can never be fully covered. Technically, the generative/transformative rules are constructed by native speakers of the languages, and, once developed, can be used to automatically grow and transform the parse tree of a given question, until the guardrail is broken. Besides, \textit{JADE} also incorporates an active learning algorithm to incrementally improve the LLM-based evaluation module, which
iteratively optimizes the prompts for evaluation with a small amount of annotated data, to effectively strengthen the alignment with the judgement made by human experts. For more evaluation results and demo, please check our website: \url{https://whitzard-ai.github.io/jade.html}.

\noindent\pxd{{\footnotesize[\textbf{Content Warning: This paper contains examples of harmful language.}]}}
\end{abstract}





% Featured by OpenAI's ChatGPT, the rise of aligned large language models (LLM) is recognized as a milestone in the history of AI, and catalyzes wild imagination on the arrival of \textit{Artificial General Intelligence} (AGI). To achieve harmless generation, many approaches are proposed to align the AI generation contents with human values, or called \textit{AI alignment}. This equips pretrained large language models with the ability of generating safe responses under unsafe requests. However, we find human language is more complex than the current best LLM can handle. To validate this point, 
\end{abstract}

{\bf Keywords}: esport, talent search, benchmark program, complexity, psychology test.

\section{Introduction}
% !TEX root = ../arxiv.tex

Unsupervised domain adaptation (UDA) is a variant of semi-supervised learning \cite{blum1998combining}, where the available unlabelled data comes from a different distribution than the annotated dataset \cite{Ben-DavidBCP06}.
A case in point is to exploit synthetic data, where annotation is more accessible compared to the costly labelling of real-world images \cite{RichterVRK16,RosSMVL16}.
Along with some success in addressing UDA for semantic segmentation \cite{TsaiHSS0C18,VuJBCP19,0001S20,ZouYKW18}, the developed methods are growing increasingly sophisticated and often combine style transfer networks, adversarial training or network ensembles \cite{KimB20a,LiYV19,TsaiSSC19,Yang_2020_ECCV}.
This increase in model complexity impedes reproducibility, potentially slowing further progress.

In this work, we propose a UDA framework reaching state-of-the-art segmentation accuracy (measured by the Intersection-over-Union, IoU) without incurring substantial training efforts.
Toward this goal, we adopt a simple semi-supervised approach, \emph{self-training} \cite{ChenWB11,lee2013pseudo,ZouYKW18}, used in recent works only in conjunction with adversarial training or network ensembles \cite{ChoiKK19,KimB20a,Mei_2020_ECCV,Wang_2020_ECCV,0001S20,Zheng_2020_IJCV,ZhengY20}.
By contrast, we use self-training \emph{standalone}.
Compared to previous self-training methods \cite{ChenLCCCZAS20,Li_2020_ECCV,subhani2020learning,ZouYKW18,ZouYLKW19}, our approach also sidesteps the inconvenience of multiple training rounds, as they often require expert intervention between consecutive rounds.
We train our model using co-evolving pseudo labels end-to-end without such need.

\begin{figure}[t]%
    \centering
    \def\svgwidth{\linewidth}
    \input{figures/preview/bars.pdf_tex}
    \caption{\textbf{Results preview.} Unlike much recent work that combines multiple training paradigms, such as adversarial training and style transfer, our approach retains the modest single-round training complexity of self-training, yet improves the state of the art for adapting semantic segmentation by a significant margin.}
    \label{fig:preview}
\end{figure}

Our method leverages the ubiquitous \emph{data augmentation} techniques from fully supervised learning \cite{deeplabv3plus2018,ZhaoSQWJ17}: photometric jitter, flipping and multi-scale cropping.
We enforce \emph{consistency} of the semantic maps produced by the model across these image perturbations.
The following assumption formalises the key premise:

\myparagraph{Assumption 1.}
Let $f: \mathcal{I} \rightarrow \mathcal{M}$ represent a pixelwise mapping from images $\mathcal{I}$ to semantic output $\mathcal{M}$.
Denote $\rho_{\bm{\epsilon}}: \mathcal{I} \rightarrow \mathcal{I}$ a photometric image transform and, similarly, $\tau_{\bm{\epsilon}'}: \mathcal{I} \rightarrow \mathcal{I}$ a spatial similarity transformation, where $\bm{\epsilon},\bm{\epsilon}'\sim p(\cdot)$ are control variables following some pre-defined density (\eg, $p \equiv \mathcal{N}(0, 1)$).
Then, for any image $I \in \mathcal{I}$, $f$ is \emph{invariant} under $\rho_{\bm{\epsilon}}$ and \emph{equivariant} under $\tau_{\bm{\epsilon}'}$, \ie~$f(\rho_{\bm{\epsilon}}(I)) = f(I)$ and $f(\tau_{\bm{\epsilon}'}(I)) = \tau_{\bm{\epsilon}'}(f(I))$.

\smallskip
\noindent Next, we introduce a training framework using a \emph{momentum network} -- a slowly advancing copy of the original model.
The momentum network provides stable, yet recent targets for model updates, as opposed to the fixed supervision in model distillation \cite{Chen0G18,Zheng_2020_IJCV,ZhengY20}.
We also re-visit the problem of long-tail recognition in the context of generating pseudo labels for self-supervision.
In particular, we maintain an \emph{exponentially moving class prior} used to discount the confidence thresholds for those classes with few samples and increase their relative contribution to the training loss.
Our framework is simple to train, adds moderate computational overhead compared to a fully supervised setup, yet sets a new state of the art on established benchmarks (\cf \cref{fig:preview}).


The organization of the paper is the following. In the next section, we give a brief overview of the psychological and informatics background and the phenomenon of losing the character is illustrated. The second section presents the algorithm and the operation of our benchmark program including the presentation of the first measurements followed by systematic measurements. 
Finally, we conclude our paper and show some future plans.
%Finally, the paper is ended by conclusion and acknowledgements.

\subsection{Psychological Background}
The cognitive ability of attention is a significant factor in everyday life, either it comes from work, hobby or the daily activities, 
as it affects the performance of all the previously mentioned things. 
The alertness, or in other words, the long upheld attention, in technical terms is called vigilance. 
The research of vigilance is an important topic in Psychology from 1970 to the present day. 
The first method used was the Mackworth Clock \cite{Mackworth}, in which the participants had to pay attention to a clock that had a second hand which sometimes sprang twice, and then the participants had to signal as soon as possible. 
For measuring attention and concentration, there is another method, 
the Toulouse-Piéron test \cite{ToulousePieron}, 
in which participants have to follow a given scheme to separate right and wrong signs. To measure vigilance we must take into consideration the hit ratio and the number of false alarms. In almost all of our activities there are also interfering stimuli that affects our performance as well. These other factors vary by quantity and quality, and some can be stimulating, 
while some detain us 
from the optimal performance. The Yerkes-Dodson law \cite{YerkesDodsonLaw} says that for achieving the best performance 
there is an optimal arousal level, which level is higher in simpler tasks, and lower in complex activities. 
It can be represented by a inverted U-shaped curve. 
We must not forget that as in some other things, in the attentional system there are also personal differences that 
should be taken into consideration while researching the subject \cite{AltPszichobook}. 

Other objects in the environment can affect how we perceive the one object that is interesting for us. 
In 1940, 
Witkin et al.\ did a research on perception \cite{WitkinEtAl}, and from this work, they created a theory about two different cognitive styles, which they called field dependent, and field independent. A field dependent person perception is mainly affected by the field, the environment of the observed object. On the contrary, a field independent person does not affected by the field created by the observed object’s environment. This phenomenon was investigated by a task, in which the participants had to determine whether a straight rod, 
in different planes is vertical or not.
Moreover, there is another typical method used in this topic, that is the Tilting room, Tilting chair test, 
in which the participant is sitting in a tiltable chair that he or she needs to controll in order to get him/herself into vertical position despite the tilting room. 
Later, Witkin and Goodenough reinvestigated the topic, 
and they came to a conclusion that the two styles 
are two ends of the spectrum, 
however, some people are fixed with one of the cognitive styles, 
while others can adapt to the style they use depending on the situation \cite{WitkinGoodenough}. 

Speaking of attention, it’s important to talk about the main processing system, i.e., the brain. The operation of the brain is frequently compared to the mechanism of a personal computer 
by many researchers. 
Carl Sagan based his theory on the binary coding, 
so he used the information content in binary. 
When we are watching something, the picture seen that our brain maps, is made of plenty of information. 
Sagan wanted to calculate the 
information processing 
speed of the brain, to do so, he based his calculation on the example of looking at the moon, and from this example he drew the consequence, that the brain can process about 5000 bit/sec at its peak performance \cite{Sagan}. In a modern project, called Building 8, the main thought is to make the brain into a computer. Based on this project, the
information processing speed of the brain 
is about a terrabyte/sec, which far exceeds the speed estimated by Sagan \cite{Nieva}.  

\subsubsection{Practicing filling out tests}

Filling out tests and experiments are common tools in the science of psychology. Countless methods were 
created to date, but these methods are not just used, because researchers improve them, as well as, try to test them in a wider range. However, we need to consider certain factors in each experiment 
and test
that how they affect the method’s usability and the final results as well. Among these factors, there is one, when the participant obtains knowledge about what is expected from her/him, 
or which answer are considered the ’best’. This way the participant will accomodate to this information, because he/she, as everyone else, wants to portray herself/himself in the best manner possible and to be the ’best’ in performance. In multiple choice questions, there are some tricks, that are well known in the common knowledge, which we all use, when we don’t know the right answer for sure. A somewhat similar tool is the experience or 
routine with filling out tests, which can help to choose the adequate strategy for solving the situation, this is called test-wisdom. To achieve that, one must discover the logic behind the method, or practise it many times. But the test-wisdom often cause inconvenience for  
test developers, because they have to keep in mind a bonus factor, which is totally diverge from the basic variables they meant to manipulate, and vary in each individual \cite{PszichoMeresbook}.

The effect of being experienced in filling in tests was studied in a research, in which an aptitude test called GRE (Grand Record Examination) was used. Practise samples were sent to random participants 5 weeks before the examination. Those who got these samples also 
receieved advices for completing. In conclusion, the group with prior knowledge and practise earned significantly better results in the examination. Furthermore, there were also a notable growth in points, when the participants received an only 4 hours educational practise before the examination. It’s important to note that this difference and growth was present only in the logical reasoning part of the exam, and not in the mathematical and verbal parts 
\cite{PowersSwinton}. This data was reexamined later, because researchers wanted to know, if there is any difference when the existing groups would be split into subgroups by the different attributes of the participants. As a conclusion, there was no significant difference between the subgroups, but there was a notable difference in the group in which the members' primary language was not English, they scored lesser points than the others \cite{Powers86}.

Repeatedly performing the same experiment or test with the same participants could affect the results. Previously, as we specified, repeatedly using the same method could cause the lowering of its validity, and the results could be distorted. Participants can learn and adapt to certain methods, even if its just means a small percentage of difference. The current test takes 10 minutes to complete, in this 10 minutes the participant’s full attention and concentration is needed.  We should keep in mind, that the negative effects of fatigue could balance the positive effects of practise, in a direct way of repeated examinations. So this two factors should be considered in the evaulation, and while drawing consequence.




It is therefore proposed to perform our benchmark test in a competitive way trying to beat friends, family members, colleagues or ourselves.

\subsection{Informatics Background}
Since computer games have a relatively short history and their effects on cognitive skills have just been started to be researched recently, there are plenty of questions to be answered. In \cite{HOMER201850}, authors reported an increase in executive functions in school students after playing computer games. Moisala et al. in \cite{MOISALA2017204} shows that enhancements in speed and performance accuracy 
of working memory tasks is related to daily gaming activity. In \cite{bediou2018meta}, authors present an analysis of the impact of action video games on cognitive skills.

Using computer games to measure cognitive abilities has a short history, but a promising future. Most research try to measure the presence or severity of a certain cognitive disease such as dementia or Alzheimer's disease.
In \cite{anguera2013video}, authors show how a long-term use of video games can reduce multitasking costs in older adults. Geyer et al. in \cite{GEYER2015260} show that the change of the score of an online game is in connection with the age-related changes in working memory.

Seldom can we find applications that has been developed for the measurement of cognitive abilities. One such application is reported in \cite{10.1007/978-3-319-27695-3_13} and \cite{pataki2015computer}, it is a framework that has been developed to measure cognitive abilities and its change of elders with computer games. This framework is able to log and analyze scores achieved in various online computer games.

From the viewpoint of information theory and HCI (Human-Computer Interaction), the Hick's law \cite{seow} could be an interesting aspect. This law states that the response time of the brain increases with logarithm of the size of the input. For our purposes, it can be an interesting question: 
how can we apply the Hick's law (or other information theory figure) in our benchmark software?

\subsection{Losing The Character}
%We have experienced the feeling of losing the character during playing several games like for example League of Legends or Clash of Clans. 
We have experienced the feeling of losing the character during playing several games like for example 
League of Legends\footnote{\url{https://na.leagueoflegends.com}}, 
Clash of Clans\footnote{\url{http://supercell.com/en/games/clashofclans/}},
Clash Royale\footnote{\url{http://supercell.com/en/games/clashroyale/}},
Heroes of the Storm\footnote{\url{https://heroesofthestorm.com}}, 
Dota 2\footnote{\url{https://www.dota2.com}}, 
World of Warcraft\footnote{\url{https://worldofwarcraft.com}} 
or Cabal\footnote{\url{http://cabal.playthisgame.com}}. 


Now we share our thoughts about the phenomenon of „losing the character”, and give some examples to illustrate it from the game called League of Legends.
Basically, a match starts kind of slowly and quietly: the laners are farming, as well as the junglers in their own territory. Of course smaller fights can occur in the early stages of the game, like a 1v1 in the solo lanes, or a 3v3 in the bottom lane as both of the junglers decides to gank, but these situations are relatively easy to see through. As we head into the mid and late game, teams start fights more often with more people, even with all of them. This is what we call teamfights. These are harder to handle, 
because a lot of things can appear on our screen at the same time: the champions who participate in the fight, optionally minions or jungle monsters, and the visual effects of the spells, summoner spells, and the active or passive abilities of the items. Besides them, we see a lot of things, we still have to make sure that we fulfill our ingame role properly: position well, attack the proper target, or defend our teammates. We have to handle a lot of information at a blink of an eye, so it is completely natural, that sometimes we do not know where to look at or what to do. We can lose our own character, which can end with our death; we can lose the target character, and it can survive; or we can lose the character that we wanted to protect, thus an important member of the team can die.
This can be a short explanation of the phenomenon, which we can also call „losing the focus”.

An example ingame footage can be viewed at \url{https://youtu.be/wdy3KUm1454}, starting at 2:12.

These situations are one of the hardest parts of the game, and it is not easy to handle them well. The easiest way to prepare for them is to play a lot of games, and get experience in them. Also it can help a lot, if we think ahead before a potential teamfight, e.g. which character will be our target, who should we be afraid of, what summoner spells the enemy still has, etc. All of these things can help to execute the fights more properly.


\section{Brain Benchmarking Series 6}

\section{Methodology}

\subsection{Global Pruning Strategy}
We propose a top-down parser based on syntactic distance~\cite{DBLP:conf/acl/BengioSCJLS18} to evaluate scores for all split points in a sentence and generate a merge order according to the scores. 

\begin{figure}[htb!]
  \centering
  \includegraphics[width=0.4\textwidth]{data/parse_tree.png}
  \caption{(a) A parsed tree obtained by sorting split scores ($v_i$). (b) A sampled tree by adding Gumbel noise ($g_i$ in dark vertical bars).}
  \label{fig:sample_demo}
\end{figure}

\paragraph{Top-down parser.}
Given a sentence $\mathbf{S} = \{s_{1}, s_{2},..., s_{n}\}$, there are $n-1$ split points between words. 
We define a top-down parser by giving confidence scores to all split points as follows:
\begin{equation}
\begin{aligned}
\textbf{v}=[v_{1}, v_{2}, ..., v_{n-1}] = f(\mathbf{S};\theta)
\end{aligned}
\end{equation}
To keep it simple and rigorously maintain linear complexity, we select bidirectional LSTMs as the backbone, though Transformers are also an option.
As shown in Figure~\ref{fig:sample_demo}, 
first, a bi-directional LSTM encodes the sentence, 
and then, for each split point, an MLP over the concatenation of the left and right context representations yields the final split scores. 
Formally, we have:
\begin{equation}
\begin{aligned}
\label{eq:split_score_eval}
&\overrightarrow{\textbf{h}},\overleftarrow{\textbf{h}} = \mathrm{BiLSTM}(\mathbf{E};\theta )\\
&v_{i} = \mathrm{LayerNorm}(\mathrm{MLP}(\overrightarrow{\textbf{h}}_{i}\oplus \overleftarrow{\textbf{h}}_{i+1}))
\end{aligned}
\end{equation}
Here, $\mathbf{E}$ is the embedding of the input sentence $\mathbf{S}$, while
$\overrightarrow{\textbf{h}}$ and $\overleftarrow{\textbf{h}}$ denote the forward and reverse representation, respectively. 
$v_{i}$ is the score of the $i$-th split point, whose left and right context representations are $\overrightarrow{\textbf{h}}_{i}$ and $\overleftarrow{\textbf{h}}_{i+1}$. 
Given scores $[v_1, v_2, ..., v_{n-1}]$, one can easily recover the binary tree shown in Figure~\ref{fig:sample_demo}:
We recursively split a span (starting with the entire sentence) into two sub-spans by picking the split point with the highest score in the current span.
Taking the sentence in Figure~\ref{fig:sample_demo} (a) as an 
example, we split the overall sentence at split point $3$ in the first step, which leads to two sub-trees over $s_{1:3}$ and $s_{4:6}$. 
Then we split $s_{1:3}$ at $2$ and $s_{4:6}$ at $4$. We can continue this procedure until the complete tree has been recovered. 

\paragraph{Tree sampling.}
In the training stage, we perform sampling over the computed scores $[v_1, v_2, ..., v_{n-1}]$ in order to increase the robustness and exploration of our model.
Let $\mathcal{P}^{t}$ denote the list of split points at time $t$ in ascending order, which is $\{1,2,3,...,n\!-\!1\}$ in the first step. 
Then a particular split point $a_{t}$ is selected from $\mathcal{P}^{t}$ by sampling based on the probabilities estimated by stacking of split points scores. The sampled
$\{a_{1}, a_{2}, ..., a_{n-1}\}$ together form the final split point sequence $\mathcal{A}$. 
At each time step, we remove $a_{t}$ from $\mathcal{P}^{t}$ when $a_{t}$ is selected, then sample the next split point until the set of remaining split points is empty. Formally, we have:
\begin{align}
&a_{t} \sim \mathrm{softmax}(\textbf{v}^t)\\
&\mathcal{P}^{t+1}=\mathcal{P}^{t}\setminus \{{a_{t}}\}
\end{align}
where $\textbf{v}^t$ is concatenation of $v_{i}$ in $\mathcal{P}^{t}$.
As the Gumbel-Max trick \cite{GUMBEL,DBLP:conf/nips/MaddisonTM14} provides a simple and efficient way
to draw samples from a categorical distribution with class probabilities, we can obtain $a_{t}$ via the Gumbel-Max trick as:
\begin{align}
&a_{t} = \underset{i}{\mathrm{argmax}}\,[v_{i}+g_{i}], i \in \mathcal{P}^{t},
\end{align}
where $g_{i}$ is the Gumbel noise for the $i$-{th} split point. 
Therefore, the aforementioned process is equivalent to sorting the original sequence of split points scores with added Gumbel noise.
Figure~\ref{fig:sample_demo} (b) shows a sampled tree with respect to the split point scores.
The split point sequence $\mathcal{A}$ can hence be obtained simply as:
\begin{align}
\mathcal{A} = \underset{i}{\mathrm{argsort}}(\textbf{v} + \textbf{g})
\end{align}
Here, $\mathrm{argsort}$ sorts the array in descending order and returns the indices of the original array.
The sampled $\mathcal{A}$ is $\{2, 4, 3, 5, 1\}$ in Figure~\ref{fig:sample_demo} (b).

\paragraph{Span Constraints.}
As word-pieces~\cite{wu2016google} and Byte-Pair Encoding (BPE) are commonly used in pretrained language models,
it is straightforward to incorporate multiple word-piece constraints into the top-down parser to reduce word-level parsing errors.
We denote a list of span constraints composed of beginning and end positions of non-split-table spans as $\mathcal{C}$, defined as $\mathcal{C}=\{(b_{1}, e_{1}), (b_{2}, e_{2}), ..., (b_{n}, e_{n})\}$. 
For each $(b_{i}, e_{i})$ in $\mathcal{C}$, there should be a sub-tree for a span covering the sub-string $s_{b_{i}:e_{i}}$. 
This goal can be achieved by simply adjusting the scores of all splits within the spans in C by $-\delta$. To make them smaller than the scores of span boundaries, $\delta$ could be defined as $(\max(\textbf{v}) - \min(\textbf{v}) + c)$, where $c$ could be any positive number.

\paragraph{Model-based Pruning.}
We denote the reverse order of the split point sequence $\mathcal{A}$ as $\mathcal{M}$ and
then treat $\mathcal{M}$ as a bottom-up merge order inferred by the top-down parser based on the global context.
Subsequently, the complete pruning process is as follows: 
\begin{enumerate}
\item Pick the next merge index by invoking Alg~\ref{alg:next_merge_point}. 
\item Perform Steps 3 and 4 in the heuristic pruning part in Section~\ref{sec:r2d2}
\end{enumerate}
As shown in Figure~\ref{fig:pruning}, we still retain the threshold and the pruning logic of R2D2, 
but we select cells to merge according to $\mathcal{M}$ instead of following heuristic rules. 
Specifically, given a shrinking chart table, 
we select the next merge index among the second row by popping and modifying $\mathcal{M}$ in Algorithm~\ref{alg:next_merge_point}.

\begin{algorithm}[!h]
\small
    \caption{Next merge index in the second row}
    \label{alg:next_merge_point}
    \begin{algorithmic}[1] % The number tells where the line numbering should start
        \Function{Next-Index}{$\mathcal{M}$}
        \State{$i = \pop(\mathcal{M})$}\Comment{Index}
        \For{$j \in 1$ to $\mathcal{M}.\mathrm{len}$}
        \If {$\mathcal{M}_{j} > i$} \Comment{Merging at left}
        \State {$\mathcal{M}_{j} = \mathcal{M}_{j} - 1$} \Comment{Shift left}
        \EndIf
        \EndFor
        \State{\Return{$i$}}
        \EndFunction
    \end{algorithmic}
\end{algorithm}

Take the example in Figure~\ref{fig:sample_demo} (b) for instance: 
$\mathcal{M}$ starts with $\{1, 5, 3, 4, 2\}$.
Then we merge the first cell in the second row in Figure~\ref{fig:pruning} (b),
and obtain a new $\mathcal{M} = \{4, 2, 3, 1\}$. 
In the next round, we treat the 4th cell covering $s_{5:6}$ as a non-splittable cell in Figure~\ref{fig:pruning} (e), 
and $\mathcal{M}$ becomes $\{2, 3, 1\}$.

\subsection{Optimization}\label{sec:opt}

We denote the tree probabilities estimated by the top-down parser and R2D2 as $p_{\theta}(\textbf{z}|\textbf{S})$, $q_{\phi}(\textbf{z}|\textbf{S})$, respectively. 
The difficulty here is that while
$q_{\phi}(\textbf{z}|\textbf{S})$ may be optimized by the objective defined in Equation~\ref{eq:bilm_loss}, 
there is no gradient feedback for $p_{\theta}(\textbf{z}|\textbf{S})$. 
To make $p_{\theta}(\textbf{z}|\textbf{S})$ learnable, an intuitive solution is to fit $p_{\theta}(\textbf{z}|\textbf{S})$ to $q_{\phi}(\textbf{z}|\textbf{S})$ by minimizing their Kullback–Leibler distance. 
While the tree probabilities of both distributions are discrete and not exhaustive,
inspired by URNNG~\cite{dblp:conf/naacl/kimrykdm19}, a Monte Carlo estimate for the gradient with respect to $\theta$ can be defined as:
\begin{equation}
\small
\begin{aligned}
&\triangledown_{\theta} \KL[q_{\phi}(\textbf{z}|\textbf{S}) \parallel p_{\theta}(\textbf{z}|\textbf{S}) ] \\
= &\triangledown_{\theta} \mathbf{E}_{z \sim q_{\phi}(\textbf{z}|\textbf{S})}[\log \frac{q_{\phi}(\textbf{z}|\textbf{S})}{p_{\theta}(\textbf{z}|\textbf{S})}] \\
\approx &-\triangledown_{\theta} \frac{1}{K}\sum_{k=1}^{K}\log p_{\theta}(\textbf{z}^{(k)}|\textbf{S})
\end{aligned}
\vspace{-1.5pt}
\end{equation}
with samples $\textbf{z}^{(1)}$, ..., $\textbf{z}^{(K)}$ from $q_{\phi}(\textbf{z}|\textbf{S})$. 
Algorithm~\ref{alg:sample} shows the complete sampling process from $q_{\phi}(\textbf{z}|\textbf{S})$.
Specifically, we sample split points with corresponding span boundaries recursively as in previous work \cite{DBLP:journals/corr/cmp-lg-9805007,DBLP:conf/emnlp/FinkelMN06,dblp:conf/naacl/kimrykdm19} 
with respect to the intermediate tree probabilities calculated during hierarchical encoding.

\begin{algorithm}[!h]
\small
    \caption{Top-down tree sampling for R2D2}
    \label{alg:sample}
    \begin{algorithmic}[1] % The number tells where the line numbering should start
        \Function{Sample}{$\mathcal{T}_{1,n}$} \Comment{Root cell}
        \State {$Q = [\mathcal{T}_{1,n}]$}
        \State {$K = []$}
        \While {$Q$ is not empty}
        \State{$\mathcal{T} = \pop(Q)$}
        \State {$i,j = \mathcal{T}.i$, $\mathcal{T}.j$} \Comment{Start/end indices}
        \State{$L = \mathcal{T}.\mathrm{splits}$} \Comment{$m$ splits at most}
        \State{$\tau=0$}
        \For {$k \in 1$ to $\len(L)$}
        \State {$w_{k} = \widetilde{p}_{i,j}^{L[k]}$} \Comment{Using Equation~\ref{eq:tree_prob}}
        \State {$\tau = \tau + w_{k}$} \Comment {Sum up all $w_{k}$}
        \EndFor
%        \State{$\tau = \sum_{k \in \mathcal{S}}^{}w_{k}$}
        \State{$idx \sim \mathrm{Cat}([w_{1}/\tau, ..., w_{\len(L)}/\tau])$} \\
        \Comment{Sample a split point}
        \State{$\push(K, (L[idx], i, j))$} \\
        \Comment{Keep the split point and span boundary}
        \If {$L[idx] > i$} \Comment{Add left child}
            \State{$\push(Q, \mathcal{T}_{i, L[idx]})$}
        \EndIf
        \If {$L[idx] + 1 < j$} \Comment{Add right child}
            \State{$\push(Q, \mathcal{T}_{L[idx]+1, j})$}
        \EndIf
        \EndWhile
        \State{\Return{$K$}}
        \EndFunction
    \end{algorithmic}
\end{algorithm}

A sequence of split points and corresponding spans is returned by the sampler. For the $k$-{th} sample $\textbf{z}^{(k)}$, let $p_{\theta}(a_{t}^{k}|\textbf{S})$ denote the probability of taking $a_{t}^{k}$ as split from span $(i_{t}^{k}, j_{t}^{k})$ at the $t$-{th} step. Formally, we have:
\begin{equation}
\small
\begin{aligned}
%p_{\theta}(a_{t}^{k}|\textbf{S}) = \softmax ([v_{i_{t}^{k}}, ..., v_{j_{t}^{k}}]) \\
p_{\theta}(a_{t}^{k}|\textbf{S}) &= \frac{e^{v_{a^{k}_{t}}}}{e^{v_{i^{k}_{t}}} + ... + e^{v_{j^{k}_{t}}}} \\
\log p_{\theta}(\textbf{z}^{(k)}|\textbf{S}) &= \sum_{t=1}^{n-1} \log p_{\theta}(a_{t}^{k}|\textbf{S}),
\end{aligned}
\end{equation}
where $i_{t}^{k}$ and $j_{t}^{k}$ denote the start and end of the corresponding span. Please note here that the $v_i$ are not adjusted by span constraints. 

\subsection{Downstream Tasks}
\label{sec:downstream}
\paragraph{Inference.}
In this paper, we mainly focus on classification tasks as downstream tasks. We consider the root representation as representing the entire sentence.
As we have two models pre-trained in our framework -- an R2D2 encoder and a top-down parser -- we have two ways of generating the representations:
\begin{enumerate}
    \item[a)] Run forced encoding over the binary tree from the top-down parser with the R2D2 encoder.
    \item[b)] Use the binary tree to guide the pruning of the R2D2 encoder, and take the root representation $e_{1,n}$.
\end{enumerate}
It is obvious that the first approach is much faster than the latter one, as the R2D2 encoder only runs $n-1$ times in forced encoding, 
and can run in parallel layer by layer, e.g., we may run compositions at $a_5$, $a_3$, and $a_4$ in parallel in Figure~\ref{fig:sample_demo} (b).
We explore both of these approaches in our experiments.

\paragraph{Training Objectives.}
As suggested in prior work \cite{radford2018improving,howard-ruder-2018-universal,gururangan-etal-2020-dont}, 
given a pretrained model, continued pretraining on an in-domain corpus with the same pretraining objective can yield a better generalization ability.
Thus, we simply combine our pretraining objectives via summation in all downstream tasks. At the same time, as the downstream task may guide R2D2 to more reasonable tree structures, we still maintain the KL loss to enable the parser to continuously update.
For the two inference methods,
we uniformly select the root representation $e_{1,n}$ as the representation for a given sentence followed by an MLP, and estimate the cross-entropy loss, denoted as $\mathcal{L}_\mathrm{forced}$ and $\mathcal{L}_\mathrm{cky}$, respectively. Let $\mathcal{L}_\mathrm{KL}$ denote the KL loss described in Section~\ref{sec:opt} and $\mathcal{L}_\mathrm{bilm}$ denote the bidirectional language model loss described in Eq~\ref{eq:bilm_loss}.
The final loss is:
\begin{equation}
\mathcal{L} = \mathcal{L}_\mathrm{forced} + \mathcal{L}_\mathrm{cky} + \mathcal{L}_\mathrm{bilm} + \mathcal{L}_\mathrm{KL}
\end{equation}

\subsection{First Measurements}

As concluded in our former preliminary study \cite{brainbs5}, one of the further developments of Series 5 is changing to full screen from fixed-size window. This modification affects the basic operation of the benchmark, 
so the first objective was to verify that whether the feeling of losing the character still appears correctly or not. On Windows systems there were no problems. Some experiments using default settings on Windows 10 can be seen in Fig \ref{brainbs6nb}, Fig \ref{brainbs6nab} and Fig \ref{brainbs6nabs}. But on GNU/Linux systems test subjects reported that the feeling of losing the character is not experienced. These observations will be detailed in a next section.



\begin{figure*}
    \centering
    \begin{subfigure}{.45\linewidth}
        \centering
%        \includegraphics[scale=0.5]{nb2}
        \includegraphics[scale=0.37]{nb2}
        \caption{With using the touchpad.}
    \end{subfigure}
        \hskip2em
    \begin{subfigure}{.45\linewidth}
        \centering
%        \includegraphics[scale=0.5]{nb3}
        \includegraphics[scale=0.37]{nb3}
        \caption{With using a standalone mouse.}
    \end{subfigure}
    \\
    \centering
    \begin{subfigure}{.45\linewidth}
            \centering
%        \includegraphics[scale=0.25]{nb2Test-6000-screenimage}
        \includegraphics[scale=0.18]{nb2Test-6000-screenimage}
        \caption{The final result was 5.45563 Kilobytes.}
    \end{subfigure}
    \hskip2em
    \begin{subfigure}{.45\linewidth}
            \centering
%        \includegraphics[scale=0.25]{nb3Test-6000-screenimage.png}
        \includegraphics[scale=0.18]{nb3Test-6000-screenimage.png}
        \caption{The final result was 6.37927 Kilobytes.}
         \label{brainbs6nblog}
    \end{subfigure}
    \caption{These tests were also performed by the first author on the same environment as in Fig  \ref{brainbs6winnb1}. All the logged data and final screenshots can be found at \url{http://smartcity.inf.unideb.hu/~norbi/BrainBSeries6/measurements/NB/}.}
 \label{brainbs6nb}
\end{figure*}




\begin{figure*}
    \centering
    \begin{subfigure}{.45\linewidth}
        \centering
%        \includegraphics[scale=0.5]{nab1}
\includegraphics[scale=0.37]{nab1}
        \caption{6.51813 Kilobytes (performed with the touchpad).}
        \label{brainbs6nab1}
    \end{subfigure}
        \hskip2em
    \begin{subfigure}{.45\linewidth}
        \centering
%        \includegraphics[scale=0.5]{nab2}
        \includegraphics[scale=0.37]{nab2}
        \caption{4.31812 Kilobytes (performed with a mouse).}
        \label{brainbs6nab2}
    \end{subfigure}
    \centering
    \begin{subfigure}{.45\linewidth}
        \centering
%        \includegraphics[scale=0.5]{nab3}
        \includegraphics[scale=0.37]{nab3}
        \caption{6.79218 Kilobytes (performed with the touchpad).}
        \label{brainbs6nab3}
    \end{subfigure}
    \caption{This test was performed with a male child (10 years old, on the same environment as in Fig  \ref{brainbs6winnb1}). The data and final screenshots can be found at \url{http://smartcity.inf.unideb.hu/~norbi/BrainBSeries6/measurements/NaB/}.  It should be noted that  test subjects with touchpad can use both hands, one for holding the button and the other for motion.}     
    \label{brainbs6nab}
\end{figure*}


\begin{figure*}
    \centering
    \begin{subfigure}{.45\linewidth}
        \centering
%        \includegraphics[scale=0.25]{nab1Test-6000-screenimage}
        \includegraphics[scale=0.18]{nab1Test-6000-screenimage}
        \caption{This final screenshot corresponds to Fig \ref{brainbs6nab1}.}
    \end{subfigure}
    \hskip2em
    \centering
    \begin{subfigure}{.45\linewidth}
        \centering
%        \includegraphics[scale=0.25]{nab2Test-6000-screenimage}
        \includegraphics[scale=0.18]{nab2Test-6000-screenimage}
        \caption{This final screenshot corresponds to Fig \ref{brainbs6nab2}.}
    \end{subfigure}
        \hskip2em
    \begin{subfigure}{.45\linewidth}
        \centering
%        \includegraphics[scale=0.25]{nab3Test-6000-screenimage}
        \includegraphics[scale=0.18]{nab3Test-6000-screenimage}
        \caption{This final screenshot corresponds to Fig \ref{brainbs6nab3}.}
    \end{subfigure}
    \caption{This test was performed with a male child (10 years old, on the same environment as in Fig  \ref{brainbs6winnb1}). The data and final screenshots can be found at \url{http://smartcity.inf.unideb.hu/~norbi/BrainBSeries6/measurements/NaB/}.}
     \label{brainbs6nabs}
\end{figure*}



\subsection{Logging Data}

The state of the BrainB benchmark can be saved at any time by pressing the S button but measured data is automatically saved after the test is finished. The program saves a screenshot of the display to a PNG file. For example, such a screenshot was shown in Fig \ref{brainbs6screen}. The data is saved to a text file that contains the information shown in Listing \ref{savedata}, where the most important lines are the following. 
The two time values tell the time when data is saved. The first one (in Line \ref{time1}) 
is expressed in 100 millisecond units and the second one (in Line \ref{time2}) is expressed in the form minutes:seconds. 
The noc value tells the number of characters (boxes).
The nop value tells the number of pause events initiated by the test subject.
The relation symbol in Line \ref{relation} indicates the fulfillment of our research hypothesis that 
the mean of the complexity of changing from lost to found is less than the mean of the changing found to lost. 

\begin{lstlisting}[language={C},numbers={left},stepnumber={1},numbersep={-1pt},basicstyle={\scriptsize\ttfamily},caption={The structure of the measured data. This log file is belong to the measurement shown in Fig \ref{brainbs6nblog}.},label={savedata},escapechar={|}]
  NEMESPOR BrainB Test 6.0.3
  time      : 6000 |\label{time1}|
  bps       : 28170
  noc       : 71
  nop       : 0
  lost      : 
  30530 31840 39910 10960 60270 71280 50340 51580 31670 
  49260 53710 41620 86830 72580 56310 70560 68870 45500 
  52480 52660 45640 46870 44660 75860 68150 70110 69610 
  47130 61980 75310 90440 75700 62670 54870 69820 75170
  84350 76990 80480 70840 54920 40720 33800 31590 28860 
  24650 27250 53490 58180 56200 57490 53930 39030 83870 
  87180 78270 70990 43600 52360 43910 33820 31120 34830
  32370 32840 37080 32390 
  mean      : 54181
  var       : 18541.5
  found     : 
  12880 22240 26690 11190 19880 36170 14930 28100 25860 
  27580 36040 34590 22250 12060 11760 8880 10660 30840 
  48000 33030 43040 26330 45880 50380 34970 45950 36610 
  46660 47980 45330 65290 57080 55340 54700 43930 34850 
  55030 43240 69500 50770 58680 54750 65470 59610 79030 
  67190 63890 61550 65590 54100 69460 69210 37390 41850 
  53130 31650 45400 46430 50490 44310 35960 53510 25760 
  38950 33250 39360 46650 63050 64890 68590 76430 50570 
  57630 57250 28830 42020 45500 67160 63310 69930 80200 
  76980 56300 44320 58340 79850 81590 69740 88200 89160 
  62640 55030 60510 39810 51660 51730 47720 62330 66150 
  47100 60470 70810 88930 75110 65290 68830 59430 63710 
  22570 36940 29450 43630 53100 55560 64750 39530 59610 
  58250 71950 62800 75250 76720 81910 31730 47010 44890 
  58490 61750 66900 69380 81650 79450 72420 
  mean      : 51442
  var       : 18616.1
  lost2found: 14930 22250 11760 43040 26330 34970 46660 
  43930 50770 61550 54100 37390 31650 44310 25760 50570 
  28830 56300 69740 62640 39810 62330 65290 59430 22570
  39530 31730 72420 
  mean      : 43235
  var       : 16826.7
  found2lost: 31840 10960 60270 51580 31670 49260 53710 
  86830 70560 68870 45500 52660 45640 46870 75860 69610 
  61980 75310 90440 54870 69820 75170 84350 80480 53490 
  56200 83870 78270 
  mean      : 61283
  var       : 18824.2
  mean(lost2found) < mean(found2lost) |\label{relation}|
  time      : 10:0 |\label{time2}|
  U R about 6.37927 Kilobytes
\end{lstlisting}


\subsection{Choosing Colors}

\definecolor{tea_green}{RGB}{214, 234, 193}
\definecolor{hint_green}{RGB}{226,246,209}
\definecolor{Madang}{RGB}{190,235,159}
\definecolor{yellow_green}{RGB}{198,222,119}
\definecolor{link_water}{RGB}{221, 232, 250}
\definecolor{celestial_blue}{RGB}{52, 152, 219}
\definecolor{shakespeare}{RGB}{85, 154, 193}
\definecolor{buttermilk}{RGB}{255,242,174}
\definecolor{chardonnay}{RGB}{250,196,114}
\definecolor{rajah}{RGB}{253,180,98}
\definecolor{fog}{RGB}{213, 193, 234}
\definecolor{melon}{RGB}{254,191,181}
\definecolor{sundown}{RGB}{249, 180, 181}
\definecolor{mona_lisa}{RGB}{246,152,134}
\definecolor{salmon}{RGB}{242,131,107}


\definecolor{saltpan}{RGB}{238, 243, 232}
\definecolor{aqua_spring}{RGB}{232, 243, 232}
\definecolor{tea_green}{RGB}{214, 234, 193}
\definecolor{Madang}{RGB}{190,235,159}
\definecolor{fringy_flower}{RGB}{194, 234, 193}
\definecolor{aero_blue}{RGB}{193, 234, 213}
\definecolor{pixie_green}{RGB}{183,214,170}
\definecolor{french_pass}{RGB}{195,232,246}
\definecolor{ice_cold}{RGB}{169,232,220}
\definecolor{pale_turquoise}{RGB}{172,240,242}
\definecolor{cruise}{RGB}{179,226,205}
\definecolor{sail}{RGB}{163,205,235}
\definecolor{spindle}{RGB}{179,205,227}
\definecolor{link_water}{RGB}{221, 232, 250}
\definecolor{periwinkle}{RGB}{203,213,232}
\definecolor{zanah}{RGB}{220, 233, 213}
\definecolor{frostee}{RGB}{217, 231, 214}
\definecolor{opal}{RGB}{199, 221, 211}
\definecolor{jet_stream}{RGB}{188, 214, 210}
\definecolor{skeptic}{RGB}{153, 187, 167}
\definecolor{hint_green}{RGB}{226,246,209}
\definecolor{snow_flurry}{RGB}{230,245,201}
\definecolor{surf_crest}{RGB}{205,230,208}
\definecolor{yellow_green}{RGB}{198,222,119}
\definecolor{cream}{RGB}{255,255,204}
\definecolor{pale_prim}{RGB}{255,255,179}
\definecolor{spring_sun}{RGB}{242,243,195}
\definecolor{portafino}{RGB}{245,237,160}
\definecolor{buttermilk}{RGB}{255,242,174}
\definecolor{cream_brulee}{RGB}{255, 229, 151}
\definecolor{dairy_cream}{RGB}{254,226,189}
\definecolor{champagne}{RGB}{254,217,166}
\definecolor{chardonnay}{RGB}{250,196,114}
\definecolor{manhattan}{RGB}{226,180,125}
\definecolor{rajah}{RGB}{253,180,98}
\definecolor{early_dawn}{RGB}{252,243,218}
\definecolor{egg_shell}{RGB}{238, 234, 215}
\definecolor{selago}{RGB}{243, 232, 243}
\definecolor{quartz}{RGB}{219,223,238}
\definecolor{fog}{RGB}{213, 193, 234}
\definecolor{languid_lavender}{RGB}{222,203,228}
\definecolor{watusi}{RGB}{254,221,207}
\definecolor{coral_andy}{RGB}{243,204,205}
\definecolor{cosmos}{RGB}{248,209,210}
\definecolor{melon}{RGB}{254,191,181}
\definecolor{azalea}{RGB}{234, 193, 194}
\definecolor{beauty_bush}{RGB}{235, 185, 179}
\definecolor{sundown}{RGB}{249, 180, 181}
\definecolor{mona_lisa}{RGB}{246,152,134}
\definecolor{salmon}{RGB}{242,131,107}


\definecolor{summer_sky}{RGB}{58, 151, 233}
\definecolor{chateau_green}{RGB}{72, 179, 96}
\definecolor{matisse}{RGB}{25, 104, 167}
\definecolor{allports}{RGB}{31, 106, 125}
\definecolor{sun_shade}{RGB}{255, 144, 68}
\definecolor{flamingo}{RGB}{237, 88, 85}
\definecolor{studio}{RGB}{128, 91, 160}



\definecolor{maya_blue}{RGB}{102, 204, 255}
\definecolor{feijoa}{RGB}{178,223,138}
\definecolor{sushi}{RGB}{117, 168, 47}
\definecolor{norway}{RGB}{158, 194, 132}
\definecolor{japanese_laurel}{RGB}{53, 116, 40}
\definecolor{see_green}{RGB}{161,228,195}
\definecolor{monte_carlo}{RGB}{135,204,194}
\definecolor{granny_smith_apple}{RGB}{150,214,150}
\definecolor{moss_green}{RGB}{170,216,176}
\definecolor{chateau_green}{RGB}{72, 179, 96}
\definecolor{opal}{RGB}{164,207,190}
\definecolor{acapulco}{RGB}{117, 170, 148}
\definecolor{viridian}{RGB}{55, 137, 122}
\definecolor{amazon}{RGB}{56, 123, 84}
\definecolor{asparagus}{RGB}{123, 160, 91}
\definecolor{fruit_salad}{RGB}{91, 160, 94}
\definecolor{puerto_rico}{RGB}{72, 179, 150}
\definecolor{mountain_meadow}{RGB}{0, 163, 136}
\definecolor{matisse}{RGB}{25, 104, 167}
\definecolor{allports}{RGB}{31, 106, 125}
\definecolor{astral}{RGB}{55, 111, 137}
\definecolor{spring_leaves}{RGB}{46, 83, 117}
\definecolor{biscay}{RGB}{44, 62, 80}
\definecolor{midnight}{RGB}{0, 29, 50}
\definecolor{amethyst}{RGB}{153, 102, 204}
\definecolor{studio}{RGB}{128, 91, 160}
\definecolor{tapestry}{RGB}{194, 109, 132}
\definecolor{atomic_tangerine}{RGB}{255, 153, 102}
\definecolor{amber}{RGB}{255, 191, 0}
\definecolor{casablanca}{RGB}{244, 178, 84}
\definecolor{california}{RGB}{233, 140, 58}
\definecolor{tomato}{RGB}{255, 97, 56} 
\definecolor{alizarin}{RGB}{233, 58, 64}



\definecolor{linen}{RGB}{251, 239, 227}
\definecolor{double_pearl_lusta}{RGB}{253, 242, 208}
\definecolor{oasis}{RGB}{253, 242, 208}
\definecolor{milan}{RGB}{255, 254, 169}
\definecolor{texas}{RGB}{245, 232, 123}
\definecolor{maize}{RGB}{249, 212, 156}

\definecolor{turmeric}{RGB}{211, 178, 76}
\definecolor{saffron}{RGB}{249,193,62}
\definecolor{my_sin}{RGB}{255, 176, 59}
\definecolor{tree_poppy}{RGB}{246, 154, 27}
\definecolor{jaffa}{RGB}{240, 131, 58}
\definecolor{crusta}{RGB}{254, 127, 44}
\definecolor{tahiti_gold}{RGB}{223, 102, 36}
\definecolor{outrageous_orange}{RGB}{255, 100, 45}
\definecolor{safety_orange}{RGB}{254, 106, 0}


\definecolor{azalea}{RGB}{251, 196, 196}
\definecolor{oyster_pink}{RGB}{238,206,205} 
\definecolor{coral_candy}{RGB}{242,208,205} 
\definecolor{baby_pink}{RGB}{246, 194, 192}
\definecolor{petite_orchid}{RGB}{223, 157, 155}
\definecolor{apricot}{RGB}{241,140,122}
\definecolor{NY_pink}{RGB}{228,136,113}
\definecolor{carmine_pink}{RGB}{231, 76, 60}
\definecolor{deep_carmine_pink}{RGB}{236, 50, 67}

\definecolor{wewak}{RGB}{244, 143, 150}
\definecolor{light_coral}{RGB}{244, 127, 123}
\definecolor{bittersweet}{RGB}{255,111,105}
\definecolor{carnation}{RGB}{245, 80, 86}
\definecolor{flamingo}{RGB}{237, 88, 85}
\definecolor{sunset_orange}{RGB}{242,89,75}
\definecolor{ku_crimson}{RGB}{243, 0, 25}
\definecolor{amaranth}{RGB}{234,46,73}
\definecolor{valencia}{RGB}{214, 87, 70}
\definecolor{chilean_firegongs }{RGB}{215, 87, 44}
\definecolor{mexican_red}{RGB}{170, 41, 37}



\definecolor{napa}{RGB}{163, 154, 137}

\definecolor{athens_gray}{RGB}{236, 240, 241}
\definecolor{gallery}{RGB}{240,240,240}
\definecolor{mercury}{RGB}{230,230,230}
\definecolor{platinum}{RGB}{228,228,228}
\definecolor{silver}{RGB}{191,191,191}
\definecolor{aluminum}{RGB}{153,153,153}
\definecolor{ship_gray}{RGB}{77,77,77}
\definecolor{tuatara}{RGB}{67, 67, 67}

\definecolor{malibu}{RGB}{110, 180, 240}
\definecolor{celestial_blue}{RGB}{52, 152, 219}
\definecolor{curious_blue}{RGB}{41, 128, 185}
\definecolor{french_blue}{RGB}{0, 112, 182}
\definecolor{matisse}{RGB}{25, 104, 167}
\definecolor{shakespeare}{RGB}{85, 154, 193}
\definecolor{seagull}{RGB}{128,177,211}
\definecolor{jelly_bean}{RGB}{45, 126, 150}
\definecolor{venice_blue}{RGB}{87, 135, 105}
\definecolor{boston_blue}{RGB}{68, 147, 161}

\definecolor{turquoise}{RGB}{41,217,194}
\definecolor{java}{RGB}{2,190,196}
\definecolor{riptide}{RGB}{141,211,199}
\definecolor{mountain_meadow}{RGB}{0, 163, 136}
\definecolor{free_speech_aquamarine}{RGB}{0, 156, 114}

\definecolor{cosmic_latte}{RGB}{222, 247, 229}
\definecolor{chinook}{RGB}{163, 232, 178}
\definecolor{padua}{RGB}{121, 189, 143}
\definecolor{ocean_green}{RGB}{79, 176, 112}
\definecolor{pastel_green}{RGB}{107, 227, 135}
\definecolor{chateau_green}{RGB}{69, 191, 85}
\definecolor{RoyalBlue}{RGB}{69, 191, 85}
\definecolor{pigment_green}{RGB}{0, 175, 79}
\definecolor{fern}{RGB}{101,197,117}
\definecolor{killarney}{RGB}{56, 113, 66}


\subsection{Known Problems with Series 6}

Despite that our benchmark is developed on Linux it is surprising that test subjects who performed it on Linux did not experience the feeling of losing the character. This problem causes the deteriorated results shown in Fig \ref{brainbs6l}. It is important to note that it has not been detected in earlier series of the application. Moreover, before Series 6, there was no Windows binary edition of BrainB program. In Series 6, changing to full screen from windowed causes the problem because Series 6  is sensitive to the different mouse sensitivity settings on Windows and Linux systems (the measurements shown in Fig  \ref{brainbs6l} were performed with a Logitech mouse with \textit{acceleration:  5/1} and \textit{threshold:  5} xset m\footnote{\url{https://www.x.org/archive/X11R7.7/doc/man/man1/xset.1.xhtml}} setting). 
A short-term solution may be to standardize the test environment used by each member 
of a given subset of test subjects. We apply this method to perform systematic measurements with Series 6 in the next section. The long-term solution will be to fine-tune the control of movements of boxes that is hardwired into the Series 6 from Series 5 at this moment.
Another possibility is to take the liberty of fine-tuning of the mouse for test subjects who thus would be able to choose their custom mouse settings in order to increase their effectiveness. This is also in well accordance with the competitive way of performing our test.  Fig \ref{brainbs6nbft} presents two measurements using custom mouse settings.

\begin{figure*}
    \centering
    \begin{subfigure}{.45\linewidth}
        \centering
%        \includegraphics[scale=0.5]{nbl}
        \includegraphics[scale=0.37]{nbl}
        \caption{The test subject was the same as in the experiment shown in Fig \ref{brainbs6nb}. The final result was 3.76904 Kilobytes.}
 \label{brainbs6l1}
    \end{subfigure}
        \hskip2em
    \begin{subfigure}{.45\linewidth}
        \centering
%        \includegraphics[scale=0.5]{nabl}
        \includegraphics[scale=0.37]{nabl}
        \caption{The test subject was the same as in the experiment shown in Fig \ref{brainbs6nab}. The final result was 3.75116 Kilobytes.}
         \label{brainbs6l2}
    \end{subfigure}
    \\
    \centering
    \begin{subfigure}{.45\linewidth}
            \centering
%        \includegraphics[scale=0.25]{nblTest-6000-screenimage}
        \includegraphics[scale=0.18]{nblTest-6000-screenimage}
        \caption{This final screenshot corresponds to Fig \ref{brainbs6l1}.}
         \label{brainbs63}
    \end{subfigure}
    \hskip2em
    \begin{subfigure}{.45\linewidth}
            \centering
%        \includegraphics[scale=0.25]{nablTest-6000-screenimage.png}
        \includegraphics[scale=0.18]{nablTest-6000-screenimage.png}
        \caption{This final screenshot corresponds to Fig \ref{brainbs6l2}.}
         \label{brainbs6l4}
    \end{subfigure}
    \caption{These tests were performed on a GNU/Linux desktop (Ubuntu 16.04, 
    SyncMaster S24B300 monitor with resolution
1920x1080). Test subjects reported that the feeling of losing the character is not experienced.}
    \label{brainbs6l}
\end{figure*}


\begin{figure*}
    \centering
    \begin{subfigure}{.45\linewidth}
        \centering
%        \includegraphics[scale=0.5]{nbl2}
        \includegraphics[scale=0.37]{nbl2}
        \caption{The final result was 5.95587 Kilobytes. xinput settings were the following \enquote{Device Accel Constant Deceleration (277): 1.000000}, \enquote{Device Accel Velocity Scaling (279): 1.000000} and \enquote{Device Accel Profile (276):	-1}}
 \label{brainbs6nbft1}
    \end{subfigure}
        \hskip2em
    \begin{subfigure}{.45\linewidth}
        \centering
%        \includegraphics[scale=0.5]{nbl3}
        \includegraphics[scale=0.37]{nbl3}
        \caption{The final result was 5.71674 Kilobytes. xinput settings were the following \enquote{Device Accel Constant Deceleration (277): 2.000000}, \enquote{Device Accel Velocity Scaling (279): 15.000000} and \enquote{Device Accel Profile (276):	-1}}
         \label{brainbs6nbft2}
    \end{subfigure}
    \\
    \centering
    \begin{subfigure}{.45\linewidth}
            \centering
%        \includegraphics[scale=0.25]{nbl2Test-6000-screenimage}
        \includegraphics[scale=0.18]{nbl2Test-6000-screenimage}
        \caption{This final screenshot corresponds to Fig \ref{brainbs6nbft1}.}        
         \label{brainbs6nbft3}
    \end{subfigure}
    \hskip2em
    \begin{subfigure}{.45\linewidth}
            \centering
%        \includegraphics[scale=0.25]{nbl3Test-6000-screenimage.png}
        \includegraphics[scale=0.18]{nbl3Test-6000-screenimage.png}
        \caption{This final screenshot corresponds to Fig \ref{brainbs6nbft2}.}
         \label{brainbs6nbft4}
    \end{subfigure}
    \caption{The test subject was the same as in the experiment shown in Fig \ref{brainbs6nb}. 
    The subject reported that the feeling of losing the character has already been experienced.}    
    \label{brainbs6nbft}
\end{figure*}

\subsection{Systematic Measurements with Series 6}

The BrainB Series 6 was measured in two groups: UDPROG and DEAC-Hackers.
The first one is a Facebook community of over 550 actual or former students of the BSc course of “High Level Programming Languages” at the University of Debrecen. 
The second one is an esport department of the University of Debrecen's Athletic Club. Participation in the BrainB Series 6 survey was voluntary in both groups. 

In the UDPROG community 33 members send back their results including the PNG screenshot and the produced text file 
within 2 days from the date of announcement (20 August 2018). 
The arithmetic mean of the final results of UDPROG participants is 4.95345. 
The mean of the number of boxes at the moment when the benchmark ends is 57.1818.
The averaged losing and finding curve for all members is shown in Fig \ref{udprogmeanFL}.
At the end of the curve the arithmetic mean values of complexity of the losing and finding events are irrelevant because the size of the sequences of losing and finding events 
are different for every participants. Fig \ref{udprognum} indicates these different sizes.

In the DEAC-Hackers community 12 esport athletes have sent back their results that can be seen in Fig \ref{deac}. 
It is important to notice that despite low sample sizes of test subjects the averaged losing and finding 
curves shown in Fig \ref{udprogmeanFL} and \ref{deacmeanFL} have already separated the losing and finding events.


\begin{figure*}
    \centering
    \begin{subfigure}{.45\linewidth}
  \centering
%    \includegraphics[scale=0.5]{udprogmeanFL}
    \includegraphics[scale=0.37]{udprogmeanFL}
  \caption{This figure shows the averaged losing and finding curve for all UDPROG participants where the losing (L) and finding (F) events are also  indicated.}
   \label{udprogmeanFL}
    \end{subfigure}
        \hskip2em
    \begin{subfigure}{.45\linewidth}
        \centering
%    \includegraphics[scale=0.5]{udprognum}
    \includegraphics[scale=0.37]{udprognum}
  \caption{The sizes of samples of losing and finding events. The x-axis shows the sizes and the y-axis shows the number of test-subjects.}
   \label{udprognum}
    \end{subfigure}
    \caption{Measurements in the community UDPROG. The arithmetic mean of the final results of UDPROG participants is 4.95345. 
    The mean of the number of boxes at the moment when the benchmark ends is 57.1818. The anonymized data can be found at \url{http://smartcity.inf.unideb.hu/~norbi/BrainBSeries6/measurements/UDPROG/}.}    
    \label{udprog}
\end{figure*}


\begin{figure*}
    \centering
    \begin{subfigure}{.45\linewidth}
  \centering
%    \includegraphics[scale=0.5]{deacmeanFL}
    \includegraphics[scale=0.37]{deacmeanFL}
  \caption{This figure shows the averaged losing and finding curve for all DEAC-Hackers participants where the losing (L) and finding (F) events are also  indicated.}
   \label{deacmeanFL}
    \end{subfigure}
        \hskip2em
    \begin{subfigure}{.45\linewidth}
        \centering
%    \includegraphics[scale=0.5]{deacnum}
    \includegraphics[scale=0.37]{deacnum}
  \caption{The sizes of samples of losing and finding events. The x-axis shows the sizes and the y-axis shows the number of test-subjects.}
   \label{deacnum}
    \end{subfigure}
    \caption{Measurements in the community DEAC-Hackers. The arithmetic mean of the final results of DEAC-Hackers  participants is 3.71036. It is surprisingly lower than expected if compared to the value 4.95345 of the examined programming community. 
    The mean of the number of boxes at the moment when the benchmark ends is 49. The anonymized data can be found at \url{http://smartcity.inf.unideb.hu/~norbi/BrainBSeries6/measurements/DEACH/}.}    
    \label{deac}
\end{figure*}


\section{Conclusion}

Our research hypothesis was that 
the mean of the complexity of changing lost to found is less than the mean of the changing found to lost. 
Fig \ref{udprogmeanFL} and \ref{deacmeanFL} show the fulfillment of this hypothesis. 
It seems very well in these figures that the averaged losing and finding curve has precisely separated 
the losing and finding events. Intuitively, this result shows that we lose the character on a higher 
complexity level then we find it on a relatively lower level again. 
This simple hypothesis has been proved by the results of this study. 

In order to further strengthen the completion of our benchmark test in a competitive way in the  following versions we are going to offer to test subjects a little more 
liberty of fine-tuning the settings. The fine-tuning of mouse settings was already mentioned earlier. A further possibility is to allow using custom colors.

The next research objective will be to verify the satisfaction of Hick's law. 
To achieve this goal it is simple enough to compare the complexity of finding and losing events with the time differences 
of these. Unfortunately, the actual version of the BrainB benchmark do not record these timestamps. 
The BrainB Series 7 will contain this feature. Our long-term research goal is to further develop our benchmark to a standard psychological test that can be used for talent search in esport.

\section{Acknowledgement}

\\
\stitle{Acknowledgements:} We thank Yifan Wu for the initial inspiration, Anant Bhardwaj for data collection, Laura Rettig on early formulations of the problem, and the support of NSF 1527765 and 1564049.

\bibliographystyle{alpha}
%\bibliographystyle{IEEEtran}
\bibliography{brainbs6}

\end{document}






