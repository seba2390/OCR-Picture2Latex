Losing the control of the character in a given gameplay situation is a very common feeling that is well known among gamers. In this situation, players cannot control their character, simply because they temporarily cannot see it due to the visual complexity of the display is unexpectedly increased and/or the game is lagging and, finally, it is also possible that the players have no sufficient experience to control the character. 
In this paper, we introduce our benchmark computer program called \textit{BrainB Test Series 6} that can abstract this feeling. In this test, game objects are symbolized by boxes as it can be seen in Fig \ref{brainbs6screen}. All boxes move according to random walks. There is a distinguished box labelled by the name \textit{Samu Entropy}. It represents the character controlled by the player. 
The benchmark test lasts for 10 minutes. 
During the test, the user must continuously hold and drag the mouse button on the center of Samu Entropy. 
If the user succeeds in this task then the benchmark program will increase the visual complexity of the display. It will draw more and more overlapping boxes which will move faster and faster. Otherwise, if the mouse pointer cannot follow the center of Samu Entropy then the visual complexity will be decreased. The test will delete more and more boxes and the remaining boxes move slower 
and slower until the user finds Samu Entropy again, i.e., clicks on Samu Entropy.

The BrainB Series 1 to 4 were developed in the family setting of the first author\footnote{For example, see \url{https://www.twitch.tv/videos/139186614}}. 
Then, in our university environment, 
we had already done a preliminary study \cite{brainbs5} on a previous (BrainB Series 5) version of our benchmark. Some of its measurements were streamed live on Twitch\footnote{For example, see \url{https://www.twitch.tv/videos/206478952}}. 
The main research goal of this study is to show that players lose the character on a higher complexity level of the display and they find it on a relatively lower complexity level.

\begin{figure*}
  \centering
    %\includegraphics[width=0.9\textwidth]{Test-6000-screenimage-norbiwin}
    \includegraphics[width=\textwidth]{Test-6000-screenimage-norbiwin}
  \caption{A screenshot of BrainB Test Series 6 in action. Can you find the box labelled by the name Samu Entropy in this picture?}
  \label{brainbs6screen}
\end{figure*}
