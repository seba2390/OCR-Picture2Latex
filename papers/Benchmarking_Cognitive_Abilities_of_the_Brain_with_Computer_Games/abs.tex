Most of the players have experienced the feeling of temporarily losing their character in a given gameplay situation when they cannot control the character, simply because they temporarily cannot see it. The main reasons for this feeling may be due to the interplay of the following factors: (1) the visual complexity of the game is unexpectedly increased compared with the previous time period as more and more game objects and effects 
are rendered on the display; (2) and/or the game is lagging; (3) and finally, it is also possible that the players have no sufficient experience with controlling the character. This paper focuses on the first reason. We have developed a benchmark program 
which allows its user to experience the feeling of losing character. While the user can control the character well the benchmark program will increase the visual complexity of the display. Otherwise, if the user lost the character then the program will decrease the complexity until the user will find the character again, and so on. The complexity is measured based on the number of changed pixels between two consecutive display images. Our measurements show that the average of bit per second values of losing and finding pairs describes the user well. 
The final goal of this research is to further develop our benchmark to a standard psychological test.