Since computer games have a relatively short history and their effects on cognitive skills have just been started to be researched recently, there are plenty of questions to be answered. In \cite{HOMER201850}, authors reported an increase in executive functions in school students after playing computer games. Moisala et al. in \cite{MOISALA2017204} shows that enhancements in speed and performance accuracy 
of working memory tasks is related to daily gaming activity. In \cite{bediou2018meta}, authors present an analysis of the impact of action video games on cognitive skills.

Using computer games to measure cognitive abilities has a short history, but a promising future. Most research try to measure the presence or severity of a certain cognitive disease such as dementia or Alzheimer's disease.
In \cite{anguera2013video}, authors show how a long-term use of video games can reduce multitasking costs in older adults. Geyer et al. in \cite{GEYER2015260} show that the change of the score of an online game is in connection with the age-related changes in working memory.

Seldom can we find applications that has been developed for the measurement of cognitive abilities. One such application is reported in \cite{10.1007/978-3-319-27695-3_13} and \cite{pataki2015computer}, it is a framework that has been developed to measure cognitive abilities and its change of elders with computer games. This framework is able to log and analyze scores achieved in various online computer games.

From the viewpoint of information theory and HCI (Human-Computer Interaction), the Hick's law \cite{seow} could be an interesting aspect. This law states that the response time of the brain increases with logarithm of the size of the input. For our purposes, it can be an interesting question: 
how can we apply the Hick's law (or other information theory figure) in our benchmark software?