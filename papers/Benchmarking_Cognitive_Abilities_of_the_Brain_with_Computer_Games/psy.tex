The cognitive ability of attention is a significant factor in everyday life, either it comes from work, hobby or the daily activities, 
as it affects the performance of all the previously mentioned things. 
The alertness, or in other words, the long upheld attention, in technical terms is called vigilance. 
The research of vigilance is an important topic in Psychology from 1970 to the present day. 
The first method used was the Mackworth Clock \cite{Mackworth}, in which the participants had to pay attention to a clock that had a second hand which sometimes sprang twice, and then the participants had to signal as soon as possible. 
For measuring attention and concentration, there is another method, 
the Toulouse-Piéron test \cite{ToulousePieron}, 
in which participants have to follow a given scheme to separate right and wrong signs. To measure vigilance we must take into consideration the hit ratio and the number of false alarms. In almost all of our activities there are also interfering stimuli that affects our performance as well. These other factors vary by quantity and quality, and some can be stimulating, 
while some detain us 
from the optimal performance. The Yerkes-Dodson law \cite{YerkesDodsonLaw} says that for achieving the best performance 
there is an optimal arousal level, which level is higher in simpler tasks, and lower in complex activities. 
It can be represented by a inverted U-shaped curve. 
We must not forget that as in some other things, in the attentional system there are also personal differences that 
should be taken into consideration while researching the subject \cite{AltPszichobook}. 

Other objects in the environment can affect how we perceive the one object that is interesting for us. 
In 1940, 
Witkin et al.\ did a research on perception \cite{WitkinEtAl}, and from this work, they created a theory about two different cognitive styles, which they called field dependent, and field independent. A field dependent person perception is mainly affected by the field, the environment of the observed object. On the contrary, a field independent person does not affected by the field created by the observed object’s environment. This phenomenon was investigated by a task, in which the participants had to determine whether a straight rod, 
in different planes is vertical or not.
Moreover, there is another typical method used in this topic, that is the Tilting room, Tilting chair test, 
in which the participant is sitting in a tiltable chair that he or she needs to controll in order to get him/herself into vertical position despite the tilting room. 
Later, Witkin and Goodenough reinvestigated the topic, 
and they came to a conclusion that the two styles 
are two ends of the spectrum, 
however, some people are fixed with one of the cognitive styles, 
while others can adapt to the style they use depending on the situation \cite{WitkinGoodenough}. 

Speaking of attention, it’s important to talk about the main processing system, i.e., the brain. The operation of the brain is frequently compared to the mechanism of a personal computer 
by many researchers. 
Carl Sagan based his theory on the binary coding, 
so he used the information content in binary. 
When we are watching something, the picture seen that our brain maps, is made of plenty of information. 
Sagan wanted to calculate the 
information processing 
speed of the brain, to do so, he based his calculation on the example of looking at the moon, and from this example he drew the consequence, that the brain can process about 5000 bit/sec at its peak performance \cite{Sagan}. In a modern project, called Building 8, the main thought is to make the brain into a computer. Based on this project, the
information processing speed of the brain 
is about a terrabyte/sec, which far exceeds the speed estimated by Sagan \cite{Nieva}.  

\subsubsection{Practicing filling out tests}

Filling out tests and experiments are common tools in the science of psychology. Countless methods were 
created to date, but these methods are not just used, because researchers improve them, as well as, try to test them in a wider range. However, we need to consider certain factors in each experiment 
and test
that how they affect the method’s usability and the final results as well. Among these factors, there is one, when the participant obtains knowledge about what is expected from her/him, 
or which answer are considered the ’best’. This way the participant will accomodate to this information, because he/she, as everyone else, wants to portray herself/himself in the best manner possible and to be the ’best’ in performance. In multiple choice questions, there are some tricks, that are well known in the common knowledge, which we all use, when we don’t know the right answer for sure. A somewhat similar tool is the experience or 
routine with filling out tests, which can help to choose the adequate strategy for solving the situation, this is called test-wisdom. To achieve that, one must discover the logic behind the method, or practise it many times. But the test-wisdom often cause inconvenience for  
test developers, because they have to keep in mind a bonus factor, which is totally diverge from the basic variables they meant to manipulate, and vary in each individual \cite{PszichoMeresbook}.

The effect of being experienced in filling in tests was studied in a research, in which an aptitude test called GRE (Grand Record Examination) was used. Practise samples were sent to random participants 5 weeks before the examination. Those who got these samples also 
receieved advices for completing. In conclusion, the group with prior knowledge and practise earned significantly better results in the examination. Furthermore, there were also a notable growth in points, when the participants received an only 4 hours educational practise before the examination. It’s important to note that this difference and growth was present only in the logical reasoning part of the exam, and not in the mathematical and verbal parts 
\cite{PowersSwinton}. This data was reexamined later, because researchers wanted to know, if there is any difference when the existing groups would be split into subgroups by the different attributes of the participants. As a conclusion, there was no significant difference between the subgroups, but there was a notable difference in the group in which the members' primary language was not English, they scored lesser points than the others \cite{Powers86}.

Repeatedly performing the same experiment or test with the same participants could affect the results. Previously, as we specified, repeatedly using the same method could cause the lowering of its validity, and the results could be distorted. Participants can learn and adapt to certain methods, even if its just means a small percentage of difference. The current test takes 10 minutes to complete, in this 10 minutes the participant’s full attention and concentration is needed.  We should keep in mind, that the negative effects of fatigue could balance the positive effects of practise, in a direct way of repeated examinations. So this two factors should be considered in the evaulation, and while drawing consequence.


