
\boldparagraph{Fragmentation Scoring} The network predicts a score with a mean-absolute deviation of 0.45 from the test labels on the fragmentation test set of 216 embryos labeled at 3,652 times (Figure~\ref{fig:frag}, right). When distinguishing between low- ($<1.5$) and high- ($\ge 1.5$) fragmentation, the network and the test labels agree 88.9\% of the time. Our network outperforms a baseline InceptionV3 by 1.9\%; focus averaging and cropping to a region-of-interest each provide a 1--1.5\% boost to the accuracy (Table~\ref{table:ablation}).

We suspect that much of the fragmentation network's error comes from imprecise human labeling of the train and test sets, due to difficulties in distinguishing fragments from small cells and due to grouping the continuous fragmentation score into discrete bins. To evaluate the human labeling accuracy, two annotators label the fragmentation test set in duplicate and compare their results. The two annotators have a mean-absolute deviation of 0.37 and are 88.9\% consistent in distinguishing low- from high- fragmentation embryos. Thus, the fragmentation CNN performs nearly optimally in light of the labeling inaccuracies.

