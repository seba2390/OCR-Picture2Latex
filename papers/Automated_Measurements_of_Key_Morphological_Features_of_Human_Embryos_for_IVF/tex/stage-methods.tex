
\boldparagraph{Stage Classification} For low fragmentation embryos, we classify the embryo's developmental stage over time using a classification CNN (ResNeXt101~\cite{xie2017aggregated}). The classifier takes the three middle focal planes as input and predicts a 13-element vector of class probabilities, with 9 classes for cleavage-stage embryos (one each for 1--8 cells and one for $\ge 9$ cells) and one class each for morula (M), blastocyst (B), empty wells (E), and degenerate embryos (D; Figure~\ref{fig:stage}, left). To account for inaccuracies in the training data labels, we trained the classifier with a soft loss function modified from the standard cross-entropy loss
\begin{equation}
\log \left( p(\ell | m) \right) = \log \left( \sum_{t} p(\ell | t) p(t | m) \right) \quad ,
\label{eqn:softloss}
\end{equation}
where $t$ is the true stage of an image, $\ell$ the (possibly incorrect) label, and $m$ the model's prediction. We measured $p(\ell | t)$ by labeling 23,950 images in triplicate and using a majority vote to estimate the true label $t$ of each image. This soft-loss differs from the regularized loss in~\cite{szegedy2016rethinking} by differentially weighting classes; for instance, $p(\ell = \textrm{1-cell} | t = \textrm{1-cell}) = 0.996$ whereas $p(\ell = \textrm{6-cell} | t = \textrm{6-cell}) = 0.907$. Using the measured $p(\ell | t)$, we then trained the network with 341 embryos labeled at 111,107 times, along with a validation set of 73 embryos labeled at 23,381 times for early stopping~\cite{goodfellow2016deep}. Finally, we apply dynamic programming~\cite{bellman1966dynamic} to the predicted probabilities to find the most-likely non-decreasing trajectory, ignoring images labeled as empty or degenerate (Figure~\ref{fig:stage}, center).

