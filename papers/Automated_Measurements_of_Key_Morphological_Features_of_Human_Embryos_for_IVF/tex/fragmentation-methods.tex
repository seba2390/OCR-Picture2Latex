
\boldparagraph{Fragmentation Scoring} With the cropped image from the zona pellucida segmentation, we score the embryo's degree of fragmentation using a regression CNN (InceptionV3~\cite{szegedy2016rethinking}). The network takes a single-focus image as input and predicts a fragmentation score of 0 (0\% fragments), 1 ($<$10\%), 2 (10-20\%), or 3 ($\ge$20\%), following clinical practice. We train the network to minimize the $L^1$ loss on cleavage-stage images of 989 embryos at 16,315 times, each labeled with an integer score from 0--3; we use a validation set of 205 embryos labeled at 3,416 times for early stopping~\cite{goodfellow2016deep}. For each time point in the movie we analyze, we run the CNN on the three middle focal planes and take the average as the final score (Figure~\ref{fig:frag}, left).

Counting and identifying cells in fragmented embryos is difficult, inhibiting the labeling of train or test data for these embryos. Moreover, since high fragmentation is strongly correlated with low embryo viability~\cite{alikani1999human}, in standard clinical practice highly fragmented embryos are frequently discarded. Thus, we only train the rest of the networks on embryos with fragmentation less than 2.

