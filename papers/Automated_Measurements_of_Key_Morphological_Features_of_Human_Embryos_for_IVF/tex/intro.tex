
One in six couples worldwide suffer from infertility~\cite{cui2010mother}. Many of those couples seek to conceive via In-Vitro Fertilization (IVF). In IVF, a patient is stimulated to produce multiple oocytes. The oocytes are retrieved, fertilized, and the resulting embryos are cultured \textit{in vitro}. Some of these are then transferred to the mother's uterus in the hopes of achieving a pregnancy; the remaining viable embryos are cryopreserved for future treatments. While transferring multiple embryos to the mother increases the potential for success, it also increases the potential for multiple pregnancies, which are strongly associated with increased maternal morbidity and offspring morbidity and mortality~\cite{norwitz2005maternal}. Thus, it is highly desirable to transfer only one embryo, to produce only one healthy child~\cite{practice2017guidance}. This requires clinicians to select the best embryos for transfer, which remains challenging~\cite{racowsky2011national}. % FIXME misplaced modifier -- is transfer challenging or selection

The current standard of care is to select embryos primarily based on their morphology, by examining them under a microscope. In a typical embryo, after fertilization the two pronuclei, which contain the father's and mother's DNA, move together and migrate to the center of the embryo. The embryo undergoes a series of cell divisions, during the ``cleavage stage.'' Four days after fertilization, the embryo compacts and the cells firmly adhere to each other, at which time it is referred to as a compact ``morula.'' On the fifth day, the embryo forms a ``blastocyst,'' consisting of an outer layer of cells (the trophectoderm) enclosing a smaller mass (the inner-cell mass). On the sixth day, the blastocyst expands and hatches out of the zona pellucida (the thin eggshell that surrounds the embryo)~\cite{elder2020vitro}. Clinicians score embryos by manually measuring features such as cell number, cell shape, cell symmetry, the presence of cell fragments, and blastocyst appearance~\cite{elder2020vitro}, usually at discrete time points. Recently, many clinics have started to use time-lapse microscopy systems that continuously record movies of embryos without disturbing their culture conditions~\cite{rubio2014clinical,dolinko2017national,armstrong2019time}. However, these videos are typically analyzed manually, which is time-consuming and subjective.

Previous researchers have trained convolutional neural networks (CNNs) to directly predict embryo quality, using either single images or time-lapse videos \cite{petersen2016development,tran2019deep}. However, interpretability is vital for clinicians to make informed decisions on embryo selection, and an algorithm that directly predicts embryo quality from images is not interpretable. Worse, since external factors such as patient age~\cite{franasiak2014nature} and body-mass index~\cite{broughton2017obesity} also affect the success of an embryo transfer, an algorithm trained to predict embryo quality may instead learn a representation of confounding variables, which may change as IVF practices or demographics change. Some researchers have instead trained CNNs to extract a few identifiable features, such as blastocyst size~\cite{kheradmand2017inner}, blastocyst grade~\cite{kragh2019automatic,filho2012method,khosravi2019deep}, cell boundaries~\cite{rad2018hybrid}, or the number of cells when there are 4 or fewer~\cite{khan2016deep,lau2019embryo}. While extracting identifiable features obviates any problems with interpretability, these works leave out many key features that are believed to be important for embryo quality. Moreover, these methods are not fully automated, requiring the input images to be manually annotated as in the cleavage or blastocyst stage.

Here, we automate measurements of five key morphokinetic features of embryos in IVF by creating a unified pipeline of five CNNs. We work closely with clinicians to choose features relevant for clinical IVF: segmentation of the zona pellucida (Fig.~\ref{fig:network}a), grading the degree of fragmentation (Fig.~\ref{fig:network}b), classification of the developmental stage from 1-cell to blastocyst (Fig.~\ref{fig:network}c), object instance segmentation of cells in the cleavage stage (Fig.~\ref{fig:network}d), and object instance segmentation of pronuclei before the first cell division (Fig.~\ref{fig:network}e). With the exception of zona pellucida segmentation, all these features are used for embryo selection~\cite{alikani1999human,racowsky2011national,amir2019time,nickkho2019hydatidiform}; we segment the zona pellucida both to improve the other networks and because zona properties occasionally inform other IVF procedures~\cite{cohen1992implantation}. The five CNNs work together in a unified pipeline, combining results to improve performance over individual CNNs trained per task by several percent.

