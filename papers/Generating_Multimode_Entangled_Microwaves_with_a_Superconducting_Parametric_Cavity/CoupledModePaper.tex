\documentclass[prl,10pt,twocolumn,superscriptaddress,notitlepage,floatfix,amssymb]{revtex4}
%Oridinary packages
\usepackage[letterpaper,hmargin={1.5cm,1.5cm},vmargin={1.5cm,2.5cm}]{geometry}
\usepackage{mathtools}
\usepackage{float}
\usepackage{amsfonts}
\usepackage{enumitem}
\usepackage{graphicx}
\usepackage{color}
%Custom commands, safe to implement
\newcommand*{\ham}{\hat{H}}
\newcommand*{\ket}[1]{|{#1}\rangle}
\newcommand*{\bra}[1]{\langle{#1}|}
\newcommand*{\braket}[3][|]{\langle{#2}{#1}{#3}\rangle}
\newcommand*{\cre}[2][a]{\hat{#1}_{#2}^{\dagger}}	%\cre[A]{out}==\hat{A}_{out}
\newcommand*{\dcre}[2][a]{\dot{\hat{#1}}_{#2}^{\dagger}}
\newcommand*{\ddcre}[2][a]{\ddot{\hat{#1}}_{#2}^{\dagger}}
\newcommand*{\ann}[2][a]{\hat{#1}_{#2}}
\newcommand*{\dann}[2][a]{\dot{\hat{#1}}_{#2}}
\newcommand*{\ddann}[2][a]{\ddot{\hat{#1}}_{#2}}
\newcommand*{\rot}[1]{e^{-i\omega_{#1}t}}
\newcommand*{\arot}[1]{e^{i\omega_{#1}t}}
\newcommand*{\Inp}[1]{\hat{I}_{#1}}
\newcommand*{\Qua}[1]{\hat{Q}_{#1}}
\newcommand*{\num}[1]{\hat{n}_{#1}}

\begin{document}
\title{Generating Multimode Entangled Microwaves with a Superconducting Parametric Cavity}



\author{C.W. Sandbo Chang}
\affiliation{Institute for Quantum Computing and Electrical and Computer Engineering, University of Waterloo, Waterloo, Canada}
\author{M. Simoen}
\affiliation{MC2, Chalmers University of Technology, G\"oteborg, Sweden}
\author{Jos{\'e} Aumentado}
\affiliation{National Institute of Standards and Technology, 325 Broadway, Boulder, Colorado 80305, USA}
\author{Carlos Sab{\'\i}n}
\affiliation{Instituto de F{\'i}sica Fundamental, CSIC, Serrano, 113-bis, 28006 Madrid, Spain}
\author{P. Forn-D\'{i}az}
\affiliation{Institute for Quantum Computing and Electrical and Computer Engineering, University of Waterloo, Waterloo, Canada}
\author{A.M. Vadiraj}
\affiliation{Institute for Quantum Computing and Electrical and Computer Engineering, University of Waterloo, Waterloo, Canada}
\author{Fernando Quijandr\'{\i}a}
\affiliation{MC2, Chalmers University of Technology, G\"oteborg, Sweden} 
\author{G. Johansson}
\affiliation{MC2, Chalmers University of Technology, G\"oteborg, Sweden}
\author{I. Fuentes}
\affiliation{School of Mathematical Sciences, University of Nottingham, Nottingham NG7 2RD, United Kingdom}
\affiliation{Faculty of Physics, University of Vienna, Boltzmanngasse 5, 1090 Vienna, Austria}
\author{C.M. Wilson}
\affiliation{Institute for Quantum Computing and Electrical and Computer Engineering, University of Waterloo, Waterloo, Canada}

\email{chris.wilson@uwaterloo.ca}

%\author{C.W. Sandbo Chang$^{1}$, M. Simoen$^2$, J. Aumentado$^{3}$, Carlos Sab{\'\i}n$^4$, P. Forn-Diaz$^{1}$, A.M. Vadiraj$^{1}$, I.F.Q. Diaz$^2$, G. Johansson$^2$, I. Fuentes$^5$, C.M. Wilson$^{1}$}
%
%\email{chris.wilson@uwaterloo.ca}
%
%\affiliation{Institute for Quantum Computing and Electrical and Computer Engineering, University of Waterloo, Waterloo, Canada}
%\affiliation{MC2, Chalmers University of Technology, G\"oteborg, Sweden}
%\affiliation{National Institute of Standards and Technology, Boulder, CO, USA}
%\affiliation{Instituto de F{\'i}sica Fundamental, CSIC, Serrano, 113-bis, 28006 Madrid, Spain}
%\affiliation{In a superposition.}


\begin{abstract}
In this Letter, we demonstrate the generation of multimode entangled states of propagating microwaves.  The entangled states are generated by parametrically pumping a multimode superconducting cavity. By combining different pump frequencies, applied simultaneously to the device, we can produce different entanglement structures in a programable fashion.  The Gaussian output states are fully characterized by measuring the full covariance matrices of the modes. The covariance matrices are absolutely calibrated using an \textit{in situ} microwave calibration source, a shot noise tunnel junction. Applying a variety of entanglement measures, we demonstrate both full inseparability and genuine tripartite entanglement of the states.  Our method is easily extensible to more modes. 
\end{abstract}

\date{\today}

 \pacs{42.50.Gy, 85.25.Cp, 03.67.Hk}       

\maketitle

The generation and distribution of entanglement is an important problem in quantum information science. For instance, distributing entangled photons is a key paradigm in quantum communication \cite{Jennewein:2000ke}.  Distributing entangled photons as a way to entangle remote processing nodes of a larger quantum computer is also a promising path towards scalability \cite{Kimble:2008if,Felicetti:2014dz}. Multimode entangled states can also be used for a variety of quantum networking protocols such as quantum state sharing \cite{Lance:2004do}, quantum secret sharing \cite{Cleve:1999cb,Tyc:2002cf}, and quantum teleportation networks \cite{Yonezawa:2004wm}. It is therefore of great interest to develop novel ways of efficiently generating propagating entangled states. In this letter, we present a microwave circuit, a multimode parametric cavity, that generates propagating bipartite and tripartite entangled states of microwave photons. Furthermore, the entanglement structure of the tripartite states can be changed \textit{in situ} by the appropriate choice of pump frequencies.  The design is easily extensible to more modes using the same principle and techniques.  The ability to generate complex multimode states with a programmable entanglement structure would potentially enable a number of interesting advances beyond those already mentioned such as microwave cluster states \cite{Bruschi:2016hu}, error-correctable logical qubits for quantum communication \cite{Bolt:2016cv,Jouguet:2013df}, and the quantum simulation of relativistic quantum information processing systems \cite{Bruschi:2013cq, Wilson:2011ir}. 

Superconducting parametric cavities have shown great promise as a quantum technology platform in recent years. Quantum-limited parametric amplifiers have become almost commonplace in superconducting quantum computation. The parametric generation of bipartite continuous-variable (CV) entanglement between two microwave modes has been demonstrated using parametric cavities \cite{Flurin:2012hq,Flurin:2015jf,Fedorov:2016kp,Fedorov:2017tk}. Other work has shown that parametric processes can coherently couple microwave signals between different modes of a single cavity or multiple cavities, including generating superposition states of a single photon at different frequencies \cite{Sirois:2015gr,ZakkaBajjani:2011in}.  The generation of multimode CV entangled states at optical frequencies has also been demonstrated in a variety of ways \cite{Aoki:2003is,Yonezawa:2004wm,Lance:2004do,Pysher:2011hn,Gerke:2016hu,Shalm:2012iaa}.

In this work, by using a multimode superconducting parametric cavity, we demonstrate the generation, calibration and verification of multimode CV entanglement, observing genuine tripartite entanglement of three propagating microwave modes. The states are fully characterized by measuring the 6-by-6 covariance matrix of the mode quadratures. The device operates in steady-state, functioning as a continuous-wave source of entanglement. Our scheme can be extended beyond three modes by simply adding more pump tones.
 
%Continuous variable entanglement presents advantages in applying quantum technologies, potentially enabling continuous variable quantum key distribution with improved efficiency and robustness (PRA 87, 022308 (2013)), (PRL 109, 100502 (2012)), meanwhile acting as a promising alternative quantum computation resource such as realizing cluster state-based universal quantum computation (Scientific Reports 6, 18349 (2016)) and continuous variable qubit (PRL 105, 053602 (2010)). Combined with the flexibility in microwave regime (Nature 474, 589–597 (30 June 2011), Appl. Phys. Lett. 92, 203501 (2008)),

\begin{figure}
\includegraphics[width=0.6\linewidth]{Circuit.pdf}	
\includegraphics[width=0.9\linewidth]{TuningCurves.pdf}	
\label{Circuit}
\caption{Top Panel: Simplified schematic of the microwave measurement setup. From the bias tee onward, the measurement chain is shared by the parametric cavity and the SNTJ calibration source. The short connections between the switch and two devices are made as physically identical as possible. The system is calibrated independently at each of the measurement frequencies. Bottom Panels: Tuning curves of the three modes of the cavity, showing the tuning of the measured resonance frequencies with external magnetic flux, $\Phi_{\text{ext}}$ (in units of the flux quantum $\Phi_0$). The maximum frequency of the three modes are $f_{1,\text{max}} = 4.217$ GHz, $f_{2,\text{max}} = 6.171$ GHz and $f_{3,\text{max}} = 7.578$ GHz.  To allow individual difference frequencies to be addressed, the mode frequencies are dispersed by modulating the impedance of the cavity along its length.}
\end{figure}

The device is a quarter-wavelength coplanar waveguide resonator \ref{Circuit} terminated by a SQUID at one end.  On the other end, it is capacitively overcoupled $\left(\text{Q}\approx7000\right)$ to a nominally $Z_0 = 50$ $\Omega$ line.  The device is made using Al and standard photolithography and e-beam lithography techniques. The fundamental mode has a relatively low frequency of around 1 GHz, giving higher modes with an average frequency spacing of 2 GHz, such that three higher-order modes are accessible within our 4-8 GHz measurement bandwidth.  Parametric processes are driven by a microwave pump inductively coupled to the SQUID, modulating the boundary condition of the resonator. Previous work demonstrated that this type of device could operate as a nondegenerate parametric amplifier operating near the standard quantum limit \cite{Simoen:2015by,Wustmann:2013bma,Wustmann:2017vka}. In a uniform cavity, the mode frequencies are equally spaced, making it difficult to address individual pairs of modes. To avoid this problem, we follow the approach of \cite{ZakkaBajjani:2011in} and modulate the impedance of the transmission line along the length of the cavity, varying the impedance from 41 to 72 $\Omega$. %Our design produces an anharmonicity as large as 600 MHz (see Figure 1). 

As has been well-documented \cite{Yamamoto:2008cr,Wilson:2010fj,Wilson:2011ir}, the SQUID parametrically couples the total flux in the cavity, $\hat{\Phi}_{c}$, to the pump flux, $\hat{\Phi}_{p}$ through its Hamiltonian  $\hat{H}_{\text{SQ}}=E_J|\cos(\pi\hat{\Phi}_{p}/\Phi_0)|\cos(2\pi\hat{\Phi}_{c}/\Phi_0)$ \cite{Johansson:2009hf}.
%\begin{equation}
%\hat{H}_{SQ}=2E_J\left|\cos\left(\pi\frac{\hat{\Phi}_{p}}{\Phi_0}\right)\right|\cos\left(2\pi\frac{\hat{\Phi}_{c}}{\Phi_0}\right),	\label{SQUIDPotential}
%\end{equation}
Starting from this relation, we can derive our interaction Hamiltonian by expanding to first order in $\hat{\Phi}_{p}$ (around the flux bias $\Phi_{\text{ext}}$) and to  second order in  $\hat{\Phi}_{c}$.  Further, applying the parametric approximation to the pump, we find
\begin{equation}
\ham_{\text{int}}=\hbar g_0\left(\alpha_{p}+\alpha_{p}^*\right)\left(\ann[a]{1}+\cre[a]{1}+\ann[a]{2}+\cre[a]{2}+\ann[a]{3}+\cre[a]{3}\right)^2	\label{Hint}
\end{equation}
where $\alpha_p$ denotes the coherent pump amplitude, the bosonic operators $\ann[a]{i},\cre[a]{i}$ correspond to the three cavity modes considered here, and $g_0$ is an effective coupling constant. Eq. (\ref{Hint}) contains a large number of terms corresponding to different physical processes. However, we can selectively activate different processes by the appropriate choice of pump frequency. For instance, by choosing the sum frequency $f_{p}=f_i + f_j$, $\hat{H}_{\text{int}}$ can be reduced to $H_{\text{DC}}=\hbar g\left(\ann[a]{i}\ann[a]{j}+\cre[a]{i}\cre[a]{j}\right)$ by using the appropriate rotating-wave approximation.  $H_{\text{DC}}$ is well-known to produce parametric downconversion, which creates (destroys) pairs of photons and has been used to produce entangled photons in a wide variety of systems.  In particular, it has been used in superconducting microwave systems to produce two-mode squeezing (TMS) \cite{Flurin:2012hq}, a form of CV entanglement.  If we instead choose to pump at the difference frequency $f_{p}=|f_i - f_j|$, $\hat{H}_{\text{int}}$ reduces to $\hat{H}_{\text{CC}}=\hbar g'\left(\ann[a]{i}\cre[a]{j}+\cre[a]{i}\ann[a]{j}\right)$. $\hat{H}_{\text{CC}}$ produces a coherent coupling between modes.  The internal cavity modes described by $\ann[a]{i}$ can be connected to the propagating modes exterior to the cavity, described by operators $\ann[a]{i,\text{o}}$, using standard input-output theory \cite{Wustmann:2013bma}.%This beam splitter-like coupling was used in (Nature Physics 7, 599-603, (2011)) to coherently swap a single photon back and forth between two modes, including producing a superposition of the photon at different frequencies. 
%Above, $g$ and $g'$ are coupling constants in the respective rotating frames. 

In this Letter, we show experimentally that these two distinct classes of parametric processes can in fact be combined by simultaneously pumping at multiple frequencies, producing multipartite interactions between multiple modes in a way that is flexible and extensible. Because $H_{\text{DC}}$ and $H_{\text{CC}}$ do not commute with each other, it is not at all obvious that it should be possible to compose these operations in a straightforward manner.  In fact, the commutator of $H_{\text{DC}}$ and $H_{\text{CC}}$ plays an important role in generating the additional dynamics needed to generate multimode entanglement. This versatile method was first suggested in \cite{Bruschi:2016hu}, where it was shown that it is theoretically possible to produce multimode entangled states including CV cluster states.  Recent work has studied the computational complexity of the generated states, showing that they can be used for classically hard computations such as boson sampling \cite{Peropadre:2018dy}. The method generalizes previous work on squeezing \cite{Bruschi:2013cq}, mode-mixing quantum gates \cite{Bruschi:2013ka}, as well as entanglement \cite{Friis:2012ft,Friis:2013fn} in cavities undergoing relativistic motion. Earlier experimental work studied the development of multimode coherence in a parametric resonator pumped at two frequencies \cite{Paraoanu:2016fqa}.  To our knowledge, this is the first experimental work demonstrating that this scheme does, in fact, produce multimode entanglement, which has important implications for the field of relativistic quantum information among others.
%In this experiment, we demonstrate the generation of entanglement between three propagating microwave modes with an entanglement structure that can be selected by the appropriate choice of pumps. 

%We present two multipartite entanglement schemes, that we call the coupled-mode and bisqueezing schemes. Both  generate entanglement between three modes, but with a correlation structure that differs between the schemes. In the coupled-mode scheme, the device is pumped simultaneously at $f_{p1}= f_1 + f_3 =\text{11.78 GHz}$ and $f_{p2}=|f_2-f_3|=\text{1.40 GHz}$.  The pump at $f_{p1}$ produces TMS between $f_1$ and $f_3$,  while the pump at $f_{p2}$ coherently couples $f_3$ with the near-vacuum mode $f_2$.  In the bisqueezing scheme \cite{Bruschi:2017jy}, the pump tones are applied at $f_{p1}=f_1 + f_2 =\text{10.39 GHz}$ and $f_{p2}=f_2+f_3=\text{13.75 GHz}$, directly producing TMS correlations between the pairs $f_1,f_2$ and $f_2,f_3$. Note that these two cases where studied at different flux bias points, so that the individual mode frequencies are somewhat different between the two cases.

We present two multipartite entanglement schemes, that we call the coupled-mode (CM) and bisqueezing (BS) schemes. Both  generate entanglement between three modes, but with a correlation structure that differs. In the CM scheme, the device is pumped simultaneously at $f_{p1}= f_1 + f_2$  and $f_{p2}=|f_3-f_1|$.  The pump at $f_{p1}$ produces TMS between $f_1$ and $f_3$,  while the pump at $f_{p2}$ coherently couples $f_3$ with $f_2$.  In the BS scheme \cite{Bruschi:2017jy}, the pump tones are applied at $f_{p1}=f_1 + f_2 $ and $f_{p2}=f_2+f_3$, directly producing TMS correlations between the pairs $f_1,f_2$ and $f_2,f_3$. 

%Note that these two cases where studied at different flux bias points, so that the individual mode frequencies are somewhat different between the two cases (see Table \ref{Measures}).
%In doing so, correlations with the structure of TMS can be detected not only between mode $f_1$ and $f_3$, but also between mode $f_1$ and $f_2$, even though the latter two do not directly interact. Correlations can be detected between the indirect modes $f_1$ and $f_3$, but with the structure of coherently swapped modes (see below).
%As a result, the combination of parametric processes can generate interesting multimode correlations. 

We will characterize the entanglement in our propagating output states within the covariance formalism \cite{Simon:1994gb}.  With the good assumption that all of our $N$ modes are Gaussian \cite{Note1}, the state is fully characterized by the $2N \times 2N$ covariance matrix $\mathbf{V}$ of the I and Q voltage quadratures of the propagating modes. For the theoretical analysis, the measured voltage quadratures are calibrated and scaled, as shown below, to produce the quantities $\ann[x]{i}=\ann[a]{i,\text{o}}+\cre[a]{i,\text{o}}$ and $\ann[p]{i}=-i\left(\ann[a]{i,\text{o}}-\cre[a]{i,\text{o}}\right)$.  By collecting the N-mode quadrature operator terms into a vector operator $\mathbf{\hat{K}}=\left(\ann[x]{1},\ann[p]{1},\ann[x]{2},\ann[p]{2},\dots,\ann[x]{N},\ann[p]{N}\right)^T$, the elements in $\mathbf{V}$ are defined as $V_{ij}=\left\langle \hat{K}_i\hat{K}_j +  \hat{K}_j\hat{K}_i\right\rangle/2$ (assuming the modes are mean zero).

To test the validity of our calibration, we can first test if our measured covariance matrices are physical. To be physical in a classical sense, $\mathbf{V}$ has to be real, symmetric and positive semidefinite. 
%The first two requirements are automatically fulfilled by our definition. The last check can be performed by verifying its minimum eigenvalue $\lambda$ is nonnegative, i.e. $\lambda_{\text{min}}(\mathbf{V})\geq0$. This test generalizes the simple idea that the covariances cannot be larger than the variances. 
To be physical in the quantum sense, $\mathbf{V}$ must also obey the Heisenberg uncertainty principle. 
%related to the commutation relations of $\mathbf{\hat{K}}$. We can write the commutation relations as $\left[\hat{K}_i,\hat{K}_j\right]=2i\Omega_{ij}$ (PhysRevA.49.1567, 1994), (PRA 72, 032334, 2005)), where $\mathbf{\Omega}$ is the symplectic form which, in our basis, can be written as $\mathbf{\Omega}=\begin{pmatrix}0&1\\-1&0\end{pmatrix}\otimes\mathbf{I}_N$.  
It has been shown \cite{Simon:1994gb} that the uncertainty principle can be expressed in terms of the symplectic eigenvalues, $\nu_i$, of $\mathbf{V}$, which are found by diagnolizing $\mathbf{V}$ through a canonical transformation of $\mathbf{\hat{K}}$.  With these definitions, the uncertainty principle simply states $\nu_i \geq 1$ for all $i$. All the measured covariances matrices below were found to be physical according to these definitions \cite{SupNote}.
%(Conveniently, the symplectic eigenvalues can also be found as the absolute values of the regular eigenvalues of $i\mathbf{\Omega V}$.) In other words, we can say that $\mathbf{V}$ is (quantum) physical if $\nu_{\text{min}}(\mathbf{V})\geq1$. 


We can now study the entanglement properties of $\mathbf{V}$.  A common measure of entanglement in CV systems is the logarithmic negativity, $\mathcal{N}$, which derives from the positive partial transpose (PPT) criterion \cite{Simon:2000fd,Adesso:2005fw}. The physical picture of the PPT criterion is that if we time-reverse a subsystem (partition) of a multimode entangled state, then the resulting total state will be \textit{unphysical}.  Testing for entanglement then corresponds to confirming that the covariance matrix of the partial transpose state $\widetilde{\mathbf{V}}$ is \textit{unphysical}. That is, the entanglement condition is  $\widetilde{\nu}_{\text{min}}\equiv \nu_{\text{min}}(\widetilde{\mathbf{V}}) < 1$ or equivalently $\mathcal{N} \equiv \textrm{max}[0,-\ln(\widetilde{\nu}_{\text{min}})] > 0$.

%In our covariance matrix description, the transpose operation, \textit{i.e.} time-reversal, applied to the density matrix corresponds to changing the signs of the $p$ quadratures of the state \cite{Simon:2000fd}.  Partial transposition then corresponds to changing the sign of a subset of the $p$ quadratures. For instance, for a 1-mode$|$(N-1)-mode partition of the total state, this corresponds to  the transformation $\mathbf{V}\rightarrow\widetilde{\mathbf{V}}=\mathbf{\Lambda}_i\mathbf{V}\mathbf{\Lambda}_i$, where $\mathbf{\Lambda}_i=\text{diag}\left(a_1,a_2,\dots,a_{2i},\dots,a_{2N}\right)$ is a diagonal matrix with all $a=1$ except $a_{2i}=-1$, corresponding to partitioning the $i$-th mode.

The PPT criterion and $\mathcal{N}$ suffice to fully characterize two-mode Gaussian states but, as is well-known, classifying entanglement quickly grows complex with increasing $N$. Limiting ourselves to three-mode states, early work suggested classifying entanglement based on applying the PPT criterion to the three possible bipartitions of the state \cite{Giedke:2001ey, vanLoock:2003hn}.  This work proposed a highest class of ``fully inseparable" states, where all bipartitions are entangled. This class can be quantified by the so-called tripartite negativity $\mathcal{N}^{tri} = (\mathcal{N}^{A} \mathcal{N}^{B} \mathcal{N}^{C})^{1/3}$, where ${A,B,C}$ label bipartitions, which is only nonzero for fully inseparable states \cite{Sabin:2008ce}.

It was later pointed out \cite{Teh:2014ij,Shchukin:2015ci,Gerke:2016hu} that, although this test rules out that any one mode is separable from the whole, it does \textit{not} rule out that the state is a mixture of states, each of which is separable. That is, there exists states of the form $\rho = a \rho_1 \rho_{23} + b \rho_2 \rho_{13} + c \rho_3 \rho_{12}$, where $a + b + c = 1$, which are fully inseparable according to the above definition \cite{Teh:2014ij}.  It was suggested that the term ``genuine" tripartite entanglement be reserved for states that \textit{cannot} be written as such a convex sum.  We note that this distinction between full inseparability and genuine entanglement only exists for mixed states, so understanding the purity of the state under study is important.

Ref  \cite{Teh:2014ij} derived a set of generalized inequalities to test for genuine tripartite entanglement. We define linear combinations of our quadratures $u = h_1 x_1 + h_2 x_2 + h_3 x_3$ and $v = g_1 p_1 + g_2 p_2 + g_3 p_3$, where the $h_i$ and $g_i$ are arbitrary real constants to be optimized. It was shown that states without genuine entanglement satisfy the inequality 
\begin{equation}
S \equiv \langle \Delta u^2 \rangle + \langle \Delta v^2 \rangle \ge 2 \min \{| h_i g_i | + | h_j g_j + h_k g_k |  \}
\label{Bound}
\end{equation}
where the minimization is over permutations of $\{i,j,k\}$. We can reduce the optimization space and simplify the bound by putting restrictions on the coefficients. For this Letter, we will use the two cases i) $h_l = g_l = 1$, $h_m = h_n = h$, $g_m = g_n = g$, $h g < 1$ and ii) $h_l = g_l = 1$, $h_m = -g_n$, $h_n = -g_m$ both with the search domain $[-1,1]$. With these restrictions, the bound simplifies to 2.

%Ref  \cite{Teh:2014ij} derived a set of generalized inequalities to test for genuine tripartite entanglement. In the simplest form, we define linear combinations of our quadratures $u = x_i + h(x_j + x_k)$ and $v = p_i + g(p_j + p_k)$, where $h$ and $g$ are arbitrary constants to be optimized and $\{i,j,k\}$ is some permutation of $\{1,2,3\}$. It was shown that with $h$ and $g$ restricted to the domain $[-1,1]$, that states without genuine entanglement satisfy the inequality 
%\begin{equation}
%S \equiv \langle \Delta u^2 \rangle + \langle \Delta v^2 \rangle \ge 2.
%\label{Bound}
%\end{equation}
%We note that this is only a sufficient condition and, in particular, a stricter bound can be found by allowing more free coefficients in the definition of $u$ and $v$, at the cost of having to optimize in a higher dimensional search space.

To operate the device, the SQUID is flux biased to within 10\% of $\Phi_0$.  The pump tones are combined and feed to the on-chip pump line. The output of the device is fed through circulators to a cryogenic HEMT amplifier. After further amplification at room temperature, the signal is split in two paths and then fed through custom-made image-rejection filters into a pair of RF digitizers.  The digitizers output $I$ and $Q$ samples with a variable bandwidth.  In this work, the bandwidth was $BW = 1$ MHz. The variances and covariances of the $I$ and $Q$ time series are then computed. The measurements are done sequentially for the three mode pairs.  We remove the effects of drift by performing a chopped measurement, with the pumps turned on for 5 seconds followed by the pumps turned off for 5 seconds. The differenced data is then averaged over many cycles, typically 1000.
%Our system noise is significant when compared to, if not larger than, the output power of our parametric cavity.  It must therefore be accurately measured and subtracted from the total signal.  This is complicated by the fact that the system output noise drifts, likely due to some combination of drift in the gain and actually noise.  However, as the drift is relatively slow ($<$10\% over 15 minutes) \textbf{(Sandbo, is the drift really 10\%?  That seems large.)}, 
%We remove the effects of drift in the measurement by performing a chopped measurement with the pump tones turned on for 5 seconds followed by the pumps turned off for another 5 seconds. When the pump is off, the device simply reflects the input noise. Thus, the corresponding OFF signal is the system noise plus the \textit{input} noise of the mode. The final data is obtained by subtracting the OFF data from the ON data for each cycle. This differenced data is then averaged over many cycles, typically 50-200.

%\begin{figure}
%\centering
%\includegraphics[width=1\linewidth]{Coupled-mode.pdf}	\label{Coupled-mode}\\
%%\includegraphics[width=1\linewidth]{Line.pdf}
%\includegraphics[width=1\linewidth]{Bisqueezing.pdf}	\label{Bisqueezing}
%\caption{The combined pumping schemes for tripartite entanglement generation. a) The coupled-mode scheme, pump 1 and pump 2 tones are applied simultaneously, with pump 1 being the two-mode down-conversion tone which is generating photons in the 4.21 GHz and 7.57 GHz modes. Pump 2 is the coherent coupling tone which swap the signal between the 6.17 GHz vacuum mode and the 7.57 GHz down-converted mode. As a result there are signals coming out in steady state from all three modes with a non trivial correlation structure. b) The bisqueezing scheme. Pump 1 and pump 2 are two individual phase-locked two-mode down-conversion tone, generating photons respectively in 4.22|6.17 GHz and 6.17|7.58 GHz mode pairs. With mode 2 being in common for the two TMDC, the three modes exhibits a similar correlation structure to the coupled-mode scheme. Both of the schemes are therefore potentially giving rise to tripartite entanglement.}
%\end{figure}

\begin{table*}
%\begin{tabular}{|c||c|c||c|c|c|c||c|c|c|c|}
\begin{tabular}{|c||c|c||c|c|c|c|}
\hline
% &\multicolumn{2}{|c||}{Frequencies} & \multicolumn{4}{|c||}{$T_{\text{in}}=60$ mK} & \multicolumn{4}{|c|}{$T_{\text{in}}=120$ mK} \\
  &\multicolumn{2}{|c||}{Frequencies} & \multicolumn{3}{|c|}{Entanglement Measures} \\
\hline
%Scheme & Modes &  Pumps & $N_Q$ & $\tilde{\nu}_{\text{min}}$ & $\mathcal{N}$ & $S$ & $N_Q$ & $\tilde{\nu}_{\text{min}}$ & $\mathcal{N}$ & $S$ \\
Scheme & Modes &  Pumps & $\tilde{\nu}_{\text{min}}$ & $\mathcal{N}^{tri}$ & $S$\\
\hline
%\hline
%TMS & 6.17, 7.57 & 13.74 & 1.01, 1.00& $0.29\pm0.01$ & $1.24$ & -- & 1.17,1.09 & $0.41\pm0.01$ & $0.89$ & -- \\
\hline
%CM & 4.20, 6.16, 7.55 & 10.36, 3.35 & 1.07, 1.01, 1.00 & .23, .30, .34 & $1.24$ & $0.80\pm0.05$ & 1.42, 1.17, 1.09 & $.47, .59, .61$ & $0.58$ & $1.66\pm0.04$\\
CM & 4.20, 6.16, 7.55 & 10.36, 3.35 & $0.48\pm0.002$, $0.39\pm0.002$, $0.57\pm0.002$ & $0.73\pm0.005$ & $1.49\pm0.01$\\
\hline
%BS & 4.20, 6.16, 7.55 & 10.36, 11.75  & 1.07, 1.01, 1.00 & .55, .43, .64 & $0.62$ & $1.94\pm0.028$ & 1.42, 1.16, 1.09 & $.83, .61, .77$ & $0.29$ & $2.72\pm0.026$ \\
BS & 4.20, 6.16, 7.55 & 10.36, 11.75 & $0.31\pm0.003$, $0.48\pm0.004$, $0.39\pm0.004$ & $0.94\pm0.012$ & $1.19\pm0.01$\\
\hline
\end{tabular}
%\begin{tabular}{|c||c|c|c|c|c|c|}
%\hline
% & \multicolumn{3}{|c|}{$T_{\text{in}}=60$ mK} & \multicolumn{3}{|c|}{$T_{\text{in}}=120$ mK} \\
%\hline
%Scheme & $\tilde{\nu}_{\text{min}}$ & $\mathcal{N}$ & $S$ & $\tilde{\nu}_{\text{min}}$ & $\mathcal{N}$ & $S$ \\
%\hline
%TMS & $.29\pm.02$ & $1.23$ & -- & $.42\pm.02$ & $.86$ & -- \\
%\hline
%CM & .23,.30,.34 & $1.24$ & $.80\pm.04$ & $.49,.61,.63$ & $.55$ & $1.71\pm.04$\\
%\hline
%BS & .55,.43,.63 & $.64$ & $1.98\pm.03$ & $.86,.64,.79$ & $.25$ & 3.06 \\
%\hline
%\end{tabular}
\caption{\label{Measures} Entanglement measures and frequencies for the various pumping schemes. 
%TMS is the two-mode case.  
CM is the coupled-mode scheme. BS is the bisqueezing scheme. The Frequencies columns list the respective mode and pump frequencies. The $\tilde{\nu}_{\text{min}}$ column reports the minimum symplectic eigenvalues for all three bipartition from the PPT tests. The $\mathcal{N}^{tri}$ column reports the tripartite negativity. The $S$ column reports the measure of genuine tripartite entanglement in Eq. 2.  The entanglement conditions are $\tilde{\nu}_{\text{min}}<1$;  $\mathcal{N} > 0$; and $S < 2$. Statistical errors are reported. See the Supplemental Material for a discussion of systematic error \cite{SupNote}. We find full inseparability and genuine tripartite entanglement for both entanglement schemes.}
\end{table*}

The entanglement tests described above compare the variances and covariances of the modes at the level of the vacuum noise.  It is therefore essential to have an accurate, absolute calibration of $\mathbf{V}$.  In this experiment, we perform this calibration using a shot noise tunnel junction (SNTJ) \cite{Spietz:2003vy,Spietz:2006ck} produced by NIST-Boulder \cite{SupNote}. 
%The reference plane for the calibration is the output of the parametric cavity. We use a microwave switch mounted on the dilution unit to switch between the parametric cavity and SNTJ.  The overall circuit\ref{Circuit} is designed such that as much of the amplifier chain as possible is shared between the SNTJ and the resonator to ensure near identical overall transmission. The short sections of line that are not shared are made as physically identical as possible and any transmission differences are calibrated. We use the SNTJ to give an absolute calibration of the total system gain, $G_i$, at each of the measurement frequencies. 
 
The quadrature voltages at room temperature, $\Inp{i}$ and $\Qua{i}$, are converted to the scaled quadrature variables $\hat{x}_i$ and $\hat{p}_i$ using the calibrated system gains, $G_i$.  Following recent work \cite{Flurin:2015jf}, the scaled variance at the device output is
\begin{equation}
\langle\ann[x]{i}^2\rangle=\frac{4\left(\langle\Inp{i}^2\rangle_{ON}-\langle\Inp{i}^2\rangle_{OFF}\right)}{G_iZ_0 h f_i BW}+\coth\frac{h f_i}{2k_BT_i}
\end{equation}
with a similar definition for $\hat{p}_i$. The $\coth()$ term here represents the input quantum noise, at temperature $T_i$, which is (unfortunately) subtracted when we subtract the reference noise measured with the pump off. Without the input noise, the output variance will be underestimated, leading to an overestimate of the degree of entanglement or even an erroneous claim of entanglement.  It is therefore critical to characterize $T_i$ \cite{Bruschi:2013cd}.  Assuming the mode is in the vacuum state is tantamount to assuming that the system is entangled. In our setup, the calibration of the system gain using the SNTJ also gives us the physical electron temperature of the SNTJ.  As detailed in the Supplemental Material \cite{SupNote}, we find values of 25-37 mK over the course of our measurements.  For our working frequencies, these temperatures are deeply in the quantum regime, giving $\coth(h f_i/2k_BT_i) = 1.00$ with at least 3 significant figures for all our measurements.

Estimating the covariances of our modes is easier since neither the input noise nor system noise is correlated at different frequencies.  The covariance is then obtained by simply rescaling the room temperature values as, e.g.,
\begin{equation}
\langle\ann[x]{i}\ann[x]{j}\rangle=\frac{4\langle\Inp{i}\Inp{j}\rangle_{ON}}{\sqrt{G_iG_j f_i f_j}Z_0 h BW}.
\end{equation} 
As $\mathbf{V}$ is symmetric, just 21 terms in the matrix need to be individually measured for $N=3$ modes.

%As a first test of our measurement setup, we tested the generation of a two-mode squeezed state, produced by pumping modes 2 and 3 with a single pump tone at 13.74 GHz. As reported in Table \ref{Measures}, we found that the output state was entangled for both temperatures.

%We obtain the directly measured 4 x 4 covariance matrix $\mathbf{V}_{6,7}-\mathbf{N}_{\text{Q}}$:
%\begin{align*}
%	\bordermatrix{
%		 & x_2 & p_2 & x_3 & p_3 &\ \ \ \ \ \\
%    x_2 & 2.73 & 0.00& {\color{blue} 3.23} &-0.01\\
%	p_2 & 0.00 & 2.75& 0.01 &{\color{red}-3.24}\\
%	x_3 & {\color{blue} 3.23}  & 0.01& 2.28 & 0.00\\
%	p_3 &-0.01 &{\color{red}-3.24}& 0.00 & 2.31}
%\end{align*}
%with $\mathbf{N}_{\text{Q}} = \text{diag}(...,\coth(h f_i/2k_BT_i),...)$ a diagonal matrix with the corresponding input quantum noise for each mode.  The correlations are color coded (online) with significant positive (negative) correlations in blue (red). For an input temperature of 120 mK, we find $\mathbf{N}_{\text{Q}}^{\text{120}} =  \text{diag}(1.19,1.19,1.10,1.10)$ and for 60 mK we find $\mathbf{N}_{\text{Q}}^{\text{60}} =  \text{diag}(1.01,1.01,1.00,1.00)$.  As reported in Table \ref{Measures}, we find that state is entangled assuming input temperatures of both 120 mK and 60 mK.
%Applying the PPT test to the 120 mK case, we find $\tilde{\nu}_{\text{min}}^{\text{120}}(\widetilde{\mathbf{V}}_{6,7})=0.42\pm0.02$, which is significantly below the bound of 1, implying that the modes are entangled even for this worst-case estimate.  The negativity is then $\mathcal{N}^{6,7}_{\textrm{120}}=0.86$ assuming 120 mK, rising to $\mathcal{N}^{6,7}_{\textrm{60}}=1.23$ assuming 60 mK.


%We can then study the tripartite CM scheme. 
To study the tripartite CM scheme, we measure the 6 x 6 matrix $\mathbf{V}_{4,6,7}$:
\begin{align*}
	\bordermatrix{
		 & x_1 & p_1 & x_2 & p_2 & x_3 & p_3\ \\
    x_1 & 2.05&0.00&{\color{blue}1.87}&0.00&{\color{blue}0.88}&0.00\ \\
	p_1 & 0.00&2.04&0.00&{\color{red}-1.87}&0.00&{\color{blue}0.88}\ \\
	x_2 & {\color{blue}1.87}&0.00&2.85&0.00&{\color{blue}1.56}&0.00\ \\
	p_2 & 0.00&{\color{red}-1.87}&0.00&2.85&0.00&{\color{red}-1.56}\ \\
	x_3 & {\color{blue}0.88}&0.00&{\color{blue}1.56}&0.00&1.79&0.00\ \\
	p_3 & 0.00&{\color{blue}0.88}&0.00&{\color{red}-1.56}&0.00&1.79\ }.
\end{align*}
The correlations are color coded (online) with significant positive (negative) correlations in blue (red). As reported in Table \ref{Measures}, this state demonstrates both full inseparability and genuine tripartite entanglement.  This is the major result of this Letter.

%As described above, we can then analyze the inseparability of the state by checking for bipartite entanglement between the 3 possible bipartitions of modes, i.e., 4-6,7, 6-4,7 and 7-4,6. Working with the worst-case $\mathbf{N}_{\text{Q}}^{\text{120}} =  \text{diag}(1.46,1.46,1.19,1.19,1.10,1.10)$ and applying time-reversal to mode 1 yields $\nu^{\text{4-6,7}}_{\text{min}}=0.49$, confirming the bipartition is entangled. Similarly, the other two bipartitions give $\nu^{\textrm{6-4,7}}_{\text{min}}=0.61$ and $\nu^{\textrm{7-4,6}}_{\text{min}}=0.63$, indicating entanglement for all three bipartitions.  We find a lower bound on the tripartite negativity of $\mathcal{N}^{467}_{\textrm{120}}=\left(\mathcal{N}^{\textrm{4-6,7}}\mathcal{N}^{\textrm{6-4,7}}\mathcal{N}^{\textrm{7-4,6}}\right)^\frac{1}{3}=0.55$. Assuming 60 mK, we find the upper bound $\mathcal{N}^{467}_{\textrm{60}} = 1.24$.  We thus find that this tripartite state is fully inseparable.

%We can also test the state for genuine tripartite entanglement, following the definitions above. Taking the mode permutation $u = x_1 + h(x_2 + x_3)$ and $v = p_1 + g(p_2 + p_3)$, we numerically minimize the statistic $S$ defined above. Again, the value depends on the temperature we assume for the input state.  For the worst-case, we find $S_{120}=1.71 \pm 0.04 < 2$ with the parameters $h = -g = -0.60$.  In the best case, we find $S_{60} = 0.80 \pm 0.04 < 2$ with the same parameters.  In either case, the state violates the inequality (\ref{Bound}) by several standard deviations, clearly indicating that it exhibits genuine tripartite entanglement. 


For the BS scheme, we have the measured matrix $\mathbf{V}_{4,6,7}$:
\begin{align*}
	\bordermatrix{
		& x_1  & p_1 & x_2  & p_2  & x_3  & p_3\ \\
    x_1 & 3.91&0.00&{\color{blue}2.34}&0.00&{\color{blue}2.78}&0.00\ \\
	p_1 & 0.00&3.91&0.00&{\color{red}-2.33}&0.00&{\color{red}-2.78}\ \\
	x_2 & {\color{blue}2.34}&0.00&2.28&0.00&{\color{blue}1.45}&0.00\ \\
	p_2 & 0.00&{\color{red}-2.33}&0.00&2.28&0.00&{\color{blue}1.45}\ \\
	x_3 & {\color{blue}2.78}&0.00&{\color{blue}1.45}&0.00&2.72&0.00\ \\
	p_3 & 0.00&{\color{red}-2.78}&0.00&{\color{blue}1.45}&0.00&2.72\ }.
\end{align*}
As shown in Table \ref{Measures}, we again find that the state demonstrates both full inseparability and genuine tripartite entanglement. % assuming $T_{\text{in}} = 60$ mK, but not for $T_{\text{in}} = 115$ mK.  Although these results are equivocal, they gives us confidence that genuine entanglement could be achieved in a future experiment.
%The entanglement check of the three bipartitions gives $\nu^{\textrm{4-6,7}}_{\text{min}}=0.86$, $\nu^{\textrm{6-4,7}}_{\text{min}}=0.64$ and $\nu^{\textrm{7-4,6}}_{\text{min}}=0.79$. The bounds on the tripartite negativity are $\mathcal{N}^{467}_{\textrm{120}} = 0.25$ and $\mathcal{N}^{467}_{\textrm{60}} = 0.61$.  We therefore find that this state is also fully inseparable. 

%We can also evaluate the genuine entanglement of this state. We find the best mode permutation to be $u = x_2 + h(x_1 + x_3)$ and $v = p_2 + g(p_1 + p_3)$.  For the worst-case, we find $S(120\text{ mK})=3.06 > 2$ with the parameters $h = -g = \text{?}$.  In the best case, we find $S(60\text{ mK})= 1.975 \pm 0.026 \sim 2$ with $h = -0.7$ and $g=0.76$.  This simplified test therefore fails to find genuine tripartite entanglement.

%We can therefore perform a more exhaustive test. Defining now $u' = x_2 + h_1 x_1 + h_3 x_3$ and $v' = p_2 + g_1 p_1 + g_3 p_3$, we can optimize $S' \equiv \langle \Delta u'^2 \rangle + \langle \Delta v'^2 \rangle$ over the four free parameters. The only complication is that now the bound, $S'_{B}$ on $S'$ depends explicitly on the values of the $h_i$,$g_i$ and must be explicitly computed for each set of values (ref). For the 120 mK case, the extended search does not improve the result. For the 60 mK case, we find an optimal value of $S' = 1.96 \pm 0.027$ at the values $h_1 = -0.7$, $h_3 = -0.8$, $g_1 = 0.8$, $g_3 = 0.7$.  At these values of the parameter the bound is $S'_{B} = 2$.  We therefore find a marginal violation at the 1.5 standard deviation level. While perhaps not strong enough to confidently claim genuine tripartite entanglement for this state, the result does give us confidence that tripartite entanglement could be achieved in a future experiment.

The main limitation on the degree of entanglement in the system seems to be the purity of the output states.  For an ideal system, pumping harder should increase the degree of squeezing without degrading the purity of the state, and therefore increase the parametric gain and degree of entanglement monotonically.  We instead see that the gain increases with pump strength, but the purity of the states simultaneous declines, limiting the maximum degree of entanglement. This suggests some form of nonideality such as higher-order nonlinearities, self-heating, or parasitic coupling to other cavity modes. These limitations can be more thoroughly investigated in future work.

The authors wish to thank B. Plourde, J.J. Nelson and M. Hutchings at Syracuse University for invaluable help in junction fabrication.  We would also like to acknowledge M. Piani for helpful discussions. CMW, CWSC, PFD and AMV acknowledge NSERC of Canada, the Canadian Foundation for Innovation, the Ontario Ministry of Research and Innovation, Canada First Research Excellence Fund (CFREF), Industry Canada, and the CMC for financial support. Financial support by Fundaci{\'o}n General CSIC (Programa ComFuturo) is acknowledged by CS as well as additional support from Spanish MINECO/FEDER FIS2015-70856-P and CAM PRICYT Project QUITEMAD+ S2013/ICE-2801. IF acknowledges support from EPSRC (CAF Grant No. EP/G00496X/2). FQ and GJ acknowledge the support from the Knut and Alice Wallenberg Foundation.



%\textbf{Conclusion}
%In conclusion, we have developed a microwave source that provides true tripartite entanglement with steady-state output over three different frequency modes. The output is verified with SNTJ to be a valid physical state, with entanglement checked qualitatively and quantified by negativity. Having the tunable frequency modes, the device can be tailored to cover applications over a wide frequency range, with the amount of entanglement further increased by improved fabrication quality. High precision control of entanglement generation and coherent coupling among the modes can be performed with standard pulsing technique to obtain specified entanglement structure, which will make this device a flexible building block for future quantum networks.
\bibstyle{apsrev}
\bibliography{CoupledModePaper,Notes}

\end{document}
