\documentclass[a4paper]{article}

\usepackage[margin=1in]{geometry} %

\usepackage{amsmath}
\usepackage{amsthm}
\usepackage{amssymb}
\usepackage{graphicx}
\usepackage{xcolor}
\usepackage{mathrsfs} %
\usepackage[utf8]{inputenc}
\usepackage{adjustbox}
\usepackage{enumerate}

\usepackage{tikz}
\usetikzlibrary{shapes,calc,math,backgrounds,matrix}

\usepackage{hyperref}


\usepackage[%
backend=biber, 
bibstyle=numeric, 
citestyle=numeric-comp,
maxcitenames=3,
maxbibnames=20,
sorting=ydnt
]{biblatex}
\addbibresource{refs.bib}

\theoremstyle{plain}
\newtheorem{theorem}{Theorem}
\newtheorem{corollary}[theorem]{Corollary}
\newtheorem{lemma}[theorem]{Lemma}
\newtheorem{claim}{Claim}[theorem]
\newtheorem{axiom}[theorem]{Axiom}
\newtheorem{conjecture}[theorem]{Conjecture}
\newtheorem{fact}[theorem]{Fact}
\newtheorem{hypothesis}[theorem]{Hypothesis}
\newtheorem{assumption}[theorem]{Assumption}
\newtheorem{proposition}[theorem]{Proposition}
\newtheorem{criterion}[theorem]{Criterion}
\theoremstyle{definition}
\newtheorem{definition}[theorem]{Definition}
\newtheorem{example}[theorem]{Example}
\newtheorem{remark}[theorem]{Remark}
\newtheorem{problem}[theorem]{Problem}
\newtheorem{principle}[theorem]{Principle}

\usepackage[linesnumbered,ruled,vlined,commentsnumbered]{algorithm2e} %

\newcommand{\EMPH}[2][red]{\textcolor{#1}{\textit{#2}}}
\newcommand{\sfTS}{{\mathsf{TS}}} %
\newcommand{\sfTJ}{{\mathsf{TJ}}} %
\newcommand{\sfTAR}{{\mathsf{TAR}}} %
\newcommand{\sfR}{{\mathsf{R}}} %
\newcommand{\calP}{{\mathcal{P}}}
\newcommand{\calG}{{\mathcal{G}}}
\newcommand{\calH}{{\mathcal{H}}}
\newcommand{\calS}{{\mathcal{S}}}
\newcommand{\calR}{{\mathcal{R}}}
\newcommand{\calL}{{\mathcal{L}}}
\newcommand{\Ip}{{I^{\prime}}}
\newcommand{\Ipp}{{I^{\prime\prime}}}
\newcommand{\Istar}{{I^{\star}}}
\newcommand{\Itil}{{\widetilde{I}}}
\newcommand{\Jp}{{J^{\prime}}}
\newcommand{\Jpp}{{J^{\prime\prime}}}
\newcommand{\Jstar}{{J^{\star}}}
\newcommand{\Gbar}{{\overline{G}}}
\newcommand{\Gtil}{{\widetilde{G}}}
\newcommand{\Gp}{{G^{\prime}}}
\newcommand{\Gpp}{{G^{\prime\prime}}}
\newcommand{\Tbar}{{\overline{T}}}
\newcommand{\Ttil}{{\widetilde{T}}}
\newcommand{\Tp}{{T^{\prime}}}
\newcommand{\Tpp}{{T^{\prime\prime}}}
\newcommand{\Tstar}{{T^\star}}
\newcommand{\ttPSPACE}{{\mathtt{PSPACE}}}
\newcommand{\ttNP}{{\mathtt{NP}}}
\newcommand{\ttP}{{\mathtt{P}}}
\newcommand{\ttFPT}{{\mathtt{FPT}}}
\newcommand{\dist}{\mathsf{dist}} %
\newcommand{\diam}{\mathsf{diam}} %
\newcommand{\opt}{\mathsf{OPT}} %
\newcommand{\reconf}[2][\sfR]{\overset{#2}{\longrightarrow}_{#1}} %
\newcommand\redsout{\bgroup\markoverwith{\textcolor{red}{\rule[0.5ex]{2pt}{2pt}}}\ULon} %



\title{\textbf{The Complexity of Distance-$r$ Dominating Set Reconfiguration}}
\author{Niranka~Banerjee$^1$ \and Duc~A.~Hoang$^2$}
\date{
	$^1$ Research Institute of Mathematical Sciences, Kyoto University, Japan\\
          \href{mailto:niranka@gmail.com}
          {\texttt{niranka@gmail.com}}\\
        $^2$ VNU University of Science, Vietnam National University, Hanoi, Vietnam\\
	\href{mailto:hoanganhduc@hus.edu.vn}{\texttt{hoanganhduc@hus.edu.vn}}
}

\begin{document}
\maketitle

\begin{abstract}
For a fixed integer $r \geq 1$, a \textit{distance-$r$ dominating set} of a graph $G = (V, E)$ is a vertex subset $D \subseteq V$ such that every vertex in $V$ is within distance $r$ from some member of $D$.
Given two distance-$r$ dominating sets $D_s, D_t$ of $G$, the \textsc{Distance-$r$ Dominating Set Reconfiguration (D$r$DSR)} problem asks if there is a sequence of distance-$r$ dominating sets that transforms $D_s$ into $D_t$ (or vice versa) such that each intermediate member is obtained from its predecessor by applying a given reconfiguration rule exactly once.
The problem for $r = 1$ has been well-studied in the literature.
We consider \textsc{D$r$DSR} for $r \geq 2$ under two well-known reconfiguration rules: Token Jumping ($\mathsf{TJ}$, which involves replacing a member of the current D$r$DS by a non-member) and Token Sliding ($\mathsf{TS}$, which involves replacing a member of the current D$r$DS by an adjacent non-member).
We show that \textsc{D$r$DSR} ($r \geq 2$) is $\mathtt{PSPACE}$-complete under both $\mathsf{TJ}$ and $\mathsf{TS}$ on bipartite graphs, planar graphs of maximum degree six and bounded bandwidth, and chordal graphs.
On the positive side, we show that \textsc{D$r$DSR} ($r \geq 2$) can be solved in polynomial time on split graphs and cographs under both $\mathsf{TS}$ and $\mathsf{TJ}$ and on trees and interval graphs under $\mathsf{TJ}$.
Along the way, we observe some properties of a shortest reconfiguration sequence in split graphs when $r = 2$, which may be of independent interest.

\noindent\textbf{Keywords:} distance-$r$ dominating set, reconfiguration problem, computational complexity, PSPACE-completeness, polynomial-time algorithm

\noindent\textbf{2020 MSC:} 05C85
\end{abstract}

\section{Introduction}
\label{sec:intro}

For the last few decades, \textit{reconfiguration problems} have independently emerged in different areas of computer science, including recreational mathematics (e.g., games and puzzles), computational geometry (e.g., flip graphs of triangulations), constraint satisfaction (e.g., solution space of Boolean formulas), and even quantum complexity theory (e.g., ground state connectivity), and so on~\cite{Heuvel13,Nishimura18,MynhardtN19,BousquetMNS22}.	
In a \textit{reconfiguration variant} of a computational problem (e.g., \textsc{Satisfiability}, \textsc{Independent Set}, \textsc{Dominating Set}, \textsc{Vertex-Coloring}, etc.), two \textit{feasible solutions} (e.g., satisfying truth assignments, independent sets, dominating sets, proper vertex-colorings, etc.) $S$ and $T$ are given along with a \textit{reconfiguration rule} that describes how to slightly modify one feasible solution to obtain a new one.
The question is whether one can transform/reconfigure $S$ into $T$ via a sequence of feasible solutions such that each intermediate member is obtained from its predecessor by applying the given rule exactly once.
Such a sequence, if exists, is called a \textit{reconfiguration sequence}.

In 1975, Meir and Moon~\cite{MeirM75} combined the concepts of ``distance'' and ``domination'' in graphs and introduced the so-called \textit{distance-$r$ dominating set} of a graph (which they called a ``$r$-covering''), where $r \geq 1$ is some fixed integer.
For a fixed positive integer $r$ and a graph $G$, a \textit{distance-$r$ dominating set} (D$r$DS) (also known as \textit{$r$-hop dominating set} or \textit{$r$-basis}) of $G$ is a vertex subset $D$ where each vertex of $G$ is within distance $r$ from some member of $D$.
In particular, a D$1$DS is nothing but a \textit{dominating set} of $G$.
Given a graph $G$ and some positive integer $k$, the \textsc{Distance-$r$ Dominating Set} problem asks if there is a D$r$DS of $G$ of size at most $k$.
\textsc{Distance-$r$ Dominating Set} remains $\ttNP$-complete even on bipartite graphs or chordal graphs of diameter $2r + 1$~\cite{ChangN84}.
Since the work of Meir and Moon, this notion of distance domination in graphs has been extensively studied from different perspectives.
We refer readers to the survey of Henning~\cite{Henning20} for more details.

Reconfiguration of dominating sets has been well-studied in the literature from both algorithmic and graph-theoretic viewpoints.
We briefly mention here some well-known algorithmic results and refer readers to~\cite{MynhardtN19} for recent developments from the graph-theoretic perspective.
Imagine that a token is placed on each vertex of a dominating set of a graph $G$ and no vertex has more than one token.
In a reconfiguration setting for dominating sets, the following reconfiguration rules have been considered.
\begin{itemize}
	\item \textbf{Token Sliding ($\sfTS$):} one can move a token to one of its unoccupied neighbors as long as the resulting token-set forms a dominating set.
	\item \textbf{Token Jumping ($\sfTJ$):} one can move a token to any unoccupied vertex as long as the resulting token-set forms a dominating set.
	\item \textbf{Token Addition/Removal ($\sfTAR(k)$):} one can either add or remove a token as long as the resulting token-set forms a dominating set of size at most some threshold $k \geq 0$.
\end{itemize}
Haddadan~et~al.~\cite{HaddadanIMNOST16} first studied the computational complexity of \textsc{Dominating Set Reconfiguration (DSR)} under $\sfTAR$ and showed that the problem is $\ttPSPACE$-complete on planar graphs of maximum degree six, bounded bandwidth graphs, split graphs and bipartite graphs.
On the positive side, they designed polynomial-time algorithms for solving \textsc{DSR} under $\sfTAR$ on cographs, forests, and interval graphs.
Bonamy~et~al.~\cite{BonamyDO21} observed that the $\sfTJ$ and $\sfTAR$ rules are equivalent under some constraints, and therefore the above-mentioned results indeed hold for \textsc{DSR} under $\sfTJ$.
Bonamy~et~al.~\cite{BonamyDO21} first studied the computational complexity of \textsc{DSR} under $\sfTS$ and showed that the hardness results of Haddadan~et~al.~\cite{HaddadanIMNOST16} hold even under $\sfTS$.
On the positive side, Bonamy~et~al.~\cite{BonamyDO21} designed polynomial-time algorithms for solving \textsc{DSR} under $\sfTS$ on cographs and dually chordal graphs (which contains trees and interval graphs).
Bousquet and Joffard~\cite{BousquetJ21} later showed that \textsc{DSR} under $\sfTS$ is $\ttPSPACE$-complete on circle graphs and can be solved in polynomial time on circular-arc graphs, answering some open questions previously asked by Bonamy~et~al.~\cite{BonamyDO21}.
Recently, K\v{r}i\v{s}t'an and Svoboda~\cite{KristanS23} improved the positive results of Bonamy~et~al.~\cite{BonamyDO21} by showing polynomial-time algorithms to find a \textit{shortest} reconfiguration sequence, if exists, between two given dominating sets under $\sfTS$ when the input graph is either a tree or an interval graph.
However, their techniques cannot be extended to dually chordal graphs.

A systematic study on the parameterized complexity of several reconfiguration problems, including \textsc{DSR}, was initiated by Mouawad~et~al.~\cite{MouawadN0SS17}.
There are two natural parameterizations: the number of tokens $k$ and the length of a reconfiguration sequence $\ell$.
In~\cite{MouawadN0SS17}, Mouawad~et~al. showed that \textsc{DSR} under $\sfTAR$ on general graphs is $\mathtt{W[1]}$-hard parameterized by $k$ and $\mathtt{W[2]}$-hard parameterized by $k + \ell$.
When parameterized by $k$ on graphs excluding $K_{d,d}$ as a subgraph for any constant $d$ (including bounded degeneracy and nowhere dense graphs), Lokshtanov~et~al.~\cite{LokshtanovMPRS18} designed an $\mathtt{FPT}$ algorithm for solving the problem.
When parameterized by $\ell$ alone, it was mentioned in~\cite{BousquetMNS22} that the problem is fixed-parameter tractable on any class where first-order model-checking is fixed-parameter tractable.
We refer readers to~\cite{LokshtanovMPRS18,BousquetMNS22} and the references therein for more details.

To the best of our knowledge, for some fixed integer $r \geq 2$, the computational complexity of \textsc{Distance-$r$ Dominating Set Reconfiguration (D$r$DSR)} has not yet been studied.
On the other hand, from the parameterized complexity viewpoint, Siebertz~\cite{Siebertz18} studied \textsc{D$r$DSR} under $\sfTAR$ parameterized by $k$ and proved that there exists a constant $r$ such that the problem is $\mathtt{W[2]}$-hard on somewhere dense graphs which are close under taking subgraphs.
On the positive side, Siebertz showed that the problem is in $\mathtt{FPT}$ on nowhere dense graphs.
From the graph-theoretic viewpoint, DeVos~et~al.~\cite{DeVosDJS20} introduced the \textit{$\gamma_r$-graph} of a graph $G$---a (reconfiguration) graph whose nodes are \textit{minimum} distance-$r$ dominating sets of $G$ and two nodes are adjacent if one can be obtained from the other by applying a single $\sfTJ$-move---and proved a number of results on its realizability.
In this paper, to obtain a better understanding of the separating line between ``hard'' and ``easy'' instances of \textsc{D$r$DSR} ($r \geq 2$) on different graphs, we study the problem under $\sfTS$ and $\sfTJ$ from the computational complexity viewpoint.
(The definitions of these rules are similar to those for \textsc{DSR}.)


\subsection*{Our Results}
In Section~\ref{sec:hardness}, we prove hardness results of \textsc{D$r$DSR} in different graph classes.
We show that most of the hardness results of Haddadan~et~al.~\cite{HaddadanIMNOST16} and Bonamy~et~al.~\cite{BonamyDO21} for $r = 1$ can be extended for fixed integer $r \geq 2$ under both $\sfTS$ and $\sfTJ$.
In Section~\ref{sec:bipartite}, we show that \textsc{D$r$DSR} ($r \geq 2$) is $\ttPSPACE$-complete on bipartite graphs (Theorem~\ref{thm:bipartite}). In Section ~\ref{sec:planar} we show that \textsc{D$r$DSR} ($r \geq 2$) on planar graphs of maximum degree six and bounded bandwidth is $\ttPSPACE$-complete. (Theorem~\ref{thm:planar}).
Recall that Haddadan~et~al.~\cite{HaddadanIMNOST16} proved the $\ttPSPACE$-completeness of \textsc{D$r$DSR} under $\sfTAR$ on split graphs (and therefore on chordal graphs) for $r = 1$.
In Section~\ref{sec:chordal}, we show that for fixed integer $r \geq 2$, \textsc{D$r$DSR} remains $\ttPSPACE$-complete on chordal graphs (Theorem~\ref{thm:chordal}) under $\sfTS$ and $\sfTJ$. 
Note that all of our hardness results also hold even when considering minimum D$r$DSs.

In Section~\ref{sec:polytime}, we prove some upper bound results of  \textsc{D$r$DSR} in different graph classes. 
In Sections~\ref{sec:power-graph} and~\ref{sec:bounded-diam}, we prove some simple observations which then imply that \textsc{D$r$DSR} can be solved in polynomial time under $\sfTJ$ on interval graphs (Corollary~\ref{cor:TJ-interval}) and under both $\sfTS$ and $\sfTJ$ on cographs (Corollary~\ref{cor:TJ-cograph}).
In Section~\ref{sec:split}, we show that for fixed integer $r \geq 2$, \textsc{D$r$DSR} can be solved in polynomial time on split graphs under both $\sfTS$ and $\sfTJ$ (Theorem~\ref{thm:TS-split} and Corollary~\ref{cor:TJ-split}). 
This result provides a surprising dichotomy to the lower bound result on split graphs of Haddadan et al. \cite{HaddadanIMNOST16}. 
We also provide examples showing that the natural lower bounds on the lengths of a shortest reconfiguration sequence are sometimes not achievable on split graphs under both $\sfTS$ and $\sfTJ$ when $r = 2$ (Theorems~\ref{thm:TS-split-shortest} and~\ref{thm:TJ-split-shortest}).
In Section~\ref{sec:trees}, we prove that \textsc{D$r$DSR} under $\sfTJ$ on trees (and forests) can be solved in polynomial time (Theorem~\ref{thm:TJ-trees}). 
In short, we show that the positive results of Haddadan et al. \cite{HaddadanIMNOST16} for $r = 1$ under $\sfTAR$ on cographs, forests, and interval graphs also hold for fixed $r \geq 2$ under $\sfTJ$.

\section{Preliminaries}
\label{sec:preliminaries}

For the concepts and notations not defined here, we refer readers to~\cite{Diestel2017}.
Unless otherwise mentioned, throughout this paper, we always consider simple, connected, undirected graphs $G$ with vertex-set $V(G)$ and edge-set $E(G)$.
For any pair of vertices $u, v$, the \textit{distance} between $u$ and $v$ in $G$, denoted by $\dist_G(u, v)$, is the length of a shortest path between them.
For two vertex subsets $X, Y$, we use $X - Y$ and $X + Y$ to indicate $X \setminus Y$ and $X \cup Y$, respectively.
If $Y$ contains a single vertex $u$, we write $X - u$ and $X + u$ instead of $X - \{u\}$ and $X + \{u\}$, respectively.
We denote by $X \Delta Y$ their \textit{symmetric difference}, i.e., $X \Delta Y = (X - Y) + (Y - X)$.
For a subgraph $H$ of $G$, we denote by $G - H$ the graph obtained from $G$ by deleting all vertices of $H$ and their incident edges in $G$.

A \textit{dominating set (DS)} of $G$ is a vertex subset $D$ such that for every $u \in V(G)$, there exists $v \in D$ such that $\dist_G(u, v) \leq 1$.
For a fixed positive integer $r$, a \textit{distance-$r$ dominating set (D$r$DS)} of $G$ is a vertex subset $D$ such that for every $v \in V(G)$, there exists $v \in D$ such that $\dist_G(u, v) \leq r$.
In particular, any D$1$DS is also a DS and vice versa.
Let $N^r_G[u]$ be the set of all vertices of distance at most $r$ from $u$ in $G$.
We say that a vertex $v$ is \textit{$r$-dominated} by $u$ (or $u$ \textit{$r$-dominates} $v$) if $v \in N^r_G[u]$.
We say that a vertex subset $X$ is \textit{$r$-dominated} by some vertex subset $Y$ if each vertex in $X$ is $r$-dominated by some vertex in $Y$.
A D$r$DS is nothing but a vertex subset $D$ that $r$-dominates $V(G)$.
We denote by $\gamma_r(G)$ the size of a minimum D$r$DS of $G$.

We say that $u$ \textit{covers} the edge $e \in E(G)$ if $u$ is an endpoint of $e$.
A \textit{vertex cover (VC)} of $G$ is a vertex subset $C$ such that for every edge $uv \in E(G)$, either $u \in C$ or $v \in C$.
Intuitively, vertices in $C$ cover all edges of $G$.
Observe that in a connected graph $G$, any VC is also a DS.
We denote by $\tau(G)$ the size of a minimum VC of $G$.

Throughout this paper, we write ``$(G, D_s, D_t)$ under $\sfR$'' to indicate an instance of \textsc{D$r$DSR} where $D_s$ and $D_t$ are two given D$r$DSs of a graph $G$ and the reconfiguration rule is $\sfR \in \{\sfTS, \sfTJ\}$.
Imagine that a token is placed on each vertex in a D$r$DS of a graph $G$. 
A \textit{$\sfTS$-sequence} in $G$ between two D$r$DSs $D_s$ and $D_t$ is the sequence $\mathcal{S} = \langle D_s = D_0, D_1, \dots, D_q = D_t \rangle$ such that for $i \in \{0, \dots, q-1\}$, the set $D_i$ is a D$r$DS of $G$ and there exists a pair $x_i, y_i \in V(G)$ such that $D_i - D_{i+1} = \{x_i\}$, $D_{i+1} - D_i = \{y_i\}$, and $x_iy_i \in E(G)$.
A \textit{$\sfTJ$-sequence} in $G$ can be defined similarly without the restriction $x_iy_i \in E(G)$.
Depending on the considered rule $\sfR \in \{\sfTS, \sfTJ\}$, we can also say that $D_{i+1}$ is obtained from $D_i$ by \textit{immediately sliding/jumping} a token from $x_i$ to $y_i$ and write $x_i \reconf[\sfR]{G} y_i$.
Thus, we can also write $\mathcal{S} = \langle x_0 \reconf[\sfR]{G} y_0, \dots, x_{q-1} \reconf[\sfR]{G} y_{q-1} \rangle$.
In short, $\mathcal{S}$ can be viewed as a (ordered) sequence of either D$r$DSs or token-moves.
(Recall that we defined $\mathcal{S}$ as a sequence between $D_s$ and $D_t$. As a result, when regarding $\mathcal{S}$ as a sequence of token-moves, we implicitly assume that the initial D$r$DS is $D_s$.) 
With respect to the latter viewpoint, we say that $\mathcal{S}$ \textit{slides/jumps a token $t$ from $u$ to $v$ in $G$} if $t$ is originally placed on $u \in D_0$ and finally on $v \in D_q$ after performing $\mathcal{S}$. 
The \textit{length} of a $\sfR$-sequence is simply the number of times the rule $\sfR$ is applied.
Additionally, the length of a \textit{shortest} $\sfR$-sequence in $G$ between two D$r$DSs $D_s$ and $D_t$ is denoted by $\opt_{\sfR}(G, D_s, D_t)$.


\section{Hardness Results}
\label{sec:hardness}

We begin this section with the following simple observation.
It is well-known that for any computational problem in $\ttNP$, any of its reconfiguration variants is in $\ttPSPACE$~\cite{ItoDHPSUU11}[Theorem~1]. 
As a result, when proving the $\ttPSPACE$-completeness of \textsc{D$r$DSR} on a certain graph class, it suffices to show a polynomial-time reduction.

\subsection{Bipartite Graphs}
\label{sec:bipartite}

\begin{theorem}\label{thm:bipartite}
	\textsc{D$r$DSR} under $\sfR \in \{\sfTS, \sfTJ\}$ on bipartite graphs is $\ttPSPACE$-complete for any $r \geq 2$.
\end{theorem}
\begin{proof}
	We give a polynomial-time reduction from \textsc{Minimum Vertex Cover Reconfiguration (M-VCR)} on general graphs, which was showed to be $\ttPSPACE$-complete by Ito~et~al.~\cite{ItoDHPSUU11}.
	Our reduction extends the one given by Bonamy~et~al.~\cite{BonamyDO21} for the case $r = 1$.
	Let $(G, C_s, C_t)$ be an instance of \textsc{M-VCR} under $\sfR$ where $C_s, C_t$ are two minimum VCs of a graph $G$.
	We will construct an instance $(G^\prime, D_s, D_t)$ of \textsc{D$r$DSR} under $\sfR$ where $D_s$ and $D_t$ are two D$r$DSs of a bipartite graph $G^\prime$.
	
	Suppose that $V(G) = \{v_1, \dots, v_n\}$.
	We construct $G^\prime$ from $G$ as follows.
	\begin{enumerate}[(a)]
		\item Replace each edge $v_iv_j$ by a path $P_{ij} = x_{ij}^0x_{ij}^1\dots x_{ij}^{2r}$ of length $2r$ ($1 \leq i, j \leq n$) with $x_{ij}^0 = v_i$ and $x_{ij}^{2r} = v_j$. 
		Observe that $x_{ij}^p = x_{ji}^{2r - p}$ for $0 \leq p \leq 2r$.
		\item Add a new vertex $x$ and join it to every vertex in $V(G)$.
		\item Attach a new path $P_x$ of length $r$ to $x$.
	\end{enumerate}
	We define $D_s = C_s + x$ and $D_t = C_t + x$.
	Clearly, this construction can be done in polynomial time.
	(See \figurename~\ref{fig:bipartite}.)
	
	\begin{figure}[ht]
		\centering
		\includegraphics[width=0.7\textwidth]{figs/bipartite}
		\caption{An example of constructing a bipartite graph $G^\prime$ from a graph $G$ when $r = 2$ in the proof of Theorem~\ref{thm:bipartite}.}
		\label{fig:bipartite}
	\end{figure}
	
	In the next two claims, we prove that our construction results in an instance of \textsc{D$r$DSR} on bipartite graphs: Claim~\ref{clm:Gp-bipartite} shows that $G^\prime$ is bipartite and Claim~\ref{clm:Cx-DrDS} implies that both $D_s$ and $D_t$ are minimum D$r$DSs of $G^\prime$.
	\begin{claim}\label{clm:Gp-bipartite}
		$G^\prime$ is a bipartite graph.
	\end{claim}
	\begin{proof}
		We show that any cycle $\mathcal{C}^\prime$ in $G^\prime$ has even length and therefore $G^\prime$ is bipartite.
		Observe that no cycle of $G^\prime$ contains a vertex from $V(P_x) - x$.
		From the construction, if $\mathcal{C}^\prime$ does not contain $x$, it follows that $V(\mathcal{C}^\prime) \cap V(G)$ must form a cycle $\mathcal{C}$ of $G$.
		Therefore, $\mathcal{C}^\prime$ is of length $2r|E(\mathcal{C})|$, which is even.
		On the other hand, if $\mathcal{C^\prime}$ contains $x$, it follows that $V(\mathcal{C}^\prime) \cap V(G)$ must form a path $\mathcal{P}$ of $G$.
		Therefore, $\mathcal{C}^\prime$ is of length $2 + 2r|E(\mathcal{P})|$, which again is even.
	\end{proof}
	
	\begin{claim}\label{clm:Cx-DrDS}
		Any set of the form $C + x$, where $C$ is a minimum VC of $G$, is a minimum D$r$DS of $G^\prime$.
	\end{claim}
	\begin{proof}
		To see this, note that, by construction, $x$ $r$-dominates every vertex in $V(G^\prime) - \bigcup_{v_iv_j \in E(G)}\{x_{ij}^r\}$.
		Additionally, since each $x_{ij}^r$ belongs to exactly one path $P_{ij}$ and $C$ is a minimum VC of $G$, it follows that $C$ $r$-dominates $\bigcup_{v_iv_j \in E(G)}\{x_{ij}^r\}$.
		Thus, $C + x$ $r$-dominates $V(G^\prime)$, i.e., $C + x$ is a D$r$DS of $G^\prime$.
		
		It remains to show that $C + x$ is minimum.
		Indeed, it is sufficient to show that $\tau(G) + 1 = \gamma_r(G^\prime)$ where $\tau(G)$ and $\gamma_r(G^\prime)$ are respectively the size of a minimum VC of $G$ and a minimum D$r$DS of $G^\prime$.
		Since $C + x$ is a D$r$DS of $G^\prime$, we have $\tau(G) + 1 = |C| + x \geq  \gamma_r(G^\prime)$.
		On the other hand, note that any minimum D$r$DS $D^\prime$ of $G^\prime$ must $r$-dominate $V(P_x)$ and therefore contains a vertex of $P_x$ (which, by the construction, does not belong to $G$).
		Moreover, from the construction of $G^\prime$, each path $P_{ij}$ ($1 \leq i, j \leq n$ and $v_iv_j \in E(G)$) is of length $2r$.
		Thus, in order to $r$-dominate all $V(P_{ij})$, $D^\prime$ needs to contain at least one vertex from each path $P_{ij}$.
		Therefore, $\gamma_r(G^\prime) = |D^\prime| \geq 1 + \tau(G)$.
		Our proof is complete.
	\end{proof}
	
	Before proving the correctness of our reduction, we prove some useful observations.
	\begin{claim}\label{clm:vertex-x}
		Let $D = C + x$ be a D$r$DS of $G^\prime$ where $C$ is a minimum VC of $G$.
		Then,
		\begin{enumerate}[(a)]
			\item $D - u + y$ is not a D$r$DS of $G^\prime$, for any $u \in C$ and $y \in V(P_x) - x$.
			\item $D - x + v$ is not a D$r$DS of $G^\prime$, for any $v \in V(G^\prime) - x$.
			\item $D - u + z$ is not a D$r$DS of $G^\prime$, for $u = v_i \in C$ and $z \notin \bigcup_{\{j \mid v_j \in N_G(v_i) - C\}}V(P_{ij})$.
		\end{enumerate}
	\end{claim}
	\begin{proof}
		\begin{enumerate}[(a)]
			\item Suppose to the contrary there exists $u \in C$ and $y \in V(P_x) - x$ such that $D^\prime = D - u + y$ is a D$r$DS of $G^\prime$.
			Since $|D^\prime| = |D| = \tau(G) + 1$, Claim~\ref{clm:Cx-DrDS} implies that $D^\prime$ is a minimum D$r$DS of $G^\prime$.
			On the other hand, from the construction of $G^\prime$, any vertex $r$-dominated by $y$ must also be $r$-dominated by $x$. 
			Thus, $D^\prime - y$ is also a D$r$DS of $G^\prime$, which is a contradiction.
			
			\item Suppose to the contrary there exists $v \in V(G^\prime) - x$ such that $D^\prime = D - x + v$ is a D$r$DS of $G^\prime$.
			From (a), it follows that $v \in V(P_x) - x$; otherwise some vertex of $P_x$ would not be $r$-dominated by any member of $D^\prime$.
			Since $C$ is a minimum VC of $G$, it follows that there exists a pair $i, j \in \{1, \dots, n\}$ such that $v_iv_j \in E(G)$ and $C \cap \{v_i, v_j\} = \{v_i\}$; otherwise every vertex of $G$ would contain a token in $C$ and therefore $C$ would not be minimum---a contradiction.
			From the construction of $G^\prime$, $x$ is the unique vertex in $V(P_x)$ that $r$-dominates $x_{ij}^{r+1}$.
			Thus, $x_{ij}^{r+1}$ is not $r$-dominated by any vertex in $D^\prime = D - x + v$ for $v \in V(P_x) - x$, which is a contradiction.
			
			\item Let $j \in \{1, \dots, n\}$ be such that $v_j \in N_G(u) - C = N_G(v_i) - C$.
			Since $C$ is a minimum VC of $G$, such a vertex $v_j$ must exist.
			From the construction of $G^\prime$, the vertex $u = v_i$ is the unique vertex in $D$ that $r$-dominates $x_{ij}^r$.
			Thus, in order to keep $x_{ij}^r$ being $r$-dominated, a token on $u$ can only be moved to some vertex in $\bigcup_{\{j \mid v_j \in N_G(v_i) - C\}}V(P_{ij})$.
		\end{enumerate}
	\end{proof}
	Intuitively, starting from a token-set of the form $C + x$ for some minimum VC $C$ of $G$, Claim~\ref{clm:vertex-x}(a) means that as long as $x$ has a token, no other token can be moved in $G^\prime$ to a vertex in $P_x - x$, Claim~\ref{clm:vertex-x}(b) implies that if $x$ has a token, it will never be moved in $G^\prime$, and Claim~\ref{clm:vertex-x}(c) says that a token cannot be moved in $G^\prime$ ``too far'' from its original position.
	
	We are now ready to show the correctness of our reduction.
	(Claims~\ref{clm:TS-bipartite} and~\ref{clm:TJ-bipartite}.)
	\begin{claim}\label{clm:TS-bipartite}
		Under $\sfTS$, $(G, C_s, C_t)$  is a yes-instance if and only if $(G^\prime, D_s, D_t)$ is a yes-instance.
	\end{claim}
	\begin{proof}
		\begin{itemize}
			\item[($\Rightarrow$)] Suppose that $\calS$ is a $\sfTS$-sequence in $G$ between $C_s$ and $C_t$.
			We construct a sequence $\calS^\prime$ of token-slides in $G^\prime$ between $D_s$ and $D_t$ by replacing each $v_i \reconf[\sfTS]{G} v_j$ in $\calS$ with the sequence $\calS_{ij} = \langle v_i \reconf[\sfTS]{G^\prime} x_{ij}^1, x_{ij}^1 \reconf[\sfTS]{G^\prime} x_{ij}^2, \dots, x_{ij}^{2r-1} \reconf[\sfTS]{G^\prime} x_{ij}^{2r} \rangle$, where $i, j \in \{1, \dots, n\}$ and $v_iv_j \in E(G)$.
			Intuitively, sliding a token $t_{ij}$ from $v_i$ to its neighbor $v_j$ in $G$ corresponds to sliding $t_{ij}$ from $v_i$ to $v_j$ in $G^\prime$ along the path $P_{ij}$.
			Since $x$ always $r$-dominates $V(G^\prime) - \bigcup_{v_iv_j \in E(G)}\{x_{ij}^r\}$ and after each move in $\calS_{ij}$ the token $t_{ij}$ always $r$-dominates $x_{ij}^r$, it follows that $\calS_{ij}$ is indeed a $\sfTS$-sequence in $G^\prime$.
			Thus, $\calS^\prime$ is our desired $\sfTS$-sequence.
			
			\item[($\Leftarrow$)] Let $\calS^\prime$ be a $\sfTS$-sequence in $G^\prime$ between $D_s$ and $D_t$.
			We describe how to construct a $\sfTS$-sequence $\calS$ in $G$ between $C_s$ and $C_t$.
			Initially, $\calS = \emptyset$.
			For each move $u \reconf[\sfTS]{G^\prime} v$ ($u \neq v$) in $\calS^\prime$, we consider the following cases:
			\begin{enumerate}[{\bf {Case} 1:}]
				\item {\bf Either $u \in V(P_x)$ or $v \in V(P_x)$.} 
				Since $\calS^\prime$ starts with $D_s = C_s + x$, Claim~\ref{clm:vertex-x} ensures that no token can be placed on some vertex in $P_x - x$ and $x$ always contains a token.
				As a result, this case does not happen.
				
				\item {\bf $u = x_{ij}^p$ and $v = x_{ij}^{p+1}$ for $0 \leq p \leq 2r$.} Recall that $x_{ij}^0 = v_i$ and $x_{ij}^{2r} = v_j$. 
				Additionally, from our construction, since $x_{ij}^p$ has a token, so does $v_i$.
				If $p = 2r-1$, append the move $v_i \reconf[\sfTS]{G} v_j$ to $\calS$.
				Otherwise, do nothing.
			\end{enumerate}
			To see that $\calS$ is indeed a $\sfTS$-sequence in $G$, it suffices to show that if $C$ is the minimum VC obtained right before the move $v_i \reconf[\sfTS]{G} v_j$ then $C^\prime = C - v_i + v_j$ is also a minimum VC of $G$.
			Suppose to the contrary that $C^\prime$ is not a vertex cover of $G$.
			It follows that there exists $k \in \{1, \dots, n\}$ such that $v_iv_k \in E(G)$, $v_k \neq v_j$, and $v_k \notin C$.
			Intuitively, the edge $v_iv_k$ is not covered by any vertex in $C^\prime$.
			On the other hand, let $D$ be the D$r$DS of $G^\prime$ obtained right before the move $x_{ij}^{2r-1} \reconf[\sfTS]{G^\prime} x_{ij}^{2r} = v_j$.
			Since $D^\prime = D - x_{ij}^{2r-1} + x_{ij}^{2r}$ is also a D$r$DS of $G^\prime$, there must be some vertex in $D^\prime$ that $r$-dominates $x_{ik}^r$, which implies $V(P_{ik}) \cap D^\prime \neq \emptyset$.
			However, from the construction of $\calS$, it follows that $v_k \in C$, which is a contradiction.
			Thus, $C^\prime$ is a vertex cover of $G$.
			Since $|C^\prime| = |C|$, it is also minimum.
		\end{itemize}
	\end{proof}
	
	\begin{claim}\label{clm:TJ-bipartite} %
		Under $\sfTJ$, $(G, C_s, C_t)$  is a yes-instance if and only if $(G^\prime, D_s, D_t)$ is a yes-instance.
	\end{claim}
	\begin{proof}
		\begin{itemize}
			\item[($\Rightarrow$)] Suppose that $\calS$ is a $\sfTJ$-sequence in $G$ between $C_s$ and $C_t$.
			It follows from Claim~\ref{clm:Cx-DrDS} that the sequence $\calS^\prime$ of token-jumps obtained from $\calS$ by replacing each move $u \reconf[\sfTJ]{G} v$ in $\calS$ by $u \reconf[\sfTJ]{G^\prime} v$ is a $\sfTJ$-sequence between $D_s = C_s + x$ and $D_t = C_t + x$.
			
			\item[($\Leftarrow$)] On the other hand, let $\calS^\prime$ be a $\sfTJ$-sequence in $G^\prime$ between $D_s$ and $D_t$.
			We describe how to construct a $\sfTJ$-sequence $\calS$ in $G$ between $C_s$ and $C_t$.
			Initially, $\calS = \emptyset$.
			For each move $u \reconf[\sfTJ]{G^\prime} v$ ($u \neq v$) in $\calS^\prime$, we consider the following cases:
			\begin{enumerate}[{\bf {Case} 1:}]
				\item {\bf Either $u \in V(P_x)$ or $v \in V(P_x)$.} As before, it follows from Claim~\ref{clm:vertex-x} that this case does not happen.
				
				\item {\bf $u = x_{ij}^p$ and $v = x_{ij}^{q}$ for $0 \leq p, q \leq 2r$.} Recall that $x_{ij}^0 = v_i$ and $x_{ij}^{2r} = v_j$. 
				If $q = 2r$, append the move $v_i \reconf[\sfTJ]{G} v_j$ to $\calS$.
				Otherwise, do nothing.
				
				\item {\bf $u = x_{ij}^p$ and $v = x_{k\ell}^{q}$ for two edges $v_iv_j$ and $v_kv_\ell$ in $G$ and $0 \leq p, q \leq 2r$.}
				Note that if ($\star$)~$v_iv_j$ and $v_kv_\ell$ are adjacent edges in $G$ and either $u$ or $v$ is their common endpoint, we are back to {\bf Case~2}.
				Thus, let's assume ($\star$) does not happen and consider the \textit{first} move of this type.
				From Claim~\ref{clm:vertex-x}, it must happen that before the move $x_{ij}^p \reconf[\sfTJ]{G^\prime} x_{k\ell}^{q}$, some move in $\calS$ places a token on $u = x_{ij}^p$ and moreover such a token must come from either $v_i$ or $v_j$.
				Additionally, again by Claim~\ref{clm:vertex-x}, if the token comes from $v_i$ then $v_j$ contains no other token, and vice versa.
				Let $D$ be the D$r$DS obtained right before the move $x_{ij}^p \reconf[\sfTJ]{G^\prime} x_{k\ell}^{q}$.
				Now, since $x_{ij}^p \reconf[\sfTJ]{G^\prime} x_{k\ell}^{q}$ is the first move of its type, it follows that the path $P_{ij}$ contains exactly one token from $D$ which is placed on $x_{ij}^p$.
				However, this means we cannot perform $x_{ij}^p \reconf[\sfTJ]{G^\prime} x_{k\ell}^{q}$, otherwise $x_{ij}^r$ would not be $r$-dominated by the resulting token-set.
				Thus, this case does not happen unless ($\star$) is satisfied.
			\end{enumerate}
			To see that $\calS$ is indeed a $\sfTJ$-sequence in $G$, it suffices to show that if $C$ is the minimum VC obtained right before the move $v_i \reconf[\sfTJ]{G} v_j$ then $C^\prime = C - v_i + v_j$ is also a minimum VC of $G$.
			Suppose to the contrary that $C^\prime$ is not a vertex cover of $G$.
			It follows that there exists $k \in \{1, \dots, n\}$ such that $v_iv_k \in E(G)$, $v_k \neq v_j$, and $v_k \notin C$.
			Intuitively, the edge $v_iv_k$ is not covered by any vertex in $C^\prime$.
			On the other hand, let $D$ be the D$r$DS of $G^\prime$ obtained right before the move $x_{ij}^p \reconf[\sfTJ]{G^\prime} x_{ij}^{2r} = v_j$.
			Since $D^\prime = D - x_{ij}^{p} + x_{ij}^{2r}$ is also a D$r$DS of $G^\prime$, there must be some vertex in $D^\prime$ that $r$-dominates $x_{ik}^r$, which implies $V(P_{ik}) \cap D^\prime \neq \emptyset$.
			However, from the construction of $\calS$, it follows that $v_k \in C$, which is a contradiction.
			Thus, $C^\prime$ is a vertex cover of $G$.
			Since $|C^\prime| = |C|$, it is also minimum.
		\end{itemize}
	\end{proof}
	Our proof is complete.
\end{proof}

\subsection{Planar Graphs}
\label{sec:planar}
\begin{theorem}\label{thm:planar}
	\textsc{D$r$DSR} under $\sfR \in \{\sfTS, \sfTJ\}$ on planar graphs of maximum degree six and bounded bandwidth is $\ttPSPACE$-complete for any $r \geq 2$.
\end{theorem}
\begin{proof}
	We give a polynomial-time reduction from \textsc{Minimum Vertex Cover Reconfiguration (M-VCR)} on planar graphs of maximum degree three and bounded bandwidth, which was (implicitly) showed to be $\ttPSPACE$-complete in the works of Hearn and Demaine~\cite{HearnD05} and van der Zanden~\cite{Zanden15}\footnote{Hearn and Demaine~\cite{HearnD05} proved that \textsc{Independent Set Reconfiguration} is $\ttPSPACE$-complete on planar graphs of maximum degree three and later van der Zanden extended their results for those graphs with the additional ``bounded bandwidth'' restriction. Since their proofs only involve \textit{maximum independent sets}, the same results hold for \textit{minimum vertex covers}.}.
	Our reduction extends the classic reduction from \textsc{Vertex Cover} to \textsc{Dominating Set}~\cite{GareyJohson1979}.
	This reduction has also been modified for showing the hardness of the problem for $r = 1$ (i.e., \textsc{Dominating Set Reconfiguration}) by Haddadan~et~al.~\cite{HaddadanIMNOST16} under $\sfTAR$ and later by Bonamy~et~al.~\cite{BonamyDO21} under $\sfTS$.
	Let $(G, C_s, C_t)$ be an instance of \textsc{M-VCR} under $\sfR$ where $C_s, C_t$ are two minimum VCs of a planar graph $G$ of maximum degree three and bounded bandwidth.
	We will construct an instance $(G^\prime, D_s, D_t)$ of \textsc{D$r$DSR} under $\sfR$ where $D_s$ and $D_t$ are two D$r$DSs of a planar graph $G^\prime$ of maximum degree three and bounded bandwidth.
	
	Suppose that $V(G) = \{v_1, \dots, v_n\}$.
	We construct $G^\prime$ from $G$ as follows.
	For each edge $v_iv_j \in E(G)$, add a new path $P_{ij} = x_{ij}^0x_{ij}^1\dots, x_{ij}^{2r}$ of length $2r$ ($1 \leq i, j \leq n$) with $x_{ij}^0 = v_i$ and $x_{ij}^{2r} = v_j$.
	Observe that $x_{ij}^p = x_{ji}^{2r-p}$ for $0 \leq p \leq 2r$.
	Intuitively, $G^\prime$ is obtained from $G$ by replacing each edge of $G$ by a cycle $\mathcal{C}_{ij}$ of length $2r+1$ formed by the path $P_{ij}$ and the edge $uv = v_iv_j$.
	We define $D_s = C_s$ and $D_t = C_t$.
	Clearly, this construction can be done in polynomial time.
	(See \figurename~\ref{fig:planar}.)
	
	\begin{figure}[ht]
		\centering
		\includegraphics[width=0.7\textwidth]{figs/planar}
		\caption{An example of constructing $G^\prime$ from a planar, subcubic, and bounded bandwidth graph $G$ in the proof of Theorem~\ref{thm:planar}. Vertices in $V(G^\prime) - V(G)$ are marked with the gray color. Each dotted path is of length $r-1$.}
		\label{fig:planar}
	\end{figure}
	
	It follows directly from the construction that $G^\prime$ is planar and has maximum degree six and both $D_s$ and $D_t$ are D$r$DSs of $G^\prime$ (and so is any minimum VC of $G$).
	Moreover, $D_s$ and $D_t$, and in general all minimum VCs of $G$, are also minimum D$r$DSs of $G^\prime$.
	To see this, it suffices to prove that $\tau(G) = \gamma_r(G^\prime)$ where $\tau(G)$ and $\gamma_r(G^\prime)$ are respectively the size of a minimum VC of $G$ and a minimum D$r$DS of $G^\prime$.
	Since any minimum VC of $G$ is also a D$r$DS of $G^\prime$, we have $\tau(G) \geq \gamma_r(G^\prime)$.
	On the other hand, from the construction of $G^\prime$, observe that for any pair $i, j \in \{1, \dots, n\}$ with $v_iv_j \in E(G)$, the vertex $x_{ij}^r$ (whose distance from both $v_i$ and $v_j$ is exactly $r$) can only be $r$-dominated by some vertex in $V(\mathcal{C}_{ij})$, which implies that one needs at least $\tau(G)$ tokens to $r$-dominate $V(G^\prime)$.
	Therefore, $\gamma_r(G^\prime) \geq \tau(G)$.
	
	Since the number of edges of $G^\prime$ is exactly $(2r+1)$ times the number of edges of $G$, the bandwidth of $G^\prime$ increases (comparing to that of $G$) only by a constant multiplicative factor, which implies that $G^\prime$ is a bounded bandwidth graph.
	
	We are now ready to show the correctness of our reduction.
	\begin{claim}\label{clm:TS-planar}
		Under $\sfTS$, $(G, C_s, C_t)$  is a yes-instance if and only if $(G^\prime, D_s, D_t)$ is a yes-instance.
	\end{claim}
	\begin{proof}
		\begin{itemize}
			\item[($\Rightarrow$)] Let $\calS$ be a $\sfTS$-sequence in $G$ between $C_s$ and $C_t$.
			Since any minimum VC of $G$ is also a minimum D$r$DS of $G^\prime$, the sequence $\calS^\prime$ obtained by replacing each move $u \reconf[\sfTS]{G} v$ in $\calS$ by $u \reconf[\sfTS]{G^\prime} v$ is also a $\sfTS$-sequence in $G^\prime$ between $D_s = C_s$ and $D_t = C_t$.
			
			\item[($\Leftarrow$)] Let $\calS^\prime$ be a $\sfTS$-sequence in $G^\prime$ between $D_s$ and $D_t$.
			We construct a sequence of token-slides $\calS$ in $G$ between $C_s = D_s$ and $C_t = D_t$ as follows.
			Initially, $\calS = \emptyset$.
			For each move $u \reconf[\sfTS]{G^\prime} v$ in $\calS^\prime$, we consider the following cases.
			\begin{enumerate}[{\bf {Case} 1:}]
				\item {\bf $u \in V(G)$ and $v \in V(G)$.} It must happen that $u = v_i$ and $v = v_j$ for some $i, j \in \{1, \dots, n\}$ such that $v_iv_j \in E(G)$. We append $v_i \reconf[\sfTS]{G} v_j$ to $\calS$.
				
				\item {\bf $u \in V(G)$ and $v \in V(G^\prime) - V(G)$.} Do nothing.
				
				\item {\bf $u \in V(G^\prime) - V(G)$ and $v \in V(G)$.} It must happen that $u = x_{ij}^{2r-1}$ and $v = x_{ij}^{2r} = v_j$ for some $i, j \in \{1, \dots, n\}$ such that $v_iv_j \in E(G)$.
				We append $v_i \reconf[\sfTS]{G} v_j$ to $\calS$.
				
				\item {\bf $u \in V(G^\prime) - V(G)$ and $v \in V(G^\prime) - V(G)$.} Do nothing.
			\end{enumerate}
			
			To see that $\calS$ is indeed a $\sfTS$-sequence in $G$, it suffices to show that if $C$ is the minimum VC obtained right before the move $v_i \reconf[\sfTS]{G} v_j$ in $G$ then $C^\prime = C - v_i + v_j$ is also a minimum VC of $G$.
			If {\bf Case 1} happens, this is trivial.
			Thus, it remains to consider the case {\bf Case 3} happens.
			In this case, suppose to the contrary that $C^\prime$ is not a VC of $G$.
			It follows that there exists $k \in \{1, \dots, n\}$ such that $v_iv_k \in E(G)$, $v_k \neq v_j$, and $v_k \notin C$.
			Intuitively, the edge $v_iv_k$ is not covered by any vertex in $C^\prime$.
			On the other hand, let $D$ be the D$r$DS of $G^\prime$ obtained right before the move $x_{ij}^{2r-1} \reconf[\sfTS]{G^\prime} x_{ij}^{2r} = v_j$.
			Since $D^\prime = D - x_{ij}^{2r-1} + x_{ij}^{2r}$ is also a D$r$DS of $G^\prime$, there must be some vertex in $D^\prime$ that $r$-dominates $x_{ik}^r$, which implies $V(P_{ik}) \cap D^\prime \neq \emptyset$.
			However, from the construction of $\calS$, it follows that $v_k \in C$, which is a contradiction.
			Thus, $C^\prime$ is a vertex cover of $G$.
			Since $|C^\prime| = |C|$, it is also minimum.
		\end{itemize}
	\end{proof}
	
	\begin{claim}\label{clm:TJ-planar}
		Under $\sfTJ$, $(G, C_s, C_t)$  is a yes-instance if and only if $(G^\prime, D_s, D_t)$ is a yes-instance.
	\end{claim}
	\begin{proof}
		\begin{itemize}
			\item[($\Rightarrow$)] Let $\calS$ be a $\sfTJ$-sequence in $G$ between $C_s$ and $C_t$.
			Since any minimum VC of $G$ is also a minimum D$r$DS of $G^\prime$, the sequence $\calS^\prime$ obtained by replacing each move $u \reconf[\sfTJ]{G} v$ in $\calS$ by $u \reconf[\sfTJ]{G^\prime} v$ is also a $\sfTS$-sequence in $G^\prime$ between $D_s = C_s$ and $D_t = C_t$.
			
			\item[($\Leftarrow$)] Let $\calS^\prime$ be a $\sfTJ$-sequence in $G^\prime$ between $D_s$ and $D_t$.
			We construct a sequence of token-jumps $\calS$ in $G$ between $C_s = D_s$ and $C_t = D_t$ as follows.
			Initially, $\calS = \emptyset$.
			For each move $u \reconf[\sfTJ]{G^\prime} v$ in $\calS^\prime$, we consider the following cases.
			\begin{enumerate}[{\bf {Case} 1:}]
				\item {\bf $u \in V(G)$ and $v \in V(G)$.} It must happen that $u = v_i$ and $v = v_j$ for some $i, j \in \{1, \dots, n\}$. We append $v_i \reconf[\sfTJ]{G} v_j$ to $\calS$.
				
				\item {\bf $u \in V(G)$ and $v \in V(G^\prime) - V(G)$.} Do nothing.
				
				\item {\bf $u \in V(G^\prime) - V(G)$ and $v \in V(G)$.} From the construction of $G^\prime$, for each pair $i, j \in \{1, \dots, n\}$ such that $v_iv_j \in E(G)$, the vertex $x_{ij}^r$ must be $r$-dominated by at least one vertex of $\mathcal{C}_{ij}$.
				Additionally, note that any token-set resulting from a move in $\calS^\prime$ must be a minimum D$r$DS of $G^\prime$.
				Thus, we must have $u = x_{ij}^p$ and $v = x_{ij}^{2r} = v_j$ for some $i, j \in \{1, \dots, n\}$ such that $v_iv_j \in E(G)$ and $1 \leq p \leq 2r-1$.	
				Now, we append $v_i \reconf[\sfTS]{G} v_j$ to $\calS$.
				
				\item {\bf $u \in V(G^\prime) - V(G)$ and $v \in V(G^\prime) - V(G)$.} Do nothing.
			\end{enumerate}
			
			To see that $\calS$ is indeed a $\sfTJ$-sequence in $G$, it suffices to show that if $C$ is the minimum VC obtained right before the move $v_i \reconf[\sfTJ]{G} v_j$ in $G$ then $C^\prime = C - v_i + v_j$ is also a minimum VC of $G$.
			If {\bf Case 1} happens, this is trivial.
			Thus, it remains to consider the case {\bf Case 3} happens.
			In this case, suppose to the contrary that $C^\prime$ is not a VC of $G$.
			It follows that there exists $k \in \{1, \dots, n\}$ such that $v_iv_k \in E(G)$, $v_k \neq v_j$, and $v_k \notin C$.
			Intuitively, the edge $v_iv_k$ is not covered by any vertex in $C^\prime$.
			On the other hand, let $D$ be the D$r$DS of $G^\prime$ obtained right before the move $x_{ij}^{p} \reconf[\sfTJ]{G^\prime} x_{ij}^{2r} = v_j$, for $1 \leq p \leq 2r-1$.
			Since $D^\prime = D - x_{ij}^{p} + x_{ij}^{2r}$ is also a D$r$DS of $G^\prime$, there must be some vertex in $D^\prime$ that $r$-dominates $x_{ik}^r$, which implies $V(P_{ik}) \cap D^\prime \neq \emptyset$.
			However, from the construction of $\calS$, it follows that $v_k \in C$, which is a contradiction.
			Thus, $C^\prime$ is a vertex cover of $G$.
			Since $|C^\prime| = |C|$, it is also minimum.
		\end{itemize}
	\end{proof}
	Our proof is complete.
\end{proof}

\subsection{Chordal Graphs}
\label{sec:chordal}
\begin{theorem}\label{thm:chordal}
	\textsc{D$r$DSR} under $\sfR \in \{\sfTS, \sfTJ\}$ on chordal graphs is $\ttPSPACE$-complete for any $r \geq 2$.
\end{theorem}
\begin{proof}
	We give a polynomial-time reduction from \textsc{Minimum Vertex Cover Reconfiguration (M-VCR)} on general graphs, which was showed to be $\ttPSPACE$-complete by Ito~et~al.~\cite{ItoDHPSUU11}.
	Let $(G, C_s, C_t)$ be an instance of \textsc{M-VCR} under $\sfR$ where $C_s, C_t$ are two minimum VCs of a graph $G$.
	We will construct an instance $(G^\prime, D_s, D_t)$ of \textsc{D$r$DSR} under $\sfR$ where $D_s$ and $D_t$ are two D$r$DSs of a chordal graph $G^\prime$.
	
	Suppose that $V(G) = \{v_1, \dots, v_n\}$.
	We construct $G^\prime$ from $G$ as follows.
	\begin{itemize}
		\item Form a clique in $G^\prime$ having all vertices $v_1, \dots, v_n$ of $G$
		\item For each edge $v_iv_j \in E(G)$, add a corresponding new vertex $x_{ij}$ to $G^\prime$ and join it to both $v_i$ and $v_j$, where $i, j \in \{1, \dots, n\}$.
		Observe that $x_{ij} = x_{ji}$.
		Furthermore, attach to each $x_{ij}$ a new path $P_{ij}$ of length exactly $r-1$. 
		\item For each vertex $v_i \in V(G)$, add a corresponding new vertex $v^\prime_i$ to $G^\prime$ and join it to any $v_j$ satisfying that $\dist_G(v_i, v_j) \leq 1$.
		Furthermore, attach to each $v_i^\prime$ a new path $Q_i$ of length exactly $r-1$. 
	\end{itemize}
	We define $D_s = C_s$ and $D_t = C_t$. 
	Clearly, this construction can be done in polynomial time.
	(See \figurename~\ref{fig:chordal}.)
	
	\begin{figure}[ht]
		\centering
		\includegraphics[width=0.7\textwidth]{figs/chordal}
		\caption{An example of constructing $G^\prime$ from a graph $G$ in the proof of Theorem~\ref{thm:chordal}. Vertices in $V(G^\prime) - V(G)$ are marked with the gray color. Vertices in the yellow box form a clique. Each red path is of length exactly $r-1$.}
		\label{fig:chordal}
	\end{figure}
	
	It follows from the construction that $G^\prime$ is indeed a chordal graph.
	More precisely, if we define $H = (K \uplus S, F)$ to be the split graph with $K = \{v_1, \dots, v_n\}$ forming a clique and $S = \bigcup_{\{i, j \mid v_iv_j \in E(G)\}}\{x_{ij}\} \cup \bigcup_{i=1}^n\{v^\prime_i\}$ forming an independent set, then $G^\prime$ is obtained from $H$ by attaching paths to each member of $S$, which clearly results a chordal graph.
	Additionally, one can verify that any minimum vertex cover of $G$ is also a minimum dominating set of $H$ and therefore a minimum D$r$DS of $G^\prime$.
	(Recall that in a connected graph, any VC is also a DS.)
	
	\begin{claim}\label{clm:chordal}
		Under $\sfR \in \{\sfTS, \sfTJ\}$, $(G, C_s, C_t)$  is a yes-instance if and only if $(G^\prime, D_s, D_t)$ is a yes-instance.
	\end{claim} 
	\begin{proof}
		\begin{itemize}
			\item[($\Rightarrow$)] Let $\calS$ be a $\sfR$-sequence in $G$ between $C_s$ and $C_t$.
			Since any minimum VC of $G$ is also a minimum D$r$DS of $G^\prime$, the sequence $\calS^\prime$ obtained by replacing each move $u \reconf[\sfR]{G} v$ in $\calS$ by $u \reconf[\sfR]{G^\prime} v$ is also a $\sfR$-sequence in $G^\prime$ between $D_s = C_s$ and $D_t = C_t$.
			
			\item[($\Leftarrow$)] Let $\calS^\prime$ be a $\sfR$-sequence in $G^\prime$ between $D_s$ and $D_t$.
			From the construction of $G^\prime$, observe that no token can be moved to a vertex in $V(G^\prime) - V(G)$; otherwise some degree-$1$ endpoint of either a $P_{ij}$ or a $Q_\ell$ would not be $r$-dominated by the resulting token-set.
			Therefore, any move $u \reconf[\sfR]{G^\prime} v$ in $\calS^\prime$ satisfies that both $u$ and $v$ are in $V(G)$, and thus can be replaced by the move $u \reconf[\sfR]{G} v$ to construct $\calS$---our desired $\sfR$-sequence between $C_s = D_s$ and $C_t = D_t$ in $G$.
		\end{itemize}
	\end{proof}
\end{proof}

\section{Polynomial-Time Algorithms}
\label{sec:polytime}

\subsection{Graphs and Their Powers}
\label{sec:power-graph}

An extremely useful concept for studying distance-$r$ dominating sets is \textit{graph power}.
For a graph $G$ and an integer $s \geq 1$, the \textit{$s^{th}$ power of $G$} is the graph $G^s$ whose vertices are $V(G)$ and two vertices $u, v$ are adjacent in $G^s$ if $\dist_G(u, v) \leq s$. 
Observe that $D$ is a D$r$DS of $G$ if and only if $D$ is a DS of $G^{r}$.
The following proposition is straightforward.
\begin{proposition}\label{prop:power-graph}
	Let $\calG$ and $\calH$ be two graph classes and suppose that for every $G \in \calG$ we have $G^r \in \calH$ for some fixed integer $r \geq 1$.
	If \textsc{DSR} under $\sfTJ$ on $\calH$ can be solved in polynomial time, so does \textsc{D$r$DSR} under $\sfTJ$ on $\calG$. %
\end{proposition}
\begin{proof}
	Since $D$ is a D$r$DS of $G$ if and only if $D$ is a DS of $G^{r}$, any $\sfTJ$-sequence in $G$ between two D$r$DSs can be converted to a $\sfTJ$-sequence in $G^r$ between two corresponding DSs and vice versa.
\end{proof}

Recall that the power of any interval graph is also an interval graph~\cite{AgnarssonGH00,ChenC10} and 
\textsc{DSR} under $\sfTJ$ on interval graphs is in $\ttP$~\cite{HaddadanIMNOST16}.
Along with Proposition~\ref{prop:power-graph}, we immediately obtain the following corollary.
\begin{corollary}\label{cor:TJ-interval}
	\textsc{D$r$DSR} under $\sfTJ$ on interval graphs is in $\ttP$ for any $r \geq 1$.
\end{corollary}

\subsection{Graphs With Bounded Diameter Components}
\label{sec:bounded-diam}

\begin{proposition}\label{prop:bounded-diameter}
	Let $G$ be any graph such that there is some constant $c > 0$ satisfying $\diam(C_G) \leq c$ for any component $C_G$ of $G$.
	Then, \textsc{D$r$DSR} on $G$ under $\sfR \in \{\sfTS, \sfTJ\}$ is in $\ttP$ for every $r \geq c$.
\end{proposition}
\begin{proof}
	When $r \geq c$, any size-$1$ vertex subset of $G$ is a D$r$DS.
	In this case, observe that any token-jump (and therefore token-slide) from one vertex to any unoccupied vertex always results a new D$r$DS. 
	The problem becomes trivial: under $\sfTJ$, the answer is always ``yes''; under $\sfTS$, the answer depends on the number of tokens in each component.
\end{proof}

Since any connected cograph has diameter at most $2$, the following corollary is straightforward.
\begin{corollary}\label{cor:TJ-cograph}
	\textsc{D$r$DSR} under $\sfR \in \{\sfTS, \sfTJ\}$ on cographs is in $\ttP$ for any $r \geq 2$.
\end{corollary}

\subsection{Split Graphs}
\label{sec:split}

In this section, we assume that for any split graph $G$, the set $V(G)$ is partitioned into two subsets $K = \{v_1, \dots, v_p\}$ and $S = \{v_{p+1}, v_{p+2}, \dots, v_{p+q}\}$ which respectively induce a clique and an independent set of $G$.
For convenience, we write $G = (K \uplus S, E)$.

\begin{lemma}\label{lem:TS-split}
	Let $D$ be a D$2$DS of a split graph $G = (K \uplus S, E)$.
	\begin{enumerate}[(a)]
		\item For every pair $u \in D \cap K$ and $v \in K - D$, the set $D - u + v$ is a D$2$DS of $G$.
		\item For every pair $u \in D \cap S$ and $v \in K - D$, the set $D - u + v$ is a D$2$DS of $G$.
		\item For every pair $u \in D \cap K$ and $v \in S - D$, the set $D - u + v$ is a D$2$DS of $G$ if $(D \cap K) - u \neq \emptyset$.
	\end{enumerate}
\end{lemma}
\begin{proof}
	By definition, $\dist_G(v, w) \leq 2$ for any $w \in V(G)$.
	Consequently, in both (a) and (b), $\{v\}$ is a D$2$DS of $G$, and therefore so is $D - u + v \supseteq \{v\}$.
	In (c), since $(D \cap K) - u \neq \emptyset$, there must be a vertex $x \in D \cap K$ such that $x \neq u$.
	Again, since $\{x\}$ is a D$2$DS of $G$, so is $D - u + v \supseteq \{x\}$.
\end{proof}

\begin{theorem}\label{thm:TS-split}
	\textsc{D$r$DSR} under $\sfTS$ on split graphs is in $\ttP$ for any $r \geq 2$.
	In particular, when $r = 2$, for any pair of size-$k$ D$2$DSs $D_s, D_t$ of a split graph $G = (K \uplus S, E)$, there is a $\mathsf{TS}$-sequence in $G$ between $D_s$ and $D_t$.
\end{theorem}
\begin{proof}
	Proposition~\ref{prop:bounded-diameter} settles the case $r \geq 3$.
	It remains to consider the case $r = 2$.
	We claim that for any pair of size-$k$ D$2$DSs $D_s, D_t$ of a split graph $G = (K \uplus S, E)$, there is a $\mathsf{TS}$-sequence in $G$ between $D_s$ and $D_t$.
	Suppose that $p = |K| \geq 1$ and $q = |S| \geq 1$.
	
	We first show how to construct a size-$k$ D$2$DS $D^\star$ and then claim that for any size-$k$ D$2$DS $D$ of $G$, there exists a $\mathsf{TS}$-sequence between $D$ and $D^\star$.
	Suppose that vertices of $G$ are arranged as $v_1, v_2, \dots, v_p, v_{p+1}, v_{p+2}, \dots, v_{p+q}$ where $K = \{v_1, \dots, v_p\}$ and $S = \{v_{p+1}, \dots, v_{p+q}\}$.
	We take $D^\star = \{v_1, \dots, v_k\}$.
	
	We describe how to construct a $\sfTS$-sequence between $D$ and $D^\star$.
	Our construction is based on the observations described in Lemma~\ref{lem:TS-split}.
	Intuitively, in each iteration $i \in \{1, \dots, k\}$ of our algorithm, we will move a token in $D$ to $v_i$ and considered it as ``settled''.
	We note that once a vertex in $D$ is ``settled'' in some iteration, it always remains ``settled'' after each next iteration.
	For each $i \in \{1, \dots, k\}$,
	\begin{itemize}
		\item Assign $D^\prime \gets D - \{v_1, \dots, v_{i-1}\}$. Note that if $i = 1$ then $D^\prime = D$.
		Intuitively, $D^\prime$ contains ``unsettled'' vertices in $D$ at that time. 
		\item Let $v^i \in D^\prime$ be such that $\dist_G(v^i, v_i) = \min_{x \in D^\prime}\dist_G(x, v_i)$.
		\begin{itemize}
			\item If $v_i \in K$ and $v^i \in K$, by Lemma~\ref{lem:TS-split}, we can directly slide the token on $v^i$ to $v_i$ and update $D \gets D - v^i + v_i$.
			
			\item If $v_i \in K$ and $v^i \in S$, let $P_i$ be a shortest $v^iv_i$-path in $G$.
			Observe that $V(P_i) \cap D = \{v^i\}$ and $P_i$ is of length at most $2$.
			If $P_i$ is of length $1$ (i.e., $P = v^iv_i$), again by Lemma~\ref{lem:TS-split}, we can directly slide the token on $v^i$ to $v_i$ and update $D \gets D - v^i + v_i$.
			If $P_i$ is of length $2$, $v^i$ and $v_i$ must have a common neighbor $w \in K$.
			Since $V(P_i) \cap D = \{v^i\}$, we have $w \notin D$.
			By Lemma~\ref{lem:TS-split}, we can immediately slide the token on $v^i$ to $w$ and then from $w$ to $v_i$, and update $D \gets (D - v^i + w) - w + v_i$.
			
			\item If $v_i \in S$, we must have $i > p = |K|$ (i.e., all vertices in $K$ are already filled with tokens) and therefore $v^i \in S$ (since all vertices in $K$ are already ``settled'').
			Thus, a shortest $v^iv_i$-path $P_i$ must be of length either $2$ or $3$ and all of its vertices except $v_i$ contain tokens (i.e., they are in $D$).
			If $P_i$ is of length $2$, $v^i$ and $v_i$ must have a common neighbor $w \in K$ and since $i > p$ we also have $w \in D$.
			Thus, by Lemma~\ref{lem:TS-split}, we can directly slide the token on $w$ to $v_i$ and then from $v^i$ to $w$ and update $D \gets (D - w + v_i) - v^i + w$.
			If $P_i$ is of length $3$, let $P_i = v^ixyv_i$ and we must have $\{x, y\} \subseteq K$ and since $i > p$ we also have $\{x, y\} \subseteq D$.
			Again, by Lemma~\ref{lem:TS-split}, we can directly slide the token on $y$ to $v_i$, then the token on $x$ to $y$, and then the token on $v^i$ to $x$, and update $D \gets ((D - y + v_i) - x + y) - v^i + x$.
		\end{itemize}
	\end{itemize}
\end{proof}

Since any $\sfTS$-sequence in $G$ is also a $\sfTJ$-sequence, a direct consequence of Theorem~\ref{thm:TS-split} and Proposition~\ref{prop:bounded-diameter} is as follows.
\begin{corollary}\label{cor:TJ-split}
	\textsc{D$r$DSR} under $\sfTJ$ on split graphs is in $\ttP$ for any $r \geq 2$.
\end{corollary}

We now consider \textit{shortest} reconfiguration sequences in split graphs when $r = 2$.
Observe that each $\sfR$-sequence ($\sfR \in \{\sfTS, \sfTJ\}$) between two D$r$DSs $D_s, D_t$ induces a bijection $f$ between them: the token on $u \in D_s$ must finally be placed on $f(u) \in D_t$ and vice versa.

\begin{theorem}\label{thm:TS-split-shortest}
	Suppose that $D_s = \{s_1, \dots, s_k\}$ and $D_t = \{t_1, \dots, t_k\}$ are two size-$k$ D$2$DSs of a split graph $G = (K \uplus S, E)$.
	Let $M^{*}_{\sfTS} = \min_{f} \sum_{i=1}^{k} \dist_G(s_i,f(s_i))$ where $f$ is a bijection between vertices of $D_s$ and $D_t$. 
	Then,
	\begin{enumerate}[(a)]
		\item $\opt_{\sfTS}(G, D_s, D_t) \geq M^*_{\sfTS}$.
		\item There exists a graph $G$ and two size-$k$ D$2$DSs $D_s, D_t$ of $G$ such that $\displaystyle\opt_{\sfTS}(G, D_s, D_t) = M^*_{\sfTS} + 1$.
	\end{enumerate}
\end{theorem}
\begin{proof}
	\begin{enumerate}[(a)]
		\item In order to slide a token from $s_i \in D_s$ to $f(s_i) \in D_t$ for some $i \in \{1, \dots, k\}$, one cannot use less than $\dist_G(s_i, f(s_i))$ token-slides.
		\item We construct a split graph $G = (K \uplus S, E)$ as follows. 
		(See \figurename~\ref{fig:TS-split}.)
		\begin{itemize}
			\item $K$ contains $3$ vertices labelled $s_1, a, b$. Vertices of $K$ form a clique in $G$.
			\item $S$ contains $2k + 1$ vertices labelled $s_2, \dots, s_k, t_1, \dots, t_k, c, d$.
			Vertices of $S$ form an independent set in $G$.
			\item Join $s_1$ to $s_2, \dots, s_k, t_1, \dots, t_k$. Join $a$ to $c$ and $t_1$. Join $b$ to $t_2, \dots, t_k$ and $d$.
		\end{itemize}
		Let $D_s = \{s_1, \dots, s_k\}$ and $D_t = \{t_1, \dots, t_k\}$.
		\begin{figure}[ht]
			\centering
			\includegraphics[width=0.6\textwidth]{figs/TS-split}
			\caption{Construction of a split graph $G = (K \uplus S, E)$ satisfying Theorem~\ref{thm:TS-split-shortest}(b).}
			\label{fig:TS-split}
		\end{figure}
		
		One can readily verify that both $D_s$ and $D_t$ are D$2$DSs of $G$.
		Observe that $M^\star_{\sfTS} = \sum_{i=1}^{k}\dist_G(s_i, t_i) = 2(k-1) + 1 = 2k - 1$.
		Additionally, any $\sfTS$-sequence $\calS$ in $G$ between $D_s$ and $D_t$ must begin with sliding the token on $s_1$ to one of its unoccupied neighbors.
		A $\sfTS$-sequence of length exactly $M^\star_{\sfTS}$ must slide $s_1$ to one of $t_1, \dots, t_k$ but that is not possible, otherwise either $c$ (if sliding to $t_2, \dots, t_k$) or $d$ (if sliding to $t_1$) would not be $2$-dominated by the resulting token-set.  
		Thus, such a sequence does not exist.
		
		On the other hand, a $\sfTS$-sequence of length exactly $M^\star_{\sfTS} + 1$ can be constructed: first sliding the token on $s_1$ to $a$, then sliding the token on $s_i$ to $t_i$ along the $s_it_i$-path for $2 \leq i \leq k$, and finally sliding the token on $a$ to $t_1$.
		Since a token is always placed on $a \in K$ (whose distance to any other vertex is at most $2$) after the first token-slide and before the final one, the above sequence of token-slides is indeed a $\sfTS$-sequence in $G$.
	\end{enumerate}
\end{proof}

\begin{theorem}\label{thm:TJ-split-shortest}
	Suppose that $D_s = \{s_1, \dots, s_k\}$ and $D_t = \{t_1, \dots, t_k\}$ are two size-$k$ D$2$DSs of a split graph $G = (K \uplus S, E)$.
	Let $\displaystyle M^{*}_{\sfTJ} = \frac{|D_s \Delta D_t|}{2}$. 
	Then,
	\begin{enumerate}[(a)]
		\item $\opt_{\sfTJ}(G, D_s, D_t) \geq M^*_{\sfTJ}$.
		\item There exists a graph $G$ and two size-$k$ D$2$DSs $D_s, D_t$ of $G$ such that $\displaystyle\opt_{\sfTS}(G, D_s, D_t) = M^*_{\sfTJ} + 1$.
	\end{enumerate}
\end{theorem}
\begin{proof}
	\begin{enumerate}[(a)]
		\item Trivial.
		\item We construct a split graph $G = (K \uplus S, E)$ as follows. 
		\begin{itemize}
			\item $K$ contains $2k$ vertices labelled $v_1, \dots, v_k, w_1, \dots, w_k$.
			Vertices in $K$ form a clique in $G$.
			
			\item $S$ contains $k^2 + k$ vertices labelled $v_{11}, \dots, v_{1k}, v_{21}, \dots, v_{2k}, \dots, v_{k1}, \dots, v_{kk}, w_{11}, \dots, w_{k1}$.
			Vertices in $S$ form an independent set in $G$.
			
			\item For each fixed $i \in \{1, \dots, k\}$, 
            \begin{itemize}
                \item for $j \in \{1, \dots, k\}$,
                \begin{itemize}
                    \item join $v_i$ to every $v_{ij}$,
                    \item join $w_i$ to every $v_{ji}$,
                \end{itemize}
                \item join $w_i$ to $w_{i1}$.
            \end{itemize}
		\end{itemize}
		Let $D_s = \bigcup_{i=1}^k\{v_{ii}\}$ and $D_t = \bigcup_{i=1}^k\{w_{i1}\}$.
		\begin{figure}[ht]
			\centering
			\includegraphics[width=0.8\textwidth]{figs/TJ-split}
			\caption{Construction of a split graph $G = (K \uplus S, E)$ satisfying Theorem~\ref{thm:TJ-split-shortest}(b). Vertices in the yellow box are in $K$.}
			\label{fig:TJ-split}
		\end{figure}
		
		To see that $D_s$ is a D$2$DS of $G$, note that each $v_{ii}$ $2$-dominates every vertex in $K \cup \bigcup_{j=1}^k\{v_{ij}\} \cup \{w_{ii}\}$.
		To see that $D_t$ is a D$2$DS of $G$, note that each $w_{i1}$ $2$-dominates every vertex in $K \cup \bigcup_{j=1}^k\{v_{ji}\}$.
		Moreover, any $\sfTJ$-sequence of length exactly $M^\star_{\sfTJ}$ must begin with a direct token-jump from some $v_{ii}$ to some $w_{j1}$ for some $1 \leq i, j \leq k$ but that is not possible, otherwise no vertices in $\bigcup_{\ell=1}^k\{v_{i\ell}\} - v_{ij}$ are $2$-dominated by the resulting token-set.
		
		On the other hand, a $\sfTJ$-sequence of length exactly $M^\star_{\sfTJ} + 1$ can be constructed: first jumping the token on $v_{11}$ to $v_1$, then for $2 \leq i \leq k$, directly jumping the token on $v_{ii}$ to $w_{i1}$, and finally jumping the token on $v_1$ to $w_{11}$.
		As before, since a token is always placed at $v_1 \in K$ after the first token-jump and before the final one, the above sequence of token-jumps is indeed a $\sfTJ$-sequence in $G$.
	\end{enumerate}
\end{proof}

\subsection{Trees}
\label{sec:trees}

\begin{theorem}\label{thm:TJ-trees}
	\textsc{D$r$DSR} under $\sfTJ$ on trees is in $\ttP$ for any $r \geq 2$.
\end{theorem}

To prove this theorem, we extend the idea of Haddadan~et~al.~\cite{HaddadanIMNOST16} for $r = 1$ under $\sfTAR$ and the linear-time algorithm of Kundu and Majumder~\cite{KunduM16} for finding a minimum D$r$DS on trees.
In particular, we employ a simpler implementation of Kundu and Majumder's algorithm presented by Abu-Affash, Carmi, and Krasin~\cite{Abu-AffashCK22}.
More precisely, based on the minimum D$r$DS $D^\star$ obtained from the implementation of Abu-Affash, Carmi, and Krasin, we construct a partition $\mathbb{P}(T)$ of $T$ consisting of $\gamma_r(T)$ vertex-disjoint subtrees, each of which contains exactly one vertex of $D^\star$.
(Haddadan~et~al. called such a set $D^\star$ a \textit{canonical} dominating set.)
For convenience, we denote by $C_x$ the member of $\mathbb{P}(T)$ whose intersection with $D^\star$ is the vertex $x$.
We claim that $\mathbb{P}(T)$ satisfies the following property: for any D$r$DS $D$ of $G$, each member of $\mathbb{P}(T)$ contains at least one vertex in $D$.
Using this property, it is not hard to design a linear-time algorithm for constructing a $\sfTJ$-sequence between any pair of size-$k$ D$r$DSs $D_s, D_t$ of $G$.
The key idea is one can transform both $D_s$ and $D_t$ into some D$r$DS $D$ that contains $D^\star$.
For instance, to transform $D_s$ into $D$, for each subtree $C_x \in \mathbb{P}(T)$ for $x \in D^\star$, we move any token in $D_s \cap V(C_x)$ to $x$.
If we handle each subtree $C_x$ based on the order of subtrees added to $\mathbb{P}(T)$ in our modified implementation, such a transformation will form a $\sfTJ$-sequence in $T$.
After this procedure, we obtain a set of tokens $D^\prime$ that contains $D^\star$ and since $D^\star$ is a minimum D$r$DS of $G$, transforming $D^\prime$ into $D$ under $\sfTJ$ can now be done easily: until there are no tokens to move, repeatedly take a token in $D^\prime - D$, move it to some vertex in $D - D^\prime$, and update both $D$ and $D^\prime$.

We now define some notations and, for the sake of completeness, describe the algorithm of Abu-Affash, Carmi, and Krasin~\cite{Abu-AffashCK22}.
In a graph $G$, for a vertex subset $D \subseteq V(G)$ and a vertex $u \in V(G)$, we define $\delta_D(u) = \min_{v \in D}\dist_G(u, v)$ and call it the \textit{distance} between $u$ and $D$.
Observe that a vertex $u$ is $r$-dominated by $D$ if $\delta_D(u) \leq r$ and therefore $D$ is a D$r$DS of $G$ if for every $u \in V(G)$ we have $\delta_D(u) \leq r$.
For a $n$-vertex tree $T$, let $T_u$ be the \textit{rooted form} of $T$ when regarding the vertex $u \in V(T)$ as the root.
For each $v \in V(T_u)$, we denote by $T_v$ the subtree of $T_u$ rooted at $v$.
In other words, $T_v$ is the subtree of $T_u$ induced by $v$ and its descendants.
We also define $h(T_v) = \max_{w \in V(T_v)}\dist_{T_u}(v, w)$ and call it the \textit{height} of $T_v$.
In other words, $h(T_v)$ is the largest distance from $v$ to a vertex in $T_v$.
The set of children of $v$ in $T_u$ is denoted by $\textit{child}(v)$.

The algorithm is described in Algorithm~\ref{algo:minDrDS}.
In short, in each iteration, it finds a subtree $T_v$ of height exactly $r$, adds $v$ to $D^\star$, and removes all the leaves of $T_u$ that are in $N^r_{T_u}[v]$.
To implement the algorithm in $O(n)$ time, a modified version of the depth-first search (DFS) algorithm was used in~\cite{Abu-AffashCK22}
(Function $\mathtt{ModifiedDFS}$ in Algorithm~\ref{algo:minDrDS}).
The procedure $\mathtt{ModifiedDFS}$ visits the vertices of $T_u$ starting from the root $u$ and recursively visits each of its children, which means vertices in $D^\star$ would be added in a ``bottom-up'' fashion.
In each recursive call $\mathtt{ModifiedDFS}(v)$, if $h(T_v) = r$ then $v$ is added to $D^\star$ and since all vertices in $T_v$ is $r$-dominated by $v$, we remove them from $T_u$ and return $\delta_{D^\star}(v) = 0$.
Otherwise ($h(T_v) \neq r$), we call $\mathtt{ModifiedDFS}(w)$ for each child $w$ of $v$, and we update $\delta_{D^\star}(v)$ and $h(T_v)$ according to these calls.
When these calls return, we have $h(T_v) \leq r$.
Then we check whether $\delta_{D^\star}(v) + h(T_v) \leq r$.
If so (which means the current $D^\star$ $r$-dominates $v$ and all its descendants in the original rooted tree $T_u$), we remove all the vertices of $T_v$ from $T_r$ and return $\delta_{D^\star}(v)$.
Otherwise ($\delta_{D^\star}(v) + h(T_v) > r$), we check again whether $h(T_v) = r$ (in case the descendants reduced the height of $T_v$ to $r$).
If so, we add $v$ to $D^\star$, remove all the vertices of $T_v$ from $T_u$, and return $\delta_{D^\star}(v) = 0$.
Otherwise ($h(T_v) < r$), we return $\infty$.
Finally, when $\delta_{D^\star}(u) = \infty$, we add $u$ to $D^\star$.


\begin{algorithm}[ht]
	\KwIn{A tree $T_u$ rooted at $u$.}
	\KwOut{A minimum distance-$r$ dominating set $D^\star$ of $T_u$.}
	\SetArgSty{textbb} 
	\DontPrintSemicolon
	
	$D^\star \gets \emptyset$\;
	\For{each $v \in V(T_u)$}{
		compute $h(T_v)$\;
		$\delta_{D^\star}(v) \gets \infty$\;
	}
	$\delta_{D^\star}(u) \gets \mathtt{ModifiedDFS}(u)$\;
	\If{$\delta_{D^\star}(u) = \infty$}{
		$D^\star \gets D^\star + u$\;
	}
	\Return{$D^\star$}
	
	\BlankLine
	
	\SetKwFunction{MDFS}{ModifiedDFS}
	\SetKwProg{Fn}{Function}{:}{}
	
	\Fn{\MDFS{$v$}}{
		\If{$h(T_v) = r$}{
			$D^\star \gets D^\star + v$\;
			$T_u \gets T_u - T_v$\;
			$h(T_v) \gets -1$\;
			$\delta_{D^\star}(v) \gets 0$\;
		}
		\Else(\tcp*[f]{$h(T_v) > r$}){
			\For{each $w \in \textit{child}(v)$}{
				\If{$h(T_w) \geq r$}{
					$\delta_{D^\star}(v) \gets \min\{\delta_{D^\star}(v), \mathtt{ModifiedDFS}(w) + 1\}$
				}
			}
			$h(T_v) \gets \max\{h(T_w) + 1: w \in \textit{child}(v)\}$\tcp*{updating $h(T_v)$}
			\If{$h(T_v) + \delta_{D^\star}(v) \leq r$}{
				$T_u \gets T_u - T_v$\;
				$h(T_v) \gets -1$\;
			}
			\If{$h(T_v) = r$}{
				$D^\star \gets D^\star + v$\;
				$T_u \gets T_u - T_v$\;
				$h(T_v) \gets -1$\;
				$\delta_{D^\star}(v) \gets 0$\;
			}
			\Else{
				$\delta_{D^\star}(v) \gets \infty$\;
			}
		}
		
		\Return{$\delta_{D^\star}(v)$}
	}
	
	\caption{$\mathtt{MinDrDSTree}(T_u)$}
	\label{algo:minDrDS}
\end{algorithm}

To illustrate Algorithm~\ref{algo:minDrDS}, we consider the example from~\cite{Abu-AffashCK22} for $r = 2$ with the tree $T_u$ rooted at $u = 1$ as described in \figurename~\ref{fig:TJ-trees}.
The first vertex added to $D^\star$ is $7$ in $\mathtt{ModifiedDFS}(7)$, since $h(T_7) = 2$.
In this call, we remove $T_7$ from $T_u$, update $h(T_7) = -1$ and return $\delta_{D^\star}(7) = 0$ to $\mathtt{ModifiedDFS}(4)$.
In $\mathtt{ModifiedDFS}(4)$, we update $\delta_{D^\star}(4) = 1$ and, after traversing vertices $6$ and $8$, $h(T_4) = 1$, and since $h(T_4) + \delta_{D^\star}(4) = 2 = r$ and $7$ is the latest vertex added to $D^\star$, we remove $T_4$ from $T_u$ and return $\delta_{D^\star}(4) = 1$ to $\mathtt{ModifiedDFS}(2)$.
In $\mathtt{ModifiedDFS}(2)$, since $h(T_2) = 3 > r$, we call $\mathtt{ModifiedDFS}(5)$ which adds $5$ to $D^\star$, removes $T_5$ from $T_u$, and returns $\delta_{D^\star}(5) = 0$.
Then, we update $\delta_{D^\star}(2) = 1$ and $h(T_2) = 0$, and since $h(T_2) + \delta_{D^\star}(2) = 1 < r$ and $5$ is the latest vertex added to $D^\star$, we remove $T_2$ from $T_u$, and return $\delta_{D^\star}(2) = 1$ to $\mathtt{ModifiedDFS}(1)$.
In $\mathtt{ModifiedDFS}(1)$, since $\delta_{D^\star}(1) = 2$ and, after traversing $3$, $h(T_1) = 1$, we return $\delta_{D^\star}(1) = \infty$ to Algorithm~\ref{algo:minDrDS}, and therefore, we add $1$ to $D^\star$.

\begin{figure}[ht]
	\centering
	\includegraphics[width=0.35\textwidth]{figs/TJ-trees}
	\caption{A tree $T_u$ rooted at $u = 1$. For $r = 2$, Algorithm~\ref{algo:minDrDS} returns $D^\star = \{7, 5, 1\}$. A partition $\mathbb{P}(T_u) = \{C_7, C_5, C_1\}$ of $T_u$ is also constructed.}
	\label{fig:TJ-trees}
\end{figure}

We now describe how to construct our desired partition $\mathbb{P}(T_u)$.
Recall that $\mathbb{P}(T_u)$ is nothing but a collection of vertex-disjoint subtrees whose union is the original tree $T_u$.
Suppose that $D^\star$ is the minimum D$r$DS of $T_u$ obtained from Algorithm~\ref{algo:minDrDS} and furthermore assume that vertices of $D^\star$ are ordered by the time they were added to $D^\star$.
For each $v \in D^\star$, we define $C_v$ (the unique member of $\mathbb{P}(T_u)$ containing $v$) as $T_v$ (the subtree of $T_u$ rooted at $v$) and then delete $T_v$ from $T_u$.
\figurename~\ref{fig:TJ-trees} illustrates how to construct $\mathbb{P}(T_u)$ in the above example.
From the construction, it is clear that each member of $\mathbb{P}(T_u)$ contains exactly one vertex from $D^\star$.
We say that two subtrees $C_x, C_y$ in $\mathbb{P}(T_u)$ are \textit{adjacent} if there exists $v \in V(C_x)$ and $w \in V(C_y)$ such that $vw \in E(T_u)$.
If a subtree contains the root $u$ then we call it the \textit{root subtree}.
Otherwise, if a subtree has exactly one adjacent subtree then we call it a \textit{leaf subtree} and otherwise an \textit{internal subtree}.

We now claim that the constructed partition $\mathbb{P}(T_u)$ satisfies the following property.
\begin{lemma}\label{lem:partition-tree}
    Let $D$ be any D$r$DS of $T_u$.
	Then, $D \cap V(C_v) \neq \emptyset$ holds for every $v \in D^\star$. 
\end{lemma}
\begin{proof}
We claim that for each $v \in D^\star$, one can find a vertex $v^\prime \in V(C_v)$ such that $N^r_T[v^\prime] \subseteq V(C_v)$.
For each $v \in D^\star$, let $D^\star_v$ be the set of all vertices added to $D^\star$ before $v$.

If $C_v$ is a leaf subtree, we take any leaf in $C_v$ of distance exactly $r$ from $v$ and regard it as $v^\prime$.
Clearly, $v^\prime$ is also a leaf of $T_u$ and is not $r$-dominated by any vertex outside $C_v$, i.e., $N^r_T[v^\prime] \subseteq V(C_v)$.

If $C_v$ is an internal subtree, we describe how to find our desired $v^\prime$.
From Algorithm~\ref{algo:minDrDS}, since $v$ is the next vertex added to $D^\star_v$, it follows that there must be some vertex in $V(C_v)$ not $r$-dominated by any member of $D^\star_v$; we take $v^\prime$ to be the one having maximum distance from $v$ among all those vertices.
By definition, $v^\prime$ is clearly not $r$-dominated by any vertex in a $C_w$ where $w \in D^\star_v$.
Since $C_v$ is an internal subtree, by Algorithm~\ref{algo:minDrDS}, the distance between $v$ and $v^\prime$ must be exactly $r$ and therefore no vertex in a $C_w$, where $w \in D^\star - D^\star_v - v$, $r$-dominates $v$.
(Recall that by Algorithm~\ref{algo:minDrDS}, since $v$ is added to $D^\star_v$, the current subtree $T_v$ must have height exactly $r$.)
Thus, $N_T^r[v^\prime] \subseteq V(C_v)$.

If $C_v$ is the root subtree, again we can choose $v^\prime$ using exactly the same strategy as in the case for internal subtrees.
The main difference here is that, by Algorithm~\ref{algo:minDrDS}, the distance between $v^\prime$ and $v$ may not be exactly $r$.
However, since $C_v$ contains the root $u$, $v$ is the last vertex added to $D^\star$.
(Intuitively, this means $C_v$ has no ``parent subtree'' above it.)
Therefore, in order to show that $N_T^r[v^\prime] \subseteq V(C_v)$, it suffices to show that no vertex in a $C_w$, where $w \in D^\star_v$, $r$-dominates $v^\prime$.
Indeed, this clearly holds by definition of $v^\prime$.

We are now ready to prove the lemma.
Suppose to the contrary that there exists $v \in D^\star$ such that $D \cap V(C_v) = \emptyset$.
Then, $D$ does not $r$-dominate $v^\prime$---a vertex in $C_v$ with $N_T^r[v^\prime] \subseteq V(C_v)$.
This contradicts the assumption that $D$ is a D$r$DS.
Our proof is complete.
\end{proof}

The following lemma is crucial in proving Theorem~\ref{thm:TJ-trees}.
\begin{lemma}\label{lem:TJ-trees}
	Let $D$ be an arbitrary D$r$DS of $T_u$.
	Let $D^\prime$ be any D$r$DS of $T_u$ that contains $D^\star$, i.e., $D^\star \subseteq D^\prime$.
	Then, in $O(n)$ time, one can construct a $\sfTJ$-sequence $\calS$ in $T_u$ between $D$ and $D^\prime$.
\end{lemma}
\begin{proof}

	We construct $\calS$ as follows. 
	Initially, $\calS = \emptyset$.
	\begin{enumerate}[{\bf {Step} 1:}]
		\item For each $v \in D^\star$, let $x$ be any vertex in $D \cap V(C_v)$.
		From Lemma~\ref{lem:partition-tree}, such a vertex $x$ exists.
		We append $x \reconf[\sfTJ]{T_u} v$ to $\calS$ and assign $D \gets D - x + v$.
		(After this step, clearly $D^\star \subseteq D \cap D^\prime$.)
		
		\item Let $x \in D - D^\prime$ and $y \in D^\prime - D$. 
		We append $x \reconf[\sfTJ]{T_u} y$ to $\calS$ and assign $D \gets D - x + y$.
		Repeat this step until $D = D^\prime$.
	\end{enumerate}
	
	For each $v \in D^\star$, let $D^\star_v$ be the set of all vertices added to $D^\star$ before $v$.
	Since any vertex $r$-dominated by $x$ and not in $C_v$ is $r$-dominated by either $v$ or a member of $D^\star_v$, any move performed in \textbf{Step~1} results a new D$r$DS of $T_u$.
	Note that after {\bf Step~1}, $D^\star \subseteq D \cap D^\prime$.
	Thus, any move performed in {\bf Step~2} results a new D$r$DS of $T_u$.
	In short, $\calS$ is indeed a $\sfTJ$-sequence in $T_u$.
	In the above construction, as we ``touch'' each vertex in $D$ at most once, the running time is indeed $O(n)$.
\end{proof}

Using Lemma~\ref{lem:TJ-trees}, it is not hard to prove Theorem~\ref{thm:TJ-trees}.
More precisely, let $(T, D_s, D_t)$ be an instance of \textsc{D$r$DSR} under $\sfTJ$ where $D_s$ and $D_t$ are two D$r$DSs of a tree $T$.
By Lemma~\ref{lem:TJ-trees}, one can immediately decide if $(T, D_s, D_t)$ is a yes-instance by comparing the sizes of $D_s$ and $D_t$: if they are of the same size then the answer is ``yes'' and otherwise it is ``no''.
Moreover, in a yes-instance, Lemma~\ref{lem:TJ-trees} allows us to construct in linear time a $\sfTJ$-sequence (which is not necessarily a shortest one) between $D_s$ and $D_t$.


\section{Concluding Remarks}
\label{sec:conclude}

In this paper, we provide an initial picture of the computational complexity of \textsc{D$r$DSR} ($r \geq 2$) under $\sfTS$ and $\sfTJ$ on different graph classes.
We extended several known results for $r = 1$ and provided a complexity dichotomy of \textsc{D$r$DSR} on split graphs: the problem is $\ttPSPACE$-complete for $r = 1$ but can be solved in polynomial time for $r \geq 2$.
The following questions remain unsolved:
\begin{enumerate}[{\bf {Question} 1:}]
	\item What is the complexity of \textsc{D$r$DSR} ($r \geq 2$) under $\sfTS$ on trees?
	\item What is the complexity of \textsc{D$r$DSR} ($r \geq 2$) under $\sfTS$ on interval graphs?
\end{enumerate}

\section*{Acknowledgment}
Niranka Banerjee is funded by JSPS KAKENHI Grant Number
JP20H05967 and Duc A. Hoang is funded by University of Science, Vietnam National University, Hanoi under project number TN.23.04.

\printbibliography
\end{document}
