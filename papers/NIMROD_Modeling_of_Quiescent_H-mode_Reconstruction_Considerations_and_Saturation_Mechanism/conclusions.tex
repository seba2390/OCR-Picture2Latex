\section{Discussion and Conclusions} 
\label{sec:conclusions}

With regard to the saturation mechanism of the unstable modes, there are two
avenues to saturation from a spectral energy perspective: (1) The unstable
perturbations can directly modify the mean fields and eliminate the source of
free energy by relaxing the profile gradients; or (2) the perturbations can
couple to stable modes that dissipate the energy (again through modification of
the mean fields or through energy flow to the boundary of the domain). 
Figure~\ref{fig:profs} shows that the perturbations in this case make
non-trivial modifications to the $n_\phi=0$ fields that relax the pressure
gradient and current profile leading to saturation. 

One complication to this picture arises when considering the steady-state
nature of the initial state from the reconstructed fields. As mentioned in
Sec.~\ref{sec:reconstruct}, these fields are assumed to be time independent
given the presence of sources, sinks and fluxes that are outside the scope of
our modeling. For a discharge state with broadband MHD activity, the
contribution of the flux from the MHD perturbations is included in the
steady-state assumption. In this sense, our modeling of the transport from the
MHD perturbations constitutes a `double counting' of this flux. For studies
that compare the level of this flux to experiment there are two approaches to
resolve this inconsistency: (1) The initial profiles could differ from the
experiment and be more unstable such that the MHD perturbations relax the
profiles to a state that resembles the reconstructed profiles; or (2) the
$n_\phi=0$ modifications from the MHD perturbations could be cancelled (or
ignored) such that the final profiles match the reconstructed values. This
first approach suffers from the difficulty of finding the more unstable state,
a priori, that relaxes to the reconstructed state.  The second approach is
the traditional way turbulent flux calculations are performed and is of interest
for future studies. As this approach eliminates mode saturation through
modification of the mean fields, the stability of the mode spectrum becomes
critical. In particular, it is likely that simulations must be performed with
an extended-MHD model that includes two-fluid, first-order
finite-Larmour-radius effects that stabilize the intermediate-$n_\phi$ modes.

Our simulations produce an MHD turbulent-like state, which is a good candidate
to at least partially explain the broadband-MHD phenomena. However, additional
comparisons to experimental data are required to confidently claim these
simulations truly model the discharge dynamics. Prior attempts to compare with
magnetic probe data proved unsuccessful as the probe measurement temporal
resolution (200 kHz) is approximately two orders of magnitude smaller
than our nonlinear simulation time period. Higher time-resolution measurements
that make local measurements of the perturbations (e.g. beam-emission
spectroscopy and Doppler reflectometry) are a more promising avenue to pursue
validation and will be the subject of future studies.

% \begin{itemize}
% \item
% n=0 transport in the context of reconstruction as steady state
% \item
% need more comparisons with experiment
% \item
% magnetic comparisons are limited by exp. time resolution vs. simulation time
% \item
% higher time resolution measurements (e.g. BES and doppler reflectometry) are a
% more promising avenue to persue validation
% \item
% first simulations show MHD turbulent state - good canidate for broadband-MHD
% \end{itemize}

