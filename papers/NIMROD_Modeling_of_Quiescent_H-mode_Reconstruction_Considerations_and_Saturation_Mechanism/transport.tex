\section{Transport induced by the MHD perturbations}
\label{sec:transport}

\begin{figure}
  \centering
  \includegraphics[width=8cm]{fieldline.png}
  \vspace{-4mm}
  \caption{[Color online]
  Poincaré plot of magnetic fieldlines at t=$40\;\mu s$ . The pedestal region
  becomes stochastic where the original LCFS and SOL-region bounding contours are
  shown for reference. Inset shows homoclinic tangle structure near the divertor
  x-point.  }
  \label{fig:fieldline}
\end{figure}

In the presence of these electro-magnetic MHD perturbations, the pedestal
region becomes stochastic as shown in Fig.~\ref{fig:fieldline} at t=$40\;\mu s$
by a magnetic field-line Poincaré (or puncture) plot. Field-lines in this
figure are followed for $10^4\;m$ or until they hit the wall and then are color
coded by their total length.  The region from near the top of the pedestal out
to the LCFS becomes stochastic. As the inset figure shows, a homoclinic tangle
structure \cite{evans02,roeder03,evans05} develops near the divertor x-point.
Given the large parallel thermal conductivity and stochastic magnetic fields,
one expects significant energy transport within the pedestal region to result.
However, as shown and discussed next, this is not the case and the energy
transport is relatively small when compared with the particle transport.

\begin{figure}
  \centering
  \includegraphics[width=7cm]{profs.png}
  \vspace{-4mm}
  \caption{[Color online]
  Toroidal average of the density, temperature, toroidal- and
  poloidal-rotation, and current-density profiles on the outboard midplane for
  the initial condition and average values during three time windows. }
  \label{fig:profs}
\end{figure}

Figure~\ref{fig:profs} shows the toroidal average of the density, temperature,
toroidal- and poloidal-rotation, and current-density profiles on the outboard
midplane for the initial conditions and average values during three time
windows. The largest profile modifications occur with the density and current
profiles, whereas the flow profiles are largely unchanged. Consistent with
experimental observations during QH-mode\cite{garofalo15}, the simulated state
leads to large particle transport relative to the thermal transport. However,
this simulation result is somewhat puzzling given the stochastic magnetic field
region within the pedestal. We posit three potential explanations that require
further investigation. The first possibility is that the simulation time is
too short for the profiles to reach a fully relaxed state. 
The second theory is that the large thermal conductivity
changes the phase of the temperature perturbation relative to the density
perturbation in such a way that the flux-surface-averaged advective transport
(where the particle flux is $(d/dV)\int_V \tilde{n} \tilde{\mathbf{v}} \cdot
\nabla \psi / |\nabla \psi| dV$ and the thermal flux is roughly 
$(d/dV)\int_V \tilde{T} \tilde{\mathbf{v}} \cdot \nabla \psi / |\nabla \psi|
dV$; here $V$ is the
volume enclosed by a flux surface) is large for the plasma density but small
for the plasma energy. A third hypothesis is that the stochastic transport is
small because the temperature in the open-field line region is large
($\simeq1\;keV$) and thus the effective temperature gradient along the
field-lines is small. For this computation, the pressure profile (and thus
implicitly the temperature profile) in the SOL was chosen to minimize the SOL
currents. Future computation will include temperatures in the open-flux region
that are at least an order of magnitude smaller. The experimentally relevant
value is somewhat difficult to determine as the ion temperature profile is not
well constrained in the open-field-line region. Given the relatively low
density outside the LCFS, the CER measurements can be corrupted by effects from
confined particles with large banana orbits from the higher-density pedestal
region.
