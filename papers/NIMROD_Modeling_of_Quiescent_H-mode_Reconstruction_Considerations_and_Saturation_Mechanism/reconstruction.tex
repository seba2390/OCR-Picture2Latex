%\section{Importing reconstructed discharges into NIMROD}
\section{Extended equilibrium reconstruction for NIMROD}
\label{sec:reconstruct}

\begin{figure}
  \centering
  \includegraphics[width=8cm]{145098_newspec.png}
  \vspace{-4mm}
  \caption{[Color online]
  Cross-power spectrum plot of the magnetic fluctuation probe measurements
  from DIII-D shot 145098. An initial phase contains coherent EHO fluctuations
  followed by a phase with broadband-MHD activity. The NIMROD simulation
  is initialized from a reconstruction during the latter phase at 4250 ms.
  }
  \label{fig:newspec}
\end{figure}

The cross-power spectrum from magnetic fluctuation probe measurements on DIII-D
during shot 145098 is shown in Fig.~\ref{fig:newspec}.  This shot is part of a
campaign to create a QH-mode discharge with ITER-relevant parameters and shape,
the ultimate results of which are summarized in Ref.~\cite{garofalo15}.  As the
torque is ramped down the mode activity transitions from EHO to broadband MHD.
An EFIT \cite{lao85,lao05} reconstruction, constrained by magnetic-probe,
motional-stark-effect, Thomson-scattering, and charge-exchange-recombination
(CER) measurements, is used to specify the initial condition in the NIMROD code.
High quality equilibria are essential for extended-MHD modeling with
initial-value codes such as NIMROD.  Typically the spatial resolution
requirements for extended-MHD modeling, which must resolve singular-layer
physics and highly anisotropic diffusion, are more stringent than the
resolution of equilibrium reconstructions from experimental discharges. To
circumvent mapping errors, we re-solve the Grad-Shafranov equation with
open-flux regions using the NIMEQ \cite{Howell14} solver to generate a new
equilibrium while using the mapped results for both an initial guess and to
specify the boundary condition. 

\begin{SCfigure}
  %\centering
  \includegraphics[width=4cm]{jp-sketch.png}
  %\vspace{-4mm}
  \caption{[Color online]
  Sketch that shows a discontinuous first derivative in pressure
  causes a discontinuous current profile when solving the Grad-Shafranov 
  equation.}
  \label{fig:jp-sketch}
\end{SCfigure}

Additionally, reconstructions commonly assume that the region outside the last
closed flux surface (LCFS) is current free. The pressure, temperature and
density profiles are specified only up to the LCFS and are assumed to be
constant outside the LCFS as illustrated in Fig.~\ref{fig:jp-sketch}.  For
discharges with large pedestal current, as is commonly found during QH-mode,
this can lead to a large discontinuity in the current density at the LCFS that
is problematic for MHD modeling. During our re-solve of the Grad-Shafranov
equation, we relax the current-free assumption outside the LCFS and include
temperature and density profiles with non-zero gradients which generate
associated small currents in the scrape-off layer (SOL) that cause the overall
current profile to be continuous. 
Modified-bump-function fits are used to smoothly extrapolate the pressure,
electron temperature and particle density in the SOL region. Derivatives of all
orders vanish for this functional fit at the edge of the SOL region and thus
the resulting current profile smoothly decays to zero. For this case, the
pressure drops from $922$ to $581\;Pa$, the electron temperature drops from
$186$ to $30\;eV$ and the density drops from $4.6\times10^{18}$ to
$2.5\times10^{18}\;m^{-3}$ in the SOL region. The half width of the electron
pressure profile is roughly $2.5\;mm$ at the outboard mid-plane and $2\;cm$ at
the divertor plate. This results in a SOL width that is slightly smaller than
the measured half width of the heat-flux during the later half of the inter-ELM
period of DIII-D ELMy H-mode discharges in Ref.~\cite{eich13}. These profiles
fits, made with a focus on the resulting current and resistivity profiles,
result in an ion-temperature profile that remains above $1\;keV$ throughout the
domain. This inconsistency will be removed in future modeling. The new solution
is an equilibrium that closely resembles the original reconstruction with the
exception of the open-flux currents and additional quantification of these
methods will be described in a future manuscript \cite{kingSOL}. 
This regenerated equilibrium is consistent with the
core profiles that are measured by the high quality diagnostics on DIII-D.

\begin{figure}
  \centering
  \includegraphics[width=7cm]{npq.png}
  \vspace{-4mm}
  \caption{[Color online]
  Density, pressure and safety-factor profiles as a function of 
  normalized flux. SOL profiles for density and pressure are also
  shown for $\psi_n>1$.}
  \label{fig:npq}
\end{figure}

The density, pressure and safety-factor profiles as a function of normalized
flux are shown in Fig.~\ref{fig:npq}, where the SOL profiles for density and
pressure are included where $\psi_n>1$. The current-profile that results
from our re-solve of the Grad-Shafranov equation is plotted in Fig.~\ref{fig:current}.
The SOL region contains small, but non-zero, currents that terminate
poloidally on the divertor.

\begin{figure}
  \centering
  \includegraphics[width=7cm]{current.png}
  \vspace{-4mm}
  \caption{[Color online]
  Initial current density (contour colors for toroidal current and arrow
  vectors for poloidal current) from the reconstructed state with an extrapolated SOL
  region. The SOL region contains small, but non-zero currents that
  terminate poloidally on the divertor. }
  \label{fig:current}
\end{figure}

\begin{figure}
  \centering
  \includegraphics[width=7cm]{ExB_Kpol_flow.png}
  \vspace{-4mm}
  \caption{[Color online]
  The reconstructed toroidal (contour colors) and poloidal (arrow vector) flow with
  extrapolated flows in the SOL region. The poloidal flow flips sign just inside
  the LCFS with flows closer to the core proceeding counter-clockwise and flows
  near and outside the LCFS proceeding clockwise. }
  \label{fig:flows}
\end{figure}

Importantly, our initial conditions include the full reconstructed
toroidal and poloidal flows as shown in Fig.~\ref{fig:flows}.
Experimentally, these flows are critical to the observation of QH-mode where,
in particular, large $\mathbf{E}\times\mathbf{B}$ flow shear is highly correlated with quiescent
operation \cite{garofalo11}. 
% JRK: full flow included
% In our simulation, $E\times B$ toroidal flow is
% retained and the diamagnetic contribution is neglected (where these two
% contributions compose the full toroidal flow) as this choice is consistent the
% single-fluid MHD model used in our nonlinear computations.  
Like the thermodynamic profiles, the flow profiles are specified up to and
are non-zero at the LCFS. Thus we extrapolate these profiles to zero within the
SOL.

% Relative to the tokamak core, the characteristic time and spatial scales
% are compressed.  However, type-I ELMs still have the instability time
% scale associated with the fast crash is an order of magnitude faster
% than the transport-time scale associated with the processes that govern
% the build up of the pedestal structure.   This separation of time scales
% still allows the standard decomposition of studying the linear
% instabilities about an equilibrium that is used in core modes as well.
% 
% Like the core modes, these long-wavelength instabilities are dominated
% by the stiffness in the ideal MHD terms, even for the cases when they
% may be strictly ideal stable.  Multiple numerical methods have been
% developed to handle this stiffness for both linear and nonlinear codes.
% For the nonlinear codes, one numerical advantage is to
% separate the fields into steady-state (e.g. the reconstructed fields)
% and time-dependent parts.  The pure steady-state terms are analytically
% eliminated resulting in the largest terms in the system to be removed
% from the numerical computations.

Typically only MHD-force balance (a Grad-Shafranov solution) is strictly
enforced for the steady state.  In practice, perturbations about a
time-independent equilibrium are evolved, and that the time-independent
equilibrium need not be a time-independent solution of the source-free
resistive MHD equations \cite{Sovinec04,charlton86}. This
effectively assumes the presence of implicit (in the sense that they are
calculable but not calculated) sources, fluxes and sinks.  With these
assumptions, if the code is run on a MHD-stable case with $n_\phi>0$
perturbations, the modification to the $n_\phi=0$ fields is insignificant.
Alternatively, when the case is MHD-unstable, the initial $n_\phi=0$ fields are
self-consistently modified by the presence of the unstable modes.

% There is no technical reason to make this decomposition into 
% steady-state equilibrium and time-dependent perturbed fields as 
The NIMROD
code has the capability to compute the extended-MHD evolution of the
reconstructed fields. However, it is well-known that physical mechanisms
outside the scope of our modeling equations mediate tokamak transport such as
neoclassical bootstrap current, toroidal viscosity, and poloidal flow damping,
neutral beam and RF drives, kinetic turbulence, and coupling to the scrape-off layer
(SOL), neutrals, impurities and the material boundary. Including these effects
requires explicit calculation of the sources, fluxes and sinks. These
transport-type calculations are possible and are becoming practical (e.g.
\cite{Jenkins12,Jenkins15,held15}), but this sort of integrated modeling
remains in the future. Thus in this work we assume that the initial
reconstructed fields are steady state and our goal is to model the evolution of
the 3D perturbations around this state.
