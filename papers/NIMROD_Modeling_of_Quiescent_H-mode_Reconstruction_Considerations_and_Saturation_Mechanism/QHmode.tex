%% LyX 2.0.6 created this file.  For more info, see http://www.lyx.org/.
%% 
%\documentclass[english,aps,superscriptaddress,showkeys,showpacs,prepri]{revtex4}
\documentclass[english,aps,superscriptaddress,showkeys,showpacs,prepri,twocolumn]{revtex4}
\usepackage[T1]{fontenc}
\pdfoutput=1
%\usepackage[latin9]{inputenc}
\usepackage[letterpaper]{geometry}
\geometry{verbose,tmargin=1in,bmargin=1in,lmargin=1in,rmargin=1in}
\setcounter{secnumdepth}{3}
\usepackage{units}
\usepackage{bbding}
\usepackage{amsmath}
\usepackage{amssymb}
\usepackage{graphicx}
\usepackage{esint}
\usepackage{sidecap}
\usepackage[colorinlistoftodos,prependcaption,textsize=tiny]{todonotes}

\usepackage[utf8]{inputenc}
\usepackage{lmodern} % load a font with all the characters

\makeatletter

%%%%%%%%%%%%%%%%%%%%%%%%%%%%%% LyX specific LaTeX commands.
%% Because html converters don't know tabularnewline
\providecommand{\tabularnewline}{\\}

%%%%%%%%%%%%%%%%%%%%%%%%%%%%%% Textclass specific LaTeX commands.
\@ifundefined{textcolor}{}
{%
 \definecolor{BLACK}{gray}{0}
 \definecolor{WHITE}{gray}{1}
 \definecolor{RED}{rgb}{1,0,0}
 \definecolor{GREEN}{rgb}{0,1,0}
 \definecolor{BLUE}{rgb}{0,0,1}
 \definecolor{CYAN}{cmyk}{1,0,0,0}
 \definecolor{MAGENTA}{cmyk}{0,1,0,0}
 \definecolor{YELLOW}{cmyk}{0,0,1,0}
}

%%%%%%%%%%%%%%%%%%%%%%%%%%%%%% User specified LaTeX commands.
%\usepackage[sort&compress,numbers]{natbib} 
\bibliographystyle{apsrev4-1} 
\usepackage{doi}
\usepackage{hyperref}

\makeatother

\usepackage{babel}
\begin{document}

\title{NIMROD Modeling of Quiescent H-mode: Reconstruction Considerations and
Saturation Mechanism}

\author{J. R. King}
\affiliation{Tech-X Corporation, 5621 Arapahoe Ave. Boulder, CO 80303, USA}

\author{K. H. Burrell}
\affiliation{General Atomics, PO Box 85608, San Diego, CA 92186–5608, USA}

\author{A. M. Garofalo}
\affiliation{General Atomics, PO Box 85608, San Diego, CA 92186–5608, USA}

\author{R. J. Groebner}
\affiliation{General Atomics, PO Box 85608, San Diego, CA 92186–5608, USA}

\author{S. E. Kruger}
\affiliation{Tech-X Corporation, 5621 Arapahoe Ave. Boulder, CO 80303, USA}

\author{A. Y. Pankin}
\affiliation{Tech-X Corporation, 5621 Arapahoe Ave. Boulder, CO 80303, USA}

\author{P. B. Snyder}
\affiliation{General Atomics, PO Box 85608, San Diego, CA 92186–5608, USA}

\date{draft \today}
\begin{abstract}
The extended-MHD NIMROD code [C.R.~Sovinec and J.R.~King, J.~Comput.~Phys.~{\bf
229}, 5803 (2010)] models broadband-MHD activity from a reconstruction of a
quiescent H-mode shot on the DIII-D tokamak [J. L. Luxon, Nucl. Fusion 42, 614
(2002)]. Computations with the reconstructed toroidal and
poloidal ion flows exhibit low-$n_\phi$ perturbations ($n_\phi\simeq1-5$) that
grow and saturate into a turbulent-like MHD state. The workflow used to project
the reconstructed state onto the NIMROD basis functions re-solves the
Grad-Shafranov equation and extrapolates profiles to include scrape-off-layer
currents. Evaluation of the transport from the turbulent-like MHD state leads
to a relaxation of the density and temperature profiles. 
Published version: Nucl. Fusion 57 022002 (2017) [\url{http://dx.doi.org/10.1088/0029-5515/57/2/022002}]
\end{abstract}

\keywords{broadband-MHD,
extended-MHD modeling,
quiescent H-mode,
tokamak pedestal}

\pacs{52.30.Ex 52.35.Py, 52.55.Fa, 52.55.Tn, 52.65.Kj}
\maketitle

%--------------------------------------------------
%%%%%%%%%%%%%%%%%%%%%%%%%%%
% names
%%%%%%%%%%%%%%%%%%%%%%%%%%%
\newcommand{\modelname}{DCN+\xspace}
\newcommand{\papertitle}{DCN+: Mixed objective and deep residual coattention for question answering}
%RS Not super sure about this title but DCN+ makes not sense and "mixed objective" is unclear and residual coattention is also not (yet) giving anybody a good reason to read the paper...
%\newcommand{\papertitle}{DCN+: Improving Dynamic Coattention Networks for Question Answering}
%\newcommand{\papertitle}{DCN+: Reinforced and Deeper Dynamic Coattention Networks for Question Answering}


\newcommand{\squad}{SQuAD\xspace}

%%%%%%%%%%%%%%%%%%%%%%%%%%%
% shortcuts
%%%%%%%%%%%%%%%%%%%%%%%%%%%
\newcommand{\todo}[1]{\textcolor{orange}{{#1}}\xspace}

%%%%%%%%%%%%%%%%%%%%%%%%%%%
% variables
%%%%%%%%%%%%%%%%%%%%%%%%%%%

\newcommand{\real}{\mathbb{R}}
\newcommand{\loss}{l}
\newcommand{\pstart}{p^{\rm{start}}}
\newcommand{\pend}{p^{\rm{end}}}

\newcommand{\lookup}{L}
\newcommand{\encoded}{E}
\newcommand{\affinity}{A}
\newcommand{\summary}{S}
\newcommand{\context}{C}

\newcommand{\devem}{{\rm EM}_{\rm dev}}
\newcommand{\devf}{{\rm F1}_{\rm dev}}
\newcommand{\testem}{{\rm EM}_{\rm test}}
\newcommand{\testf}{{\rm F1}_{\rm test}}

%%%%%%%%%%%%%%%%%%%%%%%%%%%
% colours
%%%%%%%%%%%%%%%%%%%%%%%%%%%
\definecolor{myblue}{RGB}{67,118,237}
\definecolor{myred}{RGB}{237,39,58}


%%%%%%%%%%%%%%%%%%%%%%%%%%%
% functions
%%%%%%%%%%%%%%%%%%%%%%%%%%%

\newcommand{\softmax}[1]{{\rm softmax}\left({#1}\right)}
\newcommand{\bilstm}{{\rm biLSTM}}
\newcommand{\coattn}{{\rm coattn}}
\newcommand{\concat}[1]{{\rm concat}\left({#1}\right)}
\newcommand{\emb}{{\rm emb}}
\newcommand{\proj}{{\rm proj}}
\newcommand{\dhid}{h}
\newcommand{\demb}{e}
\newcommand{\ddocument}{m}
\newcommand{\dquestion}{n}

\newcommand{\answer}[2]{\rm{ans}\left( {#1}, {#2}\right)}

%%%%%%%%%%%%%%%%%%%%%%%%%%%
% numbers
%%%%%%%%%%%%%%%%%%%%%%%%%%%

% ours
\newcommand{\devemours}{74.5\%\xspace}
\newcommand{\devfours}{83.1\%\xspace}
\newcommand{\emours}{75.1\%\xspace}
\newcommand{\fours}{83.1\%\xspace}
\newcommand{\emoursensemble}{78.9\%\xspace}
\newcommand{\foursensemble}{86.0\%\xspace}

% dcn
\newcommand{\devemdcn}{65.4\%\xspace}
\newcommand{\devfdcn}{75.6\%\xspace}
\newcommand{\emdcn}{66.2\%\xspace}
\newcommand{\fdcn}{75.9\%\xspace}
\newcommand{\emdcnensemble}{71.6\%\xspace}
\newcommand{\fdcnensemble}{80.4\%\xspace}

% bidaf
\newcommand{\devembidaf}{67.7\%\xspace}
\newcommand{\devfbidaf}{77.3\%\xspace}
\newcommand{\embidaf}{68.0\%\xspace}
\newcommand{\fbidaf}{77.3\%\xspace}
\newcommand{\embidafensemble}{73.7\%\xspace}
\newcommand{\fbidafensemble}{81.5\%\xspace}

% sedtbidaf
\newcommand{\devemsedtbidaf}{67.9\%\xspace}
\newcommand{\devfsedtbidaf}{77.4\%\xspace}
\newcommand{\emsedtbidaf}{68.5\%\xspace}
\newcommand{\fsedtbidaf}{78.0\%\xspace}
\newcommand{\emsedtbidafensemble}{73.0\%\xspace}
\newcommand{\fsedtbidafensemble}{80.8\%\xspace}

% mnemonic reader
\newcommand{\devemmr}{70.1\%\xspace}
\newcommand{\devfmr}{79.6\%\xspace}
\newcommand{\emmr}{69.9\%\xspace}
\newcommand{\fmr}{79.2\%\xspace}
\newcommand{\emmrensemble}{73.7\%\xspace}
\newcommand{\fmrensemble}{81.7\%\xspace}

% rnet
\newcommand{\devemrnet}{72.3\%\xspace}
\newcommand{\devfrnet}{80.6\%\xspace}
\newcommand{\emrnet}{72.3\%\xspace}
\newcommand{\frnet}{80.7\%\xspace}
\newcommand{\emrnetensemble}{76.9\%\xspace}
\newcommand{\frnetensemble}{84.0\%\xspace}

% docreader
\newcommand{\devemdocreader}{69.5\%\xspace}
\newcommand{\devfdocreader}{78.8\%\xspace}
\newcommand{\emdocreader}{70.0\%\xspace}
\newcommand{\fdocreader}{79.0\%\xspace}

% fastqa
\newcommand{\devemfastqa}{70.3\%\xspace}
\newcommand{\devffastqa}{78.5\%\xspace}
\newcommand{\emfastqa}{70.8\%\xspace}
\newcommand{\ffastqa}{78.9\%\xspace}

% reasonet
\newcommand{\emreasonet}{69.1\%\xspace}
\newcommand{\freasonet}{78.9\%\xspace}
\newcommand{\emreasonetensemble}{73.4\%\xspace}
\newcommand{\freasonetensemble}{81.8\%\xspace}

%ablation 
%\newcommand{\emdev}{74.3\%\xspace}
%\newcommand{\fdev}{82.5\%\xspace}

\newcommand{\emcove}{71.3\%\xspace}
\newcommand{\fcove}{79.9\%\xspace}
\newcommand{\deltaemcove}{3.2\%\xspace}
\newcommand{\deltafcove}{3.2\%\xspace}

\newcommand{\emnocoattention}{73.1\%\xspace}
\newcommand{\fnocoattention}{81.5\%\xspace}
\newcommand{\deltaemcoattention}{1.4\%\xspace}
\newcommand{\deltafcoattention}{1.6\%\xspace}

\newcommand{\emnomixedobjective}{73.8\%\xspace}
\newcommand{\fnomixedobjective}{82.1\%\xspace}
\newcommand{\deltaemmixedobjective}{0.7\%\xspace}
\newcommand{\deltafmixedobjective}{1.0\%\xspace}

\newcommand{\emnomoe}{74.0\%\xspace}
\newcommand{\fnomoe}{82.4\%\xspace}
\newcommand{\deltaemmoe}{0.5\%\xspace}
\newcommand{\deltafmoe}{0.7\%\xspace}

\newcommand{\emnoworddropout}{73.8\%\xspace}
\newcommand{\fnoworddropout}{82.1\%\xspace}
\newcommand{\deltaemworddropout}{0.5\%\xspace}
\newcommand{\deltafworddropout}{0.4\%\xspace}


%% 2 decimal places
% % ours
% \newcommand{\devemours}{74.46\%\xspace}
% \newcommand{\devfours}{83.12\%\xspace}
% \newcommand{\emours}{75.09\%\xspace}
% \newcommand{\fours}{83.08\%\xspace}
% \newcommand{\emoursensemble}{78.85\%\xspace}
% \newcommand{\foursensemble}{86.00\%\xspace}
% 
% % dcn
% \newcommand{\devemdcn}{65.40\%\xspace}
% \newcommand{\devfdcn}{75.60\%\xspace}
% \newcommand{\emdcn}{66.23\%\xspace}
% \newcommand{\fdcn}{75.90\%\xspace}
% \newcommand{\emdcnensemble}{71.63\%\xspace}
% \newcommand{\fdcnensemble}{80.39\%\xspace}
% 
% % bidaf
% \newcommand{\devembidaf}{67.70\%\xspace}
% \newcommand{\devfbidaf}{77.30\%\xspace}
% \newcommand{\embidaf}{67.97\%\xspace}
% \newcommand{\fbidaf}{77.32\%\xspace}
% \newcommand{\embidafensemble}{73.74\%\xspace}
% \newcommand{\fbidafensemble}{81.53\%\xspace}
% 
% % sedtbidaf
% \newcommand{\devemsedtbidaf}{67.89\%\xspace}
% \newcommand{\devfsedtbidaf}{77.42\%\xspace}
% \newcommand{\emsedtbidaf}{68.48\%\xspace}
% \newcommand{\fsedtbidaf}{77.97\%\xspace}
% \newcommand{\emsedtbidafensemble}{73.02\%\xspace}
% \newcommand{\fsedtbidafensemble}{80.84\%\xspace}
% 
% % mnemonic reader
% \newcommand{\devemmr}{70.10\%\xspace}
% \newcommand{\devfmr}{79.60\%\xspace}
% \newcommand{\emmr}{69.86\%\xspace}
% \newcommand{\fmr}{79.21\%\xspace}
% \newcommand{\emmrensemble}{73.67\%\xspace}
% \newcommand{\fmrensemble}{81.69\%\xspace}
% 
% % rnet
% \newcommand{\devemrnet}{72.30\%\xspace}
% \newcommand{\devfrnet}{80.60\%\xspace}
% \newcommand{\emrnet}{72.30\%\xspace}
% \newcommand{\frnet}{80.70\%\xspace}
% \newcommand{\emrnetensemble}{76.90\%\xspace}
% \newcommand{\frnetensemble}{84.00\%\xspace}
% 
% % docreader
% \newcommand{\devemdocreader}{69.50\%\xspace}
% \newcommand{\devfdocreader}{78.80\%\xspace}
% \newcommand{\emdocreader}{70.00\%\xspace}
% \newcommand{\fdocreader}{79.00\%\xspace}
% 
% % fastqa
% \newcommand{\devemfastqa}{70.30\%\xspace}
% \newcommand{\devffastqa}{78.50\%\xspace}
% \newcommand{\emfastqa}{70.80\%\xspace}
% \newcommand{\ffastqa}{78.90\%\xspace}
% 
% % reasonet
% \newcommand{\emreasonet}{69.10\%\xspace}
% \newcommand{\freasonet}{78.90\%\xspace}
% \newcommand{\emreasonetensemble}{73.40\%\xspace}
% \newcommand{\freasonetensemble}{81.80\%\xspace}
% 
% %ablation 
% \newcommand{\emdev}{74.30\%\xspace}
% \newcommand{\fdev}{82.50\%\xspace}
% 
% \newcommand{\emcove}{71.30\%\xspace}
% \newcommand{\fcove}{79.90\%\xspace}
% \newcommand{\deltaemcove}{3.00\%\xspace}
% \newcommand{\deltafcove}{2.60\%\xspace}
% 
% \newcommand{\emnocoattention}{73.09\%\xspace}
% \newcommand{\fnocoattention}{81.50\%\xspace}
% \newcommand{\deltaemcoattention}{1.21\%\xspace}
% \newcommand{\deltafcoattention}{1.00\%\xspace}
% 
% \newcommand{\emnomixedobjective}{73.81\%\xspace}
% \newcommand{\fnomixedobjective}{82.11\%\xspace}
% \newcommand{\deltaemmixedobjective}{0.49\%\xspace}
% \newcommand{\deltafmixedobjective}{0.39\%\xspace}
% 
% \newcommand{\emnomoe}{73.95\%\xspace}
% \newcommand{\fnomoe}{82.43\%\xspace}
% \newcommand{\deltaemmoe}{0.35\%\xspace}
% \newcommand{\deltafmoe}{0.07\%\xspace}
% 
% \newcommand{\emnoworddropout}{73.83\%\xspace}
% \newcommand{\fnoworddropout}{82.14\%\xspace}
% \newcommand{\deltaemworddropout}{0.47\%\xspace}
% \newcommand{\deltafworddropout}{0.36\%\xspace}   % Useful abbreviations
\section{Introduction}  \label{sec:introduction}

\newcommand\inexpIntro[3]{#1?(#2,#3).}
\newcommand\rinexpIntro[3]{*#1?(#2,#3).}
\newcommand\outexpIntro[3]{#1!(#2,#3).}
\newcommand\outatomIntro[3]{#1!(#2,#3)}

We propose a fully automated method for proving termination of \(\pi\)-calculus processes.
Although there have been a lot of studies on termination analysis for the \(\pi\)-calculus
and related calculi~\cite{Deng06IC,Demangeon07,SangiorgiTermination,KobayashiHybrid,Yoshida04IC,DBLP:journals/jlp/DemangeonHS10,Venet98SAS}, most of them have been rather theoretical,
and there have been surprisingly little efforts in developing  fully automated termination
verification methods and tools based on them. To our knowledge,
Kobayashi's \typical{}~\cite{TyPiCal,KobayashiHybrid} is the only exception that
can prove termination of \(\pi\)-calculus processes (extended with natural numbers)
fully automatically, but its termination analysis is quite limited (see Section~\ref{sec:relatedwork}).

Our method is based on a reduction to termination analysis for sequential programs:
we translate a \(\pi\)-calculus process \(P\) to a sequential program \(S_P\), so that
if \(S_P\) is terminating, so is \(P\). The reduction allows us to use
powerful, mature methods and tools
for termination analysis of sequential programs~\cite{heizmann2016ultimate,freqterm,DBLP:conf/lics/PodelskiR04,Kuwahara2014Termination,DBLP:journals/cacm/CookPR11}.

The idea of the translation is to convert a chain of communications on replicated input
channels to a chain of recursive function calls of the target sequential program.
Let us consider the following Fibonacci process:
\begin{align*}
    & \rinexpIntro{\fib}{n}{r}
        \ifexp{n<2}{ \soutatom{r}{1} \\ &\quad}
                   { \nuexp{s_1} \nuexp{s_2} (\outatomIntro{\fib}{n-1}{s_1} \PAR \outatomIntro{\fib}{n-2}{s_2} \PAR \sinexp{s_1}{x}\sinexp{s_2}{y}\soutatom{r}{x+y}) \\}
    & \PAR \outatomIntro{\fib}{m}{r}
\end{align*}
Here, the process
$\rinexpIntro{\fib}{n}{r} \ldots$ is a function server that computes the \(n\)-th Fibonacci number
in parallel and returns the result to \(r\),
and $\outatom{\fib}{m}{r}$ sends a request for computing the \(m\)-th Fibonacci number;
those who are not familiar with the syntax of the \(\pi\)-calculus may wish to consult
Section~\ref{sec:targetlanguage} first.
To prove that the process above is terminating for any integer \(m\),
it suffices to show that there is no infinite chain of communications on $\fib$:
\[
    \fib(m,r) \to \fib(m_1,r_1) \to \fib(m_2,r_2) \to \cdots.
\]
We convert the process above to the following program:\footnote{The actual translation
  given later is a little more complex.}
\begin{verbatim}
 let rec fib(n) = if n<2 then () else (fib(n-1) [] fib(n-2)) in
 fib(m)
\end{verbatim}
Here, \texttt{[]} represents the non-deterministic choice.
Note that, although the calculation of Fibonacci numbers is not preserved,
for each chain of communications on \texttt{fib}, there is a corresponding
sequence of recursive calls:
\[
\mathtt{fib}(m) \to \mathtt{fib}(m_1) \to \mathtt{fib}(m_2) \to \cdots.
\]
Thus, the termination of the sequential program above implies the termination of
the original process.
As shown in the example above, (i) each communication on a replicated input channel
is converted to a function call, (ii) each communication on a non-replicated input
channel is just removed (or, in the actual translation, replaced by a call of
a trivial function defined by \(f(\seq{x})=(\,)\)), and (iii) parallel composition
is replaced by a non-deterministic choice.
We formalize the translation outlined above and prove its correctness.

The basic translation sketched above sometimes loses too much information.
For example, consider the following process:
\begin{align*}
    & \rinexpIntro{\pre}{n}{r} \soutatom{r}{n-1} \\
    & \PAR \rinexpIntro{f}{n}{r} \ifexp{n<0}{ \soutatom{r}{1} }
                                       { \nuexp{s} (\outatomIntro{\pre}{n}{s} \PAR \sinexp{s}{x}\outatomIntro{f}{x}{r}) } \\
    & \PAR \outatomIntro{f}{m}{r}
\end{align*}
The translation sketched above would yield:
\begin{verbatim}
  let pred(n) = n-1 in
  let rec f(n) = if n<0 then () else (pred(n) [] f(*)) in
  f(m)
\end{verbatim}
Here, \texttt{*} represents a non-deterministic integer: since we have removed
the input $\sinatom{s}{x}$, we do not have information about the value of \( x \).
As a result, the sequential program above is non-terminating, although the original
process is terminating.
To remedy this problem, we also refine the basic translation above by using a refinement
type system for the \(\pi\)-calculus. Using the refinement type system,
we can infer that the value of \(x\) in the original process is less than \(n\),
so that we can refine the definition of \texttt{f} to:
\begin{verbatim}
 let rec f(n) = ... else (pred(n) [] let x=* in assume(x<n);f(x))
\end{verbatim}
The target program is now terminating, from which
we can deduce that the original process is also terminating.
We have implemented an automated tool based on the refined translation above.

The contributions of this paper are summarized as follows.
\begin{itemize}
\item The formalization of the basic translation from the \(\pi\)-calculus
  (extended with integers) to sequential programs, and a proof of its correctness.
\item The formalization of a refined translation based on a refinement type system.
\item An implementation of the refined translation, including automated refinement type
  inference based on CHC solving, and experiments to evaluate the effectiveness of
  our method.
\end{itemize}

The rest of this paper is structured as follows.
Section~\ref{sec:targetlanguage} introduces the source and target languages
of our translation.
Section~\ref{sec:approach} 
formalizes the basic translation, and proves its correctness.
Section~\ref{sec:refinement} refines the basic translation by using a refinement type system.
Section~\ref{sec:implementation} reports an implementation and experiments.
Section~\ref{sec:relatedwork} discusses related work,
and Section~\ref{sec:conclusion} concludes the paper.

%\section{Importing reconstructed discharges into NIMROD}
\section{Extended equilibrium reconstruction for NIMROD}
\label{sec:reconstruct}

\begin{figure}
  \centering
  \includegraphics[width=8cm]{145098_newspec.png}
  \vspace{-4mm}
  \caption{[Color online]
  Cross-power spectrum plot of the magnetic fluctuation probe measurements
  from DIII-D shot 145098. An initial phase contains coherent EHO fluctuations
  followed by a phase with broadband-MHD activity. The NIMROD simulation
  is initialized from a reconstruction during the latter phase at 4250 ms.
  }
  \label{fig:newspec}
\end{figure}

The cross-power spectrum from magnetic fluctuation probe measurements on DIII-D
during shot 145098 is shown in Fig.~\ref{fig:newspec}.  This shot is part of a
campaign to create a QH-mode discharge with ITER-relevant parameters and shape,
the ultimate results of which are summarized in Ref.~\cite{garofalo15}.  As the
torque is ramped down the mode activity transitions from EHO to broadband MHD.
An EFIT \cite{lao85,lao05} reconstruction, constrained by magnetic-probe,
motional-stark-effect, Thomson-scattering, and charge-exchange-recombination
(CER) measurements, is used to specify the initial condition in the NIMROD code.
High quality equilibria are essential for extended-MHD modeling with
initial-value codes such as NIMROD.  Typically the spatial resolution
requirements for extended-MHD modeling, which must resolve singular-layer
physics and highly anisotropic diffusion, are more stringent than the
resolution of equilibrium reconstructions from experimental discharges. To
circumvent mapping errors, we re-solve the Grad-Shafranov equation with
open-flux regions using the NIMEQ \cite{Howell14} solver to generate a new
equilibrium while using the mapped results for both an initial guess and to
specify the boundary condition. 

\begin{SCfigure}
  %\centering
  \includegraphics[width=4cm]{jp-sketch.png}
  %\vspace{-4mm}
  \caption{[Color online]
  Sketch that shows a discontinuous first derivative in pressure
  causes a discontinuous current profile when solving the Grad-Shafranov 
  equation.}
  \label{fig:jp-sketch}
\end{SCfigure}

Additionally, reconstructions commonly assume that the region outside the last
closed flux surface (LCFS) is current free. The pressure, temperature and
density profiles are specified only up to the LCFS and are assumed to be
constant outside the LCFS as illustrated in Fig.~\ref{fig:jp-sketch}.  For
discharges with large pedestal current, as is commonly found during QH-mode,
this can lead to a large discontinuity in the current density at the LCFS that
is problematic for MHD modeling. During our re-solve of the Grad-Shafranov
equation, we relax the current-free assumption outside the LCFS and include
temperature and density profiles with non-zero gradients which generate
associated small currents in the scrape-off layer (SOL) that cause the overall
current profile to be continuous. 
Modified-bump-function fits are used to smoothly extrapolate the pressure,
electron temperature and particle density in the SOL region. Derivatives of all
orders vanish for this functional fit at the edge of the SOL region and thus
the resulting current profile smoothly decays to zero. For this case, the
pressure drops from $922$ to $581\;Pa$, the electron temperature drops from
$186$ to $30\;eV$ and the density drops from $4.6\times10^{18}$ to
$2.5\times10^{18}\;m^{-3}$ in the SOL region. The half width of the electron
pressure profile is roughly $2.5\;mm$ at the outboard mid-plane and $2\;cm$ at
the divertor plate. This results in a SOL width that is slightly smaller than
the measured half width of the heat-flux during the later half of the inter-ELM
period of DIII-D ELMy H-mode discharges in Ref.~\cite{eich13}. These profiles
fits, made with a focus on the resulting current and resistivity profiles,
result in an ion-temperature profile that remains above $1\;keV$ throughout the
domain. This inconsistency will be removed in future modeling. The new solution
is an equilibrium that closely resembles the original reconstruction with the
exception of the open-flux currents and additional quantification of these
methods will be described in a future manuscript \cite{kingSOL}. 
This regenerated equilibrium is consistent with the
core profiles that are measured by the high quality diagnostics on DIII-D.

\begin{figure}
  \centering
  \includegraphics[width=7cm]{npq.png}
  \vspace{-4mm}
  \caption{[Color online]
  Density, pressure and safety-factor profiles as a function of 
  normalized flux. SOL profiles for density and pressure are also
  shown for $\psi_n>1$.}
  \label{fig:npq}
\end{figure}

The density, pressure and safety-factor profiles as a function of normalized
flux are shown in Fig.~\ref{fig:npq}, where the SOL profiles for density and
pressure are included where $\psi_n>1$. The current-profile that results
from our re-solve of the Grad-Shafranov equation is plotted in Fig.~\ref{fig:current}.
The SOL region contains small, but non-zero, currents that terminate
poloidally on the divertor.

\begin{figure}
  \centering
  \includegraphics[width=7cm]{current.png}
  \vspace{-4mm}
  \caption{[Color online]
  Initial current density (contour colors for toroidal current and arrow
  vectors for poloidal current) from the reconstructed state with an extrapolated SOL
  region. The SOL region contains small, but non-zero currents that
  terminate poloidally on the divertor. }
  \label{fig:current}
\end{figure}

\begin{figure}
  \centering
  \includegraphics[width=7cm]{ExB_Kpol_flow.png}
  \vspace{-4mm}
  \caption{[Color online]
  The reconstructed toroidal (contour colors) and poloidal (arrow vector) flow with
  extrapolated flows in the SOL region. The poloidal flow flips sign just inside
  the LCFS with flows closer to the core proceeding counter-clockwise and flows
  near and outside the LCFS proceeding clockwise. }
  \label{fig:flows}
\end{figure}

Importantly, our initial conditions include the full reconstructed
toroidal and poloidal flows as shown in Fig.~\ref{fig:flows}.
Experimentally, these flows are critical to the observation of QH-mode where,
in particular, large $\mathbf{E}\times\mathbf{B}$ flow shear is highly correlated with quiescent
operation \cite{garofalo11}. 
% JRK: full flow included
% In our simulation, $E\times B$ toroidal flow is
% retained and the diamagnetic contribution is neglected (where these two
% contributions compose the full toroidal flow) as this choice is consistent the
% single-fluid MHD model used in our nonlinear computations.  
Like the thermodynamic profiles, the flow profiles are specified up to and
are non-zero at the LCFS. Thus we extrapolate these profiles to zero within the
SOL.

% Relative to the tokamak core, the characteristic time and spatial scales
% are compressed.  However, type-I ELMs still have the instability time
% scale associated with the fast crash is an order of magnitude faster
% than the transport-time scale associated with the processes that govern
% the build up of the pedestal structure.   This separation of time scales
% still allows the standard decomposition of studying the linear
% instabilities about an equilibrium that is used in core modes as well.
% 
% Like the core modes, these long-wavelength instabilities are dominated
% by the stiffness in the ideal MHD terms, even for the cases when they
% may be strictly ideal stable.  Multiple numerical methods have been
% developed to handle this stiffness for both linear and nonlinear codes.
% For the nonlinear codes, one numerical advantage is to
% separate the fields into steady-state (e.g. the reconstructed fields)
% and time-dependent parts.  The pure steady-state terms are analytically
% eliminated resulting in the largest terms in the system to be removed
% from the numerical computations.

Typically only MHD-force balance (a Grad-Shafranov solution) is strictly
enforced for the steady state.  In practice, perturbations about a
time-independent equilibrium are evolved, and that the time-independent
equilibrium need not be a time-independent solution of the source-free
resistive MHD equations \cite{Sovinec04,charlton86}. This
effectively assumes the presence of implicit (in the sense that they are
calculable but not calculated) sources, fluxes and sinks.  With these
assumptions, if the code is run on a MHD-stable case with $n_\phi>0$
perturbations, the modification to the $n_\phi=0$ fields is insignificant.
Alternatively, when the case is MHD-unstable, the initial $n_\phi=0$ fields are
self-consistently modified by the presence of the unstable modes.

% There is no technical reason to make this decomposition into 
% steady-state equilibrium and time-dependent perturbed fields as 
The NIMROD
code has the capability to compute the extended-MHD evolution of the
reconstructed fields. However, it is well-known that physical mechanisms
outside the scope of our modeling equations mediate tokamak transport such as
neoclassical bootstrap current, toroidal viscosity, and poloidal flow damping,
neutral beam and RF drives, kinetic turbulence, and coupling to the scrape-off layer
(SOL), neutrals, impurities and the material boundary. Including these effects
requires explicit calculation of the sources, fluxes and sinks. These
transport-type calculations are possible and are becoming practical (e.g.
\cite{Jenkins12,Jenkins15,held15}), but this sort of integrated modeling
remains in the future. Thus in this work we assume that the initial
reconstructed fields are steady state and our goal is to model the evolution of
the 3D perturbations around this state.

%\input{linear}   
\section{Nonlinear programming for pMDPs}\label{sec:nonlinear}
In this section we formally state a general pMDP parameter synthesis problem and describe how
it can be formulated using nonlinear programming.

\subsection{Formal problem statement}\label{sec:problem} 
%We are given a pMDP $\pMdpInit$, a set of specifications $\varphi_1,\ldots,\varphi_n$ that are either probabilistic reachability properties or expected cost properties, and an objective function $f\colon\Paramvar\rightarrow\R$ over the variables $V$.
%We define the \emph{pMDP synthesis problem} as computing a \emph{well-defined valuation} $u$ for $\mdp$, and a (randomized) scheduler $\sched\in\Sched^\mdp$, such that for the MC $\mdp^\sched[u]$ induced by $\sched$ and instantiated by $u$ it holds that $\mdp^\sched[u]\models\varphi_1,\ldots,\varphi_n$ and $f[u]$ is extremal; the obtained pair $(u,\sched)$ is an \emph{optimal solution} to the pMDP synthesis problem.
%Intuitively, we want to compute a valuation of parameters and a scheduler such that all specifications are satisfied and the value of the objective is extremal.
%We refer to a pair $(u,\sched)$ that only guarantees satisfaction of the specifications, but does not necessarily extremize the objective $f$, as a \emph{feasible solution}.
\fbox{
%\begin{minipage}{\textwidth}
\begin{minipage}{\dimexpr\textwidth-4\fboxsep}
\begin{problem}\label{prob:pmdpsyn}
Given a pMDP $\pMdpInit$, specifications
$\varphi_1,\ldots,\varphi_q$ that are either probabilistic reachability
properties or expected cost properties, and an objective function
$f\colon\Paramvar\rightarrow\R$ over the variables $V$, compute a well-defined
valuation $u\in\valuations^V$ for $\mdp$, and a (randomized) scheduler $\sched\in\Sched^\mdp$
such that the following conditions hold:
\begin{enumerate}[(a)]
	\item \label{prob:pmdpsyn_a} \emph{Feasibility}:
		the Markov chain $\mdp^\sched[u]$ induced by scheduler $\sched$ and
		instantiated by valuation $u$ satisfies the specifications, \ie, $\mdp^\sched[u]\models\varphi_1 \wedge \ldots \wedge \varphi_q$.
	
	\item \label{prob:pmdpsyn_b} \emph{Optimality}:
		the objective $f$ is minimized.
\end{enumerate}
\end{problem}
\end{minipage}
}

\noindent Intuitively, we wish to compute a parameter valuation and a scheduler such
that all specifications are satisfied, and the objective is globally minimized.
%We minimize the objective $f$, but depending on the
%application, we might wish to maximize it instead.\sj{I don't like this sentence.. It remains unclear whether we can also maximize or not. Can't we just always minimize in the formal problem statement}
We refer to a valuation--scheduler pair $(u,\sched)$ that satisfies
condition~(\ref{prob:pmdpsyn_a}), \ie, only guarantees satisfaction of the
specifications but does not necessarily minimize the objective $f$, as a
\emph{feasible} solution to the pMDP synthesis problem. If both
(\ref{prob:pmdpsyn_a}) and (\ref{prob:pmdpsyn_b}) are satisfied, the pair
is an \emph{optimal} solution to the pMDP synthesis problem.

%\emph{Strategy variables} $\sigma^{s,\act}\in[0,1]$ for each state $s\in S$ and action $\act\in\Act$ define a strategy $\sigma$ by $\sigma(s,\act)=\sigma^{s,\act}$ if additional side constraints ensure well-definedness. 
%Analogously, \emph{perturbation variables} $\delta^{s,\act}\in[-1,1]$ define a perturbation via $\delta(s,\act)=\delta^{s,\act}$ for all $s\in S$ and $\act\in\Act$.	

%\subsection{Nonlinear programming}\label{sec:nonlinear_approach}
%functions over $\Var$. An NLP is an optimization problem of the form
%As mentioned, we will provide a nonlinear program (NLP) that encodes
%Problem~\ref{prob:pmdpsyn}. 
% by providing a valuation
%$u^\star$ such that $f[u^\star]\leq f[u]$ for all
%$u\in\valuations^\Var$ \sj{while adhering to the constraints $g_i$ and $h_j$}.
%The usefulness of the NLP encoding with respect to the original problem is
%captured by the concepts of \emph{soundness} and \emph{completeness}.  An NLP
%encoding is \emph{sound} if a feasible (optimal) solution of the NLP is a
%feasible (optimal) solution of the problem encoded by the NLP.  The encoding is
%\emph{complete} if every feasible (optimal) solution of the original problem is
%a feasible (optimal) solution of the encoding.


%We refer to soundness in the sense that each variable assignment that satisfies
%the constraints induces a scheduler and a valuation of parameters such that a
%feasible solution of the problem is induced. Moreover, any optimal solution to
%the NLP induces an optimal solution of the problem. 
%
%Completeness means that all possible solutions of the problem can be encoded by this NLP; while
%unsatisfiability means that no such solution exists, making the problem



\subsection{Nonlinear encoding}
We now provide an NLP encoding of Problem~\ref{prob:pmdpsyn}. A general NLP over %valuations $u \in \valuations^\Var$ of 
a set of real-valued variables $\Var$ can be written as
\begin{align}
	\text{minimize} 		&\quad f\label{eq:nl_obj} \\
	\text{subject to} 		&\notag\\
	\forall i.\, 1\leq i\leq m 	&\quad g_i \leq 0,\label{eq:nl_ineq}\\
	\forall j.\, 1\leq i\leq p 	&\quad h_j = 0,\label{eq:nl_eq}
\end{align}
where $f$, $g_i$, and $h_j$ are arbitrary functions over $\Var$, and $m$ and $p$ are the 
number of inequality and equality constraints of the program respectively. Tools like
\tool{IPOPT}~\cite{ipopt} solve small instances of such problems.



Consider a
pMDP $\pMdpInit$ with specifications $\varphi_1=\reachProplT$ and $\varphi_2=\expRewProp{\kappa}{G}$. We will discuss how additional specifications of either type can be encoded.
The set $\Var = \Paramvar \cup W$ of variables of the NLP consists of
the variables $\Paramvar$ that occur in the pMDP as well as a set $W$ of additional variables:
\begin{itemize}
	\item $\{ \sched^{s,\alpha} \mid s \in S, \act\in\Act(s) \}$,
		which define the randomized scheduler $\sched$ by $\sched(s)(\act)=\sched^{s,\act}$.
		% Note that these scheduler variables may be assigned zero.
	\item $\{ p_s \mid s \in S \}$, 
		where $p_s$ is the probability of reaching the target set 
		$T\subseteq S$ from state $s$ under scheduler $\sched$, and
	\item $\{ c_s \mid s \in S \}$, where $c_s$ is the expected cost to reach $G\subseteq S$ from $s$ under $\sched$.
\end{itemize}
%We then lift the given objective function $f$ from the domain $\Paramvar$ to $\Var$,
%yielding $f\colon \Var\rightarrow\R$.
 A valuation over $\Var$ consists of a valuation $u\in\valuations^V$ over the
pMDP variables and a valuation $w\in\valuations^W$ over the additional variables.
\begin{align}
	\mbox{minimize } &\quad f \label{eq:min_rand}\\
	\mbox{subject to}\notag \\
					 &\quad p_{\sinit}\leq \lambda,				\label{eq:lambda}\\
					 &\quad c_{\sinit}\leq \kappa,				\label{eq:kappa}\\
	\forall s\in S.	&\quad \sum_{\act\in\Act(s)}\sched^{s,\act}=1, \label{eq:well-defined_sched_rand}\\
	\forall s\in S\,~\forall\act\in\Act(s). &\quad 0 \leq \sched^{s,\act} \leq 1,				\label{eq:sched_is_dist}\\
	\forall s\in S\,~\forall\act\in\Act(s).	 &\quad \sum_{s'\in S}\probmdp(s,\act,s')=1,	\label{eq:well-defined_probs_rand}\\
	\forall s, s'\in S\,~\forall\act\in\Act(s).	 &\quad 0 \leq \probmdp(s,\act,s') \leq 1,	\label{eq:probs_is_prob}\\
	\forall s\in T.	&\quad p_s=1,															\label{eq:targetprob_rand}\\
	\forall s\in S\setminus T. &\quad p_s=\sum_{\act\in\Act(s)}\sigma^{s,\act}\cdot\sum_{s'\in S}	\probmdp(s,\act,s')\cdot p_{s'}, \label{eq:probcomputation_rand}\\
	\forall s\in G.	 &\quad c_s=0,															\label{eq:targetrew}\\
	\forall s\in S\setminus G.	&\quad c_s= \sum_{\act\in\Act(s)} \sigma^{s,\act} \cdot \Bigl(c(s,\act) + \sum_{s'\in S}	\probmdp (s,\act,s') \cdot c_{s'}\Bigr). \label{eq:rewcomputation}
\end{align}%
The NLP~\eqref{eq:min_rand}--\eqref{eq:rewcomputation} encodes Problem~\ref{prob:pmdpsyn} in the following way.
The objective function $f$ in~\eqref{eq:min_rand} is any real-valued function over the variables $\Var$. 
The constraints~\eqref{eq:lambda} and~\eqref{eq:kappa} encode the
specifications $\varphi_1$ and $\varphi_2$, respectively.
The constraints~\eqref{eq:well-defined_sched_rand}--\eqref{eq:sched_is_dist}
ensure that the scheduler obtained is well-defined by requiring that the
scheduler variables at each state sum to unity. 
Similarly, the constraints
\eqref{eq:well-defined_probs_rand}--\eqref{eq:probs_is_prob} ensure that
for all states, parameters from $\Paramvar$ are instantiated such that
probabilities sum up to one. 
(These constraints are included if not all probabilities at a state are constant.)
The probability of reaching the target for all states in the target set is
set to one using~\eqref{eq:targetprob_rand}.
The reachability probabilities in each state 
depend on the reachability of the successor states and the transition
probabilities to those states through~\eqref{eq:probcomputation_rand}.
Analogously to the reachability probabilities, the cost for each goal state $G\subseteq S$
must be zero, thereby precluding the collection of infinite cost at
absorbing states, as enforced by~\eqref{eq:targetrew}.
Finally, the expected cost for all states except target states is given by
the equation~\eqref{eq:rewcomputation}, where according to the
strategy $\sched$ the cost of each action is added to the expected cost of
the successors. 
%Note that the probability of reaching states from $G$ needs to be one in order
%for the expected cost to be defined. 
	%The NLP works as follows.
	%First, the value of the objective function is minimized~\eqref{eq:min_rand}. The probability assigned to the initial state $\sinit\in S$ has to be smaller than or equal to $\lambda$ to satisfy $\varphi=\reachProplT$~\eqref{eq:lambda}. This is analogous for the expected cost assigned to the initial state~\eqref{eq:kappa}. To have a well-defined randomized scheduler $\sched$, we ensure that the assigned values of the corresponding strategy variables at each state sum up to one~\eqref{eq:well-defined_sched_rand}. 

	%For all target states $T\subseteq S$, the probability of the corresponding probability variables is assigned one~\eqref{eq:targetprob_rand}; analogously, for each goal state $G\subseteq S$ the expected cost of the cost variables is assigned zero, otherwise infinite cost is collected at absorbing states. The probability to reach $T\subseteq S$ from each $s\in S$ is computed in~\eqref{eq:probcomputation_rand}, defining a nonlinear equation system, where action probabilities, given by the induced strategy $\sched$, are multiplied by probability variables for all possible successors. 
	
We can readily extend the NLP to include more specifications. If
another reachability property $\varphi'=\reachProp{\lambda'}{T'}$ is given, we add the set of probability variables $\{ p'_s \mid
s \in S\}$ to $W$, and duplicate the 
constraints~\eqref{eq:targetprob_rand}--\eqref{eq:probcomputation_rand} accordingly.
%is extended to all
%states $s\in T\cup T'$, and~\eqref{eq:probcomputation_rand} is copied for all
%$p'_s$, thereby computing the probability of reaching $T'$ under the same
%scheduler as for reaching $T$. 
To ensure satisfaction of $\varphi'$, we also add the constraint
$p'_{\sinit}\leq \lambda'$.
The procedure is similar for additional expected cost properties. 
By construction, we have the following result relating the NLP encoding and Problem~\ref{prob:pmdpsyn}.
\begin{theorem}
	\label{thm:soundcomplete}
	The NLP~\eqref{eq:min_rand}--\eqref{eq:rewcomputation} is sound and complete with respect to Problem~\ref{prob:pmdpsyn}.
\end{theorem}
%
We refer to soundness in the sense that each variable assignment that satisfies
the constraints induces a scheduler and a valuation of parameters such that a
feasible solution of the problem is induced. Moreover, any optimal solution to
the NLP induces an optimal solution of the problem. Completeness means that all
possible solutions of the problem can be encoded by this NLP; while
unsatisfiability means that no such solution exists, making the problem
\emph{infeasible}.
%We refer to soundness and completeness as in Section~\ref{sec:nonlinear_approach}. 

\paragraph{Signomial programs.} By Def.~\ref{def:posy} and~\ref{def:pmdp}, all constraints in the NLP consist of signomial functions.
A special class of NLPs known as \emph{signomial programs} (SGPs) is of the form~\eqref{eq:nl_obj}--\eqref{eq:nl_eq} where $f$, $g_i$ and $h_j$ are signomials over $\Var$, see Def.~\ref{def:posy}. Therefore, we observe that the NLP~\eqref{eq:min_rand}--\eqref{eq:rewcomputation} is an SGP. We will refer to the NLP as an SGP in what follows.

SGPs with equality constraints consisting of functions that are \emph{not affine} are not \emph{convex} in general. 
In particular, the SGP~\eqref{eq:min_rand}--\eqref{eq:rewcomputation} is not necessarily convex. Consider a simple pMC only having transition probabilities of the form $p$ and $1-p$, as in Example~\ref{ex:die}. The function in the equality
constraint~\eqref{eq:probcomputation_rand} of the corresponding SGP encoding is not affine in
parameter $p$ and the probability variable $p_s$ for some state $s\in S$.
More generally, the equality constraints
\eqref{eq:well-defined_probs_rand},
\eqref{eq:probcomputation_rand}, and
\eqref{eq:rewcomputation}
involving $\probmdp$ are not necessarily affine, and thus the SGP may not be a convex program~\cite{boyd_convex_optimization}.
%If the transition probability functions are polynomials in the parameters, then the constraints~\eqref{eq:probcomputation_rand} or not even bilinear.
%For more complicated problems involving higher degree transition probability functions and nondeterminism, the nonconvexity of the constraint~\eqref{eq:probcomputation_rand} is recursively worse. 
Whereas for convex programs \emph{global optimal solutions} can be
found efficiently~\cite{boyd_convex_optimization}, such guarantees are
not given for SGPs. 
However, we can efficiently obtain local optimal solutions for SGPs in our setting, as shown in the following sections.





\documentclass[prb,aps,twocolumn,amsmath,amssymb,floatfix,superscriptaddress]{revtex4}
\usepackage[dvips]{graphics}
\usepackage{color}
\definecolor{dred}{rgb}{0,0,0.6}

\begin{document}

\title{Spin half-adder}

\author{Moumita Patra}

\email{moumita.patra19@gmail.com}

\affiliation{Department of Physics, Indian Institute of Technology Bombay,\\
Mumbai, Maharashtra 400076, India}

\begin{abstract}

A new proposal is given to design a spin half-adder in a nano-junction. It is well known that
at finite voltage a net circulating current (known as circular current) appears within a mesoscopic
ring under asymmetric ring-to-electrode interface configuration. This circular current induces
a finite magnetic field at the center of the ring. We utilize this phenomenon to construct a spin
half adder. The circular current induced magnetic field is used to regulate the alignments of
local free spins, by their orientations we specify the output states of the `sum' and `carry'.
All the outputs are spin based, therefore the results get atomically stored in the system. We
also illustrate the experimental possibilities of our proposed model. 

\end{abstract}

\maketitle

\section{Introduction}
 
The ultimate goal of modern technology is to make atomic scale devices. The
continuous shrinking in the size of the channel length of a transistor has driven the industry
from the first four-function calculators to the modern laptops~\cite{datta}. The functionality
of these atomic scale devices are based on the quantum nature of the electrons. But the movement
of charge within an information processing device always associates dissipation which makes the
device energy in-efficient~\cite{SB}. Replacement of ``electrons" by ``spin" is found to be
a most suitable way to resolve these problems. In 1990~\cite{SFET}, Datta and Das came out with
a proposal of spin-field-effect-transistor (SFET), where they used the spin degree of freedom of channel
electrons instead of the charge. Starting from the idea of SFET, till now various proposals have been
reported using spin degrees of freedom, such as spin injection into semiconductors from ferromagnetic
metals~\cite{spin1,spin2,spin3,spin4,spin5}, the development of diluted ferromagnetic
semiconductors~\cite{spin6,spin7}, etc. These devices have several advantages like,
low power consumption, speedy processing, etc. compared to the commonly used semi-conducting
devices~\cite{ref6,ref7,ref8,ref9}. Apart from these advantages, the most important factor in the
context of computation is that these devices are non-volatile in nature. Therefore, unlike charge
based microprocessor, the spintronic devices can store the output itself and we do not need any extra
memory device. For example, using magneto-resistive elements, AND, OR, NAND, and NOR gates have
been constructed with non-volatile output~\cite{ref4}. Dery {\it et. al} have designed logic gate
that consisted of a semi-conductor structure with multiple magnetic contacts~\cite{lg1}. In a recent
work Datta {\it et al.}~\cite{LGS5} have proposed all spin logic devices along with storage mechanism.
Spin-orbit interaction has been used to perform a universal logic operation utilizing minimum
possible devices~\cite{lg2}. In another work A. A. Khajetoorains {\it et al.}~\cite{lg3} have combined
bottom-up atomic fabrication with spin-resolved scanning tunneling microscopy to construct and readout
atomic-scale model systems performing logic operations.

It is always important to design spin based combinational digital circuit at atomic scale. In this article
we propose a spin half-adder using circular current induced magnetic field
\begin{figure}
{\centering\resizebox*{8cm}{6cm}{\includegraphics{Model.eps}}\par}
\caption{(Color Online). Model of the half-adder where a quantum channel containing multiple
loops is connected to the electrodes. The spin orientations
of sites 4 and 6 is taken as inputs, whereas the sum and carry are specified by the alignment
of free spin namely S and C, respectively.}
\label{fig1}
\end{figure}
in a comfirmational interface. An usual half-adder consists of an AND and XOR gates which are
independently composed of various transistors, resistors, capacitors, etc. Whereas our model has a
strikingly simple design consisting of couple of loops, where the orientations of spins denotes the
low and high states of the inputs and outputs. 

Under finite bias condition, a net current~\cite{ref10,ref11,ref12,ref13,ref14}
appears within the ring, along with the transport current (or drain current). This current
is known as circular current $I$. Circular current is analogous to the persistent current
in a Aharonov-Bhom ring, where the driving force is magnetic field. The circular current
produces a net local magnetic field $B$ at the ring center. In some cases
the magnitude of $B$ reaches $\sim$Millitesla (mT) even up to the order of $\sim$T. Such high
local magnetic field can be used to manipulate the alignment of a local spin embedded at
the center of the ring or at any point on the axis (say, $Z$-axis) passing through the
center of the ring~\cite{ref11,ref13,ref15}. Using the bias induced
circular current, here we design a half-adder where all the inputs and outputs are spin based.
The system is composed of a nano-channel sandwiched between electrodes, namely source and
drain (as shown in Fig.~\ref{fig1}). The electrodes are semi-infinite and non-magnetic.
The channel consists of multiple loops. The spin orientations (viz up and down) of sites 4 and
6 are taken as inputs for the entire operation. The outputs of the half-adder i.e., `sum'
and `carry are specified by the spin orientations of two free spins S and C, respectively
which are attached at the center of the top loop containing atomic sites 3-4-5-6, and at
the center of the whole system, respectively. The sum and carry are not
orientated at the $Z$-direction and make an angle $\theta_C$ (say, $\theta_C = 30^{\circ}$).
This can be done by the application of a constant external magnetic field. We use the
circular current induced magnetic field to tune these free spins. Under finite bias
condition the circular current so as the magnetic field vanish when the loops are symmetric,
\begin{figure}[ht]
{\centering\resizebox*{5cm}{3cm}{\includegraphics{MvsTheta.eps}}\par}
\caption{(Color online) Circular current induced magnetic field at the center
of the device as a function of $\theta_4$ and $\theta_6$ for $\theta_{11} = \theta_{13} = 180^{\circ}$
(The bias voltage $V = 0.75 V$ and the temperature $T = 100\,$K).}
\label{mag}
\end{figure}
and they reappear for asymmetric loop geometry. This is the key idea behind the logical
operations. When a large magnetic field is produced in the loop, the
corresponding free spin changes its orientation towards $Z$-direction. Whereas when
the circular current (and the associated magnetic field) vanishes, the free spin again moves
back by an angle $\theta_C$ due to the applied external constant magnetic field. In the system
the asymmetry is introduced by the different spin
orientations of sites 4 and 6.
\begin{figure*}
{\centering\resizebox*{17cm}{8cm}{\includegraphics{GR1.eps}}\par}
\caption{(Color Online). The operational principle of the spin half-adder. (a) Sketch of
standard electronics half-adder consist of an XOR and AND gate. (b) Truth table for spin
half-adder. (c) - (f) The  representation of all the spin based half-adder operations
for four input conditions i.e., (0,0), (0,1), (1,0), and (1,1), respectively. Here the
input states are specified by the spin orientations of A and B (shown in green
color). Whereas the sum and carry are specified by free spins S and C (shown in
black color), respectively.}
\label{fig2}
\end{figure*}
As the Boolean operation is based on the manipulation of local spin by the
means of circular current, this device is expected to has negligible power-consumption and delay
time~\cite{mp}, and very high endurance ($> 10^{15}$ cycle) which exceeds the requirements of various
memory use cases, including high-performance applications such as CPU level-2 and level-3
caches~\cite{en}, as we find in the spin-transfer-torque random-access memory (where local spin
is regulated by spin polarized current).

An efficient readout of electronic spins denoting the outputs sum and carry, is
required for the experimental execution. The traditional magnetic resonance techniques rely on large
ensembles of nuclear spins. Though the ultimate goal is the readout of the single spin states and there
are lots of proposals available in this direction at the present time. For example,
spin readout of nitrogen-vacancy (NV) centers in diamond is achieved by the conversion of the
electronic spin state of the nitrogen-vacancy to a charge-state distribution, followed by
single-shot readout of the charge state~\cite{sr1}. A versatile single spin meter is designed which
is consisting of a quantum dot in a magnetic field under microwave irradiation combined with a charge
counter~\cite{sr2}. Several other proposals are also available based on the physical system in which
the spin is housed~\cite{sr3} or requiring certain special features, such as optical activity~\cite{sr4},
nuclear spins~\cite{sr5} or a large detector-system interaction~\cite{sr6}, etc. These proposals make us
confident about the spin state readability of our system.

There are several characteristic features which substantiate the
robustness of the half adder.

\begin{enumerate}

\item Though in this paper we consider a simplified model of 14 atomic sites,
but the results are equally true for any same kind of system having more atomic sites. But
in each loop the number of atoms should be even such that they can be symmetrically
connected to the other part of the circuit.
	
\item So far in the literature, to construct spin logic gates, the spins are used
either as input or output variables. So spin-to-charge converter
is required for every operation,  which causes the loss of efficiency. As in
our set up  all the operations are described by only spin states, no
spin-to-charge conversion is required.

\item The spin based logic devices can store the information which is very important for
non-volatile computations in computer. On the other hand the conventional charge based
computers are volatile.

\item The spin injection efficiency and the material of the channel (semi-conductor or
metal) are two extremely important aspects for any experimental application~\cite{ref14a}. 
Metal has high spin injection efficiency~\cite{LGS5}, but the spin coherence length is
smaller for a metallic channel. On the hand, semi-conductor has high coherence length, but
inadequate spin injection efficiency. In this paper, the proposed model has a metallic channel
with smaller length (only 14 atomic sites), so that it has high spin injection efficiency,
whereas the issue of coherence is solved because of its smaller system size.
		
\item The proposed model can be reprogrammed to have various other logical
operations like NAND, NOT, OR, etc.

\item As the spin states are solely described by the up and down spins so, an
efficient mechanism to rotate local spins is required. We describe an experimental setup in
section.~V, where we use the bias induced magnetic field~\cite{ref15,cite22} to regulate
the input states.

\item The results are valid at non-zero temperature, which is very crucial for
practical applications.
		
\end{enumerate}

We arrange the paper in following manner. The theoretical method is discussed in Sec~II.
In Sec~III we present all the results.
In Sec~IV, various other logical operations are demonstrated in the same setup.
An experimental proposal is demonstrated in Sec~V and finally, we give a over view in
Sec~VI.

\section{Theoretical prescription}

To calculate the circular current in a nano junction, we use the wave-guide theory.
We start by writing the tight binding Hamiltonian of the model as shown in Fig.~\ref{fig1}.
The system consists of quantum channel connected to the electrodes. Therefore the Hamiltonian
for the entire system becomes
\begin{equation}
H= H_C + H_S + H_D + H_T
\label{eq1}
\end{equation}
Here the $H_C$, $H_S$ and $H_D$ the Hamiltonians for the channel (C),
source (S) and drain and they read as:
\begin{equation}
H_{\alpha}=\sum \epsilon_{\alpha,\sigma}c_{n,\sigma}^{\dagger}c_{n,\sigma} +
\sum\left(c_{n+1,\sigma}^{\dagger} t_{\alpha,\sigma} c_{n,\sigma} + h.c. \right)
\label{eq2}
\end{equation}
\noindent where $\alpha=C, S, D$. For the electrodes, the on-site energy and
atom-to-atom coupling become : $\epsilon_{\alpha,\sigma}=\epsilon_0$ and
$t_{\alpha,\sigma}=t_0$, respectively whereas for the channel they are
$\epsilon$ and $t$, respectively. Sites 4, 6, 11,
and 13 are magnetic. The onsite potential for these sites are:
$\epsilon - h_i.\sigma$ ($h_i = \sim |h_i|$ represents the spin-flip scattering
and $\sigma$ is Pauli matrices). The last term $H_T$ of Eq.~\ref{eq1}
is the tunneling Hamiltonian. 

In the channel, the atomic sites 4, 6, 11, and 13 are magnetic, we need to calculate
all spin dependent components of circular current $I_C$. In one of our recent
works~\cite{ref14} we have put forward the methodology to calculate the spin components of
circular current. Here we follow the same
prescription.The detailed calculation of the bond current density $J_{i,i+1}$ between
the site $i$ and $i+1$ is described in Appendix~\ref{aa}. The current at bias voltage $V$ can
be written as~\cite{ref16,ref17}
\begin{equation}
I_{i,i+1}(V) = \int\limits_{-\infty}^{\infty}J_{i,i+1}(E)[f_S(E) - f_D(E)]\, dE
\label{eq6}
\end{equation}
Where, $f_{S(D)}=\Big[1 + e^{\frac{E-\mu_{S(D)}}{k_B T}}\Big]^{-1}$ is
the Fermi function ($k_B$ is the Boltzmann constant and $T$ is the temperature)
corresponding to the source and drain and $\mu_{S(D)}$ is the corresponding
chemical potential.

In the present system we need to calculate the circular current for two loops: one is for the
top loop containing atomic sites 3, 4, 5, and 6, and other for the center loop containing
atomic sites 1-2-3-6-5-7-8-9-10-13-12-14. After calculating the corresponding bond currents
we calculate net circular current of these two loops as:
\begin{equation}
I_1 = \frac{1}{4}\left(I_{3,4} + I_{4,5} + I_{5,6} + I_{6,3}\right)
\label{eq7a}
\end{equation}
and
\begin{eqnarray}
I_2 & = & \frac{1}{12}(I_{1,2} + I_{2,3} + I_{3,6} + I_{6,5} + I_{5,7} + I_{7,8}
+ I_{8,9} \nonumber \\
& & + I_{9,10} + I_{10,13} + I_{13,12} + I_{12,14} + I_{14,1})
\label{eq7b}
\end{eqnarray}
respectively. If the current goes in anti-clockwise direction then we consider it positive,
and vise-versa. 

Net local magnetic fields are established as the circular currents flow with in the rings.
Using the Biot-Savart's law we can calculate the magnetic fields as
\begin{equation}
\vec{B_n}(\vec{r_n}) = \sum\limits_{\langle i,j \rangle} \left(\frac{\mu_0}{4\pi}
\right)
\int I_{i,j}\frac{d\vec{r^{\prime}} \times(\vec{r_n}-\vec{r^{\prime}_n})}
{|(\vec{r_n}-\vec{r^{\prime}_n})|^3}
\label{eq8}
\end{equation}
$n=1,2$. $\mu_0$ is the magnetic constant. 

Now we consider a free spin is embedded at the ring center as shown
in Fig.~\ref{fig1}. The spin is initially misaligned with $Z$
direction. With the appearance of $I_C$ and associated $B$, the spin
tries to align itself along $Z$ direction. We calculate the spin
angle of rotation $\theta_C$~\cite{ref13,cite12,cite22} by the magnetic field $B$
for a time $\tau$ as
\begin{equation}
\theta_i=g \mu_B B \tau /2\hbar
\label{eq9}
\end{equation}
$i=1,2$. $g(\approx1) \rightarrow$ Lande $g$-factor; $\mu_B \rightarrow$ Bohr magneton.

\section{Numerical Results and Discussion}

As the functionality of the half-adder depends on the appearance of
circular induced magnetic field in asymmetric situation, we want to examine the dependence of the
induced magnetic field on the system-asymmetry. In Fig.~\ref{mag}, we plot the magnetic field
produced at the center
of the device with the angle of rotation of the inputs-A and B. We consider spin-11 and 13
to be down. All the $h_i$'s are $0.25\,$eV. Here we find that a large magnetic field is produced
when the system has the most asymmetry and it smoothly varies towards zero as the
orientations of A and B become similar to spins-11 and 13, and this is the key factor of our proposal.
For the execution of logic operation the appearance of zero circular current is a prime requirement.
This demands an ideal symmetric condition, which seems to be unrealistic in
real situation. But we can see in Fig.~\ref{mag} that the produced magnetic field is insufficient to
rotate the spin along $Z$ for a considerable range around the symmetry point
($\theta_4 = \theta_6 = 180^{\circ}$). Therefore we can argue that, even if there is little
asymmetries due to manufacturing imperfection and other factors, but this proposal will
still be equally valid.

Now We explain the half-adder operation.
\begin{figure}[ht]
{\centering\resizebox*{8cm}{5cm}{\includegraphics{BvsV.eps}}\par}
\caption{(Color online) (a)-(d) Produced magnetic fields $B_1$ (red curve) and
$B_2$ (black curve) associated with the sum and carry, respectively for the four input conditions.
Here we set $T = 250\,$K.}
\label{fig3}
\end{figure}
In Fig.~\ref{fig2} we present the circuit diagram and the truth table for spin half adder.
The input states are specified by the spin orientations of
spin-A and B as $\downarrow : \rightarrow 0$ and $\uparrow : \rightarrow 1$.
Whereas the output conditions are specified by the spin alignment of free
spin S (represents sum) and C (i.e., carry). The mechanisms
of sum and carry are as follows:

\vskip 0.1cm
\noindent
{\bf \underline{Sum}:} We assume that the free spins initially are not aligned
along $Z$-direction. The output for the sum is defined as: if the free spin S is in its
initial position, then the output is 0, and if it is aligned along $Z$, it
represents 1. The alignment of the individual free spin depends
on the appearance of current induced magnetic field in
each loop. So  there is no magnetic field when both A and B are
parallel, i.e., when both are either up or down (see Fig.~\ref{fig2}(c) and
\begin{table}[ht]
\caption{Truth tables for different parallel logical operations.}
$~$
\vskip -0.25cm
\fontsize{7}{11}
\begin{tabular}{|c|c|c|c|}
\hline
\multicolumn{2}{|c|}{Input} & {Sum ($|B_1|$ in mT)} & {Carry ($|B_2|$ in mT)}\\
\hline
A & B & S & C \\
 \hline
$\downarrow$ & $\downarrow$ & 0 &  62.7\\
$\downarrow$ & $\uparrow$ & 101 & 30.3 \\
$\uparrow$ & $\downarrow$ & 101 &  41.5\\
$\uparrow$ & $\uparrow$ & 0 & 0 \\
 \hline
\end{tabular}
\label{tab1}
\end{table}
(d)). In these cases S remains in its initial position such
that logical output becomes 0. When they are anti-parallel (shown in
Fig.~\ref{fig2}(e) and (f)) the corresponding loop becomes asymmetry and a
net magnetic field is produced along $Z$ direction and the free spin S follows
the field. These situations imply 1. So, this part of the circuit behaves an
XOR gate which is the sum of the half adder.

\vskip 0.1cm
\noindent
{\bf \underline{Carry}:} To generate carry we consider the
full system, and place a free spin C at the center of the whole system.
The output for the carry is assumed to be 0 when C is aligned along $Z$,
otherwise it is 1. As the lower loop contains two up spins,
the whole system becomes symmetric only when
both the inputs (i.e., A and B in upper loop) are up. So no magnetic field is
developed at the center of the circuit and C remains in its initial position
(Fig.~\ref{fig2}(f)). This situation implies 1. For all three input conditions
the system is asymmetric and a net magnetic field produced at the center which
turns the free spin C along $Z$ direction (as shown in Fig.~\ref{fig2}(c) - (e))
hence the output becomes $0$ in these cases. So AND behavior is accomplished at C
and hence the construction of half adder is accomplished. 

In Fig.~\ref{fig3} we plot the produced magnetic fields $B_1$ and $B_2$ associated
with the operation sum and carry, respectively as a function of voltage for
four different input conditions (i.e., when inputs A and B are: $(\downarrow, \downarrow)$,
$(\downarrow, \uparrow)$, $(\uparrow, \downarrow)$, and $(\uparrow, \uparrow)$ and shown in
Fig.~\ref{fig3}(a)-(d), respectively). Here we set, all site energies to zero, hopping
integral in contacting leads at $1\,$eV, and all the $t_i$s' in the ring at $0.5\,$eV and
\begin{figure}[ht]
{\centering\resizebox*{8cm}{3cm}{\includegraphics{LG.eps}}\par}
\caption{(Color online) Layouts of different logic gates.}
\label{lg}
\end{figure}
the ring-to-lead couplings at $0.5\,$eV. The magnitude of the net magnetization at sites
4, 6, 11, and 13 are $0.5\,$eV. The calculations are done considering $250\,$K temperature. The
average atomic distance $a$ is considered to be $1~\AA$. The red curve represents the magnetic
field corresponding to sum (i.e. $B_1$) whereas black one represents $B_2$ which is the
magnetic field associated to carry. Let the free spins corresponding to S and C  are
initially set at $30^{\circ}$ . When the loops become asymmetric, the circular currents
hence net magnetic fields will produce in each loop that will turn the free spins S and
C towards $Z$ direction. Considering the desired operation time as $\tau=5\,$ns~\cite{cite22},
we can
calculate the desired magnetic field to align the spin along $Z$ is $\sim 2.4\,$mT (as
follows from Eq.~\ref{eq9}). So we need at least $\sim 2.4\,$mT to execute all the logic
operations. For all the four cases of Fig.~\ref{fig3} we find large enough magnetic fields
are produced which are more than sufficient to turn S and C in appropriate cases. On the
other hand for the proper cases the magnetic fields are exactly zero, which will leave
the S and C in its initial positions. As the results remain valid for large ranges of voltage
and temperature, we can expect that the proposed model might be implemented in the
laboratory. 

The quantitative representation of half adder is shown in Table~\ref{tab1}. Here all the
parameters are chosen to be the same as Fig.~\ref{fig3} and magnetic fields are evaluated at
bias voltage $0.5\,$V.

\section{Reprogrammable spin logic gate}

In this paper our main motivation is to construct spin half-adder, though
other logical operations can be accomplished by reprogramming the same system. For
example in Fig.~\ref{lg}, we have shown the sketches for NAND and NOT gates. The
spins shown in green color represent the inputs and O represents the output. As the
logical operations follow the symmetry conditions of the ring, we accordingly
set the spin orientations of the other sites in the system. For NAND gate (Fig.~\ref{lg}(a))
spin-11 and 13 are set to be down. So only for ($\downarrow$, $\downarrow$) input condition,
there is no circular current appears at the center of the system and the output becomes
zero. But for all other three cases the outputs are 1, which is a NAND gate response.
For NOT gate (Fig.~\ref{lg}(b)), there is one input i.e., A, and other spin are considered
to be up. So if A is down, output becomes 1 and vice-versa.

In a similar fashion, we can reprogram this model to have other logical operations
also, which definitely implies the versatility of our proposal.

\section{Experimental setup of half adder}

To have the input conditions 0 and 1, we need to align the
spins in the magnetic ring selectively several prescriptions are available to control single electron spin.
For instance, using radio frequency pulses these spins can be manipulated~\cite{pl22,pl23,pl24},
though in this case relatively larger time scale is required to operate the spins. On the other hand,
\begin{figure}[ht]
{\centering
\resizebox*{4cm}{4cm}{\includegraphics{ExpModel.eps}}
\resizebox*{4cm}{2.5cm}{\includegraphics{IvstC1.eps}}\par}
\caption{(Color online). Left figure: A nano ring is connected to the electrodes in most
asymmetrically. As there exists a shunting path between the electrodes S and D, electrons
can directly hop between the them. Right figure: Modulation of $I$  and
magnetic field $B$  with the hopping parameter $t_C$. The ring has 20 atomic sites,
its radius is $10$\AA$\,$. $B$ is calculated at a distance $20\,$\AA$\,$ from the
center of the ring to the point of measurement. The applied voltage is
considered to be $V=0.5\,$V. The onsite potential $\epsilon_r$ and hopping integral
$t_r$ for the entire system are considered as: $\epsilon_r=0$ and $t_r=1\,$eV.}
\label{f5}
\end{figure}
the manipulations can be made much faster such as in the picosecond or femtosecond time scale with
the help of optical pulses~\cite{pl25,pl26,pl27}. In another pioneering work Press et al. have shown that
the selective tuning of electron spins are possible within the spins' coherence times by means of
ultrafast laser pulses~\cite{pl28}. With the availability of these various sophisticated prescriptions, we
strongly believe that the alignment of selective spins (i.e., 4 and 6) can be properly adjusted.

Apart from the above mentioned proposals here we present another suitable method for the proper regulations
of the spin specifying input conditions. We make use of the circular current induced magnetic field to
regulate the inputs and for that matter tuning of the magnetic field externally is required.
We take a nano ring with two side attached metallic electrodes (S and D) connected
to the two adjacent sites of the ring as shown in the left figure of Fig.~\ref{f5}.
As the electrodes are connected at the two neighboring sites of the ring,
with a finite probability, the electrons can directly hop from S to D. Let $t_C$ be
related hopping integral between the electrodes. By changing relative positions of the
electrodes we can tune $t_C$, and tuning $t_C$ we can regulate the circular current
induced magnetic field $B$ for a large scale. This proposal has already been discussed
in on of our previous work~\cite{ref21}. For an example, taking an 20-site ring we
show the variation of
the magnetic field with source-to-drain coupling $t_C$ in the right figure
of Fig.~\ref{f5} where we follow the same theoretical prescription for the calculation
of current and magnetic filed as described in Sec.~II. From the result we can conclude
that regulating the tunneling between the side attached electrodes the local magnetic $B$
field can be tuned a large scale which is required for flipping of spin states (i.e., up or
down). Such nano junctions needed to be put on the sites 4 and 6 (as shown in
in the right of Fig.~\ref{f5}, the distance of the ring center to the sites 4 and 6
will be $20\,$\AA$\,$) and in every cases changing the shunting paths between
source-to-drain we can specify the required input conditions.

\section{Closing Remarks}

In conclusion, we have theoretically realized a spin half-adder where all the
inputs are output conditions are completely spin based. The outputs preserve the
memories. The logic operations have been implemented in a nano channel containing multiple
loops. The basic mechanism of replies on the appearance
of circular current and associated magnetic field under a finite bias condition.
We have used the magnetic field to rotate free spins embedded at the center of the loops.
By the proper chooses of their orientations we have specified the 0 and 1 states of `sum' and
`carry' of the half-adder. The proposal is well tested at non-zero temperature and a well
suited experimental setup is discussed for input spins regulations. We have
shown various other logical operation in the same setup utilizing the bias induced circular
current.

In a real situation, many possible sources are there those may destroy
phase and spin memory of electrons, and among them the most common source is electron-phonon
(e-ph), the stray field, and other impurities. Theoretically one can incorporate these effects
by studying the dephasing effecting on the current by introducing phenomenological voltage
probes into the system. The effect of dephasing and impurities on circular current are
different compared to the transport current, which generally decreases with these factors.
On the other hand, circular current may increase in the presence of dephasing
and disorder~\cite{ref14}. Therefore we strongly believe that the results presented here
will be still valid in real experiment.

Though results displayed here are calculated for a specific system, but this proposal will be well
suited for same kind of geometry having any arbitrary number of atomic sites in each loop. The spin
half adder along with other logical operations, discussed here will certainly boost the new
generation computations along with storage mechanism.

\section{Acknowledgements}

The author gratefully acknowledge the fruitful discussions with Prof. Alok Shukla and
Prof. S. K. Maiti. The work has been done with the financial support (post-doctoral
fellowship) from Indian Institute of Technology, Bombay, India.

\appendix

\section{Circular current density}
\label{aa}

The wave guide formalism~\cite{ref14,ref18,ref19,ref20,ref21} involves a set of linear coupled
equations which are obtained from the Schr\"{o}dinger equation $H|\phi\rangle = E|\phi\rangle$
with $|\phi\rangle = \left[\sum A_{n,\sigma} a_{n,\sigma}^{\dagger} + \sum B_{n,\sigma}
b_{n,\sigma}^{\dagger} + \sum C_{i,\sigma} c_{i,\sigma}^{\dagger}\right]|0\rangle$.
The coefficients $A_{n,\sigma}, B_{n,\sigma}$, and $C_{n,\sigma}$
are the wave-amplitude corresponding to the $n$th site of the electrodes (namely, source
and drain), where as $i$ represent the site index of the ring. Let an up spin injected from the
source to the channel as a plane wave with unit amplitude. For our present setup as shown in
Fig.~\ref{fig1}, we have the equations as follows:
{
\allowdisplaybreaks
\begin{widetext}
{\footnotesize
\begin{eqnarray}
\left[\left(\begin{array}{cc}
        E & 0 \\
        0 & E
\end{array}\right) - \left(\begin{array}{cc}
    \epsilon_0 & 0 \\
    0 & \epsilon_0
\end{array}\right)\right]\left(\begin{array}{cc}
        1 + \rho_{\uparrow\uparrow} \\
        \rho_{\uparrow\downarrow}
    \end{array}\right) = \left(\begin{array}{cc}
    t_0 & 0 \\
    0 & t_0
\end{array}\right) \left(\begin{array}{cc}
    e^{-ika} + \rho_{\uparrow\uparrow}e^{ika} \\
        \rho_{\uparrow\downarrow}e^{ika}
    \end{array}\right) + \left(\begin{array}{cc}
   t_{S}  & 0 \\
    0 & t_{S}
\end{array}\right) \left(\begin{array}{cc}
   C_{1,\uparrow\uparrow}  & 0 \\
    0 & C_{1,\uparrow\downarrow}
\end{array}\right)\nonumber \\
\left[\left(\begin{array}{cc}
    E & 0 \\
    0 & E
\end{array}\right) - \left(\begin{array}{cc}
    \epsilon & 0 \\
    0 & \epsilon
\end{array}\right)\right] \left(\begin{array}{cc}
   C_{1,\uparrow\uparrow}  & 0 \\
    0 & C_{1,\uparrow\downarrow}
\end{array}\right) = \left(\begin{array}{cc}
    t_{S} & 0 \\
    0 & t_{S}
\end{array}\right)\left(\begin{array}{cc}
    1 + \rho_{\uparrow\uparrow(S)} \\
    \rho_{\uparrow\downarrow(S)}
    \end{array}\right) + \left(\begin{array}{cc}
    t & 0 \\
    0 & t
\end{array}\right) \left(\begin{array}{cc}
   C_{2,\uparrow\uparrow}  & 0 \\
    0 & C_{2,\uparrow\downarrow}
\end{array}\right) + \left(\begin{array}{cc}
    t & 0 \\
    0 & t
\end{array}\right) \left(\begin{array}{cc}
   C_{14,\uparrow\uparrow}  & 0 \\
    0 & C_{14,\uparrow\downarrow}
\end{array}\right)
\nonumber \\
\left[\left(\begin{array}{cc}
    E & 0 \\
    0 & E
\end{array}\right) - \left(\begin{array}{cc}
    \epsilon & 0 \\
    0 & \epsilon
\end{array}\right)\right] \left(\begin{array}{cc}
   C_{2,\uparrow\uparrow}  & 0 \\
    0 & C_{2,\uparrow\downarrow}
\end{array}\right) = \left(\begin{array}{cc}
    t & 0 \\
    0 & t
\end{array}\right) \left(\begin{array}{cc}
   C_{1,\uparrow\uparrow}  & 0 \\
    0 & C_{1,\uparrow\downarrow}
\end{array}\right) + \left(\begin{array}{cc}
    t & 0 \\
    0 & t
\end{array}\right) \left(\begin{array}{cc}
   C_{3,\uparrow\uparrow}  & 0 \\
    0 & C_{3,\uparrow\downarrow}
\end{array}\right)\nonumber \\
\left[\left(\begin{array}{cc}    
    E & 0 \\
    0 & E
\end{array}\right) - \left(\begin{array}{cc}
    \epsilon & 0 \\
    0 & \epsilon
\end{array}\right)\right] \left(\begin{array}{cc}
   C_{3,\uparrow\uparrow}  & 0 \\
    0 & C_{3,\uparrow\downarrow}
\end{array}\right) = \left(\begin{array}{cc}
    t & 0 \\
    0 & t
\end{array}\right) \left(\begin{array}{cc}
   C_{2,\uparrow\uparrow}  & 0 \\
    0 & C_{2,\uparrow\downarrow}
\end{array}\right) + \left(\begin{array}{cc}
    t & 0 \\
    0 & t
\end{array}\right) \left(\begin{array}{cc}
   C_{4,\uparrow\uparrow}  & 0 \\
    0 & C_{4,\uparrow\downarrow}
\end{array}\right) + \left(\begin{array}{cc}
    t & 0 \\
    0 & t
\end{array}\right) \left(\begin{array}{cc}
   C_{6,\uparrow\uparrow}  & 0 \\
   0 & C_{6,\uparrow\downarrow}
\end{array}\right) \nonumber \\
\left[\left(\begin{array}{cc}
    E & 0 \\
    0 & E
\end{array}\right) - \left(\begin{array}{cc}
    \epsilon + h_4\cos\vartheta_4 & \sin\vartheta_4e^{-i\varphi_4} \\ 
    \sin\vartheta_4e^{i\varphi_4} & \epsilon - h_4\cos\vartheta_4
\end{array}\right)\right] \left(\begin{array}{cc}
   c_{4,\uparrow\uparrow}  & 0 \\
    0 & c_{4,\uparrow\downarrow}
\end{array}\right)
  = \left(\begin{array}{cc}
    t & 0 \\
    0 & t
\end{array}\right) \left(\begin{array}{cc}
    C_{3,\uparrow\uparrow}  & 0 \\
    0 & C_{3,\uparrow\downarrow}
\end{array}\right) +  \left(\begin{array}{cc}
    t & 0 \\
    0 & t
\end{array}\right) \left(\begin{array}{cc}
    C_{5,\uparrow\uparrow}  & 0 \\
    0 & C_{5,\uparrow\downarrow}
\end{array}\right)\nonumber \\
\left[\left(\begin{array}{cc}    
    E & 0 \\
    0 & E
\end{array}\right) - \left(\begin{array}{cc}
    \epsilon & 0 \\
    0 & \epsilon
\end{array}\right)\right] \left(\begin{array}{cc}
    C_{5,\uparrow\uparrow}  & 0 \\
    0 & C_{5,\uparrow\downarrow}
\end{array}\right) = \left(\begin{array}{cc}
    t & 0 \\
    0 & t
\end{array}\right) \left(\begin{array}{cc}
    C_{4,\uparrow\uparrow}  & 0 \\
    0 & C_{4,\uparrow\downarrow}
\end{array}\right) + \left(\begin{array}{cc}
    t & 0 \\
    0 & t
\end{array}\right) \left(\begin{array}{cc}
    C_{6,\uparrow\uparrow}  & 0 \\
    0 & C_{6,\uparrow\downarrow}
\end{array}\right) + \left(\begin{array}{cc}
    t & 0 \\
    0 & t
\end{array}\right) \left(\begin{array}{cc}
    C_{7,\uparrow\uparrow}  & 0 \\
    0 & C_{7,\uparrow\downarrow}
\end{array}\right) \nonumber \\
\left[\left(\begin{array}{cc}
    E & 0 \\
    0 & E
\end{array}\right) - \left(\begin{array}{cc}
    \epsilon + h_6\cos\vartheta_6 & \sin\vartheta_6e^{-i\varphi_6} \\ 
    \sin\vartheta_6e^{i\varphi_6} & \epsilon - h_6\cos\vartheta_6
\end{array}\right)\right] \left(\begin{array}{cc}
    C_{6,\uparrow\uparrow}  & 0 \\
    0 & C_{6,\uparrow\downarrow}
\end{array}\right)
  = \left(\begin{array}{cc}
    t & 0 \\
    0 & t
\end{array}\right) \left(\begin{array}{cc}
    C_{5,\uparrow\uparrow}  & 0 \\
    0 & C_{5,\uparrow\downarrow}
\end{array}\right) +  \left(\begin{array}{cc}
    t & 0 \\
    0 & t
\end{array}\right) \left(\begin{array}{cc}
    C_{3,\uparrow\uparrow}  & 0 \\
    0 & C_{3,\uparrow\downarrow}
\end{array}\right)\nonumber \\
\left[\left(\begin{array}{cc}
    E & 0 \\
    0 & E
\end{array}\right) - \left(\begin{array}{cc}
    \epsilon & 0 \\
    0 & \epsilon
\end{array}\right)\right] \left(\begin{array}{cc}
    C_{7,\uparrow\uparrow}  & 0 \\
    0 & C_{7,\uparrow\downarrow}
\end{array}\right) = \left(\begin{array}{cc}
    t & 0 \\
    0 & t
\end{array}\right) \left(\begin{array}{cc}
    C_{5,\uparrow\uparrow}  & 0 \\
    0 & C_{5,\uparrow\downarrow}
\end{array}\right) + \left(\begin{array}{cc}
    t & 0 \\
    0 & t
\end{array}\right) \left(\begin{array}{cc}
    C_{8,\uparrow\uparrow}  & 0 \\
    0 & C_{8,\uparrow\downarrow}
\end{array}\right)\nonumber \\
\left[\left(\begin{array}{cc}
    E & 0 \\
    0 & E
\end{array}\right) - \left(\begin{array}{cc}
    \epsilon & 0 \\
    0 & \epsilon
\end{array}\right)\right]\left(\begin{array}{cc}
    C_{8,\uparrow\uparrow}  & 0 \\
    0 & C_{8,\uparrow\downarrow}
\end{array}\right)
= \left(\begin{array}{cc}
   t & 0 \\
    0 & t
\end{array}\right) \left(\begin{array}{cc}
   C_{7,\uparrow\uparrow}  & 0 \\
    0 & C_{7,\uparrow\downarrow}
\end{array}\right) + \left(\begin{array}{cc}
   t & 0 \\
    0 & t
\end{array}\right) \left(\begin{array}{cc}
   C_{9,\uparrow\uparrow}  & 0 \\
    0 & C_{9,\uparrow\downarrow}
\end{array}\right) + \left(\begin{array}{cc}
   t_D & 0 \\
    0 & t_D
\end{array}\right) \left(\begin{array}{cc}
        \tau_{\uparrow\uparrow}e^{ika} \\
    \tau_{\uparrow\downarrow}e^{ika}
    \end{array}\right)\nonumber \\
\left[\left(\begin{array}{cc}
    E & 0 \\
    0 & E
\end{array}\right) - \left(\begin{array}{cc}
    \epsilon & 0 \\
    0 & \epsilon
\end{array}\right)\right] \left(\begin{array}{cc}
   C_{9,\uparrow\uparrow}  & 0 \\
    0 & C_{9,\uparrow\downarrow}
\end{array}\right) = \left(\begin{array}{cc}
    t & 0 \\
    0 & t
\end{array}\right) \left(\begin{array}{cc}
   C_{8,\uparrow\uparrow}  & 0 \\
    0 & C_{8,\uparrow\downarrow}
\end{array}\right) + \left(\begin{array}{cc}
    t & 0 \\
    0 & t
\end{array}\right) \left(\begin{array}{cc}
   C_{10,\uparrow\uparrow}  & 0 \\
    0 & C_{10,\uparrow\downarrow}
\end{array}\right)\nonumber \\
\left[\left(\begin{array}{cc}    
    E & 0 \\
    0 & E
\end{array}\right) - \left(\begin{array}{cc}
    \epsilon & 0 \\
    0 & \epsilon
\end{array}\right)\right] \left(\begin{array}{cc}
   C_{10,\uparrow\uparrow}  & 0 \\
    0 & C_{10,\uparrow\downarrow}
\end{array}\right) = \left(\begin{array}{cc}
    t & 0 \\
    0 & t
\end{array}\right) \left(\begin{array}{cc}
   C_{11,\uparrow\uparrow}  & 0 \\
    0 & C_{11,\uparrow\downarrow}
\end{array}\right) + \left(\begin{array}{cc}
    t & 0 \\
    0 & t
\end{array}\right) \left(\begin{array}{cc}
   C_{13,\uparrow\uparrow}  & 0 \\
    0 & C_{13,\uparrow\downarrow}
\end{array}\right) + \left(\begin{array}{cc}
    t & 0 \\
    0 & t
\end{array}\right) \left(\begin{array}{cc}
   C_{9,\uparrow\uparrow}  & 0 \\
    0 & C_{9,\uparrow\downarrow}
\end{array}\right) \nonumber \\
\left[\left(\begin{array}{cc}
    E & 0 \\
    0 & E
\end{array}\right) - \left(\begin{array}{cc}
	\epsilon + h_{11}\cos\vartheta_{11} & \sin\vartheta_{11}e^{-i\varphi_{11}} \\ 
	\sin\vartheta_{11}e^{i\varphi_{11}} & \epsilon - h_{11}\cos\vartheta_{11}
\end{array}\right)\right] \left(\begin{array}{cc}
   C_{11,\uparrow\uparrow}  & 0 \\
    0 & C_{11,\uparrow\downarrow}
\end{array}\right)
  = \left(\begin{array}{cc}
    t & 0 \\
    0 & t
\end{array}\right) \left(\begin{array}{cc}
   C_{10,\uparrow\uparrow}  & 0 \\
    0 & C_{10,\uparrow\downarrow}
\end{array}\right) +  \left(\begin{array}{cc}
    t & 0 \\
    0 & t
\end{array}\right) \left(\begin{array}{cc}
   C_{12,\uparrow\uparrow}  & 0 \\
    0 & C_{12,\uparrow\downarrow}
\end{array}\right)\nonumber \\
\left[\left(\begin{array}{cc}    
    E & 0 \\
    0 & E
\end{array}\right) - \left(\begin{array}{cc}
    \epsilon & 0 \\
    0 & \epsilon
\end{array}\right)\right] \left(\begin{array}{cc}
    C_{12,\uparrow\uparrow}  & 0 \\
    0 & C_{12,\uparrow\downarrow}
\end{array}\right) = \left(\begin{array}{cc}
    t & 0 \\
    0 & t
\end{array}\right) \left(\begin{array}{cc}
   C_{11,\uparrow\uparrow}  & 0 \\
    0 & C_{11,\uparrow\downarrow}
\end{array}\right) + \left(\begin{array}{cc}
    t & 0 \\
    0 & t
\end{array}\right) \left(\begin{array}{cc}
   C_{14,\uparrow\uparrow}  & 0 \\
    0 & C_{14,\uparrow\downarrow}
\end{array}\right) + \left(\begin{array}{cc}
    t & 0 \\
    0 & t
\end{array}\right) \left(\begin{array}{cc}
   C_{13,\uparrow\uparrow}  & 0 \\
    0 & C_{13,\uparrow\downarrow}
\end{array}\right) \nonumber \\
\left[\left(\begin{array}{cc}
    E & 0 \\
    0 & E
\end{array}\right) - \left(\begin{array}{cc}
	\epsilon + h_{13}\cos\vartheta_{13} & \sin\vartheta_{13}e^{-i\varphi_{13}} \\ 
	\sin\vartheta_{13}e^{i\varphi_{13}} & \epsilon - h_{13}\cos\vartheta_{13}
\end{array}\right)\right] \left(\begin{array}{cc}
   C_{13,\uparrow\uparrow}  & 0 \\
    0 & C_{13,\uparrow\downarrow}
\end{array}\right)
  = \left(\begin{array}{cc}
    t & 0 \\
    0 & t
\end{array}\right) \left(\begin{array}{cc}
   C_{12,\uparrow\uparrow}  & 0 \\
    0 & C_{12,\uparrow\downarrow}
\end{array}\right) +  \left(\begin{array}{cc}
    t & 0 \\
    0 & t
\end{array}\right) \left(\begin{array}{cc}
   C_{10,\uparrow\uparrow}  & 0 \\
    0 & C_{10,\uparrow\downarrow}
\end{array}\right)\nonumber \\
\left[\left(\begin{array}{cc}
    E & 0 \\
    0 & E
\end{array}\right) - \left(\begin{array}{cc}
    \epsilon & 0 \\
    0 & \epsilon
\end{array}\right)\right] \left(\begin{array}{cc}
   C_{14,\uparrow\uparrow}  & 0 \\
    0 & C_{14,\uparrow\downarrow}
\end{array}\right) = \left(\begin{array}{cc}
    t & 0 \\
    0 & t
\end{array}\right) \left(\begin{array}{cc}
   C_{12,\uparrow\uparrow}  & 0 \\
    0 & C_{12,\uparrow\downarrow}
\end{array}\right) + \left(\begin{array}{cc}
    t & 0 \\
    0 & t
\end{array}\right) \left(\begin{array}{cc}
   C_{1,\uparrow\uparrow}  & 0 \\
    0 & C_{1,\uparrow\downarrow}
\end{array}\right)\nonumber \\
\left[\left(\begin{array}{cc}
    E & 0 \\
    0 & E
\end{array}\right) - \left(\begin{array}{cc}
    \epsilon & 0 \\
    0 & \epsilon_0
\end{array}\right)\right] \left(\begin{array}{cc}
    \tau_{\uparrow\uparrow}e^{ika} \\
    \tau_{\uparrow\downarrow}e^{ika}
    \end{array}\right)
 = \left(\begin{array}{cc}
   t_D & 0 \\
    0 & t_D
\end{array}\right) \left(\begin{array}{cc}
   C_{8,\uparrow\uparrow}  & 0 \\
    0 & C_{8,\uparrow\downarrow}
\end{array}\right) + \left(\begin{array}{cc}
    t_0 & 0 \\
    0 & t_0
\end{array}\right) \left(\begin{array}{cc}
    \tau_{\uparrow\uparrow}e^{2ika} \\
    \tau_{\uparrow\downarrow}e^{2ika}
    \end{array}\right) \nonumber \\
\label{eq3}
\end{eqnarray}}
\end{widetext}
The parameters $t_S$ and $t_D$ represent the coupling between source-to-channel and
channel-to-drain, respectively. $\rho$ and $\tau$ are to the reflection and transmission
probabilities, respectively. $\vartheta_i$ be the polar angle whereas $\varphi_i$
is the azimuthal angle. $k$ is wave vector and $a$ is the atomic length. By solving 
Eq.~\ref{eq3}, we get bond current densities for sites $i$ and $i+1$ of the channel as:
\begin{equation}
J_{i,i+1\sigma\sigma'}=\frac{(2e/\hbar)\mbox{Im}\left[t\,C_{i,\sigma\sigma'}^*
C_{i+1,\sigma\sigma'} \right]}{(2e/\hbar)(1/2)t_0\sin(ka)}
\label{eq4}
\end{equation}
In the above expressions, $\sigma$ is used for the incident spin, while $\sigma'$ represents
the transmitting spin.

Similarly, for the down spin incidence, we get a set of equations like
Eq.~\ref{eq3} and we evaluate $J_{i\rightarrow i+1\downarrow\downarrow}$ and 
$J_{i\rightarrow i+1\downarrow\uparrow}$. With all these components of circular bond current
densities now we define the net bond current density $J_{i,i+1}$ as $J_{i,i+1} =
J_{i,i+1\uparrow\uparrow} + J_{i,i+1\uparrow\downarrow} +
J_{i,i+1\downarrow\downarrow} + J_{i, i+1\downarrow\uparrow}$.

\begin{thebibliography}{99}

\bibitem{datta} S. Datta, Lessons from Nanoelectrctonics: A New Perspective
on Transport (World Scientific, 2012).

\bibitem{SB} S. Bandyopadhyay and M. Cahay, Nanotechnology \textbf{20}, 412001
(2009).

\bibitem{SFET} S. Datta and B. Das, Appl. Phys. Lett. \textbf{56}, 665 (1990).

\bibitem{spin1} H. J. Zhu {\it et al.}, Phys. Rev. Lett.
\textbf{87}, 016601 (2001).

\bibitem{spin2} A. T. Hanbicki {\it et al.}, Appl. Phys. Lett. \textbf{82},
4092 (2003).

\bibitem{spin3} X. Jiang {\it et al.}, Phys. Rev. Lett. \textbf{94}, 056601 (2005).

\bibitem{spin4} S. A. Crooker {\it et al.}, Science \textbf{309}, 2191 (2005).

\bibitem{spin5} X. Lou {\it et al.} Nature Phys. \textbf{3}, 197 (2007).

\bibitem{spin6} Y. Ohno {\it et al.} Nature \textbf{402}, 790 (1999).

\bibitem{spin7} T. Dietl, H. Ohno, F. Matsukura, J. Cibert, and D. Zener Ferrand,
Science \textbf{287}, 1019 (2000).

%\bibitem{ref5} M. Patra and S. K. Maiti, Europhys. Lett. \textbf{121}, 38004 (2018).

\bibitem{ref6} S. A. Wolf {\it et al.}, Science \textbf{294}, 1488 (2001).

\bibitem{ref7} D. E. Nikonov, G. I. Bourianoff, and P. A. Gargini, J. Supercond. Novel
Magn. \textbf{19}, 497 (2006).

\bibitem{ref8} M. Johnson and R. H. Silsbee, Phys. Rev. Lett. \textbf{55}, 1790 (1985).

\bibitem{ref9} M. N. Baibich {\it et al.}, Phys. Rev. Lett., \textbf{61}, 2472 (1988).

\bibitem{ref4} A. Ney, C. Pampuch, R. Koch, and K. H. Ploog, Nature \textbf{425}, 485 (2003).

\bibitem{lg1} H. Dery, P. Dalal, L. Cywi$\acute{\mbox{n}}$ski and \L. J.
Sham, Nature, \textbf{447} 573 (2007).

\bibitem{LGS5} B. Behin-Aein, D. Datta, S. Salahuddin, S. Datta, Nature
Nanotech. \textbf{6}, 266 (2010).

\bibitem{lg2} M. Kazemi, Sci. Rep., \textbf{7} 15358 (2017).

\bibitem{lg3} A. A. Khajetoorians, J. Wiebe, B. Chilian, and R. Wiesendanger,
Science \textbf{332}, 1062 (2011).

\bibitem{ref10} D. Rai, O. Hod, and A. Nitzan, J. Phys. Chem. C \textbf{114}, 20583 (2010).

\bibitem{ref11} D. Rai, O. Hod, and  A. Nitzan, Phys. Rev. B \textbf{85}, 155440 (2012).

\bibitem{ref12} M. Patra and  S.K. Maiti, Sci. Rep. \textbf{7} 43343 (2017).

\bibitem{ref13} M. Patra and S. K. Maiti, Org. Electron. \textbf{62}, 454 (2018).

\bibitem{ref14} M. Patra and S. K. Maiti, Phys. Rev. B \textbf{100}, 165408 (2019).

\bibitem{ref15} K. Tagami, M. Tsukada, Curr. Appl. Phys. \textbf{3} 439 (2003).

\bibitem{mp} M. Patra, A. Shukla, and S. K. Maiti, J. Phys. D: Appl. Phys.
\textbf{54}, 095001 (2021).

\bibitem{en} H. Cai {\it et al.}, Appl. Sci. \textbf{7}, 929 (2017).

\bibitem{sr1} B. J. Shields, Q. P. Unterreithmeier, N. P. de Leon, H. Park, and
M. D. Lukin, Phys. Rev. Lett. \textbf{114}, 136402 (2015).

\bibitem{sr2} J. Wabnig and B. W. Lovett, New J. Phys. \textbf{11}, 043031 (2009).

\bibitem{sr3} R. Hanson and D. D. Awschalom, Nature \textbf{453} 1043 (2008).

\bibitem{sr4} J. R. Maze {\it et al.}, Nature \textbf{455} 644 (2008).

\bibitem{sr5} M. Sarovar, K. C. Young K C, T. Schenkel and B. K. Whaley,
 Phys. Rev. B \textbf{78}, 245302 (2008).

\bibitem{sr6} H-Z Lu and S-Q Shen, Phys. Rev. B \textbf{77}, 235309 (2008).

\bibitem{ref14a} M. Patra and S. K. Maiti, Europhys. Lett. \textbf{121}, 38004 (2018).

\bibitem{cite22} D. A. Lidar and J. H. Thywissen, J. Appl. Phys. \textbf{96},
754 (2004).

%\bibitem{cite12} S. K. Maiti, J. Appl. Phys. \textbf{117}, 024306 (2015).

\bibitem{ref16} S. Datta, Electronic Transport in Mesoscopic Systems (Cambridge University
Press, Cambridge, 1997).

\bibitem{ref17} S. Datta, Quantum Transport: Atom to Transistor (Cambridge University Press,
Cambridge, 2005).

\bibitem{cite12} S. K. Maiti, J. Appl. Phys. \textbf{117}, 024306 (2015).

\bibitem{pl22} F. H. L. Koppens, K. C. Nowack, and L. M. K. Vandersypen, Phys.
Rev. Lett. \textbf{100}, 236802 (2008).

\bibitem{pl23} Y. Tokura, W. G. van der Wiel, T. Obata, and S. Tarucha,
Phys. Rev. Lett. \textbf{96}, 047202 (2006).

\bibitem{pl24} K. C. Nowack, F. H. L. Koppens, Y. V. Nazarov, and L.
M. K. Vandersypen, Science \textbf{318}, 1430 (2007).

\bibitem{pl25} K.-M. C. Fu, S. M. Clark, C. Santori, C. R. Stanley, M. C.
Holland, and Y. Yamamoto, Nat. Phys. \textbf{4}, 780 (2008).

\bibitem{pl26} J. Berezovsky, M. H. Mikkelsen, N. G. Stoltz, L. A.
Coldren, and D. D. Awschalom, Science \textbf{320}, 349 (2008).

\bibitem{pl27} Y. Wu, E. D. Kim, X. Xu, J. Cheng, D. G. Steel, A. S.
Bracker, D. Gammon, S. E. Economou, and L. J. Sham, Phys. Rev. Lett. \textbf{99}, 097402 (2007).

\bibitem{pl28} D. Press, T. D. Ladd, B. Zhang, and Y. Yamamoto, Nature
\textbf{456}, 218 (2008).

\bibitem{ref18} C.-M. Ryu, S. Y. Cho, M. Shin, K. W. Park, S. Lee, and E.-H. Lee, Int.
J. Mod. Phys. B \textbf{10}, 701 (1996).

\bibitem{ref19} Y. Shi and H. Chen, Phys. Rev. B \textbf{60}, 10949 (1999).

\bibitem{ref20} Y.-J. Xiong and X.-T. Liang, Phys. Lett. A \textbf{330}, 307 (2004).

\bibitem{ref21} M. Patra and S. K. Maiti, Sci. Rep. \textbf{7}, 14313 (2017).

\end{thebibliography}

\end{document}

%\input{magneticComp}

\section{Discussion and Conclusions}



Our method based on stabilizing forward and backward pass, resulted in improved accuracy over the baseline and it was able to predict optimal dampening, sharpness and tail-fatness before training. 
Our findings are coherent with the line of research that has established that stabilizing gradients and representations at initialization results in better performance \cite{glorot2010understanding, orthogonal_initialization, he2015delving, roberts2022principles, defazio2022scaling, bengio1994learning, hochreiter1997long, hochreiter2001gradient, arjovsky2016unitary, pascanu2013difficulty}. Moreover it gives an initial reply to the question raised by
\cite{surrogate2019, zenke2021remarkable}, which asked  for a theoretical justification of initialization and SG choice for Spiking Neural Networks. With a similar intention, \cite{rossbroich2022fluctuation} proposed an approach that guarantees sparsity of activity at initialization to pick the weights distribution at initialization, resulting in improved accuracy. Our method differs from theirs in that it starts from a principle of stability to derive constraints, instead of a principle of sparsity. It differs also in that we use it to define the SG shape at initialization, not only the weights distribution, and we can show mathematically how weights initialization is intertwined to the SG shape choice. Our results suggest that a tedious hyper-parameter grid-search can be often avoided by making use of sound and established principles of learning.

One of the conditions was designed to hit the most sensitive part of an SG, its center, which resulted in a low sparsity requirement at initialization. This is very uncommon in the Neuromorphic literature, since sparsity brings large energy gains \cite{henderson2020towards,blouw2019benchmarking, 9395703,taulsnn, rossbroich2022fluctuation}.
However, the energy gains of SNNs also come from their binary activity. A matrix-vector multiplication, with a $\mathbb{R}^{m\times n}$ matrix, has an energy cost of $mnE_{MAC}$ for a real vector, and of $mn\rho E_{AC}$ for a binary vector, where $\rho$ is the Bernouilli probability of the binary vector, and in our case the neuron firing rate, and $E_{AC}, E_{MAC}$ are the energies of an accumulate and a multiply-accumulate operation \cite{yin2021accurate, hunger2005floating}. Since MAC are more costly than AC, 31 times on a $45$nm complementary metal–oxide–semiconductor \cite{yin2021accurate, horowitz20141}, we have energy savings with any $\rho$, e.g., when all neurons fire ($\rho=1$) and when they fire half of the time steps ($\rho=1/2$). This gain does not depend on the simulation speed, since it compares a spiking and an analogue computation, at the same computation speed.
Typically requiring more sparsity through a sparsity encouraging loss term, leads to a measurable decrease in performance \cite{zenke2021remarkable, rossbroich2022fluctuation}. However we observed that it is actually possible to achieve higher performance with higher sparsity, by starting with a strong firing rate at initialization, since their synergy acts as a regularization mechanism. This was possible also because the sparsity encouraging loss term was introduced gradually, and because its contribution was kept comparable to the task loss towards the end of training.

We observed that the more complex the task is and the more complex the network to train is, the more drastic is the difference in performance of different SG shapes. It is known that learning is possible with a wide variety of SG shapes \cite{zenke2021remarkable} and the community has not yet settled for one shape or one method to reliably choose which SG to use in each case \cite{surrogate2019}. We showed how to apply a well known stability principle to the forward and backward pass of the simplest Spiking Neural Network, the LIF, as a starting point, but we think that the principles of good Neuromorphic initialization can be further elaborated, in order to tackle more complex tasks and networks.


 
\appendix
%--------------------------------------------------

\begin{acknowledgments}
We thank Carl Sovinec and Eric Held for discussions involving the nonlinear
dynamics of this work, Jim Myra and Dan D'Ippolito for discussions pertaining
to the SOL treatment in the initial state, Stuart Hudson and Todd Evans for
discussions regarding magnetic field structure and the reviewers for detailed
comments on the manuscript. This material is based on work supported by the
U.S.  Department of Energy Office of Science and the SciDAC Center for Extended
MHD Modeling under contract numbers DE-FC02-06ER54875, DE-FC02-08ER54972
(Tech-X collaborators) and DE-FC02-04ER54698 (General Atomics Collaborators).
This research used resources of the Argonne Leadership Computing Facility,
which is a DOE Office of Science User Facility supported under contract
No.~DE-AC02-06CH11357,  and resources of the National Energy Research
Scientific Computing Center, a DOE Office of Science User Facility supported by
the Office of Science of the U.S.  Department of Energy under contract
No.~DE-AC02-05CH11231.
\end{acknowledgments}

\bibliographystyle{apsrev4-1}
\bibliography{Biblio}

\end{document}
