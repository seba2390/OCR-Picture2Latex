\section{Nonlinear evolution}
\label{sec:nonlinear}

\begin{figure}
  \includegraphics[width=8cm,trim={0.5cm 2.5cm 2cm 2cm},clip]{energy.png}
  \vspace{-6mm}
  \caption{[Color online]
  Simulated kinetic and magnetic energy decomposed by toroidal mode number ($n_\phi$).}
  \label{fig:energy}
\end{figure}

\begin{figure*}
  \centering
  \includegraphics[width=16cm]{pres3D_v2.png}
  \vspace{-4mm}
  \caption{[Color online]
  A slice of the 3D domain that shows the evolution of the pressure contours at
eight different times. During the initial stages (t=$30$ and $40\;\mu s$), eddies
of hot, high density plasma are ejected from the pedestal and are advected
poloidally in the counter-clockwise direction. Later in time these eddies are
sheared apart and the dynamics become a more turbulent-like state with smoke-like
off gassing which precesses poloidally counter-clockwise. }
  \label{fig:pres3D}
\end{figure*}

The nonlinear simulation uses a single-fluid, single-temperature (assuming fast
equilibration in the perturbations) MHD model with a temperature-dependent
resistivity profile where the Lundquist number, $S$, in the core is
$1.1\times10^6$. Here $S=\tau_R/\tau_A$, where $\tau_A$ is the Alfvén time
($\tau_A=R_o/v_A$), $v_A$ is the Alfvén velocity ($B/\sqrt{m_i n_i \mu_0}$),
$\tau_R$ is the resistive diffusion time ($\tau_R=R_o^2 \mu_0/\eta$),
$R_o=1.748 m$ is the radius of the magnetic axis, $\eta$ is the electrical
resistivity, $\mu_0$ is the permeability of free space, $m_i$ is the ion mass,
and $n_i$ is the ion density.  This choice of resistivity is enhanced by a
factor of 100 relative to the Spitzer value for computational practicality.
The model includes large parallel and small perpendicular diffusivities in 
the momentum and energy equations. The parallel-momentum-stress contribution is 
\begin{equation}
\mathbf{\Pi}_{\parallel i}=
  m_i n_i \nu_{\parallel i}
  \left(\hat{\mathbf{b}}
  \hat{\mathbf{b}}-\frac{1}{3}\mathbf{I}\right)\left(
  3\hat{\mathbf{b}}\cdot\nabla\mathbf{v}_{i}\cdot
  \hat{\mathbf{b}}-\nabla\cdot\mathbf{v}_{i}\right)\;,
\end{equation} where $\mathbf{v}_i$ is the ion velocity,
$\hat{\mathbf{b}}=\mathbf{B}/|B|$, $\mathbf{I}$ is the identity tensor, and
$\nu_{\parallel i}=10^5\;m^2/s$. The parallel-heat-flux contribution is
\begin{equation}
\mathbf{q}_{\parallel}=
  -n_i \chi_{\parallel}
  \hat{\mathbf{b}}\hat{\mathbf{b}}\cdot\nabla T\;,
\end{equation}
where $T$ is the temperature and $\chi_{\parallel}=10^8\;m^2/s$.
The small perpendicular diffusivites are modeled as isotropic particle,
momentum and thermal diffusivities with a magnitude of $1\;m^2/s$.

% Relative to our linear computations, our nonlinear simulations use the same
% model but diffusivity parameters are adjusted for computational practicality.
% The temperature-dependent resistivity is enhanced by a factor of $100$ relative
% to the Spitzer value such that the core Lunquist number becomes
% $S=1.1\times10^6$.  The perpendicular diffusivities in the nonlinear
% computational are set to $1\;m^2/s$ and the parallel viscous coefficient is
% $1\times10^5\;m^2/s$,

The 3D nonlinear simulation is performed with a $60\times128$ high-order
(biquartic) finite element mesh packed around the pedestal region to resolve
the poloidal plane and $24$ Fourier modes in the toroidal direction. The
simulation is initialized from a linear computation of modes with a restricted
toroidal mode number range ($n_\phi=1-8$). The mode energies at $t=0s$ are
small, the largest energy is contained within the $n_\phi=4$ mode which has a
spectral kinetic energy content of $4.2\times10^{-5}\;J$ and a spectral
magnetic energy content of $4.4\times10^{-6}\;J$.

The boundary conditions, on both the inner annulus and outer wall, are no-slip
for the velocity, Dirichlet for the density and temperature and a perfectly
conducting wall boundary condition for the magnetic field.  Linear computations
show that the mode growth rates are unaffected by presence of the inner
boundary, however there is an important effect in nonlinear computations.  The
Dirichlet condition on density and temperature provides an unrestricted
particle and energy source in the core to maintain the profiles at the inner
boundary in the presence of fluctuation-induced transport. With respect to the
outer boundary, a sheath boundary condition is not applied at the divertor and
consideration of an improved divertor boundary condition is a direction for future
research.

% The full (ExB toroidal and poloidal) rotation
% profiles based on Carbon impurity rotation measurements are included in the
% simulations, as experimental observations indicate that the QH-mode operational
% regime is dependent on the rotation profile. 

The energy evolution from a nonlinear NIMROD simulation, decomposed by toroidal
mode number, of DIII-D QH-mode shot 145098 at $4250\;ms$ with broadband MHD
activity is shown in Fig~\ref{fig:energy}.   The simulations are initially
dominated by a $n_\phi=5$ perturbation that saturates at around $30\;\mu s$.
After this time a saturated turbulent-like state develops and the $n_\phi=1$
and $2$ modes become dominant through an inverse cascade.  Each toroidal mode
in the range of $n_\phi=1-5$ is dominant at a different time and continued
interplay between modes is observed as the simulation progresses, particularly in 
the kinetic energy spectrum.
Figure~\ref{fig:pres3D} plots the lower half of a poloidal cut of the 3D
pressure contours at eight different time slices. The first time slice ($30\;\mu
s$) shows a coherent
structure associated with the dominant $n_\phi=5$ mode. By $40\;\mu s$, this
structure becomes sheared apart leading to a turbulent-like state at later
times.  Higher-time resolution plots show the perturbations are advected 
in the counter-clockwise direction consistent with the direction
of the ion poloidal flow inside the LCFS with a smoke-like off-gassing behavior.

