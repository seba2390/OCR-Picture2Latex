\section{Introduction}
\label{sec:introduction}

% \begin{itemize}
% \item
% Intro to QH-mode, broadband-MHD vs. EHO
% \item
% Intro to NIMROD, why NIMROD?
% \item
% Overview of paper:
% \begin{itemize}
%   \item
%   DIII-D to NIMROD: reconstruction considerations. Resolve for fields, add SOL currents, separation into SS and perturbed parts, include ion ExB and poloidal flows.
%   \item 
%   Linear analysis (ALCF?) 
%   \item
%   Nonlinear evolution - energy plots, 3D pressure plots
%   \item
%   Transport analysis - homoclinic tangle, n=0 line-out plots (density, temp, tor/pol flow, current) 
%   \item
%   discussion (saturation mechanism and future work - further comparisons; turbulent-like calculations) and conclusions
% \end{itemize}
% \end{itemize}

It is desirable to have an ITER \cite{ITER} H-mode regime that is quiescent to
edge-localized modes (ELMs) \cite{connor98,leonard06}. ELMs deposit large,
localized and impulsive heat loads that can damage the divertor. A quiescent
regime with edge harmonic oscillations (EHO) or broadband MHD activity is
observed in some DIII-D
\cite{burrell01,burrell05,burrell09,garofalo11,burrell12,burrell13,solomon14,garofalo15},
JT-60U \cite{Sakamoto04,oyama05}, JET \cite{solano10} and ASDEX-U \cite{suttrop05}, discharge
scenarios. These ELM-free discharges have the pedestal-plasma confinement
necessary for burning-plasma operation in ITER\cite{garofalo15}. The mode
activity associated with the EHO or broadband MHD on DIII-D is characterized by
small toroidal-mode numbers ($n_\phi\simeq1-5$) and is thus suitable for
simulation with global MHD codes.  Measurements from beam-emission
spectroscopy, electron-cyclotron emission, and magnetic probe diagnostics show
highly coherent density, temperature and magnetic oscillations associated with
EHO.  The particle and impurity \cite{grierson15} transport is enhanced during
QH-mode, leading to essentially steady-state profiles in the pedestal region.

Relative to QH-mode operation with EHO, operation with broadband MHD tends to
occur at higher densities and lower rotation and thus may be more relevant to
potential ITER discharge scenarios.  While there are computational
investigations of the discharges with EHO \cite{liu15,battaglia14}, there is
less computational analysis of discharges with broadband MHD. In this paper, we
investigate the broadband-MHD state with nonlinear NIMROD
\cite{Sovinec04,Sovinec10} simulations initialized from a reconstruction of a
DIII-D QH-mode discharge with broadband MHD. These simulations include the
reconstructed flow and saturate into a turbulent-like state.

This paper is organized as follows.  As an initial-value computation, our
simulations require accurate and smooth initial conditions to avoid 
spurious instabilities. 
Section \ref{sec:reconstruct} describes how the reconstructed
fields are imported into the NIMROD spatial discretization. One of the novel
methods in our approach to this edge modeling is to extrapolate the pressure
and density profiles through the scrape-off layer (SOL) to maintain first-order
continuity and thus a continuous current profile when solving the
Grad-Shafranov equation.  The section concludes by discussing our assumptions
where the reconstructed fields are in steady state, and how our modeling
simulates the dynamics of perturbations about this steady state. In
Sec.~\ref{sec:nonlinear}, the model equations and the dynamics of our nonlinear
MHD simulations that saturate into a turbulent-like state are considered. The
magnetic stochasticity and transport induced from these perturbations are
analyzed in Sec.~\ref{sec:transport}.  Finally, we conclude with a discussion
of the implications and limitations of our present modeling and mention future
directions for this work.
 
% High quality equilibria are essential for extended-MHD modeling with
% initial-value codes such as NIMROD \cite{Sovinec04,Sovinec10}. Typically the
% spatial resolution requirements for extended-MHD modeling, which must resolve
% singular-layer physics and highly anisotropic diffusion, are more stringent
% than the resolution of equilibrium reconstructions from experimental
% discharges. To circumvent mapping errors, we re-solve the Grad-Shafranov
% equation with open-flux regions using the NIMEQ solver \cite{Howell14} to
% generate a new equilibrium while using the mapped results for both an initial
% guess and to specify the boundary conditions. Additionally, reconstructions
% commonly assume that the region outside the last closed flux surface (LCFS) is
% current free. For discharges with large edge current, as is found during
% QH-mode, this can lead to a large discontinuity in the current density at the
% LCFS that is problematic for MHD modeling. During our re-solve of the
% Grad-Shafranov equation, we relax the current-free assumption outside the LCFS
% and include temperature and density profiles with non-zero gradients which
% generate associated small currents in the scrape-off layer that cause the
% overall current profile to be continuous. The new solution is an equilibrium
% that closely resembles the original reconstruction with the exception of the
% open-flux currents. This regenerated equilibrium is consistent with the
% profiles that are measured by the high quality diagnostics on DIII-D.


