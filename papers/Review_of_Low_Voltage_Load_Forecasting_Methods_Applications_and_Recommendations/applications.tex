\section{Low-Voltage Load Forecasting Applications}
\label{sec:LVLF-applications}


Forecasts are often used within specific applications, which require estimates for planning,  operation and trading. This section overviews some of the common applications encountered as part of the review. 

\subsection{Network Design and Planning}

Forecasts are often used in grid design optimisation. Dupka et al.~\cite{Dupka2011FSP} use a Gaussian distribution forecast to minimise the costs of locating and sizing capacitors on the distribution system. Ravadanegh et al.~\cite{ravadanegh2013hao} use medium and long term load forecasts to locate the optimal site and size of a distribution substation. The authors in \cite{kavousifard2014mop} use a Gaussian model to forecast the load uncertainty for distribution feeders with wind turbines to optimise the selection of the topology and position of sectionalising switches. Kavousi-Fard and Niknam \cite{Kavousi2014msd} use the forecast of active and reactive loads to improve the reliability of the distribution network by focusing on the reconfiguration of a distribution feeder. Ahmadigorji et al. \cite{ahmadigorji2009odp} study the optimisation of the location points of portable distribution generation based on a cost/worth analysis. Load forecast uncertainty is incorporated by assuming Gaussian distribution for mean and predicted annual peak load.

The work of \cite{Lee2020ncr} aims to reduce the neutral current via phase arrangement. An LSTM model was adopted for monthly load forecasting, and phase arrangement optimisation was performed using particle swarm optimisation.

\subsection{Network Operations and Control}

\subsubsection{Control and Management}

One of the most common applications of demand forecasts is in grid management, using a forecast to help control a storage device \cite{sossan2016atd} and optimal operational planning \cite{Lopez2019psl}. Several papers focus on PV-battery systems to help micro-grids with high penetrations of distributed energy resources \cite{mohamed2017hcd}, \cite{khan2020tee}, \cite{Zafar2018mmp}. One way to minimize the grid disturbances from high PV penetration and avoid high injection peaks is to introduce a feed-in limit, which basically caps the maximum power injection into the grid and encourages PV owners to increase their self-consumption. The feed-in limit, however, can translate into curtailment losses. To deal with this problem, \cite{Riesen2017car} present a control algorithm that aims to minimize curtailment loss and maximise self-consumption, using a linear optimization scheme that depends on the forecast data for the next 48 hour of PV production. They show that in the presence of feed-in limits, adding storage can considerably reduce curtailment losses. Litjens et al.~\cite{Litjens2018aof} also use predictive control to reduce losses due to feed-in limits . Other applications found in the literature for PV-storage systems include, aims to decrease utility bills by maximising self-consumption \cite{Johnson2018ops}, to increase PV hosting capacity \cite{hashemi2018eco}, to minimise cost in economic dispatch~\cite{bersani2006mol}, and to minimise both energy import from the local grid and energy export \cite{stephen2020ngr}.  

Other storage scenarios include trying to reduce peak load \cite{kodaira2020oes,rowe2014apr, nikolovski2018abp,bao2012bes} and controlling of electric vehicles \cite{anastasiadis2017evc}, using the vehicles as dispatchable storage units. Forecasts are used in \cite{Yunusov2018sss} for smart storage scheduling and peak reduction on LV feeders. Bennett et al. \cite{bennett2015doa} propose a scheduling system for battery energy storage (BES) with a focus on peak shaving and valley filling using 10-min data from 128 residential households. Their scheduling system comprises of generating next-day load forecasts, deriving a charge and discharge schedule based on the load forecasts, and using an online controller for making scheduler adjustments. Multiple criteria can be pursued at the same time, for example Dongol et al. \cite{Dongol2018amp} focus on peak shaving, demand smoothing and maximizing the battery utilization using model predictive control. 

As well as focusing on demand and generation, a few storage applications also consider voltage control applications. The high penetration of PV systems connected to the grid, means distribution voltage profiles are now more likely to exceed the voltage limit, potentially resulting in transformers overloading during peak production. The issue of voltage limit exceedance is thus of additional concern given the ongoing growth in solar PV systems. There has, thus, been an increased interest in problems relating to PV regulation. In this regard, \cite{Ghosh2017dvr} propose a voltage regulation technique, based on PV generation forecasts. They utilize very short-term (15 sec) PV power forecasts, using a hybrid modelling scheme based on Kalman filter theory.
Zufferey et al.~\cite{zufferey2020psf} use probabilistic short-term forecasts for constrained optimal power-flow to optimize voltage control. Hu et al.~\cite{hu2003vvc} use forecasts for Volt/Var control in distributed systems, and Wang et al.~\cite{wang2013acv} use load forecasts as inputs for a Conservation Voltage Reduction to reduce peak demand and keep voltages within regulatory standards. The authors in \cite{Kim2013ccd} propose a control method using a dynamic programming algorithm, in which the distributed generator participates in steady-stage voltage control along with switching devices. Tap changes are the traditional method for controlling the voltage, the authors of \cite{Agalgaonkar2014dvc} utilize load and irradiance forecasts to propose an optimal power coordination strategy aimed at reducing the number of tap operations, thereby minimizing the likelihood of exceeding the control limit (runaway condition) and potentially increasing the life of the tap control mechanism.

Online power management for micro-grids is presented in \cite{Mohan2016snf}. They investigate the impact of uncertainty in nodal power injections on micro-grid cost and power flow variables, using a residential feeder as a test system.

\subsubsection{Anomaly Detection}

Several papers consider how forecasts can help to identify anomalies, with the main applications being theft detection and to reduce malicious attacks. The authors in \cite{Fenza2019dma} deal with the crucial issue of theft detection in the smart grid while accommodating the concept of drift (such as a change in family size, second household etc.). The relevance of this application can be gauged from a study by the Northeast Group, LLC, which involved 125 countries, estimating that utility companies lose around USD 96 billion per annum due to nontechnical losses including fraud, theft etc. The anomaly detection strategy of \cite{Fenza2019dma} comprises of 3 steps: (1) clustering load profiles using k-means, (2) applying an LSTM model on the curves of cluster centroids and forecasting individual consumption, and (3) identifying anomalies at any given instant based on forecast errors from the previous week. Forecast errors were quantified using the RMSE, while the anomaly detection accuracy was assessed using precision and recall. Li et al. \cite{Li2019ans} develop a theft detection system for an IoT-based smart home. Their three-step methodology is based on: (1) forecasting power consumption using multiple machine learning models (MLP, RNN, LSTM, and GRU), (2) using a simple moving average for identifying the anomaly, and (3) making a final decision if the theft has occurred. Model validation was based on simulations of theft scenarios (randomly ``stealing'' energy from different time periods).

Fadlullah et al. \cite{Fadlullah2011aew} propose a probabilistic modelling methodology based on Gaussian process regression to identify malicious attack events in a smart grid, the validation of which is based on simulations. The authors state that their approach could also be used for anomaly detection, such as voltage surges and fluctuations.

In \cite{Komatsu2020pda}, forecasts are used to develop an early warning system for peak electricity consumption demand, whereby the actual weather data was linearly interpolated to match the sampling rate of the load data. However, it was not justified why linear interpolation was a suitable strategy to employ. 

The authors in \cite{mota2007lbp} use a rule-based approach and fuzzy logic concepts to predict the load behaviour after blackouts of a substation in Andorinha/Campinas/Brazil for two different blackout conditions.

\subsubsection{Flexibility Applications}

%Forecasts were also shown to be useful for flexibility applications other than storage control. 
Forecasts are useful for helping understand the effect of demand-side response by predicting the unmodified demand to compare to the actual effects after an intervention is performed as in \cite{larsen2017dre}. Similarly, \cite{priolkar2020aoc} focuses on estimating the baseline load of an LV Substation feeder in Goa (India) for implementing demand response strategies.  In contrast, Garulli et al.~\cite{garulli2015mat} forecast the actual demand-side response using the active demand (the requested change in demand).  He and Petit \cite{he2019drs} propose a scheme for demand response scheduling in a grid with high penetrations of distributed generation. 

Given a load forecast at the substation level, the approach by Ponocko et al.~\cite{ponocko2018fdf} provides a method for decomposing the forecast into the controllable and the uncontrollable components by using an ANN for disaggregation. They investigate the required percentage of users that need a smart meter, finding that a coverage of 5\% is enough to forecast the composition at the substation level with sufficient accuracy.

Pinto et al.~\cite{pinto2017mpf} use conditional kernel density estimation to generate load forecasts which feed into an optimisation that provides feasible flexibility operating trajectories that determine the storage requirements, flexible appliances or consumer preferences. 



\subsection{Trading}

Peer-to-peer trading and the use of locally generated energy is expected to play a big role in future. The following papers use load forecasting for trading purposes.

Optimal solutions for multistage feeder routing problem using future loads and market prices are presented in \cite{Taghizadegan2019asc}. In  \cite{he2019mop},a real-time pricing strategy is developed. Energy trading algorithms for LV connected microgrids are discussed in \cite{feng2019hae} and a LASSO based model for household forecasts is used in \cite{kostmann2019fib} to feed blockchain designed local energy markets, which consider an auction process to match supply and demand. 


\subsection{Simulating Inputs, Missing Data, Privacy protection}
\label{otherapps}
Another major application for forecasts relevant to planning, operations, and trading is to generate inputs to provide other analysis or impute unknown demand. Often forecasts are used to estimate measurements for inclusion in power flow estimation. In \cite{Korres2011SEI} an average-based forecast with Gaussian errors is used to create pseudo-observations for unmeasured loads for state estimation. Zhao et al.~\cite{zhao2020rmv} use SVR with Gaussian radial basis functions to forecast load and improve the quality of historic data for estimation of distribution system states using multi-source data, i.e., historical, online, smart meters and from Supervisory Control and Data Acquisition (SCADA) systems, as features. In \cite{Bracale2013ABB} a Bayesian forecast method is used to calculate probabilistic future steady-state analysis for a smart grid, and in \cite{Chessmore2008VPE} an ANN based model is used to estimate the voltage profile via a power flow program. Finally, \cite{hermanns2020eod} use a simple average load forecast of the past weeks with a correction of the forecast values as new load values become available. The forecast is used for a grid state forecast software tool.

Forecasts are also used to fill in or estimate missing data, as in Borges et al.~\cite{borges2020etm} who use short-term forecasting models with adjusted features (using future values) to impute missing data for primary substations. They find that, while it differs for different data sets, generally the model with access to historical data, meteorological data and also data from other neighboring substations can improve over using only subsets of these features. Zhou et al.~\cite{zhou2020blb} use LSTM for load forecasts in their approach to providing harmonic state estimation using regression analysis for power flow calculations and sparse Bayesian learning. Finally, in \cite{wang2020rnl}, the authors consider decomposing the demand of LV substations into traditional load, flexible load and distributed generation components. An exponential smoothing load forecast is used as an estimate for the load component of the decomposition. 

Huyghues-Beaufond et al. investigate the effects of data cleansing on the forecast accuracy of LV/MV feeders of UK Power Networks \cite{huyghues-beaufond2020raa}. Their methodology consists of the following steps. First, outliers are detected using an automatic procedure which combines the Tukey labelling rule \cite{Tukey} and the binary segmentation algorithm. Next, various approaches for missing value imputation are investigated, including unconditional mean, hot-deck via \textit{k}-nearest neighbour and Kalman smoothing. Feed-forward deep neural networks for day-ahead forecast at hourly resolution are developed to assess the performance of the cleansing method. 
The proposed data cleansing framework efficiently removes outliers from the data, and the accuracy of forecasts is improved. It is found that hot deck (\textit{k}-NN) imputation performs best in balancing the bias-variance trade-off for short-term forecasting.

Chen et al. \cite{chen2014ita}  use Chebyshev’s inequality to identify inaccurate observations and use feature curves to restore these data points. It is shown that correction of data improves the overall forecast accuracy. 

Privacy protection is important for different applications, and in \cite{boustani2017sgp} the methods are developed to protect household privacy whilst preserving load profile correlations between forecasts and actuals. 
Similarly, in  \cite{hou2020anp} privacy-preservation via a model randomization scheme 
consisting of a forecasting phase and a reporting phase is presented. The approach is currently limited to multivariate polynomials. However, many different statistical and machine learning methods such as ARIMA, MLR, RBF and ANN can be represented using modifications.
Finally,  a security and privacy preserving scheme by limiting the connectivity of home area networks with the electric grid is presented in \cite{abdallah2017lsa}. 


\subsection{Summary}
\label{sec:summary}

Although forecasts are essential for the presented applications, with the exception of the storage control papers there is often very little focus on the methods or the forecast errors within the reviewed papers. In many cases, the forecast models used are not presented, and in others, it is not clear if a forecast has been generated, or simply the actual future values are used as proxies for real forecasts. When actual forecasts are developed for the applications, most use naive methods such as basic averages, or similar day methods. When forecast errors are used, they usually rely on basic Gaussian assumptions.

Given the volatility of distribution level demand, it is surprising that relatively few of the application papers utilise probabilistic forecasts, despite some cases such as \cite{nikolovski2018abp} (using an adaptive neuro-fuzzy inference system), \cite{zufferey2020psf} (probabilistic kNN), and \cite{pinto2017mpf} (conditional KDE). Lilla et al. \cite{Lilla2020dsl} consider day-ahead scheduling of a battery energy storage system with PV and, as stated by the authors, the uncertainty in forecasts was ignored in the analyses, and the prices of exchanges with the utility grid are assumed to be predefined.

An important gap apparent from this review is that there are very few examples of the impact and role the forecast's accuracy has on the outputs of the application. Most papers do not include a benchmark or different models with which to compare how the accuracy of the model affects the performance of the application. Without these investigations and comparisons, it is difficult to assess the value or importance of the forecast's quality. A few examples, including \cite{stephen2020ngr}, \cite{alasali2020acs} have shown that improved forecast accuracy can improve the performance of the application. Further \cite{kodaira2020oes} shows that using prediction intervals improves peak load reduction compared to some basic point forecast methods. However, there are no detailed studies of performance changes of probabilistic versus point forecasts. Understanding whether probabilistic forecasts offer significant improvements over point forecasts can help make decisions of whether probabilistic methods are worth the increased computational costs.
