\section{Introduction}
%%%
Increased monitoring and communications are opening up opportunities for smart energy networks. The transition to a more localised and distributed energy system helps to support the increased connection of low carbon technologies and provides an environment for new products and services such as peer-to-peer electricity markets, heat-as-a-service, smart storage, and increased renewable generation utilisation. While the uptake of low-carbon technologies (LCTs) is growing and spreading across the globe, predicting their growth in a specific location is challenging. Furthermore, smart technologies to manage LCT's impact on the grid is still in their relative infancy. Low voltage (LV) network modelling will improve network planning and facilitate better management of potential LCT connection hotspots. It could also allow local authorities to monitor their progress toward a more sustainable future. Our ability to facilitate and optimise these opportunities would necessitate access to accurate forecasts of the demand at the low voltage level, given that future estimates of load helps anticipate the core features of LV network models. 

Unfortunately, there is a dearth of literature on forecasting at the low voltage level. Smart meter roll-outs worldwide have increased the investigation into household-level demand \cite{Wang2018ros}, however, very few papers focus on the secondary or even primary substation level of the distribution network. In a handful of existing studies, forecasts on low voltage networks are typically based on aggregations of multiple smart meters. While this is encouraging, it has been shown that this approach does not provide a perfect representation of LV load as it overlooks many important features and nuances of a real LV network \cite{Haben2019stl}.

Although literature and methods are abundant for forecasting at the higher voltage and system levels \cite{Tao2020efa}, due to the increased volatility at the low voltage level, other challenges, not present at the system level, emerge.
Given these reasons, we need more advanced methods to accommodate a more complex range of patterns in energy time series at the LV level, whereby the underlying uncertainty is also communicated to the end-user in the form of probabilistic forecasts. There is a requirement for further research in this area that spans more advanced and complex methodologies. 

This paper serves to review the current research of load forecasting at the low voltage level, to identify the gaps and opportunities, highlight the challenges, and provide recommendations and best practices.

\begin{figure*}
	\includegraphics[,width=\textwidth]{figures/lvgrid.pdf}\par 
	\caption{Overview of typical high, medium and low voltage grid layout. This shaded area shows the scope of this review. }
	\label{fig:lvgrids}
\end{figure*}

Low voltage can be an ambiguous term, and hence before proceeding, it is worth defining the scope of this review. Low voltage is a relative term and is defined differently in different countries. Figure~\ref{fig:lvgrids} gives a simplified overview of the layout of electrical grids, as they are typical, for instance, in the UK and Europe. The voltages are stepped down via several substations from the very high transmission level voltages of, for example, 220kV or 300kV in Germany and up to 400kV in the UK, down to the 400V or 230V at the end customer level. The high voltage level of distribution grids has typically voltages of, for instance, 110kV (Germany) or 132 kV (UK), followed by a medium voltage level that ranges from 1 kV up to 35 kV. Given the large number of distribution systems, the layout at this medium voltage level varies across regions, even within countries. This level incorporates the 11kV and 33kV levels common in the UK, typical medium voltages levels of 10-20 kV in Germany, and the corresponding parts in US distribution systems rated at, for instance, 34.5kV, 13.2kV, and 4.16kV. 

For the scope of this review, we also include the lowest level of the medium voltage range, typically at 11kV in the UK and 10kV in Europe, down to the end-customer level. This ensures that the review remains focused on the challenging area of the "last mile'' of the distribution network with its numerous applications, heterogeneous end-customers, and relatively high volatility. The scope further retains forecasting within local area grids and microgrids that are operated at low voltages, as well as the special case of the household level.\footnote{In some cases it is not obvious whether a paper is LV by our definition and in this case, the paper is still reviewed and included as it is still at distribution and not transmission level.}

\subsection{Motivation}
\label{sec_motivation_relatedrevs}

Until recently, the focus of most research on short term load forecasting has been on the system level with the aim of balancing supply and demand \cite{Tao2020efa}. As discussed in \cite{diamantoulakis2015bda}, this was based on large aggregated load data where the individual variations have been averaged out. With more distributed generation, the focus is moving towards more consumer-centric models. The challenges of the future low carbon network are likely to be increasingly concentrated at the low voltage level and decentralisation in general.  

Moving forward, this will mean more locally-focused analytics, i.e. the so-called \textit{smart grid applications}, where capabilities such as smart storage control, demand-side response and peer-to-peer markets are expected to play a major role. A smart grid essentially uses advanced metering with two-way
communications to monitor, detect and respond to energy usage. Forecasting will be an essential component of smart grids since they will allow networks to anticipate, and hence prepare for significant changes in demand. Hence, although many of the applications defined in Section~\ref{sec:LVLF-applications} consider components of a smart grid, the smart grid is not synonymous with the low voltage network, the focus of this review.

There are a growing number of papers on load forecasting \cite{Tao2016pel, Tao2020efa}, however, contributions are mostly restricted to either very large aggregations, typically system-level loads, or at the individual household level, with a particular focus on smart meter data. The LV distribution level which lies in the middle of the network hierarchy has unfortunately received very little attention, hence this review. 

\begin{figure*}
	\begin{multicols}{2}
		\includegraphics[clip, trim=0.5cm 9cm 0.5cm 9.25cm, width=0.48\textwidth]{figures/SumRESProfilesNew2.pdf}\par 
		\includegraphics[clip, trim=0.5cm 9cm 0.5cm 9.25cm, width=0.48\textwidth]{figures/PowerLawNewReview2.pdf}\par 
	\end{multicols}
	\caption{LV level forecasts present unique challenges. On the left are examples of a week's worth of demand from aggregations of 500 households (plot a) down to a single household (plot f). On the right is a illustration of the power law relationship of relative error as a function of feeder size.}
	\label{lvplots}
\end{figure*}

The larger focus of the literature on household-level forecasting, compared to the load at the more general LV level, can mainly be attributed to an increasing amount of smart meter data. LV network feeders consist on average of about 50 households \cite{Haben2019stl} and hence are typically less volatile than smart meter data. This is illustrated in the left hand plot in Fig.~\ref{lvplots}. This shows different aggregations of households from the Irish smart meter trial~\cite{Commission2012csm} from $500$ households (labelled (a) at the top) down to a single household at the bottom (labelled (f)). Beyond 100 households the profiles are relatively smooth but below this the data is increasingly irregular and volatile with varying degrees of spikiness.   

The challenges associated with modelling load at lower levels of aggregation are further illustrated in the right hand side plot in Fig. \ref{lvplots}. On the y-axis is the relative error (e.g. normalised root mean square error (RMSE) or mean absolute percentage error (MAPE)) and on the x-axis is the size of the LV feeder (e.g. its average daily demand, peak demand, number of consumers etc.). This illustrates a common feature of LV demand forecasts (alternatively forecasts of aggregates of household demand); there is often a power law relationship between the size of the feeder and the relative error \cite{mirowski2014dfi}. This means it becomes exponentially more difficult to accurately forecast smaller feeders (in terms of average demand or number of customers connected). The paper by Wang et al. \cite{Wang2018ros} is also one of the few reviews which mention LV level forecast, and also highlights the volatile nature of the associated demand behaviour. 

As shown in Haben et al. \cite{Haben2014ane}, traditional pointwise error measures such as RMSE and MAPE may not be appropriate (or informative) to describe the accuracy of forecasts of smart meter demand (i.e. individual households) due to the so-called \textit{double-penalty effect}. It is likely that this effect also holds for demand on small feeders or aggregations of small numbers of residential smart meters. However, this has not been investigated within the literature.  

In the emerging LV demand forecast literature, lack of real data means that either only a few substations are considered or the LV substations are artificially created from aggregations of smart meter data. Studying only a few substations limits the conclusions from any analysis. As figure \ref{lvplots} illustrates, LV networks consist of a wide variety of behaviours with the number of consumers connected being one of the largest indicators of demand accuracy. Without a large enough sample very few general conclusions can be established. In addition, LV networks are not simply the aggregation of individual households but consists of many different components, including street lights, cameras, and other street furniture. These connections may also be reconfigured over time (see for instance ~\cite{mirowski2014dfi}). Further, as shown in \cite{Haben2019stl} knowledge of the types of households is vital, for example, households with overnight storage heaters can produce dramatically different behaviours. 

To the authors' best knowledge, the paper by Haben et al. \cite{Haben2019stl} is the only one which considers forecasts of a relatively large number ($100$) of real feeders. This highlighted previously unknown results, such as the effect of a high proportion of overnight storage heaters and commercial customers, and the lack of influence of temperature on the forecast accuracy. It is vital that these results are replicated and further studies are developed to better understand the limitations and features of LV level forecasts.  

In short, LV level demand has unique features compared to medium (MV) and high voltage (HV) level demand:
\begin{itemize}
	\item Increased volatility due to lower aggregation of demand.
	\item Increased variety of demands with different feeders made up of different numbers and types of consumers.
	\item Less well understood explanatory variables
	\item An increased range and variety of applications and requirements for forecasts at the LV
	level.
\end{itemize}

As will be demonstrated in this review, these features will drive major differences in the techniques and methods which are applied to forecasting LV demand compared to what has traditionally been developed for HV or system level demand forecasting. 


\subsection{Related Reviews}
\label{sec_related_rev}

Before proceeding with the core topics of this paper, we summarise the main recent reviews in the area of forecasting, smart meter forecasting and smart meter analytics. This will serve the purpose of 1) providing a high-level overview of forecasting from the system level to household level, 2) highlighting the need for this review and 3) surveying peer-reviewed methodologies for conducting a viable review, which we will emulate to provide consistency.   

Hong and Fan \cite{Tao2016pel} provide a tutorial review of probabilistic load forecasting. They give an outline of other reviews in the area, the main methodologies applied, applications, evaluation methods as well as future problems. In this list they include electric vehicles, wind and solar generation, and demand response, all topics very much within the remit of LV level.

A recent paper by Hong et al. \cite{Tao2020efa} focused on a review of smart meter data. They looked at a range of forecasting topics that are becoming more prominent (and will also feature in this review) including forecast combination (Section \ref{hybridSection}), hierarchical forecasting, and probabilistic forecasting (Section \ref{section_prob}). Further issues such as open data, the role of forecasting competitions, and publishing issues are also discussed. Wang et al. \cite{Wang2018ros} also perform a review of smart meter data analytics and highlights several open smart meter data sets. One of the aims of this review is to also highlight and identify many open data sets that researchers may use. To further support researchers, we are also publishing a list of relevant datasets with links to major papers, see Table \ref{tab:datasets}. We hope this review article, with the list of key papers and datasets, would provide a good starting point to anyone embarking on research in this important and evolving field of modelling LV load.

As with most reviews in other areas, both \cite{Tao2020efa} and \cite{Wang2018ros} use a Scopus search to identify the number of published papers and major journals that publish forecasting and smart meter research. 

A review on analysis of residential electricity consumption and applications of smart meter data is given in ~\cite{Yildiz2017rai}. This is a review/survey on analysis and applications of smart meter data, but lists some major forecast methods, common inputs to the forecasts, and gives an overview of the traditional and new error measures being applied. It contains also household level applications such as home energy management systems, anomaly detection, customer feedback and health care for the vulnerable. Again this review will consider all of these topics but within the wider LV context. 

As in other fields, deep learning approaches are getting more attention from researchers lately. An unpublished review of deep learning approaches can be found in~\cite{gasparin2019deep}. It is not limited to the LV-level but they explicitly compare deep learning approaches applied to household data.  In contrast, Yin et al.~\cite{yin2020aso} give a survey of the quite limited scope of deep learning approaches in the distribution network, presenting some examples of applications in load and renewable energy sources (RES) forecasting as well as fault detection.  

The above reviews do not investigate the low voltage distribution networks but instead consider smart meters \cite{Tao2020efa}, \cite{Yildiz2017rai}, or general forecasting for the higher voltage, system-wide or national level \cite{Tao2016pel}. As discussed in the previous Section \ref{sec_motivation_relatedrevs}, LV networks encompass a much wider range of problems and applications than associated with the above related reviews. LV network demand is much more volatile than higher voltage level demand and is extremely diverse. This is because they often serve different numbers and types of consumers, mixing residential, and small commercial consumers. As demonstrated in this review, LV demand forecast requirements can be very different to those used in more general load forecasting, requiring very different inputs, different methods and in some cases, very different error metrics. 

For smart meter forecasting, the challenges are very similar to LV forecasting. They both are typically very volatile and therefore may require similar techniques such as probabilistic forecasts to estimate their associated uncertainty properly. However, there are some key differences. Firstly, LV network demand is not simply the aggregation of individual consumers demand (e.g. from smart meters) \cite{Haben2019stl}, and the presence of street furniture and the diversity of sizes and types of LV networks gives them unique features (such as the power law in Figure \ref{lvplots}) which are not components of individual smart meter data. Secondly, the LV networks produce a whole range of applications that are not applicable at the end customer level. As will be explored in Section \ref{sec:LVLF-applications} this includes network control, microgrid energy trading, and flexibility applications. 

Given these specific requirements, this review focuses on the relatively underexplored area of low voltage level forecasts, their associated applications, the most significant openly available datasets, and challenges and recommendations going forward. It should be noted that although this paper is not focused on smart meter forecasting, smart meters have been included in this review as they illustrate some of the same challenges with LV level forecasting, in particular the increased volatility and diversity of time series.


\subsection{Literature Selection Methodology}
\label{sec:lit_selection_methodology}

As with the other relevant reviews summarised in Section \ref{sec_related_rev}, the search for relevant papers was conducted using the Scopus\footnote{\url{https://www.scopus.com/}} abstract and citation database that provided a user-friendly interface for refining and investigating our queries. The search query was applied to the article titles, abstracts and keywords of articles, and consisted of the following terms:
%
\newline
\texttt{
	(substation OR feeder OR "low voltage" OR "smart meter") AND (load OR electricity OR consumption) AND (forecast*)
},
%
\newline
where text in quotation marks indicates exact match and text followed by asterisk indicates words starting with this sub-string.
The search consists of the main keywords representing the level (LV, substation, etc.) and type (electricity, etc.) of forecasting. 

The final search before starting the reviewing process was conducted in August 2020 and it resulted in 1487 manuscripts. A breakdown of the number of papers (before filtering) is shown on a left panel in Figure \ref{fig:final_set_top_journals}, where we observe a proliferation of papers and high-interest post 2000. This is consistent with energy forecasting in general \cite{Tao2020efa} and smart meter analytics \cite{Wang2018ros}.
\begin{figure*}
	\begin{multicols}{2}
		\includegraphics[width=0.45\textwidth]{figures/original-search-count-per-year.pdf}\par 
		\includegraphics[clip, trim=0cm 0cm 0cm -2.5cm,width=0.45\textwidth]{figures/final-non-conference-top-journals.pdf}\par 
	\end{multicols}
	\caption{Publication count per year using the search terms as given in the text is shown on the left. On the right are top journals from the final set of articles. Open-access journals are depicted with a light grey colour}
	\label{fig:final_set_top_journals}
\end{figure*}

To manage the sheer number of papers, we proceeded with the following steps. We first removed any papers prior to the year 2000, which reduced the current paper count to 1362. Since we want to focus on peer-reviewed material and journals, we then filtered according to conference papers (conference paper and conference review) and others (article, article in press, books, book chapters, data paper, paper review). There are 807 conference and 555 non-conference items. For the non-conference we kept all 2020 and 2019 papers but only those older papers with five or more citations, as a proxy to impactful methods, which reduced the total to 423 non-conference papers (including 155 papers from 2019 and 2020). For the conference papers we were slightly stricter and kept only those with more than 20 citations while retaining all of those from 2020 resulting in 69 conference papers. In total, this resulted in 492 articles. 

A high level investigation of these papers identified several papers which were not about low voltage level load forecasting, could not be accessed because they were behind paywalls and could not be accessed by any others sources (including contacting the authors), or were not available in English. This provided a final list of 221 papers which were read and reviewed by the authors. It was found when reading some of these papers in detail that forecasts were not a topic of the manuscripts (typically consisting only to a possible application or only discussed conceptually). Hence the final number of reviewed papers is slightly smaller than 221. In addition, a few papers that are not connected to the keywords have been included in this review, because they tell the wider narrative such as the more general short term forecasting reviews discussed above (as well as a few methods' and dataset references). 

The most frequently occurring journals from the final list of reviewed papers are shown in Figure \ref{fig:final_set_top_journals}, where we find only two open-access journals: \textit{IEEE Access} and \textit{Energies}. \textit{IEEE Transactions on Smart Grid} and \textit{Power Systems}, respectively, are the most popular journals in the field, followed by \textit{Energies}.


\subsection{Structure of this Review}
% how this paper is structured
This review emulates several features from the previous reviews discussed in Section \ref{sec_related_rev} including investigating the methodologies, common explanatory variables, special topics like forecast combination and hierarchical forecasting but will be unique in several areas. Firstly, the applications are extended beyond the smart meter or system-level forecasting reviews, and discuss real-life planning, operations and control of LV networks. Since a major intention of this review is to encourage further research in LV level forecasting, this paper illustrates the opportunities and challenges and will establish an open community-driven list of known LV level open data sets to encourage further research and development. 

Figure~\ref{fig:flowchart} provides an overview of the structure of the paper. 
Section \ref{sec_forecasting_method_main} provides a comprehensive overview of the main methods and techniques in LV level load forecasting, including point forecasts (both statistical and machine learning methods), probabilistic forecasts, combination approaches, as well as more esoteric methods. Section \ref{sec_trends_special_topics} presents some of the emerging trends and special topics. Section \ref{secdatasets} focuses on the LV datasets which are utilised in the reviewed papers, summarising some of the most commonly available datasets and their features. It links to the open data list. Section \ref{sec:LVLF-applications} focuses on the LV level applications which utilise forecasting. Finally, this review will conclude with a discussion in Section \ref{sec_discussion} identifying some of the major challenges as well as sharing some of the authors own views and recommendations for future research. Note, that while the Methods section is quite comprehensive, by structuring the work by their applied methods, Table~\ref{tab:methodoverview} allows a quick reference of the reviewed work by the aggregation level and forecasting horizon. Further, each of the sections is self-contained, so that a reader interested in trends found, applications and datasets identified, or the conclusions drawn, can skip ahead to the respective section.

\begin{figure}
	\centering
	\includegraphics[width=0.60\linewidth]{figures/Drawing.pdf}
	\caption{The structure of this review.}
	\label{fig:flowchart}
\end{figure}
