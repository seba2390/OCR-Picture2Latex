\section{Discussion}
\label{sec_discussion}


The area of LV load forecasting has garnered growing attention in the last few years, which is evident from a significant increase in the number of publications, datasets, methodologies and applications, as discussed in this paper. While these recent developments are helping to move towards increasingly efficient, digital, and easy to monitor local energy systems, we are also faced with new challenges and subsequent opportunities. In this section, we discuss some of these challenges and share our views and recommendations. We also discuss gaps in the literature and possible (and desirable) future directions. 

\subsection{Recommendations}

\begin{itemize}
	\item \textit{Tackling single-source data bias} - A vast majority of the literature on modelling smart electricity meter time series have employed the Irish CER dataset, potentially because it was one of the first open-source repositories of such a dataset. \textbf{Recommendations:} to better gauge the practical scalability of models and the generalizability of findings reported using the Irish CER data, external validation using other datasets is needed. An interesting line of study would be to train models using say, the Irish CER data, but then validate and compare the models using multiple datasets from different sources (a validation approach that is sometimes used in the area of medical diagnostics where data from multiple independent cohorts are used).  Here, a large suitable, more recent data set is the UK Low Carbon London trial~\cite{UK2014ulc} that has been archived for better analysis and referenced in the Monash Time Series Forecasting Archive~\cite{godahewa2021mts} in a pre-processed and raw version.
	
	\item \textit{Towards clarity in problem definition} -  To improve readability, it will help if the papers clearly and concisely describe the problem statement upfront, before delving into the intricacies of the methodology. From some papers, even a plot of the time series, or details regarding the forecast horizon, were missing. \textbf{Recommendations:} information such as forecast horizon, level of aggregation (one household, feeder, region, etc.), a plot of the time series, sampling rate, error measures, data source, pre-processing steps (for handling missing observations and anomalous load profiles due to public holidays etc), and forecasting scheme (direct vs iterative) must be provided. 
	
	\item \textit{Need for benchmarks and robust validation}: Numerous articles did not employ benchmarks to compare their models against (they typically compared different variants of their presented approach). For HV load modelling, a range of naive benchmarks (such as seasonal random walk, seasonal moving average) and sophisticated benchmarks (such as SARMA, HWT exponential smoothing, ANNs) are typically used. However, this is not the case for papers dealing with LV load modelling. \textbf{Recommendations:} a comparison with an existing methodology proposed in the literature for a similar problem/data is needed, or at least some of the naive and sophisticated benchmarks specified above could be employed such as those used in \cite{Haben2019stl}. Also, how the data is split into train and test sets should be clearly stated. Very often hyper-parameter tuning was missing or hyper-parameters were chosen based on the test set, whereas a validation set should have been defined and be separate from the test set. The test dataset should be sufficiently large and representative.
	
	\item \textit{Modelling uncertainty due to weather}: A significant number of the papers that we surveyed used no weather information. A few papers that used weather variables, used weather actuals, thereby under-reporting the forecast errors. Only a handful of papers used weather predictions, but typically for one weather variable that was obtained from one weather station. \textbf{Recommendations:} weather ensemble predictions (ideally for different weather variables obtained from multiple weather stations) need to be used while modelling the LV load. Weather ensembles from weather stations are prone to be biased and under-dispersed.  There have been significant advancements in the area of numerical weather predictions in the last decade, unfortunately, these advancements have not yet translated into improved LV load forecast accuracy.  
	
	\item \textit{Moving towards probabilistic forecasting} - Research on LV load forecasting has tended to focus on generating point forecasts. However, the time series at the LV level exhibit considerable variability along with seasonality. \textbf{Recommendations:} (1) The aim of modelling should be to generate and evaluate probabilistic forecasts (and not just focus on a point estimate or a pre-specified quantile). (2) Studies focusing on peak LV load forecasts should use modified error measures that avoid the double penalty. An interesting line of work would be on probabilistic peak forecast evaluation at the LV level. (3) Model estimation based on in-sample probabilistic forecast error measures needs to be considered. Crucially, studies should report the model hyper-parameters along with details of the estimation framework, to help improve reproducibility. (4) A plot of probabilistic forecast accuracy (ideally quantified using a strictly proper scoring rule) versus forecast horizons should be provided, as different models may perform well at different horizons. (5) If needed, statistical significance tests should be used for model comparison.
	
	\item \textit{Improving access to literature}: despite generally a trend towards more open research by using preprint servers and Open Access journals, currently, the majority of journals in this area are not Open Access. This makes it expensive especially for industry, as unlike universities, they typically don't have subscriptions with the publishers. Further, several papers were found only in non-English speaking languages limiting the contribution to the international forecasting community. \textbf{Recommendations:} more journals should allow more explicitly the sharing of preprints for more open research. Research funding agencies should fund and even encourage open access publication (public money for public research), despite being often a considerably more expensive option. National funding agencies should further encourage the exchange with international research communities by encouraging publications, in both native languages and English.
	
	\item \textit{Becoming a supportive community}: a considerable proportion of papers used closed data sets \emph{and} closed code, while in many other communities the code is typically shared along with the paper. However, the papers reviewed in this work typically did not publish the modelling and algorithmic code. \textbf{Recommendations:} sharing the code or even the pseudo-code (along with model hyper-parameters) will increase the likelihood of adoption of the proposed methodologies. 
	
	\item \textit{Applications}: for many applications, naive forecast methods are used or an accurate forecast is assumed without much investigation into what role the forecast accuracy plays in the application. Often a forecasting method is not benchmarked at all, or worse still, no forecast is given. These scenarios mean that there is no clarity of the effect of the forecast. If a forecast has minimal impact then practitioners may apply far too much time and effort to develop a very accurate model. In contrast, if an accurate forecast is key, the application may be abandoned if the required results are not delivered. \textbf{Recommendations:} The accuracy of the forecast model and the comparison to a benchmark must be presented. This will allow a proper assessment of the forecast accuracy and the influence of forecast errors on the performance within the application. 
	
	\item \textit{Privacy}: one of the main obstacles to obtaining shared datasets from LV networks and households is data privacy, i.e. personal and life-style information can be deduced from the energy usage, and operational details for LV networks usually cannot be exposed due to safety or commercial reasons. Privacy-protecting analytics that mitigates those issues, such as differential privacy, federated learning and other methods are already in development and used (e.g. in life-sciences datasets to protect privacy) but not much is applied to energy data (we listed a few examples of papers that considered privacy protection measures in \ref{otherapps}). \textbf{Recommendations:} developing and evaluating methods for privacy protection in LV datasets will eventually enable opening and sharing (or at least modelling) of many more datasets. 
	
	\item \textit{Use of Computational Intelligence Models}: there is a recent trend to use computationally complex models from the deep learning domain on all kinds of problems with abundant data. With a plethora of papers on novel deep learning architectures appearing as preprints and the major machine learning conferences, it is possible to pick novel algorithms and mechanisms, apply them to the LV load forecasting problem and with sufficient tuning achieve good performance on a test set. Given the above challenges (dataset bias and missing benchmarks), it is therefore hard to judge the contribution of a novel approach. Compared to statistical models, current computational models are generally less interpretable and computationally expensive, making them less attractive to industrial and commercial organisations, as they are more costly to run, less ecologically sustainable, and harder to "trust". \textbf{Recommendations:} Instead of comparing computational approaches only to other computational approaches or to simple benchmarks, novel approaches should be compared to strong statistical models like SARMA and HWT exponential smoothing. Model evaluation should be done following cross-evaluation and a large enough test set (separate from that used for model tuning) should be used for comparison to other strong benchmarks to ensure generalisation. The evaluation should further consider not only forecasting errors, but also computational complexity by reporting e.g. running times or even energy consumption and qualitative properties like model interpretability.
	
	
\end{itemize}

\subsection{Bridging the gap and future directions}
From this wide coverage of current work on methods and applications, we have chosen the most pressing and important open problems and identified gaps and directions for future research.

Firstly, it should be noted that no individual method can currently be considered state-of-the-art in LV load forecasting, i.e. no method has been shown to consistently provide significantly improved results (relative to either application or appropriate metrics) over any other methods. In the literature, there are only a few wide-ranging comparisons, and even these typically only focus on a few models within a specific family of methods (neural networks or regression models, for example). As discussed in Section \ref{sec:LVLF-applications}, understanding which forecasting methods may be most appropriate or optimal for each application is even more unclear. Minimal focus is applied on the forecasts, which are simply treated as required inputs rather than  essential components of the respective problems. In the early development of the technical solutions to these applications, this may be understandable, however as the area matures, focus on the forecasts will be essential and will require detailed analysis of their role and structure. 

A major motivation for LV level forecasts is that in a smart grid, LV networks open up a wide range of
solutions and opportunities for new applications as shown in Section \ref{sec:LVLF-applications}. Not only
does this mean that algorithms must be tested with regard to how viable they are for the
particular application, but they might also require new and novel inputs, not usually required at
HV level, as demonstrated in Section \ref{subsec:novel_var}. Responses to explanatory variables are also largely unexplored at the LV level, as shown in Section \ref{subsec:expl_var}. 
Furthermore, given the huge recent progress in numerical weather prediction, an interesting avenue of future research would be
\begin{itemize}
	\item to use post-processed weather ensemble predictions to generate multi-step probabilistic forecasts of load at different levels of the LV hierarchy. 
	\item to investigate how different forecasting skills for renewable generation and skills for demand vary over different time-horizons and spatial resolutions combined.
\end{itemize}

Moreover, given that LV networks' loads are much more volatile, the development of probabilistic forecasts is an obvious direction for future work, together with developing error metrics suitable for training and use of those models, which are able to cope with the double penalty effect discussed in Section \ref{subsec:eval}. 

Another major topic, given the lack of data and benchmarks, is collaboration and sharing of datasets. There is an urgent need for the development of privacy-protecting analytics that could overcome commercial, safety and privacy concerns -  major barriers toward opening more LV load datasets.
Several open problems are related to this area: 
\begin{itemize}
	\item robust creation of synthetic, `look alike' data,  (e.g. by adding noise to datasets,  applying differential privacy, etc.);
	\item development of new or adaptation of existing  privacy protection techniques, such as federated learning etc.;
	\item quantification of how increasing privacy protection influences the accuracy of forecasting methods;
	\item exploration of adversarial techniques to keep those mitigation methods resilient ( i.e. showing that the data cannot be used in combination with other datasets to deduce identity etc.).
\end{itemize}
Finally, there is a need for developing pragmatic methods to achieve explainability of AI and ML methods in the LV load forecasting context. This will require  working interdisciplinarily with end users (power and system engineers), and developing a deeper understanding of the operational environment by AI scientists.
