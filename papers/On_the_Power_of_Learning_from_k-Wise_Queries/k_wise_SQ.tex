\documentclass[11pt]{article}
\usepackage{amssymb}
 %\setlength\parindent{0pt}
\usepackage{amsfonts}
\usepackage{amsmath}
\usepackage{amsthm}
\usepackage{latexsym}

\usepackage{bbm}

\usepackage{enumitem}
%\usepackage{enumerate}

\usepackage{mathtools}

\usepackage{xspace}

\usepackage[pagebackref=true]{hyperref}
\hypersetup{
    unicode=false,          % non-Latin characters in Acrobat bookmarks
    colorlinks=true,        % false: boxed links; true: colored links
    linkcolor=red,          % color of internal links (change box color with linkbordercolor)
    citecolor=blue,        % color of links to bibliography
    filecolor=magenta,      % color of file links
    urlcolor=cyan           % color of external links
}

\usepackage{hyperref,cleveref}
\usepackage{amsmath}
\usepackage{amsthm}
\usepackage{amsfonts}
\usepackage{fullpage,appendix}
%\usepackage{algorithm}
%\usepackage{algorithmic}
%\usepackage[ruled,vlined]{algorithm2e}


\DeclareMathOperator*{\argmin}{arg\,min}
\DeclareMathOperator*{\argmax}{arg\,max}

\usepackage[ruled]{algorithm}
\usepackage{algpseudocode}
\usepackage{algorithmicx}
%\floatname{algorithm}{Protocol}

%\usepackage{dsfont}
\usepackage{color}
\usepackage{tikz}

\providecommand{\remove}[1]{}
\providecommand{\eg}{{\em e.g.}~}
\renewcommand{\algorithmicrequire}{}
\newcommand{\eqdef} {\mathrel{\stackrel{\makebox[0pt]{\mbox{\normalfont\tiny
def}}}{=}}}
%\renewcommand{\thealgorithm}{}
\newcommand{\etal}{et al.\ }
\newcommand{\spn}{\S^{n-1}}
\newcommand{\tm}{\tilde{m}}
\newcommand{\ignore}[1]{}
\definecolor{corlinks}{RGB}{64,128,128}
\definecolor{cormenu}{RGB}{0,37,94}
\definecolor{corurl}{RGB}{0,46,91}
\definecolor{darkgreen}{rgb}{0,0.5,0}

%\setlength{\parindent}{0in}
\newcommand{\on}{\{-1,1\}}
\newcommand{\1}{\mathds{1}}
\newcommand{\UCC}{\mathsf{UCC}}
\newcommand{\err}{\mathsf{err}}

\newcommand{\Cov}{{\rm Cov}}

\newcommand{\bkets}[1]{\left(#1\right)}
\newcommand{\sbkets}[1]{\left[#1\right]}
\newcommand{\braces}[1]{\left\{#1\right\}}
\newcommand{\ip}[2]{\langle #1, #2 \rangle}


\newcommand{\CSDR}{\mathsf{cRSD}}
\newcommand{\RSD}{\mathsf{RSD}}
\newcommand{\SD}{\mathsf{SD}}
\newcommand{\Rcvr}{\mathsf{Rcvr}}
\newcommand{\fract}{\mathsf{frac}}
\newcommand{\Hyperplane}{\mathsf{Hyperplane}}

\newcommand{\Hyp}{\mathsf{Hyp}}
\newcommand{\Sub}{\mathsf{Sub}}
\newcommand{\LR}{\mathsf{LR}}

\newcommand{\rk}{\mathsf{rk}}

\newcommand{\dcdf}{\mathsf{dcdf}}
\newcommand{\cor}{\mathsf{cor}}
\renewcommand{\P}{\mathsf{P}}
\newcommand{\BPP}{\mathbf{BPP}}
\newcommand{\E} {\mathbb{E}}
\DeclareMathOperator*{\ex}{\mathbb{E}}
\DeclareMathOperator*{\pr}{\mathsf{Pr}}
\newcommand{\R}{\mathbb{R}}
\newcommand{\size}{\mathsf{size}}
\newcommand{\erf}{ \mathsf{erf} }
\newcommand{\D}{\mathcal{D}}
\newcommand{\calD}{\mathcal{D}}
\newcommand{\calP}{\mathcal{P}}
\newcommand{\calB}{\mathcal{B}}
\newcommand{\calV}{\mathcal{V}}
\newcommand{\calF}{\mathcal{F}}
\newcommand{\calZ}{\mathcal{Z}}
\newcommand{\calU}{\mathcal{U}}
\newcommand{\calQ}{\mathcal{Q}}
\newcommand{\calA}{\mathcal{A}}
\newcommand{\calC}{\mathcal{C}}
\newcommand{\cale}{\mathcal{E}}
\renewcommand{\H}{\mathbb{H}}
\newcommand{\cH}{\mathcal{H}}
\newcommand{\G}{\mathbb{G}}
\newcommand{\N}{\mathbb{N}}
\newcommand{\cN}{\mathcal{N}}
\newcommand{\B}{\mathbb{B}}
\newcommand{\A}{\mathcal{A}}
\newcommand{\cC}{\mathcal{C}}
\newcommand{\C}{\mathbb{C}}
\newcommand{\W}{\mathbb{W}}
\newcommand{\U}{\mathbb{U}}
\newcommand{\cU}{\mathcal{U}}
\newcommand{\SO}{\mathsf{SO}}
\newcommand{\SU}{\mathsf{SU}}
\newcommand{\Sign}{\mathsf{Sign}}
\renewcommand{\L}{\mathcal{L}}
\renewcommand{\O}{\mathcal{O}}
\newcommand{\tmu} {\tilde{\mu}}
\newcommand{\I}{\mathcal{I}}
\newcommand{\tnu}{\tilde{\nu}}
\newcommand{\M}{\mathcal{M}}
\newcommand{\norm}[1]{||#1||}
\newcommand{\poly}{\mathsf{poly}}
\newcommand{\Prob}{\Pr}
\renewcommand{\S}{\mathbb{S}}
\newcommand{\STAT}{\mathsf{STAT}}
\newcommand{\VSTAT}{\mathsf{VSTAT}}
\newcommand{\wRFA}{\mathsf{wRFA}}
%\newcommand{\CSTAT}{\mathsf{CSTAT}}
\newcommand{\F}{\mathbb{F}}
\newcommand{\calf}{\mathcal{F}}
\newcommand{\Fc}{\mathcal{F}}
\newcommand{\Dc}{\mathcal{D}}
\newcommand{\floor}[1]{\left \lfloor #1 \right \rfloor}
\newcommand{\zo}{\{0, 1\}}

\newcommand{\KL}{{\mathrm{KL}}}
\newcommand{\Div}{{\mathrm{D}}}
\newcommand{\KLR}{R_{\mathrm{KL}}}

\newcommand{\cp}{\mathsf{cp}}

\newcommand{\st}{\mathsf{st}}

\newcommand{\Ex}{\mathbb E}

%complexity classes
\newcommand{\Ppoly}{\mathsf{P/poly}}
\newcommand{\BPTIME}{\mathsf{BPTIME}}
\newcommand{\EXP}{\mathsf{EXP}}
\newcommand{\NP}{\mathsf{NP}}
\newcommand{\DTIME}{\mathsf{DTIME}}
\newcommand{\PSPACE}{\mathsf{PSPACE}}
\newcommand{\PH}{\mathsf{PH}}
\newcommand{\SIZE}{\mathsf{SIZE}}
\newcommand{\NEXP}{\mathsf{NEXP}}
\newcommand{\CSTAT}{\mathsf{CSTAT}}
\newcommand{\Maj}{\mathsf{MAJ}}
\renewcommand{\cal}[1]{\mathcal{#1}}
\newcommand{\eps}{\epsilon}

\newcommand{\Corr}{\mathsf{Corr}}
\newcommand{\Succ}{\mathsf{Succ}}

\newcommand{\twCCU}{\mathsf{2wayCCU}}
\newcommand{\wt}{\mathsf{wt}}
\newcommand{\cald}{\mathcal{D}}

\newcommand{\Disc}{\textsc{Disc}}

\newenvironment{proofof}[1]{\noindent{\bf Proof}
of #1:\hspace*{1em}}{\qed\bigskip}

\newtheorem{fact}{Fact}[section]
\newtheorem{definition}[fact]{Definition}
\newtheorem{defn}[fact]{Definition}
\newtheorem*{notation}{Notation}
\newtheorem*{untheorem}{Theorem}
\newtheorem{theorem}[fact]{Theorem}
\newtheorem{lemma}[fact]{Lemma}
\newtheorem{lem}[fact]{Lemma}
\newtheorem{corollary}[fact]{Corollary}
\newtheorem{observation}[fact]{Observation}
\newtheorem{proposition}[fact]{Proposition}
\newtheorem{problem}[fact]{Problem}
\newtheorem{claim}[fact]{Claim}
\newtheorem{property}[fact]{Property}
\newtheorem{conjecture}[fact]{Conjecture}
\newtheorem*{note}{Note}
\newtheorem{remark}[fact]{Remark}

\newtheorem{question}{Question}
\newtheorem{answer}{Answer}
\newtheorem{goal}{Goal}

\makeatletter
\newtheorem*{rep@theorem}{\rep@title}
\newcommand{\newreptheorem}[2]{%
\newenvironment{rep#1}[1]{%
 \def\rep@title{#2 \ref{##1}}%
 \begin{rep@theorem}}%
 {\end{rep@theorem}}}
\makeatother

\newreptheorem{theorem}{Theorem}


\newcommand{\MCSP} {\mathsf{MAX}\text{-}\mathsf{CSP}}
\newcommand{\tE}{\tilde{\mathbb{E}}}
\newcommand{\Span}{\mathsf{Span}}
\newcommand{\tchi}{\tilde{\chi}}
\newcommand{\tv}{\tilde{v}}
\newcommand{\dist}{\mathsf{dist}}
\newcommand{\cl}{\mathsf{cl}}
\newcommand{\Ball}{\mathsf{Ball}}
\newcommand{\bchi}{\bar{\chi}}
\newcommand{\Proj}{\mathsf{Proj}}
\newcommand{\COMM}{\mbox{BS}}

\newcommand{\CC}{\mathsf{CC}} % Communication complexity

\newcommand{\authnote}[4]{{\bf [{\color{#3} #1's Note:} {\color{#4} #2}]}}
\newcommand{\mmod}{~\mathrm{mod}~}


\newif\ifnotes\notestrue
% \newif\ifnotes\notesfalse


\ifnotes

\newcommand{\vnote}[1]{\textcolor{red}{{\bf (Vitaly:} {#1}{\bf ) }} \marginpar{\tiny\bf
             \begin{minipage}[t]{0.5in}
               \raggedright Vitaly
            \end{minipage}}}
\newcommand{\bnote}[1]{\textcolor{red}{{\bf (Badih} {#1}{\bf ) }} \marginpar{\tiny\bf
             \begin{minipage}[t]{0.5in}
               \raggedright Badih
            \end{minipage}}}
\newcommand{\gnote}[1]{\textcolor{red}{{\bf (GV:} {#1}{\bf ) }} \marginpar{\tiny\bf
             \begin{minipage}[t]{0.5in}
               \raggedright GV
                \end{minipage}}}
\else
\newcommand{\enote}[1]{}
\newcommand{\bnote}[1]{}
\newcommand{\gnote}[1]{}
\fi


\usepackage{empheq}
\newcommand*\widefbox[1]{\fbox{\hspace{2em}#1\hspace{2em}}}

\newcommand{\atomicsolverof}[2]{\textsc{AtomicSolver}{($#1$,$#2$)}\xspace}
\newcommand{\atomicsolverproc}[2]{\textsc{AtomicSolver}{($#1$,$#2$, $a_{#1}$, $b_{#1}$, $R_{#1, #2}$ )}\xspace}
\newcommand{\atomicsolver}{\textsc{AtomicSolver}\xspace}
\newcommand{\generator}{\textsc{AuxiliaryVariableGenerator}\xspace}

\title{On the Power of Learning from $k$-Wise Queries}
\author{
Vitaly Feldman \\
IBM Research - Almaden
\and
Badih Ghazi\thanks{Work done while at IBM Research - Almaden.}\\
Computer Science and Artificial Intelligence Laboratory, MIT
}

\date{\today}

\begin{document}

%\input{statement}

\newpage

\maketitle


\begin{abstract}
\label{sec:abstract}

%% 1. what is the problem 
Scientific applications that run on leadership computing facilities often face the challenge 
of being unable to fit leading science cases onto accelerator devices due to memory constraints 
(memory-bound applications).
%
% 2. what is your solution 
In this work, the authors studied one such US Department of Energy mission-critical condensed matter 
physics application, Dynamical Cluster Approximation (DCA++), and this paper discusses how device memory-bound challenges were successfully reduced  by proposing an effective 
``all-to-all'' communication method---a ring communication algorithm. 
%
This implementation takes advantage of acceleration on GPUs and remote direct memory access (RDMA) for fast data exchange between GPUs. 
%
\\Additionally, the ring algorithm was optimized with sub-ring communicators
and multi-threaded support to further reduce communication overhead and 
expose more concurrency, respectively.
%
% 3. What's the cherry-picked evaluation result you want to mention
The computation and communication were also analyzed 
by using the Autonomic Performance Environment for Exascale 
(APEX) profiling tool,  and this paper further discusses the 
performance trade-off for the ring algorithm implementation. 
%
The memory analysis on the ring algorithm shows that the allocation size for the authors' most 
memory-intensive data structure per GPU is now reduced to $1/p$ of the original size, where $p$ is the number of GPUs in the ring communicator.
%
The communication analysis suggests that 
the distributed Quantum Monte Carlo execution time grows linearly as sub-ring size increases, and the cost of messages passing through the network interface connector could be a limiting factor.


%
% \todoRed{Ronnie: Next sentence needs rewrite, too much information about Green's function that no one knows in the abstract; recommend generalizing.} \emph {However, DCA++ is currently facing memory-bound challenge as 
% a larger device array $G_t$ is limited by device memory size, where
% $G_t$ is a two-particle Green's function that allows condensed matter
% scientists to explore larger and more complex (higher fidelity)
% physics cases.}

\end{abstract}

\keywords{DCA++, Quantum Monte Carlo, GPU Remote Direct Memory Access, memory-bound issue, exascale machines}


\newpage

\tableofcontents

\newpage

Reinforcement learning has achieved great success in areas such as Game-playing \citep{silver2018general,vinyals2019grandmaster}, robotics \cite{kober2013reinforcement}, large language models \citep{ouyang2022training}, etc.
However, due to safety concerns or physical limitations, in some real-world reinforcement learning problems, we must consider additional constraints that may influence the optimal policy and the learning process \citep{garcia2015comprehensive}.
% For example, a robotic arm must not take actions that may cause harm to itself or the environments.
A standard framework to handle such cases is the constrained Markov Decision Process (CMDP) \citep{altman1999constrained}.
Within the CMDP framework, the agent has to maximize
the expected cumulative reward while
obeying a finite number of constraints, which are usually in the form of expected cumulative cost criteria.

However, we are sometimes concerned with the problem with a continuum of constraints.
For example,
the constraints we meet might be time-evolving or subject to uncertain parameters, which
cannot be formulated as an ordinary CMDP
(see Examples \ref{Example_Time_Evolving} and  \ref{Example_Uncertain}).
In this paper we would study a generalized CMDP  
to address the above problem.  Because the constraints are not only infinite-number but also lie
in a continuous set,
the generalization is not trivial. Fortunately, we find that we can borrow the idea behind semi-infinite programming (SIP) \citep{remez1934determination, hettich1993semi} to deal with the semi-infinite constraints.
Accordingly, we propose \emph{semi-infinitely constrained Markov decision processes} (SICMDPs)
as a novel complement to the ordinary CMDP framework.
%More specifically,  an SICMDP model %, we consider 
%contains a continuum of constraints whereas an ordinary CMDP contains a finite number of constraints. 

%This generalization is natural but not trivial. However, we can brows the idea  
%The idea is quite natural and can be backtracked
%to the practice of extending linear programming to linear semi-infinite programming (LSIP) %\cite{remez1934determination, GobernaLSIO1998}.
%In addition, 
%As a complementary approach to the ordinary CMDP framework, 
%SICMDP can be used to model these problems  which cannot be described by a finite number of constraints
%that are not covered by .
%For example,
%the restrictions we consider can be time-evolving or subject to uncertain parameters
%, thus
%cannot be described by a finite number of constraints but a continuum of constraints 
%(see Examples \ref{Example_Time_Evolving} and  \ref{Example_Uncertain}).

We also present two reinforcement learning algorithms to solve SICMDPs called SI-CRL and SI-CPO, respectively.
SI-CRL is a model-based reinforcement learning algorithm designed for tabular cases, and SI-CPO is a policy optimization algorithm for non-tabular cases.
% and analyze its performance both theoretically and empirically.
The main challenge is that we need to deal with a continuum of constraints, thus reinforcement learning algorithms for ordinary CMDPs do not work anymore.
In SI-CRL, we tackle this difficulty by first transforming the reinforcement learning problem to an equivalent LSIP problem, which can then be solved using methods in the LSIP literature like the dual exchange methods \citep{Hu1990,reemtsen1998numerical}.
In SI-CPO, we resort to the idea of cooperative stochastic approximation developed in \cite{lan2020algorithms, wei2020comirror}.
As far as we know, we are the first to introduce tools from semi-infinitely programming (SIP) into the reinforcement learning community for solving constrained reinforcement learning problems.

% To the best of our knowledge, we are the first to apply tools from semi-infinitely programming (SIP) to solve reinforcement learning problems.
Furthermore, we give theoretical analysis for both SI-CRL and SI-CPO.
We decompose the error of SI-CRL into two parts: the statistical error from approximating the true SICMDP with an offline dataset and the optimization error due to the fact that the solution of the LSIP problem obtained by the dual exchange method is inexact.
On the optimization side, we show that the iteration complexity of SI-CRL is $O\left(\left\{\mathrm{diam}(Y)L\sqrt{|\gS|^2|\gA|m}/\left[(1-\gamma)\epsilon\right]\right\}^m\right)$.
On the statistical side, we show that the sample complexity of SI-CRL is $\widetilde O\left(\frac{|S|^2|A|^2}{\epsilon^2(1-\gamma)^3}\right)$ if the offline dataset is generated by a generative model, and $\widetilde O\left(\frac{|S||A|}{\nu_{\min} \epsilon^2(1-\gamma)^3}\right)$ if the dataset is generated by a probability measure $\nu$ as considered in \cite{chen2019information}.
Here $\widetilde O$ means that all logarithm terms are discarded.
For SI-CPO, things become a little more complicated because other than the statistical error and the optimization error, we also need to consider the function approximation error, which comes from imperfect policy parametrizations.
It is shown if the function approximation error can be controlled to $O(\epsilon)$ order, the iteration complexity of SI-CPO is $\widetilde{O}\left(\frac{1}{\epsilon^2(1-\gamma)^6}\right)$ and the sample complexity of SI-CPO is $\widetilde{O}(\frac{1}{\epsilon^4(1-\gamma)^{10}})$.
Here our iteration complexity bound is equivalent to a typical $\widetilde O(1/\sqrt{T})$ global convergence rate.

We perform a set of numerical experiments to illustrate the SICMDP model and validate our proposed algorithms.
Specifically, we examine two numerical examples, namely the discharge of sewage and ship route planning.
Through the discharge of sewage example, we show the advantage of the SICMDP framework over the CMDP baseline obtained by naive discretization in modeling realistic sequential decision-making problems.
Moreover, we demonstrate the effectiveness of the SI-CRL and SI-CPO algorithms in such tabular environments. 
In the ship route planning example, we illustrate the benefits of the SICMDP framework and the ability of the SI-CPO algorithm to address complex continuous control tasks involving continuous state spaces with modern deep reinforcement learning techniques.

% In summary, our contributions are listed as follows.
% First, we present the SICMDP model, which can be viewed as a generalization of the ordinary CMDP model.
% Second, we propose an algorithm to perform reinforcement learning for SICMDPs, which is called SI-CRL, and we believe that we are the first to apply tools from SIP
% to solve reinforcement learning problems.
% Third, we give a theoretical analysis of SI-CRL and identify both its sample complexity and iteration complexity.
% In addition, we perform numerical experiments to illustrate the SICMDP model and validate the SI-CRL algorithm.
% \{This paragraph can be removed!!! \}







We study the problem of selling $m$ items to $n$ buyers. We denote a bundle of items as a quantity vector $\vec{q} \in \Z_{\geq 0}^m$. The number of units of item $i$ in the bundle is $q[i]$. The bundle consisting of only one copy of the $i^{th}$ item is denoted by the standard basis vector $\vec{e}_i$, where $e_i[i] = 1$ and $e_i[j] = 0$ for all $j \not= i$. Each buyer $j \in [n]$ has a valuation function $v_j$ over bundles of items. We denote an allocation as $Q = \left(\vec{q}_1, \dots, \vec{q}_n\right)$ where $\vec{q}_j$ is the bundle that buyer $j$ receives. The cost to produce $\vec{q}$ is $c\left(\vec{q}\right)$ and the cost to produce the allocation $Q$ is $c\left(Q\right)$.
Suppose there are $\kappa_i$ units available of item $i$. Let $K = \prod_{i = 1}^m \left(\kappa_i+1\right)$. We use $\vec{v}_j = \left(v_j\left(\vec{q}_1\right), \dots, v_j\left(\vec{q}_K\right)\right)$ to denote buyer $j$'s values for all of the $K$ bundles and we use $\vec{v} = \left(\vec{v}_1, \dots, \vec{v}_n\right)$ to denote a vector of buyer values. We use the notation $\cX$ to denote the set of all valuation vectors $\vec{v}$. Additive buyers have values $v_j\left(\vec{q}\right) = \sum_{i = 1}^m q[i] v_j\left(\vec{e}_i\right)$ and unit-demand buyers have values $v_j\left(\vec{q}\right) = \max_{i : q[i] \geq 1} v_j\left(\vec{e}_i\right)$. The mechanisms  we study are dominant strategy incentive compatible, so we assume that the bids equal the buyers' valuations.

There is an unknown distribution $\pazocal{D}$ over buyers' values. 
The notation $\profit_M\left(\vec{v}\right)$ denotes the profit of a mechanism $M$ on the valuation vector $\vec{v}$. We use the notation $\profit_{\dist}\left(M\right) = \E_{\vec{v} \sim \dist}\left[\profit_M\left(\vec{v}\right)\right]$ and for a set of samples $\sample$, we use the notation \[\profit_{\sample}\left(M\right) = \frac{1}{|\sample|}\sum_{\vec{v} \in \sample}\profit_M\left(\vec{v}\right).\]

We study real-valued functions parameterized by vectors $\vec{p}$ in $\R^d$, denoted as $f_{\vec{p}}:\domain \to \R.$ For a fixed $\vec{v} \in \domain$, we often consider $f_{\vec{p}}\left(\vec{v}\right)$ as a function of its parameters, which we denote as $f_{\vec{v}}\left(\vec{p}\right)$.

\section{Separation of $(k+1)$-wise from $k$-wise queries}\label{sec:pf_sep}
\newcommand{\ind}{\mathbbm{1}}

We start by describing the concept class $\calC$ that we use to prove Theorem~\ref{thm:k_wise_sep}. Let $\ell$ and $k$ be positive integers with $\ell \geq k+1$. The domain will be $\mathbb{F}_p^{\ell}$. For every $a = (a_1,\dots, a_{\ell}) \in \mathbb{F}_p^{\ell}$, we consider the hyperplane
$$ \Hyp_a \doteq \{ z = (z_1,\dots,z_{\ell}) \in \mathbb{F}_p^{\ell}: z_{\ell} = a_1 z_1 + \dots + a_{\ell-1} z_{\ell-1} + a_{\ell}\}.$$
We then define the Boolean-valued function $f_a: \mathbb{F}_p^{\ell} \to \{\pm 1\}$ to be the indicator function of the subset $\Hyp_a \subseteq \mathbb{F}_p^{\ell}$, i.e., for every $z \in \mathbb{F}_p^{\ell}$,

\[
 f_a(z) = \begin{dcases*}
        +1  & if $z \in \Hyp_a$,\\
        -1 & otherwise.
        \end{dcases*}
\]

Then, we will consider the concept classes $\calC_{\ell} \doteq \{f_a: a \in \mathbb{F}_p^{\ell}\}$. We denote $\calC \doteq \calC_{k+1}$. We start by stating our upper bound on the $(k+1)$-wise SQ complexity of the distribution-independent PAC learning of $\calC_{k+1}$.

\begin{lem}[$(k+1)$-wise upper bound]\label{le:k_ub}
Let $p$ be a prime number and $k$ be a positive integer. There exists a distribution-independent PAC learning algorithm for $\calC_{k+1}$ that makes at most $t \cdot \log(1/\eps) $ queries to $\STAT^{(k+1)}(\eps/t)$, for some $t = O_k(\log{p})$.
\end{lem}

We next state our lower bound on the $k$-wise SQ complexity of the same tasks considered in Lemma~\ref{le:k_ub}.

\begin{lem}[$k$-wise lower bound]\label{le:k_plus_1_lb}
 Let $p$ be a prime number and $\ell$, $k$ be positive integers with $\ell \geq k+1$ and $k = O(p)$. There exists $t =  \Omega\big(p^{(\ell-k)/4}\big)$ such that any distribution-independent PAC learning alogrithm for $\calC_{\ell}$ with error at most $1/2-2/t$ that is given access to $\STAT^{(k)}(1/t)$ needs at least $t$ queries.
\end{lem}

Note that  Lemma~\ref{le:k_ub} and Lemma~\ref{le:k_plus_1_lb} imply Theorem~\ref{thm:k_wise_sep}.

\subsection{Upper bound}

%Note that it is enough to prove Lemma~\ref{le:k_ub} in the case where $\ell = k+1$. This is because in the case of $\ell > k+1$, we can choose a subset $S \subseteq [\ell]$ of coordinates of size $\ell - k - 1$, and consider all $p^{\ell-k-1}$ settings of the variables $\{a_i: i \in S\}$. For each such setting, we can run the learning algorithm for the case where $\ell = k+1$. This results in a list of at most $p^{\ell-k-1}$ candidate hypotheses, one of which is guaranteed to be close to the true concept. To find this good hypothesis, we then estimate the empirical error for each candidate hypothesis using unary SQs, and output a hypothesis with small enough error. Henceforth, we assume that $\ell = k+1$ and prove Lemma~\ref{le:k_ub}.

\paragraph{Notation}
We first introduce some notation that will be useful in the description of our algorithm. For any matrix $M$ with entries in the finite field $\mathbb{F}_p$, we denote by $\rk(M)$ the rank of $M$ over $\mathbb{F}_p$. Let $(a_1,\dots,a_{k+1}) \in \mathbb{F}_p^{k+1}$ be the unknown vector that defines $f_a$ and $P$ be the unknown distribution over tuples $(z_1, \dots, z_{k+1}) \in \mathbb{F}_p^{k+1}$. 

Note that $\Hyp_a$ is an affine subspace of $\mathbb{F}_p^{k+1}$. To simplify our treatment of affine subspaces, we embed the points of $\mathbb{F}_p^{k+1}$ into $\mathbb{F}_p^{k+2}$ by mapping each $z \in \mathbb{F}_p^{k+1}$ to $(z,1)$. This embedding maps every affine subspace $V$ of $\mathbb{F}_p^{k+1}$ to a linear subspace $W$ of $\mathbb{F}_p^{k+2}$, namely the span of the image of $V$ under our embedding. Note that this mapping is one-to-one and allows us to easily recover $V$ from $W$ as $V = \{z \in \mathbb{F}_p^{k+1} \ | \ (z,1) \in W\}$.  Hence given $k+1$ examples
$$\big((z_{1,1}, \dots, z_{1,k+1}),b_1\big),\big((z_{2,1}, \dots, z_{2,k+1}),b_2\big), \dots, \big((z_{k+1,1}, \dots, z_{k+1,k+1}),b_{k+1}\big)$$  we define the matrix:
\begin{equation}\label{eq:Z_mat_def}
Z \doteq
\begin{bmatrix}
z_{1,1}       & z_{1,2} &  \cdot & z_{1,k+1} & 1 \\
z_{2,1}       & z_{2,2} & \cdot & z_{2,k+1} & 1 \\
\cdot       & \cdot & \cdot & \cdot & \cdot \\
\cdot       & \cdot & \cdot & \cdot & \cdot \\
z_{k+1,1}       & z_{k+1,2} & \cdot & z_{k+1,k+1} & 1
\end{bmatrix}.
\end{equation}
For $\ell \in [k+1]$ we also denote by $Z_\ell$ the matrix that consists of the top $\ell$ rows of $Z$.  Further, for a $(k+1)$-wise query function $\phi\big((z_1,b_1),\ldots,(z_{k+1},b_{k+1})  \big)$, we use $Z$ to refer to the matrix obtained from the inputs to the function.

Let $Q$ be the distribution defined by sampling a random example $\big((z_{1}, \dots, z_{k+1}),b\big)$, conditioning on the event that $b=1$ and outputting $(z_{1}, \dots, z_{k+1},1)$. Note that if the examples from which $Z$ is built are positively labeled i.i.d. examples then each row of $Z$ is sampled i.i.d. from $Q$ and hence $Z_\ell$ is distributed according to $Q^\ell$.
%Let $Q$ be the distribution over $\mathbb{F}_p^k$ obtained by sampling $(z_1, \dots, z_{k+1}) \sim P$, conditioning on the event that $z_{k+1} = a_1 z_1 + \dots + a_k z_k +a_{k+1}$, and outputting $(z_1, \dots, z_k)$. Equivalently, we can define the distribution $Q$ as sampling a random example, conditioning on the event that its label is $1$, and restricting to the first $k$ coordinates.  We will
We denote by $1^{k+1}$ the all $+1$'s vector of length $k+1$.

\paragraph{Learning algorithm}
We start by explaining the main ideas behind the algorithm. On a high level, in order to be able to use $(k+1)$-wise SQs to learn the unknown subspace, we need to make sure that there exists an affine subspace that contains most of the probability mass of the positively-labeled points and
that is spanned by $k+1$ random positively-labeled points with noticeable probability. Here, the probability is with respect to the unknown distribution over labeled examples. Thus, for positively labeled tuples $(z_{1,1}, \dots, z_{1,k+1})$, $(z_{2,1}, \dots, z_{2,k+1})$, $\dots$, $(z_{k+1,1}, \dots, z_{k+1,k+1})$, we  consider the $(k+1) \times (k+2)$ matrix $Z$ defined in \Cref{eq:Z_mat_def}. If $W$ is the row-span of $Z$, then the desired (unknown) affine subspace is the set $V$ of all points $(z_1, \dots,z_{k+1})$ such that $(z_1, \dots,z_{k+1}, 1) \in W$.
	
	
	If the (unknown) distribution over labeled examples is such that with noticeable probability, $k+1$ random positively-labeled points form a full-rank linear system (i.e., the matrix $Z$ has full-rank with noticeable probability conditioned on $(b_1,\dots,b_{k+1}) = 1^{k+1}$), we can use $(k+1)$-wise SQs to find, one bit at a time, the $(k+1)$-dimensional row-span $W$ of $Z$, and we can then output the set $V$ of all points $(z_1, \dots,z_{k+1})$ such that $(z_1, \dots,z_{k+1}, 1) \in W$ as the desired affine subspace (below, we refer to this step as the Recovery Procedure).
		
	 We now turn to the (more challenging) case where the system is not full-rank with noticeable probability (i.e., the matrix $Z$ is rank-deficient with high probability conditioned on $(b_1,\dots,b_{k+1}) = 1^{k+1}$). Then, the system has rank at most $i$ with high probability, for some $i < k+1$. There is a large number of possible $i$-dimensional subspaces and therefore it is no longer clear that there exists a single $i$-dimensional subspace that contains most of the mass of the positively-labeled points. However, we demonstrate that for every $i$, if the rank of $Z$ is at most $i$ with sufficiently high probability, then there exists a \emph{fixed} subspace $W$ of dimension at most $i$ that contains a large fraction of the probability under the row-distribution of $Z$ (it turns out that if this subspace has rank equal to $i$, then it should be \emph{unique}). We can then use $(k+1)$-wise SQs to output the affine subspace $V$ consisting of all points $(z_1,\dots,z_{k+1})$ such that $(z_1,\dots,z_{k+1},1) \in W$ (via the Recovery Procedure).


 %To ensure this, we prove (in \Cref{le:ex_sub}) that
%Note that in principle this can be due to the distribution over individual rows of $Z$ having small positive mass on \emph{several} subspaces each of dimension at most $i$. But in order to recover a meaningful subspace using statistical queries, we need the rows of $Z$ to lie inside a \emph{fixed} $i$-dimensional subspace with high probability.
	%Note that such a subspace corresponds to a subset of columns of $Z$ that are linearly independent with high probability. \textcolor{red}{Once we find this subspace}, we can set the redundant coordinates of $(a_1,\dots,a_{k+1})$ to $0$'s and solve for the remaining coordinates using the same ``recovery'' procedure as before: namely, we invert the lower-dimensional full-rank system and output the bits one at a time using $(k+1)$-wise SQs.
	The general description of the algorithm is given in Algorithm~\ref{alg:k_wise_SQ}, and the Recovery Procedure (allowing the reconstruction of the affine subspace $V$) is separately described in \Cref{alg:recovery}. We denote the indicator function of event $E$ by $\ind(E)$. Note that the statistical query corresponding to the event $\ind(E)$ gives an estimate of the probability of $E$.
	%\vnote{Matrix $Z$ is defined only under this condition anyway. There is some confusion here between inputs to a query and (conditioned) random variables used for analysis. It would be more clear to use separate symbols for them.}
\begin{algorithm}[H]
\caption{$(k+1)$-wise SQ Algorithm}
\label{alg:k_wise_SQ}
{\bf Inputs.} $k \in \mathbb{N}$, error probability $\epsilon > 0$.\\
{\bf Output.} Function $f:\mathbb{F}_p^{k+1} \to \{\pm 1 \}$.
\begin{algorithmic}[1]
\State Set tolerance of each SQ to $\tau = (\epsilon/2^{c\cdot(k+2)})^{(k+1)^{k+3}}$, where $c>0$ is a large enough absolute constant.
\State Define the threshold $\tau_i = 2^{c \cdot (k+2-i)} \cdot k \cdot \tau^{1/(k+1)^{k+2-i}}$ for every $i \in [k+1]$.
\State Ask the SQ $\phi(z,b) \doteq \ind(b=1)$ and let $w$ be the response.
  \If{$w \le \epsilon -\tau$}
    \State\label{st:early_term} Output the all $-1$'s function.
  \EndIf
% \State For $(z_1,\dots,z_{k+1}) \in (\mathbb{F}_p^{k+1})^{k+1}$ and $(b_1,\dots,b_{k+1}) \in \{\pm 1\}^{k+1}$, let:
\State Let $\widetilde{\phi}\big((z_1,b_1),\ldots,(z_{k+1},b_{k+1}) \big) \doteq \ind((b_1,\dots,b_{k+1}) = 1^{k+1})$.
\State\label{st:v_resp} Ask the SQ $\widetilde{\phi}$ and let $v$ be the response.
\For{$i = k+1$ down to $1$}
       % \State For $(z_1,\dots,z_{k+1}) \in (\mathbb{F}_p^{k+1})^{k+1}$ and $(b_1,\dots,b_{k+1}) \in \{\pm 1\}^{k+1}$, let:
\State Let $\phi_i\big((z_1,b_1),\ldots,(z_{k+1},b_{k+1})  \big) \doteq \ind((b_1,\dots,b_{k+1}) = 1^{k+1}\mbox{ and }\rk(Z) = i$).
	\State Ask the SQ $\phi_i$ and let $v_i$ be the response.
  \If{$v_i/v \geq \tau_i$}
    %\State\label{st:LI_col} \textcolor{red}{Using $(k+1)$-wise SQs, find subset $S \subseteq [k+1]$ of $i$ linearly independent columns of $Z$.}
    %\State\label{st:lin_comb} Using $(k+1)$-wise SQs, $\forall t \in [k]$, write $z_t$ as a linear comb. of $\{z_s: s \in S \setminus \{k+1\}\} \cup \{1^{k+1}\}$.
    %\State Let $V \subseteq \mathbb{F}_p^{k+1}$ be the resulting subspace.
	\State Run Recovery Algorithm on input $(i,v_i)$ and let $\widehat{V}$ be the subspace of $\mathbb{F}_p^{k+1}$ it outputs.
	      		\State Define function $f:\mathbb{F}_p^{k+1} \to \{-1,1\}$ by:
	\State $f(z_1,\dots,z_{k+1}) = +1$ if $(z_1,\dots,z_{k+1}) \in \widehat{V}$.
	\State $f(z_1,\dots,z_{k+1}) = -1$ otherwise.
	\State Return $f$.
  \EndIf
      \EndFor

\end{algorithmic}
\end{algorithm}
%\vnote{In this description $Z$ and $(z_{1,k+1}, z_{2,k+1}, \dots, z_{k+1,k+1})$ are inputs to the procedure. They are not and are only defined inside a query function. So the query is defined as: $\phi_{j,d}\big((z_1,b_1),\ldots,(z_{k+1},b_{k+1})  \big)$ is equal to 1 if $(b_1,\dots,b_{k+1}) = 1^{k+1}$  and $\rk(Z) = i$ and bit $j$ of the description of the solution $a$ to the system $Ma=\bar{z}_{k+1}$ is equal to $d$. Here $M$ is the matrix that is formed from $z_1,...,z_{k+1}$ in the same as you do for $Z$ (I do not use the same name to keep the random variable and its distribution separate from specific instantiations).}
%\vnote{Another small comment is that there is no need to separately compute $S$ and $a$. There is a unique subspace and its entire description is short.}
\begin{algorithm}[H]
	\caption{Recovery Procedure}
	\label{alg:recovery}
	{\bf Input.} Integer $i \in [k+1]$.\\
	{\bf Output.} Subspace $\widehat{V}$ of $\mathbb{F}_p^{k+1}$ of dimension $i$.
	\begin{algorithmic}[1]
		%\State \textcolor{red}{Let $i = |S|$.}
		 %\State\label{st:solve} \textcolor{red}{Using $i$-wise SQs, solve the system $\{z_{k+1,\ell} = \sum_{v \in S \cap [k]} a_v z_{v,\ell} + a_{k+1}: \ell \in [i]\}$ for $a \in \mathbb{F}_p^i$.}
         \State Let $m_i = (k+2) \cdot i \cdot \lceil \log p \rceil$
		 \For{each bit $j \leq m_i$}% in the binary representation of output subspace $W$}
		 \State Define event $E_j(Z) = \ind(\text{bit } j \text{ of row span of } Z \text{ is } 1)$.
		 \State Let $\phi_{i,j} \big((z_1,b_1),\ldots,(z_{k+1},b_{k+1})  \big) \doteq \ind( E_j(Z) \mbox{ and } (b_1,\dots,b_{k+1}) = 1^{k+1}\mbox{ and }\rk(Z) = i$).
		 \State Ask the SQ $\phi_{i,j}$ and let $u_{i,j}$ be the response.
		
		    \If{$u_{i,j}/v_i \geq (9/10)$}
		    \State Set bit $j$ in binary representation of $\widehat{W}$ to $1$.
		    \Else
		    \State Set bit $j$ in binary representation of $\widehat{W}$ to $0$.
		    \EndIf
		  \EndFor
		  \State Let $\widehat{V}$ be the set all points $(z_1,\dots,z_{k+1})$ such that $(z_1,\dots,z_{k+1},1) \in \widehat{W}$.
	\end{algorithmic}
\end{algorithm}


%\texttt{<do stuff>}
%\textcolor{red}{We point out that in Step~\ref{st:LI_col} of Algorithm~\ref{alg:k_wise_SQ}, the subset $S$ can be taken to be the first subset of $i$ linearly independent columns of $Z$ that includes the last column, in some canonical ordering of the subsets of columns of $Z$. Moreover, Step~\ref{st:LI_col} can be implemented using $i \cdot \lceil \log_2(k+1) \rceil$ many $(k+1)$-wise SQs by asking queries of the form ``$(b_1,\dots,b_{k+1})=1^{k+1}$ and $\rk[Z] = i$ and the $r$th bit of the $m$th element of $S$ is $1$'' (for some $r \in [\lceil \log_2(k+1) \rceil]$ and $m \in [i]$). Furthermore, Step~\ref{st:solve} of \Cref{alg:recovery} (the recovery procedure) can be implemented using $i \cdot \lceil \log_2(p) \rceil$ many \textcolor{red}{$i$-wise} SQs by asking queries of the form ``$(b_1,\dots,b_{k+1})=1^{k+1}$ and $\rk[Z] = i$ and the $r$th bit in the binary expansion of $a_v$ is $1$ where $a$ is the solution to the system  $\{z_{k+1,\ell} = \sum_{v \in S \cap [k]} a_v z_{v,\ell} + a_{k+1}: \ell \in [i]\}$'' (for some $r \in [\lceil \log_2(p) \rceil]$ and $v \in [i]$).}

% Similarly, Step~\ref{st:lin_comb} can be implemented using $k \cdot i \cdot \lceil \log_2(p) \rceil$ many $(k+1)$-wise SQs.

\paragraph{Analysis}
We now turn to the analysis of Algorithm~\ref{alg:k_wise_SQ} and the proof of Lemma~\ref{le:k_ub}. We will need the following lemma, which shows that if the rank of $Z$ is at most $i$ with high probability, then there is a \emph{fixed} subspace of dimension at most $i$ containing most of the probability mass under the row-distribution of $Z$.
%\vnote{It would be useful to explain how this Lemma is related to the high-level description.}
\begin{lem}\label{le:ex_sub}
Let $i \in [k+1]$. If $\Pr_{Q^{k+1}}[\rk(Z) \le i] \geq 1-\xi$, then there exists a subspace $W$ of $\mathbb{F}_p^{k+2}$ of dimension at most $i$ such that $\Pr_{z \sim Q}[z \notin W] \le \xi^{1/k}$.
\end{lem}

\begin{remark}
We point out that the exponential dependence on $1/k$ in the probability upper bound in Lemma~\ref{le:ex_sub} is tight. To see this, let $p = 2$, and $\{e_1, \dots , e_k\}$ be the standard basis in $\mathbb{F}_2^k$. Consider the base distribution $P$ on $\mathbb{F}_2^k$ that puts probability mass $1-\alpha$ on $e_1$, and probability mass $\alpha/(k-1)$ on each of $e_2$, $e_3$, $\dots$, $e_k$. Then, a Chernoff bound implies that if we draw $k$ i.i.d. samples from $P$, then the dimension of their span is at most $2 \cdot \alpha \cdot k$ with probability at least $1 - \exp(-k)$. On the other hand, for any subspace $W$ of $\mathbb{F}_2^k$ of dimension $2 \cdot \alpha \cdot k$, the probability that a random sample from $P$ lies inside $W$ is only $1- \Theta(\alpha)$.
\end{remark}

To prove Lemma~\ref{le:ex_sub}, we will use the following proposition.
\begin{proposition}\label{prop:ind}
Let $\ell \in [k+1]$, $i \in [\ell-1]$ and $\eta >0$. If $\Pr_{Q^{\ell}}[\rk(Z_{\ell}) \le i] \geq 1-\eta$, then for every $\nu \in (0,1]$, either there exists a subspace $W$ of $\mathbb{F}_p^{k+2}$ of dimension $i$ such that $\Pr_{z \sim Q}[z \notin W] \le \nu$ or $\Pr_{Q^i}[\rk(Z_{i}) \le i-1] \geq 1-\eta/\nu$.
\end{proposition}

\begin{proof}
Let $p \doteq \Pr_{Q^i}[\rk(Z_{i}) \le i-1]$. For every (fixed) matrix $A_i \in \mathbb{F}_p^{i \times (k+2)}$, define
$$\mu(A_i) \doteq \Pr_{Q^\ell}[\rk(Z_{\ell}) \le i ~ | ~ Z_{i} = A_i].$$
Then,
\begin{align*}
\Pr_{Q^{\ell}}[\rk(Z_{\ell}) \le i] &= p+(1-p)\cdot \Pr_{Q^{\ell}}[\rk(Z_{\ell}) \le i ~ | ~ \rk(Z_{i}) = i]\\
&= p+(1-p)\cdot \Ex_{ Q^i}\bigg[\mu(Z_i) \bigg| ~ \rk(Z_{i}) = i \bigg].
\end{align*}
%\vnote{I think the last line can be very confusing. What is the internal probability over? I think you could clear it up by first defining $\mu(A) \doteq \Pr_{Q^\ell}[\rk(Z_{\ell}) \le i ~ | ~ Z_{i} = A]$ and then using $\Ex_{ Q^i}\bigg[\mu(Z_i) \bigg| ~ \rk(Z_{i}) = i \bigg]$ (the expectation is just over $Q^i$ since you conditioning on $\rk(Z_{i}) = i$ explicitly). }
Since $\Pr_{Q^{\ell}}[\rk(Z_{\ell}) \le i] \geq 1-\eta$, we have that
$$  \Ex_{ Q^i}\bigg[\mu(Z_i) \bigg| ~ \rk(Z_{i}) = i \bigg] \geq 1 - \eta/(1-p). $$
Hence, there exists a setting $A_i \in \mathbb{F}_p^{i \times (k+2)}$ of $Z_{i}$  such that $\rk(A_{i}) = i$ and
$$\Pr[\rk(Z_{\ell}) \le i ~ | ~ Z_{i} = A_{i}] \geq 1 - \eta/(1-p).$$
We let $W$ be the $\mathbb{F}_p$-span of the rows of $A_{i}$. Note that the dimension of $W$ is equal to $i$ and that $\Pr_{z \sim Q}[z \notin W] \le \eta/(1-p)$. Thus, we conclude that for every $\nu \in (0,1]$, either $p \geq 1-\eta/\nu$ or $\Pr_{z \sim Q}[z \notin W] \le \nu$, as desired.
\end{proof}

We now complete the proof of Lemma~\ref{le:ex_sub}.
\begin{proof}[Proof of Lemma~\ref{le:ex_sub}]
Starting with $\ell = k+1$ and $\eta = \xi$, we inductively apply Proposition~\ref{prop:ind} with $\nu = \xi^{1/k}$ until we either get the desired subspace $W$ or we get to the case where $i=1$. In this case, we have that $\Pr_{Q^{\ell}}[\rk(Z_{\ell}) \le 1] \geq 1-\xi^{1/k}$ for $\ell \geq 2$. Since the last column of $Z_{\ell}$ is the all $1$'s vector, we conclude that there exists $z^* \in \mathbb{F}_p^{k+1}$ such that $\Pr_{z \sim Q}[z \neq (z^*,1)] \le \xi^{1/k}$. We can then set our subspace $W$ to be the $\mathbb{F}_p$-span of the vector $(z^*,1)$.
\end{proof}

For the proof of Lemma~\ref{le:k_ub} we will also need the following lemma, which states sufficient conditions under which the Recovery Procedure (\Cref{alg:recovery}) succeeds.
\begin{lem}\label{le:recovery}
	Let $i \in [k+1]$. Assume that in \Cref{alg:k_wise_SQ}, $v > \epsilon^{k+1}/2$ and $v_i/v \geq \tau_i$. If there exists a subspace $W$ of $\mathbb{F}_p^{k+2}$ of dimension equal to $i$ such that
	\begin{equation}\label{eq:lemma_W_assumption}
	\Pr_{z \sim Q}[z \notin W] < \frac{\tau_i} {4 \cdot (k+1)},
	\end{equation}
	then the affine subspace $\widehat{V}$ output by \Cref{alg:recovery} (i.e., the Recovery Procedure) consists of all points $(z_1,\dots,z_{k+1})$ such that $(z_1,\dots,z_{k+1},1) \in W$.
\end{lem}

We note that \Cref{le:recovery} would still hold under quantitatively weaker assumptions on $v$, $v_i/v$ and $\Pr_{z \sim Q}[z \notin W]$ in \Cref{eq:lemma_W_assumption}. In order to keep the expressions simple, we however choose to state the above version which will be sufficient to prove \Cref{le:k_ub}. The proof of \Cref{le:recovery} appears in \Cref{subsec:pf_rec_lem}. We are now ready to complete the proof of Lemma~\ref{le:k_ub}.

\begin{proof}[Proof of Lemma~\ref{le:k_ub}]
If Algorithm~\ref{alg:k_wise_SQ} terminates at Step~\ref{st:early_term}, then the error of the output hypothesis is at most $\epsilon$, as desired. Henceforth, we assume that Algorithm~\ref{alg:k_wise_SQ} does not terminate at Step~\ref{st:early_term}. Then, we have that $\Pr[b = 1] > \epsilon$, and hence $\Pr[(b_1,\dots,b_{k+1}) = 1^{k+1}] > \epsilon^{k+1}$. Thus, the value $v$ obtained in Step~\ref{st:v_resp} of Algorithm~\ref{alg:k_wise_SQ} satisfies $v > \epsilon^{k+1} - \tau \geq \epsilon^{k+1}/2$, where the last inequality follows from the setting of $\tau$. Let $i^*$ be the first (i.e., largest) value of $i \in  [k+1]$ for which $v_i/v \geq \tau_i$. To prove that such an $i^*$ exists, we proceed by contradiction, and assume that for all $i \in [k+1]$, it is the case that $v_i/v < \tau_i$. Note that $Z$ has an all $1$'s column, so it has rank at least $1$. Moreover, it has rank at most $k+1$. Therefore, we have that
\begin{align*}
1 &= \Pr[1 \le \rk(Z) \le k+1 ~ |~ (b_1,\dots,b_{k+1})=1^{k+1}]\\
&= \displaystyle\sum\limits_{i=1}^{k+1} \Pr[\rk(Z) = i ~ | ~ (b_1,\dots,b_{k+1})=1^{k+1}]\\
&\le \displaystyle\sum\limits_{i=1}^{k+1} \frac{v_i + \tau}{v - \tau}\\
&\le 2 \cdot \displaystyle\sum\limits_{i=1}^{k+1} \frac{v_i + \tau}{v}\\
&\le 2 \cdot \displaystyle\sum\limits_{i=1}^{k+1} (\frac{v_i}{v}  + \frac{2\tau}{\epsilon^{k+1}})\\
&< 2 \cdot \displaystyle\sum\limits_{i=1}^{k+1} \tau_i + 4 \cdot (k+1) \cdot \frac{\tau}{\epsilon^{k+1}}.
\end{align*}
Using the fact that $\tau_i$ is monotonically non-increasing in $i$ and the settings of $\tau_1$ and $\tau$, the last inequality gives
\begin{align*}
1 &\le 2 \cdot (k+1) \cdot \tau_1 + 4 \cdot (k+1) \cdot \frac{\tau}{\epsilon^{k+1}} < 1,
\end{align*}
a contradiction.

We now fix $i^*$ as above. We have that
\begin{align*}
\Pr[\rk(Z) \le i^* ~ | ~ (b_1,\dots,b_{k+1})=1^{k+1}] &= 1 - \displaystyle\sum\limits_{i = i^*+1}^{k+1} \Pr[\rk(Z) = i ~ | ~ (b_1,\dots,b_{k+1})=1^{k+1}]\\
&\geq 1 - \displaystyle\sum\limits_{i = i^*+1}^{k+1} \frac{v_i + \tau}{v-\tau}\\
&\geq 1 - 2 \cdot \displaystyle\sum\limits_{i = i^*+1}^{k+1} (\frac{v_i}{v}  + \frac{2\tau}{\epsilon^{k+1}})\\
&> 1 - 2 \cdot \displaystyle\sum\limits_{i = i^*+1}^{k+1} (\tau_i + 2 \cdot \frac{\tau}{\epsilon^{k+1}})\\
& \geq 1 - 4 \cdot \displaystyle\sum\limits_{i = i^*+1}^{k+1} \tau_i\\
&\geq 1 - 4 \cdot k\cdot \tau_{i^*+1}.
%&= 1 - \textcolor{red}{2^{6\cdot (k-i^*)}\cdot k^2 \cdot  \tau^{1/k^{k-i^*}}}
\end{align*}
By Lemma~\ref{le:ex_sub}, there exists a subspace $W$ of $\mathbb{F}_p^{k+2}$ of dimension at most $i^*$ such that
\begin{equation}\label{eq:notin_W}
\Pr_{z \sim Q}[z \notin W] \le (4 \cdot k)^{1/k} \cdot \tau_{i^*+1}^{1/k}.
\end{equation}

\begin{proposition}\label{prop:failure}
	For every $i \in [k]$, we have that $(k+1) \cdot (4 \cdot k)^{1/k} \cdot \tau_{i+1}^{1/k} \le \tau_{i }/4$.
\end{proposition}
We note that \Cref{prop:failure} follows immediately from the definitions of $\tau_{i}$ and $\tau$ (and by letting $c$ by a sufficiently large positive absolute constant). Moreover, \Cref{prop:failure} (applied with $i = i^*$) along with \Cref{eq:notin_W} imply that $\Pr_{z \sim Q}[z \notin W]$ is at most $\tau_{i*}/(4(k+1))$.

By a union bound, we get that with probability at least
\begin{equation}\label{eq:alg_succ_prob}
1- (k+1) \cdot \Pr_{z \sim Q}[z \notin W] \geq 1 - \frac{\tau_{i^*}}{4},
\end{equation}
all the rows of $Z$ belong to $W$.

Since $v_{i*}/v \geq \tau_{i*}$, we also have that:
\begin{align}\label{eq:cond_lb_i_star}
\Pr[\rk(Z) = i^* ~ | ~ (b_1,\dots,b_{k+1})=1^{k+1}] &\geq \frac{v_{i*}-\tau}{v + \tau}\nonumber\\
&\geq \frac{1}{2} \cdot \frac{(v_{i*}-\tau)}{v}\nonumber\\
&\geq \frac{1}{2} \cdot (\tau_{i^*} - \frac{2 \cdot \tau}{\epsilon^{k+1}})\nonumber\\
& \geq \frac{\tau_{i^*}}{3}
\end{align}
Combining \Cref{eq:alg_succ_prob} and \Cref{eq:cond_lb_i_star}, we get that the rank of $W$ is \emph{equal to} $i^*$.

Let $V$ be the affine subspace consisting of all points $(z_1,\dots,z_{k+1})$ such that $(z_1,\dots,z_{k+1},1) \in W$. By \Cref{le:recovery}, we get that \Cref{alg:recovery} (and hence \Cref{alg:k_wise_SQ}) correctly recovers the affine subspace $V$.

We note that the function $f$ output by Algorithm~\ref{alg:k_wise_SQ} is the $\pm 1$ indicator of a subspace of the true hyperplane $\Hyp_a$. To see this, note that $f$ is the $\pm 1$ indicator function of the subspace $V$, and by Equations~(\ref{eq:notin_W}) and (\ref{eq:cond_lb_i_star}), we have that with probability at least $\tau_{i*}/12$ over $Z \sim Q^{k+1}$, all the columns of $Z$ belong to $W$ and $\rk(Z) = i^*$. Since the dimension of $W$ is equal to $i^*$ and since we are conditioning on $(b_1,\dots,b_{k+1})=1^{k+1}$, this implies that the correct label of all the points in $V$ is $+1$. Hence, $f$ only possibly errs on positively-labeled points (by wrongly giving them the label $-1$). Moreover, Algorithm~\ref{alg:k_wise_SQ} ensures that the output function $f$ gives the label $+1$ to every $(z_1,\dots,z_{k+1}) \in \mathbb{F}_p^{k+1}$ for which $(z_1,\dots,z_{k+1},1) \in W$. Therefore, the function $f$ that is output by Algorithm~\ref{alg:k_wise_SQ} (when it does not terminate at Step~\ref{st:early_term}) has error at most the right hand side of (\ref{eq:notin_W}). So to upper-bound the error probability, it suffices for us to verify that the right-hand side of (\ref{eq:notin_W}) is at most $\epsilon$. This is obtained by applying the next proposition with $i = i^*+1$.
\begin{proposition}\label{prop:tau_i_pow_1_ov_k}
	For every $i \in [k+1]$, we have that $(4 \cdot k)^{1/k} \cdot \tau_{i}^{1/k} \le \epsilon^{k} $.
\end{proposition}

The proof of \Cref{prop:tau_i_pow_1_ov_k} follows immediately from the definitions of $\tau_{i}$ and $\tau$ and by letting $c$ be a sufficiently large positive absolute constant.

The number of queries performed by the $(k+1)$-wise algorithm is at most $O(k^2 \cdot \log{p})$, and their tolerance is $\tau \geq (\epsilon/2^{c\cdot(k+2)})^{(k+1)^{k+3}}$, where $c$ is a positive absolute constant. Finally, we remark that the dependence of the SQ complexity of the above algorithm on the error parameter $\eps$ is $\eps^{-k^{O(k)}}$. It can be improved to a linear dependence on $1/\eps$ by learning with error $1/3$ and then using boosting in the standard way (boosting in the SQ model works essentially as in the regular PAC model \cite{aslam1993general}).
\end{proof}

\subsection{Lower bound}
Our proof of lower bound is a generalization of the lower bound in \cite{Feldman:16sqd} (for $\ell=2$ and $k=1$). It relies on a notion of {\em combined randomized statistical dimension} (``combined" refers to the fact that it examines a single parameter that lower bounds both the number of queries and the inverse of the tolerance).
In order to apply this approach we need to extend it to $k$-wise queries. This extension follows immediately from a simple observation. If we define the domain to be $X' \doteq X^k$ and the input distribution to be $D' \doteq D^k$ then asking a $k$-wise query $\phi:X^k \to [-1,1]$ to $\STAT^{(k)}_D(\tau)$ is equivalent to asking a unary query $\phi: X' \to [-1,1]$ to $\STAT^{(k)}_{D'}(\tau)$. Using this observation we define the $k$-wise versions of the notions from \cite{Feldman:16sqd} and give their properties that are needed for the proof of Lemma~\ref{le:k_plus_1_lb}.

\subsubsection{Preliminaries}
Combined randomized statistical dimension is based on the following notion of average discrimination.
\begin{defn}[$k$-wise average $\kappa_1$-discrimination]\label{def:kappa_1_disc}
Let $k$ be any positive integer. Let $\mu$ be a probability measure over distributions over $X$ and $D_0$ be a reference distribution over $X$. Then,
$$ \bar{\kappa}_1^{(k)}(\mu,D_0) \doteq \sup_{\phi: X^k \to [-1,+1]} \bigg\{ \Ex_{D \sim \mu}[|D^k[\phi]-D_0^k[\phi]|] \bigg\}. $$
\end{defn}

We denote the problem of PAC learning a concept class $\calC$ of Boolean functions up to error $\epsilon$ by $\mathcal{L}_{PAC}(\calC,\epsilon)$. Let $Z$ be the domain of the Boolean functions in $\calC$. For any distribution $D_0$ over labeled examples (i.e., over $Z \times \{\pm 1\}$), we define the Bayes error rate of $D_0$ to be
\begin{equation*}
\err(D_0) = \displaystyle\sum\limits_{z \in Z} \min\{D_0(z,1) , D_0(z,-1)\} = \min_{h: Z \to \{\pm1 \}} \Pr_{(z,b) \sim D_0}[h(z) \neq b].
\end{equation*}

%The following statistical dimension -- defined for decision problems -- will be useful when proving Lemma~\ref{le:k_plus_1_lb}.
\begin{defn}[$k$-wise combined randomized statistical dimension]\label{def:csdr}
Let $k$ be any positive integer. Let $\calD$ be a set of distributions and $D_0$ a reference distribution over $X$. The $k$-wise combined randomized statistical dimension of the decision problem $\calB(\calD,D_0)$ is then defined as
$$ \CSDR_{\bar{\kappa}_1}^{(k)}(\calB(\calD,D_0)) \doteq \sup_{\mu \in S^{\calD}} (\bar{\kappa}_1^{(k)}(\mu,D_0))^{-1}, $$
where $S^\D$ denotes the set of all probability distributions over $\D$.

Further, for any concept class $\calC$ of Boolean functions over a domain $Z$, and for any $\epsilon > 0$, the $k$-wise combined randomized statistical dimension of $\mathcal{L}_{PAC}(\calC,\epsilon)$ is defined as
\begin{equation*}
\CSDR_{\bar{\kappa}_1}^{(k)}(\mathcal{L}_{PAC}(\calC,\epsilon)) \doteq \sup_{D_0 \in S^{Z \times \{\pm 1\}}: \err(D_0) > \epsilon} \CSDR_{\bar{\kappa}_1}^{(k)}(\calB(\calD_{\calC},D_0)),
\end{equation*}
where $\calD_{\calC} \doteq \{ P^f: P \in S^{Z}, f \in \calC\}$ with $P^f$ denoting the distribution on labeled examples $(x,f(x))$ with $x \sim P$.
\end{defn}

The next theorem lower bounds the randomized $k$-wise SQ complexity of PAC learning a concept class in terms of its $k$-wise combined randomized statistical dimension.% (introduced in Definition~\ref{def:csdr}).

\begin{theorem}[\cite{Feldman:16sqd}]\label{thm:sq_RSD}
Let $\calC$ be a concept class of Boolean functions over a domain $Z$, $k$ be a positive integer and $\epsilon, \delta > 0$. Let $d \doteq \CSDR_{\bar{\kappa}_1}^{(k)}(\mathcal{L}_{PAC}(\calC,\epsilon))$. Then, the randomized $k$-wise SQ complexity of solving $\mathcal{L}_{PAC}(\calC,\epsilon - 1/\sqrt{d})$ with access to $\STAT^{(k)}(1/\sqrt{d})$ and success probability $1-\delta$ is at least $(1-\delta) \cdot \sqrt{d} - 1$.
\end{theorem}


To lower bound the statistical dimension we will use the following ``average correlation'' parameter introduced in \cite{FeldmanGRVX:12}.
\begin{defn}[$k$-wise average correlation]\label{def:rho}
Let $k$ be any positive integer. Let $\calD$ be a set of distributions and $D_0$ a reference distribution over $X$. Assume that the support of every distribution $D \in \calD$ is a subset of the support of $D_0$. Then, for every $x \in X^k$, define $\hat{D}(x) \doteq \frac{D^k(x)}{D_0^k(x)} - 1$. Then, the $k$-wise average correlation is defined as
$$ \rho^{(k)}(\calD,D_0) \doteq \frac{1}{|\calD|^2} \cdot \displaystyle\sum\limits_{D, D' \in \calD} | D_0^k[\hat{D} \cdot \hat{D}']|. $$
\end{defn}

Lemma~\ref{lem:ub_rho} relates the average correlation to the average discrimination (from Definition~\ref{def:kappa_1_disc}).
\begin{lem}[\cite{Feldman:16sqd}]\label{lem:ub_rho}
Let $k$ be any positive integer. Let $\calD$ be a set of distributions and $D_0$ a reference distribution over $X$. Let $\mu$ be the uniform distribution over $\calD$. Then,
$$ \bar{\kappa}_1^{(k)}(\mu,D_0) \le 4 \cdot \sqrt{\rho^{(k)}(\calD,D_0)}. $$
\end{lem}



\subsubsection{Proof of Lemma~\ref{le:k_plus_1_lb}}\label{subsec:sep_lb}

Denote $X \doteq \mathbb{F}_p^{\ell} \times \{\pm 1\}$. Let $\calD$ be the set of all distributions over $X^k$ that are obtained by sampling from any given distribution over $(\mathbb{F}_p^{\ell})^k$ and labeling the $k$ samples according to any given hyperplane indicator function $f_a$. Let $D_0$ be the uniform distribution over $X^k$. We now show that $\CSDR_{\bar{\kappa}_1}(\calB(\calD,D_0)) = \Omega\big(p^{(\ell-k)/2}\big)$. By definition,
$$ \CSDR_{\bar{\kappa}_1}(\calB(\calD,D_0)) \doteq \sup_{\mu \in S^{\calD}} (\bar{\kappa}_1(\mu,D_0))^{-1}. $$
We now choose the distribution $\mu$. For $a \in \mathbb{F}_p^{\ell}$, we define $P_a$ to be the distribution over $\mathbb{F}_p^{\ell}$ that has density $\alpha = 1/(2 (p^{\ell}-p^{\ell-1}))$ on each of the $p^{\ell}-p^{\ell-1}$ points outside $\Hyp_a$, and density $\beta = 1/p^{\ell-1}-\alpha p +\alpha = 1/(2p^{\ell-1})$ on each of the $p^{\ell-1}$ points inside $\Hyp_a$. We then define $D_a$ to be the distribution obtained by sampling $k$ i.i.d.~random examples of $\Hyp_a$, the marginal of each over $\mathbb{F}_p^{\ell}$ being $P_a$. Let $\calD' \doteq \{D_a ~ | ~ a \in \mathbb{F}_p^{\ell}\}$, and let $\mu$ be the uniform distribution over $\calD'$. By Lemma~\ref{lem:ub_rho}, we have that $\bar{\kappa}_1(\mu,D_0) \le 4 \cdot \sqrt{\rho(\calD,D_0)}$, so it is enough to upper bound $\rho(\calD,D_0)$.

We first note that for $a, a' \in \mathbb{F}_p^{\ell}$, we have
\begin{align*}
D_0[\hat{D}_a \cdot \hat{D}_{a'}] &= \Ex_{(z,b) \sim D_0} [ \hat{D}_a(z,b) \cdot \hat{D}_{a'}(z,b)]\\
&= \Ex_{(z,b) \sim D_0} \bigg[ \bigg(\frac{D_a(z,b)}{D_0(z,b)}-1 \bigg) \cdot \bigg(\frac{D_{a'}(z,b)}{D_0(z,b)}-1\bigg)\bigg]\\
&= \Ex_{(z,b) \sim D_0} \bigg[ \frac{D_a(z,b) \cdot D_{a'}(z,b)}{D_0^2(z,b)} - \frac{D_a(z,b)}{D_0(z,b)} - \frac{D_{a'}(z,b)}{D_0(z,b)} +1\bigg]\\
&= \Ex_{(z,b) \sim D_0} \bigg[ \frac{D_a(z,b) \cdot D_{a'}(z,b)}{D_0^2(z,b)}\bigg] - 2 \cdot \Ex_{(z,b) \sim D_0} \bigg[\frac{D_a(z,b)}{D_0(z,b)}\bigg] +1\\
&= 2^{2k} \cdot p^{2 k \ell} \cdot \Ex_{(z,b) \sim D_0}[D_a(z,b) \cdot D_{a'}(z,b)] - 2^{k+1} \cdot p^{k \ell} \cdot \Ex_{(z,b) \sim D_0}[D_a(z,b)] + 1
\end{align*}

We now compute each of the two expectations that appear in the last equation above.

\begin{proposition}\label{prop:single_ex}
For every $a \in \mathbb{F}_p^{\ell}$,
$$ \Ex_{(z,b) \sim D_0}[D_a(z,b)] = \frac{1}{2^k} \cdot \bigg(\frac{1}{p} \cdot \beta + \bigg(1-\frac{1}{p}\bigg) \cdot \alpha\bigg)^k = \frac{1}{2^k \cdot p^{k\cdot \ell}}.$$
\end{proposition}

The proof of Proposition~\ref{prop:single_ex} appears in the appendix.

\begin{proposition}\label{prop:pair_ex}
For every $a, a' \in \mathbb{F}_p^{\ell}$,
\[
\Ex_{(z,b) \sim D_0}[D_a(z,b) \cdot D_{a'}(z,b)] = \begin{dcases*}
	& $\frac{1}{2^k} \cdot (\frac{1}{p} \cdot \beta^2 + (1-\frac{1}{p}) \cdot \alpha^2)^k$ if $\Hyp_a = \Hyp_{a'}$,\\
	& $\frac{1}{2^k}\cdot (\alpha^2 \cdot (1-\frac{2}{p}))^k$ if $\Hyp_a \cap \Hyp_{a'} = \emptyset$,\\
	& $\frac{1}{2^k} \cdot ( \frac{\beta^2}{p^2} +\alpha^2 \cdot (1-\frac{2}{p}+\frac{1}{p^2}))^k$ otherwise.
        \end{dcases*}
\]
\end{proposition}

The proof of Proposition~\ref{prop:pair_ex} appears in the appendix. Using Proposition~\ref{prop:single_ex} and Proposition~\ref{prop:pair_ex}, we now compute $D_0[\hat{D}_a \cdot \hat{D}_{a'}]$.

\begin{proposition}\label{prop:D_0}
For every $a, a' \in \mathbb{F}_p^{\ell}$,
\[
D_0[\hat{D}_a \cdot \hat{D}_{a'}] = \begin{dcases*}
	& $(p+1-\frac{1}{p-1})^k - 1$ if $\Hyp_a = \Hyp_{a'}$,\\
	& $\frac{1}{2^k} \cdot \frac{(1-\frac{2}{p})^k}{(1-\frac{1}{p})^{2k}}-1$ if $\Hyp_a \cap \Hyp_{a'} = \emptyset$,\\
	& $0$ otherwise.
        \end{dcases*}
\]
\end{proposition}

The proof of Proposition~\ref{prop:D_0} appears in the appendix. When computing $\rho(\calD,D_0)$, we will also use the following simple proposition.
\begin{proposition}\label{prop:pairs_hyp}
\begin{enumerate}
\item The number of pairs $(a,a') \in (\mathbb{F}_p^{\ell})^2$ such that $\Hyp_a = \Hyp_{a'}$ is equal to $p^{\ell}$.
\item The number of pairs $(a,a') \in (\mathbb{F}_p^{\ell})^2$ such that $\Hyp_a$ and $\Hyp_{a'}$ are distinct and parallel is equal to $p^{\ell}\cdot(p-1)$.
\item The number of pairs $(a,a') \in (\mathbb{F}_p^{\ell})^2$ such that $\Hyp_a$ and $\Hyp_{a'}$ are distinct and intersecting is equal to $p^{2\cdot \ell}-p^{\ell+1}$.
\end{enumerate}
\end{proposition}

Using Proposition~\ref{prop:D_0} and Proposition~\ref{prop:pairs_hyp}, we are now ready to compute $\rho(\calD,D_0)$ as follows
\begin{align*}
\rho(\calD,D_0) &\le \frac{1}{p^{2\cdot \ell}} \cdot \bigg[ p^{\ell} \cdot (p+1-\frac{1}{p-1})^k +p^{\ell} \cdot (p-1) + p^{2\cdot \ell} \cdot 0 \bigg]\\
&\le O\bigg(\frac{1}{p^{\ell-k}}\bigg) + \frac{1}{p^{\ell-1}}\\
&= O\bigg(\frac{1}{p^{\ell-k}}\bigg),
\end{align*}
where we used above the assumption that $k = O(p)$. We deduce that $\bar{\kappa}_1(\mu,D_0)  = O\bigg(1/p^{(\ell-k)/2}\bigg)$, and hence $\CSDR_{\bar{\kappa}_1}(\calB(\calD,D_0)) = \Omega\bigg(p^{(\ell-k)/2}\bigg)$. This lower bound on $\CSDR_{\bar{\kappa}_1}(\calB(\calD,D_0))$, along with Definition~\ref{def:csdr}, Theorem~\ref{thm:sq_RSD} and the fact that $D_0$ has Bayes error rate equal to $1/2$, imply Lemma~\ref{le:k_plus_1_lb}.



\iffalse
In order to prove \Cref{le:recovery}, we will need the following straightforward propositions.

\begin{proposition}\label{prop:frac_lb}
	For any positive real numbers $N$, $D$ and $\tau$ such that $\tau = o(N)$ and $\tau = o(D)$, we have that
	\begin{equation*}
	\frac{N-\tau}{D+\tau} \geq \frac{N}{D} \cdot (1-o(1)).
	\end{equation*}
\end{proposition}

\begin{proposition}\label{prop:frac_ub}
	For any positive real numbers $N$, $D$ and $\tau$ such that $\tau = o(D)$, we have that
	\begin{equation*}
	\frac{N+\tau}{D-\tau} \le \frac{N}{D} \cdot (1+o(1)) + o(1).
	\end{equation*}	
\end{proposition}

\begin{proposition}\label{prop:tau_i_ub}
	For every $i \in [k+1]$, we have that $\tau_i = o_c(1)$.
\end{proposition}
\begin{proposition}\label{prop:bd_tau_v_tau_i}
	If $v > \epsilon^{k+1}/2$, then for every $i \in [k+1]$, we have that
	\begin{equation*}
	\tau = o_c \bigg( (v \cdot \tau_i - \tau ) \cdot (1-\tau_i/4)\bigg).
	\end{equation*}
\end{proposition}

The proof of \Cref{prop:bd_tau_v_tau_i} follows from the definitions of $\tau$ and $\tau_i$ in \Cref{alg:k_wise_SQ}.
\fi


\section{Reduction for flat distributions}\label{sec:pf_flat}
\newcommand{\dci}{{\kappa_1}}
To prove Theorem~\ref{thm:flat} we use the characterization of the SQ complexity of the problem of estimating $D^k[\phi]$ for $D\in \D$ using a notion of statistical dimension from \cite{Feldman:16sqd}. Specifically, we use the characterization of the complexity of solving this problem using unary SQs and also the generalization of this characterization that characterizes the complexity of solving a problem using $k$-wise SQs. The latter is equal to 1 (since a single $k$-wise SQ suffices to estimate $D^k[\phi]$). Hence the $k$-wise statistical dimension is also equal to 1. We then upper bound the unary statistical dimension by the $k$-wise statistical dimension. The characterization then implies that an upper bound on the unary statistical dimension gives an upper bound on the SQ complexity of estimating $D^k[\phi]$.

We also give a slightly different way to define flatness that makes it easier to extend our results to other notions of divergence.
\begin{defn}
Let $\calD$ be a set of distributions over $X$. Define
$$R_\infty(\D) \doteq \inf_{\bar D \in S^X} \sup_{D\in \D} \Div_\infty(D\|\bar D), $$
where $S^X$ denotes the set of all probability distributions over $X$ and $$\Div_\infty(D\|\bar D) \doteq \sup_{y\in X} \ln \frac{\Pr_{x\sim D}[x=y]}{\Pr_{x\sim \bar D}[x=y]}$$ denotes the max-divergence. We say that $\calD$ is $\gamma$-flat if $R_\infty(\D) \leq \ln \gamma$.
\end{defn}

For simplicity, we will start by relating the $k$-wise SQ complexity to unary SQ complexity for decision problems. The statistical dimension for this type of problems is substantially simpler than for the general problems but is sufficient to demonstrate the reduction. We then build on the results for decision problems to obtain the proof of Theorem~\ref{thm:flat}.

\subsection{Decision problems}
The $k$-wise generalization of the statistical dimension for decision problems from \cite{Feldman:16sqd} is defined as follows.
\begin{defn}
Let $k$ be any positive integer. Consider a set of distributions $\calD$ and a reference distribution $D_0$ over $X$. Let $\mu$ be a probability measure over $\calD$ and let $\tau > 0$. The $k$-wise maximum covered $\mu$-fraction is defined as
$$ \kappa_1\text{-}\fract^{(k)}(\mu,D_0,\tau) \doteq \sup_{\phi: X^k \to [-1,+1]} \bigg\{ \Pr_{D \sim \mu}[|D^k[\phi]-D_0^k[\phi]| > \tau] \bigg\}. $$
\end{defn}
\begin{defn}[$k$-wise randomized statistical dimension of decision problems]\label{def:rdm_sd}
Let $k$ be any positive integer. For any set of distributions $\cald$, a reference distribution $D_0$ over $X$ and $\tau > 0$, we define
$$ \RSD_{\kappa_1}^{(k)}(\calB(\calD,D_0), \tau) \doteq \sup_{\mu \in S^{\calD}} ( \kappa_1\text{-}\fract^{(k)}(\mu,D_0,\tau))^{-1}, $$
where $S^\D$ denotes the set of all probability distributions over $\D$.
\end{defn}

As shown in \cite{Feldman:16sqd}, $\RSD$ tightly characterizes the randomized statistical query complexity of solving the problem using $k$-wise queries. As observed before, the $k$-wise versions below are implied by the unary version in \cite{Feldman:16sqd} simply by defining the domain to be $X' \doteq X^k$ and the set of input distributions to be $\D' \doteq \{D^k \ |\  D \in \D\}$.

\begin{theorem}[\cite{Feldman:16sqd}]\label{thm:random-algorithm2queries}
Let $\calB(\D,D_0)$ be a decision problem, $\tau > 0, \delta \in (0,1/2)$, $k \in \mathbb{N}$ and $d=\RSD^{(k)}_\dci(\calB(\D,D_0),\tau)$. Then there exists a randomized algorithm that solves $\calB(\D,D_0)$ with success probability $\geq 1-\delta$ using $d \cdot \ln(1/\delta)$ queries to $\STAT^{(k)}_D(\tau/2)$. Conversely, any algorithm that solves $\calB(\D,D_0)$ with success probability $\geq 1-\delta$ requires at least $d \cdot (1-2\delta)$ queries to $\STAT^{(k)}_D(\tau)$.
\end{theorem}

We will also need the following dual formulation of the statistical dimension given in Theorem~\ref{def:rdm_sd}.
\begin{lem}[\cite{Feldman:16sqd}]\label{fa:rcvr}
Let $k$ be any positive integer. For any set of distributions $\cald$, a reference distribution $D_0$ over $X$ and $\tau > 0$,  the statistical dimension $\RSD_{\kappa_1}^{(k)}(\calB(\calD,D_0), \tau)$ is equal to the smallest $d$ for which there exists a distribution $\calP$ over functions from $X^k$ to $[-1,+1]$ such that for every $D \in \calD$,
$$ \Pr_{\phi \sim \calP}[|D^k[\phi]-D_0^k[\phi]| > \tau] \geq \frac{1}{d}.$$
\end{lem}

We can now state the relationship between $\RSD_{\kappa_1}^{(k)}$ and $\RSD_{\kappa_1}^{(1)}$ for any $\gamma$-flat $\D$.
\begin{lem}\label{lem:k-wise-flat-decision}
Let $\gamma \geq 1$, $\tau > 0$ and $k \in \mathbb{N}$. Let $X$ be a domain, $\calD$ be a $\gamma$-flat class of distributions over $X$ and $D_0$ be any distribution over $X$. Then
$$\RSD_{\kappa_1}^{(1)}(\calB(\calD,D_0),\tau/(2k))  \leq \frac{4k \cdot \gamma^{k-1}}{\tau} \cdot \RSD_{\kappa_1}^{(k)}(\calB(\calD,D_0),\tau).$$
\end{lem}
\begin{proof}
Let $d \doteq \RSD_{\kappa_1}^{(k)}(\calB(\calD,D_0),\tau)$. Fact~\ref{fa:rcvr} implies the existence of a distribution $\calP$ over $k$-wise functions such that for every $D \in \calD$,
$$\Pr_{\phi \sim \calP}[|D^k[\phi]-D_0^k[\phi]| > \tau] \geq \frac{1}{d}.$$
We now fix $D$ and let $\phi$ be such that $|D^k[\phi]-D_0^k[\phi]| > \tau$. 

By the standard hybrid argument,  
\begin{equation}\label{eq:good_j}
\E_{j \sim [k]} \left[\left|D^{j} D_0^{k-j}[\phi]-D^{j-1}D_0^{k-j+1}[\phi]\right|\right] > \frac{\tau}{k} ,
\end{equation}
where $j \sim [k]$ denotes a random and uniform choice of $j$ from $[k]$.
This implies that
\begin{equation*}\label{eq:pull_out}
\E_{j \sim [k]} \Ex_{x_{< j} \sim D^{j-1}} \Ex_{x_{> j} \sim D_0^{k-j}} \bigg[\bigg|D[\phi(x_{<j}, \cdot, x_{> j})] - D_0[\phi(x_{<j}, \cdot, x_{> j})]\bigg|\bigg] > \frac{\tau}{k}.
\end{equation*}
By an averaging argument (and using the fact that $\phi$ takes values between $-1$ and $+1$), we get that with probability at least $\tau/(4 \cdot k)$ over the choice of $j\sim [k]$, $x_{< j} \sim D^{j-1}$ and $x_{> j} \sim D_0^{k-j}$, we have that
\begin{equation*}\label{eq:after_whp_switch}
\bigg|D[\phi(x_{<j}, \cdot, x_{> j})] - D_0[\phi(x_{<j}, \cdot, x_{> j})]\bigg| > \frac{\tau}{2 \cdot k}.
\end{equation*}

Since $\calD$ is a $\gamma$-flat class of distributions, there exists a (fixed) distribution $\bar{D}$ over $X$ such that for every measurable event $E \subset X$, $\Pr_{x\sim D}[x \in E] \le \gamma \cdot \Pr_{x\sim \bar{D}}[x \in E]$. Thus, we can replace the unknown input distribution $D$ by the distribution $\bar{D}$ and get that, with probability at least $\tau/(4 \cdot k \cdot \gamma^{k-1})$ over the choice of $j\sim [k]$, $x_{< j} \sim \bar{D}^{j-1}$ and $x_{> j} \sim D_0^{k-j}$, we have
\begin{equation}\label{eq:after_flat}
\bigg|D[\phi(x_{<j}, \cdot, x_{> j})] - D_0[\phi(x_{<j}, \cdot, x_{> j})]\bigg| > \frac{\tau}{2 \cdot k}.
\end{equation}
We now consider the following distribution $\mathcal{P'}$ over unary SQ functions (i.e., over $[-1,+1]^X$): Independently sample $\phi$ from $\calP$, $j$ uniformly from $[k]$, $x_{< j} \sim \bar{D}^{j-1}$ and $x_{> j} \sim D_0^{k-j}$, and output the (unary) function $\phi'(x) = \phi(x_{<j}, x, x_{> j})$. Then, for every $D\in \D$, we have that with probability at least $\frac{1}{d }\cdot \frac{\tau}{4k} \cdot \frac{1}{\gamma^{k-1}}$ over the choice of $\phi'$ from $\calP'$, we have that $|D[\phi'] - D_0[\phi']| > \tau/(2 \cdot k)$. Thus, by Fact~\ref{fa:rcvr} $$\RSD_{\kappa_1}^{(1)}\left(\calB(\calD,D_0),\frac{\tau}{2 \cdot k}\right) \le \frac{4 d \cdot \gamma^{k-1} \cdot k}{\tau}.$$
\end{proof}

Lemma \ref{lem:k-wise-flat-decision} together with the characterization in Theorem \ref{thm:random-algorithm2queries} imply the following upper bound on the SQ complexity of a decision problem in terms of its $k$-wise SQ complexity.
\begin{theorem}\label{thm:flat-decision-reduction}
Let $\gamma \geq 1$, $\tau > 0$ and $k \in \mathbb{N}$. Let $X$ be a domain, $\calD$ be a $\gamma$-flat class of distributions over $X$ and $D_0$ be any distribution over $X$. If there exists an algorithm that, with probability at least $2/3$ solves $\calB(\D,D_0)$ using $t$ queries to $\STAT^{(k)}_D(\tau)$, then for every $\delta>0$, there exists an algorithm that, with probability at least $1-\delta$ solves $\calB(\D,D_0)$ using $t \cdot 12k \cdot \gamma^{k-1} \cdot \ln(1/\delta) /\tau$ queries to $\STAT^{(1)}_D(\tau/(4k))$.
\end{theorem}

\subsection{General problems}
We now define the general class of problems over sets of distributions and a notion of statistical dimension for these types of problems.
\begin{defn}[Search problems]
A search problem $\calZ$ over a class $\calD$ of distributions and a set $\calF$ of solutions is a mapping $\calZ: \calD \to 2^{\calF} \setminus \{\emptyset\}$, where $2^{\calF}$ denotes the set of all subsets of $\calF$. Specifically, for every distribution $D \in \calD$, $\calZ(D) \subseteq \calF$ is the (non-empty) set of valid solutions for $D$. For a solution $f \in \calF$, we denote by $\calZ_f$ the set of all distributions for which $f$ is a valid solution.
\end{defn}

\begin{defn}[Statistical dimension for search problems \cite{Feldman:16sqd}]\label{def:search_SD}
For $\tau > 0$, $k \in \mathbb{N}$, a domain $X$ and a search problem $\calZ$ over a class of distributions $\calD$ over $X$ and a set of solutions $\calF$, we define the \emph{$k$-wise statistical dimension} with $\kappa_1$-discrimination $\tau$ of $\calZ$ as
\begin{equation*}
\SD^{(k)}_{\kappa_1}(\calZ,\tau) \doteq \sup_{D_0 \in S^X} \inf_{f \in \calF} \RSD^{(k)}_{\kappa_1}(\calB(\calD \setminus \calZ_f, D_0), \tau),
\end{equation*}
where $S^X$ denotes the set of all probability distributions over $X$.
\end{defn}

Lemma~\ref{lem:search_lb} lower-bounds the deterministic $k$-wise SQ complexity of a search problem in terms of its ($k$-wise) statistical dimension.
\begin{theorem}[\cite{Feldman:16sqd}]\label{lem:search_lb}
Let $\calZ$ be a search problem, $\tau > 0$ and $k \in \mathbb{N}$. The deterministic $k$-wise SQ complexity of solving $\calZ$ with access to $\STAT^{(k)}(\tau)$ is at least $\SD^{(k)}_{\kappa_1}(\calZ,\tau)$.
\end{theorem}

The following theorem from \cite{Feldman:16sqd} gives an upper bound on the SQ complexity of a search problem in terms of its statistical dimension. It relies on the multiplicative weights update method to reconstruct the unknown distribution sufficiently well for solving the problem. The use of this algorithm introduces dependence on $\KL$-radius of $\D$. Namely, we define
$$\KLR(\D) \doteq \inf_{\bar D \in S^X } \sup_{D\in \D} \KL(D\|\bar D), $$
where $\KL(\cdot \| \cdot)$ denotes the KL-divergence.
\begin{theorem}
[\cite{Feldman:16sqd}]\label{lem:search_ub}
Let $\calZ$ be a search problem, $\tau, \delta > 0$ and $k \in \mathbb{N}$. There is a randomized $k$-wise SQ algorithm that solves $\calZ$ with success probability $1-\delta$ using
$$ O\bigg(\SD^{(k)}_{\kappa_1}(\calZ,\tau) \cdot \frac{\KLR(\calD)}{\tau^2} \cdot \log\bigg( \frac{\KLR(\calD)}{\tau \cdot \delta}\bigg) \bigg)$$
queries to $\STAT^{(k)}(\tau/3)$.
\end{theorem}

Note that $\KL$-divergence between two distributions is upper-bounded (and is usually much smaller) than the max-divergence we used in the definition of $\gamma$-flatness. Specifically, if $\D$ is $\gamma$-flat  then $\KLR(\D) \leq \ln \gamma$.
We are now ready to prove Theorem~\ref{thm:flat} which we restate here for convenience.
\begin{reptheorem}{thm:flat}[restated]
Let $\gamma \geq 1$, $\tau > 0$ and $k$ be any positive integer. Let $X$ be a domain and $\calD$ be a $\gamma$-flat class of distributions over $X$. There exists a randomized algorithm that given any $\delta > 0$ and a $k$-ary function $\phi: X^k \to [-1,1]$, estimates $D^k[\phi]$ within $\tau$  for every (unknown) $D \in \calD$ with success probability at least $1-\delta$ using $$\tilde{O}\bigg( \frac{\gamma^{k-1} \cdot k^3}{\tau^3} \cdot \log (1/\delta)\bigg)$$
queries to $\STAT_D^{(1)}(\tau/(6 \cdot k))$.\end{reptheorem}
\begin{proof}
We first observe that the task of estimating $D^k[\phi]$ up to additive $\tau$ can be viewed as a search problem $\calZ$ over the set $\calD$ of distributions and over the class $\calF$ of solutions that corresponds to the interval $[-1,+1]$. Next, observe that one can easily estimate $D^k[\phi]$ up to additive $\tau$ using a single query to $\STAT^{(k)}_D(\tau)$. Lemma~\ref{lem:search_lb} implies that $\SD^{(k)}_{\kappa_1}(\calZ,\tau) = 1$. By Definition~\ref{def:search_SD}, for every $D_1 \in S^X$, there exists $f \in \calF$, such that $\RSD^{(k)}_{\kappa_1}(\calB(\calD \setminus \calZ_f, D_1), \tau) = 1$. By Lemma \ref{lem:k-wise-flat-decision},
 $$\RSD_{\kappa_1}^{(1)}\left(\calB(\calD \setminus \calZ_f, D_1),\frac{\tau}{2 \cdot k}\right) \le \frac{4 \cdot \gamma^{k-1} \cdot k}{\tau}.$$

Thus, Fact~\ref{fa:rcvr} and Definition~\ref{def:search_SD} imply that
$$ \SD_{\kappa_1}^{(1)}(\calZ,\frac{\tau}{2 \cdot k}) \le \frac{4 \cdot \gamma^{k-1} \cdot k}{\tau}.$$
Applying Lemma~\ref{lem:search_ub}, we conclude that there exists a randomized unary SQ algorithm that solves $\calZ$ with probability at least $1-\delta$ using at most $$O\bigg(\gamma^{k-1} \cdot k^3 \cdot \frac{\KLR(\mathcal{D})}{\tau^3} \cdot \log\bigg( \frac{k \cdot \KLR(\calD)}{\tau \cdot \delta}\bigg)\bigg)$$
queries to $\STAT^{(1)}(\tau/(6 \cdot k))$. This -- along with the fact that $\KLR(\calD) \le \ln(\gamma)$ whenever $\calD$ is a $\gamma$-flat set of distributions -- concludes the proof of Theorem~\ref{thm:flat}.
\end{proof}

\paragraph{Other divergences:} While the max-divergence that we used for measuring flatness suffices for the applications we give in this paper (and is relatively simple), it might be too conservative in other problems. For example, such divergence is infinite even for two Gaussian distributions with the same standard deviation but different means. A simple way to obtain a more robust version of our reduction is to use approximate max-divergence. For $\delta \in [0,1)$ it is defined as: $$\Div_\infty^\delta(D\|\bar D) \doteq \ln \sup_{E \subseteq X} \frac{\Pr_{x\sim D}[x\in E]-\delta}{\Pr_{x\sim \bar D}[x\in E]} .$$ Note that $\Div_\infty^0(D\|\bar D) = \Div_\infty(D\|\bar D)$. Similarly, we can define a radius of $\D$ in this divergence $$R_\infty^\delta(\D) \doteq \inf_{\bar D \in S^X} \sup_{D\in \D} \Div_\infty^\delta(D\|\bar D) .$$

Now, it is easy to see that, if $\Div_\infty^\delta(D\|\bar D) \leq r$ then $\Div_\infty^{k\delta}(D^k\|\bar D^k) \leq kr$. This means that if in the proof of Lemma \ref{lem:k-wise-flat-decision} we use the condition
$R_\infty^{\tau/(8k^2)}(\D) \leq \ln \gamma$ instead of $\gamma$-flatness then we will obtain that the event in Equation \eqref{eq:after_flat} holds with probability at least $$ \left( \frac{\tau}{4k} - (k-1) \cdot  \frac{\tau}{8k^2}\right) / \gamma^{k-1} \geq \frac{\tau}{\gamma^{k-1} \cdot 8k}$$ over the same random choices.

This implies the following generalization of Theorem \ref{thm:flat}.
\begin{theorem}\label{thm:flat-approx}
Let $\tau > 0$ and $k$ be any positive integer. Let $\calD$ be a class of distributions over a domain $X$ and $\gamma = \exp(R_\infty^{\tau/(8k^2)}(\D))$. There exists a randomized algorithm that given any $\delta > 0$ and a $k$-ary function $\phi: X^k \to [-1,1]$, estimates $D^k[\phi]$ within $\tau$  for every (unknown) $D \in \calD$ with success probability at least $1-\delta$ using $$\tilde{O}\bigg( \frac{\gamma^{k-1} \cdot k^3 \cdot \KLR(\calD)}{\tau^3} \cdot \log (1/\delta)\bigg)$$
queries to $\STAT_D^{(1)}(\tau/(6 \cdot k))$.\end{theorem}

An alternative approach is to use Renyi divergence of order $\alpha > 1$ defined as follows:
 $$\Div_\alpha(D\|\bar D) \doteq \frac{1}{1-\alpha} \cdot \ln \left (\E_{y \sim D} \left[\left(\frac{\Pr_{x\sim D}[x=y]}{\Pr_{x\sim \bar D}[x=y]}\right)^{\alpha-1}\right] \right).$$ The corresponding radius is defined as $$R_\alpha(\D) \doteq \inf_{\bar D \in S^X} \sup_{D\in \D} \Div_\alpha(D\|\bar D) .$$

To use it in our application we need the standard property of the Renyi divergence for product distributions $\Div_\alpha(D^k\|\bar D^k) = k \cdot \Div_\alpha(D\|\bar D)$ and also the following simple lemma from \cite[Lemma 1]{MansourMR09}:
\begin{lem}
For $\alpha > 1$, any two distributions $D,\bar D$ over $X$ and an event $E\subseteq X$:
$$\Pr_{x\sim D}[x\in E] \leq \left(\exp(\Div_\alpha(D\|\bar D) ) \cdot \Pr_{x\sim \bar D}[x\in E]\right)^{\frac{\alpha-1}{\alpha}}. $$
\end{lem}
We will need the inverted version of this lemma:
$$\Pr_{x\sim \bar D}[x\in E] \geq \frac{\left(\Pr_{x\sim D}[x\in E]\right)^{\frac{\alpha}{\alpha-1}}}{\exp(\Div_\alpha(D\|\bar D) )}. $$
Applying this in the proof of Lemma \ref{lem:k-wise-flat-decision} for $\gamma = \exp(R_\alpha(\D))$, we obtain that the event in Equation \eqref{eq:after_flat} holds with probability at least $$ \left(\frac{\tau}{4k}\right)^{\frac{\alpha}{\alpha-1}}/ \gamma^{k-1}. $$
This gives the following generalization of Theorem \ref{thm:flat}.
\begin{theorem}\label{thm:flat-approx}
Let $\tau > 0,\alpha >1$ and $k$ be any positive integer. Let $\calD$ be a class of distributions over a domain $X$ and $\gamma = \exp(R_\alpha(\D))$. There exists a randomized algorithm that given any $\delta > 0$ and a $k$-ary function $\phi: X^k \to [-1,1]$, estimates $D^k[\phi]$ within $\tau$  for every (unknown) $D \in \calD$ with success probability at least $1-\delta$ using $$\tilde{O}\bigg( \gamma^{k-1} \cdot \left( \frac{k}{\tau}\right)^{2+\frac{\alpha}{\alpha-1}} \cdot \log (1/\delta)\bigg)$$
queries to $\STAT_D^{(1)}(\tau/(6 \cdot k))$.\end{theorem}



\subsection{Applications to solving CSPs and learning DNF}
\label{sec:lower-bounds}
We now give some examples of the application of our reduction to obtain lower bounds against $k$-wise SQ algorithms. Our applications for stochastic constraint satisfaction problems (CSPs) and DNF learning. We start with the definition of a stochastic CSP with a {\em planted solution} which is a pseudo-random generator based on Goldreich's proposed one-way function \cite{goldreich2000candidate}.
\begin{definition}
Let $t \in \mathbb{N}$ and $P: \{\pm 1\}^t \to \{\pm 1\}$ be a fixed predicate. We are given access to samples from a distribution $P_{\sigma}$, corresponding to a (``planted'') assignment $\sigma \in \{\pm 1 \}^n$. A sample from this distribution is a uniform-random  $t$-tuple $(i_1, \dots, i_t)$ of distinct variable indices along with the value $P(\sigma_{i_1}, \dots, \sigma_{i_t})$. The goal is to recover the assignment $\sigma$ when given $m$ independent samples from $P_{\sigma}$. A (potentially) easier problem is to distinguish any such planted distribution from the distribution $U_t$ in which the value is an independent uniform-random coin flip (instead of $P(\sigma_{i_1}, \dots, \sigma_{i_t})$).
\end{definition}
We say that a predicate $P: \{\pm 1\}^t \to \{\pm 1\}$ has complexity $r$ if $r$ is the degree of the lowest-degree non-zero Fourier coefficient of $P$. It can be as large as $t$ (for the parity function).
A lower bound on the (unary) SQ complexity of solving such CSPs was shown by \cite{FeldmanPV:13} (their result is for the stronger $\VSTAT$ oracle but here we state the version for the $\STAT$ oracle).
\begin{theorem}[\cite{FeldmanPV:13}]\label{thm:FPV_csp}
Let $t, q \in \mathbb{N}$ and $P: \{\pm 1\}^t \to \{\pm 1\}$ be a fixed predicate of complexity $r$. Then for any $q >0$, any  algorithm that, given access to a distribution $D \in \{P_\sigma\ |\  \sigma  \in \{\pm 1 \}^n\} \cup \{U_t\}$  decides correctly whether $D = P_\sigma$ or $D=U_t$ with probability at least $2/3$ needs $q/2^{O(t)}$ queries to $\STAT^{(1)}_D\left(\left(\frac{\log q}{n}\right)^{r/2}\right)$.
\end{theorem}

The set of input distributions in this problem is $2$-flat relative to $U_t$ and it is one-to-many decision problem. Hence Theorem \ref{thm:flat-decision-reduction} implies\footnote{We can also get essentially the same result by applying the simulation of a $k$-wise SQ using unary SQs from Theorem \ref{thm:flat}.} the following lower bound for $k$-wise SQ algorithms.
\begin{theorem}\label{thm:csp-k-wise}
Let $t \in \mathbb{N}$ and $P: \{\pm 1\}^t \to \{\pm 1\}$ be a fixed predicate of complexity $r$. Then for any $\alpha >0$, any  algorithm that, given access to a distribution $D \in \{P_\sigma\ |\  \sigma  \in \{\pm 1 \}^n\} \cup \{U_t\}$  decides correctly whether $D = P_\sigma$ or $D=U_t$ with probability at least $2/3$ needs $2^{n^{1-\alpha} -O(t)}$ queries to $\STAT^{(n^{1-\alpha})}_D\left((2/n^{\alpha})^{r/2}  \cdot n^{1-\alpha}/4\right)$.
\end{theorem}
\begin{proof}
Let $\A$ be a $k$-wise SQ algorithm using $q'$ queries to $\STAT^{(n^{1-\alpha})}_D\left((2/n^{\alpha})^{r/2} \cdot n^{1-\alpha}/6\right)$ which solves the problem with success probability $2/3$.
We let $k=n^{1-\alpha}$ and apply Theorem \ref{thm:flat-decision-reduction} to obtain an algorithm that uses unary SQs and solves the problem with success probability $2/3$. This algorithm uses $q_0 = q' \cdot 2^{n^{1-\alpha}} \cdot n^{O(r)}$ queries to $\STAT^{(1)}_D\left((2/n^{\alpha})^{r/2}\right)$. Now choosing $q = 2^{2n^{1-\alpha}}$ we get that $\left(\frac{\log q}{n}\right)^{r/2} \leq (2/n^{\alpha})^{r/2}$. This means that $q_0 \geq q/2^{O(t)} = 2^{2n^{1-\alpha}- O(t)}$.
Hence $q' = 2^{2n^{1-\alpha}-O(t) - n^{1-\alpha}- O(r)} = 2^{n^{1-\alpha} -O(t)}$.
\end{proof}
Similar lower bounds can be obtained for other problems considered in \cite{FeldmanPV:13}, namely, planted satisfiability and $t$-SAT refutation.

To obtain a lower bound for learning DNF formulas we can use a simple reduction from the Goldreich's PRG defined above to learning DNF formulas of polynomial size. It is based on ideas implicit in the reduction from $t$-SAT refutation to DNF learning from \cite{DanielyS16}.
\begin{lemma}\label{lem:reduce-dnf}
$P: \{\pm 1\}^t \to \{\pm 1\}$ be a fixed predicate. There exists a mapping $M$ from $t$-tuples of indices in $[n]$ to $\zo^{tn}$ such that for every $\sigma \in \{\pm 1 \}^n$ there exists a DNF formula $f_\sigma$ of size $2^t$ satisfying $P(\sigma_{i_1}, \dots, \sigma_{i_t}) = f_\sigma(M(i_1, \dots, i_t))$.
\end{lemma}
\begin{proof}
The mapping $M$ maps $(i_1, \dots, i_t)$ to the concatenation of the indicator vectors of each of the indices. Namely, for $j \in [t]$ and $\ell \in [n]$, $M(i_1, \dots, i_t)_{j,\ell} = 1$ if and only if $i_j = \ell$, where we use the double index $j,\ell$ to refer to element $n (j-1) + \ell$ of the vector. Let $v_{j,\ell}$ denote the variable with the index $j,\ell$. Let $\sigma$ be any assignment and we denote by $z_j^\sigma$ the $j$-th variable of our predicate $P$ when the assignment is equal to $\sigma$. We first observe that $z_j^\sigma \equiv \bigwedge_{\ell \in [n], \sigma_\ell = 0} \bar{v}_{j,\ell}$. This is true since, by definition, the value of the $j$-th variable of our predicate is $\sigma_{i_j}$. This value is $1$ if and only if $i_j \not\in \{\ell \in [n]\ | \ \sigma_\ell = 0\}$. This is equivalent to $v_{j,\ell}$ being equal to $0$ for all $\ell \in [n]$ such that  $\sigma_\ell = 0$. Analogously, $\bar{z}^\sigma_j \equiv \bigwedge_{\ell \in [n], \sigma_\ell = 1} \bar{v}_{j,\ell}$. This implies that any conjunction of variables $z_1^\sigma,\bar{z}^\sigma_1,\ldots,z^\sigma_t,\bar{z}^\sigma_t$ can be expressed as a conjunction over variables $\bar{v}_{j,\ell}$. Any predicate $P$ can be expressed as a disjunction of at most $2^t$ conjunctions and hence there exists a DNF formula $f_\sigma$ of size at most $2^t$ whose value on $M(i_1, \dots, i_t)$ is equal to $P(\sigma_{i_1}, \dots, \sigma_{i_t})$
\end{proof}

This reduction implies that by converting a sample $((i_1, \dots, i_t),b)$ to a sample $(M(i_1, \dots, i_t),b)$ we can transform the Goldreich's PRG problem into a problem in which our goal is to distinguish examples of some DNF formula $f_\sigma$ from randomly labeled examples. Naturally, an algorithm that can learn DNF formulas can output a hypothesis which predicts the label (with some non-trivial accuracy), whereas such hypothesis cannot exist for predicting random labels. Hence known SQ lower bounds on planted CSPs \cite{FeldmanPV:13} immediately imply lower bounds for learning DNF. Further, by applying Lemma \ref{lem:reduce-dnf} together with Thm.~\ref{thm:csp-k-wise} for $t=r=\log n$ we obtain the first lower bounds for learning DNF against $n^{1-\alpha}$-wise SQ algorithms.
\begin{theorem}\label{thm:dnf-k-wise}
For any constant (independent of $n$) $\alpha >0$, there exists a constant $\beta>0$ such that
 any  algorithm that PAC learns DNF formulas of size $n$ with error $<1/2 - n^{- \beta \log n}$ and success probability at least $2/3$ needs at least $2^{n^{1-\alpha}}$ queries to $\STAT^{(n^{1-\alpha})}_D(n^{- \beta \log n})$.
\end{theorem}
We remark that this is a lower bound for PAC learning polynomial size DNF formulas with respect to some fixed (albeit non-uniform) distribution over $\zo^n$. The approach for relating $k$-wise SQ complexity to unary SQ complexity given in \cite{blum2003noise} applies to this setting. Yet, in their proof the tolerance needed for the unary SQ algorithm is $\tau/2^k$ and therefore it would not give a non-trivial lower bounds beyond $k=O(\log n)$.

%Our reduction produces a $\zo^{2tn}$
 

\section{Reduction for low-communication queries}\label{sec:sq_cc}
In this section, we prove  Theorem~\ref{thm:sq_and_cc} using a recent result of Steinhardt, Valiant and Wager \cite{SteinhardtVW16}.
Their result can be seen giving a SQ algorithm that simulates a communication protocol between $n$ parties. Each party is holding a sample drawn i.i.d.~from distribution $D$ and broadcasts at most $b$ bits about its sample (to all the other parties). The bits can be sent over multiple rounds. This is essentially the standard model of multi-party communication complexity (\eg \cite{KN97}) but with the goal of solving some problem about the unknown distribution $D$ rather than computing a specific function of the inputs. Alternatively, one can also see this model as a single algorithm that extracts at most $b$-bits of information about each random sample from $D$ and is allowed to extract the bits in an arbitrary order (generalizing the $b$-bit sampling model that we discuss in Section \ref{subsec:RFA} and in which $b$-bits are extracted from each sample at once). We refer to this model simply as algorithms that extract at most $b$ bits per sample.

\begin{theorem}[\cite{SteinhardtVW16}]\label{thm:SVW}
Let $\calA$ be an algorithm that uses $n$ samples drawn i.i.d.~from a distribution $D$ and extracts at most $b$ bits per sample. Then, for every $\beta > 0$, there is an algorithm $\calB$ that makes at most $2 \cdot b \cdot n$ queries to $\STAT^{(1)}_D(\beta/(2^{b+1} \cdot k))$ and the output distributions of $\calA$ and $\calB$ are within total variation distance $\beta$.
\end{theorem}

We will use this simulation to estimate the expectation of $k$-wise functions that have low communication complexity.
Specifically, we recall the following standard model of public-coin randomized $k$-party communication complexity.
\begin{defn}
For a function $\phi:X^k \to \{\pm 1\}$ we say that $\phi$ has a $k$-party public-coin randomized communication complexity of at most $b$ bits per party with success probability $1-\delta$ if there exist a protocol satisfying the following conditions. Each of the parties is given $x_i \in X$ and access to shared random bits. In each round one of the parties can compute one or more bits using its input, random bits and all the previous communication and then broadcast it to all the other parties. In the last round one of the parties computes a bit that is the output of the protocol. Each of the parties communicates at most $b$ bits in total. For every $x_1,\ldots,x_k\in X$, with probability at least $1-\delta$ over the choice of the random bits the output of the protocol is equal to $\phi(x_1,\ldots,x_k)$.
\end{defn}

We are now ready to prove Theorem~\ref{thm:sq_and_cc} which we restate here for convenience.
\begin{reptheorem}{thm:sq_and_cc}[restated]
Let $\phi:X^k \to \{\pm 1\}$ be a function, and assume that $\phi$ has $k$-party public-coin randomized communication complexity of $b$ bits per party with success probability $2/3$. Then, there exists a randomized algorithm that, with probability at least $1-\delta$, estimates $\Ex_{x \sim D^k}[\phi(x)]$ within $\tau$ using $O(b \cdot k \cdot \log(1/\delta)/\tau^2)$ queries to $\STAT^{(1)}_D(\tau')$ for some $\tau' = \tau^{O(b)}/k$.
\end{reptheorem}
\begin{proof}
We first amplify the success probability of the protocol for computing $\phi$ to $\delta'\doteq \tau/8$ using the majority vote of $O(\log(1/\delta'))$ repetitions. By Yao's minimax theorem \cite{Yao:1977} there exists a deterministic protocol $\Pi'$ that succeeds with probability at least $1-\delta'$ for $(x_1,\dots,x_k) \sim D^k$. Applying Theorem~\ref{thm:SVW}, we obtain a unary SQ algorithm $\A$ whose output is within total variation distance at most $\beta \doteq \tau/8$ from $\Pi'(x_1, \dots, x_k)$ (and we can assume that the output of $\A$ is in $\{\pm 1\}$). Therefore:
$$|\E[\A] - D^k[\phi]| \leq |\E[\A] - \E_{D^k} [\Pi'(x_1, \dots, x_k)]| + |\E_{D^k} [\Pi'(x_1, \dots, x_k)] - D^k [\phi]| \leq  \frac{2\tau}{8} + \frac{2\tau}{8} = \frac{\tau}{2}.$$
Repeating $\A$ $O(\log(1/\delta)/\tau^2)$ times and taking the mean, we get an estimate of $D^k[\phi]$ within $\tau$ with probability at least $1-\delta$. This algorithm uses $O(b \cdot k \cdot \log(1/\delta)/\tau^2)$ queries to $\STAT^{(1)}_D(\tau')$ for $\tau' = \frac{\tau}{8}/(2^{O(\log(8/\tau) \cdot b)} \cdot k) = \tau^{O(b)}/k$.
\end{proof}

The collision probability for a distribution $D$ is defined as $\Pr_{(x_1, x_2) \sim D^2}[x_1 = x_2]$. This corresponds to $\phi(x_1,x_2)$ being the Equality function which, as is well-known, has randomized $2$-party communication complexity of $O(1)$ bits per party with success probability $2/3$ (see, e.g., \cite{KN97}). Applying Theorem~\ref{thm:sq_and_cc} with $k=2$ we get the following corollary.
\begin{corollary}
For any $\tau, \delta >0$, there is a SQ algorithm that estimates the collision probability of an unknown distribution $D$ within $\tau$ with success probability $1-\delta$ using $O(\log(1/\delta)/\tau^2)$ queries to $\STAT^{(1)}_{D}(\tau^{O(1)})$.
\end{corollary} 

\section{Applications}\label{sec:applications}

In this section, we present applications of our results to two SDPs: Max-Cut and matrix completion, both of which are important problems in the learning domain and have been studied extensively. Interest has grown to develop efficient solvers for these SDPs~\citep{arora2007combinatorial, pmlr-v65-mei17a, hardt2013understanding, bandeira2016low}.
%General SDP solvers such as interior point or ellipsoid method can lead to slow algorithms for these problems. Instead, over the years, specialized efficient algorithms have been developed for these problems \citep{arora2007combinatorial, hardt2013understanding}. 

This work differs from previous efforts in at least two ways. First, we aim to demonstrate that Burer--Monteiro-style approaches, which are often used in practice, can indeed lead to provably efficient algorithms for general SDPs. We believe that building upon this work, it should be possible to improve the time-complexity guarantees of such factorization-based algorithms. Second, we note that several problems formulated as SDPs in fact necessitate low-rank solutions, for example because of memory concerns (as is the case in matrix completion),  and factorization approaches provide a natural means to control rank. % For such problem, existing methods do not provide low rank solutions while our method is guaranteed to give a low-rank solution. 

%We note that both of these problems have been extensively studied and for both of them there exist highly specialized algorithms that are highly efficient~\cite{arora2007combinatorial,hardt2013understanding}. Our results here do not beat them; rather our goal here is to demonstrate that the Burer-Monteiro approach can successfully solve these SDPs in polynomial time. In practice, this approach is much faster than other generic SDP solvers such as interior point method and ellipsoid method, and in addition returns low rank solutions.
\subsection{Max-Cut}

We first consider the popular Max-Cut problem which finds applications in clustering related problems. In a seminal paper, \cite{goemans1995improved} defined the following SDP to solve the Max-Cut problem: $\min_{X\in \Rnn} \ip{C}{X}, \mbox{s.t. } X_{ii} = 1 \; \forall \; 1 \leq i \leq n, X \succeq 0 $, where $n$ is the number of vertices in the given graph and $C$ is its adjacency matrix. Since the constraint set also satisfies $\trace{X}=n$, we consider the following penalized, non-convex version of the problem.
% \begin{align*}	& \min_{X\in \Rnn} \ip{C}{X} \\
%	& \mbox{s.t. } X_{ii} = 1 \; \forall \; 1 \leq i \leq n \\
%	& \qquad X \succeq 0,
%\end{align*}
\begin{align}
	\widehat{L}_{\mu}(U) \defeq \ip{C+G}{U\trans{U}} + \mu\left(\left(\ip{I}{U\trans{U}}-n\right)^2 + \sum_{i=1}^{n}\left(\ip{e_i \trans{e_i}}{U\trans{U}}-1\right)^2\right),\label{eqn:maxcut}
\end{align}
where $G$ is a random symmetric Gaussian matrix.  Let $\widehat F_{\mu}(UU^T) = \widehat L_{\mu} (U)$. After some simplifying computations, we have the following corollary of Theorem~\ref{thm:optimal_approx_compact}.
\begin{corollary}\label{cor:maxcut}
There exists an absolute numerical constant $c_1$ such that the following holds. With probability greater than $1-\delta$,
every $(\eps, \gamma)$-SOSP $U$ of the perturbed Max-Cut problem $\widehat{L}_{\mu}(U)$~\eqref{eqn:maxcut} with:
\begin{align*}\epsilon \leq \frac{1}{c_1} \left(\frac{\gamma \sigma_G^2}{\mu n}\right)^{2/3},~~ \text{ and } ~~  k = \tilde{\Omega} \left( \sqrt{n \log\left(\frac{\mu^2 \sqrt{n}}{\sigma_G}\right)}\right),
\end{align*}
satisfies $	\widehat{F}_{\mu}(UU^T) - \widehat{F}_{\mu}(X^*) \leq \gamma \sqrt{\epsilon} \trace{X^*} +\frac{1}{2} \epsilon \frob{U}$, where $X^*$ is a global optimum of $\widehat{F}_{\mu}(X)$.
%\begin{align*}
%	\widehat{L}_{\mu}(U) - \widehat{L}_{\mu}(U^*) \leq \gamma \sqrt{\epsilon} \trace{U^* \trans{U^*}} + \epsilon \frob{U}.
%\end{align*}
\end{corollary}
The above result states that for the penalized version of the perturbed Max-Cut SDP, the Burer--Monteiro approach finds an approximate global optimum as soon as the factorization rank $k = \tilde{\Omega}(\sqrt{n})$. Existing results for Max-Cut using this approach either only handle exact SOSPs~\citep{boumal2016non}, or require $k=n+1$~\citep{boumal2016globalrates}, or require $k$ that is dependent on $\frac{1}{\eps}$~\citep{pmlr-v65-mei17a}. Moreover, complexity per iteration scales only linearly with the number of edges in the graph. %However, current analysis is required to set $\epsilon$ to be fairly small which can lead to a super-linear algorithm; we leave further tightening of dependence on $\epsilon$ for future work.


\subsection{Matrix Completion}
In this section we specialize our results for the matrix completion problem \cite{candes2009exact}. The goal of a matrix completion problem is to find a low-rank matrix $M$ using only a small number of its entries, with applications in recommender systems. To ensure that the computed matrix is low-rank and generalizes well, one typically imposes nuclear-norm regularization which leads to the following SDP: 

\begin{minipage}{0.2\linewidth}
	\begin{align*}
	\min &\quad \trace{W_1} + \trace{W_2}\\ \text{s. t. }&\quad X_{ij} =M_{ij}, (i,j) \in \calS \\  &\quad \begin{bmatrix}W_1 & X \\ X^T & W_2\end{bmatrix} \succeq 0.
	\end{align*}
\end{minipage}
\begin{minipage}{0.05\linewidth}
	\begin{align*}
		\equiv \\
	\end{align*} \break
\end{minipage}
\begin{minipage}{0.6\linewidth}
	\begin{align*}
	\min & \quad \ip{I}{Z} \nonumber \\ \text{s. t. }&\quad \frac{1}{2}\ip{e_{i+n}e_{j+n}^T + e_{j+n} e_{i+n}^T}{Z} = M_{ij}, (i,j) \in \calS \nonumber \\  &\quad Z \succeq 0.
	\end{align*}
\end{minipage}
%\begin{align*} \min &\quad \trace{W_1} + \trace{W_2}\\ \text{s. t. }&\quad X_{ij} =M_{ij}, (i,j) \in \Omega \\  &\quad \begin{bmatrix}W_1 & X \\ X^T & W_2\end{bmatrix} \succeq 0.
%\end{align*} 
\noindent Here $\calS$ is the set of observed indices of $M$ and $Z\defeq \begin{bmatrix}W_1 & X \\ X^T & W_2\end{bmatrix}$. % by $Z$, we can rewrite the above SDP as \begin{align} \min & \quad \ip{Z}{I} \nonumber \\ \text{s. t. }&\quad \frac{1}{2}\ip{e_{i+n}e_{j+n}^T + e_{j+n} e_{i+n}^T}{Z} = M_{ij}, (i,j) \in \Omega \nonumber \\  &\quad Z \su,cceq 0. \label{eq:mc_sdp}\end{align}
Let
\begin{align}
	\widehat{L}_{\mu}(U) = \ip{I+G}{UU^T} + \mu \sum_{i=1}^m \left(\frac{1}{2}\ip{e_{i+n}e_{j+n}^T + e_{j+n} e_{i+n}^T}{UU^T} - M_{ij} \right)^2  \label{eq:matcomp}
\end{align}
be the corresponding penalty objective.  Let $\widehat F_{\mu}(UU^T) = \widehat L_{\mu} (U)$. The objective is positive definite with $\lambda_1(C)=\lambda_n(C)=1$. Also, since $\calA$ is a sub-sampling operator, $\|\calA\| \leq 1$. Finally, for $\eps^2 \leq \frac{\mu}{2}\sqrt{\sum_{(i,j) \in \calS} M_{ij}^2}$, the residues are bounded by: \begin{align*} B&=\|\calA\| \max \left \{ \left( \frac{2\eps} {\lambda_n(C)}\right)^2, \frac{2\mu}{\lambda_n(C)} \|\vb\|_2^2 \right \}+\|\vb\|_2 \leq \max  3\mu \sqrt{\sum_{(i,j) \in \calS} M_{ij}^2}. \end{align*}


\noindent Applying Theorem~\ref{thm:optimal_approx} for this setting gives the following corollary.
\begin{corollary}\label{cor:mc_optimal}There exists an absolute numerical constant $c_2$ such that the following holds. With probability greater than $1-\delta$,
every $(\eps, \gamma)$-SOSP $U$ of the perturbed matrix completion problem $\widehat{L}_{\mu}(U)$~\eqref{eq:matcomp} with:
\begin{align*}\sG \leq \frac{1}{4\sqrt{n \log(n/ \delta)}},~~ \eps \leq \frac{1}{c_2}\left(\frac{\gamma \abs{\calS} \sigma_G^2 }{ n  \mu }\right)^{\sfrac{2}{3}}, ~\text{ and }~ k = \tilde{\Omega} \left( \sqrt{ \abs{\calS}   \log\left(\frac{\mu^2 \sqrt{n} \sqrt{\sum_{(i,j) \in \calS} M_{ij}^2}}{\sigma_G}\right) } \right),\end{align*} satisfies $\widehat{F}_{\mu}(UU^T)  - \widehat F_{\mu}(X^*)  \leq \gamma \sqrt{\epsilon} \trace{X^*} + \frac{1}{2} \eps \|U\|_F$, where $X^*$ is a global optimum of $\widehat{F}_{\mu}(X)$.
\end{corollary}
\noindent This result shows that for the  matrix completion problem with $m$ observations, for rank $\tilde{\Omega}(\sqrt{m})$, any approximate local minimum of the factorized and penalized problem is an approximate global minimum. 

Most of the existing results on matrix completion either require strong distribution assumptions on $\calS$ and incoherence assumptions on $M$ to recover a low-rank solution \citep{candes2009exact, jain2013low}. The standard nuclear norm minimization algorithms are not guaranteed to converge to low-rank solutions without these assumptions,  which implies that the entire matrix would need to be stored for prediction which is infeasible in practice. Similarly,  generalization error bounds \citep{foygel2011concentration} as well as differential privacy guarantees  depend on recovery of a low-rank solution.

Our result guarantees finding a rank -$\tilde{\Omega}(\sqrt{m})$ solution without any statistical assumptions on the sampling or the matrix. The tradeoff is our results do not guarantee finding a lower (potentially a constant) rank solution, even if one exists for a given problem. 


%\subsection{Normalized Cut}
%
%In this section we will consider the problem of computing the Normalized cut of a given graph $\calG$ \citep{shi2000normalized}. Given a graph $G$ with $n$ vertices and $e$ edges, the normalized cut is the partition of vertices into two sets $S$ and $S^c$ that minimizes, $$\frac{cut(S,S^c)}{D(S)}+ \frac{cut(S,S^c)}{D(S^c)}. $$ Here $cut(S,S^c)$ is the number of edges between $S$ and $S^c$. $D(S)$ is the sum of degree of vertices in $S$. This problem is NP-hard in the worst case. 
%
%Let $\widehat{X}$ be a $n+1 \times n+1$ matrix, with the following structure, $\widehat{X} = \begin{bmatrix}  X & 0_{n \times 1} \\ 0_{1 \times n} & x \end{bmatrix}$. \citet{bie2006fast} proposed the following SDP relaxation to find the normalized cut.
%
% \begin{align*} \underset{\widehat{X}}{\minimize} &\quad \ip{\widehat{X}}{L}\\ \text{subject to } &\quad \widehat{X} \succeq 0 \\
%&\quad \ip{\widehat{X}}{A_i} =0, i \in [n]\\
%&\quad \ip{\widehat{X}}{A_{n+1}}=-1 \\
%\end{align*}
%Here $L$ is the graph Laplacian divided by twice the number of edges, $2e$. $A_i$ ($i \in [n]$) is a matrix with $1$ at $(i,i)$ and $-1$ at $(n+1,n+1)$ entry, with rest of the entries begin $0$s. $\displaystyle A_{n+1} =\begin{bmatrix} dd^T/(4e^2) & 0_{n \times 1} \\0_{1 \times n} & -1 \end{bmatrix}$, where $d$ is the vector of degrees of the vertices.
%
%
%


%%\appendices
\section{Pseudocode for Algorithm of Section~\ref{subsec:instantly_decodable}}
\label{app:pseudocode}

\algdef{SE}[EVENT]{Event}{EndEvent}[1]{\textbf{upon event}\ #1\ \algorithmicdo}{\algorithmicend\ \textbf{event}}%
\algtext*{EndEvent}

\begin{algorithm}
\caption{Coding for Three Users with Feedback}\label{alg:three_users}
\begin{algorithmic}[1]
\State \textbf{Initialize}: $r_i \gets 0, \forall i \in \mathcal{U}$  % \Comment{Packets received by each user}
\ForAll {$t \in [N]$}
    \While {$\nexists \ i \in \mathcal{U} \ s.t.\ Z_i = 0$}
%    \State Send $S(t)$ until at least one user receives it
    	\State Transmit $S(t)$
    \EndWhile
%    \If {$\exists \ i \in \mathcal{U}, T \in \mathbb{N} \ s.t.\ N_i(T) = 0$}
%    \If {$\exists \ i \in \mathcal{U} \ s.t.\ N_i = 0$}
        \State $Q_{\mathcal{E}} \gets Q_{\mathcal{E}} \cup \{S(t)\}$
        \State $r_j\gets r_j + 1 \qquad \forall \ j \ s.t.\  Z_j = 0$
%    \EndIf
\EndFor

%\While{$\exists \ \textrm{distinct} i, j, k \in \mathcal{U}, i\neq j \neq k \ s.t.\ Q_i \neq \varnothing \ \textbf{and} \ \Q{j,k} \neq \varnothing$}
\While{$\exists \ \textrm{distinct} \ i, j, k \in \mathcal{U}, \ s.t.\ Q_i \neq \varnothing \ \textbf{and} \ \Q{j,k} \neq \varnothing$}
    \While {$\nexists \ l \in \mathcal{U} \ s.t.\ Z_l = 0$}
%	\State Let $q_i \in Q_i$, $q_{j,k} \in \Q{j,k}$ 
%       \State Transmit $q_i \oplus q_{j,k}$ until at least one user receives it  
       \State Transmit $q_i \oplus q_{j,k}$ where $q_i \in Q_i$, $q_{j,k} \in \Q{j,k}$ 
    \EndWhile
    \State $r_l\gets r_l + 1 \qquad \forall \ l \ s.t.\  Z_l = 0$
    \If {$Z_i = 0$}
    	\State $Q_i \gets Q_i \setminus \{q_i\}$
    \EndIf
    \If {$Z_j = 0 \ \textbf{and} \ Z_k = 1$}
    	\State $Q_k \gets Q_k \cup \{q_{j, k}\}$
       \State $\Q{j,k} \gets \Q{j, k} \setminus \{q_{j,k}\}$
    \ElsIf {$Z_j = 1 \ \textbf{and} \ Z_k = 0$}
    	\State $Q_j \gets Q_j \cup \{q_{j, k}\}$
	\State $\Q{j,k} \gets \Q{j, k} \setminus \{q_{j,k}\}$
    \ElsIf {$Z_j = 0 \ \textbf{and} \ Z_k = 0$}
    	\State $\Q{j,k} \gets \Q{j, k} \setminus \{q_{j,k}\}$
    \EndIf    
\EndWhile

\While{$Q_i \neq \varnothing \  \forall \ i \in \mathcal{U}$}
	\While {$\nexists \ l \in \mathcal{U} \ s.t.\ Z_l = 0$}
		\State Transmit $q_1 \oplus q_{2} \oplus q_{3}$, where $q_i \in Q_i \ \forall i \in \mathcal{U}$
	\EndWhile
	\ForAll {$l \in \mathcal{U} \ s.t.\ Z_l = 0$}
		\State $Q_l \gets Q_l \setminus \{q_l\}$
		\State $z_l \gets z_l + 1$
	\EndFor
\EndWhile

\Event{$ \exists \ i \ s.t.\ r_i \geq 1- d_i  $} %\Comment{One user finished}
%	\State Let $j, k \in \mathcal{U} \setminus \{i\}, \ s.t.\ j \neq k$
	\State $Q_j \gets Q_j \cup \Q{i,j} \qquad \forall \ j \in \mathcal{U} \setminus \{i\}$
	\State \textbf{run} two-user algorithm of Section~\ref{sec:two_users}
\EndEvent
\end{algorithmic}
\end{algorithm}


%\newpage
\bibliographystyle{alpha}
\bibliography{refs,vf-allrefs}
\onecolumn


% \tableofcontents{}

% \newpage

\section*{Supplementary Material}
\addcontentsline{toc}{section}{Supplementary Material}


Throughout this discussion, 
we will make frequently use 
of the following standard results
concerning the exponential concentration 
of random variables:

\begin{lemma}[Hoeffding's inequality for independent RVs~\citep{hoeffding1994probability}] Let $Z_1, Z_2, \ldots, Z_n$ be independent bounded random variables with $Z_i \in [a,b]$ for all $i$, then 
    \begin{align*}
        \prob\left( \frac{1}{n} \sum_{i=1}^n (Z_i - \Expo{Z_i}) \ge t \right) \le \exp{\left( -\frac{2nt^2}{(b-a)^2} \right) }
    \end{align*} 
    and 
    \begin{align*}
        \prob\left( \frac{1}{n} \sum_{i=1}^n (Z_i - \Expo{Z_i}) \le -t \right) \le \exp{\left( -\frac{2nt^2}{(b-a)^2} \right) }
    \end{align*} 
    for all $t \ge 0$. 
\end{lemma}

\begin{lemma}[Hoeffding's inequality for sampling with replacement~\citep{hoeffding1994probability}] \label{lem:hoeffding_sampling} Let $\calZ = (Z_1, Z_2, \ldots, Z_N)$ be a finite population of $N$ points with $Z_i \in [a.b]$ for all $i$. Let $X_1, X_2, \ldots X_n$ be a random sample drawn without replacement from $\calZ$. Then for all $t \ge 0$, we have 
    \begin{align*}
        \prob\left( \frac{1}{n} \sum_{i=1}^n (X_i - \mu ) \ge t \right) \le \exp{\left( -\frac{2nt^2}{(b-a)^2} \right) }
    \end{align*} 
    and 
    \begin{align*}
        \prob\left( \frac{1}{n} \sum_{i=1}^n (X_i - \mu ) \le -t \right) \le \exp{\left( -\frac{2nt^2}{(b-a)^2} \right) } \,,
    \end{align*} 
    where $\mu = \frac{1}{N} \sum_{i=1}^{N} Z_i$. 
\end{lemma}

We now discuss one condition that generalizes the exponential concentration to dependent random variables.
\begin{condition}[Bounded difference inequality] \label{cond:BDC} Let $\calZ$ be some set and $\phi: \calZ^n \to \Real$. We say that $\phi$ satisfies the bounded difference assumption if 
there exists $c_1, c_2, \ldots c_n \ge 0$ s.t. for all $i$, we have 
\begin{align*}
    \sup_{Z_1,Z_2, \ldots,Z_n, Z_i^\prime \in \calZ^{n+1} } \abs{\phi (Z_1, \ldots, Z_i, \ldots, Z_n ) - \phi (Z_1, \ldots, Z_i^\prime, \ldots, Z_n ) } \le c_i \,.
\end{align*} 
\end{condition}

\begin{lemma}[McDiarmid’s inequality~\citep{mcdiarmid1989}] \label{lem:McDiarmid} Let $Z_1, Z_2, \ldots, Z_n$ be independent random variables on set $\calZ$ and $\phi : \calZ^n \to \Real$ satisfy bounded difference inequality (\codref{cond:BDC}). Then for all $t>0$, we have 
    \begin{align*}
        \prob\left( \phi(Z_1, Z_2, \ldots, Z_n) - \Expo{\phi(Z_1, Z_2, \ldots, Z_n)} \ge t \right) \le \exp{\left( -\frac{2t^2}{\sum_{i=1}^n c_i^2} \right) } 
    \end{align*} 
    and 
    \begin{align*}
        \prob\left( \phi(Z_1, Z_2, \ldots, Z_n) - \Expo{\phi(Z_1, Z_2, \ldots, Z_n)} \le -t \right) \le \exp{\left( -\frac{2t^2}{\sum_{i=1}^n c_i^2} \right) } \,.
    \end{align*} 
\end{lemma}


\section{Proofs from \secref{sec:ERM_training}}\label{app:proof_erm}

\textbf{Additional notation {} {}} Let $m_1$ be the number of mislabeled points ($\wt S_M$) and $m_2$ be the number of correctly labeled points ($\wt S_C$). Note $m_1 + m_2 = m$. 


\subsection{Proof of \thmref{thm:error_ERM}}


\begin{proof}[Proof of \lemref{lem:fit_mislabeled}] 
    The main idea of our proof is to regard 
    the clean portion of the data 
    ($S \cup \wt S_C$) as fixed.   
    Then, there exists an (unknown) classifier $f^*$ 
    that minimizes the expected risk
    calculated on the (fixed) clean data
    and (random draws of) the mislabeled data $\wt S_M$. 
    % 
    % 
    Formally, 
    \begin{align}
    f^* \defeq \argmin_{f \in \calF} \error_{\widecheck {\calD}} (f) \,, \label{eq:modified_ERM}
    \end{align}
    where $$\widecheck \calD = \frac{n}{m+n} \calS + \frac{m_2}{m+n} \wt \calS_C  + \frac{m_1}{m+n}\calDm \,.$$ 
    Note here that $\widecheck \calD$ is a combination 
    of the \emph{empirical distribution} 
    over correctly labeled data $S \cup \wt S_C$
    and the (population) distribution 
    over mislabeled data $\calDm$.
    Recall that 
    \begin{align}
    \wh f \defeq \argmin_{f \in \calF} \error_{\calS \cup \wt S} (f) \,. \label{eq:orig_ERM}
    \end{align}
    % 
    % 
    Since, $\widehat f$ minimizes 0-1 error 
    on $S \cup \wt S$, using ERM optimality on \eqref{eq:orig_ERM},  
    we have 
    \begin{align}
        \error_{\calS \cup \wt \calS}(\widehat f) \le \error_{
            \calS \cup \wt \calS}(f^*) \,.    \label{eq:step1}
    \end{align}
    Moreover, since $f^*$ is independent of $\wt S_M$, using Hoeffding's bound,
    % \footnote{For a fully rigorous argument,
    % refer to the complete proof in App.~\ref{app:proof_erm}.} 
    we have with probability at least $1-\delta$ that
    \begin{align}
      \error_{\wt \calS_M}(f^*) \le \error_{ \calDm}(f^*) +  \sqrt{\frac{\log(1/\delta)}{2 m_1}} \,. \label{eq:step2} 
    \end{align}
    %$ 
    %for some constant $c_1\le 1/2$. 
    Finally, since $f^*$ is the optimal classifier on $\widecheck \calD$, 
    we have 
    \begin{align}
        \error_{\widecheck \calD}(f^*) \le \error_{\widecheck \calD}(\widehat f) \,. \label{eq:step3}
    \end{align}
    Now to relate \eqref{eq:step1} and \eqref{eq:step3}, we multiply \eqref{eq:step2} by $\frac{m_1}{m+n}$ and add $\frac{n}{m+n} \error_{\calS} (f)  + \frac{m_2}{m+n} \error_{\wt \calS_C} (f)$ both the sides. Hence, 
    we can rewrite \eqref{eq:step2} as follows: 
    \begin{align}
        \error_{\calS \cup \wt\calS}(f^*) \le \error_{ \widecheck \calD}(f^*) +  \frac{m_1}{m+n}\sqrt{\frac{\log(1/\delta)}{2 m_1}} \,. \label{eq:step4} 
    \end{align}
    Now we combine equations \eqref{eq:step1}, \eqref{eq:step4}, and \eqref{eq:step3}, to get 
    \begin{align}
        \error_{\calS \cup \wt \calS}(\wh f) \le \error_{\widecheck \calD}(\wh f) +  \frac{m_1}{m+n}\sqrt{\frac{\log(1/\delta)}{2 m_1}} \,, 
    \end{align}
    which implies 
    \begin{align}
        \error_{ \wt \calS_M}(\wh f) \le \error_{\calDm}(\wh f) + \sqrt{\frac{\log(1/\delta)}{2 m_1}} \,. \label{eq:lemma1_final}
    \end{align}
    Since $\wt S$ is obtained by randomly labeling an unlabeled dataset, we assume $2m_1 \approx m$ \footnote{Formally, with probability at least $1-\delta$, we have  $(m - 2m_1)\le \sqrt{m\log(1/\delta)/2}$.}. Moreover, using $\error_{\calDm} = 1 - \error_{\calD}$ we obtain the desired result.   
    % Combining the above steps and using the fact 
    % that $\error_\calD = 1- \error_{\calDm} $, 
    % we obtain the desired result.
\end{proof}

\begin{proof}[Proof of \lemref{lem:mislabeled_error}]
    Recall $\error_{\wt S} (f) = \frac{m_1}{m} \error_{\wt S_M}(f) + \frac{m_2}{m} \error_{\wt S_C}(f)$. Hence, we have 
    \begin{align}
        2\error_{\wt S}(f) - \error_{\wt S_M}(f) - \error_{\wt S_C}(f) &= \left(\frac{2m_1}{m} \error_{\wt S_M}(f) - \error_{\wt S_M}(f)\right) + \left(\frac{2m_2}{m} \error_{\wt S_C}(f) - \error_{\wt S_C}(f)\right) \\ &= \left(\frac{2m_1}{m} - 1\right) \error_{\wt S_M}(f) + \left(\frac{2m_2}{m} - 1 \right)\error_{\wt S_C} (f) \,.
    \end{align} 
    Since the dataset is labeled uniformly at random, with probability at least $1-\delta$, we have  $\left(\frac{2m_1}{m} - 1\right) \le \sqrt{\frac{\log(1/\delta)}{2m}}$. Similarly, we have with probability at least $1-\delta$, $\left(\frac{2m_2}{m} - 1\right) \le \sqrt{\frac{\log(1/\delta)}{2m}}$. Using union bound, with probability at least $1-\delta$, we have
    % \begin{align}
    %     2\error_{\wt S} - \error_{\wt S_M}(f) - \error_{\wt S_C}(f) \le \sqrt{\frac{\log(2/\delta)}{2m}} \left(\error_{\wt S_M}(f) + \error_{\wt S_C}(f) \right) \le 2\sqrt{\frac{\log(2/\delta)}{2m}} \,. \label{eq:lemma2_final}
    % \end{align}
    \begin{align}
        2\error_{\wt S} - \error_{\wt S_M}(f) - \error_{\wt S_C}(f) \le \sqrt{\frac{\log(2/\delta)}{2m}} \left(\error_{\wt S_M}(f) + \error_{\wt S_C}(f) \right) \,. \label{eq:lemma2_prefinal}
    \end{align}
    With re-arranging $\error_{\wt S_M}(f) + \error_{\wt S_C}(f)$ and using the inequality $ 1- a\le \frac{1}{1+a} $, we have  
    \begin{align}
        2\error_{\wt S} - \error_{\wt S_M}(f) - \error_{\wt S_C}(f) \le 2\error_{\wt \calS} \sqrt{\frac{\log(2/\delta)}{2m}}  \,. \label{eq:lemma2_final}
    \end{align}

    % We obtain the desired result by using 
\end{proof}

\begin{proof}[Proof of \lemref{lem:clear_error}]
% Recall 0-1 error on each point  $(x,y) \in S \cup \wt S$ is given by $\I{ f(x)\ne y}$.
In the set of correctly labeled points $S \cup \wt S_C$, we have $S$ as a random subset of $S \cup \wt S_C$. Hence, using Hoeffding's inequality for sampling without replacement (\lemref{lem:hoeffding_sampling}), we have with probability at least $1-\delta$
\begin{align}
    \error_{\wt \calS_C} (\wh f)- \error_{\calS \cup \wt \calS_C}( \wh f) \le  \sqrt{\frac{\log(1/\delta)}{2m_2}} \,.
\end{align}
Re-writing $\error_{\calS \cup \wt \calS_C}( \wh f)$ as $\frac{m_2}{m_2 + n} \error_{\wt \calS_C }(\wh f) + \frac{n}{m_2 + n} \error_{\calS }(\wh f)$, we have with probability at least $1-\delta$
\begin{align}
   \left(\frac{n}{n+m_2}\right) \left(\error_{\wt \calS_C} (\wh f)- \error_{\calS}( \wh f) \right) \le  \sqrt{\frac{\log(1/\delta)}{2m_2}} \,.
\end{align}
As before, assuming $2m_2 \approx m$, we have with probability at least $1-\delta$ 
\begin{align}
    \error_{\wt \calS_C} (\wh f)- \error_{\calS}( \wh f) \le \left(1+\frac{m_2}{n}\right)  \sqrt{\frac{\log(1/\delta)}{m}} \le \left(1 + \frac{m}{2n}\right) \sqrt{\frac{\log(1/\delta)}{m}} \,. \label{eq:lemma3_final}
\end{align} 
\end{proof}

\begin{proof}[Proof of \thmref{thm:error_ERM}] 
    Having established these core intermediate results, we can now combine above three lemmas to prove the main result. 
    In particular, we bound the population error on clean data ($\error_\calD(\wh f)$) as follows:  
    \begin{enumerate}[(i)]
        \item First, use \eqref{eq:lemma1_final}, to obtain an upper bound on the population error on clean data, i.e., with probability at least $1-\delta/4$, we have
        \begin{align}
            \error_{ \calD} (\wh f) \le 1 - \error_{ \wt \calS_M}(\wh f) + \sqrt{\frac{\log(4/\delta)}{m}} \,. 
        \end{align}
        \item  Second, use \eqref{eq:lemma2_final}, to relate the error on the mislabeled fraction with error on clean portion of randomly labeled data and error on whole randomly labeled dataset, i.e., with probability at least $1-\delta/2$, we have 
        \begin{align}
            - \error_{\wt S_M}(f) \le \error_{\wt S_C}(f) - 2\error_{\wt S}  + 2\error_{\wt S} \sqrt{\frac{\log(4/\delta)}{2m}}  \,. 
        \end{align} 
        \item Finally, use \eqref{eq:lemma3_final} to relate the error on the clean portion of randomly labeled data and error on clean training data, i.e., with probability $1-\delta/4$, we have 
        \begin{align}
            \error_{\wt \calS_C} (\wh f)\le - \error_{\calS}( \wh f) + \left(1 + \frac{m}{2n} \right) \sqrt{\frac{\log(4/\delta)}{m}} \,. 
        \end{align} 
    \end{enumerate}

    Using union bound on the above three steps, we have with probability at least $1-\delta$: 
    \begin{align}
        \error_\calD (\wh f) \le \error_{\calS}(\wh f)   + 1 - 2\error_{\wt \calS}(\wh f)   + \left(\sqrt{2} \error_{\wt S} + 2 + \frac{m}{2n}\right)  \sqrt{\frac{\log(4/\delta)}{m}} \,.
    \end{align}
    % Note that $(1/\sqrt{2} + 2.5)$ is a loose constant. In experiments, we use the ratio $\frac{m}{n}$
    %  the exact error $\error_{\wt \calS}(\wh f)$ 
    % to evaluate R.H.S.    
\end{proof}

\subsection{Proof of \propref{prop:rademacher}}

\begin{proof}[Proof of \propref{prop:rademacher}]
    For a classifier $ f: \calX \to \{-1, 1\}$, we have $1 - 2\,\indict{ f(x) \ne y} = y \cdot f(x)$. Hence, by definition of $\error$, we have 
    \begin{align}
        1 -2\error_{\wt \calS}(f) = \frac{1}{m}\sum_{i=1}^m y_i \cdot f(x_i) \le \sup_{f \in \calF} \, \frac{1}{m} \sum_{i=1}^m y_i \cdot f(x_i)  \,. \label{eq:error_rademacher}
    \end{align}
    Note that for fixed inputs $(x_1, x_2, \ldots, x_m)$ in $\wt S$, $(y_1, y_2, \ldots y_m)$ are random labels. Define $\phi_1 (y_1, y_2, \ldots, y_m) \defeq \sup_{f \in \calF} \, \frac{1}{m} \sum_{i=1}^m y_i \cdot f(x_i)$. We have the following bounded difference condition on $\phi_1$. For all i, 
    \begin{align}
        \sup_{y_1, \ldots y_m, y_i^\prime \in \{-1, 1\}^{m+1} } \abs{ \phi_1 (y_1,\ldots, y_i, \ldots, y_m) - \phi_1 (y_1,\ldots, y_i^\prime, \ldots, y_m)  } \le 1/m \,. \label{cond1_rademacher}
    \end{align} 
    
    Similarly, we define $\phi_2 (x_1, x_2, \ldots, x_m) \defeq \Expt{ y_i \sim_U \{-1, 1\}  }{ \sup_{f \in \calF} \, \frac{1}{m}  \sum_{i=1}^m y_i \cdot f(x_i)}$. We have the following bounded difference condition on $\phi_2$. 
    For all i,
    \begin{align}
        \sup_{x_1, \ldots x_m, x_i^\prime \in \calX^{m+1} } \abs{ \phi_2 (x_1,\ldots, x_i, \ldots, x_m) - \phi_1 (x_1,\ldots, x_i^\prime, \ldots, x_m)  } \le 1/m \,. \label{cond2_rademacher}
    \end{align}
    Using McDiarmid’s inequality (\lemref{lem:McDiarmid}) twice 
    with Condition \eqref{cond1_rademacher} and \eqref{cond2_rademacher}, 
    with probability at least $1-\delta$, we have
    \begin{align}
        \sup_{f \in \calF} \, \frac{1}{m} \sum_{i=1}^m y_i \cdot f(x_i)  - \Expt{x,y}{\sup_{f \in \calF} \, \frac{1}{m} \sum_{i=1}^m y_i \cdot f(x_i) } \le \sqrt{\frac{2\log(2/\delta)}{m}} \,. \label{eq:final_rademacher}
    \end{align} 
    Combining \eqref{eq:error_rademacher} and \eqref{eq:final_rademacher}, we obtain the desired result. 
\end{proof}


\subsection{Proof of \thmref{thm:error_regularized_ERM}}

Proof of \thmref{thm:error_regularized_ERM} follows similar to the proof of \thmref{thm:error_ERM}. Note that the same results in \lemref{lem:fit_mislabeled}, \lemref{lem:mislabeled_error}, and \lemref{lem:clear_error} hold in the regularized ERM case. However, the arguments in the proof of \lemref{lem:fit_mislabeled} change slightly. Hence, we state the lemma for regularized ERM and prove it here for completeness. 

\begin{lemma} \label{lem:lemma1_reg}
    Assume the same setup as \thmref{thm:error_regularized_ERM}. 
    Then for any $\delta >0$, with probability at least  $1-\delta$ 
    over the random draws of mislabeled data $\wt S_M$, we have 
    \begin{align}
        \error_\calD(\widehat f)  \le 1 -\error_{\wt \calS_M}(\widehat f) + \sqrt{\frac{\log(1/\delta)}{m}}\,. 
    \end{align} 
\end{lemma}
\begin{proof}
    The main idea of the proof remains the same, i.e. regard 
    the clean portion of the data 
    ($S \cup \wt S_C$) as fixed.   
    Then, there exists a classifier $f^*$ 
    that is optimal over draws 
    of the mislabeled data $\wt S_M$. 

    
    Formally, 
    \begin{align}
    f^* \defeq \argmin_{f \in \calF} \error_{\widecheck {\calD}} (f)  + \lambda R(f) \,, \label{eq:modified_ERM_reg}
    \end{align}
    where $$\widecheck \calD = \frac{n}{m+n} \calS + \frac{m_1}{m+n} \wt \calS_C  + \frac{m_2}{m+n}\calDm \,.$$ That is, $\widecheck \calD$ a combination of 
    the \emph{empirical distribution} 
    over correctly labeled data $S \cup \wt S_C$
    % in $S\cup \wt S$ 
    and the (population) distribution 
    over mislabeled data $\calDm$.
    Recall that 
    \begin{align}
    \wh f \defeq \argmin_{f \in \calF} \error_{\calS \cup \wt S} (f) + \lambda R(f) \,. \label{eq:orig_ERM_reg}
    \end{align}
    % 
    % 
    Since, $\widehat f$ minimizes 0-1 error 
    on $S \cup \wt S$, using ERM optimality on \eqref{eq:orig_ERM},  
    we have 
    \begin{align}
        \error_{\calS \cup \wt \calS}(\widehat f) + \lambda R(\wh f) \le \error_{
            \calS \cup \wt \calS}(f^*) + \lambda R(f^*) \,.    \label{eq:step1_reg}
    \end{align}
    Moreover, since $f^*$ is independent of $\wt S_M$, using Hoeffding's bound,
    % \footnote{For a fully rigorous argument,
    % refer to the complete proof in App.~\ref{app:proof_erm}.} 
    we have with probability at least $1-\delta$ that
    \begin{align}
      \error_{\wt \calS_M}(f^*) \le \error_{ \calDm}(f^*) +  \sqrt{\frac{\log(1/\delta)}{2 m_1}} \,. \label{eq:step2_reg} 
    \end{align}
    %$ 
    %for some constant $c_1\le 1/2$. 
    Finally, since $f^*$ is the optimal classifier on $\widecheck \calD$, 
    we have 
    \begin{align}
        \error_{\widecheck \calD}(f^*) + \lambda R(f^*) \le \error_{\widecheck \calD}(\widehat f) + \lambda R(\wh f) \,. \label{eq:step3_reg}
    \end{align}
     Now to relate \eqref{eq:step1_reg} and \eqref{eq:step3_reg}, we can re-write the \eqref{eq:step2_reg} as follows: 
    \begin{align}
        \error_{\calS \cup \wt\calS}(f^*) \le \error_{ \widecheck \calD}(f^*) +  \frac{m_1}{m+n}\sqrt{\frac{\log(1/\delta)}{2 m_1}} \,. \label{eq:step4_reg} 
    \end{align}
    After adding $\lambda R(f^*)$ on both sides in \eqref{eq:step4_reg}, we combine equations \eqref{eq:step1_reg}, \eqref{eq:step4_reg}, and \eqref{eq:step3_reg}, to get 
    \begin{align}
        \error_{\calS \cup \wt \calS}(\wh f) \le \error_{\widecheck \calD}(\wh f) +  \frac{m_1}{m+n}\sqrt{\frac{\log(1/\delta)}{2 m_1}} \,, 
    \end{align}
    which implies 
    \begin{align}
        \error_{ \wt \calS_M}(\wh f) \le \error_{\calDm}(\wh f) + \sqrt{\frac{\log(1/\delta)}{2 m_1}} \,. \label{eq:lemma_reg_final}
    \end{align}
    Similar as before, since $\wt S$ is obtained by randomly labeling an unlabeled dataset, we assume 
    $2m_1 \approx m$. Moreover, using $\error_{\calDm} = 1 - \error_{\calD}$ we obtain the desired result. 
\end{proof}
% \begin{proof}[Proof of ]
    
% \end{proof}

\subsection{Proof of \thmref{thm:multiclass_ERM}}

To prove our results in the multiclass case,
we first state and prove lemmas
parallel to those
% We first state and prove lemmas 
% parallel 
% to the three lemmas 
used in the proof of balanced binary case. 
We then combine these results 
% in the three lemmas 
to obtain the result in \thmref{thm:multiclass_ERM}. 

Before stating the result, 
we define mislabeled distribution $\calDm$ for any $\calD$.
While $\calDm$ and $\calD$ share 
the same marginal distribution over inputs $\calX$,
the conditional distribution over labels $y$ 
given an input $x\sim \calD_\calX$ is changed as follows:
For any $x$, the Probability Mass Function (PMF) over $y$ is defined as:  
$p_{\calDm} (\cdot \vert x) \defeq \frac{1 - p_{\calD}(\cdot \vert x)}{k - 1}$, where $ p_{\calD}(\cdot \vert x)$ is the PMF over $y$ for the distribution $\calD$. 

\begin{lemma} \label{lem:fit_mislabeled_multi}
    Assume the same setup as \thmref{thm:multiclass_ERM}. 
    Then for any $\delta >0$, with probability at least  $1-\delta$ 
    over the random draws of mislabeled data $\wt S_M$, we have 
    \begin{align}
        \error_\calD(\widehat f)  \le (k-1)\left(1 -\error_{\wt \calS_M}(\widehat f)\right) + (k-1)\sqrt{\frac{\log(1/\delta)}{m}}\,. \label{eq:lemma1_multi}
    \end{align}   
\end{lemma} 

\begin{proof}
   
    The main idea of the proof remains the same.
    We begin by regarding the clean portion of the data 
    ($S \cup \wt S_C$) as fixed. 
    Then, there exists a classifier $f^*$ 
    that is optimal over draws 
    of the mislabeled data $\wt S_M$. 
    
    However, in the multiclass case,
    we cannot as easily relate the population error on mislabeled data 
    to the population accuracy on clean data.   
    While for binary classification, 
    % we could upper bound $\error_{\wt \calS_M}$ 
    % with $1-\error_\calD$ 
    we could lower bound the population accuracy $1-\error_\calD$
    with the empirical error on mislabeled data $\error_{\wt \calS_M}$ 
    (in the proof of \lemref{lem:fit_mislabeled}), 
    for multiclass classification, 
    error on the mislabeled data 
    and accuracy on the clean data 
    in the population 
    are not so directly related.  
    To establish \eqref{eq:lemma1_multi},
    we break the error on the 
    (unknown) mislabeled data 
    into two parts: one term corresponds 
    to predicting the true label on mislabeled data, 
    and the other corresponds to predicting 
    neither the true label 
    nor the assigned (mis-)label.  
    Finally, we relate these errors to their
    population counterparts to establish \eqref{eq:lemma1_multi}. 
    
    Formally, 
    \begin{align}
    f^* \defeq \argmin_{f \in \calF} \error_{\widecheck {\calD}} (f)  + \lambda R(f) \,, \label{eq:modified_ERM_reg2}
    \end{align}
    where $$\widecheck \calD = \frac{n}{m+n} \calS + \frac{m_1}{m+n} \wt \calS_C  + \frac{m_2}{m+n}\calDm \,.$$ 
    That is, $\widecheck \calD$ is a combination 
    of the \emph{empirical distribution} 
    over correctly labeled data $S \cup \wt S_C$
    % in $S\cup \wt S$ 
    and the (population) distribution 
    over mislabeled data $\calDm$.
    Recall that 
    \begin{align}
    \wh f \defeq \argmin_{f \in \calF} \error_{\calS \cup \wt S} (f) + \lambda R(f) \,. \label{eq:orig_ERM_reg2}
    \end{align}
    % 
    % 
    Following the exact steps from the proof of \lemref{lem:lemma1_reg}, 
    with probability at least $1-\delta$, we have  
    \begin{align}
        \error_{ \wt \calS_M}(\wh f) \le \error_{\calDm}(\wh f) + \sqrt{\frac{\log(1/\delta)}{2 m_1}} \,. \label{eq:lemma1_final_multi_prev}
    \end{align}
    Similar to before, since $\wt S$ is obtained 
    by randomly labeling an unlabeled dataset, 
    we assume 
    $\frac{k}{k-1} m_1 \approx m$. 
    
    Now we will relate $\error_{\calDm} (\wh f)$ with $\error_{\calD}(\wh f)$. 
    Let $y^T$ denote the (unknown) true label 
    for a mislabeled point $(x, y)$ 
    (i.e., label before replacing it with a mislabel). 
    \begin{align*}    
         \Expt{(x, y) \in \sim \calDm}{\indict{ \wh f(x) \ne y }}  &= \underbrace{\Expt{(x, y) \in \sim \calDm}{\indict{ \wh f(x) \ne y \land \wh f(x) \ne y^T}}}_{\RN{1}} \\ &\qquad \qquad + \underbrace{\Expt{(x, y) \in \sim \calDm}{\indict{ \wh f(x) \ne y \land \wh f(x) = y^T}}}_{\RN{2}} \,. \numberthis \label{eq:excess_term}
    \end{align*}
    Clearly, term 2 is one minus the accuracy 
    on the clean unseen data, i.e.,
    \begin{align}
        \RN{2} = 1 - \Expt{{x,y} \sim \calD}{ \indict{ \wh f(x) \ne y}} = 1- \error_{\calD}(\wh f) \,. \label{eq:term1}    
    \end{align}
    Next, we relate term 1 with the error on the unseen clean data. 
    We show that term 1 is equal to the error on the unseen clean data 
    scaled by $\frac{k-2}{k-1}$,
    where $k$ is the number of labels.
    Using the definition of mislabeled distribution $\calDm$,  
    we have 
    \begin{align}
        \RN{1} = \frac{1}{k-1} \left( \Expt{(x, y) \in \sim \calD}{ \sum_{i \in \calY \land i\ne y}  \indict{ \wh f(x) \ne i \land \wh f(x) \ne y}} \right) = \frac{k-2}{k-1} \error_{\calD}(\wh f) \,.\label{eq:term2}
    \end{align}    

    Combining the result in \eqref{eq:term1}, \eqref{eq:term2} and \eqref{eq:excess_term}, we have 
    \begin{align}
        \error_{\calDm}(\wh f) = 1- \frac{1}{k-1} \error_{\calD}(\wh f) \,.\label{eq:combine_terms}
    \end{align}
    Finally, combining the result in \eqref{eq:combine_terms} 
    with equation \eqref{eq:lemma1_final_multi_prev}, 
    we have with probability $1-\delta$, 
    \begin{align}
      \error_{\calD}(\wh f) \le  (k-1) \left( 1- \error_{ \wt \calS_M}(\wh f) \right)  + (k-1) \sqrt{\frac{k \log(1/\delta)}{ 2(k-1)m}} \,. \label{eq:lemma1_final_multi}
    \end{align}
\end{proof}

\begin{lemma} \label{lem:mislabeled_error_multi}
    Assume the same setup as \thmref{thm:multiclass_ERM}. 
    Then for any $\delta >0$, 
    with probability at least $1-\delta$ 
    over the random draws of $\wt S$, we have  
    % \begin{align}
        $$\abs{k\error_{\wt \calS}(\widehat f) - \error_{\wt \calS_C}(\widehat f) -  (k-1)\error_{\wt \calS_M}(\widehat f) } \le  2k\sqrt{\frac{\log(4/\delta)}{2m}}\,. $$ % \label{eq:lemma2}
    % \end{align}   
    %  for some constant $c_3 \le 1.0\,$.
\end{lemma} 


\begin{proof}
    Recall $\error_{\wt S} (f) = \frac{m_1}{m} \error_{\wt S_M}(f) + \frac{m_2}{m} \error_{\wt S_C}(f)$. Hence, we have 
    \begin{align*}
        k\error_{\wt S}(f) - (k-1)\error_{\wt S_M}(f) - \error_{\wt S_C}(f) &= (k-1)\left(\frac{k m_1}{(k-1) m} \error_{\wt S_M}(f) - \error_{\wt S_M}(f)\right) \\ & \qquad \qquad + \left(\frac{km_2}{m} \error_{\wt S_C}(f) - \error_{\wt S_C}(f)\right) \\ &= k \left[ \left(\frac{m_1}{m} - \frac{k-1}{k}\right) \error_{\wt S_M}(f) + \left(\frac{m_2}{m} - \frac{1}{k} \right) \error_{\wt S_C} (f) \right] \,.
    \end{align*} 
    Since the dataset is randomly labeled, 
    we have with probability at least $1-\delta$, 
    $\left(\frac{m_1}{m} - \frac{k-1}{k}\right) \le \sqrt{\frac{\log(1/\delta)}{2m}}$. 
    Similarly, we have with probability at least $1-\delta$, 
    $\left(\frac{m_2}{m} - \frac{1}{k}\right) \le \sqrt{\frac{\log(1/\delta)}{2m}}$. 
    Using union bound, we have with probability at least $1-\delta$
    % \begin{align}
    %     2\error_{\wt S} - \error_{\wt S_M}(f) - \error_{\wt S_C}(f) \le \sqrt{\frac{\log(2/\delta)}{2m}} \left(\error_{\wt S_M}(f) + \error_{\wt S_C}(f) \right) \le 2\sqrt{\frac{\log(2/\delta)}{2m}} \,. \label{eq:lemma2_final}
    % \end{align}
    \begin{align}
        k\error_{\wt S}(f) - (k-1)\error_{\wt S_M}(f) - \error_{\wt S_C}(f)  \le k \sqrt{\frac{\log(2/\delta)}{2m}} \left(\error_{\wt S_M}(f) + \error_{\wt S_C}(f) \right) \,. \label{eq:lemma2_final_multi}
    \end{align}

    % We obtain the desired result by using 
\end{proof}

\begin{lemma} \label{lem:clear_error_multi}
    Assume the same setup as \thmref{thm:multiclass_ERM}. 
    Then for any $\delta >0$, with probability at least $1-\delta$ 
    over the random draws of $\wt S_C$ and $S$, we have 
    % \begin{align}
        $$\abs{\error_{\wt \calS_C}(\widehat f) - \error_{\calS}(\widehat f) } \le 1.5 \sqrt{\frac{k\log(2/\delta)}{2m}}\,.$$ %\label{eq:lemma3}
    % \end{align}   
    % for some constant $c_2 \le 1.2\,$.
\end{lemma} 
\begin{proof}
    % Recall 0-1 error on each point  $(x,y) \in S \cup \wt S$ is given by $\I{ f(x)\ne y}$.
    In the set of correctly labeled points $S \cup \wt S_C$,
    we have $S$ as a random subset of $S \cup \wt S_C$. 
    Hence, using Hoeffding's inequality 
    for sampling without replacement 
    (\lemref{lem:hoeffding_sampling}), 
    we have with probability at least $1-\delta$
    \begin{align}
        \error_{\wt \calS_c} (\wh f)- \error_{\calS \cup \wt \calS_C}( \wh f) \le  \sqrt{\frac{\log(1/\delta)}{2m_2}} \,.
    \end{align}
    Re-writing $\error_{\calS \cup \wt \calS_C}( \wh f)$ 
    as $\frac{m_2}{m_2 + n} \error_{\wt \calS_C }(\wh f) + \frac{n}{m_2 + n} \error_{\calS }(\wh f)$, 
    we have with probability at least $1-\delta$
    \begin{align}
       \left(\frac{n}{n+m_2}\right) \left(\error_{\wt \calS_c} (\wh f)- \error_{\calS}( \wh f) \right) \le  \sqrt{\frac{\log(1/\delta)}{2m_2}} \,.
    \end{align}
    As before, assuming $km_2 \approx m$, 
    we have with probability at least $1-\delta$ 
    \begin{align}
        \error_{\wt \calS_c} (\wh f)- \error_{\calS}( \wh f) \le \left(1+\frac{m_2}{n}\right)  \sqrt{\frac{k\log(1/\delta)}{2m}} \le \left( 1 + \frac{1}{k}\right) \sqrt{\frac{k\log(1/\delta)}{2m}} \,. \label{eq:lemma3_final_multi}
    \end{align} 
\end{proof}

\begin{proof}[Proof of \thmref{thm:multiclass_ERM}] 
    Having established these core intermediate results, 
    we can now combine above three lemmas. 
    In particular, we bound the population error 
    on clean data ($\error_\calD(\wh f)$) as follows:  
    \begin{enumerate}[(i)]
        \item First, use \eqref{eq:lemma1_final_multi}, 
        to obtain an upper bound on the population error on clean data, 
        i.e., with probability at least $1-\delta/4$, we have
        \begin{align}
            \error_{ \calD} (\wh f) \le (k-1)\left(1 - \error_{ \wt \calS_M}(\wh f) \right) + (k-1) \sqrt{\frac{k\log(4/\delta)}{2(k-1)m}} \,. 
        \end{align}
        \item  Second, use \eqref{eq:lemma2_final_multi}
        to relate the error on the mislabeled fraction 
        with error on clean portion of randomly labeled data 
        and error on whole randomly labeled dataset, 
        i.e., with probability at least $1-\delta/2$, we have 
        \begin{align}
            - (k-1)\error_{\wt S_M}(f) \le \error_{\wt S_C}(f) - k\error_{\wt S}  + k\sqrt{\frac{\log(4/\delta)}{2m}}  \,. 
        \end{align} 
        \item Finally, use \eqref{eq:lemma3_final_multi} 
        to relate the error on the clean portion of randomly labeled data 
        and error on clean training data, 
        i.e., with probability $1-\delta/4$, we have 
        \begin{align}
            \error_{\wt \calS_C} (\wh f)\le - \error_{\calS}( \wh f) + \left(1 + \frac{m}{kn} \right) \sqrt{\frac{k\log(4/\delta)}{2m}} \,. 
        \end{align} 
    \end{enumerate}

    Using union bound on the above three steps, 
    we have with probability at least $1-\delta$: 
    \begin{align}
        \error_\calD (\wh f) \le \error_{\calS}(\wh f) + (k-1) - k\error_{\wt \calS}(\wh f)   + (\sqrt{k(k-1)} + k + \sqrt{k} + \frac{m}{n\sqrt{k}})  \sqrt{\frac{\log(4/\delta)}{2m}} \,.\label{eq:multiclass_ERM_final}
    \end{align}
    Simplifying the term in RHS of \eqref{eq:multiclass_ERM_final}, 
    we get the desired result. 
    % Note that since $\frac{m}{n\sqrt{k}}$ 
    % is much smaller than the sum of the other terms
    % the other terms in summation, 
    % we ignore $\frac{m}{n\sqrt{k}}$  
    % Z: ??? --- great
    % that 
    % them
    in the final bound. 
    % we ignore that in the final bound. 
    % Note that $(1/\sqrt{2} + 2.5)$ is a loose constant. In experiments, we use the ratio $\frac{m}{n}$
    %  the exact error $\error_{\wt \calS}(\wh f)$ 
    % to evaluate R.H.S.    
\end{proof}

\newpage
\section{Proofs from \secref{sec:linear_models}}\label{app:proof_gd}
We suppose that the parameters of the linear function 
are obtained via gradient descent on 
the following $L_2$ regularized problem: 
\begin{align}
    % n in denominator is avoided deliberately
    \calL_S(w; \lambda) \defeq \sum_{i=1}^n{(w^Tx_i - y_i)^2} + \lambda \norm{w}{2}^2 \,, \label{eq:l2_MSE_app}   
\end{align}
where $\lambda\ge0$ is a regularization parameter. 
We assume access to a clean dataset 
$S = \{(x_i, y_i)\}_{i=1}^n \sim \calD^n$ 
and randomly labeled dataset 
$\wt S = \{(x_i, y_i)\}_{i=n+1}^{n+m} \sim \wt \calD^m$. 
Let $\bX = [x_1, x_2, \cdots, x_{m+n}]$ 
and $\by = [y_1, y_2, \cdots, y_{m+n}]$. 
Fix a positive learning rate $\eta$ such that 
$\eta \le 1/\left(\norm{\bX^T\bX}{\text{op}} + \lambda^2\right)$ 
and an initialization $w_0 = 0$. 
% \todos{Assumption made for simplicty}. 
Consider the following gradient descent iterates 
to minimize objective \eqref{eq:l2_MSE_app} on $S \cup \wt S$:
\begin{align}
w_t = w_{t-1} - \eta \grad_w \calL_{S \cup \wt S} (w_{t-1}; \lambda) \quad \forall t=1,2,\ldots \label{eq:GD_iterates_app}
\end{align} 
Then we have $\{ w_t\}$ converge to the limiting solution 
$\wh w = \left( \bX^T\bX+\lambda \boldsymbol{I}\right)^{-1}\bX^T\by$. Define $\widehat f (x) \defeq f(x ; \wh w) $.  

% \subsection{\textcolor{red}{Errata}}

% We wish to correct the following error in the body:
% \codref{cond:error_stability} is not enough 
% to guarantee the result in \thmref{thm:linear}. 
% We now present a slightly stronger condition 
% called \emph{hypothesis stability} 
% under which we obtain a result 
% similar to \thmref{thm:linear}. 

% This error doesn't change the main arguments of the proof,
% where we show that the empirical train error 
% is less than or equal to the leave-one-out error.
% We need a stronger condition to relate leave-one-out error 
% with the population error of the original classifier. 
% Specifically, while \codref{cond:error_stability} 
% relates the average population error of leave-one-out classifiers 
% with the population error of the original classifier, 
% we need the new condition to show the concentration 
% of the empirical leave-one-out error 
% and average population error of leave-one-out classifiers. 
% main takeaway 

% Note that the new condition, 
% while being stronger than the previous one, 
% still doesn't imply generalization \citep{bousquet2002stability,elisseeff2003leave,abou2019exponential}. 
% Overall, the main results in \secref{sec:ERM_training} 
% and takeaways of the paper remain unaffected by the error.  

% We now present the new condition 
% and a corrected statement of \thmref{thm:linear}. 
% Recall, for a given training set $S \sim \calD^n $, 
% we use $S_{(i)}$ to denote the training set $S$ 
% with the $i^{\text{th}}$ point removed.

% \begin{condition}[Hypothesis Stability] 
%     \label{cond:hypothesis_stability}
%     We have $\beta$ hypothesis stability 
%     if our training algorithm $\calA$ satisfies the following: 
%     \begin{align*}
%     % ${\sum_{i=1}^n \frac{\error_{\calD}( f(\calA, S_{(i)}))}{n} - \error_\calD(f(\calA, S))} \le \beta\,$.
%     \forall i \in \{1,2,\ldots, n\}, \quad  \Expt{\calS, (x,y) \in \calD}{ \abs{\error\left( f(x) ,y  \right) - \error\left( f_{(i)}(x), y \right) }} \le \frac{\beta}{n} \,,
%     \end{align*}
%     where $f_{(i)} \defeq f(\calA, S_{(i)})$ and $ f \defeq f(\calA, S)$.
% \end{condition}

% \begin{theorem}[Correct statement of \thmref{thm:linear}] \label{thm:new_linear}
%     Assume that this gradient descent algorithm satisfies \codref{cond:hypothesis_stability}
%     with $\beta=\calO(1)$.  
%     Then for any $\delta >0$, with probability at least $1-\delta$ 
%     over the random draws of datasets $\wt S$ and $S$, we have:
%     \begin{align}
%         \error_\calD(\widehat f) \le \error_\calS(\widehat f) + 1 - 2 \error_{\wt\calS}(\widehat f) + \left(\frac{1}{\sqrt{2}} + 1.5 \right) \sqrt{\frac{\log(4/\delta)}{m}} + \sqrt{\frac{4}{\delta}\left(\frac{1}{m} +\frac{3\beta}{m+n} \right)}  \,. \label{eq:gd_error}
%     \end{align} 
%     % for some constant $c\le 3.2$.
% \end{theorem}

\subsection{Proof of \thmref{thm:linear}}
We use a standard result from linear algebra, 
namely the Shermann-Morrison formula 
\citep{sherman1950adjustment} for matrix inversion:  

\begin{lemma}[\citet{sherman1950adjustment}] \label{lem:sherman}
    Suppose $\bA \in \Real^{n \times n}$ 
    is an invertible square matrix 
    and $u,v \in \Real^n$ are column vectors. 
    Then $\bA + uv^T$ is invertible iff $1 + v^T \bA u \ne 0$ 
    and in particular
    \begin{align}
        (\bA + u v^T)^{-1} = \bA^{-1}  - \frac{\bA^{-1} uv^T \bA^{-1} }{ 1 + v^T \bA^{-1} u} \,.
    \end{align}   
\end{lemma}
\newcommand\byy[1]{\by_{\left(#1\right)}}
\newcommand\bXX[1]{\bX_{\left(#1\right)}}
\newcommand\ff[1]{\wh f_{\left(#1\right)}}

For a given training set $S \cup \wt S_C$, 
define leave-one-out error 
on mislabeled points in the training data 
as $$\error_{\text{LOO}(\wt S_M) } = \frac{\sum_{(x_i, y_i) \in \wt S_M} \error( f_{(i)}( x_i), y_i)}{ \abs{\wt S_M }} \,, $$
where $f_{(i)} \defeq f(\calA, (S \cup \wt S)_{(i)})$. 
To relate empirical leave-one-out error and population error 
with hypothesis stability condition, 
we use the following lemma:   

\begin{lemma}[\citet{bousquet2002stability}] \label{lem:stability_error}
    For the leave-one-out error, we have
    \begin{align}
        \Expo{ \left( \error_{\calDm}(\wh f) -\error_{\text{LOO}(\wt S_M) } \right)^2 } \le \frac{1}{2m_1}+  \frac{3\beta}{n + m}\,.
    \end{align}   
    % where $ f \defeq f(\calA, S \cup \wt S) $.
\end{lemma}

Proof of the above lemma is similar 
to the proof of Lemma 9 in \citet{bousquet2002stability} 
and can be found in \appref{app:proof_lem_error}. 
% 
% Before presenting the result, we introduce some notation. 
Before presenting the proof of \thmref{thm:linear}, 
we introduce some more notation. 
Let $\bX_{(i)}$ denote the matrix of covariates 
with the $i^{\text{th}}$ point removed. 
Similarly, let $\by_{(i)}$ be the array of responses 
with the $i^{\text{th}}$ point removed. 
Define the corresponding regularized GD solution 
as $\wh w_{(i)} = \left( \bXX{i}^T\bXX{i}+\lambda \boldsymbol{I}\right)^{-1}\bXX{i}^T\byy{i}$. 
Define $\ff{i}(x) \defeq f(x ; \wh w_{(i)}) $.

\begin{proof}[Proof of \thmref{thm:linear}]
    Because squared loss minimization does not imply 0-1 error minimization, 
    we cannot use arguments from \lemref{lem:fit_mislabeled}. 
    This is the main technical difficulty. 
    To compare the 0-1 error at a train point with an unseen point, 
    we use the closed-form expression for $\widehat{w}$ 
    and Shermann-Morrison formula 
    to upper bound training error 
    with leave-one-out cross validation error. 
    
    The proof is divided into three parts: 
    In part one, we show that 0-1 error 
    on mislabeled points in the training set 
    is lower than the error obtained 
    by leave-one-out error at those points. 
    In part two, we relate this leave-one-out error 
    with the population error on mislabeled distribution
    using \codref{cond:hypothesis_stability}.
    While the empirical leave-one-out error is an unbiased estimator 
    of the average population error of leave-one-out classifiers, 
    we need hypothesis stability 
    to control the variance 
    of empirical leave-one-out error. 
    Finally, in part three, we show 
    that the error on the mislabeled training points 
    can be estimated with just the randomly labeled 
    and clean training data (as in proof of \thmref{thm:error_ERM}).  

    \textbf{Part 1 {} {}} First we relate training error with leave-one-out error.        
    For any training point $(x_i, y_i)$ in $\wt S \cup S$, we have 
    \begin{align}
        \error(\wh f(x_i), y_i ) &= \indict{ y_i \cdot x_i^T \wh w < 0 } = \indict{ y_i \cdot x_i^T \left( \bX^T\bX+\lambda \boldsymbol{I}\right)^{-1}\bX^T\by < 0 } \\
        &= \indict{ y_i \cdot x_i^T \underbrace{\left( \bXX{i}^T\bXX{i} + x_i ^T x_i +\lambda \boldsymbol{I}\right)^{-1}}_{\RN{1}} (\bXX{i}^T\byy{i} + y_i \cdot x_i) < 0 } \,.
    \end{align}
    Letting $\bA = \left(\bXX{i}^T\bXX{i} +\lambda \boldsymbol{I}\right)$ 
    and using \lemref{lem:sherman} on term 1, we have 
    \begin{align}
        \error(\wh f(x_i), y_i ) &= \indict{ y_i \cdot x_i^T \left[\bA^{-1} -  \frac{\bA^{-1} x_i x_i^T \bA^{-1}}{ 1 + x_i ^T \bA^{-1} x_i } \right] (\bXX{i}^T\byy{i} + y_i \cdot x_i) < 0 } \\
        &= \indict{ y_i \cdot\left[ \frac{ x_i^T \bA^{-1} ( 1 + x_i ^T \bA^{-1} x_i ) -  x_i^T \bA^{-1} x_i x_i^T \bA^{-1}}{ 1 + x_i ^T \bA ^{-1}x_i } \right] (\bXX{i}^T\byy{i} + y_i \cdot x_i) < 0 } \\
        &= \indict{ y_i \cdot\left[ \frac{ x_i^T \bA^{-1}}{ 1 + x_i ^T \bA ^{-1}x_i } \right] (\bXX{i}^T\byy{i} + y_i \cdot x_i) < 0 } \,.
    \end{align}

    Since $1 + x_i^T \bA^{-1} x_i > 0$, we have 
    \begin{align}
        \error(\wh f(x_i), y_i ) &= \indict{ y_i \cdot x_i^T \bA^{-1} (\bXX{i}^T\byy{i} + y_i \cdot x_i) < 0 } \\
        &= \indict{ x_i^T \bA^{-1} x_i +  y_i \cdot x_i^T \bA^{-1} (\bXX{i}^T\byy{i}) < 0 } \\
        &\le \indict{ y_i \cdot x_i^T \bA^{-1} (\bXX{i}^T\byy{i}) < 0 } = \error(\ff{i}(x_i), y_i ) \,.\label{eq:LOO_error}
    \end{align}

    Using \eqref{eq:LOO_error}, we have 
    \begin{align}
        \error_{\wt \calS_M } (\wh f) \le \error_{\text{LOO} (\wt S_M)} \defeq \frac{\sum_{(x_i, y_i) \in \wt S_M} \error(\ff{i}(x_i), y_i ) }{\abs{\wt \calS_M}}\label{eq:LOO_error_final} \,.
    \end{align}
    \textbf{Part 2 {}{}} We now relate RHS in \eqref{eq:LOO_error_final} 
    with the population error on mislabeled distribution. 
    To do this, we leverage \codref{cond:hypothesis_stability} 
    and \lemref{lem:stability_error}. 
    In particular, we have 

    \begin{align}
        \Expt{\calS \cup \wt \calS_M }{ \left(\error_{\calDm}(\wh f) - \error_{\text{LOO} (\wt S_M)}\right)^2 } \le \frac{1}{2m_1} + \frac{3\beta}{m+n} \,.
    \end{align}

    Using Chebyshev's inequality, with probability at least $1-\delta$, we have 
    \begin{align}
        \error_{\text{LOO} (\wt S_M)} \le  \error_{\calDm}(\wh f)   + \sqrt{\frac{1}{\delta}\left(\frac{1}{2m_1} +\frac{3\beta}{m+n} \right)} \,. \label{eq:final_mislabeled_linear}
    \end{align}
    

    \textbf{Part 3 {}{}} Combining \eqref{eq:final_mislabeled_linear} and \eqref{eq:LOO_error_final}, we have 

    \begin{align}
        \error_{\wt \calS_M } (\wh f) \le \error_{\calDm}(\wh f)   + \sqrt{\frac{1}{\delta}\left(\frac{1}{2m_1} +\frac{3\beta}{m+n} \right)} \,. \label{eq:linear_parallel_lem1}
    \end{align}

    Compare \eqref{eq:linear_parallel_lem1} with \eqref{eq:lemma1_final} 
    in the proof of \lemref{lem:fit_mislabeled}. 
    We obtain a similar relationship 
    between $\error_{\wt \calS_M }$ and $\error_{\calDm}$ 
    but with a polynomial concentration 
    instead of exponential concentration. 
    In addition, since we just use concentration arguments 
    to relate mislabeled error to the errors
    on the clean and unlabeled portions 
    of the randomly labeled data, 
    we can directly use the results 
    in \lemref{lem:mislabeled_error} and \lemref{lem:clear_error}. 
    Therefore, combining results in \lemref{lem:mislabeled_error}, \lemref{lem:clear_error}, and \eqref{eq:linear_parallel_lem1} with union bound, 
    we have with probability at least $1-\delta$
    \begin{align}
        \error_\calD(\widehat f) \le \error_\calS(\widehat f) + 1 - 2 \error_{\wt\calS}(\widehat f) + \left(\sqrt{2}\error_{\wt\calS}(\widehat f) + 1 + \frac{m}{2n} \right) \sqrt{\frac{\log(4/\delta)}{m}} + \sqrt{\frac{4}{\delta}\left(\frac{1}{m} +\frac{3\beta}{m+n} \right)}  \,.
    \end{align}
    

       
\end{proof}

\subsection{Extension to multiclass classification} \label{app:multiclass_linear}
For multiclass problems with squared loss minimization, as standard practice, we consider one-hot encoding for the underlying label, i.e., a class label $c \in [k]$ is treated as $(0, \cdot, 0,1,0, \cdot, 0) \in \Real^k$ (with $c$-th coordinate being 1).  As before, we suppose that the parameters of the linear function 
are obtained via gradient descent on the following $L_2$ regularized problem: 
\begin{align}
    % n in denominator is avoided deliberately
    \calL_S(w; \lambda) \defeq \sum_{i=1}^n\norm{w^Tx_i - y_i}{2}^2 + \lambda \sum_{j=1}^k \norm{w_j}{2}^2 \,, \label{eq:l2_multiclass_MSE_app}   
\end{align}
where $\lambda\ge0$ is a regularization parameter. 
We assume access to a clean dataset 
$S = \{(x_i, y_i)\}_{i=1}^n \sim \calD^n$ 
and randomly labeled dataset 
$\wt S = \{(x_i, y_i)\}_{i=n+1}^{n+m} \sim \wt \calD^m$. 
Let $\bX = [x_1, x_2, \cdots, x_{m+n}]$ 
and $\by = [e_{y_1}, e_{y_2}, \cdots, e_{y_{m+n}}]$. 
Fix a positive learning rate $\eta$ such that 
$\eta \le 1/\left(\norm{\bX^T\bX}{\text{op}} + \lambda^2\right)$ 
and an initialization $w_0 = 0$. 
% \todos{Assumption made for simplicty}. 
Consider the following gradient descent iterates 
to minimize objective \eqref{eq:l2_MSE_app} on $S \cup \wt S$:
\begin{align}
{w_j}^t = {w_j}^{t-1} - \eta \grad_{w_j} \calL_{S \cup \wt S} (w^{t-1}; \lambda) \quad \forall t=1,2,\ldots \text{ and } j=1,2,\ldots,k  \,. \label{eq:GD_multi_iterates_app}
\end{align} 
Then we have $\{ {w_j}^t\}$ for all $j =1,2,\cdots, k$ converge to the limiting solution 
$\wh w_j = \left( \bX^T\bX+\lambda \boldsymbol{I}\right)^{-1}\bX^T\by_j$. Define $\widehat f (x) \defeq f(x ; \wh w) $.  

\begin{theorem}\label{thm:multi_linear}
    Assume that this gradient descent algorithm satisfies \codref{cond:hypothesis_stability}
    with $\beta=\calO(1)$.  
    Then for a multiclass classification problem wth $k$ classes, for any $\delta >0$, with probability at least $1-\delta$, we have:
    \begin{align*}
        \error_\calD(\widehat f) \le \error_\calS(\widehat f) &+ (k-1)\left(1 - \frac{k}{k-1} \error_{\wt\calS}(\widehat f) \right) \\ &+ \left(k + \sqrt{k} + \frac{m}{n\sqrt{k}} \right) \sqrt{\frac{\log(4/\delta)}{2m}} + \sqrt{k(k-1)} \sqrt{\frac{4}{\delta}\left(\frac{1}{m} +\frac{3\beta}{m+n} \right)}  \,. \numberthis \label{eq:gd_multi_error}
    \end{align*} 
    % for some constant $c\le 3.2$.
\end{theorem}
\begin{proof}
    The proof of this theorem is divided into two parts. In the first part, we relate the error on the mislabeled samples with the population error on the mislabeled data. Similar to the proof of \thmref{thm:linear}, we use Shermann-Morrison formula to upper bound training error with leave-one-out error on each $\wh w^j$. Second part of the proof follows entirely from the proof of \thmref{thm:multiclass_ERM}. In essence, the first part derives an equivalent of \eqref{eq:lemma1_final_multi_prev} for GD training with squared loss and then the second part follows from the proof  of \thmref{thm:multiclass_ERM}. 
    
    \textbf{Part-1:} Consider a training point $(x_i,y_i)$ in $\wt S \cup S $. For simplicity, we use $c_i$ to denote the class of $i$-th point and use $y_i$ as the corresponding one-hot embedding. Recall error in multiclass point is given by $\error(\wh f(x_i), y_i ) = \indict{ c_i \not \in \argmax x_i^T \wh w }$. Thus, there exists a $j \ne c_i \in [k]$, such that we have
     \begin{align}
        \error(\wh f(x_i), y_i ) &= \indict{ c_i \not \in \argmax x_i^T \wh w } = \indict{ x_i^T \wh w_{c_i} < x_i^T \wh w_{j}  } \\ &= \indict{ x_i^T \left( \bX^T\bX+\lambda \boldsymbol{I}\right)^{-1}\bX^T\by_{c_i} < x_i^T \left( \bX^T\bX+\lambda \boldsymbol{I}\right)^{-1}\bX^T\by_{j} } \\
        &= \indict{ x_i^T \underbrace{\left( \bXX{i}^T\bXX{i} + x_i ^T x_i +\lambda \boldsymbol{I}\right)^{-1}}_{\RN{1}} \left(\bXX{i}^T{\by_{c_i}}_{(i)} + x_i - \bXX{i}^T{\by_{j}}_{(i)}\right) < 0 } \,.
    \end{align}
    Letting $\bA = \left(\bXX{i}^T\bXX{i} +\lambda \boldsymbol{I}\right)$ 
    and using \lemref{lem:sherman} on term 1, we have 
    \begin{align}
        \error(\wh f(x_i), y_i ) &= \indict{ x_i^T \left[\bA^{-1} -  \frac{\bA^{-1} x_i x_i^T \bA^{-1}}{ 1 + x_i ^T \bA^{-1} x_i } \right]  \left(\bXX{i}^T{\by_{c_i}}_{(i)} + x_i - \bXX{i}^T{\by_{j}}_{(i)}\right) < 0 } \\
        &= \indict{ \left[ \frac{ x_i^T \bA^{-1} ( 1 + x_i ^T \bA^{-1} x_i ) -  x_i^T \bA^{-1} x_i x_i^T \bA^{-1}}{ 1 + x_i ^T \bA ^{-1}x_i } \right]  \left(\bXX{i}^T{\by_{c_i}}_{(i)} + x_i - \bXX{i}^T{\by_{j}}_{(i)}\right) < 0 } \\
        &= \indict{ \left[ \frac{ x_i^T \bA^{-1}}{ 1 + x_i ^T \bA ^{-1}x_i } \right]  \left(\bXX{i}^T{\by_{c_i}}_{(i)} + x_i - \bXX{i}^T{\by_{j}}_{(i)}\right) < 0} \,.
    \end{align}
    Since $1 + x_i^T \bA^{-1} x_i > 0$, we have 
    \begin{align}
        \error(\wh f(x_i), y_i ) &= \indict{ x_i^T \bA^{-1}  \left(\bXX{i}^T{\by_{c_i}}_{(i)} + x_i - \bXX{i}^T{\by_{j}}_{(i)}\right) < 0 } \\
        &= \indict{ x_i^T \bA^{-1} x_i +  x_i^T \bA^{-1}  \bXX{i}^T{\by_{c_i}}_{(i)}  - x_i^T\bA^{-1}  \bXX{i}^T{\by_{j}}_{(i)} < 0 } \\
        &\le \indict{  x_i^T \bA^{-1}  \bXX{i}^T{\by_{c_i}}_{(i)}  - x_i^T\bA^{-1}  \bXX{i}^T{\by_{j}}_{(i)} < 0  } = \error(\ff{i}(x_i), y_i ) \,.\label{eq:LOO_error_multi}
    \end{align}
    Using \eqref{eq:LOO_error_multi}, we have 
    \begin{align}
        \error_{\wt \calS_M } (\wh f) \le \error_{\text{LOO} (\wt S_M)} \defeq \frac{\sum_{(x_i, y_i) \in \wt S_M} \error(\ff{i}(x_i), y_i ) }{\abs{\wt \calS_M}}\label{eq:LOO_error_multi_final} \,.
    \end{align}
    
    We now relate RHS in \eqref{eq:LOO_error_final} 
    with the population error on mislabeled distribution. 
    Similar as before, to do this, we leverage \codref{cond:hypothesis_stability} 
    and \lemref{lem:stability_error}. Using  \eqref{eq:final_mislabeled_linear} and \eqref{eq:LOO_error_multi_final}, we have 
    \begin{align}
        \error_{\wt \calS_M } (\wh f) \le \error_{\calDm}(\wh f)   + \sqrt{\frac{1}{\delta}\left(\frac{1}{2m_1} +\frac{3\beta}{m+n} \right)} \,. \label{eq:linear_multi_parallel_lem1}
    \end{align}
    
    We have now derived a parallel to \eqref{eq:lemma1_final_multi_prev}. Using the same arguments in the proof of \lemref{lem:fit_mislabeled_multi}, we have 
    \begin{align}
      \error_{\calD}(\wh f) \le  (k-1) \left( 1- \error_{ \wt \calS_M}(\wh f) \right)  + (k-1)\sqrt{\frac{k}{\delta(k-1)}\left(\frac{1}{2m_1} +\frac{3\beta}{m+n} \right)}  \,. \label{eq:lemma1_linear_final_multi}
    \end{align}
    
    \textbf{Part-2:} We now combine the results in \lemref{lem:mislabeled_error_multi} and \lemref{lem:clear_error_multi} to obtain the final inequality in terms of quantities that can be computed from just the randomly labeled and clean data. Similar to the binary case, we obtained a polynomial concentration instead of exponential concentration. Combining \eqref{eq:lemma1_linear_final_multi} with \lemref{lem:mislabeled_error_multi} and \lemref{lem:clear_error_multi}, we have with probability at least $1-\delta$
    \begin{align*}
        \error_\calD(\widehat f) \le \error_\calS(\widehat f) &+ (k-1)\left(1 - \frac{k}{k-1} \error_{\wt\calS}(\widehat f) \right) \\ &+ \left(k + \sqrt{k} + \frac{m}{n\sqrt{k}} \right) \sqrt{\frac{\log(4/\delta)}{2m}} + \sqrt{k(k-1)} \sqrt{\frac{4}{\delta}\left(\frac{1}{m} +\frac{3\beta}{m+n} \right)}  \,. \numberthis \label{eq:gd_multi_error_proof}
    \end{align*} 
\end{proof}

\subsection{Discussion on \codref{cond:hypothesis_stability}} \label{app:discuss_cond1}
The quantity in LHS of \codref{cond:hypothesis_stability} 
measures how much the function learned by the algorithm 
(in terms of error on unseen point) will change 
when one point in the training set is removed. 
% Discussion on exponential concentration and stronger condition. 
% Notice that hypothesis stability implies error stability, i.e., \codref{cond:error_stability} \citep{bousquet2002stability}.  
% In summary, while error stability allowed us 
% to relate the average population error 
% of the leave-one-out classifiers 
% with the population error of the original classifier, 
We need hypothesis stability condition 
to control the variance of the empirical leave-one-out error to show concentration of average leave-one-error with the population error. 

Additionally, we note that while the dominating term in the RHS of \thmref{thm:linear} matches with the dominating term in ERM bound in \thmref{thm:error_ERM}, there is a polynomial concentration term 
(dependence on $1/\delta$ instead of $\log(\sqrt{1/\delta})$) 
in \thmref{thm:linear}. 
Since with hypothesis stability, 
we just bound the variance, 
the polynomial concentration is due 
to the use of Chebyshev's inequality 
instead of an exponential tail inequality
(as in \lemref{lem:fit_mislabeled}).
Recent works have highlighted that 
a slightly stronger condition than hypothesis stability 
can be used to obtain an exponential concentration 
for leave-one-out error \citep{abou2019exponential},
but we leave this for future work for now. 
% We leave 
% However, the constants 

% we also want to highlight  

\subsection{Formal statement and proof of \propref{prop:early_stop}} \label{app:formal_early_stop}

Before formally presenting the result, 
we will introduce some notation.  
By $\calL_{S}(w)$, we denote 
the objective in \eqref{eq:l2_MSE_app} with $\lambda=0$. 
Assume Singular Value Decomposition (SVD) of $\bX$
as $\sqrt{n} \bU \bS^{1/2} \bV^T$. 
Hence $\bX^T \bX = \bV \bS \bV^T$.
Consider the GD iterates defined in \eqref{eq:GD_iterates_app}. 
% 
We now derive closed form expression 
for the $t^\text{th}$ iterate of gradient descent:  
% 
\begin{align}
    w_t = w_{t-1} + \eta \cdot \bX^T (\by - \bX w_{t-1}) = (\bI - \eta \bV \bS \bV^T )w_{k-1} + \eta \bX^T \by \,.
\end{align}
Rotating by $\bV^T$, we get 
\begin{align}
    \wt w_t = (\bI - \eta\bS )\wt w_{k-1} + \eta \wt \by \label{eq:GD_recur},
\end{align}
where $\wt w_t = \bV^T w_t $ and $\wt \by = \bV^T \bX^T \by$. 
Assuming the initial point $w_0 = 0$ 
and applying the recursion in \eqref{eq:GD_recur}, we get
\begin{align}
    \wt w_t = \bS ^{-1} ( \bI - (\bI - \eta \bS)^k ) \wt \by \,, 
\end{align} 
Projecting solution back to the original space, we have 
\begin{align}
     w_t = \bV \bS ^{-1} ( \bI - (\bI - \eta \bS)^k ) \bV^T \bX^T \by \,. 
\end{align} 
% We will work with this GD solution at any iterate $t$ in the next proposition. 
Define $f_t(x) \defeq f(x;w_t)$ 
as the solution at the $t^{\text{th}}$ iterate. 
Let $\wt w_{\lambda} = \argmin_{w} \calL_\calS (w;\lambda) = (\bX^T \bX + \lambda \bI)^{-1} \bX^T \by = \bV (\bS + \lambda \bI )^{-1} \bV^T \bX^T \by $. 
% ) \,,$ for all $t=1,2,\ldots\,.$ 
and define $\wt f_\lambda(x) \defeq f(x;\wt w_\lambda)$ as the regularized solution. 
Assume $\kappa$ be the condition number 
of the population covariance matrix 
and let $s_\text{min}$ be the minimum positive 
singular value of the empirical covariance matrix. 
Our proof idea is inspired from recent work 
on relating gradient flow solution 
and regularized solution 
for regression problems \citep{ali2018continuous}. 
We will use the following lemma in the proof: 
\begin{lemma} \label{lem:ineq_soln}
    For all $x \in [0,1]$ and for all $ k \in \mathbb{N}$, 
    we have (a) $ \frac{kx}{1+kx} \le 1- (1-x)^k$ 
    and (b) $ 1- (1-x)^k \le 2 \cdot \frac{kx}{kx+1} $.
    %  where $g(c)$ is a constant dependent on $c$. For $c = 1$, $g(c) = 2.0$.   
\end{lemma}
\begin{proof}
    % [Proof of \lemref{lem:ineq_soln}]
    % Part (a) is easy. 
    Using $ (1-x)^k \le \frac{1}{1+kx}$, we have part (a). 
    For part (b), we numerically maximize 
    $\frac{ (1+kx ) (1 - (1-x)^k) }{kx}$ 
    for all $k\ge 1$ and for all $x \in [0, 1]$.  
\end{proof}

% 
% Next, 

\begin{prop}[Formal statement of \propref{prop:early_stop}] \label{prop:formal_early_stop}
Let $\lambda = \frac{1}{t\eta}$. 
For a training point $x$, we have 
\begin{align*}
    \Expt{x \sim \calS}{(f_t(x) - \wt f_\lambda(x))^2} &\le c(t,\eta) \cdot \Expt{x \sim \calS}{f_t(x)^2} \,, %\label{eq:early_stop}
\end{align*}
where $c(t, \eta) \defeq \min( 0.25, \frac{1}{s_\text{min}^2 t^2 \eta^2})$. 
Similarly for a test point, we have 
\begin{align*}
    \Expt{x \sim \calD_\calX}{(f_t(x) - \wt f_\lambda(x))^2} &\le \kappa \cdot c(t,\eta) \cdot \Expt{x \sim \calD_\calX}{f_t(x)^2} \,. %\label{eq:early_stop}
\end{align*}
\end{prop} 

\begin{proof}
    %%%%%%%%%%%%% 
    We want to analyze the expected squared difference output 
    of regularized linear regression 
    with regularization constant $\lambda = \frac{1}{\eta t}$ 
    and the gradient descent solution at the $t^\text{th}$ iterate. 
    We separately expand the algebraic expression 
    for squared difference at a training point and a test point. 
    % We start by considering the difference  
    Then the main step is to show that 
    $\left[ \bS ^{-1} ( \bI - (\bI - \eta \bS)^k )  - (\bS + \lambda \bI )^{-1}\right] \preceq c(\eta, t) \cdot \bS ^{-1} ( \bI - (\bI - \eta \bS)^k ) $.

    %%%%%%%%%%%%%
    
   \textbf{Part 1 {} {}} 
    First, we will analyze the squared difference 
    of the output at a training point 
    (for simplicity, we refer to $S \cup \wt S$ as $S$), i.e., 
    \begin{align}
        \Expt{ x \sim \calS }{\left(f_t(x) - \wt f_\lambda (x)\right)^2} &= \norm{\bX w_t - \bX \wt w_\lambda}{2}^2\\ &=   \norm{\bX \bV \bS ^{-1} ( \bI - (\bI - \eta \bS)^t ) \bV^T \bX^T \by - \bX \bV (\bS + \lambda \bI )^{-1} \bV^T \bX^T \by }{2}^2 \\
        &= \norm{\bX \bV \left(\bS ^{-1} ( \bI - (\bI - \eta \bS)^t ) - (\bS + \lambda \bI )^{-1} \right) \bV^T \bX^T \by  }{2} \\
        &=  \by^T \bV \bX \left( \underbrace{\bS ^{-1} ( \bI - (\bI - \eta \bS)^t ) - (\bS + \lambda \bI )^{-1}}_{\RN{1}} \right)^2 \bS \bV^T \bX^T \by \label{eq:train_GD_rel} \,.
        %  (\bX \bV \bS ^{-1} ( \bI - (\bI - \eta \bS)^k ) \bV^T \bX^T \by)^T \bX \bV \bS ^{-1} ( \bI - (\bI - \eta \bS)^k ) \bV^T \bX^T \by
    \end{align}
    We now separately consider term 1. 
    Substituting $\lambda = \frac{1}{t \eta}$, 
    we get
    \begin{align}
        \bS ^{-1} ( \bI - (\bI - \eta \bS)^t ) - (\bS + \lambda \bI )^{-1} &= \bS^{-1} \left( ( \bI - (\bI - \eta \bS)^t ) - (\bI + \bS^{-1} \lambda )^{-1}\right) \\
        &= \underbrace{\bS^{-1} \left( ( \bI - (\bI - \eta \bS)^t ) - (\bI + ( \bS t \eta)^{-1}  )^{-1}\right)}_{\bA} \,.
    \end{align}

    We now separately bound the diagonal entries in matrix $\bA$. 
    With $s_i$, we denote $i^{\text{th}}$ diagonal entry of $\bS$.
    Note that since $ \eta\le 1/\norm{S}{\text{op}}$, 
    for all $i$, $\eta s_i  \le 1$.  
    Consider $i^{\text{th}}$ diagonal term (which is non-zero) 
    of the diagonal matrix $\bA$, we have 
    \begin{align}
        \bA_{ii} = \frac{1}{s_i} \left(  1 - (1 - s_i \eta)^t - \frac{t \eta s_i}{1 + t \eta s_i } \right) &=  \frac{1 - (1 - s_i \eta)^t}{s_i} \left( \underbrace{ 1 - \frac{t \eta s_i}{(1 + t \eta s_i)(1 - (1 - s_i \eta)^t)}}_{\RN{2}} \right) \\ 
         &\le \frac{1}{2}\left[ \frac{1 - (1 - s_i \eta)^t}{ s_i} \right] \tag*{(Using \lemref{lem:ineq_soln} (b))} \,.
    \end{align} 
    Additionally, we can also show the following upper bound on term 2: 
    \begin{align}
         1 - \frac{t \eta s_i}{(1 + t \eta s_i)(1 - (1 - s_i \eta)^t)} &= \frac{(1 + t \eta s_i)(1 - (1 - s_i \eta)^t) - t \eta s_i }{(1 + t \eta s_i)(1 - (1 - s_i \eta)^t)} \\
         & \le  \frac{ 1 -  (1 - s_i \eta)^t - t \eta s_i (1 - s_i \eta)^t}{(1 + t \eta s_i)(1 - (1 - s_i \eta)^t)} \\
         & \le \frac{1}{t\eta s_i} \,. \tag{Using \lemref{lem:ineq_soln} (a)}
        %  &\le \frac{1}{2}\left[ \frac{1 - (1 - s_i \eta)^t}{ s_i} \right] \tag*{(Using \lemref{lem:ineq_soln})} \,.
    \end{align} 

    Combining both the upper bounds 
    on each diagonal entry $\bA_{ii}$, we have 
    \begin{align}
    \bA \preceq c_1(\eta, t) \cdot \bS^{-1} ( \bI - (\bI - \eta \bS)^t ) \,, \label{eq:upperbound_diagonal}
    \end{align}
    where $c_1(\eta, t ) = \min(0.5, \frac{1}{t s_i \eta })$. Plugging this into \eqref{eq:train_GD_rel}, we have 
    \begin{align}
        \Expt{ x \sim \calS }{\left(f_t(x) - \wt f_\lambda (x)\right)^2} &\le c(\eta, t) \cdot \by^T \bV \bX  \left( \bS^{-1} ( \bI - (\bI - \eta \bS)^t ) \right)^2 \bS \bV^T \bX^T \by \\
        &=   c(\eta, t) \cdot \by^T \bV \bX  \left( \bS^{-1} ( \bI - (\bI - \eta \bS)^t ) \right) \bS \left( \bS^{-1} ( \bI - (\bI - \eta \bS)^t ) \right) \bV^T \bX^T \by \\
        & =  c(\eta, t) \cdot \norm{\bX w_t}{2}^2 \\
        &= c(\eta, t) \cdot  \Expt{ x \sim \calS }{\left(f_t(x) \right)^2} \,,
    \end{align}
    where $c(\eta, t ) = \min(0.25, \frac{1}{t^2 s^2_i \eta^2 })$.

    \textbf{Part 2 {} {}} With $\bSigma$, 
    we denote the underlying true covariance matrix. 
    We now consider the squared difference of output at an unseen point: 
    \begin{align}
        \Expt{ x \sim \calD_{\calX} }{\left(f_t(x) - \wt f_\lambda (x)\right)^2} &= \Expt{x \sim \calD_{\calX}}{\norm{x^T w_t - x^T \wt w_\lambda}{2}} \\
        &=   \norm{x^T \bV \bS ^{-1} ( \bI - (\bI - \eta \bS)^t ) \bV^T \bX^T \by - x^T \bV (\bS + \lambda \bI )^{-1} \bV^T \bX^T \by }{2} \\
        &= \norm{x^T \bV \left(\bS ^{-1} ( \bI - (\bI - \eta \bS)^t ) - (\bS + \lambda \bI )^{-1} \right) \bV^T \bX^T \by  }{2} \\
        &= \by^T \bV \bX \left( \bS ^{-1} ( \bI - (\bI - \eta \bS)^t ) - (\bS + \lambda \bI )^{-1} \right) \bV^T \bSigma \bV \\ &\qquad \qquad \qquad \qquad \qquad \left( (\bI - (\bI - \eta \bS)^t ) - (\bS + \lambda \bI )^{-1} \right) \bV^T \bX^T \by \\
        &\le \sigma_{\text{max}} \cdot \by^T \bV \bX \left( \underbrace{\bS ^{-1} ( \bI - (\bI - \eta \bS)^t ) - (\bS + \lambda \bI )^{-1}}_{\RN{1}} \right)^2 \bV^T \bX^T \by \,, \label{eq:test_GD_rel}
        %  (\bX \bV \bS ^{-1} ( \bI - (\bI - \eta \bS)^k ) \bV^T \bX^T \by)^T \bX \bV \bS ^{-1} ( \bI - (\bI - \eta \bS)^k ) \bV^T \bX^T \by
    \end{align}
    where $\sigma_{\text{max}}$ is the maximum eigenvalue 
    of the underlying covariance matrix $\bSigma$. 
    Using the upper bound on term 1 in \eqref{eq:upperbound_diagonal}, 
    we have 
    \begin{align}
        \Expt{ x \sim \calD_{\calX} }{\left(f_t(x) - \wt f_\lambda (x)\right)^2} &\le \sigma_{\text{max}} \cdot c(\eta, t) \cdot \by^T \bV \bX  \left( \bS^{-1} ( \bI - (\bI - \eta \bS)^t ) \right)^2 \bV^T \bX^T \by \\
        &=   \kappa \cdot c(\eta, t) \cdot \sigma_{\text{min}}\cdot \norm{\bV \left( \bS^{-1} ( \bI - (\bI - \eta \bS)^t ) \right) \bV^T \bX^T \by}{2}^2 \\
        &\le \kappa \cdot c(\eta, t) \cdot \left[ \bV \left( \bS^{-1} ( \bI - (\bI - \eta \bS)^t ) \right) \bV^T \bX^T \right]^T \bSigma \\
        &\qquad \qquad \qquad \qquad \qquad \left[ \bV \left( \bS^{-1} ( \bI - (\bI - \eta \bS)^t ) \right) \bV^T \bX^T \right] \by \\
        & = \kappa \cdot c(\eta, t) \cdot \Expt{x \sim \calD_{\calX}}{\norm{x^T w_t}{2}} \,.
    \end{align}
% 
% 
    % Since $ \eta\le 1/\norm{S}{\text{op}}$, invoking \lemref{lem:ineq_soln} to upper bound term 1 with
\end{proof}

\subsection{Extension to deep learning} \label{appsubsec:ext_DL}
Under \asmpref{appsubsec:justifying_assumption1}, we present the formal result parallel to \thmref{thm:multiclass_ERM}. 
\begin{theorem} \label{thm:multiclass_ERM_algoA}
    Consider a multiclass classification problem 
    with $k$ classes. Under \asmpref{asmp:deep_models}, 
    for any $\delta >0$, with probability at least $1-\delta$,
    we have
    \vspace{-10pt}
    \begin{align*}
        \error_\calD(\widehat f)  \le \error_\calS(\widehat f) + (k-1) \left(1 - \tfrac{k}{k-1} \error_{\wt\calS}(\widehat f)\right) + c\sqrt{\frac{\log(\frac{4}{\delta})}{2m}} \,,\numberthis \label{eq:multiclass_ERM_deep}
    % \vspace{-20pt}
    \end{align*}
    for some constant $c \le ((c+1) k+\sqrt{k} + \frac{m}{n\sqrt{k}})$.
\end{theorem}

The proof follows exactly as in step (i) to (iii) in \thmref{thm:multiclass_ERM}.  

\subsection{Justifying~\asmpref{asmp:deep_models}} \label{appsubsec:justifying_assumption1}

Motivated by the analysis on linear models, we now discuss alternate (and weaker) conditions that imply \asmpref{asmp:deep_models}. 
We need hypothesis stability (\codref{cond:hypothesis_stability}) and the following assumption relating training error and leave-one-error: 

\begin{assumption} \label{asmp:loo_error}
Let $\wh f$ be a model obtained by training with algorithm $\calA$ on a mixture of clean $S$ and randomly labeled data $\wt S$. Then we assume we have 
\begin{align*}
    \error_{\wt \calS_M} (\wh f) \le  \error_{\text{LOO} (\wt S_M)} \,, 
\end{align*}
for all $(x_i, y_i) \in  \wt S_M$ where $\wh f_{(i)} \defeq f(\calA, S \cup {{}\wt S_M}_{(i)})$ and  $\error_{\text{LOO} (\wt S_M)} \defeq  \frac{\sum_{(x_i, y_i) \in \wt S_M} \error(\ff{i}(x_i), y_i ) }{\abs{\wt \calS_M}}$.  
\end{assumption}

% we assume this to extend our result (parallel to \thmref{thm:multi_linear}) for deep models. 
Intuitively, this assumption states that the error on a (mislabeled) datum $(x,y)$ included in the training set is less than the error on that datum $(x,y)$ obtained by a model trained on the training set $S - \{(x,y)\}$. We proved this for linear models trained with GD in the proof of \thmref{thm:multi_linear}. 
% 
\codref{cond:hypothesis_stability} with $\beta = \calO(1)$ and \asmpref{asmp:loo_error} together with \lemref{lem:stability_error} implies \asmpref{asmp:deep_models} with a polynomial residual term (instead of logarithmic in $1/\delta$): 
\begin{align}
     \error_{\calS_M} (\wh f) \le  \error_{\calDm}(\wh f)   + \sqrt{\frac{1}{\delta}\left(\frac{1}{m} +\frac{3\beta}{m+n} \right)} \,.
\end{align}
% Note that this  

\newpage 
\section{Additional experiments and details}\label{app:exp}
\newcommand\tab[1][1cm]{\hspace*{#1}}

\subsection{Datasets} \label{sec:app_dataset}

\textbf{Toy Dataset {} {}} Assume fixed constants $\mu$ and $\sigma$. For a given label $y$, we simulate features $x$ in our toy classification setup as follows: 
\begin{align*}
    x \defeq \texttt{concat} \left[ x_1, x_2\right] \quad \text{where} \quad  x_1 \sim  \calN( y \cdot \mu, \sigma^2 I_{d \times d}) \ \  \text{and} \ \  x_1 \sim  \calN( 0, \sigma^2 I_{d \times d}) \,.
\end{align*}  
% where $y$ is the true label and $x$ is the corresponding feature vector. 
In experiements throughout the paper, we fix dimention $d=100$, $\mu = 1.0 $, and $\sigma = \sqrt{d}$. Intuitively, $x_1$ carries the information about the underlying label and $x_2$ is additional noise independent of the underlying label. 

\textbf{CV datasets {} {}} We use MNIST~\citep{lecun1998mnist} and CIFAR10~\cite{krizhevsky2009learning}. 
% For binary tasks, 
We produce a binary variant from the multiclass classification problem by mapping classes $\{0,1,2,3,4\}$ to label $1$ and $\{ 5,6,7,8,9\}$ to label $-1$. For CIFAR dataset, we also use the standard data augementation of random crop and horizontal flip. PyTorch code is as follows: 

\texttt{(transforms.RandomCrop(32, padding=4),\\
\tab transforms.RandomHorizontalFlip())}

\textbf{NLP dataset {} {}} We use IMDb Sentiment analysis~\citep{maas2011learning} corpus.  

\subsection{Architecture Details} 

All experiments were run on NVIDIA GeForce RTX 2080 Ti GPUs. We used PyTorch~\citep{NEURIPS2019a9015} and Keras with Tensorflow~\citep{abadi2016tensorflow} backend for experiments. 
% , ELMo embeddings~\citep{Peters:2018}, and Hugging Face Transformers~\citep{wolf-etal-2020-transformers}. 

\textbf{Linear model {} {}} For the toy dataset, we simulate a linear model with scalar output and the same number of parameters as the number of dimensions.   

\textbf{Wide nets {} {}} To simulate the NTK regime, we experiment with $2-$layered wide nets. The PyTorch code for 2-layer wide MLP is as follows: 


\texttt{ nn.Sequential( \\
\tab     nn.Flatten(),\\
\tab    nn.Linear(input\_dims, 200000, bias=True),\\
\tab    nn.ReLU(),\\
\tab    nn.Linear(200000, 1, bias=True)\\
\tab     )}


We experiment both (i) with the second layer fixed at random initialization; (ii)  and updating both layers' weights.     

\textbf{Deep nets for CV tasks {} {}} We consider a 4-layered MLP. The PyTorch code for 4-layer MLP is as follows: 

\texttt{ nn.Sequential(nn.Flatten(), \\
\tab        nn.Linear(input\_dim, 5000, bias=True),\\
\tab        nn.ReLU(),\\
\tab        nn.Linear(5000, 5000, bias=True),\\
\tab        nn.ReLU(),\\
\tab        nn.Linear(5000, 5000, bias=True),\\
\tab        nn.ReLU(),\\
% \tab        nn.Linear(5000, 5000, bias=True),\\
% \tab        nn.ReLU(),\\
\tab        nn.Linear(1024, num\_label, bias=True)\\
\tab        )}

For MNIST, we use $1000$ nodes instead of $5000$ nodes in the hidden layer. 
% 
We also experiment with convolutional nets. In particular, we use ResNet18 \citep{he2016deep}. Implementation adapted from:  \url{https://github.com/kuangliu/pytorch-cifar.git}. 

\textbf{Deep nets for NLP {} {}} We use a simple LSTM model with embeddings intialized with ELMo embeddings~\citep{Peters:2018}. Code adapted from: \url{https://github.com/kamujun/elmo_experiments/blob/master/elmo_experiment/notebooks/elmo_text_classification_on_imdb.ipynb} 

We also evaluate our bounds with a BERT model. In particular, we fine-tune an off-the-shelf uncased BERT model~\citep{devlin2018bert}. Code adapted from Hugging Face Transformers~\citep{wolf-etal-2020-transformers}: \url{https://huggingface.co/transformers/v3.1.0/custom_datasets.html}. 


\subsection{Additonal experiments}

\textbf{Results with SGD on underparameterized linear models {} {}} 

\begin{figure*}[h]
    \centering 
    % \vspace{-15pt}
    % \includegraphics[width=0.9\linewidth]{example-image-a}
    \includegraphics[width=0.3\linewidth]{figures/lowdim-Gaussian-SGD.pdf}
    % \includegraphics[width=0.9\linewidth]{figures/{CIFAR10_rn=0.1_lr=0.2_wd=0.005}.png}
    \vspace{-5pt}
    \caption{ 
    % Predicted lower bound 
    % on different
    We plot the accuracy and corresponding bound 
    (RHS in \eqref{eq:erm}) at $\delta = 0.1$
    for toy binary classification task. 
    Results aggregated over $3$ seeds. 
    % i.e., $1-\error$ where $\error$ is the term in the RHS of \eqref{eq:erm}
    Accuracy vs fraction of unlabeled data (w.r.t clean data) 
    in the toy setup with a linear model trained with SGD. Results parallel to \figref{fig:error_binary}(a) with SGD.  }
    \label{fig:error_binary_linear}
    \vspace{-5pt}
\end{figure*}

\textbf{Results with wide nets on binary MNIST {} {}}

\begin{figure*}[h]
    \centering 
    % \vspace{-15pt}
    % \includegraphics[width=0.9\linewidth]{example-image-a}
    \subfigure[GD with MSE loss]{\includegraphics[width=0.3\linewidth]{figures/MNIST-GD_MSE.pdf}} \hfil
    \subfigure[SGD with CE loss]{\includegraphics[width=0.3\linewidth]{figures/MNIST-SGD_CE.pdf}}
    \subfigure[SGD with MSE loss]{\includegraphics[width=0.3\linewidth]{figures/MNIST-SGD_MSE-first-layer.pdf}}
    % \includegraphics[width=0.9\linewidth]{figures/{CIFAR10_rn=0.1_lr=0.2_wd=0.005}.png}
    \vspace{-5pt}
    \caption{ 
    % Predicted lower bound 
    % on different
    We plot the accuracy and corresponding bound 
    (RHS in \eqref{eq:erm}) at $\delta = 0.1$ 
    for binary MNIST classification. 
    Results aggregated over $3$ seeds. 
    % i.e., $1-\error$ where $\error$ is the term in the RHS of \eqref{eq:erm}
    Accuracy vs fraction of unlabeled data 
    for a 2-layer wide network on binary MNIST with both the layers training in (a,b) and only first layer training in (c). 
    Results parallel to \figref{fig:error_binary}(b) .  }
    \label{fig:error_binary_MNIST}
    \vspace{-5pt}
\end{figure*}

% \begin{figure*}[h]
%     \centering 
%     % \vspace{-15pt}
%     % \includegraphics[width=0.9\linewidth]{example-image-a}
%     \subfigure[GD with MSE loss]{\includegraphics[width=0.3\linewidth]{figures/MNIST.pdf}} \hfil
    
%     \subfigure[SGD with CE loss]{\includegraphics[width=0.3\linewidth]{figures/MNIST.pdf}}
%     % \includegraphics[width=0.9\linewidth]{figures/{CIFAR10_rn=0.1_lr=0.2_wd=0.005}.png}
%     \vspace{-5pt}
%     \caption{ 
%     % Predicted lower bound 
%     % on different
%     We plot the accuracy and corresponding bound 
%     (RHS in \eqref{eq:erm}) at $\delta = 0.1$
%     for binary MNIST classification. 
%     Results aggregated over $3$ seeds. 
%     % i.e., $1-\error$ where $\error$ is the term in the RHS of \eqref{eq:erm}
%     Accuracy vs fraction of unlabeled data 
%     for a 2-layer wide network on binary MNIST with just the first layer training. 
%     Results parallel to \figref{fig:error_binary}(b) with only the first layer training.  }
%     \label{fig:error_binary_MNIST}
%     \vspace{-5pt}
% \end{figure*}

\textbf{Results on CIFAR 10 and MNIST {} {}} 
% 
We plot epoch wise error curve for results in \tabref{table:multiclass}(\figref{fig:error_epoch_CIFAR10} and \figref{fig:error_epoch_MNIST}). We observe the same trend as in \figref{fig:error_CIFAR10}. Additionally, we plot an \emph{oracle bound} obtained by tracking the error on mislabeled data which nevertheless were predicted as true label. To obtain an exact emprical value of the oracle bound, we need underlying true labels for the randomly labeled data. 
% Note that our bound in \thmref{thm:multiclass_ERM}, lower bounds the accuracy as predicted by the oracle bound. 
While with just access to extra unlabeled data we cannot calculate oracle bound, we note that the oracle bound is very tight and never violated in practice underscoring an importamt aspect of generalization in multiclass problems. This highlight that even a stronger conjecture may hold in multiclass classification, i.e., error on mislabeled data (where nevertheless true label was predicted) lower bounds the population error on the distribution of mislabeled data and hence, the error on (a specific) mislabeled portion predicts the population accuracy on clean data. 
% 
On the other hand, the dominating term of in \thmref{thm:multiclass_ERM} is loose when compared with the oracle bound. The main reason, we believe is the pessimistic upper bound in \eqref{eq:lemma1_final_multi_prev} in the proof of \lemref{lem:fit_mislabeled_multi}. We leave an investigation on this gap for future. 
% of fit 

% However, oracle bound highlights two . One,  



\begin{figure}[h]
    \centering 
    % \vspace{-15pt}
    % \includegraphics[width=0.9\linewidth]{example-image-a}
    \subfigure[MLP]{\includegraphics[width=0.3\linewidth]{figures/CIFAR10-FNN.pdf}} \hfil
    \subfigure[ResNet]{\includegraphics[width=0.3\linewidth]{figures/CIFAR10-Resnet.pdf}}
    % \includegraphics[width=0.9\linewidth]{figures/{CIFAR10_rn=0.1_lr=0.2_wd=0.005}.png}
    % \vspace{-10pt}
    \caption{ Per epoch curves for CIFAR10 corresponding results in \tabref{table:multiclass}. As before, we just plot the dominating term in the RHS of \eqref{eq:multiclass_ERM} as predicted bound. Additionally, we also plot the predicted lower bound by the error on mislabeled data which nevertheless were predicted as true label. We refer to this as ``Oracle bound''. See text for more details. 
    % 
    % except for the stopping point. 
    % The bound predicted by RATT (RHS in \eqref{eq:multiclass_ERM}) is vacuous. 
    }\label{fig:error_epoch_CIFAR10}
    % \vspace{-15pt}
\end{figure}


\begin{figure}[h]
    \centering 
    % \vspace{-15pt}
    % \includegraphics[width=0.9\linewidth]{example-image-a}
    \subfigure[MLP]{\includegraphics[width=0.3\linewidth]{figures/MNIST-FNN.pdf}} \hfil
    \subfigure[ResNet]{\includegraphics[width=0.3\linewidth]{figures/MNIST-Resnet.pdf}}
    % \includegraphics[width=0.9\linewidth]{figures/{CIFAR10_rn=0.1_lr=0.2_wd=0.005}.png}
    % \vspace{-10pt}
    \caption{ Per epoch curves for MNIST corresponding results in \tabref{table:multiclass}. As before, we just plot the dominating term in the RHS of \eqref{eq:multiclass_ERM} as predicted bound. Additionally, we also plot the predicted lower bound by the error on mislabeled data which nevertheless were predicted as true label. We refer to this as ``Oracle bound''. See text for more details. 
    % 
    % except for the stopping point. 
    % The bound predicted by RATT (RHS in \eqref{eq:multiclass_ERM}) is vacuous. 
    }\label{fig:error_epoch_MNIST}
    % \vspace{-15pt}
\end{figure}

\textbf{Results on CIFAR 100 {} {}} 
% 
On CIFAR100, our bound in \eqref{eq:multiclass_ERM} yields vacous bounds. However, the oracle bound as explained above yields tight guarantees in the initial phase of the learning (i.e., when learning rate is less than $0.1$) (\figref{fig:error_CIFAR100}).  

\begin{figure}[h]
    \centering 
    % \vspace{-15pt}
    % \includegraphics[width=0.9\linewidth]{example-image-a}
    \includegraphics[width=0.3\linewidth]{figures/CIFAR100-Resnet.pdf}
    % \includegraphics[width=0.9\linewidth]{figures/{CIFAR10_rn=0.1_lr=0.2_wd=0.005}.png}
    % \vspace{-10pt}
    \caption{ Predicted lower bound by the error on mislabeled data which nevertheless were predicted as true label with ResNet18 on CIFAR100. We refer to this as ``Oracle bound''. See text for more details. 
    % 
    % except for the stopping point. 
    The bound predicted by RATT (RHS in \eqref{eq:multiclass_ERM}) is vacuous. 
    }\label{fig:error_CIFAR100}
    % \vspace{-15pt}
\end{figure}


% \paragraph{Experiments on CIFAR100} 


% \subsection{Model Selection using RATT}


\subsection{Hyperparameter Details}


\textbf{\figref{fig:error_CIFAR10} {} {}} We use clean training dataset of size $40,000$. We fix the amount of unlabeled data at $20\%$ of the clean size, i.e. we include additional $8,000$ points with randomly assigned labels. We use test set of $10,000$ points. For both MLP and ResNet, we use SGD with an initial learning rate of $0.1$ and momentum $0.9$. We fix the weight decay parameter at $5\times 10^{-4}$. After $100$ epochs, we decay the learning rate to $0.01$. We use SGD batch size of $100$. 

\textbf{\figref{fig:error_binary} (a) {} {}} We obtain a toy dataset according to the process described in \secref{sec:app_dataset}. We fix $d=100$ and create a dataset of $50,000$ points with balanced classes. Moreover, we sample additional covariates with the same procedure to create randomly labeled dataset. For both SGD and GD training, we use a fixed learning rate $0.1$.    

\textbf{\figref{fig:error_binary} (b) {} {}} Similar to binary CIFAR, we use clean training dataset of size $40,000$ and fix the amount of unlabeled data at $20\%$ of the clean dataset size. To train wide nets, we use a fixed learning of $0.001$ with GD and SGD. We decide the weight decay parameter and the early stopping point that maximizes our generalization bound (i.e. without peeking at unseen data ).  We use SGD batch size of $100$. 

\textbf{\figref{fig:error_binary} (c) {} {}} With IMDb dataset, we use a clean dataset of size $20,000$ and as before, fix the amount of unlabeled data at $20\%$ of the clean data. To train ELMo model, we use Adam optimizer with a fixed learning rate $0.01$ and weight decay $10^{-6}$ to minimize cross entropy loss. We train with batch size $32$ for 3 epochs. To fine-tune BERT model, we use Adam optimizer with learning rate $5\times 10^{-5}$ to minimize cross entropy loss. We train with a batch size of $16$ for 1 epoch.    

\textbf{\tabref{table:multiclass} {} {}} For multiclass datasets, we train both MLP and ResNet with the same hyperparameters as described before. We sample a clean training dataset of size $40,000$ and fix the amount of unlabeled data at $20\%$ of the clean size. We use SGD with an initial learning rate of $0.1$ and momentum $0.9$. We fix the weight decay parameter at $5\times 10^{-4}$. After $30$ epochs for ResNet and after $50$ epochs for MLP, we decay the learning rate to $0.01$.  We use SGD with batch size $100$. 
For \figref{fig:error_CIFAR100}, we use the same hyperparameters as 
CIFAR10 training, except we now decay learning rate after $100$ epochs. 


In all experiments, to identify the best possible accuracy on just the clean data, we use the exact same set of hyperparamters except the stopping point. We choose a stopping point that maximizes test performance. 

\subsection{Summary of experiments }

\begin{center}
    \begin{table}[H] 
        \centering
        \begin{tabular}{|c|c|c|c|} 
        \hline
        Classification type & Model category & Model & Dataset  \\ [0.5ex] 
        \hline
        \hline
        \multirow{10}{*}{Binary} & Low dimensional & Linear model & Toy Gaussain dataset  \\
                        \cline{2-4}
                         & Overparameterized 
                        %  & Linear model & Toy Gaussain dataset \\
                        %  \cline{3-4}
                        %  & & 2-layer wide net& Toy Gaussain dataset \\
                        %  \cline{3-4}
                         & \multirow{2}{*}{2-layer wide net} & \multirow{2}{*}{Binary MNIST} \\
                         & linear nets & &  
                         \\
                         \cline{2-4}                 
                         & \multirow{6}{*}{Deep nets} & \multirow{2}{*}{MLP} & Binary MNIST \\
                         \cline{4-4}
                         & &  & Binary CIFAR \\
                         \cline{3-4}
                         &  & \multirow{2}{*}{ResNet} & Binary MNIST \\
                         \cline{4-4}
                         & &  & Binary CIFAR \\
                         \cline{3-4}
                         &  & ELMo-LSTM model & IMDb Sentiment Analysis \\
                         \cline{3-4}
                         & & BERT pre-trained model & IMDb Sentiment Analysis \\
        \hline
        \multirow{5}{*}{Multiclass} & \multirow{5}{*}{Deep nets} & \multirow{2}{*}{MLP} & MNIST \\
                        \cline{4-4} 
                        & & & CIFAR10 \\                   
                        \cline{3-4}
                         &   & \multirow{3}{*}{ResNet} & MNIST \\
                         \cline{4-4}
                         &   & & CIFAR10 \\
                         \cline{4-4}
                         &   & & CIFAR100 \\
        \hline
        \end{tabular}
        % \caption{Summary of experiments performed} \label{table:experiments}
    \end{table}    
    % \footnotetext[6]{We use both MSE loss and cross-entropy loss.}
    % \footnotetext[6]{We try 2 variants: one with a fixed first layer and the other with both layers trainable.}
\end{center}

\newpage
\section{Proof of \lemref{lem:stability_error}} \label{app:proof_lem_error}

\begin{proof}[Proof of \lemref{lem:stability_error}]
    Recall, we have a training set $S \cup \wt S_C$. We defined leave-one-out error on mislabeled points as $$\error_{\text{LOO}(\wt S_M) } = \frac{\sum_{(x_i, y_i) \in \wt S_M} \error( f_{(i)}( x_i), y_i)}{ \abs{\wt S_M }} \,, $$
    where $f_{(i)} \defeq f(\calA, (S \cup \wt S)_{(i)})$. Define $S^\prime \defeq S \cup \wt S$. Assume $(x,y)$ and $(x^\prime,y^\prime)$ as i.i.d. samples from ${\calDm}$. 
    Using Lemma 25 in \citet{bousquet2002stability}, we have
    \begin{align*}
        \Expo{ \left( \error_{\calDm}(\wh f) -\error_{\text{LOO}(\wt S_M) } \right)^2 } \le & \Expt{ S^\prime, (x,y), (x^\prime,y^\prime) }{ \error(\wh f(x), y ) \error(\wh f(x^\prime), y^\prime )} - 2 \Expt{ S^\prime, (x,y) }{ \error(\wh f(x), y ) \error(f_{(i)}(x_i), y_i )} \\
        & + \frac{m_1-1}{m_1}\Expt{ S^\prime }{  \error(f_{(i)}(x_i), y_i )  \error(f_{(j)}(x_j), y_j )} + \frac{1}{m_1} \Expt{ S^\prime }{  \error(f_{(i)}(x_i), y_i ) } \,. \numberthis \label{eq:main_reln}
    \end{align*}
    We can rewrite the equation above as : 
    \begin{align*}
        \Expo{ \left( \error_{\calDm}(\wh f) -\error_{\text{LOO}(\wt S_M) } \right)^2 } \le &  \, \underbrace{\Expt{ S^\prime, (x,y), (x^\prime,y^\prime) }{ \error(\wh f(x), y ) \error(\wh f(x^\prime), y^\prime ) - \error(\wh f(x), y ) \error(f_{(i)}(x_i), y_i )}}_{\RN{1}} \\
        & + \underbrace{\Expt{ S^\prime }{  \error(f_{(i)}(x_i), y_i )  \error(f_{(j)}(x_j), y_j ) -  \error(\wh f(x), y ) \error(f_{(i)}(x_i), y_i )}}_{\RN{2}} \\ &+ \underbrace{\frac{1}{m_1} \Expt{ S^\prime }{  \error(f_{(i)}(x_i), y_i ) - \error(f_{(i)}(x_i), y_i )  \error(f_{(j)}(x_j), y_j ) }}_{\RN{3}} \,. \numberthis \label{eq:main_reln2}
    \end{align*}
    
    We will now bound term $\RN{3}$.  Using Cauchy-Schwarz's inequality, we have
    
    \begin{align}
        \Expt{ S^\prime }{  \error(f_{(i)}(x_i), y_i ) - \error(f_{(i)}(x_i), y_i )  \error(f_{(j)}(x_j), y_j ) }^2 &\le  \Expt{ S^\prime }{  \error(f_{(i)}(x_i), y_i ) }^2 \Expt{S^\prime}{1 -   \error(f_{(j)}(x_j), y_j ) }^2 \\
        &\le \frac{1}{4} \,.\label{eq:term1_lem12}
    \end{align}
    
    Note that since $(x_i,y_i)$, $(x_j ,y_j )$, $(x,y)$, and $(x^\prime, y^\prime)$ are all from same distribution $\calDm$, we directly incorporate the bounds on term $\RN{1}$ and $\RN{2}$ from the proof of Lemma 9 in \citet{bousquet2002stability}. Combining that with \eqref{eq:term1_lem12} and our definition of hypothesis stability in \codref{cond:hypothesis_stability}, we have the required claim. 
    
    
    % We now re-write term $\RN{1}$ as
    % \begin{align*}
    %         &\Expt{S^\prime, (x,y), (x^\prime,y^\prime) }{ \error(\wh f(x), y ) \error(\wh f(x^\prime), y^\prime ) - \error(\wh f(x), y ) \error(f_{(i)}(x_i), y_i )} \\ & \qquad = \Expt{ S^\prime, (x,y), (x^\prime,y^\prime) }{ \error(\wh f(x), y ) \error(\wh f  (x^\prime), y^\prime ) - \error(\wh f ^\prime(x), y ) \error(f_{(i)}(x^\prime), y^\prime )} \tag{Exchanging $(x_i, y_i)$ with $(x^\prime, y^\prime)$ in the second term} \\
    %         & \qquad = \Expt{ S^\prime, (x,y), (x^\prime,y^\prime) }{  \left(\error(\wh f(x), y )-  \error(f_{(i)}(x), y ) \right) \error(\wh f  (x^\prime), y^\prime )  } \\
    %         & \qquad  + \Expt{ S^\prime, (x,y), (x^\prime,y^\prime) }{  \left(\error(f_{(i)}(x), y ) -\error(\wh f ^\prime(x), y ) \right) \error(\wh f  (x^\prime), y^\prime )}  \\
    %         & \qquad +\Expt{ S^\prime, (x,y), (x^\prime,y^\prime) }{  \left( \error(\wh f  (x^\prime), y^\prime ) -  \error(f_{(i)}(x^\prime), y^\prime ) \right) \error(\wh f ^\prime(x), y ) }  \,, \numberthis \label{eq:term1_final}
    % \end{align*}
    % where $\wh f^\prime$ is the classifier obtained by training on $ S^\prime_{(i)} \cup \{ (x^\prime, y^\prime) \} $. Similarly we can re-write term $\RN{2}$ as 
    % \begin{align*}
    %     & \Expt{ S^\prime }{  \error(f_{(i)}(x_i), y_i )  \error(f_{(j)}(x_j), y_j ) -  \error(\wh f(x), y ) \error(f_{(i)}(x_i), y_i )} \\
    %     &\quad  = \Expt{ S^\prime, (x,y), (x^\prime,y^\prime)}{  \error(f^{\prime\prime}_{(i)}(x), y )  \error(f_{(j)}^{\prime}(x^\prime), y^\prime ) -  \error(\wh f(x), y ) \error(f_{(i)}(x_i), y_i )} \tag{Exchanging $(x_i, y_i)$ with $(x, y)$ and $(x_j, y_j)$ with $(x^\prime, y^\prime)$ in the first term}\\
    %     &\quad = \Expt{ S^\prime, (x,y), (x^\prime,y^\prime)}{  \error(f^{\prime\prime}_{(j)}(x), y )  \error(f_{(i)}^{\prime}(x^\prime), y^\prime ) -  \error(\wh f^\prime (x), y ) \error(f^\prime_{(j)}(x^\prime), y^\prime )} \tag{Exchanging $(x_i, y_i)$ and $(x_j, y_j)$ and then replacing $(x_j, y_j)$ with $(x^\prime, y^\prime)$ in the second term} \\
    %     & \quad = \Expt{ S^\prime, (x,y), (x^\prime,y^\prime) }{  \left( \error(f_{(i)}^{\prime}(x^\prime), y^\prime )   -  \error(\wh f^{\prime\prime}  (x^\prime), y^\prime ) \right)  \error(f^{\prime\prime}_{(j)}(x), y )   } \\
    %     & \quad  + \Expt{ S^\prime, (x,y), (x^\prime,y^\prime) }{  \left( \error(f^{\prime\prime}_{(j)}(x), y )  -\error(\wh f ^\prime(x), y ) \right) \error(\wh f^{\prime\prime}  (x^\prime), y^\prime )  }  \\
    %     & \quad+ \Expt{ S^\prime, (x,y), (x^\prime,y^\prime) }{  \left( \error(\wh f^{\prime\prime}  (x^\prime), y^\prime )  -  \error(f^\prime_{(j)}(x^\prime), y^\prime ) \right)  \error(\wh f^\prime (x), y ) }   \\
    %     & \quad = \Expt{ S^\prime, (x,y), (x^\prime,y^\prime) }{  \left( \error(f_{(i)}^{\prime}(x^\prime), y^\prime )   -  \error(\wh f (x^\prime), y^\prime ) \right)  \error(f_{(i)}(x_j), y_j )   } \\
    %     & \quad  + \Expt{ S^\prime, (x,y), (x^\prime,y^\prime) }{  \left( \error(f^{\prime\prime}_{(j)}(x), y )  -\error(\wh f (x), y ) \right) \error(\wh f^{\prime\prime}  (x_j), y_j )  }  \\
    %     & \quad+ \Expt{ S^\prime, (x,y), (x^\prime,y^\prime) }{  \left( \error(\wh f^{\prime\prime}  (x^\prime), y^\prime )  -  \error(f^\prime_{(j)}(x^\prime), y^\prime ) \right)  \error(\wh f^\prime (x^\prime), y^\prime ) }  \,, \numberthis \label{eq:term2_final}
    % \end{align*}
    % where $f^{\prime\prime}_{(j)}$ is trained on $S^\prime_{(j,i)} \cup {(x,y)}$, $f^{\prime}_{(i)}$ is trained on $S^\prime_{(j,i)} \cup {(x^\prime,y^\prime)}$, and $\wh f^{\prime\prime} $ is trained on $S^\prime_{(j)} \cup {(x,y)}$. Note in the last line we replaced $(x,y)$ by $(x_j, y_j)$ in the first term, replaced $(x^\prime,y^\prime)$ by $(x_j, y_j)$ in the second term and exchanged $(x_i,y_i)$ with $(x_j,y_j)$ and also $(x,y)$ and $(x^\prime, y^\prime)$
    
    
\end{proof}


% 
% 16th Century Version Control 
% 

% \onecolumn

% \section*{Supplementary Material}
% We will be using the following standard results
% on exponential concentration of random variables 
% all throughout the discussion:

% \begin{lemma}[Hoeffding's inequality for independent RVs~\citep{hoeffding1994probability}] Let $Z_1, Z_2, \ldots, Z_n$ be independent bounded random variables with $Z_i \in [a,b]$ for all $i$, then 
%     \begin{align*}
%         \prob\left( \frac{1}{n} \sum_{i=1}^n (Z_i - \Expo{Z_i}) \ge t \right) \le \exp{\left( -\frac{2nt^2}{(b-a)^2} \right) }
%     \end{align*} 
%     and 
%     \begin{align*}
%         \prob\left( \frac{1}{n} \sum_{i=1}^n (Z_i - \Expo{Z_i}) \le -t \right) \le \exp{\left( -\frac{2nt^2}{(b-a)^2} \right) }
%     \end{align*} 
%     for all $t \ge 0$. 
% \end{lemma}

% \begin{lemma}[Hoeffding's inequality for sampling with replacement~\citep{hoeffding1994probability}] \label{lem:hoeffding_sampling} Let $\calZ = (Z_1, Z_2, \ldots, Z_N)$ be a finite population of $N$ points with $Z_i \in [a.b]$ for all $i$. Let $X_1, X_2, \ldots X_n$ be a random sample drawn without replacement from $\calZ$. Then for all $t \ge 0$, we have 
%     \begin{align*}
%         \prob\left( \frac{1}{n} \sum_{i=1}^n (X_i - \mu ) \ge t \right) \le \exp{\left( -\frac{2nt^2}{(b-a)^2} \right) }
%     \end{align*} 
%     and 
%     \begin{align*}
%         \prob\left( \frac{1}{n} \sum_{i=1}^n (X_i - \mu ) \le -t \right) \le \exp{\left( -\frac{2nt^2}{(b-a)^2} \right) } \,,
%     \end{align*} 
%     where $\mu = \frac{1}{N} \sum_{i=1}^{N} Z_i$. 
% \end{lemma}

% We now discuss one condition that generalizes the exponential concentration to dependent random variables.
% \begin{condition}[Bounded difference inequality] \label{cond:BDC} Let $\calZ$ be some set and $\phi: \calZ^n \to \Real$. We say that $\phi$ satisfies the bounded difference assumption if 
% there exists $c_1, c_2, \ldots c_n \ge 0$ s.t. for all $i$, we have 
% \begin{align*}
%     \sup_{Z_1,Z_2, \ldots,Z_n, Z_i^\prime in \calZ^{n+1} } \abs{\phi (Z_1, \ldots, Z_i, \ldots, Z_n ) - \phi (Z_1, \ldots, Z_i^\prime, \ldots, Z_n ) } \le c_i \,.
% \end{align*} 
% \end{condition}

% \begin{lemma}[McDiarmid’s inequality~\citep{mcdiarmid1989}] \label{lem:McDiarmid} Let $Z_1, Z_2, \ldots, Z_n$ be independent random variables on set $\calZ$ and $\phi : \calZ^n \to \Real$ satisfy bounded difference assumption (\codref{cond:BDC}). Then for all $t>0$, we have 
%     \begin{align*}
%         \prob\left( \phi(Z_1, Z_2, \ldots, Z_n) - \Expo{\phi(Z_1, Z_2, \ldots, Z_n)} \ge t \right) \le \exp{\left( -\frac{2t^2}{\sum_{i=1}^n c_i^2} \right) } 
%     \end{align*} 
%     and 
%     \begin{align*}
%         \prob\left( \phi(Z_1, Z_2, \ldots, Z_n) - \Expo{\phi(Z_1, Z_2, \ldots, Z_n)} \le -t \right) \le \exp{\left( -\frac{2t^2}{\sum_{i=1}^n c_i^2} \right) } \,
%     \end{align*} 
% \end{lemma}


% \section{Proofs from \secref{sec:ERM_training}}\label{app:proof_erm}

% \textbf{Additional notation {} {}} Let $m_1$ be the number of mislabeled points ($\wt S_M$) and $m_2$ be the number of correctly labeled points ($\wt S_C$). Note $m_1 + m_2 = m$. 


% \subsection{Proof of \thmref{thm:error_ERM}}


% \begin{proof}[Proof of \lemref{lem:fit_mislabeled}] 
%     The main idea of our proof is to regard 
%     the clean portion of the data 
%     ($S \cup \wt S_C$) as fixed.   
%     Then, there exists a classifier $f^*$ 
%     that is optimal over draws 
%     of the mislabeled data $\wt S_M$. 
% % 
%     % 
%     Formally, 
%     \begin{align}
%     f^* \defeq \argmin_{f \in \calF} \error_{\widecheck {\calD}} (f) \,, \label{eq:modified_ERM}
%     \end{align}
%     where $$\widecheck \calD = \frac{n}{m+n} \calS + \frac{m_1}{m+n} \wt \calS_C  + \frac{m_2}{m+n}\calDm \,.$$ That is, $\widecheck \calD$ a combination of 
%     the \emph{empirical distribution} 
%     over correctly labeled data $S \cup \wt S_C$
%     % in $S\cup \wt S$ 
%     and the (population) distribution 
%     over mislabeled data $\calDm$.
%     Recall that 
%     \begin{align}
%     \wh f \defeq \argmin_{f \in \calF} \error_{\calS \cup \wt S} (f) \,. \label{eq:orig_ERM}
%     \end{align}
%     % 
%     % 
%     Since, $\widehat f$ minimizes 0-1 error 
%     on $S \cup \wt S$, using ERM optimality on \eqref{eq:orig_ERM},  
%     we have 
%     \begin{align}
%         \error_{\calS \cup \wt \calS}(\widehat f) \le \error_{
%             \calS \cup \wt \calS}(f^*) \,.    \label{eq:step1}
%     \end{align}
%     Moreover, since $f^*$ is independent of $\wt S_M$, using Hoeffding's bound,
%     % \footnote{For a fully rigorous argument,
%     % refer to the complete proof in App.~\ref{app:proof_erm}.} 
%     we have with probability at least $1-\delta$ that
%     \begin{align}
%       \error_{\wt \calS_M}(f^*) \le \error_{ \calDm}(f^*) +  \sqrt{\frac{\log(1/\delta)}{2 m_1}} \,. \label{eq:step2} 
%     \end{align}
%     %$ 
%     %for some constant $c_1\le 1/2$. 
%     Finally, since $f^*$ is the optimal classifier on $\widecheck \calD$, 
%     we have 
%     \begin{align}
%         \error_{\widecheck \calD}(f^*) \le \error_{\widecheck \calD}(\widehat f) \label{eq:step3}
%     \end{align}
%      Now to relate \eqref{eq:step1} and \eqref{eq:step3}, we can re-write the \eqref{eq:step2} as follows: 
%     \begin{align}
%         \error_{\calS \cup \wt\calS}(f^*) \le \error_{ \widecheck \calD}(f^*) +  \frac{m_1}{m+n}\sqrt{\frac{\log(1/\delta)}{2 m_1}} \,. \label{eq:step4} 
%     \end{align}
%     Now we combine equations \eqref{eq:step1}, \eqref{eq:step4}, and \eqref{eq:step3}, to get 
%     \begin{align}
%         \error_{\calS \cup \wt \calS}(\wh f) \le \error_{\widecheck \calD}(\wh f) +  \frac{m_1}{m+n}\sqrt{\frac{\log(1/\delta)}{2 m_1}} \,, 
%     \end{align}
%     which implies 
%     \begin{align}
%         \error_{ \wt \calS_M}(\wh f) \le \error_{\calDm}(\wh f) + \sqrt{\frac{\log(1/\delta)}{2 m_1}} \,. \label{eq:lemma1_final}
%     \end{align}
%     Since $\wt S$ is obtained by randomly labeling an unlabeled dataset, we assume $2m_1 \approx m$ \footnote{Formally, with probability at least $1-\delta$, we have  $(m - 2m_1)\le \sqrt{m\log(1/\delta)/2}$ }. Moreover, using $\error_{\calDm} = 1 - \error_{\calD}$ we obtain the desired result.   
%     % Combining the above steps and using the fact 
%     % that $\error_\calD = 1- \error_{\calDm} $, 
%     % we obtain the desired result.
% \end{proof}

% \begin{proof}[Proof of \lemref{lem:mislabeled_error}]
%     Recall $\error_{\wt S} (f) = \frac{m_1}{m} \error_{\wt S_M}(f) + \frac{m_2}{m} \error_{\wt S_C}(f)$. Hence, we have 
%     \begin{align}
%         2\error_{\wt S}(f) - \error_{\wt S_M}(f) - \error_{\wt S_C}(f) &= \left(\frac{2m_1}{m} \error_{\wt S_M}(f) - \error_{\wt S_M}(f)\right) + \left(\frac{2m_2}{m} \error_{\wt S_C}(f) - \error_{\wt S_C}(f)\right) \\ &= \left(\frac{2m_1}{m} - 1\right) \error_{\wt S_M}(f) + \left(\frac{2m_2}{m} - 1 \right)\error_{\wt S_C} (f) \,.
%     \end{align} 
%     Since the dataset is randomly labeled, with probability at least $1-\delta$, we have  $\left(\frac{2m_1}{m} - 1\right) \le \sqrt{\frac{\log(1/\delta)}{2m}}$. Similarly, we have with probability at least $1-\delta$, $\left(\frac{2m_2}{m} - 1\right) \le \sqrt{\frac{\log(1/\delta)}{2m}}$. Using union bound, we have with probability at least $1-\delta$
%     % \begin{align}
%     %     2\error_{\wt S} - \error_{\wt S_M}(f) - \error_{\wt S_C}(f) \le \sqrt{\frac{\log(2/\delta)}{2m}} \left(\error_{\wt S_M}(f) + \error_{\wt S_C}(f) \right) \le 2\sqrt{\frac{\log(2/\delta)}{2m}} \,. \label{eq:lemma2_final}
%     % \end{align}
%     \begin{align}
%         2\error_{\wt S} - \error_{\wt S_M}(f) - \error_{\wt S_C}(f) \le \sqrt{\frac{\log(2/\delta)}{2m}} \left(\error_{\wt S_M}(f) + \error_{\wt S_C}(f) \right) \,. \label{eq:lemma2_prefinal}
%     \end{align}
%     With re-arranging $\error_{\wt S_M}(f) + \error_{\wt S_C}(f)$ and using the inequality $ 1- a\le \frac{1}{1+a} $, we have  
%     \begin{align}
%         2\error_{\wt S} - \error_{\wt S_M}(f) - \error_{\wt S_C}(f) \le 2\error_{\wt \calS} \sqrt{\frac{\log(2/\delta)}{2m}}  \,. \label{eq:lemma2_final}
%     \end{align}

%     % We obtain the desired result by using 
% \end{proof}

% \begin{proof}[Proof of \lemref{lem:clear_error}]
% % Recall 0-1 error on each point  $(x,y) \in S \cup \wt S$ is given by $\I{ f(x)\ne y}$.
% In the set of correctly labeled points $S \cup \wt S_C$, we have $S$ as a random subset of $S \cup \wt S_C$. Hence, using Hoeffding's inequality for sampling without replacement (\lemref{lem:hoeffding_sampling}), we have with probability at least $1-\delta$
% \begin{align}
%     \error_{\wt \calS_c} (\wh f)- \error_{\calS \cup \wt \calS_C}( \wh f) \le  \sqrt{\frac{\log(1/\delta)}{2m_2}} \,.
% \end{align}
% Re-writing $\error_{\calS \cup \wt \calS_C}( \wh f)$ as $\frac{m_2}{m_2 + n} \error_{\wt \calS_C }(\wh f) + \frac{n}{m_2 + n} \error_{\calS }(\wh f)$, we have with probability at least $1-\delta$
% \begin{align}
%   \left(\frac{n}{n+m_2}\right) \left(\error_{\wt \calS_c} (\wh f)- \error_{\calS}( \wh f) \right) \le  \sqrt{\frac{\log(1/\delta)}{2m_2}} \,.
% \end{align}
% As before, assuming $2m_2 \approx m$, we have with probability at least $1-\delta$ 
% \begin{align}
%     \error_{\wt \calS_c} (\wh f)- \error_{\calS}( \wh f) \le \left(1+\frac{m_2}{n}\right)  \sqrt{\frac{\log(1/\delta)}{m}} \le 1.5 \sqrt{\frac{\log(1/\delta)}{m}} \,. \label{eq:lemma3_final}
% \end{align} 
% \end{proof}

% \begin{proof}[Proof of \thmref{thm:error_ERM}] 
%     Having established these core intermediate results, we can now combine above three lemmas to prove the main result. 
%     In particular, we bound the population error on clean data ($\error_\calD(\wh f)$) as follows:  
%     \begin{enumerate}[(i)]
%         \item First, use \eqref{eq:lemma1_final}, to obtain an upper bound on the population error on clean data, i.e., with probability at least $1-\delta/4$, we have
%         \begin{align}
%             \error_{ \calD} (\wh f) \le 1 - \error_{ \wt \calS_M}(\wh f) + \sqrt{\frac{\log(4/\delta)}{m}} \,. 
%         \end{align}
%         \item  Second, use \eqref{eq:lemma2_final}, to relate the error on the mislabeled fraction with error on clean portion of randomly labeled data and error on whole randomly labeled dataset, i.e., with probability at least $1-\delta/2$, we have 
%         \begin{align}
%             - \error_{\wt S_M}(f) \le \error_{\wt S_C}(f) - 2\error_{\wt S}  + \sqrt{\frac{\log(4/\delta)}{2m}}  \,. 
%         \end{align} 
%         \item Finally, use \eqref{eq:lemma3_final} to relate the error on the clean portion of randomly labeled data and error on clean training data, i.e., with probability $1-\delta/4$, we have 
%         \begin{align}
%             \error_{\wt \calS_C} (\wh f)\le - \error_{\calS}( \wh f) + \left(1 + \frac{m}{2n} \right) \sqrt{\frac{\log(4/\delta)}{m}} \,. 
%         \end{align} 
%     \end{enumerate}

%     Using union bound on the above three steps, we have with probability at least $1-\delta$: 
%     \begin{align}
%         \error_\calD (\wh f) \le \error_{\calS}(\wh f)   + 1 - 2\error_{\wt \calS}(\wh f)   + (1/\sqrt{2} + 2.5)  \sqrt{\frac{\log(4/\delta)}{m}} \,.
%     \end{align}
%     Note that $(1/\sqrt{2} + 2.5)$ is a loose constant. In experiments, we use the ratio $\frac{m}{n}$
%     %  the exact error $\error_{\wt \calS}(\wh f)$ 
%     to evaluate R.H.S.    
% \end{proof}

% \subsection{Proof of \propref{prop:rademacher}}

% \begin{proof}[Proof of \propref{prop:rademacher}]
%     For a classifier $ f: \calX \to \{-1, 1\}$, we have $1 - 2\,\indict{ f(x) \ne y} = y \cdot f(x)$. Hence, by definition of $\error$, we have 
%     \begin{align}
%         1 -2\error_{\wt \calS}(f) = \frac{1}{m}\sum_{i=1}^m y_i \cdot f(x_i) \le \sup_{f \in \calF} \, \frac{1}{m} \sum_{i=1}^m y_i \cdot f(x_i)  \,. \label{eq:error_rademacher}
%     \end{align}
%     Note that for fixed inputs $(x_1, x_2, \ldots, x_m)$ in $\wt S$, $(y_1, y_2, \ldots y_m)$ are random labels. Define $\phi_1 (y_1, y_2, \ldots, y_m) \defeq \sup_{f \in \calF} \, \frac{1}{m} \sum_{i=1}^m y_i \cdot f(x_i)$. We have the following bounded difference condition on $\phi_1$. For all i, 
%     \begin{align}
%         \sup_{y_1, \ldots y_m, y_i^\prime \in \{-1, 1\}^{m+1} } \abs{ \phi_1 (y_1,\ldots, y_i, \ldots, y_m) - \phi_1 (y_1,\ldots, y_i^\prime, \ldots, y_m)  } \le 1/m \,. \label{cond1_rademacher}
%     \end{align} 
    
%     Similarly define $\phi_2 (x_1, x_2, \ldots, x_m) \defeq \Expt{ y_i \sim_U \{-1, 1\}  }{ \sup_{f \in \calF} \, \frac{1}{m}  \sum_{i=1}^m y_i \cdot f(x_i)}$. We have the following bounded difference condition on $\phi_2$. For all i,
%     \begin{align}
%         \sup_{x_1, \ldots x_m, x_i^\prime \in \calX^{m+1} } \abs{ \phi_2 (x_1,\ldots, x_i, \ldots, x_m) - \phi_1 (x_1,\ldots, x_i^\prime, \ldots, x_m)  } \le 1/m \,. \label{cond2_rademacher}
%     \end{align}
%     Using McDiarmid’s inequality (\lemref{lem:McDiarmid}) twice with Condition \eqref{cond1_rademacher} and \eqref{cond2_rademacher}, with probability at least $1-\delta$, we have
%     \begin{align}
%         \sup_{f \in \calF} \, \frac{1}{m} \sum_{i=1}^m y_i \cdot f(x_i)  - \Expt{x,y}{\sup_{f \in \calF} \, \frac{1}{m} \sum_{i=1}^m y_i \cdot f(x_i) } \le \sqrt{\frac{2\log(2/\delta)}{m}} \label{eq:final_rademacher}
%     \end{align} 
%     Combining \eqref{eq:error_rademacher} and \eqref{eq:final_rademacher}, we obtain the desired result. 
% \end{proof}


% \subsection{Proof of \thmref{thm:error_regularized_ERM}}

% Proof of \thmref{thm:error_regularized_ERM} follows similar to the proof of \thmref{thm:error_ERM}. Note that the same results in \lemref{lem:fit_mislabeled}, \lemref{lem:mislabeled_error}, and \lemref{lem:clear_error} hold in the regularized ERM case. However, the arguments in the proof of \lemref{lem:fit_mislabeled} changes slightly. Hence, we state and prove a lemma parallel to \lemref{lem:fit_mislabeled} for completeness. 

% \begin{lemma} \label{lem:lemma1_reg}
%     Assume the same setup as \thmref{thm:error_regularized_ERM}. 
%     Then for any $\delta >0$, with probability at least  $1-\delta$ 
%     over the random draws of mislabeled data $\wt S_M$, we have 
%     \begin{align}
%         \error_\calD(\widehat f)  \le 1 -\error_{\wt \calS_M}(\widehat f) + \sqrt{\frac{\log(1/\delta)}{m}}\,. 
%     \end{align} 
% \end{lemma}
% \begin{proof}
%     The main idea of the proof remains the same, i.e. regard 
%     the clean portion of the data 
%     ($S \cup \wt S_C$) as fixed.   
%     Then, there exists a classifier $f^*$ 
%     that is optimal over draws 
%     of the mislabeled data $\wt S_M$. 

    
%     Formally, 
%     \begin{align}
%     f^* \defeq \argmin_{f \in \calF} \error_{\widecheck {\calD}} (f)  + \lambda R(f) \,, \label{eq:modified_ERM_reg}
%     \end{align}
%     where $$\widecheck \calD = \frac{n}{m+n} \calS + \frac{m_1}{m+n} \wt \calS_C  + \frac{m_2}{m+n}\calDm \,.$$ That is, $\widecheck \calD$ a combination of 
%     the \emph{empirical distribution} 
%     over correctly labeled data $S \cup \wt S_C$
%     % in $S\cup \wt S$ 
%     and the (population) distribution 
%     over mislabeled data $\calDm$.
%     Recall that 
%     \begin{align}
%     \wh f \defeq \argmin_{f \in \calF} \error_{\calS \cup \wt S} (f) + \lambda R(f) \,. \label{eq:orig_ERM_reg}
%     \end{align}
%     % 
%     % 
%     Since, $\widehat f$ minimizes 0-1 error 
%     on $S \cup \wt S$, using ERM optimality on \eqref{eq:orig_ERM},  
%     we have 
%     \begin{align}
%         \error_{\calS \cup \wt \calS}(\widehat f) + \lambda R(\wh f) \le \error_{
%             \calS \cup \wt \calS}(f^*) + \lambda R(f^*) \,.    \label{eq:step1_reg}
%     \end{align}
%     Moreover, since $f^*$ is independent of $\wt S_M$, using Hoeffding's bound,
%     % \footnote{For a fully rigorous argument,
%     % refer to the complete proof in App.~\ref{app:proof_erm}.} 
%     we have with probability at least $1-\delta$ that
%     \begin{align}
%       \error_{\wt \calS_M}(f^*) \le \error_{ \calDm}(f^*) +  \sqrt{\frac{\log(1/\delta)}{2 m_1}} \,. \label{eq:step2_reg} 
%     \end{align}
%     %$ 
%     %for some constant $c_1\le 1/2$. 
%     Finally, since $f^*$ is the optimal classifier on $\widecheck \calD$, 
%     we have 
%     \begin{align}
%         \error_{\widecheck \calD}(f^*) + \lambda R(f^*) \le \error_{\widecheck \calD}(\widehat f) + \lambda R(\wh f) \label{eq:step3_reg}
%     \end{align}
%      Now to relate \eqref{eq:step1_reg} and \eqref{eq:step3_reg}, we can re-write the \eqref{eq:step2_reg} as follows: 
%     \begin{align}
%         \error_{\calS \cup \wt\calS}(f^*) \le \error_{ \widecheck \calD}(f^*) +  \frac{m_1}{m+n}\sqrt{\frac{\log(1/\delta)}{2 m_1}} \,. \label{eq:step4_reg} 
%     \end{align}
%     After adding $\lambda R(f^*)$ on both sides in \eqref{eq:step4_reg}, we combine equations \eqref{eq:step1_reg}, \eqref{eq:step4_reg}, and \eqref{eq:step3_reg}, to get 
%     \begin{align}
%         \error_{\calS \cup \wt \calS}(\wh f) \le \error_{\widecheck \calD}(\wh f) +  \frac{m_1}{m+n}\sqrt{\frac{\log(1/\delta)}{2 m_1}} \,, 
%     \end{align}
%     which implies 
%     \begin{align}
%         \error_{ \wt \calS_M}(\wh f) \le \error_{\calDm}(\wh f) + \sqrt{\frac{\log(1/\delta)}{2 m_1}} \,. \label{eq:lemma_reg_final}
%     \end{align}
%     Similar as before, since $\wt S$ is obtained by randomly labeling an unlabeled dataset, we assume 
%     $2m_1 \approx m$. Moreover, using $\error_{\calDm} = 1 - \error_{\calD}$ we obtain the desired result. 
% \end{proof}
% % \begin{proof}[Proof of ]
    
% % \end{proof}

% \subsection{Proof of \thmref{thm:multiclass_ERM}}

% We first state and prove lemmas parallel to three lemmas used in the proof of balanced binary case. Then we combine the results in the three lemmas to obtain the result in \thmref{thm:multiclass_ERM}. 

% Before stating the result, we define mislabeled distribution $\calDm$ for any $\calD$. While $\calDm$ and $\calD$ share 
% the same marginal distribution over $\calX$, 
% the distribution over labels $y$ 
% given an input $x\sim \calD_\calX$ is changed.
% In particular, for any $x$, the pdf over $y$ is changed to:  
% $p_{\calDm} (\cdot \vert x) \defeq \frac{1 - p_{\calD}(\cdot \vert x)}{k - 1}$.

% \begin{lemma} \label{lem:fit_mislabeled_multi}
%     Assume the same setup as \thmref{thm:multiclass_ERM}. 
%     Then for any $\delta >0$, with probability at least  $1-\delta$ 
%     over the random draws of mislabeled data $\wt S_M$, we have 
%     \begin{align}
%         \error_\calD(\widehat f)  \le (k-1)\left(1 -\error_{\wt \calS_M}(\widehat f)\right) + (k-1)\sqrt{\frac{\log(1/\delta)}{m}}\,. \label{eq:lemma1_multi}
%     \end{align}   
% \end{lemma} 

% \begin{proof}
%     The main idea of the proof remains the same, i.e. regard 
%     the clean portion of the data 
%     ($S \cup \wt S_C$) as fixed. 
%     Then, there exists a classifier $f^*$ 
%     that is optimal over draws 
%     of the mislabeled data $\wt S_M$. 
    
%     However, we need to be careful while relating population error on mislabeled data with population accuracy on clean data.   
%     While for binary classification,  we could upper bound $\error_{\wt \calS_M}$ 
%     with $1-\error_\calD$  (in the proof of \lemref{lem:fit_mislabeled}), 
%     for multiclass classification, 
%     error on the mislabeled data 
%     and accuracy on the clean data 
%     in the population 
%     are not so directly related.  
%     To establish \eqref{eq:lemma1_multi},
%     we break the error on the 
%     (unknown) mislabeled data 
%     into two parts: one term corresponds 
%     to predicting the true label on mislabeled data, 
%     and the other corresponds to predicting 
%     neither the true label 
%     nor the assigned (mis-)label.  
%     Finally, we relate these errors to their
%     population counterparts to establish \eqref{eq:lemma1_multi}. 
    
%     Formally, 
%     \begin{align}
%     f^* \defeq \argmin_{f \in \calF} \error_{\widecheck {\calD}} (f)  + \lambda R(f) \,, \label{eq:modified_ERM_reg2}
%     \end{align}
%     where $$\widecheck \calD = \frac{n}{m+n} \calS + \frac{m_1}{m+n} \wt \calS_C  + \frac{m_2}{m+n}\calDm \,.$$ That is, $\widecheck \calD$ a combination of 
%     the \emph{empirical distribution} 
%     over correctly labeled data $S \cup \wt S_C$
%     % in $S\cup \wt S$ 
%     and the (population) distribution 
%     over mislabeled data $\calDm$.
%     Recall that 
%     \begin{align}
%     \wh f \defeq \argmin_{f \in \calF} \error_{\calS \cup \wt S} (f) + \lambda R(f) \,. \label{eq:orig_ERM_reg2}
%     \end{align}
%     % 
%     % 
%     Following the exact steps from the proof of \lemref{lem:lemma1_reg}, with probability at least $1-\delta$, we have  
%     \begin{align}
%         \error_{ \wt \calS_M}(\wh f) \le \error_{\calDm}(\wh f) + \sqrt{\frac{\log(1/\delta)}{2 m_1}} \,. \label{eq:lemma1_final_multi_prev}
%     \end{align}
%     Similar to before, since $\wt S$ is obtained by randomly labeling an unlabeled dataset, we assume 
%     $\frac{k}{k-1} m_1 \approx m$. 
    
%     Now we will relate $\error_\calDm (\wh f)$ with $\error_{\calD}(\wh f)$. Let $y^T$ denote the (unknown) true label for a mislabeled point $(x, y)$ (i.e., label before replacing it with a mislabel). 
%     \begin{align}    
%          \Expt{(x, y) \in \sim \calDm}{\indict{ \wh f(x) \ne y }}  &= \underbrace{\Expt{(x, y) \in \sim \calDm}{\indict{ \wh f(x) \ne y \land \wh f(x) \ne y^T}}}_{\RN{1}} + \underbrace{\Expt{(x, y) \in \sim \calDm}{\indict{ \wh f(x) \ne y \land \wh f(x) = y^T}}}_{\RN{2}} \,. \label{eq:excess_term}
%     \end{align}
%     Clearly, term 2 is one minus the accuracy on the clean unseen data, i.e. 
%     \begin{align}
%         \RN{2} = 1 - \Expt{{x,y} \sim \calD}{ \indict{ \wh f(x) \ne y}} = 1- \error_{\calD}(\wh f) \,. \label{eq:term1}    
%     \end{align}
%     Next, we  relate term 1 with the error on the unseen clean data. We show that term 1 is equal to the error on the unseen clean data scaled by $\frac{k-2}{k-1}$ where $k$ is the number of labels. Using the definition of mislabeled distribution $\calDm$,  we have 
%     \begin{align}
%         \RN{1} = \frac{1}{k-1} \left( \Expt{(x, y) \in \sim \calD}{ \sum_{i \in \calY \land i\ne y}  \indict{ \wh f(x) \ne i \land \wh f(x) \ne y}} \right) = \frac{k-2}{k-1} \error_{\calD}(\wh f) \,.\label{eq:term2}
%     \end{align}    

%     Combining the result in \eqref{eq:term1}, \eqref{eq:term2} and \eqref{eq:excess_term}, we have 
%     \begin{align}
%         \error_{\calDm}(\wh f) = 1- \frac{1}{k-1} \error_{\calD}(\wh f) \,.\label{eq:combine_terms}
%     \end{align}
%     Finally, combining the result in \eqref{eq:combine_terms} with equation \eqref{eq:lemma1_final_multi_prev}, we have with probability $1-\delta$, 
%     \begin{align}
%       \error_{\calD}(\wh f) \le  (k-1) \left( 1- \error_{ \wt \calS_M}(\wh f) \right)  + (k-1) \sqrt{\frac{k \log(1/\delta)}{ 2(k-1)m}} \,. \label{eq:lemma1_final_multi}
%     \end{align}
% \end{proof}

% \begin{lemma} \label{lem:mislabeled_error_multi}
%     Assume the same setup as \thmref{thm:multiclass_ERM}.  Then for any $\delta >0$, with probability at least $1-\delta$ over the random draws of $\wt S$, we have  
%     % \begin{align}
%         $$\abs{k\error_{\wt \calS}(\widehat f) - \error_{\wt \calS_C}(\widehat f) -  (k-1)\error_{\wt \calS_M}(\widehat f) } \le  2k\sqrt{\frac{\log(4/\delta)}{2m}}\,. $$ % \label{eq:lemma2}
%     % \end{align}   
%     %  for some constant $c_3 \le 1.0\,$.
% \end{lemma} 


% \begin{proof}
%     Recall $\error_{\wt S} (f) = \frac{m_1}{m} \error_{\wt S_M}(f) + \frac{m_2}{m} \error_{\wt S_C}(f)$. Hence, we have 
%     \begin{align}
%         k\error_{\wt S}(f) - (k-1)\error_{\wt S_M}(f) - \error_{\wt S_C}(f) &= (k-1)\left(\frac{k m_1}{(k-1) m} \error_{\wt S_M}(f) - \error_{\wt S_M}(f)\right) + \left(\frac{km_2}{m} \error_{\wt S_C}(f) - \error_{\wt S_C}(f)\right) \\ &= k \left[ \left(\frac{m_1}{m} - \frac{k-1}{k}\right) \error_{\wt S_M}(f) + \left(\frac{m_2}{m} - \frac{1}{k} \right) \error_{\wt S_C} (f) \right] \,.
%     \end{align} 
%     Since the dataset is randomly labeled, we have with probability at least $1-\delta$, $\left(\frac{m_1}{m} - \frac{k-1}{k}\right) \le \sqrt{\frac{\log(1/\delta)}{2m}}$. Similarly, we have with probability at least $1-\delta$, $\left(\frac{m_2}{m} - \frac{1}{k}\right) \le \sqrt{\frac{\log(1/\delta)}{2m}}$. Using union bound, we have with probability at least $1-\delta$
%     % \begin{align}
%     %     2\error_{\wt S} - \error_{\wt S_M}(f) - \error_{\wt S_C}(f) \le \sqrt{\frac{\log(2/\delta)}{2m}} \left(\error_{\wt S_M}(f) + \error_{\wt S_C}(f) \right) \le 2\sqrt{\frac{\log(2/\delta)}{2m}} \,. \label{eq:lemma2_final}
%     % \end{align}
%     \begin{align}
%         k\error_{\wt S}(f) - (k-1)\error_{\wt S_M}(f) - \error_{\wt S_C}(f)  \le k \sqrt{\frac{\log(2/\delta)}{2m}} \left(\error_{\wt S_M}(f) + \error_{\wt S_C}(f) \right) \,. \label{eq:lemma2_final_multi}
%     \end{align}

%     % We obtain the desired result by using 
% \end{proof}

% \begin{lemma} \label{lem:clear_error_multi}
%     Assume the same setup as \thmref{thm:multiclass_ERM}. 
%     Then for any $\delta >0$, with probability at least $1-\delta$ 
%     over the random draws of $\wt S_C$ and $S$, we have 
%     % \begin{align}
%         $$\abs{\error_{\wt \calS_C}(\widehat f) - \error_{\calS}(\widehat f) } \le 1.5 \sqrt{\frac{k\log(2/\delta)}{2m}}\,.$$ %\label{eq:lemma3}
%     % \end{align}   
%     % for some constant $c_2 \le 1.2\,$.
% \end{lemma} 
% \begin{proof}
%     % Recall 0-1 error on each point  $(x,y) \in S \cup \wt S$ is given by $\I{ f(x)\ne y}$.
%     In the set of correctly labeled points $S \cup \wt S_C$, we have $S$ as a random subset of $S \cup \wt S_C$. Hence, using Hoeffding's inequality for sampling without replacement (\lemref{lem:hoeffding_sampling}), we have with probability at least $1-\delta$
%     \begin{align}
%         \error_{\wt \calS_c} (\wh f)- \error_{\calS \cup \wt \calS_C}( \wh f) \le  \sqrt{\frac{\log(1/\delta)}{2m_2}} \,.
%     \end{align}
%     Re-writing $\error_{\calS \cup \wt \calS_C}( \wh f)$ as $\frac{m_2}{m_2 + n} \error_{\wt \calS_C }(\wh f) + \frac{n}{m_2 + n} \error_{\calS }(\wh f)$, we have with probability at least $1-\delta$
%     \begin{align}
%       \left(\frac{n}{n+m_2}\right) \left(\error_{\wt \calS_c} (\wh f)- \error_{\calS}( \wh f) \right) \le  \sqrt{\frac{\log(1/\delta)}{2m_2}} \,.
%     \end{align}
%     As before, assuming $km_2 \approx m$, we have with probability at least $1-\delta$ 
%     \begin{align}
%         \error_{\wt \calS_c} (\wh f)- \error_{\calS}( \wh f) \le \left(1+\frac{m_2}{n}\right)  \sqrt{\frac{k\log(1/\delta)}{2m}} \le \left( 1 + \frac{1}{k}\right) \sqrt{\frac{k\log(1/\delta)}{2m}} \,. \label{eq:lemma3_final_multi}
%     \end{align} 
% \end{proof}

% \begin{proof}[Proof of \thmref{thm:multiclass_ERM}] 
%     Having established these core intermediate results, we can now combine above three lemmas. 
%     In particular, we bound the population error on clean data ($\error_\calD(\wh f)$) as follows:  
%     \begin{enumerate}[(i)]
%         \item First, use \eqref{eq:lemma1_final_multi}, to obtain an upper bound on the population error on clean data, i.e., with probability at least $1-\delta/4$, we have
%         \begin{align}
%             \error_{ \calD} (\wh f) \le (k-1)\left(1 - \error_{ \wt \calS_M}(\wh f) \right) + (k-1) \sqrt{\frac{k\log(4/\delta)}{2(k-1)m}} \,. 
%         \end{align}
%         \item  Second, use \eqref{eq:lemma2_final_multi}, to relate the error on the mislabeled fraction with error on clean portion of randomly labeled data and error on whole randomly labeled dataset, i.e., with probability at least $1-\delta/2$, we have 
%         \begin{align}
%             - (k-1)\error_{\wt S_M}(f) \le \error_{\wt S_C}(f) - k\error_{\wt S}  + k\sqrt{\frac{\log(4/\delta)}{2m}}  \,. 
%         \end{align} 
%         \item Finally, use \eqref{eq:lemma3_final_multi} to relate the error on the clean portion of randomly labeled data and error on clean training data, i.e., with probability $1-\delta/4$, we have 
%         \begin{align}
%             \error_{\wt \calS_C} (\wh f)\le - \error_{\calS}( \wh f) + \left(1 + \frac{m}{kn} \right) \sqrt{\frac{k\log(4/\delta)}{2m}} \,. 
%         \end{align} 
%     \end{enumerate}

%     Using union bound on the above three steps, we have with probability at least $1-\delta$: 
%     \begin{align}
%         \error_\calD (\wh f) \le \error_{\calS}(\wh f) + (k-1) - k\error_{\wt \calS}(\wh f)   + (\sqrt{k(k-1)} + k + \sqrt{k} + \frac{m}{n\sqrt{k}})  \sqrt{\frac{\log(4/\delta)}{2m}} \,.
%     \end{align}
%     % Note that $\frac{m}{n\sqrt{k}}$ is much smaller than the other terms in addition. Hence, we ignore this in the final bound. 
%     % Note that $(1/\sqrt{2} + 2.5)$ is a loose constant. In experiments, we use the ratio $\frac{m}{n}$
%     %  the exact error $\error_{\wt \calS}(\wh f)$ 
%     % to evaluate R.H.S.    
% \end{proof}

% \newpage
% \section{Proofs from \secref{sec:linear_models}}\label{app:proof_gd}

% We suppose that the parameters of the linear function 
% are obtained via gradient descent on 
% the following $L_2$ regularized problem: 
% \begin{align}
%     % n in denominator is avoided deliberately
%     \calL_S(w; \lambda) \defeq \sum_{i=1}^n{(w^Tx_i - y_i)^2} + \lambda \norm{w}{2}^2 \,, \label{eq:l2_MSE_app}   
% \end{align}
% where $\lambda\ge0$ is a regularization parameter. 
% We assume access to a clean dataset 
% $S = \{(x_i, y_i)\}_{i=1}^n \sim \calD^n$ 
% and randomly labeled dataset 
% $\wt S = \{(x_i, y_i)\}_{i=n+1}^{n+m} \sim \wt \calD^m$. 
% Let $\bX = [x_1, x_2, \cdots, x_{m+n}]$ 
% and $\by = [y_1, y_2, \cdots, y_{m+n}]$. 
% Fix a positive learning rate $\eta$ such that 
% $\eta \le 1/\left(\norm{\bX^T\bX}{\text{op}} + \lambda^2\right)$ 
% and an initialization $w_0 = 0$. 
% % \todos{Assumption made for simplicty}. 
% Consider the following gradient descent iterates 
% to minimize objective \eqref{eq:l2_MSE_app} on $S \cup \wt S$:
% \begin{align}
% w_t = w_{t-1} - \eta \grad_w \calL_{S \cup \wt S} (w_{t-1}; \lambda) \quad \forall t=1,2,\ldots \label{eq:GD_iterates_app}
% \end{align} 
% Then we have $\{ w_t\}$ converge to the limiting solution 
% $\wh w = \left( \bX^T\bX+\lambda \boldsymbol{I}\right)^{-1}\bX^T\by$. Define $\widehat f (x) \defeq f(x ; \wh w) $.  

% \subsection{\textcolor{red}{Errata}}

% We wish to correct the following error in the body: \codref{cond:error_stability} is not enough to guarantee the result in \thmref{thm:linear}. We now present a slightly stronger condition called \emph{hypothesis stability} under which we obtain a result similar to \thmref{thm:linear}. 

% This error doesn't change the main arguments of the proof where we show that the empirical train error is less than or equal to the leave-one-out error. We need a stronger condition to relate leave-one-out error with the population error of the original classifier. Specifically, while \codref{cond:error_stability} relates the average population error of leave-one-out classifiers with the population error of the original classifier, we need the new condition to show the concentration of the empirical leave-one-out error and  average population error of leave-one-out classifiers. 
% % main takeaway 

% Note that the new condition, while being stronger than the previous one, still doesn't imply generalization~\cite{bousquet2002stability,elisseeff2003leave,abou2019exponential}. Overall, the main results in \secref{sec:ERM_training} and takeaways of the paper remain unaffected by the error.  

% We now present the new condition and a corrected statement of \thmref{thm:linear}. Recall, for a given training set $S \sim \calD^n $, 
% we use $S_{(i)}$ to denote the training set $S$ 
% with the $i^{\text{th}}$ point removed.

% \begin{condition}[Hypothesis Stability] 
%     \label{cond:hypothesis_stability}
%     We have $\beta$ hypothesis stability 
%     if our training algorithm $\calA$ satisfies the following: 
%     \begin{align*}
%     % ${\sum_{i=1}^n \frac{\error_{\calD}( f(\calA, S_{(i)}))}{n} - \error_\calD(f(\calA, S))} \le \beta\,$.
%     \forall i \in \{1,2,\ldots, n\}, \quad  \Expt{\calS, (x,y) \in \calD}{ \abs{\error\left( f(x) ,y  \right) - \error\left( f_{(i)}(x), y \right) }} \le \frac{\beta}{n} \,,
%     \end{align*}
%     where $f_{(i)} \defeq f(\calA, S_{(i)})$ and $ f \defeq f(\calA, S)$.
% \end{condition}

% \begin{theorem}[Correct statement of \thmref{thm:linear}] \label{thm:new_linear}
%     Assume that this gradient descent algorithm satisfies \codref{cond:hypothesis_stability}
%     with $\beta=\calO(1)$.  
%     Then for any $\delta >0$, with probability at least $1-\delta$ 
%     over the random draws of datasets $\wt S$ and $S$, we have:
%     \begin{align}
%         \error_\calD(\widehat f) \le \error_\calS(\widehat f) + 1 - 2 \error_{\wt\calS}(\widehat f) + \left(\frac{1}{\sqrt{2}} + 1.5 \right) \sqrt{\frac{\log(4/\delta)}{m}} + \sqrt{\frac{4}{\delta}\left(\frac{1}{m} +\frac{3\beta}{m+n} \right)}  \,. \label{eq:gd_error}
%     \end{align} 
%     % for some constant $c\le 3.2$.
% \end{theorem}

% \subsection{Proof of \thmref{thm:new_linear}}
% We use a standard result from linear algebra, namely Shermann-Morrison formula~\citep{sherman1950adjustment} for matrix inversion:  

% \begin{lemma}[\citet{sherman1950adjustment}] \label{lem:sherman}
%     Suppose $\bA \in \Real^{n \times n}$ is an invertible square matrix and $u,v \in \Real^n$ are column vectors. Then $\bA + uv^T$ is invertible iff $1 + v^T \bA u \ne 0$ and in particular
%     \begin{align}
%         (\bA + u v^T)^{-1} = \bA^{-1}  - \frac{\bA^{-1} uv^T \bA^{-1} }{ 1 + v^T \bA^{-1} u} \,.
%     \end{align}   
% \end{lemma}
% \newcommand\byy[1]{\by_{\left(#1\right)}}
% \newcommand\bXX[1]{\bX_{\left(#1\right)}}
% \newcommand\ff[1]{\wh f_{\left(#1\right)}}

% For a given training set $S \cup \wt S_C$, define leave-one-out error on mislabeled points in the training data as $$\error_{\text{LOO}(\wt S_M) } = \frac{\sum_{(x_i, y_i) \in \wt S_M} \error( f_{(i)}( x_i), y_i)}{ \abs{\wt S_M }} \,, $$
% where $f_{(i)} \defeq f(\calA, (S \cup \wt S)_{(i)})$. To relate empirical leave-one-out error and population error with hypothesis stability condition, we use the following lemma:   

% \begin{lemma}[\citet{bousquet2002stability}] \label{lem:stability_error}
%     For the leave-one-out error, we have
%     \begin{align}
%         \Expo{ \left( \error_{\calDm}(\wh f) -\error_{\text{LOO}(\wt S_M) } \right)^2 } \le \frac{1}{2m_1}+  \frac{3\beta}{n + m}\,.
%     \end{align}   
%     % where $ f \defeq f(\calA, S \cup \wt S) $.
% \end{lemma}

% Proof of the above lemma is similar to the proof of  Lemma 9 in \citet{bousquet2002stability} and can be found in \appref{app:proof_lem_error}. 
% % 
% % Before presenting the result, we introduce some notation. 
% Before presenting the proof of \thmref{thm:new_linear}, we introduce some more notation. Let $\bX_{(i)}$ denote the matrix of covariates with $i^{\text{th}}$ point removed. Similarly let $\by_{(i)}$ be the array of responses with $i^{\text{th}}$ point removed. Define the corresponding regularized GD solution as $\wh w_{(i)} = \left( \bXX{i}^T\bXX{i}+\lambda \boldsymbol{I}\right)^{-1}\bXX{i}^T\byy{i}$. Define $\ff{i}(x) \defeq f(x ; \wh w_{(i)}) $.

% \begin{proof}[Proof of \thmref{thm:new_linear}]
%     Because squared loss minimization does not imply 0-1 error minimization, we cannot use arguments from \lemref{lem:fit_mislabeled}. This is the main technical difficulty. To compare the 0-1 error at a train point with an unseen point, 
%     we use the closed-form expression for $\widehat{w}$ and Shermann-Morrison formula to upper bound training error with leave-one-out cross validation error. 
    
%     The proof is divided into three parts: In part one, we show that 0-1 error on mislabeled points in the training set is lower than the error obtained by leave-one-out error at those points. In part two, we relate this leave-one-out error with the population error on mislabeled distribution using \codref{cond:hypothesis_stability}. While the empirical leave-one-out error is unbiased estimator of the average population error of leave-one-out classifiers, we need hypothesis stability to control the variance of empirical leave-one-out error. Finally in part three, we show that the error on the mislabeled training points can be estimated with just the randomly labeled and  clean training data (as in proof of \thmref{thm:error_ERM}).  

%     \textbf{Part 1 {} {}} First we relate training error with leave-one-out error.        
%     For any 
%     training point $(x_i, y_i)$ in $\wt S \cup S$, we have 
%     \begin{align}
%         \error(\wh f(x_i), y_i ) &= \indict{ y_i \cdot x_i^T \wh w < 0 } = \indict{ y_i \cdot x_i^T \left( \bX^T\bX+\lambda \boldsymbol{I}\right)^{-1}\bX^T\by < 0 } \\
%         &= \indict{ y_i \cdot x_i^T \underbrace{\left( \bXX{i}^T\bXX{i} + x_i ^T x_i +\lambda \boldsymbol{I}\right)^{-1}}_{\RN{1}} (\bXX{i}^T\byy{i} + y \cdot x_i) < 0 }
%     \end{align}
%     Letting $\bA = \left(\bXX{i}^T\bXX{i} +\lambda \boldsymbol{I}\right)$ and using \lemref{lem:sherman} on term 1, we have 
%     \begin{align}
%         \error(\wh f(x_i), y_i ) &= \indict{ y_i \cdot x_i^T \left[\bA^{-1} -  \frac{\bA^{-1} x_i x_i^T \bA^{-1}}{ 1 + x_i ^T \bA^{-1} x_i } \right] (\bXX{i}^T\byy{i} + y \cdot x_i) < 0 } \\
%         &= \indict{ y_i \cdot\left[ \frac{ x_i^T \bA^{-1} ( 1 + x_i ^T \bA^{-1} x_i ) -  x_i^T \bA^{-1} x_i x_i^T \bA^{-1}}{ 1 + x_i ^T \bA ^{-1}x_i } \right] (\bXX{i}^T\byy{i} + y \cdot x_i) < 0 } \\
%         &= \indict{ y_i \cdot\left[ \frac{ x_i^T \bA^{-1}}{ 1 + x_i ^T \bA ^{-1}x_i } \right] (\bXX{i}^T\byy{i} + y \cdot x_i) < 0 } \,.
%     \end{align}

%     Since $1 + x_i^T \bA^{-1} x_i > 0$, we have 
%     \begin{align}
%         \error(\wh f(x_i), y_i ) &= \indict{ y_i \cdot x_i^T \bA^{-1} (\bXX{i}^T\byy{i} + y \cdot x_i) < 0 } \\
%         &= \indict{ x_i^T \bA^{-1} x_i +  y_i \cdot x_i^T \bA^{-1} (\bXX{i}^T\byy{i}) < 0 } \\
%         &\le \indict{ y_i \cdot x_i^T \bA^{-1} (\bXX{i}^T\byy{i}) < 0 } = \error(\ff{i}(x_i), y_i ) \,.\label{eq:LOO_error}
%     \end{align}

%     Using \eqref{eq:LOO_error}, we have 
%     \begin{align}
%         \error_{\wt \calS_M } (\wh f) \le \error_{\text{LOO} (S_M)} \defeq \frac{\sum_{(x_i, y_i) \in \wt S_M} \error(\ff{i}(x_i), y_i ) }{\abs{\wt \calS_M}}\label{eq:LOO_error_final}
%     \end{align}
%     \textbf{Part 2 {}{}} We now relate RHS in \eqref{eq:LOO_error_final} with the population error on mislabeled distribution. To do this, we leverage \codref{cond:hypothesis_stability} and \lemref{lem:stability_error}. In particular, we have 

%     \begin{align}
%         \Expt{\calS \cup \wt \calS_M }{ \left(\error_{\calDm}(\wh f) - \error_{\text{LOO} (S_M)}\right)^2 } \le \frac{1}{2m_1} + \frac{3\beta}{m+n} \,.
%     \end{align}

%     Using Chebyshev's inequality, with probability at least $1-\delta$, we have 
%     \begin{align}
%         \error_{\text{LOO} (S_M)} \le  \error_{\calDm}(\wh f)   + \sqrt{\frac{1}{\delta}\left(\frac{1}{2m_1} +\frac{3\beta}{m+n} \right)} \,. \label{eq:final_mislabeled_linear}
%     \end{align}
    

%     \textbf{Part 3 {}{}} Combining \eqref{eq:final_mislabeled_linear} and \eqref{eq:LOO_error_final}, we have 

%     \begin{align}
%         \error_{\wt \calS_M } (\wh f) \le \error_{\calDm}(\wh f)   + \sqrt{\frac{1}{\delta}\left(\frac{1}{2m_1} +\frac{3\beta}{m+n} \right)} \,. \label{eq:linear_parallel_lem1}
%     \end{align}

%     Compare \eqref{eq:linear_parallel_lem1}, with \eqref{eq:lemma1_final} in the proof of \lemref{lem:fit_mislabeled}. We obtain a similar relationship between $\error_{\wt \calS_M }$ and $\error_{\calDm}$ but with a polynomial concentration instead of exponential concentration. 
%     In addition, since we just use concentration arguments to relate mislabeled error with the error on clean portion and unlabeled portion, we can directly use the results in \lemref{lem:mislabeled_error} and \lemref{lem:clear_error}. Therefore, combining results in \lemref{lem:mislabeled_error}, \lemref{lem:clear_error}, and \eqref{eq:linear_parallel_lem1} with union bound, we have with probability at least $1-\delta$

%     \begin{align}
%         \error_\calD(\widehat f) \le \error_\calS(\widehat f) + 1 - 2 \error_{\wt\calS}(\widehat f) + \left(\frac{1}{\sqrt{2}} + 1.5 \right) \sqrt{\frac{\log(4/\delta)}{m}} + \sqrt{\frac{4}{\delta}\left(\frac{1}{m} +\frac{3\beta}{m+n} \right)}  \,.
%     \end{align}
    

       
% \end{proof}

% \subsection{Discussion on \codref{cond:hypothesis_stability}}

% The quantity in LHS of \codref{cond:hypothesis_stability} measures how much the function learned by the algorithm (in terms of error on unseen point) will change when one point in the training set is removed. 
% % Discussion on exponential concentration and stronger condition. 
% Notice that hypothesis stability implies error stability, i.e., \codref{cond:error_stability} ~\cite{bousquet2002stability}.  In summary, while error stability allowed us to relate the average population error of the leave-one-out classifiers with the population error of the original classifier, we need hypothesis stability condition to control the variance of the empirical leave-one-out error. 

% Additionally, we note that while the dominating term in the RHS of \thmref{thm:new_linear} matches with the dominating term in ERM bound in \thmref{thm:error_ERM}, there is a polynomial concentration term (dependence on $1/\delta$ instead of $\log(\sqrt{1/\delta})$) in  \thmref{thm:new_linear}. 
% Since with hypothesis stability, we just bound the variance,  the polynomial concentration is due to the use of Chebyshev's inequality instead of an exponential tail inequality (as in \lemref{lem:fit_mislabeled}).
% Recent works have highlighted that slightly stronger condition than hypothesis stability can be used to obtained an exponential concentration for leave-one-out error~\citep{abou2019exponential}, but we leave this for future work for now. 
% % We leave 
% % However, the constants 

% % we also want to highlight  

% \subsection{Formal statement and proof of  of \propref{prop:early_stop}}

% Before formally presenting the result, we will introduce some notation.  By $\calL_{S}(w)$, we denote 
% the objective in \eqref{eq:l2_MSE_app} with $\lambda=0$. 
% Assume Singular Value Decomposition (SVD) of $\bX$  as $\sqrt{n} \bU \bS^{1/2} \bV^T$. Hence $\bX^T \bX = \bV \bS \bV^T$.
% Consider the GD iterates defined in \eqref{eq:GD_iterates_app}. 
% % 
% We now derive closed form expression for the $t^\text{th}$ iterate of gradient descent:  
% % 
% \begin{align}
%     w_t = w_{t-1} + \eta \cdot \bX^T (\by - \bX w_{t-1}) = (\bI - \eta \bV \bS \bV^T )w_{k-1} + \eta \bX^T \by \,.
% \end{align}
% Rotating by $\bV^T$, we get 
% \begin{align}
%     \wt w_t = (\bI - \eta\bS )\wt w_{k-1} + \eta \wt \by \,, \label{eq:GD_recur}
% \end{align}
% where $\wt w_t = \bV^T w_t $ and $\wt \by = \bV^T \bX^T \by$. Assuming the initial point $w_0 = 0$ and applying the recursion in \eqref{eq:GD_recur}, we get
% \begin{align}
%     \wt w_t = \bS ^{-1} ( \bI - (\bI - \eta \bS)^k ) \wt \by \,, 
% \end{align} 
% Projecting solution back to the original space, we have 
% \begin{align}
%      w_t = \bV \bS ^{-1} ( \bI - (\bI - \eta \bS)^k ) \bV^T \bX^T \by \,, 
% \end{align} 
% % We will work with this GD solution at any iterate $t$ in the next proposition. 
% Define $f_t(x) \defeq f(x;w_t)$ as the solution at the $t^{\text{th}}$ iterate. 
% Let $\wt w_{\lambda} = \argmin_{w} \calL_\calS (w;\lambda) = (\bX^T \bX + \lambda \bI)^{-1} \bX^T \by = \bV (\bS + \lambda \bI )^{-1} \bV^T \bX^T \by $. 
% % ) \,,$ for all $t=1,2,\ldots\,.$ 
% and define $\wt f_\lambda(x) \defeq f(x;\wt w_\lambda)$ as the regularized solution. 
% Assume $\kappa$ be the condition number of the population covariance matrix 
% and 
% let $s_\text{min}$ be the minimum positive singular value of the empirical covariance matrix. Our proof idea is inspired from recent work on relating gradient flow solution and regularized solution for regression problems \citep{ali2018continuous}. We will use the following lemma in the proof: 
% \begin{lemma} \label{lem:ineq_soln}
%     For all $x \in [0,1]$ and for all $ k \in \mathbb{N}$, we have (a) $ \frac{kx}{1+kx} \le 1- (1-x)^k$ and (b) $ 1- (1-x)^k \le 2 \cdot \frac{kx}{kx+1} $.
%     %  where $g(c)$ is a constant dependent on $c$. For $c = 1$, $g(c) = 2.0$.   
% \end{lemma}
% \begin{proof}
%     % [Proof of \lemref{lem:ineq_soln}]
%     % Part (a) is easy. 
%     Using $ (1-x)^k \le \frac{1}{1+kx}$, we have part (a). For part (b), we numerically maximize $\frac{ (1+kx ) (1 - (1-x)^k) }{kx}$ for all $k\ge 1$ and for all $x \in [0, 1]$.  
% \end{proof}

% % 
% % Next, 

% \begin{prop}[Formal statement of \propref{prop:early_stop}] \label{prop:formal_early_stop}
% Let $\lambda = \frac{1}{t\eta}$. For a training point $x$, we have 
% \begin{align*}
%     \Expt{x \sim \calS}{(f_t(x) - \wt f_\lambda(x))^2} &\le c(t,\eta) \cdot \Expt{x \sim \calS}{f_t(x)^2} \,, %\label{eq:early_stop}
% \end{align*}
% where $c(t, \eta) \defeq \min( 0.25, \frac{1}{s_\text{min}^2 t^2 \eta^2})$. Similarly for a test point, we have 
% \begin{align*}
%     \Expt{x \sim \calD_\calX}{(f_t(x) - \wt f_\lambda(x))^2} &\le \kappa \cdot c(t,\eta) \cdot \Expt{x \sim \calD_\calX}{f_t(x)^2} \,. %\label{eq:early_stop}
% \end{align*}
% \end{prop} 

% \begin{proof}
%     %%%%%%%%%%%%% 
%     We want to analyze the expected squared difference output of regularized linear regression with regularization constant $\lambda = \frac{1}{\eta t}$ and gradient descent solution at $t^\text{th}$ iterate. We separately expand the algebraic expression for squared difference at a training point and a test point. 
%     % We start by considering the difference  
%     Then the main step is to show that  $\left[ \bS ^{-1} ( \bI - (\bI - \eta \bS)^k )  - (\bS + \lambda \bI )^{-1}\right] \preceq c(\eta, t) \cdot \bS ^{-1} ( \bI - (\bI - \eta \bS)^k ) $.

%     %%%%%%%%%%%%%
    
%   \textbf{Part 1 {} {}} 
%     First, we will analyze the squared difference of output at a training point (for simplicity, we refer to $S \cup \wt S$ as $S$), i.e. 
%     \begin{align}
%         \Expt{ x \sim \calS }{\left(f_t(x) - \wt f_\lambda (x)\right)^2} &= \norm{\bX w_t - \bX \wt w_\lambda}{2}^2 =   \norm{\bX \bV \bS ^{-1} ( \bI - (\bI - \eta \bS)^t ) \bV^T \bX^T \by - \bX \bV (\bS + \lambda \bI )^{-1} \bV^T \bX^T \by }{2}^2 \\
%         &= \norm{\bX \bV \left(\bS ^{-1} ( \bI - (\bI - \eta \bS)^t ) - (\bS + \lambda \bI )^{-1} \right) \bV^T \bX^T \by  }{2} \\
%         &=  \by^T \bV \bX \left( \underbrace{\bS ^{-1} ( \bI - (\bI - \eta \bS)^t ) - (\bS + \lambda \bI )^{-1}}_{\RN{1}} \right)^2 \bS \bV^T \bX^T \by \label{eq:train_GD_rel}
%         %  (\bX \bV \bS ^{-1} ( \bI - (\bI - \eta \bS)^k ) \bV^T \bX^T \by)^T \bX \bV \bS ^{-1} ( \bI - (\bI - \eta \bS)^k ) \bV^T \bX^T \by
%     \end{align}
%     We now separately consider term 1. Substituting $\lambda = \frac{1}{t \eta}$, we get
%     \begin{align}
%         \bS ^{-1} ( \bI - (\bI - \eta \bS)^t ) - (\bS + \lambda \bI )^{-1} &= \bS^{-1} \left( ( \bI - (\bI - \eta \bS)^t ) - (\bI + \bS^{-1} \lambda )^{-1}\right) \\
%         &= \underbrace{\bS^{-1} \left( ( \bI - (\bI - \eta \bS)^t ) - (\bI + ( \bS t \eta)^{-1}  )^{-1}\right)}_{\bA}
%     \end{align}

%     We now separately bound the diagonal entries in matrix $\bA$. 
%     With $s_i$, we denote $i^{\text{th}}$ diagonal entry of $\bS$. Note that since $ \eta\le 1/\norm{S}{\text{op}}$, for all $i$, $\eta s_i  \le 1$.  Consider $i^{\text{th}}$ diagonal term (which is non-zero) of the diagonal matrix $\bA$, we have 
%     \begin{align}
%         \bA_{ii} = \frac{1}{s_i} \left(  1 - (1 - s_i \eta)^t - \frac{t \eta s_i}{1 + t \eta s_i } \right) &=  \frac{1 - (1 - s_i \eta)^t}{s_i} \left( \underbrace{ 1 - \frac{t \eta s_i}{(1 + t \eta s_i)(1 - (1 - s_i \eta)^t)}}_{\RN{2}} \right) \\ 
%          &\le \frac{1}{2}\left[ \frac{1 - (1 - s_i \eta)^t}{ s_i} \right] \tag*{(Using \lemref{lem:ineq_soln} (b))} \,.
%     \end{align} 
%     Additionally, we can also show the following upper bound on term 2: 
%     \begin{align}
%          1 - \frac{t \eta s_i}{(1 + t \eta s_i)(1 - (1 - s_i \eta)^t)} &= \frac{(1 + t \eta s_i)(1 - (1 - s_i \eta)^t) - t \eta s_i }{(1 + t \eta s_i)(1 - (1 - s_i \eta)^t)} \\
%          & \le  \frac{ 1 -  (1 - s_i \eta)^t - t \eta s_i (1 - s_i \eta)^t}{(1 + t \eta s_i)(1 - (1 - s_i \eta)^t)} \\
%          & \le \frac{1}{t\eta s_i} \,. \tag{Using \lemref{lem:ineq_soln} (a)}
%         %  &\le \frac{1}{2}\left[ \frac{1 - (1 - s_i \eta)^t}{ s_i} \right] \tag*{(Using \lemref{lem:ineq_soln})} \,.
%     \end{align} 

%     Combining both the upper bounds on each diagonal entry $\bA_{ii}$, we have 
%     \begin{align}
%     \bA \preceq c_1(\eta, t) \cdot \bS^{-1} ( \bI - (\bI - \eta \bS)^t ) \,, \label{eq:upperbound_diagonal}
%     \end{align}
%     where $c_1(\eta, t ) = \min(0.5, \frac{1}{t s_i \eta })$. Plugging this into \eqref{eq:train_GD_rel}, we have 
%     \begin{align}
%         \Expt{ x \sim \calS }{\left(f_t(x) - \wt f_\lambda (x)\right)^2} &\le c(\eta, t) \cdot \by^T \bV \bX  \left( \bS^{-1} ( \bI - (\bI - \eta \bS)^t ) \right)^2 \bS \bV^T \bX^T \by \\
%         &=   c(\eta, t) \cdot \by^T \bV \bX  \left( \bS^{-1} ( \bI - (\bI - \eta \bS)^t ) \right) \bS \left( \bS^{-1} ( \bI - (\bI - \eta \bS)^t ) \right) \bV^T \bX^T \by \\
%         & =  c(\eta, t) \cdot \norm{\bX w_t}{2}^2 \\
%         &= c(\eta, t) \cdot  \Expt{ x \sim \calS }{\left(f_t(x) \right)^2} \,,
%     \end{align}
%     where $c(\eta, t ) = \min(0.25, \frac{1}{t^2 s^2_i \eta^2 })$.

%     \textbf{Part 2 {} {}} With $\bSigma$, we denote the underlying true covariance matrix. We now consider the squared difference of output at an unseen point: 
%     \begin{align}
%         \Expt{ x \sim \calD_{\calX} }{\left(f_t(x) - \wt f_\lambda (x)\right)^2} &= \Expt{x \sim \calD_{\calX}}{\norm{x^T w_t - x^T \wt w_\lambda}{2}} \\
%         &=   \norm{x^T \bV \bS ^{-1} ( \bI - (\bI - \eta \bS)^t ) \bV^T \bX^T \by - x^T \bV (\bS + \lambda \bI )^{-1} \bV^T \bX^T \by }{2} \\
%         &= \norm{x^T \bV \left(\bS ^{-1} ( \bI - (\bI - \eta \bS)^t ) - (\bS + \lambda \bI )^{-1} \right) \bV^T \bX^T \by  }{2} \\
%         &= \by^T \bV \bX \left( \bS ^{-1} ( \bI - (\bI - \eta \bS)^t ) - (\bS + \lambda \bI )^{-1} \right) \bV^T \bSigma \bV \\ &\qquad \qquad \qquad \qquad \qquad \left( (\bI - (\bI - \eta \bS)^t ) - (\bS + \lambda \bI )^{-1} \right) \bV^T \bX^T \by \\
%         &\le \sigma_{\text{max}} \cdot \by^T \bV \bX \left( \underbrace{\bS ^{-1} ( \bI - (\bI - \eta \bS)^t ) - (\bS + \lambda \bI )^{-1}}_{\RN{1}} \right)^2 \bV^T \bX^T \by \,, \label{eq:test_GD_rel}
%         %  (\bX \bV \bS ^{-1} ( \bI - (\bI - \eta \bS)^k ) \bV^T \bX^T \by)^T \bX \bV \bS ^{-1} ( \bI - (\bI - \eta \bS)^k ) \bV^T \bX^T \by
%     \end{align}
%     where $\sigma_{\text{max}}$ is the maximum eigenvalue of the underlying covariance matrix $\bSigma$. Using the upper bound on term 1 in \eqref{eq:upperbound_diagonal}, we have 
%     \begin{align}
%         \Expt{ x \sim \calD_{\calX} }{\left(f_t(x) - \wt f_\lambda (x)\right)^2} &\le \sigma_{\text{max}} \cdot c(\eta, t) \cdot \by^T \bV \bX  \left( \bS^{-1} ( \bI - (\bI - \eta \bS)^t ) \right)^2 \bV^T \bX^T \by \\
%         &=   \kappa \cdot c(\eta, t) \cdot \sigma_{\text{min}}\cdot \norm{\bV \left( \bS^{-1} ( \bI - (\bI - \eta \bS)^t ) \right) \bV^T \bX^T \by}{2}^2 \\
%         &\le \kappa \cdot c(\eta, t) \cdot \left[ \bV \left( \bS^{-1} ( \bI - (\bI - \eta \bS)^t ) \right) \bV^T \bX^T \right]^T \bSigma \\
%         &\qquad \qquad \qquad \qquad \qquad \left[ \bV \left( \bS^{-1} ( \bI - (\bI - \eta \bS)^t ) \right) \bV^T \bX^T \right] \by \\
%         & = \kappa \cdot c(\eta, t) \cdot \Expt{x \sim \calD_{\calX}}{\norm{x^T w_t}{2}} \,.
%     \end{align}
% % 
% % 
%     % Since $ \eta\le 1/\norm{S}{\text{op}}$, invoking \lemref{lem:ineq_soln} to upper bound term 1 with
% \end{proof}


% \newpage
% \section{Additional experiments and details}\label{app:exp}
% \newcommand\tab[1][1cm]{\hspace*{#1}}

% \subsection{Datasets} \label{sec:app_dataset}

% \textbf{Toy Dataset {} {}} Assume fixed constants $\mu$ and $\sigma$. For a given label $y$, we simulate features $x$ in our toy classification setup as follows: 
% \begin{align*}
%     x \defeq \texttt{concat} \left[ x_1, x_2\right] \quad \text{where} \quad  x_1 \sim  \calN( y \cdot \mu, \sigma^2 I_{d \times d}) \ \  \text{and} \ \  x_1 \sim  \calN( 0, \sigma^2 I_{d \times d}) \,.
% \end{align*}  
% % where $y$ is the true label and $x$ is the corresponding feature vector. 
% In experiements throughout the paper, we fix dimention $d=100$, $\mu = 1.0 $, and $\sigma = \sqrt{d}$. Intuitively, $x_1$ carries the information about the underlying label and $x_2$ is additional noise independent of the underlying label. 

% \textbf{CV datasets {} {}} We use MNIST~\citep{lecun1998mnist} and CIFAR10~\cite{krizhevsky2009learning}. 
% % For binary tasks, 
% We produce a binary variant from the multiclass classification problem by mapping classes $\{0,1,2,3,4\}$ to label $1$ and $\{ 5,6,7,8,9\}$ to label $-1$. For CIFAR dataset, we also use the standard data augementation of random crop and horizontal flip. PyTorch code is as follows: 

% \texttt{(transforms.RandomCrop(32, padding=4),\\
% \tab transforms.RandomHorizontalFlip())}

% \textbf{NLP dataset {} {}} We use IMDb Sentiment analysis~\citep{maas2011learning} corpus.  

% \subsection{Architecture Details} 

% All experiments were run on NVIDIA GeForce RTX 2080 Ti GPUs. We used PyTorch~\citep{NEURIPS2019a9015} and Keras with Tensorflow~\citep{abadi2016tensorflow} backend for experiments. 
% % , ELMo embeddings~\citep{Peters:2018}, and Hugging Face Transformers~\citep{wolf-etal-2020-transformers}. 

% \textbf{Linear model {} {}} For the toy dataset, we simulate a linear model with scalar output and the same number of parameters as the number of dimensions.   

% \textbf{Wide nets {} {}} To simulate the NTK regime, we experiment with $2-$layered wide nets. The PyTorch code for 2-layer wide MLP is as follows: 


% \texttt{ nn.Sequential( \\
% \tab     nn.Flatten(),\\
% \tab    nn.Linear(input\_dims, 200000, bias=True),\\
% \tab    nn.ReLU(),\\
% \tab    nn.Linear(200000, 1, bias=True)\\
% \tab     )}


% We experiment both (i) with the first layer fixed at random initialization; (ii)  and updating both layers' weights.     

% \textbf{Deep nets for CV tasks {} {}} We consider a 4-layered MLP. The PyTorch code for 4-layer MLP is as follows: 

% \texttt{ nn.Sequential(nn.Flatten(), \\
% \tab        nn.Linear(input\_dim, 5000, bias=True),\\
% \tab        nn.ReLU(),\\
% \tab        nn.Linear(5000, 5000, bias=True),\\
% \tab        nn.ReLU(),\\
% \tab        nn.Linear(5000, 5000, bias=True),\\
% \tab        nn.ReLU(),\\
% % \tab        nn.Linear(5000, 5000, bias=True),\\
% % \tab        nn.ReLU(),\\
% \tab        nn.Linear(1024, num\_label, bias=True)\\
% \tab        )}

% For MNIST, we use $1000$ nodes instead of $5000$ nodes in the hidden layer. 
% % 
% We also experiment with convolutional nets. In particular, we use ResNet18 \citep{he2016deep}. Implementation adapted from:  \url{https://github.com/kuangliu/pytorch-cifar.git}. 

% \textbf{Deep nets for NLP {} {}} We use a simple LSTM model with embeddings intialized with ELMo embeddings~\citep{Peters:2018}. Code adapted from: \url{https://github.com/kamujun/elmo_experiments/blob/master/elmo_experiment/notebooks/elmo_text_classification_on_imdb.ipynb} 

% We also evaluate our bounds with a BERT model. In particular, we fine-tune an off-the-shelf uncased BERT model~\citep{devlin2018bert}. Code adapted from Hugging Face Transformers~\citep{wolf-etal-2020-transformers}: \url{https://huggingface.co/transformers/v3.1.0/custom_datasets.html}. 


% \subsection{Additonal experiments}

% 1. SGD with linear models on cross entropy and MSE loss. 

% 2. CE loss and SGD. GD with MSE loss 

% 3. Binary MNIST with MLP. multiclass MNIST  

% \textbf{Results on CIFAR 10 {} {}} 
% % 
% We plot epoch wise error curve for results in \tabref{table:multiclass}. We observe the same trend as in \figref{fig:error_CIFAR10}. Additionally, we plot an \emph{oracle bound} obtained by tracking the error on mislabeled data which nevertheless were predicted as true label. To obtain an exact emprical value of the oracle bound, we need underlying true labels for the randomly labeled data. 
% % Note that our bound in \thmref{thm:multiclass_ERM}, lower bounds the accuracy as predicted by the oracle bound. 
% While with just access to extra unlabeled data we cannot calculate oracle bound, we note that the oracle bound is very tight and never violated in practice underscoring an importamt aspect of generalization in multiclass problems. This highlight that even a stronger conjecture may hold in multiclass classification, i.e., error on mislabeled data (where nevertheless true label was predicted) lower bounds the population error on the distribution of mislabeled data and hence, the error on (a specific) mislabeled portion predicts the population accuracy on clean data. 
% % 
% On the other hand, the dominating term of in \thmref{thm:multiclass_ERM} is loose when compared with the oracle bound. The main reason, we believe is the pessimistic upper bound in \eqref{eq:lemma1_final_multi_prev} in the proof of \lemref{lem:fit_mislabeled_multi}. We leave an investigation on this gap for future. 
% % of fit 

% % However, oracle bound highlights two . One,  



% \begin{figure}[h]
%     \centering 
%     % \vspace{-15pt}
%     % \includegraphics[width=0.9\linewidth]{example-image-a}
%     \includegraphics[width=0.4\linewidth]{figures/CIFAR10-FNN.pdf} \hfil
%     \includegraphics[width=0.4\linewidth]{figures/CIFAR10-Resnet.pdf}
%     % \includegraphics[width=0.9\linewidth]{figures/{CIFAR10_rn=0.1_lr=0.2_wd=0.005}.png}
%     % \vspace{-10pt}
%     \caption{ Per epoch curves for CIFAR10 corresponding results in \tabref{table:multiclass}. As before, we just plot the dominating term in the RHS of \eqref{eq:multiclass_ERM} as predicted bound. Additionally, we also plot the predicted lower bound by the error on mislabeled data which nevertheless were predicted as true label. We refer to this as ``Oracle bound''. See text for more details. 
%     % 
%     % except for the stopping point. 
%     % The bound predicted by RATT (RHS in \eqref{eq:multiclass_ERM}) is vacuous. 
%     }\label{fig:error_epoch_CIFAR10}
%     % \vspace{-15pt}
% \end{figure}


% \textbf{Results on CIFAR 100 {} {}} 
% % 
% On CIFAR100, our bound in \eqref{eq:multiclass_ERM} yields vacous bounds. However, the oracle bound as explained above yields tight guarantees in the initial phase of the learning (i.e., when learning rate is less than $0.1$). 

% \begin{figure}[h]
%     \centering 
%     % \vspace{-15pt}
%     % \includegraphics[width=0.9\linewidth]{example-image-a}
%     \includegraphics[width=0.4\linewidth]{figures/CIFAR100-Resnet.pdf}
%     % \includegraphics[width=0.9\linewidth]{figures/{CIFAR10_rn=0.1_lr=0.2_wd=0.005}.png}
%     % \vspace{-10pt}
%     \caption{ Predicted lower bound by the error on mislabeled data which nevertheless were predicted as true label with ResNet18 on CIFAR100. We refer to this as ``Oracle bound''. See text for more details. 
%     % 
%     % except for the stopping point. 
%     The bound predicted by RATT (RHS in \eqref{eq:multiclass_ERM}) is vacuous. 
%     }\label{fig:error_CIFAR100}
%     % \vspace{-15pt}
% \end{figure}


% % \paragraph{Experiments on CIFAR100} 



% \subsection{Hyperparameter Details}


% \textbf{\figref{fig:error_CIFAR10} {} {}} We use clean training dataset of size $40,000$. We fix the amount of unlabeled data at $20\%$ of the clean size, i.e. we include additional $8,000$ points with randomly assigned labels. We use test set of $10,000$ points. For both MLP and ResNet, we use SGD with an initial learning rate of $0.1$ and momentum $0.9$. We fix the weight decay parameter at $5\times 10^{-4}$. After $100$ epochs, we decay the learning rate to $0.01$. We use SGD batch size of $100$. 

% \textbf{\figref{fig:error_binary} (a) {} {}} We obtain a toy dataset according to the process described in \secref{sec:app_dataset}. We fix $d=100$ and create a dataset of $50,000$ points with balanced classes. Moreover, we sample additional covariates with the same procedure to create randomly labeled dataset. For both SGD and GD training, we use a fixed learning rate $0.1$.    

% \textbf{\figref{fig:error_binary} (b) {} {}} Similar to binary CIFAR, we use clean training dataset of size $40,000$ and fix the amount of unlabeled data at $20\%$ of the clean dataset size. To train wide nets, we use a fixed learning of $0.001$ with GD and SGD. We decide the weight decay parameter and the early stopping point that maximizes our generalization bound (i.e. without peeking at unseen data ).  We use SGD batch size of $100$. 

% \textbf{\figref{fig:error_binary} (c) {} {}} With IMDb dataset, we use a clean dataset of size $20,000$ and as before, fix the amount of unlabeled data at $20\%$ of the clean data. To train ELMo model, we use Adam optimizer with a fixed learning rate $0.01$ and weight decay $10^{-6}$ to minimize cross entropy loss. We train with batch size $32$ for 3 epochs. To fine-tune BERT model, we use Adam optimizer with learning rate $5\times 10^{-5}$ to minimize cross entropy loss. We train with a batch size of $16$ for 1 epoch.    

% \textbf{\tabref{table:multiclass} {} {}} For multiclass datasets, we train both MLP and ResNet with the same hyperparameters as described before. We sample a clean training dataset of size $40,000$ and fix the amount of unlabeled data at $20\%$ of the clean size. We use SGD with an initial learning rate of $0.1$ and momentum $0.9$. We fix the weight decay parameter at $5\times 10^{-4}$. After $30$ epochs for ResNet and after $50$ epochs for MLP, we decay the learning rate to $0.01$.  We use SGD with batch size $100$. 
% For \figref{fig:error_CIFAR100}, we use the same hyperparameters as 
% CIFAR10 training, except we now decay learning rate after $100$ epochs. 


% In all experiments, to identify the best possible accuracy on just the clean data, we use the exact same set of hyperparamters except the stopping point. We choose a stopping point that maximizes test performance. 

% \subsection{Summary of experiments }

% \begin{center}
%     \begin{table}[H] 
%         \centering
%         \begin{tabular}{|c|c|c|c|} 
%         \hline
%         Classification type & Model category & Model & Dataset  \\ [0.5ex] 
%         \hline
%         \hline
%         \multirow{9}{*}{Binary} & Low dimensional & Linear model & Toy Gaussain dataset  \\
%                         \cline{2-4}
%                          & \multirow{1}{*}{Overparameterized linear nets} 
%                         %  & Linear model & Toy Gaussain dataset \\
%                         %  \cline{3-4}
%                         %  & & 2-layer wide net& Toy Gaussain dataset \\
%                         %  \cline{3-4}
%                          & 2-layer wide net & Binary MNIST \\
%                          \cline{2-4}                 
%                          & \multirow{6}{*}{Deep nets} & \multirow{2}{*}{MLP} & Binary MNIST \\
%                          \cline{4-4}
%                          & &  & Binary CIFAR \\
%                          \cline{3-4}
%                          &  & \multirow{2}{*}{ResNet} & Binary MNIST \\
%                          \cline{4-4}
%                          & &  & Binary CIFAR \\
%                          \cline{3-4}
%                          &  & ELMo-LSTM model & IMDb Sentiment Analysis \\
%                          \cline{3-4}
%                          & & BERT pre-trained model & IMDb Sentiment Analysis \\
%         \hline
%         \multirow{5}{*}{Multiclass} & \multirow{5}{*}{Deep nets} & \multirow{2}{*}{MLP} & MNIST \\
%                         \cline{4-4} 
%                         & & & CIFAR10 \\                   
%                         \cline{3-4}
%                          &   & \multirow{3}{*}{ResNet} & MNIST \\
%                          \cline{4-4}
%                          &   & & CIFAR10 \\
%                          \cline{4-4}
%                          &   & & CIFAR100 \\
%         \hline
%         \end{tabular}
%         % \caption{Summary of experiments performed} \label{table:experiments}
%     \end{table}    
%     % \footnotetext[6]{We use both MSE loss and cross-entropy loss.}
%     % \footnotetext[6]{We try 2 variants: one with a fixed first layer and the other with both layers trainable.}
% \end{center}

% \newpage
% \section{Proof of \lemref{lem:stability_error}} \label{app:proof_lem_error}

% \begin{proof}[Proof of \lemref{lem:stability_error}]
%     Recall, we have a training set $S \cup \wt S_C$. We defined leave-one-out error on mislabeled points as $$\error_{\text{LOO}(\wt S_M) } = \frac{\sum_{(x_i, y_i) \in \wt S_M} \error( f_{(i)}( x_i), y_i)}{ \abs{\wt S_M }} \,, $$
%     where $f_{(i)} \defeq f(\calA, (S \cup \wt S)_{(i)})$. Define $S^\prime \defeq S \cup \wt S$. Assume $(x,y)$ and $(x^\prime,y^\prime)$ as i.i.d. samples from ${\calDm}$. 
%     Using Lemma 25 in \citet{bousquet2002stability}, we have
%     \begin{align*}
%         \Expo{ \left( \error_{\calDm}(\wh f) -\error_{\text{LOO}(\wt S_M) } \right)^2 } \le & \Expt{ S^\prime, (x,y), (x^\prime,y^\prime) }{ \error(\wh f(x), y ) \error(\wh f(x^\prime), y^\prime )} - 2 \Expt{ S^\prime, (x,y) }{ \error(\wh f(x), y ) \error(f_{(i)}(x_i), y_i )} \\
%         & + \frac{m_1-1}{m_1}\Expt{ S^\prime }{  \error(f_{(i)}(x_i), y_i )  \error(f_{(j)}(x_j), y_j )} + \frac{1}{m_1} \Expt{ S^\prime }{  \error(f_{(i)}(x_i), y_i ) } \,. \numberthis \label{eq:main_reln}
%     \end{align*}
%     We can rewrite the equation above as : 
%     \begin{align*}
%         \Expo{ \left( \error_{\calDm}(\wh f) -\error_{\text{LOO}(\wt S_M) } \right)^2 } \le &  \, \underbrace{\Expt{ S^\prime, (x,y), (x^\prime,y^\prime) }{ \error(\wh f(x), y ) \error(\wh f(x^\prime), y^\prime ) - \error(\wh f(x), y ) \error(f_{(i)}(x_i), y_i )}}_{\RN{1}} \\
%         & + \underbrace{\Expt{ S^\prime }{  \error(f_{(i)}(x_i), y_i )  \error(f_{(j)}(x_j), y_j ) -  \error(\wh f(x), y ) \error(f_{(i)}(x_i), y_i )}}_{\RN{2}} \\ &+ \underbrace{\frac{1}{m_1} \Expt{ S^\prime }{  \error(f_{(i)}(x_i), y_i ) - \error(f_{(i)}(x_i), y_i )  \error(f_{(j)}(x_j), y_j ) }}_{\RN{3}} \,. \numberthis \label{eq:main_reln2}
%     \end{align*}
    
%     We will now bound term $\RN{3}$.  Using Schwarz's inequality, we have
    
%     \begin{align}
%         \Expt{ S^\prime }{  \error(f_{(i)}(x_i), y_i ) - \error(f_{(i)}(x_i), y_i )  \error(f_{(j)}(x_j), y_j ) }^2 &\le  \Expt{ S^\prime }{  \error(f_{(i)}(x_i), y_i ) }^2 \Expt{S^\prime}{1 -   \error(f_{(j)}(x_j), y_j ) }^2 \\
%         &\le \frac{1}{4} \label{eq:term1_lem12}
%     \end{align}
    
%     Note that since $(x_i,y_i)$, $(x_j ,y_j )$, $(x,y)$, and $(x^\prime, y^\prime)$ are all from same distribution $\calDm$, we directly incorporate the bounds on term $\RN{1}$ and $\RN{2}$ from proof of Lemma 9 in \citet{bousquet2002stability}. Combining that with \eqref{eq:term1_lem12} and our definition of hypothesis stability in \codref{cond:hypothesis_stability}, we have the required claim. 
    
    
%     % We now re-write term $\RN{1}$ as
%     % \begin{align*}
%     %         &\Expt{S^\prime, (x,y), (x^\prime,y^\prime) }{ \error(\wh f(x), y ) \error(\wh f(x^\prime), y^\prime ) - \error(\wh f(x), y ) \error(f_{(i)}(x_i), y_i )} \\ & \qquad = \Expt{ S^\prime, (x,y), (x^\prime,y^\prime) }{ \error(\wh f(x), y ) \error(\wh f  (x^\prime), y^\prime ) - \error(\wh f ^\prime(x), y ) \error(f_{(i)}(x^\prime), y^\prime )} \tag{Exchanging $(x_i, y_i)$ with $(x^\prime, y^\prime)$ in the second term} \\
%     %         & \qquad = \Expt{ S^\prime, (x,y), (x^\prime,y^\prime) }{  \left(\error(\wh f(x), y )-  \error(f_{(i)}(x), y ) \right) \error(\wh f  (x^\prime), y^\prime )  } \\
%     %         & \qquad  + \Expt{ S^\prime, (x,y), (x^\prime,y^\prime) }{  \left(\error(f_{(i)}(x), y ) -\error(\wh f ^\prime(x), y ) \right) \error(\wh f  (x^\prime), y^\prime )}  \\
%     %         & \qquad +\Expt{ S^\prime, (x,y), (x^\prime,y^\prime) }{  \left( \error(\wh f  (x^\prime), y^\prime ) -  \error(f_{(i)}(x^\prime), y^\prime ) \right) \error(\wh f ^\prime(x), y ) }  \,, \numberthis \label{eq:term1_final}
%     % \end{align*}
%     % where $\wh f^\prime$ is the classifier obtained by training on $ S^\prime_{(i)} \cup \{ (x^\prime, y^\prime) \} $. Similarly we can re-write term $\RN{2}$ as 
%     % \begin{align*}
%     %     & \Expt{ S^\prime }{  \error(f_{(i)}(x_i), y_i )  \error(f_{(j)}(x_j), y_j ) -  \error(\wh f(x), y ) \error(f_{(i)}(x_i), y_i )} \\
%     %     &\quad  = \Expt{ S^\prime, (x,y), (x^\prime,y^\prime)}{  \error(f^{\prime\prime}_{(i)}(x), y )  \error(f_{(j)}^{\prime}(x^\prime), y^\prime ) -  \error(\wh f(x), y ) \error(f_{(i)}(x_i), y_i )} \tag{Exchanging $(x_i, y_i)$ with $(x, y)$ and $(x_j, y_j)$ with $(x^\prime, y^\prime)$ in the first term}\\
%     %     &\quad = \Expt{ S^\prime, (x,y), (x^\prime,y^\prime)}{  \error(f^{\prime\prime}_{(j)}(x), y )  \error(f_{(i)}^{\prime}(x^\prime), y^\prime ) -  \error(\wh f^\prime (x), y ) \error(f^\prime_{(j)}(x^\prime), y^\prime )} \tag{Exchanging $(x_i, y_i)$ and $(x_j, y_j)$ and then replacing $(x_j, y_j)$ with $(x^\prime, y^\prime)$ in the second term} \\
%     %     & \quad = \Expt{ S^\prime, (x,y), (x^\prime,y^\prime) }{  \left( \error(f_{(i)}^{\prime}(x^\prime), y^\prime )   -  \error(\wh f^{\prime\prime}  (x^\prime), y^\prime ) \right)  \error(f^{\prime\prime}_{(j)}(x), y )   } \\
%     %     & \quad  + \Expt{ S^\prime, (x,y), (x^\prime,y^\prime) }{  \left( \error(f^{\prime\prime}_{(j)}(x), y )  -\error(\wh f ^\prime(x), y ) \right) \error(\wh f^{\prime\prime}  (x^\prime), y^\prime )  }  \\
%     %     & \quad+ \Expt{ S^\prime, (x,y), (x^\prime,y^\prime) }{  \left( \error(\wh f^{\prime\prime}  (x^\prime), y^\prime )  -  \error(f^\prime_{(j)}(x^\prime), y^\prime ) \right)  \error(\wh f^\prime (x), y ) }   \\
%     %     & \quad = \Expt{ S^\prime, (x,y), (x^\prime,y^\prime) }{  \left( \error(f_{(i)}^{\prime}(x^\prime), y^\prime )   -  \error(\wh f (x^\prime), y^\prime ) \right)  \error(f_{(i)}(x_j), y_j )   } \\
%     %     & \quad  + \Expt{ S^\prime, (x,y), (x^\prime,y^\prime) }{  \left( \error(f^{\prime\prime}_{(j)}(x), y )  -\error(\wh f (x), y ) \right) \error(\wh f^{\prime\prime}  (x_j), y_j )  }  \\
%     %     & \quad+ \Expt{ S^\prime, (x,y), (x^\prime,y^\prime) }{  \left( \error(\wh f^{\prime\prime}  (x^\prime), y^\prime )  -  \error(f^\prime_{(j)}(x^\prime), y^\prime ) \right)  \error(\wh f^\prime (x^\prime), y^\prime ) }  \,, \numberthis \label{eq:term2_final}
%     % \end{align*}
%     % where $f^{\prime\prime}_{(j)}$ is trained on $S^\prime_{(j,i)} \cup {(x,y)}$, $f^{\prime}_{(i)}$ is trained on $S^\prime_{(j,i)} \cup {(x^\prime,y^\prime)}$, and $\wh f^{\prime\prime} $ is trained on $S^\prime_{(j)} \cup {(x,y)}$. Note in the last line we replaced $(x,y)$ by $(x_j, y_j)$ in the first term, replaced $(x^\prime,y^\prime)$ by $(x_j, y_j)$ in the second term and exchanged $(x_i,y_i)$ with $(x_j,y_j)$ and also $(x,y)$ and $(x^\prime, y^\prime)$
    
    
% \end{proof}

\end{document}

