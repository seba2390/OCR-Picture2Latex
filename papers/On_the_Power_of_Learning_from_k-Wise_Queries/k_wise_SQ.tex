\documentclass[11pt]{article}
\usepackage{amssymb}
 %\setlength\parindent{0pt}
\usepackage{amsfonts}
\usepackage{amsmath}
\usepackage{amsthm}
\usepackage{latexsym}

\usepackage{bbm}

\usepackage{enumitem}
%\usepackage{enumerate}

\usepackage{mathtools}

\usepackage{xspace}

\usepackage[pagebackref=true]{hyperref}
\hypersetup{
    unicode=false,          % non-Latin characters in Acrobat bookmarks
    colorlinks=true,        % false: boxed links; true: colored links
    linkcolor=red,          % color of internal links (change box color with linkbordercolor)
    citecolor=blue,        % color of links to bibliography
    filecolor=magenta,      % color of file links
    urlcolor=cyan           % color of external links
}

\usepackage{hyperref,cleveref}
\usepackage{amsmath}
\usepackage{amsthm}
\usepackage{amsfonts}
\usepackage{fullpage,appendix}
%\usepackage{algorithm}
%\usepackage{algorithmic}
%\usepackage[ruled,vlined]{algorithm2e}


\DeclareMathOperator*{\argmin}{arg\,min}
\DeclareMathOperator*{\argmax}{arg\,max}

\usepackage[ruled]{algorithm}
\usepackage{algpseudocode}
\usepackage{algorithmicx}
%\floatname{algorithm}{Protocol}

%\usepackage{dsfont}
\usepackage{color}
\usepackage{tikz}

\providecommand{\remove}[1]{}
\providecommand{\eg}{{\em e.g.}~}
\renewcommand{\algorithmicrequire}{}
\newcommand{\eqdef} {\mathrel{\stackrel{\makebox[0pt]{\mbox{\normalfont\tiny
def}}}{=}}}
%\renewcommand{\thealgorithm}{}
\newcommand{\etal}{et al.\ }
\newcommand{\spn}{\S^{n-1}}
\newcommand{\tm}{\tilde{m}}
\newcommand{\ignore}[1]{}
\definecolor{corlinks}{RGB}{64,128,128}
\definecolor{cormenu}{RGB}{0,37,94}
\definecolor{corurl}{RGB}{0,46,91}
\definecolor{darkgreen}{rgb}{0,0.5,0}

%\setlength{\parindent}{0in}
\newcommand{\on}{\{-1,1\}}
\newcommand{\1}{\mathds{1}}
\newcommand{\UCC}{\mathsf{UCC}}
\newcommand{\err}{\mathsf{err}}

\newcommand{\Cov}{{\rm Cov}}

\newcommand{\bkets}[1]{\left(#1\right)}
\newcommand{\sbkets}[1]{\left[#1\right]}
\newcommand{\braces}[1]{\left\{#1\right\}}
\newcommand{\ip}[2]{\langle #1, #2 \rangle}


\newcommand{\CSDR}{\mathsf{cRSD}}
\newcommand{\RSD}{\mathsf{RSD}}
\newcommand{\SD}{\mathsf{SD}}
\newcommand{\Rcvr}{\mathsf{Rcvr}}
\newcommand{\fract}{\mathsf{frac}}
\newcommand{\Hyperplane}{\mathsf{Hyperplane}}

\newcommand{\Hyp}{\mathsf{Hyp}}
\newcommand{\Sub}{\mathsf{Sub}}
\newcommand{\LR}{\mathsf{LR}}

\newcommand{\rk}{\mathsf{rk}}

\newcommand{\dcdf}{\mathsf{dcdf}}
\newcommand{\cor}{\mathsf{cor}}
\renewcommand{\P}{\mathsf{P}}
\newcommand{\BPP}{\mathbf{BPP}}
\newcommand{\E} {\mathbb{E}}
\DeclareMathOperator*{\ex}{\mathbb{E}}
\DeclareMathOperator*{\pr}{\mathsf{Pr}}
\newcommand{\R}{\mathbb{R}}
\newcommand{\size}{\mathsf{size}}
\newcommand{\erf}{ \mathsf{erf} }
\newcommand{\D}{\mathcal{D}}
\newcommand{\calD}{\mathcal{D}}
\newcommand{\calP}{\mathcal{P}}
\newcommand{\calB}{\mathcal{B}}
\newcommand{\calV}{\mathcal{V}}
\newcommand{\calF}{\mathcal{F}}
\newcommand{\calZ}{\mathcal{Z}}
\newcommand{\calU}{\mathcal{U}}
\newcommand{\calQ}{\mathcal{Q}}
\newcommand{\calA}{\mathcal{A}}
\newcommand{\calC}{\mathcal{C}}
\newcommand{\cale}{\mathcal{E}}
\renewcommand{\H}{\mathbb{H}}
\newcommand{\cH}{\mathcal{H}}
\newcommand{\G}{\mathbb{G}}
\newcommand{\N}{\mathbb{N}}
\newcommand{\cN}{\mathcal{N}}
\newcommand{\B}{\mathbb{B}}
\newcommand{\A}{\mathcal{A}}
\newcommand{\cC}{\mathcal{C}}
\newcommand{\C}{\mathbb{C}}
\newcommand{\W}{\mathbb{W}}
\newcommand{\U}{\mathbb{U}}
\newcommand{\cU}{\mathcal{U}}
\newcommand{\SO}{\mathsf{SO}}
\newcommand{\SU}{\mathsf{SU}}
\newcommand{\Sign}{\mathsf{Sign}}
\renewcommand{\L}{\mathcal{L}}
\renewcommand{\O}{\mathcal{O}}
\newcommand{\tmu} {\tilde{\mu}}
\newcommand{\I}{\mathcal{I}}
\newcommand{\tnu}{\tilde{\nu}}
\newcommand{\M}{\mathcal{M}}
\newcommand{\norm}[1]{||#1||}
\newcommand{\poly}{\mathsf{poly}}
\newcommand{\Prob}{\Pr}
\renewcommand{\S}{\mathbb{S}}
\newcommand{\STAT}{\mathsf{STAT}}
\newcommand{\VSTAT}{\mathsf{VSTAT}}
\newcommand{\wRFA}{\mathsf{wRFA}}
%\newcommand{\CSTAT}{\mathsf{CSTAT}}
\newcommand{\F}{\mathbb{F}}
\newcommand{\calf}{\mathcal{F}}
\newcommand{\Fc}{\mathcal{F}}
\newcommand{\Dc}{\mathcal{D}}
\newcommand{\floor}[1]{\left \lfloor #1 \right \rfloor}
\newcommand{\zo}{\{0, 1\}}

\newcommand{\KL}{{\mathrm{KL}}}
\newcommand{\Div}{{\mathrm{D}}}
\newcommand{\KLR}{R_{\mathrm{KL}}}

\newcommand{\cp}{\mathsf{cp}}

\newcommand{\st}{\mathsf{st}}

\newcommand{\Ex}{\mathbb E}

%complexity classes
\newcommand{\Ppoly}{\mathsf{P/poly}}
\newcommand{\BPTIME}{\mathsf{BPTIME}}
\newcommand{\EXP}{\mathsf{EXP}}
\newcommand{\NP}{\mathsf{NP}}
\newcommand{\DTIME}{\mathsf{DTIME}}
\newcommand{\PSPACE}{\mathsf{PSPACE}}
\newcommand{\PH}{\mathsf{PH}}
\newcommand{\SIZE}{\mathsf{SIZE}}
\newcommand{\NEXP}{\mathsf{NEXP}}
\newcommand{\CSTAT}{\mathsf{CSTAT}}
\newcommand{\Maj}{\mathsf{MAJ}}
\renewcommand{\cal}[1]{\mathcal{#1}}
\newcommand{\eps}{\epsilon}

\newcommand{\Corr}{\mathsf{Corr}}
\newcommand{\Succ}{\mathsf{Succ}}

\newcommand{\twCCU}{\mathsf{2wayCCU}}
\newcommand{\wt}{\mathsf{wt}}
\newcommand{\cald}{\mathcal{D}}

\newcommand{\Disc}{\textsc{Disc}}

\newenvironment{proofof}[1]{\noindent{\bf Proof}
of #1:\hspace*{1em}}{\qed\bigskip}

\newtheorem{fact}{Fact}[section]
\newtheorem{definition}[fact]{Definition}
\newtheorem{defn}[fact]{Definition}
\newtheorem*{notation}{Notation}
\newtheorem*{untheorem}{Theorem}
\newtheorem{theorem}[fact]{Theorem}
\newtheorem{lemma}[fact]{Lemma}
\newtheorem{lem}[fact]{Lemma}
\newtheorem{corollary}[fact]{Corollary}
\newtheorem{observation}[fact]{Observation}
\newtheorem{proposition}[fact]{Proposition}
\newtheorem{problem}[fact]{Problem}
\newtheorem{claim}[fact]{Claim}
\newtheorem{property}[fact]{Property}
\newtheorem{conjecture}[fact]{Conjecture}
\newtheorem*{note}{Note}
\newtheorem{remark}[fact]{Remark}

\newtheorem{question}{Question}
\newtheorem{answer}{Answer}
\newtheorem{goal}{Goal}

\makeatletter
\newtheorem*{rep@theorem}{\rep@title}
\newcommand{\newreptheorem}[2]{%
\newenvironment{rep#1}[1]{%
 \def\rep@title{#2 \ref{##1}}%
 \begin{rep@theorem}}%
 {\end{rep@theorem}}}
\makeatother

\newreptheorem{theorem}{Theorem}


\newcommand{\MCSP} {\mathsf{MAX}\text{-}\mathsf{CSP}}
\newcommand{\tE}{\tilde{\mathbb{E}}}
\newcommand{\Span}{\mathsf{Span}}
\newcommand{\tchi}{\tilde{\chi}}
\newcommand{\tv}{\tilde{v}}
\newcommand{\dist}{\mathsf{dist}}
\newcommand{\cl}{\mathsf{cl}}
\newcommand{\Ball}{\mathsf{Ball}}
\newcommand{\bchi}{\bar{\chi}}
\newcommand{\Proj}{\mathsf{Proj}}
\newcommand{\COMM}{\mbox{BS}}

\newcommand{\CC}{\mathsf{CC}} % Communication complexity

\newcommand{\authnote}[4]{{\bf [{\color{#3} #1's Note:} {\color{#4} #2}]}}
\newcommand{\mmod}{~\mathrm{mod}~}


\newif\ifnotes\notestrue
% \newif\ifnotes\notesfalse


\ifnotes

\newcommand{\vnote}[1]{\textcolor{red}{{\bf (Vitaly:} {#1}{\bf ) }} \marginpar{\tiny\bf
             \begin{minipage}[t]{0.5in}
               \raggedright Vitaly
            \end{minipage}}}
\newcommand{\bnote}[1]{\textcolor{red}{{\bf (Badih} {#1}{\bf ) }} \marginpar{\tiny\bf
             \begin{minipage}[t]{0.5in}
               \raggedright Badih
            \end{minipage}}}
\newcommand{\gnote}[1]{\textcolor{red}{{\bf (GV:} {#1}{\bf ) }} \marginpar{\tiny\bf
             \begin{minipage}[t]{0.5in}
               \raggedright GV
                \end{minipage}}}
\else
\newcommand{\enote}[1]{}
\newcommand{\bnote}[1]{}
\newcommand{\gnote}[1]{}
\fi


\usepackage{empheq}
\newcommand*\widefbox[1]{\fbox{\hspace{2em}#1\hspace{2em}}}

\newcommand{\atomicsolverof}[2]{\textsc{AtomicSolver}{($#1$,$#2$)}\xspace}
\newcommand{\atomicsolverproc}[2]{\textsc{AtomicSolver}{($#1$,$#2$, $a_{#1}$, $b_{#1}$, $R_{#1, #2}$ )}\xspace}
\newcommand{\atomicsolver}{\textsc{AtomicSolver}\xspace}
\newcommand{\generator}{\textsc{AuxiliaryVariableGenerator}\xspace}

\title{On the Power of Learning from $k$-Wise Queries}
\author{
Vitaly Feldman \\
IBM Research - Almaden
\and
Badih Ghazi\thanks{Work done while at IBM Research - Almaden.}\\
Computer Science and Artificial Intelligence Laboratory, MIT
}

\date{\today}

\begin{document}

%\input{statement}

\newpage

\maketitle


  In this paper, we explore the connection between secret key agreement and secure omniscience within the setting of the multiterminal source model with a wiretapper who has side information. While the secret key agreement problem considers the generation of a maximum-rate secret key through public discussion, the secure omniscience problem is concerned with communication protocols for omniscience that minimize the rate of information leakage to the wiretapper. The starting point of our work is a lower bound on the minimum leakage rate for omniscience, $\rl$, in terms of the wiretap secret key capacity, $\wskc$. Our interest is in identifying broad classes of sources for which this lower bound is met with equality, in which case we say that there is a duality between secure omniscience and secret key agreement. We show that this duality holds in the case of certain finite linear source (FLS) models, such as two-terminal FLS models and pairwise independent network models on trees with a linear wiretapper. Duality also holds for any FLS model in which $\wskc$ is achieved by a perfect linear secret key agreement scheme. We conjecture that the duality in fact holds unconditionally for any FLS model. On the negative side, we give an example of a (non-FLS) source model for which duality does not hold if we limit ourselves to communication-for-omniscience protocols with at most two (interactive) communications.  We also address the secure function computation problem and explore the connection between the minimum leakage rate for computing a function and the wiretap secret key capacity.
  
%   Finally, we demonstrate the usefulness of our lower bound on $\rl$ by using it to derive equivalent conditions for the positivity of $\wskc$ in the multiterminal model. This extends a recent result of Gohari, G\"{u}nl\"{u} and Kramer (2020) obtained for the two-user setting.
  
   
%   In this paper, we study the problem of secret key generation through an omniscience achieving communication that minimizes the 
%   leakage rate to a wiretapper who has side information in the setting of multiterminal source model.  We explore this problem by deriving a lower bound on the wiretap secret key capacity $\wskc$ in terms of the minimum leakage rate for omniscience, $\rl$. 
%   %The former quantity is defined to be the maximum secret key rate achievable, and the latter one is defined as the minimum possible leakage rate about the source through an omniscience scheme to a wiretapper. 
%   The main focus of our work is the characterization of the sources for which the lower bound holds with equality \textemdash it is referred to as a duality between secure omniscience and wiretap secret key agreement. For general source models, we show that duality need not hold if we limit to the communication protocols with at most two (interactive) communications. In the case when there is no restriction on the number of communications, whether the duality holds or not is still unknown. However, we resolve this question affirmatively for two-user finite linear sources (FLS) and pairwise independent networks (PIN) defined on trees, a subclass of FLS. Moreover, for these sources, we give a single-letter expression for $\wskc$. Furthermore, in the direction of proving the conjecture that duality holds for all FLS, we show that if $\wskc$ is achieved by a \emph{perfect} secret key agreement scheme for FLS then the duality must hold. All these results mount up the evidence in favor of the conjecture on FLS. Moreover, we demonstrate the usefulness of our lower bound on $\wskc$ in terms of $\rl$ by deriving some equivalent conditions on the positivity of secret key capacity for multiterminal source model. Our result indeed extends the work of Gohari, G\"{u}nl\"{u} and Kramer in two-user case.

\newpage

\tableofcontents

\newpage

% !TEX root = ../arxiv.tex

Unsupervised domain adaptation (UDA) is a variant of semi-supervised learning \cite{blum1998combining}, where the available unlabelled data comes from a different distribution than the annotated dataset \cite{Ben-DavidBCP06}.
A case in point is to exploit synthetic data, where annotation is more accessible compared to the costly labelling of real-world images \cite{RichterVRK16,RosSMVL16}.
Along with some success in addressing UDA for semantic segmentation \cite{TsaiHSS0C18,VuJBCP19,0001S20,ZouYKW18}, the developed methods are growing increasingly sophisticated and often combine style transfer networks, adversarial training or network ensembles \cite{KimB20a,LiYV19,TsaiSSC19,Yang_2020_ECCV}.
This increase in model complexity impedes reproducibility, potentially slowing further progress.

In this work, we propose a UDA framework reaching state-of-the-art segmentation accuracy (measured by the Intersection-over-Union, IoU) without incurring substantial training efforts.
Toward this goal, we adopt a simple semi-supervised approach, \emph{self-training} \cite{ChenWB11,lee2013pseudo,ZouYKW18}, used in recent works only in conjunction with adversarial training or network ensembles \cite{ChoiKK19,KimB20a,Mei_2020_ECCV,Wang_2020_ECCV,0001S20,Zheng_2020_IJCV,ZhengY20}.
By contrast, we use self-training \emph{standalone}.
Compared to previous self-training methods \cite{ChenLCCCZAS20,Li_2020_ECCV,subhani2020learning,ZouYKW18,ZouYLKW19}, our approach also sidesteps the inconvenience of multiple training rounds, as they often require expert intervention between consecutive rounds.
We train our model using co-evolving pseudo labels end-to-end without such need.

\begin{figure}[t]%
    \centering
    \def\svgwidth{\linewidth}
    \input{figures/preview/bars.pdf_tex}
    \caption{\textbf{Results preview.} Unlike much recent work that combines multiple training paradigms, such as adversarial training and style transfer, our approach retains the modest single-round training complexity of self-training, yet improves the state of the art for adapting semantic segmentation by a significant margin.}
    \label{fig:preview}
\end{figure}

Our method leverages the ubiquitous \emph{data augmentation} techniques from fully supervised learning \cite{deeplabv3plus2018,ZhaoSQWJ17}: photometric jitter, flipping and multi-scale cropping.
We enforce \emph{consistency} of the semantic maps produced by the model across these image perturbations.
The following assumption formalises the key premise:

\myparagraph{Assumption 1.}
Let $f: \mathcal{I} \rightarrow \mathcal{M}$ represent a pixelwise mapping from images $\mathcal{I}$ to semantic output $\mathcal{M}$.
Denote $\rho_{\bm{\epsilon}}: \mathcal{I} \rightarrow \mathcal{I}$ a photometric image transform and, similarly, $\tau_{\bm{\epsilon}'}: \mathcal{I} \rightarrow \mathcal{I}$ a spatial similarity transformation, where $\bm{\epsilon},\bm{\epsilon}'\sim p(\cdot)$ are control variables following some pre-defined density (\eg, $p \equiv \mathcal{N}(0, 1)$).
Then, for any image $I \in \mathcal{I}$, $f$ is \emph{invariant} under $\rho_{\bm{\epsilon}}$ and \emph{equivariant} under $\tau_{\bm{\epsilon}'}$, \ie~$f(\rho_{\bm{\epsilon}}(I)) = f(I)$ and $f(\tau_{\bm{\epsilon}'}(I)) = \tau_{\bm{\epsilon}'}(f(I))$.

\smallskip
\noindent Next, we introduce a training framework using a \emph{momentum network} -- a slowly advancing copy of the original model.
The momentum network provides stable, yet recent targets for model updates, as opposed to the fixed supervision in model distillation \cite{Chen0G18,Zheng_2020_IJCV,ZhengY20}.
We also re-visit the problem of long-tail recognition in the context of generating pseudo labels for self-supervision.
In particular, we maintain an \emph{exponentially moving class prior} used to discount the confidence thresholds for those classes with few samples and increase their relative contribution to the training loss.
Our framework is simple to train, adds moderate computational overhead compared to a fully supervised setup, yet sets a new state of the art on established benchmarks (\cf \cref{fig:preview}).


\section{Preliminaries}
Given a graph $G=(V,E)$, and vertex $u \in V$, let $\deg(u,G)$ be the degree of $u$ in $G$. 
Given a tree $T$ and $u, v \in T$, denote the $u$-$v$ path in $T$ by $\pi(u,v,T)$. When the tree $T$ is clear from the context, we may omit it and write $\pi(u,v)$. For a (possibly weighted) subgraph $G' \subseteq G$ and a vertex pair $s,t \in V$, let $\dist_{G'}(s,t)$ denote the length of the $s$-$t$ shortest path in $G'$. 

\paragraph{Fault-Tolerant Labeling Schemes.}
For a given graph $G$, let $\Pi: V\times V \times \mathcal{G} \to \mathbb{R}_{\geq 0}$ %\mtodo{a reviewer pointed out that the last element in the domain should probably be subgraphs of $G$ and not $G$, see which notation we want here.} \mertodo{I am not sure that it is needed as we consider $G$ in the fault-free setting and $G \setminus F$ in the FT setting. We can of course write $\Pi: V\times V \times \mathcal{G} \to \mathbb{R}_{\geq 0}$ where $\mathcal{G}$ is the family of all $G$-subgraphs, but I cannot see why we need it.} 
be a function defined on pairs of vertices and a subgraph $G' \subset G$, where $\mathcal{G}$ is the family of all subgraphs of $G$. For an integer parameter $f\geq 1$, an $f$-\emph{fault-tolerant labeling scheme} for a function $\Pi$ and a graph family $\mathcal{F}$ is a pair of functions $(L_{\Pi},D_{\Pi})$. The function $L_{\Pi}$ is called the \emph{labeling function}, and $D_{\Pi}$ is called the \emph{decoding function}. For every graph $G$ in the family $\mathcal{F}$, the labeling function $L_{\Pi}$ associates with each vertex $u \in V(G)$ and every edge $e \in E(G)$, a label $L_{\Pi}(u,G)$ (resp., $L_{\Pi}(e,G)$). It is then required that given the labels of any triplets $s,t, F \in V \times V \times E^f$, the decoding function $D_{\Pi}$ computes $\Pi(s,t, G \setminus F)$.  The primary complexity measure of a labeling scheme is the \emph{label length}, measured by the length (in bits) of the largest label it assigns to some vertices (or edges) in $G$ over all graphs $G \in \mathcal{F}$. An $f$-FT connectivity labeling scheme is required to output YES iff $s$ and $t$ are connected in $G \setminus F$.  In $f$-FT \emph{approximate distance labeling scheme} it is required to output an estimate for the $s$-$t$ distance in the graph $G \setminus F$. Formally, an $f$-FT labeling scheme is $q$\emph{-approximate} if the value $\delta(s,t,F)$ returned by the decoder algorithm satisfies that $\dist_{G \setminus F}(s,t)\leq \delta(s,t,F) \leq q \cdot \dist_{G \setminus F}(s,t)$.  Throughout the paper we provide randomized labeling schemes which provide a high probability guarantee of correctness for any fixed triplet $\langle s,t, F \rangle$. 


\paragraph{Fault-Tolerant Routing Schemes.} In the setting of FT routing scheme, one is given a pair of source $s$ and destination $t$ as well as $F$ edge faults, which are initially unknown to $s$. The routing scheme consists of \emph{preprocessing} and \emph{routing} algorithms. The preprocessing algorithm defines labels $L(u)$ to each of the vertices $u$, and a header $H(M)$ to the designated message $M$. In addition, it defines for every vertex $u$ a routing table $R(u)$. The labels and headers are usually required to be short, i.e., of poly-logarithmic bits. 
The routing procedure determines at each vertex $u$ the port-number on which $u$ should send the messages it receives. The computation of the next-hop is done by considering the header of the message $H(M)$, the label of the source and destination $L(s)$ and $L(t)$ and the routing table $R(u)$. The routing procedure at vertex $u$ might also edit the header of the message $H(M)$. The failing edges are not known in advance and can only be revealed by reaching (throughout the message routing) one of their endpoints. The \emph{space} of the scheme is determined based on maximal length of message headers, labels and the individual routing tables. The stretch of the scheme is measured by the ratio between the length of the path traversed until the message arrived its destination and the length of the shortest $s$-$t$ path in $G \setminus F$. In the more relaxed setting of \emph{forbidden-set routing schemes} the failing edges are given as input to the routing algorithm.

%\paragraph{Forbidden-Set Routing Schemes.} One of the key applications of labeling schemes is routing.
%In the setting of forbidden-set routing schemes, given the labels of $s$, $t$ and a set of forbidden edges $
%F$, it is required to determine the next-hop neighbor of $s$ on some short $s$-$t$ path in $G \setminus F$. The main two complexity measures are the stretch induced by the $s$-$t$ path encoded by the labels.  \mtodo{Maybe the definition should be more similar to the next one? (consider labels, tables, headers)}
%That is, given the labels of $u$, $v$ and the faults $F$, the decoder function returns the port-number of $u$'s neighbor lying on a $u$-$v$ path $P$ in $G \setminus F$ such that $P \subseteq G$ and $len(P)\leq s \cdot \dist(u,v, G \setminus F)$ for some approximation factor $s$. 


\section{Separation of $(k+1)$-wise from $k$-wise queries}\label{sec:pf_sep}
\newcommand{\ind}{\mathbbm{1}}

We start by describing the concept class $\calC$ that we use to prove Theorem~\ref{thm:k_wise_sep}. Let $\ell$ and $k$ be positive integers with $\ell \geq k+1$. The domain will be $\mathbb{F}_p^{\ell}$. For every $a = (a_1,\dots, a_{\ell}) \in \mathbb{F}_p^{\ell}$, we consider the hyperplane
$$ \Hyp_a \doteq \{ z = (z_1,\dots,z_{\ell}) \in \mathbb{F}_p^{\ell}: z_{\ell} = a_1 z_1 + \dots + a_{\ell-1} z_{\ell-1} + a_{\ell}\}.$$
We then define the Boolean-valued function $f_a: \mathbb{F}_p^{\ell} \to \{\pm 1\}$ to be the indicator function of the subset $\Hyp_a \subseteq \mathbb{F}_p^{\ell}$, i.e., for every $z \in \mathbb{F}_p^{\ell}$,

\[
 f_a(z) = \begin{dcases*}
        +1  & if $z \in \Hyp_a$,\\
        -1 & otherwise.
        \end{dcases*}
\]

Then, we will consider the concept classes $\calC_{\ell} \doteq \{f_a: a \in \mathbb{F}_p^{\ell}\}$. We denote $\calC \doteq \calC_{k+1}$. We start by stating our upper bound on the $(k+1)$-wise SQ complexity of the distribution-independent PAC learning of $\calC_{k+1}$.

\begin{lem}[$(k+1)$-wise upper bound]\label{le:k_ub}
Let $p$ be a prime number and $k$ be a positive integer. There exists a distribution-independent PAC learning algorithm for $\calC_{k+1}$ that makes at most $t \cdot \log(1/\eps) $ queries to $\STAT^{(k+1)}(\eps/t)$, for some $t = O_k(\log{p})$.
\end{lem}

We next state our lower bound on the $k$-wise SQ complexity of the same tasks considered in Lemma~\ref{le:k_ub}.

\begin{lem}[$k$-wise lower bound]\label{le:k_plus_1_lb}
 Let $p$ be a prime number and $\ell$, $k$ be positive integers with $\ell \geq k+1$ and $k = O(p)$. There exists $t =  \Omega\big(p^{(\ell-k)/4}\big)$ such that any distribution-independent PAC learning alogrithm for $\calC_{\ell}$ with error at most $1/2-2/t$ that is given access to $\STAT^{(k)}(1/t)$ needs at least $t$ queries.
\end{lem}

Note that  Lemma~\ref{le:k_ub} and Lemma~\ref{le:k_plus_1_lb} imply Theorem~\ref{thm:k_wise_sep}.

\subsection{Upper bound}

%Note that it is enough to prove Lemma~\ref{le:k_ub} in the case where $\ell = k+1$. This is because in the case of $\ell > k+1$, we can choose a subset $S \subseteq [\ell]$ of coordinates of size $\ell - k - 1$, and consider all $p^{\ell-k-1}$ settings of the variables $\{a_i: i \in S\}$. For each such setting, we can run the learning algorithm for the case where $\ell = k+1$. This results in a list of at most $p^{\ell-k-1}$ candidate hypotheses, one of which is guaranteed to be close to the true concept. To find this good hypothesis, we then estimate the empirical error for each candidate hypothesis using unary SQs, and output a hypothesis with small enough error. Henceforth, we assume that $\ell = k+1$ and prove Lemma~\ref{le:k_ub}.

\paragraph{Notation}
We first introduce some notation that will be useful in the description of our algorithm. For any matrix $M$ with entries in the finite field $\mathbb{F}_p$, we denote by $\rk(M)$ the rank of $M$ over $\mathbb{F}_p$. Let $(a_1,\dots,a_{k+1}) \in \mathbb{F}_p^{k+1}$ be the unknown vector that defines $f_a$ and $P$ be the unknown distribution over tuples $(z_1, \dots, z_{k+1}) \in \mathbb{F}_p^{k+1}$. 

Note that $\Hyp_a$ is an affine subspace of $\mathbb{F}_p^{k+1}$. To simplify our treatment of affine subspaces, we embed the points of $\mathbb{F}_p^{k+1}$ into $\mathbb{F}_p^{k+2}$ by mapping each $z \in \mathbb{F}_p^{k+1}$ to $(z,1)$. This embedding maps every affine subspace $V$ of $\mathbb{F}_p^{k+1}$ to a linear subspace $W$ of $\mathbb{F}_p^{k+2}$, namely the span of the image of $V$ under our embedding. Note that this mapping is one-to-one and allows us to easily recover $V$ from $W$ as $V = \{z \in \mathbb{F}_p^{k+1} \ | \ (z,1) \in W\}$.  Hence given $k+1$ examples
$$\big((z_{1,1}, \dots, z_{1,k+1}),b_1\big),\big((z_{2,1}, \dots, z_{2,k+1}),b_2\big), \dots, \big((z_{k+1,1}, \dots, z_{k+1,k+1}),b_{k+1}\big)$$  we define the matrix:
\begin{equation}\label{eq:Z_mat_def}
Z \doteq
\begin{bmatrix}
z_{1,1}       & z_{1,2} &  \cdot & z_{1,k+1} & 1 \\
z_{2,1}       & z_{2,2} & \cdot & z_{2,k+1} & 1 \\
\cdot       & \cdot & \cdot & \cdot & \cdot \\
\cdot       & \cdot & \cdot & \cdot & \cdot \\
z_{k+1,1}       & z_{k+1,2} & \cdot & z_{k+1,k+1} & 1
\end{bmatrix}.
\end{equation}
For $\ell \in [k+1]$ we also denote by $Z_\ell$ the matrix that consists of the top $\ell$ rows of $Z$.  Further, for a $(k+1)$-wise query function $\phi\big((z_1,b_1),\ldots,(z_{k+1},b_{k+1})  \big)$, we use $Z$ to refer to the matrix obtained from the inputs to the function.

Let $Q$ be the distribution defined by sampling a random example $\big((z_{1}, \dots, z_{k+1}),b\big)$, conditioning on the event that $b=1$ and outputting $(z_{1}, \dots, z_{k+1},1)$. Note that if the examples from which $Z$ is built are positively labeled i.i.d. examples then each row of $Z$ is sampled i.i.d. from $Q$ and hence $Z_\ell$ is distributed according to $Q^\ell$.
%Let $Q$ be the distribution over $\mathbb{F}_p^k$ obtained by sampling $(z_1, \dots, z_{k+1}) \sim P$, conditioning on the event that $z_{k+1} = a_1 z_1 + \dots + a_k z_k +a_{k+1}$, and outputting $(z_1, \dots, z_k)$. Equivalently, we can define the distribution $Q$ as sampling a random example, conditioning on the event that its label is $1$, and restricting to the first $k$ coordinates.  We will
We denote by $1^{k+1}$ the all $+1$'s vector of length $k+1$.

\paragraph{Learning algorithm}
We start by explaining the main ideas behind the algorithm. On a high level, in order to be able to use $(k+1)$-wise SQs to learn the unknown subspace, we need to make sure that there exists an affine subspace that contains most of the probability mass of the positively-labeled points and
that is spanned by $k+1$ random positively-labeled points with noticeable probability. Here, the probability is with respect to the unknown distribution over labeled examples. Thus, for positively labeled tuples $(z_{1,1}, \dots, z_{1,k+1})$, $(z_{2,1}, \dots, z_{2,k+1})$, $\dots$, $(z_{k+1,1}, \dots, z_{k+1,k+1})$, we  consider the $(k+1) \times (k+2)$ matrix $Z$ defined in \Cref{eq:Z_mat_def}. If $W$ is the row-span of $Z$, then the desired (unknown) affine subspace is the set $V$ of all points $(z_1, \dots,z_{k+1})$ such that $(z_1, \dots,z_{k+1}, 1) \in W$.
	
	
	If the (unknown) distribution over labeled examples is such that with noticeable probability, $k+1$ random positively-labeled points form a full-rank linear system (i.e., the matrix $Z$ has full-rank with noticeable probability conditioned on $(b_1,\dots,b_{k+1}) = 1^{k+1}$), we can use $(k+1)$-wise SQs to find, one bit at a time, the $(k+1)$-dimensional row-span $W$ of $Z$, and we can then output the set $V$ of all points $(z_1, \dots,z_{k+1})$ such that $(z_1, \dots,z_{k+1}, 1) \in W$ as the desired affine subspace (below, we refer to this step as the Recovery Procedure).
		
	 We now turn to the (more challenging) case where the system is not full-rank with noticeable probability (i.e., the matrix $Z$ is rank-deficient with high probability conditioned on $(b_1,\dots,b_{k+1}) = 1^{k+1}$). Then, the system has rank at most $i$ with high probability, for some $i < k+1$. There is a large number of possible $i$-dimensional subspaces and therefore it is no longer clear that there exists a single $i$-dimensional subspace that contains most of the mass of the positively-labeled points. However, we demonstrate that for every $i$, if the rank of $Z$ is at most $i$ with sufficiently high probability, then there exists a \emph{fixed} subspace $W$ of dimension at most $i$ that contains a large fraction of the probability under the row-distribution of $Z$ (it turns out that if this subspace has rank equal to $i$, then it should be \emph{unique}). We can then use $(k+1)$-wise SQs to output the affine subspace $V$ consisting of all points $(z_1,\dots,z_{k+1})$ such that $(z_1,\dots,z_{k+1},1) \in W$ (via the Recovery Procedure).


 %To ensure this, we prove (in \Cref{le:ex_sub}) that
%Note that in principle this can be due to the distribution over individual rows of $Z$ having small positive mass on \emph{several} subspaces each of dimension at most $i$. But in order to recover a meaningful subspace using statistical queries, we need the rows of $Z$ to lie inside a \emph{fixed} $i$-dimensional subspace with high probability.
	%Note that such a subspace corresponds to a subset of columns of $Z$ that are linearly independent with high probability. \textcolor{red}{Once we find this subspace}, we can set the redundant coordinates of $(a_1,\dots,a_{k+1})$ to $0$'s and solve for the remaining coordinates using the same ``recovery'' procedure as before: namely, we invert the lower-dimensional full-rank system and output the bits one at a time using $(k+1)$-wise SQs.
	The general description of the algorithm is given in Algorithm~\ref{alg:k_wise_SQ}, and the Recovery Procedure (allowing the reconstruction of the affine subspace $V$) is separately described in \Cref{alg:recovery}. We denote the indicator function of event $E$ by $\ind(E)$. Note that the statistical query corresponding to the event $\ind(E)$ gives an estimate of the probability of $E$.
	%\vnote{Matrix $Z$ is defined only under this condition anyway. There is some confusion here between inputs to a query and (conditioned) random variables used for analysis. It would be more clear to use separate symbols for them.}
\begin{algorithm}[H]
\caption{$(k+1)$-wise SQ Algorithm}
\label{alg:k_wise_SQ}
{\bf Inputs.} $k \in \mathbb{N}$, error probability $\epsilon > 0$.\\
{\bf Output.} Function $f:\mathbb{F}_p^{k+1} \to \{\pm 1 \}$.
\begin{algorithmic}[1]
\State Set tolerance of each SQ to $\tau = (\epsilon/2^{c\cdot(k+2)})^{(k+1)^{k+3}}$, where $c>0$ is a large enough absolute constant.
\State Define the threshold $\tau_i = 2^{c \cdot (k+2-i)} \cdot k \cdot \tau^{1/(k+1)^{k+2-i}}$ for every $i \in [k+1]$.
\State Ask the SQ $\phi(z,b) \doteq \ind(b=1)$ and let $w$ be the response.
  \If{$w \le \epsilon -\tau$}
    \State\label{st:early_term} Output the all $-1$'s function.
  \EndIf
% \State For $(z_1,\dots,z_{k+1}) \in (\mathbb{F}_p^{k+1})^{k+1}$ and $(b_1,\dots,b_{k+1}) \in \{\pm 1\}^{k+1}$, let:
\State Let $\widetilde{\phi}\big((z_1,b_1),\ldots,(z_{k+1},b_{k+1}) \big) \doteq \ind((b_1,\dots,b_{k+1}) = 1^{k+1})$.
\State\label{st:v_resp} Ask the SQ $\widetilde{\phi}$ and let $v$ be the response.
\For{$i = k+1$ down to $1$}
       % \State For $(z_1,\dots,z_{k+1}) \in (\mathbb{F}_p^{k+1})^{k+1}$ and $(b_1,\dots,b_{k+1}) \in \{\pm 1\}^{k+1}$, let:
\State Let $\phi_i\big((z_1,b_1),\ldots,(z_{k+1},b_{k+1})  \big) \doteq \ind((b_1,\dots,b_{k+1}) = 1^{k+1}\mbox{ and }\rk(Z) = i$).
	\State Ask the SQ $\phi_i$ and let $v_i$ be the response.
  \If{$v_i/v \geq \tau_i$}
    %\State\label{st:LI_col} \textcolor{red}{Using $(k+1)$-wise SQs, find subset $S \subseteq [k+1]$ of $i$ linearly independent columns of $Z$.}
    %\State\label{st:lin_comb} Using $(k+1)$-wise SQs, $\forall t \in [k]$, write $z_t$ as a linear comb. of $\{z_s: s \in S \setminus \{k+1\}\} \cup \{1^{k+1}\}$.
    %\State Let $V \subseteq \mathbb{F}_p^{k+1}$ be the resulting subspace.
	\State Run Recovery Algorithm on input $(i,v_i)$ and let $\widehat{V}$ be the subspace of $\mathbb{F}_p^{k+1}$ it outputs.
	      		\State Define function $f:\mathbb{F}_p^{k+1} \to \{-1,1\}$ by:
	\State $f(z_1,\dots,z_{k+1}) = +1$ if $(z_1,\dots,z_{k+1}) \in \widehat{V}$.
	\State $f(z_1,\dots,z_{k+1}) = -1$ otherwise.
	\State Return $f$.
  \EndIf
      \EndFor

\end{algorithmic}
\end{algorithm}
%\vnote{In this description $Z$ and $(z_{1,k+1}, z_{2,k+1}, \dots, z_{k+1,k+1})$ are inputs to the procedure. They are not and are only defined inside a query function. So the query is defined as: $\phi_{j,d}\big((z_1,b_1),\ldots,(z_{k+1},b_{k+1})  \big)$ is equal to 1 if $(b_1,\dots,b_{k+1}) = 1^{k+1}$  and $\rk(Z) = i$ and bit $j$ of the description of the solution $a$ to the system $Ma=\bar{z}_{k+1}$ is equal to $d$. Here $M$ is the matrix that is formed from $z_1,...,z_{k+1}$ in the same as you do for $Z$ (I do not use the same name to keep the random variable and its distribution separate from specific instantiations).}
%\vnote{Another small comment is that there is no need to separately compute $S$ and $a$. There is a unique subspace and its entire description is short.}
\begin{algorithm}[H]
	\caption{Recovery Procedure}
	\label{alg:recovery}
	{\bf Input.} Integer $i \in [k+1]$.\\
	{\bf Output.} Subspace $\widehat{V}$ of $\mathbb{F}_p^{k+1}$ of dimension $i$.
	\begin{algorithmic}[1]
		%\State \textcolor{red}{Let $i = |S|$.}
		 %\State\label{st:solve} \textcolor{red}{Using $i$-wise SQs, solve the system $\{z_{k+1,\ell} = \sum_{v \in S \cap [k]} a_v z_{v,\ell} + a_{k+1}: \ell \in [i]\}$ for $a \in \mathbb{F}_p^i$.}
         \State Let $m_i = (k+2) \cdot i \cdot \lceil \log p \rceil$
		 \For{each bit $j \leq m_i$}% in the binary representation of output subspace $W$}
		 \State Define event $E_j(Z) = \ind(\text{bit } j \text{ of row span of } Z \text{ is } 1)$.
		 \State Let $\phi_{i,j} \big((z_1,b_1),\ldots,(z_{k+1},b_{k+1})  \big) \doteq \ind( E_j(Z) \mbox{ and } (b_1,\dots,b_{k+1}) = 1^{k+1}\mbox{ and }\rk(Z) = i$).
		 \State Ask the SQ $\phi_{i,j}$ and let $u_{i,j}$ be the response.
		
		    \If{$u_{i,j}/v_i \geq (9/10)$}
		    \State Set bit $j$ in binary representation of $\widehat{W}$ to $1$.
		    \Else
		    \State Set bit $j$ in binary representation of $\widehat{W}$ to $0$.
		    \EndIf
		  \EndFor
		  \State Let $\widehat{V}$ be the set all points $(z_1,\dots,z_{k+1})$ such that $(z_1,\dots,z_{k+1},1) \in \widehat{W}$.
	\end{algorithmic}
\end{algorithm}


%\texttt{<do stuff>}
%\textcolor{red}{We point out that in Step~\ref{st:LI_col} of Algorithm~\ref{alg:k_wise_SQ}, the subset $S$ can be taken to be the first subset of $i$ linearly independent columns of $Z$ that includes the last column, in some canonical ordering of the subsets of columns of $Z$. Moreover, Step~\ref{st:LI_col} can be implemented using $i \cdot \lceil \log_2(k+1) \rceil$ many $(k+1)$-wise SQs by asking queries of the form ``$(b_1,\dots,b_{k+1})=1^{k+1}$ and $\rk[Z] = i$ and the $r$th bit of the $m$th element of $S$ is $1$'' (for some $r \in [\lceil \log_2(k+1) \rceil]$ and $m \in [i]$). Furthermore, Step~\ref{st:solve} of \Cref{alg:recovery} (the recovery procedure) can be implemented using $i \cdot \lceil \log_2(p) \rceil$ many \textcolor{red}{$i$-wise} SQs by asking queries of the form ``$(b_1,\dots,b_{k+1})=1^{k+1}$ and $\rk[Z] = i$ and the $r$th bit in the binary expansion of $a_v$ is $1$ where $a$ is the solution to the system  $\{z_{k+1,\ell} = \sum_{v \in S \cap [k]} a_v z_{v,\ell} + a_{k+1}: \ell \in [i]\}$'' (for some $r \in [\lceil \log_2(p) \rceil]$ and $v \in [i]$).}

% Similarly, Step~\ref{st:lin_comb} can be implemented using $k \cdot i \cdot \lceil \log_2(p) \rceil$ many $(k+1)$-wise SQs.

\paragraph{Analysis}
We now turn to the analysis of Algorithm~\ref{alg:k_wise_SQ} and the proof of Lemma~\ref{le:k_ub}. We will need the following lemma, which shows that if the rank of $Z$ is at most $i$ with high probability, then there is a \emph{fixed} subspace of dimension at most $i$ containing most of the probability mass under the row-distribution of $Z$.
%\vnote{It would be useful to explain how this Lemma is related to the high-level description.}
\begin{lem}\label{le:ex_sub}
Let $i \in [k+1]$. If $\Pr_{Q^{k+1}}[\rk(Z) \le i] \geq 1-\xi$, then there exists a subspace $W$ of $\mathbb{F}_p^{k+2}$ of dimension at most $i$ such that $\Pr_{z \sim Q}[z \notin W] \le \xi^{1/k}$.
\end{lem}

\begin{remark}
We point out that the exponential dependence on $1/k$ in the probability upper bound in Lemma~\ref{le:ex_sub} is tight. To see this, let $p = 2$, and $\{e_1, \dots , e_k\}$ be the standard basis in $\mathbb{F}_2^k$. Consider the base distribution $P$ on $\mathbb{F}_2^k$ that puts probability mass $1-\alpha$ on $e_1$, and probability mass $\alpha/(k-1)$ on each of $e_2$, $e_3$, $\dots$, $e_k$. Then, a Chernoff bound implies that if we draw $k$ i.i.d. samples from $P$, then the dimension of their span is at most $2 \cdot \alpha \cdot k$ with probability at least $1 - \exp(-k)$. On the other hand, for any subspace $W$ of $\mathbb{F}_2^k$ of dimension $2 \cdot \alpha \cdot k$, the probability that a random sample from $P$ lies inside $W$ is only $1- \Theta(\alpha)$.
\end{remark}

To prove Lemma~\ref{le:ex_sub}, we will use the following proposition.
\begin{proposition}\label{prop:ind}
Let $\ell \in [k+1]$, $i \in [\ell-1]$ and $\eta >0$. If $\Pr_{Q^{\ell}}[\rk(Z_{\ell}) \le i] \geq 1-\eta$, then for every $\nu \in (0,1]$, either there exists a subspace $W$ of $\mathbb{F}_p^{k+2}$ of dimension $i$ such that $\Pr_{z \sim Q}[z \notin W] \le \nu$ or $\Pr_{Q^i}[\rk(Z_{i}) \le i-1] \geq 1-\eta/\nu$.
\end{proposition}

\begin{proof}
Let $p \doteq \Pr_{Q^i}[\rk(Z_{i}) \le i-1]$. For every (fixed) matrix $A_i \in \mathbb{F}_p^{i \times (k+2)}$, define
$$\mu(A_i) \doteq \Pr_{Q^\ell}[\rk(Z_{\ell}) \le i ~ | ~ Z_{i} = A_i].$$
Then,
\begin{align*}
\Pr_{Q^{\ell}}[\rk(Z_{\ell}) \le i] &= p+(1-p)\cdot \Pr_{Q^{\ell}}[\rk(Z_{\ell}) \le i ~ | ~ \rk(Z_{i}) = i]\\
&= p+(1-p)\cdot \Ex_{ Q^i}\bigg[\mu(Z_i) \bigg| ~ \rk(Z_{i}) = i \bigg].
\end{align*}
%\vnote{I think the last line can be very confusing. What is the internal probability over? I think you could clear it up by first defining $\mu(A) \doteq \Pr_{Q^\ell}[\rk(Z_{\ell}) \le i ~ | ~ Z_{i} = A]$ and then using $\Ex_{ Q^i}\bigg[\mu(Z_i) \bigg| ~ \rk(Z_{i}) = i \bigg]$ (the expectation is just over $Q^i$ since you conditioning on $\rk(Z_{i}) = i$ explicitly). }
Since $\Pr_{Q^{\ell}}[\rk(Z_{\ell}) \le i] \geq 1-\eta$, we have that
$$  \Ex_{ Q^i}\bigg[\mu(Z_i) \bigg| ~ \rk(Z_{i}) = i \bigg] \geq 1 - \eta/(1-p). $$
Hence, there exists a setting $A_i \in \mathbb{F}_p^{i \times (k+2)}$ of $Z_{i}$  such that $\rk(A_{i}) = i$ and
$$\Pr[\rk(Z_{\ell}) \le i ~ | ~ Z_{i} = A_{i}] \geq 1 - \eta/(1-p).$$
We let $W$ be the $\mathbb{F}_p$-span of the rows of $A_{i}$. Note that the dimension of $W$ is equal to $i$ and that $\Pr_{z \sim Q}[z \notin W] \le \eta/(1-p)$. Thus, we conclude that for every $\nu \in (0,1]$, either $p \geq 1-\eta/\nu$ or $\Pr_{z \sim Q}[z \notin W] \le \nu$, as desired.
\end{proof}

We now complete the proof of Lemma~\ref{le:ex_sub}.
\begin{proof}[Proof of Lemma~\ref{le:ex_sub}]
Starting with $\ell = k+1$ and $\eta = \xi$, we inductively apply Proposition~\ref{prop:ind} with $\nu = \xi^{1/k}$ until we either get the desired subspace $W$ or we get to the case where $i=1$. In this case, we have that $\Pr_{Q^{\ell}}[\rk(Z_{\ell}) \le 1] \geq 1-\xi^{1/k}$ for $\ell \geq 2$. Since the last column of $Z_{\ell}$ is the all $1$'s vector, we conclude that there exists $z^* \in \mathbb{F}_p^{k+1}$ such that $\Pr_{z \sim Q}[z \neq (z^*,1)] \le \xi^{1/k}$. We can then set our subspace $W$ to be the $\mathbb{F}_p$-span of the vector $(z^*,1)$.
\end{proof}

For the proof of Lemma~\ref{le:k_ub} we will also need the following lemma, which states sufficient conditions under which the Recovery Procedure (\Cref{alg:recovery}) succeeds.
\begin{lem}\label{le:recovery}
	Let $i \in [k+1]$. Assume that in \Cref{alg:k_wise_SQ}, $v > \epsilon^{k+1}/2$ and $v_i/v \geq \tau_i$. If there exists a subspace $W$ of $\mathbb{F}_p^{k+2}$ of dimension equal to $i$ such that
	\begin{equation}\label{eq:lemma_W_assumption}
	\Pr_{z \sim Q}[z \notin W] < \frac{\tau_i} {4 \cdot (k+1)},
	\end{equation}
	then the affine subspace $\widehat{V}$ output by \Cref{alg:recovery} (i.e., the Recovery Procedure) consists of all points $(z_1,\dots,z_{k+1})$ such that $(z_1,\dots,z_{k+1},1) \in W$.
\end{lem}

We note that \Cref{le:recovery} would still hold under quantitatively weaker assumptions on $v$, $v_i/v$ and $\Pr_{z \sim Q}[z \notin W]$ in \Cref{eq:lemma_W_assumption}. In order to keep the expressions simple, we however choose to state the above version which will be sufficient to prove \Cref{le:k_ub}. The proof of \Cref{le:recovery} appears in \Cref{subsec:pf_rec_lem}. We are now ready to complete the proof of Lemma~\ref{le:k_ub}.

\begin{proof}[Proof of Lemma~\ref{le:k_ub}]
If Algorithm~\ref{alg:k_wise_SQ} terminates at Step~\ref{st:early_term}, then the error of the output hypothesis is at most $\epsilon$, as desired. Henceforth, we assume that Algorithm~\ref{alg:k_wise_SQ} does not terminate at Step~\ref{st:early_term}. Then, we have that $\Pr[b = 1] > \epsilon$, and hence $\Pr[(b_1,\dots,b_{k+1}) = 1^{k+1}] > \epsilon^{k+1}$. Thus, the value $v$ obtained in Step~\ref{st:v_resp} of Algorithm~\ref{alg:k_wise_SQ} satisfies $v > \epsilon^{k+1} - \tau \geq \epsilon^{k+1}/2$, where the last inequality follows from the setting of $\tau$. Let $i^*$ be the first (i.e., largest) value of $i \in  [k+1]$ for which $v_i/v \geq \tau_i$. To prove that such an $i^*$ exists, we proceed by contradiction, and assume that for all $i \in [k+1]$, it is the case that $v_i/v < \tau_i$. Note that $Z$ has an all $1$'s column, so it has rank at least $1$. Moreover, it has rank at most $k+1$. Therefore, we have that
\begin{align*}
1 &= \Pr[1 \le \rk(Z) \le k+1 ~ |~ (b_1,\dots,b_{k+1})=1^{k+1}]\\
&= \displaystyle\sum\limits_{i=1}^{k+1} \Pr[\rk(Z) = i ~ | ~ (b_1,\dots,b_{k+1})=1^{k+1}]\\
&\le \displaystyle\sum\limits_{i=1}^{k+1} \frac{v_i + \tau}{v - \tau}\\
&\le 2 \cdot \displaystyle\sum\limits_{i=1}^{k+1} \frac{v_i + \tau}{v}\\
&\le 2 \cdot \displaystyle\sum\limits_{i=1}^{k+1} (\frac{v_i}{v}  + \frac{2\tau}{\epsilon^{k+1}})\\
&< 2 \cdot \displaystyle\sum\limits_{i=1}^{k+1} \tau_i + 4 \cdot (k+1) \cdot \frac{\tau}{\epsilon^{k+1}}.
\end{align*}
Using the fact that $\tau_i$ is monotonically non-increasing in $i$ and the settings of $\tau_1$ and $\tau$, the last inequality gives
\begin{align*}
1 &\le 2 \cdot (k+1) \cdot \tau_1 + 4 \cdot (k+1) \cdot \frac{\tau}{\epsilon^{k+1}} < 1,
\end{align*}
a contradiction.

We now fix $i^*$ as above. We have that
\begin{align*}
\Pr[\rk(Z) \le i^* ~ | ~ (b_1,\dots,b_{k+1})=1^{k+1}] &= 1 - \displaystyle\sum\limits_{i = i^*+1}^{k+1} \Pr[\rk(Z) = i ~ | ~ (b_1,\dots,b_{k+1})=1^{k+1}]\\
&\geq 1 - \displaystyle\sum\limits_{i = i^*+1}^{k+1} \frac{v_i + \tau}{v-\tau}\\
&\geq 1 - 2 \cdot \displaystyle\sum\limits_{i = i^*+1}^{k+1} (\frac{v_i}{v}  + \frac{2\tau}{\epsilon^{k+1}})\\
&> 1 - 2 \cdot \displaystyle\sum\limits_{i = i^*+1}^{k+1} (\tau_i + 2 \cdot \frac{\tau}{\epsilon^{k+1}})\\
& \geq 1 - 4 \cdot \displaystyle\sum\limits_{i = i^*+1}^{k+1} \tau_i\\
&\geq 1 - 4 \cdot k\cdot \tau_{i^*+1}.
%&= 1 - \textcolor{red}{2^{6\cdot (k-i^*)}\cdot k^2 \cdot  \tau^{1/k^{k-i^*}}}
\end{align*}
By Lemma~\ref{le:ex_sub}, there exists a subspace $W$ of $\mathbb{F}_p^{k+2}$ of dimension at most $i^*$ such that
\begin{equation}\label{eq:notin_W}
\Pr_{z \sim Q}[z \notin W] \le (4 \cdot k)^{1/k} \cdot \tau_{i^*+1}^{1/k}.
\end{equation}

\begin{proposition}\label{prop:failure}
	For every $i \in [k]$, we have that $(k+1) \cdot (4 \cdot k)^{1/k} \cdot \tau_{i+1}^{1/k} \le \tau_{i }/4$.
\end{proposition}
We note that \Cref{prop:failure} follows immediately from the definitions of $\tau_{i}$ and $\tau$ (and by letting $c$ by a sufficiently large positive absolute constant). Moreover, \Cref{prop:failure} (applied with $i = i^*$) along with \Cref{eq:notin_W} imply that $\Pr_{z \sim Q}[z \notin W]$ is at most $\tau_{i*}/(4(k+1))$.

By a union bound, we get that with probability at least
\begin{equation}\label{eq:alg_succ_prob}
1- (k+1) \cdot \Pr_{z \sim Q}[z \notin W] \geq 1 - \frac{\tau_{i^*}}{4},
\end{equation}
all the rows of $Z$ belong to $W$.

Since $v_{i*}/v \geq \tau_{i*}$, we also have that:
\begin{align}\label{eq:cond_lb_i_star}
\Pr[\rk(Z) = i^* ~ | ~ (b_1,\dots,b_{k+1})=1^{k+1}] &\geq \frac{v_{i*}-\tau}{v + \tau}\nonumber\\
&\geq \frac{1}{2} \cdot \frac{(v_{i*}-\tau)}{v}\nonumber\\
&\geq \frac{1}{2} \cdot (\tau_{i^*} - \frac{2 \cdot \tau}{\epsilon^{k+1}})\nonumber\\
& \geq \frac{\tau_{i^*}}{3}
\end{align}
Combining \Cref{eq:alg_succ_prob} and \Cref{eq:cond_lb_i_star}, we get that the rank of $W$ is \emph{equal to} $i^*$.

Let $V$ be the affine subspace consisting of all points $(z_1,\dots,z_{k+1})$ such that $(z_1,\dots,z_{k+1},1) \in W$. By \Cref{le:recovery}, we get that \Cref{alg:recovery} (and hence \Cref{alg:k_wise_SQ}) correctly recovers the affine subspace $V$.

We note that the function $f$ output by Algorithm~\ref{alg:k_wise_SQ} is the $\pm 1$ indicator of a subspace of the true hyperplane $\Hyp_a$. To see this, note that $f$ is the $\pm 1$ indicator function of the subspace $V$, and by Equations~(\ref{eq:notin_W}) and (\ref{eq:cond_lb_i_star}), we have that with probability at least $\tau_{i*}/12$ over $Z \sim Q^{k+1}$, all the columns of $Z$ belong to $W$ and $\rk(Z) = i^*$. Since the dimension of $W$ is equal to $i^*$ and since we are conditioning on $(b_1,\dots,b_{k+1})=1^{k+1}$, this implies that the correct label of all the points in $V$ is $+1$. Hence, $f$ only possibly errs on positively-labeled points (by wrongly giving them the label $-1$). Moreover, Algorithm~\ref{alg:k_wise_SQ} ensures that the output function $f$ gives the label $+1$ to every $(z_1,\dots,z_{k+1}) \in \mathbb{F}_p^{k+1}$ for which $(z_1,\dots,z_{k+1},1) \in W$. Therefore, the function $f$ that is output by Algorithm~\ref{alg:k_wise_SQ} (when it does not terminate at Step~\ref{st:early_term}) has error at most the right hand side of (\ref{eq:notin_W}). So to upper-bound the error probability, it suffices for us to verify that the right-hand side of (\ref{eq:notin_W}) is at most $\epsilon$. This is obtained by applying the next proposition with $i = i^*+1$.
\begin{proposition}\label{prop:tau_i_pow_1_ov_k}
	For every $i \in [k+1]$, we have that $(4 \cdot k)^{1/k} \cdot \tau_{i}^{1/k} \le \epsilon^{k} $.
\end{proposition}

The proof of \Cref{prop:tau_i_pow_1_ov_k} follows immediately from the definitions of $\tau_{i}$ and $\tau$ and by letting $c$ be a sufficiently large positive absolute constant.

The number of queries performed by the $(k+1)$-wise algorithm is at most $O(k^2 \cdot \log{p})$, and their tolerance is $\tau \geq (\epsilon/2^{c\cdot(k+2)})^{(k+1)^{k+3}}$, where $c$ is a positive absolute constant. Finally, we remark that the dependence of the SQ complexity of the above algorithm on the error parameter $\eps$ is $\eps^{-k^{O(k)}}$. It can be improved to a linear dependence on $1/\eps$ by learning with error $1/3$ and then using boosting in the standard way (boosting in the SQ model works essentially as in the regular PAC model \cite{aslam1993general}).
\end{proof}

\subsection{Lower bound}
Our proof of lower bound is a generalization of the lower bound in \cite{Feldman:16sqd} (for $\ell=2$ and $k=1$). It relies on a notion of {\em combined randomized statistical dimension} (``combined" refers to the fact that it examines a single parameter that lower bounds both the number of queries and the inverse of the tolerance).
In order to apply this approach we need to extend it to $k$-wise queries. This extension follows immediately from a simple observation. If we define the domain to be $X' \doteq X^k$ and the input distribution to be $D' \doteq D^k$ then asking a $k$-wise query $\phi:X^k \to [-1,1]$ to $\STAT^{(k)}_D(\tau)$ is equivalent to asking a unary query $\phi: X' \to [-1,1]$ to $\STAT^{(k)}_{D'}(\tau)$. Using this observation we define the $k$-wise versions of the notions from \cite{Feldman:16sqd} and give their properties that are needed for the proof of Lemma~\ref{le:k_plus_1_lb}.

\subsubsection{Preliminaries}
Combined randomized statistical dimension is based on the following notion of average discrimination.
\begin{defn}[$k$-wise average $\kappa_1$-discrimination]\label{def:kappa_1_disc}
Let $k$ be any positive integer. Let $\mu$ be a probability measure over distributions over $X$ and $D_0$ be a reference distribution over $X$. Then,
$$ \bar{\kappa}_1^{(k)}(\mu,D_0) \doteq \sup_{\phi: X^k \to [-1,+1]} \bigg\{ \Ex_{D \sim \mu}[|D^k[\phi]-D_0^k[\phi]|] \bigg\}. $$
\end{defn}

We denote the problem of PAC learning a concept class $\calC$ of Boolean functions up to error $\epsilon$ by $\mathcal{L}_{PAC}(\calC,\epsilon)$. Let $Z$ be the domain of the Boolean functions in $\calC$. For any distribution $D_0$ over labeled examples (i.e., over $Z \times \{\pm 1\}$), we define the Bayes error rate of $D_0$ to be
\begin{equation*}
\err(D_0) = \displaystyle\sum\limits_{z \in Z} \min\{D_0(z,1) , D_0(z,-1)\} = \min_{h: Z \to \{\pm1 \}} \Pr_{(z,b) \sim D_0}[h(z) \neq b].
\end{equation*}

%The following statistical dimension -- defined for decision problems -- will be useful when proving Lemma~\ref{le:k_plus_1_lb}.
\begin{defn}[$k$-wise combined randomized statistical dimension]\label{def:csdr}
Let $k$ be any positive integer. Let $\calD$ be a set of distributions and $D_0$ a reference distribution over $X$. The $k$-wise combined randomized statistical dimension of the decision problem $\calB(\calD,D_0)$ is then defined as
$$ \CSDR_{\bar{\kappa}_1}^{(k)}(\calB(\calD,D_0)) \doteq \sup_{\mu \in S^{\calD}} (\bar{\kappa}_1^{(k)}(\mu,D_0))^{-1}, $$
where $S^\D$ denotes the set of all probability distributions over $\D$.

Further, for any concept class $\calC$ of Boolean functions over a domain $Z$, and for any $\epsilon > 0$, the $k$-wise combined randomized statistical dimension of $\mathcal{L}_{PAC}(\calC,\epsilon)$ is defined as
\begin{equation*}
\CSDR_{\bar{\kappa}_1}^{(k)}(\mathcal{L}_{PAC}(\calC,\epsilon)) \doteq \sup_{D_0 \in S^{Z \times \{\pm 1\}}: \err(D_0) > \epsilon} \CSDR_{\bar{\kappa}_1}^{(k)}(\calB(\calD_{\calC},D_0)),
\end{equation*}
where $\calD_{\calC} \doteq \{ P^f: P \in S^{Z}, f \in \calC\}$ with $P^f$ denoting the distribution on labeled examples $(x,f(x))$ with $x \sim P$.
\end{defn}

The next theorem lower bounds the randomized $k$-wise SQ complexity of PAC learning a concept class in terms of its $k$-wise combined randomized statistical dimension.% (introduced in Definition~\ref{def:csdr}).

\begin{theorem}[\cite{Feldman:16sqd}]\label{thm:sq_RSD}
Let $\calC$ be a concept class of Boolean functions over a domain $Z$, $k$ be a positive integer and $\epsilon, \delta > 0$. Let $d \doteq \CSDR_{\bar{\kappa}_1}^{(k)}(\mathcal{L}_{PAC}(\calC,\epsilon))$. Then, the randomized $k$-wise SQ complexity of solving $\mathcal{L}_{PAC}(\calC,\epsilon - 1/\sqrt{d})$ with access to $\STAT^{(k)}(1/\sqrt{d})$ and success probability $1-\delta$ is at least $(1-\delta) \cdot \sqrt{d} - 1$.
\end{theorem}


To lower bound the statistical dimension we will use the following ``average correlation'' parameter introduced in \cite{FeldmanGRVX:12}.
\begin{defn}[$k$-wise average correlation]\label{def:rho}
Let $k$ be any positive integer. Let $\calD$ be a set of distributions and $D_0$ a reference distribution over $X$. Assume that the support of every distribution $D \in \calD$ is a subset of the support of $D_0$. Then, for every $x \in X^k$, define $\hat{D}(x) \doteq \frac{D^k(x)}{D_0^k(x)} - 1$. Then, the $k$-wise average correlation is defined as
$$ \rho^{(k)}(\calD,D_0) \doteq \frac{1}{|\calD|^2} \cdot \displaystyle\sum\limits_{D, D' \in \calD} | D_0^k[\hat{D} \cdot \hat{D}']|. $$
\end{defn}

Lemma~\ref{lem:ub_rho} relates the average correlation to the average discrimination (from Definition~\ref{def:kappa_1_disc}).
\begin{lem}[\cite{Feldman:16sqd}]\label{lem:ub_rho}
Let $k$ be any positive integer. Let $\calD$ be a set of distributions and $D_0$ a reference distribution over $X$. Let $\mu$ be the uniform distribution over $\calD$. Then,
$$ \bar{\kappa}_1^{(k)}(\mu,D_0) \le 4 \cdot \sqrt{\rho^{(k)}(\calD,D_0)}. $$
\end{lem}



\subsubsection{Proof of Lemma~\ref{le:k_plus_1_lb}}\label{subsec:sep_lb}

Denote $X \doteq \mathbb{F}_p^{\ell} \times \{\pm 1\}$. Let $\calD$ be the set of all distributions over $X^k$ that are obtained by sampling from any given distribution over $(\mathbb{F}_p^{\ell})^k$ and labeling the $k$ samples according to any given hyperplane indicator function $f_a$. Let $D_0$ be the uniform distribution over $X^k$. We now show that $\CSDR_{\bar{\kappa}_1}(\calB(\calD,D_0)) = \Omega\big(p^{(\ell-k)/2}\big)$. By definition,
$$ \CSDR_{\bar{\kappa}_1}(\calB(\calD,D_0)) \doteq \sup_{\mu \in S^{\calD}} (\bar{\kappa}_1(\mu,D_0))^{-1}. $$
We now choose the distribution $\mu$. For $a \in \mathbb{F}_p^{\ell}$, we define $P_a$ to be the distribution over $\mathbb{F}_p^{\ell}$ that has density $\alpha = 1/(2 (p^{\ell}-p^{\ell-1}))$ on each of the $p^{\ell}-p^{\ell-1}$ points outside $\Hyp_a$, and density $\beta = 1/p^{\ell-1}-\alpha p +\alpha = 1/(2p^{\ell-1})$ on each of the $p^{\ell-1}$ points inside $\Hyp_a$. We then define $D_a$ to be the distribution obtained by sampling $k$ i.i.d.~random examples of $\Hyp_a$, the marginal of each over $\mathbb{F}_p^{\ell}$ being $P_a$. Let $\calD' \doteq \{D_a ~ | ~ a \in \mathbb{F}_p^{\ell}\}$, and let $\mu$ be the uniform distribution over $\calD'$. By Lemma~\ref{lem:ub_rho}, we have that $\bar{\kappa}_1(\mu,D_0) \le 4 \cdot \sqrt{\rho(\calD,D_0)}$, so it is enough to upper bound $\rho(\calD,D_0)$.

We first note that for $a, a' \in \mathbb{F}_p^{\ell}$, we have
\begin{align*}
D_0[\hat{D}_a \cdot \hat{D}_{a'}] &= \Ex_{(z,b) \sim D_0} [ \hat{D}_a(z,b) \cdot \hat{D}_{a'}(z,b)]\\
&= \Ex_{(z,b) \sim D_0} \bigg[ \bigg(\frac{D_a(z,b)}{D_0(z,b)}-1 \bigg) \cdot \bigg(\frac{D_{a'}(z,b)}{D_0(z,b)}-1\bigg)\bigg]\\
&= \Ex_{(z,b) \sim D_0} \bigg[ \frac{D_a(z,b) \cdot D_{a'}(z,b)}{D_0^2(z,b)} - \frac{D_a(z,b)}{D_0(z,b)} - \frac{D_{a'}(z,b)}{D_0(z,b)} +1\bigg]\\
&= \Ex_{(z,b) \sim D_0} \bigg[ \frac{D_a(z,b) \cdot D_{a'}(z,b)}{D_0^2(z,b)}\bigg] - 2 \cdot \Ex_{(z,b) \sim D_0} \bigg[\frac{D_a(z,b)}{D_0(z,b)}\bigg] +1\\
&= 2^{2k} \cdot p^{2 k \ell} \cdot \Ex_{(z,b) \sim D_0}[D_a(z,b) \cdot D_{a'}(z,b)] - 2^{k+1} \cdot p^{k \ell} \cdot \Ex_{(z,b) \sim D_0}[D_a(z,b)] + 1
\end{align*}

We now compute each of the two expectations that appear in the last equation above.

\begin{proposition}\label{prop:single_ex}
For every $a \in \mathbb{F}_p^{\ell}$,
$$ \Ex_{(z,b) \sim D_0}[D_a(z,b)] = \frac{1}{2^k} \cdot \bigg(\frac{1}{p} \cdot \beta + \bigg(1-\frac{1}{p}\bigg) \cdot \alpha\bigg)^k = \frac{1}{2^k \cdot p^{k\cdot \ell}}.$$
\end{proposition}

The proof of Proposition~\ref{prop:single_ex} appears in the appendix.

\begin{proposition}\label{prop:pair_ex}
For every $a, a' \in \mathbb{F}_p^{\ell}$,
\[
\Ex_{(z,b) \sim D_0}[D_a(z,b) \cdot D_{a'}(z,b)] = \begin{dcases*}
	& $\frac{1}{2^k} \cdot (\frac{1}{p} \cdot \beta^2 + (1-\frac{1}{p}) \cdot \alpha^2)^k$ if $\Hyp_a = \Hyp_{a'}$,\\
	& $\frac{1}{2^k}\cdot (\alpha^2 \cdot (1-\frac{2}{p}))^k$ if $\Hyp_a \cap \Hyp_{a'} = \emptyset$,\\
	& $\frac{1}{2^k} \cdot ( \frac{\beta^2}{p^2} +\alpha^2 \cdot (1-\frac{2}{p}+\frac{1}{p^2}))^k$ otherwise.
        \end{dcases*}
\]
\end{proposition}

The proof of Proposition~\ref{prop:pair_ex} appears in the appendix. Using Proposition~\ref{prop:single_ex} and Proposition~\ref{prop:pair_ex}, we now compute $D_0[\hat{D}_a \cdot \hat{D}_{a'}]$.

\begin{proposition}\label{prop:D_0}
For every $a, a' \in \mathbb{F}_p^{\ell}$,
\[
D_0[\hat{D}_a \cdot \hat{D}_{a'}] = \begin{dcases*}
	& $(p+1-\frac{1}{p-1})^k - 1$ if $\Hyp_a = \Hyp_{a'}$,\\
	& $\frac{1}{2^k} \cdot \frac{(1-\frac{2}{p})^k}{(1-\frac{1}{p})^{2k}}-1$ if $\Hyp_a \cap \Hyp_{a'} = \emptyset$,\\
	& $0$ otherwise.
        \end{dcases*}
\]
\end{proposition}

The proof of Proposition~\ref{prop:D_0} appears in the appendix. When computing $\rho(\calD,D_0)$, we will also use the following simple proposition.
\begin{proposition}\label{prop:pairs_hyp}
\begin{enumerate}
\item The number of pairs $(a,a') \in (\mathbb{F}_p^{\ell})^2$ such that $\Hyp_a = \Hyp_{a'}$ is equal to $p^{\ell}$.
\item The number of pairs $(a,a') \in (\mathbb{F}_p^{\ell})^2$ such that $\Hyp_a$ and $\Hyp_{a'}$ are distinct and parallel is equal to $p^{\ell}\cdot(p-1)$.
\item The number of pairs $(a,a') \in (\mathbb{F}_p^{\ell})^2$ such that $\Hyp_a$ and $\Hyp_{a'}$ are distinct and intersecting is equal to $p^{2\cdot \ell}-p^{\ell+1}$.
\end{enumerate}
\end{proposition}

Using Proposition~\ref{prop:D_0} and Proposition~\ref{prop:pairs_hyp}, we are now ready to compute $\rho(\calD,D_0)$ as follows
\begin{align*}
\rho(\calD,D_0) &\le \frac{1}{p^{2\cdot \ell}} \cdot \bigg[ p^{\ell} \cdot (p+1-\frac{1}{p-1})^k +p^{\ell} \cdot (p-1) + p^{2\cdot \ell} \cdot 0 \bigg]\\
&\le O\bigg(\frac{1}{p^{\ell-k}}\bigg) + \frac{1}{p^{\ell-1}}\\
&= O\bigg(\frac{1}{p^{\ell-k}}\bigg),
\end{align*}
where we used above the assumption that $k = O(p)$. We deduce that $\bar{\kappa}_1(\mu,D_0)  = O\bigg(1/p^{(\ell-k)/2}\bigg)$, and hence $\CSDR_{\bar{\kappa}_1}(\calB(\calD,D_0)) = \Omega\bigg(p^{(\ell-k)/2}\bigg)$. This lower bound on $\CSDR_{\bar{\kappa}_1}(\calB(\calD,D_0))$, along with Definition~\ref{def:csdr}, Theorem~\ref{thm:sq_RSD} and the fact that $D_0$ has Bayes error rate equal to $1/2$, imply Lemma~\ref{le:k_plus_1_lb}.



\iffalse
In order to prove \Cref{le:recovery}, we will need the following straightforward propositions.

\begin{proposition}\label{prop:frac_lb}
	For any positive real numbers $N$, $D$ and $\tau$ such that $\tau = o(N)$ and $\tau = o(D)$, we have that
	\begin{equation*}
	\frac{N-\tau}{D+\tau} \geq \frac{N}{D} \cdot (1-o(1)).
	\end{equation*}
\end{proposition}

\begin{proposition}\label{prop:frac_ub}
	For any positive real numbers $N$, $D$ and $\tau$ such that $\tau = o(D)$, we have that
	\begin{equation*}
	\frac{N+\tau}{D-\tau} \le \frac{N}{D} \cdot (1+o(1)) + o(1).
	\end{equation*}	
\end{proposition}

\begin{proposition}\label{prop:tau_i_ub}
	For every $i \in [k+1]$, we have that $\tau_i = o_c(1)$.
\end{proposition}
\begin{proposition}\label{prop:bd_tau_v_tau_i}
	If $v > \epsilon^{k+1}/2$, then for every $i \in [k+1]$, we have that
	\begin{equation*}
	\tau = o_c \bigg( (v \cdot \tau_i - \tau ) \cdot (1-\tau_i/4)\bigg).
	\end{equation*}
\end{proposition}

The proof of \Cref{prop:bd_tau_v_tau_i} follows from the definitions of $\tau$ and $\tau_i$ in \Cref{alg:k_wise_SQ}.
\fi


\section{Reduction for flat distributions}\label{sec:pf_flat}
\newcommand{\dci}{{\kappa_1}}
To prove Theorem~\ref{thm:flat} we use the characterization of the SQ complexity of the problem of estimating $D^k[\phi]$ for $D\in \D$ using a notion of statistical dimension from \cite{Feldman:16sqd}. Specifically, we use the characterization of the complexity of solving this problem using unary SQs and also the generalization of this characterization that characterizes the complexity of solving a problem using $k$-wise SQs. The latter is equal to 1 (since a single $k$-wise SQ suffices to estimate $D^k[\phi]$). Hence the $k$-wise statistical dimension is also equal to 1. We then upper bound the unary statistical dimension by the $k$-wise statistical dimension. The characterization then implies that an upper bound on the unary statistical dimension gives an upper bound on the SQ complexity of estimating $D^k[\phi]$.

We also give a slightly different way to define flatness that makes it easier to extend our results to other notions of divergence.
\begin{defn}
Let $\calD$ be a set of distributions over $X$. Define
$$R_\infty(\D) \doteq \inf_{\bar D \in S^X} \sup_{D\in \D} \Div_\infty(D\|\bar D), $$
where $S^X$ denotes the set of all probability distributions over $X$ and $$\Div_\infty(D\|\bar D) \doteq \sup_{y\in X} \ln \frac{\Pr_{x\sim D}[x=y]}{\Pr_{x\sim \bar D}[x=y]}$$ denotes the max-divergence. We say that $\calD$ is $\gamma$-flat if $R_\infty(\D) \leq \ln \gamma$.
\end{defn}

For simplicity, we will start by relating the $k$-wise SQ complexity to unary SQ complexity for decision problems. The statistical dimension for this type of problems is substantially simpler than for the general problems but is sufficient to demonstrate the reduction. We then build on the results for decision problems to obtain the proof of Theorem~\ref{thm:flat}.

\subsection{Decision problems}
The $k$-wise generalization of the statistical dimension for decision problems from \cite{Feldman:16sqd} is defined as follows.
\begin{defn}
Let $k$ be any positive integer. Consider a set of distributions $\calD$ and a reference distribution $D_0$ over $X$. Let $\mu$ be a probability measure over $\calD$ and let $\tau > 0$. The $k$-wise maximum covered $\mu$-fraction is defined as
$$ \kappa_1\text{-}\fract^{(k)}(\mu,D_0,\tau) \doteq \sup_{\phi: X^k \to [-1,+1]} \bigg\{ \Pr_{D \sim \mu}[|D^k[\phi]-D_0^k[\phi]| > \tau] \bigg\}. $$
\end{defn}
\begin{defn}[$k$-wise randomized statistical dimension of decision problems]\label{def:rdm_sd}
Let $k$ be any positive integer. For any set of distributions $\cald$, a reference distribution $D_0$ over $X$ and $\tau > 0$, we define
$$ \RSD_{\kappa_1}^{(k)}(\calB(\calD,D_0), \tau) \doteq \sup_{\mu \in S^{\calD}} ( \kappa_1\text{-}\fract^{(k)}(\mu,D_0,\tau))^{-1}, $$
where $S^\D$ denotes the set of all probability distributions over $\D$.
\end{defn}

As shown in \cite{Feldman:16sqd}, $\RSD$ tightly characterizes the randomized statistical query complexity of solving the problem using $k$-wise queries. As observed before, the $k$-wise versions below are implied by the unary version in \cite{Feldman:16sqd} simply by defining the domain to be $X' \doteq X^k$ and the set of input distributions to be $\D' \doteq \{D^k \ |\  D \in \D\}$.

\begin{theorem}[\cite{Feldman:16sqd}]\label{thm:random-algorithm2queries}
Let $\calB(\D,D_0)$ be a decision problem, $\tau > 0, \delta \in (0,1/2)$, $k \in \mathbb{N}$ and $d=\RSD^{(k)}_\dci(\calB(\D,D_0),\tau)$. Then there exists a randomized algorithm that solves $\calB(\D,D_0)$ with success probability $\geq 1-\delta$ using $d \cdot \ln(1/\delta)$ queries to $\STAT^{(k)}_D(\tau/2)$. Conversely, any algorithm that solves $\calB(\D,D_0)$ with success probability $\geq 1-\delta$ requires at least $d \cdot (1-2\delta)$ queries to $\STAT^{(k)}_D(\tau)$.
\end{theorem}

We will also need the following dual formulation of the statistical dimension given in Theorem~\ref{def:rdm_sd}.
\begin{lem}[\cite{Feldman:16sqd}]\label{fa:rcvr}
Let $k$ be any positive integer. For any set of distributions $\cald$, a reference distribution $D_0$ over $X$ and $\tau > 0$,  the statistical dimension $\RSD_{\kappa_1}^{(k)}(\calB(\calD,D_0), \tau)$ is equal to the smallest $d$ for which there exists a distribution $\calP$ over functions from $X^k$ to $[-1,+1]$ such that for every $D \in \calD$,
$$ \Pr_{\phi \sim \calP}[|D^k[\phi]-D_0^k[\phi]| > \tau] \geq \frac{1}{d}.$$
\end{lem}

We can now state the relationship between $\RSD_{\kappa_1}^{(k)}$ and $\RSD_{\kappa_1}^{(1)}$ for any $\gamma$-flat $\D$.
\begin{lem}\label{lem:k-wise-flat-decision}
Let $\gamma \geq 1$, $\tau > 0$ and $k \in \mathbb{N}$. Let $X$ be a domain, $\calD$ be a $\gamma$-flat class of distributions over $X$ and $D_0$ be any distribution over $X$. Then
$$\RSD_{\kappa_1}^{(1)}(\calB(\calD,D_0),\tau/(2k))  \leq \frac{4k \cdot \gamma^{k-1}}{\tau} \cdot \RSD_{\kappa_1}^{(k)}(\calB(\calD,D_0),\tau).$$
\end{lem}
\begin{proof}
Let $d \doteq \RSD_{\kappa_1}^{(k)}(\calB(\calD,D_0),\tau)$. Fact~\ref{fa:rcvr} implies the existence of a distribution $\calP$ over $k$-wise functions such that for every $D \in \calD$,
$$\Pr_{\phi \sim \calP}[|D^k[\phi]-D_0^k[\phi]| > \tau] \geq \frac{1}{d}.$$
We now fix $D$ and let $\phi$ be such that $|D^k[\phi]-D_0^k[\phi]| > \tau$. 

By the standard hybrid argument,  
\begin{equation}\label{eq:good_j}
\E_{j \sim [k]} \left[\left|D^{j} D_0^{k-j}[\phi]-D^{j-1}D_0^{k-j+1}[\phi]\right|\right] > \frac{\tau}{k} ,
\end{equation}
where $j \sim [k]$ denotes a random and uniform choice of $j$ from $[k]$.
This implies that
\begin{equation*}\label{eq:pull_out}
\E_{j \sim [k]} \Ex_{x_{< j} \sim D^{j-1}} \Ex_{x_{> j} \sim D_0^{k-j}} \bigg[\bigg|D[\phi(x_{<j}, \cdot, x_{> j})] - D_0[\phi(x_{<j}, \cdot, x_{> j})]\bigg|\bigg] > \frac{\tau}{k}.
\end{equation*}
By an averaging argument (and using the fact that $\phi$ takes values between $-1$ and $+1$), we get that with probability at least $\tau/(4 \cdot k)$ over the choice of $j\sim [k]$, $x_{< j} \sim D^{j-1}$ and $x_{> j} \sim D_0^{k-j}$, we have that
\begin{equation*}\label{eq:after_whp_switch}
\bigg|D[\phi(x_{<j}, \cdot, x_{> j})] - D_0[\phi(x_{<j}, \cdot, x_{> j})]\bigg| > \frac{\tau}{2 \cdot k}.
\end{equation*}

Since $\calD$ is a $\gamma$-flat class of distributions, there exists a (fixed) distribution $\bar{D}$ over $X$ such that for every measurable event $E \subset X$, $\Pr_{x\sim D}[x \in E] \le \gamma \cdot \Pr_{x\sim \bar{D}}[x \in E]$. Thus, we can replace the unknown input distribution $D$ by the distribution $\bar{D}$ and get that, with probability at least $\tau/(4 \cdot k \cdot \gamma^{k-1})$ over the choice of $j\sim [k]$, $x_{< j} \sim \bar{D}^{j-1}$ and $x_{> j} \sim D_0^{k-j}$, we have
\begin{equation}\label{eq:after_flat}
\bigg|D[\phi(x_{<j}, \cdot, x_{> j})] - D_0[\phi(x_{<j}, \cdot, x_{> j})]\bigg| > \frac{\tau}{2 \cdot k}.
\end{equation}
We now consider the following distribution $\mathcal{P'}$ over unary SQ functions (i.e., over $[-1,+1]^X$): Independently sample $\phi$ from $\calP$, $j$ uniformly from $[k]$, $x_{< j} \sim \bar{D}^{j-1}$ and $x_{> j} \sim D_0^{k-j}$, and output the (unary) function $\phi'(x) = \phi(x_{<j}, x, x_{> j})$. Then, for every $D\in \D$, we have that with probability at least $\frac{1}{d }\cdot \frac{\tau}{4k} \cdot \frac{1}{\gamma^{k-1}}$ over the choice of $\phi'$ from $\calP'$, we have that $|D[\phi'] - D_0[\phi']| > \tau/(2 \cdot k)$. Thus, by Fact~\ref{fa:rcvr} $$\RSD_{\kappa_1}^{(1)}\left(\calB(\calD,D_0),\frac{\tau}{2 \cdot k}\right) \le \frac{4 d \cdot \gamma^{k-1} \cdot k}{\tau}.$$
\end{proof}

Lemma \ref{lem:k-wise-flat-decision} together with the characterization in Theorem \ref{thm:random-algorithm2queries} imply the following upper bound on the SQ complexity of a decision problem in terms of its $k$-wise SQ complexity.
\begin{theorem}\label{thm:flat-decision-reduction}
Let $\gamma \geq 1$, $\tau > 0$ and $k \in \mathbb{N}$. Let $X$ be a domain, $\calD$ be a $\gamma$-flat class of distributions over $X$ and $D_0$ be any distribution over $X$. If there exists an algorithm that, with probability at least $2/3$ solves $\calB(\D,D_0)$ using $t$ queries to $\STAT^{(k)}_D(\tau)$, then for every $\delta>0$, there exists an algorithm that, with probability at least $1-\delta$ solves $\calB(\D,D_0)$ using $t \cdot 12k \cdot \gamma^{k-1} \cdot \ln(1/\delta) /\tau$ queries to $\STAT^{(1)}_D(\tau/(4k))$.
\end{theorem}

\subsection{General problems}
We now define the general class of problems over sets of distributions and a notion of statistical dimension for these types of problems.
\begin{defn}[Search problems]
A search problem $\calZ$ over a class $\calD$ of distributions and a set $\calF$ of solutions is a mapping $\calZ: \calD \to 2^{\calF} \setminus \{\emptyset\}$, where $2^{\calF}$ denotes the set of all subsets of $\calF$. Specifically, for every distribution $D \in \calD$, $\calZ(D) \subseteq \calF$ is the (non-empty) set of valid solutions for $D$. For a solution $f \in \calF$, we denote by $\calZ_f$ the set of all distributions for which $f$ is a valid solution.
\end{defn}

\begin{defn}[Statistical dimension for search problems \cite{Feldman:16sqd}]\label{def:search_SD}
For $\tau > 0$, $k \in \mathbb{N}$, a domain $X$ and a search problem $\calZ$ over a class of distributions $\calD$ over $X$ and a set of solutions $\calF$, we define the \emph{$k$-wise statistical dimension} with $\kappa_1$-discrimination $\tau$ of $\calZ$ as
\begin{equation*}
\SD^{(k)}_{\kappa_1}(\calZ,\tau) \doteq \sup_{D_0 \in S^X} \inf_{f \in \calF} \RSD^{(k)}_{\kappa_1}(\calB(\calD \setminus \calZ_f, D_0), \tau),
\end{equation*}
where $S^X$ denotes the set of all probability distributions over $X$.
\end{defn}

Lemma~\ref{lem:search_lb} lower-bounds the deterministic $k$-wise SQ complexity of a search problem in terms of its ($k$-wise) statistical dimension.
\begin{theorem}[\cite{Feldman:16sqd}]\label{lem:search_lb}
Let $\calZ$ be a search problem, $\tau > 0$ and $k \in \mathbb{N}$. The deterministic $k$-wise SQ complexity of solving $\calZ$ with access to $\STAT^{(k)}(\tau)$ is at least $\SD^{(k)}_{\kappa_1}(\calZ,\tau)$.
\end{theorem}

The following theorem from \cite{Feldman:16sqd} gives an upper bound on the SQ complexity of a search problem in terms of its statistical dimension. It relies on the multiplicative weights update method to reconstruct the unknown distribution sufficiently well for solving the problem. The use of this algorithm introduces dependence on $\KL$-radius of $\D$. Namely, we define
$$\KLR(\D) \doteq \inf_{\bar D \in S^X } \sup_{D\in \D} \KL(D\|\bar D), $$
where $\KL(\cdot \| \cdot)$ denotes the KL-divergence.
\begin{theorem}
[\cite{Feldman:16sqd}]\label{lem:search_ub}
Let $\calZ$ be a search problem, $\tau, \delta > 0$ and $k \in \mathbb{N}$. There is a randomized $k$-wise SQ algorithm that solves $\calZ$ with success probability $1-\delta$ using
$$ O\bigg(\SD^{(k)}_{\kappa_1}(\calZ,\tau) \cdot \frac{\KLR(\calD)}{\tau^2} \cdot \log\bigg( \frac{\KLR(\calD)}{\tau \cdot \delta}\bigg) \bigg)$$
queries to $\STAT^{(k)}(\tau/3)$.
\end{theorem}

Note that $\KL$-divergence between two distributions is upper-bounded (and is usually much smaller) than the max-divergence we used in the definition of $\gamma$-flatness. Specifically, if $\D$ is $\gamma$-flat  then $\KLR(\D) \leq \ln \gamma$.
We are now ready to prove Theorem~\ref{thm:flat} which we restate here for convenience.
\begin{reptheorem}{thm:flat}[restated]
Let $\gamma \geq 1$, $\tau > 0$ and $k$ be any positive integer. Let $X$ be a domain and $\calD$ be a $\gamma$-flat class of distributions over $X$. There exists a randomized algorithm that given any $\delta > 0$ and a $k$-ary function $\phi: X^k \to [-1,1]$, estimates $D^k[\phi]$ within $\tau$  for every (unknown) $D \in \calD$ with success probability at least $1-\delta$ using $$\tilde{O}\bigg( \frac{\gamma^{k-1} \cdot k^3}{\tau^3} \cdot \log (1/\delta)\bigg)$$
queries to $\STAT_D^{(1)}(\tau/(6 \cdot k))$.\end{reptheorem}
\begin{proof}
We first observe that the task of estimating $D^k[\phi]$ up to additive $\tau$ can be viewed as a search problem $\calZ$ over the set $\calD$ of distributions and over the class $\calF$ of solutions that corresponds to the interval $[-1,+1]$. Next, observe that one can easily estimate $D^k[\phi]$ up to additive $\tau$ using a single query to $\STAT^{(k)}_D(\tau)$. Lemma~\ref{lem:search_lb} implies that $\SD^{(k)}_{\kappa_1}(\calZ,\tau) = 1$. By Definition~\ref{def:search_SD}, for every $D_1 \in S^X$, there exists $f \in \calF$, such that $\RSD^{(k)}_{\kappa_1}(\calB(\calD \setminus \calZ_f, D_1), \tau) = 1$. By Lemma \ref{lem:k-wise-flat-decision},
 $$\RSD_{\kappa_1}^{(1)}\left(\calB(\calD \setminus \calZ_f, D_1),\frac{\tau}{2 \cdot k}\right) \le \frac{4 \cdot \gamma^{k-1} \cdot k}{\tau}.$$

Thus, Fact~\ref{fa:rcvr} and Definition~\ref{def:search_SD} imply that
$$ \SD_{\kappa_1}^{(1)}(\calZ,\frac{\tau}{2 \cdot k}) \le \frac{4 \cdot \gamma^{k-1} \cdot k}{\tau}.$$
Applying Lemma~\ref{lem:search_ub}, we conclude that there exists a randomized unary SQ algorithm that solves $\calZ$ with probability at least $1-\delta$ using at most $$O\bigg(\gamma^{k-1} \cdot k^3 \cdot \frac{\KLR(\mathcal{D})}{\tau^3} \cdot \log\bigg( \frac{k \cdot \KLR(\calD)}{\tau \cdot \delta}\bigg)\bigg)$$
queries to $\STAT^{(1)}(\tau/(6 \cdot k))$. This -- along with the fact that $\KLR(\calD) \le \ln(\gamma)$ whenever $\calD$ is a $\gamma$-flat set of distributions -- concludes the proof of Theorem~\ref{thm:flat}.
\end{proof}

\paragraph{Other divergences:} While the max-divergence that we used for measuring flatness suffices for the applications we give in this paper (and is relatively simple), it might be too conservative in other problems. For example, such divergence is infinite even for two Gaussian distributions with the same standard deviation but different means. A simple way to obtain a more robust version of our reduction is to use approximate max-divergence. For $\delta \in [0,1)$ it is defined as: $$\Div_\infty^\delta(D\|\bar D) \doteq \ln \sup_{E \subseteq X} \frac{\Pr_{x\sim D}[x\in E]-\delta}{\Pr_{x\sim \bar D}[x\in E]} .$$ Note that $\Div_\infty^0(D\|\bar D) = \Div_\infty(D\|\bar D)$. Similarly, we can define a radius of $\D$ in this divergence $$R_\infty^\delta(\D) \doteq \inf_{\bar D \in S^X} \sup_{D\in \D} \Div_\infty^\delta(D\|\bar D) .$$

Now, it is easy to see that, if $\Div_\infty^\delta(D\|\bar D) \leq r$ then $\Div_\infty^{k\delta}(D^k\|\bar D^k) \leq kr$. This means that if in the proof of Lemma \ref{lem:k-wise-flat-decision} we use the condition
$R_\infty^{\tau/(8k^2)}(\D) \leq \ln \gamma$ instead of $\gamma$-flatness then we will obtain that the event in Equation \eqref{eq:after_flat} holds with probability at least $$ \left( \frac{\tau}{4k} - (k-1) \cdot  \frac{\tau}{8k^2}\right) / \gamma^{k-1} \geq \frac{\tau}{\gamma^{k-1} \cdot 8k}$$ over the same random choices.

This implies the following generalization of Theorem \ref{thm:flat}.
\begin{theorem}\label{thm:flat-approx}
Let $\tau > 0$ and $k$ be any positive integer. Let $\calD$ be a class of distributions over a domain $X$ and $\gamma = \exp(R_\infty^{\tau/(8k^2)}(\D))$. There exists a randomized algorithm that given any $\delta > 0$ and a $k$-ary function $\phi: X^k \to [-1,1]$, estimates $D^k[\phi]$ within $\tau$  for every (unknown) $D \in \calD$ with success probability at least $1-\delta$ using $$\tilde{O}\bigg( \frac{\gamma^{k-1} \cdot k^3 \cdot \KLR(\calD)}{\tau^3} \cdot \log (1/\delta)\bigg)$$
queries to $\STAT_D^{(1)}(\tau/(6 \cdot k))$.\end{theorem}

An alternative approach is to use Renyi divergence of order $\alpha > 1$ defined as follows:
 $$\Div_\alpha(D\|\bar D) \doteq \frac{1}{1-\alpha} \cdot \ln \left (\E_{y \sim D} \left[\left(\frac{\Pr_{x\sim D}[x=y]}{\Pr_{x\sim \bar D}[x=y]}\right)^{\alpha-1}\right] \right).$$ The corresponding radius is defined as $$R_\alpha(\D) \doteq \inf_{\bar D \in S^X} \sup_{D\in \D} \Div_\alpha(D\|\bar D) .$$

To use it in our application we need the standard property of the Renyi divergence for product distributions $\Div_\alpha(D^k\|\bar D^k) = k \cdot \Div_\alpha(D\|\bar D)$ and also the following simple lemma from \cite[Lemma 1]{MansourMR09}:
\begin{lem}
For $\alpha > 1$, any two distributions $D,\bar D$ over $X$ and an event $E\subseteq X$:
$$\Pr_{x\sim D}[x\in E] \leq \left(\exp(\Div_\alpha(D\|\bar D) ) \cdot \Pr_{x\sim \bar D}[x\in E]\right)^{\frac{\alpha-1}{\alpha}}. $$
\end{lem}
We will need the inverted version of this lemma:
$$\Pr_{x\sim \bar D}[x\in E] \geq \frac{\left(\Pr_{x\sim D}[x\in E]\right)^{\frac{\alpha}{\alpha-1}}}{\exp(\Div_\alpha(D\|\bar D) )}. $$
Applying this in the proof of Lemma \ref{lem:k-wise-flat-decision} for $\gamma = \exp(R_\alpha(\D))$, we obtain that the event in Equation \eqref{eq:after_flat} holds with probability at least $$ \left(\frac{\tau}{4k}\right)^{\frac{\alpha}{\alpha-1}}/ \gamma^{k-1}. $$
This gives the following generalization of Theorem \ref{thm:flat}.
\begin{theorem}\label{thm:flat-approx}
Let $\tau > 0,\alpha >1$ and $k$ be any positive integer. Let $\calD$ be a class of distributions over a domain $X$ and $\gamma = \exp(R_\alpha(\D))$. There exists a randomized algorithm that given any $\delta > 0$ and a $k$-ary function $\phi: X^k \to [-1,1]$, estimates $D^k[\phi]$ within $\tau$  for every (unknown) $D \in \calD$ with success probability at least $1-\delta$ using $$\tilde{O}\bigg( \gamma^{k-1} \cdot \left( \frac{k}{\tau}\right)^{2+\frac{\alpha}{\alpha-1}} \cdot \log (1/\delta)\bigg)$$
queries to $\STAT_D^{(1)}(\tau/(6 \cdot k))$.\end{theorem}



\subsection{Applications to solving CSPs and learning DNF}
\label{sec:lower-bounds}
We now give some examples of the application of our reduction to obtain lower bounds against $k$-wise SQ algorithms. Our applications for stochastic constraint satisfaction problems (CSPs) and DNF learning. We start with the definition of a stochastic CSP with a {\em planted solution} which is a pseudo-random generator based on Goldreich's proposed one-way function \cite{goldreich2000candidate}.
\begin{definition}
Let $t \in \mathbb{N}$ and $P: \{\pm 1\}^t \to \{\pm 1\}$ be a fixed predicate. We are given access to samples from a distribution $P_{\sigma}$, corresponding to a (``planted'') assignment $\sigma \in \{\pm 1 \}^n$. A sample from this distribution is a uniform-random  $t$-tuple $(i_1, \dots, i_t)$ of distinct variable indices along with the value $P(\sigma_{i_1}, \dots, \sigma_{i_t})$. The goal is to recover the assignment $\sigma$ when given $m$ independent samples from $P_{\sigma}$. A (potentially) easier problem is to distinguish any such planted distribution from the distribution $U_t$ in which the value is an independent uniform-random coin flip (instead of $P(\sigma_{i_1}, \dots, \sigma_{i_t})$).
\end{definition}
We say that a predicate $P: \{\pm 1\}^t \to \{\pm 1\}$ has complexity $r$ if $r$ is the degree of the lowest-degree non-zero Fourier coefficient of $P$. It can be as large as $t$ (for the parity function).
A lower bound on the (unary) SQ complexity of solving such CSPs was shown by \cite{FeldmanPV:13} (their result is for the stronger $\VSTAT$ oracle but here we state the version for the $\STAT$ oracle).
\begin{theorem}[\cite{FeldmanPV:13}]\label{thm:FPV_csp}
Let $t, q \in \mathbb{N}$ and $P: \{\pm 1\}^t \to \{\pm 1\}$ be a fixed predicate of complexity $r$. Then for any $q >0$, any  algorithm that, given access to a distribution $D \in \{P_\sigma\ |\  \sigma  \in \{\pm 1 \}^n\} \cup \{U_t\}$  decides correctly whether $D = P_\sigma$ or $D=U_t$ with probability at least $2/3$ needs $q/2^{O(t)}$ queries to $\STAT^{(1)}_D\left(\left(\frac{\log q}{n}\right)^{r/2}\right)$.
\end{theorem}

The set of input distributions in this problem is $2$-flat relative to $U_t$ and it is one-to-many decision problem. Hence Theorem \ref{thm:flat-decision-reduction} implies\footnote{We can also get essentially the same result by applying the simulation of a $k$-wise SQ using unary SQs from Theorem \ref{thm:flat}.} the following lower bound for $k$-wise SQ algorithms.
\begin{theorem}\label{thm:csp-k-wise}
Let $t \in \mathbb{N}$ and $P: \{\pm 1\}^t \to \{\pm 1\}$ be a fixed predicate of complexity $r$. Then for any $\alpha >0$, any  algorithm that, given access to a distribution $D \in \{P_\sigma\ |\  \sigma  \in \{\pm 1 \}^n\} \cup \{U_t\}$  decides correctly whether $D = P_\sigma$ or $D=U_t$ with probability at least $2/3$ needs $2^{n^{1-\alpha} -O(t)}$ queries to $\STAT^{(n^{1-\alpha})}_D\left((2/n^{\alpha})^{r/2}  \cdot n^{1-\alpha}/4\right)$.
\end{theorem}
\begin{proof}
Let $\A$ be a $k$-wise SQ algorithm using $q'$ queries to $\STAT^{(n^{1-\alpha})}_D\left((2/n^{\alpha})^{r/2} \cdot n^{1-\alpha}/6\right)$ which solves the problem with success probability $2/3$.
We let $k=n^{1-\alpha}$ and apply Theorem \ref{thm:flat-decision-reduction} to obtain an algorithm that uses unary SQs and solves the problem with success probability $2/3$. This algorithm uses $q_0 = q' \cdot 2^{n^{1-\alpha}} \cdot n^{O(r)}$ queries to $\STAT^{(1)}_D\left((2/n^{\alpha})^{r/2}\right)$. Now choosing $q = 2^{2n^{1-\alpha}}$ we get that $\left(\frac{\log q}{n}\right)^{r/2} \leq (2/n^{\alpha})^{r/2}$. This means that $q_0 \geq q/2^{O(t)} = 2^{2n^{1-\alpha}- O(t)}$.
Hence $q' = 2^{2n^{1-\alpha}-O(t) - n^{1-\alpha}- O(r)} = 2^{n^{1-\alpha} -O(t)}$.
\end{proof}
Similar lower bounds can be obtained for other problems considered in \cite{FeldmanPV:13}, namely, planted satisfiability and $t$-SAT refutation.

To obtain a lower bound for learning DNF formulas we can use a simple reduction from the Goldreich's PRG defined above to learning DNF formulas of polynomial size. It is based on ideas implicit in the reduction from $t$-SAT refutation to DNF learning from \cite{DanielyS16}.
\begin{lemma}\label{lem:reduce-dnf}
$P: \{\pm 1\}^t \to \{\pm 1\}$ be a fixed predicate. There exists a mapping $M$ from $t$-tuples of indices in $[n]$ to $\zo^{tn}$ such that for every $\sigma \in \{\pm 1 \}^n$ there exists a DNF formula $f_\sigma$ of size $2^t$ satisfying $P(\sigma_{i_1}, \dots, \sigma_{i_t}) = f_\sigma(M(i_1, \dots, i_t))$.
\end{lemma}
\begin{proof}
The mapping $M$ maps $(i_1, \dots, i_t)$ to the concatenation of the indicator vectors of each of the indices. Namely, for $j \in [t]$ and $\ell \in [n]$, $M(i_1, \dots, i_t)_{j,\ell} = 1$ if and only if $i_j = \ell$, where we use the double index $j,\ell$ to refer to element $n (j-1) + \ell$ of the vector. Let $v_{j,\ell}$ denote the variable with the index $j,\ell$. Let $\sigma$ be any assignment and we denote by $z_j^\sigma$ the $j$-th variable of our predicate $P$ when the assignment is equal to $\sigma$. We first observe that $z_j^\sigma \equiv \bigwedge_{\ell \in [n], \sigma_\ell = 0} \bar{v}_{j,\ell}$. This is true since, by definition, the value of the $j$-th variable of our predicate is $\sigma_{i_j}$. This value is $1$ if and only if $i_j \not\in \{\ell \in [n]\ | \ \sigma_\ell = 0\}$. This is equivalent to $v_{j,\ell}$ being equal to $0$ for all $\ell \in [n]$ such that  $\sigma_\ell = 0$. Analogously, $\bar{z}^\sigma_j \equiv \bigwedge_{\ell \in [n], \sigma_\ell = 1} \bar{v}_{j,\ell}$. This implies that any conjunction of variables $z_1^\sigma,\bar{z}^\sigma_1,\ldots,z^\sigma_t,\bar{z}^\sigma_t$ can be expressed as a conjunction over variables $\bar{v}_{j,\ell}$. Any predicate $P$ can be expressed as a disjunction of at most $2^t$ conjunctions and hence there exists a DNF formula $f_\sigma$ of size at most $2^t$ whose value on $M(i_1, \dots, i_t)$ is equal to $P(\sigma_{i_1}, \dots, \sigma_{i_t})$
\end{proof}

This reduction implies that by converting a sample $((i_1, \dots, i_t),b)$ to a sample $(M(i_1, \dots, i_t),b)$ we can transform the Goldreich's PRG problem into a problem in which our goal is to distinguish examples of some DNF formula $f_\sigma$ from randomly labeled examples. Naturally, an algorithm that can learn DNF formulas can output a hypothesis which predicts the label (with some non-trivial accuracy), whereas such hypothesis cannot exist for predicting random labels. Hence known SQ lower bounds on planted CSPs \cite{FeldmanPV:13} immediately imply lower bounds for learning DNF. Further, by applying Lemma \ref{lem:reduce-dnf} together with Thm.~\ref{thm:csp-k-wise} for $t=r=\log n$ we obtain the first lower bounds for learning DNF against $n^{1-\alpha}$-wise SQ algorithms.
\begin{theorem}\label{thm:dnf-k-wise}
For any constant (independent of $n$) $\alpha >0$, there exists a constant $\beta>0$ such that
 any  algorithm that PAC learns DNF formulas of size $n$ with error $<1/2 - n^{- \beta \log n}$ and success probability at least $2/3$ needs at least $2^{n^{1-\alpha}}$ queries to $\STAT^{(n^{1-\alpha})}_D(n^{- \beta \log n})$.
\end{theorem}
We remark that this is a lower bound for PAC learning polynomial size DNF formulas with respect to some fixed (albeit non-uniform) distribution over $\zo^n$. The approach for relating $k$-wise SQ complexity to unary SQ complexity given in \cite{blum2003noise} applies to this setting. Yet, in their proof the tolerance needed for the unary SQ algorithm is $\tau/2^k$ and therefore it would not give a non-trivial lower bounds beyond $k=O(\log n)$.

%Our reduction produces a $\zo^{2tn}$
 

\section{Reduction for low-communication queries}\label{sec:sq_cc}
In this section, we prove  Theorem~\ref{thm:sq_and_cc} using a recent result of Steinhardt, Valiant and Wager \cite{SteinhardtVW16}.
Their result can be seen giving a SQ algorithm that simulates a communication protocol between $n$ parties. Each party is holding a sample drawn i.i.d.~from distribution $D$ and broadcasts at most $b$ bits about its sample (to all the other parties). The bits can be sent over multiple rounds. This is essentially the standard model of multi-party communication complexity (\eg \cite{KN97}) but with the goal of solving some problem about the unknown distribution $D$ rather than computing a specific function of the inputs. Alternatively, one can also see this model as a single algorithm that extracts at most $b$-bits of information about each random sample from $D$ and is allowed to extract the bits in an arbitrary order (generalizing the $b$-bit sampling model that we discuss in Section \ref{subsec:RFA} and in which $b$-bits are extracted from each sample at once). We refer to this model simply as algorithms that extract at most $b$ bits per sample.

\begin{theorem}[\cite{SteinhardtVW16}]\label{thm:SVW}
Let $\calA$ be an algorithm that uses $n$ samples drawn i.i.d.~from a distribution $D$ and extracts at most $b$ bits per sample. Then, for every $\beta > 0$, there is an algorithm $\calB$ that makes at most $2 \cdot b \cdot n$ queries to $\STAT^{(1)}_D(\beta/(2^{b+1} \cdot k))$ and the output distributions of $\calA$ and $\calB$ are within total variation distance $\beta$.
\end{theorem}

We will use this simulation to estimate the expectation of $k$-wise functions that have low communication complexity.
Specifically, we recall the following standard model of public-coin randomized $k$-party communication complexity.
\begin{defn}
For a function $\phi:X^k \to \{\pm 1\}$ we say that $\phi$ has a $k$-party public-coin randomized communication complexity of at most $b$ bits per party with success probability $1-\delta$ if there exist a protocol satisfying the following conditions. Each of the parties is given $x_i \in X$ and access to shared random bits. In each round one of the parties can compute one or more bits using its input, random bits and all the previous communication and then broadcast it to all the other parties. In the last round one of the parties computes a bit that is the output of the protocol. Each of the parties communicates at most $b$ bits in total. For every $x_1,\ldots,x_k\in X$, with probability at least $1-\delta$ over the choice of the random bits the output of the protocol is equal to $\phi(x_1,\ldots,x_k)$.
\end{defn}

We are now ready to prove Theorem~\ref{thm:sq_and_cc} which we restate here for convenience.
\begin{reptheorem}{thm:sq_and_cc}[restated]
Let $\phi:X^k \to \{\pm 1\}$ be a function, and assume that $\phi$ has $k$-party public-coin randomized communication complexity of $b$ bits per party with success probability $2/3$. Then, there exists a randomized algorithm that, with probability at least $1-\delta$, estimates $\Ex_{x \sim D^k}[\phi(x)]$ within $\tau$ using $O(b \cdot k \cdot \log(1/\delta)/\tau^2)$ queries to $\STAT^{(1)}_D(\tau')$ for some $\tau' = \tau^{O(b)}/k$.
\end{reptheorem}
\begin{proof}
We first amplify the success probability of the protocol for computing $\phi$ to $\delta'\doteq \tau/8$ using the majority vote of $O(\log(1/\delta'))$ repetitions. By Yao's minimax theorem \cite{Yao:1977} there exists a deterministic protocol $\Pi'$ that succeeds with probability at least $1-\delta'$ for $(x_1,\dots,x_k) \sim D^k$. Applying Theorem~\ref{thm:SVW}, we obtain a unary SQ algorithm $\A$ whose output is within total variation distance at most $\beta \doteq \tau/8$ from $\Pi'(x_1, \dots, x_k)$ (and we can assume that the output of $\A$ is in $\{\pm 1\}$). Therefore:
$$|\E[\A] - D^k[\phi]| \leq |\E[\A] - \E_{D^k} [\Pi'(x_1, \dots, x_k)]| + |\E_{D^k} [\Pi'(x_1, \dots, x_k)] - D^k [\phi]| \leq  \frac{2\tau}{8} + \frac{2\tau}{8} = \frac{\tau}{2}.$$
Repeating $\A$ $O(\log(1/\delta)/\tau^2)$ times and taking the mean, we get an estimate of $D^k[\phi]$ within $\tau$ with probability at least $1-\delta$. This algorithm uses $O(b \cdot k \cdot \log(1/\delta)/\tau^2)$ queries to $\STAT^{(1)}_D(\tau')$ for $\tau' = \frac{\tau}{8}/(2^{O(\log(8/\tau) \cdot b)} \cdot k) = \tau^{O(b)}/k$.
\end{proof}

The collision probability for a distribution $D$ is defined as $\Pr_{(x_1, x_2) \sim D^2}[x_1 = x_2]$. This corresponds to $\phi(x_1,x_2)$ being the Equality function which, as is well-known, has randomized $2$-party communication complexity of $O(1)$ bits per party with success probability $2/3$ (see, e.g., \cite{KN97}). Applying Theorem~\ref{thm:sq_and_cc} with $k=2$ we get the following corollary.
\begin{corollary}
For any $\tau, \delta >0$, there is a SQ algorithm that estimates the collision probability of an unknown distribution $D$ within $\tau$ with success probability $1-\delta$ using $O(\log(1/\delta)/\tau^2)$ queries to $\STAT^{(1)}_{D}(\tau^{O(1)})$.
\end{corollary} 

\section{Applications}\label{sec:applications}

Here we will see many problems \cite{peng2016approximate} that fall into the framework of $p$-extendible systems and thus, we directly have recovery results on their stable instances. Some of the problems might be hard, like Weighted Independent Set, whereas others may be easy (i.e. in $P$), having exact algorithms, however the greedy is extremely simple and fast compared to those. 



%\appendix

\section{Experimental details and more results}
\label{sec:app_exp}
We run all the experiments on Nvidia RTX 2080 Ti GPUs and V100 GPUs. Table~\ref{tab:app_testbed} summarizes the set of images used in each figure or table in the main paper.  

\captionsetup[table]{font=small}
\begin{table}[H]
    \small
    \centering
    \begin{tabular}{|p{2.5cm}|p{10cm}|}
    \toprule
         {\bf Figure/Table} & {\bf Comments}	\\
    \midrule
        Figure~\ref{fig:BN_var}a & We’ve tuned hyperparams for the attack (see Appendix~\ref{sec:app_hyperparam}) and carried out evaluations on the whole CIFAR-subset. The first sampled batch of size 16 from CIFAR-subset was used in Figure~\ref{fig:BN_var}a to demonstrate the quality of recovery for low-resolution images when BatchNorm statistics are not assumed to be known.  \\
        \midrule
        Figure~\ref{fig:BN_var}b & We’ve tuned hyperparams for the attack (see Appendix~\ref{sec:app_hyperparam}) and carried out evaluations on the whole ImageNet-subset. The best-reconstructed image in ImageNet-subset was used in Figure 1b to demonstrate the quality of recovery for high-resolution images when BatchNorm statistics are not assumed to be known.\\
        \midrule
        Figure~\ref{fig:batch_label_dist} & Percentages of class labels per batch were evaluated over the entire CIFAR10 dataset, for a random seed.	\\
        \midrule
        Figure~\ref{fig:reconstructed_labels} & The first sampled batch of size 16 was used in Figure~\ref{fig:reconstructed_labels} to demonstrate the quality of recovery when labels are not assumed to be known.	\\
        \midrule
        Table~\ref{tab:exp_summary} and Figure~\ref{fig:vis_recon} & We’ve tuned hyperparams for the attack and carried out evaluations on the whole CIFAR-subset. Table~\ref{tab:exp_summary} summarizes the performance of the attack on the whole CIFAR-subset and  Figure~\ref{fig:vis_recon} shows example images.\\
    \bottomrule
    \end{tabular}
    \caption{Summary of experimental testbed for each evaluation.}
    \label{tab:app_testbed}
\end{table}


\subsection{Hyper-parameters}
\label{sec:app_hyperparam}



\paragraph{Training.} For all experiments, we train ResNet-18 for 200 epochs, with a batch size of 128. We use SGD with momentum 0.9 as the optimizer. The initial learning rate is set to 0.1 by default, except for gradient pruning with $p=0.99$ and $p=0.999$. where we set the initial learning rate to 0.02. We decay the learning rate by a factor of 0.1 every 50 epochs.

\paragraph{The attack.}  We report the performance under different $\alpha_{\rm TV}$'s (Figure~\ref{fig:BN_tv_tune}) and $\alpha_{\rm BN}$'s (Figure~\ref{fig:BN_reg_tune}).

\begin{figure}[H]
\captionsetup[subfigure]{labelfont=scriptsize, textfont=tiny}
    \centering
    \subfloat[Original]{\includegraphics[width=0.12\textwidth]{imgs/appendix/TV/original.png}}
    \subfloat[$\alpha_{\rm TV}$=0]{\includegraphics[width=0.12\textwidth]{imgs/appendix/TV/tv_0.png}}
    \subfloat[$\alpha_{\rm TV}$=1e-3]{\includegraphics[width=0.12\textwidth]{imgs/appendix/TV/tv_1e-3.png}}
    \subfloat[$\alpha_{\rm TV}$=5e-3]{\includegraphics[width=0.12\textwidth]{imgs/appendix/TV/tv_5e-3.png}}
    \subfloat[$\alpha_{\rm TV}$=1e-2]{\includegraphics[width=0.12\textwidth]{imgs/appendix/TV/tv_1e-2.png}}
    \subfloat[$\alpha_{\rm TV}$=5e-2]{\includegraphics[width=0.12\textwidth]{imgs/appendix/TV/tv_5e-2.png}}
    \subfloat[$\alpha_{\rm TV}$=1e-1]{\includegraphics[width=0.12\textwidth]{imgs/appendix/TV/tv_1e-1.png}}
    \subfloat[$\alpha_{\rm TV}$=5e-1]{\includegraphics[width=0.12\textwidth]{imgs/appendix/TV/tv_5e-1.png}}
    
    \caption{Attacking a single CIFAR-10 images in $\rm BN_{exact}$ setting, with different coefficients for the total variation regularizer ($\alpha_{\rm TV}$'s). $\alpha_{\rm TV}$=1e-2 gives the best reconstruction.}
    \label{fig:BN_tv_tune}
\end{figure}


\begin{figure}[H]
\vspace{-5mm}
\captionsetup[subfigure]{labelfont=scriptsize, textfont=tiny}
    \centering
    \subfloat[Original]{\includegraphics[width=0.16\textwidth]{imgs/assumptions/BN/original.png}}
    \subfloat[$\alpha_{\rm BN}$=0]{\includegraphics[width=0.16\textwidth]{imgs/assumptions/BN/reconstructed_train_train_bn=0.png}}
    \subfloat[$\alpha_{\rm BN}$=5e-4]{\includegraphics[width=0.16\textwidth]{imgs/assumptions/BN/reconstructed_train_train_bn=5e-4.png}}
    \subfloat[$\alpha_{\rm BN}$=1e-3]{\includegraphics[width=0.16\textwidth]{imgs/assumptions/BN/reconstructed_train_train_bn=1e-3.png}}
    \subfloat[$\alpha_{\rm BN}$=5e-3]{\includegraphics[width=0.16\textwidth]{imgs/assumptions/BN/reconstructed_train_train_bn=5e-3.png}}
    \subfloat[$\alpha_{\rm BN}$=1e-2]{\includegraphics[width=0.16\textwidth ]{imgs/assumptions/BN/reconstructed_train_train_bn=1e-2.png}}
    \caption{Attacking a batch of 16 CIFAR-10 images in $\rm BN_{infer}$ setting, with different coefficients for the BatchNorm regularizer ($\alpha_{\rm BN}$'s). $\alpha_{\rm TV}$=1e-3 gives the best reconstruction.}
    \label{fig:BN_reg_tune}
\end{figure}


\subsection{Details and more results for Section~\ref{sec:assumption}}

\paragraph{Attacking a single ImageNet image.} We launched the attack on ImageNet using the objective function in Eq.~\ref{eq:objective}, where $\alpha_{\rm TV}=0.1$, $\alpha_{\rm BN}=0.001$. We run the attack for 24,000 iterations using Adam optimizer, with initial learning rate 0.1, and decay the learning rate by a factor of $0.1$ at 
$3/8,5/8,7/8$ of training. We rerun the attack 5 times and present the best results measured by LPIPS in Figure~\ref{fig:BN_var}.

\paragraph{Qualitative and quantitative results for a more realistic attack.} We also present results of a more realistic attack in Table~\ref{tab:exp_summary_realistic} and Figure~\ref{fig:vis_recon_realistic}, where the attacker does {\em not} know BatchNorm statistics but knows the private labels. We assume the private labels to be known in this evaluation, because for those batches whose distribution of labels is uniform, the restoration of labels should still be quite accurate~\citep{yin2021see}.
As shown, in the evaluated setting, the attack is no longer effective when the batch size is 32 and Intra-InstaHide with $k=4$ is applied. The accuracy loss to stop the realistic attack is only around $3\%$ (compared to around $7\%$ to stop the strongest attack) .


\begin{figure}[H]
\captionsetup[subfigure]{font=small}
  \centering
  \subfloat{\includegraphics[width=\textwidth]{imgs/Compare_16_32.png}}
  \caption{Reconstruction results under different defenses for a more realistic setting (when the attacker knows private labels but does not know BatchNorm statistics). We also present the results for the strongest attack from Figure~\ref{fig:vis_recon} for comparison. Using Intra-InstaHide with $k=4$ and batch size 32 seems to stop the realistic attack.}
  \label{fig:vis_recon_realistic}
\end{figure}

\captionsetup[table]{font=small}
\begin{table}[H] 
  \scriptsize
  \setlength{\tabcolsep}{2.6pt}
  \renewcommand{\arraystretch}{0.95}
  \begin{tabular}{|l|c|c|c|c|c|c|c|c|c|c|c|c|c|c|c|c|}
  \toprule
   &  \multirow{2}{*}{\bf None} & \multicolumn{6}{c|}{\multirow{2}{*}{\bf GradPrune ($p$)}} & \multicolumn{2}{c|}{\multirow{2}{*}{\bf MixUp ($k$)}} & \multicolumn{2}{c|}{\multirow{2}{*}{\bf Intra-InstaHide ($k$)}} & \multicolumn{2}{c|}{\bf GradPrune ($p=0.9$)}\\
   & & \multicolumn{6}{c|}{} & \multicolumn{2}{c|}{} & \multicolumn{2}{c|}{} & {\bf  + MixUp } & {\bf  + Intra-InstaHide}\\
  \midrule
   {\bf Parameter}  & - & 0.5 & 0.7 & 0.9 & 0.95 & 0.99 & 0.999 & 4 & 6 & 4 & 6 & $k=4$ & $k=4$ \\
   \midrule
   {\bf Test Acc.} & 93.37 & 93.19 & 93.01 & 90.57 & 89.92 & 88.61 & 83.58 &  92.31 & 90.41 & 90.04 & 88.20 & 91.37 & 86.10 \\
   \midrule
  {\bf Time (train)} & $1\times$ & \multicolumn{6}{c|}{$1.04\times$} & \multicolumn{2}{c|}{$1.06\times$} & \multicolumn{2}{c|}{$1.06\times$} & \multicolumn{2}{c|}{$1.10\times$} \\
  \midrule
  \multicolumn{14}{|c|}{\bf Attack batch size $= 16$, the strongest attack} \\
  \midrule
  {\bf Avg. LPIPS $\downarrow$}  & 0.41  & 0.41  & 0.42  & 0.46  & 0.48  & 0.50  & 0.55         & 0.50  & 0.49  & 0.69  & 0.69  & 0.62  & \best{0.73}\\
  {\bf Best LPIPS $\downarrow$}  & 0.21  & 0.22  & 0.27  & 0.29  & 0.30  & 0.29  & 0.48         & 0.31  & 0.28  & 0.56  & 0.56  & 0.37  & \best{0.65}\\
  {(LPIPS std.)}                 & 0.09  & 0.08  & 0.07  & 0.06  & 0.06  & 0.06  & 0.04         & 0.10  & 0.10  & 0.06  & 0.07  & 0.10  & 0.05\\
  \midrule
   \multicolumn{14}{|c|}{\bf Attack batch size $= 16$, attacker knows private labels but does not know BatchNorm statistics} \\
   \midrule
   {\bf Avg. LPIPS $\downarrow$}  & 0.49 & 0.51 & 0.48 & 0.51 & 0.52 & 0.56 & 0.60 & 0.71 & 0.71 & \best{0.75} & \best{0.75} & 0.74 &  0.74\\
   {\bf Best LPIPS $\downarrow$}  & 0.30 & 0.33 & 0.31 & 0.33 & 0.34 & 0.39 & 0.44 & 0.48 & 0.53 & \best{0.65} & 0.63 & 0.61 &  0.63\\
   {(LPIPS std.)}                 & 0.08 & 0.09 & 0.08 & 0.08 & 0.07 & 0.07 & 0.05 & 0.08 & 0.07 & 0.04 & 0.05 & 0.08 &  0.05\\
   \midrule
   \multicolumn{14}{|c|}{\bf Attack batch size $= 32$, the strongest attack} \\
  \midrule
  {\bf Avg. LPIPS $\downarrow$}  & 0.45  & 0.46  & 0.48  & 0.52  & 0.54  & 0.58  & 0.63         & 0.50  & 0.49  & 0.69  & 0.69  & 0.62  & \best{0.73}\\
   {\bf Best LPIPS $\downarrow$}  & 0.18  & 0.18  & 0.22  & 0.31  & 0.43  & 0.48  & 0.54         & 0.31  & 0.28  & 0.56  & 0.56  & 0.37  & \best{0.65}\\
   {(LPIPS std.)}                 & 0.11  & 0.11  & 0.09  & 0.07  & 0.05  & 0.04  & 0.04         & 0.10  & 0.10  & 0.06  & 0.07  & 0.10  & 0.05\\
    \midrule
   \multicolumn{14}{|c|}{\bf Attack batch size $= 32$, attacker knows private labels but does not know BatchNorm statistics} \\
   \midrule
   {\bf Avg. LPIPS $\downarrow$}  & 0.48 & 0.50 & 0.53 & 0.53 & 0.55 & 0.60 & 0.63 & 0.73 & 0.72 & 0.76 & 0.76 & 0.76 & \best{0.77} \\
   {\bf Best LPIPS $\downarrow$}  & 0.29 & 0.32 & 0.32 & 0.31 & 0.40 & 0.41 & 0.55 & 0.63 & 0.60 & \best{0.68} & 0.63 & 0.66 & 0.65\\
   {(LPIPS std.)}                 & 0.08 & 0.07 & 0.07 & 0.08 & 0.08 & 0.06 & 0.04 & 0.06 & 0.06 & 0.04 & 0.05 & 0.06 & 0.05\\
  \bottomrule
  \end{tabular}
  \vspace{2mm}
%   \subfloat{\includegraphics[width=0.98\textwidth]{imgs/Compare_16_32.png}}
  \caption{\small Utility-security trade-off of different defenses for a more realistic setting (when the attacker knows private labels but does not know BatchNorm statistics). We also present the results for the strongest attack from Table~\ref{tab:exp_summary} for comparison. We evaluate the attack on 50 CIFAR-10 images and report the LPIPS score ($\downarrow$: lower values suggest more privacy leakage).
  We mark the least-leakage defense measured by the metric in \best{green}.} 
  \label{tab:exp_summary_realistic}
\end{table}

\iffalse
\paragraph{Qualitative and quantitative results for private labels unknown.} Apart from the example in Figure~\ref{fig:reconstructed_labels} with batch size being 16, we provide another example for how unknown labels affect reconstruction quality in Figure~\ref{fig:assumption2_app}, with batch size being 32. We also provide quantitative measurements in Figure~\ref{tab:assumption2_app1} and~\ref{tab:assumption2_app2}.

\begin{figure}[H]
    \centering
    \subfloat[Reconstructions with and without private labels]{
    \includegraphics[width=0.95\textwidth]{imgs/assumptions/label_known_unknown_32.png}
    \label{fig:assumption2_app}
    }\\
    \subfloat[Batch size = 16]{
        \setlength{\tabcolsep}{4pt}
        \small
        \begin{tabular}[b]{|c|c|c|}
                \toprule
                  & {\bf Labels known} &  {\bf Labela unknown} \\
                \midrule
                %  {\bf Avg. PSNR $\uparrow$} & 12.45 & 12.01  \\
                %  {\bf Best PSNR $\uparrow$} & 17.42 & 14.85    \\
                 {\bf Avg. LPIPS $\downarrow$} & 0.44 & 0.58 \\
                 {\bf Best LPIPS $\downarrow$} & 0.25 & 0.32    \\
                \bottomrule
            \end{tabular}
        \label{tab:assumption2_app1}
        }
        \subfloat[Batch size = 32]{
        \setlength{\tabcolsep}{4pt}
        \small
        \begin{tabular}[b]{|c|c|c|}
                \toprule
                  & {\bf Labels known} &  {\bf Labels unknown} \\
                \midrule
                %  {\bf Avg. PSNR $\uparrow$} & 13.01 & 12.16 \\
                %  {\bf Best PSNR $\uparrow$} & 17.09 & 14.62    \\
                 {\bf Avg. LPIPS $\downarrow$} & 0.41 & 0.62 \\
                 {\bf Best LPIPS $\downarrow$} & 0.21 & 0.39    \\
                \bottomrule
            \end{tabular}
        \label{tab:assumption2_app2}
        }
    \caption{A reconstructed batch of 32 images with and without private labels known (a). We also provide quantitative measurements of reconstructions with batch size 16 (b) and 32 (c) ($\downarrow$: lower values suggest more leakage). The gradient inversion attack is weakened when private labels are not available.}
\end{figure}
\fi


% \iffalse
\subsection{More results for the strongest attack}

\paragraph{Full version of Figure~\ref{fig:vis_recon}.} Figure~\ref{fig:vis_recon_full} provides more examples for reconstructed images by the strongest attack under different defenses and batch sizes. 



\begin{figure}[H]
\captionsetup[subfigure]{font=small}
  \centering
  \vspace{-12mm}
  \subfloat[Batch size $=1$]{\includegraphics[width=\linewidth]{imgs/recon_vis_bs=1_BN_exact_small.png}}\\
  \vspace{-3mm}
  \subfloat[Batch size $=16$]{\includegraphics[width=\linewidth]{imgs/recon_vis_bs=16_BN_exact_small.png}}\\
  \vspace{-3mm}
  \subfloat[Batch size $=32$]{\includegraphics[width=\linewidth]{imgs/recon_vis_bs=32_BN_exact_small.png}}\\
  \vspace{-2mm}
  \caption{Reconstruction results under different defenses with batch size 1 (a), 16 (b) and 32 (c). Full version of Figure~\ref{fig:vis_recon}.}
  \label{fig:vis_recon_full}
  \vspace{-2mm}
\end{figure}


\paragraph{Results with MNIST dataset.} We’ve repeated our main evaluation of defenses and attacks (Table~\ref{tab:exp_summary}) on MNIST dataset~\citep{deng2012mnist} with a simple 6-layer ConvNet model. Note that the simple ConvNet does not contain BatchNorm layers. We evaluate the following defenses on the MNIST dataset with a 6-layer ConvNet architecture against the strongest attack (private labels known):

\begin{itemize}
    \item GradPrune (gradient pruning): gradient pruning sets gradients of small magnitudes to zero. We vary the pruning ratio $p$ in \{0.5, 0.7, 0.9, 0.95, 0.99, 0.999, 0.9999\}.
    \item MixUp: we vary $k$ in \{4,6\}, and set the upper bound of a single coefficient to 0.65 (coefficients sum to 1).
    \item Intra-InstaHide: we vary $k$ in \{4,6\}, and set the upper bound of a single coefficient to 0.65 (coefficients sum to 1). 
    \item A combination of GradPrune and MixUp/Intra-InstaHide.
\end{itemize}

We run the evaluation against the strongest attack and batch size 1 to estimate the upper bound of privacy leakage. Specifically, we assume the attacker knows private labels, as well as the indices of mixed images and mixing coefficients for MixUp and Intra-InstaHide. 

\begin{figure}[t]
    \centering
    \includegraphics[width=0.95\linewidth]{imgs/appendix/recon_vis_MNIST.png}
    \caption{Reconstruction results of MNIST digits under different defenses with the strongest atttack and batch size 1.}
    \label{fig:vis_recon_MNIST}
    \vspace{-5mm}
\end{figure}

For MNIST with a simple 6-layer ConvNet, defending the strongest attack with gradient pruning may require the pruning ratio $p\geq 0.9999$. MixUp with $k=4$ or $k=6$ are not sufficient to defend the gradient inversion attack. Combining MixUp ($k=4$) with gradient pruning ($p=0.99$) improves the defense, however, the reconstructed digits are still highly recognizable. Intra-InstaHide alone ($k=4$ or $k=6$) gives a bit better defending performance than MixUp and GradPrune. Combining InstaHide ($k=4$) with gradient pruning ($p=0.99$) further improves the defense and makes the reconstruction almost unrecognizable. 




\subsection{More results for encoding-based defenses}
We visualize the whole reconstructed dataset under MixUp and Intra-InstaHide defenses with different batch sizes in Figure~\ref{fig:encode_bs1}, \ref{fig:encode_bs16} and \ref{fig:encode_bs32}.  Sample results of the original and the reconstructed batches are provided in Figure~\ref{fig:mixup_vs_instahide}.

\begin{figure}[H]
    \centering
    \includegraphics[width=0.95\textwidth]{imgs/appendix/mixup_vs_instahide.png}
    \caption{Original and reconstructed batches of 16 images under MixUp and Intra-InstaHide defenses. We visualize both the original and the absolute images for the Intra-InstaHide defense. Intra-InstaHide makes pixel-wise matching harder.}
    \label{fig:mixup_vs_instahide}
    \vspace{-5mm}
\end{figure}

\begin{figure}[H]
\captionsetup[subfigure]{labelfont=scriptsize, textfont=tiny}
    \centering
    \subfloat[Original]{\includegraphics[width=0.23\textwidth]{imgs/decode_res/InstaHide/bs1_k4/originals.png}} \hspace{1mm}
    \subfloat[MixUp, $k$=4]{\includegraphics[width=0.23\textwidth]{imgs/decode_res/Mixup/bs1_k4/grad_decode.png}} \hspace{1mm}
    \subfloat[MixUp, $k$=6]{\includegraphics[width=0.23\textwidth]{imgs/decode_res/Mixup/bs1_k6/grad_decode.png}} \hspace{1mm}
    \subfloat[MixUp+GradPrune, $k$=4, $p$=0.9]{\includegraphics[width=0.23\textwidth]{imgs/decode_res/Mixup/bs1_k4_gradprune/grad_decode.png}}
    
    \subfloat[Original]{\includegraphics[width=0.23\textwidth]{imgs/decode_res/InstaHide/bs1_k4/originals.png}} \hspace{1mm}
    \subfloat[InstaHide, $k$=4]{\includegraphics[width=0.23\textwidth]{imgs/decode_res/InstaHide/bs1_k4/grad_decode.png}} \hspace{1mm}
    \subfloat[InstaHide, $k$=6]{\includegraphics[width=0.23\textwidth]{imgs/decode_res/InstaHide/bs1_k6/grad_decode.png}} \hspace{1mm}
    \subfloat[InstaHide+GradPrune, $k$=4, $p$=0.9]{\includegraphics[width=0.23\textwidth]{imgs/decode_res/InstaHide/bs1_k4_gradprune/grad_decode.png}}
    \caption{Reconstrcuted dataset under MixUp and Intra-InstaHide against the strongest attack (batch size is 1).}
    \label{fig:encode_bs1}
    \vspace{-10mm}
\end{figure}


\begin{figure}[H]
\captionsetup[subfigure]{labelfont=scriptsize, textfont=tiny}
    \centering
    \subfloat[Original]{\includegraphics[width=0.23\textwidth]{imgs/decode_res/InstaHide/bs1_k4/originals.png}} \hspace{1mm}
    \subfloat[MixUp, $k$=4]{\includegraphics[width=0.23\textwidth]{imgs/decode_res/Mixup/bs16_k4/grad_decode.png}} \hspace{1mm}
    \subfloat[MixUp, $k$=6]{\includegraphics[width=0.23\textwidth]{imgs/decode_res/Mixup/bs16_k6/grad_decode.png}} \hspace{1mm}
    \subfloat[MixUp+GradPrune, $k$=4, p=0.9]{\includegraphics[width=0.23\textwidth]{imgs/decode_res/Mixup/bs16_k4_gradprune/grad_decode.png}}

    
    \subfloat[Original]{\includegraphics[width=0.23\textwidth]{imgs/decode_res/InstaHide/bs1_k4/originals.png}} \hspace{1mm}
    \subfloat[InstaHide, $k$=4]{\includegraphics[width=0.23\textwidth]{imgs/decode_res/InstaHide/bs16_k4/grad_decode.png}} \hspace{1mm}
    \subfloat[InstaHide, $k$=6]{\includegraphics[width=0.23\textwidth]{imgs/decode_res/InstaHide/bs16_k6/grad_decode.png}} \hspace{1mm}
    \subfloat[InstaHide+GradPrune, $k$=4, $p$=0.9]{\includegraphics[width=0.23\textwidth]{imgs/decode_res/InstaHide/bs16_k4_gradprune/grad_decode.png}}
    \caption{Reconstrcuted dataset under MixUp and Intra-InstaHide against the strongest attack (batch size is 16).}
    \label{fig:encode_bs16}
\end{figure}



\begin{figure}[H]
\captionsetup[subfigure]{labelfont=scriptsize, textfont=tiny}
    \centering
    \subfloat[Original]{\includegraphics[width=0.23\textwidth]{imgs/decode_res/InstaHide/bs1_k4/originals.png}} \hspace{1mm}
    \subfloat[MixUp, $k$=4]{\includegraphics[width=0.23\textwidth]{imgs/decode_res/Mixup/bs32_k4/grad_decode.png}} \hspace{1mm}
    \subfloat[MixUp, $k$=6]{\includegraphics[width=0.23\textwidth]{imgs/decode_res/Mixup/bs32_k6/grad_decode.png}} \hspace{1mm}
    \subfloat[MixUp+GradPrune, $k$=4, $p$=0.9]{\includegraphics[width=0.23\textwidth]{imgs/decode_res/Mixup/bs32_k4_gradprune/grad_decode.png}}

    
    \subfloat[Original]{\includegraphics[width=0.23\textwidth]{imgs/decode_res/InstaHide/bs1_k4/originals.png}} \hspace{1mm}
    \subfloat[InstaHide, $k$=4]{\includegraphics[width=0.23\textwidth]{imgs/decode_res/InstaHide/bs32_k4/grad_decode.png}} \hspace{1mm}
    \subfloat[InstaHide, $k$=6]{\includegraphics[width=0.23\textwidth]{imgs/decode_res/InstaHide/bs32_k6/grad_decode.png}} \hspace{1mm}
    \subfloat[InstaHide+GradPrune, $k$=4, $p$=0.9]{\includegraphics[width=0.23\textwidth]{imgs/decode_res/InstaHide/bs32_k4_gradprune/grad_decode.png}}
    \caption{Reconstrcuted dataset under MixUp and Intra-InstaHide against the strongest attack (batch size is 32).}
    \label{fig:encode_bs32}
\end{figure}






We briefly recall the framework of statistical inference via empirical risk minimization.
Let $(\bbZ, \calZ)$ be a measurable space.
Let $Z \in \bbZ$ be a random element following some unknown distribution $\Prob$.
Consider a parametric family of distributions $\calP_\Theta := \{P_\theta: \theta \in \Theta \subset \reals^d\}$ which may or may not contain $\Prob$.
We are interested in finding the parameter $\theta_\star$ so that the model $P_{\theta_\star}$ best approximates the underlying distribution $\Prob$.
For this purpose, we choose a \emph{loss function} $\score$ and minimize the \emph{population risk} $\risk(\theta) := \Expect_{Z \sim \Prob}[\score(\theta; Z)]$.
Throughout this paper, we assume that
\begin{align*}
     \theta_\star = \argmin_{\theta \in \Theta} L(\theta)
\end{align*}
uniquely exists and satisfies $\theta_\star \in \text{int}(\Theta)$, $\nabla_\theta L(\theta_\star) = 0$, and $\nabla_\theta^2 L(\theta_\star) \succ 0$.

\myparagraph{Consistent loss function}
We focus on loss functions that are consistent in the following sense.

\begin{customasmp}{0}\label{asmp:proper_loss}
    When the model is \emph{well-specified}, i.e., there exists $\theta_0 \in \Theta$ such that $\Prob = P_{\theta_0}$, it holds that $\theta_0 = \theta_\star$.
    We say such a loss function is \emph{consistent}.
\end{customasmp}

In the statistics literature, such loss functions are known as proper scoring rules \citep{dawid2016scoring}.
We give below two popular choices of consistent loss functions.

\begin{example}[Maximum likelihood estimation]
    A widely used loss function in statistical machine learning is the negative log-likelihood $\score(\theta; z) := -\log{p_\theta(z)}$ where $p_\theta$ is the probability mass/density function for the discrete/continuous case.
    When $\Prob = P_{\theta_0}$ for some $\theta_0 \in \Theta$,
    we have $L(\theta) = \Expect[-\log{p_\theta(Z)}] = \kl(p_{\theta_0} \Vert p_\theta) - \Expect[\log{p_{\theta_0}(Z)}]$ where $\kl$ is the Kullback-Leibler divergence.
    As a result, $\theta_0 \in \argmin_{\theta \in \Theta} \kl(p_{\theta_0} \Vert p_\theta) = \argmin_{\theta \in \Theta} L(\theta)$.
    Moreover, if there is no $\theta$ such that $p_\theta \txtover{a.s.}{=} p_{\theta_0}$, then $\theta_0$ is the unique minimizer of $L$.
    We give in \Cref{tab:glms} a few examples from the class of generalized linear models (GLMs) proposed by \citet{nelder1972generalized}.
\end{example}

\begin{example}[Score matching estimation]
    Another important example appears in \emph{score matching} \citep{hyvarinen2005estimation}.
    Let $\bbZ = \reals^\tau$.
    Assume that $\Prob$ and $P_\theta$ have densities $p$ and $p_\theta$ w.r.t the Lebesgue measure, respectively.
    Let $p_\theta(z) = q_\theta(z) / \Lambda(\theta)$ where $\Lambda(\theta)$ is an unknown normalizing constant. We can choose the loss
    \begin{align*}
        \score(\theta; z) := \Delta_z \log{q_\theta(z)} + \frac12 \norm{\nabla_z \log{q_\theta(z)}}^2 + \text{const}.
    \end{align*}
    Here $\Delta_z := \sum_{k=1}^p \partial^2/\partial z_k^2$ is the Laplace operator.
    Since \cite[Thm.~1]{hyvarinen2005estimation}
    \begin{align*}
        L(\theta) = \frac12 \Expect\left[ \norm{\nabla_z q_\theta(z) - \nabla_z p(z)}^2 \right],
    \end{align*}
    we have, when $p = p_{\theta_0}$, that $\theta_0 \in \argmin_{\theta \in \Theta} L(\theta)$.
    In fact, when $q_\theta > 0$ and there is no $\theta$ such that $p_\theta \txtover{a.s.}{=} p_{\theta_0}$, the true parameter $\theta_0$ is the unique minimizer of $L$ \cite[Thm.~2]{hyvarinen2005estimation}.
\end{example}

\myparagraph{Empirical risk minimization}
Assume now that we have an i.i.d.~sample $\{Z_i\}_{i=1}^n$ from $\Prob$.
To learn the parameter $\theta_\star$ from the data, we minimize the empirical risk to obtain the \emph{empirical risk minimizer}
\begin{align*}
    \theta_n \in \argmin_{\theta \in \Theta} \left[ L_n(\theta) := \frac1n \sum_{i=1}^n \score(\theta; Z_i) \right].
\end{align*}
This applies to both maximum likelihood estimation and score matching estimation. 
In \Cref{sec:main_results}, we will prove that, with high probability, the estimator $\theta_n$ exists and is unique under a generalized self-concordance assumption.

\begin{figure}
    \centering
    \includegraphics[width=0.45\textwidth]{graphs/logistic-dikin} %0.4
    \caption{Dikin ellipsoid and Euclidean ball.}
    \label{fig:logistic_dikin}
\end{figure}

\myparagraph{Confidence set}
In statistical inference, it is of great interest to quantify the uncertainty in the estimator $\theta_n$.
In classical asymptotic theory, this is achieved by constructing an asymptotic confidence set.
We review here two commonly used ones, assuming the model is well-specified.
We start with the \emph{Wald confidence set}.
It holds that $n(\theta_n - \theta_\star)^\top H_n(\theta_n) (\theta_n - \theta_\star) \rightarrow_d \chi_d^2$, where $H_n(\theta) := \nabla^2 L_n(\theta)$.
Hence, one may consider a confidence set $\{\theta: n(\theta_n - \theta)^\top H_n(\theta_n) (\theta_n - \theta) \le q_{\chi_d^2}(\delta) \}$ where $q_{\chi_d^2}(\delta)$ is the upper $\delta$-quantile of $\chi_d^2$.
The other is the \emph{likelihood-ratio (LR) confidence set} constructed from the limit $2n [L_n(\theta_\star) - L_n(\theta_n)] \rightarrow_d \chi_d^2$, which is known as the Wilks' theorem \citep{wilks1938large}.
These confidence sets enjoy two merits: 1) their shapes are an ellipsoid (known as the \emph{Dikin ellipsoid}) which is adapted to the optimization landscape induced by the population risk; 2) they are asymptotically valid, i.e., their coverages are exactly $1 - \delta$ as $n \rightarrow \infty$.
However, due to their asymptotic nature, it is unclear how large $n$ should be in order for it to be valid.

Non-asymptotic theory usually focuses on developing finite-sample bounds for the \emph{excess risk}, i.e., $\Prob(L(\theta_n) - L(\theta_\star) \le C_n(\delta)) \ge 1 - \delta$.
To obtain a confidence set, one may assume that the population risk is twice continuously differentiable and $\lambda$-strongly convex.
Consequently, we have $\lambda \norm{\theta_n - \theta_\star}_2^2 / 2 \le L(\theta_n) - L(\theta_\star)$ and thus we can consider the confidence set $\calC_{\text{finite}, n}(\delta) := \{\theta: \norm{\theta_n - \theta}_2^2 \le 2C_n(\delta)/\lambda\}$.
Since it originates from a finite-sample bound, it is valid for fixed $n$, i.e., $\Prob(\theta_\star \in \calC_{\text{finite}, n}(\delta)) \ge 1 - \delta$ for all $n$; however, it is usually conservative, meaning that the coverage is strictly larger than $1 - \delta$.
Another drawback is that its shape is a Euclidean ball which remains the same no matter which loss function is chosen.
We illustrate this phenomenon in \Cref{fig:logistic_dikin}.
Note that a similar observation has also been made in the bandit literature \citep{faury2020improved}.

We are interested in developing finite-sample confidence sets.
However, instead of using excess risk bounds and strong convexity, we construct in \Cref{sec:main_results} the Wald and LR confidence sets in a non-asymptotic fashion, under a generalized self-concordance condition.
These confidence sets have the same shape as their asymptotic counterparts while maintaining validity for fixed $n$.
These new results are achieved by characterizing the critical sample size enough to enter the asymptotic regime.


%\newpage
\bibliographystyle{alpha}
\bibliography{refs,vf-allrefs}
\chapter{Supplementary Material}
\label{appendix}

In this appendix, we present supplementary material for the techniques and
experiments presented in the main text.

\section{Baseline Results and Analysis for Informed Sampler}
\label{appendix:chap3}

Here, we give an in-depth
performance analysis of the various samplers and the effect of their
hyperparameters. We choose hyperparameters with the lowest PSRF value
after $10k$ iterations, for each sampler individually. If the
differences between PSRF are not significantly different among
multiple values, we choose the one that has the highest acceptance
rate.

\subsection{Experiment: Estimating Camera Extrinsics}
\label{appendix:chap3:room}

\subsubsection{Parameter Selection}
\paragraph{Metropolis Hastings (\MH)}

Figure~\ref{fig:exp1_MH} shows the median acceptance rates and PSRF
values corresponding to various proposal standard deviations of plain
\MH~sampling. Mixing gets better and the acceptance rate gets worse as
the standard deviation increases. The value $0.3$ is selected standard
deviation for this sampler.

\paragraph{Metropolis Hastings Within Gibbs (\MHWG)}

As mentioned in Section~\ref{sec:room}, the \MHWG~sampler with one-dimensional
updates did not converge for any value of proposal standard deviation.
This problem has high correlation of the camera parameters and is of
multi-modal nature, which this sampler has problems with.

\paragraph{Parallel Tempering (\PT)}

For \PT~sampling, we took the best performing \MH~sampler and used
different temperature chains to improve the mixing of the
sampler. Figure~\ref{fig:exp1_PT} shows the results corresponding to
different combination of temperature levels. The sampler with
temperature levels of $[1,3,27]$ performed best in terms of both
mixing and acceptance rate.

\paragraph{Effect of Mixture Coefficient in Informed Sampling (\MIXLMH)}

Figure~\ref{fig:exp1_alpha} shows the effect of mixture
coefficient ($\alpha$) on the informed sampling
\MIXLMH. Since there is no significant different in PSRF values for
$0 \le \alpha \le 0.7$, we chose $0.7$ due to its high acceptance
rate.


% \end{multicols}

\begin{figure}[h]
\centering
  \subfigure[MH]{%
    \includegraphics[width=.48\textwidth]{figures/supplementary/camPose_MH.pdf} \label{fig:exp1_MH}
  }
  \subfigure[PT]{%
    \includegraphics[width=.48\textwidth]{figures/supplementary/camPose_PT.pdf} \label{fig:exp1_PT}
  }
\\
  \subfigure[INF-MH]{%
    \includegraphics[width=.48\textwidth]{figures/supplementary/camPose_alpha.pdf} \label{fig:exp1_alpha}
  }
  \mycaption{Results of the `Estimating Camera Extrinsics' experiment}{PRSFs and Acceptance rates corresponding to (a) various standard deviations of \MH, (b) various temperature level combinations of \PT sampling and (c) various mixture coefficients of \MIXLMH sampling.}
\end{figure}



\begin{figure}[!t]
\centering
  \subfigure[\MH]{%
    \includegraphics[width=.48\textwidth]{figures/supplementary/occlusionExp_MH.pdf} \label{fig:exp2_MH}
  }
  \subfigure[\BMHWG]{%
    \includegraphics[width=.48\textwidth]{figures/supplementary/occlusionExp_BMHWG.pdf} \label{fig:exp2_BMHWG}
  }
\\
  \subfigure[\MHWG]{%
    \includegraphics[width=.48\textwidth]{figures/supplementary/occlusionExp_MHWG.pdf} \label{fig:exp2_MHWG}
  }
  \subfigure[\PT]{%
    \includegraphics[width=.48\textwidth]{figures/supplementary/occlusionExp_PT.pdf} \label{fig:exp2_PT}
  }
\\
  \subfigure[\INFBMHWG]{%
    \includegraphics[width=.5\textwidth]{figures/supplementary/occlusionExp_alpha.pdf} \label{fig:exp2_alpha}
  }
  \mycaption{Results of the `Occluding Tiles' experiment}{PRSF and
    Acceptance rates corresponding to various standard deviations of
    (a) \MH, (b) \BMHWG, (c) \MHWG, (d) various temperature level
    combinations of \PT~sampling and; (e) various mixture coefficients
    of our informed \INFBMHWG sampling.}
\end{figure}

%\onecolumn\newpage\twocolumn
\subsection{Experiment: Occluding Tiles}
\label{appendix:chap3:tiles}

\subsubsection{Parameter Selection}

\paragraph{Metropolis Hastings (\MH)}

Figure~\ref{fig:exp2_MH} shows the results of
\MH~sampling. Results show the poor convergence for all proposal
standard deviations and rapid decrease of AR with increasing standard
deviation. This is due to the high-dimensional nature of
the problem. We selected a standard deviation of $1.1$.

\paragraph{Blocked Metropolis Hastings Within Gibbs (\BMHWG)}

The results of \BMHWG are shown in Figure~\ref{fig:exp2_BMHWG}. In
this sampler we update only one block of tile variables (of dimension
four) in each sampling step. Results show much better performance
compared to plain \MH. The optimal proposal standard deviation for
this sampler is $0.7$.

\paragraph{Metropolis Hastings Within Gibbs (\MHWG)}

Figure~\ref{fig:exp2_MHWG} shows the result of \MHWG sampling. This
sampler is better than \BMHWG and converges much more quickly. Here
a standard deviation of $0.9$ is found to be best.

\paragraph{Parallel Tempering (\PT)}

Figure~\ref{fig:exp2_PT} shows the results of \PT sampling with various
temperature combinations. Results show no improvement in AR from plain
\MH sampling and again $[1,3,27]$ temperature levels are found to be optimal.

\paragraph{Effect of Mixture Coefficient in Informed Sampling (\INFBMHWG)}

Figure~\ref{fig:exp2_alpha} shows the effect of mixture
coefficient ($\alpha$) on the blocked informed sampling
\INFBMHWG. Since there is no significant different in PSRF values for
$0 \le \alpha \le 0.8$, we chose $0.8$ due to its high acceptance
rate.



\subsection{Experiment: Estimating Body Shape}
\label{appendix:chap3:body}

\subsubsection{Parameter Selection}
\paragraph{Metropolis Hastings (\MH)}

Figure~\ref{fig:exp3_MH} shows the result of \MH~sampling with various
proposal standard deviations. The value of $0.1$ is found to be
best.

\paragraph{Metropolis Hastings Within Gibbs (\MHWG)}

For \MHWG sampling we select $0.3$ proposal standard
deviation. Results are shown in Fig.~\ref{fig:exp3_MHWG}.


\paragraph{Parallel Tempering (\PT)}

As before, results in Fig.~\ref{fig:exp3_PT}, the temperature levels
were selected to be $[1,3,27]$ due its slightly higher AR.

\paragraph{Effect of Mixture Coefficient in Informed Sampling (\MIXLMH)}

Figure~\ref{fig:exp3_alpha} shows the effect of $\alpha$ on PSRF and
AR. Since there is no significant differences in PSRF values for $0 \le
\alpha \le 0.8$, we choose $0.8$.


\begin{figure}[t]
\centering
  \subfigure[\MH]{%
    \includegraphics[width=.48\textwidth]{figures/supplementary/bodyShape_MH.pdf} \label{fig:exp3_MH}
  }
  \subfigure[\MHWG]{%
    \includegraphics[width=.48\textwidth]{figures/supplementary/bodyShape_MHWG.pdf} \label{fig:exp3_MHWG}
  }
\\
  \subfigure[\PT]{%
    \includegraphics[width=.48\textwidth]{figures/supplementary/bodyShape_PT.pdf} \label{fig:exp3_PT}
  }
  \subfigure[\MIXLMH]{%
    \includegraphics[width=.48\textwidth]{figures/supplementary/bodyShape_alpha.pdf} \label{fig:exp3_alpha}
  }
\\
  \mycaption{Results of the `Body Shape Estimation' experiment}{PRSFs and
    Acceptance rates corresponding to various standard deviations of
    (a) \MH, (b) \MHWG; (c) various temperature level combinations
    of \PT sampling and; (d) various mixture coefficients of the
    informed \MIXLMH sampling.}
\end{figure}


\subsection{Results Overview}
Figure~\ref{fig:exp_summary} shows the summary results of the all the three
experimental studies related to informed sampler.
\begin{figure*}[h!]
\centering
  \subfigure[Results for: Estimating Camera Extrinsics]{%
    \includegraphics[width=0.9\textwidth]{figures/supplementary/camPose_ALL.pdf} \label{fig:exp1_all}
  }
  \subfigure[Results for: Occluding Tiles]{%
    \includegraphics[width=0.9\textwidth]{figures/supplementary/occlusionExp_ALL.pdf} \label{fig:exp2_all}
  }
  \subfigure[Results for: Estimating Body Shape]{%
    \includegraphics[width=0.9\textwidth]{figures/supplementary/bodyShape_ALL.pdf} \label{fig:exp3_all}
  }
  \label{fig:exp_summary}
  \mycaption{Summary of the statistics for the three experiments}{Shown are
    for several baseline methods and the informed samplers the
    acceptance rates (left), PSRFs (middle), and RMSE values
    (right). All results are median results over multiple test
    examples.}
\end{figure*}

\subsection{Additional Qualitative Results}

\subsubsection{Occluding Tiles}
In Figure~\ref{fig:exp2_visual_more} more qualitative results of the
occluding tiles experiment are shown. The informed sampling approach
(\INFBMHWG) is better than the best baseline (\MHWG). This still is a
very challenging problem since the parameters for occluded tiles are
flat over a large region. Some of the posterior variance of the
occluded tiles is already captured by the informed sampler.

\begin{figure*}[h!]
\begin{center}
\centerline{\includegraphics[width=0.95\textwidth]{figures/supplementary/occlusionExp_Visual.pdf}}
\mycaption{Additional qualitative results of the occluding tiles experiment}
  {From left to right: (a)
  Given image, (b) Ground truth tiles, (c) OpenCV heuristic and most probable estimates
  from 5000 samples obtained by (d) MHWG sampler (best baseline) and
  (e) our INF-BMHWG sampler. (f) Posterior expectation of the tiles
  boundaries obtained by INF-BMHWG sampling (First 2000 samples are
  discarded as burn-in).}
\label{fig:exp2_visual_more}
\end{center}
\end{figure*}

\subsubsection{Body Shape}
Figure~\ref{fig:exp3_bodyMeshes} shows some more results of 3D mesh
reconstruction using posterior samples obtained by our informed
sampling \MIXLMH.

\begin{figure*}[t]
\begin{center}
\centerline{\includegraphics[width=0.75\textwidth]{figures/supplementary/bodyMeshResults.pdf}}
\mycaption{Qualitative results for the body shape experiment}
  {Shown is the 3D mesh reconstruction results with first 1000 samples obtained
  using the \MIXLMH informed sampling method. (blue indicates small
  values and red indicates high values)}
\label{fig:exp3_bodyMeshes}
\end{center}
\end{figure*}

\clearpage



\section{Additional Results on the Face Problem with CMP}

Figure~\ref{fig:shading-qualitative-multiple-subjects-supp} shows inference results for reflectance maps, normal maps and lights for randomly chosen test images, and Fig.~\ref{fig:shading-qualitative-same-subject-supp} shows reflectance estimation results on multiple images of the same subject produced under different illumination conditions. CMP is able to produce estimates that are closer to the groundtruth across different subjects and illumination conditions.

\begin{figure*}[h]
  \begin{center}
  \centerline{\includegraphics[width=1.0\columnwidth]{figures/face_cmp_visual_results_supp.pdf}}
  \vspace{-1.2cm}
  \end{center}
	\mycaption{A visual comparison of inference results}{(a)~Observed images. (b)~Inferred reflectance maps. \textit{GT} is the photometric stereo groundtruth, \textit{BU} is the Biswas \etal (2009) reflectance estimate and \textit{Forest} is the consensus prediction. (c)~The variance of the inferred reflectance estimate produced by \MTD (normalized across rows).(d)~Visualization of inferred light directions. (e)~Inferred normal maps.}
	\label{fig:shading-qualitative-multiple-subjects-supp}
\end{figure*}


\begin{figure*}[h]
	\centering
	\setlength\fboxsep{0.2mm}
	\setlength\fboxrule{0pt}
	\begin{tikzpicture}

		\matrix at (0, 0) [matrix of nodes, nodes={anchor=east}, column sep=-0.05cm, row sep=-0.2cm]
		{
			\fbox{\includegraphics[width=1cm]{figures/sample_3_4_X.png}} &
			\fbox{\includegraphics[width=1cm]{figures/sample_3_4_GT.png}} &
			\fbox{\includegraphics[width=1cm]{figures/sample_3_4_BISWAS.png}}  &
			\fbox{\includegraphics[width=1cm]{figures/sample_3_4_VMP.png}}  &
			\fbox{\includegraphics[width=1cm]{figures/sample_3_4_FOREST.png}}  &
			\fbox{\includegraphics[width=1cm]{figures/sample_3_4_CMP.png}}  &
			\fbox{\includegraphics[width=1cm]{figures/sample_3_4_CMPVAR.png}}
			 \\

			\fbox{\includegraphics[width=1cm]{figures/sample_3_5_X.png}} &
			\fbox{\includegraphics[width=1cm]{figures/sample_3_5_GT.png}} &
			\fbox{\includegraphics[width=1cm]{figures/sample_3_5_BISWAS.png}}  &
			\fbox{\includegraphics[width=1cm]{figures/sample_3_5_VMP.png}}  &
			\fbox{\includegraphics[width=1cm]{figures/sample_3_5_FOREST.png}}  &
			\fbox{\includegraphics[width=1cm]{figures/sample_3_5_CMP.png}}  &
			\fbox{\includegraphics[width=1cm]{figures/sample_3_5_CMPVAR.png}}
			 \\

			\fbox{\includegraphics[width=1cm]{figures/sample_3_6_X.png}} &
			\fbox{\includegraphics[width=1cm]{figures/sample_3_6_GT.png}} &
			\fbox{\includegraphics[width=1cm]{figures/sample_3_6_BISWAS.png}}  &
			\fbox{\includegraphics[width=1cm]{figures/sample_3_6_VMP.png}}  &
			\fbox{\includegraphics[width=1cm]{figures/sample_3_6_FOREST.png}}  &
			\fbox{\includegraphics[width=1cm]{figures/sample_3_6_CMP.png}}  &
			\fbox{\includegraphics[width=1cm]{figures/sample_3_6_CMPVAR.png}}
			 \\
	     };

       \node at (-3.85, -2.0) {\small Observed};
       \node at (-2.55, -2.0) {\small `GT'};
       \node at (-1.27, -2.0) {\small BU};
       \node at (0.0, -2.0) {\small MP};
       \node at (1.27, -2.0) {\small Forest};
       \node at (2.55, -2.0) {\small \textbf{CMP}};
       \node at (3.85, -2.0) {\small Variance};

	\end{tikzpicture}
	\mycaption{Robustness to varying illumination}{Reflectance estimation on a subject images with varying illumination. Left to right: observed image, photometric stereo estimate (GT)
  which is used as a proxy for groundtruth, bottom-up estimate of \cite{Biswas2009}, VMP result, consensus forest estimate, CMP mean, and CMP variance.}
	\label{fig:shading-qualitative-same-subject-supp}
\end{figure*}

\clearpage

\section{Additional Material for Learning Sparse High Dimensional Filters}
\label{sec:appendix-bnn}

This part of supplementary material contains a more detailed overview of the permutohedral
lattice convolution in Section~\ref{sec:permconv}, more experiments in
Section~\ref{sec:addexps} and additional results with protocols for
the experiments presented in Chapter~\ref{chap:bnn} in Section~\ref{sec:addresults}.

\vspace{-0.2cm}
\subsection{General Permutohedral Convolutions}
\label{sec:permconv}

A core technical contribution of this work is the generalization of the Gaussian permutohedral lattice
convolution proposed in~\cite{adams2010fast} to the full non-separable case with the
ability to perform back-propagation. Although, conceptually, there are minor
differences between Gaussian and general parameterized filters, there are non-trivial practical
differences in terms of the algorithmic implementation. The Gauss filters belong to
the separable class and can thus be decomposed into multiple
sequential one dimensional convolutions. We are interested in the general filter
convolutions, which can not be decomposed. Thus, performing a general permutohedral
convolution at a lattice point requires the computation of the inner product with the
neighboring elements in all the directions in the high-dimensional space.

Here, we give more details of the implementation differences of separable
and non-separable filters. In the following, we will explain the scalar case first.
Recall, that the forward pass of general permutohedral convolution
involves 3 steps: \textit{splatting}, \textit{convolving} and \textit{slicing}.
We follow the same splatting and slicing strategies as in~\cite{adams2010fast}
since these operations do not depend on the filter kernel. The main difference
between our work and the existing implementation of~\cite{adams2010fast} is
the way that the convolution operation is executed. This proceeds by constructing
a \emph{blur neighbor} matrix $K$ that stores for every lattice point all
values of the lattice neighbors that are needed to compute the filter output.

\begin{figure}[t!]
  \centering
    \includegraphics[width=0.6\columnwidth]{figures/supplementary/lattice_construction}
  \mycaption{Illustration of 1D permutohedral lattice construction}
  {A $4\times 4$ $(x,y)$ grid lattice is projected onto the plane defined by the normal
  vector $(1,1)^{\top}$. This grid has $s+1=4$ and $d=2$ $(s+1)^{d}=4^2=16$ elements.
  In the projection, all points of the same color are projected onto the same points in the plane.
  The number of elements of the projected lattice is $t=(s+1)^d-s^d=4^2-3^2=7$, that is
  the $(4\times 4)$ grid minus the size of lattice that is $1$ smaller at each size, in this
  case a $(3\times 3)$ lattice (the upper right $(3\times 3)$ elements).
  }
\label{fig:latticeconstruction}
\end{figure}

The blur neighbor matrix is constructed by traversing through all the populated
lattice points and their neighboring elements.
% For efficiency, we do this matrix construction recursively with shared computations
% since $n^{th}$ neighbourhood elements are $1^{st}$ neighborhood elements of $n-1^{th}$ neighbourhood elements. \pg{do not understand}
This is done recursively to share computations. For any lattice point, the neighbors that are
$n$ hops away are the direct neighbors of the points that are $n-1$ hops away.
The size of a $d$ dimensional spatial filter with width $s+1$ is $(s+1)^{d}$ (\eg, a
$3\times 3$ filter, $s=2$ in $d=2$ has $3^2=9$ elements) and this size grows
exponentially in the number of dimensions $d$. The permutohedral lattice is constructed by
projecting a regular grid onto the plane spanned by the $d$ dimensional normal vector ${(1,\ldots,1)}^{\top}$. See
Fig.~\ref{fig:latticeconstruction} for an illustration of the 1D lattice construction.
Many corners of a grid filter are projected onto the same point, in total $t = {(s+1)}^{d} -
s^{d}$ elements remain in the permutohedral filter with $s$ neighborhood in $d-1$ dimensions.
If the lattice has $m$ populated elements, the
matrix $K$ has size $t\times m$. Note that, since the input signal is typically
sparse, only a few lattice corners are being populated in the \textit{slicing} step.
We use a hash-table to keep track of these points and traverse only through
the populated lattice points for this neighborhood matrix construction.

Once the blur neighbor matrix $K$ is constructed, we can perform the convolution
by the matrix vector multiplication
\begin{equation}
\ell' = BK,
\label{eq:conv}
\end{equation}
where $B$ is the $1 \times t$ filter kernel (whose values we will learn) and $\ell'\in\mathbb{R}^{1\times m}$
is the result of the filtering at the $m$ lattice points. In practice, we found that the
matrix $K$ is sometimes too large to fit into GPU memory and we divided the matrix $K$
into smaller pieces to compute Eq.~\ref{eq:conv} sequentially.

In the general multi-dimensional case, the signal $\ell$ is of $c$ dimensions. Then
the kernel $B$ is of size $c \times t$ and $K$ stores the $c$ dimensional vectors
accordingly. When the input and output points are different, we slice only the
input points and splat only at the output points.


\subsection{Additional Experiments}
\label{sec:addexps}
In this section, we discuss more use-cases for the learned bilateral filters, one
use-case of BNNs and two single filter applications for image and 3D mesh denoising.

\subsubsection{Recognition of subsampled MNIST}\label{sec:app_mnist}

One of the strengths of the proposed filter convolution is that it does not
require the input to lie on a regular grid. The only requirement is to define a distance
between features of the input signal.
We highlight this feature with the following experiment using the
classical MNIST ten class classification problem~\cite{lecun1998mnist}. We sample a
sparse set of $N$ points $(x,y)\in [0,1]\times [0,1]$
uniformly at random in the input image, use their interpolated values
as signal and the \emph{continuous} $(x,y)$ positions as features. This mimics
sub-sampling of a high-dimensional signal. To compare against a spatial convolution,
we interpolate the sparse set of values at the grid positions.

We take a reference implementation of LeNet~\cite{lecun1998gradient} that
is part of the Caffe project~\cite{jia2014caffe} and compare it
against the same architecture but replacing the first convolutional
layer with a bilateral convolution layer (BCL). The filter size
and numbers are adjusted to get a comparable number of parameters
($5\times 5$ for LeNet, $2$-neighborhood for BCL).

The results are shown in Table~\ref{tab:all-results}. We see that training
on the original MNIST data (column Original, LeNet vs. BNN) leads to a slight
decrease in performance of the BNN (99.03\%) compared to LeNet
(99.19\%). The BNN can be trained and evaluated on sparse
signals, and we resample the image as described above for $N=$ 100\%, 60\% and
20\% of the total number of pixels. The methods are also evaluated
on test images that are subsampled in the same way. Note that we can
train and test with different subsampling rates. We introduce an additional
bilinear interpolation layer for the LeNet architecture to train on the same
data. In essence, both models perform a spatial interpolation and thus we
expect them to yield a similar classification accuracy. Once the data is of
higher dimensions, the permutohedral convolution will be faster due to hashing
the sparse input points, as well as less memory demanding in comparison to
naive application of a spatial convolution with interpolated values.

\begin{table}[t]
  \begin{center}
    \footnotesize
    \centering
    \begin{tabular}[t]{lllll}
      \toprule
              &     & \multicolumn{3}{c}{Test Subsampling} \\
       Method  & Original & 100\% & 60\% & 20\%\\
      \midrule
       LeNet &  \textbf{0.9919} & 0.9660 & 0.9348 & \textbf{0.6434} \\
       BNN &  0.9903 & \textbf{0.9844} & \textbf{0.9534} & 0.5767 \\
      \hline
       LeNet 100\% & 0.9856 & 0.9809 & 0.9678 & \textbf{0.7386} \\
       BNN 100\% & \textbf{0.9900} & \textbf{0.9863} & \textbf{0.9699} & 0.6910 \\
      \hline
       LeNet 60\% & 0.9848 & 0.9821 & 0.9740 & 0.8151 \\
       BNN 60\% & \textbf{0.9885} & \textbf{0.9864} & \textbf{0.9771} & \textbf{0.8214}\\
      \hline
       LeNet 20\% & \textbf{0.9763} & \textbf{0.9754} & 0.9695 & 0.8928 \\
       BNN 20\% & 0.9728 & 0.9735 & \textbf{0.9701} & \textbf{0.9042}\\
      \bottomrule
    \end{tabular}
  \end{center}
\vspace{-.2cm}
\caption{Classification accuracy on MNIST. We compare the
    LeNet~\cite{lecun1998gradient} implementation that is part of
    Caffe~\cite{jia2014caffe} to the network with the first layer
    replaced by a bilateral convolution layer (BCL). Both are trained
    on the original image resolution (first two rows). Three more BNN
    and CNN models are trained with randomly subsampled images (100\%,
    60\% and 20\% of the pixels). An additional bilinear interpolation
    layer samples the input signal on a spatial grid for the CNN model.
  }
  \label{tab:all-results}
\vspace{-.5cm}
\end{table}

\subsubsection{Image Denoising}

The main application that inspired the development of the bilateral
filtering operation is image denoising~\cite{aurich1995non}, there
using a single Gaussian kernel. Our development allows to learn this
kernel function from data and we explore how to improve using a \emph{single}
but more general bilateral filter.

We use the Berkeley segmentation dataset
(BSDS500)~\cite{arbelaezi2011bsds500} as a test bed. The color
images in the dataset are converted to gray-scale,
and corrupted with Gaussian noise with a standard deviation of
$\frac {25} {255}$.

We compare the performance of four different filter models on a
denoising task.
The first baseline model (`Spatial' in Table \ref{tab:denoising}, $25$
weights) uses a single spatial filter with a kernel size of
$5$ and predicts the scalar gray-scale value at the center pixel. The next model
(`Gauss Bilateral') applies a bilateral \emph{Gaussian}
filter to the noisy input, using position and intensity features $\f=(x,y,v)^\top$.
The third setup (`Learned Bilateral', $65$ weights)
takes a Gauss kernel as initialization and
fits all filter weights on the train set to minimize the
mean squared error with respect to the clean images.
We run a combination
of spatial and permutohedral convolutions on spatial and bilateral
features (`Spatial + Bilateral (Learned)') to check for a complementary
performance of the two convolutions.

\label{sec:exp:denoising}
\begin{table}[!h]
\begin{center}
  \footnotesize
  \begin{tabular}[t]{lr}
    \toprule
    Method & PSNR \\
    \midrule
    Noisy Input & $20.17$ \\
    Spatial & $26.27$ \\
    Gauss Bilateral & $26.51$ \\
    Learned Bilateral & $26.58$ \\
    Spatial + Bilateral (Learned) & \textbf{$26.65$} \\
    \bottomrule
  \end{tabular}
\end{center}
\vspace{-0.5em}
\caption{PSNR results of a denoising task using the BSDS500
  dataset~\cite{arbelaezi2011bsds500}}
\vspace{-0.5em}
\label{tab:denoising}
\end{table}
\vspace{-0.2em}

The PSNR scores evaluated on full images of the test set are
shown in Table \ref{tab:denoising}. We find that an untrained bilateral
filter already performs better than a trained spatial convolution
($26.27$ to $26.51$). A learned convolution further improve the
performance slightly. We chose this simple one-kernel setup to
validate an advantage of the generalized bilateral filter. A competitive
denoising system would employ RGB color information and also
needs to be properly adjusted in network size. Multi-layer perceptrons
have obtained state-of-the-art denoising results~\cite{burger12cvpr}
and the permutohedral lattice layer can readily be used in such an
architecture, which is intended future work.

\subsection{Additional results}
\label{sec:addresults}

This section contains more qualitative results for the experiments presented in Chapter~\ref{chap:bnn}.

\begin{figure*}[th!]
  \centering
    \includegraphics[width=\columnwidth,trim={5cm 2.5cm 5cm 4.5cm},clip]{figures/supplementary/lattice_viz.pdf}
    \vspace{-0.7cm}
  \mycaption{Visualization of the Permutohedral Lattice}
  {Sample lattice visualizations for different feature spaces. All pixels falling in the same simplex cell are shown with
  the same color. $(x,y)$ features correspond to image pixel positions, and $(r,g,b) \in [0,255]$ correspond
  to the red, green and blue color values.}
\label{fig:latticeviz}
\end{figure*}

\subsubsection{Lattice Visualization}

Figure~\ref{fig:latticeviz} shows sample lattice visualizations for different feature spaces.

\newcolumntype{L}[1]{>{\raggedright\let\newline\\\arraybackslash\hspace{0pt}}b{#1}}
\newcolumntype{C}[1]{>{\centering\let\newline\\\arraybackslash\hspace{0pt}}b{#1}}
\newcolumntype{R}[1]{>{\raggedleft\let\newline\\\arraybackslash\hspace{0pt}}b{#1}}

\subsubsection{Color Upsampling}\label{sec:color_upsampling}
\label{sec:col_upsample_extra}

Some images of the upsampling for the Pascal
VOC12 dataset are shown in Fig.~\ref{fig:Colour_upsample_visuals}. It is
especially the low level image details that are better preserved with
a learned bilateral filter compared to the Gaussian case.

\begin{figure*}[t!]
  \centering
    \subfigure{%
   \raisebox{2.0em}{
    \includegraphics[width=.06\columnwidth]{figures/supplementary/2007_004969.jpg}
   }
  }
  \subfigure{%
    \includegraphics[width=.17\columnwidth]{figures/supplementary/2007_004969_gray.pdf}
  }
  \subfigure{%
    \includegraphics[width=.17\columnwidth]{figures/supplementary/2007_004969_gt.pdf}
  }
  \subfigure{%
    \includegraphics[width=.17\columnwidth]{figures/supplementary/2007_004969_bicubic.pdf}
  }
  \subfigure{%
    \includegraphics[width=.17\columnwidth]{figures/supplementary/2007_004969_gauss.pdf}
  }
  \subfigure{%
    \includegraphics[width=.17\columnwidth]{figures/supplementary/2007_004969_learnt.pdf}
  }\\
    \subfigure{%
   \raisebox{2.0em}{
    \includegraphics[width=.06\columnwidth]{figures/supplementary/2007_003106.jpg}
   }
  }
  \subfigure{%
    \includegraphics[width=.17\columnwidth]{figures/supplementary/2007_003106_gray.pdf}
  }
  \subfigure{%
    \includegraphics[width=.17\columnwidth]{figures/supplementary/2007_003106_gt.pdf}
  }
  \subfigure{%
    \includegraphics[width=.17\columnwidth]{figures/supplementary/2007_003106_bicubic.pdf}
  }
  \subfigure{%
    \includegraphics[width=.17\columnwidth]{figures/supplementary/2007_003106_gauss.pdf}
  }
  \subfigure{%
    \includegraphics[width=.17\columnwidth]{figures/supplementary/2007_003106_learnt.pdf}
  }\\
  \setcounter{subfigure}{0}
  \small{
  \subfigure[Inp.]{%
  \raisebox{2.0em}{
    \includegraphics[width=.06\columnwidth]{figures/supplementary/2007_006837.jpg}
   }
  }
  \subfigure[Guidance]{%
    \includegraphics[width=.17\columnwidth]{figures/supplementary/2007_006837_gray.pdf}
  }
   \subfigure[GT]{%
    \includegraphics[width=.17\columnwidth]{figures/supplementary/2007_006837_gt.pdf}
  }
  \subfigure[Bicubic]{%
    \includegraphics[width=.17\columnwidth]{figures/supplementary/2007_006837_bicubic.pdf}
  }
  \subfigure[Gauss-BF]{%
    \includegraphics[width=.17\columnwidth]{figures/supplementary/2007_006837_gauss.pdf}
  }
  \subfigure[Learned-BF]{%
    \includegraphics[width=.17\columnwidth]{figures/supplementary/2007_006837_learnt.pdf}
  }
  }
  \vspace{-0.5cm}
  \mycaption{Color Upsampling}{Color $8\times$ upsampling results
  using different methods, from left to right, (a)~Low-resolution input color image (Inp.),
  (b)~Gray scale guidance image, (c)~Ground-truth color image; Upsampled color images with
  (d)~Bicubic interpolation, (e) Gauss bilateral upsampling and, (f)~Learned bilateral
  updampgling (best viewed on screen).}

\label{fig:Colour_upsample_visuals}
\end{figure*}

\subsubsection{Depth Upsampling}
\label{sec:depth_upsample_extra}

Figure~\ref{fig:depth_upsample_visuals} presents some more qualitative results comparing bicubic interpolation, Gauss
bilateral and learned bilateral upsampling on NYU depth dataset image~\cite{silberman2012indoor}.

\subsubsection{Character Recognition}\label{sec:app_character}

 Figure~\ref{fig:nnrecognition} shows the schematic of different layers
 of the network architecture for LeNet-7~\cite{lecun1998mnist}
 and DeepCNet(5, 50)~\cite{ciresan2012multi,graham2014spatially}. For the BNN variants, the first layer filters are replaced
 with learned bilateral filters and are learned end-to-end.

\subsubsection{Semantic Segmentation}\label{sec:app_semantic_segmentation}
\label{sec:semantic_bnn_extra}

Some more visual results for semantic segmentation are shown in Figure~\ref{fig:semantic_visuals}.
These include the underlying DeepLab CNN\cite{chen2014semantic} result (DeepLab),
the 2 step mean-field result with Gaussian edge potentials (+2stepMF-GaussCRF)
and also corresponding results with learned edge potentials (+2stepMF-LearnedCRF).
In general, we observe that mean-field in learned CRF leads to slightly dilated
classification regions in comparison to using Gaussian CRF thereby filling-in the
false negative pixels and also correcting some mis-classified regions.

\begin{figure*}[t!]
  \centering
    \subfigure{%
   \raisebox{2.0em}{
    \includegraphics[width=.06\columnwidth]{figures/supplementary/2bicubic}
   }
  }
  \subfigure{%
    \includegraphics[width=.17\columnwidth]{figures/supplementary/2given_image}
  }
  \subfigure{%
    \includegraphics[width=.17\columnwidth]{figures/supplementary/2ground_truth}
  }
  \subfigure{%
    \includegraphics[width=.17\columnwidth]{figures/supplementary/2bicubic}
  }
  \subfigure{%
    \includegraphics[width=.17\columnwidth]{figures/supplementary/2gauss}
  }
  \subfigure{%
    \includegraphics[width=.17\columnwidth]{figures/supplementary/2learnt}
  }\\
    \subfigure{%
   \raisebox{2.0em}{
    \includegraphics[width=.06\columnwidth]{figures/supplementary/32bicubic}
   }
  }
  \subfigure{%
    \includegraphics[width=.17\columnwidth]{figures/supplementary/32given_image}
  }
  \subfigure{%
    \includegraphics[width=.17\columnwidth]{figures/supplementary/32ground_truth}
  }
  \subfigure{%
    \includegraphics[width=.17\columnwidth]{figures/supplementary/32bicubic}
  }
  \subfigure{%
    \includegraphics[width=.17\columnwidth]{figures/supplementary/32gauss}
  }
  \subfigure{%
    \includegraphics[width=.17\columnwidth]{figures/supplementary/32learnt}
  }\\
  \setcounter{subfigure}{0}
  \small{
  \subfigure[Inp.]{%
  \raisebox{2.0em}{
    \includegraphics[width=.06\columnwidth]{figures/supplementary/41bicubic}
   }
  }
  \subfigure[Guidance]{%
    \includegraphics[width=.17\columnwidth]{figures/supplementary/41given_image}
  }
   \subfigure[GT]{%
    \includegraphics[width=.17\columnwidth]{figures/supplementary/41ground_truth}
  }
  \subfigure[Bicubic]{%
    \includegraphics[width=.17\columnwidth]{figures/supplementary/41bicubic}
  }
  \subfigure[Gauss-BF]{%
    \includegraphics[width=.17\columnwidth]{figures/supplementary/41gauss}
  }
  \subfigure[Learned-BF]{%
    \includegraphics[width=.17\columnwidth]{figures/supplementary/41learnt}
  }
  }
  \mycaption{Depth Upsampling}{Depth $8\times$ upsampling results
  using different upsampling strategies, from left to right,
  (a)~Low-resolution input depth image (Inp.),
  (b)~High-resolution guidance image, (c)~Ground-truth depth; Upsampled depth images with
  (d)~Bicubic interpolation, (e) Gauss bilateral upsampling and, (f)~Learned bilateral
  updampgling (best viewed on screen).}

\label{fig:depth_upsample_visuals}
\end{figure*}

\subsubsection{Material Segmentation}\label{sec:app_material_segmentation}
\label{sec:material_bnn_extra}

In Fig.~\ref{fig:material_visuals-app2}, we present visual results comparing 2 step
mean-field inference with Gaussian and learned pairwise CRF potentials. In
general, we observe that the pixels belonging to dominant classes in the
training data are being more accurately classified with learned CRF. This leads to
a significant improvements in overall pixel accuracy. This also results
in a slight decrease of the accuracy from less frequent class pixels thereby
slightly reducing the average class accuracy with learning. We attribute this
to the type of annotation that is available for this dataset, which is not
for the entire image but for some segments in the image. We have very few
images of the infrequent classes to combat this behaviour during training.

\subsubsection{Experiment Protocols}
\label{sec:protocols}

Table~\ref{tbl:parameters} shows experiment protocols of different experiments.

 \begin{figure*}[t!]
  \centering
  \subfigure[LeNet-7]{
    \includegraphics[width=0.7\columnwidth]{figures/supplementary/lenet_cnn_network}
    }\\
    \subfigure[DeepCNet]{
    \includegraphics[width=\columnwidth]{figures/supplementary/deepcnet_cnn_network}
    }
  \mycaption{CNNs for Character Recognition}
  {Schematic of (top) LeNet-7~\cite{lecun1998mnist} and (bottom) DeepCNet(5,50)~\cite{ciresan2012multi,graham2014spatially} architectures used in Assamese
  character recognition experiments.}
\label{fig:nnrecognition}
\end{figure*}

\definecolor{voc_1}{RGB}{0, 0, 0}
\definecolor{voc_2}{RGB}{128, 0, 0}
\definecolor{voc_3}{RGB}{0, 128, 0}
\definecolor{voc_4}{RGB}{128, 128, 0}
\definecolor{voc_5}{RGB}{0, 0, 128}
\definecolor{voc_6}{RGB}{128, 0, 128}
\definecolor{voc_7}{RGB}{0, 128, 128}
\definecolor{voc_8}{RGB}{128, 128, 128}
\definecolor{voc_9}{RGB}{64, 0, 0}
\definecolor{voc_10}{RGB}{192, 0, 0}
\definecolor{voc_11}{RGB}{64, 128, 0}
\definecolor{voc_12}{RGB}{192, 128, 0}
\definecolor{voc_13}{RGB}{64, 0, 128}
\definecolor{voc_14}{RGB}{192, 0, 128}
\definecolor{voc_15}{RGB}{64, 128, 128}
\definecolor{voc_16}{RGB}{192, 128, 128}
\definecolor{voc_17}{RGB}{0, 64, 0}
\definecolor{voc_18}{RGB}{128, 64, 0}
\definecolor{voc_19}{RGB}{0, 192, 0}
\definecolor{voc_20}{RGB}{128, 192, 0}
\definecolor{voc_21}{RGB}{0, 64, 128}
\definecolor{voc_22}{RGB}{128, 64, 128}

\begin{figure*}[t]
  \centering
  \small{
  \fcolorbox{white}{voc_1}{\rule{0pt}{6pt}\rule{6pt}{0pt}} Background~~
  \fcolorbox{white}{voc_2}{\rule{0pt}{6pt}\rule{6pt}{0pt}} Aeroplane~~
  \fcolorbox{white}{voc_3}{\rule{0pt}{6pt}\rule{6pt}{0pt}} Bicycle~~
  \fcolorbox{white}{voc_4}{\rule{0pt}{6pt}\rule{6pt}{0pt}} Bird~~
  \fcolorbox{white}{voc_5}{\rule{0pt}{6pt}\rule{6pt}{0pt}} Boat~~
  \fcolorbox{white}{voc_6}{\rule{0pt}{6pt}\rule{6pt}{0pt}} Bottle~~
  \fcolorbox{white}{voc_7}{\rule{0pt}{6pt}\rule{6pt}{0pt}} Bus~~
  \fcolorbox{white}{voc_8}{\rule{0pt}{6pt}\rule{6pt}{0pt}} Car~~ \\
  \fcolorbox{white}{voc_9}{\rule{0pt}{6pt}\rule{6pt}{0pt}} Cat~~
  \fcolorbox{white}{voc_10}{\rule{0pt}{6pt}\rule{6pt}{0pt}} Chair~~
  \fcolorbox{white}{voc_11}{\rule{0pt}{6pt}\rule{6pt}{0pt}} Cow~~
  \fcolorbox{white}{voc_12}{\rule{0pt}{6pt}\rule{6pt}{0pt}} Dining Table~~
  \fcolorbox{white}{voc_13}{\rule{0pt}{6pt}\rule{6pt}{0pt}} Dog~~
  \fcolorbox{white}{voc_14}{\rule{0pt}{6pt}\rule{6pt}{0pt}} Horse~~
  \fcolorbox{white}{voc_15}{\rule{0pt}{6pt}\rule{6pt}{0pt}} Motorbike~~
  \fcolorbox{white}{voc_16}{\rule{0pt}{6pt}\rule{6pt}{0pt}} Person~~ \\
  \fcolorbox{white}{voc_17}{\rule{0pt}{6pt}\rule{6pt}{0pt}} Potted Plant~~
  \fcolorbox{white}{voc_18}{\rule{0pt}{6pt}\rule{6pt}{0pt}} Sheep~~
  \fcolorbox{white}{voc_19}{\rule{0pt}{6pt}\rule{6pt}{0pt}} Sofa~~
  \fcolorbox{white}{voc_20}{\rule{0pt}{6pt}\rule{6pt}{0pt}} Train~~
  \fcolorbox{white}{voc_21}{\rule{0pt}{6pt}\rule{6pt}{0pt}} TV monitor~~ \\
  }
  \subfigure{%
    \includegraphics[width=.18\columnwidth]{figures/supplementary/2007_001423_given.jpg}
  }
  \subfigure{%
    \includegraphics[width=.18\columnwidth]{figures/supplementary/2007_001423_gt.png}
  }
  \subfigure{%
    \includegraphics[width=.18\columnwidth]{figures/supplementary/2007_001423_cnn.png}
  }
  \subfigure{%
    \includegraphics[width=.18\columnwidth]{figures/supplementary/2007_001423_gauss.png}
  }
  \subfigure{%
    \includegraphics[width=.18\columnwidth]{figures/supplementary/2007_001423_learnt.png}
  }\\
  \subfigure{%
    \includegraphics[width=.18\columnwidth]{figures/supplementary/2007_001430_given.jpg}
  }
  \subfigure{%
    \includegraphics[width=.18\columnwidth]{figures/supplementary/2007_001430_gt.png}
  }
  \subfigure{%
    \includegraphics[width=.18\columnwidth]{figures/supplementary/2007_001430_cnn.png}
  }
  \subfigure{%
    \includegraphics[width=.18\columnwidth]{figures/supplementary/2007_001430_gauss.png}
  }
  \subfigure{%
    \includegraphics[width=.18\columnwidth]{figures/supplementary/2007_001430_learnt.png}
  }\\
    \subfigure{%
    \includegraphics[width=.18\columnwidth]{figures/supplementary/2007_007996_given.jpg}
  }
  \subfigure{%
    \includegraphics[width=.18\columnwidth]{figures/supplementary/2007_007996_gt.png}
  }
  \subfigure{%
    \includegraphics[width=.18\columnwidth]{figures/supplementary/2007_007996_cnn.png}
  }
  \subfigure{%
    \includegraphics[width=.18\columnwidth]{figures/supplementary/2007_007996_gauss.png}
  }
  \subfigure{%
    \includegraphics[width=.18\columnwidth]{figures/supplementary/2007_007996_learnt.png}
  }\\
   \subfigure{%
    \includegraphics[width=.18\columnwidth]{figures/supplementary/2010_002682_given.jpg}
  }
  \subfigure{%
    \includegraphics[width=.18\columnwidth]{figures/supplementary/2010_002682_gt.png}
  }
  \subfigure{%
    \includegraphics[width=.18\columnwidth]{figures/supplementary/2010_002682_cnn.png}
  }
  \subfigure{%
    \includegraphics[width=.18\columnwidth]{figures/supplementary/2010_002682_gauss.png}
  }
  \subfigure{%
    \includegraphics[width=.18\columnwidth]{figures/supplementary/2010_002682_learnt.png}
  }\\
     \subfigure{%
    \includegraphics[width=.18\columnwidth]{figures/supplementary/2010_004789_given.jpg}
  }
  \subfigure{%
    \includegraphics[width=.18\columnwidth]{figures/supplementary/2010_004789_gt.png}
  }
  \subfigure{%
    \includegraphics[width=.18\columnwidth]{figures/supplementary/2010_004789_cnn.png}
  }
  \subfigure{%
    \includegraphics[width=.18\columnwidth]{figures/supplementary/2010_004789_gauss.png}
  }
  \subfigure{%
    \includegraphics[width=.18\columnwidth]{figures/supplementary/2010_004789_learnt.png}
  }\\
       \subfigure{%
    \includegraphics[width=.18\columnwidth]{figures/supplementary/2007_001311_given.jpg}
  }
  \subfigure{%
    \includegraphics[width=.18\columnwidth]{figures/supplementary/2007_001311_gt.png}
  }
  \subfigure{%
    \includegraphics[width=.18\columnwidth]{figures/supplementary/2007_001311_cnn.png}
  }
  \subfigure{%
    \includegraphics[width=.18\columnwidth]{figures/supplementary/2007_001311_gauss.png}
  }
  \subfigure{%
    \includegraphics[width=.18\columnwidth]{figures/supplementary/2007_001311_learnt.png}
  }\\
  \setcounter{subfigure}{0}
  \subfigure[Input]{%
    \includegraphics[width=.18\columnwidth]{figures/supplementary/2010_003531_given.jpg}
  }
  \subfigure[Ground Truth]{%
    \includegraphics[width=.18\columnwidth]{figures/supplementary/2010_003531_gt.png}
  }
  \subfigure[DeepLab]{%
    \includegraphics[width=.18\columnwidth]{figures/supplementary/2010_003531_cnn.png}
  }
  \subfigure[+GaussCRF]{%
    \includegraphics[width=.18\columnwidth]{figures/supplementary/2010_003531_gauss.png}
  }
  \subfigure[+LearnedCRF]{%
    \includegraphics[width=.18\columnwidth]{figures/supplementary/2010_003531_learnt.png}
  }
  \vspace{-0.3cm}
  \mycaption{Semantic Segmentation}{Example results of semantic segmentation.
  (c)~depicts the unary results before application of MF, (d)~after two steps of MF with Gaussian edge CRF potentials, (e)~after
  two steps of MF with learned edge CRF potentials.}
    \label{fig:semantic_visuals}
\end{figure*}


\definecolor{minc_1}{HTML}{771111}
\definecolor{minc_2}{HTML}{CAC690}
\definecolor{minc_3}{HTML}{EEEEEE}
\definecolor{minc_4}{HTML}{7C8FA6}
\definecolor{minc_5}{HTML}{597D31}
\definecolor{minc_6}{HTML}{104410}
\definecolor{minc_7}{HTML}{BB819C}
\definecolor{minc_8}{HTML}{D0CE48}
\definecolor{minc_9}{HTML}{622745}
\definecolor{minc_10}{HTML}{666666}
\definecolor{minc_11}{HTML}{D54A31}
\definecolor{minc_12}{HTML}{101044}
\definecolor{minc_13}{HTML}{444126}
\definecolor{minc_14}{HTML}{75D646}
\definecolor{minc_15}{HTML}{DD4348}
\definecolor{minc_16}{HTML}{5C8577}
\definecolor{minc_17}{HTML}{C78472}
\definecolor{minc_18}{HTML}{75D6D0}
\definecolor{minc_19}{HTML}{5B4586}
\definecolor{minc_20}{HTML}{C04393}
\definecolor{minc_21}{HTML}{D69948}
\definecolor{minc_22}{HTML}{7370D8}
\definecolor{minc_23}{HTML}{7A3622}
\definecolor{minc_24}{HTML}{000000}

\begin{figure*}[t]
  \centering
  \small{
  \fcolorbox{white}{minc_1}{\rule{0pt}{6pt}\rule{6pt}{0pt}} Brick~~
  \fcolorbox{white}{minc_2}{\rule{0pt}{6pt}\rule{6pt}{0pt}} Carpet~~
  \fcolorbox{white}{minc_3}{\rule{0pt}{6pt}\rule{6pt}{0pt}} Ceramic~~
  \fcolorbox{white}{minc_4}{\rule{0pt}{6pt}\rule{6pt}{0pt}} Fabric~~
  \fcolorbox{white}{minc_5}{\rule{0pt}{6pt}\rule{6pt}{0pt}} Foliage~~
  \fcolorbox{white}{minc_6}{\rule{0pt}{6pt}\rule{6pt}{0pt}} Food~~
  \fcolorbox{white}{minc_7}{\rule{0pt}{6pt}\rule{6pt}{0pt}} Glass~~
  \fcolorbox{white}{minc_8}{\rule{0pt}{6pt}\rule{6pt}{0pt}} Hair~~ \\
  \fcolorbox{white}{minc_9}{\rule{0pt}{6pt}\rule{6pt}{0pt}} Leather~~
  \fcolorbox{white}{minc_10}{\rule{0pt}{6pt}\rule{6pt}{0pt}} Metal~~
  \fcolorbox{white}{minc_11}{\rule{0pt}{6pt}\rule{6pt}{0pt}} Mirror~~
  \fcolorbox{white}{minc_12}{\rule{0pt}{6pt}\rule{6pt}{0pt}} Other~~
  \fcolorbox{white}{minc_13}{\rule{0pt}{6pt}\rule{6pt}{0pt}} Painted~~
  \fcolorbox{white}{minc_14}{\rule{0pt}{6pt}\rule{6pt}{0pt}} Paper~~
  \fcolorbox{white}{minc_15}{\rule{0pt}{6pt}\rule{6pt}{0pt}} Plastic~~\\
  \fcolorbox{white}{minc_16}{\rule{0pt}{6pt}\rule{6pt}{0pt}} Polished Stone~~
  \fcolorbox{white}{minc_17}{\rule{0pt}{6pt}\rule{6pt}{0pt}} Skin~~
  \fcolorbox{white}{minc_18}{\rule{0pt}{6pt}\rule{6pt}{0pt}} Sky~~
  \fcolorbox{white}{minc_19}{\rule{0pt}{6pt}\rule{6pt}{0pt}} Stone~~
  \fcolorbox{white}{minc_20}{\rule{0pt}{6pt}\rule{6pt}{0pt}} Tile~~
  \fcolorbox{white}{minc_21}{\rule{0pt}{6pt}\rule{6pt}{0pt}} Wallpaper~~
  \fcolorbox{white}{minc_22}{\rule{0pt}{6pt}\rule{6pt}{0pt}} Water~~
  \fcolorbox{white}{minc_23}{\rule{0pt}{6pt}\rule{6pt}{0pt}} Wood~~ \\
  }
  \subfigure{%
    \includegraphics[width=.18\columnwidth]{figures/supplementary/000010868_given.jpg}
  }
  \subfigure{%
    \includegraphics[width=.18\columnwidth]{figures/supplementary/000010868_gt.png}
  }
  \subfigure{%
    \includegraphics[width=.18\columnwidth]{figures/supplementary/000010868_cnn.png}
  }
  \subfigure{%
    \includegraphics[width=.18\columnwidth]{figures/supplementary/000010868_gauss.png}
  }
  \subfigure{%
    \includegraphics[width=.18\columnwidth]{figures/supplementary/000010868_learnt.png}
  }\\[-2ex]
  \subfigure{%
    \includegraphics[width=.18\columnwidth]{figures/supplementary/000006011_given.jpg}
  }
  \subfigure{%
    \includegraphics[width=.18\columnwidth]{figures/supplementary/000006011_gt.png}
  }
  \subfigure{%
    \includegraphics[width=.18\columnwidth]{figures/supplementary/000006011_cnn.png}
  }
  \subfigure{%
    \includegraphics[width=.18\columnwidth]{figures/supplementary/000006011_gauss.png}
  }
  \subfigure{%
    \includegraphics[width=.18\columnwidth]{figures/supplementary/000006011_learnt.png}
  }\\[-2ex]
    \subfigure{%
    \includegraphics[width=.18\columnwidth]{figures/supplementary/000008553_given.jpg}
  }
  \subfigure{%
    \includegraphics[width=.18\columnwidth]{figures/supplementary/000008553_gt.png}
  }
  \subfigure{%
    \includegraphics[width=.18\columnwidth]{figures/supplementary/000008553_cnn.png}
  }
  \subfigure{%
    \includegraphics[width=.18\columnwidth]{figures/supplementary/000008553_gauss.png}
  }
  \subfigure{%
    \includegraphics[width=.18\columnwidth]{figures/supplementary/000008553_learnt.png}
  }\\[-2ex]
   \subfigure{%
    \includegraphics[width=.18\columnwidth]{figures/supplementary/000009188_given.jpg}
  }
  \subfigure{%
    \includegraphics[width=.18\columnwidth]{figures/supplementary/000009188_gt.png}
  }
  \subfigure{%
    \includegraphics[width=.18\columnwidth]{figures/supplementary/000009188_cnn.png}
  }
  \subfigure{%
    \includegraphics[width=.18\columnwidth]{figures/supplementary/000009188_gauss.png}
  }
  \subfigure{%
    \includegraphics[width=.18\columnwidth]{figures/supplementary/000009188_learnt.png}
  }\\[-2ex]
  \setcounter{subfigure}{0}
  \subfigure[Input]{%
    \includegraphics[width=.18\columnwidth]{figures/supplementary/000023570_given.jpg}
  }
  \subfigure[Ground Truth]{%
    \includegraphics[width=.18\columnwidth]{figures/supplementary/000023570_gt.png}
  }
  \subfigure[DeepLab]{%
    \includegraphics[width=.18\columnwidth]{figures/supplementary/000023570_cnn.png}
  }
  \subfigure[+GaussCRF]{%
    \includegraphics[width=.18\columnwidth]{figures/supplementary/000023570_gauss.png}
  }
  \subfigure[+LearnedCRF]{%
    \includegraphics[width=.18\columnwidth]{figures/supplementary/000023570_learnt.png}
  }
  \mycaption{Material Segmentation}{Example results of material segmentation.
  (c)~depicts the unary results before application of MF, (d)~after two steps of MF with Gaussian edge CRF potentials, (e)~after two steps of MF with learned edge CRF potentials.}
    \label{fig:material_visuals-app2}
\end{figure*}


\begin{table*}[h]
\tiny
  \centering
    \begin{tabular}{L{2.3cm} L{2.25cm} C{1.5cm} C{0.7cm} C{0.6cm} C{0.7cm} C{0.7cm} C{0.7cm} C{1.6cm} C{0.6cm} C{0.6cm} C{0.6cm}}
      \toprule
& & & & & \multicolumn{3}{c}{\textbf{Data Statistics}} & \multicolumn{4}{c}{\textbf{Training Protocol}} \\

\textbf{Experiment} & \textbf{Feature Types} & \textbf{Feature Scales} & \textbf{Filter Size} & \textbf{Filter Nbr.} & \textbf{Train}  & \textbf{Val.} & \textbf{Test} & \textbf{Loss Type} & \textbf{LR} & \textbf{Batch} & \textbf{Epochs} \\
      \midrule
      \multicolumn{2}{c}{\textbf{Single Bilateral Filter Applications}} & & & & & & & & & \\
      \textbf{2$\times$ Color Upsampling} & Position$_{1}$, Intensity (3D) & 0.13, 0.17 & 65 & 2 & 10581 & 1449 & 1456 & MSE & 1e-06 & 200 & 94.5\\
      \textbf{4$\times$ Color Upsampling} & Position$_{1}$, Intensity (3D) & 0.06, 0.17 & 65 & 2 & 10581 & 1449 & 1456 & MSE & 1e-06 & 200 & 94.5\\
      \textbf{8$\times$ Color Upsampling} & Position$_{1}$, Intensity (3D) & 0.03, 0.17 & 65 & 2 & 10581 & 1449 & 1456 & MSE & 1e-06 & 200 & 94.5\\
      \textbf{16$\times$ Color Upsampling} & Position$_{1}$, Intensity (3D) & 0.02, 0.17 & 65 & 2 & 10581 & 1449 & 1456 & MSE & 1e-06 & 200 & 94.5\\
      \textbf{Depth Upsampling} & Position$_{1}$, Color (5D) & 0.05, 0.02 & 665 & 2 & 795 & 100 & 654 & MSE & 1e-07 & 50 & 251.6\\
      \textbf{Mesh Denoising} & Isomap (4D) & 46.00 & 63 & 2 & 1000 & 200 & 500 & MSE & 100 & 10 & 100.0 \\
      \midrule
      \multicolumn{2}{c}{\textbf{DenseCRF Applications}} & & & & & & & & &\\
      \multicolumn{2}{l}{\textbf{Semantic Segmentation}} & & & & & & & & &\\
      \textbf{- 1step MF} & Position$_{1}$, Color (5D); Position$_{1}$ (2D) & 0.01, 0.34; 0.34  & 665; 19  & 2; 2 & 10581 & 1449 & 1456 & Logistic & 0.1 & 5 & 1.4 \\
      \textbf{- 2step MF} & Position$_{1}$, Color (5D); Position$_{1}$ (2D) & 0.01, 0.34; 0.34 & 665; 19 & 2; 2 & 10581 & 1449 & 1456 & Logistic & 0.1 & 5 & 1.4 \\
      \textbf{- \textit{loose} 2step MF} & Position$_{1}$, Color (5D); Position$_{1}$ (2D) & 0.01, 0.34; 0.34 & 665; 19 & 2; 2 &10581 & 1449 & 1456 & Logistic & 0.1 & 5 & +1.9  \\ \\
      \multicolumn{2}{l}{\textbf{Material Segmentation}} & & & & & & & & &\\
      \textbf{- 1step MF} & Position$_{2}$, Lab-Color (5D) & 5.00, 0.05, 0.30  & 665 & 2 & 928 & 150 & 1798 & Weighted Logistic & 1e-04 & 24 & 2.6 \\
      \textbf{- 2step MF} & Position$_{2}$, Lab-Color (5D) & 5.00, 0.05, 0.30 & 665 & 2 & 928 & 150 & 1798 & Weighted Logistic & 1e-04 & 12 & +0.7 \\
      \textbf{- \textit{loose} 2step MF} & Position$_{2}$, Lab-Color (5D) & 5.00, 0.05, 0.30 & 665 & 2 & 928 & 150 & 1798 & Weighted Logistic & 1e-04 & 12 & +0.2\\
      \midrule
      \multicolumn{2}{c}{\textbf{Neural Network Applications}} & & & & & & & & &\\
      \textbf{Tiles: CNN-9$\times$9} & - & - & 81 & 4 & 10000 & 1000 & 1000 & Logistic & 0.01 & 100 & 500.0 \\
      \textbf{Tiles: CNN-13$\times$13} & - & - & 169 & 6 & 10000 & 1000 & 1000 & Logistic & 0.01 & 100 & 500.0 \\
      \textbf{Tiles: CNN-17$\times$17} & - & - & 289 & 8 & 10000 & 1000 & 1000 & Logistic & 0.01 & 100 & 500.0 \\
      \textbf{Tiles: CNN-21$\times$21} & - & - & 441 & 10 & 10000 & 1000 & 1000 & Logistic & 0.01 & 100 & 500.0 \\
      \textbf{Tiles: BNN} & Position$_{1}$, Color (5D) & 0.05, 0.04 & 63 & 1 & 10000 & 1000 & 1000 & Logistic & 0.01 & 100 & 30.0 \\
      \textbf{LeNet} & - & - & 25 & 2 & 5490 & 1098 & 1647 & Logistic & 0.1 & 100 & 182.2 \\
      \textbf{Crop-LeNet} & - & - & 25 & 2 & 5490 & 1098 & 1647 & Logistic & 0.1 & 100 & 182.2 \\
      \textbf{BNN-LeNet} & Position$_{2}$ (2D) & 20.00 & 7 & 1 & 5490 & 1098 & 1647 & Logistic & 0.1 & 100 & 182.2 \\
      \textbf{DeepCNet} & - & - & 9 & 1 & 5490 & 1098 & 1647 & Logistic & 0.1 & 100 & 182.2 \\
      \textbf{Crop-DeepCNet} & - & - & 9 & 1 & 5490 & 1098 & 1647 & Logistic & 0.1 & 100 & 182.2 \\
      \textbf{BNN-DeepCNet} & Position$_{2}$ (2D) & 40.00  & 7 & 1 & 5490 & 1098 & 1647 & Logistic & 0.1 & 100 & 182.2 \\
      \bottomrule
      \\
    \end{tabular}
    \mycaption{Experiment Protocols} {Experiment protocols for the different experiments presented in this work. \textbf{Feature Types}:
    Feature spaces used for the bilateral convolutions. Position$_1$ corresponds to un-normalized pixel positions whereas Position$_2$ corresponds
    to pixel positions normalized to $[0,1]$ with respect to the given image. \textbf{Feature Scales}: Cross-validated scales for the features used.
     \textbf{Filter Size}: Number of elements in the filter that is being learned. \textbf{Filter Nbr.}: Half-width of the filter. \textbf{Train},
     \textbf{Val.} and \textbf{Test} corresponds to the number of train, validation and test images used in the experiment. \textbf{Loss Type}: Type
     of loss used for back-propagation. ``MSE'' corresponds to Euclidean mean squared error loss and ``Logistic'' corresponds to multinomial logistic
     loss. ``Weighted Logistic'' is the class-weighted multinomial logistic loss. We weighted the loss with inverse class probability for material
     segmentation task due to the small availability of training data with class imbalance. \textbf{LR}: Fixed learning rate used in stochastic gradient
     descent. \textbf{Batch}: Number of images used in one parameter update step. \textbf{Epochs}: Number of training epochs. In all the experiments,
     we used fixed momentum of 0.9 and weight decay of 0.0005 for stochastic gradient descent. ```Color Upsampling'' experiments in this Table corresponds
     to those performed on Pascal VOC12 dataset images. For all experiments using Pascal VOC12 images, we use extended
     training segmentation dataset available from~\cite{hariharan2011moredata}, and used standard validation and test splits
     from the main dataset~\cite{voc2012segmentation}.}
  \label{tbl:parameters}
\end{table*}

\clearpage

\section{Parameters and Additional Results for Video Propagation Networks}

In this Section, we present experiment protocols and additional qualitative results for experiments
on video object segmentation, semantic video segmentation and video color
propagation. Table~\ref{tbl:parameters_supp} shows the feature scales and other parameters used in different experiments.
Figures~\ref{fig:video_seg_pos_supp} show some qualitative results on video object segmentation
with some failure cases in Fig.~\ref{fig:video_seg_neg_supp}.
Figure~\ref{fig:semantic_visuals_supp} shows some qualitative results on semantic video segmentation and
Fig.~\ref{fig:color_visuals_supp} shows results on video color propagation.

\newcolumntype{L}[1]{>{\raggedright\let\newline\\\arraybackslash\hspace{0pt}}b{#1}}
\newcolumntype{C}[1]{>{\centering\let\newline\\\arraybackslash\hspace{0pt}}b{#1}}
\newcolumntype{R}[1]{>{\raggedleft\let\newline\\\arraybackslash\hspace{0pt}}b{#1}}

\begin{table*}[h]
\tiny
  \centering
    \begin{tabular}{L{3.0cm} L{2.4cm} L{2.8cm} L{2.8cm} C{0.5cm} C{1.0cm} L{1.2cm}}
      \toprule
\textbf{Experiment} & \textbf{Feature Type} & \textbf{Feature Scale-1, $\Lambda_a$} & \textbf{Feature Scale-2, $\Lambda_b$} & \textbf{$\alpha$} & \textbf{Input Frames} & \textbf{Loss Type} \\
      \midrule
      \textbf{Video Object Segmentation} & ($x,y,Y,Cb,Cr,t$) & (0.02,0.02,0.07,0.4,0.4,0.01) & (0.03,0.03,0.09,0.5,0.5,0.2) & 0.5 & 9 & Logistic\\
      \midrule
      \textbf{Semantic Video Segmentation} & & & & & \\
      \textbf{with CNN1~\cite{yu2015multi}-NoFlow} & ($x,y,R,G,B,t$) & (0.08,0.08,0.2,0.2,0.2,0.04) & (0.11,0.11,0.2,0.2,0.2,0.04) & 0.5 & 3 & Logistic \\
      \textbf{with CNN1~\cite{yu2015multi}-Flow} & ($x+u_x,y+u_y,R,G,B,t$) & (0.11,0.11,0.14,0.14,0.14,0.03) & (0.08,0.08,0.12,0.12,0.12,0.01) & 0.65 & 3 & Logistic\\
      \textbf{with CNN2~\cite{richter2016playing}-Flow} & ($x+u_x,y+u_y,R,G,B,t$) & (0.08,0.08,0.2,0.2,0.2,0.04) & (0.09,0.09,0.25,0.25,0.25,0.03) & 0.5 & 4 & Logistic\\
      \midrule
      \textbf{Video Color Propagation} & ($x,y,I,t$)  & (0.04,0.04,0.2,0.04) & No second kernel & 1 & 4 & MSE\\
      \bottomrule
      \\
    \end{tabular}
    \mycaption{Experiment Protocols} {Experiment protocols for the different experiments presented in this work. \textbf{Feature Types}:
    Feature spaces used for the bilateral convolutions, with position ($x,y$) and color
    ($R,G,B$ or $Y,Cb,Cr$) features $\in [0,255]$. $u_x$, $u_y$ denotes optical flow with respect
    to the present frame and $I$ denotes grayscale intensity.
    \textbf{Feature Scales ($\Lambda_a, \Lambda_b$)}: Cross-validated scales for the features used.
    \textbf{$\alpha$}: Exponential time decay for the input frames.
    \textbf{Input Frames}: Number of input frames for VPN.
    \textbf{Loss Type}: Type
     of loss used for back-propagation. ``MSE'' corresponds to Euclidean mean squared error loss and ``Logistic'' corresponds to multinomial logistic loss.}
  \label{tbl:parameters_supp}
\end{table*}

% \begin{figure}[th!]
% \begin{center}
%   \centerline{\includegraphics[width=\textwidth]{figures/video_seg_visuals_supp_small.pdf}}
%     \mycaption{Video Object Segmentation}
%     {Shown are the different frames in example videos with the corresponding
%     ground truth (GT) masks, predictions from BVS~\cite{marki2016bilateral},
%     OFL~\cite{tsaivideo}, VPN (VPN-Stage2) and VPN-DLab (VPN-DeepLab) models.}
%     \label{fig:video_seg_small_supp}
% \end{center}
% \vspace{-1.0cm}
% \end{figure}

\begin{figure}[th!]
\begin{center}
  \centerline{\includegraphics[width=0.7\textwidth]{figures/video_seg_visuals_supp_positive.pdf}}
    \mycaption{Video Object Segmentation}
    {Shown are the different frames in example videos with the corresponding
    ground truth (GT) masks, predictions from BVS~\cite{marki2016bilateral},
    OFL~\cite{tsaivideo}, VPN (VPN-Stage2) and VPN-DLab (VPN-DeepLab) models.}
    \label{fig:video_seg_pos_supp}
\end{center}
\vspace{-1.0cm}
\end{figure}

\begin{figure}[th!]
\begin{center}
  \centerline{\includegraphics[width=0.7\textwidth]{figures/video_seg_visuals_supp_negative.pdf}}
    \mycaption{Failure Cases for Video Object Segmentation}
    {Shown are the different frames in example videos with the corresponding
    ground truth (GT) masks, predictions from BVS~\cite{marki2016bilateral},
    OFL~\cite{tsaivideo}, VPN (VPN-Stage2) and VPN-DLab (VPN-DeepLab) models.}
    \label{fig:video_seg_neg_supp}
\end{center}
\vspace{-1.0cm}
\end{figure}

\begin{figure}[th!]
\begin{center}
  \centerline{\includegraphics[width=0.9\textwidth]{figures/supp_semantic_visual.pdf}}
    \mycaption{Semantic Video Segmentation}
    {Input video frames and the corresponding ground truth (GT)
    segmentation together with the predictions of CNN~\cite{yu2015multi} and with
    VPN-Flow.}
    \label{fig:semantic_visuals_supp}
\end{center}
\vspace{-0.7cm}
\end{figure}

\begin{figure}[th!]
\begin{center}
  \centerline{\includegraphics[width=\textwidth]{figures/colorization_visuals_supp.pdf}}
  \mycaption{Video Color Propagation}
  {Input grayscale video frames and corresponding ground-truth (GT) color images
  together with color predictions of Levin et al.~\cite{levin2004colorization} and VPN-Stage1 models.}
  \label{fig:color_visuals_supp}
\end{center}
\vspace{-0.7cm}
\end{figure}

\clearpage

\section{Additional Material for Bilateral Inception Networks}
\label{sec:binception-app}

In this section of the Appendix, we first discuss the use of approximate bilateral
filtering in BI modules (Sec.~\ref{sec:lattice}).
Later, we present some qualitative results using different models for the approach presented in
Chapter~\ref{chap:binception} (Sec.~\ref{sec:qualitative-app}).

\subsection{Approximate Bilateral Filtering}
\label{sec:lattice}

The bilateral inception module presented in Chapter~\ref{chap:binception} computes a matrix-vector
product between a Gaussian filter $K$ and a vector of activations $\bz_c$.
Bilateral filtering is an important operation and many algorithmic techniques have been
proposed to speed-up this operation~\cite{paris2006fast,adams2010fast,gastal2011domain}.
In the main paper we opted to implement what can be considered the
brute-force variant of explicitly constructing $K$ and then using BLAS to compute the
matrix-vector product. This resulted in a few millisecond operation.
The explicit way to compute is possible due to the
reduction to super-pixels, e.g., it would not work for DenseCRF variants
that operate on the full image resolution.

Here, we present experiments where we use the fast approximate bilateral filtering
algorithm of~\cite{adams2010fast}, which is also used in Chapter~\ref{chap:bnn}
for learning sparse high dimensional filters. This
choice allows for larger dimensions of matrix-vector multiplication. The reason for choosing
the explicit multiplication in Chapter~\ref{chap:binception} was that it was computationally faster.
For the small sizes of the involved matrices and vectors, the explicit computation is sufficient and we had no
GPU implementation of an approximate technique that matched this runtime. Also it
is conceptually easier and the gradient to the feature transformations ($\Lambda \mathbf{f}$) is
obtained using standard matrix calculus.

\subsubsection{Experiments}

We modified the existing segmentation architectures analogous to those in Chapter~\ref{chap:binception}.
The main difference is that, here, the inception modules use the lattice
approximation~\cite{adams2010fast} to compute the bilateral filtering.
Using the lattice approximation did not allow us to back-propagate through feature transformations ($\Lambda$)
and thus we used hand-specified feature scales as will be explained later.
Specifically, we take CNN architectures from the works
of~\cite{chen2014semantic,zheng2015conditional,bell2015minc} and insert the BI modules between
the spatial FC layers.
We use superpixels from~\cite{DollarICCV13edges}
for all the experiments with the lattice approximation. Experiments are
performed using Caffe neural network framework~\cite{jia2014caffe}.

\begin{table}
  \small
  \centering
  \begin{tabular}{p{5.5cm}>{\raggedright\arraybackslash}p{1.4cm}>{\centering\arraybackslash}p{2.2cm}}
    \toprule
		\textbf{Model} & \emph{IoU} & \emph{Runtime}(ms) \\
    \midrule

    %%%%%%%%%%%% Scores computed by us)%%%%%%%%%%%%
		\deeplablargefov & 68.9 & 145ms\\
    \midrule
    \bi{7}{2}-\bi{8}{10}& \textbf{73.8} & +600 \\
    \midrule
    \deeplablargefovcrf~\cite{chen2014semantic} & 72.7 & +830\\
    \deeplabmsclargefovcrf~\cite{chen2014semantic} & \textbf{73.6} & +880\\
    DeepLab-EdgeNet~\cite{chen2015semantic} & 71.7 & +30\\
    DeepLab-EdgeNet-CRF~\cite{chen2015semantic} & \textbf{73.6} & +860\\
  \bottomrule \\
  \end{tabular}
  \mycaption{Semantic Segmentation using the DeepLab model}
  {IoU scores on the Pascal VOC12 segmentation test dataset
  with different models and our modified inception model.
  Also shown are the corresponding runtimes in milliseconds. Runtimes
  also include superpixel computations (300 ms with Dollar superpixels~\cite{DollarICCV13edges})}
  \label{tab:largefovresults}
\end{table}

\paragraph{Semantic Segmentation}
The experiments in this section use the Pascal VOC12 segmentation dataset~\cite{voc2012segmentation} with 21 object classes and the images have a maximum resolution of 0.25 megapixels.
For all experiments on VOC12, we train using the extended training set of
10581 images collected by~\cite{hariharan2011moredata}.
We modified the \deeplab~network architecture of~\cite{chen2014semantic} and
the CRFasRNN architecture from~\cite{zheng2015conditional} which uses a CNN with
deconvolution layers followed by DenseCRF trained end-to-end.

\paragraph{DeepLab Model}\label{sec:deeplabmodel}
We experimented with the \bi{7}{2}-\bi{8}{10} inception model.
Results using the~\deeplab~model are summarized in Tab.~\ref{tab:largefovresults}.
Although we get similar improvements with inception modules as with the
explicit kernel computation, using lattice approximation is slower.

\begin{table}
  \small
  \centering
  \begin{tabular}{p{6.4cm}>{\raggedright\arraybackslash}p{1.8cm}>{\raggedright\arraybackslash}p{1.8cm}}
    \toprule
    \textbf{Model} & \emph{IoU (Val)} & \emph{IoU (Test)}\\
    \midrule
    %%%%%%%%%%%% Scores computed by us)%%%%%%%%%%%%
    CNN &  67.5 & - \\
    \deconv (CNN+Deconvolutions) & 69.8 & 72.0 \\
    \midrule
    \bi{3}{6}-\bi{4}{6}-\bi{7}{2}-\bi{8}{6}& 71.9 & - \\
    \bi{3}{6}-\bi{4}{6}-\bi{7}{2}-\bi{8}{6}-\gi{6}& 73.6 &  \href{http://host.robots.ox.ac.uk:8080/anonymous/VOTV5E.html}{\textbf{75.2}}\\
    \midrule
    \deconvcrf (CRF-RNN)~\cite{zheng2015conditional} & 73.0 & 74.7\\
    Context-CRF-RNN~\cite{yu2015multi} & ~~ - ~ & \textbf{75.3} \\
    \bottomrule \\
  \end{tabular}
  \mycaption{Semantic Segmentation using the CRFasRNN model}{IoU score corresponding to different models
  on Pascal VOC12 reduced validation / test segmentation dataset. The reduced validation set consists of 346 images
  as used in~\cite{zheng2015conditional} where we adapted the model from.}
  \label{tab:deconvresults-app}
\end{table}

\paragraph{CRFasRNN Model}\label{sec:deepinception}
We add BI modules after score-pool3, score-pool4, \fc{7} and \fc{8} $1\times1$ convolution layers
resulting in the \bi{3}{6}-\bi{4}{6}-\bi{7}{2}-\bi{8}{6}
model and also experimented with another variant where $BI_8$ is followed by another inception
module, G$(6)$, with 6 Gaussian kernels.
Note that here also we discarded both deconvolution and DenseCRF parts of the original model~\cite{zheng2015conditional}
and inserted the BI modules in the base CNN and found similar improvements compared to the inception modules with explicit
kernel computaion. See Tab.~\ref{tab:deconvresults-app} for results on the CRFasRNN model.

\paragraph{Material Segmentation}
Table~\ref{tab:mincresults-app} shows the results on the MINC dataset~\cite{bell2015minc}
obtained by modifying the AlexNet architecture with our inception modules. We observe
similar improvements as with explicit kernel construction.
For this model, we do not provide any learned setup due to very limited segment training
data. The weights to combine outputs in the bilateral inception layer are
found by validation on the validation set.

\begin{table}[t]
  \small
  \centering
  \begin{tabular}{p{3.5cm}>{\centering\arraybackslash}p{4.0cm}}
    \toprule
    \textbf{Model} & Class / Total accuracy\\
    \midrule

    %%%%%%%%%%%% Scores computed by us)%%%%%%%%%%%%
    AlexNet CNN & 55.3 / 58.9 \\
    \midrule
    \bi{7}{2}-\bi{8}{6}& 68.5 / 71.8 \\
    \bi{7}{2}-\bi{8}{6}-G$(6)$& 67.6 / 73.1 \\
    \midrule
    AlexNet-CRF & 65.5 / 71.0 \\
    \bottomrule \\
  \end{tabular}
  \mycaption{Material Segmentation using AlexNet}{Pixel accuracy of different models on
  the MINC material segmentation test dataset~\cite{bell2015minc}.}
  \label{tab:mincresults-app}
\end{table}

\paragraph{Scales of Bilateral Inception Modules}
\label{sec:scales}

Unlike the explicit kernel technique presented in the main text (Chapter~\ref{chap:binception}),
we didn't back-propagate through feature transformation ($\Lambda$)
using the approximate bilateral filter technique.
So, the feature scales are hand-specified and validated, which are as follows.
The optimal scale values for the \bi{7}{2}-\bi{8}{2} model are found by validation for the best performance which are
$\sigma_{xy}$ = (0.1, 0.1) for the spatial (XY) kernel and $\sigma_{rgbxy}$ = (0.1, 0.1, 0.1, 0.01, 0.01) for color and position (RGBXY)  kernel.
Next, as more kernels are added to \bi{8}{2}, we set scales to be $\alpha$*($\sigma_{xy}$, $\sigma_{rgbxy}$).
The value of $\alpha$ is chosen as  1, 0.5, 0.1, 0.05, 0.1, at uniform interval, for the \bi{8}{10} bilateral inception module.


\subsection{Qualitative Results}
\label{sec:qualitative-app}

In this section, we present more qualitative results obtained using the BI module with explicit
kernel computation technique presented in Chapter~\ref{chap:binception}. Results on the Pascal VOC12
dataset~\cite{voc2012segmentation} using the DeepLab-LargeFOV model are shown in Fig.~\ref{fig:semantic_visuals-app},
followed by the results on MINC dataset~\cite{bell2015minc}
in Fig.~\ref{fig:material_visuals-app} and on
Cityscapes dataset~\cite{Cordts2015Cvprw} in Fig.~\ref{fig:street_visuals-app}.


\definecolor{voc_1}{RGB}{0, 0, 0}
\definecolor{voc_2}{RGB}{128, 0, 0}
\definecolor{voc_3}{RGB}{0, 128, 0}
\definecolor{voc_4}{RGB}{128, 128, 0}
\definecolor{voc_5}{RGB}{0, 0, 128}
\definecolor{voc_6}{RGB}{128, 0, 128}
\definecolor{voc_7}{RGB}{0, 128, 128}
\definecolor{voc_8}{RGB}{128, 128, 128}
\definecolor{voc_9}{RGB}{64, 0, 0}
\definecolor{voc_10}{RGB}{192, 0, 0}
\definecolor{voc_11}{RGB}{64, 128, 0}
\definecolor{voc_12}{RGB}{192, 128, 0}
\definecolor{voc_13}{RGB}{64, 0, 128}
\definecolor{voc_14}{RGB}{192, 0, 128}
\definecolor{voc_15}{RGB}{64, 128, 128}
\definecolor{voc_16}{RGB}{192, 128, 128}
\definecolor{voc_17}{RGB}{0, 64, 0}
\definecolor{voc_18}{RGB}{128, 64, 0}
\definecolor{voc_19}{RGB}{0, 192, 0}
\definecolor{voc_20}{RGB}{128, 192, 0}
\definecolor{voc_21}{RGB}{0, 64, 128}
\definecolor{voc_22}{RGB}{128, 64, 128}

\begin{figure*}[!ht]
  \small
  \centering
  \fcolorbox{white}{voc_1}{\rule{0pt}{4pt}\rule{4pt}{0pt}} Background~~
  \fcolorbox{white}{voc_2}{\rule{0pt}{4pt}\rule{4pt}{0pt}} Aeroplane~~
  \fcolorbox{white}{voc_3}{\rule{0pt}{4pt}\rule{4pt}{0pt}} Bicycle~~
  \fcolorbox{white}{voc_4}{\rule{0pt}{4pt}\rule{4pt}{0pt}} Bird~~
  \fcolorbox{white}{voc_5}{\rule{0pt}{4pt}\rule{4pt}{0pt}} Boat~~
  \fcolorbox{white}{voc_6}{\rule{0pt}{4pt}\rule{4pt}{0pt}} Bottle~~
  \fcolorbox{white}{voc_7}{\rule{0pt}{4pt}\rule{4pt}{0pt}} Bus~~
  \fcolorbox{white}{voc_8}{\rule{0pt}{4pt}\rule{4pt}{0pt}} Car~~\\
  \fcolorbox{white}{voc_9}{\rule{0pt}{4pt}\rule{4pt}{0pt}} Cat~~
  \fcolorbox{white}{voc_10}{\rule{0pt}{4pt}\rule{4pt}{0pt}} Chair~~
  \fcolorbox{white}{voc_11}{\rule{0pt}{4pt}\rule{4pt}{0pt}} Cow~~
  \fcolorbox{white}{voc_12}{\rule{0pt}{4pt}\rule{4pt}{0pt}} Dining Table~~
  \fcolorbox{white}{voc_13}{\rule{0pt}{4pt}\rule{4pt}{0pt}} Dog~~
  \fcolorbox{white}{voc_14}{\rule{0pt}{4pt}\rule{4pt}{0pt}} Horse~~
  \fcolorbox{white}{voc_15}{\rule{0pt}{4pt}\rule{4pt}{0pt}} Motorbike~~
  \fcolorbox{white}{voc_16}{\rule{0pt}{4pt}\rule{4pt}{0pt}} Person~~\\
  \fcolorbox{white}{voc_17}{\rule{0pt}{4pt}\rule{4pt}{0pt}} Potted Plant~~
  \fcolorbox{white}{voc_18}{\rule{0pt}{4pt}\rule{4pt}{0pt}} Sheep~~
  \fcolorbox{white}{voc_19}{\rule{0pt}{4pt}\rule{4pt}{0pt}} Sofa~~
  \fcolorbox{white}{voc_20}{\rule{0pt}{4pt}\rule{4pt}{0pt}} Train~~
  \fcolorbox{white}{voc_21}{\rule{0pt}{4pt}\rule{4pt}{0pt}} TV monitor~~\\


  \subfigure{%
    \includegraphics[width=.15\columnwidth]{figures/supplementary/2008_001308_given.png}
  }
  \subfigure{%
    \includegraphics[width=.15\columnwidth]{figures/supplementary/2008_001308_sp.png}
  }
  \subfigure{%
    \includegraphics[width=.15\columnwidth]{figures/supplementary/2008_001308_gt.png}
  }
  \subfigure{%
    \includegraphics[width=.15\columnwidth]{figures/supplementary/2008_001308_cnn.png}
  }
  \subfigure{%
    \includegraphics[width=.15\columnwidth]{figures/supplementary/2008_001308_crf.png}
  }
  \subfigure{%
    \includegraphics[width=.15\columnwidth]{figures/supplementary/2008_001308_ours.png}
  }\\[-2ex]


  \subfigure{%
    \includegraphics[width=.15\columnwidth]{figures/supplementary/2008_001821_given.png}
  }
  \subfigure{%
    \includegraphics[width=.15\columnwidth]{figures/supplementary/2008_001821_sp.png}
  }
  \subfigure{%
    \includegraphics[width=.15\columnwidth]{figures/supplementary/2008_001821_gt.png}
  }
  \subfigure{%
    \includegraphics[width=.15\columnwidth]{figures/supplementary/2008_001821_cnn.png}
  }
  \subfigure{%
    \includegraphics[width=.15\columnwidth]{figures/supplementary/2008_001821_crf.png}
  }
  \subfigure{%
    \includegraphics[width=.15\columnwidth]{figures/supplementary/2008_001821_ours.png}
  }\\[-2ex]



  \subfigure{%
    \includegraphics[width=.15\columnwidth]{figures/supplementary/2008_004612_given.png}
  }
  \subfigure{%
    \includegraphics[width=.15\columnwidth]{figures/supplementary/2008_004612_sp.png}
  }
  \subfigure{%
    \includegraphics[width=.15\columnwidth]{figures/supplementary/2008_004612_gt.png}
  }
  \subfigure{%
    \includegraphics[width=.15\columnwidth]{figures/supplementary/2008_004612_cnn.png}
  }
  \subfigure{%
    \includegraphics[width=.15\columnwidth]{figures/supplementary/2008_004612_crf.png}
  }
  \subfigure{%
    \includegraphics[width=.15\columnwidth]{figures/supplementary/2008_004612_ours.png}
  }\\[-2ex]


  \subfigure{%
    \includegraphics[width=.15\columnwidth]{figures/supplementary/2009_001008_given.png}
  }
  \subfigure{%
    \includegraphics[width=.15\columnwidth]{figures/supplementary/2009_001008_sp.png}
  }
  \subfigure{%
    \includegraphics[width=.15\columnwidth]{figures/supplementary/2009_001008_gt.png}
  }
  \subfigure{%
    \includegraphics[width=.15\columnwidth]{figures/supplementary/2009_001008_cnn.png}
  }
  \subfigure{%
    \includegraphics[width=.15\columnwidth]{figures/supplementary/2009_001008_crf.png}
  }
  \subfigure{%
    \includegraphics[width=.15\columnwidth]{figures/supplementary/2009_001008_ours.png}
  }\\[-2ex]




  \subfigure{%
    \includegraphics[width=.15\columnwidth]{figures/supplementary/2009_004497_given.png}
  }
  \subfigure{%
    \includegraphics[width=.15\columnwidth]{figures/supplementary/2009_004497_sp.png}
  }
  \subfigure{%
    \includegraphics[width=.15\columnwidth]{figures/supplementary/2009_004497_gt.png}
  }
  \subfigure{%
    \includegraphics[width=.15\columnwidth]{figures/supplementary/2009_004497_cnn.png}
  }
  \subfigure{%
    \includegraphics[width=.15\columnwidth]{figures/supplementary/2009_004497_crf.png}
  }
  \subfigure{%
    \includegraphics[width=.15\columnwidth]{figures/supplementary/2009_004497_ours.png}
  }\\[-2ex]



  \setcounter{subfigure}{0}
  \subfigure[\scriptsize Input]{%
    \includegraphics[width=.15\columnwidth]{figures/supplementary/2010_001327_given.png}
  }
  \subfigure[\scriptsize Superpixels]{%
    \includegraphics[width=.15\columnwidth]{figures/supplementary/2010_001327_sp.png}
  }
  \subfigure[\scriptsize GT]{%
    \includegraphics[width=.15\columnwidth]{figures/supplementary/2010_001327_gt.png}
  }
  \subfigure[\scriptsize Deeplab]{%
    \includegraphics[width=.15\columnwidth]{figures/supplementary/2010_001327_cnn.png}
  }
  \subfigure[\scriptsize +DenseCRF]{%
    \includegraphics[width=.15\columnwidth]{figures/supplementary/2010_001327_crf.png}
  }
  \subfigure[\scriptsize Using BI]{%
    \includegraphics[width=.15\columnwidth]{figures/supplementary/2010_001327_ours.png}
  }
  \mycaption{Semantic Segmentation}{Example results of semantic segmentation
  on the Pascal VOC12 dataset.
  (d)~depicts the DeepLab CNN result, (e)~CNN + 10 steps of mean-field inference,
  (f~result obtained with bilateral inception (BI) modules (\bi{6}{2}+\bi{7}{6}) between \fc~layers.}
  \label{fig:semantic_visuals-app}
\end{figure*}


\definecolor{minc_1}{HTML}{771111}
\definecolor{minc_2}{HTML}{CAC690}
\definecolor{minc_3}{HTML}{EEEEEE}
\definecolor{minc_4}{HTML}{7C8FA6}
\definecolor{minc_5}{HTML}{597D31}
\definecolor{minc_6}{HTML}{104410}
\definecolor{minc_7}{HTML}{BB819C}
\definecolor{minc_8}{HTML}{D0CE48}
\definecolor{minc_9}{HTML}{622745}
\definecolor{minc_10}{HTML}{666666}
\definecolor{minc_11}{HTML}{D54A31}
\definecolor{minc_12}{HTML}{101044}
\definecolor{minc_13}{HTML}{444126}
\definecolor{minc_14}{HTML}{75D646}
\definecolor{minc_15}{HTML}{DD4348}
\definecolor{minc_16}{HTML}{5C8577}
\definecolor{minc_17}{HTML}{C78472}
\definecolor{minc_18}{HTML}{75D6D0}
\definecolor{minc_19}{HTML}{5B4586}
\definecolor{minc_20}{HTML}{C04393}
\definecolor{minc_21}{HTML}{D69948}
\definecolor{minc_22}{HTML}{7370D8}
\definecolor{minc_23}{HTML}{7A3622}
\definecolor{minc_24}{HTML}{000000}

\begin{figure*}[!ht]
  \small % scriptsize
  \centering
  \fcolorbox{white}{minc_1}{\rule{0pt}{4pt}\rule{4pt}{0pt}} Brick~~
  \fcolorbox{white}{minc_2}{\rule{0pt}{4pt}\rule{4pt}{0pt}} Carpet~~
  \fcolorbox{white}{minc_3}{\rule{0pt}{4pt}\rule{4pt}{0pt}} Ceramic~~
  \fcolorbox{white}{minc_4}{\rule{0pt}{4pt}\rule{4pt}{0pt}} Fabric~~
  \fcolorbox{white}{minc_5}{\rule{0pt}{4pt}\rule{4pt}{0pt}} Foliage~~
  \fcolorbox{white}{minc_6}{\rule{0pt}{4pt}\rule{4pt}{0pt}} Food~~
  \fcolorbox{white}{minc_7}{\rule{0pt}{4pt}\rule{4pt}{0pt}} Glass~~
  \fcolorbox{white}{minc_8}{\rule{0pt}{4pt}\rule{4pt}{0pt}} Hair~~\\
  \fcolorbox{white}{minc_9}{\rule{0pt}{4pt}\rule{4pt}{0pt}} Leather~~
  \fcolorbox{white}{minc_10}{\rule{0pt}{4pt}\rule{4pt}{0pt}} Metal~~
  \fcolorbox{white}{minc_11}{\rule{0pt}{4pt}\rule{4pt}{0pt}} Mirror~~
  \fcolorbox{white}{minc_12}{\rule{0pt}{4pt}\rule{4pt}{0pt}} Other~~
  \fcolorbox{white}{minc_13}{\rule{0pt}{4pt}\rule{4pt}{0pt}} Painted~~
  \fcolorbox{white}{minc_14}{\rule{0pt}{4pt}\rule{4pt}{0pt}} Paper~~
  \fcolorbox{white}{minc_15}{\rule{0pt}{4pt}\rule{4pt}{0pt}} Plastic~~\\
  \fcolorbox{white}{minc_16}{\rule{0pt}{4pt}\rule{4pt}{0pt}} Polished Stone~~
  \fcolorbox{white}{minc_17}{\rule{0pt}{4pt}\rule{4pt}{0pt}} Skin~~
  \fcolorbox{white}{minc_18}{\rule{0pt}{4pt}\rule{4pt}{0pt}} Sky~~
  \fcolorbox{white}{minc_19}{\rule{0pt}{4pt}\rule{4pt}{0pt}} Stone~~
  \fcolorbox{white}{minc_20}{\rule{0pt}{4pt}\rule{4pt}{0pt}} Tile~~
  \fcolorbox{white}{minc_21}{\rule{0pt}{4pt}\rule{4pt}{0pt}} Wallpaper~~
  \fcolorbox{white}{minc_22}{\rule{0pt}{4pt}\rule{4pt}{0pt}} Water~~
  \fcolorbox{white}{minc_23}{\rule{0pt}{4pt}\rule{4pt}{0pt}} Wood~~\\
  \subfigure{%
    \includegraphics[width=.15\columnwidth]{figures/supplementary/000008468_given.png}
  }
  \subfigure{%
    \includegraphics[width=.15\columnwidth]{figures/supplementary/000008468_sp.png}
  }
  \subfigure{%
    \includegraphics[width=.15\columnwidth]{figures/supplementary/000008468_gt.png}
  }
  \subfigure{%
    \includegraphics[width=.15\columnwidth]{figures/supplementary/000008468_cnn.png}
  }
  \subfigure{%
    \includegraphics[width=.15\columnwidth]{figures/supplementary/000008468_crf.png}
  }
  \subfigure{%
    \includegraphics[width=.15\columnwidth]{figures/supplementary/000008468_ours.png}
  }\\[-2ex]

  \subfigure{%
    \includegraphics[width=.15\columnwidth]{figures/supplementary/000009053_given.png}
  }
  \subfigure{%
    \includegraphics[width=.15\columnwidth]{figures/supplementary/000009053_sp.png}
  }
  \subfigure{%
    \includegraphics[width=.15\columnwidth]{figures/supplementary/000009053_gt.png}
  }
  \subfigure{%
    \includegraphics[width=.15\columnwidth]{figures/supplementary/000009053_cnn.png}
  }
  \subfigure{%
    \includegraphics[width=.15\columnwidth]{figures/supplementary/000009053_crf.png}
  }
  \subfigure{%
    \includegraphics[width=.15\columnwidth]{figures/supplementary/000009053_ours.png}
  }\\[-2ex]




  \subfigure{%
    \includegraphics[width=.15\columnwidth]{figures/supplementary/000014977_given.png}
  }
  \subfigure{%
    \includegraphics[width=.15\columnwidth]{figures/supplementary/000014977_sp.png}
  }
  \subfigure{%
    \includegraphics[width=.15\columnwidth]{figures/supplementary/000014977_gt.png}
  }
  \subfigure{%
    \includegraphics[width=.15\columnwidth]{figures/supplementary/000014977_cnn.png}
  }
  \subfigure{%
    \includegraphics[width=.15\columnwidth]{figures/supplementary/000014977_crf.png}
  }
  \subfigure{%
    \includegraphics[width=.15\columnwidth]{figures/supplementary/000014977_ours.png}
  }\\[-2ex]


  \subfigure{%
    \includegraphics[width=.15\columnwidth]{figures/supplementary/000022922_given.png}
  }
  \subfigure{%
    \includegraphics[width=.15\columnwidth]{figures/supplementary/000022922_sp.png}
  }
  \subfigure{%
    \includegraphics[width=.15\columnwidth]{figures/supplementary/000022922_gt.png}
  }
  \subfigure{%
    \includegraphics[width=.15\columnwidth]{figures/supplementary/000022922_cnn.png}
  }
  \subfigure{%
    \includegraphics[width=.15\columnwidth]{figures/supplementary/000022922_crf.png}
  }
  \subfigure{%
    \includegraphics[width=.15\columnwidth]{figures/supplementary/000022922_ours.png}
  }\\[-2ex]


  \subfigure{%
    \includegraphics[width=.15\columnwidth]{figures/supplementary/000025711_given.png}
  }
  \subfigure{%
    \includegraphics[width=.15\columnwidth]{figures/supplementary/000025711_sp.png}
  }
  \subfigure{%
    \includegraphics[width=.15\columnwidth]{figures/supplementary/000025711_gt.png}
  }
  \subfigure{%
    \includegraphics[width=.15\columnwidth]{figures/supplementary/000025711_cnn.png}
  }
  \subfigure{%
    \includegraphics[width=.15\columnwidth]{figures/supplementary/000025711_crf.png}
  }
  \subfigure{%
    \includegraphics[width=.15\columnwidth]{figures/supplementary/000025711_ours.png}
  }\\[-2ex]


  \subfigure{%
    \includegraphics[width=.15\columnwidth]{figures/supplementary/000034473_given.png}
  }
  \subfigure{%
    \includegraphics[width=.15\columnwidth]{figures/supplementary/000034473_sp.png}
  }
  \subfigure{%
    \includegraphics[width=.15\columnwidth]{figures/supplementary/000034473_gt.png}
  }
  \subfigure{%
    \includegraphics[width=.15\columnwidth]{figures/supplementary/000034473_cnn.png}
  }
  \subfigure{%
    \includegraphics[width=.15\columnwidth]{figures/supplementary/000034473_crf.png}
  }
  \subfigure{%
    \includegraphics[width=.15\columnwidth]{figures/supplementary/000034473_ours.png}
  }\\[-2ex]


  \subfigure{%
    \includegraphics[width=.15\columnwidth]{figures/supplementary/000035463_given.png}
  }
  \subfigure{%
    \includegraphics[width=.15\columnwidth]{figures/supplementary/000035463_sp.png}
  }
  \subfigure{%
    \includegraphics[width=.15\columnwidth]{figures/supplementary/000035463_gt.png}
  }
  \subfigure{%
    \includegraphics[width=.15\columnwidth]{figures/supplementary/000035463_cnn.png}
  }
  \subfigure{%
    \includegraphics[width=.15\columnwidth]{figures/supplementary/000035463_crf.png}
  }
  \subfigure{%
    \includegraphics[width=.15\columnwidth]{figures/supplementary/000035463_ours.png}
  }\\[-2ex]


  \setcounter{subfigure}{0}
  \subfigure[\scriptsize Input]{%
    \includegraphics[width=.15\columnwidth]{figures/supplementary/000035993_given.png}
  }
  \subfigure[\scriptsize Superpixels]{%
    \includegraphics[width=.15\columnwidth]{figures/supplementary/000035993_sp.png}
  }
  \subfigure[\scriptsize GT]{%
    \includegraphics[width=.15\columnwidth]{figures/supplementary/000035993_gt.png}
  }
  \subfigure[\scriptsize AlexNet]{%
    \includegraphics[width=.15\columnwidth]{figures/supplementary/000035993_cnn.png}
  }
  \subfigure[\scriptsize +DenseCRF]{%
    \includegraphics[width=.15\columnwidth]{figures/supplementary/000035993_crf.png}
  }
  \subfigure[\scriptsize Using BI]{%
    \includegraphics[width=.15\columnwidth]{figures/supplementary/000035993_ours.png}
  }
  \mycaption{Material Segmentation}{Example results of material segmentation.
  (d)~depicts the AlexNet CNN result, (e)~CNN + 10 steps of mean-field inference,
  (f)~result obtained with bilateral inception (BI) modules (\bi{7}{2}+\bi{8}{6}) between
  \fc~layers.}
\label{fig:material_visuals-app}
\end{figure*}


\definecolor{city_1}{RGB}{128, 64, 128}
\definecolor{city_2}{RGB}{244, 35, 232}
\definecolor{city_3}{RGB}{70, 70, 70}
\definecolor{city_4}{RGB}{102, 102, 156}
\definecolor{city_5}{RGB}{190, 153, 153}
\definecolor{city_6}{RGB}{153, 153, 153}
\definecolor{city_7}{RGB}{250, 170, 30}
\definecolor{city_8}{RGB}{220, 220, 0}
\definecolor{city_9}{RGB}{107, 142, 35}
\definecolor{city_10}{RGB}{152, 251, 152}
\definecolor{city_11}{RGB}{70, 130, 180}
\definecolor{city_12}{RGB}{220, 20, 60}
\definecolor{city_13}{RGB}{255, 0, 0}
\definecolor{city_14}{RGB}{0, 0, 142}
\definecolor{city_15}{RGB}{0, 0, 70}
\definecolor{city_16}{RGB}{0, 60, 100}
\definecolor{city_17}{RGB}{0, 80, 100}
\definecolor{city_18}{RGB}{0, 0, 230}
\definecolor{city_19}{RGB}{119, 11, 32}
\begin{figure*}[!ht]
  \small % scriptsize
  \centering


  \subfigure{%
    \includegraphics[width=.18\columnwidth]{figures/supplementary/frankfurt00000_016005_given.png}
  }
  \subfigure{%
    \includegraphics[width=.18\columnwidth]{figures/supplementary/frankfurt00000_016005_sp.png}
  }
  \subfigure{%
    \includegraphics[width=.18\columnwidth]{figures/supplementary/frankfurt00000_016005_gt.png}
  }
  \subfigure{%
    \includegraphics[width=.18\columnwidth]{figures/supplementary/frankfurt00000_016005_cnn.png}
  }
  \subfigure{%
    \includegraphics[width=.18\columnwidth]{figures/supplementary/frankfurt00000_016005_ours.png}
  }\\[-2ex]

  \subfigure{%
    \includegraphics[width=.18\columnwidth]{figures/supplementary/frankfurt00000_004617_given.png}
  }
  \subfigure{%
    \includegraphics[width=.18\columnwidth]{figures/supplementary/frankfurt00000_004617_sp.png}
  }
  \subfigure{%
    \includegraphics[width=.18\columnwidth]{figures/supplementary/frankfurt00000_004617_gt.png}
  }
  \subfigure{%
    \includegraphics[width=.18\columnwidth]{figures/supplementary/frankfurt00000_004617_cnn.png}
  }
  \subfigure{%
    \includegraphics[width=.18\columnwidth]{figures/supplementary/frankfurt00000_004617_ours.png}
  }\\[-2ex]

  \subfigure{%
    \includegraphics[width=.18\columnwidth]{figures/supplementary/frankfurt00000_020880_given.png}
  }
  \subfigure{%
    \includegraphics[width=.18\columnwidth]{figures/supplementary/frankfurt00000_020880_sp.png}
  }
  \subfigure{%
    \includegraphics[width=.18\columnwidth]{figures/supplementary/frankfurt00000_020880_gt.png}
  }
  \subfigure{%
    \includegraphics[width=.18\columnwidth]{figures/supplementary/frankfurt00000_020880_cnn.png}
  }
  \subfigure{%
    \includegraphics[width=.18\columnwidth]{figures/supplementary/frankfurt00000_020880_ours.png}
  }\\[-2ex]



  \subfigure{%
    \includegraphics[width=.18\columnwidth]{figures/supplementary/frankfurt00001_007285_given.png}
  }
  \subfigure{%
    \includegraphics[width=.18\columnwidth]{figures/supplementary/frankfurt00001_007285_sp.png}
  }
  \subfigure{%
    \includegraphics[width=.18\columnwidth]{figures/supplementary/frankfurt00001_007285_gt.png}
  }
  \subfigure{%
    \includegraphics[width=.18\columnwidth]{figures/supplementary/frankfurt00001_007285_cnn.png}
  }
  \subfigure{%
    \includegraphics[width=.18\columnwidth]{figures/supplementary/frankfurt00001_007285_ours.png}
  }\\[-2ex]


  \subfigure{%
    \includegraphics[width=.18\columnwidth]{figures/supplementary/frankfurt00001_059789_given.png}
  }
  \subfigure{%
    \includegraphics[width=.18\columnwidth]{figures/supplementary/frankfurt00001_059789_sp.png}
  }
  \subfigure{%
    \includegraphics[width=.18\columnwidth]{figures/supplementary/frankfurt00001_059789_gt.png}
  }
  \subfigure{%
    \includegraphics[width=.18\columnwidth]{figures/supplementary/frankfurt00001_059789_cnn.png}
  }
  \subfigure{%
    \includegraphics[width=.18\columnwidth]{figures/supplementary/frankfurt00001_059789_ours.png}
  }\\[-2ex]


  \subfigure{%
    \includegraphics[width=.18\columnwidth]{figures/supplementary/frankfurt00001_068208_given.png}
  }
  \subfigure{%
    \includegraphics[width=.18\columnwidth]{figures/supplementary/frankfurt00001_068208_sp.png}
  }
  \subfigure{%
    \includegraphics[width=.18\columnwidth]{figures/supplementary/frankfurt00001_068208_gt.png}
  }
  \subfigure{%
    \includegraphics[width=.18\columnwidth]{figures/supplementary/frankfurt00001_068208_cnn.png}
  }
  \subfigure{%
    \includegraphics[width=.18\columnwidth]{figures/supplementary/frankfurt00001_068208_ours.png}
  }\\[-2ex]

  \subfigure{%
    \includegraphics[width=.18\columnwidth]{figures/supplementary/frankfurt00001_082466_given.png}
  }
  \subfigure{%
    \includegraphics[width=.18\columnwidth]{figures/supplementary/frankfurt00001_082466_sp.png}
  }
  \subfigure{%
    \includegraphics[width=.18\columnwidth]{figures/supplementary/frankfurt00001_082466_gt.png}
  }
  \subfigure{%
    \includegraphics[width=.18\columnwidth]{figures/supplementary/frankfurt00001_082466_cnn.png}
  }
  \subfigure{%
    \includegraphics[width=.18\columnwidth]{figures/supplementary/frankfurt00001_082466_ours.png}
  }\\[-2ex]

  \subfigure{%
    \includegraphics[width=.18\columnwidth]{figures/supplementary/lindau00033_000019_given.png}
  }
  \subfigure{%
    \includegraphics[width=.18\columnwidth]{figures/supplementary/lindau00033_000019_sp.png}
  }
  \subfigure{%
    \includegraphics[width=.18\columnwidth]{figures/supplementary/lindau00033_000019_gt.png}
  }
  \subfigure{%
    \includegraphics[width=.18\columnwidth]{figures/supplementary/lindau00033_000019_cnn.png}
  }
  \subfigure{%
    \includegraphics[width=.18\columnwidth]{figures/supplementary/lindau00033_000019_ours.png}
  }\\[-2ex]

  \subfigure{%
    \includegraphics[width=.18\columnwidth]{figures/supplementary/lindau00052_000019_given.png}
  }
  \subfigure{%
    \includegraphics[width=.18\columnwidth]{figures/supplementary/lindau00052_000019_sp.png}
  }
  \subfigure{%
    \includegraphics[width=.18\columnwidth]{figures/supplementary/lindau00052_000019_gt.png}
  }
  \subfigure{%
    \includegraphics[width=.18\columnwidth]{figures/supplementary/lindau00052_000019_cnn.png}
  }
  \subfigure{%
    \includegraphics[width=.18\columnwidth]{figures/supplementary/lindau00052_000019_ours.png}
  }\\[-2ex]




  \subfigure{%
    \includegraphics[width=.18\columnwidth]{figures/supplementary/lindau00027_000019_given.png}
  }
  \subfigure{%
    \includegraphics[width=.18\columnwidth]{figures/supplementary/lindau00027_000019_sp.png}
  }
  \subfigure{%
    \includegraphics[width=.18\columnwidth]{figures/supplementary/lindau00027_000019_gt.png}
  }
  \subfigure{%
    \includegraphics[width=.18\columnwidth]{figures/supplementary/lindau00027_000019_cnn.png}
  }
  \subfigure{%
    \includegraphics[width=.18\columnwidth]{figures/supplementary/lindau00027_000019_ours.png}
  }\\[-2ex]



  \setcounter{subfigure}{0}
  \subfigure[\scriptsize Input]{%
    \includegraphics[width=.18\columnwidth]{figures/supplementary/lindau00029_000019_given.png}
  }
  \subfigure[\scriptsize Superpixels]{%
    \includegraphics[width=.18\columnwidth]{figures/supplementary/lindau00029_000019_sp.png}
  }
  \subfigure[\scriptsize GT]{%
    \includegraphics[width=.18\columnwidth]{figures/supplementary/lindau00029_000019_gt.png}
  }
  \subfigure[\scriptsize Deeplab]{%
    \includegraphics[width=.18\columnwidth]{figures/supplementary/lindau00029_000019_cnn.png}
  }
  \subfigure[\scriptsize Using BI]{%
    \includegraphics[width=.18\columnwidth]{figures/supplementary/lindau00029_000019_ours.png}
  }%\\[-2ex]

  \mycaption{Street Scene Segmentation}{Example results of street scene segmentation.
  (d)~depicts the DeepLab results, (e)~result obtained by adding bilateral inception (BI) modules (\bi{6}{2}+\bi{7}{6}) between \fc~layers.}
\label{fig:street_visuals-app}
\end{figure*}


\end{document}

