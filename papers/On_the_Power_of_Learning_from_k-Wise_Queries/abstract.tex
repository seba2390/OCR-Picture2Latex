\abstract{
Several well-studied models of access to data samples, including statistical queries, local differential privacy and low-communication algorithms rely on queries that provide information about a function of a single sample. (For example, a statistical query (SQ) gives an estimate of $\E_{x\sim D}[q(x)]$ for any choice of the query function $q:X\rightarrow \R$, where $D$ is an unknown data distribution.) Yet some data analysis algorithms rely on properties of functions that depend on multiple samples. Such algorithms would be naturally implemented using $k$-wise queries each of which is specified by a function $q:X^k\rightarrow \R$. Hence it is natural to ask whether algorithms using $k$-wise queries can solve learning problems more efficiently and by how much.

Blum, Kalai, Wasserman~\cite{blum2003noise} showed that for any weak PAC learning problem over a fixed distribution, the complexity of learning with $k$-wise SQs is smaller than the (unary) SQ complexity by a factor of at most $2^k$. We show that for more general problems over distributions the picture is substantially richer. For every $k$, the complexity of distribution-independent PAC learning with $k$-wise queries can be exponentially larger than learning with $(k+1)$-wise queries. We then give two approaches for simulating a $k$-wise query using unary queries. The first approach exploits the structure of the problem that needs to be solved. It generalizes and strengthens (exponentially) the results of Blum \etal \cite{blum2003noise}. It allows us to derive strong lower bounds for learning DNF formulas and stochastic constraint satisfaction problems that hold against algorithms using $k$-wise queries. The second approach exploits the $k$-party communication complexity of the $k$-wise query function. 
}
