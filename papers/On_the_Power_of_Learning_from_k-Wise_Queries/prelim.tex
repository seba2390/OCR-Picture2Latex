\section{Preliminaries}
For any distribution $D$ over a domain $X$ and any positive integer $k$, we denote by $D^k$ the distribution over $X^k$ obtained by drawing $k$ i.i.d. samples from $D$. For a distribution $D$ over a domain $X$ and a function $\phi: X \to \mathbb{R}$, we denote $D[\phi] \doteq \Ex_{x \sim D}[\phi(x)]$.

%\subsection{Statistical Queries}

Next, we formally define the $k$-wise SQ oracle.

\begin{defn}\label{def:kw_sq}
Let $D$ be a distribution over a domain $X$ and $\tau > 0$. A $k$-wise statistical query oracle $\STAT^{(k)}_D(\tau)$ is an oracle that given as input any function $\phi:X^k \to [-1,+1]$, returns some value $v$ such that $|v-\Ex_{x \sim D^k}[\phi(x)]| \le \tau$.
\end{defn}

We say that a $k$-wise SQ algorithm is given access to $\STAT^{(k)}(\tau)$, if for every when the algorithm is given access to  $\STAT^{(k)}_D(\tau)$, where $D$ is the input distribution. We note that for $k = 1$, Definition~\ref{def:kw_sq} reduces to the usual definition of an SQ oracle that was first introduced by Kearns \cite{kearns1998efficient}. %The SQ complexity of a $k$-wise SQ algorithm is the maximum of the number of $k$-wise SQs made by the algorithm and the inverse of the minimum tolerance used.
The $k$-wise SQ complexity of solving a problem with access to $\STAT^{(k)}(\tau)$  is the minimum number of queries $q$ for which exists a $k$-wise SQ algorithm with access to $\STAT^{(k)}(\tau)$ that solves the problem using at most $q$ queries. Our discussion and results can also be easily extended to the stronger $\VSTAT$ oracle defined in \cite{FeldmanGRVX:12} and to more general real-valued queries using the reductions in \cite{Feldman:16sqvar}.

The PAC learning \cite{valiant1984theory} is defined as follows.
\begin{defn}
For a class $\calC$ of Boolean-valued functions over a domain $Z$, a PAC learning algorithm for $\calC$ is an algorithm that for every $P$ distribution over $Z$ and $f \in \calC$, given an error parameter $\epsilon > 0$, failure probability $\delta > 0$ and access to i.i.d.~labeled examples of the form $(x,f(x))$ where $x \sim P$, outputs a hypothesis function $h$ that, with probability at least $1-\delta$, satisfies $\Pr_{x \sim P}[h(x) \neq f(x)] \le \epsilon$.
\end{defn}

We next define one-vs-many decision problems, which will be used in the proofs in our Section~\ref{sec:pf_sep} and Section~\ref{sec:pf_flat}.
\begin{defn}[Decision problem $\calB(\calD, D_0)$]
Let $\calD$ be a set of distributions and $D_0$ a reference distribution over a set $X$. We denote by $\calB(\calD, D_0)$ the decision problem where we are given access to a distribution $D \in \calD \cup \{D_0\}$ and wish to distinguish whether $D \in \calD$ or $D = D_0$.
\end{defn}

