\section{Reduction for low-communication queries}\label{sec:sq_cc}
In this section, we prove  Theorem~\ref{thm:sq_and_cc} using a recent result of Steinhardt, Valiant and Wager \cite{SteinhardtVW16}.
Their result can be seen giving a SQ algorithm that simulates a communication protocol between $n$ parties. Each party is holding a sample drawn i.i.d.~from distribution $D$ and broadcasts at most $b$ bits about its sample (to all the other parties). The bits can be sent over multiple rounds. This is essentially the standard model of multi-party communication complexity (\eg \cite{KN97}) but with the goal of solving some problem about the unknown distribution $D$ rather than computing a specific function of the inputs. Alternatively, one can also see this model as a single algorithm that extracts at most $b$-bits of information about each random sample from $D$ and is allowed to extract the bits in an arbitrary order (generalizing the $b$-bit sampling model that we discuss in Section \ref{subsec:RFA} and in which $b$-bits are extracted from each sample at once). We refer to this model simply as algorithms that extract at most $b$ bits per sample.

\begin{theorem}[\cite{SteinhardtVW16}]\label{thm:SVW}
Let $\calA$ be an algorithm that uses $n$ samples drawn i.i.d.~from a distribution $D$ and extracts at most $b$ bits per sample. Then, for every $\beta > 0$, there is an algorithm $\calB$ that makes at most $2 \cdot b \cdot n$ queries to $\STAT^{(1)}_D(\beta/(2^{b+1} \cdot k))$ and the output distributions of $\calA$ and $\calB$ are within total variation distance $\beta$.
\end{theorem}

We will use this simulation to estimate the expectation of $k$-wise functions that have low communication complexity.
Specifically, we recall the following standard model of public-coin randomized $k$-party communication complexity.
\begin{defn}
For a function $\phi:X^k \to \{\pm 1\}$ we say that $\phi$ has a $k$-party public-coin randomized communication complexity of at most $b$ bits per party with success probability $1-\delta$ if there exist a protocol satisfying the following conditions. Each of the parties is given $x_i \in X$ and access to shared random bits. In each round one of the parties can compute one or more bits using its input, random bits and all the previous communication and then broadcast it to all the other parties. In the last round one of the parties computes a bit that is the output of the protocol. Each of the parties communicates at most $b$ bits in total. For every $x_1,\ldots,x_k\in X$, with probability at least $1-\delta$ over the choice of the random bits the output of the protocol is equal to $\phi(x_1,\ldots,x_k)$.
\end{defn}

We are now ready to prove Theorem~\ref{thm:sq_and_cc} which we restate here for convenience.
\begin{reptheorem}{thm:sq_and_cc}[restated]
Let $\phi:X^k \to \{\pm 1\}$ be a function, and assume that $\phi$ has $k$-party public-coin randomized communication complexity of $b$ bits per party with success probability $2/3$. Then, there exists a randomized algorithm that, with probability at least $1-\delta$, estimates $\Ex_{x \sim D^k}[\phi(x)]$ within $\tau$ using $O(b \cdot k \cdot \log(1/\delta)/\tau^2)$ queries to $\STAT^{(1)}_D(\tau')$ for some $\tau' = \tau^{O(b)}/k$.
\end{reptheorem}
\begin{proof}
We first amplify the success probability of the protocol for computing $\phi$ to $\delta'\doteq \tau/8$ using the majority vote of $O(\log(1/\delta'))$ repetitions. By Yao's minimax theorem \cite{Yao:1977} there exists a deterministic protocol $\Pi'$ that succeeds with probability at least $1-\delta'$ for $(x_1,\dots,x_k) \sim D^k$. Applying Theorem~\ref{thm:SVW}, we obtain a unary SQ algorithm $\A$ whose output is within total variation distance at most $\beta \doteq \tau/8$ from $\Pi'(x_1, \dots, x_k)$ (and we can assume that the output of $\A$ is in $\{\pm 1\}$). Therefore:
$$|\E[\A] - D^k[\phi]| \leq |\E[\A] - \E_{D^k} [\Pi'(x_1, \dots, x_k)]| + |\E_{D^k} [\Pi'(x_1, \dots, x_k)] - D^k [\phi]| \leq  \frac{2\tau}{8} + \frac{2\tau}{8} = \frac{\tau}{2}.$$
Repeating $\A$ $O(\log(1/\delta)/\tau^2)$ times and taking the mean, we get an estimate of $D^k[\phi]$ within $\tau$ with probability at least $1-\delta$. This algorithm uses $O(b \cdot k \cdot \log(1/\delta)/\tau^2)$ queries to $\STAT^{(1)}_D(\tau')$ for $\tau' = \frac{\tau}{8}/(2^{O(\log(8/\tau) \cdot b)} \cdot k) = \tau^{O(b)}/k$.
\end{proof}

The collision probability for a distribution $D$ is defined as $\Pr_{(x_1, x_2) \sim D^2}[x_1 = x_2]$. This corresponds to $\phi(x_1,x_2)$ being the Equality function which, as is well-known, has randomized $2$-party communication complexity of $O(1)$ bits per party with success probability $2/3$ (see, e.g., \cite{KN97}). Applying Theorem~\ref{thm:sq_and_cc} with $k=2$ we get the following corollary.
\begin{corollary}
For any $\tau, \delta >0$, there is a SQ algorithm that estimates the collision probability of an unknown distribution $D$ within $\tau$ with success probability $1-\delta$ using $O(\log(1/\delta)/\tau^2)$ queries to $\STAT^{(1)}_{D}(\tau^{O(1)})$.
\end{corollary} 