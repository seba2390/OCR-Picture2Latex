\appendix

\section{Omitted proofs}


\subsection{Proof of \Cref{le:recovery}}\label{subsec:pf_rec_lem}
	In the following, we denote by $o_c(\cdot)$ and $\omega_c(\cdot)$ asymptotic functions obtained by taking the limit as the parameter $c$ goes to infinity. In particular, $o_c(1)$ can be made arbitrarily close to $0$ by letting $c$ be large enough.
	
	Let $W$ be as in the statement of \Cref{le:recovery}. To prove the lemma, it suffices to show that each bit $j$ in the binary representation of the subspace $\widehat{W}$ constructed by \Cref{alg:recovery} is equal to the corresponding bit of $W$. Henceforth, we fix $j$. We consider the two cases where bit $j$ of $W$ is equal to $1$, and where it is equal to $0$. 
	
	First, we assume that bit $j$ of $W$ is equal to $1$, and prove that in the execution of \Cref{alg:recovery}, it will be the case that $u_{i,j}/v_i \geq 1- o_c(1)$.  We can then set $c$ to be sufficiently large to ensure that $u_{i,j}/v_i \geq  (9/10)$. Note that for any positive real numbers $N$, $D$ and $\tau$ such that $\tau = o(N)$ and $\tau = o(D)$, we have that
	\begin{equation*}
	\frac{N-\tau}{D+\tau} \geq \frac{N}{D} \cdot (1-o(1)).
	\end{equation*}
	Thus, it is enough to show that the next three statements hold:
	\begin{enumerate}
		\item[(i)] $\tau = o_c( \overline{v}_i )$,
		\item[(ii)] if bit $j$ of $W$ is $1$, then $(\overline{u}_{i,j}/\overline{v}_i)	\geq 1 - o_c(1)$,
		\item[(iii)] if bit $j$ of $W$ is $1$, then $\tau = o_c( \overline{u}_{i,j} )$,
	\end{enumerate}
	where $\overline{u}_{i,j} \triangleq \Ex[\phi_{i,j}]$ and $\overline{v}_i \triangleq \Ex[\phi_i]$.
	
	To show (i) above, note that
	\begin{align*}
	\overline{v}_i &= \Pr\bigg[(b_1,\dots,b_{k+1}) = 1^{k+1}\mbox{ and }\rk(Z) = i \bigg]\\
	&\geq v_i - \tau\\
	&\geq v \cdot \tau_i - \tau\\
	&\geq \omega_c(\tau),
	\end{align*}
	where the first inequality follows from the definition of $v_i$ and the SQ guarantee, the second inequality follows from the given assumption (in the statement of \Cref{le:recovery}) that $(v_i/v) \geq \tau_i$, and the last inequality follows from the fact that since $v > \epsilon^{k+1}/2$, for every $i \in [k+1]$, we have that
	\begin{equation*}
	\tau = o_c \bigg( (v \cdot \tau_i - \tau ) \cdot (1-\tau_i/4)\bigg).
	\end{equation*}.
	
	Recall the definition of the event $E_j(Z)$ from the description of \Cref{alg:recovery}. To show (ii) above, note that
	
	\begin{align*}
	\frac{\overline{u}_{i,j}}{\overline{v}_i} &= \Pr\bigg[E_j(Z) ~ | ~ (b_1,\dots,b_{k+1}) = 1^{k+1}\mbox{ and }\rk(Z) = i \bigg]\\
	&\geq \Pr\bigg[\text{all rows of } Z \text{ belong to } W ~ | ~ (b_1,\dots,b_{k+1}) = 1^{k+1}\mbox{ and }\rk(Z) = i \bigg]\\
	&= 1- \Pr\bigg[\exists\text{ a row of } Z \text{ that } \notin W ~ | ~ (b_1,\dots,b_{k+1}) = 1^{k+1}\mbox{ and }\rk(Z) = i \bigg]\\
	&\geq 1 - (k+1) \cdot \Pr_{z \sim Q}[z \notin W]\\
	&\geq 1 - \frac{\tau_i}{4}\\
	&\geq 1- o_c(1),
	\end{align*}
	where the first inequality uses the assumption that bit $j$ in the binary representation of $W$ is $1$ and the facts that the dimension of $W$ is equal to $i$ and that we are conditioning on $\rk[Z] = i$. The second inequality follows from the union bound, the third inequality follows from the assumption given in \Cref{eq:lemma_W_assumption}, and the last inequality follows from the fact that for every $i \in [k+1]$, we have that $\tau_i = o_c(1)$.
	
	To show (iii) above, note that
	\begin{align*}
	\overline{u}_{i,j} &= \overline{v}_i \cdot \frac{\overline{u}_{i,j}}{\overline{v}_i}\\
	&\geq \omega_c(\tau) \cdot (1- o_c(1))\\
	&\geq \omega_c(\tau),
	\end{align*}
	where the first inequality follows from (i) and (ii) above.
	
	We now turn to the (slightly different) case where bit $j$ of $W$ is equal to $0$, and prove that in the execution of \Cref{alg:recovery}, we will have that $u_{i,j}/v_i = o_c(1)$. Note that for any positive real numbers $N$, $D$ and $\tau$ such that $\tau = o(D)$, we have that
	\begin{equation*}
	\frac{N+\tau}{D-\tau} \le \frac{N}{D} \cdot (1+o(1)) + o(1).
	\end{equation*}
	Thus, it is enough to use the fact that $\tau = o_c( \overline{v}_i )$ (proven in (i) above) and to show the next statement:
	\begin{enumerate}
		\item[(iv)] if bit $j$ of $W$ is $0$, then $(\overline{u}_{i,j}/\overline{v}_i)	= o_c(1)$.
	\end{enumerate}
	
	To prove (iv), note that since bit $j$ of $W$ is $0$, we have that
	\begin{align*}
	\frac{\overline{u}_{i,j}}{\overline{v}_i} &\le \Pr\bigg[\exists\text{ a row of } Z \text{ that } \notin W ~ | ~ (b_1,\dots,b_{k+1}) = 1^{k+1}\mbox{ and }\rk(Z) = i \bigg]\\
	&\le \frac{\tau_i}{4}\\
	&\le o_c(1),
	\end{align*}
	where the first inequality above follows from the assumption that bit $j$ in the binary representation of $W$ is $0$ and the facts that the dimension of $W$ is equal to $i$ and that we are conditioning on $\rk[Z] = i$. The second inequality above follows from the union bound and the assumption given in \Cref{eq:lemma_W_assumption}, and the last inequality follows from the fact that for every $i \in [k+1]$, we have that $\tau_i = o_c(1)$. As before, we choose $c$ to be sufficiently large to ensure that this last probability is smaller than $(1/10)$.


\subsection{Proof of Proposition~\ref{prop:single_ex}}
Let $a \in \mathbb{F}_p^{\ell}$. We have that:
	\begin{align*}
	\Ex_{(z,b) \sim D_0}[D_a(z,b)] &= \Ex_{(z,b) \sim D_0}\bigg[\displaystyle\prod\limits_{i=1}^k \Ex_{(z_i,b_i) \sim D_0}[D_a(z_i,b_i)\bigg]\\
	&= \displaystyle\prod\limits_{i=1}^k \Ex_{(z_i,b_i) \sim D_0}\bigg[ D_a(z_i,b_i)\bigg]\\
	&= \displaystyle\prod\limits_{i=1}^k \Ex_{(z_i,b_i) \sim D_0}\bigg[ D_a(z_i) \cdot \ind(b_i = f_a(z_i))\bigg]\\
	&= \displaystyle\prod\limits_{i=1}^k \Ex_{z_i \sim D_0}\bigg[ D_a(z_i) \cdot \Ex_{b_i \in_R \{\pm 1\}}[\ind(b_i = f_a(z_i))]\bigg]\\
	&= \frac{1}{2^k} \cdot \displaystyle\prod\limits_{i=1}^k \Ex_{z_i \sim D_0}\bigg[ D_a(z_i) \bigg]\\
	&= \frac{1}{2^k} \cdot \left(\frac{1}{p} \cdot \beta + \left(1-\frac{1}{p}\right) \cdot \alpha\right)^k.
	\end{align*}

\subsection{Proof of Proposition~\ref{prop:pair_ex}}

	Let $a, a' \in \mathbb{F}_p^{\ell}$. First, assume that $\Hyp_a = \Hyp_{a'}$, i.e., that $a = a'$. Then,
	\begin{align*}
	\Ex_{(z,b) \sim D_0}[D_a(z,b) \cdot D_{a'}(z,b)] &= \Ex_{(z,b) \sim D_0}[D_a(z,b)^2]\\
	&= \Ex_{(z,b) \sim D_0}\bigg[\displaystyle\prod\limits_{i = 1}^k D_a(z_i,b_i)^2 \bigg]\\
	&= \displaystyle\prod\limits_{i = 1}^k \Ex_{(z_i,b_i) \sim D_0} [D_a(z_i,b_i)^2]\\
	&= \displaystyle\prod\limits_{i = 1}^k \Ex_{(z_i,b_i) \sim D_0} [D_a(z_i)^2 \cdot \ind(b_i = f_a(z_i))]\\
	&= \displaystyle\prod\limits_{i = 1}^k \Ex_{z_i}\bigg[D_a(z_i)^2 \cdot \Ex_{b_i}[\ind(b_i = f_a(z_i))] \bigg]
	\end{align*}
	Thus,
	\begin{align*}
	\Ex_{(z,b) \sim D_0}[D_a(z,b) \cdot D_{a'}(z,b)] &= \frac{1}{2^k} \cdot \displaystyle\prod\limits_{i = 1}^k \Ex_{z_i}[D_a(z_i)^2]\\
	&= \frac{1}{2^k} \cdot \displaystyle\prod\limits_{i = 1}^k \left(\frac{1}{p} \cdot \beta^2 + \left(1-\frac{1}{p}\right) \cdot \alpha^2\right)\\
	&= \frac{1}{2^k} \cdot \left(\frac{1}{p} \cdot \beta^2 + \left(1-\frac{1}{p}\right) \cdot \alpha^2\right)^k.
	\end{align*}
	Now we assume that $\Hyp_a \cap \Hyp_{a'} = \emptyset$. Then,
	\begin{align*}
	\Ex_{(z,b) \sim D_0}[D_a(z,b) \cdot D_{a'}(z,b)] &= \Ex_{(z,b) \sim D_0}\bigg[\displaystyle\prod\limits_{i = 1}^k D_a(z_i,b_i) \cdot D_{a'}(z_i,b_i) \bigg]\\
	&= \displaystyle\prod\limits_{i = 1}^k \Ex_{(z_i,b_i) \sim D_0}[D_a(z_i,b_i) \cdot D_{a'}(z_i,b_i)]\\
	&= \displaystyle\prod\limits_{i = 1}^k \Ex_{(z_i,b_i) \sim D_0}[D_a(z_i) \cdot \ind(b_i = f_a(z_i)) \cdot D_{a'}(z_i) \cdot \ind(b_i = f_{a'}(z_i))]\\
	&=  \displaystyle\prod\limits_{i = 1}^k \Ex_{z_i}\bigg[D_a(z_i) \cdot D_{a'}(z_i) \cdot \ind(f_a(z_i) = f_{a'}(z_i))  \cdot \Ex_{b_i}[\ind(b_i = f_{a}(z_i))] \bigg]\\
	&=  \frac{1}{2^k} \cdot \displaystyle\prod\limits_{i = 1}^k \Ex_{z_i}\bigg[D_a(z_i) \cdot D_{a'}(z_i) \cdot \ind(f_a(z_i) = f_{a'}(z_i)) \bigg]\\
	&= \frac{1}{2^k} \cdot \displaystyle\prod\limits_{i = 1}^k \left(\alpha^2 \cdot \left(1-\frac{2}{p}\right)\right)\\
	&= \frac{1}{2^k}\cdot \left(\alpha^2 \cdot \left(1-\frac{2}{p}\right)\right)^k
	.\end{align*}
	Finally, we assume that $\Hyp_a \neq \Hyp_{a'}$ and $\Hyp_a \cap \Hyp_{a'} \neq \emptyset$. Then,
	\begin{align*}
	\Ex_{(z,b) \sim D_0}[D_a(z,b) \cdot D_{a'}(z,b)] &= \frac{1}{2^k} \cdot \displaystyle\prod\limits_{i = 1}^k \Ex_{z_i}\bigg[D_a(z_i) \cdot D_{a'}(z_i) \cdot \ind(f_a(z_i) = f_{a'}(z_i)) \bigg]\\
	&=  \frac{1}{2^k} \cdot \displaystyle\prod\limits_{i = 1}^k ( \frac{\beta^2}{p^2} +\alpha^2 \cdot (1-\frac{2}{p}+\frac{1}{p^2}))\\
	&= \frac{1}{2^k} \cdot ( \frac{\beta^2}{p^2} +\alpha^2 \cdot (1-\frac{2}{p}+\frac{1}{p^2}))^k.
	\end{align*}
	
\subsection{Proof of Proposition~\ref{prop:D_0}}

	First, we assume that $a, a' \in \mathbb{F}_p^{\ell}$ are such that $\Hyp_a = \Hyp_{a'}$, i.e., $a = a'$. Then, by Proposition~\ref{prop:pair_ex} and by our settings of $\alpha$ and $\beta$, we have that
	\begin{align*}
	\Ex_{(z,b) \sim D_0}[D_a(z,b) \cdot D_{a'}(z,b)] &= \frac{1}{2^k} \cdot (\frac{1}{p} \cdot \beta^2 + (1-\frac{1}{p}) \cdot \alpha^2)^k\\
	&= \frac{1}{2^{2k} \cdot p^{(2\ell-1)\cdot k}} \cdot (1+\frac{1}{p-1})^k.
	\end{align*}
	Hence, $D_0[\hat{D}_a \cdot \hat{D}_{a'}] = (p+1-\frac{1}{p-1})^k - 1$, as desired.
	
	Next, we assume that $a, a' \in \mathbb{F}_p^{\ell}$ are such that $\Hyp_a \cap \Hyp_{a'} = \emptyset$. Then, by Proposition~\ref{prop:pair_ex} and by our setting of $\alpha$, we have that
	\begin{align*}
	\Ex_{(z,b) \sim D_0}[D_a(z,b) \cdot D_{a'}(z,b)] &= \frac{1}{2^k}\cdot (\alpha^2 \cdot (1-\frac{2}{p}))^k\\
	&= \frac{1}{2^{3k} \cdot p^{2 k \ell}} \cdot \frac{(1-\frac{2}{p})^k}{(1-\frac{1}{p})^{2k}}.
	\end{align*}
	Hence, $D_0[\hat{D}_a \cdot \hat{D}_{a'}] = \frac{1}{2^k} \cdot \frac{(1-\frac{2}{p})^k}{(1-\frac{1}{p})^{2k}}-1$, as desired.
	
	Finally, we assume that $a, a' \in \mathbb{F}_p^{\ell}$ are such that $\Hyp_a \neq \Hyp_{a'}$ and $\Hyp_a \cap \Hyp_{a'} \neq \emptyset$. Then, by Proposition~\ref{prop:pair_ex} and by our settings of $\alpha$ and $\beta$, we have that
	\begin{align*}
	\Ex_{(z,b) \sim D_0}[D_a(z,b) \cdot D_{a'}(z,b)] &= \frac{1}{2^k} \cdot ( \frac{\beta^2}{p^2} +\alpha^2 \cdot (1-\frac{2}{p}+\frac{1}{p^2}))^k\\
	&= \frac{1}{2^{2k} \cdot p^{2 k \ell}}.
	\end{align*}
	Hence, $D_0[\hat{D}_a \cdot \hat{D}_{a'}] = 0$, as desired.
