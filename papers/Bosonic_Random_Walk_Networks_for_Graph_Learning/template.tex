%% LyX 2.2.3 created this file.  For more info, see http://www.lyx.org/.
%% Do not edit unless you really know what you are doing.
\documentclass[12pt,english]{article}
\usepackage[T1]{fontenc}
\usepackage[latin9]{inputenc}
\usepackage{geometry}
\geometry{verbose}
\setlength{\parskip}{\smallskipamount}
\setlength{\parindent}{0pt}
\usepackage{babel}
\usepackage{amsmath}
\usepackage{amsthm}
\usepackage[unicode=true,
 bookmarks=true,bookmarksnumbered=false,bookmarksopen=false,
 breaklinks=false,pdfborder={0 0 1},backref=false,colorlinks=false]
 {hyperref}
\hypersetup{pdftitle={Everything I know about Everything},
 pdfauthor={Justin Domke}}
\usepackage{breakurl}

\makeatletter
%%%%%%%%%%%%%%%%%%%%%%%%%%%%%% Textclass specific LaTeX commands.
\numberwithin{equation}{section}
\numberwithin{figure}{section}

%%%%%%%%%%%%%%%%%%%%%%%%%%%%%% User specified LaTeX commands.
\newcommand{\lecture}[4]{
   \pagestyle{myheadings}
   \thispagestyle{plain}
   \newpage
   \setcounter{page}{1}
   \noindent
   \begin{center}
   \framebox{
      \vbox{\vspace{2mm}
    %\hbox to 6.28in { {\bf \@date} \hfill note }
    \hbox to 6.28in { {\bf \@date} \hfill Lecture #3 }
       \vspace{2mm}
       \hbox to 6.28in { {\Large \hfill #1  \hfill} }
       \vspace{2mm}
       \hbox to 6.28in { {\it Lecturer: Justin Domke} \hfill {\it Scribe: #2}  }
      \vspace{2mm}}
   }
   \end{center}
   \markboth{#1}{#1}
   \vspace*{4mm}
}

%\setlength{\oddsidemargin}{0.25 in}
%\setlength{\evensidemargin}{-0.25 in}
%\setlength{\topmargin}{-0.6 in}
%\setlength{\textwidth}{6.5 in}
%\setlength{\textheight}{8.5 in}
% modified these May 6 2017 to make more space
\setlength{\oddsidemargin}{0.0 in}
\setlength{\evensidemargin}{-0.5 in}
\setlength{\topmargin}{-0.8 in}
\setlength{\textwidth}{6.5 in}
\setlength{\textheight}{9.5 in}
\setlength{\headsep}{0.5 in}
\setlength{\parindent}{0 in}
%\setlength{\parskip}{0.1 in}
\setlength{\parskip}{0.05 in}
\usepackage{psfrag}

\usepackage{pdfsync}

\makeatother

\begin{document}
\lecture{Introduction and Overview}{Awesome TA Name Here}{1, Sep. 6, 2017}
% the fields are
% 1) title
% 2) scribe name
% 3) lecture number and date of lecture\tableofcontents{}

\section{Summary}

Last time we talked about...

We were left thinking about...

Today we will cover

\section{Empirical Risk Minimization}

The goal of Empirical Risk Minimization (ERM) is to minimize
\[
\sum_{i=1}^{n}L(x_{i},y_{i},\theta).
\]

\section{Guidlines}
\begin{itemize}
\item Make sure you include:
\begin{itemize}
\item The topic for the day
\item Your name as scrbe
\item The number for the lecture (1-26)
\item The date of the lecture
\end{itemize}
\item Use the notation from the lectures (which will match the notation
from the book as far as possible)
\item Include figures.
\begin{itemize}
\item You are free to make these any way that you find most efficient.
\item You can use powerpoint or tikz or hand-draw them and scan them in
if that's fastest.
\end{itemize}
\end{itemize}

\end{document}