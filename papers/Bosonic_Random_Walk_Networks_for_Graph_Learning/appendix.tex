\appendix

\setcounter{thm}{0}

\onecolumn


\section{Fock Space}
\label{apx:fock}
Fock basis is a construction for the state space of a variable or unknown number of identical particles from the Hilbert space of a single particle. If the identical particles are bosons, the $n$-particle states are vectors in the symmetric tensor product of $n$ single-particle Hilbert spaces \citep{fock_wiki}. In this section we provide a basic introduction to Fock spaces, and the corresponding creation and annihiliation operators. 

\begin{wrapfigure}{r}{4cm}
\vspace{-0.3cm}
\captionsetup{type=figure}
    \centering
    \includegraphics[width=\linewidth]{hatom.png}
    \caption{Configuration Space of an Atom \label{fig:hatom} }
\end{wrapfigure}

Consider a system which can accomodate particles in $M$ different states. Let $B = {k_i} i \in [0,1,..M]$ be an orthonormal basis of states in the the one-particle Hilbert space. For electronic states in an atom such as the one shown in Figure \ref{fig:hatom} these can correspond to the ground state shell , the first excited state or higher states. 

A Fock state is then defined as the state such that for each $i$, the state is an eigenstate of the particle number operator $ \widehat {N_{{\mathbf {k} }_{i}}}$ corresponding to the $i$-th elementary state $k_i$. The Fock bases of the state space of the system is then depicted by
$$\ket{n_0,n_1,n_2...n_M}$$
where $n_i$ denotes the number of particles in the state $i$. For the system in \ref{fig:hatom}  this refers to the number of electrons in different shells around the atom. Informally, the Fock bases is a counting basis i.e the bases of zero particle states, one particle states, two particle states, and so on



\paragraph{Creation and Annihilation Operators}
For the state defined earlier the  annihilation/creation operators are indexed by the state $i=1...M$ and operate as follows:
\begin{align*}
    a_i^\dagger \ket{n_0,n_1,..,n_i,..n_M} = n_i \ket{n_0,n_1,..,n_i-1,...,n_M} \\
    a_i \ket{n_0,n_1,..,n_i,..n_M} = n_i \ket{n_0,n_1,..,n_i+1,...,n_M} \\
\end{align*}


\section{Bosonic Walkers Example}
\label{apx:example}
Consider a pair of particles executing a quantum walk on a 1D grid.
The coin space of walkers then has two states $\ket{\uparrow}$ and $\ket{\downarrow}$. We assume that the shift operators act in a way such that a walker whose coin is up moves right, and a walker whose coin is down moves left.

Let the current state of the walkers after the action of the coin operator be: 
$$\ket{1,-1}\otimes\ket{\uparrow \downarrow}$$
It represents the first walker at position 1 with its coin state being up, while the second walker at position -1 with its coin state being down. From this state, the first particle can only move right and the second can only move left.

However if the particles are boson, then the aforementioned state does not exist in the Hilbert space of the system (as it is not symmetric with respect to the exchange operation). Instead the pair of bosonic walkers will be in a state $$ \frac{1}{\sqrt{2}}(\ket{1,-1}\otimes\ket{\uparrow \downarrow} + \ket{-1,1}\otimes\ket{\downarrow \uparrow})$$

From this state the first particle can move both left and right, a motion which is not possible for a non-bosonic walker. Note this was enforced by the symmetrization postulate. Such "nonlocal" correlations across non-interacting particle states is not possible in non-bosonic particles (or in classical case) and is a truly quantum phenomenon. %reminiscient of entanglement. %Nonlocality, quantum correlations, and violations of classical realism in the dynamics of two noninteracting quantum walkers

\section{Bosonic Walker Simulation}
\label{apx:bosonic}
We implement the simulation of quantum walkers as described in \cite{gamble2010twoparticle,Rigovacca_2016}. For simplicity, we shall limit the exposition to involve only 2 particles. The generalization to multiple walkers is straightforward.

A joint two boson state can be obtained from two single-particle states $\ket{\psi_1}, \ket{\psi_2}$ with the following symmetrization operation:

$$
\ket{\text{Sym}(\psi_1,\psi_2)} = \dfrac{\ket{\psi_1}\otimes\ket{\psi_2} + \ket{\psi_2}\otimes\ket{\psi_1}}{\sqrt{2(1+ |\bra{\psi_1}\ket{\psi_2}|^2)}}
$$

One should note that the aforementioned symmetrized form cannot represent all the possible states in the symmetric Hilbert space of two bosonic walkers. For example, in walks on a chain, the following state is a physically allowed one:
$$
\ket{\psi} = \dfrac{\ket{ij} + \ket{ji}}{\sqrt{2}} \otimes \dfrac{\ket{\uparrow \downarrow} + \ket{ \downarrow \uparrow}}{\sqrt{2}}
$$
where $i,j$ are indices of nodes in the chain, while $\downarrow,\uparrow$ represent the coin states. However, we shall limit ourselves to the states expressible in the aforementioned symmetrized form in this work.

Finally, during the measurement stage also, one needs to take into account the symmetrization requirement. This is done by considering a measurement involving the following projector operation::

$$
\Pi_{ij} = \frac{1}{2}(\ket{ij}\bra{ij} + \ket{ji}\bra{ji} 
$$
$$
Pr(\psi_t)[i,j] = \text{Tr}[\Pi_{ij} U_t\ket{\psi_0}\bra{\psi_0}U_t^\dagger]
$$

\section{Hyperparameter Details}
\label{apx:hyper}

\paragraph{Classification}
For the Enzyme and NCI1 experiment, we set the walk length to be 6 in both verison of the quantum network. The output neural net is a set2vec layer (for aggregation) followed by single layer. The feature and hidden layer dimensions are all set 64. In Mutag, the walk length is reduced to 4 and the layer size to 16. The GCN and DCNN are used as the input layer to a similar neural network i.e a set2vec layer followed by a hidden layer of size 64 (16 for Mutag).


\paragraph{Regression}
For QM7 we use quantum walk networks using a 4-step walk, followed by the set2vec layer with a hidden size of dimension 10. A similar setup is followed for DCNN and GCN models.

All models are trained with Adam optimizer with a learning rate of 1e-3.
