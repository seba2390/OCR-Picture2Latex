\documentclass[prl,amsmath,amssymb, notitlepage, twocolumn,
nofootinbib,
superscriptaddress,
longbibliography
]{revtex4-1}

\usepackage{amsmath}
\usepackage{tabularx,graphicx}
\usepackage{epstopdf}
% \usepackage{txfonts}
\usepackage{graphicx}
\usepackage{latexsym}
\usepackage{amssymb}
\usepackage{amsmath}
\usepackage{color, colortbl}
\usepackage{psfrag}
\usepackage{bbm}
\usepackage{bm}
\usepackage{titlesec}
\usepackage{dsfont}
\usepackage{feynmp}
\usepackage{slashed}
\usepackage{multirow}
\textwidth 6.8in
\oddsidemargin -0.25in
\usepackage{url}

\newcommand{\rb}[1]{\left( #1 \right)}
\newcommand{\mb}[1]{\mathbf{#1}}
\newcommand{\mc}[1]{\mathcal{#1}}
\newcommand{\ew}[1]{\langle #1 \rangle}
\newcommand{\beq}{\begin{eqnarray}}
	\newcommand{\eeq}{\end{eqnarray}}
\newcommand{\blg}[1]{\mathrm{lg}\left[  #1right]}
\newcommand{\svec}{\mbox{\boldmath$\sigma$}}
\newcommand{\op}[2]{| #1 \rangle \langle #2 |}
\newcommand{\sech}{\mathrm{sech}}
\newcommand{\sfrac}[2]{\begin{array}{c}\frac{#1}{#2}\end{array}}
\newcommand{\la}{\langle}
\newcommand{\ra}{\rangle}
\newcommand{\veps}{\varepsilon}
\newcommand{\vphi}{\varphi}
\newcommand{\Tr}{{\rm Tr}}
\newcommand{\tr}{{\rm tr}}
\newcommand{\re}{{\rm Re}}
\newcommand{\im}{{\rm Im}}
\newcommand{\bsp}{\begin{split}}
	\newcommand{\esp}{\end{split}}
\newcommand{\const}{{\rm const}}
\newcommand{\adj}{{\rm adj}}
\newcommand{\sgn}{{\rm sgn}}
\newcommand{\eff}{{\rm eff}}
\newcommand{\hc}{{\rm h.c.}}
\newcommand{\half}{\frac{1}{2}}
\newcommand{\ie}{{i.e., }}
\newcommand{\eg}{{e.g., }}
\newcommand{\gcs}{{\rm{CS}}_g}
\newcommand{\mO}{{\mathcal{O}}}
\newcommand{\BCS}{{\rm BCS}}
\newcommand{\DMT}{{\rm DMT}}
\newcommand{\FT}{{\rm FT}}
\newcommand{\DMMT}{{\rm DM$^2$T }}
\newcommand{\tn}{\textnormal}
\usepackage{color}
\definecolor{darkblue}{rgb}{0.,0.,0.4}
\definecolor{darkred}{rgb}{0.5,0.,0.}
\definecolor{BlueViolet}{RGB}{138,43,226}
\definecolor{SkyBlue}{RGB}{30,144,255}
\definecolor{DarkGreen}{RGB}{0,100,0}
\usepackage[pdftex,colorlinks=true,linkcolor=darkblue,citecolor=blue,urlcolor=darkred]{hyperref}
\usepackage[normalem]{ulem}

\let \oldvec \vec
\renewcommand{\vec}[1]{\bm{#1}}
\newcommand{\vechat}[1]{\hat{\bm{#1}}}

%%%% Commenting & change tracking
\newcommand{\delete}[1]{ {\color{red}  \sout{#1}   }}
\newcommand{\add}[1]{ { \color{blue}  #1 }}
\newcommand{\change}[2]{{  \color{red}   \sout{#1}   } { \color{blue}  #2}}

\newcommand{\weicheng}[1]{ { \color{BlueViolet} \footnotesize (\textsf{WY}) \textsf{\textsl{#1}} }}
\newcommand{\sungsik}[1]{ { \color{darkred} \footnotesize (\textsf{SSL}) \textsf{\textsl{#1}} }}
\newcommand{\liujun}[1]{ { \color{SkyBlue} \footnotesize (\textsf{LZ}) \textsf{\textsl{#1}} }}


%%%% Weicheng Ye Addition
\usepackage{mathrsfs}
% \usepackage{caption}
\usepackage{subcaption}
\usepackage{comment}



\begin{document}

\title{
UV/IR Mixing in Marginal Fermi Liquids  
}
	
\author{Weicheng Ye}
\affiliation{Perimeter Institute for Theoretical Physics, Waterloo, Ontario, Canada N2L 2Y5}
\affiliation{Department of Physics and Astronomy, University of Waterloo, Waterloo, Ontario, Canada N2L 3G1}

\author{Sung-Sik Lee}
\affiliation{Perimeter Institute for Theoretical Physics, Waterloo, Ontario, Canada N2L 2Y5}
\affiliation{Department of Physics and Astronomy, McMaster University, Hamilton, Ontario, Canada L8S 4M1}
	
\author{Liujun Zou}
\affiliation{Perimeter Institute for Theoretical Physics, Waterloo, Ontario, Canada N2L 2Y5}

\begin{abstract}

When Fermi surfaces (FS) are subject to long-range interactions that are marginal in the renormalization-group sense, 
Landau Fermi liquids  are destroyed,  but only barely.
With the interaction further screened by 
particle-hole excitations through one-loop  quantum corrections,
it has been believed that
these marginal Fermi liquids (MFLs) are described by weakly coupled field theories at low energies.
In this paper, we point out a possibility in which higher-loop processes qualitatively change the picture through UV/IR mixing, in which the size of FS enters as a relevant scale.
The UV/IR mixing effect enhances the coupling at low energies, such that the basin of attraction for the weakly coupled fixed point of a (2+1)-dimemsional MFL shrinks to a measure-zero set in the low-energy limit.
This UV/IR mixing is caused by gapless
virtual Cooper pairs 
that spread over the
entire FS
through marginal long-range interactions.
Our finding signals a possible breakdown of the patch description for the MFL, and questions the validity of using the MFL as the base theory  in a controlled scheme for non-Fermi liquids that arise from relevant long-range interactions.

\end{abstract}

\maketitle



{\it Introduction.} 
Non-Fermi liquids (nFLs) arise ubiquitously when Fermi surfaces (FS) are coupled to gapless collective modes that mediate long-range interactions.
The physics of nFLs is central to  the strange metallic behavior and/or unconventional superconductivity in various systems, including cuprates, heavy-fermion compounds, half-filled Landau level
and pnictides \cite{Senthil2006, Lee2017, Berg2018}. However, understanding nFLs 
has been a long-standing challenge due to
strong quantum fluctuations 
amplified by abundant gapless modes near FS \cite{
Polchinski1992,
ALTSHULER,
Kim1995,
ABANOV2,
Lee2009,
Metlitski2010}. 


Under renormalization group (RG) flow, most theories for (2+1)-dimensional nFLs flow to the strong-coupling regime at low energies,
and non-perturbative methods are required to understand their universal long-distance physics \cite{Sur2013,PhysRevX.7.021010}.
However, there is a special class of nFLs,
marginal Fermi liquids (MFLs)\footnote{In this paper, the terms ``metal", ``nFL" and ``MFL" may refer to that of either electrons or some emergent fermions, depending on the context.},
where interaction effects are relatively weak,
\ie marginal in the RG sense with logarithmic (log) perturbative corrections.
If the marginal interactions are further screened, the long-distance physics should be captured by weakly coupled theories.
While the MFL was first introduced for cuprates \cite{
VARMALI},
it is relevant in rather broad contexts.
First, metallic states realized at the half-filled Landau level and exotic Mott transitions may be related to MFLs \cite{HLR,Senthil2008}. 
Second,  MFLs have been used as a foothold to gain a controlled access to strongly coupled nFLs in an expansion scheme,
where the exponent with which long-range interactions decay in space is used as a control parameter \cite{Nayak1994,Mross2010}.

In nFLs, fermionic quasiparticles are destroyed by scatterings that are singularly enhanced at small momenta. If large-momentum scatterings are suppressed strongly enough, 
one can understand physical observables that are local in momentum space (\eg the single-particle spectral function) within local patches of FS that include the momentum point of interest (see Fig. \ref{fig: FSpatch}). 
Although this patch description becomes ultimately invalid in the presence of a pairing instability driven by short-range four-fermion couplings,
which is {\it non-local} in momentum space, 
one may hope that the dominant physics in the normal state can be captured within the patch theory without invoking the entire FS. 
So the patch theory \cite{Lee2008, Lee2009, Metlitski2010, Mross2010, Metlitski2014} has been widely used to describe a large class of nFLs (see Refs. \cite{HLR,Senthil2008,Lee1992,Lee2005} for some prominent examples) . 

In MFLs, however, the validity of the patch theory is questionable even before taking into account the short-range four-fermion couplings, because large-momentum scatterings are only marginally suppressed.
If large-momentum scatterings create significant inter-patch couplings, the patch theory fails even for the purpose of describing observables local in momentum space.
In this case, the size of the FS, a UV paramter, qualitatively modifies the IR scaling behavior, showcasing UV/IR mixing.

While such UV/IR mixing does not show up at low orders in the perturbative expansion \cite{Mross2010, Metlitski2014}, 
a systematic understanding of higher-order effects is still lacking.
This issue is also pertinent to multiple experimentally relevant problems.
First, in the context of quantum Hall physics, it is debated whether the Halperin-Lee-Read theory \cite{HLR} and the recently proposed Son's theory \cite{Son2015}
describe the same 
universal physics 
of the composite Fermi liquid (CFL).
Second, in the continuous Mott transitions reported in {\ensuremath{\kappa}}-(ET)$_{2}$Cu$_{2}$(CN)$_{3}$ \cite{Kurosaki2005, Furukawa2015} and moir\'e materials \cite{Li2021, Ghiotto2021},
the observed phenomena appear to be compatible with the predictions of the patch theory \cite{Senthil2008}, but some specific critical properties seem to disagree \cite{Li2021}. 
To resolve these issues, it is crucial to understand the behaviors of the corresponding MFL theories.
Moreover, understanding higher-order effects in nFLs in general may provide new insight on
the nature of quantum phase transitions associated with sudden jumps of the FS size \cite{Onuki2005,Fang2020}, 
and deconfined metallic quantum criticality
\cite{Zou2020, Zou2020a, Zhang2020, Zhang2020a}.

In this paper, we study the higher-order behaviors of a theory of
$N$ flavors of two-dimensional FS coupled to a dynamical $U(1)$ gauge field, whose kinetic energy scales as $k_y^{1+\epsilon}$.
For the marginal exponent ($\epsilon=0$),
we indeed find potential UV/IR mixing in four-loop processes
that renormalize the gauge coupling (see Fig. \ref{FD: double log diagrams}), with a strength logarithmically singular
in the FS size.
This is caused by gapless {\it virtual} Cooper pairs that manage to explore the entire FS, assisted by large-momentum scatterings that are only marginally suppressed.


\begin{figure}[h]
\centering
\includegraphics[width=0.36\textwidth]{FSpatch.pdf}
\caption{
In the patch theory, a FS is partitioned into multiple patches ($\sim k_F/\Lambda_y$ of them, with $k_F$ the Fermi momentum and $\Lambda_y$ the patch size). Ignoring the short-range four-fermion interactions, 
couplings between modes from different patches are weak unless they have almost parallel Fermi velocities.
In this case,
one can focus on a pair of antipodal patches 
that have nearly collinear Fermi velocities.
}
\label{fig: FSpatch}
\end{figure}

{\it Model and regularization scheme.} We denote the low-energy fermion fields near  a pair of antipodal patches by $\psi_{ip}$, 
with $p=\pm$ the patch index and $i=1,...,N$ the flavor index. 
$a$ represents the gauge field  (see Fig.~\ref{fig: FSpatch}). 
Due to the kinematic constraints, the most important interactions occur between fermionic modes and the gauge bosons 
with momenta nearly perpendicular 
to their Fermi velocity \cite{Polchinski1992, Lee2008}. 
The Euclidean action for the patch theory is \cite{ Mross2010}
\beq \label{eq: patch theory}
S = S_\psi + S_{\text{int}} + S_a,
\eeq
where 
\beq
\label{eq: Lagrangian of patch theory}
\begin{split}
S_\psi &= \int[dk] \sum_{i,p}\psi^\dagger_{ip}(k)(-i k_\tau + p k_x + k_y^2)\psi_{ip}(k),\\
S_{\text{int}} &= \int[dk_1][dk_2] \sum_{i,p}\lambda_p a(k_1) \psi^\dagger_{ip}(k_1+k_2) \psi_{ip}(k_2),\\
S_a &= \int[dk] \frac{N}{2e^2} |k_y|^{1+\epsilon} a(-k)a(k),\\
\end{split}
\eeq
with $[dk]=\frac{dk_\tau dk_x dk_y}{(2\pi)^3}$ and $\lambda_\pm=\pm 1$. The reason for the opposite signs of $\lambda_\pm$ is because $a$ couples to the currents of the fermions, and fermions from the two patches have opposite Fermi velocities. By power counting, the coupling $e^2$ is marginal (relevant) if $\epsilon=0$ ($\epsilon>0$). Physically, with decreasing $\epsilon$, 
the fermion-boson coupling gets weaker at small momenta, 
but large-momentum scatterings become stronger, 
which increases the ``risk" of UV/IR mixing.
%}
Below we primarily focus on the marginal case with $\epsilon=0$. 
Note that 
a large $N$ is still useful in organizing the calculations 
when $\epsilon=0$.

One can introduce two cutoffs.
$\Lambda$ denotes the energy cutoff\footnote{
UV cutoff associated to high-energy modes can be imposed on the 
$x$-momentum relative to the FS as well, yet this cutoff violates an emergent gauge invariance \cite{Mross2010}.
}, and $\Lambda_y$ is the cutoff of $y$-momentum.
The former is the usual UV cutoff,
while the latter represents the size of patch.
We take $\Lambda\rightarrow\infty$ for simplicity, \ie the theory is regularized by $\Lambda_y$ only. Crucially, $\Lambda_y$ also serves as IR data that measures the number of gapless modes near the FS,
and there is a priori no guarantee that low-energy observables are insensitive to
$\Lambda_y$.
If the $\Lambda_y$-dependence cannot be removed 
in low-energy observables 
by renormalization, the theory has UV/IR mixing,
and the patch description
fails.



\begin{figure}
     \centering
    \begin{subfigure}{\linewidth}
         \includegraphics[width=0.8\linewidth]{Benz.pdf}
          \caption{
          }
         \label{FD: Benz}
    \end{subfigure}
    \begin{subfigure}{\linewidth}
         \includegraphics[width=\linewidth]{3Stringe.pdf}
        %  \hfill
        %  \hfill
          \caption{
          }\label{FD: 3Stringe}
         
    \end{subfigure}
        \caption{Eq. \eqref{eq: double log} comes from these two diagrams, together with their cousins with both internal fermion loops flipped in direction, which are not shown here. Note that only when the two fermion loops in each diagram run in the same direction do they contribute  to double-log divergence. See Sec. III.C.2 in \cite{supp} for all diagrams at this order.}
\label{FD: double log diagrams}
\end{figure}



{\it UV/IR mixing.} We consider the photon self-energy, $\Pi(k)$, which is $O(N^1)$ to the leading order.
To order $N^0$, $\Pi(k)$ is finite as $\Lambda_y\rightarrow\infty$ \cite{Metlitski2010, Mross2010}, due to a kinematic constraint that is nevertheless absent at higher orders (see Eq. (30) of the Supplemental Material \cite{supp} or \cite{Holder2015b}).
At order $N^{-1}$, by exact calculations we find a {\it double}-log divergence in the diagram in Fig. \ref{FD: Benz}:
$\Pi_0(k_\tau=0, k_y)= \frac{1}{N}\frac{|k_y|}{e^2}\frac{2\alpha^4}{3\pi^2}\left(\ln\left(\frac{\Lambda_y}{k_y}\right)\right)^2$, with $\alpha\equiv e^2/(4\pi)$ (see Sec. III.C.3 in \cite{supp}). Other diagrams are harder to compute explicitly.
However, under reasonable assumptions, we argue that the only other net contribution to the double-log divergence is from Fig. \ref{FD: 3Stringe}, whose contribution is also $\Pi_0$ at $k_\tau=0$ (see Sec. III.C.4 in \cite{supp}). Double-log divergences are usually from divergences in sub-diagrams, which can then be cancelled by  diagrams with counter terms. 
But the present double-log divergences are not due to this, since the only divergent sub-diagram in Fig.~\ref{FD: double log diagrams} 
is the three-loop vertex correction, but the corresponding counter term does not contribute to the renormalization of the boson kinetic term to order $N^{-1}$. Taken all diagrams together (including the ones with counter terms), the total double-log divergence to order $N^{-1}$ is
\beq
\label{eq: double log}
\Pi(k_\tau=0, k_y)\sim \frac{1}{N}\frac{|k_y|}{e^2}\frac{4\alpha^4}{3\pi^2}\left(\ln\left(\frac{\Lambda_y}{k_y}\right)\right)^2.
\eeq


\begin{figure}[h]
     \centering
     \includegraphics[width=\linewidth]{IntegrationRegion.pdf}
         \caption{
The gray regions illustrate modes that are integrated out {\it if} we tune $\Lambda_y$ in MFL (left) or $\Lambda$ in a usual field theory without FS (right). The latter has gapless modes only at a single point in the momentum space (shown in red), while the former has gapless modes overlapping with the gray regions. 
In MFL, $\Pi(k)$ calculated at a {\it fixed} $\Lambda_y$ can have
IR singularities
stronger 
than $\ln(\Lambda_y/k)$.
}
         \label{fig: Integration Region}
\end{figure}

To better understand this result, first consider the usual renormalizable field theories without a FS, \eg $3+1$ dimensional $\phi^4$ theory.
In such theories, given a UV cutoff $\Lambda$, quantities like
$d\Pi/d\ln\Lambda$ 
are analytic in the external momentum $k\ll\Lambda$, since this derivative measures the contribution of {\it high-energy}
modes in the energy window $[\Lambda, \Lambda+d\Lambda)$ (see Fig. \ref{fig: Integration Region}). 
Consequently, the non-analyticity in $\Pi$ can at most take the form of $k^2 \ln\frac{\Lambda}{k}$, and the $\Lambda$-dependence of the results can then be eliminated by local counter terms, allowing any observable at a scale $k_1 \ll\Lambda$ to be expressed solely in terms of renormalized quantities measured at another scale $k_2 \ll\Lambda$, and the IR physics is insensitive to the UV physics.

However, the present theory has another short-distance scale, $\Lambda_y$, which measures the number of {\it gapless} modes near the FS.
Low-energy observables in general can depend on $\Lambda_y$ in a sensitive manner. Especially, $d\Pi/d\ln\Lambda_y$ does not have to be analytic in $k$
(see Fig. \ref{fig: Integration Region}). Gapless modes can not only renormalize the existing non-local term through
$|k_y| \ln(\Lambda_y/|k_y|)$,
but also generate stronger non-analyticity
in the quantum effective action, such as 
$|k_y| \ln^n(\Lambda_y/|k_y|)$ with $n>1$ ($n=2$ in Eq. \eqref{eq: double log}).
In this case, the $\Lambda_y$-dependence cannot be removed in low-energy observables through renormalization of the existing terms (local or not) in the action, signaling UV/IR mixing.

UV/IR mixing is known to arise 
in metals {\footnote{We note that UV/IR mixing with different origins is also proposed in other setups \cite{Ghamari2014, Shao2020, You2021, Shackleton2021, Zhou2021}.}. 
First, the FS size $k_F$, a UV parameter,
becomes relevant at low energies when a critical boson is coupled with FS
whose dimension is greater than one, as a boson can decay into particle-hole pairs along the ``great circle'' of FS whose tangent space includes the boson momentum \cite{Mandal2015,Mandal2016}.
Second, the FS size is important in the presence of pairing instabilities driven by short-range four-fermion interactions, via which Cooper pairs residing on the FS with zero total momentum can be scattered throughout the entire FS without violating momentum or energy conservation \cite{Cooper1956, Metlitski2014,Wang2016}.

The origin of the UV/IR mixing we find here is related to the second one but different. The contribution in Eq. \eqref{eq: double log} comes from {\it virtual} Cooper pairs (VCP), represented by the two fermion loops that come from opposite patches and run in the same direction in Fig. \ref{FD: double log diagrams}. Via the {\it marginal} long-range interactions mediated by the gauge field, these VCP spread over the entire FS, which enjoys a large phase space for scattering and can have singular contributions (see Eq. (86) in \cite{supp}).
Indeed, the double-log divergence disappears if either the VCP are absent (\eg by taking the fermion loops in Fig. \ref{FD: double log diagrams} to be in the same patch 
and/or run in opposite directions), or large-momentum scatterings are further suppressed (\eg by taking $\epsilon>0$)\footnote{
Our double-log divergence appears similar to the Sudakov double pole \cite{Peskin1995, Stewart2013} in QED and effective theories of QCD.
Both of them originate from gapless degrees of freedom.
However, the Sudakov double-log originates from small momentum modes
while our double-log divergence is from modes with large $y$-momentum.
}. 
This is reminiscent to the enhanced quasiparticle decay rate due to VCP in Fermi liquids \cite{Pimenov2021}.

{\it Consequences of UV/IR mixing.} Eq. \eqref{eq: double log} forces us to view $\Lambda_y$ as another 
{\it coupling constant} of the theory \cite{Mandal2015}. 
In particular, $\tilde \Lambda_y \equiv \Lambda_y/\mu$ plays the role of a relevant coupling as the size of FS blows up relative to the decreasing scale $\mu$.
The beta functions of the theory are (see Sec. I and II in \cite{supp} for details)
\beq\label{eq: Confusing RG term}
\frac{d \tilde \Lambda_y}{d \ln\mu} = - \tilde \Lambda_y,  ~~ 
\frac{d\alpha}{d \ln\mu} = 
\frac{2\alpha^2}{\pi N} -
\frac{8\alpha^5}{3\pi^2N^2} \ln\tilde \Lambda_y.
\eeq
Let us analyze these beta functions in the weak-coupling regime
with 
$\alpha \lesssim 1$ and low-energy limit
with $\mu\ll \Lambda_y$, together with
a large but finite $N$.
The first term 
in $d\alpha/d\ln \mu$
is the lowest-order term in $1/N$ and $\tilde \Lambda_y$-independent.
As the scale $\mu$ is lowered, 
it makes the 
gauge coupling decrease logarithmically through screening.
If the initial coupling $\alpha_0$ defined at energy scale $\mu_0$ satisfies $(\alpha_0^3/N)\cdot\ln  (\Lambda_y/\mu_0) \lesssim 1$, this term dominates and
the gauge coupling flows to zero at low energies.
On the other hand, 
for $(\alpha_0^3/N)\cdot\ln (\Lambda_y/\mu_0) \gtrsim 1$,
the second term dominates, 
which tends to enhance the coupling at low energies.
In this case,
we can ignore the first term to the leading order.
Then
the gauge coupling grows as $\alpha = \alpha_0\left[1-\frac{16}{3\pi^2 N^2}\alpha_0^4\left( \ln^2\left(\frac{\mu}{\Lambda_y}\right)-\ln^2\left(\frac{\mu_0}{\Lambda_y}\right)\right)\right]^{-1/4}$.
This solution shows a divergence of the gauge coupling with decreasing $\mu$, although it cannot be trusted in the strong coupling regime.
For theories defined at scale $\mu_0$,
the basin of attraction for the $\alpha=0$ fixed point  
is given by
$
{\cal B}_{\mu_0}
\equiv
\{
\alpha_0 | \alpha_0^3 < c N/ \ln (\Lambda_y/\mu_0) 
\}$,
with $c$ an $O(1)$ constant (see the shaded region in Fig. \ref{fig: Flow}).
The salient feature is that 
${\cal B}_{\mu_0}$
shrinks to a measure-zero set   in the low-energy limit 
(\ie $\mu_0\ll\Lambda_y$), due to the scale dependence in the beta function.
The fact that the beta function explicitly depends on $\Lambda_y$ is a hallmark of UV/IR mixing.

\begin{figure}[h]
\centering
\includegraphics[width=0.5\textwidth]{Flow.pdf}
\caption{
The flow of $\alpha=e^2/(4\pi)$ with initial condition $\alpha=\alpha_0$ at $\mu=\mu_0$. For each $\mu_0$, there is a critical value $\alpha^*\approx N/ \ln (\Lambda_y/\mu_0)$ (the dashed curve) :
 when $\alpha_0<\alpha^*$ the gauge coupling flows to zero at low energies (green), while when $\alpha>\alpha^*$ it flows to infinity (orange).}
\label{fig: Flow}
\end{figure}


In the presence of the UV/IR mixing, the FS size cannot be dropped in low-energy physical observables.
For example,
the single-fermion spectral function takes the form of
$\mc{A}(\omega, \vec k, T)
=
\omega^\Delta f\left(\frac{\omega}{k^z_\parallel},  \frac{k_\parallel}{k_F^{z'}}, \frac{\omega}{T}\right)$, 
where $k_F$ is the FS size, $k_{\parallel}$ is the distance of $\vec k$ away from the FS, $T$ is the temperature, $\Delta$, $z$ and $z'$ are critical exponents ($z'=2$ from Eq. \eqref{eq: double log}), and $f$ is a universal function  \cite{Mandal2015}.
It is interesting to test this in CFLs at various filling factors that can be realized in Chern bands \cite{Zou2020a}. 

The UV/IR mixing in MFLs also has implications for the $\epsilon$-expansion  scheme \cite{Nayak1994,Mross2010}, which has been used to approach nFLs 
with $\epsilon =1$ from MFLs with $\epsilon=0$ perturbatively in $\epsilon$.
To see it, we examine
how the UV/IR mixing in the base theory 
with $\epsilon=0$ affects the perturbative $\epsilon$-expansion.
In theories with $\epsilon > 0$, the UV/IR mixing disappears since the diagrams
in Fig. \ref{FD: double log diagrams}
are no longer divergent in $\Lambda_y$, as large-momentum scatterings are  further suppressed.
Instead, the double-log  
in Eq. \eqref{eq: double log}
is translated to a double pole in $\epsilon$ as{\footnote{See Ref. \cite{Holder2015} for a different but related calculation at $\epsilon=1$.}}
\beq\label{eq: double log epsilon}
\begin{split}
\Pi(k_\tau=0, &k_y)\sim \frac{1}{N}\frac{|k_y|}{\tilde e^2}\frac{8\tilde\alpha^4}{\pi^2}\times\\
&\left( \frac{1}{27\epsilon^2}+\frac{1}{9\epsilon}\ln(\mu/|k_y|)+\frac{1}{6}\left(\ln(\mu/|k_y|)\right)^2\right),
\end{split}
\eeq
where
$\tilde e^2 = e^2\mu^{-\epsilon}$ is  the dimensionless coupling and $\tilde\alpha=\tilde e^2/(4\pi)$. 
Since $1/\epsilon$-poles cannot be absorbed by terms already present in the action,
the naive perturbative expansion appears ill-defined. Moreover, this singular self-energy suggests that $\epsilon$ is renormalized to a larger value, further indicating that the $\epsilon$-expansion may break down.
This calls for alternative control schemes for nFLs.
See  Refs. \cite{Dalidovich2013,
Sur2014,
Lunts2017}
for the dimensional regularization scheme that has no UV/IR mixing,
and Refs. \cite{Damia2019, Aldape2020, Kim2020, Esterlis2021}
for other proposals.


{\it Summary and discussion.} 
We provide strong evidence that a $(2+1)$-dimensional MFL exhibits UV/IR mixing, caused by virtual Cooper pairs that spread over the entire FS due to large-momentum scatterings.
Our finding suggests the breakdown of the patch theory for MFLs
and a potential issue
in the $\epsilon$-expansion that uses MFLs as the base theory for nFLs.

We conclude with a few final remarks.
First, the UV/IR mixing identified in $(2+1)$-dimensional MFLs can be extended to more general cases.
Consider metals with $m$-dimensional FS (\eg a spherical or cylindrical FS has $m=2$ and a Weyl nodal line has $m=1$) coupled to a critical boson whose kinetic energy goes as $k_y^{1+\epsilon}$. The contribution of virtual Cooper pairs to loop corrections scales as $\sim\int\frac{d^mk_y}{k_y^{1+\epsilon}}$, which suggests that for
$m \geqslant 1+\epsilon$,
there exist UV divergences associated with the extended size of FS,
and UV/IR mixing can arise.
So we expect virtual-Cooper-pair-induced UV/IR mixing in $(3+1)$-dimensional gauge theories with $m=2$ and $\epsilon=1$ 
(on top of the UV/IR mixing identified in Ref. \cite{Mandal2015}).
This is relevant to quantum spin liquids \cite{Zhou2008, Lawler2008} and mixed-valence insulators \cite{Chowdhury2017}\footnote{
Since $\epsilon=1$ corresponds to local kinetic term for gauge boson, we expect that UV/IR mixing identified here does not originate from the non-analyticity of the kinetic term of the gauge boson.}.

Second, our result is obtained within the standard patch theory. 
To understand the  full consequences of
the UV/IR mixing 
caused by large-momentum scatterings, 
one should consider a general theory that keeps 
track of how the boson-fermion coupling is renormalized at {\it large} boson momenta.
For this, instead of a coupling constant, one should take into account a {\it momentum-dependent coupling function}, reminiscent of the familiar form factors in the interaction vertices in various settings \cite{Parameswaran2013}. Moreover, the four-fermion couplings, which should also be described by a coupling function, are not considered here, but they should in principle be studied on equal footing as the gauge coupling. Whether there is UV/IR mixing can depend on the microscopic details of the physical system.
What we have shown is the presence of a UV/IR mixing in systems where the two effects above are negligible. 
In the future, it will be of great interest to understand whether such UV/IR mixing exists in systems where these effects are significant and should be incorporated into the theory.
In any case, our results suggest that the physics of MFLs is richer than originally expected,
and mandates a qualitative 
improvement of the current theoretical understanding.


{\it Acknowledgement.} 
We thank Tobias Holder, Subir Sachdev and T. Senthil for helpful discussion. W.Y. would like to especially thank Timothy Hsieh for introducing the topic of non-Fermi liquid, Haoran Jiang for explaining double-log behavior in phenomenology and Minyong Guo for hospitality during his stay in Beijing where the bulk calculation was carried out. 
Research at Perimeter Institute is supported in part by the Government of Canada through the Department of Innovation, Science and Economic Development Canada and by the Province of Ontario through the Ministry of Colleges and Universities.
S.L. acknowledges the support of the Natural Sciences and Engineering Research Council of
Canada.

\bibliography{main.bib} 

\end{document}
