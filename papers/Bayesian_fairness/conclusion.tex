\section{Conclusion and future directions}
\label{sec:conclusion}

%Our central thesis is that, even though 
Existing fairness criteria may be hard to satisfy or verify in a
learning setting because they are defined for the true model. For that
reason, we develop a Bayesian fairness framework, which 
deals explicitly with the information available to the decision maker.
Our framework allows us to more adequately incorporate uncertainty into fairness considerations.
%\goran{I would remove the sentence that follows and add that our framework allows us to
%%%more adequately incorporate uncertainty into fairness considerations.}
%This allows us
%to make some new connections between, for example, similarity and
%independence definitions. %\dcp{don't we want to just use `similarity' as a conceptual
%connection rather than something to separately satisfy?}
% to some degree.
We believe that a further exploration of the informational aspects of
fairness, and in particular for sequential decision problems in the Bayesian setting, will be extremely
fruitful.

%\dcp{drop} Although in this paper we have not expanded upon the notions of
%individual versus set fairness, we believe it is an important aspect
%of fairness that has been hitherto neglected. %For that reason, further
%research into the tradeoffs that can appear in that setting is \yang{XXX}.



%%% Local Variables:
%%% mode: latex
%%% TeX-master: "subjective-fairness-aistats18.tex"
%%% End:
