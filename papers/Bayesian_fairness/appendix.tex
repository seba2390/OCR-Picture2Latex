
\newpage
\section*{\LARGE Supplementary materials for ``Bayesian Fairness''}

\setcounter{section}{0}
%\setcounter{theorem}{0}

~\\

\section{Impossibility result}

\begin{theorem}\label{thm:impossible}
Calibration and balance conditions cannot hold simultaneously, except: (i) if there exists perfect decision rules that there exists $a,y$ s.t. $ P_\param^\pol(y\mid a) = 0 $ or $ P_\param^\pol(a\mid y) = 0 $, or (ii) $z$ is independent of $y$ that for each $z$, $ P_\param^\pol(z|y) \equiv \text{const.},~\forall y$.
\end{theorem}
\begin{proof}
We prove by contradiction. Using Bayes rule we first have
\begin{align}
    P_\param^\pol(a, z \mid y) =     P_\param^\pol(y, z \mid a) \cdot \frac{P_\param^\pol(a \mid y)}{P_\param^\pol(y \mid a)} ~.\label{eqn:1}
\end{align}
Suppose calibration condition holds, that is 
\[
P_\param^\pol(y, z \mid a) = P_\param^\pol(y \mid a) P_\param^\pol(z \mid a)
\]
Plug above into Eqn. (\ref{eqn:1}) we have 
\begin{align*}
    P_\param^\pol(a, z \mid y) &=    P_\param^\pol(y \mid a) P_\param^\pol(z \mid a) \cdot \frac{P_\param^\pol(a \mid y)}{P_\param^\pol(y \mid a)} \\
    &=P_\param^\pol(z \mid a) \cdot P_\param^\pol(a \mid y).
    \end{align*}
    On the other hand, if balanced condition holds too, we have
    \begin{align*}
    P_\param^\pol(a, z \mid y) =P_\param^\pol(a \mid y) \cdot P_\param^\pol(z \mid y) 
    \end{align*}
    Together we have that 
    \[
  P_\param^\pol(z \mid a) \cdot P_\param^\pol(a \mid y)  = P_\param^\pol(a \mid y) \cdot P_\param^\pol(z \mid y) \rightarrow   P_\param^\pol(z \mid a) =P_\param^\pol(z \mid y),
    \]
    which does not hold when condition (ii) does not hold, completing the proof. 
\end{proof}

~\\

\section{Trivial decision rules for balance}
\label{sec:counterexample}
 
\begin{theorem}
  A trivial decision rule of the form $\pol(a \mid x) = p_a$ can always satisfy balance for a Bayesian decision problem. However, it may be the only balanced decision rule, even when a non-trivial balanced policy can be found for every possible $\param \in \Param$.
  \label{lem:trivial-balance}
\end{theorem}
\begin{proof}
  For the first part, notice that Eqn. \eqref{eq:balanced-rule}
  can be always satisfied trivially if $\pol(a \mid x) = p_a$, i.e. we ignore the observations when taking our actions.
  For the second part, we can rewrite Eqn. \eqref{eq:balanced-rule} as
  \begin{align*}
    \sum_x \pol(a | x) \left[P_\theta(x, z | y) - P_\theta(x | y) P_\theta(z | y) \right] &= 0\\
    \sum_x \pol(a | x) \Delta_\theta(x, y, z) &= 0,
  \end{align*}
  where the $\Delta$ term is only dependent on the parameters.  This
  condition can be satisfied in two ways: the first is if the model
  $\param$ makes $x, z$ conditionally independent on $y$. The second
  is if the vector $\pol(a \mid \cdot)$ is orthogonal to
  $\Delta_\param(\cdot, y, z)$. If $|\CX| > |\CY \times \CZ|$, then,
  for any $\param$, we can always find a policy vector
  $\pol(a \mid \cdot)$ that is orthogonal to all vectors
  $\Delta_\param(\cdot, y, z)$. However, if these vectors across $\param$ have exactly degree of freedom being 1 (since they add up to 0, thus the rank of them can be at most the full rank - 1), then no single policy can be orthogonal
  to all, as otherwise the degree of freedom for this set of vectors will be at least 2. 
\end{proof}


\begin{example}
  In this balance example, there are two models. In the first model, for some value $y$, we have:
  \begin{align}
    P_{\param}(x=0|y) &= 1/4, &
                                P_{\param}(x=0|y,z=1) &= 1/4 - \epsilon, \\
    P_{\param}(x=1|y) &= 1/4, &
                                P_{\param}(x=1|y,z=1) &= 1/4 + \epsilon, \\
    P_{\param}(x=2|y) &= 1/4, &
                                P_{\param}(x=2|y,z=1) &= 1/4 + \epsilon, \\
  \end{align}
  so that
  \begin{align}
    P_{\param}(x=0|y) -  P_{\param}(x=0|y,z=1) &= \epsilon, \\
    P_{\param}(x=1|y) - P_{\param}(x=1|y,z=1) &=  - \epsilon\\
    P_{\param}(x=2|y) - P_{\param}(x=2|y,z=1) &=  - \epsilon
  \end{align}
  Similarly, we can construct models $\param'$ and $\param''$ so that the corresponding differences are $(-\epsilon, \epsilon, -\epsilon)$ and $(\epsilon, \epsilon, -\epsilon)$.
  For any policy $\pol(a \mid x)$ consider the vector $\pol_a = (\pol(a = 1 \mid x))_{x=1}^3$. Note that we can make the policy orthogonal to the first model simply by setting $\pol_a = (1/2, 1/2, 1)$.
\end{example}

~\\

\section{Proof of Theorem \ref{noise:model}}
\begin{proof}
We show the proof for Bayes-balance condition, while the proof for Marginal-balance resembles similarities. Denote the $(1-\delta)$-event that $\theta$ drawn from $\beta(\theta)$ that is $\epsilon$ close to the true model $\theta^*$ in all the conditional probabilities $P_{\theta}(x|y,z), P_{\theta}(x|y)$ as $\mathcal E$, then we have:
%\begin{align*}
%  &~~~\sum_x \pol(a | x)
%  \int_{\mathrlap{\Param }}
%    \left[P_\param(x, z | y)
%  - P_\param(x | y) P_\param(z | y) \right] \\
%  &=\int_{\theta \in \mathcal E} \sum_x \pol(a | x)
%    \left[P_{\param}(x, z | y)
%  - P_{\param}(x | y) P_{\param}(z | y) \right]\\
%  &+  \int_{\theta \notin \mathcal E}\sum_x \pol(a | x) \left[P_\param(x, z | y)
%  - P_\param(x | y) P_\param(z | y) \right] \\
% & \leq \int_{\theta \in \mathcal E} \sum_x \pol(a | x)
%    \left[P_{\param^*}(x, z | y)
%  - P_{\param^*}(x | y) P_{\param^*}(z | y) \right] +2\epsilon\\
%  &+\int_{\theta \notin \mathcal E}\sum_x \pol(a | x) \left[P_{\param^*}(x, z | y)
%  - P_{\param^*}(x | y) P_{\param^*}(z | y) \right] + 2\delta\\
%&  \leq \alpha+2(\epsilon+\delta).% \alpha \sum_x \pol(a | x) = \alpha.
%\end{align*}
\begin{align*}
  &~~~~~\bigl|\sum_x \pol(a | x)
    \left[P_{\param^*}(x, z | y)
  - P_{\param^*}(x | y) P_{\param^*}(z | y) \right]\bigr|\\
  &= \biggl|\int_{\theta \in \mathcal E} \sum_x \pol(a | x)
    \left[P_{\param^*}(x, z | y)
  - P_{\param^*}(x | y) P_{\param^*}(z | y) \right]\\
  &+\int_{\theta \notin \mathcal E}\sum_x \pol(a | x) \left[P_{\param^*}(x, z | y)
  - P_{\param^*}(x | y) P_{\param^*}(z | y) \right]\biggr|\\
  &\leq \biggl|\int_{\theta \in \mathcal E} \sum_x \pol(a | x)
    \left[P_{\param}(x, z | y)
  - P_{\param}(x | y) P_{\param}(z | y) \right] +2\epsilon\\
  &+\int_{\theta \notin \mathcal E}\sum_x \pol(a | x) \left[P_{\param^*}(x, z | y)
  - P_{\param}(x | y) P_{\param}(z | y) \right] + 2\delta\biggr|\\
  &\le \bigl |\sum_x \pol(a | x)
  \int_{\mathrlap{\Param }}
    \left[P_\param(x, z | y)
  - P_\param(x | y) P_\param(z | y) \right]\bigr| +  2(\epsilon+\delta).
  \end{align*}
  Summing over all $a,y,z$ gives us the results. 
\end{proof}
~\\

\section{Gradient calculations for optimal balance decision}
\label{sec:gradient}

For simplicity, let us define the vector in $\Simplex^{\CA}$:
\[
c_w(y,z) = \sum_x \pol_w(\cdot \mid x) \Delta(x, y, z),
\]
so that
\[
f_\lambda(w) = \util(\bel, \pol_w) -  \lambda \sum_{y,z} c_w(y,z)^\top c_w(y,z).
\]
Now
\begin{align*}
  &~~~~\grad_w \left(c_w(y,z)^\top c_w(y,z)\right)
  \\
  &= 
  \grad_w \sum_a c_w(y,z)_a^2\\
  &= 
  \sum_a 2 c_w(y,z)_a
  \grad_w c_w(y, z)_a
  \\
  \grad_w & c_w(y, z)_a
  = 
    \sum_x \grad_w \pol_w(a \mid x) \Delta(x, y, z),
\end{align*}
while
\begin{align}
  \grad \util(\bel, \pol_w)
  &=
    \int_\CX \dd \Pr_\bel(x) \grad_w \pol_w(a \mid x) \E_\bel (\util \mid x, a) 
\end{align}
Combining the two terms, we have
\begin{align*}
  \grad_w f_\lambda(w)
  &= 
    \int_\CX \grad_w \pol_w(a \mid x)
    \bigl[
    \dd \Pr_\bel(x) \E_\bel(U \mid x, a)
  \\
  &- 2 \lambda \sum_{y,z} c_w(y, z)_a \Delta(x, y, z) \dd \Lambda(x),
    \bigr].
\end{align*}
where $\Lambda$ is the Lebesgue measure.
We now derive the gradient for the $\grad_w \pol_w$ term. We consider
two parameterizations.
\paragraph{Independent policy parameters.} When $\pol(a \mid x) = w_{ax}$, we obtain
$$\partial \pol(a' \mid x') / \partial ax = \ind{ax = a'x'}$$.
This unfortunately requires projecting the policy parameters back to  the simplex. For this reason, it might be better to use a parameterization that allows unconstrained optimization. 
\paragraph{Softmax policy parameters.} When 
$$\pol(a \mid x) = e^{w_{ax}} / \sum_{a'} e^{w_{a'x}},$$
we have the following gradients:
\begin{align*}
  \partial \pol(a \mid x) / \partial ax
  &= 
    e^{w_{ax}} \sum_{a' \neq a} e^{w_{a'x}} \left(\sum_{a'} e^{w_{a'x}}\right)^{-2}
  \\
  \partial \pol(a \mid x) / \partial a'x
  &= 
    e^{w_{ax} + w_{a'x}}\left(\sum_{a''} e^{w_{a''x}}\right)^{-2}, ~ a \neq a'\\
  \partial \pol(a \mid x) / \partial a'x'
  &= 
    0, ~ ax \neq a'x'.
\end{align*}

~\\


\section{Empirical formulation.}
For infinite $\CX$, it may be more efficient to rewrite
\eqref{eq:balanced-bayes-constraint} as
\begin{align}
  0
  &=
    \int_{\CX} \pol(a \mid x) 
    \dd \left[ P(x, z \mid y)
    - P(x \mid y) P(z \mid y)\right]
  \\
  &= 
    \int_{\CX} \pol(a \mid x) 
    \left[ P(z \mid y, x)
    - P(z \mid y)\right] \dd P(x \mid y)\\
  &=
    \int_{\CX} \pol(a \mid x) 
    \left[ P(z \mid y, x)
    - P(z \mid y)\right] \frac{P(y \mid x)}{P(y)} \dd P(x)\\
  &\approx
    \sum_{x \sim P_\param(x)} \pol(a \mid x) 
    \left[ P(z \mid y, x)
    - P(z \mid y)\right] \frac{P(y \mid x)}{P(y)}
\intertext{simplifying by dropping the $P(y)$ term:}
  0 &\approx
    \sum_{x \sim P_\param(x)} \pol(a \mid x) 
    \left[ P(z \mid y, x)
    - P(z \mid y)\right] P(y \mid x),
\end{align}
This allows us to approximate the integral by sampling $x$, and can be
useful for e.g. regression problems.

\section{Complete figures}
This section has complete versions of the figures which could not fit in the main text.
\begin{figure*}
\centering
 \subfloat[$\lambda=0$]{
   % This file was created by matlab2tikz.
%
%The latest updates can be retrieved from
%  http://www.mathworks.com/matlabcentral/fileexchange/22022-matlab2tikz-matlab2tikz
%where you can also make suggestions and rate matlab2tikz.
%
\definecolor{mycolor1}{rgb}{0.00000,0.75000,0.75000}%
%
\begin{tikzpicture}

\begin{axis}[%
width=0.951\fwidth,
height=0.75\fwidth,
at={(0\fwidth,0\fwidth)},
scale only axis,
xmode=log,
xmin=1,
xmax=100,
xminorticks=true,
xlabel={t},
ymin=4.5,
ymax=8,
ylabel={$U,F$},
axis background/.style={fill=white}
]
\addplot [color=blue, line width=2.0pt, forget plot]
  table[row sep=crcr]{%
1	5.43561523059217\\
2	5.90370595362012\\
3	5.86193877425643\\
4	6.1974251447707\\
5	6.37538745144895\\
6	6.29084098710727\\
7	6.37533818756081\\
8	6.50213010808347\\
9	6.5991724386949\\
10	6.52046523295858\\
11	6.72543938736623\\
12	6.79086221447693\\
13	6.73720035131321\\
14	6.73338708212267\\
15	6.72543680895541\\
16	6.69340595276163\\
17	6.7374401920428\\
18	6.76459636329051\\
19	6.81917107124367\\
20	6.81961739074537\\
21	6.80848937741292\\
22	6.76496273549776\\
23	6.81943224558462\\
24	6.82707956663492\\
25	6.82513922838762\\
26	6.80872080097762\\
27	6.8194279261825\\
28	6.8189260277632\\
29	6.80836851035401\\
30	6.80853114441057\\
31	6.81931954445528\\
32	6.73856115284411\\
33	6.73816264740434\\
34	6.73350236577814\\
35	6.70687112292105\\
36	6.6785416628419\\
37	6.67686803452494\\
38	6.73847232580075\\
39	6.81942547936638\\
40	6.82688492791769\\
41	6.82666694620967\\
42	6.82685885290022\\
43	6.82704019063763\\
44	6.82689924230517\\
45	6.82677823960367\\
46	6.8270597587393\\
47	6.8194560097514\\
48	6.82689843014005\\
49	6.82701602901863\\
50	6.82681169930566\\
51	6.81949797532045\\
52	6.82669354234299\\
53	6.81943329511497\\
54	6.82651491512469\\
55	6.82704770792639\\
56	6.82679743590289\\
57	6.82658758155496\\
58	6.82669828218253\\
59	6.82660948782845\\
60	6.82669798274291\\
61	6.8265894448791\\
62	6.82677142197074\\
63	6.82681491374407\\
64	6.82666114793132\\
65	6.80855816288338\\
66	6.81943481045244\\
67	6.82660396672482\\
68	6.82386355702123\\
69	6.81947471789097\\
70	6.82700074175287\\
71	6.82693449357647\\
72	6.82709850214994\\
73	6.82660499877464\\
74	6.82697649873285\\
75	6.82670593695921\\
76	6.82675415174255\\
77	6.82687238452097\\
78	6.82695391513762\\
79	6.82698434483159\\
80	6.82686631929662\\
81	6.82688791087935\\
82	6.82646811164595\\
83	6.82644252014829\\
84	6.82665183718036\\
85	6.82669758685249\\
86	6.82676665652696\\
87	6.82695751973057\\
88	6.82658946086894\\
89	6.82655885737069\\
90	6.82664771118609\\
91	6.82677188302957\\
92	6.82669723134228\\
93	6.82649131899508\\
94	6.82694362296807\\
95	6.82701100800417\\
96	6.82689276186491\\
97	6.82697942973237\\
98	6.8265959260873\\
99	6.82685566499548\\
100	6.82661794251553\\
};
\addplot [color=black!50!green, dashed, line width=3.0pt, forget plot]
  table[row sep=crcr]{%
1	5.41383459952623\\
2	6.03977740818528\\
3	6.34445896962323\\
4	6.36360081485984\\
5	6.5327140246455\\
6	6.54312011538435\\
7	6.60180812138142\\
8	6.73373741924754\\
9	6.73566894837812\\
10	6.79427286043087\\
11	6.79651547742057\\
12	6.81745557776205\\
13	6.82361089403465\\
14	6.82453008513155\\
15	6.81832088322606\\
16	6.78744690174988\\
17	6.8245307167606\\
18	6.82360223044389\\
19	6.82320123869577\\
20	6.82664800772996\\
21	6.82680811620926\\
22	6.82647925470442\\
23	6.82692086547104\\
24	6.8269265759928\\
25	6.82693425257449\\
26	6.8268229060682\\
27	6.82700552417735\\
28	6.82667175358748\\
29	6.82668179704755\\
30	6.82696045898621\\
31	6.82660833511297\\
32	6.82680589726841\\
33	6.82671589421294\\
34	6.82682294359754\\
35	6.82701707308604\\
36	6.82650493640409\\
37	6.82699220517708\\
38	6.82668021066387\\
39	6.82672209879424\\
40	6.82681624339904\\
41	6.82673440179816\\
42	6.82689211021934\\
43	6.82677720152432\\
44	6.82694542985608\\
45	6.82666056532852\\
46	6.82671478552557\\
47	6.82693547661731\\
48	6.82672478318047\\
49	6.82677490664466\\
50	6.82656216462366\\
51	6.82648634438907\\
52	6.82704568401014\\
53	6.826862274337\\
54	6.82707378292588\\
55	6.82701064453263\\
56	6.82655872230876\\
57	6.82680284995162\\
58	6.82675484205814\\
59	6.82669898677491\\
60	6.8269587402534\\
61	6.82659491807925\\
62	6.8270850859566\\
63	6.82673295666194\\
64	6.82699550615579\\
65	6.82685223576769\\
66	6.82671348441515\\
67	6.82673202844262\\
68	6.82712820153405\\
69	6.82689184429809\\
70	6.82683272593571\\
71	6.82648201507614\\
72	6.82674866309257\\
73	6.82671747591638\\
74	6.82699024215261\\
75	6.82717941910823\\
76	6.82694493886605\\
77	6.8267239335274\\
78	6.82696730470131\\
79	6.82694340122485\\
80	6.82682976358288\\
81	6.82660119141053\\
82	6.82686412764447\\
83	6.82699084765003\\
84	6.8267722735916\\
85	6.82674429780571\\
86	6.82694302876559\\
87	6.82675714405635\\
88	6.82696031751509\\
89	6.82689255300199\\
90	6.82702023326717\\
91	6.82696488326549\\
92	6.82660850972417\\
93	6.8268004525055\\
94	6.82679611886495\\
95	6.82682594191007\\
96	6.82686735298234\\
97	6.82672601577022\\
98	6.82680403091386\\
99	6.82704486981606\\
100	6.8267221923614\\
};
\addplot [color=red, dashdotted, line width=2.0pt, forget plot]
  table[row sep=crcr]{%
1	4.65679844872143\\
2	5.69596508675862\\
3	6.12283278302887\\
4	6.64373241409889\\
5	7.32558978567296\\
6	6.08163155109078\\
7	6.19748304313167\\
8	6.87388547503056\\
9	7.08389848708165\\
10	7.66590296789734\\
11	7.53380043435034\\
12	7.62021550620541\\
13	6.55198691005539\\
14	7.31957601971931\\
15	7.35795487617778\\
16	6.56844918258317\\
17	6.60268271810485\\
18	6.93824349667414\\
19	6.54115540687113\\
20	6.91918785115645\\
21	7.21517127808337\\
22	7.13583366054737\\
23	6.66321201520444\\
24	6.98638303153576\\
25	6.78324605322824\\
26	7.38817377199521\\
27	6.63843044254195\\
28	6.75672123473148\\
29	7.3764343870945\\
30	7.19570861786257\\
31	6.65820831042757\\
32	6.9430024004874\\
33	6.85869875035141\\
34	6.84454259485917\\
35	6.98880616760368\\
36	7.08431244310177\\
37	7.25857722575256\\
38	6.98686399745706\\
39	6.6537099802175\\
40	6.7214629047179\\
41	6.57818486200004\\
42	6.78856195660524\\
43	6.98657442393879\\
44	6.70851618145228\\
45	6.77231060119333\\
46	6.91138081525664\\
47	6.80900520532078\\
48	6.8955319793459\\
49	6.84792233580796\\
50	6.79275017251076\\
51	6.71426555740713\\
52	6.59004662459356\\
53	6.77652926179091\\
54	6.65083295234543\\
55	6.85126864447383\\
56	6.74854673311024\\
57	6.75003213226614\\
58	6.89126758286218\\
59	6.82121018801547\\
60	6.73429925858514\\
61	6.55866795761166\\
62	6.77542717061313\\
63	6.7505053995152\\
64	6.61001242197726\\
65	7.25454977658437\\
66	6.66630203216666\\
67	6.49701316241904\\
68	6.74842532476086\\
69	6.71504163949096\\
70	6.97439204366755\\
71	6.76367082195242\\
72	6.96309910967014\\
73	6.82447935225019\\
74	6.97808966701071\\
75	6.92487707804123\\
76	6.66720140412107\\
77	6.79555773655483\\
78	6.8885209752185\\
79	6.82374109038333\\
80	6.72908264481633\\
81	6.86784379199626\\
82	6.63745368452247\\
83	6.5131420448435\\
84	6.88807751914145\\
85	6.72286398044753\\
86	6.73792752347492\\
87	6.93224164162795\\
88	6.837553331591\\
89	6.70329794971067\\
90	6.77161494678466\\
91	6.92391725886576\\
92	6.64034499050567\\
93	6.80661355109981\\
94	6.90661745847731\\
95	6.90906010305707\\
96	6.73117108750913\\
97	6.96005426572143\\
98	6.54290590591496\\
99	6.69200091471541\\
100	6.67305945104694\\
};
\addplot [color=mycolor1, dotted, line width=3.0pt, forget plot]
  table[row sep=crcr]{%
1	4.63856673683556\\
2	7.65090449353503\\
3	7.29720941968353\\
4	6.39411547781226\\
5	5.59180859447572\\
6	5.91103616867134\\
7	5.65345033980548\\
8	6.99157334254097\\
9	6.98378088352546\\
10	6.36878678535445\\
11	6.5703513033626\\
12	6.72120527280937\\
13	7.18827355673923\\
14	7.07400805356152\\
15	7.07676946422607\\
16	6.68918545343691\\
17	6.87716882988162\\
18	7.21295461204102\\
19	6.89059757703387\\
20	6.6418700279056\\
21	6.72969261054924\\
22	6.58527483721895\\
23	6.84422565408741\\
24	6.76291708398274\\
25	6.78586446833003\\
26	6.97929349712728\\
27	6.91929251936603\\
28	6.59430908038756\\
29	6.82168411870536\\
30	6.86077963582398\\
31	6.71389518734896\\
32	6.78498708606475\\
33	6.79899086704355\\
34	6.85315001850439\\
35	6.96486748159995\\
36	6.73627790220896\\
37	6.8592233857706\\
38	6.56505333926661\\
39	6.73635246282633\\
40	6.75269318472154\\
41	6.67726773578806\\
42	6.90187689510759\\
43	6.78042022470713\\
44	6.77019613843506\\
45	6.73856118706226\\
46	6.57185409950028\\
47	6.96761071311978\\
48	6.66130453339595\\
49	6.80708379997048\\
50	6.85748230435172\\
51	6.70135826348809\\
52	6.9938569734353\\
53	6.72539117699799\\
54	6.94640793627233\\
55	6.9385051726723\\
56	6.61631698807355\\
57	6.788520931626\\
58	6.88572558664489\\
59	6.68380106971717\\
60	6.86253282851936\\
61	6.65810790284873\\
62	6.8793638877059\\
63	6.78922309796063\\
64	6.94456473813911\\
65	6.70260082131938\\
66	6.72230045188807\\
67	6.69964002067404\\
68	7.00732110672733\\
69	6.81079303773709\\
70	6.68424630783847\\
71	6.52353247112277\\
72	6.73012734112837\\
73	6.69496855949307\\
74	6.92703451903552\\
75	7.02396178656437\\
76	6.81686683630093\\
77	6.59425523559705\\
78	6.95528299863492\\
79	6.75296627805458\\
80	6.78184138131174\\
81	6.63659860007914\\
82	6.88566575528246\\
83	6.92879059314807\\
84	6.64710033562532\\
85	6.65151555001541\\
86	6.9180522788148\\
87	6.71736971916958\\
88	6.90775109592753\\
89	6.89321397542376\\
90	6.80705343437269\\
91	6.93499379578587\\
92	6.59613086758874\\
93	6.86088232672981\\
94	6.67935174396831\\
95	6.81958814386339\\
96	6.78885095717227\\
97	6.62347677209479\\
98	6.66807026959861\\
99	6.99366307038507\\
100	6.7131226265173\\
};
\end{axis}
\end{tikzpicture}%
  }
  \subfloat[$\lambda=0.25$]{
    % This file was created by matlab2tikz.
%
%The latest updates can be retrieved from
%  http://www.mathworks.com/matlabcentral/fileexchange/22022-matlab2tikz-matlab2tikz
%where you can also make suggestions and rate matlab2tikz.
%
\definecolor{mycolor1}{rgb}{0.00000,0.75000,0.75000}%
%
\begin{tikzpicture}

\begin{axis}[%
width=0.951\fwidth,
height=0.75\fwidth,
at={(0\fwidth,0\fwidth)},
scale only axis,
xmode=log,
xmin=1,
xmax=100,
xminorticks=true,
xlabel={t},
ylabel={$U, F$},
ymin=3,
ymax=8,
axis background/.style={fill=white}
]
\addplot [color=blue, line width=2.0pt, forget plot]
  table[row sep=crcr]{%
1	5.48092030379997\\
2	5.62187002690099\\
3	5.95863155946218\\
4	6.28898882529671\\
5	6.37263646084564\\
6	6.44619863535187\\
7	6.51978070369432\\
8	6.6260112803707\\
9	6.59969215942956\\
10	6.64710138462075\\
11	6.64739151299339\\
12	6.64510042770851\\
13	6.64152382540431\\
14	6.77898626490996\\
15	6.78942211106642\\
16	6.78461368559541\\
17	6.78747870326049\\
18	6.78938642097554\\
19	6.78832161718462\\
20	6.78775050690658\\
21	6.78944722876843\\
22	6.78917577314025\\
23	6.78819139814796\\
24	6.78724347966865\\
25	6.78821620436663\\
26	6.78831214850696\\
27	6.7880922368358\\
28	6.7881723236791\\
29	6.78794233482332\\
30	6.78813814586129\\
31	6.78820354351858\\
32	6.78841099428386\\
33	6.78808353541935\\
34	6.7881908225767\\
35	6.78774827696057\\
36	6.7885187243922\\
37	6.78845419838743\\
38	6.78814836495893\\
39	6.78821228904136\\
40	6.78818735684172\\
41	6.78799405430338\\
42	6.7882713816731\\
43	6.78927591624498\\
44	6.78946030227827\\
45	6.78821020841118\\
46	6.78827185230104\\
47	6.78860337850886\\
48	6.78748352584787\\
49	6.78799354670293\\
50	6.78842672785986\\
51	6.7880902860153\\
52	6.78808388647639\\
53	6.78747738661796\\
54	6.78819916095484\\
55	6.78835895289234\\
56	6.78843293590063\\
57	6.78825725240079\\
58	6.78829796145325\\
59	6.788057466024\\
60	6.78839237562291\\
61	6.78834851740735\\
62	6.78735253906824\\
63	6.78775411240988\\
64	6.78863576035963\\
65	6.78755861187669\\
66	6.78954119015546\\
67	6.78873337301612\\
68	6.78816033804725\\
69	6.78820344967906\\
70	6.78776361064147\\
71	6.78814177015364\\
72	6.7879366996741\\
73	6.7880936081835\\
74	6.78825561226921\\
75	6.78821316644409\\
76	6.78918787026003\\
77	6.78818333235925\\
78	6.78819645711209\\
79	6.78810751942358\\
80	6.78842982644288\\
81	6.78739048420254\\
82	6.78823104668009\\
83	6.78781609618644\\
84	6.78820734692313\\
85	6.78828044289443\\
86	6.78722601773148\\
87	6.78755406937019\\
88	6.78811157856524\\
89	6.78808107645364\\
90	6.789012996106\\
91	6.7881650039973\\
92	6.78821720889247\\
93	6.78807168410838\\
94	6.78810800021703\\
95	6.78793379320975\\
96	6.78826043758827\\
97	6.78807781437123\\
98	6.78818537206819\\
99	6.78850763801694\\
100	6.7884621883911\\
};
\addplot [color=black!50!green, dashed, line width=3.0pt, forget plot]
  table[row sep=crcr]{%
1	5.46839092366965\\
2	5.75703558072284\\
3	6.21187308663476\\
4	6.39102664381444\\
5	6.55586234089632\\
6	6.38413615637934\\
7	6.42300581162868\\
8	6.67498693237809\\
9	6.63039319016224\\
10	6.58773391003488\\
11	6.75296577578321\\
12	6.786443793693\\
13	6.78805569029723\\
14	6.78745585366746\\
15	6.78832428887165\\
16	6.78830066490763\\
17	6.78839106970021\\
18	6.78808802788535\\
19	6.78798694582777\\
20	6.78817693494755\\
21	6.7881008909547\\
22	6.78842032667698\\
23	6.78783128904056\\
24	6.78875615083487\\
25	6.78845727850269\\
26	6.78819401413074\\
27	6.78834196285126\\
28	6.7872101033022\\
29	6.7879980507075\\
30	6.78824430728795\\
31	6.78816970586208\\
32	6.78818801908122\\
33	6.78814017664448\\
34	6.78825078323049\\
35	6.7881484859051\\
36	6.78818691493551\\
37	6.78808987102091\\
38	6.78816298338296\\
39	6.78799846176474\\
40	6.78828094214078\\
41	6.78727871939925\\
42	6.78803975010666\\
43	6.78725571301125\\
44	6.78871518089493\\
45	6.78840508970577\\
46	6.7875707596091\\
47	6.7881728716072\\
48	6.78846952964538\\
49	6.78817772400779\\
50	6.78840021851439\\
51	6.78822370152359\\
52	6.78740754507343\\
53	6.78818214694285\\
54	6.78826729796029\\
55	6.78794709000275\\
56	6.78819769453097\\
57	6.78773687436855\\
58	6.78760780764942\\
59	6.78820024812653\\
60	6.78822638564773\\
61	6.78816361161253\\
62	6.78832451810031\\
63	6.78840398227284\\
64	6.78810379421081\\
65	6.7882192655374\\
66	6.7882812746018\\
67	6.78767994390469\\
68	6.78789414506478\\
69	6.78833152743326\\
70	6.78812570838307\\
71	6.78853051769768\\
72	6.78820591270197\\
73	6.78840134179175\\
74	6.78833646862341\\
75	6.78714175770295\\
76	6.78827649125506\\
77	6.78761416139175\\
78	6.78818828234791\\
79	6.78820835330531\\
80	6.78869172117444\\
81	6.78888773725641\\
82	6.78807839890497\\
83	6.78822911534438\\
84	6.7882605284278\\
85	6.78824522127325\\
86	6.78813733439761\\
87	6.78828150957633\\
88	6.787818875421\\
89	6.78808023524246\\
90	6.78808690828464\\
91	6.78820378585089\\
92	6.78810395152641\\
93	6.78826222843939\\
94	6.78821516715624\\
95	6.7884244130633\\
96	6.78815086989971\\
97	6.78823200461007\\
98	6.78762008169212\\
99	6.78764861589493\\
100	6.78759288324044\\
};
\addplot [color=red, dashdotted, line width=2.0pt, forget plot]
  table[row sep=crcr]{%
1	3.95263838270111\\
2	3.82503765619641\\
3	4.69915687329073\\
4	4.95179395517991\\
5	5.69532493678198\\
6	5.37334301528087\\
7	4.83235868797524\\
8	4.73554152570934\\
9	5.01931591636941\\
10	5.03432141135122\\
11	4.79254541499617\\
12	4.63914092183212\\
13	4.44750001169128\\
14	4.81685316598652\\
15	4.9147788424582\\
16	4.91327394352937\\
17	4.95995623494322\\
18	4.8945337204222\\
19	4.84879148065828\\
20	4.83909002285821\\
21	4.88259365579378\\
22	4.88389090533673\\
23	4.8443938066954\\
24	4.83840179753397\\
25	4.84558816878397\\
26	4.84905292663818\\
27	4.84147078022581\\
28	4.84372820754065\\
29	4.83890505616955\\
30	4.84232199646838\\
31	4.84523950410094\\
32	4.85078354779625\\
33	4.84135735710565\\
34	4.84520888873435\\
35	4.84215081653731\\
36	4.85535850522084\\
37	4.85185247997884\\
38	4.84260169610015\\
39	4.84620634763515\\
40	4.84523575187968\\
41	4.83902567068879\\
42	4.84701671196802\\
43	4.87609116537311\\
44	4.88206656252824\\
45	4.84539564610091\\
46	4.84731781906082\\
47	4.85657048784665\\
48	4.84464814453533\\
49	4.84034483610184\\
50	4.85098637586876\\
51	4.85234480413716\\
52	4.84223427910482\\
53	4.83645625338907\\
54	4.84534540433531\\
55	4.84989668463203\\
56	4.85152070699286\\
57	4.84710109351825\\
58	4.84850862271226\\
59	4.8417612393451\\
60	4.8505172651455\\
61	4.84854018635913\\
62	4.84166599328614\\
63	4.83795694229326\\
64	4.85812270662427\\
65	4.8507748368422\\
66	4.88437423777575\\
67	4.85926481587375\\
68	4.84382853660602\\
69	4.8450886394879\\
70	4.83777595797139\\
71	4.8436077111013\\
72	4.83723787081233\\
73	4.8419858580276\\
74	4.84700191788378\\
75	4.84679719167575\\
76	4.88422845265261\\
77	4.84492363174228\\
78	4.84542290433572\\
79	4.84308012880218\\
80	4.85170602717477\\
81	4.84406941230317\\
82	4.84661435851207\\
83	4.83788327052865\\
84	4.84567275514356\\
85	4.84723479294532\\
86	4.83903092538253\\
87	4.83854193953482\\
88	4.84165658956618\\
89	4.84096009660066\\
90	4.86812900077183\\
91	4.84417172920604\\
92	4.84595715415907\\
93	4.84059852688428\\
94	4.84207974058152\\
95	4.84058822114876\\
96	4.84739615953176\\
97	4.84066509683728\\
98	4.84408611041905\\
99	4.85435432738002\\
100	4.85140753714343\\
};
\addplot [color=mycolor1, dotted, line width=3.0pt, forget plot]
  table[row sep=crcr]{%
1	4.43942617049761\\
2	6.07517227920279\\
3	6.01511546400837\\
4	7.99001185146673\\
5	6.91853861731467\\
6	7.45736885559621\\
7	5.9696968988519\\
8	5.90044821555372\\
9	6.53253989933909\\
10	5.95793744141482\\
11	4.47853828240861\\
12	5.01187651983101\\
13	4.86994172661999\\
14	4.87457926082996\\
15	4.91605950930105\\
16	4.83664217269984\\
17	4.84487405751951\\
18	4.84572345362514\\
19	4.84496677844009\\
20	4.84859591315895\\
21	4.8439187330126\\
22	4.85436336951604\\
23	4.84095204112765\\
24	4.86334743088784\\
25	4.85229857867871\\
26	4.84514781002158\\
27	4.84965309635877\\
28	4.83928916416435\\
29	4.84189589312187\\
30	4.84704232748438\\
31	4.84549575927436\\
32	4.84486903809857\\
33	4.84236622368815\\
34	4.84766997345754\\
35	4.85758994431681\\
36	4.84579416567247\\
37	4.84096626422101\\
38	4.84505804961238\\
39	4.8440266791363\\
40	4.84775057449601\\
41	4.84209322722535\\
42	4.84050324640196\\
43	4.83980167280306\\
44	4.85923425685094\\
45	4.85111829634059\\
46	4.84132164675338\\
47	4.84370161642999\\
48	4.87223462946916\\
49	4.84432618495737\\
50	4.8500961427862\\
51	4.84586521700634\\
52	4.84466882674989\\
53	4.84385534411402\\
54	4.84618282826888\\
55	4.85639560861603\\
56	4.8456548959247\\
57	4.84125063954534\\
58	4.84248664382494\\
59	4.84531192138565\\
60	4.84571944195174\\
61	4.84383104434909\\
62	4.84870278970119\\
63	4.85126878016323\\
64	4.84210931804219\\
65	4.84625260846576\\
66	4.84761300621132\\
67	4.83951697497941\\
68	4.84891722764123\\
69	4.84990546659528\\
70	4.84234548798647\\
71	4.85552485686154\\
72	4.84534656853895\\
73	4.85214524633418\\
74	4.84833846211763\\
75	4.83836834799854\\
76	4.84747334309443\\
77	4.84577873600088\\
78	4.84566127923588\\
79	4.84656159051625\\
80	4.85932831037586\\
81	4.86458441816834\\
82	4.84154805775504\\
83	4.84631576231663\\
84	4.84721937222264\\
85	4.8458013115131\\
86	4.84218017264925\\
87	4.84713277748063\\
88	4.83723898611049\\
89	4.84187774364254\\
90	4.84171840375588\\
91	4.8452678398018\\
92	4.84182079956507\\
93	4.84699098516083\\
94	4.84549640548173\\
95	4.85158555409688\\
96	4.84409812448365\\
97	4.84640282767493\\
98	4.84023191773009\\
99	4.83639997556977\\
100	4.83899503618132\\
};
\end{axis}
\end{tikzpicture}%
  }
  \subfloat[$\lambda=0.5$]{
    % This file was created by matlab2tikz.
%
%The latest updates can be retrieved from
%  http://www.mathworks.com/matlabcentral/fileexchange/22022-matlab2tikz-matlab2tikz
%where you can also make suggestions and rate matlab2tikz.
%
\definecolor{mycolor1}{rgb}{0.00000,0.75000,0.75000}%
%
\begin{tikzpicture}

\begin{axis}[%
width=0.951\fwidth,
height=0.75\fwidth,
at={(0\fwidth,0\fwidth)},
scale only axis,
xmode=log,
xmin=1,
xmax=100,
xminorticks=true,
xlabel={t},
ymin=2,
ymax=7,
axis background/.style={fill=white}
]
\addplot [color=blue, line width=2.0pt, forget plot]
  table[row sep=crcr]{%
1	5.43626321740328\\
2	5.62955957449209\\
3	5.83892212841107\\
4	6.08455180148673\\
5	6.45389213646402\\
6	6.51404201900693\\
7	6.55425288332807\\
8	6.47841299626561\\
9	6.62069875641757\\
10	6.51422248577047\\
11	6.59838520559784\\
12	6.59032275016397\\
13	6.52849619086474\\
14	6.68240075733416\\
15	6.70628180193069\\
16	6.70366440054659\\
17	6.70078300565614\\
18	6.70264005866184\\
19	6.70272040838255\\
20	6.70215960026927\\
21	6.7035515874975\\
22	6.70405002626636\\
23	6.70380004965903\\
24	6.70280860477349\\
25	6.70188832548547\\
26	6.70154419712915\\
27	6.70290303559077\\
28	6.70099710415468\\
29	6.70193379267451\\
30	6.70120150981987\\
31	6.70173152738914\\
32	6.70159866600691\\
33	6.70148331010774\\
34	6.70024700158092\\
35	6.70162388648996\\
36	6.70122567452083\\
37	6.70119760380918\\
38	6.7008942387301\\
39	6.70207192686816\\
40	6.70206106174765\\
41	6.70118803780591\\
42	6.7011708537815\\
43	6.70121592271\\
44	6.70090701893929\\
45	6.70090952627623\\
46	6.70172791977211\\
47	6.7004604519536\\
48	6.70214417070081\\
49	6.70087927793523\\
50	6.7009280299354\\
51	6.70192571875951\\
52	6.70171886250275\\
53	6.70150820728044\\
54	6.70126159777429\\
55	6.70041242169488\\
56	6.7002853540649\\
57	6.70046583835858\\
58	6.70020648356859\\
59	6.70116548685459\\
60	6.70050519866534\\
61	6.7015615201992\\
62	6.7007685612289\\
63	6.70095050594716\\
64	6.70154935714809\\
65	6.70152349860255\\
66	6.70070897977991\\
67	6.7009292344296\\
68	6.70068914149036\\
69	6.70163551712463\\
70	6.70142476387294\\
71	6.70069739926064\\
72	6.7017336129983\\
73	6.70128008473107\\
74	6.70162543894496\\
75	6.70146658599244\\
76	6.70170456840915\\
77	6.7013768002497\\
78	6.70182944797324\\
79	6.70116360226985\\
80	6.701438682881\\
81	6.70092732271296\\
82	6.70135911617293\\
83	6.70188062229687\\
84	6.70075474316751\\
85	6.70097329900569\\
86	6.70040857565787\\
87	6.70120199869424\\
88	6.70120379847468\\
89	6.70170383955328\\
90	6.70113156603443\\
91	6.70058910840515\\
92	6.70055443379444\\
93	6.70091509145021\\
94	6.70170584002308\\
95	6.70130045964099\\
96	6.70227846487977\\
97	6.70187435109027\\
98	6.70119790741984\\
99	6.70133993723436\\
100	6.70064902183277\\
};
\addplot [color=black!50!green, dashed, line width=3.0pt, forget plot]
  table[row sep=crcr]{%
1	5.51006868459902\\
2	5.94349925651812\\
3	6.15110188956242\\
4	6.02712399624356\\
5	6.32392569534522\\
6	6.36115851019674\\
7	6.36007362082859\\
8	6.42304614951736\\
9	6.46234178919198\\
10	6.46733988080431\\
11	6.45899027408367\\
12	6.3374431395461\\
13	6.54985803215858\\
14	6.3167027540054\\
15	6.46001002676741\\
16	6.56942958143311\\
17	6.61343634104224\\
18	6.63292515036324\\
19	6.63448250199397\\
20	6.63488106426994\\
21	6.63474315974286\\
22	6.68991621855335\\
23	6.6335990864829\\
24	6.69904685642871\\
25	6.70141875769818\\
26	6.7012806526023\\
27	6.70070437469069\\
28	6.70101301806078\\
29	6.70067654491554\\
30	6.70099908199053\\
31	6.70110335421526\\
32	6.70109454031283\\
33	6.70048011400453\\
34	6.70015802246409\\
35	6.70146803260837\\
36	6.70137519147395\\
37	6.70074615877402\\
38	6.70091163421494\\
39	6.7008946179437\\
40	6.70109518272528\\
41	6.70205145033657\\
42	6.70115273392061\\
43	6.70186718313822\\
44	6.70177380992648\\
45	6.7003804780666\\
46	6.70153952966726\\
47	6.70174797129619\\
48	6.70057765354134\\
49	6.70081922974977\\
50	6.70164333602487\\
51	6.70125515905557\\
52	6.70152742403787\\
53	6.70163646612874\\
54	6.70157301629276\\
55	6.701414269396\\
56	6.70151509569165\\
57	6.70113067220305\\
58	6.70067348960663\\
59	6.70094302840244\\
60	6.70066241369534\\
61	6.70171037678393\\
62	6.7017258902379\\
63	6.70100013577623\\
64	6.70084519318045\\
65	6.70123086292297\\
66	6.70103013736592\\
67	6.70117679885823\\
68	6.70136115786824\\
69	6.70167814502445\\
70	6.70122192741145\\
71	6.70183309433243\\
72	6.70100999249549\\
73	6.70135356970916\\
74	6.7002459876897\\
75	6.70146912361473\\
76	6.70130258398882\\
77	6.70075589014854\\
78	6.70080348452352\\
79	6.70112183391194\\
80	6.70120314765165\\
81	6.70119447031542\\
82	6.70144475842244\\
83	6.70137848453271\\
84	6.70168043308462\\
85	6.70077627802382\\
86	6.70071854612631\\
87	6.70101623736037\\
88	6.70104533420828\\
89	6.70135513010366\\
90	6.70056144078923\\
91	6.70102557689087\\
92	6.7006300362861\\
93	6.70107085890124\\
94	6.70153290457477\\
95	6.70133276766025\\
96	6.70119862303415\\
97	6.70089139040385\\
98	6.70152026042981\\
99	6.70007620915822\\
100	6.70036852816079\\
};
\addplot [color=red, dashdotted, line width=2.0pt, forget plot]
  table[row sep=crcr]{%
1	2.44661771734869\\
2	2.52576353700595\\
3	2.16127647791958\\
4	3.0310290948882\\
5	3.40887425643453\\
6	3.50341814626276\\
7	3.56683307706032\\
8	3.73648862770094\\
9	3.23565565554268\\
10	3.40502086781842\\
11	3.04321944910621\\
12	3.01889669235785\\
13	3.43068866230066\\
14	3.25900115565335\\
15	3.17273095025148\\
16	3.10459827121771\\
17	3.0526650791918\\
18	3.09139487891992\\
19	3.09579296675206\\
20	3.09234915988499\\
21	3.09851430049847\\
22	3.10488187422497\\
23	3.11031833793038\\
24	3.09025546200949\\
25	3.07799538254974\\
26	3.05651836407002\\
27	3.09294115326098\\
28	3.04643784169409\\
29	3.08269460708509\\
30	3.07199875153622\\
31	3.06343319656065\\
32	3.07375106751663\\
33	3.06639676553553\\
34	3.05995126777791\\
35	3.07517841982544\\
36	3.06389299412385\\
37	3.06951122517533\\
38	3.04740118393345\\
39	3.08209796772388\\
40	3.07641205413439\\
41	3.05399563766818\\
42	3.05893438157334\\
43	3.05541018738379\\
44	3.06393384924717\\
45	3.04896819364968\\
46	3.07750178838152\\
47	3.05587980147252\\
48	3.08182658514786\\
49	3.04540555034994\\
50	3.06253296378515\\
51	3.07015540037389\\
52	3.06967956686681\\
53	3.0665229029738\\
54	3.06112004082199\\
55	3.05154570012539\\
56	3.04896094300526\\
57	3.05485326870482\\
58	3.04000064341995\\
59	3.059682219609\\
60	3.06316955979892\\
61	3.06141679688057\\
62	3.04373363846218\\
63	3.06791502934542\\
64	3.0757031610305\\
65	3.07958268848526\\
66	3.05845004880157\\
67	3.06552120542576\\
68	3.0419291877304\\
69	3.06806252792471\\
70	3.06832192190414\\
71	3.05477491131693\\
72	3.06448088366623\\
73	3.04826170050289\\
74	3.06948504914674\\
75	3.06412523565948\\
76	3.07259231360712\\
77	3.06403226648865\\
78	3.06970412697863\\
79	3.06165521805732\\
80	3.0614821477698\\
81	3.052753008387\\
82	3.06648346891204\\
83	3.07741163270134\\
84	3.04639913183469\\
85	3.06273214783606\\
86	3.06596710461126\\
87	3.04983966925001\\
88	3.06539272911671\\
89	3.06856762532641\\
90	3.05256539118647\\
91	3.03999612848695\\
92	3.05343433068084\\
93	3.05168086156649\\
94	3.06665063702698\\
95	3.05274503115123\\
96	3.08200349747352\\
97	3.07318177136839\\
98	3.06615315275989\\
99	3.05534532007844\\
100	3.05738636915808\\
};
\addplot [color=mycolor1, dotted, line width=3.0pt, forget plot]
  table[row sep=crcr]{%
1	5.10841690753339\\
2	6.30366720328818\\
3	6.59816030254869\\
4	6.17790900319201\\
5	4.46475981640557\\
6	5.00765820609084\\
7	4.30951390227002\\
8	4.04161842546795\\
9	4.01485656248218\\
10	4.34312892872975\\
11	3.66541471620045\\
12	3.517951425459\\
13	3.61233274197403\\
14	3.74518751805426\\
15	4.16236954544025\\
16	3.64187637079015\\
17	3.4741715491685\\
18	3.29478144435632\\
19	3.18876107018515\\
20	3.20397854482619\\
21	3.19320600929792\\
22	3.27036325372438\\
23	3.20516039434347\\
24	3.05980848849459\\
25	3.05167931384347\\
26	3.05717743757322\\
27	3.0412954175818\\
28	3.06149776452893\\
29	3.04849873207019\\
30	3.04518142355283\\
31	3.05500058612968\\
32	3.05073964570803\\
33	3.05686069870436\\
34	3.04945434987303\\
35	3.07163211580473\\
36	3.07857793419349\\
37	3.07135457940943\\
38	3.05417963349527\\
39	3.06161284949563\\
40	3.07797428607014\\
41	3.08166389658056\\
42	3.06906824068655\\
43	3.07685730044911\\
44	3.06959642640651\\
45	3.04382768227879\\
46	3.06171377644948\\
47	3.07010403124248\\
48	3.05731202400714\\
49	3.07062597924447\\
50	3.06229126911754\\
51	3.05690352916751\\
52	3.06922983803813\\
53	3.05873407555027\\
54	3.0560740176733\\
55	3.0796699396199\\
56	3.05903847571533\\
57	3.05346016169011\\
58	3.05274905852315\\
59	3.06334635906267\\
60	3.05083683979572\\
61	3.06929869363696\\
62	3.07363736785062\\
63	3.06771725206619\\
64	3.0555037122552\\
65	3.06363688025101\\
66	3.05429921537096\\
67	3.05786906562215\\
68	3.06089932774921\\
69	3.08142837576992\\
70	3.07119110837081\\
71	3.07819963661889\\
72	3.07592939418843\\
73	3.06934163009663\\
74	3.04161092943315\\
75	3.06115524034447\\
76	3.05413353643378\\
77	3.04342239143652\\
78	3.06695751758088\\
79	3.06223131468353\\
80	3.06790529182809\\
81	3.07235109327085\\
82	3.06559871297931\\
83	3.06886571268794\\
84	3.07193206572329\\
85	3.06176523652693\\
86	3.05312708109413\\
87	3.05450888405249\\
88	3.0615663512455\\
89	3.05752676840432\\
90	3.05975652143678\\
91	3.07211379467446\\
92	3.0525903118369\\
93	3.04773333165863\\
94	3.06121170064906\\
95	3.06467535806715\\
96	3.07540315101183\\
97	3.06243086424018\\
98	3.06862373610331\\
99	3.04294148478522\\
100	3.05839878495847\\
};
\end{axis}
\end{tikzpicture}%
  }
  \\
  \subfloat[$\lambda=0.75$]{
    % This file was created by matlab2tikz.
%
%The latest updates can be retrieved from
%  http://www.mathworks.com/matlabcentral/fileexchange/22022-matlab2tikz-matlab2tikz
%where you can also make suggestions and rate matlab2tikz.
%
\definecolor{mycolor1}{rgb}{0.00000,0.75000,0.75000}%
%
\begin{tikzpicture}

\begin{axis}[%
width=0.951\fwidth,
height=0.75\fwidth,
at={(0\fwidth,0\fwidth)},
scale only axis,
xmode=log,
xmin=1,
xmax=100,
xminorticks=true,
xlabel={t},
ymin=0,
ymax=7,
ylabel={$U, F$},
axis background/.style={fill=white}
]
\addplot [color=blue, line width=2.0pt, forget plot]
  table[row sep=crcr]{%
1	5.47461127098968\\
2	5.60478890856896\\
3	5.77191033058464\\
4	5.90011496868576\\
5	6.14377790957163\\
6	6.31041948276384\\
7	6.43552914365433\\
8	6.5151531251348\\
9	6.54729734736349\\
10	6.57798495912498\\
11	6.57849571632306\\
12	6.58765985244295\\
13	6.58900819766999\\
14	6.58807470034135\\
15	6.58544669469427\\
16	6.58533577780727\\
17	6.58370067428244\\
18	6.58971365581165\\
19	6.5878980368345\\
20	6.59048092832187\\
21	6.59125044154283\\
22	6.5896710150567\\
23	6.58987286386257\\
24	6.58946117658065\\
25	6.58999333445314\\
26	6.59031718167873\\
27	6.59139413998349\\
28	6.59090165755066\\
29	6.59051023277067\\
30	6.59219941170473\\
31	6.59124669937017\\
32	6.58902226344559\\
33	6.58930522013265\\
34	6.59156801046039\\
35	6.59140118319131\\
36	6.59071800018758\\
37	6.5904962121109\\
38	6.58994289370681\\
39	6.59067144643273\\
40	6.5908473033638\\
41	6.58935869164842\\
42	6.59079752619829\\
43	6.59045257663437\\
44	6.59038156874226\\
45	6.58922214845872\\
46	6.5920430806679\\
47	6.59061552462121\\
48	6.5910853518908\\
49	6.5904707507656\\
50	6.59147532593014\\
51	6.59088905580135\\
52	6.59228441522875\\
53	6.59101992136476\\
54	6.59064064465195\\
55	6.59183783311411\\
56	6.59023817047242\\
57	6.59061017708232\\
58	6.5906193777462\\
59	6.59140530114315\\
60	6.5912579079603\\
61	6.59103059374945\\
62	6.59073429065668\\
63	6.5902464349662\\
64	6.59060596841733\\
65	6.59015307164583\\
66	6.59133709302133\\
67	6.59065526384808\\
68	6.58958255899406\\
69	6.59157679882054\\
70	6.59033990286414\\
71	6.59149880127007\\
72	6.59081749662089\\
73	6.58978538380844\\
74	6.59092637473552\\
75	6.58927529131999\\
76	6.58850343746042\\
77	6.59070961917545\\
78	6.59053467074177\\
79	6.58991848266028\\
80	6.59074090596838\\
81	6.59116797494771\\
82	6.59120811456219\\
83	6.58951958489378\\
84	6.59076403901614\\
85	6.58992010133594\\
86	6.59063253300245\\
87	6.59074681389064\\
88	6.59131520405029\\
89	6.58979941235916\\
90	6.58994888558284\\
91	6.59168176024839\\
92	6.59066396864525\\
93	6.59105539706456\\
94	6.59127950787074\\
95	6.59064023962094\\
96	6.59008138513551\\
97	6.59140183039189\\
98	6.58838036017117\\
99	6.5909343332388\\
100	6.59170125491083\\
};
\addplot [color=black!50!green, dashed, line width=3.0pt, forget plot]
  table[row sep=crcr]{%
1	5.62056595098105\\
2	5.9249699465451\\
3	6.01582827122022\\
4	5.95571514939206\\
5	5.92294247321893\\
6	5.81452251942304\\
7	6.13426812770014\\
8	6.04595676543718\\
9	6.29227294231064\\
10	6.27799612179181\\
11	6.32051997859251\\
12	6.36527532407389\\
13	6.32925991780952\\
14	6.45465237806581\\
15	6.58223240737318\\
16	6.57987442531153\\
17	6.55874072799058\\
18	6.5753041864462\\
19	6.58119554829056\\
20	6.58728407958621\\
21	6.57809056851983\\
22	6.56764059133101\\
23	6.41013588735602\\
24	6.38917378777253\\
25	6.58721628658142\\
26	6.58203616141645\\
27	6.4153585168044\\
28	6.56214565465418\\
29	6.42451672084935\\
30	6.57980421789207\\
31	6.59042602754726\\
32	6.59044529564649\\
33	6.59080569187187\\
34	6.59089611744954\\
35	6.58916534077888\\
36	6.59030181816158\\
37	6.59088695127249\\
38	6.58937832439878\\
39	6.59148671433531\\
40	6.59073721108426\\
41	6.59093404750414\\
42	6.59087742004054\\
43	6.58928541970452\\
44	6.59058564316579\\
45	6.59132828384597\\
46	6.59058851035584\\
47	6.59110263530547\\
48	6.59147553116448\\
49	6.58668393005854\\
50	6.59062363782734\\
51	6.59101515662169\\
52	6.59176251951947\\
53	6.58857868597947\\
54	6.59013423143927\\
55	6.59111909588808\\
56	6.5911457483472\\
57	6.59003852850639\\
58	6.59143982256265\\
59	6.59048536674524\\
60	6.59039176821099\\
61	6.59137154542976\\
62	6.5912535226034\\
63	6.58981582043099\\
64	6.5908918306807\\
65	6.59083300972301\\
66	6.59079121501016\\
67	6.59030883873119\\
68	6.59129857301594\\
69	6.59128229744416\\
70	6.59095332658505\\
71	6.59023081358638\\
72	6.59081865364851\\
73	6.59094026619255\\
74	6.59040142378658\\
75	6.59163684757572\\
76	6.59110676104285\\
77	6.59119701707623\\
78	6.58989897914213\\
79	6.5918191325617\\
80	6.59000392128006\\
81	6.59005624567473\\
82	6.59175293524183\\
83	6.5908999928299\\
84	6.59095496039431\\
85	6.58972726779356\\
86	6.59027090444657\\
87	6.59089103275845\\
88	6.59083663036135\\
89	6.59079717575128\\
90	6.59042755723988\\
91	6.59032670713473\\
92	6.59144685236426\\
93	6.5909022764905\\
94	6.59173282877921\\
95	6.58975668648628\\
96	6.58944811703204\\
97	6.5909704505714\\
98	6.58887279368135\\
99	6.59056883095371\\
100	6.59080516986559\\
};
\addplot [color=red, dashdotted, line width=2.0pt, forget plot]
  table[row sep=crcr]{%
1	0.962011577752915\\
2	1.39512833468571\\
3	1.90148489293405\\
4	1.51648055618043\\
5	2.15248666521931\\
6	1.55704328138457\\
7	1.66059810154288\\
8	1.52166767379182\\
9	1.60756285452165\\
10	1.57032375887657\\
11	1.57143632706561\\
12	1.54570012712595\\
13	1.53062124352496\\
14	1.54268502801898\\
15	1.53021067085092\\
16	1.55206916663035\\
17	1.54627922395726\\
18	1.51390174557473\\
19	1.50375418924415\\
20	1.51530747681262\\
21	1.52196737407071\\
22	1.50354898705708\\
23	1.51332011123504\\
24	1.52232901674692\\
25	1.52509841996306\\
26	1.52238627498701\\
27	1.5324730708447\\
28	1.51163420208493\\
29	1.51812829528253\\
30	1.52774156845851\\
31	1.52183786551656\\
32	1.5016521070917\\
33	1.49397777104937\\
34	1.51952082470017\\
35	1.52980925719026\\
36	1.52606705237248\\
37	1.52515602796127\\
38	1.50973845097226\\
39	1.52151282867688\\
40	1.53441438633646\\
41	1.52415222096287\\
42	1.51132177031528\\
43	1.5246991042189\\
44	1.50540365539768\\
45	1.50579300125202\\
46	1.5332974431763\\
47	1.52602816377732\\
48	1.52498943304033\\
49	1.52505167508116\\
50	1.52936828998447\\
51	1.52360673777914\\
52	1.53058250934036\\
53	1.52357239030125\\
54	1.52025395181835\\
55	1.53499050028086\\
56	1.51742772358661\\
57	1.51441659560209\\
58	1.51420820595266\\
59	1.5262997058346\\
60	1.53616242827644\\
61	1.52204501826377\\
62	1.51266612225863\\
63	1.5190132195052\\
64	1.52431344447975\\
65	1.51530419794324\\
66	1.52201531315481\\
67	1.51061899113799\\
68	1.51304669429054\\
69	1.52532175891914\\
70	1.53260170132874\\
71	1.52225905085154\\
72	1.50963037140345\\
73	1.52293726685294\\
74	1.52308518163793\\
75	1.4950927590338\\
76	1.50963368680608\\
77	1.51272511429317\\
78	1.51042633785373\\
79	1.50938242359903\\
80	1.5271682127886\\
81	1.5117185420591\\
82	1.52766007832391\\
83	1.51122470668482\\
84	1.51941129140273\\
85	1.51728704039085\\
86	1.52688068863037\\
87	1.50792216758675\\
88	1.51772700059157\\
89	1.51734457771922\\
90	1.51448813532051\\
91	1.53151859073494\\
92	1.52116535307287\\
93	1.51578575724323\\
94	1.52090562538418\\
95	1.51986831699292\\
96	1.50477239990879\\
97	1.52614502228845\\
98	1.4935503656753\\
99	1.51752656157145\\
100	1.53496968961844\\
};
\addplot [color=mycolor1, dotted, line width=3.0pt, forget plot]
  table[row sep=crcr]{%
1	4.6864935622248\\
2	4.8994383088347\\
3	4.56109313648738\\
4	4.90229035547094\\
5	2.9314300803944\\
6	3.2784332597462\\
7	2.46963199592331\\
8	2.30125152271061\\
9	2.01471564245944\\
10	2.17712372051914\\
11	1.78625426786003\\
12	1.85353123296186\\
13	1.99614342802232\\
14	1.65287461428729\\
15	1.49633284810289\\
16	1.48610098565703\\
17	1.45156396181048\\
18	1.47406316737917\\
19	1.46740662240884\\
20	1.50873895369633\\
21	1.45078965236607\\
22	1.45245977605543\\
23	1.71819478818221\\
24	2.08550862826438\\
25	1.5059355521904\\
26	1.47394259570511\\
27	1.75330416344819\\
28	1.38731873977995\\
29	1.70173662068909\\
30	1.46762136054097\\
31	1.50989567888606\\
32	1.52057424298062\\
33	1.52296770227207\\
34	1.51890013441686\\
35	1.51167634567613\\
36	1.52013496455013\\
37	1.53042250209518\\
38	1.51614460001237\\
39	1.51068803218364\\
40	1.52349879587756\\
41	1.5292094373918\\
42	1.52723516835788\\
43	1.50219367701738\\
44	1.50889764085272\\
45	1.53357949409282\\
46	1.5063426671682\\
47	1.52330362351915\\
48	1.52806722140564\\
49	1.48492641984674\\
50	1.50496302104093\\
51	1.51726658029489\\
52	1.51662375013351\\
53	1.48434709843271\\
54	1.51950876152417\\
55	1.51591219634932\\
56	1.52244113525343\\
57	1.51228811211186\\
58	1.51907869642937\\
59	1.52153657235561\\
60	1.52002671698928\\
61	1.52644490901635\\
62	1.51264572580944\\
63	1.5160114727441\\
64	1.52871982636947\\
65	1.52455740252168\\
66	1.52881396551966\\
67	1.52041366900985\\
68	1.51920423367879\\
69	1.52572400377125\\
70	1.52896228076713\\
71	1.50694486971597\\
72	1.5191129634431\\
73	1.53589170731109\\
74	1.51554168107362\\
75	1.52114588847516\\
76	1.5073119787335\\
77	1.53177729643885\\
78	1.49938469465641\\
79	1.52493589459921\\
80	1.51681359788873\\
81	1.50273815004913\\
82	1.53692226281642\\
83	1.51074892880607\\
84	1.53056388532285\\
85	1.52037492759945\\
86	1.51626804395327\\
87	1.5065737477751\\
88	1.52211768630356\\
89	1.52115051067854\\
90	1.521429717609\\
91	1.50829188644711\\
92	1.53568509407678\\
93	1.51780971278461\\
94	1.5267654175131\\
95	1.51765382436885\\
96	1.51725665471583\\
97	1.51173138823043\\
98	1.50648110697348\\
99	1.50703573590319\\
100	1.5100847459919\\
};
\end{axis}
\end{tikzpicture}%
  }
  \subfloat[$\lambda=1$]{
    % This file was created by matlab2tikz.
%
%The latest updates can be retrieved from
%  http://www.mathworks.com/matlabcentral/fileexchange/22022-matlab2tikz-matlab2tikz
%where you can also make suggestions and rate matlab2tikz.
%
\definecolor{mycolor1}{rgb}{0.00000,0.75000,0.75000}%
%
\begin{tikzpicture}

\begin{axis}[%
width=0.951\fwidth,
height=0.75\fwidth,
at={(0\fwidth,0\fwidth)},
scale only axis,
xmode=log,
xmin=1,
xmax=100,
xminorticks=true,
xlabel={t},
ymin=0,
ymax=6,
axis background/.style={fill=white}
]
\addplot [color=blue, line width=2.0pt, forget plot]
  table[row sep=crcr]{%
1	5.00968895853338\\
2	5.01867889787289\\
3	4.96734479517662\\
4	5.04284731151066\\
5	4.96410589090651\\
6	4.92855692547961\\
7	4.9849530218492\\
8	5.06366975530007\\
9	5.0646593325148\\
10	5.12795539320582\\
11	4.89537719631868\\
12	5.1613757241285\\
13	4.97670201742803\\
14	5.09142292366911\\
15	4.82148690792892\\
16	5.13409097540207\\
17	5.1341222403747\\
18	4.99850448319231\\
19	4.83165018984949\\
20	5.02222433844796\\
21	5.09962536428159\\
22	5.27934133662201\\
23	4.79337511133361\\
24	4.90405968430141\\
25	4.92075754330474\\
26	5.11674761478616\\
27	5.07383774437207\\
28	5.03240729832405\\
29	5.10667372816094\\
30	5.11074333496191\\
31	4.97878759507553\\
32	5.00046708761504\\
33	4.97488261321955\\
34	5.15357615714199\\
35	4.96285974167586\\
36	5.26459895740581\\
37	5.03792711601913\\
38	5.07764265394929\\
39	4.8895291974242\\
40	5.07768367522412\\
41	5.06360471512597\\
42	5.08353527287868\\
43	4.94099433242653\\
44	5.06468411730368\\
45	5.03270474398886\\
46	4.90237266369745\\
47	4.98043974931563\\
48	5.00730122676234\\
49	5.13167761766822\\
50	5.19185466491209\\
51	5.07687243384233\\
52	5.03928528212662\\
53	5.10640525406914\\
54	4.93859699587589\\
55	5.04541892638361\\
56	5.10314807571485\\
57	5.02543661853879\\
58	5.09710412795917\\
59	4.94878463621224\\
60	4.96981990076901\\
61	5.00113821629166\\
62	4.8471272091173\\
63	4.96653825479825\\
64	4.98745229273912\\
65	4.96029634125317\\
66	5.02507297666484\\
67	5.03191485345586\\
68	4.78673918044439\\
69	5.13506346096913\\
70	4.99502019340387\\
71	5.12813088525568\\
72	4.79845087337887\\
73	5.12996383403546\\
74	4.89162622588852\\
75	5.10264675623685\\
76	5.13272098199585\\
77	4.98435276390501\\
78	5.09237223293692\\
79	5.00149253981768\\
80	4.75194294090012\\
81	5.03149061817618\\
82	5.02286030628539\\
83	5.24753847765359\\
84	5.22136217976202\\
85	5.10313535258851\\
86	5.05630707400665\\
87	4.90911944262103\\
88	4.86514337148086\\
89	4.96542277592957\\
90	5.10553585133878\\
91	4.86192590684149\\
92	4.8877163095861\\
93	5.04272626528149\\
94	5.178037742655\\
95	4.95703731128462\\
96	4.96342988704945\\
97	4.94562711869157\\
98	5.12826309188108\\
99	5.00201749924757\\
100	5.15041565106772\\
};
\addplot [color=black!50!green, dashed, line width=3.0pt, forget plot]
  table[row sep=crcr]{%
1	5.00300452628849\\
2	5.02970653965817\\
3	5.20466060694751\\
4	5.14146874082709\\
5	4.9910411604218\\
6	4.89024994332463\\
7	4.89806177913917\\
8	4.67891065080925\\
9	4.96352233768098\\
10	5.11873559062519\\
11	5.00461106803201\\
12	5.0933328909675\\
13	4.94280869042649\\
14	4.9013247242917\\
15	5.03329638813133\\
16	5.09096432042337\\
17	5.03718251660817\\
18	5.04622163155994\\
19	4.91507309045656\\
20	5.00911975802797\\
21	4.8805581550414\\
22	5.09477415426356\\
23	5.11680762101062\\
24	4.93270418462427\\
25	5.04329221125669\\
26	5.04753720370155\\
27	4.95783686340263\\
28	4.96471807927325\\
29	4.95459445567533\\
30	5.05269435827337\\
31	4.94332736994511\\
32	4.9634056628628\\
33	4.99178549700645\\
34	5.01955597449025\\
35	5.06959287330068\\
36	5.10406288706704\\
37	5.05428931447181\\
38	5.05360258993434\\
39	5.19098602663521\\
40	5.13432669039147\\
41	5.04481791625102\\
42	4.93327080495292\\
43	4.87441982812724\\
44	4.90032655762289\\
45	4.97117949746028\\
46	5.08396671092384\\
47	5.0941678600243\\
48	4.98382899854513\\
49	4.98723509356919\\
50	4.83226642481922\\
51	4.95971660930353\\
52	4.94005325564499\\
53	5.08706338833015\\
54	4.94004925909277\\
55	5.1026973859899\\
56	4.99006306135613\\
57	4.97867318077998\\
58	4.92257162631893\\
59	5.00903496641468\\
60	4.99365976457554\\
61	5.0688704006483\\
62	5.17330343058461\\
63	4.86520813721441\\
64	5.00955599336862\\
65	4.90209515329868\\
66	5.02286894177531\\
67	4.88685405319227\\
68	4.98201378929519\\
69	4.92145050447653\\
70	4.85192474695884\\
71	4.90420661056582\\
72	4.94736936590097\\
73	5.08018005144609\\
74	5.11315225414218\\
75	5.0027118275121\\
76	5.21405914094631\\
77	5.16016496621367\\
78	4.96449008382216\\
79	4.97506141225551\\
80	5.10924283664766\\
81	5.08678777404397\\
82	5.11367151608572\\
83	4.86474225359695\\
84	5.00770928873192\\
85	4.90440005893373\\
86	4.98428828645257\\
87	5.1051380891621\\
88	4.81087465332218\\
89	5.07047597133386\\
90	4.92657747251374\\
91	5.13655011265076\\
92	4.93631681428851\\
93	4.99018860905356\\
94	5.02364009840864\\
95	4.92906948745726\\
96	4.69107607363032\\
97	4.99231164146143\\
98	5.09581049122184\\
99	5.05267820097062\\
100	5.07949959098964\\
};
\addplot [color=red, dashdotted, line width=2.0pt, forget plot]
  table[row sep=crcr]{%
1	0.0079538799589155\\
2	0.029543354739046\\
3	0.0433687510715851\\
4	0.0267997530098269\\
5	0.0552496970950875\\
6	0.0426903095165145\\
7	0.0679414709532705\\
8	0.0337160620542988\\
9	0.00774590526565739\\
10	0.0133546186947674\\
11	0.00840124016438263\\
12	0.00820499270165407\\
13	0.00590896597439416\\
14	0.0176499482814558\\
15	0.00252354607061481\\
16	0.00717860489756771\\
17	0.00319735702605778\\
18	0.00187242718476113\\
19	0.00288948427503447\\
20	0.00496589148972686\\
21	0.00224283116617609\\
22	0.00150386647593838\\
23	0.00159912641771346\\
24	0.00368014203475327\\
25	0.00109444086079066\\
26	0.000446191788070531\\
27	0.000734292624807064\\
28	0.000401051896476695\\
29	0.000932649177910425\\
30	0.00141656183508094\\
31	0.000515309722339206\\
32	0.00100075848721307\\
33	0.000844245985664303\\
34	0.000208029214434817\\
35	0.000817863343109707\\
36	6.87462086350267e-05\\
37	2.66168146450068e-05\\
38	0.000119567334337687\\
39	5.30513852849204e-05\\
40	0.000150595335306735\\
41	0.000156079689737337\\
42	0.000132583609651288\\
43	0.000105671878860136\\
44	0.000363674074689149\\
45	0.000332094737739339\\
46	0.000132280430068013\\
47	0.000360764969405453\\
48	0.000202804047634601\\
49	0.000914969854600546\\
50	0.00193609011034195\\
51	0.000485374853592484\\
52	0.000669058007541339\\
53	0.000100638550352485\\
54	0.000125608139962873\\
55	5.81618915309334e-05\\
56	1.65360876138627e-05\\
57	1.52893349942399e-05\\
58	4.50288483102586e-06\\
59	4.30220492720571e-06\\
60	1.47759406493938e-05\\
61	1.13574777807691e-06\\
62	3.05814881114346e-07\\
63	8.03477068955045e-07\\
64	6.11637159899579e-06\\
65	1.5135799037908e-06\\
66	3.34000395704954e-07\\
67	2.68482160484598e-07\\
68	4.02673438368847e-07\\
69	2.57633046635655e-06\\
70	1.50530556519574e-06\\
71	4.19887918924763e-06\\
72	1.10386452954666e-06\\
73	1.237121797162e-06\\
74	8.25241630388637e-08\\
75	9.6784364482321e-08\\
76	5.18656159601938e-08\\
77	2.0863147713059e-08\\
78	2.19152522685541e-07\\
79	4.46832491850821e-08\\
80	7.72699679516323e-10\\
81	8.9349853205317e-08\\
82	5.30542790635243e-10\\
83	1.98465955523162e-08\\
84	8.02736813289686e-09\\
85	3.91370113361051e-08\\
86	8.73236313782212e-08\\
87	7.71171273296649e-09\\
88	4.60709804182315e-11\\
89	5.66442994947387e-08\\
90	2.12750392924871e-08\\
91	4.04237672076645e-08\\
92	3.88491488358296e-08\\
93	2.65185909008292e-08\\
94	2.16508821825751e-08\\
95	4.37319159845521e-09\\
96	7.49389154669673e-08\\
97	2.08453943166065e-08\\
98	2.6708469428986e-08\\
99	5.90199758325732e-09\\
100	1.66200369130425e-08\\
};
\addplot [color=mycolor1, dotted, line width=3.0pt, forget plot]
  table[row sep=crcr]{%
1	2.722153411267\\
2	1.94516327026953\\
3	1.64508920163008\\
4	2.46256694199538\\
5	2.01900980362994\\
6	0.952183343545042\\
7	0.885305409684854\\
8	1.48994419555813\\
9	0.801874165878863\\
10	1.03111506169953\\
11	0.698290341643939\\
12	1.28703555289472\\
13	0.906153080690147\\
14	0.361312065333853\\
15	0.528382063924027\\
16	0.5004332650923\\
17	0.179922228143026\\
18	0.24485383970683\\
19	0.518467911031\\
20	0.396631360348243\\
21	0.286584107530347\\
22	0.0603461255916548\\
23	0.148105826934939\\
24	0.0208517310325713\\
25	0.185801109769805\\
26	0.0389451605100446\\
27	0.0852808488988264\\
28	0.04951256976962\\
29	0.00293319542166685\\
30	0.0200069023336759\\
31	0.0167216620657047\\
32	0.0344243494755248\\
33	0.040236107222278\\
34	0.0201039315812888\\
35	0.016858757911495\\
36	0.0241656670145098\\
37	0.0198255436378841\\
38	0.0318284149608048\\
39	0.0162088393933059\\
40	0.039594321688047\\
41	0.00450617792555742\\
42	0.0109730978164902\\
43	0.0129224627755731\\
44	0.0864968986718645\\
45	0.0481343182828859\\
46	0.0556736894707966\\
47	0.0191104076358362\\
48	0.0412646735520683\\
49	0.0176149892716187\\
50	0.00441802660286454\\
51	0.000432317205929555\\
52	0.000240846243408926\\
53	0.000188863174901928\\
54	5.70154076065477e-05\\
55	0.000208560630301867\\
56	7.85284996924208e-05\\
57	2.21993073117616e-05\\
58	6.54599360020493e-06\\
59	4.69267031139676e-05\\
60	0.000122224255520404\\
61	1.29640550327843e-05\\
62	5.90687176957188e-06\\
63	4.43604109352983e-06\\
64	5.58923974059919e-07\\
65	9.0581274547844e-07\\
66	8.18019431653096e-07\\
67	4.83986305006053e-07\\
68	1.03839991204521e-06\\
69	6.2129588397099e-07\\
70	4.64933447872208e-07\\
71	1.69060089887443e-07\\
72	3.05849363198025e-07\\
73	1.20927203685865e-07\\
74	9.55692055288801e-08\\
75	7.60571618399762e-08\\
76	6.24518519058812e-08\\
77	6.93872577478441e-08\\
78	4.02800714128452e-08\\
79	1.93759546630455e-08\\
80	3.9343338870102e-08\\
81	1.50059774810078e-08\\
82	1.59401431176721e-08\\
83	7.27357793840842e-08\\
84	8.26336498923122e-09\\
85	3.80585755198375e-08\\
86	4.86056369782277e-08\\
87	3.717722801735e-08\\
88	1.62525653097387e-09\\
89	1.59693921129335e-08\\
90	6.77232274862789e-08\\
91	4.8663724644361e-08\\
92	1.14002947485849e-08\\
93	4.70024474232401e-08\\
94	1.65402175987577e-08\\
95	3.54811160226814e-08\\
96	1.2803379939206e-08\\
97	6.09303501530596e-08\\
98	1.45453130886335e-08\\
99	5.65938338638241e-08\\
100	3.36173018218328e-08\\
};
\end{axis}
\end{tikzpicture}%
  }
  \subfloat[legend]{
    \raisebox{4em}{\definecolor{mycolor1}{rgb}{0.00000,0.75000,0.75000}%

\begin{tikzpicture}

  \begin{axis}[%
    hide axis,
    xmin=10,
    xmax=50,
    ymin=0,
    ymax=0.4,
    legend style={draw=white!15!black,legend cell align=left}
    ]
    \addlegendimage{color=blue, line width=2.0pt}
    \addlegendentry{Bayes $U$};
    \addlegendimage{color=black!50!green, dashed, line width=3.0pt};
    \addlegendentry{Marginal $U$};
    \addlegendimage{color=red, line width=2.0pt, dashdotted}
    \addlegendentry{Bayes $F$};
    \addlegendimage{color=mycolor1, dotted, line width=3.0pt};
    \addlegendentry{Marginal $F$};
    
  \end{axis}
\end{tikzpicture}
}
  }
  \caption{\textbf{Synthetic data, utility-fairness trade-off.} This plot is generated from the same data as Figure~\ref{fig_exp_1}. However, now we are plotting the utility and fairness of each individual policy separately. In all cases, it can be seen that the Bayesian policy achieves the same utility as the non-Bayesian policy, while achieving a lower fairness violation.}
  \label{fig_exp_1:tradeoff_extend}
\end{figure*}

\begin{figure*}
\centering
  \subfloat[$\lambda=0$]{
    \input{figures/sequential/dirichlet-lambda-0.0.tex}
  }
  \subfloat[$\lambda=0.25$]{
    \input{figures/sequential/dirichlet-lambda-0.25.tex}
  }
  \subfloat[$\lambda=0.5$]{
    \input{figures/sequential/dirichlet-lambda-0.5.tex}
  }
  \\
  \subfloat[$\lambda=0.75$]{
    \input{figures/sequential/dirichlet-lambda-0.75.tex}
  }
  \subfloat[$\lambda=1$]{
    \input{figures/sequential/dirichlet-lambda-1.0.tex}
  }
  \subfloat[legend]{
    \raisebox{4em}{\input{figures/sequential/finite-models-legend.tex}}
  }

  \caption{\textbf{COMPAS dataset.} Demonstration of balance on the COMPAS dataset. The plots show the value measured on the holdout set for the \textbf{Bayes} and \textbf{Marginal} balance.
  Figures (a-e) show the utility achieved under different choices of $\lambda$ as we we observe each of the  6,000 training data points. Utility and fairness are measured on the empirical distribution of the remaining data and it can be seen that the Bayesian approach dominates as soon as fairness becomes important, i.e. $\lambda > 0$.  }
  \label{fig:compas-dbn_extend}
\end{figure*}


\begin{figure*}
\centering
  \subfloat[$\lambda=0$]{
    \input{figures/sequential/dirichlet-lambda-seq1-0.0.tex}
  }
  \subfloat[$\lambda=0.25$]{
    \input{figures/sequential/dirichlet-lambda-seq1-0.25.tex}
  }
  \subfloat[$\lambda=0.5$]{
    \input{figures/sequential/dirichlet-lambda-seq1-0.5.tex}
  }
  \\
  \subfloat[$\lambda=0.75$]{
    \input{figures/sequential/dirichlet-lambda-seq1-0.75.tex}
  }
  \subfloat[$\lambda=1$]{
    \input{figures/sequential/dirichlet-lambda-seq1-1.0.tex}
  }
  \subfloat[legend]{
    \raisebox{4em}{\input{figures/sequential/finite-models-legend.tex}}
  }
\caption{\textbf{Sequential allocation.} Performance measured with respect to the empirical model of the holdout COMPAS data, when the DM's actions affect which data will be seen. This means that wif a prisoner was not released, then the dependent variable $y$ will remain unseen. For that reason, the performance of the Bayesian approach dominates the classical approach even when fairness is not an issue, i.e. $\lambda = 0$.}
\label{fig:sequential-allocation_extend}
\end{figure*}





%\section{Proof of Lemma \ref{lemma:ult}}
%\begin{proof}
%  Let $a^*(x) \defn \max_a \E(\util \mid a, x)$ for some arbitrary
%  distribution. Then for any randomized policy $\pol$:
%  \begin{align*}
%  &\E^\pol(\util \mid x) =
%  \int_{\CA} \E(\util \mid a, x) \dd \pol(a \mid x)\\
%  &\leq \int_{\CA} \E(\util \mid a^*, x) \dd \pol(a \mid x)\\
%  &=\E(\util \mid a^*, x) \dd \pol(a \mid x).
%  \end{align*} 
%  We first apply this to the distribution $\bel(\param)$ to obtain the
%  result for Thompson sampling. For stochastic dominance, note that
%  $$\Pr_\bel(X > Y) = \int_\Param \Pr_\param(X > Y) \dd \bel(\param).$$
%  As stochastic dominance can be implemented by first sampling a
%  parameter and then sampling a dominant variable under this
%  parameter, we can reapply this fact and obtain the final result.
%\end{proof}
%%% Local Variables:
%%% mode: latex
%%% TeX-master: "subjective-fairness"
%%% End:
