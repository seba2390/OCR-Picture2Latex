\section{Open areas of research}
\label{sec:openResearch}

The surveyed papers  show that, despite exciting results obtained, fully understanding how to develop effective VPs for patient-doctor communication training requires further work. Reasons are related to the fact that either the technological components have not been fully explored yet or results are still inadequate to fully assess the effectiveness of different design approaches. Thus, in this section, we briefly discuss some open problems and present areas requiring further research.

\textbf{Assessment of design elements.}
%\label{sec:effectiveness}
% The general effectiveness of VPs on developing communication skills has been discussed in literature by several authors \cite{peddle2016virtual,lee2020effective,richardson2019virtualreview}. A common complaint in VP-related literature is the lack of standardized terminology and considerable heterogeneity in study design that makes retrieval and evaluation of literature a troublesome task. Despite that, both \cite{peddle2016virtual} and \cite{lee2020effective} conclude that, when appropriately contextualized in a well thought out educational context, VPs are indeed useful in developing, practising and building confidence about communication and other skills like decision making and teamwork.  
% Based on these findings, one possible question arising from our review is if the surveyed articles provide pieces of evidence about the effects on learning outcomes and efficacy of the simulation of the different instructional design elements and the technical features available. 
% Unfortunately, the answer is negative. In most of the works analyzed, authors report users' feedback or comments about a particular element/feature only and a direct comparison between different design choices is missing. The only notable exceptions are three. First, \cite{ochs2019training}, which assessed different presentation formats showing that immersive VR technologies yield superior results when compared with non-immersive ones. Second, concerning the distribution method, \cite{maicher2017developing} reports that a standalone application can provide a considerably higher level of engagement than its web-based version, thanks to the possibility of leveraging advanced technical features (voice-controlled input and large volume displays) to increase immersion and focus on the task at hand. Finally, \cite{hirumi2016advancing}, compared closed and open-option unfolding designs, reporting learners' preferences for the latter.  
As discussed in Section \ref{sec:effectiveness}, the current literature lacks a thorough evaluation of the effectiveness of alternative designs. This observation highlights the fact that further work has to be done to develop a better understanding of instructional elements and technical features that VP simulations can offer in order to achieve the desired learning outcomes. 


\textbf{Scope.} Another comment can be made on the specific communication learning context. While several core skill domains jointly contribute to a patient's health and satisfaction (like relationship building, information gathering, patient education, shared decision making and breaking bad news \cite{riedl2017influence}), most of the surveyed VP simulations  focus only on one specific domain. This observation highlights the need to develop novel approaches capable of addressing simultaneously the multiple communication challenges one has to face when interacting with a real patient, thus helping to improve the overall learner's communication skills.


%\andrea{Authoring tools for VPs.}
\textbf{Authoring tools.}
Implementing VPs is a cumbersome and complicated process, which requires taking into account several different elements (NLP, emotion modelling, affective computing, 3D animations, etc.), which, in turn, involve specific technological and technical skills. Usually, the development of a VP is a cyclical process of research, refinement and validation with experts that can take a considerable amount of time \cite{rossen2009human}. Thus, there is the need to develop simple (and effective) authoring tools that can allow developers to support clinical educators in the rapid design, prototyping and deploying of VPs in a variety of use cases. 
Examples of authoring tools for narrative-style VPs with 3D graphics are very scarce in the literature. The work presented in \cite{jeuring2015communicate} integrates a scenario builder that allows clinical educators to design the unfolding of their cases. This authoring tool exploits a domain reasoner where the response of the virtual agent is determined not only by the previous dialogue that the user chose, but also by other parameters like the agent's current emotional state. However, this tool lacks the possibility to customize the virtual environment or the VP's aesthetics.
The NERVE VP \cite{hirumi2016advancingPart2,hirumi2016advancing,kleinsmith2015understanding} is built upon the Virtual People Factory \cite{rossen2009human}, a web application that enables the users to build conversational models using an un-annotated corpus retrieval approach based on keyword matching. 
Another interesting example is SIDNIE (Scaffolded Interviews Developed by Nurses in Education \cite{dukes2016participatory}). This tool allows clinical educators to edit the patient's medical status, dialogue options and physical appearance. However, to our knowledge, SIDNIE has not been deployed in any publicly available form, and  appears to be aimed exclusively at nurse training scenarios.

In other application areas (such as building clinical skills and problem-solving abilities), the extensive use of tools such as DecisionSim, OpenLabyrinth and Web-SP %(a comparative study detailing the usage and characteristics of these tools can be found in 
\cite{doloca2015comparative} is a clear demonstration of the fact that an easy-to-use authoring tool is a determinant factor for the success of a VP application. However, compared to these areas, the specific context of patient-doctor communication training involves more complex systems, with 3D visuals and branched narratives offering a more realistic interaction, which makes the development of authoring tools in this area much more challenging \cite{talbot2012sorting}.\par
%End of Edo's Addon

\textbf{Emerging web technologies.}
%WebXR and Streaming Services}
In the previous sections, we highlighted that personal devices are coming with better and better hardware and computational power, thus helping to narrow the gap between standalone and web-based applications. Another contribution will inevitably come from recent advances in web-based technology, like, e.g., WebXR\footnote{\url{https://www.w3.org/TR/webxr/}}. WebXR is a device-independent framework that allows users to develop and share VR and AR applications over the Internet, with considerable support for different hardware and web browsers. In addition, game-streaming platforms such as Google Stadia\footnote{\url{https://stadia.google.com/}} are a very promising workaround for the limited computational capabilities of personal devices. With these platforms, the bulk of the computation is processed on the server side, then the pre-rendered output is streamed to the final user's device. The implementation of such technological solutions in the immediate future will enable the applications to combine the accessibility of current web-based software with the computational complexity of standalone applications run on a dedicated machine. \par

%edo{mi rendo conto che Virtual Humans è un termine veramente vago, perché può intendere sia NPC che Player Avatar... Non volevo usare NPC perché fa troppo gamer, però qualche articolo scientifico che usa il termine NPC l'ho trovato, quindi userei quello per evitare qualsiasi ambiguità. Riscritta tutta questa sezione}

\textbf{Multiple virtual humans.}
Interacting with a relative or another health care provider are considered crucial aspects of a clinician's communication skills \cite{hallin2011effects,kee2018communication}. However, VP simulations usually include only two actors: the learner (possibly represented by an avatar) and a unique Non-Playable Character (NPC), i.e, a virtual human not controlled by the trainee that represents the patient. The only two examples that include more than one NPC besides the patient are the Medical Interview Episode of the UTTimePortal \cite{zielke2016beyond,zielke2016using} (which incorporates a patient and a caregiver), and MPathic-VR \cite{guetterman2019medical,kron2017using} (which includes a patient's relative and a nurse). 
Beyond this observation, we should also note that another interesting future development (still untouched in the field of VPs for patient-doctor communication skills, to the best of our knowledge) could be to provide the possibility of interacting (within the simulation) with other human-controlled avatars, in a way similar to that proposed by approaches focused on inter-professional communication in emergency medical situations \cite{anbro2020using}. 


\textbf{Immersive VR and AR.}
% \andrea{Stress the relavance of IVR in particular. Immersion and presence contribute to empathic bond with the VP, which in turns have beneficial effects on the learning outcomes. This is why we stress so much these elements}
There is a general understanding among researchers that increasing the level of immersion and realism of the simulations (e.g., using large volume displays, HMDs, spatialized 3D audio, higher fidelity graphics and animations) leads to more believable human-computer interactions \cite{chuah2013exploring,johnsen2008evaluation}, which in turns help improve the users' communication and empathic skills  \cite{ochs2019training,zielke2017developing} and, ultimatley, the learning outcomes in general \cite{limniou2008full}. 
%The several potential advantages of using these technologies, with particular attention on provider-patient communication, have been extensively discussed in \cite{zielke2017developing}. 
However, surprisingly, the use of IVR technologies in this specific context appears to be quite limited. Only two VPs out of \totalVPs, i.e., Ochs \cite{ochs2019training} and CESTOL VR Clinic \cite{sapkaroski2018implementation}, mention the use of IVR, and AR appears to be completely unexplored.  The primary obstacles to the adoption of IVR or AR in VP simulations seem to be the complexity, challenges and costs of development steps \cite{zielke2017developing}.  

Fortunately, things are going to change rapidly. In recent years, the availability and quality of VR devices have increased considerably, and their cost has decreased dramatically.  These factors contribute (together with the availability of high-end development platforms such as Unity or Unreal engine) to reducing overall costs and efforts for developing IVR and AR applications. Furthermore, IVR offers currently a truly immersive, unbroken environment that can shift the cognitive load directed on imagining oneself \quotes{being there} in VR towards solving the task at hand. In turn, higher immersion and visual fidelity can have positive effects on learning \cite{coulter2007effect}, \cite{huerta2012measuring}. Thus, we expect that, soon, VR and AR will contribute to improving the state of the art in this research field.




\textbf{Fully-fledged non-verbal input.}
%Non-Verbal Input should be determinant, not an accessory.}
In our opinion, this is a major lack in current designs. The unfolding of the simulation's narrative should be dictated (in tandem) by both user's verbal and non-verbal behaviours. To this end, developers of future VPs should attempt to fully leverage non-verbal cues as a factor that actively influences the state of the agent. For instance, the same utterance should have a different outcome if the user maintains eye contact with the patient, looks in another direction, and is fidgeting or exhibiting an incoherent facial expression.
The extraction of para-linguistic factors such as tone of voice, loudness, inflection, rhythm, and pitch can provide information about the actual emotional states of the other peer in the communication. Prosody must be addressed with great attention since it is one of the main ways to express empathy and can have a considerable impact in increasing patient satisfaction \cite{kee2018communication}. Thus, computational mechanisms capable of extracting these variables from the analysis of the user's voice are sorely needed. 
The same para-linguistic factors should be also available to modulate the VP response according to its emotional states. In fact, one of the problems with present text-to-speech libraries is that they pronounce everything with the same tone, which makes it impossible to communicate feelings through voice. 
