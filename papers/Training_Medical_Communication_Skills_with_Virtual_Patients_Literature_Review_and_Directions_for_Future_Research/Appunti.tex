%%%%%%%%%%%%%%%%%%%%%%%%%% EDO's text

\andrea{Doctor - Patient Communication, applications and design principles for Virtual Patients}

\begin{itemize}
\item \emph{What are the elements of doctor-patient communication?}
\end{itemize}{}
Doctor-Patient communication can be broken down in 3 main areas (\cite{riedl2017influence}):
\begin{itemize}
    \item Relationship Building
    \item Information Gathering
    \item Patient Education
\end{itemize}{}

All three of these can be applied both to patients and their relatives. Empathy plays a fundamental role both in Relationship Building and Patient Education, for example when there is the need to break bad news.
In literature several communication protocols can be found, for example SPIKES (\cite{baile2000spikes}, \emph{find other citations}), which outlines the proper procedure for breaking bad news, or tackle end-of-life conversations.

\begin{itemize}
\item \emph{Why is Doctor-Patient communication an important skill to train?}
\end{itemize}{}

Patients rate open communication as one of the most important aspects of their relationship with their physicians (\cite{dibbelt2010patient}).
Doctor-Patient communication is important to avoid malpractice accusations and to increase patient satisfaction (\cite{kee2018communication}, \cite{fiscella2004patient}), \cite{papadakis2005disciplinary}, \cite{stelfox2005relation}, \cite{franks2005patients}, \cite{hickson2002patient}, \cite{kohatsu2004characteristics}), can have beneficial effects on patients, creating in some cases a placebo effect (\cite{kelley2009patient}), while conversely a poor grasp on communication skills can be detrimental both on the patient's and their relatives' health (\cite{kee2018communication}, \cite{judge2004affect}).
Communication is a complex phenomenon that's not restricted to the verbal domain only, as outlined by (\cite{kee2018communication}), these are all critical aspects that have been subject to patients' complaints:
\begin{itemize}
    \item Non-verbal communication (eye-contact, facial expressions, prosodic features)
    \item Choice of words, lack of pauses to let patients ask questions, lack of listening
    \item Information given, meaning of words
    \item Empathy (disrespect, poor attitudes)
\end{itemize}{}

\begin{itemize}
\item \emph{What are the methods of training communication skills for medical personnel?}
\end{itemize}{}

The main methods of training involve Standardized Patients and Virtual Patients. Standardized Patients are actors taking the role of a patient that simulate specific behaviors and/or symptoms.  

\begin{itemize}
\item \emph{What are Virtual Patients (VPs)?}
\end{itemize}{}

VP, while being more widespread recently, have been around for many years, with the first account going as far back as 1966(\cite{bitzer1966clinical}). Virtual Patient is a very broad term, encompassing learning tools from very rudimentary simple linear text-based simulations (such as those discussed in these articles \cite{wilkening2017evaluation}, \cite{georg2018virtual}, \cite{kononowicz2019virtual}) to advanced multi-sensory immersive computer applications leveraging VR headsets (\cite{jung2012evaluation}). This is a problem, as highlighted by (\cite{foronda2020virtual}), especially in medical literature as it can be misleading and lead to misunderstand exactly "what" a VP is when mentioned in an article.
VPs are used in almost every branch of medical learning (doctors, nurses, dentist, psychologists etc.). There is extensive literature on the topic, and several systematic reviews (\cite{cook2009virtual}, \cite{cook2010computerized}), some of which are very recent (\cite{kononowicz2019virtual}, \cite{foronda2020virtual}, \cite{rourke2019does}, \cite{lee2020effective}, \cite{georg2018virtual}).
In the aforementioned reviews, VPs have been compared to traditional (or, more generally, non VP) methods of teaching in terms of
\begin{itemize}
    \item Knowledge
    \item Skill
    \item User Satisfaction
    \item Critical Reasoning
\end{itemize}{}
As emerges from these systematic reviews, VPs generally yield positive results in knowledge, user satisfaction and critical reasoning, while development of skill is debatable. Compared to non-VP forms of teaching, VP applications yield comparable or slightly better results in most cases. Also these articles indicate that the employment of VPs must be carefully considered as to where and when it is implemented in medical training, and must be envisioned more as an integrative form of learning than a substitute to other more traditional devices. 
VPs may or may not involve communication elements, since a good portion of them is focused on the development of a skill, such as clinical critical reasoning (analyzing symptoms and developing a diagnosys) (\emph{citations needed}) or training in a particular procedure (injections, catheterisation \cite{ismailoglu2018comparison}, \cite{engum2003intravenous}, \cite{jung2012evaluation}) for example in a simulation aimed at nurses and dentists.
In relation to communication skills, some articles (\cite{morency2015simsensei}, \cite{stevens2006use}) argue that VPs may offer a more stress-free an repeatable environment to practice communication skills for medical personnel, as virtual humans may increase the willingness to disclose (\cite{lucas2014s}).
An advantage of VPs is obviously their repeatability, student independence (\cite{cook2010computerized}) and immediate feedback to assess efficiency of examination compared to an expert's and/or in terms of cost-effectiveness of exams and laboratory tests (\cite{cook2009virtual}). 
\cite{stevens2006use} points out several other potential advantages over Standardized Patients, namely:
\begin{itemize}
    \item limiting variability and expense associated with SP training
    \item creating an almost limitless repository of diverse and challenging virtual clinical scenarios (ie, the aggressive patient or poor historian) that are difficult to duplicate with authentic SPs (ie, infants, children, gender, ethnicity, and cultural characteristics)
    \item maintaining a computerized log of student progress with objective performance data
    \item tailoring educational methods to fit individual student learning styles and rates of progress 
    %OCCHIO QUESTO ELENCO SONO CITAZIONI LETTERALI, SE LE USI IN UN ARTICOLO VERO RISCRIVILE CON PAROLE TUE SE NO IL FILTRO ANTIPLAGIO CI frega
\end{itemize}{}

\edo{questo è un punto saliente del discorso che dobbiamo fare. Lo metto qua per reference, è letterale, da: :} \cite{hirumi2016advancing} \par
Bearman et al. 2001 \edo{che abbiamo anche noi in bibliografia: } \cite{bearman2001random} distinguished two basic approaches to virtual patient design: (a) narrative—that presents students with a coherent storyline and decisions that have direct
consequences on the patient’s outcome \edo{che fa parte delle cosiddette Non-Technical Skills, o NTS (che comprendono anche altre cose, vedi sotto)}, and (b) problem-solving—that requires students to collect information from patient histories, lab tests, and physical examinations to make diagnostic and management decisions based on their findings.
Most VPs are based on the problem-solving (PS) approach, presenting learners with multiple-choice actions to advance through different scenes that depict various provider-patient interactions (Bearman 2003, Huang et al. 2007). Problem-solving approaches are easier to develop and deliver than narrative applications (Bearman 2003), and existing solutions such as VPSim (Benedict 2010) \edo{a cui io aggiungerei anche DecisionSim, Web-SP e OpenLabyrinth} and the standard-defining group MedBiquitious (www.medbiq.org) provide powerful tools for end-users to create linear and branched experiences \edo{sottolineando l'importanza degli authroing tools e del fattore web-based}, and large case libraries that promote decision-making (Bearman 2003).
However, the PS approach feels scripted and students have little control over what to do next (Bearman 2003). By limiting responses to a defined set of choices, PS designs neither simulate authentic communications with patients, nor foster students’ ability to formulate questions.
\edo{fine citazione letterale}\par
\edo{definizione di problem solving strucutre e narrative structure dall'articolo originle della bearman:}
Problem solving: "The problem-solving approach is found in virtual patient designs concerned with teaching clinical reasoning and diagnosis. Generally the student has to collect a range of information, usually from menus of possible history questions, lab tests, and physical examinations, and make diagnostic and management decisions based on their findings"
Narrative: "The narrative approach is often found in virtual patient encounters which are concerned with cause and effect. This includes programmes that have an emphasis on decision making which results in various outcomes over time."
E, IMPORTANTISSIMO: "Narrative and problem-solving approaches can be used in tandem"!!!!
%\edo{definizione letterale di nontechnical skills (NTS), da } \cite{flin2008safety}: "the cognitive, %social and personal resource skills that complement technical skills and contribute to safe and efficient %task performance". Tenendo ben presente il fatto che non si limitano alla comunicazione, ma comprendono: %"situational awareness, decision-making, communication, teamwork, leadership, managing stress and coping %with fatigue".
%\edo{fine citazione letterale}\par
\edo{per quanto riguarda il training delle Non Technical Skills, letteralmente da }\cite{peddle2016virtual}: Historically, nontechnical skills are poorly articulated and addressed in undergraduate health professional education programs (Pearson
& McLafferty, 2011). Skills required for competent practice in the clinical setting, such as teamwork, are traditionally left to be attained through ‘‘on-thejob’’ learning experiences (Brennan et al., 2010). Early professional education in nontechnical skills may afford learners a basic understanding of the factors influencing human performance, improved team communication, and support development of solutions for minimizing risks to patients (Flin & Patey, 2009). The development of the Patient Safety Curriculum Guide by the World Health Organization, which aims to build the knowledge and skills of health professionals to better prepare them for safer practice, supports the importance of addressing these concepts in undergraduate health professional programs (Walton, 2011). Learning through simulation is suggested to provide a robust and effective strategy to successfully develop nontechnical skills (Unsworth, Melling, Allan, Tucker, & Kelleher, 2014) with examples across the learning continuum. Simulation is positively associated with improved communication in handover, the development of leadership skills (Lewis, Strachan, & Smith. 2012), improvements in team behaviors and team performance in crisis situations,
and developing situational awerness skills, \edo{fine citazione letterale}.\par
\edo{sempre a riguardo delle NTS:} Da \cite{peddle2019development}: "Traditionally, the evolution of these skills (NTS) happened over time in clinical practice as the practitioner developed from novice to expert(\cite{josephsen2014virtual}).However, current clinical practice environments necessitate health professionals learning these foundational abilities prior to graduation (\cite{josephsen2014virtual})."

\edo{spunti per quanto riguarda autenticità delle simulazioni, da:} \cite{peddle2016virtual}: "Designing simulations which provide learners with a sense of the unpredictability of the clinical environment is an area for future research."\par
da \cite{volante2016effects}: "It has been shown that human-agent interactions can evoke similar social responses to that of human-human interactions \cite{pan2011computer}, \cite{robb2013leveraging}. Generating anemotional response from participants is important when working with learning systems, as it has a strong influence on memory retention \cite{dunsworth2007fostering}. ". Sempre da questo articolo \cite{volante2016effects} possiamo tirare fuori considerazioni interessanti: il realismo è importante per l'immersione, ma in questo periodo c'è il pericolo di cadere nella Uncanny Valley, infatti qua si dimostra che gli utenti dimostrano maggior empatia verso il paziente col toon shader piuttosto che verso quello con lo shading realistico. 

\edo{appunti riguardo a metodi di training standard rispetto ai VP, con citazioni:} Da: \cite{banszki2018clinical}: "Educators have previously used simulated learning environments (SLEs) to replicate realistic scenarios for the purpose of training clinical skills within a safe environment (Theodoros et al., 2010). SLEs include the use of standardised patients (i.e., trained actors), mannequins, and virtual patients (Theodoros et al., 2010). Compared to traditional clinical placements, SLEs have the benefit of being standardised, permitting educators to deliver controlled, repeatable practice of clinical scenarios that may otherwise be considered uncommon or high-risk. " \edo{questo sembra fare riferimento esclusivamente agli speech pathologists e all'Australia}. Inoltre, da \cite{kleinsmith2015understanding}: "While VPs are constructed with the same symptoms and responses with which an SP is trained, VPs can be created to exhibit a wide range of clinical issues that are not possible for SPs, e.g., facial paralysis, ptosis (i.e., drooping of the upper eyelid), etc. "\par

dA \cite{maicher2017developing}: "Although SP interviews can be standardized, they require significant faculty effort and ongoing support In addition, SP interactions can be internally inconsistent, with quality depending on the experience and training of the actor portraying the patient." \edo{le citazioni che ci sono qua sono irreperibili, quindi questa rimane come considerazione generale}.\par

dalle conclusioni di \cite{maicher2017developing}: "These VSP encounters are not designed to replace existing training with SPs or real patients. Rather, they present an opportunity for students to gain early practice on their historytaking skills in safe, nonthreatening environments before real-life or simulated SP encounters".\par

\edo{argomentazioni e reference sull'alta fedeltà grafica}: da \cite{coulter2007effect} e \cite{huerta2012measuring}: Maggiore fedeltà della rappresentazione visuale causa maggiore immersione e ha effetti positivi sull'apprendimento.

\edo{citazione sull'uso e la validazione di applicazioni VR in ambito medico}, da \cite{zielke2015serious}: "Most validation studies of virtual simulation and training-type games focus on a narrow set of surgical skills such as laparoscopic and endoscopic training \cite{graafland2012systematic}. Very little has been published on how to implement fuller, organization-wide curriculum characteristics such as interprofessional team-based communication, perspective sharing, patient-centered “just culture” which balances safety and accountability"

\edo{altri appunti sull'immersione} da \cite{}: The level of immersion is influenced by the fidelity of the environment as a whole, in terms of graphics, audio effects and interaction with objects (Bauman 2012)


\edo{riguardo al fatto che il branched-option è più restrittivo del free-text per utenti esperti}: da \cite{jacklin2019virtual}: " The data suggest that this type of intervention  could be useful at many different stages of a professional’s career although the multiple-choice conversation style may be too restrictive for more experienced consulters" e, da \cite{carnell2015adapting}: "This study explored the creation of a selection-based VP for novice users. Our method used novice transcripts and an existing chat-based VP to produce a new selection-based VP. After creating the VP, we had actual healthcare students interview the patient, as well as two chat-based VPs. Our initial findings indicate that the selection generator can be used to design novice-oriented VPs." da \cite{jacklin2019virtual}: "Fifteen participants responded to the accompanying free-text question with most suggesting that the multiple-choice system was too restrictive". da \cite{hirumi2016advancing}: "The closed menu design has proven very useful for a number of stakeholders. First, the students see value in having the menu when they don’t know what to say. In this way, the menu is a training tool. In many ways, it is a first-step that helps students form an interview schema prior to using the open chat interface". sempre da \cite{hirumi2016advancing}: "With the open chat design, NERVE provides a sense of presence with
the patient, that the patient is awaiting what you are saying and responding to your actions.
There is a sense of pressure to perform, and the constant threat that you won’t know what
to say. The closed-menu eliminates this stress and anxiety, allowing students to focus on
the interview process. Thus, what we feel makes most sense is for the two systems to coexist, closed-menu used for learning processes and chat used for rehearsing for the actual
experience."

\edo{argomentazione sui vantaggi delle simulazioni di pazienti Web-Based:}: da \cite{kleinert20153d}, che sarebbe una technology review del 2015: "Immersive patient simulators can potentially promote learning and consolidation of procedural knowledge. Web-based simulators allow time- and location-independent learning at an individual pace". \par

\edo{questo articolo parla del potenziale uso dei serious games applicati ai virtual patients e i loro relativi benefici}: \cite{tacheci2015virtual}, testo mio: "The application of serious-game logic, and its potential benefits to learning has already been discussed by \cite{tacheci2015virtual}, although the implementation presented in the article is limited to a purely Problem-Solving VP." \par

\edo{un detrattore delle game mechanics}: da \cite{adefila2020students}: "If CSCLs become too focused on game mechanics and social interaction, they may fail to provide the social space for effective learning. In this study, some participants seemed to need the guidance of the facilitator to document
their own learning experiences. They were preoccupied with the mechanics of achieving
the task, they missed the elephant in the room."\par

\edo{risultati literature review}: 

\cite{kononowicz2019virtual} (studio in particolare sui VP non VR dal 1990 al 2018. Importante perché dice che c'è bisogno di più ricerca sugli aspetti di design, che stiamo facendo noi): "Low to modest and mixed evidence suggests that when compared with traditional education, virtual patients can more effectively improve skills, and at least as effectively improve knowledge outcomes as traditional education. Education with virtual patients provides an active form of learning that is beneficial for clinical reasoning skills. Implementations vary and are likely to be broad across pre- and postregistration education, although current studies do not provide clear guidance on when to use virtual patients. We recommend further research be focused on exploring the utility of different design variants
of virtual patients." 

\cite{kyaw2019virtual} (studio in particolare solo sui VP VR): "Although the included studies encompassed a range of participants and interventions, a lack of consistent methodological approach and studies conducted in any one health care discipline makes it difficult to draw any meaningful conclusions." E poi anche: "Our findings show that when compared with traditional education or other types of digital education, such as online or offline digital education, VR may improve postintervention knowledge and skills".

\cite{foronda2020virtual} (studio focalizzato sulle simulazioni infermieristiche, considerazioni sul fatto che la terminologia (serious game, VP, vr) è usata veramente a cazzo in letteratura. hanno cercato solo i termini relativi a "virtual simulation", non serious games o altro, comunque): "The body of evidence indicates that virtual simulation improves learning outcomes. As the relatively new science of virtual simulation progresses, more evidence is needed to substantiate best practices in methodology of virtual simulation-based education. Comparative studies involving virtual simulation versus other learning modalities are indicated to inform evidence-based teaching." 

\cite{richardson2019virtual} (studio specifico sugli effetti che i VP hanno sulla comunicazione IN CAMPO FARMACEUTICO, solo post-2000. Se servono, nella sezione discussion ci sono svariate reference sul pippone dei vantaggi dei VP: personalized objectives, opportunity of practice ecc. Stranamente nelle conclusioni si parla pochissimo di comunicazione, ma più che altro di knowledge, skill, satisfaction ecc.): "VPs can improve user’s knowledge, confidence,
skills and competency".  "VPs are an additional valuable resource to develop communication and counselling for pharmacy students; use for qualified pharmacists could not be established. Quality standards may contribute to standardised development and implementation of VPs in varied professions. The studies were not robust enough to fully establish the educational merit of VPs compared with other resources. Further research should focus on implementation and user perspectives."
Considerazione sulla scarsa standardizzazione e problematiche relative: "Similarly, a greater level of comparison and evaluation of VPs would be possible if there was a standard definition of a VP and quality
standards for their development. This would not only overcome some technical limitations but also help to develop a better literature base for their use across health professions."

\cite{cook2010computerized} (è un po' vecchio ma esamina i VP in tutto il campo medico.): I risultati sono poco quantificabili, ma si suppone che i VP siano perlomeno comparabili a metodi tradizionali. 

\cite{duff2016online}: I risultati sono promising, ci vuole più investigation, ci si chiede se sono meglio come tool di training o evaluation. 

\cite{peddle2016virtual}: si prova che i risultati sono positivi, soprattutto per quanto riguarda le nontechnical skills ed in particolare la communicazion . 

\cite{lee2020effective} (il competitor, focalizzato sulla comunicazione). " Despite these strengths, however, the simple use of VPs for communication training cannot guarantee learning outcomes in practice and this is especially the case with nonverbal communication. [...]. Our study underscores the importance of instructional elements designed in VPs such as structured tutorials, scaffolding, reflection and human feedback in achieving learning."

\cite{rourke2019does} (specifico sulle practical skills, poco indicativo per noi).  

\cite{bracq2019virtual}: (review generale su simulazioni VR su nontechnical skills fatta dal 2007 al 2017)



\edo{appunti non verbal input} 
\cite{dupuy2019virtual}: Utile per verificare lo stato emozionale degli studenti i momenti in cui bisogna essere più o meno empatici durante un'intervista medica. NOTA: non sembra che l'input non verbale influenzi lo stato del VP!

\cite{guetterman2019medical, kron2017using}: "Learners reported becoming more cognizant of their facial expressions in conversation, especially eyebrow movement, nodding, and smiling. Some learned about their own unfavorable behavior that they were not previously aware of, such as twitching and fidgeting. Beyond personal awareness, learners mentioned that they became aware of the importance of nonverbal communication in establishing rapport and conveying interest, acknowledgment, and empathy", "Learners noted that they would benefit from more training like MPathic-VR and expressed interest in interacting with the system through different scenarios. ". Molti studenti hanno ritenuto che il programma li ha aiutati a capire l'importanza degli aspetti non-verbali come sorridere o annuire, che sia per migliorare le proprie capacità o addirittura accorgersi di alcuni comportamenti che avevano di cui non erano consci. La soddisfazione è alta. Qualche remora di alcuni che hanno trovato un po' innaturale gesticolare davanti a un computer, comunque questa è una questione di abitudine che si dovrebbe stemperare una volta che uno ha confidenza con lo strumento.

\cite{maicher2017developing}: Qua usano il kinect per vedere movimenti e body posture, ma in realtà non sono citati casi in cui il comportamento non verbale dell'utente influenzi lo stato del paziente. l'unico caso citato qua è che il paziente segue con lo sguardo l'utente e ogni tanto "guarda dall'altra parte", ma non sembra che il suo stato sia influenzato.

\cite{banszki2018clinical}: Il fatto che il non-verbal input dello studente conti è implicito nel fatto che c'è un umano a guardarlo, il clinical educator menziona una studentessa che si è messa a piangere, quindi di conseguenza direi che questo aspetto c'è. Nessun feedback degli studenti, solo del clinical educator. Si menziona che alcuni studenti sono "overwhelmend" ma questo riguarda più l'output che l'input non verbale.

%APPUNTI SU FEEDBACK

\cite{banszki2018clinical}: Nel future work si discute di come un sistema di feedback interno (e non limitato a quello fornito, esternamente, dal clinical educator) possa essere utile.

%\cite{quail2016student}: il feedback è dato dal clinical educator. si sottolinea l'importanza di come il feedback debba essere guidato da un clinical educator dopo, in una fase di debriefing. \cite{king2014dynamic}, \cite{forrest2013essential}, ed è stata apprezzata dagli studenti.

%\cite{zlotos2016scenario}: Cita questo \cite{mckimm2006abc} "ABC of learning and teaching: web based learning" per l'importanza di avere feedback immediato, e anche per quanto riguarda "easily updatable content"

\cite{hirumi2016advancing} & \cite{hirumi2016advancingPart2}: %cita un altro articolo \cite{barry2005features} che dice che il feedback immediato è la feature più importante per la simulation based education (SBE) e, insieme alla ripetizione e varietà dei VP sono correlati con learning outcome migliori. 
Qui l'idea di avere un feedback in termini di scoperte fatte e scoperte ancora da fare può essere d'aiuto per il learning, in quanto fa da "guida" per lo studente. Questo è confermato dagli studenti stessi che hanno apprezzato questo sistema. Uno degli studenti ha rimarcato che sbagliare e ricevere feedback immediato sul perché ha sbagliato lo ha aiutato ad imparare meglio (questo però è relativo alla parte di quiz, non strettamente al VP). 
Si dice anche che avere questo feedback costante aiuta non solo a capire se si sta andando bene o male, ma anche come si sta andando rispetto ai propri colleghi. Considerazioni sull'aspetto di gamification come score e leaderboards! Inoltre, nei suggerimenti, si butta lì l'idea del cumulative feedback accross cases, in modo che gli studenti possano tracciare i loro miglioramenti interagento con casi diversi, che può essere un fattore motivante.
%Cita anche questo qua \cite{mcgaghie2011does}, che dice che il feedbacl informativo insieme alla deliberate practice aumenta anche le abilità di clinical reasoning in ambienti controllati.

\cite{jacklin2018improving}: cita %\cite{ericsson1993role}, dove si dice che uno dei fattori fondanti per l'apprendimento, oltre alla pratica, è il feedback. Questi si sono focalizzati molto sull'aspetto didattico dell feedback, chiedendo ad esperti, e tramite varie iterazioni l'hanno aggiustato rispetto ai commenti di questi. Il feedback è spezzato in due: emozionale e tecnico. (quello emozionale però si intende solo come animazioni del presonaggio, quello tecnico è effettivamente un feedback scritto). 

%\cite{kron2017using}: gli studenti hanno apprezzato molto il feedback dato da MPathic-VR, soprattutto per quanto riguarda la sua struttura in due run, dove nella seconda hanno potuto mettere in atto quello che hanno imparato nella prima, anche a riguardo ai loro non-verbal behaviour.

%\cite{marei2018use}: cita due studi \cite{huwendiek2009design}, \cite{botezatu2010virtual} che considerano realismo, livello di difficoltà appropriato, alta interattività e FEEDBACK come tra gli aspetti più importanti di un VP, e soprattutto più apprezzati dagli utilizzatori. La sessione di feedback (esterna alla simulazione) è considerata uno degli aspetti più utili dagli studenti (utile quanto la simulazione stessa). 

\cite{ochs2019training}: il feedback è una feature che manca, citato nel future work




%DRAFT GAMIFICATION

%Cos'è e a che cacchio serve "gamificare" un'esperienza
The idea of implementing game mechanics in an experience is, generally speaking, to make an activity that is commonly perceived as mundane more enjoyable. 
Specifically for learning scenarios, such features have been proven to make the learning experience more effective, fostering self-improvement and healthy competition between peers \cite{festinger1954theory}, \cite{benedict2013promotion}. 
%Quali sono i tre "pilastri" della gamification
The three game mechanics that are most often implemented in "gamified" experiences are scores, badges and leaderboards. 
%A cosa servono i punteggi e chi li implementa e come
Scores are a quantitative and immediate form of feedback that are an intrinsic motivator to foster the user's willingness improve his/her performance and, since they are represented as numbers, exactly \emph{how much} they can improve absolutely (if a score cap is given) or relatively (to a previous score). Scoring systems can be found in \cite{guetterman2019medical,dupuy2019virtual,jeuring2015communicate,schoenthaler2017simulated,zielke2015beyond}. Specifically in \cite{jeuring2015communicate,schoenthaler2017simulated} scores are separated for each learning goal, this is useful so that learners can understand in detail in which areas he/she is already proficient and which areas he needs to improve.
%A cosa servono le badges e chi le implementa.
In the most general sense, badges are the visual representation of an achievement. Achievements and badges are used in games to prove that the player has reached an intermediate goal on his/her road to mastery. In a gamified educational experience, badges serve a twofold purpose: they are a form of gratification to the learners, to reassure them which skills they are already proficient in and where they are in their learning path, and they also are a mean to show others (peers or educators) what they have already achieved. In this sense, obtaining a badge is both a intrinsic and extrinsic motivator for improvement.
The only simulation that implements a proper badge system is \cite{zielke2016beyond}. In a similar fashion VSPR \cite{peddle2019development} features a "certificate system" where, at the end of each learning module, the user is given a printable certificate. However, this can be seen as an "intrinsic-only" motivator, since there is no overarching structure that enables user to see each others' achievements.
%A cosa servono i rankings, perché nessuno li implementa?
Leaderboards (or rankings) are a primarily extrinsic motivator for improvement, since they provide a direct comparison between one's performance and his/her peers'. While "climbing the leaderboard" can be an extremely strong motivator for competitive individuals, implementation of a public ranking system is a feature that must be considered carefully. In fact, under-performing individuals may be demotivated and avoid an experience altogether if the gap between them and the top-players is perceived as too large.
In the examined VPs, leaderboards appear to be are completely absent. The only article that mentions leaderboards as a vehicle that could be implemented to foster competition is \cite{hirumi2016advancingPart2}, where rankings and scoring systems are cited among possible future features. 
%Altre game mechanics oltre a queste tre
On final note must be made about the VP described in \cite{adefila2020students} implementing the very peculiar idea of a Tamagotchi-style VP, that needs to be cared for in real-time over two weeks. Here caring for the patient, an elderly woman called Hollie, effectively and at regular intervals ensures an improvement of her health conditions, while neglecting her could cause her to worsen and eventually die. Here, the main game mechanic (constant care over a long period) leverages an innate sense of responsibility in the players and also reproduces quite accurately the daily tasks of a nurse. 

% DRAFT NEW PRESENTATION FORMAT

%Cosa intendiamo con presentation format (definizione)
\textbf{Presentation Format}. The surveyed works provide learners with different ways for obtaining information from the VP and for presenting the VP itself.
%Tutti quelli che abbiamo trovato sono graphic -> definizione di graphic
All of the VPs considered in this survey have some kind of \emph{graphic} representation, either through images, videos or computer graphics (CG) rendered in real time. We define \emph{graphic} VPs where physical characteristics of the patient, including observable emotional aspects such as moods and emotions, are represented through a visual medium capable of dynamically adapting to the patient's state and the unfolding narrative of the simulation. 
%Con l'eccezione di sto cazzo di foster
A notable outlier is \cite{foster2016using}, which compares three different VPs, two of which are text-based (although they include a static portrait of the patient) and one features pre-rendered CG videos created with The Sims 3 videogame.
%Esistono anche i text based, giustificazione del perché i nostri sono tutti graphic.
It must be noted that the text-based presentation format is common in exclusively Problem-Solving oriented VPs, which usually leverage platforms such as Web-SP, vpSim, DecisionSim and OpenLabyrinth. %boh, le piattaforme forse si possono anche tagliare.
On the other hand, the ample use of visual media is coherent with the requirements of VPs focused on narrative aspects and communication, since maximizing immersion and presence is crucial, as seen in all of our examples.\par
%Quali sono le sottocategorie (che sostanzialmente abbiamo già detto). La confusione tra Desktop VR e Immersive VR la chiarisco sotto il cappello di real-time 3D, dopo.
As mentioned above, \emph{Graphic} Presentation Formats can be classified according to the medium they leverage: \emph{Images}, \emph{Video} and  \emph{Real-Time 3D}.
%Categoria di graphic: image -> cos'è e chi la usa
We classified in the \emph{Image} sub-class VPs that are presented through a series of static images (either photographs or drawings), such as \cite{adefila2020students,marei2018use,albright2018using}.
%Categoria: pre-recorded video -> cos'è e chi la usa
Some VPs are presented with one or more \emph{Video}s. \emph{Video} VPs use pre-recorded footage of live actors \cite{o2019suicide,peddle2019exploring} or pre-rendered CG videos (as one of the VPs presented in \cite{foster2016using}).
%Categoria: real-time 3D -> distinzione tra DVR e IVR, cosa sono.
The \emph{Real-Time 3D} category describes simulations that present the patient and the environment as 3D models rendered in real-time. Very often this category is called Virtual Reality (VR), which is a quite generic and improper name, sometimes causing confusion, especially in medical literature, as remarked by \cite{foronda2020virtual}. In fact, a distinction must be made between \emph{Desktop VR}, which describes any 3D application running on a normal computer and displayed on a standard monitor, and proper VR (which we will call \emph{Immersive VR}), that leverages immersive technologies (Head Mounted Displays or CAVE environments) to maximize the user's immersion. 
%Sottocategoria: DVR, chi la usa, note su hw
The majority (n=13) of simulations employs real-time 3D graphics \cite{dupuy2019virtual,guetterman2019medical,hirumi2016advancing,banszki2018clinical,jacklin2019virtual,jeuring2015communicate,maicher2017developing,richardson2019virtual,schoenthaler2017simulated,szilas2019virtual,washburn2020virtual,zielke2016beyond,zlotos2016scenario}. While most VPs leverage just normal computer screens, four take advantage of large displays \cite{banszki2018clinical, quail2016student}, \cite{dupuy2019virtual}, \cite{washburn2020virtual} to make the patient appear life-sized and consequently try to portray an interaction that feels more natural.
%Sottocategoria: IVR, cos'è e chi la usa e come
We found only two simulations \cite{ochs2019training}, \cite{sapkaroski2018implementation}, leverage immersive VR technologies. In \cite{ochs2019training} the same VP is deployed in three different setups: desktop VR, immersive VR with an HMD and immersive VR in a CAVE environment, to study the different technologies' influence on the sense of presence. 
%Vantaggi/svantaggi di una rispetto all'altra (in particolare pre-recorded VS real-time e DVR vs IVR)
A note must be made on pre-recorded videos (live-action or CG): while they certainly are more realistic, they fall behind in terms of flexibility to simulations presented with real-time CG. A video with an actor cannot be re-purposed to portray a different clinical case while tweaking and expanding a simulation using 3D characters can be done in a much more modular fashion, i.e. patient appearance, environments, animations and behaviours can be modified without re-building the application from the ground up.\par 
\emph{Immersive VR} technologies certainly offer a higher degree of immersion and presence over \emph{Desktop VR} applications, however there are still great accessibility issues that justify the predominance of DVR over IVR despite te fact that, in recent years, HMDs such as Oculus Rift and HTC Vive have entered the consumer market. A VP, especially if meant to be used for rehearsal or autonomous training, cannot be realistically deployed only for IVR platforms. 




\textbf{Presentation format}. The surveyed works provide learners with different ways for obtaining information from the VP and for presenting the VP itself. A first rough subdivision is between \textit{text-based} and \textit{grahics} representations. In screen-based text simulators, where the VP is presented mainly in the form of a collection of text and structured data, with the possible inclusion of images portraying a static patient portray or exam results. However, the lack of a graphic component capable of displaying a patient that can emote as the simulation unfolds (and, consequently, change posture and facial expressions according to the current emotional state) is one of the main limitations of these approaches. 

Therefore, researchers started extending text-based simulations into learning activities with a relevant graphic component. 

\andrea{mi fermo qui che faccio fatica a capire come procedere}
\edo{We considered "Graphic" VPs where physical characteristics of the patient, including observable emotional aspects such as moods and emotions, are represented through a visual media (image, pre-recorded video or CG) capable of dynamically adapting to the patient's state and the unfolding narrative of the simulation.}

\textbf{Presentation }: This is the way the VP is presented to the user, i.e., what the user \emph{sees} while he/she is interacting with the VP. Interfaces can be \emph{Text-Based} or \emph{Graphic}:
\emph{Text-Based} interfaces present the VP to the user mainly as text and structured data. We must specify that the inclusion of a static portrait, exam results, x-ray images and such are NOT sufficient to qualify a VP for the \emph{Graphic} category, where the patient portrayed must have a substantial physical presence that can emote and/or change appearance as the simulation unfolds. Graphic is a category in itself which can be further classified in \emph{3D}, \emph{Image} and \emph{Video}. 
%All the sub-classes of \emph{Graphic} are mutually exclusive.

%\emph{Text-Based} and \emph{Graphic} are mutually exclusive values. %Veramente c'è il dannatissimo articolo foster2016using che presenta 3 VP diversi (due text-based e uno graphic) e quindi è taggato sia graphic che text based e non so bene come gestirlo
The majority (n=14) of simulations employs real-time 3D graphics, 3 use static images, two use pre-recorded live-action videos, and two belong to the IVR category. A notable outlier is \cite{foster2016using}, which compares three different VPs, two of which are Text-Based and one features pre-rendered CG videos created with The Sims 3 videogame. The ample use of visual media is coherent with the requirements of VPs focused on narrative aspects and communication, since maximizing immersion and presence is extremely important, and a text-based approach, which can be sufficient for more Problem-Solving oriented VPs, can be too restrictive. 
Two simulations \cite{ochs2019training}, \cite{sapkaroski2018implementation}, leverage immersive VR technologies. 



%OLD FEEDBACK DESCRIPTION
Simulations that offer the user some kind of feedback on their performance during the simulation, in terms of available questions, interactions, topics or feedback on the general state of the virtual agent (trust-meter, consequence of action). Also if the feedback is immediate, after or during each step or is given only at the end. This feedback relates only to the simulation itself, for example if a study in an article includes a debriefing session where users discuss their experiences, it DOES NOT count, because it is a characteristic of the study / experiment, not of the VP simulation. The same goes for any feedback given by human facilitators that are not directly part of the simulation. Also the general idea where interactions of the user generate cause-effect events in the simulation (i.e. I ask something, the Virtual Human gives feedback by responding verbally or non-verbally), are a given for any virtual human interaction and considered valid for every considered case.

%%OLD INTERACTION
The \emph{Free-Text} variant can further be split into two sub-classes: \emph{Typed} and \emph{Speech-To-Text}. The latter class refers to VP simulations that feature a Speech-To-Text module where the user can interact via natural spoken language. Also \emph{Free-Text} VPs can optionally feature \emph{Non-Verbal} communication elements. These can range from prosody to facial expression to body posture. In this class fall both non-verbal input (in the sense that the simulation can process the user's emotions via cameras or other sensors) and output (the Virtual Patient can emote non-verbally via facial expressions, body posture etc.) since they are usually paired and they are uncommon in the first place. 


%%OLD CONCLUSION
%\edo{CONCLUSION 6) I free-text sono usati ma comunque meno comuni di branched-option. Citando l'articolo di tizio caio che faceva questa considerazione, il branched option viene visto più utile per il learning dei neofiti, mentre il free text è più adatto all'evaluation di studenti già esperti. Sfatiamo questo mito dicendo che anche in un software fondamentalmente free text possono coesistere modalità learning ed evaluation senza ricorrere ai percorsi ovvi e le facilitazioni del branched-option, con il vantaggio di avere il realismo e il flow di una conversazione reale} Blablabla ciuccialà\par

%\edo{CONCLUSION 8) Gli aspetti non verbali sicuramente devono essere inclusi di più nei VP del futuro, soprattutto per quelli focalizzati sulla narrativa, l'empatia e in generale la doctor-patient communication visto che si è dimostrato che gli aspetti non verbali contribuiscono in maniera preponderante al significato di un atto comunicativo.} Blablabla ciuccialà\par

%\edo{CONCLUSION 1) Problem solving è più prominent di narrative e quelli che combinano entrambe sono anche relativamente scarsi} Blablabla ciuccialà\par

%VIA
%\edo{CONCLUSION 2) i tool singoli più usati sono quelli semplici che hanno la possibilità di fare authoring. E anche il fatto di essere web based sembra estremamente importante} As stated by (\edo{citation needed}), one of the main features that make a VP appealing to professionals in health care education is the possibility to author and customize VP simulations to their specific needs. This is supported by the fact that a significant number of articles present VPs created with DecisionSim, OpenLabyrinth and Web-SP, which are relatively low-tech text-based VP tools, but on the other hand offer easy to use interfaces to create custom VP cases. From this consideration we can gather that an user-friendly editor with flexible customization features is a determinant factor in the success of a VP application. This however doesn't mean that the VP application must be simple in itself. Principles of user-friendliness and tutorialization can and must be applied also to more complex, open-ended software featuring 3D graphics or Speech-To-Text capabilities. We can also extrapolate form the gathered data that being web-based and openly accessible online via a normal browser is a major factor that can foster the adoption of a particular VP tool. The convenience of not having the need to use external executables is apparently very desirable, also for the fact that a web-based application can be readily accessible without the need of deploying software on several different platforms, such as Android or iOS devices, a factor that can greatly increase the immediate accessibility for students and faculty. \par

%VIA
%\edo{CONCLUSION 3) I numeri di graphics non sono male, ma una parte importante sono video o photo based, che non sono molto flessibili, nel senso che se ho un video è molto difficile fare il repurposing per un nuovo caso} Blablabla ciuccialà\par


%\edo{CONCLUSION 4) VR è poco usata, e qui possiamo fare il nostro discorso per cui un ambiente più immersivo o maggiore presenza diminuiscono il carico cognitivo necessario per immaginarsi la situazione e concentrarsi sui task di comunicazione e/o problem solving. AR è usata zero (magari fai l'esempio di quell'applicazione per le operazioni al ginocchio in AR che avevi trovato, anche se non è ne' un VP ne' un articolo) e anche qua cercare di capire perché e se ha senso} Blablabla ciuccialà\par

%VIA
%\edo{CONCLUSION 5) Di Multi-Party c'è una sola istanza, e la presenza di più NPC come parenti e/o altre figure professionali sono aspetti della doctor patient communication che non vanno presi sottogamba, come dicono tizio e caio, quindi sicuramente questo potrebbe essere un filone di sviluppo interessante. Magari segnalare i casi in cui c'è collaborazione inter-professionale user-side.} Blablabla ciuccialà\par





%\edo{CONCLUSION 7) Elementi di serious game e gamification sono poco esplorati, ma vale la pena di utilizzarli di più perché hanno effetti positivi sul learning come dimostrano tizio caio e sempronio.} Blablabla ciuccialà\par

%%OLD DEFINITION OF NARRATIVE:
%The Narrative label is applied to VP simulations devoting a certain degree of attention on narrative, interpersonal and communication aspects of patient-doctor relations. These can range from empathy assessment, bad news delivery, shared decision making and others.

%%OLD DEFINITON OF PS
%This is by far the most common type of VP, since the main use case of these simulation in health care education is to train skills such as history taking, differential diagnosis, drug and exam prescriptions. The VP is basically a collection different data (sympthoms, current medications, medical history etc.) and the goal of the user is to process these data and come to a solution in terms of correct diagnosis, prescriptions or needed exams.

%OLD GAMIFICATION
%In the articles considered in this survey we found little mention to serious games and gamification elements (with \cite{adefila2020students}, \cite{szilas2019virtual}, \cite{jeuring2015communicate} being notable exceptions). Several serious games (SG) and serious gamified applications (SGA) have been employed in the field of corporate communication (communication between team members, between customer and company etc) and are reported to provide high quality outcomes \cite{uskov2014serious}.  so we find reasonable to infer that a wider adoption of such design principles may yield positive results in the field of VP simulations focused on provider-patient communication. Game-like mechanics can make the learning process more enjoyable \cite{citations here}, yield to better content retention \cite{citation here}, sense of accomplishment \cite{citations here} and features such as scoring systems, in-game achievements and rankings can foster self-improvement and healthy competition between peers \cite{festinger1954theory}. \par

%OLD STUFF ABOUT NON VERBAL INPUT
%SPOSTATO IN DISCUSSION
%While several articles mention detection of non-verbal cues as one of their features \cite{dupuy2019virtual,guetterman2019medical,kron2017using,maicher2017developing,banszki2018clinical, quail2016student}, it is important to specify that the non-verbal input by the user does NOT influence the state of the virtual human. In \cite{dupuy2019virtual} the detection of the users' facial expression is used only to asses their emotional state after the simulation has ended. In \cite{maicher2017developing} a camera tracks the users' position with the only purpose of adjusting the agent's gaze. The article also states that the users' gestures are also detected, but there is no further mention as to how this influences anything in the simulation. In the two articles presenting MPathic-VR \cite{guetterman2019medical,kron2017using} it is only stated that "MPathic-VR trains the learner in nonverbal behaviors, which the user must demonstrate to the system as picked up by a sensor before continuing", so it is unclear as what effect the user's non-verbal behaviour have on the virtual humans' responses. 

%OLD STUFF FROM FEEDBACK
%Also the general idea where interactions of the user generate cause-effect events in the simulation (i.e. I ask something, the Virtual Human gives feedback by responding or reacting in some way), are a given for any virtual human interaction and considered valid for every considered case.

%OLD EDUCATIONAL DESIGN STUFF
\item \textbf{User Target}: This category distinguishes between articles that feature VPs used by single of multiple subjects:
    \begin{itemize}
        \item \emph{Single}: VPs employed by a single user. 
        \item \emph{Collaborative}: VPs designed to be interacted with by more than one user in a collaborative fashion. This design choice can be used to foster, for example, interprofessional collaboration and communication, such as in \cite{adefila2020students}. 
    \end{itemize}
    %The \emph{Single} and \emph{Collaborative} keywords exclude one another.
    
    \textbf{Environment}: This category wants to define where and when a VP is used. The use environment can be:
    \begin{itemize}

    \item \emph{Supervised}: VPs that are not meant to be worked with by a user in his/her own time, but are, for example, employed in a classroom setting supervised by a teacher or an expert. Also, several VPs are meant to be followed by a debriefing session with an educator to go over the choices made by the learner and assess the learning outcomes, as in 
    \item \emph{Unsupervised}: A VP that can be employed by a user at any time, without the need of a supervisor or an expert.
    \end{itemize}
    %These last two keywords are mutually exclusive.
    
    
    %APPUNTI NLU /
    
\edo{NLP/NLU + Corpus Harvesting... appunti} Maicher \cite{maicher2017developing} explicitly reports the use of the NLP engine ChatScript \edo{source here?}, containing approximately 2500 rules for managing conversations, . Working -> NLP basic processing  (spell checking, canonization, determining
parts of speech, analyzing type of input [question vs statement],
managing interjections, etc) -> identify conversation domain (history, familiy etc.) based on keywords -> match against rules to determine appropriate output. Features of Chatscript -> User-friendly (to non-programmers) to create and debug dialogue, context-sensitive ("tell me more" yield results based on current discussion topic), works best when conversation domain is narrow (as patient doctor communication topics).

Cynthia Young VP \cite{foster2016using} NLP engine presented here \cite{mcclendon2014use}-> Spellchecking then -> determine paraphrase.  10 different measurements of sentence similarity to determine paraphrases of a predetermined set of questions and answers. Similarity measures obtained from a machine learning model. 4500 speech triggers (keywords?). 500 answers. 

Notes on NLP from zielke2017: "Natural language processing (NLP) is an important factor in
the development of a NUL Much has been accomplished in the
field of NLP, resulting in a recent proliferation of intelligent
assistants such as Amazon Echo, Siri, and Google Home.
Today's NLP research landscape consists of hundreds of
companies and continues to grow [20]. Speech recognition is
continuously improving and has been integrated into many
widely used applications - smartphones now allow dictation of
text messages through NLP, and Windows 10 comes with the
Cortana assistant which accepts natural language input [21]. - In
this landscape of continual research and improvement of NLP
and NUl systems, it is plausible and realistic to look toward
applying NLPINUI to VRiAR virtual patient systems." and "Despite these advances in extensive AI and language
processing research, this area remains one of the greatest
challenges faced when developing virtual patients. Accurate
natural language processing and understanding is still very
difficult to achieve. Inaccurate language processing can lead to
a frustrating, immersion-breaking experience that fails to
simulate the flow of a clinical interview, and low-performing
language systems are common, with virtual patient prototypes
demonstrating 60-75% accuracy in their language processing
[17]. Virtual patients may be extensively programmed to
recognize words and phrases and respond accordingly, but true
language understanding, conversational awareness, and natural
conversation flow remain largely out of reach"

\cite{foster2016using} also thinks to implement an automatic software (NLP based) to process empathic responses, now done by humans.

\edo{note on corpus retrieval}
\edo{\cite{lok2019can} dice che VPF ha il più grande motore di corpus retrieval del mondo, mentre \cite{hirumi2016advancing} dice che: "However, even when a question is asked for which a matched response exists, the phrasing of the question must be similar to the triggers in the database"}
Virtual People Factory, on which NERVE \cite{hirumi2016advancing,Part2hirumi2016advancing,kleinsmith2015understanding} and XXX are built upon, implements un-annotated corpus retrieval algorithms \cite{lok2019can} which is employed in the respective VPs, however, in \cite{hirumi2016advancing} the authors expresses frustration with the frequent database-matching errors, that led them to implement a guided interface at first (suggesting words/pronouns) and then a fully closed-option interface.

%APPUNTI SU IVR PERCHE' E' BELLA

\edo{APPUNTI}: zielke2017: "Believable virtual patients with a natural user interface will
undoubtedly help medical students improve their verbal and
non-verbal communication skills", "While VR (Gear VR, Oculus, Vive) headsets provide
higher fidelity renditions and a fully embodied interactive
experience, they can lack in ergonomics, potentially be
disorienting to the user, and may have unresolved safety
concerns. AR can be a powerful and successful alternative that
removes some of the above-mentioned issues but can have its
own shortcomings such as limited fidelity of holograms, limited
field of view, and limited built-in CPU/HPU. Augmented
Virtuality combines the VR experience inside the headset
together with a physical space that perfectly matches the virtual
environment's dimensions"

da \cite{lok2019can}: Researchers have studied how the
physical system (e.g., display size) used to present the virtual human impacts the interaction, with a general understanding that increased levels of immersion (e.g., larger displays, immersive displays, higher quality audio) would create a more believable interpersonal interaction \cite{chuah2013exploring}, \cite{johnsen2008evaluation}.

There is a general understanding among researchers that increasing the level of immersion and realism of simulations (e.g. using large volume displays, HMDs, spatialized 3D audio, higher fidelity graphics and animations, etc.) leads to more believable human-computer interactions \cite{chuah2013exploring}, \cite{johnsen2008evaluation}. Immersive technologies increase the sense of presence and empathy towards the virtual patient \cite{ochs2019training} and, coupled with their capability to integrate natural user interfaces, can undoubtedly yield improvements on the users' communication and empathic skills \cite{zielke2017developing}.

%APPUNTI EFFECTIVENSS -> DESIGN

(Structure) Narrative dramatic structure increases engagement: \cite{marei2018use}

(Unfolding): Having more options is better: \cite{dupuy2019virtual}, \cite{peddle2019development}
Different levels of difficulty: \cite{dupuy2019virtual}, \cite{peddle2019development}, \cite{peddle2019exploring}

(Feedback): \cite{jacklin2018improving}, \cite{quail2016student}

Live support (feedback): \cite{quail2016student}, \cite{peddle2019exploring}, \cite{adefila2020students}

(Repetition, two run structure etc:) \cite{quail2016student} \cite{kron2017using}

multiple devices: \cite{richardson2019virtual}
(Good UX (UI), easy to use): \cite{peddle2019exploring}



%...........UNMENTIONED DUE TO NON SO DOVE CAZZO METTERLI E NON SO SE POI SONO COSI' IMPORTANTI

collaborative use makes learning more effective: \cite{marei2018use} \cite{zielke2016beyond}

both-way partecipation: \cite{schoenthaler2017simulated}

unnaturalness of gesturing to computer \cite{guetterman2019medical}

%...........END OF UNMENTIONED


from \cite{quail2016student}: Students feel that the (educational) design choice of having a 'live' clinical educator support (during or in debriefing sessions) is very important. (Un po' generico e non relativo ad una scelta di desgin precisa, non so se includerlo) Also, students like the idea of consistent and individualised delivery and repetition of practice...

from \cite{dupuy2019virtual}: Feedback from students suggests that VP could have different difficulty levels, more endings, more scenarios, more options to chose from (instead of 2).

from \cite{peddle2019development}: Feedback from students suggest tha VP could have more varied options to chose from (instead of 2), especially if they are experienced (so the same thing as above: different difficulty levels).  

from \cite{peddle2019exploring}: Again, underlines the importance of clinical educator support and different difficulty levels to fit the learner experience.

also from \cite{peddle2019exploring}: students' appreciation on UX: good visual experience, easy to navigate, well organised. appreciation on opportunity for repetition. appreciation of the video format over mannequin-based interaction. 

from \cite{jacklin2019virtual}: students reported that the multiple choice system was too restrictive. 

from \cite{jacklin2018improving}: experts testing the system commented on various aspects of feedback: wording, timing, etc. (which was added later) -> particular attention on how and when to give feedback.

from \cite{adefila2020students}: underlines that presence of guidance from external facilitators is necessary to highlight the learning that has taken place through the use of the simulation.

from \cite{marei2018use}: reports that "students prefer VPs that are interactive, rich in media, authentic, have appropriate difficulty and are based on real-life scenarios relevat to their future practice". Also students highlight how important is the dramatic structure: " Students have commented on the necessity of building and arranging the incidents in the VP storyboard as in the movies", so, a story that is structured properly is fundamental for engaging the users. 
Also, collaborative structure is inductive to learning.

from \cite{washburn2020virtual}: Argues, in future work, that real voice controls over text input can lead to higher perceived autenthicity of intractions with the patient. Also in future work proposes to implement immersive technologies (HMDs) to increase the immersion. 

form \cite{schoenthaler2017simulated}: This simulation has the feature of partecipating both as the doctor and as the patient, 79percent of patient found the experience useful when talking to a real doctor. 

from \cite{richardson2019virtual}: feedback form users suggest that the application should be deployed and usable on multiple devices (implemented later). Suggestion of increased feedback with pass/fail mark at the end. 

from \cite{guetterman2019medical}: MPatich vr: High performes expressed appreciation for the theme of non-verbal communication (and communication in general), but made comments about engagement issues, namely the unnaturalness of gesturing to a computer.
from \cite{kron2017using}: mpathic vr: results confirm that the two-run structure of mpathic vr is beneficial for learning.


%


%APPUNTI EMOTIONAL MODES

\cite{dupuy2019virtual} ha un emotion recognition module, ma da quel che mi sembra non è usato per influenzare lo stato del virtual patient, ma solo per fare misurazioni sulle loro metriche. 


%(virtual patient OR simulation OR virtual reality OR serious game) AND/OR healthcare AND (communication OR nontechnical skills OR doctor-patient communication OR provider-patient communication)) queste erano le keyword come le avevo scritte prima, ma poi ho un po' ridotto se no i risultati erano numeri folli ("simulation" in particulare è una fregatura)