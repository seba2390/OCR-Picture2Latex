\section{Results and discussion}
\label{sec:taxonomy}


In this section, we present the result of our research. As stated in Section \ref{sec:reviewProtocol}, the selected VPs have been labeled according to the taxonomy of terms summarized in Tables \ref{table:instructionalDesignTable} and \ref{table:technicalDesignTable}. The definition of the identified categories (which differs to a large extent from the one presented in \cite{lee2020effective}) is introduced in the following subsections, where we also discuss the survey results relative to each group. We first introduce the instructional design elements, which are connected to the technical elements necessary to realize them; afterwards, we discuss the design choices related to the technical and technological components of the simulations. Finally, we discuss the experimental evidences related to the effectiveness of the identified design elements.



% Approaching health care provider-patient communication training through computer-based learning methods has been an active research area during the last ten years. As a result, the number of works in this area has grown significantly \cite{lee2020effective,peddle2016virtual,richardson2019virtualreview}. 

% As stated in Section \ref{sec:reviewProtocol}, during the process of reading articles and collecting data, we identified several recurring characteristics and design elements that we subsequently organized in a taxonomy of terms (summarized in Tables \ref{table:instructionalDesignTable} and \ref{table:technicalDesignTable}). The definition of the identified categories (which differs to a large extent from the one presented in \cite{lee2020effective}) is introduced in the following subsections, where we also discuss the survey results relative to each group. We first introduce the instructional design elements, which are connected to the technical elements necessary to realize them, and then we discuss the design choices related to the technical and technological component of the simulations. Finally, we discuss the experimental evidences related to the effectiveness of the identified design elements. 




% %COMPREHENSIVE TABLE WITH EVERYTHING

\begin{landscape}
%\begin{longtable}{ | p{1cm} | *{15}{l} |}
%\begin{tabularx}{\linewidth} {>{\raggedleft\arraybackslash}X >{\raggedleft\arraybackslash}X >{\raggedleft\arraybackslash}X >{\raggedleft\arraybackslash}X >{\raggedleft\arraybackslash}X >{\raggedleft\arraybackslash}X >{\raggedleft\arraybackslash}X >{\raggedleft\arraybackslash}X >{\raggedleft\arraybackslash}X >{\raggedleft\arraybackslash}X}
\tiny
{\rowcolors{3}{mywhite}{mygray}
%\begin{tabularx}{\linewidth} {X | X | X | X | X | X | X | X | X | X}
\begin{tabularx}{\linewidth}
{|>{\hsize=.5\hsize\linewidth=\hsize}X |
>{\hsize=.75\hsize\linewidth=\hsize}X |
>{\hsize=.75\hsize\linewidth=\hsize}X |
>{\hsize=1.5\hsize\linewidth=\hsize}X |
>{\hsize=1\hsize\linewidth=\hsize}X |
>{\hsize=1.5\hsize\linewidth=\hsize}X |
>{\hsize=.75\hsize\linewidth=\hsize}X |
>{\hsize=.5\hsize\linewidth=\hsize}X |
>{\hsize=.75\hsize\linewidth=\hsize}X |
>{\hsize=2\hsize\linewidth=\hsize}X |}


%\textbf{Study} & \textbf{Geographical Location and Level} & \textbf{Identification} & 
%\textbf{Multiplier Result} \\ \hline \hline
%\endfirsthead
\rowcolor{lightgray}
\textbf{Article}  & \multicolumn{4}{|l|}{\textbf{Instructional Design}} & \multicolumn{5}{|l|}{\textbf{Technical Design}} \\
\rowcolor{lightgray}
& Category & Navigation & Feedback & Gamification & Hardware & Presentation & Input Interface & Distribution & Other Tech. Features \\
%& \multicolumn{1}{|l|}{Category} & \multicolumn{1}{|l|}{Navigation} & \multicolumn{1}{|l|}{Feedback} & \multicolumn{1}{|l|}{Gamification} & \multicolumn{1}{|l|}{Hardware} & \multicolumn{1}{|l|}{Presentation} & \multicolumn{1}{|l|}{Input} & \multicolumn{1}{|l|}{Distribution} & \multicolumn{1}{|l|}{Other Tech. Features}  \\ 
%\hline 
\specialrule{.1em}{.05em}{.05em} 
\endhead

\cite{adefila2020students} & Narrative, PS & \textbf{\emph{Unclear}} &	Timed events and health status &	Tamagotchi style serious game & Any device with a web browser	& Graphic (Image) & Typed &	Web-Based & /\\ 

\cite{albright2018using} & Narrative &	Closed-Option &
	Trust meter and Virtual Instructor discussing the user’s choices & / &	Any device with a web browser & Graphic (Image) & Typed	& Web-Based & /\\ 

\cite{banszki2018clinical} + \cite{quail2016student} & Narrative	& Open-Option, Non-Verbal &	/ &	/ &	PC, 64” monitor, Microphone & Graphic (3D) & Voice-Controlled & Standalone &	Human-Controlled,Gesture and Facial Expression Output \\ 

\cite{dupuy2019virtual} & Narrative &	Closed-Option, Non-Verbal &	Errors, score, time at the end. Possibility to replay certain parts. &	Score, Time	& PC, Vertical 40” screen, Microphone, Camera & Graphic (3D) & Voice-Controlled & Standalone &	Facial Expression Detection through face tracking\\ 

\cite{foster2016using} & Narrative & Open-Option & Empathy feedback available at the end of the simulation (human-generated) & / &	Any device with a web browser & Text-Based, Graphic (Video) & Typed & Web-Based & Human-Controlled Empathy Feedback\\ 

\cite{guetterman2019medical} + \cite{kron2017using} & Narrative &	Closed-Option, Non-Verbal &	Feedback at the end giving evidence on each choice of words and a score &	Score &	PC, Microsoft Kinect, Microphone & Graphic (3D) & Voice-Controlled &	Standalone &	Facial Expression and Body Posture Detection and output, Recorded Voiceover, Multiple VHs\\ 

\cite{hirumi2016advancingPart2} + \cite{hirumi2016advancing} + \cite{kleinsmith2015understanding} & Narrative, PS &	Closed OR Open-Option & Available topics, discoveries made	& NO &	Any device with a web browser	& Graphic (3D) & Typed & Web-Based & /\\ 

\cite{jacklin2019virtual} + \cite{jacklin2018improving} & Narrative & Closed-Option &	Unspecified Feedback at the end of the simulation &	/ &	Any device with a web browser & Graphic (3D) & Typed &	Web-Based & Body Posture Output, Recorded Voiceover	\\ 

\cite{jeuring2015communicate} & Narrative & Closed-Option & Scores, recap of choices made & Envisioned as a Serious Game, Scores for each learning goal	& \textbf{\emph{Unclear}} & Graphic (3D) & Typed & \textbf{\emph{Unclear}} & Scenario Editor\\ 

\cite{maicher2017developing} & Narrative, PS &	Open-Option, Non-Verbal & /	& / & PC, Microsoft Kinect, Multi-Array Microphone & Graphic (3D) & Typed + Voice-Controlled & Web-Based + Standalone & Motion-Captured Animations, Movement and Posture Detection and Output\\

\cite{marei2018use} & Narrative	& Closed-Option &	/ &	/ &	PC & Graphic (Image) &	Typed & \textbf{\emph{Unclear}} & /\\ 

\cite{ochs2019training} & Narrative	& Hybrid & / & / & PC, HMD (Oculus Rift), CAVE, High-End Microphone & Graphic (3D, IVR) & Voice-Controlled &  Standalone & Same VP deployed in desktop, VR and Cave versions. Speech recognition is Human-Controlled, Non-Verbal output (facial expression and body posture). Text-To-Speech, Lip Synch, Virtual Playback\\

\cite{o2019suicide} & Narrative, PS &	Closed-Option &	Immediate and after-action guidance, via Virtual Instructor & / & Any device with a web browser & Graphic (Video) & 	Typed & Web-Based & /\\ 

\cite{peddle2019exploring} + \cite{peddle2019development} & Narrative, PS & Closed-Option	& Possibility to replay certain parts & / & Any device with a web browser & Graphic (Video) & Typed & Web-Based & /\\ 

\cite{richardson2019virtual} & Narrative, PS &	Closed-Option &	Personalized feedback at the end to enhance counselling ability	& / &	Any device with a web browser & Graphic (3D) &	Typed &	Web-Based & /\\

\cite{sapkaroski2018implementation} & Narrative, PS & Closed-Option & / & /	& PC, HMD (Various), Leap Motion, Oculus Touch & Graphic (3D, IVR) & Typed + Voice-Controlled + NUI & Standalone & Remote progress tracking for educators\\ 

\cite{schoenthaler2017simulated} & Narrative & Closed-Option & Virtual Instructor feedback during (discussing each choice) and after the simulation, Trust Meter & Scores for each learning goal after the simulation & \textbf{\emph{Unclear}} &  Graphic (3D) & Typed & 	\textbf{\emph{Unclear}} &	Can play as both provider and patient, Non-Verbal Output\\ 

\cite{szilas2019virtual} & Narrative &	Closed-Option &	/ &	Envisioned as a serious game &	PC	& Graphic (3D) &	Typed & Standalone & /\\ 

\cite{washburn2020virtual} & Narrative, PS & Open-Option & / &	/ &	PC or Laptop, Large Screen & Graphic (3D) & Hybrid & Standalone & Human-Transcribed Voice Controls\\ 

\cite{zielke2016beyond} + \cite{zielke2016using} & Narrative, PS &	Closed-Option &	Points and badges assigned during and after the interview & Envisioned as a Serious Game, Scores, Badges	& Any device with a web browser & Graphic (3D) & Typed & 	Web-Based & Multiple VHs, Non-Verbal Output\\ 

\cite{zlotos2016scenario} & Narrative, PS & Closed-Option	& All possible outcomes are played after the simulation is over & / & Any device with a web browser & Graphic (3D) & Typed  & Web-Based & Motion-Captured Animations, Recorded Voiceover\\ 

\hline
%\end{longtable}
\end{tabularx}
}

\end{landscape}
\normalsize
% %TABLE WITH ONLY INSTRUCTIONAL DESIGN

%\begin{longtable}{ | p{1cm} | *{15}{l} |}
%\begin{tabularx}{\linewidth} {>{\raggedleft\arraybackslash}X >{\raggedleft\arraybackslash}X >{\raggedleft\arraybackslash}X >{\raggedleft\arraybackslash}X >{\raggedleft\arraybackslash}X >{\raggedleft\arraybackslash}X >{\raggedleft\arraybackslash}X >{\raggedleft\arraybackslash}X >{\raggedleft\arraybackslash}X >{\raggedleft\arraybackslash}X}

\scriptsize 
{\rowcolors{3}{mywhite}{mygray}
%\begin{tabularx}{\linewidth} {X | X | X | X | X | X | X | X | X | X}
\begin{tabularx}{\linewidth}
{|>{\hsize=.5\hsize\linewidth=\hsize}X |
>{\hsize=.75\hsize\linewidth=\hsize}X |
>{\hsize=.75\hsize\linewidth=\hsize}X |
>{\hsize=1.5\hsize\linewidth=\hsize}X |
>{\hsize=1.5\hsize\linewidth=\hsize}X |}


%\textbf{Study} & \textbf{Geographical Location and Level} & \textbf{Identification} & 
%\textbf{Multiplier Result} \\ \hline \hline
%\endfirsthead
\rowcolor{lightgray}
\textbf{Article}  & \multicolumn{4}{|l|}{\textbf{Instructional Design}}\\
\rowcolor{lightgray}
& Category & Navigation & Feedback & Gamification\\
%& \multicolumn{1}{|l|}{Category} & \multicolumn{1}{|l|}{Navigation} & \multicolumn{1}{|l|}{Feedback} & \multicolumn{1}{|l|}{Gamification} & \multicolumn{1}{|l|}{Hardware} & \multicolumn{1}{|l|}{Presentation} & \multicolumn{1}{|l|}{Input} & \multicolumn{1}{|l|}{Distribution} & \multicolumn{1}{|l|}{Other Tech. Features}  \\ 
%\hline 
\specialrule{.1em}{.05em}{.05em} 
\endhead

\cite{adefila2020students} & Narrative, PS & \textbf{\emph{Unclear}} &	Timed events and health status &	Tamagotchi style serious game\\ 

\cite{albright2018using} & Narrative &	Closed-Option &
	Trust meter and Virtual Instructor discussing the user’s choices & /\\ 

\cite{banszki2018clinical} + \cite{quail2016student} & Narrative & Open-Option, Non-Verbal & / & /\\ 

\cite{dupuy2019virtual} & Narrative &	Closed-Option, Non-Verbal &	Errors, score, time at the end. Possibility to replay certain parts. &	Score, Time\\ 

\cite{foster2016using} & Narrative & Open-Option & Empathy feedback available at the end of the simulation (human-generated) & /\\ 

\cite{guetterman2019medical} + \cite{kron2017using} & Narrative &	Closed-Option, Non-Verbal &	Feedback at the end giving evidence on each choice of words and a score &	Score\\ 

\cite{hirumi2016advancingPart2} + \cite{hirumi2016advancing} + \cite{kleinsmith2015understanding} & Narrative, PS &	Closed OR Open-Option & Available topics, discoveries made	& NO\\ 

\cite{jacklin2019virtual} + \cite{jacklin2018improving} & Narrative & Closed-Option &	Unspecified Feedback at the end of the simulation &	/\\ 

\cite{jeuring2015communicate} & Narrative & Closed-Option & Scores, recap of choices made & Envisioned as a Serious Game, Scores for each learning goal\\ 

\cite{maicher2017developing} & Narrative, PS &	Open-Option, Non-Verbal & /	& /\\

\cite{marei2018use} & Narrative	& Closed-Option &	/ &	/\\ 

\cite{ochs2019training} & Narrative	& Hybrid & / & /\\

\cite{o2019suicide} & Narrative, PS &	Closed-Option &	Immediate and after-action guidance, via Virtual Instructor & /\\ 

\cite{peddle2019exploring} + \cite{peddle2019development} & Narrative, PS & Closed-Option	& Possibility to replay certain parts & /\\ 

\cite{richardson2019virtual} & Narrative, PS &	Closed-Option &	Personalized feedback at the end to enhance counselling ability	& /\\

\cite{sapkaroski2018implementation} & Narrative, PS & Closed-Option & / & /\\ 

\cite{schoenthaler2017simulated} & Narrative & Closed-Option & Virtual Instructor feedback during (discussing each choice) and after the simulation, Trust Meter & Scores for each learning goal after the simulation\\ 

\cite{szilas2019virtual} & Narrative &	Closed-Option &	/ &	Envisioned as a serious game\\ 

\cite{washburn2020virtual} & Narrative, PS & Open-Option & / &	/\\ 

\cite{zielke2016beyond} + \cite{zielke2016using} & Narrative, PS &	Closed-Option &	Points and badges assigned during and after the interview & Envisioned as a Serious Game, Scores, Badges\\ 

\cite{zlotos2016scenario} & Narrative, PS & Closed-Option	& All possible outcomes are played after the simulation is over & /\\ 

\hline
%\end{longtable}
\end{tabularx}
}

\normalsize


\subsection{Instructional design}
\label{sec:instructionalDesign}

This category encompasses various instructional design aspects implemented in the VP scenario, such as how the VP delivers (and facilitates) learning activities and if (and how) it provides scaffolded support to improve learner's performance.


%Instructional Table: NEW VERSION
%ditched tabularx, use only normal tabular

\begin{table} [t]
\scriptsize{
\begin{center}
    \caption{Synopsis of the reviewed VPs for each instructional design category}
    \label{table:instructionalDesignTable}
    \begin{tabular}{| p{1.8cm} | p{2cm} | p{8cm} |}
    \hline
        \rowcolor{mygray}
        \multicolumn{3}{|c|}{\textbf{Instructional design}}\\
    \hline
        \rowcolor{lightgray}
        \textbf{Category}  & \textbf{Subcategory} & \textbf{Virtual Patients}\\
    \hline
    %STRUCTURE
         \multirow{2}{*}{Structure} & \emph{Narrative} & HOLLIE \cite{adefila2020students}, AtRiskInPrimaryCare \cite{albright2018using}, Dupuy \cite{dupuy2019virtual}, CynthiaYoungVP \cite{foster2016using}, Jacklin \cite{jacklin2019virtual,jacklin2018improving}, MPathic-VR \cite{guetterman2019medical,kron2017using}, Communicate! \cite{jeuring2015communicate}, Marei \cite{marei2018use},  Ochs \cite{ochs2019training}, Szilas \cite{szilas2019virtual} \\
    \cline{2-3}
        & \emph{Narrative + Problem solving} & Banszki \cite{banszki2018clinical,quail2016student}, NERVE
        \cite{hirumi2016advancingPart2,hirumi2016advancing,kleinsmith2015understanding},  Maicher \cite{maicher2017developing}, Suicide Prevention \cite{o2019suicide}, VSPR \cite{peddle2019exploring,peddle2019development}, Richardson \cite{richardson2019virtual}, CESTOLVRClinic \cite{sapkaroski2018implementation}, Schoenthaler \cite{schoenthaler2017simulated},   Washburn \cite{washburn2020virtual},  UTTimePortal \cite{zielke2016beyond,zielke2016using}, Zlotos \cite{zlotos2016scenario}\\
    \hline
    %UNFOLDING
         \multirow{3}{*}{Unfolding} & \emph{Closed-option} & HOLLIE \cite{adefila2020students}, AtRiskInPrimaryCare \cite{albright2018using}, Dupuy \cite{dupuy2019virtual}, MPathic-VR \cite{guetterman2019medical,kron2017using}, NERVE
        \cite{hirumi2016advancingPart2,hirumi2016advancing,kleinsmith2015understanding}, Jacklin \cite{jacklin2019virtual,jacklin2018improving}, Communicate! \cite{jeuring2015communicate}, Marei \cite{marei2018use}, Suicide Prevention \cite{o2019suicide}, VSPR \cite{peddle2019exploring,peddle2019development}, Richardson \cite{richardson2019virtual}, CESTOLVRClinic \cite{sapkaroski2018implementation}, Schoenthaler \cite{schoenthaler2017simulated}, Szilas \cite{szilas2019virtual}, UTTimePortal \cite{zielke2016beyond,zielke2016using}, Zlotos \cite{zlotos2016scenario}, \\
    \cline{2-3}
        & \emph{Open-option} & Banszki \cite{banszki2018clinical,quail2016student}, CynthiaYoungVP\cite{foster2016using}, NERVE
        \cite{hirumi2016advancingPart2,hirumi2016advancing,kleinsmith2015understanding}, Maicher \cite{maicher2017developing}, Washburn \cite{washburn2020virtual} \\
    \cline{2-3}
        & \emph{Hybrid} & Ochs \cite{ochs2019training}  \\
    \hline
    %FEEDBACK
        \multirow{8}{*}{Feedback} & \emph{Replay feature} & Dupuy \cite{dupuy2019virtual}, Communicate! \cite{jeuring2015communicate}, Ochs \cite{ochs2019training}, VSPR \cite{peddle2019exploring,peddle2019development}, Zlotos \cite{zlotos2016scenario}\\
    \cline{2-3}
        & \emph{Virtual instructor} & At-Risk in Primary Care \cite{albright2018using}, Suicide Prevention \cite{o2019suicide}, Schoenthaler \cite{schoenthaler2017simulated}\\
    \cline{2-3}
        & \emph{Multiple session structure} & MPathic-VR \cite{guetterman2019medical,kron2017using}  \\
    \cline{2-3}
        & \emph{Quantitative emotional feedback} & At-Risk in Primary Care \cite{albright2018using}, Schoenthaler \cite{schoenthaler2017simulated}\\
    \cline{2-3}
        & \emph{Qualitative personalized post-feedback} & Jacklin \cite{jacklin2019virtual,jacklin2018improving}, Richardson \cite{richardson2019virtual}\\
    \cline{2-3}
        & \emph{Empathy feedback} & CynthiaYoungVP \cite{foster2016using}\\
    \cline{2-3}
        & \emph{Clinical discoveries available} & NERVE
        \cite{hirumi2016advancingPart2,hirumi2016advancing,kleinsmith2015understanding}\\
    \cline{2-3}
        & \emph{Game elements} & Dupuy \cite{dupuy2019virtual}, MPathic-VR \cite{guetterman2019medical,kron2017using}, Communicate! \cite{jeuring2015communicate}, Schoenthaler \cite{schoenthaler2017simulated}, UT-Time Portal \cite{zielke2016beyond,zielke2016using}\\
    \hline
        \multirow{3}{*}{Gamification} & \emph{Scoring system} & Dupuy \cite{dupuy2019virtual}, MPathic-VR \cite{guetterman2019medical,kron2017using}, Communicate! \cite{jeuring2015communicate}, Schoenthaler \cite{schoenthaler2017simulated}, UT-Time Portal \cite{zielke2016beyond,zielke2016using}\\
    \cline{2-3}
        & \emph{Badge system} & UTTimePortal \cite{zielke2016beyond,zielke2016using}  \\
    \cline{2-3}
        & \emph{Countdown timed events} & HOLLIE \cite{adefila2020students}\\
    \hline
     \end{tabular}
\end{center}
}
\end{table}
\normalsize

%Technical Table: NEW VERSION
%ditched tabularx, use only normal tabular

\begin{table} [t]
\scriptsize{
\begin{center}
    \caption{Synopsis of the reviewed VPs for each technical design category}
    \label{table:technicalDesignTable}
    \begin{tabular}{| p{1.8cm} | p{2cm} | p{8cm} |}
    \hline
        \rowcolor{mygray}
        \multicolumn{3}{|c|}{\textbf{Technical Design}}\\
    \hline
        \rowcolor{lightgray}
        \textbf{Category}  & \textbf{Subcategory} & \textbf{Virtual Patients}\\
    \hline
    %PRESENTATION FORMAT
        \multirow{5}{1.8cm}{Presentation format} & \emph{Image} & HOLLIE \cite{adefila2020students},  Marei \cite{marei2018use}\\
    \cline{2-3}
        & \emph{Video} & CynthiaYoungVP \cite{foster2016using}, Suicide Prevention \cite{o2019suicide}, VSPR \cite{peddle2019exploring,peddle2019development}\\
    \cline{2-3}
        & \emph{Desktop VR} & AtRiskInPrimaryCare \cite{albright2018using}, MPathic-VR \cite{guetterman2019medical,kron2017using}, NERVE
        \cite{hirumi2016advancingPart2,hirumi2016advancing,kleinsmith2015understanding}, Jacklin \cite{jacklin2019virtual,jacklin2018improving}, 
        Communicate! \cite{jeuring2015communicate}, Richardson \cite{richardson2019virtual}, Schoenthaler \cite{schoenthaler2017simulated}, Szilas \cite{szilas2019virtual}, UTTimePortal \cite{zielke2016beyond,zielke2016using}, Zlotos \cite{zlotos2016scenario}\\
    \cline{2-3}
        & \emph{Large volume display} & Dupuy \cite{dupuy2019virtual}, Banszki \cite{banszki2018clinical,quail2016student}, Maicher \cite{maicher2017developing}, Washburn \cite{washburn2020virtual}\\
    \cline{2-3}
        & \emph{Immersive VR} & Ochs \cite{ochs2019training}, CESTOLVRClinic \cite{sapkaroski2018implementation}\\
    \hline
    %INPUT INTERFACE
        \multirow{4}{1.8cm}{Input interface} & \emph{Typed} & HOLLIE \cite{adefila2020students}, AtRiskInPrimaryCare \cite{albright2018using}, CynthiaYoungVP \cite{foster2016using}, NERVE
        \cite{hirumi2016advancingPart2,hirumi2016advancing,kleinsmith2015understanding}, Jacklin \cite{jacklin2019virtual,jacklin2018improving}, Communicate! \cite{jeuring2015communicate}, Maicher \cite{maicher2017developing}, Marei \cite{marei2018use}, Suicide Prevention \cite{o2019suicide}, VSPR \cite{peddle2019exploring,peddle2019development}, Richardson \cite{richardson2019virtual}, CESTOLVRClinic \cite{sapkaroski2018implementation}, Schoenthaler \cite{schoenthaler2017simulated}, Szilas \cite{szilas2019virtual}, UTTimePortal \cite{zielke2016beyond,zielke2016using}, Zlotos \cite{zlotos2016scenario}\\
    \cline{2-3}
        & \emph{Voice-controlled} & Banszki \cite{banszki2018clinical,quail2016student}, Dupuy \cite{dupuy2019virtual}, MPathic-VR \cite{guetterman2019medical,kron2017using}, Maicher \cite{maicher2017developing}, Ochs \cite{ochs2019training}, CESTOLVRClinic \cite{sapkaroski2018implementation}\\
    \cline{2-3}
        %& \emph{NUI} & CESTOLVRClinic \cite{sapkaroski2018implementation}\\
    %\cline{2-3}
        & \emph{Hybrid} & Washburn \cite{washburn2020virtual} \\
    \cline{2-3}
        & \emph{Non-verbal} & Banszki \cite{banszki2018clinical,quail2016student}, Dupuy \cite{dupuy2019virtual}, MPathic-VR \cite{guetterman2019medical,kron2017using}, Maicher \cite{maicher2017developing}, CESTOLVRClinic \cite{sapkaroski2018implementation}\\
    \hline
    % DISTRIBUTION
        \multirow{3}{*}{Distribution} & \emph{Standalone} & Banszki \cite{banszki2018clinical,quail2016student}, Dupuy \cite{dupuy2019virtual}, MPathic-VR \cite{guetterman2019medical,kron2017using}, Maicher \cite{maicher2017developing}, Marei \cite{marei2018use}, Ochs \cite{ochs2019training}, CESTOLVRClinic \cite{sapkaroski2018implementation}, Szilas \cite{szilas2019virtual}, Washburn \cite{washburn2020virtual}\\
    \cline{2-3}
        & \emph{Web-based} & HOLLIE \cite{adefila2020students}, AtRiskInPrimaryCare \cite{albright2018using}, CynthiaYoungVP \cite{foster2016using}, NERVE
        \cite{hirumi2016advancingPart2,hirumi2016advancing,kleinsmith2015understanding}, Jacklin \cite{jacklin2019virtual,jacklin2018improving}, Maicher \cite{maicher2017developing}, Suicide Prevention \cite{o2019suicide}, VSPR \cite{peddle2019exploring,peddle2019development}, Richardson \cite{richardson2019virtual}, UTTimePortal \cite{zielke2016beyond,zielke2016using}, Zlotos \cite{zlotos2016scenario}\\
    \cline{2-3}
         & \emph{Undisclosed} &  Communicate! \cite{jeuring2015communicate}, Schoenthaler \cite{schoenthaler2017simulated} \\
    \hline
     \end{tabular}
\end{center}
}
\end{table}
\normalsize

\textbf{Structure}. The \emph{structure} defines the hierarchical organization and presentation of VP-related information within the simulation.
According to \cite{bearman2001random}, two non-mutually exclusive approaches (i.e., \emph{narrative} and \emph{problem solving}) can be defined. 
The narrative VPs are characterized by a coherent storyline, with a focus on cause-effect decisions that have a direct impact on the evolution of the simulation. These VPs present the patient as more than a mere collection of data and statistics, and devote a certain degree of attention to interpersonal and communication aspects of the provider-patient interaction.  On the contrary, the \emph{problem solving} VPs are mainly used to support inquiry-based learning scenarios such as teaching clinical reasoning, differential diagnosis, and history-taking skills. These contexts do not usually concern  portraying authentic communicative acts, since they mainly involve making questions and observations. 

Scholars and researchers recognize the power of \emph{narrative} design in the creation of meaningful learning experiences \cite{bearman2001random,marei2018use}.
Narrative-based simulations that reflect the consequences of the choices and the actions made by the learner can lead to the development of more effective VPs.  In particular, for VPs used to teach communication skills, experimental  evidence supports the value of \emph{narrative} design \cite{bearman2001random}. Thus, it is not surprising that all the VPs presented in the selected works are based on this approach. Nevertheless, it is interesting to note that 10 out the \totalVPs VPs analyzed integrate the \emph{narrative} design with a \emph{problem solving} component. 
This component aims to teach particular skills like history-taking (Cynthia Young VP \cite{foster2016using}, NERVE \cite{hirumi2016advancingPart2,hirumi2016advancing,kleinsmith2015understanding}, Maicher \cite{maicher2017developing}), clinical reasoning (VSPR \cite{peddle2019exploring,peddle2019development}, Richardson \cite{richardson2019virtual},  Washburn \cite{washburn2020virtual}, UT-Time Portal \cite{zielke2016beyond,zielke2016using}, Zlotos \cite{zlotos2016scenario}),  physical examinations (HOLLIE \cite{adefila2020students}, NERVE \cite{hirumi2016advancingPart2,hirumi2016advancing,kleinsmith2015understanding},  CESTOL VR Clinic \cite{sapkaroski2018implementation}), compilation and consultation of electronic medical records (HOLLIE \cite{adefila2020students}, Maicher \cite{maicher2017developing}), and medication administration (HOLLIE \cite{adefila2020students}). 





\textbf{Unfolding}. Given the prominence of narrative design in the development of VPs for communication skill training, another relevant design element is defining how the narrative may unfold, and how the simulation can evolve between different states. A preliminary subdivision can be made among \textit{linear} and \textit{non-linear} narratives. In the former design, VPs have a linear path to follow and the decisions, questions and options possibly presented to the learner do not influence the simulation outcome. It is clear that this design severely limits learning effectiveness, and none of the works included in this survey implemented it.

On the contrary, the \emph{non-linear} navigational structure of VPs offers learners a greater flexibility, and an higher degrees of interactivity and control. In this case, two further choices are possible. In the \textit{closed-option} design, the simulation advances to the next state by selecting one of the possible alternatives or explicit paths offered to learners. Simulation states are organized in a hierarchical structure (similar to that of the \quotes{choose your own adventure} books), which stresses the cause-effect relation of the user's choices. 
The \textit{open-option} design (sometimes referred in the literature as \quotes{free-text} \cite{jacklin2019virtual,janda2004simulation,mccoy2016evaluating}  or \quotes{open-chat} \cite{hirumi2016advancing}) can be used to develop  free-form simulations where states are organized in a partially or fully interconnected structure, and users are free to interact with the VP as they wish, thus emulating the flow of a real conversation. As we will discuss  more in detail in Section \ref{sec:technicalDesign}, learners can formulate questions and statements by either typing or having their speech transcribed into written words using speech-to-text software. Then, the application parses the text and elaborates a proper response. The VP state progression can be influenced also by non-verbal cues such as gestures, body posture, expressions and sight.


%In regards to unfolding, several works report that many users feel 'restricted' by the closed-option interface \cite{dupuy2019virtual,peddle2019development,jacklin2019virtual,hirumi2016advancing}, preferring an open-option structure or the possibility to chose between the two variants, like in NERVE \cite{hirumi2016advancingPart2,hirumi2016advancing,kleinsmith2015understanding}.


The \textit{closed-option} design characterizes most of the analyzed VPs (15), with only four works  based on an \textit{open-option} design; as for the remaining, one VP implemented both options (NERVE \cite{hirumi2016advancingPart2,hirumi2016advancing,kleinsmith2015understanding}), whereas the other one can be considered an hybrid between the two designs (Ochs \cite{ochs2019training}). One explanation for this result is the lower complexity of the \textit{closed-option} implementation, although some Authors \cite{carnell2015adapting,jacklin2019virtual} argued that a such an approach may be more suitable for novices who, for example, may still be inexperienced about the procedures to follow in a patient encounter. However, other works reported that many students feel restricted by the \textit{closed-option} interface \cite{dupuy2019virtual,hirumi2016advancing,jacklin2019virtual,peddle2019development}, preferring either an \textit{open-option} structure or the possibility to chose between the two variants. It is worth noting that implementing both options allows the use of the same VP in different educational settings. %\edo{precisazione}For instance, in NERVE \cite{hirumi2016advancingPart2,hirumi2016advancing,kleinsmith2015understanding} the two systems coexist, to leverage the less stress-inducing \textit{closed-option} variant in the learning process, and the less restrictive \textit{open-option} setting for rehearsal sessions. 
%Ho cambiato leggermente questo paragrafo perché sembrava che in NERVE ci fosse una netta distinzione tra sessioni di evaluation e learning, in realtà questa potenziale divisione è solo una proposta/considerazione degli autori
For instance, in NERVE \cite{hirumi2016advancingPart2,hirumi2016advancing,kleinsmith2015understanding}, the less stress-inducing \textit{closed-option} variant is used in the learning sessions, while rehearsal sessions leverage the less restrictive \textit{open-option} setting. 
% This concept ties to another common feedback given by users: the idea that the same simulations should have diversified levels of difficulty \cite{dupuy2019virtual}, \cite{peddle2019development}, \cite{peddle2019exploring} to appropriately challenge learners of different skill levels.
% \edo{end of addon about effectiveness}
Another interesting approach is the hybrid model implemented in Ochs \cite{ochs2019training}, where the user can freely interact through voice with the VP. Then, a human facilitator selects, from a set of possible closed-options, the utterance that semantically resembles the original phrase the most, prompting the appropriate response from the patient. The advantage of this approach is that it ease the development burden of what appears to learners as an \textit{open-option} VP, with the clear disadvantage of preventing its use as a self-learning and self-evaluation tool.



\textbf{Feedback}. With the term \emph{feedback} we refer to any form of instructional scaffolding enclosed in the simulation itself (i.e., we exclude any feedback external to the simulation, such as post-simulation debrief and reflection sessions with mentors and peers).
%(i.e., we exclude any out of simulation feedback such as post-simulation debrief and reflection sessions with mentors and peers).
Feedback can be given in many different forms, from explicit messages to discoveries made, questions answered, and visual representations of the current VP state. 

While researchers recognize the relevance of immediate and after-action feedback as an essential feature in communication-based VPs (\cite{adefila2020students,jacklin2018improving,marei2018use,peddle2019exploring,quail2016student}), 
%\cite{kleinsmith2015understanding,fiorella2012applying,mckimm2006abc,albright2018using,maicher2017developing,mcgaghie2011does,barry2005features,ericsson1993role,huwendiek2009design,botezatu2010virtual}), 
it is surprising that six of the VPs surveyed (Banszki \cite{banszki2018clinical,quail2016student}, Maicher \cite{maicher2017developing}, Marei \cite{marei2018use}, Szilas \cite{szilas2019virtual}, Washburn \cite{washburn2020virtual}, CESTOLVRClinic \cite{sapkaroski2018implementation}) do not embed any type of built-in feedback. 
The remaining works take different approaches. Four VPs (Dupuy \cite{dupuy2019virtual}, Communicate! \cite{jeuring2015communicate}, VSPR \cite{peddle2019exploring,peddle2019development}, Zlotos \cite{zlotos2016scenario}) offer the possibility, after the simulation is completed, to \textit{replay} some of its parts and analyze the outcome of different choices. 
Three simulations (At-Risk in Primary Care \cite{albright2018using}, Suicide Prevention \cite{o2019suicide},  Schoenthaler \cite{schoenthaler2017simulated}) include a \emph{virtual instructor}, i.e., a virtual tutor that gives advice or feedback based on the user's choices.
MPathic-VR \cite{guetterman2019medical,kron2017using} employs a \emph{multiple session structure}, where the first run acts as a learning phase, concluded by an automated and complete feedback provided by the system, whereas the second run (set in the same scenario) serves as an evaluation phase. This type of structure appears to be highly appreciated by students since they can immediately put into practice what they learned during the first phase, taking into account the feedback received.
Two VPs (At-Risk in Primary Care \cite{albright2018using}, Schoenthaler \cite{schoenthaler2017simulated}) feature \emph{quantitative emotional feedback} in the form of an on-screen trust meter that indicates users how effective their communication choices were at building a relation with the patient. Two VPs (Jacklin \cite{jacklin2019virtual,jacklin2018improving}, Richardson \cite{richardson2019virtual}) offer learners a \emph{personalized qualitative feedback} at the end of the simulation. CynthiaYoungVP \cite{foster2016using} uses an hybrid approach between automated and human feedback. At the end of the simulation, the students can access a web page containing \emph{empathy feedback} and scores for each response given, where scores are manually assigned by human experts. 
The approach adopted in NERVE \cite{hirumi2016advancing,hirumi2016advancingPart2,kleinsmith2015understanding} is to inform learners about the number of \emph{clinical discoveries} and empathic responses available, thus providing inexperienced users with useful guidelines on how to proceed with the conversation. Although it has not been  implemented yet, the work in  \cite{hirumi2016advancingPart2} put forth the proposition of providing \quotes{cumulative feedback} on how users are developing their skills across multiple patient scenarios. Authors suggest that this feature could both encourage repeated use of the system and act as a motivator for performance improvement. 
Finally, there is a number of VPs (HOLLIE \cite{adefila2020students}, Dupuy \cite{dupuy2019virtual}, Communicate! \cite{jeuring2015communicate}, MPathic-VR \cite{guetterman2019medical,kron2017using}, Schoenthaler \cite{schoenthaler2017simulated}, UT-Time Portal \cite{zielke2016beyond,zielke2016using}) that leverage \emph{game elements} as feedback. Since the introduction of game elements is a relevant design feature, it is discussed in detail in the  following subsection. 


%GAMIFICATION 
\textbf{Gamification}. The idea of introducing game mechanics in any learning experience is to make them more enjoyable and engaging.  Researchers and practitioners recognize that game mechanics contribute to making the learning experience more effective, fostering self-improvement and healthy competition between peers \cite{benedict2013promotion,festinger1954theory}. 
The mechanics used the most in \quotes{gamified} experiences are \emph{scores}, \emph{badges} and \emph{leaderboards}. 
Scores are a quantitative and immediate form of feedback that acts as an extrinsic motivator to foster users to improve their performance. \emph{Scoring systems} can be found in Dupuy \cite{dupuy2019virtual}, Communicate! \cite{jeuring2015communicate}, MPathic-VR \cite{guetterman2019medical,kron2017using}, Schoenthaler \cite{schoenthaler2017simulated} and UT-Time Portal \cite{zielke2016beyond,zielke2016using}. In particular, Communicate! \cite{jeuring2015communicate} and Schoenthaler \cite{schoenthaler2017simulated} provide separate scores for each learning goal (e.g., empathy control, language clarity, and pick up of patient's concerns). Such a feature can help learners to tell the areas in which they are already proficient, distinguishing them from those needing improvement.

Badges are visual representations used in games to prove that the player has reached an intermediate goal on his/her road to mastery. Their purpose is twofold: they are a form of gratification to the learners, and they allow trainees to share achievements with peers and educators. Thus, they also represent an extrinsic motivator for improvement.
%Thus, they represent both an intrinsic and extrinsic motivator for improvement.
In our survey,  the only simulation we found that implements a  \emph{badge system} is UT-Time Portal \cite{zielke2016beyond,zielke2016using}. VSPR \cite{peddle2019exploring,peddle2019development} features a system of certificates issued to users at the end of each learning module which shall be regarded as an \quotes{intrinsic-only} motivator, since there is no overarching structure that enables users to see each others' achievements.
%A cosa servono i rankings, perché nessuno li implementa?

Finally, leaderboards (or rankings) are a primarily extrinsic motivator that leverage competition with peers when they compare their performance to that of others (it should be noted that, for highly competitive individuals, the act of \quotes{climbing the leaderboard} can also be seen as a relevant intrinsic motivator independent of the context).
%Finally, leaderboards (or rankings) are a relevant intrinsic motivator for learners that leverage competition with peers when they compare their performance to that of others.
Surprisingly, despite their demonstrated benefits for learning, in our survey, we found no example of public ranking and leaderboards.

A final note is for HOLLIE \cite{adefila2020students}, which implements the very peculiar idea of a Tamagotchi-style VP the learners have to care for (adequately, at regular intervals and in real-time) over two weeks. Here, the leading game mechanics (constant care over a long period) reproduces quite accurately the daily tasks of a nurse leveraging the innate sense of responsibility in the players. 


%%TABLE WITH TECHNICAL DESGIN ONLY
%\begin{longtable}{ | p{1cm} | *{15}{l} |}
%\begin{tabularx}{\linewidth} {>{\raggedleft\arraybackslash}X >{\raggedleft\arraybackslash}X >{\raggedleft\arraybackslash}X >{\raggedleft\arraybackslash}X >{\raggedleft\arraybackslash}X >{\raggedleft\arraybackslash}X >{\raggedleft\arraybackslash}X >{\raggedleft\arraybackslash}X >{\raggedleft\arraybackslash}X >{\raggedleft\arraybackslash}X}
{\rowcolors{3}{mywhite}{mygray}
%\begin{tabularx}{\linewidth} {X | X | X | X | X | X | X | X | X | X}
\begin{tabularx}{\linewidth}
\footnotesize
{|>{\hsize=0.5\hsize\linewidth=\hsize}X |
>{\hsize=1.5\hsize\linewidth=\hsize}X |
>{\hsize=.75\hsize\linewidth=\hsize}X |
>{\hsize=.5\hsize\linewidth=\hsize}X |
>{\hsize=.75\hsize\linewidth=\hsize}X |
>{\hsize=2\hsize\linewidth=\hsize}X |}


%\textbf{Study} & \textbf{Geographical Location and Level} & \textbf{Identification} & 
%\textbf{Multiplier Result} \\ \hline \hline
%\endfirsthead
\rowcolor{lightgray}
\textbf{Article} & \multicolumn{5}{|l|}{\textbf{Technical Design}} \\
\rowcolor{lightgray}
& Hardware & Presentation & Input Interface & Distribution & Other Tech. Features \\
%& \multicolumn{1}{|l|}{Category} & \multicolumn{1}{|l|}{Navigation} & \multicolumn{1}{|l|}{Feedback} & \multicolumn{1}{|l|}{Gamification} & \multicolumn{1}{|l|}{Hardware} & \multicolumn{1}{|l|}{Presentation} & \multicolumn{1}{|l|}{Input} & \multicolumn{1}{|l|}{Distribution} & \multicolumn{1}{|l|}{Other Tech. Features}  \\ 
%\hline 
\specialrule{.1em}{.05em}{.05em} 
\endhead

\cite{adefila2020students} & Any device with a web browser	& Graphic (Image) & Typed &	Web-Based & /\\ 

\cite{albright2018using} & Any device with a web browser & Graphic (Image) & Typed	& Web-Based & /\\ 

\cite{banszki2018clinical} + \cite{quail2016student} & PC, 64” monitor, Microphone & Graphic (3D) & Voice-Controlled & Standalone &	Human-Controlled,Gesture and Facial Expression Output \\  

\cite{dupuy2019virtual} & PC, Vertical 40” screen, Microphone, Camera & Graphic (3D) & Voice-Controlled & Standalone &	Facial Expression Detection through face tracking\\ 

\cite{foster2016using} & Any device with a web browser & Text-Based, Graphic (Video) & Typed & Web-Based & Human-Controlled Empathy Feedback\\ 

\cite{guetterman2019medical} + \cite{kron2017using} & PC, Microsoft Kinect, Microphone & Graphic (3D) & Voice-Controlled &	Standalone &	Facial Expression and Body Posture Detection and output, Recorded Voiceover, Multiple VHs\\ 

\cite{hirumi2016advancingPart2} + \cite{hirumi2016advancing} + \cite{kleinsmith2015understanding} & Any device with a web browser	& Graphic (3D) & Typed & Web-Based & /\\ 

\cite{jacklin2019virtual} + \cite{jacklin2018improving} & Any device with a web browser & Graphic (3D) & Typed &	Web-Based & Body Posture Output, Recorded Voiceover	\\ 

\cite{jeuring2015communicate} & \textbf{\emph{Unclear}} & Graphic (3D) & Typed & \textbf{\emph{Unclear}} & Scenario Editor\\ 

\cite{maicher2017developing} & PC, Microsoft Kinect, Multi-Array Microphone & Graphic (3D) & Typed + Voice-Controlled & Web-Based + Standalone & Motion-Captured Animations, Movement Posture Detection and Output\\

\cite{marei2018use} & PC & Graphic (Image) &	Typed & \textbf{\emph{Unclear}} & /\\ 

\cite{ochs2019training} & PC, HMD (Oculus Rift), CAVE, High-End Microphone & Graphic (3D, IVR) & Voice-Controlled &  Standalone & Same VP deployed in desktop, VR and Cave versions. Speech recognition is Human-Controlled, Non-Verbal output (facial expression and body posture). Text-To-Speech, Lip Synch, Virtual Playback\\

\cite{o2019suicide} & Any device with a web browser & Graphic (Video) & 	Typed & Web-Based & /\\ 

\cite{peddle2019exploring} + \cite{peddle2019development} & Any device with a web browser & Graphic (Video) & Typed & Web-Based & /\\ 

\cite{richardson2019virtual} & Any device with a web browser & Graphic (3D) &	Typed &	Web-Based & /\\

\cite{sapkaroski2018implementation} & PC, HMD (Various), Leap Motion, Oculus Touch & Graphic (3D, IVR) & Typed + Voice-Controlled + NUI & Standalone & Remote progress tracking for educators\\ 

\cite{schoenthaler2017simulated} & \textbf{\emph{Unclear}} &  Graphic (3D) & Typed & 	\textbf{\emph{Unclear}} &	Can play as both provider and patient, Non-Verbal Output\\ 

\cite{szilas2019virtual} & PC	& Graphic (3D) &	Typed & Standalone & /\\ 

\cite{washburn2020virtual} & PC or Laptop, Large Screen & Graphic (3D) & Hybrid & Standalone & Human-Transcribed Voice Controls\\ 

\cite{zielke2016beyond} + \cite{zielke2016using} & Any device with a web browser & Graphic (3D) & Typed & 	Web-Based & Multiple VHs, Non-Verbal Output\\ 

\cite{zlotos2016scenario} & Any device with a web browser & Graphic (3D) & Typed  & Web-Based & Motion-Captured Animations, Recorded Voiceover\\ 

\hline
%\end{longtable}
\end{tabularx}
}



\subsection{Technical features}
\label{sec:technicalDesign}

This category explores, from a technical perspective, the different solutions that can support (and implement) the choices made in the instructional design, i.e., which are the technical features that enable the accomplishment of the envisioned learning activities. These features include the physical devices required to guarantee the exchange of information between the learner and the system, and the possible communication infrastructure needed to run the simulation.

%%Technical Table: NEW VERSION
%ditched tabularx, use only normal tabular

\begin{table} [t]
\scriptsize{
\begin{center}
    \caption{Synopsis of the reviewed VPs for each technical design category}
    \label{table:technicalDesignTable}
    \begin{tabular}{| p{1.8cm} | p{2cm} | p{8cm} |}
    \hline
        \rowcolor{mygray}
        \multicolumn{3}{|c|}{\textbf{Technical Design}}\\
    \hline
        \rowcolor{lightgray}
        \textbf{Category}  & \textbf{Subcategory} & \textbf{Virtual Patients}\\
    \hline
    %PRESENTATION FORMAT
        \multirow{5}{1.8cm}{Presentation format} & \emph{Image} & HOLLIE \cite{adefila2020students},  Marei \cite{marei2018use}\\
    \cline{2-3}
        & \emph{Video} & CynthiaYoungVP \cite{foster2016using}, Suicide Prevention \cite{o2019suicide}, VSPR \cite{peddle2019exploring,peddle2019development}\\
    \cline{2-3}
        & \emph{Desktop VR} & AtRiskInPrimaryCare \cite{albright2018using}, MPathic-VR \cite{guetterman2019medical,kron2017using}, NERVE
        \cite{hirumi2016advancingPart2,hirumi2016advancing,kleinsmith2015understanding}, Jacklin \cite{jacklin2019virtual,jacklin2018improving}, 
        Communicate! \cite{jeuring2015communicate}, Richardson \cite{richardson2019virtual}, Schoenthaler \cite{schoenthaler2017simulated}, Szilas \cite{szilas2019virtual}, UTTimePortal \cite{zielke2016beyond,zielke2016using}, Zlotos \cite{zlotos2016scenario}\\
    \cline{2-3}
        & \emph{Large volume display} & Dupuy \cite{dupuy2019virtual}, Banszki \cite{banszki2018clinical,quail2016student}, Maicher \cite{maicher2017developing}, Washburn \cite{washburn2020virtual}\\
    \cline{2-3}
        & \emph{Immersive VR} & Ochs \cite{ochs2019training}, CESTOLVRClinic \cite{sapkaroski2018implementation}\\
    \hline
    %INPUT INTERFACE
        \multirow{4}{1.8cm}{Input interface} & \emph{Typed} & HOLLIE \cite{adefila2020students}, AtRiskInPrimaryCare \cite{albright2018using}, CynthiaYoungVP \cite{foster2016using}, NERVE
        \cite{hirumi2016advancingPart2,hirumi2016advancing,kleinsmith2015understanding}, Jacklin \cite{jacklin2019virtual,jacklin2018improving}, Communicate! \cite{jeuring2015communicate}, Maicher \cite{maicher2017developing}, Marei \cite{marei2018use}, Suicide Prevention \cite{o2019suicide}, VSPR \cite{peddle2019exploring,peddle2019development}, Richardson \cite{richardson2019virtual}, CESTOLVRClinic \cite{sapkaroski2018implementation}, Schoenthaler \cite{schoenthaler2017simulated}, Szilas \cite{szilas2019virtual}, UTTimePortal \cite{zielke2016beyond,zielke2016using}, Zlotos \cite{zlotos2016scenario}\\
    \cline{2-3}
        & \emph{Voice-controlled} & Banszki \cite{banszki2018clinical,quail2016student}, Dupuy \cite{dupuy2019virtual}, MPathic-VR \cite{guetterman2019medical,kron2017using}, Maicher \cite{maicher2017developing}, Ochs \cite{ochs2019training}, CESTOLVRClinic \cite{sapkaroski2018implementation}\\
    \cline{2-3}
        %& \emph{NUI} & CESTOLVRClinic \cite{sapkaroski2018implementation}\\
    %\cline{2-3}
        & \emph{Hybrid} & Washburn \cite{washburn2020virtual} \\
    \cline{2-3}
        & \emph{Non-verbal} & Banszki \cite{banszki2018clinical,quail2016student}, Dupuy \cite{dupuy2019virtual}, MPathic-VR \cite{guetterman2019medical,kron2017using}, Maicher \cite{maicher2017developing}, CESTOLVRClinic \cite{sapkaroski2018implementation}\\
    \hline
    % DISTRIBUTION
        \multirow{3}{*}{Distribution} & \emph{Standalone} & Banszki \cite{banszki2018clinical,quail2016student}, Dupuy \cite{dupuy2019virtual}, MPathic-VR \cite{guetterman2019medical,kron2017using}, Maicher \cite{maicher2017developing}, Marei \cite{marei2018use}, Ochs \cite{ochs2019training}, CESTOLVRClinic \cite{sapkaroski2018implementation}, Szilas \cite{szilas2019virtual}, Washburn \cite{washburn2020virtual}\\
    \cline{2-3}
        & \emph{Web-based} & HOLLIE \cite{adefila2020students}, AtRiskInPrimaryCare \cite{albright2018using}, CynthiaYoungVP \cite{foster2016using}, NERVE
        \cite{hirumi2016advancingPart2,hirumi2016advancing,kleinsmith2015understanding}, Jacklin \cite{jacklin2019virtual,jacklin2018improving}, Maicher \cite{maicher2017developing}, Suicide Prevention \cite{o2019suicide}, VSPR \cite{peddle2019exploring,peddle2019development}, Richardson \cite{richardson2019virtual}, UTTimePortal \cite{zielke2016beyond,zielke2016using}, Zlotos \cite{zlotos2016scenario}\\
    \cline{2-3}
         & \emph{Undisclosed} &  Communicate! \cite{jeuring2015communicate}, Schoenthaler \cite{schoenthaler2017simulated} \\
    \hline
     \end{tabular}
\end{center}
}
\end{table}
\normalsize


\textbf{Presentation format}. The surveyed works provide learners with different types of outputs aimed to deliver VP information to the learner and presenting the VP itself. A first rough subdivision is between \textit{text-based} and \textit{graphic} representations. In screen-based text simulators, the VP is presented mainly in the form of a collection of text and structured data, with the possible inclusion of images portraying a static patient or exam results. However, the lack of a graphic component capable of displaying a patient that can express emotions as the simulation unfolds (and, consequently, change posture and facial expressions according to its current  state) is one of the main limitations of these approaches. Therefore, researchers started extending text-based simulations into learning activities with a relevant graphic component. 

All the VPs surveyed in this work fall in the \emph{graphic} category, which can be further classified in \emph{image}, \emph{video} and \emph{3D}. VPs in the \emph{image} subclass are presented through a series of static images (either photographs or drawings), such as HOLLIE \cite{adefila2020students} and Marei \cite{marei2018use}.  Some VPs present their case using \emph{video}, either in the form of live footage (Suicide Prevention \cite{o2019suicide}, VSPR \cite{peddle2019exploring,peddle2019development}) or as a computer-generated offline video (Cynthia Young VP \cite{foster2016using}). However, a clear limitation of this approach is its lack of flexibility, since the actor video cannot be re-purposed to portray a different clinical case. 

The majority of surveyed simulations fall in the \emph{3D} subclass and present the patient and the environment as 3D models rendered in real-time.  Their main advantage is that tweaking and expanding a simulation using 3D characters can be done in a much more modular fashion than with \emph{image} and \emph{video}-based VPs. Another advantage is that the sense of immersion and presence are greater than those that can be delivered by \emph{image} and \emph{video}-based VPs. 

Most of the 3D approaches rely on standard \emph{desktop VR} (DVR) settings (10, namely AtRiskInPrimaryCare \cite{albright2018using}, MPathic-VR \cite{guetterman2019medical,kron2017using}, NERVE \cite{hirumi2016advancingPart2,hirumi2016advancing,kleinsmith2015understanding}, Jacklin \cite{jacklin2019virtual,jacklin2018improving}, Communicate! \cite{jeuring2015communicate}, Richardson \cite{richardson2019virtual}, Schoenthaler \cite{schoenthaler2017simulated}, Szilas \cite{szilas2019virtual}, UTTimePortal \cite{zielke2016beyond,zielke2016using}, Zlotos \cite{zlotos2016scenario}). However, since trying to maximize the feeling of immersion and presence is extremely relevant for engaging learners and helping them achieve the expected learning outcomes, some works integrate (partially or fully) immersive technologies. Four of them (Dupuy \cite{dupuy2019virtual}, Banszki \cite{banszki2018clinical,quail2016student}, Maicher \cite{maicher2017developing}, Washburn \cite{washburn2020virtual}) take advantage of \emph{large volume displays}  to portray a life-sized and more natural interaction with the patient, and CESTOL VR Clinic \cite{sapkaroski2018implementation} uses an HMD for the same purpose. In Ochs \cite{ochs2019training}, three different setups (DVR, immersive VR with an HMD, and immersive VR in a CAVE) are compared to analyze their effect on the sense of presence. The outcome of this experiment demonstrates that immersive environments improve the sense of presence and perception of the VP, with the CAVE scoring slightly better than the HMD.
It should be noted, however, that while \emph{immersive VR} (IVR) offers a higher degree of immersion and presence over DVR, there are still accessibility issues that limit its use, in particular when the VP is intended for self-learning and self-training.


\textbf{Input Interface}. This category describes the input methods through which the user influences the unfolding of the VP simulation.
In the case of \emph{typed} interfaces, user's textual intents are entered by typing on a keyboard or selecting an item in a predefined list of choices.
\emph{Voice-controlled} simulations use natural language, which is then parsed into text through a speech-to-text module, usually offered by external Natural Language Processing (NLP) APIs \cite{foster2016using,maicher2017developing}. 
Finally, the integration within the simulation of Natural User Interfaces (NUI) allows to influence the VP state evolution through additional \textit{non-verbal} input cues such as eye contact, distance, facial expression, gestures and body posture, which can be captured with cameras and other hardware. % to provide additional input cues for the simulation.


Among the analyzed VPs, 14 feature a \emph{typed}-only input, only five are \emph{voice-controlled} (Banszki \cite{banszki2018clinical,quail2016student}, Dupuy \cite{dupuy2019virtual}, MPathic-VR \cite{guetterman2019medical,kron2017using}, Ochs \cite{ochs2019training}, CESTOL VR Clinic \cite{sapkaroski2018implementation}), Maicher \cite{maicher2017developing} has both options, and Washburn \cite{washburn2020virtual} can be considered as a hybrid solution since a human facilitator transcribes the spoken commands through a \emph{typed} interface. 
One of the reasons behind the limited use of voice controls is the fear, expressed by some Authors \cite{maicher2017developing,ochs2019training}, that  NLP systems may be technically hard to implement and prone to wrong transcriptions, which may lead to misunderstandings or unrecognized utterances, break the sense of immersion and cause frustration in the user \cite{bloodworth2010initial}.
This is why some Authors (e.g., Banszki \cite{banszki2018clinical,quail2016student}, Ochs \cite{ochs2019training}, and Washburn \cite{washburn2020virtual}) decided to have a human facilitator taking over the function of the NLP module. 
 Moreover, a VP featuring only voice controls cannot be used by learners with speech impairments \cite{maicher2017developing}. However, it should be stressed that, nowadays, speech-to-text APIs have become widely available, and their quality keeps improving; thus, problems related to imprecise transcriptions should be less and less daunting in the coming years. As for the impaired people, a smart solution to achieve maximum flexibility and accessibility could be to let the users  choose between  \emph{typed} and \emph{voice-controlled} interfaces freely. It should also be noted that IVR environments favour the use of \emph{voice-controlled} interfaces over alternative solutions such as virtual keyboards or situation-specific control boards  %, which requires for interaction the use of specific controllers, gestures or gaze/head tracking 
 \cite{sapkaroski2018implementation}, which are likely to break the sense of immersion and presence and are often cumbersome to use.

Among the analyzed VPs, five of them support also \emph{non-verbal} input, by either leveraging NUI-based approaches (e.g., using RGBD sensors, like in MPathic-VR \cite{guetterman2019medical,kron2017using} and Maicher \cite{maicher2017developing}, or standard RGB cameras, like in Dupuy \cite{dupuy2019virtual}) or having a human controller that observes the user interacting with the VP and updates the VP's response accordingly in terms of gestures and facial expressions (like in Banszki \cite{banszki2018clinical,quail2016student}). 
However, apart from Banszki \cite{banszki2018clinical,quail2016student}, it appears that this information is largely underutilized to influence the VP's behavior. In Dupuy \cite{dupuy2019virtual}, the users' facial expressions are detected to merely assess their emotional state at the end of the simulation. In Maicher \cite{maicher2017developing}, the tracked user position is simply used to adjust the agent's gaze, and there is no specific mention of the way the simulation exploits gestures. Finally, in  MPathic-VR \cite{guetterman2019medical,kron2017using}, instead of continuously capturing \emph{non-verbal} communication, learners are forced to assume specific expressions and poses when prompted by the system. In summary, the above discussion highlights that sounder ways of using \emph{non-verbal} inputs are sorely needed in this particular research field.


\textbf{Distribution}. One relevant technological parameter of the VP simulation is the way the application is distributed (and how learners can access it). In principle, there are two main options. The first option is to deploy the VP as a \emph{web-based} application that can be accessed over the Internet. Such a simulation often runs inside a web browser (which makes it device-independent), and generally requires a low amount of computational resources. This flexibility can also foster self-learning (since simulations can be accessed at places and times convenient for the learner) and helps reduce costs (since learning can be carried out online). However, since \emph{web-based} applications are required to be portable on many devices (including mobile ones), they generally sacrifice technical characteristics and computational complexity in favour of accessibility. On the other hand, \emph{standalone} applications are deployed locally on a computer or workstation. These simulations can implement more advanced and complex features since they can leverage the full computational power of a dedicated machine, and integrate external devices or sensors (such as high-quality cameras and microphones).

In our survey, we found a total of ten \emph{web-based} and eight \emph{standalone} applications; in two cases, this information was undisclosed in the paper, whereas in one case (Maicher \cite{maicher2017developing}), the VP was deployed in both variants. This latter work is interesting since it shows how a \emph{standalone} version can trade off some of the flexibility of the web one with a broad array of  features (such as voice control and gesture/posture detection). 
The Authors observed that students were significantly more engaged with the \emph{standalone} VP, whereas in the web version they had to focus on typing and reading, which make them be less prone to notice the subtle non-verbal cues manifested by the patient.

% \edo{addon about effectiveness}
% This doesn't mean that a VP simulation must sacrifice accessibility over technical capabilities at all costs, in fact \cite{richardson2019virtual} mentions user feedback asking for a simulation that is available on multiple platforms (phones, tablets, desktop). So, while a VP that incorporates many interesting technical features may be preferable for maximizing user engagement, it being deployed also in a more accessible but feature-light is a good idea to maximize its potential use cases.
% \edo{end of addon about effectiveness}

Nonetheless, it should be stressed that technology is advancing rapidly, and personal devices come equipped with ever better microphones, cameras and computational power, which can reduce the technological gap between (desktop-only) \emph{standalone} applications and \emph{web-based} ones. Further discussions on this topic are included in Section \ref{sec:openResearch}.  
    

    
\subsection{Effectiveness of design elements}
\label{sec:effectiveness}
The general effectiveness of VPs on developing communication skills has been discussed by several Authors \cite{lee2020effective,peddle2016virtual,richardson2019virtualreview}. A common complaint in VP-related literature is the lack of a standardized terminology that, coupled with a considerable heterogeneity in study design, makes the retrieval and evaluation of relevant works a troublesome task. Despite this situation, both \cite{lee2020effective} and \cite{peddle2016virtual} concluded that, when appropriately contextualized in a well thought out educational context, VPs are indeed useful for developing, practising and building confidence about communication and other skills like, e.g., decision making and teamwork.  

Based on these findings, one possible question arising from our review is if the surveyed papers provide pieces of evidence about the effects on learning outcomes and efficacy of the simulation of the different instructional design elements and the technical features available. 
Unfortunately, the answer is negative. In most of the analyzed works, the Authors reported only users' feedback or comments about a particular element/feature, and a direct comparison between different design choices is missing. The only notable exceptions are three. The first one is represented by \cite{ochs2019training}, in which  different presentation formats were assessed, showing that immersive VR technologies yield superior results when compared to non-immersive ones. The second one concerns the distribution method  \cite{maicher2017developing}. The Authors found that a standalone application can provide a considerably higher level of engagement than its web-based counterpart thanks to the possibility to leverage advanced technical features (voice-controlled input and large volume displays) to increase immersion and focus on the task at hand. The third one compared closed and open-option unfolding designs, highlighting the advantages and disadvantages of each variant \cite{hirumi2016advancing}.
%reporting learners' preferences for the latter.  %Non proprio, hirumi non si sbilancia sull'una o sull'altra, semplicemente evidenzia i vantaggi e gli svantaggi dell'una o dell'altra

% In conclusion, the current literature lacks a thorough evaluation of the effectiveness of alternative designs, and further work has to be done to develop a better understanding of instructional elements and technical features that VP simulations can offer in order to achieve the desired learning outcomes. 
% \andrea{If we have something interesting to add (something like "In our opinion, a relevant contribution to this issue would be ...") we can do it here or in the open research questions}



