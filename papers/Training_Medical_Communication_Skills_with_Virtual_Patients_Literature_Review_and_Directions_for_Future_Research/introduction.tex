\section{Introduction}
\label{sec:intro}

The communication between patients and doctors is a central component of health care practice. On the one hand, good doctor-patient communication help physicians better identify patient's needs, perceptions and expectations \cite{Longnecker2010}.
On the other hand, it is not surprising that patients rate open communication as one of the most important aspects of their relationship with the physicians \cite{dibbelt2010patient}.
Research has shown that an effective, patient-centered communication is important to increase patient satisfaction  \cite{fiscella2004patient,franks2005patients,hickson2002patient,kee2018communication,kohatsu2004characteristics,papadakis2005disciplinary,stelfox2005relation}. It can have also beneficial effects on patient's health, improving physiologic measures as blood pressure and glucose levels \cite{stewart1995effective}, increasing understanding and adherence to therapy \cite{King2013}, and even creating a placebo effect in some cases \cite{kelley2009patient}.  Conversely, a poor grasp of communication skills can be detrimental to both the patient's and their relatives' health \cite{judge2004affect,kee2018communication}, and may lead to malpractice accusations.

Communication is a complex phenomenon that's not restricted to the verbal domain. As outlined in \cite{kee2018communication}, there are several critical aspects that have been subject to patients' complaints. They include elements of non-verbal communication (e.g., lack of eye-contact, negative facial expressions, use of \quotes{improper} prosodic features), inappropriate choice of words, lack of pauses to let patients ask questions, lack of listening, issues with the information given, and lack of empathy or even disrespect and poor attitudes.


Given the relevance of the problem, a primary goal in healthcare is training clinicians and healthcare providers in developing effective communication skills. Today, standardized patients (SP, i.e., actors who are instructed to represent a patient during a clinical encounter with a healthcare provider) are considered the gold standard in such training programs. SPs provide students with the opportunity to learn and practice both technical and non-technical skills in an environment capable of reproducing the realism of the doctor-patient relationship. These simulated environments are less stressful for the students, who are not required to interact with a real patient \cite{forrest2013essential,kneebone2006human}. 
However, SPs are difficult to standardize, since their performance heavily depend on the actors' skills, and their recruitment and training can become very costly \cite{Nestel2011}.

A practical alternative to SPs is represented by virtual patients (VPs), i.e., interactive computer simulations capable of portraying a patient in a clinical scenario in a realistic way. VPs are virtual agents that have a human appearance and the ability to respond to users and engage in communication patterns typical of a real conversation. They can be equipped with external sensors capable of capturing a wide range of non-verbal clues (user's gestures and motions, expressions and line of sight) and use them to modulate the evolution of the conversation. In the field of communication skill learning, VPs have the same advantages over SPs and are characterized by comparable learning outcomes \cite{quail2016student}. They are cost-effective solutions since they can be developed once and used many times. They can be deployed as in-class or self-learning tools that students can use at their own pace and at any place. They can also be integrated into sophisticated software platforms that include automatic learner assessment, feedback and debriefing sessions. Finally, VP simulations can present, with reasonable accuracy, difficult or rare cases with a high degree of repeatability \cite{urresti2017virtual} and, compared to SPs, are also easier to standardize \cite{rogers2011developing}. 

VPs for provider-patient communication have been surveyed in several works. Bearman et al. \cite{bearman2015learning} conducted a systematic review on VPs focused on developing empathy-related skills, but the Authors did not extend their analysis to communication as a whole. A more recent review \cite{richardson2019virtualreview} investigated the specific context of pharmacist-patient counselling, focusing on the development of knowledge, skills, confidence, engagement with learning, and user satisfaction. However, the work did not discuss in depth the effects of specific design choices. Another integrative review \cite{peddle2016virtual} focused on non-technical skills like situational awareness, decision-making, teamwork, leadership, and communication, but did not consider the technical perspective and, like the previous one, it did not elaborate on the impact of specific instructional or design features.
Finally, the Authors of \cite{lee2020effective} performed a systematic review on VPs focused on communication, analyzing which features of instructional design (i.e., the definition of methods, processes and strategies that guide learners to achieve the training objectives) and technical design (i.e., the definition of the technological components aimed to support and implement the envisaged instructional design) are most effective in VP simulations. Unfortunately, the time span of the survey was limited to 2006--2018, and the number of studies remaining from the application of the inclusion and exclusion criteria was rather small (14 works, with only eight discussed in detail). 

Based on the above analysis, it was our opinion that a thorough analysis of the instructional and technical design elements as well as of the technological components (sensory system, speech understanding, interaction devices, virtual reality, VR, and augmented reality, AR, etc.) and related relevant concepts (like immersion and presence) that are involved in the development of the considered learning tools was actually missing. Hence, we tried to fill this gap through the review reported in the present paper. Similarly to \cite{lee2020effective}, our study is centered on the instructional and technical design of  VP simulations. However,  we propose a different approach to the analysis of these two components, by performing in particular a deeper investigation of the technical aspects. Then, based on our findings, we also identify current limitations and potentially unexplored areas with the aim to foster further research and developments in the field.
%Furthermore, our paper tries to understand, as much as possible, the possible effects of the relevant design elements identified on the expected outcomes (i.e., communication skills transferred and learned) and to identify potentially unexplored areas for further research and development. %Furthermore, the focus of our work is also in trying to understand, as much as possible, the possible effects of the relevant design elements on the expected outcomes (i.e., communication skills learned).

The rest of the paper is organized as follows. In Section \ref{sec:reviewProtocol}, the literature review protocol is first introduced. Afterwards, Section \ref{sec:taxonomy} presents and discusses the research results.  Section \ref{sec:openResearch}  highlights the gaps that we identified  and the directions that future studies could take in order to address
them. Finally, conclusions are given in Section \ref{sec:conclusion}.

%\small
{\rowcolors{3}{mywhite}{mygray}
\begin{tabularx}{\linewidth} {X | X | X}

\rowcolor{lightgray}
\textbf{Ref}  &\textbf{AssignedName} & \textbf{CodeName}\\
%& \multicolumn{1}{|l|}{Category} & \multicolumn{1}{|l|}{Navigation} & \multicolumn{1}{|l|}{Feedback} & \multicolumn{1}{|l|}{Gamification} & \multicolumn{1}{|l|}{Hardware} & \multicolumn{1}{|l|}{Presentation} & \multicolumn{1}{|l|}{Input} & \multicolumn{1}{|l|}{Distribution} & \multicolumn{1}{|l|}{Other Tech. Features}  \\ 
%\hline 
\specialrule{.1em}{.05em}{.05em} 
\endhead

\cite{adefila2020students} & HOLLIE & HOLLIE \cite{adefila2020students} \\ 

\cite{albright2018using} & At Risk in Primary Care &
	AtRiskInPrimaryCare \cite{albright2018using} \\ 

\cite{banszki2018clinical} + \cite{quail2016student} & Jim (?) & Banszki \cite{banszki2018clinical,quail2016student}\\ 

\cite{dupuy2019virtual} &  & Dupuy \cite{dupuy2019virtual}\\ 

\cite{foster2016using} & Cynthia Young VP & CynthiaYoungVP \cite{foster2016using}\\ 

\cite{guetterman2019medical} + \cite{kron2017using} & MPathic-VR & MPathic-VR \cite{guetterman2019medical,kron2017using}\\ 

\cite{hirumi2016advancingPart2} + \cite{hirumi2016advancing} + \cite{kleinsmith2015understanding} & NERVE & NERVE \cite{hirumi2016advancingPart2,hirumi2016advancing,kleinsmith2015understanding}\\ 

\cite{jacklin2019virtual} + \cite{jacklin2018improving} &  & Jacklin \cite{jacklin2019virtual,jacklin2018improving}\\ 

\cite{jeuring2015communicate} & Communicate! & Communicate! \cite{jeuring2015communicate}\\ 

\cite{maicher2017developing} &  & Maicher \cite{maicher2017developing}\\

\cite{marei2018use} &  & Marei \cite{marei2018use}\\ 

\cite{ochs2019training} &  & Ochs \cite{ochs2019training}\\

\cite{o2019suicide} & Suicide Prevention: Assessing
Risk with Taye Banks & Suicide Prevention \cite{o2019suicide}\\ 

\cite{peddle2019exploring} + \cite{peddle2019development} & VSPR & VSPR \cite{peddle2019exploring,peddle2019development}\\ 

\cite{richardson2019virtual} &  & Richardson \cite{richardson2019virtual}\\

\cite{sapkaroski2018implementation} & CESTOL VR Clinic & CESTOLVRClinic \cite{sapkaroski2018implementation}\\ 

\cite{schoenthaler2017simulated} & Kognito Conversation Platform (?) & Schoenthaler \cite{schoenthaler2017simulated}\\ 

\cite{szilas2019virtual} &  & Szilas \cite{szilas2019virtual}\\ 

\cite{washburn2020virtual} &  & Washburn \cite{washburn2020virtual}\\ 

\cite{zielke2016beyond} + \cite{zielke2016using} & UT-Time Portal & UTTimePortal \cite{zielke2016beyond,zielke2016using}\\ 

\cite{zlotos2016scenario} &  & Zlotos \cite{zlotos2016scenario}\\ 

\hline
%\end{longtable}
\end{tabularx}
}

\normalsize

