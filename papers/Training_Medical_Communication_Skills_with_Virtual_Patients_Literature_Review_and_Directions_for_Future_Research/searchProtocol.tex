\section{Literature review protocol}
\label{sec:reviewProtocol}


As illustrated in the previous section, we performed a literature review to document the current state-of-the-art of the use of VPs for medical communication skill learning and to identify possible areas where further research is needed. The purpose of this review was to understand the instructional and technical design principles and the efficacy of these elements in achieving the expected learning outcomes.
To this end, we developed the following guiding questions in order to help focusing information extraction.

\begin{itemize}
\item RQ1: What are the latest technical developments in the field of communica-tion-oriented VPs? 

\item RQ2: Which instructional and technical design features are employed most commonly in VP design?

\item RQ3: Which instructional and technical VP design features are more effective for learning communication skills? And which of these features are most appreciated by the users?

%\item RQ4: In the field of communication-oriented VPs, what are possible open research areas that need to be further explored?

\end{itemize}

The search process, carried out mainly between March and May 2020, started with an automated approach targeting four scientific paper databases, namely Scopus, PubMed, ACM Digital Library and IEEE Xplore. For each database we performed a search based on the main and derivative keywords (virtual patient OR (serious game AND healthcare)) AND (communication), limiting the results to papers published from 2015 onward. The choice of this date was made with the aim to survey only the most recent developments in the field and avoid excessive overlaps with previous literature reviews (e.g., with \cite{peddle2016virtual} and \cite{lee2020effective}). 

The papers found were post-processed in order to remove repeated entries and exclude reviews, editorials, abstracts, posters and panel discussions. The remaining 306 papers were analyzed by reading over their title, abstract, and introduction, and classified as either relevant or irrelevant based on the following criteria: (i) does the study relate to any of the design elements of interest (instructional or technical)? and, (ii) does the study disclose at least some of the design choices made by the Authors? If the answer to any of these questions was no, then the paper was excluded. After this step, each of the 70 accepted papers was read completely by at least one reviewer, who also assessed its quality. Its references were also analyzed according to the aforementioned screening process. 

At the end of the search process, we selected a total of \totalArticles papers. Among them, we identified a number of works that referred to the same VP, but in different experimental settings or in different phases of the development process. Since our interest was in analyzing the VP design rather than the detailed outcomes of possible experiments, papers sharing the same VP were grouped together, obtaining a total of \totalVPs VPs (17 of them had not been discussed in previous surveys, and only four of them were in common with \cite{lee2020effective}, namely Banszki \cite{banszki2018clinical,quail2016student}, CynthiaYoungVP \cite{foster2016using}, MPathic-VR \cite{guetterman2019medical,kron2017using}, and NERVE \cite{hirumi2016advancingPart2,hirumi2016advancing,kleinsmith2015understanding}). %Thus, in the following, when referring a specific VP, we will also list the articles that discuss it.

In order to capture the main characteristics of problems and solutions discussed in these papers, we introduced a taxonomy of terms for the instructional and technical design elements, whose initial version was defined based on the Authors’ expertise. Based on intermediate findings, this taxonomy was further refined into the final one introduced in Section~\ref{sec:taxonomy}. All the Authors categorized the selected VPs according to this taxonomy, and any disagreement was solved by discussion.
Finally, as a last step, we searched for references related to the open problems and potential areas of research identified during the analysis. 














% \andrea{to be done [EDO]}
% \edo{draft:}

% %In this review we focus only on VPs with narrative elements, including articles form 2015 to 2020, to assess the current state of technologies in this field and their possible future developments.
% %detto qua sembra che il fatto che sia narrative sia un search criteria, che presumo dovremmo giustificare, ma non saprei bene come. Il fatto che siano tutti narrative in realtà è incidentale, perché un VP PS basato sulla comunicazione, tecnicamente, potrebbe anche esistere.

% %Preliminary search -> objective: define classification criteria
% A preliminary search that included all types of studies and reviews regarding Virtual Patients was conducted to establish the appropriate classification criteria for design elements, to later meaningfully analyze the literature. After a number of iterations, we extrapolated 7 main categories, divided into two domains: 
% %Instructional Design, (Structure, Unfolding, Feedback and Gamification) and Technical Design (Presentation Format, Input Interface and Distribution). This preparatory phase provided an analytical framework that we later used as an inclusion criterion for studies.

% %Actual search -> include only articles that could be fully categorized according to the classification criteria we established
% The actual search process was carried out mainly between March and May 2020, targeting Scopus, PubMed (IEEE Explore, ACMDigital) and XYZ scientific databases. We carried out the search based on the main and derivative keywords (virtual patient OR simulation OR virtual reality OR serious game) AND/OR healthcare AND (communication OR nontechnical skills OR (doctor-patient communication OR provider-patient communication)), limiting the results only to articles published from 2015 onward with the objective of surveying only the most recent developments in this field and trying not to overlap results with previous literature reviews \cite{peddle2016virtual}. After this step, we performed a preliminary screening on the 131 found articles based on title and abstract, eliminating all articles that were not directly related to both VPs AND provider-patient communication. The remaining X results underwent a further full-text review by at least one author, and we removed all articles that did not provide enough information to be fully classified according to the categorization criteria we established during the preparatory phase. Reviews, editorials, abstracts, posters. 
% %se diciamo così però rimane il problema di quei maledetti due che sono undisclosed per quanto riguarda la distribuzione :(... ricontrolla edo. u
% %SOLVED: Marei è standalone.
% %jeuring communicate: l'authoring tool è di sicuro web-based (trovato in un altro articolo) l'applicazione in sé è molto poco chiaro
% %shoenthaler... molto poco chiaro (costruita con la Kognito Conversation Plaform, che può essere deployata sia standalone che web-based...)

% Finally, during the full-text review, we also searched for references reported in these articles for additional literature that could offer additional cues relevant to our research objectives. 
% At the end of the search process, a total of \totalArticles were found to be compliant with our criteria however, we identified a number of studies that used the same VP, but in different experimental settings. Since our interest lies in analyzing the VP design and not the study design, articles sharing the same VP were grouped together, obtaining a total of \totalVPs VPs. 

% %\edo{NOTA:} \cite{jacklin2019virtual} \edo{è uno 'Short-Report'. E' degno di inclusione? in ogni caso il VP non lo perdiamo, perché e' uno di quelli che ha due articoli. Io al netto di tutto lo terrei. Ma comunque chiedo un parere esterno.}
% %\cite{richardson2019virtual} \edo{anche questo è uno 'short-report'}

% %Included both experimental and non-experimental studies (qualitative and quantitative)
