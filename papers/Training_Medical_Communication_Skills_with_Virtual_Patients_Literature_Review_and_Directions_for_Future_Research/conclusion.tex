\section{Conclusion}
\label{sec:conclusion}

In our research we found many different examples of VPs focused on provider-patient communication and various approaches to their design. However, we feel that there is not a single VP that realizes the full potential of this learning tool. Some research areas still need to be explored further. The broad range of educational use cases in healthcare suggests that VP applications should be as modular and adaptable as possible. Effective and user-friendly authoring tools are very rarely implemented while being, in our opinion, a crucial feature to ensure the adoption of a VP simulation by clinical educators. 
The use of technologies such as VR, AR, and advanced NLP also needs to be explored more in depth, as they may give VP simulations the edge they need to be effectively used in daily practice. We also feel that recent developments in web-based technologies will also reduce those compromises between accessibility and advanced technical possibilities that today are still required in many situations.
%The use of technologies such as VR, AR, advanced NLP software also need to be explored more, which may give VP simulations the edge they need to be fully capable being used side-by-side (or even in place of) traditional methods without concern. We also feel that recent developments in web-based technologies will also reduce those compromises between accessibility and advanced technical possibilities that today are still made in many cases.


%From the analysis of the gathered data we have solid ground to make assumptions that answer our research questions, i.e. "What are possible future developments for VP technology that are lacking in current iterations?".\par

 