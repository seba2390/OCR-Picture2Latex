\begin{abstract}

Effective communication is a crucial skill for healthcare provi-ders %non funziona hyphenation...
since it leads to better patient health, satisfaction and avoids malpractice claims. In standard medical education, students' communication skills are trained with role-playing and Standardized Patients (SPs), i.e., actors. However, SPs are difficult to standardize, and are very resource consuming. Virtual Patients (VPs) are interactive computer-based systems that represent a valuable alternative to SPs. VPs are capable of portraying patients in realistic clinical scenarios and engage learners in realistic conversations. Approaching medical communication skill training with VPs has been an active research area in the last ten years. As a result, the number of works in this field has grown significantly. The objective of this work is to survey the recent literature, assessing the state of the art of this technology with a specific focus on the instructional and technical design of VP simulations. After having classified and analysed the VPs selected for our research, we identified several areas that require further investigation, and we drafted practical recommendations for VP developers on design aspects that, based on our findings, are pivotal to create novel and effective VP simulations or improve existing ones.


% %\textbf{Introduction} 
% \andrea{riscrivere}
% Effective communication is a crucial skill for healthcare providers since it leads to better patient health, satisfaction and avoids malpractice claims. In medical education, students' communication skills are trained with role-play, Standardized Patients (actors) and Virtual Patients (VPs). The objective of this article is to survey all the articles that can be found in recent literature (2015-2020) focused on VPs for health care providers' communication skills, assess the state of the art of this technology with a specific focus on instructional and technical design, and then to identify potentially unexplored areas for further research and development. 
% %\textbf{Taxonomy Results} 
% We found \totalArticles articles detailing studies, validations and proposals for VP simulations focused on provider-patient communication; then, we defined a set of recurring characteristics and organized them in a taxonomy to properly categorize these simulations. 
% %\textbf{Open Research & Conclusion} 
% While there are many different and compelling approaches to VPs, we feel that there is not a single virtual patient that realizes the full potential of this learning tool. After classification and analysis of all the VPs described in the articles gathered, we found several areas that need further research, and advance practical recommendations for future VP developers on design aspects that, based on our findings, are pivotal to create effective VP simulations, or improve existing ones.

\keywords{Virtual Patient \and Embodied Conversational Agent \and Provider-Patient Communication \and Instructional Design \and Technical Design}
\end{abstract}

