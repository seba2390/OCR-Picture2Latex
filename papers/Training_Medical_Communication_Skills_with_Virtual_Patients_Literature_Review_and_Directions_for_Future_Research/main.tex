% This is samplepaper.tex, a sample chapter demonstrating the
% LLNCS macro package for Springer Computer Science proceedings;
% Version 2.20 of 2017/10/04
%
\documentclass[runningheads]{llncs}
%
\usepackage{graphicx}
\usepackage[figuresright]{rotating}
\usepackage[table]{xcolor}
\definecolor{mywhite}{RGB}{255, 255, 255}
\definecolor{mygray}{RGB}{240, 240, 240}
\usepackage{tabularx}
\usepackage{supertabular}
\usepackage{longtable}
\usepackage{lscape}
\usepackage{ltablex}
\usepackage{ctable}
\usepackage{multirow}

\usepackage{hyperref}
\hypersetup{
    colorlinks=false,
    linkcolor=blue,
    filecolor=magenta,      
    urlcolor=cyan,
}

%setup per caption delle tabelle sopra alle tabelle stesse
\usepackage{floatrow}
\floatsetup[table]{capposition = top}

\urlstyle{same}


% Used for displaying a sample figure. If possible, figure files should
% be included in EPS format.
%
% If you use the hyperref package, please uncomment the following line
% to display URLs in blue roman font according to Springer's eBook style:
% \renewcommand\UrlFont{\color{blue}\rmfamily}

%%% extra latex commands
\usepackage{xcolor} 
\usepackage{soul} 
\usepackage{xspace}
\usepackage{amsmath}
\usepackage{multirow}
\usepackage{capt-of}
\usepackage{floatrow}
% Table float box with bottom caption, box width adjusted to content
%\newfloatcommand{capbtabbox}{table}[][\FBwidth]

\usepackage{hyphenat}
\hyphenation{re-pro-du-ci-bi-li-ty}
\hyphenation{com-pu-ter-ba-sed}

\newcommand{\totalArticles}{28\xspace} 

\newcommand{\totalVPs}{21\xspace} 

\newcommand{\quotes}[1]{``#1''} 

\DeclareRobustCommand{\edo}[1]
{{\sethlcolor{cyan}\hl{#1}}} 

\DeclareRobustCommand{\andrea}[1]
{{\sethlcolor{orange}\hl{#1}}} 

\DeclareRobustCommand{\francesco}[1]
{{\sethlcolor{green}\hl{#1}}} 

\begin{document}
%
\title{Training Medical Communication Skills with Virtual Patients: Literature Review and Directions for Future Research}
%
%
\titlerunning{Training Medical Communication Skills with VPs}
% If the paper title is too long for the running head, you can set
% an abbreviated paper title here
%
%Comment names for double blind rev

\author{Edoardo Battegazzorre\orcidID{0000-0002-1383-3285} \and
Andrea Bottino\orcidID{0000-0002-8894-5089} \and
Fabrizio Lamberti\orcidID{0000-0001-7703-1372}}

%\author{Authors omitted for double blind review}
%
%\authorrunning{F. Author et al.}
% First names are abbreviated in the running head.
% If there are more than two authors, 'et al.' is used.
%

%Comment institution for double blind
\institute{DAUIN, Politecnico di Torino, Corso Duca degli Abruzzi 24, 10143 Torino, Italy %\\
\email{\{edoardo.battegazzorre, andrea.bottino, fabrizio.lamberti\}@polito.it}}
%\institute{Institute omitted for double blind review\\
%\email{emails omitted for double blind review}}

%
\maketitle              % typeset the header of the contribution
%
  In this paper, we explore the connection between secret key agreement and secure omniscience within the setting of the multiterminal source model with a wiretapper who has side information. While the secret key agreement problem considers the generation of a maximum-rate secret key through public discussion, the secure omniscience problem is concerned with communication protocols for omniscience that minimize the rate of information leakage to the wiretapper. The starting point of our work is a lower bound on the minimum leakage rate for omniscience, $\rl$, in terms of the wiretap secret key capacity, $\wskc$. Our interest is in identifying broad classes of sources for which this lower bound is met with equality, in which case we say that there is a duality between secure omniscience and secret key agreement. We show that this duality holds in the case of certain finite linear source (FLS) models, such as two-terminal FLS models and pairwise independent network models on trees with a linear wiretapper. Duality also holds for any FLS model in which $\wskc$ is achieved by a perfect linear secret key agreement scheme. We conjecture that the duality in fact holds unconditionally for any FLS model. On the negative side, we give an example of a (non-FLS) source model for which duality does not hold if we limit ourselves to communication-for-omniscience protocols with at most two (interactive) communications.  We also address the secure function computation problem and explore the connection between the minimum leakage rate for computing a function and the wiretap secret key capacity.
  
%   Finally, we demonstrate the usefulness of our lower bound on $\rl$ by using it to derive equivalent conditions for the positivity of $\wskc$ in the multiterminal model. This extends a recent result of Gohari, G\"{u}nl\"{u} and Kramer (2020) obtained for the two-user setting.
  
   
%   In this paper, we study the problem of secret key generation through an omniscience achieving communication that minimizes the 
%   leakage rate to a wiretapper who has side information in the setting of multiterminal source model.  We explore this problem by deriving a lower bound on the wiretap secret key capacity $\wskc$ in terms of the minimum leakage rate for omniscience, $\rl$. 
%   %The former quantity is defined to be the maximum secret key rate achievable, and the latter one is defined as the minimum possible leakage rate about the source through an omniscience scheme to a wiretapper. 
%   The main focus of our work is the characterization of the sources for which the lower bound holds with equality \textemdash it is referred to as a duality between secure omniscience and wiretap secret key agreement. For general source models, we show that duality need not hold if we limit to the communication protocols with at most two (interactive) communications. In the case when there is no restriction on the number of communications, whether the duality holds or not is still unknown. However, we resolve this question affirmatively for two-user finite linear sources (FLS) and pairwise independent networks (PIN) defined on trees, a subclass of FLS. Moreover, for these sources, we give a single-letter expression for $\wskc$. Furthermore, in the direction of proving the conjecture that duality holds for all FLS, we show that if $\wskc$ is achieved by a \emph{perfect} secret key agreement scheme for FLS then the duality must hold. All these results mount up the evidence in favor of the conjecture on FLS. Moreover, we demonstrate the usefulness of our lower bound on $\wskc$ in terms of $\rl$ by deriving some equivalent conditions on the positivity of secret key capacity for multiterminal source model. Our result indeed extends the work of Gohari, G\"{u}nl\"{u} and Kramer in two-user case.
%
%
%

% \leavevmode
% \\
% \\
% \\
% \\
% \\
\section{Introduction}
\label{introduction}

AutoML is the process by which machine learning models are built automatically for a new dataset. Given a dataset, AutoML systems perform a search over valid data transformations and learners, along with hyper-parameter optimization for each learner~\cite{VolcanoML}. Choosing the transformations and learners over which to search is our focus.
A significant number of systems mine from prior runs of pipelines over a set of datasets to choose transformers and learners that are effective with different types of datasets (e.g. \cite{NEURIPS2018_b59a51a3}, \cite{10.14778/3415478.3415542}, \cite{autosklearn}). Thus, they build a database by actually running different pipelines with a diverse set of datasets to estimate the accuracy of potential pipelines. Hence, they can be used to effectively reduce the search space. A new dataset, based on a set of features (meta-features) is then matched to this database to find the most plausible candidates for both learner selection and hyper-parameter tuning. This process of choosing starting points in the search space is called meta-learning for the cold start problem.  

Other meta-learning approaches include mining existing data science code and their associated datasets to learn from human expertise. The AL~\cite{al} system mined existing Kaggle notebooks using dynamic analysis, i.e., actually running the scripts, and showed that such a system has promise.  However, this meta-learning approach does not scale because it is onerous to execute a large number of pipeline scripts on datasets, preprocessing datasets is never trivial, and older scripts cease to run at all as software evolves. It is not surprising that AL therefore performed dynamic analysis on just nine datasets.

Our system, {\sysname}, provides a scalable meta-learning approach to leverage human expertise, using static analysis to mine pipelines from large repositories of scripts. Static analysis has the advantage of scaling to thousands or millions of scripts \cite{graph4code} easily, but lacks the performance data gathered by dynamic analysis. The {\sysname} meta-learning approach guides the learning process by a scalable dataset similarity search, based on dataset embeddings, to find the most similar datasets and the semantics of ML pipelines applied on them.  Many existing systems, such as Auto-Sklearn \cite{autosklearn} and AL \cite{al}, compute a set of meta-features for each dataset. We developed a deep neural network model to generate embeddings at the granularity of a dataset, e.g., a table or CSV file, to capture similarity at the level of an entire dataset rather than relying on a set of meta-features.
 
Because we use static analysis to capture the semantics of the meta-learning process, we have no mechanism to choose the \textbf{best} pipeline from many seen pipelines, unlike the dynamic execution case where one can rely on runtime to choose the best performing pipeline.  Observing that pipelines are basically workflow graphs, we use graph generator neural models to succinctly capture the statically-observed pipelines for a single dataset. In {\sysname}, we formulate learner selection as a graph generation problem to predict optimized pipelines based on pipelines seen in actual notebooks.

%. This formulation enables {\sysname} for effective pruning of the AutoML search space to predict optimized pipelines based on pipelines seen in actual notebooks.}
%We note that increasingly, state-of-the-art performance in AutoML systems is being generated by more complex pipelines such as Directed Acyclic Graphs (DAGs) \cite{piper} rather than the linear pipelines used in earlier systems.  
 
{\sysname} does learner and transformation selection, and hence is a component of an AutoML systems. To evaluate this component, we integrated it into two existing AutoML systems, FLAML \cite{flaml} and Auto-Sklearn \cite{autosklearn}.  
% We evaluate each system with and without {\sysname}.  
We chose FLAML because it does not yet have any meta-learning component for the cold start problem and instead allows user selection of learners and transformers. The authors of FLAML explicitly pointed to the fact that FLAML might benefit from a meta-learning component and pointed to it as a possibility for future work. For FLAML, if mining historical pipelines provides an advantage, we should improve its performance. We also picked Auto-Sklearn as it does have a learner selection component based on meta-features, as described earlier~\cite{autosklearn2}. For Auto-Sklearn, we should at least match performance if our static mining of pipelines can match their extensive database. For context, we also compared {\sysname} with the recent VolcanoML~\cite{VolcanoML}, which provides an efficient decomposition and execution strategy for the AutoML search space. In contrast, {\sysname} prunes the search space using our meta-learning model to perform hyperparameter optimization only for the most promising candidates. 

The contributions of this paper are the following:
\begin{itemize}
    \item Section ~\ref{sec:mining} defines a scalable meta-learning approach based on representation learning of mined ML pipeline semantics and datasets for over 100 datasets and ~11K Python scripts.  
    \newline
    \item Sections~\ref{sec:kgpipGen} formulates AutoML pipeline generation as a graph generation problem. {\sysname} predicts efficiently an optimized ML pipeline for an unseen dataset based on our meta-learning model.  To the best of our knowledge, {\sysname} is the first approach to formulate  AutoML pipeline generation in such a way.
    \newline
    \item Section~\ref{sec:eval} presents a comprehensive evaluation using a large collection of 121 datasets from major AutoML benchmarks and Kaggle. Our experimental results show that {\sysname} outperforms all existing AutoML systems and achieves state-of-the-art results on the majority of these datasets. {\sysname} significantly improves the performance of both FLAML and Auto-Sklearn in classification and regression tasks. We also outperformed AL in 75 out of 77 datasets and VolcanoML in 75  out of 121 datasets, including 44 datasets used only by VolcanoML~\cite{VolcanoML}.  On average, {\sysname} achieves scores that are statistically better than the means of all other systems. 
\end{itemize}


%This approach does not need to apply cleaning or transformation methods to handle different variances among datasets. Moreover, we do not need to deal with complex analysis, such as dynamic code analysis. Thus, our approach proved to be scalable, as discussed in Sections~\ref{sec:mining}.
\section{Literature review protocol}
\label{sec:reviewProtocol}


As illustrated in the previous section, we performed a literature review to document the current state-of-the-art of the use of VPs for medical communication skill learning and to identify possible areas where further research is needed. The purpose of this review was to understand the instructional and technical design principles and the efficacy of these elements in achieving the expected learning outcomes.
To this end, we developed the following guiding questions in order to help focusing information extraction.

\begin{itemize}
\item RQ1: What are the latest technical developments in the field of communica-tion-oriented VPs? 

\item RQ2: Which instructional and technical design features are employed most commonly in VP design?

\item RQ3: Which instructional and technical VP design features are more effective for learning communication skills? And which of these features are most appreciated by the users?

%\item RQ4: In the field of communication-oriented VPs, what are possible open research areas that need to be further explored?

\end{itemize}

The search process, carried out mainly between March and May 2020, started with an automated approach targeting four scientific paper databases, namely Scopus, PubMed, ACM Digital Library and IEEE Xplore. For each database we performed a search based on the main and derivative keywords (virtual patient OR (serious game AND healthcare)) AND (communication), limiting the results to papers published from 2015 onward. The choice of this date was made with the aim to survey only the most recent developments in the field and avoid excessive overlaps with previous literature reviews (e.g., with \cite{peddle2016virtual} and \cite{lee2020effective}). 

The papers found were post-processed in order to remove repeated entries and exclude reviews, editorials, abstracts, posters and panel discussions. The remaining 306 papers were analyzed by reading over their title, abstract, and introduction, and classified as either relevant or irrelevant based on the following criteria: (i) does the study relate to any of the design elements of interest (instructional or technical)? and, (ii) does the study disclose at least some of the design choices made by the Authors? If the answer to any of these questions was no, then the paper was excluded. After this step, each of the 70 accepted papers was read completely by at least one reviewer, who also assessed its quality. Its references were also analyzed according to the aforementioned screening process. 

At the end of the search process, we selected a total of \totalArticles papers. Among them, we identified a number of works that referred to the same VP, but in different experimental settings or in different phases of the development process. Since our interest was in analyzing the VP design rather than the detailed outcomes of possible experiments, papers sharing the same VP were grouped together, obtaining a total of \totalVPs VPs (17 of them had not been discussed in previous surveys, and only four of them were in common with \cite{lee2020effective}, namely Banszki \cite{banszki2018clinical,quail2016student}, CynthiaYoungVP \cite{foster2016using}, MPathic-VR \cite{guetterman2019medical,kron2017using}, and NERVE \cite{hirumi2016advancingPart2,hirumi2016advancing,kleinsmith2015understanding}). %Thus, in the following, when referring a specific VP, we will also list the articles that discuss it.

In order to capture the main characteristics of problems and solutions discussed in these papers, we introduced a taxonomy of terms for the instructional and technical design elements, whose initial version was defined based on the Authors’ expertise. Based on intermediate findings, this taxonomy was further refined into the final one introduced in Section~\ref{sec:taxonomy}. All the Authors categorized the selected VPs according to this taxonomy, and any disagreement was solved by discussion.
Finally, as a last step, we searched for references related to the open problems and potential areas of research identified during the analysis. 














% \andrea{to be done [EDO]}
% \edo{draft:}

% %In this review we focus only on VPs with narrative elements, including articles form 2015 to 2020, to assess the current state of technologies in this field and their possible future developments.
% %detto qua sembra che il fatto che sia narrative sia un search criteria, che presumo dovremmo giustificare, ma non saprei bene come. Il fatto che siano tutti narrative in realtà è incidentale, perché un VP PS basato sulla comunicazione, tecnicamente, potrebbe anche esistere.

% %Preliminary search -> objective: define classification criteria
% A preliminary search that included all types of studies and reviews regarding Virtual Patients was conducted to establish the appropriate classification criteria for design elements, to later meaningfully analyze the literature. After a number of iterations, we extrapolated 7 main categories, divided into two domains: 
% %Instructional Design, (Structure, Unfolding, Feedback and Gamification) and Technical Design (Presentation Format, Input Interface and Distribution). This preparatory phase provided an analytical framework that we later used as an inclusion criterion for studies.

% %Actual search -> include only articles that could be fully categorized according to the classification criteria we established
% The actual search process was carried out mainly between March and May 2020, targeting Scopus, PubMed (IEEE Explore, ACMDigital) and XYZ scientific databases. We carried out the search based on the main and derivative keywords (virtual patient OR simulation OR virtual reality OR serious game) AND/OR healthcare AND (communication OR nontechnical skills OR (doctor-patient communication OR provider-patient communication)), limiting the results only to articles published from 2015 onward with the objective of surveying only the most recent developments in this field and trying not to overlap results with previous literature reviews \cite{peddle2016virtual}. After this step, we performed a preliminary screening on the 131 found articles based on title and abstract, eliminating all articles that were not directly related to both VPs AND provider-patient communication. The remaining X results underwent a further full-text review by at least one author, and we removed all articles that did not provide enough information to be fully classified according to the categorization criteria we established during the preparatory phase. Reviews, editorials, abstracts, posters. 
% %se diciamo così però rimane il problema di quei maledetti due che sono undisclosed per quanto riguarda la distribuzione :(... ricontrolla edo. u
% %SOLVED: Marei è standalone.
% %jeuring communicate: l'authoring tool è di sicuro web-based (trovato in un altro articolo) l'applicazione in sé è molto poco chiaro
% %shoenthaler... molto poco chiaro (costruita con la Kognito Conversation Plaform, che può essere deployata sia standalone che web-based...)

% Finally, during the full-text review, we also searched for references reported in these articles for additional literature that could offer additional cues relevant to our research objectives. 
% At the end of the search process, a total of \totalArticles were found to be compliant with our criteria however, we identified a number of studies that used the same VP, but in different experimental settings. Since our interest lies in analyzing the VP design and not the study design, articles sharing the same VP were grouped together, obtaining a total of \totalVPs VPs. 

% %\edo{NOTA:} \cite{jacklin2019virtual} \edo{è uno 'Short-Report'. E' degno di inclusione? in ogni caso il VP non lo perdiamo, perché e' uno di quelli che ha due articoli. Io al netto di tutto lo terrei. Ma comunque chiedo un parere esterno.}
% %\cite{richardson2019virtual} \edo{anche questo è uno 'short-report'}

% %Included both experimental and non-experimental studies (qualitative and quantitative)

\section{Results and discussion}
\label{sec:taxonomy}


In this section, we present the result of our research. As stated in Section \ref{sec:reviewProtocol}, the selected VPs have been labeled according to the taxonomy of terms summarized in Tables \ref{table:instructionalDesignTable} and \ref{table:technicalDesignTable}. The definition of the identified categories (which differs to a large extent from the one presented in \cite{lee2020effective}) is introduced in the following subsections, where we also discuss the survey results relative to each group. We first introduce the instructional design elements, which are connected to the technical elements necessary to realize them; afterwards, we discuss the design choices related to the technical and technological components of the simulations. Finally, we discuss the experimental evidences related to the effectiveness of the identified design elements.



% Approaching health care provider-patient communication training through computer-based learning methods has been an active research area during the last ten years. As a result, the number of works in this area has grown significantly \cite{lee2020effective,peddle2016virtual,richardson2019virtualreview}. 

% As stated in Section \ref{sec:reviewProtocol}, during the process of reading articles and collecting data, we identified several recurring characteristics and design elements that we subsequently organized in a taxonomy of terms (summarized in Tables \ref{table:instructionalDesignTable} and \ref{table:technicalDesignTable}). The definition of the identified categories (which differs to a large extent from the one presented in \cite{lee2020effective}) is introduced in the following subsections, where we also discuss the survey results relative to each group. We first introduce the instructional design elements, which are connected to the technical elements necessary to realize them, and then we discuss the design choices related to the technical and technological component of the simulations. Finally, we discuss the experimental evidences related to the effectiveness of the identified design elements. 




% %COMPREHENSIVE TABLE WITH EVERYTHING

\begin{landscape}
%\begin{longtable}{ | p{1cm} | *{15}{l} |}
%\begin{tabularx}{\linewidth} {>{\raggedleft\arraybackslash}X >{\raggedleft\arraybackslash}X >{\raggedleft\arraybackslash}X >{\raggedleft\arraybackslash}X >{\raggedleft\arraybackslash}X >{\raggedleft\arraybackslash}X >{\raggedleft\arraybackslash}X >{\raggedleft\arraybackslash}X >{\raggedleft\arraybackslash}X >{\raggedleft\arraybackslash}X}
\tiny
{\rowcolors{3}{mywhite}{mygray}
%\begin{tabularx}{\linewidth} {X | X | X | X | X | X | X | X | X | X}
\begin{tabularx}{\linewidth}
{|>{\hsize=.5\hsize\linewidth=\hsize}X |
>{\hsize=.75\hsize\linewidth=\hsize}X |
>{\hsize=.75\hsize\linewidth=\hsize}X |
>{\hsize=1.5\hsize\linewidth=\hsize}X |
>{\hsize=1\hsize\linewidth=\hsize}X |
>{\hsize=1.5\hsize\linewidth=\hsize}X |
>{\hsize=.75\hsize\linewidth=\hsize}X |
>{\hsize=.5\hsize\linewidth=\hsize}X |
>{\hsize=.75\hsize\linewidth=\hsize}X |
>{\hsize=2\hsize\linewidth=\hsize}X |}


%\textbf{Study} & \textbf{Geographical Location and Level} & \textbf{Identification} & 
%\textbf{Multiplier Result} \\ \hline \hline
%\endfirsthead
\rowcolor{lightgray}
\textbf{Article}  & \multicolumn{4}{|l|}{\textbf{Instructional Design}} & \multicolumn{5}{|l|}{\textbf{Technical Design}} \\
\rowcolor{lightgray}
& Category & Navigation & Feedback & Gamification & Hardware & Presentation & Input Interface & Distribution & Other Tech. Features \\
%& \multicolumn{1}{|l|}{Category} & \multicolumn{1}{|l|}{Navigation} & \multicolumn{1}{|l|}{Feedback} & \multicolumn{1}{|l|}{Gamification} & \multicolumn{1}{|l|}{Hardware} & \multicolumn{1}{|l|}{Presentation} & \multicolumn{1}{|l|}{Input} & \multicolumn{1}{|l|}{Distribution} & \multicolumn{1}{|l|}{Other Tech. Features}  \\ 
%\hline 
\specialrule{.1em}{.05em}{.05em} 
\endhead

\cite{adefila2020students} & Narrative, PS & \textbf{\emph{Unclear}} &	Timed events and health status &	Tamagotchi style serious game & Any device with a web browser	& Graphic (Image) & Typed &	Web-Based & /\\ 

\cite{albright2018using} & Narrative &	Closed-Option &
	Trust meter and Virtual Instructor discussing the user’s choices & / &	Any device with a web browser & Graphic (Image) & Typed	& Web-Based & /\\ 

\cite{banszki2018clinical} + \cite{quail2016student} & Narrative	& Open-Option, Non-Verbal &	/ &	/ &	PC, 64” monitor, Microphone & Graphic (3D) & Voice-Controlled & Standalone &	Human-Controlled,Gesture and Facial Expression Output \\ 

\cite{dupuy2019virtual} & Narrative &	Closed-Option, Non-Verbal &	Errors, score, time at the end. Possibility to replay certain parts. &	Score, Time	& PC, Vertical 40” screen, Microphone, Camera & Graphic (3D) & Voice-Controlled & Standalone &	Facial Expression Detection through face tracking\\ 

\cite{foster2016using} & Narrative & Open-Option & Empathy feedback available at the end of the simulation (human-generated) & / &	Any device with a web browser & Text-Based, Graphic (Video) & Typed & Web-Based & Human-Controlled Empathy Feedback\\ 

\cite{guetterman2019medical} + \cite{kron2017using} & Narrative &	Closed-Option, Non-Verbal &	Feedback at the end giving evidence on each choice of words and a score &	Score &	PC, Microsoft Kinect, Microphone & Graphic (3D) & Voice-Controlled &	Standalone &	Facial Expression and Body Posture Detection and output, Recorded Voiceover, Multiple VHs\\ 

\cite{hirumi2016advancingPart2} + \cite{hirumi2016advancing} + \cite{kleinsmith2015understanding} & Narrative, PS &	Closed OR Open-Option & Available topics, discoveries made	& NO &	Any device with a web browser	& Graphic (3D) & Typed & Web-Based & /\\ 

\cite{jacklin2019virtual} + \cite{jacklin2018improving} & Narrative & Closed-Option &	Unspecified Feedback at the end of the simulation &	/ &	Any device with a web browser & Graphic (3D) & Typed &	Web-Based & Body Posture Output, Recorded Voiceover	\\ 

\cite{jeuring2015communicate} & Narrative & Closed-Option & Scores, recap of choices made & Envisioned as a Serious Game, Scores for each learning goal	& \textbf{\emph{Unclear}} & Graphic (3D) & Typed & \textbf{\emph{Unclear}} & Scenario Editor\\ 

\cite{maicher2017developing} & Narrative, PS &	Open-Option, Non-Verbal & /	& / & PC, Microsoft Kinect, Multi-Array Microphone & Graphic (3D) & Typed + Voice-Controlled & Web-Based + Standalone & Motion-Captured Animations, Movement and Posture Detection and Output\\

\cite{marei2018use} & Narrative	& Closed-Option &	/ &	/ &	PC & Graphic (Image) &	Typed & \textbf{\emph{Unclear}} & /\\ 

\cite{ochs2019training} & Narrative	& Hybrid & / & / & PC, HMD (Oculus Rift), CAVE, High-End Microphone & Graphic (3D, IVR) & Voice-Controlled &  Standalone & Same VP deployed in desktop, VR and Cave versions. Speech recognition is Human-Controlled, Non-Verbal output (facial expression and body posture). Text-To-Speech, Lip Synch, Virtual Playback\\

\cite{o2019suicide} & Narrative, PS &	Closed-Option &	Immediate and after-action guidance, via Virtual Instructor & / & Any device with a web browser & Graphic (Video) & 	Typed & Web-Based & /\\ 

\cite{peddle2019exploring} + \cite{peddle2019development} & Narrative, PS & Closed-Option	& Possibility to replay certain parts & / & Any device with a web browser & Graphic (Video) & Typed & Web-Based & /\\ 

\cite{richardson2019virtual} & Narrative, PS &	Closed-Option &	Personalized feedback at the end to enhance counselling ability	& / &	Any device with a web browser & Graphic (3D) &	Typed &	Web-Based & /\\

\cite{sapkaroski2018implementation} & Narrative, PS & Closed-Option & / & /	& PC, HMD (Various), Leap Motion, Oculus Touch & Graphic (3D, IVR) & Typed + Voice-Controlled + NUI & Standalone & Remote progress tracking for educators\\ 

\cite{schoenthaler2017simulated} & Narrative & Closed-Option & Virtual Instructor feedback during (discussing each choice) and after the simulation, Trust Meter & Scores for each learning goal after the simulation & \textbf{\emph{Unclear}} &  Graphic (3D) & Typed & 	\textbf{\emph{Unclear}} &	Can play as both provider and patient, Non-Verbal Output\\ 

\cite{szilas2019virtual} & Narrative &	Closed-Option &	/ &	Envisioned as a serious game &	PC	& Graphic (3D) &	Typed & Standalone & /\\ 

\cite{washburn2020virtual} & Narrative, PS & Open-Option & / &	/ &	PC or Laptop, Large Screen & Graphic (3D) & Hybrid & Standalone & Human-Transcribed Voice Controls\\ 

\cite{zielke2016beyond} + \cite{zielke2016using} & Narrative, PS &	Closed-Option &	Points and badges assigned during and after the interview & Envisioned as a Serious Game, Scores, Badges	& Any device with a web browser & Graphic (3D) & Typed & 	Web-Based & Multiple VHs, Non-Verbal Output\\ 

\cite{zlotos2016scenario} & Narrative, PS & Closed-Option	& All possible outcomes are played after the simulation is over & / & Any device with a web browser & Graphic (3D) & Typed  & Web-Based & Motion-Captured Animations, Recorded Voiceover\\ 

\hline
%\end{longtable}
\end{tabularx}
}

\end{landscape}
\normalsize
% %TABLE WITH ONLY INSTRUCTIONAL DESIGN

%\begin{longtable}{ | p{1cm} | *{15}{l} |}
%\begin{tabularx}{\linewidth} {>{\raggedleft\arraybackslash}X >{\raggedleft\arraybackslash}X >{\raggedleft\arraybackslash}X >{\raggedleft\arraybackslash}X >{\raggedleft\arraybackslash}X >{\raggedleft\arraybackslash}X >{\raggedleft\arraybackslash}X >{\raggedleft\arraybackslash}X >{\raggedleft\arraybackslash}X >{\raggedleft\arraybackslash}X}

\scriptsize 
{\rowcolors{3}{mywhite}{mygray}
%\begin{tabularx}{\linewidth} {X | X | X | X | X | X | X | X | X | X}
\begin{tabularx}{\linewidth}
{|>{\hsize=.5\hsize\linewidth=\hsize}X |
>{\hsize=.75\hsize\linewidth=\hsize}X |
>{\hsize=.75\hsize\linewidth=\hsize}X |
>{\hsize=1.5\hsize\linewidth=\hsize}X |
>{\hsize=1.5\hsize\linewidth=\hsize}X |}


%\textbf{Study} & \textbf{Geographical Location and Level} & \textbf{Identification} & 
%\textbf{Multiplier Result} \\ \hline \hline
%\endfirsthead
\rowcolor{lightgray}
\textbf{Article}  & \multicolumn{4}{|l|}{\textbf{Instructional Design}}\\
\rowcolor{lightgray}
& Category & Navigation & Feedback & Gamification\\
%& \multicolumn{1}{|l|}{Category} & \multicolumn{1}{|l|}{Navigation} & \multicolumn{1}{|l|}{Feedback} & \multicolumn{1}{|l|}{Gamification} & \multicolumn{1}{|l|}{Hardware} & \multicolumn{1}{|l|}{Presentation} & \multicolumn{1}{|l|}{Input} & \multicolumn{1}{|l|}{Distribution} & \multicolumn{1}{|l|}{Other Tech. Features}  \\ 
%\hline 
\specialrule{.1em}{.05em}{.05em} 
\endhead

\cite{adefila2020students} & Narrative, PS & \textbf{\emph{Unclear}} &	Timed events and health status &	Tamagotchi style serious game\\ 

\cite{albright2018using} & Narrative &	Closed-Option &
	Trust meter and Virtual Instructor discussing the user’s choices & /\\ 

\cite{banszki2018clinical} + \cite{quail2016student} & Narrative & Open-Option, Non-Verbal & / & /\\ 

\cite{dupuy2019virtual} & Narrative &	Closed-Option, Non-Verbal &	Errors, score, time at the end. Possibility to replay certain parts. &	Score, Time\\ 

\cite{foster2016using} & Narrative & Open-Option & Empathy feedback available at the end of the simulation (human-generated) & /\\ 

\cite{guetterman2019medical} + \cite{kron2017using} & Narrative &	Closed-Option, Non-Verbal &	Feedback at the end giving evidence on each choice of words and a score &	Score\\ 

\cite{hirumi2016advancingPart2} + \cite{hirumi2016advancing} + \cite{kleinsmith2015understanding} & Narrative, PS &	Closed OR Open-Option & Available topics, discoveries made	& NO\\ 

\cite{jacklin2019virtual} + \cite{jacklin2018improving} & Narrative & Closed-Option &	Unspecified Feedback at the end of the simulation &	/\\ 

\cite{jeuring2015communicate} & Narrative & Closed-Option & Scores, recap of choices made & Envisioned as a Serious Game, Scores for each learning goal\\ 

\cite{maicher2017developing} & Narrative, PS &	Open-Option, Non-Verbal & /	& /\\

\cite{marei2018use} & Narrative	& Closed-Option &	/ &	/\\ 

\cite{ochs2019training} & Narrative	& Hybrid & / & /\\

\cite{o2019suicide} & Narrative, PS &	Closed-Option &	Immediate and after-action guidance, via Virtual Instructor & /\\ 

\cite{peddle2019exploring} + \cite{peddle2019development} & Narrative, PS & Closed-Option	& Possibility to replay certain parts & /\\ 

\cite{richardson2019virtual} & Narrative, PS &	Closed-Option &	Personalized feedback at the end to enhance counselling ability	& /\\

\cite{sapkaroski2018implementation} & Narrative, PS & Closed-Option & / & /\\ 

\cite{schoenthaler2017simulated} & Narrative & Closed-Option & Virtual Instructor feedback during (discussing each choice) and after the simulation, Trust Meter & Scores for each learning goal after the simulation\\ 

\cite{szilas2019virtual} & Narrative &	Closed-Option &	/ &	Envisioned as a serious game\\ 

\cite{washburn2020virtual} & Narrative, PS & Open-Option & / &	/\\ 

\cite{zielke2016beyond} + \cite{zielke2016using} & Narrative, PS &	Closed-Option &	Points and badges assigned during and after the interview & Envisioned as a Serious Game, Scores, Badges\\ 

\cite{zlotos2016scenario} & Narrative, PS & Closed-Option	& All possible outcomes are played after the simulation is over & /\\ 

\hline
%\end{longtable}
\end{tabularx}
}

\normalsize


\subsection{Instructional design}
\label{sec:instructionalDesign}

This category encompasses various instructional design aspects implemented in the VP scenario, such as how the VP delivers (and facilitates) learning activities and if (and how) it provides scaffolded support to improve learner's performance.


%Instructional Table: NEW VERSION
%ditched tabularx, use only normal tabular

\begin{table} [t]
\scriptsize{
\begin{center}
    \caption{Synopsis of the reviewed VPs for each instructional design category}
    \label{table:instructionalDesignTable}
    \begin{tabular}{| p{1.8cm} | p{2cm} | p{8cm} |}
    \hline
        \rowcolor{mygray}
        \multicolumn{3}{|c|}{\textbf{Instructional design}}\\
    \hline
        \rowcolor{lightgray}
        \textbf{Category}  & \textbf{Subcategory} & \textbf{Virtual Patients}\\
    \hline
    %STRUCTURE
         \multirow{2}{*}{Structure} & \emph{Narrative} & HOLLIE \cite{adefila2020students}, AtRiskInPrimaryCare \cite{albright2018using}, Dupuy \cite{dupuy2019virtual}, CynthiaYoungVP \cite{foster2016using}, Jacklin \cite{jacklin2019virtual,jacklin2018improving}, MPathic-VR \cite{guetterman2019medical,kron2017using}, Communicate! \cite{jeuring2015communicate}, Marei \cite{marei2018use},  Ochs \cite{ochs2019training}, Szilas \cite{szilas2019virtual} \\
    \cline{2-3}
        & \emph{Narrative + Problem solving} & Banszki \cite{banszki2018clinical,quail2016student}, NERVE
        \cite{hirumi2016advancingPart2,hirumi2016advancing,kleinsmith2015understanding},  Maicher \cite{maicher2017developing}, Suicide Prevention \cite{o2019suicide}, VSPR \cite{peddle2019exploring,peddle2019development}, Richardson \cite{richardson2019virtual}, CESTOLVRClinic \cite{sapkaroski2018implementation}, Schoenthaler \cite{schoenthaler2017simulated},   Washburn \cite{washburn2020virtual},  UTTimePortal \cite{zielke2016beyond,zielke2016using}, Zlotos \cite{zlotos2016scenario}\\
    \hline
    %UNFOLDING
         \multirow{3}{*}{Unfolding} & \emph{Closed-option} & HOLLIE \cite{adefila2020students}, AtRiskInPrimaryCare \cite{albright2018using}, Dupuy \cite{dupuy2019virtual}, MPathic-VR \cite{guetterman2019medical,kron2017using}, NERVE
        \cite{hirumi2016advancingPart2,hirumi2016advancing,kleinsmith2015understanding}, Jacklin \cite{jacklin2019virtual,jacklin2018improving}, Communicate! \cite{jeuring2015communicate}, Marei \cite{marei2018use}, Suicide Prevention \cite{o2019suicide}, VSPR \cite{peddle2019exploring,peddle2019development}, Richardson \cite{richardson2019virtual}, CESTOLVRClinic \cite{sapkaroski2018implementation}, Schoenthaler \cite{schoenthaler2017simulated}, Szilas \cite{szilas2019virtual}, UTTimePortal \cite{zielke2016beyond,zielke2016using}, Zlotos \cite{zlotos2016scenario}, \\
    \cline{2-3}
        & \emph{Open-option} & Banszki \cite{banszki2018clinical,quail2016student}, CynthiaYoungVP\cite{foster2016using}, NERVE
        \cite{hirumi2016advancingPart2,hirumi2016advancing,kleinsmith2015understanding}, Maicher \cite{maicher2017developing}, Washburn \cite{washburn2020virtual} \\
    \cline{2-3}
        & \emph{Hybrid} & Ochs \cite{ochs2019training}  \\
    \hline
    %FEEDBACK
        \multirow{8}{*}{Feedback} & \emph{Replay feature} & Dupuy \cite{dupuy2019virtual}, Communicate! \cite{jeuring2015communicate}, Ochs \cite{ochs2019training}, VSPR \cite{peddle2019exploring,peddle2019development}, Zlotos \cite{zlotos2016scenario}\\
    \cline{2-3}
        & \emph{Virtual instructor} & At-Risk in Primary Care \cite{albright2018using}, Suicide Prevention \cite{o2019suicide}, Schoenthaler \cite{schoenthaler2017simulated}\\
    \cline{2-3}
        & \emph{Multiple session structure} & MPathic-VR \cite{guetterman2019medical,kron2017using}  \\
    \cline{2-3}
        & \emph{Quantitative emotional feedback} & At-Risk in Primary Care \cite{albright2018using}, Schoenthaler \cite{schoenthaler2017simulated}\\
    \cline{2-3}
        & \emph{Qualitative personalized post-feedback} & Jacklin \cite{jacklin2019virtual,jacklin2018improving}, Richardson \cite{richardson2019virtual}\\
    \cline{2-3}
        & \emph{Empathy feedback} & CynthiaYoungVP \cite{foster2016using}\\
    \cline{2-3}
        & \emph{Clinical discoveries available} & NERVE
        \cite{hirumi2016advancingPart2,hirumi2016advancing,kleinsmith2015understanding}\\
    \cline{2-3}
        & \emph{Game elements} & Dupuy \cite{dupuy2019virtual}, MPathic-VR \cite{guetterman2019medical,kron2017using}, Communicate! \cite{jeuring2015communicate}, Schoenthaler \cite{schoenthaler2017simulated}, UT-Time Portal \cite{zielke2016beyond,zielke2016using}\\
    \hline
        \multirow{3}{*}{Gamification} & \emph{Scoring system} & Dupuy \cite{dupuy2019virtual}, MPathic-VR \cite{guetterman2019medical,kron2017using}, Communicate! \cite{jeuring2015communicate}, Schoenthaler \cite{schoenthaler2017simulated}, UT-Time Portal \cite{zielke2016beyond,zielke2016using}\\
    \cline{2-3}
        & \emph{Badge system} & UTTimePortal \cite{zielke2016beyond,zielke2016using}  \\
    \cline{2-3}
        & \emph{Countdown timed events} & HOLLIE \cite{adefila2020students}\\
    \hline
     \end{tabular}
\end{center}
}
\end{table}
\normalsize

%Technical Table: NEW VERSION
%ditched tabularx, use only normal tabular

\begin{table} [t]
\scriptsize{
\begin{center}
    \caption{Synopsis of the reviewed VPs for each technical design category}
    \label{table:technicalDesignTable}
    \begin{tabular}{| p{1.8cm} | p{2cm} | p{8cm} |}
    \hline
        \rowcolor{mygray}
        \multicolumn{3}{|c|}{\textbf{Technical Design}}\\
    \hline
        \rowcolor{lightgray}
        \textbf{Category}  & \textbf{Subcategory} & \textbf{Virtual Patients}\\
    \hline
    %PRESENTATION FORMAT
        \multirow{5}{1.8cm}{Presentation format} & \emph{Image} & HOLLIE \cite{adefila2020students},  Marei \cite{marei2018use}\\
    \cline{2-3}
        & \emph{Video} & CynthiaYoungVP \cite{foster2016using}, Suicide Prevention \cite{o2019suicide}, VSPR \cite{peddle2019exploring,peddle2019development}\\
    \cline{2-3}
        & \emph{Desktop VR} & AtRiskInPrimaryCare \cite{albright2018using}, MPathic-VR \cite{guetterman2019medical,kron2017using}, NERVE
        \cite{hirumi2016advancingPart2,hirumi2016advancing,kleinsmith2015understanding}, Jacklin \cite{jacklin2019virtual,jacklin2018improving}, 
        Communicate! \cite{jeuring2015communicate}, Richardson \cite{richardson2019virtual}, Schoenthaler \cite{schoenthaler2017simulated}, Szilas \cite{szilas2019virtual}, UTTimePortal \cite{zielke2016beyond,zielke2016using}, Zlotos \cite{zlotos2016scenario}\\
    \cline{2-3}
        & \emph{Large volume display} & Dupuy \cite{dupuy2019virtual}, Banszki \cite{banszki2018clinical,quail2016student}, Maicher \cite{maicher2017developing}, Washburn \cite{washburn2020virtual}\\
    \cline{2-3}
        & \emph{Immersive VR} & Ochs \cite{ochs2019training}, CESTOLVRClinic \cite{sapkaroski2018implementation}\\
    \hline
    %INPUT INTERFACE
        \multirow{4}{1.8cm}{Input interface} & \emph{Typed} & HOLLIE \cite{adefila2020students}, AtRiskInPrimaryCare \cite{albright2018using}, CynthiaYoungVP \cite{foster2016using}, NERVE
        \cite{hirumi2016advancingPart2,hirumi2016advancing,kleinsmith2015understanding}, Jacklin \cite{jacklin2019virtual,jacklin2018improving}, Communicate! \cite{jeuring2015communicate}, Maicher \cite{maicher2017developing}, Marei \cite{marei2018use}, Suicide Prevention \cite{o2019suicide}, VSPR \cite{peddle2019exploring,peddle2019development}, Richardson \cite{richardson2019virtual}, CESTOLVRClinic \cite{sapkaroski2018implementation}, Schoenthaler \cite{schoenthaler2017simulated}, Szilas \cite{szilas2019virtual}, UTTimePortal \cite{zielke2016beyond,zielke2016using}, Zlotos \cite{zlotos2016scenario}\\
    \cline{2-3}
        & \emph{Voice-controlled} & Banszki \cite{banszki2018clinical,quail2016student}, Dupuy \cite{dupuy2019virtual}, MPathic-VR \cite{guetterman2019medical,kron2017using}, Maicher \cite{maicher2017developing}, Ochs \cite{ochs2019training}, CESTOLVRClinic \cite{sapkaroski2018implementation}\\
    \cline{2-3}
        %& \emph{NUI} & CESTOLVRClinic \cite{sapkaroski2018implementation}\\
    %\cline{2-3}
        & \emph{Hybrid} & Washburn \cite{washburn2020virtual} \\
    \cline{2-3}
        & \emph{Non-verbal} & Banszki \cite{banszki2018clinical,quail2016student}, Dupuy \cite{dupuy2019virtual}, MPathic-VR \cite{guetterman2019medical,kron2017using}, Maicher \cite{maicher2017developing}, CESTOLVRClinic \cite{sapkaroski2018implementation}\\
    \hline
    % DISTRIBUTION
        \multirow{3}{*}{Distribution} & \emph{Standalone} & Banszki \cite{banszki2018clinical,quail2016student}, Dupuy \cite{dupuy2019virtual}, MPathic-VR \cite{guetterman2019medical,kron2017using}, Maicher \cite{maicher2017developing}, Marei \cite{marei2018use}, Ochs \cite{ochs2019training}, CESTOLVRClinic \cite{sapkaroski2018implementation}, Szilas \cite{szilas2019virtual}, Washburn \cite{washburn2020virtual}\\
    \cline{2-3}
        & \emph{Web-based} & HOLLIE \cite{adefila2020students}, AtRiskInPrimaryCare \cite{albright2018using}, CynthiaYoungVP \cite{foster2016using}, NERVE
        \cite{hirumi2016advancingPart2,hirumi2016advancing,kleinsmith2015understanding}, Jacklin \cite{jacklin2019virtual,jacklin2018improving}, Maicher \cite{maicher2017developing}, Suicide Prevention \cite{o2019suicide}, VSPR \cite{peddle2019exploring,peddle2019development}, Richardson \cite{richardson2019virtual}, UTTimePortal \cite{zielke2016beyond,zielke2016using}, Zlotos \cite{zlotos2016scenario}\\
    \cline{2-3}
         & \emph{Undisclosed} &  Communicate! \cite{jeuring2015communicate}, Schoenthaler \cite{schoenthaler2017simulated} \\
    \hline
     \end{tabular}
\end{center}
}
\end{table}
\normalsize

\textbf{Structure}. The \emph{structure} defines the hierarchical organization and presentation of VP-related information within the simulation.
According to \cite{bearman2001random}, two non-mutually exclusive approaches (i.e., \emph{narrative} and \emph{problem solving}) can be defined. 
The narrative VPs are characterized by a coherent storyline, with a focus on cause-effect decisions that have a direct impact on the evolution of the simulation. These VPs present the patient as more than a mere collection of data and statistics, and devote a certain degree of attention to interpersonal and communication aspects of the provider-patient interaction.  On the contrary, the \emph{problem solving} VPs are mainly used to support inquiry-based learning scenarios such as teaching clinical reasoning, differential diagnosis, and history-taking skills. These contexts do not usually concern  portraying authentic communicative acts, since they mainly involve making questions and observations. 

Scholars and researchers recognize the power of \emph{narrative} design in the creation of meaningful learning experiences \cite{bearman2001random,marei2018use}.
Narrative-based simulations that reflect the consequences of the choices and the actions made by the learner can lead to the development of more effective VPs.  In particular, for VPs used to teach communication skills, experimental  evidence supports the value of \emph{narrative} design \cite{bearman2001random}. Thus, it is not surprising that all the VPs presented in the selected works are based on this approach. Nevertheless, it is interesting to note that 10 out the \totalVPs VPs analyzed integrate the \emph{narrative} design with a \emph{problem solving} component. 
This component aims to teach particular skills like history-taking (Cynthia Young VP \cite{foster2016using}, NERVE \cite{hirumi2016advancingPart2,hirumi2016advancing,kleinsmith2015understanding}, Maicher \cite{maicher2017developing}), clinical reasoning (VSPR \cite{peddle2019exploring,peddle2019development}, Richardson \cite{richardson2019virtual},  Washburn \cite{washburn2020virtual}, UT-Time Portal \cite{zielke2016beyond,zielke2016using}, Zlotos \cite{zlotos2016scenario}),  physical examinations (HOLLIE \cite{adefila2020students}, NERVE \cite{hirumi2016advancingPart2,hirumi2016advancing,kleinsmith2015understanding},  CESTOL VR Clinic \cite{sapkaroski2018implementation}), compilation and consultation of electronic medical records (HOLLIE \cite{adefila2020students}, Maicher \cite{maicher2017developing}), and medication administration (HOLLIE \cite{adefila2020students}). 





\textbf{Unfolding}. Given the prominence of narrative design in the development of VPs for communication skill training, another relevant design element is defining how the narrative may unfold, and how the simulation can evolve between different states. A preliminary subdivision can be made among \textit{linear} and \textit{non-linear} narratives. In the former design, VPs have a linear path to follow and the decisions, questions and options possibly presented to the learner do not influence the simulation outcome. It is clear that this design severely limits learning effectiveness, and none of the works included in this survey implemented it.

On the contrary, the \emph{non-linear} navigational structure of VPs offers learners a greater flexibility, and an higher degrees of interactivity and control. In this case, two further choices are possible. In the \textit{closed-option} design, the simulation advances to the next state by selecting one of the possible alternatives or explicit paths offered to learners. Simulation states are organized in a hierarchical structure (similar to that of the \quotes{choose your own adventure} books), which stresses the cause-effect relation of the user's choices. 
The \textit{open-option} design (sometimes referred in the literature as \quotes{free-text} \cite{jacklin2019virtual,janda2004simulation,mccoy2016evaluating}  or \quotes{open-chat} \cite{hirumi2016advancing}) can be used to develop  free-form simulations where states are organized in a partially or fully interconnected structure, and users are free to interact with the VP as they wish, thus emulating the flow of a real conversation. As we will discuss  more in detail in Section \ref{sec:technicalDesign}, learners can formulate questions and statements by either typing or having their speech transcribed into written words using speech-to-text software. Then, the application parses the text and elaborates a proper response. The VP state progression can be influenced also by non-verbal cues such as gestures, body posture, expressions and sight.


%In regards to unfolding, several works report that many users feel 'restricted' by the closed-option interface \cite{dupuy2019virtual,peddle2019development,jacklin2019virtual,hirumi2016advancing}, preferring an open-option structure or the possibility to chose between the two variants, like in NERVE \cite{hirumi2016advancingPart2,hirumi2016advancing,kleinsmith2015understanding}.


The \textit{closed-option} design characterizes most of the analyzed VPs (15), with only four works  based on an \textit{open-option} design; as for the remaining, one VP implemented both options (NERVE \cite{hirumi2016advancingPart2,hirumi2016advancing,kleinsmith2015understanding}), whereas the other one can be considered an hybrid between the two designs (Ochs \cite{ochs2019training}). One explanation for this result is the lower complexity of the \textit{closed-option} implementation, although some Authors \cite{carnell2015adapting,jacklin2019virtual} argued that a such an approach may be more suitable for novices who, for example, may still be inexperienced about the procedures to follow in a patient encounter. However, other works reported that many students feel restricted by the \textit{closed-option} interface \cite{dupuy2019virtual,hirumi2016advancing,jacklin2019virtual,peddle2019development}, preferring either an \textit{open-option} structure or the possibility to chose between the two variants. It is worth noting that implementing both options allows the use of the same VP in different educational settings. %\edo{precisazione}For instance, in NERVE \cite{hirumi2016advancingPart2,hirumi2016advancing,kleinsmith2015understanding} the two systems coexist, to leverage the less stress-inducing \textit{closed-option} variant in the learning process, and the less restrictive \textit{open-option} setting for rehearsal sessions. 
%Ho cambiato leggermente questo paragrafo perché sembrava che in NERVE ci fosse una netta distinzione tra sessioni di evaluation e learning, in realtà questa potenziale divisione è solo una proposta/considerazione degli autori
For instance, in NERVE \cite{hirumi2016advancingPart2,hirumi2016advancing,kleinsmith2015understanding}, the less stress-inducing \textit{closed-option} variant is used in the learning sessions, while rehearsal sessions leverage the less restrictive \textit{open-option} setting. 
% This concept ties to another common feedback given by users: the idea that the same simulations should have diversified levels of difficulty \cite{dupuy2019virtual}, \cite{peddle2019development}, \cite{peddle2019exploring} to appropriately challenge learners of different skill levels.
% \edo{end of addon about effectiveness}
Another interesting approach is the hybrid model implemented in Ochs \cite{ochs2019training}, where the user can freely interact through voice with the VP. Then, a human facilitator selects, from a set of possible closed-options, the utterance that semantically resembles the original phrase the most, prompting the appropriate response from the patient. The advantage of this approach is that it ease the development burden of what appears to learners as an \textit{open-option} VP, with the clear disadvantage of preventing its use as a self-learning and self-evaluation tool.



\textbf{Feedback}. With the term \emph{feedback} we refer to any form of instructional scaffolding enclosed in the simulation itself (i.e., we exclude any feedback external to the simulation, such as post-simulation debrief and reflection sessions with mentors and peers).
%(i.e., we exclude any out of simulation feedback such as post-simulation debrief and reflection sessions with mentors and peers).
Feedback can be given in many different forms, from explicit messages to discoveries made, questions answered, and visual representations of the current VP state. 

While researchers recognize the relevance of immediate and after-action feedback as an essential feature in communication-based VPs (\cite{adefila2020students,jacklin2018improving,marei2018use,peddle2019exploring,quail2016student}), 
%\cite{kleinsmith2015understanding,fiorella2012applying,mckimm2006abc,albright2018using,maicher2017developing,mcgaghie2011does,barry2005features,ericsson1993role,huwendiek2009design,botezatu2010virtual}), 
it is surprising that six of the VPs surveyed (Banszki \cite{banszki2018clinical,quail2016student}, Maicher \cite{maicher2017developing}, Marei \cite{marei2018use}, Szilas \cite{szilas2019virtual}, Washburn \cite{washburn2020virtual}, CESTOLVRClinic \cite{sapkaroski2018implementation}) do not embed any type of built-in feedback. 
The remaining works take different approaches. Four VPs (Dupuy \cite{dupuy2019virtual}, Communicate! \cite{jeuring2015communicate}, VSPR \cite{peddle2019exploring,peddle2019development}, Zlotos \cite{zlotos2016scenario}) offer the possibility, after the simulation is completed, to \textit{replay} some of its parts and analyze the outcome of different choices. 
Three simulations (At-Risk in Primary Care \cite{albright2018using}, Suicide Prevention \cite{o2019suicide},  Schoenthaler \cite{schoenthaler2017simulated}) include a \emph{virtual instructor}, i.e., a virtual tutor that gives advice or feedback based on the user's choices.
MPathic-VR \cite{guetterman2019medical,kron2017using} employs a \emph{multiple session structure}, where the first run acts as a learning phase, concluded by an automated and complete feedback provided by the system, whereas the second run (set in the same scenario) serves as an evaluation phase. This type of structure appears to be highly appreciated by students since they can immediately put into practice what they learned during the first phase, taking into account the feedback received.
Two VPs (At-Risk in Primary Care \cite{albright2018using}, Schoenthaler \cite{schoenthaler2017simulated}) feature \emph{quantitative emotional feedback} in the form of an on-screen trust meter that indicates users how effective their communication choices were at building a relation with the patient. Two VPs (Jacklin \cite{jacklin2019virtual,jacklin2018improving}, Richardson \cite{richardson2019virtual}) offer learners a \emph{personalized qualitative feedback} at the end of the simulation. CynthiaYoungVP \cite{foster2016using} uses an hybrid approach between automated and human feedback. At the end of the simulation, the students can access a web page containing \emph{empathy feedback} and scores for each response given, where scores are manually assigned by human experts. 
The approach adopted in NERVE \cite{hirumi2016advancing,hirumi2016advancingPart2,kleinsmith2015understanding} is to inform learners about the number of \emph{clinical discoveries} and empathic responses available, thus providing inexperienced users with useful guidelines on how to proceed with the conversation. Although it has not been  implemented yet, the work in  \cite{hirumi2016advancingPart2} put forth the proposition of providing \quotes{cumulative feedback} on how users are developing their skills across multiple patient scenarios. Authors suggest that this feature could both encourage repeated use of the system and act as a motivator for performance improvement. 
Finally, there is a number of VPs (HOLLIE \cite{adefila2020students}, Dupuy \cite{dupuy2019virtual}, Communicate! \cite{jeuring2015communicate}, MPathic-VR \cite{guetterman2019medical,kron2017using}, Schoenthaler \cite{schoenthaler2017simulated}, UT-Time Portal \cite{zielke2016beyond,zielke2016using}) that leverage \emph{game elements} as feedback. Since the introduction of game elements is a relevant design feature, it is discussed in detail in the  following subsection. 


%GAMIFICATION 
\textbf{Gamification}. The idea of introducing game mechanics in any learning experience is to make them more enjoyable and engaging.  Researchers and practitioners recognize that game mechanics contribute to making the learning experience more effective, fostering self-improvement and healthy competition between peers \cite{benedict2013promotion,festinger1954theory}. 
The mechanics used the most in \quotes{gamified} experiences are \emph{scores}, \emph{badges} and \emph{leaderboards}. 
Scores are a quantitative and immediate form of feedback that acts as an extrinsic motivator to foster users to improve their performance. \emph{Scoring systems} can be found in Dupuy \cite{dupuy2019virtual}, Communicate! \cite{jeuring2015communicate}, MPathic-VR \cite{guetterman2019medical,kron2017using}, Schoenthaler \cite{schoenthaler2017simulated} and UT-Time Portal \cite{zielke2016beyond,zielke2016using}. In particular, Communicate! \cite{jeuring2015communicate} and Schoenthaler \cite{schoenthaler2017simulated} provide separate scores for each learning goal (e.g., empathy control, language clarity, and pick up of patient's concerns). Such a feature can help learners to tell the areas in which they are already proficient, distinguishing them from those needing improvement.

Badges are visual representations used in games to prove that the player has reached an intermediate goal on his/her road to mastery. Their purpose is twofold: they are a form of gratification to the learners, and they allow trainees to share achievements with peers and educators. Thus, they also represent an extrinsic motivator for improvement.
%Thus, they represent both an intrinsic and extrinsic motivator for improvement.
In our survey,  the only simulation we found that implements a  \emph{badge system} is UT-Time Portal \cite{zielke2016beyond,zielke2016using}. VSPR \cite{peddle2019exploring,peddle2019development} features a system of certificates issued to users at the end of each learning module which shall be regarded as an \quotes{intrinsic-only} motivator, since there is no overarching structure that enables users to see each others' achievements.
%A cosa servono i rankings, perché nessuno li implementa?

Finally, leaderboards (or rankings) are a primarily extrinsic motivator that leverage competition with peers when they compare their performance to that of others (it should be noted that, for highly competitive individuals, the act of \quotes{climbing the leaderboard} can also be seen as a relevant intrinsic motivator independent of the context).
%Finally, leaderboards (or rankings) are a relevant intrinsic motivator for learners that leverage competition with peers when they compare their performance to that of others.
Surprisingly, despite their demonstrated benefits for learning, in our survey, we found no example of public ranking and leaderboards.

A final note is for HOLLIE \cite{adefila2020students}, which implements the very peculiar idea of a Tamagotchi-style VP the learners have to care for (adequately, at regular intervals and in real-time) over two weeks. Here, the leading game mechanics (constant care over a long period) reproduces quite accurately the daily tasks of a nurse leveraging the innate sense of responsibility in the players. 


%%TABLE WITH TECHNICAL DESGIN ONLY
%\begin{longtable}{ | p{1cm} | *{15}{l} |}
%\begin{tabularx}{\linewidth} {>{\raggedleft\arraybackslash}X >{\raggedleft\arraybackslash}X >{\raggedleft\arraybackslash}X >{\raggedleft\arraybackslash}X >{\raggedleft\arraybackslash}X >{\raggedleft\arraybackslash}X >{\raggedleft\arraybackslash}X >{\raggedleft\arraybackslash}X >{\raggedleft\arraybackslash}X >{\raggedleft\arraybackslash}X}
{\rowcolors{3}{mywhite}{mygray}
%\begin{tabularx}{\linewidth} {X | X | X | X | X | X | X | X | X | X}
\begin{tabularx}{\linewidth}
\footnotesize
{|>{\hsize=0.5\hsize\linewidth=\hsize}X |
>{\hsize=1.5\hsize\linewidth=\hsize}X |
>{\hsize=.75\hsize\linewidth=\hsize}X |
>{\hsize=.5\hsize\linewidth=\hsize}X |
>{\hsize=.75\hsize\linewidth=\hsize}X |
>{\hsize=2\hsize\linewidth=\hsize}X |}


%\textbf{Study} & \textbf{Geographical Location and Level} & \textbf{Identification} & 
%\textbf{Multiplier Result} \\ \hline \hline
%\endfirsthead
\rowcolor{lightgray}
\textbf{Article} & \multicolumn{5}{|l|}{\textbf{Technical Design}} \\
\rowcolor{lightgray}
& Hardware & Presentation & Input Interface & Distribution & Other Tech. Features \\
%& \multicolumn{1}{|l|}{Category} & \multicolumn{1}{|l|}{Navigation} & \multicolumn{1}{|l|}{Feedback} & \multicolumn{1}{|l|}{Gamification} & \multicolumn{1}{|l|}{Hardware} & \multicolumn{1}{|l|}{Presentation} & \multicolumn{1}{|l|}{Input} & \multicolumn{1}{|l|}{Distribution} & \multicolumn{1}{|l|}{Other Tech. Features}  \\ 
%\hline 
\specialrule{.1em}{.05em}{.05em} 
\endhead

\cite{adefila2020students} & Any device with a web browser	& Graphic (Image) & Typed &	Web-Based & /\\ 

\cite{albright2018using} & Any device with a web browser & Graphic (Image) & Typed	& Web-Based & /\\ 

\cite{banszki2018clinical} + \cite{quail2016student} & PC, 64” monitor, Microphone & Graphic (3D) & Voice-Controlled & Standalone &	Human-Controlled,Gesture and Facial Expression Output \\  

\cite{dupuy2019virtual} & PC, Vertical 40” screen, Microphone, Camera & Graphic (3D) & Voice-Controlled & Standalone &	Facial Expression Detection through face tracking\\ 

\cite{foster2016using} & Any device with a web browser & Text-Based, Graphic (Video) & Typed & Web-Based & Human-Controlled Empathy Feedback\\ 

\cite{guetterman2019medical} + \cite{kron2017using} & PC, Microsoft Kinect, Microphone & Graphic (3D) & Voice-Controlled &	Standalone &	Facial Expression and Body Posture Detection and output, Recorded Voiceover, Multiple VHs\\ 

\cite{hirumi2016advancingPart2} + \cite{hirumi2016advancing} + \cite{kleinsmith2015understanding} & Any device with a web browser	& Graphic (3D) & Typed & Web-Based & /\\ 

\cite{jacklin2019virtual} + \cite{jacklin2018improving} & Any device with a web browser & Graphic (3D) & Typed &	Web-Based & Body Posture Output, Recorded Voiceover	\\ 

\cite{jeuring2015communicate} & \textbf{\emph{Unclear}} & Graphic (3D) & Typed & \textbf{\emph{Unclear}} & Scenario Editor\\ 

\cite{maicher2017developing} & PC, Microsoft Kinect, Multi-Array Microphone & Graphic (3D) & Typed + Voice-Controlled & Web-Based + Standalone & Motion-Captured Animations, Movement Posture Detection and Output\\

\cite{marei2018use} & PC & Graphic (Image) &	Typed & \textbf{\emph{Unclear}} & /\\ 

\cite{ochs2019training} & PC, HMD (Oculus Rift), CAVE, High-End Microphone & Graphic (3D, IVR) & Voice-Controlled &  Standalone & Same VP deployed in desktop, VR and Cave versions. Speech recognition is Human-Controlled, Non-Verbal output (facial expression and body posture). Text-To-Speech, Lip Synch, Virtual Playback\\

\cite{o2019suicide} & Any device with a web browser & Graphic (Video) & 	Typed & Web-Based & /\\ 

\cite{peddle2019exploring} + \cite{peddle2019development} & Any device with a web browser & Graphic (Video) & Typed & Web-Based & /\\ 

\cite{richardson2019virtual} & Any device with a web browser & Graphic (3D) &	Typed &	Web-Based & /\\

\cite{sapkaroski2018implementation} & PC, HMD (Various), Leap Motion, Oculus Touch & Graphic (3D, IVR) & Typed + Voice-Controlled + NUI & Standalone & Remote progress tracking for educators\\ 

\cite{schoenthaler2017simulated} & \textbf{\emph{Unclear}} &  Graphic (3D) & Typed & 	\textbf{\emph{Unclear}} &	Can play as both provider and patient, Non-Verbal Output\\ 

\cite{szilas2019virtual} & PC	& Graphic (3D) &	Typed & Standalone & /\\ 

\cite{washburn2020virtual} & PC or Laptop, Large Screen & Graphic (3D) & Hybrid & Standalone & Human-Transcribed Voice Controls\\ 

\cite{zielke2016beyond} + \cite{zielke2016using} & Any device with a web browser & Graphic (3D) & Typed & 	Web-Based & Multiple VHs, Non-Verbal Output\\ 

\cite{zlotos2016scenario} & Any device with a web browser & Graphic (3D) & Typed  & Web-Based & Motion-Captured Animations, Recorded Voiceover\\ 

\hline
%\end{longtable}
\end{tabularx}
}



\subsection{Technical features}
\label{sec:technicalDesign}

This category explores, from a technical perspective, the different solutions that can support (and implement) the choices made in the instructional design, i.e., which are the technical features that enable the accomplishment of the envisioned learning activities. These features include the physical devices required to guarantee the exchange of information between the learner and the system, and the possible communication infrastructure needed to run the simulation.

%%Technical Table: NEW VERSION
%ditched tabularx, use only normal tabular

\begin{table} [t]
\scriptsize{
\begin{center}
    \caption{Synopsis of the reviewed VPs for each technical design category}
    \label{table:technicalDesignTable}
    \begin{tabular}{| p{1.8cm} | p{2cm} | p{8cm} |}
    \hline
        \rowcolor{mygray}
        \multicolumn{3}{|c|}{\textbf{Technical Design}}\\
    \hline
        \rowcolor{lightgray}
        \textbf{Category}  & \textbf{Subcategory} & \textbf{Virtual Patients}\\
    \hline
    %PRESENTATION FORMAT
        \multirow{5}{1.8cm}{Presentation format} & \emph{Image} & HOLLIE \cite{adefila2020students},  Marei \cite{marei2018use}\\
    \cline{2-3}
        & \emph{Video} & CynthiaYoungVP \cite{foster2016using}, Suicide Prevention \cite{o2019suicide}, VSPR \cite{peddle2019exploring,peddle2019development}\\
    \cline{2-3}
        & \emph{Desktop VR} & AtRiskInPrimaryCare \cite{albright2018using}, MPathic-VR \cite{guetterman2019medical,kron2017using}, NERVE
        \cite{hirumi2016advancingPart2,hirumi2016advancing,kleinsmith2015understanding}, Jacklin \cite{jacklin2019virtual,jacklin2018improving}, 
        Communicate! \cite{jeuring2015communicate}, Richardson \cite{richardson2019virtual}, Schoenthaler \cite{schoenthaler2017simulated}, Szilas \cite{szilas2019virtual}, UTTimePortal \cite{zielke2016beyond,zielke2016using}, Zlotos \cite{zlotos2016scenario}\\
    \cline{2-3}
        & \emph{Large volume display} & Dupuy \cite{dupuy2019virtual}, Banszki \cite{banszki2018clinical,quail2016student}, Maicher \cite{maicher2017developing}, Washburn \cite{washburn2020virtual}\\
    \cline{2-3}
        & \emph{Immersive VR} & Ochs \cite{ochs2019training}, CESTOLVRClinic \cite{sapkaroski2018implementation}\\
    \hline
    %INPUT INTERFACE
        \multirow{4}{1.8cm}{Input interface} & \emph{Typed} & HOLLIE \cite{adefila2020students}, AtRiskInPrimaryCare \cite{albright2018using}, CynthiaYoungVP \cite{foster2016using}, NERVE
        \cite{hirumi2016advancingPart2,hirumi2016advancing,kleinsmith2015understanding}, Jacklin \cite{jacklin2019virtual,jacklin2018improving}, Communicate! \cite{jeuring2015communicate}, Maicher \cite{maicher2017developing}, Marei \cite{marei2018use}, Suicide Prevention \cite{o2019suicide}, VSPR \cite{peddle2019exploring,peddle2019development}, Richardson \cite{richardson2019virtual}, CESTOLVRClinic \cite{sapkaroski2018implementation}, Schoenthaler \cite{schoenthaler2017simulated}, Szilas \cite{szilas2019virtual}, UTTimePortal \cite{zielke2016beyond,zielke2016using}, Zlotos \cite{zlotos2016scenario}\\
    \cline{2-3}
        & \emph{Voice-controlled} & Banszki \cite{banszki2018clinical,quail2016student}, Dupuy \cite{dupuy2019virtual}, MPathic-VR \cite{guetterman2019medical,kron2017using}, Maicher \cite{maicher2017developing}, Ochs \cite{ochs2019training}, CESTOLVRClinic \cite{sapkaroski2018implementation}\\
    \cline{2-3}
        %& \emph{NUI} & CESTOLVRClinic \cite{sapkaroski2018implementation}\\
    %\cline{2-3}
        & \emph{Hybrid} & Washburn \cite{washburn2020virtual} \\
    \cline{2-3}
        & \emph{Non-verbal} & Banszki \cite{banszki2018clinical,quail2016student}, Dupuy \cite{dupuy2019virtual}, MPathic-VR \cite{guetterman2019medical,kron2017using}, Maicher \cite{maicher2017developing}, CESTOLVRClinic \cite{sapkaroski2018implementation}\\
    \hline
    % DISTRIBUTION
        \multirow{3}{*}{Distribution} & \emph{Standalone} & Banszki \cite{banszki2018clinical,quail2016student}, Dupuy \cite{dupuy2019virtual}, MPathic-VR \cite{guetterman2019medical,kron2017using}, Maicher \cite{maicher2017developing}, Marei \cite{marei2018use}, Ochs \cite{ochs2019training}, CESTOLVRClinic \cite{sapkaroski2018implementation}, Szilas \cite{szilas2019virtual}, Washburn \cite{washburn2020virtual}\\
    \cline{2-3}
        & \emph{Web-based} & HOLLIE \cite{adefila2020students}, AtRiskInPrimaryCare \cite{albright2018using}, CynthiaYoungVP \cite{foster2016using}, NERVE
        \cite{hirumi2016advancingPart2,hirumi2016advancing,kleinsmith2015understanding}, Jacklin \cite{jacklin2019virtual,jacklin2018improving}, Maicher \cite{maicher2017developing}, Suicide Prevention \cite{o2019suicide}, VSPR \cite{peddle2019exploring,peddle2019development}, Richardson \cite{richardson2019virtual}, UTTimePortal \cite{zielke2016beyond,zielke2016using}, Zlotos \cite{zlotos2016scenario}\\
    \cline{2-3}
         & \emph{Undisclosed} &  Communicate! \cite{jeuring2015communicate}, Schoenthaler \cite{schoenthaler2017simulated} \\
    \hline
     \end{tabular}
\end{center}
}
\end{table}
\normalsize


\textbf{Presentation format}. The surveyed works provide learners with different types of outputs aimed to deliver VP information to the learner and presenting the VP itself. A first rough subdivision is between \textit{text-based} and \textit{graphic} representations. In screen-based text simulators, the VP is presented mainly in the form of a collection of text and structured data, with the possible inclusion of images portraying a static patient or exam results. However, the lack of a graphic component capable of displaying a patient that can express emotions as the simulation unfolds (and, consequently, change posture and facial expressions according to its current  state) is one of the main limitations of these approaches. Therefore, researchers started extending text-based simulations into learning activities with a relevant graphic component. 

All the VPs surveyed in this work fall in the \emph{graphic} category, which can be further classified in \emph{image}, \emph{video} and \emph{3D}. VPs in the \emph{image} subclass are presented through a series of static images (either photographs or drawings), such as HOLLIE \cite{adefila2020students} and Marei \cite{marei2018use}.  Some VPs present their case using \emph{video}, either in the form of live footage (Suicide Prevention \cite{o2019suicide}, VSPR \cite{peddle2019exploring,peddle2019development}) or as a computer-generated offline video (Cynthia Young VP \cite{foster2016using}). However, a clear limitation of this approach is its lack of flexibility, since the actor video cannot be re-purposed to portray a different clinical case. 

The majority of surveyed simulations fall in the \emph{3D} subclass and present the patient and the environment as 3D models rendered in real-time.  Their main advantage is that tweaking and expanding a simulation using 3D characters can be done in a much more modular fashion than with \emph{image} and \emph{video}-based VPs. Another advantage is that the sense of immersion and presence are greater than those that can be delivered by \emph{image} and \emph{video}-based VPs. 

Most of the 3D approaches rely on standard \emph{desktop VR} (DVR) settings (10, namely AtRiskInPrimaryCare \cite{albright2018using}, MPathic-VR \cite{guetterman2019medical,kron2017using}, NERVE \cite{hirumi2016advancingPart2,hirumi2016advancing,kleinsmith2015understanding}, Jacklin \cite{jacklin2019virtual,jacklin2018improving}, Communicate! \cite{jeuring2015communicate}, Richardson \cite{richardson2019virtual}, Schoenthaler \cite{schoenthaler2017simulated}, Szilas \cite{szilas2019virtual}, UTTimePortal \cite{zielke2016beyond,zielke2016using}, Zlotos \cite{zlotos2016scenario}). However, since trying to maximize the feeling of immersion and presence is extremely relevant for engaging learners and helping them achieve the expected learning outcomes, some works integrate (partially or fully) immersive technologies. Four of them (Dupuy \cite{dupuy2019virtual}, Banszki \cite{banszki2018clinical,quail2016student}, Maicher \cite{maicher2017developing}, Washburn \cite{washburn2020virtual}) take advantage of \emph{large volume displays}  to portray a life-sized and more natural interaction with the patient, and CESTOL VR Clinic \cite{sapkaroski2018implementation} uses an HMD for the same purpose. In Ochs \cite{ochs2019training}, three different setups (DVR, immersive VR with an HMD, and immersive VR in a CAVE) are compared to analyze their effect on the sense of presence. The outcome of this experiment demonstrates that immersive environments improve the sense of presence and perception of the VP, with the CAVE scoring slightly better than the HMD.
It should be noted, however, that while \emph{immersive VR} (IVR) offers a higher degree of immersion and presence over DVR, there are still accessibility issues that limit its use, in particular when the VP is intended for self-learning and self-training.


\textbf{Input Interface}. This category describes the input methods through which the user influences the unfolding of the VP simulation.
In the case of \emph{typed} interfaces, user's textual intents are entered by typing on a keyboard or selecting an item in a predefined list of choices.
\emph{Voice-controlled} simulations use natural language, which is then parsed into text through a speech-to-text module, usually offered by external Natural Language Processing (NLP) APIs \cite{foster2016using,maicher2017developing}. 
Finally, the integration within the simulation of Natural User Interfaces (NUI) allows to influence the VP state evolution through additional \textit{non-verbal} input cues such as eye contact, distance, facial expression, gestures and body posture, which can be captured with cameras and other hardware. % to provide additional input cues for the simulation.


Among the analyzed VPs, 14 feature a \emph{typed}-only input, only five are \emph{voice-controlled} (Banszki \cite{banszki2018clinical,quail2016student}, Dupuy \cite{dupuy2019virtual}, MPathic-VR \cite{guetterman2019medical,kron2017using}, Ochs \cite{ochs2019training}, CESTOL VR Clinic \cite{sapkaroski2018implementation}), Maicher \cite{maicher2017developing} has both options, and Washburn \cite{washburn2020virtual} can be considered as a hybrid solution since a human facilitator transcribes the spoken commands through a \emph{typed} interface. 
One of the reasons behind the limited use of voice controls is the fear, expressed by some Authors \cite{maicher2017developing,ochs2019training}, that  NLP systems may be technically hard to implement and prone to wrong transcriptions, which may lead to misunderstandings or unrecognized utterances, break the sense of immersion and cause frustration in the user \cite{bloodworth2010initial}.
This is why some Authors (e.g., Banszki \cite{banszki2018clinical,quail2016student}, Ochs \cite{ochs2019training}, and Washburn \cite{washburn2020virtual}) decided to have a human facilitator taking over the function of the NLP module. 
 Moreover, a VP featuring only voice controls cannot be used by learners with speech impairments \cite{maicher2017developing}. However, it should be stressed that, nowadays, speech-to-text APIs have become widely available, and their quality keeps improving; thus, problems related to imprecise transcriptions should be less and less daunting in the coming years. As for the impaired people, a smart solution to achieve maximum flexibility and accessibility could be to let the users  choose between  \emph{typed} and \emph{voice-controlled} interfaces freely. It should also be noted that IVR environments favour the use of \emph{voice-controlled} interfaces over alternative solutions such as virtual keyboards or situation-specific control boards  %, which requires for interaction the use of specific controllers, gestures or gaze/head tracking 
 \cite{sapkaroski2018implementation}, which are likely to break the sense of immersion and presence and are often cumbersome to use.

Among the analyzed VPs, five of them support also \emph{non-verbal} input, by either leveraging NUI-based approaches (e.g., using RGBD sensors, like in MPathic-VR \cite{guetterman2019medical,kron2017using} and Maicher \cite{maicher2017developing}, or standard RGB cameras, like in Dupuy \cite{dupuy2019virtual}) or having a human controller that observes the user interacting with the VP and updates the VP's response accordingly in terms of gestures and facial expressions (like in Banszki \cite{banszki2018clinical,quail2016student}). 
However, apart from Banszki \cite{banszki2018clinical,quail2016student}, it appears that this information is largely underutilized to influence the VP's behavior. In Dupuy \cite{dupuy2019virtual}, the users' facial expressions are detected to merely assess their emotional state at the end of the simulation. In Maicher \cite{maicher2017developing}, the tracked user position is simply used to adjust the agent's gaze, and there is no specific mention of the way the simulation exploits gestures. Finally, in  MPathic-VR \cite{guetterman2019medical,kron2017using}, instead of continuously capturing \emph{non-verbal} communication, learners are forced to assume specific expressions and poses when prompted by the system. In summary, the above discussion highlights that sounder ways of using \emph{non-verbal} inputs are sorely needed in this particular research field.


\textbf{Distribution}. One relevant technological parameter of the VP simulation is the way the application is distributed (and how learners can access it). In principle, there are two main options. The first option is to deploy the VP as a \emph{web-based} application that can be accessed over the Internet. Such a simulation often runs inside a web browser (which makes it device-independent), and generally requires a low amount of computational resources. This flexibility can also foster self-learning (since simulations can be accessed at places and times convenient for the learner) and helps reduce costs (since learning can be carried out online). However, since \emph{web-based} applications are required to be portable on many devices (including mobile ones), they generally sacrifice technical characteristics and computational complexity in favour of accessibility. On the other hand, \emph{standalone} applications are deployed locally on a computer or workstation. These simulations can implement more advanced and complex features since they can leverage the full computational power of a dedicated machine, and integrate external devices or sensors (such as high-quality cameras and microphones).

In our survey, we found a total of ten \emph{web-based} and eight \emph{standalone} applications; in two cases, this information was undisclosed in the paper, whereas in one case (Maicher \cite{maicher2017developing}), the VP was deployed in both variants. This latter work is interesting since it shows how a \emph{standalone} version can trade off some of the flexibility of the web one with a broad array of  features (such as voice control and gesture/posture detection). 
The Authors observed that students were significantly more engaged with the \emph{standalone} VP, whereas in the web version they had to focus on typing and reading, which make them be less prone to notice the subtle non-verbal cues manifested by the patient.

% \edo{addon about effectiveness}
% This doesn't mean that a VP simulation must sacrifice accessibility over technical capabilities at all costs, in fact \cite{richardson2019virtual} mentions user feedback asking for a simulation that is available on multiple platforms (phones, tablets, desktop). So, while a VP that incorporates many interesting technical features may be preferable for maximizing user engagement, it being deployed also in a more accessible but feature-light is a good idea to maximize its potential use cases.
% \edo{end of addon about effectiveness}

Nonetheless, it should be stressed that technology is advancing rapidly, and personal devices come equipped with ever better microphones, cameras and computational power, which can reduce the technological gap between (desktop-only) \emph{standalone} applications and \emph{web-based} ones. Further discussions on this topic are included in Section \ref{sec:openResearch}.  
    

    
\subsection{Effectiveness of design elements}
\label{sec:effectiveness}
The general effectiveness of VPs on developing communication skills has been discussed by several Authors \cite{lee2020effective,peddle2016virtual,richardson2019virtualreview}. A common complaint in VP-related literature is the lack of a standardized terminology that, coupled with a considerable heterogeneity in study design, makes the retrieval and evaluation of relevant works a troublesome task. Despite this situation, both \cite{lee2020effective} and \cite{peddle2016virtual} concluded that, when appropriately contextualized in a well thought out educational context, VPs are indeed useful for developing, practising and building confidence about communication and other skills like, e.g., decision making and teamwork.  

Based on these findings, one possible question arising from our review is if the surveyed papers provide pieces of evidence about the effects on learning outcomes and efficacy of the simulation of the different instructional design elements and the technical features available. 
Unfortunately, the answer is negative. In most of the analyzed works, the Authors reported only users' feedback or comments about a particular element/feature, and a direct comparison between different design choices is missing. The only notable exceptions are three. The first one is represented by \cite{ochs2019training}, in which  different presentation formats were assessed, showing that immersive VR technologies yield superior results when compared to non-immersive ones. The second one concerns the distribution method  \cite{maicher2017developing}. The Authors found that a standalone application can provide a considerably higher level of engagement than its web-based counterpart thanks to the possibility to leverage advanced technical features (voice-controlled input and large volume displays) to increase immersion and focus on the task at hand. The third one compared closed and open-option unfolding designs, highlighting the advantages and disadvantages of each variant \cite{hirumi2016advancing}.
%reporting learners' preferences for the latter.  %Non proprio, hirumi non si sbilancia sull'una o sull'altra, semplicemente evidenzia i vantaggi e gli svantaggi dell'una o dell'altra

% In conclusion, the current literature lacks a thorough evaluation of the effectiveness of alternative designs, and further work has to be done to develop a better understanding of instructional elements and technical features that VP simulations can offer in order to achieve the desired learning outcomes. 
% \andrea{If we have something interesting to add (something like "In our opinion, a relevant contribution to this issue would be ...") we can do it here or in the open research questions}




%%!TEX ROOT = ../../centralized_vs_distributed.tex

\section{{\titlecap{the centralized-distributed trade-off}}}\label{sec:numerical-results}

\revision{In the previous sections we formulated the optimal control problem for a given controller architecture
(\ie the number of links) parametrized by $ n $
and showed how to compute minimum-variance objective function and the corresponding constraints.
In this section, we present our main result:
%\red{for a ring topology with multiple options for the parameter $ n $},
we solve the optimal control problem for each $ n $ and compare the best achievable closed-loop performance with different control architectures.\footnote{
\revision{Recall that small (large) values of $ n $ mean sparse (dense) architectures.}}
For delays that increase linearly with $n$,
\ie $ f(n) \propto n $, 
we demonstrate that distributed controllers with} {few communication links outperform controllers with larger number of communication links.}

\textcolor{subsectioncolor}{Figure~\ref{fig:cont-time-single-int-opt-var}} shows the steady-state variances
obtained with single-integrator dynamics~\eqref{eq:cont-time-single-int-variance-minimization}
%where we compare the standard multi-parameter design 
%with a simplified version \tcb{that utilizes spatially-constant feedback gains
and the quadratic approximation~\eqref{eq:quadratic-approximation} for \revision{ring topology}
with $ N = 50 $ nodes. % and $ n\in\{1,\dots,10\} $.
%with $ N = 50 $, $ f(n) = n $ and $ \tau_{\textit{min}} = 0.1 $.
%\autoref{fig:cont-time-single-int-err} shows the relative error, defined as
%\begin{equation}\label{eq:relative-error}
%	e \doteq \dfrac{\optvarx-\optvar}{\optvar}
%\end{equation}
%where $ \optvar $ and $ \optvarx $ denote the the optimal and sub-optimal scalar variances, respectively.
%The performance gap is small
%and becomes negligible for large $ n $.
{The best performance is achieved for a sparse architecture with  $ n = 2 $ 
in which each agent communicates with the two closest pairs of neighboring nodes. 
This should be compared and contrasted to nearest-neighbor and all-to-all 
communication topologies which induce higher closed-loop variances. 
Thus, 
the advantage of introducing additional communication links diminishes 
beyond}
{a certain threshold because of communication delays.}

%For a linear increase in the delay,
\textcolor{subsectioncolor}{Figure~\ref{fig:cont-time-double-int-opt-var}} shows that the use of approximation~\eqref{eq:cont-time-double-int-min-var-simplified} with $ \tilde{\gvel}^* = 70 $
identifies nearest-neighbor information exchange as the {near-optimal} architecture for a double-integrator model
with ring topology. 
This can be explained by noting that the variance of the process noise $ n(t) $
in the reduced model~\eqref{eq:x-dynamics-1st-order-approximation}
is proportional to $ \nicefrac{1}{\gvel} $ and thereby to $ \taun $,
according to~\eqref{eq:substitutions-4-normalization},
making the variance scale with the delay.

%\mjmargin{i feel that we need to comment about different results that we obtained for CT and DT double-intergrator dynamics (monotonic deterioration of performance for the former and oscillations for the latter)}
\revision{\textcolor{subsectioncolor}{Figures~\ref{fig:disc-time-single-int-opt-var}--\ref{fig:disc-time-double-int-opt-var}}
show the results obtained by solving the optimal control problem for discrete-time dynamics.
%which exhibit similar trade-offs.
The oscillations about the minimum in~\autoref{fig:disc-time-double-int-opt-var}
are compatible with the investigated \tradeoff~\eqref{eq:trade-off}:
in general, 
the sum of two monotone functions does not have a unique local minimum.
Details about discrete-time systems are deferred to~\autoref{sec:disc-time}.
Interestingly,
double integrators with continuous- (\autoref{fig:cont-time-double-int-opt-var}) ad discrete-time (\autoref{fig:disc-time-double-int-opt-var}) dynamics
exhibits very different trade-off curves,
whereby performance monotonically deteriorates for the former and oscillates for the latter.
While a clear interpretation is difficult because there is no explicit expression of the variance as a function of $ n $,
one possible explanation might be the first-order approximation used to compute gains in the continuous-time case.
%which reinforce our thesis exposed in~\autoref{sec:contribution}.

%\begin{figure}
%	\centering
%	\includegraphics[width=.6\linewidth]{cont-time-double-int-opt-var-n}
%	\caption{Steady-state scalar variance for continuous-time double integrators with $ \taun = 0.1n $.
%		Here, the \tradeoff is optimized by nearest-neighbor interaction.
%	}
%	\label{fig:cont-time-double-int-opt-var-lin}
%\end{figure}
}

\begin{figure}
	\centering
	\begin{minipage}[l]{.5\linewidth}
		\centering
		\includegraphics[width=\linewidth]{random-graph}
	\end{minipage}%
	\begin{minipage}[r]{.5\linewidth}
		\centering
		\includegraphics[width=\linewidth]{disc-time-single-int-random-graph-opt-var}
	\end{minipage}
	\caption{Network topology and its optimal {closed-loop} variance.}
	\label{fig:general-graph}
\end{figure}

Finally,
\autoref{fig:general-graph} shows the optimization results for a random graph topology with discrete-time single integrator agents. % with a linear increase in the delay, $ \taun = n $.
Here, $ n $ denotes the number of communication hops in the ``original" network, shown in~\autoref{fig:general-graph}:
as $ n $ increases, each agent can first communicate with its nearest neighbors,
then with its neighbors' neighbors, and so on. For a control architecture that utilizes different feedback gains for each communication link
	(\ie we only require $ K = K^\top $) we demonstrate that, in this case, two communication hops provide optimal closed-loop performance. % of the system.}

Additional computational experiments performed with different rates $ f(\cdot) $ show that the optimal number of links increases for slower rates: 
for example, 
the optimal number of links is larger for $ f(n) = \sqrt{n} $ than for $ f(n) = n $. 
\revision{These results are not reported because of space limitations.}

\section{Open areas of research}
\label{sec:openResearch}

The surveyed papers  show that, despite exciting results obtained, fully understanding how to develop effective VPs for patient-doctor communication training requires further work. Reasons are related to the fact that either the technological components have not been fully explored yet or results are still inadequate to fully assess the effectiveness of different design approaches. Thus, in this section, we briefly discuss some open problems and present areas requiring further research.

\textbf{Assessment of design elements.}
%\label{sec:effectiveness}
% The general effectiveness of VPs on developing communication skills has been discussed in literature by several authors \cite{peddle2016virtual,lee2020effective,richardson2019virtualreview}. A common complaint in VP-related literature is the lack of standardized terminology and considerable heterogeneity in study design that makes retrieval and evaluation of literature a troublesome task. Despite that, both \cite{peddle2016virtual} and \cite{lee2020effective} conclude that, when appropriately contextualized in a well thought out educational context, VPs are indeed useful in developing, practising and building confidence about communication and other skills like decision making and teamwork.  
% Based on these findings, one possible question arising from our review is if the surveyed articles provide pieces of evidence about the effects on learning outcomes and efficacy of the simulation of the different instructional design elements and the technical features available. 
% Unfortunately, the answer is negative. In most of the works analyzed, authors report users' feedback or comments about a particular element/feature only and a direct comparison between different design choices is missing. The only notable exceptions are three. First, \cite{ochs2019training}, which assessed different presentation formats showing that immersive VR technologies yield superior results when compared with non-immersive ones. Second, concerning the distribution method, \cite{maicher2017developing} reports that a standalone application can provide a considerably higher level of engagement than its web-based version, thanks to the possibility of leveraging advanced technical features (voice-controlled input and large volume displays) to increase immersion and focus on the task at hand. Finally, \cite{hirumi2016advancing}, compared closed and open-option unfolding designs, reporting learners' preferences for the latter.  
As discussed in Section \ref{sec:effectiveness}, the current literature lacks a thorough evaluation of the effectiveness of alternative designs. This observation highlights the fact that further work has to be done to develop a better understanding of instructional elements and technical features that VP simulations can offer in order to achieve the desired learning outcomes. 


\textbf{Scope.} Another comment can be made on the specific communication learning context. While several core skill domains jointly contribute to a patient's health and satisfaction (like relationship building, information gathering, patient education, shared decision making and breaking bad news \cite{riedl2017influence}), most of the surveyed VP simulations  focus only on one specific domain. This observation highlights the need to develop novel approaches capable of addressing simultaneously the multiple communication challenges one has to face when interacting with a real patient, thus helping to improve the overall learner's communication skills.


%\andrea{Authoring tools for VPs.}
\textbf{Authoring tools.}
Implementing VPs is a cumbersome and complicated process, which requires taking into account several different elements (NLP, emotion modelling, affective computing, 3D animations, etc.), which, in turn, involve specific technological and technical skills. Usually, the development of a VP is a cyclical process of research, refinement and validation with experts that can take a considerable amount of time \cite{rossen2009human}. Thus, there is the need to develop simple (and effective) authoring tools that can allow developers to support clinical educators in the rapid design, prototyping and deploying of VPs in a variety of use cases. 
Examples of authoring tools for narrative-style VPs with 3D graphics are very scarce in the literature. The work presented in \cite{jeuring2015communicate} integrates a scenario builder that allows clinical educators to design the unfolding of their cases. This authoring tool exploits a domain reasoner where the response of the virtual agent is determined not only by the previous dialogue that the user chose, but also by other parameters like the agent's current emotional state. However, this tool lacks the possibility to customize the virtual environment or the VP's aesthetics.
The NERVE VP \cite{hirumi2016advancingPart2,hirumi2016advancing,kleinsmith2015understanding} is built upon the Virtual People Factory \cite{rossen2009human}, a web application that enables the users to build conversational models using an un-annotated corpus retrieval approach based on keyword matching. 
Another interesting example is SIDNIE (Scaffolded Interviews Developed by Nurses in Education \cite{dukes2016participatory}). This tool allows clinical educators to edit the patient's medical status, dialogue options and physical appearance. However, to our knowledge, SIDNIE has not been deployed in any publicly available form, and  appears to be aimed exclusively at nurse training scenarios.

In other application areas (such as building clinical skills and problem-solving abilities), the extensive use of tools such as DecisionSim, OpenLabyrinth and Web-SP %(a comparative study detailing the usage and characteristics of these tools can be found in 
\cite{doloca2015comparative} is a clear demonstration of the fact that an easy-to-use authoring tool is a determinant factor for the success of a VP application. However, compared to these areas, the specific context of patient-doctor communication training involves more complex systems, with 3D visuals and branched narratives offering a more realistic interaction, which makes the development of authoring tools in this area much more challenging \cite{talbot2012sorting}.\par
%End of Edo's Addon

\textbf{Emerging web technologies.}
%WebXR and Streaming Services}
In the previous sections, we highlighted that personal devices are coming with better and better hardware and computational power, thus helping to narrow the gap between standalone and web-based applications. Another contribution will inevitably come from recent advances in web-based technology, like, e.g., WebXR\footnote{\url{https://www.w3.org/TR/webxr/}}. WebXR is a device-independent framework that allows users to develop and share VR and AR applications over the Internet, with considerable support for different hardware and web browsers. In addition, game-streaming platforms such as Google Stadia\footnote{\url{https://stadia.google.com/}} are a very promising workaround for the limited computational capabilities of personal devices. With these platforms, the bulk of the computation is processed on the server side, then the pre-rendered output is streamed to the final user's device. The implementation of such technological solutions in the immediate future will enable the applications to combine the accessibility of current web-based software with the computational complexity of standalone applications run on a dedicated machine. \par

%edo{mi rendo conto che Virtual Humans è un termine veramente vago, perché può intendere sia NPC che Player Avatar... Non volevo usare NPC perché fa troppo gamer, però qualche articolo scientifico che usa il termine NPC l'ho trovato, quindi userei quello per evitare qualsiasi ambiguità. Riscritta tutta questa sezione}

\textbf{Multiple virtual humans.}
Interacting with a relative or another health care provider are considered crucial aspects of a clinician's communication skills \cite{hallin2011effects,kee2018communication}. However, VP simulations usually include only two actors: the learner (possibly represented by an avatar) and a unique Non-Playable Character (NPC), i.e, a virtual human not controlled by the trainee that represents the patient. The only two examples that include more than one NPC besides the patient are the Medical Interview Episode of the UTTimePortal \cite{zielke2016beyond,zielke2016using} (which incorporates a patient and a caregiver), and MPathic-VR \cite{guetterman2019medical,kron2017using} (which includes a patient's relative and a nurse). 
Beyond this observation, we should also note that another interesting future development (still untouched in the field of VPs for patient-doctor communication skills, to the best of our knowledge) could be to provide the possibility of interacting (within the simulation) with other human-controlled avatars, in a way similar to that proposed by approaches focused on inter-professional communication in emergency medical situations \cite{anbro2020using}. 


\textbf{Immersive VR and AR.}
% \andrea{Stress the relavance of IVR in particular. Immersion and presence contribute to empathic bond with the VP, which in turns have beneficial effects on the learning outcomes. This is why we stress so much these elements}
There is a general understanding among researchers that increasing the level of immersion and realism of the simulations (e.g., using large volume displays, HMDs, spatialized 3D audio, higher fidelity graphics and animations) leads to more believable human-computer interactions \cite{chuah2013exploring,johnsen2008evaluation}, which in turns help improve the users' communication and empathic skills  \cite{ochs2019training,zielke2017developing} and, ultimatley, the learning outcomes in general \cite{limniou2008full}. 
%The several potential advantages of using these technologies, with particular attention on provider-patient communication, have been extensively discussed in \cite{zielke2017developing}. 
However, surprisingly, the use of IVR technologies in this specific context appears to be quite limited. Only two VPs out of \totalVPs, i.e., Ochs \cite{ochs2019training} and CESTOL VR Clinic \cite{sapkaroski2018implementation}, mention the use of IVR, and AR appears to be completely unexplored.  The primary obstacles to the adoption of IVR or AR in VP simulations seem to be the complexity, challenges and costs of development steps \cite{zielke2017developing}.  

Fortunately, things are going to change rapidly. In recent years, the availability and quality of VR devices have increased considerably, and their cost has decreased dramatically.  These factors contribute (together with the availability of high-end development platforms such as Unity or Unreal engine) to reducing overall costs and efforts for developing IVR and AR applications. Furthermore, IVR offers currently a truly immersive, unbroken environment that can shift the cognitive load directed on imagining oneself \quotes{being there} in VR towards solving the task at hand. In turn, higher immersion and visual fidelity can have positive effects on learning \cite{coulter2007effect}, \cite{huerta2012measuring}. Thus, we expect that, soon, VR and AR will contribute to improving the state of the art in this research field.




\textbf{Fully-fledged non-verbal input.}
%Non-Verbal Input should be determinant, not an accessory.}
In our opinion, this is a major lack in current designs. The unfolding of the simulation's narrative should be dictated (in tandem) by both user's verbal and non-verbal behaviours. To this end, developers of future VPs should attempt to fully leverage non-verbal cues as a factor that actively influences the state of the agent. For instance, the same utterance should have a different outcome if the user maintains eye contact with the patient, looks in another direction, and is fidgeting or exhibiting an incoherent facial expression.
The extraction of para-linguistic factors such as tone of voice, loudness, inflection, rhythm, and pitch can provide information about the actual emotional states of the other peer in the communication. Prosody must be addressed with great attention since it is one of the main ways to express empathy and can have a considerable impact in increasing patient satisfaction \cite{kee2018communication}. Thus, computational mechanisms capable of extracting these variables from the analysis of the user's voice are sorely needed. 
The same para-linguistic factors should be also available to modulate the VP response according to its emotional states. In fact, one of the problems with present text-to-speech libraries is that they pronounce everything with the same tone, which makes it impossible to communicate feelings through voice. 


% \vspace{-0.5em}
\section{Conclusion}
% \vspace{-0.5em}
Recent advances in multimodal single-cell technology have enabled the simultaneous profiling of the transcriptome alongside other cellular modalities, leading to an increase in the availability of multimodal single-cell data. In this paper, we present \method{}, a multimodal transformer model for single-cell surface protein abundance from gene expression measurements. We combined the data with prior biological interaction knowledge from the STRING database into a richly connected heterogeneous graph and leveraged the transformer architectures to learn an accurate mapping between gene expression and surface protein abundance. Remarkably, \method{} achieves superior and more stable performance than other baselines on both 2021 and 2022 NeurIPS single-cell datasets.

\noindent\textbf{Future Work.}
% Our work is an extension of the model we implemented in the NeurIPS 2022 competition. 
Our framework of multimodal transformers with the cross-modality heterogeneous graph goes far beyond the specific downstream task of modality prediction, and there are lots of potentials to be further explored. Our graph contains three types of nodes. While the cell embeddings are used for predictions, the remaining protein embeddings and gene embeddings may be further interpreted for other tasks. The similarities between proteins may show data-specific protein-protein relationships, while the attention matrix of the gene transformer may help to identify marker genes of each cell type. Additionally, we may achieve gene interaction prediction using the attention mechanism.
% under adequate regulations. 
% We expect \method{} to be capable of much more than just modality prediction. Note that currently, we fuse information from different transformers with message-passing GNNs. 
To extend more on transformers, a potential next step is implementing cross-attention cross-modalities. Ideally, all three types of nodes, namely genes, proteins, and cells, would be jointly modeled using a large transformer that includes specific regulations for each modality. 

% insight of protein and gene embedding (diff task)

% all in one transformer

% \noindent\textbf{Limitations and future work}
% Despite the noticeable performance improvement by utilizing transformers with the cross-modality heterogeneous graph, there are still bottlenecks in the current settings. To begin with, we noticed that the performance variations of all methods are consistently higher in the ``CITE'' dataset compared to the ``GEX2ADT'' dataset. We hypothesized that the increased variability in ``CITE'' was due to both less number of training samples (43k vs. 66k cells) and a significantly more number of testing samples used (28k vs. 1k cells). One straightforward solution to alleviate the high variation issue is to include more training samples, which is not always possible given the training data availability. Nevertheless, publicly available single-cell datasets have been accumulated over the past decades and are still being collected on an ever-increasing scale. Taking advantage of these large-scale atlases is the key to a more stable and well-performing model, as some of the intra-cell variations could be common across different datasets. For example, reference-based methods are commonly used to identify the cell identity of a single cell, or cell-type compositions of a mixture of cells. (other examples for pretrained, e.g., scbert)


%\noindent\textbf{Future work.}
% Our work is an extension of the model we implemented in the NeurIPS 2022 competition. Now our framework of multimodal transformers with the cross-modality heterogeneous graph goes far beyond the specific downstream task of modality prediction, and there are lots of potentials to be further explored. Our graph contains three types of nodes. while the cell embeddings are used for predictions, the remaining protein embeddings and gene embeddings may be further interpreted for other tasks. The similarities between proteins may show data-specific protein-protein relationships, while the attention matrix of the gene transformer may help to identify marker genes of each cell type. Additionally, we may achieve gene interaction prediction using the attention mechanism under adequate regulations. We expect \method{} to be capable of much more than just modality prediction. Note that currently, we fuse information from different transformers with message-passing GNNs. To extend more on transformers, a potential next step is implementing cross-attention cross-modalities. Ideally, all three types of nodes, namely genes, proteins, and cells, would be jointly modeled using a large transformer that includes specific regulations for each modality. The self-attention within each modality would reconstruct the prior interaction network, while the cross-attention between modalities would be supervised by the data observations. Then, The attention matrix will provide insights into all the internal interactions and cross-relationships. With the linearized transformer, this idea would be both practical and versatile.

% \begin{acks}
% This research is supported by the National Science Foundation (NSF) and Johnson \& Johnson.
% \end{acks}
%\section{Limitations and Future Work}



\label{sec:limitationsandfuture}

%\input{testchapter}


%
% ---- Bibliography ----
%
% BibTeX users should specify bibliography style 'splncs04'.
% References will then be sorted and formatted in the correct style.
%
\bibliographystyle{splncs04}\begin{thebibliography}{10}
\providecommand{\url}[1]{\texttt{#1}}
\providecommand{\urlprefix}{URL }
\providecommand{\doi}[1]{https://doi.org/#1}

\bibitem{adefila2020students}
Adefila, A., Opie, J., Ball, S., Bluteau, P.: Students’ engagement and
  learning experiences using virtual patient simulation in a computer supported
  collaborative learning environment. Innovations in Education and Teaching
  International  \textbf{57}(1),  50--61 (2020)

\bibitem{albright2018using}
Albright, G., Bryan, C., Adam, C., McMillan, J., Shockley, K.: Using virtual
  patient simulations to prepare primary health care professionals to conduct
  substance use and mental health screening and brief intervention. Journal of
  the American Psychiatric Nurses Association  \textbf{24}(3),  247--259 (2018)

\bibitem{anbro2020using}
Anbro, S.J., Szarko, A.J., Houmanfar, R.A., Maraccini, A.M., Crosswell, L.H.,
  Harris, F.C., Rebaleati, M., Starmer, L.: Using virtual simulations to assess
  situational awareness and communication in medical and nursing education: A
  technical feasibility study. Journal of Organizational Behavior Management
  pp. 1--11 (2020)

\bibitem{banszki2018clinical}
B{\'a}nszki, F., Beilby, J., Quail, M., Allen, P., Brundage, S., Spitalnick,
  J.: A clinical educator’s experience using a virtual patient to teach
  communication and interpersonal skills. Australasian Journal of Educational
  Technology  \textbf{34}(3) (2018)

\bibitem{bearman2001random}
Bearman, M., Cesnik, B., Liddell, M.: Random comparison of ‘virtual
  patient’models in the context of teaching clinical communication skills.
  Medical education  \textbf{35}(9),  824--832 (2001)

\bibitem{bearman2015learning}
Bearman, M., Palermo, C., Allen, L.M., Williams, B.: Learning empathy through
  simulation: a systematic literature review. Simulation in healthcare
  \textbf{10}(5),  308--319 (2015)

\bibitem{benedict2013promotion}
Benedict, N., Schonder, K., McGee, J.: Promotion of self-directed learning
  using virtual patient cases. American Journal of Pharmaceutical Education
  \textbf{77}(7) (2013)

\bibitem{bloodworth2010initial}
Bloodworth, T., Cairco, L., McClendon, J., Hodges, L.F., Babu, S., Meehan,
  N.K., Johnson, A., Ulinski, A.C.: Initial evaluation of a virtual pediatric
  patient system  (2010)

\bibitem{carnell2015adapting}
Carnell, S., Halan, S., Crary, M., Madhavan, A., Lok, B.: Adapting virtual
  patient interviews for interviewing skills training of novice healthcare
  students. In: International Conference on Intelligent Virtual Agents. pp.
  50--59. Springer (2015)

\bibitem{chuah2013exploring}
Chuah, J.H., Robb, A., White, C., Wendling, A., Lampotang, S., Kopper, R., Lok,
  B.: Exploring agent physicality and social presence for medical team
  training. Presence: Teleoperators and Virtual Environments  \textbf{22}(2),
  141--170 (2013)

\bibitem{coulter2007effect}
Coulter, R., Saland, L., Caudell, T., Goldsmith, T.E., Alverson, D.: The effect
  of degree of immersion upon learning performance in virtual reality
  simulations for medical education. InMedicine Meets Virtual Reality
  \textbf{15}, ~155 (2007)

\bibitem{dibbelt2010patient}
Dibbelt, S., Schaidhammer, M., Fleischer, C., Greitemann, B.: Patient-doctor
  interaction in rehabilitation: is there a relationship between perceived
  interaction quality and long term treatment results? Die Rehabilitation
  \textbf{49}(5),  315--325 (2010)

\bibitem{doloca2015comparative}
Doloca, A., {\c{T}}{\u{A}}NCULESCU, O., Ciongradi, I., Trandafir, L., Stoleriu,
  S., Ifteni, G.: Comparative study of virtual patient applications. Proc. of
  the Romanian Academy, Series A  \textbf{16}(3),  466--473 (2015)

\bibitem{dukes2016participatory}
Dukes, L.C., Meehan, N., Hodges, L.F.: Participatory design of a pediatric
  virtual patient creation tool. In: 2016 IEEE International Conference on
  Healthcare Informatics (ICHI). pp. 449--455. IEEE (2016)

\bibitem{dupuy2019virtual}
Dupuy, L., Micoulaud-Franchi, J.A., Cassoudesalle, H., Ballot, O., Dehail, P.,
  Aouizerate, B., Cuny, E., de~Sevin, E., Philip, P.: Evaluation of a virtual
  agent to train medical students conducting psychiatric interviews for
  diagnosing major depressive disorders. Journal of Affective Disorders
  \textbf{263}, ~1--8 (2020)

\bibitem{festinger1954theory}
Festinger, L.: A theory of social comparison processes. Human relations
  \textbf{7}(2),  117--140 (1954)

\bibitem{fiscella2004patient}
Fiscella, K., Meldrum, S., Franks, P., Shields, C.G., Duberstein, P., McDaniel,
  S.H., Epstein, R.M.: Patient trust: is it related to patient-centered
  behavior of primary care physicians? Medical care pp. 1049--1055 (2004)

\bibitem{forrest2013essential}
Forrest, K., McKimm, J., Edgar, S.: Essential simulation in clinical education.
  John Wiley \& Sons (2013)

\bibitem{foster2016using}
Foster, A., Chaudhary, N., Kim, T., Waller, J.L., Wong, J., Borish, M., Cordar,
  A., Lok, B., Buckley, P.F.: Using virtual patients to teach empathy: a
  randomized controlled study to enhance medical students’ empathic
  communication. Simulation in Healthcare  \textbf{11}(3),  181--189 (2016)

\bibitem{franks2005patients}
Franks, P., Fiscella, K., Shields, C.G., Meldrum, S.C., Duberstein, P., Jerant,
  A.F., Tancredi, D.J., Epstein, R.M.: Are patients’ ratings of their
  physicians related to health outcomes? The Annals of Family Medicine
  \textbf{3}(3),  229--234 (2005)

\bibitem{guetterman2019medical}
Guetterman, T.C., Sakakibara, R., Baireddy, S., Kron, F.W., Scerbo, M.W.,
  Cleary, J.F., Fetters, M.D.: Medical students’ experiences and outcomes
  using a virtual human simulation to improve communication skills: Mixed
  methods study. Journal of medical Internet research  \textbf{21}(11),  e15459
  (2019)

\bibitem{Longnecker2010}
Ha, J.F., Longnecker, N.: {{D}octor-patient communication: a review}. Ochsner J
   \textbf{10}(1),  38--43 (2010)

\bibitem{hallin2011effects}
Hallin, K., Henriksson, P., Dal{\'e}n, N., Kiessling, A.: Effects of
  interprofessional education on patient perceived quality of care. Medical
  Teacher  \textbf{33}(1),  e22--e26 (2011)

\bibitem{hickson2002patient}
Hickson, G.B., Federspiel, C.F., Pichert, J.W., Miller, C.S., Gauld-Jaeger, J.,
  Bost, P.: Patient complaints and malpractice risk. Jama  \textbf{287}(22),
  2951--2957 (2002)

\bibitem{hirumi2016advancingPart2}
Hirumi, A., Johnson, T., Reyes, R.J., Lok, B., Johnsen, K., Rivera-Gutierrez,
  D.J., Bogert, K., Kubovec, S., Eakins, M., Kleinsmith, A., et~al.: Advancing
  virtual patient simulations through design research and interplay: part
  ii—integration and field test. Educational technology research and
  development  \textbf{64}(6),  1301--1335 (2016)

\bibitem{hirumi2016advancing}
Hirumi, A., Kleinsmith, A., Johnsen, K., Kubovec, S., Eakins, M., Bogert, K.,
  Rivera-Gutierrez, D.J., Reyes, R.J., Lok, B., Cendan, J.: Advancing virtual
  patient simulations through design research and interplay: part i: design and
  development. Educational Technology Research and Development  \textbf{64}(4),
   763--785 (2016)

\bibitem{huerta2012measuring}
Huerta, R.: Measuring the impact of narrative on player's presence and
  immersion in a first person game environment. The University of Texas-Pan
  American (2012)

\bibitem{jacklin2019virtual}
Jacklin, S., Chapman, S., Maskrey, N.: Virtual patient educational intervention
  for the development of shared decision-making skills: a pilot study. BMJ
  Simulation and Technology Enhanced Learning  \textbf{5}(4),  215--217 (2019)

\bibitem{jacklin2018improving}
Jacklin, S., Maskrey, N., Chapman, S.: Improving shared decision making between
  patients and clinicians: Design and development of a virtual patient
  simulation tool. JMIR medical education  \textbf{4}(2),  e10088 (2018)

\bibitem{janda2004simulation}
Janda, M.S., Mattheos, N., Nattestad, A., Wagner, A., Nebel, D., F{\"a}rbom,
  C., L{\^e}, D.H., Attstr{\"o}m, R.: Simulation of patient encounters using a
  virtual patient in periodontology instruction of dental students: design,
  usability, and learning effect in history-taking skills. European Journal of
  Dental Education  \textbf{8}(3),  111--119 (2004)

\bibitem{jeuring2015communicate}
Jeuring, J., Grosfeld, F., Heeren, B., Hulsbergen, M., IJntema, R., Jonker, V.,
  Mastenbroek, N., van~der Smagt, M., Wijmans, F., Wolters, M., et~al.:
  Communicate!—a serious game for communication skills—. In: Design for
  teaching and learning in a networked world, pp. 513--517. Springer (2015)

\bibitem{johnsen2008evaluation}
Johnsen, K., Lok, B.: An evaluation of immersive displays for virtual human
  experiences. In: 2008 IEEE virtual reality conference. pp. 133--136. IEEE
  (2008)

\bibitem{judge2004affect}
Judge, T.A., Ilies, R.: Affect and job satisfaction: a study of their
  relationship at work and at home. Journal of applied psychology
  \textbf{89}(4), ~661 (2004)

\bibitem{kee2018communication}
Kee, J.W., Khoo, H.S., Lim, I., Koh, M.Y.: Communication skills in
  patient-doctor interactions: learning from patient complaints. Health
  Professions Education  \textbf{4}(2),  97--106 (2018)

\bibitem{kelley2009patient}
Kelley, J.M., Lembo, A.J., Ablon, J.S., Villanueva, J.J., Conboy, L.A., Levy,
  R., Marci, C.D., Kerr, C., Kirsch, I., Jacobson, E.E., et~al.: Patient and
  practitioner influences on the placebo effect in irritable bowel syndrome.
  Psychosomatic medicine  \textbf{71}(7), ~789 (2009)

\bibitem{King2013}
King, A., Hoppe, R.B.: {\textquotedblleft}best practice{\textquotedblright} for
  patient-centered communication: A narrative review. Journal of Graduate
  Medical Education  \textbf{5}(3),  385--393 (Sep 2013).
  \doi{10.4300/jgme-d-13-00072.1},
  \url{https://doi.org/10.4300/jgme-d-13-00072.1}

\bibitem{kleinsmith2015understanding}
Kleinsmith, A., Rivera-Gutierrez, D., Finney, G., Cendan, J., Lok, B.:
  Understanding empathy training with virtual patients. Computers in human
  behavior  \textbf{52},  151--158 (2015)

\bibitem{kneebone2006human}
Kneebone, R., Nestel, D., Wetzel, C., Black, S., Jacklin, R., Aggarwal, R.,
  Yadollahi, F., Wolfe, J., Vincent, C., Darzi, A.: The human face of
  simulation: patient-focused simulation training. Academic Medicine
  \textbf{81}(10),  919--924 (2006)

\bibitem{kohatsu2004characteristics}
Kohatsu, N.D., Gould, D., Ross, L.K., Fox, P.J.: Characteristics associated
  with physician discipline: a case-control study. Archives of internal
  medicine  \textbf{164}(6),  653--658 (2004)

\bibitem{kron2017using}
Kron, F.W., Fetters, M.D., Scerbo, M.W., White, C.B., Lypson, M.L., Padilla,
  M.A., Gliva-McConvey, G.A., Belfore~II, L.A., West, T., Wallace, A.M.,
  et~al.: Using a computer simulation for teaching communication skills: A
  blinded multisite mixed methods randomized controlled trial. Patient
  education and counseling  \textbf{100}(4),  748--759 (2017)

\bibitem{lee2020effective}
Lee, J., Kim, H., Kim, K.H., Jung, D., Jowsey, T., Webster, C.: Effective
  virtual patient simulators for medical communication training: a systematic
  review. Medical Education  (2020)

\bibitem{limniou2008full}
Limniou, M., Roberts, D., Papadopoulos, N.: Full immersive virtual environment
  cavetm in chemistry education. Computers \& Education  \textbf{51}(2),
  584--593 (2008)

\bibitem{maicher2017developing}
Maicher, K., Danforth, D., Price, A., Zimmerman, L., Wilcox, B., Liston, B.,
  Cronau, H., Belknap, L., Ledford, C., Way, D., et~al.: Developing a
  conversational virtual standardized patient to enable students to practice
  history-taking skills. Simulation in Healthcare  \textbf{12}(2),  124--131
  (2017)

\bibitem{marei2018use}
Marei, H., Al-Eraky, M., Almasoud, N., Donkers, J., Van~Merrienboer, J.: The
  use of virtual patient scenarios as a vehicle for teaching professionalism.
  European Journal of Dental Education  \textbf{22}(2),  e253--e260 (2018)

\bibitem{mccoy2016evaluating}
McCoy, L., Pettit, R.K., Lewis, J.H., Allgood, J.A., Bay, C., Schwartz, F.N.:
  Evaluating medical student engagement during virtual patient simulations: a
  sequential, mixed methods study. BMC medical education  \textbf{16}(1), ~20
  (2016)

\bibitem{Nestel2011}
Nestel, D., Tabak, D., Tierney, T., Layat-Burn, C., Robb, A., Clark, S.,
  Morrison, T., Jones, N., Ellis, R., Smith, C., McNaughton, N., Knickle, K.,
  Higham, J., Kneebone, R.: Key challenges in simulated patient programs: An
  international comparative case study. {BMC} Medical Education  \textbf{11}(1)
  (Sep 2011). \doi{10.1186/1472-6920-11-69},
  \url{https://doi.org/10.1186/1472-6920-11-69}

\bibitem{ochs2019training}
Ochs, M., Mestre, D., De~Montcheuil, G., Pergandi, J.M., Saubesty, J.,
  Lombardo, E., Francon, D., Blache, P.: Training doctors’ social skills to
  break bad news: evaluation of the impact of virtual environment displays on
  the sense of presence. Journal on Multimodal User Interfaces  \textbf{13}(1),
   41--51 (2019)

\bibitem{o2019suicide}
O’Brien, K.H.M., Fuxman, S., Humm, L., Tirone, N., Pires, W.J., Cole, A.,
  Grumet, J.G.: Suicide risk assessment training using an online virtual
  patient simulation. Mhealth  \textbf{5} (2019)

\bibitem{papadakis2005disciplinary}
Papadakis, M.A., Teherani, A., Banach, M.A., Knettler, T.R., Rattner, S.L.,
  Stern, D.T., Veloski, J.J., Hodgson, C.S.: Disciplinary action by medical
  boards and prior behavior in medical school. New England Journal of Medicine
  \textbf{353}(25),  2673--2682 (2005)

\bibitem{peddle2019exploring}
Peddle, M., Bearman, M., Mckenna, L., Nestel, D.: Exploring undergraduate
  nursing student interactions with virtual patients to develop
  ‘non-technical skills’ through case study methodology. Advances in
  Simulation  \textbf{4}(1), ~2 (2019)

\bibitem{peddle2016virtual}
Peddle, M., Bearman, M., Nestel, D.: Virtual patients and nontechnical skills
  in undergraduate health professional education: an integrative review.
  Clinical Simulation in Nursing  \textbf{12}(9),  400--410 (2016)

\bibitem{peddle2019development}
Peddle, M., Mckenna, L., Bearman, M., Nestel, D.: Development of non-technical
  skills through virtual patients for undergraduate nursing students: an
  exploratory study. Nurse education today  \textbf{73},  94--101 (2019)

\bibitem{quail2016student}
Quail, M., Brundage, S.B., Spitalnick, J., Allen, P.J., Beilby, J.: Student
  self-reported communication skills, knowledge and confidence across
  standardised patient, virtual and traditional clinical learning environments.
  BMC medical education  \textbf{16}(1), ~73 (2016)

\bibitem{richardson2019virtual}
Richardson, C.L., Chapman, S., White, S.: Virtual patient educational programme
  to teach counselling to clinical pharmacists: development and proof of
  concept. BMJ Simulation and Technology Enhanced Learning  \textbf{5}(3),
  167--169 (2019)

\bibitem{richardson2019virtualreview}
Richardson, C.L., White, S., Chapman, S.: Virtual patient technology to educate
  pharmacists and pharmacy students on patient communication: a systematic
  review. BMJ Simulation and Technology Enhanced Learning  (2019)

\bibitem{riedl2017influence}
Riedl, D., Sch{\"u}{\ss}ler, G.: The influence of doctor-patient communication
  on health outcomes: a systematic review. Zeitschrift f{\"u}r Psychosomatische
  Medizin und Psychotherapie  \textbf{63}(2),  131--150 (2017)

\bibitem{rogers2011developing}
Rogers, L.: Developing simulations in multi-user virtual environments to
  enhance healthcare education. British Journal of Educational Technology
  \textbf{42}(4),  608--615 (2011)

\bibitem{rossen2009human}
Rossen, B., Lind, S., Lok, B.: Human-centered distributed conversational
  modeling: Efficient modeling of robust virtual human conversations. In:
  International Workshop on Intelligent Virtual Agents. pp. 474--481. Springer
  (2009)

\bibitem{sapkaroski2018implementation}
Sapkaroski, D., Baird, M., McInerney, J., Dimmock, M.R.: The implementation of
  a haptic feedback virtual reality simulation clinic with dynamic patient
  interaction and communication for medical imaging students. Journal of
  medical radiation sciences  \textbf{65}(3),  218--225 (2018)

\bibitem{schoenthaler2017simulated}
Schoenthaler, A., Albright, G., Hibbard, J., Goldman, R.: Simulated
  conversations with virtual humans to improve patient-provider communication
  and reduce unnecessary prescriptions for antibiotics: a repeated measure
  pilot study. JMIR medical education  \textbf{3}(1), ~e7 (2017)

\bibitem{stelfox2005relation}
Stelfox, H.T., Gandhi, T.K., Orav, E.J., Gustafson, M.L.: The relation of
  patient satisfaction with complaints against physicians and malpractice
  lawsuits. The American journal of medicine  \textbf{118}(10),  1126--1133
  (2005)

\bibitem{stewart1995effective}
Stewart, A.M.: Effective physician-patient communication and health outcomes: a
  review. CMAJ : Canadian Medical Association journal = journal de
  l'Association medicale canadienne pp. 1423--1433 (1995)

\bibitem{szilas2019virtual}
Szilas, N., Chauveau, L., Andkjaer, K., Luiu, A.L., B{\'e}trancourt, M.,
  Ehrler, F.: Virtual patient interaction via communicative acts. In:
  Proceedings of the 19th ACM International Conference on Intelligent Virtual
  Agents. pp. 91--93 (2019)

\bibitem{talbot2012sorting}
Talbot, T.B., Sagae, K., John, B., Rizzo, A.A.: Sorting out the virtual
  patient: how to exploit artificial intelligence, game technology and sound
  educational practices to create engaging role-playing simulations.
  International Journal of Gaming and Computer-Mediated Simulations (IJGCMS)
  \textbf{4}(3),  1--19 (2012)

\bibitem{urresti2017virtual}
Urresti-Gundlach, M., Tolks, D., Kiessling, C., Wagner-Menghin, M., H{\"a}rtl,
  A., Hege, I.: Do virtual patients prepare medical students for the real
  world? development and application of a framework to compare a virtual
  patient collection with population data. BMC medical education
  \textbf{17}(1), ~174 (2017)

\bibitem{washburn2020virtual}
Washburn, M., Parrish, D.E., Bordnick, P.S.: Virtual patient simulations for
  brief assessment of mental health disorders in integrated care settings.
  Social Work in Mental Health  \textbf{18}(2),  121--148 (2020)

\bibitem{zielke2017developing}
Zielke, M.A., Zakhidov, D., Hardee, G., Evans, L., Lenox, S., Orr, N., Fino,
  D., Mathialagan, G.: Developing virtual patients with vr/ar for a natural
  user interface in medical teaching. In: 2017 IEEE 5th International
  Conference on Serious Games and Applications for Health (SeGAH). pp.~1--8.
  IEEE (2017)

\bibitem{zielke2016beyond}
Zielke, M.A., Zakhidov, D., Jacob, D., Hardee, G.: Beyond fun and games: toward
  an adaptive and emergent learning platform for pre-med students with the ut
  time portal. In: 2016 IEEE International Conference on Serious Games and
  Applications for Health (SeGAH). pp.~1--8. IEEE (2016)

\bibitem{zielke2016using}
Zielke, M.A., Zakhidov, D., Jacob, D., Lenox, S.: Using qualitative data
  analysis to measure user experience in a serious game for premed students.
  In: International Conference on Virtual, Augmented and Mixed Reality. pp.
  92--103. Springer (2016)

\bibitem{zlotos2016scenario}
Zlotos, L., Power, A., Hill, D., Chapman, P.: A scenario-based virtual patient
  program to support substance misuse education. American journal of
  pharmaceutical education  \textbf{80}(3) (2016)

\end{thebibliography}
%

\end{document}
