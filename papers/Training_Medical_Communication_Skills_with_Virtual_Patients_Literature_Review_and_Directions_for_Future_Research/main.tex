% This is samplepaper.tex, a sample chapter demonstrating the
% LLNCS macro package for Springer Computer Science proceedings;
% Version 2.20 of 2017/10/04
%
\documentclass[runningheads]{llncs}
%
\usepackage{graphicx}
\usepackage[figuresright]{rotating}
\usepackage[table]{xcolor}
\definecolor{mywhite}{RGB}{255, 255, 255}
\definecolor{mygray}{RGB}{240, 240, 240}
\usepackage{tabularx}
\usepackage{supertabular}
\usepackage{longtable}
\usepackage{lscape}
\usepackage{ltablex}
\usepackage{ctable}
\usepackage{multirow}

\usepackage{hyperref}
\hypersetup{
    colorlinks=false,
    linkcolor=blue,
    filecolor=magenta,      
    urlcolor=cyan,
}

%setup per caption delle tabelle sopra alle tabelle stesse
\usepackage{floatrow}
\floatsetup[table]{capposition = top}

\urlstyle{same}


% Used for displaying a sample figure. If possible, figure files should
% be included in EPS format.
%
% If you use the hyperref package, please uncomment the following line
% to display URLs in blue roman font according to Springer's eBook style:
% \renewcommand\UrlFont{\color{blue}\rmfamily}

%%% extra latex commands
\usepackage{xcolor} 
\usepackage{soul} 
\usepackage{xspace}
\usepackage{amsmath}
\usepackage{multirow}
\usepackage{capt-of}
\usepackage{floatrow}
% Table float box with bottom caption, box width adjusted to content
%\newfloatcommand{capbtabbox}{table}[][\FBwidth]

\usepackage{hyphenat}
\hyphenation{re-pro-du-ci-bi-li-ty}
\hyphenation{com-pu-ter-ba-sed}

\newcommand{\totalArticles}{28\xspace} 

\newcommand{\totalVPs}{21\xspace} 

\newcommand{\quotes}[1]{``#1''} 

\DeclareRobustCommand{\edo}[1]
{{\sethlcolor{cyan}\hl{#1}}} 

\DeclareRobustCommand{\andrea}[1]
{{\sethlcolor{orange}\hl{#1}}} 

\DeclareRobustCommand{\francesco}[1]
{{\sethlcolor{green}\hl{#1}}} 

\begin{document}
%
\title{Training Medical Communication Skills with Virtual Patients: Literature Review and Directions for Future Research}
%
%
\titlerunning{Training Medical Communication Skills with VPs}
% If the paper title is too long for the running head, you can set
% an abbreviated paper title here
%
%Comment names for double blind rev

\author{Edoardo Battegazzorre\orcidID{0000-0002-1383-3285} \and
Andrea Bottino\orcidID{0000-0002-8894-5089} \and
Fabrizio Lamberti\orcidID{0000-0001-7703-1372}}

%\author{Authors omitted for double blind review}
%
%\authorrunning{F. Author et al.}
% First names are abbreviated in the running head.
% If there are more than two authors, 'et al.' is used.
%

%Comment institution for double blind
\institute{DAUIN, Politecnico di Torino, Corso Duca degli Abruzzi 24, 10143 Torino, Italy %\\
\email{\{edoardo.battegazzorre, andrea.bottino, fabrizio.lamberti\}@polito.it}}
%\institute{Institute omitted for double blind review\\
%\email{emails omitted for double blind review}}

%
\maketitle              % typeset the header of the contribution
%
\begin{abstract}
\label{sec:abstract}

%% 1. what is the problem 
Scientific applications that run on leadership computing facilities often face the challenge 
of being unable to fit leading science cases onto accelerator devices due to memory constraints 
(memory-bound applications).
%
% 2. what is your solution 
In this work, the authors studied one such US Department of Energy mission-critical condensed matter 
physics application, Dynamical Cluster Approximation (DCA++), and this paper discusses how device memory-bound challenges were successfully reduced  by proposing an effective 
``all-to-all'' communication method---a ring communication algorithm. 
%
This implementation takes advantage of acceleration on GPUs and remote direct memory access (RDMA) for fast data exchange between GPUs. 
%
\\Additionally, the ring algorithm was optimized with sub-ring communicators
and multi-threaded support to further reduce communication overhead and 
expose more concurrency, respectively.
%
% 3. What's the cherry-picked evaluation result you want to mention
The computation and communication were also analyzed 
by using the Autonomic Performance Environment for Exascale 
(APEX) profiling tool,  and this paper further discusses the 
performance trade-off for the ring algorithm implementation. 
%
The memory analysis on the ring algorithm shows that the allocation size for the authors' most 
memory-intensive data structure per GPU is now reduced to $1/p$ of the original size, where $p$ is the number of GPUs in the ring communicator.
%
The communication analysis suggests that 
the distributed Quantum Monte Carlo execution time grows linearly as sub-ring size increases, and the cost of messages passing through the network interface connector could be a limiting factor.


%
% \todoRed{Ronnie: Next sentence needs rewrite, too much information about Green's function that no one knows in the abstract; recommend generalizing.} \emph {However, DCA++ is currently facing memory-bound challenge as 
% a larger device array $G_t$ is limited by device memory size, where
% $G_t$ is a two-particle Green's function that allows condensed matter
% scientists to explore larger and more complex (higher fidelity)
% physics cases.}

\end{abstract}

\keywords{DCA++, Quantum Monte Carlo, GPU Remote Direct Memory Access, memory-bound issue, exascale machines}

%
%
%

\section{Introduction}  \label{sec:introduction}

\newcommand\inexpIntro[3]{#1?(#2,#3).}
\newcommand\rinexpIntro[3]{*#1?(#2,#3).}
\newcommand\outexpIntro[3]{#1!(#2,#3).}
\newcommand\outatomIntro[3]{#1!(#2,#3)}

We propose a fully automated method for proving termination of \(\pi\)-calculus processes.
Although there have been a lot of studies on termination analysis for the \(\pi\)-calculus
and related calculi~\cite{Deng06IC,Demangeon07,SangiorgiTermination,KobayashiHybrid,Yoshida04IC,DBLP:journals/jlp/DemangeonHS10,Venet98SAS}, most of them have been rather theoretical,
and there have been surprisingly little efforts in developing  fully automated termination
verification methods and tools based on them. To our knowledge,
Kobayashi's \typical{}~\cite{TyPiCal,KobayashiHybrid} is the only exception that
can prove termination of \(\pi\)-calculus processes (extended with natural numbers)
fully automatically, but its termination analysis is quite limited (see Section~\ref{sec:relatedwork}).

Our method is based on a reduction to termination analysis for sequential programs:
we translate a \(\pi\)-calculus process \(P\) to a sequential program \(S_P\), so that
if \(S_P\) is terminating, so is \(P\). The reduction allows us to use
powerful, mature methods and tools
for termination analysis of sequential programs~\cite{heizmann2016ultimate,freqterm,DBLP:conf/lics/PodelskiR04,Kuwahara2014Termination,DBLP:journals/cacm/CookPR11}.

The idea of the translation is to convert a chain of communications on replicated input
channels to a chain of recursive function calls of the target sequential program.
Let us consider the following Fibonacci process:
\begin{align*}
    & \rinexpIntro{\fib}{n}{r}
        \ifexp{n<2}{ \soutatom{r}{1} \\ &\quad}
                   { \nuexp{s_1} \nuexp{s_2} (\outatomIntro{\fib}{n-1}{s_1} \PAR \outatomIntro{\fib}{n-2}{s_2} \PAR \sinexp{s_1}{x}\sinexp{s_2}{y}\soutatom{r}{x+y}) \\}
    & \PAR \outatomIntro{\fib}{m}{r}
\end{align*}
Here, the process
$\rinexpIntro{\fib}{n}{r} \ldots$ is a function server that computes the \(n\)-th Fibonacci number
in parallel and returns the result to \(r\),
and $\outatom{\fib}{m}{r}$ sends a request for computing the \(m\)-th Fibonacci number;
those who are not familiar with the syntax of the \(\pi\)-calculus may wish to consult
Section~\ref{sec:targetlanguage} first.
To prove that the process above is terminating for any integer \(m\),
it suffices to show that there is no infinite chain of communications on $\fib$:
\[
    \fib(m,r) \to \fib(m_1,r_1) \to \fib(m_2,r_2) \to \cdots.
\]
We convert the process above to the following program:\footnote{The actual translation
  given later is a little more complex.}
\begin{verbatim}
 let rec fib(n) = if n<2 then () else (fib(n-1) [] fib(n-2)) in
 fib(m)
\end{verbatim}
Here, \texttt{[]} represents the non-deterministic choice.
Note that, although the calculation of Fibonacci numbers is not preserved,
for each chain of communications on \texttt{fib}, there is a corresponding
sequence of recursive calls:
\[
\mathtt{fib}(m) \to \mathtt{fib}(m_1) \to \mathtt{fib}(m_2) \to \cdots.
\]
Thus, the termination of the sequential program above implies the termination of
the original process.
As shown in the example above, (i) each communication on a replicated input channel
is converted to a function call, (ii) each communication on a non-replicated input
channel is just removed (or, in the actual translation, replaced by a call of
a trivial function defined by \(f(\seq{x})=(\,)\)), and (iii) parallel composition
is replaced by a non-deterministic choice.
We formalize the translation outlined above and prove its correctness.

The basic translation sketched above sometimes loses too much information.
For example, consider the following process:
\begin{align*}
    & \rinexpIntro{\pre}{n}{r} \soutatom{r}{n-1} \\
    & \PAR \rinexpIntro{f}{n}{r} \ifexp{n<0}{ \soutatom{r}{1} }
                                       { \nuexp{s} (\outatomIntro{\pre}{n}{s} \PAR \sinexp{s}{x}\outatomIntro{f}{x}{r}) } \\
    & \PAR \outatomIntro{f}{m}{r}
\end{align*}
The translation sketched above would yield:
\begin{verbatim}
  let pred(n) = n-1 in
  let rec f(n) = if n<0 then () else (pred(n) [] f(*)) in
  f(m)
\end{verbatim}
Here, \texttt{*} represents a non-deterministic integer: since we have removed
the input $\sinatom{s}{x}$, we do not have information about the value of \( x \).
As a result, the sequential program above is non-terminating, although the original
process is terminating.
To remedy this problem, we also refine the basic translation above by using a refinement
type system for the \(\pi\)-calculus. Using the refinement type system,
we can infer that the value of \(x\) in the original process is less than \(n\),
so that we can refine the definition of \texttt{f} to:
\begin{verbatim}
 let rec f(n) = ... else (pred(n) [] let x=* in assume(x<n);f(x))
\end{verbatim}
The target program is now terminating, from which
we can deduce that the original process is also terminating.
We have implemented an automated tool based on the refined translation above.

The contributions of this paper are summarized as follows.
\begin{itemize}
\item The formalization of the basic translation from the \(\pi\)-calculus
  (extended with integers) to sequential programs, and a proof of its correctness.
\item The formalization of a refined translation based on a refinement type system.
\item An implementation of the refined translation, including automated refinement type
  inference based on CHC solving, and experiments to evaluate the effectiveness of
  our method.
\end{itemize}

The rest of this paper is structured as follows.
Section~\ref{sec:targetlanguage} introduces the source and target languages
of our translation.
Section~\ref{sec:approach} 
formalizes the basic translation, and proves its correctness.
Section~\ref{sec:refinement} refines the basic translation by using a refinement type system.
Section~\ref{sec:implementation} reports an implementation and experiments.
Section~\ref{sec:relatedwork} discusses related work,
and Section~\ref{sec:conclusion} concludes the paper.

\section{Literature review protocol}
\label{sec:reviewProtocol}


As illustrated in the previous section, we performed a literature review to document the current state-of-the-art of the use of VPs for medical communication skill learning and to identify possible areas where further research is needed. The purpose of this review was to understand the instructional and technical design principles and the efficacy of these elements in achieving the expected learning outcomes.
To this end, we developed the following guiding questions in order to help focusing information extraction.

\begin{itemize}
\item RQ1: What are the latest technical developments in the field of communica-tion-oriented VPs? 

\item RQ2: Which instructional and technical design features are employed most commonly in VP design?

\item RQ3: Which instructional and technical VP design features are more effective for learning communication skills? And which of these features are most appreciated by the users?

%\item RQ4: In the field of communication-oriented VPs, what are possible open research areas that need to be further explored?

\end{itemize}

The search process, carried out mainly between March and May 2020, started with an automated approach targeting four scientific paper databases, namely Scopus, PubMed, ACM Digital Library and IEEE Xplore. For each database we performed a search based on the main and derivative keywords (virtual patient OR (serious game AND healthcare)) AND (communication), limiting the results to papers published from 2015 onward. The choice of this date was made with the aim to survey only the most recent developments in the field and avoid excessive overlaps with previous literature reviews (e.g., with \cite{peddle2016virtual} and \cite{lee2020effective}). 

The papers found were post-processed in order to remove repeated entries and exclude reviews, editorials, abstracts, posters and panel discussions. The remaining 306 papers were analyzed by reading over their title, abstract, and introduction, and classified as either relevant or irrelevant based on the following criteria: (i) does the study relate to any of the design elements of interest (instructional or technical)? and, (ii) does the study disclose at least some of the design choices made by the Authors? If the answer to any of these questions was no, then the paper was excluded. After this step, each of the 70 accepted papers was read completely by at least one reviewer, who also assessed its quality. Its references were also analyzed according to the aforementioned screening process. 

At the end of the search process, we selected a total of \totalArticles papers. Among them, we identified a number of works that referred to the same VP, but in different experimental settings or in different phases of the development process. Since our interest was in analyzing the VP design rather than the detailed outcomes of possible experiments, papers sharing the same VP were grouped together, obtaining a total of \totalVPs VPs (17 of them had not been discussed in previous surveys, and only four of them were in common with \cite{lee2020effective}, namely Banszki \cite{banszki2018clinical,quail2016student}, CynthiaYoungVP \cite{foster2016using}, MPathic-VR \cite{guetterman2019medical,kron2017using}, and NERVE \cite{hirumi2016advancingPart2,hirumi2016advancing,kleinsmith2015understanding}). %Thus, in the following, when referring a specific VP, we will also list the articles that discuss it.

In order to capture the main characteristics of problems and solutions discussed in these papers, we introduced a taxonomy of terms for the instructional and technical design elements, whose initial version was defined based on the Authors’ expertise. Based on intermediate findings, this taxonomy was further refined into the final one introduced in Section~\ref{sec:taxonomy}. All the Authors categorized the selected VPs according to this taxonomy, and any disagreement was solved by discussion.
Finally, as a last step, we searched for references related to the open problems and potential areas of research identified during the analysis. 














% \andrea{to be done [EDO]}
% \edo{draft:}

% %In this review we focus only on VPs with narrative elements, including articles form 2015 to 2020, to assess the current state of technologies in this field and their possible future developments.
% %detto qua sembra che il fatto che sia narrative sia un search criteria, che presumo dovremmo giustificare, ma non saprei bene come. Il fatto che siano tutti narrative in realtà è incidentale, perché un VP PS basato sulla comunicazione, tecnicamente, potrebbe anche esistere.

% %Preliminary search -> objective: define classification criteria
% A preliminary search that included all types of studies and reviews regarding Virtual Patients was conducted to establish the appropriate classification criteria for design elements, to later meaningfully analyze the literature. After a number of iterations, we extrapolated 7 main categories, divided into two domains: 
% %Instructional Design, (Structure, Unfolding, Feedback and Gamification) and Technical Design (Presentation Format, Input Interface and Distribution). This preparatory phase provided an analytical framework that we later used as an inclusion criterion for studies.

% %Actual search -> include only articles that could be fully categorized according to the classification criteria we established
% The actual search process was carried out mainly between March and May 2020, targeting Scopus, PubMed (IEEE Explore, ACMDigital) and XYZ scientific databases. We carried out the search based on the main and derivative keywords (virtual patient OR simulation OR virtual reality OR serious game) AND/OR healthcare AND (communication OR nontechnical skills OR (doctor-patient communication OR provider-patient communication)), limiting the results only to articles published from 2015 onward with the objective of surveying only the most recent developments in this field and trying not to overlap results with previous literature reviews \cite{peddle2016virtual}. After this step, we performed a preliminary screening on the 131 found articles based on title and abstract, eliminating all articles that were not directly related to both VPs AND provider-patient communication. The remaining X results underwent a further full-text review by at least one author, and we removed all articles that did not provide enough information to be fully classified according to the categorization criteria we established during the preparatory phase. Reviews, editorials, abstracts, posters. 
% %se diciamo così però rimane il problema di quei maledetti due che sono undisclosed per quanto riguarda la distribuzione :(... ricontrolla edo. u
% %SOLVED: Marei è standalone.
% %jeuring communicate: l'authoring tool è di sicuro web-based (trovato in un altro articolo) l'applicazione in sé è molto poco chiaro
% %shoenthaler... molto poco chiaro (costruita con la Kognito Conversation Plaform, che può essere deployata sia standalone che web-based...)

% Finally, during the full-text review, we also searched for references reported in these articles for additional literature that could offer additional cues relevant to our research objectives. 
% At the end of the search process, a total of \totalArticles were found to be compliant with our criteria however, we identified a number of studies that used the same VP, but in different experimental settings. Since our interest lies in analyzing the VP design and not the study design, articles sharing the same VP were grouped together, obtaining a total of \totalVPs VPs. 

% %\edo{NOTA:} \cite{jacklin2019virtual} \edo{è uno 'Short-Report'. E' degno di inclusione? in ogni caso il VP non lo perdiamo, perché e' uno di quelli che ha due articoli. Io al netto di tutto lo terrei. Ma comunque chiedo un parere esterno.}
% %\cite{richardson2019virtual} \edo{anche questo è uno 'short-report'}

% %Included both experimental and non-experimental studies (qualitative and quantitative)

\tikzstyle{my-box}=[
    rectangle,
    draw=hidden-draw,
    rounded corners,
    text opacity=1,
    minimum height=1.5em,
    minimum width=5em,
    inner sep=2pt,
    align=center,
    fill opacity=.5,
    line width=0.8pt,
]
\tikzstyle{leaf}=[my-box, minimum height=1.5em,
    fill=hidden-pink!80, text=black, align=left,font=\normalsize,
    inner xsep=2pt,
    inner ysep=4pt,
    line width=0.8pt,
]
\begin{figure*}[t!]
    \centering
    \resizebox{\textwidth}{!}{
        \begin{forest}
            forked edges,
            for tree={
                grow=east,
                reversed=true,
                anchor=base west,
                parent anchor=east,
                child anchor=west,
                base=left,
                font=\large,
                rectangle,
                draw=hidden-draw,
                rounded corners,
                align=left,
                minimum width=4em,
                edge+={darkgray, line width=1pt},
                s sep=3pt,
                inner xsep=2pt,
                inner ysep=3pt,
                line width=0.8pt,
                ver/.style={rotate=90, child anchor=north, parent anchor=south, anchor=center},
            },
            where level=1{text width=4em,font=\normalsize,}{},
            where level=2{text width=6em,font=\normalsize,}{},
            where level=3{text width=8em,font=\normalsize,}{},
            where level=4{text width=5em,font=\normalsize,}{},
            [
                In-context Learning, ver
                [
                    Training 
                    % [
                    %     Pretrain 
                    %     [
                    %         LLMs 
                    %         [
                    %             GPT-3~\cite{gpt3}{, }GLaM~\cite{glam}{, } OPT~\cite{opt}{, } \\ AlexaTM 20B~\cite{alexatm}
                    %             , leaf, text width=33em
                    %         ]
                    %     ]
                    %     [
                    %         Pretrain Corpus
                    %         [
                    %             Pretraining Corpus Combination~cite{shin2022corpora}
                    %             , leaf, text width=25em
                    %         ]
                    %     ]
                    % ]
                    [
                        Warmup  (\S \ref{sec:warmup})
                        [
                            Supervised \\ In-context \\ Training (\S \ref{sec:s_tuning})
                            [
                                MetaICL~\cite{metaicl}{, }OPT-IML~\cite{optiml}{, }FLAN~\cite{flan}{, }\\Super-NaturalInstructions~\cite{natural}{, }Scaling Instruction~\cite{chung}{, }\\Symbol Tuning~\cite{symboltuning}
                                , leaf, text width=36em
                            ]
                        ]
                        [
                            Self-supervised \\ In-context  \\ Training  (\S \ref{sec:ss_tuning})
                            [
                                Self-supervised ICL~\cite{selfsupericl}{, }PICL~\cite{picl}
                                , leaf, text width=28em
                            ]
                        ]
                    ]
                ]
                [
                    Inference
                    [
                    Demonstration \\ Designing (\S \ref{sec:demo})
                        [
                            Organization (\S \ref{sec:organ})
                            [
                                Selecting \\ (\S \ref{sec:select})
                                [   
                                    KATE~\cite{liu2022close}{, }EPR~\cite{rubin2022learning}{, }PPL~\cite{gonen2022demystifying}{, }\\SG-ICL~\cite{kim2022self}{, }Self Adaptive~\cite{Wu2022SelfadaptiveIL}{,}MI~\cite{sorensen2022information}{, }\\Q-Learning~\cite{zhang2022active}{,} Informative Score~\cite{li2023supporting}{,}\\Topic~\cite{topic}{, }UDR~\cite{udr}
                                    , leaf, text width=36.2em
                                ]
                            ]
                            [
                                Ordering \\ (\S \ref{sec:order})
                                [
                                GlobalE\&LocalE~\cite{lu2022order}
                                , leaf, text width=15em
                                ]
                            ]
                        ]
                        [
                            Formatting (\S \ref{sec:format})
                            [
                                Instruction \\ (\S \ref{sec:formatting_instruction})
                                [
                                    Instruction Induction~\cite{induct}{, }APE~\cite{zhou2022large}{, }\\Self-Instruct~\cite{wang2022self}
                                    , leaf, text width=30em
                                ]
                            ]
                            [
                                Reasoning \\ Steps \\ (\S \ref{sec:formatting_intermediate})
                                [
                                    CoT~\cite{wang2022self}{, }Complex CoT~\cite{fu2022complexitycot}{, }\\AutoCoT~\cite{autocot}{, }Self-Ask~\cite{selfask}{, } \\
                                    MoT\citep{mot}{, } SuperICL\citep{xu2023small}\\ iCAP~\cite{wang2022iteratively}{, }Least-to-Most Prompting~\cite{least}
                                    , leaf, text width=30.5em
                                ]
                            ]
                        ]
                    ]
                    [
                        Scoring \\ Function (\S \ref{sec:scoring})
                        [
                            Channel prompt tuning~\cite{min2022noisy}{, }
                            Structrured Prompting~\cite{hao2022structured}{, }\\
                            $k$NN-Prompting~\cite{knnPrompting}
                            , leaf, text width=36em
                        ]
                    ]
                ]
            ]
        \end{forest}
    }
    \caption{Taxonomy of in-context learning. The training and the inference stage are two main stages for ICL. During the training stage, existing ICL studies mainly take a pretrained LLM as backbone, and optionally warmup the model to strengthen and generalize the ICL ability. Towards the inference stage, the demonstration designing and the scoring function selecting are crucial for the ultimate performance.  }
    \label{taxo_of_icl}
\end{figure*}
%\begin{table}[t!]
\centering
\caption{Voice conversion \& F0 manipulation results. MOS results are reported with 95\% confidence interval. VDE, and FFE are reported for F0 manipulation while PER, WER, EER, and MOS are reported for voice conversion. Notice, for VDE, and FFE higher is the better since F0 was flattened.}
\label{tab:conv}

\resizebox{1\columnwidth}{!}{
\begin{tabular}{c@{~} | c@{~} | c@{~}c@{~} | c@{~} | c@{~} ||  c@{~}c@{~} }
\toprule
\multirow{2}{*}{Dataset} & \multirow{2}{*}{Method} & \multicolumn{4}{c||}{Voice Conversion} & \multicolumn{2}{c}{F0 Manipulation} \\
\cmidrule{3-8}
& & PER~$\downarrow$ & WER~$\downarrow$ & EER~$\downarrow$ & MOS~$\uparrow$ & VDE~$\uparrow$ & FFE~$\uparrow$ \\
\midrule
VCTK & GT  & 17.16 & 4.32 & 3.25 & 4.11$\pm$0.29 & -- & -- \\
\midrule 
\multirow{3}{*}{LJ}
% & ASR-TTS   & 50.74  & --     & 66.08 & 32.96 & 1.46 \\
& CPC       & 22.22 	& 16.11 		& 0.46 		& 3.57$\pm$0.15 		& \bf 46.68 & \bf 48.71\\
& HuBERT    & \bf 19.09 & \bf 12.23 & \bf 0.31  & \bf 3.71$\pm$0.24 & 39.20 		& 48.42\\
& VQ-VAE    & 40.88 	& 36.96 		& 9.65 		& 2.90$\pm$0.17 		& 10.54 	& 12.08 \\
\midrule 
\multirow{3}{*}{VCTK} 
% & ASR-TTS   & 68.88  & --    & 41.77 & 13.55 & 6.48 \\
& CPC       &  23.58 		& 15.98 		& \bf 4.83  &  3.42 $\pm$ 0.24 		& \bf 25.29 & \bf 26.97 \\
& HuBERT    &  \bf 20.85 	& \bf 12.72 & 6.01  		& \bf  3.58 $\pm$ 0.28 	& 23.46 	& 26.67 \\
& VQ-VAE    & 36.88  		& 29.44 		& 11.56 		& 3.08 $\pm$ 0.34 		& 7.03  	& 7.80  \\
\bottomrule
\end{tabular}}
\vspace{-0.4cm}
\end{table}

\vspace{-0.1cm}
\section{Results}
\vspace{-0.1cm}
Our results cover
% We report results for 
three different settings: (i) speech reconstruction experiments; (ii) speaker conversion and F0 manipulation; (iii) bitrate analysis with subjective tests for speech codec evaluation. We employ two datasets: LJ~\cite{ljspeech17} single speaker dataset and VCTK~\cite{vctk} multi-speaker dataset. All datasets were resampled to a 16kHz sample rate.

% \paragraph*{Implementation Details.}
% \smallskip
\noindent{\bf Implementation Details\quad} 
\label{sec:impl}
We follow the same setup as in~\cite{lakhotia2021generative}. For CPC, we used the model from~\cite{Riviere2020}, which was trained on a ``clean'' 6k hour sub-sample of the LibriLight dataset~\cite{Kahn2020,Riviere2020}. We extract a downsampled representation from an intermediate layer with a 256-dimensional embedding and a hop size of 160 audio samples. For HuBERT we used a \textsc{Base} 12 transformer-layer model trained for two iterations~\cite{hsu2020hubert} on 960 hours of LibriSpeech corpus~\cite{Panayotov2015}. 
% This model encodes every 320 raw audio samples into a 768-dimensional vector. 
This model downsamples the raw audio $\times320$ into a sequence of 768-dimensional vectors. Similarly to~\cite{lakhotia2021generative}, activations were extracted from the sixth layer.

%CPC: We use a dictionary of 100 units, leading to a bitrate of 700bps.
%HuBERT: A dictionary of 100 units is used, leading to a bitrate of 350bps. 
%VQVE: The VQ-VAE discrete code operates at a bitrate of 800bps.
% For both CPC and HuBERT, the k-means algorithm is applied to convert continuous frames to discrete codes, using the LibriSpeech clean-100h~\cite{Panayotov2015} dataset. 
For CPC and HuBERT, the k-means algorithm is trained on LibriSpeech clean-100h~\cite{Panayotov2015} dataset to convert continuous frames to discrete codes. We quantize both learned representations with $K=100$ centroids. Leading to a bitrate of 700bps for CPC and 350bps for HuBERT.

% VQ-VAE
Similarly to CPC models, we trained the VQ-VAE content encoder model on the ``clean'' 6K hours subset from the LibriLight dataset. We use an encoder operating on the raw signal to extract discrete units, similar to~\cite{jukebox}. In addition, ``random restarts'' were performed when the mean usage of a codebook vector fell below a predetermined threshold. Finally, we used HiFiGAN (architecture and objective) as the decoder instead of a simple convolutional decoder, as it improved the overall audio quality. This model encodes the raw audio into a sequence of discrete tokens from 256 possible tokens~\cite{garbacea2019low} with a hop size of 160 raw audio samples. The VQ-VAE discrete code operates at a bitrate of 800bps. We additionally experimented with 100 discrete units for VQ-VAE, however results were the best for 256. This finding is consistent with~\cite{garbacea2019low}.

% verification model
The speaker verification network uses the architecture proposed in~\cite{heigold2016end}. It was trained on the VoxCeleb2~\cite{voxceleb2} dataset, achieving a 7.4\% Equal Error Rate (EER) for speaker verification on the test split of the VoxCeleb1~\cite{Nagrani17} dataset.

% pitch
Only a single F0 representation is considered across all evaluated models, trained on the VCTK dataset.
% The F0 is extracted from the raw audio using YAAPT~\cite{yaapt} algorithm, using a window size of 20ms and a 5ms hop. 
The F0 is extracted from the raw audio using a window size of 20ms and a 5ms hop. 
As a result, the F0 sequence is sampled at 200Hz. 
% We apply the quantization described at Sec.~\ref{sec:method}, using a pitch codebook of $K'=20$ tokens and an encoder that downsamples the pitch by $\times16$. 
The quantization described at Sec.~\ref{sec:method}, is applied using an F0 codebook of $K'=20$ tokens and an encoder that downsamples the signal by $\times16$. Hence, the discrete F0 representation is sampled at 12.5Hz, leading to a bitrate of 65bps. The final bitrate of the evaluated codecs is the sum of the pitch code bitrate with the content code bitrate.

% \paragraph*{Evaluation Metrics}
% \smallskip
\noindent{\bf Evaluation Metrics\quad} 
We consider both subjective and objective evaluation metrics. For subjective tests, we report the Mean Opinion Scores (MOS). In which human evaluators rate the naturalness of audio samples on a scale of 1--5. Each experiment, included 50 randomly selected samples rated by 30 raters. For objective evaluation, we consider: (i) Equal Error Rate~(EER) as an automatic speaker verification metric obtained using a pre-trained speaker verification network. We report EER between test utterances and enrolled speakers; (ii) Voicing Decision Error (VDE)~\cite{nakatani2008method}, which measures the portion of frames with voicing decision error; (iii) F0 Frame Error (FFE)~\cite{chu2009reducing}, measures the percentage of frames that contain a deviation of more than 20\% in pitch value or have a voicing decision error; (iv) Word Error Rate (WER) and Phoneme Error Rate (PER), proxy metrics to the intelligibility of the generated audio. We used a pre-trained ASR network~\cite{baevski2020wav2vec} on both reconstructed and converted samples to calculate both metrics. %To generate target phonemes, the g2p-en~\cite{g2pE2019} Grapheme2Phoneme module was used.

% \vspace{-0.1cm}
% \smallskip
\noindent{\bf Reconstruction \& Conversion}
% \vspace{-0.1cm}
We start by reporting the reconstruction performance. Results are summarized in Table~\ref{tab:recon}. When considering the intelligibility of the reconstructed signal HuBERT reaches the lowest PER and WER scores across all models, where both CPC and HuBERT are superior to VQ-VAE. However, when considering F0 reconstruction VQ-VAE outperforms both HuBERT and CPC by a significant margin. This results are somewhat intuitive, bearing in mind VQ-VAE objective is to fully reconstruct the input signal. In terms of subjective evaluation, all models reach similar MOS scores, with one exception of CPC on LJ. 

%Notice, since the same F0 units are used for each method, this result implies the VQ-VAE units contain some information about the F0 of the signal, enabling better reconstruction. Regarding speaker information, the CPC gets the lowest EER. 

To better evaluate the disentanglement properties of each method with respect to speaker identity and F0, we conducted an additional set of experiments aiming at speaker conversion and F0 manipulation. For voice conversion, we converted each test utterance into five random target speakers. Next, we employed a speaker verification network, which extracts \emph{d-vector} representation to evaluate speaker-converted utterances' similarity to real speaker utterances (low error-rate indicates good conversion), providing measurement to the speaker identity's disentanglement from the evaluated coding method. The error-rate is reported between converted test utterances and enrolled speakers. For the LJ speech single speaker dataset, we converted samples from the VCTK dataset to the single speaker and enrolled all VCTK speakers together with the single speaker. Results are summarized in Table~\ref{tab:conv} (left). Unlike resynthesis results, on voice conversion CPC and HuBERT outperform VQ-VAE on both LJ and VCTK datasets, indicating VQ-VAE contains more information about the speaker in the encoded units, hence producing more artifacts. Notice, this also affects WER, PER, and the overall subjective quality (MOS). 

Next, to evaluate the presence of F0 in the discrete units, we flattened the F0 units before synthesizing the signal and calculated VDE and FFE with respect to the original F0 values. F0 flattening was done by setting the speakers' mean F0 value across all voiced frames. In this experiment, we expected units that contain F0 information to be better at F0 reconstruction over disentangled units. Results are summarized in Table~\ref{tab:conv} (right). Notice VQ-VAE can still reconstruct the F0 almost at the same level as when using the original F0 as conditioning (5.2 vs 7.03, and 5.59 vs 7.8), in contrast to CPC and HuBERT.

\begin{figure}[t!]
\centering
\includegraphics[width=0.65\columnwidth, trim={50 20 70 20}]{figures/codec_2.pdf}
% \caption{MUSHRA subjective listening test results as a function of bitrate per second for various methods. Purple dots denote the baseline methods, and green dots the proposed SSL based method.} 
\caption{MUSHRA subjective quality results as a function of bitrate per second. Purple dots denote the baseline methods, and green dots the proposed SSL based method.} 
\label{fig:codec}
\vspace{-0.5cm}
\end{figure}

% \vspace{-0.1cm}
% \smallskip
\noindent{\bf Speech Codec}
Our final experiment evaluates the obtained speech units as a low bitrate speech codec. 
% Therefore, we evaluate how the performance varies as a function of the number of discrete units. Changing the number of units is equivalent to varying the bitrate of the encoded signal. 
We use a subjective MUSHRA-type listening test~\cite{series2014method} to measure the perceived quality of the proposed speech codec with regard to its bitrate constraints. In MUSHRA evaluations, listeners are presented with a labeled uncompressed signal for reference, a set of test samples to rate, a copy of the uncompressed reference, and a low-quality anchor. Listeners are asked to rate each test utterance and the copy of the uncompressed reference with respect to the labeled reference in a scale of 1-100.

The experiment is performed on the VCTK dataset~\cite{vctk}. For evaluation, we used 20 utterances from 5 speakers. The set of speakers in the test data is disjoint with those in the training data. For this experiment, HuBERT models with 50, 100, and 200 units were trained as described in Sec.~\ref{sec:impl}. For comparison, we included other speech codecs in our evaluation: Opus~\cite{valin2012definition} wideband at 9 kbps VBR, Codec2~\cite{rowe2011codec} at 2.4 kbps and LPCNet~\cite{valin2019real} operating at 1.6 kbps. The LPCNet model was trained from scratch on the VCTK dataset following the experimental setup in~\cite{valin2019real}. The VQ-VAE model employs the HiFiGAN decoder trained on the LibriLight dataset to match the amount of data reported in~\cite{garbacea2019low}. We compressed the anchor sample with Speex~\cite{valin2016speex} at 4 kbps as a low anchor. Fig.~\ref{fig:codec} depicts the results. HuBERT with 50 units reaches the best MUSHRA score while its bitrate is only 365bps, which is significantly lower than the baseline methods.

\section{Open areas of research}
\label{sec:openResearch}

The surveyed papers  show that, despite exciting results obtained, fully understanding how to develop effective VPs for patient-doctor communication training requires further work. Reasons are related to the fact that either the technological components have not been fully explored yet or results are still inadequate to fully assess the effectiveness of different design approaches. Thus, in this section, we briefly discuss some open problems and present areas requiring further research.

\textbf{Assessment of design elements.}
%\label{sec:effectiveness}
% The general effectiveness of VPs on developing communication skills has been discussed in literature by several authors \cite{peddle2016virtual,lee2020effective,richardson2019virtualreview}. A common complaint in VP-related literature is the lack of standardized terminology and considerable heterogeneity in study design that makes retrieval and evaluation of literature a troublesome task. Despite that, both \cite{peddle2016virtual} and \cite{lee2020effective} conclude that, when appropriately contextualized in a well thought out educational context, VPs are indeed useful in developing, practising and building confidence about communication and other skills like decision making and teamwork.  
% Based on these findings, one possible question arising from our review is if the surveyed articles provide pieces of evidence about the effects on learning outcomes and efficacy of the simulation of the different instructional design elements and the technical features available. 
% Unfortunately, the answer is negative. In most of the works analyzed, authors report users' feedback or comments about a particular element/feature only and a direct comparison between different design choices is missing. The only notable exceptions are three. First, \cite{ochs2019training}, which assessed different presentation formats showing that immersive VR technologies yield superior results when compared with non-immersive ones. Second, concerning the distribution method, \cite{maicher2017developing} reports that a standalone application can provide a considerably higher level of engagement than its web-based version, thanks to the possibility of leveraging advanced technical features (voice-controlled input and large volume displays) to increase immersion and focus on the task at hand. Finally, \cite{hirumi2016advancing}, compared closed and open-option unfolding designs, reporting learners' preferences for the latter.  
As discussed in Section \ref{sec:effectiveness}, the current literature lacks a thorough evaluation of the effectiveness of alternative designs. This observation highlights the fact that further work has to be done to develop a better understanding of instructional elements and technical features that VP simulations can offer in order to achieve the desired learning outcomes. 


\textbf{Scope.} Another comment can be made on the specific communication learning context. While several core skill domains jointly contribute to a patient's health and satisfaction (like relationship building, information gathering, patient education, shared decision making and breaking bad news \cite{riedl2017influence}), most of the surveyed VP simulations  focus only on one specific domain. This observation highlights the need to develop novel approaches capable of addressing simultaneously the multiple communication challenges one has to face when interacting with a real patient, thus helping to improve the overall learner's communication skills.


%\andrea{Authoring tools for VPs.}
\textbf{Authoring tools.}
Implementing VPs is a cumbersome and complicated process, which requires taking into account several different elements (NLP, emotion modelling, affective computing, 3D animations, etc.), which, in turn, involve specific technological and technical skills. Usually, the development of a VP is a cyclical process of research, refinement and validation with experts that can take a considerable amount of time \cite{rossen2009human}. Thus, there is the need to develop simple (and effective) authoring tools that can allow developers to support clinical educators in the rapid design, prototyping and deploying of VPs in a variety of use cases. 
Examples of authoring tools for narrative-style VPs with 3D graphics are very scarce in the literature. The work presented in \cite{jeuring2015communicate} integrates a scenario builder that allows clinical educators to design the unfolding of their cases. This authoring tool exploits a domain reasoner where the response of the virtual agent is determined not only by the previous dialogue that the user chose, but also by other parameters like the agent's current emotional state. However, this tool lacks the possibility to customize the virtual environment or the VP's aesthetics.
The NERVE VP \cite{hirumi2016advancingPart2,hirumi2016advancing,kleinsmith2015understanding} is built upon the Virtual People Factory \cite{rossen2009human}, a web application that enables the users to build conversational models using an un-annotated corpus retrieval approach based on keyword matching. 
Another interesting example is SIDNIE (Scaffolded Interviews Developed by Nurses in Education \cite{dukes2016participatory}). This tool allows clinical educators to edit the patient's medical status, dialogue options and physical appearance. However, to our knowledge, SIDNIE has not been deployed in any publicly available form, and  appears to be aimed exclusively at nurse training scenarios.

In other application areas (such as building clinical skills and problem-solving abilities), the extensive use of tools such as DecisionSim, OpenLabyrinth and Web-SP %(a comparative study detailing the usage and characteristics of these tools can be found in 
\cite{doloca2015comparative} is a clear demonstration of the fact that an easy-to-use authoring tool is a determinant factor for the success of a VP application. However, compared to these areas, the specific context of patient-doctor communication training involves more complex systems, with 3D visuals and branched narratives offering a more realistic interaction, which makes the development of authoring tools in this area much more challenging \cite{talbot2012sorting}.\par
%End of Edo's Addon

\textbf{Emerging web technologies.}
%WebXR and Streaming Services}
In the previous sections, we highlighted that personal devices are coming with better and better hardware and computational power, thus helping to narrow the gap between standalone and web-based applications. Another contribution will inevitably come from recent advances in web-based technology, like, e.g., WebXR\footnote{\url{https://www.w3.org/TR/webxr/}}. WebXR is a device-independent framework that allows users to develop and share VR and AR applications over the Internet, with considerable support for different hardware and web browsers. In addition, game-streaming platforms such as Google Stadia\footnote{\url{https://stadia.google.com/}} are a very promising workaround for the limited computational capabilities of personal devices. With these platforms, the bulk of the computation is processed on the server side, then the pre-rendered output is streamed to the final user's device. The implementation of such technological solutions in the immediate future will enable the applications to combine the accessibility of current web-based software with the computational complexity of standalone applications run on a dedicated machine. \par

%edo{mi rendo conto che Virtual Humans è un termine veramente vago, perché può intendere sia NPC che Player Avatar... Non volevo usare NPC perché fa troppo gamer, però qualche articolo scientifico che usa il termine NPC l'ho trovato, quindi userei quello per evitare qualsiasi ambiguità. Riscritta tutta questa sezione}

\textbf{Multiple virtual humans.}
Interacting with a relative or another health care provider are considered crucial aspects of a clinician's communication skills \cite{hallin2011effects,kee2018communication}. However, VP simulations usually include only two actors: the learner (possibly represented by an avatar) and a unique Non-Playable Character (NPC), i.e, a virtual human not controlled by the trainee that represents the patient. The only two examples that include more than one NPC besides the patient are the Medical Interview Episode of the UTTimePortal \cite{zielke2016beyond,zielke2016using} (which incorporates a patient and a caregiver), and MPathic-VR \cite{guetterman2019medical,kron2017using} (which includes a patient's relative and a nurse). 
Beyond this observation, we should also note that another interesting future development (still untouched in the field of VPs for patient-doctor communication skills, to the best of our knowledge) could be to provide the possibility of interacting (within the simulation) with other human-controlled avatars, in a way similar to that proposed by approaches focused on inter-professional communication in emergency medical situations \cite{anbro2020using}. 


\textbf{Immersive VR and AR.}
% \andrea{Stress the relavance of IVR in particular. Immersion and presence contribute to empathic bond with the VP, which in turns have beneficial effects on the learning outcomes. This is why we stress so much these elements}
There is a general understanding among researchers that increasing the level of immersion and realism of the simulations (e.g., using large volume displays, HMDs, spatialized 3D audio, higher fidelity graphics and animations) leads to more believable human-computer interactions \cite{chuah2013exploring,johnsen2008evaluation}, which in turns help improve the users' communication and empathic skills  \cite{ochs2019training,zielke2017developing} and, ultimatley, the learning outcomes in general \cite{limniou2008full}. 
%The several potential advantages of using these technologies, with particular attention on provider-patient communication, have been extensively discussed in \cite{zielke2017developing}. 
However, surprisingly, the use of IVR technologies in this specific context appears to be quite limited. Only two VPs out of \totalVPs, i.e., Ochs \cite{ochs2019training} and CESTOL VR Clinic \cite{sapkaroski2018implementation}, mention the use of IVR, and AR appears to be completely unexplored.  The primary obstacles to the adoption of IVR or AR in VP simulations seem to be the complexity, challenges and costs of development steps \cite{zielke2017developing}.  

Fortunately, things are going to change rapidly. In recent years, the availability and quality of VR devices have increased considerably, and their cost has decreased dramatically.  These factors contribute (together with the availability of high-end development platforms such as Unity or Unreal engine) to reducing overall costs and efforts for developing IVR and AR applications. Furthermore, IVR offers currently a truly immersive, unbroken environment that can shift the cognitive load directed on imagining oneself \quotes{being there} in VR towards solving the task at hand. In turn, higher immersion and visual fidelity can have positive effects on learning \cite{coulter2007effect}, \cite{huerta2012measuring}. Thus, we expect that, soon, VR and AR will contribute to improving the state of the art in this research field.




\textbf{Fully-fledged non-verbal input.}
%Non-Verbal Input should be determinant, not an accessory.}
In our opinion, this is a major lack in current designs. The unfolding of the simulation's narrative should be dictated (in tandem) by both user's verbal and non-verbal behaviours. To this end, developers of future VPs should attempt to fully leverage non-verbal cues as a factor that actively influences the state of the agent. For instance, the same utterance should have a different outcome if the user maintains eye contact with the patient, looks in another direction, and is fidgeting or exhibiting an incoherent facial expression.
The extraction of para-linguistic factors such as tone of voice, loudness, inflection, rhythm, and pitch can provide information about the actual emotional states of the other peer in the communication. Prosody must be addressed with great attention since it is one of the main ways to express empathy and can have a considerable impact in increasing patient satisfaction \cite{kee2018communication}. Thus, computational mechanisms capable of extracting these variables from the analysis of the user's voice are sorely needed. 
The same para-linguistic factors should be also available to modulate the VP response according to its emotional states. In fact, one of the problems with present text-to-speech libraries is that they pronounce everything with the same tone, which makes it impossible to communicate feelings through voice. 



\begin{comment}
\begin{figure}
\includegraphics[width=\linewidth]{figs/beyond_tss_lesion.pdf}
\caption[]{End-to-End runtime lesion study of the entire MNIST dataset and the FMA featurized music dataset. Each of DROP's contributions provides a runtime improvement.}
\label{fig:beyond_lesion}
\end{figure}
\end{comment}



\section{Conclusion}
\label{sec:conclusion}

Advanced data analytics techniques must scale to rising data volumes. 
DR techniques offer a powerful toolkit when processing these datasets, with PCA frequently outperforming popular techniques in exchange for high computational cost. 
In response, we propose DROP, a new dimensionality reduction optimizer. 
DROP combines progressive sampling, progress estimation, and online aggregation to identify high quality low dimensional bases via PCA without processing the entire dataset by balancing the runtime of downstream tasks and achieved dimensionality. 
Thus, DROP provides a first step in bridging the gap between quality and efficiency in end-to-end DR for downstream \red{analytics}. 

%We revisit canonical operators for time series dimensionality reduction and the measurement study of~\cite{keogh-study}, and show that PCA is more effective than popular alternatives in the data mining literature often by a margin of over $2\times$ on average on gold-standard time series benchmark data sets with respect to output data dimension. More surprisingly, we empirically demonstrate that a small number of samples are sufficient to accurately characterize directions of maximum variance and obtain a high-quality low-dimensional transformation.



%\section{Limitations and Future Work}



\label{sec:limitationsandfuture}

%\input{testchapter}


%
% ---- Bibliography ----
%
% BibTeX users should specify bibliography style 'splncs04'.
% References will then be sorted and formatted in the correct style.
%
\bibliographystyle{splncs04}\begin{thebibliography}{10}
\providecommand{\url}[1]{\texttt{#1}}
\providecommand{\urlprefix}{URL }
\providecommand{\doi}[1]{https://doi.org/#1}

\bibitem{adefila2020students}
Adefila, A., Opie, J., Ball, S., Bluteau, P.: Students’ engagement and
  learning experiences using virtual patient simulation in a computer supported
  collaborative learning environment. Innovations in Education and Teaching
  International  \textbf{57}(1),  50--61 (2020)

\bibitem{albright2018using}
Albright, G., Bryan, C., Adam, C., McMillan, J., Shockley, K.: Using virtual
  patient simulations to prepare primary health care professionals to conduct
  substance use and mental health screening and brief intervention. Journal of
  the American Psychiatric Nurses Association  \textbf{24}(3),  247--259 (2018)

\bibitem{anbro2020using}
Anbro, S.J., Szarko, A.J., Houmanfar, R.A., Maraccini, A.M., Crosswell, L.H.,
  Harris, F.C., Rebaleati, M., Starmer, L.: Using virtual simulations to assess
  situational awareness and communication in medical and nursing education: A
  technical feasibility study. Journal of Organizational Behavior Management
  pp. 1--11 (2020)

\bibitem{banszki2018clinical}
B{\'a}nszki, F., Beilby, J., Quail, M., Allen, P., Brundage, S., Spitalnick,
  J.: A clinical educator’s experience using a virtual patient to teach
  communication and interpersonal skills. Australasian Journal of Educational
  Technology  \textbf{34}(3) (2018)

\bibitem{bearman2001random}
Bearman, M., Cesnik, B., Liddell, M.: Random comparison of ‘virtual
  patient’models in the context of teaching clinical communication skills.
  Medical education  \textbf{35}(9),  824--832 (2001)

\bibitem{bearman2015learning}
Bearman, M., Palermo, C., Allen, L.M., Williams, B.: Learning empathy through
  simulation: a systematic literature review. Simulation in healthcare
  \textbf{10}(5),  308--319 (2015)

\bibitem{benedict2013promotion}
Benedict, N., Schonder, K., McGee, J.: Promotion of self-directed learning
  using virtual patient cases. American Journal of Pharmaceutical Education
  \textbf{77}(7) (2013)

\bibitem{bloodworth2010initial}
Bloodworth, T., Cairco, L., McClendon, J., Hodges, L.F., Babu, S., Meehan,
  N.K., Johnson, A., Ulinski, A.C.: Initial evaluation of a virtual pediatric
  patient system  (2010)

\bibitem{carnell2015adapting}
Carnell, S., Halan, S., Crary, M., Madhavan, A., Lok, B.: Adapting virtual
  patient interviews for interviewing skills training of novice healthcare
  students. In: International Conference on Intelligent Virtual Agents. pp.
  50--59. Springer (2015)

\bibitem{chuah2013exploring}
Chuah, J.H., Robb, A., White, C., Wendling, A., Lampotang, S., Kopper, R., Lok,
  B.: Exploring agent physicality and social presence for medical team
  training. Presence: Teleoperators and Virtual Environments  \textbf{22}(2),
  141--170 (2013)

\bibitem{coulter2007effect}
Coulter, R., Saland, L., Caudell, T., Goldsmith, T.E., Alverson, D.: The effect
  of degree of immersion upon learning performance in virtual reality
  simulations for medical education. InMedicine Meets Virtual Reality
  \textbf{15}, ~155 (2007)

\bibitem{dibbelt2010patient}
Dibbelt, S., Schaidhammer, M., Fleischer, C., Greitemann, B.: Patient-doctor
  interaction in rehabilitation: is there a relationship between perceived
  interaction quality and long term treatment results? Die Rehabilitation
  \textbf{49}(5),  315--325 (2010)

\bibitem{doloca2015comparative}
Doloca, A., {\c{T}}{\u{A}}NCULESCU, O., Ciongradi, I., Trandafir, L., Stoleriu,
  S., Ifteni, G.: Comparative study of virtual patient applications. Proc. of
  the Romanian Academy, Series A  \textbf{16}(3),  466--473 (2015)

\bibitem{dukes2016participatory}
Dukes, L.C., Meehan, N., Hodges, L.F.: Participatory design of a pediatric
  virtual patient creation tool. In: 2016 IEEE International Conference on
  Healthcare Informatics (ICHI). pp. 449--455. IEEE (2016)

\bibitem{dupuy2019virtual}
Dupuy, L., Micoulaud-Franchi, J.A., Cassoudesalle, H., Ballot, O., Dehail, P.,
  Aouizerate, B., Cuny, E., de~Sevin, E., Philip, P.: Evaluation of a virtual
  agent to train medical students conducting psychiatric interviews for
  diagnosing major depressive disorders. Journal of Affective Disorders
  \textbf{263}, ~1--8 (2020)

\bibitem{festinger1954theory}
Festinger, L.: A theory of social comparison processes. Human relations
  \textbf{7}(2),  117--140 (1954)

\bibitem{fiscella2004patient}
Fiscella, K., Meldrum, S., Franks, P., Shields, C.G., Duberstein, P., McDaniel,
  S.H., Epstein, R.M.: Patient trust: is it related to patient-centered
  behavior of primary care physicians? Medical care pp. 1049--1055 (2004)

\bibitem{forrest2013essential}
Forrest, K., McKimm, J., Edgar, S.: Essential simulation in clinical education.
  John Wiley \& Sons (2013)

\bibitem{foster2016using}
Foster, A., Chaudhary, N., Kim, T., Waller, J.L., Wong, J., Borish, M., Cordar,
  A., Lok, B., Buckley, P.F.: Using virtual patients to teach empathy: a
  randomized controlled study to enhance medical students’ empathic
  communication. Simulation in Healthcare  \textbf{11}(3),  181--189 (2016)

\bibitem{franks2005patients}
Franks, P., Fiscella, K., Shields, C.G., Meldrum, S.C., Duberstein, P., Jerant,
  A.F., Tancredi, D.J., Epstein, R.M.: Are patients’ ratings of their
  physicians related to health outcomes? The Annals of Family Medicine
  \textbf{3}(3),  229--234 (2005)

\bibitem{guetterman2019medical}
Guetterman, T.C., Sakakibara, R., Baireddy, S., Kron, F.W., Scerbo, M.W.,
  Cleary, J.F., Fetters, M.D.: Medical students’ experiences and outcomes
  using a virtual human simulation to improve communication skills: Mixed
  methods study. Journal of medical Internet research  \textbf{21}(11),  e15459
  (2019)

\bibitem{Longnecker2010}
Ha, J.F., Longnecker, N.: {{D}octor-patient communication: a review}. Ochsner J
   \textbf{10}(1),  38--43 (2010)

\bibitem{hallin2011effects}
Hallin, K., Henriksson, P., Dal{\'e}n, N., Kiessling, A.: Effects of
  interprofessional education on patient perceived quality of care. Medical
  Teacher  \textbf{33}(1),  e22--e26 (2011)

\bibitem{hickson2002patient}
Hickson, G.B., Federspiel, C.F., Pichert, J.W., Miller, C.S., Gauld-Jaeger, J.,
  Bost, P.: Patient complaints and malpractice risk. Jama  \textbf{287}(22),
  2951--2957 (2002)

\bibitem{hirumi2016advancingPart2}
Hirumi, A., Johnson, T., Reyes, R.J., Lok, B., Johnsen, K., Rivera-Gutierrez,
  D.J., Bogert, K., Kubovec, S., Eakins, M., Kleinsmith, A., et~al.: Advancing
  virtual patient simulations through design research and interplay: part
  ii—integration and field test. Educational technology research and
  development  \textbf{64}(6),  1301--1335 (2016)

\bibitem{hirumi2016advancing}
Hirumi, A., Kleinsmith, A., Johnsen, K., Kubovec, S., Eakins, M., Bogert, K.,
  Rivera-Gutierrez, D.J., Reyes, R.J., Lok, B., Cendan, J.: Advancing virtual
  patient simulations through design research and interplay: part i: design and
  development. Educational Technology Research and Development  \textbf{64}(4),
   763--785 (2016)

\bibitem{huerta2012measuring}
Huerta, R.: Measuring the impact of narrative on player's presence and
  immersion in a first person game environment. The University of Texas-Pan
  American (2012)

\bibitem{jacklin2019virtual}
Jacklin, S., Chapman, S., Maskrey, N.: Virtual patient educational intervention
  for the development of shared decision-making skills: a pilot study. BMJ
  Simulation and Technology Enhanced Learning  \textbf{5}(4),  215--217 (2019)

\bibitem{jacklin2018improving}
Jacklin, S., Maskrey, N., Chapman, S.: Improving shared decision making between
  patients and clinicians: Design and development of a virtual patient
  simulation tool. JMIR medical education  \textbf{4}(2),  e10088 (2018)

\bibitem{janda2004simulation}
Janda, M.S., Mattheos, N., Nattestad, A., Wagner, A., Nebel, D., F{\"a}rbom,
  C., L{\^e}, D.H., Attstr{\"o}m, R.: Simulation of patient encounters using a
  virtual patient in periodontology instruction of dental students: design,
  usability, and learning effect in history-taking skills. European Journal of
  Dental Education  \textbf{8}(3),  111--119 (2004)

\bibitem{jeuring2015communicate}
Jeuring, J., Grosfeld, F., Heeren, B., Hulsbergen, M., IJntema, R., Jonker, V.,
  Mastenbroek, N., van~der Smagt, M., Wijmans, F., Wolters, M., et~al.:
  Communicate!—a serious game for communication skills—. In: Design for
  teaching and learning in a networked world, pp. 513--517. Springer (2015)

\bibitem{johnsen2008evaluation}
Johnsen, K., Lok, B.: An evaluation of immersive displays for virtual human
  experiences. In: 2008 IEEE virtual reality conference. pp. 133--136. IEEE
  (2008)

\bibitem{judge2004affect}
Judge, T.A., Ilies, R.: Affect and job satisfaction: a study of their
  relationship at work and at home. Journal of applied psychology
  \textbf{89}(4), ~661 (2004)

\bibitem{kee2018communication}
Kee, J.W., Khoo, H.S., Lim, I., Koh, M.Y.: Communication skills in
  patient-doctor interactions: learning from patient complaints. Health
  Professions Education  \textbf{4}(2),  97--106 (2018)

\bibitem{kelley2009patient}
Kelley, J.M., Lembo, A.J., Ablon, J.S., Villanueva, J.J., Conboy, L.A., Levy,
  R., Marci, C.D., Kerr, C., Kirsch, I., Jacobson, E.E., et~al.: Patient and
  practitioner influences on the placebo effect in irritable bowel syndrome.
  Psychosomatic medicine  \textbf{71}(7), ~789 (2009)

\bibitem{King2013}
King, A., Hoppe, R.B.: {\textquotedblleft}best practice{\textquotedblright} for
  patient-centered communication: A narrative review. Journal of Graduate
  Medical Education  \textbf{5}(3),  385--393 (Sep 2013).
  \doi{10.4300/jgme-d-13-00072.1},
  \url{https://doi.org/10.4300/jgme-d-13-00072.1}

\bibitem{kleinsmith2015understanding}
Kleinsmith, A., Rivera-Gutierrez, D., Finney, G., Cendan, J., Lok, B.:
  Understanding empathy training with virtual patients. Computers in human
  behavior  \textbf{52},  151--158 (2015)

\bibitem{kneebone2006human}
Kneebone, R., Nestel, D., Wetzel, C., Black, S., Jacklin, R., Aggarwal, R.,
  Yadollahi, F., Wolfe, J., Vincent, C., Darzi, A.: The human face of
  simulation: patient-focused simulation training. Academic Medicine
  \textbf{81}(10),  919--924 (2006)

\bibitem{kohatsu2004characteristics}
Kohatsu, N.D., Gould, D., Ross, L.K., Fox, P.J.: Characteristics associated
  with physician discipline: a case-control study. Archives of internal
  medicine  \textbf{164}(6),  653--658 (2004)

\bibitem{kron2017using}
Kron, F.W., Fetters, M.D., Scerbo, M.W., White, C.B., Lypson, M.L., Padilla,
  M.A., Gliva-McConvey, G.A., Belfore~II, L.A., West, T., Wallace, A.M.,
  et~al.: Using a computer simulation for teaching communication skills: A
  blinded multisite mixed methods randomized controlled trial. Patient
  education and counseling  \textbf{100}(4),  748--759 (2017)

\bibitem{lee2020effective}
Lee, J., Kim, H., Kim, K.H., Jung, D., Jowsey, T., Webster, C.: Effective
  virtual patient simulators for medical communication training: a systematic
  review. Medical Education  (2020)

\bibitem{limniou2008full}
Limniou, M., Roberts, D., Papadopoulos, N.: Full immersive virtual environment
  cavetm in chemistry education. Computers \& Education  \textbf{51}(2),
  584--593 (2008)

\bibitem{maicher2017developing}
Maicher, K., Danforth, D., Price, A., Zimmerman, L., Wilcox, B., Liston, B.,
  Cronau, H., Belknap, L., Ledford, C., Way, D., et~al.: Developing a
  conversational virtual standardized patient to enable students to practice
  history-taking skills. Simulation in Healthcare  \textbf{12}(2),  124--131
  (2017)

\bibitem{marei2018use}
Marei, H., Al-Eraky, M., Almasoud, N., Donkers, J., Van~Merrienboer, J.: The
  use of virtual patient scenarios as a vehicle for teaching professionalism.
  European Journal of Dental Education  \textbf{22}(2),  e253--e260 (2018)

\bibitem{mccoy2016evaluating}
McCoy, L., Pettit, R.K., Lewis, J.H., Allgood, J.A., Bay, C., Schwartz, F.N.:
  Evaluating medical student engagement during virtual patient simulations: a
  sequential, mixed methods study. BMC medical education  \textbf{16}(1), ~20
  (2016)

\bibitem{Nestel2011}
Nestel, D., Tabak, D., Tierney, T., Layat-Burn, C., Robb, A., Clark, S.,
  Morrison, T., Jones, N., Ellis, R., Smith, C., McNaughton, N., Knickle, K.,
  Higham, J., Kneebone, R.: Key challenges in simulated patient programs: An
  international comparative case study. {BMC} Medical Education  \textbf{11}(1)
  (Sep 2011). \doi{10.1186/1472-6920-11-69},
  \url{https://doi.org/10.1186/1472-6920-11-69}

\bibitem{ochs2019training}
Ochs, M., Mestre, D., De~Montcheuil, G., Pergandi, J.M., Saubesty, J.,
  Lombardo, E., Francon, D., Blache, P.: Training doctors’ social skills to
  break bad news: evaluation of the impact of virtual environment displays on
  the sense of presence. Journal on Multimodal User Interfaces  \textbf{13}(1),
   41--51 (2019)

\bibitem{o2019suicide}
O’Brien, K.H.M., Fuxman, S., Humm, L., Tirone, N., Pires, W.J., Cole, A.,
  Grumet, J.G.: Suicide risk assessment training using an online virtual
  patient simulation. Mhealth  \textbf{5} (2019)

\bibitem{papadakis2005disciplinary}
Papadakis, M.A., Teherani, A., Banach, M.A., Knettler, T.R., Rattner, S.L.,
  Stern, D.T., Veloski, J.J., Hodgson, C.S.: Disciplinary action by medical
  boards and prior behavior in medical school. New England Journal of Medicine
  \textbf{353}(25),  2673--2682 (2005)

\bibitem{peddle2019exploring}
Peddle, M., Bearman, M., Mckenna, L., Nestel, D.: Exploring undergraduate
  nursing student interactions with virtual patients to develop
  ‘non-technical skills’ through case study methodology. Advances in
  Simulation  \textbf{4}(1), ~2 (2019)

\bibitem{peddle2016virtual}
Peddle, M., Bearman, M., Nestel, D.: Virtual patients and nontechnical skills
  in undergraduate health professional education: an integrative review.
  Clinical Simulation in Nursing  \textbf{12}(9),  400--410 (2016)

\bibitem{peddle2019development}
Peddle, M., Mckenna, L., Bearman, M., Nestel, D.: Development of non-technical
  skills through virtual patients for undergraduate nursing students: an
  exploratory study. Nurse education today  \textbf{73},  94--101 (2019)

\bibitem{quail2016student}
Quail, M., Brundage, S.B., Spitalnick, J., Allen, P.J., Beilby, J.: Student
  self-reported communication skills, knowledge and confidence across
  standardised patient, virtual and traditional clinical learning environments.
  BMC medical education  \textbf{16}(1), ~73 (2016)

\bibitem{richardson2019virtual}
Richardson, C.L., Chapman, S., White, S.: Virtual patient educational programme
  to teach counselling to clinical pharmacists: development and proof of
  concept. BMJ Simulation and Technology Enhanced Learning  \textbf{5}(3),
  167--169 (2019)

\bibitem{richardson2019virtualreview}
Richardson, C.L., White, S., Chapman, S.: Virtual patient technology to educate
  pharmacists and pharmacy students on patient communication: a systematic
  review. BMJ Simulation and Technology Enhanced Learning  (2019)

\bibitem{riedl2017influence}
Riedl, D., Sch{\"u}{\ss}ler, G.: The influence of doctor-patient communication
  on health outcomes: a systematic review. Zeitschrift f{\"u}r Psychosomatische
  Medizin und Psychotherapie  \textbf{63}(2),  131--150 (2017)

\bibitem{rogers2011developing}
Rogers, L.: Developing simulations in multi-user virtual environments to
  enhance healthcare education. British Journal of Educational Technology
  \textbf{42}(4),  608--615 (2011)

\bibitem{rossen2009human}
Rossen, B., Lind, S., Lok, B.: Human-centered distributed conversational
  modeling: Efficient modeling of robust virtual human conversations. In:
  International Workshop on Intelligent Virtual Agents. pp. 474--481. Springer
  (2009)

\bibitem{sapkaroski2018implementation}
Sapkaroski, D., Baird, M., McInerney, J., Dimmock, M.R.: The implementation of
  a haptic feedback virtual reality simulation clinic with dynamic patient
  interaction and communication for medical imaging students. Journal of
  medical radiation sciences  \textbf{65}(3),  218--225 (2018)

\bibitem{schoenthaler2017simulated}
Schoenthaler, A., Albright, G., Hibbard, J., Goldman, R.: Simulated
  conversations with virtual humans to improve patient-provider communication
  and reduce unnecessary prescriptions for antibiotics: a repeated measure
  pilot study. JMIR medical education  \textbf{3}(1), ~e7 (2017)

\bibitem{stelfox2005relation}
Stelfox, H.T., Gandhi, T.K., Orav, E.J., Gustafson, M.L.: The relation of
  patient satisfaction with complaints against physicians and malpractice
  lawsuits. The American journal of medicine  \textbf{118}(10),  1126--1133
  (2005)

\bibitem{stewart1995effective}
Stewart, A.M.: Effective physician-patient communication and health outcomes: a
  review. CMAJ : Canadian Medical Association journal = journal de
  l'Association medicale canadienne pp. 1423--1433 (1995)

\bibitem{szilas2019virtual}
Szilas, N., Chauveau, L., Andkjaer, K., Luiu, A.L., B{\'e}trancourt, M.,
  Ehrler, F.: Virtual patient interaction via communicative acts. In:
  Proceedings of the 19th ACM International Conference on Intelligent Virtual
  Agents. pp. 91--93 (2019)

\bibitem{talbot2012sorting}
Talbot, T.B., Sagae, K., John, B., Rizzo, A.A.: Sorting out the virtual
  patient: how to exploit artificial intelligence, game technology and sound
  educational practices to create engaging role-playing simulations.
  International Journal of Gaming and Computer-Mediated Simulations (IJGCMS)
  \textbf{4}(3),  1--19 (2012)

\bibitem{urresti2017virtual}
Urresti-Gundlach, M., Tolks, D., Kiessling, C., Wagner-Menghin, M., H{\"a}rtl,
  A., Hege, I.: Do virtual patients prepare medical students for the real
  world? development and application of a framework to compare a virtual
  patient collection with population data. BMC medical education
  \textbf{17}(1), ~174 (2017)

\bibitem{washburn2020virtual}
Washburn, M., Parrish, D.E., Bordnick, P.S.: Virtual patient simulations for
  brief assessment of mental health disorders in integrated care settings.
  Social Work in Mental Health  \textbf{18}(2),  121--148 (2020)

\bibitem{zielke2017developing}
Zielke, M.A., Zakhidov, D., Hardee, G., Evans, L., Lenox, S., Orr, N., Fino,
  D., Mathialagan, G.: Developing virtual patients with vr/ar for a natural
  user interface in medical teaching. In: 2017 IEEE 5th International
  Conference on Serious Games and Applications for Health (SeGAH). pp.~1--8.
  IEEE (2017)

\bibitem{zielke2016beyond}
Zielke, M.A., Zakhidov, D., Jacob, D., Hardee, G.: Beyond fun and games: toward
  an adaptive and emergent learning platform for pre-med students with the ut
  time portal. In: 2016 IEEE International Conference on Serious Games and
  Applications for Health (SeGAH). pp.~1--8. IEEE (2016)

\bibitem{zielke2016using}
Zielke, M.A., Zakhidov, D., Jacob, D., Lenox, S.: Using qualitative data
  analysis to measure user experience in a serious game for premed students.
  In: International Conference on Virtual, Augmented and Mixed Reality. pp.
  92--103. Springer (2016)

\bibitem{zlotos2016scenario}
Zlotos, L., Power, A., Hill, D., Chapman, P.: A scenario-based virtual patient
  program to support substance misuse education. American journal of
  pharmaceutical education  \textbf{80}(3) (2016)

\end{thebibliography}
%

\end{document}
