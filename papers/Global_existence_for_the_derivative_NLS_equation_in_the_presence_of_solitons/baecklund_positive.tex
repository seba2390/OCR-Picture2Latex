\subsection{B\"{a}cklund transformation for $x>0$}
What we will do in this subsection is to explore the map $u^{(1)}\leftrightarrow u^{(0)}$ for $x>0$. Therefore we introduce the functions $w^{(j)}$, $v^{(j)}$ for $j=0,1$, which are related to $u^{(j)}$ by (\ref{e def w}) and (\ref{e def v}), respectively. Next we define the matrix
\begin{equation*}
    A(x)=
    \left[
      \begin{array}{cc}
        a_{11}(x) & a_{12}(x) \\
        a_{21}(x) & a_{22}(x) \\
      \end{array}
    \right]
\end{equation*}
by
\begin{equation*}
        \left(
          \begin{array}{c}
            a_{11}(x) \\
            a_{21}(x) \\
          \end{array}
        \right)
        :=m^{(0)}(z_1;x)
        \left(
          \begin{array}{c}
            1 \\
            - \frac{2\ii\lambda_1c_1 e^{2\ii z_1x}} {z_1-\overline{z}_1} \\
          \end{array}
        \right),
        \quad
%
        \left(
          \begin{array}{c}
            a_{12}(x) \\
            a_{22}(x) \\
          \end{array}
        \right)
        :=m^{(0)}(\overline{z}_1;x)
        \left(
          \begin{array}{c}
            \frac{\overline{c}_1 e^{-2\ii \overline{z}_1 x}} {2\ii\overline{\lambda}_1(\overline{z}_1-z_1)} \\
            1 \\
          \end{array}
        \right).
\end{equation*}
In order to define the B\"{a}cklund transformation it is necessary to know that there is no $x$ such that the determinant of $A(x)$ vanishes.
\begin{prop}\label{p A inverse}
    The matrix $A$ is invertible for all $x\in\Real$. Moreover, if $\|r^{(1)}_+\|_{H^1\cap L^{2,1}}+\|r^{(1)}_-\|_{H^1\cap L^{2,1}}\leq M$, then
    \begin{equation} \label{e lower bound for det A}
    |\det(A(x))|^{-1}\leq C_M,\quad\text{for all }x>0,
    \end{equation}
    where the constant $C_M$ does not depend on $x$ and $r_{\pm}$.
\end{prop}
\begin{proof}
    Using the symmetry (\ref{e symmetrie of m}) we find
    \begin{eqnarray*}
    % \nonumber to remove numbering (before each equation)
       \left(
          \begin{array}{c}
            a_{12}(x) \\
            a_{22}(x) \\
          \end{array}
       \right)
       &=&
       \left[
         \begin{array}{cc}
           w^{(0)}(x) & 1 \\
           -|w^{(0)}(x)|^2-4\overline{z}_1 & -\overline{w^{(0)}}(x) \\
         \end{array}
       \right]
       \overline{m^{(0)}}(z_1;x)
       \left[
          \begin{array}{cc}
            0 & \frac{-1}{4\overline{z}_1} \\
            1 & 0 \\
          \end{array}
       \right]
       \left(
          \begin{array}{c}
            \frac{\overline{c}_1 e^{-2\ii \overline{z}_1 x}} {2\ii\overline{\lambda}_1(\overline{z}_1-z_1)} \\
            1 \\
          \end{array}
       \right)
       \\
       &=&
       \frac{1}{4\overline{z}_1}
       \left[
         \begin{array}{cc}
           -w^{(0)}(x) & -1 \\
           |w^{(0)}(x)|^2+4\overline{z}_1 & \overline{w^{(0)}}(x) \\
         \end{array}
       \right]
       \overline{m^{(0)}}(z_1;x)
       \left(
          \begin{array}{c}
            1\\
            \frac{2\ii\overline{\lambda}_1\overline{c}_1 e^{-2\ii \overline{z}_1 x}} {(\overline{z}_1-z_1)} \\
          \end{array}
       \right)
       \\
       &=&
       \frac{1}{4\overline{z}_1}
       \left[
         \begin{array}{cc}
           -w^{(0)}(x) & -1 \\
           |w^{(0)}(x)|^2+4\overline{z}_1 & \overline{w^{(0)}}(x) \\
         \end{array}
       \right]
       \left(
          \begin{array}{c}
            \overline{a_{11}}(x) \\
            \overline{a_{21}}(x) \\
          \end{array}
       \right).
    \end{eqnarray*}
    It follows directly that
    \begin{equation*}
        \det(A(x))=|a_{11}(x)|^2 +\frac{1}{4\overline{z}_1} |\overline{w^{(0)}}(x)a_{11}(x)+a_{21}(x) |^2.
    \end{equation*}
    The case $\det(A(x))=0$ is impossible, since due to $\im(z_1)\neq 0$ it would follow that $a_{11}(x)=a_{21}(x)=0$ and hence $\left(1,
    - \frac{2\ii\lambda_1c_1 e^{2\ii z_1x}}{z_1-\overline{z}_1}\right)^T \in\ker[m^{(0)}]$. This contradicts $\det(m^{(0)}(z;x))\equiv 1$ (see Remark \ref{r uniqueness + det=1}). Now we turn to the proof of (\ref{e lower bound for det A}). For sake of contradiction we assume that for any $d>0$ we can find $x>0$ such that $|\det(A(x))|<d$. Due to $\im(z_1)\neq 0$ and $w\in L^{\infty}$ wa can assume w.l.o.g. $|a_{11}(x)|< d$ and $|a_{12}(x)|<d$. Using (\ref{e det m=1}) we find
    \begin{eqnarray*}
    % \nonumber to remove numbering (before each equation)
      1 &=& \left|[m^{(0)}(z_1;x)]_{11} [m^{(0)}(z_1;x)]_{22}- [m^{(0)}(z_1;x)]_{12} [m^{(0)}(z_1;x)]_{21}\right| \\
       &=& \left|\left\{a_{11}(x) +\frac{2\ii\lambda_1c_1 e^{2\ii z_1x}} {z_1-\overline{z}_1}[m^{(0)}(z_1;x)]_{12}\right\} [m^{(0)}(z_1;x)]_{22}\right.\\
         &&\,
         \left.-\,[m^{(0)}(z_1;x)]_{12} \left\{ a_{21}(x)+ \frac{2\ii\lambda_1c_1 e^{2\ii z_1x}} {z_1-\overline{z}_1}[m^{(0)}(z_1;x)]_{22}\right\} \right|\\
       &=& \left|a_{11}(x) [m^{(0)}(z_1;x)]_{22}- [m^{(0)}(z_1;x)]_{12} a_{21}(x)\right|\\
       &<&d\cdot   \left\{\left|[m^{(0)}(z_1;x)]_{22}\right|+ \left|[m^{(0)}(z_1;x)]_{12}\right|\right\}\\
       &\leq&d\cdot C\cdot  \|m^{(0)}(z_1;\cdot)-1\|_{H^1(\Real_+)\cap L^{2,1}(\Real_+)}\\
       &\leq& d\cdot C_M.
    \end{eqnarray*}
    Here $C_M$ is the constant in Lemma \ref{l m(z0)-1 in H^11 and H^2} and it follows that $d$ cannot be arbitrary small. In addition we also proved the bound (\ref{e lower bound for det A}).
\end{proof}
\begin{lem}\label{l solvability of RHP N=1}
    For any scattering data $\mathcal{S}^{(1)}=\{r^{(1)}_{\pm};z_1;c_1\}$ such that $r^{(1)}_{\pm}\in L^{2,1}\cap H^1$ satisfies (\ref{e relation r+ r-})-(\ref{e r constraint}), \rh \ref{rhp m^1} admits an unique solution $m^{(1)}(z;x)$. This solution can be obtained from $m^{(0)}(z;x)$ by the following:
    \begin{equation}\label{e B�cklund for m^1}
        m^{(1)}(z;x)=A(x)\mu(z) A^{-1}(x)m^{(0)}(z;x)\mu^{-1}(z),
    \end{equation}
    where
    \begin{equation*}
        \mu(z)=
        \left[
          \begin{array}{cc}
            z-z_1 & 0 \\
            0 & z-\overline{z}_1 \\
          \end{array}
        \right].
    \end{equation*}
\end{lem}
\begin{proof}
    Let us denote by $\widetilde{m}(z;x)$ the right hand side of (\ref{e B�cklund for m^1}) and set
    \begin{equation*}
        \left[
          \begin{array}{cc}
            \tau_{11}(z) & \tau_{12}(z) \\
            \tau_{21}(z) & \tau_{22}(z) \\
          \end{array}
        \right]:=A^{-1}(x)m^{(0)}(z;x).
    \end{equation*}
    We find
    \begin{equation}\label{e Res of m tilde}
        \begin{aligned}
          \res_{z=z_1}\widetilde{m}(z;x)&=
          A(x)
          \left[
            \begin{array}{cc}
              0 & 0 \\
              (z_1-\overline{z}_1)\tau_{21}(z_1) & 0 \\
            \end{array}
          \right]
          ,\\
          \res_{z=\overline{z}_1}\widetilde{m} (z;x)&=
          A(x)
          \left[
            \begin{array}{cc}
              0 & (\overline{z}_1-z_1) \tau_{12}(\overline{z}_1) \\
              0 & 0 \\
            \end{array}
          \right]
          ,
        \end{aligned}
    \end{equation}
    and
    \begin{equation}\label{e lim m tilde c}
        \begin{aligned}
        &\lim_{z\to z_1}\widetilde{m}(z;x)
          \left(
            \begin{array}{cc}
              0 & 0 \\
              2\ii\lambda_1c_1 e^{2\ii z_1x} & 0
            \end{array}
          \right)=
          A(x)
          \left[
            \begin{array}{cc}
              0 & 0 \\
              2\ii\lambda_1c_1 e^{2\ii z_1x}\tau_{22}(z_1) & 0 \\
            \end{array}
          \right]
          ,\\
          &\lim_{z\to \overline{z}_1}\widetilde{m}(z;x)
          \left(
            \begin{array}{cc}
              0 & \frac{-\overline{c}_1}{2\ii\lambda_1} e^{-2\ii \overline{z}_1x} \\
              0 & 0
            \end{array}
          \right)=
          A(x)
          \left[
            \begin{array}{cc}
              0 &\frac{-\overline{c}_1}{2\ii\lambda_1} \tau_{11}(\overline{z}_1) \\
              0 & 0 \\
            \end{array}
          \right].
        \end{aligned}
      \end{equation}
      Using $\det m^{(0)}\equiv 1 $ it is easy to obtain
      \begin{equation*}
        \tau_{21}(z_1)=\frac{1}{\det A(x)} \frac{2\ii\lambda_1c_1 e^{2\ii z_1x}}{z_1-\overline{z}_1},\quad
        \tau_{22}(z_1)=\frac{1}{\det A(x)},
      \end{equation*}
      and
      \begin{equation*}
        \tau_{11}(\overline{z}_1)=\frac{1}{\det A(x)},\quad\tau_{12}(\overline{z}_1)= \frac{-1}{\det A(x)} \frac{\overline{c}_1 e^{-2\ii \overline{z}_1x}}{2\ii\overline{\lambda}_1 (\overline{z}_1-z_1)},
      \end{equation*}
      and thus it follows from (\ref{e Res of m tilde}) and (\ref{e lim m tilde c}) that $\widetilde{m}$ satisfies (\ref{e Res m^1}). Now we proceed with the jump on the real axis and check if point (iii) of \rh \ref{rhp m^1} is satisfied. Using the jump condition of $m^{(0)}$ (see (\ref{e jump})) and the definition (\ref{e r^0}) of $r_{\pm}^{(0)}$ we find for $z\in\Real$
      \begin{eqnarray*}
      % \nonumber to remove numbering (before each equation)
        \widetilde{m}_+(z;x) &=& \widetilde{m}_-(z;x) \mu(z)\left(
          \begin{array}{cc}
            1+\overline{r}^{(0)}_+(z)r^{(0)}_-(z) & e^{-2\ii zx}\overline{r}^{(0)}_+(z) \\
            e^{2\ii zx}r^{(0)}_-(z) & 1 \\
          \end{array}
        \right)\mu^{-1}(z) \\
         &=&  \widetilde{m}_-(z;x)\left(
          \begin{array}{cc}
            1+\overline{r}^{(1)}_+(z)r^{(1)}_-(z) & e^{-2\ii zx}\overline{r}^{(1)}_+(z) \\
            e^{2\ii zx}r^{(1)}_-(z) & 1 \\
          \end{array}
        \right).
      \end{eqnarray*}
      Next we observe
      \begin{equation}\label{e expansion of m tilde}
        \widetilde{m}(z;x)= \left[1+\frac{A(x)\:\mu(0)\:A^{-1}(x)}{z}\right] m^{(0)}(z;x)
        \left[
          \begin{array}{cc}
            \frac{z}{z-z_1} & 0 \\
            0 & \frac{z}{z-\overline{z}_1} \\
          \end{array}
        \right].
      \end{equation}
      It follows that $\widetilde{m}$ behaves for $|z|\to\infty$ as required in the point (ii) of \rh \ref{rhp m^1}. Since also the point (i) of \rh \ref{rhp m^1} is true, we conclude by the uniqueness (see Remark \ref{r uniqueness + det=1}) that $m^{(1)}(z;x)\equiv\widetilde{m}(z;x)$.
\end{proof}
The B\"{a}cklund transformation formula (\ref{e B�cklund for m^1}) is an ideal expression to extend Corollary \ref{c u in H^11 and H^2 (positive half line)} and Lemma \ref{l m(z0)-1 in H^11 and H^2} to the case where the scattering data are involving one pole $z_1$.
\begin{cor}\label{c u in H^11 and H^2 (positive half line - 1 pole)}
   Under the assumptions of Lemma \ref{l solvability of RHP N=1} the potential $u^{(1)}(x)$ reconstructed from the solution $m^{(1)}(z;x)$ of \rh \ref{rhp m dynamic} by using (\ref{e rec 1}) and (\ref{e rec 2}) lies in $H^2(\Real_+)\cap H^{1,1}(\Real_+)$. Moreover, if $\|r^{(1)}_+\|_{H^1\cap L^{2,1}}+\|r^{(1)}_-\|_{H^1\cap L^{2,1}}+|c_1|\leq M$ for some fixed $M>0$, then $u^{(1)}$ satisfies the bound
   \begin{equation}\label{e u^1 in H^11 and H^2 (positive half line)}
       \|u^{(1)}\|_{H^2(\Real_+)\cap H^{1,1}(\Real_+)}\leq C_M
   \end{equation}
   where the constant $C_M$ depends on $M$ and $z_1$ but not on $r^{(1)}_{\pm}$ and $|c_1|$.
\end{cor}
\begin{proof}
    We use (\ref{e expansion of m tilde}) and
    the expansion
    \begin{equation*}
        \left[
          \begin{array}{cc}
            \frac{z}{z-z_1} & 0 \\
            0 & \frac{z}{z-\overline{z}_1} \\
          \end{array}
        \right]=1-\frac{\mu(0)}{z}+\mathcal{O}(z^{-2}), \quad\text{as }|z|\to\infty
    \end{equation*}
    in order to find
    \begin{equation*}
        \begin{aligned}
            \lim_{|z|\to\infty}z\; \left[m^{(1)}(z;x)\right]_{12}& = \lim_{|z|\to\infty}z\; \left[m^{(0)}(z;x)\right]_{12}+ \left[A(x)\:\mu(0)\:A^{-1}(x)\right]_{12},\\
            \lim_{|z|\to\infty}z\; \left[m^{(1)}(z;x)\right]_{21}& = \lim_{|z|\to\infty}z\; \left[m^{(0)}(z;x)\right]_{21}+ \left[A(x)\:\mu(0)\:A^{-1}(x)\right]_{21}.
        \end{aligned}
    \end{equation*}
    Using the notation (\ref{e def w}) and (\ref{e def v}), we find by the reconstruction formulas (\ref{e rec 1}) and (\ref{e rec 2})
    \begin{equation}\label{e decomposition w}
       w^{(1)}(x)=w^{(0)}(x)+B_1(x),\qquad B_1(x):=-\frac{8\ii \im(z_1)a_{11}(x)a_{12}(x)}{\det(A(x))}
    \end{equation}
    and
    \begin{multline}\label{e decomposition v}
        \qquad e^{-\frac{1}{2\ii} \int_{+\infty}^x|u^{(1)}(y)|^2dy}v^{(1)}_x(x)= e^{-\frac{1}{2\ii} \int_{+\infty}^x|u^{(0)}(y)|^2dy}v^{(0)}_x(x) +B_2(x),\\
        B_2(x):=\frac{4 \im(z_1)a_{21}(x)a_{22}(x)}{\det(A(x))}.\qquad
    \end{multline}
    As it is easily to derive from the definition of $A(x)$ and Lemma \ref{l m(z0)-1 in H^11 and H^2}, we have $(A(\cdot)-1)\in L^{2,1}(\Real_+)\cap H^1(\Real_+)$ (note that $\im(z_1)>0$ is necessary). In addition, $(\det(A(\cdot))-1)\in L^{2,1}(\Real_+)\cap H^1(\Real_+)$. These two facts and \ref{e lower bound for det A} yield $B_j(\cdot)\in L^{2,1}(\Real_+)\cap H^1(\Real_+)$ for $j=1,2$. If we apply Corollary \ref{c u in H^11 and H^2 (positive half line)} to $v^{(0)}$ and $w^{(0)}$, we end up with $w^{(1)}\in H^{1,1}(\Real_+)$ and
    \begin{equation*}
        \partial_x \left(e^{-\frac{1}{2\ii} \int_{+\infty}^x|u^{(1)}(y)|^2dy } v^{(1)}_x(x)\right)\in L^2_x(\Real_+),
    \end{equation*}
    which is sufficient to conclude $u^{(1)}\in H^2(\Real_+)\cap H^{1,1}(\Real_+)$.
\end{proof}
\begin{cor}\label{c m^1(z0)-1 in H^11 and H^2}
   Let the assumptions of Lemma \ref{l solvability of RHP N=1} be valid and fix $z_2\in\Compl\setminus(\Real\cup \set{z_1,\overline{z}_1})$. Then for the solution $m^{(1)}(z;x)$ of \rh \ref{rhp m dynamic} we have $m^{(1)}(z_2;\cdot)-1\in H^1(\Real_+)\cap L^{2,1}(\Real_+)$. Moreover, if $\|r^{(1)}_+\|_{H^1\cap L^{2,1}}+\|r^{(1)}_-\|_{H^1\cap L^{2,1}}\leq M$ for some fixed $M>0$, then we also have the bound
   \begin{equation}\label{e m^1(z0)-1 in H^1 and L^2,1}
       \|m^{(1)}(z_2;\cdot)-1\|_{H^1(\Real_+)\cap L^{2,1}(\Real_+)}\leq C_M,
   \end{equation}
   where the constant $C_M>0$ depends on $M$, $z_1$, $z_2$ and $|c_1|$ but not on $r^{(1)}_{\pm}$.
\end{cor}
\begin{proof}
    (\ref{e B�cklund for m^1}) can be written as
    \begin{multline*}
        m^{(1)}(z_2;x)=m^{(0)}(z_2;x)\\-2\ii\im(z_1) A(x)
        \left[
          \begin{array}{cc}
            0 & \frac{a_{21}(x)[m(z_2;x)]_{11}- a_{11}(x)[m(z_2;x)]_{21}} {(z_2-z_1)\det(A(x))} \\
            \frac{a_{22}(x)[m(z_2;x)]_{12}- a_{12}(x)[m(z_2;x)]_{22}} {(z_2-\overline{z}_1)\det(A(x))} & 0 \\
          \end{array}
        \right].
    \end{multline*}
    $m^{(1)}(z_2;\cdot)-1\in H^1(\Real_+)\cap L^{2,1}(\Real_+)$ is now a direct consequence of $m^{(0)}(z_2;\cdot)-1\in H^1(\Real_+)\cap L^{2,1}(\Real_+)$ (see Lemma \ref{l m(z0)-1 in H^11 and H^2}), $(A(\cdot)-1)\in L^{2,1}(\Real_+)\cap H^1(\Real_+)$, $(\det(A(\cdot))-1)\in L^{2,1}(\Real_+)\cap H^1(\Real_+)$ and (\ref{e lower bound for det A}).
\end{proof} 