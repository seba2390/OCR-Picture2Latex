\section{Proof of Theorem \ref{t main} }\label{s proof}
Inductively we can add more and more poles to the \rh \ref{rhp m}. Using the B\"{a}cklund transformation for $x>0$ and $x<0$ as described in the previous section we are able to show the following Lemma.
\begin{lem}\label{l solvability of RHP N geq 2}
    For any functions $r_{\pm}\in H^1(\Real)\cap L^{2,1}(\Real)$ which satisfy (\ref{e relation r+ r-}) - (\ref{e r constraint}), for any pairwise distinct poles $\lambda_1,...,\lambda_N$ with $\im(\lambda_k^2)>0$ and for any nonzero constants $c_1,...,c_N$, the \rh \ref{rhp m} is solvable. Moreover the function $u$ which can be obtained from $m$ by using (\ref{e rec 1}) and (\ref{e rec 2}) lies in $H^2(\Real)\cap H^{1,1}(\Real)$. If, in addition, $\|r^{(1)}_+\|_{H^1\cap L^{2,1}}+\|r^{(1)}_-\|_{H^1\cap L^{2,1}}+|c_1|+...+|c_N|\leq M$ for some fixed $M>0$, then we also have the bound
    \begin{equation}\label{e u in H^2 and H^11 N=2,3,4...}
        \|u\|_{H^2(\Real)\cap H^{1,1}(\Real)}\leq C_M
    \end{equation}
    where the constant $C_M>0$ is depending on $M$ and $\lambda_k$ but not on $r_{\pm}$ and $c_k$.
\end{lem}
Now we argue analogously to \cite{Pelinovsky2016} and assume that a local solution $u(\cdot,t)\in H^2(\Real)\cap H^{1,1}(\Real)\cap\pazocal{G}$ provided by the results in \cite{Tsutsumi1980} and \cite{Hayashi1992} blows up in a finite time. That is
$$
\lim_{t\uparrow T_{\text{max}}}\|u(\cdot,t)\|_{H^2(\Real)\cap H^{1,1}(\Real)}=\infty
$$
for a maximal existence time $T_{\text{max}}>0$. By (\ref{e u in H^2 and H^11 N=2,3,4...}) we conclude
$$
\lim_{t\uparrow T_{\text{max}}}\left[\|r_{+}(\cdot,t)\|_{H^1(\Real)\cap L^{2,1}(\Real)}+\|r_{-}(\cdot,t)\|_{H^1(\Real)\cap L^{2,1}(\Real)}+\sum_{k=1}^N|c_k|\right]=\infty,
$$
which contradicts the time evolution of the reflection coefficient and of the norming constants given in Lemma \ref{l time dependence scattering data}. This argument yields the proof of Theorem \ref{t main}.
