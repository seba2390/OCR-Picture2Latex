\subsection{Scattering data}\label{ss scattering data}
We recall that $\varphi_{\pm}(\lambda;x)e^{-\ii x\lambda^2}$ and $\phi_{\pm}(\lambda;x)e^{\ii x\lambda^2}$ are solutions of the spectral problem (\ref{e Lax1}) with boundary condition (\ref{e asymptotics psi}). Taking into account $\tr(\sigma_3)=\tr(Q)=0$ we find
\begin{equation}\label{e det phi psi =1}
    \lim_{x\to\pm\infty}\det[\varphi_{\pm} (\lambda;x)e^{-\ii x\lambda^2},\phi_{\pm}(\lambda;x)e^{+\ii x\lambda^2}]=1
\end{equation}
for all $\lambda^2\in\Real$ and $x\in\Real$. Thus, in particular $\varphi_{+}e^{-\ii x\lambda^2}$ and $\phi_{+}e^{\ii x\lambda^2}$ are linearly independent and by ODE theory they form a basis of the space of solutions of the spectral problem (\ref{e Lax1}). This enables us to express the "$-$" Jost functions in terms of the "$+$" Jost functions for every $\lambda^2\in\Real$ and $x\in\Real$. According to that, there exist coefficients $\alpha,\beta,\gamma,\delta$ which satisfy:
\begin{equation}\label{e scattering relation}
    \begin{aligned}
         \varphi_{-}(\lambda;x)e^{-\ii x\lambda^2}&&=&& \alpha(\lambda)\varphi_{+}(\lambda;x)e^{-\ii x\lambda^2} &&+&&\beta(\lambda) \phi_{+}(\lambda;x)e^{\ii x\lambda^2},\\
         \phi_{-}(\lambda;x)e^{\ii x\lambda^2}&&=&& \gamma(\lambda)\varphi_{+}(\lambda;x)e^{-\ii x\lambda^2} &&+&& \delta(\lambda) \phi_{+}(\lambda;x)e^{\ii x\lambda^2}.
    \end{aligned}
\end{equation}
The matrix
$
\left[
  \begin{array}{cc}
    \alpha & \beta \\
    \gamma & \delta \\
  \end{array}
\right]
$
is referred to as the \emph{transfer matrix} in the literature and (\ref{e scattering relation}) is called \emph{scattering relation}. By (\ref{e det phi psi =1}), we verify that the determinant of the transfer matrix equals one. By Cramer's rule we find
\begin{equation}\label{e def alpha beta}
    \begin{aligned}
        \alpha(\lambda)&=\det[\varphi_{-}(\lambda;x), \phi_{+}(\lambda;x)],\\[2pt]
        \beta(\lambda)&=\det[\varphi_{+} (\lambda;x)e^{-\ii x\lambda^2}, \varphi_{-}(\lambda;x)e^{-\ii x\lambda^2}].
    \end{aligned}
\end{equation}
Making again use of $\tr(\sigma_3)=\tr(Q)=0$, we justify that $\alpha$ and $\beta$ indeed do not depend on $x$. Moreover $\alpha$ can be analytically extended to the first and third quadrant, where $\im(\lambda^2)>0$, which follows from the analytic properties of the Jost functions $\varphi_{-}$, $\phi_{+}$ in this domain. Furthermore, from the symmetry
\begin{equation}\label{e symmetries phi varphi 2}
    \phi_{\pm}(\overline{\lambda};x)=
    \left[
      \begin{array}{cc}
        0 & -1 \\
        1 & 0 \\
      \end{array}
    \right]\overline{\varphi_{\pm}(\lambda;x)},
\end{equation}
which are direct consequences of integral equations (\ref{e volterra phi varphi}), we can derive from the scattering relation (\ref{e scattering relation}) the following conservation law:
\begin{equation}\label{e alpha^2+beta^2}
    \left\{
      \begin{array}{ll}
        |\alpha(\lambda)|^2+|\beta(\lambda)|^2=1, \quad\lambda\in\Real,\\
        |\alpha(\lambda)|^2-|\beta(\lambda)|^2=1, \quad\lambda\in\ii\Real.
      \end{array}
    \right.
\end{equation}
As pointed out in \cite{Pelinovsky2016} this is indicating that the DNLS equation combines elements of the focusing and as well of the defocusing cubic NLS equation.

%We can set $x=0$ in (\ref{e def alpha beta}) and find
%\begin{eqnarray*}
%% \nonumber to remove numbering (before each equation)
%  \alpha(\lambda) &=& \det\left\{
%  \left[
%    \begin{array}{cc}
%      1 & 0 \\
%      0 & 2\ii\lambda \\
%    \end{array}
%  \right]\varphi_{-}(\lambda;0),
%  \left[
%    \begin{array}{cc}
%      \frac{1}{2\ii\lambda} & 0 \\
%      0 & 1 \\
%    \end{array}
%  \right]\phi_{+}(\lambda;0)
%  \right\} \\
%   &=& \det\left\{
%  \left[
%    \begin{array}{cc}
%      1 & 0 \\
%      0 & 2\ii\lambda \\
%    \end{array}
%  \right]\varphi_{-}(\lambda;0)-M_{-}^{\infty}(0)
%  \left(
%    \begin{array}{c}
%      \!\!1\!\! \\
%      \!\!\overline{u}(0)\!\! \\
%    \end{array}
%  \right)
%  ,
%  \left[
%    \begin{array}{cc}
%      \frac{1}{2\ii\lambda} & 0 \\
%      0 & 1 \\
%    \end{array}
%  \right]\phi_{+}(\lambda;x)-N_{+}^{\infty}(0)
%  \left(
%    \begin{array}{c}
%      \!\!0\!\! \\
%      \!\!1\!\! \\
%    \end{array}
%  \right)
%  \right\}\\
%  &&\qquad+\det\left\{
%  \left[
%    \begin{array}{cc}
%      1 & 0 \\
%      0 & 2\ii\lambda \\
%    \end{array}
%  \right]\varphi_{-}(\lambda;0)-M_{-}^{\infty}(0)
%  \left(
%    \begin{array}{c}
%      \!\!1\!\! \\
%      \!\!\overline{u}(0)\!\! \\
%    \end{array}
%  \right)
%  ,
%  N_{+}^{\infty}
%  \left(
%    \begin{array}{c}
%      \!\!0\!\! \\
%      \!\!1\!\! \\
%    \end{array}
%  \right)\right\}\\
%  &&\qquad\qquad+\det\left\{
%  M_{-}^{\infty}(0)
%  \left(
%    \begin{array}{c}
%      \!\!1\!\! \\
%      \!\!\overline{u}(0)\!\! \\
%    \end{array}
%  \right)
%  ,
%  \left[
%    \begin{array}{cc}
%      \frac{1}{2\ii\lambda} & 0 \\
%      0 & 1 \\
%    \end{array}
%  \right]\phi_{+}(\lambda;x)-N_{+}^{\infty}(0)
%  \left(
%    \begin{array}{c}
%      \!\!0\!\! \\
%      \!\!1\!\! \\
%    \end{array}
%  \right)
%  \right\}+\alpha_{\infty},
%\end{eqnarray*}
%where
%\begin{equation*}
%    \alpha_{\infty}:=\det\left\{M_{-}^{\infty}(0)
%  \left(
%    \begin{array}{c}
%      \!\!1\!\! \\
%      \!\!\overline{u}(0)\!\! \\
%    \end{array}
%  \right)
%  ,N_{+}^{\infty}(0)
%  \left(
%    \begin{array}{c}
%      \!\!0\!\! \\
%      \!\!1\!\! \\
%    \end{array}
%  \right)
%  \right\}=e^{\frac{1}{2\ii}\int_{\Real}|u(y)|^2dy}.
%\end{equation*}
%Then, by similar expressions for $\lambda\beta(\lambda)$ and $\lambda^{-1}\beta(\lambda)$ we can deduce from Corollary \ref{c properties varphi phi} the following Lemma (see \cite{Pelinovsky2016}):
%\begin{lem}\label{l alpha in H^1}
%    If $u\in H^{1,1}(\Real)$, then
%    \begin{equation}\label{e alpha,beta in H^1}
%        \alpha(\lambda)-\alpha_{\infty},\; \lambda\beta(\lambda),\; \lambda^{-1}\beta(\lambda)\in H^1_z(\Real).
%    \end{equation}
%    Moreover, if $u\in H^2(\Real)\cap H^{1,1}(\Real)$, then
%    \begin{equation}\label{e beta in L^21}
%         \lambda\beta(\lambda),\; \lambda^{-1}\beta(\lambda)\in L^{2,1}_z(\Real).
%    \end{equation}
%\end{lem}
We now continue with the definition of the reflection coefficient:
\begin{equation}\label{e def r}
    r(\lambda)=\frac{\beta(\lambda)}{\alpha(\lambda)}.
\end{equation}
This definition makes sense for every $\lambda^2\in\Real$, if $\alpha$ admits no zeros on $\Real\cup\ii\Real$, but we can not expect generally that $\alpha$ behaves like that. Therefore we define the following set:
\begin{equation}\label{e def no resonances}
    \pazocal{R}:=\set{u\in H^{1,1}(\Real):\exists A>0,|\alpha(\lambda)|>A\text{ for every }\lambda\in\Real\cup\ii\Real}
\end{equation}
Zeroes $\lambda\in\Real\cup\ii\Real$ of $\alpha$ are called \emph{resonances} in \cite{Pelinovsky2016}. Hence, the set $\pazocal{R}$ consists of those potentials, which do not admit resonances of the linear equation (\ref{e Lax1}). Let us assume from now on that $u\in \pazocal{R}$. Then,  we can rewrite the scattering relation (\ref{e scattering relation}) in the following way:
\begin{equation}\label{e alternative scattering relation}
    \Phi_+(\lambda;x)=\Phi_-(\lambda;x)(1+S(\lambda;x)), \quad\lambda^2\in\Real,
\end{equation}
where the matrices $\Phi_{\pm}$ and $S$ are given by
\begin{equation}\label{e def Phi}
    \Phi_+(\lambda;x):=
    \left[\frac{\varphi_{-}(\lambda;x)} {\alpha(\lambda)},\phi_{+}(\lambda;x)
    \right],\quad
    \Phi_-(\lambda;x):=
    \left[\varphi_{_+}(\lambda;x) ,\frac{\phi_{-}(\lambda;x)} {\overline{\alpha(\overline{\lambda})}}
    \right],
\end{equation}
and
\begin{equation}\label{e def S}
    S(\lambda;x):=
    \left\{
      \begin{array}{ll}
        \left[
          \begin{array}{cc}
            |r(\lambda)|^2 & \overline{r(\lambda)} e^{-2\ii x\lambda^2} \\
            r(\lambda) e^{2\ii x\lambda^2} & 0 \\
          \end{array}
        \right]
        , & \hbox{for }\lambda\in\Real,\vspace{2mm} \\ \left[
          \begin{array}{cc}
            -|r(\lambda)|^2 & -\overline{r(\lambda)} e^{-2\ii x\lambda^2} \\
            r(\lambda) e^{2\ii x\lambda^2} & 0 \\
          \end{array}
        \right]
        , & \hbox{for }\lambda\in\ii\Real.
      \end{array}
    \right.
\end{equation}
It is clear from the representation (\ref{e def alpha beta}) that $\alpha$ has an analytic continuation in the first and third quadrants of the $\lambda$ plane. Therefore the function $\Phi_+$ defined in (\ref{e def Phi}) can be continued analytically in the first and third quadrants, as long as there are no zeros $\lambda_0$ of the continuation of $\alpha$ with $\im(\lambda_0^2)>0$. Under the same assumption, the function $\Phi_-$ in (\ref{e def Phi}) can be analytically continued in the second and fourth quadrant. From now on we want to allow that $\alpha(\lambda)$ has finite many simple zeroes. That is $\alpha(\lambda_k)=0$ and $\alpha'(\lambda_k)\neq0$ for a finite number of pairwise different $\lambda_1,...,\lambda_N$ which are assumed to lie in the first quadrant. Note that, if $\alpha(\lambda_k)=0$, then also $\alpha(-\lambda_k)=0$. Henceforth, the continuations of $\Phi_{\pm}$ are merely meromorphic. They admit simple poles at the zeros of $\alpha$, since  $\alpha'(\lambda_k)\neq0$ for $k=1,...,N$. The prime denotes the derivative with respect to $\lambda$. We find:
\begin{equation*}
    \res_{\lambda=\pm\lambda_k}
    \Phi_+(\lambda;x)=
    \left[\frac{\varphi_{-}(\pm\lambda_k;x)} {\pm\alpha'(\lambda_k)},\;0\;
    \right].
\end{equation*}
By (\ref{e def alpha beta}), the meaning of the zeros of $\alpha$ is the following. If $\alpha(\lambda_k)=0$, then by (\ref{e def alpha beta}) the $\Compl^2$ vectors $\varphi_{-}(\lambda_k;x)e^{-\ii x\lambda_k^2}$ and $\phi_{+}(\lambda_k;x)e^{\ii x\lambda_k^2}$ are linear dependent for every $x\in\Real$. Hence,
\begin{equation}\label{e def gamma}
    \varphi_{-}(\pm\lambda_k;x)=\pm\gamma_k\; e^{2\ii x\lambda_k^2}\; \phi_{+}(\pm\lambda_k;x)
\end{equation}
for some complex constant $\gamma_k\in\Compl\setminus\set{0}$. We will refer to $\gamma_k$ as the \emph{norming constant}. The norming constants do not depend on $x$. Indeed, differentiating (\ref{e def gamma}) with respect to $x$ and using the fact that  $\varphi_{-}(\lambda_k;x)e^{-\ii x\lambda_k^2}$ and $\phi_{+}(\lambda_k;x)e^{\ii x\lambda_k^2}$ are solutions of the spectral problem (\ref{e Lax1}), we easily obtain $\partial_x \gamma_k=0$. Note also that due to the symmetry (\ref{e symmetries phi varphi 1}) the cases $+\lambda_k$ and $-\lambda_k$ do have the same norming constants upon a minus sign.  Combining (\ref{e def gamma}) and the above residue calculation we find
\begin{equation}\label{e res Phi}
    \res_{\lambda=\pm\lambda_k}
    \Phi_+(\lambda;x)=
    \left[\frac{\pm\gamma_k \; e^{2\ii x\lambda_k^2}} {\alpha'(\pm\lambda_k)}\phi_{+}(\pm\lambda_k;x),\;0\;
    \right]=
    \lim_{\lambda\to\pm\lambda_k}\Phi_+(\lambda;x)
    \left[
      \begin{array}{cc}
        0 & 0\\
        \frac{\gamma_k \; e^{2\ii x\lambda_k^2}} {\alpha'(\lambda_k)} & 0 \\
      \end{array}
    \right].
\end{equation}
Correspondingly, we can compute an analogue relation for the residue of $\Phi_-$ at $\pm\overline{\lambda}_k$.\\
By a theorem of complex analysis (see, e.g., \cite[Theorem 3.2.8]{Ablowitz2003}), the zeroes of $\alpha$ must be isolated. In addition, by \cite[Lemma 4]{Pelinovsky2016} we know $\alpha(\lambda)\to\alpha_{\infty}\neq 0$ as $|\lambda|\to\infty$. Thus, we conclude that the zeroes of $\alpha(\lambda)$ in the first quadrant form a finite set $\set{\lambda_1,...,\lambda_N}$. But the essential assumption $\alpha'(\lambda_k)\neq 0$ is generally not expectable and give rise to the following definition:
\begin{equation}\label{e def no eigenvalues}
    \pazocal{E}:=\set{u\in H^{1,1}(\Real):\alpha'(\lambda_k)\neq 0\text{ for all zeroes }\lambda_k\text{ of }\alpha\text{ with } \im(\lambda_k^2)>0}
\end{equation}
From now on, additionally to $u\in\pazocal{R}$,  we assume $u\in\pazocal{G}:=\pazocal{R}\cap\pazocal{E}$. The elements of $\pazocal{G}$ are called \emph{generic potentials} according to the classical paper \cite{Beals1984}. As remarked by the authors in \cite[Remark 5]{Pelinovsky2016}, we have $u\in\pazocal{G}$ if
\begin{equation*}\label{e estimate for no eigenv and reso}
    \|u\|_{L^2}^2+\sqrt{\|u\|_{L^1}( 2\|\partial_x u\|_{L^1}+\|u\|_{L^3}^3)}<1.
\end{equation*}
The set $\pazocal{G}$ is open and, moreover, dense in $H^{1,1}(\Real)$. Due to the availability of the transformation (\ref{e def M N}), this can be deduced from \cite{Beals1984} as explained in \cite[Proposition 4]{PelinShimaSaal2017}. However, any soliton or multi soliton is contained in $\pazocal{G}$. For those explicit solutions, the expression
\begin{equation*}
    \|u\|_{L^2}^2+\sqrt{\|u\|_{L^1}( 2\|\partial_x u\|_{L^1}+\|u\|_{L^3}^3)}
\end{equation*}
can be arbitrary large.
\medskip\\
Using the transformation (\ref{e def M N}) it is shown in \cite{Pelinovsky2016} that for $u\in H^2(\Real)\cap H^{1,1}(\Real)$ the following holds.
\begin{equation}\label{e def Phi infty}
    \Phi_{\pm}(\lambda;x)\to\Phi_{\infty}(x):=
    \left[
      \begin{array}{cc}
        e^{-\frac{1}{2\ii}\int^{+\infty}_x|u(y)|^2dy} & 0 \\
        0 & e^{-\frac{1}{2\ii}\int_{-\infty}^x|u(y)|^2dy} \\
      \end{array}
    \right]\quad\text{as }|\lambda|\to\infty.
\end{equation}
The limit has to be taken along a contour  in the corresponding domain of analyticity.

The alternative scattering relation (\ref{e alternative scattering relation}), the residue condition (\ref{e res Phi}) and finally the asymptotic behavior (\ref{e def Phi infty}) set up a \rh.
%\begin{samepage}
%\begin{framed}
%\begin{rhp}\label{rhp phi}
%Find for each $x\in\Real$, a $2\times 2$-matrix valued function $\Compl\ni \lambda\mapsto \Phi(\lambda;x)$ which satisfies
%\begin{enumerate}[(i)]
%  \item $\Phi(\lambda;x)$ is meromorphic in $\Compl\setminus(\Real\cup\ii\Real)$ (with respect to the parameter $\lambda$).
%  \item $\Phi(\lambda;x)=\Phi_{\infty}(x) +\mathcal{O}\left(\frac{1}{\lambda}\right)$ as $|\lambda|\to\infty$.
%  \item The non-tangential boundary values $\Phi_{\pm}(z;x,t)$ exist for $\lambda^2\in\Real$ and satisfy the jump relation (\ref{e alternative scattering relation}).
%  \item $\Phi$ has simple poles at $\pm\lambda_1,...,\pm\lambda_N, \pm\overline{\lambda}_1,...,\pm \overline{\lambda}_N$ with
%      \begin{equation*}
%        \begin{aligned}
%          \res_{\lambda=\pm\lambda_k}
%          \Phi(\lambda;x)&=
%          \lim_{\lambda\to\pm\lambda_k}\Phi(\lambda;x)
%          \left[
%            \begin{array}{cc}
%              0 & 0\\
%              \frac{\gamma_k \; e^{2\ii x\lambda_k^2}} {\alpha'(\lambda_k)} & 0 \\
%            \end{array}
%          \right],\\
%          \res_{\lambda=\pm\overline{\lambda}_k}
%          \Phi(\lambda;x)&=
%          \lim_{\lambda\to\pm \overline{\lambda}_k}\Phi(\lambda;x)
%          \left[
%            \begin{array}{cc}
%              0 & \frac{-\overline{\gamma}_k \; e^{-2\ii x\overline{\lambda}_k^2}} {\overline{\alpha' (\overline{\lambda_k})}}\\
%              0 & 0 \\
%            \end{array}
%          \right].
%        \end{aligned}
%      \end{equation*}
%\end{enumerate}
%\end{rhp}
%\end{framed}
%\end{samepage}
Since that \rh is somewhat unsuitable to show the existence of the inverse Scattering map, we turn again to the Zhakarov-Shabat type Jost functions $M_{\pm}$ and $N_{\pm}$ (see (\ref{e def M N}), which are functions of $z$, where we recall $z=\lambda^2$. Due to $\alpha(\lambda)=\alpha(-\lambda)$, it is alowed to define $a(z):=\alpha(\lambda)$. Of course, if $\pm\lambda_k\neq0$ are (simple) zeroes of $\alpha$, then $z_k:=\lambda_k^2$ is a (simple) zero of $a$. In order to transfer the jump condition (\ref{e alternative scattering relation}) to the Jost functions $M_{\pm}$ and $N_{\pm}$, one more explicit definition is needed:
\begin{equation}\label{e def P}
    P_{\pm}(z;x):=\frac{1}{2\ii\lambda}
    T_1(\lambda;x)T_2^{-1}(\lambda;x)N_{\pm}(z;x)=
    -\frac{1}{4z}
    \left[
      \begin{array}{cc}
        1 & u(x) \\
        -\overline{u}(x) & -|u(x)|^2-4z \\
      \end{array}
    \right]N_{\pm}(z;x).
\end{equation}
In \cite[Lemma 5]{Pelinovsky2016} it is shown, that there is no singularity in (\ref{e def P}) and moreover, $P_{\pm}(z;x)$ satisfy the following limits as $|\im(z)|\to\infty$ along a contour in the domains of their analyticity:
\begin{equation*}
    \lim_{|z|\to\infty}P_{\pm}(z;x)=
    \left(
      \begin{array}{c}
        0\\
        N_{\pm}^{\infty}(x) \\
      \end{array}
    \right).
\end{equation*}
Now we are ready to define the analogue of (\ref{e def Phi}). Instead of $\lambda\in\Real\cup\ii\Real$, now we have $z\in\Real$ and set
\begin{equation}\label{e def pi}
    \pi_+(z;x):=
    \left[\frac{M_{-}(z;x)} {a(z)},P_{+}(z;x)
    \right],\quad
    \pi_-(z;x):=
    \left[M_{_+}(z;x) ,\frac{P_{-}(z;x)} {\overline{a(\overline{z})}}
    \right].
\end{equation}
These definitions entail the following analogue of (\ref{e alternative scattering relation}) which can be checked by elementary calculations:
\begin{equation}\label{e jump of pi}
    \pi_+(z;x)=\pi_-(z;x)(1+R(z;x)), \quad z\in\Real.
\end{equation}
Herein the new jump matrix $R$ which includes new reflection coefficients $r_{\pm}$, is defined by
\begin{equation*}
    R(z;x):=
        \left[
          \begin{array}{cc}
            \overline{r}_+(z)r_-(z) & e^{-2\ii xz}\overline{r}_+(z) \\
            e^{2\ii xz}r_-(z) & 0 \\
          \end{array}
        \right].
\end{equation*}
The new reflection coefficients are given by
\begin{equation}\label{e def r pm}
    r_{+}(z):=-\frac{\beta(\lambda)} {2\ii\lambda\alpha(\lambda)},\quad
    r_{-}(z):=\frac{2\ii\lambda\beta(\lambda)} {\alpha(\lambda)},\quad z\in\Real.
\end{equation}
We have the following Lemma \cite{Pelinovsky2016}.
\begin{lem}\label{l r pm in H1 and L^21}
    If $u\in H^2(\Real)\cap H^{1,1}(\Real)\cap\pazocal{R}$, then $r_{\pm}\in H^1(\Real) \cap L^{2,1}(\Real)$.
\end{lem}
Moreover, we found directly from the definition (\ref{e def r pm}) that $r_+$ and $r_-$ are connected by
\begin{equation}\label{e relation r+ r-}
    r_-(z)=4zr_+(z),\quad z\in\Real.
\end{equation}
Furthermore, $\overline{r}_+(z)r_-(z)=|r(\lambda)|^2$ if $z>0$, whereas $\overline{r}_+(z)r_-(z)=-|r(\lambda)|^2$ if $z<0$. Additionally, using (\ref{e alpha^2+beta^2}) we obtain $1-|r(\lambda)|^2=|\alpha(\lambda)|^{-2}$. Thus, we have
\begin{equation}\label{e r constraint}
   \left\{
     \begin{array}{ll}
       1+\overline{r}_+(z)r_-(z)\geq1, & z>0, \\
       1+\overline{r}_+(z)r_-(z)\geq c_0^2, & z<0,
     \end{array}
   \right.
\end{equation}
where $c_0^{-1}:=\sup_{\lambda\in\ii\Real}|\alpha(\lambda)|$. The constraint (\ref{e r constraint}) is used in \cite{Pelinovsky2016} to obtain a unique solution to the \rh \ref{rhp m} below.\par
Analytic continuations of $\pi_{\pm}$ in $\Compl^{\pm}$ exist if there is no $z\in\Compl$ such that $a(z)=0$. Otherwise we have analogously to (\ref{e def gamma})
\begin{equation*}
    M_{-}(z_k;x)=2\ii\lambda_k\gamma_k\; e^{2\ii x z_k}\; P_{+}(z_k;x)
\end{equation*}
with the same $\gamma_k$ as in (\ref{e def gamma}). Denoting the meromorphic continuations of $\pi_{\pm}(\cdot;x)$ with the same letters we can verify the following residue condition:
\begin{equation}\label{e res pi}
    \res_{z=z_k}\pi_+(z;x)=\lim_{z\to z_k}\pi_+(z;x)
          \left[
            \begin{array}{cc}
              0 & 0 \\
              2\ii\lambda_kc_k e^{2\ii xz_k} & 0
            \end{array}
          \right],
\end{equation}
where we set $c_k:=\gamma_k/a'(z_k)$. Correspondingly we can compute an analogue relation for the residuum of $\pi_-$ at $\overline{z}_k$. Next, we have
\begin{equation*}
    \pi_{\pm}(\lambda;x)\to\Phi_{\infty}(x)\quad\text{as }|\lambda|\to\infty,
\end{equation*}
similarly to (\ref{e def Phi infty}).
We obtain our final Riemann--Hilbert problem if we normalize the boundary condition at infinity:
\begin{equation}\label{e def m}
    m(z;x):=
    \left\{
      \begin{array}{ll}
        \,[\Phi_{\infty}(x)]^{-1}\pi_+(z;x), & z\in\Compl^+, \\
        \,[\Phi_{\infty}(x)]^{-1}\pi_-(z;x), & z\in\Compl^-.
      \end{array}
    \right.
\end{equation}
The multiplication from the left by the diagonal matrix $[\Phi_{\infty}(x)]^{-1}$ changes neither the analytic properties of $\pi_{\pm}$ nor the jump or residuum conditions. Therefore, the function $m$ defined in (\ref{e def m}) solves the following Riemann--Hilbert problem:
\begin{samepage}
\begin{framed}
\begin{rhp}\label{rhp m}
Find for each $x\in\Real$ a $2\times 2$-matrix valued function $\Compl\ni z\mapsto m(z;x)$ which satisfies
\begin{enumerate}[(i)]
  \item $m(z;x)$ is meromorphic in $\Compl\setminus\Real$ (with respect to the parameter $z$).
  \item $m(z;x)=1+\mathcal{O}\left(\frac{1}{z}\right)$ as $|z|\to\infty$.
  \item The non-tangential boundary values $m_{\pm}(z;x)$ exist for $z\in\Real$ and satisfy the jump relation
      \begin{equation}\label{e jump}
        m_+=m_-(1+R),\quad\text{where }
        R(z;x):=
        \left[
          \begin{array}{cc}
            \overline{r}_+(z)r_-(z) & e^{-2\ii xz}\overline{r}_+(z) \\
            e^{2\ii xz}r_-(z) & 0 \\
          \end{array}
        \right]
      \end{equation}
  \item $m$ has simple poles at $z_1,...,z_N,\overline{z}_1,...,\overline{z}_N$ with
      \begin{equation*}%\label{e Res}
        \begin{aligned}
          \res_{z=z_k}m(z;x)&=\lim_{z\to z_k}m(z;x)
          \left[
            \begin{array}{cc}
              0 & 0 \\
              2\ii\lambda_kc_k e^{2\ii xz_k} & 0
            \end{array}
          \right],\\
          \res_{z=\overline{z_k}}m(z;x)&=\lim_{z\to \overline{z}_k}m(z;x)
          \left[
            \begin{array}{cc}
              0 & \frac{-\overline{c}_k}{2\ii\lambda_k} e^{-2\ii x\overline{z}_k} \\
              0 & 0
            \end{array}
          \right].
        \end{aligned}
      \end{equation*}
\end{enumerate}
\end{rhp}
\end{framed}
\end{samepage}
We will use the notation
\begin{equation*}
    \mathcal{S}(u)=\set{r_{\pm};\lambda_1,... ,\lambda_N;c_1,...,c_N}
\end{equation*}
and call $\mathcal{S}$ the scattering data of $u$. They consist of the \emph{reflection coefficients} $r_{\pm}$ which satisfy the constraints (\ref{e relation r+ r-}) and (\ref{e r constraint}), the \emph{poles} $z_k:=\lambda_k^2$ and the \emph{norming constants} $c_k=\gamma_k/a'(z_k)$. $\mathcal{S}$ is all information we need to know about $u$ to formulate the \rh\ref{rhp m}. In the rest of this paper we treat the problem to define the inverse map $\set{r_{\pm};\lambda_1,... ,\lambda_N;c_1,...,c_N}\mapsto u$. Therefore we will solve \rh \ref{rhp m} and apply the following reconstruction formulas:
\begin{equation}\label{e rec 1}
    u(x)e^{\ii\int_{+\infty}^x|u(y)|^2dy}= -4\lim_{|z|\to\infty}z\;[m(z;x)]_{12}
\end{equation}
and
\begin{equation}\label{e rec 2}
    e^{-\frac{1}{2\ii}\int_{+\infty}^x|u(y)|^2dy} \partial_x\left(\overline{u}(x) e^{\frac{1}{2\ii}\int_{+\infty}^x|u(y)|^2dy}\right) =2\ii\lim_{|z|\to\infty}z\;[m(z;x)]_{21}.
\end{equation}
Both, (\ref{e rec 1}) and (\ref{e rec 2}), are justified in \cite{Pelinovsky2016} and the key of Inverse Scattering. By $[\cdot]_{ij}$ we denote the $i$-$j$-component of the matrix in the brackets.\\
The miraculous fact about the forward scattering is the trivial time evolution of the scattering data if the potential $u(x,t)$ evolves accordingly to the DNLS equation:
\begin{lem}\label{l time dependence scattering data}
    Under the assumption that an initial datum $u_0\in H^2(\Real)\cap H^{1,1}(\Real)\cap \pazocal{G}$ admits a (local) solution $u(\cdot,t)\in H^2(\Real)\cap H^{1,1}(\Real)$ to the Cauchy problem (\ref{e dnls}) for $t\in[0,T]$, the scattering data of $u(\cdot,t)$ are given by
    \begin{equation}\label{e time dependence scattering data}
        \mathcal{S}_t(u)=\set{r_{\pm}(z;t)=r_{\pm}(z;0) e^{4\ii z^2t};\lambda_1,... ,\lambda_N;c_1(0) e^{4\ii \lambda_1^4t},...,c_N(0) e^{4\ii \lambda_N^4t}},
    \end{equation}
    where
    \begin{equation*}
        \mathcal{S}_0(u)=\set{r_{\pm}(z;0);\lambda_1,... ,\lambda_N;c_1(0),...,c_N(0)}
    \end{equation*}
    are defined to be the scattering data of $u_0$. In particular, the set $\pazocal{G}$ is invariant under the flow of the DNLS equation, $r_{\pm}(\cdot;t)\in H^1(\Real)\cap L^{2,1}(\Real)$ for every $t\in[0,T]$, and, finally, (\ref{e relation r+ r-}) and (\ref{e r constraint}) remain valid.
\end{lem}
The proof of this Lemma is given in \cite[Section 5]{Pelinovsky2016} and we skip it here. Plugging the time dependence (\ref{e time dependence scattering data}) into the formulas of \rh \ref{rhp m} we obtain the dynamic Riemann--Hilbert problem for the DNLS equation.
\begin{samepage}
\begin{framed}
\begin{rhp}\label{rhp m dynamic}
Find for each $(x,t)\in\Real\times\Real$ a $2\times 2$-matrix valued function $\Compl\ni z\mapsto m(z;x,t)$ which satisfies
\begin{enumerate}[(i)]
  \item $m(z;x,t)$ is meromorphic in $\Compl\setminus\Real$ (with respect to the parameter $z$).
  \item $m(z;x,t)=1+\mathcal{O}\left(\frac{1}{z}\right)$ as $|z|\to\infty$.
  \item The non-tangential boundary values $m_{\pm}(z;x,t)$ exist for $z\in\Real$ and satisfy the jump relation
      \begin{equation*}%\label{e jump}
        m_+=m_-(1+R),\quad\text{where }
        R(z;x,t):=
        \left[
          \begin{array}{cc}
            \overline{r}_+(z)r_-(z) & e^{\overline{\phi}(z)}\overline{r}_+(z) \\
            e^{\phi(z)}r_-(z) & 0 \\
          \end{array}
        \right]
      \end{equation*}
      with $\phi(z):=2\ii xz+4\ii z^2t$.
  \item $m$ has simple poles at $z_1,...,z_N,\overline{z}_1,...,\overline{z}_N$ with
      \begin{equation}\label{e Res}
        \begin{aligned}
          \res_{z=z_k}m(z;x,t)&=\lim_{z\to z_k}m(z;x,t)
          \left[
            \begin{array}{cc}
              0 & 0 \\
              2\ii\lambda_kc_k e^{\phi_k} & 0
            \end{array}
          \right],\\
          \res_{z=\overline{z}_k}m(z;x,t)&=\lim_{z\to \overline{z}_k}m(z;x,t)
          \left[
            \begin{array}{cc}
              0 & \frac{-\overline{c}_k}{2\ii\lambda_k} e^{\overline{\phi}_k} \\
              0 & 0
            \end{array}
          \right],
        \end{aligned}
      \end{equation}
      where $\phi_k:=\phi(z_k)$.
\end{enumerate}
\end{rhp}
\end{framed}
\end{samepage}
\begin{rem}\label{r uniqueness + det=1}
    Without further theory we can observe that if \rh \ref{rhp m dynamic} is solvable, then the solution is unique. In order to show the uniqueness of solutions, we firstly find the following (trivial) Riemann Hilbert problem for the map $z\mapsto \det(m(z;x,t))$:
    \begin{equation*}
        \left\{
           \begin{array}{ll}
             \det(m(z;x,t))\text{ is an entire function with respect to the parameter }z,\\
             \det(m(z;x,t))\to 1,\text{ as }|z|\to\infty.
           \end{array}
         \right.
    \end{equation*}
    By Liouville's theorem we conclude
    \begin{equation}\label{e det m=1}
        \det(m(z;x,t))\equiv 1,\text{ for all }x,t\in\Real\text{ and }z\in \Compl.
    \end{equation}
    Hence, for a possible solution $m$ of \rh \ref{rhp m dynamic}, $[m(z;x,t)]^{-1}$ exists for all $x\in\Real$ and $z\in \Compl$. If we have a second solution $\widetilde{m}(z;x,t)$, the ratio $\widetilde{m}(z;x,t)[m(z;x,t)]^{-1}$ satisfies
    \begin{equation*}
        \left\{
           \begin{array}{ll}
             \widetilde{m}(z;x,t)[m(z;x,t)]^{-1}\text{ is an entire function with respect to the parameter }z,\\
             \widetilde{m}(z;x,t)[m(z;x,t)]^{-1}\to 1,\text{ as }|z|\to\infty,
           \end{array}
         \right.
    \end{equation*}
    such that $\widetilde{m}(z;x,t)[m(z;x,t)]^{-1}\equiv 1$.
\end{rem}
We end the subsection mentioning the following symmetry:
\begin{equation}\label{e symmetrie of m}
    m(z;x)=\frac{1}{4z}
    \left[
      \begin{array}{cc}
        w(x) & 1 \\
        -|w(x)|^2-4z & \overline{w}(x) \\
      \end{array}
    \right]
    \overline{m(\overline{z};x)}
    \left[
      \begin{array}{cc}
        0 & 1 \\
        4z & 0 \\
      \end{array}
    \right],
\end{equation}
where $w(x):= u(x)\,e^{\ii\int^x_{+\infty}|u(y)|^2dy}$. The symmetry (\ref{e symmetrie of m}) is obtained when one transfers the symmetry (\ref{e symmetries phi varphi 2}) to $\pi_{\pm}$ and $m$, respectively. 