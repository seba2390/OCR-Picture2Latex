\section{Inverse scattering without poles}\label{s inverse sc}
In this section we are dealing with \rh \ref{rhp m} in the case where $N=0$. Hence, $m$ has no pole in $\Compl\setminus\Real$ and is analytic in $\Compl\setminus\Real$. We recall the associated Riemann--Hilbert problem:\\
\begin{framed}
    \begin{rhp}\label{rhp m^0}
        Find for each $x\in\Real$ a $2\times 2$-matrix valued function $\Compl\ni z\mapsto m(z;x)$ which satisfies
        \begin{enumerate}[(i)]
          \item $m(z;x)$ is meromorphic in $\Compl\setminus\Real$ (with respect to the parameter $z$).
          \item $m(z;x)=1+\mathcal{O}\left(\frac{1}{z}\right)$ as $|z|\to\infty$.
          \item The non-tangential boundary values $m_{\pm}(z;x)$ exist for $z\in\Real$ and satisfy the jump relation
              \begin{equation}\label{e jump m^0}
                  m_+=m_-(1+R),\quad\text{where}\quad
                  R(z;x):=
                  \left(
                    \begin{array}{cc}
                       \overline{r}_+(z)r_-(z) & e^{-2\ii zx}\overline{r}_+(z) \\
                       e^{2\ii zx}r_-(z) & 0 \\
                    \end{array}
                  \right).
              \end{equation}
        \end{enumerate}
    \end{rhp}
\end{framed}
For any function $h\in L^p(\Real)$ with $1\leq p<\infty$, the Cauchy operator denoted by $\pazocal{C}$ is given by
\begin{equation*}
    \pazocal{C}(h)(z):=\frac{1}{2\pi\ii}\int_{\Real} \frac{h(s)}{s-z}ds,\quad z\in\Compl\setminus\Real.
\end{equation*}
When $z$ approaches to a point on the real line transversely from the upper and lower half planes, the Cauchy operator becomes the following projection operators:
\begin{equation*}
    \Ppm(h)(z):=\lim_{\eps\downarrow 0}\frac{1}{2\pi\ii}\int_{\Real} \frac{h(s)}{s-(z\pm\eps)}ds,\quad z\in\Real.
\end{equation*}
The following proposition summarizes all properties which are needed to establish the solvability of \rh \ref{rhp m^0} and furthermore to prove estimates on the solution.
\begin{prop}\label{p cauchy operator}
    \begin{enumerate}[(i)]
      \item For every $h\in L^p(\Real)$, $1\leq p<\infty$, the Cauchy operator $\pazocal{C}(h)$ is analytic off the real line.
      \item For $h\in L^1(\Real)$, $\pazocal{C}(h)(z)$ decays to zero as $|z|\to\infty$ and admits the asymptotic
          \begin{equation}\label{e lim z Ch(z)}
            \lim_{|z|\to\infty}z\pazocal{C}(h)(z)= -\frac{1}{2\pi\ii}\int_\Real h(s) ds,
          \end{equation}
          where the limit is taken either in $\Compl^+$ or $\Compl^-$.
      \item The projection operators $\Ppm$ are linear bounded operators $L^p(\Real)\to L^p(\Real)$ for each $p\in(1,\infty)$. For $p=2$ we have $\|\Ppm\|_{L^2\to L^2}=1$.
      \item For every $x_0\in\Real_+$ and every $r\in H^1(\Real)$, we have
          \begin{equation}\label{e sup P^pm r 1}
              \sup_{x\in(x_0,\infty)} \|\langle x\rangle\Ppm (r(z)e^{\mp2\ii zx})\|_{L^{2}_z(\Real)}\leq \|r\|_{H^1},
          \end{equation}
          where $\langle x\rangle:=\sqrt{1+|x|^2}$. In addition,
          \begin{equation}\label{e sup P^pm r 2}
              \sup_{x\in\Real} \|\Ppm (r(z)e^{\mp2\ii zx})\|_{L^{\infty}_z(\Real)}\leq \frac{1}{\sqrt{2}}\|r\|_{H^1}.
          \end{equation}
          Furthermore, if $r\in L^{2,1}(\Real)$, then
          \begin{equation}\label{e sup P^pm r 3}
              \sup_{x\in\Real}\|\Ppm (zr(z)e^{\mp2\ii zx})\|_{L^{\infty}_z(\Real)}\leq \frac{1}{\sqrt{2}}\|zr\|_{L^{2,1}}.
          \end{equation}
      \item (Sokhotski-Plemelj theorem) The following two identities hold:
      \begin{equation}\label{e Sokhotski-Plemelj}
        \begin{aligned}
            &\Pp-\Pm=\text{Id}_{L^p(\Real)},\\
            &\Pp+\Pm=-\ii\pazocal{H},
        \end{aligned}
      \end{equation}
      where $\pazocal{H}:L^p(\Real)\to L^p(\Real)$ is the Hilbert transform given by
      \begin{equation*}
        \pazocal{H}(h)(z):=\lim_{\eps\downarrow 0}\frac{1}{\pi}\left(\int_{-\infty}^{z-\eps}+ \int^{\infty}_{z+\eps}\right) \frac{h(s)}{s-z}ds,\quad z\in\Real.
      \end{equation*}
      \item Let $f_+$ and $f_-$ functions defined in the upper (lower) $\Compl$-plane. If $f_{\pm}$ is analytic in $\Compl^{\pm}$ and $f_{\pm}(z)\to 0$ as $|z|\to\infty$ for $\im(z)\gtrless0$, then
          \begin{equation}\label{e Ppm of analytic functions}
            \Ppm(f_{\mp})(z)=0,\qquad\Ppm(f_{\pm})(z)=\pm f_{\pm}(z),\quad z\in\Real.
          \end{equation}
    \end{enumerate}
\end{prop}
The Cauchy operator is useful to convert \rh \ref{rhp m^0} into an integral equation. Indeed, the jump condition (\ref{e jump m^0}) can be written as
\begin{equation*}
    (m_+(z;x)-1)-(m_-(z;x)-1)=m_-(z;x)R(z;x).
\end{equation*}
Applying $\Pp$ and $\Pm$ to this equation yields by (\ref{e Ppm of analytic functions}) the following integral equation
\begin{equation}\label{e integral equation for m+-}
    m_{\pm}(z;x)=1+\Ppm(m_-(\cdot;x)R(\cdot;x))(z),\quad z\in\Real,
\end{equation}
which represents the solution of \rh \ref{rhp m^0} on the real line. The following Lemma ensures the solvability of \rh \ref{rhp m} (see Corollary 6 and Lemma 9 in \cite{Pelinovsky2016}):
\begin{lem}\label{l solvability of RHP N=0}
    Let $r_{\pm}\in H^1(\Real)\cap L^{2,1}(\Real)$ such that the relation (\ref{e relation r+ r-}) and the constraint (\ref{e r constraint}) hold. Then there exists an unique solution $m_{\pm}$ of the system of integral equations (\ref{e integral equation for m+-}). Moreover there exists a positive constant $C$ that depends on $\|r_{\pm}\|_{L^{\infty}}$ only such that $m_{\pm}$ enjoys the estimate
    \begin{equation*}
        \|m_{\pm}(\cdot;x)-1\|_{L^2}\leq C(\|r_{+}\|_{L^2} +\|r_{-}\|_{L^2})
    \end{equation*}
    for every $x\in\Real$.
\end{lem}
This Lemma yields indeed a solution of \rh \ref{rhp m^0}, since the analytic continuation of $m_{\pm}$ is found by Proposition \ref{p cauchy operator} (ii):
\begin{equation}\label{e RHP solution formula}
    m(z;x)=1+\frac{1}{2\pi\ii}\int_{\Real} \frac{m_-(y;x)R(y;x)}{y-z}dy,\quad z\in\Compl\setminus \Real.
\end{equation}
Alternatively we can factorize $1+R=(1+R_+)(1+R_-)$ with
\begin{equation}\label{e def R+ and R-}
    R_+(z;x)=
    \left(
      \begin{array}{cc}
            0 & e^{\overline{\phi}(z)}\overline{r}_+(z) \\
            0 & 0 \\
      \end{array}
    \right),\qquad
    R_-(z;x)=
    \left(
      \begin{array}{cc}
            0 & 0 \\
            e^{\phi(z)}r_-(z) & 0 \\
      \end{array}
    \right).
\end{equation}
The jump relation (\ref{e jump}) then becomes $m_+ -m_-=m_-R_+ +m_+R_-$ and applying again $\Ppm$ to this equation yields us
\begin{equation}\label{e alternative RHP solution formula}
    m(z;x)=1+\frac{1}{2\pi\ii}\int_{\Real} \frac{m_-(y;x)R_+(y;x)+ m_+(y;x)R_-(y;x)}{y-z}dy.
\end{equation}
In component form, for the non-tangential limits $z\to\Real$, we find
\begin{equation}\label{e component RHP solution formula}
    m_{\pm}(z;x)=1+
    \left[
      \begin{array}{cc}
        \Ppm\left([m_+(z;x)]_{12} r_-(z) e^{2\ii zx}\right)(z) & \Ppm\left([m_-(z;x)]_{11} \overline{r}_+(z) e^{-2\ii zx}\right)(z) \\
        \Ppm\left([m_+(z;x)]_{22} r_-(z) e^{2\ii zx}\right)(z) & \Ppm\left([m_-(z;x)]_{21} \overline{r}_+(z) e^{-2\ii zx}\right)(z) \\
      \end{array}
    \right].
\end{equation}
In the further analysis of \rh \ref{rhp m} we will meet expressions of the form
\begin{equation}\label{e def I_1,2}
    \begin{aligned}
        &I_1(r)(x):=\frac{1}{2\pi\ii} \int_{\Real}[m_-(y;x)-1]_{11} r(y) e^{-2\ii yx}dy,\\
        &I_2(r)(x):=\frac{1}{2\pi\ii} \int_{\Real}[m_+(y;x)-1]_{22}r(y) e^{2\ii yx}dy,
    \end{aligned}
\end{equation}
where $m_{\pm}$ are the unique solutions of the system of integral equations (\ref{e integral equation for m+-}) and $r$ is some given function.
\begin{prop}\label{p bound <x>^2I}
    Suppose that the assumptions of Lemma \ref{l solvability of RHP N=0} are fulfilled and take $r\in H^1(\Real)\cap L^{2,1}(\Real)$. Then the functionals defined in (\ref{e def I_1,2}) satisfy the bound
    \begin{equation}\label{e bound <x>^2I}
        \begin{aligned}
            &\|I_1(r)\|_{H^1(\Real_+)\cap L^{2,1}(\Real_+)}\leq C\|r_-\|_{H^1\cap L^{2,1}}(\|r_+\|_{H^1\cap L^{2,1}}+\|r_-\|_{H^1\cap L^{2,1}})\|r\|_{H^1\cap L^{2,1}},\\
            &\|I_2(r)\|_{H^1(\Real_+)\cap L^{2,1}(\Real_+)}\leq C\|r_+\|_{H^1\cap L^{2,1}}(\|r_+\|_{H^1\cap L^{2,1}}+\|r_-\|_{H^1\cap L^{2,1}})\|r\|_{H^1\cap L^{2,1}}
        \end{aligned}
    \end{equation}
    where $C$ is a positive constant.
\end{prop}
\begin{proof}
    For the convenience of the reader we prove this proposition although it is already proven in \cite{Pelinovsky2016}. We find by (\ref{e component RHP solution formula}) and integrating by parts
    \begin{eqnarray*}
    % \nonumber to remove numbering (before each equation)
      I_1(r)(x) &=& \frac{1}{2\pi\ii} \int_{\Real}\Pm\left([m_+(z;x)]_{12} r_-(z) e^{2\ii zx}\right)\!(y)\: r(y) e^{-2\ii yx}dy\\
       &=&   \frac{-1}{2\pi\ii} \int_{\Real}[m_+(y;x)]_{12} r_-(y) e^{2\ii zx}\Pp\left(r(z) e^{-2\ii zx}\right)(y) dy.
    \end{eqnarray*}
    Using the H\"{o}lder inequality and the estimate (\ref{e sup P^pm r 1}), we arrive at
    \begin{equation*}
        \sup_{x\in(x_0,\infty)}|\langle x\rangle^2I_1(r)(x)|\leq \|r_-\|_{L^{\infty}}\|r\|_{H^1}\sup_{x\in(x_0,\infty)} \|\langle x\rangle[m_+(y;x)]_{12}\|_{L^2_z}.
    \end{equation*}
    We know $\sup_{x\in(x_0,\infty)} \|\langle x\rangle[m_+(y;x)]_{12}\|_{L^2_z}\leq C\|r_+\|_{H^1}$ by \cite[Lemma 10]{Pelinovsky2016} which completes the proof of $I_1(r)\in L^{2,1}(\Real_+)$. The assertion $\partial_x I_1(r)\in L^{2}(\Real_+)$ is established by using again the inhomogeneous equation (\ref{e component RHP solution formula}), its $x$ derivative, integration by parts, H\"{o}lder inequality, and in the end estimates (\ref{e sup P^pm r 1}) - (\ref{e sup P^pm r 3}),
    \begin{equation*}
        \sup_{x\in(x_0,\infty)} \|\langle x\rangle[m_+(y;x)]_{12}\|_{L^2_z}\leq C\|r_+\|_{H^1}
    \end{equation*}
    and
    \begin{equation*}
        \sup_{x\in\Real} \|[\partial_x m_+(y;x)]_{12}\|_{L^2_z}\leq C(\|r_+\|_{H^1\cap L^{2,1}}+\|r_-\|_{H^1\cap L^{2,1}}).
    \end{equation*}
    The latter statement can also be found in \cite[Lemma 10]{Pelinovsky2016}.
\end{proof}
The proposition above yields directly the following fundamental result (see Lemma 11 in \cite{Pelinovsky2016}):
\begin{cor}\label{c u in H^11 and H^2 (positive half line)}
   Fix $M>0$. Under the assumptions of Lemma \ref{l solvability of RHP N=0} and if $\|r_+\|_{H^1\cap L^{2,1}}+\|r_-\|_{H^1\cap L^{2,1}}\leq M$, the potential $u$ reconstructed from the solution $m$ of \rh \ref{rhp m^0} by using (\ref{e rec 1}) and (\ref{e rec 2}) lies in $H^2(\Real_+)\cap H^{1,1}(\Real_+)$. Moreover, it satisfies the bound
   \begin{equation}\label{e u in H^11 and H^2 (positive half line)}
       \|u\|_{H^2(\Real_+)\cap H^{1,1}(\Real_+)}\leq C_M,
   \end{equation}
   where the constant $C_M$ does not depend on $r_{\pm}$.
\end{cor}
\begin{proof}
    We set
    \begin{equation}\label{e def w}
        w(x):=u(x)e^{\ii\int_{+\infty}^x|u(y)|^2dy}
    \end{equation}
    and
    \begin{equation}\label{e def v}
        v(x):=\overline{u}(x)e^{-\frac{1}{2\ii} \int_{+\infty}^x|u(y)|^2dy},
    \end{equation}
    such that the following relations hold:
    \begin{equation}\label{e relations u v w}
        \begin{aligned}
            |u(x)|&=|v(x)|=|w(x)|\\
            |u_x(x)|&\leq |v_x(x)|+\frac{1}{2}|v(x)|^3
        \end{aligned}
    \end{equation}
    Using the reconstruction formulas (\ref{e rec 1}) and (\ref{e rec 2}), Proposition \ref{p cauchy operator} (ii) and the integral equation (\ref{e alternative RHP solution formula}) we immediately find
    \begin{equation*}
        w(x)=\frac{2}{\pi\ii}\int_{\Real}\overline{r}_+(z) e^{-2\ii zx}dz\;+\;4\,I_1(\overline{r}_+)(x)
    \end{equation*}
    and
    \begin{equation*}
        e^{-\frac{1}{2\ii} \int_{+\infty}^x|u(y)|^2dy}v_x(x)
        =-\frac{1}{\pi}\int_{\Real}r_-(z) e^{2\ii zx}dz\;-\;2\ii\:I_2(r_-)(x).
    \end{equation*}
    In each of these equations the first summand on the right hand side is controlled in $H^1\cap L^{2,1}$ since $r_{\pm}\in H^1\cap L^{2,1}$. Moreover Proposition \ref{p bound <x>^2I} yields directly $w\in L^{2,1}(\Real_+)$ and $v_x\in L^{2,1}(\Real_+)$ and thus finally by (\ref{e relations u v w}) $u\in H^{1,1}(\Real_+)$. Proposition \ref{p bound <x>^2I} also leads to
    \begin{equation*}
        \partial_x\left(e^{-\frac{1}{2\ii} \int_{+\infty}^x|u(y)|^2dy}v_x(x)\right)\in L^2_x(\Real_+).
    \end{equation*}
    By a straightforward calculation we conclude $u\in H^2(\Real_+)$. The bound (\ref{e u in H^11 and H^2 (positive half line)}) is obtained from application of (\ref{e bound <x>^2I}). The proof of the Corollary is now complete.
\end{proof}
With regard to the B\"{a}klund transformation which we intend to use in the following section in order to include solitons we need the following Lemma in addition to (\ref{e u in H^11 and H^2 (positive half line)}). The only purpose in the repeating of so many details of the inverse Scattering withour poles is to deduce this Lemma which can not be found in \cite{Pelinovsky2016}.
\begin{lem}\label{l m(z0)-1 in H^11 and H^2}
   Let the assumptions of Corollary \ref{c u in H^11 and H^2 (positive half line)} be valid and fix $z_0\in\Compl\setminus\Real$. Then for the solution $m(z;x)$ of \rh \ref{rhp m^0} we have $m(z_0;\cdot)-1\in H^1(\Real_+)\cap L^{2,1}(\Real_+)$ with the bound
   \begin{equation}\label{e m(z0)-1 in H^1 and L^2,1}
       \|m(z_0;\cdot)-1\|_{H^1(\Real_+)\cap L^{2,1}(\Real_+)}\leq C_{M},
   \end{equation}
   where the constant $C_M$ depends on $z_0$ and $M$ but not on $r_{\pm}$.
\end{lem}
\begin{proof}
    Fix $z_0\in\Compl\setminus\Real$. We use (\ref{e alternative RHP solution formula}) to find
    \begin{equation}\label{e m21 m12 in L^2,1}
        \begin{aligned}
            &[m(z_0;x)]_{12}= \frac{1}{2\pi\ii} \int_{\Real}\frac{[m_-(y;x)]_{11}\overline{r}_+(y) e^{-2\ii yx}}{y-z_0}dy= \frac{1}{2\pi\ii}\int_{\Real}\widetilde{r}_+(z) e^{-2\ii zx}dz\;+\;I_1(\widetilde{r}_+)(x),\\
            &[m(z_0;x)]_{21}= \frac{1}{2\pi\ii} \int_{\Real}\frac{[m_+(y;x)]_{22}r_-(y) e^{2\ii yx}}{y-z_0}dy= \frac{1}{2\pi\ii}\int_{\Real}\widetilde{r}_-(z) e^{2\ii zx}dz\;+\;I_2(\widetilde{r}_-)(x),
        \end{aligned}
    \end{equation}
    where $\widetilde{r}_-(z):={r}_-(z)/(z-z_0)$ and $\widetilde{r}_+(z):=\overline{r}_+(z)/(z-z_0)$, respectively. Due to the fact that $\|\widetilde{r}_{\pm}\|_{H^1\cap L^{2,1}}\leq c \|{r}_{\pm}\|_{H^1\cap L^{2,1}}$, where the constant $c>0$ depends on $z_0$ only, and using Proposition \ref{p bound <x>^2I} we end up with (\ref{e m(z0)-1 in H^1 and L^2,1}) for the non diagonal entries $m_{12}$ and $m_{21}$.
    Using again (\ref{e alternative RHP solution formula}) we obtain
    \begin{equation*}
        [m(z_0;x)]_{11}=1+\frac{1}{2\pi\ii} \int_{\Real} \frac{[m_+(y;x)]_{12}r_-(y) e^{2\ii yx}}{y-z_0}dy,
    \end{equation*}
    where we can insert $[m_+(y;x)]_{12}=\Pp ([m_-(z;x)]_{11} \overline{r}_+(z) e^{-2\ii zx})(y)$ from the integral equation (\ref{e integral equation for m+-}). Then we integrate by parts and obtain
    \begin{equation}\label{e formula for m(z_0;x)-1 in the proof of the lemma}
        [m(z_0;x)]_{11}=1-\frac{1}{2\pi\ii} \int_{\Real} [m_-(y;x)]_{11} \overline{r}_+(y) e^{-2\ii yx}\;\Pm (\widetilde{r}_-(z) e^{2\ii zx})(y)\:dy,
    \end{equation}
    where we put again $\widetilde{r}_-(z):={r}_-(z)/(z-z_0)$. Furthermore we set
    \begin{equation*}
        R_+(y):=\overline{r}_+(y)\;\Pm (\widetilde{r}_-(z) e^{2\ii zx})(y).
    \end{equation*}
    To prove $R_+\in H^1\cap L^{2,1}$ we recall the continuity property $\|\Ppm\|_{L^2\to L^2}=1$. One consequence is that $\|\Pm (\widetilde{r}_-(z) e^{2\ii zx})(\cdot)\|_{L^2} \leq c \|r_-\|_{L^2}$. Additionally, we find
    \begin{equation*}
        \|\partial_z \Pm (\widetilde{r}_-(z) e^{2\ii zx})(z)\|_{L^2_z}\leq \| \Pm (\widetilde{r}'_-(z) e^{2\ii zx})(\cdot)\|_{L^2_z} +\|2\ii x \Pm (\widetilde{r}_-(z) e^{2\ii zx})(\cdot)\|_{L^2_z},
    \end{equation*}
    where we can apply the bound (\ref{e sup P^pm r 1}) of Proposition \ref{p cauchy operator} and again $\|\Ppm\|_{L^2\to L^2}=1$. Thus, we are able to control  $\Pm (\widetilde{r}_-(z) e^{2\ii zx})(\cdot)$ in $H^1$ uniformly for $x>0$. Altogether, we have shown $\|R_+\|_ {H^1\cap L^{2,1}}\leq c \|r_+\|_ {H^1\cap L^{2,1}}\|r_-\|_ {H^1}$, which is needed because we want to apply Proposition \ref{p bound <x>^2I}. Therefore we write (\ref{e formula for m(z_0;x)-1 in the proof of the lemma}) in the form
    \begin{equation*}
        [m(z_0;x)]_{11}-1=\frac{1}{2\pi\ii} \int_{\Real} R_+(y)e^{-2\ii yx}\:dy+ \:I_1(R_+\!)(x).
    \end{equation*}
    Analogously, it can be carried out in a similar way, that for $R_-(y):=r_-(y)\;\Pp (\widetilde{r}_+(z) e^{-2\ii zx})(y)$,
    \begin{equation*}
        [m(z_0;x)]_{22}-1=\frac{1}{2\pi\ii} \int_{\Real} R_-(y)e^{2\ii yx}\:dy+ \:I_2(R_-\!)(x).
    \end{equation*}
    Combining Fourier theory and the bound (\ref{e bound <x>^2I}) we have now accomplished the proof of (\ref{e m(z0)-1 in H^1 and L^2,1}) also for the diagonal entries.
\end{proof}
Estimates on the negative half-line can be found by modifying the solution $m(z;x)$ of \rh \ref{rhp m^0} in the following way:
\begin{equation}\label{e def m delta}
    m_{\delta}(z;x):=m(z;x)
    \left[
      \begin{array}{cc}
        \delta^{-1}(z) & 0 \\
        0 & \delta(z) \\
      \end{array}
    \right],
\end{equation}
where
\begin{equation}\label{e delta}
    \delta(z)=\exp\left(\frac{1}{2\pi\ii} \int_{\Real}\frac{\log(1+ \overline{r}_+(y)r_-(y))}{y-z}dy\right).
\end{equation}
In Proposition 8 in \cite{Pelinovsky2016} it is shown that $\log(1+ \overline{r}_+r_-)\in L^2(\Real)$ due to (\ref{e r constraint}). Hence, the integral in (\ref{e delta}) is well-defined and $\delta$ solves the following RHP:
\begin{framed}
    \begin{rhp}\label{rhp delta}
        Find a scalar valued function $\Compl\ni z\mapsto \delta(z)$ which satisfies
        \begin{enumerate}[(i)]
          \item $\delta(z)$ is meromorphic in $\Compl\setminus\Real$.
          \item $\delta(z)=1+\mathcal{O} \left(\frac{1}{z}\right)$ as $|z|\to\infty$.
          \item The non-tangential boundary values $\delta_{\pm}(z)$ exist for $z\in\Real$ and satisfy the jump relation
              \begin{equation}\label{e jump delta}
                 \delta_+(z)=\left[1+ \phantom{\widehat{l}}\overline{r}_+(z)r_-(z) \right] \delta_-(z).
              \end{equation}
        \end{enumerate}
    \end{rhp}
\end{framed}
Using the symmetry $\delta(\overline{z})=\overline{\delta}^{-1}(z)$ and the jump condition (\ref{e jump delta}) it is an easy exercise to verify, that the function $m_{\delta}(z;x)$ defined in \ref{e def m delta} is a solution to the following \rh:
\begin{samepage}
\begin{framed}
    \begin{rhp}\label{rhp m^0 delta}
        Find for each $x\in\Real$ a $2\times 2$-matrix valued function $\Compl\ni z\mapsto m_{\delta}(z;x)$ which satisfies
        \begin{enumerate}[(i)]
          \item $m_{\delta}(z;x)$ is meromorphic in $\Compl\setminus\Real$ (with respect to the parameter $z$).
          \item $m_{\delta}(z;x)=1+\mathcal{O} \left(\frac{1}{z}\right)$ as $|z|\to\infty$.
          \item The non-tangential boundary values $m_{\pm,\delta}(z;x)$ exist for $z\in\Real$ and satisfy the jump relation
              \begin{equation}\label{e jump m^0 delta}
                  m_{+,\delta}=m_{-,\delta}(1+R_{\delta}) ,\quad\text{where}\quad
                  R_{\delta}(z;x):=
                  \left[
                    \begin{array}{cc}
                       0 & e^{-2\ii zx}\overline{r}_{+,\delta}(z) \\
                       e^{2\ii zx}r_{-,\delta}(z) & \overline{r}_{+,\delta}(z) r_{-,\delta}(z) \\
                    \end{array}
                  \right],
              \end{equation}
              and $r_{\pm,\delta}(z):= \overline{\delta}_+(z) \overline{\delta}_-(z)r_{\pm}(z)$.
        \end{enumerate}
    \end{rhp}
\end{framed}
\end{samepage}
The new jump matrix $R_{\delta}$ admits an factorization analogously to (\ref{e def R+ and R-}). For
\begin{equation}\label{e def R+ and R- delta}
    R_{+,\delta}(z;x):=
                  \left[
                    \begin{array}{cc}
                       0 & 0 \\
                       e^{2\ii zx}r_{-,\delta}(z) & 0 \\
                    \end{array}
                  \right],\qquad
    R_{-,\delta}(z;x):=
                  \left[
                    \begin{array}{cc}
                       0 & e^{-2\ii zx}\overline{r}_{+,\delta}(z) \\
                       0 & 0 \\
                    \end{array}
                  \right],
\end{equation}
we find
$m_{+,\delta} -m_{-,\delta}=m_{-,\delta}R_{+,\delta} +m_{+,\delta}R_{-,\delta}$ for $z\in \Real$, such that analogously to (\ref{e alternative RHP solution formula}),
\begin{equation*}
    m_{\delta}(z;x)=1+\frac{1}{2\pi\ii}\int_{\Real} \frac{m_{-,\delta}(y;x)R_{+,\delta}(y;x)+ m_{+,\delta}(y;x)R_{-,\delta}(y;x)}{y-z}dy.
\end{equation*}
The following exemplary calculation shows why \rh \ref{rhp m^0 delta} can be studied in order to extend Lemma \ref{l m(z0)-1 in H^11 and H^2} and Corollary \ref{c u in H^11 and H^2 (positive half line)} to the negative half-line. We have for $z_0\in\Compl\setminus\Real$
\begin{eqnarray*}
% \nonumber to remove numbering (before each equation)
  [m_{\delta}(z_0;x)]_{12}&=& \frac{1}{2\pi\ii} \int_{\Real}\frac{[m_{+,\delta}(y;x)]_{11} \overline{r}_{+,\delta}(y) e^{-2\ii yx}}{y-z_0}dy \\
   &=&  \frac{1}{2\pi\ii} \int_{\Real}\widetilde{r}_{+,\delta}(y) e^{-2\ii yx}dy+I_{1,\delta}(\widetilde{r}_{+,\delta})
\end{eqnarray*}
where $\widetilde{r}_{+,\delta}(z):= \overline{r}_{+,\delta}(z)/(z-z_0)$ and
\begin{equation*}
    I_{1,\delta}(r):=\frac{1}{2\pi\ii} \int_{\Real}[m_{+,\delta}(y;x)-1]_{11} r(y) e^{-2\ii yx}dy.
\end{equation*}
The functional $I_{1,\delta}(r)$ satisfies the same estimates as in Proposition \ref{p bound <x>^2I} with $\Real_+$ replaced by $\Real_-$ because the operators $\Pp$ and $\Pm$ swap their places in comparison with the integral equation (\ref{e integral equation for m+-}).
\begin{lem}\label{l m(z0)-1 in in H^1 and L^2,1 (delta)}
   Fix $M>0$ and $z_0\in\Compl\setminus\Real$ and let the assumptions of Lemma \ref{l solvability of RHP N=0} be valid. If in addition $\|r_+\|_{H^1\cap L^{2,1}}+\|r_-\|_{H^1\cap L^{2,1}}\leq M$, then for the solution $m_{\delta}(z;x)$ of \rh \ref{rhp delta} we have $m_{\delta}(z_0;\cdot)-1\in H^1(\Real_-)\cap L^{2,1}(\Real_-)$ with the bound
   \begin{equation*}%\label{e m(z0)-1 in H^1 and L^2,1}
       \|m_{\delta}(z_0;\cdot)-1\|_{H^1(\Real_-)\cap L^{2,1}(\Real_-)}\leq C_{M},
   \end{equation*}
   where the constant $C_M$ depends on $z_0$ and $M$, but not on $r_{\pm}$.
\end{lem}
With respect to the potential $u(x)$ the two Riemann--Hilbert problems \ref{rhp m^0} and \ref{rhp m^0 delta} are equivalent in the following sense:
\begin{equation}\label{e equivalent rhps}
    \begin{aligned}
        \lim_{|z|\to\infty}z\;[m(z;x)]_{12}= \lim_{|z|\to\infty}z\;[m_{\delta}(z;x)]_{12},\\
        \lim_{|z|\to\infty}z\;[m(z;x)]_{21}= \lim_{|z|\to\infty}z\;[m_{\delta}(z;x)]_{21}.
    \end{aligned}
\end{equation}
This observation follows directly from the definition (\ref{e def m delta}) and leads to the following extension of Corollary \ref{c u in H^11 and H^2 (positive half line)}.
\begin{cor}\label{c u in H^11 and H^2 (negative half line)}
   Fix $M>0$. Under the assumptions of Lemma \ref{l solvability of RHP N=0} and if $\|r_+\|_{H^1\cap L^{2,1}}+\|r_-\|_{H^1\cap L^{2,1}}\leq M$, the potential $u$ reconstructed from the solution $m$ of \rh \ref{rhp m^0} by using (\ref{e rec 1}) and (\ref{e rec 2}) lies in $H^2(\Real_-)\cap H^{1,1}(\Real_-)$ and satisfies the bound
   \begin{equation}\label{e u in H^11 and H^2 (negative half line)}
       \|u\|_{H^2(\Real_-)\cap H^{1,1}(\Real_-)}\leq C_M
   \end{equation}
   where the constant $C_M$ does not depend on $r_{\pm}$.
\end{cor}

